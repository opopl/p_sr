% vim: keymap=russian-jcukenwin
%%beginhead 
 
%%file 18_06_2021.fb.antonovskij_roman.1.aleksandr_nevskij
%%parent 18_06_2021
 
%%url https://www.facebook.com/roman.antonovskiy/posts/4224455017615727
 
%%author Антоновский, Роман
%%author_id antonovskij_roman
%%author_url 
 
%%tags donbass,galickij_daniil,istoria,nevskii_aleksandr,rossia,rusmir,ukraina,vojna,zapad
%%title Князь Северо-Восточной Руси, святой Александр Невский в своё время сделал верный выбор
 
%%endhead 
 
\subsection{Князь Северо-Восточной Руси, святой Александр Невский в своё время сделал верный выбор}
\label{sec:18_06_2021.fb.antonovskij_roman.1.aleksandr_nevskij}
\Purl{https://www.facebook.com/roman.antonovskiy/posts/4224455017615727}
\ifcmt
 author_begin
   author_id antonovskij_roman
 author_end
\fi

На Руси во все времена были сторонники полной интеграции с Западом. 

Есть они и сейчас. Однако история учит нас тому, что Россия должна идти своим путём. 

Князь Северо-Восточной Руси, святой Александр Невский в своё время сделал верный выбор. 

С Ордой на Востоке он наладил вассальные отношения и заключил мир. А на Западе
вёл победоносные войны, громя тевтонов, ливонцев и шведов. 

Почти в то же самое время князь Западной Руси, Даниил Галицкий поступил ровно наоборот. 

С монголами он воевал, а с поляками и венграми заключал тактические и стратегические союзы. 

Почему же прав был Невский, а не Галицкий? 

Монголо-татары не навязывали Руси свою культуру и религию, иго не было
оккупацией. Дипломатия Невского позволила избежать войны и оккупации,
ограничившись вассалитетом с уплатой дани и предоставлением русских воинов для
походов.  

Окрепнув, Русь сбросила с себя и это иго, покорив осколки Орды в виде
Казанского, Астраханского, Крымского и Сибирского ханства. 

Запад же хотел уничтожить Русь как самобытную православную цивилизацию и прибрать в рукам ее земли.  

Восточные походы немцев, шведов и поляков были крестовыми. Они хотели огнём и мечом насадить на Руси католичество.  

Невский знал о том, как католические крестоносцы онемечили славян-венедов,
населявших Восточную Пруссию. Знал он и о том, что они разграбили столицу
Византии по пути в Иерусалим. 

А отбив Святую Землю у сарацинов, католики притесняли там не только мусульман, но и православных. 

Даниилу Галицкому в обмен на военную помощь Запада против монголов пришлось
пойти на унию. Признать главенство Папы Римского и над своей православной
церковью. 

Чем все кончилось, мы знаем. 
Из Руси Александра Невского проросла великолепная Российская Империя. 

Земли некогда сильной Западной Руси были разорены западными «союзниками» и на
долгое время стали собственностью немцев и поляков.  Там они жестко насаждали
свою культуру, подавляя все русское. По сути на западнорусских землях сложилось
две идентичности: Русины, спрятавшись в неприступных карпатских горах, хранили
верность Руси. 

Галичане восприняли западную идентичность и антирусскую пропаганду.  

В 21 веке ничего не поменялось. Есть Россия, противостоящая западной экспансии,
и есть Украина - Западная Русь, которая легла под Запад и стремительно теряет
свою русскую идентичность. 

Вплоть до того, что русскоязычные парни с Восточной Украине воюют против
русских на Донбассе в составе Азова или Айдара. Только вместо принятия
католичества сегодня Запад прогибает нас под принятие своей новой культуры
толерантного неотроцкизма и ее девиантных ценностей. 

С нашей стороны, как и во времена Невского, этому снова противостоит наша
русская православная цивилизация. 

И снова, как и тогда, не страшен нам поворот на Восток. 

Могучему Китаю Запад тоже объявил торговую и ценностную войну. И Китай
заинтересован в России как в политическом и экономическом партнере, но как и
Орда он не будет навязывать России свои ценности и пытаться оккупировать наши
территории.
