% vim: keymap=russian-jcukenwin
%%beginhead 
 
%%file 26_11_2021.fb.lukjanenko_sergej.pisatel.1.zolotoj_vek.cmt
%%parent 26_11_2021.fb.lukjanenko_sergej.pisatel.1.zolotoj_vek
 
%%url 
 
%%author_id 
%%date 
 
%%tags 
%%title 
 
%%endhead 
\subsubsection{Коментарі}

\begin{itemize} % {
\iusr{Yuri Mori}
Всё так. Но почему это сопровождается такой ненавистью между людьми?

\begin{itemize} % {
\iusr{Андрей Потапов}
\textbf{Yuri Mori} потому что это в нашей природе. Увы

\iusr{Yuri Mori}
\textbf{Андрей Потапов} в нашей природе ещё много чего есть, мне кажется. Любовь и красота. Созидание и мечты.

\iusr{Сергей Лукьяненко}
\textbf{Yuri Mori} Потому что когда мир ломается - люди ломаются вместе с ним.

\iusr{Андрей Потапов}
\textbf{Yuri Mori} в пирамиде потребностей это более высокие уровни. Если нижние не обеспечены, об этих можно забыть. Я раньше считал это циничной позицией, но потом стоило понаблюдать - и я понял...

\iusr{Yuri Mori}
\textbf{Сергей Лукьяненко} люди всегда будут разными. Кому охота сломаться и оскотиниться - тому никаких перемен вокруг не нужно. А кто хочет оставаться человеком в любых условиях, тот и останется.

\iusr{Oleg Badin}
Может, наступил Платиновый Век? Платина металл не такой эффектный, как золото... Но более технологичный... Как-то так.

\iusr{Андрей Потапов}
\textbf{Oleg Badin} не вижу резона в технологиях, когда люди злы друг к другу.

\iusr{Евгений Рахимжанов}
Резон есть. Сейчас благодаря глобальным коммуникациям все добрые люди могут собраться вместе и замочить всех злых)))

\iusr{Алекс Ломанов}
\textbf{Сергей Лукьяненко} замечательная фраза. Мир сломали люди, а людей сломал сломанный мир.

\iusr{Сергей Лукьяненко}
\textbf{Yuri Mori} Вначале прочел вместо "мечты" - "менты". Задумался. Даже решил, что понял глубинный смысл. Потом перечитал.

\iusr{Андрей Потапов}
\textbf{Сергей Лукьяненко} вот, кстати, без ментов созидание действительно хромает  @igg{fbicon.smile} 

\iusr{Alexandr Maltsev}
\textbf{Yuri Mori}, врядли "сломанные и оскотиневшиеся " люди (ну, те - кому никаких перемен для этого не нужно) себя таковыми считают. Разность людей штука субьективная, для вас кто-то плохой, а для кого- то плохой вы.

\iusr{Михаил Герасимов}
\textbf{Сергей Лукьяненко} сначала ломаются люди, а потом - мир.

\iusr{Лидия Афанасьева}
\textbf{Yuri Mori} те, кто испытывает ненависть будут вымирать. Выживут те, кто поймёт, что люди едины на уровне ноосферы - о чем говорил ещё академик Вернадский
И научатся жить и принимать решения, исходя из этого единства.

\iusr{Константин Бойдало}
\textbf{Евгений Рахимжанов} коммуникации всегда объединяют злых.

\iusr{Татьяна Глущенко}
\textbf{Yuri Mori} нет ничего больнее, чем терять иллюзии
\end{itemize} % }

\iusr{Владимир Перемолотов}

Да. Золотой Век это всегда надежда на что-то лучшее, подспудная уверенность,
что реализация надежды дело реальное. А она ушла...


\iusr{Андрей Потапов}

Мы просто ходим от крайности к крайности, как маятник. И всё хорошо, только
когда мы пересекаем золотую середину. Короткий миг счастья перед новым витком
маразма.

\iusr{Федор Бобков}
Удивительно точно вы описали мои ощущения! Магия.

\iusr{Ирина Черкашина}

Теперь я поняла, почему лет двадцать назад постап сделался так популярен...
Перемены рождаются в книгах, а уж потом в реальности  @igg{fbicon.smile} 

А что там сейчас в тренде - попаданцы?

\begin{itemize} % {
\iusr{Андрей Потапов}
\textbf{Ирина Черкашина} если современники начнут попадать в головы мировых лидеров прошлого, всё накроется медным тазом гораздо раньше  @igg{fbicon.face.grinning.sweat} 

\iusr{Георгий Саенко}
\textbf{Андрей Потапов} может они уже попали, кхм.

\iusr{Sergey Mineev}
\textbf{Андрей Потапов} В Брежнева вот 200 попыток было проникновения попаданцев, но он пил лекарство от давления, а оно не даёт вселиться попанданцам)

\iusr{Андрей Потапов}
Sergey Mineev "Настоящее попадалово" (с)  @igg{fbicon.smile} 

\end{itemize} % }

\iusr{Ярослав Вечер}
Американская мечта почила вместе с коммунизмом в общей могиле истории, убитые цифрой

\iusr{Igor Davidchuk}
Хорошо сказано. Правильно

\iusr{Грызлов Валерий}
Золотой век всегда в будущем. Или где-то в вымышленном мире.

\begin{itemize} % {
\iusr{Андрей Потапов}
\textbf{Грызлов Валерий} не соглашусь. А как же эпоха Просвещения? Как же лучшие годы СССР?

\iusr{Филипп Гиревка}
\textbf{Грызлов Валерий} не, золотой век всегда в прошлом.

\iusr{Грызлов Валерий}
\textbf{Андрей Потапов} это и есть золотой век в вымышленном мире. В реальном мире в то время продолжались войны, эпидемии, голод. В СССР просто берегли нервы советским гражданам. А эпоха Просвещения это череда изматывающих войн.

\iusr{Грызлов Валерий}
\textbf{Филипп Гиревка} у тех, у кого нет будущего.
\end{itemize} % }

\iusr{Владимир Ватрушин}
"Права человека" - это не более чем высказывание Чукчи (как персонажа анекдотов), когда он соглашается с кем-либо.

\iusr{Александр Тишков}

Золотой век? 2 мировые войны. Миллионы умерших от голода. Террор. Репрессии.
Геноцид. Ядерные бомбардировки. Ну такое... Не похоже на Золотой век.

\begin{itemize} % {
\iusr{Милослав Князев}
\textbf{Александр Тишков} так без золотого века всё это тоже было бы, только возведённое в большую сепень.

\iusr{Александр Тишков}
\textbf{Милослав Князев} Ну эдак в каждой эпохе можно разглядеть золотой век со своими недостатками.

\iusr{Милослав Князев}
\textbf{Александр Тишков} раньше было лучше (С)

\iusr{Алексей Борисов}

Ещё сто лет назад голод был постоянным спутником человечества. 90 процентов
населения выращивали еду, и тут же её сьедали. Если успевали, пока не отобрали.

А счас эти же 90 процентов могут заниматься всякой фигнёй на сытый желудок. На
фоне того постоянного голода, нынешние войны, репрессии и геноцид это
действительно золотой век.


\iusr{Александр Тишков}
\textbf{Алексей Борисов} 

Так ведь про то и речь. У автора золотой век 150 лет назад начался. А про 90
процентов сытых. Где? В Африке? Азии? Латинской Америке? Сытые? Ну-ну.

\end{itemize} % }
\iusr{Олег Лукьяненко}

Основное свойство человека разумного - не ненависть к подобным, и, к сожалению,
не любовь. Главное в нас - это приспособляемость к изменяющемуся миру. Пандемия
привела к ломке старых норм и появлению новых. Но мы и это переживем. А Золотой
Век - он во все времена был в прошлом.

\iusr{Алла Озерова}

Да, тоже так чувствую. И, к сожалению, за золотым веком всегда наступают
смутные темные времена. Нам повезло- мы застали кусочек золотого века, правда,
не понимали этого тогда

\iusr{Шеремет Глеб}

Я тут прочитал Артура Кларка "большая глубина" автор в 57 году описывал 2015.
Автор Оэочень оптимистично по отношению к человечеству наше настоящее
описывает. Но при этом характерная особенность, что "гуманизмом" движущей его
силой, в книге наделяется технический прогресс, а человек за ним поспевает.

Азимов алгоритмизируя категорический императив ещё верил в будущее, а вот Филип
Дик уже нет и описывал конфликт практически как в Идиоте у ФМ. Понятно пожиже и
пониже, но конфликт то тот же.

\iusr{Марина Чиркова}
золотой век кончится вместе с нефтью.

\iusr{Zahar Kariagin}

\obeycr
Двадцатый век был веком стали,
Титана, стронция, гептила.
Но дровосеки все сменяли
На соль, лимончик и текилу...
\restorecr

\textbf{\#ZaharKariagin}

\begin{itemize} % {
\iusr{Олег Лукьяненко}
\textbf{Zahar Kariagin} Да уж, от стронция лучше «Столичная», чем текила... )
\end{itemize} % }

\iusr{Алексей Альбинский}

Со всем уважением, но экспедиция по спасению капитана Гранта была в 1864
году...

\begin{itemize} % {
\iusr{Сергей Лукьяненко}
\textbf{Алексей Альбинский} Но книжка, если не ошибаюсь, вышла в 1868...

\iusr{Алексей Альбинский}
\textbf{Сергей Лукьяненко} Книжка да.
\end{itemize} % }

\iusr{Григорий Бурмистров}
Так и есть....

\iusr{David Starsky}

"...Золотой век закончился в 1991 году, вместе с распадом СССР, когда "история
закончилась" и напрягаться стало не нужно..." И для кого же он был в СССР
"золотым"? Для современных некрофилов из стана адептов возрождения СССР 2.0 на
основе РФ 2.ХХ ?

\iusr{Алексей Борисов}

Надо же, вся европейская философия и христианское мировоззрение вдруг взяли и
закончились. Доктор сказал Ковид, значит Ковид.

\iusr{Igor Vavilov}
Для «умеющих приспособиться» «золотой» век наступил в России в 1991-м.  @igg{fbicon.face.confused} 

\iusr{Лана Белова}
Если позолота сползает, значит, и век был не золотой, а суррогатный))

\iusr{Sergey Ivanov}
Я думал ты рассказал это ещё когда писал «Танцы на снегу»

\begin{itemize} % {
\iusr{Сергей Лукьяненко}
\textbf{Sergey Ivanov} Иногда надо повторять...
\end{itemize} % }

\iusr{Юлий Буркин}

А я бы призвал наоборот - и дальше придерживаться тех прекрасных принципов
свободы и равенства. Пусть это была и иллюзия, но людям при ней жилось лучше.

\iusr{Walera Rud}
Жуть

\iusr{Евгений Квятковский}

И в хороших, и в посредственных присутствуют одни и те же компоненты: рифмы,
размер, содержание... Но от чтения хороших стихов обязательно появляется холодок
в спине. За все, что мы с тобой и что с детьми случится – вставай, иди на бой,
в ворота гунн стучится. Наш мир давно угас, но не расстался с нами, и все, что
есть у нас – лишь камень, сталь и пламя... А. Лазарчук.

\begin{itemize} % {
\iusr{Евгений Квятковский}
Это я к тому, что от этого текста тоже холодок в спине появляется. Хороший пост, значит.
\end{itemize} % }

\iusr{Галина Лепешина}
Баланс между приспособление и сохранением в себе основ нелегок

\iusr{Михаил Шеховцов}

Король умер, да здравствует Король! Старый мир умер, зато рождается новый
прекрасный удивительный мир!

\begin{itemize} % {
\iusr{Михаил Шеховцов}
\textbf{Сергей Лукьяненко} « и треснул мир напополам», может это к лучшему?
\end{itemize} % }

\iusr{Лариса Икрянникова}

Очень похоже на правду. Но ведь будет новый золотой век, история циклична.
Хотелось бы, чтобы наши дети застали хотя бы его начало - но вряд ли, наверно,
да?


\iusr{Лариса Икрянникова}
Понравился термин "профессиональные бездельники", а я-то все думала, как эти категории назвать :о))

\iusr{Sergey Tsarev}
Что-то золотое время все короче и короче:)

\iusr{Anuarbek Skakov}
Разве не с Одиссеи и Иллиады начался золотой век?

\iusr{Эдуард Аронин}

А по-моему, даже не смотря на то,что ковид немного портит(корректирует?)
картину, золотой век это как раз сейчас.

Когда ещё в истории человечества была такая продолжительность и уровень жизни?


\iusr{Вадим Долгопятов}
Дух перемен витает...

\iusr{Светлана Харитонова}

Да, ограничения по ковиду затронули в первую очередь сферы развлечений, отдыха,
спорта, туризма - \enquote{бездельников}. А у обычных работяг особых изменений в жизни
не произошло, сужу по себе. Так что, думаю, свинец


\iusr{Павел Бритов}
Печальная мудрость! Выходит, что нам повезло, а детей жалко... (Перефразируя старый анекдот).

\iusr{Роман Литвин}
"Золотой век" возник на фоне жестоких эксплуатаций колоний и мировых войн.

\iusr{Людмила Кобизькая}

Расстраиваться рано, золотой век был, наверно, только для золотого миллиарда.
Остальные и не догадывались. А теперь золотой век в Азии. А нам следует
задуматься.

\iusr{Ilya Belykh}
Хорошие слова были сказаны. В текушей обстановке неважно кто побеждает. Для
прбеды важно лиш то кто рухнет последним.


\iusr{O'Serge Phoenix}

Замечая, как понятие "золотого" века уходит в глубину времён, можно даже
сделать вывод, что впервые он наступил ещё до "бронзовых" и "железных" веков.
 @igg{fbicon.face.nerd} 

\iusr{Den Romnov}

Вы-то правы в том, что нужно приспосабливаться... только ведь странно выглядит,
когда защищаются права определенных "каст". И, ладно бы, как в древности,
защита была у богатых или знатных, а бедные и безродные должны были
довольствоваться тем, что им дадут. Сейчас-то защита и угнетение совершенно не
подчиняются каким-то законам логики...

\iusr{Андрей Смирнов}
Конец Века Золотарей  @igg{fbicon.smile} 

\iusr{Владимир Миляев}
а может и горячая сталь

\iusr{Жуй Рябчиков}

Просто жадная, лживая, похотливая, агрессивная и завистливая обезьяна Homo, еще
и самовлюблена настолько, что мнит себя честной, щедрой, романтичной и доброй.
Иногда ее лихорадит мечтами и книгами на эту тему. Не было золотого века. Было
медленное и ограниченное распространение информации.

\iusr{Сергей Зеленин}
Золотой век России - 1814-1917.

\begin{itemize} % {
\iusr{Сергей Щетинин}
\textbf{Сергей Зеленин} 

золотой, ага.
Для 1\% дворян империи.
Для 0,9\% духовенства.
Для 0,6\% купечества.
Кто ещё? Для 7\% казаков. И 10\% мещан.

Золотой век похоже всегда и во все времена основан на рабстве, ну и ему
подобных формах эксплуатации.

Ведь золото недёшево, на всех не хватит, а для 3-7\% населения империи вполне.
Можно расцветать в искусстве, литературе, музыке и прочая.

Да, кстати, почитайте "Пошехонскую старину", там этот Золотой век вполне себе
объективно описан, непосредственным очевидцем и участником.

\iusr{Сергей Зеленин}
\textbf{Сергей Щетинин} большевичков не спрашивали.
Про рабство лучше молчите, поскольку вы, коммуняки, установили его в России.

\iusr{Сергей Зеленин}
\textbf{Сергей Щетинин} про "благодетельный" совок скажите колхозникам, которых вы держали в рабстве

\iusr{Сергей Зеленин}
\textbf{Сергей Щетинин} о, из Тамбова. Не стыдно перед отравленными газом предкам, которые как-то не очень приняли совок?

\iusr{Антон Канарейкин}
Как же булкой захрустело. Почему всё-таки все наши монархисты такие идиоты?
\end{itemize} % }

\iusr{Галина Суркова}

Но мы живы, мы живём. И радость каждого дня, радость познания, общения-все
зависит от нас, пока мы живы. И вообще-то очень живы
@igg{fbicon.monkey.see.no.evil}  @igg{fbicon.smile} 

\iusr{Дмитрий Сезганов}

А если бы Дойль, Верн и Достоевский не фантазировали тогда - сейчас бы не было
и облома.  @igg{fbicon.face.screaming.in.fear} 

\iusr{Дмитрий Д.}
Исправьте, пожалуйста, гаЛЛоны

\iusr{Андрей Перла}
Монолог героя хорошего романа

\iusr{александр тупайло}

Точно сказано, только сползает не позолота, а сползает мир, мы все. Туда, куда
раньше спускались отходы всех этих золотых, серебряных и прочих веков. Нужно
пытаться оставаться людьми, теми самыми Хомо сапиенс в любой ситуации.

\iusr{Лев Грачёв}
К сожалению..

\iusr{Дмитрий Попов}

Не уверен, что описанное применимо ко всему человечеству. Думаю, с точки зрения
большинства людей Земли это совсем не так. Достаточно попытаться посмотреть на
все глазами китайца. Или индуса. А это треть всех землян

\begin{itemize} % {
\iusr{Сергей Лукьяненко}
\textbf{Дмитрий Попов} Ну разумеется я смотрю глазами гражданина России, человека вполне русско-европейского мировоззрения. Но затронет всех.
\end{itemize} % }

\iusr{Ольга Драпико}

Великий человек - Хосе Ортега-и-Гассет в своей работе «Восстание масс»
подробнейшим образом исследовал к чему же приведёт подобный «золотой век»,
когда массам стали доступны «удовольствия» Элиты.

\begin{itemize} % {
\iusr{Сергей Лукьяненко}
\textbf{Ольга Драпико} Не читал. Но теперь гляну.
\end{itemize} % }

\iusr{Павел Яшаров}
Всё двигается по спирали. И мы на очередном витке.

\iusr{Александр Татаринцев}
Прослушал недавно «Воскресение» Толстого. Он вот не думал, что в золотом веке живёт )

\iusr{Сергей Тимченко}
Замечательно написано

\iusr{Сергей Кондрашов}
\textbf{Сергей Васильевич}, так чья это работа, прогрессоров или регрессоров?

\iusr{Helena Klaasen}
Как это печально!

\iusr{Slava Konovalov}

Золотой век - век Полдня - ещё впереди. Но перед ним - да, распад и деградация,
откат к родоплеменному строю, века мракобесия и средневековья, а затем новый
подъем сознания, расцвет наук, и ... второй заход.

Пока, мы проиграли, мрак победил.

\iusr{Павел Николенко}
Нужно запасаться патронами ?

\iusr{Юрий Нечипоренко}
Здорово!

\iusr{Anatoly Loginov}

\obeycr
Тяжелые времена рождают сильных людей.
Сильные люди создают хорошие времена.
Хорошие времена рождают слабых людей.
Слабые люди создают тяжелые времена.
\restorecr


\iusr{Андрей Малыга}

Хотел бы еще заметить, что революции, Советский союз, имеют свои и не малые
заслуги, перед Золотым веком. Именно, чтобы не допустить революций в своих
странах, были приняты все эти декларации прав человека, именно под давлением
существования Советского союза и других социалистических стран, чтобы не
потерять все, были приняты все эти социальные реформы в капиталистических
странах, и именно потому, что Советский союз погиб, погиб и Золотой век, ибо
исчезло социалистическое давление на капиталистические страны!

\iusr{Владимир Васильев}

Не возможно поверить что автор. Тот кто указан перед текстом. Или у других
трудов на самом деле другой автор

\iusr{Владимир Нечаев}

Молодые и сильные выживут. Наверное.

\iusr{Алексей Уткин}

На Земле слишком много людей.  Так или иначе их число сократится.  Последние
года два отбор идет по уму. И это не так уж плохо.


\end{itemize} % }
