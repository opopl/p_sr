% vim: keymap=russian-jcukenwin
%%beginhead 
 
%%file 16_07_2021.fb.tjagnibok_oleg.1.unr_ukraina_pravopreemnik
%%parent 16_07_2021
 
%%url https://www.facebook.com/oleh.tyahnybok/posts/4339246202762727
 
%%author Тягнибок, Олег
%%author_id tjagnibok_oleg
%%author_url 
 
%%tags istoria,sssr,ukraina,unr,ussr,verhovna_rada,vo_svoboda
%%title Про правонаступництво України щодо Української Народної Республіки
 
%%endhead 
 
\subsection{Про правонаступництво України щодо Української Народної Республіки}
\label{sec:16_07_2021.fb.tjagnibok_oleg.1.unr_ukraina_pravopreemnik}
 
\Purl{https://www.facebook.com/oleh.tyahnybok/posts/4339246202762727}
\ifcmt
 author_begin
   author_id tjagnibok_oleg
 author_end
\fi

Символічно, що саме сьогодні, 16 липня, в День проголошення Декларації про
державний суверенітет України, у Верховній Раді зареєстровано законопроект
(ініціатор - свободівка Оксана Савчук) \#5796 «Про правонаступництво України
щодо Української Народної Республіки»

\ifcmt
  pic https://scontent-cdt1-1.xx.fbcdn.net/v/t39.30808-6/217394323_4339245769429437_6746108208519253858_n.jpg?_nc_cat=110&ccb=1-4&_nc_sid=8bfeb9&_nc_ohc=WB9LZ8uOypcAX_HUIyi&_nc_ht=scontent-cdt1-1.xx&oh=1f485b9d4210eb845808af8bcd50c5c5&oe=611A8AB3
  width 0.8
\fi

Вашій увазі текст закону. 

Ваші думки, товариство:

Спираючись на багатовікову традицію українського державотворення і на основі
підтвердженого Українською нацією права на самовизначення, коли на
Всеукраїнському референдумі 1 грудня 1991 року 90,32\% громадян України
проголосували за Незалежність України, беручи до уваги безперервність
української державно-правової традиції, яка була започаткована у Руській
державі та продовжена у Галицько-Волинському князівстві, Королівстві Руському,
Великому князівстві Литовському, Руському і Жемайтійському, Війську
Запорозькому (Гетьманщині), Українській Народній Республіці,
Західно-Українській Народній Республіці, Карпатській Україні, відновлена Актом
30 червня 1941 року та Актом 24 серпня 1991 року, на підставі того, що 22 січня
1918 року Центральна Рада IV Універсалом проголосила самостійну, незалежну,
вільну, суверенну державу – Українську Народну Республіку (УНР), відновивши
національну державність Української нації, ґрунтуючись на прагненні Української
нації жити в єдиній незалежній державі, що було підтверджено Універсалом
Директорії Української Народної Республіки від 22 січня 1919 року про
об’єднання Української Народної Республіки та Західно-Української Народної
Республіки, закріплюючи невід’ємність території Криму від Української Народної
Республіки та незаконність окупації Криму військами РСФСР, пам’ятаючи, що
Голодомор 1932-1933 років в Україні є геноцидом Українського народу і був
вчинений СССР як державою-окупантом, засуджуючи порушення основоположних права
і свобод людини і громадянина комуністичним тоталітарним режимом 1917-1991
років в Україні, пам’ятаючи про інші численні злочини окупаційного режиму СССР
проти Української нації, спрямовані на руйнування її вікових традицій, духовної
культури і етнічної самобутності, визнаючи постійну збройну та політичну
боротьбу Української нації проти окупації території України Совєтським Союзом,
керуючись тим, що Держава Україна успадкувала право на державний суверенітет
від Української Народної Республіки, що підтверджено Грамотою Державного Центру
Української Народної Республіки в екзилі від 22 серпня 1992 року, протидіючи
гібридній воєнній агресії Російської Федерації проти України, вчиненій
Російською Федерацією анексії Криму та окупації нею частини території України,
керуючись Конституцією України, Верховна Рада України приймає цей Закон і
проголошує правонаступництво України щодо Української Народної Республіки та
встановлює статус СССР як держави-окупанта.

Стаття 1. Україна є правонаступницею Української Народної Республіки за принципом «tabula rasa».

Стаття 2. Континуїтет сучасної української держави був започаткований у Руській
Державі, продовжений Галицько-Волинським князівством, Королівством Руським,
Великим князівством Литовським, Руським і Жемайтійським, Військом Запорозьким
(Гетьманщиною), Українською Народною Республікою, Західно-Українською Народною
Республікою, Карпатською Україною та Українською Державою, відновленою Актом 30
червня 1941 року та Актом 24 серпня 1991 року.

Стаття 3. Українська нація є автохтонною, титульною нацією держави Україна,
спадкоємицею розвиненої духовної і матеріальної культури сотень поколінь
українців.

Стаття 4. У період з 21 листопада 1921 року по 23 серпня 1991 року Україна мала статус окупованої країни.

Українська Народна Республіка остаточно була окупована 21 листопада 1921 року,
коли війська Красной Армії (РККА) розстріляли полонених воїнів армії
Української Народної Республіки, учасників Другого зимового походу у селі Базар
Житомирської області.

Стаття 5. У період окупації території України за кордоном діяв Державний центр
Української Народної Республіки як правонаступник Директорії Української
Народної Республіки, який до відновлення незалежності Української Держави та
формування легітимних органів влади виконував функції українського уряду в
еміграції.

Стаття 6. Суверенітет належить винятково українській нації.

Окупація території Української Народної Республіки не означає, що українська
нація втратила право на національний суверенітет, втілений у формі державного
суверенітету.

Виразниками цього права залишилися як сам факт існування української нації, так
і державні інституції Української Народної Республіки, що продовжували діяти за
межами України.

Україна не може успадкувати державний суверенітет від Української Совєтської
Соціалістичної Республіки, правонаступницею якої вона є відповідно до Закону
України «Про правонаступництво України» від 12 вересня 1991 року. Українська
Совєтська Соціалістична Республіка як окупаційна адміністрація СССР державним
суверенітетом не володіла.

Стаття 7. Підставою правонаступництва України щодо Української Народної
Республіки є акт передачі 22 серпня 1992 року Президентом УНР Миколою Плав’юком
Грамоти Державного Центру Української Народної Республіки, повноважень та
атрибутів державної влади Президентові України Леоніду Кравчуку, обраному на
вільних, демократичних загальноукраїнських виборах 1 грудня 1991 року.

Стаття 8. Ознаками успадкування Україною суверенітету від Української Народної
Республіки є:

\begin{itemize}
  \item а) українці, як автохтонна, титульна нація;
  \item б) територія, як переважна частина території УНР;
  \item в) українська мова, як єдина державна мова;
  \item г) місто Київ, як столиця держави;
  \item ґ) Державний Герб, Державний Прапор та Державний Гімн України, як символи Української Народної Республіки;
  \item д) спадкоємність державно-правової традиції.
\end{itemize}

Стаття 9. Російська Федерація як правонаступниця колишнього СССР є винною у
скоєнні злочинів проти українців окупаційним тоталітарним комуністичним режимом
в період окупації території України з 21 листопада 1921 року по 23 серпня 1991
року.

Україна має право вимагати від Російській Федерації компенсації за злочини,
скоєні проти українців окупаційним тоталітарним комуністичним режимом у період
окупації території України з 21 листопада 1921 року по 23 серпня 1991 року.

Стаття 10. Україна має право переглянути свої зобов’язання за міжнародними
договорами, які до відновлення незалежності України 24 серпня 1991 року уклала
Українська ССР як суб’єкт СССР, і підтвердити ті, які не суперечать Конституції
України, цьому Закону та національним інтересам української держави.

Стаття 11. Україна не несе відповідальність за державними боргами Української Народної Республіки.

Україна має право вимагати від правонаступниці СССР Російської Федерації
повернення заощаджень громадян України, які були на рахунках Сбербанку СССР
станом на 31 грудня 1991 року, та належної Україні частки золотих та валютних
запасів, алмазного фонду, закордонного майна колишнього СССР, як компенсації за
пограбування України і українців у період окупації.

Стаття 12. Закони та інші акти органів державної влади СССР та УССР, крім
Декларації про державний суверенітет України від 16 липня 1990 року, Акта
проголошення незалежності України від 24 серпня 1991 року, актів, що стосуються
державного кордону України, актів організаційно-розпорядчого характеру та
індивідуальних актів, не діють на території України.

Стаття 13. Державним кордоном України є кордон, який станом на 16 липня 1990
року був між Українською ССР і Білоруською ССР, РСФСР, Республікою Молдова та
кордон Союзу ССР, що відмежовував територію України від інших держав.

Території, які входили до складу Української Народної Республіки і є поза
межами держави Україна, є сферою життєво важливих національних інтересів
України.

Стаття 14. Вважати день 22 січня Днем Незалежності України і щорічно відзначати його як державне свято.

Рахувати незалежність Української держави з 22 січня 1918 року, з дня
проголошення IV Універсалу Центральної Ради Української Народної Республіки,
яким відновлено національну державність Української нації.

24 серпня 1991 року вважати Днем відновлення Незалежності України і щорічно відзначати його як державне свято.

Прикінцеві положення

\begin{itemize}
\item 1. Цей Закон набирає чинності з дня, наступного за днем опублікування.
\item 2. Визнати таким, що втратив чинність Закон України «Про правонаступництво України» (№ 1543-ХІІ).
\item 3. Визнати такою, що втратила чинність, постанову Верховної Ради України від 12 вересня 1991 року 1545-ХІІ «Про порядок тимчасової дії на території України окремих актів законодавства Союзу РСР» (Відомості Верховної Ради України, 1991 р., № 46, ст. 621).
\item 3. Кабінету Міністрів України у шестимісячний термін з дня набрання чинності цим Законом:

\begin{itemize}
  \item а) привести нормативно-правові акти Кабінету Міністрів України у відповідність із цим Законом.
  \item б) забезпечити приведення міністерствами та іншими центральними органами виконавчої влади України їхніх нормативно-правових актів у відповідність з цим Законом.
  \item в) створити робочу групу щодо визначення збитків завданих України у період окупації СССР.
\end{itemize}
\end{itemize}
