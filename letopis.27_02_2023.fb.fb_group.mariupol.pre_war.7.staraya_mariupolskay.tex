%%beginhead 
 
%%file 27_02_2023.fb.fb_group.mariupol.pre_war.7.staraya_mariupolskay
%%parent 27_02_2023
 
%%url https://www.facebook.com/groups/1233789547361300/posts/1418525102221076
 
%%author_id fb_group.mariupol.pre_war,elena_mariupolskaja
%%date 27_02_2023
 
%%tags mariupol,mariupol.pre_war,kolodjazj,voda
%%title Старая мариупольская криничка
 
%%endhead 

\subsection{Старая мариупольская криничка}
\label{sec:27_02_2023.fb.fb_group.mariupol.pre_war.7.staraya_mariupolskay}
 
\Purl{https://www.facebook.com/groups/1233789547361300/posts/1418525102221076}
\ifcmt
 author_begin
   author_id fb_group.mariupol.pre_war,elena_mariupolskaja
 author_end
\fi

Старая мариупольская криничка. Сейчас она не выглядит как раньше. Обложена
красным кирпичом, струйка воды вытекает из трубы. Старая лестница ... по ней
уже не спустится.Недалеко есть лестница с металлическими ступенями (тоже уже в
не очень хорошем состоянии), по дороге к роднику. встречаются целые речушки,
стекающие с горы. Нет уже и часовенки... лишь часть стены... А когда-то ...

«Водой город снабжается из старого фонтана. Он хорошо отделан; водоем
разделяется на три резервуара: для прачек, для питьевой воды и для водопоя
домашних животных. Над средним находится маленькая часовня с иконой Божьей
Матери. В сухое время года в сутки фонтан дает 146000 ведер».  Старинная
русская мера — ведро — соответствовала 12,3 литра.

Малоимущие мариупольцы и те, кто жил неподалеку от фонтана, ходили за водой с
ведрами или кувшинами пешком. Состоятельным же горожанам ее возили прямо домой
водовозы.

%\ii{27_02_2023.fb.fb_group.mariupol.pre_war.7.staraya_mariupolskay.cmt}
