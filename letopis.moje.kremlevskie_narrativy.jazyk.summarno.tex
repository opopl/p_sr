% vim: keymap=russian-jcukenwin
%%beginhead 
 
%%file moje.kremlevskie_narrativy.jazyk.summarno
%%parent moje.kremlevskie_narrativy.jazyk
 
%%url 
 
%%author_id 
%%date 
 
%%tags 
%%title 
 
%%endhead 

\paragraph{Суммарно по ситуации с русским языком}
\label{sec:moje.kremlevskie_narrativy.jazyk.summarno}

Ну что ж, идем дальше. На данный момент, если собрать имеющуюся информацию
по проблеме украинского и русского языков (начало 2022 года, еще до войны), а
это десятки и сотни тысяч страниц текста, десятки тысяч самых разнообразных
публикаций, комментариев пользователей и разных видео/аудио материалов,
получится примерно следующая картина:

\begin{itemize} % {
\item (1) русский язык является повсеместно распространенным по всей территории
Украины, от Ужгорода до Харькова, от Чернигова до Одессы, от Киева до
Мелитополя, от Сум до Тернополя; русским языком свободно владеют представители
всех политических партий и движений, от крайне националистического Правого
Сектора, абсолютно вражески настроенного по отношению к России, и чьим лидером
в 2013-2015 годах был Дмитрий Ярош, уроженец Днепродзержинска, и Украинской
Добровольческой Армии, до Оппозиционной Платформы за Жизнь (ОПЗЖ), партии,
ориентированной на интересы жителей Юго-Востока Украины и выступающей за
дружественные отношения с Российской Федерацией; на русском языке говорят как
дети, школьники и студенты, так и бабушки и дедушки; как мужчины, так и
женщины; как люди богатые, так и люди бедные; русским языком владеют
представители всех религиозных конфессий, от УПЦ (МП) до УГКЦ, также как и
атеисты и агностики;

\item (2) русский язык является одним из двух главных языков Украины, наравне с украинским;

\item (3) на русском больше говорят в Киеве, Харькове, Чернигове, - то есть
крупных городах, и вообще на всем юго-востоке, в то время как на Западной
Украине преобладает украинский; также, украинский больше распространен в
сельской местности, в том время как в городах - больше русский;

\item (4) украинский язык является единственным государственным; на нем ведется
делопроизводство, обучение в школах, университетах; 

\item (5) русский язык, на данным момент, являясь повсеместно распространенным
по всей территории Украинской Республики, тем не менее, в значительной степени
поражен в правах на государственном уровне. Несмотря на статью 10 Конституции
Украины, гарантирующую свободное развитие, использование и защиту русского
языка, а также статью 24, говорящую в частности о том, что не может быть
привилегий или ограничений по языковому признаку, к русскому языку в Украине на
государственном уровне, а также в массовом сознании многих людей, относятся как
к чему-то второстепенному, и даже как явлению вредному и ненужному.  На нем
массово не ведется обучение в университетах и в школах, русский язык не
считается частью официальной идеологии Украины как независимой державы; на
русском языке нет государственного гимна.  Что касается школ и дошкольного
образования, на настоящий момент, насколько нам известно, в Украине осталось
довольно незначительное количество русских школ и детсадиков. 

\item (6) Отдельно что касается Киева, русскоязычные киевляне по нашим
наблюдениям своем в массе своей отлично знают украинский, и не имеют
значительных предубеждений по поводу украинского языка, как отдельного
полноценного языка.

\item (7) касательно происхождения русского языка, часто в обсуждениях среди
простого народа, продвигается тезис о том, что украинский язык ярче, богаче
русского; что украинский древнее, чем русский, и что украинский является прямым
потомком языка, на котором говорили в Киевской Руси (образцы которого,
например, запечатлены на графити Софии Киевской), в то время как русский имеет
отношение в первую очередь к Москве, и к финно-угорским племенам меря и мокша,
а не к Киеву.

\ifcmt
  tab_begin cols=3,no_fig,center,resizebox=0.8
     pic https://pravlife.org/sites/default/files/field/image/graffiti1.jpg
     pic https://kiev-foto.info/images/statii/graffiti_sofii_5.jpg
     pic https://avatars.mds.yandex.net/i?id=073e8c7290d39f0f93fff00e54249a10-5234591-images-thumbs&n=13
  tab_end
\fi

\end{itemize} % }

