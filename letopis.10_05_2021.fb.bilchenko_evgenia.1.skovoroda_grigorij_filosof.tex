% vim: keymap=russian-jcukenwin
%%beginhead 
 
%%file 10_05_2021.fb.bilchenko_evgenia.1.skovoroda_grigorij_filosof
%%parent 10_05_2021
 
%%url https://www.facebook.com/yevzhik/posts/3878458652189216
 
%%author 
%%author_id 
%%author_url 
 
%%tags 
%%title 
 
%%endhead 

\subsection{БЖ. Царский певчий - Григорию Сковороде, первому философу Руси}
\label{sec:10_05_2021.fb.bilchenko_evgenia.1.skovoroda_grigorij_filosof}
\Purl{https://www.facebook.com/yevzhik/posts/3878458652189216}

БЖ. Царский певчий.


\ifcmt
  pic https://scontent-bos3-1.xx.fbcdn.net/v/t1.6435-9/183899789_3878458595522555_6684069420747664971_n.jpg?_nc_cat=105&ccb=1-3&_nc_sid=8bfeb9&_nc_ohc=Tu78H-tYEm4AX_R24Tv&_nc_ht=scontent-bos3-1.xx&oh=adbcded61771d53ea2b28831821d76eb&oe=60C3E8E0
\fi


Григорию Сковороде, первому философу Руси
Ты говоришь, Григорий, счастье - в Господней радости:
Просто идти вдоль ковидных стен через минное поле, рай нести
В рюкзаке за плечами, Елизавете навстречу к её капелле.

Как дождь капал? А вы, дождинки серебряные, как пели?
Научи меня так, как писал ты сам в "Разговоре" своём "об истинном".
Я - не путник пятый. Во мне - нет счастья: лишь небо твоё неистовое.
Казачий потомок, больной философ, бродячий поэт - всем порчен...

Федеральная журналистка зовёт на эспрессо с пончиками.
Со мной о политике не говорят политики, брат Григорий мой.
А все думают: "Повезло, ей хватило на Черногорию
От императорской щедрой платы за чистый голос,

За мудрый Логос и за Платона, за поседевший волос".
Григорий, ты смог придти в этот Дом: на престоле, не на шезлонге?
Да святится имя твоё, товарищ, Павел-апостол, Лонгин.
Я прячусь от мира, потопа змей, в религиях философий...

А ветра такие, ветра такие...
Как на Голгофе.

7 мая 2021 г.

Фото: Анастасия Коваленко
