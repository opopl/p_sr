% vim: keymap=russian-jcukenwin
%%beginhead 
 
%%file 06_04_2021.fb.stavnijchuk_marina.1.mova_jazyk
%%parent 06_04_2021
 
%%url https://www.facebook.com/permalink.php?story_fbid=1724217091097912&id=100005289129370
 
%%author 
%%author_id 
%%author_url 
 
%%tags 
%%title 
 
%%endhead 

\subsection{Наш Руський Язик}
\Purl{https://www.facebook.com/permalink.php?story_fbid=1724217091097912&id=100005289129370}

Вже кілька днів не завершуються дискусії щодо « нашого руського язика».... 

Зауважу.  Не треба Авакову , Мендель і Ко займатися дурницями щодо
нав‘язування суспільству «нашего русского языка» Досить втягувати Україну  у
дурні внутрішні і зовнішні дискусії «не тільки  щодо борщу», який єдиний може
об‘єднати країну, як це роблять МЗС  і Мінкультури , а й щодо мови на вже
новому конфліктному витку, завдяки надалекій прессекретарці президента.  Знову
і знову , розділяючи суспільство. Підіймаючи на поверхню  вкотре Ніцой та Ко з
іншого боку.  Даючи їм безкінечно грати на людських емоціях цим крайнім , не
стриманим ,щоб не сказати більше ... так званим діячам, які підживлюють і без
того роздираючу нас, суспільну конфліктність до рівня кипіння. 

Є, як мінімум, положення статті  10, 11 Конституції України, «дух і букву »
яких варто неухильно дотримуватися в законодавстві та у практиці щодо мови,
щодо освіти в нашій державі!  Там чітко, на основі міжнародних стандартів
збалансовані мовно-культурні права громадян і міжнародні зобов‘язання України. 

Коли вже ті неуки при владі, навчаться  конституційній культурі! Скільки б
напруги було б знято в нашому суспільстві.

