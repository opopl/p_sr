%%beginhead 
 
%%file 23_01_2023.fb.fb_group.story_kiev_ua.1.a_vi_znali__shcho_v_
%%parent 23_01_2023
 
%%url https://www.facebook.com/groups/story.kiev.ua/posts/2122522904611166
 
%%author_id fb_group.story_kiev_ua,mak_poznyak
%%date 23_01_2023
 
%%tags kiev,istoria,kazaki
%%title А ви знали, що в столиці Австрії встановлено два пам'ятники на честь запорозьких козаків?
 
%%endhead 

\subsection{А ви знали, що в столиці Австрії встановлено два пам'ятники на честь запорозьких козаків?}
\label{sec:23_01_2023.fb.fb_group.story_kiev_ua.1.a_vi_znali__shcho_v_}
 
\Purl{https://www.facebook.com/groups/story.kiev.ua/posts/2122522904611166}
\ifcmt
 author_begin
   author_id fb_group.story_kiev_ua,mak_poznyak
 author_end
\fi

\textbf{ВІДНЮ НА ЗГАДКУ ВІД КИЄВА.}

А ви знали, що в столиці Австрії встановлено два пам'ятники на честь
запорозьких козаків? Обидва створені видатними київськими майстрами. Обидва
нагадують віденцям про яскраву перемогу у важливій битві, яка не обійшлася без
активної участі козаків. В одній зі скульптурних композицій присутня постать
уродженця Київщини.

Про створення пам'ятників київськими майстрами розповім далі, а для початку
трохи цікавої історії.

▫️📖 У 1683 році султан Мехмед IV вирішив повалити Габсбурзьку (австрійську)
монархію і підкорити собі християнську Європу. Незлічене військо османів під
супроводом Великого візира Кари-Мустафи захопило угорські міста і прямувало
далі. Найбажанішою ціллю для Оттоманської імперії було найбагатше і
найпрестижніше європейське місто, колиска Габсбургів — Відень, який називали
\enquote{Золотим яблуком Європи}.

▫️ Кара-Мустафа був настільки впевнений у власній непереможності, що діставшись
Відня і взявши його в облогу, він не поспішав, смакуючи кожен звитяжний момент.
150-тисячна армія османів все міцніше стискала кільце навколо європейської
столиці. Окрім того, в місті спалахнула епідемія. Ситуація була критичною,
Відень знаходився на межі падіння до ніг Мехмеда IV.

На допомогу союзникам-християнам поспішив польський король і видатний
полководець Ян ІІІ Собеський (нині національний герой Польщі). Зволікання могло
мати критичні наслідки, тому він не став чекати підкріплення литовських полків
і зібрав усіх доступних воїв — близько 25 тисяч душ, в тому числі винайняв на
українських землях 5000 запорозьких козаків.

▫️ Швидкі та рішучі дії війська Яна Собеського захопили зненацька ворога, який
кількісно переважав сили європейських (австро-німецько-польсько-українських)
союзників вдвічі, а за деякими даними — навіть втричі. Одну з ключових ролей у
важливій битві відіграли саме козаки — вони нахабно увірвались в саме серце
величезної османської армії і посіяли паніку. Потім козаки гнали загарбників до
самого Чорного моря, здобувши багато коштовних трофеїв.

Король Ян Собеський так писав про запорожців у тій битві:

✍️ \emph{«А вони (козаки) пішли так швидко, відважно і мужньо, що під димом опанували
зараз не тільки передмістя і стодоли, але й першу палісаду і браму та заткнули
в ній свої прапори з хрестами — і в усіх заслужили собі на велику славу».}

✍️ \emph{«...Турків зараз ведуть, як псів і худобу, їх моя драгунія і козаки немало
забрали».}

▫️ А свідок тих подій, польський письменник Ян Хризостом Пасек ділився у своїх
мемуарах враженнями:

✍️ \emph{«Яка то красива винахідливість і майстерність козацька!»}

▫️ Віденська перемога стала для запорожців не єдиною, але найважливішою у першій
війні цілої низки війн «Священної ліги», що точилися між Османською Імперією та
коаліцією християнських держав із 1683 по 1699 рік.

⬜️🎁 РОБОТИ КИЇВСЬКИХ МАЙСТРІВ.


\textbf{№1.}

Перший козацький пам'ятник (клік ФОТО №1) віденській громаді у 2003 році
подарувала Національна спілка художників України (адреса спілки: м.Київ, вул.
Січових Стрільців 1-5). Подарунок був приурочений до 320 річниці визволення
Відня від османської облоги, а також до 12 річниці Незалежності України.

\textbf{▫️ Автори.}

Працювати над скульптурою взявся сам голова Національної спілки художників
України — корінний киянин, професор Національної академії образотворчого
мистецтва і архітектури, народний художник України і лауреат Національної
премії ім. Шевченка — Володимир Чепелик (1945—2021). Йому допомагав київський
архітектор Володимир Скульський. На той момент обидва Володимири мали у
власному доробку немало робіт, зокрема, добре відомих киянам. Руки скульптора
Чепелика доклалися до пам'ятників Михайлу Грушевському та Олесю Гончару,
меморіальних комплексів Жертвам Чорнобильської катастрофи та Жертвам репресій.
Архітектор Скульський брав участь у створенні пам'ятників Бикову, Городецькому,
Паніковському, Голохвастову.

\textbf{▫️ Локація.}

Присвячений козакам меморіал від українських майстрів — встановили у
мальовничому віденському парку Тюркеншанцпарк, який було засновано імператором
ще у 1888 році. На тому самому місці турки зводили укріплення під час битви за
Відень у 1683 році.

\textbf{▫️ Композиція.}

Пам'ятник містить. в собі три елементи: козака, який сидить у філософській позі
й палить люльку, коня, що пасеться позаду, і камінь поруч з козаком, на якому
написано:

\emph{«Українське козацьке військо зробило важливий внесок у звільнення Відня від
турецької облоги 12 вересня 1683 р. У вирішальній битві польський король Ян III
Собеський, головнокомандувач армією союзників, покладав великі надії на
допомогу козаків. Очевидці говорили про їх надзвичайну мужність та велику
хоробрість. 2 лютого 1684 року в Римі відбулося святкове богослужіння в
присутності Папи Римського, щоб подякувати «українському козацтву, яке
допомогло перемозі над турками і татарами».}

\textbf{№2.}

Інший пам'ятник нашим пращурам-воїнам (клік ФОТО№2) встановили у Відні в 2013
році за ініціативи української діаспори та при всебічній підтримці віденського
мера Міхаеля Хойпля. Приводом стала 330 річниця \enquote{Віденської відсічі}.

\textbf{▫️ Автори.}

На цей раз київський скульптор Володимир Чепелик творив пам'ятник козакам у
парі зі своїм сином Олексієм Чепеликом (1971 р.н.). До того батько й син вже
працювали разом — над відомим пам'ятником актору Миколі Яковченко біля театру
ім. Івана Франка у Києві.

\textbf{▫️ Локація.}

Другу роботу українських майстрів встановили у північно-східній частині
Віденського лісу на горі Леопольдсберг.

\textbf{▫️ Композиція.}

Пам'ятник — це тріо озброєних козаків у людський зріст: сотник Яків Потапенко,
полковник Павло Апостол-Щуровський (уродженець Брацлавщини) і сотник Чомуха
(уродженець Київщини). Воїни споглядають з гори на Відень, під яким вони
зупинили османську навалу. Напис на пам'ятнику викарбувано двома мовами —
німецькою та українською: \emph{«1683. Присвячується українським козакам — захисникам
Відня».}

⬜️ℹ️ ДОДАТКОВІ ЦІКАВИНКИ.

✅ Неподалік від пам'ятника трійці козаків на горі Леопольдсберг — знаходиться
красива церква Святого Леопольда. На будівлі церкви ще у 1983 році відкрили
меморіальну дошку на козацьку честь до 300-річчя Віденської перемоги.

✅ Перемога у Віденській битві стала можливою завдяки ще одному українцю — Юрію
Кульчицькому, який мешкав і працював у австрійському місті, був
перекладачем-поліглотом і добре знав османську мову та звичаї. Під час облоги
Відня він перевдягнувся османом, вправно пробрався через ворожий табір і
передав важливу звістку польському королю Яну ІІІ Собеському. Після перемоги
Кульчицький отримав як винагороду османські кавові зерна, винайшов рецепт
знаменитої віденської кави, відкрив у Відні власну кав'ярню та навчив Європу
пити каву. Пам'ятник Юрію Кульчицькому теж присутній в австрійській столиці.

✅ Останніми роками доброю традицією стало святкування історичної звитяги
польською та українською громадами сумісно. Це дуже символічно на тлі
сьогоднішніх подій.

© Maks Poznyak 
