% vim: keymap=russian-jcukenwin
%%beginhead 
 
%%file 06_03_2022.fb.hozhainova_vika.1.v_ukraini_my_vdoma_usjudy
%%parent 06_03_2022
 
%%url https://www.facebook.com/vika.hozh/posts/1782253738646992
 
%%author_id hozhainova_vika
%%date 
 
%%tags __feb_2022.vtorzhenie
%%title В Україні - ми вдома усюди
 
%%endhead 
 
\subsection{В Україні - ми вдома усюди}
\label{sec:06_03_2022.fb.hozhainova_vika.1.v_ukraini_my_vdoma_usjudy}
 
\Purl{https://www.facebook.com/vika.hozh/posts/1782253738646992}
\ifcmt
 author_begin
   author_id hozhainova_vika
 author_end
\fi

Перші пару років життя у Києві я іноді брехала, звідки я родом. Казала, що я з
Києва або інших областей. Адже за відповіддю \enquote{Я з Луганська} йшло дуже багато
неприємних запитань: а де твої батьки? а як там зараз? а звідки ти знаєш
українську? тощо.

Зараз бачу, що багатьом не подобається, коли їх називають біженцями. Адже це
слово має негативне забарвлення та асоціації. Біженкою мене жодного разу не
назвали і я б за таке вбила, але переселенкою називали. Вибачте, переселенка?
Яка ж я переселенка? Україна - моя країна. Я можу жити у ній, де захочу, від
Донбасу до Закарпаття. У будь-якому куточку. Можу ці куточки змінювати протягом
життя. Що це за слово таке?

Навіть якби я поїхала за кордон, то була б емігранткою. Для усіх існують
нормальні слова, крім тих, хто переїхав через війну (хоча я - не через неї).
Біженці та переселенці. Дякую, дуже приємно. Наче для того, аби зберегти якусь
людську гідність, ти маєш залишитися на території, де йде війна. Навіть якщо не
хочеш з багатьох інших причин.

Я чесно не хотіла б, щоб емпатія досягалася такою ціною. Але плюс у тому, що
розуміння тепер стане більше, тепер всі рівні і від того - згуртованіші, зліші,
сильніші.

А біженцями і переселенцями нас з вами називати не можна. Я вірю, що не
підпадають під такі слова освічені люди, які володіють мовами, професіями,
мають певні соціальні зв'язки. Ми з вами, жоден з нас, тих, хто поїхав, не
переселенці і точно не біженці. За кордоном - ми тимчасово у гостях перед тим,
як повернутися додому. В Україні - ми вдома усюди.
