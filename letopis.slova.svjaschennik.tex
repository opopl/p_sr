% vim: keymap=russian-jcukenwin
%%beginhead 
 
%%file slova.svjaschennik
%%parent slova
 
%%url 
 
%%author_id 
%%date 
 
%%tags 
%%title 
 
%%endhead 
\chapter{Священник}

%%%cit
%%%cit_head
%%%cit_pic
\ifcmt
  tab_begin cols=3
     pic https://gdb.rferl.org/50F8A760-B3F8-438C-81D3-CE307F9914AE_w650_r0_s.jpg
     pic https://gdb.rferl.org/76E44E75-358F-48C7-AAB5-DBEE63C8B6E0_w650_r0_s.jpg
		 pic https://gdb.rferl.org/9AD38A44-F1E2-45DE-9C3E-AF7C780FB405_w650_r0_s.jpg
  tab_end
\fi
%%%cit_text
У 1775 році Достоєвські вимушені були покинути Волинь. Вони продали свій маєток
Кличковичі (це невелике село неподалік Ковеля).  Тоді прадід Федора
Достоєвського, Григорій Гомерович, перебрався на Східну Волинь, до Янушполя, де
став греко-католицьким (за тодішньою термінологією уніатським) \emph{священиком}. Два
його сини теж були \emph{священниками}. Один із них, Андрій, дід письменника, служив у
селі Війтівці на Поділлі в 1782–1820 роках.  Йому довелося перейти з унії на
православ’я. Його син Лев також служив у Війтівцях \emph{священником}. Інший син,
Михайло, батько письменника, готувався до \emph{священицької} кар’єри, навчався в
Подільській семінарії в Шаргороді. Звідти його як одного з найкращих учнів
відправили вчитися в Медико-хірургічну академію в Москву, де він осів. У Москві
й народився Федір Достоєвський. Хоча батько Федора Достоєвського опинився в
Москві, але він формувався в лоні традиційної культури українського
\emph{духовенства}. І цю культуру, принаймні почасти, він передав своєму синові. Чи не
звідти глибинна релігійність Федора Достоєвського, яка мало співзвучна з
релігійністю російською, але яка вписується в контекст релігійних традицій
України XVI–XVIII століть?
%%%cit_comment
%%%cit_title
\citTitle{200-літній ювілей. Українські корені Федора Достоєвського та «елліністична поезія» України}, 
Петро Кралюк, www.radiosvoboda.org, 11.11.2021
%%%endcit
