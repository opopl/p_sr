% vim: keymap=russian-jcukenwin
%%beginhead 
 
%%file 22_06_2020.fb.naumchuk_tanya.1.tabor
%%parent 22_06_2020
 
%%url https://www.facebook.com/manya.naumchuk/posts/671942976997260
%%author Наумчук Таня
%%author_id naumchuk_tanya
%%title Табор
%%tags poetry
 
%%endhead 

\subsection{ТАБОР}
\label{sec:22_06_2020.fb.naumchuk_tanya.1.tabor}
\Purl{https://www.facebook.com/manya.naumchuk/posts/671942976997260}
\Pauthor{Наумчук, Таня!Табор}

\ifcmt
pic https://scontent.fiev6-1.fna.fbcdn.net/v/t1.0-9/104992151_671942956997262_2850872399438722402_n.jpg?_nc_cat=111&ccb=2&_nc_sid=8bfeb9&_nc_ohc=T8ELFGWKZf0AX9W_vRL&_nc_ht=scontent.fiev6-1.fna&oh=2ffb60d7ae2b03b1249b1302792d316f&oe=5FDFF8C8
\fi

\begin{multicols}{2}
	\obeycr
Устало плетется лошадка в обозе,
Тоскливо кибитка скрипит.
Покрыта седушка плетёной рогожей
И слышен стук крепких копыт.

Под тенью кибитки не так уж прохладно,
Жара забралась и туда.
А вечер наступит, что сердцу отрадно,
И станет обоз у пруда.

Костер запылал над широкой рекою,
Вверх искры, згарая, летят
И мальчик пытается просто рукою
Поймать их раз десять подряд.

Уже в котелке закипела похлёбка,
Нарезан ломтями хлеб черный, ржаной,
Налита вина виноградного стопка,
И каждому так по одной.

По струнам гитары прошлись быстро пальцы:
"Ой, нэнэ! Ой, нэнэ"-- протяжно звучит...
А кто-то уже пошел в речку купаться,
И плещется там и кричит.

Печальную песню цыганка запела,
Качаясь то взад, то вперёд,
Быть может она в ней сказать захотела
Какой есть цыганский народ.

Детишек цыгане в кибитки загнали,
Пора уже спать им давно,
Хоть сами в дороге изрядно устали,
Но спать не хотят все равно.

Их говор становиться тише и реже...
Все слушает песню в ночи.
Лишь рядом лягушки почуется скрежет,
Иль где-то сова закричит.

Гитара умолкла, костер затухает,
Устал кочевой и весёлый народ.
Продолжит обоз, только лишь рассветает,
Свой трудный и долгий поход.

(22.06.2020). (Наумчук Таня).
	\restorecr
\end{multicols}
