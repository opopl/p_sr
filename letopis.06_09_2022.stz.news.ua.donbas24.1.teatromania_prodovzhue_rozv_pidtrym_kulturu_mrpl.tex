% vim: keymap=russian-jcukenwin
%%beginhead 
 
%%file 06_09_2022.stz.news.ua.donbas24.1.teatromania_prodovzhue_rozv_pidtrym_kulturu_mrpl
%%parent 06_09_2022
 
%%url https://donbas24.news/news/teatromaniya-prodovzuje-rozvivati-ta-pidtrimuvati-kulturu-mariupolya
 
%%author_id demidko_olga.mariupol,news.ua.donbas24
%%date 
 
%%tags 
%%title "Театроманія" продовжує розвивати та підтримувати культуру Маріуполя
 
%%endhead 
 
\subsection{\enquote{Театроманія} продовжує розвивати та підтримувати культуру Маріуполя}
\label{sec:06_09_2022.stz.news.ua.donbas24.1.teatromania_prodovzhue_rozv_pidtrym_kulturu_mrpl}
 
\Purl{https://donbas24.news/news/teatromaniya-prodovzuje-rozvivati-ta-pidtrimuvati-kulturu-mariupolya}
\ifcmt
 author_begin
   author_id demidko_olga.mariupol,news.ua.donbas24
 author_end
\fi

\ii{06_09_2022.stz.news.ua.donbas24.1.teatromania_prodovzhue_rozv_pidtrym_kulturu_mrpl.pic.front}

\begin{center}
  \em\color{blue}\bfseries\Large
За декілька місяців Народний театр \enquote{Театроманія} зміг відродити та
примножити свою діяльність в Німеччині
\end{center}

2 місяці тому засновник і режисер Народного театру \enquote{Театроманія} \href{https://www.facebook.com/profile.php?id=100001584004558}{%
Антон Тельбізов}%
\footnote{\url{https://www.facebook.com/profile.php?id=100001584004558}} відновив діяльність театру в німецькому місті Ганновер. Тепер до
колективу долучаються українські актори, які наразі перебувають в Німеччині.
Відтепер \enquote{Театроманія} з успіхом представляє свої вистави не тільки
українським, а й зарубіжним глядачам. Зокрема, на День Незалежності України під
керівництвом режисера Антона Тельбізова Народний театр \enquote{Театроманія} в
німецькому місті Ганновер показав виставу \emph{\enquote{Тіні забутих предків}}. Її
представили у зруйнованій церкві, що дуже символічно, адже рік тому спектакль
був вперше показаний у зруйнованій синагозі в Маріуполі. Будівля церкви дуже
схожа на маріупольську синагогу. Вистава відбулася при підтримці та
фінансуванні Української спілки Нижньої Саксонії в місті Ганновері.

% 1 - Антон Тельбізов у Ганновері навчає нових акторів "Театроманії"
\ii{06_09_2022.stz.news.ua.donbas24.1.teatromania_prodovzhue_rozv_pidtrym_kulturu_mrpl.pic.1}

Через те, що не всім акторам вдалося виїхати до Ганновера, були достатньо
складні вводи у виставу. Над виставою працювало 18 осіб, з них 14 — акторів, 9
— з Маріуполя. Режисерка \enquote{Театроманії} та актриса \href{https://www.facebook.com/profile.php?id=100001752890543}{%
Ольга Самойлова}%
\footnote{\url{https://www.facebook.com/profile.php?id=100001752890543}} %
зіграла маму
Івана. Самого ж Івана зіграв актор \textbf{Радислав Пономаренко}, який приїхав з Києва.
Він наразі мешкає у Варшаві. Хлопцю дуже швидко вдалося увійти в колектив та
впоратися зі складною роллю. Ще один киянин \textbf{Станіслав Міщук} блискуче зіграв
старого вівчаря. Колектив зібрався за 10 днів і репетирував у євангельській
церкві. \textbf{Марія Бойко}, яка уособлювала Землю, була голосом вистави. Її щирий спів
приємно вразив глядачів. Якщо в Маріуполі мікрофон був лише у Марії, то в
Ганновері кожен актор мав свій петличний мікрофон, що зробило звук ще більш
потужним та чітким і дозволило якісніше передати дійство для глядачів, які
дивилися виставу в онлайн форматі. Загалом використання мікрофонів на сцені
стало дуже цікавим досвідом для всього колективу.

\ii{06_09_2022.stz.news.ua.donbas24.1.teatromania_prodovzhue_rozv_pidtrym_kulturu_mrpl.pic.3}
\ii{06_09_2022.stz.news.ua.donbas24.1.teatromania_prodovzhue_rozv_pidtrym_kulturu_mrpl.pic.4}

Перед виставою \enquote{Тіні забутих предків} був ярмарок та пройшов невеличкий
концерт. Всі гроші, що були зібрані цього дня, спрямовувалися у Фонд
маріупольських дітей, які постраждали від російської агресії та втратили
батьків.

Переважна більшість глядачів були українцями, які прийшли у вишиванках. Проте
спектакль дивилися й німці та іноземні гості. Колектив підготував лібрето
англійською та німецькою мовами, завдяки якому зарубіжні глядачі могли
ознайомитися зі змістом вистави. 

% 2 - Сцена в Ганновері, де театр представив виставу «Тіні забутих предків»
\ii{06_09_2022.stz.news.ua.donbas24.1.teatromania_prodovzhue_rozv_pidtrym_kulturu_mrpl.pic.2}

\ii{06_09_2022.stz.news.ua.donbas24.1.teatromania_prodovzhue_rozv_pidtrym_kulturu_mrpl.pic.5}

\begin{leftbar}
  \begingroup
\emph{\enquote{Сама вистава пройшла на одному диханні. Панувала дуже особлива
атмосфера. Багато іноземних глядачів наголосили, що це справжнє сучасне
мистецтво, що вони нічого подібного не бачили. А українці щиро дякували
та підкреслювали, що це найкращий День незалежності в їх житті}}, —
розповіла Ольга Самойлова.
   \endgroup
\end{leftbar}

До речі, Ользі Самойловій в німецькому місті Білефельд, де вона наразі мешкає
разом із сім'єю, запропонували взяти участь в окремому театральному проєкті з
українськими дітьми.

\begin{leftbar}
  \begingroup
\emph{\enquote{Я завжди мріяв виступити в Європі, але не думав, що це буде при таких
обставинах. Насправді я відчував гордість. За Маріуполь, за країну, за нашу
націю. Я хочу всюди розповідати, що Маріуполь — це велике українське місто з
красивими, працездатними та успішними людьми. Цією виставою ми довели, що ми
маємо багату історію, свою унікальну культуру, що ми гідні того, щоб про нас
знали в усьому світі}}, — наголосив Антон Тельбізов.
   \endgroup
\end{leftbar}

\ii{06_09_2022.stz.news.ua.donbas24.1.teatromania_prodovzhue_rozv_pidtrym_kulturu_mrpl.pic.6}
\ii{06_09_2022.stz.news.ua.donbas24.1.teatromania_prodovzhue_rozv_pidtrym_kulturu_mrpl.pic.7}

Наразі колектив \enquote{Театроманії} готує лялькову казку \enquote{Сусідка}. У виставі
виступить двоє акторів Марія Бойко та її молодший брат Арсеній. Репетиції
відбуваються онлайн. Ця казка буде поставлена маріупольськими акторами для
маріупольських дітей.

\begin{leftbar}
  \begingroup
\emph{\enquote{Ми хочемо, щоб юні маріупольці зрозуміли, що попри те, що ти все
втратив, завжди можна почати все спочатку}}, — підкреслив Антон
Тельбізов. 
   \endgroup
\end{leftbar}

\ii{06_09_2022.stz.news.ua.donbas24.1.teatromania_prodovzhue_rozv_pidtrym_kulturu_mrpl.pic.8_9}

Цю казку представлять в Дніпрі та інших українських містах. Наприкінці вистави
діти отримають розмальовки від \href{https://www.facebook.com/profile.php?id=100000576599040}{Анастасії Пономарьової}%
\footnote{\url{https://www.facebook.com/profile.php?id=100000576599040}}
\enquote{Розмалюй своє місто}, де
будуть представлені яскраві архітектурні споруди Маріуполя, зокрема можна буде
знайти і яскраві двері театру з написом \enquote{Театроманія} і приміщення ПК
\enquote{Молодіжного}, де і виступав колектив з 2011 року. Після представлення вистави
в Україні її повезуть в Ганновер, де покажуть дітям з різних міст України.
Наразі в Ганновері зареєстровано 14 тисяч українських дітей. Проєкт створено в
межах програми для молоді UPSHIFT, що здійснюється Дитячим фондом ООН (ЮНІСЕФ)

До Нового Року Українська спілка Нижньої Саксонії в м. Ганновері запропонувала
\enquote{Театроманії} поставити вертеп, адже україн\hyp{}ської культури в Німеччині дуже не
вистачає. Антон Тельбізов вже познайомився з Хором українського співу в
Ганновері і має декілька унікальних ідей, які допоможуть зробити вертеп
яскравим та незабутнім для глядачів.

\begin{leftbar}
  \begingroup
\emph{\enquote{Це буде схоже на український Аватар. Я хочу витримати всі традиції і при цьому
зробити дійство насиченим та неповторним. Так біблійні сюжети спробуємо
зобразити за допомогою графіки. Вже шукаю художників}}, — поділився
Антон Тельбізов. 
   \endgroup
\end{leftbar}

Кожною своєю виставою та всією творчістю колектив \enquote{Театроманії} готовий і
надалі підтримувати розвиток культури Маріуполя і загалом України.

\ii{06_09_2022.stz.news.ua.donbas24.1.teatromania_prodovzhue_rozv_pidtrym_kulturu_mrpl.pic.10}

Нагадаємо, раніше Донбас24 розповідав, що \href{https://donbas24.news/news/bum-sinyo-zovtix-kolyoriv-novii-trend-svitovoyi-modi-ta-kulturi-foto}{синьо-жовті кольори України стали
трендом світової моди та культури}.%
\footnote{Синьо-жовті кольори України стали трендом світової моди та культури, Яна Іванова, donbas24.news, %
05.09.2022, \par\url{https://donbas24.news/news/bum-sinyo-zovtix-kolyoriv-novii-trend-svitovoyi-modi-ta-kulturi-foto}
}

\emph{Ще більше новин та найактуальніша інформація про Донецьку та Луганську області
в нашому телеграм-каналі Донбас24.}

ФОТО: Антона Тельбізова.

\ii{insert.author.demidko_olga}
%\ii{06_09_2022.stz.news.ua.donbas24.1.teatromania_prodovzhue_rozv_pidtrym_kulturu_mrpl.txt}
