%%beginhead 
 
%%file 06_12_2020.fb.fb_group.mariupol.nekropol.1.mezhd_den_volontera_pozdravlenia
%%parent 06_12_2020
 
%%url https://www.facebook.com/groups/278185963354519/posts/426554405184340
 
%%author_id fb_group.mariupol.nekropol,arximisto
%%date 06_12_2020
 
%%tags 
%%title Мы поздравляем всех волонтеров Мариупольского Некрополя!
 
%%endhead 

\subsection{Мы поздравляем всех волонтеров Мариупольского Некрополя!}
\label{sec:06_12_2020.fb.fb_group.mariupol.nekropol.1.mezhd_den_volontera_pozdravlenia}
 
\Purl{https://www.facebook.com/groups/278185963354519/posts/426554405184340}
\ifcmt
 author_begin
   author_id fb_group.mariupol.nekropol,arximisto
 author_end
\fi

Сегодня праздник – \textbf{Международный день волонтеров}. День людей, которые жертвуют
свое время и труд ради общего блага.

\textbf{Мы поздравляем всех волонтеров Мариупольского Некрополя!} Благодаря вашим
самоотверженным усилиям всеми забытое Старое кладбище превращается в настоящий
символ любви и уважения к создателям города.

Вы – \enquote{делатели}. Когда все вокруг болтают о сохранении исторического
наследия, вы его восстанавливаете собственными руками. Не на словах, а на деле.

Вы – творцы истории Мариуполя. Ибо возвращаете из забвения людей, которые его
создали.

Наконец, вы – \enquote{настоящие буйные}, как пел когда-то Владимир Высоцкий.

Ибо не ждете \enquote{манны небесной}, а действуете!

Честь вам и хвала!

До встречи, друзья!

p.s. И – подборка фотографий и видео, уже ставших историческими... 🙂

\begin{itemize} % {
\item Восстановление памятника итальянских купцов Галлеано%
\footnote{Internet Archive: \url{https://archive.org/details/video.16_07_2020.arximisto.vosstanovlenie_pamjatnik_semji_galleano}}

\item Монтаж света на кресте усыпальницы Найденовых% 
\footnote{Internet Archive: \url{https://archive.org/details/video.08_06_2020.arximist.montazh_sveta_krest_usypalnica_najdenovyh}}

\item Репортаж \enquote{Суспільне. Донбас}%
\footnote{Internet Archive: \url{https://archive.org/details/video.15_10_2020.suspilne_donbas.230_rokiv_mrpl_volontery_arximisto}}

\end{itemize} % }
