% vim: keymap=russian-jcukenwin
%%beginhead 
 
%%file slova.obrazovanie.osvita
%%parent slova
 
%%url 
 
%%author 
%%author_id 
%%author_url 
 
%%tags 
%%title 
 
%%endhead 
\chapter{Образование (Освіта)}
\label{sec:slova.obrazovanie.osvita}

%%%cit
%%%cit_head
%%%cit_pic
%%%cit_text
... Голодомор також доволі системне явище.  Смотри, я говорю о положительных
моментах. С точки зрения \emph{образования}, \emph{образование} было возможно не самым
лучшим, но он было системным. Оно давало такой комплекс знаний, который
позволял человеку в какой-то системе координат существовать. Сегодняшнее
украинское \emph{образование} оно абсолютно бессистемное, оно какой-то хаотичный
набор, эклектика, в которой не отвечают на вопрос, с чем человек идет в мир.
\emph{Советское образование} на этот вопрос отвечало, оно умело сегментировать людей
по каким-то кластерам: вот этот сюда, вот этот сюда. Если у тебя есть хорошие
математические навыки, то тебя будут продвигать в какой-нибудь институт
Баумана, причем в прямом смысле этого продвигать: тебя оценивали в топовых
вузах, исходя из твоих навыков
%%%cit_comment
%%%cit_title
\citTitle{Украина в плену консервативного мышления: почему 30 лет шли не туда и что делать}, 
Сергей Иванов; Юрий Романенко, hvylya.net, 15.07.2021
%%%endcit
