%%beginhead 
 
%%file 28_02_2023.fb.suhorukova_nadia.mariupol.2.mikita
%%parent 28_02_2023
 
%%url https://www.facebook.com/permalink.php?story_fbid=pfbid02J5XwpTe9NTM8U9yhjiiwUPjoKTbiZXSGqVkz42h4gmZbgkxZyzfy7ebTow7c8nmYl&id=100087641497337
 
%%author_id suhorukova_nadia.mariupol
%%date 28_02_2023
 
%%tags mariupol,mariupol.war
%%title Микита
 
%%endhead 

\subsection{Микита}
\label{sec:28_02_2023.fb.suhorukova_nadia.mariupol.2.mikita}

\Purl{https://www.facebook.com/permalink.php?story_fbid=pfbid02J5XwpTe9NTM8U9yhjiiwUPjoKTbiZXSGqVkz42h4gmZbgkxZyzfy7ebTow7c8nmYl&id=100087641497337}
\ifcmt
 author_begin
   author_id suhorukova_nadia.mariupol
 author_end
\fi

Юлька пішла до пологового будинку в середу. \par
І відразу ж зник зв'язок. \par
Набирати її не мало сенсу, і ми з мамою вирішили відвідати Юлину маму, мою подругу. \par
Завтра. \par
Це було якесь безглузде слово.\par
Завтра могло й не настати. Але ми з мамою вперто планували наступний день.\par
Тоді ми ховались від обстрілів у тамбурі.  \par
Дві стіни трохи приглушували пекельні звуки. \par
Десь біля ліфта кричали діти: вони не встигли спуститися на перший поверх і тому сховалися на четвертому поверсі. \par
Я обіймала себе й хиталася на табуретці. \par
Добре, що ніхто не бачив: у тамбурі хоч в око стрель. \par
Темрява взагалі стала нашою постійною супутницею.\par
Темно було і вдень. \par
Небо в  диму не пропускало сонця, а вікна ми заклали подушками та завісили ковдрами. \par
Коли прийшли морози й випав сніг, виявилося, що нам нічим укриватися, а в квартирі було страшенно холодно. \par
Довелося познімати ковдри.\par
Через кілька днів вибуховою хвилею вибило шибки на лоджії, а двері на балконі викривило так, що вони більше не зачинялися.\par
Мама весь час пропонувала кудись піти, когось провідати. \par
Спочатку ми відвели  Енджі до знайомих.\par
Вигулювати її коло нас було небезпечно. \par
Енджі не розуміла, що відбувається, і намагалась утекти якнайдалі від під'їзду, аби знайти чисту землю -  без смиття  та скла.\par
Вона піднімала лапи й дивилася на мене. Їй не було куди ступити.\par
Потім ми пішли провідати мою подругу в дев'ятиповерхівці поряд. \par
Мамі було важливо дізнатися, як почувається її дочка і кого вона народила.\par
Юлька народила Микиту. \par
У неї був кесарів розтин другого березня.\par
Четвертого вона пішла з пологового будинку. \par
Шостого прийшла з немовлям у дім наших знайомих.\par
На її сьомому поверсі було занадто страшно.\par
Юлина сестра, мама, бабуся і крихітний Микита спали в коридорі на підлозі. \par
Воду для суміші розігрівали на сухому пальному: маленький принц не міг їсти суміш, якщо її розбавляли водою, нагрітою на багатті.\par
У Юльки дуже болів шов. \par
Ми обробляли його перекисом. Я казала їй: \par
\enquote{Навіщо ти пішла з пологового будинку? Там же професіонали. Ти дуже рано виписалася}\par
Юлька спочатку відповідала, що хвилювалася за своїх, а потім сказала:\par
\enquote{Там надто великі вікна. Мені було дуже страшно}\par
За кілька днів, дев'ятого березня, рyzкий льотчик скинув на пологовий будинок авіабомбу. \par
А там залишалися жінки, які не могли поїхати додому. \par
Одна з них мешкала в селі під Маріуполем. \par
Її навіть ніхто не відвідував. Вона народила дівчинку.\par
Медперсонал годував породіллю зі своїх запасів.\par
Крихітний Микита практично жив у підвалі. \par
Іноді його виносили на поверхню, і він розглядав усе довкола винятково дорослим поглядом.\par
Микитку всі  любили. \par
Жінки носили малого на руках і заколисували. \par
Юлька зітхала: вона хотіла, щоб синочок був самостійнішим.\par
Я пам'ятаю його першу прогулянку.\par
З комбінезона стирчить лише ніс.\par
Малюк  вдихає повітря, що пахне порохом. \par
Олена, моя подруга й бабуся Микити, тримає його на руках, немовля спить, і тут лунає скрегіт заліза, ніби хтось тягне величезний  каркас...\par
Ми знали, що одразу після цього стрілятимуть. \par
Перша прогулянка Микити тривала п'ять хвилин.\par
Мабуть, він вважає, що народився у похмурому й немилосердному світі. \par
Микита постійно був  в темряві.\par
У підвалі світить тільки  ліхтарик;\par
у кімнаті на першому поверсі, куди його приносять, щоб обробити пупок, теж напівтемно:\par
вікна затулені мішками з піском.\par
Тато Микити воює за Україну, тому ми називали малюка сином полку. \par
Юлька іноді тікає дзвонити чоловікові зі сходів будинку №105. \par
Коли з'являється рідкісний зв'язок. \par
У цей час завжди обстріл. \par
Ми вмовляємо Юльку нікуди не ходити, але вона дуже хоче почути голос чоловіка.\par
Микита, здається, передчуває обстріли.\par
За кілька хвилин до  нього  -   ридає немовлячим басом, а потім одразу починають стріляти. \par
Микита нас попереджає.\par
Ми лягаємо на диван, накриваємо голови подушками і слухаємо гуркіт літака.\par
***\par
Цьому красеню через два дні виповниться рік.\par
Його чудова бабуся зуміла врятувати.\par
Вона вивезла Микиту та свою родину  із мертвого міста 15 березня.\par
Вони їхали під обстрілами.\par
Малюкові тоді  було менше двох тижнів.\par
