% vim: keymap=russian-jcukenwin
%%beginhead 
 
%%file 19_05_2021.stz.news.ua.mrpl_city.1.novyj_art_prostir_mrpl
%%parent 19_05_2021
 
%%url https://mrpl.city/blogs/view/novij-art-prostir-mariupolya
 
%%author_id demidko_olga.mariupol,news.ua.mrpl_city
%%date 
 
%%tags 
%%title Новий арт-простір Маріуполя
 
%%endhead 
 
\subsection{Новий арт-простір Маріуполя}
\label{sec:19_05_2021.stz.news.ua.mrpl_city.1.novyj_art_prostir_mrpl}
 
\Purl{https://mrpl.city/blogs/view/novij-art-prostir-mariupolya}
\ifcmt
 author_begin
   author_id demidko_olga.mariupol,news.ua.mrpl_city
 author_end
\fi

\ii{19_05_2021.stz.news.ua.mrpl_city.1.novyj_art_prostir_mrpl.pic.1}

Нещодавно відкрила для себе новий цікавий арт-простір \enquote{Міський Будда}, який
працює в Маріуполі вже цілий рік. Вважаю, що він заслуговує на особливу увагу і
стане цікавим всім творчим, креативним та активним містянам.

Арт-простір відкрився рік тому \emph{1 червня 2020 року}. Засновницею стала
маріупольська громадська діячка і в минулому організаторка святкових заходів
\emph{\textbf{Ксенія Гузь}}. Співзасновником простору є чоловік Ксенії маріупольський
громадський діяч і тренер з йоги \emph{\textbf{Максим Свєтлов}}. Центр розмістився у
старовинному будинку кінця XIX століття на вулиці Куїнджі і включає в себе
безліч різних напрямів. Те, що людям сподобалося найбільше обов'язково
повторюється. Зокрема, неодноразово були представлені віршовані батли, стендапи
та духовні зустрічі (медитації). Частими є й такі заходи, як майстер-класи,
тренінги, квести, пікніки. Щосуботи та щонеділі проводяться нові заходи. Максим
Свєтлов наголошує, що вони цілком і повністю підтримують слова Далай-лами:
\emph{\enquote{якщо діти з 8 років почнуть медитувати, припиняться всі війни}}. 

\ii{19_05_2021.stz.news.ua.mrpl_city.1.novyj_art_prostir_mrpl.pic.2}

Засновники
арт-простору вирішили спробувати поєднати творчість і духовність. Проте Максим
підкреслює, що вони не є релігійною структурою. На тренінгах в ігровій формі
організатори дуже легко і невимушено можуть говорити на досить глибинні теми:
звідки береться свідомість, як вона керує нами, що нам допомагає залишатися в
позитивному настрої і, що відбувається коли починають керувати негативні
емоції... Загалом на тренінгах піднімаються важливі психологічні теми, які
допомагають знайти відповіді на життєво-значущі питання. За основу Ксенія і
Максим взяли Будду як образ, тому що в цій міській метушні людина дуже часто
забуває про себе, їй потрібно зупинитися і послухати тишу. Тому вони і обрали
Будду, який говорив, що \enquote{не варто пірнати глибоко в релігію і не слід йти далеко
в соціальність. Потрібно знайти серединний шлях і бути щасливим}. У цьому
просторі люди можуть і подумати про щось важливе, і зустріти однодумців.

\ii{19_05_2021.stz.news.ua.mrpl_city.1.novyj_art_prostir_mrpl.pic.3}

Дуже цікавими є  театральні заходи простору: різноманітні вистави, перформанси,
поетичні вечори. Торік Ксенія провела для всіх охочих \enquote{Хвилину слави}: хтось
читав вірш, хтось показував танець. Головне, що кожен міг себе проявити. Були
навіть призові місця. Також в центрі відбуваються заходи із задіянням акторів.
Зокрема, актриса Театральної артілі \enquote{Драмком} Юлія Корсун на етновечорі читала
вірш. У червні в арт-просторі відбудеться театралізований захід – поетичний
вірш \enquote{Ода життю}, де вірші, пісні і танці завдяки одній сценарній лінії
перетворяться на яскраве сценарне дійство. Маленьке життя буде прожите за одну
пісню чи за один вірш. Загалом в центрі частими є імпровізації. Тут можна
відчути себе і актором, і режисером водночас. Цікаво, що і в будинку, і на
подвір'ї можна побачити безліч цікавих локацій. Так унікальним є місце тиші, де
можна побачити декілька портретів видатних духовних вчителів. Також подвір'я і
будинок прикрашають різноманітні міні-мурали та зображення мандал. Є куточок
біля вогнища, поруч зі сценою розташована гойдалка. Всюди дуже затишно. Наразі
будується будиночок на дереві.

%\ii{19_05_2021.stz.news.ua.mrpl_city.1.novyj_art_prostir_mrpl.pic.4}
\ii{19_05_2021.stz.news.ua.mrpl_city.1.novyj_art_prostir_mrpl.pic.4_5}

29 травня відбудеться відкриття літньої сцени простору. На відкритті
виступатиме народний ансамбль \enquote{Шумка}. А 30 травня маріупольцям буде
представлена прем'єрна моновистава \emph{\enquote{Не будь як горіх}}, в якій піднімаються
актуальні соціальні проблеми.

% pics 6,8
\ii{19_05_2021.stz.news.ua.mrpl_city.1.novyj_art_prostir_mrpl.pic.6_and_8}

Арт-простір \enquote{Міський Будда} вже має багато однодумців і дру\hyp{}зів. Цікаво, що
Максим повністю оволодів технікою майстра з ремонтних робіт, адже все робиться
власними руками. Ніяких грантів чи спонсорів вони не мають. Проте, оскільки
арт-простір \enquote{Міський Будда} відкритий для всіх охочих, кожен небайдужий може
допомогти чим зможе. Загалом всі ремонтні та будівельні роботи, упорядкування
території зроблено силами подружжя та їхніми друзями, які вважають, що новий
арт-простір дає багато цікавих можливостей для особистого розвитку.

% pics 7,8,9
\ii{19_05_2021.stz.news.ua.mrpl_city.1.novyj_art_prostir_mrpl.pic.7}
% !! in fact it is 9_10
\ii{19_05_2021.stz.news.ua.mrpl_city.1.novyj_art_prostir_mrpl.pic.8_9}
