% vim: keymap=russian-jcukenwin
%%beginhead 
 
%%file 22_01_2022.fb.dykyj_evgen.1.zi_svjatom_soborna_ukraina
%%parent 22_01_2022
 
%%url https://www.facebook.com/evgen.dykyj/posts/10159867121373808
 
%%author_id dykyj_evgen
%%date 
 
%%tags den.ukr.sobornosti,sobornist,ukraina
%%title Зі святом, Соборна Україно!
 
%%endhead 
 
\subsection{Зі святом, Соборна Україно!}
\label{sec:22_01_2022.fb.dykyj_evgen.1.zi_svjatom_soborna_ukraina}
 
\Purl{https://www.facebook.com/evgen.dykyj/posts/10159867121373808}
\ifcmt
 author_begin
   author_id dykyj_evgen
 author_end
\fi

Цей день бував різним. Далекого 1990-го ми стояли живим ланцюгом від Харкова до
Львова та мріяли про Незалежність як про щось далеке та майже недосяжне. У
2011-му наш живий ланцюг на мості Патона був викликом свіжообраному Януковичу
та першим натяком на майбутній Майдан. Ще три роки потому нам було не до
ланцюгів – в той день «Груша» кілька разів переходила з рук у руки, ми палили
шини та розливали коктейлі, і отримали перших чотирьох «двохсотих». Ще рік
потому лічба «двохсотих» йшла на тисячі, почалися бої за Дебальцеве. Минуло сім
років, і от ми рахуємо «борти», що везуть нам зброю від союзників, обліковуємо
ворожі танки на кордонах та готуємось знову захищати незалежність та
соборність. Власне, нічого нового – той самий ворог, та сама загроза, ті самі
цінності, за які ми готові помирати самі та «двохсотити» ворогів. І «расклад»
направду незрівнянно кращий аніж всі попередні рази: Совок був визнаною всіма
політичною реальністю, а незалежна Україна – для когось мрією, а для більшості
нереальною фантазією; Янукович був «живим та легітимним», а ми –
«екстремістами» та «правопорушниками»; Дебальцево боронили радянським іржавим
залізяччям, союзники ж замість «Джавелінів» та «Стінгерів» надсилали в кращому
випадку броніки та бинти. Ну а якщо порівняти всі пережиті нами Дні соборності,
і наше становище в ці дні, із тим одним днем у 1919, коли оголосили про своє
об’єднання у Соборній Україні дві українські держави, вже фізично майже не
існуючих, із вщент розбитими арміями та окупованими 90 відсотками території –
нам взагалі гріх жалітись на долю. Порівняно із всіма поколіннями до нас ми
маємо найкращі у мови та становище, і з такої позиції програти Орді – це вже
просто себе не поважати. Я не знаю, почнеться нова фаза війни сьогодні, за
тиждень, чи зусиллями Заходу відкладеться ще на рік – другий. Цього наразі не
знають ні Байден, ні Зе, ні навіть Пуйло. Але загалом фінальний двобій Соборної
Незалежної України проти недобитого Совка неминучий. На карті світу лишимось
або ми, або Імперія Зла, де і зараз, як співав 35 років тому один з останніх
порядних росіян «люді, стрєлявшиє в наших отцов, строят плани на наших дєтєй».
І шансів вижити та перемогти у нас наразі більше аніж будь-коли в попередній
історії. Тож святкуємо день нашої єдності, і спокійно, без зайвих нервів
обираємо кожний своє місце у нашому живому ланцюгу опору на той день, коли це
знадобиться. Зі святом, Соборна Україно!
