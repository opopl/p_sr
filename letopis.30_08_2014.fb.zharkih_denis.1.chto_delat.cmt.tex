% vim: keymap=russian-jcukenwin
%%beginhead 
 
%%file 30_08_2014.fb.zharkih_denis.1.chto_delat.cmt
%%parent 30_08_2014.fb.zharkih_denis.1.chto_delat
 
%%url 
 
%%author_id 
%%date 
 
%%tags 
%%title 
 
%%endhead 
\subsubsection{Коментарі}

\begin{itemize} % {
\iusr{Татьяна Чеботникова}
бывшей украине нужен Нельсон Мандела, но вы его не вырастили

\iusr{Olga Ivanova}
ооо, как мило, а что значит "бывшая Украина"? типо только росия щаз настоящая?

\iusr{Yana Beniaminova}
Ольга, создается впечатление, что вы не"усвоили" смысл прочитанного)

\iusr{Julia Eryomenko}

Бывшая Украина - это страна, где нарушены и попраны все статьи конституции и
УКК, где отсутствует мораль, где нарушается право на жизнь и свободу.

\iusr{Olga Ivanova}

Я " усвоила" , коммент пропал как-то. Ненависть - это не обо мне и большинстве
Украинцев. Это ватники ненавидят Украину. И это не "бывшая страна", Капулетти
нику отвечаю. такой вот фамилией Иванова я Украинка, и рассе нас не победить


\iusr{Виктор Телегин}
Именно так!

\iusr{Наталья Дейкина-Овчинникова}
Это красиво, но нереально ...

\iusr{Денис Жарких}
Реально Татарское иго победили духовно, потом оружием, а последнее сражение на Угре и сражением не было. Татары убежали.

\iusr{Наталья Дейкина-Овчинникова}
Не сильна в истории...

\iusr{Yana Beniaminova}

Ольга, вы способны говорить нормально? человеческим языком? я не знаю, что
такое " ватник"! какого отношения к себе вы ожидаете, используя подобную
терминологию?! здесь нормальные адекватные люди!

\iusr{Анна Юрьевская}
Ольга ИВАНОВА! С каких пор вы ненавидите Россию? Вы о ней вообще думали три месяца назад?

\iusr{Елена Рудяк}
Ненависть и ярлыки. Не сомневаютмя только дураки.

\iusr{Julia Eryomenko}
Денис, спасибо за пост. Вы прям как Довлатов

\iusr{Olga Ivanova}

Да, представляете, я вменяемо по-русски могу понимать и писать! Неожиданно,
видимо)) Так как нас обзывают бандеровцами и укропами, так и зазомбированных Пу
называют ватниками

\iusr{Yana Beniaminova}
Ольга вас здесь и сейчас кто нибудь обозвал?

\iusr{Елена Рудяк}
@Olga. Ivanova заблукала трохи, вам не сюди

\iusr{Olga Ivanova}
ой да, вы правы, у Дениса на странице я больше не буду альтернативные мнения постить))

\iusr{Julia Eryomenko}

Да, я ее обозвала. Написала правду, что в стране, где я живу всю жизнь и
вынуждена скрываться под ником, полностью отсутствуют правовые основы
государства

\iusr{Julia Eryomenko}
А где там, в чем было мнение, Ольга? Штампы и агрессия

\iusr{Денис Жарких}
Ольга, постите, я не против.

\iusr{Никита Голобоков}

Проблема в том, что и Россия всю политику по отношению к Украине за последние 15
лет свела к газовому транзиту. То есть, просто Украиной занимались
поскольку/постольку. В итоге и получили ненависть.


\iusr{Yana Beniaminova}

Ольга, вы сами того не замечая, настраиваете людей против себя! признаюсь, я не
согласна с вашими взглядами! но я не позволяю себе вас оскорблять! я хочу, что
бы это было взаимно!

\iusr{Анна Юрьевская}

С этими вылупившимися два месяца назад "украинцами"вообще интересный
вопрос. Людям внушили, что до этого они были "непоймикем",а как только
возненавидели Рашку, так стали сразу "украинцами".И эти ,собственно, прямо в рай
поступают - прощаются все грехи - и настоящие и будущие .Можно убивать даже -
Госдеп (не Господь! ) прощает. А кто же жил до этого 1200 лет тут?

\iusr{Денис Жарких}
Алла - браво! Кстати в 2004 те же аргументы были НАС ЗНАЕТ ВЕСЬ МИР!

\iusr{Елена Елена}
Вот все на лицо, оставим особо яросных и обидчевых учеников на второй год?

\iusr{Yana Beniaminova}

то что произошло, уже нельзя изменить! вопрос, как жить дальше? будь то Украина
или что либо иное, как остановить этот процесс деградации?

\iusr{Денис Жарких}
Процесс деградации останавливается мыслительным процессом и трудом.

\iusr{Yana Beniaminova}
боюсь с мыслительным процессом могут возникнуть сложности... но в конце концов, именно труд сделал из обезьяны человека) так что надежда есть)

\iusr{Olga Ivanova}
как и кого я оскорбила? термин "ватник" применили к себе? Что ж, тут я ни при чем...

\iusr{Денис Жарких}
Ольга, Вы правы. Зомбированных тут нет. Только думающие.

\iusr{Yana Beniaminova}

Ольга, хоть вы и владеете русским языком) вам не мешало бы еще овладеть умением
общаться! потому что в будущем, особенно если вы хотите жить в Европейской
стране, вам это пригодится! недопустимо применять подобные термины к
малознакомым людям! если они с вами не согласны, это не делает их хуже или
лучше! усвойте!!!


\iusr{Денис Жарких}
Если не изменяет память Тахкомыш спалил Москву после Куликовской битвы.

\iusr{Денис Жарких}
Яна, давайте не будем никого ничему учить. И вообще меньше восклицательных знаков. Мы же не укропы!!!  @igg{fbicon.smile} 

\iusr{Yana Beniaminova}
Денис, я знаете ли, наивна) согласна с вами, больше не буду, пусть резвятся

\iusr{Денис Жарких}
Чем бы женщина не тешилась, лишь бы на фронт детей не посылала.

\iusr{Olga Ivanova}

Яна с иностранной фамилией, спасибо за мнение! Извините за восклиц. знак))) (
не умею синим цветом теги ставить, чтоб ответить лично) Все так относительно,
особенно мнение о способности общаться живущим вдалеке от Украины. Также
замечу, что не делала замечаний, кто хуже, кто лучше. Посылаю лучи добра и
счастья!

\iusr{Yana Beniaminova}

Ольга, спасибо за акцент на фамилии) но это фамилия моего мужа, а не моя) по
поводу удаленности от Украины - вся моя семья находится в Луганске на
сегодняшний день, поэтому я имею моральное право высказывать мнение, и по
последнему пункту, я не сказала, что вы делаете замечания, я призвала вас,
вести себя по человечески! а пожелания взаимны)

\iusr{Sergey Serous}

Пардон, @Olga Ivanova, со странным аватаром, траурный квадрат с полосками, я
уже видел у одного тролля. Вопрос, вы украинка и работаете в Москве? Чойта? В
окружении зазомбированных ватников? Дурите их? Облапошиваете? Денюшку для
украинской армии зарабатываете?

\iusr{Залина Базоева}

Замечательный пост! О настоящем! О том, что действительно важно и существенно!
А не о том, кто прав, а кто виноват! Спасибо Вам, Денис, за то, что пишете и за
то, ЧТО пишите!

\iusr{Джулия Кошкина}

Пост, как всегда-в глаз, но очень не хватает живого общения..., я помню про
встречу...(((Кофе?))

\iusr{Ivanka Smilovich}

Милая, бедная Ольга Иванова... Таки яркий и жалкий пример того, о чем Вы,
Денис, написали. Хлебом не корми - дай поскандалить грязным ртом, дай
поненавидеть!!! Удивительная... женщина. Коих, к сожалению... тьма. Потому
ненавистью и продолжает накрывать. Не мудростью. Не умом. Искусственно
взрощенной ненавистью...


\iusr{Sergey Serous}
Какой Голодомор, японцы вон американцам Хиросиму с Нагасаки простили, по их приказу в санкциях против России участвуют ...

\iusr{Наталья Бабанина}

А слабо на выпады не отвечать? Сейчас у каждого своя правда. Пусть тысячу раз,
с нашей точки зрения, и корявая. Сейчас важно быть терпимее. Вот я так думаю.
Хотя сложно, безусловно

\iusr{Алексей Большаков}

. В стране стоит русофобская атмосфера и это явно не имеет под собой такого
масштабного основания. Это стало возможным благодаря победе сил, пришедших к
власти и навязавшей с помощью олигархов через сми подобную ересь, чтобы народ
нашел врага и не мешал им решать свои меркантильные интересы. После майдана
власть дала явный посыл, что пришла ради народа Украины и готова реформировать
страну для долгожданного за все годы независимости роста экономики и
благосостояния народа. Но это слова. Реально за это время никаких реформ не
было и народу так и не смогли объяснить, какую такую страну мы будем строить.
Зато Яценюк собрал команду и обвалил гривну, прикрываясь требованиями МВФ, но
не забывая про свой карман через очень выборочное рефинансирование. Затем в
духе авантюрных романов он и Турчинов так громко начали накручивать ситуацию с
востоком страны, что таки добились начала настоящих военных действий и теперь
никому было не интересно спрашивать о реформах и будущем страны, а все
мужественно стали бороться с внешним агрессором, наличие которого так и не
смогло дать оснований объявить эти действия войной. До сих пор мы воюем с
Россией, а гибнут и стой и с другой стороны в основном наши соотечественники.
Кому нужна вся эта война? Кто реально верит, что Россия будет присоединять
Донбасс к своей территории? Это миф высосанный из пальца. Такое развитие
событий абсолютно не выгодно России, т.к. оно не дает ей никакой выгоды кроме
целой кучи проблем. Им с головой лет на 20 хватит Крыма, чтобы подтянуть этот
регион к общероссийским стандартам. В чем причина конфликта? На востоке, как им
показалось, не приняли переворота в Киеве и вместо реальных объяснений, что это
не так, их начали травить и разжигать внутри страны ненависть. А ведь надо было
просто не на словах, а на деле показать, что грядут реальные реформы для
модернизации страны. Не на словах, а на деле отдать полномочия на места,
подкрепив их реальным бюджетом. И не было бы никаких волнений. Но, как
показывает время, эти реальные реформы как раз и не нужны новой власти. Гораздо
проще о них говорить и пенять, что им сейчас просто не дают их вершить. В
стране за полгода не произошло ни одной перемены к лучшему. Ни одна структура
не реформирована для народа. Коррупция стала опять нормой, только потоки внутри
страны изменили направление. И за это боролся народ Украины? Люди, опомнитесь!
Ведь, конечно, проще при снижении уровня жизни все валить на придуманного для
вас врага.

\iusr{Алексей Большаков}

Во всем виновата Россия. Это она развалила промышленность Украины и довела
страну до того, что если бы санкции коснулись Украины, то мы бы наконец поняли
в какой «мапупии» мы живем. Страна полностью зависит от импорта и не производит
и 20\% продукции для обеспечения жизнедеятельности страны. Как не пеняй на
Россию, но даже убрав доходы от газа и нефти их ВВП буде минимум в 2,5 раза
выше на душу населения чем у нас. А ведь Россия так далека от цивилизовано
развитых стран Европы и США. Взрослый человек, если он адекватен, в своих
проблемах винит себя, а не соседа Васю. А нам все время кто-то извне мешает
строить страну. Может стоит повзрослеть и выбирать более избирательно «слуг
народа», тогда быть может и майданы не понадобятся? Ведь, если включить мозг и
отбросить манипуляцию сознания с помощью всевозможных рейтингов через сми
олигархов, то стоит (по совету Рабиновича) выгнать метлой всех завсегдатаев ВР
и других местечек в правительственном квартале, кто последние 10 лет был
причастен к этим структурам. Они мастера мимикрии. Ведь все знают, что взрослый
человек очень тяжело меняет свои взгляды и убеждения. Так почему мы каждые 4
года верим этим «старым мордам», которые перед выборами всегда прозревают и
понимают свои ошибки. Они так доверительно обещают нам измениться, что скоро в
нашем сарае не будет места складывать грабли. Неужели наш народ будет позволять
манипулировать собой и дальше? Может быть надо им сказать: «Досыть!» Если мы
построим успешную страну, то это единственный способ доказать нашим соседям
свою полноценность. Хватит всех обвинять, давайте работать. Давайте делать
правильный выбор. Не надо выбирать за слова, давайте применять другие критерии.
Ведь репутацию человека никто не отменял и в делах мы судим о человеке именно
по ней. Так почему мы так слепы на выборах? Ведь это единственно правовой и
цивилизованный способ делать будущее своей страны. Давайте в этот парламент не
изберём тех, кто хоть раз там был и посмотрим на результат. Может он нас
удивит. Эта «новая» власть за полгода разве не подтвердила мою просьбу? Я верю
в вас народ Украины!

\end{itemize} % }
