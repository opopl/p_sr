% vim: keymap=russian-jcukenwin
%%beginhead 
 
%%file 03_02_2022.yz.maj_dnr.1.zachem_demonizirujut_rossiu
%%parent 03_02_2022
 
%%url https://zen.yandex.ru/media/id/5f8f226b1fe36c1d9e02a36b/zachem-demoniziruiut-rossiiu-61fc514545e8005f9759e38c
 
%%author_id yz.maj_dnr
%%date 
 
%%tags demonizacia,infvojna,rossia,ukraina,zapad
%%title Зачем демонизируют Россию?
 
%%endhead 
 
\subsection{Зачем демонизируют Россию?}
\label{sec:03_02_2022.yz.maj_dnr.1.zachem_demonizirujut_rossiu}
 
\Purl{https://zen.yandex.ru/media/id/5f8f226b1fe36c1d9e02a36b/zachem-demoniziruiut-rossiiu-61fc514545e8005f9759e38c}
\ifcmt
 author_begin
   author_id yz.maj_dnr
 author_end
\fi

А давайте еще раз порассуждаем на тему «Россия нападэ». Коротенько, в ритме
тоста.

Тост 1. Глобальные масс-медиа вместе с соцсетями Россию уже «приговорили» к
неизбежной войне. Украина ее давно «ловит на горячем» вместе со всем составом
ОБСЕ в 56 из 57 стран, поймать не могут, но раз Псаки сказала «агрессор
очевиден всему цивилизованному миру», значит, этот самый «цивилизованный»
обязан так и думать, иначе папа Джо ремешком накажет. Больше всего мне в этой
ситуации нравится, что Киев стопитцот раз в год заявляет про украино-российскую
войну, а потом рекомендует своим гражданам приготовиться к «чрезвычайной
ситуации», мол, вот, сейчас, завтра или на днях Россия нападет. Нормальный
человек спросит: как же нападет, если уж почти 8 лет война? Но за подобный
вопрос этот нормальный человек тут же попадет на Миротворец, как пить дать.

\ii{03_02_2022.yz.maj_dnr.1.zachem_demonizirujut_rossiu.pic.1}

Но знаете, мы тут, в Донбассе уже ничему не удивляемся.

Пропагандистский раскол Вселенной на «демократические страны», т.е.
подчиняющиеся или подконтрольные США, и остальные, которые однажды назвали
«недемократичными» и даже запилили целый каталог с указанием \% демократии, в
зависимости от того, кто в какой степени «уважает» «Вашингтонское политбюро».
Ну, и отношение к ним соответствующее. Так вот, Россия в этом каталоге где-то в
самом конце, а потому к ней можно применять, по мнению западных политиков и
СМИ, любые информационные (похоже, что не только) методы. От «Мы вас заставим
любить демократию» до «агрессор очевиден». Но их поражает другое: Россия
демонстрирует не только стойкость к многолетним информационным атакам, но и
полное отсутствие страха перед угрозой «Мы вас изолируем».

Тост 2. Обратите внимание на то, в какое состояние загнали себя сами
государства, идущие на поводу у «Вашингтонского политбюро». Я не говорю о
власти, вернее, полновластии транснациональных корпораций над разумом и самой
жизнью. Я говорю об обычном, бытовом – газовые войны, биологические войны,
банковские войны, гендерные, расовые, кредитные, да этот список можно
продолжать и продолжать. Сколько там, говорите, ежегодно замерзает в Британии
стариков? Сколько там говорите, бомжей в Европе? А как чувствуют себя белые в
США, все еще встают на колени перед афроамериканцами? Я говорю о системном,
глобальном кризисе, который остановить может только новое переформатирование
мира, по простому передел. Вот только обычный способ передела в виде войны уже
не катит – слишком опасно для самого «политбюро», ибо нет у них таких бункеров
в которые можно было бы заселиться на сотню-другую лет, пока Земля будет
очищаться от ядерной пыли.

А Россия что? А России не нужны войны и прочие дрязги – ей время расти. Через
сто лет, вот увидите, сегодняшнее время назовут «Русским Чудом», даже не смотря
на общемировой кризис, пандемию и другие, не менее глобальные проблемы. Еще раз
повторюсь для непонятливых – России война не нужна, и сейчас она делает все,
чтобы ее не втянули в мировую заваруху, вернее, чтобы этой мировой заварухи
вообще не было. И плевать ей на интересы каких-то там «обкомов», «глубинных
правительств» и прочих.

Тост 3. Диалога с НАТО не существует. Показательный возврат якобы отношений –
фикция. Путин сказал, что Россию не устраивает ответ на предложения по
безопасности, и весь мир замер в ожидании реакции Кремля. А тут еще, как назло
(ха-ха) Украина сливает в сеть полный текст этого самого ответа. Украина – не
Россия а это значит, что США, испугавшись, что Россия таки обнародует сей
документ, да еще со своими пояснениями и комментариями, поспешили сделать это
через своих вассалов. Путин не только дзюдоист, но и гроссмейстер. А значит,
геополитическая шахматная партия будет разыграна так, что «черные» будут играть
сами с собой и сами себя уберут с шахматной доски. Лишь бы только те белые
фигуры, на которых проступают черные пятна, так пешками и оставались.

Другой вопрос, что время неумолимо, а потому надо спешить. Спешить принимать
решение. Не тем, что наверху, а нам, простым людям надо решать, с кем мы. С
белыми, или готовы вымазаться в черный. С Псако-Блинкено-Зеленскими или со
своими детьми. Да, да, вопрос стоит именно так. От нашего решения, от каждого
из нас по отдельности, зависит будущее гораздо больше, чем мы даже можем себе
представить.

Тост 4. При чем тут малозначимый Зеленский? А при том, что в том самом
«политбюро» считают (пока еще считают), что шанс у них есть. Бескрайняя Россия
со всеми своими природными ресурсами – непаханное поле и лакомый кусок для тех,
кто уже испоганил свои земли сланцем, ГМО и прочими вредными для жизни
«арийцев». Ну, что поделаешь, подушатала Европа свои реальные активы, Британия
доедает последний кус, оставшийся от колоний, да и в Америке не все так гладко.
А тут, прям под носом, Сибирь с Уралом, да несметные «кладовые» из сказок
Бажова. Разве устоишь? Но напролом страшновато – 1945-й еще в памяти. Значит,
включаем «сделать виноватым и конфисковать». За что виноватить-то? А найдется.
Не знаю, обратили ли вы внимание на волны санкций, начиная от Магницкого с его
списком, Грузинских мотивов 08.08.08, Скрипаля, Боинга и прочего, вплоть до
Навального – ведь все это разработанные «политбюро» информационные атаки для
введения санкций. И сегодняшняя истерика по поводу мнимой угрозы «Россия
нападэ» - из того же ящика Пандоры. Но для каждой такой атаки нужна жертва, и
ведь заметьте, жертвы всегда разные, проекты почти не повторяются, разве что
разгон фейков идет по одной и той же методичке. Короче говоря, теперь в жертвы
выбрали Украину. И я вообще думаю, несмотря на мою отчаянную неприязнь к
Порошенко, что его заменили на Зеленского как раз по причине того, что он не
соглашался отдавать страну на заклание. Впрочем, тут не патриотизм сработал, а
принцип «такая корова нужна самому». Но факт – Зеленский не побоялся славы
Нерона, может быть, он даже сценку какую разыграет, глядя, как горит Рим,
пардон,, Украина. Не думаю, что он не понимает, на что идет – разве его не
впечатлили нью-йоркские башни, снесенные ради «демократизации» Афганистана и
Ирака?

Тост 5, самый короткий. Банальная истина: начал дело – доводи до конца. Вот
поэтому я и думаю, что, начав строить Украину как Анти-Россию, американцы и
Британцы доведут это дело до конца. И пойду на любые крайности.

Незабвенный генерал из знаменитого фильма говорил, что тост должен быть
коротким, как выстрел. А я тут вам наваяла своих размышлялок на целое застолье.
Но знаете, не желаю пить за упокой, а хочу поднять бокал за здравие. За разум.
За нравственность. За традиции. Словом, за будущее свободной и сильной России.
