%%beginhead 
 
%%file 07_08_2023.fb.muzej_literatury.ua.kiev.1.avtorskij_vechir_julii_kashpurenko
%%parent 07_08_2023
 
%%url https://www.facebook.com/NMLUmuseumlit/posts/pfbid02yiiUsCnhbfj2aSqjnKVyTa9SPGpLrT7JZoaRs27haWyYXfiZkqE6REPiP1zkkQMRl
 
%%author_id muzej_literatury.ua.kiev
%%date 07_08_2023
 
%%tags 
%%title Авторський вечір поезії Юлії Кашпуренко
 
%%endhead 

\subsection{Авторський вечір поезії Юлії Кашпуренко}
\label{sec:07_08_2023.fb.muzej_literatury.ua.kiev.1.avtorskij_vechir_julii_kashpurenko}

\Purl{https://www.facebook.com/NMLUmuseumlit/posts/pfbid02yiiUsCnhbfj2aSqjnKVyTa9SPGpLrT7JZoaRs27haWyYXfiZkqE6REPiP1zkkQMRl}
\ifcmt
 author_begin
   author_id muzej_literatury.ua.kiev
 author_end
\fi

5 серпня в Національному музеї літератури України в рамках музейного проєкту
\enquote{ритМИ і риМИ війни} відбувся авторський вечір поезії Юлії Кашпуренко. 

\href{https://www.facebook.com/profile.php?id=100002050349167}{Юлія Кашпуренко} – поетеса, військова лікарка-хірург, волонтерка та
координаторка волонтерського штабу \enquote{Свій до свого}.

\ifcmt
  ig https://i2.paste.pics/e0e1a6b5a48838cb6a27f29badecde4e.png
  @wrap center
  @width 0.9
\fi

Вечір поезій був благодійним, збір коштів відбувався для потреб 1 роти 1
батальйону 80-ої окремої десантно-штурмової бригади ДШВ ЗСУ.

Пані Юлія читала свої вишукані, ніжні та світлі поезії під супровід ніжної,
елегійної мелодії. Як наслідок – створила дивовижну, зворушливу, камерну
атмосферу, яка заполонила всіх присутніх у мистецькому виставковому залі музею.
До деяких \enquote{особливих} віршів поетеса подавала авторські коментарі, передісторію
їх написання. Особливим сюрпризом стало виконання Юлією на прохання присутніх
народної пісні \enquote{Скрипливії ворітечка}. Адже поетеса, попри свою мужню військову
професію, пише не тільки дуже гарні вірші, а й чудово співає. 

Професор Юрій Ковалів, який написав передмову до віршів пані Юлії,
опублікованих у рубриці \enquote{Книжка в газеті} культурологічного просвітницького
тижневика \enquote{Слово Просвіти}, акцентував, що поезії авторки являють нам збірний
портрет сучасної української душі, яка переживає цю війну в усіх її трагічних
вимірах. Пан Юрій підкреслив, що працюючи з текстами багатьох поетів,
спостерігав надмірні емоційність та пафос, тоді як у ліриці пані Юлії присутні
роздуми та внутрішній біль. У її віршах учений спостерігає вистраждану,
витончену лірику. \enquote{Вірші пані Юлії – це філігранна робота. Коли читаєш її
поезії, помічаєш надзвичайно високу культуру поетичного мовлення!} Юрій Ковалів
також спонукав авторку видавати свої збірки, адже її поезія заслуговує бути
прочитаною кожним. 

На заході були присутні військові, які висловили щиру вдячність авторці,
зауваживши, що вона демонструє надпотужний рівень сучасної української
літератури. Воїни подарували авторці шеврон свого батальйону. На якому було
зображено дуб – як символ могутності, дубовий вінок – як символ перемоги та
сонечко у сварзі. 

Учасник Історичного клубу \enquote{Холодний яр} Василь Ковтун, запросив Юлію Кашпуренко
доєднатись до їхнього товариства і наполіг, на важливості видання поетесою
власної збірки.

Під кінець захід відвідала резонансна поетеса, лідерка платформи \enquote{Творчий
сектор}, аеророзвідниця Збройних Сил України Мальва Кржанівська. Вона принесла
турнікети для бійців та футболки із зображеннями волонтерок, лікарок та інших
жінок, що допомагають фронту. Кошти з продажу цих футболок із зображенням Юлії
Кашпуренко будуть спрямовані на потреби бійців 80-ої окремої десантно-штурмової
бригади ДШВ ЗСУ. Також пані Мальва прочитала поезію, яку присвятила пані Юлії. 

Насамкінець Юлія Кашпуренко розповіла про те, що вона замислюється над виданням
збірки віршів. Адже всі її поезії нині розкидані в різних записничках чи то на
сторінках Facebook. А ще пані Юлія подякувала за теплу й невимушену зустріч у
музеї, що \enquote{дихала} поезією, музикою, піснею, квітами та любов'ю. І зізналася,
що така зворушлива зустріч у неї відбулася чи не вперше. 

Національний музей літератури України дякує авторці за щирий талант, витончені
вірші та величезну волонтерську працю. А гостям – за чудові слова та теплий
вечір.

Відео заходу за посиланням \url{https://youtu.be/y_E0JS0QLSs}
