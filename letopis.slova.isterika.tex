% vim: keymap=russian-jcukenwin
%%beginhead 
 
%%file slova.isterika
%%parent slova
 
%%url 
 
%%author 
%%author_id 
%%author_url 
 
%%tags 
%%title 
 
%%endhead 
\chapter{Истерика}
\label{sec:slova.isterika}

%%%cit
%%%cit_head
%%%cit_pic
%%%cit_text
Небратья и братья.
Украинская пропаганда устраивает \emph{истерию} по поводу высказывания Путина об одном
народе. \enquote{Он отказал нам в праве на существование, он желает уничтожить нас, он
никогда не признает нашего выбора} - сокрушаются они. Высказывание Путина
действительно опасно для украинской власти, и тут много моментов, на которые
важно обратить внимание.
Сама \emph{истерическая реакция} совершенно не удивительна. \emph{Истерика} по любому
высказыванию Москвы для украинской власти стала нормой. Украинская политика
похожа на вздорную женщину, которая считает, что ее непременно хотят обидеть,
изнасиловать и унизить. При этом бывший муж давно развелся и живет своей
жизнью, но дама считает, что он думает только о ней, вынашивает месть, и желает
вернуть любой ценой. Недавно Владимир Быстряков так высказался об процессе
украинской самостоятельности - \enquote{У нас отобрали нефть, газ, моря и океаны}. И
тут не поспоришь. Если развелись, то не стоит думать, что бывший муж будет и
дальше приносить зарплату. Но негодяй, непременно негодяй
%%%cit_comment
%%%cit_title
\citTitle{Что может предъявить украинская власть Путину на фразу о едином народе? / Лента соцсетей / Страна}, 
Денис Жарких, strana.ua, 04.07.2021
%%%endcit

