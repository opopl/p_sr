% vim: keymap=russian-jcukenwin
%%beginhead 
 
%%file slova.moskva
%%parent slova
 
%%url 
 
%%author 
%%author_id 
%%author_url 
 
%%tags 
%%title 
 
%%endhead 
\chapter{Москва}

Болтовня о \emph{промосковскости} и \enquote{пропутинскости} Лукашенко есть полная ерунда,
поскольку все эти годы бацька только то и делал, что балансировал между \emph{Москвой}
и Западом, всячески уворачивался от цепких объятий \emph{Москвы} и лично Путина, с
которым у него очень непростые отношения, пытаясь сохранить максимум
самостийности, но не разрушить при этом, по украинскому образцу, экономику и
социалку, во многом зависящую от рынков бывшего Союза и вообще от традиционных
связей. Один только герб Советской Белоруссии чего стоит на фоне российского
триколора!,
\textbf{Европа впала в иррациональный истеричный лукашизм}, Александр Карпец,
strana.ua, 03.06.2021

Колектив Національного культурного центру України у м. \emph{Москві} вітає з Міжнародним днем захисту дітей!, 
\textbf{Наші діти!}, ukrcentr.ru, 01.06.2021

27 травня в Національному культурному центрі України у м. \emph{Москві} відбувся
концерт лауреата міжнародних конкурсів Української народної хорової капели
\emph{Москви}. На барвисту концертну програму «Пісенне джерело натхнення» завітали
шанувальники й друзі Української народної хорової капели \emph{Москви}, які протягом
багатьох років відвідують численні музичні вечори в Культурному центрі України
та, насамперед, щомісячний Український музичний салон. Відкриваючи концертну
програму творчого колективу, зі словами вдячності до гостей вечора звернулася
художній керівник та диригент капели, начальник служби з питань культури і
діаспори НКЦУ, заслужений працівник культури України та Росії, голова РГО
«Культурно-просвітницький центр українців у м. \emph{Москві}» Вікторія Скопенко,
\textbf{Пісенне джерело натхнення}, ukrcentr.ru, 28.05.2021

