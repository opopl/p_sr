% vim: keymap=russian-jcukenwin
%%beginhead 
 
%%file 15_03_2023.fb.kipcharskij_viktor.mariupol.1.r_k_tomu___18__den_.cmt
%%parent 15_03_2023.fb.kipcharskij_viktor.mariupol.1.r_k_tomu___18__den_
 
%%url 
 
%%author_id 
%%date 
 
%%tags 
%%title 
 
%%endhead 

\qqSecCmt

\iusr{Марина Солошенко}

Чекала, що ти вийдеш на зв'язок кожного дня і раптом прорвався! У нас було
жахіття, а Маріуполь це не передати словами! Мені так хочеться, щоб твій
щоденник потрапив до росіян, невже вони не зрозуміють, що сотворив їх кумир!
Біль біль біль!!! Ти виїхав в той день, як театр був розбитий ( навіть німці
цього собі не дозволили), це буде завтра. В Харкові гради, та смерчі готили по
Салтівці, та ХТЗ, наш район залишався самим безпечним. Мої діти серед ночі
вийшли во двір і бачать, на них летять 3 ракети, вони від страху ледве в дом
зайшли, вибухнуло в центрі, в той час ракети йшли з Криму.

\begin{itemize} % {
\iusr{Віктор Кіпчарський}
\textbf{Марина Солошенко} 

Найважчим була відсутність зв'язку і, відповідно, новин. Тепер я впевнений, що
саме це нас і врятувало...

Дізнавшись про реальні події за межами нашого двору, ми би могли опустити руки...

\iusr{Светлана Водзянская-Живогляд}
\textbf{Марина Солошенко} 

це був метушливий день, спала геть погано, на те як стрибають банки ніхто вже
не звертав уваги, гупало так що між вибухами майже не було тиші. З самого ранку
я була сама не своя, не могла стояти на одному місці навіть хвилину,
прислухалася до звуків, боліло сердце, наче його хтось стискає у долонях.
Поприбирала в хаті, пішла зранку готувати борщ, чоловік здивувався, але мовчки
розвів багаття (в мене вже кілька днів нав'язлива ідея прогріти будинок, бо
коли кріпили підвісну стелю майстер сказав, що не можна опускати температуру
нижче +2, в хаті +3,5). І мені потрібне гаряче каміння в будинку, хочаб на 1
градус прогріти і відтягнути невідворотну втрату дорогоцінної підвісної стелі.
Зготувала їсти велику каструлю борщу, каміння бігом в хату, чоловік свариться,
а мене тіпає, я бігаю з тією цеглою, мию дзеркала і перекладаю речі. Вийшла до
криниці людей не багато, прилетіло кілька мін десь дуже близько, намагалася
забитися у проміжок між криницею і парканом, почула автоматні черги, багато
черг у районі 17 МКР, почало трусити. Люди кажуть що наш сусід сьогодні виїхав,
наче є коридор і він подався на здачу у Ялті. Вийшла кума за водою спитала що
зі мною, не знаю звідки вийшла наступна фраза «Там Ромик, я знаю это он». На що
отримала відповідь «Здались бы уже и все закончилось». Так боляче мені ще не
було, я знала що вони «ждуни» але сказати таке про мого брата було занадто... Не
стала витрачати сили на розмову, просто пішла у двір мовчки, дійшла до городу і
впала на коліна, ще кілька черг автомата, чоловік не розуміє що коїться, а мене
викручує від болю за грудною клітиною, я кричу йому в одежу, далі пам'ятаю
погано (кричала, билася, сварилася, казала що зараз піду вбивати русню голими
руками) через годину приходжу до тями, і намагаюся зайняти себе, скупала дітей,
зварила кашу ( та не кашу якщо чесно, а цеглу нагріти). Руки тремтять наче в
алкаша, працює авіація по 1000 дрібниць...

\iusr{Віктор Кіпчарський}
\textbf{Светлана Водзянская-Живогляд} 

Нам дала сили надія, а надію дало те, що напередодні кількасот людей з
Маріуполя дісталися до Запоріжжя. Принаймні так сказали по радіо. І незважаючи
на те, що доїхали не всі - колону накрили Градами - ми розуміли, що наше
спокійне (якщо порівняти з Вашим життям, та життям багатьох інших, хто зараз
постить свої спогади, з жахливими підробицями), триває доти, доки тримаються
вікна та дах.

Тобто, їхати треба неодмінно, незважаючи на купу якщо:

\obeycr
Якщо Епіка не заведеться.
Якщо не випустять.
Якщо палива не вистачить.
Якщо колеса пробьють уламки.
Якщо в телефонах щось знайдуть.
Цих якщо було дуже багато.
\restorecr

А противагою цим \enquote{якщо} було лише \enquote{не спробуєш - не дізнаєшся}.

Ну і головне правило водіїв, якому мене вчив тато: \enquote{Пішов на обгін - то вже не
смикайся}.

Дорогою до госпіталю в пошуках зв'язку я реально побачив що може статися.
Статися будь з ким, статися будь коли.

Тож, треба було спробувати втекти від цього.

А для цього потрібно бути спокійними самим і не давати приводу для паніки
близьким.

\end{itemize} % }

\iusr{Светлана Водзянская-Живогляд}

Старша донька(від попереднього шлюбу) в підвалі спитала в чоловіка куди літаки
скидають бомби, він навіть не подумавши відповів що 1000 дрібниць і все поруч,
вона заскаучала, як побите цуценя, її батько жив біля аллеї поруч ( авіація
вдарила сусідній будинок, його теж сгорів згодом), ледь заспокоїли казками що
батько у підвалі і з ним точно все добре. Діти нервують, молодший не дозволяє
себе перевдягнути і зняти зимового комбінезона і взуття, іграшки ховає під
подушку, книги взагалі недоторканні. Малих вклала не знаю нащо пішла мити в 3
раз підлогу в хаті, чоловік ліг з дітьми спати, я пішла біля 11 ночі. Годину
крутилася, від стресу випила за день біля 6 літрів води, як добре що криниця
поруч... намагалася заснути добре що сердце відпустило годині о 8 я хоч змогла
дихати. Старша у ві сні скиглить і всхліпує, середня сіпається іноді, молодший
спить на чоботах та у комбінезоні у позі ембріона, смердять урологічні
прокладки, які в нього замість памперса( ті що дали військові з Єви), вони йому
від живота до попереку, трохи протікають і страшенно смердять... Тільки
задрімала- почула кроки у коридорі будинку і стук у кришку підвалу....

Але це вже було завтра...

\begin{itemize} % {
\iusr{Віктор Кіпчарський}
\textbf{Светлана Водзянская-Живогляд} 

У мене на питання: \enquote{куди?}, або \enquote{це по нас?} була універсальна відповідь: \enquote{Від
нас!}. Літак над головою: \enquote{ Спокійно! Оскільки ми його чуємо - він вже нас
пролетів!}.

Міни над головою: \enquote{Ми її чуємо - то вона летить не до нас. Свою міну не почуєш!}.

І так далі.

В першу чергу, я заспокоював себе.

\iusr{Светлана Водзянская-Живогляд}
\textbf{Віктор Кіпчарський} це все вірно, але чути можна і коли літак тільки підлітає... в є хоч трохи часу сховатися від уламків

\iusr{Віктор Кіпчарський}
\textbf{Светлана Водзянская-Живогляд} 

З перехрестя Грецька-Миру ми бачили ще цілий Драм. Літаків або вибухів ані на
Нептуні, ані в центрі не чули. За новинами з телевізору хронологія така:
Нептун, а за годину - Драм.

Але це було вже без нас...

\end{itemize} % }
