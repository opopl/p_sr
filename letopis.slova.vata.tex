% vim: keymap=russian-jcukenwin
%%beginhead 
 
%%file slova.vata
%%parent slova
 
%%url 
 
%%author 
%%author_id 
%%author_url 
 
%%tags 
%%title 
 
%%endhead 
\chapter{Вата}

Коли я через декілька тижнів повернувся у Харків, застав, мабуть, останній
розгін \emph{вати}, коли містом ішов проукраїнський марш. Його перестріли \emph{ватники},
проте вони не співставили сили. Їх відгупали – і це була остаточна перемога.
Потім, у 15 році, ще був теракт, коли вибухнула міна, яка повбивала людей і
серед них був зовсім юний хлопець Даня Дідік. Я знаю його батька - він з того
часу двіжує по різних проукраїнських акціях. А ще були погрози, що типу на 9
травня теж будуть \emph{ватні} теми. Але на той час у нас вже сформувалася рота
\enquote{Східний корпус}. (Підрозділ патрульної служби поліції особливого призначення).
Вона була не дуже велика, але 9 травня вони їздили на джипах, з кулеметами з
вікон – \emph{вата}, звісно, перелякалася, 
\textbf{Доброволець Олександр Воробєй: \enquote{Війна не змінює людину, вона
просто розкриває її сутність. Якщо людина була гандоном, то після фронту вона
стане махровим п\#дарасом}},
Віка Ясинська, censor.net.ua, 02.06.2021

Коли почалися події на сході, зокрема і в моєму місті, я був у Києві. І ми
зібралися з хлопцями на так званий \enquote{Автопроїзд дружби}, який мав допомагати
ганяти \emph{вату} в Харкові, Луганську, Донецьку. Але під Чутовим (Полтавська
область) нас зупинив \enquote{Беркут} і сильно усіх відгамселив. Тоді з нами їхав мій
друг Сергій Бондар, псевдонім Тімур, а у нього була корочка помічника депутата,
\textbf{Доброволець Олександр Воробєй: \enquote{Війна не змінює людину, вона
просто розкриває її сутність. Якщо людина була гандоном, то після фронту вона
стане махровим п\#дарасом}},
Віка Ясинська, censor.net.ua, 02.06.2021

