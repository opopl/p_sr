% vim: keymap=russian-jcukenwin
%%beginhead 
 
%%file 23_11_2021.fb.fb_group.story_kiev_ua.2.mocart_v_kieve
%%parent 23_11_2021
 
%%url https://www.facebook.com/groups/story.kiev.ua/posts/1804069203123206
 
%%author_id fb_group.story_kiev_ua,chudnovskij_roman.kiev
%%date 
 
%%tags kiev,mocart.kompozitor,muzyka
%%title Моцарт в Киеве
 
%%endhead 
 
\subsection{Моцарт в Киеве}
\label{sec:23_11_2021.fb.fb_group.story_kiev_ua.2.mocart_v_kieve}
 
\Purl{https://www.facebook.com/groups/story.kiev.ua/posts/1804069203123206}
\ifcmt
 author_begin
   author_id fb_group.story_kiev_ua,chudnovskij_roman.kiev
 author_end
\fi

\index[rus]{Моцарт}
\index[rus]{Музыка}

\ii{23_11_2021.fb.fb_group.story_kiev_ua.2.mocart_v_kieve.pic.1}

\enquote{Ну какое отношение он имеет к нашему городу? - спросил меня мой внутренний и
порою изрядно вредный оппонент, когда я задумал раскрыть тему \enquote{Моцарт в Киеве},
- ведь великий композитор никогда не был у нас! НИКОГДА! Хотя в те времена, во
второй половине 18 века, его запросто могла пригласить с концертами - одним
щелчком пальцев - сама Екатерина Великая, которая дружила со многими
выдающимися европейцами и даже в Киеве побывала... Но не сложилось. Брось ты
эту тему!  Да и разве не Шопен был твоей любовью с малых лет - с его вальсом
номер 7?!}

\enquote{Что ж, Шопен действительно звучал в моей жизни, и в детстве и всегда, и этот
его гениальный вальс, - подумал я, - но сколько себя помню, и Моцарт был
рядом...}

Я жил в коммуналке на Саксаганского с мамой и бабушкой в одной большой комнате,
и у нас в ней помещался даже небольшой кабинетный рояль фирмы \enquote{Дидерихс} - ведь
бабушка в прошлом преподавала музыку. Но мне почему-то запомнилось, как на нем
иногда играла мама (совсем чуть-чуть, я бы сказал, по-любительски) разные
популярные вещи из классики, в том числе и \enquote{Турецкий  марш} Моцарта.

Иногда мы с мальчишками по холму соседнего заросшего бурьяном и заставленного
гаражами двора (ведь Киев город на холмах) добирались до музыкального училища
имени Глиера, что находилось на улице Толстого, и из его открытых окон
доносились звуки музыки - скрипка и фортепиано, и мне теперь кажется (хотя
полвека почти прошло), что там звучали и мелодии Моцарта!

Кроме того у моей мамы был, если можно так выразиться, \enquote{культ} его 40-й
симфонии, особенно ее первой части, и это фантастически прекрасное произведение
покорило и меня на всю жизнь!

А в моей коммуналке (это я добавлю аргументов в моем споре с внутренним
оппонентом) жил один уникальный сосед - Эммануил Аркадьевич Сливко, врач по
профессии, который классическую музыку любил и досконально знал, хотя сам
никогда в жизни не играл. У него я столько пластинок с классикой переслушал!
Так вот, до сих пор помню голубую коробку-альбом со скрипичными концертами
Моцарта в исполнении Давида Ойстраха и Берлинского филармонического оркестра!
Какие прекрасные вещи, какой великий скрипач! Впрочем, тогда я в нюансы
исполнительского мастерства не вникал, меня завораживал мелодизм Моцарта, и
особенно пленили меня 3-й и 4-й его концерты из той коробки...

Прошли десятки лет, и я, уже давно не живший в коммунальной квартире, открыл
для себя и полюбил и изумительные фортепианные концерты великого композитора:
18-й, 20-й, 21-й, 23-й, 24-й... В унылые вечера поздней осени, такие, как и
ныне мы часто видим за окном, столько раз - уже из колонок компьютера - слушал
я их в исполнении Рихтера - и на душе становилось легче и радостней...

И сейчас ещё вот что расскажу.

Однажды золотой осенью мы с женой вознамерились прогуляться по Саксаганского и
окрестным улицам, с тем чтоб завершить прогулку в парке Шевченко или
университетском ботсаду. Была отличная погода, но сумерки неотвратимо наползали
на короткий осенний день. 

И наверно, сам Люцефер, шелестя по-змеиному в глубях золотистых крон и вороша в
нашей памяти прошлое, коварно решил в тот день испытать наш многолетний союз на
прочность. В районе перекрестка Саксаганского и Красноармейской мне как
художнику вдруг припомнились мои некоторые натурщицы из почти забытого
прошлого... Хотя они были очень даже интеллигентными киевлянками. Возле улицы
Горького я был обвинен в позднем оформлении наших давних романтических
отношений (при этом стихи и картины в честь любимой в счёт не принимались)... А
подходя к улице Владимирской, я уже себя чувствовал обвиняемым на
межгалактическом женском суде - и только и мог повторять: \enquote{Да я же... Да это не
то, что ты подумала...}  

Вечер, как сказали в одном фильме, переставал быть томным.

И вдруг я произнес: \enquote{Послушай, это кажется Моцарт...}

Из окна старого здания (а это была, кстати, одна киевская музыкальная школа),
при повороте с улицы Саксаганского на Владимирскую, раздавались мелодичные,
немного грустные, но таки прекрасные звуки рояля! И это была чудесная Фантазия
до минор великого композитора... ( Ее, кстати, так проникновенно играет Эмиль
Григорьевич на Youtube.) 

Мы заслушались и уже, как мне кажется, вполне миролюбиво дошли до парка.

В общем, прогулка была спасена - уверен, благодаря великой музыке!..

Но история на этом не закончилась.

Во время гуляния в киевском парке я вспомнил непростую жизнь венского классика,
его короткие 35 лет жизни, могилу для бедных, где его предали земле... Он умер
в декабре, а что он чувствовал - в свою последнюю осень? Может, размышлял о
\enquote{Реквиеме},  который ему заказал таинственный \enquote{черный человек}?

И вообще, за что это нам, всему человечеству, с его злом, лицемерием и
пороками, так повезло, что золотое сердце Моцарта, его волшебная, излучающая
свет музыка и через века несёт нам надежду?

Взял я как-то \enquote{перо и бумагу} и попробовал передать в рифмованных
строчках его образ, а стих назвал - \enquote{Прогулка с Моцартом}.

Ведь хотя великий композитор жил в Вене, его присутствие - да! да! да! - я с
детства ощущал и в моем родном городе!

\headCenter{ПРОГУЛКА С МОЦАРТОМ}

\begin{multicols}{2}
\setlength{\parindent}{0pt}
\obeycr
Осень вела нашу жизнь в потемки, 
Но в глубине окна 
Моцарт, казалось, играл негромко 
И обращался к нам... 
\smallskip
Что ж, и не будем – о сокровенном, 
Лучше представь на миг: 
Век Восемнадцатый. Зала в Вене. 
Ноты и воск на них... 
\smallskip
Может, и он был тогда бессонным, 
Пил, может, грусть гоня. 
Так же, наверно, рыжие клёны 
В сини горели дня. 
\smallskip
И фейерверком его встречали, 
Лишь подышать придет... 
Он улыбался, но был печален - 
Времени видя лёт. 
\smallskip
Музыкой грезя, шуршал листвою  -
Звуков шлифуя власть -
И паутинку ловил рукою, 
Знавшей суму и страсть. 
\smallskip
День был погож, небеса так чисты, 
Воздух бодрящ слегка; 
Клены, пылая, роняли искры 
На париков снега... 
\smallskip
Грезил и вдруг озирался живо: 
Чья колокольцем – речь? 
(Бант на косице – вспорхнет игриво,
Взглядом – готов прожечь!..) 
\smallskip
Верно, казался повесой праздным
Чепчикам чинных бонн 
И признавался весьма опасным 
Для легковерных жен.
\restorecr
\end{multicols}

\ii{23_11_2021.fb.fb_group.story_kiev_ua.2.mocart_v_kieve.cmt}
