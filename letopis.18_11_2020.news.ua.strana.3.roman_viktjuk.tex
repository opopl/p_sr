% vim: keymap=russian-jcukenwin
%%beginhead 
 
%%file 18_11_2020.news.ua.strana.3.roman_viktjuk
%%parent 18_11_2020
 
%%url https://strana.ua/news/301582-smert-viktjuka-obsuzhdajut-v-seti-chto-pishut-radikaly.html
%%author 
%%author_id 
%%tags 
%%title 
 
%%endhead 

\subsection{\enquote{Родился украинцем - умер москалем}. Как националисты радовались смерти режиссера Романа Виктюка}
\label{sec:18_11_2020.news.ua.strana.3.roman_viktjuk}
\Purl{https://strana.ua/news/301582-smert-viktjuka-obsuzhdajut-v-seti-chto-pishut-radikaly.html}

12:30, 18 ноября 2020 
\index[deaths.rus]{Виктюк, Роман!режиссер!17 ноября 2020, Москва}

\ifcmt
pic https://strana.ua/img/article/3015/82_main.jpeg
caption Роман Виктюк умер 17 ноября. Фото: ТАСС 
\fi
 

После того как стало известно, что в Москве умер украинский
режиссер-легенда Роман Виктюк, радикалы начали писать в соцсети, что
скорбеть не нужно. Главная их претензия - Виктюк жил и работал в России. 

Правда, не вся \enquote{партия войны} радовалась смерти режиссера. Другие,
наоборот, использовали его имя, как доказательство \enquote{плохого} Донбасса,
приводя его отдельные цитаты.

Разбирались, как националисты восприняли смерть Виктюка и какие взгляды
высказывал при жизни сам режиссер.

\subsubsection{\enquote{Может, хорошо, что умер?}}

Лидер радикальной группы \enquote{Братство} Дмитрий Корчинский признал, что не
смотрел ни одного спектакля Виктюка, но предположил, что это хорошо, что
тот умер - ведь он работал в Москве.

\enquote{Пишут, что умер известный театральный режиссер Роман Виктюк. Я не видел
ни одного его представления, но доверяю тем, кто считал его талантливым.
Все семь лет московской военной агрессии против его Родины он продолжал
жить в Москве, продолжал в московских театрах ставить московские спектакли
для московитов. Может хорошо, что умер?} - написал Корчинский.

\ifcmt
pic https://strana.ua/img/forall/u/0/34/%D0%A1%D0%BD%D0%B8%D0%BC%D0%BE%D0%BA(380).JPG
\fi

В комментариях к посту Корчинского пишут о Виктюке, что он \enquote{родился
украинцем - умер москалем}, \enquote{будет похоронен в путинской России} и что он
\enquote{что был, что не было}.

При этом часть комментаторов отмечают, что Виктюк был гениальным
режиссером.

\ifcmt
pic https://strana.ua/img/forall/u/0/34/%D0%A1%D0%BD%D0%B8%D0%BC%D0%BE%D0%BA(381).JPG
\fi

Праворадикал Алексей Бык, который запомнился нападением на журналиста
Руслана Коцабу, а сейчас участвует в акциях \enquote{Нацкорпуса}, также написал,
что удивлен тому, что в Украине скорбят по Виктюку, ведь он жил и работал
в Москве.

\enquote{Всенародный плач по режиссеру Роману Виктюку, который от большой любви к
Украине жил, работал и умер в Москве. Люди, вы что, пьяные?} - написал
Бык.

\ifcmt
pic https://strana.ua/img/forall/u/0/34/%D0%A1%D0%BD%D0%B8%D0%BC%D0%BE%D0%BA(378).JPG
\fi

В комментариях у Быка уточнили, почему он сам скорбел по Егору Летову
(российский музыкант, лидер группы \enquote{Гражданская оборона}).

На что радикал ответил, что скорбеть было можно, поскольку Летов - не
украинец и не сидел в Москве во время войны. То есть это две главные
претензии к Виктюку.

\ifcmt
pic https://strana.ua/img/forall/u/0/34/%D0%A1%D0%BD%D0%B8%D0%BC%D0%BE%D0%BA(379).JPG
\fi

При этом некоторые, наоборот, позиционируют Виктюка как патриота,
например, часть \enquote{партии войны} цитирует его высказывание о \enquote{плохом
Донбассе}.

\enquote{Хочу жителям Донбасса сказать только одно: вы живете вне Бога, вы Его не
слышите. На секунду выключите все телеканалы, закройте глаза и побудьте в
тишине. Без зла, без насилия, без убийства. В этой тишине, приобщившись к
Небу, может быть, в вашей душе что-то произойдет. Должно произойти, если
вы украинцы. Если вы не граждане Украины, даже эта тишина вам не поможет,
тогда оставьте Украину в покое}, - привел цитату режиссера экс-министр
обороны и лидер партии \enquote{Гражданская позиция} Анатолий Гриценко.

\ifcmt
pic https://strana.ua/img/forall/u/0/34/%D0%A1%D0%BD%D0%B8%D0%BC%D0%BE%D0%BA(382).JPG
\fi

Цитату о Донбассе напомнил и консул Украины в Израиле Олег Вишняков.

\enquote{Умер львовянин, талантливый режиссер и выдающийся человек. Большой и
неповторимый Роман Виктюк. Даже в период тоталитаризма он никогда не
ставил спектаклей, которые \enquote{обслуживали систему}.

Комментируя в 2014 году ситуацию на Донбассе, Виктюк призвал местных
жителей выключить телевизоры, чтобы в тишине разобраться в происходящем, а
также рекомендовал тем, кто не считает себя гражданином Украины, оставить
страну в покое.

Высказывался он и за принятие закона о защите животных от жестокого
обращения.

Мне не хватит символов, чтобы описать всю его огромную театральную
деятельность ... Услышьте наши аплодисменты на небесах, маэстро! Надеюсь,
Вы насладились жизнью, как мы наслаждались вашим талантом}.

\ifcmt
pic https://strana.ua/img/forall/u/0/34/%D0%A1%D0%BD%D0%B8%D0%BC%D0%BE%D0%BA(383).JPG
\fi

\subsubsection{Жил в Москве, не любил Путина}

Отметим, что хотя режиссер действительно жил в Москве, о России после
развала СССР он отзывался неизменно плохо. Хотя работать в белокаменной
начал еще в 70-х.

Его постановка \enquote{Служанки} 1988 года стала знаковым событием в театральном
мире. Ее называли театральным манифестом. Это был первый спектакль театра
\enquote{Сатирикон} после смерти Аркадия Райкина. Все женские роли в постановке
играли мужчины.

В 90-х в той же Москве режиссер основал собственный театр Романа Виктюка. 

\ifcmt
pic https://strana.ua/img/forall/u/0/34/%D0%A1%D0%BD%D0%B8%D0%BC%D0%BE%D0%BA(377).jpg
caption Отреставрированный Дом культуры имени Русакова, в здание которого переехал Театр Романа Виктюка. Фото: РИА Новости
\fi


То есть в этом смысле \enquote{претензии} националистов понятны. Однако в
последние годы Виктюк активно ругал Россию и ее руководство.

О своем приезде в столицу России, на одном из памятных вечеров, Виктюк
вспоминал так:

\enquote{В Москву меня провожали, как в Сибирь. Упаковали все: перину, подушку,
ложки… А деньги всем домом зашили в сатиновые трусы... Все так прятали,
чтобы ни один москаль не мог найти, что дали ребенку из Западной Украины.
Перед самым отъездом я получил телеграмму от своих подруг, которые уехали
в Москву на два года раньше. \enquote{Не приезжай: таких, как ты, много}, -
говорилось там. Но я никому это послание не показал и поехал.

И вот стою я на вокзале без вещей, в одних тапочках, и не знаю, что делать
(вещи украли). Деньги достать не могу - это ж надо раздеваться… Попросил у
прохожих монетку позвонить, дозвонился, узнал, когда приходить на
вступительные экзамены. Сел в троллейбус и поехал… А вокруг - огни
вечерней Москвы. Проезжаю мимо домов, которые неподалеку от Кремля стоят,
и говорю себе: не смотри туда! Тебе там никогда не быть! А теперь я живу в
одном из этих домов…}

Виктюк жил в квартире принадлежащей ранее внуку Сталина - Василию
Бурдонскому. Ее помог получить актер Михаил Ульянов, который попросил
тогдашнего мэра Юрия Лужкова отдать ее известному режиссеру. Так Роман
Виктюк въехал в квартиру на Тверской. По этому поводу он называл себя
бандеровцем, который поселился в квартире главного бандита.  

В своих интервью режиссер неоднократно говорил, что делает все, чтобы
знали, что он --- украинец. На двери его квартиры были желто-голубые
ленточки. 

Критиковал Владимира Путина, заявляя, что тот \enquote{не от народа, а от черта}.
Черта он вспоминал, говоря о войне на Донбассе: \enquote{Для тех, кто начал эту
братоубийственную войну, нет никаких барьеров, нет Бога. Там черт}.

В одном из интервью он плохо отзывался о русских, которые освободили Львов
от гитлеровцев.

\enquote{Когда русские приехали в город, они грабили квартиры, находили длинные
ночные женские рубашки, одевали их как платья, надевали свои валенки, в
которых они приехали и в таком «прикиде» отправлялись в театр. Для них,
прибывших из самых дремучих российских лесов и совершенно не подозревавших
о наличии парфюма и нормального мыла, львовская действительность с её
кафе, ресторанами, обычаями и нравами, стала абсолютнейшим потрясением.
Так было!}, - говорил Виктюк.

Но при этом режиссер вспоминал, что украинские власти, в отличие от
российских, никогда не обращали на него внимания. 

\enquote{Настал день премьеры, и директор национального театра в Афинах спрашивает
меня, какой флаг повесить на театре --- русский или украинский?

Я отвечаю: \enquote{Той страны, посол которой приедет первым!}. Между тем скажу
вам, что посол Украины всегда молчал и никогда не поздравлял меня с
успехом, каким бы грандиозным он ни был, а вот посол России никогда не
забывал поздравить меня и проявить свое внимание.

Так вот, без пяти семь. Через пять минут --- начало премьеры, а корзина
цветов от посла России уже меня ждет. От украинского посла --- ни звоночка,
ни цветочка}, - рассказывал режиссер.

Над театром тогда подняли флаг России, хотя Виктюк был первым в то время
режиссером, который приехал в Грецию от Украины. Украинский флаг ему
подарили после спектакля - на сцене.

Также Роман Виктюк рассказывал, что когда был на гастролях в США во время
Оранжевой революции, то, выходя на сцену после спектакля, всегда
подчеркивал, что он украинский режиссер.

\enquote{Вы знаете, у американцев не принято ругать свою родину, страну, где ты
живешь. Для них это дико. Так вот я тоже не хочу этого делать. Потому что
я люблю Украину и ее людей и хочу, чтобы они жили в счастливой стране}, -
говорил Роман Виктюк.

Таким образом знаменитого режиссера менее всего можно было заподозрить в
симпатиях к Путину, российской власти и ее политике в отношении Украины.
Но в глазах националистов сам факт, что Виктюк жил и работал в последние
годы в Москве, является достаточным, чтобы порадоваться его смерти.
