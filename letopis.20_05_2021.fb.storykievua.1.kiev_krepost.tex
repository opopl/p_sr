% vim: keymap=russian-jcukenwin
%%beginhead 
 
%%file 20_05_2021.fb.storykievua.1.kiev_krepost
%%parent 20_05_2021
 
%%url https://www.facebook.com/groups/story.kiev.ua/permalink/1665999013596893/
 
%%author 
%%author_id 
%%author_url 
 
%%tags 
%%title 
 
%%endhead 
\subsection{Мой Киев - моя крепость}
\Purl{https://www.facebook.com/groups/story.kiev.ua/permalink/1665999013596893/}

Мой Киев - моя крепость.

Все, кто приезжает в Киев, начинают обзор города с Золотых Ворот -
символических \enquote{врат} столицы Киевской Руси. Однако, мало кто знает, что Киев
был действующей и очень сильной крепостью до середины ХІХ века. 

\ifcmt
  pic https://scontent-iad3-1.xx.fbcdn.net/v/t1.6435-0/p206x206/188821130_161886095942764_4888061821832039247_n.jpg?_nc_cat=100&ccb=1-3&_nc_sid=b9115d&_nc_ohc=TUJ5dJ-iO0wAX-Xi0Le&_nc_ht=scontent-iad3-1.xx&tp=6&oh=be853a164998a56d4d40be6003b82d2f&oe=60CA6AE2
\fi

Первые фортификационные  укрепления появились в Х веке. Намного большим был так
называемый „град Ярослава” в XI веке, с центром в Святой Софии. К его
фортификационной системе принадлежали и славные Золотые ворота. Впрочем, уже
тогда свои укрепления имели и Подол, и Копырив конец, и Печерские горы. К
сожалению, оборонительные сооружения древнего Киева не смогли противостоять
сокрушительному нашествию монголо-татар. 

\ifcmt
  pic https://scontent-iad3-1.xx.fbcdn.net/v/t1.6435-9/187955396_161886162609424_925147468846096922_n.jpg?_nc_cat=110&ccb=1-3&_nc_sid=b9115d&_nc_ohc=lLPKhxoaRi4AX_0hhUh&_nc_ht=scontent-iad3-1.xx&oh=8b5ea53c095f04b395d46fb3eefa2c84&oe=60CA69EE

	pic https://scontent-iad3-1.xx.fbcdn.net/v/t1.6435-9/187904892_161886269276080_2260395222727902220_n.jpg?_nc_cat=103&ccb=1-3&_nc_sid=b9115d&_nc_ohc=R0QVL1Kt9H0AX-AGW4Z&_nc_ht=scontent-iad3-1.xx&oh=8970d5a18324ff4aa306679f9f6463df&oe=60CCE58D

	pic https://scontent-iad3-1.xx.fbcdn.net/v/t1.6435-9/187655433_161886339276073_4476475247518106506_n.jpg?_nc_cat=101&ccb=1-3&_nc_sid=b9115d&_nc_ohc=P6H3bIyH8RcAX-zKoqY&_nc_ht=scontent-iad3-1.xx&oh=beacb45b1299c4ff4af2b91222c8d96c&oe=60CB7109
\fi

Во время польско-литовского владычества горожане начали укреплять наиболее
популярный район города - Подол. А над Подолом, на Княжеской горе вознесся
угрюмый Литовский Замок с пятнадцатью шестиугольными башнями. В 1482 году его
сжег крымский хан Менгли-Гирей. 

В средние века Киев находился в эпицентре борьбы трех враждующих стран -
Польши, Турции и России. Поэтому его оборона имела очень важное значение. В
1679 году казацкие войска под руководством гетмана Самойловича укрепили
земляной вал вокруг Печерской Лавры, поставили деревянный частокол с башнями и
бойницами, выкопали глубокий ров. Печерское укрепление соединил со
Старокиевским огромный вал - ретраншемент.

В начале XVIII века русский царь Петр І, готовясь к войне со Швецией, вновь
обратил внимание на Киев, как на один из основных опорных пунктов обороны
Малороссии. В 1706 году началось грандиозное строительство Печерской крепости.
Велось оно согласно наиболее современной к тому времени фортификационной
системе французского военного инженера Вобана. Руководил строительством гетман
Иван Мазепа. Вокруг Лавры появилась сложная система земляных валов с редутами,
бастионами, глубокими рвами. Печерский городок укрепили настолько хорошо, что в
него переехало все городское руководство и даже губернская канцелярия.

Киевская крепость расстраивалась на протяжении всего XVIII века.

В 1784-98 годах напротив входа в Киево-Печерскую Лавру, на месте разрушенного
Вознесенского женского монастыря, было построено грандиозное здание «Арсенала».
Здесь разместился штаб киевской военной инспекции, а позднее, в начале XIX в.
штаб Киевского гарнизона и канцелярия военного губернатора.

В первые годы ХІХ века генерал-губернатором Киева стал знаменитейший русский
полководец М.И.Кутузов. Не удивительно, что он начал работы по ремонту и
застройке оборонительных укреплений. А перед самым нашествием Наполеона на
Россию было возведено Зверинецкое укрепление в районе нынешнего Ботанического
сада Академии наук.

После окончания войны 1812 года русская военная верхушка хотела, было исключить
Киев из штата городов-крепостей, поскольку состояние укреплений желало лучшего.
За старую крепость вступился Великий Князь и будущий царь Николай Павлович. С
его легкой руки и началась новая строительная эпопея 1830-60-х гг.

Следует сказать, что Печерск первой половины XIX ст. представлял собой полную
противоположность нынешнему фешенебельному району. Это был уютный топкий
закоулок, почти недоступный для власти и полиции по причине полной
непроходимости дорог. Русский писатель Николай Лесков знавал места, которые «по
каким-то геологическим причинам…служили просторным вместилищем для стока
черноземной грязи, которая образовала здесь сплошное болото с вонючими озерами,
в которых плавали гуси и утки. У жителей этих забытых богом и правительством
мест, как правило, бывали «плохи пашпортишки». И жили они «неодобрительною и
даже буйною жизнью в стародавнем запорожском духе».

Чтобы освободить место для построения крепости предполагалось выселить его
колоритное население на Шулявку. По приказу генерал-губернатора Д.Г.Бибикова к
дряхлым печерским лачугам прибивались знаменитые „бибиковские доски” - дощечки
с запретом ремонта и указанием срока сноса строения. Тем не менее, доски
висели, а жители развалюх стойко держали оборону, пока не оказались окруженными
крепостными стенами.

Тем временем на Печерске шло эпохальное строительство. В основу был положен
проект известного военного инженера К.И.Оппермана. Старая крепость – цитадель,
была дополнена Васильковским и Госпитальным укреплениями, Лысогорским фортом,
многочисленными оборонительными казармами и круглыми башнями вдоль края
Печерской возвышенности. В результате был построен крупнейший в мире
каменно-земляной фортификационный комплекс площадью 700 гектаров.

К сожалению, не бывает ничего вечного. Одновременно со строительством крепости
шел бурный процесс развития военной техники. Крепость теряла свое
оборонительное значение уже в процессе сооружения и уже в 1897 году
превратилась всего лишь в крепость-склад. Только печально известный „Косой
Капонир”, часть Госпитального укрепления, продолжал функционировать как
политическая тюрьма - „киевский Шлиссельбург”.
