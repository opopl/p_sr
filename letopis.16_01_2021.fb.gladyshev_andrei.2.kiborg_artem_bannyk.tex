% vim: keymap=russian-jcukenwin
%%beginhead 
 
%%file 16_01_2021.fb.gladyshev_andrei.2.kiborg_artem_bannyk
%%parent 16_01_2021
 
%%url https://www.facebook.com/groups/2405359826402240/permalink/2837147383223480/
 
%%author 
%%author_id 
%%author_url 
 
%%tags 
%%title 
 
%%endhead 
\subsection{«Кіборг» Артем Банник і нині на передовій}
\label{sec:16_01_2021.fb.gladyshev_andrei.2.kiborg_artem_bannyk}
\Purl{https://www.facebook.com/groups/2405359826402240/permalink/2837147383223480/}

\ifcmt
  author_begin
   author_id gladyshev_andrei
  author_end
\fi

Вже за місяць після випуску із Національної академії сухопутних військ Артем
Банник відправився на війну командувати взводом. Відважному лейтенантові було
лише трохи за двадцять, у підпорядкуванні — бійці, більшість із яких «в батьки
годяться», кожен зі своїм характером, життєвими обставинами, досвідом... Та й
часу на здобуття авторитету на фронті — обмаль...

\ifcmt
  pic https://scontent-mad1-1.xx.fbcdn.net/v/t1.0-9/140015598_2882279428762906_4650172627280317013_o.jpg?_nc_cat=106&ccb=2&_nc_sid=825194&_nc_ohc=-5v4mj4ZdKYAX9C3ru4&_nc_ht=scontent-mad1-1.xx&oh=65475a6831c3bcd84794e414aa6f3783&oe=60293317
	caption «Кіборг» Артем Банник і нині на передовій, 16.01.2021
  width 0.4
\fi

За крайні шість років бачив стільки, що про його подвиги, мабуть, можна було б
написати не одну книжку... Тривала ротація в Донецькому аеропорту, Водяне,
Опитне, Піски, Красногорівка, Кримське, Новотроїцьке і ще безліч містечок
Луганщини та Донеччини... А там — штурми, перемоги, втрати, ухвалення рішень,
від яких залежать десятки життів... А ще бажання стати кращим, що привело до
проходження семимісячних командирських курсів у Штатах, і найкраща віддушина у
воєнному «нон-стопі» — народження синочка.

А коли, здавалось, що ось воно більш-менш спокійне життя — дружина, син,
службова квартира, ППД... перспективного офіцера перевели в іншу бригаду, де
він обійняв посаду комбата!

Артем, тепер вже капітан, кавалер ордена Богдана Хмельницького ІІІ ступеня — і
нині на передовій. А де ще бути бойовому комбату?

Слухаючи розповідь Артема, постійно хотілось перепитати, чи точно він 1993 року
народження. Надто серйозні речі він розповідав.

— Мабуть, на краще, що ми одразу після випуску пішли на фронт, — розмірковує
Артем Банник. — Адже ми не бачили тієї, «совкової», армії з усіма її
«принадами». Натомість одразу познайомились із фронтовиками, одягненими в
камуфляжі всіх країн світу, кожен з яких був готовий померти за Україну, з
честю і почуттям гордості.

Перше серйозне бойове завдання не змусило на себе довго чекати — участь у визволенні Авдіївки...

— Літо 2014-го, точної інформації про ворожу техніку немає, точної лінії фронту
так само... Завдання ставили «в полі» — «Ось карта, ось маршрут, тут проходите,
тут закріплюєтесь!». Ми заходили в Авдіївку серйозними силами — із трьома
танками та БМП. Звісно, не обійшлось без «форс-мажорів». Під час першої спроби
зайти в місто наш головний танк потрапив у яму, «роззувся»... Поки ми його
витягнули, поки «гусянку» одягнули, минуло більше ніж година. А ще на одній із
БМП виникла проблема з гарматою, навідник-оператор ніяк не міг справитись.
Кажу: «Давай я гляну! Та це ж елементарно!». Рукави закатав, все полагодив...
Не обійшлось і без ворожих снайперів, що чекали нас, але були вчасно знищені.
Ось так поступово я «обростав пір’ям», а віковий бар’єр між мною та підлеглими,
що у батьки годяться, зникав... У Авдіївці ми успішно справились із завданням —
зайшли, розгромили блокпост, пройшли через залізницю, вийшли до «Царської
охоти», закріпились, почали кругову оборону.

Після штурму Авдіївки багато воїнів підрозділу хотіли з’їздити у відпустки,
видихнути. Та не забарилось нове завдання — Донецький аеропорт. Щоправда,
керівництво пообіцяло, що підрозділ заходить туди усього на кілька днів, а
потім його замінять, і люди спокійно відпочинуть... Хто ж знав, що оце «кілька
днів» затягнеться на кілька місяців пекла.  — Перших два тижні нашого
перебування в ДАПі було не те, щоб спокійно, але терпимо — артилерія, обстріли,
ворожі БМП... Та поступово ситуація ускладнювалась, обстріли ставали
потужнішими, ставало зрозуміло, що для російських окупантів ДАП — питання
принципу, і вони ні перед чим не зупиняться. Під час навчання в академії у нас
завжди було своєрідне змагання між різними родами військ: десантники
переконували, що вони крутіші, піхотинці — що вони... Донецький аеропорт одразу
показав, що це все немає жодного значення, він усіх прирівняв під одну планку.
Важливо лише — чого саме ти вартий, яка ти людина. Буквально день-два тяжких
умов — і людей видно наскрізь, немов на рентген-апараті.

За кілька тижнів перебування в ДАПі Артема призначили командиром
загально-військового резерву, і список обов’язків офіцера ще більш розширився.
Це і відбиття атак, наступ, і евакуація поранених, і підвіз забезпечення...

— Там я вперше побачив, як танковий снаряд одним махом пробиває одразу сім стін, і ти залишаєшся стояти на відкритий місцевості... Окрім того, що нас громили з великою дальності важкою технікою, зблизька — закидали гранатами, то ще й працювали ворожі снайпери. Запам’ятався випадок, коли ми з головним сержантом проходили коридором, а назустріч — бойовик, без зброї, руки в карманах, став і зухвало дивиться на нас. А я ж гарячий, 21 рік, думаю вальнути його зразу! Сержант хапає мене за плече, ми падаємо на підлогу, і в цей момент спрацьовує снайпер з іншого боку! Усе за лічені секунди! Потім я вже зрозумів, що це працювала снайперська пара й вони просто чекали, поки ми виглянемо. Кажу: «Саня, ти врятував мені життя!».
Артем розповідає, що в ДАПі воювали, як проти професійних військових, так і проти тих, хто ледь не вперше взяв зброю до рук.

— Коли ворог відправляв новачків у наступ через поле, у той час, коли ми
захищені, у будівлі, це нагадувало комп’ютерну гру. Біжать у повний зріст,
невміло тримають зброю... Хлоп-хлоп — і всі лежать... Потім, коли ми обшукували
тіла, знаходили «деенерівські» документи та телефони з фотографіями з
Ростову-на-Дону, тобто чітко зрозуміло, звідки всі ці «добровольці» взялись.
Причому вже наступного дня по тому ж полю знову йшли «бійці». Вогонь! Встають
вже не сім, а чотири... Знову вогонь! Всі лежать! Наступного дня знову йдуть
тією ж дорогою... За їхніми тілами навіть ніхто не приходив. Та разом із
«добровольцями» були й професійні бойовики в хорошому обмундируванні, із
продуманим планом дій... Якось підходжу до вікна на другому поверсі, помічаю
знизу п’ятеро бойовиків, які тихенько обговорюють, як зайдуть через перший
поверх... Я бігом туди, гранату в руки, потім ще одну кинув для гарантії...

У ДАПі Артем врятував багатьох поранених, вчасно евакуювавши їх на БМП. Та, на
жаль, врятувати вдалось не всіх. На його очах загинув Герой України Сергій
Колодій під час відбиття танкової атаки противника наприкінці вересня.

— Він був командиром моєї роти. Сів за штурвал БМП, аби прикрити нас після
двогодинних перестрілок, бій тоді точився просто в коридорі, противник
підключив найбільшу за увесь час кількість озброєння та людей. БМП Колодія
знищив ворожий танк...

Непростим випробуванням був і вихід із ДАПу. Третя година ранку, українська
МТЛБ успішно проскочила найнебезпечнішу ділянку, та перед Пісками машина
«роззулась».

— Довелось під залпи «Градів» усією «компанією» одягати «гусянку». Коли вийшли
у Водяне, аж не вірилось, що вирвались живими з того пекла... Пізніше я багато
аналізував бої в Донецькому аеропорті. Можливо, можна було б вийти і з меншими
втратами або ж діяти дещо інакше, але ми вчились воювати...А далі — урочиста
зустріч, відпустки, відновлення техніки. Артема призначають командиром роти, і
починаються нові, не менш небезпечні фронтові «пригоди».

— Чого тільки не було за ці роки! І Жолобок, де відстань до ворога 70 метрів, і
позиція, що з трьох боків прицільно прострілювалась, і ще не одна ротація в
Авдіївці, яку визволяли у 2014-му, і Піски, і Новотроїцьке... Велику роль у
здобутті досвіду відіграло навчання з канадцями, коли батальйон здобув
сертифікацію НАТО, і моє навчання в Америці на капітанських курсах...

Наостанок Артем каже, що відчуває себе повністю на своєму місці. Ну не зміг би
він бути «кабінетним» комбатом. І хоч наразі, у зв’язку з перемир’ям, стало
дещо складніше розвивати батальйон, підтримувати навики підлеглих, постійно
займати їх роботою, залучати до тренувань, плани на майбутнє — амбітні.

А питання вікового бар’єра давно стало неактуальним... Так, можливо, на початку
війни Артему й не вистачало досвіду, можливо, були певні прогалини в навчанні,
та на фронті він став справжнім авторитетним бойовим командиром!
