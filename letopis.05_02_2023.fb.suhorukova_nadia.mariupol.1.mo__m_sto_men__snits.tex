%%beginhead 
 
%%file 05_02_2023.fb.suhorukova_nadia.mariupol.1.mo__m_sto_men__snits
%%parent 05_02_2023
 
%%url https://www.facebook.com/permalink.php?story_fbid=pfbid02oh52SU5zAjp1a8gdia6iHg41BUDKczUGteMzJuh2F6LF9NU1JYvDfGXndGbCyzzjl&id=100087641497337
 
%%author_id suhorukova_nadia.mariupol
%%date 05_02_2023
 
%%tags mariupol,mariupol.war
%%title Моє місто мені сниться ночами. Щасливе і затишне. Таким, яким  було раніше
 
%%endhead 

\subsection{Моє місто мені сниться ночами. Щасливе і затишне. Таким, яким  було раніше}
\label{sec:05_02_2023.fb.suhorukova_nadia.mariupol.1.mo__m_sto_men__snits}

\Purl{https://www.facebook.com/permalink.php?story_fbid=pfbid02oh52SU5zAjp1a8gdia6iHg41BUDKczUGteMzJuh2F6LF9NU1JYvDfGXndGbCyzzjl&id=100087641497337}
\ifcmt
 author_begin
   author_id suhorukova_nadia.mariupol
 author_end
\fi

Моє місто мені сниться ночами.

Щасливе і затишне. Таким, яким  було раніше. 

Я не хочу прокидатися. 

Мені здається, що воно реально існує. 

В  іншому вимірі. 

А зранку наступає час горя. 

Я розумію, що Маріуполь зараз чужий і майже мертвий. 

І  я застигаю між сном і дійсністю. 

Потім  згадую як було страшно. 

В ніч із 15 на 16 березня бомбардували без упину. 

Ми думали про те, як виїхати.

У нас була одна примарна машина.

Дев'ятеро людей і собака, яких потрібно вивезти.

І мінімум шансів дійти до гаража.

Він був у районі школи, а там лупили з усіх видів зброї і взагалі не
спинялися. 

Окупанти вибирали на місцевості квадрат і розбивали його до руїн.

У ту саму висотку влучали десятки разів. 

Присягаюся, там ніколи не було наших військових. Жодного. 

Там мешкали цивільні, які сподівалися, що бомбардувати перестануть і вони
вийдуть по воду або готувати їжу на багатті. 

А росіяни били навідліг. Ми в підвалі слухали ці звуки й задихалися від жаху:
скидалося на удари по голому тілу величезним батогом.

Війна кілька разів на день виконувала симфонію смерті.

Спочатку лунав скрегіт зубів велетня та залізні удари по даху.

Гадаю, то була репетиція, хтось готувався до виступу. 

А може, то вже була увертюра. Потім вступали "Гради". 

Тремтіла земля, тремтіли стіни.

Над нами летіли гігантські слепи  вбивці. 

Ми не могли зрозуміти куди.

Усюди ж люди. 

Для когось із них ця пекельна  музика стала останнім, що вони почули.

Для мене найстрашнішим був гуркіт літаків. 

Я накривала голову подушкою і мріяла оглухнути від сильного удару об землю. 

Земля прогиналася, а літак заходив на друге коло, і ми знову вмирали до
наступного вибуху.

Саме 15 березня ми почули абсолютно нові звуки із симфонії смерті: два
винятково потужні вибухи.

Від них усередині все перевернулося, голова стала величезною і порожньою, а
стіни підвалу вібрували ще якийсь час.

Я вирішила, що це зброя масового ураження. 

І з жахом уявляла, що побачу, коли вийду надвір.

Пізніше люди із селища біля Маріуполя розповіли, що по місту стріляли російські
кораблі. 

Нас убивали із землі, з повітря і з моря. 

Нас убивали звідусіль. 

Моє місто послідовно перетворювали на руїни.

Ми дедалі рідше виходили на поверхню. 

А вибираючись із Маріуполя лише змінювали одну смерть на іншу.

Моя подруга казала: \enquote{Ліпше загинути в дорозі, ніж здохнути у підвалі під
бомбами}.

%\ii{05_02_2023.fb.suhorukova_nadia.mariupol.1.mo__m_sto_men__snits.cmt}
