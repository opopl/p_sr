% vim: keymap=russian-jcukenwin
%%beginhead 
 
%%file 11_10_2021.stz.edu.lnr.lgaki.1.jalta_konkurs
%%parent 11_10_2021
 
%%url https://lgaki.info/dostizheniya/iz-yalty-v-moskvu-prolozhili-sebe-dorogu-studenty-kolledzha-akademii-matusovskogo
 
%%author_id 
%%date 
 
%%tags donbass,konkurs,jalta,rossia,krym,lgaki,studenty
%%title Из Ялты в Москву проложили себе дорогу студенты колледжа Академии Матусовского
 
%%endhead 
\subsection{Из Ялты в Москву проложили себе дорогу студенты колледжа Академии Матусовского}
\label{sec:11_10_2021.stz.edu.lnr.lgaki.1.jalta_konkurs}

\Purl{https://lgaki.info/dostizheniya/iz-yalty-v-moskvu-prolozhili-sebe-dorogu-studenty-kolledzha-akademii-matusovskogo}

На южном берегу Крыма подвели итоги Международного конкурса-фестиваля детского,
юношеского и взрослого творчества «Солнечный остров», в котором участвовали и
студенты музыкального отделения колледжа Академии Матусовского. Воспитанницы
преподавателя Татьяны Йовса показали себя отлично: не просто стали
победительницами в номинации «Эстрадный вокал», но и получили от организаторов
музыкального турнира приглашение на финал международных конкурсов-фестивалей
Фонда «Мир на ладони» «Кубок России», который пройдет 10-12 декабря в Москве.

\ifcmt
  pic https://lgaki.info/wp-content/uploads/2021/10/yC_OlczOj6E.jpg
  @width 0.7
  %@wrap \parpic[r]
\fi

А результаты девушек таковы:

Карина Батенёва – лауреат первой степени,

Юлиана Стороженко – лауреат первой степени,

Виктория Олейник – лауреат второй степени,

Елизавета Матычак – лауреат второй степени,

Анастасия Байда – лауреат второй степени.

Поздравляем!

 

Фото из личного архива участниц.
