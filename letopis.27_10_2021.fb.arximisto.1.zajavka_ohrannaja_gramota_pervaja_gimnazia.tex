%%beginhead 
 
%%file 27_10_2021.fb.arximisto.1.zajavka_ohrannaja_gramota_pervaja_gimnazia
%%parent 27_10_2021
 
%%url https://www.facebook.com/arximisto/posts/pfbid0NU1Yu1pLXsTjRKBpZfgVYTA1NgpEvajb1fQBX67S1fA7rKZ4SXYiCdN8v9mgn372l
 
%%author_id arximisto
%%date 27_10_2021
 
%%tags 
%%title Заявка на "охранную грамоту" для Первой гимназии будет рассмотрена 4 ноября 2021 года
 
%%endhead 

\subsection{Заявка на \enquote{охранную грамоту} для Первой гимназии будет рассмотрена 4 ноября 2021 года}
\label{sec:27_10_2021.fb.arximisto.1.zajavka_ohrannaja_gramota_pervaja_gimnazia}

\Purl{https://www.facebook.com/arximisto/posts/pfbid0NU1Yu1pLXsTjRKBpZfgVYTA1NgpEvajb1fQBX67S1fA7rKZ4SXYiCdN8v9mgn372l}
\ifcmt
 author_begin
   author_id arximisto
 author_end
\fi

Заявка на \enquote{охранную грамоту} для Первой гимназии будет рассмотрена 4
ноября 2021 года

\#новости\_архи\_города

Заявка \enquote{Архи-Города} на предоставление зданию Первой гимназии Мариуполя
(Мариинской женской гимназии) статуса объекта культурного наследия будет
рассмотрена на заседании Консультативного совета по вопросам охраны культурного
наследия при управлении культуры и туризма Донецкой областной государственной
администрации 4 ноября 2021 года. Об этом сообщили по телефону директору ГО
\enquote{Архи-Город} А. Марусову представители управления. 

В результате рассмотрения заявки управление культуры и туризма может одобрить
заявку, предоставить зданию Первой гимназии статус объекта культурного наследия
и направить в Министерство культуры и информационной политики Украины материалы
для признания здания памяткой архитектуры местного значения (с его занесением в
государственный Реестр недвижимых памяток Украины). 

В случае выявления недочетов в заявке управление может ее вернуть
\enquote{Архи-Городу} для доработки.

Напомним, что заявка была подана ГО \enquote{Архи-Город} в сентябре 2021 года. Она
включает в себя учетную документацию здания (историческая справка, материалы
фото фиксации здания, учетная карточка и т.д.), которая была разработана ООО
\enquote{НИПЦ архитектурного и градостроительного наследия} (директор – архитектор
Анатолий Изотов) летом 2021 года. 

Средства для разработки и подачи учетной документации (16 780 грн.) были
собраны двумя общественными организациями и одиннадцатью мариупольцами в июне
2021 года.

Текст учетной документации доступен по следующим ссылкам: 

\url{https://cutt.ly/7WGT1BF} (pdf)\footnote{Internet Archive: \url{https://archive.org/details/doc.2021.kyiv.oblikova_kartka.mariupol.mariinska_zhinocha_gimnazia}} или \url{https://cutt.ly/CWGTHUT} (Word, Winrar).

\#мариинке\_охранную\_грамоту
