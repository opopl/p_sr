% vim: keymap=russian-jcukenwin
%%beginhead 
 
%%file 02_08_2022.stz.news.ua.donbas24.1.serce_mrpl_mdu_dopomagaje_mrplci.txt
%%parent 02_08_2022.stz.news.ua.donbas24.1.serce_mrpl_mdu_dopomagaje_mrplci
 
%%url 
 
%%author_id 
%%date 
 
%%tags 
%%title 
 
%%endhead 

Ольга Демідко (Маріуполь)
02_08_2022.olga_demidko.donbas24.serce_mrpl_mdu_dopomagaje_mrplci
Маріуполь,Україна,Мариуполь,Украина,Mariupol,Ukraine,Kiev,Kyiv,Київ,Киев,Mariupol.MSU,Маріуполь.МДУ,Мариуполь.МГУ,Освіта,Образование,Education,date.02_08_2022

Серце Маріуполя — Маріупольський університет — активно допомагає постраждалим маріупольцям (ВІДЕО)

Журналіст і блогер Микола Осиченко завітав до Гуманітарного штабу
Маріупольського університету і розповів про основні напрями його роботи

Гуманітарний штаб Маріупольського державного університету працює вже місяць і
за цей час встиг допомогти багатьом маріупольцям. Микола Осиченко, блогер,
журналіст та громадський діяч у своєму блозі представив екскурсію штабом, де
висвітлив головні напрями роботи Гуманітарного штабу Маріупольського
університету щодо допомоги маріупольцям.

Хто кординує діяльність штабу?

Блогер поспілкувався з постійними працівницями штабу — доцентом Маріупольського
університету Анастасією Трофименко та депутатом Маріупольської міської ради від
«Слуги народу» Катериною Хлименковою.

«Майже одразу, коли університет розпочав роботу, ми зрозуміли, що до Києва
приїжджають наші студенти, співробітники, які потребують нашої підтримки, нашої
допомоги і почали шукати цю допомогу. Спочатку ми шукали тільки для наших
співробітників та студентів, але згодом до нас почали звертатися й інші
маріупольці. Саме тому ми відкрили такий Гуманітарний штаб, де кожен
маріуполець зможе таку підтримку отримати», — розповіла Анастасія Трофименко.

«Допомагати людям — це те, що зараз рятує і робить тебе сильним. Наш штаб якраз
про маріупольців. В день може прийти 50 і більше людей», — додала Катерина
Хлименкова.

У штабі на волонтерських засадах працюють як працівники Маріупольського
університету (викладачі, студенти), так і всі небайдужі люди, які мають
можливість у вільний від роботи час допомагати постраждалим маріупольцям.

Основні напрями роботи штабу

Микола Осиченко розповів, що у штабі маріупольцям можуть допомогти в декількох
напрямах. Це:

— одяг (як літній, так і зимовий);

— продуктові набори;

— дитяче харчування;

— препарати побутової хімії;

— засоби особистої гігієни;

— іграшки для дітей;

— посуд;

— ковдри, пледи, подушки, постіль, рушники.

Також можуть допомогти з працевлаштуванням, адже волонтери штабу дізнаються, чи
є вільні вакансії в Києві. Маріупольці, які втратили домівки дуже вдячні за
надану допомогу і підтримку Гуманітарному штабу Маріупольського університету.

«Ми дуже вдячні за розуміння, за допомогу, за ту турботу, яку ми отримуємо від
працівників штабу. Питають, що необхідно, все це фіксують та передзвонюють,
якщо потрібно. Тут дуже привітні та добрі люди», — підкреслили маріупольці, які
відвідали Гуманітарний штаб МДУ.

Хто допомагає Гуманітарному штабу МДУ?

Гуманітарному штабу Маріупольського університету допомагають Фонд Ріната
Ахметова, Благодійний фонд Вадима Новинського та відомі маріупольці. Зокрема,
добрим другом штабу є відомий український богатир, активний громадський та
політичний діяч Олександр Лашин.

«Великим другом для всього нашого штабу є Олександр Лашин. Нещодавно він привіз
нам багато рюкзаків від американських пілотів для маріупольських дітей, які в
нас, як гарячі пиріжки, розлетілися прямо першого ж дня», — зазначила Анастасія
Трофименко.

«Це один з маріупольських штабів. Ми, звісно, товаришуємо. Чим можу, завжди
допомагаю. Зараз приїхав спитати, що ще потрібно. Ми плануємо виїзд за кордон
за гуманітарною допомогою. І обов'язково щось привеземо до Гуманітарного штабу
МДУ для наших маріупольців. Зі штабу завжди йду в чудовому настрої, адже мені
не соромно дивитись маріупольцям в очі і я завжди радий поспілкуватися з
жителями мого міста», — прокоментував Олександр Лашин.

Періодично до штабу приїздить і Кирило Папакиця, помічник народного депутата
Вадима Новинського, щоб дізнатися, що ще потрібно маріупольцям.

«Ми постійно на зв'язку з працівниками штабу, які здійснюють допомогу
маріупольцям. Ми розуміємо потреби наших жителів, а їх ще дуже багато. Фонд
Вадима Новинського є надійним партнером Гуманітарного штабу МДУ. Я особисто, як
випускник цього університету і маріупольчанин, стараюся бувати тут частіше,
оскільки в штабі постійно зустрічаєш когось з мешканців мого рідного міста і це
дуже гріє душу. Маріуполь в нашому серці завжди був, є і буде і необхідно й
надалі старатися допомагати нашим маріупольцям», — поділився думками Кирило
Папакиця.

Як можна отримати допомогу?

Потрібно прийти до Гуманітарного штабу, який розташований за адресою: Київ,
вул. Преображенська, 6 в будні з понеділка по п'ятницю з 10:00 до 17:00, в
суботу з 11:00 до 15:00. У штабі необхідно зареєструватися (принести паспорт,
довідку переселенця), щоб 1 раз на місяць приходити до Штабу та отримувати
необхідну допомогу. Микола Осиченко зустрівся з ректором Маріупольського
державного університету Миколою Трофименком, щоб дізнатися, як з'явилася ідея
створити Гуманітарний штаб.

«До нас зверталося і звертається дуже багато наших друзів-партнерів і для того,
щоб систематизувати всю допомогу і бажання допомогти нашим співробітникам,
викладачам та студентам, їх сім'ям, ми відкрили Гуманітарний штаб, який
допоможе кординувати роботи і розширили його діяльність для допомоги всім
маріупольцям. Дійсно, Маріупольський державний університет — це центр
відновлення Маріуполя і особливо в столиці. Ми говоримо, що серце Маріуполя в
Києві б'ється на базі Маріупольського державного університету і в
Маріупольському університеті. Тому абсолютно логічним є створення такої
структури, яка надає допомогу нашим співгромадянам в столиці. В штабі працюють
волонтери, які абсолютно безкоштовно приймають всю допомогу від охочих
допомогти, і надають її всім маріупольцям, які того потребують», — розповів
ректор МДУ Микола Трофименко.

Нагадаємо, раніше Донбас24 розповідав про Луганську клінічну лікарню, яка
відкрилась у Дніпрі.

ФОТО: з відкритих джерел.
