% vim: keymap=russian-jcukenwin
%%beginhead 
 
%%file moje.prizyv
%%parent moje
 
%%url 
 
%%author_id 
%%date 
 
%%tags 
%%title 
 
%%endhead 

\section{Обращение жителей города Киева относительно нынешней российско-украинской войны}

Добрый вечер. Не знаем, как лучше начать этот текст, но давайте все-таки
как-нибудь начнем. Сейчас вечер 28 февраля 2022 года, мы находимся в городе
Киеве, и как мы все знаем, уже вовсю идет по всей территории Украины
кровопролитная война. Кровопролитная, ужасная война. Война на истощение, война
просто на убой. Война, которую начала Российская Федерация против Украины. Да,
это ужасно, что вторжение, которое многие считали абсолютно мнимым, таки
состоялось. Мы, если честно признаться, смеялись над непрерывным потоком
сообщений в западных и украинских СМИ о готовящемся вторжении, и мы ничуть не
верили, что война таки произойдет. Это мнимое вторжение уже стало просто мемом
в то довоенное время, с которого как будто прошла уже целая вечность, хотя на
самом деле прошло всего несколько дней. Да, знаете, бывает так, что иногда один
день, даже один час проносится как целый год. Да, мир резко разделился на до, и
после, так же как для Киевлян 1941 года, например. Так вот... то, во что
абсолютно не верилось, таки состоялось. И каждый час, каждый час, где-то кто-то
кого-то убивает, калечит, жжет. Горят здания, горит земля... и плачут
человеческие души...  кто-то плачет в подвале, кто-то орет от боли... и кто-то
просто умирает, потому что пуля попала в печень или селезенку, и его не успели
спасти, не успели довезти в госпиталь, и все прекрасные планы на жизнь
обнулились.  Жена, дети, работа, - все исчезло навсегда... Обычный человек...
Ну вот, скажем...  Иван Петрович, 52 лет, живет в Киеве всю жизнь... Мама и
папа - тоже киевляне.  Мама работала учительницей, а папа - на заводе Арсенал,
еще во времена СССР, оптику разрабатывал для космических стартов СССР.  А сам
Иван Петрович работал предпринимателем... держал велосипедный магазин... пошел
в тероборону с начала войны, ведь это его родной город, он тут вырос, на
Русановке или же Вознесенке... Высокой геополитикой не интересовался, просто
примерный семьянин, трое детей, жена, и старенькая мама.  Получил автомат, и
пошел защищать Город и Отчизну. Или же вот... В Киеве семья на машине недавно
пыталась уехать, их расстреляли, - это вы можете прочитать в новостях. Может
быть наши, может быть ваши, да какая разница... И что, что тут такого, скажете?
А вот и нет человека, нет людей, вот и все... Люди что-то там планировали,
мечтали, но все оборвалось в один миг...  Знаете, песня такая есть - Есть
только миг, между прошлым и будущим... и этот миг называется Жизнь. Вы ж ее в
России тоже поете, правда?  Это песня композитора Александра Зацепина на слова
Леонида Дербенёва, написанная для советского фильма «Земля Санникова» (1973).
Звучала также в телесериале \enquote{Перевал Дятлова} (2020). Наиболее
известные исполнители песни: Олег Анофриев, Олег Даль, Николай Расторгуев,
Михаил Боярский, Лев Лещенко и Игорь Наджиев.

\raggedcolumns
\begin{multicols}{2} % {
\setlength{\parindent}{0pt}
\obeycr
Призрачно все в этом мире бушующем
Есть только миг за него и держись
Есть только миг между прошлым и будущим
Именно он называется жизнь

Вечный покой сердце вряд ли обрадует
Вечный покой для седых пирамид
А для звезды что сорвалась и падает
Есть только миг ослепительный миг
А для звезды что сорвалась и падает
Есть только миг ослепительный миг

Пусть этот мир вдаль летит сквозь столетия
Но не всегда по дороге мне с ним
Чем дорожу чем рискую на свете я
Мигом одним только мигом одним
Счастье дано повстречать иль беду еще
Есть только миг за него и держись
Есть только миг между прошлым и будущим
Именно он называется жизнь

Есть только миг между прошлым и будущим
Именно он называется жизнь
\restorecr
\end{multicols} % }

И жизнь-то человеческая бесценна... Вы ж там в России поете песню Жить (песня и
клип по идее композитора Игоря Матвиенко) целым коллективом артистов, правда?

\ifcmt
  tab_begin cols=2,no_fig,center
		 @caption Киевское метро, мирное время

     pic https://avatars.mds.yandex.net/i?id=4ec9211cea2cb58ee6ce92ab1a6c3a1f-5870172-images-thumbs&n=13
		 pic https://ic.pics.livejournal.com/deletant/25270605/1797309/1797309_original.jpg
  tab_end
\fi

Так вот... Идет война, которая унесла, насколько мы можем судить по непрерывным
сообщениям в СМИ, уже тысячи жизней. Братоубийственная война... И счетчик
унесенных жизней, счетчик увечий, ранений, счетчик искалеченных судеб
непрерывно растет. Говорят, что воинов Российской Армии убито уже около трех
тысяч, а сколько убито граждан Украины, наших храбрых Защитников, наших
настоящих Богатырей, мы не знаем, но думаем, что тоже немало. 

\ifcmt
  tab_begin cols=2,no_fig,center
		 @caption Киевское метро, Киев, война, 2022

		 pic https://storage.myseldon.com/news-pict-9f/9FD1B5AE3D644CDE418BFE2D721339A6
     pic https://cs12.pikabu.ru/post_img/2022/02/27/2/og_og_1645920003214027623.jpg
  tab_end
\fi

Позвольте представиться. Мы - Киевляне, жители тысячелетнего города Киева,
Матери Городам Русским, великого Города, имеющего фундаментальное,
цивилизационное значение для Украины, России и Беларуси, в котором находятся
наши тысячелетние святыни - Киево-Печерская Лавра с пещерами и церквями, а
также музеем изобразительного исскуства, где выставлены картины Катерины
Билокур, потом, София Киевская с Орантой и графити, Выдубицкий монастырь...
Кириловская церковь... и мы сейчас сидим взаперти в одной из киевских
квартир... в одном из районов Киева. У нас относительно тихо на данный момент,
и слава Богу. Есть электричество, водопроводная вода, у нас есть запасы
продуктов на несколько недель, так что мы пока что еще не умираем с голоду, и
вообще живы. Просто живы, и это уже круто.  В наш дом не попала ракета, как в
высотный дом по Лобановского 6 (называвшийся раньше проспект Краснозвездный),
вокруг пока что не свистят пули, и знаете, такое ощущение, что вообще почти
ничего не происходит, если не читать сообщения из интернета и телеграма.  Днем
светит солнышко над нашим Вечным Городом, Городом Ярослава Мудрого и Владимира
Крестителя... Городом Булгакова и Пушкина, Гоголя и Высоцкого, Анны Ахматовой и
Николая Амосова...  Тысячелетнего Города Киева, в котором столько всего
происходило на протяжении тысячелетий, Городом... который столько раз был в
огне, который столько раз подвергался разорению и разграблению, и тем не менее,
он каждый раз возрождался.  Возрождался, чтобы стать еще красивее. В 1240 году
этот Город был сожжен дотла нашествием Хана Батыя... а сейчас... кто помнит о
татаро-монголах?  Да никто...  потому что Город возродился, и стал еще краше, и
намного больше, чем во времена Батыя.  Никто не помнит... кроме пожелтевших
страниц летописей, и кроме самого Города, потому что Город помнит все... Дело
давно минувших дней, преданья старины глубокой, как писал когда-то Пушкин... И
несмотря на войну, мы надеемся, он останется таким же красивым, как и раньше.
Что его не успеют уничтожить в нынешней войне, и что разум восторжествует над
безумием.  И кстати, Киев - это Город, который во многом строила Российская
Империя, если вы помните историю, - так, например, Андреевскую церковь в свое
время построил архитектор Б. Ф.  Растрелли по приказу Императрицы Елизаветы
Петровны в 1749-1754 годах на том месте, где, по преданию, апостол Андрей
Первозванный во время его путешествия воздвиг крест. Как вы может быть знаете,
об этом есть запись в Повести Временных Лет, который давным-давно написал наш
преподобный Нестор Летописец, чьи мощи на протяжении тысячелетий лежат в
ближних пещерах Киево-Печерской Лавры, и куда мы с удовольствием периодически
ходим, держа свечки в руках.  Может быть, знаете, в этих пещерах лежат богатырь
Илья Муромец, а также Агапит Печерский, врач безмездный, и много-много других
святых, которые в свое время всю жизнь проводили в пещерах и затворах, молясь
Богу за других людей и за спасение своих душ.

\ifcmt
  tab_begin cols=3,no_fig,center
     pic https://avatars.mds.yandex.net/i?id=f6d73333003dd92e4f1326c33f8df84c-5693613-images-thumbs&n=13
		 @caption Андреевская церковь, город Киев

		 pic https://medru.su/wp-content/uploads/2019/09/41_6_crm.jpg
		 @caption Императрица Елизавета Петровна

		 pic https://www.culture.ru/storage/images/f4014625-370f-5b3d-bc08-197aebf308a8
		 @caption Франческо Бартоломео Растрелли
  tab_end
\fi

Так вот. Снаружи поют птички, предвещая наступления скорой весны... как будто
жизнь идет своим чередом до начала войны, как будто вообще ничего не
происходило. Кроме того, что улицы пустынны, что наш район опустел, нам
кажется, что как будто все такое же самое. Кроме того, что... вдалеке воет
сирена, и слышны звуки взрывов.  И если не брать во внимание сирену, которая
призывает горожан спускаться в бомбоубежище, то может показаться... что это
просто обычный салют. У нас обычно в Киеве на праздники делают классные салюты,
ну например, на День Киева, или же когда мы празднуем победу футбольного клуба
Динамо Киева. Кроме того, у нас также есть гостиница Салют, у нее немного
вычурная архитектура, и она находится возле Дворца Детей и Юношества (Київський
палац дітей та юнацтва), возле Площади Славы, там, где также есть Аллея Славы и
Вечный Огонь Неизвестному Солдату. Чуть далее находится Музей Голодомора, - а
еще чуть дальше, - если проехать по улице Лаврской, - будет Киево-Печерская
Лавра и музей Великой Отечественной Войны.

\ifcmt
  tab_begin cols=2,no_fig,center
     %pic https://reservehall.com/images/restourants/initially/Sallut_17_1551951024.png
		 pic https://novate.ru/files/u34692/TheHotelSalutinKiev.jpg
		 @caption Гостиница Салют, город Киев

		 pic https://sun9-68.userapi.com/c636322/v636322902/24dc8/r1qc77wD26c.jpg
		 @caption Дворец Детей и Юношества, город Киев

  tab_end
\fi

Однако мы знаем, сколько сейчас несчастий происходит по всей территории
Украины. Ну что ж, надо что-то делать. Надо что-то делать, поскольку похоже все
движется к тому, что мы - украинцы и россияне, на самом деле будучи одной
крови, поскольку мы все славяне, просто перестреляем, перережем друг друга в
кровавом месиве, поскольку с обоих сторон идет мощная поддержка ресурсами,
оружием, людской поддержкой, информационной пропагандой. С нашей стороны идут
сообщения о том, что мы мочим врага, что мы обязательно победим, что наше дело
правое, что оккупант будет изгнан, и что вы все сгинете в аду. С вашей стороны
пишут немного другое, но какова истинная картина - очень трудно понять.  У нас
о войне пишут на русском и украинском языках, у вас - на русском.

С одной стороны Украине помогает - Запад, оружием, информационной поддержкой, а
также тот неоспоримый факт, что эта война для нас - это война за Отчизну, это
война справедливая, Отечественная. Да, Украинская Отечественная Война. И за нас
сейчас вообще весь мир, от Австралии до Нью-Йорка, от Лондона до Сиднея, от
Парижа до Берлина. Многие смеялись над мантрами политиков, что с нами весь мир,
но это сейчас действительно так. Абсолютно простые люди, - не политики, -
выходят и протестуют.  Быль стала явью... Так же как и мантра про злобных
москалей и страну-агрессора, увы...  Также многие россияне и беларусы против
войны, но к сожалению, их голос пока еще мало слышим в Российской Федерации и
Республике Беларусь. И как бы кто не называл эту войну в Российской Федерации,
например используя термин \enquote{cпециальная военная операция на Украине}, -
так, вроде Президент Российской Федерации Владимир Владимирович Путин ее
называет, - это война для нас есть священная война за Отчизну, а для Российской
Федерации - это война захватническая, несправедливая, сколько бы вы у себя не
повторяли про нацистов и бандеровцев. Да, действительно в Украине есть много
проблем, связанных с языком, верой, закрытием сми, а также с тем, что власти
Украины, как и многие ее жители, упорно не замечали кровоточащей раны Донбасса,
и никак не хотели ее лечить, никак не хотели мира... но... эта сегодняшняя
война началась, это уже факт, и начали ее именно вы, Российская Федерация,
сильная, мощная Держава, огромная Держава, обладающая правом вето в ООН, и
также ядерным потенциалом, способным уничтожить все живое на Земле по многу
раз. А раз Вы ее начали первыми, то все поменялось.  Нет уже больше ни
бандеровцев, ни националистов, ни ватников, нет больше проукраинских и
пророссийских партий, а есть захватчики, и те, кто по зову сердца и души воюет
за свою Отчизну. И те, кто защищает Родину, говорят и по-русски, и на
украинском, и сейчас им, - несмотря на все предыдущие эмоциональные дискусии, -
вообще не до того, красивее русский язык украинского, или же то, насколько
русский язык относится к северным племенам мокша/меря.

...Россия использует все свои ресурсы, поставив на карту, все что у России
есть. Россия играет ва-банк, и Вы пойдете до конца. Мы вас хорошо понимаем,
Россияне. У вас есть весомые аргументы относительно того, что НАТО угрожает вам
ракетами, что есть угроза того, что ракеты НАТО будут размещены на территории
Украины, и что время подлета к Москве будет весьма незначительным. Это мы
хорошо понимаем. 

С одной стороны,
