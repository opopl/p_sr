% vim: keymap=russian-jcukenwin
%%beginhead 
 
%%file 29_12_2020.news.ru.vesti.1.argentina_covid_vaccine_sputnik_v
%%parent 29_12_2020
 
%%url https://www.vesti.ru/article/2505063
 
%%author 
%%author_id 
%%author_url 
 
%%tags 
%%title Аргентина третьей в мире начала всеобщую вакцинацию от ковида
 
%%endhead 
 
\subsection{Аргентина третьей в мире начала всеобщую вакцинацию от ковида}
\label{sec:29_12_2020.news.ru.vesti.1.argentina_covid_vaccine_sputnik_v}
\Purl{https://www.vesti.ru/article/2505063}

\index[rus]{Вакцина!Спутник V!Аргентина}

\ifcmt
  pic https://cdn-st1.rtr-vesti.ru/vh/pictures/xw/308/564/6.jpg
	caption Аргентина третьей в мире начала всеобщую вакцинацию от ковида
\fi

В Аргентине\Furl{https://smotrim.ru/article/2503176} началась вакцинация от коронавируса российским препаратом "Спутник
V".

Как сообщает РИА Новости,\Furl{https://ria.ru/} кампания стартовала в 9:00 (15:00
по московскому времени) на всей территории страны. В первую очередь вакцину
получат аргентинские медики, работники отделений интенсивной терапии. Также
одним из первых привился губернатор провинции Буэнос-Айрес Аксель Кисильоф.

На прошлой неделе 300 тысяч доз вакцины были доставлены в Аргентину. Как
сообщил производитель препарата Российский фонд прямых инвестиций, эта поставка
является частью от общего объема в 10 миллионов доз. В понедельник вакцина была
отправлена в аргентинские провинции.

Таким образом, Аргентина стала третьей страной, начавшей всеобщую вакцинацию.
Первая – Россия, вторая – Белоруссия, куда была доставлена первая партия
"Спутника V".\Furl{https://smotrim.ru/article/2504946}

