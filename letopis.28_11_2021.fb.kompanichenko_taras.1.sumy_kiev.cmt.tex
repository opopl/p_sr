% vim: keymap=russian-jcukenwin
%%beginhead 
 
%%file 28_11_2021.fb.kompanichenko_taras.1.sumy_kiev.cmt
%%parent 28_11_2021.fb.kompanichenko_taras.1.sumy_kiev
 
%%url 
 
%%author_id 
%%date 
 
%%tags 
%%title 
 
%%endhead 
\subsubsection{Коментарі}

\begin{itemize} % {
\iusr{Kateryna Dovgan}
Дякуємо, Тарасе, Лицаре Світла!

\iusr{Olena Otrokh}

Я знаю купу людей які наче і розуміють, що війна і росія ворог, але фільми,
музика і товари то ж геть інше. Мені хочеться взять в одну руку весло, в
іншу-лопату @igg{fbicon.face.steam.nose} 

\iusr{Гандзя Осадча}
\textbf{Olena Otrokh},  @igg{fbicon.face.monocle}  @igg{fbicon.face.smirking}  мо ліпше екскаватор⁉️

\iusr{Любов Швайка-ширко}

Не можна так казати... але поки все совєцьке не виздихає!!! Їхала на похорон
хрещеника, учасника російсько- української війни в рідний Степань і маршрутка
Рівне- Кричільськ на зауваження за роспопсу, чуть не викинули...

\begin{itemize} % {
\iusr{Лена Ровенская}
\textbf{Любов Швайка-ширко}
Треба писати зразу на управління транспорту. Там керівник сам Герой війни, Кіборг
Федір Мисюра

\iusr{Любов Швайка-ширко}
\textbf{Лена Ровенская} керівник - це рівень свідомості у кожного в голові... там були пасажири і всі виступили проти мене...

\iusr{Лена Ровенская}
\textbf{Любов Швайка-ширко}
Як мені це знайомо....
Соромно перед тими, хто за таке насЄлєніє гине....

\iusr{Роман Павлюк}
\textbf{Любов Швайка-ширко} Є рускій реп.
Так що не все так просто.

\iusr{Nadia Stojko}
\textbf{Любов Швайка-ширко} можна казати!!!!

\ifcmt
  ig https://scontent-frx5-2.xx.fbcdn.net/v/t39.1997-6/p480x480/91521538_1030933857302751_5093925307199520768_n.png?_nc_cat=1&ccb=1-5&_nc_sid=0572db&_nc_ohc=E-EmWZR18LIAX_MAwr4&_nc_ht=scontent-frx5-2.xx&oh=fe47da42598da22510d1baf5d8f5a001&oe=61AFDEB7
  @width 0.2
\fi

\iusr{Любов Швайка-ширко}
\textbf{Лена Ровенская} За Україну вони борються!

\iusr{Лена Ровенская}
\textbf{Любов Швайка-ширко} як не сумно то теж Україна....


\end{itemize} % }

\iusr{Таня Лобода}

Треба доброї волі держави заборонити все це гімно і пустити його платними
каналами. Та наші державні "мужі" його і продукують і мають за це зелену вулицю
там запорєбриком для свого лайна.

\iusr{Андрей Фищук}

Викликати в ОДА перевізників. Поговорити із ними. Розказати про законодавство.
Про війну із Росією. Це не рівень водіїв.

\begin{itemize} % {
\iusr{Олександр Кондратюк}
\textbf{Андрій Фіщук} найімовірніше, підбірки цієї кінохалтури теж не водії складають, а хтось з адміністрації фірми-перевізника

\iusr{Андрей Фищук}
\textbf{Rostyslav Martynyuk} давати люлей треба не виконавцю, а господарю. Власника треба приводити до тями.

\iusr{Oleksiy Furmanov}
\textbf{Andriy Fishchuk} у всіх ОДА головують ватники, призначені таким самим президентом ватником, обраним такими самими ватниками. Це хіба що їх викликать щоб разом вздрочнуть на мінкультовскіх Сватів.

\iusr{Андрей Фищук}
\textbf{Oleksiy Furmanov} в Сумах Голова ОДА - \textbf{Dmytro Zhyvytskyy} . І він зовсім не ватник.

\iusr{Микола Цимбалюк}
\textbf{Андрій Фіщук} тобто, національна свідомість і просто людська гідність - то не рівень водіїв!?

\iusr{Олександр Кондратюк}
\textbf{Микола Цимбалюк} Припускаю, водії там — умовні виборці зеленського.

\iusr{Rostyslav Martynyuk}
\textbf{Микола Цимбалюк} 

це свідоме катування пасажирів. Робиться роками й неухильно. Власне тому НІКОЛИ
не пересувався маршрутками. Бо цілком міг би заїхита в потилицю того хруня за
відповіді в такому ключі. І сів би я, а не він.
@igg{fbicon.face.grinning.smiling.eyes} 

\iusr{Гандзя Осадча}
\textbf{Rostyslav Martynyuk}, 

не бачу нічого смішного, даруйте. На мій хлопський розум тут наявна складніша
психологічна проблема, як наслідок знецінення праці та панування штучного
\enquote{дефіциту}.

Працівники (водії, продавці, побутовий сервіс) з якогось дива вважають свою
роботу - персональною послугою, майже благодійністю до клієнта.

Саме тому побутує зневажливе ставлення. Тількі фінансови визиски можуть
поліпшити ситуацію!

\iusr{Андрей Фищук}
\textbf{Ганна Осадча} сто відсотків. Це психологічний пережиток совка
\end{itemize} % }

\iusr{Євген Гапоненко}

Нічого не має відбуватися у них в голові і не відбудеться. Ринок споживання
працює не так. У цих людей немає ніяких роздумів і співчувань. Це має бути
спеціально спланована, розроблена і втілена в життя державна програма по
обмеженню цього всього де тільки зможе держава дотягнутись. Хтось скаже та,
всерівно все не обмежать і знайдуть якісь лазівки. А все і не потрібно,
більшість людей не шукає це лайно ціленаправлено, а просто по замовчуванню ним
користується щоб не заморочуватись, плюс воно доступно скрізь. От якщо
прикрутити хоч трохи просту і легку доступність цього, то рузультат буде дуже
хороший. Згадайте Вконтакте, там пів країни сиділо, всі про нього говорили.
Зараз це вже просто тінь колишнього впливу.

\begin{itemize} % {
\iusr{Тригубенко Костянтин}
\textbf{Євген Гапоненко} 

хто така та ваша держава яка розроблятиме державні програми? Чи не вчорашній
вчитель історії і завзятий русофіл, чи не завзята і пробивна жіночка у якої
життя було прекрасне за срср? Ви вважаєте, що держава це щось таке абстрактне
велике і розумне яке захистить бідних і скривджених українців він
малоросіянства?? Та пів країни сплять і бачпть путіна царем тут у нас , аби
тільки не стріляв і рублі давав.

\iusr{Євген Гапоненко}

Я з вами абсолютно згоден як все є на даний момент. Я просто написав як на мою
думку спрацює колись а як не спрацює ніколи. Бо тут обговорюється варіант чи це
люди самі мають мінятися і ми всі чекаємо поки це відбудеться (на мій погляд
цього ніколи не буде) чи ми всі чекаємо поки до до влади прийдуть якісь
конкретні люди які виставлять такі обмеження (на мій погляд це все ж колись
відбудеться, саме таким чином).

\end{itemize} % }

\iusr{Руслан Б-к}

Виписувати штраф в 50000грн на підставі відеодокази, 10\% тому хто подав заяву,
і можно не скандалити

\begin{itemize} % {
\iusr{Борис Мінаєв}
\textbf{Ruslan B-k} Таксисти, що працюють по програмі знають, що їм від мене, або негативний відгук, пбо чайові. На державному рівні посадити всіх на подібні програми, але нікому це робити, всі заняті «баблом».
\end{itemize} % }

\iusr{Viktor Pushkar}

на рівні водіїв питання не вирішується. Треба штрафувати власників транспортних
засобів і конфісковувати техніку, яку використовують для примусового споживання
пасажирами російської пропаганди. В автобусах здорової людини нема одного
телевізора, який волає на весь салон. Або тиша, або індивідуальний перегляд
кіна на маленькому екрані, на навушники. Записали номер, поскаржились на
порушення прав пасажирів, за пару днів прийшла інспекція і демонтувала
подразник. Так має бути, але поки до того нема правових підстав...

\begin{itemize} % {
\iusr{Taras Kompanichenko}
\textbf{Viktor Pushkar} Дякую. Зроблю.

\iusr{Інгвар Тітус}
\textbf{Taras Kompanichenko} ще одне, друже! Вони не платять роялті за ці фільми!!! А там штраф.... ой-йой-йой!!! Таким методом можна провчити!!!
\end{itemize} % }

\iusr{Zarina Nadia}

Допомагає звернення до перевізників. Потрібно зафіксувати номер автобуса і
назву маршруту. В автобусі має бути контактний телефон ФОПа або підприємства,
що здійснює перевезення.


\iusr{Kazimierz Pawłowski}
Працювати нам і працювати в цій царині. Що ще, якщо політика держави в інформаційному просторі така.

\iusr{Олег Бардашевський}
Підтримую на всі 100!

\iusr{Оксана Проселкова}

Має бути не \#какая\_разніца, або культурний внутрішній бар'єр від тих народів,
що зазіхають на нашу землю. На жаль, ви праві, гігієна відсутня


\iusr{Slavko Polyatynchuk}

Люмпен, нажаль, то складова частина в нашому суспільстві ...тільки щоденна
просвітницька робота буде давати поступовий результат

\begin{itemize} % {
\iusr{Сергій Сухомлин}
\textbf{Slavko Polyatynchuk} 

люмпен просвіту не сприймає і ніколи не оцінить.. то особливий вироджений клас
суспільства.... з люмпенами потрібно інакше боротись... хоча це не
"гуманно"... світова спільнота таке засудить....


\iusr{Гандзя Осадча}
\textbf{Slavko Polyatynchuk} , НІТ!
ЛИШЕ штрафи велики та ганьба!

\iusr{Alexander Derkach}
сила 3.14здюлєй (с) Митець

\iusr{Петро Васьків}
\textbf{Slavko Polyatynchuk} На жаль, лише пряник тут не спрацює. Потрібен і добрячий батіг.
\end{itemize} % }

\iusr{Ivan Kalinichenko}

Повністю вас підтримую Пане Тарасе. Останні два рази як їздив тими маршрутками,
то був змушений сваритися з водієм щоб це виключили або включили щось інше. Як
виявлялося, іншого в них не було.

\begin{itemize} % {
\iusr{Taras Kompanichenko}
\textbf{Ivan Kalinichenko} 

Та мені вже набрибло виявляти ініціативу та вступати в передбачуваний конфлікт
з водієм, в якому мене прогнозованоі не підтримають співпасажири. Хотів
поспостерігати, як антрополог...


\iusr{Ivan Kalinichenko}
\textbf{Taras Kompanichenko} 

дуже нелегку ношу ви собі обираєте. Мені, після дуже довгої перерви в
споживанні московського культур продукту було надзвичайно важко дивитися (а не
дивитися ви не можете) серіал про русскіх депутатів, їхні жарти про захмарну
корупцію (бо там це норма) і прікольчікі про мєнтов.

А відносно спів пасажирів, то їхня неперебірливість і всеїдність контенту за
часту шокує... а шо, ну хай ото собі грає якась дурничка, мені ото яка разниця,
по нашому ж говорять, я ото одним оком слухаю.

\end{itemize} % }

\iusr{Ирина Бирюкова}

на цих нічого не подіє, ціннісна прірва. Маємо дбати про нащадків, займатися
ними, плекати. А цих прибере тільки час. Він не стомлюється, - вірно пише Ліна
Костенко, - а ми живі, нам треба поспішати.

\begin{itemize} % {
\iusr{Олександр Кондратюк}
\textbf{Iryna Biriukova} Без організованого тиску сам по собі час не впорається.

\iusr{Ирина Бирюкова}
\textbf{Олександр Кондратюк} і це теж вірно, мовчання і бездіяльність дозволяє цьому тривати довше.

\iusr{Oksana Klokova}
\textbf{Iryna Biriukova} нащадків вони так само налаштовують. Постійно маю розмови з учнями, для яких усе українське- погане та чуже, починаючи з мови. ..(
\end{itemize} % }

\iusr{Олег Мацьоцький}

У мене постійні битви з тупою шофернею, причому від пасажирів я ніколи не
отримував підтримки

\begin{itemize} % {
\iusr{Taras Kompanichenko}
\textbf{Олег Мацьоцький} Абсолютно тотожний досвід.

\iusr{Оксана Кротюк}
\textbf{Олег Мацьоцький} У мене те саме в магазинах: знайдеться покупець у черзі, який підтримає російськомовного касира, але мене - при тому, що зауваження роблю стримано і веду розмову аргументовано - ну, хоч би на сміх один раз хто підтримав. Такі ми...

\iusr{Гандзя Осадча}
\textbf{Taras Kompanichenko} , дуже прикро. Не вдається бойкотувати скажених перевізників, навіть скарги й публічна ганьба - не сприяють тверезому баченню життя!
Почуваєшся приниженим, згвалтованим, побитим - жахливий, болючий відчай!
Я в побібних випадках включаю голосно
Ваш спів \enquote{ВИ ПРОСИТЕ ПІСНІ},
а потому Лесю Рой \enquote{Якже ти підеш на війну} й поспіль групу Пирятин.
Хоча направду від ядучого головного болю та приниження - не рятує.
\end{itemize} % }

\iusr{Роман Павлюк}
Це у всіх автобусах таке. Просто показує рівень культури в галузі.
У Львові зробили правильно — заборонили музику в маршрутках.

\iusr{Олена Олійник}

Нічого не відбудеться в таких їхніх душах (сумно вас розчаровувати, пане Тарасе, але...)

Це люди із паралельної з нашою реальності. Для них це МИ не розуміємо простих
речей. Що какая там вайна мєжду прастимі людьмі, ета всьо палітікі, каториє
хатят нас пасоріть, а нам нада вмєстє виживать і памаґать друг другу. Що расєя
сільнєє, какой дурак протів нєйо пайдьот, а значіт русскіє наши братья (не
жартую, це стислий переказ реальних промов, почутих мною у різні з останніх
років за різних обставин). І можу ще подібних тез накидати, та гадаю, що
шановне товариство вже зрозуміло, а дехто і впізнав знайомі мотивчики.

Ви, як митець, звісно і надалі намагатиметесь достукатися до їхніх душ. Так і
має бути, і велика дяка вам за це, як і за все, що робите для нас і України.
Але часу на їхнє перевиховання в нас нема, з огляду на обставини, тому дієвим
для ЦИХ буде лише примус з боку законодавства і відповідних органів, причому
потужний.

Даруйте цей багатослівний монолог)


\iusr{Петро Васьків}

Потрібний Закон про колаборацію, згідно з яким такі перевізники, як мінімум,
добряче потрусили б свої гаманці, а як максимум - хлебтали б тюремну баланду.


\iusr{Фемий Мустафаев}

Тільки на законодавчому рівні, нехай слухають вдома росіїську попсу, водій
відповідає за життя пасажирів, водій не діджей, він повинен працювати, а
рушійні засоби не концертнії зали.

\begin{itemize} % {
\iusr{Сергій Коваленко}
\textbf{Фемій Мустафаєв} так, ці телевізори..анахонізм .
\end{itemize} % }

\iusr{Леонід Горобець}

Влітку проїхав маршруткою по маршруту Київ-Суми-Київ. Три сумних висновки:

\begin{itemize} % {

\item 1) Дійсно крутять по телевізору російське лайно

\item 2) Воно подобається пасажирам. Хотів попросити вимкнути, але багато хто з
інтересом дивився.

\item 3) По дорозі в Суми крутили російське лайно, по дорозі до Києва українське
виробництва Студия "Квартал 95". Українське лайно зняте ще гірше ніж російське.

\end{itemize} % }

\iusr{Валерій Корнєєв}

Ну, вибачте мою дегуманізацію людської сутності, але — жуйні парнокопитні.

\iusr{Гандзя Осадча}
\textbf{Валерій Корнєєв} , щирий біологічний погляд @igg{fbicon.face.monocle}  ‼ ️ 

\iusr{Wolodymyr Hucul}

Я, наприклад, про існування вови зєлєнскава дізнався із телевізора в автобусі
Автолюкса "Ужгород - Київ" далекого 2006 року. І дуже дивувався в 2019 що він
ще живий, виявляється. Я не конспіролог, але якась подлянка в цих маршрутках
ховається.


\iusr{Андрей Михненко}
\textbf{Wolodymyr Hucul} особливо, якщо згадати, що "Ласковий Май" і Шатунова розкрутили саме так-роздаючи касети провідникам,щоб ставили у потягах

\iusr{Oleh Stephaniv}

... їх змінить мабуть саме вторгнення Мордору рф, коли особисто зачепить їхні родини.
О мій народе сторозтерзний,
люблю і водночас ненавиджу тебе.

\begin{itemize} % {
\iusr{Oleksiy Furmanov}
\textbf{Oleh Stephaniv} не змінить - вони побіжуть вислужуватись колаборантами. Їм пофіг гривні чи рублі.

\iusr{Олександр Бойчук}
\textbf{Oleh Stephaniv}
Навіть вторгнення не змінить

\iusr{Oleh Stephaniv}
\textbf{Олександр Бойчук} виходить їх треба провчити вже і натомість тими хто стикається з цим бидляччям щоднини
\end{itemize} % }

\iusr{Леся Рой}

ніхто не став на бік водія - шалений прогрес: я влітку попросила стишити рос
шансон з матюками (в салоні бцли й діти), то кілька пасажирів загавкали, що їм
наравиця така музика..

що робити - зрозуміло: штраф за зафіксований випадок


\iusr{Лена Ровенская}

Вчора їхала в маршрутці, то звучав не шансон і російська попса , а пісні Квітки
Цісик. Оце я розумію. Але в більшості водіїв, нажаль вата в голові


\iusr{Eugene Ilarionov}

Така проблема дуже просто та швидко вирішується наступним чином (превірено 2
рази): у кожній маршрутці біля водія є інформаційна дошка з даними
фірми-перевізника. Обидва рази як тільки я казав водію, що поскаржуся
перевізнику та починав набирати номер телефону, що зазначений на цій дошці,
водій мовчки (ОК, зі "скрежетом зубовним") вимикав кіно чи радіо.

\iusr{Roman Vlasyuk}

Як на мене, нашим людям не підходять терміни \enquote{лагідна українізація} чи \enquote{лагідна
дерусифікація}. Потрібна законодавча нагайка, яку всі відразу зрозуміють. І за
порушення закону - жорстке невідворотнє покарання.

\iusr{Олесь Шевченко}

Бидло аргументів не розуміє. Бидлу просто треба бити морду. Це якщо по-лагіднолму.

\iusr{Ганна Дика}

На жаль, це стосується не лишецих водіїв. Кілька років тому я їхала з Полтави
під музичний супровід нижче-плинтусного гумору російською. На зауваження
водієві - нуль реакції.

\iusr{Катерина Немира}

кілька років тому, водій автобуса міжнародного сказав мені. що їм ці фільми дає
начальство.. А потім тихенько подякував за мою реакцію
несприйняття.. Тихенько, тому ,що боявся продовження теми з мого боку на рівні
власників автобуса, чи тому, що дійшло?-я не вияснила.. Я була одна на весь
повний автобус.. Отож , виглядає, що поволі. але рухаємось таки до
усвідомлення...


\iusr{Ganna Yarovenko}
Пиши офіційну скаргу на перевізника, не соромся.

\iusr{Валерий Фисун}
У головах у них "какаяразніца"

\iusr{Даля Волошинівська}
Чому

\iusr{Lesja Seifert}

Підтримую Вас повністю! Найгірше, що навіть в невеликих сільських магазинах,
кафе крутять російську попсу. Моя відпустка - це постійні зауваження таким
продавцям і водіям.


\iusr{Наталія Башмакова}
в їхній дім ще не прийшла війна...

\iusr{Андрей Михненко}

\obeycr
Це морок.
Звичайний гіпноз.
Вони глибоко сплять і не усвідомлюють.
Розбудити їх можна.
Вимкнувши той лайноконтент, що ллється з більшості каналів і в інтернеті. І дати їм українське. Естетичне. Чудове. Наповнююче. Надихаюче. А такого є-і багато!
І вони звикнуть, і будуть дивитися українське.
От Ваша, Тарасе, творчість будить. Виймає з мороку.
Головне, щоб кожен з нас усвідомив свою відповідальність за таких, зачарованих.
І звільнив з цього ментального полону таку кількість людей, яку тільки зможе.
Порадою, піснею, прикладом, фільмом, віршем, будь-чим творчо-українським ("созидательньім")
Ви то робите на всі 300\%.
І позитивна динаміка є.
Бо, згадайте, рік-півтора тому така ж(навіть ще більш дика) історія з Яриною, закінчилася тим, що їй прийшлося вийти на трасі, бо вся маршрутка була на боці водія.
А вона їхала з похорону свого Воїна...
І тільки завдяки допису у фб, ця історія набула розголосу, водія перевиховували вже на кінцевій зупинці і, здається, він був звільнений.
Так що незважаючи на неприємний інцидент, позитивна динаміка все ж є.
Тож лупаймо цю скалу!
Бо хто, як не ми?)
\restorecr

\iusr{Oksana Hotsinska}

Взагалі дивно, що ще досі десь крутять кіно у салонах бусів. Це
позапозавчорашній день. Досить з нас того інформаційного шуму,, тим більше
московського. Ніхто не має права нав'язувати свої примітивні смаки, та ще й за
наші гроші. У кожного є ґаджет і навушники. В дорозі потрібен спокій. А водій
має концентруватись на дорозі, коли він везе ЛЮДЕЙ.

Офіційна скарга на перевізника.

\iusr{Олег Ковальський}

Низька свідомість - дозволяє кремлю проводити блискучі операції гібридної
війни, невже не видно, що окремі канали тб відкрито працюють на ворога , не
говорячи за політиків... низькосортні серіали про ментів, прокурорів - бровада
геройства.... хто ми ??? Своє плюндруємо - нас не поважає світ і дивується нашій
тупості ,продовжувати можна довго... повністю підтримую Тараса, розуміючи, що
його бачення - це гнів свідомих людей


\iusr{Олександр Одрін}
Не знаю. Боюсь, що лише примус. А самі - ні. Сумно і гидко.

\iusr{Тригубенко Костянтин}
Що має статись? Російський град має розтрощити ту маршрутку, тоді трохи почнуть думати.

\iusr{Оксана Дзюбенко}
Нiчого таких суб'ектiв не виправить! Тiльки могила...

\iusr{В'ячеслав Березовський}
Цей "нарід" вибрав собі кловуна. ☹️

\iusr{Людмила Верещагіна}
Прикро, що не збільшується кількість людей, які готові підтримати, незгодних.. раби мовчать... Неначе люди подуріли..

\iusr{Роман Клочко}

Пам'ятаю, що після останнього такого скандалу в медійному просторі група
депутатів зареєструвала законопроєкт про заборону вмикати музику або фільми у
громадському транспорті. Але щось про нього тепер не чути. Маєте нагоду
нагадати. 

\url{https://www.rada.gov.ua/news/news_kom/201610.html}

\iusr{Григорій Лук'яненко}

"...відбутися в житті перевізників"... має зміна пріоритетів у їхніх ватних
головах після "чарівного пєндєля" у вигляді... - тут дописуємо кожен сам що
вважає за ефективніше...


\iusr{Yury Yamash}

Треба починати з телеканалів які за програмою гібридної віни транслюють фільми
що славлять радянсьеке - російське військо. І це практично кожного дня


\iusr{Algirdas Gaidukas}

Це дуже болюче питання. Неймовірно, але за роки Незалежної України відбулася
нищівна русифікація. Навіть більша, ніж в часи совдепії. Інколи слідкую за
,,Україна має таланти". Та двомовність- білінгвізм просто вбиває. Навіть
Притула і Руслана в щирості громадянської позиції та відданості українській
ідеї не викликають сумніву, вживають барбаризми, звороти запозичені чи
скальковані з московитською мови. Зовсім відсутні приклади бездоганної
літературної української мови на телебаченні чи в радіо. Все це мимоволі
нівелює і знецінює відчайдушні спроби українізації простору мас медіа. Робить
їх ніби несправжніми, неповноцінними, ніби для ока. А та українізація ой як же
потрібна. Справа не зрушиться з місця, доки не буде покарання гривнею. У Литві
кожне публічне відтворення музики чи фільмів є платними. Всі бари, ресторани
платять відповідну суму у скарб держави ( казначейство). Я вже не кажу про
кримінальну відповідальність за авторські права. Отже, українізація має бути не
м'якою, а жорсткою.

\iusr{Зоя Навроцька}
Намагаюсь не користуватися цим транспортом саме через цю гидоту.

\iusr{Oleksandr Biletskyi}
Цих перевізників треба батогами відшмагати. 100\% допоможе!

\iusr{Ntina Ntoubrova}

Вони так само, як і жителі ОРДЛО в 2014р. щиро вірять, що якщо почнеться війна
на їхній території - то до них снаряди не будуть долітати, солдати в їхніх
дворах не будуть ховатися та відстрілюватися, забирати їхнє майно та їжу, бо
вони, перевізники \enquote{внєпалітікі, тому нехай оці воюючі подалі від мене
розбираються, що їх там не влаштовує}.

Насправді, люди - досить егоїстичні істоти. І якщо їх не навчати співчуттю,
вони завжди будуть завважувати тільки свої інтереси. Зараз в черговий раз під
час шанування жертв Голодомору газети зарясніли статтями свідчень про те, як
одні відверталися байдуже від вмираючих, а інші - з радістю йшли до лав
карателів, бо не хотіли бути одними із знищених. Але їм, через егоїстичність та
відсутність співчуття до інших, і в голову не приходило, що можна було б
об.єднатися і влаштувати бунт - і тоді точно не було б погано усім. Бо їм
простіше вирішити питання про себе, а що там з іншими - \enquote{а мені пофіг, всім не
допоможеш}.

Точно так же оці всі перевізники впевнені, що вони зможуть \enquote{домовитися якщо
що}. Я 100\% впевнена, що більшість з них гадки не мають, що витворяли із
таксистами бойовики в Донецьку та Луганську. Бо ті репортажі вони не дивилися,
їм реготати під \enquote{Квартал} та \enquote{Комеді-клаб-Росія} було веселіше.

Тому щоб ці ілюзії \enquote{відчекатися спокійно в своєму домі, поки навколо чума та
війна} розвіялися - треба аби працювало ТБ і зокрема - художні фільми. Бо вони
найкраще людей й навчають.

\iusr{Oleg Babak}

Держава, на жаль, нічого не вдіє з цим. Принаймні, у найближчому майбутньому.
Бо держава - це люди, що в ній живуть. Це і водії цих маршруток, і пасажири, і,
власники перевізного бізнесу, etc. А у випадку Сумщини ще й близьке сусідство з
Росією дається взнаки. Я родом із Сум, з числа моїх знайомих сум'ян не так вже
й багато перейшло на українську в побуті за останні роки, та мовна ситуація,
все ж, відчутно змінилася. Більше людей ситуативно переходить на українську,
якщо до них звертатися рідною мовою. До речі, мені періодично передають із Сум
речі через водіїв цих маршруток, і часто вони зі мною також спілкуються
українською. А от щодо руснявого \enquote{культурного} продукту - тут все назагал
печально. Сум'яни аж занадто інертні в цьому плані.


\iusr{Любовь Якубович}
Насправді це жахливо. Сьомий рік війни, а у нас досі шофьори за кермом автобусів та маршруток.

\iusr{Людмила Маркевич}

Згадайте як цей,, обтяжений інтелектом елєкторат\enquote{ валив на різних
бастамонатіківполяковихвимяісциклокаменськихпотаповихсердючкових... @igg{fbicon.cry}{repeat=3} 


\iusr{Kateryna Dudka}
Боже, воділа, який затикає Тараса Компаніченка - це вище моїх когнітивних здібностей

\iusr{Юрій Свистун}
Пакет "Граду" у власний город, будинок лікує все...
Так було в Маріуполі, Краматорську....

\iusr{Lapka UA}
І молодь виховується на російській попсі тому і сприйняття таке. На жаль! Але все має змінюватись!!!

\iusr{Ірина Шевченко}
Скільки їздила, стільки і крутили. Не Голлівуд чи Ворнер бразерс, а саме отой шлак

\iusr{Олеся Чапцева-Білоус}

Іноді здається, що код нації винищили голодоморами і розстрілами, а до того ж
через різні депортації змішали з азіатською расою... Справжніх мало, але
тішить, що осередок патріотів не знищили і не зможуть!


\iusr{Пабло Павлів}
Там вистачає таких

\iusr{Olga Budnyk}
В маршрутках Київ-Рівне-Київ та сама ситуація  @igg{fbicon.anger} 

\iusr{Oleksiy Furmanov}

73\% нікуди ж не ділись. Ну тобто корона трішки покосила, але все одно всі на
місці. Їм какаяразніца чиї танки їздять по їх містах, аби "асфальт бил освєщон"

\iusr{Сергій Сухомлин}

вони вже окупували без вторгнення \enquote{чинушами і служаками в формі прокурорів
..поліції ..податківців... міністрів}....яке ще вторгнення потрібно?.... то все
зробиться руками внутрішньої окупації... і \enquote{вакциномором}.... Після вакцин
здатність народити дітей знижується дуже сильно... от такий собі план окупації і
винищення українців... А коли люди почнуть ту всю банду корупції скидувати.. тоді
російські війська прийдуть чинушам на допомогу.... а також їх чекає і \enquote{население
какая разница}...яке підживлюється російською продукцією..  

\begin{itemize} % {
\iusr{Олександр Бойчук}
\textbf{Сергій Сухомлин}
Дуже Вас прошу, не пишіть дурниць про «вакциномор», не поширюйте безграмотні плітки

\iusr{Сергій Сухомлин}
\textbf{Олександр Бойчук} 

чому? коли в ковгоспи заганяли \enquote{добровільно}.. видно теж людям тоді говорили
активісти комсомольці і парторги.. щоб люди не сперечались з тодішньою владою.. і
вступали в ковгоспи здаючи свій реманент і віддаючи коней і корів, чи не так?
Коли тільки люди починають мислити логічно і вільно.. то їм закидують що
говорять дурниці... так-так час покаже... якщо вам до вподоби приймати вакцину то
приймайте хоч кожен день...

\iusr{Олександр Бойчук}

Дякую за пораду. Кожен день не треба, є розроблена світовою наукою схема
щеплення, якій я довіряю. Чи може ви спеціаліст вірусолог і маєте відповідну
наукову ступінь? Яке ви маєте право, не будучи фахівцем, таке писати.

\end{itemize} % }

\iusr{Viktoriya Horbach}

На жаль у нас 70\% таких громадян на чолі з \enquote{зеленою пліснявою}...

\iusr{Марія Середа}

Актуально! Більшість людей, нажаль, звикли і не розуміють цієї проблеми. Допоки
вона не стоятиме на загальнодержавному рівні.

\iusr{Zoryana Yanytska}

Так виглядає, що професія перевізник об’єднує так звані «73\%». На багатьох
маршрутах Україна-Західна Європа московське сміття крутять без перерв. Навіть
якщо просиш вимкнути - таке враження що тебе висадять посеред дороги , а все
одно будуть слухати « музику».

\iusr{Oleg Klymenchukov}

Прикро, але досі не всі розуміють, що подібні фільми та інша \enquote{агітпродукція} з
середини руйнує не менше ніж реальна зброя. Проблема стосується не тільки
окремого маршруту, чи регіону, це загальна проблема суспільства. Щодо
перевізників, то вони обов'язково будуть запрошені в ОДА для відповідної
роботи.


\iusr{Андрей Каплюк}
Тупі створіння, замість голови у них шлунок....

\iusr{Виталий Мальцев}
...за наше мито , нас же і бито .
...це зверхність та зневага до пасажирів як до людей , за їхні ж гроші , та кріміналізована монополія на ринку перевезень , навіть як що підвищити платню за проїзд в рази , цього питання не вилікує ,...це питання державного рівня а не окремих перевізників

\iusr{Anna Taldenko}
Ганьбище ☹️

\iusr{Андрій Опята}
Нічого і ніколи вони не зрозуміють.

\iusr{Iryna Petrotchenko}
я знаю, що повинно статися.
коло такої людини повинен розірваться снаряд і відірвати їй ногу. тоді вона скаже: \enquote{так це ж ворог!}.
ця війна особлива.

\iusr{Наталія Антонішина}
На жаль, це сміття майже усюди..

\iusr{Роман Кушнарьов}
Підтримую

\iusr{Степан Харченко}

...що має відбутися в житті перевізників..? Ві дповідаю пане Тарасе : На жаль
вже нічого... Тільки тоді , коли ,ми, українці створимо, відбудуємо, відновимо
три найголовніші, на мою думку інституції :1. сім"ї. 2.
освіти.3.релігіїзнавства. от тоді ми матимемо нормальних , адекватних,
УКРАЇНСЬКИХ водіїв, суддів, поліціянтів, лікарів, священників, вчителів....ну
ви мене зрозуміли....


\iusr{Иван Майстренко}
Ці люди вибрали зє

\iusr{Галина Бідношей}
Багато каналів в Західній Україні віщують російською... як на зло...

\iusr{Олексій Грицишин}
\textbf{Ярина Чорногуз} вже декілька років тому піднімала питання

\iusr{Slav Sh}
Малороси звичайні
Нічого дивного

\iusr{Петро Білоус}
Саме із за таких ватників до цих пір йде війна, а найнебезпечніше, що в тилу ми її по ки що програємо.

\iusr{Гриць Коваль}
Якщо буде десь у Сумах виступ, дайте знати. Дякую!

\iusr{Pavlo Berest}
 @igg{fbicon.face.downcast.sweat}{repeat=2} 

\iusr{Юра Горнович}

\ifcmt
  ig https://scontent-frx5-2.xx.fbcdn.net/v/t39.1997-6/p480x480/105941685_953860581742966_1572841152382279834_n.png?_nc_cat=1&ccb=1-5&_nc_sid=0572db&_nc_ohc=Wvo4qfyB-uUAX-qq66q&_nc_ht=scontent-frx5-2.xx&oh=7e160b4791bca10815fcc23eb2fef84f&oe=61B0144B
  @width 0.2
\fi

\iusr{Костянтин Козак Новаленко}
Номер автобуса є?(або точний час відправлення.)
Якщо так, дай всім інфо, будемо дзвонити і гнобити його. Його звільнять, або міняють перевізника.
З повагою небайдужий активіст)

\iusr{Hanna Enko}

Дякую, що Ви є... Революція Гідності триває,вона йде країною, це Іспит на честь
та гідність кожного українця перед Богом, собою, родиною, Україною. Святе місце
порожнім не буває. Молімось, об'єднуємось, діймо...


\iusr{Алла Рубан}
Це справді ВЕЛИКА проблема. Українці виявились заслабкі проти рашинської отруйної бацили. Треба виробляти імунітет.

\iusr{Калініченко Сергій}
Найкраще допомогає удар битою по шійних хребцях....

\iusr{Валерій Четверков}
Вівці будуть крутити московитське кіно поки не отримають по морді.


\end{itemize} % }
