%%beginhead 
 
%%file 17_11_2022.fb.tkachenko_natalia.bila_cerkva.1.zaraz_pishut_naschek
%%parent 17_11_2022
 
%%url https://www.facebook.com/permalink.php?story_fbid=pfbid0LYFT7sSJr49BAGjBf567gKnrsZ5q4iQmoTC2EMVCnEQWWe2kHRQZeaY7VwWWxADvl&id=100002177778165
 
%%author_id tkachenko_natalia.bila_cerkva
%%date 17_11_2022
 
%%tags 
%%title Зараз пишуть НАСЧЕКАЄНАЙВАЖЧАЗИМА
 
%%endhead 

\subsection{Зараз пишуть НАСЧЕКАЄНАЙВАЖЧАЗИМА}
\label{sec:17_11_2022.fb.tkachenko_natalia.bila_cerkva.1.zaraz_pishut_naschek}

\Purl{https://www.facebook.com/permalink.php?story_fbid=pfbid0LYFT7sSJr49BAGjBf567gKnrsZ5q4iQmoTC2EMVCnEQWWe2kHRQZeaY7VwWWxADvl&id=100002177778165}
\ifcmt
 author_begin
   author_id tkachenko_natalia.bila_cerkva
 author_end
\fi

Зараз пишуть \#НАСЧЕКАЄНАЙВАЖЧАЗИМА. Особисто мені стає складніше від того, що
всі постійно пишуть, що буде складніше. Складніше за 99-2000, коли не існувало
генераторів, паверів, памперсів, бойлерів, обігрівачів, стіралок-автоматів і
м'якого курячого філе? При цьому не було світла, газу, гарячої води, а довбана
курка варилося 2,5 години і була жорстка. Діти не хотіли того їсти. В магазинах
не було дитячих ліжок І памперсів, одноразові пелюшки і мобільні телефони
з'явилися через пару років. А ще не було тупо роботи. Тупо продуктів і товарів
і народ шукав хто їжу хто одяг, хто що міг куди міг вез і ніс. Люди просто
виживали. Я згадую своє студентство виключно як  голод. Бо хоч я і жила на
Печерську, але світла не було постійно, а плитки електричні. Тож поїсти було
складно зварити. Тоді не було мюслі, швидкої їжи, фастфуду, електрочайників. Із
цивілізації тільки кип'ятильник. Зараз страшні часи. Але ми все ще забезпечені.
Люди пашуть, як бджоли. Замовлення з Харкова до Білий Церкви зараз приходить за
добу. Зараз складно. Проте давайте шукати слова підтримки,сили,віри та надії.
Бо ми переможемо і вистоїмо. Ми ж це все вже проходили.
