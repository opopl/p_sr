% vim: keymap=russian-jcukenwin
%%beginhead 
 
%%file 17_12_2019.stz.news.ua.mrpl_city.1.andrij_marusov_priazovja_idei
%%parent 17_12_2019
 
%%url https://mrpl.city/blogs/view/andrij-marusov-priazovyatse-pole-dlya-realizatsii-najriznomanitnishih-idej
 
%%author_id demidko_olga.mariupol,news.ua.mrpl_city
%%date 
 
%%tags 
%%title Андрій Марусов: "Приазов'я – це поле для реалізації найрізноманітніших ідей!"
 
%%endhead 
 
\subsection{Андрій Марусов: \enquote{Приазов'я – це поле для реалізації найрізноманітніших ідей!}}
\label{sec:17_12_2019.stz.news.ua.mrpl_city.1.andrij_marusov_priazovja_idei}
 
\Purl{https://mrpl.city/blogs/view/andrij-marusov-priazovyatse-pole-dlya-realizatsii-najriznomanitnishih-idej}
\ifcmt
 author_begin
   author_id demidko_olga.mariupol,news.ua.mrpl_city
 author_end
\fi

Пропоную відкрити для себе ще одного ініціативного, різнобічного і талановитого
маріупольця, який має креативне мислення і відрізняється готовністю змінювати
власне місто. Історик, соціолог, журналіст, голова і засновник громадської
організації \emph{\enquote{Архі-Місто} \textbf{Андрій Марусов}} – справжній генератор ідей та
координатор багатьох культурно-історичних проєктів.

Народився Андрій в Маріуполі. Мама – вчитель фізики, батько працював механіком
на аглофабриці заводу ім. Ілліча. Батько наприкінці 70-х років брав участь в
будівництві металургійного підприємства в Індії. Так Андрію, якому на той час
виповнилося 6 років, пощастило з сім'єю два роки пожити в країні контрастів. З
того часу збереглися найпозитивніші спогади. Жили в промисловому місті Бхилаї.
Саме там хлопець пішов до першого класу. На все життя запам'яталося як
відзначали свята води і кольорів. Також залишилися спогади про отруйних змій,
які в тій місцевості були настільки звичним явищем, що боятися їх ніхто й не
думав. Коли переїхали до Маріуполя, хлопець починає вчитися у школі № 27.
Найбільше подобалося вивчати історію та літературу. Після школи закінчив
історичний факультет Донецького університету. Юнаку пощастило повчитися в
одному з найкращих європейських вузів – в Центрально-Європейському Університеті
в Празі. Після закінчення навчання в Донецьку працював фахівцем-соціологом. У
2000 році переїхав до Києва, де продовжив роботу соціологом. Разом з другом і
колегою \emph{\textbf{Андрієм Пазюком}} почав займатися електронною демократією і онлайновою
цензурою (2004 рік). Вони провели конкурс на визначення найбільш відкритого
органу місцевого самоврядування України. Також займався розслідуванням
корупційних схем, що призвело до численних судових позовів з боку їхніх
фігурантів. У \enquote{Дзеркалі тижня} – суспільно-політичному тижневику, що видається
в Києві, він працював над висвітленням саме корупційних схем та їх відкриття. З
2012 року (до 2018 року) був обраний на посаду голови  правління \emph{\textbf{Transparency
International Ukraine}}, що є акредитованим представником глобального руху
Transparency International, який комплексно підходить до розробки і
впровадження змін задля зниження рівня корупції. Корінний маріуполець довгий
час мешкав у Києві, але про своє рідне місто і не думав забувати.  Коли події
майдану та війни сколихнули всю країну і, зокрема Маріуполь, Андрій повернувся
до своєї Малої Батьківщини. Сьогодні він вважає, що будь-яка гідна
культурно-історична ідея повинна мати вихід на міжнародний рівень. Можливо,
саме тому його проекти відомі не тільки українцям, але й багатьом іноземцям. 

Андрій реалізовує ідеї, присвячені збереженню культурно-істо\hyp{}ричної спадщини
Маріуполя і Приазов'я загалом. Завдяки його про\hyp{}єкту вдалося вшанувати пам'ять
людей, які боролися за відновлення історичної назви міста. Маріуполь став
першим в Україні і одним з перших міст в колишньому СРСР, який домігся
повернення свого імені. Подіям тих далеких, переломних років присвячена
виставка \emph{\enquote{Маріуполь: повернення імені}}, яка нещодавно була представлена у
Центрі сучасного мистецтва ім. А. І. Куїнджі. Проходила вона в рамках проєкту
\emph{\enquote{30 років вільному Маріуполю}}. Розуміючи, що в місті ніхто не збирається
відзначати тридцятиріччя, Андрій вирішив взяти ініціативу в свої руки. На
сьогоднішній день вже проведено конкурс серед учнів, організовані виставка,
круглий стіл та навіть рок-концерт. Громадський діяч зазначає, що подібні
заходи дуже важливі, оскільки теми, пов'язані з локальною історією, не
втрачають своєї актуальності. Завдяки іншому проєкту \textbf{\enquote{М.А.Р.С}} Андрію вдалося
профінансувати розкопки Маріупольської археологічної експедиції на чолі з
єдиним професійним археологом у Маріуполі \textbf{Забавіним В'ячеславом}.  2 тис.
доларів вдалося зібрати завдяки співпраці з приватними компаніями в Києві,
різними українськими громадськими організаціями. В експедиції взяли участь
українські волонтери з Києва, Вінниці, Львова та Запоріжжя. Вони були приємно
вражені історичною та археологічною спадщиною Маріуполя та Приазов'я.

\ii{17_12_2019.stz.news.ua.mrpl_city.1.andrij_marusov_priazovja_idei.pic.1}

Найближчим часом наш герой намагатиметься реалізувати про\hyp{}єкт, пов'язаний з
муралами. Чоловік сподівається, що саме завдяки муралам сірі стіни міста
отримають друге дихання.

Вільний час Андрій намагається проводити з сином, який проживає в Києві. Син
підтримує батька і його діяльність. Чоловіка надихає те, як сильно змінився
Маріуполь, а найбільше його вражають люди, які, як і він, готові до змін і
реалізації власних ідей...  

\textbf{Улюблена книга:} російська класика, найбільше полюбляє перечитувати твори А. П. Чехова.

\textbf{Улюблений фільм:} \enquote{Андрій Рубльов} (1966 рік), \enquote{Маленька Віра} (1988 рік).

\textbf{Порада маріупольцям:} \emph{\enquote{Робіть! Реалізовуйте тут і зараз, адже Приазов'я – це поле для реалізації найрізноманітніших ідей!}}.
