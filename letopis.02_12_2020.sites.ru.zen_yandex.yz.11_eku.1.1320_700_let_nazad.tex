% vim: keymap=russian-jcukenwin
%%beginhead 
 
%%file 02_12_2020.sites.ru.zen_yandex.yz.11_eku.1.1320_700_let_nazad
%%parent 02_12_2020
 
%%url https://zen.yandex.ru/media/11ecu/eto-byl-1320i-ili-700-let-nazad-5fbb568d93538c3039246a4a
 
%%author 
%%author_id yz.11_eku
%%author_url 
 
%%tags istoria,1320
%%title Это был 1320-й или 700 лет назад
 
%%endhead 
 
\subsection{Это был 1320-й или 700 лет назад}
\label{sec:02_12_2020.sites.ru.zen_yandex.yz.11_eku.1.1320_700_let_nazad}
\Purl{https://zen.yandex.ru/media/11ecu/eto-byl-1320i-ili-700-let-nazad-5fbb568d93538c3039246a4a}
\ifcmt
  author_begin
   author_id yz.11_eku
  author_end
\fi

\begin{leftbar}
  \begingroup
    \em\Large\bfseries\color{blue}
"Верните мне мой 1320-й год!"
  \endgroup
\end{leftbar}

Такое часто можно слышать от уже немногочисленных старожил, которые ещё помнят
те славные времена. Современному поколению не понять, чем так запомнилась нам
та "Осень Средневековья". Для тех, юных читателей, кто максимум, что помнит -
это осень СССР, мы и ведём сие повествование.

\ifcmt
pic https://avatars.mds.yandex.net/get-zen_doc/3839286/pub_5fbb568d93538c3039246a4a_5fc2187b39eab6574d8c1cba/scale_2400
caption Москва XiV века. Узнали?
\fi

Это сейчас Москва и Тверь для обывателя практически слились в единый центр
необъятной России. А в тот 1320 год случилось великой дело. Наконец-то Анне
Кашинской удалось умолить московского князя Юрия Даниловича отдать её гроб с
телом мужа, убиенного ещё несколько лет назад в Орде. Князь Михаил Тверской
упорно оспаривал первенство на Руси у своего московского коллеги. Но Юрий
оказался проворнее и в Орде дискредитировал Михаила. Тверича долго пытали,
неволили, а потом и убили объединенные люди хана и Юрия. И вот в 1320-м году,
наконец, Михаил, возвратился на родину, быть погребенным в родной земле, уже
широко почитаемый как мученик. 

\ifcmt
pic https://avatars.mds.yandex.net/get-zen_doc/1875669/pub_5fbb568d93538c3039246a4a_5fc2181663d5740415a6e1ac/scale_2400
\fi

В этом же году сын Михаила Дмитрий Грозные Очи женится на дочери великого
литовского князя Гедимина. Через пять лет этот тверской наследник отомстит
московскому князю Юрию, правда ценой своей жизни, но это уже совсем иная
история.

Порядка двух десятков князей разной степени взаимного подчинения и
независимости одновременно правят в разных частях земли русской. Не
прекращаются попытки захватить вотчины друг друга, но это все на тот момент
попытки реализации личных амбиций, а вовсе не великой задачи "Единения Руси".
Вряд ли кто-то всерьёз об этом помышлял. Ну присоединила Москва Углич в 1320-м
году, но разве ж это значило что-то всерьез в то время?

Гедимин укрепляет Великое княжество литовское, захватывает всё новые земли на
территории современных нам Украины и Белоруссии, подчиняет себе даже Киев, хотя
в те годы уже вряд ли к Киеву употребимо "даже". В 1320 году присоединяет
Гедимин Витебск. На фоне непрекращающихся дрязг русских князей ВКЛ выглядит в
те годы очень убедительно и сложно было себе тогда представить привычную нам в
будущем геполитическую карту. 

\ifcmt
pic https://avatars.mds.yandex.net/get-zen_doc/1856956/pub_5fbb568d93538c3039246a4a_5fc217c339eab6574d8aeebd/scale_1200
cpx Гедимин
\fi

Крепнет Золотая Орда. Правит Узбек хан. Как раз в 1320 он решает сделать ислам
государственной религией. Государственная же политика основана на жестокости и
устрашении всех подвластных народов.

На территории Китая главную роль играет Империя Юань, где правит монгольская
династия. В 1320-м году в попытке создать себе марионеточного правителя
вдовствующая императрица Даги сажает на трон 17-летнего внука Шидэбала. Но тот
оказывается вполне самостоятельным правителем. Поэтому и не исполнится ему 21
год. Правда, бабушку пережить удалось.

В семье византийского императора Андроника II больше горе. Сначала внучка
умерла, а затем один внук организовал убийство другого внука, говорят, по
ошибке. Хотел не его убить, вроде. Но узнав об этом умер и отец этих братьев,
сын императора. Не пережил такого горя. Сам Андроник II выдержал. А его
убийца-внук начнет войну против него и станет Императором, отправив деда в
монастырь.

\ifcmt
pic https://avatars.mds.yandex.net/get-zen_doc/3499786/pub_5fbb568d93538c3039246a4a_5fc2178963d5740415a6020c/scale_1200
caption Монетка тех лет (Classical Numismatic Group, Inc. http://www.cngcoins.com)
\fi

Османская империя уже зарождается, правда центр у нее ещё в городе Сёгют. Осман
I расширяет свои владения под боком у Византии.

Забыли про погоду сказать. Холодно было в те годы. Лихорадило с начала века.
Вначале, почти 10 лет была засуха. Потом начались дожди. Да такие дожди, что
вымывали все посевы. Люди оставались без урожая, зимы были чрезвычайно долгими
и суровыми, а значит...историки пишут, что гибли сотни, а то и тысячи людей
ежедневно, и если представить современную карту, то это всё происходило на
территории от Ирландии до Белоруссии. А впереди в этом веке ещё людей ждала и
пандемия Чумы. Детишки, пережившие этот голод её застанут.

Не устали ещё от событий славного 1320-го года?

Во Франции эпоха "проклятых королей". На троне Филипп V длинный. Это уже
четвертый король за 6 лет. Он предпоследний из дома Капетингов. Скоро их сменят
Валуа. А почему? Потому что сыновей нет, только дочери. Передать власть некому,
ведь сам и отстаивал ранее Салический закон, который запрещает передачу власть
женщинам или через женщин.

В Англии Эдуард II. Он заслуживает отдельного рассказа, и мы говорили об этом
монархе ранее.\Furl{https://zen.yandex.ru/media/11ecu/carstvo-za-druga-5c24cb30dd2a8e00a9004c04?integration=morda_zen_lib&place=export}

Про многих мы не сказали, конечно. 

\ifcmt
pic https://avatars.mds.yandex.net/get-zen_doc/1926321/pub_5fbb568d93538c3039246a4a_5fc21728d57ee927523844ec/scale_1200
caption Амдэ-Цыйон I
\fi

Давайте хоть упомянем кратко про Эфиопию. Там Соломонова Династия. Амдэ-Цыйон I
борется с мусульманским Египтом и фактически находящимися под его влияниями
эфиопскими исламизированными провинциями. Даже удается достичь определенных
успехов, укрепить страну.

Но стольно ещё происходит в мире интересного в 1320-м, что и не рассказать в
одном материале, да и многие уже устали читать длинный текст. 

11 ЭКЮ
***
Предлагаем Вам наши рассказы про другие 20-е года.
\begin{itemize}
  \item 20 год
  \item \href{https://zen.yandex.ru/media/11ecu/1000i-god-rus-i-mir-5f3a41d44f2e4f5b77ddee5e?utm_source=yandex.zen&utm_medium=article&utm_campaign=5f8012d35dbc67260c3d0961}{1000 год}
  \item 1120 год
  \item \href{https://zen.yandex.ru/media/11ecu/eto-byl-1220i-ili-800-let-tomu-nazad-5f4b516711a7d1138b0d7c80?integration=morda_zen_lib&place=export}{1220 год}
  \item 1420 год
  \item 1520 год
  \item 1620 год
  \item 1720 год
  \item 1820 год
  \item 1920 год
  \item 2020 год
\end{itemize}
