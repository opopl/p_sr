%%beginhead 
 
%%file 02_07_2023.fb.fb_group.story_kiev_ua.1.pereulok_desjatinnyj
%%parent 02_07_2023
 
%%url https://www.facebook.com/groups/story.kiev.ua/posts/2227972780732844
 
%%author_id fb_group.story_kiev_ua,solovjeva_tatjana.kiev
%%date 02_07_2023
 
%%tags 
%%title Переулок Десятинный
 
%%endhead 

\subsection{Переулок Десятинный}
\label{sec:02_07_2023.fb.fb_group.story_kiev_ua.1.pereulok_desjatinnyj}
 
\Purl{https://www.facebook.com/groups/story.kiev.ua/posts/2227972780732844}
\ifcmt
 author_begin
   author_id fb_group.story_kiev_ua,solovjeva_tatjana.kiev
 author_end
\fi

Переулок Десятинный\par
весь опутан паутиной,\par
в переулке пахнет глиной\par
и нагретым кирпичом.\par
В переулке - чертовщина:\par
здесь горит еще лучина\par
и тоскуют беспричинно\par
непонятно, ни о чем.\par

В этом странном переулке\par
все шаги звеняще-гулки,\par
словно музыка в шкатулке,\par
раздаются здесь шаги.\par
И ворот скрежещут втулки,\par
и рассветно пахнут булки,\par
сто ремесел в переулке,\par
хоть не видно здесь ни зги...\par

Улица Десятинная - хорошо известна, от Михайловской площади она приведет к Андреевской церкви, а переулок Десятинный? Не каждый сразу ответит, переспросит: \par
-Улица Десятинная? \par
- Переулок! \par
- Не знаю.\par
 А в стихотворении киевского поэта Риталия Заславского:\par

Переулок Десятинный -\par
столько всякой мешанины:\par
то шуршат машины шины,\par
то свистит-поет стрела.\par
Запах носится полынный,\par
и мерещатся руины -\par
где-то древняя дружина\par
здесь, наверно, полегла.\par

Переулок Десятинный - по четной стороне всего одно здание - Бизнес центр Южный Дворец - угол с Владимирской. \par

...и летит с Гончарки ветер,\par
И кричат, играя, детм,\par
И томится сердце в нетях,\par
и опять летит во тьму.\par

Короткий переулок продолжится лестницей, по которой можно не только спуститься на Подол, но и посидеть на лавочках и любоваться...\par
А настоящая поэзия она всегда вне времени и последние строки стихотворения, что написано много лет назад в Киеве, звучат и сегодня:\par

Здесь когда-то с Шуркой вместе,\par
в мономаховом предместье,\par
мы с утра ловили вести:\par
Что там в мире? Как сейчас?\par
Ветер, крыши, грохот жести.\par
Сколько боли в каждом жесте!\par
Жили-были. Честь по чести.\par
Что же будет после нас?\par

\ii{02_07_2023.fb.fb_group.story_kiev_ua.1.pereulok_desjatinnyj.cmt.pics}
