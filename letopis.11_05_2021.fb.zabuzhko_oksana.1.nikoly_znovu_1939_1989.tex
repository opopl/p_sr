% vim: keymap=russian-jcukenwin
%%beginhead 
 
%%file 11_05_2021.fb.zabuzhko_oksana.1.nikoly_znovu_1939_1989
%%parent 11_05_2021
 
%%url https://www.facebook.com/oksana.zabuzhko/posts/10159320568758953
 
%%author 
%%author_id 
%%author_url 
 
%%tags 
%%title 
 
%%endhead 
\subsection{НІКОЛИ ЗНОВУ: 1939-1989}
\Purl{https://www.facebook.com/oksana.zabuzhko/posts/10159320568758953}

НІКОЛИ ЗНОВУ: 1939-1989. Прочитайте уважненько, бдлска, допис від Olexiy
Dryomov, лінк у коментарі: це не тільки про дєньпабєди. Це - про ще одну міну
сповільненої дії, передану нам від СРСР. Тільки про цю, на відміну від
дняпабєди, ми взагалі не говоримо, відводимо очі.

(Ось саме тому я вже 5-й рік невтомно повторюю, що найважливіша українська
книжка, написана в минулому десятилітті, - це "ДОРОГА НА АСМАРУ" Сергія
Сингаївського Serhiy Synhayivsky: перший серйозний роман про "український
рахунок" у колоніальних війнах радянської імперії.)

Я знаю, що кажу: я - з "афганського покоління". "Непритомною пам'яттю тіл,
спраглих тіл ми ввесь час пам'ятали, /Що на київській крові десь родять
афганські піски", - це я написала в 1985-му (ота дівчинка в афганській дублянці
з попереднього фото написала, вірш називається "Голосом вісімдесятих"). І я
добре запам'ятала, як у 2016-му, до 25-ліття Незалежности, "Український
тиждень" опублікував статтю співробітниці Інституту історії про пам'ятники доби
Незалежности - де, без сорома казка, стверджувалось, буцімто першими "маркерами
незалежности" в Україні стали - пам'ятники афганцям (спойлер: брехня, першими
були пам'ятники жертвам Голодомору й сталінських репресій!), а "голодоморні"
буцім з'явились тільки після Указу Ющенка 2008 р., який "як би там не було,
слід було виконувати" (с) (спойлер: брехня, пам'ятники жертвам Голодомору по
селах почали з'являтись майже на 20 рр. раніше, тобто як тільки "стало можна",
а в Києві першим "пострадянським" пам'ятником стала споруджена у вересні 1993
р. "Чорна Богородиця" на Михайлівській площі, якої ні Авториця, ні Редакція УТ
не помітити просто фізично не могли, навіть якби їм ФСБ кожне око сотнею
мільйонів рублів заліпило!))... І це тільки один приклад того, як українцям
вміло проходяться по мізках отим самим "ветеранським парадом", непомітно й
поступово замішуючи "адіннарод" уже не тільки на "спільній крові", пролитій "до
9 мая", а й в усіх ПІЗНІШИХ загарбницьких війнах СРСР, де ми, мовляв, також
"билі вмєстє", "ето наша общая бєда" - "знать, подмога нє прішла, подкрєплєньє
нє пріслалі", ціла готова субкультура послужливо виє вам у вуха в підземних
переходах, тільки підспівуй... А там уже й до братання з ополчєнцамі недалечко,
казав же в Донецьку Стєшин (атож, я інколи й таке дивлюся!) - "здєсь же по обє
сторони фронта рускієлюді!" (с)

Не тільки Будапешт-56, не тільки Прага-68 - Єгипет, В'єтнам, Афганістан,
Ефіопія, де ще, що я пропустила?.. Всюди, всюди - як "чорносвиточники" при
форсуванні Дніпра Жуковим - у цих мливах совкового Мордору використовувались
(від 1939-го по 1989-й)  - наші українські хлопці. Тому наше українське "НІКОЛИ
ЗНОВУ" мало б, по справедливості, бути розширене - і охоплювати не європейських
7, а всі 50 (прописом, П'ЯТДЕСЯТ!) років: від сталінського указу 1939 р., за
яким українців стали брати на службу в ЧА за межами України (доти, нагадаю, не
брали!), - і аж до виведення з Афганістану і з Європи радянських військ, із
якими повернулись додому в Україну - бозна-скільки наших... (А скільки - не
повернулося? а скільки лишилося - на мордорській службі в непокірних
"национальных республиках"?..)

Ми досі не маємо ВЛАСНИХ СЛІВ, щоб говорити про їхні трагедії (це в Шевченка
виходило - "і тебе загнали, мій Якове добрий... не за Україну, а за її
ката...", і в Стефаника в новелах про ПСВ - ну так ми й Шевченка читати
перестали, і Стефаника...). Ми досі, як зачаровані, мовчимо - або
послуговуємось чужим (токсичним і, в кінцевому підсумку, смертельним для нас)
дискурсом - бо не знаємо, НЕ ЗНАЙШЛИ, як іще відповісти на питання, яким мене,
в кінці тих-таки 1980-х, "срєзал" таксист-афганець, коли я йому почала читати
полум'яну антиколоніальну проповідь: "Вы мне скажите, за что я своих друзей
хоронил?" (с)

Так що - не "9-м мая" єдиним, ох ні. 

І недарма вони ветеранів 1960-80-х "георгіївським маршем" пускають. І ті, що характерно, - йдуть...

Будьмо уважні (с).
