% vim: keymap=russian-jcukenwin
%%beginhead 
 
%%file 07_06_2021.fb.berdnik_miroslava.5.kaddafi_birthday
%%parent 07_06_2021
 
%%url https://www.facebook.com/permalink.php?story_fbid=1221270971658484&id=100013267772922
 
%%author 
%%author_id berdnik_miroslava
%%author_url 
 
%%tags 
%%title Сегодня день рождения Муаммара Каддафи
 
%%endhead 
 
\subsection{Сегодня день рождения Муаммара Каддафи}
\label{sec:07_06_2021.fb.berdnik_miroslava.5.kaddafi_birthday}
\Purl{https://www.facebook.com/permalink.php?story_fbid=1221270971658484&id=100013267772922}
\ifcmt
 author_begin
   author_id berdnik_miroslava
 author_end
\fi

Сегодня день рождения Муаммара Каддафи.

Из книги М. Каддафи «Деревня, деревня. Земля, земля. Самоубийство космонавта и
другие рассказы» (Триполи, 1993, с. 65-79): «В течение всей жизни следует
сопротивляться Смерти... Смерть - это мужчина, нападающий и никогда не
переходящий к обороне, даже если он потерпел поражение. Он злобен, иногда смел,
иногда труслив. Бывает, смерть терпит поражение и оказывается вынужденной
отступить. Она не побеждает в результате каждого нападения, как это обычно
считают. В скольких схватках лицом к лицу смерть изнемогала и, обессилев,
убиралась прочь! Несмотря на раны, полученные в борьбе со смертью, упорный
противник никогда не сдается. И в этом превосходство жизни над смертью...
Правильной позицией является сопротивление, а бегство даже за границу от Смерти
не спасет... Однако если она перевоплотилась в женщину... и мы ощущаем ее каждой
клеточкой тела... то тогда сопротивляться ей недостойно мужчины, и следует
уступить ей в последний момент...».

Во время правления Каддафи, ливийка, родившая ребенка, получала пособие в размере 5000 долларов для себя и малыша.

Электроэнергия предоставлялась гражданам Ливии бесплатно, а цена бензина в Ливии составляла 14 центов за литр.

\ifcmt
  pic https://scontent-cdt1-1.xx.fbcdn.net/v/t1.6435-9/197083249_1221270914991823_3590038944762274096_n.jpg?_nc_cat=101&ccb=1-3&_nc_sid=730e14&_nc_ohc=YMnny3X3ElEAX-vcCru&_nc_ht=scontent-cdt1-1.xx&oh=d03fe64664a2915f2f508f6a405a275d&oe=60E589A4
\fi

\emph{Владимир Медвидь}
Респект Каддафи! Светлая Память!

\emph{Alex Khrissanov}

вот помню 10 лет назад сидел я чтоб не в парикмахерской и вошел товарисч со
значком вот этих повстанческих сил Ливии. Я его спросил зачем, он сказал что
типа это знак сил где молодые хотят перестроить страну чтоб было лучше, акцент
был на \enquote{молодые}. Я покрутил головой, ибо Кадафи уж на что странный был, а
действительно заботился об общественном благе. После этого я людей со значками
этими и не встречал. Я думаю что если встретить того человека, он сделает вид
что и не помнит ничего такого даже.

\emph{Zinaiyda Levchishina}

\textbf{Alex Khrissanov} я тоже помнию, как спорила с серьезными людьми, которые
\enquote{ратовали} за демократию в Ливии, ты что, орали они, знаешь какая там
бедность, знаешь какой жестокий режим.... После нескольких лет известной
демократии, я спросила: ну что вы довольные обстановкой? И мне, с невозмутимым
видом сказали : ошибочная политика..

\ifcmt
  pic https://scontent-cdt1-1.xx.fbcdn.net/v/t39.1997-6/s168x128/142198391_938673180208887_8398734923463187125_n.png?_nc_cat=105&ccb=1-3&_nc_sid=ac3552&_nc_ohc=_kUx_POXRxsAX9wnsZC&_nc_ht=scontent-cdt1-1.xx&tp=30&oh=0426f476558c402b3259def1d43a080f&oe=60E4AED8
	width 0.1
\fi

\emph{Александр Берман}

\url{https://stihi.ru/2011/04/03/165}

\emph{Наталия Борщева}

А что он писал, и хорошо, я не знала. И лицо замечательное, с хорошей мечтой

\emph{Галина Карчевская}

Помним. Светлая память

\emph{Люда Малашевич}

И какая страшная смерть и муки его ожидали в конце жизненного пути .(( Как
сияла от счастья Клинтон, радовалась его смерти. Никогда не забуду ее лицо на
видео. Радость змеи.

