% vim: keymap=russian-jcukenwin
%%beginhead 
 
%%file books.alternatyvna_evolucija
%%parent body
 
%%endhead 

\section{Альтернативна Еволюція}

Для чого людина приходить у цей світ?

У чому критерій її Буття?

Яка стежка суджена нашій планеті у III тисячолітті?

Олесь Бердник намагається відповісти на ці запитання у творах, які увійшли до
цієї збірки. Автор стверджує: «Альтернативна Еволюція» — ось шлях у майбутнє.
Людина покликана засівати поле Буття квітами Любові й Радості. Але зараз над
світом гримить поєдинок Світла й Мороку, краси і потворності, — і доля світу
залежить від кожного з нас.

До збірки увійшли:

Заповіт людям Землі

Альтернативна Еволюція

Словник Ра (Воскресіння Слова)

Падіння Люцифера

Пора звести блакитиний храм!

«Астероїд свободи». Школа астросталкерів

Це вогняне слово — Свобода

Тайна Христа

Пісня надземна

ТВОРИ, ЩО УВІЙДУТЬ ДО СЕРІЇ

Поза часом і простором

Людина без серця

Привид іде по Землі

Шляхи титанів

Стріла Часу

Серце Всесвіту

Катастрофа

Марсіанські «зайці»

Подорож в Антисвіт

Сини Світовида

Діти Безмежжя

Дві безодні

Хор елементів

Вогняний вершник

Подвиг Вайвасвати

Чаша Амріти

Дике поле

Сузір’я Зелених Риб

Покривало Ізіди

Окоцвіт

Остання ніч

Зоряний Корсар

Лабіринт Мінотавра

Альтернативна Еволюція

Словник Ра

Тартар

Вогнесміх

Падіння Люцифера

Пора звести Блакитний Храм!..

Жива вода

Серце Матіоли та інші легенди

Пітьма вогнища не розпалює…

Це огненне слово — свобода

Астероїд Свободи

Камертон Дажбога

Тайна Христа

Пісня Надземна

Альтернативна Еволюція

Зоряним Братам,

грядущій Громаді Свободи,

вільним духам Всесвіту

з любов’ю.

Автор

\ii{books.alternatyvna_evolucija.intro}
\ii{books.alternatyvna_evolucija.zapovit}
\ii{books.alternatyvna_evolucija.slovo_1_mirazhi_buttya}

\ii{books.alternatyvna_evolucija.slovo_2}
\ii{books.alternatyvna_evolucija.slovo_3}

\subsection{ЗАПОВІТ ЛЮДЯМ ЗЕМЛІ}

Люди Землі!

Я народився серед вас, навчався у ваших школах, мислив вашими словами, ідеями,
дихав повітрям вашої планети. Я став сином людським. Довго, дуже довго я
намагався збагнути ваші ідеали, ваші прагнення, щоб віднайти прийнятний для
людства шлях до вселенської спів’єдності. І ось — замикається коло, що
вичерпало доцільність подальшого терпіння, бо милосердя Любові, хоч і
невичерпне, проте тримається на вищій доцільності. Нині час терпіння Любові для
Землі завершується. Планета зобов’язана дати плід.

Минули напружені тисячоліття еволюції. Ви отримали від Космосу неоціненні
скарби духу й розуму: вчення Свободи, полум’я Мужності, ідеал Героїзму.
Посланців Великої Матері ви зробили своїми богами, своїми ідолами, створивши
для них храми, музеї, мавзолеї. Але як ви скористалися тими дарунками для
осмислення свого життя, як преобразили вищими ідеалами суспільство, людську
душу, тіло, дії, слово і мисль?

Ви маєте космічну енергетику, але безсилі перед ілюзорною рукою смерті.

Ви маєте релігії Любові, але їхнім ім’ям благословляєте вбивство і ненависть.

Ви створили вчення Братерства, але ідеями того вчення розрубали єдине тіло
Людства на ворожі частки.

Ви розірвали гравітацію планети, але вихід ваших кораблів у міжпланетний
простір не звільнив вас від павутини мізерних побутових прагнень. Космічна Ера
стала для вас лише ареною нового змагання за марнолюбні земні цілі, вона лише
розширила хворобу вашого духу в зоряний простір, ставши загрозою для далеких
світів.

Ви оволоділи термінологією Синів Світла, декларуючи ідеали Нового Світу, але
старанно оберігаєте в словах, думках та діях закони Мороку. Жодне суспільство
Землі не проголосило святенність та недоторканність Життя як неповторного Вияву
Світоносної Першосуті.

Ви шукаєте панацеї від хвороб, але спокійно під захистом псевдозаконів нищите у
в’язницях, концентраційних таборах та лікарнях мільйони людей. Ви ведете
нескінченні війни, які ще жодного разу не утвердили людяного результату —
перемогою завжди користувалися найпідліші представники роду людського.

Ви створили безліч технічних пристроїв, скориставшись генієм учених та
інженерів, але ні на йоту не підняли морально-етичний рівень сучасних поколінь:
наказ деспота або парламенту легко знімає з людини обов’язок бути милосердною
істотою. Досвід останнього сторіччя яскраво свідчить про повну деградацію
мешканців планети Земля.

Всі ці страшні втрати ще можна було б виправдати прагненням до великої мети. У
вас такої мети нема.

Ідея Співстраждання, проголошена безліччю подвижників, перетворилася на свою
протилежність: вона годує Дракона Ненависті і Зла енергією Любові й Добра. Так
болісні зусилля Синів Світла вивести людство з рабства були використані для ще
ганебнішого ув’язнення Духу.

Ідею Духу ви зробили культом торгівлі з Богом.

Ідею Свободи перетворили в наріжний камінь тиранії.

Ідею Любові затьмарили хмарами ненависті і сексуальної насолоди.

Ідею Пізнання зробили рабою утилітарної технології.

Ви абсолютизували смерть. Ви розрубали в психіці поколінь нитку безперервності
життя, а отже відчужили власну планету від Єдиного Організму Космосу. Ваш світ
став гангреною, раковою пухлиною, що поглинає вогняну енергію Прасубстанції і
навзамін не творить нічого, крім психообразів мерзоти і ненависті. Ноосфера
Землі — це вампір Космосу.

Ви ввійшли в протистояння космічній Волі Єдності, а отже повністю знехтували
Космічне Право.

Ресурси планети — золотий талант Сонця — вичерпуються, нових ви не отримаєте.
Доля Землі визначається свободою волі людства. Баланс світлих і темних надбань
неминуче схиляється в бік мороку. Земля розірвала еволюційний ланцюг, вибравши
шлях самознищення. Але ще є можливість — остання можливість!

Зоряне Братерство простягає руку рятунку: хто почує, хто збагне — той увійде у
Нове Буття.

Прийдіть, мужні, безстрашні, захоплені, закохані! Прийдіть, знедолені, шукачі
небувалого, відкинуті, забуті, втомлені безглуздістю рутинного життя! Прийдіть,
мрійники і повстанці!

Створимо незламне Кільце Зоряного Братерства у незмірності наших сердець! Сійте
чисті зерна Нового Світу у поле душі, а не в лабіринти соціальних структур.

Збирайтеся в політ. Наш вирій — понад хаосом цього світу. Кличте вільних від
мізерії повсякдення, кличте їх не в пастки соціальних, політичних чи релігійних
формул, — кличте їх у Нескінченність Неба Свободи, яке нам відкриває Велика
Мати Світу,

Знаю — страшно, незвично, незбагненно. Ворог навчив думати по-земному, говорити
по-земному. Треба стати над собою, спопелити нікчемну земну логіку вогнем
прагнення до Абсолютної Свободи.

Ви боїтеся смерті? її нема. То вигадка жерців та позитивістів. І релігії, й
заперечення релігій породжені марою смерті. Я кажу вам: розрубайте в психіці, у
свідомості мару, покривало древнього Дракона — Смерть. Ви вічні, невмирущі,
безперервні, — лише забули про це у далеких мандрах серед мороку.

Пора повернутися до Батьківської Хати. Чуєте, Браття?

Ви вже не діти, Люди Землі! Якщо й тепер ви відкинете ідею Зоряного Братерства,
вся вага історичної карми бумерангом обрушиться на планету. Ви боялися
Страшного Суду, який має прийти з «неба»? Той Суд у вашому серці. Вибирайте!

Мати Світу відкриває браму до Нової Світлиці, але туди не ввійде земна мерзота.
Хто побачить ту браму, хто піде на останній Поклик Любові — той від Світла. Хто
відвернеться — залишиться в одвічному мороці.

Що ще може зробити для вас Любов, Люди Землі? Так просто. Так просто зробити
вибір, доки ще світить сонце, доки Серце Великої Матері відкрите.

Зоряне Братерство об’єднує душі в Блакитний Легіон Свободи. Де ви, Брати?!
Озовіться!!!

Олесь Бердник 1973 р.

Альтернативна Еволюція

СУМА ЛЮБОВІ

Мандрівники Духу йдуть, летять, прагнуть по Дорогах Всесвіту, їх стрічають
Друзі й Вороги, ті, що люблять, і ті, що ненавидять. Тисячі закликів звучать.
Куди? Для чого?

Хто допоможе зрозуміти — де Оаза Радості, а де пастка Змія?

Релігія, окультизм, містика, наука, теїзм, атеїзм… Усі прагнуть оволодіти
головним Скарбом Буття — живим Серцем Мислячих Істот. Саме тут — головна битва
мільйоноліть і суть Містерії віків.

Друже мій!

Хто ти — не знаю. Який ти — не відаю.

Може, ти юна дівчинка, котра очікує свого небесного принца? Чи мати багатьох
дітей? Захоплений студент, який схиляється над старовинними манускриптами,
розшукуючи там сліди вічнодіючого Духа? А може, виснажений сумнівами учений? Чи
ти дух незримого світу і лише готуєшся до тяжких земних стежок?

Хто б ти не був, де б ти не був, який би ти не був, — вітаю тебе! Коли б ти не
жив — у минулому, сучасному, в прийдешньому (бо для Вічності все в спільноті),
— я люблю тебе! І шлю тобі своє серце, відкрите всім світам та епохам.

Що хочу сказати? Адже так багато вже сказано. Океан ідей. Потоки слів. Гори книг. Тисячі кілометрів кіноплівок. Що додам до цього?

Щире слово.

Були вже суми теології. Суми технології. Суми інтелектуальних догм. Я не стану
їх повторювати.

Шлю тобі не наукове дослідження, не хитромудрий сюжет, не законодавчу доктрину.

Співаю Пісню Серця. Даю СУМУ ЛЮБОВІ.

Ти не запитаєш, яка в цьому потреба та необхідність. Ти бачиш, що світ
здригається в передсмертній агонії. Бути йому чи не бути — залежить від нас, бо
це наш світ, створений нашим духом і волею…

Друже мій!

Одного разу в моє серце ввійшов Біль Світу. І став моїм болем. І я вже не міг
позбутися його — ані вдень, ані вночі.

Я усвідомив: злочинно заперечувати цей біль, ігнорувати або забувати. Я
збагнув: його взагалі не повинно бути! Все живе (а живе — все!) палко жадає
буття без болю, без страждання, без трагедії. Біль — то розпечений шворінь,
всаджений невідомим катом у Нерв Світу.

Страждає все: мати, яка народжує дитя, пташка, що кладе яйце, зерно, котре
заперечує себе, щоб стати паростком, клітина, що віддає власне буття для
існування багатоклітинного організму; волає про справедливість оленятко,
спливаючи кров’ю в пащі лева, благають від неба рятунку вівці й корови на наших
бойнях, де готується для гурманів субстрат буття із страждання й крові,
мучиться трудівник для блага абстрактного ідола — держави, карається сонце, що
зігріває планети з їхнім безглуздим життям, корчиться в апокаліптичному трепеті
світовий простір, засіяний ранами зірок та галактик, що прагнуть в нікуди…

О, ми знаємо з тобою, друже мій, що вчені міщани та жерці — холопи грізного
небесного деспота знайдуть у своїх катехізисах чи монографіях археологічні
докази необхідності страждання, його корисності, цінності, його неминучості.
Вони притягнуть для ствердження «істини» священні книги, свідоцтва авторитетів,
хитросплетіння діалектики. Проте що нам до цього? Що нам базікання учених або
жерців світових релігій, які твердять, буцімто вони знайшли істину, але якщо
їхня духовна чи інтелектуальна деспотія не принесла людству Радості й Любові?!

Єдине, що варто шукати, в ім’я чого варто змагатися й помирати, — це РАДІСТЬ! Або ЛЮБОВ, бо це одне й те ж! А якщо істина — не Любов, то нам не потрібна істина. Проте відкинемо слова, рушимо до суті!

Я збагнув: у корені цього Світу відсутня Любов і Радість. Бо те, що породжене Любов’ю, не може й не повинне страждати!

Хто стверджує, що Світло може проявитися лише на тлі Пітьми, що Радість неможлива без Страждання, той перебуває на первісному рівні мислення, котре ґрунтується на біполярному, чорно-білому розмежуванні Світу, той ще не проник у багатомірність Світобудови, в її веселкову поліалектику, в її глибинну Свободу.

Всі «закони» детермінізму та необхідності, які вивчаються нами, — всього лиш покривало для бездуховних сліпців, полонених ілюзією механічного Космосу. Але облишимо теорію.

Необхідно знайти рішення, що лежить в корені цього світу: Злочин і Зрада — чи Випадок, який сотворив наш Всесвіт у неосяжній космічній Лотереї? Випадок відпадає, бо Світ надто постійний у своїй неухильній і нескінченній жорстокості. Воістину практично всі інформаційні шляхи Ойкумени, всі русла Софії-Мудрості узурповані хижаками мороку. Тож сподіватися на чудоподібне зникнення Світу Муки — пусте заняття.

Необхідні шляхи нові, небувалі, парадоксальні. Потрібна радісна Альтернатива,
Альтернатива Волі, Альтернатива руйнування Космічної В’язниці. Ось чому я
вирішив прийняти до своєї душі цей вічний біль і збагнути Світ, тобто —
збагнути себе!

І що ж я узрів, приклавши до Світу мірило Любові, мірило дитячого серця?

ЛАБІРИНТ МІНОТАВРА — ось справжня назва нашого світу.

Елліни залишили мудрий і глибокий міф, який відображає страшну реальність.
Звіролюдина, котра знаходиться в кожній клітинці життя, невпинно вимагає
страхітливої данини — енергії Світла, Духу, перепрацьованого в процесі еволюції
в тріпотливу тканину життя. Всі їдять всіх, щоб жити!

Але навіщо? Який сенс у цьому потворному взаємопожиранні?

Ця реальність катівського Лабіринту давно жахала духовних мандрівників. Вони
шукали виходу з вселенського черева у Світ Радості, тому що передчували його в
своєму серці. Чулися могутні заклики до Звільнення впродовж тисячоліть. І люди
прагнули услід покликам, але… ще темнішою ставала ніч, ще густішим — туман
ілюзій, ще шаленіше гарчав Мінотавр, ще вище мурувалися стіни Лабіринту.

Вселенський крематорій працює на повну потужність. У топці космічного Молоха
горять тіла, метали, душі, ідеї, держави, нації, племена, планети, зірки. Так
звана еволюція, а з нею й цивілізація — то лише шлаки страхітливого
палахкотіння Матерії у багатті Часу й Простору. Містики й окультисти, віруючі
та аскети мріють про Еру Преображення, де страждання й пошуки відшкодуються
осягненням Істини й Царства Любові. Даремні й марні ці очікування! Вони — той
комплекс ідей, що дозволяє утримувати в Лабіринті страждаючі душі людей.

Прагматична наука і скептичні філософи відкинули потойбіччя, містику, аскетизм
і захопилися земною метою. Так, вони створили безліч практичних цілей. Але
яких? Куди їхня мета веде?

Соціальні концепції оновлення людського життя на практиці показали всю марність
побудови гармонійного світу в ілюзорній сфері тління й відносності. Ці
концепції просто замінили потойбічну релігію поцейбічною, сформувавши ієрархію
уже повністю земних божків-ідолів.

Чому люди уміють мужньо захищати себе, своїх близьких, друзів, рідну країну,
чому вони можуть рятувати лахміття з пожежі, — і вони ж покірно й байдуже ждуть
загибелі всього сущого?

Дивовижний та грізний парадокс. Він мусить бути з’ясований. Тому що якщо справа
й далі піде так, то Планета й Людство приречені. Ні контроль над кількістю
населення, ні праця вчених над створенням синтетичної їжі, ні перспективи
переселення в інші світи зоряного Космосу, ні мрії про опанування термоядерної,
анігіляційної чи іншої енергії, ні проекти збереження Біосфери й захисту
вимираючих видів флори й фауни — ніщо не в силі вирішити головної проблеми:
проблеми Єдності Життя, а з нею й проблеми Єдності Сфери Розуму. Досягнення
найдивовижніших результатів в тій чи іншій окремій галузі науки, творчості,
народного господарства чи соціології можуть навіть прискорити руйнацію Життя,
якщо вони переслідують інтереси частини, а не Цілості.

Але хто і де на Землі думає й творить в ім’я Цілості? Ми бачимо лише її
словесні уламки, що носять назви «Бог», «Людство», «Космос», «Мир», «Істина» і
т. д. Але реальні дії здійснюються лише заради ілюзорних частин, тимчасових
псевдореальностей, що зникають у хвилях Хроносу. Історія тисячократно
продемонструвала людям, що всі їхні труди, війни, створення імперій, повстання,
походи, ідеологічні протистояння — даремні, що вони завжди лишаються духовно
оголеними перед оком Вічності й повинні знову й знову братися до безглуздої
Сізіфової праці, піднімаючи на гору камінь абсурдного буття.

Де причини вражаючої інерції? Втома душі? Сон Розуму? Чи гіпноз невідомого
Ворога, зацікавленого в знищенні Життя?

Необхідно знайти отруйний вірус і знищити його, щоб повернути Буття до Радості.
Заради цієї мети можна віддати все, тому що ми здобуваємо все — і навіть
більше.

Рецепт рятунку єдиний — пробудження. Повернення до Першожиття, котре вічне,
незнищенне і світоносне. Пробуджений погляд по-дитячому бачить те, що тяжко
побачити мудрецям, зануреним у вивчення дискретностей. Він може побачити й
шляхи виходу зі Світового Лабіринту.

Відверте Слово Співчуття має звучати безперервно. Чи зможе воно втілитися в
життя, — залежить від глибини сну людей, закоханих у своє хворе ілюзорне буття.

У цей час останніх можливостей, коли кожне відкрите зусилля може бути рятівним,
ми мовчання вважаємо злочином — якщо серце вимагає термінової дії для
пробудження душ, здатних до Єднання. Тільки спільно можна утвердити шлях
збереження свідомості й розуму серед вселенського хаосу, а отже — шлях до
рятунку Психосфери, що сформована в Океані Всесвіту тяжкою віковічною Еволюцією
Духу.

Чи вірить автор цих рядків в успіх?

Безумовно! Космічний переворот Свідомості здійсниться. Бажано, щоби це Дійство
відбулося з мінімальними руйнуваннями, без «архангельських труб», без кривавої
вакханалії революцій і контрреволюцій.

Учитель Серця в «Пісні Надземній» говорить: «Нитка щастя може бути зіткана.
Тому ще раз звернемося до Серця Народного».

Істинно, лише Серце Народне без лукавих мудрувань може зрозуміти велич і
незвичайність грядущого народження людини від Духу — вогняного Народження.

Окультизм і містика, релігія і марновірні течії зучили думати, що народження
згори, заповідане Христом, — це щось трансцендентне й надприродне. «Пісня
Наземна» ж говорить: «Без містики, але як Соратники Матері Світу, пройдемо в
нову Країну завзято, радісно й стрімко. І хай ніщо не зіб’є вашого ритму».

До Нового Народження покликані всі — вчені й хлібороби, будівники та інженери,
шукачі тонких технологій та романтики зоряних доріг, поети й барди, художники й
зодчі, керівники народів і всі, хто прагне Нового Світу.

Нове Небо й Нова Земля суджені. Ця віра зігрівала мільйони сердець подвижників
у всі віки тяжких духовних пошуків. Де основа такого переконання?

У надії на підтримку Духа Цілості, що незламний при будь-яких падіннях окремих
людей, народів та цивілізацій…

\subsection{Слово перше МІРАЖІ БУТТЯ}

Найбільша біда сучасності — брехливість або відносна правдивість інформаційних
каналів: періодики, книг, радіо й телевізії, шкіл, кіно, традицій, історичних
та релігійних переказів. Лише окремі мислителі здатні усвідомити фальш
суспільного буття, і мало хто має мужність заявити про це.

Свідомість людства засмічена лавиною жупелів та штампів. Більшість людей
мислить газетними заголовками, фразами з популярних пісень, дотепними
анекдотами. Тому необхідно оголити проблему, як би не було боляче. Без болю не
відкинути тисячолітню кору сплячки й невігластва.

Отже, проблема катастрофи. Під загрозою не окремий острів чи континент, не
система чи ідеологія, навіть не цивілізація. Під загрозою повного зникнення зі
скрижалей Буття — саме Життя, а з ним — і Всесвіт, що усвідомив себе як чудо
свідомої Духосфери.

Не стихії, не тварини і не рослини поставили Світ на край загибелі. Це —
людська вина й проблема. Тож, якщо Людина, будучи лідером Життя, замахнулася на
Корінь цього Життя — на Цілість, якщо їй вдалося потрясти основи Буття, то
необхідно отримати точні відповіді на запитання:

Що таке Людина?

Що таке Світ, в якому вона перебуває і який нерозривний з нею?

Життя — це випадок чи сама сутність Буття?

Свідомість — хвороба чи основа, творець структури чи її функція?

Що рухає Світ — Випадок чи певна Воля?

Чи такий Світ, яким його вивчає наука, чи ми бачимо лише сни свідомості і
вважаємо їх дійсністю?

Що таке історія — реальність світу чи тимчасова координата загальнолюдської
свідомості?

Хто формував мови, науку, релігію, технологію?

Прогрес — реальність чи соціологічна функція?

Чи існує Світовий Лабіринт? Хто його Творець — Випадок чи розум?

Чи є так звана Еволюція і що це таке? Хто досконаліший: найпростіші чи
високорозвинуті істоти? Де критерій «досконалості»?

Чи існує таємний Світовий Уряд, і хто його очолює — Брати Людства чи Вороги
Людства?

Це не пусті запитання, і є досить підстав, аби їх розглянути. Ось лише кілька
передумов:

\begin{itemize}
	

\item • Людина, нібито мисляча істота, що має незаперечну перевагу над тваринами,
упродовж всієї відомої історії займається жорстоким самознищенням.

\item • Людина, яка розуміє Єдність Життя, старанно руйнує його, знищуючи вітку, на
якій росте.

\item • Людина, маючи мужність у потрібну мить померти ради якихось проблем чи
				цінностей, полохливо виконує накази нікчемних правителів під гнітом
								незрозумілого страху.

\item • Людина, котра явно бачить катастрофічність ситуації, поглиблює її й
				переслідує тих, хто намагається розбудити сплячих.

\end{itemize}

Ці та інші нез’ясовні факти свідчать про своєрідний гіпноз, в якому перебуває
Людина, або ж про древню Світову Хворобу, кульмінацією котрої буде грядуща
катастрофа Життя.

Так, гіпноз безсумнівний. Навіяні ним МІРАЖІ БУТТЯ і спричинили різнопланове
руйнування Свідомості. Але для того, щоб збагнути сутність цих міражів,
необхідно підійти до розуміння безумовної основи Буття й мислення, визначити те
безсумнівне, що всіх нас об’єднує.

Безумовні основи Буття — Свідомість і Космос, на полі яких відбувається
містерія Життя, та естафета Життя й Розуму, що передається з вічності у
вічність. Чому ця тріада є безумовною основою?

СВІДОМІСТЬ тому, що це єдине явище, про наявність якого ніхто не сперечається:
усі проблеми — це проблеми свідомості. Будь-який процес пізнання — це питання
свідомості і відповідь відомості. Аби про щось, говорити і за щось братися —
необхідне усвідомлення. Так визначається лідерство свідомості.

КОСМОС тому, що він — загальний для всіх мислячих і не мислячих істот. Він є
загальним фактом для свідомості всіх. Його походження — це інше питання, але
саме це питання й буде найголовнішим у визначенні катастрофічності ситуації.

ЕСТАФЕТА ЖИТТЯ Й РОЗУМУ тому, що є живий ланцюг поколінь, котрий свідомо і
несвідомо тче загальну тканину Буття. Саме естафета Життя й Розуму творить
релігію, соціологію, науку, космогонію і шукає осмислені моделі Світобудови.
Але чи могла ця естафета передати крізь ланцюг поколінь правдиве відображення
Цільного Буття, щоб уникнути міражів?

Розглянемо, що ж породило Свідомість, а отже, й розуміння Космосу, і оцінку
Історичної Естафети упродовж відомого періоду часу? Естафета Життя й Розуму
розкривається в трьох фазах суб’єктивного Часу: в минулому, сучасному й
майбутньому. Жорстокий взаємозв’язок явищ, невблаганна причинність ілюструє
жахливі феномени Буття.

\subsubsection{У МИНУЛОМУ}

Світи хаосу й титанічної боротьби — в космосі й на землі. Яре кипіння
Першожиття в океанах і початок взаємопожирання: десь там уже почалася лихоманка
буття і світова Хвороба. Ускладнюючись, породжуючи химерні форми Життя, Матерія
не вилікувалася, а втягнулася у ще жахливіші цикли потрясінь і конфліктів.
Довжелезні періоди царювання рептилій, їх страшна безкінечна боротьба,
припинена згодом космічним катаклізмом; поява ссавців, а з ними й приматів —
ембріонів Людини, — не зупинила лютої хвороби буття, а перевела її в план
духовного протистояння, у Сферу Розуму. І розгорнулась тисячолітня битва Світла
й Мороку, Гармонії й Хаосу, радості й відчаю. Так страшна естафета Життя й
Розуму передалася в сучасне.

\subsubsection{У СУЧАСНОМУ}

Розтерзаний, подрібнений світ. Кровоточива Планета, що пливе в невідомість.
Зримий Космос байдужий до долі Людини й Життя на Землі. Отже, вся суть проблеми
переноситься у Внутрішній Космос Духу.

Що ми отримали від минулого?

Сотні держав та племен, відгороджених стінами ворогування.

Загальне недовір’я, породжене розбоєм, грабунками, насиллям, шовінізмом. Ріст
злочинності, зростання жадоби до зовнішніх задоволень та споживацтва. Згасання
народної культури. Розповсюдження псевдокультури з допомогою потужної індустрії
світової інформаційної мережі. Сексуальне й ідеологічне розтління юних
поколінь. Падіння кращих традицій.

Безперервні війни. Нагромадження ядерної зброї, що не дає нікому жодного шансу
на перемогу, не кажучи вже про безцільність будь-яких «перемог».

Безперспективність дипломатичних зусиль. Брехливість політиків. Взаємні підозри
і загальний страх. Відсутність загальної для планети буттєвої платформи,
достойної Людини, — і це найголовніше. Жахлива сліпота більшості лідерів. Втома
й цинізм цілих народів. Руйнування тканини Життя, зникнення тисяч видів рослин
і тварин, отруєння океанів, морів, річок, землі, атмосфери. Повне спотворення
еволюційної ритміки Природи, котра не має сили відновлювати те, що ми руйнуємо…

\subsubsection{У МАЙБУТНЬОМУ}

Мільярди жадібних душ, позбавлених духовності й співстраждання, що прагнуть
лише розкоші та насолод: океан злочинів та збочень. Ніхто й ніщо не наповнить
цю бездонну прірву бажань!

Виснаження землі. Падіння врожайності. Задуха, бо флора гине.

Голод. Смерть мільйонів. Епідемії. Вимирання тварин, бо вони не зможуть жити в
агонізуючій Біосфері, серед лахміття екологічних зв’язків.

Деградація психіки. Натовпи дегенератів, девальвація знання, бо процес його
засвоєння перетворився в «заучування». Вузькі «спеціалісти», котрі більше
нагадують біороботів. Неконтрольовані спалахи насилля, тероризму, несподівані
війни, атомна катастрофа.

Це — кінець нашого, людського світу.

Якщо Природа й справиться з жахливим зараженням й отруєнням Сфери Життя, якщо
вона й відтворить Біосферу, то це вже буде не наш історичний світ…

Про що ж ми говоримо? Невже про грізні пророцтва, невже про апокаліптичні «чаші
помсти»? Чи не достатньо чули про це люди впродовж віків?

Жахи ніколи не стимулювали людей до пошуків кращого рішення. Навпаки — вони
вбивали ініціативу й змушували кидатися в обійми містики, фанатизму й покірної
приреченості. Тому ми згадали про грядущу катастрофу не для залякування, а щоб
нагадати про можливість Преображення, про можливість порятунку того, що може
бути врятоване.

Але що рятувати?

Залишки Біосфери? Для чого? Щоб розтягнути агонію?

Чи рятувати Людину як вона є — з усіма її збоченнями й жахіттями?

Суть проблеми не в тому, щоб самозберегтися (адже тілесно так чи інакше
помирають усі); суть у тім, щоб не тільки зупинити катастрофу, але й відтворити
Єдність, а отже — Радість Життя, котра суджена як головний прояв Самопізнання.

Аналізуючи минуле, сучасне й грядуще, ми бачимо, що всі жахи Буття — це єдиний
процес розвитку Космічної Хвороби, що уразила Життя.

Певно, знайдуться вчені апологети катастрофічного світосприймання, котрі
стануть стверджувати «законність» та «необхідність» такої ситуації. Знайдуться
і «святі отці», що вкажуть на «гріхопадіння» Людини і стануть стверджувати
невідворотність світового покарання.

Ми рішуче повстанемо проти приреченості. Ми стверджуємо Святість та Радість
людського Духу. Ми стверджуємо його Праведність та Космічне Синівство. Ми
стверджуємо, що Він здатен і мусить повернути Радість Життя. Отруйний вірус
падіння і хвороби необхідно вигнати — і відновити Цілість. Але лічені роки
зосталися до грізного порогу. Сплячі — не прокинуться! Але тим, хто не спить,
ми пропонуємо разом розглянути принципи Суми Любові, що кличе нас до
Материнського Світу Радості.

Але для того, щоб зрозуміти й прийняти суть цієї Альтернативи, необхідно
розглянути МІРАЖІ ПСЕВДОБУТТЯ, ці грандіозні космічні й історичні фікції, котрі
завели людство у лабіринт, порушивши гармонію Цілого.

МІРАЖ ЕНЕРГЕТИЗМУ або ЧАСУ — головний міраж, що став основою багатьох інших.

Зруйнована в результаті світової хвороби цілісність породила ритміку обмежених
проявів — Час, а разом з ним — Енергію як динаміку цих проявів.

Ефект розпаду Цілого, хвиля сили, відчужена від Внутрішнього Космосу і
відірвана від Джерела, оформлена у зовнішні стихії — ось що таке Енергія, що
лягла в основу тканини розпорошеного світу.

Не зрозумівши Руху та його проявів (Енергії, Сили, Могутності, Напруги), Людина
запрагнула оволодіти Зовнішнім Космосом, використовуючи часово-ефемерний аспект
Руху — Енергію. Погляньте неупередженим оком на творіння так званої технічної
революції, і ви збагнете, що ми ввійшли в протистояння з Цілим, що ми стали
його паразитами, ворогами, руйнаторами.

Ми творимо енерготарани для подолання простору і часу; ми використовуємо
могутність Матерії для руйнування того, що нездатні збагнути; ми схожі на
нерозумних дітей, котрі в пороховому погребі влаштовують феєрверк, коли
бомбардуємо потоками часток атомні мішені синхрофазотронів, ґвалтуючи
мільйоноградусну плазму в магнітних тюрмах. Ми грубо вторгаємося у світ атома,
клітини, психіки, ігноруючи єдність Сущого, і не розуміємо, що безвідповідальні
вторгнення в інші сфери життя можуть викликати ланцюгову реакцію розпаду.

Необхідно збагнути, що Енергія як прояв зовнішнього руху — це лише грубий
аспект світової Динаміки Духу, Космічної Свідомості. Ми будуємо цивілізацію на
цьому примітивному, часовому підмурку і тим прирікаємо себе на неминучу
загибель, не підготувавши свою свідомість до оволодіння Духовною Динамікою.

Уся наука, вся технологія збудована на ілюзорній ідеї, буцімто Всесвіт — це
джерело невичерпної енергії. Висуваються прожекти про грядуще використання
енергії в масштабах випромінювання зірок, галактик, метагалактик. Що за чудний
міраж? І для чого це потрібно, якщо навіть сучасна людина мріє про Тишу, цю
священну Матір Цілого?!

Збільшення технічної могутності вводить Людину в конфлікт з усією Світобудовою.
Ми знаємо, що від нерозумного використання і руйнування екологічного кільця
Планети деградує і рухається до загибелі вся Біосфера. Так само і при сліпому
оволодінні тими чи іншими енергорівнями мікрокосмосу чи Макрокосмосу ми
порушуємо стихійну рівновагу Природи, масштаби, походження і призначення котрої
неможливо пізнати з допомогою нашого деформованого бачення.

Енергетична криза, що уразила цивілізацію, — не випадкова. Надії на те, що
органічне паливо буде замінене термоядерним синтезом, — наївні. Енергія
творящого синтезу — це привілей Світу Єдності, а не світу руйнації, в якому ми
живемо і котрий живимо своїм розділено-дискретним мисленням.

Вимагають перегляду самі уявлення про СВІТЛО як про певну хвилю випромінювання
(чи про потік квантів, фотонів). Скоріше за все, СВІТЛО — це посередник між
внутрішнім та зовнішнім Космосом, котрий зв’язує Світового Суб’єкта зі Світовим
Об’єктом.

Оскільки Світло — найвище (для нашої свідомості) прояв Світової Динаміки, воно
тче світ форм, проникає в них і одухотворює речовинний світ. Те, що вимірює та
вивчає фізика, — лише динаміка речовини в полі гравітації, сплячої Свідомості.
Світло невловиме, ми помічаємо лише його ілюзорний слід, тільки його
енергетичну шкаралупу, що виникає при контакті зі світом форм, «заморожених»
Часом.

Використовуючи енергію, ми виснажуємо себе, свій власний дух. Це —
самопожирання в буквальному розумінні. Кільце взаємозв’язку дуже тісне, воно
губиться в безодні Часу й Простору, і важко побачити, звідки бере початок те
Джерело, з якого ми смокчемо енергетичну кров. Але грядуще відкриє нам ту
істину, що її знали древні мудреці, котрі подарували нам символ змії, що кусає
свій хвіст, чудовий символ самопоїдання.

Ясно лише одне: що більша енергетична потужність, котру ми використовуємо, то
більш напружене протистояння Космічного Цілого. Ми узурпуємо ті основи, що на
них тримається сам Фундамент Світобудови.

Сума Любові — Альтернативна Еволюція — кличе людство до пошуку нових шляхів
Буття: неенергетичних чи наденергетичних. Залишивши в спокої стихії,
співпрацюючи з ними в процесі творення гармонійних і прекрасних форм Життя, ми
отримаємо доступ до таких таємниць і глибин Буття, про які не можуть навіть
мріяти сучасні вчені. Ми агресивно відвойовуємо крихти зруйнованих, покалічених
таємниць, а необхідно увійти в Дім Таємниці друзями. Тоді й Таємниця зникне і
зітче для нас нову чудесну форму Життя.

Отже, підсумуємо.

Увесь енергетичний Всесвіт, зорі, галактики, мікросвіти, макросвіти — все це
лише наше колективне Творення, розіп’яте на безодні Часу і Простору в спробі
утвердити свою самообмеженість у певній системі координат. Дух Людини забув про
свою Прабатьківщину-Цілість, про своє право на Повне Буття і намагається
спорудити жебрацьке буття в сферах тління й смерті.

Чи наближаємося ми до таємниці буття, нагромаджуючи енергоможливості?

Ні. Ми тільки викликаємо більшу тугу і спрагу нових джерел енергії.

Всі наші зазирання в «інтимну таємницю кожного», у надра атома, зірок, у
глибини генів, у нейрони мозку показують нам лише невичерпні лабіринти
структур, безперервне бурління речовини в ритміці просторово-часових процесів.
Ніякої таємниці в цій фантасмагорії антижиття ми не відкриємо. Наші подорожі до
«інших світів» — це лише польоти у власну пустоту, непотрібне нагромадження
інформаційного сміття, що закриває від нас власну сутність. Новий Світ рішуче
позбудеться деспотії енергетизму як породження Хроносу-Часу.

Відомий астрофізик Козирєв стверджував, що час породжує енергію. Не будемо
аналізувати, що мав на увазі вчений, розробляючи цю концепцію. Але, зі свого
боку, повністю згодимося з ним. Саме час творить енергію. Саме Хронос рубає
Світове Ціле на часово-просторові прояви, творячи з цих осколків-квантів у
міріадах психік мозаїку оманливого буття. Динаміка цього кінематографу Всесвіту
і створює енергетичний світ.

МІРАЖ ПРОГРЕСУ або ІСТОРІЇ — не менш жахливий та підступний. Людство повірило в
грубу казку про те, що Світ прогресує, вдосконалюється, йде до гармонії. На
ґрунті цього міфу творилися соціологічні моделі, що кликали героїв та
подвижників до Преображення світу, до його перебудови. Але чи не дивний намір?
Де критерій покращення світу? Де зразок такого покращення? І що покращувати:
лише людину, чи тваринний світ, чи рослини і весь комплекс біологічних
структур? Саме питання «покращення» — фікція, міраж! Адже Всесвіт — не
спеціалізована машина, функції котрої можна вдосконалити, а таємнича Єдність
Внутрішнього й Зовнішнього, що було завжди і всюди. Отже, необхідне не
«вдосконалення» чогось, а розкриття глибинної сутності Ядра Життя, котре
виправить все, що деформоване.

Ми були свідками страшних злочинів в ім’я «прогресу»: колоніальні війни і
безжальне знищення туземного населення та грабунок їхніх багатств, загарбання
Америки і майже повне знищення оригінальної культури індіанців, жорстокі
європейські війни, столітні багаття для спалення інакомислячих під знаком Того,
хто Любов проголосив основою Нового Світу, криваві війни, революційні
потрясіння XX століття в ім’я людськості і свободи, нинішні нескінченні
локальні війни, перевороти, гризня політиків, переділ світу, змова за спиною
народів і багато чого, про що прекрасно знають всі мислячі люди.

То про який прогрес можна говорити? Як може жахливе минуле породити «сяюче
майбутнє», що його обіцяють пророки земного раю? Хто спекулює міфом «прогресу»,
той просто духовний підлотник і ошуканець…

О, ми знаємо і пам’ятаємо про подвиги тих чи інших людей, рухів, груп. Ми
знаємо про великі душі, що прийняли голгофу ради друзів своїх. Ми захоплюємося
відважними дослідниками, мандрівниками, духовними подвижниками, героями, що
зневажали смерть, великими трудівниками, що подарували світові красу творінь і
плоди дивовижних відкриттів.

Але хіба прийняв світ від них естафету любові, розуму і подвигу Краси? Світова
структура, розіпнувши їх, тільки зводить замучених у ранг «святих»,
використовуючи факт подвигу для того, щоби заштопати дірку, пробиту героями в
стіні тюремного світу; використовуючи ідеї тих, що впали, ветхий світ
виготовляє підробку, одягає маску змін і під цією маскою продовжує чинити
підлість ще більш витончено, аніж раніше.

Отже, можна говорити не про прогрес, а про періодичні спроби романтиків,
героїв, подвижників руйнувати твердиню оманливого, тіньового світу, зіткану
падкою свідомістю, — і про їх невдачу. Але… невдачі накопичуються, і саме вони
несуть в собі нові можливості.

Щоб прагнути до Нового Світу, до Нового Шляху, необхідно чесно й відкрито
поглянути на історичний балаган, потрібно зрозуміти, що він створений і
управляється вмілими й підлими руками. Потрібно прослідкувати, як створювалися
історичні міфи, брехливі традиції, фальшива історіографія, котра ліпила певний
образ народів і навіть сам тип Людини.

Не може бути й мови про те, щоб цей історичний світ ввести в нову епоху,
створити з цих фрагментів, з кривавого лахміття щось ціле, гармонійне.

Необхідно прийняти в серце біль і підлість сучасності, щоб пройти крізь вогонь
очищення й спрямувати себе до Радісного Преображення.

Та чи багато тих, хто зможе збагнути таємницю Преображення, що так просто
відкривається всім зрячим у трансформації повзучої гусені в крилату
психею-метелика?

МІРАЖ РЕЛІГІЇ більш витончений. Він виростає з віри в духовну Ієрархію
Всесвіту, у наявність за всією фантасмагорією Буття Батьків — Творців Сущого.

Усі релігійні моделі використовують інтуїтивне відчуття людською душею своєї
Єдиносутності з Коренем Буття, свого Космічного Синівства. Та замість того, щоб
природно розвивати це відчуття в почуття Всеєдинства й Всеродства, жерці
релігій створили могутній апарат церковної адміністрації, що зацікавлений уже
не в духовному розкритті людини, а в її самоприниженні.

Син Бога і Спадкоємець Космічних Скарбів стає нікчемним грішником, «блудним
сином», що упродовж тисячоліть повзає в мороці земного лабіринту, очікуючи
якогось «Страшного Суду», копійованого з жорстоких трибуналів середньовіччя.

Передчуття радісної Зустрічі з Батьком та Матір’ю Буття трансформується в страх
перед Небесним Судією, а отже, деградує внутрішній Божий Світ Душі,
припиняється ріст, розвиток, розкриття Духовного Космосу.

Людина починає шукати виходу із зачарованого кола: зовнішній світ — смерть,
жах, самотність; у сфері духу — караючий суддя! Поміж цими жорнами душа
перемелюється і від болю кидається від полюсу до полюсу: або стає рабом земних
диктаторів, або рабом лжебогів, що одне й те ж.

У релігій теж є свій «історизм» або ідея своєрідного прогресу. Очевидно, ця
ідея — відгомін древнього знання космоісторії. Спочатку — блаженний стан
Людини, потім — падіння і всі жахи буття поза Богом, потім — спокутуюча Жертва
Христа і грядуще воскресіння праведних в іншому світі.

Так, все логічно. Але логічно з нашої, земної точки зору — з точки зору
жорстоких людей, позбавлених співстраждання. Ми творимо богів «за нашою
подобою», навіть не сміючи допустити, що Найвища Сутність — Єдність і радість
Буття. Чи можуть від неї йти руйнівні чи нищівні імпульси?

Згадаємо Великого Вчителя Нового Заповіту. Ось де Син Людський, осяяний Світлом
Батька Сущого! Дух, що прагне до Преображення, Дух Великої Жертви, Дух, що
підносить кожну людину, бо Він нерушимо утвердив, що «Царство Бога внутрі нас».
А де Царство — там і Цар. Отже, Бог у нас. Ми його Діти. Ми — фундатори
Всесвіту, ми продовжувачі Його Творчої Дії. Ось яке покликання Людини!

Як далеко ми відійшли від Нового Заповіту! Як багато часу й можливостей
втратили. Міраж обманних релігій має бути розвіяний, щоби Людина могла
вернутися до судженого Всебуття…

Атеїзм — інший полюс тієї ж помилки. Поклонятися зовнішньому богу чи
розвінчувати його — це одне й те ж. І апологети, й супротивники утверджують
Володаря Світу Цього: теїзм та атеїзм — його права й ліва руки. У цьому ключі
треба розглядати всі окультні доктрини, всі містичні таємні товариства, всі
«одкровення», всі феномени-обіцянки небувалої могутності та нових можливостей.
Всі можливості знаходяться в Людині, і Божественний Птах всередині тільки й
чекає, щоб розірвати шкаралупу Земного Яйця і злетіти в Небо Свободи.

Саме до цього кличе Сума Любові.

МІРАЖ НАУКИ ще більш витончений. Відкинувши ілюзії «потойбічного світу, раю,
небесного царства», людина попрямувала до зовнішньої ілюзії, створивши
аналітичну науку. Вона увірувала в те, що, руйнуючи Ціле, розпинаючи, ріжучи і
класифікуючи частини, можна збагнути Ціле, оволодіти ним.

%6

Що за безглузда помилка! І яка злобна впертість!

Чим заслужила фізична людина — двонога істота з ряду приматів — щоб увесь
Всесвіт упав до її ніг? Що несе вона Природі навзамін того, що вона постійно
бере? А брати без віддачі — це космічний злочин.

Наука не замислюється над цим. Апологети «точної науки» крушать все навколо —
атомні структури, живі клітини, лізуть у психіку і гени, в минуле й майбутнє,
прагнуть до далеких світів, щоб і там ствердити свої честолюбство й зверхність.

Що вони намагаються знайти в мікросвіті? Ті ж битви й конфлікти, що й на Землі?

Що вони шукають в далеких світах? Про що хочуть говорити з іншими розумами,
якщо не можуть домовитися з такими ж, як і самі вони?

Справді, про що говорити із Зоряними Братами? Що розказати їм про нашу Землю?
Чим похвалитися? Безкінечними війнами? Отруєнням Планети? Тим, що озброїли
хижих політиків комічною зброєю загального руйнування?

Великий злочин Науки перед Розумом і Совістю. Бо найстрашніші диктатори не
змогли б нічого зробити, якби апологети науки не вклали в їхні руки таємну силу
землі й неба.

Воістину, Наука — хижак, що прикинувся велелюбним створінням, звір, який обіцяв
народам рай земний, а натомість укинув Планету в тяжкі кола техногенного Пекла.

Світ, як Жива Істота, зникає. Він закутий у бетон, у залізо й пластмасу. Він
обплутаний сіткою дроту й обманної інформації. Він знемагає під мільйонами
коліс і гусениць, що мучать його плоть. Він отруєний пестицидами та іншими
«цидами». Він задихається від випарів отруйного газу і нескінченного потоку
всілякої мерзоти, що пливе з мегаполісів та міст.

Навіть багато учених жахнулися від того, що самі накоїли. Дехто закликає до
здорового глузду, дехто заперечує небезпеку. І ті, й інші — ошуканці й
негідники!

Необхідні не запобіжні заходи (хіба не все одно, як убивати і мучити — швидше
чи повільніше?), а рішуча Альтернатива Буття. Необхідно збагнути небезпеку
зовнішньої аналітичної «науки» як ілюзорного інструменту псевдопізнання, бо
неможливо пізнати Ціле, дроблячи його на частки, агресивно вторгатися в
сокровенні глибини Життя, викликаючи тим самим його спротив. Ми не пізнаємо
сутність Буття, а воюємо з ним.

Наука Нового Світу вивчатиме не ілюзорне плетиво тіньових часових структур, а
невичерпні глибини Внутрішнього Космосу, котрий є зодчий і Космосу Зовнішнього.

Пізнавши себе, тобто Внутрішнє, Людина пізнає і Зовнішнє, стане Розумним Духом
Усесвіту. Не «закон» диктуватиме Людині сенс дій та волінь, а Людина творитиме
динамічні закони для Зовнішнього Космосу — Закони Радості, Краси і Єдності.

Хай не гніваються друзі пізнання, а добре подумають, що за інверсія відбулася з
проблемою гнозису-пізнання, кому стали слугувати нащадки древніх мудреців, що
свято берегли таємниці Матері Світу від жадібних прислужників Мороку? Збагнувши
свою зраду, вони рішуче спрямують свій шлях до Альтернативи!..

Небезпечний і МІРАЖ ЗОВНІШНЬОЇ ТВОРЧОСТІ. Він повів людей по шляху Каїна,
котрий, за біблейським міфом, був першим будівником, творцем наук, ініціатором
мистецтв, держав і т. д. Цей міраж обманув Людину надією на творення в
зовнішньому світі того, що було втрачено у внутрішньому. І що ж ми бачимо?

Безкінечне будівництво міст, палаців, пірамід, лабіринтів, храмів, картин,
скульптур, доріг, імперій, держав, ідеологій, наук, релігій, світобачень,
розваг і багато чого іншого. І вся ця «творчість» — всепожираючому Часу на
з’їжу. Тільки пилюга над колишніми імперіями, тільки каркання зловісних птахів
над шляхами кривавих завойовників, зотлілі кістки грізних диктаторів, лише
огризки древніх «знань» і «вірувань», тільки втома народів від нескінченного
шляху в нікуди, від непотрібної, тяжкої «творчості», від безперервного
будівництва «осяйного майбутнього».

Чому ж триває безглузда «творчість», чому пливуть потоком брехливі книги,
фальшиві картини, нікчемні теорії, сірі мудрування? Чому ростуть нові міста,
поглинаючи живу природу, готуючи все нові й нові комфортні в’язниці квартир,
офісів і ресторанів, з яких людям ніколи не вирватися?

Хтось скаже, що без такої творчості людина деградує, що в цьому — сенс Буття і
покликання Людини.

Це — напівправда. Людина, воістину, суджений Творець Всесвіту, але творець
духовний. Вона могла б стати Розумом і Духом Буття, направляючи Стихії й
Елементи Природи до гармонійного співзвуччя й Краси. Але не ліпити з крихт
розтерзаної Єдності, не творити тіньові образи занепалого Світу, — а прагнути
до Само-творчості, до Саморозкриття. Як це роблять зорі, квіти, окремі
мислителі, що вказали Шлях до Світів Духу.

Сума Любові не заперечує зовнішню творчість, ще протягом багатьох літ вона буде
необхідною як обрамлення головної будівлі — Світу Самотворчості, при якому
Людина почне ліпити сама себе й нові гармонійні форми Зладованого Космосу.

А зовнішня творчість (і тому є безліч прикладів) у більшості випадків веде в
ілюзорні світи безкінечної марноти…

МІРАЖ ТІЛЕСНОСТІ, або ФОРМИ — грізний міраж. Він полонив дух Людини і
нерозривної з нею фауни ілюзією окремості. Він, роздробивши Ціле, протиставив
форму формі, він розчленив Єдину Сутність Буття на океан дискретностей, частин,
що хаотично танцюють у безмежності під пресом «законів» тяжіння чи
відштовхування.

Тіло необхідне як ступінь самоусвідомлення серед Вічності, серед безодні Буття.
Це — сходи розкриття невичерпного потенціалу Духу, це — нові горизонти Світів і
Можливостей. Тіло — інструмент Творящого Духу для життя й діяльності в певній
сфері Буття. Але це в тому випадку, якщо тіло чи форма динамічні й слухняні до
волі Кореня Життя. А якщо вони самозберігаються і закривають зір Духу, тоді
вони перетворюються на космічну в’язницю, на міраж. Тоді тілесний інтелект
створює інструмент самозбереження у вигляді наук, релігій, соціології,
традицій, здорового глузду і т. д. Ілюстрація: кокон з метеликом, що
розвивається всередині. Якщо форма-кокон переконає метелика, що її треба
«зберегти», то метелик не з’явиться на світ, а задихнеться в самоствореній
тюрмі.

МІРАЖ ФОРМИ породжує ще один — МІРАЖ МНОЖИННОСТІ. Ця фікція дискретності,
роздробленості, роздільності Світу переслідує нас постійно, щохвилини буття.
Зорі, атоми, частинки, люди, звірі, хмари в небі, зотлілі кістки дідів, камені
колишніх палаців, новонароджені, мурашки, птахи, іскри сонця на хвилях води —
все тікає, розсипається, все це тримається разом тільки завдяки творчій потузі
Світової Свідомості, але чи надовго вистачить цієї потуги? Невже до
безкінечності слуги Мороку будуть терзати частини тіла Осіріса, кидати частини
розтерзаного Орфея в потік ілюзорного буття?

Ціле втратило Себе серед безодні і намагається з’єднатися із своїми ж
частинами, шукає їх, запалюючи маяки Любові, Краси й Розуму серед страшної
пустелі Небуття. Але цього не стається. Бо у Світі, поряд з істинними частинами
«Володаря Світла» діють самозванці, актори Мороку, що намагаються запобігти
Об’єднанню Світів. Головна їхня зброя — «Закон Дзеркала». Частини «Володаря
Світла» — Першосутності Духовного Світу — відображені в дзеркалах Мороку і
породжують легіони ілюзорних двійників. Маючи можливість діяти у світі форм,
паразитуючи на таємничій сутності Життя, псевдодвійники Реального постійно
тчуть примарне Буття Зовнішнього Світу, використовуючи духовний матеріал Синів
Світла. Так вони затримують Духів Творення в тюрмі Часу й Простору, змушуючи
крутити упродовж міріад років Колесо Обманного Буття.

Результат цього — страшна самотність серед пустелі зоряного світу. Самотність —
ось бич Буття. Біль самотності штовхає весь світ до пошуків втраченої Єдності
(чи Раю, за містичною термінологією), але ці пошуки ведуться на шляхах
суєтності й марноти. Саме біль самотності породжує агресію, ненависть і любовну
пристрасть — підробку Єдності.

Так, любов теж агресія, хоч і ніжна. Вона не задовольняється сама собою, але
намагається полонити іншого, щоб передати тому свій біль, щоб забути про
необхідність пошуку.

Самотність породила тілесність, тілесність — форму, форма — енергетизм,
енергетизм — силу, сила — перевагу, перевага — ієрархію Світобудови.

Так був розтерзаний Єдиний Світ, і Міражі Псевдобуття перемогли Дух Цілості.

Але де ж джерело цього світового гіпнозу? Хто навіяв страшний космічний сон?

Чимало традицій древності стверджують дуалізм Світобудови. Вони говорять про
наявність у світі певної сили, що зачаїлася біля самих витоків Еволюції і
руйнівно впливає на процес Життя.

Так, відповімо ми. Необхідно мужньо й відкрито глянути у вічі правді. Древній
Господар Світу — космоісторична сутність, із життям якої зв’язана проблема
Падіння. Говорячи сучасною термінологією, мова йде про Інтегральний Інтелект
Планети, або про Супермозок Біосфери Життя.

\subsection{Слово друге ІНТЕГРАЛЬНИЙ ІНТЕЛЕКТ ПЛАНЕТИ}

Прадавні легенди говорять про падіння Високої Сутності Буття, про те, що після
цього Падіння у Світ прийшли смерть, руйнація і страждання. Сатана, Люцифер,
Аріман, Мара, Чорнобог — численні синоніми цього Духа. Хто ж він такий? Чи
реальна ця Індивідуальність, чи це тільки певний символ космічного явища?

Традиції юдаїзму та породженого ним християнства стверджують, що Люцифер був
Високим Архангелом, Світоносцем, Сином Зорі, Первістком Світобудови. Він
загордився і повстав на свого Творця. Після страшного поєдинку в Небесах він і
його воїни були скинуті в нижчі сфери, де вони намагаються спотворити людську
Еволюцію, що народжена Творцем як Альтернатива Грішним Могутностям.

Прямолінійне розуміння древніх переказів заводить нас у хащу схоластики та
філософських абсурдів. Звісно, могутня Світоносна Сутність не могла б повстати
на Корінь Життя, що її породив. Очевидно, що мова йде про Індивідуальність, що
виникла історично.

Очевидно, при гармонійному розвитку Світового Цілого кожен новий Цикл Буття
супроводжується і народженням нової Свідомості, що охоплює і творить все більш
глибокі й багатопланові сфери Життя і Радості. Така Свідомість не замикається в
обмеженій Сфері, а насичує її напругою творчості й дії, перетворюючи в Зерно ще
величнішого Буття. Безкінечна Пісня Радості — ось що таке життя при такій
послідовності Циклів.

Але у випадку дисгармонії, у випадку хвороби Світового Плоду, його розвиток
може бути спотворений. Змагаючись із хворобою, Світове Життя почало творити
деградовані форми і деформованих істот. Його еволюційна ритміка була
лихоманково-патологічною, його рівновага досягалася ціною взаємопожираючого
кругообігу. Були народжені не законні форми краси та єдності, а спеціалізовані
форми: хижаків і травоїдних, паразитів і їхніх господарів…

А в кульмінаційний момент якогось Циклу, коли в Сфері Життя мала виникнути
Індивідуальна Свідомість, відбулася інтеграція хворобливих елементів і
фрагментів цього «лихоманкового» життя. Так з’явився «Розум» Біосфери як певна
об’єднуюча функція елементів світового життя. Будучи психічним фокусом динаміки
спотворених форм, він з моменту свого народження не міг позбутися комплексу
неповноцінності. Власне, його можна було б назвати Духом хвороби і розпаду: так
навіть прекрасна істота під час хвороби може перетворитися в монстра.

Цілість не була знищена цією планетарною бідою, бо Корінь її недоступний для
зовнішніх деформацій. Динаміка глибин Життя вступила в тривалий, тяжкий процес
боротьби за оздоровлення.

Але появу будь-якої нової якості в біосфері Інтегральний Інтелект сприймав як
загрозу собі, тому він використав свої «адміністративні можливості», свою
потугу Супермозку для деформації цієї і нової якості відповідно до свого
розуміння і потреб хвороботворного буття. Він «планував» екологічні цикли
взаємопожирання, він встановлював жорстоку рівновагу в Природі, він змушував
Матерію творити замість гармонійних форм кошмарні зразки своєї уяви.

Та крізь ці хащі еволюційного хаосу, крізь агонію й ричання страшних чудовиськ
Світовий Зародок прагнув протиставити космічній Хворобі Розпаду свою головну
силу, здатну вилікувати Єдиний Організм. Ця сила — Радість Єдності, це Любов,
втілене Світло. І носій її — Людина.

Міфи й казки древності передають дивовижні вісті про Золотий Вік, коли люди,
тварини, птахи і рослини були єдиним Вінком Буття. Все було Всім, і Джерело
Життя було єдиним. Радість була сенсом буття і харчем, енергією руху і сутністю
саморозкриття.

Світ став виліковуватися. Це була радісна Епоха Титанів, Епоха Урана.

Та хвороба повернулася. Зародок Супермозку, мов ракова клітина, зачаївся в
глибинах Життя, і вірус, що уразив колись форму життя, зумів проникнути в розум
Людини.

Так було заражене саме Джерело Буття — Свідомість, Світлоносний Центр Життя.
Єдність знову розпалася.

Інтегральний Інтелект отримав доступ до сфери Духу через поневолені свідомості
людей. Тепер він міг формувати не лише хворобливі прояви життя, але й впливати
на Світову Цілість, на Космічний Розум. Природа, позбавлена керівного розуму
Титанів, почала керуватися Законом. Інерцію Закону використав Інтегральний
Інтелект для повного оволодіння сферою Життя та Психосферою Людства.

Свідомість мислячої істоти, розділена з Цілістю, побачила Світобудову
відчуженою і зовнішньо-ворожою. Титан став ліліпутом, духовним пігмеєм. Форма
прояву змінилася відповідно. Всі життєві елементи, озброївшись одне проти
одного, втратили можливість пластичної трансформації і вічного розкриття. Світ
заснув, завмер, ввійшовши в Колесо Самопожирання, в обійми Світового Змія.

Людина впала зі свого трону лідера Життя і стала хитрою, підступною, жорстокою
істотою. Почалася драма, про яку ми говорили раніше: історичні цикли
взаємознищення, війн, знищення «братів менших» — тварин, руйнування самої
тканини Світового Життя…

Лише Флора — рослинний покров Матері Світу — зберегла ще безпосередній зв’язок
з Джерелом Світла і пропонує своє життя, свої плоди для кривавого світу тварин
і людей.

Супермозок міг би святкувати перемогу над Єдністю. Та Корінь Життя для нього
недоступний. У надрах Таємниці зріє Зерно Нового Світу, котрому суджено
замінити хворі зерна нинішнього Циклу.

Ґерць жорстокий і нещадний. Більшість взагалі не зрозуміє, про що мова! Що за
нова космічна релігія? Що за планетарні кібернетичні монстри?

На горе людства і всього життя — це так. По всій космоісторії проходить чорна
рука Ворога Буття. Поєдинок ветхого й нового — дух будь-якого історичного
циклу. Наука вважає битву протилежностей у Природі «законом», але це — лише
прояв злісного консерватизму Деспота, що не бажає жодної трансформації.

Історія релігій та окультних товариств усіх віків чітко визначила вплив
Космічного Узурпатора на земне життя. Його втручання було занадто явним,
жорстоким і нещадним. І стихійний спротив Людського Серця і Розуму, що має
зв’язок з Коренем Буття, міг перетворитися у відкрите, свідоме протистояння.
Тому необхідно було створити ті Міражі Буття, про які ми говорили раніше.

Цілі історичні цикли пішли на формування цих ментальних та чуттєвих фікцій.
Можна коротко прослідкувати послідовність їх створення і боротьби Цілості за
повернення Єдності.

Традиції древніх народів одностайно відзначають пантеїстичний світогляд
пралюдей. Наші предки сприймали Природу як щось нероздільне з собою, своїми
відчуттями, думками, діями, прагненнями. Не було, і не могло бути відчуження
зовнішнього від внутрішнього: все було єдиним в єдиному потоці Буття.

Головний удар був нанесений саме тут. Через своїх апологетів Інтегральний
Інтелект створив «Ієрархію» — Космічну Адміністрацію. Раніше у природні рівні
Буття могла увійти будь-яка творча свідомість, що доросла до такого рівня.
Тепер же вони були блоковані грізним наказом Самозванця, чому допомогло
створення світових релігій.

Уважно проаналізуємо: ідея нікчемності та гріховності, залежності й тління;
почуття приреченості і смертності; невідворотність грядущого суду і
невизначеність вироку; неосяжність світобудови, відділеної від людини, і
слабість самої людини — ось які зерна сіяли в серця людей апологети багатьох
релігій.

Хитрість та лукавство псевдодуховних жерців ще й у тому, що вони сміливо
використали переказ про Ворога Життя, увели його в концепцію буття.
Паталогічний образ Сатани-Супротивника став лялькою-страховидлом, щоб прикрити
жорстокість самого Господаря Світу. «Бог світу цього» і його «супротивник» — це
одна й та ж сутність, це — хворий Інтелект хворого Світу, що розігрує фарс
фальшивої дуальності для того, аби обманути істинного Володаря Буття — Людину.

%%%8
%https://www.litmir.me/br/?b=199222&p=7

Процес приниження й розділення йшов на всіх планах і сферах. Дух Єдності став
Духом Протистояння. Діти однієї Матері стали ворогами. Брати стали володарями,
рабами, жебраками, вбивцями, катами, духовними шарлатанами, брехунами, творцями
і руйнаторами. Кривава ріка розділила сильне колись Серце на міріади захололих
псевдосердець.

І найголовніше: було принижене Слово.

Ми зараз практично не можемо збагнути істинну сутність Слова. Ми вивчаємо його
морфологічний аспект чи акустичний, фоновий чи синтаксичний. Та це безглузде
заняття: тінь не дає уявлення про явище, що породило її. Автор «Євангелія від
Івана» правдиво й глибоко передав дихання Істини в урочистих рядках: Спочатку
було Слово, і Слово було у Бога, і Слово було Бог.

Непромовлене Слово і є Корінь Буття, що породжує всю глибинну невичерпність
Життя. Людство, як паросток від цього Кореня, володіло даром Слова, а з ним —
можливістю бути співтворцем Світобудови. У гармонійному Світі Слово — це
сукупність Внутрішнього і Зовнішнього, повнота всіх потенцій та можливостей.

Але у світі хворому, де запанував Узурпатор, Слово було затемнене. Розділені
істоти втратили дар Єдиного Слова. Господар Світу створив Антислово, підробку
взаєморозуміння. Та навіть у цю тіньову мову безперервно вторгалися слуги
Мороку, щоб гасити то там, то тут окремі, сяючі в пітьмі іскри Бога-Логосу.

Слово, що було колись живим Творцем всіх явищ і Сукупністю їх, стало нікчемним
символом, знаком, котрий могли вільно міняти й спотворювати законодавці
Антилогосу.

Філософії, теїзми, світоглядні спекуляції — все це темна безодня Антилогосу,
Антислова, що ними Інтегральний Інтелект Планети намагається одурманити
Свідомість Синів Людських.

Скільки дивовижних зусиль і витонченості! Скільки пасток!

Міріади богів. Згодом — монотеїзм. А коли теїзм рушиться — приходить атеїзм з
його цинізмом та обмеженістю. Замість потойбічних едемів — міражі земного раю,
соціологічні мишоловки, розраховані на жадібність та честолюбство плебсу. А для
інтелектуалів — фікція пізнання: колупання в надрах хворої Матерії,
використання її занепалої могутності для цілей насичення та комфорту,
накопичення динамічної сили Речовини для руйнування світу, котрий і без того
вже в агонії. А для істинних шукачів — містичні апендикси, окультні заклики,
космічні польоти, логічні трансформації, досягнення й можливості.

Але все це — без виходу в Свободу, без порушення диктатури Інтегрального
Інтелекту, що є володарем усіх важелів влади в проявленій сфері.

Історія відзначила декілька спроб Цілості дати Людству Альтернативу. Найбільш
сильні з них — місії Будди й Христа. Будда прямо закликав до виходу з володінь
Мари, ілюзорного й хворого світу. Христос скинув деспотію Володаря цього світу
і посіяв Зерно Нового Світу у внутрішньому Космосі.

Та володар цієї сфери моментально здійснив корективу. Ім’ям Посланців зі Світу
Свободи були створені нові культи, релігії, ритуали, ієрархії. Кільце знову
замикалося. Світовий Змій торжествував.

Яка ж його мета? Чого жадає досягнути Супермозок Землі? Для чого мучить він
людей у страшних циклах самознищення й руйнацій?

Не варто шукати сенсу (у всякому разі, нашого людського) в цьому
космоісторичному хаосі. Можливо, це злобна впертість Деспота. Можливо, якийсь
задум, у якому нам, людям, відведена роль виконавців та енергоджерела, можливо,
ми упродовж тисячоліть віддаємо йому психоенергію, індуковану з допомогою
церков, соціальних об’єднань, світових воєн, походів, революційних потрясінь,
масових епідемій…

Якщо цей супермозок виник на основі хворого життя, то він ніколи не буде
зацікавлений у лікуванні, бо воскресіння гармонійного Першожиття означає його
зникнення. Тому битва за Єдність Світів для нього питання життя й смерті.

Французький вчений Тейяр де Шарден справедливо вказав на боготворчу роль
Людини: він побачив пророчим оком виникнення в ноосфері духовного вогнища
«Омега», що синтезує всі еволюційні зусилля Універсуму. Та мислитель не
відзначив руйнівного для ноосфери впливу древнього Узурпатора, котрий розділив
Потік Розуму на міріади розрізнених крапель. Чи можлива при такій ситуації
концентрація еволюційних сил навколо «Омеги»?

При тій світовій ситуації, котру ми аналізуємо, такої можливості немає. Хіба що
станеться диво. Але де основа для такого дива? Де його витоки?

Тільки серце людське несе в собі всі можливості для Преображення. Тільки воно в
силі внести в еволюційний потік нові якості, котрі стануть вирішальним
імпульсом для уздоровлення Світу і Життя.

Така нова якість і є Альтернативна Еволюція — Сума Любові.

\subsection{Слово третє АЛЬТЕРНАТИВНА ЕВОЛЮЦІЯ}

Нагромаджувати жахи — річ невдячна. Наше завдання — запропонувати Альтернативу.

Але чи багато людей збагнуть нові можливості? А збагнувши, чи запрагнуть
важкого шляху, де необхідне повне самозречення?

Занадто далеко зайшла хвороба Світу. Зовнішній ілюзіон Всесвіту зітканий так
міцно, що навіть глибокі мислителі потрапляють у полон до законів міри, числа й
ваги. Метелик б’ється всередині самоствореного кокона-космосу, і боїться
залишити його, і задихається від туги за небом, і поступово втрачає здатність
до польоту в темниці часу й простору, підмінюючи істинне життя технічними
підробками.

Необхідний вольовий вибір, що йде від Людини Розумної як

Лідера Еволюції.

Необхідно створити Альтернативний Центр, здатний протиставити себе деспотичній
владі Космічного Узурпатора. Якщо такий Центр буде створений, вся втаємничена
могутність Цілості ввіллється в його зусилля. І річ зовсім не в деклараціях, а
в щирому, глибинному прагненні до Єдності Життя.

Альтернативний рух має виходити з монолітних передумов відродження Нового
Світу:

• Усяка істота суверенна. Кожне життя священне й рівноцінне. Тому виключається
використання будь-якого життя для підтримки іншого життя. Такий екологічний
кругообіг і є біологічним лабіринтом паразитизму, за наявності якого немає
сенсу мріяти про гармонізацію Світу.

• Попередня основа Альтернативи передбачає — у перехідний період — відмову від
кривавої їжі, використання для насичення лише плодів і зерен рослин. У далекій
перспективі — самодостатність, духоенергетика, єдність внутрішнього і
зовнішнього космосу, коли Людина поверне собі Повне Буття, що переможе потреби
дискретного тіла й розуму.

• Мислення має подолати віковічні догми про «законність» тієї світової
ситуації, в якій ми перебуваємо, подолати ідеї про те, що основою буття є
«боротьба» якихось протилежностей. Боротьба є, але це боротьба Єдиного Життя
проти вікового спотворення Цілості, проти руйнівного імпульсу дискретності,
множинності, ворожнечі, що йдуть від творця Лабіринту.

• Свідомість має рішуче позбутися моделей світобудови, в яких вона закрита
серед безмежного Дому Буття, в хащі мікро- й макросвітів. Прагнучи до
Альтернативної Еволюції, ми стверджуємо центральність свідомості у всій її
космічності.

Звісно, мова йде не про часткову свідомість людини чи іншої живої істоти. Ми
говоримо про Світову Свідомість, про Дух Усесвіту. Людина виростає з цього
Духу, вона є Квіткою Дерева Життя, вона несе в собі всю могутність свого
Батька, як зерно несе всю потенцію рослини.

Але коли Розум піддався хворобі розпаду, у свідомості всіх людей сформувався
образ відчуженого Всесвіту. Утвердивши такий Всесвіт, ми стали його полоненими.
Створивши ілюзіон Буття, ми не можемо розлучитися з ним.

Згадаймо казки про сплячий світ, у центрі якого спить Царівна. Мудрість народів
передала через покоління вістку про цю світову біду. Спляча Царівна — це Душа
Світу, котра не може прокинутися від гіпнотичного сну. У казці її пробуджує
герой, що зумів подолати заслін сплячого світу і вбити темного чаклуна, винного
у злочині. У реальному бутті це має здійснити героїчний Розум, що безстрашно
заглянув у сутність життя і побачив там отруйного Ворога, що напоїв отрутою
Радість.

