% vim: keymap=russian-jcukenwin
%%beginhead 
 
%%file link.04_12_2008.stz.news.ua.pravda.tolkachev_aleksej.1.mria_pro_ukrainu
%%parent 04_12_2021.fb.tolkachev_aleksej.1.mria_pro_ukrainu
 
%%url 
 
%%author_id 
%%date 
 
%%tags 
%%title 
 
%%endhead 

\href{https://www.pravda.com.ua/articles/2008/12/4/3627035/}{%
Мрія про Україну, ОЛЕКСІЙ ТОЛКАЧОВ, pravda.com.ua, 04.12.2008%
}

\begin{multicols}{2}

\begin{textQuote}
\obeycr
\enquote{А завзяття, праця щира свого ще докаже,
Ще ся волі в Україні піснь гучна розляже,
За Карпати відіб'ється, згомонить степами,
України слава стане поміж народами}.
\restorecr
\end{textQuote}

\textSelect{Куплет, який було виключено з офіційного гімну України у 2003 році.}

Патріотом сьогодні бути дуже важко. \enquote{Любіть Україну!} - така соціальна реклама
часто зустрічається в Києві та інших містах. Але свідомість підштовхує до
питання – за що любити? Соціальна реклама не наводить жодних вагомих
аргументів.

\headTwo{За що любити Україну?}

За те, що вона стала злою мачухою для мільйонів українців, які були змушені
емігрувати за кордон? Або за роздовбані дороги?

Можливо за те, що села вимирають, а землі стоять не засіяні? А може варто
любити Україну за те, що ми маємо найбільші показники купівлі люксових дорогих
авто в Європі?

Чи за те, що українські виродки, що називають себе елітою, добре розуміються на
гламурі?

На сьогодні, після 18 років незалежної державотворчості в політиці панує
цілковитий абсурд, безідейність, цинізм та корупція, і постає питання про
збереження територіальної цілісності України.

Батьківщину люблять за те, чим можна пишатися. Дехто зазначить, що українська
співачка Руслана виграла Євробачення – цим варто пишатися.

Однак це привід любити співачку Руслану, а не Україну, яка не доклала до цього
жодних суттєвих зусиль, а потім ледь не провалила підготовку Євробачення-2005 в
Києві.

Брати Клички, футболіст Шевченко або співак Гришко – їх знає весь світ. Однак
чи причетна Україна до успіху цих талановитих українців?

Від України шарахаються в ЄС, не визнають навіть перспективи її членства. В
НАТО Україну ніби й чекають, але бажають триматися осторонь. Східний Старший
Брат цинічно насміхається над Україною, періодично навіть не визнаючи, що така
держава існує.

Сьогоденна Україна – це огидна реальність. Саме тому за даними опитувань
49,65\% українців хотіли б працювати за кордоном, з яких 25\% - залишити
Україну назавжди.

Що ж тримає в Україні тих, хто свідомо не їде? І тим більше, тих пасіонаріїв,
які щось роблять для загального блага, задля позитивних соціальних, культурних
та економічних зрушень?

\headTwo{Чи не єдине, що залишається в умовах суворої реальності – це Мрія про Україну}

Є така, Мрія про Україну. Про прекрасну щасливу багату Україну – \enquote{Країну щастя
і добра}, як співає Михайло Поплавський. Ця мрія існує в майбутньому і не має
нічого спільного з реальністю. Ця мрія вселяє віру, дає сили й надію, вона є
дороговказом життя Громадянина і країни.

На жаль, більшість громадян цю Мрію на сьогодні втратили, вона розвіялася,
розбилася об камені реальної безнадії. Тепер їм доводиться виживати, не
відчуваючи майбутнього, розраховуючи лише на власні сили. Або мріяти про
переїзд з України туди, \enquote{де нас нема}, де їм буде даровано мрію про
світле комфортне майбутнє.

Сучасна трагедія України полягає у всезагальній деморалізації та дезорієнтації.
Жодна концептуальна ідея не досягла такої величі, щоб стати загальною Мрією про
Україну, захопити і спрямувати розум і дії мільйонів Українців, перетворитися
на справжню Національну Ідею.

Політичні партії не знаходять загальної підтримки, оскільки теж не можуть
запропонувати народу гідну Мрію про Україну. Їм доводиться ґрунтувати виборчий
успіх на брутальній бездушній технології, а всі їхні ідеї зводяться до того,
хто, що і як реформує, або хто пообіцяє більшу виплату по народженню дитини.
Примітив, не здатний пробудити зернятка віри в людей і запалити їхні очі.

Політична еліта, яка б мала вести народ шляхом прогресу, натомість пропонує
зовсім інші мізерні ідеї – спробуйте мрію про Україну в складі Росії, або про
Україну в складі НАТО. Для когось мрія про Україну зводиться до ринкових реформ
або підвищення пенсії до рівня прожиткового мінімуму.

Існує мрія про Україну, де шанують жертв голодомору, а для когось Україна – це
мрія про садок вишневий коло хати і вранішній спів соловейка. \enquote{Україна для
українців!}, \enquote{Європейська Україна}, \enquote{Злагода!} - хіба це ідеї?

\headTwo{Мізерність ідеї обумовлює відсутність політичної волі для творення в Україні
чогось більшого за реальність - прекрасного і процвітаючого.}

Коли ви мрієте про що-небудь, ваша мрія обов’язково найкраща! Так мріється про
те, що Україна є такою ж казковою, як Швейцарія, або такою ж фантастичною і
багатою, як Арабські Емірати. Що реформи здійснюються такими ж темпами та
ефективністю, як у процвітаючій сьогодні Південній Кореї.

Деморалізовані обивателі (а, може, й інтелектуали) заперечать, що це неможливо.
Однак маючи багаті спостереження за європейськими країнами, автор може з
впевненістю констатувати, що в Україні немає жодних об’єктивних причин, через
які ми не можемо жити так само як у Франції, Швейцарії чи в будь-якому іншому
процвітаючому куточку планети.

\headTwo{Для відродження й розвитку нашої Батьківщини потрібна Мрія про процвітання
України. Мрія сьогодні народжується з приходом нових поколінь, нової справжньої
еліти, не закутої в лещата гламуру, стереотипів та реальності... Ця Мрія буде
варта того, щоб в неї повірили і за нею пішли.}

Кількість пасіонаріїв, які керуються Мрією, в Україні з кожним днем зростає.
Згідно з теорією Льва Гумільова про пасіонарні поштовхи, мабуть, один з таких
відбувається зараз в Україні. Дуже скоро посіонарна хвиля нової генерації
повинна сформулювати для всієї України \enquote{шлях прогресу} - Українську Мрію, нову
загальну Національну Ідею.

Якою вона буде? Поки що можна зробити лише поверхневий сумбурний нарис.

Мрія про Україну не матиме нічого спільного з девізом сучасного державного
гімну \enquote{Ще не вмерла Україна}. І не вмре! Навпаки, Україна покаже диво розвитку
та прогресу всій Європі.

Це може бути схоже на \enquote{новий курс Рузвельта} або \enquote{японську революцію Мейдзи}.
Головне, що за дуже короткий час Україна повинна опиниться на принципово іншому
рівні цивілізаційного розвитку. І це цілком реально.

Стрімкі реформи зможуть започаткувати нові форми соціальних, економічних і
політичних відносин. Основних зусиль держава докладе до підтримки і розвитку
творчого потенціалу українського народу.

Тоді світова криза виявиться не трагедією, а трампліном для України, яка, ніби
інвестиційно привабливий проект світового масштабу, залучить міжнародні
фінансові ресурси.

Україна переосмислить найпрогресивніші надбання та ідеї західної цивілізації,
проведе їхню модернізацію та реалізує на практиці. Адже майбутнє – за новітніми
формами неолібералізму та синергетичної організації світу, а також оновленими
давніми ідеями гуманізму і калокагатії.

Наукова обґрунтованість та доцільність має нарешті стати критерієм державної
політики та соціальної організації. А підтримка фундаментальної науки в Україні
дуже скоро призведе до наукових проривів, які виведуть Україну в число світових
лідерів. Бо українські вчені вже мають розробки принципово нових джерел
енергії!

Національна ідея України ґрунтуватиметься на розкритті потенціалу народу, на
прогресивних ідеях майбутнього, на ролі України для всієї Європи і
слов’янського світу, а не на сучасній трагічно-жертовній історичній риториці.
Позитивне національне мислення обумовить позитивний результат.

В основі суспільної організації лежатиме можливість кожного займатися своєю
спорідненою працею – тією, яка буде приносити задоволення та самореалізацію
людині, а не лише фінансовий ресурс на прожиття.

Уявіть собі Україну, в якій Львів – культурна столиця не тільки України, але й
всієї Європи? Виставки, вернісажі, карнавали...

Донецьк –– міжнародний фінансовий центр. Чим не новий Франкфурт-на-Майні або
Нью-Йорк? \enquote{Донецька міжнародна фондова біржа} - звучить?

Одеса може стати центром кінематографу. Що заважає? Будемо проводити міжнародні
кінофестивалі в Одесі. Буде український Голівуд на березі моря.

Харків – технополіс, університетський та науковий центр світового рівня, який
за короткий строк може не тільки відродити вітчизняну науку, але й дарувати
світу розробки майбутнього. Був же з середини 1920-х до середини 1930-х років
Харків однією із світових столиць ядерної фізики, авіабудування, біохімії,
психології та театру!

Херсон, Крим, Вінниця, Житомир, Маріуполь, Бердянськ – в майбутній Україні
місце знайдеться всім. Наприклад, у Франції намагалися придумати свою роль для
невеличкого містечка Ле Манс.

І придумали – започаткували знамениті автомобільні перегони, за якими щороку
спостерігає весь світ! Головне – розвиток інфраструктури, якою буде активно
займатися держава, інвестуючи в майбутнє.

І кожне місто України, а не тільки Київ, буде \enquote{містом квітів} не лише на
ідіотських білбордах, а й в реальності. Люди тоді не тікатимуть з провінції до
столиці у пошуках кращого життя, бо реальні шанси і можливості самореалізації
будуть скрізь!

\headTwo{Українська мрія – це ідея цивілізаційного прориву, яка забезпечить докорінну
зміну картинки за вікном, народить гармонійний і красивий світ. Українська мрія
– це гідне місце України в світі, який стане невпізнанним після світової
фінансової кризи.}

Для народження нової України потрібно не так вже й багато часу. Скажімо, за 18
років незалежності більша частина могла б бути вже пройденою... Потрібна лише
політична воля, чистий творчий розум та любов до народу. Ці риси матиме нова
генерація української еліти – пасіонарії, романтики, Творці.

До формулювання Української Мрії і народження нової України залишилося зовсім
небагато часу. Але необхідно не просто чекати, а й вірити! У нас є шанс стати
провідниками чудових змін!

\end{multicols}
