% vim: keymap=russian-jcukenwin
%%beginhead 
 
%%file 09_12_2022.fb.ohmatdet.bolnica.1.vyshgorod
%%parent 09_12_2022
 
%%url https://www.facebook.com/ndslohmatdyt/posts/pfbid02V3eCtRZgXNL2qTwLgEhZeyEMeJHqDS8pZ2zLxL7nT6irX1EcxkBC1VrMJ76dSjnXl
 
%%author_id ohmatdet.bolnica
%%date 
 
%%tags 
%%title Дочка ледь не позбулася ока, а синові розірвало вухо: в Охматдиті лікуються постраждалі з Вишгорода
 
%%endhead 
 
\subsection{Дочка ледь не позбулася ока, а синові розірвало вухо: в Охматдиті лікуються постраждалі з Вишгорода}
\label{sec:09_12_2022.fb.ohmatdet.bolnica.1.vyshgorod}
 
\Purl{https://www.facebook.com/ndslohmatdyt/posts/pfbid02V3eCtRZgXNL2qTwLgEhZeyEMeJHqDS8pZ2zLxL7nT6irX1EcxkBC1VrMJ76dSjnXl}
\ifcmt
 author_begin
   author_id ohmatdet.bolnica
 author_end
\fi

⚡️Дочка ледь не позбулася ока, а синові розірвало вухо: в Охматдиті лікуються постраждалі з Вишгорода⚡️

8-річний Сашко та 12-річна Таня уважно слідкують за повітряними тривогами та
при кожній загрозі спускаються в укриття лікарні. 23 листопада в їх дім
прилетіла російська ракета, забравши життя 7 сусідів. У цей час вся їх родина
також перебувала вдома.

«Все сталося прямо в одну секунду. До нас у квартиру влетів балконний
дерев'яний блок. Ударною хвилею мене скинуло з дивана, а над дітьми, які сиділи
подалі від вікна, нависла віконна рама, дивом їх не накривши. А ось чоловікові
бетонний блок \enquote{проїхався} по ногах»,– згадує мати Наталя.💔

Жінка на якийсь час втратила свідомість, вона отямилась, коли почула запах гарі
— сусідній під'їзд почав горіти. Її чоловік на перебитих ногах зміг розгребти
завали та вибити двері квартири. Далі він втратив свідомість. Дітей з під'їзду
допоміг винести незнайомець, який кинувся на допомогу постраждалим. Після
виписки Сашко і Таня сплетуть для свого героя патріотичні браслети, щоб
віддячити за допомогу.🙏🏻

Спочатку дітей відвезли у Вишгородську лікарню, де надали першу медичну
допомогу, потім — в Охматдит. Тут їх оглянули хірурги приймального відділення,
офтальмологи, отоларингологи, діагности, дітям провели КТ обстеження. У Наталі
була сильна контузія, Сашку зробили операцію та витягли осколки з обличчя,
зашили розірване вухо. Найбільше постраждала 12-річна Таня, яка отримала
глибокі поранення обличчя.❤️🩹

«У дівчини ураження м'яких тканин обличчя, постраждала скронева зона. Дитині
дуже пощастило, що осколки дивом не травмували очне яблуко, попри те, що 2
осколки ввійшли між стінкою орбіти і очним яблуком. Спеціалісти Охматдиту
провели операцію дівчині та зберегли зір дитині»,— розповідає завідувачка
відділення гнійної хірургії Надія Павлівна Кисіль.🏥

Перші два дні діти не говорили про те, що сталося. Замість цього вони
передавали один одному записки на папері. З постраждалими займається психолог,
вже зараз діти активно спілкуються та мають позитивний настрій і багато планів
на майбутнє. Перше до чого хочуть повернутися, так це до занять з карате. Уся
їх родина займається цим видом спорту. Діти вже неодноразово були переможцями
всеукраїнських змагань. Купа їх медалей так і лишилася висіти на стіні в їх
кімнаті, попри вибухову хвилю.⚡️

📸 Тетяна Бундзило 

Більше фото за посиланням: \url{https://photos.app.goo.gl/8b2wJjgyuDjipuPg7}

% 1,2,3
\ii{09_12_2022.fb.ohmatdet.bolnica.1.vyshgorod.pic.1}
% 4,5
\ii{09_12_2022.fb.ohmatdet.bolnica.1.vyshgorod.pic.2}
% 6,7,8
\ii{09_12_2022.fb.ohmatdet.bolnica.1.vyshgorod.pic.3}
% 9,10,11
\ii{09_12_2022.fb.ohmatdet.bolnica.1.vyshgorod.pic.4}
% 12,13

\ii{09_12_2022.fb.ohmatdet.bolnica.1.vyshgorod.orig}
\ii{09_12_2022.fb.ohmatdet.bolnica.1.vyshgorod.cmtx}
