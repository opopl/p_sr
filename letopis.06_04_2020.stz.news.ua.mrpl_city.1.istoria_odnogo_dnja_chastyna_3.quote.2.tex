% vim: keymap=russian-jcukenwin
%%beginhead 
 
%%file 06_04_2020.stz.news.ua.mrpl_city.1.istoria_odnogo_dnja_chastyna_3.quote.2
%%parent 06_04_2020.stz.news.ua.mrpl_city.1.istoria_odnogo_dnja_chastyna_3
 
%%url 
 
%%author_id 
%%date 
 
%%tags 
%%title 
 
%%endhead 

\begin{quote}
\small
\enquote{Все 3 приказа, написанные на 3-х отдельных листах Ильяшенко спрятал во
внутренние карманы своего пальто; затем стал вплотную перед писарем и сказал
\enquote{смотри}, растегнул верхние пуговицы жилета и вынул из-под него складной
медальйон из красной меди, висевший на Андреевской ленте. В этом медальйоне был
портрет Государя...

\enquote{Смотри}, говорил Ильяшенко, наступая на писаря, \enquote{ты знаєш, кто я такой, видишь
чем я награжден от Государя. Знай, что если ты известишь начальника команды до
ревизии мною суда, сегодня же будешь повешен}! С этими словами, произнесенными
со зловещим грозным шепотом, Ильяшенко быстро вышел из канцелярии. У писаря
отнялся дух, опять потемнело в глазах, застучало в голове: не донести
начальнику – беда, донести – еще хуже, мелькнуло в его сознании: недвижимый,
словно окаменелый, остался он в своей канцелярии, как будто тысяча цепей
приковали его к месту...}.
\end{quote}

