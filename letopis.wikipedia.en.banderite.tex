% vim: keymap=russian-jcukenwin
%%beginhead 
 
%%file wikipedia.en.banderite
%%parent wikipedia.en
 
%%url 
 
%%author 
%%author_id 
%%author_url 
 
%%tags 
%%title 
 
%%endhead 
\section{Banderite - Banderivtsi}
\label{sec:wikipedia.en.banderite}

Збережено: 02.06.21
\url{https://en.wikipedia.org/wiki/Banderites}

The Banderivtsi (Ukrainian: Бандерівці, Bandе́rivtsi or bandе́rovtsy, Polish:
banderowcy, Russian: Бандеровцы) are members of an assortment of right-wing
organizations in Ukraine.

The term derives from the name of Stepan Bandera (1909-1959), head of the
Organization of Ukrainian Nationalists that formed in 1929 as an amalgamation
of movements including the Union of Ukrainian Fascists.[1][2] The union, known
as OUN-B, had been engaged in various atrocities, including murder of
civilians, most of whom were ethnic Poles. This was the result of the
organization's extreme Polonophobia, but the victims also included other
minorities such as the Jews and Romani people.[3][4] The term "Banderites" was
used by the Bandera followers themselves, by others during the Holocaust, and
during the massacres of Poles in Volhynia and Eastern Galicia by OUN-UPA from
1943–1944. These massacres resulted in the deaths of 80,000-100,000 Poles and
10,000-15,000 Ukrainians.[5]

According to Timothy D. Snyder, the term continues to be used (often
pejoratively) to describe Ukrainian nationalists who sympathize with fascist
ideology and consider themselves followers of the OUN-UPA myth in modern
Ukraine.[6] 

History

Stepan Bandera (standing, third from the right) with members of the Chervona
kalyna Zahin, Lwow, 1928. Bandera was enlisted to OUN organization by Stepan
Okrimovich (Охрімович, sitting, first from the left)

The first murder operation carried out by the Organization of Ukrainian
Nationalists (OUN) with the active participation of the then 25-year-old
Bandera was the June 1934 assassination of Bronisław Pieracki, Poland's
Minister of the Interior. Bandera personally provided the assassin with the
murder weapon, a 7.65 mm caliber pistol.[7] His subsequent arrest and
conviction turned Bandera into an instant legend among the militant Ukrainian
nationalists of the Second Polish Republic. Bandera, who escaped from prison
after the German invasion of Poland in September 1939, offered his services to
Nazi Germany in exchange for ongoing financial and logistical support.[8]

On February 10, 1941, a conference for OUN leadership was held in Kraków,
Poland. Since 1939, Andriy Atanasovych Melnyk, a founder of the OUN, had been
placed at its head. He had been chosen for his more moderate and pragmatic
stance; his supporters admired Mussolini's fascism but condemned Nazism.
However, a younger and more radical Nazism-supporting faction of the OUN were
dissatisfied. It was at this conference that the schism solidified. This
radical Nazism-supporting contingent of the OUN refused to accept Andriy
Atanasovych Melnyk as head of the OUN and instead named Bandera. This led to
the split of the OUN in the spring of 1941 into two groups: OUN-B (Banderites),
who were more militant, younger and supported Bandera, and OUN-M (Melnykites),
who were generally older and more ideological. In February 1941, several months
before the German attack on the USSR, Bandera became the leader (Providnyk) of
the OUN-B faction, or the Banderivtsi. Five months later, in July 1941, Bandera
himself was arrested and sent to a concentration camp in Germany. He was
imprisoned there until 1944.

The OUN-B formed Ukrainian death squads that carried out pogroms and massacres
both independently and with support from the Germans.[9][8]

To ensure maximum impact of the systematic ethnic cleansing campaign in the
contested territory, OUN-B faction spread antisemitic, racist, and fascist
propaganda among the ordinary peasants and other Ukrainians.[9]:235-236 Aided
by Stetsko, Shukhevych, and Lenkavskyi (OUN-B propaganda chief), Bandera wrote
a manifesto entitled "Ukrainian National Revolution" that called for the
annihilation of so-called ethnic enemies. The manifesto informed the locals how
to behave and included specific instructions about the killing of Jews, Poles,
and Ukrainian opponents of fascism.[9]:237[10]

"OUN leaflets appeared on the city streets. They read: "Exterminate the Poles,
Jews and communists without mercy. Do not pity the enemies of the Ukrainian
National Revolution!" [10]

Bandera coordinated the pogroms from behind. He did not participate in them; he
remained in the area of occupied Kholmshchyna (Polish Chełm Land) further
north-west.[9]:237

The vast majority of pogroms carried out by the Banderites occurred in Eastern
Galicia and Volhynia, but also in Bukovina.[9]:237 The most deadly of them was
perpetrated in the city of Lviv by the people's militia formed by OUN with
direct participation of civilians, at the moment of the German arrival in the
Soviet-occupied eastern Poland.[11] There were two Lviv pogroms, carried out in
a one-month span, both lasting for several days; the first one from 30 June to
2 July 1941, and the second one from 25 to 29 July 1941.[12] The first pogrom
took the lives of at least 4,000 Jews.[13] It was followed by the killing of
2,500 to 3,000 Jews by the Einsatzgruppe C,[14] and the "Petlura Days" massacre
of more than 2,000 Polish Jews by the Ukrainian militants.[13][15] During the
pogrom, on June 30, 1941 Bandera declared a sovereign Ukrainian state in Lviv,
and a few days later was arrested by the Germans who opposed it. Bandera was
sent to detention in Germany. His supporters took over the command of the UPA
death squads two years later, in November 1943.[2][8] 

References

Rudling, Per A. (November 2011). "The OUN, the UPA and the Holocaust: A Study in the Manufacturing of Historical Myths". The Carl Beck Papers in Russian & East European Studies. University of Pittsburgh. Number 2107. p. 3 (6 of 76 in PDF). ISSN 0889-275X.
Cooke, Philip; Shepherd, Ben (2014). Hitler's Europe Ablaze: Occupation, Resistance, and Rebellion during World War II. Skyhorse Publishing. p. 336. ISBN 978-1632201591.
Lower, Wendy; Faulkner Rossi, Lauren (2017). Lessons and Legacies XII: New Directions in Holocaust Research and Education. Northwestern University Press. pp. 170–171, 174. ISBN 978-0810134508. "The victims of the Holocaust had a difficult time identifying precisely who intended to murder them; the usual terminology was "Banderites," which indicated adherents of a particular political tendency, or "Bulbas," which indicated the insurgent force initiated by Taras Bulba-Borovets.[p. 174]"
Risch, William Jay (2011). The Ukrainian West: Culture and the Fate of Empire in Soviet Lviv. Harvard University Press. pp. 55, 65, 69. ISBN 978-0674061262.
Rossoliński-Liebe, Grzegorz (2016). "Stepan Bandera, Dr. Andrii Portnov, and the Holocaust: Is the Bandera Myth Detached from the Person?". The American Association for Polish-Jewish Studies. Retrieved 2016-04-23. "[Andrii Portnov's article from the Frankfurter Allgemeine Zeitung (8 January 2016)] informs the readers about the use of the term Banderites in Soviet propaganda, but he [Portnov] forgets to mention that the OUN members did call themselves such and regarded Bandera as their leader when they were murdering Jews during the pogroms in summer 1941 and when they, in the uniforms of the Ukrainian police, were helping the Germans to shoot Jews in 1942 and 1943. He also ignores the fact that Ukrainian nationalists perceived themselves as Banderites and were perceived as such by others during the ethnic cleansings of the Polish population in Volhynia and eastern Galicia. Finally, he does not inform the readers that Bandera never condemned the atrocities committed by the OUN and UPA."
Snyder, Timothy (February 24, 2010). "A Fascist Hero in Democratic Kiev". The New York Review of Books.
Żeleński, Władysław (1973). The Assassination of Minister Pieracki [Zabòjstwo ministra Pierackiego]. Poland: Institut Literacki. pp. 20–22, 72. Biblioteka "Kultury" volume 233.
Motyl, Alexander J. (2000). Encyclopedia of Nationalism. Two-Volume Set. Elsevier, Academic Press. p. 40. ISBN 0080545246. With over one hundred contributors. "On February 10, 1941, Bandera called a conference of radicals in Kraków, Poland. The conference refused to accept Melnyk as leader, and named Bandera head of the OUN. This led to the split of the OUN in the spring of 1941 into two groups: OUN-B (Banderites), who were more militant, younger and supported Bandera, and OUN-M (Melnykites), who were generally older, more ideological."
Rossolinski, Grzegorz (2014). Stepan Bandera: The Life and Afterlife of a Ukrainian Nationalist: Fascism, Genocide, and Cult. Columbia University Press. pp. 112, 234–235, 236. ISBN 978-3838266848. "The OUN-B organized a militia, which both collaborated with the Germans and killed Jews independently."
Piotrowski, Tadeusz (1998). Poland's Holocaust: Ethnic Strife, Collaboration with Occupying Forces and Genocide in the Second Republic, 1918–1947. Jefferson, NC: McFarland & Company. p. 209. ISBN 0-7864-0371-3. OCLC 37195289. "OUN leaflets appeared on the city streets. They read: "Exterminate the Poles, Jews and communists without mercy. Do not pity the enemies of the Ukrainian National Revolution!""
Prof. John-Paul Himka (25 February 2013). "A few more words about the Lviv pogrom" [Ще кілька слів про львівський погром]. IstPravda.com.ua. Історична правда. With links to relevant articles. For the English original, see: John-Paul Himka (2011). "The Lviv Pogrom of 1941: The Germans, Ukrainian Nationalists, and the Carnival Crowd". Canadian Slavonic Papers. 53 (2–4): 209–243. doi:10.1080/00085006.2011.11092673. ISSN 0008-5006. S2CID 159577084.. Archived from the original on 4 March 2016.
Himka, John-Paul (2011). "The Lviv Pogrom of 1941: The Germans, Ukrainian Nationalists, and the Carnival Crowd". Canadian Slavonic Papers. Vol. 53 (2–4): 209–243. ISSN 0008-5006.
USHMM. "Lwów". The Holocaust Encyclopedia. United States Holocaust Memorial Museum. Archived from the original (Internet Archive) on 2012-03-07.
N.M.T. (1945). "Trials of War Criminals before the Nuernberg Military Tribunals" (PDF). Volume IV : "The Einsatzgruppen Case" Complete, 1210 Pages. Nuremberg Military Tribunals under Control Council Law No. 10: 542–543 in PDF (518–519 in original document) – via PDF direct download. "With N.M.T. commentary to testimony of Erwin Schulz (p. 543 in PDF)."

    Longerich, Peter (2010). Holocaust: The Nazi Persecution and Murder of the Jews. Oxford; New York: Oxford University Press. p. 194. ISBN 978-0-19-280436-5.

Further reading
	Wikimedia Commons has media related to Stepan Bandera commemoration.

    Valeriy Smoliy (1997), "Small dictionary of Ukrainian history" — Lybid.
    G. Demyian — "Banderivtsi" — Ternopil dictionary encyclopedia – G. Iavorskiy — "Zbruch", 2004-2010, 696p. ISBN 966-528-197-6.
