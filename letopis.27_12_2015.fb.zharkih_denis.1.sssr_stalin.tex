% vim: keymap=russian-jcukenwin
%%beginhead 
 
%%file 27_12_2015.fb.zharkih_denis.1.sssr_stalin
%%parent 27_12_2015
 
%%url https://www.facebook.com/permalink.php?story_fbid=1711210665759010&id=100006102787780
 
%%author_id zharkih_denis
%%date 
 
%%tags istoria,sssr,stalin_jozef
%%title Мое отношение к Сталину и СССР
 
%%endhead 
 
\subsection{Мое отношение к Сталину и СССР}
\label{sec:27_12_2015.fb.zharkih_denis.1.sssr_stalin}
 
\Purl{https://www.facebook.com/permalink.php?story_fbid=1711210665759010&id=100006102787780}
\ifcmt
 author_begin
   author_id zharkih_denis
 author_end
\fi

Мое отношение к Сталину и СССР

Сейчас много на ФБ разговоров о Сталине и СССР. Вот одни говорят - репрессии,
бедность, ограничение свобод. И это правда. Другие говорят - технологический
рывок, социальные программы, победа над фашизмом, выход в космос. И это тоже
правда. Правда, она такая, она разная. Хотим видеть репрессии - увидим, хотим
видеть мощь страны - не проглядим. Только одно и другое строилось из людей,
чаще всего эти люди различны. Кто-то мучил своих сограждан, кто-то двигал
человечество вперед, защищал его ценой своей жизни. И вопрос кого считать
нашими предками- тех или этих? Думаю, что нужно считать тех, кто защищал и
двигал, а мучителям в потомстве отказать.Да не просто отказать, а записать в
недостойное поведение, и помнить. Именно помнить, а не кичиться. 

Я сторонник того, что СССР был великой страной, но то, что с ни произошло,
закономерно. Когда его развалили никто не пикнул даже, значит, были причины.
Вон фашисты до последнего стояли, защищая свой Рейх, а коммунисты непонятно с
какого перепугу обделались. И эти перепуганные ничего уже обратно не вернут,
как бы не обещали. Исходя из исторического опыта, можно с уверенностью
предположить, что Союз вернется, но соберут его не те, кто потерял, и на других
основаниях. А для этого нужно понять, что было плохого, и что хорошего. Недаром
же в русских сказках оживление делалось не только живой водой, но и мертвой.

\ii{27_12_2015.fb.zharkih_denis.1.sssr_stalin.cmt}
