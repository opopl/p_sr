% vim: keymap=russian-jcukenwin
%%beginhead 
 
%%file 09_01_2022.fb.fb_group.story_kiev_ua.1.pamjati_ok_antonova
%%parent 09_01_2022
 
%%url https://www.facebook.com/groups/story.kiev.ua/posts/1836662036530589
 
%%author_id fb_group.story_kiev_ua,krysjkov_sergij.kiev
%%date 
 
%%tags antonov_oleg.aviakonstruktor.sssr,aviacia,kiev,pamjat
%%title Памяти Олега Константиновича Антонова...
 
%%endhead 
 
\subsection{Памяти Олега Константиновича Антонова...}
\label{sec:09_01_2022.fb.fb_group.story_kiev_ua.1.pamjati_ok_antonova}
 
\Purl{https://www.facebook.com/groups/story.kiev.ua/posts/1836662036530589}
\ifcmt
 author_begin
   author_id fb_group.story_kiev_ua,krysjkov_sergij.kiev
 author_end
\fi

Друзья, я собирался опубликовать этот рассказ ближе к дню рождения (7 февраля)
О. К. Антонова, однако материал уже готов (текст написан, негативы
отсканированы),  меня же \enquote{распирает}. Выкладываю сейчас. Фотоснимки ранее нигде
не публиковались.

Памяти Олега Константиновича Антонова...

Май 1978 года, Москва. В редакции журнала «Моделист-конструктор» раздался
телефонный звонок. Звонил парень из г. Борзна Черниговской области и попросил
пригласить к телефону консультанта по авиации. Трубку взял лётчик-испытатель
Г. С. Малиновский. Абонент сразу же приступил к делу:

- Здравствуйте. Я тут, сделал самолёт... Дайте мне, пожалуйста, инструкцию, как
летать?

Разумеется, по телефону невозможно научить летать, в чём лётчик и постарался
убедить звонившего чудака. Хоть они тогда и не пришли к согласию, произошло
главное: связь была установлена и продолжилась в дальнейшем, по почте и по
телефону. И вот, в середине лета того же года, Малиновский едет в Борзну. И
едет не один: вместе с ним, туда приехал Олег Константинович Антонов...

\ii{09_01_2022.fb.fb_group.story_kiev_ua.1.pamjati_ok_antonova.pic.1}

За пару дней до этого, в Киеве, в моей квартире зазвонил телефон. Я тогда
начинал карьеру фотографа на Украинской Студии хроникально-документальных
фильмов. Кинооператор Фёдор Фёдорович З., пригласил меня на «воскресную
прогулку» в г. Борзну и я, лёгкий тогда на подъём, сразу же согласился. 

\ii{09_01_2022.fb.fb_group.story_kiev_ua.1.pamjati_ok_antonova.pic.2}

В воскресенье, в 5 часов утра я вышел из дому и меня подобрал студийный ГАЗ-69.
В салоне ехали трое: я, Фёдор Фёдорович и водитель… Если это можно назвать
салоном – верх старенькой машины был брезентовый. Меж задних сидений,
расположенных вдоль бортов, лежали ящик с кинокамерой «Конвас-Автомат»,
запасной аккумулятор к ней, объективы, коробки с киноплёнкой, тяжёлый
киноштатив с инерционной головкой. У меня же был только один стандартный кофр,
в котором, в боевой готовности лежал Pentacon Six TL, два или три объектива к
нему, запас плёнки и приготовленный женой сухпаёк. Вспышку не взял, поскольку
съёмка предполагалась на свежем воздухе, а день намечался солнечный.

\ii{09_01_2022.fb.fb_group.story_kiev_ua.1.pamjati_ok_antonova.pic.3}

Мчались по шоссе, тряслись по просёлкам и, наконец часам к девяти, прибыли на
место. Припарковались в центре города, напротив памятника Вождю, возле то ли
горкома, то ли райкома партии. Я был новичок на киностудии – проработал всего
лишь год – поэтому командиром нашего экипажа был Фёдор Фёдорович. Он и пошёл к
партийным товарищам, согласовывать наше участие в предстоящем мероприятии. Я и
водитель остались в машине.

\ii{09_01_2022.fb.fb_group.story_kiev_ua.1.pamjati_ok_antonova.pic.4}

Пока ждали, к нам подошли 2 работника доблестной советской милиции. Началась
проверка документов, выяснение «кто такие, почему здесь?», «что за машина, надо
проверить, а может и задержать на всякий случай». У водителя изъяли права.
Похоже, эти ребята, считающие себя хозяевами города и хозяевами жизни, хотели
денег, а может просто решили показать чужакам свою власть. Дознание было
прервано товарищем в строгом чёрном костюме, подошедшим вместе с Ф.Ф., который
сказал им «кыш» и они сразу же испарились.

Приехали в чисто поле, на место предстоящего события. А там уже столпились
несколько десятков автомобилей и автобусов, толпа любопытствующих, пресса,
городские и партийные начальники. И тут я впервые увидел знаменитого О.К.
Антонова! Сразу достал камеру и стал щёлкать, не жалея плёнки. Антонов и
Малиновский долго осматривали стоящий там самолёт, изучая его конструкцию.
Длина – 15 метров, размах крыльев – 12, вес 320 кГ. Вместо фюзеляжа – длинная
балка, к которой подвешено сидение и мотор, а на другом её конце – хвостовое
оперение. Какой двигатель – не помню. Создатель сего чуда – местный житель,
молодой слесарь Виктор Юрченко. 

Пока уважаемая комиссия проводила экспертизу, я, прислушиваясь к разговорам,
успел отснять несколько плёнок. Фёдор Фёдорович тоже непрерывно стрекотал своей
камерой. Между делом, к нам подъехал милицейский «бобик», оттуда вылез
начальник городской милиции и, извинившись перед нами за действия своих
подчинённых, вернул водителю его права. Затем подошёл к градоначальнику,
козырнул ему – мол, выполнено – и уехал.

И вот, наконец, мэтры авиации дали добро на взлёт! Виктор сел на место пилота,
пристегнулся (сидение открыто со всех сторон), прогрел двигатель и начал
разбег. В качестве взлётной полосы был выбран прямой и ровный участок грунтовой
дороги, идущей через поле. И вдруг, какой-то [нехорошее слово], разворачивая
свой автобус, выехал на полосу, наперерез уже разогнавшейся «птичке».
Наткнувшись на этот автобус, самолёт сломал крыло. Присутствующие были
разочарованы, Виктор конечно, очень расстроен, но не подавлен. Он успел
признаться, что ранее уже невысоко летал и что это не первая его авария:
несколько недель назад, при посадке он сломал шасси (очевидцы того полёта
подтвердили, что он летал во-о-он над теми берёзами). Так что, это ещё не конец
испытаний, да и столь многочисленная зрительская аудитория никак не вдохновляла
молодого авиаконструктора.

Водитель злополучного автобуса многое услышал, как в свой собственный адрес,
так и по адресу его родственников до седьмого колена, самолёт же был
отбуксирован в сарай, для ремонта. Председатель горсовета пригласил гостей
города на поляну, где на открытом воздухе уже был накрыт стол на полсотни
персон. Олег Константинович отказался, так как уже был приглашён на обед
родителями Виктора. Пригласили они и нашу съёмочную группу.

В городе преобладал частный жилищный сектор, в одном из таких домов и проживал
Виктор с родителями. Небольшой земельный участок, на нём дом и большой сарай
для самолёта - Виктор шутя, называл его ангаром.

В Борзне население небольшое, все друг друга знали. Одни считали Виктора
чудаком («нафига оно тебе надо?») и насмехались, другие понимали его и
поддерживали. Ещё в школе, он сделал моторную лодку, позже построил аэросани.
Для самолёта потребовалось большое количество пластика – пошло в ход много
мыльниц, игрушек, футляров для очков, которые Виктор скупал в немыслимых
количествах, в хозтоварах и «оптике». Местный библиотекарь, бывший его учитель
физики, снабжал Виктора подшивками журналов со статьями по авиации и
самолётостроению.

За столом, с любовью сервированным радушной хозяйкой, расселись гости, Виктор и
хозяева – всего около десятка человек. Обед был вкусный и сытный, количество
принятых рюмок – небольшое, беседа – интересная и приятная, а мой сухпаёк
остался скучать в кофре. Олега Константиновича сопровождали его водитель и
фотограф с авиазавода, который снимал мало; с Малиновским была его супруга, она
во время войны была радисткой в партизанском отряде, поэтому мне, как бывшему
радисту, было о чём с ней поговорить.

Если гипотетически представить себе диалог с Антоновым человека, ничего не
знающего о нём, то тот скорее всего не догадался бы, что перед ним – выдающаяся
личность. Мои впечатления об Олеге Константиновиче – самые положительные. Он
держался скромно, просто и естественно, иногда даже казался немного
застенчивым. Никогда не повышал голос, с любым собеседником был
доброжелательным и внимательным, говорил как с равным, умел слушать не
перебивая. Беседовать с ним было легко и приятно. Осмелюсь предположить, что и
на работе он был таким же. Полагаю, такие черты свойственны большинству
гениальных людей.

После обеда, все вышли погулять на природе, погода была прекрасная. Виктор
много и долго говорил с Антоновым и Малиновским, остальные старались им не
мешать. Я много фотографировал. Олег Константинович подарил Виктору свою книгу
«Десять раз сначала», этот момент и запечатлён на выложенном здесь снимке. На
переднем плане Г.С. Малиновский, Виктор Юрченко и О.К. Антонов. Бородач на
заднем плане – фотограф-кинооператор с завода Антонова, который сопровождал
Олега Константиновича почти во всех его поездках. Снято фотоаппаратом
«Пентакон» с любимым мной длиннофокусным объективом Sonnar 2,8/180. Этот
объектив хорош тем, что позволяет снимать издали, не отвлекая объект съёмки от
его занятий и не смущая близким присутствием фотографа. Также, малая глубина
резкости помогает отделить главный объект от фона. Этот снимок нигде не
публиковался, но в соцсетях и на форумах я его выкладывал. Остальные снимки
выкладываю впервые.

Мы вернулись в Киев ближе к концу дня, ещё засветло, полные впечатлений и
эмоций. На следующий день, я поспешил проявить плёнки и отпечатал несколько
наиболее удачных фотографий, которые и принёс в редакцию газеты «Молодь
України», вместе с моей короткой заметкой об этом событии. В редакции всё это
приняли, однако почему-то затягивали с публикацией. На мой звонок ответили, что
материал крайне интересный, короткой статейки недостаточно, поэтому в Борзну
поехал их спецкор О. Ященко, для сбора более подробной и расширенной
информации. Через несколько дней, в номере газеты от 9 августа 1978 года, была
опубликована его большая статья «Десять разів спочатку», с одной из моих
фотографий. За этот снимок мне выплатили гонорар 3 рубля с копейками.

Кинооператор же, Фёдор Фёдорович З., повёл себя нехорошо. Он попросил у меня
негативы, якобы на время, а я, молодой и зелёный, поверил и отдал. Несмотря на
мои настойчивые требования вернуть плёнки, он их всё-таки зажал. У меня
сохранилось лишь несколько снимков, десяток негативов и эта газетная вырезка,
которая кстати, помогла мне сейчас освежить в памяти некоторые подробности.

С годами, мои эмоции поутихли, осталась лишь благодарность Фёдору Фёдоровичу за
приключение, о котором я сейчас рассказал.

(с)Айнцвайдрайченко, 2022

\ii{09_01_2022.fb.fb_group.story_kiev_ua.1.pamjati_ok_antonova.cmt}
