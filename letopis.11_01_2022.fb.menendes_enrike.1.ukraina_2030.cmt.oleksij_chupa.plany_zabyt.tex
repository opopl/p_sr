% vim: keymap=russian-jcukenwin
%%beginhead 
 
%%file 11_01_2022.fb.menendes_enrike.1.ukraina_2030.cmt.oleksij_chupa.plany_zabyt
%%parent 11_01_2022.fb.menendes_enrike.1.ukraina_2030.cmt
 
%%url 
 
%%author_id 
%%date 
 
%%tags 
%%title 
 
%%endhead 
\paragraph{Олексій Чупа - Поки існує РФ у сучасному вигляді, про такі плани можна забути)}

\begin{itemize} % {
\iusr{Олексій Чупа}
Поки існує РФ у сучасному вигляді, про такі плани можна забути)

\begin{itemize} % {
\iusr{Денис Журавлев}
Да, не переживайте, скоро РФ развалится. Просто перевернитесь с боку на бок, и всё будет хорошо :+)

\iusr{Олексій Чупа}
\textbf{Denis Zhuravlev} так я і не кажу, що розвалиться. Я впевнений, що нас чекають покоління гемору з агресивним сусідом, як у випадку Південної Кореї
\end{itemize} % }

\iusr{Алена Теодорович}
Откуда даты? ЕС не светит, его не будет. И не надо.

\begin{itemize} % {
\iusr{Энрике Менендес}
\textbf{Алена Теодорович} Да, 2030 это мечта. А вот 2035-2040 уже лучше

\iusr{Алексей Иванов}
\textbf{Алена Теодорович} 

думаю ЕС посыпется. Англия вышла. Во Франции то же крайне антиЕСовские
настроения. Следующим президентом может стать евроскептик. Тогда в ЕС только
Германия и останется.

\iusr{Кирилл Полевой}
\textbf{Алена Теодорович} вам в Воронеже точно не надо

\iusr{Алена Теодорович}
\textbf{Энрике Менендес} 

О вторых датах - можно тогда вообще обо всем забыть окончательно. И будет ли
вообще тогда Украина ? Если ей земля и люди не нужны, то другим нужны. Все.

\end{itemize} % }

\iusr{Андрей Николаев}
\enquote{дурень думкою багатіє}....

\begin{itemize} % {
\iusr{Энрике Менендес}
\textbf{Андрей Николаев} Ну так это про вас больше, чем про меня ))

\iusr{Андрей Николаев}
\textbf{Энрике Анатольевич Менендес} от це, навряд чі  @igg{fbicon.smile} 

\end{itemize} % }

\iusr{Valeria Kotskyh}

Только всё это надо было делать ДО начала военных действий. Теперь это
желанная, но увы, утопия.

Ботофермы ненависти получают хорошие гранты за свою работу((

\begin{itemize} % {
\iusr{Энрике Менендес}
\textbf{Valeria Kotskyh} 

Если бы удалось избежать войны вообще была бы другая история. Надо искать
выход, ведь нынешняя ситуация ещё не худшая из возможных. Если не будет так, то
какая альтернатива?

\iusr{Алексей Иванов}
\textbf{Энрике Менендес} 

думаю, что Украине нужно пройти финский или японский путь. Потери ВСУ в
Донбассе где то около 4 тысяч. Это много, но не критично. Народ ещё не испытал
болевой шок. Война не зашла в каждый двор, не дотронулась к каждому. А вот
когда пойдут потоки гробов, когда украинские вояки увидят падающее на них
железное небо, когда каждая вышиванка представит себя бегущим по улице с трупом
своего ребёнка. Тогда прийдет он - страх, и только после этого где то в уголках
мозга начнёт прорастать здравомыслие. Очень хотелось, что бы это произошло без
войны, очень. Но украинское общество невероятно одурманено. И без болевого шока
здесь видимо уже никак. Вы поймите, я был противником насилия им и остаюсь. Но
когда нет, ни то что бы диалога но даже намёка на него. То всё будет плохо. Для
всех. Но для жителей Украины особенно.

\iusr{Valeria Kotskyh}
\textbf{Энрике Менендес} боюсь, самая реальная перспектива, исходя из нынешних данных, - длительный замороженный конфликт((
Я искренне хочу, чтобы было так, как пишешь ты! Просто слабо в это верю.

\iusr{Rostyslav Pelekhovych}
\textbf{Alexey Ivanov} , 

за вашою логікою ті, кого війна все ж зачепила, повинні підтримувати ідеї
Енріке. Реальність же дуже болюча для вас: вони ненавидять московитів з
колабораціоністами, та підтримують унітарність більше ніж диванні експерти


\iusr{Алексей Иванов}
\textbf{Rostyslav Pelekhovych} 

это просто у них есть тыл. 95\% жителей Украины за пределами Донбасса война
никак не коснулась. От слова вообще. Война на Украине поддерживается лишь
только потому, что это где-то далеко и не серьезно. Первый же взрыв снаряда в
черте города Киева и всё, многие станут сразу миротворцами.

\iusr{Ирина Слюсарева}
\textbf{Rostyslav Pelekhovych} 

почему реальность должна быть \enquote{болюча}. Да плевать на вашу ненависть. Просто
плевать. Ну молотите вы по клаве, что с того. Ноль влияния.

\iusr{Petro Protsyk}
\textbf{Alexey Ivanov} 

Покажіть це вашим матерям. Хай вони знають, як путін готується закопувати їх
синів. І запитайте їх згоду.

\url{https://www.pravda.com.ua/news/2021/12/21/7318112/}

Але, перед цим розкажіть їм правду:

Донбас. Лише за півроку 2014 московини втратили тільки кадрових військових 2
тисячі та добре покалічених 3 тисячі. Їхтепертамнєт

\href{https://web.archive.org/web/20150315011807/http://bs-life.ru/rabota/zarplata/voennosluzashchie2015.html}{%
Повышение зарплаты военнослужащим в 2015 году, bs-life.ru, 2015}

До речі, скоро отримаєте гроби з Казахстану. Потім й про \enquote{здравомисліє}
поговоримо. Хоча навряд чи. Ви - раби

\iusr{Rostyslav Pelekhovych}
\textbf{Irina}, ви неуважні: ненавиджу не я, а люди з бойовим досвідом та зареєстрованою зброєю

\iusr{Rostyslav Pelekhovych}
\textbf{Alexey}, 

кажу ж вам: люди, в чиїх містах падали москальські снаряди, за всіма
опитуваннями ненавидять і московитів і місцевих колабораціоністів більше, ніж
їх мирні сусіди

Ну А ви переконуйте себе далі що справа лише в тому що до Києва не дійшли

\iusr{Алексей Иванов}
\textbf{Rostyslav Pelekhovych} 

я наверно вас очень сильно удивлю но в наших местах падали сильно
преимущественно украинские снаряды. А подконтрольный Киеву Донбасс почему то
голосует за ОПЗЖ. Это как то не вписывается в ваши бредни.

\iusr{Rostyslav Pelekhovych}
> А подконтрольный Киеву Донбасс почему то голосует за ОПЗЖ.
саме тому що до них не дійшов рускій мір, \textbf{Alexey Ivanov}, вони і голосують так як 10 років. я ж кажу про тих до кого дійшов

\iusr{Алексей Иванов}
\textbf{Rostyslav Pelekhovych} до кого дийшов? Отвечай.

\iusr{Rostyslav Pelekhovych}
\textbf{Alexey Ivanov} , скажімо до Олександра Сосницького - можете нижче його коментарі прочитати. І не тикайте

\iusr{Алексей Иванов}
\textbf{Rostyslav Pelekhovych} 

ну жил в Донбассе какой то националист Сосницкий ненавидел Россию и русских до
2014 года, ненавидит и сейчас. Что изменилось. Просто аргументация, до 2014
года ненавидел с одними аргументами, сейчас ненавидит с другими, для вселенной
ничего не поменялось. А отношение широких масс населения показали выборы.


\iusr{Rostyslav Pelekhovych}
\textbf{Alexey Ivanov} , 

відношення широких мас до федералізації, мовного питання і НАТО дійсно
показують вибори. Ви ж сподіваюсь розумієте що підтримка НАТО на Сході теж
збільшиалсь, причому найбільш динамічно зі всіх регіонів?

\iusr{Алексей Иванов}
\textbf{Rostyslav Pelekhovych} 

не буду отрицать. Как надеюсь вы не будете отрицать, что Украине не стоит
вступать в НАТО. Ибо даже за стремление туда, прийдется заплатить. И чем
сильнее вы будете туда стремится, тем больше для вас будет эта цена. Украинцев
жалко, но российскому руководству, свой народ жальче. А значит украинцы начнут
гибнуть в непропорционально больших объемах. И так будет пока не поумнеют.
Лично вас война не коснулась, когда коснётся, вы будете петь совсем другие
песни. Вы даже близко не знаете, что такое война.

\iusr{Rostyslav Pelekhovych}
> Ибо даже за стремление туда, прийдется заплатить.

Можливо, \textbf{Alexey Ivanov} . Але за подібні погрози російський народ, про який
російське керівництво з понтом турбується, ВЖЕ платить. І судчи з динаміки
відношення до війни росіян і українців, запас терпіння в нас значно більший

\iusr{Алексей Иванов}
\textbf{Rostyslav Pelekhovych} вы о чём?

\iusr{Rostyslav Pelekhovych}
\textbf{Alexey Ivanov} , я про відсутність відділень сбербанка в Криму, про санкції проти західних товарів, в тч продуктів харучування, про погіршення економічних показників і тд

\iusr{Alex Radfox}
\textbf{Alexey Ivanov} \enquote{Первый же взрыв снаряда в черте города Киева и всё, многие станут сразу миротворцами.}
Первые выстрелы и взрывы в моём городе заставил меня взять в руки оружие.

\iusr{Алексей Иванов}
\textbf{Rostyslav Pelekhovych} 

и каких западных товаров нет в России? В Крыму работают филиалы всех российских
организаций, только называются слегка по другому. Вместо РЖД-ТЖД. И так далее.
Сейчас под санкциями Россия живет лучше, чем Украина в 2013 году.

\iusr{Алексей Иванов}
\textbf{Alex Radfox} и с кем ты воевал? И в рядах чего?

\iusr{Ирина Слюсарева}
\textbf{Rostyslav Pelekhovych} 

а я не про вас и написала. Именно про людей \enquote{со зброей}. Вы ж уже 8 лет
прыгаете от злости - и что? Ну и обтекайте дальше. Можно прямо в зброе.

\iusr{Rostyslav Pelekhovych}
\textbf{Alexey Ivanov} , тих на які Росія наклала санкції? ) 

\url{https://yle.fi/novosti/3-10351310}

\iusr{Rostyslav Pelekhovych}
> В Крыму работают филиалы всех российских организаций. Вместо РЖД-ТЖД
Питав ніби про Сбєр, до чого тут РЖД? )

\iusr{Rostyslav Pelekhovych}
> Сейчас под санкциями Россия живет лучше, чем Украина в 2013 году.
отже справа не в Майдані? чудово ) але темпи зростання в Україні, попри війну на сході, вищі ніж в Росії

\iusr{Rostyslav Pelekhovych}
\textbf{Irina Sluysareva} , а люди зі зброєю не по клаві б'ють, а дуже навіть крошать московитів та колабораціоністів

\iusr{Ирина Слюсарева}
\textbf{Rostyslav Pelekhovych} правда? Смешной вы, честно. Пароль: Иловайск. Отзыв: Украина понад усе.

\iusr{Ирина Слюсарева}
\textbf{Rostyslav Pelekhovych} 

иными словами, вы признали факт: что ваши грозные воины (даже в зброе)
устрашают не всех и не всегда. И вот когда не, то с вашей стороны начинается
разная недоказуемая брехня, переходящая в нечленораздельные звуки. Ладно,
бывайте. Грустно читать так много реплик, до ужаса похожих на диагнозы.

\end{itemize} % }

\iusr{Андрей Николаев}
Україна, єдина, соборна, неподільна, унітарна країна! Все, крапка!

\begin{itemize} % {
\iusr{Ирина Слюсарева}

Напишет такой деятель \enquote{все, крапка} - и отчего-то воображает, что в реальности
будет так, как хочется именно ему. Крапка, что ли, лучше, чем у других. Но ведь
нет.


\iusr{Остап Петренко}
\textbf{Ирина Слюсарева} , да, в реальности будет именно так, потому что Россию считают врагом 72\% опрошенных!

\iusr{Денис Журавлев}
Опорошенных

\iusr{Денис Зимин}
\textbf{Ирина Слюсарева} а ведь вместо \enquote{крапка}, можно было слова халва раз 20 написать)

\iusr{Игорь Писаренко}
\textbf{Андрей Николаев} від Сяну до Збручу

\iusr{Андрей Николаев}
\textbf{Денис Зимин} 

Щось не влаштовує? Ви живете у вільній країні, тому можете абсолютно вільно
зібрати речі і виїхати на постійне проживання туди, де будете щасливими. Досить
страждати


\iusr{Андрей Николаев}
\textbf{Irina Sluysareva} Саме так і буде

\iusr{Evgeniy Akunin}
\textbf{Андрей Николаев} где-то заржали олигархи с клоуна

\iusr{Игорь Ковальчук}
\textbf{Андрей Николаев} выедут все, обязательно.
Но вместе с землёй.

\iusr{Энрике Менендес}
\textbf{Андрей Николаев} 

Так а где я написал противоположное? Унитарная может и оставаться, на самом
деле. Есть вариант как Испания - де-юре унитарная, а де-факто федерация

\iusr{Денис Зимин}
\textbf{Андрей Николаев} 

вы не поверите, но как раз именно этим \enquote{патриоты} активно и занимаются,
особенно усердно из западных областей, даже соц исследования показывают, что
уехать жить в другую страну страну больше желают именно жители западных, а не
восточных или южных регионов. Так в чем вопрос, -пусть продолжают это делать и
дальше, но тихо, без этой псевдопатриотичной глупости.


\iusr{Ирина Слюсарева}
\textbf{Остап Петренко} 

да начихать на ваших \enquote{опрошенных}. Как до вас не доходят простые вещи: просто
начихать. От их эмоций не зависит ничего. Кроме криков в ФБ.

\iusr{Ирина Слюсарева}
\textbf{Андрей Николаев} 

вы просто удивительные. Честно. А я напишу, что саме так не буде. И что далее?
Ваші репліки повертають мене в солодкі роки дитинства. Приблизно так ми
спілкувалися в дитячому садку.

\iusr{Halyna Pry}
\textbf{Irina Sluysareva} 

удивительно, что вам непонятно до сих пор, что украинцам начихать, что вам
начихать и об этом говорят не эмоции, а факты и вся социология.. Не выехать ли
в Москву, там чихать способнее, по крайней мере в унисон, а не против ветра?  @igg{fbicon.smile} 
Умные люди так и поступают, остальные чихают  @igg{fbicon.frown} 

ПС Она и так в Москве, чихание из Москвы вдвойне интересно  @igg{fbicon.face.grinning.squinting} 

\iusr{Остап Петренко}
\textbf{Ирина Слюсарева} , 

а мне на тебя начхать, россиянка, потому что мои опрошенные будут на выборах
голосовать за про-украинские партии, и это хорошо!

\iusr{Людмила Погорелова}
\textbf{Андрей Николаев} 

яка ж вона вильна???? Вакцинацию делают принудительно, переезд из одного
региона в другой запрещён, предъявлять сертификаты всем, без них будешь платить
за проезд, даже если тебе 89-90 лет... повторение времён Гитлера 33-35 года в
Германии... Вольной она станет, когда вынесут йододефицитных из В. Р.... и это
будет!!!!


\iusr{Андрей Николаев}
\textbf{Людмила Погорелова} Звідки ви цю маячню про пересування між регіонами берете?

\end{itemize} % }

\iusr{Алена Теодорович}

Даты не подходящие. Если не произойдет вхождение в ближайшие год-два, то
вхождение в Россию и это само ЕС одобрит, никому волынка уже не нужна.
И.... Россия с США обо всем уже договорились, но делают вид, что надо еще
перетирать, а на самом деле работают над тем, как красиво для всех преподнести.

\begin{itemize} % {
\iusr{Алена Теодорович}

Кому отчего-то смешно, то мы потом проверим, кто и как будет смеяться.
Николаев и Трембач -лопухи )))))) Бытов туда же. Короче.... лопухи, если не
хотите, чтобы вас имели, не будьте лопухами.

\iusr{Андрей Николаев}
\textbf{Алена Теодорович} Судячи з Ваших каментів, Вас уже досить добре мають, причому у збоченій формі  @igg{fbicon.smile} 

\iusr{Энрике Менендес}
\textbf{Алена Теодорович} ЕС не одобрит точно и совсем не понятно, зачем вхождение РФ? Какие стратегические цели решает?

\iusr{Алена Теодорович}
\textbf{Энрике Менендес} 

Потому, что не все так легко расшвыриваются богатствами)) и интересы есть более
привлекательные, чем десятилетиями уговаривать Киев взять обратно свое же....
Всех уже подташнивает. Дела люди делать хотят и денюжку зарабатывать, а из-за
баранов .которые бумаги подписывают, а потом же не хотят выполнять то, что уже
сколько раз подписали надо все себе тормозить ? ....Кому это надо ? Вы бы
уговаривали припадочного несколько десятилетий да еще себя в ущерб ? А вы не
видите планомерного и последовательного вползания ОРДЛО в РФ ? Вы думаете, что
РФ будет еще несколько десятилетий уговаривать Киев и попутно вливать свои
деньги в ОРДЛО. кормить пенсионеров и прочее. чтобы потом вот так отдать ?.Вы
серьезно ? То, что в Украине закрывают. то в ОРДЛО начинает работать и
увеличивают мощности. Ой..... Думайте, как хотите.. Но ..,или это произойдет
быстрее. или все


\iusr{Алена Теодорович}
\textbf{Андрей Николаев} Ты настолько испорчен, что в голове больше ничего не возникает ? Не впечатлил, убогий.
\end{itemize} % }

\iusr{Святослав Батов}

Внеблоковый статус без военных баз, но ЧФ РФ оставили в Крыму?) Подобный ужас,
описанный вами в посте возможен исключительно если проиграть войну

\begin{itemize} % {
\iusr{Денис Журавлев}
Так вы её уже проиграли. Ваша капитуляция назвается \enquote{Минские соглашения}.

\iusr{Алина Яковлева}
\textbf{Денис Журавлев} а лднр считает, что они проиграют, если рашка таки впихнет их в Украину по минским)

\iusr{Святослав Батов}

\textbf{Denis Zhuravlev} но именно рашу ххххх за их невыполнение)
Поглядим насколько хватит микроскопической (в мировом выражении) экономики рашки, совка хватило на 12 лет с начала Афгана

\iusr{Ирина Слюсарева}
Тот простой факт, что ЧФ РФ уже в Крыму, вас не ужасает? Или вы не в курсе? Странное детское отрицание реальности.

\iusr{Остап Петренко}
\textbf{Денис Журавлев} , а мы их выполнять не будем, как и сейчас!

\iusr{Денис Журавлев}
Точно, говорят, что договор со Славаукраиной не стоит бумаги, на которой он подписан :+))

\iusr{Святослав Батов}

\textbf{Irina Sluysareva} много разных войск в Украине, для меня нет разницы, если выводить все - то все.

\iusr{Остап Петренко}
\textbf{Денис Журавлев} , плевать что там говорят

\iusr{Денис Журавлев}
Остап, плевать что вам плевать :+)

\iusr{Остап Петренко}
\textbf{Денис Журавлев} , плевать на твоё Ольгино и на твою Россию!

\iusr{Денис Журавлев}
Смешно слышать про Ольгино от пустого аккаунта :+)))))

\iusr{Святослав Батов}

\textbf{Denis Zhuravlev} договор в Минске подписан на обратной стороне будапештского меморандума

\iusr{Остап Петренко}
а про \enquote{плевать на твою Россию} - согласен?

\iusr{Денис Журавлев}

Святослав, если вас что-то не устраивает в Будапештском меморандуме, так
отзовите свою подпись :+)) а заодно посмотрите, что согласно меморандуму нужно.
делать в случае его несоблюдения :+)

\iusr{Денис Журавлев}
Остап, площадь России более 17 млн. кв.км. волнуюсь за вашу систему слюноотделения :+)

\iusr{Остап Петренко}
\textbf{Денис Журавлев} , площадь большая, народу не так много. Нужно чтобы Китай забрал у вас лишнее

\iusr{Игорь Ковальчук}
\textbf{Святослав Батов} войну давно проиграли.

\iusr{Святослав Батов}

\textbf{Игорь Ковальчук} 

пока что недоразвитого фашистского \enquote{русского мира} нет на 92\% территории
Украины. Это точно не проигрыш  @igg{fbicon.wink} 

\iusr{Святослав Батов}

\textbf{Denis Zhuravlev} если вас что-то не устраивает в Минских соглашениях, отзовите свою подпись)
А то вы только балаболить на международных встречах можете)

\iusr{Энрике Менендес}
\textbf{Святослав Батов} 

Так Крым это территория с двойным статусом - Вы знаете как решить эту проблему
лучше? Концессию по АРК не я предложил - её предложил кто-то из американских
великих - то ли Киссинджер то ли Бжезинский.

\iusr{Святослав Батов}

\textbf{Энрике Анатольевич Менендес} 

для реализации подобных идей нужны референдумы, для референдума нужно для
начала восстановить работу органов власти на конкретной территории. Это
невозможно при наличии иностранных войск Выведение россиян обратно в их Сызрань
и другие мухосрански из Украины возможно только по причинам по которым это
произошло ранее в Германии, Афгане, Балтике и т.д. Но когда эти причины
настанут, никакая концессия уже обсуждаться не будет. Поэтому смысл обсуждать
то, что нереализуемо.


\iusr{Денис Журавлев}

Остап, ну, конечно же Китай заберёт у России Сибирь! Но случится это только в
том случае, если вы прекратите есть на ночь сырые помидоры и почитатете про
линию Хэйхэ-Тэнчун :+)


\iusr{Денис Журавлев}

Святослав, я думаю Россию как раз-таки всё устраивает в минских соглашениях
кроме того, что Украина их не выполняет :+)

\iusr{Святослав Батов}
\textbf{Denis Zhuravlev} 

как только Россия выполнит хотя бы первый пункт, дальше можно будем обсуждать и
выведение иностранного контингента и возвращение на линию разграничения по
состоянию на дату подписания. А пока вы их не выполняете, на вас будут
продолжать налагать санкции.

Но вам же от них только лучше, больше рублей за доллар, как сказал ботоксный
фюрер)


\iusr{Денис Журавлев}
А можно ссылку на то место в минских соглашениях, где сказано, что Россия должна что-либо выполнять?

\iusr{Денис Журавлев}

Тут у Святослава есть два варианта: сделать вид, что он не понимает вопроса или
слиться на обсуждение другого вопроса. Посмотрим, какой вариант он выберет :+)


\iusr{Rostyslav Pelekhovych}
\textbf{Denis Zhuravlev} , а за що на Росію санкції наклали?
дєцкій сад

\iusr{Денис Журавлев}

Ростислав, санкции наложили не за что, а для чего :+) Может вы прочитаете
Минские соглашения, и сможете назвать конкретный пункт, который должна
исполнить Россия?

\iusr{Rostyslav Pelekhovych}
\textbf{Denis Zhuravlev} , 

можливо ви заходите в ФБ через суверенний інтернет "Буратіно", то ж проінформую
: Франція та Німеччина офіційно вважають Росію стороною мінських домоленостей
(і військового конфлікту, само собою). Саме тому вони і мають юридичні підстави
дял санкцій


\iusr{Денис Журавлев}

Ростислав, вы, видимо не понимаете разницу между \enquote{за что} и \enquote{для чего}, а так
же не можете дать цитату из минских соглашений, по которой Россия должна там
что-то выполнить :+))) но вам славаукраинцам это простительно, у вас с
реальностью особые отношения :+)))))))

\iusr{Ирина Слюсарева}
\textbf{Святослав Батов} 

с вами уже консультируются относительно вывода войск из Крыма? А, пока еще нет.
Тогда о чем вы? Меня поражает в людях, подобных вам, поразительная
мечтательность и неспособность смотреть в глаза фактам. А зачем вы пишете про
\enquote{Сызрань и другие мухосрански} - это считается остроумием? Вы заметили, что
оппоненты не шутят насчет ваших Чмировок, Цицор и Крыжополей? Знаете, почему?
Потому что такие шутки - ....мягко говоря, есть несомненный признак большого
ума и хороших манер.

\iusr{Денис Журавлев}

Ирина, Вы отвлекаете Святослава от чтения Минских соглашений. Он уже почти
совсем нашёл обязательства России в данном документе. Осталось совсем
чуть-чуть.

\iusr{Petro Protsyk}
\textbf{Denis Zhuravlev} 

це пан певно має на увазі великий договір \enquote{про дружбу} з московією, яким путін
підтерся Напавши на Україну? Чи може Заключні Ґ ельсинські документи про
безпеку й співробітництво в Європі? Й ще десятки міжнародних договорів....

\iusr{Денис Журавлев}
Петро, я ни слова не сказал про договор про дружбу :+) но вы же его не расторгли, или я ошибаюсь?

\iusr{Petro Protsyk}
\textbf{Denis Zhuravlev} 

не до мене запитання та й відповідь вам, судячи з настрою, не цікава. Але все ж

10. Вывод всех иностранных вооруженных формирований, военной техники, а также
наемников с территории Украины под наблюдением ОБСЕ. Разоружение всех
незаконных групп.

Як гадаєте, чиї іноземні формування побачила на території України ОБСЄ,
підписався посол Зурабов й зобов'язався вивести путін


\iusr{Petro Protsyk}
\textbf{Denis Zhuravlev} тобто путін договір не порушив?

\iusr{Petro Protsyk}
\textbf{Denis Zhuravlev} ЗАКОН УКРАЇНИ

Про припинення дії Договору про дружбу, співробітництво і партнерство між Україною і Російською Федерацією

(Відомості Верховної Ради (ВВР), 2019, № 1, ст.2)

Зважаючи на збройну агресію Російської Федерації проти України, що становить
істотне порушення Російською Федерацією Договору про дружбу, співробітництво і
партнерство між Україною і Російською Федерацією, учиненого 31 травня 1997 року
в м. Києві та ратифікованого Верховною Радою України 14 січня 1998 року, яке
має істотне значення для здійснення об’єкта і цілей цього Договору, виходячи з
положень статті 60 Віденської конвенції про право міжнародних договорів, статті
24 Закону України "Про міжнародні договори України" та

відповідно до статті 40 Договору про дружбу, співробітництво і партнерство між
Україною і Російською Федерацією

Верховна Рада України постановляє:

1. Припинити з 1 квітня 2019 року дію Договору про дружбу, співробітництво і
партнерство між Україною і Російською Федерацією, учиненого 31 травня 1997 року
в м. Києві та ратифікованого Верховною Радою України 14 січня 1998 року
(Відомості Верховної Ради України, 1998 р., № 20, ст. 103).

2. Припинення дії Договору про дружбу, співробітництво і партнерство між
Україною і Російською Федерацією звільняє Україну від будь-якого зобов’язання
щодо його виконання і не впливає на права, зобов’язання чи правове становище
України, що виникли в результаті виконання зазначеного Договору до припинення
його дії, відповідно до статті 70 Віденської конвенції про право міжнародних
договорів.

3. Цей Закон набирає чинності з дня, наступного за днем його опублікування.

Президент України

П.ПОРОШЕНКО
м. Київ
6 грудня 2018 року
№ 2643-VIII

\iusr{Денис Журавлев}
Петро по поводу наёмников:

\href{https://www.dw.com/ru/v-ukraine-vozbudili-dela-protiv-sotni-naemnikov-v-donbasse/a-54491064}{%
В Украине возбудили дела против сотни иностранных наемников в Донбассе, dw.com%
}

\iusr{Денис Журавлев}

И во ещё:

\href{https://zn.ua/POLITICS/rada-khochet-uprostit-poluchenie-ukrainskoho-hrazhdanstva-dlja-inostrantsev-voevavshikh-na-donbasse-v-rjadakh-vsu.html}{%
Рада хочет упростить получение украинского гражданства иностранцами, воевавшими на Донбассе в рядах ВСУ, zn.ua, 02.11.2021%
}

\iusr{Денис Журавлев}

Ссылаться на \enquote{вооружённую агрессию России} при том, что миссия ОБСЕ не
подтверждает участия Российской армии в гражданской войне на Донбассе, это
сильно :+)


\iusr{Rostyslav Pelekhovych}
> ОБСЕ не подтверждает участия Российской армии
Denis , зате підтверджує російський суд

\iusr{Святослав Батов}

\textbf{Denis Zhuravlev} 

а вы считаете что я должен посвятить глупым ватникам, которые считают что
Россия не подписала минские соглашения (но ее за их невыполнение наказывают) и
бросить гораздо более интересные дела?)

\iusr{Святослав Батов}
\textbf{Denis Zhuravlev} 

миссия обсе которая постоянно фиксирует новое российское вооружение на Донбассе
должна подтвердить что?)

Разжевать жертвам киселя головного мозга что если россияне за российские деньги
экипированные Россией получая приказы от россиян стреляет из российской техники
заправленной российским топливом российскими снарядами и патронами по
украинцам, то это гражданская война?)

А участие РОА во второй мировой войне делало Великую отечественную гражданской
тоже?)))  @igg{fbicon.face.tears.of.joy}{repeat=3} 

Какой неподкованный нынче ватник пошел, вам бюджеты урезали и по объявлению у
метро набирать начали?)


\iusr{Святослав Батов}
\textbf{Петро Процик} совсем слабые пошли у рашки пропагандисты)

\iusr{Petro Protsyk}
\textbf{Denis Zhuravlev}

\enquote{Материалы Гаагского трибунала} - События в Крыму и на востоке Украины
начиная с 20 февраля 2014 года.

Крым.

\enquote{ В 2016 году Канцелярия Прокурора опубликовала свою оценку ситуации ...,
которую она квалифицировала как международный вооруженный конфликт
(международный синоним термина \enquote{война}) между Украиной и Российской Федерацией,
возникший самое позднее 26 февраля 2014 года. (или ранее). Согласно оценке ...
право международных вооруженных конфликтов применимо и после 18 марта 2014 года
в той мере, в какой ... будет фактически сохраняться состояние оккупации.}

Восточная Украина.

\enquote{Канцелярия Прокурора также привела дополнительные сведения, указывающие на
факт существования прямого военного противостояния между вооруженными силами
Российской Федерации и Украины, из чего вытекает, что самое позднее с 14 июля
2014 года ... на востоке Украины имел место международный вооруженный конфликт
(война). ...
Факт существования единого международного вооруженного конфликта на востоке
Украины предполагает применение в отношении соответствующего периода статей
Статута, касающихся вооруженных конфликтов международного характера. }

\enquote{Учитывая то, что Украина признала юрисдикцию международного суда на
неограниченный срок, Канцелярия Прокурора также продолжит регистрацию обвинений
в преступлениях, совершенных в Украине, в той мере, в какой они подпадают под
юрисдикцию Суда.}

\url{https://www.icc-cpi.int/itemsDocuments/2017-PE-rep/2017-otp-rep-PE-UKRAINE_RUS.pdf}
\href{https://informnapalm.org/en/russian-presidents-press-secretary-lied-again-about-the-war-against-ukraine}{%
Russian President’s Press Secretary lied again about the war against Ukraine, %
informnapalm.org%
}

\iusr{Petro Protsyk}
\textbf{Denis Zhuravlev} 

не спекулюйте - в п.10 Комплексу заходів....йдеться про іноземні воєнні
формування. Розумієте, що таке \enquote{іноземне воєнне форування}? А іноземці, що
воюють в наших ЗС, роблять це на законних підставах. Вони - на державній
службі. А незаконні формування це утворені не державою, як от в ордло

\iusr{Alex Radfox}
\textbf{Denis Zhuravlev} 

Для начала, хотя бы эти пункты Комплекса мер по выполнению Минских соглашений.

п. 2. Отвод всех тяжёлых вооружений обеими сторонами на равные расстояния в
целях создания зоны безопасности шириной минимум 50 км друг от друга для
артиллерийских систем калибром 100 мм и более, зоны безопасности шириной 70 км
для РСЗО и шириной 140 км для  @igg{fbicon.index.pointing.right} РСЗО
«Торнадо-С», @igg{fbicon.index.pointing.left}  «Ураган», «Смерч» и тактических
ракетных систем «Точка» («Точка У»):

для украинских войск — от фактической линии соприкосновения;

для вооружённых формирований «отдельных районов Донецкой и Луганской областей
Украины (ОРДЛО)» —  @igg{fbicon.index.pointing.right} от линии соприкосновения
согласно Минскому меморандуму от 19 сентября 2014 года.
@igg{fbicon.index.pointing.left} 

п. 10. Вывод всех иностранных вооружённых формирований, военной техники, а
также наёмников с территории Украины под наблюдением ОБСЕ. Разоружение всех
НЕЗАКОННЫХ групп.

\iusr{Petro Protsyk}
\textbf{Sviatoslav Batov} 

А мене навпаки непокоїть їх активність. Останнім часом спостерігаю на сторінках
лідерів суспільної думки ніби чергування по декілька московинських
пропагандистів, як ось тут, що з подачі автора сторінки чи з використанням
якоїсь його думки, часто вирваної з контексту, формують декілька груп, де
проштовхують московські наративи. Часто без опонування, тоді влаштовують
псевдодискусію між собою. Ну й підтримку смайликами. А утворені для протидії
ІПсО державні структури не показують своєї ефективності

\href{https://spravdi.gov.ua/dezinformacziya-u-2021-najbezgluzdishi-momenty}{%
Дезінформація у 2021: найбезглуздіші моменти, spravdi.gov.ua, 11.01.2022%
}

\iusr{Денис Журавлев}
Как и ожидалось, Святослав просто слился, не сумев найти в тексте минских соглашений обязанности России :+))

\iusr{Денис Журавлев}
\textbf{Alex Radfox}, 

и что? Там что-то сказано про обязательства России? :+))) бедные славаукраинцы
за 8 лет \enquote{войны с Россией} так и не научились отличать российское оружие от
Российской армии. Даст Бог, у них не будет возможности ощутить эту разницу.


\iusr{Денис Журавлев}
Петро, Славаукраинские реалии и словосочетание \enquote{на законных основаниях} это самое нелепое сочетание из всех возможных :+)

\iusr{Rostyslav Pelekhovych}
> Там что-то сказано про обязательства России?
те, \textbf{Denis Zhuravlev} , що від неї, як від сторони конфлікту, вимагають інші учасники нормандської четвірки - Франція та Німеччина

\iusr{Денис Журавлев}

Ростислав, того же, чего хотел добиться друг моего детства Коля Остен-Бакен от
подруги моего же детства, польской красавицы Инги Зайонц - любви.


\iusr{Alex Radfox}
\textbf{Denis Zhuravlev} 

Не могу поверить, что у вас проблемы с пониманием текста.

1. У кого на территории Луганской и Донецкой областей, кроме российской армии,
есть на вооружении РСЗО «Торнадо-С»?

2. Кто, кроме правителей России, может принудить командиров \enquote{вооружённых
формирований «отдельных районов Донецкой и Луганской областей Украины (ОРДЛО)»}
БЕЗ БОЯ оставить Дебальцево и Углегорск с окрестными сёлами и \enquote{отойти от линии
соприкосновения согласно Минскому меморандуму от 19 сентября 2014 года}?

3. \enquote{Разоружение всех НЕЗАКОННЫХ групп.} - Считаете, это должны сделать ВСУ и
\enquote{Азов} с \enquote{Айдаром}? Ну, тоже неплохой вариант. Мы так и сделаем.

\iusr{Денис Журавлев}

Не могу поверить, что вы не видите разницы между \enquote{пониманием} и
\enquote{содержанием} :+) всё ваши вопросы это всего лишь трактовка этих
соглашений, при этом во втором пункте чётко обозначены стороны конфликта:

1) украинские войска

2) вооружённые формирования «отдельных районов Донецкой и Луганской областей
Украины (ОРДЛО)». Т.е. с обеих сторон формирования Украины, следовательно и
война гражданская.

Всё остальное это ваши домыслы и спекуляции :+)) что касается
\enquote{незаконных групп}, то народная ЛДНР уже давно придали законный статус
народной милиции расформировав незаконные формирования, так что тут всё ок.

\iusr{Денис Журавлев}

Самый большой обмен пленными между Славаукраиной и ЛДНР по формуле \enquote{306 на 74}
- президент Порошенко заявляет: \enquote{мы сегодня не отдали ни одного россиянина}, и
добавляет, что пленных россиян у него аж 15! Т.е. обмениваются сотнями граждан
Украины, а в плену 15 Россиян. Обычная гражданская война..

\url{https://youtu.be/Yn30Ltn_N0A}

\iusr{Людмила Погорелова}
\textbf{Petro Protsyk} 

не отсохла ещё \enquote{рука и нога кремля}? Достали уже со своими \enquote{наративами}! Чуть
что рука, нога, кремль, агрессор.... До Пети жили-дружили, летали-ездили друг к
другу... нет, надо было прогнуться под западных \enquote{друзей}....допрыгались??? Страну
всю рвут друзья, экономика ниже плинтуса, молодежь бежит кто-куда.... от этой
власти и беззакония!!!

\end{itemize} % }

\iusr{Юрий Трембач}

Энрике, у меня в последнее время ощущение, что вы где то нашли папку с рабочими
доками партии регионов двадцатилетней давности)) прям что не пост, то флешбек)
По сути - спасибо, но нет.

\begin{itemize} % {
\iusr{Людмила Погорелова}
\textbf{Юрий Трембач} спасибо и от меня. Удивляет Юрий, почему \enquote{нет}? Вам плохо жилось при Азарове???
Прекрасные мечты у Анрике!!! Наверное навеяны ФОРМУЛОЙ СТРАНЫ Мураева?
Скоро он нам покажет свой новый труд по Конституции!
Надеюсь, ваши с его работой, пазлы совпадут.
\end{itemize} % }

\end{itemize} % }
