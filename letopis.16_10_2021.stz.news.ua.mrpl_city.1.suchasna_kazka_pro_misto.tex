% vim: keymap=russian-jcukenwin
%%beginhead 
 
%%file 16_10_2021.stz.news.ua.mrpl_city.1.suchasna_kazka_pro_misto
%%parent 16_10_2021
 
%%url https://mrpl.city/blogs/view/suchasna-kazka-pro-misto
 
%%author_id demidko_olga.mariupol,news.ua.mrpl_city
%%date 
 
%%tags 
%%title Сучасна казка про місто
 
%%endhead 
 
\subsection{Сучасна казка про місто}
\label{sec:16_10_2021.stz.news.ua.mrpl_city.1.suchasna_kazka_pro_misto}
 
\Purl{https://mrpl.city/blogs/view/suchasna-kazka-pro-misto}
\ifcmt
 author_begin
   author_id demidko_olga.mariupol,news.ua.mrpl_city
 author_end
\fi

\ii{16_10_2021.stz.news.ua.mrpl_city.1.suchasna_kazka_pro_misto.pic.1}

У нашому місті продовжується \emph{Марафон міжнародних резиденцій – 2021}, завдяки
якому створюються нові театральні проєкти. У третій резиденції бере участь
режисерка з Німеччини \textbf{Крістін Діссманн}, яка готує виставу-інсталяцію \enquote{Між
часами} що поєднає театр, музику та візуальне мистецтво з експериментальним
мистецьким досвідом у новому вимірі і буде присвячена Маріуполю.

\ii{16_10_2021.stz.news.ua.mrpl_city.1.suchasna_kazka_pro_misto.pic.2}

У мультидисциплінарному арт-проєкті німецька режисерка разом із юними акторами
\emph{народного театру \enquote{Театроманія}, першої театральної школи-студії та художниками
з Маріупольської філії Національної академії образотворчого мистецтва і
архітектури} створюють нову розповідь про Маріуполь, балансуючи між фактами та
уявою, відображаючи те, що є і що може бути.

\ii{16_10_2021.stz.news.ua.mrpl_city.1.suchasna_kazka_pro_misto.pic.3}

Координаторка резиденції \textbf{Вікторія Федорів} розповіла, що повна назва вистави –
сучасна казка про місто \enquote{Між часами}, адже основний сюжет всього дійства,
присвячений Маріуполю, тому, яким він є та яким його бачать діти. 

\begin{quote}
\em\enquote{Якщо коротко, то це можна сказати, що це Маріуполь очима дітей,
тобто вони так само, як і в попередній резиденції писали власні тексти,
але тут уже тільки їхні тексти були взяті за основу, їхні враження про
Маріуполь, їхні улюблені місця, їхні якісь побажання місту, те, як вони
його уявляють, як вони хочуть, щоб воно ставало кращим, розвивалося. І
на ці тексти під час перфомансу художники малюватимуть картини, тобто
представлятимуть мрії дітей}.
\end{quote}

Режисер народного театру \enquote{Театроманія} \textbf{Антон Тельбізов} наголосив, що всі
спектаклі із зарубіжними режисерами відрізняються за формою, змістом і
тематикою. На думку Тельбізова, у цій виставі-інсталіції використаний дуже
цікавий прийом – спільно з художниками буде створена цифрова локація, що
допоможе прибрати громіздкі декорації і наповнить простір новими ідеями.
Водночас режисер \enquote{Театроманії} підкреслив, що \emph{Крістін Діссман дуже тонко
відчуває кожного актора, вона їх відкриває по-новому, піднімає різні важливі
теми і допомагає кожному знайти власний напрям.}

\ii{16_10_2021.stz.news.ua.mrpl_city.1.suchasna_kazka_pro_misto.pic.4}

Німецька мисткиня \textbf{Крістін Діссманн} є художницею, авторкою численних п'єс,
майстер-класів, навчальних проєктів та режисеркою театрального
науково-дослідного проєкту \enquote{11 питань до України}. Крістін працює над виставою
вже три тижні, дуже скучила за рідними, проте ця резиденція для неї – дуже
цікавий і незвичний досвід. Режисерка зазначила, що \emph{процес є доволі
інтенсивним, все відбувається дуже швидко, багато нових корисних знайомств.}

Протягом резиденції учасники працювали над трансформацією малої сцени
\enquote{Гатроному} Центру сучасного мистецтва \enquote{Готель Континенталь} на
суцільний мультимедійний простір-інсталяцію. Ця резиденція покликана відкрити
імена молодих художників та розкрити їхні можливості у напрямі створення
театральної сценографії. Учасники спільно створюватимуть єдиний перформативний
акт.  Студенти Маріупольської філії НАОМА отримують тему, за якою вони повинні
зобразити певні карикатури, що допоможуть більше розкрити зміст розповідей
акторів. Малюють на графічних планшетах в диджитал-форматі, вс зображення
одразу транслюватимуться на стіни \enquote{Гастроному}. Загалом молоді
художники у захваті від проєкту. Зокрема, студентка Маріупольської філії НАОМА
\textbf{Анастасія Тимченко} зазначила, що для неї \emph{це дуже цікавий та унікальний досвід}.

Всі задіяні актори розповідатимуть власні історії про рідне місто, тому цю
виставу можна вважати документальною.

Потрапити на прем'єру \emph{вистави-інсталяції \enquote{Сучасна казка про місто
\enquote{Між часами}}} можна безкоштовно за попередньою реєстрацією на сторінці
у Facebook Маріуполь.  Велика культурна столиця України. Марафон міжнародних
резиденцій 2021 реалізується в рамках проєкту культурно-мистецьких ініціатив
\enquote{Діалог мовою мистецтва} за підтримки Українського культурного фонду.
