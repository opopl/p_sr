% vim: keymap=russian-jcukenwin
%%beginhead 
 
%%file 16_11_2022.fb.ohmatdet.bolnica.1.mark_6_rokiv
%%parent 16_11_2022
 
%%url https://www.facebook.com/ndslohmatdyt/posts/pfbid0Eb25EbVar6grF9yiTy2DvBKHHXmT5dtpmgSfmdFGLZiJvpKFfnFx5fcigUShU2R5l
 
%%author_id ohmatdet.bolnica
%%date 
 
%%tags 
%%title Лікування довжиною в 6 років: історія маленького Марка, який переніс 14 операцій
 
%%endhead 
 
\subsection{Лікування довжиною в 6 років: історія маленького Марка, який переніс 14 операцій}
\label{sec:16_11_2022.fb.ohmatdet.bolnica.1.mark_6_rokiv}
 
\Purl{https://www.facebook.com/ndslohmatdyt/posts/pfbid0Eb25EbVar6grF9yiTy2DvBKHHXmT5dtpmgSfmdFGLZiJvpKFfnFx5fcigUShU2R5l}
\ifcmt
 author_begin
   author_id ohmatdet.bolnica
 author_end
\fi

Лікування довжиною в 6 років: історія маленького Марка, який переніс 14
операцій, останню з яких — за день до початку війни💔

Половину свого життя Марк прожив в Охматдиті. Хлопчик народився з важкою
вродженою вадою — тотальною формою хвороби Гішпрунга. Його кишківник фактично
не функціонував, була вроджена кишкова непрохідність. У перший місяць життя
хлопчик пережив дві операції за місцем проживання, а в півторамісячному віці
потрапив в Охматдит у тяжкому стані.⚡️

У НДСЛ «Охматдит» протягом 6 років мультидисциплінарна команда боролася
спочатку за життя, а потім за здоров’я Марка. Дитина перебувала у реанімації,
відділенні новонароджених, відділеннях гнійної, абдомінальної та ургентної
хірургії, хлопчика постійно консультували діагности, МРТ- та УЗД-спеціалісти,
лабораторна служба лікарні.💪🏻

«Були часи, коли Марка вдвічі на тиждень забирали в операційну, він був у
жахливому стані, але лікарі продовжували боротьбу за його здоров’я та
повноцінне життя»,— згадує мама Наталя. Хлопчику було виведено тонкокишкову
стому (ілеостома), бо самостійно ходити в туалет він не міг. Майже на три роки
Марк з мамою повернулися додому, навчилися жити зі стомою: дитина займалася
танцями, ходила в садок та мала щасливе дитинство. Але мама вірила, закрити
стому вдасться.🙏🏻

Рік тому родина приїхала в Охматдит на плановий огляд. Лікарі вирішили
спробувати закрити стому, та підвищити якість життя Марка. Першим етапом
спеціалісти відділення ургентної хірургії провели видалення повністю
патологічно зміненої товстої кишки (тотальна форма хвороби Гіршпрунга) з
формуванням резервуару під захистом ілеостоми. ❤️🩹

На останню, чотирнадцяту операцію родина прибула наприкінці лютого 2022 року.
«23 лютого ми закрили ілеостому, відновивши пасаж по шлунково-кишковму тракту,
що дало змогу Марку повернутися до звичайного життя, природньо спорожнятися та
більше не носити мішечок (калоприймач) на животику»,— розповідає хірург
відділення ургентної хірургії Роман Жежера. Хірургічне втручання провели 23
лютого, а менше ніж за добу почалась повномасштабна війна.💔

Мама Марка пригадує, як змушені були з сином ховатися у підвалі, коли поруч з
Охматдитом лунали вибухи. «Ми спускалися до підвалу по 8 разів на день, зрештою
майже постійно знаходилися там. Було дуже страшно, що твоя дитина змушена
перебувати у підвалі після операції, а не в облаштованій палаті. Пам’ятаю, як
медсестри лікарні на колінах ставили крапельниці малому»,— ділиться Наталя.😭

Попри тяжке лікування, складну ваду розвитку та навіть війну, лікарям вдалося
повернути дитину до звичайного життя — без стоми. Маленький Марк став
найпалкішим шанувальником патріотичних концертів в Охматдиті. Тепер він і сам
мріє стати відомим артистом.✨

\ii{16_11_2022.fb.ohmatdet.bolnica.1.mark_6_rokiv.orig}
\ii{16_11_2022.fb.ohmatdet.bolnica.1.mark_6_rokiv.cmtx}
