% vim: keymap=russian-jcukenwin
%%beginhead 
 
%%file 26_10_2021.fb.fb_group.ukrainci_rozmov_ukrainskoju.1.moskvyny.cmt
%%parent 26_10_2021.fb.fb_group.ukrainci_rozmov_ukrainskoju.1.moskvyny
 
%%url 
 
%%author_id 
%%date 
 
%%tags 
%%title 
 
%%endhead 
\subsubsection{Коментарі}
\label{sec:26_10_2021.fb.fb_group.ukrainci_rozmov_ukrainskoju.1.moskvyny.cmt}

\begin{itemize} % {
\iusr{Olena Vermizova}

З мого досвіду, українці багато в чому поступаються росіянам. Росіяни бережуть
свою культуру, мову, поширюють її і за кордоном в діаспорах. Українці... ті, що
живуть в Україні, в більшості своїй розмовляють російською мовою. Споживають
російський культурний продукт і не абачають в тому нісого страшного. Ті, що за
кордоном, теж якось так... в польщі коли мешкала, багато українців змалечку з
дітьми розмовляли польською, бо "їм тут жити", ну і російською. Скільки
українців у Польщі зараз, і як часто я чула російську навкруги, і лічені рази -
хоча б суржик. В США як справи в нашому містечку... українці як правило не
згуртовані. Приєднуються до російськомовних ком'юніті (спільнот). А росіяни?
Вчіться у них, людоньки. Вони тут створили дитячі садочки двомовні (англійська,
російська), де діти вчать шанувати дідів:))). Є у них тут секції російськомовні
(математика, гімнастика, творчість). В бібліотеках влаштовують періодично
зустрічі, де діточкам читають книжки російською мовою. Організовують свята
разом, зібрання. І туди ж ходять українці, так. Бо то ж просто свято, всі поза
політикою. @igg{fbicon.frown}  Російський народ - близький і братній, як вони
це виправдовують. І російські садочки українці відвідують. Оці діти російські
виростуть, відчуваючи себе росіянами, зі своєю мовою і традиціями. Ну, і
малороси при них виростуть, українського походження... :-((

Але це не лише росіяни так свого тримаються. Аналогічно збираються, гуртуються
і підтримують свою культуру тут і іранці, і індуси, і китайці, японці і так
далі. Це викликає безмірну повагу.

\iusr{Дмитро Руденко}

Московити - це кури, які пишаються вдаваною величчю своєї птахофабрики.
Європейським народам це геть не притаманно.

\iusr{Михайло Шахтьор Жадан}

А ще масковити говорять "труба" - коли йдеться про недолю, про становище коли
всі справи кепські.

\begin{itemize} % {
\iusr{Михайло Шахтьор Жадан}
"Багат і магуч русскій язик" - цей вислів був написаний на стенді у моєму шкільному класі. В Україні, звісно ж...
\end{itemize} % }

\iusr{Oleksander Pavlenko}

Треба усвідомити, що більшість громадян України не є українцями, а є росіянами.
Можливо не більшість етнічних росіян, але росіян по духу. А хто це такі? Це
індивіди, які можуть мати різне походження, у т.ч. і українське, але які
розмовляють па русскі, закохані у все російське: шансон, попсу, тощо, путєна,
лєніна, Сталіна, Гагаріна, ненавидять Україну і українців, їх мову та культуру.

\begin{itemize} % {
\iusr{Ivanna Ignatenko}
\textbf{Oleksander Pavlenko} 

Не давно на лекціях в університеті обговорювали питання громадянства. І головна
наша проблема це автоматичне громадянство станом на 1991 рік. Треба було ще
тоді проявити позицію, але коли біля керма все одно комуняки, нічого з того б
не було. Деяким громадянам за 24 роки самостійності не розвиднилось в очах.

\end{itemize} % }

\iusr{Василь Давимока}

Упрощенний язик пролетарского общения


\end{itemize} % }
