% vim: keymap=russian-jcukenwin
%%beginhead 
 
%%file 05_04_2021.fb.lenkivecka_ljudmila.1.literatura
%%parent 05_04_2021
 
%%url https://www.facebook.com/groups/597204557550989/permalink/813784362559673/
 
%%author 
%%author_id 
%%author_url 
 
%%tags 
%%title 
 
%%endhead 

\subsection{Розповсюджені помилки в кінцівках літературних творів}
\Purl{https://www.facebook.com/groups/597204557550989/permalink/813784362559673/}


\ifcmt
  pic https://scontent-ams4-1.xx.fbcdn.net/v/t1.6435-9/169576711_861813624402240_2389643572635517394_n.jpg?_nc_cat=110&ccb=1-3&_nc_sid=825194&_nc_ohc=EPYoJPWZHo0AX_iDZ0u&_nc_ht=scontent-ams4-1.xx&oh=9fc81d19590ba1dac58daaebdc9034a8&oe=608FDE71
  width 0.4
\fi


1. Обман читача.
Зараз модно круто змінювати долю героя - наприклад, угробити в кінці всіх, хоча ніщо не передбачало такої розв'язки. Сміливо, несподівано, читач ахає і ... розчаровується. Розчарування замінює почуття задоволення від прочитаного, читач шкодує, що взагалі відкрив цю книгу. Обман читача відбувається і з інших причин, наприклад, коли кінцівка не відповідає заявленому жанру.
2. Заплутування читача.
Читач з цікавістю стежив за долею героя, а в кінці не може зрозуміти - вирішені завдання чи ні? Наприклад, героїня довго завойовувала серце обранця і, здається, завоювала, але роман закінчується багатозначною сценою: він і вона зустрічаються і мовчать під якусь музику. Читач вигукує: «Ну ?! Що ?!», - перегортає сторінку, а там порожньо.
3. Нахабна незавершеність.
Герой вирішує проблеми, але одразу вплутується в нові. Далі буде. Однак читач не знав на самому початку, що даний роман всього лише перша книга трилогії. Поширений зараз прийом, коли письменнику або видавництву здається, що якщо повідомити про продовження в кінці, то читач неодмінно почне бігати по крамницях в пошуках другого тому. Не почне - для цього книга має бути мегачудовою. Цей прийом можна використовувати (за умови «мегачудовості»), коли продовження вже написане, і знайти його можна одразу, а не чекати рік або два. Те, що діє для «Гри престолів», далеко не завжди діє для книг.
4. Кінцівка поза логікою.
Герой вирішує завдання настільки нестандартно, що читач взагалі перестає розуміти, про що йде мова.
Абсурдний приклад: чарівний герой всю дорогу шукає свою кохану, знаходить, вбиває її і радіє - проблеми більше немає, а він сам, виявляється, маніяк в десятому поколінні.
5. Банальність (знецінення сюжету) кінцівки.
Прикладів - море: «мені все наснилося», «сталося диво», «герой хворів, а тепер одужав», «вийшов з коми (летаргії), в якій все і привиділося», «втрутилися невідомі сили (бог, диявол, друзі, гаманець з грошима)». Сюди ж віднесемо і хеппі-енди без серйозних підстав, і прийом «скоріше всіх вб'ю, бо набридло писати».
6. Кінцівка типу «ми новий світ побудуємо» або «поцілунок на руїнах».
Поширений прийом у фільмах і книгах: парочка сидить (стоїть) на каменях зруйнованого міста, на купі обвуглених трупів, в пустелі, що була раніше планетою Земля, і радіє життю. Виникає проблема достовірності. Читач просто не вірить в «щасливий» кінець.
7. Кінцівка «каша».
Письменник вирішує, що пора б і зав'язувати. В результаті тридцять три лінії роману завершуються в одному розділі. Епізоди нагромаджуються один на інший, часто взагалі не пов'язані між собою. За обсягом зав'язка  виходить куди більшою, ніж розв'язка.
8. Кінцівка «не можу зупинитися».
Всі завдання вирішені, але письменник ніяк не може розлучитися зі своїми героями, вигадує їм нові, не пов'язані з попереднім сюжетом пригоди, пише епілог, а потім ще післямова, словничок, сиквели, приквели та інше. І не окремо - всю цю красу закидає в кінцівку.
9. Кінцівка «встигнути все».
Так, проблеми вирішуються, але письменник раптом виявив, що не реалізував до кінця свій потенціал. Він починає вставляти недоречний гумор, анекдоти, аргументи і факти, гіпотези і концепції, передавати свої знання про все на світі, часто похапцем вкладаючи все це в уста героїв, створює безліч діалогів, які нічого не додають до сюжету.
10. Кінцівка «Мораль цієї байки така ...» Стиль твору стає пафосним, а кожне рішення (розв'язка) супроводжується мораллю для читача.
***
Зрозуміти помилки просто. Чому ж письменники їх допускають? Згадувати правила хорошої кінцівки потрібно не тоді, коли вона вже написана, а ще під час планування твору. Одне з важливих правил - не починати книгу, не знаючи, що ж буде в кінці. Якраз при підході «навпаки» письменник або кидає почате, або пише кінцівку «аби як».
Джерело:
https://ficwriter.info/.../7161-kontsovka-proizvedeniya.html
#література #письменництво #поради_авторам #фінал_твору #літературна_кухня
