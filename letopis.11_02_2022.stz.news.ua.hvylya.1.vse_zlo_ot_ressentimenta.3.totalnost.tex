% vim: keymap=russian-jcukenwin
%%beginhead 
 
%%file 11_02_2022.stz.news.ua.hvylya.1.vse_zlo_ot_ressentimenta.3.totalnost
%%parent 11_02_2022.stz.news.ua.hvylya.1.vse_zlo_ot_ressentimenta
 
%%url 
 
%%author_id 
%%date 
 
%%tags 
%%title 
 
%%endhead 

\subsubsection{Тотальность Ресентимента}

Рессентимент не рефлексируется. Ресентимент блокирует мышление. Ресентимент это
своеобразная «черная дыра», которая не просто поглощает человека или группу, но
и не позволяет вырваться из нее, ибо попытка кого-нибудь, например бывших
друзей или просто доброжелателей, вызволить ресентиментала из его Ресентимента
воспринимается как проявление враждебности и ведет к еще большей злости.

В одной из авторских работ Ресентимент был обобщен до мыслительных установок и
даже ориентаций:

«Ресентимент – это, прежде всего, структура редукционных мыслительных
установок: отказ от предельных притязаний и общего масштаба; отказ от
перспективы, развития и ориентации на будущее во имя, выживания и ориентации на
прошлое; продвижение простых ответов на сложные вопросы; отказ от разнообразия
во имя однообразия; доминирование одной модели мироустройства, одной доктрины,
одной идеологии; преимущество актуальности над важностью; преимущество частного
интереса и выгоды над общим благом; переход к игре с нулевой суммой; отказ или
запрет на творчество; переход от создания к перераспределению; появление и
распространение фетишей, акцентуаций и гиперкомпенсаций; отложенное,
делегированное и заимствованное мышление вместо немедленного, непосредственного
и оригинального; отказ от сложных форм мышления во имя простейших; утрата и
отказ от ненормативного мышления во имя нормативного; запрет ненормативного
мышления.

Кроме того, ресентимент — это также патическая редукция в безразличие, уныние,
сомнение, вялость, разочарование, уныние по отношению к сложному и в
демонстративный оптимизм и безосновательную эйфорию по отношению к простому и
только потом — в оскорбление, подозрение, зависть, месть и т.д. . Ресентимент
рождает ресентиментальную идентичность и ресентиментальный язык.
Мыслительно-патический ресентимент является основанием морального ресентимента,
а не наоборот. Преодоление мыслительно-патического и нравственного ресентимента
является сложным социальным действием рефлексивно-эмоционального переустройства
общества, занимающего целый исторический период…»

Западные гуманитарии, сделав важным эмоциональный интеллект, мало внимания
уделили искривленным, извращенным, ядовитым формам эмоционального интеллекта,
имеющего группистский характер. Ресентимент как распределенный эмоциональный
интеллект своей группы таких же ресентименталов закрывает доступ к состояниям
эмоционального интеллекта других или всего общества: радость, надежда, доверие,
восхищение. Когда другим больно, им хорошо и наоборот.

Как считал Шелер, наполненный Ресентиментом человек проявляет такую
выраженную психическую дистанцию со своими жертвами,
что никогда полностью не достигает желаемого устойчивого удовлетворения,
полученного посредством его собственных неэтичных действий. «Возмездие» такого
рода уже не приносит никакой пользы, а «выражение» такого рода лишено всякой
возможности положительных результатов. «В настоящем Ресентименте нет
эмоционального удовлетворения, а только пожизненный гнев и тоска в чувствах,
сравнивающих с другими», — пишет Шелер.

Ресентимент волевой это болезнь воли, когда вместо воли зависть, месть и
порабощение других. Воля к контролю и принуждению — это чистый Ресентимент.

Ресентимент веры — это когда вместо веры фанатизм, ненависть к неверным и
обскурантизм. Это так называемая вера суженного горизонта, когда отказываются
достигать трансцендентного, сосредотачиваясь на имманентном.

Ресентимент не достигает сложной истины. Ресентимент разлагает всякую истину на
более простые истины и противопоставляет эти простые истины.

В то же время Ресентимент бывает очень творческим, потому что свобода
творчества может означать любые устремления, не обязательно положительные.

Самое главное зло Ресентимента — не в разрушении каких-либо отдельных смыслов,
а в разрушении смыслообразования как такового. Смыслы или тем более сложные
смыслы создаются в раскрепощенных и независимых мышлении-патии-воле-вере, а не
в их ресентиментальных проявлениях.

Особенно скандально нежелание западных гуманитариев рефлексировать и
осмысливать современный собственный Ресентимент в таких движениях и явлениях,
которые они считают прогрессивными и перспективными. Речь идет о Black Lives
Matter, радикальном феминизме, «культуре отмены» и кампании
конъюнктурно-ресентиментального насилия, которые можно описать на примере дела
Вайнштейна и движения «me too».

Когда Ресентимент есть сам по себе, можно надеяться, что остальные идентичные
представления могут претендовать на многомерное достоинство или многомерную
справедливость. Но когда Ресентименты накладывается друга на друга, то это
ведет к умножению их и взаимоподдержке. Так возникает Ресентиментальный
комплекс.

В предыдущую эпоху отдельные Ресентименты в революциях и войнах доказывали свою
ограниченность. Но теперь мы живем в эпоху множественного тотального
Ресентимента, который закрывает доступ к сложным формам истины, добра и красоты
и загоняет в упрощение, то есть в фашизм.

Фашизм — это контроль и принуждение для единства общества в простоте. Только
сложность требует договора и общественного контроля над принуждением. Как мы
при этом сформулируем простое единство — класс, нация, гендер, возраст, раса,
гражданство, человечность, актуальная реальность — неважно. Либеральная
демократия никак не обезопасена от фашизма. Ресентимент, которому вовремя не
дали отпор, всегда рождает фашизм, как бы он ни назывался, как бы ни скрывался
и как бы притворно не отрекался от своей сути. Цифровой Ресентимент, в который
мир погружается глобально, это практически цифровой фашизм.
