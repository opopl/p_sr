% vim: keymap=russian-jcukenwin
%%beginhead 
 
%%file 30_07_2021.fb.fb_group.story_kiev_ua.1.velosiped_shuljavka_i_vojna
%%parent 30_07_2021
 
%%url https://www.facebook.com/groups/story.kiev.ua/posts/1718229758373818
 
%%author_id fb_group.story_kiev_ua,majorenko_georgij.kiev
%%date 
 
%%tags detstvo,kiev,pamjat,velosiped,vov
%%title ВЕЛОСИПЕД, ШУЛЯВКА И ВОЙНА
 
%%endhead 
 
\subsection{ВЕЛОСИПЕД, ШУЛЯВКА И ВОЙНА}
\label{sec:30_07_2021.fb.fb_group.story_kiev_ua.1.velosiped_shuljavka_i_vojna}
 
\Purl{https://www.facebook.com/groups/story.kiev.ua/posts/1718229758373818}
\ifcmt
 author_begin
   author_id fb_group.story_kiev_ua,majorenko_georgij.kiev
 author_end
\fi

ВЕЛОСИПЕД, ШУЛЯВКА И ВОЙНА.

\raggedcolumns
\begin{multicols}{3} % {
\setlength{\parindent}{0pt}


На фотографии 20-х годов по улице Борщаговской (Шулявка) на велосипеде едет
мальчик Вова Цуканов. 

Это мой дядя, свидеться с которым мне не довелось.

На другом фото Володя Цуканов с книгой. О чем он мечтал? Кем хотел стать? Мог
ли предвидеть этот мальчишка с Шулявки свою судьбу и предположить, что через 12
лет его семья покинет Киева и он уже никогда не возвратится в город детства.

\ii{30_07_2021.fb.fb_group.story_kiev_ua.1.velosiped_shuljavka_i_vojna.pic.1}

Что отец его - Эммануил Федорович Цуканов станет профессором и завкафедрой в
Днепродзержинском металлургическом институте, а его родной брат Жорж будет
приглашен Брежневым в Москву и станет одной из самых влиятельных и загадочных
личностей в команде Генсека? Об этом более подробнее я расскажу позже...

\ii{30_07_2021.fb.fb_group.story_kiev_ua.1.velosiped_shuljavka_i_vojna.pic.2}

А пока мальчишка Вова Цуканов едет по Шулявке на велосипеде и пишет письмо
своей дорогой тетушке Вере, уехавшей с мужем из Киева далеко, далеко:

"Вера, это я еду к тебе..."

Но дорога мальчика пролегла в другую сторону.

Однажды Володя вырастет, окончит школу и в 1940 году будет призван в Красную
Армию. Проходить срочную службу будет в Белоруссии, которая первой приняла на
себя удары фашистов в сорок первом. И тогда Володя исчезнет без вести, и не
будут знать родные и близкие о его судьбе.


И уже позже выяснится, что в начале июля 1941 Владимир Цуканов попадет к
фашистам в плен, и полтора месяца спустя погибнет в страшном концлагере для
военнопленных Шталаг VIII-E, который был расположен рядом с селом Нойхаммер,
Силезия. 

Люди там содержались в нечеловеческих условиях и гибли в страшных муках от
голода, холода, издевательств и пуль.

Увы, не сохранились фотографии взрослого Володи Цуканова. Так он и остался в
нашей памяти шулявским мальчишкой из 20-х годов.

Ехал мальчик на велосипеде....

\end{multicols} % }
