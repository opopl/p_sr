% vim: keymap=russian-jcukenwin
%%beginhead 
 
%%file slova.nacia
%%parent slova
 
%%url 
 
%%author 
%%author_id 
%%author_url 
 
%%tags 
%%title 
 
%%endhead 
\chapter{Нация}
\label{sec:slova.nacia}

%%%cit
%%%cit_pic
\ifcmt
  pic https://avatars.mds.yandex.net/get-zen_doc/901899/pub_5ff4cffafe4e686f6a93f46c_5ff4d048bb14d54ffb01aba6/scale_1200
	caption Святой равноапостольный князь Владимир. Икона конца XIX века
\fi
%%%cit_text
А вот на самом деле самое главное лицо украинской \emph{нации} – это киевский
князь Владимир Красно Солнышко, это тот, который 1000 лет назад Русь крестил.
Это, по сути, единственная личность, которую современные украинцы приписывают
себе полностью, и не хотят делиться со своими братьями россиянами и белорусами,
утверждая, что это исключительно их герой.  Даже когда несколько лет назад в
Москве был открыт новый памятник Владимиру, как крестителю Руси, украинские
власти и всякие патриоты возмутились так, будто кто-то чужой покусился на их
самое святое. А на самом ли деле киевский князь Владимир Креститель был
украинским? Или хотя бы киевским?
%%%cit_title
\citTitle{Почему киевский хан Владимир Креститель считается украинским князем, когда он даже славянином не был?},
Исторический Понедельник, zen.yandex.ru, 05.01.2021 
%%%endcit

%%%cit
%%%cit_head
%%%cit_pic
%%%cit_text
Если спросить гражданина РФ, кто он по \emph{национальности}, все от тувы до питера
скажут \enquote{да, мы русские}. Здесь давно вытеснили понятие \emph{национальности} и
\emph{национальной гордости}, все малые народности, из которых состоит \enquote{русский
человек}, граждане Российской Федерации, стесняются того, что они таджики,
тувинцы, нанайцы, адыгейцы, кабардины и прочие малые народности. В СССР было
стыдно быть \enquote{нерусским}, поэтому люди стыдились своей \emph{национальности}. Так
стиралась \emph{национальная идентичность}. Вместо нее пришло \enquote{советский человек},
\enquote{русский}, как прилагательное к чему-то.  Ну, не существует \enquote{русских}. Носители
языка – да, но РФ населена не русскими, а различными малыми народностями,
\emph{национальностями}, объединенных в \enquote{гражданин Российской Федерации}
%%%cit_comment
%%%cit_title
\citTitle{Страна без исторических корней всегда будет оккупантом}, 
Олена Степова, news.obozrevatel.com, 17.06.2021
%%%endcit

%%%cit
%%%cit_head
%%%cit_pic
%%%cit_text
Это означает, что единственный путь спасения для российского государства перед
лицом явных угроз – создание собственной \emph{национальной} модели диктатуры. Однако
важнее того – выработка собственного этического кодекса, который мог бы стать
основой для такой \emph{национальной модели}
%%%cit_comment
%%%cit_title
\citTitle{Революция Духа – единственный путь спасения России}, 
Юрий Барбашов, voskhodinfo.su, 30.06.2021
%%%endcit
