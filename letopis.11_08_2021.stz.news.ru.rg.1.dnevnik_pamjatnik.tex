% vim: keymap=russian-jcukenwin
%%beginhead 
 
%%file 11_08_2021.stz.news.ru.rg.1.dnevnik_pamjatnik
%%parent 11_08_2021
 
%%url https://rg.ru/2021/08/11/dnevnik-doneckogo-voenkora-aleksandra-gricenko-o-rasstreliannom-i-voskresshem-pamiatnike.html
 
%%author_id 
%%date 
 
%%tags dnevnik,pamjatnik,doneck,donbass,zhizn,pamjatnik,pamjat
%%title  Я вернулся, мама! Дневник донецкого военкора Александра Гриценко о расстрелянном и воскресшем памятнике
 
%%endhead 

\subsection{Я вернулся, мама! Дневник донецкого военкора Александра Гриценко о расстрелянном и воскресшем памятнике}
\label{sec:11_08_2021.stz.news.ru.rg.1.dnevnik_pamjatnik}

\Purl{https://rg.ru/2021/08/11/dnevnik-doneckogo-voenkora-aleksandra-gricenko-o-rasstreliannom-i-voskresshem-pamiatnike.html}

\enquote{Родина} продолжает вести рубрику, открытую в год 365-летия воссоединения
России и Украины и напоминающую читателям об общем и славном прошлом двух
братских народов.

\begin{multicols}{2} % {
\setlength{\parindent}{0pt}

\ii{11_08_2021.stz.news.ru.rg.1.dnevnik_pamjatnik.pic.1}

\headTwo{25 апреля 2015 года}

Только что созвонился с Анатолием Яновским, заместителем главы Новоазовского
района, он сообщил: \enquote{Накрыли Саханку! Попадания по школе, детскому саду,
частные дома посекло, но в основном стекла и шифер, слава богу, ни убитых, ни
раненых... И еще... они расстреляли Памятник Братской могилы, он уничтожен
полностью...}

\ii{11_08_2021.stz.news.ru.rg.1.dnevnik_pamjatnik.pic.2}

Завтра поеду, сегодня уже не успею, дороги до Новоазовска почти нет, путь в
одну сторону не быстрее 4 часов, это каких-то 100-110 км...

...Освобождение села Саханка и соседних населенных пунктов от немецкой
оккупации началось в сентябре 1943 года силами 32-го и 34-го гвардейских
кавалерийских казачьих полков 9-й гвардейской Кубанской казачьей кавалерийской
дивизии 130-й Таганрогской стрелковой дивизии. В ходе Донбасской наступательной
операции и ожесточенных боев (только 5 сентября противник четырежды атаковал
пехотой и танками \enquote{Тигр} высоты 97,7 и 91,4) войска Южного фронта прорвали
немецкий рубеж обороны.

\ii{11_08_2021.stz.news.ru.rg.1.dnevnik_pamjatnik.pic.3}

Тридцать два бойца Красной армии, погибшие при освобождении Саханки, были
захоронены в братской могиле в центре села. Первый памятник здесь был
установлен в 1948 году. В 1989 году на его месте поднялась трогающая за сердце
5-метровая скульптура \enquote{Мать, поддерживающая раненого бойца}.

Она и была разрушена через 36 лет огнем украинской артиллерии.

\headTwo{19 февраля 2019 года}

Прибыли в Саханку с волонтёрами, привезли очередной гуманитарный груз: семенной
материал (томаты, огурцы, зелень, морковь), детское питание, пополнили,
насколько смогли, аптечку (аптеки нет, фельдшерско-акушерский пункт закрыт).

На вопрос \enquote{в чём больше всего нуждаются жители Саханки?} получили
шокирующий ответ:

- Нам бы Памятник восстановить...

Вокруг мемориала крутилась вся их жизнь. Праздники, выпускные вечера, первые
поцелуи. Нам показали остатки фундамента...

\ii{11_08_2021.stz.news.ru.rg.1.dnevnik_pamjatnik.pic.4}

Как его восстанавливать? Даже просто стоять на этой площадке опасно - она как
на ладони с той стороны. Пообещал сделать все, что в моих силах. Сомнений
больше, чем уверенности.

\ii{11_08_2021.stz.news.ru.rg.1.dnevnik_pamjatnik.pic.5}

\headTwo{6 ноября 2019 года}

Ездим по прифронтовым школам с российским бизнесменом Вадимом Хомичем. Он
привез компьютеры и школьные наборы. Маршрут от Старой Ласпы до Саханки.
Финишируем, конечно, на месте бывшего Памятника, рассказываю его историю. Вадим
внимательно слушает и говорит: \enquote{А давай попробуем! Может, получится сделать
точную копию...}

\ii{11_08_2021.stz.news.ru.rg.1.dnevnik_pamjatnik.pic.6}

\headTwo{7 ноября 2019 года}

Встретились с Вадимом в донецкой чайной \enquote{Золотой жук}. Приняли решение -
делаем! Моя задача - сформировать смету, договориться со строителями. Вадим
берёт на себя основную финансовую поддержку и начинает сбор средств. Лиха беда
начало!

\ii{11_08_2021.stz.news.ru.rg.1.dnevnik_pamjatnik.pic.7}

\figCapA{Учетная карточка воинского захоронения в Саханке.}

\headTwo{24 декабря 2019 года}

Пришли первые деньги от Вадима. Теперь можно оплатить работы по
3D-модели-рованию, которое донецкие спецы сделали по архивным фотографиям.
Надеюсь, что мы не отходим от оригинала.

% Памятник собирали по кусочкам...
\ii{11_08_2021.stz.news.ru.rg.1.dnevnik_pamjatnik.pic.8}

\headTwo{23 января 2020 года}

Выехал в Саханку с опалубкой для нового фундамента. Багажник набит цементом,
иногда машина цепляет брызговиками дорогу. Все думаю, что делать с центральной
плитой возле памятника, с фамилиями бойцов, она скоро совсем рассыплется.

\headTwo{29 января 2020 года}

Удивительное дело! 27 января написал в соцсетях о проблеме с плитой, а сегодня
получил запрос: пришлите размер плиты и детальные данные. Сейчас обмерим,
снимем фамилии и перешлём некоему Доброму Человеку.

(Уже в мае 2021 года выяснилось, что это была гранитная мастерская из Ногинска,
но имени и фамилии Человека так и не удалось узнать)

\ii{11_08_2021.stz.news.ru.rg.1.dnevnik_pamjatnik.pic.9}

\headTwo{5 мая 2020 года}

Всё готово. Загружен грузовой микроавтобус: цемент, арматура, брусья,
отпечатанный контур Памятника. Компания, занимающаяся печатью композиции,
предоставляет людей для ее установки. Это очень хорошо! Всё успеем! До Дня
Победы обещают установить... слабо верится, но они специалисты...

Прыгаем в машины, едем в Саханку!!!

% Долгожданная плита с бессмертными именами.
\ii{11_08_2021.stz.news.ru.rg.1.dnevnik_pamjatnik.pic.10}

\headTwo{7 мая 2020 года}

Около 11.00 обстрел. ВСУ нанесли удар по клубу Саханки, в то время, когда
рабочие занимались кровельными работами. Рядом оказались дети (Настя и Дима).
Дима тяжело ранен.

Эвакуируем работников компании 3D-печати...

% 13-летний Дима: хирурги извлекли из мальчишки более 40 осколков.
\ii{11_08_2021.stz.news.ru.rg.1.dnevnik_pamjatnik.pic.11}

\headTwo{8 мая 2020 года}

Дима (13 лет) доставлен в Донецк. Состояние крайне тяжёлое. Нога - аппарат
Илизарова. Из пацана достали более 40 осколков. Врачи бьются за его жизнь.
Обещают приложить все усилия, чтобы не оставить навсегда инвалидом...

Настя в больнице Новоазовска, у неё касательные ранения головы... Мама Насти -
контузия... Мужики-кровельщики - контузии + осколочные (лёгкие).

Временно останавливаем работы по Памятнику. Увы, к Дню Победы не успеем...

% Памятник собирали по кусочкам...
\ii{11_08_2021.stz.news.ru.rg.1.dnevnik_pamjatnik.pic.12}

\headTwo{18 июня 2020 года}

Гарантировать безопасность сотрудникам 3D-компании я не могу. А Памятник
поднимать нужно дальше. Руки есть. Надо ехать!

Подгорный Игорь, глава Саханки: \enquote{Одному работать - не дело!}, и уже через
полчаса ко мне присоединились двое помощников. В контуре внешней обшивки
обнаружили отверстие, а внутри памятника пулю. Долетает и стрелковое.

Сегодня вникали в технологический процесс. Провозились целый день. Поднялись на
20-25 см.

Быстро не получится.


\end{multicols} % }
