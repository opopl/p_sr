% vim: keymap=russian-jcukenwin
%%beginhead 
 
%%file 21_11_2021.fb.ljoshenko_elena.kiev.jurist.1.gidnist_svoboda_netu.cmt.text.donpress
%%parent 21_11_2021.fb.ljoshenko_elena.kiev.jurist.1.gidnist_svoboda_netu.cmt
 
%%url 
 
%%author_id 
%%date 
 
%%tags 
%%title 
 
%%endhead 

Сегодня годовщина так называемой революции достоинства.

Сегодня будут пафосные речи чиновников и высших руководителей.

Сегодня многие другие люди, которые давно уже разочаровались в «идеалах
майдана» будут говорить высокие слова и пускать искусственные слезы, прекрасно
понимая к чему привела эта революция.

Не смотря на искренние побуждения людей, которые вышли тогда, чтобы жить лучше,
а в итоге стали жить еще хуже.

Они даже на 8 год не хотят признаться даже самим себе, что это было не то что
зря, а фактически откинуло нашу страну на много лет назад.

В экономике, медицине и промышленности. Но чтобы признать это надо быть честным
прежде всего с самими собой, чему многие из нас за эти годы разучились.

Надо быть достаточно смелым и благородным, даже если ты и участвовал во всем
этом.

Признавать свои ошибки это признак действительно сильных и свободных людей,
которыми вы так хотите казаться, несмотря на отражение в зеркале...

Правда в том, что вас (нас) обманули никакого вступления в ЕС, повышения
зарплат и уровня жизни договор об ассоциации нам не дал, да и не мог дать.
Скорее наоборот. Сейчас это ясно каждому, лишь часть стесняется это признать.

Наверняка не было бы войны и потери территории.

Уж точно прошли бы выборы и тот же Янукович просто бы ушел, если бы его не
поддержали люди.

Не было бы роста тарифов, на тоже в электричество и газ в 5-10 раз.

Не было бы убийственных реформ медицины и правоохранительных органов фактически
уничтоживших их.

Не ругались бы со всеми своими соседями включая Беларусь.

Думаю, главное в этой дате.

Это память.

Это наконец-то признать свои ошибки.

Это желание знать и понимать, как не надо делать.

Это ещё один повод потребовать результаты расследования — кто же убивал
протестующих и правоохранителей, почему за это так никто и не ответил?

Это перестать жить двойной жизнью и двойной моралью.

Это и будет первым шагом к строительству, возрождению новой, доброй, радушной,
сильной и богатой Украины, которую мы тогда потеряли и которую все в мире так
любили и уважали.

Простите, что так сумбурно.

Но это от души.

Надоели эта двойная мораль.

Каждый год в этот день.
