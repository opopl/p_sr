% vim: keymap=russian-jcukenwin
%%beginhead 
 
%%file 14_06_2023.stz.news.ua.donbas24.1.mariupol_krajeznavchyj_muzej_novi_proekty
%%parent 14_06_2023
 
%%url https://donbas24.news/news/mariupolskii-krajeznavcii-muzei-rozpocinaje-novi-projekti
 
%%author_id demidko_olga.mariupol,news.ua.donbas24
%%date 
 
%%tags 
%%title Маріупольський краєзнавчий музей розпочинає нові проєкти
 
%%endhead 
 
\subsection{Маріупольський краєзнавчий музей розпочинає нові проєкти}
\label{sec:14_06_2023.stz.news.ua.donbas24.1.mariupol_krajeznavchyj_muzej_novi_proekty}
 
\Purl{https://donbas24.news/news/mariupolskii-krajeznavcii-muzei-rozpocinaje-novi-projekti}
\ifcmt
 author_begin
   author_id demidko_olga.mariupol,news.ua.donbas24
 author_end
\fi

\begin{qqquote}
Маріупольський музей почав збір цінних артефактів та спогадів маріупольців 
\end{qqquote}

Маріупольський краєзнавчий музей, заснований у 1920 році, внаслідок
широкомасштабної збройної агресії Росії через численні обстріли та
бомбардування зазнав
\href{https://donbas24.news/news/den-muzeyiv-yak-rosiyani-ruinuyut-kulturnu-spadshhinu-donbasu-foto}{суттєвих
руйнувань}. У результаті неодноразових прямих влучань в музеї спалахнула
пожежа, що охопила всі поверхи будівлі. На жаль, музей втратив свої експозиції
та численні фондові зібрання. Значних руйнувань зазнали і філії музею: Музей
народного побуту та Художній музей імені А. І.  Куїнджі, розташовані на одній
вулиці з краєзнавчим музеєм. Наразі співробітники Маріупольського краєзнавчого
музею почали низку важливих проєктів, спрямованих на відродження музеїв
міста-героя.

\textbf{Читайте також:} \href{https://donbas24.news/news/mariupolskii-muzei-im-ai-kuyindzi-vidteper-mozna-vidvidati-ne-vixodyaci-z-domu-foto}{Маріупольський музей ім. А.І. Куїнджі відтепер можна відвідати, не виходячи з дому (ФОТО)}

\subsubsection{Збір експонатів}

Частина працівників музею виїхала на підконтрольну Україні територію та
розпочала роботу, спрямовану на відновлення роботи Маріупольських музеїв. З
огляду на це КУ \enquote{Маріупольський краєзнавчий музей} намагається налагодити
співпрацю з усіма маріупольцями, які виїхали з окупованого міста і готові
поділитися артефактами та предметами (або їх зображеннями) з історії міста
Маріуполь, зокрема тими, що пов'язані з сучасною історією. 

\begin{leftbar}
	\begingroup
{\em\enquote{Відродження міста Маріуполь та його музеїв не зможе відбутися без участі
та допомоги громади міста, і хоча зараз ми перебуваємо у різних містах
та країнах, ми можемо згуртуватися та почати відроджувати
Маріупольський музей вже зараз. Матеріальна і культурна спадщина нашого
міста має бути доступною для всіх істориків, дослідників та загалом
всієї громади Маріуполя. Тільки разом ми зможемо відновити наші музейні
фонди}}, — наголосив виконучий обов'язки директора Маріупольського
краєзнавчого музею Олександр Горе.
	\endgroup
\end{leftbar}

Зараз у фондах музею вже є цінні експонати, серед яких і подарунки, які
передають з музеїв та установ інших міст України виконуючому обов'язки
директора Маріупольського краєзнавчого музею Олександру Горе. Зокрема, в музеї
вже зберігається Значок та сувенірна медаль \enquote{Одеса - перлина біля моря}. 

\textbf{Читайте також:} 

\href{https://donbas24.news/news/ukrayinska-ilyustratorka-malyuje-kartini-kavoyu-sered-yiyi-robit-arxitektura-mariupolya-foto}{%
Українська ілюстраторка малює картини кавою: серед її робіт - архітектура Маріуполя (ФОТО)}

\ii{14_06_2023.stz.news.ua.donbas24.1.mariupol_krajeznavchyj_muzej_novi_proekty.pic.1}

Протягом останніх місяців фонди музею щодня поповнюються. Одним з перших, хто
передав експонати до нової колекції музею, став Михайло Шехалі. Зокрема, він
передав унікальну книгу відомого маріупольського краєзнавця Сергія Давидовича
Бурова \enquote{Мариуполь. Былое} та унікальні маріупольські листівки ХХ сторіччя.

\ii{14_06_2023.stz.news.ua.donbas24.1.mariupol_krajeznavchyj_muzej_novi_proekty.pic.2}

\textbf{Читайте також:} 

\href{https://donbas24.news/news/muzei-zgoriv-kamyana-baba-rozsipalasya-mariupolec-slavik-dolzenko-vryatuvavsya-z-okupaciyi}{%
Музей згорів, кам'яна баба розсипалася: маріуполець Славік Долженко врятований з окупації}

\subsubsection{Збір свідчень}

Спільно з
\href{https://donbas24.news/news/vidnovlennya-knizkovogo-fondu-mariupolya-koli-vidkrijetsya-nova-biblioteka}{Центральною
міською бібліотекою ім. В. Г. Короленка м. Маріуполь} Маріупольський
краєзнавчий музей розпочав реалізацію проєкту \enquote{Маріуполь — невтрачена
пам'ять}, головна мета якого — зібрати усні свідчення та спогади маріупольців.
Проєкт передбачає проведення відео-інтерв'ю з маріупольцями, що вимушено
покинули місто і бажають допомогти в збереженні повної та об'єктивної історії
міста з метою збору та документування усної історії міста, періоду
російсько-української війни та історії міста загалом. Матеріали проєкту стануть
частиною експозиційної та виставкової роботи музею, публікацій та відкритого
архіву пам'яті міста.

\begin{leftbar}
	\begingroup
\enquote{Ми записуємо розповіді маріупольців, щоби зберегти їх свідчення та
зробити їх гучнішими. Ми збираємо живі спогади та розповіді, щоби
зробити їх частиною нашої спільної історії — історії, яку треба
розповідати}, — підкреслив Олександр Горе.
	\endgroup
\end{leftbar}

\ii{14_06_2023.stz.news.ua.donbas24.1.mariupol_krajeznavchyj_muzej_novi_proekty.pic.3}

\textbf{Читайте також:} 

\href{https://donbas24.news/news/mariupolska-kamerna-filarmoniya-dolucilasya-do-projektu-prisvyacenomu-zagiblim-mitcyam}{Маріупольська камерна філармонія долучилася до проєкту, присвяченому загиблим митцям (ВІДЕО)}

Крім цього, наразі співробітники музею проводять збір експонатів та свідчень за
темою фільтраційних заходів, що проводилися в Маріуполі загарбниками, для
створення тематичної виставки, проведення якої планується у серпні в Києві. 

\begin{leftbar}
	\begingroup
		\bfseries
			\enquote{Будемо дуже вдячні за інформацію, світлини або експонати за темою}, — додав Олександр.
	\endgroup
\end{leftbar}

Маріупольський краєзнавчий музей дуже сподівається на підтримку у відновленні
роботи та прямо зараз приймає: світлини, предмети, книги — все, що пов'язано з
історією та культурою Маріуполя.

Писати можна на пошту: \url{mamu1920@ukr.net} та сторінки у Facebook чи Instagram,
тел.: 0963 700 162. Експонати можна залишати і в центрах \enquote{Я Маріуполь}.

Раніше Донбас24 розповідав, що маріупольська актриса з українськими віршами виступає в німецьких містах.

Ще більше новин та найактуальніша інформація про Донецьку та Луганську області в нашому телеграм-каналі Донбас24.

ФОТО: з архіву Олександра Горе
