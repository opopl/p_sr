% vim: keymap=russian-jcukenwin
%%beginhead 
 
%%file slova.imja
%%parent slova
 
%%url 
 
%%author 
%%author_id 
%%author_url 
 
%%tags 
%%title 
 
%%endhead 
\chapter{Имя}
\label{sec:slova.imja}

Облишимо культурно зумовлені факти. Маркес, Борхес і Льйоса змусили мене
задуматися над іншим. Власне, над тим моментом зрілості, коли сам обираєш, як
ти називаєшся. Навіть прийнявши заготовлені родом варіанти, людина має свободу
вибору. Хтось живе як Толік, хтось — як Анатолій Васильович, хтось взагалі стає
Анатоль. Кілька літер, додаткових звуків і відразу зовсім інший портрет. Навіть
те, як людина вимовляє \emph{власне ім'я}, повільно й поважно, чи квапливо і голосно,
скаже чимало про власника і його ментально-душевний стан.  Напевне, в
дорослості, ми всі розуміємо — немає красивих і некрасивих \emph{імен}. Є особистості,
що своїм досвідом і характером перетворили власне \emph{ім’я} на символ. Символ сили
чи слабкості, перемоги чи підлості.  В дитинстві ви не раз чули, та й самі
вимовляли питання — \enquote{як ти називаєшся?} Не поспішайте його виправляти на \enquote{як
тебе звуть?} Бо то два різних питання. Останнє, усталене в дорослому
суспільстві, де інші вирішують, як тебе звуть. А перше, дитяче і вільне,
запитує суть — як ти сам називаєшся?  Тож усе, що я хочу сказати цим текстом, —
не поспішаймо з власним \emph{ім’ям},  його варто відчути, тоді й інші відчують, як ти
називаєшся і, головне, чому,
\citTitle{Як ти називаєшся?}, Анна Данильчук, day.kyiv.ua, 10.06.2021


%%%cit
%%%cit_head
%%%cit_pic
\ifcmt
  pic https://avatars.mds.yandex.net/get-zen_doc/1874110/pub_60bf11f12b92e05b9b222e4c_60bf13034adb2513a39137cb/scale_1200
	caption Художник Игорь Ожиганов
	width 0.4
\fi
%%%cit_text
Сколько у вас \emph{имен}? Одно по паспорту и всё? А вот наши предки славяне имели как
минимум, два – одно домашнее, потаённое, второе - для всех. Иной раз добавляли
и третье, когда отрок взрослел и значительно изменялся. Может, в те
достопамятные времена наши родичи и были язычниками, но смысл в сокрытии своего
истинного \emph{имени}, которое было созвучно трепетным струнам души, все-таки был,
так они защищали личное пространство от посторонних и не всегда добрых людей. В
сглаз и порчу можно не верить, но отрицать разрушительную силу злого пожелания,
все-таки, не стоит. Береженого Бог бережет
%%%cit_comment
%%%cit_title
\citTitle{Забытая красота славянских имен}, 
Материя Мысли, zen.yandex.ru, 08.06.2021
%%%endcit

