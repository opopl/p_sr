% vim: keymap=russian-jcukenwin
%%beginhead 
 
%%file 12_12_2017.stz.news.ua.mrpl_city.1.andriivski_vechornyci
%%parent 12_12_2017
 
%%url https://mrpl.city/blogs/view/andriivski-vechornitsi
 
%%author_id demidko_olga.mariupol,news.ua.mrpl_city
%%date 
 
%%tags 
%%title Андріївські вечорниці
 
%%endhead 
 
\subsection{Андріївські вечорниці}
\label{sec:12_12_2017.stz.news.ua.mrpl_city.1.andriivski_vechornyci}
 
\Purl{https://mrpl.city/blogs/view/andriivski-vechornitsi}
\ifcmt
 author_begin
   author_id demidko_olga.mariupol,news.ua.mrpl_city
 author_end
\fi

Незважаючи на вік, кожен повинен вірити в дива. Адже без них світ стане нудним
і нецікавим. Якщо в якісь моменти був поганий настрій або просто не щастило,
то, звичайно, станеться диво, яке раптом порадує. Однак найчастіше справжні
дива відбуваються у зимові свята.

На території України протягом зимових місяців відзначають дуже різні за своїм
характером і змістом зимові свята. Залежно від етнічної, релігійної, класової
приналежності українців, а також історичної епохи, змінювалися дати і характер
святкування зимових святкових днів. Однак є свята, які відомі далеко не всім
мешканцям України. Так, свято Андрія Первозванного вважається найбільшим
передноворічним святом, яке відзначається ще з часів Київської Русі 13 грудня.
Андрій Первозванний входив в число 12-ти апостолів Христа, згідно з переказами,
був братом апостолу Петру і покровителем жителів України, Росії та Шотландії. У
стародавніх літописах зустрічаються описи того, як під час останньої місійної
подорожі апостол побував на берегах Чорного моря, пропливав Дніпром у Скіфію і
навіть побував в місці, де пізніше постане величне місто Київ. Напередодні 13
грудня прийнято збиратися на \emph{Андріївські вечорниці}, з якими пов'язано безліч
неординарних звичаїв. Проте в історії Маріуполя не зафіксовано святкування
вечорниць, що з легкістю можна виправити.

\ii{12_12_2017.stz.news.ua.mrpl_city.1.andriivski_vechornyci.pic.1}

Вечорниці проходять в ніч з 12-го на 13-е грудня і в основному святкуються
тільки молоддю. Вражає той факт, що свято Андрія – один день в році, коли
молодь може бешкетувати, адже їм усе прощається. Головною формою святкування
Андрія на всьому просторі розселення українців були молодіжні зібрання зі
спільними застіллями, різноманітними забавами та іграми. Подекуди їх називали
\enquote{великими вечорницями}, \enquote{святом дівчат}, які супроводжувалися веселими співами,
довгими дівочими посиденьками, знайомствами і заграваннями молоді. Вечорниці
тривали всю ніч, оскільки наступного разу молоді люди могли поспілкуватися
тільки на гуляннях напередодні Різдва.

\ii{12_12_2017.stz.news.ua.mrpl_city.1.andriivski_vechornyci.pic.2}

За народними уявленнями, святий Андрій є втіленням чоловічого начала,
провісником долі, покровителем кохання і шлюбу. Ймовірно ці характеристики
Андрія спочатку склалися у католиків і протестантів, у яких з днем святого
пов'язується період адвенту (передріздвяного посту) і церковного Нового року –
найзручніший час для віщування й передбачення майбутнього.

Андріївські ворожіння були присвячені темі шлюбу. Дівчата намагалися дізнатися,
хто буде їхнім чоловіком, чи буде він багатим, який матиме характер, чи стануть
вони щасливі разом, тощо. 

За традицією, в цей вечір і ніч хлопці робили різноманітні дрібні капості
дівчатам (викрадали і ховали предмети одягу, прикраси). З настанням темряви
хлопці лякали їх на вулицях або стукали у вікна, говорячи на різні лади,
намагаючись таким чином звернути на себе увагу.

Дівчата ж в цей час займалися таємним обрядом – проводили ворожіння на
судженого і на результат майбутнього сімейного життя. Іноді дійством керувала
заміжня і щаслива у сімейному житті ворожка, але найчастіше дівиці виступали в
ролі ворожок самостійно. Щоб дізнатися ім'я нареченого, дівчатам було приписано
кидати знятого з ноги червоного чобота, – якнайдалі через голову. Кинуте взуття
уважно розглядали, оскільки вважалося, що суджений прийде в будинок з того
боку, куди вказує носок чобітка.

Не менш поширеними були ворожіння з використанням кільця, скрученої з паперу
квітки, ляльки і гілочок мирти. На всіх цих предметах ворожка робила
замовляння, а потім пропонувала дівчатам по черзі тягнути жереб. Вважалось, що
та, яка витягне із загальної маси кільце, – незабаром вийде заміж.

\ii{12_12_2017.stz.news.ua.mrpl_city.1.andriivski_vechornyci.pic.3}

Своєрідним символом свята Андрія є обрядове печиво, яке використовувалось у
різноманітних іграх та магічних церемоніях. Йдеться про андріївську калиту,
відомою ще під назвою \enquote{коржа}, або \enquote{коляти}. У його виготовленні намагалися
взяти участь усі дівчата, що сходилися на вечорниці. Своєю круглою формою
\enquote{калита} нагадувала сонце. Пекли її з пшеничного або житнього борошна, обсипали
маком і змащували зверху медом і маківкою, витискали орнамент. Калиту
підвішували в хаті до сволока, її охороняв пан Калитинський, тримаючи в руках
квач чи віхоть. Хлопці, дівчата і навіть господиня хати по черзі їхали \enquote{кусати}
коржа на рогачі або кочерзі. Сенс гри полягав у тому, щоб під'їхати \enquote{верхи},
відкусити без допомоги рук шматок печива і не розсміятися. Кому це не вдалося –
ставав об'єктом жартівливої кари, і пан Калитинський обмащував невдасі обличчя
сажею (глиною або крейдою).

Сьогодні в українських містах продовжують відзначати Андріївські вечорниці, але
якщо раніше хлопці ніколи не заглядали до дівчат на вогник, а саме ворожіння
було таїнством, то зараз молодь часто збирається разом. До ворожіння
відносяться жартівливо, а самі вечорниці сприймають як чудовий привід
повеселитися перед Новим роком.

Зима – час, коли віра в дива приходить до кожного. Можливо, людина сама цього
не визнає, але в глибині душі вона відчуває, що з нею ось-ось станеться щось
чарівне. Тож чому б не прискорити ці самі дива і не почати відзначати зимові
свята з Андріївських вечорниць, які наповнять будинки сміхом молоді і надією на
краще майбутнє...
