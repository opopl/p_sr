% vim: keymap=russian-jcukenwin
%%beginhead 
 
%%file 03_03_2023.fb.kipcharskij_viktor.mariupol.1.r_k_tomu__bulo_take_.cmt
%%parent 03_03_2023.fb.kipcharskij_viktor.mariupol.1.r_k_tomu__bulo_take_
 
%%url 
 
%%author_id 
%%date 
 
%%tags 
%%title 
 
%%endhead 

\qqSecCmt

\iusr{Марина Солошенко}

Фейсбук не пережив того, що маріупольці, тому і заблокував!

\iusr{Олена Крайова}

Читаю ваш щоденник кожного дня. З кожним днем мені все складніше змусити себе
його читати, страшно, але читаю. Прокоментувати не можу... слова безсилі тут

\begin{itemize} % {
\iusr{Віктор Кіпчарський}
\textbf{Олена Крайова} 

Мій щоденник має щасливе закінчення - ми усі живі.

Безмежно вдячні Вам і Ірині за Ваше терпіння, з яким Ви вислухали нас в черзі
на Палія.

Я в захопленні від Вашої сміливості: ми лише тікали від того пекла, а Ви туди
повернулися, аби витягнути рідних...

\iusr{Олена Крайова}

Всі тоді все робили на якійсь інтуїції, ніхто не знав, як треба, як можна. Може
виживеш, а може ні. Стільки людей, Боже. Я змогла заплакати перший раз від
початку війни 24 лютого цього року. Прокинулась в голові із цією фразою. І все,
мене перший раз за рік прорвало.

\iusr{Віктор Кіпчарський}
\textbf{Олена Крайова} Ми жили - тому й вижили.
Ні, ще не вдається покласти на слова те, що в голові. чи на душі.
\end{itemize} % }

\iusr{Marina Dervinskay}

Пожалуйста перешлите мне этот дневник воспоминаний, нам больше не чего
вспомнить Мариупол пл. Кирова, наш дом уже под бульдозерами.

\begin{itemize} % {
\iusr{Віктор Кіпчарський}
\textbf{Marina Dervinskay} 

Я щоночі перекладаю один день спогадів, виправляю помилки (попечені, поморожені
та набиті сокирою пальці погано слухалися) і викладаю на своїй сторінці.
Готового ще нема.

Зустрічна пропозиція: міняю на Ваші спогади, які не використаю без Вашого
дозволу. Я розумію, що згадувати важко, але це як витягати скалку: боляче, але
необхідно.

Запевняю, поділившись спогадами, Ви розділите свою біду і відчуєте полегшення.

\end{itemize} % }
