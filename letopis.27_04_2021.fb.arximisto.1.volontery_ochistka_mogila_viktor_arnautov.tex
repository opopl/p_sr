%%beginhead 
 
%%file 27_04_2021.fb.arximisto.1.volontery_ochistka_mogila_viktor_arnautov
%%parent 27_04_2021
 
%%url https://www.facebook.com/arximisto/posts/pfbid0366YATzC1TbQPpQysYdRBNEPNa47jovfkc7VkqUonMhwDSiXhKdLFQ4nvaeR2ihmbl
 
%%author_id arximisto
%%date 27_04_2021
 
%%tags 
%%title Волонтеры очистили от мусора могилу с прахом Виктора Арнаутова
 
%%endhead 

\subsection{Волонтеры очистили от мусора могилу с прахом Виктора Арнаутова}
\label{sec:27_04_2021.fb.arximisto.1.volontery_ochistka_mogila_viktor_arnautov}

\Purl{https://www.facebook.com/arximisto/posts/pfbid0366YATzC1TbQPpQysYdRBNEPNa47jovfkc7VkqUonMhwDSiXhKdLFQ4nvaeR2ihmbl}
\ifcmt
 author_begin
   author_id arximisto
 author_end
\fi

Волонтеры очистили от мусора могилу с прахом Виктора Арнаутова, всемирно
известного мариупольского художника

\#новости\_архи\_города

В Мариупольском Некрополе волонтеры очистили от мусора могилу Аделаиды Ивановны
Арнаутовой (1873-1937), матери Виктора Михайловича Арнаутова, всемирно
известного художника-муралиста, ученика Диего Риверы, автора муралов и мозаик в
Сан Франциско, Мексике и Мариуполе. Урна с его прахом была захоронена в могиле
его женой Нонной Талепоровской после его смерти в 1979 году.

В одном месте Некрополя расположены могилы Аделаиды Арнаутовой, ее дочери и
сестры художника Лидии Михайловны Астаховой (1901-1961), а также племянницы
художника Лидии Михайловны Астаховой (1933-1951).

Еще восемь лет назад на могилах стояли кресты, а весь комплекс был окружен
металлической оградой (см. фото). В прошлом году Комбинат коммунальных
предприятий очистил эту территорию от непроходимых зарослей и оказалось, что
она была пристанищем для наркоманов, бомжей и алкоголиков.

Исчезли ограда и кресты. Надгробие Аделаиды Арнаутовой и урна с прахом были
разбиты. Все пространство усеяно шприцами, тряпками и осколками бутылок.

Волонтеры вынесли более десятка мешков с мусором, как сообщил Андрей Марусов,
директор ГО \enquote{Архи-Город}, а также тщательно очистили ближайшее пространство.

Виктор Михайлович Арнаутов своим талантом прославил и Мариуполь, и приазовских
греков на весь мир. Мы шокированы состоянием его могилы, как заявила Надежда
Андреевна Чапни, председатель совета Мариупольского общества греков и участница
уборки. На ближайшем заседании совета мы будем обсуждать вопрос о создании
памятника художнику в Некрополе.

Место захоронения праха Виктора Арнаутова долгие годы было окутано тайной и
неизвестностью. Вероятная причина – незаконность захоронения его праха на
кладбище (оно было закрыто для захоронений в 1972 г.).

Как пишет Роберт Черни, профессор истории университета Сан Франциско и автор
первой англоязычной биографии художника \enquote{Viktor Arnautoff and the Politics of
Art}, \enquote{Нонна Талепоровская обещала Виктору, что после кремации его прах будет
захоронен рядом с его мамой, на Старом кладбище Мариуполя. Когда власти Жданова
не разрешили это сделать, Нонна официально зарегистрировала его захоронение на
участке ее семьи на Богословском кладбище в Ленинграде. Спустя некоторое время
она вместе с Игорем Стефановым, [дальним родственником художника], привезла
прах в Жданов и тайно поместила урну в могилу его матери}.

Напомним, что в Мариуполе Виктор Арнаутов стал автором \textbackslash со-автором знаменитой
мозаики \enquote{Покорение космоса} на Доме связи, мозаики \enquote{Учительница с детьми} на
школе № 54, а также мозаичного фриза \enquote{От скифов к космосу} в Мариупольском
аэропорту. Его картины хранятся в Художественном музее им. А. Куинджи, а также в
музеях и частных коллекциях США, Украины, России и стран ЕС. 

Мы благодарны волонтерам, участникам уборки могилы с прахом В. Арнаутова, -
Марине Головновой, Елена Сугак, Надежде Чапни, Андрею Клепикову и Александру
Сальникову, а также Андрею Никифоренко и Илья Луковенко.

Мы также признательны благотворителям, чьи взносы позволили закупить необходимый инвентарь, перчатки и мешки для уборки!  

==============

Биография Виктора Арнаутова: Robert Cherny. Victor Arnautoff and the Politics
of Art, University of Illinois Press, 2017 \footnote{\url{https://cutt.ly/bbeh2iJ}}

Мариупольское общество греков\footnote{\url{http://www.greeks.in.ua}}

\#mariupol\_necropolis\_report
