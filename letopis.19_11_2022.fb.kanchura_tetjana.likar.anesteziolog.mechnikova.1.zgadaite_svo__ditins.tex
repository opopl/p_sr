%%beginhead 
 
%%file 19_11_2022.fb.kanchura_tetjana.likar.anesteziolog.mechnikova.1.zgadaite_svo__ditins
%%parent 19_11_2022
 
%%url https://www.facebook.com/tanjushka.djan/posts/pfbid032p2zWYSyDr7GbQMStq1rZS6QmekbcTtnW8sxwfYUTUSu3cQQjsoBZoWM9fXmwLCGl
 
%%author_id kanchura_tetjana.likar.anesteziolog.mechnikova
%%date 19_11_2022
 
%%tags donor,zima
%%title Згадайте своє дитинство. Мама зимою приводить вас в поліклініку, там холодно настільки, що бульби в носі замерзають
 
%%endhead 

\subsection{Згадайте своє дитинство. Мама зимою приводить вас в поліклініку, там холодно настільки, що бульби в носі замерзають}
\label{sec:19_11_2022.fb.kanchura_tetjana.likar.anesteziolog.mechnikova.1.zgadaite_svo__ditins}

\Purl{https://www.facebook.com/tanjushka.djan/posts/pfbid032p2zWYSyDr7GbQMStq1rZS6QmekbcTtnW8sxwfYUTUSu3cQQjsoBZoWM9fXmwLCGl}
\ifcmt
 author_begin
   author_id kanchura_tetjana.likar.anesteziolog.mechnikova
 author_end
\fi

Згадайте своє дитинство. Мама зимою приводить вас в поліклініку, там холодно
настільки, що бульби в носі замерзають. Холодне всьо кушетки, кабінет лікаря,
коридори, а здати кров з пальчика в таких умовах це просто саме жахіття. Вже не
ясно од чого больніше, од укола, чи од того, що тьотя-лаборант тисне на твій
палець з усієї сили. Згадали?

А теперь уявіть, що в таких умовах працюють центри крові в прифронтових
областях. В Харкові в укритті, де приймають донорів, не має опалення. Зовсім.
Його обіцяють зробити з літа, лише обіцяють. Нажаль. В запорізькому банку крові
холодно настільки, що там не працюють аналізатори. Навіть техніка не підвозить
таких умов. І це я вам розказала лише про два регіона...

Зима буде над важкою. Потреба в крові буде рости і дуже важливо зберегти
можливість її заготівлі. Тому ми всією командою \href{https://www.facebook.com/DonorUA.Ukraine}{ДонорUA} просимо вас зігріти
донорів та долучитись до збору на обігрівачі для центрів крові. 

Що робити?

- кинути пару гривень в банку можна тут:

\url{https://send.monobank.ua/jar/9KKCWu8rhD}

- написати коментар або зробити репост.

- написати допис на своїй сторінці і розказати про цю проблему. 

Будьте котиками - зігрійте донорів. Цьом.
