% vim: keymap=russian-jcukenwin
%%beginhead 
 
%%file 06_11_2021.fb.lazarev_dmitrij.ekaterinburg.rossia.pisatel.fantast.1.aleksej_kelin_nenauchnyj_podhod
%%parent 06_11_2021
 
%%url https://www.facebook.com/ldv.writer/posts/2046791632137240
 
%%author_id lazarev_dmitrij.ekaterinburg.rossia.pisatel.fantast
%%date 
 
%%tags literatura,recenzia,rossia,rusmir
%%title Читательское. Алексей Келин «Ненаучный подход»
 
%%endhead 
 
\subsection{Читательское. Алексей Келин «Ненаучный подход»}
\label{sec:06_11_2021.fb.lazarev_dmitrij.ekaterinburg.rossia.pisatel.fantast.1.aleksej_kelin_nenauchnyj_podhod}
 
\Purl{https://www.facebook.com/ldv.writer/posts/2046791632137240}
\ifcmt
 author_begin
   author_id lazarev_dmitrij.ekaterinburg.rossia.pisatel.fantast
 author_end
\fi

Читательское. Алексей Келин «Ненаучный подход». Уже третья прочитанная мной
книга из данного цикла. Первые две - «Этикет следствия» и «Семейные
обязательства». Собственно, то, что все еще читаю, говорит, что автор марку
держит, выдавая вещи достаточно высокого уровня. Как определить жанр этой
книги? Не так-то просто. Пожалуй, это политический детектив в
фэнтезийно-историческом антураже.

\ifcmt
  ig https://scontent-frt3-1.xx.fbcdn.net/v/t39.30808-6/251329876_2046791362137267_4863927566699775694_n.jpg?_nc_cat=107&ccb=1-5&_nc_sid=730e14&_nc_ohc=8sLvKkGkOBUAX9PGHup&_nc_ht=scontent-frt3-1.xx&oh=8c869c057d345f856afe602607c64d12&oe=61AA8B1C
  @width 0.4
  %@wrap \parpic[r]
  @wrap \InsertBoxR{0}
\fi

Пройдусь по плюсам. Автор придумал очень интересный мир и знает его в
совершенстве. При этом не вываливает это свое знание в книгу в лекционном
стиле, а дает дозированно в диалогах и происходящих событиях. Очень здорово
продумана хитрейшая и многоходовая политическая интрига, на которой держится
весь сюжет. Ну и очень тесно переплетенная с интригой детективная линия также
отменно хороша. Герои, пришедшие в «Ненаучный подход» в основном из первой
книги и немного из второй, прописаны также очень качественно. Даже
второстепенные тут не являются исключением. Никакого картона – настоящие
многогранные личности со всеми их тараканами в голове, скелетами в шкафу,
рефлексией, тайными желаниями, достоинствами и недостатками – в общем, еще один
признак хорошей литературы.

Героям хочется сопереживать, читать про них интересно, равнодушными они не
оставляют. Виктор фон Берген, бывший имперский аристократ, бывший военный, а
ныне простой мент, пардон, сотрудник стражи соседнего государства. Честный и
принципиальный, но все время бегущий от прошлого, которое, зараза, догоняет.

Винс – мальчишка, бывший беспризорник, состоявший в банде, а теперь камердинер
Виктора и верный его помощник во всех делах, в том числе и следственных.

Мистрис Анна, полицейский эксперт по магии, потенциально сильнейший некромант
континента, которой принципы не позволяют брать силу из мук и смертей и которая
всего-то хочет защитить диссертацию.

Георг фон Раух – блестящий имперский кавалергард с безупречными манерами и
острым умом, способный порвать на ленточки кого угодно и всегда себе на уме.

Короче, перечислять можно долго, и у каждого героя есть своя фишка. Каждый
вовлечен в главное расследование – той самой многоходовой политической интриги,
ставки в которой запредельно высокие, но попутно решает и свои личные задачи.

Теперь к минусам. Их немного, но есть. Несколько затянуто. Главным образом за
счет диалогов. Они хороши, но… можно бы их и поменьше, особенно тех, в которых
выдается политинформация.

Темп повествования несколько рваный, и активное действие периодически сменяется
длинными участками, когда почти ничего не происходит. Грубо говоря, лошадь не
то что рысью, а шагом идет. Потом - раз, и в галоп, а потом резко опять шаг.

К финалу претензий нет. Сюжет изящно закруглен, тайны раскрыты, с логикой
детектива проблем никаких. В общем, роман качественный, любителям
вышеозначенных жанров почитать стоит.

Ссылка в первом комменте

\begin{itemize} % {
\iusr{Дмитрий Лазарев}
\url{https://author.today/work/147559}
\end{itemize} % }
