% vim: keymap=russian-jcukenwin
%%beginhead 
 
%%file 03_10_2017.stz.news.ua.mrpl_city.1.budynok_hameleon
%%parent 03_10_2017
 
%%url https://mrpl.city/blogs/view/mariupolskij-budinok-hameleon
 
%%author_id demidko_olga.mariupol,news.ua.mrpl_city
%%date 
 
%%tags 
%%title Маріупольський будинок-хамелеон
 
%%endhead 
 
\subsection{Маріупольський будинок-хамелеон}
\label{sec:03_10_2017.stz.news.ua.mrpl_city.1.budynok_hameleon}
 
\Purl{https://mrpl.city/blogs/view/mariupolskij-budinok-hameleon}
\ifcmt
 author_begin
   author_id demidko_olga.mariupol,news.ua.mrpl_city
 author_end
\fi

В Україні є багато будинків, що вражають своєю самобутністю, унікальністю та
неординарністю. Це  будинок легенд у Львові, будинок-хвиля в Одесі,
будинок-скульптура в Луцьку. До цього переліку ми сміливо можемо додати і
маріупольський будинок, який знаходиться по проспекту Миру, 40. Адже у різний
час він мав різні назви, у ньому знаходилися установи, що мали
діаметрально-протилежне значення, чим він, безумовно, дивує і вражає. Очима
цього будинку можна побачити безліч важливих культурних та політичних подій, що
відбувалися у Маріуполі з початку XX ст. і до сьогодення. Він ніби справжній
хамелеон пристосовувався до будь-якої зовнішньої ситуації, суспільства,
соціальних умов.

\ii{03_10_2017.stz.news.ua.mrpl_city.1.budynok_hameleon.pic.1}
\ii{03_10_2017.stz.news.ua.mrpl_city.1.budynok_hameleon.pic.2}

Можливо, найбільший вплив на будинок мав перший господар - відомий у Маріуполі
громадський діяч Ілля Еммануїлович Юр'єв, якому в силу власної професії часто
доводилося пристосовуватися до зовнішніх обставин.

Завдяки проведеному дослідженню відомого маріупольського краєзнавця Бурова
Сергія Давидовича, вдалося дізнатися, що Ілля Еммануїлович був присяжним
повіреним (адвокат при окружному суді або судовій палаті). Наприкінці 1906 року
Іллю Юр'єва обирають гласним у міську думу. На першому засіданні думи нового
складу, 31 березня 1907 року, його обрали головою. За масштабами невеликого
повітового міста Юр'єв був доволі впливовим громадським діячем. Він був членом
повітової училищної ради, попечителем міського початкового училища ім.
Митрополита Ігнатія, одним з трьох старшин Клубу комерційних зборів.

Відомо, що Юр'єв був холостяком, але на самоті не залишався, адже часто в гості
до Іллі Еммануїловича заходили найвпливовіші маріупольці. У святкові дні
збиралися на балконі будинку, щоб насолодитися дійствами, які відбувалися на
вулиці Катерининській – центральній та головній вулиці міста.

Після смерті Юр'єва у будинку з 1900 р. розташувалася редакція газети
\enquote{Мариупольский справочный листок}. А наприкінці 1912 р. підприємець П.
Урбанський відкрив на першому поверсі будинку кінозал \enquote{ХХ століття}. Саме тут
маріупольці могли насолодитися останніми новинками кіно та відпочити у вихідні
дні.

За радянських часів кінотеатр закрили, і у 1922 році тут розташувався  дитячий
будинок ім. Леккерта. А пізніше у будинку було створено перший піонерський
табір Маріуполя, тут маріупольці ставали піонерами і приймали урочисту присягу.

Цікаво, що у передвоєнні роки тут знаходився місцевий відділ НКВС (Народний
комісаріат внутрішніх справ), який виносив жорстокі вироки на заслання, табори,
страту ні в чому невинних людей. У ті часи маріупольці намагалися обходити
будівлю стороною. Старожили розповідали, що незабаром після того, як гітлерівці
окупували наше місто, з підвалу НКВС були витягнуті тіла розстріляних людей.
Трупи розклали на тротуарі вздовж вулиці Карла Маркса (нині – Грецької). До
цього страшного місця зігнали маріупольців. Німецький офіцер виголосив коротку
промову. Людина в цивільному переклала її зміст і повідомила, що це жертви
більшовицького терору... і що німці прийшли у місто нібито з метою врятування
всіх від знищення... Незабаром у цю будівлю вселилося СД-гестапо (Служба безпеки
– таємна державна поліція). Будівля під час окупації стала центром окупаційної
влади. І знову маріупольці обходили будинок, адже нацисти піддавали місцевих
мешканців тортурам не меншим, ніж НКВС.

На початку вересня 1943 року, відступаючі, німці спалили будинок, як і багато
інших будівель Маріуполя. Проте на початку 1950-х років будинок був
відновлений. Саме тоді було добудовано третій поверх. Можливо, це зробив Микола
Йосипович Нікаро-Карпенко, обдарований архітектор-художник. Два верхні поверхи
були відведені під житло, а на першому поверсі розмістився магазин з продажу
взуття. Молоді маріупольці швидко забули страшні події окупації і саме біля
цього будинку почали призначати побачення своїм дівчатам. Вислів: \enquote{Де
зустрінемося?} – \enquote{У взуття}, - набуває популярності у місті. Хто б міг подумати,
що до всієї насиченої біографії будинку додався ще й елемент романтики... 

\ii{03_10_2017.stz.news.ua.mrpl_city.1.budynok_hameleon.pic.3}

Сьогодні будівля продовжує дивувати неординарністю та унікальністю. Адже тепер
у ній розташований найбільш культовий заклад для інтелігенції та
інтелектуальної молоді – кафе \enquote{IZBA-ЧИТАЛЬНЯ}. Відкрив її Богдан Чабан,
вимушений переселенець з Донецьку, ветеран АТО (воював у найбільш гарячих
точках на Донбасі, в тому числі був під Іловайськом і брав участь в операції зі
звільнення Пісок). Після повернення з війни Богдана не полишала ідея перевезти
улюблений заклад до Маріуполя, що йому вдалося зробити завдяки отриманню гранту
від програми ООН на розвиток власного бізнесу. Кафе відкрили за 4 місяці, чому
посприяла колосальна підтримка від багатьох міст України. Для кафе зібрали 450
кг книг. Багато було зроблено власними руками, але це додало закладу ще більшої
креативності і самобутності. Тепер більш атмосферного і затишного місця,
прикрашеного книгами, у Маріуполі не знайти.

Напевно, у будинку по пр. Миру, 40 перетнулися всі енергетичні потоки і він
дійсно схожий на хамелеона, який вміло обирає потрібну маску. Маріупольський
будинок-хамелеон ніби притягує до себе найбільш яскраві і неординарні установи,
події, людей. Інколи він обирає для себе нові ролі вимушено, а інколи немов сам
хоче знову опинитися у вирі подій. А сама по собі будівля приємна поєднанням
пілястр з фронтонами. Тут зустрічаються бароко та сталінський ампір.

\ii{03_10_2017.stz.news.ua.mrpl_city.1.budynok_hameleon.pic.4}
\ii{03_10_2017.stz.news.ua.mrpl_city.1.budynok_hameleon.pic.5}
\ii{03_10_2017.stz.news.ua.mrpl_city.1.budynok_hameleon.pic.6}
\ii{03_10_2017.stz.news.ua.mrpl_city.1.budynok_hameleon.pic.7}
