% vim: keymap=russian-jcukenwin
%%beginhead 
 
%%file slova.nato
%%parent slova
 
%%url 
 
%%author 
%%author_id 
%%author_url 
 
%%tags 
%%title 
 
%%endhead 
\chapter{НАТО}

\enquote{Слуга народа} Безуглая назвала провокационным вопрос о количестве стран в \emph{НАТО},
Елена Вьюн, strana.ua, 02.06.2021

В ходе программы \enquote{Хард с Влащенко} телеведущая Наталья Влащенко
спросила у Безуглой, сколько стран входят в \emph{НАТО}.  \enquote{Смотрите, я вообще
сразу по эфиру и по нашей беседе прошу не задавать конкретных вопросов, потому
как мы разговариваем про то, что происходит}, - ответила нардеп.  После чего
Влащенко добавила, \enquote{Ну мы же про \emph{НАТО} сейчас говорим?}.  В ответ на это
Безуглая заявила, что такие вопросы являются провокационными,
\textbf{\enquote{Слуга народа} Безуглая назвала провокационным вопрос о количестве стран в НАТО},
Елена Вьюн, strana.ua, 02.06.2021

В \emph{НАТО} входит 30 государств: Албания, США, Бельгия, Болгария, Эстония, Испания,
Голландия, Хорватия, Исландия, Италия, Канада, Греция, Литва, Люксембург,
Латвия, Норвегия, Польша, Португалия, Франция, Румыния, Германия, Cеверная
Mакедония, Словакия, Словения, Великобритания, Дания, Чехия, Турция, Венгрия,
Черногория,
\textbf{\enquote{Слуга народа} Безуглая назвала провокационным вопрос о количестве стран в НАТО},
Елена Вьюн, strana.ua, 02.06.2021

\emph{НАТО} – не защищает, Евросоюз – сыпется, но есть план Б. Что заявил Зеленский в интервью немецкой газете,
Максим Минин, strana.ua, 01.06.2021

Ех, не вміє молоде покоління політиків тримати удар. Згадується, свого часу в
розпал дискусії про те, чи варто Україні вступати у \emph{Північно-Атлантичний
Альянс}, міністр оборони Анатолій Гриценко запитав у Леоніда Кравчука: - А Ви
можете сказати, скільки держав входять в \emph{НАТО}?  Леонід Макарович миттєво
відповів питанням на питання: - А Ви можете назвати основні засади
діалектичного матеріалізму?,
\textbf{Молодым политикам стоит поучиться держать удар у Леонида Кравчука}, Константин Бондаренко, strana.ua, 02.06.2021

Несмотря на протесты Польши, в преамбуле текста единой конституции Евросоюза,
который удалось согласовать минувшей ночью в Брюсселе, не будет упоминания о
христианских корнях вошедших в него государств. Зато там будет упоминание о
том, что НАТО было и останется «фундаментом европейской безопасности»,
\textbf{Европа сносит свою идентичность}, Мак Сим, zen.yandex.ru, 04.06.2021

\emph{НАТО} или не \emph{НАТО}.  Последние дни все больше разговоров об
Альянсе, как Украина туда хочет вступить и почему нас там хотят/не хотят
видеть.  Но наблюдая за дискуссиями в публичной плоскости наших политиков,
складывается впечатления, что большинство в душе не ..ведают, что это вообще за
организация, когда она была создана, с какой целью и главное, чем она вообще
занимается?  Напишу-ка я вам, парочку интересных фактов о \emph{НАТО}.  НАТО
или Организация Североатлантического договора (\emph{Североатлантический
Альянс}) — это военно-политический блок, объединяющий большинство стран Европы
(включая Турцию), Соединённые Штаты Америки и Канаду.  Дата основания - 4
апреля 1949 года в США.  Начиная с даты своего основания, \emph{НАТО} была
ориентирована на противодействие СССР и, позднее, также странам — участницам
возглавлявшегося СССР Варшавского договора, заключённого в 1955 году.  По
словам первого генерального секретаря Исмэя Гастингса, цель создания
\emph{НАТО} звучит так: «Держать Советский Союз вне (Европы), американцев —
внутри, а немцев — в подчиненном положении» За время существования, в
организации было 8 расширений числа стран участниц и сегодня их насчитывается -
30 государств,
\citTitle{Хорошо ли мы представляем себе, что это такое - НАТО?}, Василий Апасов, strana.ua, 08.06.2021

Этим власть говорит нам, что война не кончится никогда.  Стратегией
нацбезопасности и планом завершения войны власть называет вступление в
\emph{НАТО} - изначально нереальный сценарий. Этим власть говорит нам, что
война не закончится никогда, а сама власть планирует перекладывать
ответственность за войну на государства \emph{НАТО}, не желающие видеть Украину
в составе Альянса.  Стратегия, построенная на сослагательном наклонении, на
\enquote{если БЫ}.  Государству с такой стратегией вряд ли удастся выжить,
необходимо создать новую. Но опять же, у нас отвечает за это Данилов. Круг
замкнулся,
\citTitle{Если план окончания войны - вступление в НАТО, то это нереальный план}, Елена Дьяченко, strana.ua, 08.06.2021

Даже в пьяном дебоше солдат \emph{НАТО} они видят руку Путина.  В Эстонии
крепко \emph{пьяные НАТОвские солдаты} устроили дебош и в агрессивной форме
унижали местных жителей.  В ответ на данную ситуацию «британские эксперты» на
полном серьезе заявили, что поют алкоголем \emph{НАТОвских вояк} агенты Путина
для дискредитации военно-политического блока \emph{НАТО}))) Видимо их покусали
«украинские эксперты», ведь симптомы налицо, нужна срочная госпитализация в
психдиспансер и курс лечения от русофобии))),
\citTitle{Британские эксперты обвинили Путина в пьяном дебоше НАТОвских солдат}, Дмитрий Василец, strana.ua, 09.06.2021

\enquote{Вы не раз давали понять, что Украина в \emph{НАТО} - это красная линия. Что вы думаете об этом сейчас?} - спросил журналист. 
\enquote{Я так не говорил}, - ответил Путин. 
\enquote{Но давали понять}, - сказал журналист. 
\enquote{Каждый понимает, как считает нужным. Но что касается расширения \emph{НАТО},
приближения инфраструктуры к российским границам - чрезвычайно важная вещь для
безопасности россиян. Вот смотрите. Было две волны расширения \emph{НАТО} на Восток.
И, заметьте, тогда не было Крыма, не было никаких событий после госпереворота в
Украине. Отношения между Россией и Соединенными Штатами, отношения между
Россией и коллективным Западом тогда были ровными, вполне приемлемыми, если не
сказать партнерскими. И тем не менее все наши озабоченности были
проигнорированы. Все предварительные договоренности - правда, они носили устный
характер с Михаилом Сергеевичем Горбачевым, - приняли забвение. Не хочется
употреблять грубые слова, но просто наплевали на наши интересы}, - начал Путин,
\citTitle{Путин интервью – о вступлении Украины в НАТО, коренных народах, встрече с Зеленским. Главное}, 
Екатерина Терехова, strana.ua, 10.06.2021

%%%cit
%%%cit_pic
%%%cit_text
Вот интересно, сколько ещё раз Зеленскому нужно услышать от лидеров \emph{НАТО}, что Украине \emph{НАТО} не светит, так как:

\begin{itemize}
\item -отсутствует поддержки членства Украины в \emph{НАТО} у участников Альянса;
\item -вопрос членства в \emph{НАТО} раскалывает украинцев;
\item -у Украины двусторонние конфликты с членами \emph{НАТО}, например, с Венгрией;
\item -существует тотальное недоверие членов \emph{НАТО} к украинской власти;
\item -сложное международное положение никак не способствует расширению Альянса;
\item -лидеры Альянса осознают вредоносность присоединения Украины для самого НАТО;
\end{itemize}

Сколько еще раз нужно объяснить Зеленскому и Кулебе, что в 2008 не было РЕШЕНИЯ
о предоставлении ПДЧ для Украины, а была принята ДЕКЛАРАЦИЯ с расплывчатыми
«приветствуем стремление» и «ПДЧ будет следующим шагом» и «поддерживаем заявки» 
%%%cit_title
\citTitle{Сколько еще раз лидеры НАТО должны послать Зеленского?}, Елена Лукаш, strana.ua, 11.06.2021
%%%endcit

%%%cit
%%%cit_pic
\ifcmt
  pic https://avatars.mds.yandex.net/get-zen_doc/1936915/pub_60bf450a2b92e05b9bce0bd1_60bf482ffaeaaf49e4143718/scale_1200
  caption За секунду до начала \enquote{неуставных} отношений. ВСУ
\fi
%%%cit_text
Не могу понять, кто больше хочет, Украина в \emph{НАТО}, или доблестные солдаты \emph{НАТО}
изо всех сил пытаются соответствовать солдатам Вооруженных сил Украины (ВСУ).
Перед очередной просьбой Зеленского принять, наконец-то, незалежную в \emph{НАТО}, в
группировке ВСУ на Донбассе, так сказать, в самых передовых частях армии,
провели исследование состояния морально-психологического духа солдат. Не
знаю, уж, как они это проводили, психологи общались с личным составом, или
анонимно тестировали, или офицеры сами заполняли бумажки. Но результаты,
которые предоставили заместителю командующего группировкой ВСУ на Донбассе
генералу-майору Кидоня, скажем так, совсем неутешительные.  Можно смело
сравнить нынешнюю армию Украины с повстанческой армией имени батьки Махно,
потому что украинские солдатики сегодня живут под гордым девизом: Анархия –
мать порядка
%%%cit_title
\citTitle{Украинская армия \enquote{пьет и воюет} по натовским стандартам. А
\enquote{Боширов и Петров}, возможно, разлагают британских солдат в Эстонии}, 
Моя Поляна, zen.yandex.ru, 08.06.2021
%%%endcit

%%%cit
%%%cit_pic
%%%cit_text
Добрий вечір, малятка. Любі хлопчики та дівчатка. Президент Владимир Зеленский
в день саммита \emph{НАТО} отметил, что Украина сделала все для получения Плана
действий по членству в \emph{альянсе}. Теперь настало время конкретных решений от
\emph{НАТО}. Сам \emph{альянс} отделался дежурными вежливыми фразами. Саммит \emph{НАТО}, который
сегодня прошел в Брюсселе, подтверждает решение Бухарестского саммита 2008 о
намерении в будущем принять Украину. При этом Киеву посоветовали
сосредоточиться на реформах. В общем, по ходу, это затянувшееся «нет»
%%%cit_comment
%%%cit_title
\citTitle{Минимальная заболеваемость, ежегодная вакцинация, дорожающий бензин...}, 
Денис Безлюдько, strana.ua, 14.06.2021
%%%endcit

