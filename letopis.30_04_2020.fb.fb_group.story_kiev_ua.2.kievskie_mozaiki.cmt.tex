% vim: keymap=russian-jcukenwin
%%beginhead 
 
%%file 30_04_2020.fb.fb_group.story_kiev_ua.2.kievskie_mozaiki.cmt
%%parent 30_04_2020.fb.fb_group.story_kiev_ua.2.kievskie_mozaiki
 
%%url 
 
%%author_id 
%%date 
 
%%tags 
%%title 
 
%%endhead 
\subsubsection{Коментарі}

\begin{itemize} % {
\iusr{Дмитрий Евграфов}
В ЖЖ был один фан, который сфоткал, классифицировал и атрибутировал все мозаичные пано Киева.

\begin{itemize} % {
\iusr{Приходько Володимир}
Дан Приходько?

\iusr{Ирина Петрова}
\textbf{Приходько Володимир} дуже раділа б доповненням! Якщо є - будь ласока!

\iusr{Приходько Володимир}
\textbf{Ирина Петрова} Спрошу у сына.

\iusr{Ирина Петрова}
\textbf{Дмитрий Евграфов} ой-ой, але в групі заборонені посилання на сторонні ресурси. Я зараз для себе це збережу, дуже цікавлюсь такими речами. Дякую!! але зараз адміни попросять видалити посилання...

\iusr{Дмитрий Евграфов}
\textbf{Ирина Петрова} В правилах группы не обнаружил запрета на ссылки.

\iusr{Ирина Петрова}
\textbf{Дмитрий Евграфов} можна запитати у адміна, мені зараз з телефону важко шукати...

\iusr{Дмитрий Евграфов}
\textbf{Ирина Петрова} уже нашел. Странно, это. Ссылки информативные, не рекламные. Зачем?

\iusr{Ирина Петрова}
\textbf{Дмитрий Евграфов} такі правила. Ми всі намагаємось їх додержуватись. В цій групі такі, в іншій - інші. Це таке життя в неті. Приймаємо або ні. Вибір кожного. Я зберегла Ваше посилання, буду вивчати. Дякую.

\iusr{Дмитрий Евграфов}
\textbf{Ирина Петрова} Ссылки на Ютуб, или например на городской портал Киева тоже запрещены?

\iusr{Ирина Петрова}
\textbf{Дмитрий Евграфов} так, на всі сторонні ресурси.
\end{itemize} % }

\iusr{Ирина Петрова}

\ifcmt
  ig https://scontent-frx5-1.xx.fbcdn.net/v/t1.6435-9/95215855_3174597435907175_3928497459122143232_n.jpg?_nc_cat=111&ccb=1-5&_nc_sid=dbeb18&_nc_ohc=TjU5hUmUpWIAX9n24Ws&_nc_ht=scontent-frx5-1.xx&oh=a817500f4d99c0258a1c5095f911f9fc&oe=61B4709E
  @width 0.4
\fi

\iusr{Ирина Петрова}
Пробачте, пропустила панно автовокзалу - наша центральна площа

\iusr{Ирина Петрова}

Тільки що в особистих мені написали, що не дуже розуміють, що таке
"хрущовське..."? Може, й справді не дуже зрозуміло? Це гра слів - ампір -
вампір, барокко - барако. Кияни - вони ж на дотепи такі швидкі!))) Іноді
дивуєшся, як за мить народжуються в мережі меми та фотожаби! Кияни - ви
найкрутіші та найдотепніші!

\iusr{Ольга Склярук}
Моя родная \textbf{\#школа168} Оболонь

\ifcmt
  ig https://scontent-frx5-1.xx.fbcdn.net/v/t1.6435-9/95663820_3193804604011938_2758023935005032448_n.jpg?_nc_cat=111&ccb=1-5&_nc_sid=dbeb18&_nc_ohc=ExYHOy8PnM4AX9zug1e&_nc_ht=scontent-frx5-1.xx&oh=7609554971a50ac8b33856fa382576bc&oe=61B3B3AE
  @width 0.4
	@wrap \parpic[r]
  %@wrap \InsertBoxR{0}
\fi

\begin{itemize} % {
\iusr{Ирина Петрова}
Дуже дякую за доповнення. Може, колись хтось зробить повний альбом чудових київських мозаік? бо немає нічого вічного...

\iusr{Ольга Склярук}
\textbf{Ирина Петрова} этой мозаике 40 лет с хвостиком. Удивительно, что она сохранилась.

\iusr{Ирина Петрова}
Робили якісно)

\iusr{Ольга Склярук}
Видно как ее по- тихоньку дети колупают. Сама такой была @igg{fbicon.beaming.face.smiling.eyes} 
\end{itemize} % }

\iusr{Наташа Белоус}
Колись був кінотеатр Аврора. Була мазайка на фасаді і де вона зараз?

\begin{itemize} % {
\iusr{Ирина Петрова}
так, це не єдина вже втрачена мозаіка. Ще була на фасаді кінотеатру "Україна"...ніщо не вічно...

\iusr{Наташа Белоус}
\textbf{Ирина Петрова} сумно а чому не зберннти

\iusr{Дмитрий Евграфов}
\textbf{Наташа Белоус} этот кинотеатр снесли за один день, в месте с мозаикой, увы...

\iusr{Ирина Петрова}
Хто зна, таких питань купа...

\iusr{Наташа Белоус}
\textbf{Дмитрий Евграфов} я знаю в этом и печаль закон дуракам не писан

\iusr{Дмитрий Евграфов}
\textbf{Наташа Белоус} в духе современных десоветизаторов и оставшиеся тоже демонтируют.

\iusr{Наташа Белоус}
\textbf{Дмитрий Евграфов} можно было и музей собрать даже под открытым небом

\iusr{Дмитрий Евграфов}
\textbf{Наташа Белоус} конечно. Правда трудоёмко это - аккуратно демонтировать мозайку и а так же выложить в другом месте.

\iusr{Наташа Белоус}
\textbf{Дмитрий Евграфов} обидно. В Киеве и не только сколько уничтоженно за последнее время
\end{itemize} % }

\iusr{Дмитрий Евграфов}

Многие мозайки известны только местным. Например, детская поликлиника на
Троещине (пр. Маяковского, 18-А)

\ifcmt
  ig https://scontent-frx5-1.xx.fbcdn.net/v/t1.6435-9/95357115_2794952813950397_8473205756080947200_n.jpg?_nc_cat=110&ccb=1-5&_nc_sid=dbeb18&_nc_ohc=FhiHKxLFUmYAX8SgAlf&_nc_ht=scontent-frx5-1.xx&oh=b936b3b1a0daa91a9aa788f31960a377&oe=61B6C1BD
  @width 0.4
\fi

\iusr{Ирина Петрова}

Ось результати кропіткої роботи учасників групи КИ по відновленню мозаіки
Український танок. Дякую, друзі! Ви такі круті! Це вашими зусиллями, напрекір
усьому, мозаіка у своєму оригінальному вигляді! Пишаюся вами!

\ifcmt
  ig https://scontent-frx5-1.xx.fbcdn.net/v/t1.6435-9/95320604_3174750292558556_4227864092766896128_n.jpg?_nc_cat=111&ccb=1-5&_nc_sid=dbeb18&_nc_ohc=qtK_CU_T59MAX8khbm9&_nc_ht=scontent-frx5-1.xx&oh=780eb0ee29b6209dd763467ff632938c&oe=61B5B840
  @width 0.4
\fi

\begin{itemize} % {
\iusr{Людмила Зилинская}
\textbf{Ирина Петрова} Обожаю эту мозаику, любуюсь! Спасибо Вам!

\iusr{Ирина Петрова}
\textbf{Людмила Зилинская} 

я, к сожалению, именно к этой огромной работе группы непричастна, потому что
стала участником группы, когда было уже все сделано. Это была встреча на
2-летие Киевских историй, и наш админ Олег Коваль тогда нам рассказывал,
сколько усилий пришлось вложить в восстановление, и показал фотографию
сделанной реставрации. Спасибо всем им, кто принимал участие. Более подробно об
этом есть публикация Олега в группе.

\iusr{Людмила Зилинская}
\textbf{Ирина Петрова} Спасибо и Вам, сердце радуется!
\end{itemize} % }

\iusr{Неонила Сваток}
Хорошо сохранилась мозаика на эстакаде возле одного из корпусов института кардиологии им. Стражеско

\iusr{Віка Виктория}

Хорошо бы, если бы всем сохранившимся киевским мозаикам присвоили бы статус
памятников местного значения. Чтобы их не уродовали утеплениями квартир и
кондиционерами...

\iusr{Maryna Chemerys}
Мозаїка "Кий, Щек , Хорив" на фасаді Театру ляльок (кол. к/т Ровесник), автор Григорій Довженко

\ifcmt
  ig https://scontent-frt3-1.xx.fbcdn.net/v/t1.6435-9/95673493_2938297006236310_3614100480050855936_n.jpg?_nc_cat=106&ccb=1-5&_nc_sid=dbeb18&_nc_ohc=kZsYVeq7PZwAX-oE4YX&_nc_ht=scontent-frt3-1.xx&oh=03498c27724adf19dac26084691a808c&oe=61B44D54
  @width 0.4
\fi

\iusr{Maryna Chemerys}
Мозаїки Миколи Стороженка, вулиця Андрія Малишка

\ifcmt
  ig https://scontent-frt3-2.xx.fbcdn.net/v/t1.6435-9/95329765_2938303752902302_5644615482632830976_n.jpg?_nc_cat=101&ccb=1-5&_nc_sid=dbeb18&_nc_ohc=GseUpTvLWD0AX9NGKwd&_nc_ht=scontent-frt3-2.xx&oh=d40b37ddcb3f2a0e58401e449184f687&oe=61B4E2C6
  @width 0.4
\fi

\iusr{Андрей Недзельницкий}

а можно как0то добраться до панно в "тарелке" на Лыбедской? видел какие0то
фрагменты на общих фото в кинотеатре, который там. ее то разберут, может кто-то
поснимает

\iusr{Ирина Ирина}

\ifcmt
  ig https://scontent-frt3-1.xx.fbcdn.net/v/t1.6435-9/95683841_1130787460601182_1967818483039207424_n.jpg?_nc_cat=104&ccb=1-5&_nc_sid=dbeb18&_nc_ohc=SWpJcc-x1A8AX8-1QPD&_nc_ht=scontent-frt3-1.xx&oh=1607cd8aa8cebae39f87a40d1108618a&oe=61B40E0F
  @width 0.4
\fi

\iusr{Natasha Levitskaya}

Спасибо! Очень интересно! Какие замечательные работы - такое украшение города!
Жаль, что много потеряшек...

\iusr{Татьяна Дзюба}
Беда!

\ifcmt
  ig https://scontent-frt3-2.xx.fbcdn.net/v/t1.6435-9/95616692_691284738341667_2582099325781803008_n.jpg?_nc_cat=101&ccb=1-5&_nc_sid=dbeb18&_nc_ohc=ElpLqRjkDfMAX_OfTaR&_nc_ht=scontent-frt3-2.xx&oh=c0649eec221606e189adcf8dbacad896&oe=61B438CC
  @width 0.4
\fi

\iusr{Татьяна Дзюба}

\ifcmt
  ig https://scontent-frt3-1.xx.fbcdn.net/v/t1.6435-9/95593112_691285195008288_1780584189838491648_n.jpg?_nc_cat=107&ccb=1-5&_nc_sid=dbeb18&_nc_ohc=MrwOVzgsiYgAX9Rd3YJ&_nc_ht=scontent-frt3-1.xx&oh=192906990995dc96eb1cbc0bb5149004&oe=61B6CFFC
  @width 0.4
\fi

\iusr{Татьяна Дзюба}
А что делать с этим?

\iusr{Татьяна Дзюба}

\ifcmt
  ig https://scontent-frx5-2.xx.fbcdn.net/v/t1.6435-9/95833076_691285631674911_9125328731034877952_n.jpg?_nc_cat=109&ccb=1-5&_nc_sid=dbeb18&_nc_ohc=BXcQancjDKgAX8UALJ3&_nc_ht=scontent-frx5-2.xx&oh=cb3615f60640bae673c45a987dbc19ee&oe=61B70A13
  @width 0.4
\fi

\begin{itemize} % {
\iusr{Ирина Петрова}
\textbf{Татьяна Дзюба} 

це жах!!!! Немає законодавчих підстав вгомонити таких дурнів(((( не знаю, чи є
їхньою власністю стіна будинку? Це просто жах жахливий(((( може, стоїть ширше
постити в неті такі світлини з указанням адреси? Щоб хоча б сором збудив душу
таких дурнів? Хоча, сором... це вже анахронізм((((

\end{itemize} % }

\iusr{Алиса Святына}
Дарница

\ifcmt
  ig https://scontent-frx5-2.xx.fbcdn.net/v/t1.6435-9/95359425_10220141324497253_430320212269596672_n.jpg?_nc_cat=109&ccb=1-5&_nc_sid=dbeb18&_nc_ohc=lwt9_oJ8hdIAX8KXR-Q&_nc_ht=scontent-frx5-2.xx&oh=e256397c3336674a22cdaa77db58747d&oe=61B74B35
  @width 0.4
\fi

\iusr{Лена Примаченко}

У нас тоже сохранилась на бывшем магазине... Макаров Киевская обл. но многим не
нравится почемуто(((

\begin{itemize} % {
\iusr{Алиса Святына}
\textbf{Лена Примаченко} сфотографируйте и покажите нам ) интересно

\iusr{Ирина Петрова}
да, да, интересно бі посмотреть!

\iusr{Лена Примаченко}
Я так и думала, вечером буду в центре и сделаю фото))

\iusr{Лена Примаченко}

\ifcmt
  ig https://scontent-frt3-1.xx.fbcdn.net/v/t1.6435-9/95634308_285786462435250_6152322905268027392_n.jpg?_nc_cat=106&ccb=1-5&_nc_sid=dbeb18&_nc_ohc=qvKOeW576ucAX9zdFdi&_nc_ht=scontent-frt3-1.xx&oh=78f124c13d870f8a979ebfb06b60af81&oe=61B499A7
  @width 0.4

	ig https://scontent-frt3-1.xx.fbcdn.net/v/t1.6435-9/95666461_285786595768570_1887486312382988288_n.jpg?_nc_cat=102&ccb=1-5&_nc_sid=dbeb18&_nc_ohc=KrI3TAf4N6MAX9JX2xt&_nc_ht=scontent-frt3-1.xx&oh=bd0a717ee461e259da5fc67fe729fb2b&oe=61B777F5
  @width 0.4
\fi

\iusr{Лена Примаченко}
Вчера не дошла до центра, сегодня прогулялась))) Мозайки, одна напротив другой ...

\end{itemize} % }

\iusr{Ирина Петрова}

Хочу висловити велику подяку учасниці нашої групи з ніком Nina NinaNina за
уточнення, пояснення, інформацію до моїх фотографій в цій темі. дуже приємно
бачити, що є багато людей, які знають, люблять, вивачють історію нашого Міста!
Пані Ніно, щира подяка!

\iusr{Галина Кошманенко}

Було ще мозаїчне панно на фасаді будівлі закритого тенісного корту на території
НСК "Олімпійський" - чоловік і жінка спортивної статури й напис: "Прудкі, як
лань, міцні, як кінь, у сяйві сонячних промінь". Не знаю, хто автор. Здається
той будинок знесли. А шкода.

\iusr{Богдан Черненко}
снесенный в 2005 кинотеатр Аврора, мозайка 1966.

\ifcmt
  ig https://scontent-frx5-1.xx.fbcdn.net/v/t1.6435-9/95680900_2457081557726358_9014399904851689472_n.jpg?_nc_cat=111&ccb=1-5&_nc_sid=dbeb18&_nc_ohc=y3wa5hmmOZoAX9KE59v&_nc_ht=scontent-frx5-1.xx&oh=2a56e05ff386fffd6ceb5671e26654d4&oe=61B48E28
  @width 0.3
\fi

\begin{itemize} % {
\iusr{Ирина Петрова}
\textbf{Bogdan Kievlianin} где-то в теме как раз интересовались именно этой мозаикой. Надеюсь, увидят. Спасибо!!!
\end{itemize} % }

\iusr{Виктор Бухтияров}

Фонтан біля Палацу дітей та юнацтва повністю у цьому році відреставрували. Він
працює. Флагшток поновили. Зараз це найвищий в Києві (а, може, й в Україні)
флагшток.

\end{itemize} % }
