% vim: keymap=russian-jcukenwin
%%beginhead 
 
%%file 26_11_2021.fb.lukjanenko_sergej.pisatel.1.zolotoj_vek.cmt
%%parent 26_11_2021.fb.lukjanenko_sergej.pisatel.1.zolotoj_vek
 
%%url 
 
%%author_id 
%%date 
 
%%tags 
%%title 
 
%%endhead 
\subsubsection{Коментарі}

\begin{itemize} % {
\iusr{Yuri Mori}
Всё так. Но почему это сопровождается такой ненавистью между людьми?

\begin{itemize} % {
\iusr{Андрей Потапов}
\textbf{Yuri Mori} потому что это в нашей природе. Увы

\iusr{Yuri Mori}
\textbf{Андрей Потапов} в нашей природе ещё много чего есть, мне кажется. Любовь и красота. Созидание и мечты.

\iusr{Сергей Лукьяненко}
\textbf{Yuri Mori} Потому что когда мир ломается - люди ломаются вместе с ним.

\iusr{Андрей Потапов}
\textbf{Yuri Mori} в пирамиде потребностей это более высокие уровни. Если нижние не обеспечены, об этих можно забыть. Я раньше считал это циничной позицией, но потом стоило понаблюдать - и я понял...

\iusr{Yuri Mori}
\textbf{Сергей Лукьяненко} люди всегда будут разными. Кому охота сломаться и оскотиниться - тому никаких перемен вокруг не нужно. А кто хочет оставаться человеком в любых условиях, тот и останется.

\iusr{Oleg Badin}
Может, наступил Платиновый Век? Платина металл не такой эффектный, как золото... Но более технологичный... Как-то так.

\iusr{Андрей Потапов}
\textbf{Oleg Badin} не вижу резона в технологиях, когда люди злы друг к другу.

\iusr{Евгений Рахимжанов}
Резон есть. Сейчас благодаря глобальным коммуникациям все добрые люди могут собраться вместе и замочить всех злых)))

\iusr{Алекс Ломанов}
\textbf{Сергей Лукьяненко} замечательная фраза. Мир сломали люди, а людей сломал сломанный мир.

\iusr{Сергей Лукьяненко}
\textbf{Yuri Mori} Вначале прочел вместо "мечты" - "менты". Задумался. Даже решил, что понял глубинный смысл. Потом перечитал.

\iusr{Андрей Потапов}
\textbf{Сергей Лукьяненко} вот, кстати, без ментов созидание действительно хромает  @igg{fbicon.smile} 

\iusr{Alexandr Maltsev}
\textbf{Yuri Mori}, врядли "сломанные и оскотиневшиеся " люди (ну, те - кому никаких перемен для этого не нужно) себя таковыми считают. Разность людей штука субьективная, для вас кто-то плохой, а для кого- то плохой вы.

\iusr{Михаил Герасимов}
\textbf{Сергей Лукьяненко} сначала ломаются люди, а потом - мир.

\iusr{Лидия Афанасьева}
\textbf{Yuri Mori} те, кто испытывает ненависть будут вымирать. Выживут те, кто поймёт, что люди едины на уровне ноосферы - о чем говорил ещё академик Вернадский
И научатся жить и принимать решения, исходя из этого единства.

\iusr{Константин Бойдало}
\textbf{Евгений Рахимжанов} коммуникации всегда объединяют злых.

\iusr{Татьяна Глущенко}
\textbf{Yuri Mori} нет ничего больнее, чем терять иллюзии
\end{itemize} % }

\iusr{Владимир Перемолотов}

Да. Золотой Век это всегда надежда на что-то лучшее, подспудная уверенность,
что реализация надежды дело реальное. А она ушла...


\iusr{Андрей Потапов}

Мы просто ходим от крайности к крайности, как маятник. И всё хорошо, только
когда мы пересекаем золотую середину. Короткий миг счастья перед новым витком
маразма.

\iusr{Федор Бобков}
Удивительно точно вы описали мои ощущения! Магия.

\iusr{Ирина Черкашина}

Теперь я поняла, почему лет двадцать назад постап сделался так популярен...
Перемены рождаются в книгах, а уж потом в реальности  @igg{fbicon.smile} 

А что там сейчас в тренде - попаданцы?

\begin{itemize} % {
\iusr{Андрей Потапов}
\textbf{Ирина Черкашина} если современники начнут попадать в головы мировых лидеров прошлого, всё накроется медным тазом гораздо раньше  @igg{fbicon.face.grinning.sweat} 

\iusr{Георгий Саенко}
\textbf{Андрей Потапов} может они уже попали, кхм.

\iusr{Sergey Mineev}
\textbf{Андрей Потапов} В Брежнева вот 200 попыток было проникновения попаданцев, но он пил лекарство от давления, а оно не даёт вселиться попанданцам)

\iusr{Андрей Потапов}
Sergey Mineev "Настоящее попадалово" (с)  @igg{fbicon.smile} 

\end{itemize} % }


\end{itemize} % }
