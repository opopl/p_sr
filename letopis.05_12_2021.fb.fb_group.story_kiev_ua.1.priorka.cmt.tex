% vim: keymap=russian-jcukenwin
%%beginhead 
 
%%file 05_12_2021.fb.fb_group.story_kiev_ua.1.priorka.cmt
%%parent 05_12_2021.fb.fb_group.story_kiev_ua.1.priorka
 
%%url 
 
%%author_id 
%%date 
 
%%tags 
%%title 
 
%%endhead 
\subsubsection{Коментарі}

\begin{itemize} % {
\iusr{Yuriy Bubnov}
Фотографировал в 1986 году. Купола были ещё зелёные...

\ifcmt
  ig https://scontent-frx5-2.xx.fbcdn.net/v/t39.30808-6/263849795_1068749230583857_8920962352727047775_n.jpg?_nc_cat=109&ccb=1-5&_nc_sid=dbeb18&_nc_ohc=1rGdjQakWjIAX_nC-U8&_nc_ht=scontent-frx5-2.xx&oh=56d5a4a6d5d38298a6c75bb4efd463d9&oe=61B25036
  @width 0.4
\fi

\begin{itemize} % {
\iusr{Maksym Oleynikov}
\textbf{Yuriy Bubnov} Позолотять у 2006 році.

\iusr{Алла Диба}
\textbf{Yuriy Bubnov} А я собі думаю, чому мені в пам'яті церковні бані "якісь не такі"... Ось воно що. Так запало в душу.
\end{itemize} % }

\iusr{Вячеслав Нестеров}

Якщо придивитись уважно, навіть на цих фото, можна зрозуміти, що з чотирьох
бічних барабанів насправді уцілів лише один, з дійсно цегляними віконцями і
крикрасами, три інші є глухими циліндричними барабанами для симетрії куполів.
Жодна з реконструкцій так і не стосувалася цих елементів покрівлі.  @igg{fbicon.frown} 

\iusr{Людмила Терпелюк}
Дякую. Живу поряд і дуже цікаво було читати про історію Пріорки та Покровськоі церкви.

\iusr{Liudmila Kabanova}

Огромное спасибо! Фото замечательные, тем более мое детство прошло на Приорке,
как раз почти рядом с этим храмом. И мне там все очень близко.


\iusr{Ирина Козина}

насколько я знаю, кладбище вокруг церкви занимало значительно большую
территорию, нежели теперь. но было снесено и на его месте выстроены новые дома

\iusr{Нина Грисевич}
Дуже цікаво! Дякую! Я проживаю в цій мусцевості 43 роки і з задоволенням дізналася про Пріорку.

\iusr{Грег Станлий}

В войну в стенах этой церкви спаслось очень много людей. Следы от осколков до
сих пор хорошо видны на стенах. Очень интересно церковное кладбище.


\iusr{Yuriy Bubnov}
2011 год, июль. Там же.

\ifcmt
  ig https://scontent-frx5-1.xx.fbcdn.net/v/t39.30808-6/264489540_1068823393909774_5401748996469919249_n.jpg?_nc_cat=110&ccb=1-5&_nc_sid=dbeb18&_nc_ohc=7Jnc6rxITtQAX80gqRj&_nc_ht=scontent-frx5-1.xx&oh=3fca0524c03faa8bc02042de039183ac&oe=61B2135D
  @width 0.4
\fi

\iusr{Kate Levenko}

Дякую! Дуже цікаво. Мене хрестили в цій церкві, також у мене там було вінчання,
а моя донька народилась в тому "великому пологовому будинку" і також хрещена в
церкві на Мостицькій. І взагалі з Пріоркою в нашої родини багато пов'язано, хоч
ніколи там і не жили  @igg{fbicon.smile}  Ще тут напишу: місцевість, яка іде від перехрестя вул.
Автозаводської з Полярною і до перехрестя з Луговою, моя бабуся називала
Поперечкою (можливо тому, що там вулиці перехрещуються під прямим кутом з
трамвайною лінією). Деякі старожили ще пам'ятають цю назву, але більшість
жителів вже не розуміють про яке місце іде мова.

\iusr{Елена Фросталь}
Мою дочку крестили в этой церкви

\iusr{Инна Косянчук}
Я також дізналась дещо цікаве про Яцека Одровонжа: свого часу він був канонізований католицькою церквою як св. Гіацинт.

\iusr{Николай Гребенкин}
Прекрасный рассказ. Познавательно.  @igg{fbicon.face.smiling.hearts} Спасибо!

\iusr{Любовь Чернявская}

Когда вокруг стояли частные одноэтажные дома, храм казался больше и
величественнее. Запомнилась дорога, которая вела к нему. На фоне многоэтажек
немного потерялся...


\iusr{Олексій Степаненко}
Дякую, дуже цікаво.

\iusr{Татьяна Оксаненко}
Очень интересно! Спасибо большое!

\iusr{Людмила Джулай}
Дякую...

\iusr{Vika Sinimae}
И меня здесь крестили и дочку мою и внука, мы жили на ул, Мостицкой, а сейчас на Новомостицкой

\end{itemize} % }
