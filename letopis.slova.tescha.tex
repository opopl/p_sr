% vim: keymap=russian-jcukenwin
%%beginhead 
 
%%file slova.tescha
%%parent slova
 
%%url 
 
%%author 
%%author_id 
%%author_url 
 
%%tags 
%%title 
 
%%endhead 
\chapter{Теща}
\label{sec:slova.tescha}

%%%cit
%%%cit_head
%%%cit_pic
%%%cit_text
Самое плохое для Зеленского - имидж человека, воюющего с женщинами.  Акела
промахнулся... сериал продолжается... Обыски в офисе и у \emph{тещи} (?!!!) Медведчука –
это продолжение негативного сценария. Негативного для Зеленского \& Co.  Мы уже
видели, как «Акела промахнулся» один раз. Когда власть выдвигает громкие
обвинения – госизмена и разграбление национальных ресурсов – а потом судья на
заседании на всю страну отчитывает прокурорских за то, что нет доказательств –
это удар по репутации Зеленского \& Co.  Как выглядит вся эта история? Что
думают люди? На пальцах поясняю.  Если есть громкие обвинения, но нет
доказательств – это репрессии
%%%cit_comment
%%%cit_title
\citTitle{Если есть громкие обвинения, но нет доказательств – это репрессии / Лента соцсетей / Страна}, 
Дмитрий Марунич, strana.ua, 25.06.2021
%%%endcit
