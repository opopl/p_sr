% vim: keymap=russian-jcukenwin
%%beginhead 
 
%%file 16_10_2021.fb.fb_group.story_kiev_ua.1.veronika_chernjahivska
%%parent 16_10_2021
 
%%url https://www.facebook.com/groups/story.kiev.ua/posts/1776114439252016
 
%%author_id fb_group.story_kiev_ua,olejnikov_maksim
%%date 
 
%%tags chelovek,chernjahovskaja_veronika,istoria,kiev,kultura,ukraina,zhizn
%%title Цій вродливій і талановитій київській дівчині судилася начебто щаслива доля - Вероніка Черняхівська
 
%%endhead 
 
\subsection{Цій вродливій і талановитій київській дівчині судилася начебто щаслива доля - Вероніка Черняхівська}
\label{sec:16_10_2021.fb.fb_group.story_kiev_ua.1.veronika_chernjahivska}
 
\Purl{https://www.facebook.com/groups/story.kiev.ua/posts/1776114439252016}
\ifcmt
 author_begin
   author_id fb_group.story_kiev_ua,olejnikov_maksim
 author_end
\fi

Цій вродливій і талановитій київській дівчині  судилася начебто щаслива доля.
Ще б пак: онука відомого драматурга, фундатора українського професійного театру
Михайла Старицького, донька відомого лікаря, професора Олександра
Черняхівського і письменниці Людмили Старицької-Черняхівської, племінниця
генерального секретаря освіти УНР Івана Стешенка… Вона ще застала в живих свого
діда, бачила у своєму домі квіт української нації - Миколу Лисенка, Лесю
Українку, Михайла Грушевського, Бориса Грінченка, Михайла Коцюбинського, Івана
Франка… 


\ifcmt
  tab_begin cols=3

     pic https://scontent-mxp1-1.xx.fbcdn.net/v/t1.6435-9/246043116_935350507054652_8944242390179029310_n.jpg?_nc_cat=100&ccb=1-5&_nc_sid=b9115d&_nc_ohc=YCcRgWX7uswAX_AWNOZ&_nc_ht=scontent-mxp1-1.xx&oh=f416d377ceae39e39867a6bd2be59560&oe=6190005C

     pic https://scontent-mxp1-1.xx.fbcdn.net/v/t1.6435-9/246326405_935350663721303_5802760283840068848_n.jpg?_nc_cat=111&ccb=1-5&_nc_sid=b9115d&_nc_ohc=5G9MvG56UKkAX_6At-0&_nc_ht=scontent-mxp1-1.xx&oh=81a77db6e183d50cf2f0414a65bfb031&oe=618F3319

		 pic https://scontent-mxp1-1.xx.fbcdn.net/v/t1.6435-9/245994222_935350703721299_4949770207731205038_n.jpg?_nc_cat=100&ccb=1-5&_nc_sid=b9115d&_nc_ohc=1MGZwCi2m6QAX9A27xs&_nc_ht=scontent-mxp1-1.xx&oh=7d3be6634ac654f1fe1ab2883489038d&oe=618F8D2E

  tab_end
\fi

Але насправді доля їй випала гірка і трагічна.

Вероніка Черняхівська народилася 25 квітня 1900 року в Києві. Здобувши
якісну початкову домашню освіту, дівчинка зі знаної в місті родини вступила
до Київської приватної жіночої гімназії Аделаїди Жекуліної, а коли
відкрились перші українські гімназії, перейшла до Другої української
гімназії імені Кирило-Мефодіївського братства, яку й закінчила 1 червня 1918
року із золотою медаллю. Надалі здобула фах інженера-економіста в Київському
інституті народного господарства, водночас відвідувала курси іноземних мов,
де вивчала грецьку, німецьку, французьку і латинь.

\ifcmt
  tab_begin cols=3

     pic https://scontent-mxp1-1.xx.fbcdn.net/v/t1.6435-9/245906098_935350787054624_7975124785660253668_n.jpg?_nc_cat=109&ccb=1-5&_nc_sid=b9115d&_nc_ohc=Qt8axMXIuc0AX8hBbDX&tn=lCYVFeHcTIAFcAzi&_nc_ht=scontent-mxp1-1.xx&oh=7dc0b2a92e0a14e7e15ee8fa65f06718&oe=6191DB85

     pic https://scontent-mxp1-1.xx.fbcdn.net/v/t1.6435-9/246193055_935350853721284_156850884777495528_n.jpg?_nc_cat=109&ccb=1-5&_nc_sid=b9115d&_nc_ohc=HNih2qQjDTgAX9n8t5R&_nc_ht=scontent-mxp1-1.xx&oh=36e7bf269566794492f6361959f3a923&oe=6191D667

		 pic https://scontent-mxp1-1.xx.fbcdn.net/v/t1.6435-9/246151262_935350917054611_6647360489613186343_n.jpg?_nc_cat=108&ccb=1-5&_nc_sid=b9115d&_nc_ohc=H5Ejt3muoCcAX-eqg9t&_nc_ht=scontent-mxp1-1.xx&oh=ddbbf816eb63da8201b3c7be1ae399da&oe=61906664

  tab_end
\fi

Вероніка була вродливою, тому її руки добивалися багато залицяльників. І хоча
дівчина сама обрала чоловіка, вийшовши заміж за Івана Євлашенка, їхній шлюб був
невдалим. Її творча натура прагнула самостійності і повної самореалізації, тому
невдовзі після весілля розлучилася з чоловіком.

Щоб відволіктися від сімейних негараздів, Черняхівська поринула у заглиблене
вивчення мов, вирішила опанувати англійську, бо у гімназії не вивчала, і
зосередилася на перекладацькій справі. Видавництво «Сяйво» опублікувало її
версію рідною мовою роману Джека Лондона «Місячна долина». Батько Вероніки
влаштував доньку в Народний комісаріат охорони здоров’я
референтом-перекладачем. На цій посаді разом із татом Вероніка виїздить у
відрядження до Іспанії. По дорозі вони зупинилися в Берліні, де Вероніка, поки
батько сидів у лабораторіях, ходила в кіно, театри, на концерти. Там вона
познайомилася з банківським службовцем Теодором Геккеном, а через два роки вони
одружилися.  У шлюбі вони проживуть недовго, але через десять років цей шлюб
для неї виявиться фатальним.


\ifcmt
  tab_begin cols=3

     pic https://scontent-mxp1-1.xx.fbcdn.net/v/t1.6435-9/246151276_935350997054603_9098430341014189763_n.jpg?_nc_cat=108&ccb=1-5&_nc_sid=b9115d&_nc_ohc=2KOvlVp81acAX9I8LaQ&_nc_ht=scontent-mxp1-1.xx&oh=e6d602c69078153203d5946c7b5b6102&oe=6191A748

     pic https://scontent-mxp1-1.xx.fbcdn.net/v/t1.6435-9/246224544_935351070387929_1766481711566324382_n.jpg?_nc_cat=101&ccb=1-5&_nc_sid=b9115d&_nc_ohc=yu3IraHfaw0AX_Wasv4&_nc_ht=scontent-mxp1-1.xx&oh=f9d9cb3a05af8ff52df487d36c184831&oe=6191C4E2

		 pic https://scontent-mxp1-1.xx.fbcdn.net/v/t1.6435-9/246303602_935351157054587_6537215703433858657_n.jpg?_nc_cat=111&ccb=1-5&_nc_sid=b9115d&_nc_ohc=8o0zNqJOHEoAX8_nwSG&_nc_ht=scontent-mxp1-1.xx&oh=3b8d33012469de0cc139e8c82560e169&oe=618ED4DF

  tab_end
\fi

У 1929 році Вероніка повернулася на батьківщину. Державне видавництво України
опублікувало два романи перекладу В.Черняхівської – «Пригоди Олівера Твіста»
Ч.Діккенса та «Прорість (Жерміналь)» Е.Золя. Вона пише ліричні поезії,
друкується в альманахах. Черняхівській належать українські переклади Стендаля,
Мопассана, а також наукової медичної літератури.


%\ifcmt
  %tab_begin cols=2

     %pic 

     %pic <++>

  %tab_end
%\fi

У той період Вероніка подала на розлучення з Теодором. Причиною була  заборона
виїздити за кордон, щоб бачитися з чоловіком. Перед жінкою постала проблема
вибору: лишитися з батьками в Україні і продовжувати свою справу, чи виїхати до
чоловіка в Німеччину і назавжди поховати себе в домашніх клопотах. Черняхівська
обрала перше. 

Вероніка продовжувала плідно працювати над перекладами світових шедеврів, але
над нею, як і над усією тодішньою елітою, вже був занесений каральний меч
сталінізму. 

ОДПУ УРСР наприкінці 1920-х років сфабрикувало справу «СВУ» («Спілки
визволення України»), про вигадану антирадянську організацію серед
української інтелігенції. Тоді під маховик репресій потрапила і Вероніка –
її арештували, але згодом відпустили. Натомість за грати потрапили її
батьки. Мати, яка вже мала 62 роки, стійко витримала 29 принизливих допитів.
Подружжя Черняхівських засудили до 5 років ув’язнення і 3 років поразки у
правах. Але більшовицька влада вирішила продемонструвати свою «гуманність».
Вироки замінили на умовні і на 6 років їх вислали до Сталіно (нинішній
Донецьк). 

За період слідства Вероніку виключили із профспілки, це позбавило її роботи. Не
допомогли ні зв’язки, ні висока кваліфікація перекладача. На роботу не брали,
бо мала тавро від «процесу СВУ». Повернулась до Києва, втретє вийшла заміж – за
математика і астронома на прізвище Ганжа. Шлюб носив скоріше формальний
характер: чоловік жив під Києвом, до дружини приїздив тільки на вихідні, вона ж
і того рідше відвідувала чоловіка. Але в подальшому до кінця свого життя в
документах Вероніка фігурує з прізвищем Черняхівська-Ганжа. 

Є свідчення, що певний час вона працювала в Народному комісаріаті охорони
здоров’я, де перекладала документи, які надходили до міністерства іноземною
мовою. Платили небагато, і Вероніці вистачало тільки на необхідне, та справді
страшні часи ще були попереду.

У січні 1938р.Вероніку знову заарештували. Цього разу звинуватили у шпигунстві
на користь Німеччини. На той час «органи» викрили буцімто «російсько-німецьку
контрреволюційну группу». Серед її членів були і люди, з якими Вероніка не раз
зустрічалась колись на вечірках. На допитах двоє з них «зізналися», що
Черняхівська – німецький резидент та зв’язкова німців з «неблагонадійним
українським елементом». Ця вигадка і лягла в основу звинувачень. До того ж
жінці згадали її колишній шлюб із німецьким підданим, і за цей факт 10-річної
давнини теж вхопилися слідчі. 

Із слідчої справи: 

«3-м Отделом НКВД УССР 9 января сего года была арестована, как подозреваемая в
шпионской деятельности Черняховская-Ганжа Вероника Александровна, до ареста
переводчица с иностранных языков при Наркомздраве. Следственными действиями
установлено: 

Обвиняемая Черняховская-Ганжа является выходцем из украинской
националистической семьи, которая, начиная с 1917 года, то есть с момента
установления Советской власти, группируя вокруг себя украинскую
националистическую молодежь, при этом распространяла контрреволюционные
пораженческие слухи о гибели советской власти и реставрации капитализма…
Обвиняемая Черняховская-Ганжа, будучи недовольна существующим строем, в 1926 г.
выезжала в Берлин, где пробыла до 1928г., откуда возвратилась обратно в СССР и
с того момента и по день ареста была связана с немецким консульством в г.
Киеве, посещала квартиру немецкого консула, где собирались украинские
националисты, и на этих сборищах обсуждались вопросы борьбы с Советской
властью…».

На допитах Вероніка Олександрівна рішуче відкидала всі звинувачення. Почалися
нелюдські тортури, і врешті 26 липня 1938р. Вероніка визнала себе винною. Тоді
Києвом ширилися чутки, що Черняхівську, аби морально зламати і вибити з неї
зізнання, неодноразово гвалтували слідчі, через що вона збожеволіла. 21 вересня
1938 р. особливої «трійка» при Київському облуправлінні НКВС УРСР  засудила
Вероніку Черняхівську до вищої міри покарання з конфіскацією всього особистого
майна. Вирок виконали 22 вересня 1938 р., а наступної ночі, разом із тілами
інших замордованих, вивалили з вантажівки у заздалегідь вириті безіменні
братські могили у Биківнянському лісі під Києвом.

Батькам усно повідомили, що дочку засудили на 10 років таборів без права
листування. 22 грудня 1939 р. її 70-річний батько професор О.Г.Черняхівський
помер із горя. Два наступні роки її 72-річна мати Людмила Михайлівна
Старицька-Черняхівська шукала дочку по сибірських режимних таборах, писала
листи про помилування Вероніки до Хрущова, Молотова, Калініна і самого Сталіна.

20 липня 1941 р., коли під Києвом вже точилися бої з гітлерівцями, Людмилу
Михайлівну Старицьку-Черняхівську заарештували агенти НКВС. Разом із сестрою -
письменницею, перекладачкою і педагогом Оксаною Михайлівною Стешенко, літніх
жінок бортовою вантажівкою вивезли до Харкова, де звинуватили в антирадянській
діяльності. Потім у товарному вагоні для транспортування худоби відправили на
заслання до Акмолинська (нині – Астана), в Північний Казахстан.

Коли дорогою 73-річна Людмила Михайлівна Старицька-Черняхівська померла,
конвоїри просто викинули тіло української письменниці з вагону. Тож точна
дата смерті старшої дочки корифея українського театру Михайла Старицького
невідома. Як і те, чи є десь її могила… Оксана Михайлівна пережила сестру на
рік - доїхала до Казахстану, де померла в таборі у 1942р.

А про трагічну долю Вероніки Олександрівни Черняхівської дізналися лише у
1990р., коли на запит Музею видатних діячів української культури була видана
довідка КДБ УРСР про її реабілітацію...
