% vim: keymap=russian-jcukenwin
%%beginhead 
 
%%file 09_11_2021.fb.berdnik_miroslava.1.dostojevskij_200_let
%%parent 09_11_2021
 
%%url https://www.facebook.com/miroslava.berdnik.1/posts/988966521687806
 
%%author_id berdnik_miroslava
%%date 
 
%%tags 200_let,chelovek,dostojevskii_fedor,kultura,literatura,rossia
%%title 200-летие со дня рождения Достоевского
 
%%endhead 
 
\subsection{200-летие со дня рождения Достоевского}
\label{sec:09_11_2021.fb.berdnik_miroslava.1.dostojevskij_200_let}
 
\Purl{https://www.facebook.com/miroslava.berdnik.1/posts/988966521687806}
\ifcmt
 author_begin
   author_id berdnik_miroslava
 author_end
\fi

В 2021году 11 ноября Россия и весь литературный мир отмечают 200-летие со дня
рождения выдающегося русского писателя Фёдора Михайловича Достоевского
(1821–1881)

Юбилей Достоевского отмечается под эгидой ЮНЕСКО.

Творчество русского гения многогранно. Гениальны его высказывания о будущем
России, о будущем мира. Творчество его многоаспектно, оно неисчерпаемо.

\ifcmt
  ig https://scontent-frt3-2.xx.fbcdn.net/v/t39.30808-6/253748866_988966728354452_214486855758841341_n.jpg?_nc_cat=103&ccb=1-5&_nc_sid=730e14&_nc_ohc=yiW1kowqvRQAX-S-4ev&_nc_ht=scontent-frt3-2.xx&oh=3bb602f993743977fe9c2006fdc6db8a&oe=61AA2722
  @width 0.4
  %@wrap \parpic[r]
  @wrap \InsertBoxR{0}
\fi

Писатель имеет наибольшее представительство во «Всемирной библиотеке» – серии
Норвежского книжного клуба, включающей 100 книг из списка, составленного в 2002
году Норвежским книжным клубом совместно с Норвежским институтом имени Нобеля.

В составлении списка приняли участие 100 писателей из 54 стран мира. Целью
проекта был отбор наиболее значимых произведений мировой литературы. Каждый
писатель, участвовавший в составлении списка, предоставил собственный список из
10 книг. Из русскоязычных писателей в составлении списка участвовали Чингиз
Айтматов (1928–2008), Валентин Распутин (1937–2015) и Александр Ткаченко
(1945–2007).

Книги не разделены по значимости. Организаторы заявили, что «они все равны», за
исключением книги «Хитроумный идальго Дон Кихот Ламанчский» Мигеля де
Сервантеса, которая набрала на 50\% больше голосов, чем любая другая книга.

Во «Всемирную библиотеку» включены четыре романа Достоевского: «Преступление и
наказание», «Идиот», «Бесы», «Братья Карамазовы».

Тремя произведениями представлены Уильям Шекспир, Лев Толстой («Война и мир»,
«Анна Каренина», «Смерть Ивана Ильича») и Франц Кафка.

Из русских писателей, помимо Достоевского и Льва Толстого, во «Всемирной
библиотеке» представлены Николай Гоголь («Мёртвые души») и Антон Чехов
(«Рассказы»).

Юбилей Ф. М. Достоевского отмечается на государственном уровне в соответствии с
указом Президента Российской Федерации от 24 августа 2016 года № 424 «О
праздновании 200-летия со дня рождения Ф. М. Достоевского».
