% vim: keymap=russian-jcukenwin
%%beginhead 
 
%%file 19_09_2022.stz.news.ua.donbas24.1.zrostannja_ruhu_oporu_mrpl_molod_protystoit_okupantam.txt
%%parent 19_09_2022.stz.news.ua.donbas24.1.zrostannja_ruhu_oporu_mrpl_molod_protystoit_okupantam
 
%%url 
 
%%author_id 
%%date 
 
%%tags 
%%title 
 
%%endhead 

Зростання руху опору в Маріуполі — як молодь протистоїть окупантам

У Маріуполі об'єднується молодь, яка чекає на деокупацію та намагається
наблизити перемогу

Напередодні повномасштабного вторгнення росії в Маріуполі 22 лютого відбувся
мітинг «Маріуполь — це Україна», який об'єднав патріотично налаштованих містян
та яскраво проілюстрував настрої громади. Сьогодні в місті й досі перебувають
справжні патріоти, які створили власне антиросійське підпілля. 22 лютого вони
брали участь в мітингу у вільному українському Маріуполі, а сьогодні
намагаються боротися проти окупантів. Серед них багато молоді, адже не всі
містяни змогли виїхати і вони залишилися зі своїми батьками. Наразі юні
патріоти вимушені ховатись та щодня роблять все можливе, щоб населення
Маріуполя не втрачало віри та було правильно проінформоване.

Читайте також: В Маріуполі розгортається рух спротиву

Коли було створено та який склад підпілля? 

Насправді можна вважати, що це один з напрямів маріупольського підпілля, адже
наразі відомо багато випадків, які свідчать, що спротив в Маріуполі все ж існує
і він доволі розгалужений. Зараз же піде мова про молодь міста, яка влітку
вирішила об'єднатися для подальшої діяльності, спрямованої проти окупантів.
Учасниками цього підпілля стали юні маріупольці віком 13−16 років та їхні
батьки. Вони і до війни мали чітку патріотичну позицію, любили історію України,
присвячували пісні, вірші та прозу своїй Батьківщині. До цьогорічного Дня
Незалежності учасники підпілля надсилали своїм патріотично налаштованим
знайомим запис власного виконання Маршу українських націоналістів. Слова:
«Солодше нам у бою умирати, ніж в путах жити, мов німі раби» стали їхнім
девізом. Контрнаступ наших військових та звільнення Харківщини надихнули не
тільки всю Україну, але й маріупольських підпильників. Саме тоді вони вирішили
активізувати діяльність, склали своє перше звернення та поширили, як в
повідомленнях маріупольцям, що виїхали, так і в листівках, написаних від руки.
Щодня вони намагаються збирати інформацію про звільнені міста України та
розповсюджувати її в Маріуполі. До того ж мають власний список колаборантів
міста, з якими ведуть свою особисту боротьбу. Себе вони називають
анархо-націоналістами та сподіваються на якомога швидке звільнення Маріуполя.

Звернення від маріупольських підпільників 

«Маріуполь потерпає від наслідків російської окупації та захоплення. Усюди
воєнний та поліцейській розлад, пригнічення усього народного та українського.
Але, чим сильніша темрява, тим легше помітити в ній промінь світла. Наступ
визволителів на Харківщині наповнив серця людей надією і це допомагає тим, хто
доводить своє прагнення до волі не словом, а ділом. Зараз місцевий таємний
супротив займається розшуком та планами винищення зрадників з поліції та СБУ,
знаходженням колоборантів та розповсюдженням пропаганди волі дії, волі думки,
волі народу. Саме Анархо-Націоналістичний рух вже знає більшість зрадників, цим
щурам далеко не втекти від Правосуддя. Більшість розуміє, що, якщо б московити
не почали напад, наше місто було б цілим і зараз, не було б стільки втрат серед
цивільного населення... За офіційними розрахунками померло більше 10% мирного
населення, ще більше зникли безвісти та все що розшукуються. Проте такі цифри
не відповідають дійсності... Ми все ще не забули про розстріл цивільного
населення, знищення наших будинків та знущання над народом. Страждає не тільки
наше місто, але й Волноваха, Сартана, Широкине. Сила нашого народу — у нашій
волі і ми за цю волю готові боротися. Скоро нас звільнять від окупації, але ми
хочемо прискорити цей процес і щодня робимо для цього все можливе. Слава
Україні та смерть усім хто на перешкоді нашій свободі!»

Нагадаємо, раніше Донбас24 розповідав, як у Маріуполі підірвали окупантів.

Ще більше новин та найактуальніша інформація про Донецьку та Луганську області
в нашому телеграм-каналі Донбас24.

ФОТО: з відкритих джерел
