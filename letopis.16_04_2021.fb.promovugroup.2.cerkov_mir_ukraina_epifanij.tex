% vim: keymap=russian-jcukenwin
%%beginhead 
 
%%file 16_04_2021.fb.promovugroup.2.cerkov_mir_ukraina_epifanij
%%parent 16_04_2021
 
%%url https://www.facebook.com/epifaniy/photos/a.1617050028510560/2868499333365617/
 
%%author 
%%author_id 
%%author_url 
 
%%tags 
%%title 
 
%%endhead 
\subsection{Всі ми прагнемо встановлення справедливого миру - ПЦУ - Епифаний}
\Purl{https://www.facebook.com/epifaniy/photos/a.1617050028510560/2868499333365617/}

Всі ми прагнемо встановлення справедливого миру. Мир – природний стан людства.
Господь Ісус Христос часто звіщав, як істину і побажання: «Мир вам». Не маю
сумніву, що Україна, з милості Божої, здобуде мир.  

\ifcmt
  pic https://scontent-mxp1-2.xx.fbcdn.net/v/t1.6435-9/174041314_2868499336698950_3952935214485054253_n.jpg?_nc_cat=1&ccb=1-3&_nc_sid=730e14&_nc_ohc=nr7uHw8cO14AX8JMP_5&_nc_ht=scontent-mxp1-2.xx&oh=fe00bf971b7063fee8608779c10d7616&oe=60A15512
\fi

Але чи допоможе нам в цьому страх? Ні, страхом не перемогти.  В ці квітневі дні
ми входимо у восьмий рік війни. Війни, якої ми не хотіли, не ініціювали й не
провокували. Ми ніколи не зазіхали на чуже, не прагнули поневолити інших чи
загарбати чиюсь територію. Війни, в якій ми захищаємо Богом дану нам землю, наш
народ і наші цінності.     

І сусід-агресор добре знає, що протягом усього цього часу українці виявили таку
силу духу та єдності, якої мало хто від нас очікував. Нас не знищили обстріли й
не зламала брехня агресора. Тож вже вкотре в паніці російська влада ширить
«вірус страху», який століттями вирощує й культивує у підвладних їй сама, –
погрожує знищити, брязкає зброєю, збільшує психологічний тиск. Страх – дійсно
небезпечний «вірус», що може вбити зсередини. Але страх – це всього лиш
інструмент, яким агресор хоче підкорити нас. 

Не бійтеся! 

Не падайте духом! 

Ми – сильний народ і вміємо долати виклики. Пригадайте події 2014-го року, як
тоді поширювали «вірус страху», але перед лицем справжньої небезпеки страх
зник, натомість молитви Церков та єдність народу, сила армії та мудрість
дипломатів стали стіною супроти планів агресора. Як результат – ми зберегли
державність, вибороли духовну незалежність, увічнену в Томосі, завоювали
міжнародну підтримку, виховали сильну молодь, зуміли сформувати потужну оборону
на лінії фронту і в серцях мільйонів співвітчизників. 

Немає сумніву, переможемо й цю навалу. Наша Церква молиться за визволення від
чужинців. Наші капелани опікуються  військовими. Українська Церква – з
українським народом.  

«Не страшіться і не бійтеся їх; Господь, Бог ваш, іде перед вами; Він буде
боротися за вас...» (Втор. 1: 29 – 30).
