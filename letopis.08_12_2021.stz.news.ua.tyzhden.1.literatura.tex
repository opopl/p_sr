% vim: keymap=russian-jcukenwin
%%beginhead 
 
%%file 08_12_2021.stz.news.ua.tyzhden.1.literatura
%%parent 08_12_2021
 
%%url https://tyzhden.ua/Columns/50/253787
 
%%author_id nestelejev_maksim
%%date 
 
%%tags literatura,kultura,kniga,chelovek
%%title Не сподівайтеся позбутися літератури
 
%%endhead 
\subsection{Не сподівайтеся позбутися літератури}
\label{sec:08_12_2021.stz.news.ua.tyzhden.1.literatura}
\Purl{https://tyzhden.ua/Columns/50/253787}

\ifcmt
 author_begin
   author_id nestelejev_maksim
 author_end
\fi

Матеріал друкованого видання № 48 (732) від 1 грудня.

\textSelect{Здавна лунає хор голосів, який не до ладу виспівує про те, що книжки читати
необов’язково.}

Їм підспівує інший хор, який звіряється, що читати варто тільки нонфікшн, а
гаяти час на художню літературу не треба. Мовляв, і так життя коротке, довкола
«чудесний світ новий», у якому стільки незрозумілого, що хотілось би бодай
якось у цьому розібратися, а не марнувати час на чужі емоції й фальшиві
переживання. Проте мені це видається безнадійним хотінням відсікти голову від
серця, спростивши людину до бездушного інтелекту.

Утилітарний підхід до всього — споконвічний. Власне, він і забезпечив наше
виживання впродовж тисячоліть, коли спочатку уважно прислухалися до думок
велемудрих стариганів, а потім дбайливо вчитувалися у фоліанти з тією самою
метою: дізнатися про життя, осягнути істини, пережиті іншими, і зрозуміти те,
що нас оточує.

\zzrule

\begin{multicols}{2}
\em\color{magenta}

\ifcmt
  ig https://knigamir.com/upload/iblock/5a6/5a66c497a018093697dcecbc46a56b67.jpg
  @width 0.3
\fi

У роки, коли я тільки що почав викладати в своєму рідному місті, я завів собі
друга, а спільність наших смаків робила його мені дуже дорогим. Був він мені
ровесником і знаходився в тому ж цвітінні квітучої юності... Я відхилив його
від дійсної віри, — у нього, хлопці, вона не була глибокою і справжньою, — до
тих згубних і забобонних казок, які змушували матір мою плакати наді мною.
Разом з моєю заблукала і його душа, а моя не могла вже обходитися без нього.

І ось Ти, що по п'ятах наздоганяєш тих, хто біжить від Тебе, Бог помсти і
джерело милосердя, що обертає нас до себе чудовими способами, ось Ти узяв його
з цього життя, коли ледве виповнився рік нашій дружбі, що була для мене солодше
всього, що було солодкого в тодішньому моєму житті... Страждаючи лихоманкою,
він довго лежав без пам'яті, в смертному поту. Оскільки в його одужанні
зневірилися, то його охрестили в несвідомому стані. Я не звернув уваги на це,
розраховуючи, що в душі його швидше вдержиться те, що він пізнав від мене, ніж
те, що робили з його безпам'ятним тілом.

Сталося, проте, зовсім інакше. Він поправився і видужав, і як тільки я зміг
говорити з ним (а зміг я зараз же, як зміг і він, тому що я не відходжував від
нього, і ми не могли відірватися один від одного), я почав було глузувати з
хрещення, яке він прийняв зовсім без свідомості і без пам'яті. Він вже знав, що
він його прийняв. Я розраховував, що і він посміється разом зі мною, але він
відсахнувся від мене в жаху, як від ворога, і з дивною і раптовою незалежністю
сказав мені, що якщо я хочу бути його другом, то не повинен ніколи говорити
йому таких слів. Я, вражений і збентежений, вирішив відкласти свій натиск до
тих пір, поки він видужає і зможе, сповна видужавши, розмовляти зі мною про що
завгодно. Але через декілька днів, в мою відсутність, він знову захворів
лихоманкою і помер, віднятий у мене, безумного, щоб жити у Тебе на втіху мені

\ifcmt
  ig https://avatars.mds.yandex.net/i?id=d1be1275d86789ea7ed9cd172c7e440a-5233315-images-thumbs&n=13
  @width 0.3
\fi

Воскресенье, 14 июня 1942 г.

Начну с того момента, когда увидела тебя на столе среди других подарков ко
дню рождения. (То, что я накануне сама выбрала тебя  в магазине, не имеет
значения).

В пятницу 12 июня я проснулась в шесть утра. И не удивительно, ведь это
был день  моего  рождения! Но встать  так  рано я не могла, так что пришлось
усмирять  свое  любопытство  до  без  четверти  семь.  Терпеть  дольше  было
невозможно, и я  спустилась в столовую, где меня радостно приветствовал  наш
кот Морши.  В семь часов  я  заглянула  к маме с папой, а потом  в гостиную,
чтобы рассмотреть подарки. Первым увидела тебя -  мой самый  лучший подарок.
Еще были букет  роз  и  два букета пионов. От родителей  я получила  голубую
блузку, настольную игру  и  бутылку виноградного  сока,  которое,  по-моему,
напоминает по вкусу  вино  (вино ведь  тоже делают из винограда), настольную
игру, баночку крема, 2.5 гульдена и талон на две  книжки.  Еще книгу "Камера
Обскура",  которую  я, правда,  поменяла, потому что она уже  есть у  Марго,
домашнее печенье (испеченное, разумеется, мною, я - мастерица по печенью!),
много сладостей и мамин клубничный торт. А также письмо от бабули, пришедшее
именно в этот день, случайно, разумеется.

\end{multicols}

\zzrule

Нонфікшн у широкому значенні — усі жанри, що не базуються на вигадці (фікції)
чи принаймні не дуже нею зловживають. Це може бути і біографія, і сповідь (як в
Августина), і есеї (як у Монтеня з його «Пробами»), і щоденник (як в Анни
Франк), і спогади, і журналістика, і документалістика, але останнім часом під
цим поняттям розуміють передусім науково-популярну літературу.

«Ікони» наукпопу змінюються залежно від нових викликів, однак спільним
лишається прагнення знайти нового світського месію, який навчить жити та
сформулює те, що ми давно усвідомлювали, однак не могли дібрати потрібних слів.
Так, у Радянському Союзі, де були чудові серійні видання нонфікшну («Еврика»,
«У світі науки і техніки», бібліотечка «Квант» тощо), до безтями зачитувалися
Дейлом Карнеґі з його лицемірними «Як здобувати друзів і впливати на людей»
(«How To Win Friends and Influence People», 1936). Ще пам’ятаю, як під час
перебудови модними були «Закони Паркінсона» та різні брошурки про дієти, а
особливо цінною вважали книженцію Пола Бреґґа про голодування.

Згодом світ здивував сам себе, захопившись виданням на не надто прості теми
чорних дір, суперструн і часопростору, — «Короткою історією часу» («A Brief
History of Time», 1988) харизматичного Стівена Гокінґа. 1997-го в нонфікшну
нібито відбулося друге народження, коли інтерес до нього перезавантажили дві
монографії: «Тріска: Біографія риби, яка змінила світ» («Cod: A Biography of
the Fish That Changed the World») Марка Курланського та «Зброя, мікроби і харч»
(«Guns, Germs, and Steel: The Fates of Human Societies») Джареда Даймонда.
Згодом прогриміло ім’я Малколма Ґладуелла, зокрема після книжки «Поворотний
момент. Як дрібні зміни спричиняють великі зрушення» («The Tipping Point: How
Little Things can make a Big Difference», 2002) та «Спалах! Сила несвідомих
думок, або Як не заважати мозку приймати рішення» («Blink: The Power of
Thinking Without Thinking», 2007). Того ж року вийшов «Чорний лебідь» («The
Black Swan: The Impact of the Highly Improbable») Нассіма Талеба, що одразу ж
став бестселером, а з настанням світової економічної кризи тільки утвердився в
цьому статусі. Сучасна зірка наукпопу Ювал Ной Гарарі став таким, бо пропонує
читачам такий собі універсальний нонфікшн, об’єднання всіх теорій під однією
обкладинкою в книжках «Людина розумна. Історія людства від минулого до
майбутнього» («Sapiens: A Brief History of Humankind», 2014) та «Людина
божественна. За лаштунками майбутнього» («Homo Deus: A Brief History of
Tomorrow», 2016). Хоч, звісно, нині багатьом досить лише «Маленької книги хюґе»
(«The little book of hygge: the Danish way to live well», 2016) Міка Вікінґа чи
подібного, як раніше всі божеволіли від фен-шую.

Ознайомившись із певною кількістю різного нонфікшну, я, зрештою, можу висловити
до нього дві глобальні претензії: 1) кожну з таких книжок можна суттєво
скоротити десь до розміру невеличкої статті на вікіпедії; 2) у
науково-популярній літературі є все, крім дива.

Якщо перше, гадаю, більш-менш зрозуміле, то щодо другого поясню на прикладі.
Років п’ятнадцять тому журналісти спонтанно об’єднали провідних
авторів-атеїстів, які пишуть нонфікшн (Сема Гарріса, Деніела Деннета,
Крістофера Гітченса, Річарда Докінза), під назвою «Чотири вершники». Докінз,
найвідоміший серед них на наших теренах, любить повторювати фразу фантаста
Артура Кларка: «Будь-яка достатньо розвинена технологія не відрізняється від
магії». Література факту — це саме технологія, яка створюється за незмінними
законами, принципами й формулами; майстерна вигадка — це завжди магія, адже є
загальні рецепти й рекомендації, як її написати. Проте кожен шедевр художньої
літератури — це завжди їхнє дивовижне порушення. Нонфікшн може здивувати хитрим
зіставленням доктрин чи вірогідною гіпотезою, тоді як класика світової
літератури дивує саме тим, що її успіх непередбачуваний, а вплив
незапрограмований. Жадібне поглинання самого наукпопу створює ілюзію того, що
ти швидко пізнаєш світ, проте це не панацея від новочасного невігластва, адже
лихо з розуму починається тоді, коли люди відсікають емоції на догоду логіці та
переоцінюють власну здатність критично мислити. Читачі лише мотиваційних чи
бізнес-текстів, безумовно, орієнтуються в психології чи економіці, проте
обкрадають себе тим, що не звідають, як, скажімо, у поезії відкриття
відбувається якщо не в кожному рядку, то принаймні на кожній сторінці, а
винаходи стаються щоразу, коли несподівано стикаються думка та почуття.
Обмежуючись тільки нонфікшном, ігноруєш ту суттєву дрібничку, яка й робить
людиною, істотою розумною, але ж і співчутливою, розважливою, однак і
нерозсудливою, головастою та водночас сердешною. 
