% vim: keymap=russian-jcukenwin
%%beginhead 
 
%%file 19_05_2019.sites.ru.rusichi_info.1.ostanovit_nravstvennoje_rastlenije
%%parent 19_05_2019
 
%%url https://rusichi.info/952-ostanovit-nravstvennoe-rastlenie.html
 
%%author 
%%author_id 
%%author_url 
 
%%tags russia,molodezh
%%title ОСТАНОВИТЬ НРАВСТВЕННОЕ РАСТЛЕНИЕ
 
%%endhead 
 
\subsection{Остановить нравственное растление}
\label{sec:19_05_2019.sites.ru.rusichi_info.1.ostanovit_nravstvennoje_rastlenije}
\Purl{https://rusichi.info/952-ostanovit-nravstvennoe-rastlenie.html}

В настоящее время против России, а точнее против русов и других коренных
народов России, идёт необъявленная война, она ведётся врагами человечества уже
не одно столетие. Особое внимание в программе геноцида уделяется нравственному
растлению молодого поколения. Самое страшное, что эта необъявленная война –
невидима для большинства. Большинство людей, особенно молодое поколение, не
понимают, что происходит. Многие и не хотят понимать. Для того чтобы наши
действия были адекватны происходящему, важно правильно понимать, что
происходит. Мы должны видеть действия наших врагов, чтобы не позволять им
осуществлять задуманное. 

В музее славянской культуры им. Константина Васильева
в Москве участниками Движения была проведена лекция «Предотвращение
нравственного растления молодого поколения. Будущее России в наших руках» – с
целью рассказать об осуществляемом сегодня геноциде русского и других коренных
народов России через нравственное растление молодого поколения, о методах и
способах его реализации. Чем живёт сегодня молодое поколение, чем «дышит»?
Какими принципами руководствуется в жизни? Многие живут по принципам: «Моя хата
с краю, ничего не знаю», «Живём только один раз, и я должен попробовать всё в
этой жизни», «Мне важнее всего мои сиюминутные желания, и меня совершенно не
волнует, что будет потом» и т.п. – поэтому многие пьют, курят, развлекаются, не
задумываясь о последствиях. И это уже считается нормой в молодёжной среде. И
многим нет дела до того, что не входит в круг целенаправленно сформированных у
них интересов! В различных областях и сферах нашей жизни происходит навязывание
ложной информации, ложных ориентиров. 

В молодёжной среде происходит самая настоящая подмена ценностей. Самые высокие
морально-нравственные качества, которые были у наших предков, заменили самыми
низшими: хамство, лицемерие, ложь, невежество, неуважение к старшим, эгоизм,
предательство, распущенность нравов – подобное сегодня можно наблюдать сплошь и
рядом. Всё это является следствием применяемой социальными паразитами
технологии, с помощью которой они мечтают сделать из людей рабов своей системы.
Рабы не думают, не стремятся к развитию, ими легко управлять, а главное – они и
не знают, что они уже рабы.

Сегодня воздействие на молодёжь идёт многоплановое, во всех сферах жизни
социума расставлены ловушки. 

В СМИ, Интернет, индустрии развлечений, системе
образования, сфере культуры и искусства, через различные «модные» молодёжные
направления, субкультуры, в других областях нашей жизни – везде применяются
методы направленного воздействия на сознание и подсознание подрастающего
поколения, формирующие их мировоззрение в заданном направлении, с определёнными
ценностными ориентирами, установками поведения. Для молодёжи специально создают
определённую социальную среду, очень комфортную для существования в ней – с
целью увести в сторону от реального развития, от настоящих жизненных ценностей
человека, дезориентировать, оторвать от корней, заменив правду полуправдой и
откровенной ложью, отучить самостоятельно мыслить – сделать человека
пожизненным заложником этой «новой реальности», не имеющей ничего общего с
реальностью настоящей! Чтобы не попадаться в эти ловушки, нужно как минимум о
них знать, видеть их, понимать, что происходит. 

К сожалению, навязываемые
ложные ориентиры и ценности, недостаток правдивой информации – не позволяют
подрастающему поколению разобраться, что к чему, и вовремя выйти из-под этого
влияния. Многие так и остаются на всю жизнь заложниками этой «реальности», и их
жизнь становится пустоцветом. Дело в том, что в процессе своего развития
подросток проходит определённую стадию развития. В этот период структура
сущности молодого человека неустойчива и более подвержена разрушительным
влияниям извне, потому что резонирует с, условно, более «низкими» потоками
материй. Здесь под «низкими» потоками можно понимать музыку, где преобладают
низкие частоты, рваные ритмы, опять же, условно, тяжёлые звуки, отсутствие
смыслового наполнения и т.д. Подростков привлекает такая музыка, появляется
желание идти вразрез с общечеловеческими нормами поведения, выходят на первый
план не самые лучшие качества человека, характерно недостойное поведение – в
этот период инстинкты преобладают над разумом. И вот для них специально
создаётся такая особая среда (с определёнными ценностями, нормами поведения и
образом жизни и т.п.), которая им нравится, с которой они резонируют в этот
период своего развития, но выйти из-под влияния которой им крайне нелегко, в
том числе с учётом вышесказанного.

По сути, молодёжь целенаправленно задерживают
в стадии развития на уровне «разумных животных». Если не пройти эту стадию
вовремя, то можно задержаться в ней на всю оставшуюся жизнь. Такой человек уже
никогда не сможет выйти на более высокий уровень своего развития – уровень
«собственно человека». И конечно, таким обществом легко управлять. Всё это
прекрасно понимают социальные паразиты, используя особенности этого периода
развития человека для нанесения непоправимого удара по подрастающему поколению
населения России. 

\ifcmt
pic https://i.mycdn.me/i?r=AyH4iRPQ2q0otWIFepML2LxR9W_4asyK3hTOu15qo-ikmA
\fi

\subsubsection{СМИ}

В современном обществе СМИ являются одними из основных инструментов в
формировании мировоззрения молодого поколения. СМИ активно создают иллюзию
того, что они отражают различные мнения, представляют интересы всех
социо-культурных групп общества. Но есть мнение, что, возможно, за этим
разнообразием на самом деле хорошо маскируется насильственно внедряемое
единомыслие. 

Телевидение формирует наше мировоззрение, через него навязываются
те или иные установки и шаблоны поведения. При помощи художественных фильмов и
развлекательных шоу, почти все из которых или произведены в США, или сотканы по
американскому образцу, осуществляется скрытное и эффективное распространение
«правильных» представлений о ценностях и «правильного» мировоззрения. 

В художественных фильмах сюжет представляется так, как это «удобно», эмоции
высвобождаются в «нужном» направлении. Если потребуется, то о любом событии
можно снять фильм, в котором в занимательной форме будут «правильно объяснены»
первопричины. Граница между вымыслом и реальностью зачастую осознанно
стирается. 

Особенно сильное воздействие оказывается на детей. И не напрасно целые школьные
классы загоняют в кино на «ценные в воспитательном отношении» фильмы. Фильмы
оказывают гораздо более эффективное манипуляционное воздействие, чем новостные
репортажи. Репортажи читают или смотрят осознанно, в то время как
художественный фильм всегда целится в подсознание. Событие смешивается с
эмоциями и поэтому врезается в память. Потом уже не знаешь «где именно, но
видел это собственными глазами». 

Если разобраться в вопросе, какие мультфильмы сегодня предлагает индустрия
развлечений для просмотра детям, какие игрушки, то станет очевидной их
направленность на разрушение психики ребёнка. В противоположность тому
исконному укладу, который был у нас веками, внедряется всё чуждое нашему
менталитету (менталитету коренного населения, проживающего в России). 

К сожалению, современные, вечно занятые родители зачастую не уделяют должного
внимания развитию своих детей, предоставляя их самим себе, или, «чтобы не
мешался», просто включают ему телевизор или мультфильмы. И даже не
задумываются, какие образы передаются детям в мультфильмах. В женских образах
наши традиционные качества, такие как целомудрие, нежность, застенчивость,
безкорыстие, скромность, материнство, как таковое, заменяются на диаметрально
противоположные, навязывается «агрессивная сексуальность» в поведении.

\ifcmt
pic https://i.mycdn.me/i?r=AyH4iRPQ2q0otWIFepML2LxR5ghYdcKQSklav592easAEg
\fi

Пропаганда в СМИ насилия, жестокости, распущенности, потребительского образа
жизни, растлевает общество, в особенности детей и подростков. Внимательно
изучив молодёжные печатные издания и посмотрев телевизор, вы легко заметите,
что активней всего нашим детям навязываются следующие установки поведения:
необходимость раннего начала половой жизни, вред полового воздержания в юном
возрасте, необходимость гражданского или пробного брака, одобрительное
отношение к сексуальным меньшинствам. 

Задача такого социального моделирования –
создание с помощью средств массовой информации человека с заранее заданными
нормами поведения. 

Программы и проекты по сокращению численности населения С
далеко идущей целью «создания условий для захвата нашего государства, земель,
ресурсов» в России внедряются технологии искусственного сокращения численности
населения нашей страны. 

Последние годы внутри нашей страны происходит активное
лоббирование на законодательном уровне вредоносных программ по разрушению
института семьи, растлению молодёжи через извращённое сексуальное и
морально-нравственное воспитание и образование, не прекращающихся попыток
введения ювенальных технологий.

\ifcmt
pic https://i.mycdn.me/i?r=AyH4iRPQ2q0otWIFepML2LxRBfDuereDiPor8bztfLtMRg
\fi

Форсайт-проект «Детство 2030», как «олицетворение России будущего», активно
продвигается в нашей стране и набирает обороты. Основными авторами проекта
являются иностранные организации: заказчик – Благотворительный фонд «Моё
Поколение», исполнитель – Международная Методологическая Ассоциация при
поддержке Общественной Палаты Российской Федерации (сайт проекта «Детство
2030»).

В статье на сайте «Русская народная линия» говорится о будущем, которое
уготовили российским детям разработчики этого проекта. По всей России сегодня,
при поддержке Правительства РФ и других государственных структур, созданы
представительства организации РАПС (Российской Ассоциации Планирования Семьи),
приоритет в деятельности которой отдаётся секспросвещению молодёжи, пропаганде
и распространению небезопасных по своей сути методов и средств, направленных
против деторождения (о Международной федерации планирования семьи (МФПС) и
Российской ассоциации планирования семьи (РАПС), как подразделения МФПС). 

Центр медико-психологической помощи населению «Холис» разрабатывает совершенно
чудовищные материалы (мультфильмы, брошюры и т.п.) и методики обучения для
внедрения их в школах на уроках секспросвета (правовая оценка деятельности
Центра «Холис» и экспертное заключение).

\ifcmt
pic https://i.mycdn.me/i?r=AyH4iRPQ2q0otWIFepML2LxR-hrey4RbKVVIWbHb4BkwPw
\fi

На уроках секспросвета в школах детям рассказывают об устройстве половой
системы мужчины и женщины, школьникам постарше объясняют особенности
взросления, делая акцент на безопасности сексуальных контактов и необходимости
использовании презерватива, чтобы себя обезопасить. Под прикрытием борьбы со
СПИДом подросткам навязывают мысль об опасности незащищённого секса,
рассказывают о разных видах секса между делом, проводя мысль о том, что половые
отношения в этом возрасте, как бы сами собой разумеются. 

Создаваемая в России система ювенальной юстиции – очень серьёзная угроза,
направленная на уничтожение и семьи, и всего государства. Ювенальные технологии
в России пытаются внедрить уже давно в разных формах и упаковках. В статье
«Ювенальный Садизм» анализируется современное положение вещей: «Два года назад
совместными усилиями родителей и общественных организаций наш народ отбил атаку
питающегося детьми и семьями западного чудовища – ювенальной юстиции,
[заблокировав принятие в Госдуме лоббируемого финансируемыми Западом
организациями (такими как РАПС и ХОЛИС) законопроекта о ювенальных судах]. Это
была большая победа. 

Но за эти два года враги русского народа учли свои ошибки, как следует
подготовились и предприняли новую атаку, попытавшись замаскировать суть новых
инициатив. В Государственную Думу были внесены законопроекты, представляющие
собой ту же самую ювенальную юстицию, только как бы поделённую на части и
переименованную-замаскированную. Народ общественным обсуждением новых
законопроектов и, вообще, информированием о них решили на этот раз «не
утомлять». А между тем принятие этих законов напрямую угрожает всем семьям и
гражданам России!»

\ifcmt
pic https://i.mycdn.me/i?r=AyH4iRPQ2q0otWIFepML2LxRv0lgu73SVE20Btxf45a6aA
\fi

Несмотря на изменение стратегии и тактики действий, суть от этого не меняется:
внедряемая в России система ювенальной юстиции – это «система, нацеленная, с
одной стороны, на максимальное смягчение отношения к малолетним
правонарушителям, а с другой – на внедрение технологии узаконенного изъятия из
семьи любого ребёнка под предлогом защиты его интересов». Помимо бюджетного
финансирования всех этих программ щедрые пожертвования идут из-за границы: от
МВФ через ООН (Фонд народонаселения и ЮНЕСКО), от Международной федерации
планирования семьи и Ассоциации планирования семьи США, от английского фонда
НОУ-ХАУ, от Агентства по международному развитию (США) и многих других
иностранных организаций. Все они составляют материальный многомиллионный базис
для обеспечения геноцида населения в России через создания условия резкого
падения рождаемости. 

\ifcmt
pic https://i.mycdn.me/i?r=AyH4iRPQ2q0otWIFepML2LxRmfJFRd3AeecBbZS-LTQkSQ
\fi

\subsubsection{Телегония }

Ещё одним инструментом сокращения численности населения является замалчивание и
дискредитация информации о тех или иных реальных природных явлениях и
процессах. Такую информацию целенаправленно замалчивают, чтобы люди не знали
реальных последствий своих действий и, по незнанию, совершали эти действия,
нанося урон своему здоровью и генетике будущих поколений. Так и произошло с
явлением телегонии, о котором сегодня нигде открыто не говорят, наоборот,
выставляют это реальное явление как не соответствующее действительности.Об этом
явлении знали наши предки, в традициях которых отражено понимание природных
процессов и последствий тех или иных действий. 

Эти традиции были залогом их психического и физического здоровья, позволяли им
долгое время сохранять свои Роды от вырождения. После периода замалчивания и
сокрытия от людей правды, стали появляться доказательства реальности явления
телегонии. Всё больше открытий независимых учёных и исследователей (П. Гаряев,
В. Авдеев, Н. Левашов и др.), новых методов (газоразрядной визуализации –
эффект Кирлиана; спектральный анализ и др.), фиксирующих наличие у человека
сущности (совокупности материальных тел человека) – подтверждают реальность
этого явления, суть которого заключается в том, что во время близости между
мужчиной и женщиной происходит контакт не только на физическом уровне, но и на
уровне их сущностей, и мужчина оставляет мощный отпечаток своей сущности на
сущности женщины, «программируя» женщину на рождение потомства именно от этого
мужчины. Несмотря на то, что существует уже достаточно доказательств реальности
этого явления, и всё больше и больше появляется новых, этот вопрос по-прежнему
замалчивается, следовательно, люди, в частности, молодёжь, в общей своей массе
по прежнему об этом не знают и продолжают совершать ошибки, которые сказываются
на их здоровье, здоровье их детей, и значит – на будущем целой нации.

\ifcmt
pic https://i.mycdn.me/i?r=AyH4iRPQ2q0otWIFepML2LxRayNJFm9CTiK5Ff5k0QDfYA
\fi

\subsubsection{Заключение}

Молодому поколению без осознания того, что происходит в различных областях и
сферах жизни, без верных ориентиров, основанных на настоящих знаниях, избежать
целенаправленного всестороннего «прессинга» и выбраться из западни, в которую
они попадают в подростковый период, – очень трудно, учитывая особенности
процесса развития в этот период. Родители же, сами мало понимая, что на самом
деле происходит сегодня в нашем обществе, не имея достоверной информации,
продолжают пребывать в иллюзии, что ничего плохого с их детьми случиться не
может; вечно занятые решением кажущихся им более важными задач, они зачастую не
понимают всей серьёзности положения вещей, не придавая реальным угрозам
должного значения. 

Какой же выход из всего этого у нас есть? Что мы можем сделать? • Прежде всего,
нам самим повышать нашу осознанность, разбираться во всём, что происходит
вокруг и внутри нас, вникать во всё, что касается наших детей. Понимание
происходящего вокруг нас поможет нам увидеть систему, которая делает из людей
рабов, и самим выйти из-под её влияния. • Задуматься, какой пример жизни мы
подаём окружающим – в семье, детям, родным и близким, друзьям, сослуживцам,
какие качества мы проявляем. • Создавать полноценные семьи, которые будут
являться лучшей защитой детям в период их развития и формирования их
мировоззрения. • Информировать своё окружение о том, что происходит, о разных
формах и методах геноцида. • Важна гражданская позиция каждого человека, она
должна быть активной – нельзя проходить мимо несправедливости, нельзя быть
безразличным – только так можно что-то изменить. Мы должны знать

\ifcmt
pic https://i.mycdn.me/i?r=AyH4iRPQ2q0otWIFepML2LxRIybMkHhk0RXCFpO-8ZdXYg
\fi
