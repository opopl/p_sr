% vim: keymap=russian-jcukenwin
%%beginhead 
 
%%file 2016.uinp.22_sichnja_den_sobornosti_ukrainy.0.intro
%%parent 2016.uinp.22_sichnja_den_sobornosti_ukrainy
 
%%url 
 
%%author_id 
%%date 
 
%%tags 
%%title 
 
%%endhead 

\begin{center}
\textbf{Український інститут національної пам'яті пропонує вважати 2016-й Роком Державності України.}
\end{center}

\ii{2016.uinp.22_sichnja_den_sobornosti_ukrainy.pic.1}

2016 року Україна відзначає 25-річчя нинішньої державності. 24 серпня 1991 року
прийнято Акт проголошення незалежності України, який українці схвалили на
референдумі 1 грудня. Ця подія стала відправною точкою для відліку історії
сучасної Української державності.

Утім, історики цілком справедливо відзначають, що 24 серпня 1991 року насправді
відбулося відновлення державної незалежності України.

Вперше у XX столітті українська незалежність була проголошена 22 січня 1918
року IV Універсалом Української Центральної Ради, а вже за рік (22 січня 1919
року) на Софійському майдані в Києві відбулася не менш вагома подія –
об'єднання Української Народної Республіки (УНР) і Західно-Української Народної
Республіки (ЗУНР) в одну державу.

З метою гідного відзначення 25-річчя незалежності України на виконання Указу
Президента України \enquote{Про відзначення 25-ї річниці незалежності України} від 03
грудня 2015 року № 675 Український інститут національної пам'яті пропонує
вважати \textbf{2016 рік Роком Державності України}.

Важливо говорити не тільки про 24 серпня 1991 року – акт, який став логічним
завершенням боротьби українців за незалежність, а й про цілу низку визначальних
для державотворення історичних подій ХХ століття:

\begin{itemize} % {
\item 22 січня 1918 року – проголошення незалежності Української Народної Республіки;

\item 22 січня 1919 року – проголошення Акта злуки Української Народної Республіки і
Західно-Української Народної Республіки;

\item 15 березня 1939 року – проголошення незалежності Карпатської України;

\item 30 червня 1941 року – проголошення Акта відновлення української державності;

\item 16 липня 1990 року – ухвалення Декларації про державний суверенітет України;

\item 1 листопада 1918 року – день \enquote{Листопадового чину}, українського
повстання у Львові, за результатами якого постала Західно-Українська Народна
Республіка;

\item 1 грудня 1991 року – Всеукраїнський референдум на підтвердження Акта
проголошення незалежності України від 24 серпня 1991 року.
\end{itemize} % }

\textbf{Однією з найважливіших історичних дат є саме 22 січня} (проголошення Першої
Незалежності у 1918 році та проголошення Акта злуки УНР і ЗУНР) — День
Соборності.

\textbf{Заходи з цієї нагоди мають розпочати ювілейні вшанування в рамках Року Державності України.}

