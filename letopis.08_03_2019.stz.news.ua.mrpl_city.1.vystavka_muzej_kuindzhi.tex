% vim: keymap=russian-jcukenwin
%%beginhead 
 
%%file 08_03_2019.stz.news.ua.mrpl_city.1.vystavka_muzej_kuindzhi
%%parent 08_03_2019
 
%%url https://mrpl.city/news/view/v-mariupole-otkrylas-unikalnaya-vystavka-iz-proshlogo-veka-foto
 
%%author_id news.ua.mrpl_city
%%date 
 
%%tags 
%%title В Мариуполе открылась уникальная выставка из прошлого века (ФОТО)
 
%%endhead 
 
\subsection{В Мариуполе открылась уникальная выставка из прошлого века (ФОТО)}
\label{sec:08_03_2019.stz.news.ua.mrpl_city.1.vystavka_muzej_kuindzhi}
 
\Purl{https://mrpl.city/news/view/v-mariupole-otkrylas-unikalnaya-vystavka-iz-proshlogo-veka-foto}
\ifcmt
 author_begin
   author_id news.ua.mrpl_city
 author_end
\fi

Лучшие работы графиков второй половины ХХ века представлены в Мариупольском
художественном музее им. А.И. Куинджи.

На выставке экспонируются более 100 графических работ народных, заслуженных
художников Украины из киевской коллекции дирекции художественных выставок
Национального союза художников.

Экспонаты выполнены в технике акварели, литографии, линогравюры, акватинты,
офорта и других направлений графики.

Руководитель музея Татьяна Були на открытии выставки отметила значимость
события в культурной жизни Мариуполя, так как подобная коллекция работ такого
высокого уровня экспонируется в городе впервые.

\enquote{На выставке представлены работы известных мастеров графического искусства. Они
отличаются филигранной утонченность и виртуозной техникой. Интересны острые и
разнообразные композиции, психологическая выразительность образов, легкая
непринужденность рисунка. Многие имена связаны с развитием книжной графики}, -
отметила Татьяна Були.

Выставка \enquote{Графика украинских художников} ждет посетителей до 14 апреля.

Напомним, что \href{https://mrpl.city/news/view/proekt-genetic-cod-2-vizualnye-spetse-ffekty-udivili-mariupoltsev-foto}{визуальные спецэффекты проекта \enquote{Genetic Cod-2} удивили
мариупольцев}. Ранее публиковалось, что в Мариуполе появились картины художника,
популярного в аукционных домах \href{https://mrpl.city/news/view/-341}{\enquote{Кристис} и \enquote{Сотбис}}.
