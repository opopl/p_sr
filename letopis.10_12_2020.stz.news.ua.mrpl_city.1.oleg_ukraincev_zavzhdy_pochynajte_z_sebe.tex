% vim: keymap=russian-jcukenwin
%%beginhead 
 
%%file 10_12_2020.stz.news.ua.mrpl_city.1.oleg_ukraincev_zavzhdy_pochynajte_z_sebe
%%parent 10_12_2020
 
%%url https://mrpl.city/blogs/view/oleg-ukraintsev-zavzhdi-pochinajte-z-sebe
 
%%author_id demidko_olga.mariupol,news.ua.mrpl_city
%%date 
 
%%tags 
%%title Олег Украінцев: "Завжди починайте з себе!"
 
%%endhead 
 
\subsection{Олег Украінцев: \enquote{Завжди починайте з себе!}}
\label{sec:10_12_2020.stz.news.ua.mrpl_city.1.oleg_ukraincev_zavzhdy_pochynajte_z_sebe}
 
\Purl{https://mrpl.city/blogs/view/oleg-ukraintsev-zavzhdi-pochinajte-z-sebe}
\ifcmt
 author_begin
   author_id demidko_olga.mariupol,news.ua.mrpl_city
 author_end
\fi

\emph{\textbf{На початку 2020 року я вирішила створити серію нарисів,
присвячених жителям Маріуполя, закоханих у своє місто. За рік встигла
висвітлити 53 біографії непересічних та яскравих містян, чия діяльність і
життєвий шлях приємно вражають і надихають на нові звершення.}} \textbf{Наприкінці року
пропоную познайомитися ще з одним} \emph{ініціативним, різнобічним і талановитим
чоловіком, який має креативне мислення, завжди готовий прийти на допомогу і
підтримати в будь-якій ситуації...}\par\emph{\textbf{Олег Украінцев}} – справжній генератор ідей
і енергії. Він став координатором великої кількості загальноміських проєктів,
зміг посприяти громадсько-культурному розвитку міста і активізації молоді, як
Маріуполя, так і загалом України. Вважає найвищою цінністю свободу, завдяки
якій людина може реалізувати власний потенціал повною мірою.

У Олега є багато однодумців, при цьому в кожному напрямі вже давно є \enquote{свої
перевірені люди}. Водночас його ентузіазм і натхненна діяльність посприяли
реалізації багатьох суспільно значущих проєктів.

\ii{10_12_2020.stz.news.ua.mrpl_city.1.oleg_ukraincev_zavzhdy_pochynajte_z_sebe.pic.1}

Народився Олег у Білорусі в 1976 році в дружній родині. Доля нагородила не
тільки люблячими батьками, але і старшим братом. Батько – \emph{\textbf{Анатолій
Володимирович}} – кадровий військовослужбовець. Він, ветеран ракетних військ, і
зараз працює воєнруком в ЗОШ № 65. У минулому командував ракетним дивізіоном в
Білорусі. Цікаво, що коли батько Олега був ще молодим лейтенантом, його
заохотили санаторною путівкою до Криму, де Анатолій Украінцев відпочивав в
одному заїзді разом з \emph{Юрієм Гагаріним}. У 1986 році сім'я переїхала до
Маріуполя. Спочатку приїхали просто у відпустку, але зрозумівши, що у цьому
місті хочеться жити, вирішили остаточно переїхати. В дитинстві, використовуючи
велику бібліотеку батьків, читав дуже багато, здебільшого фантастику,
детективи. У школі найбільше любив вивчати історію. Олег з дитинства був дуже
активним хлопцем: займався боксом, парашутним спортом. Довгий час хотів стати
льотчиком, але пізніше вирішив, що професія слідчого йому ближча. У 1997 року
закінчив стаціонар Харківського національного університету внутрішніх справ,
спеціальність \enquote{юрист-правознавець}. Працював оперуповноваженим карного розшуку
Маріуполя. Цей період окремий, насичений у біографії. Насправді знайти своє
покликання чоловік зміг завдяки \emph{\textbf{Свірському Борису Михайловичу}} (нині професору
МДУ) та \emph{\textbf{Юлії Гетьман}} (доньці Бориса Михайловича). З Юлею він сидів у школі за
однією партою, а її батько часто приходив до школи з розповідями про роботу
правоохоронних органів (тоді працівник Жовтневого райвідділу капітан
Свірський). Коли Олег заходив до Юлі додому, її батько ненав'язливо давав
почитати хлопцю цікаві детективи. Завдяки Борису Михайловичу, який своїми
розповідями про будні міліціонера зачепив школяра, Олег і вирішив стати слідчим
(тоді ж не знав що сищик і слідчий – це не одне і теж). У 2000 році чоловік був
особисто запрошений \emph{\textbf{Володимиром Семеновичем Бойком}} на роботу провідним
референтом генерального директора ВАТ \enquote{ММК ім. Ілліча}, де працював та
продовжує працювати вже на посаді начальника відділу донині. У 2006 р. заочно
закінчив Донецький державний університет управління (економічний факультет). У
2014 році закінчив аспірантуру Національної академії правових наук України. У
лютому 2015 року в Національній академії правових наук України успішно захистив
дисертацію на тему: \emph{\enquote{Договір концесії морських портів}}. Одним з опонентів на
захисті був академік, доктор юридичних наук Шишка Роман Богданович. а у вченій
раді був присутній доктор юридичних наук Руслан Стефанчук, котрий згодом зробив
стрімку політичну кар'єру, а тоді дебати торкалися лише наукових моментів.
Незабаром Олегу було присвоєно звання кандидата юридичних наук, а також з
урахуванням змін в системі освіти України особисто від академіка Національної
академії правових наук України \emph{Володимира Васильовича Луця} отримав диплом
доктора філософії в галузі права.

\ii{10_12_2020.stz.news.ua.mrpl_city.1.oleg_ukraincev_zavzhdy_pochynajte_z_sebe.pic.2}

Олег вдячний долі, що був знайомий з легендою маріупольського і не тільки
розшуку, \emph{\textbf{Чорним Миколою Олександровичем}}, який часто допомагав і порадою та
надихав особитим прикладом. Також чоловіку пощастило працювати з \emph{Валерієм
Андрущуком, Андрієм Шелеховим, Василем Зваричем, Миколою Побійним} – справжніми
професіоналами своєї справи.

У 2015 році Олег Анатолійович видав монографію \enquote{Договір концесії морських
портів України}. Сфера наукових інтересів: право інтелектуальної власності,
авторське право, теоретичні проблеми юридичних осіб, договірне право, дозвільна
діяльність у сфері господарювання, митні режими та процедури, відповідальність
у цивільному праві. Однак Олег є автором не тільки накових доповідей, також
чоловік пише вірші і прозу. А починалося все  зі статей і заміток, присвячених
туризму, написаних на прохання відомого маріупольського журналіста \emph{\textbf{Юрія Яковича
Некрасовського}}. Деякі вірші нашого героя можна знайти у збірці \textbf{\enquote{По живому.
Околовоенные дневники}}.  яка стала важливою для всіх авторів, що увійшли до
книги (всі автори збірки – мешканці прифронтової або окупованої території).
Вірші Олега Украінцева передають щирі переживання автора і дійсно чіпляють за
живе...

\begin{quote}
\em\enquote{І нехай під пташиний плач,

Очі бриз полоскоче сіллю,

Він бажає всім море удач,

Хто живе тут з надією та біллю.

Спорожніє осінній пляж,

\enquote{Єгоза} похитнеться місцями

І заллє,мов басейн,бліндаж

Та відбитки АТО із рубцями}...
\end{quote}

Масштабна презентація збірки відбулася у багатьох містах Ук\hyp{}раїни: від Одеси до
Луцька, від Краматорська до Львова. І усюди авторів зустрічали небайдужі та
щирі слухачі. \enquote{По живому} – це не комерційний, а соціальний проєкт. Тому всі
партнери допомагали на волонтерських засадах. Охочих допомогти було і є дуже
багато. Наприклад, переклад книги литовською мовою, а згодом і презентацію в
Литві зробила Спілка письменників Литви абсолютно безкоштовно. Цікаво, що
англійською збірку переклав американський перекладач \emph{\textbf{Райлі Костіган}}, який є
перекладачем книги \emph{\textbf{Сергія Жадана}} та до якого стоїть неймовірна черга. Він
переклав більшу частину прози. Причому, як всі на волонтерських засадах, ще й
подякував за надану можливість перекладати таку потужну книгу.

Але, перш за все, Олег дуже вдячний співавторці проєкту \textbf{Оксані Стоміній!}
Чоловік наголошує, що \emph{у Оксани безмежно позитивна енергетика, яка має цілющі
властивості.}

\ii{10_12_2020.stz.news.ua.mrpl_city.1.oleg_ukraincev_zavzhdy_pochynajte_z_sebe.pic.3}

Наступним проєктом творчої команди, яку назвали \emph{\enquote{Паперові сходи}} (до Олега і
Оксани приєднався \emph{\textbf{Дмитро Паскалов}}) – став дитячий проєкт, розрахований на
участь дітей прифронтових територій, де зараз найбільш гостро відчувається
підвищена тривожність і нервова напруга серед дорослих і дітей. Організатори
проєкту вирішили створити унікальну книгу, написану підлітками, але адресовану
дорослим. Вона так і називається \emph{\textbf{\enquote{Лист дорослому}}}. Запропонували дітям написати
листа на будь-яку тему з питаннями, побажаннями, зауваженнями, болем і радістю.
Ця книжка  вийшла невимовно зворушливою.

Також командою підготовлені дитяча кишенькова брошура \emph{\textbf{\enquote{Це моє право}}} і збірка
\emph{\textbf{\enquote{Маю право}}}. Олег впевнений, що ці видання є справді корисними, інформативними,
зрозумілими для будь-якої дитини і зручними для використання. В книжках йдеться
не лише про права дитини та її обов'язки, але й про порушення прав, булінг,
кібербулінг тощо. Найважливіше, що книги дають дитині чіткі і прості поради та
роз'яснення про те, як діяти та куди звертатися по допомогу у випадках
порушення прав або жорстокого поводження із нею. Книги від ГО \enquote{Паперові сходи}
опинилися серед переможців ювілейного Десятого відкритого міського конкурсу
\emph{\textbf{\enquote{Книга і преса року}.}}

\ii{10_12_2020.stz.news.ua.mrpl_city.1.oleg_ukraincev_zavzhdy_pochynajte_z_sebe.pic.4}

Серед незавершених проєктів є один особливо цікавий. Це\par\noindent фільм \emph{\textbf{\enquote{Жив був
будинок}.}} Його ідею з Оксаною Олег довго обговорював, і почав реалізовувати з
відомим в Маріуполі журналістом, режисером \emph{\textbf{Андрієм Кіором}} – засновником
творчого об'єднан\hyp{}ня \emph{\enquote{Адамаха Фiльм}}, яке створило в Маріуполі
художньо-докумен\hyp{}тальний фільм \enquote{Таємниця козацького храму}, а також кілька
короткометражних художніх фільмів. Для фільму Олега та Оксани вже відзнято
багато матеріалу. Тому можна сподіватися, що незабаром маріпольчани зможуть
його побачити.

Сьогодні Олег – щасливий чоловік та батько. Дружина – \emph{\textbf{Олена}} – членкиня
Національної спілки художників України, невтомно відкриває Маріуполь для
українських та зарубіжних міст з кращих сторін. Старший син Єгор – студент
ХНУРЕ, молодший Тимур –  ще школяр. Нещодавно родина поповнилася – тепер сім'ю
охороняє чарівне щеня Алмаз. У Маріуполі найбільше любить море. Він катається
на лижах, грає у настільний теніс, любить велосипед, водні походи, останнє
особливо підтримують сини.

В будинку Олега, у теплій дружній атмосфері, у вечірнього каміна траплялося
бувати і художникам, і оперним співакам, гостювали поети і письменники.
Зокрема, Сергій Жадан, Олександр Ірванець, Григорій Семенчук, Володимир
Рафеенко, Вадим Джувага і, звичайно, Оксана Стоміна.

\ii{10_12_2020.stz.news.ua.mrpl_city.1.oleg_ukraincev_zavzhdy_pochynajte_z_sebe.pic.5}

Планів у талановитого маріупольців ще багато, адже зупинятися на досягнутому
він не збирається. Впевнена, що поперед, ще багато креативних проєктів, які
стануть корисними та важливими для нашого міста.

\begingroup
\em
\textbf{Улюблена книга:} Ільф і Петров  \enquote{Дванадцять стільців} (1928).

\textbf{Улюблений фільм:}  \enquote{Пролітаючи над гніздом зозулі} (1975), \enquote{Форест Гамп} (1994). ,

\textbf{Порада маріупольцям:}  

\begin{quote}
\enquote{Завжди починайте з себе! Знаходьте для себе час. Людина – це поле для реалізації найрізноманітніших ідей. І ще -  важливо не придумувати героїв, а знаходити та відкривати їх в своєму житті!}.
\end{quote}
\endgroup
