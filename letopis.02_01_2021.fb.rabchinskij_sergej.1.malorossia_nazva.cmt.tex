% vim: keymap=russian-jcukenwin
%%beginhead 
 
%%file 02_01_2021.fb.rabchinskij_sergej.1.malorossia_nazva.cmt
%%parent 02_01_2021.fb.rabchinskij_sergej.1.malorossia_nazva
 
%%url 
 
%%author 
%%author_id 
%%author_url 
 
%%tags 
%%title 
 
%%endhead 
\subsubsection{Коментарі}
\label{sec:02_01_2021.fb.rabchinskij_sergej.1.malorossia_nazva.cmt}

\begin{itemize}
%%%fbauth
%%%fbauth_name
\iusr{Ирина Будаева}
%%%fbauth_url
%%%fbauth_place
%%%fbauth_id
%%%fbauth_front
%%%fbauth_desc
%%%fbauth_www
%%%fbauth_pic
%%%fbauth_pic portrait
%%%fbauth_pic background
%%%fbauth_pic other
%%%fbauth_tags
%%%fbauth_pubs
%%%endfbauth
 
"Русь була одна з центром в Києві. І крапка." Коли ж настане час і натхнення повертати власну назву?..


\begin{itemize}
%%%fbauth
%%%fbauth_name
\iusr{Сергей Рабчинский}
%%%fbauth_url
%%%fbauth_place
%%%fbauth_id
%%%fbauth_front
%%%fbauth_desc
%%%fbauth_www
%%%fbauth_pic
%%%fbauth_pic portrait
%%%fbauth_pic background
%%%fbauth_pic other
%%%fbauth_tags
%%%fbauth_pubs
%%%endfbauth
 
\textbf{Ирина Будаева} 

Коли вистачить розуму вивчити власну історію.

Не знаю хто сказав (помилково приписується Геббельсу): “Відбери у народу
історію - і через покоління він перетвориться на натовп, а ще через покоління
ним можна управляти, як стадом”. Чи не здається вам що це уж дуже схоже на
українців?
\end{itemize}

%%%fbauth
%%%fbauth_name
\iusr{Евгений Романов}
%%%fbauth_url
%%%fbauth_place
%%%fbauth_id
%%%fbauth_front
%%%fbauth_desc
%%%fbauth_www
%%%fbauth_pic
%%%fbauth_pic portrait
%%%fbauth_pic background
%%%fbauth_pic other
%%%fbauth_tags
%%%fbauth_pubs
%%%endfbauth
 
Так!

%%%fbauth
%%%fbauth_name
\iusr{Denis Nevyadomski}
%%%fbauth_url
%%%fbauth_place
%%%fbauth_id
%%%fbauth_front
%%%fbauth_desc
%%%fbauth_www
%%%fbauth_pic
%%%fbauth_pic portrait
%%%fbauth_pic background
%%%fbauth_pic other
%%%fbauth_tags
%%%fbauth_pubs
%%%endfbauth
 
Так украинцы шарахаются от Руси, как черт от ладана, как вернуть имя и историю?

\begin{itemize}
%%%fbauth
%%%fbauth_name
\iusr{Сергей Рабчинский}
%%%fbauth_url
%%%fbauth_place
%%%fbauth_id
%%%fbauth_front
%%%fbauth_desc
%%%fbauth_www
%%%fbauth_pic
%%%fbauth_pic portrait
%%%fbauth_pic background
%%%fbauth_pic other
%%%fbauth_tags
%%%fbauth_pubs
%%%endfbauth
 
\textbf{Denis Nevyadomski} А как вернуть людям мозги? Человечество ничего кроме
просвещения не придумало. Помните давнешний анекдот, в котором на вопрос
сколько надо учиться для того, чтобы стать по настоящему интеллигентным
человеком, был получен ответ, что для этого надо, чтобы было три высших
образования: высшее образование вашего деда, вашего отца и ваше.
\end{itemize}

%%%fbauth
%%%fbauth_name
\iusr{Максим Мосин}
%%%fbauth_url
%%%fbauth_place
%%%fbauth_id
%%%fbauth_front
%%%fbauth_desc
%%%fbauth_www
%%%fbauth_pic
%%%fbauth_pic portrait
%%%fbauth_pic background
%%%fbauth_pic other
%%%fbauth_tags
%%%fbauth_pubs
%%%endfbauth
 
Кто вообще эту карту придумал?!

\begin{itemize}
%%%fbauth
%%%fbauth_name
\iusr{Сергей Рабчинский}
%%%fbauth_url
%%%fbauth_place
%%%fbauth_id
%%%fbauth_front
%%%fbauth_desc
%%%fbauth_www
%%%fbauth_pic
%%%fbauth_pic portrait
%%%fbauth_pic background
%%%fbauth_pic other
%%%fbauth_tags
%%%fbauth_pubs
%%%endfbauth
 
\textbf{Максим Мосин} 

Когда в споре со сторонниками Русского мира об "исконно русских Донбассе и
Крыме", а заодно и территорий нынешних Белгородской, Воронежской, Ростовской
областей и Краснодарского края приводишь данные Переписи населения Российской
империи 1897 года, предваряя их вопросом с вполне предсказуемым ответом -
считают ли они царя Николая II украинским националистом, то реакция - как
правило, немая сцена из "Ревизора", авторства спорного украино-русского
писателя Гоголя. The numbers are what they are (С цифрами не поспоришь). И
только у самых заматерелых руссомирцев начинается истерика в стиле "вы все
врете". Это надо понимать: я и государь Император Николай II.

Украинская Народная Республика в начале 1918 года согласно международно
признанным границам по Берестейскому договору от 9 февраля 1918 года с
Германской, Австро-Венгерской, Османской империями и Болгарским царством,
признанным РСФСР по Брестскому миру от 3 марта 1918 года с теми же Центральными
державами, включала в себя все этнические украинские земли по переписи
Российской империи 1897 года.

А это, кроме нынешней территории Украины (естественно с аннексированным
"исконно русским" татарским Крымом), ещё и Берестейщина (территория нынешней
Бресткой области Беларуси), Северщина и Стародубщина (территории нынешних
Гомельской области Беларуси и Брянской, Курской, Белгородской, Орловской и даже
Тульской областей России), Слобожанщина (территории тех же Белгородской-Курской
и Воронежской областей), Донщина (территория Ростовской области), Кубань и
Ставропольщина (ну это надеюсь сами догадаетесь где).

А ведь был ещё и образованный в середине 1917 года украинский Зеленый Клин или
Закитайщина на Дальнем Востоке (территории нынешних Хабаровского и Приморский
края, Амурской и Сахалинской областей, республики Саха-Якутия) - впоследствии
Украинская Дальневосточная Республика.

Кстати, данные переписи Российской империи 1897 года относительно этнических
украинских земель не сильно отличаются от данных переписи СССР 1926 года,
которая проводилось до принудительной руссификации.

Больше - по ссылкам:
\url{https://www.facebook.com/100017678566688/posts/363940834205199?sfns=mo},
\url{https://www.facebook.com/100017678566688/posts/369151777017438?sfns=mo} и в
комментариях к ним.

Аналогично и с нашими польскими соседями в споре об "исконно польских Волыни и
Галиции" надо ссылаться на данные как этой переписи, так и переписи
Австро-Венгерской империи 1900 года. А заодно, совсем не лишним было бы задать
им вопрос - считают ли они императора Франца Иосифа I, а также руководителей
стран Антанты - государств победителей в Первой Мировой войне: Клемансо, Вудро
Вильсона, Ллойд Джорджа и лорда Керзона с его известной линией, украинскими
националистами.

Сдаётся мне, что реакция адептов "Польска от можа до можа" не сильно будет
отличаться от реакции адептов “русского мира”.

Больше - по ссылке: \url{https://www.facebook.com/sergii.rabchynskyi/posts/161464341119517}

Так что, мой юный друг, не надо ничего придумывать. Нужно просто учить историю
не по учебникам советской ну очень средней школы, а заодно пользоваться
статистическими данными, а не рассуждениями всяких убогих невежд и неучей:)

\end{itemize}

%%%fbauth
%%%fbauth_name
\iusr{Andrew Okara}
%%%fbauth_url
%%%fbauth_place
%%%fbauth_id
%%%fbauth_front
%%%fbauth_desc
%%%fbauth_www
%%%fbauth_pic
%%%fbauth_pic portrait
%%%fbauth_pic background
%%%fbauth_pic other
%%%fbauth_tags
%%%fbauth_pubs
%%%endfbauth
 
a naxuja na latynku?

\begin{itemize}
%%%fbauth
%%%fbauth_name
\iusr{Сергей Рабчинский}
%%%fbauth_url
%%%fbauth_place
%%%fbauth_id
%%%fbauth_front
%%%fbauth_desc
%%%fbauth_www
%%%fbauth_pic
%%%fbauth_pic portrait
%%%fbauth_pic background
%%%fbauth_pic other
%%%fbauth_tags
%%%fbauth_pubs
%%%endfbauth
 
\textbf{Andrew Okara} Неужто так слабо все это сказать \enquote{на нормальном, на гражданском языке}?

\href{https://www.youtube.com/watch?v=PvR9_cmZtck}{%
А теперь скажи на нормальном языке, Видеотека, youtube, 29.01.2016%
}

\end{itemize}

%%%fbauth
%%%fbauth_name
\iusr{Наталия Приходькина-Ильяшенко}
%%%fbauth_url
%%%fbauth_place
%%%fbauth_id
%%%fbauth_front
%%%fbauth_desc
%%%fbauth_www
%%%fbauth_pic
%%%fbauth_pic portrait
%%%fbauth_pic background
%%%fbauth_pic other
%%%fbauth_tags
%%%fbauth_pubs
%%%endfbauth
 

\ii{fbicon.heart.stars}


%%%fbauth
%%%fbauth_name
\iusr{Денис Вадимович}
%%%fbauth_url
%%%fbauth_place
%%%fbauth_id
%%%fbauth_front
%%%fbauth_desc
%%%fbauth_www
%%%fbauth_pic
%%%fbauth_pic portrait
%%%fbauth_pic background
%%%fbauth_pic other
%%%fbauth_tags
%%%fbauth_pubs
%%%endfbauth
 

Такой ненависный соседям националист Корчинский похожие тезисы доносил ещё в
90х. Дон, Кубань, Белгород и Воронеж показывал на схожей карте как украинские
земли. Историю пишут победители. С былью история в книгах не часто совпадают.

Другой момент в том, что Китай постепенно аннексирует дальний восток РФ - со
временем может развалить соседа.

Ещё один момент - как Польша забирает нас без войны. Обнищавшая страна теряет
ещё и своих граждан за гроши от злотых. Только подняв экономику и избавившись
от внешнего влияния мы вернём заробитчан, Донбасс, Крым и не только. Бюджет
сильнее танков

\begin{itemize}
%%%fbauth
%%%fbauth_name
\iusr{Сергей Рабчинский}
%%%fbauth_url
%%%fbauth_place
%%%fbauth_id
%%%fbauth_front
%%%fbauth_desc
%%%fbauth_www
%%%fbauth_pic
%%%fbauth_pic portrait
%%%fbauth_pic background
%%%fbauth_pic other
%%%fbauth_tags
%%%fbauth_pubs
%%%endfbauth
 
\textbf{Денис Вадимович} 

И к чему этот весь пафос? Своё мнение о наших новоявленных националистах, в том
числе о перманентном провокаторе и предателе всех и вся - я написал в
публикации.

А экономику мы «поднимаем» все последние 30 лет, да видно «поднималка» не
работает у рукожопых. Знаете что плохому танцору мешает? «Внешнее влияние»! А
так бы он ещё как плясал. Ну прямо как Фред Астор!

\end{itemize}

%%%fbauth
%%%fbauth_name
\iusr{Світлана Самусенко}
%%%fbauth_url
%%%fbauth_place
%%%fbauth_id
%%%fbauth_front
%%%fbauth_desc
%%%fbauth_www
%%%fbauth_pic
%%%fbauth_pic portrait
%%%fbauth_pic background
%%%fbauth_pic other
%%%fbauth_tags
%%%fbauth_pubs
%%%endfbauth
 

Не будете заперечувати, що українці сьогодні, то нащадки русинів, бо Україна -
в минулому Русь?

\begin{itemize}
%%%fbauth
%%%fbauth_name
\iusr{Сергей Рабчинский}
%%%fbauth_url
%%%fbauth_place
%%%fbauth_id
%%%fbauth_front
%%%fbauth_desc
%%%fbauth_www
%%%fbauth_pic
%%%fbauth_pic portrait
%%%fbauth_pic background
%%%fbauth_pic other
%%%fbauth_tags
%%%fbauth_pubs
%%%endfbauth
 
\textbf{Світлана Самусенко} А ви уважно прочитали публікацію до того як писати комент? І про що вона?

%%%fbauth
%%%fbauth_name
\iusr{Світлана Самусенко}
%%%fbauth_url
%%%fbauth_place
%%%fbauth_id
%%%fbauth_front
%%%fbauth_desc
%%%fbauth_www
%%%fbauth_pic
%%%fbauth_pic portrait
%%%fbauth_pic background
%%%fbauth_pic other
%%%fbauth_tags
%%%fbauth_pubs
%%%endfbauth
 
\textbf{Сергій Рабчинський} Так. Моє запитання таке, тому що Ви написали, що вважаєте назву нашого народу українцями принизливою.

%%%fbauth
%%%fbauth_name
\iusr{Сергей Рабчинский}
%%%fbauth_url
%%%fbauth_place
%%%fbauth_id
%%%fbauth_front
%%%fbauth_desc
%%%fbauth_www
%%%fbauth_pic
%%%fbauth_pic portrait
%%%fbauth_pic background
%%%fbauth_pic other
%%%fbauth_tags
%%%fbauth_pubs
%%%endfbauth
 
\textbf{Світлана Самусенко} І я пояснив чому. Жоден народ в світі не
називається за назвою територіі. Саме території мають назву від назви народу,
який на них мешкає. Русь - від русі, Франкія (Франція) від франків (французів),
Англія від англійців, а не навпаки!

%%%fbauth
%%%fbauth_name
\iusr{Світлана Самусенко}
%%%fbauth_url
%%%fbauth_place
%%%fbauth_id
%%%fbauth_front
%%%fbauth_desc
%%%fbauth_www
%%%fbauth_pic
%%%fbauth_pic portrait
%%%fbauth_pic background
%%%fbauth_pic other
%%%fbauth_tags
%%%fbauth_pubs
%%%endfbauth
 
\textbf{Сергій Рабчинський} 

Щодо історичного тлумачення все зрозуміло. Але новітня історія державотворення
України будувалася саме з волі українців. ""Верховна Рада України від імені
Українського народу - громадян України всіх національностей, виражаючи
суверенну волю народу,

спираючись на багатовікову історію українського державотворення і на основі
здійсненого українською нацією, усім Українським народом права на
самовизначення,

дбаючи про забезпечення прав і свобод людини та гідних умов її життя,

піклуючись про зміцнення громадянської злагоди на землі України та
підтверджуючи європейську ідентичність Українського народу і незворотність
європейського та євроатлантичного курсу України,

(Абзац п'ятий преамбули із змінами, внесеними згідно із Законом № 2680-VIII від 07.02.2019)

прагнучи розвивати і зміцнювати демократичну, соціальну, правову державу,
усвідомлюючи відповідальність перед Богом, власною совістю, попередніми,
нинішнім та прийдешніми поколіннями, керуючись Актом проголошення незалежності
України від 24 серпня 1991 року, схваленим 1 грудня 1991 року всенародним
голосуванням, приймає цю Конституцію - Основний Закон України."- Преамбула
Конституції України.

%%%fbauth
%%%fbauth_name
\iusr{Сергей Рабчинский}
%%%fbauth_url
%%%fbauth_place
%%%fbauth_id
%%%fbauth_front
%%%fbauth_desc
%%%fbauth_www
%%%fbauth_pic
%%%fbauth_pic portrait
%%%fbauth_pic background
%%%fbauth_pic other
%%%fbauth_tags
%%%fbauth_pubs
%%%endfbauth
 
\textbf{Світлана Самусенко} 

І навіщо ви мені це пишите? Що ви намагаєтесь цим довести?

Що наші можновладці неосвічені невігласи, демогиги та брехуни, які не розуміють
значення слів, які вживають? Так це і так відомо будь-якій розумній людині!

Яка «багатовікова історія українського державатворення»? Скільки років загалом
всім державному утворенню, які використовували назву Україна, у тому числі
автономним чи псевдо на кшалт СРСР? Зараз це максимум 103 роки! А у 1991 було
ще менше. Це скільки століть? Любий школяр скаже: одне, а аж ніяк не «багато».

А про які «права і свободи людини та гідні умови її життя» ці крадіі колись
дбали? Яке таке гідне життя для людини вони майже за 30 років створили?

А яку «громадянську злагоду» вони зміцнили? Невже не бачите що рівень злоби і
ненависті у нашему суспільстві зашкалює? Один український сегмент з його
хамством чого вартий!

А що там зі «зміцнюванням соціальної та правової держави»? Які особисто у вас є
соціальні та правові гарантії? Як і де ви їх можете захистити? В нашому
українському суді? Не смішить моі капці!

Так що менше пафосу і більше розуму і здорового глузду. І вчить історію. Там знайдете багато відповідей.

Больше - по ссылке: 

\url{https://www.facebook.com/100017678566688/posts/754293851836560/?d=n}, 

\url{https://www.facebook.com/100017678566688/posts/504229466843001/?d=n}, 

\url{https://www.facebook.com/100017678566688/posts/316670825598867/}

%%%fbauth
%%%fbauth_name
\iusr{Сергей Рабчинский}
%%%fbauth_url
%%%fbauth_place
%%%fbauth_id
%%%fbauth_front
%%%fbauth_desc
%%%fbauth_www
%%%fbauth_pic
%%%fbauth_pic portrait
%%%fbauth_pic background
%%%fbauth_pic other
%%%fbauth_tags
%%%fbauth_pubs
%%%endfbauth
 

И еще несколько слов о философско-историческо-культурологически-этнографическом
вопросе почему Украина-Русь не родила своих отцов-основателей а ни в середине
XVII, а ни в начале XX и XXI веков.

Конечно можно было бы отделаться словами британского писателя, публициста,
историка и философа Томаса Карлайла (Thomas Carlyle) - "Революции часто
задумывают романтики, осуществляют фанатики, а пользуются ее плодами отпетые
негодяи (Revolutions are often initiated by idealists, carried out by fanatics
and hijacked by scoundrels)".

Однако, мне кажется, что на наших теренах к этому добавляется ещё и убогость
всей этой нашей, так называемой, политической элиты (включая, как они себя
называют, оппозиционеров), всей этой с менталитетом до информационной
(цифровой) эры замусоленной колоды "птенцов гнезда Кучмова" - "нанайских
мальчиков", вот уже более четверти века взаимовыгодно борющихся за пользование
мажоритарным пакетом ПАО "Украинское государство" (проигравший становится
миноритарием, хоть и с несколько меньшим объёмом дивидендов, но явно не
голодающим - как говаривал булгаковский Шариков: "Я без пропитания оставаться
не могу, где же я буду харчеваться?"), и с мировоззрением/миропониманием,
находящемся в лучшем случае в координатах горбочевского реформирования,
ускорения и перестройки середины 80-х годов прошлого века, с его известными
улучшить, углубить и расширить, которая увы не доросла до уровня
отцов-основателей.

Поэтому и не стали наши Майданы революциями (хоть На граните, хоть Оранжевой,
хоть Достоинства). Потому что, революция - это коренной и резкий переворот в
общественно-политических отношениях, приводящий к смене общественного
строя/системы. А у нас никого изменения системы в результате Майданов не
произошло. Одни "нанайские мальчики" сменили у "кормушки" других.

А вот почему Украина-Русь не родила своих отцов-основателей а ни в середине
XVII, а ни в начале XX и XXI веков - это действительно уже вопрос философский,
а может быть ещё исторический, культурологический и этнографический.

Конечно это вопрос комплексный и сложный. Наверное для его понимания ученые ещё
напишут свои труды, если конечно урапатриоты и вышиватники их не заклюют. Но
некоторыми своими мыслями попробую поделиться.

Народ (общество) и его элита безусловно находятся в диалектическом
взаимодействии. И в этом смысле, перефразируя известное выражение графа Жозефа
де Местра, можно сказать - каждый народ имеет ту элиту, которую он заслуживает.

С другой же стороны, формирование элиты - это процесс штучный (единичный),
кропотливый и длительный. Помните давнешний анекдот, в котором на вопрос
сколько надо учиться для того, чтобы стать по настоящему интеллигентным
человеком, был получен ответ, что для этого надо, чтобы было иметь 3 высших
образования: высшее образование вашего деда, вашего отца и ваше.

После практически поголовной гибели элиты древней Руси в битве на Ворскле 1399
году была утеряна связь поколений элит. Следующая элита Руси - литовского
происхождения хотя первоначально и приняла язык, обычаи и веру русинов (все эти
Вишневецкие, Чарторыйские, Володыевские etc.), но впоследствии ополячилась и
окотоличилась и стала элитой совсем другого, враждебного Руси государства -
Речи Посполитой, а фактически Королевства Польского.

В связи с этим (а народ и общество не возможны без элиты) возникла новая элита
- казацкая. Молодая и по настоящему ещё не сформировавшаяся она достаточно
быстро была коррумпирована и попала под влияние уже другой внешней силы -
Московского царства. Более того, эта русинская элита приняла самое активное
участие в создании Российской империи (тот же Феофан Прокопович, который был
чуть ли главным идеологом империи) и стала одним из её становых хребтов (как до
этого Речи Посполитой). Достояно напомнить, что Российскую империю фактически
создали немка - императрица Екатерина и русин - ее канцлер Безбородько.

Эта тенденция продолжилась и во времена другой империи - Советской, в которой
выходцы из Украины всегда занимали далеко не последние места (один
днепропетровский клан чего стоит).

Таким образом, всё что мы сейчас имеем в виде украинской элиты - это наследники
коррумпированных в течении более чем трёх с половиной сотен лет
российско-советской империей манкуртов, которые как в овечьи шкуры "вирядилися
у вишиванки та почали розмовляти украінською", но остались, в лучшем случае,
"агентами влияния" Москвы. Вот в этом, как я понимаю, и есть наша особенность,
а я бы даже сказал - беда:((

% -------------------------------------
\ii{fbauth.solodovnik_jurij.energodar.ukraina}
% -------------------------------------

Плюс кровавый коммунистический террор, ГУЛАГи, голодомор, ВМВ, извращенная
совковая пропаганда - и имеем выжженое поле и народ, пока еще не ставший нацией
с большой буквы.


%%%fbauth
%%%fbauth_name
\iusr{Сергей Рабчинский}
%%%fbauth_url
%%%fbauth_place
%%%fbauth_id
%%%fbauth_front
%%%fbauth_desc
%%%fbauth_www
%%%fbauth_pic
%%%fbauth_pic portrait
%%%fbauth_pic background
%%%fbauth_pic other
%%%fbauth_tags
%%%fbauth_pubs
%%%endfbauth
 
\textbf{Юрій Солодовник} Это все следствие. А причина - отсутствие элиты, отцов основателей и государства.

%%%fbauth
%%%fbauth_name
\iusr{Світлана Самусенко}
%%%fbauth_url
%%%fbauth_place
%%%fbauth_id
%%%fbauth_front
%%%fbauth_desc
%%%fbauth_www
%%%fbauth_pic
%%%fbauth_pic portrait
%%%fbauth_pic background
%%%fbauth_pic other
%%%fbauth_tags
%%%fbauth_pubs
%%%endfbauth
 
\textbf{Сергій Рабчинський} Україна-Русь.

\url{https://uk.m.wikipedia.org/wiki/Україна-Русь}

%%%fbauth
%%%fbauth_name
\iusr{Сергей Стрелец}
%%%fbauth_url
%%%fbauth_place
%%%fbauth_id
%%%fbauth_front
%%%fbauth_desc
%%%fbauth_www
%%%fbauth_pic
%%%fbauth_pic portrait
%%%fbauth_pic background
%%%fbauth_pic other
%%%fbauth_tags
%%%fbauth_pubs
%%%endfbauth
 
\textbf{Світлана Самусенко} 

Ну и зачем мне совать Wikipedia? Для вас беллетристика (или чтиво для
домохозяек, как вам больше нравиться) - истина в последней инстанции? Это раз.

Ну а коль ссылаетесь на что-то, то будьте добры в этом хотя бы разобраться. Я
пишу у термине Русь, существовавшим с X века, о термине Украина - с XVI! И к
чему ваш гибридный новояз XIX Украина-Русь? Что вы этим хотите доказать? И это
два.



%%%fbauth
%%%fbauth_name
\iusr{Світлана Самусенко}
%%%fbauth_url
%%%fbauth_place
%%%fbauth_id
%%%fbauth_front
%%%fbauth_desc
%%%fbauth_www
%%%fbauth_pic
%%%fbauth_pic portrait
%%%fbauth_pic background
%%%fbauth_pic other
%%%fbauth_tags
%%%fbauth_pubs
%%%endfbauth
 
\textbf{Сергій Рабчинський} 

11 січня 2021 року у Відні відбулись урочисті заходи зі святкування 60-річчя
онука останнього Австро-Угорського імператора, Голови будинку Габсбургів Карла
фон Габсбурга - Лотарінгського. Ігор Смешко привітав із 60-річчям Президента
Пан-Європейського Союзу Карла фон Габсбурга. Серед офіційних титулів якого,
поряд із імператорським, є також титули короля Галичини і Волині та герцога
Буковини. Олег Кікавець:"Ось так. Надзвичайно приємно дізнатись, що династійна
нитка не загублена і до цього часу пов'язує нашу Україну-Русь з династією
Габсбургів..." 

\href{https://youtu.be/8PVoUoxJq5w}{%
Ігор Смешко привітав із 60-річчям Президента Пан-Європейського Союзу Карла фон Габсбурґа, %
Ihor Smeshko, youtube, 11.01.2021%
}

%%%fbauth
%%%fbauth_name
\iusr{Сергей Стрелец}
%%%fbauth_url
%%%fbauth_place
%%%fbauth_id
%%%fbauth_front
%%%fbauth_desc
%%%fbauth_www
%%%fbauth_pic
%%%fbauth_pic portrait
%%%fbauth_pic background
%%%fbauth_pic other
%%%fbauth_tags
%%%fbauth_pubs
%%%endfbauth
 
\textbf{Світлана Самусенко} 

Вам не настобридло демонструвати власну неосвіченість?

По-перше, якщо так полюбляєте Wikipedia, то непогано було б звернути увагу на
офіційне титулування Карла фон Габсбурга за її англійською версією:

His ancestral titles are Imperial Highness Archduke of Austria, Prince of
Hungary, Bohemia, Dalmatia, Croatia, Slavonia, Galicia, Lodomeria, Illyria, and
Jerusalem, etc.; Grand Duke of Tuscany and Cracow, Duke of Lorraine, Salzburg,
Styria, Carinthia, Carniola and of the Bukovina, Grand Prince of Transylvania,
Margrave of Moravia, Duke of Upper and Lower Silesia, of Modena, Parma,
Piacenza and Guastalla, of Auschwitz and Zator, of Teschen, Friuli, Ragusa and
Zara, Princely Count of Habsburg and Tyrol, of Kyburg, Gorizia and Gradisca,
Prince of Trent and Brixen, Margrave of Upper and Lower Lusatia and in Istria,
Count of Hohenems, Feldkirch, Bregenz and Sonnenberg, etc.; Lord of Trieste,
Cattaro and in the Windic March, Grand Voivode of the Voivodeship of Serbia,
etc., etc., etc.

Із багатьох його титулів наступні пов’язані з теритроією сьогоднішньої України:
Prince of Galicia, Lodomeria, Duke of the Bukovina! Так що нема серед його
\enquote{офіційних титулів}, а ні \enquote{короля}, а ні \enquote{Галичини та Волині}, які \enquote{висмоктані
з пальця} українською Wikipedia.

Чому \enquote{висмоктані}? Тому що українська Wikipedia не містить жодних посилань на
джерело цієї дурні. А ось в англійській - джерело титолування наведено: Franz
Gall: Österreichische Wappenkunde. Handbuch der Wappenwissenschaft. (Austrian
heraldry. Handbook of coat of arms science - German) Böhlau,
Vienna/Cologne/Weimar 1992, ISBN 3-205-05352-4, p 105 f.

А тому, \enquote{короля Галичини та Волині} свог, як казав один персонаж
Акуніна, \enquote{ви можете запхати собі ... у портмоне. Пардон}.

А по-друге, ще раз, для особливо обдарованих, українською: “якщо посилаєтесь на
щось, то будьте ласкаві в цьому хоча б розібратися”!

Жоден з численних титулів голови Будинку Габсбурґ-Лотаринґен не містить
\enquote{притягнену за вуха} таким же як ви невігласом Олегом Кікавцем Україну-Русь! Ну
не було ніякої \enquote{України-Русі} у складі Австро-Угорщини.

%%%fbauth
%%%fbauth_name
\iusr{Світлана Самусенко}
%%%fbauth_url
%%%fbauth_place
%%%fbauth_id
%%%fbauth_front
%%%fbauth_desc
%%%fbauth_www
%%%fbauth_pic
%%%fbauth_pic portrait
%%%fbauth_pic background
%%%fbauth_pic other
%%%fbauth_tags
%%%fbauth_pubs
%%%endfbauth
 
\textbf{Сергей Стрелец} 

По-перше, якщо Ви себе вважаєте освіченим, то Ваше хамське ставлення до мене
про це не свідчить. 

По-друге, не на Вікіпедію останній раз посилалася. 

По-третє, вчіться аналізувати сказане іншими і логічно мислити. Особливо, коли
говорить людина рівня, яку знають і поважають у світі. 

І четверте. Україна це Русь. Українці - нащадки русинів. Читайте великих
українців, видатних істориків, письменників різних країн і українських теж.

Наприклад, відомому українцю Котляревському більше 250 років. Якщо Ви
відчуваєте приниження від того, що є українцем, то напевне Ви ним не є, Сергій
Рабчинський чи Сергей Стрелец, чи хто Ви там є насправді. Дискусії моїй з Вами
кінець. Дякую, що прочитали.


%%%fbauth
%%%fbauth_name
\iusr{Helen Oszeczinszki}
%%%fbauth_url
%%%fbauth_place
%%%fbauth_id
%%%fbauth_front
%%%fbauth_desc
%%%fbauth_www
%%%fbauth_pic
%%%fbauth_pic portrait
%%%fbauth_pic background
%%%fbauth_pic other
%%%fbauth_tags
%%%fbauth_pubs
%%%endfbauth
 
\textbf{Світлана Самусенко} Осві ченість, без Т !

%%%fbauth
%%%fbauth_name
\iusr{Сергей Стрелец}
%%%fbauth_url
%%%fbauth_place
%%%fbauth_id
%%%fbauth_front
%%%fbauth_desc
%%%fbauth_www
%%%fbauth_pic
%%%fbauth_pic portrait
%%%fbauth_pic background
%%%fbauth_pic other
%%%fbauth_tags
%%%fbauth_pubs
%%%endfbauth
 
\textbf{Світлана Самусенко} 

Как там у Михаила Афанасьевича по поводу таких умников как вы -

\enquote{Вы, Шариков, чепуху говорите и возмутительнее всего то, что говорите
ее безапелляционно и уверенно}. Это раз.

\href{https://youtu.be/zO2il_HLMpk}{%
Вы, Шариков, чепуху говорите, TheJustSnake, youtube, 07.06.2014%
}

Не с Wikipedia взяли эту дурню про "короля Галичини та Волині"? неужто сами
придумали? Значит у вас с украинской Wikipedia мысли случайно сошлись...😞 Это
два.

Может Олега Кикавца и \enquote{знають і поважають у світі}, однако Google \enquote{людини рівня
Кікавця} не известна. Так что и мне прийдется остаться в неведении. И это три.

И ещё раз для особо одаренных: \enquote{З початку XIX століття, аби підкреслити свою
окремішність від росіян (великоросів), харківська українська інтелігенція стала
використовувати назву \enquote{українці} як загальний етнонім замість старої назви
\enquote{русини}. Ця нова самоназва остаточно затвердилася у Наддніпрянській Україні
після національно-визвольного руху 1917 - 1920 років та радянської
українізації. Теж саме відбулося на Галичині і Буковині у 1918 - 1940 роках
завдяки діяльності українських націоналістичних організацій} - это с вашей
любимой украинской Wikipedia
(\url{https://uk.wikipedia.org/wiki/Українці})

Так что не мог Иван Петрович Котляревский 1769 года рождения быть украинцем по
определению! Ну не придумали к моменту его рождения ещё этого слова для
обозначения народа. А это четыре.

Позволю себе ещё процитировать нашего великого земляка, который не считал себя украинцем:

\enquote{Вы стоите на самой низкой ступени развития, [...] Вы ещё только формирующееся,
слабое в умственном отношении существо, все ваши поступки чисто звериные, и вы
в присутствии... людей с университетским образованием позволяете себе с
развязностью совершенно невыносимой подавать какие-то советы космического
масштаба и космической же глупости... [...] ...вам нужно молчать и слушать, что вам
говорят. Учиться и стараться стать хоть сколько-нибудь приемлемым членом
социального общества}?

И это все о таких как вы, убогих...:))

\href{https://www.youtube.com/watch?app=desktop&v=Wm2nN9hfxoY}{%
Вы стоите на самой низкой ступени развития, Филипп Преображенский, youtube, 09.10.2018%
}

И да, чрезвычайно признателен, что больше не будите меня донимать своими глупостями. Успехов!

%%%fbauth
%%%fbauth_name
\iusr{Сергей Стрелец}
%%%fbauth_url
%%%fbauth_place
%%%fbauth_id
%%%fbauth_front
%%%fbauth_desc
%%%fbauth_www
%%%fbauth_pic
%%%fbauth_pic portrait
%%%fbauth_pic background
%%%fbauth_pic other
%%%fbauth_tags
%%%fbauth_pubs
%%%endfbauth
 
\textbf{Helen Oszeczinszki} Але вміє «аналізувати сказане іншими і логічно мислити»!

%%%fbauth
%%%fbauth_name
\iusr{Helen Oszeczinszki}
%%%fbauth_url
%%%fbauth_place
%%%fbauth_id
%%%fbauth_front
%%%fbauth_desc
%%%fbauth_www
%%%fbauth_pic
%%%fbauth_pic portrait
%%%fbauth_pic background
%%%fbauth_pic other
%%%fbauth_tags
%%%fbauth_pubs
%%%endfbauth
 
\textbf{Світлана Самусенко} і додайте" боячись вразити росіян......

%%%fbauth
%%%fbauth_name
\iusr{Світлана Самусенко}
%%%fbauth_url
%%%fbauth_place
%%%fbauth_id
%%%fbauth_front
%%%fbauth_desc
%%%fbauth_www
%%%fbauth_pic
%%%fbauth_pic portrait
%%%fbauth_pic background
%%%fbauth_pic other
%%%fbauth_tags
%%%fbauth_pubs
%%%endfbauth
 
\textbf{Helen Oszeczinszki} Дякую! Виправила. На українську перейшла повністю досить недавно.

\end{itemize}

%%%fbauth
%%%fbauth_name
\iusr{Тетяна Кузьмішкіна}
%%%fbauth_url
%%%fbauth_place
%%%fbauth_id
%%%fbauth_front
%%%fbauth_desc
%%%fbauth_www
%%%fbauth_pic
%%%fbauth_pic portrait
%%%fbauth_pic background
%%%fbauth_pic other
%%%fbauth_tags
%%%fbauth_pubs
%%%endfbauth
 

\ifcmt
  ig https://scontent-cdg2-1.xx.fbcdn.net/v/t1.6435-9/136328840_729160488000271_8909858510489776434_n.jpg?_nc_cat=108&ccb=1-3&_nc_sid=dbeb18&_nc_aid=0&_nc_ohc=IIs3z5fzVwoAX9gnKef&_nc_ht=scontent-cdg2-1.xx&oh=81283ee860e00ada8dbdfa462aa067d3&oe=612C5AFF
  width 0.3
\fi

%%%fbauth
%%%fbauth_name
\iusr{Сергей Рабчинский}
%%%fbauth_url
%%%fbauth_place
%%%fbauth_id
%%%fbauth_front
%%%fbauth_desc
%%%fbauth_www
%%%fbauth_pic
%%%fbauth_pic portrait
%%%fbauth_pic background
%%%fbauth_pic other
%%%fbauth_tags
%%%fbauth_pubs
%%%endfbauth
 

\href{https://youtu.be/TKIDJ_HJees}{%
Куди зникли сотні тисяч українців після перепису 1897 року в Російської Імперії, %
Відеожурнал Версаль, %
youtube, 13.03.2021%
}

%%%fbauth
%%%fbauth_name
\iusr{Сергей Стрелец}
%%%fbauth_url
%%%fbauth_place
%%%fbauth_id
%%%fbauth_front
%%%fbauth_desc
%%%fbauth_www
%%%fbauth_pic
%%%fbauth_pic portrait
%%%fbauth_pic background
%%%fbauth_pic other
%%%fbauth_tags
%%%fbauth_pubs
%%%endfbauth
 
No comments...

\ifcmt
  ig https://scontent-cdg2-1.xx.fbcdn.net/v/t1.6435-9/167215623_138682481520461_5286168359166894270_n.jpg?_nc_cat=104&ccb=1-3&_nc_sid=dbeb18&_nc_ohc=oLIUR3-CrSMAX8ILrTb&_nc_ht=scontent-cdg2-1.xx&oh=8ec0a887451043c91731df1a25e2fa24&oe=612E752E
  width 0.3
\fi

%%%fbauth
%%%fbauth_name
\iusr{Helen Oszeczinszki}
%%%fbauth_url
%%%fbauth_place
%%%fbauth_id
%%%fbauth_front
%%%fbauth_desc
%%%fbauth_www
%%%fbauth_pic
%%%fbauth_pic portrait
%%%fbauth_pic background
%%%fbauth_pic other
%%%fbauth_tags
%%%fbauth_pubs
%%%endfbauth
 
Поширюю, це лікбез!

\end{itemize}

