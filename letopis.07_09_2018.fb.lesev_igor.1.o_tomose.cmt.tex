% vim: keymap=russian-jcukenwin
%%beginhead 
 
%%file 07_09_2018.fb.lesev_igor.1.o_tomose.cmt
%%parent 07_09_2018.fb.lesev_igor.1.o_tomose
 
%%url 
 
%%author_id 
%%date 
 
%%tags 
%%title 
 
%%endhead 
\subsubsection{Коментарі}

\begin{itemize} % {
\iusr{Павло Тарасюк}
Друже, це той рідкий випадок, коли я з Тобою погоджуюсь

\begin{itemize} % {

\iusr{Игорь Лесев}
это тот редкий случай, когда Паша лайкнул пост... надо скрин сделать)

\iusr{Алла Фидельская}
Я и не сомневалась, что заокэанськи "друзи" подсуетятся

\iusr{Павло Тарасюк}
\textbf{Алла Фидельская} а я, щиро кажучи, мав сумніви

\end{itemize} % }

\iusr{Alexandr Bondarenko}

Всё конечно может быть. Может будет томос, может не будет томоса. Совершенно
ясно одно Украина в новом своём виде, в виде Антироссии, полностью
ориентирована на Запад. И будет здесь только одна католическая церковь,
напрямую подчинённая Папе, и никакой униатской тем более православной. ни с
томосом ни без томоса. Только католицизм и только латиница. без вариантов.

\begin{itemize} % {
\iusr{Игорь Лесев}

а с чего здесь должен появиться католицизм? Вы заблуждаетесь. Будет своя версия
британского протестантизма (англиканства), только на православный манер

\iusr{Матвей Кублицкий}
\textbf{Игорь Лесев} но в любом случае папа римский выиграл

\iusr{Павло Тарасюк}
Якщо не буквально, то суть процесу Ви зрозуміли чудово

\iusr{Alexandr Bondarenko}
\textbf{Игорь Лесев} 

как всегда на этом белом свете. что-то где-то появляется. где его раньше не
было - исключительно волевым импульсом тех кто в этом заинтересован. Конечно
без вариантов это будет страна с латиницей со всеми этими этерами и прочим и
конечно только католическая. На какое-то время может и томос пригодиться, хотя
бы ради того. чтобы архитектурно православные соборы привести к внешнему облику
близкому к католическим стандартам.


\iusr{Игорь Лесев}
\textbf{Матвей Кублицкий} скажем так, Кирилл и Ко точно проиграли

\iusr{Alexandr Bondarenko}
\textbf{Игорь Лесев} Кирилл и ко тут объекты. Объекты по определению в проигрыше.

\iusr{Матвей Кублицкий}
\textbf{Игорь Лесев} странная увереность в данном моменте времени

\iusr{Игорь Лесев}
\textbf{Alexandr Bondarenko} 

я точно не самый большой спец в церковных делах, но думаю, что наши в бабловые
дела никакого Папу подпускать не будут... у них есть уже свои приходы среди
греко-католиков, там пусть и кормятся. А делить недвиж и паству УПЦ МП сумеют
как-то без католиков

\iusr{Матвей Кублицкий}
\textbf{Игорь Лесев} ты пока не осознаешь основное... и ты прав - логически!. украина ушла от единого православия - и в самом деле - нафига оно нужно....

\iusr{Alexandr Bondarenko}
\textbf{Игорь Лесев} 

это не моё конечно дело, но возможно вы слишком сильно фокусируетесь на
"бабловых делах". Когда речь идёт о большой политике, а это именно большая
политика, деньги не просто уходят на второй план, их просто нет. от слова -
совсем. Рвать все экономические связи с РФ, покупать российский газ на реверсе
у словаков и донбасский уголь через Пенсильванию, это очень слабо связано с
выгодой и бабловыми делами, для страны в целом)) В общем не буду спорить, я
лишь высказал свою точку зрения - я совершенно на 100\% убеждён, что никакой
кириллицы тут не будет и никакого православия ни в каком виде тут ни будет.

\end{itemize} % }

\iusr{Матвей Кублицкий}

знаешь, Игорь, тоже не понимал - зачем Путин так активно поддерживает
православную церковь. и тут украина - показатель. да чтоб не было сект. не было
турчинова и яценюка

\begin{itemize} % {
\iusr{Игорь Лесев}

любая секта, как и террористическая организация - это вопрос признания и
легитимации. Баптисты у нас формально тоже секты, а в Штатах - полноправная
церковь. УПЦ КП признают на международном уровне- и вот тебе уже раскольники
превращаются в полноправную церковь. Протестанты тоже когда-то сектантами
считались, а всю Европу в 16 веке кровью чуть окропили и все - договорились кто
где бабло косит)

\iusr{Матвей Кублицкий}
\textbf{Игорь Лесев} я не буду спорить с твоей точкой зрения. просто когда в стране много религий - будет много проблем. не вы первые
\end{itemize} % }

\iusr{Марина Прохорова}
\url{https://www.facebook.com/diak.kuraev/posts/2238844756143606}

\iusr{Stella Aksenova}
Просто очередная трещина- раскол...не больше.
Москва может говорить, что угодно. Что делает? - Вот вопрос.
И ответ сразу.

\begin{itemize} % {
\iusr{Stella Aksenova}
Да. И танец -возмущение (от Захаровой)был 2 года назад и совершенно по другому поводу)

\iusr{Stella Aksenova}
Не в обиду... просто совсем уж... как сми
\end{itemize} % }

\iusr{Олег Хавич}

Никакого бунта не будет, даже вонять не будут. Большая часть клира УПЦ МП
перейдет в ЕППУЦ, большей части мира - глубоко похристу.

\begin{itemize} % {
\iusr{Игорь Лесев}

да, плюс/минус тоже думаю, что так и будет... но у нас же в строю всегда есть
черные лебеди, которые умеют сделать даже там, где все само идет в руки -
непредвиденные кровавые разборки. Это тоже надо учитывать

\iusr{Олег Хавич}
\textbf{Игорь Лесев}, УПЦ МП не смогла защитить НИ ОДИН храм, даже в селах, где её верующие составляли 80-90\%. Подставили щёки - подставят и ж*пы.

\iusr{Игорь Лесев}

так себе наезд... а как ты им предлагаешь защищать храмы? Они вообще должны
этим заниматься? У нас же нельзя позвонить в ментовку, чтобы твои права
отстояли. По себе вот знаешь, сидя в Польше)

\iusr{Олег Хавич}
\textbf{Игорь Лесев}, 

я прекрасно знаю, что храмы захватывали при поддержке мусоров. Но против тысячи
бабушек ни одни мусора не устоят - это ж не меня одного потоптать. А бабушек
никто не поднимал, и сейчас не поднимет.

\iusr{Игорь Лесев}
\textbf{Олег Хавич} я как-то изучал материалы по захвату храмов на западной. Бабушки сколько там могут стоять? Ну день, ну два. А так - загоняют молодняк с битами, те заводят своего батюшку и все - будьте здоровы. Дальше идут юридические срачи, но по факту храмом владеет уже новая контора.

\iusr{Игорь Лесев}
\textbf{Олег Хавич} но это жесткий вариант. А чаще просто батюшку перекупают, он объявляет переход вместе "с прихожанами и храмом". И все - досвидос

\iusr{Олег Хавич}
\textbf{Игорь Лесев}, сейчас с этих батюшек ещё плату за вход в ЕППУЦ брать будут  @igg{fbicon.smile} 

\iusr{Natali Maslova}
\textbf{Marina Keller}, как донецкий аэропорт.... "так не доставайся же ты никому"...

\end{itemize} % }

\iusr{Дмитрий Коломийченко}
Зря вы так пишите об ответе. Когда он будет, нам всем мало не покажется.

\begin{itemize} % {
\iusr{Игорь Лесев}
и согласитесь, от того что и как я пишу, мало зависит, кому и что мало не покажется))

\iusr{Дмитрий Коломийченко}
\textbf{Игорь Лесева} Безусловно, ведь история объективна. И да всё это уже здесь было после унии. Ничего нового, но от этого еще печальнее.

\iusr{Gavrilova Irina}
будет, будет... Русские медленно запрягают да быстро едут. Слыхали, да

\iusr{Diana Wanadi}
Ой все...
Дада будет будет ответ.
Вам на голову срут уже четвертый год откровенно, а вы все на западных партнеров оглядываетесь
\end{itemize} % }

\iusr{Владимир Колесниченко}

Будет галопом деградировать церковная жизнь. Те священники, что окажутся
принципиальны уедут, а бизнесмены от церкви просрут паству. Останется что-то
лубочное с жовтоблакытным, и в это верить не будут.

\begin{itemize} % {
\iusr{Олег Хавич}
Больше 90\% священиков останутся. Епископов поменьше - но не потому, что не захотят, а филаретовцы не всех возьмут.

\iusr{Владимир Колесниченко}
А через какое-то время, как всегда, через десятилетие, вернётся настоящая церковь, типа РПЦ

\iusr{Владимир Колесниченко}
\textbf{Олег Хавич} 

в каждом приходе, наверняка есть свои подвижники искрение, за которыми идут
верующие, не просто забубонные бабушки. Так те будут жертвовать собой, в чем
сила христианства. Можем увидеть духовных подвижников, истинно верующих. И,
кстати, христианство прогрессирует когда находиться под давлением.

\end{itemize} % }

\iusr{Елена Несветайлова}
Томос это что?

\begin{itemize} % {
\iusr{Игорь Лесев}
термос, только виртуальный

\iusr{Елена Несветайлова}
Я серьёзно спрашиваю.)

\iusr{Игорь Лесев}

Ну вот тебе выдержка из вики. "Томос - духовный термос, предназначенный для
утоления жажды оральным способом во избежания гнева Божьего. Выдается в
Стамбуле. Изготовление - Тайвань. При получении - пить до дна".

\iusr{Елена Несветайлова}
Так можно двинуться умом.

\iusr{Игорь Лесев}
Вот-вот. Люди за термос сегодня готовы друг друга убивать. Сам удивлен.

\iusr{Елена Несветайлова}
Прекращай иронизировать. Ну, подпишут. И хрен с ним. Утопическая страна.
\end{itemize} % }

\iusr{Gavrilova Irina}

А как же Новинский?

\begin{itemize} % {
\iusr{Игорь Лесев}
фундаментальный вопрос... действительно, зачем все это без Новинского
\end{itemize} % }

\iusr{Станислав Бочкур}
Какое дело Кремлю до ваших Томосов? Хоть иудаизм принимайте, лишь бы газовую трубу не трогали до поры до времени.

\iusr{Mike Schachmann}

выскажусь по данному вопросу. никогда раньше не слыхал за этот томас. как и 99
проц остальных тоже. но это неважно. понятно, шо всем этим мурикам на печерске
надо дальше поднимать температуру в котелке. и они расчитывают на этого томаса.
марнi сподiвання. я жил в укр. когда филя организовал свою лавочку и было
достаточно шума из за этого. помню как по крещаьику ходили новые "епископы". но
сейчас. дай хоть томаса андерса. это большинству уже до задницы. во первых
потому, шо народу реально меньше стало. во вторых это не перекроет тему войня и
курс гривны.


\iusr{Матвей Кублицкий}

половина православных в мире - россияне. так что автокефалия Украины будет
означать конец константинопольского патриархата в его нынешнем виде. Вряд ли на
это кто решится. будут спекулировать этой темой и всё

\begin{itemize} % {
\iusr{Игорь Лесев}

"Константинопольский патриархат" - это фикция. Патриархат на территории
мусульманской страны - это уже смешно. Как староверы в современной России.
Название есть, а самого понимания что это и с чем едят - никакого. Решение
принималось в Вашингтоне и оно сугубо геополитическое. Так что флюиды добра
посылай не к несчастному Варфоломею))

\iusr{Матвей Кублицкий}

Значит произойдет историческая справедливость и центр православия переедет в
Москву. А константинопольский патриархат станет ничем. Чем он и должен был
стать после присяге папе римскому перед нашествием османов. А на Украине так и
останется две церкви.

\iusr{Игорь Лесев}

центр православия и так в России, и расколу в мировом православии таки быть...
а вот по Украине пока много вопросов, вообще, тут предсказывать тяжело - всякое
может пойти, от "ничего не изменилось" до новой кровищи

\iusr{Матвей Кублицкий}
\textbf{Игорь Лесев} на счет "предсказывать тяжело" - согласен полностью. много неожиданного, много неизвестной информации. Так что продолжим наблюдать. В любом случае автокефалия - очередное ухудшение ситуации на Украине. То есть то что нужно Порошенко....
\end{itemize} % }

\iusr{Владислав Клочков}
Ну разрыв народов, жалко что ли

\iusr{Василий Челишев}

Весь движ "майдана" был не борьба с коррупцией, нет. Это было уйти от России
любыми способами. Хай чорт, абы нэ москаль. И пока у них получается при
какой-то апатии в Москве.

\iusr{Igor Maximov}
Есть мнение

\href{https://nahnews.org/1006988-ukrainskii-raskol-cerkvi-ili-kievskaya-avtokefaliya-ssha-tolknuli-varfolomeya-na-voinu-s-moskvoi}{%
Раскол церкви, или Томос для Украины: США толкнули Варфоломея на войну с Москвой, nahnews.org, 12.10.2018%
}


\end{itemize} % }
