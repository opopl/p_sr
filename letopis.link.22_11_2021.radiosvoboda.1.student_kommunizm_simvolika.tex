% vim: keymap=russian-jcukenwin
%%beginhead 
 
%%file link.22_11_2021.radiosvoboda.1.student_kommunizm_simvolika
%%parent 25_11_2021.fb.uljanov_anatolij.1.usaid_ukr_propaganda.cmt
 
%%url 
 
%%author_id 
%%date 
 
%%tags 
%%title 
 
%%endhead 

\href{https://www.radiosvoboda.org/a/komunizm-totalitarnyy-rezhym-symvoilka/31573870.html}{%
Студент похизувався радянським прапором і образив захисників України. Чи буде кримінальне провадження?, %
Галина Терещук, radiosvoboda.org, 22.11.2021%
}

\begin{multicols}{2}
\ifcmt
  ig https://gdb.rferl.org/B3465B44-D536-4907-A96E-423C749BDDA8_w1023_r1_s.jpg
  @width 0.4
\fi

\figCaption{Під час акції біля Конституційного суду України, який взявся перевіряти закон
про декомунізацію на відповідність Конституції. Київ, 21 травня 2019
року}

\textSelect{Заборонена комуністична символіка, зневажливі висловлювання про українську мову
і події в Україні ‒ деякі молоді люди використовують їх для власної
популяризації у соціальних мережах, як інструмент привернення уваги. Окремі
дописи стають підставою для відкриття кримінального провадження. А більшість
залишається поза увагою правоохоронців. Переважно реагують, коли є звернення
громадян. Чому деякі молоді люди, які не жили в радянський час, використовують
комуністичні символи на одязі чи на своїх сторінках у соцмережах?}

19-річний киянин рік тому зайшов у львівську кав’ярню у шапці з кокардою, на
якій були зображені серп і молот у червоній зірці. Його зауважив небайдужий
львів’янин і викликав поліцію. Слідчі відкрили провадження за статтею 436
(виготовлення, поширення комуністичної, нацистської символіки та пропаганда
комуністичного та націонал-соціалістичного (нацистського) тоталітарних режимів)
Кримінального кодексу України.

У травні цього року управління СБУ у Львівській області затримала 35-річного
мешканця Золочева, який намагався продати червоний прапор із зображенням Лєніна
і герба СРСР. За повідомленням СБУ, радянську символіку могли використати 9
травня. Поліція задокументувала зловмисника в момент збуту забороненої речі.
Теж було відкрите кримінальне провадження за статтею «виготовлення і поширення
комуністичної символіки».

Цьогоріч Галицький районний суд Львова виніс вирок 22-річному чоловіку за
носіння футболки з комуністичною атрибутикою. Львів'янина затримали у центрі
міста торік у вересні.

Ще один скандал був спричинений львівською блогеркою, яка заявила, що
ненавидить ветеранів АТО, бо вони мають право пільгового проїзду у транспорті,
також молода особа назвала окупований Крим ‒ російським. Після сотень обурень
людей, вона вибачилася. «Я каюсь за те що говорила! Дуже дякую людям, які
захищають нашу Україну від російської агресії», – написала вже у своєму
дописі-вибаченні.

\headTwo{Студент політехніки образив захисників}

Днями соціальні мережі вибухнули емоціями щодо відео, оприлюднених студентом
університету «Львівська політехніка». Хлопець навчається на другому курсі
інституту «Архітектури та дизайну». Він поширив відео у ТікТок із зображенням
на стіні в його помешканні радянських атрибутів, якими він хизується.

Також студент опублікував фрагмент відео з маршу, присвяченого Дню захисника
України, назвавши його учасників «тваринами». На тлі звучить музика, яка
асоціюється для людей, які жили в СРСР, з російською програмою у «Світі
тварин». Та й студент російською мовою зробив досить образливий підпис на
адресу учасників маршу. Молодий чоловік з батьками проживає поблизу Львова, в
селі Оброшино. Свої розповіді веде російською.

Оскільки у львівській політехніці заняття проводять дистанційно, застати
студента в університеті не вдалося. Радіо Свобода розмовляло з його батьком
Ігорем Волосецьким. Але чоловік дуже коротко сказав, що «проблеми теоретично
немає», що син десь учинив неусвідомлено, батькові важко було підібрати слова.
Пообіцяв детальніше порозмовляти, але більше так і не відгукнувся на телефонні
дзвінки кореспондентки Радіо Свобода.

\headTwo{Провели бесіду і готують кодекс культури}

У соціальних мережах піднялась хвиля обурень щодо студента і дописувачі
вимагали навіть його звільнення з університету. Деякі відео студент прибрав.

«Керівництво інституту «Архітектури та дизайну» провело дві бесіди з батьками
студента, говорили з ним телефоном. Студент вів свою приватну сторінку і якоїсь
агресії чи недоброї поведінки на парах він не демонстрував. Тобто, відреагувати
на це важко. Ця ситуація обурює всіх в університеті та інституті і тому ми
зустрілись з батьками, вони сказали, що даватимуть раду. Хлопець вступив із
високими балами на державну форму навчання, але зараз його успішність
знизилась. У нас є кодекс корпоративної культури, який поширюється на
працівників. Зараз ми будемо розробляти і для студентів», ‒ каже Наталія
Павлишин, керівниця центру комунікацій університету «Львівська політехніка».

Радіо Свобода поцікавилось думкою студентів університету «Львівська
політехніка» щодо змісту відео, які оприлюднив молодий чоловік, який навчається
поруч із ними у виші:

– Я чула про цей випадок, але не дуже вникала. Обговорювали цю тему. Мабуть, з
минулого це відгомін. Можливо, це якась психологічна травма з дитинства, зараз
бракує уваги. У суспільстві зараз у моді безкарність, роби що хочеш і межі
дозволеного дуже і дуже розтягнулись і на такому можемо собі дозволити
хайпанути.

– Я до такого ставлюсь погано.

– Я ніяк на таке не реагую, це якась класична провокація. Комусь щось важливе у
минулому. Це інтернет, перехідний вік і кожен хоче привернути до себе увагу
нетолерантними темами. Колись це не було так нахабно, а тепер стерті фільтри, ‒
говорили студенти.


\end{multicols}
