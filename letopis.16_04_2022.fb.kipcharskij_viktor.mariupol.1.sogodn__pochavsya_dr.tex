%%beginhead 
 
%%file 16_04_2022.fb.kipcharskij_viktor.mariupol.1.sogodn__pochavsya_dr
%%parent 16_04_2022
 
%%url https://www.facebook.com/permalink.php?story_fbid=pfbid0CBq5Yv2HKbu4VeacsAcsMr9b3ZyfM5mTypmkZcJpe9UKcUwfmRepfcNaviPx98Gul&id=100006830107904
 
%%author_id kipcharskij_viktor.mariupol
%%date 16_04_2022
 
%%tags mariupol,mariupol.war
%%title Сьогодні почався другий місяць з того часу, як ми доїхали із блокованого Маріуполя до Вільної України
 
%%endhead 

\subsection{Сьогодні почався другий місяць з того часу, як ми доїхали із блокованого Маріуполя до Вільної України}
\label{sec:16_04_2022.fb.kipcharskij_viktor.mariupol.1.sogodn__pochavsya_dr}

\Purl{https://www.facebook.com/permalink.php?story_fbid=pfbid0CBq5Yv2HKbu4VeacsAcsMr9b3ZyfM5mTypmkZcJpe9UKcUwfmRepfcNaviPx98Gul&id=100006830107904}
\ifcmt
 author_begin
   author_id kipcharskij_viktor.mariupol
 author_end
\fi

Сьогодні почався другий місяць з того часу, як ми доїхали із блокованого
Маріуполя до Вільної України.

Ми повернулися до таких зручних подарунків цивілізації, як світло, зв'язок,
тепло, газ, магазини, лікарні, аптеки тощо.

Ніби усе добре, але болить на душі... Що з рідним містом? Що з друзями, що там
залишилися?

Друзі час від часу з'являються у мережі - і це викликає неабияку радість.

А от фото та відео зруйнованого міста - навіюють сумні думки...

Півжиття тому я працював у Польщі, у Ченстохові. Кожного місяця я їздив до
Варшави по зарплатню. Після вирішення службових проблем, я мав кілька годин до
вечірнього потягу до Ченстохови, які витрачав на мандрівки містом. 

Старе Място (історичний центр Варшави), був вщент зруйнований під час
придушення повстання. Суцільні руїни. Сім'я мого польського друга (ще до його
народження) кілька років мешкала під сходами до півниці (підвалу) - це єдине,
що залишилося від будинку і могло хоч якось прикрити згори від дощу та снігу.
По боках дідусь Юрека с синами (батьком та дядком Юрека) з уламків цегли
зробили якісь стіни і там жили кілька років. Дідусь Юрека був чоботарем, і
годував сім'ю, ремонтуючи людям взуття.

На початку 60-х років польські кінематографісти зняли стрічку про Кохання
польської дівчини (бійчині спротиву) та німецького хлопця (солдата вермахту).
Стрічка про трагічне кохання молодих людей отримала нагороду міжнародного
кінофестивалю і її творці вирушили країнами світу представляти її.

Десь у Південній Америці (здається, у Аргентині) їх запитали про те, як вони
зробили такі натуральні декорації знищеного міста. Поляки відповіли, що вони и
знімали у Варшаві.

- Але як зробили такі масштабні декорації?

- А не було декорацій - знімали простота у місті.

Люди ,вдумайтеся: після звільнення Варшави минуло 15+ років, центр міста ще у
руїнах!!!

Весь цей час поляки збирали фотографії людей (сімейні знімки, тощо), по яких
ретельно відтворювали образ знищеного міста, аби відбудувати його у історичному
вигляді.

Дуже зворушливо читати: "будинок відбудовано, але фрагмент стіни на другому
поверсі - справжній).

Я щиро вірю, що по війні Старий Маріуполь так саме буде відбудовано - з
історичними підробицями.

Може колись ми, або наші діти та онуки пройдуть Старою частиною Маріуполя,
спустяться Земською до будинку Гампера...

Вірю, що мрії здійснюються...

%\ii{16_04_2022.fb.kipcharskij_viktor.mariupol.1.sogodn__pochavsya_dr.cmt}
