% vim: keymap=russian-jcukenwin
%%beginhead 
 
%%file 09_11_2021.fb.nitsoi_larysa.1.movna_stijkistj
%%parent 09_11_2021
 
%%url https://www.facebook.com/larysa.nitsoi/posts/4844518618913565
 
%%author_id nitsoi_larysa
%%date 
 
%%tags bandera_stepan,diktant,mova,ukraina,ukrainizacia
%%title Як виробити в себе мовну стійкість?
 
%%endhead 
 
\subsection{Як виробити в себе мовну стійкість?}
\label{sec:09_11_2021.fb.nitsoi_larysa.1.movna_stijkistj}
 
\Purl{https://www.facebook.com/larysa.nitsoi/posts/4844518618913565}
\ifcmt
 author_begin
   author_id nitsoi_larysa
 author_end
\fi

Я сиджу за столом Степана Бандери, набираюся сили і (я ж учителька) пишу вам
усім домашку, яку треба виконувати протягом дня. Сьогодні, в День української
мови, завдання усім таке: утриматися повністю від іншомовного контенту і
слухати, дивитися, слати мемчики/відоси/кліпчики, картинки у «вайбері»,
поширювати дописи, читати й розмовляти - ЛИШЕ УКРАЇНСЬКОЮ. 

\ifcmt
  ig https://scontent-mia3-2.xx.fbcdn.net/v/t1.6435-9/254779794_4844527302246030_2363577860636419151_n.jpg?_nc_cat=110&ccb=1-5&_nc_sid=730e14&_nc_ohc=rKcxh5n6IY4AX9LDGO-&_nc_ht=scontent-mia3-2.xx&oh=532ed7409277802e21bd1da34749e84f&oe=61AF9CB9
  @width 0.3
  %@wrap \parpic[r]
  @wrap \InsertBoxR{0}
\fi

Якщо вам пришлють відосик, посиланнячко чи іншу хрінь знаєте якою мовою,
видаляйте, не дивлячись. 

Нам усім треба напрацьовувати МОВНУ СТІЙКІСТЬ. У народів, які пережили геноцид,
мовна стійкість слабка – і це стосується нас. А мовна стійкість народів, які
були імперіями, сильна. Мовна стійкість – це ментальна стійкість. Це стійкість
індентична. Мовна стійкість – це альфа і омега нашої нації, нашої сили, нашого
утвердження на своїй землі. 

 @igg{fbicon.triangle.red.up} Як виробити в себе мовну стійкість? Три простих правила. Ви. Ваш простір.
Ваше оточення.

1. ВИ. Слідкуйте за собою, говоріть лише українською.

2. ВАШ ПРОСТІР. Почистіть свій простір (книги, телевізор, інтернет, мемчики,
відосики – все звучання і відеоряд) від самі знаєте якої мови, про яку не можна
говорити. Безжально перемикайте канали, хвилі, сторінки і переходьте на ті, де
звучить українська. 

3. ВАШЕ ОТОЧЕННЯ. Просіть своє оточення розмовляти З ВАМИ українською і не
присилати ВАМ жодних котиків, які підписані самі знаєте якою мовою. Бо так, як
вони впєрєді сваіх танков поставили дітей і жінок, так вони впєрєді котиків та
іншої симпатичної лабуди ставлять свою мову. Це гібридна зброя проти вас. Щоб
вас привчити до постійної її присутності у вашому просторі засобами милих
картинок.

 @igg{fbicon.triangle.red.up} Сьогодні, в День української мови, саме час узятися за себе, свій простір і
своє оточення.

Буде здорові та мовно стійкі. Усіх обіймаю. Слава Україні!

Поширте, будь ласка, цей допис між своїми рідними й друзями, нехай наша мовна
стійкість міцнішає.

\ii{09_11_2021.fb.nitsoi_larysa.1.movna_stijkistj.cmt}
