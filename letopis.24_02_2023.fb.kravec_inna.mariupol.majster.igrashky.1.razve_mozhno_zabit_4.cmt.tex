% vim: keymap=russian-jcukenwin
%%beginhead 
 
%%file 24_02_2023.fb.kravec_inna.mariupol.majster.igrashky.1.razve_mozhno_zabit_4.cmt
%%parent 24_02_2023.fb.kravec_inna.mariupol.majster.igrashky.1.razve_mozhno_zabit_4
 
%%url 
 
%%author_id 
%%date 
 
%%tags 
%%title 
 
%%endhead 

\qqSecCmt

\iusr{Viktoria Ihnatenko}

Тю, они же такие добрые... пришли миротворцы освобождать людей от жизни и
жилья. Они же спасатели, спасали Вас...

Конечно это сарказм... нет слов, чтобы поддержать Вас, т. к. что можно сказать
в таком случае?! Страшно обидно и больно. Ненависть на века.

Боюсь спросить, это ваш дом в посте на фото?

\begin{itemize} % {
\iusr{Инна Кравец}
\textbf{Виктория Игнатенко} 

да, дом мой и улица моя, и дальше районы такие же... Не знаю как там сейчас,
долго наш район считался серой зоной и туда не пускали. Сейчас, вроде бы,
открыли, но информации у меня нет(

\iusr{Viktoria Ihnatenko}
\textbf{Инна Кравец} ужас....т.е Вам некуда возвращаться....боже,,,,мрази и твари

\iusr{Инна Кравец}
\textbf{Виктория Игнатенко} да, мы уже не вернёмся, некуда.

\iusr{Viktoria Ihnatenko}
\textbf{Инна Кравец} не знаю какие подобрать слова, чтобы поддержать😔
\end{itemize} % }

\iusr{Sveta Svetlana}

Очень страшно было там находиться 2 месяца, было страшно за жизни детей, не
верилось, что такое может с нами происходить. Было больно все потерять, тяжело
покинуть родных людей и родные земли, начать все сначала, учить чужой язык.
Тяжело нам и теперь. Но только надежда на будущее, которое обещает Бог,
помогает справляться со всеми трудностями.

\begin{itemize} % {
\iusr{Инна Кравец}
\textbf{Sveta Svetlana} это точно! Только надеждой и живём, делаем всё, что от нас зависит и ждём) Вы большие молодцы! Я очень рада, что вы выбрались)
\end{itemize} % }

\iusr{Яна Рыжикова}

Инночка, хочу только обнять🫂♥️ Поймут только те кто там был. А остальным и не
нужно ничего рассказывать. Человек который не пережил этот Ад до конца никогда
не поймёт...

\begin{itemize} % {
\iusr{Инна Кравец}
\textbf{Яна Рыжикова} обнимаю взаимно 🥰 самое главное, что мы живы, что мы понимаем друг друга и у всех нас есть надежда)❤️

\iusr{Яна Рыжикова}
\textbf{Инна Кравец} Да!🙏🏿

\end{itemize} % }

\iusr{Сергей Штамбур}

70 дней. Дальше города уже не было.

\begin{itemize} % {
\iusr{Инна Кравец}
\textbf{Сергей Штамбур} 😢😢😢😢😢😢

\iusr{Инна Кравец}
\textbf{Сергей Штамбур} 😢😢😢😢
\end{itemize} % }

\iusr{Natalya Brueva}

Держитесь Инночка, у Вас Все́ Будет Хорошо!!!

\begin{itemize} % {
\iusr{Инна Кравец}
\textbf{Наталья Бруева} спасибо большое! У нас всё хорошо. Мы очень хотим, чтобы у вас всё было хорошо, мы очень переживаем о вас(((
\end{itemize} % }

\iusr{Валерий Гришко}

Скоро, в Новом Мире мы об этом забудем.

\iusr{Natalia Ulyakina}

Точно сказано, кто там не был, тот этого не поймёт. Инночка, Тарас, Марк мы вас
помним и любим, то что мы пережили это был действительно АД. И лагерь беженцев
пройденная школа, только мы там задержались на пять дней и спали на картоне,
ели один раз и то когда хлеб 🍞 увидели, плакали не поверили, что держим его в
руках. После пережитого в подвале, где жуткий холод,голод и смертью пахло,
лагерь беженцев показался раем, там где было тепло. Этого кошмара никогда не
забудешь! И только с помощью Бога мы смогли оттуда выбраться.

\begin{itemize} % {
\iusr{Инна Кравец}
\textbf{Natalia Ulyakina} 

Дорогие, мы тоже вас помним и любим))) так страшно, что столько родных людей
пережили страшные события ((( И так печально вспоминать сколько людей плакало
над хлебом и как прижимали его к сердцу 😢 А лагерь с картонками действительно
был раем после всего... Этот кошмар даже не придумать просто так. А свет в
лагере, а тишина без взрывов, это было невероятным счастьем заснуть на полу, но
в безопасности!(((

\end{itemize} % }

