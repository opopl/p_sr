% vim: keymap=russian-jcukenwin
%%beginhead 
 
%%file 12_01_2022.fb.fb_group.story_kiev_ua.1.kiev_restorany_sssr.cmt
%%parent 12_01_2022.fb.fb_group.story_kiev_ua.1.kiev_restorany_sssr
 
%%url 
 
%%author_id 
%%date 
 
%%tags 
%%title 
 
%%endhead 
\zzSecCmt

\begin{itemize} % {
\iusr{Tetyana Kilesso}

Цукерки \enquote{Надія} я ніколи не куштувала. Відбивні з свинини так, замовляла вцьому
ресторані. А зараз замовила б заливну осетрину з лимоном.


\iusr{Светлана Манилова}

В упоминаемый Вами период в рестораны я еще не ходила. @igg{fbicon.smile} Впервые в \enquote{Столичном}
оказалась в новогоднюю ночь 1988 года и второй раз в июне 1989 года, когда
отмечали с группой окончание университета. Впечатления хорошие, как и
обслуживание с кухней.

\iusr{Вадим Ляшенко}

Многие блюда сейчас у нас в диковину.

\iusr{Светлана Ветрова}

Приємні спогади цін за порції можна було спокійно посидіти со спиртним на
червонец на двох але треба туди попасти а це іноді було скрутно

\iusr{Вадим Ляшенко}
 @igg{fbicon.thumb.up.yellow} 

\iusr{Катя Бекренева}

Не часто посещала рестораны) В основном на свадьбах) По хронологии: Пектораль,
два раза Краков, Млын (защита), Верховина, Украина (просто так), Братислава,
два раза на юбилеях, Столичный (просто так)

Это советское время. Понравилась кухня в Млыне.

\begin{itemize} % {
\iusr{Света Медецкая}
\textbf{Катя Бекренева} Верховина, которая на Павловской была???)))

\iusr{Катя Бекренева}
\textbf{Света Медецкая} нет, на 5-ой просеке, в Святошино
\end{itemize} % }

\iusr{Ирина Таировна}
Крутой был ресторан на то время,
Говорят,у директора ресторана сОбака ходила с золотым ошейником.

\iusr{Катя Бекренева}
\textbf{Ирина Таировна} и с бриллиантами) Шефиня выходила её выгуливать по ночам. Громкое дело было.

\iusr{Ирина Озерянская}

нас с чоловіком не пустили до цього ресторану, бо були у джинсах  @igg{fbicon.smile}  1981р.

\iusr{Таня Сидорова}

На мене з меню ні чого не справило враження. А інтер'єр був дуже простий.

\iusr{Басанько Любов}
Я была в ресторане \enquote{Столичном} после окончания института.
Это был 1965 год.
Сдавали, как сейчас помню, по 10 рублей.
Меню, конечно, не помню, но помню, что стол ломился от еды.
А как было весело!
Мы же тогда были совсем юные и беззаботные. Эх...

\iusr{Алла Вайнерман}
Наверное, нам так уже не жить!

\begin{itemize} % {
\iusr{Ольга Красюк}
\textbf{Алла Вайнерман} Конечно, не жить) мы ведь немножко стареем) в юности все иначе) но у каждого времени своя прелесть)

\iusr{Алла Вайнерман}
\textbf{Ольга Красюк}. Я о ценах, а вы о возрасте.
\end{itemize} % }

\iusr{Михайлина Голуб}

\enquote{Столичный} мы с друзьями иногда посещали в студенческие годы - 68-72-ой. Да,
студенты иногда могли себя это позволить. Цены там были демократичнее, чем,
например, в \enquote{Динамо} или \enquote{Театральном}. Правда, если честно сказать
обслуживание и кухня соответствовали этой демократичности. И еще в этом
ресторане в оркестре играл на трубе мой бывший однокласник Гриша. Вообще-то он
с блеском окончил музыкальную школу по класу скрипки, но она ему надоела. Он
окончил эстрадно-цирковое училище - разговорный жанр. По окончанию весь их курс
ра пределили в Архангельск. Там он с успехом выступал, но все же захотел
вернуться в Киев. А тут оказалось, что артисты-разговорники не нужны. А семью
кормить нужно было. Он очень быстро освоил трубу (вообще он играл на многих
инструментах) и стал, как тогда говорили, рестораным лабухом. Мы с
одноклассниками иногда приходили в \enquote{Столичный}, чтобы послушать его. А потом он
уехал за океан и в \enquote{Столичный} мы больше не ходили. И все-таки было жалко,
когда его здание снесли.

\begin{itemize} % {
\iusr{Ольга Красюк}
\textbf{Михайлина Голуб} Добрая история... жизненная..

\iusr{Михайлина Голуб}
\textbf{Ольга Красюк}  @igg{fbicon.grin}  @igg{fbicon.face.grinning.big.eyes}  @igg{fbicon.face.grinning.smiling.eyes}  @igg{fbicon.beaming.face.smiling.eyes}  @igg{fbicon.laugh.rolling.floor} 
\end{itemize} % }

\iusr{Ігор Кулаков}
Завжди всміхаюся, коли бачу підпис калькулятора

\iusr{Ed Edward}
Ресторан «Лейпциг».........?

\iusr{Наташа Гресько}

иногда с однокусниками обедали в Столичном: там были самые дешевые \enquote{комплексы}.
хотя обычно так далеко не забирались - ходили в Ленинградскую. или в Украину -
если \enquote{при деньгах}) а еще, там же Крыжинка была рядом!


\iusr{Катя Бекренева}

А вот самые вкусные и дефицитные конфеты нам приносила родственница, она
работала в ателье \enquote{Коммунар}. В открытой продаже таких я не встречала.


\iusr{Viktor Kozhevnikov}

В Столичном был фирменный салат и напиток Троянда. На основе его меню написали
книгу \enquote{современная украинская кухня}.

\begin{itemize} % {
\iusr{Леонид Красиловский}
\textbf{Viktor Kozhevnikov} да салат Столичный- типа Оливье но вместо колбасы и гороха было мясо и вроде как пекинской капусты ( которую ещё наверное не выращивали...)))
\end{itemize} % }

\iusr{Инна Валентиновна}

Мне все блюда в \enquote{Столичном} нравились: заливное, котлета по - киевски,
салат столичный и т. д. А что его снесли, я считаю преступлением перед нашей
памятью.  Никому он не мешал, разве что делкам строительного бизнеса.

\begin{itemize} % {
\iusr{Николай Гребенкин}
\textbf{Инна Валентиновна} 

А плюс к тому, это здание закрывало Крещатик от холодных северных ветров. Это
здание было щитом Крещатика @igg{fbicon.face.smiling.hearts} 

\iusr{Инна Валентиновна}
\textbf{Николай Гребенкин} согласна 100\%.
\end{itemize} % }

\iusr{Марина Набока}

конфеты \enquote{Надежда} даже не помню, а \enquote{птичье молоко} люблю до сих пор. Ресторан
Столичный тоже помню, неоднократно бывала и с друзьями, и просто поесть, на
Крещатике не сильно много было доступных мест общепита, а в Столичном часто
были банкеты и ресторан переставал быть доступным. Такой себе ресторан средней
руки, но нормально поесть можно было

\begin{itemize} % {
\iusr{Лариса Павловская}
\textbf{Marina Naboka} Я часто заходила в столовку на первом этаже, а в самом ресторане у нас свадьба была

\iusr{Марина Набока}
\textbf{Лариса Павловская} во-во, это было популярное место, я там тоже на чьей-то свадьбе была, уже и не помню чьей.

\iusr{Лариса Павловская}
\textbf{Marina Naboka} Жаль, что нет его уже
\end{itemize} % }

\iusr{Ольга Dzhun}

Обратитесь в ресторан Будьмо, руководство этого ресторана было у истоков
ресторанного дела в Киеве, они вам, думаю, помогут

\iusr{Анна Земко}

У нас був випуск інститутській у Столичному в червні 1990 року) Меню взагалі не
пам'ятаю( Памятаю якісь дивани півколами навколо столів (а може теж плутаю  @igg{fbicon.grin} )

\iusr{Надежда Коваленко}

А мы в \enquote{Столичном} свою свадьбу справляли.

Тогда, а это 27 апреля 1985 год, удивлялись в ресторане, что мы сами себе
торжество заказывали, без родителей. Сами платили, сами выбирали блюда. Хотя до
этого не посещали рестораны. Была мечта... вот такая, наивная и простая.

Там 3 зала было и общий танцпол. Оркестранты сказали, что мы самая красивая
пара ....Было очень приятно. А еще, нам предложили выбрать любую песню, чтоб
они для нас сыграли бесплатно, честь такой красивой пары.

Мы были молодыми и счастливыми... и не думали, что счастье долгим не бывает...
Чернобыль и все изменится

\begin{itemize} % {
\iusr{Ирина Мундирова}
\textbf{Надежда Коваленко} И у нас такая же история 02. 09. 1984 г.

\iusr{Людмила Зобенко}
\textbf{Надежда Коваленко} А какую же вы выбрали песню?

\iusr{Надежда Коваленко}

Я отдала предпочтение на выбор песни своей мвме, ту, которая ей на тот момент
нравилась.

Прошло столько лет, и не важна песня, важнее те эмоции, что мы все чувствовали
в тот момент, в тот день, в то время.... они/чувства/ незабываемы.

Все были живы, счастливы и молоды! А эмоции, песни тех лет - были прекрасны!
\end{itemize} % }

\iusr{Григорий Владимирович}

В 70- х участвовал в разработке \enquote{Дело Ёлкиной} ( Шахиня) в ресторане
Столичный.

\begin{itemize} % {
\iusr{Zoya Pshenichny}
\textbf{Григорий Владимирович} 

Думаю мало тех, кто помнит об этом деле.

А вообще классный был ресторан, у меня там был выпускной техникума.

Соседка рабо ала в кафе на первом этаже, кассиршей.

\begin{itemize} % {
\iusr{Larisa Kodryansky}
\textbf{Zoya Pshenichny} 

У меня была свадьба в кафе на первом этаже в 1972 году. Спустя год или два, уже
не помню, моего папу вызывали по делу \enquote{шахини}.

\iusr{Zoya Pshenichny}
\textbf{Larisa Kodryansky} 

Тогда многих вызывали свидетелями, особенно у кого были банкеты.

В музее МВД после дела \enquote{Шахини}, были выставлены конфискованные у нее вещи,
драгоценности.

\iusr{Larisa Kodryansky}
\textbf{Zoya Pshenichny} 

Нервы тогда потрепали изрядно. Я уже и не помню, чем закончилось дело \enquote{шахини}.

\end{itemize} % }

\iusr{Катя Бекренева}
\textbf{Григорий Владимирович} 

Я в году так 77-78 читала книгу, мне было тогда 15, не помню чьи воспоминания
об уголовных делах, написал кто-то из следователей, там было и дело Шахини
(шефини), врезалось в память.

\iusr{Надежда Владимир Федько}
\textbf{Григорий Владимирович} Пам'ятаю)

\begin{itemize} % {
\iusr{Григорий Владимирович}
\textbf{Надежда Владимир Федько} Мы даже со следаком по её просьбе возили на допрос на Лукьяновку лимоны.

\iusr{Надежда Владимир Федько}
\textbf{Григорий Владимирович} Як я розумію, то Ви працювали в ОБХСС ))
\end{itemize} % }

\iusr{Zoya Pshenichny}
\textbf{Larisa Kodryansky} 

Пересажали и ресторан трест иАвтоматторг.

Она долго отсидела, но вышла.

Там даже были расстрельные статьи.

\begin{itemize} % {
\iusr{Григорий Владимирович}
\textbf{Zoya Pshenichny} По Пушкарскому Автоматторг отдельная история.

\iusr{Zoya Pshenichny}
Но приблизительно в одно время.

\iusr{Надежда Владимир Федько}
\textbf{Zoya Pshenichny} Шахиня померла в колонії, відсидівши 8 років.

\iusr{Zoya Pshenichny}
\textbf{Надежда Владимир Федько} Вы правы,я уже побзабыла столько лет прошло.

\iusr{Надежда Владимир Федько}
\textbf{Григорий Владимирович} 

Можливо Ви пам'ятаєте \enquote{справу Давидова}? Це десь 1971-й рік... Поділ.

Він \enquote{працював} за схемою Остапа Бендера. Досьє на керівних працівників
торгівлі - пропозиція викупити за 20\%...

\iusr{Григорий Владимирович}
\textbf{Надежда Владимир Федько} Нет, я ещё служил в СА.

\iusr{Григорий Владимирович}
\textbf{Надежда Владимир Федько} А Пушкарский вышел после 10 лет и умер больным уже на воле.

\iusr{Надежда Владимир Федько}
\textbf{Григорий Владимирович} Я знаю...
\end{itemize} % }

\end{itemize} % }

\iusr{Сергей Борткевич}
Херово мы жили!

\iusr{Ольга Линникова}
За 3 рубля можно было посидеть вечером

\iusr{Татьяна Аверкиева}

Ещё на Крещатике было кафе где подавали очень вкусное жаркое в горшочке.
Название кафе к сожалению не помню.

\iusr{Елена Харченко}

Училась в КТЭИ и часто сдачу экзаменов отмечали в ресторанах. «Днепровский»,
«Млын», «Динамо», «Столичный», «Дубки», «Краков», «Украина», «мыслывець». А в
«Лейпциге» у нас был выпускной. В 80-х г. на левом берегу открылись рестораны
«Червона Рута» и «Олимпиада-80». Они работали то ли до 3-х часов ночи, то ли до
утра (точно не помню). Но тогда это были очень модные места. И еще «Горыще»

\begin{itemize} % {
\iusr{Владимир Мареев}
\textbf{Елена Харченко} в Лейпциге часто бывало настоящее немецкое пиво из ГДР бутылочное.

\iusr{Марина Набока}
Горыще очень модное место было, попасть было трудно...

\iusr{Alla Zgurzhnitsky}
\textbf{Marina Naboka} никак не вспомню где был этот ресторан

\iusr{Елена Харченко}
А в Дубках пела Любовь Успенская

\iusr{Alla Zgurzhnitsky}
\textbf{Elena Kharchenko} до 77 года

\iusr{Елена Харченко}

Ресторан Прага на ВДНХ всегда был рестораном \enquote{высокого класса} с
хорошей кухней и прекрасным обслуживание. Да и сейчас, пожалуй, такой же

\end{itemize} % }

\iusr{Галина Шевченко}

Я что-то пропустила? Первый раз слышу о конфетах \enquote{Надежда}... В рестораны
ходила редко, хотя это действительно было доступно...

\begin{itemize} % {
\iusr{Alla Zgurzhnitsky}
\textbf{Галина Шевченко} я тоже, хотя большой любитель. Наверное привозили

\iusr{Людмила Гнынюк}
\textbf{Галина Шевченко} и я удивилась... впервые слышу
\end{itemize} % }

\iusr{Alla Zgurzhnitsky}
Динамо наверное был из лучших по кухне и музыке.

\begin{itemize} % {
\iusr{Владимир Мареев}

По кухне и музыке наверное лучшим был ресторан в гостинице Киев.

\begin{itemize} % {
\iusr{Alla Zgurzhnitsky}
\textbf{Владимир Мареев} 

не согласна. Динамо был лучше. Профессиональные музыканты. То что готовили в
Динамо не готовили ни в одном другом ресторане. Куриные фаршированные ножки,
жульены с грибами.


\iusr{Tatyana Shpak}
\textbf{Alla Zgurzhnitsky} Жюльены с грибами и курицей готовили и в \enquote{Интуристе}, и весьма прилично.

\iusr{Владимир Мареев}
\textbf{Alla Zgurzhnitsky} на вкус и цвет товарищей нет. В Киеве во многих ресторанах готовили отлично. Киев, Днепр, Украина....да что там говорить.

\iusr{Alla Zgurzhnitsky}
\textbf{Владимир Мареев} как-то Интурист прошёл мимо, так же как и Днепр, где была один раз. Не помню, что там было.

\iusr{Alla Zgurzhnitsky}
\textbf{Владимир Мареев} в общем-то да. Готовили хорошо не только в центральных. Был такой скромненький ресторанчик за Октябрьским универмагом на Брест-литовском, назывался Пектораль. Такой вкусной грибной солянки нигде больше не ела. Работала недалеко, ходили иногда на обед.

\iusr{Владимир Мареев}
\textbf{Alla Zgurzhnitsky} да, помню такой. Бывал по моему.

\iusr{Владимир Мареев}
\textbf{Alla Zgurzhnitsky} в Интуристе такой не большой, уютный ресторан был.

\iusr{Alla Zgurzhnitsky}
\textbf{Владимир Мареев} на Ленина? Вечером сбегали с занятий и направлялись, где были знакомые, чтобы наверняка. Как-то до интуриста не доходили.

\iusr{Ілона Луценко}
\textbf{Владимир Мареев} Подавали самые вкусные комплексные обеды.

\iusr{Ілона Луценко}
\textbf{Владимир Мареев} Отменная была залевная осетрина.
\end{itemize} % }

\end{itemize} % }

\iusr{Владимир Мареев}

В Интуристе на ул. Ленина подавали замечательные котлеты по Киевски. Пожалуй
лучшие в городе. 70 е годы.

\iusr{Ілона Луценко}
\textbf{Владимир Мареев} Да очень вкусные.

\iusr{Alla Zgurzhnitsky}

Был ещё ресторан Москва, сейчас Украина. У меня там были знакомые официанты.
Можно было попасть вечером в выходные. Можно было взять второе блюдо и 100
грамм за 5-7 рублей. Часто отмечали там дни рождения. Банкет до 15 рублей с
человека, как ресурсы позволяют.

\begin{itemize} % {
\iusr{Alla Kobak}
\textbf{Alla Zgurzhnitsky} , 15 рублей были не такие уж и мылые деньги. Не все могли себе это позволить. Так же как и сейчас.

\iusr{Alla Zgurzhnitsky}
\textbf{Alla Kobak} не каждый день. Вместо того чтобы стоять у плиты, а потом убирать за гостями, было приятно провести вечер в ресторане. Тем более, что гости знали подарочный код. Подарки деньгами только. В те времена ничего достойного в магазине нельзя было купить за 10-15 рублей.

\iusr{Alla Zgurzhnitsky}
До 15.

\iusr{Ілона Луценко}
\textbf{Alla Zgurzhnitsky} У меня в ресторане Москва была свадьба. Хорошая была кухня.
\end{itemize} % }

\iusr{Tatianz Moysa}
Там для своих в 80-е годы пекли миндальное печенье, Лидия Мироновна Ровнер для нас їх замовляла

\iusr{Lyudmila Logginova}

Я занималась спортивной гимнастикой и во время сборов нам выдавали талоны на
питание. Не помню сколько это в ценовом выражении. Но! В Столичном эти талоны
мы отоваривали сладостями: шоколадками, пироженками... За что и любили не
ресторан, а кафе на первом этаже.

\iusr{Oleg Kheyfets}
\textbf{Lyudmila Logginova} 2.5 р в сутки

\iusr{Hennadii Tsurkov}

Не совсем понимаю ностальгию за какими-то московскими конфетами. В совке были
чековые магазины, в Киеве - \enquote{Каштан}. Коньяк \enquote{Наполеон} - 13
чеков, шоколадные наборы - 1- 2,5 чека. Курс чеков у барыг был 1,7 - 1,8 к
деревянному. Не для простых совков, разумеется. \enquote{Ах, Запад. Не пот, а
запах} (С)

\ifcmt
  ig https://scontent-frt3-2.xx.fbcdn.net/v/t39.30808-6/271752599_338898611424252_2166613003451509305_n.jpg?_nc_cat=101&ccb=1-5&_nc_sid=dbeb18&_nc_ohc=IuXfM8HnZEIAX-HlKbs&_nc_ht=scontent-frt3-2.xx&oh=00_AT8Ri6DtfmBbU_AhZzH5tz90w5_gICz-7xtLxb--lp-s2w&oe=61E5892B
  @width 0.4
\fi

\iusr{Надежда Владимир Федько}

Их нравы)

\ifcmt
  ig https://scontent-frt3-1.xx.fbcdn.net/v/t39.30808-6/271762388_4902887099770709_8581125607043554006_n.jpg?_nc_cat=104&ccb=1-5&_nc_sid=dbeb18&_nc_ohc=4G7GBtIfI_AAX9JTv2L&_nc_ht=scontent-frt3-1.xx&oh=00_AT8eQKoBNFijkpq0T40oTPy80h82mcKFWwJm3TL1Xu4JBA&oe=61E51044
  @width 0.4
\fi

\begin{itemize} % {
\iusr{Alla Zgurzhnitsky}
\textbf{Nadegda Volodymyr Fedko} Это интересный документ. Забыли написать где, кто, и что исполняли на еврейских свадьбах в той же Юности.
\end{itemize} % }

\iusr{Тамара Кизя}

В1969 году в этом ресторане отмечали свадьбу в одном из залов. Все было
прекрасно: много гостей,, университетских друзей, много музыки и много-много
счастья. Теперь только воспоминания по черно-белым фотографиям. Это было
недавно, это было давно... В 70-е внимательно следили за \enquote{делом Елкиной}.

\begin{itemize} % {
\iusr{Надежда Владимир Федько}
\textbf{Тамара Кизя} А на фотографіях інтер'єр зали видно?

\iusr{Тамара Кизя}
\textbf{Nadegda VolodymyrFedko} к сожалению нет, в основном действо

\iusr{Надежда Владимир Федько}
\textbf{Тамара Кизя} Жаль. Дуже рідко можна побачити фото з інтер'єром залів...
\end{itemize} % }

\iusr{Валентина Валентина}
 @igg{fbicon.heart.eyes} 

\iusr{Наталия Педос}

Ми у 1975 році святкували закінчення КПІ. Складалися по 10 крб. і були дуже
задоволені всім, що було на столі.

Тепер про \enquote{Столичний} тільки спогади залишилися...


\iusr{Irina Fon}

Стандартный набор-салат столичный, буженина и коньяк. Так отдыхали. Не знаю
какой это год, но при средней зп 150 р заоблочные цены.

\begin{itemize} % {
\iusr{Alla Zgurzhnitsky}
\textbf{Irina Fon} не каждый же день

\begin{itemize} % {
\iusr{Irina Fon}
\textbf{Alla Zgurzhnitsky} все равно. При зп 150, взять за 99 осетрину. Ту мач. Джинсы стоили 100р, так их носили по три года.  @igg{fbicon.smile} 

\iusr{Alla Zgurzhnitsky}
\textbf{Irina Fon} 150 - это выше средней. Где вы видели в магазине осетрину за 99?

\iusr{Irina Fon}
\textbf{Alla Zgurzhnitsky} вы о тогда или о сейчас? Представте, минималка сейчас 4000 грн. Взять осетрину за 3200? Нет таких цен. Кг осетрины стоит 1400. Кило! А там максимум 150 гр.

\iusr{Alla Zgurzhnitsky}
\textbf{Irina Fon} я вообще сейчас в другой стране. Я говорю о временах Советского Союза. 70е-80е годы. С этого тема начиналась.

\iusr{Irina Fon}
\textbf{Alla Zgurzhnitsky} вы не согласны, что это очень и очень дорого 99р за 150гр осетрины при зп 150р выше среднего?

\iusr{Yuriy Austin}
\textbf{Alla Zgurzhnitsky} 

Девочки, не спорьте, как говорится. Обед в ресторане на четверых с бутылкой
водки и бутылкой сухого вина, обходился немногим более 100 рублей. Это конец
70-х, начало 80-х. Обычная зарплата молодого, в то время человека, была 120 -
140 рублей, минус 13\% подоходного налога. Из этого видно, что сходить в
ресторан тогда было чрезвычайно дорого.


\iusr{Alla Zgurzhnitsky}
\textbf{Irina Fon} а где вы видели такие цены?

\iusr{Irina Fon}
\textbf{Alla Zgurzhnitsky} в прeйскурантe на картинкe под постом, посмотритe @igg{fbicon.face.upside.down} 

\iusr{Irina Fon}
\textbf{Yuriy Austin} 

имeнно в этот ресторан черезвычайно дорого. В Лейпциге мы сидели в четвером за
50 рублей. А тут одна осетрина 99. Капец


\iusr{Yuriy Austin}
\textbf{Irina Fon} 

Я уже и забыл, где этот Лейпциг был... Зарплата советского человека была так
составлена, чтобы мы никогда не забывали, кто мы такие есть. Поэтому все
немножко приробатывали, приворовывали или вязали на дому спицами. Кто что мог и
умел. Я, немножко ювелирил, но такой приработок был во первых опасен, во вторых
легко пропивался.


\iusr{Irina Fon}
\textbf{Yuriy Austin}  @igg{fbicon.grin}  я и сейчас вяжу вохотку. Раньше да, это была необходимость, без которой не выжить. В голодные 90-е я даже так подрабатывала, вязала свитера на заказ. Помогало выживать.

\iusr{Irina Fon}
\textbf{Yuriy Austin} Лейпциг на углу Прорезной и Владимирской, такое красивое старое здание увенчаное башенкой

\iusr{Людмила Гаценко}
\textbf{Yuriy Austin}

Прекрасний ресторан німецької кухні \enquote{Лейпциг}, в якому в першому у Києві почали
відзначати Новий рік цілу ніч для відвідувачів, був у сумнозвісному красивому
початку 20 ст. червоному будинку, який от уже котрий рік стоїть огороджений
парканом і пустий на розі Володимирської і Прорізної.

\iusr{Yuriy Austin}
\textbf{Людмила Гаценко} 

Да, я уже вспомнил, конечно. Меня там, даже, один раз арестовали во время
\enquote{Андроповских} рейдов по дневным кинотеатрам и ресторанам. Пришлось в
трёхдневный срок устраиваться на работу. Три месяца после этого на киностудии
Довженко декорации строил. @igg{fbicon.grin} 

\iusr{Надежда Владимир Федько}
\textbf{Irina Fon} ШАНОВНІ! Вибачте, але ви читати вмієте? Порція осетрини коштувала 99 копійок!
Дивіться уважно меню!

\iusr{Irina Fon}
\textbf{Надежда Владимир Федько} мы уже все поняли. Дочитайте до конца а потом выступайте
\end{itemize} % }

\iusr{Yuriy Austin}
А мы с женой и детьми, уехали в Америку в 1993-ем. Тоже помогло выжить.

\begin{itemize} % {
\iusr{Irina Fon}
\textbf{Yuriy Austin} 

у нас была тоже возможность, но родители болели, невозможно бвло их ни тащить
ни оставить, так и живем. В Америку уже не уедешь, а Израиль так себе вариант.
@igg{fbicon.grin} 

\iusr{Yuriy Austin}
\textbf{Irina Fon} 

Мои родители приехали в США на год позже чем мы. И довольно счастливо прожили
здесь более 20-ти лет. Тоже болели, конечно. А кто в старости не болеет? Мой
отец был немец и мы могли уехать в Германию без никаких проблем. Но мама-
еврейка и отец говорил, что своих еврейских детей он в Германию не повезёт. В
Израиль не хотела моя жена, украинка. Америка была компромиссом для нас всех.

\iusr{Надежда Владимир Федько}
\textbf{Yuriy Austin} А ми нікуди не збиралися їхати з України. І нічого не трапилося. Вижили) І до 2021-го жили нормально.
\end{itemize} % }

\iusr{Irina Fon}

Сейчас и тут неплохо. По крайней мере пока война не у порога, лучше, чем в
Израиле. Уважения ср стороны цивилизованого мира, конечно меньше, но не нужно
прятаться в бункере от бомб. Пока.

\iusr{Oksana Dekhtiar}
\textbf{Irina Fon} 

Ирусь, мне кажется, что там указана цена 99 коп. Посмотри - там в начале
прочерк стоит. Т.е. ноль рублей 99 коп.

\begin{itemize} % {
\iusr{Irina Fon}
\textbf{Oksana Dekhtiar} осетрина то? Не может быть

\iusr{Oksana Dekhtiar}
И все же это 99 коп.

\iusr{Irina Fon}
\textbf{Oksana Dekhtiar} там селедка 19.не копеек же. Срсиска стоила 50коп в кафе

\iusr{Oksana Dekhtiar}
\textbf{Irina Fon} Ир, думаю, что копеек. Я вспоминаю свое ресторанное прошлое. Горячее стоило не дороже 2,5 руб. Закуска - вполне могла стоить 99 коп. Наценка была ограничена. К тому же, мы не знаем какой выход у блюда. Возможно, 100г.

\iusr{Irina Fon}
\textbf{Oksana Dekhtiar} посмотри, там всюду прочерк, это просто тире перед ценой. Осетрина не может стоить рубль. Повторяю, сосиска в кафе стоила 50коп. А тут ресторан.

\iusr{Oksana Dekhtiar}
\textbf{Irina Fon} а на других страницах вместо прочерка - цифра.

\iusr{Irina Fon}
\textbf{Oksana Dekhtiar} хочешь сказать, что селедка стоила 19коп? В ресторане? Ну, тогда это цены 50-х. Но никак не 70-х

\iusr{Oksana Dekhtiar}
\textbf{Irina Fon} все возможно. Мы были в первом классе в 75м и можно только догадываться. В те времена была ресторанная наценка - не более 20 или 25\% от себестоимости. Я не помню порядок цифр, но точно были ограничения.

\iusr{Oksana Dekhtiar}
\textbf{Irina Fon} и повторяю, выход блюда не указан. А стандартная порция 130-150г.

\iusr{Irina Fon}
\textbf{Oksana Dekhtiar} там написано 75год. Тогда это очень дешево  @igg{fbicon.grin} . У меня в школьной столовой уходил рубль. А тут осетрина за рубль. Можно жить @igg{fbicon.grin} 

\iusr{Oksana Dekhtiar}
\textbf{Irina Fon} Рестораны в СССР являлись государственными заведениями общественного питания. Аренду не платили. З/п работникам тоже из собственного кармана не платили... Можно было баловать публику!

\iusr{Oksana Dekhtiar}
\textbf{Irina Fon} Да может быть там на рубль той осетрины было граммов 70. А учитывая, что заливная... Думаю, 50 - максимум. А остальное - желатин))

\iusr{Irina Fon}
\textbf{Oksana Dekhtiar} да не сильно побалуешь. Два салата и рыбная и мясная нарезка. Сейчас так кормят в кафе. Для ресторана простенько. Нр, учитывая, чтр продукты в дифиците-неплохо.  @igg{fbicon.grin} 

\iusr{Iryna Osadcha}
\textbf{Irina Fon} Конечно копеек. А про сосиску - так они 2,20 руб. за кило. Ну и про какие года нужно уточнять.

\iusr{Irina Fon}
\textbf{Iryna Osadcha} да, поняла уже. Срсиска в 70-е стоила 50коп с хлебом и горчицей.  @igg{fbicon.grin} 
\end{itemize} % }

\end{itemize} % }

\iusr{Надежда Владимир Федько}

З дружиною ходили... 10 рублів на двох.

\iusr{Alla Zgurzhnitsky}
\textbf{Nadegda Volodymyr Fedko} скромненько поужинать, второе, 50 грамм, салатик. И потанцевать.

\iusr{Наташа Драпушко}

А была в Столичном много раз. Еще тогда, когда Любовь Успенская там пела и
объявляли Любомира. Оркестр был лучший. Игорь руководитель, фамилию не помню.


\iusr{Alla Zgurzhnitsky}
\textbf{Наташа Драпушко} если бы не уехала и не стала петь в ресторанах, она бы так и осталась ресторанной певицей. По сути она такая и есть.

\iusr{Zoya Pshenichny}
\textbf{Наташа Драпушко} И в \enquote{Дубках} она пела.

\iusr{Алина Цветкова}
В этом ресторане мы гуляли свадьбу моей сестрички ! Очень хороший ресторан был! @igg{fbicon.heart.suit}

\iusr{Людмила Серая}

Отмечала в \enquote{Столичном} 18-летие с друзьями. Это было в 1963 году. Мы были
студентами 1-го курса КПИ. Меню, конечно, не помню. Но было приятно, играл
оркестр. Как молоды мы были...

\iusr{Нонна Юшкова}
Кухня была супер

\iusr{Elena Veremenko}

В молодости была несколько раз в этом ресторане. А мне запомнилось кафе на
первом этаже где можно было недорого пообедать. А ещё выступление цыган которые
пели в стеклянной пристройке.

\iusr{Татьяна Домаскина}

У меня в ресторане Столичный была свадьба в 1986, а на первом этаже было
кафе-мороженое Снежинка по-моему называлось. Мы студентками туда бегали

\begin{itemize} % {
\iusr{Oksana Dekhtiar}
\textbf{Татьяна Домаскина} Крижинка

\iusr{Татьяна Домаскина}
\textbf{Oksana Dekhtiar} точно! Спасибо!
\end{itemize} % }

\iusr{Григорий Владимирович}

А мы в 70-х молодые ходили в \enquote{Петушок} по соточке с бутербродом или в
Курени.

\begin{itemize} % {
\iusr{Irina Fon}
\textbf{Григорий Владимирович} на месте Петушка теперь памятник Лобановскому @igg{fbicon.grin} 

\iusr{Юрий Надточий}
\textbf{Григорий Владимирович}, Да, Гриша, маршрут этот присутствовал частенько.
\end{itemize} % }

\iusr{Инна Далина}

Не было никаких конфет \enquote{Надежда}. На фоне тотального дефицита всего были
остродефицитные, но известные всем продукты - черная икра, палтус, Киевский
торт, конфеты Трюфель и Лещина (киевские). Из Москвы привозили только два
сладких дефицита - шоколад Балет и шоколад Бабаевский.

\begin{itemize} % {
\iusr{Zoya Pshenichny}
\textbf{Инна Далина} Вот и я не помню конфет \enquote{Надежда}.

\begin{itemize} % {
\iusr{Elena Tsirulnik}
\textbf{Zoya Pshenichny} Я тоже не помню...

\iusr{Zoya Pshenichny}
\textbf{Elena Tsirulnik} Не было таких, хорошо помню ассортимент ф-ки Карла Маркса.

\iusr{Надежда Владимир Федько}
\textbf{Zoya Pshenichny} Я таких цукерок не бачив! Якби такі були, то я б запам'ятав, оскільки моя дружина - Надія (Надежда).
\end{itemize} % }

\iusr{Elena Tsirulnik}
\textbf{Инна Далина} 

Не любила Бабевский. Самые вкусные конфеты и шоколад- киевские, и до сих пор покупаю только киевские.

\begin{itemize} % {
\iusr{Zoya Pshenichny}
\textbf{Elena Tsirulnik} \enquote{Светоч} тоже были вкусными.
А вообще самыми вкусными для меня были лимонные дольки, трехслойный мармелад, зефир, \enquote{Золотой ключик}

\iusr{Elena Tsirulnik}
\textbf{Zoya Pshenichny} Да, точно забыла что были сладости от Свiточ львовские, по-моему. Тоже были вкусные.
\end{itemize} % }

\iusr{Владимир Саркисян}
\textbf{Инна Далина} Зачем же так категорично? Такие конфеты были.

\ifcmt
  ig https://scontent-mxp1-1.xx.fbcdn.net/v/t39.30808-6/271852262_2106043052888004_3778145972416608223_n.jpg?_nc_cat=104&ccb=1-5&_nc_sid=dbeb18&_nc_ohc=X9bhEcZd3z8AX9BJRmy&_nc_ht=scontent-mxp1-1.xx&oh=00_AT9gDiWn3L9ROrsPlaA-KxO_uqRjl45KAbBxllerGKQGyQ&oe=61E51012
  @width 0.3
\fi

\begin{itemize} % {
\iusr{Zoya Pshenichny}
\textbf{Владимир Саркисян} 

В Киеве таких не было, или очень очень давно.

\iusr{Инна Далина}
\textbf{Vladimir Sarkisyan} 

у меня старшая сестра работала в типографии, где печатались этикетки для конфет
и алкоголя. Вот она приносила пачки бракованный этикеток мне и мы с друзьями
играли с ними. Вот она рассказывала, что чем дороже и дефицитнее конфеты,
шоколад и алкоголь, тем более сложная печать этикеток и оберток,что над
дизайном работают лучшие художники и контролируют процесс печати, потому, что
там и позолота и тиснение и самая лучшая бумага. Возможно конфеты Надежда и
существовали, но никто их не помнит ни в качестве любимых, ни в качестве
дефицитных. И обёртка явно дешёвая и не дизайнерская. Поэтому зачем писать
очередную ложь о том чего не было.

\iusr{Євгенія Словачевська}
\textbf{Владимир Саркисян} Названия не помню, но описанный вкус - да! Очень ярко вспомнился
\end{itemize} % }

\iusr{Лариса Бригинец}
\textbf{Инна Далина} Был ещё шоколад Вдохновение, состоял из нескольких продолговатых брусочков

\begin{itemize} % {
\iusr{Zoya Pshenichny}
\textbf{Лариса Бригинец} Синяя упаковка, а на ней балерина.

\iusr{Инна Далина}
\textbf{Zoya Pshenichny} точно! Вдохновение, а я назвала эти шоколадки Балетом, потому, что балерина была на обертке
\end{itemize} % }

\iusr{Ирина Шевцова}
\textbf{Инна Далина} и шоколад Вдохновение!

\iusr{Ирина Власенко}
\textbf{Инна Далина}
И карамель привозили
И конфеты из халвы

\end{itemize} % }

\iusr{Al Pol}

Ресторан СТОЛИЧНЫЙ только один из многих .... В Киеве прошла моя студенческая
юность ... Рестораны и кафе - часть этой жизни .... В семидесятые годы
отрывался в ресторанах МЛЫН, СТОЛИЧНЫЙ, Пектораль, Дубки на Нивках,
Театральный, Жокей (на ипподроме), ВiТРЯК, ЧЕРВОНА РУТА, шашлычная КОЛЫБА на
Гидропарке, ДИНАМО на одноименном стадионе.

В ресторане Динамо был отличный оркестр. Помню YESTERDAY в их исполнении ...
Отличный вокал, шикарный английский.....

Как молоды мы были ....

\begin{itemize} % {
\iusr{Алла Сахарова}
\textbf{Al Pol} А ПРАГА на Выставке?

\iusr{Al Pol}
\textbf{Алла Сахарова} Помню этот ресторан ... К сожалению позажигать в нем не удалось.
\end{itemize} % }

\iusr{Jul Solo}
Сыр Голландский дорогой

\iusr{Нина Харченко}
\textbf{Jul Solo} ,3.60 кг )

\iusr{Лариса Квитко Лыскина}
Свадьбу мою в 76 гуляли
Затем там же защиту диплома отмечали.

\iusr{Людмила Билык}

Мы отмечали выпускной в 1976... Это был мой первый поход в ресторан... Не помню
меню, но меня очень впечатлила котлета \enquote{по-киевски}...  @igg{fbicon.grin} ... Потом, конечно,
были и \enquote{Динамо}, и \enquote{Червона Рута}, и \enquote{Прага}, и \enquote{Дубки} и ещё... Эх
молодость!!!

\iusr{Людмила Гаценко}

27 квітня того самого 1986-го відзначала день народження. Ще не знали, що з
нами трапилося. Деякі гості не прийшли на день народження, бо вже їх вночі
відправили на ЧАЕС.

\iusr{Nina NinaNina}
Фантик конфеты \enquote{Надежда}

\ifcmt
  ig https://scontent-mxp1-1.xx.fbcdn.net/v/t39.30808-6/271767863_6724226657652666_3873424130635359527_n.jpg?_nc_cat=101&ccb=1-5&_nc_sid=dbeb18&_nc_ohc=sDkHgh0QYe0AX8QKpse&_nc_ht=scontent-mxp1-1.xx&oh=00_AT-Hkcx_WWcUr5xdKC-maas0rU06Jwj0XGDLciKmmFTN-g&oe=61E4CE62
  @width 0.2
\fi

\iusr{Anatoly Petrikovsky}
Хочу машину времени-в 1975 г поужинать.

\iusr{Klara Mezhebovsky}
Какая прелесть

\iusr{Elena Tsirulnik}
По воскресеньям после прогулки заходили туда обедать с родителями.

\iusr{Arcadia Olimi}

Я еще застала ресторан Столичный))), за 25 р. можно было прекрасно поужинать с
вином, кухня прекрасная, вкусно!!!

Справа от ресторана к майдану примыкала шикарная кулинария  @igg{fbicon.cherries} 

Я специально спускалась 16 трол. заходила, обеспечивала семье ужин малой
кровью, в смысле стояния у плиты)). А какие булочки с корицей  @igg{fbicon.face.savoring.food}{repeat=3} там были.

Со всеми этими сокровищами тащилась в Академ. городок.

Оно того стоило  @igg{fbicon.heart.beating}{repeat=3} 


\end{itemize} % }
