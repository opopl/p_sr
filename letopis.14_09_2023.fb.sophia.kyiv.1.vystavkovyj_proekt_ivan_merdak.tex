%%beginhead 
 
%%file 14_09_2023.fb.sophia.kyiv.1.vystavkovyj_proekt_ivan_merdak
%%parent 14_09_2023
 
%%url https://www.facebook.com/SophiaKyiv/posts/pfbid0JqYqBRFcNrWD52aVmKDNC2mm2X4sknkMghh6VRLfnhwQhm9SDxobxXPmTzYwh7kDl
 
%%author_id sophia.kyiv
%%date 14_09_2023
 
%%tags 
%%title Виставковий проєкт - роботи Івана Мердака
 
%%endhead 

\subsection{Виставковий проєкт - роботи Івана Мердака}
\label{sec:14_09_2023.fb.sophia.kyiv.1.vystavkovyj_proekt_ivan_merdak}

\Purl{https://www.facebook.com/SophiaKyiv/posts/pfbid0JqYqBRFcNrWD52aVmKDNC2mm2X4sknkMghh6VRLfnhwQhm9SDxobxXPmTzYwh7kDl}
\ifcmt
 author_begin
   author_id sophia.kyiv
 author_end
\fi

Сьогодні, 14 вересня, стартував новий виставковий проєкт Національних
заповідників \enquote{Софія Київська} і \enquote{Замки Тернопілля}, присвячений 90-ій річниці з
дня народження заслуженого майстра народної творчості України Івана Мердака.
(1933-2007р.р).

У залах \enquote{Хлібні} представлено роботи автора, виконані із природного матеріалу –
дерева. Майстер перевтілив природні форми дерева у величні постаті наших
древніх предків та поетичні образи із українського фольклору. 

В урочистому відкритті виставки взяли участь представники  Лемківщини - ректор
Київської академії мистецтв, народна артистка Ольга Бенч, учений секретар НЗ
\enquote{Замки Тернопілля} Надія Макарчук, директор Музейного комплексу \enquote{Лемківське
село} Віра Дудар, голова Київського товариства \enquote{Лемківщина} Михайло Мацієвський
та інші. 

Провела захід генеральний директор Національного заповідника \enquote{Софія Київська}
Неля Куковальська, яка отримала від лемківців у подарунок статуетку - виконану
в техніці різьблення фігуру козака.

Приємним доповненням до експозиції став художній виступ студенток Теребовлянського фахового коледжу.
