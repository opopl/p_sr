% vim: keymap=russian-jcukenwin
%%beginhead 
 
%%file 27_12_2020.news.ru.vesti.zenin_sergei.1.ogon_blokbaster
%%parent 27_12_2020
 
%%url https://www.vesti.ru/television/article/2504354
 
%%author 
%%author_id zenin_sergei
%%author_url 
 
%%tags film,ogon,russia
%%title Блокбастер "Огонь": так работу пожарных еще не показывал никто
 
%%endhead 
 
\subsection{Блокбастер \enquote{Огонь}: так работу пожарных еще не показывал никто}
\label{sec:27_12_2020.news.ru.vesti.zenin_sergei.1.ogon_blokbaster}
\Purl{https://www.vesti.ru/television/article/2504354}
\ifcmt
  author_begin
   author_id zenin_sergei
  author_end
\fi

\ifcmt
pic https://cdn-st1.rtr-vesti.ru/vh/pictures/xw/308/714/2.jpg
caption В прокат вышел новый блокбастер "Огонь", 27.12.2020
\fi

\begin{leftbar}
  \begingroup
    \em\Large\bfseries\color{blue}
В прокат вышел новый блокбастер "Огонь".\Furl{https://smotrim.ru/article/2499820} О трудных буднях пожарных и
спасателей. Невероятно зрелищная российская кинолента.
  \endgroup
\end{leftbar}

Масштабные лесные пожары в России – каждый год. И ежегодно сгорают до 8
миллионов гектар леса. Это чуть больше, чем территория современной Чехии. При
таких темпах лет через 20 у нас совсем не останется лесов. Поэтому идет
настоящая война со стихией.

"Огонь" – это классический фильм-катастрофа. О простых людях в экстремально
непростой ситуации и об их защитниках.

"Мне кажется, там не только у моего персонажа, там у всех персонажей огонь
внутри. То, как люди гасили в себе или поддерживают, или бьются с ним. Огонь, с
ним не договоришься. Самая главная вещь. С водой, с огнем – с ними не
договоришься", – говорит народный артист России Константин Хабенский.

Есть все: смелость, страх, улыбки, слезы, ненависть, любовь. Но категория –
"6+". Поэтому вся страсть – в одном лишь поцелуе. Когда он не попрощался перед
вылетом, и тогда она встала на взлетной полосе, чтобы остановить самолет.

"Такая махина к тебе приближается, ты знаешь что все это отрепетировано, что
там за рулем профессиональные люди, которые умеют вовремя остановиться, нажать
на тормоз, но, конечно, чисто психологически это тяжело, причем не просто идти,
а бежать на него", – вспоминает актриса Стася Милославская.

Они каждый раз прощаются так, как будто в последний. Но то, что показано в
фильме, рассказано настоящими пожарными, бывает и в жизни, когда огонь повсюду.

"Нам сделали огромный павильон, выстроенный из настоящего леса, то есть клочок
земли, поставили на какие-то такие платформы огромный кусок леса и нажимали
кнопку. Он горел, когда нужно, чтобы он переставал гореть, нажимали кнопку – он
переставал гореть. Я такого никогда не видел. Это какое-то чудо света было в
кино. Это правда, это нереально! Все полыхало, горело", – говорит актер Иван
Янковский.

Так тренируются, чтобы потом уже не было места для страха, чтобы на автопилоте,
потому что всегда время идет на минуты и даже на секунды. Говорят, что страшно
только первый раз. И именной в этот, в первый раз, ты понимаешь, твоя это
профессия или нет. И либо уходишь на гражданку, либо продолжаешь служить.

Вот о них это кино, о тех, кто в мирное время готов рисковать своей жизнью, для
которых этот риск – просто работа.

"Это не госзаказ, и мы просто делали про это кино, а оно оказалось к 30-летию
МЧС. Так получилось. Мы об этом не думали. Но я рад, если личному составу и
руководству, которое нам помогало, это как-то приглянется, тогда наша миссия
выполнена", – отметил Леонид Верещагин, гендиректор кинокомпании "Студия
ТриТЭ".

Так работу этих людей еще не показывал никто. И так не сопереживали. Когда
крошечный вертолет, у которого есть имя ("Лариса Ивановна"), в самом жерле
пожара пытается оторваться от земли, весь зрительный зал непроизвольно ему
помогает. Хоть на руках, но вынести вертушку с детьми из этого ада. Снято
честно. Веришь.

"Мне кажется, что это очень важная картина, потому что она рассказывает о тех,
кто спасает других, о величайшем подвиге людей, которые идут в огонь ради
спасения жизни других, – подчеркнул Никита Михалков, народный артист России,
продюсер фильма "Огонь". – Люди уже смотрели этот фильм. И я знаю реакцию: люди
смотрят, не могут оторваться, люди плачут, переживают. Мне кажется, что это то
кино, которое нам сегодня очень нужно. Надеюсь, вы мне поверите и пойдет в
кинотеатры".

Фильм принимали стоя. Глаза на мокром месте не только у женщин, но и у бывавших
в похожих передрягах, мужчин в форме. "Работа реально – огонь. Передали
доблесть, силу, мужество пожарных. Переполняют эмоции. Каждый день парни
рискуют", – говорит один из зрителей. "Когда закончился фильм, первая мысль,
которая была: я хочу замуж за спасателя! Такого парня каждой бы пожелать", –
признается одна из зрительниц.

Премьерный показ в Таганроге. Для сотрудников МЧС и авиационного завода имени
Бериева, где выпускается пожарный Бе-200 и где частично проходили съемки фильма
"Огонь". И те, и другие играли в эпизодах.

Зритель на фильм идет. И если бы не ограничения, введенные сейчас в
кинотеатрах, сборы побили бы все рекорды. Ведь, как известно, бесконечно долго
можно смотреть на три вещи: как трудятся люди, как течет вода и как горит
огонь. И этом смысле, единственное, что объединяет все три, – это работа
пожарных.
