% vim: keymap=russian-jcukenwin
%%beginhead 
 
%%file 28_01_2022.fb.fb_group.story_kiev_ua.1.kiev_visim_sekretiv.5.muzej_kirilovska_cerkva
%%parent 28_01_2022.fb.fb_group.story_kiev_ua.1.kiev_visim_sekretiv
 
%%url 
 
%%author_id 
%%date 
 
%%tags 
%%title 
 
%%endhead 
\subsubsection{5. МУЗЕЙ КИРИЛІВСЬКА ЦЕРКВА}

Це не просто оповита легендами місцина, де билинні українські герої перемагали
зміїв гориничів, а одне із найсвятіших місць Києва.

Саме тут на Дорогожичах (вул.Теліги, 12) розміщено київський храм Святого
Кирила XII ст.  Один із найважливіших об’єктів часів Руси та і загалом нашої
історії (бо таких збереглося - на пальцях рук перерахувати).

Стіни його увібрали своїм корінням і зберігають пам’ять про те, що відбувалося
тут протягом останніх майже тисячу років. Намолений сакрум зберігає накопичену
віками інформацію, треба тільки вміти прочитати її – через архітектуру,
унікальний живопис, численні, ще не повністю розшифровані написи графіті, через
ще не досліджені потаємні підземні ходи та поховання, частина яких з'явилась
задовго до, століттями формували Кирилівський цвинтар, а відтак і історію
функціонування Києво-Кирилівської Свято-Троїцької обителі.

Про історію монастиря і церкви я писав багато, як і про важкі сторінки святині
і повзучу окупацію.

Про головних героїв завдяки яким храм зберігся до наших днів (Острозьких, Савву
Туптало, Мелетія Дзика, Івану Мазепу ін.), а також і про те, як мені вдалося
допомогти святині і відновити в ній пам’ять про видатні родини Тупталів і
Дзиків.

Церква вражає історією часів Руси, до якої можна буквально доторкнутись,
княжими похованнями, а також і неймовірною палітрою витончених художніх
стінописів і робіт, від XII ст, через плеяду видатних художників XVII-XIX ст.,
і аж до сьогодні.

Прямо зараз у стінах музею відкрилась виставки «ЗІРКА, ЩО ВІД КИЄВА ВОЗСІЯЛА»,
яка стала одним із етапів «Меценатського проекту відновлення реліквій
Кирилівської церкви», та присвячена родині Тупталів. Тут можна бачити не лише
портрети колишнього ігумена, просвітника і інтелектуала Димитрія Ростовського
(Туптала), а і його батька - Савви Туптала. Який мені вдалося відновити разом з
художником О. Ольховим і І. Марголіною. Скоро ще з’являться барельєфи на фасаді
церкви, які будуть уславлювати козацькі родини Тупталів і Дзиків. Цю роботу
також вже профінансували і чекаємо результатів.

Прошу всіх сходити в Кирилівську церкву-музей, подивитись виставку (діятиме до
травня), оглянути унікальні стінописи храму, рівних яким мало в Європі, а тоді
написати мені відгук. Це буде дуже приємно для мене, а для вас пізнавально,
цікаво і корисно.
