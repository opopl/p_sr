% vim: keymap=russian-jcukenwin
%%beginhead 
 
%%file 29_10_2020.fb.denis_zharkih.1.nazism_fascism
%%parent 29_10_2020
%%url https://www.facebook.com/permalink.php?story_fbid=2840697346143664&id=100006102787780
%%author 
%%tags 
%%title 
 
%%endhead 

\subsection{Фашизм и нацизм}
\label{sec:29_10_2020.fb.denis_zharkih.1.nazism_fascism}

\Purl{https://www.facebook.com/permalink.php?story_fbid=2840697346143664&id=100006102787780}

Обожаю "интеллектуалов", которые постоянно пускаются в рассуждения, что фашизм
и нацизм совершенно разные вещи, и путают их только некультурные люди. Я так
понимаю, что эти "культурные люди" фашистам служили бы, а нацистам - ни-ни (или
наоборот). 

Да, советская пропаганда разницы не видела, и советский солдат (страшно
несведующий в этом смысле) гнал эту нечисть до Берлина, особо не спрашивая
тонкости политической ориентации: Погань, оно и есть погань. 

Разве сложно понять, что все это обеление фашизма, нацизма и коллаборационизма
нужно только для того, чтобы сломать волю народа к сопротивлению? Нас грабят,
заставляют умирать (различными способами) за чужие интересы, унижают, лишаю
прав, а сопротивлятся мы не должны. Мы же интеллигентные люди, а не совки
какие-то, которые могли и по зубам дать, и сильно. 

Ведь как некультурно! По зубам нам давать могут только патриоты, а мы ни-ни.
Утерся и шасть под лавку. И там читать труды Гитлера и Муссолини и находить
разницу. И на выборы идти голосовать либо за нового Гитлера, либо за Муссолини,
поскольку у нас демократия. 

И "умным людям" европейской направленности давно пора понять, что День Победы
праздник стариков, а не воспоминание, как народ поднялся и навалял
завоевателям. Не так ведь, поднимались, не так сражались, не так умирали... Мы
же знаем, мы умные. 

И главное - не называйте героя героем, а грабителя грабителем. Сегодня
грабитель это благодетель. И не надо праздника чествования героев, нет у
нас героев, кончились... 

Не кончились. Пока находится мужество идти против такого течения есть у
нас еще герои, а подлецы были всегда, без них героев не было бы!
