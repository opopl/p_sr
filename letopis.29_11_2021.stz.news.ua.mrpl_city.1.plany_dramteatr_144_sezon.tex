% vim: keymap=russian-jcukenwin
%%beginhead 
 
%%file 29_11_2021.stz.news.ua.mrpl_city.1.plany_dramteatr_144_sezon
%%parent 29_11_2021
 
%%url https://mrpl.city/blogs/view/pro-plani-mariupolskogo-dramatichnogo-teatru-na-144-ij-teatralnij-sezon
 
%%author_id demidko_olga.mariupol,news.ua.mrpl_city
%%date 
 
%%tags 
%%title Про плани Маріупольського драматичного театру на 144-ий театральний сезон
 
%%endhead 
 
\subsection{Про плани Маріупольського драматичного театру на 144-ий театральний сезон}
\label{sec:29_11_2021.stz.news.ua.mrpl_city.1.plany_dramteatr_144_sezon}
 
\Purl{https://mrpl.city/blogs/view/pro-plani-mariupolskogo-dramatichnogo-teatru-na-144-ij-teatralnij-sezon}
\ifcmt
 author_begin
   author_id demidko_olga.mariupol,news.ua.mrpl_city
 author_end
\fi

11 листопада відбувся конкурс на посаду директора-художнього керівника
Донецького академічного обласного драматичного театру (м. Маріуполь). Я
вирішила розпитати переможця конкурсу \emph{\textbf{Володимира Володимировича Кожевнікова}} про
плани театру на новий 144-ий сезон.

Директор-художній керівник театру одразу підкреслив, що він сподівається цього
сезону розширити гастрольну та фестивальну діяльність колективу. На його думку,
було б непогано, якщо б одна вистава була показана в театрі і того ж дня
колектив представляв ще один спектакль в іншому місті Донецької області. На це
у театру є і творчі, і технічні можливості. До речі, наразі відбувається
процедура закупки нового звукового обладнання. Керівництво сподівається, що
встигне його отримати до Нового року.

\ii{29_11_2021.stz.news.ua.mrpl_city.1.plany_dramteatr_144_sezon.pic.1}

Незважаючи на пандемію та карантин за попередній театральний сезон в театрі
встигли відремонтувати дах та сцену, зробили нову пожежну систему, систему
пожежогасіння та систему сповіщення. Цього театрального сезону ремонтні роботи
будуть продовжені. В планах ремонт фасаду. Зараз продовжується
науково-дослідницька робота щодо можливості ремонтних робіт, адже будівля
театру є пам'яткою архітектури місцевого значення і необхідно зробити все
можливе для її збереження. Водночас у директора-художнього керівника є плани
щодо розширення жанрового спрямування театру. Володимир Кожевніков наголосив,
що 
\begin{quote}
\em\enquote{дуже великим попитом користуються у глядача саме вистави, де багато
музичних номерів, танців, де є можливість відпочити і душею, і вухами}.
\end{quote}
Тому, ймовірно, згодом театр перетвориться із драматичного в му\hyp{}зично-драматичний.

У новому 144-ому театральному сезоні також планується відкриття нових сценічних
просторів, адже наразі доводиться працювати в умовах пандемії, тому для
збереження здоров'я глядачів малу сцену майже не використовують. Зокрема, вже у
2022 році планується відкрити сцену у фойє, де відбуватиметься спілкування з
глядачем. Будуть стояти столики з кавою, чаєм, печивом. На такій сцені
показуватимуть імпровізаційні вистави, які дозволять залучати глядачі до
дійства. Формат цієї сцени буде постійно змінюватися: від музичних та
драматичних вистав до тематичних вечорів. Ще є ідея відкрити сценічний
майданчик під сценою. В театрі наразі працює сцена на сцені, а це буде сцена
під колом. Невелика кількість глядачів зможе потрапити на таку сцену (десь до
20 осіб). Проте атмосфера цього сценічного простору підійде під детективні
вистави. І третій сценічний майданчик – це третій поверх театру, так звана
сцена під дахом. За словами директора-художнього керівника, там вже сьогодні
можна працювати, брати вистави, які йдуть у форматі малої сцени. Ці вистави
будуть швидко трансформуватися і їх можна буде побачити не тільки в приміщенні,
але й на вулиці.

\ii{29_11_2021.stz.news.ua.mrpl_city.1.plany_dramteatr_144_sezon.pic.2}

Враховуючи те, що зараз саме в період пандемії другий рік поспіль театр не має
тих прибутків, які отримував до цього, керівництво сподівається, що глядач не
залишить театр і ходитиме на вистави частіше.

До нового року театр готує виставу \emph{\enquote{Кіт у чоботях}} українською мовою. На день
святого Миколая покажуть спектакль \emph{\enquote{Іде Святий Миколай}}. Також у новому сезоні,
якщо не завадить пандемія, прем'єрними стануть такі унікальні спектаклі, як
\emph{\enquote{Ромео і Джульєтта}} і \emph{\enquote{Мина Мазайло}} та мюзикли \emph{\enquote{Весілля в Малинівці}} і \emph{\enquote{Моя
чарівна леді}}. На всі вистави пускатимуть тільки глядачів, які мають зелені
COVID-сертифікати. Потрапити до театру можна як з паперовим сертифікатом, так і
з електронним у \enquote{Дії}.
