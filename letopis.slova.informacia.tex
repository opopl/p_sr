% vim: keymap=russian-jcukenwin
%%beginhead 
 
%%file slova.informacia
%%parent slova
 
%%url 
 
%%author 
%%author_id 
%%author_url 
 
%%tags 
%%title 
 
%%endhead 
\chapter{Информация}

%%%cit
%%%cit_head
%%%cit_pic
%%%cit_text
А  шефиня Центра \emph{информационной} безопасности (ЦИБ) Людмила Цибульская
сообщила, что будет заниматься тремя направлениями. А именно: стратегическими
коммуникациями, противодействием \emph{дезинформации} путем постоянного
\emph{информирования} и адвокацией (чем-чем? – Г.Б.) усилий государства и
общественных организаций на международном уровне. И поведала, что орган успел
запустить несколько «важных проектов».  Интересно следующее. «ЦПД может
получать в установленном порядке от органов исполнительной власти,
правоохранительных и разведывательных органов, прокуратуры, органов местного
самоуправления, предприятий и организаций, независимо от форм собственности и
подчинения, статистические данные, \emph{информацию}, справочные материалы и
т.п.» То бишь любая \emph{информация}, представляющая интерес для СНБО и его
\emph{«дезинформационного»} отростка, будет использоваться, даже несмотря на
то, что персональные данные граждан Украины вроде бы защищены законодательством
%%%cit_comment
%%%cit_title
\citTitle{Как борьба с «дезой» на Украине множит сущности}, Георгий Бородин, odnarodyna.org, 07.01.2021
%%%endcit


