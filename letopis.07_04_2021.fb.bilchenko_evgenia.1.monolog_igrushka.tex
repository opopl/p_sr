% vim: keymap=russian-jcukenwin
%%beginhead 
 
%%file 07_04_2021.fb.bilchenko_evgenia.1.monolog_igrushka
%%parent 07_04_2021
 
%%url https://www.facebook.com/yevzhik/posts/3785660131469069
 
%%author 
%%author_id 
%%author_url 
 
%%tags 
%%title 
 
%%endhead 

\subsection{БЖ. Монолог игрушки}
\label{sec:07_04_2021.fb.bilchenko_evgenia.1.monolog_igrushka}
\Purl{https://www.facebook.com/yevzhik/posts/3785660131469069}


\ifcmt
  pic https://scontent-mad1-1.xx.fbcdn.net/v/t1.6435-9/170077230_3785660004802415_8347635053857895220_n.jpg?_nc_cat=107&ccb=1-3&_nc_sid=8bfeb9&_nc_ohc=rLYXB5me0VcAX-5Z-ZH&_nc_ht=scontent-mad1-1.xx&oh=a548dc9acf6af6ab3f4b5bfce176631c&oe=6093B55F
\fi


У меня не вышло стать терминатором.
Проливать свою кровь гранатовую.
Восстанавливаться из шариков жидкой ртути.
Говорить коротко и по сути.
У меня не вышло стать резидентом.
Запоминать пароли, скользить одетым
В серые плащ и шляпу по всяким Прагам,
Выискивать явочные кофейни, глотать бумагу.
У меня не вышло уйти в повстанцы.
В тИре я могу выиграть только зайца.
Я попадаю в тройку из десяти.
Я не обучена толерантно, легко уйти.
У меня не вышло ходить, держась за перила.
Ночью я боюсь спать: пот пропитал перину
Родного больного. Нет никого роднее.
Лучше болезнь бы взяла меня: мне привычно с нею.
После бунтика на меня ни одна не купилась партия:
Все сказали, что я - гибрид юродивого и парии.
Это - не для Кремля, Брюсселя и Вашингтона.
Это - разве что для Шопена или Платона.
Все считают, что я - богата: насасываю, аскаю.
Я не крашусь уже два года, я на себе таскаю
Кульки с яйцами и батоном из магазинной ссылки.
После тарифов мне остаётся на салицилку.
Все считают, что хайп - мой хлеб, а идол - моя свобода.
Но больше всего я люблю закат на кожице небосвода:
Апельсиновый и красивый, то багровый, то сивый.
Они обижаются, но закат есть моя Россия.
Да, я люблю внимание: артисту нужны подмостки
Для выживания тела, для тренировки мозга.
Да, я люблю педагогику: учителю дети - счастье.
И люблю науку: крошиться ночью терминами на части.
Грех ли - любить знание, сцену, аудиторию?
Если меня готовили к такой вот смешной истории:
Преподавателя, доктора и писателя?
У меня не вышло стать Бодхисаттвой.
У меня не вышло даже стать дорогой женой.
Вылечить мужа, когда он совсем больной.
Заболеть, наконец, самой, ковидом и биполяркой,
Отключиться в коме: тащить я больше не в силах лямку.
Месяц назад мне давали убежище аж в Берлине.
Но там надо спать без мужа - ковидного, на перине.
А я не могу без мужа - это, наверно, дурка.
И ещё: я умираю без Петербурга.
Вы будете насмехаться, жалеть и ластиться.
На гидазепаме я до рассвета как аноним фрилансю
Чужие докторские: "Объект, предмет, новизна, гипотеза..."
Их авторам страшно стоять со мной рядом в зале, чтоб не испортиться.
Девочка Аня, спрашивая, искренно ждёт ответа:
"Что толку терпеть экзекуцию в этом треклятом гетто?"
Как объяснить ей, что я взорвала тысячи их Бастилий -
Тайно, и явно, и вперемежку, - лишь бы они простили.
Взорвала жизнь свою, как балда, собственно, ради них -
Не нарочно, не героично: просто так обернулся миг:
Тупо и ради себя частично, но за истины свет святой...
Не любите меня, - у меня не вышло стать для вас золотой.
8 апреля 2021 г.
