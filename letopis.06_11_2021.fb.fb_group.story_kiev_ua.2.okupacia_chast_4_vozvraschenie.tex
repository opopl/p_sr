% vim: keymap=russian-jcukenwin
%%beginhead 
 
%%file 06_11_2021.fb.fb_group.story_kiev_ua.2.okupacia_chast_4_vozvraschenie
%%parent 06_11_2021
 
%%url https://www.facebook.com/groups/story.kiev.ua/posts/1791667447696715
 
%%author_id fb_group.story_kiev_ua,bubnov_jurij
%%date 
 
%%tags 1943.kiev.osvobozhdenie,kiev,okkupacia,rkka,semja,sssr,vov,zhizn
%%title Моя семья. Жизнь и приключения во время немецкой оккупации. Часть 4, возвращение
 
%%endhead 
 
\subsection{Моя семья. Жизнь и приключения во время немецкой оккупации. Часть 4, возвращение}
\label{sec:06_11_2021.fb.fb_group.story_kiev_ua.2.okupacia_chast_4_vozvraschenie}
 
\Purl{https://www.facebook.com/groups/story.kiev.ua/posts/1791667447696715}
\ifcmt
 author_begin
   author_id fb_group.story_kiev_ua,bubnov_jurij
 author_end
\fi

Моя семья. Жизнь и приключения во время немецкой оккупации. 

(окончание)

Часть 4. Возвращение

Узнав от красноармейцев, что Киев освобожден, отец приказал нашей «команде»
немедленно собираться в дорогу и возвращаться домой. Нам предстояло идти в
сторону шоссе через большое поле, на котором ночью произошел бой.

\ii{06_11_2021.fb.fb_group.story_kiev_ua.2.okupacia_chast_4_vozvraschenie.pic.1}

И опять началось все сначала. В загруженную двуколку впрягся отец, Валя толкал
повозку сзади, мама со мною на руках, остальные взяли, кто что смог – кто узлы,
кто чемоданы. Несмотря на все пережитое, настроение было хорошее – во-первых,
шли домой, во-вторых, я перестал кашлять.

Мы двинулись к дороге через поле, которое казалось бесконечным. Шли мы очень
медленно, так как земля уже была замерзшая и повсюду были глубокие воронки от
снарядов, поэтому двуколка ехала по ней как по стиральной доске – подпрыгивала
и тряслась. Впереди, в метрах двухстах от нас, стоял подбитый танк и несколько
сгоревших машин, из которых валил черный дым. Подойдя поближе, мы увидели
страшную картину – повсюду лежали трупы немецких солдат. Лежали они в разных
позах, как их застала смерть, многие с открытыми глазами., почти у всех
вывернуты карманы и снята обувь, оружия не было. Ветер разносил по полю
какие-то бумажки. Мы старались не смотреть на убитых, обойти их, но они были
всюду. Больше всего поразил труп одного немца, к которому мы подошли совсем
близко. Был он без головы, в одной руке держал фотографию какой-тот женщины,
другой руки не было.

\ii{06_11_2021.fb.fb_group.story_kiev_ua.2.okupacia_chast_4_vozvraschenie.pic.2}

Пройдя еще немного вперед, увидели запряженную двумя лошадьми телегу и трех
солдат-красноармейцев, стоявших рядом с ней. У одного из них на рукаве была
повязка с красным крестом. На телеге лежало несколько тел, накрытых брезентом,
из-под которого были видны валенки. Оказалось, это похоронная команда, которая
ищет наших погибших солдат. Немного поговорили с ними. Они поинтересовались,
откуда мы и куда идем, угостили крестного махоркой.

Вскоре, где-то вдалеке, уже можно было видеть дорогу, по которой ехали колонны
автомашин. И тут случилась неприятность, которая должна была случиться – у
тележки отвалилось колесо. Доконало ее поле с замёрзшими комками земли и
воронки. Кое-как двуколку удалось починить, но когда до шоссе оставалось пройти
совсем немного - развалилась окончательно, и мы остались со своим скарбом без
транспорта. 

Мимо нас по дороге проезжали колонны автомашин с орудиями и
солдатами, но на наши просьбы ни одна из них не остановилась. Ближе к вечеру,
движение по дороге почти прекратилось – проезжали лишь одиночные машины. Мы
сели на обочине и стали думать, что делать дальше. Понятно, с некоторыми
вещами, как бы ни было их жаль, придется расстаться. Решили взять самое
необходимое и двигаться дальше. 

Шли уже более часа, когда позади услышали шум
приближающейся машины. Немного обогнав нас, остановилась грузовая крытая
машина. 

\ii{06_11_2021.fb.fb_group.story_kiev_ua.2.okupacia_chast_4_vozvraschenie.pic.3}

Из кабины вышел военный в форме, похожей на немецкую, и направился в
нашу сторону. Мы были уверены, что это опять немцы, но бежать было некуда -
кругом поле и дорога. Подойдя к нам, он на русском языке с небольшим акцентом,
сказал, чтобы мы их не боялись - они из Чехословацкой армии и воюют вместе с
Красной армией против немцев. (Примечание мое. 1-я чехословацкая бригада
впервые приняла участие в боях под Киевом 5 ноября 1943 года). 

Спросили куда мы
идем, и узнав, что направляемся в Киев, предложили нас подвезти. Радости нашей
не было предела. Более того, два солдата-чеха, которые ехали в кузове, помогли
погрузить в машину наш скарб. Маму со мной посадили в кабину, остальных в
кузов. Ехали, не замечая холода и того, что целый день ничего не ели, хотелось
плакать от счастья, что наши мытарства скоро закончатся.

В Киеве машина остановилась у завода «Большевик». Офицер сказал, что им нужно
повернуть направо, и дальше мы должны добираться сами. Они даже представить не
могли, как мы были им благодарны. Солдаты выгрузили наши вещи, дали две буханки
хлеба и три банки консервов, на которых были надписи на незнакомом нам языке.

\ii{06_11_2021.fb.fb_group.story_kiev_ua.2.okupacia_chast_4_vozvraschenie.pic.4}

Мы шли по Брест-Литовскому шоссе в сторону Галицкой площади, было уже совсем
темно и очень холодно. По обе стороны дороги горело несколько домов, и жар
исходил от них, как от печки. Хотелось бросить все, стоять и греться. Но нужно
было идти. Изредка попадались навстречу или обгоняли нас, такие же беженцы как
мы, но было их совсем немного. Несколько раз попадались наши солдаты, но они
проходили мимо, как бы не замечая нас.

Наконец, падая от усталости и мучаясь от холода и голода, добрались до нашего
двора – Дмитриевской, 16. Окна домов зияли темными отверстиями, под ногами
хрустели осколки разбитых стекол. Мы разговаривали негромко, но каждое слово
отдавалось сильным эхом. Кругом ни души. Стало страшно. И вдруг, кто-то увидел
на пятом этаже нашего дома в квартире №14 слабый огонек. Там жил наш сосед
сверху доктор Ребров.

Поднялись к себе на четвертый этаж, в 12-ю квартиру. Двери были открыты
настежь. Вещи оставили на кухне, а сами пошли наверх, где были радостно
встречены хозяином. Доктор Ребров поставил на примус огромный чайник, и мы до
утра сидели за столом, рассказывая друг другу, что пережили за эти последние
два месяца. Оказывается, он не покидал запретную зону, а прятался то у себя в
квартире, то в подвале дома, где и дождался прихода наших войск.

Утром, когда стало светло, зашли в свою квартиру. Хотя все оставалось на своих
местах - даже большой обеденный сервиз стоял в буфете, было видно, что в ней
побывали немцы – на столе стояли тарелки с остатками еды, пустые бутылки из-под
вина и стаканы, там же стоял небольшой кинопроектор и несколько металлических
круглых коробок с киноплёнкой. 

\ii{06_11_2021.fb.fb_group.story_kiev_ua.2.okupacia_chast_4_vozvraschenie.pic.5}

На полу были разбросаны немецкие газеты, бумаги,
обмундирование, а на диване лежал портфель, в котором оказались какие-то
документы, печать и несколько коробочек, в которых лежали награды – кресты,
связка писем и нож в чехле со свастикой.

Примечание мое. Как мы узнали позже, нож в чехле - это был немецкий офицерский
кортик. Валя умудрился его спрятать на верху печки в квартире, и он у нас
прослужил как кухонный нож до начала 60-х, правда уже без свастики. Что же
касается кинопроектора, который оказался известной немецкой фирмы "AGFA", он
оставался у нас до середины 50-х. Куда всё это исчезло потом – неизвестно. 

\ii{06_11_2021.fb.fb_group.story_kiev_ua.2.okupacia_chast_4_vozvraschenie.pic.6}

Отец приказал бабушкам собрать, как он выразился всю эту «гадость», имея ввиду
награды, завернул их в какую-то тряпку, спустился во двор и выбросил в
общественный туалет. Газеты, бумага и документы пригодились очень кстати - все
пошло на растопку в печку, вместе с досками, которые отец принес со двора.
Прибрали немного в квартире, одеялами занавесили разбитые окна и затопили
печку. Стало очень уютно и спокойно. На столе расставили тарелки, порезали
хлеб, открыли консервы и пригласили доктора Реброва.

За столом сидели не долго, нервное напряжение спало и всех свалил сон. Спали
как убитые, весь день и всю ночь. 

Утром мама спустилась по лестнице двумя этажами ниже посмотреть, не вернулась
ли соседка - Аня Жолковская с семьёй. Долго стучала в дверь, но никто не
открыл. Отец вышел во двор, надеясь увидеть своего товарища Севу Атамановского.
Жил он с женой Надей и двумя детьми – Ттамарой и Юрой здесь же, в
полуподвальном помещении флигеля. 

\ii{06_11_2021.fb.fb_group.story_kiev_ua.2.okupacia_chast_4_vozvraschenie.pic.7}

Дом был пуст. Отец уже возвращался домой, как вдруг увидел в тупике двора
брошенный немцами мотоцикл с коляской. Так как в баке не оказалось бензина
(видимо, поэтому немцы его и бросили), отец закатил его в подвал нашего дома.
Позже выяснилось, что это был американский мотоцикл «Indian», который прослужил
нашей семье до конца 50-х. Каким образом американский мотоцикл оказался в
оккупированном немцами Киеве – неизвестно. 

Постепенно наш и близлежащие дома начали заполняться возвращающимися беженцами.
Вскоре на Дмитриевскую, 18 вернулся в свою коморку часовщик-еврей, которого все
считали погибшим. Оказалось, его также приютила и спасла киевская семья.
Возвращению часовщика очень обрадовался мой крестный - во-первых, это был его
старый знакомый, во-вторых, его чудом сохранившиеся старинные карманные часы
«Longines» нуждались в починке.

\ii{06_11_2021.fb.fb_group.story_kiev_ua.2.okupacia_chast_4_vozvraschenie.pic.8}

Пожалуй, это и все, что я слышал от родных об их жизни в оккупации, за
исключением отрывочных, никак не связанных между собой эпизодов, которые до сих
пор хранятся в моей памяти.

Фотографии. 

1. 1943 год, ноябрь. Жители Киева возвращаются после его освобождения.
Наверное, примерно такой вид имела и наша семья в те же дни.  Фотография из
открытых источников. 

2. 1943 год, ноябрь. Бульвар Шевченко (видно ограду ботанического сада).
Беженцы возвращаются. Фотография из открытых источников. 

3. 1948 год. Набережная Днепра. На том самом мотоцикле \enquote{Indian}–
отец, мама, моя сестричка Наташа и я. 

4. 1948 год. Набережная Днепра. Отец, мама, Валя, сестричка Hаташа и я. 

5. 1949 год. Бабушка Настя, Валя и сестричка Наташа. 

6. 1955 г. Бабушка Маня 

7. 1949 год. Отец, мама, сестричка Наташа и я.

8. 1949 год. Валя и сестричка Наташа.

\ii{06_11_2021.fb.fb_group.story_kiev_ua.2.okupacia_chast_4_vozvraschenie.cmt}
