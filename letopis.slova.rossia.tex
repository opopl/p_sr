% vim: keymap=russian-jcukenwin
%%beginhead 
 
%%file slova.rossia
%%parent slova
 
%%url 
 
%%author 
%%author_id 
%%author_url 
 
%%tags 
%%title 
 
%%endhead 

Новости Крымнаша. Предатели в Крыму пришли к выводу, что \emph{россияне} — хуже
глистов, obozrevatel.com, 24.05.2021

Луганский академический театр кукол совместно с коллегами из \emph{РФ} и ДНР
провел онлайн-выставку рисунков юных зрителей, посвященную Дню славянской
письменности и культуры, lug-info.com, 24.05.2021

\enquote{Луганский академический театр кукол совместно с Брянским областным
театром кукол и Горловским городским театром кукол организовал онлайн-выставку
детских рисунков \enquote{Мы - славяне!}, посвященную Дню славянской
письменности и культуры, который ежегодно широко отмечается в \emph{России} и
других странах 24 мая}, - говорится в сообщении, lug-info.com, 24.05.2021.

Сейчас мы переживаем время не менее судьбоносное. И от того, вспомнит ли народ
свои изначальные культурные коды, зависит судьба наша в новом веке – судьба не
только \emph{России}, но и всей человеческой цивилизации, vz.ru, 24.05.2021

С этого времени \emph{Русь} ощущает себя уже не просто некоей «срединной
землей» между Востоком и Западом, Азией и Европой, но неким особым миром между
землей и небом, между началом и концом истории. И здесь, между небом и землей,
началом и концом истории, начинает она искать свои берега: «Открылась бездна,
звезд полна. Звездам числа нет, бездне – дна», vz.ru, 24.05.2021

Свободная и богатая за счет транзитной торговли, \emph{Русь} переживает бурный
рост, на пике которого и является «Слово о законе и благодати» (между 1037 и
1050 годами) митрополита Илариона – первое слово \emph{русского} самосознания,
сказанное первым этнически \emph{русским} митрополитом, vz.ru, 24.05.2021

Не просто Україна, а Україна-\emph{Русь}. Як Михайло Грушевський захищає
державні кордони?  radiosvoboda.org, 23.05.2021

А вот принципиального различия между укладами современной \emph{Россией} и
Украиной нет. Тут, как в Крыму или на Донбассе - были украинцы сверху, стали
\emph{русские}, а по сути, так примерно одни и те же люди. Народу, по большому
счету, начхать, сегодня \emph{русские} записываются в украинцы, завтра украинцы
запишутся в \emph{русские} - делов-то, \textbf{Дело Протасевича - еще один
повод раздуть Холодную войну}, Денис Жарких, strana.ua, 24.05.2021

Ця традиція іде ще від апостола Андрія – дві тисячі років тому він першим
провістив сакральну природу Києва. У ХІ столітті у знаменитій промові «Слово
про закон і благодать» перший \emph{Руський} митрополит Іларіон возвеличив Київ
як «Город Святий Всеславний», radiosvoboda.org, 09.10.2018

Добре, хоч \emph{Мордору} нічого не дали, але перед Литвою соромно! Так не
роблять! Лариса НиЦой, facebook, 23.05.2021

Я голосувала окрім деяких інших і за Литву, вони класні. А журі заангажовані,
це очевидно. І що бісить, Україна з року в рік продовжує давати бали
\emph{Русні}, комментарий (Ірина Чубарова), пост Ларисы НиЦой, facebook,
23.05.2021

Если \emph{Россия} не захочет быть мощной, имперской страной, она станет
Украиной. Само по себе майданное государство с \emph{российского} горизонта не
исчезнет, а предположение, будто украинская проблема рассосется сама по себе,
за семь лет доказало свою несостоятельность, \textbf{Надоела Украина? Ежедневное «иди и
смотри»}, Константин Кеворкян, 24.05.2021

Двери закрываются. Жителей \enquote{Л/ДНР} в \emph{России} ждёт ГУЛАГ 2.0,
obozrevatel.com, 25.05.2021

Помимо этого, как юрист, Лилия Корнилова раскрывает природу инструментария
\emph{российского} агрессора на международной арене и методы противостояния
этим провокациям, \url{www.obozrevatel.com/person/liliya-kornilova.htm}

«В \emph{России} отличное бездорожье» — как автобус не смог довезти детей до
школы, regnum.ru, 25.05.2021

Полузабытый \emph{русский} гений, которого в США считают своим — фото с
выставки, regnum.ru, 25.05.2021

\emph{Россия} лайкающая, Что на самом деле смотрят и ищут \emph{россияне} в
интернете, lenta.ru, 14.05.2021

Для поставок в Европу нефтегазового сырья, удобрений и металлов \emph{Россия}
использует не грузовые автомобили, а трубопроводы, железную дорогу и танкеры в
морских портах. \emph{Россия} в последние годы активно переводила свои грузы с
белорусской железной дороги на собственные морские порты на Балтике, vz.ru,
25.05.2021

Возмущение патриотов поддержал министр культуры и информационной политики
Александр Ткаченко, заявив, что в \emph{России} в очередной раз пытаются
присвоить себе национальное украинское блюдо, odnarodyna.org, 25.05.2021

У примітках до «Полтави» Пушкін зазначає: «Дорошенко, один з героїв давньої
Мало\emph{росії}, непримиренний ворог \emph{російського} панування»,
radiosvoboda.org, 25.05.2021

Не все, що звучить по-\emph{російськи} чи не все, що схоже на \emph{російську}
чи подібне до слова, яке існує в \emph{російській}, було сюди принесено з
\emph{російської} мови, pravda.com.ua, 25.05.2021

Якщо розбирати, яке слово де-факто прийшло з української, а яке запозичене,
вийдуть дуже прикольні речі. Вийде, що слово \enquote{щур} в українській мові скоріш за
все з \emph{російської}, а \emph{російське} слово \enquote{крыса} скоріш за все з української, просто
свого часу ми ними помінялися, pravda.com.ua, 25.05.2021

Ступка, вічна пам'ять, в кожній другій сцені, де він з'являється в ролі Тараса
Бульби, повторює про \emph{"русскую душу"}, \emph{"русскую землю"},
\emph{"русскую силу"}. Будь-яка компліментарність щодо України в цьому випадку
– це імперська, шовіністична компліментарність до меншого брата. Гоголь
спочатку був не про це, Бортко зробив його про це, і цей контекст змінює всі
повідомлення, pravda.com.ua, 25.05.2021

Сладкие парочки: \emph{Россия} и Украина, mikle1.livejournal.com, 21.11.2009

Даже я понял, что он лично виноват в том, что \emph{Россия} до сих пор не
развалилась на демократические Чечни и Запопинские республики Окрайны, Китайны
и прочая, mikle1.livejournal.com, 21.11.2009

Записки киевлянки: людей накручивают - «26 мая затмение, вот тогда
\emph{Россия} и нападет! - kp.ru, 15.04.2021

Истории Олеся Бузины: Почему рухнула Киевская \emph{Русь}, sedognya.ua, 14.06.2014

Я уже писал в одной из предыдущих статей, что название «Киевская \emph{Русь}»
придумал только в XIX веке московский историк Михаил Погодин. До него никто
даже не подозревал, что она «киевская». Современники называли эту страну просто
\emph{Русью} или \emph{Русской} землей, Олесь Бузина, segodnya.ua, 14.06.2014

Пельмени-мутанты, \emph{русский} гимн и \enquote{Кин-дза-дза} – англичанин
попал в \emph{Россию} и признался: \enquote{I was not gotov...}, tsargrad.tv,
12.04.2021

«Наглые \emph{русские}»: мир протестует против формы \emph{российских}
олимпийцев, woman.ru, 16.04.2021

«Наш слоган — \enquote{Окна в \emph{Россию} будущего}: интервью с автором \emph{русской}
кибердеревни, www.mirf.ru, 16.04.2021

Фигурное катание. \enquote{Невозможно победить \emph{Россию}. Их фигуристы на
запредельном уровне}, - Японцы высказались о КЧМ-2021, sport.ru, 16.04.2021

Первое упоминание названия \emph{Россия} - X век н.э. Константин
Багрянородный, zhenziyou.livejournal.com, 19.03.2016

Как Киевскую \emph{Русь} превратили в Украину, а потом в Анти\emph{Россию},
kp.ru, 17.06.2019

\enquote{Одержимость вот такая русофобская и навязчивая идея обвинить \emph{Россию} во всем и
вся. Наверное, уже скоро дойдет до того, что \emph{Россия} будет обвиняться просто в
факте своего существования}, - заявил Песков, strana.ua, 25.05.2021

«Даже обидно. На Украине из всех утюгов воюют с \emph{Россией}, а \emph{РФ} на нее плевать», - пишет Иосиф Эглис,
\textbf{Реакцию россиян на человека с символикой Украины сняли на видео}, riafan.ru, 13.05.2021

«А теперь попробуйте прогуляться по Львову с \emph{российской} символикой», - предлагает Татьяна Сергиенко,
\textbf{Реакцию россиян на человека с символикой Украины сняли на видео}, riafan.ru, 13.05.2021

Єдине «тішить» у цій ситуації, що й українці дали «достойну відповідь»
слов'янам. Українські глядачі лише проголосували за виступ від однієї
слов'янської країни – \emph{Росії}. Хоча \emph{Росію} (що показово!) представляла співачка,
яка не є слов'янкою. Але ж українці без «братньої \emph{Росії}» жити не можуть
– хоча й воюють із нею. Тому й дали співачці Маніжі від \emph{Росії} 4 бали.
Добре, що не 12, - \textbf{Про слов'янську солідарність?}, Петро Кралюк, day.kiev.ua,
24.05.2021

Империя воскресла: почему \emph{Россия} вычеркивает диссидентов из своей
истории, glavred.info, 24.05.2021

Ці «сайти-помийки» теж часто перетворюються на рупори \emph{російської}
пропаганди. Чи то піарячи того-таки Пальчевського з його \emph{проросійськими}
тезами, чи то поширюючи фейк Медведчука про американські біолабораторії, чи
цитуючи псевдоекспертів про фашизм в Україні, чи просто напряму даючи слово
Пєскову, \textbf{Сотні тисяч. Яка аудиторія (про)російських медіа в Україні},
texty.org.ua, 12.05.2021

Ще частіше незаконно використовують \emph{російську} мову замість української
посадовці обласного рівня, \textbf{Місцеві ради, телебачення, спорт та інше.
Хто порушує закон про мову і що робити, щоб він виконувався}, texty.org.ua,
20.05.2021

Партизанська війна? Як Зеленський готує ЗСУ до можливого масштабного наступу
\emph{Росії}, radiosvoboda.org, 26.05.2021

Може, якби наші 210 тисяч вояків були озброєні такою технікою, як Армія оборони
Ізраїлю, то, може, і \emph{Росія} тремтіла б перед нами, розумієте?» – зазначає
Жданов, radiosvoboda.org, 26.05.2021

Вице-президент Ассоциации туроператоров \emph{России} (АТОР) Дмитрий Горин
рассказал, в какие пляжные страны из тех, с которыми \emph{Россия} возобновила
авиасообщение с 25 мая, можно будет лететь отдыхать. Об этом сообщает «Москва
24», lenta.ru, 26.05.2021

Накануне создания \emph{Русского} царства в Северо-Восточной \emph{Руси} было
три великих княжества: Московское, Тверское и Рязанское, \textbf{Как немецкий
принц стал черниговским князем и русским святым}, Подумалось мне часом,
zen.yandex.ru, 24.03.2021

Великое княжество Рязанское много веков было пограничным краем, встречавшим
всех завоевателей. К югу от рязанских земель начиналось Дикое Поле, откуда
испокон веков на \emph{Русь} набегали кочевники, \textbf{Как немецкий принц
стал черниговским князем и русским святым}, Подумалось мне часом,
zen.yandex.ru, 24.03.2021

\emph{Россия} большая, сильная, она с нефтью и газом, современным вооружением,
лучшими в мире вакцинами и много чем ещё. Много кому от нас что-то надо, все
едут, просят, договариваются, Мак Сим, zen.yandex.ru, 16.05.2021

У середу, 26 травня, активісти «КримSOS» провели акцію «Ходіння по колу» біля
посольства \emph{Росії} в Києві. Це стало 58-м заходом на підтримку жертв
насильницьких зникнень в окупованому \emph{Росією} українському Криму та їхніх рідних,
\textbf{«Ходіння по колу»: посольство Росії пікетували на підтримку жертв насильницьких зникнень у Криму – фоторепортаж},
radiosvoboda.org, 26.05.2021

У 1905 році Рада міста Львова вирішила урочисто відзначити 250-річчя оборони
Львова під час облоги міста військами Богдана Хмельницького. Офіційно це
звучало як \enquote{порятунок міста від козаків та \emph{росіян}}, \textbf{Дві
історії, які нічого не вчать}, zaxid.net, 25.05.2021

Українець Феофан Прокопович назавжди змінив обличчя \emph{Росії}, перетворивши
її на імперію. Навіть відновлення патріаршества, революція, епоха радянського
атеїзму не змогли вбити його напрацювання, \textbf{Прокопович та інші українці,
що будували Росію}, Володимир Володько, zrada.org, 04.03.2011

Но остается открытым вопрос: почему страна, которая \enquote{не \emph{Россия}},
так сильно озабочена в основном \emph{российской} или, на худой конец,
бела\emph{русской} повесткой?
\textbf{Протасевич нам важнее, чем состояние собственной экономики}, Владислав Михеев, 
strana.ua, 27.05.2021

\textbf{Протасевич нам важнее, чем состояние собственной экономики}, Владислав Михеев, 
strana.ua, 27.05.2021

Массовые расстрелы в школах и других общественных местах — пугающая проблема
современного общества, с которой в последние годы столкнулась и \emph{Россия},
lenta.ru, 27.05.2021

В то же время, Пивоваров сообщил, что закрытие организации вовсе не означает,
что ее члены опустили руки. Они по прежнему готовы делать все возможное
\enquote{чтобы Россия стала свободной}. При этом он выразил уверенность в том,
что \enquote{Россия обязательно будет свободной}, Организация \enquote{Открытая
Россия} закрылась, strana.ua, 27.05.2021

Добре, що ми навчилися відкидати \enquote{\emph{русский} мир}. Погано, що ми ще
не навчилися боротися з нашими рідними терористами-популістами, і країна
залишається у них в заручниках, Сергій Фурса, gazeta.ua, 25.05.2021

Населення території цілком допускає можливість, що \emph{рускій} начальнік
повернеться, Іван Семесюк, gazeta.ua, 25.05.2021

Как сообщал OBOZREVATEL, на Донбассе вооруженные формирования \emph{Российской}
Федерации ранили украинского военного, obozrevatel.com, 27.05.2021

\enquote{Ненависть окутала их глаза}. В Кремле прокомментировали слова
президента Польши о \enquote{\emph{России}-агрессоре}, strana.ua, 27.05.2021

Ранее мы рассказывали, что Кабинет министров \emph{Российской} Федерации
утвердил список недружественных \emph{России} стран. Однако Польши в этом
перечне нет, \textbf{\enquote{Ненависть окутала их глаза}. В Кремле
прокомментировали слова президента Польши о \enquote{России-агрессоре}},
strana.ua, 27.05.2021

Судьба \emph{России} будет решаться на улицах, Виталий Портников,
obozrevatel.com, 27.05.2021

Ведь власть пытается информационно пристегнуть его к \enquote{зраде} Медведчука. И,
даже если вины лидера ОПЗЖ доказать не удастся, медийно Порошенко будут
\enquote{полоскать} и обвинять в сотрудничестве с \emph{Россией}, strana.ua, 27.05.2021

\emph{Россиянин} полез целоваться с лошадью и остался с откушенным носом, strana.ua, 10.12.2020

Как \emph{Россия} воюет на Донбассе: 30 главных расследований,
radiosvoboda.org, 27.05.2021

Князь Николай Трубецкой ещё до войны снискал себе славу одного из наиболее
проникновенных \emph{русских} историков и религиозных философов. Его занятия
привели его к мысли, что Европа в \emph{русском} обществе переоценена именно в
духовно-практическом смысле, тогда как Азия имеет непосредственное влияние на
формирование \emph{русского} менталитета, \textbf{«Евразийская концепция
\emph{русской} истории». Черниговский евразиец Петр Савицкий}, ukraina.ru,
27.05.2021

Подругу Романа Протасевича, администрировавшую канал со сливами личных данных
бело\emph{русских} силовиков и арестованную вместе с ним, зовут София Сапега,
\textbf{У подруги Протасевича историческая фамилия - Сапега}, strana.ua,
28.05.2021

У звіті Facebook сказано, що \emph{Росія} залишається світовим лідером у
виробництві дезінформації, спрямованої на втручання до інших країн. Переважно
цілями її закордонних операцій, окрім США та України, є Велика Британія, Лівія
та Судан, radiosvoboda.org, 28.05.2021

«Роды просто раздавили», Одиночество, нищета и депрессия: истории
\emph{россиянок}, которых заставили рожать нежеланных детей, lenta.ru,
28.05.2021

«Он невероятно умелый манипулятор», Как \emph{Россия} может воспользоваться
окончательным разрывом Лукашенко с Западом?, lenta.ru, 28.05.2021

\emph{Россиянка} выиграла квартиру и тут же ее лишилась, lenta.ru, 28.05.2021

Як хочеться потрапити у світ без \emph{Русской} общины Украины, \emph{Русского}
блока та їм подібних московських шавок.  Як хочеться почути живих Драгоманова
та Ключевського; тих людей, які займались історією як наукою, а не як повією,
\textbf{Забыть все = Забить на всё. Як хочеться усе забути}, zrada.org,
24.07.2010

Саме в цей день рівно 7 років тому наші герої зробили вибір на користь свободи,
навіть якщо за неї треба було віддати життя. \emph{Російські} бойовики
намагалися взяти під контроль міжнародний аеропорт Донецька, вимагали в
українських військових скласти зброю. Наші захисники здаватися не збиралися та
мужньо тримали оборону аеропорту. Підрозділи спецназу за підтримки авіації
завдали першого потужного удару по терористах. Того дня \emph{російські}
окупанти відступили, зазнавши великих втрат, \textbf{26 травня 2014 року – один
з переломних моментів війни на Сході}, bigkyiv.com.ua, 26.05.2021

Более того, \emph{российская} вакцина и разговоры о ней являются элементом
гибридной войны против Украины. В частности, таковым может быть и заявление
Лукашенко, считает Николенко, \textbf{\enquote{В русле подрывных усилий}}...
strana.ua, 28.05.2021

Более того, поскольку мы уже столько лет ведем войну с \emph{Россией}, самый
прям удачный момент отменить вместе с бензином электричество, например. И
антрацит с коксом. В рамках выбранного курса, так сказать, \textbf{Сейчас самое
время отменить вместе с бензином электричество}, Дмитрий Заборин, strana.ua,
29.05.2021

Уже давно, к сожалению, прошло то время, когда \emph{российские} и украинские патриоты
обменивались историческими шуточками и только ими могли и ограничиваться. Одна
из них, с \emph{российской} стороны, была такая:
Если вы так недовольны Москвой, то все ваши претензии – к вашему же
«украинскому» киевскому князю Юрию Долгорукому, который её основал! - zen.yandex.ru, 

Далее человек, выдающий себя за \emph{российского} оппозиционера, требует у
европейцев биткоины и подписаться на его YouTube-канал, учит делать Путину
больно и показывать голый зад \emph{российскому} посольству. Седые господа
вежливо слушают.  Связь никто не разрывает, \textbf{Наши европейские партнеры
готовы договариваться хоть с чертом лысым}, Максим Могильницкий, strana.ua,
29.05.2021

Как было не раз сказано, оная история почти повторяет разворот и обыск
\emph{белорусского} самолёта, устроенного СБУ в 2016-м. Но, как мне (примерно)
написал один из комментаторов с флагом ЕС на аватарке, \enquote{Да мне
наплевать на \emph{белорусский} самолёт. \emph{Беларусь} не член ЕС. А то был
европейкий самолёт, с европейскими гражданами на борту.} Отлично раскрыта тема
кто какого сорта, я считаю.  Но это ещё не всё, \textbf{История с Протасевичем
объемно показывает, кто чего стоит}, Роман Подолян, strana.ua, 29.05.2021

Ан-26 — самолёт со славным прошлым. Машина, засветившаяся даже в голливудском
кино — к примеру, в боевике «Неудержимые».  Но время не стоит на месте. И
скоро, в том числе и из-за провальной политики украинского руководства и
разрыва производственных связей с \emph{Россией}, господство в небе могут
захватить другие машины, \textbf{Привет «Антонову»: \emph{Россия} лишит Украину
важного рынка}, ukraina.ru, 28.05.2021

\emph{Російська} дезінформаційна кампанія в Європі проти вакцини Pfizer.
Розслідування Радіо Свобода встановило дійових осіб, radiosvoboda.org,
29.05.2021

Хоча жодні докази не свідчать про причетність \emph{російських} державних
установ чи чиновників до кампанії в соціальних мережах, їхні меседжі та способи
подачі звучать суголосно до того, що робить PR-кампанія \emph{Російського}
фонду прямих інвестицій, який фінансує і просуває на ринок вакцини Sputnik V,
\textbf{\emph{Російська} дезінформаційна кампанія в Європі проти вакцини
Pfizer. Розслідування Радіо Свобода встановило дійових осіб}, radiosvoboda.org,
29.05.2021

Прессекретар Кремля Дмитро Пєсков прокоментував ці звинувачення. «\emph{Росія}
нікого не дезінформує, \emph{Росія} з гордістю говорить про свої успіхи, і
\emph{Росія} ділиться своїми успіхами щодо першої в світі зареєстрованої
вакцини», – сказав він, \textbf{\emph{Російська} дезінформаційна кампанія в
Європі проти вакцини Pfizer. Розслідування Радіо Свобода встановило дійових
осіб}, radiosvoboda.org, 29.05.2021

Я люблю читать новости. Какими бы они ни были. Хорошими, плохими, леденящими
кровь - без разницы. Читая новости, я ощущаю биение жизни и понимаю, что от
кого бы мы ни произошли, \emph{русские} люди, обезьяны были нашими пра-пра или
инопланетяне, без разницы, наша степень психической надежности - как у
хранилища Форт Нокс, хотя мы и по эту сторону океана, \textbf{Мы - несгибаемая
нация. Нас - только дустом}, Из Питера С Любовью. Юля, zen.yandex.ru,
01.04.2021

Мы вышли из одного древнего государства, приняли одно крещение и в то время
были не просто вместе - мы были одним народом. Нам доводилось разделяться на
разные княжества, бороться между собой за первенство власти, воевать за города
и земли, но как народ мы не разделялись и никто нас не разделял. Для самих себя
и для окружающих мы все именовались \emph{Русью}, \emph{русскими}, ну а
феодальная раздробленность народов была тогда явлением распространенным. Для
нашего народа такая раздробленность добром не кончилась - от внешнего врага
отбиться не смогли, \textbf{\emph{Россия} и Украина - утраченное Содружество},
Igor Novikov, zen.yandex.ru, 29.05.2021

Как ожидание Второго Пришествия в 1492-ом году изменило курс \emph{русской}
истории, Открытая семинария, zen.yandex.ru, 20.05.2021

Пелагея, душа народа, очень тронула ее новая песня, Люблю ее всем сердцем и
душой! Достояние \emph{России}! Слушая ее, я забываю обо всем, душа летает от
такого неимоверного сильного голоса, zen.yandex.ru, 11.05.2021

Тилль из «Rammstein»: любовь к \emph{Росиии} и песни на \emph{русском} языке,
zen.yandex.ru, 26.04.2021

В глазах у девочки Васнецов увидел столько одиночества и чисто \emph{русской}
печали, что прямо ахнул, Юлия Варенцова, zen.yandex.ru, 20.03.2021

Президент Польши Анджей Дуда во время визита в Грузию назвал \emph{Россию}
\enquote{ненормальной страной}.  \enquote{Действия \emph{России} — агрессивные,
имперские, которые отбирают у людей нормальную жизнь, приводят к разрушению
государств, к военным положениям, к гибели людей. <...> \emph{Россия} ненормальная
страна. Это не то государство, которое нормально себя ведет},— сказал господин
Дуда, obozrevatel.com, 29.05.2021

Когда я подготовил в 1985 году свой первый семинар по истории \emph{русских}
интеллектуальных течений, 127 студентов подвои заявки на 12 мест.....  Я не
рассчитывал, что они будут что-то знать о \emph{России}, но считал, что человек
интересующийся историей интеллектуальных течений, должен быть знаком с
классикой мировой литературы, \textbf{Американская элита совершенно оторвана от мировой культуры}, Игорь Заславский, strana.ua, 30.05.2021
