% vim: keymap=russian-jcukenwin
%%beginhead 
 
%%file 13_12_2020.news.ua.obozrevatel.ostapovec_galina.1.covid_pik
%%parent 13_12_2020
 
%%url https://www.obozrevatel.com/society/golubovskaya-myi-nahodimsya-na-pike-ognya-i-ne-mozhem-sderzhat-epidemiyu-teper-nam-nado-spasat-lyudej.htm
 
%%author Остаповец, Галина
%%author_id ostapovec_galina
%%author_url 
 
%%tags covid,ukraina
%%title Голубовская: Украина оказалась на пике огня COVID-19, надо спасать людей, а не обвинять Минздрав
 
%%endhead 
 
\subsection{Голубовская: Украина оказалась на пике огня COVID-19, надо спасать людей, а не обвинять Минздрав}
\label{sec:13_12_2020.news.ua.obozrevatel.ostapovec_galina.1.covid_pik}
\Purl{https://www.obozrevatel.com/society/golubovskaya-myi-nahodimsya-na-pike-ognya-i-ne-mozhem-sderzhat-epidemiyu-teper-nam-nado-spasat-lyudej.htm}
\ifcmt
	author_begin
   author_id ostapovec_galina
	author_end
\fi

\ifcmt
pic https://i.obozrevatel.com/news/2020/12/13/filestoragetemp-2020-12-13t171221-614.jpg?size=972x462
\fi

Массовое тестирование на
коронавирус\Furl{https://www.obozrevatel.com/topic/koronavirus-2019-ncov/} уже
неэффективно, а официальная статистика по заболевшим давно не отображает
настоящих реалий. Руководство медицинской отраслью последние 30 лет разными
главами Минздрава привело к ее полной разрухе. А те, кто сейчас во всем винит
министра Максима
Степанова,\Furl{https://www.obozrevatel.com/person/maksim-stepanov.htm}
намерено манипулируют обществом, сказала в интервью OBOZREVATEL ведущий
инфекционист Украины Ольга
Голубовская.\Furl{https://www.obozrevatel.com/person/olga-golubovskaya.htm}

\textbf{– Эпидемия коронавируса бушует в Украине уже 10-й месяц. Какая у нас ситуация
по состоянию на декабрь?}

– Ситуация напряженная не только в Украине, но и во всем мире,\Furl{https://www.obozrevatel.com/abroad/hronika-koronavirusa-v-ukraine-i-mire-13-dekabrya-bolezn-udarila-novoj-volnoj-obnovlyaetsya.htm} особенно в
странах Северного полушария. У нас она приближается к пику, сейчас один из
наиболее тяжелых периодов. \textbf{Нагрузка на систему здравоохранения колоссальная,
больных очень много. Увеличивается количество тяжелых больных, причем молодых,
до 30 лет}. Два месяца назад такого еще не было. Заболевание постоянно меняет
клиническую картину и пока что в сторону утяжеления течения.

\textbf{– Молодые до 30 лет – это те, у кого есть сопутствующие болезни?}

– Бывают относительно здоровые. Например, небольшая избыточная масса тела,
бывает нет никаких видимых заболеваний. Многое зависит от нашей генетики, но
вообще, развитие инфекционного заболевания зависит от дозы возбудителя, его
вирулентности и реакции иммунитета на него. Это очень сложное взаимодействие.

\textbf{Весь мир и мы проходим адаптацию к новому возбудителю, а он адаптируется к нам.}
Чем это все закончится – точно неизвестно, есть разные прогнозы.

Надеемся, что \textbf{где-то к 2022 году вирус немного потеряет свою агрессивность}, как
это было в период испанского гриппа. Я не очень люблю такое сравнение, это
совершенно разные вирусы, но есть надежда, что он будет циркулировать рядом с
нами пятым коронавирусом.

\textbf{– В последнюю неделю статистика по новым заболевшим в Украине немного
стабилизировалась. По крайней мере она не растет быстрыми темпами, как раньше.
Почему?}

– Не смотрю на \textbf{официальные цифры по заболевшим. Они совершенно не соответствуют
никакой действительности}. В таком масштабной заболевании, чтобы иметь
плюс-минус правильную статистику, надо тестировать огромное количество людей.
Не вижу в этом большого смысла.

\textbf{– Почему же?}

– Тестирование нужно для постановки диагноза и выявлению контактных лиц. В
первом случае тестирование важно, если человек болеет не тяжело. Например,
легкий насморк, еще что-то. Да, желательно его протестировать, хотя в 30 или
даже в 40\% случаях результат может быть ложно-отрицательным.

Врачам, чтобы госпитализировать пациента, по большему счету, лабораторного
подтверждения не нужно. При наличии типичных клинических признаков,
классическая пневмония, контакт с возбудителем – все уже знают, что это. Грипп
мы ведь тоже ставим по классической клинической картине, а не по ПЦР-тесту.

Массовое тестирование важно было проводить вначале заболевания, как это было в
Китае. Для чего? Чтобы быстро локализовать инфекцию. Но у нас для этого ноль
ресурсов, просто ноль. У нас разрушена система, она до сих пор не создана, все
разбалансировано, мы потеряли огромное количество кадров, новых в последние
пять лет не набираем. \textbf{Мы в Украине находимся на пике огня. Сдержать эпидемию
уже не можем, теперь нам надо спасать людей.} Для них ведь это самое важное,
чтобы им было куда лечь, чтобы их спасли и была оказана медицинская помощь.

\textbf{– В какой точке этого процесса находимся?}

– В одном из самых тяжелых периодов. Но это касается не только Украины, но в
целом и всех стран. Посмотрите, что творится в США. Я когда читаю их сводки,
мне кажется, там еще хуже, чем у нас. У них, наверное, втрое меньше больничных
коек на 10 тыс. населения, чем у нас. Там больные лежат в больничных часовнях,
в гаражах – где только есть какое-то помещение – все занято больными. Многие
страны открыли стадионы – все так, как было 100 лет назад. Ничего не изменилось
в противодействии таким болезням.

\ifcmt
pic https://i.obozrevatel.com/gallery/2020/12/13/gettyimages-1290911093.jpg
\fi

Американцы не привыкли к такому наплыву инфекционных больных, как, например,
наши страны. У нас часто были войны-революции, мы очень страдали от многих
эпидемий. Но побороли их, построив самые сильные системы противоинфекционной
защиты в мире. Посмотрите, как в 70-е годы быстро локализовали вспышку холеры в
Одесской области. Заболело меньше тысячи человек и все, дальше это заболевание
не распространилось.

Точно так же, только в гораздо большем масштабе, сделала китайская сторона. Как
только у них кто-то заболевает – они тут же его локализуют. Тут же.

В целом, боюсь, мы входим в новый цикл пандемий мирового масштаба.

\textbf{– Но пока что весь мир занят тем, что спасает людей от нынешней?}

– Да, это так. \textbf{В той же Америке такое количество трупов – по 3 тыс. в сутки. В
Украине плюс-минус статистика по смертям адекватная}. Плюс-минус, всякое
попадается, но не такое как в США, где уже ездят переездные морги, хранящие
погибших. Это страшные вещи, страшные.

Нам сейчас надо концентрироваться на спасении людей. И вот для этого ресурса
начинает не хватать всего и всех – медики валятся с ног, болеют, и не только
потому, что заражаются… Не люблю всяких спекуляций, но когда хотят разруху 30
лет повесить на министра Степанова… А за последние 5 лет все уничтожено просто
дотла! Сколько кадров уехало? Это беспрецедентная цифра. А сколько врачей
выслушало всяких гадостей от предыдущих руководств Минздрава? Мол, все плохие –
и медики, и студенты, а вот они придут и сделают нам все замечательно. Сделали!
Если бы не эта пандемия – не было бы инфекционных стационаров и инфекционистов,
как не стало эпидемиологов. И это в стране хронических эпидемий! Это диверсия
против государства и преступление против народа Украины!

И самое главное – люди, которые в этом участвовали, устраивают истерики, чтобы
снять министра во время "боевых действий"! На голову не налазит! Это те,
которым плевать на людей и врачей, которые поддерживают министра, понимая, в
каких рамках ему приходится работать.

\textbf{– Руководству Минздрава все настолько тяжело дается?}

– Украинцы должны быть благодарны министру за то, что он подписал протокол по
лечению больных. Это первый нормативный документ за последние 5 лет! Вы только
вдумайтесь, каких папуасов из нас сделали! Как это так, чтобы у врачей не было
инструмента для лечения больных? Такой протокол не просто написать в данных
условиях. Даже американцы признают, что благодаря тому, что они научились
лечить, удалось снизить смертность с 7\% до 1,9\%. За счет правильного лечения,
за счет протоколов. Самое страшное, что это все пришлось сделать в таком
галопирующем росте больных.

Но написать протокол – это одно, а обучить врачей всех специальностей, даже
акушеров-гинекологов инфекционному протоколу – это ведь не просто. А эти
погибшие больные, которых мы не успели спасти – остаются у нас в памяти на всю
жизнь. Мы бы их не потеряли, если бы…

\textbf{Не надо тупо копировать так, как в Америке. Так, как у них, у нас никогда не
будет}. В США 15-20\% ВВП отдают на медицину, и при этом коек у них в три раза
меньше. Доступ к медицинской помощи, которую красиво нам показывают в
телевизоре, есть только у 35\% населения. Остальные тыкаются, не могут
проплачивать эти страховки и т.д.

Кстати сказать, \textbf{а при чем тут Степанов, что где-то в каком-то регионе в Украине
нет кислорода}, вот скажите? Министр Степанов в этом виноват? Но ведь у нас все
децентрализовано, у нас каждый на месте маленький гетман, который думает, что
руководит, а на самом деле не понимает проблем.

\ifcmt
pic https://i.obozrevatel.com/gallery/2020/12/13/stepanov1.jpg
\fi

В прошлом году путем огромного сопротивления нам удалось предотвратить принятие
законопроекта, который нам подготовили грантоеды, по децентрализированию систем
противоинфекционной защиты.

{\bfseries 
– Чем бы это грозило?
}

– В каком-то поселке местный голова решал бы: вот эту вспышку мы будем
расследовать, а эту – нет. И это на местах, где нет денег. Люди, которые хотели
принять такое, должны понести полную ответственность. Сейчас они говорят, что
не понимали, у них сознание перевернулось… А мы же их предупреждали. Но
единственное, что они качественно делали – воевали и боролись с
профессионалами.

Во второй части интервью с Ольгой Голубовской, которая выйдет в ближайшее
время, читайте, какой сценарий развития эпидемии коронавируса реализуется в
Украине в ближайшее время и что будет, если начнется массовая передача COVID от
людей к животным и наоборот.
