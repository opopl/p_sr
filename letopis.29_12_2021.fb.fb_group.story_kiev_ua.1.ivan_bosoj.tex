% vim: keymap=russian-jcukenwin
%%beginhead 
 
%%file 29_12_2021.fb.fb_group.story_kiev_ua.1.ivan_bosoj
%%parent 29_12_2021
 
%%url https://www.facebook.com/groups/story.kiev.ua/posts/1828880557308737
 
%%author_id fb_group.story_kiev_ua,gorbov_vadim.kiev
%%date 
 
%%tags kiev
%%title Блаженный Иван Босой, Христа Ради Юродивый
 
%%endhead 
 
\subsection{Блаженный Иван Босой, Христа Ради Юродивый}
\label{sec:29_12_2021.fb.fb_group.story_kiev_ua.1.ivan_bosoj}
 
\Purl{https://www.facebook.com/groups/story.kiev.ua/posts/1828880557308737}
\ifcmt
 author_begin
   author_id fb_group.story_kiev_ua,gorbov_vadim.kiev
 author_end
\fi

\ii{29_12_2021.fb.fb_group.story_kiev_ua.1.ivan_bosoj.pic.1}

Прогуливались  с кузиной по чудом сохранившемуся в первозданном виде тихому и
патриархальному Старообрядческому кладбищу на улице Лукьяновской. Среди
заросших могил киевских купцов и каменных крестов в виде стволов деревьев на
купеческих надгробиях  XIX века сестра сразу  обратила внимание на заботливо
ухоженную могилу с живыми цветами, лампадами и иконами под скромным деревянным
православным крестом.

• Кто здесь упокоен, дорогой кузин? - спросила моя очаровательная спутница.

• Иван Иванович Расторгуев, - уверенно ответил ваш непокорный слуга.

• Чем он был знаменит?

• Иван Иванович более известен как Блаженный Иван Босой, Христа Ради Юродивый.
Знаменитый ктевский блаженный позапрошлого века. Круглый год ходил босиком в
любую погоду, носил цепи-вериги. Похоронен, кстати,  вместе  с веригами.
Покровитель киевскиъ паломников, калек, больных. Благодаря своей славе и
пожертвоаниям имел возможность кормить, лечить и одевать сотни сирых и убогих.
Его келья и приют располагались на Андреевском спуске в основании Андреевской
церкви. Ежедневно там кормили бесплатными обедами до пятисот паломников.
Упокоен в 1849 году и был торжественно похоронен на Щекавицком кладбище. Могила
Ивана Босого сразу стала местом поклонения. Некоторые больные по вере их
исцелялись прямо у могилы.

• А здесь, у старообрядцев,  как он оказался?

• - В 1935 году, когда Щекавицкое кладбище большевики окончательно уничтожили,
честные останки  Иван Иваныча верующие перенесли на соседнее кладбище. Теперь
Святый Иван Босый и с небес отсюда молится о Киеве и его жителях.

Сегодня, 29 декабря, Правславная Церковь отмечает День Памяти Блаженного
Иоанна Босого, Киевского.

\ii{29_12_2021.fb.fb_group.story_kiev_ua.1.ivan_bosoj.cmt}
