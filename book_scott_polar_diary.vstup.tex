% vim: keymap=russian-jcukenwin
%%beginhead 
 
%%file vstup
%%parent body
 
%%url 
 
%%author 
%%author_id 
%%author_url 
 
%%tags 
%%title 
 
%%endhead 

\section{Вступ}
\Purl{https://www.facebook.com/ScottPolarDiary}

Перші три тижні листопада промайнули у такому гармидері, що я закинув щоденник
і можу тепер відновити його лиш із пам’яті.

Дати тут значення не мають, але весь цей час офіцери та команда корабля були
постійно заклопотані.

Після прибуття з корабля вивантажили усе приладдя берегової партії, включно з
хатинами, саньми тощо. Протягом п’яти днів він стояв у доку. Боверс ретельно
оглянув, перебрав і пересклав суднові припаси, вивільнив багато місця,
розвантаживши багато ящиків і склавши їхній вміст у лазареті. Тим часом наш
добрий приятель Міллер оглянув течу і простежив її аж до форштевня. Ми знайшли
тріщину у фальстемі, а в одному місці отвір під наскрізний стрижень виявився
для нього завеликим. З цією трудністю, як я й очікував, Міллер впорався
блискуче, і, оскільки судно вже було на плаву й завантажене, після цього теча
значно зменшилась. Корабель і досі протікає, але води проникає не сильно
більше, ніж варто очікувати від старого дерев’яного судна.

Струм води, який було видно й чутно на кормі, повністю зупинили. Без пари зараз
течу можна стримати ручною помпою, прикладаючись до неї двічі на день протягом
чверті години чи двадцяти хвилин. У тому стані, в якому корабель був до
ремонту, та за нинішнього навантаження це точно займало б не менше
трьох-чотирьох годин щоденно.

Перш ніж судно вийшло з доку, Боверс та Ваєтт знову взялися до роботи із
вантажниками у повітці, перебираючи й сортуючи припаси берегової партії. Все
ніби минуло без заминок. Усі подарунки та покупки з Нової Зеландії було
зібрано: масла, сири, бекон, шинка, м’ясні консерви та язики.

Тим часом на пустищі за межами порту було зведено хатини. Усі деталі
полагодили, відсортували та наново позначили, аби уникнути труднощів на півдні.
Цим ділом займалися наш бездоганний тесля Девіс, Ебботт і Кіоган. Спорудили
великий зелений намет та звели для нього годящі опори.

Коли корабель вийшов із доку, на ньому закипіла робота. Офіцери й команда судна
з групою портових вантажників були заклопотані завантаженням трюмів. Люди
Міллера будували стайні, конопатили палуби, закріплювали рубки, вкладали
стрижні та різні малі пристосунки. Механіки та люди Андерсона працювали у
машинному відділенні, науковці облаштовували лабораторії, кухар переобладнував
камбуз і так далі – на кораблі не лишилося жодного клаптика, не зайнятого
робітниками.

Ми почали завантажувати припаси таким чином: у головному трюмі склали всю
провізію берегової партії та частину хатин; понад ним на головній палубі
надзвичайно щільно спакували решту деревини для хатин, сани, мандрівне
спорядження і більші інструменти й машини, необхідні науковцям. Це значно
зменшило простір для людей, але вони самі визначили його обсяг; вони переказали
через Еванса, аби на них не зважали: вони готові до таких випроб, і кілька
кубічних футів простору ролі не відіграють, – такий їхній дух.

Таким чином, відведений людям простір протягнувся від носового люка до форштевня на головній палубі.

Під баком розмістили стійла для п’ятнадцятьох поні – на більше вже не було
місця. Вузький вільний простір спереду щільно утрамбували кормом.

Одразу позаду бакової перегородки розташований маленький люк, що за негоди був єдиним входом до кают-компанії. За ним йде фок-щогла, а поміж нею та носовим люком – камбуз і лебідка; ліворуч від носового люка стоять стійла для чотирьох поні – дуже міцна дерев’яна споруда.

Ззаду за носовим люком – льодовня. Для цієї кімнати ми роздобули 3 тонни льоду,
162 баранячі туші та 3 коров’ячі, а також кілька коробок із солодким м’ясом та
нирками. Туші складено ярусами, а між ярусами лежать дерев’яні планки – це
справжній тріумф упорядкованого зберігання, і я щиро сподіваюся, що це
забезпечить нам свіжу баранину на всю зиму.

Обабіч від головного люка неподалік від льодовні стоять двоє із трьох наших моторових саней; треті ж стоять впоперек юта там, де раніше була лебідка.

Там же влаштовано склад баків із бензином; інший склад, увінчаний тюками з
кормом, розташовано між головним люком та грот-щоглою, а баки з бензином, гасом
та спиртом стоять уздовж усіх проходів.

Ми розмістили 405 тонн вугілля у вугільній ямі та головному трюмі, 25 тонн у
вільному просторі носового трюму і трохи більше 30 тонн на верхній палубі.

Ці мішки з вугіллям та згадані раніше припаси утворюють на палубі дуже важкий
вантаж, тож острах щодо цього цілком природний; втім, усе, що можна було
зробити, аби закріпити його й убезпечити, ми зробили.

Веремію на палубі довершують наші 33 собаки, прикуті до опор і стрижнів на
льодовні та головному люку, поміж моторовими саньми.

З усіма цими припасами на борту корабель все ще стояв на два дюйми вище
вантажної марки. Баки заповнені стисненим фуражем, окрім одного, в якому
зберігалися 12 тонн прісної води – її, сподіваємось, стане нам до самої криги.

Фураж. Спершу я замовив 30 тонн пресованого вівсяного сіна із Мельбурна. Отс
поступово переконав нас, що цього не досить, і зрештою вага корму для поні
зросла до 45 тонн, не враховуючи тих 3 чи 4 тонн, що призначені для негайного
вжитку. До цього додаткового фуражу входять 5 тонн сіна, 5 чи 6 тонн макухи, 4
чи 5 тонн висівок та трохи подрібненого вівса. Кукурудзи не беремо.

Нам вдалося втиснути всі собачі галети загальною вагою близько 5 тонн; Мірз
уперто не хоче годувати собак тюлениною, але, мабуть, взимку таки доведеться це
робити.

Ми зупинилися у сімейства Кінсі в їхньому будинку “Te Han” у Кліфтоні. Будинок
стоїть на краю урвища за 400 футів над морем, і з нього відкривається вид на
рівнини Крайстчерча та довгий північний пляж, що їх обмежує. Поблизу однієї з
них – пересип гавані та звивистий лиман двох маленьких річок Ейвон та
Ваймакарірі. Далеко за рівнинами видніються завжди мінливі гори, а ще далі за
ними за північним вигином моря у ясну погоду можна побачити прекрасні снігові
шапки піків Кайкура. Видиво неймовірно чарівливе, і такий вид із захищеного
сонячного куточка у саду, що пашить силою-силенною червоних та золотих квітів,
спонукає до відчуття невимовної втіхи усім сущим. 

Вночі ми спали у цьому саду просто мирного чистого неба; вдень я відбув до
своєї контори в Крайстчерчі, а тоді, може, до корабля чи на острів, а потім уже
додому гірською стежкою понад Порт-Гіллз. Цей період приємно згадувати,
незважаючи на перерви, і тоді я мав вдосталь часу для необхідних нарад із
Кінсі. Його зацікавленість в експедиції просто дивовижна, і такий інтерес від
дуже проникливої ділової людини – це надбання, яким я скористався у повні. За
моєї відсутності Кінсі буде моїм представником у Крайстчерчі; я дав йому
звичайну довіреність і, здається, надав у його розпорядження повну інформацію.
Його доброту до нас не описати словами.

\ifcmt
pic https://scontent.fiev6-1.fna.fbcdn.net/v/t1.0-9/127812838_102832275006922_3153222035467068855_n.jpg?_nc_cat=109&ccb=2&_nc_sid=730e14&_nc_ohc=CQ4tnOlOpdEAX-YVhKR&_nc_ht=scontent.fiev6-1.fna&oh=e327d87d54032034840fb5b67ec31d45&oe=5FE3AB37
\fi


