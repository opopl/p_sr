% vim: keymap=russian-jcukenwin
%%beginhead 
 
%%file 10_12_2021.fb.fb_group.story_kiev_ua.2.samyj_samyj_podol.cmt
%%parent 10_12_2021.fb.fb_group.story_kiev_ua.2.samyj_samyj_podol
 
%%url 
 
%%author_id 
%%date 
 
%%tags 
%%title 
 
%%endhead 

\zzSecCmt

\begin{itemize} % {
\iusr{Tatiana V Nskaia}
 @igg{fbicon.heart.growing}  я ходила в 100 школу в шахматную секцию...

\iusr{Таня Солодкая}
всегда любила Подол, и даже сейчас, с Оболони возим детей в 100 лицей

\iusr{Sophie Tiupyushev}
Cпасибо!

\iusr{Павел Мар}
Хаотичность впечатляет! Суперское фото!

\iusr{Ella Zalavsky}

Спасибо за экскурсию в прошлое, я тоже родилась на Подоле, улица Мирная. Скучаю
по родному городу и очень благодарно за возможность пройтись по знакомым местам

\iusr{Alina Monroz}
Живу на Подолі і на фото побачила свій будинок...

\iusr{Iryna Tymoshok}

Спасибо, получилось очень душевно... и моя улица Юрковская, дальше набережная,
где прошли мои детство и юность, очень переплетаются с Вашими воспоминаниями. И
гимназия 19, которую закончила наша доченька, и Ильинская церковь, где они с
мужем венчались... теплом накрыло...

\iusr{Ира Цирульницкая}
Я тоже кореная подолянка, но все очень изменилось, узнать тяжело, сердце болить
все это видеть где прошла молодость.

\iusr{Елена Мирошниченко}
Спасибо!

\iusr{Виктор Задворнов}
Чудесная экскурсия, благодарю!

\iusr{Татьяна Иванова}
Спасибо!

\iusr{Виктория Луцкая}

Я с вами, я с Боричева тока 28, наш дом ( коммуналки) расселили - ощущения те
же, а в 100 школе моя сестра отучилассь

\begin{itemize} % {
\iusr{Vitali Andrievski}
\textbf{Виктория Луцкая} 

я тоже родился и жил в этом доме. Но почему то вас не помню. Напомните
пожалуйста. Или мы с вами жили там в разные времена.

\begin{itemize} % {
\iusr{Виктория Луцкая}
\textbf{Vitali Andrievski} 

моя бабушка Куканова, квартира 32, я у нее была до 4лет, а потом наездами, все
выходные, а мои двоюродные там жили всю дорогу Игорь Кордин и Оксана, помните
их? Это Оксана отучилась в 100 школе, закончила в 91

\end{itemize} % }

\iusr{Vitali Andrievski}

Сапсибо. Вспомнил. Наша квартира была 29 а над нами жила семья отца Глаголева.
Его дочка Мария с мужем Володей и детьми.

\begin{itemize} % {
\iusr{Виктория Луцкая}
\textbf{Vitali Andrievski} 

это вы на 1этаже жили? Потому что священник со второго мою сестру на дому
крестил, тогда нельзя было, папа коммунист, а мама с бабушкой покрестили 65год

\iusr{Виктория Луцкая}
\textbf{Vitali Andrievski} А я с первого этажа никого тоже не помню, вам сколько лет?

\iusr{Vitali Andrievski}
\textbf{Виктория Луцкая} уже 67. Вы конечно моложе. Может быть я видел вас еще совсем ребенком
\end{itemize} % }

\end{itemize} % }

\iusr{Елена Чурюмова}

Спасибо, за теплые слова о Подоле. Люблю приезжать сюда ранним утром и бродить
среди домиков, сохранивших воспоминания, деревьев осталось очень мало. Утреннее
солнце ярко заливает улицы, брызжет и сбивает с ног.

\iusr{марина зак}
Душевно. Спасибо

\iusr{Маргарита Дуда}

Навещая Подол, я как будто в гостях у своїй бабушки. Тепло, пахне детством.
Безусловно, 'модные' тенденции портят сей патриархальный уклад... но флюиды
Подола, это нечто.

\iusr{Зинченко Виктор}
Боже, какой слог - замечательно! Спасибо

\iusr{Света Буренок}
В 100школе учился мой муж Подгурский Сергей а жил он на Дехтярной 13

\iusr{Лідія Ковмір}

Дякую! Наш дім там де наша душа.. мені пощастило залишитись жити там де моя
душа. А ваш допис зачепив до сліз...

\iusr{Ольга Яворовская}

Спасибо!!! Вспомнила. какие чудесные новогодние елки были в \enquote{Пищевике}
во время каникул.

\iusr{Елена Харченко}
Да, не одни зимние каникулы не проходили без посещения \enquote{Елки} в \enquote{Пищевике}.

\iusr{София Бойко}
Спасибо. Родилась на ПОДОЛЕ. Но судьба распорядилась и \enquote{переселила на берег
ЛЕВЫЙ}. Душа осталась на правом берегу @igg{fbicon.heart.growing}  @igg{fbicon.hearts.revolving}  @igg{fbicon.heart.beating} 

\ifcmt
  ig https://scontent-lhr8-1.xx.fbcdn.net/v/t39.1997-6/s168x128/240595295_280280763862242_7752062239402781192_n.png?_nc_cat=1&ccb=1-5&_nc_sid=ac3552&_nc_ohc=kLYYHfRHdI0AX_VjWLe&tn=lCYVFeHcTIAFcAzi&_nc_ht=scontent-lhr8-1.xx&oh=00_AT8Zg5cd6hGW97_8CT9shG_2Ccz64wN9abgkN0sLr8Vl8A&oe=61F1861D
  @width 0.1
\fi

\iusr{Татьяна Гордиенко}
Как я вас понимаю! Подол в моем сердце навсегда.

\iusr{Stella Stavitsky}
Спасибо за экскурс, увы, ничего уже не узнаю .....

\ifcmt
  ig https://scontent-lhr8-1.xx.fbcdn.net/v/t39.1997-6/s168x128/197820122_1581651598693950_778064872271246176_n.png?_nc_cat=1&ccb=1-5&_nc_sid=ac3552&_nc_ohc=rMWx7qMwrsIAX8Nk7WF&_nc_ht=scontent-lhr8-1.xx&oh=00_AT_UtYmxyRz7rAxOIZOzmrxINtjxRJHZ4rTxhMUkVzO-Ow&oe=61F13969
  @width 0.1
\fi

\iusr{Оксана Билык}
Чудесный кадр!!

\iusr{Natalie Lordkipanidze}
 @igg{fbicon.hearts.revolving}  @igg{fbicon.flame}  @igg{fbicon.hearts.revolving} 

\iusr{Владимир Картавенко}
Место Крещения Руси!

\iusr{Нина Кубанова}
А я буду покороче.
Помнится, взобравшись на Цыганку (кажется, она тогда была выше), мы ЭТО всё
видели. И пытались узнать в зелени деревьев свой дом.

\iusr{Нина Бондаренко}
Чудовий нарис і прекрасна світлина. Трошки сумні, але дорогі спогади.

\iusr{Ирина Громовик}
Дякую за допис.

\iusr{Егор Красава Тебякин Егор}
Спасибо!

\iusr{Галя Александрова}
20 и 9 школ, 17 возле Жовтня

\iusr{Петр Кузьменко}
\textbf{Галя Александрова}, я уже признал ошибку. Перепутал. Я писал и о Константиновской.

\iusr{Lena Zabelina}
\textbf{Галя Александрова} молодееец @igg{fbicon.face.tears.of.joy} 

\iusr{Калерия Ходос}

Жили на Подоле на ул, Мирной. К сожалению на месте нашего дома стоит монстр - дом
который построила Билозир просто ужас. К сожалению очень много снесли хороших
домов хоть и старинных. А ведь Подол это сердце Киева и хотелось бы чтоб его
сохранили, а не строили монстрв, жаль...

\iusr{Людмила Бабич}
Благодарю! Мило! Трогательно и очень тепло... это о нашем родном Киеве, хотя я
родом не с Подола, а с ул. Чкалова( с Сенного...!)

\iusr{Анна Загорулько}

Чудный вид на Подол! Контрактовал площадь теперь угадывается по колесу
обозрения. А вот что за красное пятно на переднем плане, понять не могу, это
крыша такого цвета?

\iusr{Алла Власова}
Все мы родом из детства и родные места самые лучшие. А Подол особенный район Киева.

\iusr{Yurii Kadochnikov}
За клубом КВВМПУ видніється церквушка. Там дочку хрестили

\iusr{Татьяна Гурьева}
Спасибо, увидела все любимое Вашими глазами подолянина. В груди защемило...

\iusr{Galina Golubitskaya}

\ifcmt
  ig https://scontent-lhr8-1.xx.fbcdn.net/v/t39.1997-6/s168x128/46949788_1074619346032831_6110986696902311936_n.png?_nc_cat=1&ccb=1-5&_nc_sid=ac3552&_nc_ohc=7FTt2PqfqqcAX9F-k29&_nc_ht=scontent-lhr8-1.xx&oh=00_AT_y6k1cJs-NM2fEvXaSu1Z9RR-bUziAb4gKlyqJdW9tGw&oe=61F169B7
  @width 0.1
\fi

\iusr{Татьяна Домаскина}

Спасибо! Люблю гулять на Подоле, рассматривать старинные дома! Сама родилась на
Зоологической и местом наших игр был зоопарк. @igg{fbicon.face.smiling.eyes.smiling}  Лазили через дырку в заборе!
Счастливое детство!!!


\iusr{Vitali Andrievski}
Петр спасибо. С такой любовью, точностью и легкой ностальгией описан почти каждый дом нашего Подола.

\iusr{Natalija Grouzman}

Вам всё удалось! Прекрасное фото и очень интересное, трогательное до слез
описание Подола. Очень Вам признательна за рассказ.


\iusr{Ирина Касьянова}
 @igg{fbicon.hands.applause.yellow}  Прекрасная ремарка получилась @igg{fbicon.face.smiling.eyes.smiling} 

\iusr{Татьяна Сирота}

Хорошо написано.
Тепло,с любовью!
Спасибо,Петр,за прекрасный пост!
И фото замечательное,его автор большой молодец!

\iusr{Людмила Пятыгина}

Как прекрасно написано! Как вроде мы в один класс в сотой школе ходили. Все как
моими глазами, спасибо! Очень тронуло! @igg{fbicon.heart.red}

\iusr{Светлана Сокур}
Огромное спасибо!!!

\iusr{Oksana Shostal}
Дякую @igg{fbicon.heart.suit}

\iusr{Александра Пихур}
СПАСИБО за фото, можно взять в свою ленту?

\iusr{Петр Кузьменко}
\textbf{Александра Пихур}, нужно! @igg{fbicon.wink} 

\iusr{Светлана Александренко}
Спасибо за прекрасный пост, за красоту и любовь к нашему Киеву!

\iusr{Инна Валентиновна}

Ну, что такое Киев без Подола? Как святой Владимир без креста.. Нас все меньше
, кто так трепетно и душевно ценит Подол и старый Киев. Носителям памяти
старого города, низкий поклон.

\iusr{Инна Шполянская-Кисиленко}

Очень хорошая панорама. Фото отличное. Да взгляд на наше детство и молодость.
Когда приезжаю в Киев всегда гуляю здесь. Ноги сами знают дорогу.  @igg{fbicon.100.percent}{repeat=4} 

\iusr{Людмила Старовойтенко}
Спасибо за рассказ и такие искренние воспоминания классссс

\iusr{Лилия Клименко}
красивый вид

\iusr{Ирина Медына}

Подол нельзя не любить...

\iusr{Михайло Присенко}

«..остался подолянином в душе...»! Так! Це на все життя і в мене, і в друзів
мого зростання! Дякую Вам!

\iusr{Светлана Никитаева}

Когда смотришь на это великолепие - душа радуется. Наш красавец Киев Ведь это
наше родное Спасибо огромное

\iusr{Светлана Никитаева}

\ifcmt
  ig https://scontent-lhr8-1.xx.fbcdn.net/v/t39.1997-6/s168x128/240595295_280280763862242_7752062239402781192_n.png?_nc_cat=1&ccb=1-5&_nc_sid=ac3552&_nc_ohc=kLYYHfRHdI0AX_VjWLe&tn=lCYVFeHcTIAFcAzi&_nc_ht=scontent-lhr8-1.xx&oh=00_AT8Zg5cd6hGW97_8CT9shG_2Ccz64wN9abgkN0sLr8Vl8A&oe=61F1861D
  @width 0.1
\fi

\iusr{Римма Грановская}
Спасибо за рассказ @igg{fbicon.thumb.up.yellow} 

\iusr{Валентина Ненько}

Спасибо Вам большое за прекрасное описание нашего родного Подола!
 @igg{fbicon.hands.pray}{repeat=3}  @igg{fbicon.thumb.up.yellow}  @igg{fbicon.rose}{repeat=3}  @igg{fbicon.christmas.tree}  @igg{fbicon.evergreen.tree} @igg{fbicon.heart.red}{repeat=3}

\ifcmt
  ig https://scontent-lhr8-1.xx.fbcdn.net/v/t39.30808-6/266199089_451036536419486_4590411649613146539_n.jpg?_nc_cat=103&ccb=1-5&_nc_sid=dbeb18&_nc_ohc=uEGuYnik87kAX_aAyX4&_nc_ht=scontent-lhr8-1.xx&oh=00_AT_qVNpyTzUDBGDh6w-lXViC4NhZOVgEDgNQDtfgQH3MIg&oe=61F2F86B
  @width 0.2
\fi

\iusr{Петр Кузьменко}

Благодарю Вас за хороший отзыв о моих публикациях! Тётя Валя! Надеюсь Вы
помните нас с \textbf{Нелли}? Здоровья Вам и всего самого доброго! @igg{fbicon.heart.sparkling} 

\iusr{Ирина Шах}

Петя, я восхищаюсь твоими постами, они такие душевные и искренние, пропитаны
любовью к жизни, к нашему родному городу, Подолу. И все это не просто написано
красиво и литературно правильно, а когда читаешь, то погружаешься в сам
рассказ, проживаешь все твои воспоминания и ощущаешь твои чувства. Спасибо, за
твой талант за твою душу и доброе сердце, за твои воспоминания. Пиши
обязательно. Творческих тебе успехов.

\begin{itemize} % {
\iusr{Петр Кузьменко}

Ирочка, благодарю! Для меня очень важно твоё мнение! Очень хотелось бы свои
скормные эпистолярные пробы дотянуть до твоих поэтических успехов! @igg{fbicon.heart.sparkling} 

\iusr{Ирина Шах}
\textbf{Петр Кузьменко} ой, не выгонят меня в краску, я любитель

\iusr{Петр Кузьменко}
\textbf{Ирина Шах}! Любитель - конкурсов победитель!  @igg{fbicon.heart.sparkling} 
\end{itemize} % }

\iusr{Любовь Белоцерковец}
Спасибо! Ваша ностальгия передалась и мне, хотя я не жила на Подоле...

\ifcmt
  ig https://scontent-lhr8-1.xx.fbcdn.net/v/t39.1997-6/s168x128/47270791_937342239796388_4222599360510164992_n.png?_nc_cat=1&ccb=1-5&_nc_sid=ac3552&_nc_ohc=G1GVKIjvpyEAX_1fHdn&_nc_ht=scontent-lhr8-1.xx&oh=00_AT_hkEcpXbAVd_Sxri5DICuIQvUYlW393nkBm8Bs0RV-AA&oe=61F27636
  @width 0.1
\fi

\iusr{Ольга Красникова}

Отец моего мужа, родился на Подоле, там жила вся его большая семья. В Ильинской
церкви он был крещён, позже уже его сын, а теперь и его внук там крещён и
венчан. Когда дедушка, как мы его все называли, был уже в почтенном возрасте, и
жизнь его подходила к закату, сын решил деду напоследок сделать приятное, устроив
ему экскурсию на автомобиле, так как самостоятельно он уже никуда не
выходил... Для старика было огромным счастьем побывать на родном Подоле, и даже
встретиться с соседкой по дому где когда-то жил, побывать на месте, где
работал, вообщем, дедушка был счастлив до слёз). Все мы родом из детства...


\iusr{Ирина Тарасенко}

Политехническая, Полевая, Борщаговская... Спасибо. что ассоциативно напомнили о
родных местах

\ifcmt
  ig https://scontent-lhr8-1.xx.fbcdn.net/v/t39.30808-6/266366656_681813499864657_7558197347169573387_n.jpg?_nc_cat=111&ccb=1-5&_nc_sid=dbeb18&_nc_ohc=wbOJf4n2BQ4AX_Pk1n9&_nc_ht=scontent-lhr8-1.xx&oh=00_AT-xkp7MsTy_mCnTdghq8db8s5jHZQc_5ncl0BOQYURyzg&oe=61F23012
  @width 0.3
\fi

\iusr{Gennadiy Smertenko}
А без Подола наш Киев невозможен, как святой Владимир без креста !!!!!

\iusr{Алла Горлач}
@igg{fbicon.heart.red}

\iusr{Евгения Королёва}
Благодарим!!!

\iusr{Pavel Prischepa}
'А без Подола Киев НЕВОЗМОЖЕН'

\iusr{Yefim Rozovsky}

В одном позвольте не согласиться. Рядом с 20й школой была 9яшкола. 17я
находилась за кинотеатров Жовтэнь.

\begin{itemize} % {
\iusr{Петр Кузьменко}
\textbf{Yefim Rozovsky} , Я уже признал эту ошибку. Писал о 17 школе в статье о Константиновской. Тут перепутал просто.

\iusr{Yefim Rozovsky}
\textbf{Петр Кузьменко} тогда извиняюсь Я не читал другую статью. О чём очень сожалею. В 20 Я учился. В где то 59- 60е годы.

\iusr{Петр Кузьменко}
\textbf{Yefim Rozovsky}! В нашей группе, нажав на имя или аватарку автора, можно прочитать всё его статьи.

\iusr{Yefim Rozovsky}
\textbf{Петр Кузьменко} СПАСИБО!

\iusr{Мила Бормотова}
\textbf{Yefim Rozovsky} а я в эти годы училась в 107 школе на улице Ратманского, а жила на Почайнинской, но сейчас подол не нравится ,за его омоложение.
\end{itemize} % }

\iusr{Olga Pyankova}

Спасибо за такой классный рассказ! Обожаю Киев, 35 лет в нём прожила. Душевно
очень написано, трогательно до слёз! Пусть всегда будет возможность побродить
по родным и любимым местам!@igg{fbicon.heart.red}

\iusr{Людмила Гриценко}

Рідні місця. Набережно-Микільська, тепер Сковороди, останній двір. Коли
приходили баржі з кавунами, знайомі вантажники стукали у вікно і прямо на ліжко
вкидали найбільші кавуни. Всі фільми здебільшого дивились в \enquote{Пищевике} ....Потім
ще з обов'язкового, влітку, морозиво, кульки, в кафешці на місці пам'ятника
Сковороді. Бабуся працювала в 15 лікарні, дідусь на мельниці,

А потім у нашому дворі було, здається, відділення ГАІ і рідною стала Русанівка.
Але Поділ - це любов на завжди

\iusr{Александр Бубенев}

Моряк, это хорошо, но когда моряк еще и добротный литератор, это очень хорошо.
Чувствуется тонкая организация души автора. Респект.

\iusr{Зоя Хужко}

Звмечательный снимок! Современный Древний Подол.  Родитель Киева!!!. Мало чего
сохранилось из прошлого, К сожалению, но мысль о том что, именно от сюда
начинался прекрасный город Киев волнует и радует история его возникновения

\iusr{Елена Гонж}

А без Подола Киев невозможен... Спасибо за фото и трогательные воспоминания...

\iusr{Ольга Рожненко}
Дякую!!!

\iusr{Vadim V. Bandrin}

Вот можете меня забанить, но эта цитата из блатной песенки: \enquote{А без Подола Киев
невозможен} уже достала меня по самое НЕ МОГУ! Удачи вам, ПОДОЛЯНЕ!

\begin{itemize} % {
\iusr{Александра Пихур}
\textbf{Vadim V. Bandourin} не завидуйте, грех!

\iusr{Светлана Манилова}
\textbf{Vadim V. Bandourin}, тише, ша... Замените слово на синоним... @igg{fbicon.face.hand.over.mouth} 

\begin{itemize} % {
\iusr{Vadim V. Bandrin}
\textbf{Светлана Манилова} Заменил. А толку-то?!  @igg{fbicon.smile} 

\iusr{Светлана Манилова}
\textbf{Vadim V.}, смысл тот же, но глазу и слуху приятнее. @igg{fbicon.smile} 

\iusr{Vadim V. Bandrin}
\textbf{Светлана Манилова} ОК!  @igg{fbicon.smile} 
\end{itemize} % }

\iusr{Олег Коваль}
\textbf{Vadim V.}, мы ценим Ваше мнение и блокировать не будем. Это не демократично...

\begin{itemize} % {
\iusr{Vadim V. Bandrin}
\textbf{Hélg Smith} 

Ну я честно высказал свою мысль. Самоцензуры у меня нет и не было. Честно же
всё рассказал! Можно спорить со мной. Можно ругаться со мной, но - зачем?


\iusr{Олег Коваль}
\textbf{Vadim V.}, Да.

\iusr{Олег Коваль}
\textbf{Vadim}, 

хочу немного акцентировать Ваше внимание на том, что мы, в нашей группе, не
устраиваем святые войны, мы просто вспоминаем нашу историю... Такой, какой она
была, без купюр.

\iusr{Vadim V. Bandrin}
\textbf{Hélg Smith} 

Ясное дело. Просто у меня отвращение врождённое к блатным песенкам. Я об этом и
высказался. Не особо понимаю, зачем об этом так долго говорить. Просто странно,
что \enquote{НАРИД} в каждом третьем посте вспомнил вот этот блатнячок. Ну что же
непонятного, если я ПРОТИВ этого? Это - моё мнение по поводу этой песни, и я
его высказал.

\iusr{Олег Коваль}
\textbf{Vadim}, совершенно верно.
\end{itemize} % }

\iusr{Vitali Andrievski}
\textbf{Vadim V. Bandourin} согласен

\iusr{Lena Pavlikova}
\textbf{Vadim V. Bandourin} А вы пропойте дальше, - \enquote{Как Святой Владимир без креста!}. И все станет на свои места .

\iusr{Vadim V. Bandrin}
\textbf{Lena Pavlikova} 

Ой, нет. Увольте. Уже пропел не полегчало.  @igg{fbicon.smile}  Ну не люблю я приблатнённую
тематику. Мне вот не сложно взять гитару и пропеть сие. Только смысла в этом
движении не вижу!  @igg{fbicon.smile} 

\end{itemize} % }

\iusr{Татьяна Приймак}
Мой Подол!!! Люблю !!!

\iusr{Оксана Фурман}
Очень душевно. Очень.

\iusr{Tetyana Zerova-Lyubimova}
Дякую Вам! Ви чудово написали!

\iusr{Зоя Луценко}

Вы про скверик с памятником Зое Космодемьянской упомянули, в 1964 году меня
назвали этим красивым именем, в честь неё. И жили мы на Жданова, 29. Всем
привет!!!! @igg{fbicon.face.smiling.hearts}{repeat=4} 

\iusr{Oksana Didkivska}

Фото прекрасне! Кольорова гама гармонійно поєднується з чудовою архітектурою
древнього міста.

\iusr{Анатолий Коробка}
БРАВО!!! Где еще можно погуляти ...

\iusr{Татьяна Зубко Маркина}

Спасибо. прошлась по Подолу. На Андреевском пробабка. На углу Волошской и
Хоревой близкие родственники. Мама жила. С отцом познакомились. Работали. Море
воспоминаний

\iusr{Марина Михайловская}

\ifcmt
  ig https://scontent-lhr8-2.xx.fbcdn.net/v/t39.30808-6/266523213_461805098880545_976800246212230376_n.jpg?_nc_cat=102&ccb=1-5&_nc_sid=dbeb18&_nc_ohc=hHUBe_RtKkAAX8FxMNd&_nc_ht=scontent-lhr8-2.xx&oh=00_AT8CPkKPsiPfnwTAU8Wdxi67Tq9aud_2b2FXEw2PRae39Q&oe=61F21352
  @width 0.3
\fi

\iusr{Татьяна Долгополова}
Очень трогательно и искренне

\iusr{Лідія Шевченко}

ЯК ТЕБЕ НЕ ЛЮБИТИ МІЙ ПОДОЛ.

\iusr{Валентина Нагорна}

Дякую за цікаву прогулянку!

\iusr{דמיטרי קוגן}

Классный пост. Как будто вернулся в детство. Только при мне не было домов
пионеров а были девятая и двадцатая школы. А в непрезентабельном доме
Стрельбицкого я прожил пять лет у бабушки. Только тогда он казался нам
таинствннным замком с его чёрными ходами и проходняком в шестнадцатый номер. А
когда подросли бегали во двор курить на переменах. Да.... Детство моё прошагало и
нет.

\iusr{Nadejda Bublik}
Любимый, Родной, подол лучшие годы прошли на подоле.

\iusr{Мария Корнева}
ПОДОЛ это что-то! Спасибо за ремарку и душевность.

\iusr{Barbara Novokhatska}

Спасибо большое, также думаем, вспоминаем, тоскуем по молодости, ее не вернёшь,
а любимый кусочек Подола всегда с нами...

\iusr{Надежда Воробьева}
Чудово, дякую
Такого міста, як у нас і на карті не існує, дякую автору!)

\iusr{Ирина Ицковская}

Спасибо огромное, словно окунулась в беззаботное, подольское детство,
вспомнила "Пищевик, где занимались волейболом и бегали в кино.... Спасибо, так
стало тепло

\begin{itemize} % {
\iusr{Vitali Andrievski}
\textbf{Ирина Ицковская} в пищевике на новогодних елках были самые хорошие и вкусные подарки.

\iusr{Ирина Ицковская}
\textbf{Vitali Andrievski} согласна!
\end{itemize} % }

\iusr{Світлана Сабліна}

Чудовий допис... ніби знову пройшлася подільськими вулицями...


\iusr{Julia Panchul}
техника растёт. прекрасный вид Подола. Спасибо

\iusr{Людмила Васьковська}

Я родилась и выросла на Полевой. А с Подолом подробно познакомилась, когда мои
внуки начали учиться в 100том лицее. Мне тоже тепло и уютно на этих улицах и в
старинных церквях. Здесь особая аура. Спасибо харизматичное место нашего
города.


\iusr{Галя Лищинская}
Обожнюю Поділ!

\iusr{Oksana Duksenko}
Красиво, спасибо

\iusr{Candro Nomadi}
Я тоже родился на Подоле, жил, учился и живу в родном районе.
Правда жизнь бросала в разные города и веси, но в душе и сердце я житель Подола!

\iusr{Тетяна Леонова}

Спасибо за ваши преданньіе детству и Подолу чувства и дар их душевно изложить и
поделиться с когортой однодумцев!..

\iusr{бАШИНСКАЯ Марина}

100-я школа. И директриса Нина Алексеевна Федорова. И прогулки на \enquote{продленке}
на Боричевом Току. И любимый Подол. Кто здесь не родился, тот и не поймет.
Спасибо, Петр.

\iusr{Татьяна Опанасенко}
За ремарку спасибо! Я не подолянка, но с детства люблю эти места.

\iusr{Nik Nik}
Очень красиво

\iusr{Галина Процько}
Потрясающая тональность съёмки!!! И красиво безусловно!!!

\iusr{Елена Троян}

Хм... памятником Зои Космодемьянской Вы меня чрезвычайно удивили. Неужели в
Киеве их было два? Поскольку один из бетона или гипса, выкрашенный серой
масляной краской много лет стоял в скверике на углу Чапаева и Коцюбинского,
потом сменился какой-то фривольной грудастой бабой в летнем платьице, потом по
требованию общественности заменён нынешним памятником (перенесено в сквер на
углу Чкалова и Хмельницкого) и в конце концов заменен памятником Гончару. Что ж
за памятник был на Подоле?

\begin{itemize} % {
\iusr{Петр Кузьменко}
\textbf{Елена Троян} 

стандартный советский. Таких было много. Даже сейчас, в нашем нынешнем городке,
буквально в нескольких сотнях метров от нас есть такой же, у школы. На фото.

\ifcmt
  ig https://scontent-lhr8-1.xx.fbcdn.net/v/t39.30808-6/265276975_4805934949458751_3540308907747162039_n.jpg?_nc_cat=100&ccb=1-5&_nc_sid=dbeb18&_nc_ohc=uhwWJqabpewAX9OmKuo&_nc_ht=scontent-lhr8-1.xx&oh=00_AT9B3i5ufKi4d1X5Yk55VSzpFYgCNjZkNDzU-QCbV6xNSw&oe=61F3439D
  @width 0.2
\fi

\iusr{Елена Троян}
\textbf{Петр Кузьменко} 

Спасибо! Как удивительно, именно такой стоял в сквере у нас. И краска не серая,
а серебряная. А какой это год? Я 1956 го и помню его с начала 60-ых. Может он
переехал?

\iusr{Петр Кузьменко}
\textbf{Елена Троян} 

нет. Их было два в Киеве. Это те, про которые я знаю точно. Может и больше.
Тогда штамповали много по всей стране.

\end{itemize} % }

\iusr{Сергей Янчуковский}
Хорошо сказано...

\iusr{Виктория Луцкая}

В конце 80х все было заброшено, в замке Ричарда мы играли в жмурки, а в церкви
напротив 100 школы катались на цепях от люстры... А вообще и я моя семья с
Подола... Мамы уже нет... а она родилась в 43 на Межигородской... а потом семья
переехала на Боричев ток

\iusr{Александра Пихур}
\textbf{Виктория Луцкая} на Межигорской

\iusr{Ирэн Васильевна}
Красота

\iusr{Людмила Билык}

Спасибо большое, Пётр! Замечательное фото и замечательный рассказ... Очень
люблю Подол... Раньше это было любимое место прогулок...  @igg{fbicon.thumb.up.yellow}  @igg{fbicon.smile}  А, ещё,
вспомнилось, как мы с подружками бегали на танцы в Высшее Военно-Морское
Училище...  @igg{fbicon.wink} 


\iusr{Greg Trosman}

Я вырос в этом районе Подола. Жил на Братской улице недалеко от старейшей и
бывшей складом тогда Ильинской церкви. Закончил 19 школу, катался на одной лыже
с Андреевского Спуска, маршировал вместе моряками из Полит училища, когда они
готовились под духовой оркестр к параду, и у меня до сих пор есть бескозырка,
играл в Народном Театре клуба Пищевиков, ходил туда же на танцы и ходил в
секцию бокса Спартака, что был возле 100 школы и тд и тп. Это было в настоящем
детстве в далёкие 60е. Память храню до сих пор!


\iusr{Dina Shapovalov}
Спасибо @igg{fbicon.hands.pray} @igg{fbicon.heart.red}

\iusr{Алина Маркина Сыченко}
Очень важная для души фотография...
Спасибо!

\iusr{Татьяна Башинская}
За что прощать то? Это Ваша жизнь, боль, радость, детство, юность!! Очень
красиво, такой обзор!!!!!


\iusr{Антон Бобровский}

Просто Чудесно! Спасибо Вам за тёплый рассказ, который греет душу!)

Не останавливайтесь и по возможности берите со \enquote{своим бездушным} товарищем
дроном Новые кадры Нашего Любимого Города!!!

\iusr{Юрий Шаповалов}

Да, Петр, покорили вы нас своим рассказом о Подоле! Действительно, это самое
известное место в Киеве. А с какой любовью вы нам обо всем поведвли! Я в 1962
-1963 годах проживал на Кругло-Университетской, но даже это короткое время на
всю жизнь осталось в моей памяти, как о самом светлом времени... Спасибо вам
большое за ввше творчество!

\iusr{Ivan Kisel}

Великолепный вид и ракурс! Выглядит объёмно. Мысленно, глядя на фото, можно
побродить подольскими улочками... Спасибо.

\iusr{Лида Жукова}

да..... так и остались..... я до14 лет жила на верхнем валу..... дворик..... В 8. 00
приезжала во двор машина с молоком ....дети за молоком с бидончиком

\iusr{Ирина Курасова}
Спасибо огромное! Прекрасные воспоминания о родном Подоле! И снимок чудесный!

\iusr{Лида Жукова}
и вообще подол есть. был и останется

\iusr{Смирнова Вера}

Спасибо за искреннюю любовь к Подолу и, соответственно, к Киеву. Потому что,
как известно: \enquote{А без Подола Киев невозможен...}

\iusr{Валентина Прищепа}

Вот это настоящий патриотизм! Тепло, душевно с мельчайшими подробностями, так как
их воспринимает живой человек. У каждого из нас есть такой кусочек планеты.
Браво!


\iusr{Iryna Osadcha}
Фото чудове! Без злобудов рідний Київ.

\iusr{Valentina Ch}
Прекрасная картина! И ремарка совсем не тоскливая, а душевная

\iusr{Viktoria Terpylo}
У вас получился чудесный пост и фото!

\iusr{Рона Ильинична}

Отличная фотография.

\iusr{Sergey Shestopal}
До слез

\iusr{Любовь Александрова}
Спасибо, хорошее фото и прекрасное описание Подола!

\iusr{Eugene Babin}

Фото очень свежее. Вырос я, изменился город.. Раньше каждый день много лет
подряд ходил из дома по улицам, которые тут все видны, в ясли, садик, школу, до
метро/трамвая в институт.. А теперь ещё и езжу по новому мосту, где тут горят
огни, что уносит куда-то далеко-высоко ))


\iusr{Любовь Григорьева}
\textbf{Петр Кузьменко}, 

как же Вы любите свой Подол! Он того стоит! Да же не знаю, рассказ о Вашей
жизни или о Подоле, все так переплетается. И фото замечательное! Спасибо!

\iusr{Наталья Яблонская}
Спасибо за рассказ о старом милом Подоле.

\iusr{Mamane Vita}
Очень хорошая фотография! Пройти бы здесь...

\iusr{Лариса Богдан}
Місто народження моє це подол проживала до 1976. Все моє дитинство і юність пройшла на подолі

\iusr{Александр Сорокин}

Большое спасибо автору. Всегда с душевной теплотой пишете о Родине. Счастья,
доброго здоровья, любви, удачи и радости жизни многая лета.


\iusr{Александр Венге}
Отличная получилась ремарка.

\iusr{Анна Сидоренко}
Красивая фотография, почти весь Подол охватила.

\iusr{Лариса Дзюбенко}
Спасибо!

\iusr{Валентина Іванівна}
Прекрасно написано, так начебто ходила з Вами тими вулицями.

\iusr{Nadiya Kolosova}
Прекрасный рассказ! Спасибо за фото!! Оно действительно отличное! Я тоже его помню таким !

\iusr{Наталия Платонова}
Нескончаемая любовь. Спасибо!

\end{itemize} % }
