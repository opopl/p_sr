% vim: keymap=russian-jcukenwin
%%beginhead 
 
%%file books.slovnyk_RA
%%parent body
 
%%endhead 

\section{СЛОВНИК РА (ВОСКРЕСІННЯ СЛОВА)}

Містерія

Від упорядника

«Словник РА» (інша назва — «Воскресіння Слова»[2]) — не науково-фантастичний
трактат про витоки мови. За допомогою «символічної етимології» Олесь Бердник
шукає в сучасній українській мові слова, що якнайкраще зберегли залишки
праслов’янського сонячного культу (автор називає його «Ра»). Тож цей твір можна
назвати культурно-світоглядною мовознавчою розвідкою, яка не претендує на
науковий статус.

В Україні «Словник РА» публікується вперше.

Спочатку було Мовчання,

З Мовчання з’явилося Слово,

Слово стало Матір’ю Мови,

Мова народила Людину.

Людина — батько Духу,

Дух у сяйві Слова

Повертається до Великого Мовчання.

Де зачинається Слово?

Хто дав йому путь і гарт?

З якої виходять основи

Пісня, роздум і жарт?

Стежки не паперові

Слово вперед ведуть, —

Буйне дерево Мови

Корінням сягає в Суть…

Суть — Материнське Лоно

Дії, світла, життя,

Те, що над всі кордони

Творить Вічне Життя!..

Слово — Дитя Великого Мовчання — заблукало на манівцях людської псевдоісторії.
Сини Хаосу нав’ючили на Слово вантаж безглуздя, жорстокості та суєтності,
перетворили Боже Дитя на віслюка для своїх забаганок, знущань, підлих замірів
та мізерної втіхи. Туман нерозуміння оповив свідомість людей, бо правдива суть
Слова затьмарена брехнею і відносністю. А затьмарене Слово жити не може у світі
облуди й ганьби: воно задихається і вмирає на хресті щоденного рабства, у
темниці безпросвітності.

Зруйнувати домовину Слова, повернути Боже Дитя Людині, воскресити в слові
безцінну вартість Материнської Першосуті — мета і обов’язок творчого мислення.

Ходімо ж тисячолітніми стежками рідного народу, спробуємо розшукати на них
загублені скарби материнської полум’яної Мови.

Слово — диво Всесвіту.

Його появу жадають пояснити хитромудрими науковими теоріями, начіплюють на
Слово безліч одежин різних епох і народів, аж доки воно забуває про своє
первородство і стає кріпаком того чи іншого мовного насильника.

Слово — не раб потреби, не дитя необхідності, а втаємничене серце Вічного
Буття. Кореневі слова в будь-якій мові не просто стихійне поєднання звуків, а
природне звучання тої Першооснови, котра стоїть за явищем, річчю, істотою,
суттю. Видозміна слів у різних народів — лише різне забарвлення того ж таки
вияву завдяки переломленню в розмаїтих скельцях прагнень і психік.

Зовнішній Космос (природній всесвіт) і Внутрішній Космос душі тісно, нерозривно
пов’язані. Той зв’язок виявляється в слові, а опісля — в думці, у дусі.

Абетка — ніби чарівна арфа. Кожна буква — струна. Сонце, зорі, дерева, квіти,
птахи, ріки, моря, вітри, люди, звірі, гори, поля — весь розмаїтий світ Великої
Матері — торкається вібруючих струн, творячи чаклунську симфонію Мови. У
природному всесвіті ми знаємо ту першооснову, котра лягла підвалиною буття,
стала чарівною ниткою, що з неї зіткано наше життя: то є дарунок
Світовида-Ярила, невидима Божа вібрація, вічне джерело радості, енергії, сили —
ПРОМІНЬ СВІТЛА.

Променю в мові відповідає звукосполучення РА. Він і є вібраційною основою
абетки, котра збуджує до дії, до життя інших своїх братів, напоює їх силою і
певністю, радістю і повнотою вияву. Доки людина не оволоділа в дитинстві звуком
РА, вона інтуїтивно відчуває, що мова її ще неповноцінна.

Могутній вібраційний звук РА і стає по праву першонародження символом сонця,
світла, радості.

Бог Сонця РА відомий багатьом народам.

Самоназва сонця — РА — могла зникати, замінювалася іншими (Світовид, Ярило,
Сонце, Дажбог і т. д.), але всі мовні сонячні діти залишилися в душі народу
протягом тисячоліть.

Гляньмо…

Радість (Ра-дість, Ра-дає… тобто це почуття дає Ра, світло, сонце).

РОБОТА (рос. РАБОТА) — (Ра-бути, бути в обіймах Ра, буття радості, світла,
натхнення). Так високо пращури оцінили роботу, творчість, як продовження місії
сонячного променя на землі. Ось чому РОБІТНИК, ТВОРЕЦЬ — істинні діти Сонця,
його воїни, його виконавці, його промені. Ось чому за ними майбуття, а не за
тими, хто віками визискував їх, не маючи повноважень РА-СВІТОВИДА. Оскільки РА
є основою самого буття, то перемога робітників, творців неминуча, а поразка
синів темряви — лише справа часу.

РОЗУМ (рос. РАЗУМ) — дивовижне поєднання понять УМ (уміти, збагнути, оволодіти)
і РА/РО. Ра з Ум дають Людину Мислячу, котра дарунок світла — РА сполучує з
умінням творити, поєднувати хаотичні елементи стихій у гармонію осмисленого
буття. Це слово, крім ознаки наявності осмислення, усвідомлення, творення, —
означає і об’єднання (РАЗОМ, тобто воєдино, вкупі). Отже, місія Розуму є
всеоб’єднання світу, всіх його проявів. Людині Мислячій доручено бути
господарем, будівником Космосу.

РАЗ. Початок лічби. Вона природно бере початок від Сонця. Кожен бачить ЄДИНЕ
джерело буття щоденно, тому звук РА законно очолює число.

Разом з тим пращури інтуїтивно відчували небезпеку роздрібнення світу, тому

РАХУВАТИ — це Ра-ховати, затемнити єдиний промінь, подрібнити його в
нескінченних проявах розділеного світу (Розділити — це РА — ділити).

РАДА — від слова РАДІСТЬ. Радитись — це опромінитись РА, радісно об’єднатися в
спільному прагненні знайти основу, прийнятну для всіх. Не може бути «злої
ради», «поганої ради», бо це вже ЗРАДА, тобто відхід від РАДИ, РАДОСТІ, від РА.
Отже, зрадники прирікають себе на вічну загибель, бо вони перестали бути синами
Світла, розірвали пуповину життя.

Не вживайте поєднань «погана порада», «недобра рада» тощо. Рада, порада можуть
бути лише світоносні, добрі, щирі, сердечні.

РАЯТИ, РОЗРАЯТИ — вернути в душу людини РА, світло, Радість. Ра — це також і
країна Світовида, Сонця, Ра, казкова країна щастя.

РАЙДУГА — світоносна дуга, ворота Ра, брама Сонця.

РАНОК — схід Ра над світом.

РАЛО, РАЛИТИ — орати землю (О-РА-ТИ), підготувати її до весняної сонячної пори,
пори Ра.

ПОРА — по-Ра, після Ра (час після сходу Ра, коли людина зобов’язана встати до
праці. Пізніше слово «пора» стало універсальним, кажеться, мовиться навіть
«нічна пора», а це неможливо по суті слова: ПОРА — лише сонячний ранок. Ось
чому «на порі» — це означає «в розквіті сил, радості, творчості, життя».

ПАРУБОК — юнак, котрий на порі, готовий до одруження, що шукає собі пари.

ПАРА — подружжя, об’єднане променем РА, освячене вогнем. Хлопець і дівчина
прадавніх часів питали одне одного:

— Ти радий мені?

— Ти рада мені?

— Стрибнеш зі мною через РА?

— Стрибну.

Вони бралися за руки, стрибали через буйне багаття і ставали ПАРОЮ, обвінчаною
богом РА.

ПРА — найдревніший, найдавніший. Той, що почався ще в епоху Ра — першосвітла. Ось чому

ПРАЩУР (П-РА-ЩУР, Пра-чур). Чур — хранитель роду, предок. Пращур — сонячний предок.

РАМА — той, хто несе промінь РА Прекрасне поєднання понять МА (мати, воліти, бути в наявності) і світла Ра. У нашій мові лишилося лише побутове значення цього слова (віконна рама, що пропускає світло в хату). Певно, візантійська навала винищила первородне поняття, котре, все-таки, лишилося на сході в епосі про героїчного Рама, посланця сонячного бога Вішну, котрий утвердив сонячний культ у герці з втіленням темряви і зла — Раваною3. (Вішну — це Вищий, Вишній, Горішній, тобто знову той же таки Ра.)

ПРАВДА — слово, котре вживають усі жерці релігій, усі соціологи, всі філософи, всі апологети містичних чи політичних вчень. Мається на увазі (здебільшого) правда як точна інформація, відомості про те чи інше явище, ситуацію, подію. Але це лише поверхове значення дивовижного слова, котре складається з ПРА — тобто найдавніший, той, що народжений від РА, і ВЕДА — знання, розуміння.

3 Виникає запитання: хто ж тоді РА-ВАНА? Але це запитання риторичне, над яким можуть подумати ті, хто захоче продовжити мовознавчі розвідки автора. — Прим. упор.

Отже, правда — це глибинне, вічне, первородне знання, притаманне самому життю.

Разом з тим:

КРИВДА — це і криве знання, і прикрите (КР), і вкрадене.

РА, РО, РУ входило в мову пращурів не лише як основа вогненності, чистоти,
вірності, сонячності, а й як єдність, об’єднання, зведення докупи різних
чинників або як ознака спільноти, рідності, єдинокровності.

Наприклад.

БРАТ, СЕСТРА, безсумнівне родичання через РА. Тим більше, що родити — це ДАТИ у
світ нове РА (Ра — дати), звідси — рід (род), народ, родина.

РЯД — як зведення роздрібненості в спільну шеренгу, випрямлення.

ПРЯСТИ — готувати спільну нить з розтріпаного волокна.

СТАРАТИСЯ — діяти для РА.

ОБРАТИ — взяти за основу РА.

Можна знайти ще багато слів у рідній мові, у мовах інших народів, котрі
ствердять вібраційну суть втаємниченого звуку РА (РО, РУ, РЕ, РІ), але справа
не в нагромадженні прикладів, а в знаходженні ключа розуміння того, що повинен
збагнути кожний творець: мова є першожиття народу, і саме вона формує дух
нації. Хто зрікається рідної мови, той перестає бути Людиною.

Ходімо далі. Розглянемо, яка основа лягає в динаміку буття? Безумовно, рух.

РУХ — основний прояв РА у зовнішньому світі. Звідси — РУКА (ру-х-а) — орган
людини, який рухає, діє, творить,

РУШАТИ, РУШИТИ, РУШЕННЯ — мінлива гра поняття, як у напрямку творчому, так і в
руйнівному.

РІКА — як рух води.

РІК — рух часу.

РІЧ, РЕКТИ — рух мови.

РОК, РОКОВАНІСТЬ — течія долі.

КРАЙ — межа, до якої сягає РА. Звідси, КРАЇНА — земля, напоєна силою, полум’ям,
щедрістю РА, земля, рідна для даного народу. Ось чому УКРАЇНА, УКРАЙНА —
дивовижна, сокровенна самоназва, в якій безодня значень. Це і земля, не
обмежена видимими кордонами, а та, котра кличе до РА, вслід за сонцем, це і
РАЇНА — багата, сонячна; це і УКРИТА, невидима, прихована в лоні РА. УКРАЇНА —
це не тільки географічний терен, держава, сукупність степів, лісів та рік, а,
передусім, духовна країна Свободи, Радості, Мужності, Пошуку, небесна країна
РА.

Саме тому козаки Січі Запорізької боронили не тільки межі рідної землі, а й
головну суть буття України — ВОЛЮ, РАДІСТЬ НЕПІДЛЕГЛОСТІ. Вони могли
відтворити, відродити НОВУ УКРАЇНУ де завгодно — на Дунаї, в Малій Азії, на
Кубані, в Америці, аби лише вдалося зберегти Зерно Свободи. Як відомо, цього не
сталося, земні козаки загинули навіки, а небесні, легендарні, пісенні рушили
шляхом РА у вічну мандрівку до Небесної УКРАЇНИ…

РА Святий, прийди у нашу хату,

Розбуди нещадно і зігрій.

Хай всміхнеться Вічно Суща Мати

Усміхом кохання і надій!

Хай у небі, на землі, в безмежжі

Загримить всепереможний РА.

Темні розпадаються мережі —

Народилась Радості пора!

А тепер перейдемо до укладення невеликого словника, в якому спробуємо
прослідкувати утворення пишних китиць, гірлянд, віток слів від первісних,
кореневих понять, котрі випливають із самої суті буття.

За яким принципом укладати словник? За абетковим? Але ж у житті, у природі
такого принципу немає. Космічна Першосуть, котра розгортається в бутті Людини й
Природи, не знає і не має деспотичних програм: вітки її Древа розкриваються за
глибинним законом родинної послідовності, любовної спорідненості, а насамперед
— важливості тої чи іншої частки буття, котра має проявитися тепер, у цю мить,
у цьому місці. Наприклад, коли розпукується зерно, то воно запускає в землю
корінець, щоб дати паросткові опертя: змінити послідовність цих проявів
неможливо. У мові — те саме: є слова-корені, на них спираються інші слова,
котрі творять стовбур, гілки, а потім і листя та квіти велетенського Древа. Це
такі слова, як БУТТЯ, СУТЬ, ВЕЛИКИЙ, МАЛИЙ, РУХ, ГОДІ і т. д. Значення РА ми
вже збагнули в заспіві до словника.

Коли пращури вийшли з лона Великої Матері-Лади, то первісні парості звуків
гуртувалися, наростали довкола урочистого кореня Р: усі голосні просто-таки
липнули до цієї могутньої вібрації, творячи найважливіші поняття: АРА, УРУ,
ОРО, ІРІ, ЕРЕ. З цих звукосполучень ми й почнемо наш СЛОВНИК РА, Словник
Радості.

Ар’ї, райці чи ірійці, де ви?

Лиш у мові пломенять сліди.

Зерна обтрусило буйне Древо,

А саме пропало назавжди.

Не пропало! Повстає у Слові,

У піснях, легендах і казках.

Скоро, скоро загримить ізнову

Під полками Рами битий шлях!

Віто, Віто — життєдайна сило,

Заховалась Ти у глибину.

У словесну запхано могилу

Душу Світу вічно-вогняну.

Воскресай, пресвітла Диво-Мати,

І життя нетлінне розбуди!

В зореносних, огнетканих шатах

Із глибин незмірності гряди!

*

А-а-а! — Первонароджена голосна. Перший паросток мови, котрий одразу спирається
на корінець «Р».

Ар, Ара. Перша тріада, священна трійця Мови, таємниче зерно Внутрішнього
Космосу душі. Від цього зерна починається мова.

А оскільки мова формує Людину, то Людина й називає себе дитиною АРА, арійцем,
сином Вічного Вогню, Світлоликого Сонця-Ра, променем Незримого багаття.

Але звук «А» — Первонароджений, він — Дитя Боже, його не можна оголювати перед
стихіями ще хаотичного світу. Інтуїтивно розуміючи це, наші пращури заховали
священну тріаду в покривала вторинних звукосполучень (жар, кара, мара, дар і т.
д.), а там, де «А» мусило стояти попереду, вони поставили один з останніх
звуків, котрий означав саму людину, Особу — «Я». Це й означало, що «А» — Син
Бога, схований в «Я»2, прикритий ним, захищений від навали стихій.

Отже, «Я» прикрило своєю плоттю «А», захистило його заборолом, твердинею інших
звуків. Рідна мова УКРАЙНИ священно берегла цей заповіт пращурів, і лише
пізніше, коли голос першопредків загубився в далині, ми впустили в огненні
шеренги Мови чужаків, таких як: авторитет, аборт, агностик, актуальний, акція,
акт, антимонія і так далі, і тому подібне…

Отже, пращури не називали себе арійцями. Це можна бачити з того, що всі
споріднені з АРА слова прикриті символом Особи — «Я».

ЯРИЛО — сонце, бог Світовид, Дажбог, пломеніюче джерело життя в небі. Яке
могутнє слово! І які повнокровні, нерушимі гірлянди слів воно породило!

ЯРИЙ — сонячний, вогняний, нестримний.

ЯРІТИ — пломеніти.

ЯРИНА — сонячний, весняний посів.

ЯРКИЙ — сліпучо-ясний (ми даремно відмовилися від цього чарівного слова).

ЯРМО — навіть це слово, що стало ознакою рабства, означає засіб для оранки
(ЯРАНКИ) землі під ярину.


ЯТРИТИ (ярити, наярювати) — допікати, розпікати.

ЯРИТИСЯ, РОЗ’ЯРИТИСЯ — шаленіти, лютувати, набиратися вогню, мужності, відваги.

Отже, ми не арійці, ми — ЯРІЙЦІ. Ми — РАЙЦІ. Ми — ІРІЙЦІ. Це останнє —
найточніше, бо слово ІРІЙ — далечінь, глибінь, безмежність, той край, куди йде
Ра-Сонце, Ярило, — породило багато-багато прегарних слів, котрі ми вже
промовляємо мертвими вустами, не відчуваючи, яка полум’яна сила закладена в
них.

Гляньте самі:

ІРІЙ — далечінь, невідома земля, безкрайня глибінь, невивчена країна, неохопний
простір; туди мандрує ЯРИЛО-РА, туди ховаються вдень зорі, туди летять птахи,
туди прямують душі людей, коли їх закликає до себе, на нову РА-СТЕЖКУ великий
Пращур-Світовид.

Саме цією дорогою, вслід за РА, повів свої загони за мрією воєвода РАМА. Той
шлях вивів його через хребти Кавказу до плоскогір’я ІРАНУ (країна Ірію), а далі
— через ріки Інд, Ганг у долину між найвищими горами світу, де було засновано
могутнє царство Ар’яварта. ВАРТА — збереглося в рідній мові теж, як чатування
біля АРА, біля вогню. Отже, АР’ЯВАРТА — твердиня аріїв, іріїв, синів Ра. Там
пощастило нащадкам РАМИ зберегти осколки прадавніх ВЕД, батьківських традицій,
втаємничені манускрипти філософських прозрінь тих далеких часів, коли суть
світу вогненно розкривалася перед щирими (Щ-ИРИ-ми, тими, що бачать далеко)
очима наших предків.

А на материнську землю, на РУСЬ-УКРАЙНУ, котилися й котилися хаотичні, злобні
вали напасників, завойовників, щоб знищити синів Світовида, їхню сонячну
культуру, їхню ПЕРШОМО-ВУ. Який жахливий історичний герць! Які втрати! Яке
широке поле, вкрите трупами вояків, зрадників, безликих боягузів,
переродженців, духовних сліпців та поодиноких героїв! Дим над згарищами, туман
над мертвим полем. І навіть вороння вже не крякає над трупами, бо звикло
випивати очі звитяжців, а нині довкола трупи сліпців, у котрих замість очей
були при житті глибокі пустоти!

На що ж опертися, щоб повернути рідній землі ЯРУ СИЛУ?

Лише на РА, на його одвічну зброю — МОВУ, в котрій він заховав полум’яні стріли
та списи! Недаром же СИН РА — ТАРАС, Кобзар України сказав пророче:

\obeycr
Возвеличу
малих отих рабів німих,
Я на сторожі коло них
поставлю СЛОВО.
\restorecr

Прилітайте, прилітайте здалека, птахи великого РАМИ, принесіть на крилах своїх вістку від ЯРИЛА. Прилітайте з таємничого ІРІЮ…

ЗІРКА — та, що сяє з ІРІЮ, здалека, з глибини неба.

ЗІР — те, що дозволяє, дає можливість дивитися в ІРІЙ, очі людини — фізичні й духовні, котрі пізнають невичерпну глибінь природи.

СІРИЙ — мряка, туман, схований ІРІЙ.

ВІРА — почуття, притаманне душі, серцю, котре веде нас в ІРІЙ, у безодню пошуку, почуття, котре не вимагає доказу, а випливає із самої суті буття.

МІРА — те, що обмежує ІРІЙ, дробить його для вивчення, вводить в число, вагу та в інші визначники.

ОБРІЙ — обрис, коло, доки ми можемо заглянути в ірій.

ДІРА — отвір, що веде невідомо куди.

ЩИРИЙ — відкритий до ІРІЮ, завжди готовий слухати голос безміру.

Ми бачимо, як слово ІРІЙ, ознака безміру, всесвіту в його незмірності, природно
лягло в основу філософських узагальнень пращурів. Вони виводили космогонічні
поняття не з штукарських, хитромудрих, псевдологічних побудов, а з першородних
відчуттів серця й розуму, котрі бачили правду буття такою, яка вона є, як
бачить батька й матір дитя, котре починає вчитися розмовляти…

ПРІРВА — невимірна глибінь.

ПІРНУТИ — заглибитися.

МИР — все, що далеко, у стані єдності, спокою, злагоди. Означає і стан дружби, і взагалі всесвіт.

Ми наводимо лише для прикладу жменьку кореневих слів від основного поняття ІРІ, але кожна мисляча людина може збагнути, як багато живих віток та листків відгалузилося від них, творячи древо мови.

Вище ми зазначили, що першонароджену трійцю АРА пращури сховали в тканину слів. Але ознака вогненності, цільності, спільноти, мужності і, нарешті, майстерності у всіх творчих виявах неодмінно супроводжується цим звукосполученням.

ЖАР — не потребує пояснень. Олово прадавнє: хто мав жар, той був з вогнем. З жару роздувалося багаття, АР.

ПАРА — над аром, над вогнем. Так же називалися і хлопець та дівчина, що стрибали над РА (ми вже згадували цей прекрасний звичай пращурів).

ДАР — цінне, щире, чисте, дороге приношення.

ДОРОГЕ — теж від ДАРУ, але воно спочатку не означало цінність грошову чи якусь
іншу, — лише духовну, сердечну.

ЦАР — той, хто володіє чарами влади (ЦАР-ЧАР). А влада вважалася суттю вогненної сили, АРА. Хто володів такою силою переконливості та мудрості, ставав водієм, царем. І саме слово «володар» означає те саме — ВОЛОДІТИ АРОМ. Ще одне тотожне поняття — ГОСПОДАР.

ЗНАХАР — той, що ЗНАЄ вогняну науку, а вогняними, тобто АРНИМИ, були всі тайни
лікування, ворожіння, поклоніння прадавнім богам.

СТАРЕЦЬ — СУТЬ АРЕЦЬ — людина, котра, проживши вік, сповнена мудрістю АРА.
Після введення візантійства це слово стало лайливим, презирливим, хоч воно
означало в епоху сонячного культу мудрих мандрівних людей.

ЧАРИ — наука про таємний вплив на людину намови чи певного зілля,
приготовленого над вогнем. ЧАРОЮ досі називають в деяких селах України
сковороду.

ЖАРТИ, ЖАРТУВАТИ — веселитися, тішитися, бути щирим, сміхотливим, вогняним. По
цьому слову видно, що пращури не знали злих жартів, поганих жартів, жарт може
бути лише доброзичливий, хоч і пекучий.

ГАРНИЙ — ясний, чистий, красивий (знову ж таки РА). КАРА — відправити К АРУ,
спалити, знищити. МАРА — приховування світла, вогню, ясності, сонця. Якщо РА
стоїть спереду, як у слові РАМА, то це означає перемогу, першість світла, а
якщо М, МА спереду — то це перевага темряви. Наприклад, МРЯКА, ХМАРИ, МОРОСЬ,
МОРОК, МАРНИЙ (тобто безнадійний). Навіть МРІЯ — те, що заховане в далекому
ІРІЮ, те, що ледве видно. Але в ІРІЮ ще видно, хоч і дуже далеко, отже за мрією
йдуть, щоб досягти її.

ГОНЧАР, ЧИНБАР, ТЕСЛЯР, БОНДАР — назви майстрів майже завжди закінчувалися на АР, як ознака завершеності, досконалості, АРНОСТІ.

СТАРАННІСТЬ, СТАРАТИСЯ — невтомно, як вогонь, як АР, творити, працювати.

ГАРЯЧИЙ — ясно без зайвих слів.

ВАРТИЙ, ВАРТА. Слова ніби різні, але вони тісно зв’язані між собою. ВАРТА — це чатування біля вогню, захист АРА. А ВАРТІСТЬ має лише те, що захищене.

ГАРТ, ГАРТУВАТИ — випробувати, зміцнити вогнем.

КАРБУВАТИ — випалювати вогнем.

МАРНУВАТИ — спопеляти, розтринькувати.

ЗМАРНІТИ — худнути під впливом життєвого горіння, а разом з тим перетворитися
на примару, різко змінитися на гірше.

Проте, досить наводити приклади. Ми не воліємо охопити всі можливі значення і
прикладення вогняної трійці АРА. Це зробить кожна мисляча людина.

Перейдемо до освоєння ще кількох кореневих понять, котрі є незамінними у сформуванні цілої лавини слів.

Наприклад, БУТТЯ.

Кожному ясно, що це — основа основ і для осмислення життя Людини у Всесвіті, і
для визначення тих чи інших явищ, відповідно до їх буттєвої значимості.

БУТИ — означає «знаходитися тут, усвідомлювати себе наявним у житті світу», але
разом і багато іншого. Наприклад:

БИТВА — герць за ствердження буття. Отже, Буття — це вічний бій з силами хаосу
і темряви.

УБИТИ — усунути істоту з буття.

БАТЬКО, БАТЯ, АТА, ТАТО, ОТА — той, хто дає буття, хто зачинає життя. Тут ми
маємо можливість збагнути прекрасні слова, поняття ОТАМАН І ГЕТЬМАН (ГОТАМАН).
ОТА, АТА — батько і МАН — ЛЮДИНА. Отже, ВОЖДЬ ЛЮДЕЙ, батько людей. Ми в
історичному поході втратили МАН, МАНУ. Воно лишилося тільки в слові МАНДРІВНИК,
МАНДРУВАТИ (людина, що йде за РА, за сонцем), або МАНИТИ (кликати, звати), або
навіть МАНА (як примара, обман). Чітко можна прослідкувати вплив візантійства,
яке прискорило деградацію священних прадавніх понять, що залишилися тільки на
Сході: МАНУ як першопредок людей. Людина з великої літери, і АТМАН — як
Першодух, що веде людство до шляху Буття. Проте, у козаків Запорізької Січі
збереглося священне визначення поняття ОТАМАН, і це ще раз підтверджує, що саме
тут, в Україні, була колиска РАМИ та його сподвижників.

ДБАТИ — діяти, щоб бути.

БАГАТТЯ — БОГ АТА, Бог-Батько. Таке чарівне визначення дали вогневі, полум’ю
наші пращури, бо, справді, без вогню не була б сформована Людина.

БУДУВАТИ — творити для буття.

БУДУЧЧИНА (майбуття) — те, що має БУТИ.

СВОБОДА — дуже прадавнє слово, складається з понять СВІЙ, СВОЄ і БУТТЯ, БУТИ.
Отже, СВОЄ БУТТЯ, незалежне ні від кого. Ще одне визначення — САМОПРОБУДЖЕННЯ,
адже БУДИТИ теж походить від БУТТЯ, бо лише той, хто не спить, по-справжньому
відчуває Буття.

БАДЬОРИЙ — пробуджений, несплячий. Це поняття понесли наші предки на Схід,
звідки воно вернулося до нас у легенді про БУДДУ, що означає теж — пробуджений,
просвітлений, чатуючий.

При переході Б в П виникає ще одна гірлянда слів, дуже гарних і глибоких: ПУТЬ,
ПУТТЯ і т. д.

ПУТЬ — дорога Буття. Щоб бути, треба вічно йти, рухатися.

ПУТТЯ, ПУТНІЙ, ДО ПУТТЯ — те, що годиться для буття, для життєвого шляху.

ПУТО — те, що не дає рухатися, вийти в путь.

Від БУТТЯ народилася також чарівна гілка ВІТА, ЖИТТЯ. Наша мова не зберегла це
слово в його першозначенні, але воно міцно зросло в безліч необхідних, живих
донині слів та понять.

ВІТАЮ, ВІТАТИ, ПРИВІТАТИ, ПРИВІТНИЙ — все це означає життєдайний, зичення
життєдайності.

СВІТ (с-віт) — означає ВСЕ ЖИТТЯ. Предки так і розуміли, що вся видима природа
— єдина жива істота.

СВІТЛО — (з вітою, з життям) — те, що несе життя. Вражаюче мудре прозріння, до
якого ми на основі наукової аналізи прийшли тільки недавно: життя на землі
породжено світлом. А про те, що пращури знали, як саме енергія променя
засвоюється на землі, свідчать слова

ВІТТЯ, ВІТИ — життєдайні, ті, що засвоїли, прийняли СВІТЛО.

КВІТИ — ті, що несуть в собі життя.

ВІТЕР — життєвий, енергійний.

ПОВІТРЯ — теж життєдайне, бо без нього нема життя.

РОЗВИТОК — розкриття ВІТИ, ЖИТТЯ, котре заховане в зерні, в яйці або в пуп’янку
рослини, птаха чи істоти, чи навіть в Яйці Всесвіту, як про це мислили наші
предки.

ВИТИСЯ, ВИТОК — теж зв’язано з ВІТОЮ. Пращури, безумовно, знали про спіральний
хід розкриття життєвих потенцій.

ЗВІТУВАТИ — оповідати про життя.

СОВІТ, СОВІТУВАТИ — допомогти в житті. Даремно гадають, що це російське слово —
наш народ ніколи не зрікався його. Близьке поняття, настільки ж космічне —
РАДА.

Ясно, що від цих слів кожен виведе безліч похідних, які іскряться в нашій мові,
показуючи її глибінь і мудрість предків.

Від ВІТИ, а отже від БУТТЯ походить також група слів, зв’язаних з поняттям
ВЕДА, ВІДА — тобто знання про Всесвіт. Адже процес ВЕДИ, ПІЗНАННЯ, ВІДАННЯ
нерозривний з БУТТЯМ, з ВІТОЮ, зі СВІТЛОМ, з можливістю ВИ-ДІТИ, БАЧИТИ ЗАКОНИ
ЖИТТЯ.

До візантійської навали, котра знищила пращурівську культуру і все, що було з
нею пов’язане (традиції, перекази, пісні, казки, засоби письма, храми сонячного
культу), мудрість віків передавали від покоління до покоління ЗНАХАРІ, ВЕДУНИ і
ВІДЬМИ. Про знахарів ми вже казали (той, що знає науку вогню), до них ще
ставилися поблажливо. А от на долю ведунів та відьом дісталося безліч лих — цим
найменням називали всіх чоловіків та жінок, котрі зберігали крихти прадавнього
знання. Адже

ВЕДУН, ВІДЬМА — це ті, хто відає, хто має (МА) веду, знання.

ВОДІЙ (ВЕДІЙ), ВОЖДЬ — також походить від ВЕДИ і ВІТИ, бо лише той може бути
ВОЖДЕМ, хто ВІДАЄ ЖИТТЯ (ВІТУ), хто ВИДИТЬ далеко, хто знає, якими шляхами
ВОДИТИ.

Закони дивовижної видозміни слів, але також і їхню неодмінну спільність у
походженні показує слово, яке ми вже згадували — БАТЬКО. Ось його
трансформація: БАТЬКО, БАТЯ, ТАТО, АТА, ОТА, ОТЕЦЬ, ВІТЕЦЬ. Усім ясно, що БАТЯ
від БУТИ, а ВІТЕЦЬ од ВІТИ, ЖИТТЯ. Так замикається коло єдності між всеохопним
буттям всесвіту і життям людини.

Перш ніж перейти до розгляду інших кореневих груп слів, подумаємо, як саме
народжуються слова від спільного зерна, скажімо, такі поняття, як горішній,
долішній і все, що зв’язане з ними. Все відбувається природно просто:
найближчими, найжиттєвішими у всі віки для людей були вогонь і вода. ГОРІТИ —
від АР, це вже ми знаємо. Пращури спостерігали, що полум’я завжди лине в небо,
отже напрямок горіння і став горішнім, що дало назву ГОРА, ГІРСЬКИЙ, ГОРБ.

Вода ллється (ли-ється) вниз. Той напрямок названо дольним (ДЕ ЛЛЄТЬСЯ). Звідси
долина, долоня і т. д.

Від горіння також походять ГОРЕ (пече, як вогонь), ГІРКИЙ (теж пекучий), ГІРШИЙ
та інші.

Це показує, як пращури підмічали споріднені явища і природно, інтуїтивно давали
їм назви, котрі нерушимо відповідали їхній суті.

До речі, про слово СУТЬ. Воно означає основу явища, речі, істоти. Народ
спочатку додавав слово «суть» до тих понять та слів, які хотів підкреслити як
головні, визначальні, а пізніше «СУТЬ» почало зливатися з кореневим словом,
творячи спільність, наприклад:

СТАРИЙ (СУТЬ АРИЙ, тобто мудрий, завершений), ми вже казали про це.

СУТІНЬ — на суті тінь (у даному разі суть — це світло, світло затьмарене, отже
настали сутінки).

Оскільки СУТЬ означає НЕРУШИМІСТЬ, то народжуються відповідні слова:

СТІНА, СТІЛ, СТАН, СТАВОК, СТАВАТИ, ПОВСТАННЯ — у корені всіх цих слів СУТЬ,
котра підкреслює, що ця річ чи явище стійкі, надійні, нерушимі.

Тепер визначимо головніші слова й поняття, що творили та творять наше мислення,
а отже сприйняття й формування Буття.

ГОЛОВНІШІ — від слова ГОЛОВА. Отже, треба визначити походження слова ГОЛОВА.
Безумовно, це найдавніше слово, поряд з РА, ВІТА тощо. Пращури прекрасно знали
значення голови, її функції, її роль у творенні усвідомлення Буття — і в
позитивному, і в негативному плані.

ГОЛОВА — складається з двох слів: ГО (гея-земля, грецькою мовою — гйо або гьо)
і ЛОВИТИ. Отже, СФОРМОВАНИЙ ЗЕМЛЕЮ ВЛОВЛЮВАЧ ПОЧУТТІВ. А ЛОВИТИ походить від ОБ
— яйце. Звідси — ОВИД, ОБІД (до колеса), ОБІДДЯ, ОБВОДИТИ, ПОВНИЙ (п-овний,
круглий). ЛОВИТИ — це Л-ОВИТИ, вводити в коло потреби, затримувати. Значить,
ГОЛОВА — ЗЕМНЕ ЯЙЦЕ, де зароджуються почуття первісного сприйняття світу. Але,
оскільки все затримане, ув’язнене, не може бути вільним, самим собою, то ясно,
що почуття людини розірвані, розп’яті, роздрібнені на частки. Цільність
Всесвіту зникає. Промінь РА, котрий увійшов у земне тіло, затьмарився, вмер,
пропав. До речі, «розіп’ясти» зовсім не означає «прибити до хреста», бо було б
«четвертувати». А РОЗІП’ЯСТИ — це РОЗДІЛИТИ НА П’ЯТЬ. Отже, ГОЛОВА розділяє
єдність світу на п’ять почуттів. Це знали наші пращурі-мудреці — ВЕДУНИ. Від
них це знання було запозичене, передане на Схід арамейцями (АРА-МА, той, що з
РА, син РАМИ). Основна думка ясно вирізьблюється. ГОЛОВА або череп
по-арамейсько-му ГОЛГОФА (майже точне звучання). Там розпочинається Син РА, там
же його мучителі ділять між собою цільнотканий хітон (отже, руйнують єдність
світу). Проте, коли форма, яка поглинула світло РА, розпалася, лягла в землю,
ПЕРШОПРОМІНЬ повстає в усій красі та могутності, щоб повернутися до БАТЬКА —
РА.

Усім відомо, які страшні перетворення пережила протягом тисячоліть ця прекрасна
містерія Духу, як вона полонила людей, спотворивши головну ідею сонячного
культу РА.

Ту ж саму драму відображено в переказі про Голіафа й Давида. ГОЛІАФ — велетень,
якого ніхто не може здолати — та ж сама ГОЛОВА-ГОЛГОФА, котра всіх розпинає,
вбиває, обманює. Виступає супроти велетня ДАВИД (ДЕВИД — син Світла, бо Дева,
Діва — це божественний, променистий, світлий, чистий). Він вбиває насильника з
допомогою ПРАЩІ (ПРАЩУР), тобто з допомогою одвічного знання РА.

Те ж саме і в сказанні про САМСОНА (САМ І СОНЦЕ), котрого вороги полонили й
відтяли йому волосся (символ променів). Знову роздрібнення єдності світла.
Вороги змушують осліпленого Самсона (поховане світло) грати їм (символ
підневільної творчості світлих сил). Але богатир, знайшовши точку опертя,
руйнує палац ворогів і нищить їх усіх. Та сама ідея визволення через руйнацію
форми, котра полонила суть.

До речі, назви ХРИСТОС, КРІШНА, слово ХРЕСТ — означає ПРИХОВАНА або ЗАХОВАНА
СУТЬ, поглинута суть (ХР і СТ). Таким чином стає ясною вся вражаюча тисячолітня
містерія Людини, в голові і серці якої відбуваються вселенські трагедії Буття.
Не потрібно робити з Космічного Дива містичних жупелів, котрі привели до
кровопролиття і обману, до створення безлічі принизливих культів, що зробили з
Людини мізерну ляльку кровожерливих «богів».

Проте, може, хтось зауважить, що ГЕЯ не слов’янське слово? Мовляв, наше слово —
ЗЕМЛЯ. Це не так. Земля — пізніше поняття, і воно не мало спочатку космічних
масштабів, адже земля — це і просто — ґрунт. ЗЕМЛЯ — це ЗОО і МА, та, що має
життя. А ГЕЯ-ГО — це вже вселенське поняття, котре вбирає в себе і значення
планети, і матерії, і живої суті світу, що проявляється в безлічі різних
створінь. У нашій мові значення ГЕЯ, як планета, не збереглося, воно перейшло в
Елладу (знов-таки наше слово, котре означає ЛАД, спільність, спілку
міст-полісів), але збереглися похідні визначення, такі як:

ГОДУВАТИ — тобто ГО-ДАВАТИ, ГО ДАЄ. Предки знали, що Земля — Го — єдина
годувальниця, що дає їжу, тіло, притулок, спокій, вдоволеність. Тому
з’являється прекрасне слово, котре означає завершеність, самодостатність —

ГОДІ — тобто досить, не треба більше, наявність повноти. А далі — ціла лавина
слів:

ГОДИТИ — вдовольняти.

ГОД — повний цикл обігу землі-Геї довкола сонця.

НЕГІДНИЙ — незавершений.

ПОГОДА, НЕГОДА — різні фази стану природи — приємний і неприємний.

ЗГОДА — об’єднання в спільному почуттів і думок людей.

ЗЛАГОДА — тобто досить зла. заборона для зла.

ЛАГОДИТИ — відновити цільність, єдність чогось. Отже не може бути поганого
лагодження, а лише ГОДНЕ.

Друга вітка від ГО-ГЬО-ГЕЇ дає теж цікаві вислови, котрі підтверджують
прадавній КУЛЬТ і значення цього поняття:

ГАЙ — ліс, хаща, зарості, котрі завжди вважалися нашими предками шатами
Матері-Землі. Звідси вже інші слова, споріднені: ГАЙВОРОННЯ, ГАЙОВИК, ГАІР,
ГАЙСТЕР. А далі -

ГАЙНУТИ — повіятися, щезнути в хащах життя. Звідси і

ГАЙДА — іти по землі, ходімо, рушили вперед, хутко, нестримно.

ГАЙДАМАКИ, ГАЙДУКИ — вільні люди, що мандрують по землі, лісові, гайові люди.

Але ГАЯТИСЬ — це зупинятись, марнувати час, отже —

ГО, ГАЯ, ГЕЯ — це непорушність, сталість, бо земля завжди на вид спокійна,
недвижна, а

ГАЙ і ЙДЕ — це порушення спокою, рух вперед. ГОЙДАТИ — де теж дати рух, вивести
зі спокою. ГЕЙ, ГОЙ — вигук, що означає «вперед».

Цікаве також походження слова «гой», яким євреї називають не юдеїв. Безумовно,
це означає Син Землі, хлібороб. Тим більше, що євреї завжди вважали себе
прибульцями на Землі, отже всі інші племена були для них «гої» — земні.

Вернемося до визначення головних слів, які дадуть ключ до похідних, другорядних
понять.

Звідки взялося слово БОГ? Що воно означає? На це питання не можна відповідати,
притягуючи теологічні догми ортодоксальних, загальновідомих релігій. Поняття
БОГ у нашій мові було ще до введення візантійства, отже воно щось мало значити.
Наприклад,

БІГ-МЕ — маю бога. Так промовляють, коли хочуть ствердити, що слово їхнє чисте,
непорушне, святе. Отже, людина запевняє, що БОГ в НЬОМУ, можна не турбуватися.

СПАСИ БІГ — слово вдячності, означає: БОГ ТОБІ ВІДДЯЧИТЬ, ВРЯТУЄ.

НЕБІЖ, НЕБОГА — означають неповне родичання, віддаленість від якоїсь цільності,
єдності.

ЗБІЖЖЯ — тобто — від Бога.

Отже, у всіх цих словах можна прослідкувати, що БОГ (БІГ) — це предок, пращур,
котрий живе в нашому серці, дає нам силу, міць, мудрість, світло. Хто ж це
такий? Це — вогень, вогонь, огень, агні, що при переході В Б дало БО-ГЕНЬ, або
скорочено БОГ, БІГ (звідси і БІГ, БІГТИ — це нестримно рухатись, бути жвавим,
як вогонь).

Значить, це вогняний предок РА, ПРАЩУР. Він дає врожай (ЗБІЖЖЯ), він просвітлює
нашу душу (БІГМЕ, СПАСИ БІГ), він зникає з серця, коли з’являється РОЗБІЖНІСТЬ.

БОЖЕВІЛЬНИЙ — вільний від бога, отже людина з розірваним, необ’єднаним
мисленням.

БОГАТИЙ (багатий) — той, що має все необхідне, забезпечена людина. БАГАЧІ — так
звуться люди, котрі володіють зовнішнім добробутом, речами, грошима, їжею. Але
раніше БОГАТИЙ був той, хто мав БОГА в серці, тобто чиста, світла людина.
Недарма кажуть — має Бога в душі, або не має Бога в душі, без Бога в серці і т.
д.

Отже, БОГ, БІГ — це ВОГОНЬ, одвічний рух природи, таємнича прихована сила суті
Буття, котра дає життя, радість і смисл.

Звідси БІГ (швидкий рух), Бігучий, Біжучий, нестримний, вічноплинний. Тому
предки називали цим найменням ріки — БУГ, БОГ. Двіна — означає те саме, бо
походить від ДВИГАТИСЬ, рухатись, а двигатись — від ДЕВИ (бог, світло, рух).
Так само й інші ріки, в корені яких ДОН, ДАН, ДУН: всі вони від ДАНИЙ, ДАВАТИ,
а дає лише ДЕВА — тобто СВІТЛО, вічний рух. Отже, вода, як вічноплинна суть,
завжди пов’язувалася з таємничою силою природи, нестримним БІГОМ стихій.

Гляньмо також, як утворюються найрізноманітніші (на перший погляд) слова від
однозначних слів: ВЕЛИКИЙ — МАЛИЙ.

Від ВЕЛИКИЙ походять:

ВОЛЯ — тобто свобода, самодостатність, бо лише великий дух може бути вільний.

ВЕЛІННЯ, ВЕЛІТИ — наказ, що виходить від великої сили.

ВАЛИТИ — переважати. ВАЛ — потужний наступ стихій, чи людей. ВІЛ — теж означає
сильний, великий.

Від МАЛИЙ народжуються слова:

МОЛОТИТИ, МОЛОТИ — тобто МАЛИТИ, дрібнити. МЛІТИ, МЛЯВИЙ — втрачати силу.
МІЛКИЙ — неглибокий, мало глибини. МОЛОДИЙ — теж походить від «ще невеликий»,
незрілий.

Визначимо також такі прадавні поняття, як ніч, вечір, ранок, день, час,
вічність.

НІЧ — НИЧ — дотепер діалектне слово, яке означає «нічого»; тобто, темна пора
доби, коли нічого не роблять.

ВЕЧІР — ВЕЧУР — час, коли предок (ЧУР) велить спочивати, забороняє діяти.

РАНОК — схід сонця РА над світом.

ДЕНЬ (ДАНИЙ) — даний світлом для життя, творчості, праці.

ВІЧНІСТЬ — від слова ВІЧЕ (спільнота) — єдність природи і всього живого.

ЧАС — від слова ЧАСТКА. Отже — розірвана єдність. Те ж саме слово в мові інших
народів. Наприклад, ХРОНОС, КРОН поглинає дітей своїх, нищить єдиний потік
життя, котрий породжує РЕЯ, РАЯ, РІЯ. Проте, ЗЕВС (ЗОО-С), ЖИТТЯ повстає
супроти ХРОНА-ЧАСУ і змушує його повернути всіх знищених Дітей.

ВОРОГ — означає те саме (корінь ВР — воровать, враг, вредить).

БРАХМА — бог творення у індусів — теж має в корені БР.

БРЕХАТИ (БРАХ-МА). Слово «брама» в нашій мові лишилося як синонім слова
«ворота», що не пускають до мети. Недарма перекази Сходу оповідають, що
бог-творець (БРАГМА — ворог РАМИ) створив незаконний світ, поділивши частку
єдиного Всесвіту.

Він запустив недосконалу еволюцію, котра не має повноти вогненної сили РА. Ось
чому у нас на Землі таке безладдя, ненависть, кровопролиття, лавина злобних
життєвих форм і проявів. БРАГМА (БРАХМА) обманює (бреше) ВІШНУ (вишній,
найвищий) і ШИВУ (ЖИВА, суть Життя), тобто не дає цьому світові вернутися в
Єдине лоно Світла-РА. Але Вішну посилає своїх воїнів, синів РАМИ, котрі в
спілці з ШИВОЮ (Життям) руйнують незаконний світ, щоб знову було єдине Буття,
сповнене Радості.

Так усі перекази народів повторюють одне й те ж: Буття всеосяжне, воно повинне
бути радісним, щасливим. Те, що цього ще немає, свідчить про обман, котрий
таїться в самій глибині життя, в його сприйнятті, творенні. Знову всі ниті
ведуть до Людини, її розуму й серця, бо все горе і безладдя світу
трансформується в душі Мислячої Істоти.

Недарма ШИВА (ДЖИВА, ЖИВА — ось де з’явилося ім’я прадавньої української Матері
Світу ЖИВИ) зветься руйнатором, але разом з тим найвищим творцем, істинним
творцем. Він руйнує засмічене псевдожиття, псевдоеволюцію, псевдопоступ,
псевдорелігії псевдонауку, псевдотіла. Все це повинне бути просвітлене,
прояснене, переправлене вогнем ВІЧНОГО РА, щоб осягнути ПЕРВОЗДАННЕ,
НЕСОТВОРЕНЕ, РАДІСНЕ БУТТЯ.

З повним правом стверджуємо, що у мові народу закладено гнозис (знання, веду) і
смисл буття, призначення Людини, її суть. Саме тому темрява завжди намагається
зруйнувати мову, Слово, бо знає, що, втративши слово, народ перестає бути
єдиним організмом, духом, бо вже не має ведучої ниті РА, ВОГНЮ ЄДИНОГО ЖИТТЯ.

Але зберігати, відстоювати слід не просто мову, а кореневу Серцевинну МОВУ, а
отже — очистити її від навали чужоземщини. Треба переглянути необхідність
велетенської маси слів, котрими користується нині література, радіо, преса,
побут. Зайві слова, а тим більше сірі слова з невизначеним чітко значенням
розпорошують думку, не можуть її сконцентрувати на суті речення й поняття.

