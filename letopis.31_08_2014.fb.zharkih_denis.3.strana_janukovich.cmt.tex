% vim: keymap=russian-jcukenwin
%%beginhead 
 
%%file 31_08_2014.fb.zharkih_denis.3.strana_janukovich.cmt
%%parent 31_08_2014.fb.zharkih_denis.3.strana_janukovich
 
%%url 
 
%%author_id 
%%date 
 
%%tags 
%%title 
 
%%endhead 
\subsubsection{Коментарі}
\label{sec:31_08_2014.fb.zharkih_denis.3.strana_janukovich.cmt}

\begin{itemize} % {
\iusr{Джулия Кошкина}

Все к рыбаку с Женевского озера, но многое зависит от переговоров, посмотрим. На
народ рассчитывать не приходиться, в массах царит разброд и шатание...

\iusr{Хельга Хельга}
Только идиот не может видеть очевидное

\iusr{Джулия Кошкина}
Хельга, таких идиотов больше чем полстраны....

\iusr{Елена Рузина}
Очень точно, Денис. А то у многих такой сумбур в головах!!!

\iusr{Natalia Gorbokon}

Янукович всегда думал своей головой и при нём украинцы сами выбирали свой путь
((( молодец, Денис. Столько текста, а суть всё та же, мог бы одним предложением
ограничиться.

\iusr{Dimitry Kurdoyakov}

провинция - одно слово, независимость на словах только, а на деле не нужна она
никому - факт. Независимость для проекта нужна, как у коммунистов в 17 году. У
Новороссии подобный проект назревает - без олигархов, зато всем понятно, не
надо писать декрет о земле и мире:-)

\iusr{Тамара Журавлева}

Замечательная статья. С одним не могу согласиться. Государственные институции как
таковые на Украине давно отсутствовали, поэтому и рушить вообщем-то было
нечего. Точнее майдан разрушил те жалкие остатки институций, которые оставались
ещё при Януковиче. В целом за все годы независимости Украина выезжала на
институциях СССР, но с каждым годом неразумного, а то и преступного правления, не
только законодательной, но и исполнительной власти, эти институции тихо ,но
уверенно разрушались, при этом новые не создавались. И надо отметить, что
происходило это при молчаливом согласии каждого из нас.

\iusr{Dimitry Kurdoyakov}

Новая концепция нужна Украине, разваливается она неспроста.

\iusr{Oksana Pankova}

а) "плохое управление команды Януковича" - это эвфемизм для словосочетания
"полный беспредел", да?

б) по-хорошему, если институты власти плохи, то их нужно ломать и заменять, а
не улучшать, ибо доведя до совершенства плохое, на выходе получаешь очень
плохое. Но расскажите, пожалуйста, подробней, какие именно институты власти
были сломаны? Были изгнаны конкретные люди с конкретных стульев.

в) считать, что при Януковиче существовало какое-либо демократическое
устройство страны и с народом кто-то считался (о чём, конечно, свидетельствуют
разовые выбросы гречки избирателям), может либо очень наивный человек, либо
циник. Никто с народом не считался - народ использовали в виде инструмента
достижения своих целей, а именно - получения необходимого количества голосов
электората, чтобы продолжать сидеть у депутатской кормушки.

\iusr{Денис Жарких}

Оксана, да что сказать... Скачите дальше... этот текст все же требует некоторых
знаний об основах института государственности...

\iusr{Oksana Pankova}

Ваш ответ, Денис, в стиле "скачите дальше", говорит, во-первых, о Вашем высоком
уровне социального общения, а во-вторых, о том, что Вам нечего ответить по
существу. Конечно же, проще перейти на хамство. В таком случае, "пишите
дальше".

\iusr{Денис Жарких}

Спасибо большое, и Вам удачи  @igg{fbicon.smile} 

\iusr{Денис Жарких}

С меня пример не берем, я перед Оксаной Панковой извинился в личке. Коммент
Сергея Филоненко удаляю за мат не по существу.


\iusr{Олег Аксёнов}

Денис, ты в многом с Путиным согласился - он о " хрупкости демократических и
властных конструкций" в постсоветском пространстве в феврале- марте
высказывался.

\iusr{Олег Аксёнов}

Да, и вот из Путина, по памяти " - в России есть коррупция, но в Украине
Януковича она просто зашкаливала. "

\iusr{Olga Ivanova}

Всякий кто хочет устранить только то, что причиняет ему страдание, не
достаточно дальновиден, ибо благо не обязательно идет следом за злом; за ним
может последовать новое зло, и при том еще худшее, как это случилось с убийцами
Цезаря, которые ввергли республику в соль великие бедствия, что им пришлось
раскаиваться в своем вмешательстве в государственные дела.

(Мишель Эйкем де Монтель)

Столько бед от не знания истории.

\iusr{Э. Парэссэ}
\textbf{Denis Zharkih}, спасибо за ваш пост!

Я рассчитывал на более личностный ответ, но и такой экскурс в историю очень
интересен.

Позвольте мне уточнить, все ли я правильно понимаю.

1. Для противников Майдана важны: мир, стабильность банковской системы,
единство Украины и удовлетворение основных жизненных потребностей граждан
(минимальный комфорт и безопасность).

2. Противники Майдана поддержали бы протест против Януковича, если бы не
лозунги о евроинтеграции.

3. Выход жителей Юго-Востока с российскими флагами и призыв к России вмешаться
был зеркальным отражением действий Майдана по отношению к ЕС.

4. Обе стороны (и сторонники, и противники Майдана) проявили национальную
незрелость, обратившись за помощью к третьим лицам, а потому, как дети,
потеряли право что-либо решать самостоятельно.

5. Бабе Мане сложно понять, что к этому причастна Россия.  @igg{fbicon.smile} 

6. Такого беспорядка до Майдана не было, значит, вина за происходящее в Украине
лежит именно на нем.

7. Майдан оказался зациклен на свержении Януковича и ограничен в своих
интеллектуальных возможностях, т.к. он не смог сформулировать четкое видение
будущего и позволил себя обдурить радикалам, популистам и банкирам.

8. Майдан не представлял и не имел права представлять среднестатистического
украинца.

9. Действия Запада (чьи интересы представлял Киев) и России (чьи интересы
представлял Симферополь и, по аналогии, Донбасс) зеркально симметричны.

10. Существующие государственные институты власти хороши, только иногда в них
попадаются относительно плохие люди (пусть даже и подкупающие электорат, но, в
целом, заигрывающие с ним), но это не умаляет основополагающей ценности этих
институтов и органов, стоящих на их защите.

11. Майдан (особенно его западное крыло) начал разрушение институтов власти,
которые сегодня пытаются подменить своими бизнес-структурами олигархи.
Последние и ведут войны между собой на территории Украины.

12. Победа над группой олигархов может быть достигнута (в порядке убывания
вероятности) либо пророссийскими силами с неподвластным олигархии менталитетом,
либо одним из олигархов, либо украинцами.

Надеюсь, что я не исказил смысл сказанного вами. Если вы подтвердите
корректность вышеприведенных положений или уточните их, то я бы мог для
наглядности так же схематично отразить еще и точку зрения сторонников Майдана.
Думаю, такое сравнение многим было бы интересно.

\iusr{Михаил Сергеев}

прежде всего, Янукович, при всех его возможных недостатках, не был ни чьей
марионеткой. Был весьма неудобным для госдепа и ЕС, и достаточно лояльным
России. По сравнению с оппонентами, его команда не была популистской. Говорил
только конкретно и по делу. Никого из политических оппонентов не злословил. Уже
за то что не подписал соглашение об ассоциации в той кабальной для Украины
редакции достоин уважения. За это его и сместили.

\iusr{Денис Жарких}
\textbf{Э. Парэссэ} 

Спасибо за диалог. Некоторые пункты изложены неверно, но некоторые вполне
соответствуют сказанному. Я думаю, что правильнее пригласить Вас к нам в клуб
для личной встречи, чтобы Вы пришли со своими АДЕКВАТНЫМИ единомышленниками и
была бы интересная встреча. Согласитесь, что обсуждать все 12 пунктов, это
отдельный пост. А клубу было бы интересно это обсудить.

\iusr{Э. Парэссэ}

Денис, живая дискуссия - это замечательно, но только когда она ограничена
заранее и четко определенной темой. Я как раз и пытаюсь прояснить позиции
сторон, чтобы можно было дальше вести предметный разговор (коли у сторон
возникло такое желание). Вы дали ретроспективу событий, я попытался
последовательно вычленить из вашего текста ключевые моменты (достаточно
принципиальный пятый пункт, уж извините за неуместный юмор, сформулирован с
дословной точностью, но с обратным вкладываемому смыслом  @igg{fbicon.smile}  ). Если вы все же
подправите мои тезисы (вашего поста), то я постараюсь осветить эти же положения
с другой стороны. После этого можно было бы и обсудить результат (спасибо за
приглашение в клуб).

\iusr{Денис Жарких}
Андрей, неплохо.

\iusr{Денис Жарких}

\textbf{Э. Парэссэ} Даю коррективы. 1. Противников Майдана нельзя считать одним целым
типом. Первый пункт это для пенсионеров, которые, просто не хотят потрясений, а
спокойно пожить. На самом деле среди антимайдановцев (не нанятых, а идейных)
полно представителей бизнеса, людей, которые вполне могут и хотят сделать
карьеру и интеллектуалов. Они, как оказалось, были правы, что изгнание
Януковича не улучшит, а ухидшит ситуацию. Куда хуже - война, люди гибнут. Эти
люди НЕ ВИДЕЛИ В МАЙДАНЕ НИКАКИХ ПЕРСЕКТИВ. Поэтому, первый пункт некорректен.

\iusr{Денис Жарких}

Второй пункт. Среди Антимайдановцев не так уж много еврофобов. Поэтому,
евроромантика пугала немногих. Но, если говорить об интеллектуалах, то им до
лампочки был протест против Януковича. Им нужен был протест против системы,
чего Майдан не предложил, и не сделал. Второй пункт также некорректен.

\iusr{Денис Жарких}

3 и 4 пункты изложены правильно. 5 пункт - Россия постоянно действующий фактор,
но уровень ее влияния на Украину экспертами то занижается, то завышается.
Поэтому большинство предпочитает судить, как баба Маня.

\iusr{Денис Жарких}

6. В приципе, правильно. Майдан, вроде, как пришел навести порядок. А на деле
получился беспорядок. Баба Маня считает, что беспорядок навела Россия. А кто-то
все-таки считает, что беспорядок навел Майдан.


\iusr{Денис Жарких}

7. пункт - правильно, 8 - не представлял, но имел право представлять, если бы
заступался за интересы этого самого украинца. Реально Майдан представлял
интересы олигархов, которых давила семья Януковича.

\iusr{Денис Жарких}

9. В целом - да. И запад и Россия работали по беспределу. Западу было наплевать
на законную украинскую власть, России наплевать на территорриальную целостность
Украины.

\iusr{Денис Жарких}

10. Существующие государственные институты не хороши. Во- первых, их уже не
существует - сломаны. Во-вторых, их нужно было развивать, а не ломать.
Развивать так, чтобы нечестных людей они выталкивали, а честных и
профессиональных притягивали. А если сломали институты - сломали и государство.
Кто Вас ребята просил государство уничтожать, патриоты хреновы????! Новое
построить? А вы знаете как? Так какого хрена ломали?????!!!!

\iusr{Денис Жарких}

11. пункт верно, 12 пункт ничего не говорил о вероятности того или иного
сценария. Просто перечислил возможные.

\iusr{Э. Парэссэ}
\textbf{Денис Жарких}, 

я рассчитывал, что вы просто выправите неудовлетворительные фразы, а пришлось
это сделать мне (т.е. опять могут быть неточности). В связи с чем, желательна
еще одна итерация в согласовании позиций противников Майдана (пункты, которые
вы уже одобрили, я практически не изменял).

1. Противники Майдана неодинаковы. Одним при Януковиче нравились порядок,
стабильность банковской системы и возможность удовлетворения основных жизненных
потребностей (доступные цены, минимальный комфорт и безопасность). Другим,
идейным, нравились приемлемые условия для ведения бизнеса и карьерного роста.
Общее между ними: Майдан не перспективен, т.к. свержение Януковича может только
ухудшить ситуацию.

2. Противники Майдана стали бы его сторонниками, если бы у них спросили «чего
они хотят?», а не навязывали силой лозунги о евроинтеграции (критично для
одних) и не подменяли протест против системы протестом против Януковича (важно
для других).

3. Выход жителей Юго-Востока с российскими флагами и призыв к России вмешаться
был зеркальным отражением действий Майдана по отношению к ЕС.

4. Обе стороны (и сторонники, и противники Майдана) проявили национальную
незрелость, обратившись за помощью к третьим лицам, а потому, как дети,
потеряли право что-либо решать самостоятельно.

5. Противники Майдана признают Россию постоянно действующим фактором, уровень
влияния которого на Украину то занижается, то завышается экспертами.
Большинство антимайдановцев все же предпочитает заниженные оценки причастности
России как к организации внутреннего противостояния двух частей украинского
общества, так и к самому противостоянию.

6. Ни такого беспорядка, ни войны до Майдана не было, значит, вина за
происходящее в Украине лежит именно на нем.

7. Майдан оказался зациклен на свержении Януковича и ограничен в своих
интеллектуальных возможностях, т.к. он не смог сформулировать четкое видение
будущего и позволил себя обдурить радикалам, популистам и банкирам.

8. Майдан реально представлял интересы олигархов, которых давила семья
Януковича, а не среднестатистического украинца.

9. Действия Запада (чьи интересы представлял Киев) и России (чьи интересы
представляли Симферополь и Донбасс) зеркально симметричны. Обе стороны работали
по беспределу: Западу было наплевать на законную украинскую власть, России
наплевать на территориальную целостность Украины.

10. Все годы независимости в Украине разрушались государственные институты и к
моменту начала Майдана они практически уже отсутствовали. Однако это не умаляет
основополагающей ценности институтов власти и органов, стоящих на их защите. Их
нужно было развивать, а не ломать, чтобы нечестных людей они выталкивали, а
честных и профессиональных притягивали. Сломанные институты - сломанное
государство. Не зная как построить новое государство, Майдан не имел права
ломать старое.

11. Майдан (особенно его западное крыло) начал разрушение институтов власти,
которые сегодня пытаются подменить своими бизнес-структурами олигархи.
Последние и ведут войны между собой на территории Украины.

12. Победа над группой олигархов может быть достигнута либо пророссийскими
силами с неподвластным олигархии менталитетом, либо одним из олигархов, либо
украинцами.

Денис, а что вы можете добавить по поводу русского языка и притеснений
русскоговорящих, а также националистической/фашистской идеологии, которая
установилась на Майдане и возобладала в Центре и на Северо-Западе Украины после
него?

\end{itemize} % }
