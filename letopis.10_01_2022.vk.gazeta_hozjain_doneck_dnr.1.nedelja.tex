% vim: keymap=russian-jcukenwin
%%beginhead 
 
%%file 10_01_2022.vk.gazeta_hozjain_doneck_dnr.1.nedelja
%%parent 10_01_2022
 
%%url https://vk.com/hozyaindnr?w=wall-165246115_12124
 
%%author_id gazeta_hozjain_doneck_dnr
%%date 
 
%%tags 
%%title Отличного начала рабочей недели, друзья!
 
%%endhead 
\subsection{Отличного начала рабочей недели, друзья!}
\label{sec:10_01_2022.vk.gazeta_hozjain_doneck_dnr.1.nedelja}

\Purl{https://vk.com/hozyaindnr?w=wall-165246115_12124}
\ifcmt
 author_begin
   author_id gazeta_hozjain_doneck_dnr
 author_end
\fi

Отличного начала рабочей недели, друзья!

Сегодня на Руси отмечали Рождественский мясоед. Мясоед не был праздничным днём,
но с него начинался длительный период определённого образа жизни православного
крестьянина — времени, когда в пищу можно употреблять мясо. Кроме религиозных
мотивов, это было связано также с тем, что в первые дни января было принято
забивать скотину. Следовательно, на столе появлялось мясо — в первую очередь,
свинина. Однако это не отменяло еженедельный пост по средам и пятницам. В
народе говорили: «Мясное — в мясоед, но постному — свой черёд».

\ii{10_01_2022.vk.gazeta_hozjain_doneck_dnr.1.nedelja.pic.1}

В это время начиналась пора свадеб. Когда, как не в Мясоед, семьи молодых могли
накрыть богатый стол для гостей? А поскольку Рождественский мясоед был долгим
(он продолжался до Масленицы), времени для весёлых брачных обрядов и застолий
вполне хватало.

Мясоед называли также Домочадцевым днём. Вся семья собиралась в доме: сообща
выполняли работы по хозяйству, вместе садились за стол. Этот день напоминал о
том, что семейное согласие — самое дорогое, что может быть у человека. «Когда в
семье лад, то и дело спорится, и достаток родится», — говорили люди.

А накануне Домочадцева дня на Красную горку, которая была возле каждой деревни,
приносили колья и вбивали их в землю. Утром на горку приходили семьями: каждый
повязывал на кол лоскуток ткани и загадывал желание. Продолжались в Мясоед и
святочные гулянья. В домах собирались гости и играли в разные игры.

Существовали и приметы на погоду для этого дня. Считалось, что если в этот день
на скирдах ляжет иней, то лето будет дождливым.

Именины в этот день: Александр, Аркадий, Ефим, Игнатий, Леонид, Никанор,
Никодим, Николай, Пётр

\begin{verbatim}
	#Хозяин_ДНР #ХозяинРеспублика #ДНР #садоводство #зима #сезон #Хозяин
	#2022_январь #народныйкалендарь #именины #2к22
\end{verbatim}
