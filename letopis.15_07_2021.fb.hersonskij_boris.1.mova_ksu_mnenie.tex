% vim: keymap=russian-jcukenwin
%%beginhead 
 
%%file 15_07_2021.fb.hersonskij_boris.1.mova_ksu_mnenie
%%parent 15_07_2021
 
%%url https://www.facebook.com/borkhers/posts/4515135041854620
 
%%author Херсонский, Борис
%%author_id hersonskij_boris
%%author_url 
 
%%tags jazyk,konstitucia,ksu,mova,obschestvo,ukraina,ukrainizacia
%%title Мова - Решение Конституционного Суда - Мнение
 
%%endhead 
 
\subsection{Мова - Решение Конституционного Суда - Мнение}
\label{sec:15_07_2021.fb.hersonskij_boris.1.mova_ksu_mnenie}
 
\Purl{https://www.facebook.com/borkhers/posts/4515135041854620}
\ifcmt
 author_begin
   author_id hersonskij_boris
 author_end
\fi

Прочитав, що Конституційний Суд України визнав Мовний закон відповідним Конституції у тому вигляді, в якому він зараз існує.

Мої друзі добре знають мою позицію щодо зміцнення статусу Державної мови в
нашій країні. А мої вороги - ще краще. В 2014 році, у вересні я вперше сказав,
що відчуваю, як російськомовний поет, "лінгвістичний дискомфорт". Й обіцяв
поперед усього сам собі, що буду робити все можливе для того, щоб поліпшити й
удосконалити моє володіння українською. Обіцяв, що буду використовувати
державну мову у публичному просторі. І, хоч до 2014 року не думав, що почну
писати вірші українською, розпочав цю - повірте, друзі - зовсім нелегку працю.
І, як бачите, ту обітницю виконав, як міг.

Але вважаю мої успіхи обмеженими. Моя рідна мова залишається російською, а
українська - вивченою. І, хоч я й читаю лекції з психології не тільки
російською, але й державною, незважаючи на те, що спеціальна література
українською й досі обмежена, а недосконалість деяких перекладів психологічної
літератури українською абсолютно ясна навіть мені.

Все ж таки, я усвідомлюю, що рівень моїх лекцій і моїх поетичних текстів
недосконалий. Русізми, помилкові наголоси залишаються симптомами моєї "мовної
хвороби".

І ось що я скажу - я готовий відмовитися від викладання, незважаючи на великий
досвід моєї викладацької і чесної (ви розумієте, що я маю на увазі! бо купівля
й продаж піборобок під науку розповсюджена до крітичного рівня)  наукової
праці. Прийдіть юні дівчата. які слухали колискові українською, й починайте
робити те, що робив я - досконалою державною мовою. Якщо відчуваєте себе
спроможними це зробити.

Але я не буду святкувати цю велику перемогу над людським розумом. Ця війна з
розумом триває в нас багато років. Вважаю, що руйнування  в сфері гуманітарних
наук буде тривати. 

Мені сімдесят. Й може остаточне втілення мовного закону в життя то є дійсно
сигнал старому поетові й вченому залишити свою діяльність й піти на відпочинок.

Але мій вік дає мені право висловити думку відверто. Моя громадянська позиція
всім відома. Я заплатив за неї досить велику ціну й жодної миті не жалкував з
приводу того, що залишився громадянином України.

Зараз я вважаю рішення Конституційного Суду ні, не помилковим, а гірше -
політичним й дискримінаційним. Так воно є. Я вважаю, що ніякі адмінистративні
заходи неспроможні посилити любов до Державної мови.  Я досить часто писав про
це. Час - повторити.

Я усвідомлюю, що цей текст теж має багато похибок. Хай так воно буде. На завершення - мій вірш українською.

Молитва

\obeycr
Господи Боже наш, що відпустив блудниці,
промовивши - йди й надалі вже не гріши!
Відкрий Україні ласки Твоєї скарбниці,
В годину сум’яття самих нас не залиши.
Господи, що освятив у Кані весілля,
для молодят на вино воду перетворив,
не залиши нас в обіймах безглуздя, свавілля,
Аби власний наш розум нас не перехитрив.
Господи, що самаритянці біля криниці
дав напитись, щоб спраги вже не спізнала вона,
що дав силу леву та крила - небесній птиці,
змилуйся на Україною - вона в нас одна.
Одна-єдина, іншої бути не може.
З нею ми помираємо, щоб повстати в останній час.
Не забирай в нас Вітчизну, бо Ти - милосердний Боже,
Ти - Отець й сирітками Ти не покинеш нас.
Чи не молить Тебе за нас Твоя Милосердна Мати?
Чи не стоять за нас на небі наші святі?
Дай нам прапор Вітчизни міцно в руках тримати,
а не дивитися мовчки у долоні пусті!
Чи не молять Тебе за нас славетні гори Карпати?
Чи не молить Тебе за нас сивий старий Дніпро?
Ти землю нам дав, щоби на ній орати,
в хаті - добро, як то кажуть, і в полі - ядро.
Дай Боже цей день прожити у спокої й тиші.
Дай до тями прийти, дай Живої Твоєї води.
Нехай же за Україну молить Тебе все, що дише.
Це - наша молитва, а Воля - Твоя назавжди.
\restorecr

2019

\ii{15_07_2021.fb.hersonskij_boris.1.mova_ksu_mnenie.cmt}
