% vim: keymap=russian-jcukenwin
%%beginhead 
 
%%file 11_05_2021.fb.fb_group.story_kiev_ua.1.hochu_baobabov
%%parent 11_05_2021
 
%%url https://www.facebook.com/groups/story.kiev.ua/posts/1659659300897531
 
%%author_id fb_group.story_kiev_ua,ugrjumova_viktoria.kiev.pisatel
%%date 
 
%%tags baobab,kiev
%%title ХОЧУ БАОБАБОВ, или ОЗЕЛЕНЕНИЕ ПО-КИЕВСКИ
 
%%endhead 
 
\subsection{ХОЧУ БАОБАБОВ, или ОЗЕЛЕНЕНИЕ ПО-КИЕВСКИ}
\label{sec:11_05_2021.fb.fb_group.story_kiev_ua.1.hochu_baobabov}
 
\Purl{https://www.facebook.com/groups/story.kiev.ua/posts/1659659300897531}
\ifcmt
 author_begin
   author_id fb_group.story_kiev_ua,ugrjumova_viktoria.kiev.pisatel
 author_end
\fi

ХОЧУ БАОБАБОВ, или ОЗЕЛЕНЕНИЕ ПО-КИЕВСКИ

Не многие люди на свете могут пожаловаться, что пострадали от моей любви к ним.
Наверное, это происходит в основном от того, что людей в целом я люблю не
сильно. А если человек к тому же держит в руках вот эту заунывную мотокосу, то
я с особой силой начинаю не сильно его любить. Тут, помимо моего неприятного
характера, виноват здравый смысл. Вот если бы, скажем, люди с мотокосами носили
вместо этих мотокос мешки для мусора и этот самый мусор в них собирали, то даже
объединенными усилиями наших сограждан не удалось бы добиться той степени
загрязненности окружающей среды, которой они с успехом добиваются сейчас.
Конечно, нам говорят, что чисто не там, где убирают, а там, где не мусорят, но
убирать все-таки тоже надо иногда.

Но наши дворники мусор не любят. Они его настолько не любят, что принципиально
избегают каких-либо встреч с ним. Наш дворник держит метлу нарочито неумело и с
оттенком высокомерной брезгливости – дескать, леди не обязана знать, что это за
предмет такой – лопата. Или вот метла эта противная. Все несъедобные,
непривлекательные и бесполезные отбросы человеческой цивилизации они
рассматривают как личное оскорбление и оживляются только тогда, когда в поле их
зрения попадает абрикосовый, яблоневый, липовый, вишневый цвет и опавшие
листья. Причем опавшие листья они преследуют с меньшим энтузиазмом – потому что
листьев много как счастья и хватит на всех. А вот цвет норовит исчезнуть
бесследно. И многие дворники проявляют просто чудеса трудового героизма,
преследуя розовые вишневые лепестки, прежде чем те унесутся в синюю даль. А
потом, когда откружатся мимолетные душистые метели, на газонах начинает
пробиваться молодая упругая изумрудная трава. Одуванчики. Незабудки всякие.
Клевер, подорожник, спорыш и прочие элементы гербария. И вот тут появляются они
– люди с мотокосами. И начинают озеленять. В чем-чем, а в отсутствии интереса к
озеленению и благоустройству газонов их обвинить нельзя. 

После озеленения по-киевски газон выглядит страшным и лысым как голова
тифозника. Или вот помните еще – после второй мировой войны во Франции,
Голландии и прочих просвещенных европах было популярно брить женщин, - которые
оказывали как платные, так и бесплатные сексуальные услуги представителям
оккупационных властей – ножницами для овец. Чтобы клоками – пострашнее и
полысее и издалека видно. Вот это типичный киевский газон после
благоустройства. Он немедленно приобретает цвет выгоревшей соломы – впрочем, он
и есть выгоревшая солома: стерня. И в этом виде он мил и любезен взгляду и
сердцу дворника.

Казалось бы, слышишь, как день и ночь надрываются за окном мотокосы, вгрызаясь
в только что выросшую траву и эти вот одуванчики с незабудками – так радуйся.
Люди благоустраивают, озеленяют, облагораживают, не покладая рук – трезвые,
пьяные, полувменяемые, нервные, злые, задерганные – но косят на радость всем. А
то вот есть среди наших граждан отдельные экземпляры, я бы их назвала, вот как
есть, невзирая на лица – воинствующие оптимисты. 

Вот, смотрите. Вышел ты к газону, на котором что-то непредусмотренно пробилось
сквозь благоустройство, повздыхал о sic transit и что vita brevis – если ты
такой, блин, умный, увидел на горизонте озеленителя с косой – и иди, пей свою
валерьянку. Так нет же! Что-то копают, разрыхляют, втыкают, сажают втихаря. И
даже поливают, что уже вообще ни в дугу, ни в красную армию. Один гражданин
недавно такое учудил, не знаю, как это описать. У него, видимо, были пятерки по
урокам ручного труда в школе. И по ботанике, наверное, тоже успевал. Так вот
он, понимаете, натыкал по всему равномерно и прекрасно пустому газону каких-то
кустиков и соцветий, а вокруг – именно в этом вопросе я хочу быть правильно
понятой - флажки. По пять штук прекрасно склеенных и отлично заметных издалека
флажков вокруг каждого этого младенческого ростка. Ростки еще не видны, а по
периметру стена какая-то из этих самых флажков – смотреть противно. Хорошо, и
не пришлось. Дворник с моткосой, добрая душа, все скосил: и траву, и росточки
эти, и флажки. Так и неизвестно, что оно могло вырасти, какая бяка ядовитая
тропическая.

Я своим читателям давно уже надоела своим гудением об уничтожении киевских
парков, сквериков, бульварчиков, а также отдельных кустов, деревьев и прочих
представителей флоры, без которых уже совершенно самостоятельно и весьма быстро
исчезают представители фауны. 

- А что вы, собственно, хотите за… - ну и озвучивают какую-нибудь
душераздирающе крохотную сумму вознаграждения за душераздирающе тяжелый
дворников труд.

Нети, оно конечно безусловное да-да-да-да. Мизерная зарплата. Удручающие
условия. Но я что-то одного не понимаю. Их всех где-то пленили в прекрасных
странах, а потом закованными в цепи доставили в наш отдельный ад? То есть вот
эта женщина, самозабвенно уродующая кусты жасмина, она что - еще недавно
бродила среди развалин Парфенона и занималась наукой философией на благо
человечества? А вот этот немолодой человек был захвачен и продан в рабство
непосредственно у телескопа? А тот, с мотокосой, перевернул недавно мировую
математическую мысль, но злая судьба сделала его гребцом на галерах, после чего
его за хорошее поведение и сильную морскую болезнь перевели на сушу – уродовать
мир, в котором я живу? И на самом деле за всем этим проглядывают козни злобных
инопланетян-сатанистов, которые освоили передовое учение маркса-энгельса-ленина
и хотят разрушить до основанья, после чего уже не планируется никакое затем?

Думаю, все гораздо проще. И именно поэтому страшнее. Но великие законы физики,
которые не смогла отменить даже передовая социалистическая наука, гласят, что
на каждое действие имеется свое противодействие. Так я вот чего. Я хочу завести
баобабов. Много-много баобабов. Потому что из авторитетного литературного
источника нам известно, что баобабы колосятся каждое утро на том месте, где ты
их вчера выполол, и растут с такой скоростью, что волосы дыбом. Но это смотря
где. Если ты, скажем, живешь на крохотной аккуратно убранной, любимой тобой
планете вместе с любимой тобой розой, то баобабы тебе опасны. А если тебя
занесло на шарик, заселенный безумными газонокосильщиками, и Стивен Кинг даже
близко не постиг всю степень ужаса, который они вызывают у отдельных землян –
то баобабы – единственное спасение.

Я закрываю глаза, и перед моим внутренним взором встает следующая пасторальная
картина: утро, газон, скошенные флажки, вытоптанные незабудки, искореженная
вдребадан мотопила, офонаревший озеленитель, мы с Маленьким принцем, и над нами
– огромный, зеленый, могучий, неуничтожимый Баобаб, под густой сенью которого
так приятно дышать полной грудью, потому что есть чем дышать. И мечтать о
будущем, потому что, возможно, оно теперь тоже есть.
