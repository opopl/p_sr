% vim: keymap=russian-jcukenwin
%%beginhead 
 
%%file 12_04_2023.stz.news.ua.donbas24.1.mrpl_teatr_conception_vystava_ternopil.txt
%%parent 12_04_2023.stz.news.ua.donbas24.1.mrpl_teatr_conception_vystava_ternopil
 
%%url 
 
%%author_id 
%%date 
 
%%tags 
%%title 
 
%%endhead 

Маріупольський театр «Conception» показав у Тернополі свою виставу (ФОТО)

Колектив театру авторської п'єси «Conception» виступив на тернопільській сцені

7 квітня маріупольський театр авторської п'єси «Conception» на сцені
Тернопільського драматичного театру імені Т. Г. Шевченка представив свою
виставу «Життя переселенське». Цей документальний спектакль про труднощі,
небезпеку та непередбачені ситуації по дорозі до вільної від агресорів
території приємно вразив глядачів, які довго не хотіли відпускати маріупольців
зі сцени.

Читайте також: Маріупольські актори зіграли благодійну виставу в Києві (ФОТО)

Ідея поїхати з виставою у Тернопіль належить директорці Департаменту
культурно-громадського розвитку Маріупольсьскої міської ради Діані Тримі, яка
наголосила, що ця поїздка відбулася в рамках проєкту «Культурна деокупація»,
оскільки наразі відбувається формування загального інформаційного простору для
об'єднання маріупольців та однодумців. Був обраний саме спектакль «Життя
переселенське», адже він є зрозумілим багатьом українцям, які вимушені покидати
свої домівки на окупованих територіях.

«Ці емоції повною мірою неможливо передати ні картинами, ні словами на папері.
Я бачила транформацію людей, які сиділи в залі. До та після перегдяду "Життя
переселенське". Мистецтво під час війни відзеркалює події сьогодення та не дає
нам змоги забувати про важливі для нас самих речі», — наголосила Діана Трима.

Читайте також: У Києві відновили маріупольську виставу (ФОТО)

За словами акторів театру, це своєрідне продовження тематики повномасштабної
війни у творчості колективу.

«Найскладніше для кожного з наших акторів було знову занурюватися у ті події,
які всі ми пережили, коли виїжджали з Маріуполя під обстрілами. Знову і знову
повертатися у важке минуле — це дійсно морально важко. Але ми намагаємося
впоратися, бо в нас є дуже велика мета — продовжувати розповідати свою
історію», — зауважила актриса театру Марина Говорущенко.

Художній керівник і режисер театру авторської п'єси «Conception» Олексій Гнатюк
зауважив, що гастрольна діяльність важлива для колективу, оскільки це завжди
сприяє розвитку.

«Нас дуже тепло зустріли. Без особливих зусиль нам вдалось відтворити
декорації. Ми відчули, що глядачі співпереживали нашій історії. А після
спектаклю ми змогли ближче познайомитися з чудовим містом Тернопіль, погуляли
його вулицями і парками. Залишились дуже приємні спогади і позитивні емоції», —
поділився Олексій Гнатюк. 

Читайте також: Чи любив Марко Кропивницький Приазов'я: до дня створення
українського реалістичного театру

У глядацькій залі були присутні і маріупольці, і запрошені гості, які були
захоплені історією про важкі випробування під час переселення. Глядачі відчули
весь біль та невпевненість, які пережили маріупольські переселенці, але
водночас були вражені силою духу і мужністю, завдяки яким героям вдалося знайти
вихід з нелегких ситуацій.

«Ви неймовірні! Дякую всім учасникам театру! Така віддача! Пройшли по всім
болючи точкам... Наші історії однакові. Ми пережили, ми вижили, всі разом... ще
раз», — поділилась емоціями одна з глядачок.

Читайте також: «Театроманія» продовжує розвивати та підтримувати культуру
Маріуполя

Найближчим часом театр авторської п'єси «Conception» планує відновити виставу «Страхи в стилі ню» — психологічну драму на одну дію, яка не залишить нікого байдужим. Також планується підготувати документальний фільм, в основу якого ляжуть спогади режисера і акторів про пережите в Маріуполі та важкий виїзд з окупованого міста.

Раніше Донбас24 розповідав унікальні факти про театральну культуру Маріуполя.

Ще більше новин та найактуальніша інформація про Донецьку та Луганську області в нашому телеграм-каналі Донбас24.

Фото: Євгена Сосновського, Театру авторської п'єси «Conception» та з відкритих джерел
