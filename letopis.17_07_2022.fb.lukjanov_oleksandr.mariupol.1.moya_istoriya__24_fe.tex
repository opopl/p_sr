%%beginhead 
 
%%file 17_07_2022.fb.lukjanov_oleksandr.mariupol.1.moya_istoriya__24_fe
%%parent 17_07_2022
 
%%url https://www.facebook.com/permalink.php?story_fbid=pfbid05yJmTXik7V72uxHuEb4xSRWdaP4V1bEL6hxXGeKsWx6NgTspVvAJY8F1AZPh51KDl&id=100034577676194
 
%%author_id lukjanov_oleksandr.mariupol
%%date 17_07_2022
 
%%tags mariupol,24.02.2022,mariupol.war
%%title Моя история. 24 февраля
 
%%endhead 

\subsection{Моя история. 24 февраля}
\label{sec:17_07_2022.fb.lukjanov_oleksandr.mariupol.1.moya_istoriya__24_fe}

\Purl{https://www.facebook.com/permalink.php?story_fbid=pfbid05yJmTXik7V72uxHuEb4xSRWdaP4V1bEL6hxXGeKsWx6NgTspVvAJY8F1AZPh51KDl&id=100034577676194}
\ifcmt
 author_begin
   author_id lukjanov_oleksandr.mariupol
 author_end
\fi

Моя история. 24 февраля.

Холодно. Утро. Страшный грохот, взрывы.

Началось.

Ракета упала во двор.

Вылетели стекла в соседнем доме, машины ранены осколками и засыпаны землей и
камнями. Мелькнула мысль. Какого хрена. Шок.

Прячемся в квартире за стенами коридора подальше от окон.

Стихло.

Думаем, что это не может долго продолжаться. 

Тв новости твердят: 10 дней, не больше. Все верят.

Есть вода, свет, газ, интернет, тв, связь. Жить можно. Запас еды на 2 недели.

Переждем 10 дней.

В восточном районе города настоящий ад, постоянные обстрелы, люди бегут в
центр. Первые десятки погибших.

Четвертый день. Пропала связь, свет, газ, вода. Начинаем думать, как выжить.

У жены и дочери истерика, "папа, давай уедем". Но как? Обстрелы в нашем районе
совсем рядом. Прячемся в подвале дома, в темноте при свечах. Людей становится
все больше.

Холодно и сыро.

Там человек 25. Душно.

Дом трясет от снарядов. Попадают в магазины, соседние дома, люди под разрывами
грабят магазины с едой, лишь бы выжить. Снег, настоящая зима в марте минус
10.Трупы везде. Мертвых ложат в воронки под домами. Закапывать невозможно.
Раненым не повезло, помощи ждать неоткуда. Холодно. Ветер.

Сильный ветер. Природа тоже объявила войну.

Обстрелы не прекращаются ни днем, ни ночью. Сердце колотится от этого грохота.
Десять дней прошло. Все страшнее и страшнее. Что дальше? Неизвестность пугает
больше, чем реальность.

Объявили зеленый коридор до Запорожья, бегу в гараж за машиной( хоть бы
уцелела). Соседние гаражи завалило. Нам повезло. Берем самое необходимое и
морскую свинку Кити. До Запорожья 250 км. В голове одна мысль: СКОРО
ВЕРНЕМСЯ!!! Это не надолго.

На выезде из города обстреливают колонну "градами". Попали в передние машины.
Дым. Люди бегут обратно.  Опять трупы, машины горят. Паника. Разворачиваемся.
Хоть бы уцелеть! Связи нет! Едем к старшей дочери на квартиру это западный
район, у них пожар, нет окон и дверей. Переезжаем в одноэтажный район порта -
поселок моряков. Через два дня город окружили. Мы опять оказались на передовой.
Воронки от снарядов 3 метра, выворачивает деревья с корнями. Прячемся в тесном
подвале 15 человек стоя. Двое 3-х летних детей. Разрывы близко. У всех в глазах
СТРАХ.  Сил нет. Усталость. Страшная апатия.

Еду готовим на костре, пригенаясь, под свистом пуль и снарядов. Уличные бои. У
зятя военные забирают машину. Еды хватит на неделю, никто не ожидал, что все
так затянется.

Вечер, уже темно. Снаряд разорвался возле дома, слышно, как вылетели окна,
через минуту второй попадает в дом, в соседнюю комнату. Страшный грохот, темно,
треск, дым, пыль что-то валится. Кто то  кричит ГОРИМ! ВСЕ НА УЛИЦУ! Паника.
Толкатня.Выносим все, что можно в темноте. Крыша может рухнуть в любой момент.

Бегу к машине- на капоте огромная часть стены, пробую завести. Не получается. С
крыши падают горящие обломки, воды нет. С ужасом смотрю, как загорелись колеса.
Полная беспомощность...

Это конец.

Кричу: "Вынесли Кити?"

Ответ: "Нет"

Дочка рыдает в ужасе.

Документы остались в куртке на кровати. Секунды. Думать некогда.

Открываю разбитое окно и прыгаю в черную стену дыма. Бегу, что то схватил и
назад в окно. Господи, получилось. Долго кашляю не хватило вдоха. Одну побитую
машину удалось откатить от горящего дома. Огромное пламя, черный дым, ветер.
Мороз пронизывает насквозь.

Все молча смотрим, как догорает дом.

Оглушительная тишина. Понимаю - слава богу живы! Сгорело все - еда, вещи,
одеяла. Что теперь?

Дети не плачут. 

Соседи дали комнату( температура +5). Никто не говорит, как-будто ничего не
произошло. В глазах ужас и полное опустошение. Все молчат.

Утром люди принесли крупу, макароны, какую то одежду. Все сочувствуют. В
соседнем доме на воротах мелом написано: "В доме мертвая 8-ми летняя Даша,
дедушка и бабушка". В другом погибли мать и сын. И так весь район через дом.
Или сгорели, или разрушены. На улице появились трупы местных жителей и военных.
Из города под обстрелами потянулась бесконечная вереница машин и людей. Старики
и дети пешком, как муравьи бегут от войны. Гуманитарный конвой в город не
пускают. В Мариуполе мародерство и голод, воды нет. Оставаться невозможно.
Решаем за два раза переехать в Бердянск и дальше на Запорожье. НАДЕЕМСЯ
ВЕРНУТЬСЯ, когда закончится война. 100 км до Бердянска добрались за сутки.
Дальше колонной автобусов. Бесконечные унизительные проверки ряженых
"освободителей". За двое суток добрались до Запорожья. 

Заметил - ковидом никто не болеет. Война - лучшее лекарство. Появилась связь.
Увидели фото нашего дома и города, вернее все, что осталось. УЖАС! Поняли -
возвращаться уже НЕКУДА. НЕТ НИЧЕГО.

Жизнь замедлилась до предела. Хочется просто сидеть в тишине и смотреть на
море, оно осталось прежним. А мы нет. Илон Маск мечтает о начале новой супер
жизни на МАРСЕ, а мы теперь думаем о простом существовании на ЗЕМЛЕ. ЖИЗНЬ
ОБНУЛИЛАСЬ. Мариуполь март 2022.
