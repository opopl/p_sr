% vim: keymap=russian-jcukenwin
%%beginhead 
 
%%file topics.vojna.my.7.matrica.nashe.novosti
%%parent topics.vojna.my.7.matrica.nashe
 
%%url 
 
%%author_id 
%%date 
 
%%tags 
%%title 
 
%%endhead 

\paragraph{15:51:17 15-08-22 НЕ КАЦАП И НЕ МОСКАЛЬ}

Украинские летчики уже «ходят по головам» россиян.

За последние несколько недель украинцев не переставали радовать приятными
новостями с фронта. Подрывы складов, командных пунктов и, на конец, аэродрома в
Крыму стали главными событиями первой половины августа. Немалую роль в этих
«хлопках» сыграли украинские летчики. И их работа может стать еще лучше, когда
Украина получит самолеты от Словакии, способные нести на борту ракеты для ЗРК
NASAMS.

Как изменилось поведение украинской авиации за время войны? Почему выхода на
границы 1991 года будет недостаточно для завершения войны? Об этом «Апостроф»
рассказал полковник запаса ВСУ, военный летчик-инструктор, военный советник
главы Донецкой области в 2014 году Роман Свитан.

– Говорят, что в последнее время украинские летчики начали дерзко вести себя.
Отмечается, что теперь они не только парами выполняют задачи. Что этому
поспособствовало?

– Летают, действительно, не только парами, а и звеньями. Пара – это два
самолета, звено – это четыре самолета, эскадрилья – это 12 самолетов. Пока
летали звеньями по 8 самолетов. Работают в основном на правом берегу в районе
Херсона и Херсонской области. Прореживаем ту группировку, которая в последнее
время перешла на правый берег через мосты, пока они еще были целыми. Работают
довольно-таки неплохо.

Последние полторы-две недели, после того, как зашли хорошие противолокационные
ракеты HARM для МиГ-29, то действительно российское ПВО в том месте было
подавлено. Наши самолеты уже вольготно чувствуют себя. Как говорят летчики,
«ходят по головам» той пехоты, тех вдвшников российских, которые сейчас
находятся на правом берегу.

– Насколько успешной является работа украинской авиации? Результатом семи
ударов в Херсонской области стало пять попаданий в пять мест сосредоточения
живой силы.

– Это на хорошо выполненная задача, это действительно хорошая работа наших
авиаторов. На старых машинах Су-25, хотя и надежных, они все-таки работают по
земле довольно неплохо. Прикрывают их истребители МиГ-29 и Су-27. Летчики
научены, и за восемь лет мастерство было поднято до максимума для этих
самолетов.

– С 24 февраля оккупанты часто отчитывались об уничтожении украинской авиации.
Но мы видим, что украинские летчики дают прикурить оккупантам. При этом, мы
ожидаем еще каких-то самолетов?

– Да, нам Словакия должна передать 11 самолетов МиГ-29. Чем они хороши? Это
хоть и старые советские самолеты, но у них новая авионика, которая позволяет
работать с натовскими ракетами воздушного базирования. И эти ракеты, которые
придут вместе с этими самолетами, нам помогут завоевать превосходство не только
на земле, а еще и в воздухе, а также прикрыть Украину от большинства ракет
российской авиации. Эти самолеты могут нести ракету AIM-120. Это базовая
ракета, которая используется и в зенитном комплексе NASAMS, которые также к нам
будут заходить.

Пока еще не говорим о поставках самолетов F-16 и F-15. Но, я думаю, и они у нас
точно будут..(H)
