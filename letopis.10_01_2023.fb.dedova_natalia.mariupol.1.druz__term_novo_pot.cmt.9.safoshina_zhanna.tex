% vim: keymap=russian-jcukenwin
%%beginhead 
 
%%file 10_01_2023.fb.dedova_natalia.mariupol.1.druz__term_novo_pot.cmt.9.safoshina_zhanna
%%parent 10_01_2023.fb.dedova_natalia.mariupol.1.druz__term_novo_pot.cmt
 
%%url 
 
%%author_id 
%%date 
 
%%tags 
%%title 
 
%%endhead 

\clearpage
\paragraph{9 - Жанна Сафошина}

\raggedcolumns
\begin{multicols}{2} % {
\setlength{\parindent}{0pt}

\begin{itemize} % {
\iusr{Hanna Krushynska}

Жанна Сафошина, погибла 9 марта со своей невесткой Викторией, от прямого
попадания в дом, осиротели 3 детей, и внук. Жанна была моей соседкой, частный
сектор, рядом с Приморьем. Накануне войны ездила с детьми на отдых в Карпаты,
вернулись в Мариуполь 25 февраля. Работала в порту.

\ifcmt
  igc https://scontent-ams4-1.xx.fbcdn.net/v/t39.30808-6/323139069_3478477732382439_4526322327744669314_n.jpg?_nc_cat=107&ccb=1-7&_nc_sid=dbeb18&_nc_ohc=hXcysuznlykAX-ehMhQ&_nc_ht=scontent-ams4-1.xx&oh=00_AfBeoF9KofXQDrulIy6Z_cY4WZ8eo9D7MAlv9DjGRO-5Pg&oe=63D0B8FB
\fi

\begin{itemize} % {
\iusr{Helen Verniievskaya}
\textbf{Hanna Krushynska} мы вернулись с Карпат в Мариуполь 25 февраля

\iusr{Hanna Krushynska}
\textbf{Helen Verniievskaya} 

я поправила. Запомнила, потому что видела пост, и подумала, что наоборот,
успели уехать, когда были поезда 24 и 25 февраля. Ошиблась... Жанна очень
позитивным, тёплым была человеком, я помню тот страшный обстрел 9 марта, мы
рядом жили, на улице Федорова, на углу, рядом с Пионерской.

\iusr{Helen Verniievskaya}
\textbf{Hanna Krushynska} 

Жанны не было с нами. С нами ездили ее дочки - Ира и Полина и они спешили домой
к маме! Так страшно об этом думать! Я не представляю как они это пережили!

\iusr{Hanna Krushynska}
\textbf{Helen Verniievskaya} а, поняла... это все сплошной кошмар...
\end{itemize} % }

\iusr{Полина Полина}

\href{https://www.facebook.com/profile.php?id=100030010052486}{Ирина Ушань}

\ifcmt
  igc https://i.paste.pics/ec4ba405e75d250306b34750e9ed626f.png
\fi

\ifcmt
  igc https://scontent-ams2-1.xx.fbcdn.net/v/t39.30808-6/323278749_484239520324063_7956476776045093085_n.jpg?_nc_cat=104&ccb=1-7&_nc_sid=dbeb18&_nc_ohc=iOZ5m5GPELsAX-BOmFT&_nc_ht=scontent-ams2-1.xx&oh=00_AfDLI1tiCmGVj1FZONUDsclRHhqApdIQ7ToaiTv4ydFZvQ&oe=63D08A14
\fi

\iusr{Hanna Krushynska}

Лейла Икаева, погибла от обстрела. Фото нет, к сожалению.

\begin{itemize} % {
\iusr{Алина Жук}
\textbf{Hanna Krushynska} а сколько ей лет было ? Эта не та что со зрением проблемы были ?

\iusr{Hanna Krushynska}
\textbf{Алина Жук} не помню точно, лет 45

\iusr{Hanna Krushynska}

Я её лет 20 не видела, что со зрением не знаю. У неё вроде двое детей было,
старшего сына Томаса знаю. Мама Людмила, отец Томас, осетин

\iusr{Алина Жук}
\textbf{Hanna Krushynska} а не не она, я поняла

\end{itemize} % }

\iusr{Stefaniya Kostiuk}

Боже, якій жах читати коментарі

\iusr{Maria Kutnyakova}

Люся Тюрина та її хлопець Андрій Гречин. Загинули в себе в квартирі по пр Мира
від авіабомби. Вона фахівчиня туристичної галузі, раніше працювала у Вежі, а
він комік та ведучий

\ifcmt
  igc https://scontent-ams2-1.xx.fbcdn.net/v/t39.30808-6/324442975_566803838303305_2019024588533648309_n.jpg?_nc_cat=100&ccb=1-7&_nc_sid=dbeb18&_nc_ohc=-Zf421xYIPEAX8ZItlU&_nc_oc=AQlbhE2IFKo8LVFU5yoRYY8HTXra_UBbCTyHuMA1mS7TlK7mU8bvbmzmMa0-0LdeWlo&_nc_ht=scontent-ams2-1.xx&oh=00_AfC1WkwMkwzp3c1Pe8mDlajBEx1KrqQGeK-2oCDLfckukQ&oe=63D0EDEF
\fi

\begin{itemize} % {
\iusr{Christina Zheleznyak}
\textbf{Maria Kutnyakova} 😭😭😭
\end{itemize} % }

\iusr{Юлія Кальницька}

Івлеви Олександр та Вікторія, загинули під час обстрілу
Назаренко Вадим, загинув під час обстрілу
Анісімов Олександр, пішов, почався обстріл, пропав без вісті
Дудченко Валерій, загинув під час обстрілу
Козлові Ігор з матір'ю загинули під час обстрілу

\begin{itemize} % {
\iusr{Ольга Красовская}
\textbf{Юлія Кальницька}, виправлю вас. Козлов Ігор і жінка Олена ( мати Ігоря, слава Богу, жива).

\iusr{Юлія Кальницька}
\textbf{Ольга Красовская} ми напевно про різних говоримо. В цього не було жінки, на скільки я знаю

\iusr{Ольга Красовская}

Вони не були одружені. Були разом с сином у підвалі, коли знаряд потрапив
туди. Їх врятували не вдалося, а син ( Олексій) живий.

\iusr{Юлія Кальницька}
\textbf{Ольга Красовская} Лівий берег? Мені все таки здається, що ми про різних людей кажемо. Працював на Азовсталі

\iusr{Ольга Красовская}
\textbf{Юлія Кальницька}, так. Моя мати Дученко Тетяна, з її слів.

\iusr{Юлія Кальницька}
\textbf{Ольга Красовская} о, ну тоді добре
\end{itemize} % }

\iusr{Nataliia Formaniuk}

Форманюк Павел Вадимович. 22.12.1989-6.03.2022

Попал под обстрел на роднике, возле мечети.

\ifcmt
  igc https://scontent-ams4-1.xx.fbcdn.net/v/t39.30808-6/323633974_978940866843867_615054568788684530_n.jpg?_nc_cat=103&ccb=1-7&_nc_sid=dbeb18&_nc_ohc=BPOPj5UWBNUAX_bfvCe&_nc_oc=AQkJ07S5F8TsgqIwbD6teGu5JS3BecDWMxRVonlwhvOa4C8CFL3fGITpEgjKV7PGsDo&_nc_ht=scontent-ams4-1.xx&oh=00_AfDRi5eXkr2V3eayxwEm9cx2mtDSpy3GFQFtPbP6iPHYPg&oe=63D0266F
\fi

\begin{itemize} % {
\iusr{Julia Garkusha}
\textbf{Nataliia Formaniuk} ооо в той день на джерелах біля мечеті загинуло чимало... Павло один з загиблих (( тепер щнаємо

\iusr{Olga Yalo}
\textbf{Julia Garkusha} возле мечети? в тот день много погибло на противоположных родниках - от ул.Кронштадской вниз в балку, там страшный обстрел был 06.03

\iusr{Julia Garkusha}
\textbf{Olga Yalo} 

мій друг пішов спочатку у бік кроншика, але почув звуки обстрілу, розвернувся і
пішов в бік мечеті, при ньому жінку та чоловіка вбило, друг встиг відкотитися,
кинув ту воду побіг до дому... Мінометний обстріл. Так можливо, більше там,
аніж там.... коротше кажучи, весь береень як суцільний пиздець...

\iusr{Natalya Dedova}
\textbf{Julia Garkusha} ми писали інтерв'ю з кумою Павла, вона розповідала цю жахливу історію. 🖤

\iusr{Nataliia Formaniuk}
\textbf{Наталя Дєдова} возможно есть свидетели, кто видел с какой стороны летели снаряды?

\iusr{Natalya Dedova}
\textbf{Nataliia Formaniuk} сложно сказать.....
\end{itemize} % }

\iusr{Mary Rjevskaia}

И после этого, они пишут, что в Мариуполе жизнь наладилась- возвращаетесь .....😢

\begin{itemize} % {
\iusr{Lev Volk}
\textbf{Mary Rjevskaia} мерзоти

\iusr{Lev Volk}
Я бачив як люди лежали на землі, розумів що багато погибло, але коли дивишся фото то це тяжко

\iusr{Евгений Котлубей}
\textbf{Mary Rjevskaia} оставшихся-налаживателей ненавижу как и орков
\end{itemize} % }

\iusr{Ludmila Karpova}

Надежда Петровна Крамар завуч школы.

\begin{itemize} % {
\iusr{Ирина Нагорная}
\textbf{Ludmila Karpova} а какой номер школы? А фото есть? По-моему это моя первая учительница... я училась ОШ64 1-Е класс (1985)

\iusr{Ludmila Karpova}
\textbf{Ирина Нагорная} да я не могу фото скинуть. Ее фото есть на моей странице в друзьях.

\iusr{Ludmila Karpova}
Она была учитель начальных классов.

\iusr{Ирина Нагорная}
\textbf{Ludmila Karpova} к сожалению не нашла... Царствие небесное..

\iusr{Ludmila Karpova}
\textbf{Ирина Нагорная} да царствие небесное. Она с сыном была в чужом доме. Не успели выйти был перелет. Ее придавила плиткой. Сын выжил

\iusr{Ludmila Karpova}
В 64 она работала
\end{itemize} % }

\end{itemize} % }

\end{multicols} % }
