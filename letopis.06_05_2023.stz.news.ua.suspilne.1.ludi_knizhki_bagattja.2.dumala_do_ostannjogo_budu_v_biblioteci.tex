% vim: keymap=russian-jcukenwin
%%beginhead 
 
%%file 06_05_2023.stz.news.ua.suspilne.1.ludi_knizhki_bagattja.2.dumala_do_ostannjogo_budu_v_biblioteci
%%parent 06_05_2023.stz.news.ua.suspilne.1.ludi_knizhki_bagattja
 
%%url 
 
%%author_id 
%%date 
 
%%tags 
%%title 
 
%%endhead 

\subsubsection{\enquote{Думала, що я буду до останнього в бібліотеці}}

Будівля бібліотеки була така на той час, що я вважала, що вона вистоїть. Вона
вистояла 1941-1945 роки й окупацію Маріуполя. Там були такі дуже гарні
підвальні приміщення, броньовані двері, броньовані стіни. І ми пішли додому,
взяли щось там поїсти, переодягнутися й прийшли на роботу, я з донькою, вона з
донькою й сторож, який прийшов в нічну зміну.

Так ми і жили десь тиждень, поки ще був зв'язок. Ходили додому, дивились свої
квартири. Я сказала нікому не виходити на роботу. Час так швидко минав, що ми
нічого не розуміли, ми тільки бачили: ось води вже немає, ось черги величезні в
магазинах. В магазинах полки змітали всі, під банкоматами стояли такі черги, що
просто неймовірно. 

\ii{06_05_2023.stz.news.ua.suspilne.1.ludi_knizhki_bagattja.pic.3}

А десь 5-го березня до нас постукали — поліція. Сказали, дівчата, ми будемо
сюди переїжджати, бо тут у вас більш менш безпечне приміщення, бо там де вони
знаходились не було підвальних приміщень. А вони виїжджали на виклики, ще якось
контролювали ситуацію.

Воєнний стан — я нічого не змогла, хоча було страшнувато. Все ж таки ми —
цивільні, а тут — військові. Але нікуди дітись, вони переїхали. На ранок
привезли свої сім'ї, ми розташувалися усі у підвальних приміщеннях, де були у
нас книгосховища. Там, де не було вікон, ми всі знімали книжки, опускали
стелажі, складали туди книжки, якось там розташовувались. Десь 50 нас чоловік в
цих приміщеннях було.

Виходити з підвалів вже було дуже-дуже страшно. 7 березня вже було так гучно й
в центрі, що до нас прибігли рано вранці, десь о сьомій, люди, які жили біля
будівлі бібліотеки по проспекту Мира, 45-43. Сказали, що в їхні підвали не
вміщаються — люди біжать з інших районів, заскакують в підвал, де тільки можна.
І вже дехто не пускає до себе. Тому питали, чи можна у нас розташовуватися.

Я відкрила все, що в мене було, всі приміщення. Десь ще ховала книжки, якусь
техніку зберігала, занесла до себе в кабінет й пускала туди людей. Так ми і
жили усі разом десь два тижні. Ну як жили, пережидали так скажемо.

Зв'язку не було вже десь після 4-5-го березня. Але у хлопців були рації, й
завдяки цьому ми ще якось знали, де що робиться у місті. Знали, де бомбануло й
чий будинок вже розвалився. Коли вони виїжджали й повертались ввечері, в них не
було лиця, бо вони збирали ще поранених людей, вони збирали трупи, ще й звозили
там по госпіталях. Вони бачили весь цей жах. Ми — не бачили, бо не могли вийти
з підвалів.

\ii{06_05_2023.stz.news.ua.suspilne.1.ludi_knizhki_bagattja.pic.4}

Десь, щось ще працювало, але чим далі, тим було складніше й складніше. Ми вже
чули оці літаки, ці бомбардування. Все літало над нашими головами. Ночі ці були
безсонні.

\begin{qqquote}
От воно летить й летить, летить й летить, а ти лежиш укутаний, бо мороз -12, а
воно бабахкає, а воно стріляє — і лежиш думаєш: \enquote{Господи, яка ж тебе мати
народила}. Оце я точно пам'ятаю, як я про це все думала. \enquote{Яка ж тебе мати
народила?!} — оце питання в мене досі. Яких там матері народжують?	
\end{qqquote}

Потім люди почали дізнаватись, що можна виїжджати (бо офіційних коридорів не
було). 14 березня прибігла одна з жінок й каже: можна виїжджати на узбережжя,
через Мангуш, через Портовське. У них була маленька дитина, й її чоловік каже:
ну, що мені треба рятувати сім'ю. Поїхав — і не повернувся, ми не знали, що з
ним. Лише 15 рано вранці він повернувся й сказав, що так дійсно можна виїхати
на Мелекіне, на Ялту, на Урзуф. Каже, що покаже дорогу тим, у кого були машини.
У деяких людей вони у нас у дворі стояли. Ми у дворі бібліотеки мали гаражі, й
ховали там машини, зокрема, і поліцейські. Тоді хто міг, той посідав на машини,
і поїхав. І вже з 15 березня нас все менше й менше ставало.

Бібліотека була буквально за 500 метрів від драмтеатру. І дівчата пішли
вигулювати собачку й побачили, що біля драмтеатру була церква й там прямо під
куполом люди ловлять зв'язок. Вони туди залізли, спіймали зв'язок, попередили
наших чоловіків, що ми живі, де ми знаходимось. Але в той момент у мене не було
думки, що я буду кудись виїжджати. Я думала, що я буду до останнього в
бібліотеці, що тут надійно, що тут все добре, двері броньовані.

Я підіймалася на 2-3 поверх, там вже не було ні вікон, ні дверей. Але в нашому
бункері було надійно. І тут дівчина прибігла і каже: \enquote{Сказав ваш негайно
збиратись, їхати. Сьогодні буде машина}.

Збиратись так збиратись, я розумію, що треба рятувати доньку. Тим паче, що ми
вже знали, що \enquote{ДеНеРи} вже недалеко від центру знаходяться, і мікрорайоні
Кірова, недалеко від мого будинку, а це вже і до нас близько. Але чекали-чекали
ми машина, а так вона й не приїхала. А вранці в мене раптом думка, що я повинна
піти додому і хоч якісь речі з собою взяти. І ось, коли тільки-тільки почався
світанок, ми з дочкою пішли додому.

\begin{qqquote}
А я 10 діб не виходила з бібліотеки. І коли ми вийшли, я побачила весь жах
оцього Маріуполя. Це був не Маріуполь, це було мертве місто. Там валялось все,
що можна, від дротів до будинків.	
\end{qqquote}

\ii{06_05_2023.stz.news.ua.suspilne.1.ludi_knizhki_bagattja.pic.5}

Все свистіло, летіло, але я дійшла до дому. Знаєте, якась така думка була:
потрібно попрощатися. Я піднялась до квартири, закрила вибиті вікна, рами
позакривала, бо в нас там вже багато чого повилітало. Взяли з дитиною якісь
речі, й так під обстрілами повернулись на роботу до бібліотеки. Й буквально за
пів години приїхала машина. Коли я виїжджала з бібліотеки, там залишалось
всього восьмеро людей.

