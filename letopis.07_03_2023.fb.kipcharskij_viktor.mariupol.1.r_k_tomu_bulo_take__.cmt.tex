% vim: keymap=russian-jcukenwin
%%beginhead 
 
%%file 07_03_2023.fb.kipcharskij_viktor.mariupol.1.r_k_tomu_bulo_take__.cmt
%%parent 07_03_2023.fb.kipcharskij_viktor.mariupol.1.r_k_tomu_bulo_take__
 
%%url 
 
%%author_id 
%%date 
 
%%tags 
%%title 
 
%%endhead 

\qqSecCmt

\iusr{Марина Солошенко}

Твій щоденник треба опублікувати в росії, ми в цей час збирались виїжджати з
Харкова, але морози нас зупинили! Дивитись новини про Маріуполь було важко,
думки були з вами! Над нашим домом літали ракети, школи онука вже не було!

\begin{itemize} % {
\iusr{Віктор Кіпчарський}
\textbf{Марина Солошенко} Твої спогади не менш важливі: запиши їх, наговори на диктофон - сбережи.
\end{itemize} % }

\iusr{Светлана Водзянская-Живогляд}

Для мене день почався дуже рано, я навіть не зрозуміла чи спала бо всю ніч я
прокидалася від того що знову і знову переступала мертвих людей, чоловік казав,
що стогнала всю ніч. Ранок почався \enquote{стандартно} від вибухів такої сили що 3
літрові банки з закрутками підстрибнули см на 20, ми разом з ними і в страшній
тиші я чула як обсипалася штукатурка стелі на підвісну стелю в будинку бо
кришка льоху закрита на \enquote{викрутку} ( підставили щоб легше було відкривати). На
мить здалося що зарад будинок не витримає. Витримав! Було страшно вилазити,
почекали 15 хвилин- тиша, повна. Потроху вилізла- будинок цілий, взривною
хвилею винесло двері які ми свого часу зробили в надії того року вигнати
прибудову. Вискочила на вулицю- горить 9 поверхівки поруч, людей біля криниці
нікого. Кинулися на дерево( старезна жерделя на якій чоловік побудував дітям
будинок на дереві) подивилася на будинок хрещених- диму нема, дерева наче не
посічені, вискочили дивитись чи цілі куми. Всі ціли!! Виявилося що у 9
поверхівки скинули кілька авіа бомб, наче мало було що тиждень до того там були
прильоти точок \enquote{У} і загинули підлітки. Далі ми цілий день не знали що робити
бо вийти з дому лячно, а дивитись як горить будинок все навкруги затягує
кіптявою. Вночі подув штормовий вітер роздмухуючи вогонь і вбиваючі надію, що
він сам може затихне..

Питання про обігрів муляло мене весь день, газу більше нема, світла вже
здається цілу вічність і діти починають шморгати носами. Цілий день мучилась
цим питанням, чоловік до ночі просто оскаженів від моїх ідей. І тут я зрозуміла
користь романів про кохання історичних!! Здається у Дюма героїня їхала у кареті
поклавши ноги на гаряче каміння!! Їжу готували на кострі, а не на мангалі, а
склали з цегли яку при купівлі будинку витягли з розваленої печі (шамотна
здається) Спробували і о диво!!! Стало тепліше аж на 5 градусів. Втомлена
намагалася вмовити 2 річного сина вмовити зняти взуття та комбінезон, він
кричав щосили \enquote{не}!!! Годину вмовляли... домовились зняти і сховати під подушку.
Малий з тієї ночі часто ховає свої речі і тут в Іспанії у любій стресовій
ситуації хапається за свої речі...
