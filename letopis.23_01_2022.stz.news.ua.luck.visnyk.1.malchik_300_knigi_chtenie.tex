% vim: keymap=russian-jcukenwin
%%beginhead 
 
%%file 23_01_2022.stz.news.ua.luck.visnyk.1.malchik_300_knigi_chtenie
%%parent 23_01_2022
 
%%url https://visnyk.lutsk.ua/news/ukraine/regions/volyn/68558-10-richniy-khlopchik-z-lutska-prochitav-300-knizhok/
 
%%author_id news.ua.luck.visnyk
%%date 
 
%%tags chtenie,deti,kniga,kultura,literatura,luck,ukraina
%%title 10-річний хлопчик з Луцька прочитав 300 книжок
 
%%endhead 
 
\subsection{10-річний хлопчик з Луцька прочитав 300 книжок}
\label{sec:23_01_2022.stz.news.ua.luck.visnyk.1.malchik_300_knigi_chtenie}
 
\Purl{https://visnyk.lutsk.ua/news/ukraine/regions/volyn/68558-10-richniy-khlopchik-z-lutska-prochitav-300-knizhok/}
\ifcmt
 author_begin
   author_id news.ua.luck.visnyk
 author_end
\fi

10-річний лучанин Володя Каліш обожнює читати книжки. 

А ще хлопчик разом із мамою веде відеоблог на YouTube, у якому ділиться
враженнями від прочитаного та знайомить підписників з літературними новинками,
пише ІА \enquote{Волинські Новини}. 

\ii{23_01_2022.stz.news.ua.luck.visnyk.1.malchik_300_knigi_chtenie.pic.1}

«Обожнюю читати. Читаю книги різних авторів, різних видань та з різноманітними
сюжетами, – розповідає хлопчик. – Детективи, фентезі, історія – мені все до
вподоби. Комікси я не надто полюбляю, але читаю і їх, цікаві трапляються.
Багатьох вони приваблюють тим, що коротенько розповідають якісь захопливі
історії, ще й малюнки яскраві».

10 січня Володя з мамою святкували невеличкий ювілей – їхньому блогу
виповнилося п’ять років. Хлопчик зізнається: ідея була мамина, а він радо
долучився.

\ii{23_01_2022.stz.news.ua.luck.visnyk.1.malchik_300_knigi_chtenie.pic.2}

«Книжки ми читаємо разом. Хоча іноді буває, що я чимось зайнята, і син мене
переганяє. Потім каже: «Я прочитав 70 сторінок, тепер наздоганяй мене», –
розповідає мама хлопчика Наталка. – Зазвичай огляди записуємо одразу після
прочитання, як то кажуть, першим дублем. Коли ще свіжі враження, без підготовки
– тоді й емоція яскравіша, щиріша».

«Аудиторія у нас не надто велика – близько 900 підписників. Але ми віримо, що
їх з часом буде значно більше, – сподівається юний блогер. – Мама записує наші
блоги на свій телефон, купила триногу, аби якість запису була кращою. А тато
нам подарував мікрофони. Хоча зараз ми їх не використовуємо, бо мені трохи не
сподобався звук».

Володя додає: вони з мамою старанно ведуть статистику прочитаних книг. І в
листопаді минулого року записали огляд вже трьохсотої книжки. Каже, часто
підписники в коментарях просять прочитати й розказати про те чи те видання.
