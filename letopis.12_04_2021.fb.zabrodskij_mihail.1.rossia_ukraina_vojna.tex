% vim: keymap=russian-jcukenwin
%%beginhead 
 
%%file 12_04_2021.fb.zabrodskij_mihail.1.rossia_ukraina_vojna
%%parent 12_04_2021
 
%%url https://www.facebook.com/permalink.php?story_fbid=290284532664755&id=100050496309386
 
%%author 
%%author_id 
%%author_url 
 
%%tags 
%%title 
 
%%endhead 

\subsection{Вітаю, шановні друзі!}
\label{sec:12_04_2021.fb.zabrodskij_mihail.1.rossia_ukraina_vojna}
\Purl{https://www.facebook.com/permalink.php?story_fbid=290284532664755&id=100050496309386}

\ifcmt
  pic https://scontent-mxp1-2.xx.fbcdn.net/v/t1.6435-9/172735259_290284412664767_8259785924056622895_n.jpg?_nc_cat=101&ccb=1-3&_nc_sid=730e14&_nc_ohc=4b9EdnMDuO8AX-8nnyC&_nc_ht=scontent-mxp1-2.xx&oh=5fb2be837c911115749592762dad80c9&oe=609CE9E0
\fi

У інформаційному штормі навколо активізації дій проросійських найманців і
переміщень підрозділів ЗС РФ вздовж наших кордонів, як виявилося, достатньо
легко втратити орієнтування.

З одного боку на це впливає неабиякий потік пропагандиських  закидів зі сторони
противника, з іншого – поляризовані і занадто бадьорі/панічні реакції чисельних
вітчизняних “експертів” і знавців сучасних  таємничих військово-політичних
методик. Третім фактором дезорієнтації, як і завжди, виступає невизначеність і
протиріччя в інформаційному просторі.

Спробуємо розібратись. Розглядаймо виключно два факти, які вже є визнаними.
Перший – активне перекидання військ вздовж наших кордонів і в окупований Крим,
другий – більш ніж помітна активізація бойових дій на Сході. Обидва з них є
частинами одного і того ж сценарію. І їх розділення має скоріше
військово-географічну природу. Об’єднує їх, безперечно, те, що вони складаються
в єдину картину демонстрації готовності до рішучих і масштабних дій проти
України. А точніше, - комусь дуже хочеться щоб така картина складалася.

Не станемо розглядати всі можливі варіанти розгортання подій. Відмітимо лише окремі моменти. 

Схід України навряд чи стане головним стратегічним напрямком. Цілком безглуздо
планувати наступальні дії там, де зосереджені найбільш підготовлені і боєздатні
військові частини і з’єднання наших Збройних Сил. Тим більше, що накреслення
наших кордонів пропонує значно ширший перелік: Сумський і Харківський напрямки,
поліський операційний район, Кримські перешийки, Чорноморське узбережжя тощо. 

Кінцевою метою вторгнення навряд чи може бути встановлення контролю над всією
територією нашої країни. Мова може йти про Лівобережжя і південні регіони. При
цьому, за визнанням навіть російських аналітиків, “повний розгром і капітуляція
Збройних Сил України є малоймовірними”. Це в свою чергу може означати, що
навіть після перших нищівних ударів, супротив Збройних Сил і населення нашої
країни загарбникам буде лише наростати.

Допомога партнерів, перш за все США і окремих країн-членів НАТО, буде
вирішальною. Відповідні показові кроки вже здійснені і продемонстрована
готовність до наступних. І мова може йти точно не про бойові дії на суходолі. Є
відпрацьований механізм забезпечення розвідувальними даними, існує можливість
практично безмежної ресурсної підтримки, відпрацьовано питання закриття
повітряного простору і вирівнювання балансу з російськими ВКС тощо.

Спробуємо поставити питання по іншому. Чому знову – Україна? Чому зараз? Чого можна домогтися?

Теперішній стан конфлікту на українському Сході мав би цілком задовільнити
кремлівських гібридних стратегів. Протистояння триває, втрати з обох сторін,
ходіння по колу в мінському/нормандському форматах може тривати вічно. Але
противнику цього замало. Кінцева мета – зробити Україну або значну її частину
підконтрольним Кремлю утворенням. Відмовитися від мети – неможливо. Теперішній
стан справ: існування “ЛНР/ДНР” є просто омріяним для 2014-го, але ресурсно
обтяжливим і політично безперспективним для 2021-го. Для путінського “рейху”
Україна завжди була і залишається тим елементом, якого так не вистачає.

Новий імпульс гідридній операції проти України дає, як не дивно, відносне
затишшя в інших точках “проекції сили” РФ, таких як наприклад Сирія.
Перевантаженій власними внутрішніми проблемами імперії важко існувати без
чергової версії зовнішньої загрози і протистояння з усім світом. Тому
драматичні відео бомбардувань сирійських міст будуть ще довго систематично
змінюватись повідомленнями про “вбитих українським БПЛА дівчат” або про
готовність Об’єднаних Сил до міфічного наступу. В РФ, як добре відомо,
телевізійні вистави і накручування суспільної думки є перевіреними способами
консолідації і відволікання від невтішної реальності.

Тепер – про можливий кінцевий результат. На другий рік своєї каденції правляча
владна команда в Україні взялася за протидію російській агентурі. Закриття,
нажаль лише окремих, рупорів кремлівської пропаганди не могло не викликати
відповідної реакції. Небезпідставно вважаючи теперішніх українських
можновладців слабкою командою, Кремль намагається примусити політичне
керівництво “дати задній хід” цим процесам і, цілком можливо, припинити
укріплення зв’язків із стратегічним партнером – США.

В цій, вже традиційно складній для нас, обстановці потрібно не вишукувати
найбільш сприятливий для нас варіант розвитку ситуації. Потрібно в масштабі
всієї держави ГОТУВАТИСЯ до самого складного з них!..
