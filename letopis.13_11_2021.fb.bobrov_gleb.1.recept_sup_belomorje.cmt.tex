% vim: keymap=russian-jcukenwin
%%beginhead 
 
%%file 13_11_2021.fb.bobrov_gleb.1.recept_sup_belomorje.cmt
%%parent 13_11_2021.fb.bobrov_gleb.1.recept_sup_belomorje
 
%%url 
 
%%author_id 
%%date 
 
%%tags 
%%title 
 
%%endhead 
\subsubsection{Коментарі}

\begin{itemize} % {
\iusr{Алексей Матвеев}

Спасибо Глеб, обязательно попробую с молоком. Я обычно рыбу варю очень долго, а
потом ещё настаиваю на горячей печи, ну это в деревне, и только с перлухой.
Молотят все!)) Отдельное спасибо за совет борьбы с пеной на кастрюле.


\iusr{Николай Ардальянов}
Глеб молодец, что назвал это суп..... уха готовится по другой технологии...

\iusr{Alexander Tumanov}

Народная практика, однако, советует пользовать не менее 5 сортов рыбы.
Неизгладимое впечатление произвела на меня уха, еденая в Венгрии, именно из 5
сортов рыбы и без всяких наполнителей, типа ракушек или крупы. С большим
количеством острого красного перца.

\begin{itemize} % {
\iusr{Глеб Бобров}
Вот, я об этом тоже думаю, Саш! В следующий раз добавлю в рыбный набор из молок, красной рыбы и тресковых - сельдь и лакедру.
\end{itemize} % }

\iusr{Наталья Паваляева}

\ifcmt
  ig https://scontent-frx5-2.xx.fbcdn.net/v/t39.1997-6/p480x480/105941685_953860581742966_1572841152382279834_n.png?_nc_cat=1&ccb=1-5&_nc_sid=0572db&_nc_ohc=dMmZHIhxBS8AX-dHIb0&_nc_ht=scontent-frx5-2.xx&oh=2e70a53db06234fede2a96c8ba0d6a8a&oe=61AA258B
  @width 0.1
\fi

\iusr{Елена Промыслова}
Завидую...
Мужу запрещены все бульоны кроме овощного, так что у нас только постные супы. А ведь хочется!  @igg{fbicon.face.confused} 

\begin{itemize} % {
\iusr{Nadya Carpenter}
Сделайте себе отдельно:)

\iusr{Елена Промыслова}
\textbf{Nadya Carpenter}
Так ему же тоже хочется, мне мужа жалко.
\end{itemize} % }

\iusr{Евгений Марченко}

Глеб Леонидыч, салам! Как азиатский сведловчанин луганчанину. Вот всё ж
зашибись, но зачем молоко в рыбий суп?

И второе, моркву на тёрке, либо как корейскую, мелко соломкой пошинковать?

Рахмет тебе душевнейший, однако попытаюсь разобраться без применения молочки
)))

\begin{itemize} % {
\iusr{Глеб Бобров}
Весь север от Беломорья до Балтики и выше готовит рыбные супы с молоком - это классика.

\iusr{Евгений Марченко}
\textbf{Глеб Бобров}
Йок вопросов. Я живу в Азии. Чуть южнее и восточнее. Просто для меня это... непонятно.

\iusr{Глеб Бобров}
Евгений с чего это Ёбург - вдруг Азия )))))))))))

\iusr{Евгений Марченко}
\textbf{Глеб Бобров}

Глеб Леонидыч, между нами. Официальная граница Европы -Азии проходит по
пермской трассе в 12-ти верстах западнее города. Чуть не доезжая до объездной.
Я там в огороды езжу еженедельно. Ты уж мне поверь пожалуйста на слово.

Фактическая граница проходит ещё на 12 вёрст западнее, на своротке на
Первоуральск. Там тоже есть знак.

Свердловск это Азия. Как бы кто как к этому факту ни относился )))

\iusr{Евгений Марченко}

\ifcmt
  ig https://scontent-frx5-2.xx.fbcdn.net/v/t39.30808-6/254944658_3957199841048316_8719938668755779080_n.jpg?_nc_cat=109&ccb=1-5&_nc_sid=dbeb18&_nc_ohc=DB9RhuKSsUEAX9R4-uu&_nc_ht=scontent-frx5-2.xx&oh=f0314f6e23c6c8e338988d350d141781&oe=61A92815
  @width 0.4
\fi

\iusr{Евгений Марченко}
И да, я азиат. Русский. Мне нисколько не стыдно.

\end{itemize} % }

\iusr{Андрей Третьяков}
Я после того, как у Похлёбкина рецепт прочёл, подсел на калью

\begin{itemize} % {
\iusr{Глеб Бобров}
У меня пока руки и голова не дошла да кальи.

\iusr{Андрей Третьяков}
\textbf{Глеб Бобров}
Очень просто и вкусно

\iusr{Глеб Бобров}
Андрей, обязательно попробую и доложу по результатам.

\end{itemize} % }

\iusr{Дмитрий Винокуров}
Супер, \textbf{Глеб Бобров} !
Интересно, завтра попробую. Тебе пора за кулинарную книгу браться. В смысле писать)

\begin{itemize} % {
\iusr{Глеб Бобров}
После Похлёбкина и Сталика???
Не, я не эпигонствую - готовлю для души ))))))))

\iusr{Дмитрий Винокуров}
\textbf{Глеб Бобров} и напиши для души)))
\end{itemize} % }

\end{itemize} % }
