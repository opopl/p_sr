% vim: keymap=russian-jcukenwin
%%beginhead 
 
%%file 01_04_2018.stz.news.ua.mrpl_city.1.vsmatrivajas_v_otkrytki
%%parent 01_04_2018
 
%%url https://mrpl.city/blogs/view/vsmatrivayas-v-otkrytki
 
%%author_id burov_sergij.mariupol,news.ua.mrpl_city
%%date 
 
%%tags 
%%title Всматриваясь в открытки
 
%%endhead 
 
\subsection{Всматриваясь в открытки}
\label{sec:01_04_2018.stz.news.ua.mrpl_city.1.vsmatrivajas_v_otkrytki}
 
\Purl{https://mrpl.city/blogs/view/vsmatrivayas-v-otkrytki}
\ifcmt
 author_begin
   author_id burov_sergij.mariupol,news.ua.mrpl_city
 author_end
\fi

Почтовую открытку, как утвер­ждает знаменитый филокартист Николай Спиридонович
Тагрин в своей книге \enquote{Мир в открытке} (М., \enquote{Изобразительное искусство}, 1978),
изобрел обер-почтмейстер Пруссии Генрих Штефан. Это он 30 ноября 1865 года на
почтовой конференции в г. Карлсруэ предложил ввести новый \enquote{почтовый листок},
предназначенный для открытых сообщений. Но первые открытки были выпущены в
свет лишь 22 сентября 1869 года австрийским правительством, которое
воспользовалось предложением Штефана.

С конца девяностых годов XIX века получило широкое распространение издание
открыток с видами городов и ландшафтов разных уголков мира. Прибыльным делом
стали заниматься крупные полиграфические фирмы, причем не только для
удовлетворения по­требителей своих стран, но и на экспорт. Н. С. Тагрин пишет:
\enquote{Известная шведская фирма \enquote{Гранберг} до 1917 года издавала многочисленные виды
русских городов исключительно для русского рынка. Частично это делалось по
специальным заказам русских издателей}. Делали  такие заказы \enquote{Гранбергу} и
мариупольцы, например владелец табачного магазина Нерофиди. Кроме Нерофиди,
изданием открыток с изображением различных мест Мариуполя занимались владельцы
типографского и книготоргового дела братья Гольдрины, книготорговец
Приходько-Пыхненко, книжный магазин Мариупольского уездного земства. Печатали
открытки не только в Стокгольме. Известны, например, \enquote{мариупольские сюжеты},
изготовленные московской фирмой \enquote{Фототипия Шерер, Набгольц и Ко}.

По сведениям известного коллекционера Л. П. Морозова, до 1917 года было издано
без малого пятьсот почтовых карточек с мариупольскими видами, правда, с учетом
их разновидностей. Кстати, Л. П. Морозов является автором и издателем
уникального в своем роде иллюстрированного каталога \enquote{Мариуполь на видовых
открытках}, вышедшего тремя изданиями (последнее – в 2009 году). Большими
собраниями почтовых открыток с изображениями нашего города располагают
Мариупольский краеведческий музей и некоторые коллек­ционеры, отдельные
экземпляры встречаются у старожилов. Всмотревшись в открытки, можно узнать
много интересного о прошлом.

\ii{01_04_2018.stz.news.ua.mrpl_city.1.vsmatrivajas_v_otkrytki.pic.1}

Вот на открытке городской театр (он находился там, где теперь на проспекте
Ленина (\emph{ныне - пр. Мира. - Прим. ред.}) стоит многоэтажный дом № 24 с отделением
Ощадбанка на первом этаже) - афиши, на одной можно прочитать фамилию Родичева,
вероятно, актера-бенефицианта, на других - названия пьес из репертуара
театра: \enquote{Идиот} и \enquote{Свадьба Кречинского}. Хороший был вкус у мариупольских
театралов. Над воротами, примыкающими к зданию театра, можно различить вывеску:
\enquote{Фортепианная мастерская \enquote{Аккорд}: переделка, прокат, Казачковский}. У входа в
театр стоят франт в \enquote{котелке}, двое мальчишек в гимназической форме, мимо
проходит девочка-нянька с ребенком на руках.

\ii{01_04_2018.stz.news.ua.mrpl_city.1.vsmatrivajas_v_otkrytki.pic.2}

Другая почтовая карточка. На ней снимок четной стороны Торговой улицы между
Георгиевской и Екатерининской. На этом месте в пятидесятые годы был построен
четырехэтажный дом, адрес которого - проспект Ленина, 10/20. Что мы видим на
открытке? Магазины, вывески: \enquote{Г.С. Брон и Л.А. Кирш}, \enquote{Обувь и галоши, Р. И
Д. Брон}, \enquote{Конфекцион готового платья}, \enquote{Конфекцион Универсаль}. Человек в
белой сорочке с ношей на плече. На мостовой - пролетка, ожидающая седока, две
подводы, у входа в магазин несколько бочек.

\ii{01_04_2018.stz.news.ua.mrpl_city.1.vsmatrivajas_v_otkrytki.pic.3}

Еще один вид нашего города на открытке. Так в начале XX века выглядел
перекресток проспекта Ленина и улицы Греческой. Дом с аптекой, он существует до
сих пор, правда, существенно перестроенный. Приставная лестница, несколько
мужчин, всадник на лошади. В глубине – церковь св. Марии Магдалины. Она стояла
в сквере, который теперь называется сейчас Театральным. В центре открытки –
часовня. Местный люд называл ее капличкой. В апреле 1891 года, на
путешествующего по Японии наследника российского престола, будущего императора
Николая II, было совершено покушение, но, как сказано было народу, произошло
чудо: великий князь остался жив. Вот в честь этого события Мариупольская
городская дума и решила воздвигнуть часовню на месте вконец обветшавшей старой
церкви. Церковь разобрали, возвели часовню и 5 мая 1895 года при большом
стечении народа освятили. Часовня простояла до 1933 года, пока ее не снесли.
Предлог был благовидный: она мешала прокладке трамвайного пути. Да, на
центральном проспекте города ходил когда-то трамвай.

\ii{01_04_2018.stz.news.ua.mrpl_city.1.vsmatrivajas_v_otkrytki.pic.4}

Это трехэтажное здание и сейчас стоит на углу Торговой и Николаевской улиц.
Его современный адрес - Николаевская улица, 7/17. Давайте же посмотрим на
открытку с его изображением тех времен, когда в нем было частное реальное
училище Василия Ивановича Гиацинтова. Внешне почти ничего не изменилось. По обе
стороны двери две картины, на одной изображен господин, на другой – шикарная
дама. Видна еще вывеска с надписью \enquote{Центральное депо готового платья}.
Она укреплена над входом.  Тумба с наклеенными афишами и объявлениями.
Несколько мальчишек-реалистов на тротуаре. Почему они не на занятиях?

Согласитесь, увлекательное это дело - рассматривать старинные почтовые открытки
с фотографиями уголков Мариуполя, давно ушедших в небытие.
