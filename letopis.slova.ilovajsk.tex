% vim: keymap=russian-jcukenwin
%%beginhead 
 
%%file slova.ilovajsk
%%parent slova
 
%%url 
 
%%author 
%%author_id 
%%author_url 
 
%%tags 
%%title 
 
%%endhead 
\chapter{Иловайск}
\label{sec:slova.ilovajsk}

%%%cit
%%%cit_head
%%%cit_pic
%%%cit_text
Есть такой пропагандистский фильм: \enquote{\emph{Иловайск} 2014. Батальон
Донбасс}. Вы про него не слышали, потому что лента провалилась в прокате - как
и все остальные поделки украинского патриотического кино. Если кратко, это
рассказ о том, как фельдмаршал Семен Семенченко победил Путина во время боев за
этот донбасский город. Именно так все было, даже не сомневайтесь.  Локациями
для съемок \enquote{\emph{Иловайска}} стали кварталы разрушенного города, по
которым бегали большие и маленькие сепаратисты. Так вот, внезапно оказалось,
что они проходили не в прифронтовой зоне, а в заброшенном городе Цукроваров,
расположенном в 280 километрах от Киева. Там при совке был мощнейший сахарный
завод, а потом все декоммунизировали, и теперь это заброшенный город-призрак.
Ну а в качестве массовки для фильма использовали уцелевшее местное население
%%%cit_comment
%%%cit_title
\citTitle{Разрушенный Иловайск снимали в 280 километрах от Киева}, 
Андрей Манчук, strana.ua, 15.06.2021
%%%endcit

