% vim: keymap=russian-jcukenwin
%%beginhead 
 
%%file 18_12_2020.news.ua.strana.3.pfizer_vaccine_obmorok
%%parent 18_12_2020
 
%%url https://strana.ua/news/307514-video-kak-v-ssha-medsestra-upala-v-obmorok-posle-vaktsinatsii-preparatom-pfizer.html
 
%%author 
%%author_id 
%%author_url 
 
%%tags covid_vaccine,pfizer,obmorok
%%title На видео попало, как в США медсестра упала в обморок после вакцинации препаратом Pfizer
 
%%endhead 
 
\subsection{На видео попало, как в США медсестра упала в обморок после вакцинации препаратом Pfizer}
\label{sec:18_12_2020.news.ua.strana.3.pfizer_vaccine_obmorok}
\Purl{https://strana.ua/news/307514-video-kak-v-ssha-medsestra-upala-v-obmorok-posle-vaktsinatsii-preparatom-pfizer.html}

\ifcmt
pic https://strana.ua/img/article/3075/video-kak-v-14_main.jpeg
caption Женщина заверила, что обморок не связан с вакцинацией. Скриншот из видео 
\fi

Старшая медсестра Тиффани Довер из клиники в городе Чаттануга (штат Теннесси,
США) упала в обморок во время пресс-конференции, посвященной вакцинации
медработников от коронавируса. Накануне встречи с журналистами ей ввели вакцину
американской компании Pfizer. 

Видеозапись инцидента появилась на YouTube 17 декабря.

Молодая женщина почувствовала себя плохо примерно через 17 минут после введения
вакцины от COVID-19. В это время она как раз говорила о вакцинации, но внезапно
прервала речь, приложила руку ко лбу, прикрыла глаза и извинилась. А в
следующую секунду начала падать. К счастью, рухнуть и удариться обо что-нибудь
ей не дали стоявшие рядом врачи.

Обморок медсестры длился недолго. Она вскоре пришла в себя и смогла продолжить
общение с журналистами.

"У меня закружилась голова, но сейчас мне лучше, и боль в руке (от инъекции -
Ред.) прошла", - рассказала о своем самочувствии Тиффани.

Также она добавила, что терять сознание в случае появления боли - это
особенность ее организма.

Врачи клиники, где работает медсестра, пояснили, что обморок не связан с
введением вакцины Pfizer. А по данным ученых, потеря сознания довольно часто
сопутствует введению любых вакцин. Кратковременная потеря контроля над собой
может быть реакцией организма на боль или стресс.

\video{https://youtu.be/_Cs7nwfKKg4}

Еще один тревожный инцидент случился недавно на Аляске. Там у медработницы
после прививки препаратом Pfizer развилась сильная аллергическая реакция.\Furl{https://strana.ua/news/307108-v-ssha-u-medrabotnitsy-posle-vaktsinatsii-pfizer-razvilas-allerhicheskaja-reaktsija.html}
Женщину даже госпитализировали.

Аллергические реакции наблюдали и у двух жителей Великобритании, привитых
вакциной Pfizer. В связи с этим медицинские власти страны рекомендовали
склонным к аллергии людям воздержаться от вакцинирования от коронавируса
препаратом этой компании.\Furl{https://strana.ua/news/306362-v-velikobritanii-allerhikov-prizvali-ne-delat-vaktsinu-ot-koronavirusa.html}
