% vim: keymap=russian-jcukenwin
%%beginhead 
 
%%file 21_02_2022.fb.uljanov_anatolij.1.ldnr
%%parent 21_02_2022
 
%%url https://www.facebook.com/dadakinder/posts/5278009475551506
 
%%author_id uljanov_anatolij
%%date 
 
%%tags __feb_2022.putin.priznanie,dnr,donbass,lnr
%%title Радоваться тому, что происходит, нечего, но и отчаиваться нет причин
 
%%endhead 
 
\subsection{Радоваться тому, что происходит, нечего, но и отчаиваться нет причин}
\label{sec:21_02_2022.fb.uljanov_anatolij.1.ldnr}
 
\Purl{https://www.facebook.com/dadakinder/posts/5278009475551506}
\ifcmt
 author_begin
   author_id uljanov_anatolij
 author_end
\fi

1. Провозглашение «суверенитета» Л/ДНР означает крах Минских соглашений, и
дальнейший территориальный распад Украины. 

\ii{21_02_2022.fb.uljanov_anatolij.1.ldnr.pic.1}

2. Поскольку реальные суверенитеты не провозглашаются за границей, – и тем
более в столицах империй, – говорить о суверенитете Л/ДНР в результате
провозглашения такового в Москве не приходится. Вероятность их вхождения в
состав России высока, но политически выгодней для неё сохранять их в статусе
«суверенных колоний», которые станут перманентным фактором дестабилизации
Украины – только уже без опции возвращения в её состав. Таковая ушла вместе с
отказом от выполнения Минских соглашений и внеблоковой политики. 

3. Провозглашение «суверенитета» Л/ДНР – это Пиррова победа для украинских
националистов, к которым сегодня относятся уже самые «левые» из наших
либералов. Наконец-то кричалка «Чемодан-Вокзал-Россия» воплотилось. Регионы,
населённые «ватой», «совками», «сепарами», «колорадами», «быдлом» и прочими
«неправильными» украинцами уходят. Визг по этому поводу связан исключительно с
потерей территорий, а не с потерей людей, которые на них проживают.

4. Продолжение территориального распада Украины означает деградацию украинского
разнообразия. И речь не только о потере части «иного» населения. В результате
признания «суверенитета» Л/ДНР мы можем ожидать ожесточения «охоты на ведьм» в
нашем обществе, обострения репрессии и дальнейшего подавления демократической
перспективы в регионе.          

5. Освобождение от дегуманизирующей киевской политики посредством интеграции в
московскую не предвещает жителям «суверенных колоний» ничего, кроме России со
всеми «прелестями» её рыночной автократии.  

6. Возникновение «суверенных колоний» России наносит удар и по остаткам
украинского суверенитета. Украина оказывается разделена между двумя империями,
и становится страной двух колоний в имперском клинче. Дальше ругаться
«суверенитетами» будут уже откровенные куклы.

7. Угроза войны возрастает, и зависит уже не от Украины, а от способности
Вашингтона и Москвы договориться о разделении сфер влияния. Принципиальным
индикатором в этом процессе является всё, что касается «Северного потока-2».

8. Всё это прискорбный, но предсказуемый исход запущенных майданами
исторических процессов, которые, в свою очередь, являются продуктами 1991-го
года. Их жертвами были и остаются не олигархи, а мирные люди.

9. Радоваться тут нечему, но и отчаиваться не стоит. Как учит нас один
политический деятель начала 20-го века, «отчаяние свойственно тем, кто не
понимает причин зла, не видит выхода, и не способен бороться». Там, где
капиталисты душат народы, всегда будут те, кто борется с их империей. Ночь
темна. Но утро неизбежно.

10. Обострение имперской агрессии требует от нас соответствующего обострения
антивоенной, инклюзивной, демократической повестки, но главное – социальной
организации, способной её продвигать и воплощать.
