% vim: keymap=russian-jcukenwin
%%beginhead 
 
%%file 21_02_2023.stz.news.ua.donbas24.3.ljudy_zdyvuvaly_za_rik_vijny_dobrym_sercem
%%parent 21_02_2023
 
%%url https://donbas24.news/news/lyudi-yaki-zdivuvali-za-rik-viini-svoyim-dobrim-sercem-foto
 
%%author_id ivanova_jana.mariupol,news.ua.donbas24
%%date 
 
%%tags 
%%title Люди, які здивували за рік війни своїм добрим серцем
 
%%endhead 
 
\subsection{Люди, які здивували за рік війни своїм добрим серцем}
\label{sec:21_02_2023.stz.news.ua.donbas24.3.ljudy_zdyvuvaly_za_rik_vijny_dobrym_sercem}
 
\Purl{https://donbas24.news/news/lyudi-yaki-zdivuvali-za-rik-viini-svoyim-dobrim-sercem-foto}
\ifcmt
 author_begin
   author_id ivanova_jana.mariupol,news.ua.donbas24
 author_end
\fi

\ii{21_02_2023.stz.news.ua.donbas24.3.ljudy_zdyvuvaly_za_rik_vijny_dobrym_sercem.pic.front}
\begin{center}
  \em\color{blue}\bfseries\Large
Українці за рік війни та незламності показали щирі серця та рятували один одного
\end{center}

Донбас24 пропонує дізнатися про надзвичайних людей, які з початку
повномасштабного вторгнення робили усе можливе, аби врятувати та допомогти
людям, які опинилися у біді через війну.

Російська агресія круто змінила життя та сенси українців. Ті, хто раніше
займався музикою, читав лекції або просто вів спортивні заняття, несподівано
брав зброю або починав волонтерити.

\textbf{Читайте також:} \href{https://donbas24.news/news/micnisi-za-zalizo-yak-zsu-geroyicno-boronyat-doneccinu-v-umovax-zimi-foto}{\emph{Міцніші за залізо: як ЗСУ героїчно боронять Донеччину в умовах зими}}%
\footnote{Міцніші за залізо: як ЗСУ героїчно боронять Донеччину в умовах зими, Яна Іванова, donbas24.news, 20.02.2023, \par%
\url{https://donbas24.news/news/micnisi-za-zalizo-yak-zsu-geroyicno-boronyat-doneccinu-v-umovax-zimi-foto}%
}

% 1 - 18-річний опікун В'ячеслав
\ii{21_02_2023.stz.news.ua.donbas24.3.ljudy_zdyvuvaly_za_rik_vijny_dobrym_sercem.1}

% 2 - Денис Мінін — витягував людей з Маріуполя
\ii{21_02_2023.stz.news.ua.donbas24.3.ljudy_zdyvuvaly_za_rik_vijny_dobrym_sercem.2.minin}

% 3 - П'ятирічна волонтерка Марія
\ii{21_02_2023.stz.news.ua.donbas24.3.ljudy_zdyvuvaly_za_rik_vijny_dobrym_sercem.3.maria}

\subsubsection{Віктор Разживін — вивіз 25 тисяч українських книжок}

Цей чоловік з Донеччини \href{https://donbas24.news/news/pereselenec-zi-slovyanska-vryatuvav-vid-viini-25-tisyac-ukrayinskix-knizok-foto}{\emph{здивував усю Україну}}%
\footnote{Переселенець зі Слов'янська врятував від війни 25 тисяч українських книжок, Яна Іванова, donbas24.news, 04.07.2022, \par\url{https://donbas24.news/news/pereselenec-zi-slovyanska-vryatuvav-vid-viini-25-tisyac-ukrayinskix-knizok-foto}}
своїм вчинком. Коли у рідному для нього Слов'янську почалися перші обстріли, він зрозумів, що варто рятувати не
тільки людей, але й українську мову. Книги для кандидата філологічних наук та
власника магазину Віктора Разживіна завжди були сенсом. Він мав книгарню у
Слов'янську і розумів, що в разі жахливого розвитку подій росіяни не будуть
берегти українські книжки.

\begin{leftbar}
\emph{\enquote{Уявімо, що це все залишилося у місті — для мене була б трагедія, якби всі
книжки згоріли. Рукописи не горять, а книжки горять і дуже гарно. Вони не
повинні горіти, вони повинні читатися. І мені взагалі не хотілося, щоб подібне
відбулося}}, — розповідав він свого часу сайту Донбас24.
\end{leftbar}

\ii{21_02_2023.stz.news.ua.donbas24.3.ljudy_zdyvuvaly_za_rik_vijny_dobrym_sercem.pic.4}

Так, специфічний вантаж пакували три дні і за один раз доправили у Київ. Так 25
тисяч книжок з українськими творами були врятовані.

\textbf{Читайте також:} \href{https://donbas24.news/news/slovo-2022-roku-yakii-viraz-stav-golovnim-v-rik-viini}{\emph{Слово 2022 року: який вираз став головним в рік війни}}%
\footnote{Слово 2022 року: який вираз став головним в рік війни, Яна Іванова, donbas24.news, 10.01.2023, \par%
\url{https://donbas24.news/news/slovo-2022-roku-yakii-viraz-stav-golovnim-v-rik-viini}%
}

Нагадаємо, що \enquote{Укрпошта} випустила \href{https://donbas24.news/news/rik-nezlamnosti-ukrposta-vipustila-marku-do-ricnici-viini-z-poslannyam-do-putina-foto}{\emph{марку до річниці війни}}%
\footnote{Рік незламності: \enquote{Укрпошта} випустила марку до річниці війни з посланням до Путіна, Тетяна Веремєєва, donbas24.news, 20.02.2023, \par\url{https://donbas24.news/news/rik-nezlamnosti-ukrposta-vipustila-marku-do-ricnici-viini-z-poslannyam-do-putina-foto}} з посланням до Путіна.

Ще більше новин та найактуальніша інформація про Донецьку та Луганську області
в нашому телеграм-каналі Донбас24.

Головне фото — pon.org.ua

%\ii{21_02_2023.stz.news.ua.donbas24.3.ljudy_zdyvuvaly_za_rik_vijny_dobrym_sercem.txt}
