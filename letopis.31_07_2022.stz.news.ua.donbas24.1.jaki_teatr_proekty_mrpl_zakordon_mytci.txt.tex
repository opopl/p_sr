% vim: keymap=russian-jcukenwin
%%beginhead 
 
%%file 31_07_2022.stz.news.ua.donbas24.1.jaki_teatr_proekty_mrpl_zakordon_mytci.txt
%%parent 31_07_2022.stz.news.ua.donbas24.1.jaki_teatr_proekty_mrpl_zakordon_mytci
 
%%url 
 
%%author_id 
%%date 
 
%%tags 
%%title 
 
%%endhead 

Які театральні проєкти та культурні заходи, присвячені Маріуполю, реалізуються закордонними митцями

Андрій Палатний, куратор численних театральних проєктів розповів, як європейські режисери підтримують Маріуполь

У 2021 році в рамках фестивалю iStage 2021 та Марафону міжнародних резиденцій
до Маріуполя приїжджали європейські режисери і разом з Народним театром
«Театроманія», Першою театральною школою-студією та Театром ляльок створювали
унікальні вистави. З початком повномасштабного вторгнення росії в Україну
Крістін Дісмен, Мадлен Бонгард та Евангелос Космідіс вирішили підтримати місто,
яке вони встигли побачити на власні очі розвиненим та квітучим. За декілька
місяців війни від Маріуполя залишилися руїни і про цю трагедію тепер
розповідають режисери у своїх виставах. Андрій Палатний, куратор театральної,
перфомативної та цифрової програм «Гогольfest dream», фестивалю iStage 2021 та
Марафону міжнародних резиденцій підкреслив, що через війну багато планів не
вдалося реалізувати.

«24 лютого я зустрів в потязі. Їхав до Маріуполя. Потяг зупинився серед поля.
Було незрозуміло, що далі. Ввечері я все ж доїхав до міста, але всіх пасажирів
повезли одразу ж назад до Києва. Наприкінці лютого вже мав відкриватися Центр
сучасного мистецтва "Готель Континенталь" і я планував зустрітися з
керівництвом ЦСМу, щоб обговорити подальші плани, адже всі вистави Марафону
міжнародних резиденцій мали стати репертуаром „Готелю Континенталь“. В перший
день війни я побачив востаннє Маріуполь цілим і неушкодженим», — розповів
Андрій Палатний.

Через два тижні Андрій організував зустріч з режисерами, які приїздили до
Маріуполя, і було вирішено створити власну команду. Спочатку вони зробили сайт
mariupol.life англійською мовою, який має на меті познайомити європейських
жителів з Маріуполем, його культурою до війни. Зокрема, на сайті можна побачити
події, що відбувалися в рамках Марафону міжнародних резиденцій 2021. Також на
сайті висвітлені нищівні руйнування Маріуполя.

«Маріуполь бомбили з перших днів війни. Там зруйновано близько 90% міста. Це
місце гуманітарної катастрофи», — зазначено на сайті.

Найголовнішою задачею команди — було допомогти маріупольській молоді,
талановитим акторам та актрисам вибратися з облоги міста. Далі кожен з
режисерів почав проводити заходи на підтримку Маріуполя в різних містах та за
власні кошти. Перша масштабна культурна подія відбулася на легендарній сцені
Deutsches Theater в Берліні. Там вразив всіх своїм виступом режисер Народного
театру «Театроманія» Антон Тельбізов та свою історію через відео-зв'язок
розповіла актриса Марія Бойко. Директорка Junges DT Birgit Lengers підтримала
ініціативу проведення заходу, адже сама з виставою приїздила до Маріуполя в
2019 році, яку представляли в «Колегіум-школі» № 1.

«Через культурні події, вистави ми почали розповідати світу про Маріуполь, про
його культурні події та повне знищення російськими військами», — додав
Палатний.

Кульмінацією стала подія, що відбулася 27 березня, Deutsches Theater Berlin
організував вечір під назвою «Українські голоси для Маріуполя», присвячений
портовому місту, де тиждень тому бомбами було зруйновано театр, який служив
укриттям для тисячі маріупольців.

21 травня в Національному театрі в Празі під час конференції Європейської
театральної конвенції була представлена вистава «Аляска» режисера-постановника
Національного театру Грецї в Афінах та Муніципального театру Пірея Евангелоса
Космідіса, яку він вперше представив у серпні 2021 року в Маріуполі. Виставу
було перекладено англійською мовою. Спектакль залишається актуальним, адже
включає реальні думки та коментарі маріупольських дітей, які зіткнулися з
війною. Наразі виставу перекладено різними мовами, щоб її змогло побачати
якомога більше глядачів.

А 17 червня було відкрито центр Hotel Continental — Art Spase in Exile, в якому
була представлена виставка, присвячена Маріуполю з 2014 до 2022 років. Центр
було створено німецькою режисеркою Крістін Діссман спільно з Андрієм Палатним.
Він повинен реалізувати декілька функцій: стати домом культури для українських
митців та платформою для діалогу між українськими та європейськими діячами.

«Кожен проєкт має свою стратегію розвитку і дає можливості рухатися далі», — пояснив Андрій Палатний. 

В підтримку Маріуполя і загалом України проводить зустрічі в zoom Мадлен
Бонгард — швейцарська режисерка, акторка та педагогиня. Також вона випустила
подкаст, присвячений місту. Влітку 2021 року в Маріуполі вона провела серію
акторських практик, що досліджували усвідомлення присутності.

У жовтні 2021 року режисерка з Німеччини Крістін Діссманн представила в
Маріуполі виставу-інсталяцію «Між часами» що поєднала театр, музику та
візуальне мистецтво з експериментальним мистецьким досвідом у новому вимірі і
була присвячена Маріуполю. Режисерка продовжує ставити цю виставу за участі
українських, білоруських та німецьких акторів.

Європейські митці намагаються підтримути маріупольську молодь. Зокрема, завдяки
Europen theatre convention акторка Народного театру «Театроманія» Марія Бойко
протягом 2 тижнів стажувалася з грецькими режисерами. 

«Ще дуже багато планів. Зокрема, хочемо поставити виставу, присвячену художній
керівниці Театру ляльок Ірині Руденко та працюємо над мультфільмом, який
присвятимо Маріуполю. Весь матеріал, який у нас був зібраний з „Гогольfest
dream“, фестивалю iStage 2021 та Марафону міжнародних резиденцій ми змонтували,
протитрували англійською та французькою і наразі представляємо на європейських
медіаплатформах», — додав Палатний.

Також відбуваються перемовини з Шекспіровським фестивалем у Гданську —
містом-побратимом Маріуполя, для подальшого представлення там театральної
програми Маріуполя. 

Нагадаємо, раніше Донбас24 розповідав про знаменитого маріупольського
архітектора Віктора Нільсена.

ФОТО: з відкритих джерел та особистих архівів Андрія Палатного і Ольги Демідко.
