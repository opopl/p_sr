% vim: keymap=russian-jcukenwin
%%beginhead 
 
%%file 06_10_2018.stz.news.ua.mrpl_city.1.dress_kod_dlja_pedagogov
%%parent 06_10_2018
 
%%url https://mrpl.city/blogs/view/dress-kod-dlya-pedagogov
 
%%author_id burov_sergij.mariupol,news.ua.mrpl_city
%%date 
 
%%tags 
%%title Дресс-код для педагогов
 
%%endhead 
 
\subsection{Дресс-код для педагогов}
\label{sec:06_10_2018.stz.news.ua.mrpl_city.1.dress_kod_dlja_pedagogov}
 
\Purl{https://mrpl.city/blogs/view/dress-kod-dlya-pedagogov}
\ifcmt
 author_begin
   author_id burov_sergij.mariupol,news.ua.mrpl_city
 author_end
\fi

Как-то пяток лет назад в мариупольских газетах и телевизионных каналах
промелькнуло сообщение, что управление образования горсовета намерено ввести
дресс-код для педагогов подведомственных ему учебных заведений. Дресс-код –
словечко, попавшее в русский разговорный язык, думается, из телевизионных
сериалов. Может быть, это и не так? На всякий случай  приведем определение его,
почерпнутое из Яндекса. Итак, \emph{дресс-код  — форма одежды, требуемая при
посещении определённых мероприятий, организаций, заведений...}

Тема дресс-кода педагогов вообще-то малоинтересна для пенсионера, чей возраст
пересек восьмидесятилетний рубеж, но она стала той искрой, которая зажгла
костер воспоминаний о виденных фотографиях с изображениями наставников
мариупольских школяров разных лет, об образах собственных и своих сверстников
учителей и учительниц, к сожалению, давно покинувших наш бренный мир.

\textbf{Читайте также:} 

\href{https://mrpl.city/news/view/v-mariupole-vybrali-samogo-luchshego-uchitelya-priazovya-foto-plusvideo}{%
В Мариуполе выбрали самого лучшего учителя Приазовья, Роман Катріч, mrpl.city, 05.10.2018}

Пришла на память старинная фотография из фондов Мариупольского краеведческого
музея, как бы мы сейчас сказали, педагогического коллектива мариупольской
Мариинской женской гимназии. На ней изображены семнадцать учительниц и
восемнадцатая – начальница гимназии \textbf{Александра Александровна Генглез},
приглашенная \textbf{Ф. А. Хартахаем} учительница Санкт-Петербургской Петровской гимназии
для организации подобной ей учебного заведения в нашем городе. Как они одеты?
Их платья самых разнообразных фасонов, но что общее - юбки до пола и глухой
верх, заканчивающийся стоячим воротничком. Прически у всех, за исключением
Александры Александровны, пышные - по моде начала ХХ века...

% 1 - Учительницы женской Мариинской гимназии. Фото из фондов Мариупольского краеведческого музея
\ii{06_10_2018.stz.news.ua.mrpl_city.1.dress_kod_dlja_pedagogov.pic.1}

1 сентября 1945 года, полдень. Перед двухэтажным зданием школы (ул.
Митрополитская, 5) стоят четыре колонны взявшихся за руки по двое наголо
остриженных мальчишек. Через несколько минут их заведут в классные комнаты, и
они станут полноправными первоклассниками Мужской неполной средней школы №3,
попросту – семилетки. Перед этим же зданием утром тут же стояли такие же
колонны, но состоящие из девочек, только им предстояло стать ученицами не 3-й
школы, а Женской неполной средней школы №1. Никаких церемоний, речей, букетов
цветов. Время суровое, на Дальнем Востоке идет еще война с японцами. Люди,
собравшиеся у порога учебного заведения, ведь не знали, что завтра, 2 сентября
1945 года, на борту американского линкора \enquote{Миссури} в Токийском заливе
будет подписан акт капитуляции Японии, и кровавая мировая война закончится.

% 2 - Первоклассники 1945 года
\ii{06_10_2018.stz.news.ua.mrpl_city.1.dress_kod_dlja_pedagogov.pic.2}

Кто запомнился из первого школьного дня? Конечно же, учительница - \textbf{Анна
Дмитриевна Бабенко}. Она как перед глазами стоит. В легком голубом платье в
белый горошек, гладко зачесанные волосы, подобранные заколками, на губах нет и
признака помады. Был ли у нее другой летний наряд? Не зацепила память. В
холодное время года она носила фабричной вязки серо-голубую кофту-жакет. Иногда
Анна Дмитриевна приходила на работу в спецовке своего брата. Это теперь
понимаешь, что кофту нужно было стирать, что не всегда она высыхала вовремя, а
кроме нее в гардеробе нашей любимой наставницы ничего не было. И надо же было
случиться, что в тот день, когда нас – первоклашек  пришли фотографировать, на
ней была спецовка брата. Так и осталась эта фотография, как свидетельство
сурового времени. Очень сурового.

\textbf{Читайте также:} 

\href{https://mrpl.city/news/view/mariupolskij-uchitel-nagrazhden-prezidentom-ordenom-za-zaslugi}{%
Мариупольский учитель награжден президентом орденом \enquote{За заслуги}, Роман Катріч, mrpl.city, 05.10.2018}

С пятого класса уже появилось несколько учителей.  Математике нас учила \textbf{Анна
Александровна Горохова}. Из разговоров взрослых довелось слышать, что она
окончила так называемый педагогический класс Мариинской женской гимназии,
дававший право его выпускнице преподавать предмет, в котором она проявила
наибольшие успехи. Видимо, нашей учительнице  лучше всего давалась математика.
Анна Александровна была старой девой. Неизменным ее нарядом было строгое
коричневое платье без малейших признаков украшений. Злые языки поговаривали,
мол, как на нее надели в гимназии форменное платье, так она его до сих пор и
носит.

\textbf{Мария Филипповна Шатова} преподавала украинский язык и литературу и была нашей
классной руководительницей. Это была статная, слегка полноватая женщина с
благородными чертами лица. Как она одевалась? Особенно запомнилось серое  из
добротной шерсти платье, подчеркивающее ее фигуру. Были у нее и другие наряды.
Но что-то не припоминается она в классе в ярком цветастом наряде или, Боже
сохрани, в сарафане с голыми до плеча руками.

Предметом обожания школяров была учительница русского языка и литературы
\textbf{Валентина Спиридоновна Хонахбей}. Мы не знали тогда, что она, едва окончив
педагогический институт, попала на фронт и в школу пришла после демобилизации.
Стройная, изящная, с темными выразительными глазами, с волнами вьющихся волос.
Она носила черный шерстяной сарафан в сочетании с блузками, сшитыми из светлой
ткани в полоску, но чаще белые, украшенные жабо из узких оборок или небольшим
бантом из той же ткани, что и блузка. Валентина Спиридоновна была подтянута,
строга, необыкновенно аккуратна во всем, что и требовала от своих учеников.
Тетради по русскому языку должны были быть обернуты в белую бумагу, на их
лицевой стороне наносился синим карандашом круг, в котором писались фамилия и
имя ученика, а также обозначение класса. Она добивалась, чтобы изложение, -
письменная работа для проверки знаний по литературе, - предварялось эпиграфом,
планом и, конечно же, без грамматических ошибок.  Спасибо ей за то, что она
научила грамотно писать, научила излагать мысли на бумаге. 

Наставником по немецкому языку была \textbf{Инна Петровна Мичурина}. Как говорили, она
была замужем за большим начальником, и поэтому одевалась чуть лучше, чем ее
коллеги.  Она довольно часто меняла вязаные кофточки, правда, все они были
скромных расцветок. Единственно, что позволяла себе наша \enquote{немка} – это блузки
из пестрой ткани, стыдливо выглядывающие из кофточки.  Инна Петровна обладала
яркой красотой, черные волосы были расчесаны на прямой пробор, заплетены в две
косы и закреплены на голове в виде короны. Кожа на лице была матово-белая с
румянцем на щеках, большие черные печальные глаза довершали ее облик. Она была
добра и снисходительна, особенно не требовала знаний, но всегда застенчиво
просила тишины. Через пару лет после окончания семилетки мы узнали, что Инна
Петровна умерла. Умерла совсем молодой, ей не было и сорока лет. Она стала
жертвой туберкулеза легких. Вот откуда у нее был румянец.

Историк \textbf{Раиса Ивановна Ясирова} столь темпераментно излагала свой предмет, что
порой казалось, попадись под руку ей враг Александра Македонского или Юлия
Цезаря, она проткнула бы его насквозь длинной указкой, с которой не
расставалась.

Географичка \textbf{Татьяна Алексеевна}, к сожалению, забылась ее фамилия, сравнительно
молодая женщина, печальная, никогда не повышающая голоса, терпеливо излагала
свой предмет среди галдящего класса. 

Преподаватель биологических дисциплин и химии \textbf{Нина Дементьевна Счастная},
запомнившаяся тем, что безжалостно \enquote{лепила} двойки,  особенно по ботанике. И
дело было не столько в лени учеников, сколько в малом количестве учебников. Их
выдавали из расчета – одна книжка на три ученика. Правда, у нее можно было
заработать \enquote{пятерку}, нарисовав плакат с изображением какого-нибудь хвоща или
ламинарии. Как-то выветрилось из головы, во что одевались эти учительницы. Одно
можно сказать твердо, что юбки их были значительно ниже колен, что их блузы и
платья не имели глубоких декольте, что маникюр был не ярок, что на шее не было
медальонов, золотых цепочек и бус, и что-то не помнятся кольца на пальцах  рук.
Правда, нужно не забывать, что после окончания тяжелейшей войны прошло всего
каких-то шесть-семь лет. Какие там кольца?

\textbf{Читайте также:} 

\href{https://mrpl.city/news/view/deti-iz-okkupirovannogo-donbassa-smogut-obuchatsya-distantsionno}{Дети из оккупированного Донбасса смогут обучаться дистанционно, Олена Онєгіна, mrpl.city, 05.10.2018}

Ну, а педагоги-мужчины? Вид у них был более чем спартанский. Директор школы
\textbf{Иван Спиридонович Журавлев} обычно был одет в полувоенную гимнастерку
темно-коричневого цвета, подпоясанную кавказским ремешком. В особо
торжественных случаях он облачался  в видавший виды серый двубортный костюм в
редкую полоску с пришпиленным к нему орденом Ленина. Поговаривали, что эту
награду наш директор получил за выслугу лет.

\textbf{Петр Кондратьевич Новицкий}, физрук и военрук  - рыжий худощавый  мужчина,
комиссованный из армии, из-за тяжелых ранений, - донашивал фронтовые галифе и
гимнастерку. У него был зычный голос с хрипотцой, которым он отдавал четкие
команды, будто перед ним не мальчишки, а бойцы его роты на передовой.

\ii{06_10_2018.stz.news.ua.mrpl_city.1.dress_kod_dlja_pedagogov.pic.3}

На всю жизнь запечатлелся образ \textbf{Алексея Павловича Ракова}, который преподавал
физику, запомнился и его кремовый чесучовый пиджак с наружным нагрудным
карманом, из которого торчали авторучка и карандаш. Свой предмет он излагал
мастерски. Особенно интересны были уроки, на которых Алексей Павлович показывал
опыты, иллюстрирующие физические законы.

Бытовало ли столь пуританское отношение к внешнему виду учителей только в
Мужской неполной средней школе №3? Нет, конечно. Просматривая общие фотографии
классов других мариупольских школ,  замечаешь на учителях те же скромные наряды
и прически, лишенные каких-либо изысков. Что это -  осознание своего
предназначения педагога, результат воспитания, требование начальства или ...?
Нет ответа.

\textbf{Читайте также:} 

\href{https://mrpl.city/news/view/mariupolskie-uchitelya-otmetili-professionalnyj-prazdnik-vmeste-s-me-rom-foto}{Мариупольские учителя отметили профессиональный праздник вместе с мэром, Олена Онєгіна, mrpl.city, 04.10.2018}
