% vim: keymap=russian-jcukenwin
%%beginhead 
 
%%file 13_12_2020.news.ru.ukraina_ru.lubenskii_andrei.1.humanitarnaja_sfera_oruzhie
%%parent 13_12_2020
 
%%url https://ukraina.ru/interview/20201213/1029940040.html
 
%%author Лубенский, Андрей
%%author_id lubenskii_andrei
%%author_url 
 
%%tags volga_vasilii
%%title Василий Волга: Гуманитарная сфера — это оружие массового поражения, пора нам это понять
 
%%endhead 
 
\subsection{Василий Волга: Гуманитарная сфера — это оружие массового поражения, пора нам это понять}
\label{sec:13_12_2020.news.ru.ukraina_ru.lubenskii_andrei.1.humanitarnaja_sfera_oruzhie}
\Purl{https://ukraina.ru/interview/20201213/1029940040.html}
\ifcmt
	author_begin
   author_id lubenskii_andrei
	author_end
\fi

\index[rus]{Русь!Национальная Идея!Василий Волга, 13.12.2020}

\ifcmt
pic https://cdn1.img.ukraina.ru/images/102992/59/1029925943.jpg
\fi

\begin{leftbar}
	\begingroup
\large\em
\textbf{Россию и Украину стравливают те же силы, которые разрушили СССР. Не добившись
полной победы в 90-х годах прошлого века, они продолжают своё дело, и нам, если
мы хотим сохраниться как исторический народ, надо научиться им противостоять.
Об этом в интервью изданию Украина.ру сказал политик и общественный деятель
Василий Волга}
	\endgroup
\end{leftbar}

— Василий, сейчас в связи с очередной годовщиной распада СССР много споров: мог
ли Союз сохраниться и как бы мы в таком случае жили сегодня?

— Это очень странный вопрос. Мог ли «Титаник» не затонуть? Мог, если бы не
встретил айсберг. Мог ли голкипер не пропустить пенальти в последнем матче?
Мог, если бы прыгнул в другую сторону. Мог ли СССР не погибнуть? Мог, если бы
процессы, которые привели к его гибели, двигались бы в противоположном
направлении.

Если бы СССР не погиб, если бы он пережил кризис конца 1980-х годов, то мы бы
жили сегодня в самой передовой стране мира. И это, конечно, при том условии,
если бы мы вновь не наделали ворох ошибок.

\textbf{— В чём заключаются истинные причины распада СССР?}

— Истинная и главная причина всегда одна. Всё остальное — это уже не причины, а
следствия из этой единственной главной Первопричины.

Так вот, главной причиной распада СССР была утрата веры в возможность
построения коммунистического общества. То есть это когда верхушка государства
изображала эту веру и день в день талдычила о том, что «учение Маркса истинно,
потому что оно верно», положив на самом деле раздвижной телескоп и на Маркса, и
на его учение, и на объективную реальность; а народные массы, так те уже даже
ничего и не изображали, а только рассказывали друг другу анекдоты про
«кремлёвских старцев».

То есть это такое положение, которое уничтожило смысл самого СССР, ибо Союз
создавался как объединенная воля народов, направленная в «светлое будущее». А
когда над «светлым будущим» стали смеяться, когда даже наверху в него перестали
верить, но ничего другого взамен не предложили, то пропала и сама цель. Живые
же человеческие сообщества, без цели, их объединяющей, существовать не могут,
они неизбежно разрушаются. Доказано историей.

Так, например, Великая Российская Империя была рождена Великой Православной
Идеей «Москва — третий Рим», Великий Советский Союз был рожден идеей построения
Царствия Божия на земле — Коммунизмом. Причем даже коммунизм имел совершенно
русский окрас.

То есть я хочу сказать, что первопричиной крушения любой империи, в том числе и
СССР, есть потеря веры народов в Национальную Идею.

\textbf{— Почему не нашлось никого, кто смог бы защитить страну?}

— Отчего же? Были и ГКЧП, и Руцкой, и Хасбулатов, и расстрел Белого дома, но
тут к месту будет вспомнить Бисмарка: «На штыках можно прийти к власти, на
штыках усидеть нельзя». К тому же, когда на протяжении предыдущих лет Горбачев
со товарищи вели системную подрывную работу против СССР, продав его фактически
за рекламу пиццы, в массы, под видом идеи свободы, была внедрена жажда
разрушения.

То есть у Горбачева все получилось. Массы словно вырвались на эту свободу,
круша всё на своем пути, и наша страна сделалась лёгкой добычей американских
ястребов.

\textbf{— Как оцениваете последствия распада СССР для Украины и для России?}

— Последствия для наших стран оказались катастрофическими. Но по-разному мы с
ними справились. Россия на сегодня уже почти в четыре раза превысила уровень
ВВП 1991 года. Мы же, Украина, достигли на сегодня только уровня 63\% от выпуска
ВВП 1991 года. Для Украины распад СССР стал непоправимой трагедией.

\textbf{— Теперь нас — Украину и Россию — буквально стравливают, тащат во вражду. Кому
это выгодно и, главное, почему это получается?}

— Это делают те же, кто разрушил СССР, и, не добившись своего, полной победы в
1990-х, продолжают свое дело. Получается это дело у них в первую очередь
потому, что они все свои силы бросили в гуманитарную сферу, то есть в область
Веры, Идеи, Духовности. Мы же, Россия и русские на Украине, не смогли этому
противостоять, поскольку даже сегодня мы считаем гуманитарную сферу чем-то
второстепенным, само собой разумеющимся. Запад же относится к ней как к оружию
массового поражения, разработав и предложив сегодня Украине новую Национальную
Идею: «Украина превыше всего».

\textbf{— Возможно ли восстановление единства, в том или ином виде, наших народов в исторически обозримой перспективе?}

— Оно неизбежно!

\textbf{— Что для этого необходимо сделать?}

— Осмыслить и предложить новую Национальную Идею, которая должна быть не чем иным, как синергией двух предыдущих национальных идей.

{\bfseries 
— Кто же сможет разработать новую национальную идею и убедить людей в её правильности и спасительности?
}

— Такие силы есть. На них просто нужно обратить внимание.

