% vim: keymap=russian-jcukenwin
%%beginhead 
 
%%file 04_01_2019.stz.news.ua.lb.1.arhitekturnyj_atlas_dorevoljucijnogo_mariupolja
%%parent 04_01_2019
 
%%url https://lb.ua/culture/2019/01/04/415883_arhitekturniy_atlas.html
 
%%author_id stanislavskyj_ivan.mariupol.fotograf.zhurnalist,news.ua.lb
%%date 
 
%%tags 
%%title Архітектурний атлас дореволюційного Маріуполя
 
%%endhead 
 
\subsection{Архітектурний атлас дореволюційного Маріуполя}
\label{sec:04_01_2019.stz.news.ua.lb.1.arhitekturnyj_atlas_dorevoljucijnogo_mariupolja}
 
\Purl{https://lb.ua/culture/2019/01/04/415883_arhitekturniy_atlas.html}
\ifcmt
 author_begin
   author_id stanislavskyj_ivan.mariupol.fotograf.zhurnalist,news.ua.lb
 author_end
\fi

\begin{quote}
\bfseries
В массовом сознании за Мариуполем давно и железно закрепился образ флагмана
украинской металлургии. В последние годы к этому добавился еще более суровый
имидж переднего края боевых действий. Приазовский город часто упоминается в
контексте военных событий или промышленных показателей. Но где-то там, под
прахом веков и слоем заводской пыли, есть ещё один Мариуполь – старый город
шумных базарных площадей, крутых мощеных улочек и солидных купеческих
особняков. Он заслонен артефактами индустриализации в историческом,
географическом и ментальном смыслах.

В отличие от заводских труб, старый Мариуполь легко не заметить, потому что
архитектурных свидетельств существования этого города осталось не так уж и
много. Вернее, большая их часть находится в непрезентабельном состоянии, а то,
что можно с гордостью демонстрировать - скорее разрозненные фрагменты, по
которым трудно представить целое. Судьба старого Мариуполя такая же сложная и
исковерканная, как и судьба современного.

LB.ua продолжает серию архитектурных гидов по украинским городам.
\end{quote}

Мариуполь появился на картах в конце ХVІІІ века. В 1882 году к побережью
Азовского моря от станции Еленовка дотянулась ветка Екатерининской железной
дороги. А в 1889 заработал современный глубоководный порт. Эти два
инфраструктурных проекта послужили прологом к большому будущему. Они сделали
город привлекательным для иностранных инвесторов, а местному купечеству открыли
мировые рынки. Мариуполь стал расти, как на дрожжах. Задымили два
металлургических завода, открывались иностранные консульства и торговые
представительства, в лавках можно было купить товары со всей Европы. Вскоре
появились местные миллионеры. Интересно отметить, что историко-культурный центр
Мариуполя всю досоветскую эпоху совпадал со старыми укреплениями Кальмиусской
паланки.

Позже по идеологическим причинам связи с козацким прошлым были разрушены. От
этого центра в западном направлении исходили три главных городских улицы,
образовывая в плане трезубец. Именно в этом районе стали появляться самые
значимые архитектурные объекты. Похвастать согласованностью архитектурных
решений Мариуполь не может, так как строительство велось хаотично, в
соответствии со вкусами и желаниями каждого отдельного заказчика. Большинство
построек того периода выполнено в эклектичном стиле, но иногда можно четко
заметить влияние неоклассицизма, модерна и даже неоготики. Кирпичная застройка
рубежа ХІХ – ХХ веков это как раз та самая \enquote{мариупольская архитектурная
старина}, которая досталась в наследство современным горожанам.

К началу прошлого века многие \enquote{акулы} местного бизнеса уже могли себе позволить
масштабное капитальное строительство. Спустя сто лет частные особняки и
коммерческая недвижимость богатейшей части городской элиты составили
значительную часть архитектурного наследия Мариуполя.

% Дома Трегубова у Театрального сквера
\ii{04_01_2019.stz.news.ua.lb.1.arhitekturnyj_atlas_dorevoljucijnogo_mariupolja.1.doma_tregubova}

% Особняк британского вице-консула Вильяма Вальтона
\ii{04_01_2019.stz.news.ua.lb.1.arhitekturnyj_atlas_dorevoljucijnogo_mariupolja.2.osobnjak_valtona}

% Особняк Гиацинтова
\ii{04_01_2019.stz.news.ua.lb.1.arhitekturnyj_atlas_dorevoljucijnogo_mariupolja.3.osobnjak_giacintova}

% Особняк судовладельца Регира
\ii{04_01_2019.stz.news.ua.lb.1.arhitekturnyj_atlas_dorevoljucijnogo_mariupolja.4.osobnjak_sudovladelca_regira}

% Дом врача Гампера
\ii{04_01_2019.stz.news.ua.lb.1.arhitekturnyj_atlas_dorevoljucijnogo_mariupolja.5.dom_vracha_gampera}

% Гостиница «Континенталь»
\ii{04_01_2019.stz.news.ua.lb.1.arhitekturnyj_atlas_dorevoljucijnogo_mariupolja.6.gostinica_kontinental}

% Кинотеатр «Победа»
\ii{04_01_2019.stz.news.ua.lb.1.arhitekturnyj_atlas_dorevoljucijnogo_mariupolja.7.kinoteatr_pobeda}

% Дом у Городского сада
\ii{04_01_2019.stz.news.ua.lb.1.arhitekturnyj_atlas_dorevoljucijnogo_mariupolja.8.dom_u_gorodskogo_sada}

% Дом со львами
\ii{04_01_2019.stz.news.ua.lb.1.arhitekturnyj_atlas_dorevoljucijnogo_mariupolja.9.dom_so_lvami}

% Духовное училище
\ii{04_01_2019.stz.news.ua.lb.1.arhitekturnyj_atlas_dorevoljucijnogo_mariupolja.10.duhovnoje_uchilische}

% Александровская гимназия
\ii{04_01_2019.stz.news.ua.lb.1.arhitekturnyj_atlas_dorevoljucijnogo_mariupolja.11.aleksandr_gimnazia}

% Реальное училище
\ii{04_01_2019.stz.news.ua.lb.1.arhitekturnyj_atlas_dorevoljucijnogo_mariupolja.12.realnoje_uchilische}

% Епархиальное училище
\ii{04_01_2019.stz.news.ua.lb.1.arhitekturnyj_atlas_dorevoljucijnogo_mariupolja.13.eparhialnoje_uchilische}

% Управление завода «Русский Провиданс»
\ii{04_01_2019.stz.news.ua.lb.1.arhitekturnyj_atlas_dorevoljucijnogo_mariupolja.14.upravlenia_zavoda_russkij_providans}

% Казенный винный склад
\ii{04_01_2019.stz.news.ua.lb.1.arhitekturnyj_atlas_dorevoljucijnogo_mariupolja.15.kazennyj_vinnyj_sklad}

% Водонапорная башня
\ii{04_01_2019.stz.news.ua.lb.1.arhitekturnyj_atlas_dorevoljucijnogo_mariupolja.16.vodonapornaja_bashnja}

% Большая хоральная синагога
\ii{04_01_2019.stz.news.ua.lb.1.arhitekturnyj_atlas_dorevoljucijnogo_mariupolja.17.bolshaja_horalnaja_sinagoga}

***

Нельзя сказать, что мариупольская общественность равнодушна к судьбе
архитектурного достояния. Местные краеведы уделили достаточно много времени
изучению истории старых зданий. Также в городе существует актив небезразличных
граждан. Но пока что обстоятельства сильнее. Общая тенденция печальна – старый
Мариуполь постепенно исчезает.

В качестве показательного примера можно привести дом дореволюционного
градоначальника Попова, который тихонько разваливался у центрального сквера на
глазах всего города целых пять лет. Немногочисленные прецеденты восстановления
исторических сооружений видятся скорее случайностями, чем стратегией. Всего
пять зданий в Мариуполе имеют статус архитектурной памятки местного значения, и
только три относятся к рассматриваемому нами периоду. Но благодаря бюрократии и
эти объекты могут лишиться своей защиты. Кроме отсутствия официального статуса
существует ещё один фактор способствующий упадку исторического центра Мариуполя
– это география.

Практически вся архитектурная старина находится в зоне накрываемой выбросами
металлургического завода \enquote{Азовсталь}. Из исторического центра, под давлением
экологической обстановки переселяется всё больше людей, бросая древние дома на
произвол. Кое-где старый Мариуполь напоминает город-призрак. Район становится
всё менее привлекательным как для жилья, так и для любой хозяйственной
деятельности. А значит, историческая энтропия неизбежно разложит многие
архитектурные памятки на элементарные частицы. Наступит время и останется
только смотреть на фотографии, и вздыхать о потерянной красоте Мариуполя.

Іван Станіславський, журналіст

%\ifcmt
  %ig https://i2.paste.pics/81a7dae62fd6266f17ecbb6f00ff0f18.png
  %@wrap center
  %@width 0.9
\fi
