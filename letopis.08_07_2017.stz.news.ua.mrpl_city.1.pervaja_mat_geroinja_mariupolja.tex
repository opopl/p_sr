% vim: keymap=russian-jcukenwin
%%beginhead 
 
%%file 08_07_2017.stz.news.ua.mrpl_city.1.pervaja_mat_geroinja_mariupolja
%%parent 08_07_2017
 
%%url https://mrpl.city/blogs/view/odna-iz-pervyh-materej-geroin-mariupolya
 
%%author_id burov_sergij.mariupol,news.ua.mrpl_city
%%date 
 
%%tags 
%%title Первая "Мать-героиня" Мариуполя
 
%%endhead 
 
\subsection{Первая \enquote{Мать-героиня} Мариуполя}
\label{sec:08_07_2017.stz.news.ua.mrpl_city.1.pervaja_mat_geroinja_mariupolja}
 
\Purl{https://mrpl.city/blogs/view/odna-iz-pervyh-materej-geroin-mariupolya}
\ifcmt
 author_begin
   author_id burov_sergij.mariupol,news.ua.mrpl_city
 author_end
\fi

\ii{08_07_2017.stz.news.ua.mrpl_city.1.pervaja_mat_geroinja_mariupolja.pic.1}

Наталия Антоновна Гах первой из жительниц Мариуполя была удостоена высокого
звания \enquote{Мать-героиня}. Такое звание было учреждено Указом Президиума Верховного
Совета СССР 8 июля 1944 года и присваивалось многодетным матерям, родившим и
воспитавшим десять и более детей. Грамота о присвоении почетного звания
Наталии Антоновне была подписана в Кремле 6 декабря 1945 года. На момент этого
радостного для дружной семьи события все дети Наталии Антоновны и Леонтия
Аввакумовича были живы-здоровы. Кроме самой старшей дочери – Варвары, ушедшей
из жизни еще в 40-м году из-за несчастного случая.

Николай Леонтьевич Гах - предпоследний ребенок у своих родителей. Он-то и взял
на себя миссию сохранить для будущих поколений историю своей семьи и передал
Мариупольскому краеведческому музею реликвии своей мамы - грамоту Президиума
Верховного Совета СССР и орден \enquote{Мать-героиня}, а также фотографии родителей и
всех братьев и сестер с биографическими сведениями о каждом из них.

Наталия Антоновна родилась в 1893 году в селе Никольском. Ее мать рано умерла,
мачеха оказалась женщиной очень жестокой, и Наталии довелось хлебнуть немало
горя. Поэтому, когда молодой парень Леонтий Гах из села Благовещенка – этот
населенный пункт сейчас принадлежит  Бильмакскому району Запорожской области -
предложил ей руку и сердце, она без раздумий пошла с ним под венец. Обвенчали
молодых в 1910 году, а на следующий год появилась на свет божий их дочь – Варя.
Следующей родившейся дочке дали имя Анастасия.

С началом Первой мировой войны в 1914 году Леонтий Аввакумович был призван в
действующую армию. В одном из ожесточенных боев был тяжело ранен. Хирурги
решили ампутировать правую ногу, Леонтий наотрез отказался. С гноящейся раной
он отправился домой. В вагоне поезда вместе с ним ехала знахарка (теперь мы бы
ее назвали народной целительницей), которая взялась излечить раненого воина.
Свое обещание она выполнила. Правда, всю жизнь Леонтий Аввакумович прихрамывал,
но зато ходил на своих ногах.

Вернувшись в родное село, Леонтий Аввакумович занялся привычным для него делом
– стал крестьянствовать. Пошли дети. В 1916 году родился сын Иван, в 1917 году
– дочь Пелагея, в 1919-м – Евдокия, в 1923-м – Мария, в 1925-м – Василий, в
1928-м – Анна, в 1930-м – Андрей, в 1932-м – Раиса, в 1936-м – Николай и,
наконец, в 1937 году – Александр. Гражданская война, голодные годы - 1921-й,
1933 – 1934-е - вынудили семью четыре раза переезжать из Приазовья в
Волгоградскую область и обратно. Там, на Дону, были украинские поселки. Только
в 1944 году  Леонтий Аввакумович и  Наталия Антоновна с младшими детьми осели в
Мариуполе. Николай Леонтьевич, благодаря которому и удалось воссоздать краткую
историю семьи Гахов, вспоминал: \enquote{Отец работал разнорабочим, чаще всего был
объездчиком. Объезжал поля для сохранения урожая. Мама занималась домашним
хозяйством. Когда была возможность, работала на стрижке овец. Под Сталинградом,
где мы жили, был овцеводческий совхоз. Стрижка овец – изнурительная, пыльная и
грязная работа}.

Особые испытания пришлось пережить семье в годы Великой Отечественной войны.
Вскоре после ее начала отец многодетного семейства был мобилизован в трудовую
армию на танкоремонтный завод под Сталинградом. Немцы захватили завод, Леонтий
Аввакумович в числе других рабочих и служащих попал в плен. Ему удалось бежать
из лагеря военнопленных. Измученный голодом и холодом, он едва добрался до
дома, слег и несколько месяцев болел. Об этом времени  Николай Леонтьевич
рассказывал: \enquote{После тяжелых боев, которые были под Сталинградом, там, где мы
жили, началась голодовка. Немцы ограбили всех и все -  ничего не оставили для
пропитания. Вот почему родители вернулись с нами на Украину, здесь все-таки
корни и большинство родственников}.

Конечно, и Леонтий Аввакумович, и дети  были рады, когда пришло сообщение, что
их мама получит правительственную награду. Но со временем довелось в связи с
этим испытать и огорчение. Об этом лучше узнать от Николая Леонтьевича:
\enquote{Вручение наград состоялось в 1946 году в мариупольском горисполкоме. Маме
воспевались дифирамбы, обещали дом нам построить и прочее, прочее. Дело в том,
что к грамоте  и ордену \enquote{Мать-героиня} прилагалась орденская книжка, в которой
перечислялись льготы, положенные награжденной. В том числе, например,
бесплатный проезд в любой конец Советского Союза, туда и обратно. Но с приходом
к власти Никиты Сергеевича все льготы были отменены. Так все обещанные льготы
остались пустыми словами. Когда мама обратилась насчет оформления пенсии, ей
руководитель собеса сказал: \enquote{Какая пенсия? Вы только успевали детей рожать. 
У вас нет трудового стажа}.  Так пенсию маме и не дали}.

Как же сложилась судьба детей Наталии Антоновны и Леонтия Аввакумовича? Иван
Леонтьевич был призван в армию еще в 1936 году. С первого и до последнего дня
Великой Отечественной войны, за исключением тех из них, когда лечили его раны в
госпиталях, был на фронте. Войну майор Иван Гах закончил с семью ранениями и
орденом Красного Знамени, орденом Отечественной войны, двумя орденами Красной
Звезды, медалью \enquote{За отвагу}, медалью \enquote{За освобождение Праги}. Уже после войны
он был награжден за безупречную  воинскую службу и другими орденами и медалями.
С 1948 года семь лет служил на острове Сахалин. Окончил военно-политическую
академию имени Ленина. В 1955 году назначен заместителем начальника
Харьковского гарнизона по спецвойскам, с этой должности через одиннадцать лет
вышел в отставку в звании полковника.

Василий Леонтьевич. Попал на фронт в 1944 году. За отличное выполнение задания
по разведке ближних тылов противника с взятием \enquote{языка} сержант Гах награжден
орденом Славы III степени,  в бою под Ковелем тяжело ранен в ногу, с большой
потерей крови был доставлен в полевой госпиталь, где ему ампутировали ногу.
Став в девятнадцать лет инвалидом, он не впал в отчаяние. Получил среднее
образование в школе рабочей молодежи, окончил стоматологический институт,
прошел путь от рядового врача до главного стоматолога бассейнового
медобъединения. Несмотря на пережитое Василий Леонтьевич был жизнелюбом,
человеком жизнерадостным. В молодые годы он с удовольствием играл для друзей на
аккордеоне. Потому и был душой компании. Он был прекрасным врачом, и люди были
ему признательны за это.

Андрей Леонтьевич после окончания  Мариупольского металлургического техникума
был направлен в город Уральск. Затем окончил Ленинградский кораблестроительный
институт, стал инженером. И до последнего времени работал заместителем
директора завода им. Ворошилова по качеству.

Николай Леонтьевич с отличием окончил тот же техникум, что и его брат Андрей, с
\enquote{красным дипломом} - металлургический институт. В этом вузе работал
инженером-исследователем, на \enquote{Тяжмаше} был заместителем начальника одного из
цехов. Особая глава его биографии – длительная служебная командировка в Индию.
Там, будучи человеком деятельным,  он находил время петь в хоре, организованном
в городке советских специалистов, серьезно занимался фотографией, однажды даже
в объектив его фотоаппарата попала Индира Ганди.

Александр Леонтьевич – самый младший в семье. Он также окончил металлургический
техникум, но дальше учиться не захотел. Прикипело его сердце к автомобилям.
Начинал он свою водительскую карьеру в совхозе, продолжил ее на армейских
машинах, после демобилизации работал на городских автобусных маршрутах. Затем
пересел на автобусы, совершавшие междугородные рейсы, а со временем –
международные. Вот тогда-то он и повидал дороги и города Болгарии и Румынии,
Польши и Турции.

Вот какие слова довелось услышать от Николая Леонтьевича о его сестрах: \enquote{Сестры
посвятили свою жизнь нам - братьям, чтобы мы получили образование, поэтому они
работали в совхозе на разных работах, даже грузчиками, только Анна, ушедшая
совсем недавно из жизни, стала закройщицей}.

Итак, сестры Гах. У Евдокии Леонтьевны трудовой стаж начался в четырнадцать
лет. Повзрослев, она последовала примеру Паши Ангелиной, научилась управлять
трактором. В годы войны она активно участвовала в эвакуации сельхозтехники за
Волгу. В личной жизни ей не повезло. Она так и не вышла замуж, парни – ее
сверстники - сложили головы на войне. По той же причине осталась незамужней и
Мария Леонтьевна.

Наталия Антоновна не умела ни писать, ни читать. Она была сиротой, и некому
было отвести ее в школу, а потом вся жизнь была посвящена детям. Правда, она
обладала даром с молниеносной скоростью в уме складывать, вычитать, делить и
умножать многозначные цифры. И вместе с тем, четверо ее сыновей получили высшее
образование. А уж среди ее внуков есть и те, кто увенчан учеными званиями. Сын
Ивана Леонтьевича Геннадий – доктор физико-математических наук, сын Анны
Леонтьевны  Владимир – кандидат биологических наук, сын Николая Леонтьевича –
доктор физико-математических наук.

Сегодня из двенадцати детей Наталии Антоновны и Леонтия Аввакумовича в живых
остался только один  - Николай. Да, время берет свое.
