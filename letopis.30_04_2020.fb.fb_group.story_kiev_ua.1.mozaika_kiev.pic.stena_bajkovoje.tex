% vim: keymap=russian-jcukenwin
%%beginhead 
 
%%file 30_04_2020.fb.fb_group.story_kiev_ua.1.mozaika_kiev.pic.stena_bajkovoje
%%parent 30_04_2020.fb.fb_group.story_kiev_ua.1.mozaika_kiev
 
%%url 
 
%%author_id 
%%date 
 
%%tags 
%%title 
 
%%endhead 

\begin{fminipage}{0.9\textwidth}

\ifcmt
  ig https://scontent-frt3-2.xx.fbcdn.net/v/t1.6435-9/95101677_3174559155911003_5629330394106036224_n.jpg?_nc_cat=101&ccb=1-5&_nc_sid=b9115d&_nc_ohc=4EY2-hiM8hYAX8JdvVf&_nc_ht=scontent-frt3-2.xx&oh=c46b49b12f566edcfcb8e2d649e69052&oe=61B5118A
	@width 1.0
\fi

\iusr{Марина Подлинева}

Оооо! А я каждый раз смотрю на эту стену и думаю - странно, то ли делали - не
доделали, то ли что-то пытались замазать.... Хорошо, что пытаются восстановить!
Стена, мягко выражаясь, грустная... Ну и что, что кладбище, но ходят туда живые
люди. На Байковом нашли покой столько известных выдающихся людей. Можно было бы
рассказать о них на этой стене.

\iusr{Ирина Петрова}

Це було надзвичайно сильне по емоціях панно, наші читачі розмістили фото - сама
його наживо не бачила, тільки чула тоді цю сумну історію, це жахливе варварство
невігласів, на превеликий жаль....

\end{fminipage}
