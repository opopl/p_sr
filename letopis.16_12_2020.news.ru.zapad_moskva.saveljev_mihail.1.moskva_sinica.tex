% vim: keymap=russian-jcukenwin
%%beginhead 
 
%%file 16_12_2020.news.ru.zapad_moskva.saveljev_mihail.1.moskva_sinica
%%parent 16_12_2020
 
%%url https://na-zapade-mos.ru/1031270-samaja-moskovskaja-ptica-sinica
 
%%author Савельев, Михаил
%%author_id saveljev_mihail
%%author_url 
 
%%tags moskva,ptica,sinica
%%title Самая московская птица – синица
 
%%endhead 
 
\subsection{Самая московская птица – синица}
\label{sec:16_12_2020.news.ru.zapad_moskva.saveljev_mihail.1.moskva_sinica}
\Purl{https://na-zapade-mos.ru/1031270-samaja-moskovskaja-ptica-sinica}
\ifcmt
	author_begin
   author_id saveljev_mihail
	author_end
\fi

\ifcmt
  pic https://na-zapade-mos.ru/files/data/user/AiF/olga.k/files/2020/2020.12.15-1608065685.4988_bolshaja-sinica.jpg
  caption Большая синица
  width 0.5
  fig_env wrapfigure
\fi

В числе коротающих с нами время до весны синица – самая распространённая и всем
известная зимующая птица Москвы. Она особенно облюбовала заказник «Воробьёвы
горы». Часто можно её встретить в Москворецком природно-историческом парке, в
Тропарёвском ландшафтном заказнике, в «Долине реки Сетунь». Впрочем, когда
прижимают холода, синицы начинают собираться в стайки, чтобы было легче
пережить зиму, и поближе подлетают к домам, привлекают их и устроенные на
балконах кормушки – так что сейчас время, наиболее удобное для наблюдений. Не
каждый знает, что синица – это только «фамилия», «имён», то есть её видов,
обитающих на природных территориях Москвы, – шесть, и большинство из них
занесены в Красную книгу столицы.

\subsubsection{Большая синица}

Большая синица действительно самая крупная из своего рода. У неё жёлтая грудка
с чёрным «галстуком», который значительно шире у самцов. Весной и летом
питается насекомыми, а осенью и зимой переходит на растительный корм.

\subsubsection{Хохлатая синица, или гренадёрка}

\ifcmt
pic https://na-zapade-mos.ru/files/data/user/AiF/olga.k/files/2020/2020.12.15-1608065685.7038_xoxlataja-sinica-mixail-ezdokov.jpeg
\fi

Очень большая редкость для московских парков, названа так из-за хорошо
заметного хохолка на голове, похожего на шапку гренадёра. Перья на спине у
птицы однотонные, буровато-серые, иногда с лёгким оливковым либо рыжеватым
оттенком, а на животе желтовато-белые, с охристым оттенком на боках.

\begin{leftbar}
	\begingroup
		\em\bfseries\large *Шкала категорий Красной книги для оценки состояния видов: 0 – вероятно
				исчезнувшие; 1 – под угрозой исчезновения; 2 – сокращающиеся в
				численности; 3 – редкие.
	\endgroup
\end{leftbar}

\subsubsection{Московка}

\ifcmt
pic https://na-zapade-mos.ru/files/data/user/AiF/olga.k/files/2020/2020.12.15-1608065685.7019_moskovka.jpg
\fi

Самая московская по названию птица напоминает большую синицу, но имеет более
ледяную окраску и плотное телосложение. Специалисты считают, что шапочка, или
маска, дала птице первоначальное название «масковка», которое впоследствии было
изменено и приобрело «столичный» оттенок. В Красной книге Москвы – под 2-й
категорией.

\subsubsection{Обыкновенная лазоревка}

\ifcmt
pic https://na-zapade-mos.ru/files/data/user/AiF/olga.k/files/2020/2020.12.15-1608065685.5848_lazorevka.jpg
\fi

Мельче большой синицы, опознать её можно по характерной особенности: на голове
лазурная шапочка. Лазоревки нередко летают в одной стайке с другими синицами,
вместе защищаются от врагов и добывают пропитание. Ещё один вид – белая
лазоревка, или князёк – в Красной книге Москвы под 1-й категорией.

\subsubsection{Буроголовая гаичка, или пухляк}

\ifcmt
pic https://na-zapade-mos.ru/files/data/user/AiF/olga.k/files/2020/2020.12.15-1608065685.5009_gaichka.jpg
\fi

Своё второе название эта синица получила из-за манеры сильно распушать оперение
в ненастную погоду. Да и вообще из-за плотного телосложения и короткой шеи
птичка похожа на пушистый шарик. Шапочка у пухляка матово-чёрная, а спина,
плечи, крылья, поясница и надхвостье буровато-серые. Занесена в Красную книгу
Москвы под 2-й категорией.

\subsubsection{Длиннохвостая синица, или ополовник}

\ifcmt
pic https://na-zapade-mos.ru/files/data/user/AiF/olga.k/files/2020/2020.12.15-1608065685.7027_opolovnik.jpg
\fi

Эта мелкая птица названа так из-за очень длинного хвоста. Второе название дано
ей действительно в честь столового прибора, поскольку форма тела птички весьма
его напоминает. Окрас головы, шеи и живота у птицы белый, спинка и крылья
чёрные с розовыми и коричневыми пестринами, а хвост и бока нежно-розовые. Имеет
3-ю категорию редкости в Красной книге Москвы.

\subsubsection{Водоплавающие соседи}

Сосчитать, сколько всего в Москве синиц, задача, конечно, непростая. В ходе
общеевропейского учёта птиц (EuroBirdwatch), проходившего в январе этого года,
в Москве экологи и добровольцы-москвичи всего насчитали примерно 4000 пернатых
38 видов. А вот водоплавающих, не улетающих на зимовку в тёплые края, сосчитать
легче. В парке «Москворецкий», например, сообщили экологи, охотно зимуют
утки-кряквы: примерно 450 особей – в Строгинском затоне, 100 – у Строгинского
мыса со стороны Москвы и 200 – у причала Троице-Лыково. В природном заказнике
«Долина реки Сетунь» учёт в январе 2020-го прошёл на участке реки от МКАД до
Рябиновой улицы. Здесь отметили 252 кряквы: 156 самцов, 96 самок. Также учёт
провели на реке Очаковке у Большого Очаковского пруда. Там зимует 171 кряква:
91 самец и 80 самок.

\ifcmt
  pic https://na-zapade-mos.ru/files/data/user/AiF/olga.k/files/2020/2020.12.15-1608065744.699_dsc-6078.jpg
  width 0.5
	cpx Фото: Эдуард Кудрявицкий
  fig_env wrapfigure
\fi

Кстати,численность и места обитания зимующих в столице водоплавающих пернатых –
это один из индикаторов экологической обстановки. Зимой 2019-2020 годов был
установлен рекорд – на зиму у нас остались 9750 водоплавающих. Следующий
подсчёт – в январе 2021-го.

\subsubsection{Чем покормить?}

Подкармливать синиц и других зимующих птиц нужно не раньше, чем установятся
значительные минусовые температуры и постоянный снежный покров. Для синичек
подойдёт несолёное сало, просо, цельный овёс, семена льна и ячменя, подсушенные
крошки белого хлеба. Всем птицам вредны любые жареные, солёные, перчёные,
копчёные продукты, ржаной хлеб, сдоба, свежий белый хлеб, пшено, фрукты,
миндаль.

