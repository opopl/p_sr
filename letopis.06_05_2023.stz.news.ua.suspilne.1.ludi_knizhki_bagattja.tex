% vim: keymap=russian-jcukenwin
%%beginhead 
 
%%file 06_05_2023.stz.news.ua.suspilne.1.ludi_knizhki_bagattja
%%parent 06_05_2023
 
%%url https://suspilne.media/465458-ludi-knizkami-rozpaluvali-bagatta-istoria-vtrat-ta-vidnovlenna-mariupolskoi-biblioteki-aka-stala-bomboshovisem
 
%%author_id news.ua.suspilne,ukolov_vladislav
%%date 
 
%%tags 
%%title "Люди книжками розпалювали багаття". Маріупольської бібліотеки ім. Короленка відновлюється в евакуації
 
%%endhead 
 
\subsection{\enquote{Люди книжками розпалювали багаття}. Маріупольської бібліотеки ім. Короленка відновлюється в евакуації}
\label{sec:06_05_2023.stz.news.ua.suspilne.1.ludi_knizhki_bagattja}
 
\Purl{https://suspilne.media/465458-ludi-knizkami-rozpaluvali-bagatta-istoria-vtrat-ta-vidnovlenna-mariupolskoi-biblioteki-aka-stala-bomboshovisem}
\ifcmt
 author_begin
   author_id news.ua.suspilne,ukolov_vladislav
 author_end
\fi

\ifcmt
  %ig https://cdn4.suspilne.media/images/resize/1040x1.78/5879b9fc1292b84a.jpg
	ig https://i2.paste.pics/57ae501361a32b94f26a8e31f6f49904.png
  @caption Вікторія Лісогор — директорка Маріупольської центральної бібліотеки ім. Короленка. . Фото: Суспільне Донбас
  @wrap center
  @width 0.8
\fi

\begin{qqquote}
Маріупольська центральна публічна бібліотека ім. Короленка — це саме серце
міста, його центральна частина. Поблизу драмтеатр, музей Куінджі. Зведений на
початку минулого століття будинок, що на розі вулиці Грецької та проспекту
Миру, вистояв Другу світову, але не повномасштабне вторгнення РФ: на нещодавно
оновлених зображеннях Google Maps — добре видно, що будівля потрощена. У перші
тижні війни підвали бібліотеки стали прихистком для десятків людей. Директорка
Вікторія Лісогор провела в стінах рідної книгозбірні двадцять днів війни.
Глибокі підземні сховища, міцні стіни, броньовані двері — все це створювало
певний час відчуття відносної безпеки, згадує маріуполька. \enquote{Я не думала, що
доведеться покинути бібліотеку}, — каже вона. Покинула місто поспіхом 16
березня. Шлях до Запоріжжя був складний і небезпечний. Нині відновлює
бібліотеку у Дніпрі і збирає книжки, з якими планує повернутись у рідне місто.

Яким був початок повномасштабної війни у Маріуполі, як вдалося виїхати з міста
та пройти блокпости \enquote{ДНР} та як працює бібліотека в евакуації — директорка
Маріупольскої центральної публічної бібліотеки ім. Короленка Вікторія Лісогор
розповіла в інтерв'ю Суспільному. Далі — пряма мова. 
	
\end{qqquote}

\subsubsection{24 лютого. Четвер. Санітарний день}

Напередодні 24 лютого 2022 року якісь розмови ходили, у місті, та я знаю, й в
області, й Україні було неспокійно. Але до останнього ми не вірили, що таке
може трапитись. Бо Маріупольці такі люди, які знають, що це таке: ми живемо в
такій ситуації з 2014 року.

24 числа я почула рано вранці вибух, вскочила — чоловіка поруч немає. Я виходжу
на кухню, а він там п'є каву, курить. Кажу: що ти сидиш, щось бахкає. Він каже:
перший раз чи що, може, знову в Широкіне.

Я відкриваю \enquote{Фейсбук}, включаю телевізор, а там вже і Дніпро, і по всій Україні
— кругом. Почали мені дзвонити мої співробітники з Талаківки, Сартани, там було
дуже гучно, це підступи до Маріуполя. Кажуть, ми не можемо добратись, ми не
виходимо вже з підвалів, що ми будемо робити.

\ii{06_05_2023.stz.news.ua.suspilne.1.ludi_knizhki_bagattja.pic.1}

Я як зараз пам'ятаю, це четвер — у нас був санітарний день. Я сказала: сидіть
всі по домівках, будемо щось вирішувати далі. В чаті усім співробітникам
написала, щоб всі залишились вдома, сама зібралась й пішла на роботу. До роботи
мені 15 хвилин, і моє керівництво також сказало, що зараз буде, щось
вирішувати, бо ніхто не очікував цього.

Прийшовши на роботу, я побачила, що співробітники центральної бібліотеки також
прийшли на роботу, бо люди не можуть залишатись вдома: ніхто нічого не розуміє
— що робити, роботи, звісно, ніякої не було: ходили всі читали якісь
Телеграм-канали, якісь новини. Дівчата, у кого хлопці були в лавах ЗСУ, якусь
інформацію нам передавали.

Я жила на сьомому поверсі, тому було якось страшнувато. Мої вікна всі виходили
на Азовсталь. А так вже дуже-дуже гучно було в передмісті Маріуполя. І от
прийшла співробітниця, каже: вибачте, але можна я прийду сюди з донькою, бо я
дуже боюсь, я живу на 8 поверсі. Я кажу, а я на 7. Вона каже, так може ми тут
будемо.

\ii{06_05_2023.stz.news.ua.suspilne.1.ludi_knizhki_bagattja.pic.2}

\subsubsection{\enquote{Думала, що я буду до останнього в бібліотеці}}

Будівля бібліотеки була така на той час, що я вважала, що вона вистоїть. Вона
вистояла 1941-1945 роки й окупацію Маріуполя. Там були такі дуже гарні
підвальні приміщення, броньовані двері, броньовані стіни. І ми пішли додому,
взяли щось там поїсти, переодягнутися й прийшли на роботу, я з донькою, вона з
донькою й сторож, який прийшов в нічну зміну.

Так ми і жили десь тиждень, поки ще був зв'язок. Ходили додому, дивились свої
квартири. Я сказала нікому не виходити на роботу. Час так швидко минав, що ми
нічого не розуміли, ми тільки бачили: ось води вже немає, ось черги величезні в
магазинах. В магазинах полки змітали всі, під банкоматами стояли такі черги, що
просто неймовірно. 

\ii{06_05_2023.stz.news.ua.suspilne.1.ludi_knizhki_bagattja.pic.3}

А десь 5-го березня до нас постукали — поліція. Сказали, дівчата, ми будемо
сюди переїжджати, бо тут у вас більш менш безпечне приміщення, бо там де вони
знаходились не було підвальних приміщень. А вони виїжджали на виклики, ще якось
контролювали ситуацію.

Воєнний стан — я нічого не змогла, хоча було страшнувато. Все ж таки ми —
цивільні, а тут — військові. Але нікуди дітись, вони переїхали. На ранок
привезли свої сім'ї, ми розташувалися усі у підвальних приміщеннях, де були у
нас книгосховища. Там, де не було вікон, ми всі знімали книжки, опускали
стелажі, складали туди книжки, якось там розташовувались. Десь 50 нас чоловік в
цих приміщеннях було.

Виходити з підвалів вже було дуже-дуже страшно. 7 березня вже було так гучно й
в центрі, що до нас прибігли рано вранці, десь о сьомій, люди, які жили біля
будівлі бібліотеки по проспекту Мира, 45-43. Сказали, що в їхні підвали не
вміщаються — люди біжать з інших районів, заскакують в підвал, де тільки можна.
І вже дехто не пускає до себе. Тому питали, чи можна у нас розташовуватися.

Я відкрила все, що в мене було, всі приміщення. Десь ще ховала книжки, якусь
техніку зберігала, занесла до себе в кабінет й пускала туди людей. Так ми і
жили усі разом десь два тижні. Ну як жили, пережидали так скажемо.

Зв'язку не було вже десь після 4-5-го березня. Але у хлопців були рації, й
завдяки цьому ми ще якось знали, де що робиться у місті. Знали, де бомбануло й
чий будинок вже розвалився. Коли вони виїжджали й повертались ввечері, в них не
було лиця, бо вони збирали ще поранених людей, вони збирали трупи, ще й звозили
там по госпіталях. Вони бачили весь цей жах. Ми — не бачили, бо не могли вийти
з підвалів.

\ii{06_05_2023.stz.news.ua.suspilne.1.ludi_knizhki_bagattja.pic.4}

Десь, щось ще працювало, але чим далі, тим було складніше й складніше. Ми вже
чули оці літаки, ці бомбардування. Все літало над нашими головами. Ночі ці були
безсонні.

\begin{qqquote}
От воно летить й летить, летить й летить, а ти лежиш укутаний, бо мороз -12, а
воно бабахкає, а воно стріляє — і лежиш думаєш: \enquote{Господи, яка ж тебе мати
народила}. Оце я точно пам'ятаю, як я про це все думала. \enquote{Яка ж тебе мати
народила?!} — оце питання в мене досі. Яких там матері народжують?	
\end{qqquote}

Потім люди почали дізнаватись, що можна виїжджати (бо офіційних коридорів не
було). 14 березня прибігла одна з жінок й каже: можна виїжджати на узбережжя,
через Мангуш, через Портовське. У них була маленька дитина, й її чоловік каже:
ну, що мені треба рятувати сім'ю. Поїхав — і не повернувся, ми не знали, що з
ним. Лише 15 рано вранці він повернувся й сказав, що так дійсно можна виїхати
на Мелекіне, на Ялту, на Урзуф. Каже, що покаже дорогу тим, у кого були машини.
У деяких людей вони у нас у дворі стояли. Ми у дворі бібліотеки мали гаражі, й
ховали там машини, зокрема, і поліцейські. Тоді хто міг, той посідав на машини,
і поїхав. І вже з 15 березня нас все менше й менше ставало.

Бібліотека була буквально за 500 метрів від драмтеатру. І дівчата пішли
вигулювати собачку й побачили, що біля драмтеатру була церква й там прямо під
куполом люди ловлять зв'язок. Вони туди залізли, спіймали зв'язок, попередили
наших чоловіків, що ми живі, де ми знаходимось. Але в той момент у мене не було
думки, що я буду кудись виїжджати. Я думала, що я буду до останнього в
бібліотеці, що тут надійно, що тут все добре, двері броньовані.

Я підіймалася на 2-3 поверх, там вже не було ні вікон, ні дверей. Але в нашому
бункері було надійно. І тут дівчина прибігла і каже: \enquote{Сказав ваш негайно
збиратись, їхати. Сьогодні буде машина}.

Збиратись так збиратись, я розумію, що треба рятувати доньку. Тим паче, що ми
вже знали, що \enquote{ДеНеРи} вже недалеко від центру знаходяться, і мікрорайоні
Кірова, недалеко від мого будинку, а це вже і до нас близько. Але чекали-чекали
ми машина, а так вона й не приїхала. А вранці в мене раптом думка, що я повинна
піти додому і хоч якісь речі з собою взяти. І ось, коли тільки-тільки почався
світанок, ми з дочкою пішли додому.

\begin{qqquote}
А я 10 діб не виходила з бібліотеки. І коли ми вийшли, я побачила весь жах
оцього Маріуполя. Це був не Маріуполь, це було мертве місто. Там валялось все,
що можна, від дротів до будинків.	
\end{qqquote}

\ii{06_05_2023.stz.news.ua.suspilne.1.ludi_knizhki_bagattja.pic.5}

Все свистіло, летіло, але я дійшла до дому. Знаєте, якась така думка була:
потрібно попрощатися. Я піднялась до квартири, закрила вибиті вікна, рами
позакривала, бо в нас там вже багато чого повилітало. Взяли з дитиною якісь
речі, й так під обстрілами повернулись на роботу до бібліотеки. Й буквально за
пів години приїхала машина. Коли я виїжджала з бібліотеки, там залишалось
всього восьмеро людей.

\subsubsection{Мангуш. Хліб, ковбаса \enquote{Макєєвська} та сльози}

Сіли в машину. Нас було п'ятеро: я з донькою, моя подруга з донькою й ще один
хлопець. Нам водій одразу сказав: ляжте на підлогу й не висовуйтесь зовсім. Ну,
як же не висовуватись. Ми дивились у вікна — це був жах: дуже багато було
загиблих людей, вони просто лежали на вулиці, їх ніхто не збирав, бо дівати
було нікуди. Коли ми виїхали за Маріуполь, він тоді сказав: все, видихайте, вже
більш менш безпека. Це було Портовське, дорога на Мангуш.

Ми виїхали з бібліотеки о пів на дванадцяту, і до Мангуша ми потрапили о 8-й
годині вечора, бо там була колона. що просто все тихенько, тихенько, все було в
полях це все ми стояли довго й довго. Й там в полях пробився перший зв'язок. Я
тоді змогла зв'язатись з чоловіком, з мамою, яка залишилась там під Маріуполем.
Дізнались, що живі.

\ii{06_05_2023.stz.news.ua.suspilne.1.ludi_knizhki_bagattja.pic.6}

У Мангуші ми зустріли друга чоловіка, він нас поселив. І ми помились перший раз
за ці всі дні. Були трохи здалеку вибухи, але ми знали, що сюди вже нічого не
прилетить, бо тут — вони вже окупували Мангуш. Це перша ніч, яку ми більш менш
спокійно спали, хоча все чули, знали, що бомблять вже Азовсталь.

Як раз в цей день, 16 березня, коли ми виїхали — прилетіло у драмтеатр. Дві
доби ми провели в Мангуші. І от вранці ми із подругою пішли щось купити поїсти,
і ми пройшли по Мангушу, побачили всіх цих "денерівців" з цими білими
пов’язками, ці черги за хлібом. Нічого не схотілось, прийшли з нею мовчки,
лягли й 2-й годині і просто мовчали, бо це був, якийсь такий, якась така
ситуація, що ми не вірили, що це все з нами проходить. Знайшли там у хазяїв,
щось там перекусили, так пройшов день.

Телефонуємо чоловікам: що нам далі. Кажуть, ми шукаємо, як вас забрати. Бо тоді
ще не розуміли, як їхати, це тільки-тільки початок тиках виїздів був, ще не
було фільтрації.
