% vim: keymap=russian-jcukenwin
%%beginhead 
 
%%file slova.materj
%%parent slova
 
%%url 
 
%%author 
%%author_id 
%%author_url 
 
%%tags 
%%title 
 
%%endhead 
\chapter{Матерь}
\label{sec:slova.materj}

%%%cit
%%%cit_head
%%%cit_pic
%%%cit_text
Майор.
Очень долго думал, стоит об этом писать или нет. Потому что есть темы, где
удобней не топтаться. Но все же напишу, потому что это касается всех нас.
На днях на Донбассе погиб украинский военный. Майор Богдан. И так вышло, что
это сын одной из лучших подруг моей \emph{мамы} тети Оли. Она его всегда называла
Бодя. И все мое подростковое детство в Гайсине прошло рядом с Богданом. Не то,
чтобы мы с ним дружили, но дружили \emph{наши мамы}. А значит все дни рождения,
дежурные походы в гости – это все проходило у нас совместно.
Когда я закончил школу – я старше Богдана на 4 года – и уехал из Гайсина, с ним
мы практически больше не виделись. Помню, Богдан как-то даже стал заочной
причиной скандала между мной и \emph{мамой}. Уже студентом приехав на пару дней в
Гайсин, я увидел, что не хватает каких-то моих машинок и солдатиков.
Оказывается \emph{мама} дала их поиграть (с возвратом и он, действительно, потом их
вернул) Богдану. А я орал, что это моя частная собственность и вообще, какого
хера
%%%cit_comment
%%%cit_title
\citTitle{Наши ребята гибнут там, где у власти нет ни стратегической, ни тактической цели}, 
Игорь Лесев, strana.ua, 08.07.2021
%%%endcit

%%%cit
%%%cit_head
%%%cit_pic
\ifcmt
  pic https://strana.ua/img/forall/u/0/36/2021-07-27_12h58_59.png
  width 0.4
\fi
%%%cit_text
Причем он рассказал, что упрекал в русскоязычности и собственную \emph{мать}.
"Один раз в садике, где \emph{мама} разговаривала на русском языке, а наши друзья из
садика разговаривали на украинском, я сказал: "\emph{Мама}, как тебе не стыдно, ты
разговариваешь на русском, а с тобой разговаривают на украинском. И с тех пор я
начал \emph{маму} перевоспитывать на украинском. \emph{Мама} сначала тоже не понимала, почему
нельзя разговаривать на русском и любить Украину", – вспомнил Михаил
%%%cit_comment
%%%cit_title
\citTitle{Песни на русском – дерьмо. Как на канале Порошенко с ребенком о национализме говорили}, 
Анна Копытько, strana.ua, 27.07.2021
%%%endcit

