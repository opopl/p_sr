% vim: keymap=russian-jcukenwin
%%beginhead 
 
%%file 10_05_2023.tg.kiev.funtime.1.dvorik_s_voronami
%%parent 10_05_2023
 
%%url https://funtime.kiev.ua/pogulyat-v-gorode/dvorik-s-voronami
 
%%author_id kiev.funtime
%%date 
 
%%tags kyiv.voron_krum
%%title Дворик с воронами
 
%%endhead 
 
\subsection{Дворик с воронами}
\label{sec:10_05_2023.tg.kiev.funtime.1.dvorik_s_voronami}
 
\Purl{https://funtime.kiev.ua/pogulyat-v-gorode/dvorik-s-voronami}
\ifcmt
 author_begin
   author_id kiev.funtime
 author_end
\fi

В столице Украине имеется масса удивительных, красивых, интересных мест, о
некоторых из которых не знают даже местные жители. Подобные локации,
относительно Печерской Лавры, Площади Независимости, Золотых ворот, нового
пешеходного моста и других значимых объектов, не столь популярны, однако
вызывают неподдельный интерес, особенно среди туристов и гостей города.
Поскольку место находится практически в самом центре Киева, после посещения
дворика с воронами можно отправиться на прогулку по историческим районам,
побывать на площади Независимости, сфотографироваться у фонтанов, посетить
многочисленные музеи, театры и т.д.

\textbf{Описание дворика с воронами} 

О существовании дворика с воронами знают немногие. Его популярность в последнее
время больше обусловлена упоминанием о нем в социальных сетях. Посмотреть на
удивительных птиц, которые охотно гоняют местных котов, забравшихся в клетку с
целью полакомиться колбасой, готовы присоединиться к распитию пива, могут
имитировать звуки автомобильной сигнализации, доводить до бешенства местных
жителей, передразнивая их, съезжаются не только дети, но и взрослые. В
некотором роде дворик стал еще одной достопримечательностью столицы, которой, к
сожалению, пока еще нет в официальных путеводителях.

Виновниками столь пристального внимания являются трое воронов по имени Корбин,
Карлуша и Кирилл. Небезразличные люди в свое время подобрали птенцов и выходили
в огромной клетке, расположенной посреди двора жилого малоэтажного дома. На
протяжении двух десятков лет сюда прибывали новые обитатели и улетали
повзрослевшие, окрепшие и восстановившиеся вороны. Уход за птицами выполняют
местные жители, небольшая инициативная группа. В кормлении дополнительно
принимают участие гости, получающие взамен лакомств возможность увидеть таланты
обитателей дворика.

Каждый ворон имеет свою судьбу, которая зачастую складывалась трагически. В
частности Карлуха был привезен сюда охотниками, которые нашли его абсолютно
беспомощными в лесу. Вероятно, птица была ранена или детьми, или недалекими
людьми. Изначально \enquote{опекун} Корбина поселил ворона на дереве и привязал
к нему 50 метровую веревку, чтобы тот случайно не заблудился. Обычно, во время
очередных полетов, местные голуби в панике разлетаются в стороны, что вызывает
определенный восторг у местных жителей и случайных зевак. Ворон охотно идет на
живое общение с людьми.

Годом позже, во дворе появился еще один птенец, которого назвали Карлуха. Он
был принесен киевлянами, которые приметили его на асфальте, предположив, что
тот выпал из гнезда. Именно в этот момент появилась идея строительства огромной
клетки для двух птиц, которая успешно была воплощена в жизнь. Спустя еще пару
лет к дуэту присоединился Кирилл. Его небезразличные люди принесли практически
умирающего из местного зоопарка. После восстановления воронов, их новый опекун
побоялся выпускать птиц на волю, полагая, что они не смогут самостоятельно
выжить в дикой природе. 

История дворика характеризуется и чередой трагических событий. В частности
ранее в клетке обитал еще четвертый ворон. Из хулиганских побуждений, в 2010
году, злоумышленники повредили клетку и выпустили из нее птиц. Поскольку
пернатые привыкли к своему постоянному месту обитания, они вскоре вернулись
назад. А вот найти четвертого ворона так и не удалось. Его дальнейшая судьба, к
сожалению не известна. Регулярно добрые люди приходят сюда, приносят птенцов
нуждающихся в реабилитации, помощи и постоянном уходе. Каждому их них стараются
найти свое место.

К великому разочарованию, уже в скором времени столица может потерять столь
необычную достопримечательность. Дело в том, что земля, участок площадью в 24
сотки, на которой находится дворик с воронами и еще три отдельно стоящих
здания, выставлена на продажу за 1 000 000 долларов США. Вероятно, здесь
появится или новый многоэтажный дом, или развлекательный центр. О судьбе же
пернатых думают только волонтеры. Ведутся переговоры о переносе клетки в другое
место, однако к конкретному решению до сих пор не пришли. Пока же у воронов
есть дом, небезразличные люди, которые за ними ухаживают, кормят и не забывают.

\textbf{Чем заняться?} 

В некотором роде дворик с воронами является местом постоянного паломничества
киевлян и гостей столицы. Сюда приходят только лишь для того, чтобы воочию
увидеть огромных птиц, покормить, посмотреть на их таланты и получить от этого
огромное удовольствие. Регулярно локацию посещают экскурсионные группы,
самостоятельные туристы, простые зеваки и случайные прохожие. Рядом имеется
музей \enquote{Золотые ворота}, музей \enquote{Медуз}, кинотеатры, фонтаны,
торговые центры, скверы, парковые зоны, видовые площадки, ухоженные аллеи и
много других интересных мест. В каждом их них обязательно захочется побывать.

\textbf{Как добраться} 

Дворик с воронами находится в центральном районе Киева, на улице Рейтарская, 9,
всего в 5 минутах ходьбы от станции метро \enquote{Золотые ворота} зеленой ветки.
Добраться сюда можно относительно просто на общественном транспорте, автобусах
и троллейбусах, регулярно \enquote{приходящим} по улице Владимирской и Софиевской. В
шаговой доступности находится бульвар Шевченко и Крещатик. Практически с любой
точки столицы доехать сюда реально не более чем за полчаса. Относительно
недалеко расположен центральный железнодорожный вокзал, куда пребывают составы
с пригорода и соседних областей.

\textbf{Итог}

Побывать в дворике с воронами должен каждый турист или гость Киева. Место
достаточно интересное и необыкновенное в принципе. Аналогов ему в нашей стране
не существует. Трое воронов всегда с удовольствием встречают посетителей,
позируют под фотокамеры и охотно идут на любое общение. Они очень любят
лакомства, за которые готовы показывать свои таланты. Необходимо помнить, что,
хоть птички и смотрятся безобидными созданиями, они имеют достаточно мощную
хватку клюва. В 2015 году на одной из стен близлежащего дома известный художник
– Александр Бритцевый создал уникальный мурал, на котором изображены черные
вороны. Свое творение он назвал \enquote{Вестником жизни}.

\textbf{Видео двора с воронами на Рейтарской}

\textbf{Галерея изображений}

Галерея фотографий места для прогулки \enquote{Дворик с воронами}

\ii{10_05_2023.tg.kiev.funtime.1.dvorik_s_voronami.pic.1.galereja}
