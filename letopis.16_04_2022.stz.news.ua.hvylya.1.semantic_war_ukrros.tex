% vim: keymap=russian-jcukenwin
%%beginhead 
 
%%file 16_04_2022.stz.news.ua.hvylya.1.semantic_war_ukrros
%%parent 16_04_2022
 
%%url https://hvylya.net/analytics/250892-semanticheskaya-voyna-ukrainy-i-rossii
 
%%author_id berger_dmitrij,news.ua.hvylya
%%date 
 
%%tags 
%%title Семантическая война Украины и России
 
%%endhead 
 
\subsection{Семантическая война Украины и России}
\label{sec:16_04_2022.stz.news.ua.hvylya.1.semantic_war_ukrros}
 
\Purl{https://hvylya.net/analytics/250892-semanticheskaya-voyna-ukrainy-i-rossii}
\ifcmt
 author_begin
   author_id berger_dmitrij,news.ua.hvylya
 author_end
\fi

\begin{zznagolos}
Мышление американца скорее юридическое, а украинское — семантическое. Уже в
этом определении заложена проблема семантики, о которой мы и поговорим.	
\end{zznagolos}

\ifcmt
  ig https://hvylya.net/crops/6a11f9/834x0/1/0/2022/04/16/RzWyE2kGdV8yQGoOQswGkutj2zrQ3aJ53UYKinwn.jpeg
	@caption Война Украины и России. Фото: Depositphotos.com
  @wrap center
  @width 0.8
\fi

По моим ненаучным наблюдениям мышление американца скорее юридическое, а
украинское — семантическое. Уже в этом определении заложена проблема семантики,
о которой мы и поговорим.

Например, президент Байден не назвал преступления российских оккупантов в
Украине сейчас геноцидом. Он подчеркнул, что по его личному мнению это геноцид,
но окончательное заключение он оставляет специалистам по международному праву.

Специалисты, кстати, тоже бываю разные. Большинство людей, называющими себя
аналитиками и экспертами, как правило всего лишь комментаторы, выражающие свою
субъективную точку зрения. Что хорошо и нужно, а иногда даже приносит им славу
и деньги, дай им бог здоровья, но нужно понимать, что как ни интересно они бы
не излагали, это не делает их мнение чем-то более, чем личное мнение.

И неизбежно, что оказавшись посреди урагана войны, даже самый замечательный
эксперт, охваченный эмоциями, начнёт терять былую объективность. А вот это уже
плохо. Когда мои друзья, образованные и обычно сдержанные люди, вдруг призывают
к совершению преступлений, оправдывая это тем, что по отношению к подлому врагу
любые средства сойдут, их удельный вес в обществе приводит не только к потерям
имиджа страны, но вполне может вести к настоящим преступлениям. А они были.

Дело не в том, что и как мы называем. Мы можем что угодно называть как угодно,
тут никаких ограничений нет. Проблема в том, что сказав слово, люди решают, что
то значение, которое они вкладывают в него лично, как правило эмоционально в
конкретном контексте, обязательно должно также восприниматься другими, без
эмоций и в другом контексте.

Нам свойственно употреблять термины, которые для нас несут определенную
эмоциональную составляющую. Это естественно, мы же не можем заниматься
этимологией каждого слова, мы же не лингвисты-филологи. Мы, например, знаем,
что слово геноцид — негативное, означающее для нас массовые убийства, и в таком
общем значении его используем. И удивляемся, почему другие этого не принимают.
Возможно потому, что в политическом контексте слово «геноцид» имеет другие,
более детальные определения, а в юридическом потребуется намного больше
конкретных деталей взаимосвязи многих элементов событий и личностей с набором
неопровержимых документированных доказательств конкретных слов и действий. При
этом все будут правы, потому что все будут говорить о разных вещах, используя
одно и тоже слово. К тому же, геноцид, преступления против человечности и
военные преступления совершенно разные вещи с юридической точки зрения. Но из
неё исходит обычно принятие политических решений, а не из нашего хотения.

По сообщениям, точнее одному сообщению, российский беспилотник в Мариуполе
сбросил нечто, похожее на отравляющее вещество, от которого три человека
почувствовали себя плохо. Можно в этом увидеть доказанное начало химической
войны и требовать немедленной, желательно военной, реакции, мирового
сообщества. А можно увидеть одно сообщение от некоторого источника о
случившемся или неслучившемся происшествии, где может было, а может и не было
использовано химическое оружие. Для нас такой информации может и достаточно, а
для других — нет. Потому что мы все вкладываем свои значения в слова.

Наверное, разница между специалистом и дилетантом в том, что специалист
значения в слова не вкладывает, а изучает и осознаёт их изначальное значение.
Что позволяет не обманывать себя, по крайней мере.

Наблюдая за обвинениями ООН в том и сём, приходится удивляться тому, что люди
считают ООН эдаким мировым правительством, которое оперирует некими законами,
которые все обязаны выполнять. И страшно расстраиваются, когда ты им
объясняешь, что ООН — всего лишь форум, где, по идее, у каждого члена
организации есть возможность высказаться и быть услышанным. Всё. По большему
счету ничего больше ООН, как организация делать не может. У неё нет собственных
вооруженных сил, она полагается на армии её членов.

Члены в ООН тоже ещё те, и пришлось придумать право вето для больших держав.
Хотя бы потому, что абсолютное большинство стран в мире не являются нормальными
демократиями и на международной арене будут кучковаться по религиозным,
геополитическим или просто коррупционным признакам. Без права вето главной
силой в ООН была бы Саудовская Аравия, возможно, которой вполне по силам
сплотить и скупить сколько нужно голосов.

ООН была создана, чтобы не допустить войны. Но когда война или геноцид пошёл,
ООН мало что в состоянии сделать. Особенно с Россией в Совете Безопасности.
Убрать из него Россию? Можно, но тогда и ООН можно закрывать. Можно обойтись и
без всемирного общего форума, чего нет, но кому от этого станет лучше? Уберем
возможность общения как такового и разбредёмся по своим углам? Какая в этом
выгода и кому я не знаю.

Упрекать других в том, что они исходят из своих представлений, возможностей и
интересов возможно только тогда, когда ты сам предъявляешь себе такие же
претензии. Обвинять Германию в том, что она систематично и добровольно посадила
себя на российскую ресурсную иглу, можно и нужно. Только стоит добавлять, что
тем же занималась и Украина, хотя Россия угрожала не Германии.

Можно обижаться до бесконечности, что Украине помогают только потому, что она
не сдалась, а сражается. Возможно, что если бы она сражалась в 2014, ей бы
стали помогать тогда? Какой смысл помогать оружием сдающемуся?

Какой смысл копаться в истории, которую ты даже не особо и знаешь, в культуре,
о которой у тебя довольно туманные представления, или в вопросах языкознания, в
которых ты ничего не понимаешь, если у тебя в стране конкретный враг,
намеренный её уничтожить, как таковую? Как в своё время Сталин, узнав о
заявлениях римского папы, поинтересовался язвительно: «А сколько у римского
папы танковых дивизий?» Ибо войны выигрывают дивизии, а не мерения у кого
древнее предки.

Называйте агрессоров как вам угодно: орками, ордой, мокшами, рашистами,
фашистами или нацистами. Главное помните, что это всё ваши слова, в которые вы
вкладываете свои личные эмоции, которые выражают ваше состояние сегодня и никак
не отражают объективную реальность в которой есть Российская Федерация,
российская армия и российский президент, фамилию забыл. Это то, с чем
приходится иметь дело в действительности, конкретным оружием сбивать конкретные
самолёты, сжигать конкретные танки, топить конкретные корабли, обращаться к
конкретным парламентам и на конкретных форумах проталкивать свою конкретную
повестку. Безотносительно к Андрею Боголюбскому и Валуевскому циркуляру.

Лепить обидные ярлычки на противника, конечно, забавно, но это больше похоже на
чесание собственного эго или даже комплекс неполноценности. Гораздо полезнее
топить корабли противника. Даже не посылая их на х@й!
