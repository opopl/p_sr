% vim: keymap=russian-jcukenwin
%%beginhead 
 
%%file 06_05_2022.stz.pc.ua.dou.1.litachok.3.wiki
%%parent 06_05_2022.stz.pc.ua.dou.1.litachok
 
%%url 
 
%%author_id 
%%date 
 
%%tags 
%%title 
 
%%endhead 

\subsubsection{Додаток - стаття з Вікіпедії, версія 16.05.2022}
\label{sec:06_05_2022.stz.pc.ua.dou.1.litachok.3.wiki}

\url{https://uk.wikipedia.org/wiki/PD-2}

\ii{06_05_2022.stz.pc.ua.dou.1.litachok.pic.3.wiki}

PD-2

Матеріал з Вікіпедії — вільної енциклопедії.

\textbf{UKRSPECSYSTEMS PD-2} — український багатоцільовий БпЛА. Призначений для
здійснення повітряної розвідки та бойового застосування як носій бомбового
навантаження з бойовою частиною вагою до 3 кг[1]. Може злітати зі злітної смуги
(«по-літаковому»), також може оснащуватися модулями системи вертикального
зльоту та посадки, що робить його апаратом вертикального злету та приземлення
(англ. VTOL — \emph{Vertical Take-Off and Landing}).

\ii{06_05_2022.stz.pc.ua.dou.1.litachok.pic.4.wiki}

\paragraph{Історія розробки}

У 2014 році почалася російська збройна агресія проти України. Бойові дії
показали потребу в безпілотних апаратах, перш за все для розвідки. Волонтери
закупляли цивільні апарати, які або використовували як є, або адаптували до
потреб війни[2].

\ii{06_05_2022.stz.pc.ua.dou.1.litachok.pic.5.wiki}

Волонтерська організація «Народний проект» брала участь у допомозі українським
військовим на сході України. У спробах імпорту сучасних БпЛА, організація
зіткнулася з небажанням деяких іноземних компаній продавати свої апарати. Тоді
у співпраці з компанією «UkrSpecSystems» було розроблено безпілотний авіаційний
комплекс PD-1, який станом 2020 рік випускався для потреб Збройних сил
України[джерело?].

У 2020 році компанія «UkrSpecSystems» розробила безпілотник PD-2, який є
глибокою модернізацією PD-1 зі збільшенням корисного навантаження, дальності
зв'язку та іншими вдосконаленнями[1].

У листопаді 2020 року новий літальний апарат проходив визначальні
випробування[3]. 

\paragraph{Опис комплексу}

Наземна станція управління, що входить до складу комплексу, може забезпечувати
одночасне управління двома БПЛА. Літаки комплексу оснащені оптичною станцією з
гіростабілізованою платформою, розробленою та створеною в Україні. Підвіс
оснащений кількома камерами та тепловізором, що дозволяє проводити
спостереження на висоті до кількох тисяч метрів у будь який час доби. Вбудовані
датчики положення та спеціальне програмне забезпечення дозволяють забезпечити
чіткість та якість зображення[1]. 

\paragraph{Опис апарату}

PD-2 побудований за нормальною аеродінамічною схемою — одномоторний високоплан
з триопорним шасі з переднім колесом, шасі в польоті не прибирається. Крило
оснащене закінцівками. Хвостове оперення двобалкове, обернене V-подібне. Силова
установка зі штовхаючим гвинтом, який обертається двигуном внутрішнього
згоряння.

Літак комплектується швидкозйомними модулями з додатковими електричними
двигунами (системою вертикального зльоту та посадки — СВЗП), які дозволяють
БПЛА здійснювати вертикальний зліт та посадку[1]. У такій комплектації він стає
апаратом вертикального зльоту та посадки.

Відповідно до класифікації НАТО (STANAG 4670 (ATP 3.3.7)) БпЛА належить до
\textbf{класу I} (≤150 кг), категорії \textbf{малий} (>15 кг), масштаб
застосування \textendash \textbf{тактична частина} (батальйон, полк)[4].

\paragraph{Технічні характеристики}

\textbf{Загальні характеристики}

\begin{itemize}
  \item Екіпаж: 4
  \item Вантажопідйомність: до 19 кг[5]
  \item Розмах: 5 m (16 ft 5 in) (у конфігурації з СВЗП)
  \item Максимальна злітна вага: 55 kg (121 lb) (у конфігурації з СВЗП)
  \item Повітряний гвинт: 2-лопастевий
\end{itemize}

\textbf{Льотні характеристики}

\begin{itemize}
  \item Максимальна швидкість: 140 км/год (87 миль/год; 76 kn)
  \item Крейсерська швидкість: 100 км/год (62 миль/год; 54 kn)
  \item Швидкість звалювання: 65 км/год (40 миль/год; 35 kn)
  \item Дальність: 1 000 km (621 mi; 540 nmi)
  \item Тривалість польоту: 10 годин (без СВЗП)
  \item Практична стеля: 5 000 m (16 404 ft)
  \item Швидкопідйомність: 3 m/s (590 ft/min)
  \item Швидкість зниження: 3 m/s (590 ft/min)
  \item Інше: дальність зв'язку — 220 км[5]
\end{itemize}

\paragraph{Див. також}

\begin{itemize}
  \item RQ-7 Shadow
  \item PD-1
\end{itemize}

\paragraph{Примітки}

\begin{itemize} % {

\item 1. ukr.defense.news@gmail.com (08 листопада 2020). Оновлений БПЛА PD-2
оснастили сучасною гіростабілізованою платформою українського виробництва.
Defense Express. Архів оригіналу за 8 листопада 2020. Процитовано 6 травня
2022.

\item 2. Вячеслав Масний (21.02.2019). Українські військові дрони: від склянки з
гранатою до ударного хайтеку. Укрінформ. Архів оригіналу за 21 січня 2022.
Процитовано 6 травня 2022. «Як Україна розвиває власні військові безпілотники
та технології боротьби з ворожими»

\item 3. Люксіков Михайло (2 грудня 2020). Вітчизняний БПЛА PD-2 готується до
експлуатації в ЗСУ. Український мілітарний портал. Архів оригіналу за 3 грудня
2020. Процитовано 3 грудня 2020.

\item 4. Збірник наукових праць Державного науково-дослідного інституту авіації.
2016. Вип.№ 12(19)/ МЕДВЕДЄВ Г. А. — Аналіз досвіду НАТО щодо стандартизації в
галузі безпілотної авіації

\item 5. \href{https://www.unmannedsystemstechnology.com/company/ukrspecsystems/pd-2-uas-vtol-fixed-wing-uas/}{Ukrspecsystems. PD-2 UAS VTOL Fixed-Wing UAS. Unmanned Systems Technology}.
Процитовано 6 травня 2022.

\end{itemize} % }
