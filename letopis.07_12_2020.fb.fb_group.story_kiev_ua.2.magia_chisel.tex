% vim: keymap=russian-jcukenwin
%%beginhead 
 
%%file 07_12_2020.fb.fb_group.story_kiev_ua.2.magia_chisel
%%parent 07_12_2020
 
%%url https://www.facebook.com/groups/story.kiev.ua/posts/1533267773536685/
 
%%author_id fb_group.story_kiev_ua,petrova_irina.kiev
%%date 
 
%%tags 1977,chislo,kiev,magia,novyj_god
%%title Магия Чисел
 
%%endhead 
 
\subsection{Магия Чисел}
\label{sec:07_12_2020.fb.fb_group.story_kiev_ua.2.magia_chisel}
 
\Purl{https://www.facebook.com/groups/story.kiev.ua/posts/1533267773536685/}
\ifcmt
 author_begin
   author_id fb_group.story_kiev_ua,petrova_irina.kiev
 author_end
\fi

М А Г И Я Ч И С Е Л

Я верю в магию чисел.

Наступающий, 1977-й, с двумя счастливыми семёрками, должен, просто обязан был
принести что-то необыкновенное, неожиданное. Я думала об этом, убаюканная гулом
двигателей самолёта, возвращаясь из командировки 30 декабря.

Киев встречал меня метелью, морозцем, предновогодней суетой. Радостное
предпраздничное настроение несколько омрачилось на ёлочном базаре – весь
ассортимент составляли несколько «рыбьих скелетиков» с прилипшими желтыми
иголками.

\ifcmt
  ig https://scontent-frt3-1.xx.fbcdn.net/v/t1.6435-9/130310063_3856110331089212_2537657755165185086_n.jpg?_nc_cat=104&ccb=1-5&_nc_sid=825194&_nc_ohc=32RpZV4CZDYAX9oCTZY&_nc_ht=scontent-frt3-1.xx&oh=c73f7f87ca7965dd45b1d5d0af02ab96&oe=61B3B563
  @width 0.4
  %@wrap \parpic[r]
  @wrap \InsertBoxR{0}
\fi

Подруга, которой я пожаловалась на такую неудачу, недоуменно хмыкнула: «Лиз, а
зачем тебе ёлка? Ты ведь в ресторан идешь, а второго в Карпаты едешь. Приедешь
– сразу надо её разбирать. Не вижу смысла. Поставь в вазу лапу – да и делов!»
Правда, она еще не знала, что в Карпаты я не еду, умудрившись по телефону
поругаться с Юрием, наотрез отказаться от поездки и от всей прочей совместной
жизни. И как же это – Новый Год без запаха ёлки? Нет, нет, надо что-то
придумать. И тут меня озарило – ведь уже почти вечер, 31-е декабря, все уже
поставили и принарядили свои ёлочки. А может пройтись по соседним Заньковецкой,
Станиславского (какое театральное общество), если у кого-нибудь увижу ёлку на
балконе – значит она лишняя!

Я вышла в сиреневые сумерки декабря и медленно пошла, внимательно осматривая
балконы домов. Все куда-то бежали, спешили, суетились, из магазинов
высовывались хвосты очередей, кто-то там, внутри кричал: «Только по полкило в
руки!», вспотевшие и счастливые вываливались с шампанским и апельсинами в
авоськах и бежали дальше. А я медленно шла, разглядывая дома, со стороны,
видимо, смахивая на туристку. Мне не надо было ни полкило в руки, ни
шампанского, ни апельсинов – мне нужна была ёлка. Пройдя вверх по своей крутой
киевской улице, по двум другим, по небольшой площади, я уже начала сомневаться
в своей идее.

Но две счастливые семёрки – это не одна!

Свернув на соседнюю улочку, на балконе второго этажа старинного дома я увидела
небольшую сосенку. Вот она, удача! Вернее, половина удачи. Теперь надо убедить
хозяев продать мне деревце.

На двери нужной мне квартиры я увидела номер – 14. Опять две семёрки, мне
определенно начинало везти. Дверь широко распахнулась, и первое, что я увидела,
так же широко распахнутые глаза цвета летнего моря. Он был невысокого роста, в
смешном переднике с зайцем, и вопросительно глядел на меня. После секундного
онемения, я начала что-то невразумительно бормотать про командировку, ёлочные
скелеты и традиции детства, но тут из глубины квартиры донесся женский голос:
«Игорь, кто там?» У меня мгновенно испортилось настроение, я не видела еще ЭТУ
женщину, а уже ненавидела её. Но счастье было прочно заложено в счастливых
числах – он ответил: «Мама (!!!), это пришли за ёлкой!» Мама, мама, мама… Мне
захотелось просто расцеловать эту невидимую маму. Это еще не отрицало наличие и
жены, но как-то не так было горько. Естественно, маме стало любопытно, кто это
пришел за ёлкой и почему за ёлкой? Еще бы за колбасой пришли. Невысокая
темноволосая женщина в таком же смешном переднике с зайцем вышла к двери.
Очевидно, распахнутые глаза цвета морской волны передавались в этой семье по
наследству.

Представившись, я повторила свою неубедительную историю, продолжая заикаться от
подступишего волнения. Мама и сын рассмеялись одинаковым смехом, пригласили
меня войти, в комнате уже стояла нарядная ёлка. На спинке стула я увидела
китель лётчика с майорской звездочкой на погонах. Это был его китель, выправку
военного человека я сразу отметила. Как объяснила мне ЕГО мама, они с Игорем,
не сговариваясь, купили два деревца, но она - ёлку, а он принес сосну.
Поставили, конечно, ёлочку, и мне отдадут ненужную сосенку. Из ЕГО рук в ту
минуту я готова была принять даже кактус. Посмеялись, вручили мне находку,
обменялись наилучшими пожеланиями – и дверь закрылась.

Я шла, обнимая деревце, мне было так радостно, счастливо, что даже звонок Юрия
не расстроил меня. Он не сдал мой билет, просил прощения, убедил меня в
сказочности поездки, я сдалась. Мне было уже все равно, куда и с кем ехать – я
была счастлива.

Незаметно пролетел праздничный январь, февральские метели утихали, 23 февраля я
зашла в тот подъезд и в почтовый ящик 14-й квартиры бросила открытку. «С
праздником! Всего наилучшего. Л.» - вот и всё, что я написала в ней.

Мы помирились с Юрием, всё стало по-прежнему, но были теперь у меня в жизни
глаза цвета летнего моря. Это было только моё, я никому об этом не рассказала.
К метро я ходила теперь по новому маршруту, проложенному через соседнюю улочку.
Правда, это был небольшой крюк, но зато можно было смотреть на его балкон.

Утром 7 марта я сначала не поняла, почему к перилам этого балкона прикреплена
вазочка с ветками мимозы. Через пару шагов я остановилась, как вкопанная.
Конечно, это ведь Я знаю, где он живет. А своего адреса я ведь не называла, вот
и пришлось отвечать на моё поздравление таким образом. Я любила в тот день всё
человечество. И теперь цветы всегда появлялись на этом балкончике – нарциссы
апреля, охапки сирени в мае, летом это были пеоны и ромашки, осень встретила
меня астрами и хризантемами, а 30-го декабря на балконе стояла небольшая
сосенка. Все было ясно, меня ждали и звали, но, идти туда я не могла.

Гремучий воз прошлой жизни тащился за мной, частые поездки в командировки и
возвращения, прощания и встречи давали ненужную подпитку нашим отношениям с
Юрием, плюс моё безволие и инертность характера. Но, перед каждым праздником я
бросала в почтовый ящик 14-й квартиры открыточку с добрыми пожеланиями. Так я
показывала, что вижу цветы и благодарна за это.

Пролетел год, потом еще один. Цветы на балконе были всегда, эта традиция давала
мне силы и ощущения поддержки. Наступил год Олимпиады.

И вдруг цветов не стало. Май для меня превратился в ноябрь. Конечно, не может
же хозяин таких глаз дарить цветы незнакомке, ничего не получая взамен. Это
нормально, мужчина не должен жить какими-то сказочными мечтами. Всем нужна
семья. По привычке, я еще с полгода ходила по той улице, но, цветы на балконе
не появлялись. Что-то ушло из жизни, и теперь я окончательно порвала с Юрием.
Трудно объяснить логику этого поступка, но, без цветов на балконе 14 квартиры
мне ничего не было нужно.

В декабре на балконе не было ёлочки.

Февральским утром я ехала в метро, мой взгляд упал на газету в руках мужчины,
сидевшего рядом. Меня как током ударило – с газетной фотографии смотрели широко
распахнутые глаза, четверо лётчиков стояли у самолета, шлемы держали в руках. Я
успела заметить только название газеты , даже заголовок не прочла, мужчина
поднялся и вышел. Ехать мне было еще несколько остановок, но на следующей
станции я выскочила из вагона, через ступеньку бегом поднялась по эскалатору и
бросилась к газетному киоску. Лихорадочно развернула газету, увидела
фотографию, впилась в неё взглядом, несколько минут смотрела в это, ставшее
родным, лицо. Потом прочла заголовок «Прощайте, боевые товарищи!», и…очнулась
уже в машине «Скорой».

«Что ж ты, девонька, падаешь-то прям посерёд улицы?» - склонилась надо мной
пожилая медсестра. «Молода еще так-то обморочиться, но сердчишко хорошее, щас
укольчик поставим, полежишь, касатка, да и с Богом.»

Я выплывала из какого-то липкого омута, мне делали уколы, что-то лили в рот, но
всё в мире было сосредоточено на газете, которую я судорожно зажала в руке. «А
чего ж ты так в газетку-то вцепилась, милая? Уж ни в ней ли причина?» -
участливо ворковала медсестричка. Она мягко разжала мои пальцы, увидела статью
и вздохнула. «Да, сколько их, соколиков наших, в том проклятом Афганистане уж
полегло, что ж делается-то, прости нас Господи!»

Их самолет сбили над ущельем, его звали Горчаков Игорь Никандрович, последние
полтора года он выполнял интернациональный долг в горной стране, ненавидящей
нас и поныне.

На работе мне удалось выпросить у шефа длительную командировку, на полгода я
уехала из дома, чужие места, работа как-то отвлекали, но в сердце был ноющий
осколочек. И еще, я теперь не могла видеть море, его цвет был для меня
непереносим.

Прошло два года, жизнь равнодушно продолжалась. Работа, работа, работа была
моим занятием, любовью, спасением, отдушиной. В Новый год ёлку я не ставила, в
Карпаты поехала, жила вроде как по привычке.

И всё же я верю в магию чисел, а две семёрки – это не одна. Седьмого июля я шла
по улице, а навстречу невысокая женщина, с иссиня-седой головой и распахнутыми
глазами цвета морской волны, катила инвалидную коляску.

У моего свёкра красивое старинное имя – Никандр, мой пятилетний Игорь Игоревич
мечтает стать лётчиком, как папа. Свекровь рассказывала мне: «Лиза, первое, что
попросил Игорь, придя в себя после нескольких тяжелейших операций, чтобы я
поставила цветы на балконе. Я так и сделала. Он очень ждал твоих открыточек.»

На Новый Год у нас всегда есть вторая ёлочка на балконе.

Иллюстрация из архива автора.

Киев, 2004

\ii{07_12_2020.fb.fb_group.story_kiev_ua.2.magia_chisel.cmt}
