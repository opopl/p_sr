% vim: keymap=russian-jcukenwin
%%beginhead 
 
%%file poetry.rus.sokor.ato_oos_vojna
%%parent poetry.rus.sokor
 
%%url http://maysterni.com/publication.php?id=134438
%%author 
%%tags 
%%title 
 
%%endhead 

\subsubsection{АТО-ООС и одна война}
\label{sec:poetry.rus.sokor.ato_oos_vojna}
\Purl{http://maysterni.com/publication.php?id=134438}

Идет бой. Это война.
Разрыв снаряда, блеск огня.
Глаза закрыла пелина,
И сильна боль прожгла меня.
Но я опять рванулся в бой,
Друзья кричали мне:- Постой!.
Как будь-то я бегу тропой,
Что из школы шел домой.
Меня мама там встречала,
В школу из школы проводжала.
Иду сейчас я стороной,
Мама вскрикнула:- Сынок, постой!
И я вздрогнул, и я застыл,
А женский голос говорил:-
Ну что, солдат, пойдем со мной.
Но я слышу идет бой.
Двадцать третий мне пробил,
Четыре года в АТО жил.
Друзей своих я хоронил.,
Но ведь еще, я не любил.
Как же мне любить красу,
И гладить девичью косу.
Мои глаза покрыла мгла,
Свечу любви мне не зажгла.
Хочу коснуться ее руки,
Своей девчонки и любви.
И быть обятый красотой,
Любовью девичьей и теплотой.
Но надомной хлопочет врач,
Слышу рядом стон и плач.
Сквозь окон слышно - „пей до дна“.
-Что, окончилась война?
- Нет, сынок, то слуги власті,
Они, пируют каждого дня.
2013-2018г.
