% vim: keymap=russian-jcukenwin
%%beginhead 
 
%%file 24_11_2018.lj.ugryumova.1.stekljannyj_kljuch
%%parent 24_11_2018
 
%%url https://ugryumova.livejournal.com/249105.html
 
%%author_id 
%%date 
 
%%tags 
%%title ЗАНИМАТЕЛЬНАЯ БИБЛИОГРАФИЯ - СТЕКЛЯННЫЙ КЛЮЧ
 
%%endhead 
\subsection{ЗАНИМАТЕЛЬНАЯ БИБЛИОГРАФИЯ - СТЕКЛЯННЫЙ КЛЮЧ}
\label{sec:24_11_2018.lj.ugryumova.1.stekljannyj_kljuch}

\Purl{https://ugryumova.livejournal.com/249105.html}

\begin{zznagolos}
Бывают такие странные люди, влюбленные в автобусные остановки, в афишные тумбы,
в таблички с названиями улиц, в витрины, в скамейки, во всю эту чушь и чешую,
чешую города

Вяч. Курицын Операция на сердце
\end{zznagolos}

Первое, что бросилось мне в глаза в кабинете Роднянского, была раритетная афиша
«Касабланки». У стены стоял аквариум. Окна выходили на Институтскую, взбегающую
от Крещатика к Смольному; тогда еще старые деревья моего детства росли вдоль
всей улицы. А само здание, где размещалась администрация и
литературно-сценарный отдел телеканала «1+1», стояло на Крещатике, аккурат
посреди Козьего болотца, у перекрестка семи дорог, откуда как раз татары в
недалеком 1240 шли на последний приступ.

\ii{24_11_2018.lj.ugryumova.1.stekljannyj_kljuch.pic.1}

Всякий раз, когда я вижу, как медленный троллейбус ползет по довольно крутой
Михайловской по направлению к Софии, и знаю, как скоро становятся видны ее
мощные, но невысокие стены – я думаю, что так же неспешно и неотвратимо
поднимались по этому склону тумены Субудая и Джебэ-нойона, тянулись бесконечным
змеиным потоком к Золотым и Лядским воротам – и вот уже Киев пал. Как недавно,
в сущности, все это было. Просто наша жизнь еще короче, и проносится с
крейсерской скоростью. Как сказала однажды моя мудрая бабушка – «Смотри, смотри
внимательно и запоминай. Просвистит мимо – не заметишь». Мне было года четыре,
дни тянулись медленно-медленно, как стенобитные орудия монголов в гору, и
казалось, что все, что я вижу вокруг, будет стоять всегда. А осталось
неизменным только небо над Софией – оно тут всегда особенное, наверное,
какой-то оптический фокус из-за золотых куполов – но здесь это как бы лоскут,
вырезанный в небосводе, цвета чище и совершенно другие.

Собственно, речь на встрече как раз и шла о том, чтобы написать сценарий
мини-сериала, серии на четыре – и чтобы в рамках развлекательного жанра
показать тот, старый, любимый нами Киев. Со всеми его каштанами, павильонами,
беседками, фонтанами, в которых плавали когда-то золотые рыбки, с парками, по
которым носились ручные совершенно белки и важно вышагивали разговорчивые
скворцы. С симфоническими оркестрами и – главной достопримечательностью города
– теми старыми киевлянами, которые к тому моменту, когда происходил наш
разговор, стали такой же исчезающей натурой, как и множество киевских
памятников старины.

Вот только не помню – был ли еще виноград на нотном магазине или уже нет?
Скорее всего, еще был, вился во всей красе по всему фасаду, сообщая зданию
удивительную элегантность и привлекательность, которую оно тут же утратило,
лишившись своего уникального зеленого одеяния.

Я, наверное, потому все время пишу о чудесах, что значительная часть моей жизни
разыгрывается именно в жанре чуда. Я не просто верю, что чудеса бывают, я
доподлинно знаю, что они случаются постоянно.

Вот и «Стеклянный ключ» не исключение.

Как вы уже поняли, это сценарий. Случилось так, что после выхода в виде
газетного романа «Мужчин ее мечты», где то и дело упоминались киевские реалии,
руководитель одного из ведущих украинских телевизионных каналов «1+1», продюсер
и режиссер Александр Роднянский, предложил написать сценарий – и кто бы
отказался, и я впервые столкнулась с изнанкой телевизионной кухни.

Несмотря на кажущуюся схожесть, телевидение и газеты – совершенно разные миры.
Тут действуют абсолютно различные физические законы и случаются совершенно
непредсказуемые химические реакции. Так вышло, что сценарий был написан, но
снимать сериал не стали. Однако в договоре был предусмотрительно вставлен пункт
о праве автора на беллетризацию текста, и этим правом я нахально
воспользовалась.

Дело в том, что для меня этот сценарий был поводом рассказать об удивительных
людях и удивительном городе, который я видела в своем сказочном детстве. Так
что самые простые вещи как раз выдумка, а самые невероятные – как раз нет.
Потому что есть вещи, которые придумать невозможно. Их можно только наблюдать,
а после – записать, запечатлеть, оставить в памяти других людей хотя бы в таком
виде. Хоть как-то сохранить.

Теперь, во времена всеобщей мобильной связи, когда все как один пассажиры метро
делают то же, что и Джонни Мнемоник – водят руками по экранам с целью получить
нужную информацию, и это реальность, данная нам в ощущениях, а не
фантастическое кино – трудно объяснить, зачем бы это в приличном доме у
балконных дверей стояла кастрюля, в которой лежали огромные старинные ножницы с
привязанной к ним веревкой. Веревка уходила в неизвестность. Это было не так
давно, хоть и в прошлом веке и тысячелетии, и тоже называлось «телефон». Когда
Елизавета Алексеевна из соседнего парадного (она жила в прекрасном эркере
этажом ниже, чем мы)  хотела поговорить, она дергала за веревку. Ножницы
громыхали в кастрюле, мы выходили на балкон. Общение – в основном при помощи
жестов, отточенное годами практики – доставляло обеим сторонам невероятное
удовольствие. У нее же аппарат выглядел иначе. Под огромной пальмой-хаммеропсом
стоял высокий медный кувшин, привезенный из поездки по Самарканду в 1911 году.
К веревке был привязан серебряный половник с баронским гербом времен
крестоносцев. Не зря в медные колокола при литье добавляли толику серебра –
звук ее «телефона» был изумительно звонок и прекрасен.

Енот Поля, выведенный в романе под именем енота Поли, сидит в кресле вместе с
зеленым змеей Анютой (Анюта – мужеского полу, и бантик у него голубой),
медведем Хрумкиным и быком Бакандором (который стал и быком, и Бакандором за
несколько лет до того, как появилась на свет первая страница «Некромерона»). В
книге дан его точный словесный портрет.

Пара влюбленных – она с огромной роскошной косой, уложенной вокруг головы в
форме короны, он – незрячий, жили на Печерске. Я писала о них в рассказе
«Вечный город».

Беседку на Владимирской горке, знаменитую Кокоревскую беседку, в которой раньше
играл военный оркестр, под звуки которого чинно прогуливались под липами и
каштанами нарядно одетые пары, недавно, как написали в газетах, «сломали
вандалы», часть ее сдали на металлолом. Для японцев это не стало такой бы
трагедией — главное, считают они, память, а беседку можно восстановить в
прежнем виде; но я не японец — я всегда буду знать, что это уже совсем другая
беседка. А вот ее близнеца, беседки из Купеческого сада, где летом, среди роз и
фонтанов играл духовой оркестр — золотые, сверкающие на солнце инструменты,
белые мундиры — вообще давно не существует. Если с этим смириться окончательно,
если признать как данность, то ее и не станет.

Именно поэтому «Стеклянный ключ» — это не только «что», но и «где».

Люди часто пропускают описание пейзажей. Их можно понять. Пейзаж — это место
действия, а для описания места, думаем мы, хватит и одного-двух слов. Река,
лес; или, например, парк. Мы вообразим сами. Потому что у каждого свой
собственный лес и свой парк, как свои бабушкины пирожки и детские сказки. Но я
пишу, как прошу, не пропускайте, пожалуйста, этот абзац, прочтите. Чем больше
людей его прочтут, тем больше запомнят этот кусочек мира, и тем верней он
сохранится в чьей-то памяти, потому что другого способа сохранить его — уже
нет.

Старая аптека в Пассаже, в месте, с которым связано столько воспоминаний, что
их не уместишь даже в четырехсерийном сериале – или в романе всего-то на 576
страниц – описана до деталей такой, какой она когда-то была. Как жаль, что у
меня не осталось ни одной открытки, ни одной фотографии. Теперь там тоже
аптека, но та, прежняя – будто бы вышла из романа Рэя Бредбери «Вино из
одуванчиков». На ней еще лежал налет предыдущей эпохи. Конечно, так, слабая
тень, даже тень тени, но что-то еще оставалось от тех аптек, где ели мороженое,
пили сельтерскую или воду «Фиалка» (я еще застала ее – сиреневая душистая вода,
от которой колет в носу), покупали газеты, где стояли по полкам белые с синим
фаянсовые сосуды и ступки с пестиками; а под стеклом лежали тюбики самой
вкусной детской апельсиновой зубной пасты «Зоопарк», брикетики «Гематогена» и
разные занимательные штучки. Главным же в той аптеке действительно был аквариум
– старый, громоздкий, в железной раме, огромный такой. И в нем плавали пухлые
золотые рыбки. Думаю, именно невероятная нежность к этой давно исчезнувшей
аптеке побудила меня однажды впоследствии стать шеф-редактором именно
медицинского издательского дома, и писать статьи об устройстве аптек разных
стран под псевдонимом Николай Смелов.

Конечно, это уже совсем другая история о Маугли, и о ней не было бы смысла
вспоминать тут, но именно в Музейном переулке (тогда это была улица Кирова) жил
до войны Николай Смелов, ушедший 23 июня 1941 на войну; инженер-капитан,
несколько раз ранен, несколько раз награжден. В 1944 году, когда моя бабушка со
всеми домочадцами и сундуком, набитым нотами и старыми письмами, въехала в
разгромленный Киев, в центре которого уцелело всего несколько домов, то первый,
кого она встретила, был ее племянник, уезжавший на фронт после ранения. И прямо
на колене он написал ей бумагу, что позволяет своей родственнице поселиться в
Музейном переулке, в его комнате. А через три дня погиб в бою, и наша семья так
и осталась жить в доме, у музея с каменными львами. Так что диковинным образом
Николай Смелов до сих пор имеет отношение и  к издательскому дому «Профессор
Преображенский», и к роману «Стеклянный ключ».

Так что все эти кнопочки-звоночки с анекдотическим подбором имен и фамилий —
читайте в ряд «Я. Штейн», «И. Я. Штейн», «А. Я. Финкельштейн», — гроздьями
висящие на дверях; мраморные ступени, дубовые перила, чугунная решетка старого
лифта, к которому до 1973 года прилагалась настоящая живая лифтерша; и черные
ходы, необъятные кухни, голландская печь с изразцами цвета сливок и узором из
фиалок, этажерка с книгами, засунутыми туда в количестве, вдвое превышающем
расчетные размеры, безделушки, утюжки для кружев, разрозненные чашки и блюдца –
трогательные остатки былого семейного счастья, — и нанкинский голубой шелковый
ковер… Словом, не могу же я заново переписывать весь роман – так вот все это
писано с натуры в натуральную же величину.

А вот синее файдешиновое платье, которые так изумительно шло к сапфирово-синим
глазам моей бабушки, пришло в американской посылке уже после 1944 года – их
присылали еще несколько лет после войны, причем эта помощь шла в огромных
количествах, и в этих ящиках из Америки и Европы можно было обнаружить что
угодно – платья, туфли, кремы, тушенку, галеты, косметику, ноты, книги, сахар,
чай, кофе… Так вот синее файдешиновое платье до сих пор живет у меня на
антресолях, и рука не поднимается его выбросить, хотя товарный вид, конечно,
уже не тот. А изумительной работы кольцо с изумрудом ушло из семьи намного
раньше войны – в голодных двадцатых бабушка обменяла его в Торгсине на огромную
запечатанную бутыль с подсолнечным маслом, а вместо масла в бутыли оказалась
коровья моча. Но рекламаций в Торгсине, как вы понимаете, не принимали. Сережки
обменяли в 1943 на хлеб.

Письма, прекрасные поэтичные письма влюбленного короля, правда, не Людовика
XIV, а Генриха IV погибли со многим прочим имуществом в Ташкенте, в 1920 году.

Многое не сохранилось, увы. Про некоторые вещи трудно было предположить, что их
не станет – тем более, так внезапно, быстро. Особенно грустно, что не осталось
фотографий, картин – чего-нибудь. Например, прекрасной деревянной эстрады в
Мариинском парке (я писала о ней во многих эссе и рассказах, напишу еще, но что
с того – ее не вернуть). Деревянная раковина с великолепной акустикой. Позади,
в зеленых зарослях уютно умостился бюстик Глинки. Слева – если стоять лицом к
эстраде – чугунный «стакан» — крохотный стеклянный киоск в оплетке из чугунного
литья, где выдавали бесплатно свежую прессу. После стали взимать две, что ли,
или три копейки. Подобное чугунное литье можно было увидеть на киосках,
стоявших по обе стороны Прорезной. В одном продавался табак, сигары и сигареты.
В другом – о, какое это было чудо, в другом киоске: крутящаяся стойка с
длинными мензурками с фруктовыми и ягодными сиропами и стеклянный сосуд с
газировкой. Помните «Подкидыша»? Раневскую? Муля, не нервируй меня! А теперь –
вишневую. И малиновую. Вот-вот-вот, сколько там было сиропов – точно больше
десятка: смородиновый, крыжовенный, мандариновый!

Но возвратимся к эстраде. На ней выступали знаменитые на всю Европу дирижеры:
Турчак, Рахлин, Косточка Симеонов. Рахлин, кстати, самоучка, в отряде
конногвардейцев с Котовским прошел от и до, освоив по дороге почти все
музыкальные инструменты. Так вот, в первых рядах на их концертах действительно
сидела сестра звезды «Большого вальса» Милицы Корриус.; и когда она умерла,
кто-то привязал к скамейке черную ленту, и место это довольно долго пустовало.
И у дирижеров считалось хорошим тоном слегка поклониться, адресуясь этому
месту.

И на самом деле в близлежащих к парку магазинах шли бои среди меломанов за
светлую оберточную бумагу – ею выстилали лавки у эстрады, чтобы не испортить
костюмы от мадам Тимукиной. И Анна Васильевна – женщина, по-бунински не
решившаяся уехать от мужа с молодым офицером и помнившая эти несколько дней
невыносимого счастья всю оставшуюся невыносимую жизнь, – тоже была. Как был и
особнячок рядом с домом графа Уварова – просто в описанной мной комнатке жила
другая  – с судьбой еще более удивительной и фамилией гораздо более известной.
А соседкой ее была, между прочим, теща знаменитого партизана Ковпака, имени
которой я так никогда и не запомнила, потому что бабушка и ее приятельницы
неизменно звали тещу Феклой.

А вот Трояновский – тот воевал, в Первую мировую особо отличился. Мать его
мечтала женить сына на одной из княжон Кантакузиных (маму даже где-то можно
понять: Кантакузины — потомки византийских императоров, род древний и богатый
до неприличия) — а он сопротивлялся как германскому наступлению и даже грозился
посадить за стол рядом с собой «башибузука», своего денщика Ахмета, преданного
ему до невозможности. То есть, я забыла упомянуть, что Трояновский служил
офицером в татарском полку Дикой Дивизии, то есть был кавалергардом. Потом он
принял революцию, поехал комиссаром в Среднюю Азию, Ахмет поехал с ним и
подцепил где-то тиф. Трояновский Ахмета выходил, а сам от тифа умер. Вот такая
еще была история. 

В детстве, своем сказочном детстве, я видела многих удивительных людей и
слышала массу невероятных историй, в том числе – любовных. Вот на улице Розы
Люксембург, возле упомянутого уже мною особняка графа Уварова и неподалеку от
дома любовницы сахарозаводчика Терещенко, госпожи Обезьяниновой, жила Елизавета
Дмитриевна. Когда я узнала ее, она была уже весьма почтенной женщиной. Весьма
счастливой почтенной женщиной. Замуж выходила пять раз – «я была счастлива в
браке 26 лет, на это у меня ушло пять мужей» — как раз ее случай. Еще
гимназисткой она влюбилась в Романа, Ромашку – он был чуть старше, потому,
когда началась Первая мировая ушел на фронт добровольцем. Потом случилась целая
жизнь – наша, особенная жизнь, с 1924, 1937, 1941, 1945. И вот, когда Елизавете
Дмитриевне было столько, что она с ностальгией вспоминала свои пятьдесят, на
улице напротив Мариинского парка она встретила Ромашку. Что восхитительно, они
не стали терять времени: подали заявление на той же неделе, быстро поженились,
по-моему, в «шоколадном домике» — тогдашнем центральном ЗАГСе им даже не стали
давать время на размышление. И они прожили долгую прекрасную жизнь. А потом,
однажды, когда они пили шампанское и слушали музыку, Ромашка поднялся и вышел
на кухню со словами «Я тебя жду». И умер, разумеется. А Елизавета Дмитриевна
как-то почти без слез его похоронила, через неделю купила торт, шампанское,
надела праздничное платье – и... легко-легко, светло-светло...

Многое изменилось, многое ушло безвозвратно, но пока мы живы и говорим об этих
людях мы, таким образом, говорим за них перед Богом. Полагаю, это главная
работа тех, кто помнит. Это главное дело тех, кто пишет.

Он стоял со своим мольбертом в Мариинском парке в любую погоду. Зимой – в
ушанке и теплом кожухе, руки у него коченели на морозе, и он хлебал что-то
исходившее на морозе паром из трофейного немецкого термоса с фарфоровой чашкой.
Летом, весной, осенью – всегда стоял с кистями, мольбертом, стульчиком, которым
никогда не пользовался, палитрой, целым чемоданчиком с тюбиками краски...

Многое я помню как бы в удвоенном виде: просто как виденное мною и как
отражение в его картине. Особенно хорошо помню любимую аллею от фонтана с
маскаронами (ой – это тоже целая история: когда их отливали, то сделали
маскароны с физиономии мастера, который жутко допекал рабочим) и сам дворец
императрицы Марии Федоровны. Бабушка его называла только так. Помню картину –
осень, золото, лазурь, хрустальная вода, ручейком льющаяся по груде камней.
Тогда в Мариинском был еще маленький фонтан, похожий на сказочный прудик, и
статуя Леси Украинки стояла под огромной ивой, словно раздвигая ветви, словно
выходя к пруду...

Он писал истово, каждый день, каждую деталь, так, что я до сих пор не могу
забыть эти свои детские впечатления.

Вот бы мне научиться писать свое так же.

Пространством называется то, что окружает тела.

Временем — то, как они исчезают

Самуил Лурье

Фото - Доктор Айболит в витрине Детской аптеки (прибалтийский \enquote{близнец} киевского оформления)
