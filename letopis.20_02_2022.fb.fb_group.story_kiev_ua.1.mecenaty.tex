% vim: keymap=russian-jcukenwin
%%beginhead 
 
%%file 20_02_2022.fb.fb_group.story_kiev_ua.1.mecenaty
%%parent 20_02_2022
 
%%url https://www.facebook.com/groups/story.kiev.ua/posts/1865795256950600
 
%%author_id fb_group.story_kiev_ua,dubinina_oksana
%%date 
 
%%tags gorod,kiev,mecenat
%%title Агов, де ви, шановні меценати?...
 
%%endhead 
 
\subsection{Агов, де ви, шановні меценати?...}
\label{sec:20_02_2022.fb.fb_group.story_kiev_ua.1.mecenaty}
 
\Purl{https://www.facebook.com/groups/story.kiev.ua/posts/1865795256950600}
\ifcmt
 author_begin
   author_id fb_group.story_kiev_ua,dubinina_oksana
 author_end
\fi

Агов, де ви, шановні меценати?...

Серце стискається, коли ми бачимо, як сумними темними очима-вікнами дивиться на
Андріївський узвіз Замок Річарда, як гине садиба Івана Терещенка на бульварі
Шевченка, як руйнуються садові місточки ландшафтного парку в садибі Більського
(колишня дача Хрущова), коли очі бачать руїни Замку Рудольфа Штейнгеля і безліч
інших пам’яток київської архітектури, дуже хочеться, щоб і в Києві
реставрували, берегли, щоб допомагали підтримувати  меценати Історію для наших
нащадків. 

\ii{20_02_2022.fb.fb_group.story_kiev_ua.1.mecenaty.pic.1}

І дуже приємно, що ми маємо хороші приклади меценацтва в Україні. Адже всі
реставровані об’єкти можна використовувати, як івент-простір, таке модне нині
слово, для розвитку, відпочинку та туризму. (event – англ. подія)

\ii{20_02_2022.fb.fb_group.story_kiev_ua.1.mecenaty.pic.2}

Свого часу, Ольга Богомолець шукала для своєї колекції ікон приміщення в Києві
і не знайшла, адже ціни були непомірно високі на приміщення. Ось так у 2007
році викупила і почала потроху реставрувати напіврозвалену старовинний млин в
Житомирській області, який через довгих 4 роки – у 2011 році перетворився на
чарівний за красою та задумкою туристичний та культурний центр.

\ii{20_02_2022.fb.fb_group.story_kiev_ua.1.mecenaty.pic.3}

Історико-культурний комплекс «Замок Радомисль». Житомирська обл., м.Радомишль.

Трохи більше 100 км від Києва знаходиться це чудове і дуже цікаве місце.
Казковий замок і ландшафтний парк радують очі та душу. Такі місця дарують
насолоду та збагачують енергією.

\ii{20_02_2022.fb.fb_group.story_kiev_ua.1.mecenaty.pic.4}

Я в захваті від цієї ідеї та від втілення прекрасної мрії лікарем та політиком
Ольгою Богомолець. Справа в тому, що замок – це відреставрована будівля старого
млина (майже його розвалин), збудованого у 1902р на старому фундаменті старої
Папірні – паперової фабрики, що згадувалася в старовинних хроніках, як
Радомишльська паперова фабрика (збудована у 1612р). Дивно, але ця Папірня була
збудована, як оборонна споруда і має всі ознаки фортеці. Можливо все ж первинно
ця будівля, зроблена на підземній гранітній скелі, призначалася саме для
оборони...

\ii{20_02_2022.fb.fb_group.story_kiev_ua.1.mecenaty.pic.5}

В музеї замку знаходиться колекція ікон, зібрана Ольгою Богомолець, та тисячі
історичних експонатів українського побуту. Інтер’єр виконаний в середньовічному
стилі. 

В замку є цех виготовлення паперу старовинним методом.

\ii{20_02_2022.fb.fb_group.story_kiev_ua.1.mecenaty.pic.6}

Дуже гарний ландшафтний парк з місточками, гранітними скульптурами, спокійними,
неначе дзеркало, ставками та бурхливими водоспадами.

\ii{20_02_2022.fb.fb_group.story_kiev_ua.1.mecenaty.pic.7}

Історична коротка довідка. Місто Радомишль знаходиться на лівому березі річки
Тетерів. Перша історична згадка у 1150 році під назвою Мичеськ, а потім –
Микгород, (саме так зараз називається один з районів міста). Назва Микгород від
старої річки Мика, оскільки на древлянській мові «микатися» означало колись
дуже стрімкий та норовливий характер цієї річки. 

\ii{20_02_2022.fb.fb_group.story_kiev_ua.1.mecenaty.pic.8}

З першої половини XVI століття після татарського нападу місто перенесли на
високі береги річки Тетерів і назва Радомисль пішла від словосполучення
«радісна мисль» від більш вигідного та захищеного розташування.

\ii{20_02_2022.fb.fb_group.story_kiev_ua.1.mecenaty.pic.9}

Назва міста Радомишль з 1946 р.

P.S. Щодо дороги до Радмишлю повинна сказати, що із 100 км – 70 майже ідеальна
Житомирська траса, але якихось 25-30 км будьте готові долати несподівані
перешкоди бездоріжжя. Але враження від цього залишилися все одно пречудові.

\ii{20_02_2022.fb.fb_group.story_kiev_ua.1.mecenaty.pic.10}
\ii{20_02_2022.fb.fb_group.story_kiev_ua.1.mecenaty.pic.11}
