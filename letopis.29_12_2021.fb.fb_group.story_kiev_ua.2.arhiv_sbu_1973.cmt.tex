% vim: keymap=russian-jcukenwin
%%beginhead 
 
%%file 29_12_2021.fb.fb_group.story_kiev_ua.2.arhiv_sbu_1973.cmt
%%parent 29_12_2021.fb.fb_group.story_kiev_ua.2.arhiv_sbu_1973
 
%%url 
 
%%author_id 
%%date 
 
%%tags 
%%title 
 
%%endhead 
\zzSecCmt

\begin{itemize} % {
\iusr{Igor Popell}
\enquote{эксперЕмент}
Какой пассаж, мон колонель! И как стыдно перед мсье Пьером!  @igg{fbicon.laugh.rolling.floor} 

\begin{itemize} % {
\iusr{Yuriy Bubnov}
\textbf{Igor Popell} 

Товарищ подколонель где-то слышал слово \enquote{экстремент} и малость перепутал. А
фамилия какая - Киселюк-Торжевский... Звучит как Бендер-Задунайский. Из князей,
что ли?

\end{itemize} % }

\iusr{Вадим Сандино}
Исп. Шишов до сих пор сидит в том же кабинете. Полковник сбу небось...

\iusr{Інна Лапань}
Саву

\iusr{Владимир Писаренко}

Неужели с бликами поляризационный фильтр и длиннофокусная оптика не помогали?
Лентяи какие-то в КГБ работали.

\begin{itemize} % {
\iusr{Игорь Козиянчук}
\textbf{Владимир Писаренко} 

они пользовались советской пленкой( видимо шпионы до иностранной не
дотягивали), а она имеет чувствительность 65 единиц! Чуть тень и жопа


\iusr{Александра Цымбал}
\textbf{Владимир Писаренко} шторы то никто не запрещал

\iusr{Владимир Писаренко}
\textbf{Александра Цымбал} , 

в отчёте КГБ в шторы не упоминаются, а в окне они закрывают только часть
помещения.

Скорее, у сотрудников не было особого рвения к слежке.

В моей скромной практике был похожий случай - обратились за помощью люди, у
которых не получилось восстановить удаленную информацию средствами СБУ. Очень
удивлялись, когда у меня получилось. Видимо, те не очень старались...

\end{itemize} % }

\iusr{Nadia Kuchmy}
Фильм вспомнился \enquote{Est-Ouest} (1999)

\begin{itemize} % {
\iusr{Yuriy Bubnov}
Одного из комментаторов эта публикация очень задела. Уж не он ли печатал донесение?
\end{itemize} % }

\iusr{Valentina Urban}

Да это же дом в котором я жила и выросла, Лабораторный переулок \#24 кв. 15.
Первые 40 квартир по 2 квартиры на каждом из пяти этажей были сданы в
эксплуатацию в 1957 куда мы переехали с частного дома на ул. Предславинской.
Продолжение к нашему дому построили в 1959-60 гг. Мимо окон этой квартиры
проходили ежедневно направлялась на ул. Щорса. В последнем парадном на первом
этой пристройки в 3- х комнатной угловой квартире проживала семья французов. Их
огромный кот всегда грелся на солнышке расположились на подоконнике. Прочитав
статью так хотелось бы узнать кто проживал в квартире \# 82 в те времена, ведь
мы дети практически знали всех жильцов и нашего дома, и соседних еще четырех
домов. Это был маленький островок домов с детской площадкой, овощным магазином,
входом на верхнее поле Центрального стадиона, гаражами и входом в военный
госпиталь. Я прожила в этом доме 1957-1972 гг.

\begin{itemize} % {
\iusr{Ольга Кирьянцева}
\textbf{Valentina Urban} 

у меня есть знакомый, который тоже в этом доме вырос и жил как раз в эти же
годы. Только я не знаю номера квартиры. Знаю только, что у него был брат и отец
военный.

\begin{itemize} % {
\iusr{Valentina Urban}
\textbf{Olga Kiryantseva} 

Насколько я помню в нашем доме жили семьи у которых отец был военным : Это
Виктор и его старший брат Паляница, Славик и его старший брат Родины, Саша и
Коля Олейник, Сережа или Саша и его младший брат Матвиенко, также в нашем
парадном проживала семья Сидоровых, но у них было две дочери. Может кого-то из
перечисленных фамилий вспомните.

\iusr{Ольга Кирьянцева}
\textbf{Valentina Urban} 

нет, среди перечисленных. Они жили в угловом подъезде, как он говорил, на
\enquote{изгибе} дома. Братья Михаил и Сергей. Их отец был подполковником, создал и
заведовал военной кафедрой у нас в институте.

\iusr{Valentina Urban}
\textbf{Olga Kiryantseva} 

К сожалению я таких братьев абсолютно не помню. В годы когда я там жила в
изгибе было только одно парадное, вход в которое для жильцов был с обратной
стороны дома. На первых двух этажах размещались детские ясли и сад, а на
остальных 3- х этажах жили: Родины, еще какая-то то семья без детей, и на
верхнем пятом этаже гимнаст Юрий Титов с женой теннисисткой Лерой. Возможно
вышеупомянутые Вами Михаил и Сергей поселились перед нашим отъездом поэтому я
их не знала.


\iusr{Вадим Глина}
\textbf{Ольга Кирьянцева} а я жил на Щорса 23-а

\iusr{Valentina Urban}
\textbf{Vadim Glina} 

В прежние годы я хорошо знала ул. Щорса, параллельную Лабораторному переулку,
однако прошло так много лет. Интересно где находился дом Щорса 23-А, что было
рядом? Кажется в доме \#29 по ул. Щорса было ателье мод, однако вполне возможно
что я уже призабыла.

\iusr{Вадим Глина}
\textbf{Valentina Urban} Угловой дом, сразу за моргом Военного госпиталя, перед и-том «Укргипросахар».

\iusr{Valentina Urban}
\textbf{Vadim Glina} 

Спасибо, прекрасно помню этот 9-ти этажный дом построенный где-то в середине
60-х. Вход в парадные был со двора, а в полуподвальном помещении с лицевой
стороны был кажется склад галантирейной продукции.

\end{itemize} % }

\end{itemize} % }

\iusr{Barbara Novokhatska}
Спасибо, очень интересно

\iusr{Сергей Андреещев}

Вот этот кот и был самым главным шпиёном. Я знаю, он за детским садом на Щорса
следил, а французы так, оперативное прикрытие для отвлечения внимания.
Недоглядел Федорчук, прогавил ...

\iusr{Gennadiy Radchenko}
Кислюк-Торжевский не очень то парился, без огонька подглядывал.

\iusr{Леся Сагайдачная}
Вот уж точно

\iusr{Мария Иванова}
На лабораторном общежитие комитетовское было.

\iusr{Наталія Таша}
да что ж такое! порубать эти деревья!

\iusr{Sasha Ty}
Хм, я мешкав в такому ж будинку і за вікном були гаражі з танками.
Виявляється, можно було цим зкористуватись
 @igg{fbicon.smile} 

\iusr{Вадим Глина}

Я знал эту семью, бывал у них дома. Полноватый француз, который работал на
радио, его худощавая жена и две дочки-скелетины. И куча французского винила. В
69 году это был клондайк! Потом, как-то разошлись наши пути. И куда он делся,
этот шпийон-француз - даже не знаю...

\begin{itemize} % {
\iusr{Вадим Глина}

А жили они а первом этаже, как заходишь в подъезд, поднимаешься на пол марша и
сразу направо. У него было полно винила из Франции- Энрико Масиас, Азнавур,
Эдит Пиаф, Саша Дистель, Джони Холидей, Сильвия Вартан, Далида, Гинзбург, Матье
и многие другие. Я брал у француза, уже и не помню, как его и звали, эти винилы
и переписывал на наш семейный Дніпро 10. Это было не то, что круто, это было
что-то невообразимое! А потом папины и мамины сослуживцы с наших бобин
переписывали эти записи и себе.

\begin{itemize} % {
\iusr{Valentina Urban}
\textbf{Vadim Glina} 

В вашем доме кажется на 6 или 7- м этаже жил Гена фарцовщик. Вы его случайно не знали?

\iusr{Вадим Глина}
\textbf{Valentina Urban} 

Мы жили на 7 этаже в первом парадном. Гена был наш сосед- виолончелист из
оперного театра. Но он не был фарцом. В соседнем подъезде жил Сергей Крыницкий.
Но в 1972 году он, «по пьяне», выпал с балкона 7 этажа. А в 1973 году мы
переехали на площадь Толстого, в проходняк. С тех пор у меня о Щорса 23А нет
никаких сведений.


\iusr{Valentina Urban}
\textbf{Vadim Glina} 

Спасибо за ответ, речь шла о другом Гене, который действительно был
фарцовщиком.

И тем не мнее, большое Спасибо за общение и прекрасные воспоминания дорогих
памяти мест.

С Новым здоровым, счастливым и мирным Годом Вас и всех близких!
\end{itemize} % }

\iusr{Valentina Urban}
\textbf{Vadim Glina} 

Спасибо за напоминание, отца семейства не помню, а вот помню что дочки были
действительно очень худыми и не общительными.

\begin{itemize} % {
\iusr{Вадим Глина}
\textbf{Valentina Urban} Да, дочки были нелюдимыми и страшненькими и полностью разбивали стереотипы о красоте французских женщин!)))
\end{itemize} % }

\end{itemize} % }

\iusr{Вадим Глина}

А если пройти дальше по Щорса, перед хлебозаводом, был дом с магазином. В нём
жил кореец, женатый на нашей девушке. Он был моим первым учителем по восточным
единоборствам в 1968-70 годах. Так вот его арестовали за шпионаж. Я пришёл к
нему домой, а жена говорит, что ночью был обыск и мужа куда-то увезли. Больше я
ни его, ни его жену не видел(((

\begin{itemize} % {
\iusr{Yuriy Bubnov}
\textbf{Вадим Глина} 

М.б. его арестовали именно за то, что был учителем по восточным единоборствам?
В те времена, это было под запретом, наряду с карате и культуризмом.

\begin{itemize} % {
\iusr{Вадим Глина}
\textbf{Yuriy Bubnov} Нет, я потом видел его соседей и они сказали, что его арестовали именно за шпионаж.

\iusr{Yuriy Bubnov}
\textbf{Вадим Глина} Возможно, и так. Хотя, для советского обывателя, все иностранцы были \enquote{шпиЁнами}.))

\iusr{Вадим Глина}
\textbf{Yuriy Bubnov} Это точно. Даже из \enquote{соцлагеря}!
\end{itemize} % }

\iusr{Валентин Ореховский}
\textbf{Вадим Глина} Ты опасный человек Вадим,!!!


\begin{itemize} % {
\iusr{Вадим Глина}
\textbf{Валентин Ореховский} Не, не очень)))
\end{itemize} % }

\iusr{Valentina Urban}
\textbf{Vadim Glina} Вы наверное занимались единоборствам в спортзале возле детского сада напротив дома где жили?

\begin{itemize} % {
\iusr{Вадим Глина}
\textbf{Valentina Urban} Нет, туда нас не пускали. Занимались у него дома.

\iusr{Валентин Хромогин}
\textbf{Valentina Urban} 

а згадайте як в газеті Молодь України (не впевнений що вірно написав) на останній
сторінці вели школу кврвте, з фотографіями. Було круто.

\end{itemize} % }

\end{itemize} % }

\iusr{Жалил Жукенов}

И это в то время когда наши космические корабли бороздили бескрайние просторы
Вселенной и передавали фотоснимки с объектов вне Солнечной системы))


\iusr{Владимир Дубровский}

Фотоаппараты \enquote{ФЭД}, \enquote{Смена}, \enquote{Зоркий}, и ещё им подобные на таком расстоянии не
снимали.


\end{itemize} % }
