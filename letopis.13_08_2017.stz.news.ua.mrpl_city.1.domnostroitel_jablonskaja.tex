% vim: keymap=russian-jcukenwin
%%beginhead 
 
%%file 13_08_2017.stz.news.ua.mrpl_city.1.domnostroitel_jablonskaja
%%parent 13_08_2017
 
%%url https://mrpl.city/blogs/view/domnostroitel-yablonskaya
 
%%author_id burov_sergij.mariupol,news.ua.mrpl_city
%%date 
 
%%tags 
%%title Домностроитель Яблонская
 
%%endhead 
 
\subsection{Домностроитель Яблонская}
\label{sec:13_08_2017.stz.news.ua.mrpl_city.1.domnostroitel_jablonskaja}
 
\Purl{https://mrpl.city/blogs/view/domnostroitel-yablonskaya}
\ifcmt
 author_begin
   author_id burov_sergij.mariupol,news.ua.mrpl_city
 author_end
\fi

\ii{13_08_2017.stz.news.ua.mrpl_city.1.domnostroitel_jablonskaja.pic.1}

В предвоенные годы Яблонскую в Мариуполе знали все. Если не в лицо, то хотя бы
понаслышке. И много лет спустя после того, как она уехала из нашего города, ее
вспоминали, особенно те, кому пришлось работать с этой удивительной женщиной на
строительстве \enquote{Азовстали}. Вспоминали разное. И ее неженскую хватку, и
блестящий инженерный талант. А кто запомнил, что она курила, носила мужскую
одежду и порой бывала грубой, что перед ней трепетали нерадивые прорабы,
бригадиры и десятники, что держалась она всегда независимо — стоял ли перед ней
рабочий или представитель наркомата.

\ii{13_08_2017.stz.news.ua.mrpl_city.1.domnostroitel_jablonskaja.pic.2}

В экспозиции музея комбината \enquote{Азовсталь} была выставлена фотография Яблонской:
гладко зачесанные волосы, усталый взгляд широко поставленных глаз, на губах
полуулыбка, из ворота плотно застегнутого жакета выглядывает краешек белого
воротничка блузки да узел галстука, завязанного на мужской манер. Любовь
Викторовна Яблонская приехала в Мариуполь в 1932 году, чтобы возглавить
\enquote{Доменстрой} — трест по нашим понятиям, перед коллективом которого была
поставлена задача — возвести доменный цех нового металлургического гиганта,
строящегося на берегу Азовского моря. К этому времени она накопила более чем
солидный инженерный и организационный багаж. Направленная после окончания в
1926 году Московского высшего технического училища им. Баумана на Керченский
металлургический завод на должность инженера-конструктора, Яблонская потратила
немало усилий и времени, чтобы перейти из конструкторского бюро непосредственно
на производство. По бытующим в ту пору взглядам, женщина и доменный цех были
понятиями абсолютно несовместимыми.

\ii{13_08_2017.stz.news.ua.mrpl_city.1.domnostroitel_jablonskaja.pic.3}

Вопреки этому, казалось бы, незыблемому постулату переход оказался весьма
плодотворным: под ее руководством на Керченском металлургическом заводе были
построены три доменные печи с механизированной загрузкой шихтовых материалов и
первая в Советском Союзе агломерационная фабрика ленточного типа. Более того,
Любовь Викторовна после пуска в эксплуатацию доменного цеха успела поработать в
нем начальником смены, помощником начальника цеха и не с чужих слов, а на
собственном опыте эксплуатационника могла оценить все огрехи, допущенные при
строительных работах и монтаже оборудования. Керченский период деятельности она
завершила, поработав некоторое время главным инженером управления капитального
строительства.

За шесть с небольшим лет работы на \enquote{Азовстали} — с апреля 1932-го по июнь 1939
года — Яблонская практически полностью повторила все то, что ей пришлось делать
в Керчи.

И в 60-80-е годы сооружение доменных печей относилось к числу весьма сложных
инженерных задач. Это-то при широком применении средств механизации:
экскаваторов с многокубовыми ковшами, мощных бульдозеров, строительных кранов
большой грузоподъемности, значительного парка тяжелых автосамосвалов, а главное
— при наличии высококвалифицированных и образованных кадров: рабочих, техников
и инженеров различных специальностей.

Яблонской и ее соратникам по строительству доменных печей \enquote{Азовстали} пришлось
довольствоваться работой малограмотных землекопов — вчерашних крестьян,
раскулаченных и высланных в так называемые трудовые лагеря \enquote{для перевоспитания}
на ударную стройку первой пятилетки, и грабарей, чьим орудием труда была
\enquote{грабарка} — телега с запряженной лошаденкой. Автомобили-\enquote{полуторки} и
заграничные экскаваторы-драглайны на стройке были большой редкостью.
Руководящий состав состоял главным образом из молодежи, только что окончившей
техникумы и вузы, студентов-практикантов из строительных институтов определяли
на должности бригадиров и десятников.

Уступ за уступом вгрызались лопатами в твердь земли тысячи людей, нескончаемая
вереница телег-\enquote{грабарок} отвозила выкопанный грунт в отвалы, день и ночь не
переводя дыхания. Именно так были вырыты котлованы под сами домны и их
воздухонагреватели, под другие сооружения, именно таким способом был прорыт
судоходный канал существующего и поныне азовстальского порта. Когда началось
бетонирование фундаментов, люди, обхватив друг друга руками за плечи, от зари
до зари утаптывали колышущуюся желеобразную бетонную массу ногами. Позже
десятки клепальщиков приступили к соединению вручную металлоконструкций в
единое целое: в те годы электросварка только начинала пробивать себе дорогу в
технике.

Действия многотысячной армии строителей направлялись несгибаемой волей женщины
— Любови Викторовны Яблонской. Она успевала повсюду: и разобраться в чертежах,
только что доставленных проектировщиками, и побывать там, где наметился прорыв
на стройке, и устроить \enquote{разнос} замешкавшемуся прорабу, и дозвониться в Москву,
требуя немедленной поставки материалов. Трудно представить, откуда она черпала
время и силы на преподавание на различных курсах повышения квалификации
рабочих, на проведение политзанятий, на участие в соответствующих духу времени
многочисленных партийных и профсоюзных заседаниях и конференциях. Она считала
своим долгом посещать общежития-бараки, чтобы хоть как-нибудь улучшить полный
лишений быт первостроителей.

Вот как выглядит характеристика, выданная ей в августе 1939 года директором
завода П.А. Комовым: \enquote{Инженер Яблонская Любовь Викторовна работала на заводе
\enquote{Азовсталь} в должностях начальника \enquote{Доменстроя} с 1 апреля 1932 г. по 16
октября 1937 г., помощником прораба \enquote{Ремзаводстроя} — с 10 октября 1937 г. по 1
апреля 1938 г., главным инженером отдела капитального строительства — с 1
апреля 1938 г. по 1 июня 1939 г.  За время работы на заводе тов. Яблонская
выполняла все свои обязанности командира-производственника честно, с полным
знанием своего дела, широко применяя стахановские методы, поднимала высоко
техническую и хозяйственную культуру на производстве}.

Именно в нашем городе первая в мире женщина инженер-доменщик Любовь Яблонская
достигла вершины жизненного успеха, в Мариуполе ей во второй раз в жизни
довелось увидеть плоды своего самозабвенного труда, труда без выходных и
праздников, без отдыха, труда на грани человеческих возможностей. Можно
представить, какие чувства испытывала она, когда построенные под ее
руководством первые три доменные печи \enquote{Азовстали} выдавали первый чугун. Именно
в нашем городе за достигнутые трудовые успехи Яблонская была удостоена высшей
награды страны. Правда, здесь ей довелось пережить обиду: смещение с поста
начальника доменного цеха на незначительную и малоответственную должность
помощника прораба второстепенного строительного управления, но об этом позже…

Какая же жизненная \enquote{кузница} выковала железный характер этой удивительной
женщины? Любовь Викторовна родилась в 1897 году. Училась в гимназии, с
четырнадцатилетнего возраста давала платные уроки отстающим ученикам. Значит,
семья была со скромным достатком. Еще в детстве обнаружилась музыкальная
одаренность. Училась игре на фортепиано, по воспоминаниям близких, ее учителем
некоторое время был выдающийся композитор и пианист Сергей Рахманинов. После
окончания в 1916 году гимназии стала слушательницей Высших женских курсов в
Петрограде. Кто знает, как бы сложилась ее судьба, если бы не революция? Может
быть, стала учительницей, а может — пианисткой? Но как мы знаем, революция
свершилась. Самодержавие было свергнуто.

Май 1917 года. Вчерашняя курсистка Люба Яблонская становится членом партии
левых эсеров, ставших в скором будущем соратниками большевиков по Октябрьскому
перевороту. С сентября 1917-го по май 1918 года она депутат Саратовского Совета
рабочих, крестьянских и солдатских депутатов, выполняет поручение Губисполкома,
работает в отделе по расформированию прибывающих с фронта воинских частей и
охране военного имущества развалившейся царской армии. В мае 1918 г. выходит из
партии левых эсеров, в том же году, в сентябре, начинает работу в Москве
инструктором по внешкольному образованию и членом коллегии губернского отдела
народного образования.

В январе 1919 года Краснопресненский райком РКП(б) принимает ее в партию
большевиков, а уже в августе молодая большевичка выезжает вместе с мужем —
военкомом запасных войск Московского округа — в одну их частей Красной Армии
под Тулу. Неожиданный прорыв белых - Яблонская попадает в плен, чудом избегает
расстрела. И снова она в Москве. Как агитатор Московского губкома РКП(б)
выступает на митингах, собраниях, в рабочих общежитиях.

Гражданская война окончилась, промышленность и сельское хозяйство — в состоянии
полной разрухи. Для их восстановления требуются не просто квалифицированные
специалисты, но еще и люди, беззаветно преданные делу партии. Наиболее
перспективных молодых коммунистов направляют на учебу в технические вузы. Так в
1921 году Любовь Яблонская оказывается среди студентов Московского высшего
технического училища имени Баумана — в высшей степени авторитетного учебного
заведения по подготовке инженерных кадров еще с царских времен. О том, что
произошло в ее жизни между окончанием вуза и 1 июня 1939 года — датой, когда
Любовь Викторовна получила полный расчет на заводе \enquote{Азовсталь}, чтобы
отправиться к месту нового своего назначения в Москву, рассказано выше...

В автобиографии, составленной Яблонской в январе 1957 года, написано: \enquote{С 1939
года приказом по Наркомчермету была откомандирована на работу в Главное
управление металлургической промышленности Юга и Центра. За время работы в
Министерстве черной металлургии (так после 1946 г. стали именоваться Народные
комиссариаты. — С.Б.) с 1939 по 1954 г. в качестве начальника сектора и
старшего инженера руководила капитальным строительством и восстановлением
металлургических заводов Донбасса...}

Отметим, что Яблонская была в чем-то человеком везучим. Когда по стране в
1937-1938 годах вовсю гуляла \enquote{ежовская} коса, она, имевшая непролетарское
происхождение, бывшая эсерка, близкая родственница \enquote{врагов народа} (родная
сестра Любови Викторовны Франя и ее муж Ермаков были репрессированы в
Ростове-на-Дону еще в 1936 году), непосредственная подчиненная еще одного
\enquote{врага народа} — бывшего начальника строительства, а затем директора завода
\enquote{Азовсталь} Якова Гугеля, не попала ни под \enquote{расстрельную} статью, ни даже за
решетку. Да, это можно объяснить только везением. Правда, ее резко понизили в
должности — об этом сказано в приведенной ранее характеристике, выданной ей
директором \enquote{Азовстали} Комовым, да в феврале 1938 года Мариупольский горком
КП(б)У вынес ей строгий выговор \enquote{за потерю бдительности}. Впрочем, это
наказание уже в июне 1939 года было снято тем же партийным органом...

Итак, Любовь Викторовна летом тридцать девятого года переезжает в Москву.
Казалось бы, после нескольких лет напряженнейшего труда на стройках и заводах,
когда для сна оставались считаные часы, труда без выходных, праздников и
отпусков, можно было бы ей отдохнуть в тиши наркоматовского кабинета под шелест
перелистываемых страниц томов отчетов и справок. Ведь от нее ничего более не
требовалось, чем исполнительность чиновника. Но нет, Яблонскую не устраивала
тихая жизнь. Она использует малейшую возможность, чтобы побывать на заводах.
Проверяет и перепроверяет столбцы цифр в документах, присланных с предприятий.
Ищет закономерности в успехах одних цехов и \enquote{провалах} — других. Дельным
советом старается помочь отстающим, радуется внедренным новшествам, прилагает
все усилия, чтобы новшества эти получили как можно большее распространение. Ей
поручено руководство капитальным строительством на заводах, входивших в систему
Главного управления металлургической промышленности Юга и Центра Наркомата
черной металлургии СССР.

Началась Великая Отечественная война. В первые же дни войны ее единственный сын
Лев, несмотря на то что имел бронь — он был студентом четвертого курса
Московского института связи, ушел добровольцем на фронт. Любовь Викторовна
одобрила этот поступок. В 1941 году в тяжелейших боях под Ржевом Лев был убит.
Невероятную, раздирающую душу и мозг боль Яблонская заглушает работой. В уже
упоминавшейся автобиографии 1957 года Любовь Викторовна пишет: \enquote{В период
Отечественной войны по поручению Наркомата выполняла особо важные задаиия на
Косогорском металлургическом заводе и по вводу в действие новых мощностей на
металлургических заводах Урала (Челябинском, Златоустовском, Чебаркульском)}. С
осени 1943 года части Красной Армии в ожесточенных боях начали освобождать от
фашистских оккупантов один за другим металлургические центры Донбасса и
Приднепровья. Отступая, гитлеровцы превращали цеха заводов и города в груды
развалин. В великий подвиг восстановления черной металлургии Украины внесла
значительный вклад и Любовь Викторовна.

После войны карьера Яблонской идет по нисходящей: все ниже должности, все менее
ответственные. В 1954 году Главное управление металлургической промышленности
Юга и Центра ликвидируется. Штаты Министерства черной металлургии СССР резко
сокращаются. Старшему инженеру производственного отдела Л.В. Яблонской без
предварительной беседы объявляют, что с 14 июля 1954 года она увольняется из
министерства. Сохранился черновик письма Любови Викторовны в адрес заместителя
председателя Совета Министров СССР И.Ф. Тевосяна. С этим человеком ее связывали
многие годы совместной работы по созданию современной черной металлургии
страны. Нельзя без волнения читать строки, словно летящие по листам бумаги,
исполненные горечи и обиды. Яблонская пишет: \enquote{В 1941 году я потеряла
единственного сына (погиб на фронте Великой Отечественной войны), и
единственное, что у меня осталось, — это труд}. И далее: \enquote{Прошу Вас, Иван
Федорович, вмешаться и дать соответствующее указание руководству Министерства
черной металлургии СССР}. Тевосян не вмешался, и Любовь Викторовна лишилась
работы.

Другой человек, может быть, и смирился с обстоятельствами, но только не героиня
этого очерка. Ей было уже 57 лет, когда она блестяще окончила редакторское
отделение вечернего факультета Московского полиграфического института и
приступила к исполнению обязанностей научного редактора журнала \enquote{Сталь} в
издательстве \enquote{Металлургия}. В этом качестве она трудилась еще двадцать три
года. Невероятное трудолюбие, глубокое знание металлургических проблем, общая
эрудиция обеспечили ей за короткое время уважение как со стороны членов
редакционной коллегии авторитетного журнала, так и со стороны авторов,
присылавших свои работы для опубликования.

Умерла Любовь Викторовна Яблонская 30 декабря 1980 года...

На этом можно было бы и завершить повествование о легендарной
женщине-доменщике, составленное на основе документов, хранящихся в архиве
Народного музея комбината \enquote{Азовсталь}. Да, они позволили воссоздать ее
жизненный и производственный путь, даже в какой-то мере дают возможность
представить черты ее неординарного характера. Но образ Л. В. Яблонской
становится еще более ярким, когда знакомишься с воспоминаниями Елизаветы
Павловны Валуевой, к сожалению, недавно ушедшей в мир иной.

Вот они: "Можно сказать, что вся моя сознательная жизнь в общем-то связана с
Любовью Викторовной. И я благодарна судьбе, что она меня свела с этой
замечательной, редкой женщиной. Знаю я о ней по рассказам своего отца,
покойного Павла Алексеевича Валуева, и сама лично. Говорили, что в Мариуполь из
Керчи, где она до того работала, приехала Яблонская по просьбе Серго
Орджоникидзе. Сразу же отправилась на \enquote{Азовсталь}, облазила строящуюся домну
сверху донизу. Работала она много, и большую часть суток, в том числе и в
воскресные дни, находилась на заводе. Лишь поздно вечером отправлялась домой.
Жила она в одноэтажном доме в переулке Республики. Тогда это была окраина
города. У нее в это время рос маленький сын. Зимой он жил у сестры Любови
Викторовны в Москве и только на лето приезжал в Мариуполь.

Она выглядела строгой и сухой. Носила синюю спецовку, мужские ботинки, берет,
который был укреплен большой английской булавкой. И никто не догадывался, что
под беретом у нее скрывается прекрасная светлая коса. При внешней строгости и
неприступности была она человеком, всегда готовым помочь людям в трудную
минуту. Мои родители вспоминали не раз о таком случае. Когда я родилась, а это
время было невероятно скудным, меня не во что было завернуть, чтобы забрать из
родильного дома. Отец метался, искал, где бы достать хоть какой-нибудь кусок
материи на пеленки. Любовь Викторовна от кого-то узнала об этом и распорядилась
без лишних слов выписать кусок шторной ткани и солдатское одеяло.

Незадолго перед войной Яблонскую решили перевести в Главк, в Москву. Она ни за
что не хотела уезжать с \enquote{Азовстали}, много сил она вложила в этот завод,
нравилась ей эта работа. И все же вынуждена была покинуть Мариуполь. С тех пор
Любовь Викторовна  жила в Москве, но часто приезжала в наш город в
командировки. Запомнились ее приезды к нам в дом после войны, в сороковых
годах, когда я была уже достаточно взрослой. Как всегда, более чем скромно
одетая, с маленьким чемоданчиком в руке. Моих родителей с Любовью Викторовной
связывала многолетняя дружба. Вечерами они подолгу засиживались, говорили о
работе, об азовстальских делах. Это была их жизнь.

Потом я поступила учиться в энергетический институт в Москве. И получилось так,
что Любовь Викторовна взяла меня на первое время к себе. Нужно сказать, что мне
было с ней временами нелегко. Она старалась воспитывать меня в своем духе. Я не
имела права делать маникюр, красить помадой губы. Я должна была быть предельно
скромна. Но за что я ей благодарна, так за приобщение к классической музыке. В
молодости Любовь Викторовна училась в консерватории. Была очень музыкальна. Она
доставала билеты на самые великолепные концерты и спектакли в Большой театр.

Самой большой и оберегаемой реликвией в ее доме был самодельный радиоприемник,
собранный ее погибшим сыном. Любовь Викторовна ездила на место его гибели и
привезла оттуда землю. Я уже не могу сказать — с могилы или просто землю,
опаленную огнем. Она об этом никогда не говорила. Вообще никогда и ни с кем не
делилась своим горем. Когда я помогала ей переезжать из громадной коммуналки,
где кроме нас жили еще восемь семей, в две комнаты, выделенные ей с сестрой в
трехкомнатной квартире, узелок со священной для нее землей она не доверила
никому, несла его сама.

Что еще было свойственно для Любови Викторовны, так это забота о близких. Когда
ее сестра Полина овдовела, оставшись с четырехлетней дочуркой на руках, она
забрала их к себе. Так они с Полиной и прожили всю жизнь вместе. Еще до войны
старшая сестра Франя была арестована вместе со своим мужем, а потом
расстреляна. Осталась ее дочь Ася — семнадцатилетняя студентка первого курса
Московского медицинского института. Любовь Викторовна без раздумий забрала Асю
к себе. И тогда ее стали вызывать раз за разом в печально известные органы,
требовали, чтобы она отреклась от своей племянницы, как ей говорили, дочери
\enquote{врагов народа}. В ответ получали решительный отказ. В один из таких вызовов ей
предложили положить на стол партбилет. И снова отказ. От нее отстали только
после вмешательства Джапаридзе — дочери одного из расстрелянных белогвардейцами
двадцати шести бакинских комиссаров, которая к тому времени занимала важный
государственный пост и была вхожа к самому Сталину. Асе дали возможность жить и
учиться в Москве, но сразу, же после окончания института ее выселили в
Свердловск.

Еще один штрих ее характера. Любови Викторовне было за восемьдесят, когда ее
сбил автомобиль. Последствия этой аварии были тяжелыми. У нее был перелом
основания черепа. Самое страшное — она потеряла память. Но эта женщина
продолжала бороться. Я видела ее в этом состоянии. Она писала слова, она их
переписывала сотни раз, чтобы вспомнить эти слова и повторить. Она старалась
заговорить, она старалась смотреть на губы собеседника, чтобы опять быть в
строю...".

Л.В. Яблонская жила и работала в нашем городе чуть больше, чем шесть лет, но
привязалась к нему на всю жизнь. В 1944 году она приезжала в Мариуполь в
составе государственной комиссии, принимавшей в эксплуатацию после
восстановления 3-ю доменную печь завода \enquote{Азовсталь}. Она и позже навещала город
своей молодости и по делам, и во время отпусков. Один из таких визитов, можно
сказать, зафиксирован документально. 6 мая 1969 г. газета \enquote{Приазовский рабочий}
писала: \enquote{У нас в гостях побывала бывший начальник \enquote{Доменстроя} на сооружении
завода \enquote{Азовсталь} в тридцатых годах, первая в стране женщина-домностроитель
Любовь Викторовна Яблонская. Вместе с азовстальцами она приняла участие в
первомайских торжествах в нашем городе}.
