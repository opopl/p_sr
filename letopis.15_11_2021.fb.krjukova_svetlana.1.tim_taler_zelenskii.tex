% vim: keymap=russian-jcukenwin
%%beginhead 
 
%%file 15_11_2021.fb.krjukova_svetlana.1.tim_taler_zelenskii
%%parent 15_11_2021
 
%%url https://www.facebook.com/kryukova/posts/10159780290888064
 
%%author_id krjukova_svetlana
%%date 
 
%%tags chelovek,krjus_dzhejms.pisatel,politika,smeh,svoboda_slova,tim_taler,ukraina,zelenskii_vladimir
%%title Тим Талер и Зеленский
 
%%endhead 
 
\subsection{Тим Талер и Зеленский}
\label{sec:15_11_2021.fb.krjukova_svetlana.1.tim_taler_zelenskii}
 
\Purl{https://www.facebook.com/kryukova/posts/10159780290888064}
\ifcmt
 author_begin
   author_id krjukova_svetlana
 author_end
\fi

Недавно мне в руки попалась замечательная детская книга немецкого писателя
Джеймса Крюса «Тим Тайлер или проданный смех» - про мальчика, который продал
свой звонкий неповторимый смех незнакомцу в обмен на способность выигрывать
любое пари. Мальчик разбогател, стал несчастным и половину книги потратил на
то, что пытался вернуть себе свой смех обратно.

Угадайте, про кого я подумала, не дочитав до развязки. 

И как вы думаете, чего стоило нашему президенту отказаться от смеха, юмора,
свободы слова?

А ведь именно на галерах юмора, свободы, Зеленский построил свой бизнес.

Свобода слова сделала его популярным. 

Свобода слова позволила ему стать президентом.

А теперь он стал ненавистником этого понятия, настолько, что угробил своим
эффектом самоцензуры юмор 95 квартала, способным сегодня конкурировать разве
что с шоу Поплавского. Прошло то время, когда мы слали друг другу ссылки,
потешаясь над тонким местами грубым, но всегда смешным юмором.

Прошло то время, когда отрывки шоу 95 квартала становились новостным событием.
И теперь главный вопрос заключается в том, осознаёт ли президент губительные
последствия собственных поступков? 

И какую цену он готов заплатить за то, чтобы вернуть себе свой смех?

\ii{15_11_2021.fb.krjukova_svetlana.1.tim_taler_zelenskii.cmt}
