% vim: keymap=russian-jcukenwin
%%beginhead 
 
%%file 29_11_2020.sites.ru.zen_yandex.yz.valerongrach.1.pop_14_vek.comments
%%parent 29_11_2020.sites.ru.zen_yandex.yz.valerongrach.1.pop_14_vek
 
%%url 
 
%%author 
%%author_id 
%%author_url 
 
%%tags 
%%title 
 
%%endhead 

\subsubsection{Комментарии}

\begin{itemize}
\iusr{Victor S}

1. Похмельнъ — то есть "похмелен" или "пьян".

2. Покуша — это скорее не от "ясти", а от "кушати" — "пробовать". Возможно, он снова болеет, но старается писать. А может, это — имя собственное, вроде Кукши?

3. Не "плечо болит", а "плечи болят", в обоих случаях.

4. "Не могу", скорее всего, тоже к болезни относится. Представляю себе, как тяжело писать объемные тексты. Да еще уставом, даже не полууставом, и тем более не скорописью, как было уже, скажем, в 17 веке.

5. "О горѣ мруть попи" — видимо, о страшной эпидемии "черной смерти" в 14 веке,
по-настоящему выкашивавшей, в отличие от нынешней самовенчанной заразы, не
просто деревни, но и города, и целые области. В Псков она пришла из Ганзы или
Швеции. А потом в Новгород.

\iusr{Иван Тургенев}

Victor S, цитата: "Покуша — это скорее не от "ясти", а от "кушати" —
"пробовать". А "пробовать" - это, случайно, не "отведать из чарки"? :)

\iusr{сигизмунд паперный}

Victor S, Автор упоминает "болить плечо",а я вижу "болЯть плечИ".Кто прав?

\iusr{Ольга Комарницкая}

сигизмунд паперный, болят плечи - верно

\iusr{Victor S}

Иван, значение слова "кушать" близко к "пробовать" (ср.
"покуситься/покушаться"), а не "есть", профессор Зализняк по
материалам берестяных грамот говорил об этом и приводил
примеры.

\iusr{Andre Sher}
отредактировано

А попик похоже сию книжицу переписывал и на сем имел доход хоть и невелик, был
попик вельми грамотен, а писание сие имел в имении...посему
приписки на полях оставил, что книжицей сам и пользовался....А
то что переписывал знаем по болезни , имеемой по причине
усталости плеч, поелику согбен . А попадье с попиком скучно
обретаться, вот и наладилась баба по гостям ходить, о чём попик
и скорбел весма...

\iusr{Юлия Кузьмина}

Когда читаю древние тексты, прихожу в восторг, когда встречаю "человеческое": иронию и нежность.

\iusr{Юлия Анс}

Спасибо за статью, приписки эти умилительны, как привет из глубины веков))

\iusr{роман попов}
отредактировано

Это был первый официальный блог, гусиным пером писанный.

Самое время под надписями рисовать сердечки.

\iusr{Sandro Clarice}

«Смиренный игуменъ Пафнутiй руку приложилъ»(с)

\iusr{Алла Полякова}

Это удивительное чувство - связь времён! У меня такое бывает, когда смотрю на
что-то сделанное руками человека - каменную или деревянную резьбу или другие
предметы старины. Вот, думаю, старался, делал это человек, живший за много
веков до меня. И как бы внутренним взглядом вижу его - как останавливается,
чтобы передохнуть, вытирает пот со лба, смотрит на небо, давая отдых глазам и
мыслям. Перебрасывается шутками с другими мастерами или ругается с нерадивыми
подмастерьями. Трогает загрубевшими пальцами свою работу - хорошо ли сделано?
Долго ли прослужит людям, будет радовать глаз? Хорошо, безымянный мастер! Я
посмотрела на дело рук твоих и порадовалась, и представила тебя, и на минуту ты
вернулся из небытия. Вспомнит кто-нибудь меня через несколько веков? Вряд ли.
Нечего мне оставить на столь долгую память. Жаль, но ничего не поделаешь, время
безжалостно. "Но много нас ещё живых, и нам причины нет печалится..."

\iusr{Дима Н}

О трех предметах не спеши рассуждать: о Боге, пока не утвердишься в вере; о
чужих грехах, пока не вспомнишь о своих, и о грядущем дне, пока не увидишь
рассвета.

Святитель Николай Сербский (Велимирович)

\ifcmt
pic https://avatars.mds.yandex.net/get-zen_pictures/2977420/993072345-1607548030116/orig
width 0.3
\fi

\iusr{Александр Юрьевич Попов}.

Живой был человек. Дай Бог Царствия Небесного.

\iusr{Сергей Голубев}

Спасибо. Занятный пример. Прочитал как Ваше, так и "попатворение"....Слово
"псати" сразу навело на мысль - что не только у будущего сотрудника
администрации Байдена - украинские=славянские корни.....Ну ..буквочка
поменялась, а "сюжет тот же "- Мели Емеля-твоя неделя!

\iusr{Соколов Э.}

Это наверное местное и возрастное. Влажность высокая, теплообмен
непредсказуемый. Что-то перегревается, что то мерзнет. Мне 58 во Пскове.
Последний год начали плечи болеть. Два раза в неделю перегреваюсь в ванне.
Сердце пока тянет. Не пишет поп про баню. А, баня это еще и глубокий массаж и
"диклофенирование" листьями. Все как сейчас рекомендуют, только пластмассовыми
методами. Эх, попил бы я с попом пивка...

\iusr{Наталия Кириченко}

И правда,так просто и душевно!

\iusr{Павел Новиков}

он отвечает кому то... и оправдывается, возможно напарнику - переписчику... при
помощи записей на ,, полях,,... это же очевидно...

\iusr{Роман}

Товарищ А, Знаменский Завещанная река. В лето 7216-е[1] от сотворения мира
великие бунты были на Дону и Слободской Украине, умылось кровью Дикое Поле. А
потом лютая зима прошла, и с полой водой ждали царя в замирившемся Черкасске.

Я убежден,что верным счет вести летам, нужно до Петровской реформы от С.М.З.Х

\iusr{Иван Иванов}

«О горе мрут попы» От горя мрут попы. Так же как и все мужики связаны с
природой. Чаще всего повышенная чувствительность и восприимчивость ко всему.
Сердечко чаще всего не выдерживает.

\iusr{Надежда Жигарина}

Но слово —поп, тогда не существовало.Были: писцы, батюшки, пономарь,
псаломщик.Даже слово "дьяк" означало служивый при царских приказах, это типа
современных министерств.

\iusr{111111 111111111}

Как же трогательно! За безликим переписанным текстом увидеть живого,
подверженного всем чувствам, страстям и недугам человека!
\end{itemize}
