% vim: keymap=russian-jcukenwin
%%beginhead 
 
%%file 17_04_2022.stz.news.ua.pravda.2.budynok_volontery_istorii_harkov
%%parent 17_04_2022
 
%%url https://life.pravda.com.ua/authors/6244d751ecd43
 
%%author_id lapshin_oleksandr,news.ua.pravda
%%date 
 
%%tags 
%%title Як мешканці одного будинку стали волонтерити разом. Історії з Харкова
 
%%endhead 
 
\subsection{Як мешканці одного будинку стали волонтерити разом. Історії з Харкова}
\label{sec:17_04_2022.stz.news.ua.pravda.2.budynok_volontery_istorii_harkov}
 
\Purl{https://life.pravda.com.ua/authors/6244d751ecd43}
\ifcmt
 author_begin
   author_id lapshin_oleksandr,news.ua.pravda
 author_end
\fi

\ii{17_04_2022.stz.news.ua.pravda.2.budynok_volontery_istorii_harkov.pic.1}

\begin{zznagolos}
У Харкові активно працюють волонтерські організації. Всі вони об'єднані
спільною метою допомоги, але по-різному її втілюють.

Одна з історій волонтерства – про співмешканців одного житлового будинку, які
згуртувалися та стали єдиним організмом заради допомоги іншим.

Розповідаємо, як сусіди, які до війни активно не спілкувалися, стали опорою
один для одного та для інших жителів міста.
\end{zznagolos}

\ii{17_04_2022.stz.news.ua.pravda.2.budynok_volontery_istorii_harkov.pic.2}

\textbf{Від автора:}

Ми познайомилися з Льохою через біг. Спочатку бачили та коментували тренування
один одного у Strava (мобільний застосунок для спортсменів), а потім домовилися
про спільний довгий, недільний крос. Це було на початку 2021 року.

\ii{17_04_2022.stz.news.ua.pravda.2.budynok_volontery_istorii_harkov.pic.3}

За приблизно дві години бігу харківськими вулицями та пагорбами ми багато про
що встигли поговорити з Олексієм.

Зараз я вже не пригадаю усіх деталей, але в моїй пам'яті добре закарбувалося
загальне враження про міцну духом, загартовану фізично, вольову людину. І ще
одна деталь – у переважно російськомовному Харкові, Олексій спілкувався
українською. І це викликало повагу.

Після цього ми довгий час з ним не бачилися. Лише продовжували
\enquote{лайкати} нові тренування.

Так тривало аж до початку березня цього року, коли ми випадково зустрілися з
Льохою біля одного з харківських перехресть. Ця подія в той день мене дуже
вразила, і я написав на своїй сторінці у Facebook.

Після цього я був певен, що Льоха зараз в ТРО. І через власне фотографічне
бажання побачити, як там у них все облаштовано, чим саме живуть хлопці та як
несуть службу, писав та пропонував йому зустріч. І за деякий час вона таки
відбулася!

Але те що я побачив, та про що дізнався з багатьох годин розмов, вразило мене
більше? аніж очікуваний військовий побут. Я дізнався про стихійно створену
волонтерську організацію.

Її засновниками, учасниками, натхненниками та постійним працівниками є звичайні
люди – мешканці однієї багатоповерхівки. Саме такі, які кожного мирного ранку
їхали поруч з вами в вагоні метро на роботу. А зараз залишились в Харкові та,
ризикуючи власним життям, допомагають іншим.

Далі – пряма мова мешканців будинку.

\raggedcolumns
\begin{multicols}{3} % {
\setlength{\parindent}{0pt}

\textbf{Аня Верба}

\emph{фармацевт-косметолог}

\ifcmt
  ig https://i2.paste.pics/c1f702821ffc73958bc6e09ce02e82d1.png
	@caption Аня Верба
\fi

Історія з волонтерством виникла досить спонтанно. Хтось знав когось. Чиїсь
знайомі передали нам гуманітарку. Ми її роздали. А потім спрацювало правило
\enquote{П'яти рукостискань}. І у нас в домі почали з'являтися все нові і нові пакети з
продуктами, ми їх сортували та роздавали.

Серед мешканців утворимся справжній живий організм. Ми доповнюємо один одного!
Нічого б не вийшло без знайомих Льоші, Юриної ксіви, Саніних рук, розуму
Богдана!

У нас вже було декілька психологічних стадій, коли ми розуміли, що якщо один з
команди випадає, то починає хворіти весь організм. Тому ми всіляко підтримуємо
та намагаємось допомагати один одному у всіх питаннях!

Зараз я займаюсь ліками, але все частіше подорожую містом у справах разом з
хлопцями.

\textbf{Ростислав}

\emph{директор музичної школи. Викладає хоровий спів}

\ifcmt
  ig https://i2.paste.pics/bdecfa307ea2a48230c81456e9bf1191.png
	@caption Ростислав
\fi

Я мешкаю у сусідньому будинку, але мої друзі мешкають тут. До початку війни ми
з ними домовилися, що якщо вона почнеться, ми залишимося в Харкові.

Коли пролунали перші вибухи, я прийшов до друзів. Спускатися до підвалу нам не
хотілося, тому ми сиділи у коридорі. По ньому постійно ходили мешканці, ми
віталися та розмовляли.

Так серед іншого дізналися, що Льоха, Юра, Богдан мали бажання записатися до
ТРО, але там вже не було місць. А раз немає можливості стати у стрій, то треба
знаходити інший спосіб допомагати.

Так ми дізналися про волонтерів з Харківської державної адміністрації та
домовилися з ними про продукти для мешканців будинку. Коли їх привезли, стало
зрозуміло, що для нас їх забагато – через евакуацію людей у будинку стало
менше.

Надлишки ми роздали через своїх знайомих. А потім коробки та пакети з
продуктами продовжували надходити, і ми продовжували їх роздавати. \enquote{Сарафане
радіо} розпочало свою роботу!

Ми з хлопцями провели нараду та вирішили розподілити обов'язки. Саша,
наприклад, займається фасуванням продуктів. А я організував склад ліків.

Хоча раніше не мав до них жодного стосунку, але коли їх стало забагато і вони
були звалені аби як, вирішив навести порядок. Приніс стіл, знайшов
\enquote{Довідник ліків} та розсортував ліки по \enquote{частинах тіла}.

\textbf{Олена та Богдан}

\emph{Олена – артистка хору. Богдан – інженер-проектувальник металоконструкцій}

\ifcmt
  ig https://i2.paste.pics/a3082da67de7d24d99a5d98261317698.png
	@caption Олена та Богдан
\fi

Коли розпочалася війна, ми розгубилися. Зрозуміли, що зовсім до неї не готові.
З готівкових грошей у нас на руках було лише 100 грн. Тому ми відразу пішли до
банкомату. І це добре, що напередодні ми встигли заправити машину.

Інтуїтивно відчували, що будемо корисні в місті, тому і не виїхали.

Перші дні війни ми провели частково в квартирі, а частково в коридорі. У нас
був старий, маленький радіоприймач, і ми його постійно слухали. Ми винесли до
коридора лише матрац і вважали, що війна триватиме не більше 3 днів.  Але на 10
день ми стали облаштовувати наш побут у коридорі вже більш ретельно.

Нагальні потреби об'єднували мешканців. Так, наприклад, нам було необхідно
вимкнути зовнішнє освітлення будинку. У мене була драбина, у когось знайшлася
сокира. Деякі ліхтарі ми погасили, перебивши силові дроти до них.

Першу гуманітарну допомогу привезли Юра та Льоха. Лише для мешканців будинку її
було забагато, тому ми розпочали пропонувати продукти знайомим та розвозити їх.
Процес спочатку був хаотичним, але згодом через таблиці, чати, фотозвіти ми
вибудували систему.

Декілька разів у нас були загальні зібрання мешканців, де ми вирішували
нагальні питання та призначали відповідальних за напрямками.

Олена стала займатися побутової хімією. Богдан відповідає за логістику та
комунікацію з водіями. Але трапляються випадки, коли одночасно доводиться
складати маршрут, розвантажувати-завантажувати машини та спілкуватися
телефоном.

\textbf{Саша}

\emph{колишній комірник}

\ifcmt
  ig https://i2.paste.pics/6d57048e5350096a5205b45355f050d8.png
	@caption Саша
\fi

Я мешкаю у цьому будинку разом з дружиною та донькою. Зараз вони у близьких в
області.

Я не можу сказати, що раніше знав своїх сусідів. Так, ми віталися, коли
зустрічали один одного, але близького спілкуванням не було.

Коли Юра з Льохою почали привозити продукти, ми стали значно більше
спілкуватися. Я вирішив їм допомагати. Зараз я отримую, сортую, розміщую
продукти, які надходять з різних  джерел. Час від часу разом з хлопцями їзджу
їх розвозити.

На жаль, іноді трапляються випадки, коли гуманітарну допомогу замовляють молоді
та здорові люди, а ще й вередують, коли її отримують, що не той сорт макаронів
або не  той виробник дитячого харчування.

Вони сприймають нашу роботу не як благодійність від людини до людини, а як мій
обов'язок, наче ми мусимо це робити та їм допомагати!

Але в більшості випадків ми отримуємо щиро вдячність та тепло від людей, яким
ми привозимо продукти! Саме ця людська подяка надає нам сил! Бо інколи ми
можемо поснідати десь о 17 чи 18 годині.

\textbf{Юра та Олег}

\emph{працювали у сфері громадського харчування}

\ifcmt
  ig https://i2.paste.pics/52542331edefbef5029de7f1d5eead71.png
	@caption Олег
\fi

Коли розпочалася війна, купили тютюну та сіли грати в \enquote{доту}. Потім вирішили
піти та записатися до ТРО. Три дні поспіль приходили до держадміністрації, але
нас розвертали.

На четвертий день, коли ми вранці знову збиралися туди, пили каву, о 8:02
пролунав гучний вибух. А о 8:05 нам зателефонували та сказали, що до ХОДА можна
не приходити – її розбомбили.

Потім ми збиралися виїхати з міста, але на вокзалі було занадто багато людей. А
потім нам зателефонував Олексій та запросив допомогти розвантажити вагони з
гуманітаркою.

Більшу її частину забрали волонтери, але щось залишилось, і ми її забрали.
Перевезли все до будинку, розсортували та почали розвозити.

Цим я і займаюся до сьогодні.

\textbf{Аня Калина}

\emph{працювала в IT}

\ifcmt
  ig https://i2.paste.pics/65598ab4b7d95dda0708a12da333ec68.png
	@caption Аня Калина
\fi

Ми з Льохою – сусіди. У нього з дружиною є двоє кішок, і в мене є кіт. Через це
ми і познайомилися. Коли хтось їхав з міста, інший доглядав за домашнім
улюбленцем. А ще я знала про Аню. У неї маленька собака і я чула її гавкіт.

Коли розпочалася війна, до мене зайшла Аня і спитала "Що будемо робити?".

Ми спустилися у підвал, де на фоні стресу від того що відбувається навколо,
мешканці починали спілкуватися. Люди проявляли себе дуже щиро та сердечно.
Підтримували та допомагали.

У ці дні Льоха відразу ввімкнувся.  Вирішував питання та заспокоював людей.

Я бачила, як навколо нього формується коло охочих допомагати іншим. Але сама
вирішила до нього не приєднуватися.

Я бачила, що хлопці стали справжніми годувальниками для тих мешканців, які
залишилися у будинку. Вони постійно заклопотані, у роз'їздах та повертаються
надвечір втомленими. І в цей момент я вирішила потурбуватися про хлопців.

Хоча раніше я не дуже любила хатні справи, але зараз я із задоволенням готую
для хлопців та підтримую чистоту.

Завдяки хлопцям у нашому будинку збереглося відчуття острова безпеки у місті. Я
не збираюся їхати з міста. За моїм внутрішнім відчуттям я маю залишатися та
допомагати.

\textbf{Юра}

\emph{співробітник Управління дотримання прав людини Національної поліції України}

\ifcmt
  ig https://i2.paste.pics/9445f567294d391736db9b21c50c8904.png
	@caption Юра
\fi

Я мешкаю у цьому будинку приблизно рік, але майже ні з ким не спілкувався.

Рано виходив на роботу, пізно повертався. 24 лютого я цілий день провів в
Управлінні. Ввечері ми розходилися і не розуміли що буде далі.

"Чекайте подальших розпоряджень", - так нам сказали.  Наступного ранку я поїхав
до райвідділу. Там всі нервуються та чекають, але не зрозуміло чого. Протягом
дня мені телефонували батьки, друзі, знайомі та наполягали, що б я їхав з
міста. Але для себе я вирішив, що нікуди не поїду.

Ввечері повернувся додому. Розпочалися обстріли, я пішов у підвал. Раніше і не
підозрював, що він взагалі тут існує.

Там вже зібралося багато мешканців. І я став ходити від одних дверей до інших,
стежачи за порядком. Оскільки на мені була частина службових речей, а частина
цивільних, мене стали зупиняти та питати: "А ви часом не російський шпигун?"
То ж довелося показати документи, жетон та заспокоїти своїх сусідів.

Ми познайомилися з Льохою, разом вирішили йти записуватися до ТРО, але не
встигли цього зробити, бо в ХОДА влучили ракетою. Тоді ми вирішили допомогти
мешканцям з продовольством.

Розшукували контакти волонтерів. Домовлялися про перші поставки. Розвантажували
вагони, забирали залишки, розвозили.

З часом з'являлися все нові і нові джерела гуманітарної допомоги. Але попит на
допомогу завжди випереджає наші можливості. Ми намагаємося розподіляти той
об'єм, який ми маємо та допомагати найбільш нужденним.

Зараз по запитам від цивільних ми допомагаємо продуктами, побутовою хімією,
ліками, одягом, кормами для тварин.

Серед мешканців нашого будинку деякі чоловіки вступили та зараз несуть службу у
ТРО. Ми їм допомагаємо. А також підтримуємо побратимів з поліції та ЗСУ.
Знаходимо та привозмо хлопцям  необхідні обладунки та техніку.

Доводиться багато спілкуватися та залучати свої службові зв'язки.

\textbf{Льоха}

\emph{директор в міжнародній компанії}

\ifcmt
  ig https://i2.paste.pics/116f72bc8de8eeebb1cfe1d1a8b26f3f.png
	@caption Льоха
\fi


Я знав, що буде війна з Росією. Заздалегідь готувався. Як фізично, так і
технічно. Офіційно купив гвинтівку, отримав всі необхідні документи. Регулярно
тренувався.

Планував, що коли розпочнеться війна, я виїду під Київ, де зустрінуся зі своїми
друзями, та ми разом будемо чинити опір загарбникам.

Дружину з двома котам я відправив в евакуацію за два дні до 24.02. Думав, що
росіяни розпочнуть 25.02. Але трохи помилився.

Коли побачив, що коїться на залізничному вокзалі, зрозумів, що мені з
гвинтівкою буде вкрай важко виїхати. Тому залишився у місті. Почалися обстріли,
я спустився у підвал – а далі ви вже все знаєте!

\end{multicols} % }

P.S. Це лише декілька історій серед мешканців будинку, які активно займаються
волонтерством. За роботою волонтерської команди Volonter\_d20 можете стежити в
Instagram.

Олександр Лапшин, спеціально для УП. Життя

Вас також може зацікавити:

\begin{itemize}
\item \href{https://life.pravda.com.ua/society/2022/03/25/247961/}{\enquote{Щоб Україні зберегти незалежність, треба розвалити рашку}. Записки із Харкова. ФОТОРЕПОРТАЖ}
\item \href{https://life.pravda.com.ua/society/2022/03/29/248010/}{\enquote{Лишила ключ сусідці, а через день її знайшли мертвою}: історії українок з евакуаційного автобуса}
\item \href{https://life.pravda.com.ua/society/2022/04/15/248248/}{Робити трохи більше, ніж можеш. Історії волонтерів найбільшого хабу Help Ukraine Center в Польщі}
\item \href{https://life.pravda.com.ua/society/2022/03/12/247782/}{На Харківщині дівчина народила дочку під обстрілами: окупанти не пропустили швидку}
\end{itemize}

