% vim: keymap=russian-jcukenwin
%%beginhead 
 
%%file 07_07_2023.stz.news.ua.donbas24.1.svitlana_otchenashenko_birthday.txt
%%parent 07_07_2023.stz.news.ua.donbas24.1.svitlana_otchenashenko_birthday
 
%%url 
 
%%author_id 
%%date 
 
%%tags 
%%title 
 
%%endhead 

День народження актриси, без якої складно уявити історію Маріуполя (ФОТО)

7 липня народилася видатна маріупольська актриса Світлана Отченашенко

Сьогодні, 7 липня, свої 78 років відзначала б почесна громадянка Маріуполя,
народна артистка України, лауреатка премії ім. Заньковецької Світлана Іванівна
Отченашенко, якби не вторгнення рф. Сьогодні без Світлани Іванівни складно
уявити театральне життя Маріуполя.

Читайте також: У Києві відкриється виставка, присвячена театральному життю
Маріуполя (ФОТО)

Дитинство і юність актриси

Світлана Отченашенко народилася 7 липня 1945 року в селі Ісківці Полтавської
області. Юна Світлана брала участь в шкільній самодіяльності. З 7 років почала
приміряти на себе образ актриси. Раз на два роки мати вивозила доньку із села
до Ленінграда, де водила в театр. Враження від театру й вистав з його
комплексним мистецьким впливом на свідомість стануть найсильнішими спогадами
юності. Бачила на сцені живих Ольхіну та Смоктуновського у виставі «Ідіот» у
Великому драматичному театрі, що було для дівчини дарунком долі.

Освіта майбутньої актриси розтяглась на декілька років і пройшла у різних
містах. Вступила до Всеросійського державного інституту кінематографії у
Москві, але не витримавши напруги мегаполісом, перебралась до Києва, де
закінчила студію імені Франка. Серед викладачів Світлани Отченашенко — народний
артист УРСР Покотило Михайло Федорович та народний артист СРСР Степанков
Костянтин Петрович. Після закінчення студії працювала в Полтавському театрі,
але як початківець грала дрібні ролі. Коли Світлана приїхала до Маріуполя,
вирішивши спробувати свої сили у місцевому театрі, з нею трапився курйоз.
Оскільки вона не знала, де знаходиться театр, їй довелося їхати на таксі.
Однак, незважаючи на невелику відстань від вокзалу до театру, таксист вирішив
показати місто вродливій дівчині і катав її 40 хвилин. На відміну від Полтави,
у Маріуполі молоду Світлану Отченашенко відразу ж ввели в поточний репертуар
театру.

Читайте також: Маріупольська актриса і режисерка очолила новий театр у
Німеччині (ФОТО)

Творче і особисте життя

У репертуарі актриси низка яскравих ролей: Аркадіна, Раневська, Аббі Патнем. Та
ключовою для себе Світлана Іванівна вважала роль Марії Каллас («Майстер Клас»,
«Страсті по Марії»). Загалом, про зіграних персонажів Світлани Отченашенко
можна сказати, що всі вони сильні, неоднозначні та незабутні.

Народна артистка дуже любила Маріуполь і вважала його найріднішим містом. Саме
у Маріуполі, куди вона приїхала з Полтавської області, щоб почати працювати в
театрі, відбувалися найважливіші події в її житті. Тут вона познайомилася зі
своїм чоловіком Харабетом Юхимом Вікторовичем, скульптором і медальєром,
заслуженим діячем мистецтв України, який підтримував у всьому кохану жінку. Тут
народила і виховувала сина. Завдяки енергійній діяльності актриса змогла
зберегти історичну театральну спадщину Маріуполя, надихнувши телеканал «Сигма»
на створення унікального циклу, присвяченого історії та людям Маріупольського
театру.

Попри те, що Світлана Отченашенко була актрисою далекого провінційного театру,
їй вдалося стати відомою в столичних професійних колах. Водночас вона була
літературним критиком. Крім цього, на рахунку актриси ціла низка моноспектаклів
(вечори поезії програми віршів Анни Ахматової, Марини Цвєтаєвої, «Мати» за
однойменним твором Олександра Довженка тощо). У актриси був досвід і в кіно.
Вона зіграла бабу Зою у фільмі «Зачароване кохання» (2008 рік).

Читайте також: Унікальні факти про театральну культуру Приазов’я (ФОТО)

«Дорогі мої маріупольці, всі українці, збережіть живою вашу душу. Не пускайте
туди ненависть і злість, адже зруйнована душа це набагато гірше зруйнованого
тіла!», — наголошувала Світлана Іванівна ще до повномасштабного вторгнення рф в
Україну.

Коли рф щодня бомбила Маріуполь, народна артистка залишалася у своїй квартирі.
Вона не захотіла покидати рідне місто і не могла подумати, що будівля театру, в
якій вона пропрацювала 61 рік, може бути знищена. Померла Світлана Отченашенко
7 квітня 2022 року після життя в блокадному Маріуполі внаслідок набряку легенів
у 2 міській лікарні Маріуполя.

Нагадаємо, раніше Донбас24 розповідав, що Луганський обласний театр представив
Україну на Міжнародному фестивалі «New Wave Theatre» в Румунії.

Найсвіжіші новини та найактуальнішу інформацію про Донецьку й Луганську області
також читайте в нашому телеграм-каналі Донбас24.

Фото: з особистого архіву Світлани Отченашенко
