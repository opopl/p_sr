% vim: keymap=russian-jcukenwin
%%beginhead 
 
%%file 16_01_2022.stz.news.ua.hvylya.1.anatomia_vraga.11.itogi
%%parent 16_01_2022.stz.news.ua.hvylya.1.anatomia_vraga
 
%%url 
 
%%author_id 
%%date 
 
%%tags 
%%title 
 
%%endhead 

\subsubsection{Промежуточные итоги}

Мы, конечно же, очень схематично и кратко рассмотрели главные принципы, на
которых основана государственная модель нашего идентичностного врага России.
Однако, подводя эти промежуточные итоги, попробуем выделить главное:

Современная геополитическая стратегия Российской Федерации, как государственной
модели, с точки зрения Москвы, находится в совершенно \enquote{рациональной} плоскости
и в общих чертах не отличается от предыдущих исторических периодов:

1. Непрерывная экспансия мировоззрения \enquote{единоначальной стабильности}, как
долгосрочная и системная государственная стратегия защиты от \enquote{территории хаоса}
либерализма и демократии. От этого зависит сложившийся уклад жизни российской
элиты.

2. Увеличение сферы своего геополитического влияния и нанесение \enquote{приемлимого}
экономического урона странам запада, принципы и смыслы общественно-политической
организации которых, несут угрозу обрушения государственной модели РФ в целом.
От этого зависит сложившийся уклад жизни российской элиты.

3. Максимальная консервация традиционных принципов структуры государственного
управления внутри страны и в своей сфере геополитического влияния, базирующихся
на жесткой \enquote{монархической} вертикали беспрекословного принятия легитимных
решений, касающихся внутренней и внешней политической повестки. От этого
зависит сложившийся уклад жизни российской элиты и лояльных элит в сфере
влияния Москвы.

Основой государственной конструкции Российской Федерации являются:

1. Принцип \enquote{сакральности} единоначальной вертикали власти, как фундамента
государственности и формы защиты общества/элит от \enquote{территории хаоса}.

2. \enquote{Вотчинное сознание} народа коллективиста, являющееся опорой,
легитимизирующей любые государственные решения вертикали власти, как
непреложную и единственно возможную форму общественного договора.

3. Культ войны, как идеологема осознанной необходимости государственной
стратегии - экспансии своего мировоззрения.

Эндогенными силами \enquote{развития} России являются жестокость властной вертикали и
непрерывность экспансии. Только в процессе экспансии Москва приобретает не
только новые ресурсы, но укрепляет вертикаль и стратагему выживания в целом,
обеспечивающую элиту комфортным укладом жизни.

Управленческая \enquote{либерализация} вертикали тормозит экспансию. Если экспансия
прерывается или теряет темп, в долгосрочной перспективе, российской вертикали
власти, а следовательно, элитам угрожает падение внутренней легитимности, а
государственной модели России это угрожает территориальными и ресурсными
потерями, что ведет модель к смуте, \enquote{оттепели} или революции.

Пределы экспансии Российской Федерации напрямую зависят от:

1. Уровня общественно-политической организации государственных моделей
противников

2. Способностей государственных элит противников проявлять инициативу и
выстраивать долгосрочные стратегии защиты

3. Осознания обществом противника необходимости платить ресурсную цену за свой
суверенитет. В том числе, в измерении человеческих жизней своих граждан.

Сильными сторонами современной России являются

1. Работающие государственные институции, сцементированные единоначальной
вертикалью власти, что дает возможность существенно сократить отрезок времени
между принятием решения и его реализацией.

2. Способность государственной системы к долгосрочному планированию

3. Гибкость в умении выстраивать долгосрочные и многоуровневые геополитические
стратегии с их последовательной реализацией с применением всех мер и
возможностей национального воздействия.

4. Осознанная готовность Российской Федерации, как государственной и
общественной модели, платить высокую цену за возможность реализации базовой
\enquote{иррациональной} цели - экспансии мировоззрения \enquote{единоначальной стабильности}.

Эта цена выражается в количестве материальных и человеческих ресурсов, которые
Кремль готов тратить на кажущуюся многим у нас непрактичную \enquote{иррациональность}
в своем масштабном противостоянии с западом и за восстановление \enquote{исторической
России} на \enquote{постсоветском пространстве}, о которой постоянно открыто говорит
Путин.

В реальности, которую мы предпочитаем не замечать, у России в настоящий момент
четко выстроена самоидентификация граждан, базирующаяся на исторически
оправданном алгоритме выживания их государственной модели, вне зависимости от
этнической культуры конкретного индивидуума. Русские, в отличии от нас, четко
знают кто они, за что они и зачем они. Возможность уничтожения государственных
смыслов \enquote{экспансии единоначалия}, является для русских тем, за что они будут
высокую платить цену. Кремль имеет жесткую вертикаль принятия и реализации
своих решений. Москва обладает способностью к выстраиванию долгосрочных,
многоуровневых стратегий и работающими институциями. Для реализации своей
базовой стратегии экспансии Россия обладает практически неограниченными
ресурсами, работающей экономикой и культом войны, с постоянно перевооружающейся
армией, имеющей обновляемый боевой опыт в различных горячих точках планеты от
Сирии, Мали и Ливии до Чечни и Донбасса. Такова реальность. Таков враг, с
которым мы имеем дело.

Зачастую, русских у нас в Украине называют геополитическими \enquote{гопниками},
\enquote{медведями} в шапках ушанках и лаптях, которые с грацией слона в посудной лавке
\enquote{непонятно почему} терроризируют нас и \enquote{цивилизованный мир}. Культурной,
европейской и толерантной Украине, якобы, следует лишь подождать пока какой-то
там \enquote{цивилизованный мир} успокоит \enquote{иррационального гопника}. Так выглядит наша
государственная \enquote{стратегия} защиты по сегодняшний день. Некоторые даже думают,
что находясь в такой позиции, мы выждем для себя удобный нашему социуму
результат - сидеть тихонько и \enquote{якось воно минеться}.

Это крайне опасное заблуждение, так как современная путинская Россия, находясь
в \enquote{посудной лавке} стремительно изменяющегося \enquote{цивилизованного мира} и
используя грубые методы \enquote{торговли ужасом}, на самом деле, комплексно реализует
очень прагматичную и по-своему рациональную, долгосрочную стратегию реванша и
возврата к своей исторической сфере геополитического влияния.

\enquote{Я не знаю как при Вас, а вот при нас ни одна пушка в Европе, без нашего на то
дозволения выпалить не смела} (с)

Эти слова, между прочим, принадлежат этническому украинцу, уроженцу города
Глухова, одному из инициаторов разделов Речи Посполитой, канцлеру Российской
империи Александру Безбородко. XVIII век.

Используя внутренние противоречия и слабости общества потребления, съедающего
эндогенную энергию современного запада, Россия агрессивно навязывает свои
инициативы в перекраивании мировых раскладов, а основные западные акторы, в
силу отсутствия мотивации проводить дальнейшую экспансию либерального порядка,
инициативу утратили. Именно поэтому мы сейчас наблюдаем лишь реактивную реакцию
США в их стремлении уйти от эскалации противостояния с РФ, не платить высокую
цену, сконцентрироваться на решении своих внутренних проблем и неизбежном,
заведомо проигрышном противостоянии Китаю.

В Кремле это отлично понимает и давят на западных \enquote{визави}, привычно торгуя
\enquote{ужасами войны} в обмен на возврат геополитического влияния на всю Европу,
ценой которого, в том числе, может стать и украинская государственность, а
упорное нежелание \enquote{элит} и общества в Украине понимать серьезность и общие
контуры российской стратегии, делает бесперспективной даже дискуссию о
выстраивании эффективной украинской стратагемы противостояния с Москвой.


