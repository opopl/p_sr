% vim: keymap=russian-jcukenwin
%%beginhead 
 
%%file 28_10_2020.news.ru.kommersant.1.sputniv_v_zarazhenije_koronavirus
%%parent 28_10_2020
 
%%url https://www.kommersant.ru/doc/4550045
 
%%author 
%%author_id 
%%author_url 
 
%%tags covid,vaccine,sputnik_v
%%title Несколько участников испытания вакцины «Спутник V» заразились коронавирусом
 
%%endhead 
 
\subsection{Несколько участников испытания вакцины «Спутник V» заразились коронавирусом}
\label{sec:28_10_2020.news.ru.kommersant.1.sputniv_v_zarazhenije_koronavirus}
\Purl{https://www.kommersant.ru/doc/4550045}

\index[rus]{Коронавирус!Вакцина!Спутник V}

Некоторые добровольцы, испытывающие вакцину «Спутник V», получили положительный
результат теста на коронавирус COVID-19, сообщил директор Центра Гамалеи
Александр Гинцбург. Он уточнил, что зараженные могли получить не вакцину, а
плацебо. В ходе испытаний для проверки эффективности вакцины плацебо получат 10
тыс. из 40 тыс. участников.

«По статистике какие-то ПЦР-положительные есть и какие-то заболевания, но им
могли ввести плацебо. Предварительные данные мы сможем подвести только в
середине ноября»,— сказал господин Гинцбург «РИА Новости». Он сообщил, что
первый компонент вакцины получила уже половина добровольцев, около 9 тыс.
провакцинированы и второй раз. Вакцина предполагает введение двух компонентов
препарата с разницей в три недели.

Минздрав зарегистрировал вакцину «Спутник V» Центра имени Гамалеи в августе. 27
октября Российский фонд прямых инвестиций направил во Всемирную организацию
здравоохранения заявку об ускоренной регистрации «Спутник V». В этот же день в
России началось производство второй вакцины от COVID-19 «ЭпиВакКорона», ее
разработал новосибирский центр «Вектор».
