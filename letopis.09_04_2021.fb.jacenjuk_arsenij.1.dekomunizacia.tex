% vim: keymap=russian-jcukenwin
%%beginhead 
 
%%file 09_04_2021.fb.jacenjuk_arsenij.1.dekomunizacia
%%parent 09_04_2021
 
%%url https://www.facebook.com/yatsenyuk.arseniy/posts/2075409559279835
 
%%author 
%%author_id 
%%author_url 
 
%%tags 
%%title 
 
%%endhead 
\subsection{Декомунізація - це не лише відновлення справедливості, а й основа боротьби з «рускім міром» і російською пропагандою}
\label{sec:09_04_2021.fb.jacenjuk_arsenij.1.dekomunizacia}
\Purl{https://www.facebook.com/yatsenyuk.arseniy/posts/2075409559279835}

\ifcmt
  pic https://scontent-mxp1-2.xx.fbcdn.net/v/t1.6435-9/170703532_2075409532613171_5243336400021933738_n.jpg?_nc_cat=101&ccb=1-3&_nc_sid=8bfeb9&_nc_ohc=oW8gD8CSSPMAX-7vr0f&_nc_ht=scontent-mxp1-2.xx&oh=5fe959fd6cfb53efbe10a2b81d942983&oe=609C42C4
\fi

Свою декомунізацію України я розпочав у 2007-2008 роках. Як очільник МЗС, а
потім голова Верховної Ради, демонтував радянські символи в приміщеннях
міністерства та залі засідань парламенту. 

Шість років тому було покладено початок системній декомунізації усієї держави.
Це була одна з ключових наших реформ після Революції Гідності й одна з моїх
давніх мрій. 

Довгих чверть століття Україна чекала на відновлення історичної пам’яті у
назвах населених пунктів, вулиць і площ. Два з половиною десятиліття на
центральних майданах українських міст височіли пам‘ятники катам, зрадникам і
негідникам, які вчиняли страшні злочини проти нашого народу. 

Декомунізація - це не лише відновлення справедливості, а й основа боротьби з
«рускім міром» і російською пропагандою, які завжди прикрашені історичними
вигадками й відвертою брехнею про минуле. Знищуючи радянські символи, ми
вибивали з рук агресора гуманітарну зброю. Це був ще один фронт протидії
кремлівським зайдам, на якому ми завдали їм нищівної і остаточної поразки. 

Політика декомунізації є великою інвестицією в майбутнє. Наші діти і нащадки
повинні зростати, знаючи правду про свій народ, про наших друзів та ворогів.
Перемоги і процвітання України неможливі без правдивих знань про нашу історію. 

Відзначу Volodymyr Viatrovych Володимира В’ятровича і співробітників
Український інститут національної пам'яті. Ця реформа набула таких масштабів і
успіху завдяки їхній наполегливості та системному підходу.  Ці зміни були б не
такими вражаючими, якби не підтримка місцевого самоврядування і тисяч
волонтерів та людей доброї волі, які разом зробили цю велику справу.
