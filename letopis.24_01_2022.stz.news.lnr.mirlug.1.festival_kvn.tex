% vim: keymap=russian-jcukenwin
%%beginhead 
 
%%file 24_01_2022.stz.news.lnr.mirlug.1.festival_kvn
%%parent 24_01_2022
 
%%url https://mir-lug.info/hot/festival-komand-kvn-studencheskij-marafon-proveli-v-luganske
 
%%author_id news.lnr.mirlug
%%date 
 
%%tags donbass,festival,kultura,kvn,lnr,lugansk,molodezh
%%title Фестиваль команд КВН «Студенческий марафон» провели в Луганске
 
%%endhead 
 
\subsection{Фестиваль команд КВН «Студенческий марафон» провели в Луганске}
\label{sec:24_01_2022.stz.news.lnr.mirlug.1.festival_kvn}
 
\Purl{https://mir-lug.info/hot/festival-komand-kvn-studencheskij-marafon-proveli-v-luganske}
\ifcmt
 author_begin
   author_id news.lnr.mirlug
 author_end
\fi

В Луганском государственном университете имени Владимира Даля 24 января
состоялся фестиваль команд КВН «Студенческий марафон». Его проведение
инициировали представители Общественной организации «Молодая Гвардия».

\ii{24_01_2022.stz.news.lnr.mirlug.1.festival_kvn.pic.1}

В фестивале приняли участие семь команд из разных городов и районов ЛНР, а
именно:

\begin{itemize}
  \item «Исключение из правил» (посёлок Червоный прапор);
  \item «Сборная Лутугино» (Лутугино);
  \item «Сборная Перевальска» (Перевальск);
  \item «Перевальский район» (Перевальск);
  \item «Невидимая сборная» (Луганск);
  \item Old school (Луганск);
  \item «Своим ходом» (Луганск).
\end{itemize}

Выступления команд оценивало жюри, в состав которого вошли депутат Народного
Совета ЛНР от Общественного движения «Мир Луганщине» Иван Санаев, советник
Министра образования и науки ЛНР Константин Кучер, начальник управления спорта
и молодёжи Министерства культуры, спорта и молодёжи ЛНР Олег Шеренешев,
председатель Общественной организации «Молодая Гвардия» Даниил Степанков,
начальник организационного отдела аппарата Республиканского исполнительного
комитета ОД «Мир Луганщине» Сергей Гербеев, чемпион Высшей лиги КВН Валдес
Романов.

\ii{24_01_2022.stz.news.lnr.mirlug.1.festival_kvn.pic.2}

Команды выступили в конкурсах «Приветствие» и «Разминка».

Особенно удивила аудиторию команда «Исключение из правил», ведь все её
участники – младшеклассники Червонопрапорской средней школы №34.

Участник этой команды Назар рассказал, что придумывать шутки им помогали
заместитель директора школы и учитель. Мальчик отметил, что участвовать в
фестивале команде понравилось, ребята от души повеселились и рады, что их
выступление понравилось зрителям.

Ещё одной особенностью «Студенческого марафона» стала команда «Невидимая
сборная» из Луганска. Участники этой команды презентовали зрителям
аудиовыступление, не выходя на сцену вовсе.

Даниил Степанков сообщил, что организаторам поступило более 20 заявок от
команд, которые хотели принять участие в фестивале. Он добавил, что команды,
которые выступили на фестивале, отобрали редакторы. Председатель Общественной
организации «Молодая Гвардия» поблагодарил участников команд за отличную
подготовку к фестивалю.

Победителем «Студенческого марафона» стала команда «Перевальск», на втором
месте – команда «Сборная Перевальского района», а на третьем – «Сборная
Лутугино». Все участники получили грамоты. Команде, занявшей первое место,
вручили сертификат на получение портативной акустической системы. За занятое
второе и третье место команды получили сертификаты на портативные колонки.

Самой юной команде фестиваля «Исключение из правил» вручили сладкий подарок и
игрушку.

– На подготовку у нас была неделя. Шутки писали сами, некоторые взяли из
предыдущих своих выступлений. Я уже давно играю в КВН и не единожды одерживал
победу. Думаю, если ты выбрал свой путь и решил играть в КВН, то это уже
навсегда. Я очень рад, что нашу команду оценили по достоинству и мы взяли кубок
за первое место. Люблю КВН за то, что он даёт возможность дарить людям радость
и искренний смех, – прокомментировал участник команды «Перевальск» Алексей
Науменко.

Валдес Романов подчеркнул, много критериев для оценивания команд у жюри не
было, главное то, насколько смешным было выступление.

Участник команды «Сборная Лутугино» Дмитрий Смольский рассказал, что в КВН
играет ещё со студенческих лет и вообще любит выступать на сцене, а шутки он
придумывает вместе со своими товарищами.

Участница команды «Сборная Перевальска» Жанна Рещик поделилась своими
впечатлениями:

– Я получила море эмоций сегодня. Выступать на сцене – это что-то невероятное,
зал аплодировал, что очень приятно. Мы немного переживали, но, когда вышли на
сцену, вовсе забыли о волнении. Приятно, что нас поддержали, мы видели, что
шутки нравятся зрителям. Спасибо организаторам фестиваля за такую возможность
проявить себя, попробовать свои силы и посоперничать с такими классными
командами. Все выступления ребят были достойными, команды действительно хорошо
подготовились.
