% vim: keymap=russian-jcukenwin
%%beginhead 
 
%%file 18_07_2020.fb.lnr.9
%%parent 18_07_2020
 
%%endhead 
\subsection{Во французском Нанте загорелся собор Святых Петра и Павла XV века}
\label{sec:18_07_2020.fb.lnr.9}
\url{https://www.facebook.com/groups/LNRGUMO/permalink/2856097951168427/}
  
\vspace{0.5cm}
{\small\LaTeX~section: \verb|18_07_2020.fb.lnr.9| project: \verb|letopis| rootid: \verb|p_saintrussia|}
\vspace{0.5cm}

Пожар возник в кафедральном соборе во французском городе Нант на западе страны,
на место происшествия выехали пожарные, сообщает телеканал BFMTV со ссылкой на
операционный пожарный и спасательный центр.

Как сообщили очевидцы, пожар начался около 7:30 по местному времени (8:30 мск).
На место происшествия направились пожарные, о причинах не сообщается.

Пожарная служба департамента Атлантическая Луара в своем Twitter отметила, что
пожарные "в настоящее время занимаются пожаром в соборе Нанта". Пожарные
призвали не подходить к месту происшествия.

Радиостанция France Bleu в свою очередь сообщает, что на месте работают около
60 пожарных и шесть пожарных машин. Периметр оцеплен.

Собор святого Петра и Павла в Нанте, строительство которого началось в XV веке
и продолжалось 457 лет, - одна из крупнейших готических церквей во Франции. По
высоте она на шесть метров ниже собора Парижской Богоматери. 
