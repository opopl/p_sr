% vim: keymap=russian-jcukenwin
%%beginhead 
 
%%file 15_10_2020.news.ru.lenta.1_covid
%%parent 15_10_2020
%%url https://lenta.ru/articles/2020/10/15/vaccine_diplomacy/
 
%%endhead 

\subsection{«Это как холодная война» Как создание вакцины от коронавируса
превратилось в инструмент международной политики}

\url{https://lenta.ru/articles/2020/10/15/vaccine_diplomacy/}

Пандемия коронавирусной инфекции стала настоящим испытанием не только для
глобальной экономики, но и для всех стран в отдельности, их правительств и
политических режимов. Спасательным кругом в нынешней ситуации должны стать
вакцины, которые остановят распространение заражения. Спустя десять месяцев
борьбы с пандемией сразу несколько вакцин находятся в заключительной стадии
клинических испытаний, а число предзаказов исчисляется миллиардами доз. За
возможность первыми получить долгожданную панацею развернулась серьезная борьба
между странами: одним вакцина поможет защитить население и спасти падающие
рейтинги правительства, другие рассчитывают усилить свои внешнеполитические
позиции и обрести новых союзников. О том, как в новом постковидном мире вакцина
становится действенным инструментом внешней и внутренней политики, — в
материале «Ленты.ру».

Мало кто догадывался, что ждет мир, когда в конце декабря 2019-го — начале
января 2020-го в сети появлялись видео из Китая, на которых люди теряли
сознание на улице из-за неизвестного ранее заболевания, а медики в костюмах
биологической защиты разгоняли зевак и уносили несчастных в неизвестном
направлении. Никто не мог подумать, что новый вирус приведет к жесткому
карантину и беспрецедентному для современного мира закрытию границ, обрушит
экономику даже самых развитых стран и унесет более миллиона жизней.

Природа глобализации такова, что опасный патоген, впервые замеченный в
китайском городе-миллионнике Ухане, стремительно распространился по разным
уголкам мира — от крошечных островных государств до поселений индейцев
Амазонки. Уже 30 января Всемирная организация здравоохранения (ВОЗ) признала
распространение коронавирусной инфекции чрезвычайной ситуацией, имеющей
международное значение, а 11 марта назвала вспышку неизвестного ранее патогена
пандемией.

Приблизительно в это же время крупные фармакологические компании и
исследовательские центры вступили в гонку по поиску и разработке вакцины от
смертельного вируса. Обычно этот процесс занимает годы (а то и десятилетия), и
лишь 10 процентов «кандидатов», как правило, не проваливают клинические
испытания и доходят до населения. Сейчас же, учитывая высокую контагиозность
вируса SARS-CoV-2 и разрушительное воздействие пандемии на глобальную
экономику, разработчики прикладывают максимум усилий для ускорения создания
вакцины, а регуляторы существенно упростили процедуру регистрации новых
препаратов. Однако повлиять на средние показатели успешности разработок гораздо
сложнее.

Более 190 вакцин от COVID-19 в настоящее время проходят доклинические и
клинические испытания

Десять из них находятся в заключительной — самой масштабной и решающей — фазе
испытаний, которая призвана окончательно ответить на вопрос об их эффективности
и безопасности, особенно в случае детей, пожилых и людей со слабым здоровьем.
Чисто теоретически одна из этих вакцин — те самые 10 процентов — и может стать
долгожданной панацеей.

Сразу четыре из 10 вакцин-лидеров гонки разрабатываются в Китае. Над еще тремя
работают американские компании — Novavax, Moderna и Pfizer (в последнем случае
— совместно с немецкой BioNTech и китайской Fosun Pharma). Кроме того, в
заключительных стадиях разработки находятся британская вакцина от компании
AstraZeneca и Оксфордского университета, российская вакцина от Национального
исследовательского центра эпидемиологии и микробиологии имени Н.Ф. Гамалеи, а
также препарат, разработанный бельгийской Janssen Pharmaceutical Companies —
дочкой американской Johnson\&Johnson.

«Гонка по разработке вакцины похожа на космическую гонку между США и Советским
Союзом... Это как холодная война», — оценил происходящее Брэд Лонкар,
основатель американского биотехнологического фонда, ориентированного на Китай.
По его словам, для китайских чиновников это, в свою очередь, «не только предмет
национальной гордости и важный шаг для защиты их собственного здоровья, но и
способ продемонстрировать свое превосходство».

Успех вакцины не просто обогатит компанию-разработчика, но и наделит
государства, которые получат ее первыми, значительным влиянием и одновременно
колоссальной ответственностью

Сколько вакцин направить в другие страны, а сколько оставить себе? Если
делиться с другими, то по какому принципу? Что требовать взамен? Ответы на эти
непростые вопросы определят имидж и геополитическое влияние стран на ближайшие
годы, сформируют новые партнерства и альянсы и усугубят существующие
противоречия.

\subsubsection{Кто первый встал, того и вакцина}

«На протяжении десятилетий одни и те же уставшие голоса предлагали одни и те же
неудачные решения, преследуя глобальные амбиции за счет собственного народа. Но
вы можете найти истинную основу для сотрудничества только тогда, когда вы
заботитесь о собственных гражданах. Как президент я отверг неудачные подходы
прошлого, и я с гордостью ставлю Америку на первое место так же, как и вы
должны ставить свои страны превыше всего. В этом нет ничего страшного — это то,
что вы должны делать», — с такими словами в разгар всемирной пандемии президент
США Дональд Трамп обратился к 75-й сессии Генеральной Ассамблеи ООН.

Лидерство в гонке за вакциной и обеспечение ею американцев в приоритетном
порядке — как и просто подобные обещания — могут стать серьезным подспорьем для
Трампа в борьбе на президентских выборах, до которых уже меньше месяца.
Особенно важно это на фоне растущей критики того, как его администрация
справляется с пандемией в стране.

«Думаете, электорат Трампа будет довольствоваться стратегией вакцинации,
которая не "сделает Америку снова великой"?» — отмечает Дэвид Солсбери, бывший
директор по иммунизации британского Министерства здравоохранения, который в
настоящее время является сотрудником лондонского аналитического центра Chatham
House.

Допустим, твоя страна разработает вакцину раньше других, но сразу же поставит
ее в другое государство, — такое явно голосов не прибавит Дэвид Солсбери

По всей видимости, исходя из этих соображений американское правительство уже
заключило предварительные соглашения о поставках вакцин с компаниями Novavax,
Pfizer, Moderna, Johnson\&Johnson и Sanofi. Причем речь идет о 100 миллионах
доз каждой после одобрения регуляторами. А с AstraZeneca власти США
договорились о поставках целых 300 миллионов доз — и это при том, что все
население страны насчитывает приблизительно 330 миллионов человек. Кроме того,
в июне Штаты скупили почти все мировые запасы препарата ремдесивир, который
используется для лечения больных COVID-19.

Дональд Трамп снимает маску на балконе Белого дома после выписки из больницы,
куда он попал из-за заражения коронавирусом

США — не единственная страна, решившая заблаговременно запасаться сразу
несколькими видами вакцин. Так, Великобритания договорилась о поставках в общей
сложности 340 миллионов доз от пяти производителей (при населении в 68
миллионов человек). Евросоюз должен будет получить 600 миллионов доз от двух
разработчиков и ведет переговоры с еще четырьмя. Япония договорилась с Pfizer и
AstraZeneca о закупках по 120 миллионов доз каждой из вакцин и обсуждает с
Moderna поставки еще 40 миллионов (при населении в 126 миллионов). А Австралия
смогла гарантировать поставки 85 миллионов доз двух видов для 25-миллионного
населения.

Аналогичные двусторонние соглашения с производителями заключает и ряд
развивающихся стран, например, Индонезия, Вьетнам и Бразилия. Индия, чья
фармакологическая компания Serum Institute of India будет производить вакцины
зарубежных разработчиков, также надеется получить порядка 500 миллионов доз к
середине следующего года.

Такой ажиотаж и острое желание правительств заблаговременно и гарантированно
договориться о поставках обусловлены не только негативным влиянием пандемии на
экономику, но и не в последнюю очередь политическими соображениями. Во многих
странах карантинные меры и растущая безработица привели к существенному падению
рейтингов правящих партий и кабминов, и вместе с очередной волной пандемии
недовольство лишь будет расти. Поэтому обещания скорейшей вакцинации и
возвращения к привычному порядку вещей становятся сейчас, пожалуй, самым
действенным инструментом зарабатывания политических очков и удержания власти.

Справедливости ради, предзаказы на число доз, существенно превышающее
численность населения, также могут быть объяснены тем, что в ряде случаев
вакцину нужно вводить в два этапа. Так, например, препарат Института имени
Гамалеи предполагает введение второй, «бустерной» дозы через несколько недель
после первой. К тому же пока нельзя сказать однозначно, надолго ли люди
получают иммунитет и как часто нужно будет прививаться.

Тем не менее в ВОЗ подобные шаги ряда государств окрестили «вакцинным
национализмом». «Необходимо в первую очередь вакцинировать некоторых людей во
всех странах, а не всех людей в некоторых странах», — сказал глава организации
Тедрос Аданом Гебреисус в начале сентября.

Если вакцинами не будут обеспечены люди в странах с низким и средним уровнем
дохода, то в нашем взаимосвязанном мире вирус продолжит убивать людей и
препятствовать восстановлению экономики во всем мире
\textbf{Тедрос Аданом Гебреисус}

По его словам, применение вакцин в качестве глобального общественного блага
отвечает национальным интересам каждой без исключения страны. «Вакцинный
национализм продлит, а не сократит пандемию», — добавил Гебреисус.

Контроль качества на производстве китайской фармкомпании Sinovac Biotech Фото:
Thomas Peter / Reuters

\subsubsection{Дипломатия вакцин}

Вместе с тем новые препараты, а точнее их поставки в приоритетном или льготном
порядке, очевидно, становятся важным инструментом дипломатии. Если раньше
традиционным проявлением благих намерений и дружеского отношения Китая к стране
было дарение панд, то сейчас милых пухлых зверюшек потеснило китайское
портфолио вакцин.

«Разработка и внедрение вакцины от COVID-19 в Китае, когда она будет доступна,
станет глобальным общественным благом. Это станет вкладом КНР в обеспечение
физической и ценовой доступности вакцины в развивающихся странах», — сказал
китайский лидер Си Цзиньпин в своем обращении к 75-й сессии ГА ООН.

Пекин уже пообещал, что страны Африки получат китайские вакцины первыми и
совершенно бесплатно. К тому же за время пандемии Китай направил на африканский
континент миллионы средств защиты и тестов, а также договорился с местными
компаниями об их производстве. Это неслучайно — в последнее десятилетие страна
инвестировала миллиарды долларов в африканскую экономику, особенно в добычу
полезных ископаемых и другого сырья. Кроме того, КНР видит в Африке динамично
растущий рынок сбыта и плацдарм для расширения своего геополитического влияния
и военного присутствия (в частности, у китайцев есть военно-морская база в
Джибути).

Помимо Африки Китай обещал направить свои вакцины в приоритетном порядке в ряд
азиатских государств — на Филиппины, в Камбоджу, Лаос, Мьянму, Таиланд и
Вьетнам. Также их доставят в те страны, где препараты сейчас проходят
клинические испытания: в ОАЭ, Бахрейн, Перу, Марокко, Турцию, Бангладеш,
Бразилию и Индонезию. А странам Латинской Америки и Карибского бассейна Пекин
обещал выдать кредит на миллиард долларов для покупки своих вакцин.

Китай, позиционируя себя как «великодушного вакцинатора» развивающихся стран,
по всей видимости, стремится вытеснить негативные ассоциации — образ очага и
даже виновника пандемии

«Правительство определенно заинтересовано в том, чтобы Китай преуспел в
производстве хорошей вакцины и чтобы многие страны ее захотели», — считает
эпидемиолог и эксперт RAND по Китаю Дженнифер Хуанг Боуи. По ее словам, «это
пойдет на пользу его дипломатии и смене нарратива вокруг COVID-19».

По мнению некоторых наблюдателей, в стремлении сделать свои разработки
«глобальным общественным благом» Китай может обойти стороной США и Австралию,
поскольку там уже не раз обвиняли Пекин в сокрытии вспышки коронавируса, что
якобы позволило ей перерасти в пандемию.

Дипломатия вакцин, по всей видимости, становится важным элементом и внешней
политики России, первой в мире зарегистрировавшей свой препарат, который сейчас
проходит заключительную стадию клинических испытаний. Буквально каждую неделю
Российский фонд прямых инвестиций (РФПИ), финансирующий разработку вакцины,
объявляет о новых договоренностях на ее поставки как в страны, традиционно
поддерживающие дружеские отношения с Москвой, так и в государства, исторически
находившиеся в орбите США.

По последней информации, Россия получила около 50 заявок на вакцину из разных
регионов мира. Были достигнуты договоренности о поставке 32 миллионов доз в
Мексику, до 50 миллионов — в Бразилию, 100 миллионов — в Индию, до 35 миллионов
— в Узбекистан, по 25 миллионов — в Египет и Непал. Кроме того, по данным РФПИ,
российский препарат будут испытывать в ОАЭ, Саудовской Аравии, на Филиппинах и
в Индии, а Белоруссия и Венесуэла уже получили первые дозы для вакцинации
добровольцев. А 14 октября было объявлено о регистрации второго российского
препарата.

Первый этап испытания вакцины от COVID-19 в США, март 2020 года
Фото: Ted S. Warren / AP

Разумеется, есть попытки предотвратить национализацию вакцин и гарантировать их
получение всеми странами одновременно на уровне международных организаций. Так,
в конце апреля ВОЗ совместно с европейскими лидерами объявила об инициативе по
ускорению доступа к средствам для борьбы с COVID-19 (ACT). Она призвана
обеспечить равный доступ всех стран к тестам, лекарствам и вакцинам.

В настоящее время более 170 стран присоединились к механизму COVAX, с помощью
которого планируется поставить два миллиарда доз вакцин в рамках ACT. Однако ей
по-прежнему остро не хватает финансирования: на экстренные расходы прямо сейчас
нужно 14 миллиардов долларов.

Однако далеко не все верят в успех этой инициативы. «Дело не только в том, что
механизму COVAX не хватает финансирования, (...) что у него нет четких
полномочий. Дело в том, что все правительства сталкиваются с огромным давлением
общественного мнения — с тем, чтобы они в первую очередь заботились о
собственных гражданах», — написал в своей колонке Ричард Хаас, президент
частной американской организации «Совет по международным отношениям».

Вакцинный национализм почти наверняка одержит победу над многосторонним подходом к вакцинам
\textbf{Ричард Хаас}

По мнению ведущих эпидемиологов, COVID-19 вряд ли станет последней пандемией,
число подобных вспышек новых патогенов будет расти, а значит — будет возрастать
роль фармакологических компаний и значимость их разработок. Весьма вероятно,
что они станут столь же влиятельными инструментами внутренней и внешней
политики государств, как в свое время социальные сети.

