% vim: keymap=russian-jcukenwin
%%beginhead 
 
%%file poetry.rus.sokor.zabvenie_god_33
%%parent poetry.rus.sokor
 
%%url http://maysterni.com/publication.php?id=136069
%%author 
%%tags 
%%title 
 
%%endhead 

\subsubsection{Забвение, год 1933}
\Purl{http://maysterni.com/publication.php?id=136069}

Помни трагичный год — тридцать третий.
Людей придали голоду и смерти.
Стон раздавался, детский плач,
В Кремле ликовал убийца-палач.

Люди спасались в поездках, прошениях.
Их домой возвращали люди в шинелях.
В надежде остались только поклоны.
Так, голодной смертью погибли миллионы.

Вспомни голодный год - тридцать третий,
Как люди просили хлеба иль смерти.
Умирал от голода стар и млад,
Так потухал семейный очаг.

И каждый день, и на утро -
Везде валялись трупы, трупы...
Где одиноко, где несколько лежали.
Там мама мёртвая, а дети рядом засыпали.

О, Земля! Ты принимала не омытых,
Рукой голода убитых,
Тела твоих детей.
Жертвы убийцы и палачей?

Скорблю о ранней гибели усопших,
Дождями слёзы лью.
Я приняла в объятья изнемогших,
Чтоб облегчить им тяжкую судьбу.

Лежу у ваших ног я плодородна.
Для жизни всем полна.
Сила я - великого, свободного народа,
На благо жизни, мира и добра.

Из воспоминаний моих родителей, трагичной жизни Украинского народа.

