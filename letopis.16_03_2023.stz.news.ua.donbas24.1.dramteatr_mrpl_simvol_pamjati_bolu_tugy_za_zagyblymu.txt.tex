% vim: keymap=russian-jcukenwin
%%beginhead 
 
%%file 16_03_2023.stz.news.ua.donbas24.1.dramteatr_mrpl_simvol_pamjati_bolu_tugy_za_zagyblymu.txt
%%parent 16_03_2023.stz.news.ua.donbas24.1.dramteatr_mrpl_simvol_pamjati_bolu_tugy_za_zagyblymu
 
%%url 
 
%%author_id 
%%date 
 
%%tags 
%%title 
 
%%endhead 

Драмтеатр Маріуполя — символ пам'яті, болю і туги за загиблими маріупольцями
(ФОТО)

16 березня російський літак скинув авіаційну бомбу на Маріупольський
драматичний театр

Рік тому відбулася жахлива трагедія, яка стала однією із найстрашніших сторінок
історії оборони Маріуполя. Й досі важко уявити, що будівля драмтеатру, яка була
справжнім серцем культурного життя та улюбленим місцем відпочинку маріупольців
і гостей міста, знищена... Маріупольський драмтеатр — одна лише згадка про
нього викликає сум та сльози у людей по всій Україні. Точну кількість загиблих
наразі неможливо встановити. У річницю трагедії важливо згадати і про історію
самої будівлі, і про ту страшну подію.

Читайте також: Окупанти знесли драматичний театр у Маріуполі (ВІДЕО)

Історія будівлі 

Насправді поява в Маріуполі Драматичного театру має своєрідну історичну долю.
Згідно із законодавством 1950-х років лише обласні міста мали право на
будівництво власного театру. Завдяки великій кількості клопотань та неабиякій
наполегливості мешканців міста на чолі з Костянтином Олейниченком (на той час
очолював міську Комуністичну партію України), врешті-решт було дозволено
будівництво театру в Маріуполі, але під назвою «Донецький обласний...».
Будівництво почалось у 1956 році за проєктом архітекторів Київського ДІПроміста
О. Малишенко та О. Крилової на місці зруйнованого більшовиками храму св. Марії
Магдалини. Театр було побудовано в стилі монументального класицизму з великою
кількістю ліпних декоративних елементів та глядацькою залою на 700 місць. Об'єм
будівлі становив 37 тис. м3. Центральну фасадну частину будівлі з портиком
коринфського ордера на квадратних колонах прикрашала гармонійна скульптурна
композиція, де чільне місце відводилося металургам і хліборобам, як основним
працівникам Приазов'я.

Читайте також: 7 липня єдиній народній артистці України в Маріуполі Світлані
Отченашенко виповнилося б 77 років... (ФОТО)

2 листопада 1960 року відбулось урочисте відкриття храму Мельпомени виставою за
п'єсою О. Арбузова «Іркутська історія». Відтоді площа у самому центрі міста
отримала назву «Театральна» та стала символом культурного життя Маріуполя:
центром родинного відпочинку, культурних подій, фестивалів та розваг. Першим
головним режисером театру став народний артист УРСР Петро Вєтров. Видатні
актори театру — народний артист УРСР і заслужений артист УзРСР Б. Сабуров,
народна артистка УРСР Р. Под'якова, заслужена артистка УРСР Є. Вєтрова,
заслужена артистка УРСР М. Алютова, заслужений артист УРСР В. Бугайов, артисти
А. Сорокко, Л. Луцький, О. Чернов, М Земцов, І. Зенков, С. Отченашенко, Г.
Гащенко, Т. Щекатурова — щоразу підкорювали вибагливих маріупольських глядачів.
Загалом Маріупольський театр впродовж 1970-х років набуває неофіційного статусу
одного з найкращих драмтеатрів України, чому сприяли гастрольні маршрути
колективу. Це був період підкорення елітних столичних вершин: Київ, Баку, Рига,
Таллінн, Вільнюс, Мінськ. І майже всіх обласних центрів України. Успіх театру
був переконливим і вражаючим — повні зали і захоплені відгуки театральної
критики.

28 грудня 1983 року будівлі драматичного театру у місті Маріуполі було надано
статус пам'ятки архітектури місцевого значення. У 1985 році у драмтеатрі була
відкрита мала зала. Причина використання малої сцени пояснювалася процесом
подальшого розвитку театру та постійними пошуками форм зближення сцени і
глядацької зали. На малій сцені актори змушені активізувати всі свої сили —
духовні і фізичні, бо міра відповідальності за виставу майже повністю лягає на
їхні плечі. А 12 листопада 2007 року наказом Міністерства культури і туризму
театру було присвоєно заслужений статус Академічного.

Читайте також: «Більше немає мого рідного театру»: Станіслав Боклан поділився
спогадами про Маріуполь

До повномасштабного вторгнення на сцені театру грали непересічні постаті. 85%
реперту ставилося українською мовою. Такі вистави, як «Слава героям», «Біла
ворона», «Майдан інферно», «Мокошева колисанка», «Маруся», «Мир Вашому дому»,
«Останній подвиг Ланцелотта» підкорювали не тільки маріупольців, але й
приїжджих українців з інших міст. У 2022 році планувалося відкрити сцену у
фойє, де відбуватиметься спілкування з глядачем. Там мали стояти столики з
кавою, чаєм, печивом. На такій сцені хотіли показувати імпровізаційні вистави,
які дозволять залучати глядачі до дійства. Формат цієї сцени повинен був
постійно змінюватися: від музичних та драматичних вистав до тематичних вечорів.
Ще була ідея відкрити сценічний майданчик під сценою. Але всім цим планам не
судилося втілитися у життя...

Серце культурного життя Маріуполя

Приміщення драмтеатру було серцем культурного життя міста та умовним центром
Маріуполя, орієнтиром як для мешканців міста, так і для туристів. Саме біля
будівлі драмтеатру маріупольці призначали побачення, проводили фестивалі та
концерти. Приміщення драматичного театру завжди було прикрасою міста. Сюди
полюбляли приходити в найбільш вишуканому одязі, щоб насолодитися театральним
мистецтвом завдяки виставам Донецького академічного обласного драматичного
театру (Маріуполь) чи приїжджим театральним колективам. У будь-яку пору року
його полюбляли фотографувати з різних ракурсів, писати з нього картини і навіть
присвячувати йому вірші. Будівля прикрашала більшість маріупольських листівок
та календарів. Біля маріупольського драмтеатру святкували День міста, Новий
Рік, тут гуляли сім'ями, годували голубів і насолоджувались відпочинком.
Особливо красиво і гармонійно будівля театру виглядала в темний час доби, або
взимку завдяки добре організованому підсвічуванню будівля органічно виглядала і
поблизу, і на відстані.

Читайте також: Укрпошта випустить марки з драмтеатром Маріуполя та скитом
Святогірська

Трагедія, яка сколихнула весь світ

З перших днів повномасштбаного вторгнення будівля драматичного театру Маріуполя
перетворилася на унікальний осередок солідарності і найбільший прихисток для
маріупольців. 5 березня маріупольці сподівалися, що біля драмтеатру стоятимуть
евакуаційні автобуси, але зрозумівши, що евакуації не буде, вони лишились у
приміщені Драматичного театру. На той час у будівлі перебувало понад 2000 осіб,
які лишились без даху над головою. Вже через декілька днів глядацька зала була
зруйнована, адже потрібно було розмістити якомога більше людей. 15 березня
театр покинуло близько 1000 осіб. Комендантом у театрі була художниця зі світла
Євгенія Забогонська, яка досконало знала будівлю, адже пропрацювала в театрі
двадцять років. Допомагав їй актор і її чоловік Сергій Забогонський. Вони
фіксували тих, хто прибув у театр, але не записували тих, хто покидав будівлю.
16 березня 2022 року близько 10 години ранку сталася трагедія, що сколихнула
весь світ — російський літак скинув надпотужну авіаційну бомбу на
Маріупольський драматичний театр, в результаті чого загинула велика кількість
людей, а будівлю майже повністю знищено. Супутникові знімки свідчать, що перед
та позаду будівлі драматичного театру в Маріуполі були написи великими літерами
«Дети» для пілотів авіації. Очевидно, що напис бачили коригувальники вогню та
пілоти російської авіації, якщо його легко було побачити із супутників.
Загибель великої кількості маріупольців та зруйнування будівлі театру
залишається однією з найбільших трагедій за весь час повномасштабного
вторгнення.

Читайте також: Місто в театрі — у Києві представили короткометражний фільм про
російський авіаудар по драмтеатру

У липні 2022 року загарбники здивували новою цинічною заявою — росіяни
оголосили, що в театрі загинуло всього 14 людей, а трупний запах виник через
схований у театрі склад протухлої риби. Спочатку окупанти обіцяли відновити
театр і навіть проводили якісь роботи. Ремонтом їх назвати складно — вони
просто заливали трупи людей бетоном, попередньо обробивши хлоркою. Згодом на
могилі сотень людей відбулися перші вистави. Поруч — проводили свята та
концерти. Росіяни цинічно вбили людей і одразу ж намагалися стерти навіть
пам'ять про невинно закатованих та загиблих маріупольців. Відбудови також не
сталося — врешті-решт театр в серці міста вирішили знести. Тепер він назавжди
залишиться символом болю, пам'яті та туги за загиблими маріупольцями.

У цей складний час історія українського Маріуполя продовжує писатися на
території, підконтрольній Україні, завдяки маріупольцям, які виїхали з
окупованого міста. І попри численні намагання окупантів знищити культурну
спадщину Маріуполя, ми мусимо зберегти нашу пам'ять, культуру та історію, яка
створювалася не одне століття. Пам'ятати про історію будівництва драматичного
театру, про важливість цієї будівлі для маріупольців та трагедію, яка сталася
рік тому — наш обов'язок. Адже займатися збереженням національної пам'яті —
сьогодні є одним з головних завдань кожного українця.

Раніше Донбас24 розповідав, що стартап Skeiron відтворив будівлю зруйнованого
Маріупольського драмтеатру.

Ще більше новин та найактуальніша інформація про Донецьку та Луганську області
в нашому телеграм-каналі Донбас24.

Фото: з архіву Ольги Демідко та відкритих джерел
