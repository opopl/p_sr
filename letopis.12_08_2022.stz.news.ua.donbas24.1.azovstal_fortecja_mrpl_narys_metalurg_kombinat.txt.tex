% vim: keymap=russian-jcukenwin
%%beginhead 
 
%%file 12_08_2022.stz.news.ua.donbas24.1.azovstal_fortecja_mrpl_narys_metalurg_kombinat.txt
%%parent 12_08_2022.stz.news.ua.donbas24.1.azovstal_fortecja_mrpl_narys_metalurg_kombinat
 
%%url 
 
%%author_id 
%%date 
 
%%tags 
%%title 
 
%%endhead 

Алевтина Швецова (Марiуполь)
12_08_2022.alevtina_shvecova.donbas24.azovstal_fortecja_mrpl_narys_metalurg_kombinat
Маріуполь,Україна,Мариуполь,Украина,Mariupol,Ukraine,Азовсталь,Azovstal,History,Історія,История,date.12_08_2022

«Азовсталь» — фортеця Маріуполя: нарис про металургійний комбінат (ВІДЕО)

12 серпня 2022 року комбінату «Азовсталь» виповнюється 89 років.

Азовсталь... До повномасштабного вторгнення — металургійний гігант і
містоутворююче підприємство, зараз — символ непокори, мужності та незламності.
Донбас24 до дня народження «Азовсталі» пропонує вам більше дізнатися про
історію комбінату, його спадщину та значущість в обороні Маріуполя.

З чого починався комбінат «Азовсталь»

Дуже часто металургійний гігант «Азовсталь» називають містом в місті (через те,
що площа «Азовсталі» складає близько 11 квадратних кілометрів), але майже сто
років тому, коли побудова заводу лише планувалася, ці землі знаходилися поза
межами тодішнього Маріуполя.

«Азовсталь» розташовується на лівому березі річки Кальміус і водночас на березі
Азовського моря. Будівництво заводу тісно пов'язане з курсом на
індустріалізацію, який почав проводитися урядом СРСР з кінця 20-х років. У
лютому 1930 року Президія вищої ради народного господарства СРСР прийняла
рішення про будівництво нового металургійного заводу в Маріуполі, оскільки саме
це місто замикало найкоротший шлях морем руди з Камиш-Бурунського родовища на
Керченському півострові.

7 листопада 1930 року почалося укладання бетону в фундамент першої доменної
печі заводу «Азовсталь» на лівому березі річки Кальміус. На будівництво
«Азовсталі» було виділено 292 млн радянських рублів, в роботу були залучені
найкращі спеціалісти того часу. Навіть Серго Орджонікідзе особисто приїжджав до
Маріуполя 2 лютого 1933 року, аби контролювати процес будівництва. Згодом район
Маріуполя поблизу «Азовсталі» буде названо Орджонікідзевським, а біля
заводоуправління «Азовсталі» довго стоятиме металевий монумент Орджонікідзе.
Але все це буде десятиліття потому.

12 серпня 1933 року доменна піч № 1 видала перший чавун.

Цей день вважається днем народження «Азовсталі».

Першим керівником майбутнього металургійного гіганта був Яків Гугель.

Він зробив вагомий внесок в будівництво та перші визначні роки роботи
«Азовсталі», але попри це, доля його склалася трагічно — у 1937 році Якова
Гугеля розстріляли в застінках НКВС.

Але повернемося до визначних дат в історії заводу:

• 1934 рік — в експлуатацію введено доменну піч № 2;

• 1935 рік — почало діяти сталеплавильне виробництво, вступила в дію перша в
СРСР 250-тонна мартенівська піч, що гойдається;

• 1939 рік — на «Азовсталі» встановлено світовий рекорд із продуктивності
доменної печі на добу, домна № 3 видала 1614 тонн чавуну за один день;

• 1941 рік — на заводі діяли 4 домни і 6 мартенівських печей, будувалися
прокатні цехи.

Під час Другої світової війни у зв'язку з наближенням до Маріуполя лінії
фронту, у період з 12 вересня по 5 жовтня 1941 року колектив працівників провів
демонтаж, відвантаження та вивіз на Урал обладнання, що зайняло 650 вагонів.

У жовтні 1941 року Маріуполь було окуповано. Завод «Азовсталь» став називатися
«Азовським заводом № 1» і був переданий в управління концерну Альфріда Круппа.
Окупанти змогли відновити електростанцію, механічний, монтажний,
електроремонтний, кисневий цехи «Азовсталі», але відновити мартенівські печі та
ввести підприємство в повноцінну експлуатацію німцям було не під силу.

7 вересня 1943 року, перед відступом німецьких військ із міста, «Азовсталь» був
практично повністю знищений. Загальні збитки, завдані заводу, становили 204 млн
рублів. Вже після 1945 року «Азовсталь» відновили, розширили та провели
модернізацію виробництва.

19 листопада 1945 року — видала свою першу плавку відновлена мартенівська піч
№ 1.

1946 рік — відновлення роботи домни № 4.

1948 рік — введення в експлуатацію блюмінга та рейкобалкового цеху (завод став
підприємством з повним металургійним циклом).

1949 рік — введення в експлуатацію ще двох домен і аглофабрики.

У 50-ті роки:

• запрацювали ще дві домни,

• почав роботу цех рейкових скріплень,

• аглофабрика заводу «Азовсталь» виготовила перший власний агломерат,

• введений в експлуатацію великосортний прокатний цех,

• вперше в СРСР на заводі «Азовсталь» було освоєно виробництво залізничних
рейок завдовжки 25 метрів.

На початку 1960-х років шлаки підприємства почали переробляти для отримання
будівельних матеріалів. Почала зростати на березі Азовського моря відома
азовстальська шлакова гора.

60-ті та 70-ті роки запам'яталися розбудовою заводу, проведенням на базі
«Азовсталі» наукових конференцій і конкурсів молодих спеціалістів з багатьох
країн, нагородженням заводу почесними відзнаками за досягнення високих
показників у виробництві. Також були введені в експлуатацію доменні печі № 5 і
6, стартувала робота унікального листопрокатного комплексу стану «3600»,
запрацювала перша черга киснево-конвертерного цеху.

Під час цієї великої розбудови підприємства з'явилися й бомбосховища на
«Азовсталі», це обумовлювалося постійною готовністю СРСР до війни. А
бомбосховища б могли забезпечити безперервне функціонування металургійного
виробництва навіть під час воєнних дій.

У травні 1984 року завод «Азовсталь» було перетворено на металургійний
комбінат.

Після проголошення незалежності України комбінат «Азовсталь» продовжував
потужно працювати та розвиватися: завдяки сертифікації продукції у 90-х роках
минулого століття азовстальські прокат і сталь вийшли на світовий ринок. У
нульових роках нового тисячоліття були впроваджені нові технології та
модернізація виробництва, що в майбутньому підвищило конкурентоспроможність
продукції.

Металургійний комбінат «Азовсталь» був єдиним виробником рейок в Україні. До
речі, приготуйтеся до приголомшливого факту — у 2001 році виторг «Азовсталі»
становив понад 500 млн доларів США!

У 2005 році відбулося злиття комбінату з маріупольським коксохімічним заводом
«Маркохім».

У 2006 році комбінат «Азовсталь» увійшов до «Дивізіону сталі та прокату групи
"Метінвест"», акціонером якої є фінансово-економічна група System Capital
Management (власник Рінат Ахметов).

Цікаві факти про «Азовсталь»

При побудові заводу «Азовсталь» у 30-тих роках минулого століття було виявлено
маріупольський могильник, датований 3 тисячоліттям до нашої ери. Згодом були
проведені розкопки та винайдені унікальні артефакти та поховання епохи
енеоліту.

Історія кожного цеху комбінату «Азовсталь» заслуговує на окремий літопис. Адже
усі заводські підрозділи за часи свого існування мають визначні дати, а головне
— пов'язані з долею людей, які «підіймали завод». Однією з таких визначних
постатей був Володимир Лепорський. З 1938 року він працював на «Азовсталі»
спочатку начальником зміни в мартенівському цеху, з 1955 року — головним
інженером, а з 1956 року до 1981 — директором заводу «Азовсталь». Саме
Володимир Лепорський визначив міць комбінату, традиції, характер і перспективи
на десятиліття вперед. А ще — ідею створити стислий і впізнаваний логотип
«Азовсталі», в колективі підхопили цю думку і розробили знак «АС», який
асоціюється з комбінатом.

Окрім того, були на заводі і виробництва, які не дуже притаманні металургії.
Мова йде про цех виготовлення кришталю. Свого часу азовстальський кришталь був
дуже популярним! Келихи, чарки, вази — ще в середині 2000-х не було кращого
сувеніра з Маріуполя. А у 80-х роках кришталеві келихи були вищим показником
добробуту в домі. 

Кіноісторія Маріуполя також щільно пов'язана з комбінатом «Азовсталь». Вперше
кіношники прибули до міста у 1949 році, коли в мартенівському цеху заводу
«Азовсталь» розпочалися зйомки епізодів для кіноепопеї «Падіння Берліна». Також
на «Азовсталі» знімали художні фільми «Вогонь», «Біле коло», «Найспекотніший
місяць», «Велика розмова». Можна побачити «Азовсталь» і у сценах фільму
«Маленька Віра», який часто ототожнюють з Маріуполем.

Але будемо відвертими: чи велика кількість людей впізнала «Азовсталь» у
перелічених кінокартинах? Ні. Натомість фільм українського захисника Дмитра
Козацького з позивним Орест, який він зняв за день до виходу з заводу у травні
2022 року, облетів весь світ.

Цей фільм-прощання показує, на що перетворився один з азовстальських цехів
після обстрілів рашистів. Орест разом з іншими українськими воїнами провів на
«Азовсталі» 84 дні.

«Азовсталь» і Маріуполь

Розбудова відомого заводу безпосередньо відображалася в історії Маріуполя та
зведенні в місті спортивних, культурних, житлових об'єктів.

Яхт-клуб «Азовсталь». Він з'явився на мапі міста у 1959 році за ініціативою
тодішнього директора «Азовсталі» Володимира Лепорського. Згодом яхт-клуб став
міським водноспортивним комплексом, де до початку повномасштабного вторгнення
розвивався парусний спорт, веслування на ялах, вейкбордінг, а територія
колишнього яхт-клубу стала майданчиком для проведення міських заходів. І це ще
не все, адже найпопулярніший пірс східного узбережжя знаходився саме в цій
локації. До тимчасової окупації міста російськими військами це було справжнє
місце сили для маріупольців і гостей міста!

Центр сучасного мистецтва «Готель Континенталь». Саме тут з 1933 року
розміщувався Клуб металургів заводу «Азовсталь». У роки окупації Маріуполя
німцями будівля була зруйнована, після відновлення в ній почав працювати Палац
культури і техніки металургійного комбінату «Азовсталь», який із 2010 року став
Палацем культури «Молодіжний». На жаль, циклічність історії призвела до
руйнування визначної будівлі знову — рашисти зруйнували «Готель Континенталь» у
березні 2022 року. Зараз архітектурна пам'ятка Маріуполя перетворена на руїни.

Маріупольська камерна філармонія. Будівля була зведена у 1958 році і мала назву
Палац культури коксохімічного заводу. Для дітей працівників комбінату
«Азовсталь» це місце свого часу було пов'язане зі спогадами про новорічні свята
і хороводи. Продовжуючи аналогію з сьогоденням, варто підкреслити, що під час
обстрілів російською армією Маріуполя в філармонії знаходили прихисток тисячі
містян — культурний заклад працював як бомбосховище. А після захоплення міста
рашисти подейкують, що саме в цій будівлі вони проводитимуть показові судилища
над українськими захисниками.

Кооператив «Азовсталець». У 80-х роках на Лівобережжі з'явився гаражний
кооператив біля моря. Працівники заводу отримували тут ділянки, де здебільшого
розташовували гаражі не з метою зберігання автівок, а для відпочинку —
риболовля, човники, дозвілля на Азовському узбережжі.

Одною з головних міських артерій Лівобережжя є вулиця Азовстальська. Які
об'єкти можна тут побачити? Починається Азовстальська вулиця зі Східних
прохідних комбінату «Азовсталь», неподалік розташований стадіон «Азовсталь», ще
тут є Центр культури «Лівобережний», навчальні заклади (школа № 10 і
школа-інтернат № 2), ранковий ринок «Привоз» і пам'ятник будівельникам. До 24
лютого 2022 року це був звичайний житловий район, після — суцільний жах і
руйнування. Мешканців на цій вулиці майже не лишилося.

«Азовсталь» — фортеця Маріуполя

Після початку АТО у 2014 році велику увагу було приділено бомбосховищам і
підземним тунелям на території «Азовсталі». В інтерв'ю ВВС генеральний директор
«Азовсталі» Енвер Цкітішвілі розповів, що на комбінаті налічується 36
бомбосховищ, в яких одночасно може укриватися близько 12 000 осіб.

Після 24 лютого 2022 року, коли на околицях Маріуполя вже були чутні обстріли
росіян, прийняли рішення відкрити прохідні «Азовсталі», аби забезпечити вільний
доступ цивільних до бомбосховищ. На той час в укриттях був зроблений запас
питної води та сухпайків.

Після повномасштабного вторгнення росії були підняті радянські архіви
будівництва азовстальських бомбосховищ і виявлено, що 5 з них є найбільш
захищеними від ворожих атак (можуть витримати навіть потрапляння одного прямого
ядерного удару).

Чому цивільні шукали прихистку саме в укриттях «Азовсталі»? Можливо, і через
те, що з трьох боків «Азовсталь» оточено водою — річка Кальміус і Азовське море
не дозволяли російській армії одразу захопити територію комбінату, перебування
в азовстальських бомбосховищах здавалося цивільним маріупольцям надійнішим,
аніж перебування у підвалах житлових будинків.

На початку березня російські війська оточили місто, заблокувавши можливість
покинути територію заводу. Тоді про подвиг і оборону Маріуполя говорили у
всьому світі. Мужні воїни «Азову», 36-ї бригади морської піхоти, прикордонники,
поліцейські, бійці ТрО та співробітники СБУ тримали оборону на території
металургійного гіганта. 

До середини травня росіяни зосередили всі свої сили на знищенні останньої
цитаделі українців в Маріуполі. Обстріли «Азовсталі» не припинялися ні на
хвилину: колосальна кількість авіаударів, фосфорних бомб, ракет, усіх видів
артилерії припадала на територію комбінату. Увесь світ страждав від безсилля та
відчаю, люди в багатьох країнах виходили на мирні акції з закликом врятувати
цивільних і військових з «Азовсталі». «Help Mariupol, help Azovstal», —
закликали зі сцени Євробачення 2022 Kalush Orchestra.

6 травня почалася евакуація цивільних з «Азовсталі», а 16 травня почалася
операція з порятунку військових, які були заблоковані на території заводу
«Азовсталь». За словами Ганни Маляр, евакуація військових проходила для
подальшого повернення українських захисників і захисниць додому, але цинізм і
жорстокість росіян після теракту в Оленівці 29 липня 2022 року ставлять під
сумнів усі угоди, домовленості та гарантії з боку міжнародних організацій.

Історії героїв і героїнь «Азовсталі» пролунали на весь світ. Їхні трагічні
долі, мужність і сміливість, одруження в азовстальському бункері, надія та віра
день за днем вражали та змушували затамовувати подих. Варто не забувати, що
порятунок героїв «Азовсталі» продовжується і зараз, в той час, коли окупанти
цинічно влаштовують на руїнах металургійного гіганта недолугі показові
концерти. 

Які сторінки стануть наступними в історії видатного підприємства — покаже час.
Але його розквіт, відродження та відбудова з попелу можлива лише під
українським прапором.

Нагадаємо, раніше Донбас24 розповідав, як по фото «азовця» створили картину за
номерами — всі кошти з продажу ідуть на волонтерство.

Ще більше новин та найактуальніша інформація про Донецьку та Луганську області
в нашому телеграм-каналі Донбас24.

ФОТО: малюнок на обкладинці АrtLizabeta; pastvu.com, з відкритих джерел
