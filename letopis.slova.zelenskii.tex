% vim: keymap=russian-jcukenwin
%%beginhead 
 
%%file slova.zelenskii
%%parent slova
 
%%url 
 
%%author 
%%author_id 
%%author_url 
 
%%tags 
%%title 
 
%%endhead 
\chapter{Зеленский}
\label{sec:slova.zelenskii}

Костянтине, а що Ви хочете від неосвіченого самодура, який не запнувшись ТРИЧІ
прочитав з папірця: \enquote{КонцерТ Оборонпром} (виступ при представлені Голови
Харківської ОДА у 2019)?
Він уяви не має що таке наукове середовище. Не вважаю \emph{О. С. Зеленського} світочем
науки, але навіть він міг йому міг пояснити ідиотизм задумки,
коментар, Lon Lon, \textbf{Щодо вчорашнього указу про \enquote{Президентський університет}}, 
Костянтин Матвієнко, pravda.com.ua, 01.06.2021

\emph{Зеленський} і його оточення теоретично не спроможні на ретельну наполегливу
щоденну ефективну, а головне, СИСТЕМНУ роботу. Їм в усьому потрібен хайп – по
щучому \enquote{велению}, по їхньому \enquote{хотенію}, не злазячи з печі, в'їхати у щасливе
майбутнє. Мрійники з горщика, з правом на помилку))))
У даному випадку ці зелепухи у страшному сні не знають, яка величезна і
непроста робота необхідна для утворення пристойного вишу. Схоже на те, що
Зеленський повірив у те, що він \enquote{найвеличніший хтось-там сучасності}. Підписав
височайший указ, зайняв грошенят в Європах – і все, з печі можна не злазити до
наступного хайпу)))
коментар, \textbf{Щодо вчорашнього указу про \enquote{Президентський університет}}, 
Костянтин Матвієнко, pravda.com.ua, 01.06.2021

Президент \emph{В. Зеленський} отримав унікальну в українській історії можливість
провести швидкі та ефективні реформи. Натомість його політична сила і він сам
грузнуть у внутрішніх конфліктах, примітивному меркантилізмі та
некомпетентності команди, що змушує їх до намагань відтворювати модель взаємин
влади і суспільства за прикладом сумнозвісних попередників,
\textbf{Знову \enquote{Україна не Росія}, проте багатьом дуже хочеться стерти цю відмінність},
Костянтин Матвієнко, pravda.com.ua, 02.02.2021

В этом вопросе проявилась еще и некомпетентность \emph{Зеленского} и его советников,
которые при подготовке к интервью не подсказали ему, что экспортные поставки
немецкого оружия – это крайне чувствительная и тонкая тема. После второй
мировой войны Германия систематически отказывается посылать свои войска за
пределы страны даже на просьбы США. С экспортными поставками оружия – все еще
более точечно и чувствительно. В ежегодном отчете правительства Германии по
экспорту оружия есть специальный раздел под названием «Отказано в сделках». Там
можно найти очень четкий перечень оснований, которые являются причиной решения
об отказе в продаже оружия. Например, «Проблемы с соблюдением прав человека»,
«Внутреннее напряжение или вооруженные конфликты в пределах страны»,
«Возможность использования вооружения внутри страны или продажи его за ее
пределы по неприемлемым условиям». Попадаем мы под эти перечисленные критерии?
Так почему Офис и советники \emph{Зеленского} не удосужились изучить вопрос и
подготовить гаранта, чтобы через пару часов после опубликования интервью он не
получил публично по носу от МИД Германии?,
\textbf{Это странно, но Зеленский совершенно не умеет давать интервью}, Светлана Крюкова, 
strana.ua, 03.06.2021

Тогда почему Зеленский для себя решил, что Германия должна мощно заниматься
переоснащением нашей армии, а не своей? Общий вывод – к интервью и
отечественным, и особенно крупным глобальным по охвату аудитории мировым СМИ
нужно очень серьезно готовится. Для них ответы в стиле КВНовских и квартальных
экспромтов категорически недопустимы. Если у нас - прокатит, то там - нет,
\textbf{Это странно, но Зеленский совершенно не умеет давать интервью}, Светлана Крюкова, 
strana.ua, 03.06.2021

Это странно, но \emph{Зеленский} совершенно не умеет давать интервью. Он и
высказаться толком не умеет, и что говорить, не понимает. Прочла интервью
\emph{Зеленского} немецкому изданию. Это за него Мендель отвечала, как я поняла?
Изучила текст по форме и по сути. Пара наблюдений,
Светлана Крюкова, strana.ua, 03.06.2021

Мягкость к олигархам, \emph{Зеленский} как будильник, налог на Google..., Денис
Безлюдько, strana.ua, 03.06.2021

\emph{Зеленский} обнажил еще одну свою ипостась – работать будильником для высшего
звена госслужащих. По его словам, он каждый день просыпается рано в 5:30 или
6:00 и знакомится с новой информацией, после чего звонит министрам. Иногда
будит их даже раньше, чем сигналы телефонов: «Иногда я даю свои сигналы ранее
их iPhone». \emph{Зеленский} сказал, что они уже привыкли. Вообще отличная услуга. Я в
гостинницах обычно такие заказываю. Я бы на месте Портрета деньги бы брал за
побудку. Большие,
\textbf{Мягкость к олигархам, \emph{Зеленский} как будильник, налог на Google...}, Денис
Безлюдько, strana.ua, 03.06.2021

Я часто потом троллил госслужащих запросами о выписках из журнала ))) Ржал как
конь! Жаль отменили этот ИДИОТИЗМъ... А было весело. Но видимо \emph{Зелик}
вернет мне мое счастье ))) НацДержСлужба даже разъяснение официальное тогда
распространила!,
\textbf{В законе об олигархах всплыл старый идиотизм},
Михаил Чаплыга, strana.ua, 04.06.2021

Это крайне интересный поворот. Получается, что у \emph{Зеленского} по свистку
из США готовы лишать гражданства украинцев, которые не угодили американцам. А
также - наложить на них дополнительные санкции.  Судя по всему, в первую
очередь эта практика может ударить по тем бывшим регионалам, кто после Майдана
уехал в политическую эмиграцию. И сейчас критикует украинскую власть, 
\textbf{Врагам Америки пересмотрят гражданство Украины. Главные решения нового СНБО}, 
Максим Минин; Игорь Рец, strana.ua, 04.06.2021

\enquote{Нужен Песков на минималках}. Кого и как искал Офис президента на должность пресс-секретаря Зеленского,
Анастасия Товт, strana.ua, 04.06.2021

Вместо реальной борьбы с олигархатом обществу предлагается очередной перфоманс,
который \emph{Зеленский} устраивает для того, чтобы отвлечь внимание от того
позорного факта, что ни одного предвыборного обещания он не выполнил — «эпоха
бедности» не преодолена, никакого роста экономики и занятости не произошло от
слова совсем, тарифы на коммуналку рвутся вверх, а к зиме в тарифной вакханалии
может утонуть подавляющее большинство населения. И это уже не говоря о так
называемом \enquote{рынке земли} с 1 июля, который может свестись к
экспроприации земельных наделов мелких собственников под вывеской так
называемой консолидации. При этом, насколько можно понять, \emph{Зеленский}
хочет пойти на второй срок, а еще больше его второго срока хочет его окружение,
причем каденция уже подходит к экватору. В этих условиях \emph{Зе} и Ко в
привычной для себя манере пытаются раскрутить красочное шоу и переключить
внимание на «борьбу с олигархами», 
\textbf{Народу предложили пиар-клоунаду потешной Зе-деолигархизации},
Александр Карпец, strana.ua, 07.05.2021

Хочу звернутись до \emph{Зеленського} напряму! Навіщо ви все знищуєте?! Зупиніться! - Тітаренко,
\url{https://www.youtube.com/watch?v=zbt9t4BJywE}, 06.06.2021

Президент \emph{Владимир Зеленский} анонсировал «Большую посадку». В прямом смысле
этого слова. Деревьев то бишь. Видимо это и имелось в виду на предвыборных
бордах «Весна прийде – саджати будемо». \emph{Гарант} подписал указ о сохранении и
восстановлении лесов. В рамках этой программы за три года должны высадить
миллиард деревьев. Это где-то 10 деревьев в секунду. Как писал Лесь
Подеревянский в пьесе «До хуя масла»: «Скорость охуевающая». В целом, \emph{президент}
в рамках программы «Зеленая страна» хочет увеличить площадь лесов на 1 млн
гектаров,
\citTitle{Большая посадка, беспритульный Голос, отступающий коронавирус...}, Денис Безлюдько, strana.ua, 07.06.2021

\emph{Зеленский} - это антиномия по-украински. Точнее, абсурдопедия, Андрей Головачев, strana.ua, 08.06.2021

Неприятное удивление Байденом и обида из-за НАТО. О чем новое интервью \emph{Зеленского} американцам,
Максим Минин, strana.ua, 07.06.2021

\enquote{Удивлен и разочарован}.  Начал Зеленский с удивления решением Байдена
не вводить санкции против \enquote{Северного потока - 2}. Президент Украины
заявил, что газопровод – это \enquote{оружие}. И удивился, что пули к нему
помогают отливать США.  \enquote{Неожиданно, что пули к этому оружию
предоставила такая великая страна, как Соединенные Штаты. Потому что это
образцовая цивилизация, самая образцовая демократия в мире}, – заявил
Зеленский.  Пассаж о самой демократической цивилизации – не первый у
Зеленского. По иронии судьбы, он заявлял то же самое после штурма Капитолия
сторонниками Трампа. Только в поддержку Байдена. А сейчас Зеленский начал
использовать тот же посыл для осторожной критики американского президента,
\citTitle{Зеленский интервью Axios – зрада по Северному потоку, НАТО и встрече с Путиным}, Максим Минин, strana.ua, 07.06.2021

\begin{itemize}
\item - На пресконференції \emph{Зеленський} сказав: «Якщо засудили незаконно,
то ви знаєте, я буду серйозно говорити з Аваковим, чи він може
залишитися в такій ситуації. І він це знає, і ми розуміємо
складність цього процесу». Тобто тепер Аваков «повинен
закінчити справу Шеремета».
\item - Ну ось так от.
\item - Коли все це сталося, ви були готові до найгіршого варіанту розвитку подій, тобто до позбавлення волі на довгі роки?
\item - А чому я маю бути до цього готовий, якщо ані я, ані дівчата не причетні до вбивства? Хай краще ті, хто вигадав нашу історію, думають, як будуть з неї виходити.
\end{itemize},
\citTitle{Справа Шеремета - Ріффмастер Андрій Антоненко розповів подробиці в інтерв'ю Фактам}, 
Ольга Бесперстова, fakty.ua, 06.04.2021

А сейчас, говорят, \emph{пан президент} настолько голову потерял (из-за чего бы
это?), что уже даже начал угрожать своим старшим хозяевам! То Макрона с Меркель
обвинит, что слабо стараются на благо Украины, то Джо из-за океана упрекает,
что тот ему свидание не назначил перед встречей с Путиным. Даже не позвонил,
представляете, чтобы посоветоваться с \emph{мудрым правителем Украины}? А ведь
\emph{молодой президент} так на них надеялся, так им верил! Не читал он,
видимо, предисловие к «Герою нашего времени» Михаила Лермонтова, где тот всю
подноготную дипломатии раскрывает. Поверил \emph{пан Зеленский}, что его не
только на раз-два в НАТО и ЕС примут, но и деньгами завалят, солдат и пушек
дадут, чтобы украинский «гетьман» прошел победным маршем от Львова до
Симферополя и Донецка. А сейчас, как понял, что для каждого правителя своя
страна – ближе к телу, тотчас разнюнился, стал возмущаться: вы же обещали!,
\citTitle{Форма не главное – главное содержание! И в спорте тоже...}, Мысли Бабы Яги, zen.yandex.ru, 07.06.2021

Если им это поможет побеждать, нехай бегают. Нам от этого ни холодно, ни жарко.
Всё-равно, это останется для них лишь мечтой!  А \emph{Зе} до того жалок, что аж
противно его видеть и слышать. С шавками мировые проблемы не обсуждают!,
\citComment{Елена},
\citTitle{Форма не главное – главное содержание! И в спорте тоже...}, Мысли Бабы Яги, zen.yandex.ru, 07.06.2021

Очень правильные слова говорит Титаренко, но пока этих гос.воров не начнут
сажать вместе с  вором, аферистом, проходимцем, клятым вредителем, мафиозе,
\emph{зеленским}, добра в стране не будет!!!,
\citComment{Anatolii}, 
\citTitle{Cpoчнo! PE3KOE oбpaщeниe Tитapeнкo к 3eлeнcкoмy ШOKИPOBAЛO Kиeв!},
youtube.com, 07.06.2021

Бондаренко: У \emph{Зеленського} немає ні серця, ні совісті! Йому все одно, що буде з цією країною!,
телеканал ZIK, youtube.com, 08.06.2021

Президент Украины \emph{Владимир Зеленский}, который днем ранее предложил посадить
миллиард деревьев за три года, выступил с предложение создать в Украине 1551
парк. Об этом глава государства заявил на заседании Палаты регионов Конгресса
местных и региональных властей в Днепре во вторник, 8 июня,

И ладно бы, только дороги. \emph{Жулье Зеленского} теряет берега, попутно держа нас с
вами за натуральных имбецилов. Каждую новую неделю они нам озвучивают очередную
схему по распилу денег. При этом схемы такие сумасбродные, что на них не
повелись бы в аналогичной ситуации даже такие прожженные аферисты вроде
Порошенко.  Вот просто давайте представим ситуацию. На дворе какой-то 16 год и
условный Сережа Березенко заносит шефу потустороннее предложение – а давайте
забабахаем Президентский университет. Да, первые вопросы были бы – где мой
интерес и какая у меня там доля? Но даже Порошенко затем задавал бы какие-то
предметные вопросы. А какой бюджет? 7 ярдов! Откуда? Из бюджета? Сережа, ты
свихнулся? А кого там будем готовить? Управленцев? Подожди, но у нас уже есть
Академия госуправления при Президенте Украины. Что-что? Еще там будем готовить
специалистов по аэронавтике? А нахера они там нужны, у нас ведь уже есть НАУ с
точно таким же факультетом... ,
\citTitle{Эта власть построила нам кунсткамеру в смартфоне}, Игорь Лесев, strana.ua, 09.06.2021

%%%cit
%%%cit_pic
%%%cit_text
Даже супруга \emph{Зеленского} для меня чистая копия супруги воришки-завхоза, которую
тот ласково кликал Сашхен. Она всегда рядом и готова поддержать мужа во всех
начинаниях, помогает создать образ этакого рубахи парня то ли в косоворотке, то
ли в вышиванке. Когда западных политиков председатель-завхоз Володимир с
супругой приглашают за стол «отобедать, чем бог послал», так и вижу эти круглые
личики со стеснительными улыбками и умилением на лицах, предлагающие
\enquote{правильный} украинский борщ и исключительно \enquote{украинские} вареники
%%%cit_title
\citTitle{Застенчивая улыбка Зеленского похожа на улыбку \enquote{голубого воришки} Альхена. 
\enquote{Ты кому Украину продал, жулик?!}}, Моя Поляна, zen.yandex.ru, 01.06.2021
%%%endcit

%%%cit
%%%cit_pic
%%%cit_text
Блогер Сергей Черкасский замечает, что вчера все уже забыли про политику, но
\emph{Зеленский} о ней напомнил. \enquote{\emph{Зеленский} после матча: 
\enquote{Каждый украинец был с вами,
дальше только победа, потому что то, что написано у вас на форме, не стереть и
не убрать. Это про Украину, это про вас}.  И тут я почувствовал всю мерзость
ситуации. Вот тут я почувствовал насколько далеко находится президент от
обычного народа.  Вчера вечером все реально забыли про политику! Хорошая игра
сборной Украины реально откликнулась во всех. Было приятно наблюдать за
ребятами. Вся моя ватная лента, все русские не написали ни одного плохого
слова! Только объективная оценка игры и похвала парней.  Я даже как-то забыл
про политику, забыл про колхозников, сильно болеющих за Бельгию... Правда, до
сегодняшнего утра, пока не проснулся и не увидел мерзкий пост этого, я не знаю,
как его обозвать, чтобы не присесть потом на 15 лет.  Но вот он стал тем, кто
заинтересован в конфликте. Никто вчера не вспоминал про эту форму, а ему нужно
было вспомнить.  Потому что ему нужно, чтобы конфликт продолжался}
%%%cit_comment
%%%cit_title
\citTitle{Украина - Нидерланды на Евро 2020. Отзывы на матч в Украине и России}, 
Оксана Малахова, strana.ua, 14.06.2021
%%%endcit

%%%cit
%%%cit_pic
%%%cit_text
\emph{Зеленский} любит врать, как в Украине никто не запрещает русский язык. Но на
самом деле при нём с 1 сентября 2019 запрещено среднее образование на русском
языке, а с 16 января — всю сферу обслуживания перевели на украинский язык.  И
вот, с 16 июля, все фильмы и сериалы, снятые на русском языке, закон обязывает
перевести на украинский. И так бы и произошло, если бы не \enquote{Квартал 95}
\emph{Зеленского}, который снимает все свои фильмы и сериалы только на русском языке.
Это же не людишки, а личный карман.  Поэтому \enquote{слуги} \emph{Зеленского} уже 15 июня
проголосуют \enquote{за} закон, который отменит обязательный перевод фильмов и сериалов
на украинский язык. Ничто не должно мешать \emph{Зеленскому} снимать фильмы для России
и продавать их на телеканалах, в отношении которых он лично ввёл санкции. А на
людей ему плевать
%%%cit_comment
%%%cit_title
\citTitle{Во вторник Слуги народа проголосуют против обязательного перевода фильмов на украинский язык}, 
Александр Скубченко, strana.ua, 13.06.2021
%%%endcit

%%%cit
%%%cit_pic
%%%cit_text
А чи цікавить суспільство, хто підставив президента \emph{Зеленського} щодо участі
президенського полку при захороненні колишнього есесівця?! Та суспільству
усерівно, хто підставив президента. Це їх внутрішня кухня. Інститут нацпамяті!
Апарат президенського офісу - пів тисячі осіб, купа начальників, які не роблять
своєї роботи! От що головне!!!! Президент \emph{Зеленський} у цій історії виглядає
антигуманистично і антиконституційно! От що найважливіше! Це принципово!!!
%%%cit_comment
%%%cit_title
\citTitle{Президентский полк участвовал в захоронении бывшего ССовца}, 
Марина Ставнийчук, strana.ua, 14.06.2021
%%%endcit

%%%cit
%%%cit_pic
%%%cit_text
\emph{Заленский}, кстати, когда стал Президентом, повторил красивую фразу Рейгана:
\enquote{Правительство не может решать проблемы, правительство и есть сама
проблема} Произнести, то он произнес, но, как и Рейган, не смог решить проблему
роста бюрократии. При Рейгане, кстати, выросли и налоги и государственные
расходы и численность аппарата. А \emph{Зеленский} вообще пытается выстраивать
авторитарный режим, а это означает всегда резкий рост численности бюрократии,
потому что основа авторитаризма, это бюрократия. С бюрократией можно бороться
только одним единственным путем: надо лишать ее контролирующих функций.
Полностью и навсегда!
%%%cit_comment
%%%cit_title
\citTitle{Война Зеленского с бюрократией потерпела крах, не начавшись}, 
Андрей Головачев, strana.ua, 14.06.2021
%%%endcit

%%%cit
%%%cit_head
%%%cit_pic
%%%cit_text
Компашка \emph{Зеленского} оставила все ту же практику Петра разделения и
унижения своих сограждан. Унижения и разделения. Каждый день. В каждом
публичном выступлении. Свои и чужие
%%%cit_comment
%%%cit_title
\citTitle{Бог Данилова - это бог расчеловечивания}, 
Игорь Лесев, strana.ua, 20.06.2021
%%%endcit

%%%cit
%%%cit_head
%%%cit_pic
%%%cit_text
Об этом \emph{Зеленский} заявил в интервью Наталье Мосейчук на телеканале \enquote{1+1}.  \enquote{Что
касается альтернативных планов. План \enquote{Б} - мы в нем, я считаю, находимся. А те,
кто предлагают, например, стену, как план \enquote{Б}, или план \enquote{С} – неважно - стену в
любом формате ... Стена - это полное расторжение отношений в том или ином виде. Я
считаю, что этот план может быть, но решение о запуске этого плана должно
принимать народ Украины}, - сказал Зеленский.  По его словам, \enquote{если не
получатся любые альтернативные договоренности с США, с Россией и параллельно в
Нормандском формате - будет он работать или нет - то за \enquote{стену} будет
голосовать народ Украины}
%%%cit_comment
%%%cit_title
\citTitle{Зеленский допускает референдум о расторжении отношений с Донбассом}, Карина Вольтер, strana.ua, 24.06.2021
%%%endcit

