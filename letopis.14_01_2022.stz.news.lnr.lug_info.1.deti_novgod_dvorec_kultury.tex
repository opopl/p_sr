% vim: keymap=russian-jcukenwin
%%beginhead 
 
%%file 14_01_2022.stz.news.lnr.lug_info.1.deti_novgod_dvorec_kultury
%%parent 14_01_2022
 
%%url https://lug-info.com/news/bolee-tysaci-detej-pobyvali-na-novogodnem-predstavlenii-v-luganskom-dvorce-kul-tury
 
%%author_id news.lnr.lug_info
%%date 
 
%%tags deti,novyj_god,kultura,lugansk,lnr,donbass
%%title Более тысячи детей побывали на новогоднем представлении в Луганском Дворце культуры
 
%%endhead 
\subsection{Более тысячи детей побывали на новогоднем представлении в Луганском Дворце культуры}
\label{sec:14_01_2022.stz.news.lnr.lug_info.1.deti_novgod_dvorec_kultury}

\Purl{https://lug-info.com/news/bolee-tysaci-detej-pobyvali-na-novogodnem-predstavlenii-v-luganskom-dvorce-kul-tury}
\ifcmt
 author_begin
   author_id news.lnr.lug_info
 author_end
\fi

Более тысячи юных жителей столицы Республики побывали на праздничном
представлении \enquote{Новогодняя сказка Шахерезады}, показы которого с 23 декабря
проходили в Луганском Дворце культуры (ДК). Об этом сообщила пресс-служба
Министерства культуры, спорта и молодежи ЛНР со ссылкой на учреждение.

\ii{14_01_2022.stz.news.lnr.lug_info.1.deti_novgod_dvorec_kultury.pic.1}

\enquote{Луганский дворец культуры с 23 декабря по 14 января провел 13 мероприятий,
посвященных новогодним праздникам. С учетом ограничения посадочных мест в
зрительном зале на представлениях побывали более тысячи зрителей}, – говорится
в сообщении.

\ii{14_01_2022.stz.news.lnr.lug_info.1.deti_novgod_dvorec_kultury.pic.2}

В ДК отметили, что новогодние спектакли посмотрели воспитанники Луганского
центра социальной реабилитации детей-инвалидов \enquote{Возрождение}, республиканского
социально-реабилитационного центра для несовершеннолетних, учащиеся 10
столичных школ, а также учебно-воспитательного объединения \enquote{Академия детства}.

\ii{14_01_2022.stz.news.lnr.lug_info.1.deti_novgod_dvorec_kultury.pic.3}

Сказку публике представили ведущие творческие коллективы учреждения культуры:
народный цирк \enquote{Родник}, образцовые танцевальные коллективы \enquote{Фиеста}, \enquote{Катюша},
\enquote{Карамельки}, студия восточного танца \enquote{Нейла}, детская вокальная студия
\enquote{Ассорти} и сотрудники ДК.

\enquote{Зрелищное театрализованное представление украсили профессиональное
художественное оформление сцены, музыкальный материал, видеоряд, спецэффекты и,
конечно, вдохновляющая игра самодеятельных артистов. И артисты, и зрители
получили от каждой встречи огромный заряд позитивных эмоций, подарили друг
другу незабываемые ощущения праздника, сказки и торжества силы искусства}, –
рассказали в Луганском дворце культуры.
