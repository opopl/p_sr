% vim: keymap=russian-jcukenwin
%%beginhead 
 
%%file 06_05_2014.lj.kungurov.1.donbass_chechnja_dva
%%parent 06_05_2014
 
%%url https://kungurov.livejournal.com/85155.html
 
%%author_id lj.kungurov
%%date 06_05_2014
 
%%tags donbass,vojna,ukraina,chechnja
%%title Донбас и Чечня: в чем разница (часть 2)
 
%%endhead 
\subsection{Донбас и Чечня: в чем разница (часть 2)}
\label{sec:06_05_2014.lj.kungurov.1.donbass_chechnja_dva}

\Purl{https://kungurov.livejournal.com/85155.html}

\ifcmt
 author_begin
   author_id lj.kungurov
 author_end
\fi

Начало \href{https://kungurov.livejournal.com/84905.html}{тут}. Итак, в чем же
разница между чеченским и донецким (шире – юго-восточным) сепаратизмом? Разница
принципиальная: на Украине в настоящий момент идет гражданская война, в то
время как обе чеченские войны фактически были войнами между двумя
государствами, пусть даже одно из них было непризнанным. Вторую чеченскую войну
уже можно было назвать гражданской, но она была гражданской для самих чеченцев,
различные кланы которых сводили счеты друг с другом – одни тейпы встали на
сторону бандитов, другие поддержали федералов.

Кратко напомню историю вопроса. В 1990 г. Верховный Совет Чечено-Ингушской АССР
принимает декларацию о независимости, объявляя республику субъектом СССР.  В
июне 1991 г. Общенациональный конгресс чеченского народа (ингушей почему-то
забыли) во главе с генералом Дудаевым объявляет о низложении ВС ЧИАССР и
переходе власти к ОКЧН. Двоевластие продолжается до августа, когда Верховный
совет, поддержавший ГКЧП, разгоняется дудаевцами, поддержавшими Ельцина. ОКЧН
заявил, что власть принадлежит им и объявил о независимости Чечни. Фактически с
сентября 1991 г. Чечня является неподконтрольной Москве территории. Ингушетия,
превратившись в отдельную республику, осталась в составе РСФСР. В октябре
чеченцы избирают своим президентом Дудаева. По всей Чечне идет захват военных
складов и военных частей. Первую партию оружия, кстати, дудуаецы получили,
разгромив здание республиканского КГБ.

9 ноября 1991 г. была предпринята первая попытка подавления Чечни, когда в
Ханкале приземлились военно-транспортные самолеты с советскими (пока еще
советскими) солдатами, которые были блокированы боевиками, подконтрольными
Дудаеву. В результате Москва фактически капитулировала, войска бывшей советской
армии к июню 1992 г. были выведены с территории непризнанной республики. Боевая
техника и оружие по приказу ельцинского министра обороны Грачева (вот, спасибо,
сука!) передаются дудаевцам.

Чеченцы распорядились независимостью весьма своеобразно, создав на территории
республики криминальный анклав. Движение поездов через Чечню прекращено по
одной причине – только за 8 месяцев 1994 года было совершено 120 вооруженных
нападений, разграблено 1156 вагонов и 527 контейнеров. Крупные криминальные
группировки, действующие на территории России, обустраивали в Чечне свои базы.
Процветала работорговля. Единственная территория, на которой после распада СССР
происходил геноцид русского населения – Чечня. В Грозном из 370 тысяч населения
210 тысяч были русские.  Сегодня их порядка 7 тысяч. Комментарии излишни. Между
собой чеченцы тоже почти непрерывно воевали, оппозиция создала свои вооруженные
группировки (вооружала их Россия), которые пытались свергнуть Дудаева, но
безуспешно. Поэтому когда в 1994 г. началась первая чеченская война, то в целом
было понятно кто с кем и ради чего воюет. В российской армии не было уроженцев
Чечни (по крайней мере чеченцев), население республики справедливо
воспринималось, как вражеское, а население видело в русских солдатах не
освободителей, а оккупантов. Так что назвать эту войну братоубийственной никак
не получится.

На Украине расклад совсем иной. Конфликт по сути носит не политический и не
национальный характер, а мировоззренческий. Столкнулись два менталитета –
революционный прозападно-нацистско-украинский и, если можно так сказать,
консервативно-пророссийский. Менталитет Юго-Вотстока ни в коем случае нельзя
называть русским. Он – постсоветский.

В этой связи перед хунтой стоит совершенно неразрешимая проблема: использовать
для подавления сепаратистов государственные структуры – армию, МВД практически
невозможно, потому что люди в форме в половине случаев являются уроженцами тех
самых территорий, ментальность которых не вписывается в новый украинский
порядок. Поэтому армия и милиция не боеспособны. Часть военных добровольно
переходит на сторону противника, часть воюет на отъебись, не имея к тому
никаких моральных стимулов. Собственно, «воюет» армия лишь до тех пор, пока ей
противостоят плохо организованные ополченцы, вооруженные преимущественно
стрелковым оружием. Как только сепаратисты получат более серьезное вооружение,
нежели охотничьи ружья и трофейные РПГ, армия Украины просто прекратит свое
существование даже не под ударами ополченцев, а просто из-за тотального
саботажа. Есть такое понятие, как моральная упругость войск, то есть
способность их выдерживать напряжение боя и нести потери. У укро-вояк эта самая
моральная упругость близка к нулю. Поэтому никаких успехов в проведении
антитеррористической операции на Юго-Востоке Киев не добился, и не сможет
добиться в принципе.

Еще один важный момент. В любую гражданскую войну вовлечено обычно очень
незначительное число граждан. Так в гражданской войне в России на всех сторонах
активно участвовало порядка 1,5\% населения.  Остальные жили по принципу «моя
хата с края» или были лишь сочувствующими, причем сочувствовали обычно тому,
кто побеждает.  Население мятежных украинских областей в подавляющей массе
столь же пассивно. И если бы Киев не шел сознательно на конфронтацию с
населением, то весь вялый протест, как обычно, ушел бы в гудок. Но когда против
населения брошена армия, весь расклад тут же меняется.

В Чечне российская армия и МВД тоже фактически воевали против населения
(правда, назвать его мирным язык не поворачивается), однако это население
изначально было активно-враждебным, и решительное применение насилия градус
этой враждебности понижало. Собственно, русские с чеченцами по-крупному воевали
уже в третий раз, и исторический опыт показывает только одно: лишь насилие
является для горцев убедительным аргументом. Что поделать – менталитет у них
такой.

Но на Украине основная масса населения юго-восточных областей в массе своей
пассивно-лояльна по отношению к хунте.  Поэтому ни Запорожье, ни Одесса, ни
Николаев с Херсоном не делали до последнего времени никаких попыток не то что
выйти из состава страны, но даже заявить какую-то внятную политическую позицию.
Даже Харьков, который имел шанс после бегства Януковича стать центром
политического сопротивления хунте, ведет себя крайне вяло. Озабоченность
населения в большей степени вызывает экономическое положение, то есть падение
уровня жизни.

Однако хунта прикладывает титанические усилия для того, чтобы превратить
население во врага. Скоро у обывателя на уровне рефлекса начнет срабатывать:
если видишь людей в форме с желто-голубым или красно-черным флагом – беги,
иначе они тебя убьют.  Киевский режим сознательно взял курс на развязывание
гражданской войны в интересах своих хозяев, о чем я писал в прошлом посте. Но,
делая 20-миллионное население Юго-Востока  врагом, он делает гражданскую войну
изначально проигранной для себя. Не сомневаюсь, что скоро разразится громкий
скандал, связанный с участием в военных карательных акциях иностранных
наемников (или скажем более деликатно – иностранных военных специалистов). В
моральном смысле это очень сильно подорвет позиции хунты.

Еще раз повторюсь: в Киеве сидят американские марионетки, которые старательно
исполняют желания своих хозяев по устройству на Украине «югославии». В
противном случае применение армии для подавления сепартистов не имеет ни
малейшео смысла. Как вообще армия может воевать с враждебным населением? В
Чечне федералы во время второй кампании в общем и целом придерживались такой
схемы: окружали населенный пункт, оставляли коридор для выхода населения,
отфильтровывая его, разумеется, после чего все оставшиеся внутри кольца люди
считались террористами и подлежали методичному уничтожению. Типичный пример
подобной операции (включая типичные ошибки) дает битва за село Комсомольское,
где были применены даже огнеметные системы «Буратино».  Итог сражения:
уничтожено и взято в плен до 1000 чеченских боевиков включая участников банды
Сейфуллы, пытавшейся прийти на помощь окруженным гелаевцам, потери федеральных
сил – порядка 50 убитых.

Теперь давайте порассуждаем, способны ли украинские силовики провернуть нечто
подобное, не то что в миллионном Донецке, а хотя бы в 100-тысячном Славянске.
Совершенно очевидно, что население из Славянска никуда не уйдет, даже если
укро-вояки потребуют этого. Да и куда оно денется, даже если захочет уйти?
Соответственно, силы, подчиненные хунте вынуждены будут атаковать город с
мирным населением. Как в условиях уличного боя можно отличить мирного жителя от
комбатанта? Никак. Поэтому атакующие будут вести огонь на поражение по всему,
что движется.

Вообще-то такие действия, иначе как геноцид, не квалифицируются, и отношение
будут вызывать соответствующее. Но, допустим, что хунте удастся взять под
контроль город. Смогут ли они при этом разгромить ополченцев? Совершенно точно
– не смогут. Ополченцы просто разойдутся по домам, или уйдут из города, причем
вместе с оружием. Уйдут каратели из Славянска – они снова возьмут его под
контроль. Могут ли каратели зачистить Славянск, изъяв все оружие и арестовав
ополченцев? Нет.  Представьте себе, сколько времени и сил потребуется для
зачистки города, в котором проживает 117 тысяч человек (это порядка 50 тысяч
жилищ, сотни производственных и административных зданий!!!). Да и разве
ополченцы такие идиоты, что будут прятать оружие у себя в квартирах? В частных
домах спрятать оружие не проблема – закопал на огороде или в ближайшем овраге –
попробуй найди.  В Донецкой области нет ни гор, ни крупных лесных массивов,
однако карательную операцию это отнюдь не облегчает, поскольку регион очень
урбанизирован – это фактически одни сплошные каменные джунгли, рай для
партизан.

Ладно, давайте предположим, что каким-то чудом карателям удалось полностью
выдавить сепаратистов из Славянска. Это означает лишь одно – при зачистке
соседнего Краматорска сил они будут иметь меньше (часть надо оставить для
контроля Славянска), а противник будет более многочисленным. С оперативной
точки зрения задача по умиротворению Донбасса военными средствами нерешаема в
принципе, даже если речь идет о локальных очагах сопротивления. А представьте
себе, что в «чечню» превратилось сразу Луганская и Донецкая области с
населением (ставшим реально враждебным хунте) почти в 7 миллионов человек!!!

Следующий фактор, который следует принимать во внимание: Россия могла
действовать в Чечне, имея относительно спокойный тыл. Чеченцы за годы
дудаевской и ахмадовской вольницы изрядно настроили против себя соседние
народы, включая даже родственных ингушей. А картельные силы на Украине не могут
иметь никакого спокойного тыла, поскольку соседние с Донецкой  Харьковская,
Днепропетровская и Запорожская области потенциально очень опасны в смысле
сепаратизма. На этих территориях доминирует все та же постсоветская
ментальность, и принимать необандеровскую национальную идеологию население этих
регионов не спешит, донецкие «пророссийские» хохлы для них ближе и роднее, чем
галицкие «соотечественники». А если галицкие «братья» устроят там что-то вроде
одесского барбекю, то лояльности хунте это никак не добавит.

Немаловажный фактор в любой войне, а в войне гражданской, тем более –
поддержка общественного мнения действий властей. Во время второй чеченской
войны правительство России пользовалось безусловной поддержкой населения. Если
армию поддерживают 145 миллионов, то задавить миллионный мятежный народ вполне
возможно. А на Украине тенденция к тому, что силовые мероприятия против
сепаратистов будет поддерживать меньшее число граждан, нежели сочувствует самим
сепаратистам. Как при таком раскладе можно победить?

Но решающий фактор в украинском кризисе находится за пределами страны. Этот
фактор – Россия. Я не говорю, что РФ должна немедленно начать интервенцию, хотя
даже домохозяйки, разогретые зомбоящиком, открыто требуют от Путина
оккупировать сопредельные территории (текст короткий, но весьма
показательный). Нет, вторжение крайне нежелательно. Но при любом раскладе РФ
может влиять на ход событий, даже не вмешиваясь в них напрямую, дипломатические
и экономические рычаги против слабеющей с каждым днем Украины могут быть более
действенны, нежели танковые клинья. Наконец, что может помешать Путину
поддерживать сепаратистов по сирийскому образцу?

Хунта не в состоянии контролировать дырявую границу с Россией, поэтому в любой
момент там могут оказаться сотни и тысячи «добровольцев» с  подозрительно
хорошей выучкой, вооружением и организационной структурой. Вова Путин будет
лишь разводить руками и говорить, что он не может воспрепятствовать народному
порыву русских людей защитить братьев-славян от фашистских карателей. Дескать,
РФ не закрывает границу с Украиной, российские граждане легально пересекают
границу, а чем они там занимаются, ему неведомо. Если украинские власти не
считают желательным присутствие на своей территории иностранцев, пусть сами и
запирают свою границу на замок. И никто ничего Кремлю по этому поводу
предъявить не сможет.

Так что по факту между Чечней и Донбассом нет практически ничего общего.
