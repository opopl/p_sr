% vim: keymap=russian-jcukenwin
%%beginhead 
 
%%file slova.dnepr
%%parent slova
 
%%url 
 
%%author 
%%author_id 
%%author_url 
 
%%tags 
%%title 
 
%%endhead 
\chapter{Днепр}
\label{sec:slova.dnepr}

Когда тонешь на середине \emph{Днепра} - хватаешься за проплывающие пластиковые
бутылки, Так сейчас барахтается и экономика Украины, Александр Гончаров,
strana.ua, 30.05.2021

Дамба на Северо-Крымском канале, построенная в Херсонской области, начала
разрушаться, из-за этого вода из \emph{Днепра} начала утекать на полуостров. Об
этом идёт речь в сюжете «5 канала», опубликованном на YouTube. Временную
плотину возвели в 2014 году, затем в 2015 году начали сооружать новую дамбу, но
работы так и не завершили. Недавно местные жители обнаружили, что плотина
перестала выдерживать напор воды из \emph{Днепра} и та утекает в Крым.
Конструкцию возвели из мешков с песком и металлоконструкций. Спустя почти семь
лет временная плотина перестала удерживать воду из Днепра на украинской
стороне,
\citTitle{Вода из Днепра начала поступать в Северо-Крымский канал}, , sharij.net, 07.06.2021

Ещё можно вспомнить эпический \enquote{вал Яценюка} из того же материала, что и
упомянутая выше дамба. Конечно эта стена защитила Украину от российской
агрессии, ведь наши солдаты, завидя ее, падали на землю и начинали дико ржать.
Так же на ум приходят украинские \enquote{броневики}, сваренные из мусорных баков с
присобаченной к ней дверным глазком. \enquote{Мрия} без крыльев, \enquote{построенные}
космодромы в Бразилии и Австралии, масштабная добыча нефти и газа у себя в
стране, превращение Украины во вторую Францию. Что с вами не так, потомки
древних Шумер? Раньше без проблем могли море вырыть, а тут дамбу в канале
осилить не смогли...  Кстати, Крым отказывается от \emph{днепровской} воды,
справедливо опасаясь отравления химикатами, а то и радиоактивными веществами.
Прорыв знаменитой дамбы очень символичен, он предвосхищает постепенное
прекращение блокады Крыма и другими странами,
\citTitle{Блокада Крыма дала течь...}, Мак Сим, zen.yandex.ru, 08.06.2021

%%%cit
%%%cit_pic
%%%cit_text
Авторские песни.  Ревёт и стонет \emph{Днепр} широкий Слова Тараса Шевченко
Ревёт и стонет \emph{Днепр} широкий, Сердитый ветер песнь поёт, К земле он
вербы гнёт высокие, Горами волны вдаль несёт.  И бледный месяц на ту пору,
Сквозь тучи редко проглядал, Как будто лодка в синем море, То вынырял, то
потопал.  И третьи петухи не пели, Никто нигде не гомонел, Сычи в лесу
перекликались, Да ясень, ветру в такт скрипел
%%%cit_comment
%%%cit_title
\citTitle{Украинские Песни Русскими Словами}, 
БРАТИНА, zen.yandex.ru, 15.12.2020
%%%endcit


%%%cit
%%%cit_head
%%%cit_pic
%%%cit_text
Але штучні водосховища, які наповнювалися \emph{дніпровської} водичкою, вони будуть
залишатися порожніми. Звичайно ж, ніхто не помре від спраги – це однозначно, –
але води для стабільного водопостачання Криму як і раніше буде недостатньо в
зв'язку з тим, що без \emph{дніпровської} води вирішити водну проблему в Криму
практично неможливо.  Також є проблема з тим, що якість питної води з кожним
роком погіршується, і в цьому році це погіршення також триватиме
%%%cit_comment
%%%cit_title
\citTitle{Чому затопило Ялту і Керч? Чи стане після злив в Криму більше води? Пояснює колишній кримський міністр}, 
Настоящее Время, www.radiosvoboda.org, 22.06.2021
%%%endcit

