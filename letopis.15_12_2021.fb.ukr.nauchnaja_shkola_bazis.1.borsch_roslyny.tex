% vim: keymap=russian-jcukenwin
%%beginhead 
 
%%file 15_12_2021.fb.ukr.nauchnaja_shkola_bazis.1.borsch_roslyny
%%parent 15_12_2021
 
%%url https://www.facebook.com/sciencebasis/posts/1235106423651125
 
%%author_id ukr.nauchnaja_shkola_bazis
%%date 
 
%%tags biologia,borsch,obrazovanie,rastenia,shkola,ukraina
%%title Приготування борщу з вегетативних органів рослин
 
%%endhead 
 
\subsection{Приготування борщу з вегетативних органів рослин}
\label{sec:15_12_2021.fb.ukr.nauchnaja_shkola_bazis.1.borsch_roslyny}
 
\Purl{https://www.facebook.com/sciencebasis/posts/1235106423651125}
\ifcmt
 author_begin
   author_id ukr.nauchnaja_shkola_bazis
 author_end
\fi

Чи можна придумати кращий спосіб вивчення вегетативних органів рослин та їх
видозміни, ніж приготування з них борщу?

\ii{15_12_2021.fb.ukr.nauchnaja_shkola_bazis.1.borsch_roslyny.pic.1}

Для шестикласників «Базису» це уже традиція! Щороку перед новорічними святами
вони на уроці біології разом готують борщ!  

\ii{15_12_2021.fb.ukr.nauchnaja_shkola_bazis.1.borsch_roslyny.pic.2}

Під час ретельної підготовки інгредієнтів, діти знайомляться з видозмінами
органів рослин. Наприклад, усі ми знаємо, що морква та буряк – це коренеплід,
але для багатьох дітей було несподіванкою, що головка капусти – це лише
величезна брунька цієї рослини. 

\ii{15_12_2021.fb.ukr.nauchnaja_shkola_bazis.1.borsch_roslyny.pic.3}

Для більшості дітей це перший досвід приготування борщу. І який же він смачний
виходить, якщо його зварити власноруч! Аромат запашної страви розноситься по
всій школі, і за лічені хвилини біля мультиварки збирається черга охочих
скуштувати цю корисну смакоту.

\#Basis

\begin{itemize} % {
\iusr{Марія Стаднійчук}
Неймовірно! Гадаю, діти надовго запам'ятають такий урок.

\iusr{Виктория Надточенко}

Отличный способ подать тему да ещё и навыки приготовления борща дать!:) Можно
по всем классам выше такой урок?:)


\iusr{Науково-дослідницька школа Базис}
\textbf{Виктория Надточенко} Всі шості класи свого часу проходять через це)
\end{itemize} % }
