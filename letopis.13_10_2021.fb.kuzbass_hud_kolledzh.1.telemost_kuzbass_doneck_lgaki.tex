% vim: keymap=russian-jcukenwin
%%beginhead 
 
%%file 13_10_2021.fb.kuzbass_hud_kolledzh.1.telemost_kuzbass_doneck_lgaki
%%parent 13_10_2021
 
%%url https://www.facebook.com/koxk42/posts/601553227928078
 
%%author_id kuzbass_hud_kolledzh
%%date 
 
%%tags donbass,doneck,lgaki,rossia,telemost
%%title Телемост - Кузбасс - Донецк - ЛГАКИ
 
%%endhead 
 
\subsection{Телемост - Кузбасс - Донецк - ЛГАКИ}
\label{sec:13_10_2021.fb.kuzbass_hud_kolledzh.1.telemost_kuzbass_doneck_lgaki}
 
\Purl{https://www.facebook.com/koxk42/posts/601553227928078}
\ifcmt
 author_begin
   author_id kuzbass_hud_kolledzh
 author_end
\fi

Мы долго к этому шли и вчера, во вторник, наши ожидания стали реальностью.
Состоялся значимый для развития международных связей Телемост между нашим
учебным заведением, Донецким художественным колледжем и колледжем Луганской
государственной академии культуры и искусств имени М. Матусовского.

\ifcmt
  tab_begin cols=4

     pic https://scontent-mxp1-1.xx.fbcdn.net/v/t39.30808-6/245180279_601552981261436_5935261821030449804_n.jpg?_nc_cat=110&ccb=1-5&_nc_sid=730e14&_nc_ohc=Rv-8Y19gnSgAX-RfPIl&_nc_ht=scontent-mxp1-1.xx&oh=e62b4b72f2912787d90c5bb88b032a78&oe=618E24C3

     pic https://scontent-mxp1-1.xx.fbcdn.net/v/t39.30808-6/245171520_601553021261432_4533776364895026133_n.jpg?_nc_cat=109&ccb=1-5&_nc_sid=730e14&_nc_ohc=ZfmZzpSPri0AX9qv47A&_nc_oc=AQlOiGm-qdEa5Q40NRJaqcxqNHjEIDqE6grKZfMVIl3upzKdj8z8Lu1-DerLt62VaVY&_nc_ht=scontent-mxp1-1.xx&oh=870469ed8fe46684dc58a1dd87a90900&oe=618F0956

     pic https://scontent-mxp1-1.xx.fbcdn.net/v/t39.30808-6/245179157_601553007928100_5888368071667802786_n.jpg?_nc_cat=109&ccb=1-5&_nc_sid=730e14&_nc_ohc=LfJwsA9vWcwAX9T7A3m&_nc_ht=scontent-mxp1-1.xx&oh=ac53ec15706f09fee0d49fbc38831d80&oe=618E23FC

     pic https://scontent-mxp1-1.xx.fbcdn.net/v/t39.30808-6/245163627_601553004594767_2316614995367203417_n.jpg?_nc_cat=105&ccb=1-5&_nc_sid=730e14&_nc_ohc=L2eMz6O_wbsAX_s0w90&tn=lCYVFeHcTIAFcAzi&_nc_ht=scontent-mxp1-1.xx&oh=195ee6a78f8c7b3607dfd4d23473b737&oe=618D35E6

  tab_end
\fi

Темой прозвучавших презентаций и докладов стала индустриальная составляющая
творчества местных мастеров — профессионалов изобразительного искусства, а
также студенческих проектов, и продуктов, выполненных по этим проектам.

Этот своеобразный триалог начался со знакомства, с презентационных роликов,
представляющих колледжи. И с самых первых слов протянулись теплые лучики
узнавания, радости и волнения от встречи, ведь индустриальная тематика является
значимой и для Кузбасса, и для Донбасса. Кроме того, в Донецком колледже
училась Инна Акимова, наш замечательный и любимый преподаватель.

\ifcmt
  tab_begin cols=2

     pic https://scontent-mxp1-1.xx.fbcdn.net/v/t39.30808-6/243151019_601553147928086_3619908156110551676_n.jpg?_nc_cat=105&ccb=1-5&_nc_sid=730e14&_nc_ohc=PHue0Y-_NxMAX9hezTv&_nc_ht=scontent-mxp1-1.xx&oh=2ea560c660a3159e4dc914bbd820cff3&oe=618E6AAA

     pic https://scontent-mxp1-1.xx.fbcdn.net/v/t39.30808-6/245167025_601553151261419_8406978232100878974_n.jpg?_nc_cat=103&ccb=1-5&_nc_sid=730e14&_nc_ohc=gUqEuzJRqKYAX9ADxPr&_nc_ht=scontent-mxp1-1.xx&oh=5f3eda6341f548212f61f14adb71cba7&oe=618DEB6E

  tab_end
\fi

Обратили на себя внимание искусствоведческие доклады преподавателя Кузбасского
колледжа Людмилы Оленич и старшего методиста Елены Щербаковой, фундаментальные,
проблемные, тяготеющие к развертыванию в полноценные научно-популярные
ретроспективы.

Мы очень рады знакомству и с надеждой на новые встречи смотрим в будущее!
