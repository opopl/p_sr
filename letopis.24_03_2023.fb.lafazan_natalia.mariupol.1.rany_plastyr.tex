%%beginhead 
 
%%file 24_03_2023.fb.lafazan_natalia.mariupol.1.rany_plastyr
%%parent 24_03_2023
 
%%url https://www.facebook.com/permalink.php?story_fbid=pfbid02etsjfDGwskrmXhZGmjLfxHBW18k79y5cT67LgBsFmDBEmMBzo8wEKQNgjLKi9A4Xl&id=100030592628843
 
%%author_id lafazan_natalia.mariupol
%%date 24_03_2023
 
%%tags mariupol,mariupol.war,dnevnik
%%title Иногда эти душевные раны хочу заклеить пластырем
 
%%endhead 

\subsection{Иногда эти душевные раны хочу заклеить пластырем}
\label{sec:24_03_2023.fb.lafazan_natalia.mariupol.1.rany_plastyr}

\Purl{https://www.facebook.com/permalink.php?story_fbid=pfbid02etsjfDGwskrmXhZGmjLfxHBW18k79y5cT67LgBsFmDBEmMBzo8wEKQNgjLKi9A4Xl&id=100030592628843}
\ifcmt
 author_begin
   author_id lafazan_natalia.mariupol
 author_end
\fi

Иногда эти душевные раны хочу заклеить пластырем. Иногда мечтаю проснуться и
понять - это сон. Так было пару раз. Это страшнее. Это не забывается. Это была
не война. Это был геноцид. Горящие дома, районы, девчонки, которые ищут
прокладки в 21 веке, люди,  умирающие без лекарств, только что родившие мама,
прижимающие своих детей, дежурные у костров в момент становились трупами. В
квартирах горели люди. Они горели не потому, что они не хотели выйти - их двери
блокировало взрывной волной, или на них падали плиты. Сотни людей в разных
подвалах под завалами. Люди без воды и еды. Никто не знал, пройдет ли этот
квест, который навязали рашбратья. Никто не знал, где безопаснее - в подвале
гибли сотнями, в квартирах одиночно. Но сколько было таких квартир? Сотни
тысяч. Это нельзя называть войной. С самолётов рашбратьеа скидывали бомбы, их
танки стреляли хаотично и давили ещё пригодные машины. Людей расстреливали. Мы
для них нацики. Для тех ублюдков, которые пришли нас убивать. Только те, кто
выехал из Мариуполя, со временем смогут рассказать, что тем сволочам нравилось
на́с убивать. Им доставляло это удовольствие. Они громили все. Убивали всех.
Азов и ВСУ долго держали оборону. В нелёгком бою. Наши парни ложили свои жизни
- только ради нас. Но братзкиц народ убивал нещадно. Это война только для наших
военных, которые поддерживали город в всем, до последнего. Для мирных жителей -
убийство и геноцид, которым руководил кхарлик  ах. Для тех типо братьев -
Сафари. Для мера - отель в Запорожье. Для мариупольцев. - страшные события,
смерти, потеря близких, потеря города, и шанс выжить. Шанс выжить братьzя дали
не всем.

%\ii{24_03_2023.fb.lafazan_natalia.mariupol.1.rany_plastyr.cmt}
