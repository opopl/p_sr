% vim: keymap=russian-jcukenwin
%%beginhead 
 
%%file 16_01_2022.fb.fb_group.story_kiev_ua.1.ljubite_li_vy_teatr.cmt
%%parent 16_01_2022.fb.fb_group.story_kiev_ua.1.ljubite_li_vy_teatr
 
%%url 
 
%%author_id 
%%date 
 
%%tags 
%%title 
 
%%endhead 
\zzSecCmt

\begin{itemize} % {
\iusr{Елена Соколова}
Осень люблю театр, это не обьясняется. Просто люблю...

\begin{itemize} % {
\iusr{Ирина Петрова}
\textbf{Елена Соколова} 

Театр можно обсуждать, осуждать, любить, какой-то спектакль может вызвать
чувство катарсиса, какой-то - омерзения. Уж сколько великих говаривало, что
театр - наша жизнь. С ними не поспоришь)

\iusr{Елена Соколова}
\textbf{Ирина Петрова} полностью согласна с вами
\end{itemize} % }

\iusr{Александр Асатуров}

не пора ли нам замахнуться на Вильяма нашего, понимаете, Шекспира? Ведь
насколько Ермолова могла бы играть лучше вечером, если бы днем она, понимаете
ли, работала у шлифовального станка!

\begin{itemize} % {
\iusr{Светлана Манилова}
\textbf{Александр}, \enquote{Берегись автомобиля}. Как хорошо ты вспомнил. @igg{fbicon.smile} 

\iusr{Александр Асатуров}
\textbf{Светлана Манилова} Евстигнеев - бог!

\iusr{Ирина Петрова}
\textbf{Александр Асатуров} и не поспоришь ведь @igg{fbicon.wink} 
\end{itemize} % }

\iusr{Ирина Иванченко}

Спасибо, отличные тетральные истории, приятно было прочесть про дорогих вашему
сердцу людей, про народный театр, про Пистоньку, и фотографии прикольные.

\iusr{Ирина Петрова}
\textbf{Ирина Иванченко} 

спасибо, Ирина! Такие теплые слова очень нужны автору @igg{fbicon.wink}  театральные байки -
необъятный мир, и, даже на уровне небольшого самодеятельного театрика их можно
рассказывать бесконечно.

\iusr{Виктор Задворнов}

Прекрасный рассказ. Думаю, если бы существовал до сих журнал \enquote{Самодеятельный
театр}, Ваши очерки украсили бы не один номер. Огромное человеческое спасибо!
Роли Предаевич я обожал смотреть в Русской драме. Оказывается, она еще и
режиссировала. Представляю, какой насыщенной была Ваша творческая жизнь. Желаю
всех благ!

\begin{itemize} % {
\iusr{Ирина Петрова}
\textbf{Виктор Задворнов} 

прежде всего - огромная благодарность и поклон за добрую память о нашей любимой
Верочке Леонидовне! Это был такой подарок судьбы - знакомство с Верой
Леонидовной, Александром Герасимовичем ( помню, как он приходил на репетиции, с
тростью, всегда с сигаретой, замечания говорил негромко Вере Леонидовне,
актеров всегда хвалил), с Сашей Онуровым мы начинали репетиции Ионеско \enquote{Король
умирает}, помню свою первую выходную реплику \enquote{здесь мусор и пыль...}

Не сложилось. Саши нет уже шесть лет ...

Спасибо за отклик, это греет.

Вера Леонидовна в конце репетиций всегда говорила нам: \enquote{До свидания!
Кланяйтесь непременно маме ( или супругу, или супруге)}. Это навсегда вошло и в
мой обычай, теперь, правда, вызывающий недоумение...

Светлая им память, Актёрам!
\end{itemize} % }

\iusr{Alik Perlov}

С утра прочёл)))) Спасибо тебе за рассказ @igg{fbicon.bouquet}{repeat=3}
@igg{fbicon.hands.applause.yellow}{repeat=10} Браво Бис.  Народная вы Наша
@igg{fbicon.face.blowing.kiss}{repeat=3} 


\iusr{Ирина Петрова}
\textbf{Alik Perlov} 

Алик, поклон с авансцены! Мерси, мон анж! @igg{fbicon.heart.eyes}
@igg{fbicon.face.blowing.kiss} @igg{fbicon.heart.red} ️@igg{fbicon.face.smiling.eyes.smiling}
@igg{fbicon.heart.with.ribbon} 

\iusr{Тамара Градобоева}

\ifcmt
  ig https://scontent-frt3-1.xx.fbcdn.net/v/t39.1997-6/s168x128/16781161_1341101952618574_7704631035023065088_n.png?_nc_cat=1&ccb=1-5&_nc_sid=ac3552&_nc_ohc=hM1uiDFLgRoAX_BmOxP&_nc_ht=scontent-frt3-1.xx&oh=00_AT82pUAwD2i2YMFZih_O8Tj_FhSyFI6vMKWz5V8oPXHLKg&oe=620524AB
  @width 0.1
\fi

\iusr{Елена Сидоренко}

Браво, Ирочка! Огромное спасибо за рассказ. @igg{fbicon.heart.sparkling} 

\ifcmt
  ig https://scontent-frt3-1.xx.fbcdn.net/v/t39.1997-6/s168x128/17639224_1652591041433963_4031073656246370304_n.png?_nc_cat=1&ccb=1-5&_nc_sid=ac3552&_nc_ohc=GKJX96ftDGkAX-8nBAY&_nc_ht=scontent-frt3-1.xx&oh=00_AT_4CzpxvEY92DYVcpshdr3kcH2lvHYRqs59bCrsAbDiQw&oe=62057ADF
  @width 0.1
\fi

\iusr{Ирина Петрова}
\textbf{Елена Сидоренко} спасибо за теплый отклик!

\iusr{Tetiana Petrovska}
Спасибо. Замечательние воспоминания. Люблю театральное закулисье  @igg{fbicon.flame} 

\iusr{Ирина Петрова}
\textbf{Tetiana Petrovska} 

Спасибо за отклик! И Вы совершенно правы - мир, который зритель видит из зала,
и мир, который есть щакулисами - это, как говорят в Одессе - две большие, я бы
сказала, две огромнейшие разницы. Волею судеб, мне сейчас повезло участвовать
в подготовке и проведении театрального фестиваля. Проходит фестиваль в стенах
любимого театра. Запахи, звуки, встречи - это просто невероятные ощущения.

\iusr{Елена Муравьева}

самодеятельность, как мафия, неискоренима. я \enquote{нечаянно} стала режиссером
самодеятельного театра и вот уже восемь лет мы ставим спектакли. возраст
артисток (в труппе только дамы) за шестьдесят

\iusr{Елена Муравьева}

самодеятельность, как мафия, неискоренима. я \enquote{нечаянно} стала режиссером
самодеятельного театра и вот уже восемь лет мы ставим спектакли. возраст
артисток (в труппе только дамы) за шестьдесят

\begin{itemize} % {
\iusr{Ирина Петрова}
\textbf{Елена Муравьева} 

а вот теперь поподробнее!))) Где, когда, почему нет рекламы? Любопытно ж!!! Где
выступаете? Сейчас есть спектакли? Очень хотелось бы посмотреть!

Ой, увидела на Вашей страничке! Почти понятно) а будут ещё спектакли?

\iusr{Елена Муравьева}
\textbf{Ирина Петрова} 

каждые полгода делаем новую постановку. выступаем перед новым годом и в мае.
один-два спектакля и все. к сожалению, удержать репертуар не получается. пока
мои артистки учат новые роли старые забываются. но процесс творчества увлекает
больше выступлений

\iusr{Ирина Петрова}
\textbf{Елена Муравьева} 

все равно, это очень здорово! Вы - крутые молодцы!!! Если что - я в первых
зрителях, ок? @igg{fbicon.wink}  @igg{fbicon.heart.eyes}
@igg{fbicon.hands.applause.yellow} 

\iusr{Елена Муравьева}
\textbf{Ирина Петрова} конечно
\end{itemize} % }

\iusr{Татьяна Петрюк}

Восторг ! Вы не только талантливая актриса, но и блистательная рассказчица !!!
Браво !

\begin{itemize} % {
\iusr{Ирина Петрова}
\textbf{Татьяна Петрюк} 

Танюша, спасибо, дорогая! Насчёт талантливой актрисы - это преувеличение, а
рассказать могу, что да, то да @igg{fbicon.wink}  Спасибо
@igg{fbicon.heart.eyes} @igg{fbicon.heart.red}

\begin{itemize} % {
\iusr{Татьяна Петрюк}
\textbf{Ирина Петрова}, 

чтобы рассудить нас: нужно попасть на спектакль с Вашим участием, и,
конечно, новых увлекательных историй ! Ваши Рассказы реально вносят в нашу
жизнь лучи добра тепла и света !!!

\iusr{Ирина Петрова}
\textbf{Татьяна Петрюк} 

сонечко, наверное, сцена уже только в воспоминаниях. Я ужасно рада, что именно
в группе нашла девочку из нашего театра. Это уже не первый раз, когда группа
дарит мне людей, с которыми жизнь развела. Честно - иногда во сне выхожу из
кулис, и...не помню слова, мало того - не могу понять, что играем! От всхлипа и
ужаса просыпаюсь. Почти 30 лет уже прошло, а есть ещё в уголочке мозга вот это
- выйти из кулис. Спасибо за теплые слова, они поддерживают в наше непростое
время!

\iusr{Татьяна Петрюк}
\textbf{Ирина Петрова} 

не устаю повторять - группа позитива. Уверенна, что после 30-ти лет творческого
актёрского «простоя» Вам удастся подарить зрителям приятные минуты настоящего
творчества. Сравнивать нельзя, но я когда-то занималась фигурным катанием,
через лет эдак почти 50 повезла внучку кататься на роликах ... чтобы не терять
время зря в ожидании - взяла на прокат ролики и поехала  @igg{fbicon.face.tongue} ... я испытала
невероятный душевный подъем и гордость, словами сложно передать ...

Поэтому уверенна - собирайте своих девочек и на репетиции. Зрители уже готовы и
ждут @igg{fbicon.heart.red}.
\end{itemize} % }

\end{itemize} % }

\iusr{Дмитрий Бартюк}

Есть мнение, что народные театры вскоре вытеснят, наконец, театры
профессиональные! И это правильно! Актер, не получающий зарплаты, будет играть
с большим вдохновением. Ведь кроме того, актер должен где-то работать.
Неправильно, если он целый день, понимаете, болтается в театре!

\end{itemize} % }
