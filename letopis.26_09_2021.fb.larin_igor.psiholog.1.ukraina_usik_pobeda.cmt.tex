% vim: keymap=russian-jcukenwin
%%beginhead 
 
%%file 26_09_2021.fb.larin_igor.psiholog.1.ukraina_usik_pobeda.cmt
%%parent 26_09_2021.fb.larin_igor.psiholog.1.ukraina_usik_pobeda
 
%%url 
 
%%author_id 
%%date 
 
%%tags 
%%title 
 
%%endhead 
\subsubsection{Коментарі}

\begin{itemize} % {
\iusr{Lidiya Giedroyc}
\textbf{Игорь Ларин} - всегда читаю Ваши посты, и всегда они находили во мне
отклик, до этого. Я вообще не согласна с формулировками, которые Вы приводите.
Александр, давно не мальчик, ни в каком смысле не мальчик! Не для чего не
мальчик! Не для того чтоб иметь свою позицию в отношении жизни, ни в отношении
семьи и матери, если использовать Ваши образные сравнения. Никуда его не
шатает. Он давно определился. Это особенно красочно видно из его интервью после
боя. Он говорит у него озера, он хочет рыбу ловить и жену чаще любить. О стране
- матери не было ни слова. Не о матери, ни о тех кто на передовой, делает все,
чтоб он мог вот просто этот флаг поднять! А то что он все еще в стране, которая
его воспитала, это, как-раз и говорит о лояльности к нему. О лояльности
любящего родителя к своему недоразумению - ребенку. Я думаю, что те дети
которые могут попасть в плохую компанию, и впасть в пограничные состояния, -
они как раз возможны в тех семьях что потеряли в АТО своих отцов.... Извините,
- задело!

\begin{itemize} % {
\iusr{Игорь Ларин}
\textbf{Lidiya Giedroyc} у нас нет никакого родителя. У нас нет взрослой стороны в этом контексте. Совсем.
Зрелый и устойчивый взрослый всегда находит ясные и прямые способы и ресурсы для обозначения своих границ.
И делает конкретные шаги.
Если не находит, а сам занимает позицию подростка, понимаете да? Значит речь идёт не о зрелой стороне.
Мой пост не про Усика, мой пост про незрелое общество, которое рождает незрелых людей и само же с ними потом борется.
Мой пост про нас.
То, что задело лишнее тому подтверждение

\iusr{Lidiya Giedroyc}

Я думаю зрелость это как раз присутствие позиции, иначе чем делиться с
ребенком? А задело, потому что пример о "незрелости общества" не самый, на мой
взгляд, удачный. От Вас, как профессионала. всегда жду сравнений "в яблочко". В
контексте этой темы "Усык и его пограничные состояния" выглядит не убедительно.
Извините!

\iusr{Володимир Інжир}
\textbf{Игорь Ларин} 

общество никого не рождает! Личность формируется сама. Она решает, что ей нра,
а что нет. Осознанно или нет, не важно. Это ее выборы. И последствия их тоже
ее. Как-то так

\iusr{Игорь Ларин}
\textbf{Володимир Інжир}, 

никогда больше нигде этого не повторяйте, Владимир)

Личность никак не может сформироваться отдельно от среды. Никак.

Среда имеет огромное влияние на формирование личности. Человек может это
регулировать, если может, но не быть частью среды не может по определению.

\iusr{Антон Антонов}
\textbf{Игорь Ларин} 

Это не незрелое общество, а общество с трагической историей. То противостояние,
которое происходит в этом обществе, - это следствие влияния различных империй.
Это РАЗНЫЕ сформировавшиеся мировоззрения. И не на что-то, а именно на
собственную страну. Это результат русификации ( и даже не в плане языка, а в
плане центра своей идентичности) со стороны РИ левобережной Украины и половины
украинцев. Поэтому русифицированные украинцы и нерусифицированные, грубо
говоря, бьют лица друг другу. Это очень серьезная и глубокая, с глубокими
историческими корнями, проблема. Которую никто не хочет не то что решать, а
вообще поднимать, вскрывать и объяснять.

Поэтому Усик – это не незрелый мальчик. Это вполне взрослый человек со своим
сложившимся моровоззрением (которое сложилось под влиянием вышеописанных
факторов). И именно ЕГО Украина, его гопак, его оселедец – это совсем не то,
что значат эти символы для вас или для кого-то другого. Его Украина – это
именно путинская Украина: ОДИННАРОД, россияне братья, злые заокеанские силы
разделили триединый народ, украинцы – те же русские, правильная только УПЦ МП и
т.п.

И конечно, травить его за это смысла нет. Таких половина Украины, просто Усик –
выдающийся спортсмен, поэтому заметно всё, что он говорит. Тут хоть бы для
следующих поколений кто-нибудь придумал какую-нить объединяющую идею.

\iusr{Galina Voropaeva}
\textbf{Антон Антонов} предпоследний абзац- супер!!

\iusr{Володимир Інжир}
\textbf{Игорь Ларин} вы о чем то своём @igg{fbicon.wink} 

\iusr{Джулия Кошкина}
\textbf{Игорь Ларин} ,я вижу вокруг себя абсолютно незрелое общество, которое
регулярно попадает в плохие компании. Вечный подросток с прыщами, которого
хороводят, кому только не лень. Пора рости, начинать думать и анализировать. Но
нет, и так сойдёт. А Усик как раз молодец, имеет право. И пусть ходит в какую хочет
церковь, это его личное дело. И не наше дело совать свой нос в веру человека. Мы
же за свободу и толерантность, правда?

\iusr{Boris Patychenko}
\textbf{Lidiya Giedroyc} , спасибо, лучше и не скажешь

\end{itemize} % }

\iusr{Anna Zhydovlenkova}

Этот пост вроде про спасение человека, страдающего алкогольной зависимостью.
Вместо того, чтобы думать как себя спасти (Украину), нужно видимо искать
причины в себе и спасать другого, в то время как тот второй получает истинное
удовольствие и ему все норм.

Пост 100\% манипулятивный. Усик не подросток)) а латентный сепаратист. И все что
он делает и говорит публично - ОСОЗНАНО. И Украина в этом не виновата, и не
нужно украинцам его спасать и выискивать любовь. Это нормально, когда на
поступки и слова человека есть соответствующая реакция у другого. Так в Украине
одна реакция на Усика, а вот в России другая. Единственное плохо - что он
выступает под украинским флагом. А в Украине есть кого уважать и кем дорожить-
украинскими воинами, которые под украинским флагом защищают страну, рискуя
своей жизнью, а не под украинским флагом зарабатывают деньги.

И ещё, а вы уверены, что именно Украина для него мама?)

\begin{itemize} % {
\iusr{Natali Mickiewicz}
\textbf{Anna Zhydovlenkova} Абсолютно з Вами згідна!
\end{itemize} % }

\iusr{Татьяна Александрук}

До приступов православия головного мозга он был почти национальным героем.
Внимание, уважение, восхищение, любовь - все это у него было. Что еще мы можем
дать?

\begin{itemize} % {
\iusr{Игорь Ларин}

Когда ребенка начинает плющить, взрослому рядом, крайне важно оставаться
устойчивым. Если его тоже начинает шатать-это беда для обоих.

Что мы можем?

Не впадать в такие же крайности. От восхищения до ненависти.

\iusr{Татьяна Александрук}

Согласна. Ненависть крайне редко оправданна. Тем более в этом случае. Но,
боюсь, даже у взрослых может случиться усталость, и , как следствие,
безразличие.

\iusr{Игорь Ларин}

Нет, не должна и не случается. Если речь идет о взрослом и ребенке.
Но если встречаются два ребенка, тогда да. Тогда война.
Второй ребенок в этом случае-это наше незрелое общество.

\iusr{Татьяна Александрук}
\textbf{Игорь Ларин} 

Общество незрелое. И очень опасно, когда незрелая личность, будучи публичным
человеком, в незрелом обществе распространяет сомнительные, скажем мягко, идеи.
Это сильно уменьшает желание любить и восхищаться. К сожалению.

\iusr{Игорь Ларин}
\textbf{Татьяна Александрук}, это, без сомнения, так.
Но самое главное во всем этом направленность фокуса.
Не зрелое общество, должно не личностью, пусть и не зрелой, заниматься. А исключительно собой.
Иначе выхода нет.

\iusr{Татьяна Александрук}
\textbf{Игорь Ларин} Согласна. Но способно ли на это общество, в котором взрослые и адекватные в меньшинстве? И они, судя по последним событиям, явно не вытягивают всех. Что-то оптимизма у меня не много. Но... Делаем, что должно и будь что будет. @igg{fbicon.face.smiling.eyes.smiling} 
\end{itemize} % }

\iusr{Таня Богданова}
У каждого свой путь познания себя и бога!

\iusr{Olesia Vessna}

Коли в тебе пів життя в Криму, а потім окупація, розрив сім‘ї, все розірвано і
болить, болить вже сьомий рік, звісно все чорно-біле, все на краях полюсів,
важко витримувати і бути дорослим. Подекуди впадаєш у стан розщеплення, бо так
легше... навіть коли розумієш це все. Тому хоча б не коментуєш, і не встрягаєш в
срач. Як то важко бути дорослим. Дякую за такий пост.

\begin{itemize} % {
\iusr{Игорь Ларин}
Так, це надзвичайно тяжко. Надзвичайно. Але родину та близьких людей, в скрутні
часи врятує тільки одне-це щірість, любов та прийняття.

\iusr{Olesia Vessna}
\textbf{Игорь Ларин} можливо так і буде...


\iusr{Игорь Ларин}
буде обов’язково

\iusr{Olesia Vessna}
\textbf{Игорь Ларин} !) я згодна

\iusr{Boris Patychenko}
\textbf{Olesia Vessna} простие, у кого болит? У усика?

\iusr{Olesia Vessna}
\textbf{Boris Patychenko} я не вам писала))) автор все зрозумів.
\end{itemize} % }

\iusr{Владимир Карпенко}
Чемпион пограничник….

\begin{itemize} % {
\iusr{Игорь Ларин}
Так в большом спорте это сплошь и рядом

\iusr{Dmitry Zloy}
\textbf{Игорь Ларин} покажите хоть одного россиянина пограничника - выступающего с триколором и ходящего в ПЦУ
\end{itemize} % }

\iusr{Dima Malashko}
Чемпион от Бога. Но несёт парень точно полную чушь… будем ждать взросления

\begin{itemize} % {
\iusr{Игорь Ларин}
Нам не ждать нужно, а учавствовать в этом

\iusr{Dima Malashko}
\textbf{Игорь Ларин} Саша, к сожалению, пока слышит только себя и тех, кто ему в уши вкладывает инфу, которую он потом вещает... это оч хорошо видно из его интервью и сообщений... поэтому всегда есть своё время участвовать или ждать...

\iusr{Яр Галина}
\textbf{Dima Malashko} йому це не загрожує!

\iusr{Игорь Ларин}

Позволю себе предположить, что Саша себя не слышит, в первую очередь.

К сожалению вы правы про инфу. Ограниченный доступ, ограниченные контакты,
ограниченное время это все работает не на формирования личности. Особенно если
у кого то конкретного стоит такая задача-сформировать свой взгляд на себя и на
мир.

Но всё это становится ядерной бомбой и суперпитательной средой для всякого рода
проходимцев, когда все это умножается на отсутствие поддержки близких, когда
самые родные начинают тебя критиковать, обесценивать, оскорблять.


\iusr{Galina Voropaeva}
\textbf{Dima Malashko} И до какого его возраста??? Целовать руки кремлядским фсб-шным попам???
\end{itemize} % }

% -------------------------------------
\ii{fbauth.slipchenko_sergej.kiev.ukraina.kirovograd.nlp.biznes.konsultant}
% -------------------------------------

Слава богу , что он не читает такие посты, где он незрелый подросток/
гражданин, а вокруг него незрелое /зрелое общество, которое по-отечески на него
смотрит, как на сбившегося с пути, истинного. Живет так, как хочет , делает то,
что умеет, любит то, что делает. С верой в Бога и в окружении любимых близких.

Парадокс: счастливого человека оценивает несчастное общество, зато
психологически образованное, знающие слова : принятие, толерантность, границы...
и хайпующее на нем же ) так вот- вчера победил Александр Усик, украинский
боксёр. 

А Украина победит тогда, когда у людей мозги начнут появляться. И пусть каждый
начнёт с себя, Усик с своей жизнью как-нибудь справится

\begin{itemize} % {
\iusr{Олександр Борисов}
\textbf{Sergey Slipchenko}, парадокс в тому, що люди живучи в Україні, хочуть зробити з неї московію. А отакі " підлітки" цьому сприяють.

\iusr{Сергей Слипченко}
\textbf{Олександр Борисов} Усик: а) не подросток, б) не относится к людям, которые хотят сделать из Украины московию в) любит эту страну и прославляет ее на весь мир

\iusr{Олександр Борисов}
\textbf{Sergey Slipchenko}, я вкурсі що він не підліток. Тому й виніс це слово в лапки.
Так, ми не знаємо, що він хоче, чи не хоче. Та своїми не обдуманими, а може й обдуманими й свідомо вибраними, висловами формує певні думки в його шанувальників.

\iusr{Сергей Слипченко}
\textbf{Олександр Борисов} чтобы делать выводы, сначала нужно аргументировать суждения. На каком основании Ваши выводы ?

\iusr{Олександр Борисов}
\textbf{Sergey Slipchenko}, де Ви побачили в моїх словах висновки. Я сказав свою думку. А сформулював я свою думку про нього, на основі його неоднозначної поведінки і неодноразових висловів, про " мижебратя", в різних ефірах, які я гадаю можна й зараз знайти.

\iusr{Сергей Слипченко}
\textbf{Олександр Борисов} найдите . Это будет основанием . Неоднозначная поведение - это не основание, это оценка. Я надеюсь Вы понимаете разницу между этими значениями.

\iusr{Олександр Борисов}
\textbf{Sergey Slipchenko}, я тут не збираюсь Вам, чи будь кому, що небудь доводити. Захочете - знайдете, на просторах інтернету. Не так важко.)
А то що це моя особиста думка, Я ніби написав два рази.
Так оціночна і так, не об'єктивна. Як і будь яка думка не може бути об'єктивною.)
А ще, Ви мабуть тут бачите тільки себе. Бо мене Ви тут не бачите. І знову оціночне судження. )

\iusr{Лена Лагута}
\textbf{Олександр Борисов} , мої певні думки формує завзятий націоналізм. Дуже неприємна річ. Розумію тих, хто міняє свою життя та державу

\iusr{Олександр Борисов}
\textbf{Лена Лагута}, це мабуть добре, що розумієте.)
Та мені особисто, набагато приємніший український націоналізм, аніж російський шовінізм. З яким не одноразово стикався в росії.

\iusr{Светлана Лепехова}
\textbf{Сергей Слипченко} Я бы добавила сюда: " А что сделал ты для своей любимой неньки Украины?" Чтоб предыдущему оратору дать возможность похвастаться своими заслугами, а не обесценивать труд других.

\end{itemize} % }

\iusr{Oksana Kowalski}

Все дело в том, что он уже не ребенок. Совсем не ребенок. Все, что вы говорите,
справедливо, пока речь идёт именно о ребенке - человеке незрелом,
несамостоятельном. Лет после 20-25 это уже взрослый человек, который делает сам
себя, свои слова, свои поступки, и сам за них отвечает, и расхлебывет
последствия. Детство имеет предел, потом сам за все в ответе.

\begin{itemize} % {
\iusr{Галина Яковенко}
\textbf{Oksana Kowalski} не всегда, далеко не всегда возраст в паспорте определяет взрослость и зрелость . Люди и в 60 могут с трудом различать, что с ними происходит , отвечать за свои слова и поступки.


\iusr{Игорь Ларин}
\textbf{Oksana Kowalski}, есть такое понятие в психотерапии-не сформированная идентичность. Например мужская или женская.
О чём это, если коротко?
По разным причинам, сейчас не о них, человек перестаёт расти внутренне, психологически.
Человек,
вне зависимости от его биологического возраста, функционально может быть очень силён, в том числе и финансово, во внешнем мире, а психоэмоционально оставаться в подростковом возрасте. Чаще всего 16-19 лет.
И вот все вызовы внешнего мира, которые идут не в его функциональность, а в его эго, личный внутренний мир он встречает тем, что имеет.
Это, мягко говоря, очень усложняет жизнь такому человеку и зачастую является разрушителем его функциональных достижений.
\end{itemize} % }

\iusr{Жанна Зубко}

Как же вы правы!!! Но как же трудно родителям ушедшего в притон подростка,
баллансировать между любовью, страхом за его жизнь из-за возникающей
наркозависимости и одновременно построением здоровых границ, когда он, "побитый
" последствиями своего выбора, приходит домой, чтобы отоспаться и отогреться...
И затем опять уйти.

И как же трудно этому подростку по возвращению, баллансировать на грани между
любовью к родителям и раздражением, злостью, когда они пытаются только
заикнуться о том, что ему уже нужна специализированная помощь...

\begin{itemize} % {
\iusr{Игорь Ларин}
Главное, вот это чувство дома. Если оно есть ничего не потеряно
\end{itemize} % }

\iusr{Олександр Борисов}
А може цей "підліток" і не хоче дорослішати?! І є варіант, що й зовсім не захоче. Навіть зі всім нашим терпінням і розумінням, що до нього. Що тоді?

\iusr{Timofej Kowtun}
Даже великие психологи, пишут иногда, полную откровенную хуйню.

\begin{itemize} % {
\iusr{Игорь Ларин}
\textbf{Timofej Kowtun}, Ваша оценка говорит, что Вы поняли о чём текст, правильно понимаю?
Можете в двух фразах обозначить?

\iusr{Timofej Kowtun}
\textbf{Игорь Ларин} 

Ваша фантазия, на счет того кем является Усик для Украины, это всего лишь ваша
фантазия. В моей картине мира это не укладывается, т.к. считаю его личностью
полностью сформировавшейся и знающей и самое главное в ясности понимающей, чего
он хочет и как. И только благодаря тому, что он Украинец — он и является
Победителем! Сменив команду — он сменит и энергию. А на этой энергии он уже
максимум будет чемпионом штата. Ваше мнение субъективно. Про религию скажу
коротко, смелости ему не занимать. И учить не стану. А я за него просто
порадовался, без этих тирад, про то как нужно! Моя субьективная агрессия в ваш
адрес.

\end{itemize} % }

\end{itemize} % }

