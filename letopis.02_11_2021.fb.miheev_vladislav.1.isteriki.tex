% vim: keymap=russian-jcukenwin
%%beginhead 
 
%%file 02_11_2021.fb.miheev_vladislav.1.isteriki
%%parent 02_11_2021
 
%%url https://www.facebook.com/vladislav.mikheev.5/posts/4598726533527419
 
%%author_id miheev_vladislav
%%date 
 
%%tags chelovek,isterika,obschestvo,strana,ukraina
%%title Больше истерик, хороших и разных!
 
%%endhead 
 
\subsection{Больше истерик, хороших и разных!}
\label{sec:02_11_2021.fb.miheev_vladislav.1.isteriki}
 
\Purl{https://www.facebook.com/vladislav.mikheev.5/posts/4598726533527419}
\ifcmt
 author_begin
   author_id miheev_vladislav
 author_end
\fi

Больше истерик, хороших и разных!

Ну, так совсем не интересно... 

"За что ж я гиб и мер в семнадцатом году?"

Никому и нигде - ни в метро, ни в ТРЦ, ни на фудкортах - не интересно, есть ли
у меня результат ПЦР или сертификат.

Для страны, давно наплевавшей на Конституцию,  права человека и врачебную
тайну, это крайне не типично.

Или с вирусом у нас решили бороться 

тайно, примерно как с олигархами и Северным потоком? То есть ярко, шумно,
истерично, на грани массового психоза и - без видимых последствий для вируса,
олигархов и Газпрома. И даже для собственных граждан:  меняются лишь внешние
декорации и инфоповоды для нагнетания истерии - в реальной жизни, если не
считать цен на коммуналку, не меняется ничего.

Занятная статистика примерно годичной давности обнаружилась на Интерфакс.
Согласно ей, еще год назад  иммунитетом от COVID-19 обладали в среднем 52\%
украинцев. То есть, ещё год назад, по мнению специалистов, Украина "вплотную
подошла к появлению коллективного иммунитета".

И вдруг раз - осенью 21 года она резко отошла от него на приличное расстояние.

Отошла резко от перемирия на Донбассе.

Отошла от энергобезопасности и запасов угля и газа.

Отошла от преисполненного достоинства и любви к демократии хамского отношения к
последнему диктатору Европы, в руках у которого оказался волшебный
электрорубильник.

 Отошла от ЕС и Германии, четко разделяющей экономический интерес и
 демократические ценности.

... и никуда до сих пор так и не пришла под крики "все  пропало, потому что..."
(нужное подчеркнуть).

Впрочем, ни один индивид в состоянии истерики никуда прийти не может. Он может
только метаться из стороны в сторону, совершая хаотические телодвижения. Это,
собственно и делает сейчас украинское государство во главе с новыми
эффективными управленцами. Напомню, их выбрали, рассчитывая на то, что они
будут разительно отличаться от предшественников. 

И они действительно отличаются. Примерно так пациент психбольницы отличается от
главврача.

Барыга во главе страны это плохо. Но поступки барыги, даже если они
продиктованы личным , а не коллективным интересом, прагматичны и вполне
предсказуемы. Петр Алексеевич  при всем этом обладал ещё и талантом имитировать
неподдельную защиту и заботу о нации.

Но украинцы, за 30 лет так и не научившиеся отличать имитацию от подлинника,
выбрали очередную имитацию  - "несистемных политиков".

Единственное, что сегодня с системным успехом получается у "фабрики грез",
работающей под куполом украинского управленческого цирка - это плодить
когнитивные диссонансы.

Обратите внимание, что коммуникация с обществом была системно провалена всеми
президентами Украины. Не только потому, что позиционирование в украинских
политических реалиях не предполагает позиции. Но и потому, что
коммуницировать-то особо и не с кем - общества, как реального, взрослого,
ценностно ориентированого и самоосознанного субьекта в Украине нет.

Есть отдельные талантливые, умные, реально мыслящие бизнесмены, врачи,
философы... А общества реально мыслящего нет.

Украинцы живут и "мыслят" аффектами, руководствуются и мотивируются ими же.

Осознанно или неосознанно власть, которая плоть от плоти народной, управляет
украинцами по лебоновским  законам толпы - через аффекты и массовый психоз. 

В итоге по каждой проблеме народ впадает в истерику и поляризуется. Бесконечные
помаранчевые и белосиние, ватники и вышиватники, украиномовни и русскоязычные,
ваксеры и антиваксеры, те, кто за Усика и против Усика...

Манипулировать населением таким образом можно. Захватить и какое-то время
удерживать власть тоже можно. Эффективно управлять государством и построить
цивилизованное общество -  нельзя!

Конспирология для Украины - избыточная роскошь. Никто не способен навредить нам
больше, чем мы сами себе. Даже мировое правительство и антихрист. 

В стране объявляют карантин и красные зоны, но ничего из озвученных мер не
работает. Потому что не ради борьбы с вирусом были все эти истерики о
переполненных больницах, о несознательных и диких невакцинированых варварах,
которые вымирают целыми семьями, успев перед смертью заразить миллионы
сограждан!

Эти истерики, также как и любые другие, были ради самой истерики, а не ради
управленческой эффективности и реального результата.

Вышеописанный управленческий стиль - наше все, как Пушкин у россиян. Массовые
истерики получаются у нас лучше всего, как автомобили у японцев или сыр у
голландцев. Если бы в мире был спрос на  коллективную панику - мы бы стали
мировым монополистом!

\ii{02_11_2021.fb.miheev_vladislav.1.isteriki.cmt}
