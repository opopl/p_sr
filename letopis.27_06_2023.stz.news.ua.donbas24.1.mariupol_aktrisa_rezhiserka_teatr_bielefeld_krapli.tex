% vim: keymap=russian-jcukenwin
%%beginhead 
 
%%file 27_06_2023.stz.news.ua.donbas24.1.mariupol_aktrisa_rezhiserka_teatr_bielefeld_krapli
%%parent 27_06_2023
 
%%url https://donbas24.news/news/mariupolska-aktrisa-i-reziserka-ocolila-novii-teatr-u-nimeccini
 
%%author_id demidko_olga.mariupol,news.ua.donbas24
%%date 
 
%%tags 
%%title Маріупольська актриса і режисерка очолила новий театр у Німеччині (ФОТО)
 
%%endhead 
 
\subsection{Маріупольська актриса і режисерка очолила новий театр у Німеччині (ФОТО)}
\label{sec:27_06_2023.stz.news.ua.donbas24.1.mariupol_aktrisa_rezhiserka_teatr_bielefeld_krapli}
 
\Purl{https://donbas24.news/news/mariupolska-aktrisa-i-reziserka-ocolila-novii-teatr-u-nimeccini}
\ifcmt
 author_begin
   author_id demidko_olga.mariupol,news.ua.donbas24
 author_end
\fi

\ifcmt
  ig https://i2.paste.pics/0234f0327185bc2f20dc09a22fee74a3.png
  @wrap center
  @width 0.9
\fi

\begin{center}
  \em\color{blue}\bfseries\Large
  У місті Білефельд запрацював новий український театр \enquote{Краплі} 
\end{center}

Народний театр Маріуполя \enquote{Театроманія} після повномасштабного
вторгнення рф
\href{https://archive.org/details/08_06_2023.demidko_olga.donbas24.mariupol_aktrisa_virshi_nimecchyna_vystupy_olga_samojlova}{продовжує
розвивати та популяризувати театральну культуру Маріуполя}. Зокрема, у
німецькому місті Гановер працює \enquote{Театроманія 2.0}, якою керує режисер
Антон Тельбізов, а в м. Білефельд створено філію \enquote{Театроманії} театр
\enquote{Краплі} під керівництвом актриси і режисерки Ольги Самойлової. Обидва
театри працюють з українською молоддю! Головною метою театрів є поширення
української культури, мови, традицій в країні, яка прихистила їх від війни!

\textbf{Читайте також:} 

\href{https://archive.org/details/26_04_2023.demidko_olga.donbas24.teatr_conception_linii_zhyttja}{%
Театр авторської п'єси Conception представив поетичну імпрезу \enquote{Лінії життя}, %
Ольга Демідко, donbas24.news, 26.04.2023%
}

\subsubsection{Історія створення театру \enquote{Краплі}}

За словами Ольги Самойлової, театр виник випадково. У актриси вже було декілька
проєктів, серед яких один вона реалізовувала з учнями профучилища. За один
тиждень рижисерка поставила з ними виставу. Згодом про це дізналися
представники Українського волонтерського центру, які запросили Ольгу працювати
до себе. Там актриса почала вести театральний гурток. Спочатку учасниками стали
5 дівчаток і 1 хлопчик. Але актриса розмістила оголошення, в якому розповіла
про роботу гуртка. Завдяки цьому на одному із занять Ольги вже було 19 осіб, з
яких троє — хлопці. Назву \enquote{Краплі} театру запропонував чоловік Ольги —
Валерій Мережаний.

\begin{leftbar}
	\begingroup
		\bfseries
\qbem{Ми, як краплі одного чогось великого, ніби цілого океану. І ці краплі у
різному вигляді: у вигляді дощу, чи у вигляді інших водойм знаходяться
в різних куточках землі. Ми як маленькі шматочки одного чогось великого
цілого}, — розповіла Ольга.
	\endgroup
\end{leftbar}

Режисерка своїх учнів називає \enquote{мої крапельки}. Наразі у колективі 15
акторів і актрис, яких об'єднує одне бажання — розповідати про страшну війну,
що перевернула життя кожного з них. У театрі грають хлопці і дівчата з різних
міст України: Одеси, Вінниці, Харкова, Луганська та Маріуполя. 

\textbf{Читайте також:} 

\href{https://donbas24.news/news/novi-projekti-teatromaniyi-20-ta-zittya-kolektivu-za-kordonom}{%
Нові проєкти \enquote{Театроманії 2:0} та життя колективу за кордоном, %
Ольга Демідко, donbas24.news, 04.05.2023}

\ii{27_06_2023.stz.news.ua.donbas24.1.mariupol_aktrisa_rezhiserka_teatr_bielefeld_krapli.pic.1}

\subsubsection{Перша вистава театру \enquote{Краплі}}

24 червня театр \enquote{Краплі} було запрошено на місцевий телевізійний канал \enquote{21}
разом з українськими музикантами і співаками — родиною Катерини та Макса
Тітових, де в програмі були виконані українські пісні та номер, присвячений
подіям в Україні \enquote{Незламна}. Ольга Самойлова створила музичний номер, в якому
центральне місце посів образ України. Його втілила дівчинка, яка пригощала то
яблуками, то зерном своїх сусідів. Потім сусіди вирішили, що вони можуть і самі
взяти все, що забажають. І тоді вони порушили кордони та зайшли на її
територію. Дівчинка намагалася захистити свою територію, але бій був нерівний,
тому що тих, хто напав, було дуже багато. Фінал номера вийшов досить
символічний: Україна перебуває в огні, а дівчина, що співає, підходить до неї і
вкладає в її руку квітку, які є прообразом життя, після чого дівчинка встає і
протягує руку до сонечка. Ця квітка зцілює та рятує дівчину-Україну.

У Білефельді дуже підтримують театр. Глядачі, які вперше побачили \enquote{Незламну},
наголосили, що це театр справжніх професіоналів, роботи яких вони будуть з
нетерпінням чекати.

\begin{leftbar}
	\begingroup
		\bfseries
\qbem{Вони так грають, тому що це болить. А коли болить, людина, яка навіть є
непрофесійним актором, буде віддаватися на всі 100\%. І я хочу на цій
хвилі зробити ще одну виставу}, — додала Ольга. 
	\endgroup
\end{leftbar}

\textbf{Читайте також:} 

%!!!
\href{https://donbas24.news/news/zittya-pereselenske-mariupolskii-teatr-pidgotuvav-novu-premjeru}{%
\enquote{Життя переселенське} — маріупольський театр підготував нову прем'єру, Ольга Демідко, %
donbas24.news, 28.02.2023}

