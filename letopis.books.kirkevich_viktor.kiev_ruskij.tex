% vim: keymap=russian-jcukenwin
%%beginhead 
 
%%file books.kirkevich_viktor.kiev_ruskij
%%parent books
 
%%url 
 
%%author 
%%author_id 
%%author_url 
 
%%tags 
%%title 
 
%%endhead 
\chapter{Виктор Киркевич - История Киева. Киев руський}

\section{История Киева. От мифов к реальности}

Давным-давно я услышал, что история повторяется, потому что не хватает
историков с фантазией. Думаю, это соответствует истине. Каждый из крупных
ученых, взявшихся за написание глобальной истории своей страны или хотя бы
края, использует труды предыдущих. А иначе нельзя: добавляет свое важное,
бывает, сокращает что-то малоинтересное, маловразумительное. Этим Соловьев,
Ключевский или Грушевский мало отличаются от первых летописцев, переписывавших
труды предшественников, добавлявших при этом частицу своей души, интеллекта.
Поэтому позволю себе смелость писать историю Киева по моему разумению. (Какое
хорошее старое, позабытое слово – разумение.)

Я философ, а это значит, что у меня есть вопросы на любой ответ. Мне так
хочется открыть внимательному читателю сокровенную тайну города, которую я
попытался прояснить во время написания книги, да что там – всю сознательную
жизнь. Это отчаянная попытка. Набрался смелости озаглавить книгу так: История
Киева. По-моему. Пусть кого-то возмутит моя излишняя самоуверенность. Тут,
скорее, самоутверждение и немалый опыт в написании исторических произведений.
Почему имеет право существовать история по Карамзину, а по Киркевичу
исключается? Из-за того, что у меня отсутствует венценосный покровитель? А у
великого историка он был – Николай I. Автора абсолютно не прельщает, чтобы
поколения библиофилов смотрели на нетронутое и нечитанное «девственное»
творение, стоящее в шкафу, в полукоже и с золотым обрезом. Тут – другое! Книги
я пишу и издаю для своего читателя, ему и адресую эти произведения. И ему,
читателю, судить: получилось ли рассказать о бесконечно дорогом мне городе…

Абсолютно уверен, что моя книга не будет одобрена Министерством образования и
не будет рекомендована школьникам для внеклассного чтения. Хотя в прошлом два
моих произведения прошли аттестацию: в хрестоматию по украинской литературе для
5—9-х классов вошел очерк «З iсторiї “Велесової книги”», а в пособие для
внеклассного чтения – «Схована культура, або Нiмецький вплив на Київ».

Можно историю писать в манере «от Ханенко до Хаменко», то есть от хана до хама…
Сразу скажу, что имею много знакомых со схожей фамилией, и ответственно
заявляю, что все они милые и интеллигентные люди. Хотя последние годы Украина
терпит потери и страдания от золотоордынского хана Хама… Давно считаю: что сам
вижу, слышу, нужно описывать в книгах… Записывать, что ты разглядел и чему
свидетель, и что будет после сего…

Уже не раз говорилось, что святая обязанность государства – образование народа.
Если у нации отсутствует своя культура, то она не станет полнокровной. И будет
колонизирована более сильным народом, живущим своей историей, культурой.
Чувство гордости за свой город, край, страну необходимо обязательно прививать в
начальной школе. Задача воспитания патриотизма – научить людей, и не только
молодежь, полюбить и суметь познать свой край, стать на защиту своей Родины.
Причем уверен, что окурок, брошенный в урну, важнее вывешивания флажка.
Полезнее для Родины созидать, чем разрушать! Как необходим нам современник,
любящий Украину и при этом надежно защищенный историческими, культурными,
этнографическими знаниями о ней. Имея право изучать свою страну, имеем ли эту
возможность? При массовом выпуске книг об истории нашего Отечества очень мало
авторов, которым можно доверять, и еще меньше тех, которые, избегая ложной
сенсационности, дают подробный живой рассказ о сложных путях формирования
страны. Поэтому в данной книге больше разбираются ключевые моменты нашего
прошлого. Мы рассмотрим происхождение Руси, основание Киева, первых его
правителей, появление письменности, упадок Киевской Руси, развитие и разделение
трех народов, значение… А также еще много вопросов, важных и занимательных.

За долгие годы чтения курсов истории Российской империи, СССР, а потом и
Украины я заметил, что в них меняется не только содержание, но и акценты,
идеологические подходы. В современной стране, стремящейся в Европу, это
недопустимо. И дело не в идеологических привязанностях того или иного министра
образования или даже президента, а в конъюнктурном подходе к освещению
исторического процесса. Читатель, часто с несформированными представлениями и
жизненной позицией, теряется в этих трактовках, как корабль, которому
необходимо проложить курс в безбрежном море… А без карты и компаса это весьма
затруднительно. Так вот, учебник истории должен быть правдивым, без излишних
идеологизаций. Тогда маршрут правильно проложат и корабль доплывет до цели. Это
в монографиях можно ломать копья. Я не согласен с профессором Н. Яковенко, что
ученику нужно дать разные трактовки, чтобы он выбрал правильную. Это под силу
лишь уникуму. Что говорить о школьниках, когда столпы исторической науки не
могут высказать свои обоснованные суждения по ряду важных ключевых моментов
человеческого бытия.
	
Вспомним «основной вопрос философии»: что первично – сознание или материя… Тут
важно решить, что раньше возникло – Киевская Русь (по моему мнению, предтеча
Украины) или Киев как ее столица (стольный град). В этой книге впервые
приводятся мнения не только ученых, но и любителей истории, которые часто
высказывают довольно трезвые мысли, а также важные и занимательные гипотезы,
предложенные забытыми историками, часто отражающие их взгляд на прошлое. Как ни
вспомнить М. Грушевского, который, досконально зная прошлое, не смог
разобраться в окружающих его соратниках Центральной Рады и событиях,
происходивших в стране, где он стал руководителем. В тенденциозно изложенной
истории из-за ее политизации часто отсутствует непредвзятый, трезвый взгляд на
те или иные события, деяния великих людей, мало учитывается роль негодующей
толпы, которую вульгарные марксисты называли народом. Народ-то всегда
безмолвствует, а бунт, являющийся характерной особенностью отечественного
исторического процесса, подымают деклассированные элементы или купленные
какой-то политической силой боевики. Так примитивно считали и продолжают
считать советские историки, стараясь не вспоминать, что «революционный путь»
Сталина начался с ограбления банков. СССР давно нет, но его идеологи и сейчас
занимают ключевые места в науке и политике, часто, на нашу беду, совмещая две
эти сферы, являясь плохими политиками и бездарными учеными.

Во всей путанице противоречивых «представлений», «мнений», «доктрин», реальных
и фальсифицированных находок и открытий рассмотрим один неоспоримый факт:
славянский алфавит лег в основу украинской, русской и белорусской культуры, и
это единственное, что объединило их, оправдало надуманное, предвзятое название
– «Киевская Русь – колыбель братских народов». Зададимся вопросом: насколько
важной была грамотность в те далекие времена? Скорее всего, в Киевской Руси,
как и в ряде западных стран, носителем «письменности» являлось духовенство,
развивавшее преимущественно традиции византийской церковной литературы.
Представители духовенства вели и летописание. Известны имена Нестора, игумена
Сильвестра, одного из создателей замечательного Галицко-Волынского летописного
свода епископа Иоанна и других. Основным источником так называемой Начальной
летописи явились хроника Георгия Амартола, «Летописец» патриарха Никифора,
житие Василия Нового и т. д.

Зачем писались летописи? Это волновало многих, в частности Пушкина, создавшего
образ Пимена-летописца в «Борисе Годунове», пишущего «в стол» и надеявшегося,
что его произведения когда-нибудь прочтут и узнают правду о текущих событиях…

Из летописей мы узнаем о значительной роли духовенства. Церковь активно
вмешивалась в семейное и имущественное право, имея и сугубо практический
интерес: венчание, крещение, пострижение в монашество. Иногда приходилось
изменять собственные постановления – так отменили из-за выкидышей «земные
поклоны» для женщин на сносях. Но, по большому счету, влияние церкви на
семейное право не имело особого успеха; ей волей-неволей пришлось ограничивать
свои требования и запросы.
	
Думный дьяк Григорий Котошихин (XVII в.) писал о полигамных семьях как о норме
еще в XVII в., что происходило с молчаливого согласия духовника. Византийские
церковники планировали править Рюриковичами или вместе с ними. А зась! Князья
понимали значение института церкви, неуклонно укрепляющей их абсолютную власть.
Поэтому происходила русификация службы, появились местные святые. Смело
утверждаю, что для первых Рюриковичей-христиан летописание и вообще книжность,
письменность были больше обрядовыми, культовыми принадлежностями их «новой
веры». Не забудем и то, что книга на то время была рукописной, писалась долго и
стоила целое состояние. Классическим примером служит Остромирово Евангелие,
которое стало вместе с иконами престижным родовым достоянием, его получали в
приданое или по завещанию. И это творение переписчиков, художников-оформителей,
переплетчиков, ювелиров не было книгой для чтения, а использовалось
исключительно в торжественных случаях, для важных обрядов. Ее берегли как
зеницу ока! Недаром одна из таких книг попала во Францию в качестве приданого
Анны – дочери князя Ярослава, и стала национальной святыней.

Для прочтения столь дорогих и больших по размеру книг был специальный книжный
аналой, за которым стояли чаще духовные лица. Считалось, что чтение,
переписывание, рисование заглавных букв, рисунков – это серьезное духовное
дело, к нему допускали не каждого. Это было достойным послушанием,
подвижничеством. Правители могли указать, чтобы запись в летописи отразила то
или иное важное для них событие. Чтение для развлечения не практиковалось, для
этого были сказители – «славные певцы», «бахари» – рассказчики занимательных
историй. Были и среди дружинников музыканты и певцы, которые в сопровождении
простых мелодий повествовали о подвигах своих предводителей и их
друзей-соратников. Ошибочно мнение, что приход монголо-татар замедлил развитие
культуры Киевской Руси. Факты говорят о том, что в регионах и обществах, где
идеологически доминируют греко-восточная ортодоксия, иудаизм, ислам, всегда
блокировалось развитие «мирской культуры», в частности литературы… Однако и без
надзора духовенства могли создаваться шедевры, например, «Слово о полку
Игореве», былины и сказания.

Давно пора отвергнуть интеллигентско-буржуазное представление, что «быть
образованным – хорошо!» Первые Рюриковичи предпочитали умение вести бой на
мечах умению складывать слова. Нужно было иметь какой-нибудь физический
недостаток, чтобы прослыть Мудрым или Осмомыслом. Лишь после принятия пострига
или ухода в монастырь для князя становилось возможным, а иногда и просто
необходимым, читать и писать. Привычным стало представление о грамотности
населения в прошлом. Позволю себе опровергнуть это. При Владимире появились
первые школы, в них училось, возможно, десятки юношей. Таких школ в лучшие
времена было несколько, и то при монастырях. Ряд князей – Ярослав Мудрый и
Ярослав Осмомысл, Всеволод, Мстислав, Мономах и другие – старались научить
своих детей «уму-разуму», но подобное происходило далеко не во всех семьях.
Мальчиков обучали владеть оружием, а девочек… думаю, вышиванию. Легенды об
образованности киевских княжон вполне обоснованы, но вспоминаются из них лишь
единицы.

Круг чтения знати формировался скорее как душеполезный: «Великий Покаянный
Канон» Андрея Критского или «Луг Духовный» Иоанна Мосха. Было кое-что из
античности: Аристотель, Платон, Гиппократ. Отечественная литература уже к ХIII
в. дает замечательные образцы иного, кардинального подхода к «красному слову»
светского содержания. Описание «образованности» в летописных вставках,
выдержанных в житийном стиле, является данью живописному «канону». Подобная
характеристика могла в житиях приписываться правителю, иногда и вовсе не
умевшему читать; «образованность» – часть «иноческой ипостаси», обязательная
черта «святого правителя», принявшего монашество… Это членение образа на
несколько «ипостасей», каждая из которых связана с определенным именным
прозванием, четко прослеживается в летописной традиции в отношении Рюриковичей.
Причем, как правило, акцентируется какое-либо имя, и не всегда возможно понять,
почему задержалось имя данного князя, а не другого… Замечены три «именные
ипостаси»: «княжая», «титульная», выраженная старым, языческим прозванием,
христианское мирское имя и, наконец, иноческое имя, правда, не у всех. В
рассматриваемое время развитие литературы мирской, светской для праздного
чтения – не происходит. Они появились позднее в Европе. Тем не менее письменная
традиция уже в XIII в. дает замечательный образец кардинального развития
литературы – это Галицко-Волынский летописный свод, где присутствует «институт
авторства».
	
	
* * *

Начальная история города – это вереница князей из династии Рюриковичей. О самых
первых Киевичах сведения отсутствуют или они гипотетичны, весьма смутны и
отрывочны. В описании раннего этапа формирования Киева отмечаем интересную
закономерность: каждая социальная формация давала свою, обоснованную ими самими
версию времени основания города. Но почему-то отсутствует политическая сила,
которая утвердила бы своей Кирилловскую стоянку. Яркий пример абсолютного
незнания – прошлое моего города да и происхождение большинства населенных
пунктов «нашей необъятной Родины». Это выражение, популярное в советское время,
отражает захватническую политику СССР, который планировал охватить весь земной
шар, чтобы во льдах Антарктиды сеять «доброе, разумное, вечное». Была тяга к
Северному и Южному полюсам, где еще не знали общественных формаций. Это четкое
империалистическое мышление не совпадает с менталитетом украинца, у которого
«хата с краю» и он «ничего не знает» о нуждах эфиопского пастуха, афганского
торговца или кубинского плантатора…

Я не могу согласиться с тем, что историю нельзя пересматривать. История – это
живой процесс. В ней нет истин и догм: историкам приходится иметь дело с
человеческим материалом, с людьми, которые когда-то жили и творили. Человек не
может быть по своей сути положительным или отрицательным. Поэтому единого
взгляда на ту или иную личность быть не может. Каждый важный исторический
персонаж постоянно под пристальным вниманием исследователей. «Нет пророка в
своем отечестве» – звучит лучше?! Александр Невский признан святым, но это не
оправдывает его за коллаборационизм с татарами Золотой Орды и за войну с родным
братом. Исторические «подвиги» многих личностей очень далеки от церковных
канонов настолько же, насколько иконописный лик – от реального облика.
Украинская история, как, впрочем, и любого государства – очень сложная
субстанция. К ней нельзя подходить с едиными мерками. Она должна
переписываться. В истории, если она объективна, нельзя умолчать о Симоне
Петлюре, Степане Бандере, Романе Шухевиче – точно так же, как не вспоминать
Артема (Сергеева), Лазаря Кагановича, Петра Шелеста, Владимира Шербицкого,
Сидора Ковпака… Каждый в отдельности может разъединить Украину. Но только
собранные воедино эти личности могут дать ответ на вопросы: «Что же такое
история Украины? И что такое Украина вообще?» Все указанные персонажи были
настоящими сынами своей страны, уверенные, что правда на их стороне. И без них
наше недалекое прошлое превращается в историю Малороссии. Или историю
Недорослии.

90 \% населения Земли уверены, что Америку открыл Колумб, а одна десятая – что
ее открыли викинги, и никому в голову не придет мысль, что ее открыли индейцы.

Если вам кажется, что мои шутки и высказывания вы уже слышали, то не судите
слишком строго. Они стали классикой еще до того, как я их придумал. У меня
иногда возникает желание пойти на изящную научную провокацию, которая
встревожила бы читателя, заставила бы его задуматься вместе с автором, самому
продолжить его мысль, а иногда прийти к своему, вполне возможно, иному мнению.
Ведь история, в отличие от точной науки, сугубо индивидуальна, она
воспринимается каждым по-своему. Конечно, в недалеком прошлом исторические
процессы утверждались, а не обсуждались, то есть та или иная личность получала
не исторически обоснованную объективную оценку, согласно мнению
ученого-специалиста, а субъективную трактовку какого-то партийного бонзы. Так,
например, в конце 1930-х годов начали шельмовать Богдана Хмельницкого, даже
ставили вопрос о замене киевского памятника ему на «героев Перекопа». В начале
же 1943 г., когда на заседании Ставки ставился вопрос о ближайшем освобождении
Украины от гитлеровских захватчиков, И. Сталин сказал: «Нужно ввести орден в
честь какого-нибудь украинского полководца…» Пауза. Вождь задумался, пытаясь
вспомнить хоть одного, и наконец медленно произнес: «…Богдана Хмельницкого»,
которого вспомнил, безусловно, только из-за памятника. И после этого появился
орден, гетмана стали изображать на плакатах, открытках… И кто бы осмелился
иначе посмотреть на этого полководца, если на него мудро указал сам вождь?!

Перелопатив горы научных трудов и исследований, я обнаружил, что не существует
ни одной яркой исторической личности, которой все, без исключения, авторы
давали положительную или отрицательную оценку, имели единственное четко
определенное мнение, о которой говорилось бы в одной плоскости. Нет, каждый
исторический персонаж – политик, военачальник, государственник и т. п. –
характеризуется по-разному, и часто мнения о них полностью полярные.

Опираясь на труды известных специалистов, я постараюсь изложить свое видение
исторических проблем. Мне импонируют труды М. Брайчевского, который четко
замечал существенные недостатки научных схем, построенных тем или иным
исследователем, но никогда не прибегал к тривиальной критике. Его методом было
создание альтернативной концепции, которая отражала бы оригинальное авторское
видение исторического процесса, вела бы от археологических деталей к широким
историко-философским обобщениям.

\section{Происхождение славян}

Важнейший вопрос, относящийся к истории Киева косвенно, – это происхождение
русских и украинцев. Он будоражит всех уже не одно столетие. Еще в ХIX в. Н.
Погодин утверждал, что древнерусское государство было российским, в ХII в.
«русские» переселились в московские земли, а «южнорусские» земли были заселены
украинцами. Эта концепция, не имеющая никакого научного обоснования, тем не
менее муссируется российским телевидением, где горе-журналисты добавляют свои,
высосанные из пальца, подробности. Эти «истории» любят все, кто, читая книгу
обязательной внешкольной программы, доходил только до 3—4-й страницы, с
описанием стоянки неандертальца. Для малограмотных, видевших книгу последний
раз в школе или в институте, эта стоянка – единственное, что осталось в памяти,
и неуч пытается найти сходство ее с площадкой для автомобиля. Для таких и
пишутся «правдивые и новые истории» всяких «фоменко». Я, как автор книги, имею
право на свою концепцию происхождения украинского и русского народов еще и
потому, что являюсь профессионалом с историческим университетским образованием,
но излагать ее не буду, так как не академик. А вот поддержать ту гипотезу,
какая мне ближе, вполне правомочен или могу покритиковать неприемлемую с моей
точки зрения.

В настоящее время разработки и открытия ученых не столь интересны для
всемирного «общества потребления», если они напрямую не связаны с данным
«обществом». Чаще всего науку считают успешной лишь в том случае, когда она
приводит к созданию новых средств массового уничтожения людей. Мне бы хотелось,
чтобы Адам и Ева были украинцами, но… А вот начало формирования украинского
этноса в глубокой древности – во времена трипольской культуры, древних ариев –
мне ближе. Но трипольцев еще неправильно называть не то что протоукраинцами, но
и протославянами. Хотя имеются все основания утверждать, что зачатки, некоторые
немногочисленные тенденции украинской этнической культуры все-таки
проскальзывают. Разберемся вначале, что такое этнос. Это историческая общность,
которая характеризуется собственным самосознанием, языком, этнической
территорией, материальной и духовной культурой, особенностями характера,
определенными экономическими связями. Исследования этногенеза украинцев, а
также белорусов и русских было и есть сложной научной проблемой, которая всегда
вызывала и будет вызывать острые дискуссии. Суровая правда жизни: сын русского
и узбечки – метис, сын русского и негритянки – мулат, сын русского и еврейки –
еврей. Тогда кто по национальности сын украинки и русского?!

Особую роль играют этнические и интеграционные процессы в давние периоды нашей
истории.

В этом вопросе очень сильно замешана политика. В 1948 г. В. Мавродин
сформировал концепцию древнерусской народности. По его мнению, в древнерусской
державе сформировалась этническая общность, имевшая единый язык, территорию,
материальную и духовную культуру, экономические связи, осознающая свое
единство. Очевидные недостатки концепции – игнорирование локальных отличий и
наличие многочисленных неславянских компонентов, малая часть исследованных
процессов V—IX вв. и так далее. Однако субъективное мнение ученого вполне
совпадало с партийной идеологией и представлением товарища Сталина по
национальному вопросу. Мавродина поддержало партийно-идеологическое руководство
СССР. Недалеко от утвержденной Сталиным партийной концепции отошел П. Толочко в
1991 г. в статье «Чи iснувала давньоруська народнiсть», где сформулировал тезис
о том, что древнерусская народность формировалась, однако этот процесс не
достиг логического завершения и украинская, русская, белорусская народности
начали образовываться в недрах древнерусской народности задолго до распада
Киевской Руси. Таким образом, имелось наличие локальных изменений и
динамичность этнических процессов в описываемом обществе. А вот, например, П.
Моця, в 1995 г. отстаивая концепцию древнерусской народности, указывает на
единство языка (при сохранении отдельных диалектов), общность территории,
религии, материальной и духовной культуры, определенную экономическую
целостность, одинаковые традиции, суд, военное устройство, совместную борьбу
против внешних врагов, полные элементы самосознания, чувство патриотизма (в
первую очередь среди князей, государственной и интеллектуальной элиты). Однако
он считает, что формирование украинского и русского народов началось еще во
времена Киевской Руси и было ускорено в результате монгольского нашествия.

Еще в ХIX в. существовало мнение, что каждый славянский народ имел свой
этнический субстрат в границах древнерусского государства, на базе которого
сформировались украинцы, белорусы, русские. Этого мнения придерживался М.
Грушевский, который подчеркивал, что к формированию украинского общества
привело сближение южнорусских княжеств ХII—ХIII вв., а предками украинцев
считал антов.

Восточные славяне были наибольшим и наимощнейшим этнично-социальным массивом в
Восточной Европе, и в I тыс. н. э. именно они определили основное направление
исторического развития на протяжении наполненного драматизмом периода Великого
переселения народов. В начале новой эры славяне занимали громадные просторы
лесостепи и значительной части Полесья. По уровню развития они разделялись на
два основных массива, которые оформились в течение II ст. Лесостепная полоса,
способствующая хозяйственной деятельности, была занята племенами, которые
сохранили памятки черняховской культуры и полнее отражали прогрессивные
тенденции эпохи. Полесье характеризовалось более архаичными формами социальной
организации, отраженными памятниками киевской культуры. Для нас особенно важной
является первая группа племен, поскольку именно в ее недрах зародился
общественный организм, который с VII в. получил название Русь. Определение
демографического потенциала черняховских племен, названных в византийских
источниках антами, связано с большими трудностями из-за отсутствия
статистических данных. Но считается, что густота населения представителей этой
культуры составляла около 10 чел. на 1 км2. Среди рядовых семейств выделяется
определенная группа, которая постепенно и неуклонно захватывает общественные
богатства. Общество становится социально неравным. Зарождающимся феодальным
отношениям принадлежало будущее, поэтому появилась новая система – государство,
приходящее на смену окончательно отжившей родоплеменной организации.

Тогда, в VII в., источники фиксируют два больших объединения древних славян –
антское и склавинское. Это были политические объединения полуварварского типа,
которые демонстрировали, пусть и в зачаточном состоянии, все функции
государственной структуры. Антский союз состоял из шести больших «племенных»
групп (уличей, тиверцев, волынян, бужан, дулебов, белых хорватов) и занимал всю
основную часть восточноевропейской лесостепи на запад от Днепра. Во главе этого
союза стояли цари (рексы), окруженные свитой влиятельных людей. Это
объединение, пережившее наивысший подъем в конце V–VII вв., перестало
существовать в результате аварского нашествия. Но причины, вызвавшие Антский
союз к жизни, существовали и в новых деформированных условиях.
	
