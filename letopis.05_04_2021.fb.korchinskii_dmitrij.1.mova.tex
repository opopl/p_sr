% vim: keymap=russian-jcukenwin
%%beginhead 
 
%%file 05_04_2021.fb.korchinskii_dmitrij.1.mova
%%parent 05_04_2021
 
%%url https://www.facebook.com/korchynskyi/posts/3864475956951241
 
%%author 
%%author_id 
%%author_url 
 
%%tags 
%%title 
 
%%endhead 

\subsection{Кожний захисник моковитської - є ворожим танкістом}
\Purl{https://www.facebook.com/korchynskyi/posts/3864475956951241}

Прес-секретарку президента мендель дратує українська мова. Це є зрозуміло, природньо і очікувано, адже вона - не українка. 

Проте, чому вона обстоює московитську говірку, адже мендель не московитка?

Заради своїх предків, вона мала б непокоїтися за ідиш - мову, більшість носіїв якої колись мешкали в Україні. 

Нині ця милозвучна мова майже зникла під натиском саме московитської та ще й івриту. 

Втім, мендель тут не є самотньою - інші українські політики та урядовці, попри
свої рідні ідиш, вірменську, татарську, чухонську, захищають московитську проти
української. 

Ми маємо пам‘ятати, що нема різниці між підтримкою московитської мови і
підтримкою московитських танків. Бо для московитів їхня мова є таким самим
засобом агресії, як танк. 

Кожний захисник моковитської - є ворожим танкістом.
