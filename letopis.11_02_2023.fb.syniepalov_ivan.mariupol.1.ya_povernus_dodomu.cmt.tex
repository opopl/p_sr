% vim: keymap=russian-jcukenwin
%%beginhead 
 
%%file 11_02_2023.fb.syniepalov_ivan.mariupol.1.ya_povernus_dodomu.cmt
%%parent 11_02_2023.fb.syniepalov_ivan.mariupol.1.ya_povernus_dodomu
 
%%url 
 
%%author_id 
%%date 
 
%%tags 
%%title 
 
%%endhead 

\qqSecCmt

\iusr{Zlata Osipova}

У мене теж є фото з мітингу 22го, робила скріни з телефону. І хоч іноді
здається, що цього не станеться, але вірю, що повернусь і поплачу над кожним
будинком, якого немає... це моє місто сили, а поки сили є, значить і є це місце.

\begin{itemize} % {
\iusr{Іван Синєпалов}
\textbf{Zlata Osipova} і в Сніжне теж з'їздимо)

\iusr{Zlata Osipova}
\textbf{Іван Синєпалов} навіть не знаю... в Донецьк так само... страшно, хоч і знаю там тих, хто все ще «свої»... але думаю, що з'їздимо 💙💛
\end{itemize} % }

\iusr{Євген Овчаренко}

+1

\iusr{Євген Овчаренко}

А я щє в Антрацит хочу. Малу відвести, хрещеного показати, той самий коров'ячий став показати.

\begin{itemize} % {
\iusr{Іван Синєпалов}
\textbf{Євген Овчаренко} сфоткати стіну на Штергресі і стирити ліхтарі у Хрустальному)

\iusr{Євген Овчаренко}

Стирити слолб чугуний, так! ти чо плани здаєш, а? тобі теж столб треба?)

\iusr{Євген Овчаренко}
\textbf{Іван Синєпалов} а які там люди! ЛЮДИ!

\iusr{Євген Овчаренко}

ну тобто не Людміли, а просто люди)

\iusr{Lilia Syniepalova}
\textbf{Іван Синєпалов} а що з ліхтарями в Хрустальному? Що я проґавила?
А ще обов'язково зупинитися на спуску в Укуріне, подихати степом.

\iusr{Іван Синєпалов}
\textbf{Lilia Syniepalova} красиві там ліхтарі, на дачі такі потрібні)

\iusr{Євген Овчаренко}
\textbf{Lilia Syniepalova} В Штєровкє ліхтарі з 20х років щє,як оркотня не вкрала щє

\iusr{Євген Овчаренко}
\textbf{Іван Синєпалов} шариш!

\iusr{Lilia Syniepalova}
\textbf{Іван Синєпалов} не пам'ятаю. От так(
\end{itemize} % }


\iusr{Вікторія Кисіль}
\textbf{Oleksandra Vorotilova}

\iusr{Надія Умриш}

Не можу стримати сліз....течуть, гарячі та пекучі...

Коли я виїжджала з Маріуполя у 2019-му, точно знала, що н і к о л и не приїду
більше у це місто. Але після всього, що сталося, відчуваю такий біль... Серцем
з кожним, з ким звели мене події з 2014-го... І тепер хочу приїхати до
Маріуполя. Обов'язково.

\iusr{Lilia Syniepalova}

В мене ще одне місце сили є. Було. Є. Туди я теж повернуся. Можливо тільки для
того, щоб попити водички на Сифонній.

Все життя як перекоти-поле.

Але я повернуся. І туди, де прожила перші 17 років свого життя, і в Маріуполь,
де прожила 18. Який би він не був. Це моє місто.

\iusr{Serge Danylov}

аха, я приїду і випадково зустріну тебе на веліку, як завжди

\begin{itemize} % {
\iusr{Іван Синєпалов}
\textbf{Serge Danylov} звучить як план)
\end{itemize} % }

\iusr{Victoria Narizhna}

Поплакала. Я ніколи не була в Маріуполі, і це так болить тепер. Не думала, що
так болітиме місто, якого і не знала.

\begin{itemize} % {
\iusr{Іван Синєпалов}
\textbf{Victoria Narizhna} 

не вигадав нічого кращого, ніж у відповідь дістати з кишені Тичину:\par
Ще буде: неба чистої блакиті,\par
добробут в нас підніметься, як ртуть,\par
заблискотять косарки в житі,\par
заводи загудуть...\par
І я життям багатим розсвітаюсь,\par
пущу над сонцем хмарку, як брову...\par
Я стверджуюсь, я утверждаюсь,\par
бо я живу.\par
\end{itemize} % }

\iusr{Natalya Domina}

Як же боляче це читати. Але і не читати неможливо. Тримайся

\begin{itemize} % {
\iusr{Іван Синєпалов}
\textbf{Natalya Domina} дякую!
\end{itemize} % }

\iusr{Vlademir Tremazul}

Оце гарний текст для всеукраїнського диктанту

\begin{itemize} % {
\iusr{Іван Синєпалов}
\textbf{Vlademir Tremazul} забагато авторських розділових знаків)
\end{itemize} % }

\iusr{Lilia Syniepalova}

Скільки має пройти? Мій життєвий досвід підказує, що для того, щоб повністю
змиритися зі втратою, має пройти ± такий же відтинок часу, скільки в твоєму
житті було присутнє те втрачене. Ну, принаймні, в мене якось так.

Сподіваюся, Маріуполь знов українським стане набагато раніше.

\iusr{Игорь Дворниченко}

Море у поля давно не чув !!!!!!!!!!!завжди така назва в голові !!!

Красава!!!!!!!

А то місто Марії (асоціація з Бойченко)(((((((((

\iusr{Роман Перетятько}

Дім сниться все частіше, але інший.... По відчуттям це точно він (Маріуполь)
але зовсім інші будівлі...

\begin{itemize} % {
\iusr{Lilia Syniepalova}
\textbf{Роман Перетятько} 

я про сни вже писала. Мені ще задовго (не менше як за рік) до лютого почав
снитися один і той же сон періодично. Місто візуально незнайоме, але я чітко
знала куди йти і на чому куди мені треба їхати. Шукала подібне місце вже тут,
але ніт. Думаю, що то був Маріуполь. Тільки майбутній.

\iusr{Євген Овчаренко}
\textbf{Роман Перетятько} ви вірите в віщі сни?

\iusr{Роман Перетятько}
\textbf{Євген Овчаренко} ніт, як казав мій дід «дурне спить, дурне сниться»... але хто зна.

\iusr{Євген Овчаренко}
\textbf{Роман Перетятько} 

мені з 10+/- років сниться з періодичнінстю раз в місяц один і той же сон. що я
прихожу в свій двір, а він не зовсім мій. Ну типу дім не такий трошки, то там
не те, то тут не то. Як двір з якогось паралельного Всесвіту, умовного. І
станом на зараз я певен, що дрімз кам тру. Я побачу свій сон наяву. Бо двір
мій, але буде іншим.

А за пару років до повномасштабки - літаки які бомблять мій дім, літаки з Азову
летіли...

І що то є... як воно сниться. чи просто рандом шарашить і іноді вгадує.

такоє, як казав класик, втомився він від цих питань в общєм)

\iusr{Роман Перетятько}

Я думаю, що майбутнього немає, воно твориться саме зараз і кожної миті із всіх
подій що відбуваються... тобто зплетається як шарф чи светр із ниток(подій)... тому
ми не можемо бачити наперед, але можемо відчувати та вірити. Якось так.

\iusr{Євген Овчаренко}
\textbf{Роман Перетятько} дуже крута відповідь.
ми те що ми робимо, і ми станемо тими на що заслужили.
дуже класно сказані базові речі.

\iusr{Роман Перетятько}
\textbf{Євген Овчаренко} от пан \textbf{Іван Синєпалов} щось напише і ти сидиш до пів ночі коментуєш)

\iusr{Євген Овчаренко}
\textbf{Роман Перетятько} ага, а він то мабуть спить, і \textbf{Іван Синєпалов} сповіщєння вимкнув

\iusr{Іван Синєпалов}
\textbf{Євген Овчаренко} спить і бачить уві сні Сніжне)

\iusr{Victoria Rohozhyna}

І я тепер уві снах ходжу нашим містом, і так само знаю, куди йти, хоча все
навколо інакше. Підхожу до батьківської зруйнованої хати, заходжу в середину —
а там все ціле й неушкоджене.

\iusr{Lilia Syniepalova}

Народ, а уявляєте, якщо так і є і ми ходитимемо саме по таких місцях, котрі зараз бачимо в снах?

\iusr{Victoria Rohozhyna}
\textbf{Lilia Syniepalova} ага, я повірю у віщі сни ))

\iusr{Євген Овчаренко}
\textbf{Lilia Syniepalova} думаю +/- так і буде. не таке 1в1 що в снах бачимо, але відчуття будуть такіж самі.

\iusr{Lilia Syniepalova}
\textbf{Євген Овчаренко} нехай би скоріше.

\iusr{Євген Овчаренко}
\textbf{Lilia Syniepalova} у мене завжди сни літні. тепло на вулиці завжди було. ну то скоріш за все того що я під одеялком сплю, але все ж таки)

\iusr{Lilia Syniepalova}
\textbf{Євген Овчаренко} аналогічно

\iusr{Ольга Фомичева}
\textbf{Євген Овчаренко} 

все може бути.... за неділю до початка війни, я прийшла до дому з роботи, і
трохи прилегла відпочити, я ні спала, ні дрімала, була у повній свідомості і
тут переді мною, біля кроваті стала моя покійна мама. Вона була як монахіня вся
в чорному, вона так пронизливо подивилась мені в очі, і я відразу поняла, що
буде шось не добре. Це було якісь секунди і вона зникла. Але вона прийшла
попередити мене. Як в це не вірити???

\end{itemize} % }

\iusr{Сергей Карпенко}

Дякую що зісканував мої думки і написав цього поста. Якось так

\iusr{Сергiй Одаренко}

Як добре, що там були такі люди, як добре, що там будуть такі люди.

\iusr{Olexandr Chyzh}

Твоя фота надіслана 23.02.22. Не змінилось нічого, я щітаю

\ifcmt
  igc https://scontent-fra3-1.xx.fbcdn.net/v/t39.30808-6/330794660_728568561982979_5279384637252719755_n.jpg?_nc_cat=108&ccb=1-7&_nc_sid=dbeb18&_nc_ohc=aG7GmtWlrRcAX8kcdPb&_nc_ht=scontent-fra3-1.xx&oh=00_AfAdCy-ndezuSPIQJD9Ol98fTUd6nVdStdUfVYlWOla3gg&oe=63F26088
  @width 0.6
\fi

\begin{itemize} % {
\iusr{Іван Синєпалов}
\textbf{Olexandr Chyzh} іґзектлі!
\end{itemize} % }

\iusr{Іван Синєпалов}

Увічнили

\url{https://maidan.org.ua/2023/02/ya-povernus-dodomu/}

\ifcmt
  tab_begin cols=2,no_fig,center
    pic https://i.paste.pics/db3db172719d3bc91758392474c9818f.png
    pic https://i.paste.pics/3d7259c74af31aead9116e737cdac9c9.png
  tab_end
\fi

\iusr{Іван Синєпалов}

Thanks to \textbf{Ildiko Eperjesi} for the Hungarian translation.

\url{https://facebook.com/story.php?story_fbid=3318387385071045&id=100006994478705}

\ifcmt
  tab_begin cols=3,no_fig,center
     pic https://i.paste.pics/4d9152bd4ef6185c787e4a089ed3e785.png
     pic https://i.paste.pics/948af7079b465af1c7c3db61d019ab38.png
     pic https://i.paste.pics/2d665bc55f51a1472f1c77c0941b5acc.png
  tab_end
\fi

\begin{itemize} % {
\iusr{Ildiko Eperjesi}
\textbf{Іван Синєпалов} May God bless you and protect you all, dear people.
\end{itemize} % }

\iusr{Ілля Кишенський}

Я повернусь. Хоча і дехто цього дуже боїться 😂

\begin{itemize} % {
\iusr{Іван Синєпалов}
\textbf{Ілля Кишенський} вірю, що ти повернешся з вітерцем)

\iusr{Ілля Кишенський}
\textbf{Іван Синєпалов} дуже влучний вираз 💪
\end{itemize} % }

\iusr{Анна Мурликіна}

О, Ваня...

\iusr{Анна Мурликіна}

А мені сон подарував уяву, що я буду відчувати після повернення. Одна чудова
жінка уві сні зробила мені екскурсію. Все-все показала. І коли я дійшла пішки
від Західного до проспекту Миру, то почала так ридати уві сні, що аж
прокинулась. То я тепер точно знаю, що зі мною буде коли... Не знаю тільки, чи
витримає моє серце

\begin{itemize} % {
\iusr{Людмила Иванова}
\textbf{Анна Мурликіна} все у вас буде добре....
\end{itemize} % }

\iusr{Людмила Иванова}

\ifcmt
  igc https://scontent-fra3-1.xx.fbcdn.net/v/t39.1997-6/326776304_638151904732718_4188399286249340620_n.webp?stp=dst-webp_s180x540&_nc_cat=106&ccb=1-7&_nc_sid=ac3552&_nc_ohc=lVrwqD2eVl4AX-4P83X&tn=p3ZpkpW_rS8sgziF&_nc_ht=scontent-fra3-1.xx&oh=00_AfClmuEwLSMqzKUNhFDIjn2tI7FGrHr_7qXID1tSDVAd2g&oe=63F19567
  @width 0.2
\fi

\iusr{Анна Коваленко}

Я повернуся!!! І ми відбудуємо Наше місто...

А ще я повернусь в рідне місто Амвросієвку, це недалеко від м. Снідне...

\iusr{Оксана Спивак}

Я повернусь до реднесинького Марика!!!💖🙌🙏🙏🙏✌

\ifcmt
  igc https://scontent-fra3-1.xx.fbcdn.net/v/t39.30808-6/330969888_5627155920746203_6428853723623437407_n.jpg?_nc_cat=102&ccb=1-7&_nc_sid=dbeb18&_nc_ohc=7A5A1mKDIKgAX-1zb4y&_nc_ht=scontent-fra3-1.xx&oh=00_AfDf2V0Sx6l7d7DNNaONNSxxS-69MFiZEN5SlyO0RLyVNA&oe=63F13E79
  @width 0.6
\fi

\iusr{Наталья Никулина}

Моя родина теж живе надією на повернення до рідного міста Маріуполя!

\iusr{Ольга Мазур}

2014 виїжджали із Донецька, я не знала, чи зможу це пережити, двоє маленьких
дітей і слова чоловіка, ТРЕБА ЇХАТИ........ Але Маріуполь зустрів нас, тут ми
знайшли свій другий дім, знайшли нових друзів. Я дякую Маріуполю!!! Я вірю, що
ми повернемося! Я вірю, що я повернуся і в свій рідний Донецьк, який був, є і
буде тільки українським містом! Слава Україні 💙💛🇺🇦

