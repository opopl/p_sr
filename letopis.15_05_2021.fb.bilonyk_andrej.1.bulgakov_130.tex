% vim: keymap=russian-jcukenwin
%%beginhead 
 
%%file 15_05_2021.fb.bilonyk_andrej.1.bulgakov_130
%%parent 15_05_2021
 
%%url https://www.facebook.com/bilonyk.a/posts/3930766310369777
 
%%author 
%%author_id 
%%author_url 
 
%%tags 
%%title 
 
%%endhead 

\subsection{Сьогодні 130 річниця з Дня Народження М. А. Булгакова}
\label{sec:15_05_2021.fb.bilonyk_andrej.1.bulgakov_130}
\Purl{https://www.facebook.com/bilonyk.a/posts/3930766310369777}

\ifcmt
  pic https://scontent-frt3-1.xx.fbcdn.net/v/t1.6435-0/p526x296/186870591_3930766233703118_3490930605816142431_n.jpg?_nc_cat=109&ccb=1-3&_nc_sid=730e14&_nc_ohc=mZOlYZQC5a4AX9WmO-k&_nc_ht=scontent-frt3-1.xx&tp=6&oh=a08b6687aea640dd17925ac56b46ff89&oe=60C58BD5
\fi

З ким його ще відмічати, як не з найбільшим його адептом, чарівною Lada Luzina?

Варто сказати, що за весь тисячолітній період існування Києва, про нього був
написаний лише один геніальний роман. І це  "Біла гвардія" Булгакова, а не те,
що ви подумали.

Майстер надто любив Київ, щоб на сторінках його книг відображалися ще якісь
міста (хоч і писав він вже не перебуваючи в Києві, і назви там більшістю
московські, але ми ж знаємо?)

Так, він любив свій Київ. Білий. І проукраїнські Директорія Петлюри чи
Гетьманат Скоропадського сприймалися Майстром виключно вороже.

На жаль...
