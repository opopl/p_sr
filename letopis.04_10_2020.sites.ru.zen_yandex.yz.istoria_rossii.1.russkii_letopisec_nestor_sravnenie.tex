% vim: keymap=russian-jcukenwin
%%beginhead 
 
%%file 04_10_2020.sites.ru.zen_yandex.yz.istoria_rossii.1.russkii_letopisec_nestor_sravnenie
%%parent 04_10_2020
 
%%url https://zen.yandex.ru/media/history_russian/sravnenie-russkogo-letopisca-s-povestiu-vremennyh-let-o-chem-sovral-nestor-5f15a5212a244b7334c53fda
 
%%author 
%%author_id yz.istoria_rossii
%%author_url 
 
%%tags istoria,russia
%%title Сравнение «Русского Летописца» с «Повестью временных лет». О чём соврал Нестор?
 
%%endhead 
 
\subsection{Сравнение «Русского Летописца» с «Повестью временных лет». О чём соврал Нестор?}
\label{sec:04_10_2020.sites.ru.zen_yandex.yz.istoria_rossii.1.russkii_letopisec_nestor_sravnenie}
\Purl{https://zen.yandex.ru/media/history_russian/sravnenie-russkogo-letopisca-s-povestiu-vremennyh-let-o-chem-sovral-nestor-5f15a5212a244b7334c53fda}
\ifcmt
	author_begin
   author_id yz.istoria_rossii
	author_end
\fi

\index[rus]{Книги!Русский Летописец, 1649!Сравнение «Русского Летописца» с «Повестью временных лет», 04.10.2020}

На страницах разных интернет-изданий в последние два года публикуется текст
недавно найденной старообрядческой рукописи. Называется он «Русский летописец»
и составлен в 1649 году. Некоторые отрывки этого произведения произвели фурор и
позволили отдельным дилетантам говорить, что якобы найдена летопись,
рассказывающая о подлинной, а не придуманной истории Руси.

О чём же повествует «Русский Летописец»?


\ifcmt
  pic https://avatars.mds.yandex.net/get-zen_doc/3420563/pub_5f15a5212a244b7334c53fda_5f7992ca61e6d41ef5058621/scale_2400
	caption Русский Летописец, рукопись, 1649
	width 0.6
\fi

\subsubsection{Версия начала русской истории}

После разделения человеческого рода на 72 языка, описанного в Ветхом Завете,
один из правнуков Яфета Скиф поселился у Чёрного моря. Страна, населённая его
потомками, стала именоваться Великой Скифией. Возникла междоусобная распря
между ними. И тогда два брата Словен и Рус, которые «всех своих родичей
превзошли мудростью и храбростью», взяли своих людей и отправились с ними
искать новой земли для поселения.

Ходили они 14 лет и наконец обосновались на большом озере, которое Словен
назвал Ирмерь (Ильмень) в честь своей сестры. Там же они основали Словенск
Великий (Новгород). А от Руса произошло название города Старая Русса. «Русский
Летописец» объясняет в духе народной этимологии и другие названия Новгородской
земли, выводя их от имён родичей Словена и Руса.


\ifcmt
  pic https://avatars.mds.yandex.net/get-zen_doc/2808638/pub_5f15a5212a244b7334c53fda_5f79949571c44f0829183350/scale_1200
	width 0.6
\fi

Произошло это, по словам «Русского Летописца», в 3113 году от Адама. Потом
княжили сыновья Словена и Руса – Великосан, Асан и Авахосан. «Русский
Летописец» не уточняет, кто из них был чей сын, равно как не называет имена их
матерей. Эти три князя завоевали всю Сибирь, а также ходили в походы вплоть до
Египта и Греции.

В это же самое время был знаменитый царь Александр Македонский и он направил
словено-русским князьям послание, в котором признавал их властителями всех
земель от Варяжского (Балтийского) до Хвалынского (Каспийского) моря, но
заповедал вступать им в иные земли. Великосан, Асан и Авахосан приняли условие
Александра Великого и его грамоту торжественно повесили в храме бога Велеса в
Ростове.

Затем в Словенской земле наступил великий мор, и опустела земля Словенская, а
уцелевшие жители бежали на Дунай. Спустя какое-то время словене, скифы и
болгары во множестве снова возвратились к озеру Ильмень и заселили
Словено-Русскую землю, а князем своим избрали Гостосмысла. 


\ifcmt
tab_begin cols=2
  pic https://avatars.mds.yandex.net/get-zen_doc/3700994/pub_5f15a5212a244b7334c53fda_5f79958161e6d41ef509bee0/scale_1200
	width 0.45

  pic https://avatars.mds.yandex.net/get-zen_doc/1576786/pub_5f15a5212a244b7334c53fda_5f79980771c44f08291d8a3c/scale_1200
	width 0.4
tab_end
\fi

И заново основали Словенск Великий, которых с тех пор называется поэтому
Новгородом. И много других племён словенских возникло от переселенцев. А когда
Гостосмысл стал умирать, то не стал завещал власть своим сыновьям по причине
междоусобицы в своём народе, но приказал призвать на царство Рюрика, потомка
кесаря Августа, из Прусской земли.

Дальнейшая история Руси по «Русскому Летописцу» уже принципиально не отличается
от версии «Повести временных лет».

\subsubsection{Сказка – ложь, да в ней намёк}

Вопреки дилетантским утверждениям, что «Повесть временных лет» исказила
сведения, сообщаемые «Русским Летописцем», знакомство с текстом того и другого
памятника не оставляет сомнения, что именно ПВЛ служила источником для
«Русского Летописца». Хотя, правда, не единственным источником.

Наличие предания о Гостосмысле (Гостомысле) показывает, что Иоакимовская
летопись не была выдумкой Татищева. В России, наряду с традицией, выводящей
русскую государственность из Киева, параллельно развивалась традиция о начале
Руси, связанном с Новгородом Великим. И то, что в итоге возобладала «киевская»
версия, было, очевидно, делом политической конъюнктуры, связанной с большим
притоком в Московское государство книжных людей из Киева в 17 веке.

\ifcmt
tab_begin cols=2
  pic https://avatars.mds.yandex.net/get-zen_doc/1852523/pub_5f15a5212a244b7334c53fda_5f79987b8d3ae5589b5d70a2/scale_1200
	width 0.3

  pic https://avatars.mds.yandex.net/get-zen_doc/168279/pub_5f15a5212a244b7334c53fda_5f799aeb8d3ae5589b611167/scale_1200
	width 0.4
tab_end
\fi

Несомненно, что Кий, Щек и Хорив – такие же выдуманные персонажи, как Скиф,
Словен, Рус, Волхов, Жилотуг и прочие герои «Русского Летописца». Кстати, по
нему Кий, Щек и Хорив жили во времена Вещего Олега и были им убиты, после чего
Олег стал править в Киеве. Про Аскольда и Дира «Русский Летописец» не ведает.

По всему строению и сообщаемым «фактам» «Русский Летописец» имеет образ
народной сказки. Но это не значит, что ему нужно во всём предпочесть «Повесть
временных лет». Наоборот, фантастические и наивно-исторические моменты
«Русского Летописца» (вроде завоевания всей Византии князем Игорем или
основания Москвы Вещим Олегом) должны послужить основой для критического
анализа самой ПВЛ. Там немало подобных же сказочных сюжетов, вроде того же
предания об основании Киева Кием или призвания в Новгород неких непонятно
откуда варягов-Руси. 


