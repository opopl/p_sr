% vim: keymap=russian-jcukenwin
%%beginhead 
 
%%file 03_07_2018.fb.lesev_igor.1.chej_krym
%%parent 03_07_2018
 
%%url https://www.facebook.com/permalink.php?story_fbid=1980280312003062&id=100000633379839
 
%%author_id lesev_igor
%%date 
 
%%tags krym,rossia,ukraina
%%title Чей Крым?
 
%%endhead 
 
\subsection{Чей Крым?}
\label{sec:03_07_2018.fb.lesev_igor.1.chej_krym}
 
\Purl{https://www.facebook.com/permalink.php?story_fbid=1980280312003062&id=100000633379839}
\ifcmt
 author_begin
   author_id lesev_igor
 author_end
\fi

Чей Крым?

В принципе, это уже не вопрос, а история. История одного большого
переформатирования и одного быдло-предательства.

Начнем с России. Русские уже с Крымом пропетляли. Успешно и с малыми потерями
на тактическом уровне. Санкции не работают. Изоляции нет. Ни одна страна мира
дипотношения с РФ не разорвала. Даже Украина. Что особенно чудесно. В Минских
соглашениях о Крыме ни слова. Да и в целом, мировая дискуссия сворачивает в
сторону «а давайте на все это посмотрим чуть иначе».

\ifcmt
  ig https://scontent-frx5-1.xx.fbcdn.net/v/t1.6435-9/36603027_1980280032003090_4739134929508499456_n.jpg?_nc_cat=105&ccb=1-5&_nc_sid=730e14&_nc_ohc=Flan519JywEAX_kbtRE&_nc_oc=AQk2M5CFU6mtpnfCf0Z9yycD1t3lu7X2zUJXgCETlraZxMCKn2iubJqjf_BqLCW1b4o&_nc_ht=scontent-frx5-1.xx&oh=e980b9f0a352ff3b10024559c88136f0&oe=61B749CD
  @width 0.4
  %@wrap \parpic[r]
  @wrap \InsertBoxR{0}
\fi

Россия в Крыму законтачилась железобетонно. Но одновременно, взяв Крым,
ру-элита открестилась от теории «русского мира» и русскости, как таковой.
Кремль продолжил линию российскости, заложенной еще Ельциным, где бабло и
бабло-проекты доминируют над иррациональными идеями «единой общности».
Российскость берет то, что выгодно, а не то, что должно. На себе это прекрасно
ощутили ЛДНР-овцы, которые не так встали.

Но сейчас не о Донбассе. Присоединением Крыма Путин решил еще одну важнейшую
задачу во внутренней политике. Он заложил тренд на политическую элиту,
окончательно выбив оттуда даже теоретическую угрозу либерального реванша.
Вопрос «чей Крым?» гораздо важнее не у нас, а у них. Особенно это наглядно
продемонстрировали последние президентские выборы в РФ с подсадным участием
Ксюши Собчак. До Собчак любой либерал в России на вопрос «что будете делать с
Крымом?» отвечал витиевато-непонятно. «Ну нам нужно как-то договариваться с
Украиной. Мы не совсем красиво поступили. Но и мнение людей в Крыму тоже надо
учитывать. Это сложный вопрос. Мы должны уважать международное право…». Ну вот
где-то так говорили о Крыме Навальный и Явлинский.

Собчак поставила водораздел, назвав Крым однозначно украинской территорией и
даже делала показной в своем идиотизме запрос в СБУ с просьбой посетить
полуостров через украинскую территорию. В итоге, либеральное движение в России
девальвировано и полностью маргинализировано. Вообще, слово «либерал» в России
уже дважды прошло словарь новояза. Первым над ним поиздевался Жириновский со
своей ЛДПР. А окончательно добил Путин руками/языком Собчак. И теперь любые
сменщики Путина через 6+ лет отвечают на первый стратегический вопрос – чей
Крым?

Теперь переходим к Украине. Наша страна переживает двойную трагедию. Идиоты и
воры у власти оказались еще трусами и предателями. Сама история присоединения
Крыма к России – это история двухнедельного опасливого выжидания со стороны
России и безвольного договорняка с последующей сдачей полуострова, со стороны
высшего украинского руководства.

Потеря Крыма также произвела тектонические сдвиги в украинском обществе. Но в
отличие от российского, где произошло закрепление элиты путинского образца, в
украинском обществе заработали лифты для проходимцев, популистов и просто
тупиц. Украинское общество, жадное на хорошие и просто обнадеживающие новости,
оказалось в плену сказочников.

Разберем сказки о Крыме на соседнем турецком примере. Итак, киевские
элит-сказочники назвали Крым «временно оккупированной территорией». Словесная
эквилибристика, греющая раненую душу украинских патриотов. Оккупированная.
Значит там есть внутреннее сопротивление. Да еще и временно. Значит, как Иисус
второй раз придет на Землю, так и в Крыму засияют лучи счастья и свободы
желто-синего цвета.

Другая сказка от наших сказочников – весь мир с нами. Каждое утро бельгиец,
индонезиец и перуанец, спросонья тянется к радио/ТВ и лихорадочно выискивает
новую информацию об Украине и Крыме. Ну же, что там у них? Украинцы держатся?
Русские оккупанты не продвинулись чуток в Херсонской области? Путин еще не
сдох? И вот так каждый день. Каждый день весь мир с нами и держит за нас
кулачки. А мы все – неблагодарная быдлота из Жмеринки и Шепетовки – должны в
ноги кланяться новой постмайданной элите за мирное небо над нашими головами, за
мирный Париж и за возможность попить кофе в Братиславе.

По соседству через море от нас лежит остров Кипр. Проблемный был остров, но
сейчас не о его истории. Так вышло, что в 1974 году на его севере высадились
«вежливые люди» из соседней Турции. И с тех самых пор по эту самую минуту есть
два Кипра. Тот что южнее – даже член Евросоюза. И зоны евро. А те что севернее,
почему-то все равно не бьются в падучей, что хотят обратно воссоединиться и
стать членом одной большой европейской страны.

Турция, которая по факту и оккупировала северную часть Кипра, не испытывает
АБСОЛЮТНО никакого международного давления. Всем пох. Ну, в 70-х чуть мировое
сообщество повозмущалось. И то, очень аккуратно. Потому что у Турции проливы. А
теперь у Турции газопроводы и все те же проливы. И экономика в первой
двадцатке. И Северный Кипр в мировом разрезе по большому счету никого не ипет.

И таких примеров в мире как говна. Марокко контролирует 80\% Западной Сахары.
Объявили своей территорией. ООН, типа, не согласны. И че? Пакистан у Индии с
40-х годов хапанул западную часть Кашмира и даже формально создал там
самопровозглашенное государство. Какие есть предпосылки, что Индия с гигантской
экономикой и такой же гигантской армией вернет то, что по факту давно уже ей не
принадлежит?

Из истории с Крымом надо извлечь два урока. Первый – просранное не
возвращается. И второй. Если не наказать виновных за 14-й год, история может
повториться.

\ii{03_07_2018.fb.lesev_igor.1.chej_krym.cmt}
