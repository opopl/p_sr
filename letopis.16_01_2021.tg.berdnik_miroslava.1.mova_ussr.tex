% vim: keymap=russian-jcukenwin
%%beginhead 
 
%%file 16_01_2021.tg.berdnik_miroslava.1.mova_ussr
%%parent 16_01_2021
 
%%url 
 
%%author_id 
%%date 
 
%%tags 
%%title 
 
%%endhead 

Несколько дней назад глубоко уважаемый мной Михаил Погребинский поразил меня
тем, что заявил, что с получением независимости украинский язык нуждался в
защите и его надо было защищать после советской русификации.

Я хочу напомнить о том, как в советское время угнетался украинский язык.
Считаю, что являясь носителем украинского языка, имею на это полное право.

\ii{16_01_2021.tg.berdnik_miroslava.1.mova_ussr.pic.1}
 
Так случилось, что я - сама украиноязычная украинка. Причем, выросшая в семье
классика украинской литературы и выраставшая и впитавшая любовь к родному -
настоящему украинскому языку, а не "дерьмове", на коленях классиков, которых
изучают в учебниках многие десятилетия: я вырастала на коленях Лины Костенко,
Андрея Малышко, лучшего детского писателя 20 века Ивана Сенченко. Сидела у
колен украинской еврейской писательницы Марии Пригары в доме творчества в
Ирпене, когда она писала лучшие детские книги про казаков \enquote{Михайлик - джура
козацький} и \enquote{Козак Голота}. Мне сын Ивана Франко Тарас Франко дарил свои книги
и рассказывал об отце. У культовых украинских историков Олени Апанович
(написавшей монографию о запорожском казачестве \enquote{Збройні сили України} Олени
Компан встречалась с Михаилом Брайчевским и внучкой Михаила Коцюбинского. Я
сама слушала на концерте в Музыкально-Хоровом обществе в 1973-м году
триумфальный концерт тех самых народных кобзарей, о которых сегодня сочиняют,
что их в 30-е расстреляли.

С бандеровцами (а также с теми, кто с ними боролся) я тоже знакома не
понаслышке. В тех самых кругах, еще ребенком я встречалась с \enquote{последним
командиром УПА}, как его сегодня называют - Василем Куком и любовницей
Шухевича, сдавшей его МГБ - Даркой Гусак. Поэтому согласно теории о шести
рукопожатиях можно сказать, что между мной и Бандерой с Шухевичем - всего по
одному рукопожатию.

А чтобы  не сочиняли об уничтожении украинского языка, проведу для вас
небольшой ликбез: В 70-е – 80- е в год выходило примерно 9 тыс. наименований
книг общим тиражом около 190 млн. Из них 100 млн. – на украинском, 90 – на
русском. Сегодняшним пЫсьменнЫкам и не снились тиражи по 65 тыс. экземпляров,
а такие как Драч, Павлычко, Загребельный, Олесь Гончар и другая писательская
номенклатура имели тиражи гораздо выше. «Двадцать лет запрещенный», как
сегодня лживо пишут, роман Гончара «Собор», вышел в год издания аж три раза –
сначала в мягкой обложке в серии «Романи та повісті», потом два раза – в
твердой обложке. Это не говоря о всесоюзном бенефисе этого романа. К примеру,
книги моего отца выходили массовыми тиражами по 65 тыс. экземпляров.

Все республиканские и областные телеканалы были на украинском языке. «Промінь»
и другие радиоканалы, думаю, помнят многие. На украинском выходило большинство
республиканских, областных и районных газет Украины. Выходила куча ярких,
интересных журналов: «Дніпро», «Всесвіт», «Київ», «Вітчизна», «Прапор», «Наука
та релігія», «Україна», «Перець». Вообще, для русскоязычных авторов на Украине
выходил только один толстый журнал – «Радуга». Для детей, в отличие от
сегодняшего «нЭзалежного бутя» выходили на украинском языке десятки газет и
журналов. Стоимостью от 1 копейки за газету «Зірка» до 20-30 копеек за журнал.
Среди них для детей и молодежи: уже названная «Зірка», «Барвінок», «Малятко»,
«Піонер», «Зміна», «Піонерія», «Знання та праця», «Ранок», «Наука і техніка» и
многие другие. Я не упоминаю специализированные типа журнала «Археологія
України», областные или ведомственные.

На проспекте Победы находился огромный издательский комбинат «Радянська
Україна», в его помещении – кроме редакций десятков газет и журналов – логика
ведь подсказывает, что если на украинском выходила газета «Говорить і показує
Україна», то и передачи были на украинском. В здании комбината также находилось
огромное, с корреспондентской сеткой по всему миру украиноязычное
Радиотелеграфное агенство Украины (РАТАУ). Теперь это малоконкурентное агенство
Укрінформ.

Назовем теперь некоторые издательства, которые миллионными тиражами выпускали
на украинском зыке классику мировой литературы, а также украинских и советских
авторов. Издательство детской литературы «Веселка» (раньше «Дитвидав»),
издательства для молодежи «Молодь» и «Дніпро». «Радянський письменник»,
«Укрполітвидав», «Каменяр», «Музична Україна» выпускала ноты, песни, а также
литературу этнографических экспедиций по украинской глубинке. «Радянська школа»
и «Вища школа» - учебники для средней и высшей школы, а также многочисленную
техническую и другую литературу. Издательство «Мистецтво» - альбомы живописи и
литературу по искусству. «Академкнига», «Наукова думка» - научную литературу.
Практически в каждой области существовали областные издательства, к примеру
издательство \enquote{Крим}, которое выпускало прекрасные книги на украинском языке.

Приведу примеры названий нескольких книжных серий, которые на протяжении многих
лет выпускались для бесправных безъязыких и бессловесных украинских «рабов
Москвы». В серии «Вершини світової лірики» выходила в переводах лучших
украинских переводчиков лучшая зарубежная поэзия. «Вершини світового
письменства» говорят сами за себя. В серии «Шкільна бібліотека» массовыми
тиражами выходила украинская и зарубежная классика на украинском языке. Точно
так же, как и в серии «Бібліотека художніх творів для підлітків та юнацтва
«Джерело». В серии «Бібліотека поета» выходила классика украинской поэзии.
Название серии «Життя славетних» говорит само за себя. Так же, как и серии
\enquote{Давньоруські та українські літописи}.

Я назвала далеко не все. Но это и есть итоги «советской русификации Украины».
Но тогда, когда люди имели право свободно выбирать на каком языке разговаривать
и можно было привычно видеть, как один человек говорил по-украински, другой
отвечал ему по-русски (и наоборот) и они даже не замечали этого, никому в
голову не могло прийти ненавидеть украинский язык, как средство принуждения и
порабощения.

На фото (\url{https://t.me/Varjag2007/15146})  книги моих отца и мамы
