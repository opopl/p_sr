% vim: keymap=russian-jcukenwin
%%beginhead 
 
%%file 28_09_2021.fb.fb_group.story_kiev_ua.1.gogol_i_kiev.cmt
%%parent 28_09_2021.fb.fb_group.story_kiev_ua.1.gogol_i_kiev
 
%%url 
 
%%author_id 
%%date 
 
%%tags 
%%title 
 
%%endhead 
\subsubsection{Коментарі}

\begin{itemize} % {
\iusr{Victor Gregory}
С географией у Коли было так себе, раз по пути из Полтавы в Нежин проскочил Киев.

\iusr{Milana Piskalenko}
Браво!

\iusr{Tatiana Berlin}
Во торкнуло.....

\iusr{Ольга Кирьянцева}

Просто замечательно! Тонкая ирония Вам очень удалась.
@igg{fbicon.hands.applause.yellow}  @igg{fbicon.face.happy.two.hands} 

\iusr{Vladimir M. Pushkarev}
Маленькая неточность: Паниковский там не жил, а работал.

\begin{itemize} % {
\iusr{Аркадій Друкер}
\textbf{Vladimir M. Pushkarev} а рядом жил. Потомушо у него нога была больная. Куда же ему с больной ногой? Поезжайте в Киев и спросите.

\iusr{Vladimir M. Pushkarev}
\textbf{Аркадій Друкер} Рядом жить он не мог - не по Сеньке шапка!

\iusr{Lara Ilich}
\textbf{Vladimir M. Pushkarev} и до сих пор работает...)

\iusr{Yan Nudelman}
\textbf{Vladimir M. Pushkarev} Конечно жил. Мы были соседями по коммунальной квартире.

\iusr{Аркадій Друкер}
\textbf{Yan Nudelman} Теперь я догадываюсь, кто у него трость оторвал. За что вы его так не любили?!

\iusr{Vladimir M. Pushkarev}
\textbf{Yan Nudelman} Тогда коммунальных квартир еще не было - большевики еще не угнездились.
\end{itemize} % }

\iusr{Victor Gregory}
Кстати, на углу Крещатика и Кадетской ещё перепичку не продавали, иначе он бы ни в какие Италии не уехал.

\iusr{Наталья Дрягун}
Милый рассказ  @igg{fbicon.smile} 

\iusr{Ирина Иванченко}

Вот до чего прекрасна страна/ народ, когда про её / его лучшие красоты /
чесноты самым талантливым мозгам которой/ ого гораздо продуктивней/ эффективней
описывать их ( красоты)из - за границы!

\begin{itemize} % {
\iusr{Vadim Basovskiy}
\textbf{Ирина Иванченко}

\enquote{Почему иностранец менее стремится жить у нас, чем мы в его земле? Потому, что
он и без того уже находится за границей}. Козьма Прутков.


\iusr{Michael Viz}
\textbf{Vadim Basovskiy} это про Россию. Не перепутали?

\iusr{Dubizhansky Ludmila}
\textbf{Ирина Иванченко} Дык большое видится на расстоянии! Это еще классик заметил

\iusr{Галина Петрова}
вот чего?...- ни чего!)

\end{itemize} % }

\iusr{Тамара Ар}

Как всегда, типа в духе \enquote{сатиры}, прошлись по Николаю Васильевичу Гоголю!
Автор, вам до него все равно далеко! Не утруждайтесь

\begin{itemize} % {
\iusr{Ірина Кравець}
\textbf{Тамара Ар} А Ваше мнение истина в последней инстанции?

\iusr{Аркадій Друкер}
\textbf{Toma Ar} но вот вы же утруждаетесь. Не пишите. И будет вам щасте.

\iusr{Tatiana Thoene}
\textbf{Toma Ar} и Вы не утруждайтесь читать. Пропустите и промолчите.

\iusr{Kate Levenko}
\textbf{Toma Ar} Підкажіть, а ви буваєте чимось задоволені? Ну там, зранку встали і все добре, сонечко світить, пташки співають? Що не коментар - негатив, вже почала вас впізнавати за аватаркою  @igg{fbicon.beaming.face.smiling.eyes}  А автор молодець, тонко підмітив такий от типаж  @igg{fbicon.wink} 

\iusr{Наталия Веремеенко}
\textbf{Тамара Ар}
Нате і мою баночку на
свячену воду!.. @igg{fbicon.face.eyebrow.raised} 

\iusr{Наталья Эгатова}

\ifcmt
  ig https://scontent-mxp1-1.xx.fbcdn.net/v/t39.1997-6/p480x480/105941685_953860581742966_1572841152382279834_n.png?_nc_cat=1&ccb=1-5&_nc_sid=0572db&_nc_ohc=_iE3XtYBdGYAX_yWoLc&_nc_ht=scontent-mxp1-1.xx&oh=c51abc2b22b58bf5986c988f967a31d1&oe=61B7FD4B
  @width 0.2
\fi

\iusr{Наталья Твердохлеб}
\textbf{Тамара Ар} Мне кажется, что автор и не думал соревноваться с Николаем Васильевичем. @igg{fbicon.smile}  Он просто хотел сделать наше утро добрым. Я, например, благодарна @igg{fbicon.smile} 

\iusr{Света Арсеньева}
\textbf{Тамара Ар} Как по мне Золотушкин покруче Гоголя.  @igg{fbicon.wink} 

\iusr{Alik Perlov}
\textbf{Тамара Ар} С утра негатив у вас))). Съешьте конфетку Шоколадную, попустит ))

\iusr{Yan Nudelman}
\textbf{Тамара Ар} Дык это написано не для вас.

\end{itemize} % }

\iusr{Татьяна Вакуленко}
Вот хорошо же!
Сделали мое пасмурное утро!

\iusr{Надежда Кальная}
Мне понравилось! Короткий, но всеобъемлющий экскурс в жизнь великого! @igg{fbicon.hands.applause.yellow} 

\iusr{Ірина Кравець}
Какая с утра недобрая! Какая-то желчная и вдругих коментах/как-то уж заметно/не
хотелось писать, но уж очень явно

\begin{itemize} % {
\iusr{Анатолий Золотушкин}
\textbf{Ірина Кравець} О них и написано
\end{itemize} % }

\iusr{Валерий Якимчук}
\enquote{У нас тут своих с носами некуда девать}.(( Увы, это исторически не
достоверно... @igg{fbicon.wink} 

\iusr{Энна Кот}
Треба цінувати своїх геніїв!

\begin{itemize} % {
\iusr{Аркадій Друкер}
\textbf{Enna Kot} але це не означає, що не можна жартувати, використовуючи їх імена. Гадаю, вони не образилися би, маючи гарне почуття гумору.
\end{itemize} % }

\iusr{Татьяна Тихонова}
Точно, ёмко, в меру саркастично, спасибо!

\iusr{Виктор Задворнов}

Ой, зря вы так о Нежине. Как давно вы там были? Да, это не столица. Здесь
уютнее, спокойнее... И главное для меня, в местном драмтеатре иногда и Гоголя
дают.

\begin{itemize} % {
\iusr{Анатолий Золотушкин}
\textbf{Виктор Задворнов} прошу прощения за Нежин. Просто к слову пришлось

\begin{itemize} % {
\iusr{Виктор Задворнов}
\textbf{Анатолий Золотушкин} 

Нежин - прекрасный городок. Во все времена и сезоны. Навестите, убедитесь.
Удручает лишь одно место. Желдорвокзал. Увы: бомжи \enquote{слетаются} сюда по вечерам.
Неспокойно даже в зале для отдыха (за деньгу). А так - масса локаций для
неутомимых туристов.

\iusr{Наталья Твердохлеб}
\textbf{Виктор Задворнов} По этой части наш киевский вокзал все равно победить невозможно.  @igg{fbicon.smile} Думаю, по сравнению с ним, нежинский — Букингемский дворец.  @igg{fbicon.smile} 

\iusr{Виктор Задворнов}
\textbf{Наталья Твердохлеб} 

если бы полиция не выгоняла бомжей с киевского вокзала, их бы вообще не было на
нежинском, где полиция отсутствует. СообЧающиеся сосуды - это о вокзалах.

\iusr{Наталья Твердохлеб}
\textbf{Виктор Задворнов} 

С «черного» входа в вокзал (со стороны улочки Пестеля), где все эти «обитатели»
открыто торгуют наркотиками, пьют и спят в вечной грязи и мусоре, никогда нет
никакой полиции. Стражи порядка «служат народу» с другой стороны, с
центральной. Так что, с нашими все хорошо. @igg{fbicon.smile} 

\iusr{Alik Perlov}
\textbf{Виктор Задворнов} А река в Нежине есть??

\iusr{Виктор Задворнов}
\textbf{Alik Perlov} а как же! Хотите вплавь добраться!?))

\iusr{Alik Perlov}
\textbf{Виктор Задворнов} Переселиться )) реку люблю, рыбалку @igg{fbicon.wink} 

\iusr{Kate Levenko}
\textbf{Наталья Твердохлеб} 

Не бували ви на Ніжинському вокзалі  @igg{fbicon.smile}  У нас, звісно, то все
в масштабі, але на Ніжинському настільки концентровано... Ууух. Останній раз
була там років 10 тому, і, чесно кажучи, не скучила
@igg{fbicon.beaming.face.smiling.eyes}  Але якщо їхати в Ніжин на поїзді, чи
\enquote{підвищеній} електричці, можна на вокзал не заходити - бігом по перрону
і в місто  @igg{fbicon.wink} 

\iusr{Наталья Твердохлеб} 
\textbf{Kate Levenko} 

Врахую @igg{fbicon.smile}
Дякую)

\iusr{Kate Levenko}
\textbf{Alik Perlov} 

З Ніжина всі їздять на Десну, багато містян мають дачі в селах в тому напрямі,
ціни на будиночки дуже низькі. Там рибалка супер, навіть я, не фанат, люблю
туди їздити рибалити. Бо ж ловиться. Лящі, щуки, а карасі, красноперка там - то
як сорна риба, для котів  @igg{fbicon.smile}  Кладьківка-Авдіївка-Салтикова Дівиця - пляжі кращі
за наше Чорноморське узбережжя, неймовірно красива природа, все дешево. Ну і
навколо Ніжина ліси, де реально повно грибів, ось цього року чоловікові батьки
знайшли білий на 600 грамм. Річка Остер в місті чисто для краси... В Графському
парку озеро, рибаки сидять, але ну таке...


\iusr{Alik Perlov}
\textbf{Kate Levenko} щиро дякую за росповидь ))) Треба помиркувати над цим)))

\iusr{Alik Perlov}
\textbf{Kate Levenko} Скильки километров вид Киева???

\iusr{Kate Levenko}
\textbf{Alik Perlov} 

Ніжин десь 150, села так само, від Ніжина до Десни десь 45-50 км. Схожі місця є
у нас вздовж Десни, Вища Дубечня і далі, біля Остра - але людей, звісно,
більше, ніж там. Ну і ціни там на нерухомість вражають, батьки чоловіка кілька
років тому купили будинок і 40 сотих за 2500\$, перед будинком тільки поле і
далі Десна. Влітку живуть більше там, зробили ремонт, террасу і насолоджуються
звідти заходом сонця  @igg{fbicon.smile} 


\iusr{Alik Perlov}
\textbf{Kate Levenko} Дякую за инфу)))

\iusr{Kate Levenko}
\textbf{Alik Perlov} 

Там є кілька баз відпочинку - можна просто поїхати подивитись на місця, \enquote{Дім
хаскі}, \enquote{Убідь} (цивілізація таки наступає). В домі хаскі прогулянки з хаскі по
луках, катання на конях, ми колись їздили на екскурсію - особливо дитина
вражена була.

\end{itemize} % }

\iusr{Ирина Иванченко}
\textbf{Виктор Задворнов} , недавно Оксана Дубинина сделала обзор экскурсии по нежинским местам, очень, очень мило, не отказалась бы её,,рейд" повторить.

\iusr{Tatiana Thoene}
\textbf{Виктор Задворнов} ведь написали о Нежине во времена Гоголя; а Вас ведь тогда там не было ))

\begin{itemize} % {
\iusr{Виктор Задворнов}
\textbf{Tatiana Thoene} вчитайтесь в \textbf{Анатолий Золотушкин}!!!

\iusr{Tatiana Thoene}
\textbf{Виктор Задворнов} вчитываюсь и с удовольствием.
\end{itemize} % }

\iusr{Анна Загорулько}
\textbf{Виктор Задворнов} Согласна! Нежин обижать не надо! Отличный город!

\iusr{Kate Levenko}
\textbf{Виктор Задворнов} 

Так, приємне містечко, мій чоловік звідти, дуже жаль, що не дуже популярне у
туристів і туристичної інфраструктури майже немає. Тут посилання не можна, але
за пошуком в ютьюбі можна знайти його відео: \enquote{Тисячолітнє місто Ніжин}. Іще
нещодавно траплялась справжня екскурсія по місту, можна в ютьюбі знайти за
запитом: \enquote{Город Нежин. Ритм жизни}.


\iusr{Lara Ilich}
\textbf{Виктор Задворнов} у меня сын знакомых в Нежине учится. Как ему там, спрашиваю - \enquote{сильно хлопци все бьются } - ответил.


\iusr{Виктор Задворнов}
\textbf{Michael Viz} ЛИцей

\iusr{Michael Viz}
\textbf{Виктор Задворнов} Царскосельский Лицей

\iusr{Michael Viz}
\textbf{Виктор Задворнов} т9 и не такое может выдать, дяк

\iusr{Виктор Задворнов}
\textbf{Michael Viz} Т9, коньячно

\iusr{Michael Viz}
Шупер!!

\iusr{Виктор Задворнов}
\textbf{Michael Viz} это вы куда?

\iusr{Аркадій Друкер}
\textbf{Виктор Задворнов} то же самое хотел отметить )))

\end{itemize} % }

\iusr{Наталия Бакалейко}
 @igg{fbicon.laugh.rolling.floor}  @igg{fbicon.hands.applause.yellow}{repeat=3} 

\iusr{Tatiana Thoene}
Золотушкину - браво! Легко и с юмором. А у кого чувства юмора нет или оно их покинуло - искреннее сочувствие.

\iusr{Анатолий Золотушкин}
\textbf{Tatiana Thoene} спасибо @igg{fbicon.face.grinning.big.eyes} 

\iusr{Радикс Рад}
Как можно уехать из Киева, погуляв по Владимирской горке!!! Как?!!

\begin{itemize} % {
\iusr{Анна Загорулько}
\textbf{Радикс Рад} Молча - возвращаться будет приятно)

\iusr{Радикс Рад}
\textbf{Анна Загорулько} возвращаться можно на Монмартр, а на Владимирской нужно периодически гулять, для восполнения сил и энергии)

\iusr{Анна Загорулько}
\textbf{Радикс Рад} Я вижу, Вы поклонник именно Владимирской горки! Я тоже люблю ее, но не только)

\iusr{Радикс Рад}
\textbf{Анна Загорулько} не только. Есть масса любимых мест. Вчерась гулял в Муромце - релакс  @igg{fbicon.smile} 

\iusr{Alik Perlov}
\textbf{Радикс Рад} И как было?? Как кухня? Или это не ресторан ? @igg{fbicon.wink} 

\iusr{Радикс Рад}
\textbf{Alik Perlov} я про парк. Есть на острове и рестораны, но пока ещё мною не проверялись)
\end{itemize} % }

\iusr{Нина Волченкова}

\ifcmt
  ig https://i2.paste.pics/6fb9dacbae68a3773bc4959d0c8f011a.png
  @width 0.3
\fi

\iusr{Elen Zhorniak}
Браво

\ifcmt
  ig https://i2.paste.pics/12e5d3fc740dd48c05d934a581eac88b.png
  @width 0.4
\fi

\iusr{Elen Zhorniak}
Родилась под Полтавой, Киев не проскочила, прямо и давно...
@igg{fbicon.face.tears.of.joy}{repeat=3} А вот Рим, можно подумать уже.... Ибо
@igg{fbicon.flame}{repeat=4} 

\iusr{Елена Чичик}

\ifcmt
  ig https://scontent-mxp1-1.xx.fbcdn.net/v/t39.1997-6/s168x128/17640308_1652591141433953_2515677274297073664_n.png?_nc_cat=1&ccb=1-5&_nc_sid=ac3552&_nc_ohc=yClL-RbVaBQAX_Kfil-&_nc_ht=scontent-mxp1-1.xx&oh=78de06a53fd416e63ac5f72a2358a9d3&oe=61B78E2E
  @width 0.1
\fi

\iusr{Наталия Кирьянова}

\ifcmt
  ig https://i2.paste.pics/a9825c78009aabbb1b5f1f8865c3a1d4.png
  @width 0.3
\fi

\iusr{Сергей Кравцов}
Золотушкин говорите?

\iusr{Анатолий Золотушкин}
\textbf{Сергей Кравцов} ?

\iusr{Alik Perlov}
\textbf{Анатолий Золотушкин} Павлины говоришь.... Хе))))

\iusr{Сергей Кравцов}
Вот и я о том же!

\iusr{Игорь Семенов}
Гоголь со своей птицей, предвосхитил разом всех модернистов, видение такое у художника

\iusr{Rayisa Dmytrenko}
Його звали Миколка.

\iusr{Сергей Кравцов}
Это какой такой теперь Гоголь? Писательница теперь Ницой! И не важно, что она еще и рыбой торгует.

\iusr{Аркадій Друкер}
\textbf{Сергей Кравцов} так и Гоголь не только поэмы писал.

\iusr{Savelyeva Olena}
Отличный текст!

\ifcmt
  ig https://scontent-mxp1-1.xx.fbcdn.net/v/t39.1997-6/s168x128/17634213_1652591098100624_731967241620291584_n.png?_nc_cat=1&ccb=1-5&_nc_sid=ac3552&_nc_ohc=wNTmUISdYpQAX8GkND6&_nc_ht=scontent-mxp1-1.xx&oh=53378bb964a0f43a6e01ea3af1efacb2&oe=61B800F7
  @width 0.1
\fi

\iusr{Анатолий Золотушкин}
\textbf{Savelyeva Olena} спасибо @igg{fbicon.heart.red}

\iusr{Арт Юрковская}

Ещё его упрекнули, почему он на русском языке написал, а не на державний. Ведь
украинский хлеб ел и вообще был украинцем. А то, что ездил работать в Россию -
"чемодан, вокзал, Россия". Но он чемодан, вокзал, Рим" решил по безвизу. И там
написал: "Ну, шо сынку, помогли тебе твои ляхи?" И тут почалось...

\begin{itemize} % {
\iusr{Юрий Прокопец}
\textbf{Арт Юрковская} он ещё и Яновским был , т.е. поляком . Да .

\iusr{Людмила Лысюк}
\textbf{Art Yurkovska} 

перед тем как делать подобные "умозаключения", неплохо бы получить историю и
разобраться, а не демонстрировать свое невежество. А так выходит по принципу:
оплюю кого-нибудь "великого" - так и сам поднимусь в своих глазах."

\begin{itemize} % {
\iusr{Арт Юрковская}
\textbf{Людмила Лысюк} Я ни на кого не плевала.Начинается?

\iusr{Арт Юрковская}
\textbf{Людмила Лысюк} 

А юмор и иронию отменили? В ФБ люди серьезные, сгрызут и не то что Гоголя. Знаю
по себе. За публикацию рецепта борща из мемуаров Вертинского, чуть не убили. И
отбили, но не почки, конечно, а охоту писать интересное.

\iusr{Аркадій Друкер}
\textbf{Art Yurkovska} Не этот ли рецепт? 5 л НЕГАЗИРОВАННОЙ воды...

\ifcmt
  ig https://scontent-mxp1-1.xx.fbcdn.net/v/t39.30808-6/243012674_10219418056188512_2057625143861819247_n.jpg?_nc_cat=101&ccb=1-5&_nc_sid=dbeb18&_nc_ohc=7kX6dXlPQysAX9t8juB&_nc_ht=scontent-mxp1-1.xx&oh=00bd17a6fd9489d575e6dce410e32cb0&oe=61B97D15
  @width 0.4
\fi

\iusr{Арт Юрковская}
\textbf{Аркадій Друкер} 

Нет. Там на свекольном квасе с балабушками из щуки и грибными ушками.)))

\end{itemize} % }


\iusr{Oleksa Vovk}
\textbf{Art Yurkovska} В те мрачные времена державным был расейский язык.

\begin{itemize} % {
\iusr{Владимир Шмуклер}
\textbf{Oleksa Vovk} для Гоголя в Риме - мрачные времена? За державный счет? Это Тараса Григоровича за словесную, да еще и на недержавном языке, невоздержанность к царским особам отправили на курорт Мангышлак. А так он тоже не бедствовал в Италии.
\end{itemize} % }

\end{itemize} % }

\iusr{Сергей Кравцов}

А вообще еврем очень не любят Гоголя. За его их описание, в том же \enquote{Тарасе
Бульбе}. И вполне понятно, почему из Хайфы такие о нем тексты.

\begin{itemize} % {
\iusr{Анатолий Золотушкин}
\textbf{Сергей Кравцов} ошибаешься, любезный. Люблю Гоголя и текст написан с любовью

\iusr{Сергей Кравцов}
С любовью... большой любовьюю, к ненавистникам Гоголя.

\iusr{Yan Nudelman}
\textbf{Сергей Кравцов} Этот текст не для всех. Вы в этом списке.

\iusr{Сергей Кравцов}
\textbf{Yan Nudelman} ок, помещаю и вас в список

\iusr{Тетяна Леонова}
\textbf{Сергей Кравцов} ,расслабьтесь. Вьі еще Резника покритикуйте,у вас получиться... Люди ульібнуться по-доброму не прочь с утра, но всегда начеку и бдят правдолюбители с дегтем...

\iusr{Tatiana Thoene}
\textbf{Сергей Кравцов} да бросьте! Он употреблял слово «жид», которое в те времена было обычным, а не ругательством. И персонажа Янкеля он описал с симпатией.

\iusr{Сергей Кравцов}
Гм, если я сейчас с такой же симпатией опишу Золотушкина, то меня тут же обвинят. Увы, у Гоголя было больше свободы.
\end{itemize} % }

\iusr{Жанна Павильч}
 @igg{fbicon.thumb.up.yellow} салатик

\iusr{Hennadii Tsurkov}
Неплохо и в духе Хармса)

\iusr{Сергей Кравцов}
\textbf{Hennadii Tsurkov} Ювачев Золотушкина того бы в Пакины бы определил. Ну или в Ракукины.

\iusr{Татьяна Сирота}

\ifcmt
  ig https://i2.paste.pics/b1a3ed54e082ad2607e7c1222a33437c.png
  @width 0.3
\fi

\iusr{Юрий Панчук}

А мне Нежин понравился, не знаю как насчет жить там, но учиться Гоголю было
наверно комфортно.

\begin{itemize} % {
\iusr{Kateryna Masliechkina}
\textbf{Юрий Панчук} Как раз не очень было комфортно. Он скучал очень в Нежине и не слишком хорошо отзывался о качестве обучения. Мечтал о Петербурге. Но Нежин действительно сейчас красивый старинный и уютный городок.
\end{itemize} % }

\iusr{Сергей Кравцов}

Кстати, Ирка Фарион о Гоголе такое же навалять могла, но на мове. Она баба
старой коммунистической закалки и Гоголя ненавидеть умеет правильным образом.


\iusr{Петр Закалюжний}
за Нежин обидно...

\begin{itemize} % {
\iusr{Анатолий Золотушкин}
\textbf{Петр Закалюжний} извините, хотел исправить, но не получилось
\end{itemize} % }

\emph{Евгения Ерёменко}

Простите, уважаемые одногруппники, не кажется ли нам, что подобный тон
преподнесения материала нас не возвышает?


\end{itemize} % }
