% vim: keymap=russian-jcukenwin
%%beginhead 
 
%%file 24_09_2022.stz.news.ua.donbas24.1.mrpl_ce_ukr_akcia
%%parent 24_09_2022
 
%%url https://donbas24.news/news/mariupol-ce-ukrayina-yak-v-mistax-ukrayini-proxodila-nacionalna-akciya-foto
 
%%author_id demidko_olga.mariupol,news.ua.donbas24
%%date 
 
%%tags 
%%title "Маріуполь — це Україна": як в містах України проходила національна акція
 
%%endhead 
 
\subsection{\enquote{Маріуполь — це Україна}: як в містах України проходила національна акція}
\label{sec:24_09_2022.stz.news.ua.donbas24.1.mrpl_ce_ukr_akcia}
 
\Purl{https://donbas24.news/news/mariupol-ce-ukrayina-yak-v-mistax-ukrayini-proxodila-nacionalna-akciya-foto}
\ifcmt
 author_begin
   author_id demidko_olga.mariupol,news.ua.donbas24
 author_end
\fi

%\ii{24_09_2022.stz.news.ua.donbas24.1.mrpl_ce_ukr_akcia.txt}
\ii{24_09_2022.stz.news.ua.donbas24.1.mrpl_ce_ukr_akcia.pic.front}

\begin{center}
  \em\color{blue}\bfseries\Large
Загальнонаціональна акція відбулася за ініціативою маріупольської спільноти у
тих містах, де створені центри \enquote{Я — Маріуполь} 
\end{center}

24 вересня у Києві, Дніпрі, Львові, Вінниці, Черкасах, Одесі,
Івано-Франківську, Тернополі та Калуші пройшла всеукраїнська акція \enquote{Маріуполь —
це Україна}. Зважаючи на заходи безпеки, не змогли провести акції лише в
Кривому Розі та Запоріжжі. У кожному місті маріупольці зібралися біля центрів
\enquote{ЯМаріуполь} з синьо-жовтими прапорами, щоб виступити проти псевдореферендумів
та розповісти свої історії. Маріупольська спільнота обрала саме цю дату для
проведення акції-мітингу, адже щорічно в останню суботу вересня у Маріуполі
відбувалося святкування Дня міста. Цей день об'єднав всіх маріупольців, нехай і
в різних містах України. Вони вкотре одноголосно заявили про те, що Маріуполь
був, є та буде українським.

\ii{24_09_2022.stz.news.ua.donbas24.1.mrpl_ce_ukr_akcia.pic.1}
\ii{24_09_2022.stz.news.ua.donbas24.1.mrpl_ce_ukr_akcia.pic.2}

\ii{insert.read_also.shvecova.pochujte_golos_ludej_evak}
\ii{24_09_2022.stz.news.ua.donbas24.1.mrpl_ce_ukr_akcia.pic.3}

У Києві акція розпочалась виступом мешканки Маріуполя, депутатки міської ради,
підприємиці \href{https://www.facebook.com/olga.pikula.7}{Ольги Пікули}.%
\footnote{\url{https://www.facebook.com/olga.pikula.7}}

\begin{leftbar}
\emph{\enquote{Цього року ми не відкриваємо нові набережні, нові дитячі майданчики чи
лікарні. Нас позбавили цієї можливості. Всім нам було дуже важко
вибиратися з міста. Але ми вижили і дуже вдячні нашим захисникам, які,
впевнена, незабаром звільнять Маріуполь і ми всі зможемо повернутися та
з новими силами почати його відбудовувати та відновлювати}}, —
підкреслила Ольга.
\end{leftbar}

\ii{24_09_2022.stz.news.ua.donbas24.1.mrpl_ce_ukr_akcia.pic.4}
\ii{24_09_2022.stz.news.ua.donbas24.1.mrpl_ce_ukr_akcia.pic.5}

\textbf{Читайте також:} \emph{Зростання руху опору в Маріуполі — як молодь протистоїть окупантам}%
\footnote{Зростання руху опору в Маріуполі — як молодь протистоїть окупантам, Ольга Демідко, donbas24.news, 19.09.2022, \par%
\url{https://donbas24.news/news/zrostannya-ruzu-oporu-v-mariupoli-yak-molod-protistoyit-okupantam}%
}

Переважна більшість маріупольців, а це понад 200 тисяч людей, наразі
перебувають на підконтрольній території України. Всі вони вже зробили свій
вибір. Кожен маріуполець має свою історію виживання у блокадному місті.
Російські окупанти знищили їхні будинки та принесли руйнацію в місто, але
виїхавши з Маріуполя, люди не втратили віри, що в місто можливо повернутися.
Головне, щоб його звільнили! 

\begin{leftbar}
\emph{\enquote{Я готовий зробити все можливе для відновлення найбільшого Палацу культури
на Східній Україні. \enquote{Український дім} ще відкриє свої двері і почне
нове життя. Головне повернути Маріуполь до України, а відбудувати і
відновити все зможемо! Для цього є і сили, і бажання}}, — наголосив
директор МПК \enquote{Український дім} \href{https://www.facebook.com/pavlukigor}{Ігор Павлюк}.%
%\footnote{\url{https://www.facebook.com/pavlukigor}}
\end{leftbar}

\begin{leftbar}
\emph{\enquote{Перша лікарня Маріуполя теж відновить свою роботу з поверненням Маріуполя
до України. Всі лікарі нашої лікарні одразу ж повернуться до міста і
розпочнуть роботу}}, — зазначила директорка Міської лікарні Маріуполя
№ 1 \href{https://www.facebook.com/profile.php?id=100014146715674}{Лариса Мамаєва}.%
%\footnote{\url{https://www.facebook.com/profile.php?id=100014146715674}}
\end{leftbar}

\ii{24_09_2022.stz.news.ua.donbas24.1.mrpl_ce_ukr_akcia.pic.6}
\ii{24_09_2022.stz.news.ua.donbas24.1.mrpl_ce_ukr_akcia.pic.7}
\ii{24_09_2022.stz.news.ua.donbas24.1.mrpl_ce_ukr_akcia.pic.8}
\ii{24_09_2022.stz.news.ua.donbas24.1.mrpl_ce_ukr_akcia.pic.9}
\ii{24_09_2022.stz.news.ua.donbas24.1.mrpl_ce_ukr_akcia.pic.10}

\ii{insert.read_also.shvecova.alyona_alyona_ukrainer_klip_mrpl}

В окупованому місті й досі залишається багато маріупольців, які не визнають
фейкових референдумів. Всі вони чекають на ЗСУ та вірять в Україну, проте не
можуть про це вільно говорити через небезпеку.

\begin{leftbar}
\emph{\enquote{Я продовжую спілкуватися з людьми, які лишилися в Маріуполі. Вони чекають
деокупацію, але з різних причин не змогли виїхати. Деякі лишилися через
власний стан здоров'я, деякі - через стан здоров'я батьків. А цей
референдум — це справжній фарс, адже він проходить під дулами
автоматів}}, — розповів маріуполець, який нараз перебуває в Києві,
\href{https://www.facebook.com/profile.php?id=100000675250442}{Владислав Колесніков}.
\end{leftbar}

\ii{24_09_2022.stz.news.ua.donbas24.1.mrpl_ce_ukr_akcia.pic.11}
\ii{24_09_2022.stz.news.ua.donbas24.1.mrpl_ce_ukr_akcia.pic.12}
\ii{24_09_2022.stz.news.ua.donbas24.1.mrpl_ce_ukr_akcia.pic.13}
\ii{24_09_2022.stz.news.ua.donbas24.1.mrpl_ce_ukr_akcia.pic.14}
\ii{24_09_2022.stz.news.ua.donbas24.1.mrpl_ce_ukr_akcia.pic.15}

На мітинги-акції в різних містах України прийшло дуже багато дітей, молоді і
людей похилого віку, які читали вірші, ділилися власними переживаннями та
історіями. Всі вони зараз перебувають в різних містах, але їх об'єднує сильне
бажання повернутися до українського Маріуполя та віра, що це станеться
найближчим часом.

\begin{leftbar}
\emph{\enquote{Я народився і виріс у Маріуполі і не уявляю свого життя без рідного міста.
Хочу жити в Україні. Вся моя сім'я повернеться до Маріуполя, коли його
звільнять, адже не уявляємо свого життя без нього}}, — поділився думками
9-ти річний маріуполець Денис Креньов.
\end{leftbar}

\ii{24_09_2022.stz.news.ua.donbas24.1.mrpl_ce_ukr_akcia.pic.16}
\ii{24_09_2022.stz.news.ua.donbas24.1.mrpl_ce_ukr_akcia.pic.17}
\ii{24_09_2022.stz.news.ua.donbas24.1.mrpl_ce_ukr_akcia.pic.18}
\ii{24_09_2022.stz.news.ua.donbas24.1.mrpl_ce_ukr_akcia.pic.19}
\ii{24_09_2022.stz.news.ua.donbas24.1.mrpl_ce_ukr_akcia.pic.20}

Нагадаємо раніше Донбас24 розповідав про \href{https://donbas24.news/news/tvoya-sila-pragnuti-peremogi-trivaje-onlain-vistavka-ukrayinskoyi-molodi}{онлайн-виставку української молоді}.%
\footnote{Твоя сила — прагнути перемоги: триває онлайн-виставка української молоді, Алевтина Швецова, donbas24.news, 24.09.2022, \par%
\url{https://donbas24.news/news/tvoya-sila-pragnuti-peremogi-trivaje-onlain-vistavka-ukrayinskoyi-molodi}%
}

Ще більше новин та найактуальніша інформація про Донецьку та Луганську області
в нашому телеграм-каналі Донбас24.

ФОТО: з відкритих джерел

\ii{insert.author.demidko_olga}
