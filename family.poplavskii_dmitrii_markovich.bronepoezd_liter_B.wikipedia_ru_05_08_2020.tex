% vim: keymap=russian-jcukenwin
%%beginhead 
 
%%file poplavskii_dmitrii_markovich.bronepoezd_liter_B.wikipedia_ru_05_08_2020
%%parent poplavskii_dmitrii_markovich
 
%%endhead 
\subsection{Бронепоезд Литер Б}
\url{https://ru.wikipedia.org/wiki/%D0%91%D1%80%D0%BE%D0%BD%D0%B5%D0%BF%D0%BE%D0%B5%D0%B7%D0%B4_%C2%AB%D0%9B%D0%B8%D1%82%D0%B5%D1%80_%D0%91%C2%BB}
  
Бронепоезд «Литер Б» — второй бронепоезд народного ополчения, построенный в
Киеве и принимавший участие в обороне города в ходе Великой Отечественной
войны.  

\subsubsection{Технические Данные}

% align: l,c,r 
\begin{longtable}{|l|l|}
				\multicolumn{2}{c}{\textbf{Бронепоезд «Литер Б»}} \\
Принадлежность      & СССР                                    \\
Подчинение          & соединениям Юго-Западного фронта        \\
Эксплуатация        & 20 июля — 22 сентября 1941 года         \\
Изготовитель        & Киевский паровозо-вагоноремонтный завод \\
Участие в           & Киевская оборонительная операция        \\
Известные командиры & Василевский, Леонид Владиславович       \\
				\multicolumn{2}{|c|}{\textbf{Технические данные}}                  \\
Силовая установка       & Бронепаровоз типа Ов          \\
Мощность                & 600 лошадинных сил            \\
Бронирование            & незакаленая сталь             \\
Количество броневагонов & 2 артиллерийско-пулемётных    \\
Экипаж                  & 77 человек\cite{Kajnaran2012} \\
\multicolumn{2}{c}{Вооружение} \\
Лёгкое вооружение         & 28 пулемётов \cite{Kolomiec2010}\\
Артиллерийское вооружение & 
2 76-мм пушки образца 1914/15 годов (Лендера) \cite{Kajnaran2012,Kolomiec2010,Kolomiec2010a}\\
\end{longtable}

\subsubsection{Строительство, формирование экипажа}

23 июня 1941 года рабочие Киевского паровозо-вагоноремонтного завода им.
Январского восстания (КПВРЗ) на митинге приняли решение о строительстве
бронепоезда. Разрешение на постройку было дано 4 июля Лазарем Кагановичем,
тогда главой наркомата путей сообщения\cite{Kolomiec100}. После этого начались работы по
строительству и оборудованию. 20 июля бронепоезд вошёл в строй\cite{Kajnaran2012,Kolomiec100}.

Экипаж набирался из добровольцев-железнодорожников, многие из них окзалаись
мобилизованы по линии НКПС. То есть часть экипажа де-юре оставалась
гражданскими лицами. Соответственно потери среди таких бойцов не отображались в
армейских документах о безвозвратных потерях. К таким относится Василевский Л.
В., бывший на должности командира бронепоезда и погибший в 1942 году. В целом в
обслугу бронепоезда вошло 77 человек\cite{Kajnaran2012}. 

\subsubsection{Служба}

21 июля бронепоезд присоединился к другому киевскому бронепоезду «Литер А» на
ж/д ветке Киев — Фастов для оказания поддержки частям 26-й армии, оборонявшим
район Фастова\cite{Kajnaran2012}.

23 июля оба бронепоезда поддерживали атаку 73-го погранотряда у станции
Клавдиево. На следующий день пограничники и десантная группа с бронепоездов
перешли в атаку при артиллерийской поддержке «Литер А» и «Литер Б». Врага
удалось отогнать, были захвачены штабные документы. В ночь с 3 на 4 августа
вместе с бронепоездом «Литер А» совершил удачный налёт на станцию
Боярка\cite{Kajnaran2012}.

Во время первого генерального штурма Киевского укрепрайона, который начали
войска 29-го армейского корпуса вермахта тром 4 августа 1941
года\cite{KajnaranMuravovJuschenko2017}, «Литер Б» переходит на участок Жуляны
— Тарасовка\cite{Kajnaran2012}. Немецкие источники также подтверждают
активность киевских бронепоездов. 71-я пехотная дивизия, наступавшая восточнее
села Жуляны, попадала под огневые налёты бронепоезда и принуждалась
останавливать свои атаки\cite{Haupt2007}. Но ситуация у южных окраин Киева
оставалась напряжённая, и бронепоезд действует до 15 августа совместно с
другими бронепоездами у Пирогово — Мышеловка — Лысая гора в целях поддержки
пехоты 147-й и 284-й стрелковых дивизий, 2-й воздушно-десантной
бригады\cite{KajnaranMuravovJuschenko2017}.

С 16 августа «Литер Б» выполняет боевые задания на ж/д линии Киев — Коростень,
взаимедействуя с 27-м стрелковым корпусом, а также с 20-м погранотрядом,
который в те дни оборонял рубежи Киевского укрепрайона у ж/д моста через реку
Ирпень. А 23—24 августа — снова близ села Пирогово. Вечером 23 августа немцы
захватили плацдарм на левом берегу Днепра напротив местачка Горностайполь.
Поэтому киевские бронепоезда получают задание действовать на ветке Киев -
Кобыжча и не давать противнику расширять захваченный участок. Вскоре во времы
боя близ сёл Бобровица — Заворичи «Литер Б» получил серьёзные повреждения и
ушёл в Киев на ремонт\cite{Kajnaran2012}.

К 14 сентября 1941 года танковые клинья немцев образовали киевский котёл. Более
того пехотные части противника рассекали район окружения на отдельные очаги. В
этой ситуации 29-й армейский корпус вермахта начал 16 сентября 1941 года второй
штурм КиУР\cite{KajnaranMuravovJuschenko2017}. Бронепоезд «Литер Б» действует с
17 сентября близ станции Пост-Волынский, погамая отразить вражеский
штурм\cite{Kajnaran2012}.

18 сентября войска 37-й армии по приказу командования начинают отход из
Киева\cite{KajnaranMuravovJuschenko2017}. Бронепоезда, оставшиеся в Киеве,
включая и бронепоезд «Литер Б», проследовав 20 сентября через Борисполь, пошли
на прорыв на восток вместе с основными силами 37-й армии\cite{Kajnaran2012}.
Ожесточённые бои шли за Барышевку, мост через реку Трубеж, Березань. 22
сентября повреждённые бронепоезда «Литер А» и «Литер Б» были захвачены
противником восточнее ж/д станции Березань на 84-м километре ж/д от
Киева\cite{Kajnaran2012}.

Экипажи, собрав личное оружие, организованно отошли на юго-запад в болота в
пойме реки Трубеж. Там в болотах, по воспоминаниям начальника штаба бронепоезда
«Литер А» Арефьева К. А., группа соединилась со сводным отрядом под
руководством командующего 37-й армии Власова А. А. 2 октября под командой
Власова киевские железнодорожники двинулись на прорыв на
восток\cite{Kajnaran2012}. 
