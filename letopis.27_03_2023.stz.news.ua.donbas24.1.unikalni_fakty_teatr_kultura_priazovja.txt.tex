% vim: keymap=russian-jcukenwin
%%beginhead 
 
%%file 27_03_2023.stz.news.ua.donbas24.1.unikalni_fakty_teatr_kultura_priazovja.txt
%%parent 27_03_2023.stz.news.ua.donbas24.1.unikalni_fakty_teatr_kultura_priazovja
 
%%url 
 
%%author_id 
%%date 
 
%%tags 
%%title 
 
%%endhead 

Унікальні факти про театральну культуру Приазов'я (ФОТО)

27 березня щорічно відзначається Міжнародний день театру

У 1961 році IX конгресом Міжнародного інституту театру (МІТ) був встановлений
Міжнародний день театру. Діяльність організації, згідно з її уставом, повинна
бути спрямована на «укріплення миру і дружби між народами, на розширення
творчого співробітництва всіх діячів світу». Проте це не просто професійне
свято митців сцени, сьогодні це свято і мільйонів глядачів. На жаль, через
повномасштабне вторгнення рф в Україну багато театральних колективів
призупинили свою діяльність чи були вимушені продовжити працювати в інших
містах. Проте їхня історія продовжується на підконтрольній Україні території чи
закордоном. З нагоди свята важливо згадати унікальні факти театральної культури
Маріуполя і загалом Приазов'я, театральна історія якого є унікальною і дуже
багатою.

Читайте також: «Від папірусу до гаджету» — новий проєкт Марії та Олександра
Сладкових

1. У Маріуполі наприкінці XIX століття була відкрита найпопулярніша театральна
сцена у губернії

Датою народження головного театру Приазов'я слід вважати 1878 рік, коли відомий
антрепренер В. Шаповалов побудував Зимовий театр — перший стаціонарний театр у
місті, ним же була зібрана власна трупа, він же став антрепренером і режисером
Маріупольського театру. На цій сцені грали спектаклі найбільших антреприз
України — М. Кропивницького, І. Карпенка-Карого, П. Саксаганського, М.
Старицького. Тут блищали обдарування великих Г. Федотової та В. Мейєрхольда.
Завдяки центральній та місцевій періодиці відомо, що до Зимового театру в
Маріуполі їхали з усієї Катеринославської губернії. Наприкінці XIX століття він
став найпопулярнішою театральною сценою в усій Катеринославській губернії.

Читайте також: «Театроманія» продовжує розвивати та підтримувати культуру Маріуполя

2. У трупі Марка Кропивницького виступали актори з Маріуполя

Засновнику першої професійної трупи міста, Василю Шаповалову належить заслуга
першовідкривача таланту українських акторів Любові Ліницької та Івана
Загорського, які були запрошені у трупу визнаного корифея українського театру
Марка Кропивницького. Любов Павлівна Ліницька грала в російських та українських
виставах, але перевагу надавала останнім. Найбільший успіх вона мала у п'єсах:
«Наталка Полтавка», «Назар Стодоля», «Невільник», «Шельменко-денщик», «Сватання
на Гончарівці». Бібліограф і літературознавець Ю. Меженко зазначав, що у трупі
Шаповалова Ліницька була на перших ролях і мала бенефіс. У неї навіть був
власний сценічний гардероб.

Читайте також: Чи любив Марко Кропивницький Приазов’я: до дня створення українського реалістичного театру

3. Особливою популярністю у Приазов'ї користувалися вистави Театру корифеїв

Особливе місце в театральному житті Приазов'я наприкінці XIX ст.  належало
українському музично-драматичному театру. Гастролі великих майстрів української
сцени Марка Кропивницького, Івана Карпенка-Карого, Панаса Саксаганського, Марії
Садовської-Барілотті дали можливість містянам ознайомитися з національною
театральною драматургією. Найбільшу популярність мали вистави «Наталка
Полтавка», «Запорожець за Дунаєм», «Сватання на Гончарівці», «Доки сонце зійде
— роса очі виїсть», «Наймичка». На сцені концертної зали Маріуполя грали А.
Рютчі, Я. Козельський, Л. Манько та ін.

У Музеї театрального, музичного та кіномистецтва України зберігається годинник
із лаконічним написом: «Кропивницькому від маріупольців 29.09.1889 р.». Цей
подарунок є матеріальним свідченням того, що дійсно в 1889 році М.
Кропивницький виступав у Маріуполі. Влітку 1908 року антрепренер М. Кононенко
запросив М. Кропивницького приїхати в Маріуполь. У міській газеті
«Маріупольське життя» повідомлялося, що М. Кропивницький приїздить на гастролі
і що він виступить у ролі Виборного («Наталка Полтавка»), Карася («Запорожець
за Дунаєм»), а у виставі «Доки сонце зійде — роса очі виїсть» зіграє одразу дві
ролі — колишнього селянина-кріпака Максима Хвортуни й поміщика Воронова.
Вистави мали проходити у приміщенні театру братів Яковенків і всі квитки були
негайно розпродані. Неперевершений майстер сцени цілковито виправдав очікування
публіки, яка постійно щиро винагороджувала гру Марка Кропивницького бурхливими
оплесками, що переходили в овації. Вистави відбувалися при переповнених залах.

Читайте також: Які театральні проєкти та культурні заходи, присвячені
Маріуполю, реалізуються закордонними митцями

4. У регіоні працювала спеціальна театральна комісія, яка сприяла приїзду
видатних іноземних гастролерів

У 1904 році була створена спеціальна театральна комісія, яка укладала угоди з
театральними колективами, — приміщення театру здавали їм безкоштовно, при
безкоштовному опаленні та освітленні. Це притягувало в Приазов'я чимало цікавих
театральних труп, мешканці міста могли ознайомитися з різноманітними
театральними об'єднаннями, в яких брали участь талановиті українські, російські
та іноземні актори. Несподіванкою для маріупольців стали гастролі алжирської
актриси Галіми у квітні 1912 року. Вона виконала кілька вокальних партій,
причому своїм унікальним голосом імітувала гру на флейті-окарині. Виконавиця
дуже майстерно представила бездоганну імітацію інструментальної музики.

У грудні 1912 року маріупольці в театрі Уварова щиро вітали трупу
імператорського театру з Токіо, який привіз виставу «Прислуга Отаке». Особливо
вразила своєю грою японська акторка пані Ганаке, яка володіла неймовірною
пластикою та відтворювала спів птахів. Гастрольні тури все частіше тішили
шанувальників Мельпомени в Приазов'ї.

Наприкінці 1912 року італійська трупа Гонсалеца у театрі Уварова познайомила
маріупольців з класичним оперним репертуаром. Важливою подією у культурному
житті регіону став гастрольний візит 24 жовтня 1918 року видатних акторів
німого кіно Віри Холодної та Осипа Рунича. Слід наголосити, що виступи
гастролерів — українських та іноземних зірок сцени — сприяли популяризації та
збагаченню театральної культури, перетворенню Маріуполя на культурний центр
регіону.

5. У Приазов'ї почав працювати перший Державний Грецький театр

Протягом 1932−1937 років у Маріуполі працював Державний грецький театр.
Ініціатором створення грецького театру був румейський радянський письменник,
поет і драматург, засновник національної літератури греків України — Георгій
Антонович Костоправ. Біля витоків театрального мистецтва греків Приазов'я
стояли два головних режисери — Данило Теленчі та Юрій Дранга. Трупа театру
майже цілком складалася з учорашніх аматорів, тільки три актори — Данило
Теленчі, Юрій Дранга та Георгій Дегларі — виявилися досвідченими
професіоналами. Але це не применшувало значення і якості гри акторів.

Читайте також: Як в Маріуполі проходив фестиваль «Театральна брама» — деталі

Всюди вистави грецького театру проходили з великим успіхом. Цей успіх був
продиктований ще й тим, що грецькі працюючі маси вперше побачили вистави рідною
мовою. Великий успіх у глядачів мали актори С. Янгічер, М. Чіча, які дивували
виразним читанням та грецькими народними танцями; Ф. Кашкер вражав румунським
народним танцем, а Л. Карнаухова — сольним співом. Багато акторів блискуче
володіли музичними інструментами. Особливий успіх мала Людмила Карнаухова —
вона була єдиною українкою в грецькій трупі. Але завдяки високим професійним
якостям та енергійності вона стала примою державного грецького театру,
улюбленицею публіки.  6. Актриса Маріуполя була нагороджена вищою відзнакою
Міжнародного комітету Червоного Хреста — медаллю «Флоренс Найтінгейл»

Одна з найбільш яскравих акторок маріупольського театру Людмила Антонівна
Радіонова стала безпосередньою учасницею воєнних дій. Її бойовий шлях
відзначений орденом Леніна, двома орденами Червоної Зірки, сімома бойовими
медалями, медаллю Міжнародного Червоного Хреста. Працювала вона в
Маріупольському театрі з 20-х років. У статистах ходити їй не довелось; у
Маріупольському театрі її кар'єра ропочалась блискуче — відразу з головних
ролей. З початком війни Л. Радіонова стала санінструктором 75-стрілецького
полку. У селі Миколаївці під Таганрогом вона винесла з поля бою і врятувала від
неминучої загибелі сорок п'ять поранених бійців і командирів. Була нагороджена
вищою відзнакою Міжнародного комітету Червоного Хреста — медаллю «Флоренс
Найтінгейл». Ця нагорода носить ім'я однієї із засновниць Міжнародного
Червоного Хреста і вручається медсестрам і санітаркам, які «відзначилися
виключною самовідданістю при догляді за пораненими і хворими під час війни або
громадського лиха». У 27 років вона стала інвалідом. Театр довелося полишити.
Та Людмила Антонівна Радіонова стала невід'ємною частиною історії театрального
мистецтва Приазов'я.

Читайте також: Надихаючі історії про непересічних жінок Приазов'я

7. Деякі театральні діячі Маріуполя вчилися у Леся Курбаса

Серед маріупольських акторів були досить кваліфіковані особистості з великим
досвідом роботи у театральних установах. Наприклад, Андрій Ірій (Авраменко),
який очолював Маріупольський музично-драматичний театр ім. Т. Шевченка у
1942−1943 роках, пройшов велику школу акторської та режисерської роботи,
зокрема в колективі, очолюваному Лесем Курбасом. Водночас він вступив до ОУН та
організував товариство «Просвіта», прищеплював любов до української мови,
організував вечори пам‘яті Тарасу Шевченку, оформив приміщення театру
жовто-блакитними прапорами та тризубом. Художній керівник і актор театру В.
Волгрик до війни працював у Чернігівському театрі, здобув звання заслуженого
артиста республіки. Репертуар Маріупольського музично-драматичного театру ім.
Т. Шевченка складався переважно з народно-побутових вистав, що виконувалися
здебільшого українською мовою. Всього за час окупації відбулося 18 вистав.
Найбільш популярні твори, що виконувалися на сцені театру: «Синочок», «Наталка
Полтавка», «Сватання на Гончарівці», «Пошились у дурні», «Наймичка», «Маруся
Богуславка», «Ревізор», «Рожеве павутиння», «Безприданниця», «Вій» «Біда від
ніжного серця», «Закрита вистава», «Підступність і кохання».

Читайте також: Драматичний театр Маріуполя виступив у Польщі — актори взяли
участь у міжнародному фестивалі 8. У регіоні працювали унікальні самодіяльні
театральні колективи 

8. У регіоні працювали унікальні самодіяльні театральні колективи 

У 1949 році у Приазов'ї припинили діяльність абсолютно всі професійні
колективи. Причин було декілька: і брак коштів, і відсутність приміщення. Не
останню роль відіграла горезвісна постанова ЦК КП (б)У від 12 жовтня 1946 року
«Про репертуар драматичних і оперних театрів УРСР. Заходи щодо його
поліпшення», у якій йшлося й про те, що «у „старій українській спадщині“ є
чимало п'єс, які відображають життя неправильно і фальшиво, прикрашають
минуле…, проповідують застарілі ідеї та відсталі погляди…». Проте відсутність
професійних театрів в регіоні заповнювали колективи художньої самодіяльності.
50-ті роки стали часом розквіту самодіяльних колективів. Вистави йшли на сценах
клубів і палаців культури «Азовсталь», «Будівельник», ім. Карла Маркса,
Коксохім-заводу, клубу АМП. У місті працювали як українські, так і російські
самодіяльні гуртки. Соціальний склад аматорських драматичних колективів був
досить різноманітним. Тут були представники інтелігенції, робітників,
інженерів, службовців, домогосподарки. Однак всі вони були єдині в любові до
театрального мистецтва.

Читайте також: У Києві відновили маріупольську виставу (ФОТО)

9. Будівництво театру в Маріуполі — виключення з правил

Згідно з чинними на 1950-ті роки правилами новий театр могли побудувати лише в
обласному центрі. Виходячи з цього положення, перспектив створення власного
професійного театрального колективу у Маріуполя не було. Однак завдяки великій
кількості клопотань самих маріупольців, інколи навіть у найвищі інстанції, цю
директиву вдалося обійти. В усіх клопотаннях йшлося про створення обласного
драматичного театру, а вже потім зазначалося, що з різних об'єктивних причин
приміщення для нової установи культури доцільно звести у прилеглому Маріуполі.
Саме з тих часів на афішах театру з'явилася назва: «Донецький російський
драматичний театр», а в дужках — «місто Маріуполь». Відкриття театру у 1960
році стало справжнім святом для жителів міста та було обумовлене
загальноісторичним і культурним підґрунтям Приазов'я. Репертуар трупи складався
з п'єс визначних класиків драматургії.

Читайте також: Драмтеатр Маріуполя — символ пам'яті, болю і туги за загиблими
маріупольцями (ФОТО)

10. Один з найкращих драмтеатрів України

Маріупольський театр впродовж 1970−1980-х років отримав неофіційний статус
одного з найкращих драмтеатрів України, чому сприяли гастрольні маршрути
колективу. Це був період підкорення елітних столичних вершин: Київ, Баку, Рига,
Таллінн, Вільнюс, Мінськ. І майже всіх обласних центрів України. Успіх театру
був переконливим і вражаючим — повні зали і захоплені відгуки театральної
критики. До того ж, у Маріупольському театрі 10 років працював народний артист
України Станіслав Боклан (з 1984 року). За роки роботи проявив себе актором
широкого діапазону з неосяжними творчими здібностями. Глядачі і фахівці
відзначали його яскраві сценічні дії, своєрідну пластику, емоційність, вірне
відчуття жанру. Особливо яскравою вийшла робота актора над роллю Венсана у
музичній шоу-комедії «Чао», яку художня рада Маріупольського театру визнала
найкращою чоловічою роллю театрального сезону. У драмтеатрі до його знищення
зберігалися і костюми, в яких виходив на сцену Стас Боклан на початку 90-х
років.


Раніше Донбас24 повідомляв, що маріупольські актори зіграли благодійну виставу
у Києві.

Ще більше новин та найактуальніша інформація про Донецьку та Луганську області
в нашому телеграм-каналі Донбас24.

Фото: з відкритих джерел
