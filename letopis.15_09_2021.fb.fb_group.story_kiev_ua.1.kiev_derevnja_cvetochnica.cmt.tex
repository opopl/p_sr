% vim: keymap=russian-jcukenwin
%%beginhead 
 
%%file 15_09_2021.fb.fb_group.story_kiev_ua.1.kiev_derevnja_cvetochnica.cmt
%%parent 15_09_2021.fb.fb_group.story_kiev_ua.1.kiev_derevnja_cvetochnica
 
%%url 
 
%%author_id 
%%date 
 
%%tags 
%%title 
 
%%endhead 
\subsubsection{Коментарі}

\begin{itemize} % {
\iusr{Тома Храповицкая}

Спасибо Вам Фарид, Вы чувствуете! Всё прекрасное притягивается и проходит
сквозь Вас, при этом оставляя частичку себя! Обязательно! Это магия,
подталкивающая и направляющая, указывающая деликатно и при этом требующая и
дающая Вдохновение и Силы для его Величества Творчества! Дерзайте, творите, Вы
чувствуете!!!

\begin{itemize} % {
\iusr{Фарид Степанов}
\textbf{Тома Храповицкая}
Спасибо. Очень вдохновляюще.

\end{itemize} % }

\iusr{Max Gopencko}
Двух "Киевских цветочниц" из коллажа знаю, одну даже в группу выкладывал. А
третья какая же?

\begin{itemize} % {
\iusr{Фарид Степанов}
\textbf{Max Gopencko}
Как понимаю, картин на эту тему было три.
1897, 1906 и 1908 годов.
Супруга Пимоненко, Александра Владимировна, вспоминала, что "Цветочница" 1906 года была приобретена киевским коллекционером И. Мацневым. И дальнейшие её следы утеряны.
\end{itemize} % }

\iusr{Марина Набока}
если память мне не изменяет , то "Киев – большая деревня" сказал Тарапунька на
КВН, он вроде судьей был и спасал укр.команду


\iusr{Люся Киевская}

« это было недавно - это было давно « (с) Теперь уже никак не назовёшь так Киев
! Даже если не считать левобережье - Киевом ( мое личное мнение), то и на
правом / историческом берегу - это огромный город ! ...

\begin{itemize} % {
\iusr{Фарид Степанов}
\textbf{Люся Киевская}
Под "большой деревней", во всяком случае я, понимаю: все друг друга знают, связаны узами знакомства или родства.
Тут нет ничего постыдного или уничижительного.
Да, Город вырос, изменились его жители. Но...
\end{itemize} % }

\iusr{Irina Kalinkovitsky}

Моя мама всю жизнь проработала врачом-психиатром в детском психоневрологическом
санатории. Он находился на Гоголевской 28. Это был парк и в глубине его
двухэтажный дом. Как-то она застала в парке экскурсию, которую проводила
Э.Рахлина. Оказалось, что это бывшая усадьба художника Пимоненко. Так ли это, и
что сейчас находится по этому адресу, не знаю.

Я часто прибегала к маме на работу. У меня с этим связано много воспоминаний.
Даже номер телефона почему-то помню: Б4 - 02 -64. Тогда вместо первой цифры
называли букву.

\begin{itemize} % {
\iusr{Лена Олейник}
Огромное спасибо.

\iusr{Анна Сидоренко}
Как всегда у вас фоточки очень крвсивые.
\end{itemize} % }

\iusr{Георгина Стягайло}
Дуже цякавий допис... з любов'ю до Києва.... Дякую...

\begin{itemize} % {
\iusr{Фарид Степанов}
\textbf{Георгина Стягайло}
Дякую.
\end{itemize} % }

\iusr{Irina Somova}
Спасибо. Сколько имён, которые на слуху. Интересно и познавательно.

\begin{itemize} % {
\iusr{Фарид Степанов}
\textbf{Irina Somova}
Спасибо, Ирина.

\iusr{Svitlana Plieshch}
\textbf{Irina Somova} удивительно, как все связано. Очень интересно. Хочется продолжения. Спасибо

\end{itemize} % }

\iusr{Анна Сидоренко}
Спасибо за рассказ.

\iusr{Татьяна Оржеховская}
Благодарю. Очень познавательно

\iusr{Светлана Ямковая}

Спасибо, Фарид. Всегда с интересом читаю Ваши посты. Альбом с работами Н.
Пимоненко храним дома, это один из любимых художников мужа. Эти работы
вдохновляли его, некоторые он неоднократно копировал и дарил друзьям.

\iusr{Фарид Степанов}
\textbf{Светлана Ямковая}
Спасибо, Светлана.

\iusr{Раиса Карчевская}
Спасибо большое. Очень интересно и познавательно и за прекрасные фотографии

\iusr{Фарид Степанов}
\textbf{Раиса Карчевская}
Спасибо.

\iusr{Валентина Луконина}
Замечательный текст и прекрасные фото. Как всегда, очень тонко и интересно.
Спасибо, Фарид!

\iusr{Фарид Степанов}
\textbf{Валентина Луконина}
Спасибо, Велентина.

\iusr{Vova Nesterenko}

\ifcmt
  ig https://scontent-frt3-1.xx.fbcdn.net/v/t1.6435-9/242093716_175785291338195_7986887375401674980_n.jpg?_nc_cat=108&_nc_rgb565=1&ccb=1-5&_nc_sid=dbeb18&_nc_ohc=cB8d1TN9SoYAX-qlolO&_nc_ht=scontent-frt3-1.xx&oh=89f905825567581d6a53d1e9617a7fea&oe=61699B59
  @width 0.3
\fi


\end{itemize} % }
