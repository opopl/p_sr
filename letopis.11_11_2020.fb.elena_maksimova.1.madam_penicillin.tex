% vim: keymap=russian-jcukenwin
%%beginhead 
 
%%file 11_11_2020.fb.elena_maksimova.1.madam_penicillin
%%parent 11_11_2020
 
%%url https://www.facebook.com/permalink.php?story_fbid=731498297723568&id=100025902134790
%%author 
%%tags 
%%title 
 
%%endhead 

\subsection{Зинаида Ермольева - «Мадам Пенициллин»}
\Purl{https://www.facebook.com/permalink.php?story_fbid=731498297723568&id=100025902134790}
\Pauthor{Павел Знаменский, Павел}

\index[names.rus]{Ермольева, Зинаида!Мадам Пенициллин}
\index{Лекарства!Пенициллин}
\index{Война!Великая Отечественная}
\index{Война!Вторая Мировая}

\ifcmt
img_begin 
	url https://scontent-waw1-1.xx.fbcdn.net/v/t1.0-9/125195366_731498274390237_8976190340506373361_n.jpg?_nc_cat=110&ccb=2&_nc_sid=8bfeb9&_nc_ohc=Yy9yeqdHO-YAX_UtjMI&_nc_ht=scontent-waw1-1.xx&oh=f06cd5588f28edddd92a26e211956aec&oe=5FD284AD
	caption Мадам Пенициллин - Зинаида Ермольева
	width 0.7
img_end
\fi

Наверняка многие слышали о героине романа Вениамина Каверина «Открытая книга»
докторе Татьяне Власенковой. Но мало кто знает, что прообразом её была
выдающаяся женщина, чьи достижения в развитии медицины спасли жизни миллионам
людей - Зинаида Ермольева.

\subsubsection{Панацея Второй мировой}

1943 год – разгар Великой Отечественной войны. Каждый день на всех фронтах от
ранений погибали тысячи советских офицеров и солдат. Самым страшным врагом для
медиков была инфекция. Выживать удавалось единицам. Жизненно необходим был
препарат, который смог бы остановить вспышки заражения. И таким лекарством стал
пенициллин.

На западе препарат стали использовать с 1942 года, и случаи смертности
значительно снизились. Правительство Советского Союза вело активные переговоры
по его приобретению, но союзники затягивали со сроками поставки. Выход был один
– придумать «свой пенициллин». Эта работа было поручена Зинаиде Ермольевой, в
то время возглавлявшей лабораторию в Институте экспериментальной медицины
Москвы. Уже опытной Ермольевой, не раз останавливавшей вспышки холеры и
эпидемии, предстояло сделать невозможное – в кратчайшие сроки изобрести
пенициллин. В сентябре 1943 года препарат был получен и ровно через год признан
лучшим в мире.

\subsubsection{Наша землячка – донская казачка}

Научные открытия, звания и учёные степени были ещё впереди. Но и тогда, когда
наукой Зинаида Виссарионовна ещё не занималась, она имела нечто, что всегда
помогало ей и что она сохранила до конца своих дней - ермольевский характер:
силу воли, жажду знаний, целеустремлённость и невиданную трудоспособность.

По происхождению Зинаида Ермольева - донская казачка. Родилась она на хуторе
Фролов Донской области. Окончила Новочеркасскую гимназию. На выпускном балу,
когда звучал её любимый «Сентиментальный вальс» Чайковского, она вдруг
неподвижно застыла у окна. В это время она зачитывалась биографией выдающегося
композитора. Такой гениальный человек, и такой безрадостный конец – страшная
смерть, а затем гроб, залитый известью...

«Тяжёлая болезнь, которая плохо поддаётся лечению» - вспоминала она строки из
книги. Мать Чайковского тоже умерла от холеры. «Неужели это наследственная
болезнь? Нет, конечно. А может быть, передаётся предрасположенность к ней?»

Мысли Ермольевой на выпускном балу определили окончательный выбор её будущей
профессии – она будет врачом. А как же искусство, которое так трогало её душу?
Оно останется с ней, как радость и помощник в трудах всей её жизни.

О, сколько нам открытий чудных…

Зинаида Ермольева начала свой трудовой путь в качестве ассистента кафедры
микробиологии при Северо-Кавказском бактериологическом институте. «Ещё будучи
студенткой, я вставала ни свет ни заря и пробиралась через форточку в
лабораторию, чтобы лишние пару часов отдать опытам».

Когда Зинаида Виссарионовна окончила институт, она решила остаться в родных ей
стенах и стала заведующей кафедрой микробиологии. Её всё больше и больше
интересовала новая, малоизученная область микробиологии - биохимия микробов.

Но не только лабораторной жизнью жила Ермольева. В 1922 году Ростов-на-Дону
охватила вспышка холеры. Зинаида Виссарионовна изучала болезнь не по книгам, а
в реальной жизни, самолично осматривая заражённых людей. В то время
«классический холерный вибрион» был уже известен, но практика показывала, что у
него есть «сородичи» - холероподобные. Но вопросы о том, где искать их и как
обезвреживать, по-прежнему оставались без ответов.

Зинаида Ермольева взялась за их поиски. Она провела целый ряд лабораторных
опытов, но был необходим кардинальный эксперимент – испытания на человеке.
Опасный опыт она решила поставить на себе. «Эксперимент, который едва не стал
трагическим, доказал, что некоторые холероподобные вибрионы способны
становиться «истинными» - вызывающими заболевание», - было написано в отчёте к
эксперименту, ставшему позже научным открытием.

Микробиолог Павел Лащенков в 1909 году получил из куриного яйца вещество,
которое блокировало распространение и развитие некоторых микробов. Чуть позже
британский бактериолог Александр Флеминг выделил это вещество в тканях печени,
сердца, лёгких, а также железах и слюне человека. Это вещество он назвал
«лизоцимом», но значения в использовании на практике ему не придал.  

Ермольева разработала метод концентрации и выделения лизоцима, установив его
природу, и стала впервые применять его в практических целях. Вещество было
выделено в хрене, редьке, репе и т.д., таким образом, объяснялись многие
«чудодейственные» рецепты народной медицины. Уже значительно позже Ермольева
смогла получить лизоцим-кристалл, который стали применять в хирургии, педиатрии
и других областях медицины.

По предложению Ермольевой лизоцим также стали использовать в сельском хозяйстве
и пищевой промышленности. Ей выдали авторские свидетельства на «способы
консервирования икры» и «мочки льна». В 1935 году Зинаиде Ермольевой была
присуждена докторская степень, а четырьмя годами позже её утвердили в звании
профессора.

\subsubsection{В годы Второй мировой}

Ермольева получила от Иосифа Сталина срочное донесение с приказом вылететь в
Сталинград «для осуществления профилактических мер по предотвращению
распространения холеры». Правительство страны приняло решение «выдать холерную
сыворотку всему населению города и находящимся в нём войскам» и эта миссия была
возложена на хрупкие плечи Зинаиды Ермольевой.

Но как же это сделать, ведь привезённой с собой сыворотки было явно
недостаточно, а эшелон, шедший из Москвы, по слепой воле случая разбомбили
вражеские самолёты. Оставался единственный выход – организовать производство
препарата в стенах осаждённого фашистами города. Несмотря на тяжелейшие
условия, в которых приходилось работать людям, и не без помощи Ермольевой,
производство препарата было налажено. Каждый день его принимали почти 50 тысяч
человек, чего в истории ещё никогда не было.

Зинаида Виссарионовна наблюдала за ранеными солдатами и видела, что большинство
из них умирает не от полученных в бою ран, а от заражения крови. Она прекрасно
понимала, что необходимо лекарство для того, чтобы спасти раненых. В 1929 году
всё тот же Александр Флеминг получил из плесени пенициллин, но выделить его в
чистом виде у него не получилось, так как препарат оказывался весьма нестойким.
Соотечественники Ермольевой уже давно заметили лечебные свойства плесени: ею
врачевала одна из сподвижниц Степана Разина - Алёна Арзамасская; с плесенью
работали сотрудники петербургской военно-медицинской академии.

Зинаида Ермольева задалась целью получить пенициллин из отечественного сырья и,
после долгих экспериментов, она его получила. «Первый советский
пенициллин-крустозин, который мы получили в нашей лаборатории, творил чудеса.
Он значительно задерживал рост микробов, вызывающих заражение крови, воспаление
лёгких и газовую гангрену».

Ермольевой удалось не только создать препарат, но и организовать и наладить его
промышленное производство. Делалось всё это в тяжелейший для Советского Союза
период Второй мировой войны. Пенициллин активно использовался на фронте и,
возможно, сделал для победы больше, чем десятки солдатских дивизий. Доступность
лекарства позволила уменьшить число смертности от инфекции ран. «Солдат Малышев
поступил в клинику с серьёзным осколочным ранением коленного сустава. Через две
недели, после наложения гипса, обнаружено гнойное воспаление, которое стало
распространяться на голень. Мы начали лечить его пенициллином, и через десять
дней температура пришла в норму. Рана заживает. Полное выздоровление».

Новость о чудодейственном препарате в считанные дни разнеслась по всем
госпиталям. Отовсюду шли письма от солдат с просьбой скорее вернуть их на
фронт.

В 1944 году Москву посетил создатель «импортного» пенициллина – профессор
Флори. Он понятия не имел о том, что в Советском Союзе есть свой пенициллин, и
потому «по секрету» рассказал, что ему удалось создать чудодейственное
лекарство и что он в качестве бескорыстного дара привёз с собой несколько доз.
И каково же было его удивление, когда в ответ на это вместо благодарностей и
восторгов Зинаида Ермольева спокойным голосом сказала, что в Москве почти целый
год работает пенициллиновый завод.

Британцы не поверили ей и предложили проверить эффективность советского
пенициллина. В тот же день состоялся эксперимент. В Яузской больнице были
отобраны 12 солдат с заражением крови. Их положили в одной палате – по шесть с
каждой стороны. Одних лечили британским лекарством, других советским. Оба
препарата в итоге показали одинаковые результаты, но пенициллин советского
производства требовал при лечении меньшей концентрации. Этот эксперимент
показал, что отечественный препарат более эффективен.

\subsubsection{Наследие}

Зинаида Виссарионовна Ермольева была главным редактором издания «Антибиотики» и
членом редколлегии международного «Журнала антибиотиков». На «Седьмом
международном конгрессе химиотерапии» её наградили медалью за «выдающиеся
достижения в области медицины». Зинаида Ермольева написала 535 научных работ.
Её товарищ Вениамин Каверин говорил: «Она необыкновенно щедрый человек, как в
науке, так и в жизни».В научной щедрости талантливого учёного убедились
миллионы спасенных ею людей, что же касательно личностных качеств, то здесь
Зинаиду Виссарионовну запомнили как отзывчивую, добрую, и не чуждую проблемам
других женщину. Профессор Ведьмина – ученица Ермольевой - рассказывала:
«Зинаида Виссарионовна могла вас просто не замечать в те минуты, когда у вас
было всё хорошо. Но когда ваш рабочий или душевный комфорт что-то нарушало –
она была тут как тут. Всегда помогала, хлопотала, да и просто… была рядом».

