% vim: keymap=russian-jcukenwin
%%beginhead 
 
%%file 25_10_2021.fb.volga_vasilii.1.rasterjannost.cmt.kulagina.putin
%%parent 25_10_2021.fb.volga_vasilii.1.rasterjannost.cmt
 
%%url 
 
%%author_id 
%%date 
 
%%tags 
%%title 
 
%%endhead 
\paragraph{Елена Кулагина - Путин умный человек, ответил честно - не знает}

\begin{itemize} % {
\iusr{Елена Кулагина}

Путин умный человек, ответил честно - не знает . Сейчас не знает. Дальше посмотрим.

\iusr{Михаил Иванов}

Ответы есть. И не стоит приписывать свою растерянность другим.

То, что человек не стал отвечать и вежливо ушел от ответа еще не означает , что
последнего нет.

А то, что окраина прилепилась большинством к НАТО - к Западу, так это уже
проходили в 1941 - 1945 гг....(И тогда против России воевала объединенная
европа , а так называемые "союзники" продавали им оружие.....Роль англосаксов
также можно разложить подробно.)

История повторяется!!!

\begin{itemize} % {
\iusr{Василий Александрович Волга}
\textbf{Михаил Иванов} он не ушёл от ответа. Он ответил: «Я не знаю, что делать».

\iusr{Михаил Иванов}
\textbf{Василий Александрович Волга} , зачем же так плоско мыслить (или узко - если угодно).

Зачем ему выдавать все решения , тем более, если они есть.

Могу говорить лишь за себя: когда у меня есть решение , я его никогда не озвучу, так как :

1. Даю возможность опоненту прийти в себя и предпринять адекватные шаги.

2. Если последнему нет дела до здравого смысла - чести и достоинства, то тем
более не стоит даже намекать на решение или как-либо его озвучивать. (Подлецы
извратят все сказанное. Логичнее действовать.)

\iusr{Василий Александрович Волга}
\textbf{Михаил Иванов} 

мыслю я плоско в силу ограниченности своих мыслительных способностей (уж
извините), и просто наблюдаю, поскольку, в отличие от Вас, Михаил, не умею
читать мысли вождей. А суть наблюдений пока следующая: 

1. Путин заявляет о том, что он не знает ответа на вопросы об Украине. 

2. ЛДНР, как пороховая бочка гудит недовольством, переходящим в ненависть к
Путину. 

3. НАТОвские структуры разворачиваются на Украине невзирая на все красные линии. 

4. 8 лет разговоров о тайном плане Путина - это две ВОВ. Вот, примерно, так


\iusr{Михаил Иванов}
\textbf{Василий Александрович Волга} ,

Все верно. (Если интересно, у меня больше вопросов и даже претензий ...,
вопросов предельно конкретных по конкретным действиям - ошибкам имеющим
длительные и разрушительные следствия. Но я это не выливаю публично, так как
помню , что , во-первых, мысль материальна , а , во-вторых, сознание масс не
только слабо (окраина это подтверждает), но еще и склонно не к силам
Созидания...!)

Вопрос к вам лично. У вас идет война , ваши "земляки" (сожители, со....ки...-
как угодно) убивают женщин, детей и стариков. Что сделали при этом лично вы,
чтобы спасти одного ребенка или старика (кроме постов в фейсбуке ...,Когда
рядом с вами на улице бьют старика, женщину или ребенка , а вы , стоя рядом ,
рассуждаете , как это плохо... Это ,мягко говоря, бездействие.)?

P.S. При этом и то, что хотя бы пишите, я уже ценю.

\end{itemize} % }

\iusr{Алена Ленина}

Путину и не надо знать ,зайдет в фейсбук ,почитает всех комментаторов и
сделает.. Великие спецы по всем вопросам сидят у Василия Александровича, да и
не только у него..)))

\iusr{Геннадий Секерин}

Неправильно цитируете Василий Путин дословно сказал пока не знаю что ключевое
слово тут пока. А что вы ещё хотели? Чтобы он выложил план поэтапной
денацификации Украины?

\begin{itemize} % {
\iusr{Василий Александрович Волга}
\textbf{Геннадий Секерин} Это не так. Он ответил: «Я Вас разочарую. Я не знаю ответа на этот вопрос».

\iusr{Ольга Мелник}
\textbf{Василий Александрович Волга} Смените пластинку..

\iusr{Василий Александрович Волга}
\textbf{Ольга Мелник} т.е.? Вам не нравится эта цитата? Так он несколько раз повторил её.
\end{itemize} % }

\iusr{Юлия Рудникова}

Павел Глоба сказал,что 2024 году Донбасс присоединится к России и будет
укрепление ее на международной арене в последующие 10 лет за счёт присоединения
союзных государств, но состав их будет не такой как при СССР. На Украине
лояльная к России власть появится не раньше 2031. В 2024 году есть угроза
переизбрания Зеленского на второй срок, если поменяется дата выборов. Вот такие
его неутешительные прогнозы....для Украины

\begin{itemize} % {
\iusr{Любовь Гурьева}
\textbf{Юлия Рудникова} после того, как Глоба развелся с женой- не верю ни слову. Он когда женился не смог просчитать свое будущее.

\iusr{Юлия Рудникова}
\textbf{Любовь Гурьева} 

Он историк по образованию , и делает неплохой анализ с точки зрения истории и
астрологии .75 процентов его предсказаний сбываются. Астрология -это не гадание
на кофейной гуще, а наука, дающая определенные тенденции вероятности тех или
иных событий. Есть события , которые в личной астрологической карте человека
избежать не получиться. Я тоже занимаюсь профессионально астрологией и его
анализ считаю достаточно компетентным . Хотя сама видя астрологическую карту
Украины , думаю там перспективы менее радужные .

\end{itemize} % }

\iusr{Нина Ярушина}
Желательно провести референдум.думаю,что на 1-2\% будет больше против нато...

\iusr{Любовь Гурьева}
потому, что слишком высокий процент тех, кто против России.

\iusr{Tatjana Wolkowa}

Населению Украины бросили сладкую печенюжку - « безвиз», а с ним возможность
работать за границей и этим быстро улучшить своё материальное положение! Так же
возможность « зацепиться» в европейской стране, получить вид на жительство. Это
и есть главный приоритет и главный козырь. Теперь люди пойдут на все, чтобы
отстоять эту « печенюжку». Все банально - люди хотят лучшей жизни любой ценой,
и мало кого интересует, кто будет их « хозяином», главное- кто больше заплатит.
НАТО уже в Украине, красная черта уже пройдена.

\iusr{Константин Пулин}

Ничего не надо делать. Активные пророссийские граждане Украины уедут с Украины
в Россию. России от этого будет только хорошо, т.к. рабочих рук не хватает.
Остальные активные украинцы уедут в Европу на заработки. Европе будет хорошо.
Плюс рождаемость ниже смертности. Население Украины от таких процессов
сокращается с тройной скоростью. Зачем что-то делать с оставшимся пассивным
населением и пенсионерами? Опасности они не представляют. Да ракеты станут
ближе к Москве. Ну и что? Вот Эстония - тоже страна НАТО и близко к
Санкт-Петербургу. Никто же этого не боится почему-то?

Пусть Европа боится принимать Украину в НАТО. Потому, что пока Украина не
является членом НАТО, вся ответка на Украинские провокации прилетит Украине. А
когда Украина станет членом НАТО, тогда ответка будет прилетать в Брюссель.
Принимая Украину в НАТО, они же и ответственность за неё берут, правильно? И
платить за неё будут. Не Украина же сама будет закупать оружие? У Украины на
современное вооружение денег нет  @igg{fbicon.smile}  Были бы деньги, они бы уже и сейчас
закупали американское оружие, не ждя членства в НАТО. Украинцы потому и
стремятся в НАТО потому, что ждут, что за них и тут заплатят  @igg{fbicon.smile}  Европейцы всё
понимают. И потом не спешат брать ответственность за Украину в НАТО и платить
за неё. Может и не возьмут никогда - одними обещаниями кормить будут.

А России не надо пытаться принять в свои объятия тех, кто этого не желает. Есть
желающие - Турция, Иран - с ними и надо объединяться. Эти страны раз в 5 более
мощные и полезные для ЕвразЭС, чем Украина.

\iusr{Любовь Сидельникова}

А что он вообще может ? Семилетнюю войну остановить не может . С ,, Талибаном
,, нашёл точки соприкосновения ,, и партнёрство заключил а с Украиной ничего
решить не может . Приедут , дадут команду , решит , как остаться Ни причём ,
при деньгах , власти , сырьевых потоках ...... Америка в суп не кинет курицу
несущую золотые яйца .

\iusr{Alexander Merzliakov}

Никто никогда не говорит о своих планах. Было бы слишком опрометчиво.
У них свои игры, это противостояние будет ещё долго.

\iusr{Игорь Казаков}

Ошибочное суждение о рассеянности. Никакой рассеянности нет и очевидно, быть не
может... Традиционные штампы в сложившейся многосложной ситуации не сработают.
Именно этот аргумент раскрывает содержание промежуточного суждения Президента:
"не знаю, что делать". Необходимы нетрадиционные решения развязки, которая
рождается из недр самого внутри украинского конфликта и международных
отношений, влияющих на них. В каждом временном отрезке истории существует
единственная точка - момент, определяющая структуру эффективного действия.
Путин ждёт этого решения, которое само себя предложит в своём наиболее
завершённом виде, и это произойдет уже в ближайшее время. Если бы Президент
руководствовался исключительно правовыми и военно-политическими навыками, он
никогда не был бы тем человеком, которого не понимает весь мир, но слушает
каждое его слово, затаив дыхание. Решение будет. Решимость Президента тогда
проявляется, когда он осознает законченность, завершённость сюжета, который и
знаменует собой рождение ответа на искомый вопрос. Нужно уметь ждать. Нужно
уметь не совершать непоправимых ошибок. Путин умеет!

\begin{itemize} % {
\iusr{Василий Александрович Волга}
\textbf{Игорь Казаков} дай Бог. Кто ж против? Будем ждать. Пока есть один ответ: «Я не знаю».

\iusr{Игорь Кириченко}
\textbf{Василий Александрович Волга} 

Вы очевидно запамятовали, что Путин стратег и показное его "беспомощное
неведение" это тактический ход на этот промежуток времени не более того. Он
давно уже обыграл как минимум несколько вариантов решения "украинской проблемы"
и уже имеет стратегическое видение окончательного акта этой политической игры.
Он выжидает подходящего момента для последнего хода "в дамки", но для этого ему
нужны союзники, кроме ОПЗЖ и самого Медведчука. Такими внештатными помощниками
для следующего госпереворота по кремлёвскому сценарию могут стать нынешние
формально опальные украинские олигархи. Возможно когда их масса из количества
перейдёт в качество всё и случится.

\iusr{Василий Александрович Волга}
\textbf{Игорь Кириченко} 

Игорь Глебович! Вам поверю. Только вот не могу представить, что будут делать с
народом Украины те, кто совершит Переворот? Как они тех 60\%, о которых говорит
Погребинский (а их гораздо больше, если быть честным до конца), которые
ненавидят Россию, приведут в другое состояние?

% -------------------------------------
\ii{fbauth.kazakov_igor.haysyn.vinnia_obl.ukraina.bogoslovie.psiholog}
% -------------------------------------

\textbf{Василий Александрович Волга} 

В природу человека Господь заложил способность творческого развития. Это
касается в первую очередь духовности, как причины всякого творчества. Иерархия
ступеней развития человека имеет большую протяжённость от ступени животного,
органического сознания, до проявления Божественной сущности человека, его
Обожения. Существующий на данный момент уровень развития большинства
человечества варьируется между ступенями индивидуумов и личностей. Ни то, ни
другое не является отражением духовности человека. Личность может быть
выдающейся, но живущей на уровне земных приоритетов развития сознания.
Большинство людей все же находятся на ступени развития индивидуальности. Это
тот уровень для которого существенным является лишь его индивидуальность,
мировоззрение, ощущения, привязанности, привычки, стереотипы мышления. Именно
данной категории сознания человека присуща основополагающая черта или свойство
- ВНУШАЕМОСТЬ! Для данной ступени развития это главное, что оказывает влияние
на всего человека. Доминирующим фактором является воздействие на него
информационной среды. Человек попадает в сеть доминирующих отношений, взглядов,
поступков, и спустя некоторое время, эти воздействия индивидуальность
воспринимает в себя полностью, адаптируя их в себе, как внутренние наработки
души. Это своего рода программирование человека на уровне доминирующих полей
сознания. Сердце человека, его душа подобно магниту притягивает к себе из мира
все, что мир на него извергает. Напомню, речь идёт в данном случае о ступени
развития человеческой индивидуальности. На данном уровне развития
созидательный, творческий и критический анализ, выбор и действие не прописаны.
Это означает, что бесконечное количество сменяющихся отношений, взглядов,
принципов, воздействующих на человека, будут восприниматься им со временем, как
его собственные внутренние структуры сознания. Там будет продолжаться
бесконечно долго до тех пор, пока человек не выйдет из замкнутого поля
индивидуального сознания. Для этого ему предстоит через личностный рост
сознания преодолеть свою ограниченность и замкнутость, и выйти к осознанию
целостности и единства духовного мироздания. До этой ступени большинство людей
не доходит, а значит на каждом историческом витке будет вновь и вновь
повторяться "Свадьба в Малиновке". Человек будет окрашиваться в то состояние -
облачение, которое на данный период времени является доминирующим. Это грустное
заключение, но оно от этого не становится менее реалистичным.

Вывод: Изменившиеся условия материального, социального, политического
культурологического доминирования неизбежно приведут к изменению в сознании и
отношении человека к обновлённым условиям существования. Пока человек не возрос
в своей природе до Богочеловеческого сознания, не стал носителем духовности, он
остаётся на ступени индивидуальности с неограниченной степенью внушаемости
сознания. Это признаки раннего возраста души, когда подобно ребёнку, взрослым
людям нужны поводыри, которые ему все все расскажут, покажут, объяснят и
поведут "куда надо" . Пока мы такие... Но это также отвечает на вопрос, что
должно измениться в целом народе, чтобы он изменил свои взгляды и отношения -
ДРУГАЯ СРЕДА!

\iusr{Василий Александрович Волга}
\textbf{Игорь Казаков} 

слов много, и много пустых, но главная мысль верна. И что? Я ведь именно об
этом. Теперь поменяйте среду (кто это сделает), включите новую социальную
инженерию и через тридцать лет будет результат.

\iusr{Игорь Казаков}
\textbf{Василий Александрович Волга} 

Социальная инженерия не имеет никакого отношения к реалиям. Господь проводит
народ через то поприще, которое на грани боли возродит в нем память о Источнике
всех разрешений и перемен. Тогда и случится смена реалий и среды. Вся история
человечества об этом. Сроки известны только Богу, но и они индивидуальны для
каждого человека. Жатва идёт...

\iusr{Игорь Кириченко}
\textbf{Василий Александрович Волга} 

60\% ненавидящих Россию это пропагандистский фейк. В Центральной и Юго-восточной
Украине " статистика слишком меркантильная и не отвечает действительности.
Такое приблизительное число русофобов можно собрать только на Западе Украины,
но они там были всегда мнения разные. Надо поехать в Буковель и спросить ждут
ли они туристов из РФ и всё сразу станет понятным с этой статистикой, а то, что
большинство добробатовцев погибших и раненых на Донбассе в АТО-ООС были выходцы
из Западной Украины, так это не секрет и делалось это специально. В программе
"Антироссия" ненависть на этнической основе к "русскому миру" занимает
центральное место, а окрашенная кровью ещё больше углубляет раскол в украинском
обществе.

\end{itemize} % }

\iusr{Татьяна Вишня}
\textbf{Михаил Иванов}

История повторяется, потому что они с 90-х уничтожили всё, что против
повторения десятилетиями копилось. Растащили Россию по забуграм на виллы и
пуза! Дети их не там? Счета? Они этого больше боятся, что их погонят вместе с
отпрысками, привыкшими жиреть на бедах и смертях. Оружие им пофиг. Разве что
хвалятся советской кнопкой да ад обещают. Это могут. А над народом издеваются и
держат на выживании. Западу задницу все десятилетия лизали. Начиная с
ельцинских слов о благслпавлении Америки. Нынче ельцинский центр, Наина и
разрушение уже полное. Выполнили завет хозяина.

\iusr{Павел Коптев}

"Есть ли у вас план, мистер Фикс?
У меня целых три плана!".
Но, что то пошло не так и не один из планов не сработал... Голод и холод очень хороший учитель.

\iusr{Любовь Истомина}

А почему это Путина должно волновать что делать с Украиной он что уже в России
все для людей сделал пусть заботится о россиянах и о России а то позавидовать
тоже нечему особо .

\begin{itemize} % {
\iusr{Валентина Титова}
\textbf{Любовь Истомина} Путин волнуется за Украину ( и за другие страны) не из альтруистических побуждений, а в интересах России. Жаль, что в России много таких "обиженных", как вы.

\iusr{Александр Харченко}
\textbf{Валентина Титова} не надо волноваться за Украину! От слова совсем!

\iusr{Валентина Титова}
\textbf{Александр Харченко} вот если бы в Украине не было таких, как вы, не было бы ни войны, ни разрухи и жили бы Украина с Россией тихо и мирно, как это было ДО майдана. Так что, сидите уже и помалкивайте

\iusr{Александр Харченко}
\textbf{Валентина Титова} Валентина выймите вату из головы, она вам мешает думать!

\iusr{Александр Харченко}
\textbf{Валентина Титова} и еще таких как вы хочется послать матом! Но вы же не поймете, за что!
\end{itemize} % }

\iusr{Vitali Satir}
Украина сама с собой справиться.

\begin{itemize} % {
\iusr{Paul Isaev}
\textbf{Vitali Satir} в смысле, сама себя добьёт? Тогда, да, согласен.
\end{itemize} % }

\iusr{Aleks Schmidt}

о деградации Украины можно судит по ее президентам. так что ничего не надо
предпринимать. В последние семь лет деградация присходит с космической
скоростью Так что Путин сидит на берегу и ждет когда труп Украины проплывет
мимо.

\begin{itemize} % {
\iusr{Александр Харченко}
\textbf{Aleks Schmidt} скорее Украина увидит в мутной воде труп путина! Еще совсем не вечер!

\iusr{Paul Isaev}
\textbf{Александр Харченко} "Надежды юношей питают." (С) @igg{fbicon.man.facepalming}  

@igg{fbicon.shrug}  @igg{fbicon.face.grinning.squinting}{repeat=2} 

\end{itemize} % }

\iusr{Wl Mikriukov}

Где там Волга увидел растерянность Путина? Там был ясный посыл, выраженный
очень дипломатично: ,, - В своем бардаке раzъ@бывайтесь сами!"  @igg{fbicon.wink} 

Путин и его окружение положили большой болт на Украину еще в то время, когда Волга был еще при власти и оббивал московские властные пороги, выбивая гранты на совершенно бесполезные проэкты! С 2004 года, если не раньше, было ясно куда идет Украина и бесполезно было подпаивать боржомом, а тем более вкладывать деньги!  @igg{fbicon.wink} 

\begin{itemize} % {
\iusr{Kampan Darchanskiyy}
\textbf{Wl Mikriukov} Путин и на Россию, и на свой народ положил большой болт.))

\iusr{Wl Mikriukov}
\textbf{Kampan Darchanskiyy} а на Кампана он большой )(уй положил, шоб не гавкал!  @igg{fbicon.wink} 

\ifcmt
  ig https://scontent-lhr8-1.xx.fbcdn.net/v/t1.6435-9/248319953_629771888406251_2766247553455934773_n.jpg?_nc_cat=106&ccb=1-5&_nc_sid=dbeb18&_nc_ohc=Q4NTWboosW4AX_BCR6c&_nc_ht=scontent-lhr8-1.xx&oh=b38902f96513bc3fd84c46678c418452&oe=619D4AE5
  @width 0.4
\fi

\iusr{Василий Александрович Волга}
\textbf{Wl Mikriukov} а Вы посмотрите ещё раз. Только розовые очки снимите перед просмотром.

\end{itemize} % }

\iusr{Kalisto Kalisto}
В 2014 надо было башкой думать. Сейчас, конечно, не знает.

\iusr{Вячеслав Супряга}
Зато он знает что Меркель надо идти на выборы

\iusr{Vadim Truhachev}
Путин понимает мозгом, что придется бомбить, но сердце этому не может поверить.

\iusr{Владимир Кузьмин}

17 марта1990 года 78\% населения УССР проголосовали за сохранение СССР, 2
декабря 1991года это же население 92\% проголосовало за независимость. Сколько
же людей объективно думающих на этой территории? Остальное просто
конформисты.))) Ещё жаренный петух не клюнул в жопу. Когда это случится, тогда
начнется Свадьба в Малиновке.)))

\begin{itemize} % {
\iusr{Василий Александрович Волга}
\textbf{Владимир Кузьмин} ну, да...

\iusr{Анна Мисюк}
\textbf{Владимир Кузьмин}, наверное, хорошо жить в мире иллюзий...

\iusr{Владимир Кузьмин}
\textbf{Анна Мисюк} В смысле?)))

\iusr{Анна Мисюк}
\textbf{Владимир Кузьмин}, я о тех, кто пребывает в иллюзии о том, что ещё не время, вот посмотрите, как сработает в час "Х" ХПП.

\iusr{Владимир Кузьмин}

Причем тут ХПП - это вы написали.)) Две науки изучайте - историю и теорию
управления, особенно теорию устойчивости системы, тогда реальность не будет
казаться иллюзией.))

\iusr{Василий Александрович Волга}
\textbf{Владимир Кузьмин} т.е. надо ждать не 8 лет, а 88?

\iusr{Анна Мисюк}
\textbf{Василий Александрович Волга}, у меня тот же вопрос  @igg{fbicon.smile} 

\iusr{Владимир Кузьмин}
\textbf{Василий Александрович Волга} 

Возвращаемся в начало. Общество конформистов - ждуны.)) Согласно теории
управления, раздел устойчивость стстемы - устойчивая та система (читай
государство) если внутреннее сопротивление больше или равно давлению внешней
среды. Так как внешнее и внутреннее управление данной системы уменьшает ресурсы
данной системы, т.е. уменьшает сопротивление, то данная система неизбежно будет
разрушена. Это объективная реальность и неизбежность. Ждать этого или не ждать,
личное дело каждого.)))

\iusr{Владимир Кузьмин}
\textbf{Анна Мисюк} Если то, что я описал выше, по вашему ХПП, то это не так. Это ВВП знает законы управления системами.)))

\iusr{Василий Александрович Волга}
\textbf{Владимир Кузьмин} 

ой. как умно. со мной можно проще. высокие материи - это для людей умных. не с
моим-то свиным рылом да в ваш калашный. Одна только небольшая поправочка. Вы, и
люди подобные Вам, Владимир, ничего не понимаете о происходящем на Украине,
именно по той причине, что думаете о ждунах. Так вот, украинцы ничего не ждут.
Они действуют. Причем, самым активным образом. Они только за последние двадцать
лет две революции организовали, чтобы не быть вместе с Россией, а быть в Европе
и НАТО. Так что в молоко Вы, со всей Вашей многоумной лексикой, дорогой
Владимир, говоря о ждунах. С уважением.

\iusr{Владимир Кузьмин}
\textbf{Василий Александрович Волга} 

Ах!)) Вы Василий определитесь Вы к умным или красивым.)) Кто организовал два
переворота украинцы? Это стыдно Василий не знать называя это революциями с
Вашими левыми взглядами.)) А те активисты, что бегали по майдану, это те 12\%,
которые голосовали против СССР! И даже те которые искренне боролись с
"злостными режимами", боролись не за НАТО и Европу, а за свой вишневый садочек
на своих шести соток.)) И последнее, как говаривал классик: "Простота хуже
воровства!" Так не надо из себя строить просточка не к лицу Вам. С
благодарностью.

\end{itemize} % }

\iusr{Павел Павел}

Не стоит паниковать: все не так ужасно. Активное меньшинство всегда заказывает
музыку, а пассивное большинство т. н. болото под эту музыку танцует. Активных
передавят силой и конформистское большинство пойдет за этой силой. Большевики
получили меньшинство на Учередительном собрании, но когда они его разогнали и
взяли власть силой в их партию желающих отбоя не было, что даже пришлось
чистить ряды. Так и здесь. После распада Союза многие переобулись. Переобуются,
если что, еще раз.

\begin{itemize} % {
\iusr{Василий Александрович Волга}
\textbf{Павел Павел} продолжайте фантазировать. Иногда это помогает. Только вот не нам, а им.
\end{itemize} % }

\iusr{Сергей Сергиенко}
План должен быть всегда ! Что делать ? Перестать лизать жо .. своим "партнёрам" .Идти своим курсом не обращая внимания на лай собак !

\iusr{Leniev Madalid}
Ну при вступлении Украины в НАТО у нее еще что отвалится

\begin{itemize} % {
\iusr{Kampan Darchanskiyy}
\textbf{Leniev Madalid} Кум Путина бандеровец Медведчук.))

\iusr{Leniev Madalid}
\textbf{Kampan Darchanskiyy} хех
\end{itemize} % }

\iusr{Сергей Алферов}

Ну он так и ответил на вопрос. Вы серьезно? Он должен был раскрыть планы? Все
правильно, пусть думают, что планов нет и может быть все что угодно. А как
иначе?

\begin{itemize} % {
\iusr{Василий Александрович Волга}
\textbf{Сергей Алферов} промолчать, например, пошутить, рассказать басенку... Он же только растерянно сказал, что не знает. Пересмотрите этот момент.

\iusr{Анна Мисюк}
\textbf{Сергей Алферов}, уже 8 лет бродит эта легенда о ХПП. Скажу Вам по секрету - и дальше будет бродить.
\end{itemize} % }

\iusr{John Depp}

Просто вслух не говорит! А вот, что с вашей позицией, это большой вопрос ?!?!
Что-то точно случилось?!

\begin{itemize} % {
\iusr{Василий Александрович Волга}
\textbf{John Depp} я не отношусь к тем людям, которые предпочитают не мыслить, а быть в толпе, несмотря ни на что. Всё просто. Вы же можете оставаться там, где Вам комфортно.
С уважением.
\end{itemize} % }

\iusr{Антон Гарапко}

Растерянность -синоним встревоженность, беспомощность или что -то наподобии!
Это точно не о Путине!!!

\begin{itemize} % {
\iusr{Василий Александрович Волга}
\textbf{Антон Гарапко} значит всё хорошо. Просто что делать - непонятно.

\iusr{Антон Гарапко}
\textbf{Василий Александрович Волга}
Этот вопрос затянулся на годы, но чем дальше в лес,.... Вас же всё устраивает!

\iusr{Антон Гарапко}
\textbf{Василий Александрович Волга}, Ну и началось наступлЕНИЕ всу!
\end{itemize} % }

\iusr{Сергей Моисеев}
Что делать? Забрать девять областей

\begin{itemize} % {
\iusr{Василий Александрович Волга}
\textbf{Сергей Моисеев} Вы так поняли его ответ?

\iusr{Сергей Моисеев}
\textbf{Василий Александрович Волга} 

я всегда понимал что девять областей надо было еще в 14 м забрать и делал все
что мог в этом направлении. Кто теперь помнит что Смоленск сто лет был под
Литвой? Если бы Иван Третий жевал сопли то и Калуга нам бы была известна как
литовский город.

\end{itemize} % }

\iusr{Любовь Колисниченко}
Что тут непонятного. Только полетит с нашей территории, что нибуть в сторону России , нам конец

\iusr{Любовь Колисниченко}
И не надо обольщаться, что Путин в растеренности. Вы все забыли, что он чекист? Он что не умеет скрывать эмоции?

\begin{itemize} % {
\iusr{Василий Александрович Волга}
\textbf{Любовь Колисниченко} ну, так скрыл бы. Он же наоборот, показал...(

\iusr{Юрий Беспалюк}
\textbf{Любовь Колисниченко} если он Чекист, то пусть поставит памятник Дзержинского на место
\end{itemize} % }

\iusr{Anna IMikhalkova}

Иногда полезно ничего не делать. Пусть себе вступает, тогда появится очевидный враг и будет беспощадная борьба .
А сегодня это невозможно: братья-небратья..

\iusr{Алексей Ярашев}

\obeycr
Вездесущ, всемогущ и всеведущ,
Окутан голубыми небесами.
Глядит на Украину с райских кущ,
И думает раз.. буйтеська сами.
\restorecr

Об этом и сказал В. В. Путин. Ни на секунду не сомневаясь и тем более не впадая в растеряность.

\begin{itemize} % {
\iusr{Василий Александрович Волга}
\textbf{Алексей Ярашев} и Слава Богу!!
\end{itemize} % }

\iusr{Kampan Darchanskiyy}

Не мешайте Россию с Путиным, как говорится не путайте Отечество с его
Превосходительством. Россия - это Великая страна, а Путин - серое ничтожество
волею судеб, вернее по воле российских аллигаторов, посаженное на трон. Ну, а
то что народонаселение РФ полностью зомбировано, так это по Геббельсу: " Дайте
мне средства массовой информации и я с людей сделаю стадо баранов" и " Чем
ужаснее ложь, тем толпа охотнее в неё верит", а лжет Путин постоянно.

\begin{itemize} % {
\iusr{Александр Островский}
\textbf{Kampan Darchanskiyy} хороший бот, жирный  @igg{fbicon.smile} 

\iusr{Галина Шулаева}
\textbf{Александр Островский} и завистливый)))

\iusr{Kampan Darchanskiyy}

О, как путиноголовые возбудились, прямо, как свидомиты на своего Бандеру.))
Путинизм и майдаун головного мозга - это одна болезнь, только по разному
называется в Шумерии и Путиляндии.)) А так то вы побратымы. " Россияне и
украинцы один народ." ( В.В.Путин).  @igg{fbicon.face.tears.of.joy} 

\end{itemize} % }

\iusr{Юрий Чумаченко}

Украина не вступит в НАТО. Как государства ее время на исходе. Базы НАТО на ее
территории будут разрушены на стадии начала строительства... то, что Украина
переродилась, даже хорошо... достаточно лицемерных заявлений о славянском
братстве. Украинец враг - это честно. Друг, товарищ и брат - ложь и лицемерие,
и пора ему положить конец.

\begin{itemize} % {
\iusr{Галина Шулаева}
\textbf{Юрий Чумаченко} 

не могу я принять "украинец это враг".Во мне, по-большому счету, нет ни капли
русской крови: 50\% украинской, кварта белорусской и кварта польской. И я как я
сама себе буду врагом?

Бандеровец враг-да. Согласна.

\iusr{Юрий Чумаченко}
\textbf{Галина Шулаева} . 

Мне жаль. Но украинец - это русский, проживающий на Украине. Ни у кого в голову
почему-то не приходит мысль назвать сибиряков, волжан, жителей Кубани, Рязани,
Курска и т.д. этносом. Они все русские, как и жители Украины.


\iusr{Василий Александрович Волга}
\textbf{Юрий Чумаченко} 

чушь это, извините меня, Юрий. Украина укрепляется и идеологически и
экономически. Базы НАТО строятся. Народ становится все более гомогенней
структурой, устремлённой в будущее и ненавидящее Россию. Даже слово ненависть
здесь не подходит. Презирают и не хотят иметь ничего общего. ((

\iusr{Юрий Чумаченко}
Да не Украина укрепляется, Василий... Украину укрепляют.

\iusr{Василий Александрович Волга}
\textbf{Юрий Чумаченко} пусть так

\iusr{Юрий Чумаченко}

Ну так это существенно... Переориентировать большинство украинского население
на Россию больших усилий не потребует, даже времени. Понятно, что я говорю о
русских, если хотите - православных, а не об униатах , с которыми не
договориться и которых придётся ущемить в правах, как национальное меньшинство.
( я не вкладываю религиозный смысл в термины, только идеологический).


\iusr{Сергей Концеропятов}
\textbf{Василий Александрович Волга} , 

Вы и правы, и не правы одновременно. Украину укрепляют, несомненно. Одно но: её
укрепляют, как французы и англичане взаимно в прерии и лесах Сев.Америки
строили и укрепляли форты, проецирующие влияние и силу на окрестных индейцев.
Участь местных их интересовала ровно до границ обеспечения жизнедеятельности
форта. Так и здесь. У меня на Украине масса друзей и знакомых. И я очень
опасаюсь, что идиоты в Киеве спровоцируют действительно тяжёлый силовой ответ
Москвы, в составе которого произойдёт серия ударов "по центрам принятия
решения". Как только это произойдёт, англосаксы по своей давней традиции
эвакуируют своих - и дело с концом. Списывали они активы и подороже... IMHO,
естественно.

\iusr{Андрей В. Марочкин}
\textbf{Василий Александрович Волга} нет, не чушь! Просто история, как наука — это ОДНО, а политика, завернутая политиканами в выдуманное ими «прошлое» (даже которого никогда не было) — это совсем ИНОЕ

\iusr{Василий Александрович Волга}
\textbf{Юрий Чумаченко} 

а вот про это забудьте от слова совсем. Так же, как Запад работал на Украине 30
лет, при полном бездействии России, так же и России придется работать 30 лет,
чтобы воспитать новых людей. Забудьте. Ну, или если Вам это тяжело понять и
необходимо фантазировать ради собственного психологического равновесия, то
можете предаваться этим фантазиям. С уважением к Вам.

\iusr{Василий Александрович Волга}
\textbf{Сергей Концеропятов} 

да уже восемь лет об этих угрозах разговоры. А воз и ныне там. Да еще и
президент ваш заявляет, что он в полной растерянности и не имеет ответа на
вопрос, что делать(?), случае вступления Украины в НАТО. А Украина и не будет
"вступать", она уже вползла наполовину и лет через десять завершить этот
процесс полностью.

\iusr{Юрий Чумаченко}

Да бросьте, Василий. Уже своим соседством Россия влияет на Украину,
родственными связями, не говоря уже о ТВ, кино, гастролях известных артистов...
Книги ещё не сожгли, речь на русском не истребима. Напуганы, унижены,
безвольные и беззащитные. И русские ( где-то далеко внутри, но которые вылезут
наружу... когда наши придут).

\iusr{Андрей В. Марочкин}
\textbf{Юрий Чумаченко} поддерживаю.

\iusr{Василий Александрович Волга}
\textbf{Юрий Чумаченко} 

можете пребывать в этой иллюзии. Она ведь ни на что не влияет. Будете Вы так
думать, или иначе. Пока же украинцы голосованием показывают свои предпочтения.
И самое главное - дети.

Но фантазии часто имеют терапевтический эффект. В нашем с Вами возрасте это бывает крайне необходимо. Так что фантазируйте)

С уважением.


\iusr{Юрий Чумаченко}
Это не фантазии. Все это уже было и не раз. И в России и Германии, если рассматривать близко.


\end{itemize} % }

\iusr{Ольга Борисовна Чинарова}
Путин знает ,что делать и не обязан раскрывать свой план преждевременно

\begin{itemize} % {
\iusr{Василий Александрович Волга}
\textbf{Ольга Борисовна Чинарова} 

действительно! 8 лет (почти половину его пребывания у власти) ему удаётся
прятать от всех свой план. Может и ещё двадцать восемь лет прятать. Главное,
чтобы мы вместе с ним дожили.

\iusr{Любовь Колисниченко}

А что Вы от него хотите ? Может наше население, как крымчане не допустили НАТО
в страну . Очаков знаете? 15 американских лаборатоиий в стране у нас находится,
они наверное антибиотики выращивают? И всё всех устраивает

\end{itemize} % }

\iusr{George Petrenko}

Перестаньте. Там не один план с расписанным до мелочей алгоритмом действий.
Вспомните Крым: без радиосвязи, синхронно, как единое целое, отработала масса
людей и подразделений.

\begin{itemize} % {
\iusr{Василий Александрович Волга}
\textbf{George Petrenko} это Вы Путину? Чтобы он перестал говорить о том, что не знает, что делать?

\iusr{Анна Мисюк}
\textbf{George Petrenko}, неужели Вы не понимаете, почему и как, на самом деле, случился Крым?
\end{itemize} % }

\iusr{Станислав Прокопчук}
Скорее Украина не знает что делать с Донбассом. Как чемодан без ручки и бросить жалко и нести тяжело.

\iusr{Владимир Шелемин}
Путин не знает, а Волга с Погребинским знають

\begin{itemize} % {
\iusr{Василий Александрович Волга}
\textbf{Владимир Шелемин} Вы о чём? Вы смотрели выступление Путина? Так вот, это не Волга сказал или Погребинский. Это сказал он сам о себе.
\end{itemize} % }

\iusr{Владислав Дылов}

Вот, ВВП, взял и все вам рассказал. План есть всегда. Не обольщайтесь. Всё
произойдёт очень быстро...

\begin{itemize} % {
\iusr{Василий Александрович Волга}
еще через 8 лет? или через 16? а мы доживем? а Путин?

\iusr{Владислав Дылов}
\textbf{Василий Александрович Волга} Всему свое время. Красные линии прочерчены... Вы хотите ускорить процесс краха Украинской государственности???

\iusr{Василий Александрович Волга}
\textbf{Владислав Дылов} я давно об том говорю. Ибо я отрицаю построенное на Украине государство. Путин признает, но не я.
\iusr{Владислав Дылов}
\textbf{Василий Александрович Волга} К сожалению, что Украинанская власть, что народ, привыкли что-то или кого-то ждать. Америку, Евросоюз, Россию. А сами то, что??? Где национальная или, хотя-бы, человеческая гордость???

\iusr{Василий Александрович Волга}
\textbf{Владислав Дылов} а кто Вам вообще сказал, что на Украине кто-то кого-то ждёт? Ждали в Крыму. Или крымчане, по-вашему, сами все решили?
На Украине же полным ходом идёт движение в НАТО.

\iusr{Владислав Дылов}
\textbf{Василий Александрович Волга} Крымчане, начали все сами. По НАТО, удачи и флаг в руки.

\iusr{Василий Александрович Волга}
\textbf{Владислав Дылов} а Одесситы не начали?

\iusr{Владислав Дылов}
\textbf{Василий Александрович Волга} В Одессе, бандерлоги опередили Русский мир. Картавый был в Одессе и руководил событиями. В Крым не пустили их, слава Богу. Поэтому, обошлось малой кровью.
\end{itemize} % }

\iusr{Раиса Стесенко}

Я понимаю вашу боль, Василий Александрович! Но кому, как не вам знать, что не
всегда говорят то, что думают. А тем более-раскрывают планы!!!

\begin{itemize} % {
\iusr{Василий Александрович Волга}
8 лет разговоров, Раиса, о некоем тайном Плане, а воз и ныне там. Две Великие Отечественные Войны уместились бы в это время. Думается мне, что если бы собиратели Земель Русских так сильно конспирировали бы свои планы, то никогда бы Россия и создана не была.

\iusr{Раиса Стесенко}
\textbf{Василий Александрович Волга}
Я и сама в глубоких раздумьях
\end{itemize} % }

\iusr{Леонид Зарапа}
Ответ здесь наверно один. Будем ждать нападения Украины на Россию, как в Грузии при Саакашвили. Тогда всё и решится.

\iusr{John Depp}
А можно Ставки ставить на эту новость? Вступит или нет

\iusr{Виктория Павликова}
Россия может только одно, создать уровень жизни лучше чем в Украине, а пока... шило на мыло.

\iusr{Елена Дигальцева}
Отдать нельзя удержать
Поставьте запятую

\end{itemize} % }
