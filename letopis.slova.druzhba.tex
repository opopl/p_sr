% vim: keymap=russian-jcukenwin
%%beginhead 
 
%%file slova.druzhba
%%parent slova
 
%%url 
 
%%author 
%%author_id 
%%author_url 
 
%%tags 
%%title 
 
%%endhead 
\chapter{Дружба}
\label{sec:slova.druzhba}

%%%cit
%%%cit_head
%%%cit_pic
%%%cit_text
У проросійському болоті велика заруба. Чубляться Портнов/Лукаш і Шарій. Ще
вчора вони були найліпшими \emph{друзями}, а сьогодні вже один одного тваринками
називають. Що ж сталося? Там всього намішано. І президентські амбіції
Портнова, і \enquote{туалетна історія} Шарія, і бажання їхніх учорашних спільних \emph{друзів}
вислужитись перед Зеленським, десь і Медведчук є на радарах
%%%cit_comment
%%%cit_title
\citTitle{Шарія кинули всі. Втече до Москви?}, Олексій Братущак, blogs.pravda.com.ua, 20.06.2021
%%%endcit

%%%cit
%%%cit_head
%%%cit_pic
\ifcmt
  tab_begin cols=2
		width 0.4
     pic https://strana.ua/img/forall/u/0/36/8870e29b30de20a65403b30c30b4b37560dc7956d45f7795881761.jpg
		width 0.46

     pic https://strana.ua/img/forall/u/0/36/2021-07-01_09h35_48.png
  tab_end
\fi
%%%cit_text
Подробности нападения. 29 июня на матч Украина-Швеция, который завершился
выходом украинской сборной в четвертьфинал Евро-2020, пришли не только
болельщики команд-соперниц, но и российский фанат Денис Смоленский вместе со
своими \emph{друзьями} - украинцем и поляком. Трое \emph{товарищей} выбрали
яркие образы, чтобы показать приверженность к разным странам, но в то же время
единство.  Россиянин выбрал одежду с российской символикой, которая отличала
его на \enquote{сине-желтом} секторе - он был в красной футболке сборной
России, с флагом РФ и в шапке-ушанке.  И, как видно по снимкам, в начале матча
украинские болельщики спокойно реагировали на болельщика, укутанного в триколор
%%%cit_comment
%%%cit_title
\citTitle{На Евро 2020 избили российского фаната Дениса Смоленского, что известно об инциденте}, 
Анна Копытько, strana.ua, 01.07.2021
%%%endcit

%%%cit
%%%cit_head
%%%cit_pic
\ifcmt
  pic https://strana.ua/img/forall/u/0/36/2021-07-03_14h11_09.png
	width 0.4
\fi
%%%cit_text
Главное, чтобы хорошо играли. Первые, кого мы встретили были две
\emph{подружки}, которые говорят на разных языках. Одна - на русском, другая -
на украинском. Они ответили, что для них вообще не принципиально, на каком
языке будут говорить на пресс-конференции украинские футболисты. Главное -
чтобы было понятно.  \enquote{Все равно, кому как удобно. Кто-то говорит на
украинском, кто-то - на русском.  Может, кто-то не знает украинский... А
вообще, вроде бы классно играли. Я видела в Инстаграме все истории были только
о футболе}, - получили мы первый ответ
%%%cit_comment
%%%cit_title
\citTitle{Что говорят украинцы о пресс-конференциях футболистов на русском языке. Опрос Страны}, 
Антонина Белоглазова, strana.ua, 03.07.2021
%%%endcit

%%%cit
%%%cit_head
%%%cit_pic
\ifcmt
  pic https://strana.ua/img/forall/u/0/36/2021-07-03_14h15_31.png
	width 0.4
\fi
%%%cit_text
Двое \emph{товарищей} также были солидарны с девушками и предложили в первую очередь
разобраться с теми, кто критикует украинских футболистов за язык.  При этом
ребята в повседневной жизни общаются на украинском языке.  \enquote{Как им удобно,
пусть так и говорят. В принципе много людей на русском языке говорят}, -
ответили нам. И добавили, что критику за язык в основном считают хайпом
%%%cit_comment
%%%cit_title
\citTitle{Что говорят украинцы о пресс-конференциях футболистов на русском языке. Опрос Страны}, 
Антонина Белоглазова, strana.ua, 03.07.2021
%%%endcit

%%%cit
%%%cit_head
%%%cit_pic
%%%cit_text
\enquote{Такі рішення стратегічно важливі, адже на сьомому році війни з Росією у нас
був пам'ятник \emph{дружби} з Москвою. Звісно, зараз у росіян і ватників бомбитиме,
про що свідчить навіть ейджизм і сексизм у депутата від ОПЖЗ, який таким чином
пробував зірвати розгляд цього рішення. Але такі рішення правильні і цього
декілька років безуспішно добивалась громада району}, - розповіла LB.ua
ініціаторка рішення, депутатка від фракції \enquote{Голос} Аліна Михайлова
%%%cit_comment
%%%cit_title
\citTitle{\enquote{Київрада вирішила демонтувати пам'ятник дружби Києва та Москви}}, 
Олександр Рудоманов, lb.ua, 08.07.2021
%%%endcit

%%%cit
%%%cit_head
%%%cit_pic
%%%cit_text
Одним из важнейших и ошибочных шагов украинской власти была денонсация Договора
о \emph{дружбе}, сотрудничестве и партнерстве между РФ и Украиной. Таким
образом, страна осталась фактически без работающего инструмента влияния и
юридически оформленных отношений с крупнейшим соседом.  Денонсация этого
договора в 2018 году укладывается в логику нагнетания антироссийской истерии,
милитаризации государственной политики. Однако для экономики Украины, для
позиции страны на международной арене, для интересов миллионов граждан такое
решение имело катастрофические последствия. Такой шаг свел на нет все усилия,
направленные на поиск компромиссов и мирное урегулирование конфликта на
Донбассе
%%%cit_comment
%%%cit_title
\citTitle{О будущем украинского и русского народов}, 
Виктор Медведчук, strana.ua, 15.07.2021
%%%endcit

%%%cit
%%%cit_head
%%%cit_pic
%%%cit_text
Доброго времени суток, \emph{друзья}. Украинцы – это русские говорят все,
начиная от Путина и заканчивая простым русским человеком. Трубят эту формулу
всюду, и я решил сказать свое мнение, которое имеет определенный алгоритм и
тональность.  Нет, украинцы – это не русские, иначе бы они называли себя
русскими и назывались бы Россией, а не Украиной. Да, на Украине много русских,
как раз коренные жители этой территории, остальные пришлые, именно укры от
слова окраина. Территория Украины – это Малороссия для России и окраина для
польско литовского княжества, поэтому укры не русские, это холопы польско
литовских князьков.  Проведу небольшое сравнение, чтобы понять разницу.
Чемпионат по футболу, Украина играет с Австрией, спрашивают российского
болельщика
%%%cit_comment
%%%cit_title
\citTitle{За что я люблю Малороссию (Украину)}, Вестник, zen.yandex.ru, 14.07.2021
%%%endcit
