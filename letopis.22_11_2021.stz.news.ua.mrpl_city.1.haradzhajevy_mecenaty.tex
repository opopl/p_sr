% vim: keymap=russian-jcukenwin
%%beginhead 
 
%%file 22_11_2021.stz.news.ua.mrpl_city.1.haradzhajevy_mecenaty
%%parent 22_11_2021
 
%%url https://mrpl.city/blogs/view/haradzhaevi-veliki-hlibotorgovtsi-i-grandiozni-metsenati
 
%%author_id demidko_olga.mariupol
%%date 
 
%%tags mariupol,mariupol.istoria
%%title Хараджаєви: великі хліботорговці і грандіозні меценати
 
%%endhead 
 
\subsection{Хараджаєви: великі хліботорговці і грандіозні меценати}
\label{sec:22_11_2021.stz.news.ua.mrpl_city.1.haradzhajevy_mecenaty}
 
\Purl{https://mrpl.city/blogs/view/haradzhaevi-veliki-hlibotorgovtsi-i-grandiozni-metsenati}
\ifcmt
 author_begin
   author_id demidko_olga.mariupol
 author_end
\fi

Продовжуючи цикл, присвячений видатним маріупольцям, які поховані на
найстарішому міському кладовищі, хочу розповісти про видатного діяча, який
першим був удостоєний звання \enquote{Почесний громадянин міста Маріуполя}. Це
Олександр Давидович Хараджаєв – міський голова Маріуполя (1860–1864), купець
першої гільдії і меценат. І сам Олександр Давидович, і його син, і інші члени
родини зробили унікальний внесок в розвиток торговельної справи, будівельної
галузі, освіти і медицини Маріуполя.

Життєвий шлях, роль і внесок Хараджаєвих у своїх розвідках у різний час
дослідували історики та краєзнавці С.Буров, Л. Яруцький, С.Новікова, В.Волонець
та інші.

Олександр Хараджаєв народився в Маріуполі 15(27) жовтня 1826 року у сім'ї
грецького купця \textbf{\emph{Давида Антоновича Хараджи}} (1775–1857). Завдяки участі у
торгівлі зерновим хлібом Давиду Антоновичу вдалося зміцнити добробут родини та
виділяти значні суми на благодійність. Займаючи посаду церковного старости, у
1838 році він пожертвував 5 000 руб. для будівництва собору Святого Харлампія –
найбільшого храму Маріуполя. А у 1852 році направив 5 150 крб. на спорудження
іконостасу головного храмового престолу.

\emph{\textbf{Олександр Давидович Хараджаєв}} зміг не тільки продовжити, а й примножити справу
батька.  Він заснував торгову контору із закупівлі зерна у селах Приазов'я та
продажу його за кордон і став власником понад десятка невеликих вітрильних
суден та кількох пароплавів. Серед них - 6 вітрильників каботажного та
закордонного плавання: шхуни \enquote{Агіос Георгіос}, \enquote{Десна}, \enquote{Мімі Мембелі}, \enquote{Святий
Антоній}, \enquote{Святий Миколай} та бот \enquote{Свята Надія}. Під його прапором ходив цілий
приватний торговий флот. О. Хараджаєв сприяв проведенню через Маріуполь
телеграфної лінії Ростов-на-Дону – Одеса для покращення умов торгівлі на
Азовському морі.
