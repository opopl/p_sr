% vim: keymap=russian-jcukenwin
%%beginhead 
 
%%file 10_11_2020.sites.ru.zen_yandex.yz.rusichi.1.beresta_gramoty_ukraina
%%parent 10_11_2020
 
%%url https://zen.yandex.ru/media/id/5e14d3e83d5f6900b6bfbb38/berestianye-gramoty-na-ukraine-5faaeb6c12611310853672c1
 
%%author Русичи (Яндекс Zen)
%%author_id yz.rusichi
%%author_url 
 
%%tags ukraina,beresta,gramota
%%title Берестяные грамоты на Украине
 
%%endhead 
 
\subsection{Берестяные грамоты на Украине}
\label{sec:10_11_2020.sites.ru.zen_yandex.yz.rusichi.1.beresta_gramoty_ukraina}
\Purl{https://zen.yandex.ru/media/id/5e14d3e83d5f6900b6bfbb38/berestianye-gramoty-na-ukraine-5faaeb6c12611310853672c1}
\ifcmt
	author_begin
   author_id yz.rusichi
	author_end
\fi

Все три берестяные грамоты с надписями, которые были найдены на территории
современной Украины, были обнаружены 1988-89 годах экспедицией под руководством
И.К. Свешникова на раскопках в Львовской области в селе на месте древнерусского
города Звенигорода.

Грамота №1 была найдена 26 июля 1988 года в культурном слое, датируемым
1110-1130-ми годами. Она состояла из трёх разорванных частей. Эти части были
составными большего документа, от которого осталась лишь её середина. На
оставшейся части читается небольшой текст.

\ifcmt
  pic https://avatars.mds.yandex.net/get-zen_doc/2431229/pub_5faaeb6c12611310853672c1_5faaec80126113108538581e/scale_2400
  caption Берестяная грамота из Звенигорода-Галицкого №1
\fi

Вот грамота Звен. №1. Нач. XII в.

…ънъ а мъне не [н]адобе сем[о] ч…

Перевод на современный язык: "… а мне не нужно сюда..."

Самое интересное в этой грамоте, что написано мънЕ, а не мънѢ. Наличие этого
эффекта в галицко-волынском документе (т. е. в зоне, где ъ никогда не совпадал
с е) - наглядное свидетельство того, что такая замена по крайней мере в
части случаев имела чисто графический характер.

И. К. Свешников предполагал, что «ънъ» - это последние буквы слова «гривънъ».

Грамоту №2 нашли 28 июля 1989 года на другой усадьбе того же места и с той же
датировкой культурного слоя, что и предыдущая грамота. В отличие от предыдущей,
это грамота является цельным документом. По мнению И. К. Свешникова, это
деловое письмо, начинающее без типичных вежливых слов для Новгорода. Оно было
отослано из неизвестного древнерусского города, а адресатом был житель
Звенигорода.

\ifcmt
  pic https://avatars.mds.yandex.net/get-zen_doc/2993437/pub_5faaeb6c12611310853672c1_5faaedf491ab8a65f78d3624/scale_2400
  caption Берестяная грамота из Звенигорода-Галицкого №2
\fi

Перевод на современный язык:

От Говеновой [вдовы] к Неженцу. Дай шестьдесят кун ладейных (т. е. за ладью или
на ладью). [Так] сказал Говен перед смертью, а поп записывал. Дай [их] Луке.
Если же не дашь, то я возьму у князя отрока и вместе [с ним] приеду  это
тебе станет в большую сумму.

Нежнич задолжал Говену 60 кун, после смерти Говена его вдова потребовала от
него вернуть долг через Луку. При нежелании отдавать вдова грозилась обратиться
к князю и вместе с его отроком вытребовать уже большую сумму. При такой
трактовке грамоту можно считать древнейшим известным нам древнерусским
завещанием. (Словосочетание ида на соудо в данном контексте следует понимать не
буквально, а в значении "умирая".)

Согласно аргументации И. К. Свешникова, вдова Говена могла жить в Перемышле, а
князем в этом случае выступает Володарь Ростиславич, перемышльский князь в
1092-1124 годах. Тогда письмо можно датировать между 1110 и 1124 годами.

Отрок - это должностное лицо. Письмо, очевидно, было доставлено адресату
Лукой, которому Говенова вдова поручила получить долг.

Об участии отрока и о возможной большой сумме для Неженца также можно сравнить
с текстом статьи №74 Пространной Правды, где регулируются расходы при судебном
разбирательстве на отрока, писца и перекладного.

Для загадочного слова прижь А.А. Зализняком высказана гипотеза о том, что это
при + ж<е>, где при выступает в роли наречия (а не предлога) со значением
"вместе с кем-то, "в чьем-то сопровождении.

Грамота № 3 найдена в 1989 году на месте постройки с теми же характеристиками,
как и две предыдущие грамоты. На этой грамоте изображены человек и буква «а». 

\ifcmt
  pic https://avatars.mds.yandex.net/get-zen_doc/3557661/pub_5faaeb6c12611310853672c1_5faaf01099041f23a4208e2f/scale_2400
  caption Берестяная грамота из Звенигорода-Галицкого №3.
	fig_env wrapfigure
	width 0.3
\fi

В данном месте были найдены вещи, в том числе деревянную часть от ведра, где
было написано имя «Іоанъ». И. К. Свешников полагает, что владелец этой
постройки Иван использовал берест для тренировки. Изображённый человек имеет
одежду напоминающую одежду священника.

В XXI веке на Украине была найдена береста без текста. В 2008 в Бужске тоже на
Львовщине и в 2010 году на киевском Подоле. В 1960-х в Киеве была найдена кора
с детскими рисунками.

При написании поста автор активно использовал материалы сайта \url{gramoty.ru}.
