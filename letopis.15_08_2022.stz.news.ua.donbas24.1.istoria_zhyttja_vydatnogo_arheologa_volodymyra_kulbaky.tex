% vim: keymap=russian-jcukenwin
%%beginhead 
 
%%file 15_08_2022.stz.news.ua.donbas24.1.istoria_zhyttja_vydatnogo_arheologa_volodymyra_kulbaky
%%parent 15_08_2022
 
%%url https://donbas24.news/news/istoriya-zittya-vidatnogo-arxeologa-volodimira-kulbaki
 
%%author_id demidko_olga.mariupol,news.ua.donbas24
%%date 
 
%%tags 
%%title Історія життя видатного археолога Володимира Кульбаки
 
%%endhead 
 
\subsection{Історія життя видатного археолога Володимира Кульбаки}
\label{sec:15_08_2022.stz.news.ua.donbas24.1.istoria_zhyttja_vydatnogo_arheologa_volodymyra_kulbaky}
 
\Purl{https://donbas24.news/news/istoriya-zittya-vidatnogo-arxeologa-volodimira-kulbaki}
\ifcmt
 author_begin
   author_id demidko_olga.mariupol,news.ua.donbas24
 author_end
\fi

\ii{15_08_2022.stz.news.ua.donbas24.1.istoria_zhyttja_vydatnogo_arheologa_volodymyra_kulbaky.pic.front}
\begin{center}
  \em\color{blue}\bfseries\Large
Щорічно 15 серпня відзначається професійне свято археологів України 
\end{center}

До 2008 року свято в Україні було неофіційним. Його було встановлено указом
Президента від 6 серпня 2008 року з метою підтримки розвитку вітчизняної
археології і дослідження та популяризації археологічної спадщини України. З
нагоди Дня археолога ми вирішили розповісти про життєвий шлях \textbf{Володимира
Кульбаки}, видатного вченого-археолога і талановитого викладача Маріупольського
державного університету.

% 1 - Життєвий шлях
\ii{15_08_2022.stz.news.ua.donbas24.1.istoria_zhyttja_vydatnogo_arheologa_volodymyra_kulbaky.1}

% 2 - Наукові здобутки вченого-археолога
\ii{15_08_2022.stz.news.ua.donbas24.1.istoria_zhyttja_vydatnogo_arheologa_volodymyra_kulbaky.2}

\subsubsection{Викладацька діяльність}

\ii{15_08_2022.stz.news.ua.donbas24.1.istoria_zhyttja_vydatnogo_arheologa_volodymyra_kulbaky.pic.3}

Завжди усміхнений, веселий, талановитий, енергійний і позитивний викладач...
Таким Володимир Костянтинович Кульбака запам'ятався студентам Маріупольського
державного університету. Він не тільки прищеплював інтерес до таємничої науки
археології, але й розумів молодь, тому завжди з легкістю знаходив спільну мову
зі студентами. Особисто для мене (а мені пощастило бути студенткою В. К.
Кульбаки) пари Володимира Костянтиновича були найулюбленішими, час летів
непомітно. У студентські роки моя група не усвідомлювала, як нам пощастило,
адже у нас викладав не просто цікавий лектор, а й видатний вчений, який значно
розширив знання про історію Приазов'я.

У 2009 році Володимир Костянтинович передчасно пішов з життя. Це стало
справжнім шоком не тільки для його сім'ї, але й для його колег, студентів. Всі,
хто знав цю Людину особисто, не міг повірити, що більше не почує цікавих лекцій
улюбленого викладача, що не буде з ним душевних розмов... Володимир Костянтинович
вже давно став душею історичного факультету Маріупольського державного
університету. Учень і послідовник вченого, доцент кафедри історичних дисциплін
МДУ \textbf{В'ячеслав Олегович Забавін}, на щастя, продовжив справу Володимира
Костянтиновича.

\begin{leftbar}
\emph{\enquote{Важко переоцінити внесок Володимира Костянтиновича у світову науку.
Захопленим своєю справою, талановитим, відкритим і веселим ми запам'ятали його
назавжди}}, — наголосив В'ячеслав Забавін.
\end{leftbar}

До повномасштабного вторгнення росії в Україну археологічна експедиція
Маріупольського державного університету на чолі з В'ячеславом Олеговичем
Забавіним завдяки щорічним розкопкам робила нові наукові відкриття.
Сподіваємося, що після деокупації Маріуполя В'ячеслав Забавін продовжить справу
свого вчителя.

Нагадаємо, раніше Донбас24 розповідав про \href{https://donbas24.news/news/azovstal-v-malyunkax-xudoznikiv-i-ilyustratoriv-z-usyogo-svitu-foto}{\enquote{Азовсталь} в малюнках художників і
ілюстраторів з усього світу}%
\footnote{\enquote{Азовсталь} в малюнках художників і ілюстраторів з усього світу, Алевтина Швецова, donbas24.news, 13.08.2022, \par\url{https://donbas24.news/news/azovstal-v-malyunkax-xudoznikiv-i-ilyustratoriv-z-usyogo-svitu-foto}}
та \href{https://donbas24.news/news/medbrat-z-mariupolya-shhodnya-mi-priimali-do-150-poranenix-pomirali-diti-foto}{історію медпрацівника обласної лікарні Маріуполя}.%
\footnote{Медбрат з Маріуполя: \enquote{Щодня ми приймали до 150 поранених, помирали діти}, Яна Іванова, donbas24.news, 13.08.2022, \par\url{https://donbas24.news/news/medbrat-z-mariupolya-shhodnya-mi-priimali-do-150-poranenix-pomirali-diti-foto}}

ФОТО: з відкритих джерел.

\ii{insert.author.demidko_olga}
%\ii{15_08_2022.stz.news.ua.donbas24.1.istoria_zhyttja_vydatnogo_arheologa_volodymyra_kulbaky.txt}
