% vim: keymap=russian-jcukenwin
%%beginhead 
 
%%file slova.monarh
%%parent slova
 
%%url 
 
%%author 
%%author_id 
%%author_url 
 
%%tags 
%%title 
 
%%endhead 
\chapter{Монарх}
\label{sec:slova.monarh}

%%%cit
%%%cit_head
%%%cit_pic
%%%cit_text
З часів \emph{монархії} сприйняття влади у Росії незмінне: вона є уособленням
сакрального. «Помазанник Божий» цар Олександр відрізняється від Сталіна тільки
тим, що перед ним несли хоругви з Христом, тоді як перед Сталіним – зі самим
Сталіним. Парадоксально, але наскрізь релігійна Російська імперія в цьому сенсі
була більш безбожна, ніж Радянський Союз. Перша сакралізувала владу царя,
помазаного Богом, – другий зробив Богом самого «царя».  Якщо вдатися до алюзії
марксистського «товар – гроші –товар», то тут вийде «Бог – цар – Бог». В цьому
сенсі російський народ дійсно є продовженням свої влади, яка – за діалектичним
принципом – є наслідком духовних потреб росіян. Незалежно від того, хто
очільник Росії – Романови, Хрущов або Путін – шапки народу завжди лежать на
землі.  В Україні ж ними закидують владу
%%%cit_comment
%%%cit_title
\citTitle{Відповідь Путіну. У чому різниця між українцями і росіянами?}, 
Станіслав Асєєв, www.radiosvoboda.org, 04.08.2021
%%%endcit
