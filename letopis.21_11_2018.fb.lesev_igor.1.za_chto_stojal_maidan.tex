% vim: keymap=russian-jcukenwin
%%beginhead 
 
%%file 21_11_2018.fb.lesev_igor.1.za_chto_stojal_maidan
%%parent 21_11_2018
 
%%url https://www.facebook.com/permalink.php?story_fbid=2181523205212104&id=100000633379839
 
%%author_id lesev_igor
%%date 
 
%%tags 2014,cennosti,maidan2,russofobia,ukraina
%%title За что стоял Майдан?
 
%%endhead 
 
\subsection{За что стоял Майдан?}
\label{sec:21_11_2018.fb.lesev_igor.1.za_chto_stojal_maidan}
 
\Purl{https://www.facebook.com/permalink.php?story_fbid=2181523205212104&id=100000633379839}
\ifcmt
 author_begin
   author_id lesev_igor
 author_end
\fi

За что стоял Майдан?

Пять лет спустя ватный и славаукраинский сегмент ФБ люто спорит, за что же
стоял Майдан. Первые злорадствуют в стиле симпатиков разбитого белого движения,
по каким-то причинам оставшимся в Союзе. «Ну мы же говорили, вы тупое
неотесанное быдло, развалившее чудесную страну и строящее абсолютно
некомфортное общество».

\ifcmt
  ig https://scontent-frx5-1.xx.fbcdn.net/v/t1.6435-9/46486352_2181523131878778_7007488358901350400_n.jpg?_nc_cat=111&ccb=1-5&_nc_sid=730e14&_nc_ohc=Asl4MKgXPj8AX_bPqwx&_nc_ht=scontent-frx5-1.xx&oh=4bc79ef95c1c08e0be3255976bdbe784&oe=61B70BAD
  @width 0.4
  %@wrap \parpic[r]
  @wrap \InsertBoxR{0}
\fi

Вторые рассуждают, что их чуть-чуть обманули. «Стояли мы за все хорошее, но
почему вышло все именно так, хер поймешь». Но эти же обязательно добавляют – «И
все-таки Майдан был не зря».

Майдан, действительно, был не зря, хотя бы потому что он добился своей базовой
задачи. Майдан обозначил свою стаю. Провел селекцию. Потому что Майдан – это
таки революция. Но не какого-то там «достоинства» (революционеры не могут
обходиться без пафоса). Майдан – это Великая русофобская революция. И, к слову,
не без стараний Москвы.

Русофобия первого Майдана была завуалирована под пересчет голосов. Но пересчет
лидера своей стаи. И пересчет только таким образом, когда любой результат кроме
победы, не признавался.

Русофобия второго Майдана особо уже даже не маскировалась. Все только
марафетилось пудрой. Но иррационально свои стали вместе со своими, а чужаки –
пусть и сограждане одной страны – не могли понять, почему люди с профессорскими
корочками и успешно работающим бизнесом стоят в одной шеренге с малолетними
мародерами и открытыми неонацистами.

Но «-фобия», как и «-филия» не зависят от социального статуса, образования или
финансового положения. Это как сексуальное влечение. Ты хочешь трахать или
девушку, или парня. Или ребенка. Или гуся. А кто ты – директор фирмы или
случайный прохожий, идущий мимо этой фирмы – на твое влечение никак не влияет.

Поэтому Майдан очень органичен тем задачам, которые подспудно миллионы его
участников для себя поставили. Ограничение/искоренение всего русского среди
своих соотечественников. Их фильтрация. Маргинализация. А самых непробивных –
вынуждение к миграции.

Другое дело, что сами майдановцы ведь не знали, что они собой представляют без
ру-составляющей в украинском сообществе. А они оказались не большевиками 20-х,
а совками 80-х. Строить расово правильное этническое общество, безусловно,
можно. Его можно строить даже без рабочей экономики и с коррупционной
составляющей. Даже нищета – барьер, но не принципиальный.

Но в постмайданной стерильно русофобской Украине нет эстетики для своих же
победителей. Построенное общество некрасиво. Некомфортно. И непрактично.
Разгромив русскость в Украине, победители сами массово валят из Украины без
ру-составляющей.

Все, конечно же, не свалят. Будут новые циклы. Новые метания. И новые
потемкинские деревни. Но в целом, Великая русофобская революция в Украине
состоялась успешно. Майдановцы построили именно ту машину, которую и хотели 5
лет назад. Да, она не ездит. Салон не отапливается. Но зато работает радио на
правильном языке. Разве это не повод сегодня праздновать победу?

\ii{21_11_2018.fb.lesev_igor.1.za_chto_stojal_maidan.cmt}
