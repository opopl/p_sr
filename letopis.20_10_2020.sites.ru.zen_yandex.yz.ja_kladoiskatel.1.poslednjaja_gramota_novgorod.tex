% vim: keymap=russian-jcukenwin
%%beginhead 
 
%%file 20_10_2020.sites.ru.zen_yandex.yz.ja_kladoiskatel.1.poslednjaja_gramota_novgorod
%%parent 20_10_2020
 
%%url https://zen.yandex.ru/media/yakladoiskatel/posledniaia-naidennaia-novgorodskaia-berestianaia-gramota-otkryla-informaciiu-o-kotoroi-ranee-izvestno-ne-bylo-5f84a5ebcae5a83b554a41d9
 
%%author Я Кладоискатель (Яндекс Zen)
%%author_id yz.ja_kladoiskatel
%%author_url 
 
%%tags novgorod,beresta,russia
%%title Последняя найденная Новгородская берестяная грамота открыла информацию, о которой ранее известно не было
 
%%endhead 
 
\subsection{Последняя найденная Новгородская берестяная грамота открыла информацию, о которой ранее известно не было}
\label{sec:20_10_2020.sites.ru.zen_yandex.yz.ja_kladoiskatel.1.poslednjaja_gramota_novgorod}
\Purl{https://zen.yandex.ru/media/yakladoiskatel/posledniaia-naidennaia-novgorodskaia-berestianaia-gramota-otkryla-informaciiu-o-kotoroi-ranee-izvestno-ne-bylo-5f84a5ebcae5a83b554a41d9}
\ifcmt
	author_begin
   author_id yz.ja_kladoiskatel
	author_end
\fi
\verb|20 October 171k full reads 1,5 min|
\index[cities.rus]{Новгород, Великий!Берестяная грамота, последняя найденная, 20.10.2020}
\index[rus]{Русь!История!Берестяные грамоты}

На данный момент в Новгороде найдено более 800 берестяных грамот. Но сегодня
пойдет речь о послании, найденном во время последней археологической
экспедиции. В обнаруженной грамоте исследователи прочитали информацию, о
которой ранее известно не было.

Берестяные грамоты, это новый вид исторических источников, о котором до
середины 20го века ученые даже и не подозревали. Они являются средневековыми
письмами. Это был дешёвый материал и на нем легко можно было нацарапать
послание.

В начале осени 2020 года археологи обнаружили удивительную находку во время
проведения раскопок на Софийской стороне города Великий Новгород. Ею оказалась
берестяная грамота, вернее её фрагмент, который датируется XIV веком.

Изучая новгородский культурный слой, археологи пришли к выводу, что в
средневековье в этой местности часто случались крупные пожары, так как дома, в
большинстве случаев, строили из дерева. В одном слое археологи зафиксировали 3
разрушительных пожара.

\ifcmt
pic https://avatars.mds.yandex.net/get-zen_doc/1852544/pub_5f84a5ebcae5a83b554a41d9_5f84a6088e35355ad19414c8/scale_2400
caption https://www.pomorie.ru/2020/10/01/5f75e7fc92a11b328d183fe2.html
\fi

Перед застройкой жилого дома в современном Новгороде, археологи провели
исследования участка, где ранее находилась центральная часть старого Новгорода.
Именно здесь и был обнаружен остаток берестяной грамоты, у которой, к
сожалению, сохранилась только нижняя часть текста (около 2х строк). Что было
выше остается только догадываться.

\ifcmt
  pic https://avatars.mds.yandex.net/get-zen_doc/1875669/pub_5f84a5ebcae5a83b554a41d9_5f84a6d1a144c35a2714c516/scale_1200
  caption Берестяная грамота, фото: Институт археологии РАН
\fi

В ней упоминается о рати, которая идет через Заволочье или Волок. Автор задает
вопрос: «Чи пойдёт рать за Волок» и просит осведомить его, чего ожидать от
этого военного отряда. Эта находка уникальна тем, что из документов редко
удается получить сведения о ходе каких-либо военных операций. Да и вообще в
источниках очень мало сведений о Заволочье.

Этот фрагмент наводит на мысли о каком-то военном конфликте в те времена.

\ifcmt
  pic https://avatars.mds.yandex.net/get-zen_doc/3985451/pub_5f84a5ebcae5a83b554a41d9_5f84a722cae5a83b554cb428/scale_2400
  caption https://www.pomorie.ru/2020/10/01/5f75e7fc92a11b328d183fe2.html
\fi

Рядом с фрагментом была найдена печать новгородского посадника, который занимал
эту должность в 1354-1380 годах, это и помогло датировать документ. Археологи и
сейчас находят много интересных артефактов XIV-XVI веков, среди которых чаще
всего встречаются нательные кресты и различные украшения. Все эти предметы
говорят о достойном социальном статусе жителей, которые, возможно, были
торговцами или занимали высокие посты.

Археологи зафиксировали уже более 800 берестяных грамот X-XIV вв. Написаны они
представителями всех социальных классов, даже низших сословий, что говорит о
достаточно высоком уровне грамотности.

Из таких, казалось бы, незначительных находок и строится наша история – история
России.

Загляните в наш блог)

\begin{itemize}

\item \cusr{Иван Корн}

Автор несколько поскромничал относительно "достаточно высокого уровня
грамотности". Грамотность была всеобщей. Известны грамоты,
написанные детьми. В раннем Средневековье! В то время, когда в
Европе редкий король умел сносно читать, а писать и того реже.
Более того, филологами установлено, что в текстах грамот (а их
найдено около 1200), нет ни единой грамматической ошибки! Ни
единой! При том, что это не официальные документы, а бытовая
переписка горожан, такое средневековое СМС. Проверьте свои
СМС-ки, много ошибок делаете? Вот то-то. Так что, эти скромные
артефакты - свидетельства невероятно высокого культурного
уровня Древней Руси.

\item \cusr{Отец Этлау}

Иван Корн, да, а потом пришли Романовы и сначала всех сделали безграмотными и
загнали в кабалу, а потом переписали историю под себя. Только
благодаря кровавым совкам и их наследию у нас есть на
сегодняшний момент оставшиеся крохи образования, науки и
медицины... Но ничего, ещё пару реформ и будет как при
Романовых.

\item \cusr{Андрей Виноградов}

Иван Корн,Так уж и всеобщей? Города где найдены берестяные грамоты можно
пересчитать по пальцам. В основном это северо-запад,частично
северо-восток. Основная масса в Новгороде ,потом в городах ,где
новгородцы имели торговые связи Псков16 шт),Старая Русса(51
шт), Смоленск(16) ,Псков (8)Торжок(19) ,Тверь(9) Москва,Рязань
и тд единичные находки.Всё. То есть берестяные грамоты это
чисто новгородский феномен и общерусского значения не имел.Да и
сами новгородцы судя по тем же грамотам русскими себя не особо
считали. В одной грамоте так и написано-"Еду во Владимир в Русь
из Новгорода". или в новгородской летописи-Въ лѣто 6657 [1149].
Иде архепископъ новъгородьскыи Нифонтъ въ Русь, позванъ
Изяславомь и Климомь митрополитомь: ставилъ бо его бяше
Изяслав. И таких примеров можно привести далеко не один.Все они
говорят,что сами новгородцы Новгород Русью не считали и так
почти до 15 века.

\iusr{Павел К.}

Отец Этлау, не надо клеветать на Романовых. Все произошло задолго до них.
Новгородская республика, отличавшаяся высоким уровнем
грамотности, была уничтожена московским князем Иваном III, а
затем повторно вырезана уже царем, Иваном Грозным (новгородский
погром). Эти разрушители русской культуры были не Романовы, а
Рюриковичи.

\iusr{Павел К.}

Андрей, что значит "новгородцы не считали себя Русью"? Да именно они начтоящей
русью и были. Русское государство было создано Рюриком в 862 г.
в городе Ладога (сейчас это Ленинградская обл.), а затем Рюрик
переместился в Новгород. Это и есть исконная Русь. Другие
территории, включая Киев, были завоеваны позже.



\iusr{Андрей Виноградов}

Павел К., То и значит.Я вам привёл два примера,где прямо так и написано-Из Новгорода в Русь. Хотите поищите ещё примеры ,уверен найдёте. Новгородцы считали себя отдельным народом.Мало того при анализе берестяных грамот лингвисты ,в частности Зализняк делают вполне обоснованные выводы ,что к 14-15 векам в Новгороде складывался отдельный язык,родственный русскому,но самостоятельный.И не присоедини Иван 3 Новгород к Москве сейчас бы существовало не три(русский,украинский и белорусский) а четыре восточнославянских языка.Да кстати когда было летописное призвание Рюрика никаких русских вообще не существовало как единого народа а были племенные союзы полян,древлян,радимичей ,и т.д. Варягов призывали четыре племенных союза словене( безусловные славяне),кривичи(до сих пор не определили в какой степени они славяне) чудь и меря(не славяне,финно-угры). И никаких русских.

\iusr{Сола Камели}

Андрей, новгородцы действительно были несколько в стороне от Руси, что
Внутренней, что Внешней, если вспомнить деление её Константином Багрянородным.
Но "во Владимир в Русь" должно подсказать Вам, где была подлинная Росия,
известная и грекам, и арабам.

\iusr{Павел К.}

Андрей, Вы невнимательно читали ПВЛ. Племена призвали не просто варягов, а
варягов-русь. Причем во-первых, ясно указано, что русь - это название народа, а
во вторых, что русь и до этого была там, собирая дань с окрестных племен. То
есть русь сначала изгнали на время, а потом уже призвали обратно. Русские - это
и есть союз пяти племен: руси, словен, кривичей, чуди и веси (иногда мерю
добавляют), названный по главному народу.

Насчет завоевания Новгорода Москвой. Эта политика привела не к объединению Руси
(как утверждают московские историки), а к смутному времени, и полному развалу
русского государства. Русь была объединена уже союзом ополчений - Минина и
Пожарского, Ляпунова, и прочих - рязанских, нижегородских и т.д. После чего и
был избран царь. Надеюсь вы знаете, что Москва возвысилась над русскими
городами не как русский город, а как ордынский - там был центр сбора дани с
русских городов. Поэтому "родственный русскому язык" мог возникнуть только в
Москве, но никак не в исконно русском Новгороде. А все обвинения в адрес
Новгорода - не более чем пропаганда Московского княжества. Представьте себе,
пропаганда в то время уже существовала.

Ну и насчет ученых, которые делают глубокомысленные выводы на основании пары
грамот. Я недавно читал у одного из украинцев, что жители Львова считают
киевлян москалями, а жители более западных мест, считают москалями самих
львовян. Вот примерно такой же уровень оценок по бытовым записям тех времен,
как и по современным постам. Для диссертации и создания научной школы годится,
но действительность не отражают.

\iusr{Андрей Виноградов}

Павел К., Ну и чушь вы нагородили. Во первых русь указывается в списке .И
список этот не содержит ни одного славянского.В основном это германцы. Во
вторых - русь это не племя,а профессия и слово имеет явно финское
происхождение.Все остальные англы, фряги, и т.д. и только варяги-русь. Именно
так финны называли шведов. Роутси-гребцы. Кстати именно так они шведов называют
по сей день роутсилайнен,а нас по сей день вения .В третьих Язык менялся и в
раннем древнерусском наречии не было слова русь вообще.Это слово псалось через
букву "от"(сейчас не используется) и читалось как Руотсь или Роутси.

\iusr{Андрей Виноградов}

Павел К., Я уж не говорю об отрывке с описанием прихода "послов от кагана
русов" к византийцам,где все русские почему-то оказываются германцами
-скандинавами."Мы от рода русского — Карлы, Инегелд, Фарлаф, Веремуд, Рулав,
Гуды, Руалд, Кари, Фрелав, Руар, Актеву, Труан, Лидул, Фост, Стемид— посланные
от Олега, великого князя русского..."

\iusr{Константин}

Андрей, где в словосочетании "Еду во Владимир в Русь из Новгорода" есть хотя бы
намёк на "разные народы"? Это как прочитав "еду в Сибирь, в Новосибирск"
считать, что Сибирь - это народ. Вы и другого бреда понаписали, ещё и слова
воспитанного, неглупого и явно образованного человека (Павел К) чушью
называете. Дождётесь, насуют и опустят вас в этом чятике :). Лучше исчезните,
пока не поздно :)

\iusr{Lora Shurpik}

Андрей, а в голову не приходит , что русы - это и есть союз племен, и язык
русский - это продукт смешения племенных языков.

А уж украинцев не было и тогда, когда Русь уже сложилась. Малая Русь, окраинная
- была.

\iusr{станислав}

Андрей, большинство так называемых летописей были сфабрикованы в основном при
Софии Августе Фредерике Ангальт-Цербстской, хотя повесть временных лет была
состряпана в городе кёнигсберг к приезду петра. все остальное это списки со
списков с дополнениями и исправлениями так что не надо лохматить бабушку.
ученые не в курсе где на самом деле был Великий Новгород, ну так на минуточку.

Павел К., а вы за Софию Августу Фредерику, принцессу Ангальт-Цербстскую не
хотите рассказать, убила мужа захватила престол и самые страшные годы
крепостного права были именно при ней. так что не гони волну они и романовыми
не были никогда, от слова совсем

\iusr{Андрей Виноградов}

Lora Shurpik, А в голову не приходит ,что свои бредовые предположения надо
подтверждать ссылкой на источники? А украинцы тут ни при чём.Как впрочем и
русские с белорусами. Никого из них ещё не было века до14-15.Была древнерусская
народность и то не сразу,а постепенно складывалась из племенных союзов не одно
столетие.

\iusr{Александр}

Иван Корн, за 300 лет при всеобщей грамотности новгородцы должны были написать
НЕСКОЛЬКО МИЛЛИОНОВ берестяных грамот. Найдено 1200. Допускаю, что нашли лишь
один процент, но это 120 тыс. Т.е по 400 грамот каждый год на 40-50 тыс
населения. Да, возможно каждый тридцатый был грамотным или даже каждый десятый
в 15 веке ( наибольший расцвет Новгорода), но не больше.

Для тех времен это прилично. В европейских городах примерно также. Ведь Франсуа
Вийон в 14 веке не только для дворян писал.

\iusr{Олег Ермаков}

Павел К., Да ладно! По какому же недомыслию в Правде Ярославовой в пункте
первом читаем -

1. Оубьеть моужь моужа, то мьстить братоу брата, или сынови отца, любо отцю
сына, или братоучадоу (а), любо сестриноу сынови; аще не боудеть кто мьстя, то
40 гривенъ за голову; аще боудеть роусинъ, любо гридинъ, любо коупчина (б),
любо ябетникъ, любо мечникъ, аще (в) изъгои боудеть, любо словенинъ, то 40
гривенъ положити за нь.

Русин ниразу не словенин выходит! О какЪ!


\iusr{Евгений Гришуков}

Отец Этлау, Это точно! По нем образование это,, аксиомы,, договорных величин..
Что по сути не какого отношения к настоящему образованию не имеет . Но, есть
ещё кое-что, обратная сторона этой медали, которую почему-то считают
образованием, это суждения обыкновенных невежд, кто во что горазд, в меру своей
некомпетентности... Вы правы, образованные бывают совершенно разные, то есть
это проблема в нас, способность отождемтаить, понять, определить факты, а не
проблема окружающего нас Мира! Интереснв е которые комментарии, о Руси, России
, один плюс, в том что задумались о том что собственно из себя представляет
Русь, Россия, Русская Цивилизация!Судя по комментариям мы не знаем в какой
Стране Живем!? Но тем не менее, где то на глубинах сознания осознаем,
сокральную суть России!

\iusr{Владимир Владимиров}

Андрей, Да, да, да, русских не было, но именно праславяне создали тот
праиндоевропейский язык, отца всех индоевропейских языков, что был суть
праславянский, о чем уже начинают говорить даже западные лингвисты, как
например П. Фридрих,утверждающий ,что санскрит родился на территории
праславяне, что подтвердила новейшая наука днк-генеалогия,доказав ,что
арийский,корневой субклад z645 R1a имел местом своего зарождения центральную
часть современной РФ.

Славяне ведут свою историю с корневого примерно субклада z645(z283)

\ifcmt
  pic https://avatars.mds.yandex.net/get-zen_pictures/3403730/1093869923-1603251844160/orig
\fi

\iusr{Владимир Владимиров}

Андрей, Вандалы- славяне,о чем существует множество свидетельств

Вильгельма Рубрука из книги \enquote{Путешествие в восточные страны} (13 век):

Язык русских, поляков, чехов (Воеmorum) и славян один и тот же с языком
вандалов, отряд которых всех вместе был с гуннами

Так, к примеру, Блонд говорит, что вандалы названные так по имени реки Вандал,
впоследствии стали называться славянами. Иоанн Великий Готский пишет, что
вандалы и славяне одной нации и отличаются только по названию. М. Адам во II
кн. «Истории церкви» говорит, что славяне — это те, кто прежде назывались
вандалы.

Его соотечественник, писатель XII в. Гельмольд в полном согласии с ним говорит,
что славян в древности называли вандалами, а в его время - винитами, или
винулами.

Фламандский монах Рубрук писал в 1253 г., что «язык русинов, поляков, богемов
(чехов. - С. Ц.) и славян тот же, что и у вандалов».

Также и уроженец словенской Каринтии Сигизмунд Герберштейн (первая половина XVI
в.) утверждает, что в период своего могущества вандалы «употребляли... русский
язык и имели русские обычаи и религию». Далее он поясняет, что немцы именуют
всех славян «виндами, вюндами и виндитами, производя их имена от одних только
вандалов».

Об этом же пишет, ссылаясь на не дошедшую до нас «Историю вандалов» Альберта
Кранция, хорватский просветитель из Далмации Мавро Орбини (XVII в.): «Вандалы
имели не одно, а несколько различных названий, а именно: вандалы, венеды,
венды, генеты, венеты, виниты, славяне и, наконец, валы». Для подкрепления
своего утверждения о тождестве вандалов и славян он приводит выдержки из
вандало-славянского словаря Карла Вагрийского, свидетельствующие о языковой
близости этих двух народов.

Схожее наблюдение принадлежит географу XVI в. Меркатору, который заметил о
языке населения острова Рюген, что у них в ходу «славянский да виндальский»
языки.

Этно-языковое родство вандалов и славян утверждается также во многих
средневековых русских источниках и славянском фольклоре - в частности, об этом
говорит легенда о старейшине Словене и его сыне Вандале.

\iusr{Геннадий Н. Б.}

Берестяные записки говорят о высоком проценте грамотности русского населения
тех времён и, естественно, о глубоких древних культурных его корнях,
существовавших со времён языческой Руси, которое, пришедшее ему на смену
христианство, уничтожило.

Самый действенный способ уничтожения реализовался с переходом письменности от
глаголицы к кириллице. Именно это привело к потере всей древнейшей
многотысячной истории Руси и утраты поголовной грамотности её населения.
Множество сказаний, мифов и былин Древней Руси христианская кириллица отвергла
и, естественно, нигде не зафиксировала с применением новой письменности,
заменив их древнесемитскими сказками и мифами христиан. Вот таким простым
способом лишили нас великого исторического прошлого!

Сейчас подобное происходит в бывших советских республиках, но уже с заменой
русского алфавита на латинский, ставя крест на всём советском прошлом, потому
что никто из них не будет переводить на латиницу все старые тексты, а только
то, что укладываются в их современную национальную парадигму. Так что можно с
уверенность ожидать - все они скоро перейдут из орбиты влияния России, на
орбиту американцев, да и у наших, которые исподволь поднимают подобный вопрос,
надо думать, тоже такие задумки есть!

\iusr{валентина полякова}

Самым большим и охраняемым секретом России - является её реальная история!

КТО СОЗДАВАЛ НАМ ИСТОРИЮ?

Сейчас мы последовательно перечислим ВСЕХ АКАДЕМИКОВ-ИСТОРИКОВ Российской Академии наук, как иностранцев, так и Отечественных, начиная от её основания в 1724 году вплоть до 1918 года. (справочное издание, книга 1) Мы приводим также год избрания.

1) Коль Петр или Иоганн Петер (Kohl Johann Peter), 1725,

2) Миллер или Мюллер Федор Иванович или Герард Фридрих (Mu»ller Gerard Friedrich), 1725,

3) Байер Готлиб или Теофил Зигфрид (Bayer Gottlieb или Theophil Siegfried), 1725,

4) Фишер Иоганн Эбергард (Fischer Johann Eberhard), 1732,

5) Крамер Адольф Бернгард (Cramer Adolf Bernhard), 1732,

6) Лоттер Иоганн Георг (Lotter Johann Georg), 1733,

7) Леруа Людовик или Пьер-Луи (Le Roy Pierre-Louis), 1735,

8) Мерлинг Георг (Moerling или Mo»rling Georg), 1736,

9) Брем или Брэме Иоганн Фридрих (Brehm или Brehme Johann Friedrich), 1737,

10) Тауберт Иван Иванович или Иоганн Каспар (Taubert Johann Caspar), 1738,

11) Крузиус Христиан Готфрид (Crusius Christian Gottfried), 1740,

12) ЛОМОНОСОВ МИХАИЛ ВАСИЛЬЕВИЧ, 1742,

13) Модерах Карл Фридрих (Moderach Karl Friedrich), 1749,

14) Шлецер Август Людвиг (Schlo»zer Auguste Ludwig), 1762,

15) Стриттер или Штриттер Иван Михайлович или Иоганн Готгильф (Stritter Johann Gotthilf), 1779,

16) Гакман Иоганн Фридрих (Hackmann Johann Friedrich), 1782,

17) Буссе Фомич или Иоганн Генрих (Busse Johann Heinrich), 1795,

18) Вовилье Жан-Франсуа (Vauvilliers Jean-Francois), 1798,

19) Клапрот Генрих Юлий или Юлиус (Klaproth Heinrich Julius), 1804,

20) Герман Карл Федорович или Карл Готлоб Мельхиор или Карл Теодор (Hermann Karl Gottlob Melchior или Karl Theodore), 1805,

21) Круг Филипп Иванович или Иоганн Филипп (Krug Johann Philipp), 1805,

22) Лерберг Август или Аарон Христиан (Lehrberg August Christian), 1807, 23) Келер Егор Егорович или Генрих Карл Эрнст (Ko»ler Heinrich Karl Ernst), 1817,

24) Френ Христиан Данилович или Христиан Мартин (Fra»hn Christian Martin), 1817,

25) ЯРЦОВ ЯНУАРИЙ ОСИПОВИЧ , 1818,

26) Грефе Федор Богданович или Христиан Фридрих (Gra»fe Christian Friedrich), 1820,

27) Шмидт Яков Иванович или Исаак Якоб (Schmidt Isaac Jacob), 1829, 28) Шенгрен Андрей Михайлович или Иоганн Андреас (Sjo»rgen Johann Andreas), 1829,

29) Шармуа Франц Францевич или Франсуа-Бернар (Charmoy Francois-Bernard), 1832,

30) Флейшер Генрих Леберехт (Fleischer Heinrich Lebrecht), 1835,

31) Ленц Роберт Христианович (Lenz Robert Christian), 1835,

И т.д...

\iusr{Тата Мир}

Ндяяя !У некоторых диванность в полной форме !

Кирик Новгородец (нач. XIIв.) в своих "Вопрошаниях" к новгородскому епископу
Нифонту : " Нъсть ли въ томь гръха, аже по грамотамъ ходити ногами аже кто
изрЪзавъ помечеть, а слова будуть знати? "

Тем , кто топит за исключительно новгородский феномен (?) рекомендую
поинтересоваться не только грамотами (малое количество в других др.городах
объясняется ТМН , составом почвы , застройкой местности и т.д. ), но и
средневековыми граффИти на стенах церквей , и таки , не только Вел.Новгорода .

//...Зализняк делают вполне обоснованные выводы ,что к 14-15 векам в Новгороде
складывался отдельный язык,родственный русскому,но самостоятельный...//-
глупость несусветная от А.Виноградова !Зализняк и т.д. пишут не об отдельном
языке , а о новгородском ДИАЛЕКТЕ , причём как раз таки к XIV-XV в.в. различия
в диалекте от "стандартного" древнерусского становятся МЕНЕЕ заметны , чем в
Xl-Xllв.в.

Судя по некоторым , скажем так мягко - сепаратистским глупостям об
исключительности Вел.Новгорода и плохих "москалях" , кто-то начитался
нью-хронолоджи от ВКЛ ? Вопрос риторический !

\iusr{riktiki r}

Уже Великие Шумеры набежали. 100 лет отроду с папой Лениным но память уже
девичья. Ничерта не помнят. Все путают. Доказывать будут что их Куев или Х..в
незакрытый пуп земли и Роcсия не Россия, и Русь ненастоящая, и до Куева, и
письменности не было:)

\iusr{валерий чирков}

riktiki r, верно, но вы пишите о хохлах а не о шумерах. Паны хохлы к шумерам
никакого отношения не имеют, к хазарам да!

\iusr{riktiki r}

валерий, Ну их давно в шутку так зовут. Да и к хазарам навряд ли они относятся.
Ту степь монголо-татары зачистили качественно. Земли заселялись заново из
глубинки России.

«… а всех юродивых и убогих ссылать на Окраину, там им дуракам место»

Иван Грозный

\iusr{Руслан Солнышко}

Здравствуйте. Извините за беспокойство. Хочу поделиться замечательным городом.
Город Чухлома - это небольшой, но очень провинциальный городок Костромской
области. В нашем уютном городке проживают чуть более пяти тысяч человек. Его
называют глубинкой Костромского края. В Чухломе есть красивая Успенская
церковь. Она украшает город и работает практически каждый день, кроме четверга.
Ещё в городе есть восхитительный парк, который находится недалеко от озера. В
этом удивительном парке растут красивые цветы: тюльпаны, нарциссы, красные и
жёлтые розы, за которыми очень хорошо ухаживают садоводы. Ещё есть сказочное
озеро - это замечательно место для купания жителей города и всего Чухломского
района. В Чухлому приезжают люди за десятки и сотни километров, чтобы погулять
по улицам и полюбоваться красивыми домиками. У нас ведь много крутых домов с
изумительными наличниками на окнах. А недавно, лет пять назад, появился большой
народный праздник - Пуговка. Чухломичи, а также все приезжие, желающие
насладиться этим прекрасным праздником, приезжают посмотреть его. Также
работает кинотеатр, где показывают 3D и даже 5D фильмы. Ещё в городе есть
несколько больших магазинов - Пятёрочка, Дикси, 10 Баллов, Магнит и
Перекрёсток. Не так давно этих магазинов не было. Их построили в 21 веке. Кроме
супермаркетов, в городе работают минимаркеты: Фортуна, Лесная Сказка,
СуперЛюкс, Надежда, Эврика, Алекс, Улыбка и другие магазины. Есть также и
просторные аптеки: Красные Цены, Антей, Фарм Лига, Губернская, и куча
салонов-парикмахерских, и очень хороший краеведческий музей, где научные
сотрудники рассказывают историю о нашем городе, русской старине, а также о
знаменитых людях, творческих деятелях и так далее. В городе есть недорогая
сауна - это прекрасное место для оздоровления любого человека. Вот так мы и
живём - очень хорошо и достаточно богато. Мы жители приспособленные, нам всё,
что надо - хватает. Работаем для себя и для других - стараемся! Также в городе
есть Майкова гора - это прекрасный лес. Летом здесь мы собираем землянику,
малину и подосиновики. А зимой катаемся на лыжах и на ватрушках. Приезжайте к
нам пожить, у нас тихо и спокойно, в отличии от Костромы. А возможностей очень
много, как в крупном городе. Также в Чухломе есть три больших гостиницы: две в
центре города, а одна на автозаправочной станции. Все гостиницы работают и
недорогие.

\iusr{Юлия Л.}

Ну, по поводу тотальной грамотности без единой ошибки в комментариях
погорячились.

Я пишу иконы. Часто это списки с очень древних икон. Так вот, авторы их были
грамотными, но ошибки в орфографии встречаются часто, поэтому при изготовлении
списка древней иконы надо проверять надпись на ошибки. И нельзя ее просто
списать.

Свободно читаю на старославянском, но надписи бывают очень древние, старше
нынешней версии церковно-славянского языка.

Иногда надписи приносят филологам, такое там бывает, что без специалиста не
прочитать.

Новгородская школа в силу древности сохранилась не очень хорошо, мало икон 12
века. Есть еще деревенское письмо, которое писали в северных деревнях. Вот там
ошибок бывает много.

В общем, помимо летописей, берестяных грамот есть еще пласт текстов на образах.

И я с ним имела дело довольно долго.

Бывали и очень забавные случаи при прочтении текстов на музейных иконах.
Особенно радует Южно-Русская школа. Они любили писать многофигурные иконы для
храмов, в руки Святых вкладывать назидательные свитки. Так, то свиток с
молитвой не тому Святому, то не тот Праздник. Орфографические ошибки по
сравнению с этими свитками - уже мелочь мелкая.

Так что по работе много имела дела с очень древней орфографией.

Южно-Русская школа писала слабей Новогорода 12 века, слабей Московской школы,
поэтому списков с нее обычно не делают, не шедевры, но изучают.

\iusr{Владимир}

Хорошая статья, только к ней надо писать предисловия, всё-таки не все хорошо
знают историю и историю Новгорода особенно. (Я уж про ЕГЭ не говорю) И в
советской школе это изучение было поверхностным. Да, типа: "попадаются
отдельные свитки, грамотными были 5 человек на весь город, а вот благодаря
достижениям социализма - у нас поголовная грамотность"!!!

Во-первых, начать надо с того, что Новгород и Русь (Московия) - дав абсолютно
разных государства. Я думаю это знают 30\% не более.

Во-вторых, эти разные государства населяют два различных народа!!!

А вот это, УПС!, знают в лучшем случае 1-2\%.

И именно берестяные грамоты дали возможность подтвердить этот факт.

Хотя если подумать - он вполне очевиден.

Русь заселялась с Киева. Введенный термин Киевская Русь в корне неверен,
критики (а особенно нэзалэжники) совершенно справедливо указывают что такого
упоминания нет ни у кого из соседей, и делают неверный вывод: Киевской Руси не
было.

Да, её и не было!!! Была просто Русь (и упоминаний об этом - море) со столицей
в Киеве.

И была залесская Русь, со столицей в Ярославле, Суздале, даже Твери, и в конце
концов в Москве. Так вот пришедший с Киева Князь приводил с собой своих
дворовых людей (слуг), в последствии ставших дворянами.

Новгород заселялся с запада, из Литвы.

Этому есть несколько подтверждений, не вполне очевидных:

Во-первых, политика Новгорода была ориентирована на Запад (сейчас её почему-то
называют антирусской, наверное в угоду современной политике).

Во-вторых, как раз те самые грамоты!!! Причем в конце 20-го века научное мнение
сходилось на том, что грамотными были только верхи (грамот было найдено мало),
да и те были безграмотными (писали с ошибками!!!). И именно в ошибках и речь -
это с точки зрения русского языка они писали с ошибками, а с точки зрения
новгородцев - они то писали грамотно, просто язык был не русский.

И вот два подтверждения:

Прямо эта грамота: что такое "чи"? Нет в русском языке такого слова!!! Значит
безграмотно? Ан, нет!! это просто другой язык. И "чи" есть и в литовском, и в
белорусском, и в польском.

Другой пример, в другой грамоте: "иос". Нет в русском такого слова и тут уже
говорить о безграмотности трудно.

Зато в литовском есть "jos".

И как только кто-то догадался, что новгородский язык не русский, а
западно-славянский, грамоты сразу стали написаны грамотно, просто на другом
языке.

Правда, не соглашусь с автором о "100\% грамотности", и пример с СМС - просто
супер. Грамоты это и есть СМС XV века!!! И ошибок в них ДОЛЖНО столько же

\iusr{Ипполит Воробьянинов}

Владимир, в то время русского языка в нашем понимании ещё не было. Язык в
каждом княжестве отличался от другого. Они, конечно были похожи. Люди друг
друга понимали. Но было очень много диалектов. Это не удивительно. Ибо люди
жили всю жизнь на одном месте. В другие края не ездили, ну, кроме купцов,
которых совсем не много. Ну, ещё совместные военные походы. В них тоже было
задействовано не так много людей. Современный язык для всей страны появился
лишь где-то в 19 веке. Увеличилась мобильность населения. Крепче стала
центральная власть. Так что новгородский язык тогда отличался от тверского или
владимирского не больше, чем те друг от друга.


\iusr{Aborigen СССР}

Ипполит,

Гавариш "русСкого языка в нашем понимании ещё не было"?

А какой тогда был? Можэт ты тагда жыл и на нём гаварил???

"Но было очень много диалектов." Каких? На пример хоть не много, а десяток воспроизведи для общево развития!

Выдумки свои сочиняеш людям на смех!!!

\iusr{nice.ninel}

Владимир, ещё в 60 тых годах в каждой деревне был свой говор. И даже недалёкий
пример, в 90 году из Казахстана приехала в глухую деревню Алтайского края и
встретилась с незнакомыми словами на бытовом уровне, так что не смешите с"
разными" государствами

\iusr{Владимир}

Ипполит, То что вы пишите это настолько очевидно, что как то и обсуждать не
ловко. Но я поясню написанное мною. Русский язык Киева, это один язык. Русский
язык Владимиро-Суздальской Руси, а в описываемый период Московского княжества
это другой язык. Русский язык Великого Новгорода это совсем не Русский язык
Московского княжества. Вот об этом я и пишу.

Можно было бы написать киевский язык, московский язык, новгородский язык, но
тогда это бы было бы совсем не понятно большинству читающих, они также, как вы
считают что это был некий усреднённый Русский язык, который все понимали, такой
своеобразно лингвистический "сферический конь, в вакууме".

Однако исторические описания говорят об обратном. Московские князья для общения
в новгородскими послами приглашали "литовских толмачей", что переводится, как
переводчиков из ВКЛ (Великого княжестве Литовского). Были ли они предками
литовцев или белорусов неизвестно.

И кстати, отправляясь на Переяславскую раду, московские бояре тоже брали с
собой "литовских толмачей". Здесь подозреваю это были уже малороссы или
черкассы.

\iusr{Яков POIZON}

А кто может утверждать, что новгородские грамоты написаны грамотно. Грамматики,
в современном понимании этого слова, тогда не существовало. Писали, как
говорили, не разделяя слов, не ставя ни запятых, ни точек. Современные СМСки
порой ничуть не грамотнее.

Новгородская земля не была Русью. Только в 15 веке известные московские князья
(два Ивана), частью истребив коренные новгородские рода, частью выслав их в
центральные области, заселив Новгород московскими людьми, присоединили Новгород
к Великому княжеству московскому (тогда еще не России).

Так же глупо называть Старую Ладогу столицей Руси. Эта крепость была всего лишь
резиденцией "князя". Это был наемник из варягов, который не имел никаких
наследственных прав в Новгородской земле. Русью (Rootsi) финны называли всех
германоязычных скандинавов. Финны и по сей день называют так шведов. Это слово
означало тогда не национальность, а род занятий - варяг (Varanger).

После смерти Рюрика новгородцы не стали «подписывать контракт» ни с его
малолетним сыном, ни с Олегом, военачальником. Видимо полагали, что Олег
склонен к узурпации власти. И не зря полагали. Олег отправился вниз по Днепру,
беря города. Наконец, жестоким обманом захватил власть в Киеве. Вот там он и
решил обосноваться. Киев и есть «мать городов русских», т.е. первая столица
Руси.

Берестяные грамоты хорошо сохранились во влажной земле новгородчины. Этим и
объясняется большое количество находок.

\iusr{kostik k}

Яков POIZON, грамотно,значит по Русски.Тогдаипрелки писали так,ка слышится -
так и пишется! Измени хоть одну букву и все слово поменяет
смысл.Например,"свабода"- это достижение Сварги небесной,божественной
бесконечности."Свобода"- свод,ограничение . Вот так нам и поменяли на
противоположные смыслы все слова "гусские" лингвисты...

\iusr{Владимир Колесников}

если все коменты прочитать, можно сделать вывод: что Русь, как и любое другое
гос-во (Германия, Франция, Англия) в силу каких-то исторических событий (умный,
сильный, хитрый) образовалось как гос-во сначала в Киеве, а затем в Москве. И
как ты себя не называй и на каком языке не говори , все мы один народ и чё тут
спорить

\iusr{Сергей П.}

Пра-русичей называли народом говорящим, мол, они не имели письменности, но
берестяные грамоты свидетельствуют об обратном. Хотя если быть откровенными, но
тогда грамотность была не поголовной, а царь Александр 3 даже запретил для
простолюдин достойного обучения.

Высшее образование и обучение в гимназии оставалось практически недоступным для
большинства населения России вплоть до начала ХХ века, а в 1880-х годах внук
Николая I, Александр III, продолжил политику запретов – при нём появился
циркуляр Министерства народного просвещения, закрывший доступ в гимназии и
реальные училища городской бедноте. Государство по-прежнему считало достаточным
для крестьян лишь начальное образование в объёме церковно-приходской или
земской школы

\iusr{сергей}

Романовы сделали из русов безграмотный и раболепный народ, а им еще и памятники ставят.

\iusr{клим ворошилов}

Патриарх кирилл в одном интервью утверждал, что славяне были безграмотные, жили
в ямах, вместо слов говорили бар бар. После этого интервью я его видеть не
хочу. Он прекрасно знает о наличии грамот и нагло лицемерит. Буквы в грамотах
отличаются от современных, а кирилл утверждает, что пришли кирилл и мифодий и
дали славянам грамоту.

\iusr{Владислав Большаков}

\ifcmt
pic https://avatars.mds.yandex.net/get-zen_pictures/3436280/141310478-1606417533098/orig
\fi

\iusr{Владислав Большаков}

Московско - Татарское нашествие: До нашей эры и Москва была бандитская !!!
Москва верни Великому Новгороду Вечевой Колокол и ВСЁ НАГРАБЛЕННОЕ в Новгороде,
Пскове, Твери, !!!

Картина с утерянной фото древнего журналиста !!!

\ifcmt
pic https://avatars.mds.yandex.net/get-zen_pictures/3435364/141310478-1606417439752/orig
\fi

\iusr{Павел Л.}

\enquote{В ней упоминается о рати, которая идет через Заволочье или Волок. Автор задает
вопрос: «Чи пойдёт рать за Волок» и просит осведомить его, чего ожидать от
этого военного отряда. Эта находка уникальна тем, что из документов редко
удается получить сведения о ходе каких-либо военных операций. Да и вообще в
источниках очень мало сведений о Заволочье.

Этот фрагмент наводит на мысли о каком-то военном конфликте в те времена.

Рядом с фрагментом была найдена печать новгородского посадника, который занимал
эту должность в 1354-1380 годах, это и помогло датировать документ.}

Псковско-полоцкая война 1350-55

В 1323 Псков и ВКЛ выступили союзниками в войне с ливонцами. В 1341 Ольгерд
приходил с войсками на помощь Пскову. Андрей Ольгердович стал Псковским князем,
а наместником стал литовский князь Юрий Витовтович. Андрей вскоре обосновался в
Полоцке. Весной 1349 Юрий Витовтович погиб в стычке с ливонцами. Псковичи
перестали признавать Андрея Полоцкого князем. В ответ в Полоцке и других
литовских городах были арестованы псковские купцы, а затем отпущены за выкуп. В
1350 полоцкое войско совершило набег на волости псковского пригорода Воронача.
Мор 1352 г не дал осуществить масштабные ответные действия. В 1354 и 1355
псковичи во главе с князем Остафием совершили набеги на полоцкие волости.
Вероятно, происходили и другие взаимные набеги меньшего масштаба.

Вероятно, именно к этому периоду и относится текст грамоты.

А место сосредоточения войск ВКЛ, вблизи волока, может располагаться у изгиба
реки Великой. Здесь, в более поздние времена, проходила дорога из Полоцка на
Новгород и Псков.

В 1517 году, по ней проезжал Сигизмунд Герберштейн (первое посольство в
Московию).

\ifcmt
pic https://avatars.mds.yandex.net/get-zen_pictures/2977420/17960301-1606336263581/orig
\fi

\iusr{лв}

Были в Новгороде и Пскове. Потрясены! Экскурсоводы много цитируют переводы
древних грамот. Оказывается, даже мальчишки, что обувь чистили, умели писать.

\iusr{Валентин Мащенко}

По всей вероятности эта всеобщая грамотность, когда немцы (англичане, французы, австрийцы, собственно немцы) столкнулись с этой грамотностью при посещении России, то видимо всё это их здорово испугало. Они отлично понимали к чему это может привести в дальнейшем. Не случайно Иван Грозный собирался открыть в Москве университет и, видимо, также не случайны все эти события смутного времени, да и сейчас не которые деятели Запада говорят, что в России слишком высокая грамотность населения. А наши либералы (Греф, Чубайс и др.) просто заявляют, что Россия тратит слишком много денег на образование.

\iusr{Руслан Солнышко}

Всем Здравствуйте! Не за горами Новый 2021 год. Знаете ли вы, что с 1 января
2021 года наступит настоящий рай на Земле? Бог продлил трудную Земную жизнь от
2 октября 2020 года до 1 января 2021 года. 2020 год был юбилейным, високосным и
новоцикловым годом. Как написано в Священном писании, Иисус придёт второй раз
на Землю, и сразу восстановится порядок на планете. Зимой было очень мало
снега, закат и восход солнца красней стали, коронавирус появился, красная луна
вечером, частые дожди, аномальная жара и многое другое - это признаки Второго
Пришествия Иисуса Христа. Все люди должны приносить Богу плод добра и зла,
который не принесли Адам и Ева в раю Эдеме. Также Иисус умирал за грехи всех
людей. Мы перерождаемся в прах или воскресаем на небеса. С 1 января 2021 года
болезнетворные бактерии не будут вызывать болезней, сильные не будут обижать
слабых, вода не сможет утопить, огонь не сможет обжечь, Земля не сможет
поглотить своими пропастями.Также будет всё расти бесплатно в обилии: фрукты,
овощи и ягоды. Причём даже на севере мы будем питаться и бананами, и кокосами,
и клубникой и даже картошкой, которую сажать не придётся. Наша Земля
превратится в настоящий райский сад! Животные будут питаться фруктами, овощами
и ягодами. Умирать никто не будет, но и рождаться будут реже. В Библии известно
точное число воскресших людей - 144 тысячи. Возможно, через миллиарды лет,
будет жизнь на других планетах и даже звёздах, так как вселенная бесконечная.
Это точные данные, рай непременно на Земле наступит! А пока желаю благополучия
и процветания! Пусть Новый 2021 год принесёт море счастья и богатства для всех
людей и животных! И поможет в этом Всемогущий Добрый Бог. Аминь.

\iusr{Дмитрий Шульман}

Борис, ты еблан? Сестра, несите морфий... какую х...ню я сейчас прочитал  Боже мой... идите с богом, Не возвращайтесь

\iusr{Руслан Солнышко}

Дмитрий Шульман, Я Солнышко и Красавчик! А не еблан.

\iusr{Валерий Баженов}

Руслан, писец придёт. Вот что придёт. Отнесите вашу книгу обратно в Израиль.
Нам она не нафик не нужна.

\iusr{Руслан Солнышко}

Валерий Баженов, Бог нужен всем. Только люди живут по законам Государства.

\iusr{Вячеслав}

Вот интересно откуда там кресты берутся в таком количестве... Неужели наши
пращуры так небрежно относились к предметам культа? Да и с берестой не очень-то
понятно, дерево (берёза) лет за 10 гниёт в праз, а береста 300-400 лет лежит
себе возле крестиков и ничего им не делается... Просто чудеса в решете

\iusr{Александр Романов}

Вячеслав, так береста и не гниет. И береза если не в бересте и в земле сырой не
гниет. У нас колодцы из березы да осины сруб делают. Веками стоят. Гниет только
верх над землей и то лет 50 стоит не меньше. А в коре любая древесина сгнивает
за год, через 2 года труха. А кора любого дерева дольше не гниет. Это как кожа
у животных. Животные гниют быстро а кожа снятая с них дажн не выделанные шкуры
годами не гниют. Даже в земле. Учите таки химию и физику.

\iusr{Дмитрий Д.}

Александр Романов, \enquote{И береза если не в бересте и в земле сырой не гниет. У нас
колодцы из березы да осины сруб делают. Веками стоят. Гниет только верх над
землей и то лет 50 стоит не меньше.}

Берёза наихудший вариант для сруба колодца, отслужит максимум 20 лет, под водой
гниёт, читайте справочную информацию

\iusr{Вячеслав Гузиков}

Не верю. Не верю не тексту грамоты, а самой грамоте.

Почему не верю? Грамоты находят в глубине культурного слоя: где-то 3-9 и более
метров. Почему вход в Софийский собор находится на современной отметке? Собор,
который построили раньше того периода, к которому относятся грамоты.

\iusr{Игорь Венедиктов}

Вячеслав Гузиков, а грамоты знаешь где часто находят? На помойке, то есть в
вырытой яме. Туда их выбрасывали. К тому же культурный слой хорошо растёт явно
не у церкви, все таки предки старались не мусорить там, где молятся.

\iusr{Тата Мир}

Вячеслав Гузиков, ващето уровень древнего пола в Софийском Соборе находится на
2 метра ниже современного.

\iusr{Наталья}

остается позавидовать нашим предкам по поводу грамотности. Особенно, когда
читаешь статьи на Дзене. Сегодня обнаружила еще одно слово - слово молодёжЬ
было написано без мягкого знака. Вспомнилась история из 90-х. Моя знакомая
работала фельдшером на полставки на кирпичном заводе, а после обеда в конторе
подрабатывала, выполняя обязанности кадровика и расчётного бухгалтера. На
летние сезонные работы принимали школьников, тех , кто хотел подработать. В
конце сезона им старались выдать заработанное деньгами ( с 1994 года по 2000
год официально зарплату выдали в 1997 году за два месяца. Те, кто мог, выбирали
зарплату кирпичом, потом его продавали гораздо дешевле). Некоторые ребята не
хотели ждать три месяца и писали заявления с просьбой выдать им кирпич в счёт
заработной платы. Так вот, практически во всех заявлениях школьников, за редким
исключением, слово КИРПИЧ писалось с мягким знаком. Это было отличительной
чертой учащихся Краснополянской средней школы.

Ни в коем случае не осуждаю тех, кто делает ошибки. Сама грешу этим.

\iusr{Звёзды Светят}

\ifcmt
pic https://avatars.mds.yandex.net/get-zen_pictures/3435364/43052997-1603305308062/orig
width 0.3
\fi

\iusr{роман попов}

А грамоту специально для \enquote{археологов} подложили. Вот так прямо и сказал
настоятель монаху - \enquote{поклади сею грамоту в землю у мостовой храма, для
археологов}. А он, шельмец, половинкой подтерся, а вторую поклал.

_

Ну что ж. Дело Шлимана живет и процветает. По закладке \enquote{древностей} в места
где орудуют археологи. Молодцы чО!

_

Хохлы моря копают и в песках Днепра нержавый железный меч находят (ну рюрика
конечно, на крайняк святослава), секта Фоменковцев давно чётко по хронологии
все разложила, Клесов носится со своим ДНК, не зная кому ее впрыснуть,
последователи Шлимана бересту находят там где ее на розжиг используют.

Завидую - весело живут люди. Я все жду, когда же эти клоуны объединятся? Ну
типа - найдена древняя береста, специально положенная на не менее древнюю
мостовую (куда ж еще) в которой варяго-аннунаки благословляют свято-волхвов на
подчинение сармато-русам на идгар-земле во славу фоменко-клесовцев!!!

\iusr{Ипполит Воробьянинов}

Рать идёт через Заволочье или Волок. Видимо, это современный Вышний Волочок.
Там суда перетаскивали из Мсты в Тверцу. Мста - бассейн Волхова, а Тверца
впадает в Волгу. По ней новгородцы шли вниз по течению до Торжка. Это был
последний пункт. Дальше уже были тверские земли. Там новгородцы не ходили. В
Торжке они продавали товар тверским купцам. А из Твери уже купцы (Афанасий
Никитин и прочие) везли его вниз по Волге. Или через Волок Ламский
(Волоколамск) в Москву и далее на Оку. Ибо через Нижний Новгород попасть в
Москву вверх по Оке было бы слишком далеко и небезопасно. Таким образом, волоки
были стратегическими местами. И тот, кто занимал его, мог контролировать весь
трафик товаров.

\iusr{Владислав Горбачев}

Да, ребята, да. Новгород, Псков и Вятская область процветали. Так же как
итальянские города-государства, так же как швейцарские кантоны. Так же
грамотны, богаты и счастливы. А почему? Потому что были городами-государствами.
Ещё и с республиканским строем.

Поразмыслите над этим.

\iusr{orekh712}

В бересте IX века бытовая записка- по жанру - \enquote{сговор}: \enquote{я тебя хочу, ты меня
хочешь, и брат то знает}. Вот так простые люди писали. Массовая грамотность
была.

\iusr{Irina Pankratova}

Анна Ярославна французский двор пыталась хоть чуть-чуть "причесать", библиотеку
собирала, хотела их окультурить. Сама несколько языков знала, а муженек крестик
ставил на указах и упивался незамысловатыми плотскими животными радостями.
Очень пеняла в письмах батюшке, что сослал ее в грязную деревню Париж к диким
необразованным варварам...

Образование на Руси уничтожила межусобица, иго и особенно крепостное право. На
севере, которого это все не коснулось, была абсолютная грамотность, читать
умели и женщины, и дети. Кстати, женщины имели очень высокий статус и принимали
активное участие в жизни. Ну и Ломоносов не на пустом месте такой умный
народился.

А Новгород был республикой с самой что ни на есть настоящей демократией. Это
возможно только при достаточном среднем уровне образования.

\iusr{Юрий П}

Поясню для недовольных.

В Новгороде существовал совет господ («госпо́да») — своеобразная новгородская
высшая палата. В состав совета входили:

- архиепископ (владыка) — один из руководителей государства и хранитель
государственной казны, контролировал эталоны мер и весов.

- посадник — исполнительный орган веча, постепенно его полномочия расширялись,
и с XIII века он уже избирался Советом Господ.

- тысяцкий

- кончанские старосты

- сотские старосты

- старые посадники и тысяцкие.

K XV веку все решения веча обычно заранее готовились Советом господ, это и есть
\enquote{управляемая демократия}. Народу нужно было только проголосовать.

\iusr{Евгений Гришуков}

Irina, Да, отчасти это так, но это скорее инструменты уничтожения
распространённого образования, странным образом, с появлением университетов по
западным лекалам падало общее распространение образования, то есть идея
золотого миллиарда, грамотность для избранных это не изобретение
современности... Хорошо это можно показать на институте, храме Весты , в
древнем Риме, своего рода изъятие образа жизни из народа, обязали определенную
групу людей, женщин в данном примере, служить, одновременно жестоко спрашивая
за традиции, хотя это уже искажение, сути образа Весты, одновременно всех
остальных это как бы уже не касалось... В России в первоночалтном смысле это
ещё пока есть , в образе НеВест! Похоже падение всеобщей грамотности в России,
это влияние западной псевдо цивилизации, и после Октябрьской Революции, Россия
трудно но верно возвращалась к себе самой, кто бы что не говорил, но СССР
наиболее точно отвечал концепции Русской Цивилизации, оговарбсь, сейчас многое
пытаются свалить на Романовых, дескать это они во многом виноваты, да
несомненно были не просто ошибки, а провальные просчеты, но лукавят те кто все
сваливает на них, как и в СССР, да далеко не все было идиально, но все-таки
направление было более чем верно,... Хорошо нужно присмотреться к тем кто
пытается навязать Русскому народу, народам все возможные комплексы, и назначить
виноватых, это есть попытка обыкновенных манипуляций! Но,, друзья,, России
просчитываются всегда, как раньше думали что Россия Царская, оказалось что это
Царь, институт Царской власти были Российскими, который в свою очередь был
ответов на вызовы времени! А Россия, Русь это есть не что больше Веры, то есть
её просто Верой назвать нельзя, это концепция Мировозрение, состояние Души,
Русский это Светлый, но не по внешнему виду а по состоянию Души! Поэтому и не
могут определить, своего рода прикрепить, назначить какую то точку начала Руси,
Русь, Свет Утренней Звезды это образ начала Мира! Поэтому по крайней мере у
Руси несколько центров, это и Причерноморье, и дальний Север, и даже побережье
Тихого Океана, и этому есть все докощательства... Но как и спросила Царевна
Лебедь а не расстаешься потом? Русским быть тяжело, и Русскость легко можно
потерять, Украина тому пример, они сами отказались от Русскости, да и то
Правда, какие они Русские? Русским нужно быть, а не казаться! Поэтому в те
времена, беременных грамот, образование было само собой разумеющееся явление...

\iusr{Irina Pankratova}

Евгений, я не сильна в древневедических знаниях, привыкла доверять фактам,
например, берестяным грамотам. То, что история переписывается каждым правителем
и каждой формацией, давно понятно -- только на нашей жизни это произошло
несколько раз. А что культура и идеология изменяются в соответствии с
требованиями времени и многочисленными внутренними и внешними влияниями в
обществе, это норма, заложенная самой природой нашей изменчивости,
приспособляемости. Главное -- человеком оставаться и не уничтожать в угоду этим
изменениям ни ближнего, ни дальнего своего. Может, когда-нибудь человечество
научиться этому.

\iusr{александр авдеев}

Юрий, в былине о Садко об управлении в Новгороде много прописано.

\iusr{Сергей Долгов}

Берестяные грамоты были открыты ещё до революции. Перед Первой мировой войной
появился музей новгородских грамот. В Гражданскую войну, или сразу после неё
дом был разорён и сожжён. Считается, что большевиками. Поэтому в СССР этот факт
замалчивался.

\iusr{Георгий Давыдов}

На этом примере надо было бы сравнить либеральную РФ и СССР, где были
поголовная грамотность и самая читающая нация, а ныне - русский крест - нищета,
деградация и вымирание. Вероятно, эти процессы проходили и ранее, например, при
крещении, подобно, вероятно, фашистскому крещению в европейские ценности. Когда
во власти антинародная шушера, то вряд ли стоит ждать что-то хорошее
народу-труженику, которого либералы, например, после расстрела Парламента и
Народа, назвали быдлом. А это говорит о том, что, как и требует демократия,
власть должна принадлежать Народу-Труженику, на не народу-паразиту, который,
грабя трудящихся, богатеет, увеличивая миллионеров и миллиардеров,
коррупционеров и т. д. ЗА ДЕМОКРАТИЮ ДЛЯ НАРОДА, В ИНТЕРЕСАХ НАРОДА И ПОД
КОНТРОЛЕМ НАРОДА-ТРУЖЕНИКА!!!

\iusr{Василий Смирнов}

Интересно в текстах берестяных грамот Древней Руси и то, что они читаются и
легко понимаются современными русскими людьми. Украинцы, так рьяно ратующие за
свою древность и \enquote{древность} своей мовы ни прочесть ни понять эти тексты не
могут. И нет ни одной древней летописи написанной на языке приближенном к мове.
Все без исключения на древнерусском языке.

\iusr{Георгий Давыдов}

Лариса Хорошева, по фамилии надо бы во всём хорошо разбираться! А то на частном
примере выводите общезначимые истины. Вам просто дают возможность подумать и
решить: если ныне власть стремится в той или иной мере восстановить то, что
было в советское время, то это потому, что после переворота, а ныне власть
несколько иная, либералы повели дело по эгидой внешнего врага на уничтожение.
Ибо после контрреволюционного переворота и установления политики, приведшей к
русскому кресту на основе шоковой терапии и МРОТ на уровне выживания, по словам
Заславской только молодых мужчин умерло двенадцать миллионов. А добавьте к ним
детей, которых приучили кайфовать, дыша химией, женщин и стариков при мизерных
зарплатах, пенсиях и пр. И если либералы это списывали на то, что не вписались
в рынок, а грабить общенародное добро могли единицы, то они - либералы,
свободные от чести, совести и пр., не стесняясь оперировали миллионными
числами, как и фашисты, в прогнозах по оптимизации численности населения. И
если они построили социальное государство, где одни получали возможность
становиться миллионерами и миллиардерами, то для невписавшихся - МРОТ, а для
того, чтобы не было социального взрыва, государство взяло на себя обязанность
бесплатных - за счёт бюджета, похорон. При этом ввело монетизацию льгот. А для
чего? Правильно, если вынуждены платить МРОТ, то для сокращения качества жизни
заставить из него оплачивать и разные услуги. Вот об этом и речь. Поэтому, если
та либеральная власть вела политику русского креста - обнищания, деградации и
вымирания народа, которого после расстрела законной власти и народа называла
быдлом, что соответствует фашистским планам понижения биологической силы
славянских народов, то нынешние государственники стараются как-то этому
воспрепятствовать и сделать государство социальным - в интересах народа, а не
избранных и успешных. Вот об этом речь. Вот это и надо обсуждать! Или не так?

\iusr{Кондрат К.}

НЕ прочитал вот этот коммент

\enquote{Андрей Виноградов

Павел К., Ну и чушь вы нагородили. Во первых русь указывается в списке .И
список этот не содержит ни одного славянского.В основном это германцы. Во
вторых - русь это не племя,а профессия и слово имеет явно финское
происхождение.Все остальные англы, фряги, и т.д. и только варяги-русь. Именно
так финны называли шведов. Роутси-гребцы. Кстати именно так они шведов называют
по сей день роутсилайнен,а нас по сей день вения .В третьих Язык менялся и в
раннем древнерусском наречии не было слова русь вообще.Это слово псалось через
букву \enquote{от}(сейчас не используется) и читалось как Руотсь или Роутси.
}

ЦИТАТА с верхнего коммента: \enquote{... Во вторых - русь это не племя,а профессия..} -
Ну да от тех же горе историков и специалистов есть ещё и другие которые выводят
\enquote{русь} от русла... есть и такие которые пишут, что само название славяне - это
\enquote{не римляне} (это кстати ученик Трубачёва)...

Нет это как бы авторитетные учёные и не с ругаемого кому-то только не лень дзена...

И никому не вдомёк, что понятие РУСый и РУСЬ очень близки ..

А вот из коммиентов к
\url{https://zen.yandex.ru/media/drevotmb/zagadka-slova-rus-obiasnenie-etimologii-ot-anatoli-5f3d5b9f72abdf3b1029039c}

Обоснованный (или подкреплённый) АВТОРИТЕТОМ

"...stanislaw.glushkov

2 месяца

Оле́г Никола́евич Трубачёв — советский и российский лингвист-славист,
исследователь этимологии славянских языков и славянской ономастики; специалист
в области сравнительно-исторического языкознания, лексикограф. Доктор
филологических наук, член-корреспондент АН СССР (1972), академик РАН
(1992).

\enquote{Имя русский восходит к корню славянскому и индоарийскому рукс- или
рокс-, что значит - «белый, светлый». То есть русы - народ белый, народ Света.
Согласно описаниям арабских источников, в которых задолго до появления
славянской письменности впервые зафиксировано имя русы, это были высокие люди
со светлой кожей, светлыми - русыми волосами, синеглазые; в буквальном смысле
слова - белый, светлый народ. И сами русы называли свою страну - Русь -
буквально «белый свет, единственно возможное для жизни место, Родина».

На русском севере до сих пор говорят \enquote{Вынести на Русь}, что значит вынести на
свет....}

То есть СВЕТЛЫЙ = БЕЛЫЙ.. СВЕТЛЫЙ и противоположность ЧЁРНЫЙ (СВОЙ и ЧУЖОЙ)...
и значит РУССКИЙ и НЕРУССКИЙ...

\iusr{Альберт Каракчиев}

В 1397 году новгородцы сходили \enquote{за волок}, разрушили город Орлец, поэтому в
году где-то 1399 младший брат двинского князя Ивана Никитича, Добрыня,
названный в новгородских летописях Анфалом (так как был предан анафеме) сжег
Новгород, причем, жег его не один раз. Наверное, разборка была из-за двинской
казны, увезенной новгородцами. Эта история очень интересная, погуглите...

\iusr{Обоев Рулон}

В тех краях только три волока....

И все реки через Ильмень...

Две с Руси, одна с Киева....

Вот и гадай откуда беда

\iusr{Догор}

\enquote{ Чи пойдет рать за Волок} Какого тысяцкого рать. Или какого боярина
рать.допустим 20 воинов. Может разбойников банду гонять чтоб не грабили. Или
допустим за берестой ходить, охотиться в землях где племена воинственные есть.
Мало ли. В летописях же часто упоминается что новгородцы ходили ратью на \enquote{чудь}
Помню один такой воевода \enquote{Даньслав} ходил на чудь было около 500 бронников. В
кальчугах воинов. Кальчуга хорошо держит удары дубинки,ножа,меча. О нему
удар-он удержал и в ответ бьёт по не защищённому с ножем. Железо было дорого.
Люди плодились хорошо. Жили племенами. В глубине леса. В дали от богатых и
страшных войнов новгородцев. По любому лесные племена защищались отстойным
луками со стрелками с кости наконечник,ножом купленным за кучу меха,
копьём.рагатиной,камнем. Топором с камня. Ещё новгородцы приводили после
походов на чудь \enquote{полон} пленных. Которых потом использовали как рабов или
прислуги.

\iusr{Иеремия}

Грамотность на Руси христианской была повсеместной до 10 в. до \enquote{крещения},
когда князь Владимир лёг под Византию. Свою деятельность попы
византийские-греки начали с запрета обучения русскому языку. Держали оборону до
15 века в Великом Новгороде и на землях Войска Донского. Инквизиция пришла на
Русь Через Софью - жену Ивана III с уничтожением христиан Великого Новгорода, с
внедрением рабства-крепостничества Иваном Грозным, активным наступлением на
казачество. Из Руси успели спастись тысячи лучших её представителей в Литве, в
Пруссии, на Дону. Уничтожение остатков христианства продолжали Романовы,
большевики, СССР, нынешние капиталисты.

\iusr{Wayne Gilkrist}

Современный русский язык происходит не от поднепровских племенных диалектов, а
от новгородского и псковского диалектов древнерусского языка. Носители этих
диалектов - ильменские словене - пришли с запада, с южного побережья Балтики. И
то, и другое давно доказано археологически и лингвистически.

\iusr{Юрий Червов}

Всеобщее образование тогда наш народ с детства получал безмездно в монастырях,
коих по Руси уже в 13-14 в. в. было достаточно для зтого. И в зтих монастырях
дети получали именно ОБРАЗОВАНИЕ (от слова ОБРАЗ), а не сумму знаний и
способность читать и писать. И так продолжалось до смены династии. И вот пришли
Романовы, и началось! Царь Петр отменил патриаршество, и уже при последующих
императорах и императрицах началось повальное закрытие монастырей. При
Рюриковичах, если монастырь терпел нужду, ближайший князь ссужал зтому
монастырю деньги или помогал продуктами. А при Романовых такой нищий монастырь
банкротили, т.е., закрывали. А, впрочем, закрывали монастыри не только из-за их
бедности. Потому что: ну зачем немцам грамотный и образованный русский народ?

\iusr{Андрей Виноградов}

Скучно с вами ,девушки-это я про обсуждение и большинство комментов в нём. Всё
те же избитые"мы русы мы самые великие" "нашу историю украли","проклятые
Романовы переписали летописи","Новгород-не Новгород" и прочий идиотский лепет.
А самое смешное и печальное,что зайди к"великим украм" и "к великим тюркам" и
прочим "великим" -слово в слово одно и то же разве тольк скажут что "историю
украли москали" да Фоменко или Задорнова заменят на какого-нибудь Бебика.
Интернет и Ютюб противопоказан большинству,пустые черепные коробки загружаются
националистическим,антинаучным дерьмом.

\iusr{Валерий Бочкарев}

При Советской власти мы были гражданами своего государства, а сейчас мы -
батраки и рабы иностранной антисоветской, антирусской (еврогитлеровской) власти
и идеологии, политики! Если в рамках борьбы с безграмотностью (мама в 17 лет
начала учить грамотности в школе) писали: мы - не рабы! То теперь в школах надо
учить детей писать:Мы-рабы! Как это пишут в церковных школах и визжат попы в
церквях.

\iusr{Николай Северов.}

Умение писать никак не говорит о культурном уровне человека. У нас сегодня в
России все поголовно читают и пишут, но уровень пещерности зашкаливающий. А вы
позволяете себе на основе одного факта делать выводы о \enquote{невероятно высоком
уровне культуры Древней Руси}.

\end{itemize}
