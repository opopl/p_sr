% vim: keymap=russian-jcukenwin
%%beginhead 
 
%%file 17_12_2016.stz.news.ua.mrpl_city.1.rybnaja_ploschad
%%parent 17_12_2016
 
%%url https://mrpl.city/blogs/view/rybnaya-ploshhad
 
%%author_id burov_sergij.mariupol,news.ua.mrpl_city
%%date 
 
%%tags 
%%title Рыбная площадь
 
%%endhead 
 
\subsection{Рыбная площадь}
\label{sec:17_12_2016.stz.news.ua.mrpl_city.1.rybnaja_ploschad}
 
\Purl{https://mrpl.city/blogs/view/rybnaya-ploshhad}
\ifcmt
 author_begin
   author_id burov_sergij.mariupol,news.ua.mrpl_city
 author_end
\fi

Площадь за швейной фабрикой до начала 20-х годов двадцатого же столетия
отделяла Бондарную улицу от улицы Итальянской. В советское время и Бондарная, и
Итальянская были объединены общим именем - Кузьмы Апатова, красного командира.
Тогда-то и канула в Лету Бондарная.

\ii{17_12_2016.stz.news.ua.mrpl_city.1.rybnaja_ploschad.pic.1}

Ну а площадь? Напрасно искать ее название на картах Мариуполя. Его просто там
нет, но зато  оно есть в издании \enquote{Адрес-календарь: весь Мариуполь и его уезд}
за 1910 г. Там указано, что рыбные лавки Акинитова и Бонтара, Кечеджи и
Копилевича, Пасько и Соколовского, Тютюникова и Филизанова, Челпанова то ж,
находятся... на Рыбной площади. Почему Рыбная площадь не обозначена на картах
города? Видно, название было неофициальным.

Какую же рыбу здесь можно было купить? Да разную. Правда, некоторые виды ее,
которые ниже будут перечислены, известны нашим современникам только понаслышке.
Итак, на Рыбной площади продавали судаков, в Мариуполе эту рыбу называли сулой,
лещей, у нас они именовались чебаками. А еще - вяленых рыбцов, жирные спинки
которых просвечивались на солнце, чехонь – рыбу на любителя, шемаю,
изумительный вкус которой невозможно передать словами, камбалу -  странноватое
произведение природы, по тем временам совсем нередкое и, следовательно,
недорогое. Красную рыбу - осетров или севрюгу - покупали  только в живом виде.
Бытовало мнение, что \enquote{снулой} красной рыбой можно отравиться.

Перечислим еще несколько видов съедобных обитателей глубин Азовского моря,
бывших на прилавках Рыбной площади. Черноморская сельдь, азовский пузанок,
кумжа, угорь, вырезуб, синец, сом, жерех, азовский анчоус-хамса. Это, конечно,
далеко-далеко не полный перечень того, чем довелось лакомиться нашим пращурам.
От внимательных читателей вполне можно услышать законный вопрос: почему забыты
бычки и тюлька? Безусловно, они в нашем море водились, и водились в изобилии.
Но промышленного лова их до революции не было. Разве что мальчишки могли
удочками \enquote{надергать} десяток-другой бычков для котов, а если юный рыболов
происходил из бедной семьи, то и на стол. Тюльку тоже не трогали – это же корм
более крупной рыбы. Ну а тарань? Свежей таранью не торговали, что же касается
сушеной, то ее, говоря современным языком,  реализовывали не по десятку, а
кулями, мешками по два пуда (32 килограмма). Сохранилась почтовая открытка
начала двадцатого века, на которой изображена часть Рыбной площади. Почему она
малолюдна? Да ведь лето.  Запрет на вылов рыбы. А запрета азовские рыбаки
придерживались не по принуждению, а ради сбережения богатства их кормильца –
моря.

Газета \enquote{Мариупольская жизнь} 22 марта 1909 года писала: \enquote{В магазине Общества
потребителей (Николаевская улица, дом Земства) к праздникам получены сельди
королевские, дунайские, шотландские, керченские, рижские и серебрянка
(лососинообразная рыба, промысел которой велся в Атлантическом и Тихом океанах,
а также в Средиземном море  - С.Б.), анчоусы в банках, рыбные консервы}.
Странные люди - мариупольцы, у них под боком водоем, кишащий рыбой, а им
подавай нечто заморское. Покупали эти деликатесы, естественно, очень
состоятельные люди. Все остальные довольствовались дарами Азовского моря.

Нечетная сторона Итальянской улицы - от Рыбной площади до улицы Торговой - была
также занята рыбными магазинами. Вот воспоминание об одном из них, услышанное
от отставного полковника Льва Исаевича Гецонока: \enquote{Мне было лет восемь, когда
наша семья осталась без отца. Он погиб при обстоятельствах невыясненных. Меня
пристроили в рыбный магазин мальчиком на побегушках. Хозяйки или их кухарки
выбирали рыбу, платили за нее, а моя задача была как можно быстрее доставить
покупку по указанному адресу. За доставку давали пятачок, а то и гривенник \enquote{на
чай}. Это был дополнительный заработок.  Владелец заведения был человеком
добрым, перед выходными днями и праздниками всем работникам, и мне в том числе,
выдавал по рыбине. Но если он замечал, что кто-то украл даже маленькую рыбешку,
то провинившегося увольнял немедленно}.

Нам неведомо, когда рыбный торг переместился на соседнюю Базарную площадь.
Видимо, вскоре после этого постепенно стало забываться словосочетание \enquote{Рыбная
площадь}. Старое место тоже не пустовало. Здесь, наверное, до середины 50-х
годов, если не позже, торговали керосином для ламп, примусов и керогазов, а
также скобяными и москательными товарами. В базарные дни труженики пригородных
колхозов и совхозов здесь ставили свои полуторки, потрепанные ЗиСы, а то и
подводы, наполненные плодами полей, садов и огородов.
