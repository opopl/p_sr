% vim: keymap=russian-jcukenwin
%%beginhead 
 
%%file 03_07_2020.fb.lesev_igor.1.ulichnye_bandy
%%parent 03_07_2020
 
%%url https://www.facebook.com/permalink.php?story_fbid=3332798090084604&id=100000633379839
 
%%author_id lesev_igor
%%date 
 
%%tags banda.ulichnaja,nacionalist,nacionalizm,napadenie,partia_sharija,politika,radikalizm,radikaly,sharij_analotij,strana,ukraina
%%title О наших и ваших уличных бандах
 
%%endhead 
 
\subsection{О наших и ваших уличных бандах}
\label{sec:03_07_2020.fb.lesev_igor.1.ulichnye_bandy}
 
\Purl{https://www.facebook.com/permalink.php?story_fbid=3332798090084604&id=100000633379839}
\ifcmt
 author_begin
   author_id lesev_igor
 author_end
\fi

О наших и ваших уличных бандах

По стране прокатилась серия нападений на партийцев Шария. Под раздачу также
попала офисная недвижка ОПЗЖ. И последние, в частности Кива, заговорили об идее
создания неких «отрядов самообороны», которые де будут противостоять уличным
националистам. Давайте разберем, почему эта идея тухлая и нерабочая.

Итак, начнем с природы уличных банд националистов. А это, действительно, банды,
потому что в основе их движа – противозаконная деятельность, совершаемая
группой лиц.

\ifcmt
  pic https://scontent-frt3-1.xx.fbcdn.net/v/t1.6435-9/106942897_3332797876751292_7247696611332112152_n.jpg?_nc_cat=107&ccb=1-5&_nc_sid=730e14&_nc_ohc=_aO5eMJyU5QAX88eu0t&_nc_ht=scontent-frt3-1.xx&oh=7aeac85618d0b7b8ba898a0df6320e75&oe=61A71470
  @width 0.4
\fi

Но ведь эти банды появились не вдруг и имеют четко выраженный институциональный
характер. Националистические структуры вышли из Майдана и сыграли ключевую роль
в трансформации власти от Януковича к Порошенко. Если проводить аналогию с
Октябрем 1917-го, то это была такая наша местная разновидность революционных
матросов.

Легитимность новой власти, даже при тупой поддержке западных посольств, была
крайне сомнительной. И даже одержав тактическую победу, выбив Януковича из
Киева, страну неминуемо ждал бы Большой общественный договор. Ну хотя бы потому
что «идеалы Майдана» были идеалами только для половины страны. А учитывая
качество человеческого материала, дорвавшегося до власти (тут нет расизма,
просто цитирую главу комитета от «Слуги народа» Галюсю Третьякову),
антимайданный реванш состоялся бы, несомненно, раньше, чем через пять лет
чьего-то там президентства.

Но нам всем подсобили братушки из Кремля сделав финт с Крымом. И то что было
банальным военным переворотом, тут же превратилось для майданной части страны в
«отечественную войну украинского народа против российских оккупантов».

Тут же сменился и характер противостояния с украинскими националистами.
Особенно это видно на войне в Донбассе. И проект Новороссия, и республики ЛДНР
– это все история расчленения страны А с последующей передачи оторванных кусков
стране Б.

Ну вот давайте проведем аналогию с Гражданской войной в Испании. Франция,
например, топила тогда за республиканцев, а Италия за франкистов. Итальянские
регулярные части даже воевали на территории Испании. Это была прямая
интервенция. И все же война в Испании была именно гражданской, несмотря на то,
что там были и интербригады, и люфтваффе, и наши советские летчики, и целые
дивизии из Италии. А все потому, что Франция не пыталась присоединить к себе
Каталонию, а Италия Балеарские острова. В первую очередь испанцы месили друг
друга, чтобы построить свое «идеальное общежитие».

Ну а события 2014 году у нас у всех перед глазами. Мы все знаем, кто и за что
воевал. Одни – за очень специфическую Украину, а другие за то, чтобы не было
вообще никакой Украины.

Но ближе к уличным бандам. Националистические организации, начиная с 14-го
года, вошли в такое себе государственно-частное партнерство на взаимовыгодных
условиях. Я говорю сейчас о нескольких уровнях реализации государственной
политики с точки зрения ценностей идеалов Майдана.

Есть публичная политика. Респектабельные лица, пусть и в чуть помятых пиджаках,
которые рассуждают о демократических ценностях, европейском выборе и прочих
чудесностях.

Дальше идут менты. Более закрытая и консервативная организация, которую даже за
6 лет не смог полностью заменить Аваков новыми майданными кадрами. Ну хотя бы
потому что нет в стране столько майданных ментов. К тому же менты хоть и не
добры, но понимают, что всегда и везде будут крайними, а потому исключительно
редко по своей инициативе переступают уголовный кодекс. Да, они могут не
замечать, как прямо перед ними бьют журналистов. Но если уж им дают какое-то
очевидно палевное дело, вроде похождений Стерненко, козлами быть решительно
отказываются.

Есть еще более глубинный уровень – это СБУ. Абсолютно закрытая организация и
при Порошенко перестроенная в индивидуальный карательный орган пятого
президента. Сам Порошенко очень любит себя сравнивать с польской
«Солидарностью», Лехом Валенсой и эдаким народным мессией-любимчиком. Но на
самом деле Порошенко из СБУ построил натуральную кальку службы безопасности
ПНР, а его «Солидарность» - это, скорее, ПОРП времен Ярузельского.

Везде, где происходили громкие убийства в нашей стране – Бузины или Шеремета –
непременно маячат ушки СБУ. В Польше 80-х происходило такое же дерьмо,
например, дикое по своей нелепости убийство священника Ежи Попелюшко.

Но опять же, СБУ хоть и закрытая, и полностью уходящая от любого уголовного
преследования своих сотрудников контора, но все же это официальная организация.
Ну и беспредельничать своими руками – это ведь тоже карма и не по фен-шую.

Вот, собственно, мы и дошли до самого дна – уличных националистических банд,
которые замыкают государственно-частное партнерство в стране, которую построил
Порошенко. Пацанва там разношерстная. Кто-то беспределит за идеалы, кто-то по
приколу, и почти все за материальные няшки. Такие себе черносотенцы времен
поздней Российской империи, которые могли себе позволить «идеи ради» устроить
какой-то там Кишиневский погром.

При этом нужно понимать одну особенность, которую так и не понял Петр
Алексеевич. Идеалы майданной Украины в ее кристальном виде – это как чистый
спирт. Мы все бухаем, но бухать чистый спирт – это удел особенных алкоголиков.
Отсюда и 73% за Зеленского в 19-м. Но чтобы поддерживать режим Порошенко все 5
лет в должном тонусе, без уличных банд националистов было не обойтись. Они как
раз и были конечным силовым движком его режима.

Зеленскому вся эта структуру досталась по наследству. И с ней так просто не
попрощаешься. Ну это как с нашей космической программой. Не закрывать же было
сразу «Южмаш» в 1991-м? Он и сейчас есть. Только нихера не выпускает. Но в
массовом сознании «Южмаш» - это наша гордость. Самый крупный ракетостроительный
завод на планете. Был. Ну как его нах можно закрыть? Только гондон такое может
предложить. Но пояснить, как и кому он будет строить ракеты, ну и за какие шишы
– никто на планете не может.

Уличные банды националистов в массовом майданном сознании – это тот же «Южмаш».
Когда-то они приняли самое действенное участие в разгроме сепаратистского
движения в Украине. Но разгромить движение – это не разгромить само
сепаратистское настроение, которое, все равно в стране присутствует. А значит
эти банды в сознании майданной тусовки выполняют функцию «крепителя»
государства А.

Функции, к слову, бестолковые, как у тех же черносотенцев. Заменять
государственные институты параллельными ЧВК – это значит разрушать функционал
всей госсистемы. Те же большевики воспользовались «услугами» морячков и
вооруженных рабочих, но дальше уже создавали Красную армию на регулярной
основе. И Лех Валенса, придя к власти и вычистив силовые структуры от наиболее
раздражительных коммунистических элементов, не стал заменять их на вооруженные
отряды «Солидарности».

Уличным бандам в нашей стране уготован эволюционный этап отмирания,
приблизительно также, как демонтировался франкистский режим в Испании в 70-е.
Причина этому банальна – они рудимент. Пятое колесо, которое давно уже ничего
не скрепляет.

Правда, ряд факторов может реанимировать их значимость. Первый и от нас
независящий – это внешний. Ну, например, если наши братушки из Кремля решат еще
чем-то нам помочь. Расширить те же ЛДНР до их «законных границ». Ну или Байден
в случае своего избрания решит, что над горсоветом Донецка давно уже не реет
украинский флаг.

И второй фактор – это то самое внутреннее противостояние, о котором фантазирует
Кива. Во-первых, создавая подобные «отряды самообороны», ОПЗЖ просто
подставляет ребят. Ведь за ними не будет стоять СБУ, менты и провластная
политическая тусовка. А значит в случае первого лобового столкновения, одних
будут судить так, как судят Стерненко, а вот других уже по закону.

А значит отсюда выходит во-вторых. Подобные «отряды самообороны» могут быть
эффективны только при одном условии – если они сами будут
беспредельно-криминальны. Ну что-то типа «красных бригад» в Италии времен тех
же 70-х. А те могли и вокзал взорвать, и премьер-министра похитить и затем его
мертвым подкинуть в багажнике автомобиля.

Но то, что идея альтернативной улицы витает в воздухе – это факт. И то, что мы
сами – без внешней помощи, умеем разносить свое же государство – тоже факт. Но
спасибо очень дорогому мировому сообществу. Без них мы бы справлялись
значительно дольше.

\ii{03_07_2020.fb.lesev_igor.1.ulichnye_bandy.cmt}
