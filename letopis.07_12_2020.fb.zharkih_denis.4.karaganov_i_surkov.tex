% vim: keymap=russian-jcukenwin
%%beginhead 
 
%%file 07_12_2020.fb.zharkih_denis.4.karaganov_i_surkov
%%parent 07_12_2020
 
%%url https://www.facebook.com/permalink.php?story_fbid=2874041556142576&id=100006102787780
 
%%author Жарких, Денис
%%author_id zharkih_denis
%%author_url 
 
%%tags 
%%title Караганов и Сурков
 
%%endhead 
 
\subsection{Караганов и Сурков}
\label{sec:07_12_2020.fb.zharkih_denis.4.karaganov_i_surkov}
\Purl{https://www.facebook.com/permalink.php?story_fbid=2874041556142576&id=100006102787780}
\ifcmt
	author_begin
   author_id zharkih_denis
	author_end
\fi

Вдогонку поста о статье Караганова нашел пару минут написать, что же там такого
революционного. А для этого статью надо сравнить с прошлогодней статьей Суркова
\enquote{Долгое государство Путина} с которой \enquote{Наступление в войне идей} Караганова
явно спорит. 

Сурков, как истинный царедворец, прославляет Путина и поет долгие лета его
системе государства. Сам он выделяет четыре типа российского государства
Московию Ивана Третьего, Империю Петра Первого, СССР Ленина и Россию Путина.
При этом он совершенно не поддает анализу историческую роль этих государств.
Ну, были и были. Кстати, о России Путина в исторической ретроспективе можно
говорить только когда она закончится, сегодня важнее ее роль в мире, о которой
говорит Караганов. 

Ведь Россия переходила с низшего уровня на высший, от княжества до империи, от
империи к мировому полюсу влияния. Так вот что в этом отношении государство
Путина? Более высокая ступень или менее? Например, просто тормоз к общему
распаду и ухода с политической карты? Была же Россия Ивана Грозного, которая не
справилась с задачей выхода на более высокий уровень, но подготовила все к
рывку Петра Первого. С одной стороны Грозный не вывел Россию на новый уровень,
а с другой не дал сойти на нет. 

Я сейчас не хочу обсуждать эти вопросы, тут важнее другое. Караганов требует
каких-то действий, его мысль направлена в будущее, а Сурков хлопает в ладоши и
приговаривает \enquote{Ай да мы!}, он совершенно ни о каком будущем не говорит.
Путин сделал все на века, даже если сам этого не понимает (понятно, кто это
понимает).  Вот американцы дуются со своей демократией, а сами жлобы, а мы
давно жлобы, чего не стесняемся, и в этом наша сила. Ну это если на пальцах
мысли Суркова разложить.  Мы имеем два стиля, которые есть при любом управлении
- логический и эмоциональный. Сурков же не строит логических конструкций, он
намекает, соблазняет, вводит в ощущения. Это и есть византийство в чистом виде.
Для тех, кто забыл, напомню: на месте Константинополя, центра цивилизации
Раннего Средневековья, Стамбул, такой себе мусульманский город. Так что внутри
Кремля кипят нешуточные страсти. А нам остается наблюдать. Ну интересно ведь!
