% vim: keymap=russian-jcukenwin
%%beginhead 
 
%%file 14_07_2021.fb.krjukova_svetlana.1.statja_putina_mnenie.cmt.popelhov_novgorod
%%parent 14_07_2021.fb.krjukova_svetlana.1.statja_putina_mnenie.cmt
 
%%url 
 
%%author 
%%author_id 
%%author_url 
 
%%tags 
%%title 
 
%%endhead 
\paragraph{Михаил Попелхов - Великий Новгород - главный центр Древней Руси}

\begin{itemize}
%%%fbauth
%%%fbauth_name
\iusr{Михаил Попелхов}
%%%fbauth_url
%%%fbauth_place
%%%fbauth_id
%%%fbauth_front
%%%fbauth_desc
%%%fbauth_pic
%%%fbauth_pic portrait
%%%fbauth_pic background
%%%fbauth_pic other
%%%fbauth_tags
%%%endfbauth
 

Главным политическим и финансовым центром Древней Руси домонгольского периода
был Великий Новгород, где смыкаются границы 4 из 5 "РЯДОВ" (впоследствии,
"ПЯТИН") Новгородской земли, на начальном этапе формирования государственности
бывшей ФЕДЕРАЦИЕЙ ПЯТИ СЕВЕРНЫХ ПЛЕМЁН. 

ИСТИННЫМ ПРАВИТЕЛЕМ Древней Руси, с точки зрения новгородского "правительства
гостомыслов", являлся КНЯЗЬ-РЯДНИК, с которым заключал РЯД (ДОГОВОР) Великий
Новгород в лице "правительства гостомыслов". Перенос Военной Ставки клана
русских князей из клана Рюрика в Киев в 882 году (где она пробыла до 1155 года
и откуда была перенесена во Владимир-на-Клязьме) в этом плане ничего не менял. 

- Именно столичное правительство (Великого Новгорода) и легитимированные им
КНЯЗЬЯ-РЯДНИКИ определяли цели и методы осуществления экономической политики
Древнерусского государства. 

С точки зрения новгородского "правительства гостомыслов", (в
домонгольский период истории) князья из клана Рюрика провозглашённые "Великими
Князьями Киевскими" на Киевском Вече, но не являвшимися "новгородскими
рядниками", являлись КНЯЗЬЯМИ-УЗУРПАТОРАМИ, которые должны были быть смещены
(изгнаны из Киева или убиты) ЛЕГИТИМНЫМИ ВОЕННЫМИ ПРАВИТЕЛЯМИ
(КНЯЗЬЯМИ-РЯДНИКАМИ), примеры чего нам хорошо известны. 

Наибольшими финансовыми
средствами в этот период обладал Великий Новгород, выделявший "целевое
бюджетное финансирование" для снаряжения армии КНЯЗЯ-РЯДНИКА, отправлявшегося
из Новгорода свергать власть КНЯЗЯ-УЗУРПАТОРА, и содержания части гарнизонов
Киевского укрепрайона (Верхнего города Киева, Вышгорода и др.). 

Финансовые
ресурсы КНЯЗЕЙ-УЗУРПАТОРОВ позволяли им на начальном этапе вступления в права
"Великих Князей Киевских" выпустить "коллекционную" партию монет со своим
именем, но не позволяли им финансировать найм дополнительного количества войск
для победы в неизбежной гражданской войне между Новгородом и Киевом. 

- Для
найма дополнительных войск киевские КНЯЗЬЯ-УЗУРПАТОРЫ вынуждены были либо
расплачиваться с союзниками частью территории (так были отданы Польше
КНЯЗЕМ-УЗУРПАТОРОМ Святополком Окаянным Червенские города), либо отдавать
союзникам (печенегам и пр.) "на поток" (на разграбление) принадлежащие
Древнерусскому государству сёла. - БОЛЬШИЕ ДЕНЬГИ ГОСУДАРСТВА, позволявшие ему
при необходимости финансировать армию, НАХОДИЛИСЬ В РУКАХ Великого Новгорода. 

И Великий Новгород ("правительство гостомыслов") финансировал свои военные и
экономические проекты в Южной Руси в домонгольский период русской истории точно
так же, как это делали в более позднее время правительства московских царей,
петербургских императоров и московских генсеков. Русские летописцы не знают
слов "налоги" и "целевое бюджетное финансирование", а используют вместо обоих
этих понятий слово "дань", что и сбивает с толку наших историков и просто
обывателей, интересующихся историей нашего государства. 

- Где возникла Москва?

- Москва возникла в южной части "Большой Суздальщины", главным военным центром
которой (местом Военной Ставки русских князей клана Рюрика) с 1155 года являлся
Владимир-на-Клязьме. Впоследствии главный военный (и политический) центр
"Большой Суздальщины" был перенесён из Владимира-на-Клязьме в Москву. Поскольку
(главные) военные правители Древней Руси после её разгрома армией монголов
базировались сначала во Владимире-на-Клязьме, а затем в Москве и официально
признавались монгольскими ханами в качестве главных местных правителей, то они
и "тянули мазу" ( несли ответственность и собирали дань для выплаты ордынским
ханам) перед Ордой ЗА ВСЕ НАХОДИВШИЕСЯ ПОД ИХ РУКОЙ ТЕРРИТОРИИ.



\begin{itemize}
%%%fbauth
%%%fbauth_name
\iusr{Элина Медведева}
%%%fbauth_url
%%%fbauth_place
%%%fbauth_id
%%%fbauth_front
%%%fbauth_desc
%%%fbauth_pic
%%%fbauth_pic portrait
%%%fbauth_pic background
%%%fbauth_pic other
%%%fbauth_tags
%%%endfbauth
 
\textbf{Михаил Попелхов} крюковой это не интерестно, она заснет. Ей интерестнее фантазировать))
\end{itemize}


%%%fbauth
%%%fbauth_name
\iusr{Ольга Алчиева}
%%%fbauth_url
%%%fbauth_place
%%%fbauth_id
%%%fbauth_front
%%%fbauth_desc
%%%fbauth_pic
%%%fbauth_pic portrait
%%%fbauth_pic background
%%%fbauth_pic other
%%%fbauth_tags
%%%endfbauth
 

Когда народ Украины перестанет любить/ ненавидеть другие страны и их лидеров, а
начнет любить свою землю и работать на ее процветание, создавать взаимовыгодные
отношения со всеми странами? Мнение Светланы ни чуть не удивило, украинский
политикум субстанция пограничных взглядов вата- антивата, меняющихся иногда
туда-сюда, а среднего не дано?

%%%fbauth
%%%fbauth_name
\iusr{Александр Шамрай}
%%%fbauth_url
%%%fbauth_place
%%%fbauth_id
%%%fbauth_front
%%%fbauth_desc
%%%fbauth_pic
%%%fbauth_pic portrait
%%%fbauth_pic background
%%%fbauth_pic other
%%%fbauth_tags
%%%endfbauth
 

Это Ваша личная точка зрения, или ресурс Страна.уа тоже имеет к этому отношение?)

%%%fbauth
%%%fbauth_name
\iusr{Николай Моисеенко}
%%%fbauth_url
%%%fbauth_place
%%%fbauth_id
%%%fbauth_front
%%%fbauth_desc
%%%fbauth_pic
%%%fbauth_pic portrait
%%%fbauth_pic background
%%%fbauth_pic other
%%%fbauth_tags
%%%endfbauth
 

Оу, прям как газету "День" за 2006 открыл. Если такое на НАШем задвините,
Нацрада утонет в соплях счастья и даже квоты мовы оккупанта персонально
подымет.

%%%fbauth
%%%fbauth_name
\iusr{Michell Kane-Korso}
%%%fbauth_url
%%%fbauth_place
%%%fbauth_id
%%%fbauth_front
%%%fbauth_desc
%%%fbauth_pic
%%%fbauth_pic portrait
%%%fbauth_pic background
%%%fbauth_pic other
%%%fbauth_tags
%%%endfbauth
 

Светлана это лучший текст что я читала.Но простите за нескромность я б хотела
что б чтото подобное прочитал наш президент .И я уверенна что если б он вас
попросил воспользоватся вашим текстом что б огласить это миру вы б были не
против.

\begin{itemize}
%%%fbauth
%%%fbauth_name
\iusr{Геннадий Басов}
%%%fbauth_url
%%%fbauth_place
%%%fbauth_id
%%%fbauth_front
%%%fbauth_desc
%%%fbauth_pic
%%%fbauth_pic portrait
%%%fbauth_pic background
%%%fbauth_pic other
%%%fbauth_tags
%%%endfbauth
 
\textbf{Michell Kane-Korso} Ваш президент это кто? Если Зеля, то все потуги напрасны! У него с котом один мозг на двоих!

%%%fbauth
%%%fbauth_name
\iusr{Michell Kane-Korso}
%%%fbauth_url
%%%fbauth_place
%%%fbauth_id
%%%fbauth_front
%%%fbauth_desc
%%%fbauth_pic
%%%fbauth_pic portrait
%%%fbauth_pic background
%%%fbauth_pic other
%%%fbauth_tags
%%%endfbauth
 

\textbf{Геннадий Басов} я не гражданка Украины и никогда небыла ней ,если вы о выборе
президента Украины ,то я к этому выбору Зеленский ,или кто другой не
причасна. Очень странно что вы повернули в это русло тему .Значит я неправильно
выразила мыль .Те кто голосовал за Зе могли и за кота проголосовать лишь бы не
то что было раньше. Проходят сотли лет разные эпохи религии социум а люди верят
в ,,святую чесную политику,,. которая возможна. Я поражаюсь иногда мыслям
человечества и их оптимизму.

Я не знаю что скажут дипломаты отвечать ли Путину .Видимо ничего не отвечать. Но
если отвечать то текс Светланы прекрасен. В нём широкий спект образов и
истрии. Мне понравилось поэтому выразила своё восхищение.
\end{itemize}

%%%fbauth
%%%fbauth_name
\iusr{Лариса Полякова}
%%%fbauth_url
%%%fbauth_place
%%%fbauth_id
%%%fbauth_front
%%%fbauth_desc
%%%fbauth_pic
%%%fbauth_pic portrait
%%%fbauth_pic background
%%%fbauth_pic other
%%%fbauth_tags
%%%endfbauth
 

Флаг в руки Крюковой, сделавшей свой собственный вывод- анализ. Владение
буквами, не значит владение разумом. Пусть дальше развивает свои домыслы,
стремясь к счастью быть умной.

% -------------------------------------
\ii{fbauth.moroz_aleksandr.garapena.ukraina}
% -------------------------------------

Все гораздо проще. Нехер было тырить газ без Газпрома в начале независимости.

За все надо платить.

\end{itemize}

