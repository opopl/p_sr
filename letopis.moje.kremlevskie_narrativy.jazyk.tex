% vim: keymap=russian-jcukenwin
%%beginhead 
 
%%file moje.kremlevskie_narrativy.jazyk
%%parent moje.kremlevskie_narrativy
 
%%url 
 
%%author_id 
%%date 
 
%%tags 
%%title 
 
%%endhead 

\subsubsection{Положение русского языка в Киеве, в частности, и вообще в Украине}

Тут, к сожалению, ситуация совершенно удручающая с точки зрения поддержки
государством, на содержание которого мы исправно платим налоги. Язык, на
котором писал свои сочинения Григорий Саввич Сковорода, наш великий
философ-просветитель, который в свое время учился в Киевской Академии, и чей
памятник стоит на Подоле, как раз возле Андреевского Спуска, - по факту
систематически изгоняется из сферы образования, науки, культуры. Ну, точнее его
пытаются изгнать, унизить на государственном уровне, хотя априори очевидно, что
это абсолютно бессмысленная и глупая затея, поскольку Украина, - родина Николая
Гоголя и Тараса Шевченка, - как говорила на двух родственных языках, - на
русском и на украинском, так и будет говорить. Украина говорит на русском, - от
Ужгорода до Одессы, от Львова до Харькова, от Чернигова до Днепропетровска, от
Кировограда до Пирятина, от Сум до Херсона. Так было вчера, так есть сейчас, и
так будет всегда. Но... есть хорошая пословица - научи дурака молиться, и он
лоб расшибет об стену, и у нас действительно есть много дураков, очень
энергичных и пассионарных дураков, между прочим, которые несмотря ни на что,
продолжают заниматься этой глупой затеей (вытеснение русского языка и так
называемая украинизация), вместо того чтобы тратить свои духовные силы и свою
бесценную жизнь на что-то более творческое и созидательное. Плевки на русский
язык в Украине, - это примерно как плевки в потолок, лежа на диване.  Лежишь,
плюешься, - и чувствуешь себя хорошо. Хотя по факту, - просто диванный
идиотизм. И потолок-то придется же когда-нибудь вытирать!

\ifcmt
  tab_begin cols=2,no_fig,center
     pic https://scontent-frt3-1.xx.fbcdn.net/v/t39.30808-6/274088649_1938418763005602_9087982079840120782_n.jpg?_nc_cat=104&ccb=1-5&_nc_sid=730e14&_nc_ohc=LVcOvIp3pa8AX8eUnPc&_nc_ht=scontent-frt3-1.xx&oh=00_AT-nR4qMJHxWtl8z3Qr7sEqtYMlymi_1qaJIwBPkGHq4vA&oe=621D4471

     pic https://pbs.twimg.com/media/D6Y1aJ0WsAAzlva.jpg
  tab_end
\fi

Так вот, знаете, сейчас как раз уже 300 лет нашему
великому философу, и это поразительно, просто поразительно, что язык, на
котором он писал о самых сокровенных вещах - Человеке, Боге, Душе -
систематически оплевывается и унижается. Некоторые недалекие люди говорят, а
иные даже исступлено кричат, испуская потоки желчи по бескрайнему интернету,
что это язык московских оккупантов, говорят, что это московский (москвинский,
мокшанский) язык. Чушь собачья! Русский язык, раз уж вспомнили о Сковороде, это
- прежде всего язык Сковороды, поскольку на 90 процентов или более язык его
произведений совпадает со современным русским языком. Не верите? Ну так
почитайте сами, откройте в оригинале Сад Божественных Песен, глаза ж вам даны
не только для того, чтобы смотреть передачи ТСН, не так ли? Конечно, в его
произведениях есть примеси украинских (малороссийских) слов, и грамматические
конструкции немного старомодны, все-таки писал он более двух столетий назад; но
по факту, язык его произведений - с нашей (Киевской) точки зрения, это самый настоящий русский язык, это уж
будьте уверены. И значит, если унижают русский язык, то по факту унижают язык
Григория Саввича, а значит, это плевок в самого Сковороду.  Задумайтесь об
этом, господа патриоты. Вот, например, песнь 18-ая, красиво, правда?

\raggedcolumns
\begin{multicols}{2} % {
\setlength{\parindent}{0pt}
\obeycr
Ой ты, птичко жолтобоко,
Не клади гнезда высоко!
Клади на зеленой травке,
На молоденькой муравке.
\smallskip
От ястреб над головою
Висит, хочет ухватить,
Вашею живет он кровью,
От, от! кохти он острит!
\smallskip
Стоит явор над горою,
Все кивает головою.
Буйны ветры повевают,
Руки явору ломают.
\smallskip
А вербочки шумят низко,
Волокут мене до сна.
Тут течет поточок близко;
Видно воду аж до дна.
\smallskip
На что ж мне замышляти,
Что в селе родила мати?
Нехай у тех мозок рвется,
Кто высоко в гору дмется,
\smallskip
А я буду себе тихо
Коротати милый век.
\smallskip
Так минет мене все лихо,
Щастлив буду человек.
\restorecr
\end{multicols} % }

\ifcmt
tab_begin cols=2
  @caption Памятник Григорию Сковороде в городе Киеве

  ig https://avatars.mds.yandex.net/get-altay/2776464/2a00000170f4a99f4b8134bb851eb03ec81d/XXL
  ig https://ic.pics.livejournal.com/s_a_plotnikov/85688553/71932/71932_original.jpg

tab_end
\fi

Далее, что касается русского языка, является ли он своим или чужим, мы скажем,
что русский, - это один из двух основных Киевских языков, да, языков Города
Киева, наравне с украинским, на котором, кстати, написан гимн Киева, Як тебе не
любити, Києве Мій:

\raggedcolumns
\begin{multicols}{3} % {
\setlength{\parindent}{0pt}
\obeycr
Грає море зелене,
Тихий день догора.
Дорогими для мене
Стали схили Дніпра,
Де колишуться віти
Закоханих мрій
Як тебе не любити,
Києве мій!
\smallskip
В очі дивляться канни,
Серце в них переллю.
Хай розкажуть коханій,
Як я вірно люблю.
Буду мріяти й жити
На крилах надій
Як тебе не любити,
Києве мій!
\smallskip
Спить натомлене місто
Мирним, лагідним сном.
Ген вогні, як намисто,
Розцвіли над Дніпром.
Вечорів оксамити,
Мов щастя прибій
Як тебе не любити,
Києве мій!
\restorecr
\end{multicols} % }

\ifcmt
  tab_begin cols=3,no_fig,center
     pic https://i2.paste.pics/3435a4bd2b8d179dd31668aa7dd9c70f.png
     pic https://i2.paste.pics/78cf109cef08a466679e274394180c4c.png 
     pic https://i2.paste.pics/04f319315c3b814bc8cdca8acd50b23b.png
  tab_end
\fi

Ничуть не желая унизить или умалить украинский язык (мы про
него напишем в другом месте), мы тем не менее скажем, что русский - это Язык,
великий язык на котором говорят миллионы Киевлян каждый день. В разговорах в
семье, на улицах; в школах и на свиданиях; во время телефонных разговоров и
также во время разговоров по скайпу, во время переписки по вайберу или по
телеграму. На русском языке в Киеве говорят уборщицы и депутаты Верховной Рады;
говорят сотрудники СБУ и врачи скорой; говорят везде, и тут, и там. В
маршрутках говорят и в метро тоже; говорят в институтах Академии Наук, и в
Милицейской Академии тоже; Говорят зимой, и летом; осенью, и весной. На русском
языке матерится Петр Порошенко, когда ему вручают повестку, и на русском же
языке Антон Геращенко призывает россиян остановить Путина. Да, представте себе,
дамы Фарион и Ницой, в Киеве говорят на прекрасном, удивительном русском языке,
и воистину поразительно, что нам нужно вообще об этом писать и напоминать,
поскольку обилие и разнообразие разговорного и письменного русского языка в
Киеве - это очевидный факт. 

Киев - это возможно вообще самый богатый русским
языком Город в мире, и неудивительно, - поскольку в Киеве была же первая библиотека, первые школы
на Руси, и первый свод законов, - Русская Правда... И именно Киевляне - а
Киевлянин это не только тот человек, кто родился здесь, - а в более широком
смысле - это вообще любой человек, кто душевно связан с Киевом, кто любит Киев,
кому Киев дорог, - Булгаков, родившийся здесь, Гоголь, мечтавший занять кафедру
при Киевском Университете, Пушкин, бывавший здесь в юности, Высоцкий, гостивший
в детстве у своей киевской бабушки, - и многие другие - преображали русскую
культуру и язык, каждый по-своему выводя ее на новый уровень.  На русском языке
поют Киевляне Потап и Настя, на русском языке долгие годы выступает Студия
Квартал 95; на русском языке был снят замечательный сериал Слуга Народа - к
сожалению, он был использован Зеленским в итоге не в лучшую сторону... Молодой
талантливый русскоязычный актер, пришедший на волне народных чаяний, так изящно
и смешно пародировавший украинских политиков, и предавший, - увы, -
впоследствии все идеалы, на которых он пришел к власти...    

\ifcmt
  tab_begin cols=3,no_fig,center
     pic https://lh4.googleusercontent.com/-UBWNb92DrJY/ULExscthAxI/AAAAAAAAFNw/s7085Li2Sak/s900/20120617-114000.jpg
     pic https://lh6.googleusercontent.com/-bB20pabQmw4/ULExURIT2zI/AAAAAAAAFLw/_ypSk3bdu3w/s900/20120616-113915.jpg
     pic https://lh4.googleusercontent.com/-WZTmPXZnwak/ULExanzAt6I/AAAAAAAAFMQ/f1arqxd0jbU/s700/20120616-114441.jpg
  tab_end
\fi

Русский язык абсолютно Киеву органичен и Киеву совершенно свой. На русском
говорят и пишут Антон Геращенко и Арсен Аваков, Петр Толочко и Олеся Медведева,
Алексей Арестович и Борислав Береза, Олег Волошин и даже Юрий Бутусов. На
русском языке был написан призыв Мустафы Найема, с которого начался Майдан-2, а
значит, русский язык, - это язык Майдана.  На русском языке говорили воины
Красной Армии, которые ценой огромных жертв освободили Киев, и на русском языке
писал свои книги Михаил Булгаков. Это наш - Киевский язык, корни которого идут
прямиком в Киевскую Русь, во времена, когда на стенах Софиевского Собора
простые Киевляне оставляли свои надписи. Вы хотите сказать, что это все
староукраинский язык, а русский язык - это миф путинской пропаганды, что это
все навязанное, не свое, родное. Что была именно Украина-Русь, и именно так, а
не иначе.  Ну а мы тогда ответим, что Украина-Русь - это какой-то непонятный
конструкт, какой-то вообще выдуманный мираж. Был Киев, Стольный Град Киев,
который столько раз грабили и уничтожали, а он все возрождался, Киев,
объективный, реальный, вот он - здесь - прямо под ногами! 

\ifcmt
  tab_begin cols=4,no_fig,center

     pic https://scontent-frt3-1.xx.fbcdn.net/v/t39.30808-6/274298591_5398943100149786_6052439259617304698_n.jpg?_nc_cat=104&ccb=1-5&_nc_sid=730e14&_nc_ohc=fZoy-V3nYC8AX-cLUe7&_nc_ht=scontent-frt3-1.xx&oh=00_AT-_LmgNiwF-JpZVtqveovUTA6BjPpbCG_bUziH_5vFmqQ&oe=621E8523

     pic https://scontent-frt3-1.xx.fbcdn.net/v/t39.30808-6/274541079_5398687480175348_4058488227587905878_n.jpg?_nc_cat=108&ccb=1-5&_nc_sid=730e14&_nc_ohc=wJ1kHYt2DV4AX_xt40Q&_nc_ht=scontent-frt3-1.xx&oh=00_AT-vRH1Q7BKxNILDVCBEkh-x-KhByEpGZRIKiSvrbaaTjg&oe=621E4628

     pic https://scontent-frt3-2.xx.fbcdn.net/v/t39.30808-6/274078774_5390065007704262_9192112163879752157_n.jpg?_nc_cat=101&ccb=1-5&_nc_sid=730e14&_nc_ohc=c_5cGMcePbgAX_Yzw5l&_nc_ht=scontent-frt3-2.xx&oh=00_AT-Aey1JU_1W78A7j3JSnhJvISUI4nAoe4VC1rr9hvpfhQ&oe=621CC761

     pic https://scontent-frx5-2.xx.fbcdn.net/v/t39.30808-6/274154149_5390028534374576_4309070714717242551_n.jpg?_nc_cat=109&ccb=1-5&_nc_sid=730e14&_nc_ohc=d-Y3WzxY3kYAX-nZstG&tn=Sz-7xQ5oxc85gbiv&_nc_ht=scontent-frx5-2.xx&oh=00_AT9M3n4NvN-3nc9i1pDRqy9eYwUwzXUTr9NGSQttuTHRUg&oe=621E70BD

  tab_end
\fi

Град Киев, куда язык
человеческий вел, стоило только спросить...  а как нам добраться до Киева, не
подскажете? И тут же путнику охотно указывали, - сначала пройди вон тот лес, -
ты ж не балбес? - потом поверни направо, потом - налево, потом будет речка с
русалками, - потом будет еще один лес, - только гляди, такой тебе совет, - не
попади ты волкам на обед, - ну аж потом ты наконец-то увидишь Киев...  И он
здесь стоит уже тысячу лет, наш Киев, в котором были киевляне и князья, бояре и
смерды, тиуны и дружинники. И все это было связано со Псковом и Новгородом,
Суздалем и Ярославлем, и другими городами Северо-Восточной Руси.  И все это
вместе в массовом сознании было Русь. Просто Русь. А потом, - постепенно Русь
разделилась, раздробилась, раскололась на части, и каждая часть пошла своей
дорогой, - одна путем вольницы и освободительных войн, а другая - путем
жесткой, централизованной государственности и имперскости, - потом соединившись
вместе, а потом снова в наши времена разойдясь... А пресловутая Украина-Русь?
Вот именно так, Украина-Русь, или же Русь-Украина, через дефис, и никак иначе,
а то Грушевский на пару с Арестовичем обидятся?  Да чепуха это все. Был Киев -
а потом все остальное. И что сейчас важно, по-настоящему важно, это не то,
насколько именно сегодняшний Киевский русский язык подобен языку графити Софии
Киевской, а то, что Киев сейчас является невероятно замусоренным, загрязненным,
разбитым городом, чье нынешнее запущенное состояние (можно еще вспомнить про
Чернобыль рядом) абсолютно не соответствует его истинной значимости, являясь
наглядным примером порочности и примитивности нынешней украинской
государственной идеологии, а также культурного обнищания и духовной деградации
украинского народа в целом.

\ifcmt
  tab_begin cols=3,no_fig,center

     pic https://gordonua.com/img/article/15213/29_big.jpg?v1601729183
     pic https://avatars.mds.yandex.net/i?id=515efdebec88624056a69b00ecc6e3b0-5886792-images-thumbs&n=13
     pic https://avatars.mds.yandex.net/i?id=a951e988dcb2b7168b7301d363197d0a-5747194-images-thumbs&n=13

  tab_end
\fi

Далее, касательно притеснений русского языка и в Киеве, и по Украине, это
действительно правда, и Кремль здесь особо ни причем, поскольку же именно ты,
государство Украина, управляющее Городом Киевом последние 30 лет, хорошо
постаралась в этом, закрывая русские школы, вводя глупые украинизаторские
законы, и вообще систематически унижая русский язык - кровный язык Киева! - как
второстепенный, иностранный язык.  Знаешь, Украина, есть такая пословица:
насильно мил не будешь. Это касательно развития украинского языка. К сожалению,
тоталитарная репрессивная политика в отношении статуса русского языка и русской
культуры не привела к настоящему, духовному подъему украинской культуры и
языка, как ты хотела, не привела. Совсем наоборот.

\ifcmt
  tab_begin cols=3,no_fig,center
     pic https://i2.paste.pics/c4004e2585224bc31f681c37e8d849bc.png
     pic https://i2.paste.pics/5de7ab7812afc20b9788023f41992509.png
     pic https://i2.paste.pics/b7b038ded8df6885ce858fc8fc4054af.png
  tab_end
\fi

Alas! Украинский язык - певучий, удивительный язык, - проникающий в самую глубь
сердца, озаряющий самые потаенные уголки души, - стоит только открыть томик
Леси Украинки или же Олександра Олеся, или послушать какую-нибудь песню
Русланы, - ну например, старую добрую песню Свитанок, или же, Знаю Я -
превратился в государственный фетиш, в мертвый символ, в канцелярский язык, в
язык, на котором заставляют говорить насильно, в язык, который многие люди
вообще абсолютно искренне, от души ненавидят, поскольку он ассоциируется с
ненавистью, грабежами, убийствами и насилием, творимым незаконными военными
действиями ВСУ, а также с людоедскими заявлениями Фарион, Дроздова, Ницой, и их
многочисленных единомышленников, которые готовы резать москалей и малороссов,
которые повсюду видят предателей, живя во мраке мысленных чудовищ, порожденных
ими самими... Знаешь, Украина, апостол Павл сказал когда-то, что, \enquote{если
буду говорить красиво, но не буду иметь любви, то буду подобен меди звенящей},
- то есть, даже если ты говоришь самым изысканным украинским языком, но при
этом ты зол и глуп, и душа твоя пуста и не имеет сострадания и сочувствия, а
только бесконечное - росия-ворог, путин..., москалей-на-ножи, то какой толк
тогда в твоем украинском? Как ты думаешь, Украина, если бы например Леся Украинка перенеслась
машиной времени в наше время, она бы тоже орала кричалки, которые орут фанаты
на матче Украина-Россия?

\ifcmt
  tab_begin cols=3,no_fig,center
     pic https://avatars.mds.yandex.net/i?id=2a00000179f5e312985d9263bd58639efd1b-4613271-images-thumbs&n=13
     pic https://ye.ua/images/news/poster_photo_title-1496732268.jpeg
     pic https://www.megakniga.com.ua/uploads/cache/Products/Product_images_474227/0c4864_w600.jpg
  tab_end
\fi

Украинский язык в итоге стал скучным пропагандистским орудием... Он потерял
живительную силу сердца и души, потерял то, благодаря чему он жил и развивался
столетиями, несмотря на все преграды, превратившись в итоге в средство унижения
и ненависти. Парадоксально, но факт. Лучшие произведения на украинском языке
были созданы в те времена, когда Украина не имела своей государственности,
будучи разделенной между империями. Именно в те времена жили и творили Леся
Украинка, Иван Франко, Тарас Шевченко и многие другие. Они не имели никакого
понятия, что означает выражение Крым Наш, ничего не ведали про страну-агрессора
и про электоральные санкции Запада, но, тем не менее их сочинения будут жить
вечно... Получив же независимость, Украина дала государственный статус
украинскому языку, пнув при этом ногой русский. Пинок вначале был
интелигентный, мягкий, на уровне Конституции только; но постепенно удары
становились все жестче и жестче... все наглее и наглее... 

\ifcmt
  tab_begin cols=2,no_fig,center
     pic https://avatars.mds.yandex.net/get-zen_doc/1708669/pub_5cfdc85eaff15000afe57c46_5cfdc898babd4000b092e106/scale_1200
     pic https://img5tv.cdnvideo.ru/webp/shared/files/202104/1_1298133.jpg
  tab_end
\fi

И чем больше ты, Украина, пинала русский язык, насмехалась над русской
культурой, своей собственной культурой, тем больше ты, Южная Россия, на самом
деле зверела, тем больше ты теряла свою собственную душу, свое сердце, Украйна,
тем больше ты тупела, превращаясь в итоге просто в раздолбанную лодку без весла
и паруса, которую швыряет куда-сюда по просторам информационного океана, разве
не так? И что в итоге получилось на сегодняшний день, вы и так отлично знаете.
С другой стороны... русский-то язык в Украине стал только еще живее, еще
интереснее, еще богаче, чем был во времена имперские и советские, его бьют,
издеваются, а он только крепнет, и пускает все новые побеги, - одна только
Олеся Медведева уже более тысячи видео наваяла, - а если взять еще добавить
Шария... а вот украинский язык - потускнел, поверь, потемнел, почернел...
Знаете, много книжек нынче издается на украинском, можно зайти например в
Книгарню Є... потом, походу заметить где-то в глубине полочку с русскими
книжечками и интеллигентной маркировкой \enquote{Книги Іноземними Мовами},
потом пройти дальше, ибо то, что надо, украинская таможенная служба уже
отправила обратно прямо в Кремль... повыбирать украинских книжек и журналов,
сколько хочешь, - есть и очень хорошие, правда, но также немало просто
пропагандистского мусора...  Увы!  Ложка дегтя таки портит всю бочку с медом!
Держи-мова и генетический стержень расовых украинцев, едрена мать! 

Слыша сейчас
периодически звуки взрывов над Киевом и вой сирены на второй день войны, ну что
ж, скажем вам так - ваша теория правильных украинцев и умственно неполноценных
диких московитов, отлично работает!  Посеешь ветер, пожнешь бурю!  Хотели
злобных москалей, так получите! Мордор, орду и орков вызывали тысячами, если не
миллионами, комментов и постов по телеграму и фейсбуку? Так вот они уже тут,
прямо за Оболонью.
Российско-украинскую войну вызывали? Вот пожалуйста, они уже тут как тут,
резвятся вовсю по всей песочнице, и танки, и десант, и артиллерия, и бог знает
что еще. Люди-то понятливые, хотя и долго запрягают. Вы думали, что это вы -
воины света, а они - просто стадо тупых рабов? 

\ifcmt
  tab_begin cols=4,no_fig,center
     pic https://via-midgard.com/uploads/posts/2014-04/otvet-ukrainke-leonid-kornilov_2.jpg
     pic https://avatars.mds.yandex.net/get-zen_doc/3380298/pub_6217742eb81a3f57cece8769_62177a0302e17778b2ffe068/scale_1200
     pic https://rusvesna.su/sites/default/files/styles/orign_wm/public/1_dom_zhulyany_udar.jpg
     %pic https://gordonua.com/img/article/15972/41_big.jpg?v

     pic https://scontent-frt3-2.xx.fbcdn.net/v/t39.30808-6/274681293_5532411990106883_4041197137484819698_n.jpg?_nc_cat=101&ccb=1-5&_nc_sid=8bfeb9&_nc_ohc=5203f8z5YD8AX_1PBJz&_nc_ht=scontent-frt3-2.xx&oh=00_AT800GpJX7sXdwGt_YTtS6EOiuvVVgk63R87tReVezAl4Q&oe=621ED91B

  tab_end
\fi

Так они тоже нация воинов, и дело не только в приказах Путина, или его
собственном мировозрении, поверьте ( картинка с паспортом и солдатом взята с
одной публикации на Яндекс Дзене, самый обычный человек писал ). Вам нравилось
оскорблять русских, весь русский народ, плясать целыми площадями \enquote{хто
не скаче той москаль}, - нравилось \enquote{москаляку на гилляку}, нравилось
массово орать \enquote{путин - хйло}, думали, что обычные русские, от Москвы до
Владивостока, просто такие себе безвольные бараны без сердца, разума и глаз? Вы
ж наверное не думали, что это все возможно видит и смотрит вся Россия, видит не
только Кремль и МИД с Путиным и Захаровой, но также и простые московские или
саратовские пацаны или девушки, и также какая-нибудь бабушка в Иркутске? И вот
все ваши кричалки в итоге и материализовались... Анастасия Дмитрук так
интересно читала стихи про небратьев в свое время, вот и получила Настенька,
что хотела... Нравится тебе, Анастасия, что в Киеве на третий день войны
происходит? Ты в каком районе живешь, может быть, слышишь гром канонады над
ночным Киевом недалеко от больницы Охматдет, как будто праздничный салют в день
Киева или после победы Динамо Киев, да? Киев не бросила, надеемся, твой родной
город, в сей трудный час? У себя на кухне сидишь, коктейли Молотова готовишь с
друзьями? А... извини... мы так поняли, что ты во Львове? Ну ладно.

\ifcmt
  tab_begin cols=3,no_fig,center
     pic https://i2.paste.pics/803d6ba55e25b3932d422a2784a56d62.png
     pic https://i2.paste.pics/ba6f8da4cf889225670ad6c0a713e79c.png
     pic https://i2.paste.pics/03b693c3a796d20869846ac43a6ba7cf.png
  tab_end
\fi

Сорри, отвлеклись... продолжая тему... надо признать, что украинские
интеллектуалы, да и просто обычный народ, особенно его \enquote{патриотическая}
часть в массе своей загнали себя в тупик самого слова \enquote{Украина},
разделив свою реальность на Украину и Россию, проукраинские и пророссийские
нарративы, украинцев, малороссов, какаяразница, украина-понад-усе и тому
подобное... загнав поэтому и сам украинский язык в ловушку, в своего рода
информационное гетто, лишив его живительной силы сердца и искренности, разума и
мудрости, чего Украине так не хватает в наше время! А ведь украинский язык же
не появился просто так, - а напротив, он же ж именно такой и есть певучий и
красивый, поскольку в давние времена люди хотели вольно жить на своей земле,
хотели любить и жить в гармонии с природой, - но им приходилось все время
воевать и сражаться за право быть на своей земле, - и отсюда красота языка,
\enquote{смачность} языка,  и также божественная красота украинских песен, -
как отображение внутренних волевых, чистых порывов души и сердца. Вот,
например, такая есть песня с комментариями под видео... 

\href{https://www.youtube.com/watch?v=--iWRp8Jr6k}{%
Українські пісні онлайн. Чом ти не прийшов, youtube}

\raggedcolumns
\begin{multicols}{4} % {
\setlength{\parindent}{0pt}
\obeycr
Чом ти не прийшов,
Як місяць зійшов?
Я тебе чекала.
Чи коня не мав,
Чи стежки не знав,
Мати не пускала?
\smallskip
І коня я мав,
І стежку я знав,
І мати пускала.
Найменша сестра,
Бодай не зросла,
Сідельце сховала.
\smallskip
А старша сестра
Сідельце знайшла,
Коня осідлала:
\enquote{Поїдь, братику,
До дівчиноньки,
Що тебе чекала}.
\smallskip
Тече річенька
Невеличенька,
Схочу - перескочу.
Віддайте мене,
Моя матінко,
За кого я схочу.
\restorecr
\end{multicols} % }

\ifcmt
  tab_begin cols=2,no_fig,center
     pic https://i2.paste.pics/39b5912a65683fffc41863c68bfc5f26.png
     pic https://i2.paste.pics/dd441b74a982c23a31f5cd313bc89e3d.png
  tab_end
\fi

\begin{itemize} % {
\item Боже, как я люблю украинские песни. Я сразу возвращаюсь в детство, когда все
еще были живы-здоровы, а когда за столом собирались родственники..., украинские
песни лились рекой одна за другой.... как же было здорово!!!

\item Процветания тебе, Украина  @igg{fbicon.heart.sparkling} ... с Любовью и
уважением из России...

\item Люблю Украину и украинские песни. Они самые лучшие, душу выворачивают. Слова и
музыка завораживает. Класс!!!!!!!!!

\item Песня звучит, как звучат струны души у прекрасных и красивый людей. Хочется
слушая, подпевать, хотя  не  всё слова понимаю.  @igg{fbicon.hands.pray} 

\item Я сама русская. Но очень люблю украинские песни.

\item молодцы только украинцы так поют сказочно

\item Слушаю, подпеваю с удовольствием! У меня русско-украинские корни. Люблю Россию,
люблю Украину!!! Господь все управит, все у наших народов будет хорошо!!!
\end{itemize} % }

Чувствуете, как красиво звучит? А комментарии заметили, особенно из России?

Ну что ж, идем дальше. На данный момент, если собрать имеющуюся информацию
по проблеме украинского и русского языков (начало 2022 года, еще до войны), а
это десятки и сотни тысяч страниц текста, десятки тысяч самых разнообразных
публикаций, комментариев пользователей и разных видео/аудио материалов,
получится примерно следующая картина:

\begin{itemize} % {
\item (1) русский язык является повсеместно распространенным по всей территории
Украины, от Ужгорода до Харькова, от Чернигова до Одессы, от Киева до
Мелитополя, от Сум до Тернополя; русским языком свободно владеют представители
всех политических партий и движений, от крайне националистического Правого
Сектора, абсолютно вражески настроенного по отношению к России, и чьим лидером
в 2013-2015 годах был Дмитрий Ярош, уроженец Днепродзержинска, и Украинской
Добровольческой Армии, до Оппозиционной Платформы за Жизнь (ОПЗЖ), партии,
ориентированной на интересы жителей Юго-Востока Украины и выступающей за
дружественные отношения с Российской Федерацией;

\item (2) русский язык является одним из двух главных языков Украины, наравне с украинским;

\item (3) на русском больше говорят в Киеве, Харькове, Чернигове, - то есть
крупных городах, и вообще на всем юго-востоке, в то время как на Западной
Украине преобладает украинский; также, украинский больше распространен в
сельской местности, в том время как в городах - больше русский;

\item (4) украинский язык является единственным государственным; на нем ведется
делопроизводство, обучение в школах, университетах; 

\item (5) русский язык, на данным момент, являясь повсеместно распространенным
по всей территории Украинской Республики, тем не менее, в значительной степени
поражен в правах на государственном уровне. Несмотря на статью 10 Конституции
Украины, гарантирующую свободное развитие, использование и защиту русского
языка, а также статью 24, говорящую в частности о том, что не может быть
привилегий или ограничений по языковому признаку, к русскому языку в Украине на
государственном уровне, а также в массовом сознании многих людей, относятся как
к чему-то второстепенному, и даже как явлению вредному и ненужному.  На нем
массово не ведется обучение в университетах и в школах, русский язык не
считается частью официальной идеологии Украины как независимой державы; на
русском языке нет государственного гимна.  Что касается школ и дошкольного
образования, на настоящий момент, насколько нам известно, в Украине осталось
довольно незначительное количество русских школ и детсадиков. 

\item (6) Отдельно что касается Киева, русскоязычные киевляне по нашим
наблюдениям своем в массе своей отлично знают украинский, и не имеют
значительных предубеждений по поводу украинского языка, как отдельного
полноценного языка.

\item (7) касательно происхождения русского языка, часто в обсуждениях среди
простого народа, продвигается тезис о том, что украинский язык ярче, богаче
русского; что украинский древнее, чем русский, и что украинский является прямым
потомком языка, на котором говорили в Киевской Руси (образцы которого,
например, запечатлены на графити Софии Киевской), в то время как русский имеет
отношение в первую очередь к Москве, и к финно-угорским племенам меря и мокша,
а не к Киеву.

\ifcmt
  tab_begin cols=3,no_fig,center,resizebox=0.8
     pic https://pravlife.org/sites/default/files/field/image/graffiti1.jpg
     pic https://kiev-foto.info/images/statii/graffiti_sofii_5.jpg
     pic https://avatars.mds.yandex.net/i?id=073e8c7290d39f0f93fff00e54249a10-5234591-images-thumbs&n=13
  tab_end
\fi

\end{itemize} % }

Можно было б уже сделать целую книжку на десятки тысяч страниц, была
бы воля. Что касается обсуждения в народе этой проблемы, то можно выделить такие тезисы, как, например:

\begin{itemize} % {
\item русский язык часто называется языком московским, мокшанским, московитским - то есть, считается, по-видимому,
что он ведет свое начало от Москвы, что он также был исскуственно насажден в Украине;
\item также, русский язык называют языком оккупантов;
\item настойчиво проталкивается мысль о том, что во времена СССР была так
называемая \enquote{русификация}, что русскоязычные жители крупных городов,
таких, как Харьков, Одесса, Киев, являются \enquote{зросійщеними};
\item есть тезис о том, что Россия (\enquote{Московия}), в различные
исторические периоды постоянно и последовательно пыталась
\enquote{уничтожить} украинский язык и культуру, приводятся длинные
списки ограничений и запретов, так например Эмский указ или же
Валуевский циркуляр;
\end{itemize} % }

\ifcmt
  tab_begin cols=2,no_fig,center

     pic https://i.ytimg.com/vi/CrXtezJ8gKQ/maxresdefault.jpg
     @caption Екатерина Жарких, киевский журналист и блоггер

     %pic https://pbs.twimg.com/media/EB8RmtpXsAAucfs.jpg
     pic https://pbs.twimg.com/media/Esb3HRgXAAEuv6e.jpg
     @caption Евгения Бильченко, профессор культурологии, университет Драгоманова в Киеве 

  tab_end
\fi

Стоит заметить, что комментирование постов по языковую тематику (например, если
взять наиболее популярные посты Евгении Бильченко или Екатерины Жарких) обычно
сопровождается необычайным накалом эмоций, с откровенными оскорблениями
личности оппонента, что, безусловно, не есть хорошо. В ответ на природные
человеческие аргументы, защищающие русский язык и его носителей, например, что
русский язык в Украине имеет право на существование, что русский язык -
является для его носителей родным языком, на котором мама пела колыбельные, или
же что русский язык нужно сделать вторым государственным, поскольку его 
носители наравне вносят свой вклад в строительство государства, или же то, что
русский язык защищен 10-ой статьей Конституции, и тому подобное, в ответ можно
услышать такие словесные конструкции (некоторые есть буквальные комментарии
пользователей):

\begin{itemize} % {
\item Не подобається мова, геть до Ростова; 
\item Чемодан-вокзал-Россия;
\item Думай хоть на арабском или испанском, а если ты живёшь в Украине, то будь
любезная общайся, когда надо, на українській мові;
\item Тільки бидло не може вивчити мову своєї Країни;
\item Якщо ви живете в Україні, Ви повинні знати українську мову;
\item Той, що не знає української, прстійно мешкаючи в Україні називається розумово неповноцінним, а той, що не бажає ­ звичайним покидьком­шовіністом;
\item Ніколи не забороняли російську, а навпаки ­ більше 100 указів було заборони української мови в Україні і робили це з росії... ніхто не обурюється, як завжди, тільки москалі;
\item Для применения русского языка имеется огромная территория под названием РФ;
\item Ненавидять українську мову тільки малороси;
\item Тільки окупант татаромонгольского віросповідання не приймає національні традиції і всюди насаджує свій ЯЗЫК, який набагато молодший і бідніший за українску мову;
\item брєд говорити російською в Україні, а не українською, як і брєд, що в Україні половину рускоговорящих;
\item Всі руцкацєлюсниє та лаптєногіє, на родіну!!!!
\item марш звідси в ростов. Якшо не влаштовує українська мова вперед за поребрик;
\item Для допиту полонених рузька підійде:))
\item Російська мова як мова корінної титульної нації повинна бути в РФ. А в Україні ­ українська мова ­ мова корінної титульної нації;
\item На нашій планеті та, скоріш за все, і в усій Галактиці прийнято в тій країні де ти проживаєш розмовляти мовою тієї країни, а не своєю рідною;
\item Це взагалі дикунство жити в Україні і виступати проти мови корінного населення;
\end{itemize} % }

Как видно по эмоциальной окраске подобных комментариев, вопрос об
сосущенствовании украинского и русского языков является в Украине одним из
наиболее острых в психологическом плане. Украинский язык считается одним из
стержневых факторов Украинской государственности, самой гарантией независимости
Украины как суверенного государства, в то время как русский - наоборот, как
инструмент, подрывающий саму идею Украины, как государства, и ставящий под
сомнение само существование украинской нации и культуры, как совершенно
отдельных от русской нации и культуры. Люди, ратующие за русский язык в
Украине, а именно, за предоставление русскому языку статус государственного, за
то, чтобы русский язык преподавался в школах и университетах, объявляются
агентами Кремля/Путина, на них навешиваются ярлыки пророссийских/прокремлевских
партий/организаций. И с момента объявления независимой Украины, на
государственном уровне шел непрерывный процесс укрепления позиций украинского
языка, то есть, принятие дополнительных законов, таких как закон о тотальной
украинизации, или же введение квот на телевидении и радио. Также в школах,
постепенно, уменьшалось количество часов, выделенных на изучение русского языка
и литературы, русский язык и литература были объявлены как \enquote{иностранный
язык}, и \enquote{зарубежная литература}, соответственно, и есть предположение, что глубинной
целью этого процесса в Украине является вообще полное упразднение изучения
русского языка и культуры, совершенно и всецело, и как следствие, воспитание
поколения, говорящего исключительно по-украински.

\ifcmt
  tab_begin cols=2,no_fig,center
     @caption Голосеевский район, Киев, взято из интернета

     pic https://www.5travel.net/uploads/images/00/00/02/2013/05/16/00060c.jpg
     pic https://kievvlast.com.ua/project/resources/2018/05/LdYAFiyX.jpg
  tab_end
\fi

(Небольшое отступление) Знаете, сейчас уже пошел четвертый день войны,
воскресение. Светит солнце, птички поют. Мы видим наш район в Киеве таким же,
каким он был и до войны...  Но слышатся звуки взрывов вдалеке, и никого нет на
улицах. Только солнце, ясное небо, да легкий ветерок.  А что там происходит
вдалеке от нас, мы видим только по интернету, и по телеграму.  Сейчас
комендатский час, и никуда выходить нельзя, поэтому остается сидеть дома, пить
чай, и...  Периодически смотреть, что происходит...  Происходят разные ужасные
события, невероятные события. И мы думаем, а связано ли это все с тем, о чем мы
пишем, то есть, об языковой проблеме в Украине, и вообще об отношении к русским
и России? И знаете, думаем, что связано. Потому что...  посеешь ветер, а
пожнешь бурю. Посеешь характер - пожнешь судьбу. Слова это же не просто так.
Каждое, любое слово обладает определенной духовной составляющей, которая
направляет дальнейшие действия человека, и в итоге может вообще определить всю
дальнейшую судьбу человека. Ну, если бы дело было только в одном-единственном
человеке... Ну, сказанул ты что-то, ну и что? Проехали и забыли... Но. Людей-то
миллионы, и каждый день миллионы людей пишут, пишут...  пишут всякое. Я тебя
люблю, и я тебя ненавижу.  Ты козел, и ты душка. Россия - враг, украинцы - все
предатели. Москалей на ножи, - или же, а че там нынче в Хохляндии происходит?
Ну и... к чему вы ведем? Вот к чему. Поскольку мы Киевляне, нам важно, что
говорят и пишут именно в Киеве, в Украине, в первую очередь, а также то, как
это все воспринимается в мире. Нам бы хотелось, чтобы Украину в мире любили и
уважали... Так вот. Если постоянно говорить со всех колоколен, если тысячи,
сотни тысяч людей каждый день будут думать, какая Россия ужасная и дикая
страна, что Россия - вечный злобный враг, что русские - все поголовно отсталые
дикие люди, что они всегда хотели уничтожить Украину, - несмотря на то, что
(скажем, в году 2018) никто Киев не бомбит, а дети спокойно резвятся на пляже в
Одессе, - как вы думаете, к чему это может привести в итоге? Если тысячу раз целой страной
повторять как буддистскую мантру - орки, мордор, путлер, пуйло, вата, лугандон, смерть-ворогам, росия-ворог, вы
же не думали, что когда-нибудь быль станет явью, - что Васильков будет в огне,
в Харьков войдут российские танки, что по всей Украине будут наступать
российские войска, а над Киевом будут грохотать взрывы. Вы не думали, да? 

\ifcmt
  tab_begin cols=3,no_fig,center
     pic https://s00.yaplakal.com/pics/pics_original/2/9/0/12291092.jpg
     pic https://a-a-ah-ru.s3.amazonaws.com/uploads/items/106922/132212/large_597e597d437b4549831988.jpg
     pic https://wallbox.ru/resize/1280x1024/wallpapers/main/201141/multfilmy-0b2495245387.jpg
  tab_end
\fi

И если каждый день оскорблять Путина и русских идиотскими песнями, вы ж не
думали, что он тоже человек, как и весь русский народ состоит из людей, а не
баранов, или орков, что довольно много русских действительно любят Путина,
потому что он их президент, и банально у него (них) может просто лопнуть всякое
терпение, да? -  и тут еще мы не вспомнили о конфликте на Донбассе, который
есть огромная трагедия. Если каждый день усердно, искренне от души взывать к
богу войны, купаясь в злых словах и нескончаемой желчи, он же придет
когда-нибудь. Вот он и пришел... И если каждый божий день тыкать иголками в
русского медведя, в общем-то довольно дружелюбного и миролюбивого, несмотря на
грозный внешний вид, и желающего, чтобы его оставили в покое, потому что его
ждет малина и ежевика вон в том лесу, он же ж может и взвыть в ярости в
конце-концов, и броситься на обидчика, разве не так? И если веселая Маша из
мультика Маша и Медведь, вместо того чтобы просто играть с Медведем в разные
интересные игры, начнет исподтишка засовывать с невинным видом острые иголки в
спину, как вы думаете, разве Медведь в какой-то момент не заревет и не выбросит
пинком Машу в ближайший лес?

\ifcmt
  tab_begin cols=2,no_fig,center

     pic https://upload.wikimedia.org/wikipedia/commons/thumb/0/00/%D0%9F%D0%BE%D1%81%D0%BB%D0%B5%D0%B4%D1%81%D1%82%D0%B2%D0%B8%D1%8F_%D1%83%D0%B4%D0%B0%D1%80%D0%B0_%D1%80%D0%B0%D0%BA%D0%B5%D1%82%D1%8B_%D0%BF%D0%BE_%D0%93%D0%BE%D0%BB%D0%BE%D1%81%D0%B5%D0%B5%D0%B2%D1%81%D0%BA%D0%BE%D0%BC%D1%83_%D1%80%D0%B0%D0%B9%D0%BE%D0%BD%D1%83_%D0%9A%D0%B8%D0%B5%D0%B2%D0%B0_%285%29.jpg/1280px-%D0%9F%D0%BE%D1%81%D0%BB%D0%B5%D0%B4%D1%81%D1%82%D0%B2%D0%B8%D1%8F_%D1%83%D0%B4%D0%B0%D1%80%D0%B0_%D1%80%D0%B0%D0%BA%D0%B5%D1%82%D1%8B_%D0%BF%D0%BE_%D0%93%D0%BE%D0%BB%D0%BE%D1%81%D0%B5%D0%B5%D0%B2%D1%81%D0%BA%D0%BE%D0%BC%D1%83_%D1%80%D0%B0%D0%B9%D0%BE%D0%BD%D1%83_%D0%9A%D0%B8%D0%B5%D0%B2%D0%B0_%285%29.jpg
     @caption Последствия падения ракеты в Голосеевском районе Киева, война, 2022
     
     pic https://kievvlast.com.ua/project/resources/2022/02/H1bfawdO.jpg
     @caption ул. Кошица, 7а в Дарницком районе, Киев, война, 2022

  tab_end
\fi

Ну что ж. Вы наверное хотите сказать, что мы оправдываем Путина и россиян,
страстно желающих навести порядок в Украине, отомстить за детей Донбасса и за
одесскую трагедию второго мая, и вычистить оттуда всяких нацистов и
бандеровцев? Провести денацификацию и демилитаризацию? Вы сидите в Киеве, ваш
город атакуют, перепуганные люди пачками сидят в метро и подвалах, и вы еще
смеете сказать что-то хорошее про Россию? Да вы никакие не украинцы, не
патриоты, так, ватники, малоросы, жалкие трусы. Расстрелять бы вас по законам
военного времени. Вместо того, чтобы записаться в тероборону, вы сидите на
диване, читаете книжки, пьете чай, и молча задумчиво смотрите на солнышко,
прислушиваясь к взрывам вдалеке.  Ну...  мы поправим, - мы Киевляне.  И здесь
Киевляне живут уже тысячу лет, да, на этих прекрасных склонах Днепра есть город
Киев, и в нем живут Киевляне.  А вы че, не знали, панове украинцы?  Оправдываем
- не оправдываем... Если вам так хочется услышать, то, что вы хотите услышать,
безусловно, вы услышите.  Попадание в жилые дома - это преступление со стороны
Российской Федерации.  Все, что вообще происходит сейчас в Украине - это ужасно
и это преступление Российской Федерации, и лично Путина, со всех точек зрения,
и международного права, и общечеловеческих ценностей, и Путин безусловно
понесет ответственность, и Россия уже несет колоссальные убытки по всем
фронтам, и вполне возможно, его даже сместят или убьют в итоге, если Россия
начнет массово бунтовать против войны. Что будет дальше, никто не знает на
данный момент вечера воскресения 27 февраля 2022. И мы, сидящие взаперти в
киевских квартирах, точно не знаем, как оно будет. Война, которую сейчас ведет
Украина против российских войск на своей территории, - это война праведная, за
Отчизну. Именно так это сейчас воспринимается в Украине, и особенно в Киеве.
Спящий дух Хмельницкого и Кривоноса, гайдамаков и защитников Киева 1240 года,
пробуждается, русские незваные пришельцы. 

\ifcmt
  tab_begin cols=2,no_fig,center
     pic https://avatars.mds.yandex.net/get-zen_doc/3636601/pub_5ffa0df7f906b16872c5559a_5ffb4d2af906b1687279a87c/orig
     pic https://irecommend.ru/sites/default/files/imagecache/copyright1/user-images/198291/ItHmqPWINdp2HJZYPIWaKA.jpg
  tab_end
\fi

Обычные оболонские пацаны, которые в мирное время обычно собираются где-нибудь
попить пивка и послушать какой-нибудь незамысловатый музон, или же поиграть на
гитарке для девчонки из дома напротив, готовы надрать вам зад, парни из
Грозного и Иркутстка, и уже таки здорово надирают. Да, конечно, вы - храбрые
русские воины, и ничего не боитесь, но и они - воины. Обычные Киевские
богатыри. Коса нашла на камень...  Они могут быть абсолютно равнодушны к
Бандере и даже к этой гребаной мове, и думать, что Тарас Шевченко был пьяница и
полный отстой, но они просто хотят надрать вам зад, они очень злы, и они хотят
отправить вас в ад, захватчики. И они не сдадутся, никогда. Потому что
отступать некуда, потому что это Киев, это Киев. Ладно... как оно будет дальше
- мы не знаем. Допиваем чай, и возвращаемся к теме языка.

А что насчет языка? Надо бы закончить как-нибудь, и перейти к другим темам. Как
говорил Гоголь, перед вами громада - русский язык! Не трогала б ты, Украина,
русский язык и русскую культуру, - не было бы войны. Не оскорбляла бы россиян,
Россию, русскую культуру тысячами комментов, и ведрами помоев, не было бы
вторжения. Не орали бы военные на параде в Киеве песенку про Путина, - Путин
был бы добрее к нам, он же ж тоже человек, и украинский язык и культуру, и
Украину, он, мы уверены, в душе любит... Еще тут можно добавить...  Не орали бы
протестующие толпами на Майдане \enquote{хто не скаче - той москаль}, - не
утопили бы мирные революционеры в Киеве дом Профсоюзов в огне, не выламывали бы
из улицы Институтской булыжники, которые эти люди не клали, - зачем было
портить то, что ты не строил? - не было бы ни отторжения Крыма, ни восстания на
Донбассе. Потому что посеешь ветер - пожнешь бурю, слово не воробей - вылетит,
не поймаешь. Семь раз отмерь - один раз отрежь. И лучше синица в руках, чем
журавль в небе. И не стоит лезть со своим уставом в чужой монастырь, если тебя
об этом не просили. Лучше вежливо постучать сначала.

И вместо того, чтобы плакать за павшими защитниками Отчизны, и за сгоревший
самый большой самолет Мрия в Гостомеле, ты сейчас бы уже строила наверное уже
десятую по счету Мрию, и все было бы хорошо и расчудесно. Нам в Киеве на
четвертый день войны, сидя за ужином и чашкой гарячего какао, это так кажется.
Никуда выходить нельзя, так чем же еще заниматься, как не размышлениями и
разговорами о том, о сем?  Может быть, завтра будет еще все намного хуже, хотя
вроде назначили переговоры в Припяти, и неизвестно, сколько нам еще отпущено
жить, но там, где мы живем, сегодня было тихо. И война-войной, а ужин по
расписанию.

А Вам как кажется?

\ifcmt
  tab_begin cols=2,no_fig,center
     pic https://avatars.mds.yandex.net/i?id=2a0000017a066a147e86318381462bc5d895-4936013-images-thumbs&n=13
     pic https://i.pinimg.com/originals/00/ce/23/00ce23a5a6393b760d9a0e7371d340bf.jpg
  tab_end
\fi
