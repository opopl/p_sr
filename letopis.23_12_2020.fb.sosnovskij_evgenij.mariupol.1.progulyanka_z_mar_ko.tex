%%beginhead 
 
%%file 23_12_2020.fb.sosnovskij_evgenij.mariupol.1.progulyanka_z_mar_ko
%%parent 23_12_2020
 
%%url https://www.facebook.com/evgeny.sosnovsky/posts/pfbid024njBgKoXRDGuFm8BEEevn2xHLN3PByrZcknCZjSY4ebjNmkYAbQPtERgnjAYFNdrl
 
%%author_id sosnovskij_evgenij.mariupol
%%date 23_12_2020
 
%%tags kniga,kniga.mariupol.proguljanka_z_marikom,mariupol,mariupol.pre_war,2020,podarki,chtenie,literatura,kultura,novyj_god
%%title Прогулянка з Маріком - маріупольський подарунок від Оксана Стомина та всієї її команди
 
%%endhead 

\subsection{Прогулянка з Маріком - маріупольський подарунок від Оксана Стомина та всієї її команди}
\label{sec:23_12_2020.fb.sosnovskij_evgenij.mariupol.1.progulyanka_z_mar_ko}

\Purl{https://www.facebook.com/evgeny.sosnovsky/posts/pfbid024njBgKoXRDGuFm8BEEevn2xHLN3PByrZcknCZjSY4ebjNmkYAbQPtERgnjAYFNdrl}
\ifcmt
 author_begin
   author_id sosnovskij_evgenij.mariupol
 author_end
\fi

До Нового Року ще тиждень, а я на додаток до цілого комплекту ігр від
\href{https://www.facebook.com/Cultural.and.educational.platform.sense}{Культурно-освітня
платформа \enquote{Сенс}}  отримав під ялинку ще один чудовий і теж
по-справжньому маріупольський подарунок від
\href{https://www.facebook.com/oksana.stomina}{Оксана Стомина} та всієї її
команди, яка працювала над книгою \enquote{Прогулянка з Маріком}. І мені
здається, що жодне місто України не має такого унікального видання, яке
створили маріупольці, закохані у своє місто!

Тож тепер я впевнений, що ніколи не заблукаю вулицями Старого Маріуполя, бо зі
мною буде цей веселий путівник, який із захопленням написали Оксана та Юлія
Стоміни, неперевершено проілюструвала
\href{https://www.facebook.com/profile.php?id=100000576599040}{Anastasiya
Ponomareva}, дуже корректно відкоректувала
\href{https://www.facebook.com/tetiana.petrushkina}{Татьяна Петрушкина},
професійно протестувала 8-річна Емілія Балжи і виданням якого по-керівницьки
керував Дмитрий Паскалов (ГО \enquote{Паперові Сходи}).

І окрема подяка всім тим, хто зберіг і зберігає історію нашого міста і завдяки
кому стало можливим видання такої книги: Лев Яруцький, Аркадій Проценко,
\href{https://www.facebook.com/profile.php?id=100005851476252}{Сергей Буров},
\href{https://www.facebook.com/bogko.raisa}{Раиса Божко}, Олена Дейнеченко,
\href{https://www.facebook.com/vadim.dzuvaga}{Вадим Джувага},
\href{https://www.facebook.com/profile.php?id=100009080371413}{Olga Demidko}.

\#тутварто бачити \#Маріуполь \#Mariupol \#Мариуполь \#ТакеКрасивеМісто \#ПрогулянкиЗМаріком \#ПаперовіСходи
