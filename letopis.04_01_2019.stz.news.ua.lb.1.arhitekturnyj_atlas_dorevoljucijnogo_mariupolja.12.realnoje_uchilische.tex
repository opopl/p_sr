% vim: keymap=russian-jcukenwin
%%beginhead 
 
%%file 04_01_2019.stz.news.ua.lb.1.arhitekturnyj_atlas_dorevoljucijnogo_mariupolja.12.realnoje_uchilische
%%parent 04_01_2019.stz.news.ua.lb.1.arhitekturnyj_atlas_dorevoljucijnogo_mariupolja
 
%%url 
 
%%author_id 
%%date 
 
%%tags 
%%title 
 
%%endhead 

\subsubsection{Реальное училище}

Здание еще одного дореволюционного учебного заведения стоит на углу
Николаевской и Торговой улиц. Трехэтажный дом, принадлежавший Оксюзову, в 1906
году арендовало частное реальное училище Василия Ивановича Гиацинтова. Училище
функционировало до 20-х годов ХХ века, при советской власти его ликвидировали,
а в помещениях стали заниматься школьники. Во время оккупации здание заняли под
казарму немцы, а отступая из города осенью 1943 года его сожгли. Кирпичный
сгоревший остов простоял на перекрестке до начала 60-х годов, пока дождался
своей очереди на восстановление. В наше время помещения бывшего реального
училища сдавались в аренду. Совсем недавно практически полностью дом заняло
представительство \enquote{Международного Красного Креста}. Именно это событие можно
назвать счастливым случаем в жизни здания. Фасад капитально отреставрировали и
теперь дом на перекрестке выглядит как с открытки. Красный Крест спасает не
только людей.

\ii{04_01_2019.stz.news.ua.lb.1.arhitekturnyj_atlas_dorevoljucijnogo_mariupolja.12.realnoje_uchilische.pic.1}
\ii{04_01_2019.stz.news.ua.lb.1.arhitekturnyj_atlas_dorevoljucijnogo_mariupolja.12.realnoje_uchilische.pic.2}
\ii{04_01_2019.stz.news.ua.lb.1.arhitekturnyj_atlas_dorevoljucijnogo_mariupolja.12.realnoje_uchilische.pic.3}
\ii{04_01_2019.stz.news.ua.lb.1.arhitekturnyj_atlas_dorevoljucijnogo_mariupolja.12.realnoje_uchilische.pic.4}
