% vim: keymap=russian-jcukenwin
%%beginhead 
 
%%file moje.prizyv
%%parent moje
 
%%url 
 
%%author_id 
%%date 
 
%%tags 
%%title 
 
%%endhead 

\section{Обращение жителей города Киева относительно нынешней российско-украинской войны}

Добрый день. Не знаем, как лучше начать этот текст, но давайте все-таки
как-нибудь начнем. Сейчас вечер 28 февраля 2022 года, и уже вовсю идет по всей
территории Украины кровопролитная война. Кровопролитная, ужасная война. Война
на истощение, война просто на убой. Война, которую начала Российская Федерация
против Украины. Да, это ужасно, что вторжение, которое многие считали абсолютно
мнимым, таки состоялось. Мы, если честно признаться, смеялись над непрерывным
потоком сообщений в западных СМИ о готовящемся вторжении, и мы ничуть не
верили, что война таки произойдет. Так вот... то, во что абсолютно не верилось,
таки состоялось.

\ifcmt
  tab_begin cols=2,no_fig,center
     pic https://avatars.mds.yandex.net/i?id=4ec9211cea2cb58ee6ce92ab1a6c3a1f-5870172-images-thumbs&n=13
		 pic https://ic.pics.livejournal.com/deletant/25270605/1797309/1797309_original.jpg
  tab_end
\fi

Война, которая унесла, насколько мы можем судить по непрерывным сообщениям в
СМИ, уже тысячи жизней. И счетчик унесенных жизней, счетчик увечий, ранений,
счетчик искалеченных судеб непрерывно растет.  Говорят, что россиян убито уже
около четырех тысяч, а сколько убито граждан Украины, наших храбрых защитников,
наших настоящих Богатырей, мы не знаем, но думаем, что приблизительно того же
порядка.

Позвольте представиться. Мы - Киевляне, жители тысячелетнего города Киева,
Матери Городам Русским, великого Города, в котором находятся наши тысячелетние
святыни - Киево-Печерская Лавра, София Киевская, и мы сейчас сидим взаперти в
одной из киевских квартир... в одном из районов Киева. У нас относительно тихо
на данный момент, и слава Богу. Есть электричество, водопроводная вода, у нас
есть запасы продуктов на нескольке недель, так что мы пока что еще не умираем с
голоду, и вообще живы. Просто живы, и это уже круто. В наш дом не попала
ракета, и знаете, такое ощущение, что вообще почти ничего не происходит, если
не читать сообщения из интернета и телеграма. Днем светит солнышко над нашим
Вечным Городом, Городом Ярослава Мудрого и Владимира Крестителя... Города
Булгакова и Пушкина, Гоголя и Высоцкого, Анны Ахматовой и Николая Амосова...
Тысячелетнего Города Киева, в котором столько всего происходило на протяжении
тысячелетий, Города... который столько раз был в огне, который столько раз
подвергался разорению и разграблению, и тем не менее, он каждый раз
возрождался. Возрождался, чтобы стать еще красивее. В 1240 году этот Город был
сожжен дотла, а сейчас - это прекрасный красивый Город. И кстати, Город,
который во многом строила Российская Империя, если вы помните историю, - так,
например, Андреевскую церковь в свое время построили по приказу Императрицы
Елизаветы Петровны в 1749-1754 годах на том месте, где, по преданию, апостол
Андрей Первозванный во время его путешествия воздвиг крест. Как вы может быть
знаете, об этом есть запись в Повести Временных Лет, 

Так вот. Снаружи поют птички, предвещая наступления скорой весны... как будто
жизнь идет своим чередом до начала войны, как будто вообще ничего не
происходило. Кроме того...  что улицы пустынны, что наш район опустел, как
будто все такое же самое. Кроме того, что... вдалеке воет сирена, 

Однако мы знаем, сколько сейчас несчастий происходит по всей территории
Украины. Ну что ж, надо что-то делать. Надо что-то делать, поскольку похоже все
движется к тому, что мы - украинцы и россияне, на самом деле будучи одной
крови, поскольку мы все славяне, просто перестреляем, перережем друг друга в
кровавом месиве, поскольку с обоих сторон идет мощная поддержка ресурсами,
оружием, людской поддержкой, информационной пропагандой. С нашей стороны идут
сообщения о том, что мы мочим врага, что мы обязательно победим, что наше дело
правое, что оккупант будет изгнан, и что вы все сгинете в аду. У нас о войне
пишут на русском и украинском языках, у вас - на русском. 

С одной стороны Украине помогает - Запад, и тот факт, что эта война для нас -
это война за Отчизну, это война справедливая, Отечественная. Да, Украинская
Отечественная Война. Как бы кто не называл эту войну, например используя термин
\enquote{cпециальная военная операция на Украине}, - или как там Президент
Российской Федерации Владимир Владимирович Путин ее назвал, - это война для нас
есть священная война за Отчизну, а для Российской Федерации - это война
захватническая, несправедливая, сколько бы вы у себя не повторяли про нацистов
и бандеровцев. 

Ну а Россия использует все свои ресурсы, поставив на карту, все что у России
есть.  Мы вас хорошо понимаем, Россияне. У вас есть весомые аргументы
относительно того, что НАТО угрожает вам ракетами, что есть угроза того, что
ракеты НАТО будут размещены на территории Украины, и что время подлета к Москве
будет весьма незначительным. Это мы хорошо понимаем. 

С одной стороны,
