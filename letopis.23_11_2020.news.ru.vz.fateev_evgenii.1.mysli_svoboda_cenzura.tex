% vim: keymap=russian-jcukenwin
%%beginhead 
 
%%file 23_11_2020.news.ru.vz.fateev_evgenii.1.mysli_svoboda_cenzura
%%parent 23_11_2020
 
%%url https://m.vz.ru/opinions/2020/11/23/1071901.html
 
%%author Фатеев, Евгений
%%author_id fateev_evgenii
%%author_url 
 
%%tags cenzura,svoboda
%%title Жрецы «свободы» цензурируют яйцеголовых
 
%%endhead 
 
\subsection{Жрецы «свободы» цензурируют яйцеголовых}
\label{sec:23_11_2020.news.ru.vz.fateev_evgenii.1.mysli_svoboda_cenzura}
\Purl{https://m.vz.ru/opinions/2020/11/23/1071901.html}
\ifcmt
	author_begin
   author_id fateev_evgenii
	author_end
\fi

\ii{pic.author.fateev_evgenii}

\ifcmt
pic https://img.vz.ru/upimg/soc/soc_1071901.jpg
\fi

Пора уже перестать удивляться тому, что всемирный демократический спектакль
саморазоблачился. Надеюсь, те, кто еще имел какие-то иллюзии, уже благополучно
от них избавились. Те же, кто до сих пор пребывает в плену иллюзий, не
выберутся из этого плена уже никогда. Этих мы уже потеряли.

Глобальные соцсетевые среды – это уже не пространства свободы. Там, очевидно,
живет цензура. Есть некоторая водевильность и затасканность в этом слове –
«цензура», но тут ничего не попишешь. Глобальные соцсети – абсолютно
дисциплинарные пространства с неминуемыми санкциями. В глобальных соцсетях
осуществляется самая настоящая социнженерия.

Нужно как-то привыкнуть к тому, что в момент успешности нелиберальный канал без
малейших объяснений снесут в Youtube. Уже многие и многие привыкли к регулярным
банам в Facebook по самым странным и неожиданным поводам. Причем в группе риска
пока находятся как раз успешные нелиберальные проекты.

Убивается сама возможность истории медийного успеха для избранных – в плохом
смысле слова. И это не столь невинные вещи, как многим кажется. Подается знак
всем молодым честолюбцам: не суйтесь за флажки допустимого, иначе лишитесь
всего, что строили и развивали не один год. Средствами соцсетевой монетизации
создается экономика либеральной экспансии.

Пожалуй, относительно свободными соцсетевыми площадками остаются Telegram,
«ВКонтакте» и «Одноклассники». До сих пор не совсем ясно, телеграмная свобода –
это еще эхо относительно недавнего стратапа этой соцсети, а все соцсети
начинались как места жительства свободы мнений, или осознанная политика? И что
будет, если владелец вдруг решит продаться большим глобальным дядям? И не
захочет ли он тоже начать игру в какие-нибудь мутные «алгоритмы»?

Мои друзья из одной очень успешной региональной телекомпании озаботились
проблемой диверсификации. Три года назад Youtube уже сносил их успешный канал с
несколькими сотнями тысяч подписчиков. Ребята оправились весьма быстро и
создали новый канал плюс подстраховочный. Сейчас у них 420 тыс. подписчиков в
Youtube, общее количество просмотров около 1 млрд, весьма успешные паблики в
«ВКонтакте» и «Одноклассниках». Но закавыка в том, что это патриотические
региональные новости, а значит, они всегда готовы к тому, что их без малейших
объяснений в любой момент могут снести. Именно так было с Anna news, News
front, «Царьградом»... И вот две недели назад мои друзья попробовали
приземлиться еще и на новые соцсетевые площадки – «Яндекс.Эфир» и «Тик-Ток». И
что же из этого вышло? 

Разумеется, «Тик-Ток» – это очень особенная и еще относительно новая площадка.
Мои друзья загрузили десять коротких видео. Три из них платформа не пропустила
и отцензурировала. Это, кстати, в пределах нормы. На Youtube у них примерно
такой же процент отсева. Зато в течение недели ролики на «Тик-Ток» буквально
взлетели, собрав 3 млн просмотров.

Самое интересное случилось с «Яндекс.Эфиром», который вроде бы наш,
отечественный, суверенный. Во всяком случае, у нас многие так считают. Из 30
загруженных видео в эфир были пропущены лишь три. Остальные – заблокированы.
90\% процентов видео не были выпущены! Просмотровая же статистика там была у
моих друзей минимальной – всего 12 просмотров. Не тысяч, не сотен, а просто 12.
Для сравнения, за тот же период в Youtube те же самые ролики собрали 3 млн
просмотров. На «Тик-Токе», напомню, тоже 3 млн.

Хочется надеяться, что дело пока в не до конца отстроенных алгоритмах –
все-таки площадка новая. И, кажется, уже только ленивый не говорит о
необходимости русского Youtube, но пока мы видим, что происходит.

Единственной угрозой для свободы слова считалось государство, которое
злонамеренное и вообще. Составная часть этой злонамеренности – цензура.
Цензуру, само собой, способно осуществлять только государство. Но это давно не
так! Цензура и, соответственно, угроза свободе слова давно исходит от
медиакорпораций.

Не знаю, что там с цензурой для господ журналистов, а вот я уже давно живу в
ситуации институциализированной цензуры. Постепенно я ощущаю, что цензура
становится моим бытом. Только цензура эта осуществляется не государством, а как
раз оттуда, откуда не ждали. Я уже регулярно сталкиваюсь с цензурой Facebook и
«Твиттера», Youtube и «Википедии». 

Но эта цензура не номинируется, не называется таковой. Она не обсуждается
широко. Жрецы «свободы» не призывают массы яйцеголовых осмартфоненных воспылать
праведным гневом.

Я учусь быть партизаном. Я уже знаю, что бессилен перед лицом институтов, а
потому буду партизанить, изо всех сил стараясь отстаивать свою самость,
сопротивляться превращению в безмозглое, осмартфоненное существо. 

\begin{itemize}
\iusr{Петр Волков} Москва 23 дня назад  

Если в прошлом ответ на вопрос велико ли различие между вежливостью и
политкорректностью мог бы быть неоднозначным, то сегодня ответ на него
достаточно однозначен, поскольку сегодняшнюю политкорректность вполне можно
назвать псевдополиткорректным лицемерием, которая выродилась в настоящую
цензуру. И в сегодняшних СМИ цензуры не меньше, чем было в СССР, а маскировка
цензуры красивым словом политкорректность никого не обманывает. Конечно, случаи
грамотного и разумного применения цензуры есть, но, когда во главу угла
поставлена идеология (неолиберализма или коммунизма не имеет значения) цензура
превращается в заведомый абсурд.

\iusr{Андрей Васильев} 23 дня назад

Цензура тогда становится эффективным инструментом, когда нет альтернативных
источников информации. А чтобы их не было, надо содержать НКВД.  Вот тогда и
наступит светлое будущее.

\iusr{Дэвид Нездешний} 23 дня назад

Андрей Васильев, Союзу НКВД/КГБ не помогло ) Всё равно энтузиасты ловили вражьи
голоса. И благодаря цензури верили на слово всему, что трепали оные голоса.

\iusr{Андрей Васильев} 23 дня назад

Дэвид Нездешний, при НКВД голосов не ловили. А кто за бугром громко кукарекал,
к тому докторов засылали. Умели работать!  Только изначально вся идея тухлой
была - и с цензурой, и с НКВД. Из-под палки любви к Родине не может получиться.
Когда каждый знает, что за ним завтра могут прийти, ему не до любви.

\iusr{Дэвид Нездешний} 23 дня назад  

Думаю, властям не нужно вводить цензуру и бороться с враньём в соцсетях.

Нужно дискредитировать сами соцсети, как явление. Понятно, шо в современном
мире полностью избавиться от них уже не получится, люди всё равно жаждут
общения и всегда есть ленивые или просто замкнутые неудачники, которым сложно
нормально общаться с живыми людьми и они будут удовлетворять жажду общения
через интернетики.

Однако, надо широко просвещать массы о том, что из себя представляют соц-сети.
Что сказанное или показанное там просто чьё-то личное мнение, истиной по
умолчанию не является, а зачастую оказывается нарочитой ложью. Если массы людей
утратят слепую веру в непогрешимость интернетовского вранья, то подобная
пропаганда потеряет всякую силу.

\iusr{Тигровая Пчела} Санкт Петербург 23 дня назад  

Второй раз пытаюсь написать тут коментарий. Они исчезают вникуда. Это
оскорбительно. И мне больно. Потому что ну за что?

\iusr{Александр Канис} 23 дня назад  

Я чего-то не понимаю может быть, но по моему так было всегда. Вы приходите в
редакцию и говорите я написал 10 статей, разместите, а редактор в ответ: Нам
они не подходят!

\end{itemize}


