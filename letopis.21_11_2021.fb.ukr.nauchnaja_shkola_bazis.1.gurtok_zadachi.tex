% vim: keymap=russian-jcukenwin
%%beginhead 
 
%%file 21_11_2021.fb.ukr.nauchnaja_shkola_bazis.1.gurtok_zadachi
%%parent 21_11_2021
 
%%url https://www.facebook.com/sciencebasis/posts/1220047728490328
 
%%author_id ukr.nauchnaja_shkola_bazis
%%date 
 
%%tags deti,fizika,issledovanie,issledovatel,nauka,obrazovanie,shkola,ukraina,zadacha
%%title Гурток «Дослідницькі задачі»!
 
%%endhead 
 
\subsection{Гурток «Дослідницькі задачі»!}
\label{sec:21_11_2021.fb.ukr.nauchnaja_shkola_bazis.1.gurtok_zadachi}
 
\Purl{https://www.facebook.com/sciencebasis/posts/1220047728490328}
\ifcmt
 author_begin
   author_id ukr.nauchnaja_shkola_bazis
 author_end
\fi

Ваша дитина захоплюється фізикою? Тоді їй неодмінно потрібно на гурток
«Дослідницькі задачі»!

Відкрита реєстрація на новий цикл занять, який розпочнеться 6 грудня:

\url{https://forms.gle/A4gXKQZoixqXwQCF9}

Практичні заняття розраховані на учнів 7, 8, 9, 10, 11-х класів. Проходять вони
щотижня в фізичній лабораторії науково-дослідницької школи «Базис». 

\ii{21_11_2021.fb.ukr.nauchnaja_shkola_bazis.1.gurtok_zadachi.pic}

«Дослідницькі задачі» – це як лабораторні роботи в улюбленого вчителя фізики,
але в 10 разів крутіше!

Уявіть, що вам потрібно визначити скільки зернин рису знаходиться в
кілограмовому пакеті. Можна звичайно й перерахувати, але фізика має кращі
варіанти, і якщо ви знаєте фізику, то можете значно зекономити час. Або ж
спробуйте дізнатися довжину нитки в клубку, не розмотуючи його повністю, чи
вирахувати масу води в одній клітинці мокрого паперу...

Подібні завдання охоплюють весь шкільний курс фізики. Вони поєднують олімпіадну
задачу з фізичного експерименту, шкільну лабораторну роботу та наукове
дослідження. А ще такі завдання обожнюють діти, адже під час їх розв’язання
можна проявити усю свою кмітливість, спостережливість та винахідливість. 

Умови задач другого циклу:

\url{https://www.facebook.com/permalink.php?story_fbid=127496433030048&id=100073090756647}

За детальнішою інформацією про курс звертайтеся в Telegram: 050 714 50 02 або
\url{5nauk.kiev@gmail.com}

Творча лабораторія "5 НАУК"
