% vim: keymap=russian-jcukenwin
%%beginhead 
 
%%file 16_01_2021.fb.gladyshev_andrei.3.zsu_memory_zahysnyky
%%parent 16_01_2021
 
%%url https://www.facebook.com/groups/2405359826402240/permalink/2837051843233034/
 
%%author 
%%author_id 
%%author_url 
 
%%tags 
%%title 
 
%%endhead 
\subsection{Дзвін Пам’яті пролунав дев’ять разів... В Міноборони вшанували загиблих захисників}
\label{sec:16_01_2021.fb.gladyshev_andrei.3.zsu_memory_zahysnyky}
\Purl{https://www.facebook.com/groups/2405359826402240/permalink/2837051843233034/}
\ifcmt
  author_begin
   author_id gladyshev_andrei
  author_end
\fi

\textbf{Сьогодні, 16 січня}, на території Міністерства оборони України відбувся ранковий
церемоніал вшанування військовослужбовців Збройних Сил України, на знак подяки
та данини шани воїнам, які загинули цього дня у різні роки та внаслідок
російської збройної агресії на Сході України.

На щоденному ранковому церемоніалі були присутні родичі загиблих Українських
Героїв, керівництво Міноборони та ЗС України, офіцери та працівники структурних
підрозділів Міністерства оборони та Генерального штабу ЗС України,
військовослужбовці Київського гарнізону.

Присутні вшанували пам’ять захисників України, які ціною власного життя
виборювали волю і незалежність нашої держави від підступного російського
окупанта.

Сьогодні, 16 січня, Дзвін Пам’яті пролунав дев’ять разів...

В цей день 2015 року загинули:

\begin{itemize}
  \item капітан 1 рангу ОЛЕФІРЕНКО Юрій Борисович,
  \item старший сержант ШТАНСЬКИЙ Микола Миколайович,
  \item старший солдат ТИГНЯН Володимир Олександрович,
  \item солдат БОСИЙ Олександр Любомирович,
  \item солдат КАРАКУЛА Віталій Олександрович,
  \item солдат СЛОБОЖЕНКО Сергій Володимирович,
  \item солдат СУСЛОПАРОВ Олександр Олександрович.
\end{itemize}

В 2018 році загинув доброволець ВИТВИЦЬКИЙ Андрій Ярославович.

В 2020 році віддав своє життя захищаючи Україну старший солдат ХІМІЧУК Ігор Ігорович.

Присутні вшанували пам’ять захисників України хвилиною мовчання та поклали квіти до Стели Пам’яті.

Вічна пам’ять Героям, які полягли в боях за нашу вільну, незалежну Україну!

Довідка. Зала пам’яті захисників України створена на вшанування пам’яті
військовослужбовців, які загинули за свободу, незалежність і територіальну
цілісність України, світовий мир та порядок. Її спеціально розташовано таким
чином, щоби й військовослужбовці, працівники Міноборони та Генерального штабу,
й пересічні громадяни могли зайти до Зали та віддати шану своїм захисникам.


\ifcmt
tab_begin cols=3
  pic https://scontent-mad1-1.xx.fbcdn.net/v/t1.0-9/139508010_2882170485440467_6766578097521666190_o.jpg?_nc_cat=102&ccb=2&_nc_sid=b9115d&_nc_ohc=kHFfokZro1UAX8Z6SIy&_nc_ht=scontent-mad1-1.xx&oh=fcace7bf91bbd52d981bd94da8d866a2&oe=60275BC7

	pic https://scontent-mad1-1.xx.fbcdn.net/v/t1.0-9/139569425_2882170622107120_6335871367352707820_o.jpg?_nc_cat=100&ccb=2&_nc_sid=b9115d&_nc_ohc=vvAWulxrUz4AX_wN9Xr&_nc_ht=scontent-mad1-1.xx&oh=a2a35813fde23fa1546a6e8f402fab73&oe=602A38F6

	pic https://scontent-mad1-1.xx.fbcdn.net/v/t1.0-9/139786474_2882170728773776_2872251714626016466_o.jpg?_nc_cat=100&ccb=2&_nc_sid=b9115d&_nc_ohc=yDPChxolj8sAX95pvia&_nc_ht=scontent-mad1-1.xx&oh=8799b60fdf1bffba2ff856a5b9bbe660&oe=6028A7E1

	pic https://scontent-mad1-1.xx.fbcdn.net/v/t1.0-9/139659512_2882170858773763_1885680453882843547_o.jpg?_nc_cat=108&ccb=2&_nc_sid=b9115d&_nc_ohc=UHcZvuTJkwwAX_-VMVO&_nc_ht=scontent-mad1-1.xx&oh=1afcfbf994231235695f4105f566ffe8&oe=6026FFFA
\fi


Біля споруди Зали пам’яті облаштовано церемоніальний майданчик із Дзвоном
Пам’яті та Стелою, у якій вмонтовано осколки снарядів. Щоденний поминальний
ритуал здійснюється ударами дзвону, залежно від кількості загиблих у цей день.

В середині Зали пам’яті, на спеціальних подіумах, розміщено Книги пам’яті з
іменами загиблих. Структура Книги побудована за принципом календаря: один
розворот — один день, де згадуються всі загиблі у цей день за всі роки.
