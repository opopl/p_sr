% vim: keymap=russian-jcukenwin
%%beginhead 
 
%%file 30_12_2021.stz.news.ua.mrpl_city.1.pidsumky_2021_rubrika_mrpl_teatralnyj
%%parent 30_12_2021
 
%%url https://mrpl.city/blogs/view/pidsumki-2021-roku-dlya-rubriki-mariupol-teatralnij
 
%%author_id demidko_olga.mariupol,news.ua.mrpl_city
%%date 
 
%%tags 
%%title Підсумки 2021 року для рубрики "Маріуполь театральний"
 
%%endhead 
 
\subsection{Підсумки 2021 року для рубрики \enquote{Маріуполь театральний}}
\label{sec:30_12_2021.stz.news.ua.mrpl_city.1.pidsumky_2021_rubrika_mrpl_teatralnyj}
 
\Purl{https://mrpl.city/blogs/view/pidsumki-2021-roku-dlya-rubriki-mariupol-teatralnij}
\ifcmt
 author_begin
   author_id demidko_olga.mariupol,news.ua.mrpl_city
 author_end
\fi

\ii{30_12_2021.stz.news.ua.mrpl_city.1.pidsumky_2021_rubrika_mrpl_teatralnyj.pic.1}

Незабаром в рубриці \emph{\textbf{\enquote{Маріуполь театральний}}}, що виходить в рамках програми
\enquote{Ранок Маріуполя} на Маріупольському телебаченні, вийде підсумковий річний
сюжет. Про головні здобутки, перші рекорди і подальші плани рубрики я вирішила
розповісти і в своєму блозі.

Цього року рубрика розпочала нову традицію – ходити в гості до акторів додому.
До речі, започаткувати цю чудову традицію запропонував президент
Маріупольського телебачення \textbf{Микола Осиченко}, за що йому дуже вдячні всі
театрали нашого міста. Водночас ми ходили до акторів  і в гримерки. У звичній
для них обстановці знайомилися з ними ближче. Насправді саме ці сюжети набирали
найбільше переглядів, тому цю традицію обов'язко\hyp{}во продовжимо і у 2022 році.

\ii{30_12_2021.stz.news.ua.mrpl_city.1.pidsumky_2021_rubrika_mrpl_teatralnyj.pic.2}
\ii{30_12_2021.stz.news.ua.mrpl_city.1.pidsumky_2021_rubrika_mrpl_teatralnyj.pic.3}

У рубрики \enquote{Маріуполь театральний} за 2021 рік з'явилися власні  рекорди.
Зокрема, найкоротшим стало інтерв'ю із заслуженим артистом України \textbf{Анатолієм
Шевченком}. Він досить коротко і чітко відповідав на в питання, уклався у 10
хвилин. Найдовшим було інтерв'ю з художнім керівником і режисером Народного
театру \enquote{Театроманія} \textbf{Антоном Тельбізовим}. Назбиралося матеріалу на 2 години.
Насправді, інтерв'ю з Антоном було настільки захоплюючим, що у нас мало не
згорів чайник. Але, на щастя, все обійшлося.

\ii{30_12_2021.stz.news.ua.mrpl_city.1.pidsumky_2021_rubrika_mrpl_teatralnyj.pic.4}

Під час створення сюжетів вдома та гримерках було дуже багато незабутніх
моментів. Наші яскраві театральні діячі багато сміялися, ділилися спогадами. Ми
знайомилися з домашніми улюбленцями акторів  і дізнавалися  про їхні хобі. А
мені та нашим операторам ще й пощастило посмакувати смачними десертами, що
приготували наші актриси та  побачити неймовірну кількість чудових фото з
особистих архівів  театральних діячів. Інколи траплялися дуже унікальні
світлини, на яких актори були в досить незвичних образах. Зокрема, ніколи не
забуду фото Євгена Сосновського. Далеко не всі глядачі його бачили в образі
Баби-Яги.

\ii{30_12_2021.stz.news.ua.mrpl_city.1.pidsumky_2021_rubrika_mrpl_teatralnyj.pic.5}

До речі, під час зйомок часу розслаблятися не було. Інколи і мити посуд
доводилося. Насправді, кожен сюжет, присвячений окремому театральному діячу
міста, є унікальним. Дуже вдячна \textbf{Наталі Квятковській, Антону Тельбізову, Вірі
Шевцовій, Ользі Самойловій, Наталі Гончаровій, Валерію Капінусу, Анні Німайєр,
Ігорю Курашку, Дмитру Нестеренку, Надії Лаврененко, Ларисі Колесник, Євгену
Сосновському,  Михайлу Загребі, Анатолію Шевченку, Дар'ї Івановій, Сергію
Мусієнку, Сергію Забогонському та Артуру Войцеховському,} що погодилися бути
героями сюжетів, ділилися своїми спогадами, були справжніми та щирими.

\ii{30_12_2021.stz.news.ua.mrpl_city.1.pidsumky_2021_rubrika_mrpl_teatralnyj.pic.6}

За моїми підрахунками найбільше коментарів у різних сюжетах, присвячених іншим
акторам та актрисам, прем'єрам театру давав талановитий \textbf{Артур
Войцеховський}.  Він ніколи не відмоляв і завжди знаходив потрібні і дуже
влучні слова. Є в рубриці ще один рекорд. Це кількість згадок про одну й ту
саму людину від різних акторів. Найбільше цього року, хоча я впевнена, цей
рекорд повториться і наступного року, згадували справжнього майстра,
талановитого режисера і педагога \textbf{Костянтина Добрунова}. Окремою
важливою  подією стало створення телевізійно-театрального проєкту, міні-фільму
\enquote{Ваша Леся} створеного спільними зусиллями Донецького академічного
обласного драматичного театру (м. Маріуполь) та Маріупольського телебачення.
Завдяки цьому фільму глядач здійснив екскурс головними етапами життя Лесі
Українки.

\ii{30_12_2021.stz.news.ua.mrpl_city.1.pidsumky_2021_rubrika_mrpl_teatralnyj.pic.7}

Ще однією невеликою традицією стали окремі сюжети з корисними порадами і
вправами від наших акторів. До речі, ці сюжети також збирали багато переглядів.
Разом з тим ми дізналися більше про роботу важливих цехів театру –
Декораційного, Перукарського та Костюмерного.

\ii{30_12_2021.stz.news.ua.mrpl_city.1.pidsumky_2021_rubrika_mrpl_teatralnyj.pic.8}

Цьогоріч  в рамках рубрики \enquote{Маріуполь театральний} ми висвітлювали головні
театральні події міста. 2021 рік довів, що Маріуполь має дійсно свою унікальну
і виняткову театральну культуру. Ми розповідали про iStage 2021 –  п'ятиденний
фестиваль-лабора\hyp{}торія ідей актуального театру, Марафон міжнародних резиденцій
2021, Другий відкритий обласний фестиваль театрального мистец\hyp{}тва \enquote{Театральна
брама}-2021, міський фестиваль  аматорських театрів \enquote{Ма\hyp{}ріуполь театральний}.
Також завдяки рубриці глядачі могли дізнатися більше про всі прем'ри Донецького
академічного обласного драматичного театру (м. Маріуполь), Народного театру
\enquote{Театроманія}, Театральної артілі \enquote{Драмком}, театру \enquote{Фенікс}, Театру юного
актора, Театру ляльок, Першого недержавного театру Донеччини Terra Incognita
cвій театр для своїх, Театру авторської п'єси \enquote{Концепція}.

\ii{30_12_2021.stz.news.ua.mrpl_city.1.pidsumky_2021_rubrika_mrpl_teatralnyj.pic.9}

Ми розповідали і про ті театральні простори, які мають свою багату історію і
користуються неабиякою популярністю серед глядачів, проте не є досить відомими
в Маріуполі. Зокрема, це Грецька театральна студія, яка працює в Сартані. Також
висвітлювали діяльність театрів танцю Це Зразковий театр естрадного танцю
\enquote{Релеве} та студія танців \enquote{Фламенко}.

\ii{30_12_2021.stz.news.ua.mrpl_city.1.pidsumky_2021_rubrika_mrpl_teatralnyj.pic.10}

Хочу відзначити роботу всіх операторів, які працювали протягом року  над
рубрикою. Це \textbf{Микола Козін, Олександр Ратушний,  Роман Бабіч, Олексадр Шишман,
Микола Лупаненко, Лія Полухіна, Олексій Павлюк, Анастасія Решетілова, Сергій
Макартичан.} Всі вони справжні майстри своєї справи, адже саме завдяки їхній
роботі завжди створювалася чітка та якісна картинка.

Окремо хочу відзначити роботу незмінного і талановитого монтажера рубрики
\enquote{Маріуполь театральний} \textbf{Олексія Бахеркіна}. Саме мистецьке бачення Олексія, його
художні та професійні навички сприяли створенню цілісних і унікальних сюжетів.
Мабуть, Олексій вже  сам став трошки театралом. Також хочу подякувати всім
театральним діячам міста, які сприяли створенню нових сюжетів рубрики, ділилися
ідеями та своїми матеріалами. Дуже вдячна начальниці бюро реклами та оператору
Донецького академічного обласного драматичного театру (м. Маріуполь) Ірині
Сухаревській та Льву Сандалову за всі надані фото та відеоматеріали.

\ii{30_12_2021.stz.news.ua.mrpl_city.1.pidsumky_2021_rubrika_mrpl_teatralnyj.pic.11}

Розкрию деякі плани на 2022 рік. На початку року ми дізнаємося більше про ролі
в кіно заслуженого артиста України \textbf{Андрія Луценка}, познайомимося ближче з новим
актором Донецького академічного обласного драматичного театру (м. Маріуполь)
\textbf{Даміром Суховим.}

Розкажемо про нову прем'єру Театральної артілі Драмком. Спо\hyp{}діваюся, підемо в
гості до \emph{Володимира Кожевнікова, Анжеліки  Добрунової, Ірини Руденко,
Олександра Арутюняна та Ганни Архипової}. Також започаткуємо нову традицію – про
яку дізнаємося трохи згодом. Ну що ж всіх вітаю з новорічними святами і бажаю
здійснення всього найзаповітнішого. До нових зустрічей!
