% vim: keymap=russian-jcukenwin
%%beginhead 
 
%%file poetry.rus.dnr.vladislav_rusanov.pogrebenie_jarla
%%parent poetry.rus.dnr.vladislav_rusanov
 
%%url https://stihi.ru/2018/05/20/4874
%%author 
%%tags 
%%title 
 
%%endhead 

\subsubsection{Погребение ярла}
\label{sec:poetry.rus.dnr.vladislav_rusanov.pogrebenie_jarla}
\Purl{https://stihi.ru/2018/05/20/4874}

Конец --- это только начало.
На досках смола-живица.
А воин уйдёт в Вальхаллу,
неважно, когда он родился.

В каком бы не вырос веке —
в шестом или двадцать первом,
какие бы только реки
не стали его купелью.

Эйнхериев круг не тесен,
всегда в нём найдётся место
для тех, кто отважен и честен,
из нужного слеплен теста.

Ударит ли в горло срезень,
вопьётся в сердце осколок,
достанет у скальдов песен,
в них будет мало мирского.

В них будут белые башни,
в них будет разбитая «промка».
Кому-то там будет страшно,
кому-то --- излишне громко.

Там волны в погоне шалой
уносят горящий дреки,
То воин плывёт в Вальхаллу
тропою далёких предков.

2018
