% vim: keymap=russian-jcukenwin
%%beginhead 
 
%%file 07_08_2020.news.strana_ua.beirut_video_slow
%%parent 07_08_2020
 
%%endhead 
\subsection{Появилось замедленное видео взрыва в порту Бейрута высокого качества}
\label{sec:07_08_2020.news.strana_ua.beirut_video_slow}
\url{https://strana.ua/news/283061-video-vzryva-v-portu-bejrute-vysokoho-kachestva-snjatoe-mestnym-zhitelem.html}

\index{Катастрофы!Взрыв в Бейруте}
  
\vspace{0.5cm}
 {\ifDEBUG\small\LaTeX~section: \verb|07_08_2020.news.strana_ua.beirut_video_slow| project: \verb|letopis| rootid: \verb|p_saintrussia| \fi}
\vspace{0.5cm}

21:24, сегодня Апокалипсис начался с пожара. Скриншот из видео 

Житель Бейрута Немет Агостон, чей балкон выходит на порт Бейрута, снял видео
пожара 4 августа и последующего взрыва в высоком качестве. Оно позволило
замедлить прокрутку ролика и увидеть страшную силу взрывной волны.

Ролик Агостон опубликовал в YouTube.

Добиться хорошего качества съемки автору, видимо, удалось с помощью хорошей
техники и штатива. На кадрах виден сначала дым от пожара, затем краткие
вспышки, ужасающей силы взрыв и взрывная волна. Последняя буквально смела
автора видео. К счастью, ни он, ни его родные не пострадали.

Замедленный фрагмент

Как писала "Страна", число жертв взрыва выросло до 149 человек.

Мы также сообщали, что власти Ливана заговорили о других причинах катастрофы,
кроме основной - детонации запасов аммиачной селитры
  
