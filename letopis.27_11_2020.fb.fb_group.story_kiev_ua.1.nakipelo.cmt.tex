% vim: keymap=russian-jcukenwin
%%beginhead 
 
%%file 27_11_2020.fb.fb_group.story_kiev_ua.1.nakipelo.cmt
%%parent 27_11_2020.fb.fb_group.story_kiev_ua.1.nakipelo
 
%%url 
 
%%author_id 
%%date 
 
%%tags 
%%title 
 
%%endhead 
\zzSecCmt

\begin{itemize} % {
\iusr{Тома Храповицкая}
Браво Виктория!!!!!!!!!

\begin{itemize} % {
\iusr{Виктория Угрюмова}
\textbf{Тома Захарова} спасибо)))))

\iusr{Тома Храповицкая}
\textbf{Виктория Угрюмова} реально, читая Вас, не могу ничего добавить, только подтвердить, всё читаю!!!!!!!!

\iusr{Александр Бергельсон}
Чёртовы глисты...

\iusr{Тома Храповицкая}
\textbf{Александр Бергельсон} согласна с Вами!!!!!!

\iusr{Галина Гурьева}
\textbf{Александр Бергельсон},Тома Захарова, поддеживаю 100\%!!! Как же они надоели ( мягко сказано (((

\iusr{Тома Храповицкая}
\textbf{Галина Гурьева} да, надоели это ещё мягко сказано! Они мутируют, плодятся как кубло, ненавидя всё и всех, это страшные творения живущие и паразитирующие на всём чем можно поживиться, оставляя после себя только разруху, ненависть и заражая слабых духом!!!

\iusr{Олексій Мачинський}
\textbf{Тома Захарова} все буде добре, гарантую!

\iusr{Тома Храповицкая}
\textbf{Олексій Мачинський} дякую Вам за гарантію, а як же інакше? Головне залишатися людиною з палким серцем і ладом в голові!!!

\iusr{Галина Гурьева}

\ifcmt
  ig https://scontent-frx5-2.xx.fbcdn.net/v/t39.1997-6/p480x480/105941685_953860581742966_1572841152382279834_n.png?_nc_cat=1&ccb=1-5&_nc_sid=0572db&_nc_ohc=EsITRl0vkWIAX_ZscYV&_nc_ht=scontent-frx5-2.xx&oh=00_AT-XMGTI0xGKW37JdMZNxAs23FYgWlcLjHaw72XcKPYT1w&oe=61C1E08B
  @width 0.2
\fi

\end{itemize} % }

\iusr{Леся Фандралюк}
Я с Вами, дорогая Виктория!))

\iusr{Ирина Гукасян}
Метко, «не в бровь, а в глаз».... поддерживаю!!!

\iusr{Тетяна Кутишенко}
Благодарю Вас

\iusr{Svetlana Naulko}
Невероятная Виктория! С благодарностью  @igg{fbicon.hands.raising}  @igg{fbicon.thumb.up.yellow} @igg{fbicon.heart.red}{repeat=3}

\iusr{Марина Прокопенко}
Виктория, полностью Вас поддерживаю!

\iusr{Ольга Сімонова}
Спасибо! Дякую! @igg{fbicon.heart.red} @igg{fbicon.rose}  @igg{fbicon.heart.eyes} 

\iusr{Andriy Demydenko}

Моих предков расстреливали просто посмотрев на руки, но это не повод всех
ненавидеть и усугублять разделение на горожан и приехавших, крестьян и
ремесленников. Разделять надо только профессионалов и бездельников

\begin{itemize} % {
\iusr{Lola Madino}
\textbf{Andriy Demydenko} одного мого прадіда - навіть в цивільному одязі - за виправку...

\iusr{Олексій Мачинський}
\textbf{Lola Madino} а розстрілювали в обличчя, бо такі часи були

\iusr{Lola Madino}
\textbf{Олексій Мачинський} дівчата з гарних родин вчились носити червону косинку та сидіти на безкінечних зборах з відкритим ротом - щоб обличчя не було занадто розумним...
\end{itemize} % }

\iusr{Анна Сидоренко}
Лести не переношу, но с вами согласна.

\iusr{Валентина Луконина}
Я в восторге от Вашего комментария. Я с Вами, Виктория!

\iusr{Наталія Топоровська}
Дякую щиро!

\iusr{Iryna Busol}

Пані Вікторіє, не витрачайте свої сили та нерви на таких людей, прошу Вас! Вони
були, є і будуть. Хай собі проживають у своїй реальності.

Читати Вас - насолода. Для мене честь бути знайомою з Вашими батьками, родиною,
неймовірним дядьком). Про запахи Ви написали так, що я просто зупинилася в
сторонці на тротуарі і не зрушила з місця, доки не дочитала.

Щиро дякую Вам за спогади, у які ніби переносишся і переживаєш сам  @igg{fbicon.hands.applause.yellow}  ☀ ️ 

\begin{itemize} % {
\iusr{Виктория Угрюмова}
\textbf{Iryna Busol} 

я дякую всім моїм читачам, але написала як модератор - бо знали б ви, скільки
такі люди додають адміністративного клопоту, і скільки часу життя вони
намагаються відібрати, точачи скарги - ви пам'ятаєте - пан Олег нещодавно
вивішував комент однієї пані. Справа не в емоціях, я просто озвучую такий
момент - наша група засновувалась саме для спогадів, а з них нічого не
викинеш)) дґякую вам за теплі слова

\end{itemize} % }

\iusr{Valentyna Duhnovska}
Пані Вікторіє! Пишіть. Дуже цікаво. Дякую ВАМ

\iusr{Олег Коваль}

В любом случае, нужно следить за гигиеной общения и гнать от себя таких индивидуумов.

\begin{itemize} % {
\iusr{Ольга Кирьянцева}

Олег, каким образом мы можем их гнать от себя? Пока все эти \enquote{глисты} в группе,
нам остаётся только молча возмущаться, или поддержать, когда кто-то открыто не
начнет протестовать, как в данном случае Виктория или раньше Александр Б.


\iusr{Олег Коваль}
Ольга, в группе существует функция, \enquote{пожаловаться на комментарий}. Действуйте.

\iusr{Анна Солдатова}
\textbf{Олег Коваль} вы модератор. Баньте ботов. Их видно

\iusr{Анна Солдатова}
Я не знала о функции \enquote{пожаловаться на комментарий}. Группа загадка троллями

\iusr{Олег Коваль}
\textbf{Анна Солдатова}, повторяю, я не в силах читать комментарии, делайте, как я просил.
\end{itemize} % }

\iusr{Светлана Мед}
Поддерживаю !!!

\iusr{Мария Андрушко}
Підтримую, дякую, а спогади тільки на краще.

\iusr{Владимир Новицкий}

Дуже гарно сказано, скiльки ще людей якi залишилися там в старому i не
розумiють, що то був страшний час, який калiчив людей не тiльки фiзично, а як
видно i духовно, якщо навiть серед молодих людей ми помiчаемо ознаки пристрастi
до того часу.

\begin{itemize} % {
\iusr{Andriy Demydenko}
\textbf{Vladimir Novitsky} все они понимают, им просто там нравится, как 70 миллионам, проголосовавших за Трампа, как 73\% наших соотечественников  @igg{fbicon.frown} 

\iusr{Олег Коваль}
\textbf{Andriy}, вот, давайте не будем сейчас делить людей на проценты по политическим убеждениям!

\iusr{Andriy Demydenko}
\textbf{Hélg Smith} это не убеждения, у Трампа нет убеждений, да и у нас
\enquote{какая разница} вместо убеждений

\iusr{Олег Коваль}
\textbf{Andriy}, в нашей группе нет места политике.

\iusr{Andriy Demydenko}
\textbf{Hélg Smith} не я начал про политические убеждения

\iusr{Олег Коваль}
\textbf{Andriy}, я сейчас закончу.

\iusr{Анна Сидоренко}
Олег, я вам сочувствую, очень трудно сдержать людей, когда они входят в раж. Я
тоже была заблокирована, но не в Киевские истории.
\end{itemize} % }

\iusr{Натали Сайкевич}
 @igg{fbicon.hands.applause.yellow}{repeat=3}  Браво!
Неперевершено!
Влучно!

\iusr{Svetlana Batrak}

Правильные слова. Особенно сейчас. Вот странно.. в сложный период наоборот
нужно поддерживать друг друга и быть добрее. Но озлобленных становится все
больше((((

\iusr{Irena Visochan}
\textbf{Виктория Угрюмова} Браво-Брависсимо! Полностью поддерживаю, с
благодарностью! @igg{fbicon.heart.suit}

\iusr{Ольга Вакуленко}
100\% правда!

\iusr{Михайло Наместник}

Влучно сказано, пані Валерія. Совок багатьом роз'їв мозок та їхніми руками і
нас відпускати не хоче. Але будьмо доброї думки - на них чекає доля динозаврів.

\iusr{Ольга Юровских}

Благодарю! Меня тоже злит, что детские и юношеские воспоминания, превращают в
политические бессмысленные дебаты. Да мы все родом из СССР, но это не повод
пинать нас, за то что мы помним себя, своих друзей и мороженое по 19 копеек...

Моя бабушка тоже помнила цены до революции, по мы, внучки, относились к этому с
пониманием. Бабушка не любила советскую власть, а дети были коммунистами...

\begin{itemize} % {
\iusr{Irena Visochan}
\textbf{Ольга Юровских} 

Олечка! Браво! Точно так и моя бабушка рассказывала о жизни до революции, всегда
говорила, что при Царе жизнь была лучше, всякий раз просила об этом молчать. И мы
относились с пониманием и почтением! И проценты, и про Голодомор, и про многое
другое, но всегда предупреждала, что нужно молчать, Слава богу, что мы не забываем
наших предков и большинство в нашей группе КИ-единомышленники. @igg{fbicon.hands.pray} 

\iusr{Борис Лушельский}
\textbf{Ольга Юровских}

Оля, Вы абсолютно правы. Мы, наше поколение, родом из СССР. Как бы ни было хорошо
или плохо, но мы выучились, выросли и, теперь, можем критиковать. Нужно помнить и
уважать то, что было.

\iusr{Люда Бабич}
\textbf{Ольга Юровских} 

согласна с Вами на все 100! Мы не тоскуем за колбасой и мороженным, не нужно
упрекать нас \enquote{соком}. Мы выросли в Нашем Городе и просто очень любим тот
старый Киев котрого больше нет. С красавцами каштанами, огромными клубами с
красными каннами и розами, тенистыми дворами в центре и чистыми окраинами
.Очень многие наши оппоненты напоминают мне шариковых и швондеров - всё
разрушить до основания а затем. .....

\iusr{Виктория Угрюмова}
\textbf{Люда Бабич} и главное - мы не можем поменять время и место рождения)

\iusr{Наталя Кудря}

тим більше що вся наша влада корінням прямо в той совок і найгірше з нього на
жаль все залишилося, а от те, що створене працею наших батьків і дідів-медицина
.освіта. підприємства- все знищено- тому й провокують-щоб не сильно питали де
поділося!! словеса вигадають - не буду цитувати - вивертає ...

\end{itemize} % }

\iusr{Наталия Марченкова}
Спасибо, Виктория. Не обращайте внимания на дураков

\iusr{Любов Білоножко}
 @igg{fbicon.hands.applause.yellow}{repeat=3} !

\iusr{Tamara Ivanova}
Написано на одному подиху

\iusr{Владимир Картавенко}

Батьки мого батька були вчителями. Їх знищили безслідно на Біломор каналі бо в
них була садиба в передмісті. Моя дружина писала текстові нормативно правові
документи про Пам"ятки історії Києво-Подолу. Її вбили в день народження нашого
сина, коли син, солдат УБД АТО був на військових зборах. Документи на квартиру
в будинку Пам"ятка історії де була явочна квартира групи Кушніров на Подолі
Юрківька-Фроліківська було підроблено. Я вже не пишу про інше, бо Кирілівка
навпроти мого дому. А питання полягає лише в тому щоби покласти на стіл ДВА
НАЯВНІ рішення Київради що до ДВОХ парків на Рибальському. 5.32 Гектари.


\iusr{Марина Прокопенко}
И ещё раз БРАВО!!!
Прочла с таким удовольствием...
 @igg{fbicon.face.happy.two.hands}{repeat=3} 

\iusr{Олег Коваль}
Сергей Скот заблокирован! Жду новых претендентов.

\iusr{Татьяна Милана}

Это специфика социальных сетей - поиск врага и загон инакомыслящих. Тупости
человеческой предела нет. Либо тут же банить, либо использовать обсценную
лексику, по выбору.

\begin{itemize} % {
\iusr{Andriy Demydenko}
\textbf{Тетяна Свербілова} divide and rule было задолго до соцсетей

\iusr{Олексій Мачинський}
\textbf{Тетяна Свербілова} добривечір, сусідко!
Як на мене, ліпше брати на глузи!

\iusr{Татьяна Милана}
\textbf{Олексій Мачинський} 

Доброго ранку, сусіде  @igg{fbicon.smile}  Кому що більше подобається
@igg{fbicon.smile}  В мене вже немає сил читати все те лайно, що вважають за
необхідне публікувати тут користувачі Фейсбук. Перетворили укрнет на суцільну
війну з уявним ворогом.

\end{itemize} % }

\iusr{Светлана Семенец}
Виктория, спасибо, полностью поддерживаем.

\iusr{Таня Васильева}

\ifcmt
  ig https://scontent-frx5-2.xx.fbcdn.net/v/t39.1997-6/s168x128/16781161_1341101952618574_7704631035023065088_n.png?_nc_cat=1&ccb=1-5&_nc_sid=ac3552&_nc_ohc=tv2YBZ97BCkAX9owrAM&_nc_ht=scontent-frx5-2.xx&oh=00_AT9dSuEQ7aqL9DCWzcotuclbkJx5-6WL59oeeXuu1UIdbQ&oe=61C1E82B
  @width 0.1
\fi

\iusr{Евгения Бочковская}
КРИК ДУШИ СТРАНЫ... голосом и душой Виктории !
Полностью согласна...
Самое легкое, ...облив грязью, уничтожить...
Просто удивительно..., как же не любят
умных людей !
А ОНИ, к сожалению, предпочитают не пачкаться...

\iusr{Tatyana Shpak}
Згодна на всі  @igg{fbicon.100.percent} 

\iusr{Людмила Каштан}

Я согласна. Практически все комментарии в конце концов сводятся к политике.
Забывают, что мы все киевляне и просто любим город, в котором родились и где
прошла жизнь.


\iusr{Микола Амфіногенов}
Мой любимый фонтан в Мариинском парке. Его всегда чистили осенью.

\iusr{Борис Лушельский}

Это часть нашей с вами истории и забывать ее нельзя. К тому же ,наша группа не
политизирована и давайте, дамы и господа не забывать этого.

У каждого из нас и наших предков было своё отношение к тому или другому
событию. Если начать винить, то можно винить всех без исключения: и
коммунистов, и националистов и колобарационистов. Все они погрязли в крови.

Не надо высказывать всё, что думается. Нужно уважать и других людей.

Кто не согласен, может меня забанить.

\begin{itemize} % {
\iusr{Олег Аптекарь}
\textbf{Борис Лушельский} 

это значит, борюсик, что общество больно ненавистью в очень тяжелой форме.. и
врядли при нашей жизни что-то изменится к лучшему... почему немцы не бросаются
друг на друга при виде старых фотографий?.. или французы не обвиняют друг друга
при упоминании маршала петена?... а некоторая часть украинцев, безусловно
меньшая, но жутко агрессивная до умопомешательства, считает себя точкой
отсчета правды, судьей и палачем в одном лице.. это хвороба, которая не
лечится..

\end{itemize} % }


\end{itemize} % }
