% vim: keymap=russian-jcukenwin
%%beginhead 
 
%%file 29_11_2020.fb.smolij_andrij.1.davnja_istorija_nkvd
%%parent 29_11_2020
 
%%url https://www.facebook.com/permalink.php?story_fbid=3556730097698100&id=100000834337338
 
%%author Смолій, Андрій
%%author_id smolij_andrij
%%author_url 
 
%%tags 
%%title Це давня історія, проте, яка актуальна навіть сьогодні.
 
%%endhead 
 
\subsection{Це давня історія, проте, яка актуальна навіть сьогодні}
\label{sec:29_11_2020.fb.smolij_andrij.1.davnja_istorija_nkvd}
\Purl{https://www.facebook.com/permalink.php?story_fbid=3556730097698100&id=100000834337338}
\ifcmt
	author_begin
   author_id smolij_andrij
	author_end
\fi

\index[rus]{Украина!НКВД, бандеровцы}

Це давня історія, проте, яка актуальна навіть сьогодні. 

Це історія про страшну пропаганду і брехню, в яку на жаль, досі вірить частина населення.

***

Колишній енкаведист, який у формі бандерівця вбивав мирних людей в Галичині в
40-50-х роках перед смертю прийшов на сповідь до священника. Що сказав
священник? Читайте і поширюйте! 

\ifcmt
pic https://scontent.fiev6-1.fna.fbcdn.net/v/t1.0-9/128525686_3556729971031446_8506692187304956441_n.jpg?_nc_cat=1&ccb=2&_nc_sid=8bfeb9&_nc_ohc=u3PG65W5megAX-lxg4H&_nc_ht=scontent.fiev6-1.fna&oh=d14df0cc424fb74e0eae0ea6196c5f45&oe=5FEB1D37
\fi

СПОВІДЬ КОЛИШНЬОГО ПСЕВДОБАНДЕРІВЦЯ На ім’я голови сільської ради села Н.
Заліщицького району зі Східної України надійшов лист такого змісту: «Я -
колишній солдат спецгрупи НКВД, яка під маркою бандерівців у 1944-1945 рр.
проводила масові вбивства невинних людей на Тернопільщині. Наша група позбавила
життя не один десяток людей, яких ми вбивали ніби за симпатії до совітської
влади. І ось нині, на схилі літ, стоячи над могилою, хочу висповідатись,
розповісти правду, хто насправді проводив оті криваві акції. Мені привиджуються
ті нещасні, які просили в нас пощади, але пощади їм не було... Часто у снах чую
крик чотирирічного хлопчика: «Не бий мого татка!!!». Він вирвався із рук нашого
старшини, при тім укусив його за палець. Тоді старшина вхопив хлопчика за ніжки
і з усього маху вдарив об стінку головою… Мозок із розбитої голови дитини
бризнув по хаті - на нас, на наші руки, на обличчя. Я за той час розстріляв
двох сестричок 10-ти і 12-ти років. Вони заціпеніли від жаху і навіть не
розуміли, що відбувається. Батько і мати на колінах благали нас, щоб ми
пощадили хоча б дітей і теж не розуміли, за що їм така кара - просили: «Хлопці,
схаменіться, ми ж ні в чому не винні...» (вони думали, що ми - бандерівці). А
ми звинувачували їх у тому, що їхні старші два сини пішли на фронт. Хоч ми
добре знали, що людей призовного віку забирали силою, не питаючи, хоче хтось
іти на війну чи ні. Таких випадків було багато, але мені запам’ятався особливо
оцей. Голос того хлопчика: «Не бий мого татка!». Зойки нещасних сестричок та
їхніх батьків останнім часом переслідують мене вдень і вночі. Пішов я у церкву
і висповідався. Старенький священик зблідлими вустами прошепотів: «Сину, великі
твої гріхи, але оскільки ти каєшся і тебе змушували скоїти такий гріх твої
командири, я, як священик, можу дати тобі розгрішення. І дам, але тільки тоді,
коли ти напишеш у ті села, де ви проливали невинну кров, та ще й під чужу
марку». Я знаю, пане голово, що цього люди не простять мені ніколи - надія
тільки на Бога. Адже ішов я убивати людей не сам від себе. Це вони, наші
командири-чекісти, змушували нас. Завжди, коли ми, солдати, відмовлялися когось
убивати, погрожували: «Хотітє бить чістєнькімі?!! Расстреляєм саміх как
собак!!!». Ми боялися один одного і ніколи поміж собою не обговорювали свої
вчинки, навіть під час пиятики. Але коли через якийсь час наш старшина по
необережності сам у себе вистрелив з автомата і в муках помер, ми мовчки
переглянулись, і я подумав: «Оце тобі, гаде, за тих невинних людей, за того
хлопчика...». Хоч у самих теж руки були в крові. Розкажіть, пане голово, усім у
селі про мого листа. Хай знають люди, хто насправді знищив їхніх сусідів. Не
бандерівці, ні!!! Це ми, чекісти, так криваво провокували наших людей. А винна
в тому сатанинська, кривава, совітська влада...

Р. S. І треба ж, головою сільської ради був якраз син одного із братів, що
пішли 1944 року на фронт, батьків яких, сестричок і братика було знищено
нелюдами-чекістами у березні 1945 року. Ридав, читаючи листа, голова, і в
розпуці бив кулаками об стіл. У неділю пішов до церкви і на сповіді покаявся за
свій гріх, що протягом багатьох років він і його родина носили в серці образу
та гнів, що проклинали невинних героїв-повстанців, які полягли у боротьбі за
волю України і до смерті їхньої родини, як нарешті з’ясувалося, не мали ніякого
відношення. На сповіді ридали обидва - голова і священик. Розповідь
тернопільчанина опрацював Левко ПАРАЩАК. Часопис «Прикарпаття», 30 квітня 2005
р.
