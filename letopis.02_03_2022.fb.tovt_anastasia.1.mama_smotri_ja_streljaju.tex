% vim: keymap=russian-jcukenwin
%%beginhead 
 
%%file 02_03_2022.fb.tovt_anastasia.1.mama_smotri_ja_streljaju
%%parent 02_03_2022
 
%%url https://www.facebook.com/anastasija.tovt/posts/4933782776706753
 
%%author_id tovt_anastasia
%%date 
 
%%tags 
%%title Мама, смотри, я стреляю!
 
%%endhead 
 
\subsection{Мама, смотри, я стреляю!}
\label{sec:02_03_2022.fb.tovt_anastasia.1.mama_smotri_ja_streljaju}
 
\Purl{https://www.facebook.com/anastasija.tovt/posts/4933782776706753}
\ifcmt
 author_begin
   author_id tovt_anastasia
 author_end
\fi

- Мама, смотри, я стреляю! 

Маленький мальчик в очереди к продуктовому магазину держит в руках игрушечный
красный автобус. Направляет его на маму и прохожих, и громко изображает звуки
выстрелов: пшш-пшш!

- Зайка, тише, - испуганно оглядывается мама по сторонам, - не надо так громко.
Да и автобусы не умеют стрелять, - аккуратно возражает она.

- Мой - умеет, - мальчик обиженно складывает губы в трубочку и высовывает язык.
Это - его протест против неуверенной попытки мамы разрушить его игру. 

- Я просил купить мне пистолет. Ты не купила. Стреляю из того, что есть.
Пшш-пшш! Сдавайся! Я победил!

И не поспоришь. Он уже победил. Страх - так точно.

Киев. Седьмой день войны. 

Дети все понимают. 

Дети играют в войну. 

Для взрослых война - не игрушка. Это новая реальность. К которой мы начинаем
привыкать. И от этого осознания - одновременно и легче, и ужаснее. 

Столица понемногу оживает. Адаптируется к жизни в войну. Чувство страха будто
атрофировалось вместе с инстинктом самосохранения. Им на смену приходит упрямое
желание выжить. И при этом максимально сохранить привычный уклад своей жизни,
свои бытовые привычки. Уж это мы путину не отдадим.

В перерывах между воздушными тревогами, и даже во время них киевляне спокойно
выгуливают собак. Они уже привыкли к сиренам. Появилось какое-то ленивое
желание на них не реагировать. 

Научились не вздрагивать от залпов и не приседать каждый раз, когда где-то
вдалеке слышится взрыв. Эти звуки стали обычным фоном, каким раньше была попса
из мелких кафе на районе. Сейчас - пугающая тишина.

И все же, как приятно снова видеть на улицах столицы людей!  Даже если все они
к обеду столпились в одном магазине - единственном работающем на весь район. 

Многие будто на зло игнорируют комендантский час. И затариваются в
супермаркетах - тех, что работают, - неспешно и рассудительно. Без паники и не
как «в последний раз». Как в былые времена. 

Вот на телефон в пятый раз за день приходит уведомление «внимание, в Киеве
воздушная тревога! Все в укрытие!» 

- Мне пох. Стоим. Я без сигарет не уйду, - комментирует вслух женщина с полной
тележкой продуктов в очереди на кассу. Такое же уведомление с разницей в
секунды получили в магазине все. Никто не вышел. 

В супермаркете очередь. Хоть уже меньше, чем вчера и позавчера. Хлеба
по-прежнему нет, но в местной пекарне приготовили роллини - с вишней, курицей,
грибами. Горячие - только из печи. Запах свежей выпечки - пожалуй,
единственное, что туманит голодный разум. Через 10 минут полки уже пусты. 

По оставшимся продуктам уже видно, как сменились потребительские приоритеты. В
отделе фруктов разобрали все, кроме манго по 200 грн. А кому эта экзотика
сейчас нужна? Разве что бросаться манго в оккупантов из окна. И добивать
кокосами. 

Поймала себя на мысли, что скучаю по традиционным атрибутам столичной жизни,
которые раньше страшно бесили. По киевским пробкам, например. По нервному звуку
клаксона, когда кто-то не включил поворотник или просто тупит. Оказывается,
этот звук не так уж плох по сравнению с похоронным воем сирен. 

Но что сейчас вдохновляет - это люди. 

Война - время героев. 

Герой сегодня каждый, кто несмотря ни на что продолжает делать своё дело. 

Армянин из соседнего подъезда, который сегодня под вой сирен разложил на улице
лотки с картошкой и соленьями. 

Кассир в супермаркете. 

Охранники стоянки. «Начальство ещё в первый день сказало: всем стоять на посту!
Чтобы никто не вздумал сбежать! Охраняем режимный объект». И они охраняют. 

Водитель автобуса. 

Врач, проводящий обходы в метро, превратившемся в бомбоубежище. 

Журналист, освещающий войну. 

Простые мужчины и женщины, которые собирают еду и лекарства для тех, кому
сейчас это нужно.

Смельчаки, которые с первых дней пошли в тероборону.

Их жены, бадяжащие коктейли Молотова и плетущие сетки для врагов.

Солдаты, защищающие страну и всех нас в горячих точках по всей Украине.

И этот мальчик с игрушечным автобусом. Тоже герой. Потому что не позволяет
никому - ни испуганной маме, ни тем более путину, - отобрать у него воображение
и детство.

Киев. 02.03.2022.
