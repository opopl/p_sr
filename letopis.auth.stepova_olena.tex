% vim: keymap=russian-jcukenwin
%%beginhead 
 
%%file auth.stepova_olena
%%parent 12_04_2021.fb.groznyj_valerij.1.k_jebenjam_den_kosmonavtiki
 
%%url 
 
%%author 
%%author_id 
%%author_url 
 
%%tags 
%%title 
 
%%endhead 

\subsection{Степова Олена}
\label{sec:auth.stepova_olena}

\url{https://www.obozrevatel.com/person/olena-stepova.htm}

Олена Степова – украинская писательница, блогер
Олена Степова – кто это

Олена Степова – творческий псевдоним популярной украинской писательницы и блогера. Своего настоящего имени Степова не раскрывает, опасаясь преследования со стороны террористов “Л/ДНР” и российской власти.

Известно, что она родилась в Должанске, Луганской области, в апреле 1971 года. Была членом политической партии Народный Рух Украины. Долгое время работала на Донбассе предпринимателем, а в 2000-м году занялась правозащитной деятельностью. Вскоре после начала российско-украинской войны на Донбассе переехала в Киевскую область.

Замужем, воспитывает двух дочерей. С родителями прекратила общение из-за их пророссийских взглядов.
Правда о Донбассе

Олена Степова стала чрезвычайно популярной благодаря своему блогу, в котором она пишет о текущей ситуации на Донбассе. Она подробно рассказывает о проблемах, которые регулярно возникают у террористов в ОРДЛО – от загнивания медицинской отрасли до уничтожения заводов и предприятий; не стесняясь в выражениях публикует доказательства присутствия российских и осетинских “ихтамнетов” и раскрывает военные и общественные преступления террористов “ДНР” и “ЛНР”. Ее рассказы о жизни на оккупированной территории приобрели огромную известность в сети. Книга “Все будет Украина! Или истории из зоны АТО”, которая является сборником сетевых историй Олены Степовой, разошлась крупными тиражами.
Блог, творчество и активность в сети

Кроме официального сайта и блога на OBOZREVATEL, Олена Степова ведет свою страницу в Facebook, на которую уже подписано более 17 тысяч человек. За ее вторым аккаунтом в Фейсбуке, на котором публикуются стихи Степовой, следит свыше 36 тысяч человек. Кроме того, у писательницы есть свой блог и на популярном ресурсе Blogspot.

Кроме “Все будет Украина! Или истории из зоны АТО”, Олена Степова написала ряд не менее популярных книг. Среди них:

    “Мышкины сказки” – сборник сказок, рассказанных детям в зоне АТО в период активных боевых действий.

    “Плач Марсельезы” – повесть о страшном и тяжелом быте шахтерского поселка.

    “Автобиография бунта” – подробный литературный анализ истоков сепаратизма на Донбассе.

    “Время В” – сборник стихов и сказок.

    “Свет родного дома” – две детские сказки, опубликованные в формате раскрасок.

Также она давала многочисленные интервью таким медиаресурсам, как “UA:Перший”, журнал “Фокус”, “ZIK” и “Ukrlife.tv”.
