% vim: keymap=russian-jcukenwin
%%beginhead 
 
%%file 10_03_2018.stz.news.ua.mrpl_city.1.doktor_shirjaeva
%%parent 10_03_2018
 
%%url https://mrpl.city/blogs/view/doktor-shiryaeva
 
%%author_id burov_sergij.mariupol,news.ua.mrpl_city
%%date 
 
%%tags 
%%title Доктор Ширяева
 
%%endhead 
 
\subsection{Доктор Ширяева}
\label{sec:10_03_2018.stz.news.ua.mrpl_city.1.doktor_shirjaeva}
 
\Purl{https://mrpl.city/blogs/view/doktor-shiryaeva}
\ifcmt
 author_begin
   author_id burov_sergij.mariupol,news.ua.mrpl_city
 author_end
\fi

\ii{10_03_2018.stz.news.ua.mrpl_city.1.doktor_shirjaeva.pic.1}

Травматология и ортопедия – тяжелая врачебная специальность. Кроме знаний,
выдержки при виде растерзанного человеческого тела, она требует еще и
недюжинной физической силы. Поэтому, наверное, в этой области медицины
специализируются, как правило, мужчины. Но, правду говорят, что в каждом
правиле есть исключения. И таким счастливым исключением была
ортопед-травматолог \textbf{Валерия Николаевна Ширяева}. У Валерии Николаевны было много
друзей и среди исцеленных ею людей, и среди коллег, и среди сослуживцев военной
поры. Но были у нее еще друзья особенно близкие. Такие, как Ирина Михайловна
Броновицкая, которая встретилась Валерией Николаевной девочкой-подростком,
затем молодым врачом училась у нее, позже, уже заведуя гематологическим
отделением медсанчасти завода имени Ильича, сотрудничала с нею как коллега. Она
всегда с теплотой и любовью говорит об этой необыкновенной женщине и враче. Вот
воспоминания Ирины Михайловны о докторе Ширяевой:

\begin{quote}
\em
	
\enquote{Родилась Валерия Николаевна в Литве. Там провела детские годы. Ее отец был
врачом. После Октябрьской революции семья переехала в Крым. Там Николай Ширяев
занимался не только врачебной деятельностью, но и организацией медицины. И
нужно сказать, очень плодотворно. Работая в Крыму, он помогал становлению
Советской власти, это было 17-18 годах. Как-то доктор Ширяев с несколькими
спутниками отправился в командировку. На них напали бандиты, связали, а потом
один из нападавших присмотрелся и воскликнул: \enquote{Так это ж наш доктор Ширай!}
Плененного врача тотчас развязали, к костру пригласили. А он говорит: \enquote{Как же
так? Меня развязали, а товарищи мои связанными лежат}. Ну, товарищей
освободили.

Именем Николая Ширяева названа улица в Ялте, недалеко от этой улицы стоит
домик, в котором проживала с родителями Валерия Николаевна, ее брат и сестра.
Семья дружила с местными жителями, крымскими татарами. Татары с большим
уважением относились к семье русского врача. У Ширяевых даже во дворе многое от
татарского осталось, деревьев и цветов много. Татары – большие знатоки крымской
земли, они как-то умели брать влагу от этой земли. У них были прекрасные сады и
виноградники.

Валерия Николаевна окончила Харьковский медицинский институт. Потом приехала по
назначению в Мариуполь. Работала она в травматологическом отделении, очень
добросовестно и с чуткостью относилась к больным. А больные отвечали доверием и
любовью. В 1941 году Валерию Николаевну призвали в армию. Работала вначале в
нашем городе в эвакогоспитале. За день до занятия Мариуполя гитлеровцами она
вместе с госпиталем эвакуировалась на Кавказ, в Азербайджан. Это были очень
трудные годы. Приходилось работать очень много, спасая жизнь раненых бойцов.
Именно в Азербайджане Валерия Николаевна стала большим профессионалом.

После окончания войны, в 1945 году, Валерия Николаевна вернулась в Мариуполь,
стала работать в травматологическом отделении врачом-ординатором. Была очень
спокойной, выдержанной, я никогда не слышала, чтобы она повысила голос. За этим
спокойствием и выдержкой стояли ее знания и профессионализм. Опекала молодых
врачей, старалась им помочь. Уже в очень преклонном возрасте – ей уже было
восемьдесят и более – она продолжала работать, не теряя своего мастерства,
искусства врачевать больных. Я помню, как-то я дежурила в больнице, еще будучи
молодым врачом. Привезли тяжелейшего больного, травмированного тракториста.
Какая-то растерянность была у меня – дежурного хирурга. Что делать, с чего
начинать оказывать помощь? Решили пригласить Валерию Николаевну. Несмотря на
поздний час, она пришла, посмотрела, и все стало на место. Сказала, что и как
нужно делать. И все совершенно спокойно. Без всякого назидания, а тем более
упреков в неумении. Как-то с ее появлением все почувствовали себя спокойными и
уверенными. Операция прошла успешно.

Валерия Николаевна была необыкновенно обаятельной женщиной. Она всегда на
равных держалась с молодежью, не было в ней налета высокомерия, присущего
иногда людям пожилым. Она прекрасно играла на фортепиано. Иногда мы, молодежь,
под ее аккомпанемент пели и танцевали. Она была награждена орденом \enquote{Знак
Почета}, несколькими медалями, в том числе и медалью \enquote{За оборону Северного
Кавказа}. Но надевала эти заслуженные ею награды, когда отправлялась на встречи
со своими однополчанами. Такие встречи она старалась не пропускать, и ехала с
таким приподнятым настроением.

Одевалась очень просто, особенно в преклонные годы. Очень аккуратная была, во
всем. Если куда-то собирается уезжать, и остается какая-то бумажка, соринка,
обязательно все нужно убрать, все расставить по местам, чтобы было чисто. В
доме у нее всегда так было: не единой лишней вещи. Что не нужно было, все
немедленно выбрасывалось. Очень скромно жила, но во всем был идеальный порядок
и чистота. Гимнастикой занималась до самых последних дней, каждое утро,
независимо от настроения и самочувствия, начинала с гимнастики.}

\end{quote}

Однако время брало свое. С годами силы постепенно покидали Валерию Николаевну.
Несмотря на то что друзья и родственники оказывали ей внимание и поддержку,
оставаться ей одной становилось все сложнее. И она решила покинуть Мариуполь, в
котором прошли, быть может, самые ее лучшие годы. Она уехала в город своего
детства и юности, в крымскую Ялту, к дочери. В Ялте она тихо ушла из жизни.

У друзей доктора Ширяевой сохранились ее фотографии разных лет. Что обращает на
себя внимание, когда рассматриваешь эти тронутые временем снимки? Ни на одном
из них Валерия Николаевна не позирует фотографу, стараясь показаться красивее,
чем она есть на самом деле. На общих фотографиях, сделанных на праздничных
встречах медперсонала военных госпиталей, она где-то  в сторонке, затерявшаяся
среди других. На некоторых снимках можно рассмотреть  ее руки. Руки
необыкновенно правильной формы, не потерявшие изящества и красоты и в
преклонные годы. Наверное, именно они давали возможность Валерии Николаевне
Ширяевой делать тончайшие операции даже крохотным пациентам.

Такой была доктор Ширяева.

\ii{10_03_2018.stz.news.ua.mrpl_city.1.doktor_shirjaeva.pic.2}
