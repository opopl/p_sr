% vim: keymap=russian-jcukenwin
%%beginhead 
 
%%file 08_12_2020.news.dnr.dan.dnr_dan_news.1.tvorcheskaja_molodezh_dnr
%%parent 08_12_2020
 
%%url https://dan-news.info/world/tvorcheskaya-molodezh-dnr-zavoevala-priznanie-na-bolee-sotni-onlajn-ploshhadkax-rf-chexii-i-bolgarii.html
 
%%author Донецкое Агентво Новостей (ДАН)
%%author_id dnr_dan_news
%%author_url 
 
%%tags lnr,youth
%%title Творческая молодежь ДНР завоевала признание на более сотни онлайн-площадках РФ, Чехии и Болгарии
 
%%endhead 
 
\subsection{Творческая молодежь ДНР завоевала признание на более сотни онлайн-площадках РФ, Чехии и Болгарии}
\label{sec:08_12_2020.news.dnr.dan.dnr_dan_news.1.tvorcheskaja_molodezh_dnr}
\Purl{https://dan-news.info/world/tvorcheskaya-molodezh-dnr-zavoevala-priznanie-na-bolee-sotni-onlajn-ploshhadkax-rf-chexii-i-bolgarii.html}
\ifcmt
	author_begin
   author_id dnr_dan_news
	author_end
\fi

\ifcmt
pic https://dan-news.info/wp-content/uploads/2020/12/dan-news.info-2020-12-08_12-23-52_429742-6661.jpg
\fi

Донецк, 8 дек – ДАН. Творческая молодежь ДНР в период пандемии COVID-19 приняла
участие в более чем 100 творческих конкурсах, в том числе международных,
которые проходили в режиме онлайн. Об этом сегодня сообщили в министерстве
культуры Республики.

«В режиме повышенной готовности широкое распространение в
культурно-просветительской деятельности образовательных учреждений сферы
культуры получила практика использования информационно-коммуникативных
технологий. Это позволило представлять ДНР более чем на 100 творческих
исполнительских площадках. Молодые дарования и преподаватели стали лауреатами
мероприятий в Болгарии, Чехии, ЛНР и различных регионах РФ», – сказано в
сообщении.

В частности, концертмейстер Донецкой государственной музыкальной академии имени
Прокофьева Лада Курилова стала лауреатом I cтепени, а студентка музакадемии
Александра Арих – II степени, на международном конкурсе Czech Ballads,
организованном Чешской Республикой; студенческий экспериментальный театр
«Отражения» музакадемии стал лауреатом I степени ряда международных
конкурсов-фестивалей в России; фортепианные дуэты Донецкого музыкального
колледжа имени Прокофьева признаны лауреатами I степени фестиваля «Вместе в XXI
веке» в Болгарии, конкурса Shine в Санкт-Петербурге, проходившем под эгидой
Европейской ассоциации культуры, а также конкурса-фестиваля «Artнаследие» в
Москве.

Стоит выделить также участие ученика Донецкой республиканской
специализированной музыкальной школы-интерната для одаренных детей Кирилла
Рогового в Международном конкурсе имени Рубинштейна «Миниатюра в русской
музыке», проходившем в Санкт-Петербурге. Он удостоен гран-при в номинации
«Фортепиано».

Коронавирусная инфекция распространяется в мире с декабря 2019 года. Так
называемый нулевой пациент был зафиксирован в Китае. По последним данным, на
планете вирусом заразились более 68 млн человек. Сотни тысяч новых случаев
инфицирования регистрируются ежедневно. Большинство стран в связи с эпидемией
живет в изоляции: приостановлено ж/д и авиасообщение, отменены гастроли и
другие мероприятия. В ДНР для недопущения распространения заболевания также
действует ряд ограничений. *зх*мк
