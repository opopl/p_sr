% vim: keymap=russian-jcukenwin
%%beginhead 
 
%%file 21_09_2021.fb.nikonov_sergej.10.bilchenko_orlenok_antifashist_kino
%%parent 21_09_2021
 
%%url https://www.facebook.com/alexelsevier/posts/1586923911652918
 
%%author_id nikonov_sergej,bilchenko_evgenia
%%date 
 
%%tags antifashist,bilchenko_evgenia,fashizm,kino,kinoteatr,kultura,orlenok,rusmir,ukraina
%%title БЖ. Русский Орленок versus итальянская ласточка: антифашистское кино, жизнь в фашизме
 
%%endhead 
 
\subsection{БЖ. Русский Орленок versus итальянская ласточка: антифашистское кино, жизнь в фашизме}
\label{sec:21_09_2021.fb.nikonov_sergej.10.bilchenko_orlenok_antifashist_kino}
 
\Purl{https://www.facebook.com/alexelsevier/posts/1586923911652918}
\ifcmt
 author_begin
   author_id nikonov_sergej,bilchenko_evgenia
 author_end
\fi

Пост Евгении Витальевны про антифашистское кино в украинском кинотеатре.

БЖ. Русский Орленок versus итальянская ласточка: антифашистское кино, жизнь в
фашизме

Сегодня моя бывшая коллега по одному из университетов, которая, естественно,
рядом со мной не фотографируется, купила нам, спасибо ей, кино на европейские
короткометражки. Они называются странным словом "мокьюментари", но в целом
ничего необычного для экранной культуры в этом слове нет. В обществе
перформанса, где реальную трагедию от 11 сентября и Беслана до Курска и Перми
превращают на либеральных СМИ в шоу (за это запретили мою школу), создаются
фильмы, где, наоборот, вымыслу придается псевдодокументальный вид.

Моя коллега любит самое беспомощное из слов - "постпостмодерн". Это, когда
никто не знает, что такое "постмодерн", но тупой, ещё тупее, дискурс, добавляет
к ещё более неизвестному свое блабла/постпост. Дискурс людей без корней.
Постпостмодерн мокьюментари показало ряд слабоватых фильмов в стиле цветного
тренда BLM/new lefts, но одна короткометражка ("Кожа") о белых фа и черных
антифа во взаимном экстремизме потрясла: маленький мальчик, воспитанный
отцом-расистом, превосходит отца в расизме, и застреливает его. Вот таких детей
готовят украинские школы Стерненко и компании, ждите.

Но мне больше всего понравился фильм об итальянском сопротивлении фашизму
"Ласточка". Там нет увязшего в зубах культа "невинной жертвы". Люди идут на
смерть сознательно, их не гонят. И женщина в конце не плачет, а улыбается.
Победа. Триумф, а не травма. Фильму в зале захлопала только я. В ответ
украинские слушатели захихикали.

Моя коллега сказала, что не способна партизанить. Это ее проблемы. Стало всё
безразлично.

Ещё меня поразила, что вторая моя коллега, украинка, попадья, жена священника
УПЦ, который молится за наших атошников (я б молилась, чтобы они оттуда ушли и
за донецких детей), сказала, что не может воспользоваться моей статьей, потому
что в ней употребляется слово "ризома". Оно русскомирское типа. Очевидно, Путин
придумал ризому и написал "Капитализм и шизофрению" под псевдонимом Феликса
Гваттари.

"Ризома" - понятие Жиля Делеза. Коллега - культуролог. Это всё, что надо знать
об образовании в краине мрий.

Хороший фильм про итальянских партизан. Но я все время пересматриваю "Зою".
Потому что она вспоролась, сама (без пацанов) вернулась и пошла на задание, не
схитрила, не успела поджечь все хаты, не приспособилась к разведке. Был ее
подвиг абсурдом и фанатизмом? Было ли жалко местных?

ДА. НЕТ.

ЗОИН ОРЛЕНОК был Победой Духа. А мужики на это способны сейчас? Вот когда
будете способны, звоните, пишите, без длинных речей "о высших моральных
ценностях вне политики". Может, реабилитирую.

\#рецензииотБЖ
\#самойсебекупилацветы
