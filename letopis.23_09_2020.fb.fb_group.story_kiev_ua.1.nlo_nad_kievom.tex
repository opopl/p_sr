% vim: keymap=russian-jcukenwin
%%beginhead 
 
%%file 23_09_2020.fb.fb_group.story_kiev_ua.1.nlo_nad_kievom
%%parent 23_09_2020
 
%%url https://www.facebook.com/groups/story.kiev.ua/posts/1465959443600852
 
%%author_id fb_group.story_kiev_ua,ugrjumova_viktoria.kiev.pisatel
%%date 
 
%%tags kiev,nlo
%%title НЛО НАД КИЕВОМ
 
%%endhead 
 
\subsection{НЛО НАД КИЕВОМ}
\label{sec:23_09_2020.fb.fb_group.story_kiev_ua.1.nlo_nad_kievom}
 
\Purl{https://www.facebook.com/groups/story.kiev.ua/posts/1465959443600852}
\ifcmt
 author_begin
   author_id fb_group.story_kiev_ua,ugrjumova_viktoria.kiev.pisatel
 author_end
\fi

НЛО НАД КИЕВОМ

(отрывок из авторского эссе НЛО над Киевом, Буквография Киева том 2, ссылка
обязательна. Спасибо)

\begin{zznagolos}
Большая официальная заметка об НЛО над СССР (Ровно в 4.10) в советской прессе
была напечатана в газете «Труд» 30 января 1985 года, по–моему, в среду — но
тут я могу ошибаться. А до этого дня сообщения о появлении «летающих тарелок»
выискивали, где только могли и как только умели. НЛО в стране, которой нынче
нет на карте, были представителями не только и не столько внеземных
цивилизаций, сколько символом свободы мысли и принадлежности к другому
пространству идей.
\end{zznagolos}

Мы, поколение людей, родившихся в прошлом веке и тысячелетии в той самой
стране, не являлись собственными современниками. Об этом с блистательной тоскою
написал еще Самуил Лурье и доказал как дважды два на собственном примере:
«Великий диктатор» с Чарли Чаплином посмотрел на 60 лет позже остального
человечества; «Доктора Живаго» прочитал на тридцать два года позже, чем
Пастернаку присудили за него Нобелевскую премию; и как человек прошелся по
Луне, даже краешком глаза не увидел.

\ii{23_09_2020.fb.fb_group.story_kiev_ua.1.nlo_nad_kievom.pic.1}

Именно поэтому советским интеллектуалам гораздо важнее было, чем тем же
американцам, знать, что внеземные цивилизации есть. Пускай советские граждане
не участвовали в жизни остального мира, зато так же, как все, жили во
Вселенной, занимали такое же место в пространстве и вечности. Перед необъятным
космосом, как перед господом, все становились действительно равны, приобретали,
что ли, одинаковую соразмерность. Думаю, именно против этого ощущения, а не
против каких–то конкретных марсиан так боролись лекторы общества «Знание» и
издатели единственной переводной книги об НЛО — «Правда и ложь о летающих
тарелках» Дональда Мензела.

Противный американец приводил массу фактов, а после их, как Воланд на печально
известном сеансе в Варьете, развенчивал; однако же книгу Мензела в СССР
прочитали весьма своеобразно: скрупулезно изучили факты и оставили без внимания
нелепые объяснения, что это дескать оптическая иллюзия зашла на кухню к
домохозяйке или блик от оконного стекла в соседнем квартале полчаса жужжал над
крышей фермерской усадьбы.

На Мензеле книги об НЛО в отечестве закончились. Апологетам внеземных
цивилизаций оставался самиздат и выписки–заметки.

Ах, как собирали эти заметки из иностранных и — позже и крайне редко —
советских газет и журналов, с каким азартом перепечатывали по ночам, а порой
переписывали от руки самиздатовские переводы «Колесницы богов» Эрика фон
Деникена или «Тварей» Айвена Сандерсона; с каким усердием изучали хроники и
летописи в рукописных отделах библиотек и закрытых фондах; с каким восторгом
говорили потом о своих находках на крошечных кухнях, перебивая друг друга,
торопясь сообщить последние новости. И с каким почтением взирали на редких
счастливцев, воочию наблюдавших ту самую Баальбекскую веранду или железную
колонну в Дели.

Теперь, когда огромный мир целиком и полностью умещается в компактном планшете,
когда небо утыкано спутниками, а шотландские озера — камерами, реагирующими на
движение; когда ни инопланетянин, ни морское чудовище не могут избежать
удовольствия быть сфотканным на селфи какого–нибудь чудака, принимающего их как
очередную данность, объяснить ценность пожелтевшей газетной заметки можно
только такому же, как ты — выходцу из мезозоя, еще помнящему, зачем клеить
марку на письмо и крутить диск с прорезанными в нем круглыми дырочками, в
которых мелькают полустертые цифры.

Теперь сообщение о том, что очередная кавалькада летающих тарелок проследовала
со стороны Багам в сторону Амазонки, не станет событием, даже став фактом.

А ведь совсем недавно, с точки зрения Вселенной, так и вовсе сейчас, коллекция
газетных и журнальных вырезок на разных языках, со всякими завлекательными
цветными картинками, почему–то не производимыми нашей полиграфией, ценилась на
вес золота.

Теперь, разглядывая собственные полки с отборными книгами на эту тему, я не
могу отвязаться от мысли о том, какое несказанное удовольствие, какое даже
счастье доставил бы нам, скажем, томик Чарлза Форта, обошедшийся мне в
полдоллара на развале букинистического рынка на Петровке. Не в цене даже дело —
в полной бесхозности и невостребованности книги, которая лежала в картонном
ящике, сыром после недавнего дождя, и которая была бы бесценной пару десятков
лет назад. То есть, вообще вчера. А по меркам Вселенной, так и вовсе — сейчас.

Сразу вслед за этим вспоминаю с нежностью и о том, как в 1991 году в
Симферополе, на раскладке возле книжного магазина, ясным и солнечным
августовским днем купила книгу о снежном человеке — все четные страницы в ней
были вклеены вверх ногами, и уже через день добрая половина пляжа зачарованно
следила за загадочными манипуляциями, которые я с ней проделывала.

Думаю и о том, что Киев моего детства и юности был городом, еще умудрявшимся
сохранять ту сказочную ауру, то трогательное и невероятное ощущение волшебства,
чуда, которым и прославился — именно за этим ощущением ехали сюда
многочисленные иностранные туристы, а вовсе не поглядеть на архитектурные
шедевры и великие монументы или приобщиться к высокому в наших музеях.

\begin{zznagolos}
Кстати, именно увлечение НЛО подарило Киеву один из самых известных нынче
архитектурных объектов, построенных в ХХ веке, точнее в середине 1970-х годов.
Он так и называется — Летающая тарелка. Шестнадцатиэтажное здание
научно-технической библиотеки на улице Горького, 180, само по себе довольно
скучное и ничем не примечательное, но вот мультифункциональный концертный зал,
который задумывался как светомузыкальный театр, а использовался, в основном,
как кинолекторий, так и остался совершенно особенным сооружением среди прочих
киевских построек такого же назначения. Зал буквально парит в воздухе, как
зависшее над землёй или приземлившееся на крышу НЛО. Он отличается
замечательной акустикой (специально спроектированные поворотные устройства
панелей подвесного потолка позволяют настраивать её под конкретные требования),
и в нем выступающий может говорить без микрофона. За свой проект архитектор
Флориан Юрьев получил награду «За новаторство в архитектуре».
\end{zznagolos}

Но главным шедевром, которым не мог похвастаться, пожалуй, никакой другой
город, был незабываемый киевский пейзаж — огромные парки, поросшие древними
деревьями склоны, золотые купола и вечное небо, отражающееся в вечном Днепре.
Легко было поверить, что в том Киеве запросто можно было встретить и Солоху на
метле, и черта с украденным Месяцем, и летающий объект, этот Месяц изучающий.

И как тут не припомнить к месту еще одну байку: уже году в 1996 мы с мужем
рыбачили на Черторое. Как–то внезапно набежали тучи, мы не успели добраться до
дороги, когда хлынул проливной дождь и нам пришлось спрятаться под каким–то
фигурным мостиком, который ни через что не вел, а просто стоял красоты для
посреди лесопарковой зоны, надежно спрятанный от всех, кто мог бы им
полюбоваться. Через пару часов стало ясно, что это уже не дождь, а увертюра
всемирного потопа; сидим в глуши, под фиктивным мостом, беседуем. Никто нас
видеть не может. И вдруг — картина маслом: из леса выходит и неспешной походкой
бредет через луг, начисто игнорируя потоки воды, льющейся с небес, натуральный
зеленый человечек.

Человек.

Зеленый. Цвета еще такого симпатичного — мятного. С легким серебристым
оттенком, замечательно гармонирующим с серым небом, серебристыми тучами и
серебряными нитями вполне возможно, что и грибного дождя. Идет он от нас на
довольно большом расстоянии, да еще стена ливня — как я уже сто раз упомянула.
Но зрелище захватывающее. Поблескивает так слегка, лицо — даже издалека видно,
— другого оттенка, такое, знаете, слегка фиолетовое. Я же и говорю, конец
света.

В общем, это оказался аквалангист, застигнутый дождем, который разумно решил не
снимать свой костюм для подводного плавания, а спокойно и непромокаемо
прошествовать в нем до самой троллейбусной остановки. Просто в то время цветные
костюмы из неопрена были в наших пределах огромной редкостью, проще было
поверить в НЛОнавта. А, может, это был настоящий НЛОнавт, прикинувшийся
раритетным аквалангистом. 

И поскольку я склонна думать, что Вселенная нисколько не одномерна, и что любая
жизнь в общих чертах повторяется в ней миллионы раз, будто отражается в
миллионах зеркал, мне забавно и весело представлять, что где–то сейчас есть я,
которая пишет эти заметки; где–то – я, которая читает их с улыбкой, как привет
из далекого прошлого; а где–то – я, которая, вооружившись ножницами, вырезает
из газеты заметку об НЛО над Киевом.

\ii{23_09_2020.fb.fb_group.story_kiev_ua.1.nlo_nad_kievom.cmt}
