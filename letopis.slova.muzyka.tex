% vim: keymap=russian-jcukenwin
%%beginhead 
 
%%file slova.muzyka
%%parent slova
 
%%url 
 
%%author 
%%author_id 
%%author_url 
 
%%tags 
%%title 
 
%%endhead 
\chapter{Музыка}
\label{sec:slova.muzyka}

\emph{Музыка} и кинематограф всегда были тесно связаны между собой и представить
сейчас отличный фильм без качественного саундтрека довольно-таки сложно. Тем
более, что порой главная \emph{музыкальная} композиция картины становится настолько
популярной, что продолжает жить в сердцах слушателей уже отдельно от фильма.
Но проблема \emph{музыкальной} темы в кинематографе - это порой забвение исполнителя
шлягера. Другими словами - артисты, исполняющие песни за кадром, остаются
абсолютно неизвестными, что очень жаль. Вот именно это и произошло с героиней
нашего сегодняшнего рассказа, которую зовут Татьяна Дасковская,
\citTitle{Забвение, нищета и эмиграция в Германию. Как сложилась судьба певицы,
исполнившей знаменитую песню \enquote{Прекрасное далеко}}, Rock Story,
zen.yandex.ru, 11.05.2021

%%%cit
%%%cit_pic
\ifcmt
  pic https://img.strana.ua/img/article/3405/na-kontserte-basty-31_main.jpeg
  width 0.4
\fi
%%%cit_text
В свою очередь, националисты утверждают на том, что российский рэпер якобы
поддерживал аннексию Крыма Россией.  \enquote{Человек, который участвует в жюри,
которое выбирает гимн для Крымского моста, вы считаете это нормально? Вам не
стыдно?} - спросил националист у стоящей в очереди на концерт Басты женщины.
\enquote{Мы за \emph{музыку}, мы не за политику. Мы не до чего не договоримся. Я люблю
\emph{музыку}. Вы за свое стоите, я за свое}, - ответила девушка
%%%cit_comment
%%%cit_title
  \citTitle{На концерте Басты начались перепалки с националистами}, , kiev.strana.ua, 25.06.2021
%%%endcit
