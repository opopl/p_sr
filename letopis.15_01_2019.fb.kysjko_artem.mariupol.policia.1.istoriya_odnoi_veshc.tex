%%beginhead 
 
%%file 15_01_2019.fb.kysjko_artem.mariupol.policia.1.istoriya_odnoi_veshc
%%parent 15_01_2019
 
%%url https://www.facebook.com/permalink.php?story_fbid=1995520454075542&id=100008528200769
 
%%author_id kysjko_artem.mariupol.policia
%%date 15_01_2019
 
%%tags mariupol.istoria,istoria,mariupol.pre_war,policia,vysockij_vladimir.bard,muzej.mariupol.policii
%%title ИСТОРИЯ ОДНОЙ ВЕЩИ
 
%%endhead 

\subsection{ИСТОРИЯ ОДНОЙ ВЕЩИ}
\label{sec:15_01_2019.fb.kysjko_artem.mariupol.policia.1.istoriya_odnoi_veshc}

\Purl{https://www.facebook.com/permalink.php?story_fbid=1995520454075542&id=100008528200769}
\ifcmt
 author_begin
   author_id kysjko_artem.mariupol.policia
 author_end
\fi

ИСТОРИЯ ОДНОЙ ВЕЩИ. 

Сегодня был очень приятно удивлён !

\enquote{Будет сидеть!!! Я сказал}  - живой и настоящий, опер до мозга костей с
обостренным чувством справедливости Глеб Жеглов в исполнении Владимира
Высоцкого вдохновил не одно поколение молодых людей выбрать  профессию борца с
преступностью.

Но оказалось  история культового фильма \enquote{Место встречи изменить нельзя}
связана с современной полицией Донетчины. Знаете ли вы, что одна из
деталей образа легендарного оперативника  угро - хранится в Мариуполе?

Сегодня полиция возрождает музей органов внутренних дел Донецкой области, где
будут собраны раритетные экспонаты с начала 20 века. В ходе пополнения
коллекции мы наткнулись на уникальную вещь – плащ Жеглова. 

Раритет обнаружил помощник начальника полиции Донецкой области Николай
Побойный. Все эти годы плащ бережно хранился у нашего коллеги –
оперуполномоченного уголовного розыска Сергея Евгеньевича Кобылецкого, ныне
пенсионера МВД, сотрудника одного из мариупольских вузов. Сам он из Дружковки,
но давно живет в Мариуполе. 

Это далеко не простой предмет одежды - он имеет свою историю и является
семейной реликвией. В этом плаще  пришел с войны отец Сергея Кобылецкого,
который дошел до Германии.

Трофейное немецкое пальто Сергей дал своему другу, известному в СССР
мариупольскому актеру Евгению Стежко, который  как раз пробовался на роль
Жеглова. Кожаный плащ четко вписывался в образ сотрудника МУРа послевоенных
лет.  В результате на главную роль утвердили бессмертного Владимира Высоцкого,
который оценил эффектную деталь и одолжил ее на время съемок. Сам Евгений
Стежко сыграл в фильме роль Топоркова.

Сергей Кобылецкий передал ценный реквизит в дар музею полиции Донетчины. Теперь
он, как символ легендарного опера, будет хранить память о наших мужественных
ветеранах, которые сражались с преступностью исключительно при помощи
интеллекта, а также вдохновлять молодежь.

И напоследок, еще одна цитата от непревзойденного Жеглова: \enquote{Если есть на свете
дьявол, то он не козлоногий рогач, а он дракон о трёх головах, и головы эти —
хитрость, жадность, предательство. И если одна прикусит человека, то две другие
доедят его дотла}. – Советовал бы запомнить)

%\ii{15_01_2019.fb.kysjko_artem.mariupol.policia.1.istoriya_odnoi_veshc.cmt}
