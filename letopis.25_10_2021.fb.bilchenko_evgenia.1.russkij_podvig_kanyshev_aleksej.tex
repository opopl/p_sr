% vim: keymap=russian-jcukenwin
%%beginhead 
 
%%file 25_10_2021.fb.bilchenko_evgenia.1.russkij_podvig_kanyshev_aleksej
%%parent 25_10_2021
 
%%url https://www.facebook.com/yevzhik/posts/4376060082429068
 
%%author_id bilchenko_evgenia
%%date 
 
%%tags cerkov,chelovek,dusha,kanyshev_aleksej.rossia.strannik,puteshestvie,rossia,rusmir,russkie,russkij_chelovek,vera
%%title Светлана Пикта - Русский  подвиг - репост
 
%%endhead 
 
\subsection{Светлана Пикта - Русский  подвиг - репост}
\label{sec:25_10_2021.fb.bilchenko_evgenia.1.russkij_podvig_kanyshev_aleksej}
 
\Purl{https://www.facebook.com/yevzhik/posts/4376060082429068}
\ifcmt
 author_begin
   author_id bilchenko_evgenia
 author_end
\fi

Светлана Пикта Русский  подвиг \#репост

Александр Сергеевич Канышев, 51 год, российский странник, прошел на коленях
путь от Дивеевского монастыря до Троице-Сергиевой Лавры длиной в 400 км за 9,5
месяцев в 1995-1996 годах.

Последовал призыву Богородицы.

Заслуженный деятель искусств России, академик Российской Академии
кинематографических искусств «Ника», лауреат многих престижных кинофестивалей,
режиссёр документального кино Вячеслав Васильевич Орехов снял об этом событии
документальный фильм «Очарованный странник».

\ifcmt
  ig https://scontent-frt3-1.xx.fbcdn.net/v/t1.6435-9/248476288_4376060042429072_5397620590543115324_n.jpg?_nc_cat=104&ccb=1-5&_nc_sid=8bfeb9&_nc_ohc=Gf1a0LHDOCoAX-AvBtU&_nc_ht=scontent-frt3-1.xx&oh=9e4a6cf7afba44442938ca52f7312c76&oe=619E36F5
  @width 0.4
  %@wrap \parpic[r]
  @wrap \InsertBoxR{0}
\fi

Кадры его ассоциируются с отдельными эпизодами «Андрея Рублёва» Тарковского:
почерневшие, мокрые от сырого снега деревенские дома и заборы, заснеженная
дорога, по которой мчатся грузовые машины. А по краю дорожного полотна, толкая
перед собой тележку, доверху наполненную скарбом, медленно и упорно, опираясь
второй рукой о посох, передвигается на коленях пожилой человек. Впереди тележки
— Распятие и икона Господа Вседержителя.

Скрипят самодельные брезентовые наколенники на разбитых в кровь коленях этого
странного для конца безбожного двадцатого столетия русского человека. Звучит
величественная, торжественная музыка, напоминающая о Вечности. Впереди – храм.
Люди по сторонам дороги останавливаются и глядят на ползущего с молитвой
согбенного человека.

Кто-то подаёт ему еду, просит помолиться о близких. Кто-то смотрит на него с
любопытством, кто-то – с состраданием. Одна старушка пытается помочь
подтолкнуть медленно ползущую тележку. Некоторые откровенно подсмеиваются над
чудаком.

Женщины относятся к крестоходцу с большим, чем мужчины, сочувствием. Многим
непонятно: кто этот человек – сумасшедший, юродивый или святой? Он ползёт, а
впереди него бегут слухи: убил жену и детей, зарезал ножом, зарубил топором (на
что ещё способен обезумевший русский мужик?). Вот и замаливает теперь свой
смертный грех.

Собравшимся около него людям странник говорит: «Я из Красноярского края, из
Кежемского района. В тюрьме никогда не сидел, никого не убивал. Грехи у меня
такие же, как и у всех других людей. Иду по послушанию Пресвятой Богородице.

Я был неверующим, но увидел, как Богородица пришла к Богу и сказала, что в
такие тяжёлые времена, как сейчас, нужен человек, который бы пошёл с молитвой
на коленях за Россию, за весь мир.

Бог показал ей меня, моё лицо, и сказал, что я некрещёный, грешный,
богохульник, но больше идти некому. … Я покрестился в Дивеево, стал на колени и
иду. Если люди придут в церковь и покаются в своих грехах, значит, я прошёл не
зря. У меня любовь к людям».

Его спросили: «Не страшно вам погибнуть на дороге?» Он ответил: «Страшно, когда
ничего не сделал…»

Девять с половиной месяцев шёл он на коленях от женского монастыря в Дивеево до
Троице-Сергиевой Лавры, в Сергиев посад. С молитвой за Россию, за мир,
погрязший во грехах…

Душа болела за целый мир.

Когда режиссёру Орехову сказали, что есть такой человек, он поехал искать его
на трассу. Кино, как и вся наша страна, в середине 90-х было в глубочайшем
кризисе. Орехову и его оператору выдали на киностудии одну коробку плёнки на
десять минут съёмки. Знакомый дал видеокамеру и несколько кассет.

«Мороз за 25 градусов, а он ночует в лесу. Расчищает лопатой место, ставит
палатку, разводит костёр и кое-как, с муками и неудобствами, коротает ночь.
Утром — снова на колени и дальше, дальше, вперёд…Вячеслав Орехов назвал свой
фильм в честь героя повести Николая Лескова. Очарованный странник XIX века, по
словам Людмилы Белкиной, автора статьи о режиссёре Орехове на портале
православной газеты «Благовест» — это русский человек с изломами судьбы,
горячо, всей душой любивший Россию.

Очарованный странник конца ХХ века более трагичен и даже апокалиптичен. Мир
катится в пропасть, а он идёт, чтобы люди покаялись и остались живы, ибо Чаша
Господнего гнева полна до краёв. Он знает и видит, что это зависит от него. Он
говорит просто и устало: «За мной стоит 150 миллионов русских людей. Если я
брошу, они погибнут».

У него нет гордыни, он говорит о себе так: «Я маленькое существо, грешный
человек, богохульник. Но всем нам нужно думать о спасении мира, о спасении
наших душ». Один из зевак у дороги сказал ему: «Ни за какие миллионы долларов я
бы не смог пройти такой путь». Да, ради денег и славы такой крестный ход не
пройти…

Когда он пришёл в Сергиев Посад, поклонился мощам Преподобного Сергия
Радонежского и отправился в Гефсиманский скит, в палатке на развилке дорог
прожил ещё один месяц, проповедуя людям покаяние. Его часто видели потом в
Дивеево, в Москве, у Матронушки Московской. В течение нескольких лет ходил он в
крестный ход от Липецка до Задонского монастыря, к святителю Тихону Задонскому.

О прошлом Александра Канышева известно совсем мало: был одиноким человеком,
семьи не создал. Работал строителем, ездил по стране с шабашниками, строил
дома, школы, котельные. Когда его бригада возводила церковь в Ставропольском
крае, ему в первый раз явилась Пресвятая Богородица, повелела окреститься. Он
крестился, стал пономарём в церкви. В следующее своё явление Она отправила его
странствовать по России. И он последовал Её призыву.

Почему пошёл в крестный ход на коленях? По его словам, на идущего по обочине
дороги человека люди не обратят внимания. Мало ли кто так ходит? А вот на
ползущего с молитвой человека, с крестом и иконой Богоматери на груди –
обратят. Это заставит людей вспомнить о Боге и задуматься о своей жизни.

К стёртым в кровь коленям прикрепил наколенники из нескольких слоёв брезента, к
сапогам приделал крохотные колёсики, чтобы было полегче передвигать больные
ноги. Ночевал в лесопосадках около дороги в любую погоду. Физически страдал
невероятно, но как будто кто-то свыше не давал ему тёплый кров и горячую
домашнюю пищу.

«Если бы я зашёл в дом и лёг на мягкую тёплую постель, наверняка не смог бы
потом подняться и ползти дальше, — признавался он знакомым. – Немощен и
малодушен человек».

В августе 2012 года Александр Сергеевич Канышев прошёл до Задонской обители
крестным ходом в последний раз. Проникнувшись сочувствием и уважением к
молитвенному подвигу Александра, одна из паломниц омыла его больные ноги.
Смущаясь от непривычной заботы, он заснул со счастливой детской улыбкой на
устах. Через полгода он умер.

А. С. Канышев ночью со второго на третье января в 23-55 2012 года скончался -
Великий крестоходец сын Божий Александр Канышев. Царствие ему Небесное и
светлая память!

\ii{25_10_2021.fb.bilchenko_evgenia.1.russkij_podvig_kanyshev_aleksej.cmt}
