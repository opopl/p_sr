% vim: keymap=russian-jcukenwin
%%beginhead 
 
%%file 21_08_2021.fb.fb_group.story_kiev_ua.1.zamkovaja_gora.pic.20.cmt
%%parent 21_08_2021.fb.fb_group.story_kiev_ua.1.zamkovaja_gora
 
%%url 
 
%%author_id 
%%date 
 
%%tags 
%%title 
 
%%endhead 

\iusr{Татьяна Сирота}
\figCapA{На Фроловской}

\iusr{Юрий Черниенко}
На Флоровской

\iusr{Татьяна Сирота}
\textbf{Юрий Черниенко} Флоровский монастырь!
А улица называется Фроловская. ☺ ️ 

\iusr{Юрий Черниенко}
\textbf{Татьяна Сирота} 

Вулиця відома з XVI століття, разом із Фролівським провулком складала вулицю
Чорна Грязь (від очеретяного болота, що було поряд, на місці теперішнього
Фролівського монастиря). ... З 20-х років XX століття вживається тільки назва
«Фролівська вулиця», від Києво-Флорівського Вознесенського жіночого монастиря.

Архітектурні пам'ятки: Фролівська, 22, колишня прохідна фабрики «Юність»,
182...

Назва на честь: Фролівського монастиря

Місцевість: Поділ

\iusr{Татьяна Сирота}
\textbf{Юрий Черниенко} Всё верно Вы написали.
Улица Фроловская!
