% vim: keymap=russian-jcukenwin
%%beginhead 
 
%%file 28_12_2020.fb.bilchenko_evgenia.1.kulturologia_ukraina_tragedia
%%parent 28_12_2020
 
%%url https://www.facebook.com/yevzhik/posts/3511487778886307
 
%%author Бильченко, Евгения
%%author_id bilchenko_evgenia
%%author_url 
 
%%tags bilchenko_evgenia,kultura,kulturologia,tragedia,ukraina
%%title БЖ. "Индуизм как древнегреческая религия", или Трагедия культурологии в Украине
 
%%endhead 
 
\subsection{БЖ. \enquote{Индуизм как древнегреческая религия}, или Трагедия культурологии в Украине}
\label{sec:28_12_2020.fb.bilchenko_evgenia.1.kulturologia_ukraina_tragedia}
\Purl{https://www.facebook.com/yevzhik/posts/3511487778886307}
\ifcmt
 author_begin
   author_id bilchenko_evgenia
 author_end
\fi

БЖ. \enquote{Индуизм как древнегреческая религия}, или Трагедия культурологии в Украине.

Когда-то мы с преподавателями и студентами классической академической школы
религиоведения столкнулись с таким потешным рефератом учащегося и решили
основать нечто вроде журнала студенческих казусов под одноименным названием.
Тогда казусы были относительной редкостью. Это было стыдно и это добровольно и
взаимно высмеивалось в солидарном полемическом профессиональном коллективе.
Например, я до сих пор помню, как на государственном экзамене студентка, не
посещавшая семинары, в ответ на вопрос светлой памяти профессора Заковича:
\enquote{Кто руководит митрополитами в православной церкви?} - ответила
покрасневшему от смеха Мыколе Михайловичу: \enquote{Филареты} (так в ее
сознании звучало слово \enquote{патриархи}). 

Сейчас, на терминальной стадии колонизации Украины и тотальной идеологизации
всей системы образования и науки последняя превратилась в один сплошной казус.
Нынче Будда рождается в Черкассах, Геродот повествует о предках \enquote{moskovitov},
школа П. Толочко и классический марксизм находятся под административным
запретом или подвергаются гонениям на уровне структур и выслуживающихся перед
ними индивидов, критика фашизма изгоняется из дискурса как \enquote{нежелательная}. И
это вещают не бедные маленькие студенты, а взрослые и даже пожилые люди с
претензией на научные звания. Доносы и угрозы инакомыслящим коллегам стали
массовой практикой.

Не обошла беда и нашу культурологию. И не могла обойти. Наша наука и так
образовалась на месте вакуума, порожденного крахом догматического
квазимарксизма-ленинизма, что уже говорить о степени политических чисток в ней
сейчас? Запрещаются целые культурологические школы, занимающиеся критикой
идеологии, вопреки тенденциям развития культурологии - не то, что в России, а в
столь почитаемой в колонии Западной (!) Европе. Как вожделенная Европа
развивается, никто толком не знает: английским мало кто владеет. А если и
владеет, то читаeт методички по Фукуяме.

Ветто на критику идеологии превращает культурологию в собственно идеологию.
Культурология либо становится приложением к вульгарной националистической
пропаганде Института рекламной памяти, либо же приложением к хитроумным
либеральным арт-практикам многочисленных соросовских центров. Причем,
националистическая пропаганда внедряется в жизнь с командной топорностью
вульгарного советского стиля (не путать с цивилизацией СССР), ибо ее
сторонники, поменяв воображаемую идентичность и перекрасив флаги с алых на
сине-желтые, сохранили маразматическую риторику и структуру мышления комсоргов
в их самом нелицеприятном смысле слова. Такая диалектика: лучшее забываем,
худшее наследуем. 

Их неолиберальные кураторы - более изысканны: пропаганда проводится через
прокси-сервера \enquote{современного искусства} и имеет рассеянный, якобы \enquote{аполитичный}
вид, будучи при этом насквозь политической и агитационной и табуируя все иные
школы: от традиционалистов до Франкфурта и Люблян. В первом случае
культурология - гимн русофобии и Бандере, во втором - какой-то хилый
эклектичный рэп из кейсов для училища арт-кураторов. О русофобии я вообще
молчу: имена русских мыслителей отбираются с тщательной внимательностью
офицеров Вермахта.

Что в этих условиях творится с неокрепшими головами студентов первых курсов,
которые уже не помнят, кто такой Бахтин, можете себе представить. Негласный и
противоречащий Конституции запрет на русские научные тексты в творчестве
взрослых ученых приводит к тому, что ребята утрируют ситуацию в гротеске. Дети
ведь - наше кривое зеркало. И они уже вышли из дурдома нынешних отечественных
школ, где бывшие коммунистки в лице национал-патриоистерических училок ныне
вещают про ОУН-УПА. Так появляются уже не казусы, а казусища: если вы не
знаете, что такое \enquote{давньояпонський театр Але}, то знайте: это гугль-перевод
русской статьи о театре Но. Если вы не знаете, что такое \enquote{Ромео и Джульетта},
это - американский роман про субкультуру подростков, чей автор - неизвестен. 

В общем, мне безумно жаль мою страну. Не имеющий памяти и лишенный знания не обладает правом на будущее.

Фото моё.

\ifcmt
  pic https://scontent-lga3-2.xx.fbcdn.net/v/t1.6435-9/133568482_3511487718886313_5337704400586146850_n.jpg?_nc_cat=108&ccb=1-3&_nc_sid=8bfeb9&_nc_ohc=4krMvO78Pa4AX8fnEaS&_nc_ht=scontent-lga3-2.xx&oh=9efde6111a734e06a7be33af61a76a6c&oe=60CB3F12
\fi

\emph{Константин Клаус}

Просто гениальный текст.. Настоящий интеллектуальный самородок..

\emph{Yurri Sofin}
а таки ..мдааа.

\emph{Denis Dunaev}
Прекрасно написано! Правильно, Женя! По мордасам хлещите этих псевдонауковцив!

\emph{Евгения Бильченко}

\textbf{Denis Dunaev} Спасибо, батюшка, пока приходится только подставлять.
Возможно, потом все изменится, если продержимся. Привет тебе от нашей
молодежки. Вчера видались.

\emph{Evgeny Vinogradov}

Всё соответствует печальной действительности. Оторванная ветка, в общем-то
имела шансы и возможности прижиться на новой почве и пустить корни, но этого не
случилось. Корни не пущены, кора содрана, соки спущены - ветка засыхает.

Причём, применительно к данному социуму, она засыхает не смиренно, а в ней есть
внутренняя сила саморазрушения и она сейчас правит бал, так как имеет для этого
некий запас сил. Она мала в относительном числе, но крайне пассионарна, поэтому
умудряется ещё здоровые клетки ветки держать в подчинении общей тенденции
засыхания. И неважно, что делает это под лживыми лозунгами возрождения, роста,
процветания - они не стоят выеденного яйца. Невозможно сказать гениальнее
классиков МЛ - \enquote{верхи очень хотят ТАК жить, а низы ещё могут так жить}, хоть и
кряхтят. Это объективный процесс, стихия. Чтобы изменить вектор требуется
революционная ситуация и революционные силы. Ни того, ни другого сейчас нет.
Первое ещё как-то может случиться, но со вторым - вопрос, ибо рабочий класс,
как таковой, исчез с лица социумов, заменённый на мэнагеров. Только ему, как
известно, терять было нечего кроме своих цепей. Из этого я вывожу, что
подлинная революция невозможна, а только клановый передел, если исходить из
данных границ. Но не всё, на мой взгляд, так плохо. Точнее, плохо в ближнем
рассмотрении и для людей, изнемогающих внутри. Если же точку зрения поднять
выше, то случайные метаморфозы и приключения ветки не так и важны. Есть, слава
богу, древо, оно хранит генетическую память, традиции, в нём много
неоторвавшихся и создавших прекрасную крону веток. Когда-то, скорее рано, чем
поздно, муки кончатся, ибо силы разрушения, как хворост, перегорят, дерьмо
перегниёт и согнутые пружины разогнутся, явив миру новое. Каков точно будет его
вид - отдельный разговор, но так будет.

\emph{Alexander Violin}
Благодарю сестра!!! 100\% в цель!

\emph{Евгения Бильченко}
\textbf{Alexander Violin} Да, братишка.

\emph{Yuri Rag}

Нормальные постмодерновые казусы. Французский философ свободы Мишель Ельчанинов
Главный редактор Philosophie Magazine: \enquote{Запретить RT France - ослабляет
демократию!}

\emph{Михаил Михайлович}

Преподобный Максим родился в 580 году в знатной константинопольской семье.
Обладая выдающимся умом и редкими способностями к высоким философским
размышлениям, он 10 лет провел в безмолвии, а затем вместе c учеником
Анастасием перебрался в маленький монастырь святого Георгия в Кизике. Там он
положил начало своим первым произведениям: аскетическим трактатам о борьбе со
страстями, о молитве, бесстрастности и святом милосердии. В 626 году совместное
наступление персов и аваров на Константинополь (отраженное благодаря чудесному
вмешательству Божией Матери) вынудило монахов покинуть обитель. Некоторое время
преподобный Максим провел на Крите, где начал борьбу за православную веру,
выступив против монофизитских богословов. Затем перебрался на Кипр и в 632 году
окончательно поселился в Карфагене.

В этом городе он познакомился со святителем Софронием. Преподобный Максим
возводит грандиозную богословскую систему. Согласно ей, человек, помещенный
Богом в этот мир, чтобы быть священником космической литургии, призван собирать
смыслы всех существ, чтобы преподносить их Божественному Слову, их
Первопричине, в диалоге свободной любви. Таким образом человек исполняет
предназначение, ради которого и был создан: то есть, осуществляя союз c Богом,
приводит также всю вселенную к ее совершенству во Христе-Богочеловеке.

Когда на престол взошел император Ираклий, он направил усилия на реорганизацию
пошатнувшейся империи, проведя ряд административных и военных реформ. Особенное
внимание Ираклий уделил восстановлению христианского единства.

Константинопольский патриарх Сергий получил от императора задачу найти
компромиссную догматическую формулу, способную удовлетворить монофизитов, не
отрицая при этом Халкидонский собор. Патриарх предложил учение о моноэнергизме,
согласно которому человеческая природа Христа оставалась пассивной и
нейтральной, так как ее собственное естество было поглощено энергией Слова
Божия. На самом деле речь шла о том же монофизитстве, но немного
завуалированном, то есть термин \enquote{природа} был заменен на термин \enquote{энергия}.

После интронизации патриарх Софроний опубликовал окружное послание, в котором
указал, что каждая природа наделена собственной энергией и Христос имеет одно
Лицо, но две природы и, соответственно, два действия (энергии).

Тем временем преподобный Максим оставался в Карфагене и потихоньку втягивался в
догматическую борьбу в поддержку духовного отца. Не нарушая запрета на тему о
двух энергиях, он тонко проводил мысль, что \enquote{Христос по-человечески совершает
то, что есть Божественное (совершая чудеса) и по-Божественному – то, что есть
человеческое (Животворящие Страсти)}.

Политическая ситуация после завоевания Египта арабами стала более чем
когда-либо непрочной. Поэтому император Констант II (641–668) опасался
открытого разрыва c Римом и в ответ на вмешательство папы издал Типос (648), в
котором любому христианину под страхом сурового наказания запрещалось
дискутировать о двух природах и двух волях. Затем начались преследования и
гонения на православных, в особенности на монахов и друзей преподобного
Максима.

Исповедник встретился в Риме c новым папой Мартином I (память 13 апреля),
который был твердо настроен поддерживать правую веру. По его инициативе в 649
году был созван Латеранский собор, осудивший монофелитство и отменивший
императорский эдикт.... В возрасте 82 лет в крепости Схимар преподобный Максим
окончательно соединился со Словом Божиим. Он так так любил Христа, Животворящим
Страстям Которого подражал своим исповеданием веры и мученичеством. По
преданию, каждую ночь три лампады – символ Святой Троицы – сами собой
зажигались над его гробницей. Святая десница преподобного Максима находится в
монастыре святого Павла на горе Афон.

\emph{Олександр Давиденко}

Занудило деяких марксистів-гуманітаріїв епохи перестройки від кількох тем у
курсі історичного матеріалізму та й вигадали вони нову науку культурологію. А
через чверть століття цією наукою зайнялася Женя Більченко :)

\emph{Славко Артеменко}

\textbf{Олександр Давиденко} АТО ж!
