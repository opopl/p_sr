% vim: keymap=russian-jcukenwin
%%beginhead 
 
%%file 12_01_2022.fb.menendes_enrike.1.radikaly
%%parent 12_01_2022
 
%%url https://www.facebook.com/e.menendes/posts/6803520383023663
 
%%author_id menendes_enrike
%%date 
 
%%tags radikalizm,ukraina
%%title Радикалы
 
%%endhead 
 
\subsection{Радикалы}
\label{sec:12_01_2022.fb.menendes_enrike.1.radikaly}
 
\Purl{https://www.facebook.com/e.menendes/posts/6803520383023663}
\ifcmt
 author_begin
   author_id menendes_enrike
 author_end
\fi

Вчерашний текст «Украина-2030» наверное побил мой рекорд по количеству
комментариев (на данный момент 861). Вообще мне часто пишут замечания в стиле
«вот смотри сколько агрессивных и неадекватных комментариев, как ты можешь
рассчитывать, что у твоих идей будет поддержка». Это вызывает во мне желание
рассказать, как я отношусь к комментариям в Фейсбуке и посоветовать вам, мои
дорогие читатели, взять себе на вооружение мой способ.

Итак, в первую очередь, нужно понимать саму природу социальных сетей. Это
виртуальное пространство, в котором люди позволяют себе значительно больше, чем
в реальной жизни. Здесь люди бывают более агрессивными и менее склонными к
конструктиву. Природа алгоритмов сыграла злую шутку, когда вместо налаживания
диалога с разными точками зрения, они создали вокруг каждого из нас
информационный кокон, в стиле «все кого я знаю, думают как я, а те кто думает
не так - маргиналы».

Второе важное замечание общего характера также связано с природой социальных
сетей. Представители радикальной точки зрения более склонны высказать её вслух,
тогда как представители умеренного лагеря предпочитают воздержаться от
дискуссии – они просто читают и мотают себе на ус. Именно поэтому часто мы
видим в комментариях настоящую вакханалию.

Ещё одно важное качество радикалов – и тут я имею в виду радикалов вне
зависимости от той стороны, которую они выбирают, - это умение сбиваться в
стаю. Помимо своей шумности они ещё и умеют создавать эффект массовости, кочуя
своим цыганским табором из поста в пост, набрасываясь на свою жертву, как стая
гиен. Опять же, люди с конструктивной позицией к такому поведению не склонны.
Это плюс для реальной жизни, но в сетевых баталиях это выглядит как минус,
потому что создаёт ложное ощущение, что нас меньше.

В количественном соотношении важно не забывать, что в фейсбуке представлено
абсолютное меньшинство жителей страны. Помните, как в 2019 году порохоботы
задавили всех числом и активностью в социальных сетях? А результаты выборов
помните? Вот-вот, именно к этому я и веду. Фейсбук – не репрезентативен. Здесь
сидят и лидеры мнений и т.н. пассионарии, которые безусловно влияют на общую
повестку (я люблю употреблять слово дискурс). Но когда дело дойдёт до больших
процессов, это вовсе не значит, что народная воля будет именно на их стороне.

В целом, говоря о радикалах я понял за эти годы одну вещь – они
недоговороспособны. Их нет смысла убеждать, но и нельзя игнорировать. Радикалы
усмиряются противовесом из конструктивных граждан, которые не хотят позволить
агрессивному меньшинству управлять своей жизнью. Это достигается только одним
способом – умением объединяться и предлагать свой собственный нарратив. Если
взять за мерило проблему Донбасса, то согласно социологии, можно отнести к
недоговороспособному меньшинству всего 15\%. Это не мало, но и не много. Вполне
по силам загнать их за Можайск.

И последнее, что я хотел сегодня сказать. Радикальные идеи со временем
коллапсируют. Т.к. они ничего не производят, а чаще всего только разрушают,
жизненный цикл их недолог и вскорости их энергия выгорает и начинает пожирать
своих детей.

Зная эти несколько простых тезисов, я думаю, вы будете относится спокойнее к
сетевым баталиям. И знать, что время для конструктивных идей уже не за горами.

\ii{12_01_2022.fb.menendes_enrike.1.radikaly.cmt}
