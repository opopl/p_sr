% vim: keymap=russian-jcukenwin
%%beginhead 
 
%%file 01_04_2022.fb.bengus_jurij.1.ukr_harkov_dosch
%%parent 01_04_2022
 
%%url https://www.facebook.com/bengusyuri/posts/4983994145026533
 
%%author_id bengus_jurij
%%date 
 
%%tags 
%%title 01.04.2022. В українському Харкові зранку дощить
 
%%endhead 
 
\subsection{01.04.2022. В українському Харкові зранку дощить}
\label{sec:01_04_2022.fb.bengus_jurij.1.ukr_harkov_dosch}
 
\Purl{https://www.facebook.com/bengusyuri/posts/4983994145026533}
\ifcmt
 author_begin
   author_id bengus_jurij
 author_end
\fi

01.04.2022. В українському Харкові зранку дощить. 

7 година ранку. Минулий день в околицях нашого будинку пройшов без \enquote{прильотів}
(хоч в межах офіційного району пишуть що \enquote{прильоти} були, мабуть так і є).

Зранку, серед тиші, сусідка запитала: \enquote{А чому не чути нашої далекобійної арти?
Може не дай Бог їм прилетіло?} 

Відповів: \enquote{Думаю просто немає по кому бити, всіх напередодні відігнали.
Полізуть росіяни знов - почуємо!} І дійсно, невдовзі, по дорозі до рибного
магазину - почули знайомі \enquote{бабахи}. Все працює! Наші щоразу змінюють позиції,
бо чути їх щоразу по-різному. 

Оскільки \enquote{прильотів} було зовсім не чути - дружина зібралась за покупками
(зелень закінчилася). В рибному зелені не було, проте купили черевця лосося,
горішки двох видів, сині родзинки (виявилися дешевшими ніж на ринку) та різні
інші смаколики (у зятя - день народження!). В аптеці завезення товару, тож
докупили те, що раніше шукали. В улюбленому супермаркеті дружини - фантастичне
різноманіття всього, і зокрема свіжа петрушка, укроп і пекінська капуста (для
неї це важливо, і мене привчає  @igg{fbicon.wink} ). А ще дивні темні сицилійські апельсини. Не
думали всього багато купляти, але у наплічник все не вмістилося  @igg{fbicon.wink} 
(перелічувати не буду). Розрахунок карткою - лише для самообслуговування, або
черга. Довелося пробивати все самім  @igg{fbicon.smile}. 

На шляху додому - вирішив підтримати \enquote{місцевого підприємця}, що продавав м'ясо.
Тут вже розрахунок лише \enquote{живими} грошима.

Новини з фронту обнадійливі. На багатьох ділянках ворога женуть. Але є ділянки,
де, за рахунок кількісної переваги росіян, нашим дуже важко. Премію Дарвіна
отримують окупанти, які вирішили сконцентрувати свої сили у \enquote{рудому лісі} і там
окопатися. Я, як ліквідатор 1986-1987 років, добре знаю, що копати там - це
самогубство. Через кілька тижнів рашисти і їхня техніка набралися радіації
достатньо, щоб отримати променеву хворобу. Тепер вони повезли забруднену
техніку і хворих у Білорусь (і там все забруднять). Але  самі винні, ми їх до
нас не кликали.

З районів поблизу Барвенково завчасно евакуйовано багато мешканців. Там зараз
назріває велика битва. Ситуація там дуже складна, бо ворог атакує силами,
знятими з інших фронтів.

Є приємні міжнародні новини. Ще одна країна (Норвегія) дає нам новітню зброю.
Якби ще винищувачі, то взагалі було б добре! Мабуть \enquote{не на часі}, але вибачте,
згадаю бравого вусатого міністра оборони, який колись пишався кількістю знятих
з озброєння і проданих за кордон українських літаків і гвинтокрилів...

\enquote{Браві} осетини їхали допомагати рашистам воювати, але виявилось, що українці
вбивають ворогів. Тому всі 300 кокойтівців кинули зброю і у повному складі
автостопом поїхали додому.

Не варто реагувати на кожні вислови Арахамії, Єрмака чи Аристовича, головне
дивитися що робить Залужни та ЗСУ, їм найбільша довіра.

Продовжується міжнародна ізоляція рашистів і збільшуються санкції. Пуйло вже
пропонує свою нафту за пів-ціни. Європа не бере, але в Азії ще думають...

А в цей час у Харкові прибирають місто. Хотів сфотографувати розбите скло біля
одного з гуртожитків, а там все чисто, вікна затягнуті плівкою. Вулицею їздять
машини, що прибирають пил. У нас знов вивезли сміття, але рознесені вітром
папірці довелося прибирати самім. Відкрилися магазинчики дитячого одягу,
взуття.

Починаються заняття у дистанційному форматі, тож новин на моїй сторінці
поменшає, а ботаніки, мікології та екології - побільшає. Не всі студенти
матимуть змогу бути присутніми на заняттях, тому більшість матеріалів дублюємо
у Фейсбуці, МУДЛі і ЮТУБі. Після війни - надолужимо все, що не встигли.

Україна перемагає. Слава Україні!

\ii{01_04_2022.fb.bengus_jurij.1.ukr_harkov_dosch.cmt}
\ii{01_04_2022.fb.bengus_jurij.1.ukr_harkov_dosch.cmtx}
