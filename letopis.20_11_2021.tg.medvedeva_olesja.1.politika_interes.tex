% vim: keymap=russian-jcukenwin
%%beginhead 
 
%%file 20_11_2021.tg.medvedeva_olesja.1.politika_interes
%%parent 20_11_2021
 
%%url https://t.me/Medvedeva_Olesya/6700
 
%%author_id medvedeva_olesja
%%date 
 
%%tags 
%%title 
 
%%endhead 
\subsection{О том как я заинтересовалась политикой...}
\label{sec:20_11_2021.tg.medvedeva_olesja.1.politika_interes}
\Purl{https://t.me/Medvedeva_Olesya/6700}

\ifcmt
 author_begin
   author_id medvedeva_olesja
 author_end
\fi

Проникновение (простите, иначе не назовёшь) в политику для меня было сродни
принятия чего-то неизбежного, и я прошла все 5 стадий. Опишу в общих чертах.
Если углубиться, то так можно и книгу написать:

1. Отрицание. До 22 лет политика была от меня сильно далека. Проекты, городские
мероприятия, детские дома, экологические субботники и молодежные организация -
это я. Политика - это не я, не про меня, и на оно мне на. Я на неё никак не
влияю, она не трогает меня. 

2. Гнев. Я была очень злая, нет, я была просто в ярости, когда в 2014-ом году
все начало сыпаться у меня на глазах. Моя семья, работа, город, страна. Я
злилась на этих гребенных политиков и политику как явление, ведь она была
виновна в том, что произошло. И я осознала, что она как раз и влияет на всех,
вне зависимости от нашего участия или не участия в ней.

3. Торг. Я искала возможности и пути. Думала о том, что я могу сделать, как я
могу повлиять на весь этот хаос, который вокруг. Уже в Киеве я пробовала разные
профессии, видела, что политика - это единственное, что реально за короткий
промежуток времени может повлиять на ситуацию. Но не хотела быть причастной к
этому. Пусть кто-то другой сделает, а мне надо как-то выжить. У меня ребёнок, и
остатки семьи.

4. Депрессия. Ситуация для меня и моей семьи была максимально неблагоприятная.
В стране начали все переворачивать с ног на голову, натягивать сову на глобус,
переименовывать улицы, сносить памятники и не сдавать в аренду квартиру
«донецким». Мое обостренное чувство справедливости кричало на меня внутри меня,
обвиняла в том, что я бездействую.

5. Принятие. Оценив свои возможности, имеющиеся способности и потенциал к
быстрому обучению я решила много читать все, что касается политологии,
политтехнологии и политической психологии. Ходила на разные тематические
мероприятия, семинары и открытые лекции. Воля случая свела меня с людьми из
сферы медиа и политики. Потом была партия, были политические медиа, потом
политическая журналистика. 

Шло время, я росла, росло и мое обострённое чувство справедливости, принципы,
ценности и цели.

Рассказы о том, что человек сам ко всему пришёл и вообще, он сам все сделал и
такой молодец - ложь. Рядом со мной всегда были люди, которые помогали, или
наоборот мешали. И те и те дали мне бесценный опыт и знания с которыми я иду
вперёд. Мне кажется, что есть в этом мире нечто неизбежное. Как, например,
политика для меня.
