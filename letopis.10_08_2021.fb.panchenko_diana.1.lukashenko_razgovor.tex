% vim: keymap=russian-jcukenwin
%%beginhead 
 
%%file 10_08_2021.fb.panchenko_diana.1.lukashenko_razgovor
%%parent 10_08_2021
 
%%url https://www.facebook.com/permalink.php?story_fbid=1981170848701187&id=100004248730374
 
%%author 
%%author_id panchenko_diana
%%author_url 
 
%%tags belarus,lukashenko_aleksandr,panchenko_diana,press_konferencia,ukraina
%%title 8 часов 15 минут. Таки большой разговор Лукашенко
 
%%endhead 
 
\subsection{8 часов 15 минут. Таки большой разговор Лукашенко}
\label{sec:10_08_2021.fb.panchenko_diana.1.lukashenko_razgovor}
 
\Purl{https://www.facebook.com/permalink.php?story_fbid=1981170848701187&id=100004248730374}
\ifcmt
 author_begin
   author_id panchenko_diana
 author_end
\fi

8 часов 15 минут. Таки большой разговор Лукашенко. Настолько, что последние
часы мои мысли были заняты прежде всего попытками разгадать физиологическую
стойкость собравшихся. 

Из наблюдений. Задать вопросы могли все желающие. И лояльно, и враждебно
настроенные. Так, слово дали CNN, BBC и другим западным СМИ.  Более того, между
первыми и вторыми регулярно вспыхивали дебаты. 

И это впечатлило. Подобного я никогда не видела на пресс-конференциях президентов. 

Впрочем, это была таки не пресс-конференция, а именно разговор. Скорее
эмоциональный, чем по факту. Лукашенко хотелось выговориться. Чувствовалось,
что Батька обижен, чего не скрывал. На оппозицию и тех, кто ее поддержал, на
западные и иногда российские СМИ. Фраза, которая врезалась в память: «Не
волнуйтесь, Лукашенко уйдёт. Возможно, скоро»

Меня, конечно, интересовала украинская тематика. Украина уже перешла красную
линию в отношениях с Беларусью. Когда, даже не дожидаясь ЕС, прекратила
авиасообщение с Минском. Об этом Александр Лукашенко сказал, отвечая на мой
вопрос. 

Сегодня Отношения Украины и Беларуси на критически низком уровне. Я спросила,
при каких обстоятельствах может быть хуже. Где та самая точка невозврата. Так
вот, на его взгляд, она уже пройдена. 

Лукашенко ещё раз подтвердил намерение открыть прямое авиасообщение с Крымом.
Поясню, почему назвала этот вопрос болезненным для Киева. Это не просто
самолёты в Крым, это маркёр. Все эти годы беларусские самолёты не летали
напрямую на полуостров. Лукашенко удавалось лавировать между Украиной и
Россией, сохраняя нейтралитет и позицию миротворца. 

«Вы сами открыли мне дорогу в Крым..»Можно не любить Лукашенко и не разделять
его политику, но запретив полеты беларусским авиакомпаниям, Украина потеряла
друга и союзника. 

Навсегда ли? 

В октябре 19-го года Батька сказал, что никогда не допустит вторжения третьей
страны   в Украину через беларусскую границу. «Разве что на тракторе» - пошутил
Григорьевич на встрече с Зеленским. Я спросила, может ли он сейчас подтвердить
свои слова. 

«Вам точно не стоит опасаться угрозы со стороны Беларуси.» - таким был ответ
Лукашенко. 

За 2 коротких визита в Беларусь я успела проникнуться этой страной. Я правда
болею за неё и желаю светлого будущего. Каким путём к нему идти - дело самих
беларусов. 

Но сигналов о том, что над Минском снова сгущаются тучи - предостаточно. И
главный, к сожалению, трагичный - раскол в обществе. В Украине он привёл к
трагедии. 

А напоследок Лукашенко ответил на нашу просьбу рассмотреть вопрос о возможности
трансляции Первого Независимого в Беларуси. Это было бы здорово. Чтоб канал,
который по беспределу закрыли в Украине, могли смотреть соседи.
