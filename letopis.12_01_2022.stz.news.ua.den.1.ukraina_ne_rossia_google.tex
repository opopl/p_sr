% vim: keymap=russian-jcukenwin
%%beginhead 
 
%%file 12_01_2022.stz.news.ua.den.1.ukraina_ne_rossia_google
%%parent 12_01_2022
 
%%url https://day.kyiv.ua/uk/blog/polityka/ukrayina-taky-ne-rosiya-svidchennya-google
 
%%author_id grabovs'kij_sergіj
%%date 
 
%%tags ukraina,rossia,google,internet,tradicii,mova,jazyk
%%title Україна – таки не Росія: свідчення Googlе
 
%%endhead 
\subsection{Україна – таки не Росія: свідчення Googlе}
\label{sec:12_01_2022.stz.news.ua.den.1.ukraina_ne_rossia_google}

\Purl{https://day.kyiv.ua/uk/blog/polityka/ukrayina-taky-ne-rosiya-svidchennya-google}
\ifcmt
 author_begin
   author_id grabovs'kij_sergіj
 author_end
\fi

На початку 2022 року пошукова система Googlе незаперечно засвідчила, що Україна
– не Росія. Звісно, ми це і так знаємо (втім, деякі наші земляки в тому досі
сумніваються), проте приємно. коли «електронний розум» - попри всі хиби
програмування, де-факто на користь «русского мира», адже пошуковик не дуже
розрізняє російську й українську мови, засвідчує те саме.

Отже: за передріздвяний тиждень Googlе зафіксував на теренах України (навіть на
окупованих територіях, окрім Севастополя) домінування пошуку за ключовими
словами «різдво», «риздво», «риздвом», «різдвом», «різдва», «риздва» над
«рождеством», «рождестве» (чогось ці слова «електронний розум» подає з малої
літери, мабуть, його довго вчили політкоректності...). І навіть на Донбасі та в
Криму наявне таке домінування! Так само зі словами «святвечір», «святвечер»,
«святий вечір», «святой вечер» - знов-таки, лише Севастополь демонструє
перевагу у пошуковику слова «сочельник».

Що ж, зайвий раз підтверджено те, що понад 30 років тому писав російський
академік Сергій Аверінцев: обрядовість різдвяного циклу в Європі єдина від
Лісабону до Конотопу, а Росія – це інший культурний материк. Навіть назви
елементів різдвяних свят в українців (значної кількості російськомовних у тому
числі) не збігаються з російськими; ба більше, відмінності наразі вже рушили на
схід і південь, далеко за Конотоп.

Цікаво, що по практично всій Росії домінує, ясна річ, «сочельник» - але при
цьому ціла низка регіонів РФ дивовижним чином випала з огляду пошукової системи
Googlе, немовби це якісь інші країни; втім, схоже, що інші справді інші –
йдеться про Північний Кавказ, Чукотку, Туву й іще цілу низку національних
автономій...

А разом із тим на теренах Білорусі слово «каляди» беззаперечно домінує над
«рождеством». Отож, видається, чутки про повну перемогу Лукашенка над
білоруською нацією та не менш повну русифікацію цієї країни перебільшені.

Видається, ще більш разючими стали б результати обробки запитів у Googlе
впродовж наступного, післяріздвяного тижня. Адже що відзначають у Росії?
Правильно, «старий новий рік». Абсурдне словосполучення, проте хіба мало в
житті абсурді? В Україні також говорять про «старий новий рік» - раніше більше,
особливо у радянські часи, зараз менше. А от питомо українськими святами цього
періоду є «Маланка», або ж «Меланка» (так здавна зве народ це свято), яке в
цьому столітті припадає на 13 січня за григоріанським і новоюліанським
календарями, і наступне «Василя», що відзначається 14-го. Тут існує своя давня
обрядовість, яка включає ритуальний стіл, щедрування й засівання, ворожіння,
рядження тощо. На відміну від Святвечора, святковий стіл обов’язково повинен
містити й скоромні страви, адже це був щедрий або ж багатий вечір. А ще за
народними повір’ями, в ніч з 13 на 14 січня «відкривається небо» і в Бога можна
попросити що завгодно, навіть щоби «вода в криницях перетворюється на вино».
Можна назвати й чимало різноманітних звичаїв та повір’їв, пов’язаних із днем
Василя. Ясна річ, що чоловіків, яких звали Василями, цього дня вітали, а на
Закарпатті їх ще й тричі піднімали вгору. Одне слово, веселі свята (не
випадкове ж різдвяне побажання – «веселих свят», не в однині, а у множині) – не
те що якийсь там «старий новий рік» з горілкою, наїдками та банальними тостами…

Звісно, «ожили» ще далеко не всі українські традиції новорічно-різдвяного циклу
(точніше, різдвяно-новорічного, бо саме така черговість «трьох празників», які
йдуть у гості до нас, - спершу Різдво Христове, потім святого Василя, потім
Святе Водохреще). Проте так чи інакше фактом є істотна різниця цього циклу – як
на лексичному рівні, так і на рівні обрядовості – від російських звичаїв. А
тому Путін з його «одним народом» справді живе у своєму химерному світі, що не
стикується з реальністю, проте вперто прагне підігнати реальність під свій
копил. І що йому тисячолітні традиції, що йому зафіксована пошуковою системою
Googlе у передріздвяний час різниця між росіянами й українцями (і росіянами й
білорусами також)! У нього свої свята – а у нас свої. І щедрівка на «Маланку» –
то не про нього і не до нього:

Щедрий вечір,

Добрий вечір,

Добрим людям

На здоров’я...
