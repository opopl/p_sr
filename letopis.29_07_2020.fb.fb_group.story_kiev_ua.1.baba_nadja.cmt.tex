% vim: keymap=russian-jcukenwin
%%beginhead 
 
%%file 29_07_2020.fb.fb_group.story_kiev_ua.1.baba_nadja.cmt
%%parent 29_07_2020.fb.fb_group.story_kiev_ua.1.baba_nadja
 
%%url 
 
%%author_id 
%%date 
 
%%tags 
%%title 
 
%%endhead 
\zzSecCmt

\begin{itemize} % {
\iusr{Наталья Рыбкина}
злобно

\begin{itemize} % {
\iusr{Андрей Журавлев}
я бы это назвал скорее сарказмом.

\iusr{Андрей Шиян}
\textbf{Наталья Рыбкина} не согласен! Очень человечно и с юмором!

\iusr{Володимир Шуневич}
\textbf{Наталья Рыбкина} Вы правы. Я бы сказал резче - говнистенько как-то...

\iusr{Андрей Журавлев}
\textbf{Володимир Шуневич} огромное спасибо за ваш отзыв. Я серьезно. Кто то в комментах рассказ хвалит, а кто то ругает. И это нормально. Значит рассказ получился!)

\iusr{Наталья Рыбкина}
\textbf{Володимир Шуневич} Следующая публикация этого автора подтверждает Ваше наблюдение.

\iusr{Андрей Журавлев}

Человек без чувства юмора - это тяжёлый случай. Жизнь видать не сильно балует.
А насчёт юмора в рассказах - я заранее именно с ВАМИ согласен, что там его и
близко нет)))

\end{itemize} % }

\iusr{Олег Московко}

Школа \#66 возле Индустриального моста - тоже английский с первого класса, а в
десятом - до десяти уроков в неделю... те же \enquote{допросы военнопленных} и изучение
структуры американской армии, правда директора были \enquote{нормальными}...но доставало
по 6-7 человек в подгуппе на уроках английского, приходилось \enquote{готовиться}. Хотя
со знанием английского выше \enquote{Ландон - из зе кепитал...} оканчивали не многие. @igg{fbicon.smile} 

\begin{itemize} % {
\iusr{Андрей Журавлев}

я вам скажу так - подобные школы давали очень хорошую БАЗУ языка. Я закончил
школу 40 лет назад, по английскому у меня было 4. Язык мне понадобился в начале
2000-х (я - инженер). Буквально за месяц-другой практики я вспомнил почти все.
Я сам был очень удивлен. В итоге спикаю, пишу и т.д. очень даже гут.


\iusr{Олег Московко}
\textbf{Андрей Журавлев} 

ключевое - \enquote{понадобился}, я закончил в 77, нженерил, керувал... но английским
пользовался только дабы узнать сколько ликера надо для коктейля.


\iusr{Андрей Журавлев}
\textbf{Олег Московко} все правильно. Но без хорошей языковой базы я бы \enquote{учил} все по новой неизвестно сколько времени. А так хватило пару месяцев.
\end{itemize} % }

\iusr{Alla Alla}

Прочла с мыслью, что наша директриса была точный клон бабы-Цеткин, значит их
делали под копирку и садили на директорские места, чтобы школьное время
запомнилось, как ужас всей жизни @igg{fbicon.laugh.rolling.floor}{repeat=3} 

\begin{itemize} % {
\iusr{Володимир Шуневич}
\textbf{Alla Alla} 

Очень Вам соболезную. Для нас школьные годы были лучшим, самым счастливым
временем жизни. Обьездили с учителями и директором на школьном автобусе
полСоюза от Ленинграда до Сухуми и от Бреста до Волгограда. Спортом занимались
весьма успешно, учителя были интересные, кружки. Директор тоже был по-своему
оригинал, но все любили его как отца. И школа была одной из лучших в
Житомирской области... Многие киевские - хлевы по сравнению с ней. Кстати, у
нас английский преподавали со 2-го класса...

\iusr{Маргарита Мышанская}

Вспоминаю школу с содроганием. Бесконечное унижение личности детей и
подростков. Бесконечные вычитывания по любому поводу (как в анекдоте \enquote{почему в
шапке, почему без шапки?}), бестактность и наглость учителей. С облегчением
вздохнула, когда все закончилось. Институт (КПИ, ТЭФ) показался раем. Стараюсь не
думать и не вспоминать школу вообще... Знакомимся-школа №4, Соломенка. @igg{fbicon.face.pensive} 

\end{itemize} % }

\iusr{Ольга Даненко}

Все бы ничего, но украинский язык в исполнении такой личности как вы ее описали
звучит весьма неправдоподобно.

\begin{itemize} % {
\iusr{Олег Московко}
\textbf{Ольга Даненко} все \enquote{спецшколы} были україномовными.

\iusr{Hardashnyk Irene}
\textbf{Олег Московко} Неправда! 

Я окончила 32-ю, т. н. английску. ю базовую школу ин яза, на ул. Федорова, школа
была русская, а вот недалеко находящиеся 56-я и 130-я, действительно были
украинскими.

\iusr{Андрей Журавлев}
\textbf{Hardashnyk Irene} Все вы правильно говорите, но 32-ю вряд ли можно назвать спецшколой. Программа по инглиш там была хорошая, но менее насыщенная, чем, например, у нас в 87-й. У вас не было техн. перевода и военного перевода.

\iusr{Олег Московко}
\textbf{Hardashnyk Irene} исключения только подтверждают правила! @igg{fbicon.smile} 

\iusr{Ольга Даненко}
\textbf{Andriy Zhuravlyov} 

я не знаю сколько вам лет и настолько ли вы старый как я, но я прекрасно помню
отношение к украинской мове в советские годы и тем более в Киеве. Так что
извините, но я вам не верю.

\iusr{Ольга Сахарова}
\textbf{Ольга Даненко} 

Были украинские школы, были говорящие на украинском языке учителя, дети,
семьи... \enquote{Бабу Надю} помню плохо, но была свидетелем, как говорили между собой
другие учителя по-украински. Отношение было чаще к конкретным людям.

\iusr{Андрей Журавлев}
\textbf{Ольга Даненко} 

это ваше личное дело. Кстати, в комментах выше есть 2 человека, которые учились
в моей школе. Они почему то поверили. А я в этой школе отучился с 1969 по 1979
годы. все было именно так. Другое дело - отношение к укр языку в быту. тут я с
вами соглашусь.

\iusr{Андрей Журавлев}

Ольга, у нас была украинская школа и почти все учителя говорили по украински
даже на переменках. В том числе и баба Надя. Как они и она общались в быту - я
не знаю.

\iusr{Олег Московко}
56, 130, 87, 66, 80 ...судя по комментам - україномовні.

\end{itemize} % }


\end{itemize} % }
