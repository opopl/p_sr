% vim: keymap=russian-jcukenwin
%%beginhead 
 
%%file 23_12_2020.news.ru.ukraina_ru.stojakin_vasilii.1.sikorskii_kievljanin
%%parent 23_12_2020
 
%%url https://ukraina.ru/history/20201223/1030083312.html
 
%%author 
%%author_id stojakin_vasilii
%%author_url 
 
%%tags sikorskii_igor,kiev
%%title Как в Петербурге родился небесный богатырь, придуманный киевлянином
 
%%endhead 
 
\subsection{Как в Петербурге родился небесный богатырь, придуманный киевлянином}
\label{sec:23_12_2020.news.ru.ukraina_ru.stojakin_vasilii.1.sikorskii_kievljanin}
\Purl{https://ukraina.ru/history/20201223/1030083312.html}
\ifcmt
	author_begin
   author_id stojakin_vasilii
	author_end
\fi

\index[names.rus]{Сикорский, Игорь!Авиаконструктор, 23.12.2020}
\index[rus]{Илья Муромец!Самолет, 23.12.2020}

\ifcmt
pic https://cdn1.img.ukraina.ru/images/103008/33/1030083399.jpg
\fi

Корпусный аэродром Санкт-Петербурга находился в районе нынешней станции метро
«Электросила». Именно здесь 23 декабря 1913 года впервые поднялся в воздух
самолёт конструкции Игоря Сикорского С-22, названный «Илья Муромец». Самолёт,
ставший этапным для всей мировой авиации

Разумеется, гигантские самолёты \textbf{Сикорского} могли быть построены только
в большой стране, которой была Российская империя. Однако они прочно связаны с
Киевом.

о-первых, происхождением самого Игоря Ивановича — русского человека, жившего в
Киеве на Большой Подвальной улице (сейчас это Ярославов Вал).

Во-вторых, тем, что исторический прототип былинного Ильи Муромца служил
киевскому князю, под конец жизни стал монахом Печерского монастыря и похоронен
в Верхних пещерах Киево-Печерской лавры.

«Илья Муромец» не стал первым в мире четырёхмоторным самолётом только потому,
что ему предшествовал взлетевший 26 мая того же года С-21 «Русский витязь», на
конструкцию которого Сикорский, собственно, опирался. «Илья Муромец» строился
на Русско-балтийском вагонном заводе, что несколько отразилось на его
конструкции и внешности.

Когда «Витязь» взлетел, в авиационном мире того времени это сочли неудачной
шуткой — с первого полёта братьев Райт прошло всего 10 лет, никто просто не мог
себе представить самолёт с несколькими моторами. Чтобы пресечь слухи и собрать
деньги на развитие проекта, Сикорский усадил на борт всех желающих и провёз их
над Невским проспектом. Движение в городе остановилось — такого ещё никто не
видел. Ещё бы! Ширина Невского у Мойки — 25 метров, а размах верхнего крыла
«Витязя» — 27 метров (у «Муромца» — 34,5).

Моторная схема самолёта менялась. Сначала он был двухмоторным с тянущими
винтами, потом четырёхмоторным с тянущими и толкающими винтами и наконец —
четырёхмоторным с тянущими.

Век «Витязя» был недолгим. 11 сентября 1913 года на 3-м конкурсе военных
самолётов сорвавшийся с пролетавшего над «Русским витязем» самолёта «Меллер-II»
двигатель упал на левую коробку крыльев русского самолёта, сильно повредив её.
Самолёт не стали восстанавливать.

Конструкция нового самолёта и его двухъярусная коробка фанерных крыльев с
установленными на нижней консоли четырьмя моторами остались без особых
изменений. Разве что двигатели «Аргус» имели мощность не 100, а 140 л.с., что
позволило разогнать самолёт до 105 км/ч против 90 у «Витязя». Выросли также
потолок (до 3000 м) и полезная нагрузка (с 600 до 1500 кг).

Изначально самолёт задумывался как пассажирский. Впервые в истории авиации он
был оснащён отдельным от кабины салоном, спальными комнатами и ванной с
туалетом. В пассажирской кабине имелось отопление, электрическое освещение и
даже небольшой «бульвар» — пассажиры в полёте могли выходить на палубу,
оборудованную на хвостовой балке (на крылья для контроля состояния двигателей
механики выходили ещё и через 20 лет после «Ильи Муромца», когда скорости
самолётов выросли вдвое).

12 декабря 1913 года «Илья Муромец» установил рекорд грузоподъёмности в 1100 кг
(предыдущий рекорд составлял 653 кг), 12 февраля 1914 года в воздух было
поднято 16 человек и собака, общим весом 1290 кг. Второй самолёт в июне 1914
года совершил перелёт из Петербурга в Киев с одной пересадкой.

К началу Первой мировой войны было выпущено 4 самолёта, из которых один был
переоборудован в поплавковый гидроплан.

23 декабря 1914 года император \textbf{Николай II} утвердил решение военного совета о
создании Эскадры воздушных кораблей. Теперь это День дальней авиации России.

За годы войны в войска поступило 60 машин (всего построено 73). Военные
варианты самолёта несли от двух до шести пулемётов разного типа и от 350 до
1500 кг бомб или иной нагрузки (против кавалерии, например, использовались
кованые стрелы). Для сброса бомб был разработан электросбрасыватель. На двух
самолётах были установлены 37 мм пушки для борьбы с германскими цепеллинами, но
испытания показали их низкую эффективность, и в боевых условиях они не
применялись.

Эскадра совершила 400 боевых вылетов, сбросила 65 тонн бомб и уничтожила 12
вражеских истребителей. В частности, при попытке атаковать «Муромца» погиб
гауптман фон Маккензен, сын командующего 9-й германской армией, будущего
маршала. При этом за всю войну непосредственно истребителями неприятеля была
сбита всего 1 машина (которую атаковало сразу 20 самолётов), а подбито — 3.
Полной статистики по потерям С-22 нет, но по меньшей мере 13 самолётов были
разбиты в лётных происшествиях, один — сбит зенитным огнём, два — уничтожены
экипажами ввиду приближения неприятеля при невозможности взлететь.

Военные, например Алексей Брусилов, полагали, впрочем, что свою функцию
«Муромец» не выполнил. Эффект от применения огромных воздушных кораблей
ожидался большим, но тут многое зависело от тактики и управления, а придумать
их было нельзя — только выработать с опытом.

Игорь Сикорский, как известно, Октябрьскую революцию не принял и уехал в США.
Молодая Советская республика не имела необходимых ресурсов, чтобы наладить
эксплуатацию этих сложных машин. В ходе Гражданской войны было сделано всего
несколько боевых вылетов. Самолёты Эскадры воздушных судов Украинского
государства Павла Скоропадского и вовсе не совершили ни одного полёта. Впрочем,
за всё время существования гетманата удалось только наладить учёт и
складирование имущества.

После войны непродолжительное время осуществлялись полёты по маршрутам
Екатеринбург — Сарапул и Москва — Харьков. Последние полёты С-22 относятся к
1922–1923 годам и были учебными.

Дальнейшие попытки строить тяжёлые самолёты подобной схемы к особым успехам не
привели. Сама конструкция быстро устарела, и от неё отказался и сам Сикорский.

Разумеется, ни один самолёт по сей день не сохранился, но в 1979 году режиссёр
Даниил Храбровицкий (широко известна его картина «Укрощение огня» о Сергее
Королёве) решил снять фильм о Сикорском. Поскольку персонаж этот был
антисоветский, чтобы обойти требования Главлита, был исполнен хитрый
идеологический манёвр: по сценарию Сикорский познакомился с Андреем Туполевым
ещё до революции (чего по ряду причин быть не могло) и на протяжении всего
фильма происходит соревнование двух знаменитых конструкторов. В фильме, кстати,
появляется и другой беженец — Сергей Рахманинов. В «Поэме о крыльях» снялись
замечательные актёры — Юрий Яковлев (в роли Сикорского), Владислав Стржельчик
(Туполев), Олег Ефремов (Рахманинов).

Естественно, к такому актёрскому коллективу надо было придать и соответствующее
техническое оформление. В 1978 году студентами Рижского института инженеров
гражданской авиации по чертежам, сделанным в КБ Миля (сотрудники которого
выступали консультантами — ведь Сикорский был зачинателем и мирового
вертолётостроения), был изготовлен полноразмерный макет «Ильи Муромца»,
снабжённый настоящими двигателями чешского производства. Макет способен делать
пробежки по аэродрому и использовался в съёмках в варианте с двумя и четырьмя
моторами.

С 1985 года этот аппарат помещён в экспозицию Центрального музея ВВС России в
Монино.

