% vim: keymap=russian-jcukenwin
%%beginhead 
 
%%file 01_11_2022.stz.news.ua.donbas24.1.mrpl_hudozhnyci_nova_vystavka_dortmund.txt
%%parent 01_11_2022.stz.news.ua.donbas24.1.mrpl_hudozhnyci_nova_vystavka_dortmund
 
%%url 
 
%%author_id 
%%date 
 
%%tags 
%%title 
 
%%endhead 

Ольга Демідко (Маріуполь)
01_11_2022.olga_demidko.donbas24.mrpl_hudozhnyci_nova_vystavka_dortmund
Маріуполь,Україна,Мариуполь,Украина,Mariupol,Ukraine,Выставка,Виставка,Exhibition,Олена Украінцева,Художник,Artist,Німеччина,Дортмунд,date.01_11_2022

Маріупольські художниці представили нову виставку в Дортмунді (ФОТО)

У німецькому місті Дортмунд відкрилася виставка, де можна побачити роботи,
присвячені життю в Маріуполі під час війни

28 жовтня в Шахті-музеї «Цехе Цоллерн», у Дортмунді, відкрилася виставка
PostMost, яка була створена учасницями попередніх проєктів в період
арт-резиденції в гостинному просторі ArToll міста Бедбург-Хау. Про це
повідомила маріупольська художниця Олена Украінцева.

«Мета виставки — з одного боку вкотре привернути увагу світу до подій, що
відбуваються в Україні, емоційно включити глядачів в ситуацію. З іншого боку —
це цілюща зустріч для маріупольських художниць та важливий крок в їхній
реабілітації після пережитих жахів», — наголосила Олена.

Читайте також: Як у Німеччині створюються футболки, присвячені Маріуполю —
деталі

Учасниці резиденції здебільшого є маріупольськими художницями, що через війну
залишили рідне місто та наразі перебувають в Європі. На виставці у вигляді
постерів були представлені і роботи маріупольських художників, які перебувають
в Україні.

У виставці взяли участь Антоніна Дурнєва, Анастасія Шишкіна, Олександр
Малаховський, Hilmenna, Simona, Інна Абрамова, Марина Черепченко, Ivetta Becker
та Олена Украінцева. Перед виставкою художниці протягом двох тижнів працювали
разом в арт-резиденції АрТолл в Бедбург-Хау.

Читайте також: Окупанти в Маріуполі майже знищили мурал «Мілана»: навіщо вони це роблять (ФОТО)

Також на виставці можна побачити інсталяцію-розповідь про життя в Маріуполі під
час війни.

«Це окрема болюча тема як для митців, так і для глядачів. Троє з наших
художниць виживали в Маріуполі в підвалах. Це жахіття їм довелося згадувати,
аби відтворити умови побуту, в яких знаходилися деякий час», — розповіла Олена
Украінцева.

Звуки обстрілів доповнювали загальне враження. Так художники намагалися більш
детально донести сенси, що закодовані в їхніх роботах. На відкриття виставки
прийшло багато українців-біженців з різних міст України. Також захід відвідали
й ті українці, які приїхали жити до Німеччини ще до повномасштабного вторгнення
рф в Україну. Всі глядачі залишилися під сильним враженням від побаченого.

Читайте також: Художник Денис Метелін долучився до благодійного проєкту зі
збору коштів для ЗСУ (ВІДЕО)

«Я сам з України, до Німеччини приїхав багато років назад. Для мене все, що
відбувається в Україні — велика трагедія. Ця виставка — це вперше за всі 8
місяців така потужна рефлексія-відповідь від маріупольців. Роботи художників
дозволили по-справжньому зрозуміти, що пережили люди. Ми бачимо на прикладі
маріупольців, які виїхали, що українці намагаються висвітлювати правду. І це
дає велику надію нам тут», — поділився думками відвідувач виставки Олексій
Кісліцин.

Виставка експонуватиметься до 24 березня 2023 року.

Раніше Донбас24 розповідав, як маріупольський художник Олексій Лукьянов
відтворює трагедію Маріуполя у своїх картинах.

Ще більше новин та найактуальніша інформація про Донецьку та Луганську області
в нашому телеграм-каналі Донбас24.

ФОТО: з архіву учасниць проєкту.
