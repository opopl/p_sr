% vim: keymap=russian-jcukenwin
%%beginhead 
 
%%file 30_12_2021.fb.fb_group.story_kiev_ua.1.rizdvo_derevo
%%parent 30_12_2021
 
%%url https://www.facebook.com/groups/story.kiev.ua/posts/1829737310556395
 
%%author_id fb_group.story_kiev_ua,olejnikov_maksim
%%date 
 
%%tags istoria,kiev,novyj_god,prazdnik,rizdvo,tradicii
%%title ЯК РІЗДВЯНЕ ДЕРЕВО СТАЛО НОВОРІЧНИМ
 
%%endhead 
 
\subsection{ЯК РІЗДВЯНЕ ДЕРЕВО СТАЛО НОВОРІЧНИМ}
\label{sec:30_12_2021.fb.fb_group.story_kiev_ua.1.rizdvo_derevo}
 
\Purl{https://www.facebook.com/groups/story.kiev.ua/posts/1829737310556395}
\ifcmt
 author_begin
   author_id fb_group.story_kiev_ua,olejnikov_maksim
 author_end
\fi

ЯК РІЗДВЯНЕ ДЕРЕВО СТАЛО НОВОРІЧНИМ.

Зимові свята наші предки-кияни починали святкувати з 25 грудня, коли за діючим
тоді календарем відмічалось Різдво Христове, яке у дореволюційному Києві було
центральним зимовим святом, набагато важливішим, аніж Новий рік, бо майже всі
люди були віруючими. Оскільки свято передувало зустрічі Нового року, то ялинка
вважалася не новорічною, а саме різдвяною і з’являлася в домівках киян до 24
грудня. Хоча традицію святкування Нового року 1 січня з ялинками, вітаннями і
подарунками ввів ще Петро І, у Києві Новий рік почали святкувати лише з
середини ХІХст. До того часу не лише у Києві, а й по всій імперії новорічні
святкування були рідкістю. Натомість Різдво Христове було одним із самих
улюблених і шанованих свят. У «Киевских календарях», що видавалися наприкінці
ХІХ - на початку ХХ століть щорічно, у списку «неприсутственных», тобто
вихідних днів для всіх установ і підприємств міста значилося не лише Різдво, а
й наступні дві доби. 

\begin{multicols}{2} % {
\setlength{\parindent}{0pt}

\ii{30_12_2021.fb.fb_group.story_kiev_ua.1.rizdvo_derevo.pic.1}
\ii{30_12_2021.fb.fb_group.story_kiev_ua.1.rizdvo_derevo.pic.1.cmt}

\ii{30_12_2021.fb.fb_group.story_kiev_ua.1.rizdvo_derevo.pic.2}
\ii{30_12_2021.fb.fb_group.story_kiev_ua.1.rizdvo_derevo.pic.2.cmt}
\end{multicols} % }

Вперше згадка про ялинки у квартирах киян з’явилась у «Киевскіх губернскіх
вєдомостях» у січні 1855 року. В газеті писалося, що більш-менш заможні батьки
прикрашали ялинки стрічками і іграшками, які купувалися в основному у іноземних
магазинах. Згодом більшість прикрас люди почали робити самі, з підручних
матеріалів. Так на ялинці з’явилися іграшки із соломи, картону, вати, шишок,
саморобні гірлянди, а найвідомішими ялинковими прикрасами Києва були позолочені
горішки, які в великій кількості продавалися на базарах напередодні свята.
Верхівку різдвяної ялинки прикрашала Віфлеємська зірка. Під ялинкою або в іншій
частині будинку робили вертеп – сценку, яка відтворює народження немовляти
Ісуса.   

\ii{30_12_2021.fb.fb_group.story_kiev_ua.1.rizdvo_derevo.pic.3}

На звання головної ялинки Києва в той час тоді претендували два дерева. Перше
встановлювали в Дворянському зібранні на Хрещатику – цей будинок не зберігся
(у радянський час на його місці побудували Будинок профспілок). Друга велика
ялинка красувалася в Купецькому зібранні – будівлі, де сьогодні розташовується
Національна філармонія. Саме там влаштовували найбільше дитяче різдвяне
дійство.

% Дореволюційна різдвяна листівка. Дата свята вказана – 25 грудня.
\ii{30_12_2021.fb.fb_group.story_kiev_ua.1.rizdvo_derevo.pic.4}

Між Різдвом і Водохрещем проходили веселі Святки, на час яких і випадав Новий
рік. На відміну від ночі перед Різдвом, коли всі святкували вдома у сімейному
колі, в новорічну ніч діяли усі розважальні заклади міста, проходили вечори і
бали-маскаради. Бажаючі веселилися в театрі-варьєте «Аркадія» на Бібіковському
бульварі, в кафе «Шато-де-Флер», ресторанах і т.п. З часом стали святкувати
прихід Нового року в гімназіях і училищах, дитячих сиротинцях (на благодійні
пожертви від балів і передсвяткових розпродажів). На відміну від Різдва, коли
вся родина збиралася вдома, в новорічну ніч було прийнято відвідувати бали,
наносити візити друзям і родичам. 

З початком Першої Світової війни, в яку Росія вступила у 1914р., ялинка раптом
потрапила в немилість. В країні почалася антинімецька кампанія, і ялинку
оголосили «ворожою», бо вона свого часу прийшла до Росії з Німеччини... Та що там
якась ялинка, якщо навіть столицю імперії тоді в патріотичному угарі
перейменували, щоб звучала не на німецький кшталт («-бург»), а на «ісконно
скрєпний» - Петроград.

Після більшовицького перевороту заборону на ялинку відмінили. Хто з мого
покоління не пам’ятає сусальні оповідання про Леніна на дитячій ялинці в
Сокольниках?...

Реформа в січні 1918 року (в Україні – в лютому 1918р.) ввела григоріанський
календар, який православна церква брати до уваги не захотіла, продовжуючи
відзначати Різдво за старим, юліанським календарем. І день 25 грудня – дата
традиційного святкування – припав на 7 січня за новим стилем. У підсумку
виникла плутанина. А слідом за реформою календаря для радянської влади і
різдвяні традиції виявилися класово чужими, і з ними почали боротися. Різдво та
ялинка, однак, формально не заборонялися, але вважалися проявом нелояльності та
«політичної незрілості». Радянський службовець, а особливо партієць, якого б
викрили на встановлені ялинки для своїх дітей, міг й партквиток на стіл
покласти та позбутися посади. А от хтось із «бувших» — й до таборів загриміти
«за релігійну пропаганду».

\ii{30_12_2021.fb.fb_group.story_kiev_ua.1.rizdvo_derevo.pic.5}

У 1926р. ЦК ВКП(б) назвав звичай встановлювати «так называемую рождественскую
ель» антирадянським, а роком пізніше почалася антирелігійна кампанія, що
висміювала традиції церковних свят. Викорінювалися народні звичаї, зокрема і
виконання «музыкальных произведений на религиозно-библейскую тематику», тобто
колядок і щедрівок. Замість них почали з’являтися от такі «шедеври»: 

«Добрий вечір тобі, вільний пролетарю!

Нова радість стала, яка не бувала:

Довгождана зірка волі в Жовтні засіяла.

Де цар був зажився, з панством вкорінився,

Там з голотою простою Ленін появився!».

\ii{30_12_2021.fb.fb_group.story_kiev_ua.1.rizdvo_derevo.pic.6}

Замість релігійного свята влада стала проводити для школярів та молоді різні
військово-спортивні заходи. А дорослим запропонували проводити різдвяні дні на
антирелігійних лекціях і самодіяльних сценічних постановках про революцію.

\ii{30_12_2021.fb.fb_group.story_kiev_ua.1.rizdvo_derevo.pic.7}

Однак пересічні громадяни, незважаючи на політику державного атеїзму,
продовжували святкувати у сімейному колі Різдво. Кияни часто йшли а хитрість і
святкувати потайки. Різдвяні ялинки продовжили встановлювати та прикрашати,
однак їх ховали від допитливого ока в комірчині або навіть у шафі. А містом
ходили пильні комсомольські і профспілкові патрулі, які перевіряли, чи немає де
таємно встановленої ялинки – «релігійного хламу». Паралельно випускали
агітаційні матеріали, покликані повернути увагу громадян в інше русло. Мовляв,
нічого тут святкувати, краще займатися спортом на свіжому повітрі.

\ii{30_12_2021.fb.fb_group.story_kiev_ua.1.rizdvo_derevo.pic.8}

Відразу викоренити звичні народові свята було не так-то просто. Тому ще не один
рік головні церковні дати вважалися «красными днями» нарівні з державними
святами. Календар на 1928 рік в переліку неробочих днів мав не лише 1 травня,
7-8 листопада чи, скажемо. День Паризької комуни, але й Пасху, Трійцю, Різдво...

\ii{30_12_2021.fb.fb_group.story_kiev_ua.1.rizdvo_derevo.pic.9}

Але в календарі різдвяні дні припадали на 25-26 грудня, тобто відповідно до
григоріанського календаря, тоді як православна церква традиційно святкувала
Різдво за старим стилем, згідно юліанського календаря. Тому, коли підходили
календарні свята 25-26 грудня, робітники і службовці заявляли, що не бажають
відпочивати у ці дня і просять перенести вихідні на інші числа. 

\ii{30_12_2021.fb.fb_group.story_kiev_ua.1.rizdvo_derevo.pic.10}

Радянська
пропаганда втішалася таким наче успіхам на антирелігійному фронті: ось, мовляв,
як швидко трудящі позбавилися пережитків минулого. Проте, отримавши два
відгули, наші земляки відразу використовували їх 7-8 січня і відмічали Різдво
згідно до традицій. 

\ifcmt
tab_begin cols=3,no_fig,center

	ig https://scontent-frt3-1.xx.fbcdn.net/v/t39.30808-6/270382570_979105049345864_8024241186352231019_n.jpg?_nc_cat=108&ccb=1-5&_nc_sid=b9115d&_nc_ohc=C8tMFF2rDtAAX_LuhsC&tn=lCYVFeHcTIAFcAzi&_nc_ht=scontent-frt3-1.xx&oh=00_AT9ZbX0yiTBrPgiDKgBCSrmKZVCBNNJZPqILXJBjLKlKdA&oe=61D48479
	@caption_begin
		1 січня 1937 року в Москві в Колонному залі Будинку Спілок була встановлена 15-метрова ялинка і влаштоване свято, 
		на якому вперше гостей вітав дід Мороз.
	
		Час минав, і вже через рік мало хто думав про те, що дозволена ялинка тепер
		виступала у новій якості – не як різдвяна, а як новорічна, і що на її верхівці
		тепер горить не Віфліємська зірка, що колись привела волхвів до новонародженого
		Ісуса, а символ радянської влади – червона зірка...
	@caption_end

%Maksym Oleynikov
%Час минав, і вже через рік мало хто думав про те, що дозволена ялинка тепер виступала у новій якості – не як різдвяна, а як новорічна, і що на її верхівці тепер горить не Віфліємська зірка, що колись привела волхвів до новонародженого Ісуса, а символ радянської влади – червона зірка…

  ig https://scontent-frt3-2.xx.fbcdn.net/v/t39.30808-6/270216842_979104626012573_3698098551955291656_n.jpg?_nc_cat=101&ccb=1-5&_nc_sid=b9115d&_nc_ohc=e8sfY1xYp0IAX_ax23B&tn=lCYVFeHcTIAFcAzi&_nc_ht=scontent-frt3-2.xx&oh=00_AT-B_IVAAypncV-mqVCQ9BX64TUOW7JvPmgjdToN-nzHtQ&oe=61D35933
	@caption_begin
		Шарж Б.Єфімова на Павла Постишева. За радянською міфологією саме Постишев «повернув дітям ялинку».
		\footnote{
		В лютому 1938р. Постишев, переведений на той час з України на пост 1-го
		секретаря Куйбишевського обкому і міськкому партії, був арештований, а ще
		через рік – розстріляний, як «японський шпигун» і «правий троцькіст».
		Моя мама згадувала, як в школі вони співали пісню на слова М. Рильського: «...Чи
		бачите ви посмішки і сяєво очей?  Павло Петрович Постишев сидить серед дітей».
		Та потім їх змусили вирвати з підручників сторінку з портретом Постишева, а
		якось на перерві приїхала машина і співробітники НКВС забрали вчителя, який цю
		пісню розучував з учнями...
		}
	@caption_end

  ig https://scontent-frt3-1.xx.fbcdn.net/v/t39.30808-6/270355812_979104722679230_4217775424205662027_n.jpg?_nc_cat=106&ccb=1-5&_nc_sid=b9115d&_nc_ohc=p6p-KAJGSS8AX-GoNlr&_nc_ht=scontent-frt3-1.xx&oh=00_AT9HQxXYctPGENjABClcpplE2x8f0PL4-GOLZVM6tLuakg&oe=61D41E4A
	@caption Обкладинка журналу «Крокодил» № 1, 1936р. Ялинка вже «реабілітована», наче й не було тривалої заборони...

tab_end
\fi

%\begin{multicols}{2} % {
%\setlength{\parindent}{0pt}

%\ii{30_12_2021.fb.fb_group.story_kiev_ua.1.rizdvo_derevo.pic.11}
%\ii{30_12_2021.fb.fb_group.story_kiev_ua.1.rizdvo_derevo.pic.11.cmt}

%\ii{30_12_2021.fb.fb_group.story_kiev_ua.1.rizdvo_derevo.pic.12}
%%\ii{30_12_2021.fb.fb_group.story_kiev_ua.1.rizdvo_derevo.pic.12.cmt}

%%\ii{30_12_2021.fb.fb_group.story_kiev_ua.1.rizdvo_derevo.pic.13}
%%\ii{30_12_2021.fb.fb_group.story_kiev_ua.1.rizdvo_derevo.pic.14}
%\end{multicols} % }

Тому влада, роздратована небажанням людей позбутися «релігійних пережитків
минулого», починає новий, більш жорсткий етап боротьби з релігією. У результаті
продаж ялинок опиняється під забороною, з магазинів зникає і вся різдвяна
мішура. Відміняє вихідні на релігійні свята Постанова РНК СРСР 1929р.: «В день
нового года и дни всех религиозных праздников (бывших особых дней отдыха)
работа проводится на общих основаниях».

\ii{30_12_2021.fb.fb_group.story_kiev_ua.1.rizdvo_derevo.pic.15}

Із завершенням НЕПу і посиленням атеїстичного режиму церковні свята були
викреслені з радянського календаря. Загалом до середини 1930-х років у СРСР
склалася ситуація, наче Різдво з ялинкою та іншими витребеньками —
напівзаборонені, а Новий рік — взагалі не свято, а лише заміна відривного
календаря на новий, ще «не початий».

\ii{30_12_2021.fb.fb_group.story_kiev_ua.1.rizdvo_derevo.pic.16}

Пройшло кілька років, і ялинку несподівано повернули в радянське життя. 28
грудня 1935р. в «Правді» - самій головній газеті країни, в якій нічого не могло
зявитися просто так, була опублікована невелика стаття 2-го секретаря ЦК КП(б)
України Павла Постишева під назвою: «Давайте организуем к новому году детям
хорошую елку!».  

Щоправда у країні «войовничого атеїзму» встановлювати ялинку на честь Різдва
було неможливо, тому П.Постишев у своїй статті наголошував: ялинка має бути
новорічною і переважно для дітей. Вже наступного дня, 29 грудня, в усіх великих
містах СРСР оперативно відкрилися ялинкові базари, де торгували не лише
лісовими красунями, а й прикрасами.

Якщо ялинки можна оперативно зрубати, то продаж ще й іграшок засвідчував, що
все готувалося заздалегідь. В усі дитячі заклади полетли телеграми з приписами
негайно провести новорічні свята. Негайно «Комсомольськая правда»  оприлюднила
постанову ЦК ВЛКСМ, що давала вказівку комсомольським організаціям влаштовувати
новорічні ялинки «весело і без занудства». В останні дні 1935 року всі
радянські газети накрила передноворічна хвиля. Чиновники, підприємства,
установи, партійні, комсомольські та профспілкові організації одне перед одним
рапортували: скільки ялинок встановлено, скільки прикрас фабриками та артілями
випущено та як всі дружно готуються до нового радянського свята...

Газета «Правда» від 1 січня 1936 року над фотографією Сталіна розмістила
привітання: «С Новым годом, товарищи, с новыми победами под знаменем Ленина -
Сталина!». У тому ж році вперше з новорічним привітанням по радіо звернувся
голова ЦВК СРСР М.Калінін. (Сталін, до речі, ніколи не вітав народ з Новим
роком).

Так колишнє різдвяне дерево відродилося в новому статусі – радянської
новорічної ялинки, головної складової свята. Уже напередодні 1937 року цей день
відзначили з великим розмахом – в Кремлівському палаці прикрасили величезну
ялину і провели грандіозне свято.

Ялинка перетворилась на атрибут світського, можна навіть сказати державного
свята Нового року, яке разом з Першотравнем і Жовтнем увійшло до тріади
головних свят держави. Віфліємську зірку, яку за легендою запропонував ще
Мартін Лютер у далекому XVIст., замінили на п’ятикутну червону – таку ж, як
зірки на кремлівських баштах. 

Слова «Різдво» в новорічні дні намагалися всіляко уникати. Наприклад, коли у
1961р. екранізували хрестоматійний твір Гоголя «Ніч перед Різдвом», стрічку
назвали нейтрально: «Вечори на хуторі біля Диканьки». Навіть народні тексти
підлягали цензурі. Популярна колядка «Ой, радуйся, земле» залишалась в
репертуарі українських ансамблів, але від тексту її зосталася ледве половина, а
у приспіві замість «Син Божий народився» тривалий час звучало «Рік Новий
народився». 

Звісно, хто бажав – той тихенько відмічав Різдво, тим більше, що в школах ще
продовжувались зимові канікули, а на підприємствах і в установах перші дні
після Нового року все одно йшли на розкачку...

Упродовж тривалого часу що влада, що суспільство вважали Новий рік виключно
дитячим святом, до якого дорослі були не дуже причетними. Яке там новорічне
застілля, якщо завтра до роботи? Лише 23 грудня 1947 указом Президії Верховної
Ради СРСР 1 січня оголосили вихідним днем. Відтоді новорічне свято стало
найулюбленішим і у дорослих, оскільки було практично позбавлене політичного
забарвлення.

Власне, з повоєнних часів, коли підросло покоління тих, для кого  ялинка вже
була виключно «новорічною» й аж ніяк не «різдвяною», Новий рік стає загальним
та сімейним святом. Ця ситуація цілковито зберігається й сьогодні, коли феномен
Нового року як свята № 1 маємо виключно на теренах колишнього СРСР. Але і у нас
ялинка все частіше набуває свого первісного, різдвяного змісту, залишаючись при
цьому невід’ємним новорічним атрибутом.
