% vim: keymap=russian-jcukenwin
%%beginhead 
 
%%file 05_04_2021.fb.bizin_nikolaj.1.anna_tuv_donbass
%%parent 05_04_2021
 
%%url https://www.facebook.com/nik.bizin/posts/3662808260511970
 
%%author 
%%author_id 
%%author_url 
 
%%tags 
%%title 
 
%%endhead 

\subsection{03.04.19 Заметка от журналистов}
\label{sec:05_04_2021.fb.bizin_nikolaj.1.anna_tuv_donbass}
\Purl{https://www.facebook.com/nik.bizin/posts/3662808260511970}


\ifcmt
  pic https://scontent-bos3-1.xx.fbcdn.net/v/t1.6435-9/168428263_3662805167178946_5780686263473496151_n.jpg?_nc_cat=101&ccb=1-3&_nc_sid=730e14&_nc_ohc=fDSmdsGZs2sAX_L_L-3&_nc_ht=scontent-bos3-1.xx&oh=6366369c9291114f6f5a6c64df76a5b1&oe=60983225
\fi


"Анна Тув. "Исповедь террористки." Исповедь Анны Тув из Донбасса: убили семью, выдвинута на Нобелевку. В Горловке во время обстрела она лишилась мужа и дочки и нашла уничтоживших их украинских военных. Украина отдала бы многое за то, чтобы тогда, в мае 2015-го, эта женщина осталась лежать под обломками своего дома в Горловке вместе с мужем и старшей дочерью. Чтобы ее не было вообще.
Включенная в украинскую базу «Миротворец» как террористка и сепаратистка, 35-летняя Анна Тув тем не менее ездит по всей Европе с антивоенными выступлениями. Там о ней вышли десятки статей и репортажей.
После речи Анны в женевском Дворце Наций на сессии Совета по правам человека в ООН в 2017 году зал молчал.
Как ни странно, но только мы, в России, почти ничего не знаем об Анне Тув — кандидате на Нобелевскую премию мира 2019 года. На эту награду ее тоже выдвинули европейцы.
Она — боль Донбасса. Единственный его голос, который услышали в мире.
Выдвижение простой жительницы Горловки должно помочь в освещении в СМИ конфликта на Востоке Украины, заявил в интервью господин Эннио Бордато, председатель итальянской благотворительной организации «Помогите спасти детей».
Кандидатуру Анны, как это положено по правилам Нобелевки, предложил член итальянского национального парламента Вито Коменчини. Дело Анны Тув по обвинению украинских военных, убивших ее семью, стало одним из самых громких в Страсбурге.
Одной уцелевшей рукой Анна Тув откопала новорожденную Милану.
«Каждое утро мы просыпались с мыслью о том, что вчера был самый страшный день», — начинает Анна Тув свою историю.
Анна — невысокая, худенькая, вместо руки — бионический протез. У нее негромкий голос. И на первый взгляд в ней нет ничего выдающегося — не Орлеанская дева с горящим взором.
Обычная жена и мать, обычная женщина. Чаще всего судьба и выбирает самых обычных.
Десятки, сотни раз ей приходилось выступать перед людьми, которые не знают русского, рассказывать о том, о чем многие предпочли бы вовсе не знать.
То, что она говорит сегодня мне — про оранжевых пищащих цыплят в клетке, которых пыталась спасти за мгновение до того, как снаряды обрушились на ее дом, как осела взвесь пыли от бомбежки на ее обескровленное лицо, как с рукой, висящей на одной коже, с разорванными барабанными перепонками откопала в земле свою новорожденную дочку, которой было всего несколько недель от роду, держа второй, целой рукой вырывающегося, потерявшего от ужаса рассудок двухлетнего сына, как увидела, что от мужа и старшей девочки осталась ровно половина, все это она повторяла перед другими бесчисленное число раз...
«Каждое утро мы были уверены в том, что вчера был самый страшный день», — говорит она почти без выражения. Почти не смотрит на меня. Вглядываясь в кого-то, невидимых и бесплотных, стоящих сзади.
«Я — Анна Тув, жительница Донбасса»… Свою речь она выучила наизусть. Произнесенная десятки, сотни раз, та отскакивает от зубов. Как будто это пересказ газетной статьи или страшной книжки. Иначе бы, я думаю, Анна просто сошла с ума.
«Муж не пускал меня на митинги»
«Я родилась и выросла в украинской Горловке. Мои родители шахтеры, их еще маленькими привезли на Донбасс из Курской области. По образованию я медик и большую часть своей жизни проработала медсестрой в хирургическом отделении городской больницы.
С мужем мы жили хорошо. Держали большое хозяйство, свиней, кур, огород, имели небольшой бизнес — муж занимался автомобилями, я параллельно основной работе распространяла косметику.
Мы были абсолютно аполитичны ко всему происходящему на Украине, не всегда было время даже телевизор посмотреть. Мы не ходили на митинги и демонстрации, ни за что не боролись. То, что происходило в 2014 году в Киеве, мы восприняли как очередную «оранжевую революцию» — помитингуют и прекратят. Нашей собственной жизни это никак не касалось».
На референдум по самоопределению Донецкой республики муж Анну не пустил тоже — боялся провокаций.
«Я была то беременна, то кормящая, и Юра не разрешал мне включать новости, чтобы не перегорело молоко».
Донецк, Славянск — это все было как будто в другой реальности. Да, ее семья тоже видела летящие в небе военные самолеты. Но они не думали, что те несут смерть.
«А потом бомбы начали падать на центр нашего города. Мы испугались, побросали сонных детей в машину и поехали в Крым. Мы были уверены, что через неделю все успокоится — и мы сможем вернуться: муж считал, что нельзя бросать хозяйство надолго без присмотра».
Июль 2014-го. Разрушенная Горловка. Убитая малышка на руках мертвой матери, пытавшейся в последнем объятии спасти своего ребенка, — фотография, облетевшая весь мир. «Горловская Мадонна» называется этот снимок. Убийцами этой молоденькой и ни в чем не виноватой девушки и ее дочери называли то Украину, то Россию, то ополченцев. В зависимости от конъюнктуры и политических интересов...
Погибшую девушку звали Кристина Жук, и ей было 23 года. Ее дочке Кире было 11 месяцев. Фотограф, случайный прохожий, запечатлевший ад, вспоминал, как в его ушах стоял нечеловеческий крик бабушки в старом поношенном платье, растерянно и нелепо державшей в руках кусок оконной рамы, смотревшей на дыру в здании, на том месте, где стоял ее дом.
«Нас считали сепаратистами только по факту рождения и проживания на Донбассе, авиабомбы прилетали с той стороны, в то лето Украина выжигала нас целыми кварталами, кто мог, тот уезжал, на восьми соседних улицах остались всего семей пять, — продолжает Анна Тув свой рассказ. — Я не хотела быть беженкой в Крыму, я хотела обратно домой. Мы же не бомжи, не бездомные, не бродяги. Через неделю Юра все-таки уехал, а мы с детьми остались».
Она работала барменшей в прибрежной кафешке. Денег не было, оформить пособия по беременности и родам можно было только на территории Украины, но туда было нельзя попасть, каждые три часа 10-летняя Катя приносила маме на кормление маленького Захара.
«Сезонная работа заканчивалась. Муж много раз приезжал и уговаривал меня вернуться в Горловку. Если бы я отказалась, наша семья разрушилась бы. Дети плакали и просились к отцу. Я подчинилась, но это было наше общее решение».
Они вернулись к осени 2014 года. Так как были уверены, что только что подписанный сторонами конфликта первый Минский протокол, предусматривающий прекращение огня на территории Донецка и Луганска, принесет конец войне.
Другие люди, обнадеженные переговорами, возвращались тоже. Анна забеременела третьим ребенком. А потом опять начался ад.
«Бомбили каждый лень. Об этом нигде не говорили, потому что официально-то был «Минск». Подвала у нас не было, мы прятались у друзей на соседней улице, ночью бежали по снегу, сидели в холодном погребе. Той зимой дети переболели воспалением легких... Мы были истощены. Мы потеряли бдительность. Мы устали бояться. Со временем привыкаешь ко всему. Каждый день мы просыпались с мыслью, что вчера был последний обстрел. Нас держала только эта вера».
На учет по беременности Анна встала за день до родов. Находиться в больнице было тоже небезопасно. Она родила Милану и через час под очередным обстрелом уехала домой — палили прицельно по гражданским объектам.
Старшую дочь отправляли на уроки с запиской в рюкзаке, как нужно вести себя, если начнут стрелять по школе. Анна с нетерпением ждала наступления лета, чтобы не отпускать Катю от себя.
21 мая Кате исполнилось 11 лет. Через пять дней ее не стало.
«Мы отпраздновали день рождения. Пришли соседские дети. А вечером опять началась канонада, — продолжает Анна. — Я приняла решение, что Катя закончит четвертый класс, и мы уедем хоть куда-нибудь... Нам не хватило всего дня до начала ее каникул».
Помнит ли она тот день?
Катя пришла из школы радостная, в предвкушении последнего звонка. Привезли сено, его нужно было ворошить для просушки, поливали огород, низко летал украинский разведчик-беспилотник. Все было как обычно.
«Мы не успели сориентироваться, как снова начали стрелять. Я быстро отнесла с улицы в сарай клетки с цыплятами, дочка выбежала во двор мне помогать. Я попросила Юру сделать телевизор погромче, чтобы только не слышать шума падающих снарядов.
Анна говорит, что раздался громкий свист — и она побежала к дому. Она едва успела заскочить за порог и захлопнуть за собой дверь. И... ничего не стало.
Бомба попала между коридором и детской. Молодую женщину выбросило взрывной волной наружу. Она потеряла сознание. «Очнулась от сильного визга в ушах и от запаха газа. Вся в пыли и в извести. Я не могла открыть глаза. Прямо мне в лицо была направлена перебитая газовая труба, она шипела, я не могла отвернуться и пошевелиться».
Сосед помог высвободиться из-под остатков двери, которая упала на нее. Анна слышала, как где-то рядом задыхается Захар, засыпанный землей. Она кричала и звала Юру и Катю. Но ответа не было.
«Рядом у холодильника разорвался еще один снаряд. Я не помню, как мне удалось освободить Захара. Он очень сильно кричал и все показывал на мою левую руку. Я не понимала, что это с ним, мне совершенно не было больно, смотрю — то, что осталось от руки, висит на одном лоскуте кожи».
Мобильные телефоны утопило в аквариуме. Его стекло лопнуло, и все вокруг было в воде. В результате второго взрыва Анна оказалась заблокирована в доме вместе с двумя детьми. Перевязала руку колготками Захара, откопала Милану. «Я не могла оставить сына и бежать искать Катю и Юру, потому что Захар не отпускал меня ни на секунду, босыми ногами в крови он бегал по осколкам, и мне пришлось прижать его к себе сильно-сильно, чтобы он затих. Я понимала — это все, что я могу сделать в данный момент».
Она услышала, как кто-то из мужчин-спасателей, прорываясь к ним, громко закричал: «Ребенок, ребенок!» Анна была уверена, что это нашли Катю и она без сознания. Но спасатели были белые, как мел.
«И я увидела, как от моей Кати осталась половина туловища. Рядом лежал Юра, лицом на шлакоблоках. Без рук, без ног. Он был разорван на части на пороге своего дома».
После гибели мужа и дочери Анна не хотела жить.
Долгая дорога домой
В этот день не стало не только семьи Анны Тув. 26 мая 2015 года от обстрела ВСУ погибли еще пять мирных жителей Горловки. Всю ночь в городе не прекращали гудеть «скорые», развозя раненых.
Анна отказывалась ехать в больницу без младших детей. Но их поместили в разные корпуса. Придя в себя после ампутации, она умоляла спустить ее вниз — к малышам.
«Я находилась на седьмом этаже и понимала, что если сюда сейчас влетит фугас, то ничем не смогу помочь своим детям. Ничем. Я работала в этом здании, я знала каждый его закоулок, я одна могла их спасти».
Милана была изрезана картечью, у Захара западал язык, и он весь был как ежик в железных осколках, побивших плечи и спину. Мальчик разучился говорить. Он был не в себе. Потом ему даже поставят диагноз «аутизм» — как следствие шока.
Анна не хотела жить. Уйти из жизни ей не дал английский репортер Грэм Филлипс — он все время находился рядом, в ее палате, писал хронику войны.
«После похорон гуманитарный батальон «Ангел» вывез меня с детьми в Донецк. Поселили в квартире моряка, уехавшего на работу в Европу. Хозяин жилья сказал так: «Если придет человек, попавший в трудную жизненную ситуацию, дайте ему мои ключи».
Это стало для нее с детьми временным убежищем. Безопасным местом на много месяцев. Их возили по разным госпиталям, пытаясь реабилитировать психику Захара.
Первый протез Анне поставили в Санкт-Петербурге. Рабочий тяговый протез. Не получилось сразу сделать бионический, так как еще не сформировалась культя.
Год ее не могли вывезти за границу, на чем настаивали европейцы, готовые оплатить операцию. Но заграничный паспорт можно было оформить только в Киеве, а там Анна считалась сепаратисткой, ее даже внесли в базу «Миротворец», утверждая, что она не пострадала от бомбежки, не потеряла семью, а была медсестрой ополчения ДНР.
Да, за это время она узнала, кто убил ее мужа и Катю. Не какой-то гипотетический Киев, а конкретные имена и звания украинских офицеров. Расследование этого преступления вел Следственный комитет РФ и правоохранительные органы ДНР.
По их данным, в тот день обстрелом по Горловке командовал подполковник Виктор Юшко, командир 1-го гаубичного артиллерийского дивизиона 44-й отдельной артиллерийской бригады вооруженных сил Украины (Тернополь). Это структурное подразделение входило в состав 44-й отдельной артиллерийской бригады, командиром которой является полковник Олег Лисовой. Именно он и отдал приказ стрелять по мирным жителям. Юшко его выполнил.
Наверняка у них тоже есть где-то семьи... Они целуют своих детей и жен, возвращаясь из очередной командировки на Донбасс. Живые люди, не бездушные роботы, не автоматы.
Как могут они простить себя? Чем оправдать?
«Они прекрасно знали, что среди нас нет ополченцев. Обстрел производился после того, как наш район облетел дрон-разведчик, то есть им было известно, что здесь расположены обычные жилые дома».
Общественники из Донецка направили в Страсбург материалы расследования и свидетельские показания Анны на видео вместе с юристом Дамьеном Вегье. Европейский суд по правам человека принял их, дело Анны Тув в ЕСПЧ № 56288/15, но таких исков и обращений — десятки, сотни... И это не в русле сегодняшней политической повестки дня — обвинять правительство Украины.
«У моей подруги в Горловке погибло сразу трое детей, так что мне еще повезло», — горько усмехается Анна.
Ради памяти своих близких, все, что она могла отныне, — это говорить.
Свидетель обвинения
Италия, Сан-Марино, Германия, Франция, Швейцария... Мир — тоже расколотый надвое. Где, с одной стороны, — правозащитники и журналисты, осветившие ее историю. А с другой — сытые бюргеры, функционеры, которые точно «ведали», как оно было на самом деле и кто «виноват» в этой войне, изменить их отношение к происходящему на Донбассе до сих пор невероятно тяжело.
Если бы Анна была настоящим политиком, наверное, эти люди даже не стали бы ее слушать, какие бы убедительные доводы она ни приводила.
Но в том-то и дело, что она была обычной женщиной. Вырванной страшным взрывом из своей обычной жизни, где огород, цыплята, скорые каникулы старшей дочери, кормление по часам младшей...
«Последняя моя речь в ООН в Женеве, там были спикеры из разных стран, в том числе и из Украины. Ту сторону представила женщина, которая сама уже 10 лет живет в США. Она убеждала присутствующих, что на Донбассе стоит российская армия и что украинские войска обстреливают нежилые дома и объекты сепаратистов, а не детей. Но у меня есть снимки, доказательства, свидетельства гибели моей семьи и моих знакомых, данные баллистической экспертизы, с какой стороны прилетели снаряды. Этой женщине было нечего мне возразить, ведь у нее все хорошо».
Впервые зал заседания ООН молчал. После выступления Анне не задали ни одного вопроса.

Французы делали интервью, рассказывая трагическую историю Анны Тув в своих СМИ,
итальянцы устраивали уличные протесты, требуя отдать под трибунал Петра
Порошенко и впервые — национальные украинские батальоны, признав их
террористическими, немецкие парламентарии обещали донести страшную правду до
своих избирателей...

Маленькая хрупкая женщина — не громогласные российские политики — прорвала
информационную блокаду в Европе правдой о своей истекающей кровью родине.

После выступления Анны в Вероне Петр Порошенко был лишен звания почетного итальянского гражданина.

«Мне пишут через соцсети тысячи людей, моих земляков, о том, что происходит
сейчас. Ничего не закончилось еще. Мира нет. Конечно, многим на Западе и в
Киеве это не нравится. Мне постоянно поступают угрозы. Сначала было очень
трудно выезжать за границу, так как я являлась гражданкой Украины и меня
старались не выпускать. Только в конце 2018 года, наконец, удалось получить
гражданство России как носительнице русского языка. Для этого пришлось
доказывать, что мои родители из Курской области, что я имею на это полное
право. У меня русская де...

