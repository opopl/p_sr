%%beginhead 
 
%%file 19_08_2023.fb.kplavra.1.kyivska_soroka_v_sadah_kyiv_pechersk_lavra
%%parent 19_08_2023
 
%%url https://www.facebook.com/kplavra/posts/pfbid02GbTwRjVnMAhRxA7WE56wRGCsnGHtSoUqVU51p4jxC7vnHaEpL7XAUHmigjaJEcNNl
 
%%author_id kplavra
%%date 19_08_2023
 
%%tags 
%%title Київська "Сорока" в садах Києво-Печерської лаври
 
%%endhead 

\subsection{Київська \enquote{Сорока} в садах Києво-Печерської лаври}
\label{sec:19_08_2023.fb.kplavra.1.kyivska_soroka_v_sadah_kyiv_pechersk_lavra}

\Purl{https://www.facebook.com/kplavra/posts/pfbid02GbTwRjVnMAhRxA7WE56wRGCsnGHtSoUqVU51p4jxC7vnHaEpL7XAUHmigjaJEcNNl}
\ifcmt
 author_begin
   author_id kplavra
 author_end
\fi

🍏Київська \enquote{Сорока} в садах Києво-Печерської лаври 

🍎На свято Преображення, Яблучний Спас, в лаврських садах дозрівали багаті та
різноманітні плоди. Серед них – давній сорт яблук під назвою \enquote{Сорока}.

🍏Як зазначав відомий український вчений і садівник Левко Платонович Симиренко,
цей сорт майже ніде не траплявся окрім Києва і переважно Києво-Печерської
лаври. Саме з лаврських садів Симиренко 1910 р. взяв посадковий матеріал Сороки
для свого унікального колекційного саду в Млієві на Черкащині. 

🍎Яблуні цього сорту могли з'явитися у Печерському монастирі ще за часів
архімандрита Петра Могили, \enquote{воєводича молдовського}, і бути названими
за молдавським містечком Сороки на Дністрі.  

🍏Плід Сороки має схожу на яйце форму, що різко звужується до носика. Рум'яне
строкате яблучко густо покрито червоними смужками, ніби пір'ям. Духмяні, ніжні,
кисло-солодкі на смак яблука Сороки користувалися великим попитом у Києві.
Більша частина врожаю йшла на виготовлення знаменитого київського варення –
зацукрованих фруктів, що розходилися далеко за межами Києва і України.  

🍎З плином часу Сорока, як і багато інших стародавніх сортів, зникла з
київських садів. Рідкісний екземпляр яблуні Сорока росте в Ботанічному саду
імені академіка Фоміна, на ділянці формово-декоративного саду. Куратор плодових
рослин Олександр Чекалін визначив його за описом у фундаментальній праці Л.
Симиренка \enquote{Помологія}.

Побажайте рідним і близьким на Яблучний Спас Божої благодаті та миру і
обов'язково пригостіть яблуками!

✍️Матеріал підготовлено Олександром Чекаліним, співробітником Ботанічного саду
ім. Фоміна, який досліджує тему Митрополичого саду у Національному заповіднику
Києво-Печерська лавра
