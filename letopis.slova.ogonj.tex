% vim: keymap=russian-jcukenwin
%%beginhead 
 
%%file slova.ogonj
%%parent slova
 
%%url 
 
%%author 
%%author_id 
%%author_url 
 
%%tags 
%%title 
 
%%endhead 
\chapter{Огонь}

%%%cit
%%%cit_head
%%%cit_pic
\ifcmt
  pic https://strana.ua/img/forall/u/0/34/_119058151_crimea_russia_russian_640_v2-nc.png
  caption Маршрут британского \enquote{Дефендера}
\fi
%%%cit_text
С чем и связана некоторая путаница в заявлениях Лондона. Там явно хотят
сохранить лицо. Поэтому и следуют угрожающие обещания и дальше \enquote{ходить
на Крым}.  Впрочем, далеко не факт, что попытка номер два последует в скором
времени.  В Украине после керченского инцидента тоже обещали снова отправить
корабли в пролив, но так этого и не сделали. Поскольку если дальше повышать
ставки, ситуация может выйти из-под контроля, и Россия откроет \emph{огонь на
поражение}. Чему вряд ли обрадуется блок НАТО, который никакой войны с Москвой
в ближайшей перспективе не планирует
%%%cit_comment
%%%cit_title
\citTitle{Эсминец Дефендер у берегов Крыма  - как менялась риторика Британии}, 
Оксана Малахова, strana.ua, 26.06.2021
%%%endcit

