% vim: keymap=russian-jcukenwin
%%beginhead 
 
%%file 13_12_2021.fb.fb_group.story_kiev_ua.2.raf_marshrutka.cmt
%%parent 13_12_2021.fb.fb_group.story_kiev_ua.2.raf_marshrutka
 
%%url 
 
%%author_id 
%%date 
 
%%tags 
%%title 
 
%%endhead 
\zzSecCmt

\begin{itemize} % {
\iusr{Татьяна Соловьева}

Какой прекрасный рассказ-воспоминание, спасибо большое @igg{fbicon.heart.red}!

\iusr{Muravov Dmitriy}
Супер, не знал, что маршрутки были в 70-е. Всегда думал, что это изобретение 90-х

\begin{itemize} % {
\iusr{Ксения Погорлецкая}
\textbf{Muravov Dmitriy} Маршрутки были и во второй половине 60-х (раньше не помню) и стоимость проезда была 10 коп., это уже в середине-конце 70-х она стала 15 коп.

\iusr{Татьяна Петрюк}
\textbf{Ксения Погорлецкая} да когда цена на бензин повысилась и повысилась цена в такси за км .

\iusr{Жанетта Половникова}
\textbf{Ксения Погорлецкая} да, к нам на Шелковичную (Карла Либкнехта) можно было только маршруткой добраться. А потом, - никак

\iusr{Ксения Погорлецкая}
\textbf{Жанетта Половникова} А потом мы ножками топали либо по лестнице Крутого Спуска, либо по Клугло и дворами срезали углы. С тяжелыми сумками то еще было удовольствие.. Наверное, мы жили рядом.

\iusr{Жанетта Половникова}
\textbf{Ксения Погорлецкая}, да соседи. И кому тогда помешали маршрутки? А сейчас там, вообще жуть. Мы переехали за город и я не могу морально бывать на своей улице. Сердце болит, во что её превратили.

\iusr{Ксения Погорлецкая}
\textbf{Жанетта Половникова} Переехала с Лютеранской на которой прожила полжизни в начале 90. Все изменилось... Больно на этом смотреть, чужое стало.
\end{itemize} % }

\iusr{Алекс Вельт}

Такая маршрутка ходила у нас от Центрального Автовокзала по ул.Казацьей до
Платочной фабрики. Именно шо 15 копеек. Народу набивалось полно. Это было в
1973-1978 годах. Прекратила она ходить в 1987 году по моему и пустили автобус.
но маршрут изменили.

\iusr{Елена Сидоренко}

Помним, спасибо Вам, Петя! День рождения дочери - выходной день в Украине. @igg{fbicon.heart.beating} 

\begin{itemize} % {
\iusr{Петр Кузьменко}
\textbf{Елена Сидоренко} 

да. Это всегда приятно. Катя шутит, что её Днюшка государственный праздник.
@igg{fbicon.laugh.rolling.floor} 

\iusr{Елена Сидоренко}
\textbf{Петр Кузьменко} 

а мой сын родился 30 июня, и когда депутаты долго \enquote{принимали} день Конституции,
я хотела, чтобы не спешили, совещались, и у ребёнка день рождения был бы всегда
выходной. Но.. они после бессонной ночи 28 июня приняли основной закон...

\end{itemize} % }

\iusr{Светлана Комина}

В 80-х проезд был 20 копеек в маршрутке. А в автобусе и метро-5
копеек. Называлось-маршрутное такси.

\begin{itemize} % {
\iusr{Старр Диггер}
\textbf{Светлана Комина} 15 коп

\iusr{Ксения Погорлецкая}
\textbf{Светлана Комина} 10. а потом уже 15 копеек.
\end{itemize} % }

\iusr{Виктория Рыженко}

Я тоже должна была родиться 1 июля, а родилась 28 июня, только раньше на 13
лет. На душе тепло от такого совпадения.


\iusr{Ирина Слюсаренко}

Такая же маршрутка ходила от Бессарабского рынка вверх по Круглоуниверситетской
улице. И стоила она 10 копеек.

\begin{itemize} % {
\iusr{Юляшка Белкова}
\textbf{Ирина Слюсаренко} точно! Ехала через Карла Либкнехта, Богомольца, Чекистов на Розы Люксембург и обратно на Бессарабку

\begin{itemize} % {
\iusr{Ксения Погорлецкая}
\textbf{Юляшка Белкова} Да, так и было. Проезд стоил 10, потом уже 15 копеек. На конечной на Р.Люксембург была также конечная маршрутки кторая ездила на площадь Б. Хмельницкого. Как было удобно!

\iusr{Александр Казимир}
\textbf{Ксения Погорлецкая} Подол - Бесарабка було по десять коп. 60- годи. 14
\end{itemize} % }

\iusr{Светлана Комина}
Может цена была от дальности маршрута? 10, 15, 20 копеек.

\begin{itemize} % {
\iusr{Ксения Погорлецкая}
\textbf{Светлана Комина} Цены никогда ( в то время) не зависели от дальности маршрута. Обычное подорожание маршруток тогда же когда и такси подорожало.
\end{itemize} % }

\end{itemize} % }

\iusr{Анна Сидоренко}
Как поётся в песне - не забывается такое никогда.... Спасибо за рассказ.

\iusr{Татьяна Ховрич}
Чудесный рассказ! Спасибо!

\iusr{Светлана Недайбида-Бучко}
Прекрасный рассказ! И маршруты такие знакомые! Спасибо за воспоминания!@igg{fbicon.heart.red}

\iusr{Раиса Карчевская}
Прекрасные воспоминания

\iusr{Al Nik}
 @igg{fbicon.thumb.up.yellow}{repeat=3} 

\iusr{Ирина Горлова}

\ifcmt
  ig https://i2.paste.pics/f3de7d341baaefbd8b4a034a463b8c58.png
  @width 0.2
\fi

\iusr{Нелли Кузьменко}
 @igg{fbicon.oncoming.automobile} 

\iusr{Нелли Кузьменко}

\ifcmt
  ig https://scontent-frt3-2.xx.fbcdn.net/v/t39.1997-6/s180x540/264431318_4384014421697465_4187384674482235271_n.png?_nc_cat=101&ccb=1-5&_nc_sid=ac3552&_nc_ohc=oImTDOJpT3cAX-7-XMb&_nc_ht=scontent-frt3-2.xx&oh=00_AT89OQuxj80UnGtS0cKLo3AidN3QHgIK1cYBo-G4KUPa3g&oe=61C0B49D
  @width 0.1
\fi

\iusr{Виктория Дергаль}
Завжди цікаво вас читати! Дякую за історію  @igg{fbicon.face.smiling.eyes.smiling} 

\iusr{Semyon Belenkiy}
И как всегда, спасибо за память!

\ifcmt
  ig https://i2.paste.pics/f2e9797a30eb988093db31f5c69fcc26.png
  @width 0.2
\fi

\iusr{Tetiana Desiatka}
Дякую Вам.

\ifcmt
  ig https://scontent-frx5-2.xx.fbcdn.net/v/t39.1997-6/p480x480/105941685_953860581742966_1572841152382279834_n.png?_nc_cat=1&ccb=1-5&_nc_sid=0572db&_nc_ohc=Kq8EgqfccboAX8GywK8&_nc_ht=scontent-frx5-2.xx&oh=00_AT8Yg5oaecqkbnAWymVzdFe4OjX4-Gve_6nXaUl68BqHaA&oe=61C1E08B
  @width 0.2
\fi

\iusr{Alexandr Bouuyhuk}
Часто пользовался с Бессарабки наверх на Липки)

\iusr{Gary Sorokin}
Спс  @igg{fbicon.hands.pray} 

\iusr{Владимир Гонтар}
Приветствие от КВОКДКУ выпуск 1976

\iusr{Елена Мельникова}
Ностальгия...

\iusr{Ярослав Игнатиенко}
такая же ездила от дв. Украина до Леси, по Патриса)

\iusr{Ereena Shvarts}

Спасибо, приятная и добрая история... я хорошо помню эти миниатюрные автобусики...
31 Декабря 1970 года мы переехали с подола на Лесной массив, ( как говорила моя
мама, коренная подолянка, которая родилась и выросла на Жданова, \enquote{на выселки} в
новую, долгожданную квартиру на улицу академика Курчатова, дом 11... ) от станции
метро \enquote{комсомольская} до нашего дома можно было добираться только на
маршрутном такси, так называемой, маршрутке. Автобусы тогда ещё не ходили.

\iusr{Арт Юрковская}

А что это за куколки на заднем плане?

\begin{itemize} % {
\iusr{Петр Кузьменко}
\textbf{Арт Юрковская} я написал. У жены коллекция кукол в народных и национальных костюмах. Многие она делала сама. Их уже намного больше чем моих моделей, выселяют! @igg{fbicon.face.wink.tongue} 

\iusr{Ксения Погорлецкая}
\textbf{Петр Кузьменко} У нас с Нелей -оказывается - одно увлечение: куклы в национальных костюмах.

\iusr{Елена Чичик}

Арт, это куклы из патрворков - были до 2013 года целые серии издательства
ДеАгостини: журнал плюс игрушка, на разные темы. Куклы такие были в выпуске
\enquote{Дамы Эпохи} - по произведениям писателей, там были Скарлетт, Анна Каренина,
Джейн Эйр и так далее. Потом на их же основе были куклы в национальных
костюмах. Всего было выпусков по 50-60 в каждой теме. Куклы ростом в 17 см,
кажется. Кроме этого, был интересный выпуск Викторианский кукольный домик в
масштабе 1/12 с полным комплектом мебели ко всем шести комнатам. Это
европейский проект, после триумфа в Европе его запустили в странах бывшего
СССР. Но если в европейских выпусках вся мебель была деревянная, то в наших -
часть шла пластиковая, что очень удешевляло внешний вид. Для мужчин и мальчиков
выпускались очень подробные серии \enquote{собери фрегат}, там было более 130 журналов
со вложениями и шикарный полуметровый корабль после сборки. Были и коллекции
машин, и часов, и минералов и даже пробников духов. Хорошие фигурки солдат
времен Наполеона для раскрашивания, насекомые, залитые к оргалитовые брелки и
многое другое. Неужели не видели в киосках Союзпечати?

\ifcmt
  ig https://scontent-frx5-2.xx.fbcdn.net/v/t39.30808-6/265990856_2965547070442403_2449348268668719369_n.jpg?_nc_cat=109&ccb=1-5&_nc_sid=dbeb18&_nc_ohc=Vqhs1jDtb9sAX_Y363s&_nc_ht=scontent-frx5-2.xx&oh=00_AT-BIE_Lw3V3WkFfDE1Lqw1PX188fn_0nMo1o4w7Fz9zuw&oe=61C12AD5
  @width 0.4
\fi

\end{itemize} % }

\iusr{Ярослав Игнатиенко}

\obeycr
Это мы придумали Windows
Это мы обьявили дефолт
Нам играют живые Битлз
И стареющий Эдриан Пол
Наши матери в шлемах и латах
Бьются в кровь о железную старость
Наши дети ругаются матом
Нас самих почти не осталось
\restorecr

\iusr{Богдан Головченко}
\textbf{Iar Ig} разве не \enquote{нестареющий}?!

\iusr{Алла Поповская}
Вот, вот спасибо вам недавно кто задавал вопрос о маршрутках от Бессарабки, я
помню маленькие автобусики марку конечно не знала @igg{fbicon.face.happy.two.hands}  @igg{fbicon.hands.applause.yellow} 

\iusr{Marsha Messinger}
\textbf{Алла Поповская} да, 10копеек

\iusr{Любовь Муравьева}
Помню маршрутки. Если стоимость проезда была 3- 5 копеек. Стоимость маршрутки - 10копеек. @igg{fbicon.grin} 

\begin{itemize} % {
\iusr{Ольга Агафонова}
\textbf{Любовь Муравьева} нет, 15

\iusr{Ксения Погорлецкая}
\textbf{Ольга Агафонова} Сначала стоимость была 10 копеек потом уже 15.

\iusr{Любовь Муравьева}
Это смотря какой маршрут

\iusr{Наташа Гресько}
\textbf{Любовь Муравьева} 15, кажется.

\iusr{Любовь Муравьева}
\textbf{Наташа Гресько}  @igg{fbicon.face.hand.over.mouth} 
\end{itemize} % }

\iusr{Виктория Зайцева}

Я помню, что в советское время эти модели были очень дорогие. По крайней мере,
те которые я видела. Мы купили моему двоюродному брату в подарок модель или
\enquote{Чайки} или \enquote{Москвича}, уже не помню. У неё все двери открывались, внутри
сидения были, руль. Багажник открывался и капот. Мечта любого мальчишки.


\iusr{Люда Невзгляд}
Помню такой автобус ходил с Бессарабки на Печерск, за 15 коп. В 60 ые годы Очень удобно

\iusr{Marsha Messinger}
\textbf{Люда Невзгляд} по 10

\iusr{Люда Невзгляд}

\ifcmt
  ig https://i2.paste.pics/dbc418816e913aac3697c2ea99a4e1cb.png
  @width 0.2
\fi

\iusr{Андрей Больбот}
Класс!!!! Был маршрут ул. Ивана Кудри - Тверской тупик - Владимирский рынок. 15 коп.

\iusr{Андрей Больбот}
Коллекция чудесна!!!!

\iusr{Алла Вайнерман}
Был маршрут и от Воздухофлоского моста до Печерского. Тоже, по-моему, 15 коп.

\iusr{Богдан Головченко}

Помню, ездили с родителями в середине 80-х на маршрутном такси РАФике от
Бессарабки по Круглоуниверситетской до ул. Карла Либкнехта, где через забор от
77 школы был папин дом. Шибко дорого было, 15 копеек, вместо обычных 5, но
иногда мама с папой позволяли себе шиканкуть!)

\begin{itemize} % {
\iusr{Ксения Погорлецкая}
\textbf{Богдан Головченко} Через забор (справа) от 77 школы. был т.н. ЦКовский дом, построенный в начале 70-х а слева, дом в котором на первом этаже была сберкасса.

\iusr{Богдан Головченко}
К.Либкнехта 21. По рассказам моих старших, дом был построен в конце 40-х пленными немцами.

\iusr{Ксения Погорлецкая}
\textbf{Богдан Головченко} Это большой темно-серый дом и с тыльной стороны есть забор отделяющий от школьной территории. Ваши родные правы, дом был восстановлен пленными но, скорее всего, румынами или венграми. Они восстанавливали дома рядом на Лютеранской и вечерами играли на скрипках (мама рассказывали). Может быть жили там же в уже восстановленных частях зданий (как не додумалась спросить когда было у кого; ведт пленных приводили и уводили с работы вечером).

\iusr{Сергей Оборин}

Работал в ИСМ (ин-т св. твердых материалов на Куреневке). У нач. отдела была
как бы секретарша негласная. Так она мне рассказала историю, приключившуюся с
ней где-то в 70-х. Начальник отдела послал ее в местную командировку. Иди,
говорит, там на проходной тебя РАФик ждет. Она, не долго заморачиваясь, выходит
на проходную, видит, человек стоит. Она его спрашивает, - Вы Рафик?. Он
отвечает, - я - Витя, а РАФик, вон там стоит.


\iusr{Greg Trosman}
\textbf{Сергей Оборин} а причём здесь ИСМ? @igg{fbicon.face.wink.tongue} 

\end{itemize} % }

\iusr{Рона Ильинична}
Спасибо за захватывающую
публикацию, за удачные фотографии.

\iusr{Рона Ильинична}

\ifcmt
  ig https://i2.paste.pics/6c878a88196ecef1fc64a4ab078000aa.png
  @width 0.2
\fi


\iusr{Оксана Яновская}
Улыбаюсь, теплые воспоминания! Спасибо!

\iusr{Тамара Мельничук}
Це була практично і моя історія тільки чуть раніше теж їздили на цій маршрутці до Бессарабки

\iusr{Микола Веселий}

Да, в таких прозрачных упаковках красивых были модельки. Раф точно. Двери
открывались и багажник. Чайка, Жигули, Москвичи, Камаз. Остальные всех не
упомнишь. Даже наложенным платежом мне присылали. Масштаб 1:43


\iusr{Татьяна Барабаш}
Петр, очень люблю читать ваши воспоминания! Спасибо!

\iusr{Валентина Валентина}

\ifcmt
  ig https://i2.paste.pics/cca4f15cad9ebe8128bed7f4d5172b3a.png
  @width 0.2
\fi

\iusr{Нина Яценко-Кончаковская}
Вы человек с большой силой воли  @igg{fbicon.face.smiling.eyes.smiling}{repeat=2} 

\iusr{Петр Кузьменко}
\textbf{Нина Яценко-Кончаковская} спасибо.

\iusr{Svetlana Dombrovskaya}
Хороший, теплый и душевный рассказ, спасибо!

\iusr{Валентина Коваленко}

\ifcmt
  ig https://i2.paste.pics/edab13ce93a7a55320154166ba1799a9.png
  @width 0.2
\fi

\iusr{Наталия Водяницкая}

Какое доброе и прекрасное воспоминание!.. Мы ровесники и дети наши,
оказывается, тоже!..  @igg{fbicon.grin}  А я сына родила тогда через неделю 05.07.85! Маршрутки
тоже помню. Часто пользовалась маршруткой, которая ходила от Бессарабского
рынка вверх по ул. Круглоуниверситетской за 15 коп. Мы жили на ул. Бассейная, 11.
Мама моя работала в МВД на ул.Богомольца. Туда можно было попасть или на этой
маршрутке, или пешком в др. сторону по ул. Бассейной и наверх мимо Октябрьской
больницы (сейчас Александровская). Спасибо, Пётр, за рассказ.  @igg{fbicon.smile} @igg{fbicon.heart.red}

\iusr{Alena Sidleckaya}

Единственный общественный транспорт детства, который я очень любила. Было это в
начале 80 х Раз в неделю мы с бабушкой садились на маршрутку у ТЮЗа и ехали на
Бессарабку за творогом. Самым вкусным и самым жирным. Потом шли в подземный
переход, который вёл на Крещатик м покупали конфеты ассорти на развес.
Садились снова на маршрутку и ехали назад. До сих пор помню мягкие диванчики и
любимое место у окна. Светлые моменты ну уж очень счастливого детства. Спасибо,
что поделились и помогли этим вспомнить.

\iusr{Наталия Ковалева}
Маршрутки были удобные и дешевые!



\end{itemize} % }
