% vim: keymap=russian-jcukenwin
%%beginhead 
 
%%file 26_01_2022.fb.golovachev_andrej.1.strashnyj_son_ameriki
%%parent 26_01_2022
 
%%url https://www.facebook.com/permalink.php?story_fbid=2790142391295521&id=100008993618796
 
%%author_id golovachev_andrej
%%date 
 
%%tags amerika,imperia,kitaj,rossia,usa
%%title Самый страшный сон Америки
 
%%endhead 
 
\subsection{Самый страшный сон Америки}
\label{sec:26_01_2022.fb.golovachev_andrej.1.strashnyj_son_ameriki}
 
\Purl{https://www.facebook.com/permalink.php?story_fbid=2790142391295521&id=100008993618796}
\ifcmt
 author_begin
   author_id golovachev_andrej
 author_end
\fi


Самый страшный сон Америки

Что будут делать США, если после Олимпиады Си Дзинпин и Путин сделают
совместное заявление, что однополярный мир официально закончился и что Китай и
Россия, как мировые центры силы, наряду  с Америкой,  берут ответственность за
ситуацию в мире в свои руки?

В сущности это будет означать, что Китай будет активно продвигать свою,
альтернативную Америке, социально - экономическую модель. Это будет означать
конец монополии США на на свою \enquote{единственно верную идеологию}. Это и
будет означать конец гегемонии США и конец однополярного мира. 

Каждой стране придется выбирать к какому центру  ей примыкать и на какую
идеологическую модель ей ориентироваться.

Это и есть самый страшный сон Америки. Интуиция мне подсказывает, что нечто
подобное  в этом году это произойдет.

Последние годы США проводили, на мой взгляд, примитивную, негибкую,
прямолинейную, слишком идеологизированную и неадекватную  их нынешней силе   и
авторитету   внешнюю политику и не смогли избежать самого неприятного для них
сценария,- союза Китая и России. США просто загнали Россию в этот союз.
Последствия для США такой политики будут плачевными.

Вообще любая империя, которая находится на закате,  напоминает 50-летнюю,
некогда красивую женщину, которая никак не может смириться с тем, что молодость
и  красота ее безвозвратно ушли и продолжает по прежнему  носить короткие юбки
и глупо кокетничать, как молодуха. А  чары уже не те,  спроса уже нет. Так и
империи, они не чувствуют момента, что их \enquote{молодость и обаяние} уже
ушли, но они по прежнему носят короткие юбки и злятся, что клиенты уже не
ведутся.
