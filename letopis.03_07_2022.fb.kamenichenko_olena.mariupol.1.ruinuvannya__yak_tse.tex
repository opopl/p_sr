%%beginhead 
 
%%file 03_07_2022.fb.kamenichenko_olena.mariupol.1.ruinuvannya__yak_tse
%%parent 03_07_2022
 
%%url https://www.facebook.com/nimue.errtru/posts/pfbid0uYVC6QdUW8FVaFjLFC34RpCiUEz7TXPgQjbR7CK5Rozktrkfv7SbcivbYYYcdLSWl
 
%%author_id kamenichenko_olena.mariupol
%%date 03_07_2022
 
%%tags mariupol
%%title Руйнування. Як це взагалі може бути?
 
%%endhead 

\subsection{Руйнування. Як це взагалі може бути?}
\label{sec:03_07_2022.fb.kamenichenko_olena.mariupol.1.ruinuvannya__yak_tse}

\Purl{https://www.facebook.com/nimue.errtru/posts/pfbid0uYVC6QdUW8FVaFjLFC34RpCiUEz7TXPgQjbR7CK5Rozktrkfv7SbcivbYYYcdLSWl}
\ifcmt
 author_begin
   author_id kamenichenko_olena.mariupol
 author_end
\fi

Руйнування

Як це взагалі може бути?

Я дивлюсь  у вікно на затишне подвір'я, де облаштована літня кухня, стоїть
гойдалка-диванчик, поруч сусідське подвір'я, таке ж охайне і красиве. 

Потім гуркіт. Бомбардування. Біжимо у підвал. Виходжу і опиняюсь в іншому
світі. Як все так могло змінитись? Де літня кухня? На її місті купа цегли й
пилу, уламки яскравих садових прикрас змішалися з уламками від ракет. Дах
сусіднього будиночка провалився всередину, однієї стіни нема, землю встеляє
скло. Воно скрізь. І де його стільки береться? У залишки вікон вириваються і
майорять білі завіски, немов прапори: "не бийте нас, ми мирні", - просять вони.
Але даремно. У зброї нема серця.

Жалібно і тихо, наче примара, з під завалів, мявчить кіт.

На ватяних ногах я виходжу на двір і не вірю своїм очам. Як це могло статись? Я
пересуваюсь немов по болоту, немов один невірний крок і мене може затягнути на
дно. Під ногами скрипить скло, попереду непролазні гори, які щойно були чиєюсь
домівкою. Та чого чиєюсь? Я знала тих людей! Я озираюсь і дивуюсь, якою
потужною зброєю нас накрило, як уламки снаряда розбилися на маленькі
смертоносні шматочки, і ті шматочки прошили наскрізь товсті дерев'яні будови,
металеві забори та двері. Я бачу  наскільки глибокі ями залишили на цеглинах
домів.

Бачу товсту книжку, яка своїми сторінками зупинила уламок розміром з кулак. Цей
уламок застряг і вплавився посеред рядків про кохання. 

Оглядаюсь далі та бачу людей. Те що залишилось від них. До цього моменту я
жаліла будівлі, які люди будували майже все життя, вкладали гроші та зусилля, а
у той момент зрозуміла і повторювала на протязі  кожного маріупольського дня:
"нічого страшного, що зруйновані дома, аби люди жили, люди - найважливіше, інше
відбудуємо колись".  Але люди гинули та довго лежали там, де їх наздогнала
смерть, тому що вийти на вулицю - то був страшний ризик. Потім, у часи затишшя,
їх забирали й ховали на подвір'ях. Це те що вдавалося приховувати від дітей, те
чого не повинна бачити жодна людина, але ж ми бачили.

Коли ворожі гармати розбили домівку мого старенького сусіда, він відразу
розпочав відбудову. Обстріли інших районів міста продовжувалися, я з дітьми
була у підвалі, а той сусід з рання виходив на морозну вулицю, (у березні у нас
було -10) та розбирав завали, ставив нове скло у вцілівши рами, прибирав
подвір'я. Це надихало і давало надію на те що бомбардувань більше не буде і ми
виберемось. Ми вибрались з міста, і сусід із жінкою також. Та незабаром наші
будиночки, і мій і його, зруйнували та спалили.

На фото будинок моєї мами, місце де я росла.

\#Маріуполь
