% vim: keymap=russian-jcukenwin
%%beginhead 
 
%%file slova.algoritm
%%parent slova
 
%%url 
 
%%author_id 
%%date 
 
%%tags 
%%title 
 
%%endhead 
\chapter{Алгоритм}
\label{sec:slova.algoritm}

%%%cit
%%%cit_head
%%%cit_pic
%%%cit_text
\emph{Алгоритмы} часто называют черными ящиками, потому что мы до конца не
можем предугадать результат их работы. Как думаете, сами компании контролируют
\emph{алгоритмы}, или они тоже видят в них черные ящики?  Компании частично
контролируют их. Они могут контролировать их на уровне определения основных
функций или влиять на поведение алгоритмов в конкретных ситуациях. К примеру,
недавно YouTube начал определять конспирологический контент. Платформа может
модерировать результаты поиска по сайту или точечно влиять на поведение
алгоритма. Но объяснить, по какой причине в ваших рекомендациях появилось
какое-то видео, невозможно. Параметров невероятно много. Но разработчик сможет
обосновать, по каким причинам в рекомендациях всплывает такой тип контента
%%%cit_comment
%%%cit_title
\citTitle{«У людей должен быть выбор» Как соцсети взламывают мозг человека и кто на самом деле их контролирует?: Coцсети: Интернет и СМИ: Lenta.ru}, 
Федор Тимофеев, lenta.ru, 28.10.2021
%%%endcit
