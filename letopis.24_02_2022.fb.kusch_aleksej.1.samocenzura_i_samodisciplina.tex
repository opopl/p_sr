% vim: keymap=russian-jcukenwin
%%beginhead 
 
%%file 24_02_2022.fb.kusch_aleksej.1.samocenzura_i_samodisciplina
%%parent 24_02_2022
 
%%url https://www.facebook.com/permalink.php?story_fbid=4825443167536037&id=100002112044676
 
%%author_id kusch_aleksej
%%date 
 
%%tags __feb_2022.vtorzhenie,edinstvo,narod,ukraina
%%title Самоцензура и самодисциплина
 
%%endhead 
 
\subsection{Самоцензура и самодисциплина}
\label{sec:24_02_2022.fb.kusch_aleksej.1.samocenzura_i_samodisciplina}
 
\Purl{https://www.facebook.com/permalink.php?story_fbid=4825443167536037&id=100002112044676}
\ifcmt
 author_begin
   author_id kusch_aleksej
 author_end
\fi

Самоцензура и самодисциплина.

Возьму на себя смелость сказать несколько слов.

Не сеять панику. 

Не выплескивать в социальные сети свои страхи, ведь они могут инфицировать
сотни людей. 

То есть, соблюдаем самоцензуру.

Не скупаем больше чем нужно: в магазинах и аптеках. Тому, кому реально
необходимо, может не хватить.

Не устраиваем заторы на дорогах. Они могут понадобиться для более нужных целей.

Помогаем слабым и нуждающимся. Например, престарелому соседу. 

Становимся единой семьей, национальным организмом.

Верим в себя.

Я искренне поражен стойкостью нашего народа. 

Да, пробки на житомерке. Но больше киевлян осталось.

Да, очереди в аптеках и магазинах. Но самое необходимое есть.

Люди на улицах максимально собраны. Паники нет.

Ближайшие дни будут решающими в нашей истории.

Пройдем их достойно. Кажый на своем месте.
