% vim: keymap=russian-jcukenwin
%%beginhead 
 
%%file 07_04_2022.fb.gin_anna.harkov.1.dnepr_sirena
%%parent 07_04_2022
 
%%url https://www.facebook.com/AnnaGin74/posts/7275839445821462
 
%%author_id gin_anna.harkov
%%date 
 
%%tags 
%%title В Днепре уже час воет сирена. Тут она какая-то особенно тревожная
 
%%endhead 
 
\subsection{В Днепре уже час воет сирена. Тут она какая-то особенно тревожная}
\label{sec:07_04_2022.fb.gin_anna.harkov.1.dnepr_sirena}
 
\Purl{https://www.facebook.com/AnnaGin74/posts/7275839445821462}
\ifcmt
 author_begin
   author_id gin_anna.harkov
 author_end
\fi

В Днепре уже час воет сирена. Тут она какая-то особенно тревожная. Звук, как
будто саундтрек к фильму о Второй Мировой. 

Я стою возле бомбоубежища, в чужом городе, курю и слушаю, слушаю, слушаю. 

Помню, как в детстве иногда включали сирену. По радио (еще та, пыльная
коробочка, которая висела на каждой кухне) диктор пафосно объявляла: \enquote{Сегодня в
московском районе города Харькова будет осуществляться  проверка
противовоздушной тревоги}. 

Я спрашивала у отца, для чего ее проверяют. Он отвечал "на случай войны" и
обязательно улыбался. Его улыбка означала, что это не всерьез, это так,
формальность, не о чем волноваться.  

И мы не волновалась. 

Для нас слово \enquote{война} было каким-то устаревшим что ли. Оно было из фильма
Быкова \enquote{в бой идут одни старики}, из стихотворения Симонова \enquote{жди меня}, из
папиных книг Суворова на полке, из учебника истории за 6-й класс, из парадов на
9 мая, из рассказов ветеранов, которые приходили к нам в школу, звеня орденами. 

Когда включали сирену во время уроков, нас выводили из класса цепочкой и мы,
держась за руки, спускались в спортзал.

– Внимание, эвакуация по схеме! – раздраженно командовала завуч Валентина
Ивановна в крыле младшей школы. В руках она держала большой  чертеж и
хмурилась, разглядывая его. 

А мы хихикали и топали по лестнице, как слоны. Уроки были сорваны – праздник.
Классно, если сирена приходилась на контрольную по математике. 

– Прикиньте, свезло пятому бэ! – вслух завидовали мы. Нас-то проверка застала
всего лишь на безобидной ботанике. Но все равно здорово.  

В спортзале было шумно, все по классам рассаживались на разложенные маты. Ну,
малышня. Старшеклассники убегали курить за теплицы. 

Это был 1985 год, через сорок лет после Войны. 

Спустя еще почти столько же, я стою возле бомбоубежища, беженка, курю и слушаю
сирену. Не учебную. Настоящее предупреждение о приближающейся смерти. 

Русская ракета летит прямо сейчас.  В любой город, в любой поселок, в любой
дом,  детский сад,  больницу,  театр, спортзал. В любую точку на карте Украины. 

43-й день войны. А я всё равно иногда не могу. В это. Поверить.

\ii{07_04_2022.fb.gin_anna.harkov.1.dnepr_sirena.cmt}
\ii{07_04_2022.fb.gin_anna.harkov.1.dnepr_sirena.cmtx}
