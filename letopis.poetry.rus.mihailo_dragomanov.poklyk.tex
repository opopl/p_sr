% vim: keymap=russian-jcukenwin
%%beginhead 
 
%%file poetry.rus.mihailo_dragomanov.poklyk
%%parent poetry.rus.mihailo_dragomanov
 
%%url https://textbook.com.ua/literatura/1481549892/s-2?page=1
%%author 
%%tags 
%%title 
 
%%endhead 

\Purl{https://textbook.com.ua/literatura/1481549892/s-2?page=1}

Гей, не дивуйтесь, добрії люди,
Що на Вкраїні повстало,
Що Україна по довгій дрімоті
Голову славну підняла!

Гей, українець просить не много:
Волі для люду і мови;
Но не лишає він до всій Русі
I к всім слов'янам любові.

З Північною Руссю не зломить союзу – 
Ми з нею близнята по роду, 
Ми віки ділили і радість, і горе 
І вкупі приймаєм свободу.
Ти, русин московський, один із всіх братів
Велике зложив государство,
Неси ж свою силу, де треба, на поміч,
На захист усьому слов'янству.
Клади свою славу і силу в освіті,
В краєвій і людській свободі,
Не думай ніколи неважити душу
З'єднаних з тобою народів.
По волі слов'янські й чужії народи
Поручно з тобою ітимуть:
І стане на волю литвин і естонець,
А лях в тобі брата обніметь.
Гей, слухай, ляше, ми тільки за наше
Лягали в степах головами!
Ми ж не хотіли, щоб панували
Вороги ваші над вами.
Ой, признай, ляше, ти руськеє право
На руськую землю і мову,
І з тої хвилини для власної долі
Заложиш міцную основу.
Гей, брате чеху, ми твого Жижки
Табора ще не забули,
Де ми за чашу громадську стояли
Й козацького строю здобули.
З тобою, сербине, не раз ми братались
На суші й на Руському морі,
Як бились за волю народну юнаки
На Січі й в твоїй Чорногорі.
З таким же запалом, з яким ми вмирали
За волю свою і за братську,
Ми раді повстати і кості зложити,
Щоб волю здобуть всеслов'янську.
Ой гляньте, слов'яни, встають на Заході
Німецькії грізнії хмари;
Ставаймо ж докупи, забудемо, браття,
Навіки нікчемнії свари.
Ой, не даваймо урвать ні частини
Слов'янського рідного краю,
Но проклят із нас той, хто волю зневажить.
Чужої землі забажає!
Зберімося, браття, в сім'ю рівноправну,
І крикнем на братському пиру:
«Ми хочем для себе й для цілого світу
Лиш волі, освіти і миру!»
Ой мати Вкраїно, ой Київ наш чесний,
Коли б довелось серед тебе
Те слово почути і вольнеє знам'я
Слов'янське підняти до неба!
Гейдельберг, 23 августа 1871.
