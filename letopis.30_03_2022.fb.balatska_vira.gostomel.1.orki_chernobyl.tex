% vim: keymap=russian-jcukenwin
%%beginhead 
 
%%file 30_03_2022.fb.balatska_vira.gostomel.1.orki_chernobyl
%%parent 30_03_2022
 
%%url https://www.facebook.com/Balatska.lyrics/posts/620961052875395
 
%%author_id balatska_vira.gostomel
%%date 
 
%%tags 
%%title Послали орків на Чорнобиль. Наказ короткий: захопіть!
 
%%endhead 
 
\subsection{Послали орків на Чорнобиль. Наказ короткий: захопіть!}
\label{sec:30_03_2022.fb.balatska_vira.gostomel.1.orki_chernobyl}
 
\Purl{https://www.facebook.com/Balatska.lyrics/posts/620961052875395}
\ifcmt
 author_begin
   author_id balatska_vira.gostomel
 author_end
\fi

\headCenter{Послали орків на Чорнобиль. Наказ короткий: захопіть!}

\raggedcolumns
\begin{multicols}{3} % {
\setlength{\parindent}{0pt}

\ifcmt
  ig https://scontent-mxp1-1.xx.fbcdn.net/v/t39.30808-6/277675158_620961092875391_5186006536741278247_n.jpg?_nc_cat=107&ccb=1-5&_nc_sid=8bfeb9&_nc_ohc=ZZtIYVpDh5kAX9_-shL&_nc_ht=scontent-mxp1-1.xx&oh=00_AT-bVZRhJ1zPJ8B1rwDeKp3y759OkLIGIc_8KyLq9n6Vzw&oe=626EADF8
\fi

\obeycr
Послали орків на Чорнобиль.
Наказ короткий: захопіть!
\smallskip
Тридцятий день... Здобутків обмаль:
рудих соснових верховіть
глузливий шум, окоп, лопати,
фургон червивих сухарів.
\smallskip
А капітан у військкоматі
казав - їх приймуть як царів!
Десятий день болить в макітрі,
холодний піт - немов ріка.
\smallskip
Сержант жартує: ваша світлість!
А й правда - світиться рука.
А потім дупи заясніли
(в кущі відбіг - і вже ганьба).
\smallskip
Від сухарів осточортілих
у шлунках з'їхала різьба \textendash
напав пронос, як божа кара,
нудота, слабкість... Караул!
\smallskip
А ще в лісах якась почвара
гарчить на танки упритул,
шкребе броню, гризе екрани,
на башті вила цілу ніч...
\smallskip
На ранок, в мокрому тумані,
розбіглись орки врізнобіч.
Рвонув і \enquote{світлість}. Пер, мов зомбі.
Та де ж ті села, хоч ти плач?
\smallskip
Сарай в профнастильній оздобі,
й дідок старенький - наглядач:
\enquote{Ласкаво просимо в реактор!}
\smallskip
Лайнувся орк, протер баньки:
примарний Анатолій Дятлов
на сонці гріє маслаки!
\smallskip
\enquote{Я що, помер?} - гикнув сердега.
"Хіба ж тут мруть? Ти просто здох.
Тепер душа твоя - бентега,
зневіра, біль, переполох.
\smallskip
Ти ж ніс в ці землі смерть і відчай?
То й став марою. Тінню, сном.
Забудеш доньчине обличчя.
\smallskip
Все, що любив ти, порохном
до ніг обсипеться. Дружина
цноту за сорок срібняків
продасть. Жебрацька домовина
дочасно проковтне батьків.
\smallskip
Ти будеш з відчаєм волати,
та звідси звук не виліта.
\smallskip
Така тепер твоя розплата:
марудна, затхла суєта.
\smallskip
Фантомно нитиме вітчизна,
яку ти кинув без жалю.
\smallskip
Твій труп згниє. Ворони тризну
уже святкують. \enquote{Зароблю!} \textendash
казав батькам, беручи зброю.
\smallskip
Платіж - скорбота і ганьба..." \textendash
прокаркав Дятлов, і марою
майнула тінь його ряба.
\smallskip
...Війна дотлілє. Перемога
вдихне життя в цей сірий край.
\smallskip
А орк немертвий засторогу
братам шептатиме впотай:
про гріх війни, туртури втрати,
про біль німого забуття...
\smallskip
Рудих лісів відмерлі ґрати
порвуть примару на шмаття.
І мирно спатиме Чорнобиль
в обіймах Прип'яті-ріки.
\smallskip
А в лісі гнитимуть утроби,
ординці, погань. Будяки...
\smallskip
30.03.22
\restorecr

\end{multicols} % }
