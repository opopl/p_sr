% vim: keymap=russian-jcukenwin
%%beginhead 
 
%%file 06_04_2023.fb.suhorukova_nadia.mariupol.1.golovne_scho_zhyvyj.cmt
%%parent 06_04_2023.fb.suhorukova_nadia.mariupol.1.golovne_scho_zhyvyj
 
%%url 
 
%%author_id 
%%date 
 
%%tags 
%%title 
 
%%endhead 

\qqSecCmt

\iusr{Алина Любченко}

Что-то не видно, чтобы оккупанты что-нибудь делали для жителей города.

\begin{itemize} % {
\iusr{Nadia Sukhorukova}
\textbf{Алина Любченко} самое лучшее, что они могут сделать - свалить из Мариуполя навсегда.

\iusr{Галина Малько}
\textbf{Алина Любченко} та ну що Ви а бордюри, шило на мило

\iusr{Evelina Konovka}
\textbf{Алина Любченко} как раз фото хорошо демонстрирует, что они сделали.

\iusr{Алина Любченко}
\textbf{Nadia Sukhorukova} понимаю. Сарказм.

\iusr{Olena Perkova}
\textbf{Алина Любченко} вони в своїх пропагандиських сюжетах дуже хваляться який гарний руський мир \enquote{наконец то пришёл в Мариуполь}!

\iusr{Настя Гонгало}
\textbf{Алина Любченко} вони з 9 років розграбували Луганськ, все вивезли, всі заводи, а людей зробили бомжами
\end{itemize} % }

\iusr{Вячеслав Приходченко}

\ifcmt
  igc https://i2.paste.pics/39c4452d22d41cdd35b6f4cbb943ea61.png
  @width 0.1
\fi

\iusr{Lena Omelchenko}

коли я бачу ці фото, я розумію, що вижити в цього аду було 1\%.......

\iusr{Yana Lupinova}

Узнала каждый куточек, выросла в том районе

\begin{itemize} % {
\iusr{Nadia Sukhorukova}
\textbf{Яна Лупинова} Яна, не можу зрозуміти де це? Що поблизу?

\iusr{Виктория Костоглодова}
\textbf{Nadia Sukhorukova}, Це мабуть ті будинки що за АТБ були на площі перемоги.

\iusr{Yana Lupinova}
\textbf{Nadia Sukhorukova} 

это пересечение Московской и Меотиды. Там еще виден магазин игрушек. На
последнем фото видно то, что осталось от здания, в котором когда-то был
компьютерный клуб Вирус. Все остальные дома это дворы этого самого района

\end{itemize} % }

\iusr{Елена Иванова}
🥺😥🙏🏻

\iusr{Галина Малько}

Так і було і чорний манікюр і нечесна голова, іноді і не впізнати, бо ходили як бомжі

\iusr{Борис Николаевич Яковенко}

Мариупольский смертоносный март, когда выходили из дома и не знали вернёмся ли
мы домой. Каждый день ходил к дочери, дорога занимала 15 минут. Иногда шёл
спокойнно, иногда с перебежками. Во время обстрелов падал в грязь и лужи, на
это ни кто не смотрел, главное что ты смог выжить. на улицах лежали трупы а
смерть ходила за нами по пятам.

\begin{itemize} % {
\iusr{Татьяна Коровина}
\textbf{Борис Николаевич Яковенко} но зачем так било рисковать ?

\iusr{Alina Beskrovna}
\textbf{Татьяна Коровина} Потому что не важно, где убьёт. На улице по дороге, у костра, или завалит в подвале. Поэтому риск везде одинаковый и лучше сходить проведать.

\iusr{Татьяна Коровина}
\textbf{Alina Beskrovna} но за чем ? Война

\iusr{Alina Beskrovna}
\textbf{Татьяна Коровина} Вы были в Мариуполе в прошлом марте?

\iusr{Татьяна Коровина}
\textbf{Alina Beskrovna} да
\end{itemize} % }

\iusr{Ирина Деревянко}

А до нас на район, десь після двадцятого березня, приходили двоє. Регулярно,
десь через два-три дні. Як тільки більш, менш тихіше було.

Вони приносили повідло густе, що ножем можна різати, шматочки у плівці, масло
невідомого походження, вони божилися що масло, але то був дешевий маргарин,
горілку, цигарки... Це було так огидно, я їх ненавиділа, все було дуже, дуже
дорого...

Ми їх називали перекупами. Але потім зрозуміла, що то все було награбоване і
вони просто наживалися.

Нічого не купувала у них жодного разу... Коли чоловік купив у них пачку цигарок
за 400 грн, я розплакалася. І зовсім не за гроші... Так огидно було, що
наживаються на нас... щурів підвальних...😢

\iusr{Елена Лукьяненко}

Моя сестричка з племінником були в підвалі Приват банка, на Нахімова. Вийшли
пішки на Метро 7 квітня. Сестра бігала до церкви біля поліклиники Водніков і
там купувала свічки по 40-50 грн за штуку! Ось так в війну церкви обдирали
людей.

\iusr{Irina Shvydka}

Рашистам не знакомо понятие Созидание. В их генетическом коде есть понятие
разрушение хаос и бардак

\iusr{Irina Baranova}

Який вигляд... У мене був улюблений комплект шапка та снуд, сиренево-рожевий,
він виявився дуже теплим. Я більше не змогла на нього дивитися. В ньому був наш
Ад. Пекло. Після блокади окупація. Там, у Маріуполі, всі виїхали, а ми ні.
Дивлюсь на чоловіка, коли він переодягається. А там шкіра і кісті, на спині
бачу ребра. Плачу. Я не така, бо він цього не допустив. Ми в Україні, молодша
донька навчається у школі, онлайн. Ходити у школу офлайн відмовляється, бо
дразнять однокласники "дите Донбасса". А вона дивом вижила. Який у нас вигляд?
Я не знаю. Купила на зиму іншу шапку і погладшила, бо хлібу дуже хочеться...

\begin{itemize} % {
\iusr{Nadia Sukhorukova}
\textbf{Ірина Баранова} однокласники бовдури. Поцілуйте доньку від мене. Вона дуже красива

\iusr{Irina Baranova}
\textbf{Nadia Sukhorukova} велике дякую! 💜

\iusr{Ольга Орел}
\textbf{Елена Лукьяненко} 

Ми жили біля драмтеатру. Мій чоловік ходив до церкви, яка поруч. Питав, чи
можна туди людей привести, думав, що там підвали міцні

Там була охорона, яка нікого не пускала. Ось так..

\iusr{Елена Лукьяненко}
\textbf{Ольга Орел} вірю. Як вірно Надія Сухорукова написала - \enquote{Ми в Маріуполі були одягнуті в душу...}

\iusr{Nadia Sukhorukova}
\textbf{Елена Лукьяненко} але в когось душі не було.

\iusr{Луіза Гамага}
\textbf{Ірина Баранова} 

В якому це місті України такі "однокласники"? Жах
просто. Вам треба поговорити з вчителем, так не повинно бути, дітям треба
розповідати про трагедію Маріуполя і пояснити, що фраза "дите Донбасса" - це
булінг. Ваша донечка - дитина України, як і вони. А дитині треба живе спілкування з
дітками, щоб забувати весь жах, що вона пережила і адаптувалася на новому
місці. Можливо їй потрібна допомога психолога. Сил вам і терпіння! Ви
вижили, вирвались з того пекла-це головне і все у вас буде обов'язково добре!

\iusr{Julia Mostipan}
\textbf{Ірина Баранова} 

обов'язково треба йти до директора і вимагати, щоб в класі показати фільм про
Маріуполь. Нехай знають про дітьо. Вони щось чули,але не знають реальності.

\iusr{Iryna Iryna}
\textbf{Julia Mostipan} 

хіба це допоможе. Це діти, якщо дорослі такі самі.... Бережіть доньку, нехай
навчається онлайн, це тимчасово. Дитяче серце і так стільки горя пережило

\iusr{Julia Mostipan}
\textbf{Iryna Iryna} якщо нічого не робити, нічого і не допоможе

\iusr{Неля Мачай}
\textbf{Ірина Баранова} 

у меня дочь работает в школе под Киевом. Очень много переселенцев и никто их не
обижает. А мариупольские дети имеют огромный багаж знаний.

\iusr{Iryna Iryna}
\textbf{Julia Mostipan} 

а ви багато чуєте в широкому загалі про маріупольську трагедію? Чи якісь є інші
фільми, які не створено маріупольськими журналістами про Маріуполь

Чи хтось усвідомлює в Україні рівень знищення міста. Яке не звільнене швидко,
як Буча, яке захопили, руйнують щодня! І цьому не видно краю.

І ще,чи хтось буде поважати нас ВПО, де допомога всім однакова 2000, чи людина
переїхала з Одеси, наприклад, чи такі, як ми з Маріуполя. Нас навіть міська
влада ніяк не підтримує.

Я, щоб не бути "дітьо Домбаса", нікому й не кажу звідки я. Бо тільки уважніше
роздивляються.

\end{itemize} % }

\iusr{Євдокія Фрушичева}

Ми вискочили з будинку, побігли і тільки потім я помітила, що син в одній
кофтинці, я в домашньому халаті, зимовій куртці і домашніх лосинах, донька в
пижамі і куртці. Коли дійшли до центру, пройшов перший шок і ми почали дуже
сильно мерзнути. Ніч в Драмтеатрі була дуже холодною. Тулилися один до одного,
лежачі на картонці на полу, і не могли зігрітися. В повній темряві нічого не
було видно, тільки шепіт і плач дітей... Так ми і прийшли наступного дня до
родичів. Мене прийняли спочатку за якусь стареньку жіночку, не впізнали в
заплаканому лиці знайомих рис. Голос був зірваний від крику і холоду... Читаю
ваші пости і все згадується, наче це було вчора.

А ще я сховала у внутрішньому кармані куртки всі наші заощадження і постійно
повторювала дітям, що якщо щось трапиться зі мною страшне, щоб діти зняли з
мене куртку і ховалися. Я постійно їм повторювала, що майбутнє їх виживання
залежить від моєї куртки і маленького чорного рюкзачка з документами. 😭

\begin{itemize} % {
\iusr{Nadia Sukhorukova}
\textbf{Евдокия Фрушичева} стільки пишуть люди своїх історій. Дуже хочу їх собрати. Щоб о Маріуполі ще більш почули. Ви така молодець. Пишіть 
\end{itemize} % }

\iusr{Наталья Середа}

А мы с мужем жили на московской. Муж в этот момент лежал в областной
нейрохирургии с ужасными болями. Ему постоянно кололи в позвоночник блокады и
собирались отправлять на операцию. Я так была занята поездками в больницу, что
даже не поняла что происходит. Он просто позвонил ночью и сказал война приезжай
забирай меня надо освободить палаты под раненых. Я конечно прилетела и забрала
но что делать с болями ни кто не сказал. И начались поездки по аптекам с
огромными очередями мне хотелось скупить все обезболивающие уколы и шприцы но
этого уже не было. Мне пришлось мгновенно научиться делать уколы с одного
шприца потому что запасных было очень мало. Ему они практически не помогали но
он стойко держался и старался не подавать вида и на равные со всеми мужчинами
подъезда бегал под взрывами таскал доски для костра. Хотя есть не могли кусок в
горло не лез. А ещё если выдавался момент бежали через аллею на Волгодонскую к
моей и на Азовстальскую к его маме. Как уже писали падая от взрывов и надеясь
что пронесёт и мы добежим. В подвал мы не спускались там и так было много людей
детей мы просто ложились в коридоре на пол открывали двери прощались каждую
ночь прося друг у друга прощения. Но самое тяжёлое в это время была
неизвестность

У нас в Харькове на последнем месяце беременности была дочь и что с ней мы
нечего не знали. Просто садились и всё. Это время на удивление принесло нам
новых хороших друзей которые жили в соседнем подъезде с нами много лет а теперь
это члены нашей семьи. Выбраться из ада правда сначало в оккупацию удалось
только в последних числах марта. И там мы наконец узнали что зять вез дочь
ночью в день начала войны и Харькова Довёз до Полтавы и там начались схватки
она очень трудно рожала когда всех отправляли в подвал а её оперировали
отважные врачи. И о счастье она подарила нам внука. А дальше как у всех
фильтрации и длинное путешествие нас Мариупольских бомжей по Европе. Которое не
как не кончается. Жизнь до и после.
