% vim: keymap=russian-jcukenwin
%%beginhead 
 
%%file 04_01_2022.fb.fb_group.story_kiev_ua.1.istoria_odnoj_kievljanki.cmt
%%parent 04_01_2022.fb.fb_group.story_kiev_ua.1.istoria_odnoj_kievljanki
 
%%url 
 
%%author_id 
%%date 
 
%%tags 
%%title 
 
%%endhead 
\zzSecCmt

\begin{itemize} % {
\iusr{Sergij Tikhy}
Все было бы очень мило, если бы от Д’Аннунцио очень скоро не стартовал Муссолини.

\begin{itemize} % {
\iusr{Виктор Киркевич}
\textbf{Sergij Tikhy} Они развивались паралельно, было исотрудничество. Но мне писатель интересен из-за Донателлы, которая была ещё та штучка.

\begin{itemize} % {
\iusr{Sergij Tikhy}
\textbf{Виктор Киркевич} Могу себе представить... Но о ней очень мало известно. И в Вики ее нет

\iusr{Sergij Tikhy}
Относительно Муссолини... Боюсь, что сотрудничеством єто не назовешь. Он - один из столпов режима

\iusr{Виктор Киркевич}
\textbf{Sergij Tikhy} Он, по происхождению - выхрест, не был столпом. Он более декоратор внешней атрибутики.

\iusr{Виктор Киркевич}
\textbf{Sergij Tikhy} Сережа! В моей книги "Киев "Кіевлянинъ" о ней очерк.
\end{itemize} % }

\iusr{Надежда Владимир Федько}
\textbf{Sergij Tikhy} Муссоліні проявив себе ще до Першої світової війни, коли порвав з марксизмом.
Габриеле д’Аннунцио поділяв Доктрину фашизму, тому свідомо примкнув до Муссоліні. Він був здібним пропагандистом, про свідчить збірка його статей і промов - «Тримаю тебе, Африка» (1936).
Дуче цінував д’Аннунцио і в 1924 році він отримав титул князя, а в 1937 році він очолив Королівську академію наук.
Габриеле д’Аннунцио був урочисто похований.

\iusr{Александр Венге}
\textbf{Сергій Тихий} Все же не очень мило и до Муссолини, если помнить о материальных источниках карнавальной жизни. Для меня интересен феномен массовой поддержки подобного устройства жизни. Многое напоминает.

\end{itemize} % }

\iusr{Дмитрий Даен}
Занимательно...

\iusr{Алексей Сыч}
Цiкаво, дякую.

\iusr{Natalie Michael}
Дуже цікаво переплетення доль. Світла, доброзичлива Пархомівка і д'Аннунціо.

\iusr{Иван Петренко}
\textbf{Виктор Киркевич} Жаль, нет подписей под иллюстрациями

\begin{itemize} % {
\iusr{Виктор Киркевич}
\textbf{Иван Петренко} Голубева, Ида Рубиншейн, Габриэль Д'Анунцио - В.Серов. Мадамузель Голубева -О.Роден.

\iusr{Иван Петренко}
\textbf{Виктор Киркевич} Благодарю Вас!

\iusr{Maksim Pestun}
\textbf{Виктор Киркевич} мой дедушка работал у Родена, знал Голубеву и Иду Рубинштейн.

\iusr{Виктор Киркевич}
\textbf{Maksim Pestun} Как его звали?

\iusr{Maksim Pestun}
\textbf{Виктор Киркевич} Теофил Фраерманн (Teo Fra)

\iusr{Виктор Киркевич}
\textbf{Maksim Pestun} Там работала Элеонора Блох. Интересно сохранился ее работы бюст Пушкина в школе напртив старого Охмадета?
\end{itemize} % }

\iusr{Barbara Novokhatska}
Очень интересно. Спасибо.

\iusr{Татьяна Оржеховская}
Благодарю

\iusr{Надежда Владимир Федько}

Габриеле д’Аннунцио...

\ifcmt
  ig https://scontent-frt3-1.xx.fbcdn.net/v/t39.30808-6/270371261_4879999762059443_3375497440802950793_n.jpg?_nc_cat=107&ccb=1-5&_nc_sid=dbeb18&_nc_ohc=9zB8Z3XzvWsAX-6qk3k&_nc_ht=scontent-frt3-1.xx&oh=00_AT9UUDuNFhiDH2iXVuqFwT_rfryl7DktuWRyT5yhHA5izg&oe=61E497EF
  @width 0.3
\fi

\iusr{Vladimir Feldman}

О как, а фашистов правыми называют...

\enquote{Согласно Конституции, гражданам гарантируется: личная свобода; бесплатное
начальное образование; оплата труда, обеспечивающая достойную жизнь;
гражданские права в полном объеме вне зависимости от пола, расы и религиозной
принадлежности; прожиточный минимум для безработных. Конституционно
закрепляется своеобразная концепция прав собственности: отныне никто не может
претендовать на имущество, если оно не было приобретено непосредственно за счет
личных трудовых усилий. Д’Аннунцио выдвигает лозунг «труд без утомления».
Фундаментальным принципом организации государства объявляется музыка...}


\iusr{Vadim Vadim}
Креативные были люди, спасибо за статью, удивительно.

\iusr{Таиса Солодовник}
Были там. На самой церкви в с. Пархомовка много масонских символов.

\iusr{Кретов Андрей}
Спасибо. \enquote{Скрещение судеб}.

\iusr{Maksim Pestun}

Как интересно!

\iusr{Александр Андриевский}
Niesamowicie...!

\iusr{Владимир Дубровский}
О такой республике можно смело сказать: " В дурдоме каникулы!

\iusr{Наталья Писная}
Да, не представляла даже об этих исторических событиях.. интересно! И опять же связь с Украиной!

\iusr{Світлана Куликова}

\ifcmt
  ig https://i2.paste.pics/ca60fd6eacc9022e6f6e27d9bd18bd53.png
  @width 0.2
\fi

\end{itemize} % }

