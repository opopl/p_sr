% vim: keymap=russian-jcukenwin
%%beginhead 
 
%%file 05_04_2021.fb.safinaz_alieva.1.krym_tatary_shkola
%%parent 05_04_2021
 
%%url https://www.facebook.com/safinaz.alieva/posts/3811807952238378
 
%%author 
%%author_id 
%%author_url 
 
%%tags 
%%title 
 
%%endhead 

\subsection{Крым - Татары - Школа - Депортация - Унижение}
\Purl{https://www.facebook.com/safinaz.alieva/posts/3811807952238378}

Селям алейкум! Хочу рассказать о том, как учительница в нашей школе занимается
разжиганием межнациональной розни. Вся наша семья возмущена произошедшим, и мы
не собираемся оставлять это в таком положении. Обращаемся к общественности и
планируем обратиться в соответствующие инстанции, чтобы максимально придать это
огласке.


\ifcmt
  pic https://scontent-bos3-1.xx.fbcdn.net/v/t1.6435-9/169127405_3811807712238402_1860531958439511556_n.jpg?_nc_cat=102&ccb=1-3&_nc_sid=8bfeb9&_nc_ohc=yab3tGjeEpYAX8xMb3g&_nc_ht=scontent-bos3-1.xx&oh=59a14536ca706d6d233b88ac83aeb576&oe=60968B4D

	pic https://scontent-bos3-1.xx.fbcdn.net/v/t1.6435-9/169603222_3811807875571719_2050911405441589358_n.jpg?_nc_cat=100&ccb=1-3&_nc_sid=8bfeb9&_nc_ohc=XkyHGQWoSSgAX_ZeTkr&_nc_ht=scontent-bos3-1.xx&oh=e44c965a1b4b4d37df892e021c857a45&oe=6094CC92
\fi


Моя дочь Самира Алиева учится в средней школе села Дачное (Таракъташ)Судакского
района в 4 классе.  Какое-то время назад всем ученикам класса классный
руководитель и она же учительница по литературе Руденко Елена Алексеевна дала
задание подготовить рассказ о своих дедушках и бабушках, которые прошли вторую
Мировую войну. 

Сегодня, 5 апреля 2021 года, моя дочь рассказала свою историю про нашего
дедушки, которую приготовила в качестве домашнего задания. Она объясняла
одноклассникам и учителю, что её прадедушка воевал на фронте, получил очень
серьёзное ранение и вернулся домой. После возвращения так произошло, что весь
народ был депортирован и вместе со всем народом были также высланы из Крыма
наши дедушка и бабушка. 

Дедушка в депортации прожил недолго - всего лишь год и умер. А бабушка осталась
с пятью детьми на руках. Она одна их подняла, отучила, воспитала. Дочь вкратце
написала до скольки лет бабушка дожила, сколько внуков, правнуков и
праправнуков увидела и таким образом завершила свой рассказ. 

Во время её рассказа учительница остановила и сказала: «Сейчас пусть Самира
закончит, и я вам расскажу из-за чего крымских татар выслали». Как только моя
дочь закончила свой рассказ, учительница, обращаясь ко всему классу, где 9
крымскотатарских детей учится и 7 учеников славянской национальности,
обратилась к детям со следующим комментарием.

«Я вам сейчас расскажу почему крымских татар выслали. Их выслали потому, что
они были предателями, а их предательство заключалось в том, что они помогали
немцам, оккупантам захватить Крым», - заявила всему классу учительница.

Соответственно, моя дочь буквально сразу почувствовала осуждающие взгляды своих
одноклассников – не крымских татар. Тоже самое я узнала от одного из родителей
нашего класса, который позвонил моему мужу. Их дочь пришла заплаканная. Наши
дети получили травму психологическую и моральную. Все родители в нашем классе
возмущены. Мы требуем привлечь учителя к ответственности за подобные действия,
а министерство образования Крыма дать действиям педагога соответствующую
оценку.

