% vim: keymap=russian-jcukenwin
%%beginhead 
 
%%file 28_09_2021.fb.fb_group.story_kiev_ua.1.gogol_i_kiev.cmt
%%parent 28_09_2021.fb.fb_group.story_kiev_ua.1.gogol_i_kiev
 
%%url 
 
%%author_id 
%%date 
 
%%tags 
%%title 
 
%%endhead 
\subsubsection{Коментарі}

\begin{itemize} % {
\iusr{Victor Gregory}
С географией у Коли было так себе, раз по пути из Полтавы в Нежин проскочил Киев.

\iusr{Milana Piskalenko}
Браво!

\iusr{Tatiana Berlin}
Во торкнуло.....

\iusr{Ольга Кирьянцева}

Просто замечательно! Тонкая ирония Вам очень удалась.
@igg{fbicon.hands.applause.yellow}  @igg{fbicon.face.happy.two.hands} 

\iusr{Vladimir M. Pushkarev}
Маленькая неточность: Паниковский там не жил, а работал.

\begin{itemize} % {
\iusr{Аркадій Друкер}
\textbf{Vladimir M. Pushkarev} а рядом жил. Потомушо у него нога была больная. Куда же ему с больной ногой? Поезжайте в Киев и спросите.

\iusr{Vladimir M. Pushkarev}
\textbf{Аркадій Друкер} Рядом жить он не мог - не по Сеньке шапка!

\iusr{Lara Ilich}
\textbf{Vladimir M. Pushkarev} и до сих пор работает...)

\iusr{Yan Nudelman}
\textbf{Vladimir M. Pushkarev} Конечно жил. Мы были соседями по коммунальной квартире.

\iusr{Аркадій Друкер}
\textbf{Yan Nudelman} Теперь я догадываюсь, кто у него трость оторвал. За что вы его так не любили?!

\iusr{Vladimir M. Pushkarev}
\textbf{Yan Nudelman} Тогда коммунальных квартир еще не было - большевики еще не угнездились.
\end{itemize} % }

\iusr{Victor Gregory}
Кстати, на углу Крещатика и Кадетской ещё перепичку не продавали, иначе он бы ни в какие Италии не уехал.

\iusr{Наталья Дрягун}
Милый рассказ  @igg{fbicon.smile} 

\iusr{Ирина Иванченко}

Вот до чего прекрасна страна/ народ, когда про её / его лучшие красоты /
чесноты самым талантливым мозгам которой/ ого гораздо продуктивней/ эффективней
описывать их ( красоты)из - за границы!

\begin{itemize} % {
\iusr{Vadim Basovskiy}
\textbf{Ирина Иванченко}

\enquote{Почему иностранец менее стремится жить у нас, чем мы в его земле? Потому, что
он и без того уже находится за границей}. Козьма Прутков.


\iusr{Michael Viz}
\textbf{Vadim Basovskiy} это про Россию. Не перепутали?

\iusr{Dubizhansky Ludmila}
\textbf{Ирина Иванченко} Дык большое видится на расстоянии! Это еще классик заметил

\iusr{Галина Петрова}
вот чего?...- ни чего!)

\end{itemize} % }

\iusr{Тамара Ар}

Как всегда, типа в духе \enquote{сатиры}, прошлись по Николаю Васильевичу Гоголю!
Автор, вам до него все равно далеко! Не утруждайтесь

\begin{itemize} % {
\iusr{Ірина Кравець}
\textbf{Тамара Ар} А Ваше мнение истина в последней инстанции?

\iusr{Аркадій Друкер}
\textbf{Toma Ar} но вот вы же утруждаетесь. Не пишите. И будет вам щасте.

\iusr{Tatiana Thoene}
\textbf{Toma Ar} и Вы не утруждайтесь читать. Пропустите и промолчите.

\iusr{Kate Levenko}
\textbf{Toma Ar} Підкажіть, а ви буваєте чимось задоволені? Ну там, зранку встали і все добре, сонечко світить, пташки співають? Що не коментар - негатив, вже почала вас впізнавати за аватаркою  @igg{fbicon.beaming.face.smiling.eyes}  А автор молодець, тонко підмітив такий от типаж  @igg{fbicon.wink} 

\iusr{Наталия Веремеенко}
\textbf{Тамара Ар}
Нате і мою баночку на
свячену воду!.. @igg{fbicon.face.eyebrow.raised} 

\iusr{Наталья Эгатова}

\ifcmt
  ig https://scontent-mxp1-1.xx.fbcdn.net/v/t39.1997-6/p480x480/105941685_953860581742966_1572841152382279834_n.png?_nc_cat=1&ccb=1-5&_nc_sid=0572db&_nc_ohc=_iE3XtYBdGYAX_yWoLc&_nc_ht=scontent-mxp1-1.xx&oh=c51abc2b22b58bf5986c988f967a31d1&oe=61B7FD4B
  @width 0.2
\fi

\iusr{Наталья Твердохлеб}
\textbf{Тамара Ар} Мне кажется, что автор и не думал соревноваться с Николаем Васильевичем. @igg{fbicon.smile}  Он просто хотел сделать наше утро добрым. Я, например, благодарна @igg{fbicon.smile} 

\iusr{Света Арсеньева}
\textbf{Тамара Ар} Как по мне Золотушкин покруче Гоголя.  @igg{fbicon.wink} 

\iusr{Alik Perlov}
\textbf{Тамара Ар} С утра негатив у вас))). Съешьте конфетку Шоколадную, попустит ))

\iusr{Yan Nudelman}
\textbf{Тамара Ар} Дык это написано не для вас.

\end{itemize} % }

\iusr{Татьяна Вакуленко}
Вот хорошо же!
Сделали мое пасмурное утро!

\iusr{Надежда Кальная}
Мне понравилось! Короткий, но всеобъемлющий экскурс в жизнь великого! @igg{fbicon.hands.applause.yellow} 

\iusr{Ірина Кравець}
Какая с утра недобрая! Какая-то желчная и вдругих коментах/как-то уж заметно/не
хотелось писать, но уж очень явно

\begin{itemize} % {
\iusr{Анатолий Золотушкин}
\textbf{Ірина Кравець} О них и написано
\end{itemize} % }

\iusr{Валерий Якимчук}
\enquote{У нас тут своих с носами некуда девать}.(( Увы, это исторически не
достоверно... @igg{fbicon.wink} 

\iusr{Энна Кот}
Треба цінувати своїх геніїв!

\begin{itemize} % {
\iusr{Аркадій Друкер}
\textbf{Enna Kot} але це не означає, що не можна жартувати, використовуючи їх імена. Гадаю, вони не образилися би, маючи гарне почуття гумору.
\end{itemize} % }

\iusr{Татьяна Тихонова}
Точно, ёмко, в меру саркастично, спасибо!

\iusr{Виктор Задворнов}

Ой, зря вы так о Нежине. Как давно вы там были? Да, это не столица. Здесь
уютнее, спокойнее... И главное для меня, в местном драмтеатре иногда и Гоголя
дают.

\end{itemize} % }
