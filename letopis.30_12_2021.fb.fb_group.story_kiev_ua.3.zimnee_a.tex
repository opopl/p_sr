% vim: keymap=russian-jcukenwin
%%beginhead 
 
%%file 30_12_2021.fb.fb_group.story_kiev_ua.3.zimnee_a
%%parent 30_12_2021
 
%%url https://www.facebook.com/groups/story.kiev.ua/posts/1829949167201876
 
%%author_id fb_group.story_kiev_ua,kaganovich_anatolij
%%date 
 
%%tags deti,kiev,semja,zima
%%title Зимнее...
 
%%endhead 
 
\subsection{Зимнее...}
\label{sec:30_12_2021.fb.fb_group.story_kiev_ua.3.zimnee_a}
 
\Purl{https://www.facebook.com/groups/story.kiev.ua/posts/1829949167201876}
\ifcmt
 author_begin
   author_id fb_group.story_kiev_ua,kaganovich_anatolij
 author_end
\fi

Зимнее...

\ii{30_12_2021.fb.fb_group.story_kiev_ua.3.zimnee_a.pic.1}

Темным декабрьским вечером 1965 года из горбатенького ЛАЗика вывалился
закутанный первоклассник с маленькой скрипочкой в чехольчике и с торбочкой для
сменной обуви. Это я возвращался домой с урока музыки.

На платные уроки игры на скрипке я ходил уже 3 месяца. Уроки были в моей 71-й,
школе и подзарабатывал этим местный учитель музыки. Откуда он нашел во мне хотя
бы малейшие зачатки музыкального слуха – не знаю, оставим это на его совести.
Правда, и я немного помог ему убедить в этом мою маму, безошибочно,
отвернувшись, и на слух, определяя нажимаемые клавиши пианино. А то, что я
видел все клавиши в зеркальном шкафу – оставим это на моей совести, тем более
что я не знал, чем для меня обернется такая выходка.

\ii{30_12_2021.fb.fb_group.story_kiev_ua.3.zimnee_a.pic.2}

Теперь я должен был 3 раза в неделю, вернувшись домой со школы и пообедав,
отправляться обратно на музыку. Ладно, еще золотой осенью такие походы были
даже увлекательны, но в снежном и морозном декабре, пройти от нашего дома на
Гарматной, 18 до 71-й школы, а это больше километра по узенькой протоптанной
тропке между громадными сугробами частного сектора улицы Металлистов, а потом
еще столько же по Индустриальной и Брест-Литовскому, мимо 500-летнего дуба, под
новеньким мостом «Большевик», мимо дома с гастрономом, стоявшего впритык к
мосту, и через проходные дворы к школе, которая располагалась, как раз напротив
киностудии Довженко - это было серьезное испытание для первоклашки.

Конечно, я мог бы пойти на Выборгскую, на остановку, и постоять минут 10-15 в
ожидании заветной «Двоечки», автобуса №2, который ходил по настроению, и в
который с пожеланиями, вроде: -Держи крепче пацана! - меня точно бы посадили.
Но это если еще «Двоечка» остановится... 

Поэтому я, чаще всего, и проходил этот путь пешком.  

71-я школа – это первая «образцовая школа» построенная в Киеве при Советской
власти. Проект Алешина, с участием Каракиса! «Конструктивизм, здание нового
типа – светлое, просторное, функционально удобное. В школе была столовая,
актовый зал на 500 человек и кинозал на 200 учеников.»


Я поступал туда в 6 с половиною лет и был принят, только после проверки лично
директором школы моих уверенных знаний письма и чтения.

За два с половиной года учебы, мне вспоминается только свой букварь с
портретом, еще Хрущева, на первой странице, яркие новогодние карнавальные
вечера, и свои шикарные, мастерски сшитые мамой по ночам из подручного
материала, костюмы (на одном фото «Черный принц» и «Айболит» в центре второго
фото).

\ii{30_12_2021.fb.fb_group.story_kiev_ua.3.zimnee_a.pic.3}

И совсем не вспоминаются эти отвратные уроки музыки и что, и как вымучивалось
мной из несчастной «четвертушечки». Помню, что дурость этих уроков меня достала
и домой я возвращался очень злой.

За несколько дней до этих событий, ранним морозным утром, выставленный,
уходящими на работу родителями, за дверь, я пешком и совсем замерший, преодолел
героически свой школьный маршрут. Я даже успел до звонка. Только школа была
пустая. Как объяснила техничка: - Занятия отменили. Холодно. Минус 24! 

Откуда об этом могли знать мои родители, слушавшие по утрам развлекательную
радиопередачу «Опять 25», а не тусклые хохмы местного радио?

Как сейчас помню себя сидящем на полу в пустом холле школы и ревущим от обиды.
А потом, отревевшись и отогревшись, поехал домой. 

Путь к дому был более комфортным и цивилизованным. «Двоечка» от «Большевика»
шла не очень полная, а в вестибюле новенькой станции метро всегда можно было и
погреться в ожидании. Горбатенький ЛАЗик довез меня до верхней точки Гарматной,
до сберкассы под номером 10, которую позже ограбят в фильме «Инспектор
уголовного розыска». Нужно было осилить еще какие-то 300м, но сил у меня уже не
было.  Я так уютно устроился в плотных и защищающих от ветра кустах терновника,
растущих на газоне рядом с остановкой, так вкусно согрелся и заснул, что даже
брыкался усилиям молодого парня, разбудить меня.  Но Парень оказался упорным,
отнес меня на руках к дому и выпустил только в теплом парадном.

\ii{30_12_2021.fb.fb_group.story_kiev_ua.3.zimnee_a.pic.4}

Этот недавний случай и скрипичное отвращение придали мне решимости.

Когда я с порога заглянул в нашу комнатку- папа лежал на диване и читал газету.

- Паап, я больше не пойду на скрипку – угрюмо заныл я.

- Почему? - спросил папа, не отрываясь от газеты.

- Холодно, далеко, не интересно – зачастил я.

- Пойдешь, уже все заплачено – отрубил папа, не отрываясь от газеты.

- Не пойду! – закипал я.

- Пойдешь - нажимал папа.

Тогда я взял за гриф свою «четвертушечку» и плотно тюкнул ее об угол шкафа.
Сломанное дерево отозвалось гулко. Папа отложил газету, посмотрел на повисшую
на струнах скрипочку и сказал: - Хорошо. Ты больше не пойдешь на эту музыку. 

И снова закрылся газетой. 

Маме мы сказали, что скрипочка сломалась в сугробе…

П.С.  В 1945-м, когда семья деда вернулась с эвакуации, все пришлось начинать с
нуля. Их комнату на Коминтерна, 12, забрал фронтовик, а из мебели удалось
отбить у соседей только буфет. Дед снова пошел работать на Гарматную на родной
43 авиазавод.  

Жить пришлось на заводе в бараке на углу Гарматной и Берестейского
(Брест-Литовского) шоссе. Бабушка, сразу же вскопала огород и уже скоро были
свои огромные помидоры и картошка. В 1946-м авиазавод был перенесен на новую
площадку и теперь это наш знаменитый «Антонов», а на его бывших площадях было
начато строительство комбината «Промстройдеталь» (КАМЗ). Мой дед - Айзек
Гринберг, к тому времени, был зав. складом, только образованного комбината.

Вот к нему на склад и привезли в 1945г. статую Ленина. Именно того Ленина,
которого 5 декабря 1946 года   установят на Бессарабке. И моя 10-летняя мама
год по нему бегала. 

А в 1948 г. пленные румыны достроили заводской дом на Гарматной, 18, и семья
покинула барак. 

П.П.С. Трехэтажный Г-образный дом с глубокими подвалами был построен пленными
румынами на совесть. 

Хорошо помню огромный двор, почти полностью затемненный огромными кленами. И
наши дворовые команды, увлеченно играющие в «штандера» и «квача», делающие
набеги на ближайший стекольный завод за трубочками-плювалками и  подсыпающие
сахар в бензобак новенькой «Волги», живущему по соседству известному футболисту
Вячеславу Семенову, когда он изменил родному «Динамо» с луганской «Зарей».

П.П.П.С. В 3х минутной нарезке из уцелевшего до оцифровки, большого 8мм.
папиного фильма можно еще  разглядеть и наш зимний двор, именно 1965-66гг, т.е.
времен описанных событий, и игры с братиком в 10 метровой комнатке, в которой
мы жили вчетвером, и соседей, и любимую горку, и друга Аркашку и меня на лыжах,
и еврейскую бабушку кормящую внуков, и папу с  любимым семейным жестом и мои
потуги,  ученические и утренние.
