% vim: keymap=russian-jcukenwin
%%beginhead 
 
%%file 07_04_2022.fb.fb_group.story_kiev_ua.1.misto_i_pamjatnyky_chastyna_1.cmt
%%parent 07_04_2022.fb.fb_group.story_kiev_ua.1.misto_i_pamjatnyky_chastyna_1
 
%%url 
 
%%author_id 
%%date 
 
%%tags 
%%title 
 
%%endhead 
\zzSecCmt

\begin{itemize} % {
\iusr{Владимир Картавенко}

За постановою Кабінету Міністрів України №929 від 10.10.2012 до Державного
реєстру нерухомих пам'яток України внесено пам'ятку науки, техніки та історії
Національного значення монітор «Железняков», охоронний номер 260062-Н.

29.12.2020 Наказом №19 начальника гарнізону міста Києва, генерал-полковника
Олександра Сірського корабель – пам'ятник Національного значення монітор
«Железняков» закріплений за військовою частиною А1225 ГУР МО України
дислокацією на Рибальському півострові Києва.

За наказом Міністерства культури України №604 від 04.07.2013 до Державного
реєстру нерухомих пам'яток України внесено пам'ятник Історико – меморіальний
комплекс Київської гавані давньої Руси Парк Моряків площею 2.77 Га., охоронний
номер 923-Кв.

3 вересня 2021 року Міністерством культури України за №700 до Державного
реєстру нерухомих пам'яток України внесено пам'ятник морякам героїчної
Дніпровської військової флотилії, охоронний №3066-Кв. Пам'ятник про історичні
події та героїчно загиблих у боях з нацистами військових моряків втілений у
матеріально-технічній структурі та мистецькому рішенні монументальної
композиції в традиціях української реалістичної школи історичного жанру у
скульптурі, створено на зібрані народом кошти, 06.10.1979 урочисто відкрито на
Поштовій площі і 08.05.2013 перенесено до Історико Меморіального комплексу
Київської гавані давньоруської доби Парк Моряків.

\end{itemize} % }
