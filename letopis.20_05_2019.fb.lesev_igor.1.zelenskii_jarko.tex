% vim: keymap=russian-jcukenwin
%%beginhead 
 
%%file 20_05_2019.fb.lesev_igor.1.zelenskii_jarko
%%parent 20_05_2019
 
%%url https://www.facebook.com/permalink.php?story_fbid=2456809134350175&id=100000633379839
 
%%author_id lesev_igor
%%date 
 
%%tags politika,ukraina,vybory,zelenskii_vladimir
%%title Зеленский – это ярко
 
%%endhead 
 
\subsection{Зеленский – это ярко}
\label{sec:20_05_2019.fb.lesev_igor.1.zelenskii_jarko}
 
\Purl{https://www.facebook.com/permalink.php?story_fbid=2456809134350175&id=100000633379839}
\ifcmt
 author_begin
   author_id lesev_igor
 author_end
\fi

Зеленский – это ярко. При этом не путаем эпитет «ярко» со словами «хорошо»,
«правильно» и, тем более, «справедливо». Ярко – это то, что запоминается. Что
продается.

Чем отличается «кока-кола» от «пепси»? Рекламой.

Зеленский через 5 лет после Порошенко в своей инаугурационной речи по сути
сказал ТО ЖЕ САМОЕ, что ТОГДА говорил всегда поддатый Черчилль. Мир, труд, май.
И даже к жителям Донбасса – там же дебилы живут, могут чего недопонять – было
такое же обращение на русском языке.

\ifcmt
  ig https://scontent-frt3-2.xx.fbcdn.net/v/t1.6435-9/60506127_2456809067683515_9058964107557863424_n.jpg?_nc_cat=103&ccb=1-5&_nc_sid=730e14&_nc_ohc=b04oSIRnaI4AX_8PI_p&_nc_ht=scontent-frt3-2.xx&oh=5488479ab18a4c46e2dd141a5211864e&oe=61BABEA4
  @width 0.4
  %@wrap \parpic[r]
  @wrap \InsertBoxR{0}
\fi

Но разница все же есть. Это как два разных человека рассказывают один и тот же
анекдот. Один весело. Другой смотрится посмешищем.

Порошенко все 5 лет был посмешищем и именно потому появился Зеленский.

Но все же Зеленский не так прост. Он яркий продавец бессмысленных образов.
Любимчик.

А еще он, похоже, учится. Доверяет команде. И попал в хорошие руки скульпторов.

Вот сравните с Кличко и Вакарчуком. Две грустные единицы. Что в них не
вкладывай, а достаешь из черного ящика или трусы, или часы. Никакого развития,
никакой генеральной идеи, даже никаких фишечек. Трусы и часы. УДАР и «Голос».

А Зеленский продает эмоции. В изнасилованной Порошенко Украине это дохрена
какой ценный товар.

Ничего не сказать и понравиться, это как пофлиртовать с кассиром на кассе. Ты
ведь все равно не тратишь времени дольше обычного, а человек, делая ту же самую
работу, что и для других, уже тебя чуть-чуть любит.

Итак, что хомячки запомнили по итогу инаугурационной речи Зеленского?

Первое. Рада идет найух. Этого хотят все. Это в агрессивно-художественной форме
озвучил Зе. Что он теряет? Ни-че-го. Послать нахер Раду – это бесплатное
удовольствие. Не требующее ресурсных затрат. Репутационно безукоризненное.

Второе. Мир на Донбассе любой ценой. Но без «уступок территории». Как это можно
сделать? Никак. Но кто хочет мира? Все, кто не в теме контрабаса и не поклонник
Черчилля. Короче, почти все.

Третье. Вся мразота – нахер с пляжа. Гройсман, Луценко, Грицак и прочие
гиперлуповичи, ответственные за дороги. Ведь именно этого ждут хомячки. И Зе
подал сигнал – я не играю в договорняк. Так ли это на самом деле, уже не важно.
Ведь продаются эмоции.

В итоге Зеленский продолжает умело делать именно то, что у него наиболее
красочно и художественно получается. Он борется с драконом. Борется ярко. Со
спецэффектами. И удачными импровизациями, потому что не всегда даже в
сумасшедшем доме встретишь таких полезных идиотов, как Ляшко.

И повестка у Зеленского выстроена предельно правильно. Хомячки увлечены
падением дракона. И совсем не интересуются, а кто же станет новым драконом.

А напрягаться теперь серьезно нужно двум персонажам. Коллективному
Бойко-Медведчуку и Юле.

По Бойко. Он и так скучный как черт. У ребят вообще с лицами постоянно какая-то
волосатая жопа выходит. Как не Янукович, так Бойко. А если не Бойко, то Вилкул.
И давайте прямо, пацана с такими рожами выходит в дамки только по
содержательной теме. Забрать у Бойко тему языка и мира и все – нет больше
Бойко. Вернее, нет больше 10\%-го Бойко.

А Зеленский в своей обтекаемой риторике как раз эту тему на 2 месяца у Бойко
может выхватить. И сохранить свои текущие 40\%+ процентов до дня Х.

И самое паршивое для связки Б+М, что Зе им прямо сейчас абсолютно нечем гасить.
Взятки гладки. Пацан ни за что не в ответе. И никому ничего не обещал. А ведет
себя так, чтобы понравиться восточным. Не упоротым, а именно болотным хомячкам.
А таких всегда большинство.

Вот это сейчас большая проблема для Бойко-Медведчука. У них нет времени на
раскочегариться. Им эта досрочка как раз нах и не упала. Им бы осень. Да
тарифы. Да дождаться спада эйфории по Зе... А тут тебе такое вот глянцевое
западло.

Но в еще большей печальке Юля. Знаете, можно спутать клиренс с климаксом. Слова
хотя бы похоже звучат. Но Юля спутала секс с петтингом. И умудрилась трахнуть
саму себя.

Юля замарала себя мовным законом. Не в том плане, хорошо это или нет. А в том,
что это был Петин закон и Петина повестка. Это был парад Черчилля в Галичине.
Ему, похоже, больше уже ничего и не нужно.

А вот что добилась Юля? Отрезала себе выход на умеренных хомячков на востоке.
Это раз. И не стала патриотичнее Пети. Это два.

Вот смотрите. Есть три цвета в светофоре. Красный, зеленый, ну и пусть не
желтый, а синий. Петя, Вова и Юра. Чувствуешь в сердце дух Бандеры – тебе с
Петей. Любишь в разной форме Владимира Владимировича – для тебя есть Юра. А
Вова – для всех остальных, потому что он не за Бандеру, и не за Путина.

Ну а где здесь Юля? Она застряла в болоте из абстракций, которые НЕКОМУ
продавать. Нет, 10\% наскребется. Но что ты с ними будешь делать, когда
пожелания, как у владычицы морской?

В общем, у Юли только одна надежда. Если прямо сейчас начнется уголовный разнос
команды Черчилля. Только обнуление Пети даст возможность Юле подзатариться
голосами до польской границы.

Ну а Зеленский таки да, молодец. Только за одни эти кислые морды в парламенте
на его инаугурации за него стоило проголосовать. Даже осознавая, что Зеленский
вместе с этим публичным буллингом пещерных неандертальцев порождает
одновременно нового дракона.

\ii{20_05_2019.fb.lesev_igor.1.zelenskii_jarko.cmt}
