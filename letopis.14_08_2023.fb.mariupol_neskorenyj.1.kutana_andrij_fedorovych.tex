%%beginhead 
 
%%file 14_08_2023.fb.mariupol_neskorenyj.1.kutana_andrij_fedorovych
%%parent 14_08_2023
 
%%url https://www.facebook.com/100066312837201/posts/pfbid02uR5CXFjd1cULi6vpJrTTEsiHBXAixawYc9DD3udTop17LukSdVcu9CXHdzTbaBTPl
 
%%author_id mariupol_neskorenyj
%%date 14_08_2023
 
%%tags 
%%title Кутана Андрій Федорович
 
%%endhead 

\subsection{Кутана Андрій Федорович}
\label{sec:14_08_2023.fb.mariupol_neskorenyj.1.kutana_andrij_fedorovych}

\Purl{https://www.facebook.com/100066312837201/posts/pfbid02uR5CXFjd1cULi6vpJrTTEsiHBXAixawYc9DD3udTop17LukSdVcu9CXHdzTbaBTPl}
\ifcmt
 author_begin
   author_id mariupol_neskorenyj
 author_end
\fi

📎Ще один учасник проєкту \enquote{Маріуполь нескорений} – художник із міста
Слов'янськ Кутана Андрій Федорович.

Andre Kutana – сильний, виразний, зрілий майстер натюрморту і пейзажу. Він
віртуозно володіє пензлем, тонко відчуває кольори, точно передає стани природи.
В його картинах – тонка лірика пейзажів Приазов'я.

✅️Маріуполь – рідне місто художника, там він народився у 1962 році, навчався у
металургійному інституті, але пізніше переїхав до Слов'янська. Саме у
Слов'янську починається новий етап у житті Андрія Федоровича – творчий. Він
знайомиться з донецькими митцями і, дуже скоро, перетворює у реальність свою
мрію – стає художником, учасником багатьох культурних проєктів і значимих
виставок.

🖼🎨2014 рік вплинув на життя Андрія Федоровича, він виходить зі складу
Донецького творчого об'єднання художників, стає членом Маріупольської
організації НСХУ. Саме у цей час у Маріупольському художньому музеї ім. Куїнджі
проходять дві персональні виставки майстра, його картини беруть участь у
Міжнародних фестивалях мистецтв \enquote{Меморіал А. І. Куїнджі}, він дарує музею свої
картини.

😪24 лютого 2022 року перевернуло і його життя. Рятуючи свою родину, він зміг
виїхати зі Слов’янська, але весь творчий спадок майстра залишився в місті, яке
опинилося в епіцентрі бойових дій. Треба було думати, як рятувати картини. І
тут на допомогу прийшли волонтери. Влітку 2022 року їм вдалося вивезти з
Донеччини до Дніпра твори багатьох художників, і тим самим врятувати їх. Але і
Дніпро не знаходиться у безпеці. Тож, завдяки небайдужим людям, врятовані твори
сучасних художників Донбасу попрямували в інші безпечніші регіони України.

У середині серпня 2022 року полотна Кутана Андрія Федоровича приймали участь у
дуже знаковій виставці \enquote{Схід. Рівень свободи}, що проходила на Хмельниччині.
Глядачам запам'яталась картина Андрія Федоровича \enquote{Святогірська Лавра}, та сама
лавра, так жорстоко обстрілювана ворогом.

Андрій Федорович не припинив творчої діяльності, він активно працює. Зараз
картини Андрія Кутана беруть участь у мегапроєкті \enquote{Маріуполь нескорений}. Це
роботи, створені за останні місяці. В них і біль, і смуток за втраченим мирним
життям, за рідним Маріуполем, який ворог майже знищив. Це трагічні міські
пейзажі воєнного часу \enquote{Зруйноване життя (Маріуполь)} та
\enquote{Будинок мого дитинства (Маріуполь)}.

🗓 Ці та інші картини художників можна буде побачити на виставці, яка
відкривається у київський Галереї мистецтв \enquote{Лавра} 21 серпня поточного року.

📝Біографічна довідка:

Андрій Кутана народився у м. Маріуполі у 1962 році. У1984 р. закінчив
Жданівський металургійний інститут, за направленням переїжджає до Слов'янська.
2004 р. – перша персональна виставка художника. 2013 р. – прийнятий до членів
Національної Спілки художників України. Працює у галузі станкового живопису,
акварелі, графіки. Член творчого об'єднання художників і народних умільців
\enquote{Натхнення}. Член лігі діячів грецької культури \enquote{Галатея}. Учасник
Всеукраїнських і міжнародних виставок. У 2022-23 рр. провів персональні
виставки у м. Коломиї та у художній школі м. Хмельницького, брав участь у
пленерах та творчих зустрічах українських митців. У листопаді 2022 року на
запрошення національної Галереї Мистецтв Захента (Варшава, Польща) працював у
музеї сучасного мистецтва Варшави, активно сприяв популяризації українського
мистецтва у Польщі. Картини художника зберігаються в художніх фондах музеїв
Слов'янська, Краматорська, Маріуполя, в приватних колекціях України та
зарубіжжя.

Департамент культурно-громадського розвитку Маріупольської міської ради\par
Diana Tryma\par
Галерея мистецтв \enquote{Лавра}\par
Andre Kutana\par
Костянтин Чернявський\par
КУ \enquote{Маріупольський краєзнавчий музей}\par
ЯМаріуполь\par
Місто Марії\par
\#Маріупольнескорений \#виставка \#культурнадеокупація \#Маріуполь \#Київ \#художники\par
