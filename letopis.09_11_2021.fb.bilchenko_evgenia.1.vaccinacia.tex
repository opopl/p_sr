% vim: keymap=russian-jcukenwin
%%beginhead 
 
%%file 09_11_2021.fb.bilchenko_evgenia.1.vaccinacia
%%parent 09_11_2021
 
%%url https://www.facebook.com/yevzhik/posts/4422036117831464
 
%%author_id bilchenko_evgenia
%%date 
 
%%tags bilchenko_evgenia,covid,vaccinacia
%%title БЖ. Давайте опять про вакцинацию?
 
%%endhead 
 
\subsection{БЖ. Давайте опять про вакцинацию?}
\label{sec:09_11_2021.fb.bilchenko_evgenia.1.vaccinacia}
 
\Purl{https://www.facebook.com/yevzhik/posts/4422036117831464}
\ifcmt
 author_begin
   author_id bilchenko_evgenia
 author_end
\fi

БЖ. Давайте опять про вакцинацию? 

Беда у меня, духовные сестры и братья, прямо вот не детская. Не могу привиться
и не могу не привиться. Ниже я писала, что не признаю противостояние ваксеров и
антиваксеров в постапокалиптическом мире, как не признаю конфликта красных и
белых в истории земли русской, ибо дед был красным украинским военным
пролетариев, а бабушка - белой русской цивильной дворяночкой. Но, если красное
и белое для меня, - как две розы в одном дивном саду, то ваксеры и антиваксеры,
- как два пациента в одной палате уважаемого господина Илона Маска. Они
предусмотрены. Или как пел один рэпер: "Люди делятся на два типа, мне сегодня
наплевать на оба".

\ifcmt
  ig https://scontent-lga3-1.xx.fbcdn.net/v/t1.6435-9/254687752_4422036037831472_8496886510773447139_n.jpg?_nc_cat=105&ccb=1-5&_nc_sid=8bfeb9&_nc_ohc=ckUdfKOLcQcAX_SYmMQ&_nc_ht=scontent-lga3-1.xx&oh=93aac9520d8629d3450cda01fed1e9ce&oe=61B14507
  @width 0.4
  %@wrap \parpic[r]
  @wrap \InsertBoxR{0}
\fi

Потому чисто политическую сторону вакцины я хочу убрать совсем и навсегда: я
все знаю про планы Форрестера и Медоуза, про стратегии Бжезински и Тоффлера,
про всех драгоценных теоретиков золотого миллиарда, разрежающих мир, и даже
читала на английском ту самую скандальную статью в медицинском издании Scopus,
на которую ссылался господин Осташко, но я за блогерами, особенно нервными, как
я сама, проверять привыкла: да, все real, разработка медиков США и Уханя, 2015
год, закрытая лаборатория, скрещение вирусной пневмонии с летучей мышкой.
Вообще, я крутой антиглобалист, за это и уволена из альма-мачехи украинской,
потому книгу "Сентиментальное насилие либерализма: от шока к китчу" в лавках
Питера, Москвы и даже Киева сыскивайте, там про пандемию - немного тоже. 

Я также продолжаю почитать теорию "одной рукой притопи - второй брось
спасательный круг", которую господин Жижек, пока ещё он не начал эко-закос под
new lefts, весьма смело показал на примере филантропии со стороны ядра
голодающим и измождённый гражданской войной странам периферии, сначала
превращенным в колониальные дурдома метрополии, а потом щедро одаренным
национал-либеральными психоделиками и прочими фосфорицирующими во тьме
факелками. 

Но оставим это за скобками. Случайная или специальная, опасность вируса УЖЕ
ТУПО И ДАВНО ЕСТЬ, и отрицать ее - такая же истерия, как гипертрофировать.
Подсознательно ваксер полагает, что смерть обойдется без него. Он лишает себя
права наслаждаться последней формой жизни - смертью. Антиваксер же - это
малолетний ребенок, искренно верящий, что всё решит большой дядя, потому я
такой, без маски, яху. Но Сталина уже нет, никто не придет и не сделает
прививку насильно. 

Хотя я была бы (Машенька, закрой уши) не против. Выбор без выбора (это и есть
выбор между короной и вакциной) так изнуряет квазисвободой, что очень хочется
из этой извращённой формы насилия выйти в пространство прямого честного
пенделя: чтобы прямо домой пришел врач, похожий на моего деда, закатал мне
рукав, сделал укол и потом сидел с мной, ругаясь, не полчаса, а полдня, и чтобы
потом ему звонить по любой фигне фобической, как бате родному, и ночью тоже, и
бесплатно. Вот тогда у меня бы не было этой муки сомнения... Ибо: "бегство от
свободы", Эрих Фромм, уход от Реального, Жак Лакан, и всё такое. 

А вот моё дражайшее киевское НИИ сообщило, что должна у меня быть прививка,
естественно, файзером или чем-то, на Украине признанным, чего я не хочу (дело
уже не в антиглобализме, а в том, что на мне всегда кончаются котлеты, и тот
файзер, что не из азота, а из холодильника как раз на мне кончится). Иначе я
вылетаю. Причем, до 8 ноября, то есть, все, конец. Это, Слава Богу, что я взяла
отпуск.

Плюс у меня ещё одна фигня: хронический этот сепсис у меня по трем системам,
который можно вылечить, но не на Украине, и за большие деньги. Деньги нужны
везде на нормальные обследования. А их и нет. При таком состоянии, когда я вне
инфекции, живущей у меня внутри (я не заразная), нахожусь неделю за месяц
(остальные три недели я труп, тому есть свидетели), непонятно как прививаться
(сказал врач киевского Медикома и стих на слове "потом", причем "стих" - это
глагол). В этом сепсисе мне одинаково опасны и вакцина, и корона.

Я выбрала вакцину, я реально не антиваксер, ну, убейте меня: мне нужен этот
новый гидазепам. Я, внучка заслуженных советских врачей, ни разу ни от чего не
прививалась: мои бабуля с дедом меня оберегали, потому что я с детства
выкидывала кренделя с аллергиями и прочими золотухами. Но я выбрала Спутник,
ибо: он не западный, он - лекарство. Это рациональный медицинский выбор, я могу
его обосновать без высоких патриотических чувств, хотя и "Сережа тоже". Но вряд
ли мой украинский НИИ это возлюбит. Тут такой прикол: без вакцины я все время
делаю ПЦР и боюсь, вакцины тоже боюсь, ибо болею, болея, делаю ПЦР, замкнутый
круг, их, ПЦР, количество у меня - два десятка за год. Теперь это уже шиза,
наверное. В Медикоме меня с дверей узнают. 

Вот вопрос: кто за вакцину, кто за Спутник, кто по болезни не может привиться и
кого все равно уволят: если не за отсутствие вакцинации, то за выбор именно
данной вакцины? Так, лес рук не поднялся, ок. Кстати: Спутник есть на Украине.
Надо знать места и людей. Везде все есть. Роман о Есенине Захара ваш раб
покорный получал УкрПочтой из Ленинградского рок-клуба, на что великий писатель
спросил что-то типа: "А чё так можно было?"

В общем, если я пущу-таки по венам Спутник, предварительно ещё раз отмучив
докторов своей несчастной болячкой, эффект будет тот же, что был у нас в НИИ,
когда я прислала клип Елизарова "Сталинский костюм" на адрес ученого совета в
четыре утра, - ну, люблю я шутить невинно, вообще, что с меня взять? 

Итак, люди, без нервов, оставим угрозы на работе и политику: кто относится к
прививке как к гидазепаму? И кто не может привиться, потому что болеет все
время, как дурак? Есть такие уникумы, или через ж... Только у БЖ? 

Не болейте. Или болейте смиренно: тельцу не прикажешь. Душа летит - тельце
тормозит.

\ii{09_11_2021.fb.bilchenko_evgenia.1.vaccinacia.cmt}
