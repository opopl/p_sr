% vim: keymap=russian-jcukenwin
%%beginhead 
 
%%file 03_04_2023.fb.korovanenkova_natalia.mariupol.1.31_03_2022_blok_post.cmt
%%parent 03_04_2023.fb.korovanenkova_natalia.mariupol.1.31_03_2022_blok_post
 
%%url 
 
%%author_id 
%%date 
 
%%tags 
%%title 
 
%%endhead 

\qqSecCmt

\iusr{Mariya Balalova}
🙏🙏💔

\iusr{Nadezhda Gorchakova}

Бердянськ и для нас был другой планетой!

\begin{itemize} % {
\iusr{Natali Korovanenkova}
\textbf{Надежда Горчакова} так, меньше чем 100 км.. почему за что нас так(( или кто так решил....вопросов у всех много

\iusr{Oxana Erofeeva}
\textbf{Natali} так даже в Володарске, Мангуше била совсем другая жизнь. Єто 20 км максимум от Мариуполя((((

\iusr{Natali Korovanenkova}
\textbf{Оксана} и Урзуфе и Косе

\iusr{Nadezhda Gorchakova}
\textbf{Натали Корованенкова} 

ми ж як раз у Володарського районі і були. Звичайно краще, але у нас свої
жахіття були. Але не порівняти з Маріуполем 😭😭😭

\iusr{Natali Korovanenkova}
\textbf{Оксана Ерофеева} 

в конце марта приехала соседка с Урзуфа, мы только первый день вышли с подвала,
ее двое детей были с нами в подвале все это время, так сталось( она приехала на
машине чистая и пахнущая, увидев нас она не могла поверить, ее свекровь была
еще в подвале она спустилась к ней, у нее был шок от увиденного, она плакала и
не могла произнести ни слова, потом сказала что в Урзуфе есть свет и вода и
продукты .. и они думали, что у нас так же(


\iusr{Oxana Erofeeva}
\textbf{Natali Korovanenkova} 

мои соседки после подвала смогли виехать в Никольское. И били там больше
месяца, т.к. не могли без фильтрации вернуться. Они приехали как из другой
жизни. Хотя бомбежки, подвал, больницу ми пережили вместе. Там бил свет, вода,
еда, связь

\iusr{Natali Korovanenkova}
\textbf{Оксана Ерофеева} отож

\end{itemize} % }

\iusr{Mariy Dubinskay}

Я помню... ты пропала.. смс не доходили - уведомление, что смс дошло было
единственным знаком, что Вы живы... Мы с девочками на связи и каждая шлёт смс..
и все ждём... и у кого прошла смс - сообщает всем и мы радуемся... а тут
тишина... у меня был приём, и тут звонок и твоё имя!!! Плохо слышно, но я
практически догадываюсь что ты говоришь. А главное - ТЫ ГОВОРИШЬ!! Это был
единственный раз за войну, когда я плакала... Вы живы.

\begin{itemize} % {
\iusr{Natali Korovanenkova}
\textbf{Мария Дубинская} обіймаю💕
\end{itemize} % }

\iusr{Karik Valentyna}

Господи, Наташа, мы с вами не знакомы, но я слежу всё время за вами, переживаю
жутко, а сейчас читаю ваши записи - что вы пережили - рыдаю... как сама рядом с
вами... Господи, спаси и сохрани вас🙏 даже представить всё это страшно, а вы это
всё прочувствовали на себе. Не умею писать много), но сердце просто на куски(((

\begin{itemize} % {
\iusr{Анна Филина}
\textbf{Karik Valentyna} вот и я знакома с Наташей только заочно, но этот год слежу и переживаю как за родного человека. Сначала Мариуполь, теперь Никополь.🙏И каждый её пост слëзы.

\iusr{Natali Korovanenkova}
\textbf{Анна Филина} плакати не треба, ми сильні та мужні, це вже безперечно💞обіймаю
\end{itemize} % }

\iusr{Larisa Sinchuk}

Наталка, коли читаю, відбирає мову. Плачу. Суцільний біль...

\iusr{Tatyana Tirsenko}

Наталочка, я только вчера хотела спросить у тебя, а где же продолжение истории
вашего выезда. Я, как тот мазохист, жду с нетерпением продолжение вашей
истории, читаю и реву, переживая все с вами заново, как-будто это было вчера.
😢 Каждое ваше слово, каждая мысль, похожи на наши. Наверное, все, выезжающие в
то время из Мариуполя, переживали одни и те же эмоции. Мы приехали в Бердянск
17 марта ночью, а выехали из Мариуполя в 10 утра. А 18 марта у меня был юбилей,
50 лет. Это был действительно день рождения, как-будто заново... Ты понимаешь о
чем я конечно же ❤️

\begin{itemize} % {
\iusr{Natali Korovanenkova}
\textbf{Татьяна Тырсенко} ми все кто вижил понимаем друг друга с полуслова, не просто понимаем о чем, ми єто чувтвуем 💔

Мне сегодня написал в приват человек, которий жил в Доме с Часами, пару слов,
но этот дом для каждого из нас, часть нас, кто не делал фото в Сказке, хотя бы
раз в жизни💔 так больно..

\end{itemize} % }

\iusr{Mila Pilguy}

Як згадаю, коли ти \enquote{пропала}..., я тоді теж вже виїхала з Харкова...

Коли дізналася що ти жива! Я кричала від щастя! До цього я взагалі ніяких
емоцій не відчувала..., якась пуста безодня була всередині. Не повіриш, після
розмови з тобою, з'явилася надія. Не можу пояснити чому саме тоді... Але була
надзадача вивезти тебе з окупованої території.... Як це складно, знають тільки
ті, хто пройшов через це. Скільки часів тривали перемовини з волонтерами які
вивозили людей з окупованих територій, як їм важко було дозвонитися, їхні
історії про те Пекло, що там коїться, скільки їх взяли у полон, про їх розбиті
автівки, про все. Я їх всіх вислуховувала, бо їм треба було виговорюватися у
вільні вуха про весь той біль. Але тоді, коли мені треба, в них не було поїздок
або все було під зав'язку. Хочу сказати, що вони всі мені перезвонили/написали
що зможуть вивезти вас, але ти вже виїхала на той час, Слава Богу 🙏 .

\begin{itemize} % {
\iusr{Natali Korovanenkova}
\textbf{Mila Pilguy} 

ти завжди була на зв'язку, коли його не було, не було зовсім, це було загадкою,
як?🤔пам'ятаєшь скільки ми про це говорили?🥰

\iusr{Mila Pilguy}
\textbf{Natali Korovanenkova} 

так пам'ятаю, моя рідна! Все пам'ятаю... Озираючись назад, все думаю, як ми то
все пережили, як ти все це пережила? А тоді не замислювалися, просто діяли, бо
не було іншого виходу. Тупити то не наше 😉. Зараз я точно знаю, з ким можна
йти у розвідку і хто справжній 💪💛💙

\iusr{Natali Korovanenkova}
\textbf{Mila Pilguy} це так

\end{itemize} % }

\iusr{Ирина Нотич}
🙏🙏🙏❤🐾🐾


\iusr{Наталья Клименко}
Реву((


\iusr{Michael Mvd}

тебе обязательно надо свести все твои рассказы в одном месте. книгу вряд ли
выпустишь, но эти воспоминания надо хранить. позже размываются мелочи из
которых складывалось существование.

и обязательно копируй на несколько носителей.

я думаю, что многие сейчас пишут. многие не пишут. многие собираются написать
все позже, но так и не запишут. а может получится потом тебе сделать книгу
экземпляров 50.. только не заморачивайся с редактурой.

редакторы вместе с ошибками убирают все краски пережитого.

довелось как-то проверить на себе)))

\begin{itemize} % {
\iusr{Natali Korovanenkova}
\textbf{Michael Mvd} спасибо Миш😘дякую
\end{itemize} % }

\iusr{Elena Nazarenko}

Господи, какое счастье, что вы выбрались! Без слёз невозможно читать...

\iusr{Наталья Чернышёва}

Читаю ваши записки, у меня каждый раз мурашки по телу и слезы на глазах!😢
