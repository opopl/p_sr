% vim: keymap=russian-jcukenwin
%%beginhead 
 
%%file 31_12_2021.fb.miheev_vladislav.1.ostavatjsja_ljudmi
%%parent 31_12_2021
 
%%url https://www.facebook.com/vladislav.mikheev.5/posts/4810027025730701
 
%%author_id miheev_vladislav
%%date 
 
%%tags chelovechnost,chelovek,novyj_god,obschestvo
%%title Оставаться Людьми
 
%%endhead 
 
\subsection{Оставаться Людьми}
\label{sec:31_12_2021.fb.miheev_vladislav.1.ostavatjsja_ljudmi}
 
\Purl{https://www.facebook.com/vladislav.mikheev.5/posts/4810027025730701}
\ifcmt
 author_begin
   author_id miheev_vladislav
 author_end
\fi

Прогнозы на следующий год и итоги года уходящего, как правило, пытаются
схватить глобальные тренды и знаковые события. От прорыва с искусственным
интеллектом до скумбрии по восемь.

Но в какую бы точку ни привёл нас социально-исторический процесс, важно лишь
одно - останемся ли мы в этой точке людьми?

Не функциями  государственной системы,  не цифровыми алгоритмами матрицы, не
борцами за \enquote{свои} убеждения, монетизированные другими...

Для меня событиями года стали встречи с людьми, которые остаются ... людьми.

Вот женщина, которая едет за тридевять земель вымаливать у святого ребёнка. Не
для себя -  для бездетной подруги. И вымаливает. При этом беременеют обе
одновременно. )

Удивить могут не только мистические, но и вполне бытовые вещи...

Кассир, которая помогает снять деньги в банке:  мне очень нужно - много и
быстро! Девочка-кассир, что называется, светится изнутри. По глазам видно,
хорошая, искренняя, добрая девочка.  Помню, точно такие же глаза были у двух
сестер - монахинь 15 и 16 лет, которых мне довелось видеть в одном из
монастырей...  Конечно, девочка-кассир никакая не святая. Но в том аду, которым
стал Приватбанк после национализации, ее неформальное, клиентоориентированное
поведение граничит со святостью.

Офицер на паспортном контроле в аэропорту,  не  мордующий людей соблюдением
инструкций. Да, такое бывает: телефон садится прямо у паспортного окна, Дiю
открыть невозможно. \enquote{Ладно, проходите}, -  нормальный поступок для человека и
противоестественный для адского винтика.

Чиновник социальной службы, который идёт против начальства, формальных
предписаний и равнодушия системы, добивается прекращения насилия над детьми в
приёмной семье. Идет до конца, отстаивая интересы чужих детей. Да, в нашем
социально-бюрократическом аду и такое бывает.

На все вышеперечисленное искусственный интеллект, конечно же, не способен.
Стать катализатором социального-бюрократического ада - это запросто, хотя
человек и сам не плохо с этим справляется. А вот отменить, руководствуясь
эмпатией, идиотскую норму - нет! Это даже не всем носителям естественного
интеллекта доступно.

Способен ли искусственный интеллект понять ваши молитвы?  Или помолиться за
вас? Вот здесь, мне кажется, и пролегает граница между ИИ и человеком -
существом в целом предсказуемым даже в самой своей непредсказуемости...

А тем временем формализованные нормы современного человечества становятся все
более идиотскими. 

Знаковое событие  произошло в Британии. Солдат Её Величества сменил в
результате хирургической  операции пол на женский. После чего этот
гей-трансгендер принял ислам. Некоторые, ортодоксально настроенные люди,
восприняли его поступок как личное оскорбление.

Хотя какая вам, в сущности, разница?

Главное же, что человек вакцинирован дважды от ковида, платит налоги и готов
защищать Европу от Путина. Формально он может быть достоин самой высокой
оценки системы социального рейтинга. Верно?

Хуже другое -  когда лишенный эмпатии человек становится морально-нравственным
\enquote{трансгендером}. 

Когда статус, инструкция, убеждение совершают противоестественную операцию над
его человечностью. 

Желаю всем нам в наступающем году сохранить нормальную человеческую ориентацию.

Ну, и раз уж мы вступаем в год Тигра, всем  - хорошей охоты!

В том числе и на скумбрию по восемь.  @igg{fbicon.smile} 

\ii{31_12_2021.fb.miheev_vladislav.1.ostavatjsja_ljudmi.cmt}
