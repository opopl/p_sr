% vim: keymap=russian-jcukenwin
%%beginhead 
 
%%file 14_09_2022.stz.news.ua.donbas24.1.student_mdu_oleksandr_denisov_14_medalej.txt
%%parent 14_09_2022.stz.news.ua.donbas24.1.student_mdu_oleksandr_denisov_14_medalej
 
%%url 
 
%%author_id 
%%date 
 
%%tags 
%%title 
 
%%endhead 

Ольга Демідко (Маріуполь)
14_09_2022.olga_demidko.donbas24.student_mdu_oleksandr_denisov_14_medalej
Маріуполь,Україна,Мариуполь,Украина,Mariupol,Ukraine,Mariupol.MSU,Маріуполь.МДУ,Мариуполь.МГУ,Sport,Спорт,Плавання,Німеччина,date.14_09_2022

Студент МДУ Олександр Денисов завоював 14 медалей на двох чемпіонатах Німеччини

Студент Маріупольського державного університету блискуче виступив на змаганнях
з плавання у Німеччині

Про успішний виступ Олександра Денисова повідомила пресслужба Маріупольського
державного університету.

Хлопець навчається на ІІ курсі та здобуває освіту одразу за двома
спеціальностями — «Середня освіта. Фізична культура» та «Економіка». Він почав
займатися плаванням з семи років, а на чемпіонатах подібного рівня виступає з
11. Важливо, що війна не завадила Олександру повернутися до змагань, при цьому
він виступив навіть на більш високому рівні, ніж раніше.

«Повернутися на доріжки було не важко. Я ніколи не зупинявся тренуватися.
Змагання залишили чудові враження. Німеччина докладає багато зусиль, щоб
українські спортсмени почували себе добре і не втрачали сили духу», — розповів
Олександр Денисов.

Читайте також: Маріупольський університет активно допомагає постраждалим
маріупольцям

Олександр успішно виступив на двох чемпіонатах Німеччини з плавання — Баварії
та Середньої Франконії. Обидва чемпіонати проходили між спортсменами Німеччини
та України. Попри те, що німецькі плавці мали кількісну перевагу, українці
стали справжніми підкорювачами блакитних доріжок завдяки якісній підготовці та
справжній майстерності. За результатами змагань Олександру Денисову вдалося 14
разів підніматися на п'єдестал пошани. За успішне подолання дистанцій у Баварії
Олександр отримав шість медалей: чотири за перше місце (дві перемоги здобуті
командно в естафетному плаванні), по одній за друге і третє місця. На змаганнях
Середньої Франконії маріупольський спортсмен блискуче обійшов суперників і
здобув перемогу вісім разів у двох вікових категоріях — своїй (2002−2004) та
абсолютній (серед усіх учасників змагань).

До речі, торік Олександр виступав у місті Шаморін, де проходили змагання з
плавання Slovakia swimming cup, які об'єднали спортсменів Чехії, Австрії,
Польщі, Угорщини, Словаччини та України. Під час змагань хлопецю вдалося
вибороти дві бронзові медалі на дистанціях 400 метрів комплексним плаванням і
800 метрів вільним стилем.

Читайте також: Як під час війни працює Маріупольський державрний університет?

Ще більше новин та найактуальніша інформація про Донецьку та Луганську області
в нашому телеграм-каналі Донбас24.

ФОТО: з сайту mdu.in.ua
