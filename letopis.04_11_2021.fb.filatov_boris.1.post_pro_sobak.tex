% vim: keymap=russian-jcukenwin
%%beginhead 
 
%%file 04_11_2021.fb.filatov_boris.1.post_pro_sobak
%%parent 04_11_2021
 
%%url https://www.facebook.com/permalink.php?story_fbid=4503552843059911&id=100002157183088
 
%%author_id filatov_boris
%%date 
 
%%tags 1957,istoria,kosmos,planeta,sobaka,sssr,zemlja.planeta,zhivotnoje
%%title Этот мой текст по собак и космос почти никто не читал
 
%%endhead 
 
\subsection{Этот мой текст по собак и космос почти никто не читал}
\label{sec:04_11_2021.fb.filatov_boris.1.post_pro_sobak}
 
\Purl{https://www.facebook.com/permalink.php?story_fbid=4503552843059911&id=100002157183088}
\ifcmt
 author_begin
   author_id filatov_boris
 author_end
\fi

\obeycr
Этот мой текст по собак и космос почти никто не читал.
...
Он написан 8 лет назад, в те счастливые времена, когда я ещё не был мэром и у меня ещё не было под 300 тысяч фолловеров. 
И поэтому у этого текста всего 31 лайк))
Прочитайте его сегодня. 
Нет уже ни Малыша, ни Руслана-Оленя- Паши, ни Егора, ни Куклы, ни двух терьеров :((
Но собак у меня все равно семь:))
Не проходите мимо брошенного щенка.
\restorecr

\ii{04_11_2021.fb.filatov_boris.1.post_pro_sobak.pic.1}

\obeycr
...
«Инопланетяне рядом.
...
Обожаю собак. Любых. Маленьких и огромных, породистых и не очень, домашних и бездомных. Всяких.
Их у меня семь. Они живут по всем моим домам и дачам. И не я выбираю их, а они выбирают меня...
Малыш с зубами как у крокодила средних размеров. Жена нашла его на улице с перебитым бедром и привезла грязного и дрожащего на дачу в салоне своего ослепительно белого "Порше". Малыш так и не стал любить людей, но когда видит жену, начинает трястись от счастья...
Любимец всей семьи Егор. В одно счастливое утро он пришел ниоткуда.
Странный бедолага с тройным именем Олень-Руслан-Паша. Откликается на все три клички. Никто уже не помнит, почему он стал Оленем и Русланом, а вот третьего имени он был удостоен охранниками в честь командира спецназа...
Паскуда Кукла. Она уходит и приходит раз в полгода. И ее невозможно удержать ни будкой, ни колбасой, ни добрым словом.
Ну и конечно три йоркширских терьера. У них своя стая и иногда мне кажется, что они включили в нее нас, хозяев, по очень большому снисхождению.
...
Вчера, 3 ноября, был знаменательный день. Именно 3 ноября 1957 года в первый раз в истории человечества живое существо облетело Землю в космосе. И это была собака. 
Можно встретить утверждение, что это был первый полет в космос живого существа, но это НЕПРАВДА. ЕЩЕ в 1949 году американцы запустили в суборбитальный полет на высоту 133 километра (!) обезьяну. 
Правда ракета у американцев была краденная. 
Знаменитая ФАУ-2.
Если коммунисты отбирали «смельчаков» в полет, в первую очередь, по рабоче-крестьянскому происхождению, то к животным они были еще более немилосердны. 
Обезьяны коммунистов не устраивали, потому что были неусидчивые и плохо поддавались дрессировке. 
Поэтому коммунисты остановились на собаках.
Собак отбирали из питомника для бездомных животных. Основное требование, чтобы собака была белого цвета, потому что «так она будет лучше выглядеть в кадре». 
Затем отобранных животных белого цвета стали "выбраковывать" тренировками на центрифугах, вибростендах и барокамерах. 
Остановились на трех: Альбине, Лайке и Мухе. Муха не подошла из-за небольшой кривизны лап (этот «дефект» выглядел в кадре нефотогенично). Ее сделали "технологической собакой" и тестировали на ней работу различных систем и аппаратуры.
Альбину пожалели – она в тот момент ждала потомство, поэтому ее записали в «дублеры».
Лететь пришлось Лайке. 
На момент старта Лайке было около 2 лет. Она погибла после 4 витков вокруг Земли. Сгорела заживо от перегрева, так как в космическом аппарате отсутствовала система терморегулирования. 
Предполагалось, что Лайка проживет около недели, но в любом случае она была обречена. Это был билет в один конец... 
...
Газета «Нью Йорк Таймс» назвала Лайку «самой одинокой и несчастной собакой в мире». Американцы имели право так писать, потому что оборудовали свои космические аппараты с обезьянами парашютными системами. Мы же отправляли смертников.
Спутник с мертвой Лайкой совершил 2370 витков вокруг Земли и сгорел в атмосфере только спустя полгода - 14 апреля 1958 года.
Специальная комиссия из ЦК КПСС и Совета Министров СССР не поверила, что собака умерла из-за ошибки конструкторов и потребовала провести эксперименты с похожими условиями на Земле. 
В результате погибли еще две собаки.
Примечательно, что коммунисты врали всему миру (впрочем, как обычно), что Лайка пережила весь космический полет и потом ее усыпили прямо в космосе.
...
Обожаю собак. Парадокс в том, что еще несколько лет назад я их не очень любил.
Люди думают, что отправили собак в космос. Однако иногда мне кажется, что все наоборот. Это собаки первыми прилетели к нам из космоса. Мы упорно ищем инопланетян, а они ходят рядом по улицам.
Подумайте над этим, когда в следующий раз пройдете по улице мимо брошенного щенка...”
Borys Filatov, 2013(с)
\restorecr
