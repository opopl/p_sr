% vim: keymap=russian-jcukenwin
%%beginhead 
 
%%file slova.zhurnalist
%%parent slova
 
%%url 
 
%%author 
%%author_id 
%%author_url 
 
%%tags 
%%title 
 
%%endhead 
\chapter{Журналист}
\label{sec:slova.zhurnalist}

\emph{Журналистики} в Украине давно уже нет. И дело тут не только в том, что
все значимые СМИ принадлежат крупным олигархическим кланам, и поэтому их
\enquote{журналисты} это уже не \emph{журналисты}, а обычные пропагандисты,
манипуляторы и демагоги, работающие строго на политический заказ. Это участники
политической борьбы, отстаивающие за деньги интересы работодателя. Проблема
сложнее. По сути, \emph{журналистика}, как профессия, умирает по всему миру.
Просто в Украине она умирает быстрее. Точнее, она даже не родилась после
распада СССР. О чем идет речь?  \emph{Журналист} это в первую очередь
профессиональный исследователь. Он должен брать общественно значимые факты,
новости, проблемы, подробно исследовать их затем рассказывать обществу
результаты своих исследований,
\textbf{День журналистики: праздник есть, а журналистики уже нет},
Андрей Головачев, strana.ua, 06.05.2021

И о стандартах профессии помнят далеко не все...  Поздравляю коллег с \emph{Днем
Журналиста}, с учетом уровня свободы слова на Украине, желаю Вам друзья хороших
адвокатов, принципиальных следователей, порядочных прокуроров и справедливых
судей. Чтоб исполнение преступных приказов с \enquote{высоких кабинетов} в отношении
адекватных \emph{журналистов}, на практике превращалось в фарс.  По состоянию на 2021
год, \emph{журналистское сообщество Украины} все также расколото. По одну \enquote{сторону
баррикад} те, кто способствуют продолжению вооруженного конфликта на Донбассе,
сеют рознь и ненависть по языковому и религиозному признаку, пытаясь доказать
украинцам, что \enquote{мы воюем с Россией}, а не \enquote{нами воюют}. Они считают себя
\enquote{информационными солдатами}, воюющими на стороне \enquote{Западного мира} в лице США,
полагают что врать украинцам для победы США над Россией приемлемо и
необходимо, а внешнее управление Украиной, грантовские деньги и коррупция -
это благо,
\citTitle{Журналистское сообщество Украины, к сожалению, остается расколотым}, 
Дмитрий Василец, strana.ua, 06.06.2021

По другую \enquote{сторону баррикад}, \emph{журналисты} читавшие журналистские стандарты, в
которых четко написано: \emph{журналист} своей работой ДОЛЖЕН способствовать миру,
быть объективным и освещать ситуацию с разных сторон конфликта. Которые
категорически против использования украинского народа в противостоянии
сверхдержав, в виде \enquote{пушечного мяса}, под лозунги популистов \enquote{Украина щит
Европы от России}. Такие \emph{журналисты} говорят сильно горькую правду для власти, о
том, что \enquote{нами воюют}, вскрывая выгодополучателей конфликта и рассказывая про
\enquote{прелести} внешнего управления Украиной, в виде коррупции, мародерства и
экономического геноцида.  Надеюсь, в будущем все больше \emph{журналистов} будут
читать \emph{журналистские стандарты} и работать для сохранения мира, во благо
украинского народа, наперекор \enquote{грантодателям} и олигархам.  Для этого,
рекомендую коллегам все больше обращать внимание на возможности, которые дают
современные интернет платформы, как по донесению информации людям, так и по
монетизации,
\citTitle{Журналистское сообщество Украины, к сожалению, остается расколотым}, 
Дмитрий Василец, strana.ua, 06.06.2021

В \emph{День журналистов} хочу адресовать пожелание власти. Желаю вам не
избегать неудобных и сложных медийщиков, а стремиться понять нас и научиться
говорить с нами. Как из той известной фразы о том, как нужно заводить друзей.
Выбирайте тех, кто сильнее вас - будете мучаться, но расти.  С \emph{Днём
журналиста}, коллеги и пострадавшие!, 
\textbf{В День журналиста власти важно понять, какие СМИ ей нужны},
Светлана Крюкова, strana.ua, 06.05.2021

%%%cit
%%%cit_head
%%%cit_pic
\ifcmt
  pic https://img.strana.ua/img/article/3186/tkachenko-anonsiroval-zapusk-39_main.jpeg
	caption Сейчас идет набор сотрудников в новую структуру. Фото: Лига 
\fi
%%%cit_text
По словам Ткаченко, запуск нового органа он анонсировал в эфире телепрограммы
\enquote{Право на власть} на канале 1+1.  \enquote{Темой эфира, в частности,
была так называемая \enquote{Пятая колонна}, а именно СМИ, которые работают в
Украине, но против Украины, используя российские нарративы.  Работа таких
людей, называющих себя \enquote{журналистами}, \enquote{блогерами},
\enquote{лидерами общественного мнения} на самом деле ничего общего с
журналистикой не имеет. Это лишь инструмент для вбрасывания фейков и
дезинформации, замаскированный под СМИ}, - пишет чиновник
%%%cit_comment
%%%cit_title
\citTitle{Ткаченко анонсировал запуск Центра противодействия дезинформации}, 
, strana.ua, 19.02.2021
%%%endcit


%%%cit
%%%cit_head
%%%cit_pic
%%%cit_text
Опять получается, что есть титульные языки, а есть языки недолюдей.
\emph{Титульная журналистка}.  Вот как оно. Есть титульные языки, есть языки
недолюдей. Вот так мыслит \emph{журналист} «Прямого» (подсказали, что уже ушел
с \enquote{Громадського}). Вопрос, какое может быть общее благо с таким ресстровым
порохоботом, который дискриминирует огромную часть населения Украины, живя за
ее счёт. Да с BBC такое существо выперли бы за сутки с волчьим билетом на всю
жизнь. Поэтому Дацюк был абсолютно прав - такое поведение обществом должно
караться и преследоваться
%%%cit_comment
%%%cit_title
\citTitle{Великие победы не имеют значения, главное - на каком языке говорит тренер / Лента соцсетей / Страна}, Юрий Романенко, strana.ua, 30.06.2021
%%%endcit

%%%cit
%%%cit_head
%%%cit_pic
\ifcmt
  pic https://strana.ua/img/forall/u/10/91/photo_2021-07-01_14-26-32_(2).jpg
  width 0.6
\fi
%%%cit_text
Во время общения с \emph{журналистом} Сергеем Черных Зеленский высказал мнение, что русские и украинцы - один народ.  
\enquote{...Мы в принципе не можем быть против русского народа, потому что мы один
народ. Проблемы правительства России и его отношения, в силу различных
обстоятельств, к правительству Украины вылились на нас с вами. И наша
\enquote{квартальская} позиция – не за правительство Украины, а за украинский народ. И
не против русского народа, а против российского правительства. Как мы можем его
не любить? Мы же не идиоты!..}
%%%cit_comment
%%%cit_title
\citTitle{Зеленский о русских и русском языке. Что будущий президент говорил в 2014 году}, 
Екатерина Терехова, strana.ua, 03.07.2021
%%%endcit

