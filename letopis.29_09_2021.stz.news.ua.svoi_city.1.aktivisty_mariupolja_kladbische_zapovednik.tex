% vim: keymap=russian-jcukenwin
%%beginhead 
 
%%file 29_09_2021.stz.news.ua.svoi_city.1.aktivisty_mariupolja_kladbische_zapovednik
%%parent 29_09_2021
 
%%url https://svoi.city/articles/167061/mariupolskij-nekropol
 
%%author_id korotenko_pavlo.slovjansk.zhurnalist
%%date 
 
%%tags 2021,aktivist,mariupol,mariupol.pre_war,nekropol
%%title Активисты из Мариуполя расчищают старейшее кладбище востока. Хотят превратить его в заповедник
 
%%endhead 
 
\subsection{Активисты из Мариуполя расчищают старейшее кладбище востока. Хотят превратить его в заповедник}
\label{sec:29_09_2021.stz.news.ua.svoi_city.1.aktivisty_mariupolja_kladbische_zapovednik}
 
\Purl{https://svoi.city/articles/167061/mariupolskij-nekropol}
\ifcmt
 author_begin
   author_id korotenko_pavlo.slovjansk.zhurnalist
 author_end
\fi

\ii{29_09_2021.stz.news.ua.svoi_city.1.aktivisty_mariupolja_kladbische_zapovednik.pic.1}

\begin{qqnagolos}

С прошлого года активисты мариупольской  общественной организации
\enquote{Архи-город} начали расчистку заброшенного старогородского кладбища.
Локации присвоили название
\href{https://www.facebook.com/groups/278185963354519}{\enquote{Мариупольский
Некрополь}} и хотят сделать из него историко-культурный заповедник.
	
\end{qqnagolos}

\subsubsection{Сохранилось, но заросло}

Уникальность старогородского кладбища Мариуполя в том, что это одно из
старейших сохранившихся кладбищ на востоке Украины.

\enquote{\em Этому кладбищу, можно сказать, повезло. На момент создания, в 1811-м
году, оно находилось за несколько километров от города, и только в середине 20
века его охватила городская застройка. Сейчас кладбище расположено практически
в центре. В других индустриальных городах Донбасса на месте дореволюционных
кладбищ советская власть строила парки и стадионы. А наше кладбище под эту
волну не попало - на нем продолжали хоронить аж до 1972-го года}, —
рассказывает председатель общественной организации
\href{https://www.facebook.com/arximisto}{\enquote{Архи-город}} \textbf{Андрей Марусов}.

\ii{29_09_2021.stz.news.ua.svoi_city.1.aktivisty_mariupolja_kladbische_zapovednik.pic.2}

Последние десятилетия территория кладбища заросла дикой сиренью и мусором.
Охранный статус на его территории имеют только пять памятников - три могилы
Героев Советского Союза, а также памятники воинам Второй мировой войны и
подпольщикам. Хотя, по мнению краеведов, там сохранилось достаточно много до- и
послереволюционных надгробий, под которыми покоятся останки не менее достойных
памяти горожан.

\enquote{\em Охранный статус этому объекту обязательно нужно делать. Это хоть какая-то
гарантия, что здесь ничего не будут строить или перестраивать. Сейчас кладбище
числится просто как коммунальный объект, и за разрушение здешних архитектурных
сооружений грозит только административная ответственность. К тому же, это будет
автоматический сигнал для туристов о том, что это место имеет историческую
ценность. А мы ведь вроде претендуем на то, чтобы стать туристическим городом},
— считает Андрей Марусов.

В заповедник \enquote{Мариупольский некрополь} он и его коллеги предлагают также
включить старое еврейское кладбище, где тоже сохранилось достаточно много
ценных надгробий.

\subsubsection{Денег нет, но будем искать}

С инициативой создать на территории старогородского кладбища
культурно-исторический заповедник активисты обратились к руководству города.

\enquote{\em Мы им сказали - посмотрите, у нас в центре города 16 гектаров земли, которую
можно сделать привлекательной для горожан и туристов. Тем более, что кроме
объектов старины, там можно высадить деревья и сделать \enquote{легкие} Мариуполя. Мэру
идея понравилась, но все уперлось в финансы. Деньги искать пообещали, но честно
предупредили, что быстро это не произойдет. Мы и сами понимаем, что финансы там
нужны колоссальные, потому что простым благоустройством не обойдешься}, —
говорит краевед.

\ii{29_09_2021.stz.news.ua.svoi_city.1.aktivisty_mariupolja_kladbische_zapovednik.pic.3}

Чтобы не сидеть сложа руки, мариупольцы решили начать действовать
самостоятельно. Создали группу \enquote{Мариупольский Некрополь} (так между
собой назвали этот объект) и кинули клич на сбор средств и единомышленников,
чтобы начать исследовательские и восстановительные работы.

\enquote{\em Работать мы начали в марте-апреле прошлого года. На объявления в соцсетях
откликнулись человек 20-30: кто-то занимался исследованиями, кто-то работал
лопатой, пилой или топором. Инвентарь нам дали знакомые предприниматели, к
которым мы обратились с такой просьбой. Потом подключилась Федерация греческих
обществ. Мы решили заняться одной из самых старых аллей - расчистили ее,
привели в порядок за счет пожертвований мариупольцев 10 памятников, где не
требовалась работа реставратора, и начали высаживать вдоль аллеи платаны. Эти
деревья многие ассоциируют с Грецией, а кроме того, это один из самых древних
видов растений на планете, что тоже символично}, — рассказывает Андрей Марусов.

\ii{29_09_2021.stz.news.ua.svoi_city.1.aktivisty_mariupolja_kladbische_zapovednik.pic.4}

\subsubsection{История под слоем мусора}

Поскольку старогородское кладбище не археологический, а обычный коммунальный
объект, вести на нем раскопки запрещено, поэтому волонтеры только убирали дикую
поросль и очищали надгробья от мусора и наносов земли. 

\enquote{\em Мы понимаем, что у некоторых горожан наша работа может вызвать негативную
реакцию. У многих здесь похоронены родственники, поэтому мы все делаем
максимально открыто. После каждой акции публикуем в своей группе снимки участка
до и после уборки. Обязательно рассказываем сколько и на что потратили денег,
которые нам жертвуют меценаты. На сегодняшний день мариупольцы перечислили уже
около 10 тыс. грн}, — продолжает Андрей.

\ii{29_09_2021.stz.news.ua.svoi_city.1.aktivisty_mariupolja_kladbische_zapovednik.pic.5.uchastok}

По его словам, во время расчистки Хараджаевского участка кладбища активисты
сделали несколько ценных открытий.

\enquote{\em Самое старое захоронение, которое мы нашли - 1834-35 годы. Мы обнаружили
надгробную плиту Гавриила Сахаджи, первопоселенца, одного из основателей
города. Позже мы нашли могилы его сына Ивана и некой Анастасии Сахаджи,
родственную связь которой еще предстоит установить. Там расположены захоронения
5 старинных родов. Причем удивительно, что греческие, немецкие и славянские
могилы расположены вперемешку - это нетипично и мы пока не можем сказать,
почему хоронили именно так}, — признает Андрей.

\ii{29_09_2021.stz.news.ua.svoi_city.1.aktivisty_mariupolja_kladbische_zapovednik.pic.6.plita_sahadzhi}

\subsubsection{Идентификация и привязка}

Параллельно с работой по расчистке могил, волонтеры работают с архивами.
Вернее, с тем минимумом, который сохранился по старогородскому кладбищу.

\enquote{\em Мы изучаем метрические книги, пытаемся восстановить генеалогию похороненного
человека. Проблема в том, что нет никаких карт кладбища, ни советского периода,
ни уж тем более дореволюционного времени. Есть только кладбищенские книги и то
только начиная с 1943-го года, но в них нет привязки. Мы надеялись, что за
какие-то факты сможем зацепиться и понять, где и какой участок, ряд и т.д. Но
там зачастую просто имя человека, когда и отчего он умер. На этом все}, —
описывает сложности работы Андрей Марусов.

\ii{29_09_2021.stz.news.ua.svoi_city.1.aktivisty_mariupolja_kladbische_zapovednik.pic.7.allei}

Сейчас база данных организации \enquote{Архи-город} содержит сведения о полутысяче
людей, похороненных на старогородском кладбище. Причем родственники этих людей
живут сейчас по всему миру. К примеру, далекий потомок Гавриила Сахаджи живет в
Ташкенте и уже отреагировал на новость об обнаружении могилы его предка.

\enquote{\textbf{Мариуполь} - это город множества диаспор. Сначала он набирал людей, а примерно
в 1970-е годы начал отдавать. Потомки мариупольцев живут во многих странах
мира, но старогородское кладбище все равно остается частью истории их рода. Не
говоря уже о тех родственниках, которые продолжают жить в городе. Поэтому наша
мечта - создать тематический портал, на котором можно будет разместить имена и
истории людей, похороненных на этом кладбище. Идея такая, чтобы дать
возможность каждому умершему рассказать о себе ныне живущим. Это дорогостоящий
проект, поэтому сейчас мы отчаянно ищем для него финансирование}, — говорит
глава организации \enquote{Архи-город}.

\ii{29_09_2021.stz.news.ua.svoi_city.1.aktivisty_mariupolja_kladbische_zapovednik.pic.8.volontery_vyhodnyje}

Этим летом волонтеры снова работали на Хараджаевском участке. Из-за обильных
дождей он снова зарос дикой растительностью, местами скрыв следы прошлогодней
работы. Климат меняется, констатируют активисты, и если так будет продолжаться,
старогородское кладбище может превратиться в непроходимые джунгли. Поэтому
всеми доступными способами они пытаются продвигать идею о необходимости спасти
этот объект и создать здесь историко-культурное пространство.
