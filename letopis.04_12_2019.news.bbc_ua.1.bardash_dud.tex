% vim: keymap=russian-jcukenwin
%%beginhead 
 
%%file 04_12_2019.news.bbc_ua.1.bardash_dud
%%parent 04_12_2019
 
%%url https://www.bbc.com/ukrainian/news-50655749
 
%%author 
%%author_id 
%%author_url 
 
%%tags 
%%title 
 
%%endhead 

\subsection{Юрій Бардаш дав інтерв’ю Дудю, розмова наробила галасу. Про що всі сперечаються?}
\Purl{https://www.bbc.com/ukrainian/news-50655749}
4 грудня 2019
\index[names.rus]{Бардаш, Юрий!Продюсер!Интервью Дудю}

\ifcmt
pic https://ichef.bbci.co.uk/news/800/cpsprodpb/15CB/production/_109997550_89107ba3-c15d-4fc5-97b2-78b11d3f485b.jpg
caption Юрий Бардаш
\fi

\begin{leftbar}
	\bfseries Український продюсер Юрій Бардаш дав інтерв'ю російському блогеру і
журналісту Юрію Дудю. Розмова викликала чималий ажіотаж в українських
соцмережах - одні обурені, інші в захваті.
\end{leftbar}

\subsubsection{Хто такий Юрій Бардаш?}

Юрій Бардаш - музичний продюсер і виконавець. Його найуспішніший проєкт - група
\enquote{Гриби}, яка за нетривалий період існування випустила одразу кілька пісень, що
отримали величезну популярність в Україні та Росії.

Найвідоміша з них - \enquote{Тает лёд} - вийшла навесні 2017 року. Станом на зараз кліп
на цю пісню лише на Youtube зібрав понад 211 мільйонів переглядів.

Бардаша упродовж останніх років називали одним із найуспішніших музичних
продюсерів: профільні ЗМІ та музичні критики нерідко писали, що Бардашу вдалося
зробити величезний внесок у те, аби змінити вектор розвитку музичної індустрії
на пострадянському просторі.

Попри велику увагу до себе, Бардаш тривалий час уникав публічності: розмова з
Юрієм Дудем стала його першим великим інтерв'ю за останні шість років.

\subsubsection{Що він сказав?}

Інтерв'ю з Дудем триває понад дві години. В ньому Бардаш розповідає, зокрема,
про своє дитинство в Луганській області, захоплення музикою та танцями, появу і
розпад \enquote{Грибів}, а також сольну творчість та особисте життя.

Проте найбільший резонанс викликали його роздуми про політику, про
українсько-російські стосунки. Ось кілька найяскравіших цитат.

\begin{itemize}
    \item "Болить душа за мій регіон, за людей, які потрапили в "заміс". До "ЛНР" ніяк не ставлюся. Я ставлюся до людей, з якими я виріс. Мене вони цікавлять. До них я ставлюся чудово. Мені їхні погляди знайомі і зрозумілі. Я не за "ЛНР" і "ДНР", я за людей, які точно хочуть миру. А тих, хто щось ділить, вистачає з обох боків".

    \item \enquote{Що я спостерігаю в Україні останні п'ять років? Максимальне свавілля. Я дуже часто чую такий вираз 
						\enquote{типове донбаське бидло}. Але, що я знаю про у цих людей? Це люди, які працюють все життя, завжди. 
						Їхнє життя - завод, робота. Так, вони \enquote{сурові}, так, вони не такі культурні, як у столиці. 
						Але вони працюють - ось, хто для мене ці люди}.

    \item \enquote{Зараз в Україні заборонили всі російські школи. Але 45, 50 чи 60 відстоків українців, не знаю, - російськомовні люди. Це їхня мова. Україна - їх рідна країна, але їхня мова - російська. Відчувається жорсткий тиск на цих людей: щось сказати - і ти одразу
\enquote{сепаратист}, 
\enquote{ватник} або \enquote{вишиватник}. 
На фейсбуці сидять ці *** диванні, оце військо диванне, і сиплять оце все. Відбувається цькування людей}.

		\item \enquote{В мене дід воював. Я зростав з усвідомленням того, що я
						переможець, що мій народ переможець. А тепер пам'ятники зносять,
						вулиці називають прізвищами людей, які допомагали німцям. В нас
						тепер трішки інші герої в країні, нам це все нав'язується}.

		\item \enquote{Ми їздили в Росію відкрито, коли інші артисти сиділи,
притиснувши хвости. Держава з телевізора диктує митцю політику, і
ти робиш те, що кажуть. Я не такий. Я розумію людей, над якими
рвалися бомби, і прилітали вони з обох сторін. Я не хочу, щоб
між нами, сусідами, були рамси}.
\end{itemize}
