%%beginhead 
 
%%file 01_07_2023.fb.fb_group.story_kiev_ua.1.kievskij_zoopark
%%parent 01_07_2023
 
%%url https://www.facebook.com/groups/story.kiev.ua/posts/2227284440801678
 
%%author_id fb_group.story_kiev_ua,sirota_tatjana.kiev
%%date 01_07_2023
 
%%tags 
%%title Киевский Зоопарк
 
%%endhead 

\subsection{Киевский Зоопарк}
\label{sec:01_07_2023.fb.fb_group.story_kiev_ua.1.kievskij_zoopark}
 
\Purl{https://www.facebook.com/groups/story.kiev.ua/posts/2227284440801678}
\ifcmt
 author_begin
   author_id fb_group.story_kiev_ua,sirota_tatjana.kiev
 author_end
\fi

Когда старший внук был маленький, дети жили возле \enquote{Охмадета}. И я часто
приходила сюда с малышом на прогулку: зверей посмотреть и просто погулять в
парке.

Прошли годы, дети переехали в другой район города, в семье появился ещё один
мальчик, которого я называю \enquote{младшенький}.

И уже с ним я стараюсь, хотя бы, пару раз в год побывать здесь.

Уже догадались? Сегодня мы едем в Киевский Зоопарк!

День будний, посетителей немного.

И можно без помех и толчеи подойти к вольерам с животными и птицами,
внимательно рассмотреть их, понаблюдать за ними.

За пять часов обошли весь парк. Проведали любимого жирафа и слоника Хораса, с
восторгом понаблюдали за медведями, навестили льва Геркулеса и его гарем из трёх
прекрасных львиц, посетили террариум, птичник и ещё много чего... 

В общем, получили море ярких впечатлений и позитива.

История Киевского зоопарка началась в апреле 1907 года, у его истоков стояли
активисты Киевского общества любителей природы. Преподаватели Политехнического
университета, учителя городских школ, врачи и общественные деятели Киева
выступили с инициативой создания подобного учреждения. 

Весна 1909 года считается официальным периодом основания зоопарка в Киеве.
Именно тогда он получил свою территорию и возможность собирать деньги на
содержание животных. В 1912 году Зоопарк переехал на то место, где находится по
сей день. Несмотря на то, что рельеф выделенного участка земли был очень
разнообразным (с равнинами, холмами и оврагами), его удалось рационально использовать. В
оврагах выкопали пруды для водоплавающих, на возвышенностях - были установлены
клетки и вольеры. Официальное новоселье состоялось в 1913 году.  А в 1919 году
парк получил статус \enquote{Киевский зоологический парк}. Первым его директором был
назначен Венантий Бурдзинский.

Сейчас в зоопарке живет около 3400 животных более 350 видов, среди которых
млекопитающие, птицы, рыбы, пресмыкающиеся, амфибии, беспозвоночные.

На его территории произрастает около 130 видов деревьев и кустарников. Помимо
развлекательной и познавательной, зоопарк выполняет еще и научную функцию.
Здесь проводятся исследования по акклиматизации экзотических животных и
размножению редких видов.

30 июня 2023 года.
