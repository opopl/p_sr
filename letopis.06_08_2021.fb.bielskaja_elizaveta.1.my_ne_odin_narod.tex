% vim: keymap=russian-jcukenwin
%%beginhead 
 
%%file 06_08_2021.fb.bielskaja_elizaveta.1.my_ne_odin_narod
%%parent 06_08_2021
 
%%url https://www.facebook.com/elizabeth.bielska/posts/4225742467461220
 
%%author Бельская, Елизавета
%%author_id bielskaja_elizaveta
%%author_url 
 
%%tags odin_narod,rossia,turcia,ukraina
%%title МИ НЕ ОДИН НАРОД
 
%%endhead 
 
\subsection{МИ НЕ ОДИН НАРОД}
\label{sec:06_08_2021.fb.bielskaja_elizaveta.1.my_ne_odin_narod}
 
\Purl{https://www.facebook.com/elizabeth.bielska/posts/4225742467461220}
\ifcmt
 author_begin
   author_id bielskaja_elizaveta
 author_end
\fi

МИ НЕ ОДИН НАРОД.

Туреччина охоплена вогнем. У цей час українці Туреччини збирають їжу і постіль
для тих, кого евакуювали. А їхні діти передають власні улюблені іграшки тим
турецьким малюкам, які лишилися без дому. Бо змалечку вміють співпереживати. В
цей час володарка титулу "Міс Росія" в інстаграмі викладає світлину в бікіні на
тлі гір, охоплених димом, зі словами "Я така гаряча, що аж усе довкруж горить".
Інша російська блогерка, котра має майже 700 тисяч читачів, підписує своє фото
з густими клубами диму словами "тупо за гарячою путівкою святкую свій день
народження". 

Україна безкоштовно надає Туреччині пожежні літаки, а приватна компанія
Aviation Company Ukrainian Helicopters безоплатно направила гвинтокрили та
екіпаж із 40 осіб, щоб гасити пожежі цілодобово. Росія у цей час на весь світ
заявляє про свою величезну допомогу літаками, за кожен день якої отримує від
Туреччини 150 000 доларів, просто здаючи свої повітроплави в оренду. Росія, до
слова, одна з небагатьох країн, яка не погодилася благодійно надавати допомогу
зі стихійним лихом, у якому гине не тільки природа, а й люди й тварини. 

Українці з'ясовують потреби в кожному регіоні й передають пожертви - все, як
було на Майдані. Паралельно домовляються про колективні молитви.

Ні, ми точно не один народ. На щастя. І мова - єдине що допомагає нам
відмежуватися.

\ii{06_08_2021.fb.bielskaja_elizaveta.1.my_ne_odin_narod.cmt}
