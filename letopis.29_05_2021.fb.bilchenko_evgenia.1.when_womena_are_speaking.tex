% vim: keymap=russian-jcukenwin
%%beginhead 
 
%%file 29_05_2021.fb.bilchenko_evgenia.1.when_womena_are_speaking
%%parent 29_05_2021
 
%%url https://www.facebook.com/yevzhik/posts/3934156773286070
 
%%author Бильченко, Евгения
%%author_id bilchenko_evgenia
%%author_url 
 
%%tags 
%%title БЖ. Когда говорят женщины/Коли говорять жінки / When women are speaking
 
%%endhead 
 
\section{БЖ. Когда говорят женщины/Коли говорять жінки / When women are speaking}
\label{sec:29_05_2021.fb.bilchenko_evgenia.1.when_womena_are_speaking}
\Purl{https://www.facebook.com/yevzhik/posts/3934156773286070}
\ifcmt
 author_begin
   author_id bilchenko_evgenia
 author_end
\fi

БЖ. Когда говорят женщины/Коли говорять жінки / When women are speaking.

\verb|@bilchenkozhenya|
\verb|@bilchenko_z|

Только не подумайте, что Бильченко свихнулась на Жижеке окончательно, и сейчас
начнется леволиберальная размазня о феминистической солидарности. Хотя и она
иногда полезна, если речь идет о Мужестве Женщины. Я не понимаю одного: почему
в последнее время вокруг меня вращается всё меньше сильных мужчин и все больше
смелых женщин, способных высказать позицию, противоречащую тренду гегемонии.
Способных не быть рабами общественного мнения. Почему мужчины все время мне
говорят: \enquote{Осторожно, Женя} или: \enquote{Женя, давай уйдем в чистую
философию/поэзию/эстетику}?

\ifcmt
  pic https://scontent-frt3-2.xx.fbcdn.net/v/t1.6435-9/192130434_3934149399953474_1043200797340306335_n.jpg?_nc_cat=110&ccb=1-3&_nc_sid=730e14&_nc_ohc=LYKlnRCrVWYAX8tf3FC&tn=ntrKbsW_7ChXu3v-&_nc_ht=scontent-frt3-2.xx&oh=46c9c678c83f8906ad89e7fc97c4e14e&oe=60D722C7
\fi

Ведь это же и есть двойной стандарт чистой воды. Там, где больше всего
\enquote{аполитичности}, под бархатным платочком чистоплюйства сидит идеологический
монстр - идеология, которая не называет себя так. Двойным является отношение
гегемонии ко всему: к регионам и идентичностям, к национальным государствам и
целым цивилизациям. Мультикультурализм, который давал обет уважать
Чужого/Иного, делит нас на \enquote{хороших других} и \enquote{плохих других}, прирученных и не
прирученных. Чем это отличается от схемы \enquote{свои - чужие}? Последняя хотя бы
честна. Мне удалось за последнюю неделю узнать двух женщин поближе.

Одна - простая американка, военный в американской армии. 9 мая она празднует
Победу (а не день всеобщей тирании покаяния) и держит фотографию своего деда.
На ней - та самая запретная в нашей стране лента. В центре Вашингтона. Не
двойной ли стандарт? Она - не политическая элита, она - простая американка. Ее
зовут Даниэлла Эдисон и она говорит: \enquote{My great grandfather was a pilot and I'm
trying my best to honor him and other people who won WWII}. Она не боится
говорить нетрендово.

Вторая - осмелюсь сказать, мой заочный друг, депутат Европарламента, -
Катержина Конечна, чешка. Вчера она жестко высказалась о двойных стандартах по
отношению к журналистам на примере прецедента в Минске: что прощается
\enquote{демократическим} странам, не прощается \enquote{недемократическим}. Причем, критерии
\enquote{демократичности} тоже устанавливает современная цензура. Катержина тоже не
боится говорить нетрендово:

"Дорогие единомышленники,

От многих из вас я получаю информацию о гражданине Белоруссии Романе
Протасевиче, который был высажен с самолета, направлявшегося в Вильнюс. Как вы,
кто следит за мной в течение долгого времени, знаете, у меня нет потребности
гоняться за сенсациями. В отличие от политиков, которые всегда знают кто
виновен и как его стоит наказать в зависимости от того, кому данный человек
служит или против кого выступает, я всегда сначала проверяю факты.

Прежде, чем я перейду к названному молодому человеку, которого СМИ называют
диссидентом и журналистом, я хотел бы сначала остановиться на самой посадке
самолета. Судя по заголовкам, западные страны воспринимают это как \enquote{похищение}
или даже \enquote{террористический акт}. Я не знаю, даже если самолет действительно
приземлился только из-за Протасевича, что ни авиакомпания, ни ее пилоты пока не
подтвердили, это не первый случай, когда мы сталкиваемся с подобным случаем. 8
лет назад в 2013 году президента Боливии Эво Моралес заставили приземлиться в
Вене. А насильно приземлить спец. борт правительства - это совсем другое дело.
Предполагалось, что на борту самолета будет Эдвард Сноуден, человек, который
разоблачил самую гнусную мерзость, которая была спрятана в недрах Соединенных
Штатов Америки. Кстати, в том числе и шпионаж против своих партнеров -
возможно, это заметили и «бестолочи» (в оригинале «čučkaři» - на чешском сленге
– «неспособные, бесполезные люди»), простите сотрудники чешской контрразведки
BIS. Интересно, что в тот раз журналисты не говорили о \enquote{государственном
терроризме} или посягательстве США на суверенитет стран, но заголовки носили
гораздо более мягкий тон \enquote{незапланированная посадка}. Ведь в Украине в 2016
году был вынужден приземлиться самолет по стечению обстоятельств, белорусский,
из-за решения украинских спецслужб, которые впоследствии задержали неудобного
украинскому режиму армянского журналиста.

Теперь к Роману Протасевичу. Как я уже сказала, я проверяю факты, но до сих пор
я нигде не сталкивалась с тем, что фотографии, которые вы сейчас видите на
своих экранах, не являются подлинными. Наверное, излишне говорить, что это за
приветствие на фотографиях. Однако нам нужно сосредоточиться на том, что флаги
и знаки, изображенные на фотографии, на самом деле связаны с указанным
приветствием. Этот флаг использовался во время оккупации Беларуси нацистской
Германией. К сожалению, им пользуется и так называемая белорусская оппозиция,
которую, кстати, также слепо поддерживает чешское правительство. Я не понимаю
Латвию, которая, задействуя спорт и чемпионат мира по хоккею, позволила себе
вместо того, чтобы вывесить официальный флаг страны, использовать красно-белый
флаг. Я не могу представить себе флаг фашистского протектората, висящий на
каком-либо чемпионате вместо чешского национального флага, что сравнимо с тем,
что сделала Латвия.

Я получила различные электронные письма от многих из вас, раскрывающие прошлое
Романа Протасевича. В письмах рассказывается о его причастности к Майдану, то
есть к перевороту на Украине, и даже о его активном членстве в батальоне
\enquote{Азов}, состоящем из ультраправых. Об этом, а также о его предполагаемом
обучении в Чехии (разумеется, за ваши деньги - граждане), о чём,
(Коммунистическая Партия Чехии и Моравии, единственная политическая партия в
Парламенте Чехии отправила запрос) до сегодняшнего дня ответ так и не получили.

Но если подтвердится, что он является членом этого фанатичного батальона,
разрешенного после Майдана на Украине, то для меня это явный сигнал, с каким
человеком мы имеем дело. Просто напомню вам о том, чем является военизированный
батальон \enquote{Азов}. И это проверенная и отслеживаемая информация. Как я уже
сказала, он была создана в 2014 году, и в ее рядах мы можем найти, в частности,
неонацистов, невменяемых националистов, короче говоря, представители крайне
правых. Последние этого не скрывают даже в рамках использования своей
символики, которая на нацистский режим А. Гитлера прямо указывает. История
этого батальона довольно короткая, но тем более кровавая, потому что уже 15
ноября 2014 года, менее чем через полгода после организации этой
военизированной группировки, управление Верховного комиссара ООН по правам
человека опубликовало заявление, в котором говорится, что \enquote{Азов нарушает
правила ведения войны, например, массовыми грабежами, пытками и похищениями
журналистов.} Затем этот же орган еще несколько раз расследовал преступления
батальона \enquote{Азов} и пришёл к выводу, что \enquote{с сентября 2014 года по февраль 2015
года было задокументировано массовое разграбление жилых домов мирных жителей. В
августе-сентябре 2014 года было зафиксировано жестокое обращение, изнасилования
и другие формы сексуального насилия, применяемые членами батальонов \enquote{Азов} и
\enquote{Донбасс}. Для получения признательных показаний от задержанных гражданских лиц
\enquote{Азов} применял пытки и избиения, в том числе применял электрический ток и
удушение водой.}

Нет, я не буду защищать режим Лукашенко, но те же, кто проливают слезы на
телевидении о человеке, в биографии которого, по крайней мере, выражаясь
эвфемизмом, - темные места, намеренно игнорируют уже упомянутого Эдварда
Сноудена или Джулиана Ассанжа. Этот случай - всего лишь еще один пазл в мозаике
лицемерия и двойных стандартов западной неолиберальной политики и, к сожалению,
большинства средств массовой информации, которые поставили перед собой цель
влиять на общественное мнение, а не беспристрастно информировать".

Вот ссылки на ее речь:  

\url{https://www.youtube.com/watch?v=afCiALe7pFc} 
\url{https://www.facebook.com/119624098506/posts/10158405786928507/}

И последнее. Человек, девушка, воспитанная на моих лекциях, в моей
антиглобалистической школе, ныне запрещенной в Украине, за эту же школу
получает несколько стипендий ЕС и становится председателем Совета аспирантов в
Польской Академии Наук, считая меня учителем. Не скажу, кто и какой институт,
чтобы ей не писали.  

Это не значит, что я совпадаю с упомянутыми женщинами полностью: они - более
левые, меня все больше клонит в сторону Логоса и Космоса русской поэзии,
метафизики и традиции, но суть не в этом. Мы же не делим так людей.

Суть в ином. Почему в самое темное время перед началом грандиозного мирового
противостояния мужчины всё больше прибегают к невнятной речи? 

Дорогие мужчины, я вас очень прошу, как парень - парней: не дожидайтесь, пока
мы вас будем считать не братьями, а сестрами. Поддержите, пожалуйста, таких
людей, как Даниэлла и Катержина, - своей силой. Она нам очень необходима. Ваша
мужская сила. Неужели нужна еще одна Зоя Космодемьянская, чтобы по эту сторону
глобальной периферии вы сделали для нас какое-то чудо, хотя бы словесное? 

Мне не надо говорить: \enquote{Бильченко, осторожно!} - со мной рядом надо стоять.
Стоять за спиной, если надо. Я не боюсь впереди. Но стоять. 

Если хотя бы одно украинское оппозиционное СМИ заинтересуют более подробные
источники, фото и оригиналы, которыми я пользуюсь, - я всегда к их услугам. Я -
не журналист. Я - за правду. Но журналист же должен что-то делать за правду.
Мужчина-журналист. Но мне почему-то кажется, что первой на историю Даниэллы и
Катержины отзовется снова женщина.

PS. \enquote{Любишь кататься - родись груздем, говорят...
Можешь летать - не бойся в калашный ряд}.

(Из сегодняшнего посвященного мне текста одного украинского поэта, которая
тусит с \enquote{правильными} писателями и которой я не хочу портить жизнь, выставляя
здесь ее имя, - тоже женщины, кстати)

Олег Скуратов

Браво, Женя, это было круто!

Владимир Морозов

Много текста, прочитал, но сути не уловил. О чем текст? О мужчинах, которые
слепо не поддерживают Вашу позицию, твою правду? Вы пишите о двойных стандартах
и типа осуждаете их, за их не принципиальную позицию. Вроде как все
происходящее должно соответствовать доктринам кем то ранее установленным. Но мы
же взрослые люди и понимаем, что в мире есть добро и зло и это не две силы,
которые постоянно борются между собой, это два состояния в каждом из нас и мы
их олицетворяем через свои действия, мысли и поступки. Только окружающие нас
люди дают оценку нас нам самим же на основе своего видения справедливости и
мироощущений. Фашисты, коммунисты, демократы в принципе тоже люди, которые в
свое время поддались пропаганде, клюнули на чью то правду и заняли чью то
сторону. А самые продвинутые среди них пошли дальше и научились использовать
это в своих интересах. Выживает сильнейший, не так ли?

Евгения Бильченко

Владимир Морозов 

Нет, \enquote{не так ли}, потому что вы с циничной улыбкой используете самую
тривиальную либеральную риторику: крайний релятивизм для морального оправдания
фашизма. Поверьте, это было сделано уже сотни раз до вас, не стоит повторяться.
А разбивается этот прием очень легко: логика относительности, свойственная для
вашей иронии, предполагающая уважение к Другому, противоречит логике
\enquote{выживает сильнейший}, где есть четкий бинаризм
\enquote{сильный-слабый}, \enquote{свой-чужой}. Могу вас разочаровать: текст не
о мужчинах, занявших позицию, отличную от моей, Текст о том, что по вашей
логике: а. я даю негативную оценку вашему мнению, ибо как \enquote{окружающий}
я имею право на это, следуя приему \enquote{только} (называется это
\enquote{контекстуальная истина}, вы мне сами его подбросили).  б. более
продвинутый, чем вы (а такие всегда есть, не обольщайтесь насчет своей удачной
адаптации), использует вас и выбросит, следуя принципу дарвинизма.

Владимир Морозов

Евгения Бильченко Вы отрицаете действительность, заменяя ее более глубоким, но
все же -Своим- пониманием? Да Ваша действительность так и так отличается от
моей и от миллиардов других похожих на людей

Евгения Бильченко

Владимир Морозов Мы не существуем вне своего понимания. Если у вас нет позиции,
значит, она у вас есть, просто сформирована не вами. Это воображаемая
действительность.

Евгения Бильченко

Владимир Морозов Мы не существуем вне своего понимания. Если у вас нет позиции,
значит, она у вас есть, просто сформирована не вами. Это воображаемая
действительность.

Владимир Морозов

Евгения Бильченко я считаю, что любая моя позиция в меня вшита от рождения. Вы
можете меня перепрограммировать? - Я, уверен, Вы, скажете: "Да, могу" Ведь Вы
именно этим тут и занимаетесь, Вы призываете людей меняться, а если они не
меняются под стать Вам, тогда Вы их всё-таки, наверное-осуждаете. Что это,
разве не программирование, не меняние мнений?

Просто Ваша тема ещё не очень популярная, но видимо для нового поколения
молодых, она станет похожей на смысл жизни... - Это к слову о фашизме, он сначала
тоже был не очень популярным, как и социализм (но их поддержало необразованное
большинство, массой задавили)

Владимир Морозов

Евгения Бильченко я Вашей деятельности рискну дать определение в духе времени:
\verb|#политическийстартап|

Igor Shnurenko

Всё верно. Обнаружилось, что против Нового мирового порядка и цифрового
концлагеря в основном выступают женщины. Мужчины предпочитают не иметь
собственного мнения - это даёт им свободу лавировать вместе с линией партии. То
есть «свободу» следовать сигналам роевого управления.

Евгения Бильченко

Игорь Шнуренко Я бы не хотела использовать гендер, это не мой принцип мышления,
но получается, что swarming (роение со скрытым хабом) очень четко распознается
женским стилем мышления, потому что американские авторы техники роя Аркилла и
Шафрански считали, что, чтобы \enquote{бороться с сетью, надо стать сетью}, а женское
мышление - не линейно, оно изначально ризомно. Она уже сеть - она узнает эти
манипуляции. Невольно. Я говорю сейчас отстраненно, потому что мне трудно это
давалось, у меня мужской склад ума (слишком прямой), только эмоциональность
женская.

Алексей Кучма

Игорь Шнуренко Открытая система, предпочитает оставлять анонимные уведомления.
Никто, кроме самой открытой системы, не удерживает критического восприятия
функционирования её механизмов. Рой, как закрытая система, наглядно
демонстрирует свою функциональность. Но, отдельно взятая пчела, не может
соотнести свой опыт с окружающим миром. Потому, для нее рой выступает, скорее
системой открытой. Так и с людьми (\enquote{И каждый в мире собой обольщен, И каждый
только лишь видит сон}). Иерархия необходима, для продолжения функционирования
системы. В сущности, переживаемая субъектом, она возможна только в пределах
замкнутой системы, но если же, каким-то образом, индивид догадается о
существовании мира «за гранью» этой реальности, он не только будет шокирован,
но и его шокированость станет очагом сопротивления, криком о помощи умирающего
ягненка. Тут-то его и прирежут. Получается просветляться, себе же дороже. И
дело тут не в гендере. Как бы это апокалиптично не звучало, но вне роя мы не
только не выживаем, а становимся угрозой и нас умерщвляют. Так что
мимикрировать, не самый плохой вариант. С учетом существующей системы
коммуникации, протест-маскарад, ничего общего не имеет с революционной борьбой,
а времена антиутопий уже наступили, что неутешительно.

Сергей Никонов

У сильных мужчин есть много мотивов для молчания и медленных решений. Судьба
каждого и его мировоззрение индивидуальны. Если разные формы стояния. Можно
стоять молча, но очень грозно и враги почувствуют мысль, а придраться не
смогут. Можно спокойно делать вид подчинения и единомыслия, но показать себя в
самый неприятный для врага момент. И многое другое. Важно ощущение поддержки.
