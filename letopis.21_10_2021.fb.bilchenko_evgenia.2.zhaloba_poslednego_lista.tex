% vim: keymap=russian-jcukenwin
%%beginhead 
 
%%file 21_10_2021.fb.bilchenko_evgenia.2.zhaloba_poslednego_lista
%%parent 21_10_2021
 
%%url https://www.facebook.com/yevzhik/posts/4365121413522935
 
%%author_id bilchenko_evgenia
%%date 
 
%%tags bilchenko_evgenia,poezia
%%title БЖ. Жалоба последнего листа
 
%%endhead 
 
\subsection{БЖ. Жалоба последнего листа}
\label{sec:21_10_2021.fb.bilchenko_evgenia.2.zhaloba_poslednego_lista}
 
\Purl{https://www.facebook.com/yevzhik/posts/4365121413522935}
\ifcmt
 author_begin
   author_id bilchenko_evgenia
 author_end
\fi

БЖ. Жалоба последнего листа

\ifcmt
  ig https://scontent-lga3-2.xx.fbcdn.net/v/t39.30808-6/244716094_4365121536856256_5978966323530548700_n.jpg?_nc_cat=107&ccb=1-5&_nc_sid=8bfeb9&_nc_ohc=UqAh5CAzlFsAX8JDNoM&_nc_ht=scontent-lga3-2.xx&oh=da65b647b061990400c51a6b6d26bfd9&oe=6177F42F
  @width 0.4
  %@wrap \parpic[r]
  @wrap \InsertBoxR{0}
\fi

Я похожа на этот лист: медь и золото, с червоточиной.
Я ощущаю вину за всех, не виденных мною воочию.
Я ощущаю вину за тех, кто меня воспитал и вырастил:
За отлучённых, за неофитов, за грешников и за выкрестов.
Я ощущаю вину, если я болею и дохну клячей.
Ощущаю вину за сдачу, за то, что боюсь, но клянчу.
За то, что падаю от работы, когда плоть вопит от труда.
За то, что в кране течет вода, а с мамой - запой, беда.
За то, что я лгу разведчиком, а правда бурлит по венам.
За то, что я себя обнажаю - бездарно и откровенно.
За то, что недуги свои ношу во чреве, как бабы деток,
За то, что взгляд мой бывает страшен, а юмор - едок.
За то, что стесняюсь сказать, что мне нужен хороший врач.
А ещё -  коньяк и Нева, Саврасов, соцреализм и грач.
Чтобы ко мне они прилетели, все они прилетели,
Чтобы любимому я могла без стыда говорить о теле.
Я ощущаю вину - даже, если покупаю себе пальто -
При полном безденежье, по дешевке, - лазоревое зато.
Я ощущаю вину, что я без гидазепама веско
Не в силах сказать, как отрезать, не в силах больше нести ответственность
За всё и за всех, за всех и за всё, отныне и навсегда...
За то, что плывут суда. 
За то, что течет вода.
За майдан, за АТО, за ополчение, за Крым, за Донбасс, за скрепы,
За склепы и за семью, за плохую начитку рэпа.
Я больше не верю в себя как в ученого, но когда мне
Звонят: "Выходи к нам на конференцию", - хрипло, сдавленно
Я отвечаю: "Буду!" - и комплексом - сердце в пятки.
(А в голове: "Диссидент вам нужен или мой Интернет, ребятки?")
Я больше не верю в себя как в поэта, потому что мои живые
Друзья стали злыми, а мертвые - цветы давно полевые.
Я вырвалась из поэтики, из языкового метра:
Ни муж, ни читатель не успевают за тоникой моей, ветра
Быстрее, что мчит меня по фонемам. Я хочу прекратиться. Вовсе
Жить после подвига - не фонтан. К тому же, такую осень
Сулят мне Болдино и Васильевский: я ведь всё к сорока успела...
Успела, да потеряла. Лист мой червивый спелой
Монетой на землю вот-вот падёт, - и тогда я совсем угасну.
Нет больше подросткового: "Не плачь обо мне напрасно!"
Моя вина есть моя гордыня. Нрав ей причина ль, ген ли?
Я очень просила, но мне никто не придумал сюжет О'Генри.
21 октября 2021 г.
Фото: Grigory Belov
