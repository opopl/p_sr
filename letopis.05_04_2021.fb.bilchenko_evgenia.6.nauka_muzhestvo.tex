% vim: keymap=russian-jcukenwin
%%beginhead 
 
%%file 05_04_2021.fb.bilchenko_evgenia.6.nauka_muzhestvo
%%parent 05_04_2021
 
%%url https://www.facebook.com/yevzhik/posts/3777057732329309
 
%%author 
%%author_id 
%%author_url 
 
%%tags 
%%title 
 
%%endhead 
\subsection{БЖ. Когда наука - это мужество быть}
\label{sec:05_04_2021.fb.bilchenko_evgenia.6.nauka_muzhestvo}
\Purl{https://www.facebook.com/yevzhik/posts/3777057732329309}


\ifcmt
  pic https://scontent-bos3-1.xx.fbcdn.net/v/t1.6435-9/169465073_3777057665662649_5175755313758554717_n.jpg?_nc_cat=106&ccb=1-3&_nc_sid=8bfeb9&_nc_ohc=GQvmxIt6an4AX8TqQl9&_nc_ht=scontent-bos3-1.xx&oh=72a75bc998500ebef185e3bf6c314ba7&oe=6094CEBA
\fi


Владимир Возняк, Вера Лимонченко, Геннадий Лобастов - лучшие ильенковцы нашей постсоветской духовной Родины, большие ученые, - спасибо вам за этот день щедрой красоты общения, трепетной дискуссии, умного диалога и прозорливой доброты посреди прожорливого мира. 
Я полдня плакала вчера над "Рублевым", всю ночь писала доклад о нем глазами Андрея Тарковского и целый день провела в конференции в зуме, похожем на старинное Новгородское вече имени сестры нашей Родины Софии и брата нашего Мира Павла.
Сергей Возняк
 , твой брат вырос в том старом городе Франнике, где и я, где ещё не было всего вот этого... И он тоже - православный и не неолиберальный. Я кайфую от этой истины. От науки, поэзии, этики. Мой доклад очень высоко оценили, но это я не хвастаю (гордыня - грех): я просто ВПЕРВЫЕ писала о Тарковском (я профан в нем) и ВПЕРВЫЕ выступала перед такими людьми после травли. Перетрусила.
Честно? Я плакала во время доклада, просто голову подкидывала, чтобы не стекало, говоря о восстановлении Бориской Колокола после ордынской руины, - боялась разрыдаться в этот треклятый зум. А то подумают, что я биполярная, как Мамоновский герой.
Это вот, что было, - это то самое, из научной моей весны, то забытое, то настоящее. Это - одна из лучших международных конференций в моей жизни, где ни марксистам, ни патристам не было тесно. #Любовь ибо. По сравнению с этим - потуги коллег-американистов по экранной культуре... Ну, такое... Промолчу и напишу Иное. Пусть запрещают. Рублева тоже не жаловали.
PS. БОЛЬШОЙ ПРИВЕТ ВИКТОРУ ВСЕЯ РУСИ АРОНОВИЧУ МАЛАХОВУ ОТ ВСЕХ! 
PPS. Николай Топало
  , спасибо, что ты был, а то украинские студенты со своими пинчуками скоро вообще онтологическим раком заболеют. Сергей Никонов
 , я тебя оценила. Юна Щудло
 , ты много пропустила. Придется читать доклад отдельно, но там остальные все доклады - тоже звездочки... А мы увидим небо в алмазах? А мы отдохнем? \verb|#русскимбытьмодно|

