% vim: keymap=russian-jcukenwin
%%beginhead 
 
%%file 11_11_2020.news.ua.5ua.1.syn_ivana_franka
%%parent 11_11_2020
 
%%url https://www.5.ua/svit/syn-franka-i-zhinka-khorunzhyi-ukraintsi-pershoi-svitovoi-viiny-iakykh-rosiia-vykynula-z-nashoi-istorii-228885.html
%%author 
%%author_id 
%%tags 
%%title 
 
%%endhead 

\subsection{Син Франка і жінка-хорунжий: українці Першої світової війни, яких Росія викинула з нашої історії}
11.11.2020 08:00

\Purl{https://www.5.ua/svit/syn-franka-i-zhinka-khorunzhyi-ukraintsi-pershoi-svitovoi-viiny-iakykh-rosiia-vykynula-z-nashoi-istorii-228885.html}

\index[names.rus]{Франко!Петро!Син Івана Франка}

\ifcmt
pic https://www.5.ua/media/pictures/1140x641/200400.jpg
caption Син Франка і жінка-хорунжий: українці Першої світової війни, яких Росія викинула з нашої історії
\fi

{
\bfseries
\lettrine[lines=3]{З}{ранку 11 листопада 1918 р.} пролунали останні залпи Першої світової війни –
салют на честь переможців. Найкривавіша на той час в історії війна завершилася.
Але чому її називають забутою? І чому нині \enquote{не модно} говорити про цей
конфлікт? Як Росія доклала руку до того, щоби Перша світова й імена
українців-учасників стиралися з нашої пам'яті?
}

Перша світова була глобальним протистоянням військово-політичних блоків Антанти
і Троїстого (згодом Четверного) союзу. Українці воювали за чужі інтереси по
обидва боки фронту, адже українські землі були розділені між Російською та
Австро-Угорською імперіями, які представляли відповідно Антанту і Четверний
союз.

Цю війну часто називають \enquote{забутою}, хоча за масштабами вона поступається лише
Другій світовій. 4 роки, 3 місяці і 2 тижні невпинного жертвопринесення.

\begin{leftbar}
	\bfseries
В Україні пам'ять про цю війну систематично викорінювали з масової свідомості
понад 70 років більшовицької влади.
\end{leftbar}

Для влади СРСР Перша світова не була чимось героїчним. Ленін ще на початку
обіцяв війну імперіалістичну перетворити на громадянську. Тому героїв Першої
світової радянські ідеологи замінили героями війни громадянської, пояснює
історик Андрій Руссак.


\begin{leftbar}
	\bfseries
\enquote{Шкільна програма була побудована так, що було важко зрозуміти, де і коли Перша
світова починається і чим закінчується. Згідно з офіційною радянською
доктриною, Перша світова у Росії переросла у громадянську, а в ній перемогла
Червона армія, тобто т. зв. \enquote{наші}, яких, власне, за версією радянської влади,
і треба було вшановувати як нових героїв}, – пояснює історик. 
\end{leftbar}

Водночас на Заході не було кардинальних соціальних катаклізмів, не
переглядалась історія через зміну політичної ситуації. Тому всі війни США і
західноєвропейських країн займають належне місце в історії цих держав, а
солдати є героями.

\begin{leftbar}
	\bfseries
\enquote{У Першій світовій загинуло до 1,5 млн. українців. Але немає жодного
меморіалу загиблим на території України. Хоча би чисто з людської точки
зору цю тему потрібно вивчати. Адже 1,5 млн. людей – це величезна цифра. І
про них не лишилося жодного сліду}, – каже історик. 
\end{leftbar}

\ifcmt
pic https://www.5.ua/media/pictures/820x546/200387.jpg
caption Перша світова війна, відкриті джерела
\fi

Є хіба що пам'ятники, встановлені ще під час війни австрійцями, німцями
або росіянами на честь полеглих своїх або чужих вояків. Вони збереглись на
старих цвинтарях Західної України.

\enquote{У Ковелі, наприклад, на старому міському цвинтарі збереглось декілька
пам'ятників бійцям, які померли у місцевих госпіталях. Збереглися
меморіали австрійським і німецьким воякам. Там є меморіал російським
воякам, який, уявіть, поставили австро-угорці на честь полеглих ворогів.
\enquote{Полеглим російським героям від цісарської і королівської 4-ї Армії, 1918
рік} – так написано}, – розповідає Руссак.

У Києві, поблизу Ботсаду ім. Гришка, є церква св. Миколи. Її будували у
пам'ять про померлих у госпіталях Києва. Навколо церкви відбувалися
поховання. 1917 р. вже мало би початися оздоблення церкви всередині, але
більшовики перекреслили цю справу – Перша світова була для них чужою.

\enquote{Цей меморіал так і не був завершений. Цвинтар зруйнували. У 1950-ті рр.
на його місці побудували житлові будинки, і коли рили котловани, весь час
знаходили людські кістки і черепи. Лишилося тільки незавершена церква,
котра тепер використовується як інститутська лабораторія}, – пояснює
історик.

\begin{leftbar}
	\bfseries
За підрахунками істориків, загалом понад 4,5 млн. українців були
мобілізовані до російської і австро-угорської армій. Але ця війна і досі
незнана в українській історії.
\end{leftbar}

Точної статистики щодо кількості українців, які воювали в арміях Росії і
Австро-Угорщини, скільки загинуло, було поранено, потрапило в полон –
немає.

\ifcmt
pic https://www.5.ua/media/pictures/820x546/200388.jpg
\fi

\enquote{У Російській імперії українці становили 18\% населення, в армії їх було
25-30\% від загальної численності. Можемо розрахувати: якщо Росія
мобілізувала 15 млн людей, українців серед них мало бути до 4 млн.
Загинуло 2 млн 250 тис. людей, отже можемо говорити, що близько 600 тис.
українців загинули в російській армії. Поранених – 3,8 млн, тоді близько 1
млн українців зазнали поранень у російській армії}, – пояснює він.

\begin{leftbar}
	\bfseries
Під час Першої світової понад 1 млн цивільних в імперії загинули від
хвороб і голоду. Скільки серед них було українців, говорити складно.
Перша світова охопила Царство Польське, Білорусь, Західну Волинь, Литву,
Курляндію, тому частка українців може бути менша за 30\%, але все ж
мовиться про кілька сотень тисяч загиблих, каже історик.
\end{leftbar}

\ifcmt
pic https://www.5.ua/media/pictures/820x546/200401.jpg
\fi

Серед 52 млн мешканців Австро-Угорщини українці становили 8\%. Можна
вирахувати, що з 9 млн мобілізованих українців могло бути 700 тис. людей.
Загинули в Австро-Угорщині 1,5 млн людей, тобто українців – близько 120
тис., поранено – 2,5 млн, отже українців – десь 200 тис., каже пан Руккас.

Галичина, Буковина були театром воєнних дій. Через голод, бої, перебування
в таборах могли загинути 200-300 тисяч. Тобто з боку Австро-Угорщини
загинуло до 0,5 млн українців – військових і цивільних.

\ifcmt
pic https://www.5.ua/media/pictures/820x546/200402.jpg
\fi

\begin{leftbar}
	\bfseries
\enquote{В цілому, за можливо трохи завищеними оцінками, до 1,5 млн українців
втратили своє життя під час Першої світової війни}, – пояснює дослідник.
\end{leftbar}

Більшість мобілізованих українців билися на
російсько-австрійсько-німецькому фронті, що пролягав через Волинь,
Галичину, Закарпаття, Поділля, Буковину і Бессарабію.

Українці, піддані Австрії, воювали на Балканах, у північній Італії, на
флоті в Середземномор'ї. У лавах російської армії – від Балтійського до
Чорного морів, на Закавказзі, в Ірані, у складі експедиційного корпусу в
Македонії та в північній Франції. Представники української діаспори билися
в арміях Канади, Австралії, США.

\ifcmt
pic https://www.5.ua/media/pictures/820x546/200386.jpg
\fi

\begin{leftbar}
	\bfseries
З 1914 до 1917 р. єдиною національною українською військовою частиною
був легіон Українських січових стрільців (УСС) у складі австро-угорської
армії. За час війни через його лави пройшли понад 9 тис. осіб. Втрати
вбитими, пораненими й полоненими перевищили 3 тисячі.
\end{leftbar}

Січові стрільці запекло билися проти російських військ. Відзначилися в
обороні гір Маківка (1915) і Лисоня (1916). У складі легіону була й жіноча
чота (жінки-стрільці й санітарки – щонайменше 36 осіб). Надалі "усуси"
стали активними учасниками Української революції, однією з найкращих
формацій українського війська.

\ifcmt
pic https://www.5.ua/media/pictures/820x546/200385.jpg
\fi

У російській армії не було жодного українського національного формування.
Тобто 4 млн мобілізованих українців були розпорошені по всіх фронтах.

У межах території сучасної України ареною бойових дій стали 7 областей.
Найбільше постраждали території сучасних Волині й Тернопільщини. Значних
збитків зазнали землі сучасних Львівської, Івано-Франківської,
Чернівецької і частини Рівненської областей.

Війна закінчилася перемогою Антанти – Франції, Великої Британії, США,
Італії, Японії та їхніх союзників. А Російська, Австро-Угорська, Німецька
та Османська імперії зазнали краху. Вкотре було перерозподілено світ,
виникла Ліга Націй (попередниця ООН).

\ifcmt
pic https://www.5.ua/media/pictures/820x546/200384.jpg
\fi

На мапі з'явилися Фінляндія, Естонія, Латвія, Литва, Польща,
Чехословаччина, Австрія, Угорщина, Югославія. На переважній частині
колишньої Російської імперії запанував комуністичний більшовицький режим. 

Перша світова дала поштовх до початку Української національної революції
1917-1921 рр. Постали УНР, Українська Держава, ЗУНР. Хоча їм не вдалося
втриматися надовго, українці здобули безцінний досвід державотворення.

\begin{leftbar}
	\bfseries
Повстанська боротьба тривала в Україні і після революції, представники
політичної та військової еліти в еміграції були носіями ідеї української
незалежності. На чужині діяв Державний Центр УНР, який у серпні 1992 р.
передав свої повноваження урядові незалежної України.
\end{leftbar}

Останні ветерани Першої світової відійшли у засвіти геть нещодавно – на
початку 2010-х рр. Це були канадець Джон Бебкок (1900-2010), американець
Френк Баклз (1901-2011), австралієць Клод Шулз (1901-2011) і британка
Флоренс Грін (1901-2012). Останнім учасником із-поміж українців вважається
уродженець Полтавщини \textbf{Мусій Кричевський}, який воював у війську Російської
імперії. Він помер у Донецьку 26 грудня 2008 на 112-му році життя.

\subsubsection{Яків Гандзюк}

Виходець із вінницьких селян Яків Гандзюк (1873-1918) – герой ще
японсько-російської війни 1904-1905 рр., де на полі бою вислужив спадкове
дворянство – під час Першої світової командував різними піхотними
частинами і з'єднаннями царської армії.

У боях 9 разів діставав поранення і контузії, зокрема тяжкі: шрапнеллю у
праву частину тулуба, кулею в кисть лівої руки навиліт, контузію важким
артилерійським снарядом із глибокою втратою свідомості. За бойові заслуги
був підвищений до генерал-майора і нагороджений Георгіївською зброєю.

\begin{leftbar}
	\bfseries
Президент Франції Раймон Пуанкаре вшанував його однією з найвищих
військових нагород своєї держави – Військовою медаллю (Médaille
militaire). З початком революції Гандзюк вступив до Армії УНР. У лютому
1918 р. його в Києві розстріляли більшовики. Ще живого генерала
по-звірячому добивали багнетами.
\end{leftbar}

\ifcmt
pic https://www.5.ua/media/pictures/820x546/200418.jpg
\fi

\subsubsection{Павло Скоропадський}

Нащадок давнього гетьмансько-старшинського роду Павло Скоропадський
(1873-1945) відзначився вже на початку війни.

19 серпня 1914 р. в бою під Краупішкеном (нині Ульяново Калінінградської
області РФ) він на чолі лейб-гвардії Кінного полку, попри жорстокий
артилерійський, кулеметний і гвинтівковий вогонь німців, захопив їхні
позиції та завдав противникові значних втрат. За це Скоропадського
нагородили орденом Святого Георгія IV ст.

\begin{leftbar}
	\bfseries
Під час революції 1918 р. він став Гетьманом Української Держави.
\end{leftbar}

\ifcmt
pic https://www.5.ua/media/pictures/original/200408.jpg
\fi

\subsubsection{Олександр Козаков}

Дворянин, уродженець Херсонщини Олександр Козаков (1889-1919) став
найкращим повітряним асом Російської імперії, особисто збивши 17 німецьких
і австрійських літаків. Кавалер російських, британських і французьких
орденів.

\begin{leftbar}
	\bfseries
31 березня 1915 р. в небі над Польщею здійснив другий (після Петра
Нестерова) в історії повітряний таран. Козаков не лише знищив німецький
аероплан, але й здійснив вдалу посадку – вперше у світовій практиці,
адже Нестеров під час тарану загинув.
\end{leftbar}

\ifcmt
pic https://www.5.ua/media/pictures/820x546/200407.jpg
\fi

\subsubsection{Софія Галечко}

Уродженка Лемківщини Софія Галечко (1891-1918) з початком війни
добровільно вступила до легіону Українських січових стрільців.

\begin{leftbar}
	\bfseries
Служила санітаркою, розвідницею, стрільцем, отримала звання фенріха
(хорунжої). Відзначилася у боях проти росіян у Карпатах зимою 1914-15
рр. та в бою за гору Маківку, була нагороджена срібною медаллю \enquote{За
хоробрість}.
\end{leftbar}

Трагічно загинула, потрапивши у вир під час купання в річці Бистриця
Надвірнянська на Івано-Франківщині.

\ifcmt
pic https://www.5.ua/media/pictures/820x546/200417.jpg
\fi

\subsubsection{Петро Франко}


Молодший син українського письменника Івана Франка, Петро Франко
(1890-1941) служив поручником УСС.

Відзначився в боях проти росіян під Семиківцями на Теребовлянщині восени
1915 р. Нагороджений офіцерською \enquote{Медаллю Воєнних Заслуг}.

\begin{leftbar}
	\bfseries
Під час Української революції був організатором авіації Галицької армії.
\end{leftbar}

У міжвоєнний час став викладачем, науковцем-хіміком, писав художні твори.

За основною версією, розстріляний радянськими спецслужбістами.

\ifcmt
pic https://www.5.ua/media/pictures/820x546/200416.jpg
\fi

\subsubsection{Ярослав Окуневський}
\index[names.rus]{Окуневський, Ярослав!(1860-1929)}

Виходець із родини буковинських священиків Ярослав Окуневський (1860-1929)
став адміралом, очолював медичну службу австро-угорського
військово-морського флоту.

Розробив медичний статут флоту, який, із невеликими змінами, чинний і
досі у багатьох європейських країнах.

Кавалер орденів Австро-Угорщини, Німеччини, Іспанії, Китаю.

Був волонтером. Жертвував значні кошти на легіон УСС, керував Українською
санітарною місією у Відні. 1919 р. організував поставки життєво необхідних
медикаментів для Армії УНР.

\ifcmt
pic https://www.5.ua/media/pictures/820x546/200415.jpg
\fi

\subsubsection{Пилип Коновал}
\index[names.rus]{Коновал, Пилип!(1888-1959)}


Селянин із Поділля Пилип Коновал (1888-1959) перед війною емігрував до
Канади. Був добровольцем Канадського експедиційного корпусу.

\begin{leftbar}
	\bfseries
Протягом року брав участь у \enquote{траншейній війні} на півночі Франції,
отримав звання капрала. Здійснив подвиг у кінці серпня 1917 р. в бою
поблизу міста Ланс особисто \enquote{зачистив} два кулеметних гнізда, вбив
щонайменше 16 німців, захопив один кулемет, а другий знищив гранатами.
Цим самим надав можливість своєму батальйону продовжити наступ.
\end{leftbar}

Удостоєний Хреста Вікторії – найвищої військової нагороди Великої
Британії, яку йому в Лондоні особисто вручив король Георг V (дідусь
Єлизавети ІІ). Єдиний в історії кавалер Хреста Вікторії родом зі Східної
Європи. Меморіальні дошки на честь українця встановлені в Канаді і
Франції. У Лансі 2017 р. відкрили Алею Коновала.

\ifcmt
pic https://www.5.ua/media/pictures/820x546/200414.jpg
\fi

\subsubsection{Бонус: два цікавих факти}

\begin{itemize}
	
\item  Німеччина завершила виплату репарацій, передбачених Версальським
договором 1919 р. після Першої світової, аж 3 жовтня 2010 р. Тоді
Берлін перевів останню велику суму – 70 млн євро. Більшу частину
грошей отримали Франція і Бельгія.

Версальський договір, складений країнами-переможцями, зокрема США,
Францією, Великобританією та Бельгією, німці підписали 28 червня 1919 р.
Він офіційно поклав край Першій світовій. Крім повернення низки територій
і своїх колоній переможцям, німці зобов'язалися виплатити репарації на
суму 226 млрд рейхсмарок. До 1952 р. борг частково погасили. Після розділу
території Німеччини німецькі платежі було припинено до її возз'єднання.

Зазначається, що за даними ВВС, величезні репарації, накладені переможцями
Першої світової, частково стали причиною розвалу німецької економіки та
приходу до влади Адольфа Гітлера, який відмовився їх виплачувати.

\item  Британський танкіст малював битви, у яких брав участь, і залишив
детальний щоденник. Лейтенант британської королівської армії Кеннет
Едвін Вуттон був танкістом Західного фронту.

\begin{leftbar}
	\bfseries
Вуттон брав участь і у Іпрській битві (літо 1917 року), де загинуло пів
мільйона солдатів з обох боків. Ось що він писав про це:
\end{leftbar}

\enquote{Ми їхали крізь розриви, але довкола нас тільки падали шматки землі.
Німецькі позиції були розтрощені до невпізнання – ми долали вирву за
вирвою, і майже всі вони були заповнені багнюкою і людськими тілами.
Доїхавши до вершини пагорба, ми виявили, що німці досі утримують ліс. Я
бігав, ховаючись за танком, поки довкола рвалися снаряди. Варто було б
залізти всередину, але я ненавидів задушливу тисняву машини, мені
здавалося безпечніше назовні}.

Лейтенант також описав Різдвяне перемир'я 1914 р., коли командування з
обох сторін намагалося подолати раптовий вибух людяності в солдатах і
погрожувало жорстокими покараннями за будь-яку спробу братання з ворогом.

\begin{leftbar}
	\bfseries
А ось його спогади про Різдво 1916-го. Продуктами вже не ділилися, але й
не стріляли один в одного:
\end{leftbar}

\enquote{У нас був чудовий стіл на Святвечір у бліндажі – індичка, різдвяний
пудинг, пиріг, фрукти і шампанське. Обидві сторони не стріляли. Патрулював
із півночі до третьої, у святковому настрої}.

\ifcmt
tab_begin cols=3
	caption Малюнки Кеннета Вуттона, Фото: istpravda.com.ua
	pic https://www.5.ua/media/pictures/1140x641/200406.jpg
	pic https://www.5.ua/media/pictures/original/200405.jpg
	pic https://www.5.ua/media/pictures/1140x641/200404.jpg
tab_end
\fi

Останні записи в щоденнику зроблені в госпіталі в Руані, куди
важкопоранений Кеннет Вуттон потрапив після вибуху. Потім сліди його
загубилися. Аж 2010 року щоденник випадково знайшла правнучка лейтенанта.

\end{itemize}

За інформацією джерел \enquote{Історична правда} та \enquote{Канал 24}

Читайте також: Чому Москві не можна вірити: урок гетьмана Івана
Брюховецького

\paragraph{Читайте за темою}

\begin{itemize}
	
\item Без \enquote{корінних зубів} і \enquote{виручки}: росіянізми в українській мові, яких \enquote{совєти}
нас змусили не помічати

\item Як жили кияни під час нацистської окупації – 19 унікальних світлин столиці
1941-1943 рр.

\item \enquote{Всюди культ великоросійського преобладанія}: гнівний лист до ЦК КПУ –
				чому його автор спалив себе на Хрещатику
\end{itemize}

