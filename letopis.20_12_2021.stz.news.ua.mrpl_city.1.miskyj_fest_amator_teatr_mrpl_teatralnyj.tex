% vim: keymap=russian-jcukenwin
%%beginhead 
 
%%file 20_12_2021.stz.news.ua.mrpl_city.1.miskyj_fest_amator_teatr_mrpl_teatralnyj
%%parent 20_12_2021
 
%%url https://mrpl.city/blogs/view/miskij-festival-amatorskih-teatriv-mariupol-teatralnij
 
%%author_id demidko_olga.mariupol,news.ua.mrpl_city
%%date 
 
%%tags 
%%title Міський фестиваль аматорських театрів "Маріуполь театральний"
 
%%endhead 
 
\subsection{Міський фестиваль аматорських театрів \enquote{Ма\hyp{}ріуполь театральний}}
\label{sec:20_12_2021.stz.news.ua.mrpl_city.1.miskyj_fest_amator_teatr_mrpl_teatralnyj}
 
\Purl{https://mrpl.city/blogs/view/miskij-festival-amatorskih-teatriv-mariupol-teatralnij}
\ifcmt
 author_begin
   author_id demidko_olga.mariupol,news.ua.mrpl_city
 author_end
\fi

Щорічний міський фестиваль аматорських колективів \enquote{Маріуполь театральний}, що
проводиться за підтримки Департаменту культурно-громадського розвитку
Маріупольської міської ради, завершено. Час дізнатися його результати. Головна
мета фестивалю – популяризація творчості аматорських театральних колективів
міста, на мою думку, досягнута. Цьогоріч фестиваль довів, що попри всі
карантині обмеження він є важливим і затребуваним. Директорка Департаменту
громадсько-культурного розвитку Маріупольської міської ради \emph{\textbf{Діана Трима}}
зазначила, що 

\begin{quote}
\em\enquote{фестиваль проводиться одинадцятий рік і на нього дуже чекають.
Кількість заявлених  вистав зросла. У нас навіть залишається та ж сама
кількість режисерів, але зростає кількість колективів, з якими працює цей
режисер}. 
\end{quote}

Директорка Центру культури \enquote{Лівобережний} \emph{\textbf{Тетяна Обєдкова}} підкреслила, що 

\begin{quote}
\em\enquote{такі
фестивалі дуже потрібні. Кожен аматорський колектив хоче бачити себе на великій
сцені. І ми докладаємо всіх зусиль, щоб рівень цих колективів відповідав
великим сценам}.
\end{quote}

Оскільки колективи, які стали учасниками фестивалю, є не зовсім рівними за
своїм складом, віком та виконавським рівнем організатори вирішили відмовитися
від оцінювання театрів. Головним суддею став глядач, адже саме завдяки
кількості відвідувачів можна визначити, який з аматорських театрів є найбільш
популярним і затребуваним у публіки. Всі театри отримали дипломи учасників та
цінні подарунки, які будуть корисними для роботи театрального колективу. Вже
другий рік поспіль Департамент культурно-громадського розвитку Маріупольської
міської ради намагається робити унікальні незвичні подарунки всім колективам.
Їх створюють на замовлення з логотипом фестивалю.

Основними локаціями фестивалю стали Центр сучасного мистецтва \enquote{Готель
Континенталь} та Маріупольський ляльковий театр. Начальниця
культурно-освітнього відділу Департаменту\par\noindent культурно-громадського розвитку
Маріупольської міської ради \emph{\textbf{Євгені Бузук}} пояснила, чому вирішили обрати лише
дві локації. Вона наголошує, що

\begin{quote}
\em\enquote{у кожного фестивалю є свої вимоги. І приїхати
грати на невідому сцену – для кожного колективу – випробування. Втім повинен
з'явитися азарт, театр намагатиметься зробити касу і зіграти краще та
потужніше}.
\end{quote}

На фестиваль Міські аматорські театри, серед яких Перша театральна
школа-студія, театр авторської п'єси  \enquote{Conception}, народний театр Театроманія,
театр-студія \enquote{Фенікс}, Театр юного актора, Маріупольський театр ляльок, театр
\enquote{Тезіс}, Театральна артіль \enquote{Драмком} і театр \enquote{Грані} підготували зовсім різні
вистави – драми, комедії, перформанси, а також дитячі спектаклі. Деякі з них
встигли полюбитися глядачам, а деякі – ще зовсім \enquote{свіжі}. Загалом у підготовці
до фестивалю взяло участь понад 100 акторів. Для кожного колективу важливо
представити свою творчість і почути думки глядачів.

Першим представив свою виставу Зразковий театр \enquote{Грані}. Режисерка \emph{\textbf{Наталя
Данілова}}, талановита педагогиня, яка зуміла прищепити юним маріупольцям любов
до театрального мистецтва. Участь у подібних заходах для колективу дуже
потрібна і важлива, адже через карантин Зразковий театр \enquote{Грані} вже давно не
представляв свою творчість. 16 юних акторів віком від 10 до 16 років на чолі з
режисеркою підготували яскраву прем'єру – казку \enquote{Маленька баба Яга}.

\ii{20_12_2021.stz.news.ua.mrpl_city.1.miskyj_fest_amator_teatr_mrpl_teatralnyj.pic.1}

Театр авторської п'єси \enquote{Концепція} представив веселу легку, динамічну виставу
\enquote{Сільські комедії}. Художній керівник і режисер колективу \emph{\textbf{Олексій Гнатюк}}
заснував ще один колектив – театр юного актора \enquote{Тезіс} (режисерка \emph{\textbf{Ганна
Лафазан}}), який теж став учасником фестивалю і виступив зі спектаклем \enquote{Лев,
чаклунка та платтяна шафа}.

\ii{20_12_2021.stz.news.ua.mrpl_city.1.miskyj_fest_amator_teatr_mrpl_teatralnyj.pic.2}

Учні першої театральної школи та актори Театру юного актора за участю \textbf{Ірини
Анатоліївни Руденко}, \textbf{Михайла Загреба} та актриси Народного театру \enquote{Театроманія}
\emph{\textbf{Марії Бойко}} виступили з виставою \enquote{Аляска}, присвячену пошуку щастя. Це дійство
підготували в рамках проєкту  Марафон міжнародних резиденцій. Її режисер
Евангелос Космідіс наголосив, що з моменту прем'єри змінив деякі акценти та
зробив спектакль ще більш цілісним. Цікаво, що \emph{\textbf{Евангелос Космідіс}} буде й надалі
ставити вистави з акторами театру юного актора та театру ляльок, адже режисер
налагодив контакт з усіма дітьми і відпускати його не хочуть.

\ii{20_12_2021.stz.news.ua.mrpl_city.1.miskyj_fest_amator_teatr_mrpl_teatralnyj.pic.3}

Наразі вже відомо, яка вистава мала найбільший попит. Це спектакль Театру
ляльок \enquote{Як Петрушка вчитись не хотів} (художня керівниця і режисерка –
заслужена діячка естрадного мистецтва України – \emph{\textbf{Ірина Руденко}}).

Для кожного режисера цей фестиваль є цінним і корисним досвідом. Зокрема,
художній керівник і режисер Народного театру \enquote{Театроманія} \emph{\textbf{Антон Тельбізов}}
наголоси, що 

\begin{quote}
\em\enquote{цей фестиваль для нас – це дуже знакова театральна подія, бо
хочеться подивитися і на інші колективи, і також показати свій рівень}.
\end{quote}

Колектив \enquote{Театроманії} показав постановку \emph{\textbf{Альони Горгоц}}
\enquote{У відкритому морі} за п'єсою Славомира Мрожека. Сюжет, на перший
погляд, простий: троє елегантно одягнених чоловіків дрейфують на плоту у
відкритому морі. Дивом переживши корабельну аварію, вони залишилися практично
без провізії. Коли їстівні запаси добігають кінця, перед джентльменами постає
питання, як вижити. Акторам Народного театру Театроманія було нелегко працювати
з таким серйозним матеріалом, але для них це дуже цікавий і цінний досвід.

Театр юного актора (художня керівниця і режисерка – заслужена діячка естрадного
мистецтва України – \emph{\textbf{Ірина Руденко}}) представив виставу \enquote{Ввечері третього
дня} за розповідями Надії Тефі та Аркадія Аверченка. Дія відбувається в одній
із дворянських садиб, адже раніше не було оздоровчих таборів, батьки привозили
всіх дітей в садибу і залишали на гувернантку та нянечку. акторам  було не
зовсім легко зануритися в минулу епоху, адже і постава, і мелодика мови, і
загалом виховання були в той час зовсім іншими.

\ii{20_12_2021.stz.news.ua.mrpl_city.1.miskyj_fest_amator_teatr_mrpl_teatralnyj.pic.4}

Театр \enquote{Фенікс} (режисер – \emph{\textbf{Ігор Курашко}}) виступив з виставою \enquote{А ви буваєте в
театрі?}. Події спектаклю розгортаються на початку XX ст. Вистава створена за
оповіданнями відомого російського драматурга Аркадія Аверченка, розповідає про
театр, життя артистів та жінок, які прагнуть бути потрібними і коханими. І
режисер, і актори сподіваються, що яскравий комедійний сюжет сподобається їхнім
глядачам.

\ii{20_12_2021.stz.news.ua.mrpl_city.1.miskyj_fest_amator_teatr_mrpl_teatralnyj.pic.5}

Останнього дня фестивалю свою виставу \enquote{Гра} представив колектив Театральної
артілі \enquote{Драмком} (художня керівниця та режисерка – \emph{\textbf{Наталя Гончарова}}). Спектакль
поставлений за мотивами п'єси британського журналіста і автора детективних
розповідей Ентоні Шаффера. Літній аристократ, процвітаючий автор детективних
романів, який звик сприймати життя як гру і граючи принижувати людей, не
очікував зустріти гідного суперника, здатного відповісти тією ж монетою. І в
цій непростій історії, звичайно ж, замішана жінка... У виставі задіяно всього
два актори. Спектакль тримає в напрузі до самого кінця і піднімає важливі
актуальні соціально-психологічні теми.

Для режисерів та акторів цей фестиваль є унікальною можливістю яскраво
представити свій колектив і розширити свою аудиторію. Зокрема, учень першої
театральної школи т артист Театру Юного актора \emph{\textbf{Євген Мінаєв}} поділився, що дуже
хоче, щоб про театр дізналося більше жителів Донецькій області і, щоб після
фестивалю колектив зміг працювати ще потужніше.

\ii{20_12_2021.stz.news.ua.mrpl_city.1.miskyj_fest_amator_teatr_mrpl_teatralnyj.pic.6}

Організатори сподіваються, що наступного року фестиваль матиме  продовження.
Директорка Департаменту культурно-громад\hyp{}ського розвитку Маріуполя \emph{\textbf{Діана Трима}}
наголосила, що вона \emph{дуже радіє, що, попри карантинні обмеження, фестиваль все ж
вдалося провести}. Всі учасники, які були заявлені, змогли взяти участь у
фестивалі. Всього на фестивалі виступили 9 колективів, серед них: З дитячих
колективи, 3 підліткових та 3 дорослих. На усіх виставах були повні зали. Майже
повністю розпродані квитки. Це свідчить про високий інтерес містян до
аматорського театрального мистецтв Маріуполя. Діана Володимирівна, вважає, що
цього року фестиваль відбувся набагато краще, ніж раніше і сподівається, що
наступний міський фестиваль аматорських театрів \enquote{Ма\hyp{}ріуполь театральний} стане
ще більш масштабним та яскравим.

\emph{Фото Євгена Сосновського, Народного театру \enquote{Театроманія}, театру \enquote{Фенікс} та Ольги Демідко.}  
