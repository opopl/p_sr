% vim: keymap=russian-jcukenwin
%%beginhead 
 
%%file slova.karta
%%parent slova
 
%%url 
 
%%author 
%%author_id 
%%author_url 
 
%%tags 
%%title 
 
%%endhead 
\chapter{Карта}

%%%cit
%%%cit_head
%%%cit_pic
\ifcmt
  pic https://avatars.mds.yandex.net/get-zen_doc/4888095/pub_60bdf74e7b0ba72b6fd0d2cb_60be012bbcbf42494e3b1c20/scale_1200
\fi
%%%cit_text
Хочу добавить к перечисленному \emph{карту} Африки. Хорошо очерченное
побережье, с указанием городов.  В Сахаре достаточно много небольших озёр и
рек. Обозначен пролив Гибралтар.  То есть \emph{карта} написана не из фантазий
автора, а по, достаточно точным, измерениям.  В Африке в это время идёт борьба
с работорговлей против европейских колонизаторов. Как и во всём мире, в Африке
16 века сформировано несколько государств. Развита металлургия, животноводство
и земледелие.  Но с приходом европейских колонизаторов эти государства пришли в
упадок
%%%cit_comment
%%%cit_title
\citTitle{Карты 1570 года. Тартария, Америка и материк на Северном Полюсе}, 
Нешкольная История, zen.yandex.ru, 08.06.2021
%%%endcit
