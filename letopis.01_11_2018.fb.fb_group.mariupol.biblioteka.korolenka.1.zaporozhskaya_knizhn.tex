%%beginhead 
 
%%file 01_11_2018.fb.fb_group.mariupol.biblioteka.korolenka.1.zaporozhskaya_knizhn
%%parent 01_11_2018
 
%%url https://www.facebook.com/groups/1476321979131170/posts/1939047669525263
 
%%author_id fb_group.mariupol.biblioteka.korolenka,lunina_tetjana.mariupol
%%date 01_11_2018
 
%%tags mariupol,kniga
%%title Запорожская книжная толока - 2018
 
%%endhead 

\subsection{Запорожская книжная толока - 2018}
\label{sec:01_11_2018.fb.fb_group.mariupol.biblioteka.korolenka.1.zaporozhskaya_knizhn}
 
\Purl{https://www.facebook.com/groups/1476321979131170/posts/1939047669525263}
\ifcmt
 author_begin
   author_id fb_group.mariupol.biblioteka.korolenka,lunina_tetjana.mariupol
 author_end
\fi

Вернувшись в родной город, сотрудники \#Центральной\_библиотеки\_им\_В\_Г\_Короленко,
спешат поделиться с Вами своими впечатлениями от посещения ежегодной
\#Запорожской\_книжной\_толоки - 2018, которая проходила в выставочном центре
\enquote{Козак – Палац} города Запорожье.

Сложно передать словами эмоции библиотекаря, который попал в зал заставленный
столами, каждый из которых манит к себе пестрыми книгами. Глаза разбегаются в
разные стороны и ты словно растворяешься в толпе таких же любителей книг.
Каждая книга так и призывает тебя взять ее в свои руки и хоть на минуту
окунуться в ее историю. Но отдельным подарком для поклонников украинской
литературы от организаторов, стала возможность пообщаться и взять автограф у
своего любимого писателя. Ведь книжную толоку в этом году посетили многие
талантливые представители украинской литературы, такие как: Андрей Кокотюха,
Ирен Роздобудько, Галина Вдовиченко, Вахтанг Кебуладзе, Тарас Лютый и это еще
не весь список известных нам имен.

Программа книжной толоки была наполнена интересными встречами и обсуждениями.
Сотрудники Центральной библиотеки им. В.Г. Короленко побывали на форумах
менеджеров культуры, которые были посвящены актуальным темам в сфере нашей
деятельности. Своим опытом поделились представители разных городов Украины и
директор фестиваля Игорь Гармаш.

Конечно же мы не могли пропустить встречу с одной из самых известных и читаемых
украинских писательниц – Ирен Роздобудько, которая презентовала на толоке свою
новую книгу «Прилетіла ластівочка». И теперь эта книга с автографом автора,
станет приятным дополнением на книжной полке сектора абонемента.

Еще одним значимым событием стала встреча с талантливым и харизматичным автором
многих интересных произведений, нашедших немало своих поклонников среди
читателей библиотеки – Андреем Кокотюхой. Хочется отметить что это человек,
имеющий четкие позиции и умеющий отстаивать свою точку зрения. Это автор,
который использует в своих книгах исторические факты, создавая таким образом не
просто увлекательный роман, но еще и познавательную книгу. 

Подводя итоги поездки хочется отметить, что книжная толока в Запорожье покорила
своим разнообразием программы, наполненностью и информативностью, а книголюбам
пожелать не упустить возможность побывать на ней в следующем году.
