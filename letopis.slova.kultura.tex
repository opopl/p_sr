% vim: keymap=russian-jcukenwin
%%beginhead 
 
%%file slova.kultura
%%parent slova
 
%%url 
 
%%author 
%%author_id 
%%author_url 
 
%%tags 
%%title 
 
%%endhead 
\chapter{Культура}
\label{sec:slova.kultura}

%%%cit
%%%cit_pic
%%%cit_text
Оскільки було силове впровадження росмовної \emph{субкультуру}, має бути силове витіснення її з України
%%%cit_comment
Ярослав Трінчук
%%%cit_title
\citTitle{...ми ще на стадії формування українського музичного ринку - відвойовуємо його у Росії}, 
Про Мову, facebook, 18.06.2021
%%%endcit

%%%cit
%%%cit_head
%%%cit_pic
\ifcmt
  pic https://img.vz.ru/upimg/m11/m1104023.jpg
\fi
%%%cit_text
Когда Путин в послании Федеральному собранию сказал фразу о Киплинге, которая
сразу стала мемом, он, с одной стороны, отсылал, понятно, к нравам «Книги
джунглей», прозрачно намекая на ситуацию в современной международной политике.
Но, с другой стороны (и этот более глубокий месседж также, думаю, был считан
всеми, кому он предназначался), реагировал на ту беспрецедентную атаку на
традиционную христианскую европейскую \emph{культуру}, которую мы наблюдали весь 2020
год, и которая сегодня продолжает лавинообразно нарастать
%%%cit_comment
%%%cit_title
\citTitle{Россия должна жить по заветам Киплинга}, 
Владимир Можегов, vz.ru, 18.06.2021
%%%endcit

%%%cit
%%%cit_head
%%%cit_pic
%%%cit_text
Не менее сложной и весьма затратной задачей является суверенизация
информационного пространства и установление государственного контроля над ним.
Феномен \emph{культурного влияния} на общество через социальные сети, видеоплатформы,
которые ранее заявлялись «свободными», а оказались четко регулируемыми и жестко
модерируемыми инструментами информационного и \emph{культурного воздействия}, доказал,
что власти Китая были полностью правы, когда распространения такого влияния на
своей территории не допустили
%%%cit_comment
%%%cit_title
\citTitle{Революция Духа – единственный путь спасения России}, 
Юрий Барбашов, voskhodinfo.su, 30.06.2021
%%%endcit

%%%cit
%%%cit_head
%%%cit_pic
%%%cit_text
Результаты незаконченной \emph{культурной вестернизации} России в 19-начале 20 века
были полностью уничтожены во времена сталинской деспотии. Начиная с хрущевских
реформ, страна опять стала открываться миру, \emph{культурная} вестернизация
постепенно возобновилась.  Новой попыткой движения в Европу стала Перестройка.
\enquote{Жить как на Западе} - это в то время была идея-фикс советских людей
%%%cit_comment
%%%cit_title
\citTitle{Как разведенный диктатор-отравитель собирается защищать \enquote{традиционные ценности}}, 
Игорь Эйдман, opinions.glavred.info, 06.07.2021
%%%endcit

%%%cit
%%%cit_head
%%%cit_pic
%%%cit_text
Ключовим тут є слова «\emph{одна культура}». У свідомості президента українська
культура за імперської доби не існувала окремо від російської, \emph{культура} була
спільною для обох народів. Фраза про популярність в Росії «неймовірної
української літератури» належить до порожньої, позбавленої сенсу риторики, бо
українськими творами цікавились у РСФСР хіба що поодинокі українці, що там
мешкали
%%%cit_comment
%%%cit_title
\citTitle{Українська мова в культурному просторі держави. Протистояння триває}, 
Лариса Масенко, www.radiosvoboda.org, 11.07.2021
%%%endcit

%%%cit
%%%cit_head
%%%cit_pic
%%%cit_text
Строить правовую, а не моноэтническую страну.  Современная политическая
ситуация в Украине ставит перед миллионами русскоязычного населения непростую
дилемму. Либо сохранить свой язык и \emph{культуру}, но предать украинское
государство, стать гражданами другой страны, в данном случае России, тем более
что из Украины их выталкивают, а в Россию приглашают. Либо, если они не хотят
предавать Украину, то должны забыть свой язык, свою историю, свою \emph{культуру} и
быть до конца покорными украинской власти, а что самое страшное – быть вместе с
радикалами и националистами, с их красно-черными знаменами, атаками на русский
язык, нетерпимостью к «неправильному православию», оправданием нацизма и
презрением ко всему советскому без разбора, будь то признание подвига
советского народа в победе в Великой Отечественной войне или понимание, что
весь нынешний промышленный потенциал Украины был построен советскими людьми в
советское время по приказу советской власти. Радикалы и власть имущие провели
четкую границу: все хорошее – например, успехи «Динамо» (Киев), полет в космос
Поповича, творчество композитора Ивасюка, самолет «Мрия» – украинское, а все
плохое – Чернобыльская трагедия или Голодомор – наследие советской эпохи. Дошло
до того, что украинские власти заявили, что Освенцим освободили войска Первого
Украинского фронта, не упомянув, что это был один из фронтов Красной армии
%%%cit_comment
%%%cit_title
\citTitle{О будущем украинского и русского народов}, 
Виктор Медведчук, strana.ua, 15.07.2021
%%%endcit

%%%cit
%%%cit_head
%%%cit_pic
%%%cit_text
Чому протягом 2017–2020 рр. Держкомтелерадіо дозволив ввезти в Україну 31 тис.
найменувань книг з Росії? Нібито більшість із дозволених книг – це твори
світової класики, дитяча та науково-популярна література. І все ж, подібна
\enquote{нейтральна} експансія нічого доброго не принесе в Україну, оскільки навіть
такий книжковий продукт РФ, що нібито не є загрозою національній безпеці, є
сутнісним складником політики \enquote{руського міра}. Адже \enquote{закохані} в російську
мову, літературу, \enquote{велікую \emph{культуру}}, \enquote{історіческоє велічіє} українські
громадяни не є носіями українськості, а радше їх світогляд формується в ніші
панросійськості. Саме цей людський капітал (матеріал), створений поза
українською мовою та \emph{культурою}, кремлівці розглядають як плацдарм для своїх
майбутніх агресій і проголошення сурогатних лялькових держав типу
\enquote{придністров’я, абхазії, північної осетії, лнр чи днр}. Гуманітарна зачистка на
території України через мову, освіту, інформацію, книговидавництво, концертну
діяльність, кіно, телеканали, радіо-FM, компрадорську інтелігенцію та
чужомовних владоможців, що і донині продовжують упевнено \enquote{правити} слухняним і
мовчазним народом, і є тією найбільш небезпечною і безупинною війною проти
України
%%%cit_comment
%%%cit_title
\citTitle{Правда в Рідному Слові}, Георгій Філіпчук, slovoprosvity.org, 12.07.2021
%%%endcit

%%%cit
%%%cit_head
%%%cit_pic
%%%cit_text
Активная \emph{культурная} жизнь – один из показателей жизнеспособности
государства. С этим в Донецкой народной республике все в порядке, здесь
развитию \emph{культуры} и искусства уделяют пристальное внимание.  В честь Дня
города, Дня шахтера и своего 10-летнего юбилея художественный музей
«Арт-Донбасс» проведет самую масштабную выставку из фондовой коллекции музея
«Квинтэссенция АРТа».  На выставке будет представлено более 250 работ донецких
художников, за практически семидесятилетний период, с начала 50-х годов XX
века, до наших дней, многие из них будут выставлены на суд зрителей впервые. По
результатам выставки будет издан каталог, в который войдут изображения всех
работ, представленных на вернисаже
%%%cit_comment
%%%cit_title
\citTitle{Донецк отметит День города культурными событиями}, , voskhodinfo.su, 26.08.2021
%%%endcit

%%%cit
%%%cit_head
%%%cit_pic
%%%cit_text
— Он будет в порядке, — сказал Джоунз. — Свои вопросы он еще сможет задать
позже. Он уже понял, что здесь нечто большее.  — Хорошо, — согласился Сторож. —
Мы могли бы создать величественную \emph{культуру}, единственную в Галактике, а может
и во всей Вселенной. Потому что у нас у первых появился разум и мы начали
намного раньше остальных. Мы могли бы создать образ жизни, превосходящий всякое
воображение, но среди нас в древние времена нашлись мудрецы, которые поняли,
что если мы пойдем таким путем, то окажемся в одиночестве, в изоляции от
остальных разумных существ. Мы должны были принять решение, и оно было принято.
Мы решили жить не для себя, а для других разумов, которые могли возникнуть в
Галактике
%%%cit_comment
%%%cit_title
\citTitle{Зачарованное паломничество}, Клиффорд Саймак
%%%endcit

%%%cit
%%%cit_head
%%%cit_pic
%%%cit_text
Сам Шевченко никогда не разграничивал малороссов и русских. Нигде и никогда не
назвал себя «украинцем». Поэт прекрасно знал русский язык, литературу и
\emph{культуру} в целом, которые были полноправными преемницами древнерусского языка и
\emph{культуры}. Большая часть прозы Шевченко, а также часть стихов написаны на
русском языке. Южнорусский поэт был «продуктом» именно русской культуры.
Представители \emph{русской культуры} (Жуковский, Брюллов, Григорович) и другие
помогли ему освободиться от крепостной неволи, стали учителями, помогли стать
на ноги. Сам Шевченко был частью столичной интеллигенции. В итоге поэт никогда
не разделял «Милую Украйну» и Россию. Даже в своём дневнике свою родину он
только пару раз называет Украиной, а остальных случаях Малороссией
%%%cit_comment
%%%cit_title
\citTitle{Шевченко без украинизма}, , topwar.ru, 09.03.2019
%%%endcit
