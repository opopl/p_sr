%%beginhead 
 
%%file 25_08_2021.fb.fb_group.mariupol.biblioteka.korolenka.1.prem_rna_vistava__m
%%parent 25_08_2021
 
%%url https://www.facebook.com/groups/1476321979131170/posts/4226561587440515
 
%%author_id fb_group.mariupol.biblioteka.korolenka,shulga_inna.mariupol
%%date 25_08_2021
 
%%tags mariupol,kultura,teatr,teatr.mariupol.drama
%%title Прем'єрна вистава "Маруся" Донецького академічного обласного драматичного театру
 
%%endhead 

\subsection{Прем'єрна вистава \enquote{Маруся} Донецького академічного обласного драматичного театру}
\label{sec:25_08_2021.fb.fb_group.mariupol.biblioteka.korolenka.1.prem_rna_vistava__m}
 
\Purl{https://www.facebook.com/groups/1476321979131170/posts/4226561587440515}
\ifcmt
 author_begin
   author_id fb_group.mariupol.biblioteka.korolenka,shulga_inna.mariupol
 author_end
\fi

В рамках святкування 30-річчя Незалежності України бібліотекарі Центральної
міської публічної бібліотеки ім. В. Г. Короленка разом з маріупольцями, які
мають функціональні обмеження по зору, побували на прем'єрній виставі «Маруся»
Донецького академічного обласного драматичного театру. 

Всі події, що відбувалися на сцені, тифлокоментувала співробітниця бібліотеки.
Коментування здійснювалося через спеціальні навушники, що дозволяло «дивитися»
спектакль нарівні з усіма глядачами. Коментатор, який знаходився в окремій ложі
бельетажу, вбудовувався в паузи між репліками, щоб не перекривати голоси
акторів. Таким чином коментування не заважало глядачам насолоджуватися
поетичними текстами Ліни Костенко, піснями Марусі Чурай та українськими
обрядовими танцями.
