% vim: keymap=russian-jcukenwin
%%beginhead 
 
%%file 25_10_2019.news.pravda_com_ua.SPARTA.psih_bolnica_KGB_bratstvo
%%parent 25_10_2019.news.pravda_com_ua.SPARTA
 
%%url 
%%author 
%%tags 
%%title 
 
%%endhead 

\subsubsection{Психбольница, КГБ и ``Братство Кандидатов в Люди''}

Будущий основатель "С.П.А.Р.Т.Ы." Юрий Давыдов родился в Саранске в 1954-м. До
16 лет жил в городе Веневе под Тулой, окончил училище по специальности
тракторист-механизатор. В 1973-м был призван в армию, служил в Латвии и в
Литве.

\ifcmt
img_begin 
	url https://img.pravda.com/images/doc/e/8/e8580d7-18sparta.jpg
	caption Прежде, чем создать свое движение, будущий основатель "С.П.А.Р.Т.Ы." выучился на тракториста-механизатора
	width 0.7
img_end
\fi

"Юра говорил: "Я жил в шести республиках, в 15 населенных пунктах", –
рассказывает Тамара. --- На месте не сидел, потому что папа был
политрепрессирован, удерживался в ГУЛАГе в Кемерово с 1942 по 1952 годы.
Реабилитировали только в 1956-м --- мать ездила в Кремль, добивалась. 

Но джугашвилинисты никуда не делись, давление на политзаключенных было, и
родители частенько перемещались. Когда в армию шел, спросил военного комиссара,
что он может сделать для Родины, а тот сказал ему: "Разработай Теорию Счастья,
чтобы воспитать нового советского человека". С того момента он начал по
кусочкам собирать ее".

В 1978-м после возвращения из армии Давыдов поступил учиться на
промышленно-гражданское строительство в Мордовский университет в Саранске
заочно. Параллельно менял работы и города --- работал обрубщиком в Саранске,
докером в Клайпеде, станкостроителем в Харькове. 

Помимо учебы и работы занимался спортом: боксировал в Саранске, прыгал с
парашютом в Харькове, бегал марафоны в Ростовской области. В 1980-м женился, но
брак продлился всего год.

"Юра придерживался теории, что дамочек нужно проверять на верность, –
утверждает Тамара. --- Она жила в Саранске, он расписался, уехал в Харьков и
проверял, что она будет без него делать. Из этого у него родились исследования
"История Одной Любви", "История Шаблонной Любви" и "История Потребительской
Любви". На уровне бытовой семьи мыслителю, у которого глобальные мечты и цели,
невозможно реализоваться".

В это время Давыдов также составил первую версию опросника "Знаете ли Вы
окружающий мир и себя? Можете ли Вы построить собственное Счастье?", в который
в итоге войдет 1500 вопросов. Сегодня этим вопросником в "С.П.А.Р.Т.Е."
тестируют новичков.

"Перворазнику" даем анкету --- "откровенник" из 50 вопросов, --- объясняет Тамара.
– Как только ты его заполнил --- условно за час --- тебе выдается "Методика
заполнения теста", чтобы подготовить к заполнению опросника.

Методика расписана на 28 страницах текста. Был период, когда предлагалось за
два дня ее скопировать от руки --- уровень самопожертвования проверялся. Затем
дается опросник из 1500 вопросов. Ты его переписываешь, пусть месяц на это
уйдет, и дальше заполняешь ответник --- тоже условно за месяц. Получается
трансформация мышления за 90 дней".

Одновременно с социологическими исследованиями Давыдов работал над различными
техническими инновациями. Тем не менее, его эксцентричные идеи пришлись не по
нраву советской власти и в 1984-м его отправили в психбольницу.

"Кандидат в Настоящие Великие Люди придумал проект "Лужок", чтобы улучшить
снабжение жителей Киевского района Харькова, и презентовал его в обкоме, –
вспоминает Тамара. 

В обкоме вызвали медбратьев и те нагрянули на квартиру к его отцу. Юра сел на
поезд и поехал в Мордовию. Во второй раз они приперлись на квартиру к Юре.
КГБшники просили его подписать бумажку, что он не будет заниматься
социологическими исследованиями. Юра не подписал и его на три месяца поместили
в харьковскую психушку".

\ifcmt
img_begin 
	url https://img.pravda.com/images/doc/1/5/1540df0-2sparta.jpg
	caption Советская власть пыталась лечить основателя движения Юрия Давыдова в психбольнице
	width 0.7
img_end
\fi

Выйдя из лечебницы, Давыдов продолжил исследовать человеческую природу и
окружающий мир. К концу 80-х он написал шесть томов "Теории Счастья" –
причудливый симбиоз философии, поэзии и математики. 

Одновременно с этим работал над своей физической формой --- преодолевал
100-километровые сверхмарафоны, бегал босиком по углям, практиковал раджа- и
бхакти-йогу, занимался каратэ. Искал единомышленников и в 1987-м основал
предтечу "С.П.А.Р.Т.Ы." --- организацию "Б.К.Л." --- "Братство Кандидатов в
Люди".

