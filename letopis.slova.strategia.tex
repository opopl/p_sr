% vim: keymap=russian-jcukenwin
%%beginhead 
 
%%file slova.strategia
%%parent slova
 
%%url 
 
%%author 
%%author_id 
%%author_url 
 
%%tags 
%%title 
 
%%endhead 
\chapter{Стратегия}

%%%cit
%%%cit_head
%%%cit_pic
\ifcmt
  pic https://odnarodyna.org/sites/default/files/styles/adaptive/public/article/2021/%D0%A1%D1%83%D1%89%D0%BD%D0%BE%D1%81%D1%82%D0%B8%20%D0%BC%D0%BD%D0%BE%D0%B6%D0%B8%D1%82.jpg?itok=ddZnY3nV
	width 0.4
\fi
%%%cit_text
«Триумфом дезы», вероятно, можно было назвать прошедший ещё в мае
пропагандистский форум «Украина 30», посвящённый «\emph{Стратегии} национальной
безопасности», которая стала идеей фикс для Зеленского и его челяди. О нём не
очень охотно говорили украинские СМИ, и понятно почему: у них небольшой выбор –
либо встраиваться в информационную политику команды Зе, либо повторить судьбу
трёх закрытых по велению и хотению президента оппозиционных телеканалов (и,
добавлю, ещё множества изданий, особенно в интернет-пространстве, расправа с
которыми была не столь заметной).  \emph{Стратегия}, как и большинство
идефиксов Зеленского, зародилась в недрах Совета национальной безопасности и
обороны Украины (СНБО) и была утверждена соответствующим указом гаранта 14
сентября 2020 года
%%%cit_comment
%%%cit_title
\citTitle{Как борьба с «дезой» на Украине множит сущности}, Георгий Бородин, odnarodyna.org, 07.01.2021
%%%endcit
