% vim: keymap=russian-jcukenwin
%%beginhead 
 
%%file 06_12_2021.stz.edu.lnr.lgau.1.za_dobryje_dela
%%parent 06_12_2021
 
%%url http://lnau.su/novosti/lgau-za-dobrye-dela
 
%%author_id 
%%date 
 
%%tags 
%%title ЛГАУ за добрые дела!
 
%%endhead 
\subsection{ЛГАУ за добрые дела!}
\label{sec:06_12_2021.stz.edu.lnr.lgau.1.za_dobryje_dela}

\Purl{http://lnau.su/novosti/lgau-za-dobrye-dela}

\textSelect{В Луганском государственном аграрном университете (ЛГАУ) 6 декабря состоялась
дискуссия в форме круглого стола «Волонтер – доброволец», которая была
посвящена Дню добровольца (волонтера) и Международному Дню инвалидов, где
приняли участие студенты и сотрудники ЛГАУ, а также приглашенные гости.}

\ii{06_12_2021.stz.edu.lnr.lgau.1.za_dobryje_dela.pic.1}

Глава Луганской Народной Республики (ЛНР) Леонид Пасечник своим указом
установил праздник – День добровольца (волонтера), который ежегодно будет
праздноваться в ЛНР 5 декабря.

6 декабря в ЛГАУ состоялась дискуссия в форме круглого стола «Волонтер –
доброволец», которая также была приурочена к этим знаменательным датам.

На мероприятие были затронуты вопросы кто такие волонтеры и являются ли они
добровольцами?

Целью данной встречи является развитие у студентов нравственных качеств путем
пропаганды идей добровольческого труда на благо общества и привлечения их к
решению социально значимых проблем, через участие в социальных, экологических,
гуманитарных, культурно-образовательных, просветительских проектах и
программах.

\ii{06_12_2021.stz.edu.lnr.lgau.1.za_dobryje_dela.pic.2}

Задачами мероприятия выступили:

– формирование и распространение духовно-нравственных ценностей;

– получение необходимого опыта и навыков для реализации собственных идей и проектов;

– знакомство с деятельностью волонтерских организаций в Республике, популяризация идеи добровольчества.

С приветственным словом к присутствующим обратился первый проректор ГОУ ВО ЛНР
ЛГАУ, кандидат сельскохозяйственных наук, доцент \textSelect{Сергей Гнатюк}:

\begin{displayquote}
\em\color{magenta}
− Дорогие друзья, вчера мы отмечали такой важный и необходимый праздник,
который называется «Международный день добровольца» или как сегодня принято
называть Международный день волонтера. Этот праздник имеет длительную
историю и празднуется с 1985 года. В этот день обычно принято почитать и
благодарить тех людей, которые неравнодушны к чужим проблемам. Это те люди,
которые в ущерб своему собственному времени помогают людям, которые
нуждаются в их помощи. И, самое главное, это те люди, которые никогда не
требуют взамен слов благодарности.  Наш университет всегда занимал одну из
лидирующих позиций в планах волонтерского движения. И не побоюсь этого
слова продолжаем ее занимать. Волонтерство всегда входило в план и
программу воспитательной работы нашего подрастающего поколения. С каждым
годом наши волонтерские отряды пополняются новыми студентами, которые
поступают в наш университет. Наши волонтеры, добровольцы, задействованы в
большом количестве работ различного рода. Это и облагораживание территорий
нашей Луганской Народной Республики. Все мы с вами знаем, что наш вуз
курирует, убирает и облагораживает скверы: это сквер 205 на квартале
Ольховский, это Дендропарк, который находится по улице Андрея Линева, это
помощь Луганскому детскому дому №1. Мы регулярно собираем туда подарки для
детей в виде канцелярских принадлежностей, игровых наборов и игр, в
обязательном порядке это сладости. Мы проводим на базе этого детского дома
всевозможные мероприятия для того, чтобы скрасить весьма сложную участь
детей. Кроме того, наша помощь оказывается и ветеранам ВОВ, да даже всем
нуждающимся сотрудникам нашего университета. Наши волонтерские отряды
всегда готовы прийти на помощь! Очень много в адрес администрации нашего
университета приходило благодарственных писем за помощь, которую мы
оказываем, и это очень приятно. Сегодня мы проводим очень важное
мероприятие, дискуссию, которая направлена на пропаганду волонтерский и
добровольческой деятельности. Все мы с вами знаем фразу из советского
мультфильма: «Кто людям помогает, тот тратит время зря», поэтому задача
сегодняшнего нашего мероприятия является как раз опровергнуть это выражение
для того, чтобы каждый студент нашего университета понимал, что, помогая
людям мы время зря не тратим. Я хочу всех волонтеров нашего университета
поздравить с их профессиональным праздником. Я хочу всех вас поблагодарить
за ваше милосердие, сострадание и за ваше большое сердце, потому что вся
волонтерская деятельность основана именно на этих человеческих качествах. Я
хочу вам пожелать, чтобы все то добро, которое вы творите, возвращалось вам
в обязательном порядке. С праздником всех вас!
\end{displayquote}

В дискуссии приняли участие представители проекта «Волонтер» общественного
движения «Мир Луганщине».

\begin{displayquote}
\em\color{magenta}
− Хотелось бы поприветствовать всех на данной дискуссии. Такая тема как
волонтерство очень интересная, и она сейчас развивается в нашей Луганской
Народной Республике достаточно активно. На республиканском уровне у нас
есть разные акции, в которых мы участвуем. Нам бы хотелось пригласить
студентов вашего вуза также принять участие в наших проектах. Вы очень
активные ребята и наша организация с удовольствием с вами бы сотрудничала в
этом направлении, − рассказала Алена Назина, помощник координатора проекта
«Волонтер» общественного движения «Мир Луганщине».
\end{displayquote}

Также среди присутствующих была и активистка проекта «Волонтер» общественного
движения «Мир Луганщине» − \textSelect{Ирина Стремилова}, которая поделилась
своим опытом участия в республиканских мероприятиях и призвала ребят
присоединятся к их проекту. В своем выступлении она отметила, что волонтеры
Республики активно взаимодействуют с Российской Федерацией.

Координатор студенческого самоуправления \textSelect{Илья Корнеев} рассказал
всем присутствующим об активной деятельности волонтерских отрядов ЛГАУ и
подчеркнул важность подобных дискуссий как сегодня.

\ii{06_12_2021.stz.edu.lnr.lgau.1.za_dobryje_dela.pic.3}

Отметим, что гостем на нашем мероприятии также стала заслуженный художник
Луганской Народной Республики \textSelect{Татьяна Ктитарева}, которая
представила свои творческие работы.

\begin{displayquote}
\em\color{magenta}
– Хотелось бы поблагодарить администрацию аграрного университета за то, что
действительно оказывает мне всестороннюю помощь. Дело в том, что многие люди не
смогли бы выжить без волонтеров. Знаете, волонтеры – это такой народ, который
действительно очень необходим всем нам. Я инвалид на коляске и не могу даже
руку поднять, однако я активно занимаюсь творческой деятельностью. Общество
должно понимать это все и поддерживать таких людей как я. Все мы люди, всем нам
нужна забота и человечность. Спасибо студентам ЛГАУ за то, что помогают мне, –
отметила \textSelect{Татьяна Ктитарева}.
\end{displayquote}

Кроме того, студенты факультета общественных профессий подготовили творческие
номера, приуроченные к этому дню, где исполнили песни о добре и мире, а также о
том, как важна помощь людям (Видео доступно по ссылке:
\href{https://www.youtube.com/watch?v=cmIQQcDz4us}{%
День волонтера ЛГАУ, ЛГАУ ЛНР, youtube, 04.12.2021%
}).

Отдельно хотелось бы выразить благодарность Первичной профсоюзной организации
сотрудников ЛГАУ за организацию сладкого стола для всех участников мероприятия.	

\textSelect{\em Пресс-центр университета, фото Игоря Одинцова}

\headTwo{Видео - День волонтера ЛГАУ, ЛГАУ ЛНР, youtube, 04.12.2021}

\ii{06_12_2021.stz.edu.lnr.lgau.1.za_dobryje_dela.scr.1}

