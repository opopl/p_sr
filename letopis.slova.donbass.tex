% vim: keymap=russian-jcukenwin
%%beginhead 
 
%%file slova.donbass
%%parent slova
 
%%url 
 
%%author 
%%author_id 
%%author_url 
 
%%tags 
%%title 
 
%%endhead 

Он также рассказал, что в съёмках хроники медицинских будней им помогал
реальный врач из Сибири, а точнее с Крайнего Севера. Медик болеет за
\emph{Донбасс}, помогает и бывает здесь. Сам о себе он говорит – просто доктор,
просто «Потапыч», \textbf{Премьера песни от группы «Зверобой» – Шанс на
спасение}, miaistok.su, 18.06.2020

Мир и воссоединение с \emph{Донбассом} еще впереди, Андрей Ермолаев, strana.ua,
30.05.2021

\enquote{Почта ЛНР} определила победителей конкурса \enquote{Люблю тебя, мой
край родной, Донбасс!}, lug-info.com, 31.05.2021

Детей и стариков \emph{Донбасса} жаль больше всего.. Гончаренко все равно ответит за
то, что делал и делает.. не верю,что ему не страшно.. Сны наверное снятся..
комментарий, \textbf{Опять Шария достали. Что дальше?} Анатолий Шарий, youtube.com, 31.05.2021

02 июня 2014г. - день, когда был окончательно потерян \emph{Донбасс},
\textbf{Ровно семь лет назад Украина навсегда потеряла Донбасс, - Помните женщин с оторванными ногами у здания Луганской ОГА?} 
Андрей Головачев, strana.ua, 02.06.2021

Можно сколько угодно играть в игру под названием Минские Соглашения, формула
Штайнмайера, план Козака-Ермака и т.д., но приходится признать, что Донбасс
ушел и нет силы, которая могла бы его вернуть,
\textbf{Ровно семь лет назад Украина навсегда потеряла Донбасс, - Помните женщин с оторванными ногами у здания Луганской ОГА?} 
Андрей Головачев, strana.ua, 02.06.2021

\enquote{Точка невозврата для \emph{Донбасса}}. Семь главных вопросов об авиаударе ВСУ по Луганску 7 лет назад,
Максим Минин, strana.ua, 02.06.2021

Так получилось, что почти вся моя жизнь и воспоминания связаны с войной, из-за
чего у меня нет сожалений и печали о прошлом. Я живу настоящим и изредка думаю
о будущем, в котором есть место наивной и глупой мечте, вызывающей улыбку.
Вполне реальная, согревающая и почти осязаемая она позволяет не отчаиваться
даже в самые тяжёлые времена. Я хочу, чтобы в небе \emph{Донбасса} летали пассажирские
самолёты, а не бумажные фонарики. Любая мечта должна превращаться в реальность.
Так должно быть и я верю, что так и будет,
\textbf{Фаина Савенкова. Верить и надеяться}, antifashist.com, 02.06.2021

