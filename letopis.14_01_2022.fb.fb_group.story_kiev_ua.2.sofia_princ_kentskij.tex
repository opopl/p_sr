% vim: keymap=russian-jcukenwin
%%beginhead 
 
%%file 14_01_2022.fb.fb_group.story_kiev_ua.2.sofia_princ_kentskij
%%parent 14_01_2022
 
%%url https://www.facebook.com/groups/story.kiev.ua/posts/1839993592864100
 
%%author_id fb_group.story_kiev_ua,atojev_konstantin.kiev
%%date 
 
%%tags kiev,sofia_sobor
%%title СОФИЯ КИЕВСКАЯ И ПРИНЦ МАЙКЛ КЕНТСКИЙ: КОГДА ТЫСЯЧЕЛЕТИЯ НЕ СВОДЯТ С ВАС ГЛАЗ
 
%%endhead 
 
\subsection{СОФИЯ КИЕВСКАЯ И ПРИНЦ МАЙКЛ КЕНТСКИЙ: КОГДА ТЫСЯЧЕЛЕТИЯ НЕ СВОДЯТ С ВАС ГЛАЗ}
\label{sec:14_01_2022.fb.fb_group.story_kiev_ua.2.sofia_princ_kentskij}
 
\Purl{https://www.facebook.com/groups/story.kiev.ua/posts/1839993592864100}
\ifcmt
 author_begin
   author_id fb_group.story_kiev_ua,atojev_konstantin.kiev
 author_end
\fi

СОФИЯ КИЕВСКАЯ И ПРИНЦ МАЙКЛ КЕНТСКИЙ: КОГДА ТЫСЯЧЕЛЕТИЯ НЕ СВОДЯТ С ВАС ГЛАЗ

В июле 2013 г. Киев посетил двоюродный брат Елизаветы II принц Майкл Кентский.
Члены английского авиаклуба и принц прилетели тогда в Киев по случаю фестиваля
ретро-самолетов на 25-ти уникальных машинах времен Первой и Второй мировых
войн, которых в мире осталось раз-два и обчелся. Принц посетил завод
«Антонова», отметил свой день рождения в отеле «Хаятт», принял участие в
авиашоу в Жулянах, осмотрел Софийский собор, который на него произвел сильное
впечатление. 

\ii{14_01_2022.fb.fb_group.story_kiev_ua.2.sofia_princ_kentskij.pic.1}

Говоря про Майкла Кентского, часто вспоминают, что его дед был
двоюродным братом Николая II, как-то, упуская из виду, что среди его предков
есть креститель Руси князь Владимир. Потомку Карла Великого, Вильгельма
Завоевателя, Алиеноры Аквитанской, конечно, не привыкать посещать захоронения
своих далеких родичей - хотя бы аббатство Вестминстера, но София Киевская,
пожалуй, одно из самых древних мест, сохранившихся практически в аутентичном
виде, где покоятся его предки. Не знаю, насчет Владимира Мономаха (как и в
случае с Ярославом Мудрым сохранился лишь его саркофаг), но где-то в криптах
собора покоится сын Ярослава Мудрого Великий князь Киевский Всеволод. «Повесть
временных лет» содержит запись о его погребении в церкви святой Софии.  

\raggedcolumns
\begin{multicols}{3} % {
\setlength{\parindent}{0pt}

\ii{14_01_2022.fb.fb_group.story_kiev_ua.2.sofia_princ_kentskij.pic.2}
\ii{14_01_2022.fb.fb_group.story_kiev_ua.2.sofia_princ_kentskij.pic.2.cmt}

\columnbreak
\ii{14_01_2022.fb.fb_group.story_kiev_ua.2.sofia_princ_kentskij.pic.3}

\end{multicols} % }

Когда-то Скотт Фицджеральд написал:  «Очень богатые люди не похожи на нас с
вами». Люди, знающие свою генеалогию на протяжении многих веков - тоже
особенные. Представьте себе, каково это, подойти к зеркалу, вглядеться в его
«муть и сон туманящий» и вдруг почувствовать на себе взор, прошедший сквозь
тысячелетие, какого-нибудь Мстислава Великого или Давида Строителя, Фридриха
Барбароссы, княгини Ольги. Все мы, раз уж появились на свет, имеем
генеалогический след, тянущийся сквозь тысячелетия, но лишь небольшой круг
людей может проследить его, хотя бы на несколько веков. Встречая такого
человека, невольно ощущаешь, что его глазами на тебя смотрит вечность. И принц
Майкл как раз из тех, людей, через которых в нас без осечки разряжают взгляд
миллениумы.

\ii{14_01_2022.fb.fb_group.story_kiev_ua.2.sofia_princ_kentskij.pic.4}

Впервые его имя всколыхнуло постсоветское информационное пространство в конце
90-х, после выхода книги Ф. Форсайта \enquote{Икона}. Помню с каким воодушевлением
рассказывал о ней на НТВ Евгений Киселев. Речь в книге шла о претенденте на
наследство Романовых, наиболее генетически близком к царю Николаю II - Майкле
Кентском. Благодаря своей матери Марине принцессе Греческой, одновременно
правнучке Александра II и его брата Константина, принц Майкл стал обладателем
самого большего процента крови Романовых. Он удовлетворял целому ряду
требований, чем не могли похвастаться его конкуренты: был рожден православной
матерью; знал русский язык; служил в армии. Принц родился 4 июля 1942 г. за 6
недель до гибели его отца Георга в авиакатастрофе. Учился в военной академии
Сандхерст, в 1961 - 1963 гг., служил в Германии, Гонконге и на Кипре, а также в
военной разведке. Уволился из армии в 1981 г. в звании майора. Он стал известен
благодаря романтической истории, когда подобно своему дяде Эдуарду VIII
отказался от прав на престол ради брака с любимой женщиной. Отметим,
справедливости ради – в очереди на трон он был пятнадцатым. В 1978 г. он
женился на католичке (чуть ли не первый такой брак с XVIII века в королевской
семье).
\ii{14_01_2022.fb.fb_group.story_kiev_ua.2.sofia_princ_kentskij.pic.5}

Ее звали Мария Кристина фон Рейбниц. Ее дед по материнской линии граф Фридрих
Сапари был последним послом Австро-Венгрии в России, уведомившим 24.07.1914 г.
об объявлении войны. Среди ее предков были Диана де Пуатье, Екатерина Медичи,
император Людвиг ІV, художник Рубенс, курфюрст Бранденбурга Альбрехт ІІІ
Ахиллес, и другие знаменитости. Иоанн Павел ІІ в 1983 г. дал благословение на
этот брак» в римско-католической церемонии. Из-за брака с католичкой права
принца на престол были аннулированы. Он был в них восстановлен лишь в 2015 г.
после вступления нового Закона о Короне 2013 г. 

\ii{14_01_2022.fb.fb_group.story_kiev_ua.2.sofia_princ_kentskij.pic.6}

Судя по тому, что пишут о
принцессе таблоиды это личность незаурядная, всегда идущая против течения. Если
в семье королевы культ собак, то она единственная отдает свое предпочтение
кошкам. Когда все говорят о правах животных, она утверждает, что о них надо
заботиться, но у них нет прав. Когда слова Black Lives Matter («Жизни чёрных
важны») стают фразой года, а расовая толерантность превращается в святая святых
современного мира, она может напомнить о солнечной Африке, расшумевшимся в пабе
афроамериканцам, или прийти на обед с Меган Маркл в брошке Blackmoor в виде
фигурки африканца с намеком на колониальные времена. Она заявляет, что в ней
\enquote{больше королевской крови, чем у любого человека, вступившего в семью королевы
со времен принца Филиппа}. Она часто отпускает шпильки в адрес двора, и тот
платит ей взаимностью, награждая прозвищем «наша Валькирия». Конечно, в
дружелюбии ее сложно заподозрить. Но быть 40 лет единственной католичкой при
дворе, немкой, жить за счет субсидии, выделяемой двором, и не потеряться,
сохранить индивидуальность – это не может не вызывать уважения. В последние
годы принцесса занимается благотворительностью, помощью животным и литературным
творчеством. Она издала Анжуйскую трилогию об эпохе Иоланды Арагонской, Карла
VII, Жака Кера и Агнессы Соррель.

\ii{14_01_2022.fb.fb_group.story_kiev_ua.2.sofia_princ_kentskij.pic.7}

Да, люди, которые знают тысячелетнюю историю своего рода не такие как мы. Не
такие и потому, что нахождение их предков на вершинах власти в течении многих
веков, часто приводило  к огромному количеству загубленных жизней. Все это
лежит тяжким бременем на многих аристократических семействах. Нередко, думаешь,
может и к лучшему, что ты не потомок этих  военных элит - рыцарей, штурмовавших
стены Иерусалима и Константинополя, конкистадоров, осваивавших золотые копи
Нового света, шляхты, сажавшей на кол восставших холопов. Впрочем, хорошо это
или плохо будут судить уже там, где последние станут первыми, а первые -
последними. А пока... Принц Майкл обменялся взглядом с Богоматерью-защитницей
Киева Орантой, смотрящей на него с надалтарья, глянул на фреску детей Ярослава
Мудрого на стенах собора, подумал, что когда-то и его далекий предок смотрел на
это великолепие, и еще о том, что между такими сакральными местами как София и
Вестминстер много общего, по наполняющей их божьей благодати, перекрестился и
вышел из храма в жаркое киевское лето.

\ii{14_01_2022.fb.fb_group.story_kiev_ua.2.sofia_princ_kentskij.cmt}
