% vim: keymap=russian-jcukenwin
%%beginhead 
 
%%file 20_09_2018.stz.news.ua.mrpl_city.1.muralizacia_foto_video
%%parent 20_09_2018
 
%%url https://mrpl.city/news/view/muralizatsiya-mariupolya-kak-steny-goroda-stali-kartinami-foto-plusvideo
 
%%author_id news.ua.mrpl_city,gerasimenko_jaroslav.mariupol
%%date 
 
%%tags mariupol,mariupol.pre_war,mural
%%title Мурализация Мариуполя: как стены города стали картинами (ФОТО+ВИДЕО)
 
%%endhead 
 
\subsection{Мурализация Мариуполя: как стены города стали картинами (ФОТО+ВИДЕО)}
\label{sec:20_09_2018.stz.news.ua.mrpl_city.1.muralizacia_foto_video}
 
\Purl{https://mrpl.city/news/view/muralizatsiya-mariupolya-kak-steny-goroda-stali-kartinami-foto-plusvideo}
\ifcmt
 author_begin
   author_id news.ua.mrpl_city,gerasimenko_jaroslav.mariupol
 author_end
\fi

Ярослав Герасименко | 20 вересня 2018 в 15:00

В последние годы в Мариуполе появился десяток философских, патриотических и
абстрактных муралов.

Ведущая \href{https://www.facebook.com/morning.live.mtv}{\enquote{Утро LIVE}}
Алевтина Швецова разыскала все мариупольские муралы и сделала по ним гид.

Самый первый мурал появился около 15 лет назад по проспекту Строителей, 98
напротив центрального корпуса Мариупольского государственного университета. Его
автор остался неизвестным. На мурале изображен великан, держащего карлика с
дудкой на плечах, а рядом с ними – женщина.

\ii{20_09_2018.stz.news.ua.mrpl_city.1.muralizacia_foto_video.pic.1}

Один из первых муралов в центре города появился в 2015 году на стене сгоревшего
здания городского совета. Мурал призывает защитить себя в будущем и голосовать
за Украину.

\ii{20_09_2018.stz.news.ua.mrpl_city.1.muralizacia_foto_video.pic.2.golosuj_za_ukrainu}

Пик появления муралов в городе пришелся на 2016 год. Так, на пересечении
проспекта Мира и улицы Громова недалеко от военного комиссариата в 2016 году
волонтеры проекта \enquote{Будуємо Україну Разом} (БУР) нарисовали казака с трубкой. На
груди казака нанесен герб Украины.  

\ii{20_09_2018.stz.news.ua.mrpl_city.1.muralizacia_foto_video.pic.3.kozak}

Еще один мурал, созданный в 2016 году, можно увидеть возле Драматического
театра. Там изображено морское чудовище, на спине которого плывут мальчик и
девочка.

\ii{20_09_2018.stz.news.ua.mrpl_city.1.muralizacia_foto_video.pic.4.pisarchuk.chudovische_morskoje}

\textbf{Читайте также:} \href{https://mrpl.city/news/view/zhkh-art-v-mariupole-kak-zhiteli-ukrashayut-dvory-foto-plusvideo}{%
ЖКХ-арт в Мариуполе: как жители украшают дворы?}

Во время фестиваля \enquote{З країни в Україну} на доме по пересечению проспектов Мира
и Строителей была создана картина с яхтой под флагом Мариуполя. Ее нарисовал
художник Александр Писарчук, который является автором еще нескольких муралов в
городе.

\ii{20_09_2018.stz.news.ua.mrpl_city.1.muralizacia_foto_video.pic.5.pisarchuk.kartina_s_jachtoj}

Здание судоремонтного завода тоже решили украсить муралом. Картина вышла
лирической – волосы девушки представлены морем, по которому плывет судно.

\ii{20_09_2018.stz.news.ua.mrpl_city.1.muralizacia_foto_video.pic.6.asrz}

В прошлом году забор вблизи металлургического комбината в Кальмиусском районе
стал полотном для мурала с фантастическим сюжетом.

\ii{20_09_2018.stz.news.ua.mrpl_city.1.muralizacia_foto_video.pic.7}

В старом городе тоже есть мурал. На одной из сталинок изображен велосипедист,
который \enquote{перевозит} на спине большой дом. Мурал можно посмотреть на пересечении
проспекта Мира и улицы Торговой.

\ii{20_09_2018.stz.news.ua.mrpl_city.1.muralizacia_foto_video.pic.8.velosipedist}

Жильцы одного из мариупольского ОСМД сами захотели украсить свой дом муралом.
Они обратились к активисту Максиму Светлову, который вместе со своими друзьями
из арт-студии \enquote{4 путь} изобразил на стене девушку-весну.

\ii{20_09_2018.stz.news.ua.mrpl_city.1.muralizacia_foto_video.pic.9.devushka_vesna}

\textbf{Читайте также:} \href{https://mrpl.city/news/view/zhkh-art-v-mariupole-kak-zhiteli-ukrashayut-dvory-foto-plusvideo}{В центре Мариуполя разрушили мурал с птицей (ФОТОФАКТ)}

Сюрреалистический мурал можно увидеть на стене дома по улице Шевченко, 64, если
зайти с улицы Куинджи. На островах, парящих над лесом, находятся традиционные
украинские дома. Острова объединены флагом Украины.

\ii{20_09_2018.stz.news.ua.mrpl_city.1.muralizacia_foto_video.pic.10}

На стенах школы №17 на Правом берегу изображены космос и Украина. Вход в школу
разрисовали приазовским орнаментом.

\ii{20_09_2018.stz.news.ua.mrpl_city.1.muralizacia_foto_video.pic.11}

Самый восточный мурал Мариуполя находится на стене 68 школы. Его создал по
мотивам народной сказки \enquote{Рукавичка} японский художник Миязаке Кенске.

\ii{20_09_2018.stz.news.ua.mrpl_city.1.muralizacia_foto_video.pic.12.rukavychka.fragment}

Два мурала украшают двор 37 школы. На одной стене изображена девушка, а на
другой – море и корабль. Идея второго рисунка принадлежит восьмикласснице
Марине Матяш.

\ii{20_09_2018.stz.news.ua.mrpl_city.1.muralizacia_foto_video.pic.13.shkola37}

В этом году на стене 9 школы киевский художник Саша Корбан и команда \enquote{Sky Art
Foundation / art space} создали мурал, призывающий беречь воду. На картине она
выливается из большой пластиковый бутылки, в которой \enquote{помещена} улица обычного
города.

\ii{20_09_2018.stz.news.ua.mrpl_city.1.muralizacia_foto_video.pic.14.korban.voda}

В процессе создания – мурал на пересечении Мира и улицы Леваневского. Недавно
на стене одного из домов появилось изображение девочки с игрушечным мишкой. На
рисунке изображена Милана Абдурашитова, которая потеряла маму и пострадала сама
при обстреле микрорайона \enquote{Восточный} в 2015 году.

\ii{20_09_2018.stz.news.ua.mrpl_city.1.muralizacia_foto_video.pic.15.milana}

Ранее MRPL.CITY
\href{https://mrpl.city/news/view/zhkh-art-v-mariupole-kak-zhiteli-ukrashayut-dvory-foto-plusvideo}{сообщал},
что в Мариуполе открылась выставка, где экспонируются более 100 произведений 30
мариупольских художников современного искусства.

\url{https://www.youtube.com/watch?v=BP1OL1Q7PGQ}

