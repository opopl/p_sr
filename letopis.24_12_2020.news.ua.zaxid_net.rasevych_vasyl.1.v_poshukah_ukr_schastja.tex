% vim: keymap=russian-jcukenwin
%%beginhead 
 
%%file 24_12_2020.news.ua.zaxid_net.rasevych_vasyl.1.v_poshukah_ukr_schastja
%%parent 24_12_2020
 
%%url https://zaxid.net/v_poshukah_ukrayinskogo_shhastya_n1512368
 
%%author 
%%author_id rasevych_vasyl
%%author_url 
 
%%tags ukraina
%%title В пошуках українського щастя
 
%%endhead 
 
\subsection{В пошуках українського щастя}
\label{sec:24_12_2020.news.ua.zaxid_net.rasevych_vasyl.1.v_poshukah_ukr_schastja}
\Purl{https://zaxid.net/v_poshukah_ukrayinskogo_shhastya_n1512368}
\ifcmt
	author_begin
   author_id rasevych_vasyl
	author_end
\fi

\ifcmt
pic https://zaxid.net/resources/photos/news/202012/1512368.jpg?202012281711&fit=cover&w=960&h=540&q=100%201920w
caption В пошуках українського щастя, Василь Расевич, 24.12.2020
\fi

\begin{leftbar}
  \begingroup
    \em\Large\bfseries\color{blue}
    Не можна стати по-справжньому щасливим і успішним, упосліджуючи інших
  \endgroup
\end{leftbar}

Після майже тридцяти років блукань, які запальні патріоти назвуть не інакше як
боротьбою, стало зрозумілим, що українські потуги будуть успішними тільки за
умови інклюзивності українського національного проєкту. Не треба боятися слова
інклюзія – воно означає включеність. Тобто створення таких умов, коли українці
з різних регіонів, груп і бульбашок навчаться не тільки терпіти один одного
поруч, але й толерувати вибір кожного. Не намагаючись при цьому оголосити єдино
правильним своє бачення актуальної ситуації, минулого і майбутнього. Не
наділяючи себе і свою групу ексклюзивним правом монополіста. А тим більше, не
пробуючи накинути до виконання в обов’язковому порядку свій варіант іншим. І
найголовніше – не намагатися когось сегрегувати тільки за те, що той, на його
думку, має неправильне походження, його мова не пасує до власного уявлення про
щасливе майбутнє, а бачення минулого істотно відрізняється.

Чи це можливо, покаже вже найближчий час. А поки що наприкінці чергового
важкого року подумаймо про те, що ж насправді завадило нам стати міцним,
консолідованим суспільством? Чому з таким невтішним результатом закінчувалися
ледве не всі спроби змінити на ліпше ситуацію? Чому в Україні і надалі
отримують політичні дивіденди дії, спрямовані не на консолідацію суспільства, а
на розкол і роз’єднання? І чому політична мобілізація під час виборів неодмінно
супроводжується розбурхуванням міжрегіональних антагонізмів? Думаю, що основна
причина криється в нечесності навіть зі собою.

З ранньої академічної юності мені важко давалися до розуміння окремі слова і
терміни. Вони для мене були або порожнім звуком, або по-нехлюйському
підігнаними під примітивний копил конструктами. До таких належали поняття:
«націєтворення», «державотворення», зрештою, і «національне відродження». З
часом до ненаповнених змістом додалося також поняття «соборність». Стало
поганим сигналом, коли українські академічні вчені бралися за написання
монументальних праць, як-от «історія державності» або ще гірше – «історія
соборності». Обидва напрями не тільки приписують українську національну
свідомість сучасного зразка давнім князям і козакам, але й зводять історію
українців до соціальних повстань і бунтів, безпросвітного нужденного існування,
надаючи при цьому всім невдахам виразного національного забарвлення.

Націєтворення – це така собі примітивна калька з англійського national
building, коли передбачається, що нації можуть народжуватися, занепадати,
відроджуватися і навіть перемагати. Коли є бажання, можна уявити, що є той
хтось, хто за своїм задумом націю будує і видозмінює. Часто в ролі деміурга
націй виступає Бог. Він уже початково роздає людям національність і землю, що
має їх не тільки годувати, але й народжувати. А вже звідси походять фанаберії
про «автохтонів» і «титульних» аж до «зайд» і «наїжджого елементу». Такий
підхід за означенням містить у собі положення про етнічну чистоту і,
відповідно, рано чи пізно веде до етнічних чисток. Особливо драматично це
виглядає в етнічно строкатих регіонах. Таких, як Україна. Україна – це простір,
де часто мінялися народи, межі, кордони і правлячі еліти. Особливих змін зазнав
традиційний уклад різних історичних регіонів, коли їх звели докупи в УРСР. Коли
модернізація українців відбувалася в умовах радянської, формально
інтернаціональної системи.

Тому в результаті розвалу СРСР і появи незалежної України надзвичайно загострилися конфлікти пам’ятей у різних регіонах. І замість того, щоб запропонувати інклюзивний варіант, зайнятися деконструкцією усталених кліше та міфів, у країні влаштували конкурентну боротьбу за нав’язування етнонаціонального або інтернаціонального (насправді російсько-радянського) варіантів. Це допомагало легко здобути політичну підтримку в базовому регіоні, але унеможливлювало суспільний консенсус та консолідацію. Це зробило практично неможливим створення якоїсь спільної історичної основи для всіх громадян України. Стимулювало появу великої кількості некритичних, а подекуди брехливих, історичних концепцій.

Причому подібними були дії зі всіх боків. Національні пуристи взялися за плекання національно-державницької концепції, де було місце Київській Русі, князям, «національно-визвольній війні під проводом Богдана Хмельницького», Мазепі, УНР, Державі гетьмана Скоропадського, Директорії. Потім до цього доштукували Галицьку Русь, національний поступ під Габсбургами, УСС, ЗУНР, митрополита Андрея Шептицького. Важче пішло з радянським періодом. Бо радянська модернізація ніяк не в’язалася з появою у Другій Речі Посполитій радикального українського націоналізму. Ще важче було схрестити між собою успіхи українців у радянському проєкті з діяльністю збройних формацій західних українців у складі нацистського вермахту. У цьому ключі дуже дивною виглядала «історія соборності».

Якщо ще можна було задурити дітей у школі тим, що козаки Хмельницького двічі
разом з татарами та росіянами брали в облогу Львів для національного добра, то
з Другою світовою не склалося. Так історія із чинника, що, за задумом
патріотичних істориків, мав консолідувати і надихати націю на великі звершення,
перетворилася на найбільш вибухову субстанцію, здатну розділити українців на
непримиренні табори. Чим не раз скористалися політики та й, що гріха таїти,
відверті вороги.

Не сприяло консолідації і часто вульгарне трактування різних форм національної
ідентичності: русини, малороси, українці. У підсумку нібито необхідного
спрощення поняття «русини» і «малороси» набули виразного негативного
забарвлення. Ще більшими негативними наслідками обернулася повна негація
української радянської ідентичності. Хоча саме в цей час відбувалися
інституалізація, урбанізація, формування академічної науки, масової культури,
укладання літературного канону.

Намагання повністю заперечити здобутки радянського періоду, акцентуючи увагу на
боротьбі українських національних збройних формувань, а також спроби
представити часи УРСР періодом окупації, завдали непоправної суспільної шкоди.
Оскільки значна частина українського суспільства не згодилася з такою штучною
підміною. Адже не можна одним махом відібрати в людей їхню родинну історію,
особисті уподобання, юність і молодість. Зрештою, заперечити ті матеріальні
блага, які вдалося в той час акумулювати. І це притому, що майже всі
погоджувалися, що ні комуністична ідеологія, ні КПРС для тих, хто гордиться
своїм минулим, ніякої ролі не відіграють. Тому зводити історію до ідеологічного
протистояння комуністів і націоналістів, протиставляти українцям росіян і
навпаки було справжньою диверсією проти майбутнього України.

На жаль, основною технологією стали розкол і протиставлення. Уявні і надумані
цінності стали домінувати над справді важливим. Бо не може бути справедливим
позбавлення рівних прав когось іншого за кольором шкіри, розрізом очей, статтю,
сексуальною орієнтацією чи наявністю фізичних вад. Так само не можуть надавати
виняткового статусу в суспільстві етнічне походження, факт народження в
конкретному регіоні, як і належність до релігійної конфесії, політичної сили
або відданість одній з історичних ліній. Не варто обирати депутатом людину
тільки за те, що вона когось з минулого любить або проклинає. Демонстрація
відсутності матеріального інтересу свідчить, що він десь прихований, але на
нього розраховують. Що насправді важливими є порядність, професійні
кваліфікації, досвід і чітке бачення своїх посадових функцій. Час зрозуміти, що
гасла, які апелюють до минулого – це подразники, здатні викликати рефлекторні
реакції, не більше. Що треба вимагати не плану помсти внутрішнім і зовнішнім
ворогам за історичні кривди, а викладу програми реформ і важливих суспільних
змін.

Варто усвідомити, що етнічність не може бути особистою заслугою чи вродженим
ганджем. Не можна робити велетенських узагальнень і наділяти конкретними
людськими рисами нації, соціальні та релігійні групи або покоління. Насправді
такий підхід є абсурдним. Що легко побачити завдяки кільком найпростішим
прикладам. Спробуймо зробити узагальнення на підставі того, що президент
Дональд Трамп – рудий. Після цього нам нічого іншого не залишається, як
оголосити всіх рудоволосих авторитарними самодурами, нахабами, закоханими в
маніакальні методи правління Путіна. Або ж на підставі того, що Василь
Симоненко був видатним українським поетом, оголосити всіх носіїв цього прізвища
справжніми патріотами України. Але що тоді робити з компартійним функціонером
Петром Симоненком? Що робити з етнічним українцем Судоплатовим, який вбив
Євгена Коновальця? А що з росіянином Донцовим, творцем українського
інтегрального націоналізму?

Українська історія не дає однозначних відповідей і не пропонує героїв та
ворогів за етнічною ознакою. Вона є, якщо дозволите, гібридною мішанкою всіх і
всього. Вона є конгломератом різних людей, об’єднаних сподіваннями на краще і
справедливе майбутнє.

То що треба зробити, щоб виправити ситуацію? Грубо кажучи, треба усвідомити, що
рацію мають ті, хто готує салат олів’є до святкового столу, і ті, хто не уявляє
циклу зимових свят без куті. І ті, хто йде до церкви в ніч з 24-го на 25
грудня, і ті, що святкують Святвечір з 6-го на 7 січня. Так само арелігійні, що
трактують ці дні як законні вихідні, не є ані поганими українцями, ані
непатріотами. Абсолютно рівні права в суспільстві мають ті, що люблять
творчість Булгакова й Антонича. Лесі Українки і Маріни Цвєтаєвої. Тараса
Шевченка і Адама Міцкевича. І навіть ті, що про таких і не чули, але дбають про
гармонійні відносини в суспільстві і примножують в ньому добро.

Що якщо й сегрегувати, то лише злочинців, злодіїв, корупціонерів. І то на
підставі вироків незалежного і справедливого суду. Відчуття щастя має наставати
від спільних звершень, а не через упослідження призначених кимось ворогів,
конкурентів або «нетаких». Має бути нульова толерантність до порушників закону,
а не до тих, хто думає, виглядає або поводиться інакше. Усвідомлення
необхідності інклюзивного підходу потрібне для того, щоб відчути стійкий ґрунт
під ногами. Щоб мати міцне запілля. Щоб не бути легким предметом внутрішніх і
зовнішніх маніпуляцій. Це і буде дорога до щастя.
