% vim: keymap=russian-jcukenwin
%%beginhead 
 
%%file slova.vrag
%%parent slova
 
%%url 
 
%%author 
%%author_id 
%%author_url 
 
%%tags 
%%title 
 
%%endhead 
\chapter{Враг}
\label{sec:slova.vrag}

%%%cit
%%%cit_head
%%%cit_pic
%%%cit_text
Компашка Зеленского оставила все ту же практику Петра разделения и унижения
своих сограждан. Унижения и разделения. Каждый день. В каждом публичном
выступлении. Свои и чужие.  И это удобно. \emph{Враг} – это не оппонент. Его надо
гасить безжалостно и без аргументов. А все что не получается – это потому что
враг повсюду. В СМИ. В Раде. В мелком и среднем бизнесе. На уличном пикете.
\emph{Враг} уничтожает нашу любимую Украину и нашу единственно правильную власть.
\emph{Врага} надо бить. И правильно вычленять, чтобы затем расчеловечивать
%%%cit_comment
%%%cit_title
\citTitle{Бог Данилова - это бог расчеловечивания}, 
Игорь Лесев, strana.ua, 20.06.2021
%%%endcit

%%%cit
%%%cit_head
%%%cit_pic
%%%cit_text
Так вот, Украина считает \emph{врагом} своей безопасности Филиппа Киркорова.
КИРКОРОВ.  По мнению минкульта и правоохранителей угрожает национальной
безопасности. Так, повторим, национальной безопасности он.  Угроза.
Национальной безопасности. Шо.  Шутка «яка країна, такі теракти» теперь куда
смешнее. Помните, что о человеке можно судить по его \emph{врагам}?  \emph{Враг} нашей страны
– Киркоров.  Теперь все, кто напевает его песни, славит агрессора.  Хуже только
имя божье выговаривать впустую.  Удивительная реальность
%%%cit_comment
%%%cit_title
\citTitle{Филипп Киркоров назван угрозой национальной безопасности Украины}, Дмитрий Раимов, strana.ua, 24.06.2021
%%%endcit

%%%cit
%%%cit_head
%%%cit_pic
%%%cit_text
Принцип общего блага простой – это наличие самой Украины. Это благо. Если
кто-то торгует родиной с \emph{врагом}, то это не благо. И он должен сидеть в тюрьме
или быть расстрелянным по законам военного времени. Не существует никаких
компромиссов. Не существует. Понимаете? И в вопросе \emph{врага} тоже не существует
компромиссов. Если это \emph{враг}, то вы должны пойти на любые издержки вплоть до
бомбардировок, чтобы \emph{врага} наказать и отстоять свою землю. Вот не существует.
Как только вы втягиваетесь в обсуждение цены, то есть «а давайте не
бомбардировки, давайте будем вести годы переговоров – ну давайте». Тогда
начинаются торги. И в торгах вы теряете всю страну. Нельзя этого делать,
понимаете? Или принцип есть и тогда есть Украина или принципа нет, тогда
Украины нет
%%%cit_comment
%%%cit_title
\citTitle{Сергей Дацюк: Украина сегодня - не просто попрошайка, она на мусорнике истории}, 
Сергей Дацюк; Людмила Немыря, hvylya.net, 28.06.2021
%%%endcit

%%%cit
%%%cit_head
%%%cit_pic
%%%cit_text
Водночас, все це не відміняє того факту, що РПЦ – \emph{ворожа організація}, яка
працює в інтересах країни-агресора. Купа прикладів про поширення \emph{ворожих}
наративів, \emph{ворожої} пропаганди, я вже мовчу про періодичні схрони зі зброєю в
священнослужителів. Але з цим не мирянки в хустках мають боротися, а
правоохоронні органи та СБУ. Але щось тишина. Можливо, тому, що добра половина
політичного естеблішменту хрестила своїх дітей в РПЦ, а інша була на тих
хрестинах хресними. А український політик і державний діяч, як відомо, Бога не
боїться, але дуже переживає що батюшка скаже
%%%cit_comment
%%%cit_title
\citTitle{Не прочанки в хустках мають боротися з РПЦ, а правоохоронні органи}, 
Андрій Андрушків, gazeta.ua, 28.07.2021
%%%endcit

%%%cit
%%%cit_head
%%%cit_pic
%%%cit_text
Замість розбудови привабливого та самодостатнього українського світу, ми
часто-густо намагались побудувати маленьку кальку "русского мира" під
жовто-блакитним прапором. Так само корумповану, заздрісну та злу, яку тримає до
купи тільки наявність екзистенційного \emph{ворога}. Тільки якщо в уяві більшості
росіян таким ворогом є Америка та колективний Захід, то для українців – Росія.
І тут раптом виявилось, що 40\% українців (і навіть 44\% тих, хто народились і
вирісли у незалежній Україні) не хочуть жити в парадигмі анти-Росії
%%%cit_comment
%%%cit_title
\citTitle{Фіаско етнічної українізації}, 
Генадій Друзенко, analytics.hvylya.net, 28.07.2021
%%%endcit

%%%cit
%%%cit_head
%%%cit_pic
\ifcmt
  pic https://day.kyiv.ua/sites/default/files/news/01112021/250393842_1230103024148978_5748365102449624854_n.jpg
  @width 0.4
\fi
%%%cit_text
Позиції наших захисників двічі потрапили під обстріл біля Світлодарська,
поблизу Луганського, Майорська, Золотого-4, Зайцевого, Новотошківського, в
районі Новоолександрівки, неподалік Мар'їнки та двічі біля Шумів і
Новолуганського.  Крім того, на Луганщині зафіксовано проліт \emph{ворожого}
безпілотного літального апарату, ймовірно «Орлан-10», з перетином лінії
зіткнення.  Внаслідок \emph{ворожих} дій один військовослужбовець зазнав
бойового травмування.  Воїн перебуває в лікувальному закладі. Стан його
здоров’я – задовільний.  На збройні провокації противника наші захисники
відкривали вогонь у відповідь.  Завдяки їхнім зусиллям рубежі наших підрозділів
залишилися незмінними
%%%cit_comment
%%%cit_title
\citTitle{Доба в ООС: 14 порушень, травмований воїн}, , day.kyiv.ua, 01.11.2021
%%%endcit

%%%cit
%%%cit_head
%%%cit_pic
%%%cit_text
Ми говоримо не про війну, ми говоримо про агресію. На війні етику треба
відсовувати, а бути жорстокими, категоричними і прямолінійними. Ми пам’ятаємо,
що робили з письменниками, дисидентами. Якщо \emph{ворог} переможе з нами буде те
саме, нас просто знищать.  Як я вже згадувала, що нещодавно повернулася з
Донеччини. У Залізному до війни було 25 тисяч населення. Зараз – приблизно 5.
Всі жили за рахунок шахт. Зараз цього немає. Там суцільний алкоголізм, суцільна
наркоманія. Сьогодні треба принципово ставити ці питання. Нині, коли йде сильна
інформаційна агресія, потрібен адекватний опір з нашого боку.  У нас є сили,
здоров’я, можливості, знання. Нам треба писати, знімати, показувати відзняте.
Нам треба займати простір позитивом. Там, де буде пусте місце, його займе
негатив
%%%cit_comment
%%%cit_title
\citTitle{Як виграти інформаційну війну? – Слово Просвіти}, ,slovoprosvity.org, 01.11.2021
%%%endcit

%%%cit
%%%cit_head
Съездил в Россию – враг
%%%cit_pic
\ifcmt
  pic https://avatars.mds.yandex.net/get-zen_doc/1219682/pub_618428eaa205653cbea4948e_61843cd3b50152750d708e7e/scale_1200
  @width 0.4
\fi
%%%cit_text
Почему меня на российской границе никто не убивает за украинский паспорт?
Спокойно подала в окошко два паспорта: Россия, Украина. «Пожалуйста,
проходите». Никаких вопросов ко мне. Спросили только, какова была цель поездки
на Украину. «Поездка не состоялась», - ответила я. «Вас не впустили»? – смотрит
понимающе.  При мне одного россиянина-паренька не впустили на Украину. Не
впустили российские пограничники с российской стороны. Говорил с ним инспектор,
разъяснял его права, основания для въезда – очень вежливо, доброжелательно, без
давления и шмонания сумок. Какое доверие к своим гражданам, какое уважение.  На
украинской границе не чувствуется, что мы, с украинскими паспортами, их
граждане. Мы \emph{враги}. Съездил в Россию – \emph{враг}. Ручонки свои тянут, всё пощупать
хотят, всё проверить. Карманы вывернут, а заодно и тебя наизнанку.  А видели бы
вы, какие там бабушки едут! Бедные украинские женщины. Видно, что из последних
копеек живут, по три сумки прут от детей, которые в России устроились, а сами
еле ходят, по ступенькам карабкаются. В руках куча бумажек – несут на отлёте,
бережно, чтобы не помялись – наверное, основание для въезда-выезда. У них и так
жизнь тяжелая, зачем их дополнительно мордовать? Преступниц нашли
%%%cit_comment
%%%cit_title
\citTitle{Полицейское государство}, Юлия Вельбой, zen.yandex.ru, 04.11.2021
%%%endcit

%%%cit
%%%cit_head
%%%cit_pic
%%%cit_text
Кто такой \emph{враг}? Как появляется \emph{враг}? Нужно ли воевать с врагом?
Как победить \emph{врага}? Можно ли и как можно жить без \emph{врага}?  Кто
такой \emph{враг}?  \emph{Враг} по своему происхождению не натурален. Никто не
рождается \emph{врагом} никого.  \emph{Врагами} становятся, причем вполне
сознательно. \emph{Враги} создаются, причем намеренно.  Хищник не враг жертвы.
Кровавый маньяк, убийца, бандит, насильник, вор, хулиган, коррупционер,
обманщик не являются \emph{врагами} своих жертв. Их жертвы случайны, выбранные
в той или иной степени социально искаженным интересом на фоне страстных
желаний.  Девианты не ведут войну, их действия асоциальны. \emph{Враг} же
жестко социальное явление.  Далеко не все, с кем мы конфликтуем или воюем,
\emph{враги}. Если мы набили морду хулигану, он нам еще не \emph{враг}.
\emph{Враг} может не случиться, даже если случилась война. И война может не
случиться, даже если случился \emph{враг}.  Подчинение одной стороной другой
без сопротивления другой стороны тоже не содержит \emph{вражды}.  Оккупируемые
или подавляемые путем войны за доминирование группы, страны или народы могут не
рассматриваться как \emph{враги}. И наоборот со столетними \emph{врагами} может
и не случиться войны.  \emph{Враг} есть продукт различия и свободы разных
сторон, доведенный до крайности упрощенным мышлением, когда одна сторона
представляет настолько длительную и устойчивую угрозу или опасность для другой
стороны, чтобы эта угроза была обнаружена, зафиксирована и персонифицирована
%%%cit_comment
%%%cit_title
\citTitle{Враг}, Сергей Дацюк, analytics.hvylya.net, 18.11.2021
%%%endcit
