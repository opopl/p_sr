% vim: keymap=russian-jcukenwin
%%beginhead 
 
%%file 05_06_2021.stz.news.ua.mrpl_city.1.premjery.pic.4
%%parent 05_06_2021.stz.news.ua.mrpl_city.1.premjery
 
%%url 
 
%%author_id 
%%date 
 
%%tags 
%%title 
 
%%endhead 

\ifcmt
  ig https://mrpl.city/uploads/posts/redactor/urazitxuh1ctrggw.jpg
  @wrap center
  @width 0.6
\fi
