% vim: keymap=russian-jcukenwin
%%beginhead 
 
%%file 13_09_2021.fb.baumejster_andrej.kiev.filosof.1.filosofia_istoria_socializm_marks
%%parent 13_09_2021
 
%%url https://www.facebook.com/andriibaumeister/posts/4242508692537324
 
%%author_id baumejster_andrej.kiev.filosof
%%date 
 
%%tags baumejster_andrej.filosof.kiev,chtenie,filosofia,istoria,kniga,marks_karl,obschestvo,socializm,ukraina
%%title Сегодня, по выражению Даймонда, пора учиться "быть честными с самими собой", а не приукрашивать собственную историю
 
%%endhead 
 
\subsection{Сегодня, по выражению Даймонда, пора учиться \enquote{быть честными с самими собой}, а не приукрашивать собственную историю}
\label{sec:13_09_2021.fb.baumejster_andrej.kiev.filosof.1.filosofia_istoria_socializm_marks}
 
\Purl{https://www.facebook.com/andriibaumeister/posts/4242508692537324}
\ifcmt
 author_begin
   author_id baumejster_andrej.kiev.filosof
 author_end
\fi

В угоду сиюминутным политическим интересам историю можно у нас
идеологизировать, мифологизировать и фальсифицировать. Но не менее
распространенным методом создания узкой идентичности является метод
сознательного замалчивания и намеренных "провалов исторической памяти". 

Хотя, как вы знаете, мои взгляды близки правым консерваторам, я сейчас поговорю
о левых идеях. Когда в 2018 году на Западе активно отмечали 200-летие рождения
Карла Маркса, книжные полки европейских магазинов пестрели роскошными изданиями
"Капитала" и новыми исследованиями о философии Маркса и марксизма. Естественно,
в Германии есть улицы Маркса и памятники Марксу. Марксизм является важным
элементом европейской интеллектуальной традиции и сегодня он, по всей
вероятности, будет вновь становиться влиятельным идейным течением. Но в
украинских книжных вы напрасно будете искать книги Маркса. И уж тем более книги
Ленина. И научных биографий Маркса, Энгельса, Ленина и Троцкого вы не найдете.
Прежде чем некоторые из моих читателей воспылают праведным гневом (это уже
рефлекс и эмоциональная потребность), напомню, что в современных немецких
магазинах продаются не только книги и биографии этих авторов. В моем любимом
берлинском "Дуссмане" несколько шкафов занимает литература о Третьем рейхе и
национал-социализме. Тут и многотомные дневники Геббельса, и десятки новых
научных биографий Гитлера, воспоминания функционеров и военачальников
нацистской Германии. Потому что очень важно понимать, как и почему
национал-социализм пришел к власти в Германии. В чем была его притягательная
сила для многих немцев. 

Здесь я не провожу параллели между национал-социализмом и коммунизмом. Меня
интересует другой вопрос: почему левые идеи были так популярны в Украине в
начале ХХ века? Без преувеличений можно сказать, что главные действующие
персонажи войн за независимость в 1918-1921 годах были левые. Грушевский,
делавший ставку на эсеров (и вернувшийся в большевистскую Россию, чтобы стать
влиятельным функционером на целых 10 лет), социал-демократ Винниченко,
радикальный левый Петлюра. Почитайте универсалы УНР. Повнимательнее. Например
ключевой тезис третьего универсала (7 (20) ноября 1917) провозглашает отмену
собственности на землю (то есть экспроприацию земель у собственников!!!):
"Однині на території Української Народньої Республіки існуюче право власности
на землі поміщицькі та инші землі нетрудових хазяйств сільськогосподарського
значіння, а також на удільні, кабінетські та церковні землі - касується.
Признаючи, що землі ті єсть власність усього трудового народу й мають перейти
до нього без викупу...". И так далее. 

Четвертый универсал, уже перед лицом большевистской угрозы, только подтверждает
чисто большевистские идеи (точнее, идеи эсеров, тогда еще союзников
большевиков): "В справі земельній комісія, вибрана на останній сессії нашій,
уже виробила закон про передачу землі трудовому народові без викупу, прийнявши
за основу скасування власности соціалізацію землі, згідно з нашою постановою на
сьомій сессії". Только Скоропадский нанес удар по этим радикально-левым
принципам (увы, не надолго). 

Сегодня, когда мы, с одной стороны, подчеркиваем преемственность между
современной Украиной и УНР, а с другой - готовим рынок земли, некоторые, мягко
говоря, несоответствия стараются не замечать. Здесь уместно подчеркнуть, что я
не против рынка земли (хотя вопрос требует более серьезного обсуждения). Я
просто показываю, что поверхностные идеологические нарративы, которые пытаются
представить украинскую историю неким единым процессом с общими мотивами и
целями, - вещь сомнительная и даже опасная. 

Сегодня, по выражению Даймонда, пора учиться "быть честными с самими собой", а
не приукрашивать собственную историю (представляя ее в упрощенной и гладкой
версии "многовековой борьбы за свободу"). Это факт, что среди украинских
литераторов второй половины XIX века и украинских политиков начала ХХ века
левые идеи (причем в самых радикальных формах) были чрезвычайно популярны. А
значит в основании украинского государства полагались иные идеи, по сравнению с
нынешней эпохой. Национализация и приватизация, отмена частной собственности на
землю и введение частной собственности на землю, государственная экономика и
идеология свободного рынка, - все это, согласитесь, разные вещи. А еще
универсалы говорят о "буржуазных правительствах" (негативно), о "трудящемся
народе" (как своем главном адресате). А еще монополия на торговлю: "За старого
ладу торговці та ріжни посередники наживали з бідних пригноблених клясів
величезні капитали. Однині Українська Народня Республіка бере в свої руки
найважнійші галузі торговлі і весь доход з неї потратить на користь народу". А
как вам идея национализации банков? "Приписуємо встановити державно-народній
контроль над всіма банками, які кредітами, позиками нетрудовим масам допомагали
визискувати кляси трудові (разве это не большевистский лозунг? - А.Б.). Однині
позичкова поміч банків має даватися головним чином на піддержку трудового
населення та розвиток народнього господарства Української Народньої Республіки,
а не на спекуляцію та ріжні банкові експльоатації і визиск" (то есть извлечение
прибыли - моральное зло? - А. Б). 

Не буду вдаваться в подробные разборы и приводить исторические факты. Но,
надеюсь, никто не будет спорить, что Украина в начале ХХ века создавалась как
радикально левый проект (сравните с новыми государствами Европы после 1919-1921
годов). 

Так о чем это я? О честном разговоре с самими собой. И о том, что на полках
украинских книжных (надеюсь, их будет становится все больше) должны появиться
не только Маркс и Энгельс, но и Ленин с Троцким. Чтобы мы не ходили все время
по кругу и не впадали в историческую и интеллектуальную амнезию...

\ii{13_09_2021.fb.baumejster_andrej.kiev.filosof.1.filosofia_istoria_socializm_marks.cmt}
