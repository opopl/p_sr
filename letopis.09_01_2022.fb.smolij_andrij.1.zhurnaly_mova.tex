% vim: keymap=russian-jcukenwin
%%beginhead 
 
%%file 09_01_2022.fb.smolij_andrij.1.zhurnaly_mova
%%parent 09_01_2022
 
%%url https://www.facebook.com/permalink.php?story_fbid=4772512039453227&id=100000834337338
 
%%author_id smolij_andrij
%%date 
 
%%tags jazyk,mova,smi.ukraina,ukraina,ukrainizacia
%%title За тиждень - всі загальнодержавні журнали та газети повинні видаватись УКРАЇНСЬКОЮ
 
%%endhead 
 
\subsection{За тиждень - всі загальнодержавні журнали та газети повинні видаватись УКРАЇНСЬКОЮ}
\label{sec:09_01_2022.fb.smolij_andrij.1.zhurnaly_mova}
 
\Purl{https://www.facebook.com/permalink.php?story_fbid=4772512039453227&id=100000834337338}
\ifcmt
 author_begin
   author_id smolij_andrij
 author_end
\fi

За тиждень - всі загальнодержавні журнали та газети повинні видаватись
УКРАЇНСЬКОЮ.

Чергова норма Закону про мову набирає чинності з 16 січня 2022 року.

@igg{fbicon.exclamation.mark} Згідно Закону всі всеукраїнські та регіональні
друковані видання повинні мати ОБОВ‘ЯЗКОВУ версію державною мовою. 

\ii{09_01_2022.fb.smolij_andrij.1.zhurnaly_mova.pic.1}

Наразі частина видань вже перейшло на українську ще у грудні. Серед них «ELLE
Україна», «Наталі», «Единственная», «Pink» та кілька інших. 

Хтось перейшов на українську кілька місяців назад, як журнали «НВ» та Marie
Claire. 

Журнал Forbes видається двома мовами вже півтора роки.

За моїми даними наразі оголосили про повний перехід на українську близько
80-90\% загальноукраїнських видань. Серед них журнали:

«Добрі поради» (Добрие совєти), Burda, Сабріна, Playboy Ukraine, За рулем,
Фокус та з десяток інших. 

Вже з 16 січня починаю моніторити виконання цієї норми Закону друкованими ЗМІ. 

Також, з 16 січня ви можете вимагати, щоб в місцях продажу преси - мінімум 50\%
видань мають бути українською. 

В другій половині січня поінформую вас про ситуацію, а також про можливих
порушників по яких будумо цільово працювати.

І пам‘ятайте: 

Борітеся - поборете! 

Україна має бути українською!
