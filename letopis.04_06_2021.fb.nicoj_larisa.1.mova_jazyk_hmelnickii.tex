% vim: keymap=russian-jcukenwin
%%beginhead 
 
%%file 04_06_2021.fb.nicoj_larisa.1.mova_jazyk_hmelnickii
%%parent 04_06_2021
 
%%url https://www.facebook.com/nitsoi.larysa/posts/927677304465309
 
%%author Ницой, Лариса
%%author_id nicoj_larisa
%%author_url 
 
%%tags 
%%title Прийдете сьогодні додому і скажете: «Тато і мама, ми ж українці, живемо в Україні, пора переходити на українську мову
 
%%endhead 
 
\subsection{Прийдете сьогодні додому і скажете: «Тато і мама, ми ж українці, живемо в Україні, пора переходити на українську мову}
\label{sec:04_06_2021.fb.nicoj_larisa.1.mova_jazyk_hmelnickii}
\Purl{https://www.facebook.com/nitsoi.larysa/posts/927677304465309}
\ifcmt
 author_begin
   author_id nicoj_larisa
 author_end
\fi

Мені сердито написали:

- Пані Ларисо, що ви мовчите? Тут антимовні питання потураєв у Раді просуває, а вам що, подобається? Де ваша реакція?

Товариство, та я не встигаю всюди. Я в цей час була у Хмельницькому серед
школярів. Домовлялася з меншими учнями, щоб вони своїх російськомовних батьків
учили української мови. 

\ifcmt
  pic https://scontent-cdt1-1.xx.fbcdn.net/v/t1.6435-9/195464976_927677271131979_7397519433280763589_n.jpg?_nc_cat=101&ccb=1-3&_nc_sid=8bfeb9&_nc_ohc=qHboAgXqNO0AX92M7MA&_nc_ht=scontent-cdt1-1.xx&oh=4a7a825f9fc76c62ce764be0cb21f5ae&oe=60E0CA82
\fi

- Ви он як гарно українською розмовляєте! – хвалю дітей після того, як
поговорили про книжки. - Ви повинні своїх рідних так само навчити. Знаєте, чому
вони розмовляють російською?

- Чому? – питають діти.

Стишую голос.

- Бо вони не вчили або забули українську і тому соромляться, що скажуть не так,
а ви будете з них сміятися.

Малі личка видовжуються від співчуття. Бачу, вони вже знають, як це неприємно,
коли з тебе хтось сміється.

- Я думаю, ви можете своїм татам і мамам допомогти, - кажу довірливо.

- Як?! – пожвавлюються діти.

- Прийдете сьогодні додому і скажете: «Тато і мама, ми ж українці, живемо в
Україні, пора переходити на українську мову. Не бійтеся, що скажете якось не
так. Я вам допоможу. Я буду підказувати вам українські слова». 

Діти весело сміються. Питаю в них:

- Домовилися? 
- Так! – хором обіцяють діти. 

А потім були старші школярі. Але це інша тема.  

Тому, фейсбучне товариство, я була поза новинами, і не знала, що там у нашій
країні відбувається. А своє \enquote{фе} потураєву я вже висловила трохи раніше, на
телебаченні. Публічно. На всю країну. Але ви сказали туди не ходити. То я й не
ходжу. Ходжу в школи.

\begin{itemize}
\iusr{Ярослав Вішталюк}

Сьогодні в Києві повз парочку проходив - бабця з усіх сил намагалась
розговорити онучка українською, слова підказувала, виправляла, наводила, але
онучок вперто шпарив їй окупантською без жодного українського слова.  Так що
коли хоча б діти розмовляють - то вже непогано.

\iusr{Дмитрий Храмцов}

Ярослав Вішталюк За внучка можно не волноваться-его будущее вполне обеспечено! Английский бы еще подтянуть...

\iusr{Людмила Котубей}

Нашому синочку 4 роки. Бабусі-62. Вчаться розмовляти українською правильно
разом..  І досить часто в нас можна почути \enquote{Бабуся, ти не так. От я
зараз скажу, а ти так повтори!}

\iusr{Константин Попов}

файний вiдкош потурьайєву

\end{itemize}
