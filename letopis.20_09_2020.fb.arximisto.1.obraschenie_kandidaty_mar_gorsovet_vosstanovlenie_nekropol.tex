%%beginhead 
 
%%file 20_09_2020.fb.arximisto.1.obraschenie_kandidaty_mar_gorsovet_vosstanovlenie_nekropol
%%parent 20_09_2020
 
%%url https://www.facebook.com/arximisto/posts/610963659601764
 
%%author_id arximisto
%%date 20_09_2020
 
%%tags 
%%title Обращение к кандидатам в Мариупольский городской совет - о восстановлении Мариупольского Некрополя
 
%%endhead 

\subsection{Обращение к кандидатам в Мариупольский городской совет - о восстановлении Мариупольского Некрополя}
\label{sec:20_09_2020.fb.arximisto.1.obraschenie_kandidaty_mar_gorsovet_vosstanovlenie_nekropol}

\Purl{https://www.facebook.com/arximisto/posts/610963659601764}
\ifcmt
 author_begin
   author_id arximisto
 author_end
\fi

РУС \textbackslash\ УКР

Обращение к кандидатам в Мариупольский городской совет - о восстановлении
Мариупольского Некрополя

Уважаемые земляки!

В ходе предвыборной кампании вы будете чествовать мариупольцев предыдущих
поколений - за их подвиги и свершения. Ибо мы существуем благодаря их
достижениям. 

При этом, ваши представления о героях прошлых времен будут разными, иногда
абсолютно противоположными.  

Это естественно. В нашем обществе каждый имеет право на собственное мнение. В
том числе, об истории и ее героях. Многие мариупольцы отдали за это свою жизнь.

Но мы считаем недопустимым, когда в полном запустении пребывает символ памяти о
создателях нашего города - Мариупольский Некрополь (Старое \textbackslash\ Клиновое
кладбище)! 

В забвении и мусоре лежат и греки-первопоселенцы, и строители советских
металлургических гигантов, герои Первой и Второй мировой войн, советские
подпольщики и члены подполья ОУН, мэры города и десятки тысяч рабочих,
инженеров, моряков, учителей и врачей, немцев, греков, украинцев, итальянцев,
русских, евреев, поляков...

Призываем вас принять участие в восстановлении Мариупольского Некрополя - как
во время выборов, так и после того, как вы будете избраны в совет Мариупольской
громады.

Это наша общая ответственность - чтобы память о создателях Мариуполя жила, а их
вклад высоко ценился. 

Звернення до кандидатів в Маріупольську міську раду - щодо відновлення
Маріупольського Некрополя

Шановні земляки!

Протягом передвиборчої кампанії ви будете вшановувати маріупольців попередніх
поколінь - за їхні звершення та подвиги. Бо ми існуємо завдяки їхнім
досягненням.

Разом з тим, ваші уявлення про героїв минулих часів будуть різними, іноді
навіть абсолютно протилежними. 

Це природно. У нашому суспільстві кожен має право на власну думку. В тому
числі, щодо історії та її героїв. Багато маріупольців виборювали це право ціною
власного життя.

Але ми вважаємо неприродним та неприпустимим, коли в самому центрі міста в
повному запустінні перебуває символ памяті  про творців нашого міста -
Маріупольський Некрополь (Старе \textbackslash\ Клинове кладовище)! 

Серед сміття лежать і греки, яких переселили з Криму наприкінці XVIII ст., і
будівельники радянських металургійних гігантів, герої Першої та Другої світових
війн, радянські підпільники та члени підпілля ОУН, міські голови та десятки
тисяч робітників, інженерів, моряків, вчителів та лікарів, німців, греків,
українців, італійців, росіян, євреїв, поляків...

Ми закликаємо вас взяти участь у відновленні Маріупольського Некрополя - як під
час виборів, так і після того, як ви будете обрані в раду Маріупольської
громади.

Це наша спільна відповідальність - щоб пам'ять про творців Маріуполя жила, а
їхній внесок високо цінувався!

Контакти ГО \enquote{Архі-Місто}: arximisto@gmail.com, 096 463 69 88
