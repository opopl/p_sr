% vim: keymap=russian-jcukenwin
%%beginhead 
 
%%file 28_12_2021.stz.news.ua.volyn.1.lisova_egoigrashka_konkurs
%%parent 28_12_2021
 
%%url https://www.volyn.com.ua/news/201932-u-kolkakh-na-volyni-vyznacheno-peremozhtsiv-konkursu-lisova-eko-ihrashka-foto
 
%%author_id 
%%date 
 
%%tags konkurs,isskustvo,kultura,igrushka,ekologia
%%title У Колках на Волині визначено переможців конкурсу «Лісова еко-іграшка» (Фото) 
 
%%endhead 
\subsection{У Колках на Волині визначено переможців конкурсу «Лісова еко-іграшка» (Фото)}
\label{sec:28_12_2021.stz.news.ua.volyn.1.lisova_egoigrashka_konkurs}

\Purl{https://www.volyn.com.ua/news/201932-u-kolkakh-na-volyni-vyznacheno-peremozhtsiv-konkursu-lisova-eko-ihrashka-foto}

Подивитися справді є на що...

Сьогодні конкурсною комісією було відзначено кращих учасників конкурсу «Лісова
еко-іграшка», що оголошений Волинським обласним управлінням лісового та
мисливського господарства спільно з обласною організацією Товариства лісівників
України та обласною організацією профспілки працівників лісового господарства,
повідомляє Прес-служба ДП «Колківське лісове господарство».

\ii{28_12_2021.stz.news.ua.volyn.1.lisova_egoigrashka_konkurs.pic.1}

\begin{zznagolos}
Конкурс проводиться з метою популяризації еколого-лісівничих знань серед
учнівської молоді, виховання любові та шанобливого ставлення до природи,
формування у дітей екологічного світогляду, відповідального ставлення до
навколишнього природного середовища, звичаїв та природи рідного краю.	
\end{zznagolos}

Усі роботи варті уваги, адже кожна по своєму відрізняється оригінальністю та
креативністю. Конкурс проводиться з метою популяризації еколого-лісівничих
знань серед учнівської молоді, виховання любові та шанобливого ставлення до
природи, формування у дітей екологічного світогляду, відповідального ставлення
до навколишнього природного середовища, звичаїв та природи рідного краю.

\ii{28_12_2021.stz.news.ua.volyn.1.lisova_egoigrashka_konkurs.pic.2}

Отже, за результатами конкурсу:

І місце заслужено посіла робота вихованки зразкової художньої студії «Палітра»,
учениці 8-А класу ОЗ «Колківський ліцей» Ангеліни Бортник, оригінальна
еко-іграшка «Новорічне диво»;

ІІ місце посіла спільна робота учениці 9 класу Рудниківського ліцею Валентини
Винарської та учениці 8 класу Ангеліни Кушнір із назвою «Лісові мешканці»;

ІІІ місце посіла робота учениці ОЗ «Колківський ліцей» Злати Лободи під назвою
«Новорічний подарунок».

Усі переможці обов’язково будуть нагороджені цінними подарунками від
підприємства.

\ii{28_12_2021.stz.news.ua.volyn.1.lisova_egoigrashka_konkurs.pic.3}

\begin{zznagolos}
Кращі роботи направлятимуться на адресу Волинського обласного управління
лісового та мисливського господарства для участі у конкурсі на обласному рівні.	
\end{zznagolos}

Кращі роботи направлятимуться на адресу Волинського обласного управління
лісового та мисливського господарства для участі у конкурсі на обласному рівні.

Тому є можливість на подвійну перемогу!
