% vim: keymap=russian-jcukenwin
%%beginhead 
 
%%file slova.geroj
%%parent slova
 
%%url 
 
%%author 
%%author_id 
%%author_url 
 
%%tags 
%%title 
 
%%endhead 
\chapter{Герой}
\label{sec:slova.geroj}

блин, а есь хотя бы одно предсказание, где мы будем жить вечно, радоватся,
плясать, бухать и балдеть, после чего, по синьке, завоюем галактику, станем
\emph{Героями} и о нас будут как о богах слагать песни и легенды!!?? Почему
всегда какие то страсти!!??)) Я к примеру за такое развитие дальнейших
событий!!))
\citComment{Шура Иванов},
\citTitle{Монах предсказал ужасное будущее для всего мира}, ZOLEF, zen.yandex.ru, 07.06.2021

\enquote{Слава героям! Слава Героям! Честь и достоинство не продаются за газ и
нефть!} - пишет пользователь Jais McKlon. \enquote{Какая ваша цена,
\enquote{паны} из УЕФА? может рубль, может два? Слава Украине!!!! Героям
Слава!!!!} - возмущается Антин Трубчик. \enquote{Слава Украине! Героям слава!
Украина защита для Европы и цивилизации мира! Эти слова - святые слова!} -
считает Дмитрий Конопацкий.  \enquote{Слава Украине! Слава Героям! каждый из
нас, кто от имени Украины борется за ее имя ГЕРОЙ, и не имеет значения или
политик, музыкант, футболист, или военный}, - высказался Barry Doran.
\enquote{Почему бы не требовать изменений в командной форме России и не снять
ее с чемпионата всей командой внутри? сколько бы вы взяли за ЭТО? мы начнем
финансирование толпой. Слава Украине! Слава героям!} - написал Dmitry
Klimovsky,
\citTitle{Украинцы раскритиковали требование УЕФА убрать с формы сборной слова Героям слава!}, Юлия Супрун, strana.ua, 10.06.2021

%%%cit
%%%cit_head
%%%cit_pic
%%%cit_text
Мы -- нация \emph{Героев}, мы -- россияне! Нам управлять этой огромной громадой, нашей
великой страной, и нам в ней жить справедливо и радостно, на зависть остальному
Миру, который не желает разделять наших замечательных жизненных ценностей.
Виват, Россия! Виват, Россияне! Бог с нами! А это самое главное
%%%cit_comment
%%%cit_title
\citTitle{Слава России! Мы на вершине могущества и спускаться более не намерены}, 
Алексей Наст, zen.yandex.ru, 12.06.2021
%%%endcit

%%%cit
%%%cit_head
%%%cit_pic

\ifcmt
  tab_begin cols=2
	width 0.3

     pic https://strana.ua/img/forall/u/0/36/rue4fdd579808.jpg

     pic https://strana.ua/img/forall/u/0/36/2021-06-30_12h47_06.png

  tab_end
\fi

%%%cit_text
В эфире программы \enquote{Великий футбол} \emph{герой матча} Довбик признался, что до
конца не осознает все, что произошло на поле и постепенно приходит в себя. 
\enquote{Наверное, это какая-то судьба. У меня просто нет слов. Такое стечение
обстоятельств, что надо было выходить и помогать партнерам, ждать свой шанс. Я
его все же дождался}, - поделился впечатлениями спортсмен.  Довбик также
вспомнил напутственные слова своего тренера Андрея Шевченко, который сказал,
чтобы Артем всегда находился в штрафной площадке и ждал шанс, старался
цепляться за мячи.  В Киеве у огромных экранов ТРЦ \enquote{Gulliver} фанаты были вне
себя от счастья
%%%cit_comment
%%%cit_title
\citTitle{Артем Довбик - что известно про футболиста, который вывел Украину в четверть финала Евро}, 
Анна Копытько, strana.ua, 30.06.2021
%%%endcit

%%%cit
%%%cit_head
%%%cit_pic
%%%cit_text
Річинський вказував на українські родинні корені Достоєвського по лінії батька.
Говорив і про те, що Достоєвський у «своїх писаннях навіть виявляє недобре
володіння російською мовою». До речі, це твердження, як уже говорилося,
небезпідставне. У творах Достоєвського можемо зустріти слова, які мали
українське походження. Також у цих творах можна побачити й граматичні форми,
які не дуже вписуються в рамки російської літературної мови. Достоєвський,
схоже, розумів, що його російська мова не дуже правильна. Але це по-своєму
виправдовував. Коли одному з \emph{героїв} роману «Біси» закидають те, що він
неправильно говорить російською мовою, той відповідає, що завжди так розмовляв.
Певно, так само міг себе виправдати й Достоєвський
%%%cit_comment
%%%cit_title
\citTitle{Арсен Річинський про українські аспекти творчості Федора Достоєвського (1821–1881)}, 
, www.radiosvoboda.org, 25.07.2021
%%%endcit

%%%cit
%%%cit_head
%%%cit_pic
%%%cit_text
Перед тим, як розпочати розповідь про моїх \emph{героїв}, наших сучасників, читачу, я
хочу переказати слова Ані — Сина Зірниці, однієї з легендарних постатей
шумерського епосу. Ці слова співзвучні тому, що я хочу висловити у книзі. У
приблизному перекладі афористичні твердження прадавнього мудреця й \emph{героя}
звучать так: «Мудрість — то безстрашність мислення. Правдиво мудрий лише той,
хто не творить для себе ідолів мислі. Хіба можна назвати мудрим того, хто лише
ровторює, заучує чужі міркування та переконання? Правдива мудрість знає, що
кожна проявлена думка — умовність. Нерушиме лише прагнення свідомості охопити
собою всебуттєвість, тобто — наповнити любов’ю й радістю кожну цятку всесвіту.
Збагнувши так, мудрий знає, що він на шляху правди, і той шлях неминуче приведе
його до храму тайни; де всі і все зустрічається.  Що шукає, що воліє древо,
виростаючи з маленького зерна?
%%%cit_comment
%%%cit_title
\citTitle{Вогнесміх}, Олесь Бердник
%%%endcit

%%%cit
%%%cit_head
%%%cit_pic
%%%cit_text
Досить часто \emph{героями} призначають відвертих невдах або навіть злочинців. Бо якщо
оглянутися в минуле і провести критичний аналіз діяльності ОУН, то виясниться,
що стратегія і тактика цієї організації були, м’яко кажучи, хибними. А ідея
«перманентної революції» найбільше біди завдала саме українцям, оскільки члени
ОУН бажали українцям нестерпних умов життя в міжвоєнній Польщі. Щоб вони не
мали іншого вибору і всі як один піднялися на націоналістичну революцію. Так
само фатальними виявилися тактика революційного терору та орієнтація на
нацистську Німеччину. У підсумку, проливши ріки крові і наражаючи на страшну
небезпеку українців, радикальні націоналісти нічого не досягли. Вони змушені
були забратися з української землі, а ті, що лишилися, обрекли себе на вірну
смерть. Тому приклад ОУН і її боротьба аж ніяк не надаються для розбудови
української держави у ХХІ столітті
%%%cit_comment
%%%cit_title
\citTitle{Мертві герої проти живих людей}, 
Василь Расевич, zaxid.net, 29.10.2021
%%%endcit

%%%cit
%%%cit_head
%%%cit_pic
%%%cit_text
Ось де були його \emph{герої}, ось де був його світ! Їх він кохав, як не кохав
ні брата, ні сестри, ні коханки. Їх славу невмирущу, їх одчайдушність безжурну
тих степових лицарів. Їх суворість варязьку, їх завзяття козацьке, їх віру
гарячу в сильного Бога, Бога слави і в Його справедливість. Їх славні або
замучені тіні виводить довгими рядами перед нами, Дорошенка, Гордієнка, Палія,
Швачку, Гамалію, Тараса Трясила, Залізняка, Гонту, Мазепу, Гайдая; їх, чия душа
незламна гарцювала в нім самім. Їх викликав з царства тіней, роздираючи завісу
минулого, запечатану многими печатями, їх, чиї блискучі постаті мов
бенгальським вогнем освічували тьму і неволю сучасности. До них звертається і
своїм Заповітом, бо ніхто, лиш вони могли пірвати кайдани ганьби, напоїти
вражою кровю землю нашу. Вони тільки могли створити з племені рабів вільну
націю, де була б своя правда, своя сила і своя воля, не воля й сила займанця,
ні його брехня
%%%cit_comment
%%%cit_title
\citTitle{Заповіт Шевченка}, , pravyysektor.info, 10.03.2018
%%%endcit

%%%cit
%%%cit_head
%%%cit_pic
\ifcmt
  pic https://volyn.com.ua/content/thumbs/800x/n/th/owvh5t-undefinedxundefined-undefinedxundefined-6zntxokb2xnukvcgpdafwsm45h45kthn.jpeg
  @width 0.4
\fi
%%%cit_text
\emph{Герої} її віршів – це люди й звірі, рідна школа, дощик і сонечко, космос, день і
ніч... Промінцем до реальності починає стелитися тема любові до України, рідно
мови, світу без насилля, в якому «мати не плаче, що сина нема». Ще вона просить
зникнути зі світу коронавірус. 49 поезій, які представлені до уваги
шанувальників поетичного слова, вражають своєю чистотою, дитячою щирістю та
відвертістю. На обкладинці книги використано ілюстрації Дарини Гав’янець
%%%cit_comment
%%%cit_title
\citTitle{Одинадцятирічна волинянка видала збірку власних віршів}, , www.volyn.com.ua, 01.11.2021
%%%endcit

%%%cit
%%%cit_head
%%%cit_pic
%%%cit_text
Кандидат филологических наук Анна Шестак предложила мне покинуть Украину, так
как мне не нравится Бандера. Свой коммент она оставила под постом о предстоящих
в Киеве ежегодных Бандеровских чтениях.  Анна пишет, что для части Украины
Бандера - \emph{герой}. И если я с этим не согласен, то лучше мне уехать.
Заодно она проводит параллель: если для меня Бандера - не \emph{герой}, значит,
я люблю советскую власть.  Действительно, часть украинцев считают Бандеру
\emph{героем}. Это, в основном, люди, которые ничего не знают о деятельности
ОУН-УПА, или, наоборот, знают, но считают, что массовые убийства женщин и детей
- благо.  Это не удивительно. В СССР богом считали Сталина, в нацистской
Германии - Гитлера. В Украине годы дезинформации и промывания мозгов тоже не
прошли зря и появились литературоведы - бандеровки требующие, как принято в
европейской и демократической стране, чтобы я покинул свою родину
%%%cit_comment
%%%cit_title
\citTitle{За нелюбовь к Бандере - вон за пределы Украины! / Лента соцсетей / Страна}, 
Эдуард Долинский, strana.news, 06.12.2021
%%%cit_url
\href{https://strana.news/opinions/365623-za-neljubov-k-bandere-von-za-predely-ukrainy.html}{link}
%%%endcit
