% vim: keymap=russian-jcukenwin
%%beginhead 
 
%%file 13_02_2023.fb.fb_group.mariupol.pre_war.8.kto_pomnit_eti_batis.cmt
%%parent 13_02_2023.fb.fb_group.mariupol.pre_war.8.kto_pomnit_eti_batis
 
%%url 
 
%%author_id 
%%date 
 
%%tags 
%%title 
 
%%endhead 

\qqSecCmt

\iusr{Ирина Сариджа}

Кажется каштаны на этой аллее всегда были большими...

\begin{itemize} % {
\iusr{Клавдия Макарова}
\textbf{Ирина Сариджа} Они наверно с вами росли и вы не заметили, как они стали большими.

\iusr{Ирина Сариджа}
\textbf{Клавдия Макарова} они выросли с моими детьми
\end{itemize} % }

\iusr{Ірина Хоменко}

А мы памперсы сыну.

\begin{itemize} % {
\iusr{Anzhelika Drotianko}
\textbf{Ірина Хоменко} там было всё в этих батискафах, не удивлюсь
\end{itemize} % }

\iusr{Ірина Хоменко}
Неее, не там, ошиблась

\iusr{Maryna Holovnova}

Ухти!!!!

\iusr{Zhanna Savelyeva}

В 93 купила там... свитер баленсиагу)

\iusr{Елена Аврамова}

Яке дивовижне фото, дякую!!! Я це пам'ятаю!

\begin{itemize} % {
\iusr{Anzhelika Drotianko}
\textbf{Елена Аврамова} это очень ранние 90е

\iusr{Павел Гаврилюк}
\textbf{Anzhelika Drotianko}

Да, это начало девяностых.

Были специально фирмы запущены в ход, по изготовлению этих батискафов, и
недешево они стоили, дорогие были.

И вот предприниматели вложили немало денег, но вскоре поменялись
градоначальники, и эти батискафы демонтировали. Позже эти батискафы можно было
видеть на заброшенных стоянках.

\end{itemize} % }

\iusr{Zhanna Savelyeva}

И коммерческий магазин был в конце дома, там много интересных одежек было

\begin{itemize} % {
\iusr{Ирина Ирина}
\textbf{Zhanna Savelyeva}, вы хотели сказать комиссионный?

\iusr{Zhanna Savelyeva}
\textbf{Ирина Война} 

в 90-е они стали называться... ммм.. кооперативные или коммерческие, там
продавали вещи челночники. А комиссионные - это вещи б/у или новые, но их
кто-то сдал. Сложные были варианты кооперации😆

\iusr{Ирина Ирина}
\textbf{Zhanna Savelyeva}, просто в 70х помню на перекрестке этой ул. Энгельса и Варганова через дорогу от башни был Комиссионный, я туда вещи сдавала. Думала, что у Вас описка. Я лично в этих киосках только шампунь купила, поэтому не помню чем там торговали.

\iusr{Zhanna Savelyeva}
\textbf{Ирина Война} да! Первыми коммерческими стали комиссионные, да. Вспомнилаааа! Их называли "скупками", туда сдавали вещи челночники. Как давно это было... память дает сбои...

\iusr{Ирина Ирина}
\textbf{Zhanna Savelyeva}, я такого не помню, что челночники туда вещи сдавали. Помню, что обычные люди свои б/у вещи туда носили. Я несколько раз в 70х, 80х сдавала. Тогда челночников не было. А челночницы носили вещи к нам на работу и по другим разным предприятиям и учреждениям, куда был доступ.

\iusr{Olga Yalo}
\textbf{Ирина Война} нет, комиссионки были отдельно, а еще были \enquote{скупки}, как отдельные магазинчики, так и отделы в тех же комиссионках. Там были только новые вещи.

\iusr{Ostapenko Vlad}
\textbf{Ирина Ирина} В 90-х и сигареты, жвачки, косметика, ткани, джинса и прочее... Всё что Моряки наколядовали в рейсе!😉

\end{itemize} % }


\iusr{Павел Гаврилюк}

Была эпопея с этими батискафами.

Один градоначальник устанавливает своё что-то, а сменяющийся новый мэр
демонтирует их, и на место их ставит что-то своё. И так по спирали.

\iusr{Наталия Галактионова}

В 1992 году там покупала косметику в этих киосках

\iusr{Дмитрий Шереметьев}

Дуже дякую, навіть не здогадувався, що а нашому місті було таке

\begin{itemize} % {
\iusr{Anzhelika Drotianko}
\textbf{Дмитрий Шереметьев} как я выше написала, это были очень ранние 90е🤗

\iusr{Talik Kravchenko}
\textbf{Anzhelika Drotianko} чего ранние? Где-то с года 93, если не ошибаюсь. В ранних там, как помню, люди с земли торговали.

\iusr{Anzhelika Drotianko}
\textbf{Талик Кравченко} ну да, это ранние...92-93...в 98- 99 м уже их не было

\iusr{Ирина Ирина}
\textbf{Талик Кравченко}, я там во времена перестройки шампунь покупала. Вверху мужчина написал, что эти киоски то ставили, то убирали. Хотя я этого не помню.
\end{itemize} % }

\iusr{Anatolii Shcherbina}

Оо, раннее детство!)

\iusr{Derevianko Ira}

Фото назыаается \enquote{почувствуй себя пенсионером}...🤣

\iusr{Talik Kravchenko}

Видеокассеты там покупали и так по мелочи.

\iusr{Marina Schreiner}

А какие там плакаты были! Оооооо! Как в музей поглазеть ходили, комиссионка там
была рядом прям.

\begin{itemize} % {
\iusr{Яна Шульга}
\textbf{Marina Schreiner} я купувала там плакати і фотографії Сандри і Сабріни. А ще там можно було стати в чергу на пральну машину
\end{itemize} % }

\iusr{Виктория Победа}

И в переходе было много киосков)))

\iusr{Люба Копейкина}

Я приехала в Мариуполь в 1993 году, они уже стояли)

\iusr{Ekaterina VS}

Да, если бы не напомнили, точно не вспомнила бы...

\iusr{Ирина Ирина}

Я помню как во времена Горбачева все стало дефицитом, и в одном из этих киосков
вдруг появились французские духи, еще что то импортное но и советские средства
гигиены. Мы тогда из всего этого старались сделать запасы. Я в тот день была в
другом месте, а подошла туда ближе к обеду. В очереди за французскими духами
стояли двое коллег с самого утра. Я к ним пристроилась, у меня шампунь
заканчивался. Духи вскоре закончились, коллеги решили уйти, а я осталась. Мне
пришлось простоять на морозе до вечера, когда стемнело. В итоге купила бутылку
нашего шампуня Селена. Об импорте к тому времени все забыли. Вот так Горбачев
довел страну. Больше в тех киосках ничего не покупала.

\iusr{Sergey Drovorub}

Я хорошо помню. Уникальная форма. Жевательные резинки Турбо, Бомбибом, Дональд,
Типи Тип - был основной товар, который мог интересовать, тогда еще некурящих
пацанов. А рядом игровой клуб с компьютерами Атари и игры по очереди, на деньги
выданные родителями или тетей с бабушкой и дедушкой. Карате или формула один
какая-нибудь или бродилка. В том же помещении и видеосалон, где приобщались к
новой тогда культуре, с теми же восточными единоборствами или фильмами ужасов,
кому что больше по душе.

\begin{itemize} % {
\iusr{Ostapenko Vlad}
\textbf{Sergey Drovorub} 5 стеклянных бутылок нашёл потом сдал по 20 копеек и 10 минут на Atari зависаешь у Дяди Миши... Ещё и звукозапись была!!!😉

\iusr{Sergey Drovorub}
\textbf{Ostapenko Vlad} или так. А в кинотеатре Родина, мы с пацанами иногда подвязывлись на замену афиш в окрестностях. Новые афиши разносили и сменяли ими старые, которые приносили художнику обратно. За это нам был свободный вход на любые сеансы, после их начала, когда можно было присесть на свободные кресла, если билеты не распроданы.
\end{itemize} % }

\iusr{Sergey Drovorub}

Их на Азовстали изготавливали?

\begin{itemize} % {
\iusr{Zhanna Savelyeva}
\textbf{Sergey Drovorub} кажется, Азовмаш, машиностроители

\begin{itemize} % {
\iusr{Sergey Drovorub}
\textbf{Zhanna Savelyeva} спасибо! 🙂
\end{itemize} % }

\iusr{Клавдия Макарова}
\textbf{Sergey Drovorub} 

Нет их изготавливали на частном предприятии перед Азовстальским мостом там где
был раньше синичный завод 🏭. Да был у нас такой его ещё его называли
ультрамариновый там делали синьку. До войны потом были гаражи и база \enquote{Маяк}.Что
сейчас там могу только догадываться. На пересечении ул. Шевченко и Торговой прям
на углу слева был маяк.

\begin{itemize} % {
\iusr{Sergey Drovorub}
\textbf{Клавдия Макарова} спасибо! Точно! Припоминаю, там их было валом 🙂

\iusr{Zhanna Savelyeva}
\textbf{Клавдия Макарова} помню! Люди столько не живут)). Моя бабуля, чтоб селина была серебряной, немного подсинивала воду для мытья головы. Стайл!)

\iusr{Клавдия Макарова}

Ну вы даёте. Это вы про меня? .Столько не живут. Я там жила с 1983г. моей дочке
было 4 года когда его закрыли. Люди разнесли его по кирпичикам, строили дачи.
Потом там было просто поле дети играли там в мяч. Потом частное предприятие
которое сваривало вот эти киоски. Так что и с памятью у меня всё в порядке.

\end{itemize} % }

\end{itemize} % }

\iusr{Александр Куровский}

Ничего ж себе... Календарь-плакат что справа в первом киоске, я себе покупал.
Именно его и именно в нем. Зачетная барышня. Год эдак 90-91.

\begin{itemize} % {
\iusr{Sergey Drovorub}
\textbf{Александр Куровский} точно! Про плакаты я забыл! У меня по-ммрем был нинзя с мечем, на фоне бамбука или тростника, там купленый.
\end{itemize} % }

\iusr{Лариса Дятченко}

Я хорошо помню батискафы, тогда они не очень радовали глаз, а сейчас ...память.

\iusr{Людмила Кудрявцева}

Стояли в очередь за китайськими полотенцами

\iusr{Юля Захарченко}

Дивіться, що зараз відбувається у Маріуполі!

\url{https://fb.watch/iFXZcsfvHL/}

\iusr{Natali Linnyk}

Спасибо, что напомнили!

\iusr{Вікторія Різник}

А я щось такого не пам'ятаю 🤷. Хоча в 91 році мені було 25 років

\iusr{Liliya Liliya}

Наше поколение помнит😉🤗

\iusr{Юрий Свиридов}

Если не ошибаюсь, их проектировали в Донецком НИИЧермете в начале 80-х

\iusr{Геннадий Скалозуб}

Да, было время.

\iusr{Екатерина Воробьёва}

Та усі, кому за.. пам'ятають)

\iusr{Yulia Kofa}

Вот это да! Какая фотография! Помню эти батискафы😅🥳

Но как же стало свободно на аллее, когда их убрали. Любила наш Арбат

\iusr{Svitlana Shmidt}

Не пам'ятаю

\iusr{Андрій Прокопов}

Я багато таких \enquote{батискафів} наробив, коли був студентом. Не ці самі, бо
в 1989-му, здається, вони вже стояли.

\iusr{Nataliia Volkova}

Помню....

\iusr{Сергій Андрієнко}

Лихие девяностые...

\iusr{Геннадий Скалозуб}

Анжелика! Благодарю за фото. Всех земных благ.

\iusr{Іванова Яна}

На їхньому фоні ще робили фото фотографи)

\iusr{Leonid Ehdelshteyn}

О дефиците 90-х годов. Я не помню, стояли в 1990 году батискафы или нет,
кажется что да, но там, на Арбате, была в 1990 году анонсирована
предварительная запись в живую очередь, на многие тогда дефицитные товары, типа
холодильников, стиральных машин и другого такого же дефицита. А я обещал жене и
детям поездку в Ленинград на 15 дней. Мы уже и отпуска оформили на это время.
Вроде бы даже, первые из этой живой очереди что-то и получили, ежедневно
отмечаясь несколько дней. Мы же послали эту запись лесом и рванули в Ленинград.

И ещё о дефиците. Один раз я в такой очереди постоял. В 1991 году, возле ЦУМа,
была запись в очередь за импортной мужской и женской обувью. Стали в неё с
женой в 18.00 вечера. Получили номера. В 21. 00. отправил жену спать, сам
остался на ночь. Таких как я, оставшихся, было пол очереди. Списки вела
знакомая моей жены, она их вела до утра, где-то в 7. 00 утра, потом наша
знакомая вырубилась и ушла спать, передав списки мне. Потом, где-то в 10.00
пришла жена, затем пришла выспавшаяся знакомая, но я уже дирижировал этой
очередью так, как будто всю жизнь этим занимался. Привезли обувь, где-то в 12.
00, и началась веселуха. Я с двумя добровольными помощниками, а потом сам, они
получили свою обувь раньше меня, по своей очереди, только тем и занимался, что
выводил тех, кто лез без очереди. Все дружно указывали на нахалов, на некоторых
из них действовало доброе слово, некоторых приходилось выводить, ну а самых
наглых выбрасывать из очереди, в прямом смысле. Я с женой, оторвали три пары
обуви, давали по одной в руки, я взял лишнюю, за боевые заслуги, никто и не
подумал возражать. Хватило этой обуви на одну зиму, мне, жене и подросшей дочке
(третья пара)

\begin{itemize} % {
\iusr{Яна Шульга}
\textbf{Леонид Эдельштейн} 

да, я точно помню, что там записывали в очередь на технику. А мы обувь покупали
в магазине Взуття. Я с подругой дежурила в очереди несколько дней, в итоге
взяли маме 2 пары сапог не ее размера, какие достались))) они лежали лет 10
пока не выкинули.

\end{itemize} % }

\iusr{Ірина Калініна}

Вообще не помню.... И деревья еще маленькие на аллее...

\iusr{Марина Подвысоцкая}

А мне куртку очень крутую купили... радости было)))) это 1993 год

\iusr{Лариса Голота}

А я любила все мариупольские парки. Они были один другого краше. Каждый имел
свою изюминку. Не знаю, доведется ди когда-нибудь посетить хотя-бы один из них,
а лучше все...

\iusr{Irina Koshkalda}

А я там рядом жила! Мой дом за башней виднеется.. Ларьки на нашем местном \enquote{арбате} стояли.

\iusr{Людмила Нікітіна}

не помню

\iusr{Roman Pykhtin}

А какой это год ??

\iusr{Roman Pykhtin}

Это были батискафы частные или от мариупольского торга ?

