% vim: keymap=russian-jcukenwin
%%beginhead 
 
%%file 24_01_2022.fb.bereza_borislav.1.ne_nado_panikovat
%%parent 24_01_2022
 
%%url https://www.facebook.com/borislav.bereza/posts/7631200400239333
 
%%author_id bereza_borislav
%%date 
 
%%tags panika,rossia,ugroza,ukraina,vtorzhenie
%%title Не надо паниковать. Надо трезво смотреть на вещи и понимать, что происходит
 
%%endhead 
 
\subsection{Не надо паниковать. Надо трезво смотреть на вещи и понимать, что происходит}
\label{sec:24_01_2022.fb.bereza_borislav.1.ne_nado_panikovat}
 
\Purl{https://www.facebook.com/borislav.bereza/posts/7631200400239333}
\ifcmt
 author_begin
   author_id bereza_borislav
 author_end
\fi

Ладно, я дозрел таки до этого поста. После 3716 звонка с вопросом о том, "а
будет ли война?", я понял, что проще один раз написать, чем отвечать каждому. 

Начнём с того, что война и так идёт уже 8 год. Но многие не чувствуют её из-за
того, что войны нет в их жизни, в информационном поле и в ежедневной повестке.
Отдых в Египте или на Буковеле в их жизни есть, а войны нет. Поход в кино на
новый блокбастер есть, а сводок с фронта нет. Ужин в ресторане или шашлыки на
природе есть, а сбора помощи для фронта нет. А ещё есть поход в музей, в театр,
на футбол или чтение книг, но нет всего того, что напоминает о том, что наши
воины ежедневно находятся под обстрелом и рискуют здоровьем или жизнью защищая
право всех нас на все выше перечисленное. 

В моей жизни все это есть. И отдых, и  шашлыки, и кино, и передача помощи для
фронта, и противодействие российским информатакам, и все остальное. Да, всё это
в моей жизни с 2014 года. И я живу в постоянном ожидании полномасштабной войны.
Я очень её не хочу. Но она более чем вероятна. В 2014 году мало кто верил, что
Кремль захватит Крым и часть Донбасса, хорошо вооружит сепаратистов и перекинет
в Украину свои войска, предварительно сняв с них опознавательные знаки, а так
же будет убивать украинцев и системно проводить в Украине ИПСО. А все так и
произошло. Поэтому я верю в реальность полномасштабной войны развязанной
Кремлём. Но я вижу и иное. Нет причин утверждать, что война стопроцентно будет.

По периметру Украины собраны большие силы врага. Но этих сил явно недостаточно
для полноценной успешной операции по захвату Украины. И тут ключевое слово
"УСПЕШНОЙ". Более того, эти силы не обеспечены всем необходимым, для начала
длительных активных боевых действий на нашей территории. Но не надо меня
спрашивать откуда я это знаю. Просто поверьте, что это более чем надёжная
информация. Я больше недели ежедневно контактирую с офицерами разведки,
Генштаба и вооруженных сил Украины. И они, как раз спокойны и не паникуют. Хотя
именно им и их подопечным придётся защищать Украину в случае эскалации
ситуации. И они уверены в своих силах. А ещё они напоминают, что в свое время
Кремль не смог победить Чечню, которая меньше, чем Черниговская область. Так,
что говорить об их возможности победить Украину? Порыв они организовать могут,
но это не даст им победы, зато обеспечит десятки тысяч гробов в Москву, Питер,
Рязань, Грозный, Ебург и другие города РФ. Поэтому не паникуйте. Вот это важно.
Не паниковать!

Есть так называемая "Военная доктрина Герасимова". Эта доктрина переосмысливает
современное понятие межгосударственного конфликта, а военные действия ставит в
один ряд с политическими, экономическими, информационными и другими невоенными
действиями. Доктрина стала известной после публикации статьи Герасимова
«Ценность науки в предвидении» в издании «Военно-промышленный курьер» в феврале
2013 года и последующих действий России по отношению к Украине, полностью
совпадающих с тезисами этой доктрины. В общем, это описание той самой гибридной
войны, которую с 2014 года ведёт против Украины Кремль. Так вот, в ней есть все
то, что сегодня мы видим - создание паники в информационном пространстве,
хакерские атаки на инфраструктуру, психологическое давление через массовое
ложное минирование и концентрацию сил на границе, давление на экономику и
энергошантаж, дискредитация лидеров общественного мнения и формирование страхов
у населения, а так же многое другое. Знакомо? Да, все это Кремль реализует в
противостоянии с Украиной. Но запомните главное, если в вашей голове уже живёт
паника и российские танки уже в ваших мыслях въехали в Киев, то значит ФСБ и
ГРУ добились своей цели. Поэтому главное не паниковать и верить в нашу армию. А
она у нас есть. И она сегодня во много раз сильнее и опытнее, чем в 2014-ом. К
тому же есть порядка 300 000 атошников с опытом утилизации оккупантов. Огромное
количество оружия на руках и в схронах. Плюс активно формируются силы
теробороны. А это что значит? Что нами подавятся. Вот это должно доминировать в
вашей голове. А иначе российские спецслужбы вас уже победили. 

Хотя, кто-то может сказать, что постоянно жить в ожидании войны очень тяжело.
Тяжело, но возможно. Посмотрите на Израиль. Они живут в таком состоянии уже
более 70 лет. В Израиле у меня есть близкий друг Миша Левин. У Миши маленький
бизнес. Он учитель вождения. Но если завтра начнется война и враги нападут на
Израиль, то Миша пойдёт защищать свое государство. А ещё у Миши есть два сына
Яша и Дудик. И они тоже пойдут под ружье. И Эдик, муж моей двоюродной сестры,
возьмёт тавор в руки и оденет форму. Просто у них нет другой страны, а свою они
терять не хотят. Уехать они конечно могут. Но это точно не их выбор. И я знаю,
что говорю. Вот поэтому и я никуда не уеду. Это моя страна. И я буду защищать
Украину. 

Не надо паниковать. Надо трезво смотреть на вещи и понимать, что происходит.
Да, угроза есть. Но это ещё не 100-ная гарантия, что полноценная война будет.
Особенно если враг понимает, что он не добился паники, лёгкой прогулки у него
не будет, а победить и держать Украину под контролем он не сможет. И помните,
что никто кроме нас самих Украину не защитит. Союзники дадут нам орудие и
наложат санкции на Кремль, но вонвать мы будем сами. Надо быть готовым защитить
свою семью, свой дом, своих друзей и близких, свои города и села, свою Украину!
И свои жизни. Мир уважает и считается с теми, кто проявляет силу и характер.
Значит будем такими. И не паникуйте. Ведь это бесперспективно и не несёт ничего
хорошего. Как говорится: предупрежден - значит вооружён. Вы теперь вооружены.
Как минимум информационно. Вот и делитесь этой информацией с окружающими и
сейте в окружающих уверенность в своих силах. Потому что победа начинается
именно с этого. И с веры в Вооруженные Силы Украины!
