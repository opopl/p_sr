% vim: keymap=russian-jcukenwin
%%beginhead 
 
%%file 11_12_2020.news.ua.strana.medvedeva_olesja.1.todurov_interview
%%parent 11_12_2020
 
%%url https://strana.ua/articles/interview/305908-boris-todurov-intervju-olese-medvedevoj-video-i-tekst.html
 
%%author Медведева, Олеся
%%author_id medvedeva_olesja
%%author_url 
 
%%tags todurov_boris
%%title Борис Тодуров: "Супрун к нам прислали как агента международных корпораций"
 
%%endhead 
 
\subsection{Борис Тодуров: \enquote{Супрун к нам прислали как агента международных корпораций}}
\label{sec:11_12_2020.news.ua.strana.medvedeva_olesja.1.todurov_interview}
\Purl{https://strana.ua/articles/interview/305908-boris-todurov-intervju-olese-medvedevoj-video-i-tekst.html}
\ifcmt
	author_begin
   author_id medvedeva_olesja
	author_end
\fi

\index[names.rus]{Тодуров, Борис!Директор Института Сердца, Киев}

\ifcmt
  pic https://strana.ua/img/article/3059/boris-todurov-intervju-8_main.jpeg
  caption Борис Тодуров и Олеся Медведева. Кадр из фильма-интервью 
\fi

Знаменитый украинский кардиохирург Борис Тодуров дал интервью спецкору "Страны"
и автору блога "Ясно.Понятно" Олесе Медведевой. 

Олеся провела день в Институте сердца, которым руководит Борис Михайлович.

Беседой дело не ограничилось – Тодуров показал журналистке "Страны", как он
делает свои операции: практически нон-стоп, перемещаясь из одной операционной в
другую.

Часть Института сердца сегодня переведена под ковидных больных. В перерывах
между операциями мы поговорили с Борисом Тодуровым о том, когда закончится
пандемия в Украине и как дошло до такой ситуации с коронавирусом, которую мы
имеем сейчас.  

Врач рассказал "Стране", почему уникальный институт с первых же дней работы в
Минздраве невзлюбила Ульяна Супрун. И какой вклад ее реформ в ту медицинскую
катастрофу, которая наблюдается сегодня в ходе пандемии. 

Также Тодуров объяснил, что Украине нужно делать с российской вакциной от
коронавируса. Детальнее – в фильме Олеси Медведевой.

\video{https://youtu.be/M2z7UjtjZqw}

Также ниже приводится текстовая расшифровка интервью. 

\subsubsection{\enquote{Мы взяли в пример худшую в Европе медицину}}

– Когда мы входили в пандемию, наша система здравоохранения оказалась абсолютно
к этому не готова. Несмотря на то, что вот уже несколько лет подряд вроде как
всем рассказывали о том, что была очень эффективная и продуктивная медицинская
реформа. Почему так вышло? Почему мы оказались совершенно не готовыми к тому,
что мы имеем сейчас?

– Несколько лет назад началась так называемая реформа, на самом деле большая
авантюра, которая привела к полной потере управленческой вертикали. В медицине,
в отличие от административных вещей, нельзя было делать децентрализацию, потому
что Украина – большая многомиллионная страна. Мы имеем опыт Чернобыля, мы имеем
опыт техногенных катастроф, и мы должны понимать, что что-то аналогичное может
случиться в любой момент. У нас четыре атомных станции. У нас огромное
количество заводов. В том числе тех, которые работают с химией, с какими-то
препаратами и так далее. Мы уже пережили несколько эпидемий, и мы должны быть
готовыми к тому, чтобы эта управленческая вертикаль работала безукоризненно.

У нас была до 2017 года служба санэпидемстанции, которая контролировала
санитарно-эпидемиологическую обстановку, которая могла реагировать адекватно на
изменение этой обстановки, которая контролировала основные инфекции, которые
появлялись в Украине. Это и дифтерия, и корь, и грипп, и многое другое.

Эта служба была заточена с советских времен на борьбу с этими эпидемиями. И
нужно отдать ей должное: та советская система, которую сейчас все так хают,
именно она победила эти основные эпидемии в то время в Советском Союзе в
послевоенные и довоенные годы. И туберкулез победила, и корь, и дифтерию, и
сибирскую язву, и брюшной тиф, и много-много того, что уносило тысячи жизней. И
это была абсолютно четко построенная система, рассчитанная на профилактику
заболевания и на предотвращение заболевания.

Когда я был школьником, были диспансерные осмотры. Ты рождался, на следующий
день после выписки приходила патронажная сестра из детской поликлиники, она
контролировала все прививки, была карточка в детской поликлинике на ребенка до
16 лет, делались все прививки, никто это не обсуждал, ни у кого это не
спрашивали, и была ситуация, которая до сих пор нас защищает.  

Те прививки АКДС (обеспечивают ребенку иммунитет от коклюша, дифтерии и столбняка. – Ред.) до сих пор защищают бывших жителей Советского Союза даже от сегодняшнего ковида. Вы знаете, что меньше болеют. 

– Так это доказанный факт? Просто очень много противоречивых слухов по этому поводу. 

– Да, статистика по Восточной и Западной Германии говорит о том, что в
Восточной Германии, где живет большинство людей, привитых АКДС, болеют на 30\%
меньше. 

Эта система каким-то образом существовала до последних лет, и вдруг кто-то
решил, что не нужно вкладывать в эту систему деньги. Она, собственно,
существовала на очень экономные деньги – порядка 2,8-3\% от бюджета. И эта
система была более-менее рабочей. Кто-то сказал, что "плохая система",
советская, типа система Семашко не годится никуда, давайте будем делать
реформу.

– А как вы думаете, кто сказал, что нам это не подходит?

– Это те, кто нам последние годы говорят после революции, куда нам надо
двигаться, какую демократию строить, на каком языке разговаривать, какому Богу
молиться, в какую церковь ходить. Кто-то решил из этих людей, что нужно и здесь
немножко подкорректировать, забыв при этом профинансировать эту реформу. 

Такая реформа требовала как минимум 7\% в бюджет. Так, как это делается в
западных странах. Минимум 7\%, а в некоторых странах и 12\% из бюджета страны
тратится на медицину. Вот это могла бы быть реформа, когда под нее бы подложили
какую-то экономическую базу, какой-то экономический расчет. Но, как правило, у
нас этим сильно не озадачиваются. Надо сломать. Знаете, вот последние годы у
нас не строят, а ломают. Вот это было плохо, это был Совок, его надо поломать.

Но мало у кого есть представление, что же построить вместо этого. И начали
строить некую английскую якобы систему, которая в Англии из английского бюджета
потребляет 12\% или 13\%. В десятку бюджетов входит мировых и считается в
Европе наихудшей медициной.

И вот мы решили вдруг построить именно английский вариант с 3\% вместо 13\%,
разрушив при этом полностью и убрав, поломав всю санитарно-эпидемиологическую
службу, а это лаборатории, это кадры, это санстанции в каждом районе, области,
городе. И огромное количество людей, которые контролировали санитарное
состояние.

\ifcmt
pic https://strana.ua/img/forall/u/0/92/%D0%A2%D0%BE%D0%B4%D1%83%D1%80%D0%BE%D0%B2_%D0%BA%D1%80%D1%83%D0%BF%D0%BD%D1%8B%D0%B9_%D0%BF%D0%BB%D0%B0%D0%BD.png
caption Борис Тодуров. Кадр из фильма Олеси Медведевой 
fig_env wrapfigure
width 0.4
\fi
