% vim: keymap=russian-jcukenwin
%%beginhead 
 
%%file 24_07_2020.fb.lnr.4
%%parent 24_07_2020
 
%%endhead 
\subsection{Сезон захвата заложников на Украине продолжается. Кто такой
«полтавский террорист» Роман Скрипник}
\url{https://www.facebook.com/groups/LNRGUMO/permalink/2872270772884478/}

\index{Авторы! Павел Котов}

В четверг, 23 июля, спустя всего два дня после резонансной истории с захватом
автобуса в Луцке Волынской области, в Полтаве произошло новое похожее ЧП. На
этот раз в заложниках оказался начальник областного уголовного розыска
Нацполиции

Не успело украинское общество прийти в себя после событий на западе Украины,
где террорист Максим Кривош удерживал 13 человек до момента выполнения его
требований лично президентом Владимиром Зеленским, как на первых строчках
местных СМИ утвердился новый антигерой — бывший военнослужащий, неоднократно
судимый наркоман Роман Скрипник.  

\subsubsection{Неудавшийся угон}
\label{sec:24_07_2020.fb.lnr.4}

В этот день события на улице Остапа Вишни в Полтаве развивались стремительно.
Все началось с попытки банального угона машины, которая была вовремя пресечена
полицией. Однако относительно рядовое происшествие практически сразу же вышло
из-под контроля правоохранителей.

Оказалось, что преступник, которым, как позднее выяснилось, был Скрипник, имел
при себе гранату. Попытка задержания злоумышленника успеха не умела, вместо
этого он каким-то образом захватил в заложники находящегося вблизи него
сотрудника уголовного розыска и стал угрожать подорвать себя вместе с ним.
Полиция тут же отказалась от силового решения и вступила в переговоры с
бандитом. 

Результатом этого стало то, что Скрипнику предоставили по его требованию
служебный полицейский автомобиль, после чего он согласился обменять заложника —
место сотрудника угрозыска занял его непосредственный начальник, руководитель
уголовного розыска Полтавской области, полковник Виталий Шиян. Вдвоем они сели
в машину и отбыли в неизвестном направлении про полном бездействии силовиков.

\subsubsection{Странные действия полиции}

Как и в случае со спецоперацией в Луцке, действия полиции в Полтаве также
вызывают немало вопросов. До сих пор не понятно, как Скрипник, даже при наличии
у него гранаты, сумел захватить заложника из числа правоохранителей, удерживать
его все это время, произвести обмен, сесть в автомобиль и спокойно скрыться с
места происшествия.  Но на этом странности в этой истории не заканчиваются.
Вместо каких-либо действий по освобождению Шияна и нейтрализации преступника
силовики лишь пассивно сопровождали машину со Скрипником, которого в СМИ к тому
моменту уже окрестили «полтавским террористом».

По словам главы Полтавской облгосадминистрации Виктора Синегубова, нападавшему
дали скрыться ради того, чтобы он не устроил взрыв в черте города. Тем не
менее, даже за пределами Полтавы никакой спецоперации не произошло, хотя в
украинских Telegram-каналах сообщалось, что на уровне руководства МВД и СБУ
будто бы принято решение о жестком подавлении преступника, вплоть до его
ликвидации с помощью снайперов, которые якобы занимают позиции по пути
следования авто.

Кроме того, было обнародовано единственное требование террориста — сделать так,
чтобы он не оказался в тюрьме. Естественно, речи об этом идти не могло, о чем и
заявили власти.

В итоге Скрипник с заложником свободно перемещались по области в течение почти
трех часов, не встречая никаких преград на своем пути. В какой-то момент
появилась даже информация, что они уже успели въехать на территорию Киевской
области, а в сети иронично предложили полиции устроить злоумышленнику зеленый
коридор прямо в столицу.

Еще более необычно выглядит промежуточная развязка этой истории. После
блужданий по проселочным дорогам Скрипник принял решение отпустить своего
заложника и бросить машину, после чего террорист благополучно скрылся в лесу и
пустился в бега. Чем занимались и где находились в этот момент силовики и
обещанные снайперы, остается неизвестным. 

\subsubsection{Бандит из десанта}

Учитывая хронологическую близость событий в Полтаве к захвату заложников в
Луцке, аналогии между двумя дерзкими вылазками преступников были неизбежны.
Однако анализ аккаунтов Скрипника в соцсетях и обнародованные сведения о нем со
стороны МВД говорят о том, что ни о какой идейности, даже в той форме, которую
продемонстрировал двумя днями ранее Максим Кривош, говорить не приходится.

«Полтавский террорист» Роман Скрипник родился в 1988 году. Как рассказали
журналистам знавшие его с детства соседи, преступные наклонности у него стали
проявляться еще в детстве.

«Любит что-нибудь стащить. Не знаю, какой он, адекватный или нет, но еще когда
был в школьном возрасте, то убегал из дома. Несколько лет не было его здесь. А
потом появился», — сообщила одна из них.

Всю свою сознательную жизнь Скрипник провел в тесной связке с криминалом. В
2007 году его впервые осудили за мелкую кражу, он находился под подпиской о
невыезде и отделался небольшим штрафом. Спустя лишь год он снова попал под суд,
на этот раз за мошенничество, отправившись в итоге на два года в тюрьму. Однако
пребывание в местах не столь отдаленных не отбило у него охоту к преступлениям.
В 2014 году — новый срок за наркотики, приговор — ограничение свободы на
несколько лет. Помимо этого на его счету также множество административных
нарушений.

Спустя год в криминальной биографии Скрипника случается крутой поворот. Украина
к тому моменту уже во всю вела боевые действия против восставших народных
республика Донбасса, и молодого человека, несмотря на весь его сомнительный
послужной список, призывают в армию и направляют в 25-ю днепропетровскую
воздушно-десантную бригаду. Однако боевых подвигов он не сыскал и уже в 2016
году вышел в запас.

О том, что Скрипник умеет обращаться с оружием, также свидетельствуют
многочисленные снимки в соцсетях. На многих из них он позирует с автоматами и
пистолетами, нередко одновременно с алкоголем.

Служба в АТО также объясняет наличие при нем гранаты в момент захвата
заложника, ведь именно с Донбасса в больших количествах в последние годы
нелегально вывозятся боеприпасы и продаются на сторону, оседая на руках
населения.

\subsubsection{Что же случилось в Полтаве}

Учитывая всю известную о произошедшем информацию, а также личность самого
преступника, можно говорить о том, что события в Луцке и Полтаве нельзя считать
частью одного целого. Если Максим Кривош долго и тщательно готовил свой
преступный план и даже подвел под него некую теоретическую базу, то действия
Скрипника являются скорее импровизацией, продиктованной страхом за свою жизнь и
нежеланием оказаться в тюрьме.

Однако можно предположить, что уступки властей в Луцке могли породить у
некоторых мнение о том, что террористическими актами можно добиваться своих
целей. То есть тот самый «ящик Пандоры», о котором многие вспомнили после того,
как президент Зеленский выполнил требование луцкого захватчика. На этом фоне
еще больше возникает вопросов к украинским спецслужбам и правоохранительным
органам и к их готовности предупреждать и отражать атаки террористов. 

По своеобразной иронии судьбы, в разгар истории с «полтавским террористом», в
Луцке суд избирал меру пресечения для другого террориста — Кривоша, который
отправился в СИЗО на два месяца. 

В своем выступлении он пообещал, что объявленный им 21 июля «день антисистемы»
будет продолжен. Как показывает история со Скрипником, пока на Украине
по-прежнему называют откровенных бандитов из АТО патриотами и защитниками
страны, есть вероятность, что луцкий террорист окажется прав.

Павел Котов
  
