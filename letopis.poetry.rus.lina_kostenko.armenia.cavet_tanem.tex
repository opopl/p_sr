% vim: keymap=russian-jcukenwin
%%beginhead 
 
%%file poetry.rus.lina_kostenko.armenia.cavet_tanem
%%parent poetry.rus.lina_kostenko
 
%%url https://www.ukrlib.com.ua/books/printit.php?tid=5746
%%author 
%%tags 
%%title 
 
%%endhead 

\subsubsection{Цавет танем!}
\label{sec:poetry.rus.lina_kostenko.armenia.cavet_tanem}
\Purl{https://www.ukrlib.com.ua/books/printit.php?tid=5746}
\Pauthor{Костенко, Ліна}

\obeycr
\emph{Сільві Капутікян}

Згоріли їхні селища, пропали їхні мули.
Бредуть, бредуть вигнанці в дорогу неблизьку.
Щоб мову свою рідну їх діти не забули,
їм літери виводять вірменки на піску.

А вітер, вітер, вітер!..
Який палючий вітер!..
Обвуглені обличчя січе, січе, січе!..
Лиш виведеш те слово із тої в'язі літер,
а слово ж без коріння, покотиться, втече.

І десь, в якійсь пустелі, з'їдять його верблюди.
Забудуть його діти, і виростуть німі.
Бредуть, бредуть вигнанці...
бредуть бездомні люди.
Ні даху ж, ні притулку, --- буквар їм на умі!

Згоріли їхні храми. Мужчини їхні вбиті.
Втонули їхні дзвони у озері Севан.
О, як їм далі жити? На тім кровопролитті
не місяць в небі сходить --- турецький ятаган.

А вітер, вітер, вітер!..
Як шарпає той вітер!..
Куди їх ще, вигнанців, недоля заведе?
Нема коли писати отих маленьких літер.
Немає чим писати. Нема писати де.

І тільки на привалі, в ті рідкісні хвилини,
коли ще в свої тачки жінки не запряглись,
ті буковки вірменські виводять, як стеблини,
і слізьми поливають, аби лиш прийнялись.

В пісках пустили корінь --- а вітер, вітер, вітер!..
Бредуть, бредуть вигнанці в дорогу неблизьку!..
А скрізь по всій пустелі
тоненькі стебла літер,
як трави, проростають в палючому піску.

Їх топчуть ситі коні, дзвенять чужі стремена.
А букви проростають в легенди і пісні.
"Цавет танем!" --- як кажуть, прощаючись, вірмени.
Твій біль беру на себе. Печаль твоя в мені.
\restorecr

