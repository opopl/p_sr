% vim: keymap=russian-jcukenwin
%%beginhead 
 
%%file 03_12_2021.fb.kuzmin_renat.1.oshybka_prezident
%%parent 03_12_2021
 
%%url https://www.facebook.com/RRKuzmin/posts/4530817933661130
 
%%author_id kuzmin_renat
%%date 
 
%%tags oshybki,prezident,ukraina,zakon,zelenskii_vladimir
%%title В чём ошибка президента Зеленского?
 
%%endhead 
 
\subsection{В чём ошибка президента Зеленского?}
\label{sec:03_12_2021.fb.kuzmin_renat.1.oshybka_prezident}
 
\Purl{https://www.facebook.com/RRKuzmin/posts/4530817933661130}
\ifcmt
 author_begin
   author_id kuzmin_renat
 author_end
\fi

В чём ошибка президента Зеленского? 

Основной ошибкой президента Зеленского является то, что он начал осуществлять
свои президентские полномочия не по закону, а пошёл по пути беззакония и
произвола. 

Если бы президент Зеленский стал на сторону Закона и действовал исключительно
по закону, мы бы увидели совсем другого президента, совсем другую страну и
совсем другие перспективы. 

Сегодня мы все оказались заложниками президентских амбиций и непонимания
президентом своих полномочий, функций, обязанностей и роли Института Президента
в системе государственной власти. 

Президент — гарант Конституции, он обязан действовать по закону, обязан править
исключительно по закону. Именно соблюдение закона даёт ему право требовать
исполнение закона всеми гражданами Украины. Только соблюдая Конституцию и
законы, президент в праве требовать от нас соблюдения Конституции и законов.
Только уважая нас, граждан Украины, президент может требовать уважения к себе. 

К сожалению, у нас президент позволяет себе называть граждан своей страны
ОСОБЯМИ. А что такое особь? Это отдельно функционирующий организм, тело. То
есть он нас с вами, людей, которые не согласны с его политикой, считает не
людьми, а ОРГАНИЗМАМИ, ТЕЛАМИ. И при этом требует уважения к себе. 

Вспомните, как президент доказывал военным, что он «не лох». Это что за слово в
устах президента страны? Зековское, воровское, на преступном жаргоне означает
потерпевший или «терпила». Может президент употреблять такие выражения? 

А когда президент страны сообщает на пресс-конференции, что он в обход закона
принимает решения вместе со своим СНБО? Отвечая на вопрос, почему он в суд не
обращается, сообщает, что суды это долго, а ему нужно быстро, поэтому он сам
примет решение и ему этого хватит. Это лишь означает, что президент страны
отошёл от закона и Конституции и стал на путь произвола. Фактически преступный
путь, который сегодня проходит президент, привёл страну к беззаконию и
произволу. 

Теперь мы все видим, что президент позволяет себе нарушать закон в собственных
интересах — политических, экономических, личных. Но, если президенту можно, то
чего президент вдруг решил, что всем остальным нельзя? Потому что у него власти
больше, чем у всех остальных?

ВИДЕО - Ренат Кузьмин

\ii{03_12_2021.fb.kuzmin_renat.1.oshybka_prezident.cmt}
