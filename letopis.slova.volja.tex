% vim: keymap=russian-jcukenwin
%%beginhead 
 
%%file slova.volja
%%parent slova
 
%%url 
 
%%author 
%%author_id 
%%author_url 
 
%%tags 
%%title 
 
%%endhead 
\chapter{Воля}
\label{sec:slova.volja}

\emph{Воля}. Это слово состоит из двух понятий: стремления (\enquote{сила воли},
\enquote{\emph{воля} к победе}) и свободы (\enquote{вольный ветер}, \enquote{\emph{воля
вольная}}). В отличие от более социальной \enquote{свободы}, \enquote{воля} это
природная категория (можно сравнить выражения \enquote{свобода слова} и
\enquote{дать волю словам}). Свобода - это право, но оно ограничивается правами
других людей. \emph{Воля} же не имеет никаких ограничений и не связана с законом. О
том, кто покинул тюрьму по окончании срока, мы говорим \enquote{вышел на
свободу}, а о том, кто сбежал - \enquote{вырвался на \emph{волю}}. Русский менталитет
определённо предпочитает волю как ничем не сдерживаемую силу. \emph{Воля} - это
русские бескрайние просторы, «где гуляем лишь ветер... да я!»,
\citTitle{Русский языковой менталитет: 10 основных понятий}, Языковедьма, zen.yandex.ru, 08.06.2021

%%%cit
%%%cit_head
%%%cit_pic
%%%cit_text
Я думаю, Страна, в которой сегодня не удалось еще создать глупого
человека - робота, это Россия. Нам постоянно нужно поделать что-то для души,
потому что хочется. Находясь на работе, в большинстве своем мы думаем только об
одном, когда этот чертов день закончится и побыстрее уже свалить отсюда. Для
нас работа находится далеко не на первом месте. В первую очередь для нас важно
пожить, вечно поспорить, что-то обсудить, в целом чувствовать себя живым.
Россия, как и во все времена, сегодня полна яркими и живыми личностями, каждая
из которых имеет свой уникальный характер. Начиная от простого работяги и
вплоть до нерадивых чиновников. Вот это всегда и напрягало запад - русская
непокорность и \emph{Воля}. Иначе говоря - проявление характера.  Вот друзья, в чем
заключается по мнению брата главное отличие человека русского от человека
запада. И я с ним пожалуй соглашусь
%%%cit_comment
%%%cit_title
\citTitle{После 11 лет жизни в США понял самое важное на мой взгляд отличие между американцами и русскими}, 
За Пределами Онлайна, zen.yandex.ru, 06.06.2021
%%%endcit


%%%cit
%%%cit_head
%%%cit_pic
%%%cit_text
Австрийцы показали, как нужно бороться на этом Евро.  Сколько было разговоров
(надеюсь, несерьёзных) о «гениальном стратеге Шевченко», который все якобы
сделал для того, чтоб не выйти на «непроходных» итальянцев. Австрийцы показали,
как нужно играть с этой, без спора, неизменно великой сборной.  Они настоящие
герои матча. И если б им чуть больше везло, команда, впервые в своей истории
вышедшая из группы на серьезном первенстве, отправила бы итальяшек домой.
\emph{Железная воля} и готовность бороться до конца - вот что необходимо, чтоб за тебя
твою судьбу не решали другие
%%%cit_comment
%%%cit_title
\citTitle{Воля и решимость - вот что нужно, чтобы твою судьбу за тебя не решали другие / Лента соцсетей / Страна}, 
Олег Волошин, strana.ua, 27.06.2021
%%%endcit

%%%cit
%%%cit_head
%%%cit_pic
%%%cit_text
В жизнь социума пришла психология и психотерапия. Из жизни социума исчезли
слова «смелость» и \emph{«воля»}. Волевой компонент растренировывается у взрослых,
родители с отсутствующей \emph{волей} не формируют её у детей. Сказать «мы растим
ребёнка свободной личностью, поэтому ничего не запрещаем», несравнимо легче,
чем устанавливать ему границы и удерживать эти границы, выдерживая детский визг
и истерики. Дальше не имевший границ растекается лужицей, не умея себя собрать.
И как-то плавно и незаметно мы пришли к серьезной подмене. Люди ищут, как
сделать так, чтобы было не страшно выходить из привычной норы в незнакомое,
сбегают в изучение и разглядывание внутренних процессов и приходят в терапию
потому что им нужно решить, как принять решение
%%%cit_comment
%%%cit_title
\citTitle{Из жизни социума исчезли такие понятия, как смелость и воля / Лента соцсетей / Страна}, 
Светлана Комарова, strana.ua, 29.06.2021
%%%endcit


%%%cit
%%%cit_head
%%%cit_pic
%%%cit_text
Введу ще одну категорію – \emph{волю}. Я визначаю її в цьому контексті як \enquote{нічим не
обумовлену силу, яка формує обставини реальності людини}. Тобто це рушійна
сила, яка не є викликаною зовні, яка будує нашу реальність.  Так ось слідування
принципу є проявом \emph{волі}. І власне її тренуванням.  Оскільки аргументи, чи
слідувати прийнятому рішенню, чи ні, змінюються від контексту до контексту, то
якщо людина слідує принципу незалежно від контексту, це становиться проявом
необумовленості. А необумовленість є саме тим аспектом \emph{волі}, що формує
реальність.  Наприклад, людина вирішує для себе завжди переходити дорогу на
зелене світло світлофору. Ніби нічого складного. Але часто виникають ситуації,
коли ти поспішаєш, світло – червоне, машин немає. І тоді якщо ти обираєш таки
чекати зеленого світла, тобто слідувати принципу, в тобі прокидається щось нове
— якась необумовленість, яку ти сприймаєш як якусь в собі силу
%%%cit_comment
%%%cit_title
\citTitle{Чи можна примусити збірну України розмовляти державною мовою?}, 
Євген Лапін, www.pravda.com.ua, 06.07.2021
%%%endcit

%%%cit
%%%cit_head
%%%cit_pic
%%%cit_text
«Україна завжди прагнула бути \emph{вільною}». Ці слова Вольтера присвячені
Івану Мазепі – великому гетьману, який робив перші державотворчі кроки на шляху
до незалежності, проголошеної 24 серпня 1991 року.  Показово, що, попри три
століття часу, засади державного будівництва Івана Мазепи і до сьогодні
лишаються незмінно актуальними.  Щоправда, про них в незалежній Україні дуже
мало говорять і ще менше знають. Ба більше – у столиці, яка коштом і зусиллями
Мазепи та його найближчих соратників отримала дивовижної краси архітектурний
лик, донині немає пам’ятника великому будівничому Української держави
%%%cit_comment
%%%cit_title
\citTitle{Гетьман Іван Мазепа – будівничий незалежності України і ворог імперії}, 
Ірина Костенко; Ірина Халупа, www.radiosvoboda.org, 01.08.2021
%%%endcit

%%%cit
%%%cit_head
%%%cit_pic
%%%cit_text
— Те, що трапилося зі мною, — знову озвався юнак, — підтверджує слушність
древнього афоризму: «Людина сама кує своє щастя або нещастя». І молот, і
ковадло у кожного є. Це — наша \emph{воля}. Ми звикли пливти у течії узгодженого
суспільного життя. Ми добуваємо фах, їжу, будуємо житло. А коли щось зникає —
розгублені. Втрачаємо віру в себе. А міра всього — сама людина. Це знали ще
древні. Якщо є якийсь сенс у житті, то він повинен бути не в зовнішніх
можливостях, зручностях, успіхах, досягненнях. Основне — внутрішнє визначення
мети і \emph{воля} до дії. Ви скажете те, що спочатку думав і я: що може діяти каліка,
непорушний пень, прикутий до ліжка? Вся справа в тому, що я звик бачити дію
поза собою і забув (або не міг вбагнути), що в кожній істоті, в її глибинній
сутності приховано корінь будь-якої дії. Це наш розум, дух. І якщо ми
позбавлені однієї з можливостей дії, дух може знайти десять, сто інших. Це дуже
велично й масштабно розумів Каменяр. Пам’ятаєте? «Дух — вічний революціонер!»
Якщо він пробуджений, то хто його може втримати, зупинити, закувати,
приспати?..
%%%cit_comment
%%%cit_title
\citTitle{Вогнесміх}, Олесь Бердник
%%%endcit
