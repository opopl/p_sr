%%beginhead 
 
%%file 04_04_2022.fb.golovnjova_maryna.mariupol.1.08_03_2022
%%parent 04_04_2022
 
%%url https://www.facebook.com/m.holovnova/posts/pfbid0QBPWdhvt7DNbpuDGiEVvZ8VHPKkHAFT4omdy89VNEM7q53bMojFAXj8cp2iqrV31l
 
%%author_id golovnjova_maryna.mariupol
%%date 04_04_2022
 
%%tags mariupol,mariupol.war,dnevnik
%%title 08.03.2022
 
%%endhead 

\subsection{08.03.2022}
\label{sec:04_04_2022.fb.golovnjova_maryna.mariupol.1.08_03_2022}

\Purl{https://www.facebook.com/m.holovnova/posts/pfbid0QBPWdhvt7DNbpuDGiEVvZ8VHPKkHAFT4omdy89VNEM7q53bMojFAXj8cp2iqrV31l}
\ifcmt
 author_begin
   author_id golovnjova_maryna.mariupol
 author_end
\fi

08.03.2022

На 8 березня у Маріуполі вже не було не лише електрики, опалення, зв'язку,
представників місцевої влади (за виключеннями кількох справжніх Людей), але й
газу.

Ми спали в кімнаті без вікон і сподівалися, що дві стіни вбережуть нас від
снаряду. Точніше, ми майже не спали взагалі, а ледве дихали від страху, від
холоду, від того, як ударні хвилі трусили стіни і меблі, від гулу літаків над
містом. Намагалися  вгадати, чи розриви снарядів, які ми без перестану чуємо,
були від нас чи до нас. Ночі були найстрашнішими, і коли я нарешті бачила через
дверний отвір світло, то ставало трохи легше – ми дожили до ранку.

\ii{04_04_2022.fb.golovnjova_maryna.mariupol.1.08_03_2022.pic.1}

40 хвилин намагалися скип'ятити каструльку з водою на сирих дровах біля
під'їзду. На вулиці -8 і знов випав сніг. О 8 годині як завжди побігли до
драмтеатру, аби побачити друзів і дізнатися бодай якісь новини. Йти було досить
близько, але по дорозі кілька разів доводилось забігати у під'їзди – повітряна
тривога. Про сирени мова не йдеться, я взагалі вперше їх почула, діставшись
Запоріжжя 17го березня. У Маріуполі ми дізнавалися про загрозу авіанальоту,
коли безпосередньо чули сам бомбардувальник. Ховались і слухали, коли впаде
перша бомба, потім друга, адже бомбардувальник несе їх дві. Потім ще пару
хвилин – раптом бомбардувальник не один. Якщо було тихо, то в принципі можна
виходити і далі йти по своїх справах. Навіть діти знали цю схему.

На драмі ми годинами стояли на морозі і чекали хоч якоїсь інформації: що
взагалі відбувається в країні, у місті, чи буде сьогодні або завтра зелений
коридор. 

В обід приїхали військові, привезли дітям у Драмі памперси, ковдри і печиво –
що вдалося зібрати по місту на розкрадених мародерами складах. Роздали
інформаційні листівки з новинами. Як завжди бадьоро розказали про ситуацію на
передовій. \enquote{Ми їх скоро відіб'ємо, ми кожного дня робимо все, аби от-от уже
розкільцювати місто і ви всі могли виїхати}. 

Наш новий друг, який врятує за кілька днів нашу родину, приніс букет квітів! Це
ж 8 березня. Я тримала в руках ті квіти, а люди дивувалися, підходили і просили
просто понюхати їх. Не вірили, що вони справжні, що в цьому ніби забутому всіма
місті ще залишилось щось із нормального життя.

Пізніше на Драм приїхали й поліцейські. Якийсь чоловік розпитував у одного з
них, що йому робити з трупом сусідки. Вона вийшла у двір розігріти їжу, поруч
упав снаряд, жінку розірвало на шматки, він зібрав їх у пакет і відніс у гараж.
Я тоді подумала, що зараз і сюди може прилетіти снаряд, нас усіх так само
розірве на шматки і про це навіть ніхто ніколи не дізнається.

\ii{04_04_2022.fb.golovnjova_maryna.mariupol.1.08_03_2022.pic.2}

Новин про коридор у той день так і не було, і ми ні з чим пішли назад.
Попрощалися з друзями до завтра. Я щоразу думала, що ми так легко обіцяємо одне
одному побачитися завтра, але попереду ще півдня і ще ніч, і хтозна, куди
росіяни скинуть чергову бомбу, і хто до того завтра доживе. 

До темноти і комендантської години треба було встигнути щось приготувати їсти.
Уже пару годин тривало затишшя. Біля дитячого майданчику було кілька вогнищ, на
решітках з холодильників стояли сковорідки, чайники, каструльки. Хтось навіть
виніс пляшку шампанського і пластикові стаканчики. Це ж 8 березня. Шампанського
вистачило тільки жінкам, чоловіки посміхались і говорили приємні слова. 

Один з сусідів раптом сказав: \enquote{Ми тут з сином на утрєнік до 8 березня готували
вірша. Шкода, що пропадає!} Його малий тут же вліз на найвищий турнік посеред
двору, ми всі обступили його і притихли, а він голосно і виразно почав читати
вірш. Кожне слово відбивалося чіткою і дзвінкою луною від дев'ятиповерхівок
навколо. Вірш був українською мовою, про маму, про весну, про квіти. В
останньому рядку було щось про рідну красиву Україну. Усі стояли мовчки ще
кілька секунд, потім почали аплодувати. Малий не збився жодного разу і явно був
страшенно гордий собою. Хтось витягнув з кармана і дав йому дефіцитну шоколадну
цукерку. 

\enquote{Як добре, що вірш не пропав}, – сказав сусід, потріпав малого по плечу і
повернувся до свого вогнища.

%\ii{04_04_2022.fb.golovnjova_maryna.mariupol.1.08_03_2022.cmt}
