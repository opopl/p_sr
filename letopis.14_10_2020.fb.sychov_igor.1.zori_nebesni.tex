% vim: keymap=russian-jcukenwin
%%beginhead 
 
%%file 14_10_2020.fb.sychov_igor.1.zori_nebesni
%%parent 14_10_2020
 
%%url https://www.facebook.com/groups/167555693733355/permalink/921029278385989/
 
%%author 
%%author_id 
%%author_url 
 
%%tags poezia,ukraina,vojna
%%title 
 
%%endhead 

\subsection{Зорі небесні, попри прокльони}
\label{sec:14_10_2020.fb.sychov_igor.1.zori_nebesni}
\Purl{https://www.facebook.com/groups/167555693733355/permalink/921029278385989/}

\ifcmt
  author_begin
   author_id sychov_igor
  author_end
\fi

\ifcmt
pic https://scontent.fiev6-1.fna.fbcdn.net/v/t1.0-9/121194408_995927267592891_2588934569153106800_n.jpg?_nc_cat=101&ccb=2&_nc_sid=825194&_nc_ohc=qyr0bxfALssAX-gYOKS&_nc_ht=scontent.fiev6-1.fna&oh=60136cbfb260863f7cd8cd4ddcfe2649&oe=5FF2E60E
fig_env wrapfigure
caption Зорі небесні, попри прокльони, Попри війну, печаль та зневіру Ви передайте мої поклони Тій, що кохаю душею щирою
\fi

\obeycr
Зорі небесні, попри прокльони,
Попри війну, печаль та зневіру
Ви передайте мої поклони
Тій, що кохаю душею щирою.

На чергуванні, наприкінці ночі,
Доповідаю по рації пошепки,
Й згадую знову смарагдові очі
І таємничу лагідну посмішку.

Зіроньки, прошу, шлях їй освітлюйте
І нагадайте про зустрічі наші.
Серце її наповніть надією,
Як би їй зараз не було важко.

Зорі небесні, скажіть їй, що війни
Мають початок і мають закінчення.
Після випробувань будуть обійми -
Люди у цьому віками досвідченні.

Брама до неба — голкові вуха,
Шлях до зірок — завжди через терня...
Тільки б вона ніколи не слухала
Тих, хто не вірить у моє повернення...

Зорі небесні, порою ранньою
Попри лихої війни забаганки
Ви передайте моєму коханню:
Хай не сумує - близько світанок.

Автор - Ігор Сичов
\restorecr
