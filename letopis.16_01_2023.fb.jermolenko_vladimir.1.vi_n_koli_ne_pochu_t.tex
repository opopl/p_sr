%%beginhead 
 
%%file 16_01_2023.fb.jermolenko_vladimir.1.vi_n_koli_ne_pochu_t
%%parent 16_01_2023
 
%%url https://www.facebook.com/volodymyr.yermolenko/posts/pfbid0UEp4akkG9UFLktnuGppJi3nv39jzeWPKHCER2jy5gqgKgTc6Yqh9Y6igR4wEHUmJl
 
%%author_id jermolenko_vladimir
%%date 16_01_2023
 
%%tags gorod,pamjat,tragedia
%%title Ви ніколи не почуєте цього. Як це - бути містом, у якого вкрали мир
 
%%endhead 

\subsection{Ви ніколи не почуєте цього. Як це - бути містом, у якого вкрали мир}
\label{sec:16_01_2023.fb.jermolenko_vladimir.1.vi_n_koli_ne_pochu_t}

\Purl{https://www.facebook.com/volodymyr.yermolenko/posts/pfbid0UEp4akkG9UFLktnuGppJi3nv39jzeWPKHCER2jy5gqgKgTc6Yqh9Y6igR4wEHUmJl}
\ifcmt
 author_begin
   author_id jermolenko_vladimir
 author_end
\fi

***

Ви ніколи не почуєте цього. Як це - бути містом, у якого вкрали мир. Відкривати
свої темні підземелля убивцям і катам. Впускати їх у себе, як впускають у тіло
отруту. А потім носити тіла своїх людей у собі, як ніколи не народжений плід.

Ви не почуєте цього болю, бо ви ще не знаєте мови міст. Мови їхніх протяжних
сирен. Мови свисту дронів над головами їхніх будівель. Мови темної тиші, де все
перекреслюється, і лишається тільки звук і дотик. Мови криків у квартирі
поверхом нижче, яка стала катівнею. Мови тремтіння стін, падіння піщинок пилу
зі стель, коли останні ноти ударної хвилі трусять скло твоїх вікон. Мови
будинків, вивернутих навиворіт, розрізаних навпіл ворожими ракетами і людською
назавжди тишею.

Міста навчилися говорити, чи може кричати, - тишею, темрявою, зламаними
стінами, уламками часу, якого вже не буде. Там, у своїх глибинах, серед хащі
коріння дерев, вони зберігають історії, що їх ніхто окрім них не розповість.
Вони дають їм останній притулок, гріють чай для тих, хто все ще має їм’я, але
вже не має життя. Там, у підвалах і підземних сховках, у лабіринтах зруйнованих
коридорів, під руїнами колосів-будівель, сховалося місце вічного болю,
незашитої рани, незагоєної безодні. Міста знають їхню таємну мову. Вони
назавжди бережуть острови, де люди віддали тепло своїх тіл.

Міста, німі, мовчазні, потоплені у темряву, навчилися мови, своєї, тихої,
темної, вогкої, невблаганної.

Вони все збережуть. Вони нікому не пробачать
%\ii{16_01_2023.fb.jermolenko_vladimir.1.vi_n_koli_ne_pochu_t.cmt}
