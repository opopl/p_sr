% vim: keymap=russian-jcukenwin
%%beginhead 
 
%%file 07_12_2020.fb.zharkih_denis.1.skazki_ischenko
%%parent 07_12_2020
 
%%url https://www.facebook.com/permalink.php?story_fbid=2874341966112535&id=100006102787780
 
%%author Жарких, Денис
%%author_id zharkih_denis
%%author_url 
 
%%tags politics,ukraine,russia
%%title Сказки Ищенко
 
%%endhead 
 
\subsection{Сказки Ищенко}
\label{sec:07_12_2020.fb.zharkih_denis.1.skazki_ischenko}
\Purl{https://www.facebook.com/permalink.php?story_fbid=2874341966112535&id=100006102787780}
\ifcmt
	author_begin
   author_id zharkih_denis
	author_end
\fi

\index[names.rus]{Погребинский, Михаил}

Не лез в комментарии по поводу высказывания Михаила Погребинского по УПА, хотя
было, что сказать, тем более, что он потом уточнил, о чем говорил, чем,
конечно, не исчерпал тему, но все-таки ее как-то пригладил. Но меня просто
переполняют эмоции по поводу опуса Ростислава Ищенко относительно
Погребинского, Лукашенко и Медведчука. Свой вывод скажу в конце, а пока
несколько соображений. 

Выступать против национализма с позиций крайнего национализма такой же бред,
как фашизмом лечить фашизм.

Ищенко обличает Лукашенко, Погребинского и Медведчука в умеренном национализме,
который хоть и умеренный, но повинен в смерти многих людей, хитрый и продажный.
Пока выносим за скобки реальные позиции всех трех, давайте проанализируем с
каких позиций выступает сам Ищенко. А выступает он с радикальных позиций
русского/российского национализма: Мы поглотим кого угодно, а нас вам не
поглотить;  мы народ, мы сила, мы правда, а вам еще надо доказать свою
причастность к народу и право на жизнь, кайтесь, грешники, ибо мы святые; вам
нужны только деньги, которым вы служите, вы против народа, который мы
представляем, этот народ по глупости вас кормит, а должен был удавить, и т.д. и
т.п. 

Так вот, все это я каждый день слышу от украинских националистов. Они, как
Ищенко, ставят вопрос ребром - если ты мне чем-то не нравишься, то ты за
Россию, предал Украину, народ и жить не достоин. Ищенко пребывает в похожей
истерике, не знаю, как в России такое, но у нас подобное по сто раз на дню,
привыкли уже. 

Украинские националисты считают Россию вечной угрозой Украине, отчего и
находятся в перманентной истерике. Ищенко зеркально считает независимую
Белоруссию и Украину угрозой интересам России, и отрицает всяческий
государственный и национальный интерес этих стран. При этом интеграция Украины
и Белоруссии в Россию должны происходить на российских условиях и без учета
каких-либо интересов этих двух стран. 

Ищенко много употребляет слов типа \enquote{не знают, не понимают}, а понимает
ли сам Ищенко, что интеграция без условий для одной стороны называется
оккупацией. Так вот, он является сторонником оккупации Украины и Белоруссии, о
чем орет во весь голос. Если бы я думал, как Ищенко, то сказал бы, что это и
есть позиция Кремля - оккупация всех пограничных территорий. Но я понимаю, что
позиция Ищенко одно, а позиция Кремля другое, а Ищенко просто подставляет
Кремль на международной арене, о чем \enquote{не знает, не понимает, не
догадывается}. 

Вишенкой на торте является вопрос Ищенко \enquote{Чей Крым?}, как маркер \enquote{свои-чужие}.
Тут явно плагиат у украинских националистов, они первые начали использовать
этот маркер. Ищенко явно играет по их правилам. 

\subsubsection{Позиция Лукашенко, Медведчука и Погребенского - разные позиции}

У меня как-то хватает ума не считать позицию Ищенко позицией Путина,
Жириновского и Трампа. Почему Ищенко, который, вроде как, специалист по
Украине, не видит разницы между позициями Погребинского и Медведчука, не говоря
уже о Лукашенко.  

Лукашенко досталось от Ищенко только за то, что он защищал, как понимал,
интересы своей страны. Так должность у него такая, эти интересы защищать. Или
ему нужно защищать интересы России, а о своей стране и народе не заботиться? Ну
так и украинские националисты в упор не желают видеть интересы чужой страны, а
только своей. В этом Ищенко форменный бандеровец. 

С Погребинским и Медведчуком немного сложнее, они граждане одной страны, но
вовсе не сиамские близнецы, как считает Ищенко. И говорят далеко не
равнозначные вещи. Погребинский известен своими либеральными взглядами, что,
собственно, и породило его высказывание про УПА, к которому мы вернемся позже. 

У Медведчука относительно этого совершенно другая, даже противоположная
формула. Он не предлагает мириться с бандеризацией страны, он предлагает
вынести за скобки все спорные моменты (Бандеру в первую очередь) и
договариваться о том, о чем можно договориться. Очень разумная и прагматичная
позиция. 

Вот есть у меня друг Русла Коцаба, которого я считаю героем современной
Украины. Вот никогда с ним о Бандере не говорили. Просто не было времени. Мы
говорим о том, как нам потушить пожар ненависти и войны, и у нас это
получается, мы понимаем друг друга. Он не называет меня оккупантом, я его
бандеровцем. При этом и я и он, в отличие от Ищенко, находимся под постоянным
психологическим давлением и опасностью, как пишут мне каждый день, \enquote{повторить
подвиг Бузины}.  Может так случиться, что о Бандере мы с ним никогда и не
поговорим. Но нам было, есть и будет, что обсудить и без этого. У нас есть
общие цели, а не ставить друг друга на колени и заставить каяться. И если мы
считаем, что нам не в чем каяться перед радикальными украинскими
националистами, то, простите, нам не в чем каяться и перед российскими
националистами и лично Ищенко. 

\subsubsection{В чем правота Погребинского}

Еще раз подчеркну, что Медведчук в публичной плоскости выступает значительно
резче Погребинского, он резко осуждает русофобию, подмену истории, ущемление
прав русскоязычных граждан. И это он делает не в Москве, как героический
Ищенко, а в Украине, куда я Ищенко приглашаю выступить открыто со своими
соображениями. Не обязательно на Западе, можно в Киеве, Одессе (Крым не
предлагать). 

И Медведчук заставил свою позицию уважать, сегодня украинская политика без
Медведчука невозможна (не знаю возможна ли российская политика без Ищенко).
Погребинский в отличие от Медведчука и не позиционирует себя политиком. 

Так вот о Погребинском. Я могу придраться к его высказыванию, как многие, но у
меня на это нет времени. Достаточно будет и того, что скажу - не совсем с ним
согласен. Но опять же применим формулу Медвелчука - ищем не где не согласен, а
где согласен. Так вот, я согласен с Погребинским в том, российское телевиденье,
некоторые политики (и некоторый Ищенко) ведут жесткую антиукраинскую риторику с
позиций великорусского национализма. Эта риторика потом используется не только
против оппозиционных политиков в Украине, но и против украинской политики в
России. 

Часто об Украине говорят с позиции старшего брата, который рано или поздно
поглотит Украину. И тут же великодержавные слезы, как дорого это обойдется
русскому народу. 

Нет, господа хорошие, вы или трусы оденьте, или крестик снимите. Если вы (лично
вы, а не Кремль) собрались поглощать, оккупировать, русифицировать, то скажите
сколько и чем вы собираетесь за это платить. Это ведь не только восстановление
экономики (как там экономика ЛДНР, Приднестровья, Абхазии, Северной Осетии?),
но и прямые столкновения с вооруженными украинцами, которых Запад с готовностью
вооружит. Платить придется не только деньгами, но жизнями российских солдат.
Готовы? Такие, как Ищенко, воевать не пойдут, это мы знаем по своим идеологам. 

А если поглощение Украины отменяется, то тогда зачем нагнетать обстановку
криками, что Украина это неправильная Россия. У нас тут полно всяких, которые
кричат, что Россия, это неправильная Украина (больше ничего эти субъекты делать
не умеют давно). Крики из России о том, что Украину надо разбить, расчленить,
уничтожить никак не помогают тем политикам и смелым людям, которые имеют
мужество говорить, что с Россией нужно восстанавливать отношения, что пора
прекращать политику ненависти и взаимных обвинений, что пора прекращать убивать
людей, а потом со смаком вешать трупы на политических оппонентов.  

И тут Погребинский прав (я не про УПА, это отдельная тема) необходимо снижать
градус накала, нужна разрядка. А Ищенко разрядка не нужна. Я знаю многих из
украинской диаспоры в Москве и российских граждан, озабоченных украинской
темой, которые спят и видят, как въедут в Украину на российских танках и будут
тут наводить порядки, учить жить и т.д. Мне такая позиция понятна, у нас полно
городских сумасшедших, которые мечтают на американских танках въехать в Москву
и учить россиян жизни и европейским стандартам. Их у нас даже государство
поддерживает (а как с этим у Ищенко?). 

Что мне сказать этим господам? А приезжайте сюда без танков, и посмотрим на что
вы способны. Вот живут тут без российских танков Виктор Медведчук, Юрий Бойко,
Ренат Кузьмин, Михаил Погребинский, Руслан Коцаба, Инна Иваночко, Владимир
Быстряков, Любовь Титаренко, Саша Лазарев да всех не перечислишь. И их тут за
это уважают. А кидаться говном с московского дивана в людей, которые доказали
свою порядочность на деле, много ума не надо. Нехорошо это, гадко.
