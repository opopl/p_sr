% vim: keymap=russian-jcukenwin
%%beginhead 
 
%%file poetry.rus.anastasia_lagan_1954.kiev_podija_taka
%%parent poetry.rus.anastasia_lagan_1954
 
%%url http://maysterni.com/publication.php?id=28688
%%author 
%%tags 
%%title 
 
%%endhead 

\subsubsection{А була в Києві подія така}
\label{sec:poetry.rus.anastasia_lagan_1954.kiev_podija_taka}
\Purl{http://maysterni.com/publication.php?id=28688}
\index{Поэзия!Киев!Майдан}

Була подія
В Києві така,
Що називалася
Помаранчева революція.

Київська молодь
Вийшла на Майдан
І сказала своє слово:
“Влади не віддам”.

Допоможіть нам,
Сестри та брати,
Якщо до долі України
Не байдужі ви.

Цей заклик лунав
До всіх кутків...
І першим відізвалось
Княже місто Львів.

Не хвилюйтеся кияни
Підмога буде.
З усіх кутків України
Народ в столицю йде.

Перепони на дорогах
Повсюду стоять,
А вони з країн сусідніх
Літаком летять.

Заклики лунали
Навкруги такі:
“Якщо ви українці”,
Будьте мужніми.

“Розумні й терпеливі”, –
Лунали такі слова,
Щоб не почалась
Громадянська війна.

А стояли на Майдані
Наші діти, сестри і брати.
Хоч було люто-холодно,
Але не відступали вони.

Кияни, кияни!
Велика вам хвала.
Історія запам’ятає
Ваші славні імена.

Допомагали Майдану
Чим могли.
Навіть, картоплю гарячу
Приносили Ви.

Українська молодь
Інтелегентна така.
Показала всьому світу
Яка згуртована вона.

Одного разу італійці
Запитали мене:
“Хто керує “помаранчевими”,
Що така дисципліна є?”

Той, кого хочуть за Президента
Українці молоді
І тільки його накази
Виконують вони.

Дякуємо Богу
Щиро завжди,
Що без пролиття крові
До перемоги прийшли.

Просимо мудрості
У Бога ми,
Щоб правильно керували
Державні керманичі.

Всемилостивий Боже,
Своєї ласки нам подай.
Щоб економічно розквітав
Наш український край. 
