% vim: keymap=russian-jcukenwin
%%beginhead 
 
%%file slova.tradicia
%%parent slova
 
%%url 
 
%%author 
%%author_id 
%%author_url 
 
%%tags 
%%title 
 
%%endhead 
\chapter{Традиция}
\label{sec:slova.tradicia}

%%%cit
%%%cit_head
%%%cit_pic
%%%cit_text
Світлана Ославська ─ письменниця та журналістка. Анна Ільченко — фотографка та
відеографка. Вони об’єднали зусилля, щоб зберегти пам’ять про \emph{традиції} і
архітектуру старих українських осель.  Майже за рік авторки проєкту Old khata
projekt провели близько десяти експедицій: Карпати, Волинське Полісся,
Чернігівське Полісся, Наддніпрянщина, Донбас.  Починалося все зі сторінки в
Instagram, де Світлана та Анна збирали фотографії хат з різних регіонів. Та
згодом замислилися над більш глобальним проєктом, який зрештою має стати
фотокнигою. Любов до старих хат з’явилася у них ще в дитинстві
%%%cit_comment
%%%cit_title
\citTitle{Шукайте хату. Як авторки Old khata projekt подорожують Україною і фіксують час}, 
Влад Яценко, life.pravda.com.ua, 17.06.2021
%%%endcit

%%%cit
%%%cit_head
%%%cit_pic
\ifcmt
  pic https://strana.news/img/forall/u/0/36/617a8bb5e33c0.jpg
  @width 0.4
\fi
%%%cit_text
Если раньше украинские власти просто обходили тот момент, что решающую роль в
победе над нацизмом сыграла Красная армия (так сегодня снова сделал и
Зеленский), то сегодня пошел новый тренд: заявлять, что все победы - это
заслуга антигитлеровской коалиции. В которую, кроме СССР, входили Британия и
США.  По крайней мере, такую \emph{традицию} заложил сегодня премьер Денис Шмыгаль,
который обычно вообще отмалчивается на такие темы - начисто игнорируя, к
примеру, День победы 9 мая.  Но такая концепция, конечно, полностью
антиисторическая. Украину 76 лет назад окончательно освободили в ходе
Восточно-Карпатской операции советских войск. Да и весь период наступательных
боев за Украину - с января 1943 года по октябрь 1944 года - осуществлялся
исключительно силами Красной армии.  Всего она провела 15 наступательных
операций, включая самую крупную - битву за Днепр. Карпатская операция стала
последней: она началась 9 сентября, а уже 27 октября был освобожден Ужгород - и
на следующий день советские войска вышли на нынешние западные границы УССР
%%%cit_comment
%%%cit_title
\citTitle{Высадка союзников в Ужгороде. Кого Зеленский и Шмыгаль поблагодарили за освобождение Украины от нацистов}, 
Анна Копытько, strana.news, 28.10.2021
%%%endcit

%%%cit
%%%cit_head
%%%cit_pic
%%%cit_text
Смотрим. Многие ценности, к которым мы так привыкли, - исчезают, рассеиваются и
гаснут, как свет люстры в театре.  Например, исчезает нормальная экономика..,
привычная мораль.., формальная логика.., уважаемая элита.., умное
образование.., человеческая совесть.., аналоговое мышление.., честные законы..,
любимые \emph{традиции}.., здоровое питание.., чистая вода.., выверенная
история.., институциональная семья.., поведенческая адекватность.., глубокий
сон.., понятная правда.., прогрессивная наука.., географические границы..,
интимные тайны.., исчезают...  И ещё 100 таких же явлений.  Этот
футуристический проект «100 дней до конца света», в котором я рассматривал 100
самых важных явлений, среди которых мы живем, несмотря на свою диковинность,
по-прежнему не теряет актуальности.  А теперь хорошие новости. По закону
сохранения энергии, всегда после любого Конца Света наступает Начало Света. И
мы с вами уже скоро явимся свидетелями нарождения чего-то приятного и нового.
И света в наступающей эпохе для нас хватит.  Только Свет будет иной
%%%cit_comment
%%%cit_title
\citTitle{Конец света наступает постепенно, обволакивая нас / Лента соцсетей / Страна}, 
Владимир Спиваковский, strana.news, 22.12.2021
%%%cit_url
\href{https://strana.news/opinions/368254-konets-sveta-nastupaet-postepenno-obvolakivaja-vsekh-nas.html}{link}
%%%endcit
