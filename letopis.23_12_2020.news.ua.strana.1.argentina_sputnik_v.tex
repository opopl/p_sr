% vim: keymap=russian-jcukenwin
%%beginhead 
 
%%file 23_12_2020.news.ua.strana.1.argentina_sputnik_v
%%parent 23_12_2020
 
%%url https://strana.ua/news/308423-kakuju-vaktsinu-ot-koronavirusa-odobrili-v-arhentine-.html
 
%%author 
%%author_id 
%%author_url 
 
%%tags covid_vaccine,argentina,sputnik_v
%%title В Аргентине одобрили использование российской вакцины от коронавируса "Спутник V"
 
%%endhead 
 
\subsection{В Аргентине одобрили использование российской вакцины от коронавируса \enquote{Спутник V}}
\label{sec:23_12_2020.news.ua.strana.1.argentina_sputnik_v}
\Purl{https://strana.ua/news/308423-kakuju-vaktsinu-ot-koronavirusa-odobrili-v-arhentine-.html}
\index[rus]{Вакцина!Спутник V!Аргентина, 23.12.2020}


\ifcmt
  pic https://strana.ua/img/article/3084/kakuju-vaktsinu-ot-23_main.jpeg
	caption В Аргентине одобрили экстренное применение российской вакцины. Фото: freepik 
\fi

В среду, 23 декабря, в Аргентине Министерство здравоохранения в экстренном
порядке одобрило применение российской вакцины от коронавируса "Спутник V".\Furl{https://strana.ua/news/307886-rossijskaja-vaktsina-sputnik-v-effektivna-protiv-mutirovavsheho-koronavirusa.html}

Об этом сообщает\Furl{https://tn.com.ar/salud/2020/12/23/coronavirus-en-la-argentina-el-ministerio-de-salud-autorizo-el-uso-de-emergencia-de-la-vacuna-rusa/} телеканал Todo Noticias.

Данное решение было принято после получения рекомендации от Национальной
администрации медицинских технологий лекарственных средств.

"Правительство готовится запустить план вакцинации, в рамках которого на первом
этапе вакцина будет применена к 300 тысячам человек. Затем в январе они получат
вторую дозу", - добавили в материале.

В настоящее время самолет, который вылетел, чтобы доставить первые дозы
российской вакцины в Аргентину, находится в Москве. Ожидается, что он прибудет
в Буэнос-Айрес в четверг, в 10:00-10:30.

Также "Страна" сообщала, что президент Аргентины первым в стране привьется
российской вакциной от Covid-19.\Furl{https://strana.ua/news/305917-prezident-arhentiny-pervym-v-strane-privetsja-ot-covid-19-rossijskoj-vaktsinoj.html} Он подчеркнул, что не сомневается в ее
качестве.

Напомним, что разработчики российской вакцины "Спутник V" объявили об итогах
третьего этапа испытаний.\Furl{https://strana.ua/news/306588-itohovye-ispytanija-pokazali-effektivnost-rossijskoj-vaktsiny-sputnik-v-bolee-90.html} Итоговый анализ показал эффективность вакцины
"Спутник V" выше 90\%, а против тяжелых случаев 100\%.
