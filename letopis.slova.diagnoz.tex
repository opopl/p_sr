% vim: keymap=russian-jcukenwin
%%beginhead 
 
%%file slova.diagnoz
%%parent slova
 
%%url 
 
%%author 
%%author_id 
%%author_url 
 
%%tags 
%%title 
 
%%endhead 
\chapter{Диагноз}
\label{sec:slova.diagnoz}

%%%cit
%%%cit_head
%%%cit_pic
%%%cit_text
Увы, но у меня нет однозначного \emph{диагноза} нашей общественной хвори
(импотенции?), позволяющего немедленно начать эффективное лечение. Могу только
отметить, что причина — не в зацикленности населения на тривиальном выживании,
ведь экономические трудности как раз и порождают в других странах истинно
революционные процессы по смене моделей власти. И дело не в манипулируемости
наших граждан, ведь манипулировать ими возможно и в правильную сторону, но
этого ведь не происходит.  Есть рассуждизмы, что наша неспособность выстроить
успешное общество, в котором комфортно и справедливо себя чувствуют как рядовые
граждане, так и правители, как предприниматели, так и пенсионеры и прочие
социальные группы — это вроде родового геополитического проклятия. Но поверить
в это я пока не готов. Хотя уже близок
%%%cit_comment
%%%cit_title
\citTitle{В Украине была только одна настоящая революция - в 1991-м}, 
Александр Кочетков, strana.ua, 16.07.2021
%%%endcit

