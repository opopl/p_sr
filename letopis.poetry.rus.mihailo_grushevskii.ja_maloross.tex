% vim: keymap=russian-jcukenwin
%%beginhead 
 
%%file poetry.rus.mihailo_grushevskii.ja_maloross
%%parent poetry.rus.mihailo_grushevskii
 
%%url https://www.m-hrushevsky.name/ru/Fiction/Poetry/JaMaloross.html
%%author 
%%tags 
%%title 
 
%%endhead 

\subsubsection{Я малоросс}
\Purl{https://www.m-hrushevsky.name/ru/Fiction/Poetry/JaMaloross.html}

Я малоросс, моя страна –
Гнездо борцов за честь и волю,
Лежит теперь порабщена,
Горюя, плачась на недолю.
В презреньи милый наш язык,
Молчать мы уж давно привыкли,
Шутить над нами всяк привык
И мы с насмешками уж свыклись.
Уж много слабых сыновей
Тебя забыли, Украина,
Те – аргусы земли ж своей,
Те руку лижут господина,
Одни молчат в неволе этой,
Те отреклися от тебя,
Чем заслужила эту кару
Отчизна бедная моя?
Не тем ли, что в годину скорби
Детей на битву ты слала,
Что средь печалей, разорений
Ты твердо знамя все ж несла,
Боролась ты за честь и волю,
Несла знаменья ты креста.
За что ж тебя теперь постигли,
Неволя, рабство, нищета?
За то ль, что кровью обливаясь
И волю уж отбив себе,
Москве дала на сохраненье,
На радость гибельной судьбе
Соединясь с Москвой коварной,
Хранила честно договор,
Лья кровь из-за неблагодарной,
Как льешь ее и до сих пор.
И за покорность, за смиренье
Ты оскорбленья лишь несла
И чашу горя, унижений
По каплям выпила до дна.
Терпи, Украйна, уж недолго
Тебе осталося терпеть.
Уж скоро, скоро крик призывный
Повсюду должен прогреметь.
Настанет час, и крик призывный
Вдруг прогремит во все края,
В союз мы станем неразрывный
За тебя, родина моя.
Перед порывом силы честной
Москва недолго устоит,
Падет она, Украйна встанет
И вновь свобода зацарит.
Волною гордой в край из края
Промчится нашей речи звук,
Подымем мы свободы знамя
И уж не выпустим из рук!
2.ХІ.1883 
