% vim: keymap=russian-jcukenwin
%%beginhead 
 
%%file slova.cennosti
%%parent slova
 
%%url 
 
%%author 
%%author_id 
%%author_url 
 
%%tags 
%%title 
 
%%endhead 
\chapter{Ценности}

Что делать с \emph{ценностями}? Смотрим. ...  Как же их объединить в одной душе
и в одном мозге?  Печаль заключается в том, что обретая \emph{европейские
ценности}, придется чуть-чуть пожертвовать своими, \emph{национальными}, так
как они не всегда совпадают.  Если руководствоваться лишь своими \emph{личными
ценностями}, то можно войти в клинч с общечеловеческими.  Отсюда вопрос: в
какой пропорции должны быть эти \emph{ценности}? (А-Б-С-Д) Одни пишут так
(10-20-30-40), другие наоборот 40-30-20-10. Третьи – все поровну, по 25\%.  А у
Вас какая \emph{пропорция ценностей}?,
\textbf{Где баланс между европейскими, национальными и личными ценностями?},
Владимир Спиваковский, strana.ua, 06.05.2021

%%%cit
%%%cit_head
%%%cit_pic
%%%cit_text
«Сегодня мы живём в мире, где зачастую под светлыми идеями скрываются тёмные
намерения извратить базовые человеческие потребности и \emph{ценности}. Чуждая
традиционной украинской культуре западная идеология под предлогом борьбы с
гендерным неравенством и любой дискриминацией навязывает нам приятие
нетрадиционной сексуальной ориентации, однополых браков и тотальной половой
распущенности. Мы подробно и понятно будем объяснять, как это делается и как от
этого защитить себя и близких, но главное — мы дадим информационную
альтернативу своими образовательными и творческими инициативами», — говорится в
публикации на странице «Миряне» в Facebook
%%%cit_comment
%%%cit_title
\citTitle{«Невидимое большинство». Православных на Украине призывают проснуться}, 
Андрей Лубенский, ukraina.ru, 09.06.2021
%%%endcit

