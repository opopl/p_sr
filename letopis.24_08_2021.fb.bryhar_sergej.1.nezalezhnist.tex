% vim: keymap=russian-jcukenwin
%%beginhead 
 
%%file 24_08_2021.fb.bryhar_sergej.1.nezalezhnist
%%parent 24_08_2021
 
%%url https://www.facebook.com/serhiibryhar/posts/1771585723041372
 
%%author Бригар, Сергей
%%author_id bryhar_sergej
%%author_url 
 
%%tags nezalezhnist,ukraina
%%title Сьогодні 30-та річниця Відновлення Незалежності України!
 
%%endhead 
 
\subsection{Сьогодні 30-та річниця Відновлення Незалежності України!}
\label{sec:24_08_2021.fb.bryhar_sergej.1.nezalezhnist}
 
\Purl{https://www.facebook.com/serhiibryhar/posts/1771585723041372}
\ifcmt
 author_begin
   author_id bryhar_sergej
 author_end
\fi

Скільки себе пам'ятаю, чую: "какая "нєзалєжность", от кого?" Ну і далі по
тексту, щось на зразок: "...да ми жилі пріпєваючі, а тєпєрь -  бардак..." От
тільки раніше це лунало повсюдно, а тепер - усе рідше. І кількість носіїв
відповідного світогляду природним чином зменшується, і навіть вороги чудово
розуміють "що, чому і від кого" ми маємо. 

\ifcmt
  pic https://scontent-cdg2-1.xx.fbcdn.net/v/t39.30808-6/240399952_1771585699708041_340610404300162533_n.jpg?_nc_cat=100&_nc_rgb565=1&ccb=1-5&_nc_sid=8bfeb9&_nc_ohc=dc8E5ykc4b4AX_2b9r0&_nc_ht=scontent-cdg2-1.xx&oh=8311f60f3495980791cd37de5f687ba2&oe=612B10E6
  width 0.4
\fi

Ще якихось років 20, та навіть 15, тому через українську мову в моєму місті
можна було нарватися на агресію, і навіть отримати тілесні ушкодження. Тепер це
вже малоймовірно...

Для когось Незалежність - цінність, для когось - просто даність, або й даність
неприємна, але майже ніхто вже її публічно не висміює, не знецінює, не
заперечує. І це очевидний прогрес.

Сьогодні 30-та річниця Відновлення Незалежності України! Саме відновлення!
Продовження! 30 років тому постала держава, що є спадкоємицею багатовікових
традицій, починаючи від Київської Русі, й закінчуючи УНР, Карпатською Україною.

Українці - давня нація, що є невід'ємною частиною європейської цивілізації.
Культурно, ментально, світоглядно ми ніколи не були, не є, і не станемо
частиною московської орди. Отже, обов'язково відіб'ємося від цього агресивного
утворення і збережемо свою Незалежність!

Війна триває, і для того, щоб бути солдатами українського світу, не обов'язково
йти на фронт. Війна всюди. Лінії розривів і проваль проходять по наших
населених пунктах, вулицях, дворах, а часом і оселях... Як на мене,
найваєливіше, що можна зробити для України - виховати україномовну, заглиблену
в національний контекст, свідому дитину, тобто - справжнього громадянина своєї
держави! Але варіантів бути корисним, звісно, багато: писати книги, знімати
фільми, організовувати фестивалі, допомагати військовим чи активістам, вести
просвітницьку чи агітаційно-мотиваційну роботу, підтримувати гривнею сучасну
вітчизняну культуру, і зрештою просто розмовляти українською - спокійно,
впевнено, голосно...

Головне в нашій ситуації - продовжувати розбудовувати Українську Україну! Не
зупиняючись. Це критично важливо. Саме цей шлях робить нас впевненішими і
сильнішими, адже відчуття додає відчуття гідності, окремішності, виключної
важливості для світу. Саме проти Української - україномовної, україноцентричної
- України ворог безсилий.

Вітаємо зі святом, пані та панове!

Бажаємо сил, впевненості, малих і великих перемог!

Україна - це чудова земля, багатовікова історія, і ми з вами. Наші діти - її
майбутнє. Отже, бережімо себе, своїх рідних та близьких і, радіймо любові та
братерству і, не зважаючи на масу несприятливих факторів, йдемо вперед!
