% vim: keymap=russian-jcukenwin
%%beginhead 
 
%%file 20_04_2021.fb.zharkih_ekaterina.1.ruslan_bortnik_beseda
%%parent 20_04_2021
 
%%url https://www.facebook.com/watch/?v=465479734779765
 
%%author 
%%author_id 
%%author_url 
 
%%tags 
%%title 
 
%%endhead 

\subsection{Психология урезанной Украины, победа села над городом и готовность
воевать. С Русланом Бортником о состоянии украинского общества}
\Purl{https://www.facebook.com/watch/?v=465479734779765}

Парадоксальное двоемыслие и глубочайшая социальная шизофрения — одна из самых
выразительных черт, которая описывает состояние и психологию украинского
общества, полагает политолог Руслан Бортник.

Мы хотим жёсткой руки и максимальной демократии одновременно. Большинство из
нас в душе скорее \enquote{совки и ватники}, но чтобы чувствовать себя в безопасности,
мы носим вышиватный камуфляж. 

А ещё мы так разочаровались в обещаниях и самих основах государства —
правопорядке, справедливости, законе, — что всё больше готовы противостоять
своим оппонентам, внешним или внутренним, силой кулаков и оружия. Причём даже
больше, чем в 2014 году. 

И эта черта накладывает отпечаток не только на результаты соцопросов , но и на
политические векторы, в коридоре которых движется слабая, несамостоятельная, но
агрессивная украинская элита. 

В этой беседе с директором Украинского Института политики Екатерина Жарких
попыталась выяснить, насколько пропитано войной и готовностью воевать
украинское общество. Но в процессе увлекательной беседы получила гораздо
больше: комплексную картину трансформации ментальности Украины образца 2021
года, которая даже физически не совпадает со страной весны 2014-го. 

Из \enquote{мини-империи} Украина без Крыма и части Донбасса с 15\% населения, стала
более гомогенной. Националистическая прослойка из маргинальной превратилась в
доминирующую политическую группу. Наш патриотизм умышленно и не без внешнего
участия выкрашен в чёрно-красные цвета радикального национализма. 

Наше преклонение перед героями, готовыми рисковать жизнью, велико в
независимости от того, ради чего и с какими идеями они это делают.  Потому что
большинство из нас на такое неспособны. Поэтому героика Евромайдана остаётся
цементирующим мифом нового государства и его элит. А все остальные будут
камуфлировать ради комфорта и безопасности.  

Как мы пришли к этому, и какие ещё тренды заметны в эволюции украинского
общества? Почему система \enquote{всемогущего СНБО} очень хрупка, и главное, что с этим
всем делать, чтобы вернуть стране шанс стать суверенной и комфортной для всех
граждан?

Ответы ищут и находят Екатерина Жарких и её гость политолог Руслан Бортник. 

20 апреля 2021, Киев
