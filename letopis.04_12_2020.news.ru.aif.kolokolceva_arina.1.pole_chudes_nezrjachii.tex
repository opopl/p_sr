% vim: keymap=russian-jcukenwin
%%beginhead 
 
%%file 04_12_2020.news.ru.aif.kolokolceva_arina.1.pole_chudes_nezrjachii
%%parent 04_12_2020
 
%%url https://altai.aif.ru/society/slepym_nelzya_nezryachego_muzykanta_ne_vzyali_na_pole_chudes
 
%%author Колокольцева, Арина
%%author_id kolokolceva_arina
%%author_url 
 
%%tags pole_chudes,russia
%%title «Я смогу». Алтайский незрячий музыкант добился приглашения на «Поле чудес»
 
%%endhead 
 
\subsection{«Я смогу». Алтайский незрячий музыкант добился приглашения на «Поле чудес»}
\label{sec:04_12_2020.news.ru.aif.kolokolceva_arina.1.pole_chudes_nezrjachii}
\Purl{https://altai.aif.ru/society/slepym_nelzya_nezryachego_muzykanta_ne_vzyali_na_pole_chudes}
\ifcmt
	author_begin
   author_id kolokolceva_arina
	author_end
\fi

\ifcmt
  pic https://aif-s3.aif.ru/images/022/113/8a6ead22a1721886bb9b5389fa3a5bf1.png
  caption  Александр - руководитель народного хора «Ивушка». © / Александр Данилов / личный архив 
  width 0.5
  fig_env wrapfigure
\fi

\index[names.rus]{Данилов, Александр!Незрячий музыкант, руководитель народного хора ``Ивушка'', Поле Чудес, 04.12.2020}

\textbf{Житель алтайского Змеиногорска Александр Данилов}, инвалид первой группы по
зрению, давно мечтал попасть на «Поле чудес». Музыкант, певец и руководитель
народного хора «Ивушка» недавно написал письмо Леониду Якубовичу, в котором
поздравил ведущего с прошедшим юбилеем программы, пожелал крепкого здоровья, а
также поделился мечтой принять участие в передаче и, если посчастливится,
получить особый приз от любимого ведущего.

\subsubsection{Надежда на волшебство}

В жизни Данилова уже случалось доброе волшебство. Так, юристы незрячей певицы
Дианы Гурцкой однажды помогли его семье получить однокомнатную квартиру.

«Я подумал, а может быть, смогу поучаствовать в новогодней игре и получить
подарок. Вдруг случится чудо, и призом станет особое устройство для незрячих
«ЭльБрайль», я осмелился написать про это в письме», - поделился  Александр
Данилов с корреспондентом «АиФ-Алтай».

Устройство, о котором он мечтает, по словам инвалида, позволяет незрячему
человеку не только общаться с друзьями, но и читать, редактировать и выполнять
другие действия на компьютере наравне со зрячими пользователями.

\subsubsection{Не увидите табло}

\ifcmt
pic https://static1-repo.aif.ru/1/f7/1671912/c/8579b4a8bb7f3e34e47b315d12409552.jpg
cpx Фото: личный архив/ Александр Данилов 
\fi

Какова же была радость слепого музыканта, когда вечером 2 декабря ему позвонил
редактор программы «Поле Чудес». Но после разговора Александр упал духом и даже
не мог сдержать слез.

Молодой человек, представившийся ему как Леонид, сказал, что съемки программы
будут 25-27 декабря.

«Я задал вопрос: «Скажите, пожалуйста, а инвалиды по зрению могут принимать
участие в программе «Поле чудес»? - рассказывает Данилов. - В ответ услышал:
«Нет. У нас правила, вы не можете принимать участие в передаче. Потому что вы
не увидите табло, которое перед вами, и не сможете прочитать слово». Я ответил,
что у меня прекрасная память, я запоминаю названные буквы и могу составить
слово, но молодой человек предложил приехать коллективу и чтобы кто-нибудь
играл вместо меня».

В телефонном разговоре с земляками Данилова (в редакции  «АиФ-Алтай» имеется
аудиозапись) Леонид пояснил, что речь идет даже не о правилах, а просто, мол, у
программы нет технической возможности сделать так, чтобы незрячий человек в ней
участвовал. И зрителям это будет непривычно…

«Не понимаю этого, - удивляется Александр, - я с детства не видел, но часто
решал кроссворды, и картинка у меня складывается в голове, я могу решать задачу
на слух. И барабан могу крутить, и слово отгадывать. На мой взгляд, это просто
дискриминация какая-то».

%\clearpage
\subsubsection{«Поле чудес» ждет!}

Между тем чудо все-таки случилось. Редакция телепередачи вечером 4 декабря
(после звонка корреспондента «АиФ-Алтай») изменила свое мнение и теперь ждет
Александра на съемки!

«В практике программы еще ни разу не происходило, чтобы игроком был незрячий
человек, которому придется  отгадывать слово на табло. Но учитывая горячее
желание Александра принять участие в программе и его уверенность в том, что он
справится с задачей, мы приняли решение пригласить его на съемки», - сказала
продюсер программы «Поле чудес» Галина Кузнецова.

Редакция «АиФ-Алтай» будет следить за эфиром и желает Александру победы.
