% vim: keymap=russian-jcukenwin
%%beginhead 
 
%%file 28_01_2022.fb.fb_group.story_kiev_ua.1.kiev_visim_sekretiv.8.muzej_getmanstva
%%parent 28_01_2022.fb.fb_group.story_kiev_ua.1.kiev_visim_sekretiv
 
%%url 
 
%%author_id 
%%date 
 
%%tags 
%%title 
 
%%endhead 
\subsubsection{8. ОКРАСА ПОДОЛУ - \enquote{МУЗЕЙ ГЕТЬМАНСТВА}}

8. ОКРАСА ПОДОЛУ - \enquote{МУЗЕЙ ГЕТЬМАНСТВА}. 

Дуже красива будівля, яка була занедбана до кінця 80х, коли допомогою діаспори
її вдалося відновити і вона постала мов Фенікс з попелу, щоб осяяти собою
Києво-Поділ.

Музей декларує, що з’явився: \enquote{з метою правдивого висвітлення проблем
державотворення в Україні, надання неупередженої оцінки її видатних діячів,
розкриття специфічної форми правління – Гетьманщини}.

Тут постійно діють наукові виставки: \enquote{Гетьман Іван Мазепа},
\enquote{Пилип Орлик – Гетьман, автор першої демократичної конституції України}
і \enquote{Павло Скоропадський та Українська держава 1918 року}. 

Хоча фонди Музею налічують понад 9000 музейних предметів, але зали дуже
невеликі і обійти їх можна менше як за годину. Цікавими є наукові зустрічі, які
регулярно проходять тут, де люди можуть обговорити питання історії нашої
Гетьманщини-України. А ще, кажуть, тут відбуваються романтичні вечори в стилі
Гетьманської епохи і задушевні розмови про цю прекрасну сторінку нашої історії.

Зняв для вас фото будівлі музею дроном, бо мало зустрічав її таких красивих
ракурсів (додаю). 

На цьому поки закінчимо прогулянку Києвом.

Сподіваюсь цей мій путівник стане багатьом у нагоді.

Пізнавайте, досліджуйте наше місто і нашу історію. Вони того варті.

З любов'ю до Києва і України.

