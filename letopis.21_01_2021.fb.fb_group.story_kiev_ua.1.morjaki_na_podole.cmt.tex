% vim: keymap=russian-jcukenwin
%%beginhead 
 
%%file 21_01_2021.fb.fb_group.story_kiev_ua.1.morjaki_na_podole.cmt
%%parent 21_01_2021.fb.fb_group.story_kiev_ua.1.morjaki_na_podole
 
%%url 
 
%%author_id 
%%date 
 
%%tags 
%%title 
 
%%endhead 
\zzSecCmt

\begin{itemize} % {
\iusr{Vitali Andrievski}

Молодец Петр. Я подолянин, но к Морполиту никакого отношения не имел. Зато
много лет работал на Днепре на круизных теплоходах, в основном на ВУЧЕТИЧЕ. Да
тогда Днепр был очень оживленный. Начальником Главречфлота был Н. А. Славов,
благодаря которому и Днепр и многое на Подоле оживилось. Жаль, что после него
все это пришло в запустение, пассажирского флота практически не существует

\begin{itemize} % {
\iusr{Віта Дімінська}
\textbf{Vitali Andrievski} Мій батько працював в відділі кадрів при А. Н. Славові. Ще був теплохід 25 з'їзд

\iusr{Vitali Andrievski}
\textbf{Віта Дімінська} 

був не тiльки 25 зiзд, на якому до речi, менi теж довелося працювати. Загалом
було 12 круiзних теплоходiв. На жаль зараз немае жодного з них


\iusr{Віта Дімінська}

Як казав мій батько, було три флагмани Главрічфлота :25, ВУЧЕТИЧ... третього не
пам'ятаю! На вихідних ходили на Канів, там і зустрічались.


\iusr{Vitali Andrievski}
\textbf{Віта Дімінська} 

було загалом 12 теплоходiв 301 i 302 проектiв. Першим був у 1976 роцi \enquote{Евгений
Вучетич} а потiм \enquote{25 сьезд}, \enquote{Ленин}, \enquote{советская Россия}, \enquote{добролюбов}. Это
были 301 проекта а потом пошли 302 проекта \enquote{ Глушков}, \enquote{Ватутин}, \enquote{Рыбалко},
\enquote{Ватченко}, \enquote{кошевой}, \enquote{лавриненков} и последний был \enquote{Т. Шевченко}. Загалом 12
теплоходiв. Жодного на жаль зараз на Днiпрi не залишилось

\end{itemize} % }

\iusr{Евгений Мартынович}
Спасибо за содержательный рассказ!

\iusr{Елена Свец Баркан}

Я в свое время работала в училище на Андреевской и была в совете молодых
педагогов Подольского района, отвечала за культмассовую работу. Организовывала
вечера отдыха, как они тогда назывались. Проходили они чаще всего в кафе
\enquote{Ярославна} на Константиновской. Молодые педагоги обычно были женского пола и,
для того, чтобы вечера проходили весело и приятно, приходилось прибегать к
помощи комсоргов КВВМПУ. Ходила к ним на проходную, договаривалась о количестве
ребят. Не знаю, может кто-то таким образом свою судьбу устроил?)

\begin{itemize} % {
\iusr{Петр Кузьменко}
\textbf{Елена Свец Баркан} не сомневаюсь! @igg{fbicon.wink} 

\iusr{Анатолий Кашпур}
Так вы, Елена, свахой работали?  @igg{fbicon.beaming.face.smiling.eyes} 

\iusr{Елена Свец Баркан}
\textbf{Анатолий Кашпур} скорее организатором мероприятий (массовиком-затейником) @igg{fbicon.grin} 

\iusr{Нелли Кузьменко}
\textbf{Елена Свец Баркан} Не поверишь, я на танцы ходила только тогда, когда Петюня на вахта стоял дежурный по клубу.

\iusr{Нелли Кузьменко}
\textbf{Елена Свец Баркан} И когда наши ребята, с Петькиного класса, на дискотеки играли, группа \enquote{Наследники}

\iusr{Роман Воловенко}
\textbf{Елена Свец Баркан} 

тогда модно было проводить случки между муж и жен уч заведениями зарождались
новые семьи нас авиакурсантов возили на случку в пердильное училище в Кр Рогу.

\end{itemize} % }

\iusr{Ольга Беляева}

А я с подружками на дискотеку в КВВМПу ходила... Потом выпускной, ребята в
форме с аксельбантами и кортиком-красавцы!!! А еще розы шикарные у главного
входа!!!! Как вчера...


\iusr{Марина Бортницкая}

Спасибо, чудесно написали повествование, Я жила во 2 номере Андреевского
спуска, мои окна выходили на Военно-морское училище, часто смотрела, с братом на
подготовки к парадам, жаль, всё ушло, и якоря убрали, только остались пару кустов
роз,,,

\begin{itemize} % {
\iusr{Петр Кузьменко}
\textbf{Марина Бортницкая} это мой родной дом. О нём есть мой отдельный пост в группе.

\iusr{Марина Бортницкая}
\textbf{Петр Кузьменко} 

спасибо поищу, с удовольствием прочитаю, на Андреевском спуске 2 жили с 1974 года
до отселения 1990 года, училась в 20 школе, мой брат в 100 школе, каждый раз, когда
бываю на Андреевском находит грусть, и в моем доме сейчас типа модное заведение
реберня Пьяная вишня,

\iusr{Петр Кузьменко}
\textbf{Марина Бортницкая} 

просто нажмите на мою аватарку в группе. Увидите все мои публикации.
Практически в каждом моём посте есть упоминание о моём родном доме.

\iusr{Марина Бортницкая}
\textbf{Петр Кузьменко} спасибо, мне интересно

\iusr{Марина Бортницкая}
\textbf{Петр Кузьменко} Спасибо большое, нашла, очень интересно
\end{itemize} % }

\iusr{Оксана Тернавская}
Спасибо большое Вам за рассказ от семьи военного моряка)))

\iusr{Олександр Громов}
Я со второго выпуска, 1972 год. Когда мы поступили было только два здания.

\iusr{Олег Гринюк}

Я жил в этом доме Волошская - Григория Сковороды, ранее
Набережно-Никольская 12/4, т. е. до 1977 года, когда наш дом и был передан КВВМПУ, и
поступал в 1975 экзамены сдавали в Лютеже.

\begin{itemize} % {
\iusr{Павел Пауль}
\textbf{Олег Гринюк} дом известный. Жили там замечательные люди.

\iusr{Олег Гринюк}
\textbf{Павел Пауль} 

дом Водников, капитаны Теплоходов, мой отец закончил в 1955 КИевский речной
техникум, потом ЛИВТ БЫЛ Начальником ОТК КССРЗ им. Сталина, потом просто
судостроительный завод., а в доме действительно жили Прекрасные люди, научные
работники, профессора, механики СТешек-барж, их капитаны, двор был очень дружный.


\iusr{Павел Пауль}
\textbf{Олег Гринюк} 

брат моего деда на Ленкузне мастером котельного цеха был. Марчук Евгений
Григорьевич. Содеди его Русаковы. Начальником Гимс был Виктор. Часто бывал там
в детстве.


\iusr{Олег Гринюк}
\textbf{Павел Пауль} 

Русаковы жили в нашем подъезде, Вова был гребцом бойдарочником чемпионом мы
дружили, потом они жили там где Ген. Прокуратура на Подоле, Раиса Соя гребчиха то
же чемпионка и ее семья мои соседи по коммуналке, да были люди в наше время....


\iusr{Павел Пауль}
\textbf{Олег Гринюк} я старше, а то могли бы и встречаться. Батя мой одно время тоже там жил. Он механиком на теплоходах работал.

\iusr{Олег Гринюк}
\textbf{Павел Пауль} значит моего отца точно знал.

\iusr{Павел Пауль}
\textbf{Олег Гринюк} однозначно. Директором завода помню Матяш был, деканом института Платонов А. М.

\iusr{Олег Гринюк}
\textbf{Павел Пауль} это друзья моего отца, я их хорошо помню, пацаном был,

\iusr{Павел Пауль}
\textbf{Олег Гринюк} на охоте пацаном с ними рос. Начальником пароходства Бобровникова ещё был. Марчук Вильям Павлович-отец мой.

\iusr{Олег Гринюк}
\textbf{Павел Пауль} Бобровникова если не ошибаюсь Анатолием звали.

\iusr{Павел Пауль}
\textbf{Олег Гринюк} Бобровникова Николай Андреевич.
\end{itemize} % }

\iusr{Ольга Кирьянцева}

Спасибо большое Вам за рассказ. Так замечательно написано! С огромной любовью и
гордостью за свое училище. Я жила некоторое время у родственников на Подоле,
все пути мои проходили через Красную площадь. И мне всегда хотелось хоть одним
глазком подглядеть, где и как учатся эти молодые, красивые, подтянутые
ребята-курсанты). Это ж столько девчачьих сердечек замирало и таяло при виде
тех красавцев!)) Знаю по себе)). А если удавалось попасть на их построение на
площади, то это был настоящий празник для души)). Спасибо! Вы были (и
остаётесь, я уверена,) - классными!

И с праздником Вас и всех Ваших друзей - выпускников училища.

\iusr{Елена Лясковская}

Да, попасть к вам на дискотеку было нелегко ! Но мне с подружкой однажды
повезло в далёком 1985 году @igg{fbicon.thumb.up.yellow}  @igg{fbicon.woman.dancing} 

\iusr{Ольга Писанко}
А мы были пару раз на дискотеке с подругой в году 79!

\iusr{Недоля Алексей}

А у нас со второго курса 12 роты, 1 группы Рижского Мореходного Училища в
Киевский Морполит перевёлся в 1987 году Литовченко Нормунд. Удивительная
история: я коренной Киевлянин уехал в Ригу, он, Латыш в Киев.

Хотелось бы узнать его судьбу. Может кто учился с ним вместе?


\iusr{Юлия Лепашева}

А у нас в доме батюшка живет, Сейчас батюшка, а в конце 80 ых бегал в форме
Морполита.:):):) Наверное полит. работники и духовники миссию выполняют
одинаковую.:):):)

\begin{itemize} % {
\iusr{Петр Кузьменко}
\textbf{Юлия Лепашева} об этом я писал в комментариях к посту о библиях в нашей группе. Еще в 1980-м олимпийским году я собирался поступать в Духовную семинарию.  @igg{fbicon.hands.shake} 
\end{itemize} % }

\iusr{Ілона Луценко}

Мои дедушка с бабушкой жили на Спасской улице и всегда гуляли в парке моряков.
Дедушка очень хотел что я его Внучка вышла замуж за военного моряка. Его мечта
сбылась через много лет.

\iusr{Alik Gurevich}

Пётр! Можно поправить несколько неточностей в Вашем замечательном
повествовании... С 1-го июня 1967 года началось формирование училища,
торжественное открытие состоялось 1-го октября того же года! Знаю точно, так
как в январе 1967 года был направлен в школу ОСНАЗ ВМФ и до середины июня
проходил обучение и службу там, в здании на Красной площади. А после, нас
перевели оканчивать обучение на Рыбальский .... Так как начался ремонт зданий и
образование училища.

\begin{itemize} % {
\iusr{Петр Кузьменко}
\textbf{Alik Gurevich} 

спасибо за уточнение, которое уверен с удивлением воспримут наши выпускники.

\iusr{Владимир Картавенко}
\textbf{Петр Кузьменко} 

вначале был командующий Днепровской военной флотилией матрос Полупанов, потом
младший морполит и две школы на Рыбальском. Решением Щербицкого В.В. подписаны
постановления о создании КВВМПУ и 316 УО ОСНАЗ в/ч 20884 ЧФ.

\end{itemize} % }

\end{itemize} % }
