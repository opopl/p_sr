% vim: keymap=russian-jcukenwin
%%beginhead 
 
%%file 21_07_2018.stz.news.ua.mrpl_city.1.viktor_kalognomos
%%parent 21_07_2018
 
%%url https://mrpl.city/blogs/view/viktor-kalognomos
 
%%author_id burov_sergij.mariupol,news.ua.mrpl_city
%%date 
 
%%tags 
%%title Виктор Калогномос
 
%%endhead 
 
\subsection{Виктор Калогномос}
\label{sec:21_07_2018.stz.news.ua.mrpl_city.1.viktor_kalognomos}
 
\Purl{https://mrpl.city/blogs/view/viktor-kalognomos}
\ifcmt
 author_begin
   author_id burov_sergij.mariupol,news.ua.mrpl_city
 author_end
\fi

\ii{21_07_2018.stz.news.ua.mrpl_city.1.viktor_kalognomos.pic.1}

14 сентября 1996 года по довольно крутой лестнице профилактория Азовского
морского пароходства спускался вприпрыжку человек небольшого роста с типичной
фигурой гимнаста – мощный торс, узкий таз и голова, покоящаяся над развитыми
плечами. И только прическа под бобрик, посеченная сильной проседью, выдавала
его почтенный возраст, на первый взгляд эдак лет под шестьдесят. Это был
заслуженный работник физической культуры Российской Федерации, ведущий инженер
Всероссийского НИИ по спортивно-техническим изделиям, судья международной
категории по спортивной гимнастике, главный судья и директор соревнований по
этому виду спорта самого высокого уровня, автор ряда книг, с названиями которых
досужий читатель может познакомиться в Интернете, \textbf{уроженец Мариуполя Виктор
Иванович Калогномос.}

Он приехал с женой в родной город, чтобы отдохнуть. При ближайшем знакомстве с
Виктором Ивановичем выяснилось, что он родился в 1906 году, стало быть, ему
тогда было девяносто лет. Между прочим, он объяснил свое долголетие занятием
спортом, рациональным питанием, а также тем, что за свою продолжительную жизнь
не пробовал ни табака, ни напитков, содержащих хотя бы малую толику спирта.

\textbf{Читайте также:} \href{https://mrpl.city/blogs/view/fizkultura-vozrastu-ne-pomeha}{Физкультура возрасту не помеха, Алина Егоза, mrpl.city, 18.06.2018}

В тот же день оператор ТРКК \enquote{Сигма} Виктор Дедов записал на видеокамеру беседу
со знаменитым земляком, которая стала основой для очередного выпуска программы
\enquote{Мариуполь. Былое}. С той поры прошло ровно пятнадцать лет, но, к счастью,
сохранилась стенограмма воспоминаний знаменитого земляка. Она-то и предлагается
вниманию читателям с минимальной правкой.

\begin{quote}
\em\enquote{Я родился в Мариуполе в родительском доме, который и сейчас стоит на
Итальянской улице. С десяти лет начал заниматься гимнастикой. Поэтому вся моя
жизнь была посвящена этому виду спорта. Гимнастике на снарядах. И я хотел бы
поделиться воспоминаниями, чтобы мариупольцы знали, какое значение эта
гимнастика в Мариуполе оказала на развитие спорта в Украине. Начало развития
гимнастики в Мариуполе положил Виталий Поликарпович Попович. Он учился два года
в Праге в Сокольском учебном заведении}.
\end{quote}

Отвлечемся пока от повествования Виктора Ивановича, чтобы разъяснить термин
\enquote{сокольский}. Сокольское молодёжное спортивное движение, основано в Праге в
1862 году М. Тыршем. С перерывами продолжает работать по настоящее время. Хотя
официально оно является неполитическим, фактически было носителем и
распространителем идей чешского национализма и панславизма. Именно в недрах
этого движения широко культивировалась спортивная гимнастика.

\begin{quote}
\em
\enquote{По приезду в Мариуполь Виталий Поликарпович начал преподавать в Мариупольской
Александровской мужской гимназии. Его занятия с большой охотой посещали
гимназисты. Он проводил занятия и в других учебных заведениях города, в
частности в частной женской гимназии Валентины Епифановны Остославской. Попович
организовывал показательные выступления своих питомцев – учеников местных
гимназий на большом пустыре рядом с городской больницей, где теперь построены
многоэтажные дома. Выступления были интересными, их с удовольствием смотрели
горожане. С моей точки зрения, это было начало той гимнастики, которая
появилась в Мариуполе, а далее в Украине и в дореволюционной и
послереволюционной России. У Виталия Поликарповича были выдающиеся ученики,
среди них и Виктор Чукарин, спортсмен мировой славы, Президент федерации
гимнастики СССР, заведующий кафедрой физвоспитания Университета дружбы народов
имени П. Лумумбы В. П. Фролов, А. С. Мишаков.

Александр Семенович Мишаков родился в Мариуполе, жил в поселке при заводе имени
Ильича. Там был кружок по гимнастике, где преподавал Попович, у него занимался
Александр. В дальнейшем он переехал в Киев, стал заслуженным тренером Украины и
Советского Союза. Тренер Мишаков – известен во всем мире. Он подготовил
неоднократных олимпийских чемпионов Ларису Латынину и Бориса Шахлина. Это был
великий тренер, сделавший огромное дело для развития гимнастики в Украине и
Советском Союзе.

Я начал учиться в мариупольской мужской гимназии в 1914 году. В следующем 1915
– 16 учебном году старшеклассниками был организован гимнастический кружок,
который положил начало сокольскому гимназическому обществу в Мариуполе. И это
общество устраивало большие показательные выступления. В 1917 году на месте
нынешнего стадиона около Городского сада, тогда там был земляной велотрек,
выступали гимнасты разного возраста. В том числе и младшая группа, которой
руководил я. И в моей группе был Ваня Остославский, в будущем он стал видным
учёным в области аэродинамики, доктором технических наук, профессором,
заслуженным деятелем науки и техники РСФСР. Он выступал с нами, малышами.

В 1923 году я начал работать в 1-й Мариупольской образцовой школе. Она
находилась на углу Греческой и Николаевской улиц. Она отличалась высоким
уровнем преподавания. Там же рядом я построил большую гимнастическую площадку.
На ней мы занимались гимнастическими упражнениями.}

\end{quote}

\textbf{Читайте также:} 

\href{https://mrpl.city/news/view/mariupolskogo-gimnasta-igorya-radivilova-v-tretij-raz-priznali-luchshim-sportsmenom-ukrainy}{%
Мариупольского гимнаста Игоря Радивилова в третий раз признали лучшим спортсменом Украины, Ярослав Герасименко, mrpl.city, 02.04.2018}

\begin{quote}
\em\enquote{Позднее, в начале 1924 года появился в Мариуполе Красный спортивный клуб имени
Подвойского. В нем было несколько сокольских спортивных снарядов. Зал был
небольшой на втором этаже одного из домов на Большой улице. Клубу этому никто
средств не отпускал. Чтобы как-то его поддержать, мы устраивали еженедельно
гимнастические выступления, танцевальные вечера. За счет платы за вход от
посетителей этих развлечений удавалось как-то содержать клуб. В Мариуполе я
пробыл до 1930 года.

В 1930 году переехал в Москву. Поселился на квартире у своего дяди Георгия
Ивановича Челпанова, известного во всем мире ученого-психолога, где и прожил
некоторое время, пока не нашел своего собственного жилья...

В Москве я преподавал в высших учебных заведениях, заведовал кафедрами
физвоспитания разных вузов. В это же самое время я начал заниматься судейством
спортивной гимнастики. Я был членом президиума Всесоюзной коллегии судей, а
последние сорок лет, - до развала Советского Союза – ее председателем. Сейчас я
- член исполкома федерации России по спортивной гимнастике. Еще в 1946 году
получил звание судьи всесоюзной категории, а в 1965 году – звание судьи
международной категории. Мне довелось проводить множество соревнований в
качестве главного судьи и их директора.

Надо сказать, что соревнования по гимнастике очень громоздкие, они используют
большое количество специального оборудования. На первый взгляд требования к ним
просты. Но на самом деле они формируются из соображений безопасности
спортсменов. Снаряды должны иметь прочность, упругость, форму в пределах
определенной нормы. Даже требования к гимнастическим матам, которые
используются на состязаниях высокого уровня, оговорены специальными
техническими условиями. В разработке этих документов довелось и мне
участвовать. Кроме того, для соревнований мы разработали и изготовили
гимнастическую судейскую аппаратуру. Она помогает подсчитывать результаты
участников соревнований, дает возможность судье быстро производить все
достаточно сложные арифметические вычисления. Кроме того, эта аппаратура
информирует зрителей и участников о ходе соревнований, передавая их на табло.
Вот этим, собственно говоря, я занимался и продолжаю заниматься. За это я
получил несколько патентов и авторских свидетельств об изобретениях.

Я провел очень много соревнований. Впервые я был главным судьей в 1949 году в
Ленинграде на чемпионате Советского Союза. После этого я провел два чемпионата
мира, два чемпионата Европы, международные студенческие соревнования, в 1957
году соревнования по гимнастике на VΙ Всемирном фестивале молодежи и студентов
в Москве, в рамках которого были проведены ΙΙΙ Международные дружеские игры
молодежи. На Олимпиаде-80 в Москве я также был главным судьей и директором
соревнований по гимнастике. Будучи главным судьей на Играх доброй воли в
Москве, я столь удачно провел соревнования гимнастов, что американцы пригласили
меня в Америку на очередные Игры в качестве почетного гостя. Ну, а если
говорить о моем судействе внутри страны, то это тридцать или более чемпионатов
Советского Союза, десять Спартакиад народов СССР}.	
\end{quote}

Когда съемочная аппаратура была собрана, и оставалось только дождаться
автомобиля, чтобы ехать на телестудию, Виктор Иванович после небольшой паузы
продолжил разговор. Он рассказал, что его дед Фотий Калогномос приехал в
Мариуполь из Константинополя во второй половине ΧΙΧ века. Здесь он довольно
успешно занимался коммерцией. По действующим в то время в Российской империи
законам для иностранцев его наследники имели право на обучение в
государственных учебных заведениях за казенный счет. Этим правом
воспользовались все его дети. Дочери Екатерина, Марфа, Анастасия и Евпраксия
окончили Мариинскую женскую гимназию, а сыновья Иван и Диомид – Александровскую
мужскую гимназию. Иван, обретя аттестат зрелости, приступил к работе агента
хлеботорговой фирмы, а Диомид отправился в Киев, где был принят на медицинский
факультет университета Св. Владимира...

Приехала наша машина. Прощаясь с нами, Виктор Иванович сказал, что он согласен
ответить и на другие вопросы, связанные с его пребыванием в родном городе. Но
поскольку его отдых заканчивается, лучше обращаться, когда он приедет в Москву.
По пустякам Виктора Ивановича беспокоить не хотелось, и только через год был
набран номер телефона его квартиры. На той стороне провода прозвучал голос его
жены: \emph{\enquote{Виктор Иванович умер}}. 
