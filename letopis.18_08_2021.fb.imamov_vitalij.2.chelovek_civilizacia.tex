% vim: keymap=russian-jcukenwin
%%beginhead 
 
%%file 18_08_2021.fb.imamov_vitalij.2.chelovek_civilizacia
%%parent 18_08_2021
 
%%url https://www.facebook.com/permalink.php?story_fbid=2838631919745118&id=100007950128049
 
%%author_id imamov_vitalij
%%date 
 
%%tags chelovechnost,chelovek,civilizacia
%%title Людство в тіні марива цивілізації
 
%%endhead 
 
\subsection{Людство в тіні марива цивілізації}
\label{sec:18_08_2021.fb.imamov_vitalij.2.chelovek_civilizacia}
 
\Purl{https://www.facebook.com/permalink.php?story_fbid=2838631919745118&id=100007950128049}
\ifcmt
 author_begin
   author_id imamov_vitalij
 author_end
\fi

\begin{multicols}{2}
\obeycr
Людство в тіні марива цивілізації.
Або
Не випадково в олімпійський девіз додали "разом" саме зараз.
Стихійні лиха, крах/надлом суспільного порядку, -  на короткий час розвіюють ілюзію впорядкованості життя певної людської спільноти.
Ця ілюзія лежить в головах і підкріплюється реальними здобутками цивілізації та конкретної культури.
Такі наднові висвітлюють поточний стан людства і узагальнений стан Людини, як "виду".
І що ж видно?
\ifcmt
  ig https://scontent-frx5-1.xx.fbcdn.net/v/t39.30808-6/239474880_2838631889745121_23723609738559986_n.jpg?_nc_cat=100&ccb=1-5&_nc_sid=8bfeb9&_nc_ohc=4vxQjMUyI-UAX90zahe&_nc_ht=scontent-frx5-1.xx&oh=a2fc3c9f485f7c7497600043765ccc2a&oe=613DD0BC
  @width 0.4
\fi
Така, узагальнена Людина зустрічає форс-мажорні зовнішні виклики:
- або втечею;
- або спробою зіграти на стороні виклику (там де виклик організовують люди);
- або спробою прикинутися "мертвою", затаїтися, невисовуватися.
Тобто Людина воліє покладатися (спертися на):
- на вдачу, що вдасться втекти від сили;
- на вдачу, що сила прийме за "свого";
- на вдачу, що сила не помітить.
А це вже знайома модель поведінки для Людини і людства.
Ми вже кілька разів її трансформували в іншу модель поведінки.
Щоправда, доводиться це робити щоразу в інших обставинах.
Коли такою силою виступав 
- шаблезубий тигр чи  ще купа всяких звірів
чи
- вороже налаштовані до ранніх християн язичники  
ми змогли напрацювати нову модель поведінки.
І це була опора на свою спільноту. Люди жили спільнотою, по справжньому цікавилися один одним, і проживали та йшли свій життєвих шлях поруч з особливими ближніми.
Тобто вони покладалися на те, що:
- разом вистоять супроти сили. 
Звісно і зараз, деяким людям щастить, хоча б на певний період в житті бути частиною подібної спільноти.
Людство знову має прийти до співжиття в таких спільнотах. Та тепер це треба зробити врахувавши та зберігши цивілізаційні надбання.
Потрібні спільноти, що об'єднують людей не задля безпеки, не за ідеологіями чи вірою, а за відчуттям чогось цінного.
Цінності втілюють - це постійний процес, результатом якого є створене цінне.  А цінне можна підтримувати, захищати, розвивати.
Знайдеться праця для душі, для кожного.
Людина має навчитися бачити в Людині опору. 
Опору, яка фокусує силу. 
Свою силу та силу іншого.
Зокрема для цього і створив Бог суб'єктів. Він через нас, зокрема, фокусує власну силу.
\restorecr
\end{multicols}
