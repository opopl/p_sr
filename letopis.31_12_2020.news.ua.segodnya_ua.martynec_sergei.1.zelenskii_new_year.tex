% vim: keymap=russian-jcukenwin
%%beginhead 
 
%%file 31_12_2020.news.ua.segodnya_ua.martynec_sergei.1.zelenskii_new_year
%%parent 31_12_2020
 
%%url https://politics.segodnya.ua/politics/novogodnee-obrashchenie-zelenskogo-k-ukraincam-polnyy-tekst-1498465.html
 
%%author 
%%author_id martynec_sergei
%%author_url 
 
%%tags 
%%title Новогоднее обращение Зеленского к украинцам: полный текст
 
%%endhead 
 
\subsection{Новогоднее обращение Зеленского к украинцам: полный текст}
\label{sec:31_12_2020.news.ua.segodnya_ua.martynec_sergei.1.zelenskii_new_year}
\Purl{https://politics.segodnya.ua/politics/novogodnee-obrashchenie-zelenskogo-k-ukraincam-polnyy-tekst-1498465.html}
\ifcmt
	author_begin
   author_id martynec_sergei
	author_end
\fi

* *  *  * Глава государства нашел, что сказать украинцам после такого тяжелого года

Президент Владимир Зеленский поздравил украинцев с Новым годом. Он
пообещал, что 2021 год будет лучше уходящего, но в то же время вспомнил,
каких выдающихся людей забрал у нас 2020-й. Читайте полный текст
новогоднего обращения главы государства к народу.


Дорогие украинцы!

До конца 2020 года остаются считанные минуты. Время подводить итоги,
делать выводы, говорить о планах и целях... Скучно? Согласен! Скажу
понятно для вас. И это логично, ведь все, что мы делаем – делаем для вас.
Мы живем в большой, прекрасной стране, где невероятная природа, щедрая
земля, где умные и талантливые дети, и где все президенты в своих
новогодних поздравлениях всегда говорят фразу: "Дорогие украинцы, это был
тяжелый год...".

Каким же был этот год? В нем были слезы от боли и слезы от гордости. Было
то, за что нам стыдно, и то, чем мы гордимся. То, что хочется забыть, и
то, что никогда не забудем. И были те, с кем мы стали еще большими
друзьями. Те, с кем стали стратегическими партнерами. Те, с кем нас
пытались поссорить, но не удалось. Те, кто поддерживал Украину всегда и
продолжает это делать.

2020-й напомнил, как на самом деле много героев среди нас. Весной мы
начали все болеть. Оказалось, что в больницах и аптеках много чего нет.
Почему? Просто некоторые потеряли обоняние еще задолго до ковида. И вот
все мы были как в фильмах, когда кажется, что спасти могут только
супергерои. Взрослые не верят в их существование. Но в этом году, как и в
2014-м, мы еще раз убедились – супергерои существуют. И к нашим военным
присоединились наши медики, наши суперврачи, суперлаборанты, суперводители
"скорой", суперпилоты, суперпограничники, суперучителя, суперпожарные и
суперспасатели. И у нас нет другого варианта, чем быть супер Украиной.

Да, это был год, где, к сожалению, катаклизмы разрушали жилье и природу. И
год, где, к счастью, мы разрушали стереотипы. Например, что в Украине
невозможно делать хорошие дороги. Этот миф пошел на все четыре стороны, по
всем четырем тысячах километров новых дорог. Когда-то вам будет 18, и вы
сможете водить авто по украинским дорогам. Мы старались, чтобы за время,
пока вы повзрослеете, эти дороги не постарели. И не устали.

Этот год показал, что детскую больницу "Охматдет" можно достроить и не
передавать, словно булаву каждому следующему президенту Украины. Что можно
"снести" памятник коррупции, который установлен еще в 2004-м, и за восемь
месяцев превратить легендарный долгострой страны на новый мост в
Запорожье. А по всей Украине отстроить 150 мостов. И мы не слушаем
метателей грязи на вентилятор соцсетей. После кого останутся посты, а
после кого-то останутся мосты.

Этот год показал, что для граждан из Донбасса и Крыма можно сделать
современные КПВВ, где им станет окончательно ясно, кто относится к ним как
к людям, а кто – как к заложникам. Этот год показал, что можно об армии
говорить, а можно сделать для безопасности и обороны крупнейший в истории
бюджет. Что государство может начать совместное строительство корветов и
впервые за время независимости заказать у "Антонова" три новых украинских
самолета. А проценты кредитов для бизнеса могут быть не 20, 18 или 14\%, а
маленькими, такими, как вы – 5, 7 или 9\%.

2020-й снова доказал, что Украина своих не оставляет. И нам неважно, откуда
возвращать своих людей – подвалы ОРДЛО, российская тюрьма или китайский Ухань,
это пленные моряки танкера из Ливии или погибший экипаж и пассажиры самолета из
Ирана. И мы все равно счастливы – возвращаем 250 000 граждан, застрявших из-за
карантина по миру, или одного украинца. Нашего защитника – непокоренного
Виталия Маркива.

Этот год не раз поднимал вверх украинский флаг. Мы становились первыми в
плавании, боксе, спортивной и художественной гимнастике, легкой атлетике,
борьбе, на велотреке и даже в шашках. А когда будете спорить, кто сильнее –
Бэтмен, Росомаха или Дэдпул... на самом деле самый сильный украинец, наш
богатырь Алексей Новиков, который стал в этом году самым сильным человеком на
планете Земля. А мы не имеем другого варианта, чем стать самой сильной страной.

И самое главное. Этот год напомнил нам, что такое тишина, слова "Сегодня не
стреляли". Дни, когда в новостях говорят: "За сутки нет потерь".  Недели, когда
матери, жены и дети не плачут. Месяцы, когда наши военные не погибают. 158 дней
прекращения огня на Донбассе. Самое продолжительное с начала войны. 158 дней
перемирия. Небезупречного? Да, и это правда. Но разве из-за этого – ненужного?
Нет. И это тоже правда. Знаете, вы еще маленькие и, возможно, не до конца
понимаете, почему так происходит. Но сейчас должны точно знать одно: украинские
защитники – лучшие в мире, они смелые и очень-очень сильные. Но, к сожалению,
так происходит, и некоторых из них Бог забирает к себе на небо – лучших и
отважных. В этом году мы очень старались, чтобы Бог оставил как можно больше из
них в Украине.  Чтобы они и дальше могли нас защищать и принести мир. Он
возможен. Он близко. Он обязательно будет.

Знаете, все президенты в новогоднем обращении говорят одну фразу: "Дорогие
украинцы! Это был тяжелый год... Но следующий будет лучше". И вот с этим
согласен. Он будет лучшим для всей Украины. И как бы там ни было, давайте
поблагодарим 2020-й и не будем его ругать. Да, к сожалению, он забрал у нас
киевлянина Патона, одессита Жванецкого, львовян Скорика и Виктюка. Но подарил
нам почти 300 000 новых граждан – мальчиков и девочек, родившихся в Украине в
этом году. Давайте ради них выключим свой взрослый прагматизм и загадаем для
Украины то очень смелое и почти фантастическое. Так, как это умеют делать наши
дети.

(Дальше говорят дети)

Пусть все в новом году будут здоровы!

Пусть у всех украинцев будет много-много счастья, а у меня – много-много
конфет.

Пусть всех плохих преступников посадят в тюрьму.

Пусть всем украинцам родители купят собаку, а мне – трансформера.

Чтобы у всех было много-много денег.

Мы преодолеем COVID. И коррупцию.

Желаю Украине быть успешной. И пусть будет много-много счастья.

Скоро будет вакцинация. Держитесь! Стоп COVID-19!

Пусть в Украине наступит мир.

(Дальше снова говорит президент)

Всего этого хотят наши дети. Ну а мы? А мы все это должны выполнить.

Дорогие украинцы! В следующем году мы будем отмечать 30-летие нашей
независимости. Мы – великий народ. Мы храбрость Святослава, величие Владимира,
мудрость Ярослава. У нас есть Шевченко. Более того – у нас есть два Шевченко! И
быть президентом такого прекрасного народа – это большая гордость. И очень
большая ответственность. И я буду делать все, чтобы и через год, и во всех
последующих годах в этот же день и время мне было не стыдно смотреть вам в
глаза. Я хочу не говорить на Банковой в Киеве, что в Украине наступит мир. А
сказать на Артема в Донецке, что мир в Украине наступил. И написать "Крым – это
Украина" – не в интернете, а на песке на пляже в Ялте. На украинском песке. На
украинском пляже украинской Ялты. Я знаю, что Донецк, Луганск и Крым сейчас, во
всех смыслах этого слова, живут в другом времени. И уже почти час, как по
телевизору вам сказали, что Новый год наступил. Но я знаю, что уже почти час,
как вы ждете нас, чтоб встретить Новый год вместе. Как одна семья. Как один
народ. Как одна страна. Донбасс и Крым! Переведите часы назад. Будьте с нами!
Готовы? Мы тоже.

Я искренне поздравляю всех украинцев с Новым годом! Пусть все будут здоровы.
Пусть все будут счастливы.
