% vim: keymap=russian-jcukenwin
%%beginhead 
 
%%file letters.27_01_2023.dedova
%%parent letters.27_01_2023
 
%%url 
 
%%author_id 
%%date 
 
%%tags 
%%title 
 
%%endhead 

Добрый день, Наталья! Слава Украине!

Спасибо Вам огромное за все, что вы делаете. Это очень важно, чтобы мы все,
народ Украины, сохранили память об ужасной трагедии Мариуполя, рассказали о ней
всему миру, и по полной наказали и привели к ответу россию и путинский режим за
все, что россия уже успела натворить на нашей благословенной земле.

Меня зовут Иван, я программист из Киева.  Недавно я прочитал Ваш пост от 10
января по списку погибших в Мариуполе, и меня потрясли комментарии под постом.
За время войны, я уже как то немного привык к постоянному потоку плохих
новостей о том, что там что то прилетело, кого то убило... столько уже всего
произошло... Буча, Харьков, Львов, Одесса, Изюм, Винница... Херсон, Запорожье,
Днепр, Бровары... и тем не менее, я не мог сдержать слез, когда читал эти
комментарии.  Столько загубленных жизней, столько ужасных историй...

Простые слова, вместе с фотографиями, врезаются в душу...  Знаете, как острый
нож врезаются в душу...  Родители...  Друзья... Папа, мама.. сын, дочь...
Снайпер... мина...  Вышел за водой, убит...  Готовили еду во дворе...
Прилетело... и все, людей нет...  Огромная, невообразимая трагедия Мариуполя,
прекрасного, красивейшего Города у Моря, в котором, к сожалению, я ни разу так
и не успел побывать, хотя я и был во многих городах Украины. Персонально насчет
меня, к счастью, мою психику и мою семью война не так сильно затронула, как
многих других, поэтому у меня нет моей собственной истории, которую я мог бы
Вам послать. Тем не менее, многим другим повезло гораздо меньше. Погибло
огромное число людей, миллионы людей стали беженцами, и были вынуждены начинать
жизнь заново. Многие потеряли отца, мать, сына, дочь, друзей. И если начинать об
этом думать, то можно так целый день плакать об этом... Ну ладно. Долго плакать
я не привык, я привык действовать.

Как я сказал уже выше, я программист. Кроме моей основной работы, я также на
волонтерских началах уже довольно длительное время занимаюсь разработкой
методов, а также программного обеспечения для записи по датам, авторам и
категориям самых разных публикаций в интернете. А зачем это нужно вообще? - Мне
всегда нравилось читать, читать много, книжки, статьи, публикации - и я в какой
то момент решил для себя, что не нужно просто читать, а нужно также записывать
то, то есть, сохранять файлы, что мне интересно, чтобы оно не пропало, и всегда
можно было вернуться назад и прочитать снова. И особенно это важно касательно
публикаций в соцсетях, когда так часто кого то банят, просто ни за что,
безжалостно сносят аккаунты, как это было у Надежды Сухоруковой.

Когда началась война, я решил записывать, то, что происходило и происходит
доныне - в первую очередь, не только мои впечатления, которые и так у меня в
голове - а именно то, что пишут другие люди. В общем, получился проект Летописи
Войны, которым я занимаюсь в свободное от работы и других дел время. Пока что
он еще в довольно начальной своей фазе, потому что объем информации по войне
просто невообразимо огромен, а я пока что еще занимаюсь этим сам по себе, когда есть время.
Технология, которую я использую, называется LaTeX (произносится Л-а-т-е-х). Это
фактически отдельный язык программирования документов. Это довольно старая
технология, ей уже более 30 лет, но она повсеместно используется в научном мире
учеными (физиками, математиками) для записи и публикации своих статей.
Преимущество LaTeX над, например, Microsoft Word, в том, что вы в принципе
можете записать в один документ сколько информации пожелаете, хоть сто страниц,
хоть тысячу. А как это работает в научном мире, вы можете увидеть например,
здесь https://www.arxiv.org - это архив научных препринтов, там миллионы статей
по самым разным областям науки.

И теперь касательно Вашего поста.  Я его записал, вместе с фото, и всеми
комментариями, в один pdf-документ. В нем 130 страниц, это целая книжка
получилась. Также я отдельно записал скриншоты и оригинал поста. Теперь Ваш
пост не пропадет, даже если весь интернет рухнет. И в целом, если мы хотим,
чтобы память об этой ужасной трагедии была сохранена для потомков, если мы
хотим, чтобы весь мир узнал о Мариуполе, нужно начинать систематически
записывать. Из фейсбука, телеграма, инстаграма. Потому что ни фейсбук, ни
телеграм не являются надежными хранилищами памяти, вся эта информация может
пропасть в любое время.

И касательно Вашего проекта. Буду рад помочь Вам, чем смогу, как программист,
в плане записи, сохранения всей доступной информации об ужасной трагедии
Мариуполя, чтобы как можно больше людей по всему миру смогли узнать об этом.

Здесь я прикладываю усеченный вариант документа с вашим постом, там 16 страниц.
Полная версия - 50 Мегабайт, 130 страниц, доступен в облачном хранилище по
ссылке:

https://mega.nz/file/kiwkRT6L#bBzTfJlXZa36O7YGu8VgUdxi3fsHPzMHC-dwuCXNeq4

А другие мои документы - выставлено не очень много - по войне - вы можете найти
по ссылке

https://mega.nz/folder/NjJ0mLab#SR5BVg9x-Oxsd71im5gSXw

Также, в телеграм у меня есть канал с ссылками @kyiv_fortress_1

На ок я подписан как Иван Демидов Киев, можете меня также там найти.

И скажу еще такое. Этой ночью я решил перед сном пересмотреть для успокоения
психики фильм Звездные Войны - Эпизод 9 - The Rise of SkyWalker. Там в начале
идет заставка, и слова The dead speak! (мертвые говорят!). И знаете, мертвые на
этих фото... они живы. Живы, пока мы о них помним...

С уважением,

Иван.
