% vim: keymap=russian-jcukenwin
%%beginhead 
 
%%file 28_10_2021.fb.dzhangirov_dmitrij.kiev.1.istoricheskaja_pravda.cmt
%%parent 28_10_2021.fb.dzhangirov_dmitrij.kiev.1.istoricheskaja_pravda
 
%%url 
 
%%author_id 
%%date 
 
%%tags 
%%title 
 
%%endhead 
\subsubsection{Коментарі}

\begin{itemize} % {
\iusr{Сергей Падалкин}
Кий, Щек, Хорив и Лыбидь похоже на Дыр бул щыл убеш щур

\begin{itemize} % {
\iusr{Энлиль Маратович}
\textbf{Сергей Падалкин} А так же: мене, текел, фарес...

\iusr{Сергей Падалкин}
\textbf{Энлиль Маратович} заклинание?

\iusr{Энлиль Маратович}
\textbf{Сергей Падалкин} оно
\end{itemize} % }

\iusr{Александр Олейник}
Хорошо подмечено!

\iusr{Андрей Мишин}
Варяги - первые спойлеры на копипасте, однако. Идею у Вечного города Рима позаимствовали.

\begin{itemize} % {
\iusr{Владимир Цхведиани}
\textbf{Андрей Мишин} Кий, Шек и Хорив варягами не были. Туркменские ученые считают, что они были туркменами из племени Кайы. А если серьезно — то да, тюрки появились на берегах Днепра куда раньше варягов.

\iusr{Андрей Мишин}
\textbf{Владимир Цхведиани} Всё точно. Кий - хазарская крепость, основатель наёмник из Дербента. Так в византийских летописях с описаниями военных укрепления в пограничье. А торговый город вокруг крепости Кий - Самватас ( или Суббота на старом иврите, мифическая последняя река двух пропавших еврейских колен). Все ж основатели матери городов русских - иудеи, а не тюрки Хазарского каганата.  @igg{fbicon.smile} 
\end{itemize} % }

\iusr{Дмитрий Воронин}
Банк такой то... История, древность))))))

\iusr{Татьяна Шеховцова}
Ну, судя по теперешнему названию Киева, не Кий, а Куй.

\iusr{Якубовский Юрий Георгиевич}
Кий, Щекн и Хорив?

\iusr{Andrey Khrenov}
А может это заранее приготовленные погребальные курганы?

\end{itemize} % }
