% vim: keymap=russian-jcukenwin
%%beginhead 
 
%%file 07_07_2021.fb.zadorozhnaja_natalia.1.slova_o_glavnom_v_zhizni
%%parent 07_07_2021
 
%%url https://www.facebook.com/permalink.php?story_fbid=1194381524413546&id=100015251282349
 
%%author_id zadorozhnaja_natalia
%%date 
 
%%tags chelovek,zhizn
%%title Слова о главном в жизни
 
%%endhead 
 
\subsection{Слова о главном в жизни}
\label{sec:07_07_2021.fb.zadorozhnaja_natalia.1.slova_o_glavnom_v_zhizni}
 
\Purl{https://www.facebook.com/permalink.php?story_fbid=1194381524413546&id=100015251282349}
\ifcmt
 author_begin
   author_id zadorozhnaja_natalia
 author_end
\fi

\obeycr
Такое сегодня вышло нечаянно особое настроение.
Слова о главном в жизни - о памяти по ушедшим, о любви к ним и о том, что становится порою неожиданно пусковым моментом ощущений особенных, захлёстывающих внезапно острой тоской вперемешку с любовью.
Воспоминания.
Поминание.
Да. 
Так и работает.
Поминальное приходит именно так - просто - и это слышно остро, собою, будто каждой клеточкой.
Поминальная еда.
Поминальные трапезы и слова поминальных молитв.
У каждого свои. Вне канонов.
Они навсегда в нас растворены, а в такие моменты ещё и рядом. Тогда глотается хоть и с любовью, а тяжело. 
Сглатывается. 
Вместе с любовью и тоской. 
Застревая в сердце и в глотке. 
Царапая их недостатком любви в опустевших без них навсегда жизнях наших.
И простое - оно всегда самое настоящее какое-то: из вкусного, из памяти, из любви состоит. 
\restorecr

\ii{07_07_2021.fb.zadorozhnaja_natalia.1.slova_o_glavnom_v_zhizni.pic}

\obeycr
Из любви наших любимых. 
Кто понимает это при жизни - всё равно досыта любовью их не напьётся. Всегда мало - потому что любимые. А уж после… после, когда номер в памяти есть, номер, который всю жизнь в Донецке отвечал родными голосами, а ответить некому. Нет здесь голосов самых любимых моих больше. В памяти только. А номер есть. Но квартира продана за копейки в 2018 уже, люди в ней чужие. И когда-то зарекались продавать - дом, родной дом считай, не квартира. Для всех.
А война и произошедшее с людьми всё изменило.
И родителей не стало. Вывезти в 2014 в июле успела. 
А … 
и не стало… 
Вначале папы, а год назад мамы.
14 июля маме годовщина первая. 
Учусь жить без них, да не выходит особо. Так, как любят родители, никто больше не умеет. 
Нехватка любви потом.
Прижизненная.
Пожизненная…
•
\restorecr

\ii{07_07_2021.fb.zadorozhnaja_natalia.1.slova_o_glavnom_v_zhizni.pic.bottom}

