% vim: keymap=russian-jcukenwin
%%beginhead 
 
%%file 21_01_2022.yz.tkachev_andrej.protoierej.1.isskustvo.cmt
%%parent 21_01_2022.yz.tkachev_andrej.protoierej.1.isskustvo
 
%%url 
 
%%author_id 
%%date 
 
%%tags 
%%title 
 
%%endhead 
\zzSecCmt

\begin{itemize} % {
\iusr{Александр Шилов}

 @igg{fbicon.thumb.up.yellow}  @igg{fbicon.hands.applause.yellow}{repeat=3} 

О целебном действии классической музыки говорилось давно.

Я думаю, что в отличие от рока или попсы это так и есть.

\begin{itemize} % {
\iusr{Канис Александр}

Александр, Рамштайн чудесно помогает от уныния...
\iusr{Александр Шилов}

Канис, на мой взгляд, от уныния лучше помогут AS/DS или QUEEN.

Но классика, считается более полезной для здоровья.
\iusr{Eugen Nikiforov}

Александр, почему именно классической? Любая музыка в классическом
инструментальном исполнение приобретает очень приятные и лечебные свойства.
Ведь дело в самих звуках и манере исполнения, а какая там мелодия это уже на
вкус и цвет)
\end{itemize} % }

\iusr{Теперь местный}

Нужно, конечно нужно! Но все должно быть разумно, от души и безо всякого,
прости Господи, хайпа!!!

\begin{itemize} % {
\iusr{Eugen Nikiforov}

Теперь местный, вот этого к слову и не хватает, а на поверхности как всегда
всякий мусор плавает, самые золотые залежи где-то на дне океана)
\end{itemize} % }

\iusr{Вадим Алекперов}

Ура! Наконец-то созвучный мне пост! Спасибо!

\iusr{Людмила Калинина}

Тема сродни вопросам - нужен ли человеку воздух? Или вода? И простые ответы
-смотря каким воздухом дышать и какую воду пить... Так и искусство.

\iusr{Иван Иванов}

Младшему внуку четыре года... Когда везу его в машине, всё время говорит мне -
дед, пой! Пой, как в храме! Еду и пою, а он сидит слушает, и о чём то думает...
Ребёнка не обманешь. И ещё. Я руководитель русского народного хора...
Собираются мои артисты на репетицию. Приходит одна женщина, устала - шла через
метель... Говорит: \enquote{Иду и думаю, - за хлебом не пошла бы!} Вот и рассуждайте -
нужно ли нам искусство?

\iusr{Анна М.}

Подскажите, пожалуйста, где простому народу послушать по радио классическую
музыку? Есть ли такое радио и на каких широтах? Интернет не предлагать, он
слабый, на диски денег нет, на филармонию - нет и денег и времени.

\iusr{Alex Solace}

Искусство дано нам для того, чтобы мы не умерли от правды (мира сего).

Правда Божия дана нам для того, чтобы мы не умерли от искусства (человеческого,
слишком человеческого).

\iusr{Александр Андреев}

Исскуство это очень сильная вещь и обладает очень большим воздействием на
людей. Оно увлекает их за собой. Композиторы, поэты, скульпторы, художники,
драматургии, хореографы и прочие люди творческих мрофессий. должны понимать
какую ответственность они несут и какое влияние их произведения оказывают на
человека. Композитор может написать прекрасную музыку, на неё ложаться странные
стихи и исполняет песню Шура. Или например песня \enquote{Секшн революшн} где хорошая
музыка внедряет в голову наивной молодёжи мерзкие мысли. Нужно понимать, для
чего будет использоваться, твоё творение. И чем гениальный твоё творение, тем
большее воздействие оно окажет на людей и нужно приложить все усилия, чтобы
влияние было не со знаком \enquote{-}. Часто человеку даётся дар Свыше и он
злоупотребляя им неожиданно для себя его лишается. Но \enquote{дар} от демонов не
исчезает, он овдадевает человеком и от этой страсти трудно избавиться. Культура
это инструмент и очень важно в чьих руках он находится.

\iusr{Билли Миллиган}

риторические вопросы, не нуждающиеся в ответах.

Нужен человечеству шоколадный торт или не нужен? вот в чем вопрос! он и
красивый и вкусный и радует и прямо уже под разряд прикладного искусства
попадает. А какой вредный! и для сосудов и для печени. А настроение как
поднимает?

так нужен шоколадный торт или не нужен?))) Ю.С.

Билли Миллиган, торт желанен, но не нужен ))

\iusr{Алла Тарасова}

Автор? Вы от чьего имени просите перевести сумму? От своего? Так укажите своё
имя, а не имя известного уважаемого проповедника.

\iusr{Надежда Панкова}

Все виды адекватных и искусств человеку необходимы.
\iusr{Людмила хо}

Уберите эту рекламу дуратскую, это ж не сайт о еде???? Раздражает прост
\iusr{Людмила хо}

Конечно нужно, с этим душа проявляется. Вот только препятствий много. Шурпа эта
дуратская тут зачем???? Все мысли и чувства перебивает.

\iusr{ALEX Kо}

Странный текст... Автор не знает и сомневается? Если конечно речь про
искусство, а не про его суррогаты.

Или этот пост просто чтобы что нибудь написать....

\iusr{R}

Нужно.. стих прекрасный. Все в меру!
\iusr{Вонифатий Кандид}

Искусство нужно, а вот Софрино казённо-сусальное нет.
\iusr{Василий Веточкин}

Религия как алкоголь задурманивает разум и как Сусанин ведёт в пропасть. Люди
будте бдительны, и не доверяйте шарлатанам.

\iusr{Сергей}

Василий, а есть другие варианты ? Вы чем задурманиваете свой разум?

\iusr{Andrey Maksimov}

Как сказал некто из известных, прежде чем дискутировать, давайте определимся с
терминами...  @igg{fbicon.wink}  Сегодня искусством называют почти всё, что \enquote{выливается} в
видео, текстах, на подмостках, полотнах, скульптуре... Я воспитывался в СССР и
для меня часто непонятно, как \enquote{это} может быть искусством. Если в \enquote{поэзии}
используется нецензурная лексика, если коверкаются слова и ударения,
используется музыка, ввергающая человека в депрессию; если на сцене вовсю
эксплуатируется эротический контент, призванный хоть как-то прикрыть отсутствие
смысла в песне и удержать внимание публики; если в театре и кино допускают
использование шок-контента, граничащего с порно и т.п., насилие; на полотнах -
мазня, которую легко рисуют дети в процессе познания мира и никто не признаёт
это искусством, пока какой-нибудь известный художник не сотворит подобное,
тогда здесь - искусство, в котором пытаются найти смысл... тогда лично мне
такое искусство не нужно, и я не хотел бы, чтобы мои потомки росли на нём!

Но если текст, видеоряд, постановка на сцене... вызывает трепет души, слёзы
сопереживания (радости ли, горя ли), заставляет задуматься о смысле жизни,
стать лучше - тогда это действительно искусство! Такого бы побольше в наше
время! Но, к сожалению, можно видеть обратную картину, когда в погоне за
прибылью, на потребу обществу выпускается с конвейера поток психологического и
эмоционального дерьма, отупляющего и оболванивающего народ, прошу прощения за
экспрессию... @igg{fbicon.face.kissing.closed.eyes} 

\iusr{Татьяна Гончарова}

Только не надо в руководители ставить таких деятелей, как Шнуров. Тогда и моргенштерны не появятся.

\iusr{Алексей Покровский}

Нам - нужно. А руководству страны нашей это как нож в брюхо.

\iusr{Сергей Колесов}

Видел Владимирскую Богоматерь в Третьяковке - еще тогда, когда она просто
висела в зале. Был поражен мучением иконы - такая национальная святыня должна
быть с почетом в храме. Хорошо, что потом это поняли. А храм Покрова на Нерли -
неземная красота и под стать ему окружение - идешь к нему через поля и вот он
стоит он на берегу старицы и отражается в ее воде. Даже когда еду в поезде в
Москву - всегда его высматриваю. Хоть издали посмотреть как мелькнет его
маковка.

\end{itemize} % }
