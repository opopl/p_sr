% vim: keymap=russian-jcukenwin
%%beginhead 
 
%%file 05_11_2021.fb.fb_group.story_kiev_ua.6.kiev_gorod_foto.cmt
%%parent 05_11_2021.fb.fb_group.story_kiev_ua.6.kiev_gorod_foto
 
%%url 
 
%%author_id 
%%date 
 
%%tags 
%%title 
 
%%endhead 
\subsubsection{Коментарі}

\begin{itemize} % {
\iusr{Елена Сидоренко}
Того же мнения! Руки прочь от памятников архитектуры, старинных домов, а тех, кто покушается на Подол, вообще судить надо!

\iusr{Александр Асатуров}
Тыж каперанг! Шо за ио?

\begin{itemize} % {
\iusr{Петр Кузьменко}
Александр! Дак низя. Саня! Хоть ты мне друг, но правила превыше! @igg{fbicon.face.wink.tongue} 
\end{itemize} % }

\iusr{Günther Schroeder}

я собственно не понимаю проблему, тут раздутую. на фото точно Киев. Вопрос
закрыт. Все мнения - только мнения))))))

\begin{itemize} % {
\iusr{Анна Анна}
\textbf{Günther Schroeder} и не поймете

\begin{itemize} % {
\iusr{Günther Schroeder}
\textbf{Anna Vrublevska} ну вы же мне объясните, не правда ли?

\iusr{Анна Анна}
\textbf{Günther Schroeder} зачем?) здесь или понял, или нет....

\iusr{Günther Schroeder}
\textbf{Anna Vrublevska} ну ваше мнение тоже, осталось только вашим мнением)))
\end{itemize} % }

\end{itemize} % }

\iusr{Ольга Дебда}

Здесь Киев отвратительно похож на Нью-Йорк, как его изображали на суперобложках
некоторых книг издания 90-х...(((( Я не люблю такой Киев... ☹️ Хоть тут и не
поспоришь - фото само по себе яркое.... Но нету в нем души и духа Города...

\begin{itemize} % {
\iusr{Gary Sorokin}
\textbf{Olga Debda} Нью йорк столиці мира, и такого уродства не наблблюдается

\begin{itemize} % {
\iusr{Ольга Дебда}
\textbf{Gary Sorokin} к сожалению, я там не была @igg{fbicon.shrug}  но написала - что напоминает суперобложки книг, а не фото...)))

\iusr{Gary Sorokin}

\ifcmt
  ig https://scontent-frt3-2.xx.fbcdn.net/v/t39.30808-6/246753869_4464597183619597_8275833142990137959_n.jpg?_nc_cat=101&ccb=1-5&_nc_sid=dbeb18&_nc_ohc=XQLK9ahzBPIAX9Z_oTH&_nc_ht=scontent-frt3-2.xx&oh=8425d453d56c4f5253d27918e50080fc&oe=618C2A18
  @width 0.4
\fi

\iusr{Ольга Дебда}
\textbf{Gary Sorokin} благодарю)))

\iusr{Наталья Домбровская}
\textbf{Gary Sorokin} кому то ближе Нью Йорк, кому то старый , не тронутый « цивилизацией», уютный, тёплый и родной город

\iusr{Gary Sorokin}
\textbf{Наталья Домбровская} в нй если что-то строят и находят куски пола или старую стену то пытаются вставить это в интерьер нового здания

\ifcmt
  ig https://scontent-frt3-2.xx.fbcdn.net/v/t39.30808-6/252886844_4464783653600950_3035489431477329544_n.jpg?_nc_cat=103&ccb=1-5&_nc_sid=dbeb18&_nc_ohc=RU5vsAM0XtMAX8ejQ-W&_nc_oc=AQkxCYYIplJCj7YDLM3P76E183pSu6xesp80hYpUEho1k0w5z5LaVxEG-uBsnM6Z7Mw&_nc_ht=scontent-frt3-2.xx&oh=60cecfb812d4f223ff1faf897d4f0660&oe=618C8FF5
  @width 0.4
\fi

\end{itemize} % }

\iusr{Леся Сагайдачная}
\textbf{Ольга Дебда} А я НЙ очень люблю и он иной

\begin{itemize} % {
\iusr{Ольга Дебда}
\textbf{Леся Сагайдачная} 

он молодой и имеет право быть эпатажным.... Тут суть поста, как я поняла в
сохранении былого - а в Киеве все только рушат.... ((( возьмём хотя бы тот же
Андреевский, Десятинную, аллею художников, Воздвиженскую, Гончарную и тд....
Почему рушат, в угоду новостроям, которые совсем не вписываются в атмосферу и
архитектуру... Где дом, в котором родился Булгаков???? Нет его.... Уже... Хотя
в 98, когда нас в 11 классе водили на экскурсию - был ещё жив, не авариен и
вполне бодр...

\iusr{Леся Сагайдачная}
\textbf{Ольга Дебда} он уже в 98 был не очень.

\iusr{Ольга Дебда}
\textbf{Леся Сагайдачная} но был же)))

\iusr{Леся Сагайдачная}
\textbf{Ольга Дебда} 

Ну тут проблема комплексная, просто откровенно говоря я не вижу сейчас её
решения. Вот совсем. Это к сожалению. Последний норм мэр был Косаковский, его
бодро ушли. Ну вот и всё.

\iusr{Liliya Golub} \textbf{Леся Сагайдачная} 

можно было сделать реконструкцию, как это сделали бы в цивилизованной стране, а
у нас у руля стоят подлецы и дикари, у которых в глазах только \$\$\$\$\$, к
сожалению.

\end{itemize} % }

\end{itemize} % }

\iusr{Sara Shpilman}
А за спиной фотографа дом, в котором выросла Голда Меир...

\begin{itemize} % {
\iusr{Нинел Шаевич}
\textbf{Sara Shpilman} прекратите эти байки. Выросла она в Америке. Да, ребенком жила в Киеве.

\begin{itemize} % {
\iusr{Sara Shpilman}
\textbf{Нинел Шаевич} почему байки? Ей было восемь, когда они покинули Киев. Возможно мне следовало написать «росла». Биография этой женщины в открытом доступе, и желающие вполне могут уточнить детали.

\iusr{Ольга Чекрыгина}
\textbf{Sara Shpilman} ну между Киевом и Америкой семья Голды ещё жила в Пинске...

\iusr{Sara Shpilman}
\textbf{Ольга Чекрыгина} да. Родители Голды познакомились в Пинске, ее бабушки и дедушки жили там, и там они прятались от погромов. У Голды богатая на переезды биография. Но родилась она всё-таки в Киеве. И на доме, который с ней тесно связан, ещё цела мемориальная доска.
\end{itemize} % }

\end{itemize} % }

\iusr{Lara Ilich}

Бассейная... - преображённая, современная. Всю жизнь - несколько десятилетий
тут живу и вижу, как меняется. Но меня не раздражает - что-то уходит, умирает,
что-то появляется. Фотография отличная

\begin{itemize} % {
\iusr{Olha Kuharets}
\textbf{Лара Іліч} , есть архитектура старого города, есть новое,, но когда все в перемешку, без четкой транспортной схемы с нарушением всех градостроительных норм, это не красота, а медленная смерть

\iusr{Наташа Гресько}
\textbf{Лара Іліч} а я ненавижу Бассейную, стараюсь обойти десятой дорогой. и папа так делал - хоть не застал еще всех чудных преображений. наша семья оттуда, с прапрабабушек.

\begin{itemize} % {
\iusr{Lara Ilich}
\textbf{Наташа Гресько} ну знаете, я 60 лет тут живу. и что, всё время ненавидеть? так жить не умею...

\iusr{Наташа Гресько}
\textbf{Лара Іліч} вы там живете и иначе воспринимаете.

\iusr{Lara Ilich}
\textbf{Наташа Гресько} никогда в Киеве не было гармоничной архитектуры. сплошная эклектика стилей. На островке между Мечникова и Бассейной был дом Терещенко (приют для бездомных), Потом - роддом с поликлиникой. Там работала мама моей одноклассницы - весь класс у неё зубы лечил. В 90-х его снесли. Так что, плакать?

\ifcmt
  ig https://scontent-frt3-1.xx.fbcdn.net/v/t1.6435-9/253859562_4565426323500864_794901069282545496_n.jpg?_nc_cat=102&ccb=1-5&_nc_sid=dbeb18&_nc_ohc=yMyqW7dxH9IAX-bToA0&_nc_ht=scontent-frt3-1.xx&oh=298d9b0e1a720a0c08dfa9bb391a9845&oe=61AD3416
  @width 0.4
\fi

\iusr{Наташа Гресько}
\textbf{Лара Іліч} 

не стану спорить по поводу эклектики) но Бессарабку изуродовали знатно: бульвар
исчез, да вообще у нее лица теперь нет. так, проходное место.

\iusr{Нонна Юшкова}
\textbf{Lara Ilich} плакать - не плакать, но разве можно сравнить бульвар, который был, с тем, что сейчас

\end{itemize} % }

\iusr{Liudmila Donchevskaja}
\textbf{Lara Ilich} Да, уж Бассейная была отвратительно задрипанная ,сейчас намного лучше

\iusr{Lara Ilich}

близость рынка всегда придавало свой колорит этой улице. Причём, когда
Бессарабка была более живой и демократичной - я всегда пробегала мимо неё с
содроганием, а это длилось на протяжении лет 30 ежедневно. Ведь даже
легендарный Парижский рынок "Чрево Парижа" снесли в 70-е. В том числе и из-за
крыс. А ведь от них не могли избавиться в больнице на углу. Как и во двориках и
домах на Бассейной. Нет его, прежнего мира, уходит - иногда резко, иногда
постепенно.

\end{itemize} % }

\iusr{Oksana Kozakova}

Согласна на сто процентов. Исторический центр Киева уродовали и продолжают
уродовать этими монстрами. Действительно, европейские столицы обязательно
сохраняют здания центра, реставрируют их, они приносят городу доход от
туристов. А здесь просто распродажа с рук домов, земель под застройку этими
уродами. Очень и очень жаль Киев.

\begin{itemize} % {
\iusr{Владимир Кабасин}
\textbf{Oksana Kozakova}, а вы в упор не хотите замечать, что уродство на фото не современные небоскрёбы, а жлобские балконы в частных квартирах?

\iusr{Mykhaylo Dym}
\textbf{Oksana Kozakova} А у нас в центре после 20-90х осталось что сохранять, особенно на Крещатике? Или вы советские новострои приравниваете к историческим ценностям?
\end{itemize} % }

\iusr{Тамара Петрова}
Как Вы правы!

\iusr{Леся Сагайдачная}

Ну такое, чего уже стонать? Поздно. Да и последний киевлянин мэр ну или
градоначальник был если мне не изменяет память в середине 19 века.

\begin{itemize} % {
\iusr{Нонна Юшкова}
\textbf{Леся Сагайдачная} Косаковский

\begin{itemize} % {
\iusr{Леся Сагайдачная}
\textbf{Нонна Юшкова} а вот я не помню киевлянин он или нет. Но вот такого бардака при нём не было. Потому его и сменили на Омелю. И очень некрасиво. Я тогда как раз в мэрии работала

\iusr{Нонна Юшкова}
\textbf{Леся Сагайдачная} киевлянин и отстаивал север, который сейчас под Парусом

\iusr{Нонна Юшкова}
Сквер т9

\iusr{Леся Сагайдачная}
\textbf{Нонна Юшкова} я помню, что он с Кучмой зацепился, с кем-то из его людей. Ну его и сняли с поста главы КМДА. И разделили эти должности. А сняли так как сейчас пытаются Виталика убрать. Уголовные дела на замов завели. Вот. Это я хорошо помню. К нам приходили. Печальная история
\end{itemize} % }

\end{itemize} % }

\iusr{Виктория Зайцева}

Я смотрю на фото Парижа и вижу, как там берегут исторический центр. У нас
исторический центр Киева изуродован.

\begin{itemize} % {
\iusr{Ольга Дебда}
\textbf{Viktoria Zaytseva} давайте вспомним ещё, как минимум ближайших соседей - Польшу, Чехию и Венгрию.... Мне после каждого приезда домой хотелось плакать, глядя на центр Киева(((

\begin{itemize} % {
\iusr{Ilya Gutnik}
\textbf{Olga Debda} Вы были в Гданьске? Поляки отреставрировали 500-летние дома в центре города. Выглядит центр великолепно. Приятно погулять, провести время.

\iusr{Irina Vaiyman}
\textbf{Ilya Gutnik} вот это верно!
\end{itemize} % }

\iusr{Irina Vaiyman}
\textbf{Viktoria Zaytseva} 

вы забыли, что в 19 веке снесли практически весь средневековый Париж и
построили уже такие знаменитые парижские бульвары ( потрясающее , на мой
взгляд, однообразие, но это на Мой взгляд). Хотя хотелось бы, чтобы старый
Киев, Киев моего детства, сохранился

\begin{itemize} % {
\iusr{Виктория Зайцева}
\textbf{Irina Vaiyman} Ну в Киеве ничего средневекового тоже не сохранилось. Вот хотя бы 19 век уберечь, тоже проблема.

\iusr{Irina Vaiyman}
\textbf{Viktoria Zaytseva} да я к тому, что и Европу не всегда можно за образец принимать
\end{itemize} % }

\iusr{Владимир Баваровский}
\textbf{Виктория Зайцева}, особенно стеклянная пирамида перед Лувром) Знали бы Вы, что по этому поводу говорят парижане)) Берегут исторический центр.... Впрочем когда-то они очень возмущались "громоотводом", а теперь символ)

P. S. Ни в коем случае не защищаю современную уродскую застройку Киева, но про
Париж не очень сильное сравнение.

\begin{itemize} % {
\iusr{Alexey Paschenko}
\textbf{Владимир Баваровский} Да и Эйфелева башня в своё время не вписывалась в архитектуру Парижа и была сильно критикуем а современниками

\iusr{Владимир Баваровский}
\textbf{Alexey Paschenko}, да, это я её громоотводом и назвал @igg{fbicon.smile} 
\end{itemize} % }

\iusr{Елена Махова}

Исторический центр берегут не только в Париже, но и во многих других
европейских городах. Я лично видела центр Бордо, Малаги, Брюгге.

Малага- маленький городок, но какой уютньій и удобньій. В центре местньіх очень
мало, одни туристьі со всего мира. И почему-то современньіе ресторанчики и
бутики не вьітеснили старьіе здания. В старьіх зданиях размещеньі галереи и
музеи. В общем, красота. Я бьіла в Малаге 3 дня еще до пандемии, и очень хочу
туда вернуться и провести там хотя бьі дней 10.

\iusr{Alexey Paschenko}
\textbf{Виктория Зайцева} А Эйфелева башня?

\end{itemize} % }

\iusr{Oksana Shevchuk}

Не узнала! А ведь я жила в 70х на Бассейной 11. Это был новый ( в то время) дом
с загс ом на первом этаже

\begin{itemize} % {
\iusr{Нинел Шаевич}
\textbf{Oksana Shevchuk} согласитесь, не самый красивый дом на на Бесарабке, втиснуты между домами старой постройки. А для вас он родной и близкий,

\begin{itemize} % {
\iusr{Oksana Shevchuk}
\textbf{Shayevich Ninel}
Абсолютно согласна! Но тогда он мне казался оригинальным. К тому же квартиры были лучше и большего размера чем обычные.
\end{itemize} % }

\end{itemize} % }

\iusr{Olga Lubos}
Фото под моим домом. 11 номер улицы Бассейная

\iusr{Valentina Urban}

Абсолютно не узнала ул. Бассейную, Киев очень преобразился. Мой папа работал в
50-60 гг. начальником отдела реставрации Киевской ювелирной фабрики
расположенной на углу ул. Бассейнов и ул. Шота Руставели. Мой отец страдал
заболеваниями язв желудка и двенадцатиперстной кишки, и мы с мамой каждый день
приносили ему приготовленный свежий обед, поэтому прекрасно помню эту и
близлежащие улицы.

\iusr{Лариса Лобановська}

Вверх по Леси за Гулливером, и напротив Гулливера, шикарные дома советской
постройки. А вот они Вам Киев не уродуют? И только когда среди них, этих 5-9
этажных убожеств построили Гулливер, Вас это возмутило?

\begin{itemize} % {
\iusr{Леся Сагайдачная}
\textbf{Лариса Лобановська} вот да

\iusr{Петр Кузьменко}
\textbf{Лариса Лобановська} разве в советские времена люди что-то решали? Впрочем как и сейчас. Но сейчас, хоть говорить об этом можно...

\begin{itemize} % {
\iusr{Лариса Лобановська}
\textbf{Петр Кузьменко} 

вот именно, стало можно говорить. Кроме того, подозреваю, что для большинства
недовольных ТОТ Киев, с хрущевками, 9-этажками, бетонными Дворцами спорта и тд,
и был настоящим, родным и который нельзя менять. И признать, что он был
уродлив, сложно, тк он родной и это город Вашей молодости. А моя бабушка
терпеть не могла послевоенный Крещатик, говорила, испортили Киев...

% -------------------------------------
\ii{fbauth.kit_gross.berlin.germania.kiev}
% -------------------------------------

\textbf{Петр Кузьменко} 

и раньше можно было говорить. ФБ дал возможность говорить селюкам, на которых в
другой обстановке никто не обратил бы внимания. Это разные вещи.

\iusr{Kit Gross}
\textbf{Лариса Лобановська} 

красота не в хрущевках. В них была просто срочная необходимость. Но тогда
строили для людей, сейчас для денег. В этом большая разница.

\iusr{Света Медецкая}
\textbf{Kit Gross} в точку!!!)))

\iusr{Петр Кузьменко}
\textbf{Kit Gross} 

я не о возможности говорить на страницах Интернета, хотя и её лишены соседи
Украины, мною имелась ввиду возможность высказывать претензии власти в лицо. А
за "селюка" спасибо.

\iusr{Петр Кузьменко}
\textbf{Лариса Лобановська} 

на Бассейной дома дореволюционной постройки. Были... А на моём Подле и в
историческом центре сжигают и доводят до развала совсем не хрущёвки, а
памятники архитектуры. В угоду постройке уродливых монстров. @igg{fbicon.face.sad.but.relieved} 

\iusr{Kit Gross}
\textbf{Петр Кузьменко} 

но вы же не про власть писали. А скахать можно было и тогда и сейчас, но тогда
сейчас толку в этом было ноль. Только тогда люди умнее были на фигню не
распылялись. А теперь одно хамство и все уверены в своей правоте. А кому из
соседей не дают высказаться в интернете? Или вы о нашей пропаганде, которая так
народу говорит? Но говорить и факты это же две большие разницы. А вот вам
кажется это одним и тем же. В этом действительео сейчас стало по-другому.
Многим кажется раз сказал, то прав. Хотя именно селюки так и раньше считали.
Только ьнз интернета их видно было, а сейчас можно маскироваться. Хотя
нормальным их и в интернете сразу видно.

\iusr{Петр Кузьменко}
\textbf{Kit Gross} реальные сроки за посты и репосты в Интернете получили несколько граждан России и Беларуси. Загуглите.

\iusr{Kit Gross}
\textbf{Петр Кузьменко} А также в Австралии и США. А Асанжа уже лет 10 как взааерти держат. Беспредела не должно быть. Все правильно, если они херню писали.

\iusr{Лариса Лобановська}
\textbf{Kit Gross} это когда раньше можно было говорить, в советское время?! Когда, например, посносили в Киеве все церкви?!

\iusr{Kit Gross}
\textbf{Лариса Лобановська} 

всегда. Просто из-зв сказанного всегда есть последствия. Когда человек говорит
он доожен думать, зачем и с какой целью он это говорит. Беспредел не то же
самое, что свобода.

\iusr{Станислав Малюченко}
\textbf{Петр Кузьменко} 

немного оффтоп, но почему, по-вашему мнению, модераторы КИ не банят таких
персонажей как данный ваш собеседник, ведь "селюки" - это классический русский
шовинизм-почтифашизм, но зато оперативно удаляют и банят комменты и людей
другого типа( вам первая часть вопроса)

\iusr{Лариса Лобановська}
\textbf{Kit Gross} 

тогда почему, по Вашему мнению, раньше молчали? Были умнее, у них был
прекрасный вкус, и поэтому им нравилась советская застройка Киева, те же
панельки в историческом центре Киева? Или они просто боялись последствий своих
выступлений, поэтому молчали? Как Вы считаете?

\iusr{Kit Gross}
\textbf{Stanislav Maliuchenko} cелюки это некая замена слова "жлобы". Я понимаю почему многим это может не нравится. А банят и правильно за нападки на людей и хамство, национализм, ксенофобию, и т.д. В вашем комменте очевидная ксенофобия.

\iusr{Kit Gross}
\textbf{Лариса Лобановська} а почему вы считаете что молчали? Или вы просто так думаете? Считались с этим мнением или нет это уже другой вопрос.

\iusr{Лариса Лобановська}
\textbf{Kit Gross} 

значит, мне просто случайно ни разу не попались свидетельс тва каких-то
выступлений, например, против строительства безобразного жилого дома напротив
Оперного, где Центральная, аптека. А Вы о таком слышали? Или Вам нравится этот
дом, уродующий Владимирскую улицу? Вот прямо народ выступал против, а на его
мнение наплевали?

\iusr{Kit Gross}
\textbf{Лариса Лобановська} вот, а вы говорили что молчали. Нет, не молчали. Но на это не реагировали так же как и сейчас.

\iusr{Лариса Лобановська}
\textbf{Kit Gross} я вообще-то спросила, не слышали ли Вы, чтоб кто-то был против этого дома. Я не слышала. И вообще не слышалв, чтоб кто-то против чего-то выступал. Если Вы слышали, то напишите, кто и когда выступал.

\iusr{Kit Gross}
\textbf{Лариса Лобановська} 

я видел и слышал многие выступления против разного, в том числе и в советское
время. Но я не считал нужным собирать эти свидетельства. Зачем? Улицы в Киеве
называют в честь бандеровцев, а не тех, кто что-то делали для страны и Киева.
Но разве то что вы не знаете или не слыхали говорит о том чего не было?
Например, вы знаете что такое фонон или солитон, или что такое бигдата? Если
да, вы молодец, а если нет, то это не значит, что этого нет.

\iusr{Ольга Чекрыгина}
\textbf{Лариса Лобановська} 

по поводу этого дома "выступали" многие. Только раньше это происходило в
курилках и на кухнях, а теперь для этого есть ФБ. А Киев жалко. Пусть бы
строили эту "новую красоту " где-нибудь на новых массивах. И глазу было бы
приятно, и в центре не торчали бы эти одоробла

\iusr{Лариса Лобановська}
\textbf{Kit Gross} 

выступления в курилках - да, открыто - нет. Насчет фонона знаю только, что это
какая-то частица, я не физик, а бигдата.. Извините, я программист... Но - ни
один человек не может знать ВСЕ, но и музыкант, и физик, и рабочий, живущие в
одно время, знают, чем живет общество и о чем гочорят так что Ваши сравнения со
знаниями физики и тп очень хромают. И я, жившая в Киеве при совке, прекрасно
знаю, ЧТО говорили люди и где.

\iusr{Kit Gross}
\textbf{Лариса Лобановська} 

знаете так знаете, но как вы сами сказали не все. Даже на курилку согласились.
А то что в курилках все решения принимались и принимаются вы не знали? А вот
быть Коперником, чтобы тебя сожгли на костре - глупо. Лучше быть Галлилеем,
который против власти не пошел, но вклад в развитие сделал гораздо больший, чем
Коперник который в лицо инквизиции "за правду науки" агитировал. И сейчас
Ассанж сидит и он не в Союзе жил. И в целом я сочувствуб людям жившив в совке.
Я , родившись в Союзе, на тех кто жил в совке смотрю свысока. Жалко их.

\iusr{Лариса Лобановська}
\textbf{Kit Gross} 

мы говорим о разных курилках. Треп инженеров в НИИ не решал ничего. Как и треп
рабочих и крестьян, которые были типа хозяева страны. И прекрасно, что это
время ушло.

\iusr{Kit Gross}
\textbf{Лариса Лобановська} 

у каждого свой уровень. Вы правы на своем, а я на своем. Все развивается,
поэтому ваше сравнение об ушедшем Союзе хорошо для ФБ комментов. А вот если
соавнить Союз каким он мог бы быть сегодня и сегоднчшнюю Украину. То сравнение
вполне могло бы быть не в пользу Украины. Но последнее только сослагательное
наклонение.

\iusr{Лариса Лобановська}
\textbf{Kit Gross} 

так весь наш разговор и есть ФБ каменты. И тауи да, у нас совем ращное
отношение ко всему. Я не смотрю свысока и не жалею живших при совке, но рада,
что его нет. А у Вас как-то все наоборот... Совков жалко, но лучше бы сейчас
был Союз. Ну очень странная логика

\iusr{Kit Gross}
\textbf{Лариса Лобановська} 

нет, это у вас не правильная обработка данных. Я писал другое, вы захотели
услышать свое. Это ваша проблема, а не моя логика. Будьте внимательней к тому
что написано, а не к тому, что вы хотели бы прочитать.

\iusr{Kit Gross}
\textbf{Лариса Лобановська} 

только что увидел и навеяло. Вот советская постройка. Красота... Уничтоженная
независимой Украиной. И я точно знаю, что люди были против и говорили и не
только в курилках. Так что, глядя на эту красоту, напрашивается вывод, что в
Союзе было хорошо. А в совке так же плохо, равно как в независимой Украине.
Одни жили в совке, другие в Союзе. Как и сейчас, селюкам и в сегодняшнем Киеве
хорошо, это же не их хлев. Нет ничего идеального, но если бы не было этого
совковства тогда и этого селюковства сегодня, что одно и то же, то и сейчас все
по-другому было бы. Одни видят всегда полупустой стакан, а для меня он всегда
наполнен на половину и я хочу в него долить и взять еще один. Каждый живет
своей жизнью и на своем уровне. Можно страдать из-за хрущевок, а можно в них
пожить, а не в послевоенных руинах, и двигаться дальше, жить на вершине при
этом сносить хорошо отслужившие свое хрушевки и сторить новое. Правда многих
совков обратно в их бандеровские схроны или сараи тянет, им новое не нужно,
лишь бы у всех разруха была... Вот это действительно достает и мешает. Всегда
так было.

\ifcmt
  ig https://scontent-frt3-1.xx.fbcdn.net/v/t39.30808-6/252279306_10208804079975738_3286266690591894306_n.jpg?_nc_cat=104&ccb=1-5&_nc_sid=dbeb18&_nc_ohc=0UKNdbQvaBkAX_hMkOS&_nc_ht=scontent-frt3-1.xx&oh=4faffd2fce866c0e2ecdc2b211bbc791&oe=618B25BA
  @width 0.4
\fi

\iusr{Valentina Zajfertová}
\textbf{Kit Gross} вас давно забанить надо небандеровец

\iusr{Валентина Тищенко}
\textbf{Kit Gross} 

дядя, сиди мовчки в себе в Берліні, пость інтелектуала Бужанського, а ми тут
розберемося чиїми іменами називати вулиці. І до речі: БАТЬКО НАШ БАНДЕРА ,
УКРАЇНА МАТИ. МИ ЗА УКРАЇНУ БУДЕМ ВОЮВАТИ!!!

\iusr{Люсянка Балашова}
\textbf{Kit Gross} 

бассейн, открытый ,возле стадиона Динамо-две ванны-для плавания, и для
прыжков... работал КРУГЛОГОДИЧНО... все с любовью вспоминают походы в этот
знаменитый, единственный бассейн(под открытым небом )-снесли ,для чего? стоянка
нужна была олигархам

\iusr{Люсянка Балашова}

Убрали этот бассейн, а почему не сделать реконструкцию стадиона Динамо-сделать
крышу, туалеты... ведь в парковой зоне стоит красавец-стадион

\iusr{Валентина Тищенко}
\textbf{Kit Gross} та куди нам до вас з інтелектуалом Бужанським.

\iusr{Kit Gross}
\textbf{Валентина Тищенко} поэтому вас никто и не звал, молчите мимо.

\iusr{Людмила Шевченко}
\textbf{Валентина Тищенко} 

Нешановна, ви не навчені поважати чужу думку. І, як вам подобається Бандера, то
це не означає, що він батько для всіх. Мало про Бандеру знаєте, а висловлюєтеся
так , ніби всі з вами згодні. І взагалі, не розпалюйте нациське полум*я,
нерозумна жінко.

\iusr{Валентина Тищенко}
\textbf{Людмила Шевченко} спочатку вивчіть на зубок, яка різниця між "націоналістичним" і " нациським", потім будете щось варнякати.

\iusr{Kit Gross}
\textbf{Валентина Тищенко} 

читай пункт 3 там четко написано, что это чучело гитлеровский фашистский холуй.
С ним фашиками в частности с бандерюганами воевала вся Украина. На стенку себе
повесь.

\ifcmt
  ig https://scontent-frt3-1.xx.fbcdn.net/v/t39.30808-6/248550617_10208805204243844_1395217481386753924_n.jpg?_nc_cat=108&ccb=1-5&_nc_sid=dbeb18&_nc_ohc=lVtqtQWTiBgAX-IllP3&_nc_ht=scontent-frt3-1.xx&oh=ef8815be31ef9b9dddbbf23e14e9aab4&oe=618BEA09
  @width 0.5
\fi

\iusr{Валентина Тищенко}
\textbf{Kit Gross} ну ну, парад нацистів з червоними забули. І в результаті націоналісти боролися з нацистами і червоними.Так що вішай собі на лобі.

\iusr{Марія Рябоконенко}
\textbf{Kit Gross} По каким критериям вы отличает "нормальных" от "селюков"? Поделитесь.

\iusr{Людмила Шевченко}
\textbf{Валентина Тищенко} 

То я була права, коли думала, що ви таки нерозумна. Вам потрібно навчитися
відрізняти національний рух від націоналістичного. І спілкуватися з нерозумним
створінням, яке здатне тільки на лайку - це не поважати себе. З вами розмова
закінчена, бо дурні нічого не розуміють.

\iusr{Валентина Тищенко}
\textbf{Людмила Шевченко} 

мовчи , за розумну зійдеш. Гугл в допомогу, вивчай історію , ще раз на зубок "
націоналістичний", " нациський".

\end{itemize} % }

\iusr{Нинел Шаевич}
\textbf{Лариса Лобановська} Гулливер в самом деле убожество, но и далее вверх по Леси новострой 60х.

\begin{itemize} % {
\iusr{Лариса Лобановська}
\textbf{Shayevich Ninel} согласна. Одно убожество на фоне другого...
\end{itemize} % }

\iusr{Kit Gross}
\textbf{Valentina Zajfertová} если нечего сказать, переходят на личности. У вас что-то внятное есть? Если нет, то я вас не держу.

\begin{itemize} % {
\iusr{Valentina Zajfertová}
\textbf{Kit Gross} я перешла на личности? Пересмотрите свои комментарии...забываете,что пишете и повторяете и подчеркиваете.вы не оппонент вы провокатор и плевать вам на Киев

\iusr{Kit Gross}
\textbf{Valentina Zajfertová} это не на личности? Повторяю, разумное что-то из вас выйдет или ваше х@мство только будет усугубляться? Если так, повторяю, я вас не держу.

\ifcmt
  ig https://scontent-frx5-1.xx.fbcdn.net/v/t39.30808-6/253727640_10208804739872235_7381167816787187080_n.jpg?_nc_cat=100&ccb=1-5&_nc_sid=dbeb18&_nc_ohc=kQ1638Lg2-MAX-XPKLe&_nc_ht=scontent-frx5-1.xx&oh=123b00c63e534e1730b5864b8ec29741&oe=618AE383
  @width 0.4
\fi

\end{itemize} % }

\iusr{Марианна Носаль}

Именно в этом месте Киева мне нравится, но согласна, что нужно такие здания
строить не в центре. Но тогда не нужно было строить дома в той форме, как
строили в 1970-1980-х , как по мне такие здания более уродуют города, особенно
в наше время

\begin{itemize} % {
\iusr{Олена Харченко}
\textbf{Марианна Носаль} ті будинки вже побудовані давно. І однакові. А це випирає
\end{itemize} % }

\iusr{Viktoria Terpylo}

А мне нравится это фото. И особенно вечерний вариант. Думаю что днём фото могло
выглядеть безликим, а вечером очень даже...

К сожалению невозможно представить современный город без этих высоток. В каждом
городе мира они есть!... вопрос как они впишутся в конъюнктуру города и что бы
не навредить историческим постройкам

\begin{itemize} % {
\iusr{Олена Харченко}
\textbf{Viktoria Terpylo} в тому то й річ, що не вписуються. От на харківському масиві або троєщині чи оболоні було б нормально, а тут жахіття. Особливо, коли згадуєш, як при будівництві людина загинула

\iusr{Tatiana Thoene}
\textbf{Viktoria Terpylo} да, они есть, но не в исторических районах; да и парки там под высотки не вырубают; и дома исторические не сжигают. Вот не найдёте вы уродливого, как на Андреевском спуске, черного куба ни в Риме, ни в Кракове, ни в Стокгольме.
\end{itemize} % }

\end{itemize} % }

\iusr{Лариса Драган}
Погоджуюсь, що решали вже задовбали

\iusr{Нинел Шаевич}
А за "Парусом" мой теперешний Печерск и мои любимые. И я уже не та и мой Печерск давно не тот, каким был в дни моей юности. Грустно @igg{fbicon.heart.broken} 

\begin{itemize} % {
\iusr{Наталия Бахмацкая}
\textbf{Shayevich Ninel} 

Время не стоит на месте... всё меняется и жизнь и быт наш поменялся не надо
жалеть о прошлом потому что у нас есть настоящее и будущее... люблю тебя...

\iusr{Нинел Шаевич}
\textbf{Наталия Бахмацкая} И я тебя @igg{fbicon.rose}  @igg{fbicon.kiss.mark} 
\end{itemize} % }

\iusr{Julia Panchul}
всему своё место. Здесь нет гармонии, тесно, не к месту

\iusr{Ольга Севериновская}
Простите, много раз уже говорил а.., Это не мой город.

\iusr{Светлана Светличная}

Кличко - киевлянин. может не место рождения определяет поведение человека а
наличие совести и любви к стране и людям, киевлянам?!

\begin{itemize} % {
\iusr{Maksim Pestun}
\textbf{Светлана Светличная} родился в Киргизии, учился в Переяславе... но во второй части Вы правы

\begin{itemize} % {
\iusr{Светлана Светличная}
\textbf{Maksim Pestun} его семья из Киева. бабушка растреляна в Киеве. отец военний, поєтому много катались.

\iusr{Maksim Pestun}
\textbf{Светлана Светличная} откуда такие данные? Вы Смелу с Киевом перепутали. И не расстреляна, а спаслась...

\iusr{Светлана Светличная}
\textbf{Maksim Pestun} по матери отца Кличко еврей. он не любит офишировать єтот факт. бабушка била расстреляна в Бабьем Яру немцами. єто я слишала он него в одном интервью

\iusr{Maksim Pestun}
\textbf{Светлана Светличная} кто-то из вас перепутал... все дело происходило в Черкасской области, откуда родом его отец. И расстрелян был дядя. А так все верно...

\iusr{Светлана Светличная}
\textbf{Maksim Pestun} я говорю то, что озвучил сам Кличко в одном интервью с Гордоном, где винужден бил признать ,что тоже имеет еврейские корни. и еще одно интервью, забила канал, где сказал, что его семья пострадала от расьрелов в Бабьем Яре. била растреляна его бабушка. я только озвучиваю то, что говорил сам Кличко.

\iusr{Maksim Pestun}
\textbf{Светлана Светличная} официальная биография приводит другие данные. Евреи и киевляне, хоть и схожие понятия, но не тождественные.

\iusr{Светлана Светличная}
\textbf{Maksim Pestun} я доверяю вашему мнению всецело

\end{itemize} % }

\iusr{Ольга Чекрыгина}
\textbf{Светлана Светличная} может он себя и считает киевлянином, но ведёт себя, как слон в посудой лавке.

\begin{itemize} % {
\iusr{Maksim Pestun}
\textbf{Ольга Чекрыгина} опять же в защиту Виталия. От него мало что зависит. Вернее, не все и не от него... @igg{fbicon.smile} 
\end{itemize} % }

\iusr{Галина Гурьева}

Кличко учился в 69 школе на Донецкой. А киевлянином так и не стал. Наверное, это
на генетическом уровне~любовь к городу или есть, или нет!

\begin{itemize} % {
\iusr{Maksim Pestun}
\textbf{Галина Гурьева} 

даже если и есть любовь к городу, в чем я лично не сомневаюсь, то нет
политической воли и реальной возможности что-то изменить

\iusr{Светлана Светличная}
\textbf{Галина Гурьева} думаю именно генетически он город очень любит. иначе давно уехал жить в Германию .тут другое
\end{itemize} % }

\end{itemize} % }

\iusr{Tatiana Leschenko}

Киев навсегда утратил свою уютную неповторимость! Центр города просто
изуродовали уродливые, бетонно-стеклянные монстры и переделки под старину!
Такое впечатление создаётся, что это делалось и делаеться специально, что бы и
намёка на историческую древность города не осталось! Как буд-то мстят городу,
уродуя Киев! Или может это обычная неуёмная жажда денег и жлобство!? - "И после
нас хоть потоп!"

\begin{itemize} % {
\iusr{Наталья Чулкова}
\textbf{Tatiana Leschenko} Не мстят. Просто им наплевать, лишь бы огрести максимально денег
\end{itemize} % }

\iusr{Людмила Мозговая}
Да... это Киев. Так много в центре свечек настроили...

\iusr{Helen Gorbenko}
Подчеркнуто про Крым, дальше можно не читать. Беспринципность - это бич и отрыжка совдепа. Проходите мимо.

\iusr{Светлана Миргород}

Да. И я ,Киевлянка в 5 поколении, прихожу в ужас от того, что с моим любимым
городом творят временщики то донецкие то ещё какие... Только настоящего
КИЕВЛЯНИНА, мера, я ещё не видела..

Памятники истории, старинные дома снесли, предварительно устроив поджег...
Возле святынь Лавры понастроили дерьма..

В заповедных зонах отморозки из криминальных кругов, " Сильные мира сего ",
теперь депутаты, понастроили псевдодворцов а-ля Пшонка...

\begin{itemize} % {
\iusr{Alexey Paschenko}
\textbf{Светлана Миргород} ни Бибиков, ни Фундуклей коренными киевлянами не были. А стали настоящими Киевлянами, с большой буквы.
\end{itemize} % }

\iusr{Екатерина маковецкая}
как же я с Вами согласна!!!! спасибо! я, кстати тоже Киев-Крым))))

\iusr{Георгий Готесман}
Какой Идиот превратил ресторан "Театральный" в салон "Астон-Мартин"?!

\iusr{Владимир Картавенко}

На месте Историко Мемориального комплеса "Памятник давнеруской Киевской гавани"
випускники политсучилища предложили, а Киевсовет утвердил проект застройки
места крещения Руси! ВОЮЕМ.

\ifcmt
  ig https://scontent-frt3-1.xx.fbcdn.net/v/t1.6435-9/252755723_6408487145889195_4381985815705930593_n.jpg?_nc_cat=106&ccb=1-5&_nc_sid=dbeb18&_nc_ohc=zj4B1396LyMAX-FTLMp&_nc_ht=scontent-frt3-1.xx&oh=f1d1f2f08ae2d5b99ea7b80e72194f4a&oe=61AB815D
  @width 0.3
\fi

\begin{itemize} % {
\iusr{Нинел Шаевич}
\textbf{Владимир Картавенко} даже зданием нельзя назвать, монстр @igg{fbicon.face.screaming.in.fear} 

\iusr{Владимир Картавенко}
\textbf{Shayevich Ninel} Админисуативно-офисний комплекс ПОБЕДА на базе российской субмарини СС-310 в акватории ГУР МО Украини!

\iusr{Тина Шевченко}
\textbf{Владимир Картавенко} Владимир!!! ГорДюсь Вами!!! Правда, без шуток!!!! Таких бы, как Вы, активных боевых за правду "старичков" да побольше!!!!!!!!!!!!

\iusr{Нинел Шаевич}
\textbf{Тина Шевченко} Прекрасный возраст, есть опыт,и навыки и знания.

\iusr{Владимир Картавенко}
\textbf{Shayevich Ninel} послушайте сегодня Юрия Бутусова.

\end{itemize} % }

\iusr{Наталия Водяницкая}

За спиной у фотографа мой дом (ул. Бассейная, 11), где я прожила 10 счастливых
школьных лет (потом разменяли квартиру). Окна выходили на улицу, 5 этаж. Мне до
сих пор снятся сны (раз в году) с видом на летнюю, солнечную улицу - очень
уютную, несмотря на дорогу, с зеленым сквериком посередине!... Полностью
согласна с автором. Нельзя было строить такие дома в центре - хватает
новостроек вокруг Киева. Какая красивая, зелёная и уютная была ул. Бассейная!..
И - любимая школа 78!.. @igg{fbicon.heart.red}

\begin{itemize} % {
\iusr{Inna Zalutsky}
\textbf{Наталия Водяницкая} , Школа была видна из моего двора , Красноармейской 5  @igg{fbicon.monkey.see.no.evil}  @igg{fbicon.face.tears.of.joy} 

\iusr{Наталия Водяницкая}
\textbf{Inna Zalutsky} из комнаты брата (в торце дома, квартиры) тоже была видна моя школа, конечная остановка троллейбусов 14,15 и Бессарабский рынок. На 1-м этаже нашего дома был ЗАГС.  @igg{fbicon.smile} @igg{fbicon.heart.red}

\iusr{Татьяна Рязанцева}
\textbf{Наталия Водяницкая} Бессарабка и шк.78, детство и юность.
Был самый уютный и родной район, сейчас он самый изуродованный, уничтоженый и самый некомфортный

\iusr{Наталия Водяницкая}
\textbf{Татьяна Рязанцева} мой тоже - самый уютный, родной и любимый район! Знала все дворы, проходные подъезды, чердаки, крыши... Столько изучено, избегано в детстве! @igg{fbicon.smile} 
\end{itemize} % }

\iusr{Тина Шевченко}

Мне тоже больно.. Но, как можем, воюем против безпредела с другими активистами
массива и города за сохранения парков, против безпредела злобудовників. Мы не
возмущаемся, мы боремся реально.. Возмущение и ностальгирование результатов не
дает!!!

Третьего октября этого года от ВЗрады до Майдана прошел "Марш за Киев"!!! В
многомиллионном городе нас собралось всего 3 500 реально действующих
активистов!!! Как вы думаете, такое кол-во людей может остановить
безпредел?!?!?! Где остальные неравнодушные киевляне и мешканці?!?!?!?!?!

\iusr{Анна Власова}

Согласна на все 100\% , наш Киев изуродовали, как могли. К сожалению, и
продолжают это делать .

\iusr{Нина Сухецкая}

Это пример уродования города. Улица Старонаводницкая. Ужасная коробка возле
Родины - матери, возле кладбища. Ужас! А центр города во что превратили!

\ifcmt
  ig https://scontent-frx5-1.xx.fbcdn.net/v/t39.30808-6/254636588_3026854580918004_1661489851723600498_n.jpg?_nc_cat=105&ccb=1-5&_nc_sid=dbeb18&_nc_ohc=Z17w0svem1oAX_QXkns&_nc_ht=scontent-frx5-1.xx&oh=a3ebb8618ad53d55afd219633b768ca3&oe=618C55FC
  @width 0.4
\fi

\iusr{Татьяна Сирота}

Фото красивое!
Но... это место, как по мне, стало неуютным.
О прежнем бульваре на Бассейной можно теперь только вспоминать и сожалеть.
Эти "громады" давят.
Такая архитектура хороша для новых районов, а здесь она чужая, притянута "за уши".

\iusr{Андрей Аникеев}

Все началось с Александра Омельченко, с "великого градоначальника" . Начавшего
своё преобразования Киева с укладки тротуарной плитки на Крещатике , разрушил
площадь и построил антенну на площади... за тем , вместо парка , который должен
был быть у Днепра , застроили всю набережную домами и даже особняками . И,
назвали это -Оболонской набережной. Тем самым Он показал , как зарабатывать
деньги на строительстве... а сейчас просто зарабатываются деньги , застраивая
каждый пустырь , уничтожая скверы , парки , каждый закуток... во что превратили
центр города , Ул. Владимирскую , Подол... города с много-вековой историей
развиваются в ширь , за счет окраин . Киев застраивается в верх и глубину , тем
самым нарушал и уничтожая гармонии в Киевской архитектуре ,Киевский колорит ,
от которого скоро мало-что останется !!!!

\begin{itemize} % {
\iusr{Анна Борисенко}
\textbf{Андрей Аникеев} да, сколько ходила по этой плитке на каблуках, столько его матюкала. А потом узнала, что у сына Омельченко завод по изготовлению тротуарной плитки.
\end{itemize} % }

\iusr{Liliya Golub}
Да, к сожалению, это так. Сердце кровью обливается, когда смотришь, как уничтожают исторический центр.

\iusr{Ірина Дебкалюк}
Це у минулому корпуси заводу « Більшовик» промзона! Що тут треба будо зберігати?

\iusr{Ольга Малевская}
На 100\% поддерживаю автора!

\iusr{Alexey Paschenko}

Не трогайте Киева...

Дайте ему быть таким, каким его создала история. Направьте вашу ревность на
созидание, а не на разрушение.

Рядом со старым воздвигайте новое, хотя бы и в архитектурном стиле. И если
новое будет лучше, как оно и должно быть, старое само собой склонит перед ним
свою седую голову.

(из книги Василия Шульгина "Три столицы")

\iusr{Irina Vynogradova}

Странный пост. выпускник военно-политического училища (живущий в Саках) ничего
хорошего в Киеве и не увидит @igg{fbicon.wink} . То-то дело запах детства - дворовых туалетов и
обязательная продлёнка 100й школы @igg{fbicon.wink} 

\begin{itemize} % {
\iusr{Петр Кузьменко}
\textbf{Irina Vynogradova} мне Вас жаль...
\end{itemize} % }

\iusr{Nataliya Tkachuk}
Так поступают почти все европейские столицы

\iusr{Nataliya Tkachuk}
Забавно, почему брежневки и хрущевки, втиснутые в историческую застройку, не
вызывают такого потока эмоций. Привыкли? Глаз замылился?

\begin{itemize} % {
\iusr{Петр Кузьменко}
\textbf{Nataliya Tkachuk} об этом писали и спорили. Смотрите выше.

\iusr{Nataliya Tkachuk}
\textbf{Петр Кузьменко} ну нет, тратить время на такое

\iusr{Лариса Лобановська}
\textbf{Nataliya Tkachuk} потому как возмушающиеся нынешним беспределом (а он имеет место быть, увы), выросли среди этих хрущевок, бетонных коробок всяких институтов-заводов и тп, поэтому они милы их сердцу и входят в понятие "мой старый милый Киев".

\iusr{Nataliya Tkachuk}
\textbf{Лариса Лобановська} ну я там же выросла, но ностальгии к бетонным коробкам не испытываю

\iusr{Лариса Салий}
\textbf{Петр Кузьменко} берегите сердце...

\iusr{Петр Кузьменко}
\textbf{Лариса Салий} Благодарю!

\iusr{Іллона Зейкан}
Так ось, звідки ноги ростуть!!!!!!!!!!!!!!!!!!!!!!!! КРИМ!!!!!!!!!!!!!!! Тепер зрозумілі вибачення за українську! Ганьба!

\iusr{Петр Кузьменко}
\textbf{Іллона Зейкан} дивіться ширше. Мені шкода Вас.

\iusr{Егор Красава Тебякин Егор}
\textbf{Петр Кузьменко} 

Мне их тоже жаль . Оболваненные и зомбированные неандертальской селючьей
идеологией национал-идиотов, у которых три класса образование с длинным
коридором.

И что может расказать это заезжее быдло коренным киевлянам.

Да ничего. Кроме того что кумир их батько Степан Бандера пуп земли.

Поэтому не обращайте на них внимания, они продолжают жить в своем селе.

\iusr{Sunny Smith}
\textbf{Іллона Зейкан} 

а що не так? Людина не може висловитись бо хтось з телевізора вам сказав, що ці
люди погані?  @igg{fbicon.man.facepalming}  Думайте власним розумом

\end{itemize} % }

\iusr{Таня Терещенко}
Привет с Большой Житомирской, 12.
Согласна полностью  @igg{fbicon.hands.pray} 

\iusr{Татьяна Васильевна Зубко Маркина}

Написано правдиво. Живете в Саках, привет всем. Последний раз была 2014 г,
больше не хочется, хоть скучаю. У нас воздух Свободы. Удачи

\begin{itemize} % {
\iusr{Петр Кузьменко}
\textbf{Татьяна Васильевна Зубко Маркина} благодарю Вас. Я вынужден тут жить большую часть года. По здоровью. Увы, Киев, хоть и идёт в Европу, но пока абсолютно не приспособлен для жизни людей с ограниченными возможностями.
\end{itemize} % }

\iusr{Александр Бергельсон}
Яркое и жуткое зрелище... Увы

\iusr{Егор Красава Тебякин Егор}
Согласен с Вами полностью.
Об этом сам часто пишу.

\iusr{Anna Rymarenko}

Лучше не сказать, действительно очень больно видеть последствия управления
городом временщиками! Вот Александр Омельченко был вроди и киевлянин, а что он
сделал с Сенным рынком? До сих пор не могу смириться! Так , что дело не только
киевлянин мэр или нет- дело в личности и в том, есть совесть у человека или нет

\begin{itemize} % {
\iusr{Петр Кузьменко}
\textbf{Anna Rymarenko} не был он киевлянином.

\iusr{Sara Shpilman}
\textbf{Anna Rymarenko} а сейчас и Житний ринок род угрозой...

\iusr{Петр Кузьменко}
\textbf{Sara Shpilman}! Для меня и многих подолян, и не только, Житний базар - святое место! Уничтожить его - убить дух древнего торгового Подола!  @igg{fbicon.anger} 
\end{itemize} % }

\iusr{Serhii Mo}
У меня бабушка на андреевском 2б жила, на втором этаже

% -------------------------------------
\ii{fbauth.rozenfeld_mihail.ashkelon.izrail.kiev.muzykant}
% -------------------------------------

Бедненький, так возмущается "застроили ему Киев", а сам в Крым ездит жить.
Киевлянин, очень принципиальный... скажите дорогие а Где бы все жили если Не
новостройки? вы думаете маленький уютный зеленый и милый привычный с детства
Киев должен оставаться таким навсегда, чтобы никаких жителей новых там Не
появлялось? Кто же будет убирать улицы, чинить вам обувь, кормить вас,
подвозить вам еду в ваши офисы?..

\begin{itemize} % {
\iusr{Петр Кузьменко}
\textbf{Михаил Розенфельд} ,

почему езжу в Саки, посмотрите на моей странице внимательно. Я же ничего не
имею против Вашего проживания в Ашкелоне, дорогой киевлянин. И не вынужденного,
а добровольного. А города нужно строить и развивать , бережно сохраняя и
оберегая исторический центр и архитектурные памямятники. Как это делают во всём
цивилизованном мире. Неужели это не понятно? Дозастраивались до того, что
ЮНЕСКО хочет исключить Софию Киевскую из списка памятников мирового культурного
наследия. @igg{fbicon.anger} 

\iusr{Ivan Volk}
\textbf{Михаил Розенфельд} В державі Ізраїль, у славному місті Єрусалимі є дуже багато людей, які можуть розказати бажаючим, як розселити багато нових жителів не знищуючи памяті старих. Не за так звичайно.

\iusr{Владимир Терлецкий}
\textbf{Михаил Розенфельд} какая в Ваших словах забота о киевлянах. А Вы оказывается улицы убираете, скажите какие?

\iusr{Михаил Розенфельд}
Дорогие киевляне. Я в вас верю. Вы не дадите свой город в обиду. И это радует. А где я живу и почему - долго рассказывать. Не в рамках комментов

\iusr{Elena Kopachinskaya}
\textbf{Михайло Розенфельд} Інфраструктура, комунікації, дороги Києва вже занадто перевантажені.
Ці новобудови ліплять без дотримання жодних норм.

\iusr{Михаил Розенфельд}

Я знаю. I це справедливо дратуе. Обирайте у мери достойну людину. Пiдтримуйте
громади в мiкрорайонах. Це ваше мiсто. Я сам в ньому буваю кожного року. Але
iнфрастуктура та транспорт не вiдповiдають його потребам. Це помiтно.

\end{itemize} % }

\iusr{Алла Титаренко}
Вы правы у власти решалы с жаждой наживы, но фото очень красивое  @igg{fbicon.thumb.up.yellow}{repeat=3} 

\iusr{Наталія Кузенкова}

Да, очень жаль что в Киеве не ценят старые исторические здания, поджигают и
сносят, строят небоскребы которые не вписывается в архитектуру. Выделили бы
район для таких строений, например как в Париже Дю Фанс, и изощрялись бы там .

\iusr{Sunny Smith}

Согласна, к сожалению по всей Украине так.... ( Уничтожают исторические
памятки, дома, деревья. Чтоб построить огромные человейники с минимальными
дворами....

\begin{itemize} % {
\iusr{Владимир Картавенко}
\textbf{Sunny Smith} есть по Ленкузне, о ее сносе, решение Верховного суда Украини...
\end{itemize} % }

\iusr{Ольга Береснева}

Сами работники управления памятников культуры (когда то там работала) говорят
что Киев изуродовали... очень жаль, разбазарили все. Европейские города
сохранили вековую историю в архитектуре, там гулять по улицам одно
удовольствие, в Киеве очень мало осталось красивых мест

\iusr{Inna Maistrouk}

Не пишите ваше мнение как последнее слово правды ...Киев шикарен и был и есть и
будет ...волнуйтесь лучше о Крыме

\end{itemize} % }
