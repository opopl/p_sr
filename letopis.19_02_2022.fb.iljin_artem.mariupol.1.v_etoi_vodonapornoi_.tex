%%beginhead 
 
%%file 19_02_2022.fb.iljin_artem.mariupol.1.v_etoi_vodonapornoi_
%%parent 19_02_2022
 
%%url https://www.facebook.com/IlyinArtem/posts/pfbid028SBRz1T6mDKEkRBixk4qFu2hLRhLj7quHMWYuHJBueFN7S9EpwKvAJjj4A4XhsQyl
 
%%author_id iljin_artem.mariupol
%%date 19_02_2022
 
%%tags mariupol.vezha,mariupol,gorod,mariupol.pre_war,bank.ua.privat,belgia,zapad
%%title В этой водонапорной башне, построенной по проекту архитектора Нильсена, когда-то находилось отделение ПриватБанка
 
%%endhead 

\subsection{В этой водонапорной башне, построенной по проекту архитектора Нильсена, когда-то находилось отделение ПриватБанка}
\label{sec:19_02_2022.fb.iljin_artem.mariupol.1.v_etoi_vodonapornoi_}

\Purl{https://www.facebook.com/IlyinArtem/posts/pfbid028SBRz1T6mDKEkRBixk4qFu2hLRhLj7quHMWYuHJBueFN7S9EpwKvAJjj4A4XhsQyl}
\ifcmt
 author_begin
   author_id iljin_artem.mariupol
 author_end
\fi

Зашёл на старую работу 🙂 

В этой водонапорной башне, построенной по проекту архитектора Нильсена,
когда-то находилось отделение ПриватБанка. Я там иногда помогал коллегам в 2002
году, хотя работал в другом отделении, на Азовмаше. А на четвёртом этаже (там
где над головой железное дно водного резервуара) однажды был на совещании по
развитию частного бизнеса - депозиты, потребкредиты, карточки и тд. Но помню
лишь то, что директор филиала угощал всех пивом с чипсами.

Сегодня впервые поднялся аж на самый верхний шестой уровень. Весь Мариуполь -
как на ладони.

Погода отличная. Люди гуляют. Тишина и красота!

ЗЫ

Башня построена по заказу и при финансовом участии металлургического завода
Русский Провиданс, учредителями которого была бельгийская компания.

Поэтому металл для цистерны привезли из Бельгии.

Это на память о том, кто стоил Мариуполь. И какая роль Запада

%\ii{19_02_2022.fb.iljin_artem.mariupol.1.v_etoi_vodonapornoi_.cmt}
