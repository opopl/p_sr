% vim: keymap=russian-jcukenwin
%%beginhead 
 
%%file 27_09_2021.fb.nik_melnik.1.istoria_skoropadskii_mannergeim
%%parent 27_09_2021
 
%%url https://www.facebook.com/nik.mel1958/posts/1305392736560382
 
%%author_id nik_melnik
%%date 
 
%%tags finlandia,istoria,mannergeim.karl_gustav,mudrost,narod,skoropadskij_pavel.getman,sssr,ukraina
%%title Ось і вся історія, а її, як правило, знати дуже актуально...
 
%%endhead 
 
\subsection{Ось і вся історія, а її, як правило, знати дуже актуально...}
\label{sec:27_09_2021.fb.nik_melnik.1.istoria_skoropadskii_mannergeim}
 
\Purl{https://www.facebook.com/nik.mel1958/posts/1305392736560382}
\ifcmt
 author_begin
   author_id nik_melnik
 author_end
\fi

У листопаді 1918 року "не мудрий" фінський народ прийняв генерал-лейтенанта
царської російської армії поміщика і переконаного монархіста Карла Густава
Маннергейма. Який навіть не був фіном і з великими труднощами говорив
по-фінськи. А соціалістів і комуняк "не мудрий" фінський народ вигнав до чортів
собачих.

\ifcmt
   pic https://scontent-yyz1-1.xx.fbcdn.net/v/t39.30808-6/242783628_1305392709893718_6584638047473482106_n.jpg?_nc_cat=103&_nc_rgb565=1&ccb=1-5&_nc_sid=730e14&_nc_ohc=ETLAVbSzoFAAX_-EfQa&_nc_ht=scontent-yyz1-1.xx&oh=636ed3524ec578d79880efc183df2856&oe=6159B0D9
	 @width 0.6
\fi

У листопаді 1918 року "мудрий" український народ вигнав до чортів собачих
генерал-лейтенанта царської російської армії поміщика і переконаного монархіста
Павла Петровича Скоропадського, який цілком пристойно українською говорив "у
доброму полтавському діалекті". А популістів - соціалістів та комуняк "мудрий"
український народ прийняв. Бо ті пообіцяли "халяву".

У результаті Фінляндія відстояла свою незалежність. Навіть після поразки у
війні з радянським монстром 1939-1940 року Маннергейм примудрився зберегти
Фінську державу. І зараз Фінляндія є однією з найзаможніших країн світу.

А Україна стала "республікою-сестрою" у складі СРСР, пережила три штучних
голодомори з трупожерством та людожерством, втратила у Другій світовій війні,
до якої її втягнув Кремль, 14,5 мільйонів осіб, враховуючи вторинні втрати та
більше 3 тисячі у війні Афганській. Зараз Україна є однією з найбідніших кран
світу...

P.S Ось і вся історія, а її, як правило, знати дуже актуально...

Павло Бондаренко
