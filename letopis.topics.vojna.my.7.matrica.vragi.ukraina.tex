% vim: keymap=russian-jcukenwin
%%beginhead 
 
%%file topics.vojna.my.7.matrica.vragi.ukraina
%%parent topics.vojna.my.7.matrica.vragi
 
%%url 
 
%%author_id 
%%date 
 
%%tags 
%%title 
 
%%endhead 

\paragraph{19:44:59 15-08-22 Владимир Усынин}

Сегодня пообщался с человеком, который, по счастью, меня не знал, и потому
разговор получился непринуждённый. Это был молодой парень, только две недели,
как перебравшийся из Украины на нашу сторону. Оказывается, это возможно:
теневые коридоры работают, хоть процесс и сопряжён с риском.

Незрелые рассуждения о причинах войны я пропустил, делая скидку на молодость,
сформировавшуюся в украине. Куда больше меня заинтересовал вопрос, почему
парень оттуда сбежал... На языке вертелось спросить: смылся от мобилизации? Но
пожалел самолюбие, и услышал его собственную версию. Оказывается, он не верит в
будущее этой страны; не понимает, что может ждать там молодёжь, кроме
потенциальной эмиграции.

Я впервые общался с типичным представителем украинской молодёжи, выросшем при
последних восьми годах противостояния. Он не является носителем никаких идей,
как и подавляющее большинство молодых людей по обе стороны баррикад, но его
прагматичная оценка ситуации наилучшим образом иллюстрирует будущее украины.
Будущее, которого нет.

\paragraph{20:00:36 15-08-22 Геннадий Поляцковой}

Ванёк если ты не в курсе из за ограниченной информации для жителей Украины, я
тебе скажу следующее: Крым в составе России и это с повестки давно ушло и не
обсуждается, ДЛНР это признанные Россией, а этого достаточно независимыми
государствами в следствии этого идёт СВО по полному освобождению этих
государств. Херсонская, Запорожская , и Харьковская области в полне может войти
по итогам референдума в состав России. А дальше видно будет...
