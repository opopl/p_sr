% vim: keymap=russian-jcukenwin
%%beginhead 
 
%%file 05_07_2021.fb.krivda_anna.1.zdanija_obajanie
%%parent 05_07_2021
 
%%url https://www.facebook.com/anna.krivda.7/posts/4890977457595802
 
%%author Кривда, Анна
%%author_id krivda_anna
%%author_url 
 
%%tags arhitektura,gorod,kiev,obajanie,ponimanie,ukraina
%%title Раньше я не понимала красоты старых зданий, их особого обаяния
 
%%endhead 
 
\subsection{Раньше я не понимала красоты старых зданий, их особого обаяния}
\label{sec:05_07_2021.fb.krivda_anna.1.zdanija_obajanie}
 
\Purl{https://www.facebook.com/anna.krivda.7/posts/4890977457595802}
\ifcmt
 author_begin
   author_id krivda_anna
 author_end
\fi

Раньше я не понимала красоты старых зданий, их особого обаяния. Все они
казались мне невыносимо грустными, как будто груз прожитых лет и боль живших в
них людей осталась там, внутри. 

Сама я выросла среди новостроек на бульваре Леси Украинки. Конечно сейчас эти
дома, времён развитого социализма, новостройками уже никто не назовёт. Именно
их однообразие и бедность архитектурного замысла высмеивались в советском
аналоге «Этой замечательной жизни», заменённом затем на «Один дома» — любимом
одними и яростно ненавидимом другими фильме Эльдара Рязанова «С лёгким паром». 

\ifcmt
  tab_begin cols=3
	width 0.3

     pic https://scontent-lga3-1.xx.fbcdn.net/v/t1.6435-9/211164356_4890966770930204_4341956602021540009_n.jpg?_nc_cat=104&ccb=1-3&_nc_sid=730e14&_nc_ohc=_AsS8ADh4m4AX9j0tV9&_nc_ht=scontent-lga3-1.xx&oh=12b5c3f8f5fdbde351d6250165befb42&oe=60FE3C8A

     pic https://scontent-lga3-1.xx.fbcdn.net/v/t1.6435-9/212220686_4890966764263538_5665748633787688784_n.jpg?_nc_cat=111&ccb=1-3&_nc_sid=730e14&_nc_ohc=Y663Livob20AX_F_AqZ&_nc_ht=scontent-lga3-1.xx&oh=05ab153c85e25b76248584d7c918b308&oe=60FE408D
	width 0.35

		 pic https://scontent-lga3-1.xx.fbcdn.net/v/t1.6435-9/213078641_4890966734263541_6606295814817740091_n.jpg?_nc_cat=109&ccb=1-3&_nc_sid=730e14&_nc_ohc=gFKphNgKlwgAX8KAaL8&_nc_ht=scontent-lga3-1.xx&oh=3a14c542859f413f1c1b6f511d45848e&oe=60FCF185

  tab_end
\fi

Новые, радостные дома, полные солнца — вот то, что окружало меня с детства и
ничего другого я не искала. Однако мы не вольны распоряжаться нашей жизнью в
полной мере, особенно в юные годы. 

Ветер судьбы помотав сначала по городу, занёс меня на мамину малую родину, в
центр, в район улицы Артёма. И вот, я каждый день ездила на троллейбусе, мимо
прекраснейших образцов архитектуры XIX – начала ХХ ст. Каждый дом, который я
видела в окно, обладал своей индивидуальностью, уникальным замыслом, в каждом
отразились вкус архитектора и фантазия бывшего владельца. Лепнина, конические
башенки, пятигранные балконы, полукруглые окна, арки, барельефы — все это не
могло не поражать воображение. Я ехала и влюблялась. 


\ifcmt
  tab_begin cols=3
	width 0.2

     pic https://scontent-lga3-1.xx.fbcdn.net/v/t1.6435-9/212938030_4890966710930210_4260920883927321153_n.jpg?_nc_cat=107&ccb=1-3&_nc_sid=730e14&_nc_ohc=TeDARjxqEkEAX8ZEZRj&_nc_ht=scontent-lga3-1.xx&oh=5cfc800533289f0d8585699666e1e7a9&oe=60FCAE28

     pic https://scontent-lga3-1.xx.fbcdn.net/v/t1.6435-9/210524606_4890968607596687_7875514295475792872_n.jpg?_nc_cat=102&ccb=1-3&_nc_sid=730e14&_nc_ohc=tr4mT1wAIGoAX8tapaK&tn=lowUrFCbCbt-jOWu&_nc_ht=scontent-lga3-1.xx&oh=3739ab877d0d7ee95b77790fb0d01f0d&oe=60FD280D
	width 0.35

		 pic https://scontent-lga3-1.xx.fbcdn.net/v/t1.6435-9/212276489_4890977337595814_9064489441385693361_n.jpg?_nc_cat=109&ccb=1-3&_nc_sid=730e14&_nc_ohc=iqMYlSL2sYoAX_fT3Iz&_nc_ht=scontent-lga3-1.xx&oh=1de165a681452c0293f1402d5c3cdbf3&oe=60FCACAD

  tab_end
\fi


Особенно сильное впечатление произвели на меня дом с квадратным балконом
забранным решёткой, поскольку был до невозможности абсурден, но и в этой
абсурдности прекрасен и Дом со змеями. Его часто так и называют — Дом со змеями
и каштанами. В фасаде последнего было что-то завораживающе готическое, готика
сквозила в самом его замысле, хотя он относится, безусловно, скорее к ар-нуво.
Тогда перед ним еще была сложная фигурная ограда, впоследствии уничтоженная.
Все эти впечатления привели к тому, что я увлеклась фотографией. Мне захотелось
навсегда увековечить эти неповторимые улицы, с их настроением, атмосферой,
окружающей их аурой загадки.

Я даже была среди финалистов какого-то фотоконкурса в Днепропетровске, с
фотографиями этого дома. Эти места помогли мне по-настоящему понять и полюбить
городскую архитектуру.

Тем более обидно, что все это варварски разрушается и уничтожается день за днём
у нас на глазах. В частности, Дом со змеями, а ведь это памятник архитектуры
фантастической красоты. Архитектор, построивший его, Игнатий Ледоховский,
наряду с Городецким, считается основателем киевской школы модерна.


\ifcmt
  tab_begin cols=4
	width 0.25

     pic https://scontent-lga3-1.xx.fbcdn.net/v/t1.6435-9/211662520_4890968727596675_667936110743636931_n.jpg?_nc_cat=108&ccb=1-3&_nc_sid=730e14&_nc_ohc=-Q8y0j_mqycAX9aNhLF&tn=lowUrFCbCbt-jOWu&_nc_ht=scontent-lga3-1.xx&oh=d8d045ebfb097080cf023fb6285ca696&oe=60FDB7CD

     pic https://scontent-lga3-1.xx.fbcdn.net/v/t1.6435-9/213938191_4890968767596671_3253718627010008641_n.jpg?_nc_cat=102&ccb=1-3&_nc_sid=730e14&_nc_ohc=1P49Z4Au5SUAX9KydUH&_nc_ht=scontent-lga3-1.xx&oh=88b698338f49497124cee9c93b16a319&oe=60FD783F

		 pic https://scontent-lga3-1.xx.fbcdn.net/v/t1.6435-9/212309040_4890977357595812_3690445247973411729_n.jpg?_nc_cat=104&ccb=1-3&_nc_sid=730e14&_nc_ohc=zfav6zjtUJQAX9eZ1pz&_nc_ht=scontent-lga3-1.xx&oh=2e424da8bed9f9f8d5dad7c5ed53f40c&oe=60FD9137
	width 0.1

     pic https://scontent-lga3-1.xx.fbcdn.net/v/t1.6435-9/211307197_4890968540930027_3209356653299223391_n.jpg?_nc_cat=106&ccb=1-3&_nc_sid=730e14&_nc_ohc=WI72UdGoU7gAX-XhqCC&_nc_ht=scontent-lga3-1.xx&oh=3f3b931ab1e96bc47045e4f653e9f99c&oe=60FC9BED
	width 0.1

  tab_end
\fi


Сначала видеть все это мне было так больно, что я начинала плакать проезжая
мимо. Дом со змеями и сейчас находится в чудовищно плачевном состоянии и никого
из киевских властей это не волнует. В то время, как на новостройки тратятся
огромные деньги, на реставрацию, и, хотя бы попытку сохранить архитектурное
наследие Киева не тратится ни копейки. 

Более того, памятники архитектуры часто разрушаются сознательно, намеренно и
целеустремлённо. Делается это с целью освободить место для новостроек, то есть
ради наживы. Безусловно люди способные на такое не имеют ни сердца, ни вкуса,
ни совести. 

Конечно строить жилые дома прибыльно, но в Киеве есть огромное количество
площадок, где это не уродовало бы лицо города, и не уничтожало его историю.
Если так пойдёт дальше — в скором времени от Киева, каким мы его знаем, ничего
не останется. Перед нашими глазами возникнет город, лишённый истории,
памятников архитектуры, красоты и души — безликий и никому не нужный. Я

родилась в этом городе, в этом городе родились мои родители, бабушки и дедушки
и я не могу равнодушно смотреть на происходящее, но все наши правители и
застройщики очевидно откуда-то приехали и для них ничего не дорого и ничто не
свято, кроме мамоны. Люди, которые подобным образом относятся к городу не имеют
права не то, что управлять им, а даже к нему приближаться. 

Каждый день я хожу по этого калечимому, уродуемому и не любимому его хозяевами
городу и вижу везде чудовищные раны умирающих домов. Каждый день я слышу о том,
что мы идём в Европу, но ни в одной Европейской стране не относятся подобным
образом к своему историческому наследию. Город приведённый в порядок мог бы
послужить еще большему наплыву туристов и в конце концов приносить прибыль и
прославлять Украину. То, что происходит сейчас с моим родным городом — просто
стыдно. Стыдно, страшно и больно. 

О чудовищном состоянии Дома со змеями и каштанами уже писали и ранее. Тогда
было установлено, что он отдан в аренду ТОВ «Акваплаз», которое сознательно
доводило его до уничтожения с целью получить разрешение на снос. Вопрос о
чудовищности подобного отношения к архитектурному наследию Украины уже
неоднократно поднимался, проводилось расследование, об этом писали и мне
казалось, что раз что-то делается и люди знают о происходящем, этому кошмару
будет положен конец. 

С тех пор прошло уже вероятно лет 10, но не изменилось ничего. Когда-то я
советовалась с юристом о том, что нужно делать в таком случае. На мои
сетования, что я не знаю, как лучше написать об этом она ответила: «А ты
напиши, как есть. Опиши то, что ты чувствуешь». Это я и пытаюсь сделать сейчас.
Пишу и слезы наворачиваются на глаза.

Может быть этот текст поможет обратить внимание, если не на чудовищность всей
ситуации в целом, то хотя бы на Дом со змеями и каштанами, который мне особенно
дорог.

Дом с змеями и каштанами находится по адресу Большая житомирская 32. Он
построен в 1911-1912 гг. Находится в государственном реестре памятников
архитектуры. Охранный номер 316.

(Фотографии, как мои, так и собранные с просторов интернета).

\ii{05_07_2021.fb.krivda_anna.1.zdanija_obajanie.cmt}
