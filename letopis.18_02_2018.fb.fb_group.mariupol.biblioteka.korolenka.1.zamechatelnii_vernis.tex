%%beginhead 
 
%%file 18_02_2018.fb.fb_group.mariupol.biblioteka.korolenka.1.zamechatelnii_vernis
%%parent 18_02_2018
 
%%url https://www.facebook.com/groups/1476321979131170/posts/1590043041092396
 
%%author_id fb_group.mariupol.biblioteka.korolenka,kibkalo_natalia.mariupol.biblioteka.korolenko
%%date 18_02_2018
 
%%tags mariupol,isskustvo,plastilin,vystavka
%%title Замечательный вернисаж
 
%%endhead 

\subsection{Замечательный вернисаж}
\label{sec:18_02_2018.fb.fb_group.mariupol.biblioteka.korolenka.1.zamechatelnii_vernis}
 
\Purl{https://www.facebook.com/groups/1476321979131170/posts/1590043041092396}
\ifcmt
 author_begin
   author_id fb_group.mariupol.biblioteka.korolenka,kibkalo_natalia.mariupol.biblioteka.korolenko
 author_end
\fi

%! rotate pics

Замечательный вернисаж

Настоящий Праздник искусства прошел в Мобильной галерее «Мир увлечений»
Центральной библиотеки им. В. Г. Короленко - открытие новой увлекательной
выставки «Мир в пластилине». Ее инициатором стала аматор Жанна Юрковская, более
известная в городе как «Тетушка Жаннет». 

Предоставленные ею для экспозиции около сотни творческих работ: собственные, а
также работы учеников – детей и взрослых – знакомят мариупольцев с картинами,
выполненными в технике «пластилинографии» - относительно новой, нетрадиционной
художественной техники. 

Ведущая вернисажа представила Жанну Юрковскую, рассказав об ее активном
жизненном пути, в котором всегда находилось место творчеству, об участии в
волонтерском движении. Жанна Юрковская – из команды создателей СП «Халабуда»,
она – автор познавательно-развлекательной программы «В гостях у тетушки
Жаннет»,  второй год вплотную занимается пластилинографией.  Жанна
Александровна обучает детей и взрослых этой удивительной художественной технике
в  «Халабуде», в школе-интернате № 2, Центральной детской библиотеке, планирует
давать мастер-классы в ЦБ им. В. Г. Короленко.

К Жанне было много вопросов, чувствовалось, что пластилиновая живопись очень
заинтересовала гостей вернисажа.  Оказалось, что пластилинография – это не
только красиво, но и чрезвычайно полезно: тренируется моторика мелких мышц
кисти и параллельно идет развитие речи, поскольку  центры движения и речи
находятся рядом,  и раздражение одного, заставляет работать интенсивнее и
другой. Работа с пластилином развивает пространственное мышление, фантазию,
избавляет от накопившегося негатива. 

На вернисаже звучали многочисленные поздравления с открытием выставки, добрые
пожелания дальнейших творческих успехов. Высокую оценку «пластилиновым»
картинам дала Татьяна Молокова - профессиональный живописец, руководитель
изостудии "АКВА" ЦДЮТ «Западный» Городского центра внешкольной работы по месту
жительства.  Поздравила Жанну Юрковскую с выставкой известный в Мариуполе
«пластилиновых дел мастер» Наталья Прокопишена. У Натальи Семеновны совсем
другой почерк в пластилиновой живописи: фон и персонажи ее картин более
выпуклые, рельефные, даже можно сказать барельефные. О своей любви к
пластилиновой живописи рассказала Юлия Вячеславовна Крючкова, психолог по
образованию. По традиции Мобильной галереи библиотекари подарили Жанне
Юрковской памятный персональный буклет.   

Ну, и какой же праздник без музыки? Учащиеся музыкальной школы № 5 Екатерина
Соловьева и Полина Кириченко своим искренним исполнением трогательных детских
песенок еще больше украсили вернисаж живописи.  

До новых встреч в Мобильной галерее «Мир увлечений»!
