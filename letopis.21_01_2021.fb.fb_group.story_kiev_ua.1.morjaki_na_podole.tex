% vim: keymap=russian-jcukenwin
%%beginhead 
 
%%file 21_01_2021.fb.fb_group.story_kiev_ua.1.morjaki_na_podole
%%parent 21_01_2021
 
%%url https://www.facebook.com/groups/story.kiev.ua/posts/1581096825420446
 
%%author_id fb_group.story_kiev_ua,kuzmenko_petr
%%date 
 
%%tags kiev
%%title Моряки на Подоле
 
%%endhead 
 
\subsection{Моряки на Подоле}
\label{sec:21_01_2021.fb.fb_group.story_kiev_ua.1.morjaki_na_podole}
 
\Purl{https://www.facebook.com/groups/story.kiev.ua/posts/1581096825420446}
\ifcmt
 author_begin
   author_id fb_group.story_kiev_ua,kuzmenko_petr
 author_end
\fi

Моряки на Подоле. 

Хотя наш Город не пронизан океанскими ветрами и не просолен морскими
брызгами назвать Киев сухопутным нельзя. В первую очередь потому, что он
раскинулся на берегах Днепра. Древний Славутич в старину был главным водным
шляхом \enquote{из Варяг в Греки}. А ещё совсем недавно Днепр представлял собой
основную водную транспортную артерию Украины. Правда инфраструктуру, включавшую
большой флот плавсредств разного назначения, прекрасный порт, замечательный
речной вокзал и многое другое практически уничтожили за последние десятилетия.
И не столько экономические обстоятельства, сколько горе - руководители разного
звена и ранга, которых объединял общий непрофессионализм и наплевательское
отношение к порученному делу. Но, будем надеяться на лучшее и с нетерпением
ждать восстановления и приумножения былого замечательного речного флота.

\raggedcolumns
\begin{multicols}{2} % {
\setlength{\parindent}{0pt}

\ii{21_01_2021.fb.fb_group.story_kiev_ua.1.morjaki_na_podole.pic.1}
\ii{21_01_2021.fb.fb_group.story_kiev_ua.1.morjaki_na_podole.pic.1.cmt}

\end{multicols} % }

Вернувшись к цели моего повествования скажу, что вторым в очереди, но не по
значению я считаю тот факт, что на древнем и вечно молодом любимом киевлянами
Подоле на протяжении практически всего двадцатого столетия встретить человека в
морской форме можно было не реже чем в Ленинграде, Владивостоке или
Севастополе. 

\ii{21_01_2021.fb.fb_group.story_kiev_ua.1.morjaki_na_podole.pic.2}

Попробую перечислить основные причины столь частого появления
моряков в нашем Городе. При знаменитом киевском заводе \enquote{Арсенал}, долгое время
выпускавшем навигационные комплексы для ВМФ работала система переподготовки
офицеров штурманской службы, с большим удовольствием приезжавших в Киев со всех
флотов страны на обучение или переобучение. На Подольском Рыбальском
полуострове был ещё один заслуженный судостроительный завод «Ленинская
кузница», на котором создавались и ремонтировались многие проекты малых
надводных кораблей и катеров для Военно-Морского Флота и промыслового флота.
Принимать корабли приезжали экипажи со всех флотов и флотилий. 

\ii{21_01_2021.fb.fb_group.story_kiev_ua.1.morjaki_na_podole.pic.3}

В годы
существования Днепровской военной флотилии её боевые катера базировались на
левом берегу Гаванской бухты (или гавани особого назначения, сокращённо – ГОН,
как между собой называли её киевляне) возле Киевского маяка. Впоследствии там
же на Рыбальском полуострове находились 316-й учебный отряд ОСНАЗ ВМФ,
включавший 97-ю школу младших специалистов ОСНАЗ и корабельных разведчиков
срочной службы, в которой кстати проходил обучение и часть флотской службы
Василий Шукшин, а также школу техников ВМФ, выпускавшую мичманов той же
специальности. На Печерске до 1989гда существовали 10 - е курсы
усовершенствования политсостава ВМФ. Там проходили двухмесячную подготовку
пропагандисты и секретари комитетов ВЛКСМ флотов. 

\ii{21_01_2021.fb.fb_group.story_kiev_ua.1.morjaki_na_podole.pic.4}

Под Киевом, в Броварах,
дислоцировался 108-й отряд радиоразведки ОСНАЗ ВМФ. Ещё в Городе работали
научно-производственное объединение «Славутич» и Киевский военно-морской
техникум связи, ковавший кадры специалистов по радиотехническим и
гидроакустическим системам, в том числе, и для создателей кораблей ВМФ.

\ii{21_01_2021.fb.fb_group.story_kiev_ua.1.morjaki_na_podole.pic.5}

Киевское речное училище, основанное в 1912 г. на Подоле, обучало судоводителей,
судомехаников, электромехаников, специалистов водных путей сообщения.

Существовал киевский филиал Ленинградского института водного транспорта и
судостроительный техникум, ныне преобразованные в Академию водного транспорта.
В годы Гражданской и Великой Отечественной войн в историческом здании в центре
Подола работал штаб Днепровской военной флотилии, о героическом пути которой
напоминают памятник морякам речной флотилии, и корабль - памятник монитор
«Железняков» на Рыбальском полуострове в Парке моряков. 

\raggedcolumns
\begin{multicols}{2} % {
\setlength{\parindent}{0pt}

\ii{21_01_2021.fb.fb_group.story_kiev_ua.1.morjaki_na_podole.pic.6}
\ii{21_01_2021.fb.fb_group.story_kiev_ua.1.morjaki_na_podole.pic.6.cmt}

\ii{21_01_2021.fb.fb_group.story_kiev_ua.1.morjaki_na_podole.pic.7}
\ii{21_01_2021.fb.fb_group.story_kiev_ua.1.morjaki_na_podole.pic.7.cmt}

\ii{21_01_2021.fb.fb_group.story_kiev_ua.1.morjaki_na_podole.pic.8}
\ii{21_01_2021.fb.fb_group.story_kiev_ua.1.morjaki_na_podole.pic.8.cmt}

\end{multicols} % }

В стенах древней легендарной Киево-Могилянской академии в 1947 — 1957 гг.
располагалось среднее 2-е Военно-морское политическое училище, а после него –
учебный отряд ВМФ. И, наконец, в тех же зданиях и на ещё одном подольском
квартале с 21 января 1967 года Альма - Матер многих офицеров - воспитателей
моряков, в том числе и автора этих строк, Киевское Высшее Военно - Морское
политическое училище. Именно сегодня все причастные к этому Вузу отмечают день
его основания. О нашем Морполите, как называли его киевляне, расскажу
подробней.

Училище было замечательным и стать его курсантом было престижно и весьма не
просто. Особенно для нас, вчерашних школьников. Это было обусловлено тем, что
принимали учиться в кузницу кадров бойцов идеологического и морально -
психологического фронта в основном ребят уже послуживших на Флоте или в войсках
и прочувствовавших на своей шкуре тяготы и лишения срочной службы. Но я,
подольский мальчишка выросший рядышком с училищем и с детства восхищавшийся
красивой морской формой, выправкой, статью и прекрасными манерами курсантов
тогда этого до конца себе не представлял. Хотя с замиранием сердца слушал
рассказы регулярно приглашаемых в нашу 100 - ю школу преподавателей и курсантов
КВВМПУ, и посещал несколько лет военно - патриотическое объединение "Океан",
где нашими наставниками тоже были курсанты - моряки. Итак, выдержав большой
конкурс, успешно сдав вступительные экзамены и пройдя "курс молодого бойца" Ваш
покорный слуга и другие новоиспечённые военнослужащие приняли Присягу в Парке
Славы у Вечного огня и летом 1982 года стали курсантами Морполита. Подробно
рассказывать о перипетиях нашей учёбы в этом посте я не стану. Скажу лишь что
обучали и воспитывали нас, превращая из юнцов в флотских офицеров, настоящие
профессионалы. Прекрасные, глубоко знающие и беззаветно любящие свои предметы
преподаватели, многие с учёными степенями докторов и кандидатов наук, опытные и
заслуженные морские волки. А воинскую дисциплину и премудрости службы нам очень
доходчиво и по - отечески втолковывали великолепные отцы - командиры. Учёные
мужи с погонами старших офицеров Флота научили нас главному - правильно
пользоваться своими мозгами. В первую очередь думать, анализировать,
сопоставлять, делать выводы и говоря о предмете, явлении, событии или документе
насколько возможно изучить его досконально, а потом уже доводить свои
умозаключения до подчинённых, сослуживцев или начальства. 

Система наша, как
часто между собой мы называли родное училище, была довольно своеобразной и
уникальной. Выпускаясь мы получили по сути по два образования. Гуманитарное и
техническое. В дипломах наша специальность называлась политическая -
штурманская ВМФ. Гражданская квалификация была преподаватель истории и
обществоведения и штурман флота. 

Как я уже отмечал, располагалось училище на
двух территориях учебной и жилой. Учебная была в исторических корпусах одного
из первых учебных заведений в Европе Киево-Могилянской академии основанной
на базе Киевской братской школы, в 1615 году митрополитом Петром Могилой.
Конечно за многие века корпуса и аудитории неоднократно перестраивались,
строились наново и ремонтировались, но тот старинный дух бурсаков, их неуёмное
стремление к знаниям и особая гордость за обучение и пребывание в знаменитом
историческом месте древнего Подола величественного Киева странным образом
передались курсантам сквозь века. Я уже неоднократно упоминал свою Альма -
Матер в моих постах в нашей группе и получил в комментариях просьбы от нынешних
преподавателей, выпускников и студентов Киево-Могилянской академии
рассказать в каких корпусах Вуза и какие науки мы постигали. Постараюсь
припомнить, насколько смогу. Если ошибусь или о чём-то забуду, прошу коллег,
заканчивавших наш родной Морполит в разные годы исправить и дополнить прорехи
моей памяти в комментариях. 

Итак, основной вход в главный корпус училища располагался там же где и сейчас.
По обе стороны от него красовались выкрашенные и вымытые до блеска якоря. В
вестибюле первого этажа было знамя КВВМПУ и часовой пост номер один. Стоя в
карауле у знамени, днём или ночью в любое время года я видел сквозь большие
окна те самые свои родные и близкие с детства пару кварталов, о которых уже
писал. Были хорошо видны боковые окна моей детской квартиры на Андреевском
спуске 2, универмаг и гастроном, о которых я поведал в предыдущей статье,
Красная площадь, сейчас вернувшая историческое название Контрактовая, сквер
моего детства, конечно изменившийся к тому времени, но близкий и родной. На
этом караульном посту моё сердце всегда бились учащённо. Я понимал, что
нахожусь в своём маленьком Подольском мирке, но уже совсем в другом качестве,
как защитник Родины, главным символом которой для меня было и навсегда
останется это место. 

Простите за лирическое отступление
из-за нахлынувшей в очередной раз ностальгии. 

Итак, продолжу. Чугунные литые
ажурные ступеньки центральной широкой металлической лестницы, ведущей с первого
на второй этаж главного учебного здания училища, сохранились до наших дней. На
каждой из них была выкована надпись: «Донатъ, Липковский и К. 1912». С этой
надписью связана одна из переходящих из поколения в поколение курсантов
училищная байка об обманутой девушке, которая искала на КПП и перед
выстроенными на плацу курсантами своего неверного Доната Липковского. 

Кстати
была ещё одна история, которая действительно произошла в училище и стала
притчей в языцех. История о том как году эдак в 1972 (если ошибаюсь, знающие
поправят) наш курсант, будущий проводник коммунистического мировоззрения в
массы, выкрал, разумеется с согласия и по обоюдному любовному влечению
послушницу из расположенного неподалёку женского Флоровского монастыря, женился
на ней и увёз на Флот, где бывшая инокиня стала ему верной боевой подругой.

Снова упускаю нить повествования и надеюсь на прощение читателей. На первом
этаже слева от знамени был коридорчик с рубкой дежурного по училищу и
вспомогательные помещения. Далее поворот и длинный коридор первого этажа, вдоль
которого первой слева располагалась аудитория химической и радиационной защиты.
Кстати (ну, вот опять!) с ней и нашим преподавателем капитаном второго ранга
Радзиховским связана уже далеко не смешная история о том, как перед самым нашим
выпуском в апреле 1986 года в этом кабинете он вдруг обнаружил как, без
помещаемого для убедительности работы прибора слабого источника радиации сам по
себе заработал КРАБ. Прибор корабельной дозиметрической системы обнаружения
радиационного альфа, бета-излучения. О чём было доложено по команде в Киевский
Военный округ. Помнится они несколько дней присылали офицера соответствующего
рода войск, который записывал показания нашего КРАБа. Не знаю уж зачем. Как мы
тогда думали, потому, что у сухопутных военных не было такой чувствительной
техники. Слухи о чернобыльской катастрофе ещё не поползли по Киеву. Продолжаю
рассказ. 

Далее располагался один из кабинетов партийно - политической работы,
одной из основных наших дисциплин. Далее, после бокового выхода из корпуса на
плац, была училищная библиотека, занимавшая весь оставшийся первый этаж здания.
С правого борта первого этажа, если мне не изменяет память, были кабинеты ТУЖЭК
(теории, устройства, живучести и электрооборудования корабля) и ТСК
(технических средств кораблевождения). Кстати лаборатории живучести с имитацией
корабельных отсеков находились в подвале первого корпуса. 

Со стороны улицы, в подвале была и барокамера в которой нас тестировали на
профпригодность к службе на подводных лодках, при поступлении в училище. Рядом
с красивой лестницей, слева второго этажа находилась большая 217 аудитория,
оборудованная телевизорами, кинопроектором и экраном. 

Помню, как в ноябре 1982 мы в ней стоя и скорбя наблюдали прямую трансляцию
похорон Брежнева. Сразу несколько аудиторий второго этажа принадлежали кафедре
кораблевождения. Они были оборудованы рабочими местами штурмана, для прокладки
курса корабля. 

Ещё на втором и третьем этажах размещались кабинеты философии, ТСК, мореходной
астрономии, психологии, литературы, этики и эстетики. С обратной стороны
первого корпуса, в отдельной пристройке, находился кабинет морской практики.
Рядом с ним боковые ворота, выходившие на улицу Ильинскую и баня, маленькие
зарешечённые окошки которой смотрели на швейную фабрику им. Смирнова -
Ласточкина, бывшую одним из основных \enquote{поставщиков} курсантских невест. 

Во втором корпусе находились аудитории математики, физики, иностранных языков,
превращавшиеся во второй половине дня в кабинеты самоподготовки. Ещё в нём был
магазинчик по типу военторга где продавалось всё. От печенья и конфет до
форменной фурнитуры и фотоаппаратов. Отдельно на первом этаже располагались
караульное помещение, почта и книжный магазин. За вторым корпусом была военно -
морская полоса препятствий и сетчатый забор за которым мы видели научную
библиотеку в сохранившемся здании храма Святого Духа. В циркульном здании
располагалась офицерская столовая ныне превращённая в кафе \enquote{Академия}, кафедры
и кабинеты преподавателей, аудитории кафедр научного коммунизма, истории КПСС,
политэкономии, технических средств пропаганды и другие, тоже становившееся в
послеобеденное время классами самоподготовки. 

Осталось написать о здании
изнутри примыкающем и соединённом с циркульным. В нём был спортзал и, полагаю
сейчас уже можно об этом писать, кабинеты и лаборатории секретной кафедры номер
четыре, где мы изучали тактику флота и корабельное оружие. Склоняю в низком
поклоне голову перед терпеливыми читателями, осилившими эту довольно нудную
часть моего повествования. Но, я обещал это описать немалому количеству наших
одногруппников, чья жизнь связана с Киево  -  Могилянской академией. 

Далее вскользь остановлюсь на жилой территории училища, которая занимала целый
квартал, обрамлённый улицами Волошской, Григория Сковороды, Почайнинской и
Ильинской. Там в правом здании, если стоять лицом к КПП, был строевой отдел,
санчасть, ателье замечательно шившее офицерскую форму выпускникам и командно -
преподавательскому составу, и комната для встреч с посетителями. 

В угловом сером здании со стороны Ильинской находился камбуз (кухня на
сухопутном языке), столовая и клуб с большим актовым и просмотровым кинозалом,
красивым фойе третьего этажа, где желающие факультативно серьёзно занимались
бальными танцами (как вы понимаете, недостатка в партнёршах наши танцоры не
испытывали) и большим балконом с видом на древнейшую в Киеве Ильинскую церковь
прямо, и на церковь Святого Николая Набережного левее, находящиеся уже по
Почайнинской. На стороне этой улицы при нас располагались жилые кубики с
соответствующей инфраструктурой общего пользования и наш замечательный музей на
первом этаже.  Ещё в нём был чипок, что на флотском жаргоне означает кафе. 

В угловом здании по
улицам Григория Сковороды и Волошской, сначала шёл ремонт, в котором мы
принимали посильное участие, потом были жилые кубики и классы. Во внутреннем,
примыкающем к нему здании, было общежитие некоторых офицеров и гражданского
персонала. Посредине красовался огромный плац, где мы оттачивали строевую
подготовку. Фух! Докладывая языком того времени отрапортую: \enquote{Старший курсант
Кузьменко основной доклад закончил!} 

Друзья, квумпари разных лет киевляне и
дорогие участники Киевских историй! Пожалуйста, по возможности, поправляйте и
дополняйте мой пространный рассказ о нашем родном Морполите в комментариях.
Нижайше прошу прощения у читателей за свои воспоминания, превратившиеся для вас
длинющий нудный рассказ. Но, повторюсь ещё раз, я выполнял просьбу наших
читателей - одногруппников. 

Напоследок хочу сердечно поздравить всех
преподавателей, командиров и выпускников КВВМПУ, среди которых есть до сих пор
несущие флотскую службу офицеры и адмиралы, политики, писатели, художники,
руководители разного ранга, бизнесмены и предприниматели, служащие различных
силовых и ведомственных структур и многие другие разлетевшиеся по всему Миру
настоящие мужчины,  для которых словосочетание \enquote{Честь имею!} далеко не пустой
звук, а также киевлян, неравнодушных к Морполиту с сегодняшней годовщиной
создания этого, на протяжении многих лет, знакового для Подола и Киева военного
училища.

