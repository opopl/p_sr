%%beginhead 
 
%%file 20_04_2018.fb.fb_group.mariupol.biblioteka.korolenka.1.zhit_v_zeleni___zhit
%%parent 20_04_2018
 
%%url https://www.facebook.com/groups/1476321979131170/posts/1652584261504940
 
%%author_id fb_group.mariupol.biblioteka.korolenka,kibkalo_natalia.mariupol.biblioteka.korolenko
%%date 20_04_2018
 
%%tags mariupol,mariupol.pre_war,biblioteka,kultura,ekologia,master_klass
%%title Жить в зелени – жить красиво!
 
%%endhead 

\subsection{Жить в зелени – жить красиво!}
\label{sec:20_04_2018.fb.fb_group.mariupol.biblioteka.korolenka.1.zhit_v_zeleni___zhit}
 
\Purl{https://www.facebook.com/groups/1476321979131170/posts/1652584261504940}
\ifcmt
 author_begin
   author_id fb_group.mariupol.biblioteka.korolenka,kibkalo_natalia.mariupol.biblioteka.korolenko
 author_end
\fi

Жить в зелени – жить красиво!

В Центральной библиотеке им. В.Г. Короленко продолжается цикл «Весенних
экологических ассорти». 20 апреля ученики младших классов ОШ № 7 (директор
Папазова О. Л.) во главе с учителем Гринько Е.И.  стали участниками экоассорти
«Жить в зелени – жить красиво!». Библиотекари рассказали школьникам о том, что
такое экология, о различных «экологических» датах апреля, познакомили с
творческими работами большого любителя природы Татьяны Александровны Чебановой. 

А затем дети отправились в небольшое видео-путешествие по Городскому саду
Мариуполя. 18–22 апреля в Украине отмечаются Дни заповедников и национальных
парков, и, хотя Городской сад не является ни заповедником, ни национальным
парком, этот прекрасный зеленый уголок очень любят дети и взрослые Мариуполя.

Ребята с большим удовольствием приняли участие в создании цветочных композиций
из морских ракушек. Вообще даже самые простые мастер-классы – это уникальные
мероприятия для детей, в которых они фантазируют, экспериментируют, открывают
удивительные тайны обычных предметов.

В завершении «Весеннего экологического ассорти»  все его участники дружно
решили любить и беречь природу.
