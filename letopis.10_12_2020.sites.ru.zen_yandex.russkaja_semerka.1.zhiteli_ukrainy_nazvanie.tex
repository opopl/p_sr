% vim: keymap=russian-jcukenwin
%%beginhead 
 
%%file 10_12_2020.sites.ru.zen_yandex.russkaja_semerka.1.zhiteli_ukrainy_nazvanie
%%parent 10_12_2020
 
%%url https://zen.yandex.ru/media/russian7/kak-do-vossoedineniia-s-rossiei-nazyvali-sebia-jiteli-ukrainy-5fd1af23117746797687a06d
 
%%author Русская Семерка
%%author_id russkaja_semerka
%%author_url 
 
%%tags istoria,ukraine,russia
%%title Как до воссоединения с Россией называли себя жители Украины
 
%%endhead 
 
\subsection{Как до воссоединения с Россией называли себя жители Украины}
\label{sec:10_12_2020.sites.ru.zen_yandex.russkaja_semerka.1.zhiteli_ukrainy_nazvanie}
\Purl{https://zen.yandex.ru/media/russian7/kak-do-vossoedineniia-s-rossiei-nazyvali-sebia-jiteli-ukrainy-5fd1af23117746797687a06d}
\ifcmt
	author_begin
   author_id russkaja_semerka
	author_end
\fi

\begin{verbatim}
Русская Семёрка 183 883 подписчика
\end{verbatim}

\ifcmt
pic https://avatars.mds.yandex.net/get-zen_doc/3413538/pub_5fd1af23117746797687a06d_5fd1af57fe22070c49f01702/scale_1200
\fi

Различные варианты слова «украины», «оукрайны» до XVII века обозначали
приграничные земли Руси. «Украинными городами» были даже поселения в Сибири и
возле Астрахани. В московских документах XVII века к «украинным городам»
причислялась, например, Тула. А те, кто жил на территории нынешней Украины, в
разные времена называли себя по-разному.

\subsubsection{Русины}

«Русь», «рось» – это изначально не название государства, а собирательный
термин для обозначения людей, живущих на определенной территории. Впервые
он встречается в договоре князя Олега с Византийской империей в 911-912
гг. Тогда бок о бок жили «мордва», «литва», «чудь». В Суздальской летописи
было деление на «русь» и «бесурмен».

«Русью» и «русинами» на современных украинских землях сперва называли
жителей Переяславского и Киевского княжеств. В других княжествах,
располагавшихся на территории нынешней Украины, люди называли себя,
соответственно, «галичанами», «волынцами», «черниговцами».

Однако если в договорах с иностранцами нужно было охарактеризовать людей,
живущих на более широком пространстве, тоже использовали этноним «русины».
Постепенно слово распространялось, и к концу XII – началу XIII веков слово
«русин» стало самоназванием жителей почти всей территории современной
Украины. Сейчас этот этноним сохраняется на западе Украины, соседствуя на
отдельных территориях с его вариантом «руснаки».

\subsubsection{Люди Руские, «руський народ».}

Пока существовала Киевская Русь и в течение долгого периода после ее
распада, жители многих земель называли себя «рускими», «русами». В
«Хронографе» (копия 1512 года) говорится о самоназывании людей по названию
реки Рось – правого притока Днепра: «И нарекошася своим именем Русь ради
Русы».

В украинском фольклоре названия «русины», «руськие», «русы» присутствуют
параллельно. В документах нескольких столетий упоминаются «духовенство
Руское», «вера Руская».

В XVII веке самоназвание людей вошло в официальное обозначение
государственного образования «Великое княжество Литовское и Русское» (его
полное название записано в «Статуте» 1529 года).

В договоре с Польшей гетмана Ивана Выговского (Гадячский договор 1658
года.) говорится о населении современной Украины как о «народе Руском»,
которому должно быть позволено сохранить старую греческую веру и свой
язык. Сам гетман тоже называл себя «гетманом войск Руских» и «княжества
Руского». Это было записано и на его личной печати.

Многие иностранцы тоже предпочитали такой этноним – в польских и литовских
грамотах и договорах слово «русские» встречается неоднократно. В то время
не существовало единого свода грамматических правил, поэтому в летописях и
законодательных актах слова писались то с одной «с», то с двумя, то с
мягким знаком, то без него.

В XIV веке молодые люди, жившие на территории современной Украины, начали
поступать в европейские университеты. Тогда запись этнической
принадлежности осуществлялась по желанию, ведь привычный нам «перечень
национальностей» еще не сформировался. За несколько веков в
университетских списках появились студенты «рутенской нации», «русины»,
«роксоланы», «русиняки». Часть этих самоназваний – формы, возникшие на
основе латинских обозначений. С начала XVIII века в этих списках уже
встречаются «казаки» и «украинцы».

\begin{itemize}
	\item \cusr{владимир яковенко}

Хоспода Квасные демагоги имени СпИцына - Мединского прежде чем что-то сочинять
								историческое - запомните следующее ...

= не берите инфу и наших псевдо-древних летописей - там фсё 146\% лжи некие
								правдивые факты там попадаются только случайно - но для этого
								надо знать - как их определить ...

= киевской руси никогда не было - так как город Куяй-гард строился с 1690 года
								как торгашеский город входящий в состав Ганзейского союза
								Вольных торговых городов

= коренным историческим народом нонешней Украины - являются ПОЛОВЦЫ - а
								историческое название этой территории - Восточная Поляния -
								ближайшие геннетические родственники нонешних поляков - их
								настоящими древними столицами являются Львив - Переяславль -
								Чернигов

= исходя из этого - вывод - УКРАИНЦЫ -ПОЛОВЦЫ - кас геннетические славяне - не
								имеют ничего общего ни с прибалтийскими латинянами - ни с
								карпатскими русинами - ни тем паче с фино-угорскими народами от
								Ижоры - Карелии и до Урала ....

= территория нонешней типа беллоруссии - это территория Восточной Пруссии - до
								1730 года - плюс Псков - Калуга - Смоленск - Тула - Тверь - Н
								Новгород - все эти города до 1730 года входили в состав
								прусского Ганзейского торгового союза - столицей Восточной
								Пруссии был г Манкерман - ноне это лже-полоцк ....

УЧИТЕ ИСТОРИЮ ПО НАСТОЯЩИМ ИСТОЧНИКАМ - А НЕ ПО ПРОПАГАНДОНСКИМ БРЕДНЯМ ДЛЯ
								ХОЛОПОВ И БЫДЛА ....
\end{itemize}
