% vim: keymap=russian-jcukenwin
%%beginhead 
 
%%file 05_04_2022.fb.pshemyskyj_oleksij.1.bucha_mykola
%%parent 05_04_2022
 
%%url https://www.facebook.com/oleksiy.pshemyskiy/posts/10158334472141852
 
%%author_id pshemyskyj_oleksij
%%date 
 
%%tags 
%%title Йому дали 20 хвилин аби зібрати по частинах свого друга і поховати його
 
%%endhead 
 
\subsection{Йому дали 20 хвилин аби зібрати по частинах свого друга і поховати його}
\label{sec:05_04_2022.fb.pshemyskyj_oleksij.1.bucha_mykola}
 
\Purl{https://www.facebook.com/oleksiy.pshemyskiy/posts/10158334472141852}
\ifcmt
 author_begin
   author_id pshemyskyj_oleksij
 author_end
\fi

Йому дали 20 хвилин аби зібрати по частинах свого друга і поховати його. 

Це Микола. Йому 53. Він живе у Бучі. Микола сильно заїкається. Лише закуривши
цигарку, починає говорити.

Він живе у підвалі вже 34 дні. Міг виїхати, але чоловік - управдом у
п'ятиповерхівці. Каже - не міг не лишитися. У перший день боїв у Бучі в його
вікна прилетів снаряд, пробив стіну і застряг у дитячому ліжку, яке загорілося.
На щастя дітей він на той час вже евакуював. Вогонь потушили, й одразу разом з
трьома друзями спустили усіх стареньких і жінок у підвал, і самі перебралися
туди. 

\ii{05_04_2022.fb.pshemyskyj_oleksij.1.bucha_mykola.pic.1}

Коли росіяни захопили місто, вони почали вриватися у кожен будинок. Чоловіків
виводили на вулицю, роздягали, шукали татуювання. Двом його друзям, Леоніду та
Сергію, було за 50, ще один, теж Леонід, був значно молодший. Побачивши паспорт
Леоніда - сказали, що йому нема 50, а отже може воювати. Поставили на коліна і
вистрелили у голову. 

Леоніда Микола поховав першим. Прямо на дворі. Біля трансформаторної будки. На
місці злочину досі видніється пляма крові. 

За кілька днів не стало й Сергія. Чоловік вийшов на вулицю перекурити - і
отримав кулю. Просто так. Без жодних слів і попереджень.

\ii{05_04_2022.fb.pshemyskyj_oleksij.1.bucha_mykola.pic.2}

Коли бої посилилися росіяни не мов оскаженіли, каже Микола. До цього люди іноді
виходили з підвалу - приготувати їсти чи просто подихати повітрям. Але тоді
вирішили запертися. Під вечір солдати почали гатити. Кричали аби відкрили. Перш
ніж залишити місто йомвірно хотіли удертися аби розстріляти усіх. Так сталося в
одному з будинків на сусідній вулиці, каже чоловік.

Не змігши виламати двері кинули на сходи гранату. З іншого боку вхід міцно
тримав другий Леонід. Єдиний з чоловіків, який лишився живий разом з Миколою.
Вибух. А після нього тиша. Цілу добу його тіло пролежало на скривавлених
сходах. Лише на наступний день у двері постукали знов, сказавши - у вас є 20
хвилин аби усе прибрати. Тоді Микола вийшов і побачив, що його другу відірвало
голову і розкидало ноги.

Рештки Микола зібрав у пакет та викопав нову могилу. Вже третю. Каже, розкопати
землю глибоко не зміг, мало часу, та і роки вже не ті. А тому тепер найбільше
переймається, що коли піде дощ, пісок розмиє, і прийдуть бродячі собаки.

\ii{05_04_2022.fb.pshemyskyj_oleksij.1.bucha_mykola.pic.3}

Посеред жаху, болю і смерті мабуть легко втратити надію. Але сьогодні Микола -
мій герой. Бо побачивши найгірше у людях він не втратив людяність сам. Говорячи
з нами на камеру, чоловік ледве стримував сльози. А коли запис закінчився -
розплакався і подякував, що його просто вислухали. 

Дивлячись на братські могили, де з під піску стирчать окачанілі руки, легко
втратити віру в людство, але саме такі, як Микола, повертають її.
