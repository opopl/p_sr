% vim: keymap=russian-jcukenwin
%%beginhead 
 
%%file 10_04_2023.stz.news.ua.donbas24.2.velykyj_obmin_polonenymy_80_geroiv_20_geroinj.txt
%%parent 10_04_2023.stz.news.ua.donbas24.2.velykyj_obmin_polonenymy_80_geroiv_20_geroinj
 
%%url 
 
%%author_id 
%%date 
 
%%tags 
%%title 
 
%%endhead 

10_04_2023.natalia_sorokina.donbas24.velykyj_obmin_polonenymy_80_geroiv_20_geroinj
Україна,Війна,Полон,Обмін полоненими,date.10_04_2023
Наталія Сорокіна

Великий обмін полоненими: в Україну повернулися ще 80 героїв та 20 героїнь
(ФОТО)

Серед звільнених з російського полону — оборонці Маріуполя, Бахмута, ЧАЕС, ЗАЕС
та з інших напрямків

Україні сьогодні, 10 квітня, вдалося провести черговий великий обмін
полоненими. На Батьківщину повернулися 80 наших захисників та 20 захисниць.
Серед них, зокрема, оборонці Маріуполя, Бахмута, Запорізької АЕС, Київщини,
острова Зміїний.

Читайте також: «Заборонив втрачати себе» — воїн ЗСУ родом з Донецька про те, як
вдалося пережити 321 день полону (ФОТО)

Про це передає Донбас24 з посиланням на Координаційний штаб з питань поводження
з військовополоненими.

Як повідомляється, визволити вдалося 24 нацгвардійців, 22 прикордонника, 22
представника Військово-морських сил, 21 військовослужбовця Збройних сил України
та 11 тероборонців. Вони брали участь у боях на Херсонському, Запорізькому,
Донецькому, Харківському, Сумському, Київському напрямках.

Майже половина з колишніх бранців мають важкі поранення, хвороби або зазнали
катувань. Наймолодшому зі звільнених героїв — всього 19 років. 

«Це був непростий обмін, і я вдячний всій команді, Координаційному штабу з
питань поводження із військовополоненими за те, що кожен робить те, що часто
багатьом може здаватися неможливим. Робить свою роботу, дуже важливу сьогодні,
— написав у своєму телеграм-каналі керівник Офісу Президента України Андрій
Єрмак. — Особливо радий повідомити чудову новину про обмін родичам звільнених з
полону людей. Вони довго чекали своїх чоловіків, жінок, батьків вдома.
Очікування — це завжди дуже складний та нервовий процес».

Нагадаємо, що Україна докладає максимум зусиль, щоб якнайшвидше повернути з
російського полону усіх своїх громадян. Обмін полоненими з країною-агресоркою
відбувається на постійні основі: минулий було проведено 3 квітня.

Ще більше новин та найактуальніша інформація про Донецьку та Луганську області
в нашому телеграм-каналі Донбас24.

Фото: Координаційний штаб з питань поводження з військовополоненими
