% vim: keymap=russian-jcukenwin
%%beginhead 
 
%%file 27_10_2020.news.ua.pravda.1.kgb_kultura
%%parent 27_10_2020
%%url https://life.pravda.com.ua/projects/houseofeurope/2020/10/27/242729/

%%endhead 

\subsection{Репресоване мистецтво: як українські дослідники показують культуру
через призму архівів КҐБ}

{\small
\vspace{0.5cm}
\url{https://life.pravda.com.ua/projects/houseofeurope/2020/10/27/242729/}
\vspace{0.5cm}
}

СПЕЦПРОЄКТ, Тамара Черкасова
\index[authors.rus]{Черкасова, Тамара}

Як червоний терор вплинув на розвиток української культури та що нового про
митців можна дізнатися з офіційних документів спецслужб, розповіли в.о.
директорки Центру досліджень визвольного руху Анна Олійник та директор архіву
СБУ Андрій Когут.

Центр досліджень визвольного руху спільно з колегами з Грузії та Чехії готує
онлайн-виставку "Репресоване мистецтво", яка покаже життя митців через призму
агентурних документів. Широкий загал побачить архівні документи навесні 2021
року, але підготовка до запуску проєкту вже ведеться.

Проєкт "Репресованого мистецтва" має привернути інтерес до архівних документів
через історії митців, які потрапляли у поле зору комуністичних спецслужб. 

З чого ж все почалося? Ще п'ять років тому Центр досліджень визвольного руху
працював над розробкою закону про доступ до архівів КҐБ, який зробив можливим
роботу з документами спецслужб. Потім перед співробітниками Центру постала нова
задача -— не тільки повідомити про відкриття архівів, а й навчити людей
працювати з офіційними документами. Для цього було створено посібник "Архіви
КҐБ для медіа".

"Що ми маємо тепер? Найбільший відкритий архів КҐБ в світі! А це не тільки про
історію. Це, насправді, ключ до того, щоб зрозуміти, як взаємодіяти з тими
викликами та проблемами, які у нас є сьогодні. Наприклад, з масивними
дезінформаційними кампаніями, які вкидає Росія, маніпулюючи історичними темами.
Спростування цих "фактів" можна знайти саме у відкритих архівах", --- розповідає
Андрій.

"Дезінформаційний наратив Росії, насправді, не змінився ще з часів СРСР —
лекала та заготовки ті ж самі. Змінилися інструменти, джерела поширення
інформації, але канва та методи роботи Росії прочитуються і у документах КҐБ",
--- додає Анна.

\ifcmt
pic https://life.pravda.com.ua/images/doc/6/6/6664be0-pic61.jpg
\fi

\subsubsection{Про "Репресоване мистецтво"}
\label{sec:27_10_2020.news.ua.pravda.1.kgb_kultura}

"Ми не культурологи та не мистецтвознавці, тому не беремо на себе експертизу.
Але хочемо показати, як українське мистецтво розвивалося попри цензуру,
партійні обмеження та радянські канони. Наше мистецтво було динамічним та
яскравим, навіть не дивлячись на те, яка трагічна доля спіткала багатьох
українських митців", --- пояснює Анна.

Проєкт "Репресоване мистецтво" покаже, яким чином українські культурні діячі
фігурували у документах радянських спецслужб. Це дозволить відкрити ще одну
дослідницьку грань.

Організатори не будуть давати оцінку діям митців, а просто покажуть їх життя
через призму документів. Хоча планується залучати експертів з мистецького
середовища, висновки має зробити сама аудиторія.

"Але "Репресоване мистецтво" --- це також спроба культурної дипломатії, адже ми
робимо проєкт спільно з колегами з чеською організацією Gulag.cz та грузинською
організацією SovLab. Більш того, онлайн-виставка буде перекладена не тільки
мовою всіх учасників проєкту, а й англійською. Тобто, ми зможемо показати
українське мистецтво на ширшому рівні", --- розповідає Анна.

Поки що у Центрі дослідження визвольного руху тільки обирають героїв --- шукають
цікаві показові справи, візуальний матеріал. 

"Ми хотіли б долучити до виставки митців, які потрапляли у поле зору КҐБ, і
можуть зараз дати коментар про це. Насправді, зараз процес відбору героїв більш
схожий на складання пазлів. Ми шукаємо митців, яких об'єднує Україна, Чехія та
Грузія, та місточки зв'язку між ними. У квітні 2021 року ми представимо їх --- по
десять митців з різних країн", --- пояснює Анна.

"Gulag.cz" вже давно співпрацює з Центром досліджень визвольного руху. Вони
запросили нас долучитися до проекту "Репресоване мистецтво", і ми із
задоволенням це зробили. Це чудова ідея! Ми готуємо чеську частину
онлайн-виставки спільно з Чеським архівом служб безпеки, де знаходяться справи
чехословацької комуністичної таємної поліції. Наші історики зараз працюють з
архівом та підбирають чеських (і словацьких) художників, репресованих
комуністичним режимом. Є багато справ і багато архівних документів, наприклад,
фотографії з відеоспостереження. Певен, ми виберемо цікаві справи, які будуть
цікаві людям з України. Ми намагаємось знайти певні зв’язки того часу між
художниками з Чехословаччини та радянської України та Грузії, але проблема в
тому, що того часу не було стільки перехресть", --- пояснює Степан Черноусек з
організації Gulag.cz.

"Попередньо ми хочемо включити до проєкту Сергія Параджанова. В архівах
збереглося не так багато документів про нього, але ми знайшли чудові фото, які
співробітники спецслужб не могли точно інтерпретувати", --- додає Андрій.

"Не дивлячись на те, що в пострадянські часи в грузинському суспільстві завжди
був інтерес до репресованих художників і письменників, у нас все ще лишився
загальний стереотип про творчу інтелігенцію як головну мішень масового терору. 

Ми розуміємо, що громадськість у Грузії та інших країнах не до кінця розуміє,
наскільки складними були стосунки між митцями та радянською владою та яким було
місце мистецтва в кампаніях масового терору. Тож ми хочемо показати всі ці
темні плями тоталітарного минулого на прикладі десяти представників
грузинського мистецтва, які були жертвами радянського терору або зазнали
гноблення з боку режиму", --- каже Іраклі Хвадагіані, представник SovLab.

\ifcmt
pic https://life.pravda.com.ua/images/doc/5/2/5243cb0-pic62.jpg
\fi

\subsubsection{Про те, чому на митців тих часів не можна дивитися лише з одного боку}

Всі герої майбутньої виставки --- митці, які потрапили у поле зору КҐБ. Це
означає, що вони вже виходили за ті межі, які ставили спецслужби. З іншого
боку, бути митцем у ті часи вже було приводом для відкриття справи. В архівах,
наприклад, є три справи на Остапа Вишню --- спостереження за ним, кримінальна та
справа на "агента Вишню". 

"При цьому, слід розуміти, що у Вишні не було шансу не бути агентом КҐБ і не
потрапити до Сибиру вдруге. Він стояв перед вибором --- або він живе на волі та
співпрацює зі спецслужбами, або відправляється у заслання. І тут постає
моральне питання. Адже в суспільній свідомості агент КҐБ --- це погана людина.
Але ж треба розуміти умови, в яких тоді опинялися люди, та те, як вони
виконували свої обов'язки. 

Наприклад, Остап Вишня як агент КҐБ так нікого і не здав", --- розповідає Андрій.

\subsubsection{Про важкість роботи з документами КҐБ}

У протоколах допитів митців, яких було репресовано у 30-х роках, дуже часто
можна прочитати те, що хотів записати слідчий, а не те, що мав на увазі
підслідний. Також слід враховувати те, що все могло бути сказано під тиском,
або вирване з контексту.

Наприклад, в архівах є документи, які підтверджують, що співробітники КҐБ
розстріляли Леся Курбаса за слова, вирвані з контексту. Більш того, вони знали
про це.

\ifcmt
pic https://life.pravda.com.ua/images/doc/f/e/fe74790-pic57.jpg
\fi

\subsubsection{Про спадок радянської України}

Взагалі, як бути зі спадком радянської України? Це наше чи не наше? Ми точно не
маємо забувати про те, що було, але повинні розібратися у тому, як з цим бути.

Через документи спецслужб ми можемо дізнатися, яким чином розвивалося наше
мистецтво та в які рамки воно було поставлено. Адже саме спецслужби дуже чітко
фіксували, що було з митцями, які перетинали межі дозволеного.

Бо коли кажуть про те, що у СРСР мистецтво розвивалося, нам потрібно розуміти,
якою була ціна цього розвитку.

"Зараз неможливо уявити, як це було, але треба враховувати тиск на культуру.
Розуміти, як функціонувала система і як це --- бути митцем у СРСР. Всі люди
мистецтва в ті часи усвідомлювали, що в будь-який момент їх можуть "прибрати",
--- каже Андрій.

Фото: Даша Дот
