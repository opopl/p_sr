% vim: keymap=russian-jcukenwin
%%beginhead 
 
%%file 16_03_2023.fb.kipcharskij_viktor.mariupol.1.r_k_tomu__den_21__16.cmt.1
%%parent 16_03_2023.fb.kipcharskij_viktor.mariupol.1.r_k_tomu__den_21__16.cmt
 
%%url 
 
%%author_id 
%%date 
 
%%tags 
%%title 
 
%%endhead 

\iusr{Елена Девина}

Саме в цей день, 16 березня техніка ворога наблизилася ближче до нашіх
домівок. Вдень стояв гул від її пересування. А коли стало темно, побачили їх
великий ліхтар метрів за 400 від нас. Попереду була найстрашніша ніч... Про Юру
Єрмака дізналися вже пізніше... Страшна смерть... Світла пам'ять... Він був
справжнім! Вже пізніше дізналися, що ворожій снайпер лишив життя ще одного
нашого туриста - Юрія Демченка... Царство небесне...

\iusr{Светлана Водзянская-Живогляд}

Драмтеатру не стало ще вранці, на початку 11 ми приїхали до нього як місця
збору, на під'їзді бачили людей які йшли у бік Автовокзалу по Куїнджі хтось
геть чорний у копоті, хтось білий у пилюці повність, тих людей єднав погляд -
повністю пустий без ознак розуму ..

\begin{itemize} % {
\iusr{Віктор Кіпчарський}
\textbf{Светлана Водзянская-Живогляд} 

Так, читав, що першу бомбу на Драм скинули о 10:05. За моїм щоденником, ми
виїхали від Нептуна от 9:45. Може виїхали о 8:45?

Але ж на Нептун бомбу скинули за годину до Драма.

Щось хронологія не збігається.

Я шукав час бомбування Нептуна в Драма у соцмережах і поки що не знайшов.

Навздогін: щоденник я писав "щовечора" але це не виключає можливість помилки.

\iusr{Светлана Водзянская-Живогляд}
\textbf{Віктор Кіпчарський} 

ні, скинули раніше 10 це точно!!! Я ближче до вечора напишу свої спогади, бо ми
жили ближче і я чула ТІ літаки...

\iusr{Віктор Кіпчарський}
\textbf{Светлана Водзянская-Живогляд} 

Коли я прочитав про 10:05 - я почав шукати в Інтернеті. Шукаю вже третю годину.
Буду вдячний за допомогу: якщо щось знайдете - киньте мені.

Пишу Вам у Месенджері.

\iusr{Марина Солошенко}
\textbf{Віктор Кіпчарський} про драмтеатр я спочатку впізнала від тебе, по новинам сказали в середині дня, був шок!

\iusr{Віктор Кіпчарський}
\textbf{Марина Солошенко} Я вже писав, що якби ми дивилися новини...

\iusr{Віктор Кіпчарський}
\textbf{Светлана Водзянская-Живогляд} Тут написано, що Драм знищили близько 10-ї:

\href{https://dn.gov.ua/news/16-bereznya-rokovini-znishchennya-rosiyanami-mariupolskogo-dramteatru}{%
16 березня - роковини знищення росіянами Маріупольського драмтеатру, dn.gov.ua, 16.03.2023%
}

\ifcmt
  tab_begin cols=3,no_fig,center,separate

     pic https://i.paste.pics/f84c49251226c550826cad5ea80a6552.png
		 pic https://i.paste.pics/6289f64ff86a92dde23c4c9b05cc78f1.png

  tab_end
\fi

Тобто я таки помилився.

\iusr{Светлана Водзянская-Живогляд}
\textbf{Віктор Кіпчарський} у цей час я була біля Жотневого РОВД, і не чула настільки потужного вибуху, вибух страшної сили був о 8:30-8:50

\iusr{Светлана Водзянская-Живогляд}

Хоча в цей час постійно працювала авіація - центральний ринок, пішохідний
переход, 1000 дрібниць. Весь цей шлях був у густому диму і гуркіт літаків не
стихав ні на мить здається

\iusr{Віктор Кіпчарський}

В мене помилка - Нептун знищили після Драма.

Можу виправдатися лише тим, що на той час ми майже виїхали.

\iusr{Светлана Водзянская-Живогляд}
\textbf{Віктор Кіпчарський} 

всьорашній день я закінчила стукотом у кришку підвалу...

Чоловік встав, відкрив кришку і до майже повній темряві у світлі садового
ліхтарика спускається мій хрещений, а хто за ним я не бачу доки він не
відходить у бік. Це мій брат!!! Боже я навіть очі протерла і кинулася на шию,
вдихнула запах парфумів на куртці і диму і тільки потім зрозуміла що він у
цивільному одязі. Наступні пів години він розповідав новини що у хлопців
позавчора розбомбили базу-сховище бункерною бомбою, там були всі залишки їжі,
боєприпаси... і моє сердце боліло саме в той час коли мій брат був у тому районі
і намагалися стримати прорив. Припасів і їжі більше нема, хлопці не їли 2 дні.
Командир загинув ще 4 дні назад, їх бригада вирішила по одному виходити з
оточення, і посилити групу підкріплення( всі вірили що скоро звільнять місто).
Розповів що дізналися про плани 16 бомбити саме Кірова, тому треба їхати,
спитала чі надовго? Сказав пара тижнів, максимум місяць... Насипала у каструльку
борщу і вони пішли збиратися до себе. На дворі ніч, чоловік каже лягай помпи бо
ти не спала попередню ніч, а потім будешь збиратися, крутилася з годину, ні не
можу я спати, гуркіт літаку та думки не дають дихати. Взяла наложниць ліхтарик
і почала збиратися, не знаю що я збирала 6 годин, все було як у ваті, ходила
наче мене прибило, збирала речі лише ті що покраще, навіть постільну білизну що
отримала 20 лютого ще закриту, медикаменти, зарядні нащось лишила останніми.
Часто сідала і відпочивала так наче під час марафону, чомусь легше було цеглу у
дворі перекладати ніж збирати дитячі футболки і штани. О 9 годині домовились з
братом піти дізнатися на Драмтеатр про «зелений коридор»

\iusr{Светлана Водзянская-Живогляд}

Десь о 8 прокинулися діти, чоловік сказав зібрати іграшки, щоб не лякати дітей
нічого не сказали доки не сказали бігом сідати в машину. Дав їм горіхи та
відкрив компот, пішов збирати свої інструменти. Я на кухні збираю свій
кондитерський скарб, чую літак і далі вибух настільки потужний що будинок
підскочив, а склопакет на вікні біля якого я стояла надавлять як гумова кулька,
думала що всьо мені, бо з мене би вийшов дрібний фарш, якби воно не витримало.
Я гадаю що це і був момент знищення Драмтеатру, бо це був найпотужніший вибух
того ранку. Це було 8:30-8:50, точніше не скажу. На початку 10 прийшов брат з
побратимом, сказали що ідуть з моїм чоловіком, вмовила що йду я бо 3 чоловіка
викличуть занепокоєння в людей. Вийшли, на вулиці холодно, з носа тече «юшка»
бігти через це важко бо дихаю ротом. Тільки спустилися з пагорба до сгорілої 9
поверхівки чую свист, прямо над головою пролітаю міна і влучає у горище
будинку, на голову летить каміння, зраділа що вдягнула товсту шапку, побігли
далі ще швидше, по дорозі вирви від авіа бомб, градів, мін, частина будинків
згоріла, машини повз проїздять на шаленій швидкості і в білих стрічках. Дійшли
до Жовтневого РОВД і зрозуміли що йти далі це гаяти час, люди відвертаються і
не хочуть розмовляти, бачили як заводили машину яку придавило шматком плити
зверху, плиту зняли, водій сів майже боком, назад запхали 2 дітей і жінку, бо
місце біля водія було ще більше сплющене, їх проводжали старі люди, мабудь
батьки, всі плакали, ми допомогли виштовхати машину щоб вона завелася. Сказали
що їдуть просто в нікуди через точку сбору, в іншій машині чоловік побачив
хлопців і не схотів розмовляти, від відчаю розплакалася просто благала
розказати що й до чого бо мені треба вивезти дітей, розповів що теж їде на
Драмтеатр, бо вчора був там, було дуже багато людей всередині, ті хто мали
машини збиралися їхати о 9 колоною.

\iusr{Светлана Водзянская-Живогляд}

Вирішили що цього досить, обстріли посилюються, треба прискоритися в виїзжати.
Прибігла додому в побачила ефект гальма на прикладі свого чоловіка, надписів на
машині нема, речі так і стоять у коридорі, він гріє борщ і нікуди не поспішає...
трохи не кинулася з кулаками, брат заспокоїв, Сергій (побратим брата) почав
писати надписи, я шукати простирадло, кидати ще якісь речі в торби, просити
сусідів кормити тварин, віддала кумам ключі (моя велика помилка, бо мій будинок
вони добряче пограбували, винесли все що їм могло знадобитися чі просто
сподобалося), а тварини лишилися голодними бо їх не годували, і одна тварина
згодом померла від голоду.

\iusr{Марина Солошенко}
\textbf{Светлана Водзянская-Живогляд} 

ви пережили таке лихо! Остались живі! Все прийде, не шкодуйте за речами і тоді
все повернеться з лишком, повірте! Тварин шкода, у мене собака тулиться до ніг
при кожному розриві ракети, як би хоче захисту, як їй зрозуміти що війна?!

\iusr{Светлана Водзянская-Живогляд}

Покидали речі, чоловік посадив дітей, за кермо моєї машини сів Сергій, я з
дітьми. У цей час хрещена мала поїхати і забрати бабусю, ми поїхали до її
будинку, заїхали у двір- автобуса нема, бігом назад уздовж трамвайних колій,
оминаючи вирви і обірвані дроти, бо вже бачили кілька машин з пробитими
колесами їх просто кидали і йшли пішки. Побачили наш бусік намагаємося його
наздогнати, сигналили, а вони не чують, раптом постріл і міна лягає у 2 метрах
на дорозі прямо перед бусом і не розривається, він вихряє і лишає її у кількох
см збоку, бо загальмувати просто не встигає, розвернувся і побачив нас, ми
махаємо їхати за нами, по проспекту Металоургів не поїхали бо видно розриви
авіації у районі ринку, там все у чорному димі. Їдемо по Куїнджі, всю дорогу я
забороняю дітям піднімати очі, кричу коли не слухаються і піднімають, мені
страшно що вони побачать все що коється навкруги і всю дорогу ч як мантру кажу
« глаза вниз!! не поднимать!!! Смотри на пол!!» вигляд 36 школи та автовокзалу
викликає жах . на під'їзді до Драмтеатру ми бачимо людей повнівтю чорних, чі
білих наче у борошні, міркуємо що прилетіло у будинок, біля Драмтеатру купа
сміття і .... Далі я вкусила себе за долоню аби не закричати... від театру просто
друзки, на купі каміння що від нього лишився купа людей, когось виносять я
кладуть на асфальт поруч, когось подалі, когось ведуть під руки... чути як хтось
плаче, стогне. Бачимо хвіст колони, точніше 2 автівки, їдемо за ними, бо як
їхати ми не знаємо, через 15 хвилин за нами колона машин 30, навздогнали на
Моряках кінець колони, машини стоять, попереду працює град, над нами низько
літають міни. Вискакую з машини прикріпити білі стрічки-ганчірки на дзеркала.

\iusr{Віктор Кіпчарський}
\textbf{Светлана Водзянская-Живогляд} 

Якби ви не залишили ключі сусідам - невже їх би це їх стримало?

Ми залишили ключі сусідові - він навіть квіти поливає! А ще увечері приходить
до нашої квартири, запалює світло то в одній кімнаті, то в інший, деякий час
сидить та читає книжки - створює враження, що там живуть хазяї.

Вибачався, що закопчені на багатті каструлі виніс до підвали, відмити їх не
вдалося, а у хазяїв такого не може бути - аби не насторожити газовиків та
сантехників які запускали свої системи (опалення).

Сусіди різні...

Навздогін: коли він \enquote{звітував}, я від здивування у присутності своєї дружини
розкрив йому усі свої схованки, де був алкоголь! Після повернення доведеться
влаштовувати нові!

\iusr{Светлана Водзянская-Живогляд}
\textbf{Віктор Кіпчарський} 

не просто сусідам, а кумам з якими ділили навпіл продукти та постійно
допомагали один одному, дивилися дітей одно одного... це дуже боляче було, бо
були як рідні люди.

\iusr{Віктор Кіпчарський}
\textbf{Светлана Водзянская-Живогляд} Чужа людина зрадити не може - тільки близька.
І ще один чийсь афоризм: в темні часи видно світлих людей.
Забудьте про злих, як забувають про видалений зуб, тощо.
Як мене повчав мій найкращий Друг (в мене всі Друзі - найкращі! А інші - то знайомі) - відпусти це.

\iusr{Светлана Водзянская-Живогляд}

Вискочила кріпили «стрічки» на ручки та дзеркала, на наші машини начепили, біжу
до бусу і не бачу скрізь вікно бабусю, ривком відкриваю дверцята, та її все
одно нема і заплакана бліда хрещені дивиться у вічі. «Вона відмовилась,
категорично відмовилась», сказала «вивозьте дітей, а я зі своєї хату. Нікуди не
піду». Поки розуміння дійшло до мозку, пройшло кілька хвилин, було бажання
розвернути машину і їхати забирати, навіть якщо буде відпиратися і перемогла
повага, бо мої бабуся і дідусь заслуговують поваги і їхні рішення завжди
ВІРНІ!! Сіла в свою машину, за звичкою сховалася в свій схронок, за кермом
Сергій, він розуміючи поклав руку на плече.

Колона почала рухатися потроху, уздовж дороги йдуть люди, з клунками,
велосипедами, валізами, трохи подалі уздовж дороги хрести, в одній з вирв
(видно що зовсім свіжа) на тачці привезли трупи у простирадлах, у іншій
вигружають повний багажник, жінка кріпить на дошку папірець у файлі з іменами,
один файл на дошку, інший на груди… Виїхали з міста, блок- пост розбитий, за
містом людей трохи більше йде, машини в кого є місця зупиняються і підбирають
людей з дітьми та стареньких ( в них найменше шансів дійти), дме холодний
вітер, пронизує до кісток. Дивлюся байдуже у вікно, переварю свій внутрішній
біль, тут мене наче Бьюти у груди, йдуть наче звичайні люди і у жінкі трохи
попереду з шарфу визирає маленький носик, він білий від холоду, кричу «СТІЙ»,
Сергій звичний до команд нажимає на гальма, вистрибують з машини, йде жінка а
на руках річний малюк, в неї куртка розкрита і вона намагається гріти ніжки
малюка, а він так змерз що навіть не кричить, просто течуть сльози маленькі з
оченят які мама накрила своєю шапкою поверх його. Кричу щоб сідала в машину,
бігом бо працює град і колона зупинилася, вона з чоловіком, мамою і свекрухою,
забираємо її до себе, чоловік сідає до мого чоловіка, бабусі у бусик.

Оля (так звуть дівчинку) з синочком йшли вже кілька годин, їх трохи підвезли
від Драмтеатру до виїзда з міста, далі йшли пішки, вибачилася і приклала малого
до грудей, бо він голодний, а кормити на такому протязі небезпечно. Малюк поїв,
зігрітився і заснув, а Оля розповідала як вони вижили. Та розказала як вона у
голос йшла і молилися по дорозі в голос, бо малий їсти хоче, всі змерзли і ноги
вже не йдуть від голоду. -«тільки сказала «Амінь» і бачу різко зупиняється Ваша
машина».

\iusr{Светлана Водзянская-Живогляд}

В'їхали в Маншуш і все.. годин 7 тягнучка, 3 пости попереду, машини оглядають.

Діти їдять печиво що лишилося у чоловіка у рахунок зарплатні з роботи,
запивають холодною колодязьною водою. Знайшли пів палки сухої ковбаси, діти
їсти без хліба не хотять, пройшла 10 машин вперед и біля 20 машин назад, хліба
нема ні в кого, у декількох машинах люди сказали що не їли кілька днів вже,
взяла печиво- віднесла, дякували і плакали. Трохи подалі до машини підійшов
чоловік, виявилося що він живе поруч, пропонував зайти випити гарячого чаю чи
зігрітися, набрати кип'ятку, казав що є гаряча каша вівсяна, більше нічого не
має, але хто голодний то нагодує чим є безкоштовно.

Далі був 1 пост.

В машину зазирнув ДНРовець, автомат дивиться у лоба, і з посмішкою питає
документи, подивився, віддав і питає «вам щось потрібно?». Сергій вчепив мене
за руку до синця, бо я була готова наговорити до розстрілу... чоловік каже що в
них теж питали, та брудними руками малому простягав галєтне печиво, малий
закричав «НЕ» і розплакався. Відпустив. Далі ще постими підїзжали перші і
казали що ми 2 машини та бус -одна сім'я, 5 дітей, 5 кішок, 5 собак і бабусі.
Дивилися сумки, багажник, на багатьох Оля діставала грудь і давала малому чи
щіпала його аби він плакав і нас відпускали. Страшний пост був на Бердянську,
там я посивіла..

\iusr{Віктор Кіпчарський}
\textbf{Светлана Водзянская-Живогляд} Коли це було?

\iusr{Светлана Водзянская-Живогляд}

Туди ми під'їхали на початку 1 ночі, у обличчя святять потужним ліхтарем,
солдати там злі і наче під наркотою, кричать, сміються, принижують, хамлять.
Нашу машину подивились, пореготали з Оліних грудей, хвилин 10 доглядали машину
з чоловіком і дітьми, я навіть вийшла з машини спитати що хочуть. Наставили 2
автомати, наказали сісти в машину бо зараз розстріляють обидві машини, сіла,
наказали обом машинам їхати далі щоб нас не бачили. Поїхали, а бус стоїть і
стоїть, я бачу що 2 чоловіків вивели з автобусу- хрещеного і брата і кудись
повели. Хвилин 10 пройшло, а їх нема!!!! Мене калатає, сердце болить, в голові
стучить, Сергій тримає мене за рукав бо спіймав майже біля посту, я хочу
забрати звідти хрещену і племінницю, не дай Боже відкриють вогонь. Через 5
хвилин почалася істерика, чоловік і Сергій тримали вдвох, ще через 5 хвилин
побачила як роздягнуті лише у спідньому і взутті з речами у руках вони сідають
у буса і дядя Рома дає по газах( на вулиці -10)

Виявилося що в салоні перегоріла лампочка і коли їм сказали увімкнути світло у
салоні то вони намагалися пояснити що не працює, але ніхто не слухав, чоловіків
вивели, повели за будівлю, там біля брудною стіни наказали роздягнутися до
спіднього, шукали татуювання, в брата величезний тигр на грудях, доказував що
це просто художне тату, наказали стати обличчям до стіни( вона у крові),
хрещений попросив дати 10 секунд поцілувати сина на останок, не дозволили,
перекинули затвори і у цю секунду вийшов російський командир і спитав «що
коїться», хрещений сказав що просто вивозить сім'ю з пекла, маму, дружину,
сина,онучку, племінницю з дітьми, підібрали стареньких по дорозі. Військовий
зиркнув на ДНРівців і наказав за 10 секунд прибратися з посту хрещеному з
бусом.

\iusr{Светлана Водзянская-Живогляд}
\textbf{Віктор Кіпчарський} 16 ввечорі

\iusr{Віктор Кіпчарський}
\textbf{Светлана Водзянская-Живогляд} 

Це, напевно, було 16-го, судячи з попередніх дописів.

А час виїзду приблизно по 11-й?

Ви їхали по Шевченка, потім по Куінджі, а потім? Проспектом Миру? До АС-2?

\iusr{Светлана Водзянская-Живогляд}
\textbf{Віктор Кіпчарський} по Куїнджі до театру, вниз до Котовського, понад морем, до Гагарін і до самого виїзду з міста на Мелекине.

\iusr{Светлана Водзянская-Живогляд}
\textbf{Віктор Кіпчарський} так виїзд близько 11 здається

\iusr{Віктор Кіпчарський}
\textbf{Светлана Водзянская-Живогляд}

Виїхали 16-го березня ввечері?

За містом - на Портовське, а звідти на Мангуш?

\iusr{Светлана Водзянская-Живогляд}
\textbf{Віктор Кіпчарський} 

перед Мелекіне на Портовське і потім Мангуш, як шла колона, дехто кудись
звертав, нам було страшно на підьїзді до Мангушу бачили кілька батарей Градів,
вони обстрілювали місто

\iusr{Віктор Кіпчарський}
\textbf{Светлана Водзянская-Живогляд} Ми виїхали на пару годин раніше, коли таких жахів ще не було...

\end{itemize} % }

\begin{center}
\begin{minipage}{\textwidth}
\iusr{Sergey Drovorub}

\url{https://ru.wikipedia.org/wiki/Бомбардировка_Мариупольского_театра}

\ifcmt
  ig https://i.paste.pics/bb78d6bc0599e6662c0c98b3638aa7da.png
  @wrap center
  @width 0.8
\fi

\end{minipage}
\end{center}
