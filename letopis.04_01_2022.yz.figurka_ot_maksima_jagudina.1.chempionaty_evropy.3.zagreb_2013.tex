% vim: keymap=russian-jcukenwin
%%beginhead 
 
%%file 04_01_2022.yz.figurka_ot_maksima_jagudina.1.chempionaty_evropy.3.zagreb_2013
%%parent 04_01_2022.yz.figurka_ot_maksima_jagudina.1.chempionaty_evropy
 
%%url 
 
%%author_id 
%%date 
 
%%tags 
%%title 
 
%%endhead 

\subsubsection{Чемпионат Европы 2013 (Загреб)}

Закончился 6-летний период без женских российских медалей на чемпионатах
Европы. Очень легко запомнить - женщины стали снова выигрывать медали, когда
мужчины перестали завоевывать золото с окончанием карьеры Плющенко.

\ii{04_01_2022.yz.figurka_ot_maksima_jagudina.1.chempionaty_evropy.3.zagreb_2013.pic.5}

Костнер в пятый и последний раз становится лучшей в Европе. При этом 16-летняя
Аделина Сотникова, которая к этому моменту была уже трехкратной чемпионкой
России, выиграла у Каролины по технике и короткую, и произвольную, но при этом
проиграла в сумме по компонентам почти 8 баллов.

\ii{04_01_2022.yz.figurka_ot_maksima_jagudina.1.chempionaty_evropy.3.zagreb_2013.pic.6}

Несмотря на лишь 3 место больше всего поразила конечно же еще одна дебютантка
главных взрослых стартов 16-летняя Лиза Туктамышева, которая набрала по технике
в произвольной невиданные на тот момент 70 баллов с 7 уверенными тройными
прыжками и каскадом 3+3 лутц-тулуп. 

По сегодняшним временам мы бы ждали на Чемпионате Мира от наших двух молодых
звездочек как минимум медалей, но тогда все было по другому. На Чемпионате Мира
в Лондоне Аделина с Лизой выступили неудачно, с большим количеством ошибок и
расположились друг за другом на 9 и 10 местах, но самое обидное из-за этой
неудачи наши девушки в итоге участвовали в Олимпийском сезоне лишь с двумя
квотами. 

