% vim: keymap=russian-jcukenwin
%%beginhead 
 
%%file slova.golova
%%parent slova
 
%%url 
 
%%author_id 
%%date 
 
%%tags 
%%title 
 
%%endhead 
\chapter{Голова}

%%%cit
%%%cit_head
%%%cit_pic
%%%cit_text
Лестница заскрипела, и Корнуэлл потянулся к рукоятке меча. Вихрь сена пронесся
мимо. Краем глаза он увидел, как Енот с выпущенными когтями прыгнул на
\emph{голову}, показавшуюся в отверстии сеновала. Енот опустился на эту голову,
и послышался приглушенный крик. Корнуэлл прыгнул к лестнице и начал быстро
спускаться. На полпути он уловил блеск вил, но увернулся. У основания лестницы
человек, поднимавшийся по ней, пытался освободиться от Енота, который вцепился
в голову и лицо своей жертвы. Свободной левой рукой Корнуэлл вырвал вилы у
этого человека.  Три кричащие фигуры устремились на него, и одна из них
выхватила меч. Корнуэлл отвел назад левую руку с вилами, а потом с силой бросил
их вперед. Держа перед собой меч, он побежал навстречу врагам. Щит все еще
висел у него за спиной, ему некогда было взять его в руки
%%%cit_comment
%%%cit_title
\citTitle{Зачарованное паломничество}, Клиффорд Саймак
%%%endcit

%%%cit
%%%cit_head
%%%cit_pic
%%%cit_text
Клеонові II не було діла до прецедентів. Скільки б їх не трапилося в минулому,
йому не ставало від цього ні крапельки легше. Не втішала його і думка про те,
що його прадід командував зграєю розбійників на якійсь третьосортній планеті, а
він, Клеон II, живе в розкішному палаці Амменетіка Великого і є спадкоємцем
давньої імператорської династії. Не радувало Клеона і те, що його батько
вилікував Імперію від міжусобиць і встановив в ній мир, подібний до того, яким
вона насолоджувалася при Станнеллю VI. Не приносив задоволення спокій у
віддалених провінціях, нічим не порушуваний ось уже двадцять п'ять років.
Імператор Галактики, Повелитель всього сущого застогнав і відкинув
\emph{голову} на подушки. Трохи заспокоївшись, він сів у ліжку і похмуро
втупився у дальню стіну спальні. Занадто велика кімната, в ній незатишно
одному. Інші кімнати теж занадто великі.  Втім, краще сидіти одному в спальні,
ніж дивитися на чепурних придворних, терпіти їх надто щедрі співчуття, слухати
порожню балаканину. Краще самотність, ніж суспільство неживих масок, під якими
ретельно прораховуються варіанти його смерті і успадкування престолу
%%%cit_comment
%%%cit_title
\citTitle{Фундація та Імперія}, Айзек Азімов
%%%endcit

%%%cit
%%%cit_head
%%%cit_pic
%%%cit_text
Я пішов до свого "Клубу Диких", але перед тим, як увійти туди, сперся на
поруччя Адельфі-террас і, замислившися, втупив погляд у брунатну, масну від
нафти річку. На вільному повітрі моя \emph{голова} завжди краще працює. Я витяг з
кишені список подвигів професора Челенджера і ще раз перечитав його при світлі
електричного ліхтаря. І раптом на мене зійшла ідея. Я чудово розумів, що мені,
як репортерові, не "пощастить знайти спільну мову з вередливим професором. До
того ж, наскільки я довідався з коротенької біографії, він був фанатик науки.
Може, в цьому слід пошукати його вразливого місця? Варто спробувати.  Я ввійшов
до клубу. Тільки-но вибило одинадцяту годину. У великій залі вже було майже
повно людей, хоч нові відвідувачі надходили й надходили. В кріслі проти каміна
я побачив високого, худорлявого чоловіка. Коли я підсунув крісло до нього, він
обернувся в мій бік. То був саме той, кого я найбільше хотів бачити — Тарп
Генрі з редакції журналу "Природа", він завжди радо допомагав усім своїм
знайомим. Не гаючи часу, я завів з ним розмову
%%%cit_comment
%%%cit_title
\citTitle{Утрачений світ}, Артур Конан Дойл
%%%endcit
