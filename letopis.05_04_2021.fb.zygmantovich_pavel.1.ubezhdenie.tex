% vim: keymap=russian-jcukenwin
%%beginhead 
 
%%file 05_04_2021.fb.zygmantovich_pavel.1.ubezhdenie
%%parent 05_04_2021
 
%%url https://www.facebook.com/zygmantovich/posts/3605777122867448
 
%%author 
%%author_id 
%%author_url 
 
%%tags 
%%title 
 
%%endhead 

\subsection{ПОЧЕМУ МЫ ВЕРИМ В ТО, ВО ЧТО ВЕРИМ}
\label{sec:05_04_2021.fb.zygmantovich_pavel.1.ubezhdenie}
\Purl{https://www.facebook.com/zygmantovich/posts/3605777122867448}

\ifcmt
  pic https://scontent-amt2-1.xx.fbcdn.net/v/t1.6435-9/166827417_3605776732867487_7073895421346230632_n.jpg?_nc_cat=111&ccb=1-3&_nc_sid=730e14&_nc_ohc=V3gnuIRKLnoAX8DR7Ub&_nc_ht=scontent-amt2-1.xx&oh=8a6ffacf61166f5c816e0f77d739d839&oe=608EE798
  width 0.4
\fi

Вопрос, который мне часто задают — «Как переубедить человека?»
Предполагается, что переубеждаемый где-то набрался какой-то гадости, поверил в неё и теперь его близкие от этого страдают. Давайте оставим за скобками выяснение истинны — кто на самом деле верит в какую-то гадость. Просто зафиксируем — такой вопрос есть, люди хотят переубеждать других людей.
Общий ответ на этот вопрос таков: по большому счёту — никак. Есть, конечно, разные мелкие приёмы, но в целом переубедить человека крайне трудно.
Потому, например, что огромную роль в нашей убежденности играют эмоции — особенно положительные. Если при первом знакомстве информация вызывала сильные положительные эмоции — пиши пропало. Изменить отношение к этой информации (сиречь убеждения) станет очень трудно.
Представьте себе такой эксперимент — участники в случайном порядке разбиваются на подгруппы и получают текст для прочтения. В одной группе это текст о вымышленном водном животном, в котором описан его ареал, способ размножения, средний размер и другие энциклопедические сведения.
В друго подгруппе читают текст о том же вымышленном животном. Но — в виде истории об игре этого животного с ныряльщиком.
Нетрудно предсказать итог эксперимента — положительное отношение к зверюшке было больше у второй группы. И именно у этой группы оное отношение было крепче.
Именно такой результат получили учёные-психологи из США, Мэтью Д. Роклейдж и Эндрю Латтрелл.
Вы скажете, что выдуманная животинка — это ерунда? Они с вами согласятся. И покажут другие аналогичные эксперименты.
Например, отношение к подарку на Рождество было тем крепче, чем сильнее были эмоции при получении оного.
То же самое проявилось, когда исследователи изучили онлайн-отзывы — чем больше эмоций в первом отзыве, тем устойчивее отношение к продукту или компании.
Что всё это значит? Что наша убеждённость в чём-либо далеко не всегда основана на размышлениях, логике и трезвых выводах.
Часто (если не почти всегда) основой становится эмоция. Мы как бы говорим себе: «Это сообщение меня очень удивило — удивление сильное — я поверю в это сообщение». А про что-то другое: «Эта информация меня не задела — я ничего не чувствую — не стану в это верить». 
Скажем, прочитал человек какую-нибудь теорию заговора. Она его впечатлила, пощекотала нервы. Он крестцовым отделом мозга тут же заключает — это всё правда, вон ведь как нервы щекочек, какие эмоции вызывает!...��Разубеждать его теперь — почти невыполнимая задача...
Разумеется, исследование не означает, что наши взгляды никогда не меняются. Меняются, не волнуйтесь. Исследование о том, что не все наши взгляды основаны на достоверных фактах. Да и вообще — на фактах.
Следовательно, стоит быть со своими взглядами поосторожнее. Чего вам и желаю.
***
Другие интересные исследования — #доказательнаяпсихология_Зыгмантович
Источник: Rocklage MD, Luttrell A. Attitudes Based on Feelings: Fixed or Fleeting? Psychological Science. 2021;32(3):364-380.
