% vim: keymap=russian-jcukenwin
%%beginhead 
 
%%file 27_10_2022.stz.news.ua.donbas24.1.chy_ljubyv_marko_kropyvnyckyj_priazovja_stvorennja_ukr_realistic_teatr.intro
%%parent 27_10_2022.stz.news.ua.donbas24.1.chy_ljubyv_marko_kropyvnyckyj_priazovja_stvorennja_ukr_realistic_teatr
 
%%url 
 
%%author_id 
%%date 
 
%%tags 
%%title 
 
%%endhead 

\ii{27_10_2022.stz.news.ua.donbas24.1.chy_ljubyv_marko_kropyvnyckyj_priazovja_stvorennja_ukr_realistic_teatr.pic.front}

\begin{center}
  \em\color{blue}\bfseries\Large
Трупа Марка Кропивницького досить часто виступала як в Маріуполі, так і в навколишніх селищах
\end{center}

27 жовтня 1882 року в Єлисаветградському театрі (нині Кропивницький) відбулась
перша \emph{\textbf{вистава українського театру}} з тріумфальної постановки \enquote{Наталка Полтавка}
з Марією Заньковецькою у головній ролі. 27 жовтня й досі вважається Днем
народження \textbf{\emph{першого професійного національного реалістичного театру на
українських землях}}. Згодом театральна трупа і її філії об'їздили майже всю
територією Південної України з виставами. У 1885 р. трупа розпалася на дві
частини: однією керував \textbf{Марко Кропивницький}, другою — \textbf{Михайло Старицький}. За
їхнім прикладом пішли інші і вже через кілька років на Україні діяло понад 30
мандрівних театрів. Виявляється, видатний український письменник, драматург,
театральний режисер і актор Марко Кропивницький полюбляв приїжджати до
Маріуполя та його околиць, адже приазовські глядачі обожнювали вистави
українського реалістичного театру.

\ii{insert.read_also.demidko.donbas24.teatromania_rozv_pidtrym_kulturu_mrpl}
