% vim: keymap=russian-jcukenwin
%%beginhead 
 
%%file 25_12_2021.stz.news.ua.volyn.1.chess_ukrainec_pobeda
%%parent 25_12_2021
 
%%url https://www.volyn.com.ua/news/201741-iunyi-ukrainets-stav-svitovym-chempionom-z-shakhiv
 
%%author_id 
%%date 
 
%%tags 
%%title Юний українець став світовим чемпіоном з шахів 
 
%%endhead 
\subsection{Юний українець став світовим чемпіоном з шахів}
\label{sec:25_12_2021.stz.news.ua.volyn.1.chess_ukrainec_pobeda}

\Purl{https://www.volyn.com.ua/news/201741-iunyi-ukrainets-stav-svitovym-chempionom-z-shakhiv}

\begin{multicols}{2} % {
\setlength{\parindent}{0pt}

\ii{25_12_2021.stz.news.ua.volyn.1.chess_ukrainec_pobeda.pic.1}

Юний українець став світовим чемпіоном з шахів. Він здолав гросмейстера-рекордсмена із США.

Український шахіст Ігор Самуненков здобув сенсаційну перемогу в Суперфіналі
юнацького Гран-прі серед спортсменів до 12 років, передає «Суспільне».

\begin{zznagolos}
Українець переміг завдяки результату в очній зустрічі.	
\end{zznagolos}

Як пише «Суспільне», цей турнір став завершальним у циклі змагань з рапіду
серед кадетів та юнаків у цьому сезоні.

Серед майже 3000 учасників до Суперфіналу відібрали по шість найкращих з кожної
вікової категорії.

Фаворитом відкритих змагань у категорії U12 був 12-річний американець Абґіманью
Мішра, який цього року побив рекорд Сергія Карякіна, ставши наймолодшим
гросмейстером в історії спорту.

Українець переміг завдяки результату в очній зустрічі.

Ігор Самуненков (Україна) — 6,5;

Абґіманью Мішра (США) — 6,5;

Іван Зємлянскій (Росія) — 3,5.
\end{multicols} % }
