% vim: keymap=russian-jcukenwin
%%beginhead 
 
%%file 27_09_2021.fb.bojechko_andrej.1.istoria_z_usikom
%%parent 27_09_2021
 
%%url https://www.facebook.com/Boiechko/posts/4976520845710318
 
%%author_id bojechko_andrej
%%date 
 
%%tags __sep_2021.usik.pobeda.dzhoshua,donbass,usik_aleksandr,vojna,volonter
%%title Історія з Усиком
 
%%endhead 
 
\subsection{Історія з Усиком}
\label{sec:27_09_2021.fb.bojechko_andrej.1.istoria_z_usikom}
 
\Purl{https://www.facebook.com/Boiechko/posts/4976520845710318}
\ifcmt
 author_begin
   author_id bojechko_andrej
 author_end
\fi

Є в мене кілька історій, про які я ніколи нікому не розповідав.

Історія з Усиком - одна з них.

Про неї знає лише Аліна Шатернікова, яка і познайомила нас  з Сашком вже  в
далекому  2015 році.

Моя волонтерська група готувала  тоді перше нагородження орденом Народний Герой
України, і щотижня на фронт ми відсилали сотні кілограмів продуктів, одягу
медикаментів та різного специфічного військового обладнання. 

Щодня дівчата відвідували поранених в Київському військовому госпіталі,
годували хлопців, одягали і всіляко допомагали морально та психологічно.

Саме в цей час «Українські отамани» брали участь у Всесвітній серії боксу, за
якою я уважно стежив на Фейсбук сторінці Аліни Шатернікової.

Ми знайомі ще з перших Ігор Патріотів, інколи переписувалися, тому на початку
лютого 2015-го, відразу після Дебальцево, коли в госпіталь щодня почали
прибувати десятки поранених, я написав їй коротке повідомлення - «Аліна, чи не
могла б ти запросити наших бійців на один з поєдинків Отаманів? Думаю, що їм
інколи треба «відволікатися» від лікування в госпіталі.»

За кілька днів надійшла відповідь - «Я все зробила. З радістю допоможу, всі
хлопці і Федерація підтримали, маєш 22 квитки на зустріч з кубинцями. Перед
боєм зустрінетесь з командою,  поспілкуєтесь і отримаєте подарунки. Крім того,
Олександр Усик також хоче познайомитися з бійцями  і підтримати їх.»

На той вечір нам дали найкращі місця в залі Палацу Спорту, Аліна нас зустріла і
познайомила з командою, ведучий вечора оголосив всім глядачам, що в залі за
боями спостерігають Герої війни, один з бійців навіть вийшов на ринг і побажав
Отаманам перемоги.

А перед боєм Аліна вручила кожному з боксерів знаменитий оберіг нашої місії
Народний Герой  - підвіску «Тризуб».

Ми провели чудовий вечір, трохи запізнилися в госпіталь, за що я отримав
«сувору догану» (більше ніколи з вами поранених не відпустимо, вони пропустили
вечерю і прийом ліків) від керівництва, але всі ми були тоді  щасливі  та
вражені підтримкою і шанобливим ставленням до воїнів не лише з боку боксерів,
але й усієї зали. Хлопцям довго аплодували стоячи.

Ми зустрілися не лише з командою Отаманів, які виходили в той вечір на ринг,
але й встигли познайомитися та поспілкуватися з Сашком Усиком, з Василем та
Романом Вірастюками та іншими відомими на весь світ спортсменами.

Хлопці були в захваті і ще довго згадували той вечір.

Наступного після бою ранку один з бійців написав мені повідомлення - «Андрію, я
дуже давно є фанатом Олександра Усика, чи не могли б ви попросити його залишити
для мене автограф, учора було якось ніяково просити…»

Я відіслав це повідомлення Аліні. Вона відповідно Сашкові.

В той же день отримав відповідь - Сашко підписав  бійцеві футболку і передав
тобі дещо для Народного Героя. 

Увечері ми зустрілися. «Дещо» виявилося грошима і різними сувенірами з
підписами Олександра Усика.

В свою чергу я попросив Аліну передати Сашкові, як нагороду, нашу підвіску і
побажати успіху в майбутньому поєдинку.

Вже наступного дня вона це і зробила.

З того часу минуло вже 6 років.

Як кажуть, багато води спливло… 

Чи змінився за цей час Сашко?

Я вважаю, що ні. Як був відвертим і чесним, так і залишився.

\ifcmt
  tab_begin cols=4

     pic https://scontent-yyz1-1.xx.fbcdn.net/v/t1.6435-9/243149486_4976479042381165_1189552192483462257_n.jpg?_nc_cat=111&_nc_rgb565=1&ccb=1-5&_nc_sid=8bfeb9&_nc_ohc=Z9t-F98U1b4AX8FTjqD&_nc_oc=AQn2FMFbPdOz0R1d0pv9AgeWqGDjAfA_XuOB7VJ5bi5FZPICGQCPY37xqBZj7TqGFNs&_nc_ht=scontent-yyz1-1.xx&oh=1a038bc12ed41f0d675e3f5a64423dbd&oe=617E167C

     pic https://scontent-yyz1-1.xx.fbcdn.net/v/t1.6435-9/243170886_4976479292381140_4139564203601566421_n.jpg?_nc_cat=110&_nc_rgb565=1&ccb=1-5&_nc_sid=8bfeb9&_nc_ohc=IgQtIqxHFCsAX8cEcWq&_nc_oc=AQmu2ALvDof31OLmK47ChMA-65Esom0IW1hGrH0Kcz0eBSmRHRTAs70r-TDUxqxfqwQ&_nc_ht=scontent-yyz1-1.xx&oh=231afe45b3ca4ae2c4810390a99cf643&oe=617F077F

		 pic https://scontent-yyz1-1.xx.fbcdn.net/v/t1.6435-9/243300520_4976479409047795_8441165449485658045_n.jpg?_nc_cat=111&ccb=1-5&_nc_sid=8bfeb9&_nc_ohc=5OFXljBiq5kAX8RM9zz&tn=lCYVFeHcTIAFcAzi&_nc_ht=scontent-yyz1-1.xx&oh=b49e953b1a31ec171b6120335db13203&oe=617EB869

		 pic https://scontent-yyz1-1.xx.fbcdn.net/v/t1.6435-9/243148124_4976479149047821_6951824273167516595_n.jpg?_nc_cat=105&ccb=1-5&_nc_sid=8bfeb9&_nc_ohc=TsO_LTxzseIAX8-qpZ7&tn=lCYVFeHcTIAFcAzi&_nc_ht=scontent-yyz1-1.xx&oh=3eaf761f26fc7619f85820d589a8846c&oe=617EB730

  tab_end
\fi

По різному можна ставитися до його відомих висловлювань щодо Криму і «братніх
народів», підтримувати чи засуджувати його ставлення до церкви чи віри (я
вважаю, що він має право на особисту точку зору щодо будь чого), але
спростувати  очевидне неможливо  - він є одним з найвідоміших і найславетніших
українських спортсменів, який назавжди залишиться в історії нашого народу!


\ifcmt
  tab_begin cols=4

     pic https://scontent-yyz1-1.xx.fbcdn.net/v/t1.6435-9/243178830_4976479032381166_4413428741863218840_n.jpg?_nc_cat=110&ccb=1-5&_nc_sid=8bfeb9&_nc_ohc=imJDjC7OWFYAX-HA-m2&_nc_ht=scontent-yyz1-1.xx&oh=1a25253ee12c654884e664abb85b9008&oe=61801172

     pic https://scontent-yyz1-1.xx.fbcdn.net/v/t1.6435-9/243130624_4976503715712031_8528246570895251620_n.jpg?_nc_cat=105&_nc_rgb565=1&ccb=1-5&_nc_sid=8bfeb9&_nc_ohc=qqB-Ze-Uy4wAX9nlqD9&_nc_ht=scontent-yyz1-1.xx&oh=9e9c7495c05bad1441bfd7bdfcd13ff3&oe=617DB57C

		 pic https://scontent-yyz1-1.xx.fbcdn.net/v/t1.6435-9/243194656_4976510162378053_8951435495224563619_n.jpg?_nc_cat=104&ccb=1-5&_nc_sid=8bfeb9&_nc_ohc=3_uUbkj1veYAX8TIAIU&_nc_ht=scontent-yyz1-1.xx&oh=c15f27d05272b90b9ff65630dbf0a746&oe=617FDA23

		 pic https://scontent-yyz1-1.xx.fbcdn.net/v/t1.6435-9/243176757_4976510185711384_4503548586653363709_n.jpg?_nc_cat=108&_nc_rgb565=1&ccb=1-5&_nc_sid=8bfeb9&_nc_ohc=fREUPE8NuB0AX9GfwmX&_nc_ht=scontent-yyz1-1.xx&oh=d36251dc6a00c52e0a1e6d8cfc8a7401&oe=617FE78D

  tab_end
\fi

За останні 5-7 років ніхто не прославив  у світі Україну так, як це зробив
Олександр Усик. 

На його плечі Тризуб, він танцює гапака, він розмовляє (хоча й нечасто)
українською, він  загортається в український прапор  після перемог і виходить
на бої під козацьку пісню! Українську козацьку пісню!

Про нього говорить увесь світ! Про українця Олександра Усика, не малороса, не
росіянина, а про Украіїнця!


\ifcmt
  tab_begin cols=2

     pic https://scontent-yyz1-1.xx.fbcdn.net/v/t1.6435-9/243176757_4976510185711384_4503548586653363709_n.jpg?_nc_cat=108&_nc_rgb565=1&ccb=1-5&_nc_sid=8bfeb9&_nc_ohc=fREUPE8NuB0AX9GfwmX&_nc_ht=scontent-yyz1-1.xx&oh=d36251dc6a00c52e0a1e6d8cfc8a7401&oe=617FE78D

     pic https://scontent-yyz1-1.xx.fbcdn.net/v/t1.6435-9/243130624_4976503952378674_4594156869291806251_n.jpg?_nc_cat=102&_nc_rgb565=1&ccb=1-5&_nc_sid=8bfeb9&_nc_ohc=D3tesEKTLesAX_yAZal&_nc_ht=scontent-yyz1-1.xx&oh=e82255daa755de74634e901d46cfa801&oe=618104C8

  tab_end
\fi

І мені цього цілком достатньо, щоб поважати його і дякувати за його працю і
звитягу.

Учора він створив сенсацію! Чесно, мало хто вірив, що він переможе Джошуа. Але
він переміг, і переміг переконливо та яскраво! 

Всі ЗМІ та інформаційні агенції  світу написали про цей бій і досі обговорюють
його. 

На шпальтах усіх найвідоміших газет та журналів світу в один день знову
з‘явилася Україна!

Хіба не це є тим маркером, який робить з людини героя? Лише завдяки Усику світ
сьогодні говорить про славетного українця, який здійснив диво. 

От лише один з прикладів.

BBC. « Українець Олександр Усик шикарно розібрався з Джошуа і назавжди вписав
своє ім'я в історію боксу, повернувши українцям славу королів суперважкої ваги,
яку завоювали брати Клички».

Майте повагу до нього, зробіть хоча б тисячну долю того, що він зробив для
України, тоді й засуджуйте. 

Я пишаюся тим, що знайомий з ним,  і тим, що колись давно вручив йому підвіску
Тризуб. 

Впевнений, невдовзі він стане Героєм України, бо заслуговує на це звання.

Можете кидати в мене каміння.

Я звик.

Але я маю свою власну думку і відкрито її висловлюю.

Слава Україні!

\ii{27_09_2021.fb.bojechko_andrej.1.istoria_z_usikom.cmt}
