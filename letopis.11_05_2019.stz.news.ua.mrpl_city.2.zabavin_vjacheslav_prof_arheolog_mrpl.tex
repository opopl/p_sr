% vim: keymap=russian-jcukenwin
%%beginhead 
 
%%file 11_05_2019.stz.news.ua.mrpl_city.2.zabavin_vjacheslav_prof_arheolog_mrpl
%%parent 11_05_2019
 
%%url https://mrpl.city/blogs/view/zabavin-vyacheslavprofesijnij-arheolog-mariupolya
 
%%author_id demidko_olga.mariupol,news.ua.mrpl_city
%%date 
 
%%tags 
%%title Забавін В'ячеслав – професійний археолог Маріуполя
 
%%endhead 
 
\subsection{Забавін В'ячеслав – професійний археолог Маріуполя}
\label{sec:11_05_2019.stz.news.ua.mrpl_city.2.zabavin_vjacheslav_prof_arheolog_mrpl}
 
\Purl{https://mrpl.city/blogs/view/zabavin-vyacheslavprofesijnij-arheolog-mariupolya}
\ifcmt
 author_begin
   author_id demidko_olga.mariupol,news.ua.mrpl_city
 author_end
\fi

\ii{11_05_2019.stz.news.ua.mrpl_city.2.zabavin_vjacheslav_prof_arheolog_mrpl.pic.1}

Одна з найбільш загадкових наук – \emph{археологія} – багатьох приваблює можливими
відкриттями та унікальними знахідками. Проте сьогодні, на жаль, не так багато
спеціалістів, які мають потрібну освіту і необхідні навички, щоб професійно
займатися цією цікавою і романтичною наукою. Зокрема, в Маріуполі працює лише
один професійний археолог, який завдяки невичерпному ентузіазму та енергійній
діяльності робить чимало відкриттів на теренах Північного Приазов'я та сприяє
збереженню археологічної спадщини регіону. Продовжуючи відкривати для себе
біографії видатних маріупольських сучасників, пропоную вам, дорогі читачі,
познайомитися ближче з \textbf{Забавіним В'ячеславом Олеговичем} – кандидатом історичних
наук, старшим викладачем, директором музею історії та археології МДУ і
начальником єдиної офіційної археологічної експедиції Маріуполя, яка невтомно
примножує надбання найбільш пригодницької науки – археології.

\ii{11_05_2019.stz.news.ua.mrpl_city.2.zabavin_vjacheslav_prof_arheolog_mrpl.pic.2}

Народився В'ячеслав в Маріуполі, який вважає своєю Малою Батьківщиною. Любов до
історії йому прищепив батько – \textbf{Олег Миколайович}. Коли був школярем, батько
розповідав хлопцю іс\hyp{}торичні легенди і розповіді, наприклад, про Спартака. Хтось
радив юнаку стати лікарем, хтось – інженером, лише Олег Миколайович підтримав
інтерес сина до неприбуткової професії. Мабуть, чоловік згадав, що його свого
часу не підтримали, адже і він хотів стати істориком.

\ii{11_05_2019.stz.news.ua.mrpl_city.2.zabavin_vjacheslav_prof_arheolog_mrpl.pic.3}

В'ячеславу довелося служити на флоті 3 роки. Прийшов в археологію взимку 1986
року, саме тоді він вступив до археологічного гуртка
\href{https://archive.org/details/21_12_2018.olga_demidko.mrpl_city.volodymyr_kulbaka}{Володимира
Костянтиновича Кульбаки}%
\footnote{Володимир Кульбака: видатний маріупольський археолог і улюблений викладач, Ольга Демідко, mrpl.city, 21.12.2018, %
\par\url{https://mrpl.city/blogs/view/volodimir-kulbaka-vidatnij-mariupolskij-arheolog-i-ulyublenij-vikladach}, \par%
Internet Archive: \url{https://archive.org/details/21_12_2018.olga_demidko.mrpl_city.volodymyr_kulbaka}%
},
який став для В'ячеслава вчителем і другом одночасно.
Польові археологічні розкопки розпочав з 1987 року. Відтоді наш герой вирішив
назавжди пов'язати своє життя з археологією.

У 1998 році В'ячеслав закінчив історичний факультет Донецького державного
університету.

Археолог підкреслює, що без підтримки сім'ї – дружини і сина – багато чого не
вдалося б. Пам'ятає, як писав курсові роботи, лежачи на підлозі, а маленький
син повзав на спині.

У Маріупольському державному університеті В'ячеслав Забавін не тільки викладає
археологію, ще й очолює діючу на постійній основі археологічну експедицію,
об'єктами досліджень якої є стародавні пам'ятники археології Північного
Приазов'я і загалом України. Юні археологи разом з улюбленим викладачем з
великим запалом відправляються в археологічні розвідки. І це не дивно, адже
вечірні посиденьки біля вогнища, пісні, розмови,смачна польова кухня і
неймовірна романтична атмосфера не можуть не приваблювати. Може тому літня
археологічна практика для студентів є найбільш улюбленою і очікуваною.
Пощастило, що Північне Приазов'я багате на унікальні знахідки. Саме можливість
робити нові відкриття та бажання спілкуватися з однодумцями допомагають
В'ячеславу Олеговичу не зупинятися на досягнутому. А досягнень вже дійсно
багато. Вченому вдалося відкрити на базі Маріупольського державного
університету унікальний музей історії та археології, завдяки педагогічному
таланту об'єднати навколо себе декілька поколінь юних археологів-початківців.
Водночас археолог щороку невтомно досліджує різні періоди в первісній історії
України, роблячи цінні знахідки. Зокрема, в останні роки на території
Мангушського та Нікольського районів вдалося дослідити поселенські та
поховальні пам'ятки доби бронзи, знайти залишки доісторичного мамонту, безліч
унікальних фрагментів керамічного посуду, крем'яних знарядь праці епохи пізньої
бронзи та багато іншого.

У майбутньому археолог планує систематизувати всі знайдені матеріали і
дослідити декілька курганів доби бронзи, зайнятися курганом під Ялтою (ІІІ-ІІ
тис до н.е.). Надихають В'ячеслава кішки, Азовське море та природа Північного
Приазов'я. Коли творча криза – виїжджає в археологічну розвідку, або
відправляється подорожувати.

Незважаючи на спартанські умови праці та недостатнє фінансування, В'ячеслав
Олегович не втрачає ентузіазму і з великим задоволенням та неабияким завзяттям
продовжує займатися улюбленою справою. Для багатьох колег і студентів він є
справжнім взірцем для наслідування. Це той рідкий випадок, коли людина знайшла
своє покликання і цим надихає інших.

\ii{11_05_2019.stz.news.ua.mrpl_city.2.zabavin_vjacheslav_prof_arheolog_mrpl.pic.4}

\textbf{Улюблені книги:} 

\enquote{Бравий вояк Швейк}, \enquote{12 стільців}, \enquote{Золоте теля},
книги Сті\hyp{}вена Кінга, і, звичайно, наукова література.

\textbf{Курйозний випадок:} 

\begin{quote}
\em\enquote{Пам'ятаю, як зі студентами ходили в археологічну розвідку. Нам
довелось стрибати через пересохлий струмок, а я не дострибнув і по пояс загруз
в трясовині. Потім чекали настання темряви, щоб дістатися електричкою до
Маріуполя}.
\end{quote}

\textbf{Порада маріупольцям:} 

\begin{quote}
\em\enquote{Маріупольцям можу побажати щиро любити своє місто, рідне Приазов'я, вивчати
його історію, не бути байдужими. Оскільки через призму любові до малої
Батьківщини як раз і відбувається формування особистості, як
громадянина}.
\end{quote}

\ii{insert.read_also.demidko.sladkovy}
