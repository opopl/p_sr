% vim: keymap=russian-jcukenwin
%%beginhead 
 
%%file slova.kniga
%%parent slova
 
%%url 
 
%%author 
%%author_id 
%%author_url 
 
%%tags 
%%title 
 
%%endhead 
\chapter{Книга}
\label{sec:slova.kniga}

%%%cit
%%%cit_pic
%%%cit_text
Это мы разговариваем цитатами из \emph{Книг}, мультфильмов и кино, а в глубине души
сочувственно смотрим на тех, кто спрашивает: «А откуда это?».  Это для нас
работала целая индустрия детских киностудий, и лучшие актёры изощрялись в
умении петь, танцевать и изображать наших любимых героев – лишь бы мы росли
нормальными людьми!  Мы ещё умеем дружить и не прощаем предательства.  Мы ещё
помним, что любить человека – отнюдь не синоним слова «спать» с ним.  Мы
пережили развал страны, 90-е, парочку мировых кризисов и несколько воин
(особенно внутренних).  До нас, наконец, дошло, что наши родители делали для
нас всё, что могли. Даже если не делали ничего.  Мы теперь уже точно осознали,
что «круто» – это не клубы каждые выходные, не градусы в стакане, а здоровый
цвет лица, крепкий сон и дорогие люди рядом
%%%cit_comment
%%%cit_title
\citTitle{Советскому поколению повезло родиться до того, как у детей отобрали право выбора}, 
Елена Лукаш, strana.ua, 13.06.2021
%%%endcit


%%%cit
%%%cit_pic
%%%cit_text
Про найдавнішу \emph{рукописну книгу} періоду України-Русі в російських
джерелах, а також у перекладах інформації з російської в англомовних джерелах
можна прочитати наступне: «Остромирово Евангелие – памятник культуры мирового
значения – хранится в Санкт-Петербурге в Российской национальной библиотеке.
Эта \emph{рукописная книга}, созданная в XI в., занимает совершенно особое
место в ряду самых важных памятников культурного наследия, составляющих
бесценное достояние России.  ...Остромирово Евангелие действительно стоит у
истоков русской письменности и культуры». Це інформація з офіційного сайту
Російської національної бібліотеки і в ній представлено дві традиційні для
ідеології «русского мира» міфологеми. Спростуємо їх почергово
%%%cit_comment
%%%cit_title
\citTitle{Чому написане у Києві Остромирове Євангеліє – «достояние России»?},
Ірина Костенко; Ірина Халупа, radiosvoboda.org, 13.06.2021
%%%endcit

%%%cit
%%%cit_head
Сергей Жадан - Новые Книжки
%%%cit_pic
\ifcmt
  pic https://life.pravda.com.ua/images/doc/6/a/6a3bd1a-jadan-serhiy-4.jpg
	caption Збірка поезії Сергія Жадана \enquote{Псалом авіації}
\fi
%%%cit_text
Наприкінці весни видавництво Meridian Czernowitz порадувало прихильників
творчості Сергія Жадана одразу двома його поетичними \emph{Книжками}: новою збіркою
поезії \enquote{Псалом авіації} та перекладами вибраних віршів класика німецької
літератури Бертольта Брехта \enquote{Похвала діалектиці}.  Олег Поляков спеціально для
\enquote{Української правди. Життя} порозмовляв з автором про його стосунки з релігією;
з'ясував, що значать ландшафти і краєвиди в його віршах і в його житті;
подискутував про вік ліричного героя. А також розпитав, чому Сергій обрав для
перекладу вірші Брехта і чи вплинув поет, драматург і прозаїк Брехт на поета,
драматурга і прозаїка Жадана
%%%cit_comment
%%%cit_title
\citTitle{Письменник Сергій Жадан: Мені цікаво говорити про речі прості й очевидні}, 
Олег Поляков, life.pravda.com.ua, 13.06.2021
%%%endcit

%%%cit
%%%cit_head
%%%cit_pic
\ifcmt
  pic https://avatars.mds.yandex.net/get-zen_doc/5231833/pub_61732213917354359870146a_61732274563eeb7229bae351/scale_1200
  @width 0.4
\fi
%%%cit_text
Если бы регион не знал войны, то эта девочка-вундеркинд писала бы \emph{книги}
о чем-то другом, о чем-то мирном и спокойном, может на обложке её \emph{книги}
были бы изображены цветы и играющие дети...  Вполне возможно, что её взгляды
были бы не проросийские, может Фаина вообще бы писала по-украински. Кто знает?
Обстреливая её родной город Украина навсегда потеряла и её, и весь регион.
Увы, Киев ни за что не пойдет по пути мира, Украина выбрала иной путь
%%%cit_comment
%%%cit_title
\citTitle{Или нормальное детство, или сайт "миротворец".}, Мак Сим, zen.yandex.ru, 23.10.2021
%%%endcit

%%%cit
%%%cit_head
%%%cit_pic
%%%cit_text
Український інститут книги оголосив перелік 30 знакових \emph{книг} для української
незалежності.  Проект присвячений літературним творам та виданням, які були
опубліковані з 1991 по 2021 роки.  В Інституті зазначають, що це унікальний
список видань, що вплинули на формування та розвиток країни та її громадян.  До
онлайн-голосування долучилося понад 34 тисячі людей, які залишили понад 267
тисяч голосів.  Пепередньо експерти книговидавничої галузі, літературознавці,
перекладачі та історики літератури сформували довгий список зі 100 видань.  А
за підсумками отриманої кількості голосів сформувався фінальний список з 30
\emph{книжок}
%%%cit_comment
%%%cit_title
\citTitle{Оголосили список 30 знакових книг для української незалежності}, 
Дарія Поперечна; Катерина Хорощак, life.pravda.com.ua, 27.10.2021
%%%endcit

%%%cit
%%%cit_head
%%%cit_pic
%%%cit_text
Мій текст у цій \emph{книжці} називається «Союз неможливий» і теж розповідає
драматичну історію, варту окремого фільму, Миколи Скрипника, наркома з освіти,
який активно займався українізацією, а з її закриттям, зі стартом «червоного
терору» і сталінських репресій був вимушений покінчити життя самогубством. І за
неперевіреними переказами, його останні слова були: «Не можна бути українцем і
комуністом водночас». «Детокс» — це \emph{книга} сильних сенсів, зібраних
разом під дуже чутливим і тонким керівництвом Лариси Олексіївни, яка знову
випереджає наше сьогодення у сенсах, які потрібні нам. І хоч скільки б я
розповідала про авторів і сенси, які тут закодовані, найважливіше, що ви можете
зробити, — це зануритися у читання цієї \emph{книги}. Я впевнена, що ви
отримаєте величезне задоволення від того, що ми є спадкоємцями великої і
сильної держави, й упродовж тисячоліть незалежність завжди до нас повертається
%%%cit_comment
%%%cit_title
\citTitle{«Книга сильних сенсів»}, 
Марія Чадюк, day.kyiv.ua, 28.10.2021
%%%endcit

%%%cit
%%%cit_head
%%%cit_pic
%%%cit_text
Он поднял входной клапан палатки и поманил за собой Корнуэлла. Внутри палатка
оказалась больше, чем можно было представить снаружи. Она была уставлена
множеством приборов. В углу стояла походная койка, рядом с ней стол и стул. В
центре стола находился подсвечник с толстой свечей, пламя которой дрожало от
сквозняка. В углу лежала груда \emph{книг} в кожаных переплетах. Рядом с
книгами были открытые ящики. На столе, кроме подсвечника, не оставляя места для
письма, находился какой-то странный предмет. На столе, как заметил Корнуэлл, не
было ни пера, ни чернильницы, ни песочницы, и это показалось ему странным. В
противоположном углу стоял большой металлический шкаф, а рядом с ним у
восточной стены часть помещения была отгорожена плотной черной тканью.  — Здесь
я готовлю свой фильм, — пояснил Джоунз.  — Не понимаю, — напряженно сказал
Корнуэлл.  — Взгляните.  Джоунз подошел к столу и взял из одного ящика
пригоршню квадратных листов.  — Вот это фотографии, о которых я говорил. Не
рисунки — фотографии. Давайте берите и смотрите
%%%cit_comment
%%%cit_title
\citTitle{Зачарованное паломничество}, Клиффорд Саймак
%%%endcit

%%%cit
%%%cit_head
%%%cit_pic
%%%cit_text
"Треба ставити Україну вище всього. Наше просвітянське лобі повинно врешті мати
місце у Верховній Раді України. Тоді з’являться постанови, якими буде
регулюватися інформаційна політика.  Хотілося б поговорити про \emph{книгу}. Дійсно, в
останні роки суттєво покращилося \emph{книгодрукування}. Але все ж відчувається брак
художньої літератури. Тут говорили, що потрібна українська художня література
нашим друзям на сході. Ми зробимо все, аби в них було її більше. Вона допоможе
загартуватися в інформаційній війні тим, хто сьогодні надягає погони і буде
захищати нашу незалежність. Дуже важливо, щоб про такі зустрічі повідомляли
завчасно. У нас є хороші доповідачі, які розкажуть про ті події, які
відбувалися на Західній Україні. Зокрема, про героїзм українців. Та інформація,
яка сьогодні передається на схід, переважно викривлена. Історики повинні
говорити про це, посилаючись на літературні джерела, вказувати, де знаходиться
ця інформація. Треба, щоб ми готували таких доповідачів"
%%%cit_comment
%%%cit_title
\citTitle{Як виграти інформаційну війну? – Слово Просвіти}, ,slovoprosvity.org, 01.11.2021
%%%endcit
