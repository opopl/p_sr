% vim: keymap=russian-jcukenwin
%%beginhead 
 
%%file letters.mariupol.olena_vitko
%%parent letters
 
%%url 
 
%%author_id 
%%date 
 
%%tags 
%%title 
 
%%endhead 

%https://www.facebook.com/umrysh/posts/pfbid0iVfsPUCa87jbNCK9CyLFBvLYvE1n5QKY9YHveXKczca7bY2wKn7H7XcYdHXUwmVol


%08:00:59 15-03-23

З 9 березня час доби особисто для мене практично  перестав змінюватись. Завжди
темно, бо  увесь час я переховувалась то у коридорі, то у ванній кімнаті.  Там
не так було чутно вибухи. З'являлося хибне відчуття безпеки.  Але в ніч з 13 на
14 березня я ночувала вже у підвалі.  Бо цілодобові обстріли та бомбардування
не давали  жодної хвилини сну. Сил практично не залишилося. У підвалі було дуже
холодно. Щось близько  0⁰. Знайшли вузькі двері. Настілили ковдр та я  сиділа
там. На вулицю виходила тільки, щоб Євуся сходила в туалет. Та іноді покурити
біля під'їзду. Ніч була жахлива. Гриміло та вибухало десь зовсім поряд. Іноді
здавалося, що будинок завалиться . Було таке  враження, що він ворушиться.. Але
поспати трохи вдалося. Голова,  після удару об ФЕМи  на роддомі,  боліла
нестерпно. Обличчя нило та смикалося, особливо те місце, де кістка на скроні.
Тупий біль, який постійно  був зі мною. Він дуже заважав. Око сіпалось. Але
ніч якось промайнула швидко.  Може тому, що там у тому підвалі нас було багато.
Горіли свічки з олії. Люди перешептувались. Так ми дожили  і до ранку.
Наступного дня  почали зникати машини.. люди виїжджали.  Ліпили на скло написи
''ДІТИ" , різали білі простині і їхали. Їхали з цього пекла.. на  свій страх та
ризик. Дехто не доїхав. Мої  батьки жили у будинку поряд. Наш дім був 23, вони
жили у 21. То була сталінка. І підвал там був інший. Але місця у ньому не було.
Він був забитий людьми. Зранку 14 березня прибіг батько.  Є місце у підвалі.
Переходьте туди. Швидко, наскільки це було можливо,  ми зібрали ковдри та
перебіжками між  обстрілами, перебралися до них. Усі залишки їжі зберігалися на
балконі.  Ще був шматок сала.   Декілька консервів. Лимон. Чай. Чотири яйця.
Плавлений сирок. Яблука.  Сушена, ще з осінньої херсонської рибалки, рибка.
Банка перетертої калини та банка чорної  смородини  з цукром. Цілий скарб. Дуже
допомагав термос. Закип'ятивши воду на вогнищі зранку, ми заливали в один
термос шипшину .  В інший просто кип'яток. І можно було не йти на вулицю. Цей
день виходило сонце. І я навіть підійшла до людей, які готували їжу у дворі.
Тоді ми усі і  побачили, як літак випустив бомби над містом. Знаю з власного
досвіду, що літака, який скидає бомбу практично на тебе, ти не побачиш.  Ти
тільки почуєш жахливий звук та відчує вибухову хвилю. І це не епіцентрі вибуху.
А метрів за 70..   Спочатку ми і не зрозуміли, що воно таке блищить. Один..
два.. п'ять.. і тільки потім дійшло.. він скидає бомби . У світі сонця вони
відблискували.  Скидали їх над Азовсталлю.  Через декілька секунд вони
вибухнули.. жахливий звук. Потім вже ми дізналися, що скидали фугаси від 500 до
3000 кг. Ночували цю ніч ми у підвалі,  разом із  батьками. Людей  поменшало.
Тут було не холодно.  І навіть зручно було лежати. Спати не спали. Дрімали. А
на дворі гупало, літало, вибухало.  Я дуже боялася, що прилетить у будинок і
нас завалить плитами. І думка була одна. Якщо вже прилетить, то щоб відразу,
щоб не мучатись.  То була наша передостання ніч у місті, яке залишилось один на
один з Сатаною..

Olena
Olena Vitko

%08:01:05 15-03-23

Спогади березня 2022 закінчились 4 числа. Відтоді ми були відрізані від України
та світу інформаційно та знаходились у повній облозі.. У цей день, рік тому, ми
таки дійшли до наших друзів. Попри шалені обстріли.. Вони мешкали зовмім поряд.
Хвилин 7 йти до них. Плакали та обіймалися.  Над головою літало все, що
можливо. Але бажання побачитись переважило.  Газу вже не було два дні. Їжу
готували на вогнищі у дворі.  Взяли із собою качку, яка ще зберіглася у
морзилці. Та усі гриби, які ми назбирали у Святогіррі..пили вино. Руки чорні,
готували їжу на мангалі.  День був сонячний, хоч і холодний.  Сніг лежав та
нестерпно блищав.. А літаки вже літали містом та скидали бомби на людей.
Напевно тоді ми були щасливі, якщо можно так сказати.  Ми приготували ще і суп
для себе та батьків.. ту качку та гриби насмажили на мангалі та розділили на
усіх. То були останні продукти з холодильнику, які ми готували. Їх нам
вистачило на чотири наступних дні. Мороз на вулиці допоміг зберегти їх на
балконі.  Сьогодні, згадуючи все це, здається, що то був страшний та нереальний
сон..💔

А ці фото від 8 березня  2020 року з нашої дачі. Ми тоді вперше після зими
поїхали з ночівкою. Смажили шашлик, пили шампанське.  Було дуже тепло. І так
затишно..


