%%beginhead 
 
%%file 09_08_2023.fb.mariupol_neskorenyj.1.sergij_oleksijovych_barannyk
%%parent 09_08_2023
 
%%url https://www.facebook.com/100066312837201/posts/pfbid0LGKee1h169FGUaeQs2SZLRr1ZNoBNnFjVwZC31x6vEPxfUJmEu3rDBoVhziGK81jl
 
%%author_id mariupol_neskorenyj
%%date 09_08_2023
 
%%tags 
%%title Сергій Олексійович Баранник
 
%%endhead 

\subsection{Сергій Олексійович Баранник}
\label{sec:09_08_2023.fb.mariupol_neskorenyj.1.sergij_oleksijovych_barannyk}

\Purl{https://www.facebook.com/100066312837201/posts/pfbid0LGKee1h169FGUaeQs2SZLRr1ZNoBNnFjVwZC31x6vEPxfUJmEu3rDBoVhziGK81jl}
\ifcmt
 author_begin
   author_id mariupol_neskorenyj
 author_end
\fi

🖼🎨 Ще один учасник проєкту \enquote{Маріуполь нескорений} – Сергій Олексійович Баранник.

Усе творче життя Сергій Олексійович будує свій власний світ емоцій, почуттів у
малярстві та графіці, експериментує з колірною гамою і формою, що надає його
творам неповторного авторського стилю.

📎 Війна принесла Сергію Олексійовичу тяжкі випробування. Як і багато
маріупольців він не очікував, що ворог буде перетворювати у попіл рідне місто,
тому прийняв рішення відправити родину у більш безпечне місце, а самому
залишитися у Маріуполі, щоб оберігати художні фонди маріупольської організації
і власний творчий спадок. Всі маріупольці пам'ятають лютий-березень 2022 року –
інформаційний вакуум, відсутність їжі, води, ліків, страшенний холод, постійні
авіабомбардування, обстріли з танків та градів. Не минули ці події і Сергія
Баранника.

@igg{fbicon.heart.white.middle} Одного дня, наприкінці березня, роздався
страшенний вибух і це вже було зовсім поруч, у сусідній кімнаті. Зазирнувши
туди, Сергій Олексійович побачив, що всі речі, які були на підлозі, якимось
чином опинилися приклеєними до стелі.

😪Квартира стала горіти. Треба було вибиратися, але на вулиці вже були ворожі
солдати і кожної миті можна було чекати автоматної черги. Схопивши домашнього
улюбленця – кошеня і щось з речей, він покинув свій дім. Йому пощастило
вибратися з Маріуполя. А далі дороги війни – Бердянськ, Василівка, Запоріжжя,
Дніпро. Маріупольці пам'ятають цей шлях.

✅️ Зараз Сергій Олексійович живе у м. Стрий Львівської області, створює нові
картини і бере участь у різних мистецьких проєктах, наприклад, \enquote{Обереги
України}, \enquote{Маріуполь – душа України} тощо. Тепер в його роботах досвід мирного
життя поєднується зі страшними враженнями воєнного часу.

📍 У мегапроєкті \enquote{Маріуполь нескорений} беруть участь пейзажі та
фігуративні композиції художника: \enquote{Знак війни}, \enquote{Знак
Маріуполя}, \enquote{Знак Юрія Воїна}, \enquote{Знак захисту}, \enquote{Знак
Архангела Михаїла} та інші. Вони створені вже під час війни, в них розкриває
митець свої світоглядні позиції. Свої образи він називає знаками, це умовне,
узагальнене, асоціативне відображення сутностей і понять. Але, все одно, його
асоціативне мислення ґрунтується на реальності. Побачене, придумане, те, що
залишилось у свідомості і підсвідомості, він зберігає у мовчазливих, лаконічних
символах і знаках. Ця художня інформація створена для передачі глядачам,
розрахована на їх розуміння, фантазію, асоціативне мислення.

🗓 Виставка \enquote{Маріуполь нескорений} відкривається у київській Галереї мистецтв
\enquote{Лавра} 21 серпня поточного року, і всі бажаючі зможуть скласти власну думку
про картини легендарного метра українського андеграунду Сергія Баранника.

⏩️ Біографічна довідка:

Сергій Баранник (1952 р.н.) – графік, живописець, монументаліст. Один з
засновників творчої групи альтернативного мистецтва \enquote{Маріуполь-87}. Член
Національної Спілки художників України з 1992 року, а з 2009 року має почесне
звання – Заслужений художник України. Також Сергій Баранник – голова філії
Спілки художників України у м. Маріуполь. З 1991 року веде активну виставкову
діяльність в Україні і в світі. Багато картин митця зберігаються у приватних і
музейних колекціях України, Канади, США, Італії, Франції, Литви, Китаю, Греції.

\#Маріупольнескорений \#виставка \#культурнадеокупація \#Маріуполь \#Київ \#художники
