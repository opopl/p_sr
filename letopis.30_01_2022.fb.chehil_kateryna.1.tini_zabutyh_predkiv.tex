% vim: keymap=russian-jcukenwin
%%beginhead 
 
%%file 30_01_2022.fb.chehil_kateryna.1.tini_zabutyh_predkiv
%%parent 30_01_2022
 
%%url https://www.facebook.com/permalink.php?story_fbid=3145633439096488&id=100009495874278
 
%%author_id chehil_kateryna
%%date 
 
%%tags kniga,kocjubinskij_mihail.ukr.pisatel,kultura,literatura,ukraina
%%title Тіні забутих предків
 
%%endhead 
 
\subsection{Тіні забутих предків}
\label{sec:30_01_2022.fb.chehil_kateryna.1.tini_zabutyh_predkiv}
 
\Purl{https://www.facebook.com/permalink.php?story_fbid=3145633439096488&id=100009495874278}
\ifcmt
 author_begin
   author_id chehil_kateryna
 author_end
\fi

\ii{30_01_2022.fb.chehil_kateryna.1.tini_zabutyh_predkiv.pic.1}

«Тіні забутих предків» повість, що запроторить вас у середовище, де над
будь-якими людськими здобутками панували боги, а ще мольфари, відьми, чугайстри
і мавки. Гуцульський побут, традиції, унікальне поєднана релігій, захоплює своє
органічністю. А цей міфічний характер сприйняття світу??? 

Я не вперше перечитую цю повістю, і не вперше відкриваю для себе незбагненні
достоту рядки. Михайло Коцюбинський майстер слова! Абсолютно всі його роботи є
психологічно напруженими. Але якими винятковими!!! Вбачаю істину в його прозах,
новелах, творах. 

І вкотре повторюся: це є ті самі українські літературні визиски, що належать до
найвищих досягнень української класичної прози і оминути їх увагою - то є
великий гріх).

І ще, принагідно додам: найкращою роботою Михайла Коцюбинського, я вважаю
повість «Дорогою ціною»!!! Повість, яка змінює сприйняття і уявлення про
кохання!

Шануйтеся, ну і читайте! @igg{fbicon.wink} 
