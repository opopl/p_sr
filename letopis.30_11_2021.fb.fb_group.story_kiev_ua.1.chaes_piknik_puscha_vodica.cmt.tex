% vim: keymap=russian-jcukenwin
%%beginhead 
 
%%file 30_11_2021.fb.fb_group.story_kiev_ua.1.chaes_piknik_puscha_vodica.cmt
%%parent 30_11_2021.fb.fb_group.story_kiev_ua.1.chaes_piknik_puscha_vodica
 
%%url 
 
%%author_id 
%%date 
 
%%tags 
%%title 
 
%%endhead 
\zzSecCmt

\begin{itemize} % {
\iusr{Таня Слюсаренко}
Жах. Ми нічого не знали і першого з дітьми на параді, а вони вже своїх вивезли.

\begin{itemize} % {
\iusr{Roksolana Vasilishina}
\textbf{Таня Слюсаренко} 

я дізналась про вибух 26 квітня, в суботу, на Дні народження подружки. В неї
батько працював в одному з Київських НДІ що обслуговували станцію, і бригада
інженерів з його віддіду виїхала зранку. Але по телевізору про це сказали
тільки в понеділок, 28 квітня. І сказали що \enquote{нічого страшного, аварія
ліквідується}. Але пам'ятаю що по Києву вже були слухи: казали тримати вікна
зачинені і мазатись йодом. І в школі вже уроки йшли ненормально. Бо діти
розпитували вчителів про аварію.

\iusr{Владимир Дубровский}
Боялись паники.
\end{itemize} % }

\iusr{Elena Oleynik}
А студентам київського меду ніхто не зателефонував, тому вони були 1 травня на Дніпрі ..((

\iusr{Лидия Васильева}
Вот это был
\enquote{Вагнегейт}!!!

И никто звука не издал в сторону руководства страны, все молча все проглотили. А
сегодня страну на уши ставят на разных шоу, такая беда, сорвана операция
ГРУ.!!!

Позор

\begin{itemize} % {
\iusr{Михайло Наместник}
\textbf{Лидия Васильева} 

ничего удивительного - в те времена население жрало с лопаты, что начальство
даёт. А сейчас зарождается нация и эти фокусы так просто уже не пройдут.

\begin{itemize} % {
\iusr{Грег Станлий}
\textbf{Михайло Наместник} ага, как раз два с половиной года назад она зародилась, заметно было.))

\iusr{Надежда Владимир Федько}
\textbf{Грег Станлий} Вона відокремилась...

\iusr{Грег Станлий}
\textbf{Надежда Владимир Федько} в размере статистической погрешности, разве что.)

\iusr{Надежда Владимир Федько}
\textbf{Грег Станлий} Нє. По результатам виборів)) зрозуміло.

\iusr{Грег Станлий}
\textbf{Надежда Владимир Федько} 

если вы имеете в виду 25\%, то зарождающейся там нации немного. Просто там те,
кто тупо не хотел голосовать за клоуна.

\iusr{Надежда Владимир Федько}
\textbf{Грег Станлий} 

Це окрема тема... Я перехворів цим у 2019-му... Щодо нації, яка відроджується,
я так би сказав... то її представників 7-10\%.

\iusr{Грег Станлий}
\textbf{Надежда Владимир Федько} да, эта цифра близка к истине.

\iusr{Надежда Владимир Федько}
\textbf{Грег Станлий} 

Свої думки щодо ситуації я виклав у блозі \enquote{Падіння з зірок... в реалії сучасної
України}: «Етнос-25», «73-й квартал», «Сірий сектор» на сайті \enquote{Народний
оглядач}.

Якщо цікаво, то по назві можете знайти... (Посилання тут заборонені!)

\iusr{Грег Станлий}
\textbf{Надежда Владимир Федько} спасибо, непременно найду.

\iusr{Надежда Владимир Федько}
\textbf{Грег Станлий} 

Я, до речі, і розірвав стосунки з сайтом із-за ідіотської тези, що 25\% - це
зародження нової нації... ельфів, що втрутилися у вибори... та іншої маячні.

\iusr{Грег Станлий}
\textbf{Надежда Владимир Федько} правильно сделали!

\iusr{Надежда Владимир Федько}
\textbf{Грег Станлий} 

Я перевірив... Посилання працює. назву давайте в пошуку без лапок... Там багато
і інших моїх матеріалів... Я був пов'язаний з проектом НО 20 років...

\iusr{Грег Станлий}
\textbf{Надежда Владимир Федько} да, я уже кое-что нашел, как раз читаю.

\iusr{Михайло Наместник}

Да Нації ви, шановна, жодного відношення не маєте. Живіть спокійно та не плутайтесь під ногами.
\end{itemize} % }

\iusr{Надежда Владимир Федько}
\textbf{Лидия Васильева} 

\enquote{Вагнегейт} - це зрада на найвищому рівні! Якби таке сталося в США, то був би
імпічмент, спецпрокурор, розслідування і судовий процес, пожиттєве ув'язнення!

\begin{itemize} % {
\iusr{Петр Мишура}
\textbf{Надежда Владимир Федько} Я так понимаю вы эксперт в разведке?

\iusr{Надежда Владимир Федько}
\textbf{Петр Мишура} 

НЕ буду брати на себе звання...

Що Вас цікавить?

\iusr{Петр Мишура}
\textbf{Надежда Владимир Федько} Щоб вести балачки про зраду, треба знати всі деталі...

\iusr{Надежда Владимир Федько}
\textbf{Петр Мишура} Як старший офіцер, я знаю достатньо, щоб робити висновки.
Але не маю бажання вести публічну розмову))

\iusr{Петр Мишура}
\textbf{Надежда Владимир Федько} Так і не треба каламутить тоді товариш старший офіцер...

\iusr{Петр Мишура}
\textbf{Надежда Владимир Федько} Якщо кожен старший офіцер знає подноготную операції, то керівника треба було ще раніше гнати...

\iusr{Надежда Владимир Федько}
\textbf{Петр Мишура} Вам ще раз повторити, що я не маю жодного бажання вести публічну розмову на цю тему))

\iusr{Петр Мишура}
\textbf{Надежда Владимир Федько} 

Лидия Васильева \enquote{Вагнегейт} - це зрада на найвищому рівні! Якби таке сталося в
США, то був би імпічмент, спецпрокурор, розслідування і судовий процес,
пожиттєве ув'язнення! А це шо?Не публічно?


\iusr{Golum Parasut}
\textbf{Nadegda Volodymyr} Fedko Все попереду. Тільки, як завжди доведеться активній меншості брати відповідальність за долю країни, яку байдужа більшість віддала на поталу

\iusr{Лидия Васильева}
\textbf{Надежда Владимир Федько}
но мы далеко не американцы и живем далеко не в Америке

\iusr{Надежда Владимир Федько}
\textbf{Лидия Васильева} Де ми живемо політологія мовчить))

\iusr{Лидия Васильева}
\textbf{Петр Мишура}
а сдача Крыма, Донбасса, - это было что, не гейт?

\iusr{Лидия Васильева}
\textbf{Петр Мишура}
А если еще и Савик это елозит на шоу- то это вообще....
\end{itemize} % }

\end{itemize} % }

\iusr{Алла Титаренко}

Мы тоже пережили это в Киеве на парад правда не ходили, потом младшую сестру
вывезли со школой в Скадовск и до сентября дети « оздоравливались» на море. Как
ваш тесть пережил, просто интересно кондратий не хватил?

\begin{itemize} % {
\iusr{Надежда Владимир Федько}
\textbf{Алла Титаренко} Коли Юлю відправили в піонерський табір, то мирився зі мною... А в часи "перебудови" перестав платити партійні внески і ходити на партзбори))

\iusr{Алла Титаренко}
\textbf{Надежда Владимир Федько} Ну и слава богу главное мир в семье. Очень жаль стариков и пенсионеров которые так верили Партии и Правительству. Такое разочарование  @igg{fbicon.cry} 
\end{itemize} % }

\iusr{Марина Иващенко}

А мы жили на Нивках... и часто ездили в лес в Пущу гулять... Мне было 8 лет.. не
помню... но возможно, мы тоже могли в это время где-то там и гулять... А вот на
майские мы уже были в Коктебеле... убежали с мамой и сестрой.


\iusr{Виктория Зайцева}

А я уже знала об аварии на следующий день. Я была из тех людей, что слушали
\enquote{ворожи голоси} - Немецкая волна, Голос Америки и Радио Свобода. Так что мы уже
были в курсе, и хотя мы далеко от Киева, но было очень не приятно и больно. Я
была в шоке от произошедшего. Помню, что кто-то советовал побольше употреблять
витамина С и его в аптеке уже было сложно купить.

\iusr{Надежда Владимир Федько}
\textbf{Виктория Зайцева} Я теж слухав \enquote{ворожі голоси}))

\iusr{Tatyana Shpak}
Кальченко АльвиНа, а не АльвиРа.

\begin{itemize} % {
\iusr{Надежда Владимир Федько}
\textbf{Tatyana Shpak} Вибачаюся, описка))

\iusr{Tatyana Shpak}
\textbf{Надежда Владимир Федько}  @igg{fbicon.bouquet} 

\iusr{Надежда Владимир Федько}
\textbf{Tatyana Shpak} 

Дякую! Стільки років пройшло... Я дружив з Альвіною і з її донькою, Ириною.
Фотографував на заняттях... Бував у них вдома.

До речі, я почав фотографувати балет з 1975 року. І застав Альвіну ще на
сцені...

\iusr{Оля Лемишко}

У мене з 28 квітня починалася відпустка. Їхала до родичів, зранку ще побігла в
пошуках дитячих колготок ( у сестри малому було 2 роки), знайшла, бо кінець
місяця викинули, в руки дві пари. На автобус і вперед. Всі плани шкереберть, з
відпустки повернулася вже на початку червня. І в Києві на вулиці тиша, дітей
вивезли хто куди зміг. Літо, сонце і не чути дітей. Відчуття було дивне.

\end{itemize} % }

\iusr{Александр Петренко}
Да, уж! Очень жалко джинсы (два) и кроссовки (два)!

\begin{itemize} % {
\iusr{Надежда Владимир Федько}
\textbf{Александр Петренко} З того пікніка залишилася \enquote{живою} тільки туристична сокирка, якою користуюся і досі))

\iusr{Александр Петренко}
\textbf{Надежда Владимир Федько} Спасибо. Хорошо написано, аж зачитался. Прям детектив какой-то. Завидую памяти на всякие мелочи, хотя и понимаю, что кое-где, как говорится, художественный вымысел.

\iusr{Надежда Владимир Федько}
\textbf{Александр Петренко} 

Десь у нашій голові захований диск, на якому записується інформація про кожну
міллісекунду нашого життя. Одночасно, ця інформація шляхом
енерго-інформаційного обміну поступає до Господа, який любить і опікує усіх
нас.

Коли у людини виникає гостра потреба щось пригадати, навіть через десятки
років, то Господь відкриває доступ до записаної інформації і яскраві спогади
випливають у нашій пам'яті.

Але рівень усвідомлення, сприйняття і інтерпретації спогаду залежить від
інтелектуального рівня людини.

Це суто індивідуально. Це як відчуття і розуміння музики, співу чи танцю. Це як
відчуття і розуміння літератури чи образотворчого мистецтва.

\iusr{Александр Петренко}

\enquote{...Одночасно, ця інформація шляхом енерго-інформаційного обміну поступає до
Господа...} - в интернете это называется хранение информации в облаке.

\end{itemize} % }

\iusr{Тамара Ар}

К счастью, когда автор 26, 04, 1986 гулял в Пуще, радиационный ветер не дул в
сторону Киева, а вот 30 апреля, 1 мая и дальше, гулять было уже очень не
рекомендовано! Направление ветра поменялось, слухи просачивались, окончательно
все узнали 3 мая, что произошло,,,,,,,все киевляне помнят те дни на всю жизнь

\begin{itemize} % {
\iusr{Надежда Владимир Федько}
\textbf{Тамара Ар} 

Це дуже цікава тема. Можливо мої спогади спонукають киян різного віку теж
пригадати події того трагічного літа.

Коли Юля повернулася додому з піонерського табору, то розповідала, що у них у
всіх був шок від розлучення з батьками і невідомості, що буде далі...

\begin{itemize} % {
\iusr{Светлана Манилова}
\textbf{Надежда Владимир Федько}, в группе за почти 4 года ее существования было много публикаций на эту тему.

\iusr{Надежда Владимир Федько}
\textbf{Светлана Манилова} А вони під тегом?

\iusr{Светлана Манилова}
\textbf{Надежда Владимир Федько}, нет.

\iusr{Надежда Владимир Федько}
\textbf{Светлана Манилова} Я ж неофіт)) Тільки 17-го листопада прийшов у групу.

\iusr{Светлана Манилова}
\textbf{Надежда Владимир Федько}, знаю. @igg{fbicon.smile} 

\iusr{Тамара Ар}
\textbf{Надежда Владимир Федько} 

Это трагические воспоминания,,,,,,, с 3 мая поступила рекомендация от городских
властей проводить влажную уборку в помещениях, квартирах по несколько раз на
день, не выходить с детьми на улицу, выезжать по возможности из города,,,,,,, я
с годовалым ребенком уехала на 4 месяца, как и многие киевляне с
малышами,,,,,, билеты купить было трудно, хотя пустили дополнительные поезда по
разным направлениям


\iusr{Тамара Ар}
\textbf{Надежда Владимир Федько} 

в группе в апреле, мае, действительно, есть очень много воспоминаний об этом
печальном событии

\end{itemize} % }

\end{itemize} % }

\iusr{Lora Wilhelm}
пам'ятаю, Київ наче вимер. І дихати важко було.

\iusr{Кордюков Дмитрий}
На балконе пришлось срезать металлические перила потому что фонили страшно.

\iusr{Yaromari Seaspan}
я в Жданове жил тогда, и в Кр Гвоздике был с мая по август...

\begin{itemize} % {
\iusr{Виктория Зайцева}
\textbf{Yaromari Seaspan} А где этот лагерь был тогда? Название знакомое а вспомнить не могу.
\end{itemize} % }

\iusr{Alex Jäger}
В этот день там были соревнования по спортивному ориентированию среди школ Подольского района, может кто помнит

\iusr{Дмитро Левицький}

26 ще в Києві був нормальний рівень. Тільки по деяких вулицях автобуси з
ураженими встигли трохи «нанести». Вітер змінився в напрямку Києва 28-го от
тоді вже і.....


\iusr{Юрий Панчук}

Все як і у вас, приблизно за тим ж сценарієм! Виїхали до Москви к родичам. Мені
тоді було 13 і мама відпускала мене самостійно гуляти по Москві. Я був в шоці,
наскільки асортимент товарів був кращим за Київ. ЖД квитки батьки взяли з
великими складнощами, а потім виявилося, що на ті самі місця є інші пасажири
(електронної системи продажу ще не було, а мародерство вже було). Дідусь,
старий партієць, який вирішив нас супроводити до племінника пішов до начальника
поїзда і домовився, щоб нас якось розмістили. В результаті нам з дідом
прийшлося лізти на третю полку (діду на той час було вже 78 і я досі дивуюся
його гарній фізичній формі). Був я і на параді 1 травня, нам в школі видали
красиву форму, тренували красиво ходити. Потім 2 травня відпочивали всією сім
єю в Гідропарку. Дякую, дуже гарні фото.

\iusr{Елена Махова}

А я закінчувала школу. Випускні класи вчилися до останнього та складали іспити.
Було дуже незвично 25 травня, у свято останнього дзвоника, коли нас,
випускників, не вітали першокласники, та й саме свято відбувалося не на вулиці,
а в актовому залі.

І в школі було порожньо й тихо, бо тільки одні випускники приходили на уроки.


\iusr{Игорь Коршиков}

Я бухал два дня на Русановских садах, в понедельник утром только на работе
узнал, в среду укатили в Одессу и там на въезде нам поливали машину и проверяли
документы, спрашивали член КПСС смешно


\iusr{Татьяна Чернышева}

У меня в голове только одна дурацкая мысль- зачем в лес на пикник надевать
новые джинсы и кроссовки? @igg{fbicon.monkey.see.no.evil}  @igg{fbicon.face.screaming.in.fear}{repeat=3} 

\begin{itemize} % {
\iusr{Татьяна Бригинец}
\textbf{Татьяна Чернышева} 

а где ещё было хвастаться тем, что у тебя есть джинсы и кроссовки?! В школу,
например, на субботник надевали лучшее. Иначе как себя показать мальчику из
параллельного? Все знают, что в параллельном мальчики лучше! А то и
старшекласснику на глаза попадешь...


\iusr{Надежда Владимир Федько}
\textbf{Татьяна Бригинец} Тонко підмічено!)

\iusr{Надежда Владимир Федько}
\textbf{Татьяна Чернышева} 

На той час це не було моєю вихідною одежею) Це була повсякденна одежа! Так -
круто! Але джинси і кросівки - це повсякденна одежа! Я носив з джинсами і
шкіряний піджак. Але, знову ж таки, це була повсякденна одежа!

Святкова одежа - костюм, сорочка, краватка...

***

P.S.

Зарад нарід шастає у спортивних костюмах...

\begin{itemize} % {
\iusr{Татьяна Чернышева}
\textbf{Надежда Владимир Федько} у цих костюмів назва є- карантинка. Вліз у карантинку і почимчикував хоч у школу, хоч на побачення, і в магазин, і в ліс, і на роботу)))
\end{itemize} % }

\end{itemize} % }

\iusr{Ковальская Татьяна}
Все так і було, але це ще один факт, що владі до людей було начхати!

\begin{itemize} % {
\iusr{Надежда Владимир Федько}
\textbf{Ковальская Татьяна} Зараз теж!

\iusr{Виктория Зайцева}
\textbf{Надежда Владимир Федько} Любой власти на людей начхать. Государство ставит интересы государства всегда выше интересов отдельных людей.

\iusr{Ковальская Татьяна}
\textbf{Виктория Зайцева} Це не інтереси були, це було життя і здоров'я людей! Свої сім'ї ці падлюки вивозили вже на другий день!

\iusr{Надежда Владимир Федько}
\textbf{Виктория Зайцева} Все трохи складніше... Безумовно, що інтереси існування держави є вищими за інтереси окремої особи.
Але в правильно організованій державі інтереси окремої особи органічно поєднуються з інтересами держави.
\end{itemize} % }

\iusr{Olena Elovskaya}

А мы 26 бегали кросс в Пуще на день физкультурника. А потом шли пешком. А мимо
ехала колонна автобусов, мы еще удивлялись откуда их так много везут.

\end{itemize} % }
