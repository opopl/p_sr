% vim: keymap=russian-jcukenwin
%%beginhead 
 
%%file 31_03_2021.fb.fb_group.story_kiev_ua.1.ukr_hleb
%%parent 31_03_2021
 
%%url https://www.facebook.com/groups/story.kiev.ua/posts/1630285413834920
 
%%author_id fb_group.story_kiev_ua,kuzmenko_petr
%%date 
 
%%tags hleb,hleb.ukrainskij,kiev,kievljane,ukraina
%%title Український хліб
 
%%endhead 
 
\subsection{Український хліб}
\label{sec:31_03_2021.fb.fb_group.story_kiev_ua.1.ukr_hleb}
 
\Purl{https://www.facebook.com/groups/story.kiev.ua/posts/1630285413834920}
\ifcmt
 author_begin
   author_id fb_group.story_kiev_ua,kuzmenko_petr
 author_end
\fi

Шановні одногрупники та однодумці! Сьогодні, у своєму пості, я хочу зупинитись
на візитівці нашого улюбленого Міста, яка досі була обділена достатньою увагою
у дописах авторів нашої чудової групи. Я веду розмову про Український хліб.

Так, саме цей різновид найнеобхіднішого з продовольчих товарів став улюбленим
делікатесом (я навмисно вжив тут саме це слово) не тільки киян, а і мешканців
усієї України. Та що там казати про Україну. Кияни, волею долі розкидані по
усьому світу, вважають за необхідне коли хтось їде до них, попросити першим
серед київських гостинців саме круглий буханець рідного смачного Українського
хліба. Його неповторний смак кияни пам'ятають з дитинства все життя. 

\begin{multicols}{2} % {
\setlength{\parindent}{0pt}

\ii{31_03_2021.fb.fb_group.story_kiev_ua.1.ukr_hleb.pic.1}
\ii{31_03_2021.fb.fb_group.story_kiev_ua.1.ukr_hleb.pic.1.cmt}
\ii{31_03_2021.fb.fb_group.story_kiev_ua.1.ukr_hleb.pic.2}

\ii{31_03_2021.fb.fb_group.story_kiev_ua.1.ukr_hleb.pic.3}
\ii{31_03_2021.fb.fb_group.story_kiev_ua.1.ukr_hleb.pic.3.cmt}

\ii{31_03_2021.fb.fb_group.story_kiev_ua.1.ukr_hleb.pic.4}
\ii{31_03_2021.fb.fb_group.story_kiev_ua.1.ukr_hleb.pic.4.cmt}

\ii{31_03_2021.fb.fb_group.story_kiev_ua.1.ukr_hleb.pic.5}
\ii{31_03_2021.fb.fb_group.story_kiev_ua.1.ukr_hleb.pic.5.cmt}

\ii{31_03_2021.fb.fb_group.story_kiev_ua.1.ukr_hleb.pic.6}
\ii{31_03_2021.fb.fb_group.story_kiev_ua.1.ukr_hleb.pic.6.cmt}

\end{multicols} % }

Особисто для мене та нашої родини, яка зараз мешкає у різних сторонах
далеченько від рідного Подолу, смак українського хліба уособлює собою смак
Києва, дитинства, безтурботності та щастя. Це не дивно. 

Взагалі, у величезному асортименті
хлібобулочних виробів виробництва Київхліб найбільші об’єми продажів,  серед
житньо-пшеничного хліба, має саме хліб Український столичний подовий.
Традиційний Український хліб хлібзаводи Міста почали випускати на початку 60-х
років XX століття. 

У післявоєнні роки промислове хлібопечення в Києві
виготовляло лише житній хліб, через складне економічне становище та високі ціни
на пшеничне борошно. З поступовою відбудовою та певною стабільністю промислове
хлібопечення запропонувало киянам житньо-пшеничний хліб Український, який
неабияк сподобався на смак споживачам. 

Рецептура хліба Українського від початку його виробництва і до сьогодні
залишилася незмінною. Він готується на заквасці з житнього та пшеничного
борошна з натуральних інгредієнтів, з дотриманням стандартів.  Тривалість
виробництва становить 12 годин через складність виробничого процесу. Свого
часу, після успіху хліба Український почали з’являтися схожі за рецептурою
хліби: «Дарницький», «Чумацький», «Селянський».

Попри те, що Український має просту рецептуру, він і сьогодні отримує
прихильність людей за традиційний смак та користь для організму. Хліб
Український має меншу калорійність, ніж пшеничний хліб. До його складу входять
вітаміни груп B і РР, амінокислоти, цинк, залізо, йод, калій, натрій. А чималий
вміст клітковини в продукті сприяє збалансованій роботі травної системи. Але,
не тільки неповторний смак та користь Українського хліба закохали у нього киян
та більшість тих, хто його хоч раз коштував. 

Я особисто з теплою хвилею
ностальгії завжди згадую те, як маленьким хлопчиком батьки відправляли мене за
саме Українським хлібом і я біг вверх по Андріївському узвозу у магазинчик,
який мама з папою називали \enquote{урядові об'їдки}. Якщо Українського там не було,
або він був не свіжим, я підіймався вище у хлібний магазинчик на Кожум'яках,
або біг униз на Жданова. Там був великий за мірками тих часів \enquote{фірмовий}
хлібний магазин. А повертаючись додому завжди дорогою об'їдав хрумку ароматну
скоринку ще теплого запашного буханця. 

Трохи пізніше, коли у школі закохався у свою майбутню дружину та став
проводжати її додому на Почайнинську, я звернув увагу, що Неллі мешкає зовсім
поруч з хлібзаводом розташованим наприкінці Щекавицької, практично на її розі з
Набережно-Хрещатицькою. У повітрі усього кварталу завжди витав аромат
свіжоспеченого хліба. Біля заводу був і дотепер існує маленький магазинчик
продукції підприємства. З того часу я полюбив гарячий Український хліб,
найсвіжіший. Згодом, я приносив його зранку, після звільнень з ночівлею додому,
друзям - курсантам у наш Морполіт. 

Сніданок з гарячим свіжим Українским хлібом замість звичних порізаних цеглинок
був чарівним. І серед шанувальників нашого Українського з'явились вихідці з
усіх кінців тодішньої величезної країни. Ще у магазинчик хлібзаводу за гарячим
Українським завжди любила бігати наша донечка Катруся. Тепер цим з задоволенням
займаються наші онуки, коли приїздять додому на рідний київський Поділ. 

Дуже дивне відчуття піднесення завжди охоплює мене, коли бачу онука який гризе
хрустку скоринку буханця теплого Українського хліба, дорогою додому з
магазинчика хлібзавода, який досі працює неподалік нашого подільського будинку. 

P. S. Дописуючи останні рядки свого посту я дуже яскраво уявив собі як деякі з
читачів - одногрупників потягнулись за купленим сьогодні Українским хлібом.
Відрізавши смачну скибку може хтось покладе на неї декілька тонесеньких
шматочків свіженького м'якенького сала, можливо навіть з Житнього базару.
Якщо дехто з чоловіків ще наллє собі маленьку чарочку достойного українського
напою, що чекав схожої слушної нагоди у дальньому куточку холодильника, моя
біла заздрість буде безмежною.

\ii{31_03_2021.fb.fb_group.story_kiev_ua.1.ukr_hleb.cmt}
