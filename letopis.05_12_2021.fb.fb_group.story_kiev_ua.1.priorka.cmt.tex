% vim: keymap=russian-jcukenwin
%%beginhead 
 
%%file 05_12_2021.fb.fb_group.story_kiev_ua.1.priorka.cmt
%%parent 05_12_2021.fb.fb_group.story_kiev_ua.1.priorka
 
%%url 
 
%%author_id 
%%date 
 
%%tags 
%%title 
 
%%endhead 
\subsubsection{Коментарі}

\begin{itemize} % {
\iusr{Yuriy Bubnov}
Фотографировал в 1986 году. Купола были ещё зелёные...

\ifcmt
  ig https://scontent-frx5-2.xx.fbcdn.net/v/t39.30808-6/263849795_1068749230583857_8920962352727047775_n.jpg?_nc_cat=109&ccb=1-5&_nc_sid=dbeb18&_nc_ohc=1rGdjQakWjIAX_nC-U8&_nc_ht=scontent-frx5-2.xx&oh=56d5a4a6d5d38298a6c75bb4efd463d9&oe=61B25036
  @width 0.4
\fi

\begin{itemize} % {
\iusr{Maksym Oleynikov}
\textbf{Yuriy Bubnov} Позолотять у 2006 році.

\iusr{Алла Диба}
\textbf{Yuriy Bubnov} А я собі думаю, чому мені в пам'яті церковні бані "якісь не такі"... Ось воно що. Так запало в душу.
\end{itemize} % }

\iusr{Вячеслав Нестеров}

Якщо придивитись уважно, навіть на цих фото, можна зрозуміти, що з чотирьох
бічних барабанів насправді уцілів лише один, з дійсно цегляними віконцями і
крикрасами, три інші є глухими циліндричними барабанами для симетрії куполів.
Жодна з реконструкцій так і не стосувалася цих елементів покрівлі.  @igg{fbicon.frown} 

\iusr{Людмила Терпелюк}
Дякую. Живу поряд і дуже цікаво було читати про історію Пріорки та Покровськоі церкви.

\iusr{Liudmila Kabanova}

Огромное спасибо! Фото замечательные, тем более мое детство прошло на Приорке,
как раз почти рядом с этим храмом. И мне там все очень близко.


\iusr{Ирина Козина}

насколько я знаю, кладбище вокруг церкви занимало значительно большую
территорию, нежели теперь. но было снесено и на его месте выстроены новые дома

\end{itemize} % }
