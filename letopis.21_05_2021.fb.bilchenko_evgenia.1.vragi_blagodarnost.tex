% vim: keymap=russian-jcukenwin
%%beginhead 
 
%%file 21_05_2021.fb.bilchenko_evgenia.1.vragi_blagodarnost
%%parent 21_05_2021
 
%%url https://www.facebook.com/yevzhik/posts/3909922882376126
 
%%author Бильченко, Евгения
%%author_id bilchenko_evgenia
%%author_url 
 
%%tags bilchenko_evgenia
%%title БЖ. Благодарность врагам
 
%%endhead 
 
\subsection{БЖ. Благодарность врагам}
\label{sec:21_05_2021.fb.bilchenko_evgenia.1.vragi_blagodarnost}
\Purl{https://www.facebook.com/yevzhik/posts/3909922882376126}
\ifcmt
 author_begin
   author_id bilchenko_evgenia
 author_end
\fi

БЖ. Благодарность врагам.

В их понимании
В их разумении
В их прозябании
В их освинении...

Егор Летов


\ifcmt
  pic https://scontent-lga3-2.xx.fbcdn.net/v/t1.6435-0/p526x296/189666564_3909922815709466_6809881940113552018_n.jpg?_nc_cat=107&ccb=1-3&_nc_sid=8bfeb9&_nc_ohc=CtCbdE1cN48AX9Kmkjq&_nc_ht=scontent-lga3-2.xx&tp=6&oh=b5102180ffcdbb966b266a96f94b8687&oe=60CCED6C
\fi


Почему я говорю вам \enquote{спасибо}? Потому что, если бы не Вы, мои дорогие враги, я
бы не узнала, что такое мир. Мне бы никогда не пришло в голову, что можно
публиковаться в Мюнхене, Копенгагене и Вашингтоне. Мне бы никогда не сказали,
что у меня прекрасные работы в Москве, Екатеринбурге, Питере, Наарии и
Ханты-Мансийске. Я бы не общалась с учёными, к которым раньше даже боялась
обратиться на \enquote{Вы}, а теперь между нами устанавливается царское товарищеское
\enquote{Ты}.

Если бы не Вы, мои дорогие враги, я бы не узнала, что такое друзья. Друзья -
это те люди, которые приезжают в любое время дня и ночи и в ответ на твой мат
спокойно говорят: \enquote{Так. Дальше действовать буду я}. Друзья - это те, кто на
втором конце моей необъятной Родины делают за тебя твои дела. Друзья - это те,
кто вместе с тобой покидает поле боя, жертвуя научной карьерой или, хохоча,
саботирует его. Друзья - это те, кто ставит твое имя на своих работах,
исключается за это из мероприятий и ржёт их организаторам в лицо. Друзья - это
те, кто ругается матом по телефону, чтобы потом не бояться светить тебя во всех
своих благопристойный компаниях. Друзья - это те, кто говорит: \enquote{На}, - и
пропадает, чтобы я не ощущала себя должником. Друзья - это те, кто готов
рискнуть за тебя так, как ты рискнул за него. Друзья - это те, кто остаётся
после жёсткого отсева огромного множества людей в стиле \enquote{и не друг, и не враг,
а так}.

Если бы не вы, мои дорогие враги, я бы не узнала, что такое мужество. Мужество
- это не только атака. Это ещё и терпение, перенесение непереносимого,
переживание того, что пережить не под силу, смирение и прощение. Это - умение
уходить, не оглядываясь. Это - способность быть жёсткой с бесами, вселившимися
в людей, не теряя жалости к самим людям. Это - умение расставаться навсегда без
злобы и сожаления. Это ежедневно вырабатываемая в себе способность, будучи на
войне, преодолевать боязнь смерти, ибо сила в преодолении оной, а не в атрофии.

Если бы не вы, мои дорогие враги, я бы так и оставалась скучным педагогом с
двумя ставками и четырнадцатью дисциплинами. Я бы никогда не узнала, что такое
писать без цензуры, подаваться, куда хочешь, и наслаждаться свободным
творчеством. Я бы не приобрела массу полезных скиллов по донейшену,
краудфандингу, фандрайзингу и фрилансу. Во мне бы умерли честный учёный,
педантичный литературный редактор, живой научный консультант, менеджер по
продаже своих книг, автор поэтической мастерской, копирайтер, блогер, которому
доступны закрытые ресурсы, и многое другое.

И самое главное, мои дорогие враги, чему вы меня научили: отсутствию страха. Не
потому что я стала более храброй (и это тоже), а потому что меня больше нечем
взять. 

Поэтому я благодарна вам за жизнь, за смерть, за весну, за цветы. С днём
Победы, дорогие враги мои. Теперь я знаю, что Девятое Мая - это не день, а
смысл, не праздник, а веха, не дата, а бытие. Я предпочитаю Быть, а не Иметь,
как сказала Марина Цветаева.

\subsubsection{Комментарии}

\begin{itemize}
	
\iusr{Anna Korf Stepanova}

Какая ты...

...я даже представить себе не могла, что в такой маленькой девочке могут
уникальный дар, элитарный интеллект и редкий педагогический талант сочетаться
с таким мужеством, честью, мудростью, добротой...

Я в искреннее приятном шоке... СПАСИБО, ЖЕНЕЧКА.

\iusr{Мария Лелека}

Я бы еще добавила, что друзья всегда поддержат тебя, даже если не разделяют до
конца твоих убеждений. Потому, что им важна ты, как личность,
твое здоровье, твой покой, и они всегда готовы понять и принять
и помочь, чем смогут. Жень, хоть видемся нечасто, но ты хороший
Учитель и Друг! Спасибо за твое творчество.

\iusr{Ирина Жуковская}

Брава! Брависсима! Женечка, вот сейчас крылья расправлены! Во всю ширь. До
горизонта. Я знала, что так будет, я этого ждала. И какое
счастье, что дождалась!

\iusr{Светлана Кллер}

Жень ....ты КРАСАВА! Во всех отношениях! Горжусь тем, что ты победила в этой
войне внутри себя и вне. Читаю и слезы градом катятся из глаз. Читаю и горжусь
тем человеком, который возрастает внутри тебя и обретает истинную свободу. Ты
смогла, ты сумела! Спасибо, Господи, что ты внял нашим молитвам и совершил это
.ибо одному человеку не под силу было бы вынести все то. Обнимаю тебя, sestРА!
Ты яркий пример мужества и стойкости нашим дням. Да благословит Господь щедро
твою жизнь, да дарует тебе свой мир и свою радость в каждом дне, да устоит все
твои пути во благо. Мир тебе, дорогая, мир дому твоему, мир сердцу твоему.
Обнимаю. Люблю. Жду скорых радостных вестей.

\end{itemize}
