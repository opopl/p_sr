% vim: keymap=russian-jcukenwin
%%beginhead 
 
%%file 07_10_2017.stz.news.ua.mrpl_city.1.sunduki
%%parent 07_10_2017
 
%%url https://mrpl.city/blogs/view/sunduki
 
%%author_id burov_sergij.mariupol,news.ua.mrpl_city
%%date 
 
%%tags 
%%title Сундуки
 
%%endhead 
 
\subsection{Сундуки}
\label{sec:07_10_2017.stz.news.ua.mrpl_city.1.sunduki}
 
\Purl{https://mrpl.city/blogs/view/sunduki}
\ifcmt
 author_begin
   author_id burov_sergij.mariupol,news.ua.mrpl_city
 author_end
\fi

В дедушкином доме было три сундука. Не таких, что предлагают сегодня различные
магазины через Интернет. По сравнению с дедушкиными – это не сундуки, а
сундучата. Те были огромны. Постелив кожух или рядно, на эти допотопные
предметы мебели спокойно мог устроиться спать взрослый мужчина, если, конечно,
он не был баскетбольного роста. Самый главный из сундуков стоял в комнатушке,
называемой в семье детской. В ней  еще помещались ремесленная швейная машинка
фирмы \enquote{Зингер}, на которой, сменяя иглы и лапки, можно было шить и шинельное
сукно в два слоя, а то и пристегать к сукну тонкую кожу, и тончайший батист.
Тут же были кровать со спинками, украшенными никелированными шариками. На ней
спали дедушка с бабушкой. Узкий старинный платяной шкаф, где доживали свой век
потраченные молью сюртуки и пальто  образца 1913 года, довершал обстановку
комнаты.

Сундук был пуст. Его крышка была застелена шинельным сукном, \enquote{украшенным}
многочисленными подпалинами от утюга. На этом заменителе портновского стола
дедушка кроил и утюжил тяжеленными чугунными утюгами произведения своего труда.
Необычная особенность конструкции этого сундука состояла в том то, в нижней
части его были встроены два ящика. В одном из них находились инструменты: два
молотка – большой и маленький, клещи, почему-то называемые обитателями дома на
искаженном немецком языке \enquote{абценьками}, пара отверток, ключ для выполнения
несложных сантехнических работ по прозванию \enquote{зверек}, пила-ножовка с обломанной
ручкой. Этот ящик был особенно любим представителями младшего поколения дома.
Пощелкать клещами, постучать молотком – разве это не удовольствие? Кто-то из
них и обломал ручку ножовки, пытаясь, вероятно, распилить полено или какую-то
другую деревяшку, приготовленную для растопки печи. Второй ящик был заполнен
книжками, оставленными племянником дедушки перед уходом на фронт. Это были
учебники да потрепанный томик \enquote{Кобзаря} Тараса Шевченко.

В передней стоял еще один сундук, чуть поменьше, чем в детской. Это было
приданое бабушки. Как она рассказывала, в нем на день венчания находилась икона
Божьей Матери с Младенцем, освященная  в Ладанском Покровском женском
монастыре, расположенном на правом берегу речки Удай в нескольких верстах от ее
родного хутора Руда. Для этого она специально ходила пешком в монастырь. Кроме
того, в скрыне (только так она по-украински называла сундук) находились ленты,
праздничная плахта, рубель и качалка – инструменты для деревенского глажения
белья  и более двадцати сорочек из льняного полотна. Для этих сорочек будущая
невеста на выделяемом отцом клочке земли каждую весну с детских лет сеяла лен.
Когда он вызревал, убирала его, теребила, околачивала, расстилала, сушила,
мяла, трепала, очесывала. Сама пряла, пряжу обрабатывала (в этот процесс, в
частности, входило и вымораживание зимой), ткала на ручном ткацком станке,
холсты летом выбеливала на чистом лугу, у реки, часто раз за разом макая холсты
в воду. Своими руками кроила и шила сорочки. А узоры на них вышивала на
вечерницах – собраниях сельской молодежи, так ярко описанных Николаем
Васильевичем Гоголем в \enquote{Вечерах на хуторе близ Диканьки}.

- Бабушка, а как ты успела всю эту работу проделать?

- А у мене ж богато часу було. Я ж була перестарком, тому и вийшла замiж за
рябого, – и хитровато при этом улыбалась.

У дедушки действительно на лице было несколько отметин – следов от пережитой в
детстве оспы. Много позже, когда бабушки уже не было в живых, зная ее год
рождения и год появления на свет ее первенца, удалось определить возраст
\enquote{перестарка}. Бабушке, когда она пошла под венец, был всего лишь двадцать один
год.

К тому времени, когда у бабушки появились внуки, а старший из них стал всюду
сующим свой нос подростком, от ее приданого остался только сундук без
первоначального своего содержимого. Украшение его в виде набитых накрест
полосок из тонкой жести поизносилось. Кое-где они надорвались, и зазубрины
порой больно ранили руки и цеплялись за одежду. Поблекла некогда темно-зеленая
краска. Теперь он был наполнен разной рухлядью, самой примечательной из них
была доха - верх из черной овчины, а внутренняя сторона – из волчьего меха. Ее
приобрел себе папа, когда работал несколько лет на Кольском полуострове после
окончания строительного института.

Над сундуком была приколочена вешалка. На ней в холодную пору вешали пальто
женщин и москвички мужчин – изобретения советского швейпрома, а также детские
кожушки, перешитые дедушкой для внучат из поизносившихся овчинных шуб взрослых.
Чтобы одежда не пачкалась о побеленную стену, к ней была прибита карта мира.
Она была интересна тем, что на всех природных зонах, обозначенных на ней, были
изображены звери и птицы, обитающие в каждой из них. Белые медведи, северные
олени и песцы - в приполярной области, соболя и медведи - в Сибири, амурские
тигры - на Дальнем Востоке, верблюды и сайгаки - в Средней Азии, слоны, жирафы,
носороги - в Африке, в Северной Америке - кондоры и медведи-гризли, в Австралии
- кенгуру и утконосы и так далее. Рассматривание карты было особенно интересно
после прочитанных приключенческих романов французов Луи Буссенара и Луи
Жаколио. Их книги, отпечатанные до революции с \enquote{ятями} в середине слов и
«ерами» после согласных букв в конце слов, давала почитать мамина приятельница.
Правда, насладиться в полной мере рассматриванием карты можно было только
летом, когда всю зимнюю одежду выносили во двор на просушку.

Но самым интересным был сундук, стоявший на холодной веранде. Интересным прежде
всего тем, что при проворачивании массивного ключа замок, врезанный в толщу
стенки, издавал мелодичные звуки. Это было вместилище съестных припасов. Там
были мешочки с кукурузной крупой и гречневой, перловой и ячневой. Рядом стоял
мешок побольше, с мукой. С ним соседствовала четверть с постным, то есть
подсолнечным маслом. Пожалуй, для молодых читателей нужно дать объяснение этому
сосуду, ушедшему в далекое прошлое. Так вот, четверть - это бутыль емкостью в
переводе на метрическую систему измерений чуть-чуть больше трех литров
жидкости. В царское время этой жидкостью была казенная водка, а в годы
Гражданской войны и позже – самогон. Было там еще самодельное жестяное ведерко
с медом, несколько пластов сала, обернутого в холстину, - остатки кабанчика,
заколотого перед Рождеством.

Вдоль узкой стенки сундука, у самого верха его, был приторочен ящичек. Все
домашние называли его прискрынком. В нем находились сморщенные стручки красного
перца, подвязанные на шнурок, букетик чесночных головок, несколько пачек соды и
кусочки пожелтевшего старого сала, которые использовались для приготовления
заправки борща. Заправку делали, растирая это сало с чесноком или луком в
глиняной чашке по выбору едоков дома. И как же аппетитно выглядела тарелка с
борщом, подернутым золотистой пленкой, источающей легкий парок и непередаваемый
аромат.

Кроме названных предметов, в сундуке стояла макитра, она всегда была пуста. Ее
извлекали в двух случаях. В первом случае, когда пекли пампушки – маленькие
хлебцы – праздничное сопровождение украинского борща. Тогда бабушка в макитре
толкла чеснок, добавляя постепенно по каплям постное масло. Когда чесночная
подливка была готова, в макитру сбрасывали с листа только что испеченные
пампушки, глиняная емкость, покрытая изнутри ярко зеленой поливой, накрывалась
холщевым полотенцем, сложенным вдвое. Оставалось только хорошо потрясти
макитру, чтобы пампушки равномерно покрылись растертым чесноком. Вторым случаем
применения упоминавшегося выше предмета домашней утвари было приготовление
кушанья во время поста – коржиков с маком. Коржики пекли из пресного теста. В
своем первозданном виде они были необычайно тверды, но сдобренные смесью мака и
меда, растертого в макитре, предварительно тщательнейшим образом помытой и
пропаренной, приобретали мягкость и вкус, запомнившийся на всю жизнь. Однако
для детей из содержимого сундука с \enquote{певучим} замком самым притягательным была
сушка – искривленные жарким летним солнцем пластинки яблок. Какие он были
сладкие!

Вспомнились сундуки, а за ними пошли одна за другой разновременные картины
прошлого. Дедушка, высокий, сухощавый, с впавшими щеками, коротким бобриком
волос на голове, с седеющей бородкой, в старомодном двубортном пиджаке, в
лацкан которого заткнуто несколько иголок. Бабушка, грузная, неспешно
двигающаяся не из-за природной медлительности, а по той причине, что у нее
больны ноги. Тетя, нянчащая свою младшую дочку, пытается ее убаюкать. Дядя, с
намыленным лицом \enquote{правящий} на солдатском ремне опасную бритву. Папа, после
ужина сидящий на стуле и пытающийся прочесть газету, он время от времени
подремывает, газета вот-вот выпадет из рук, вздрагивает, снова пытается читать.
Папа очень рано уезжает на работу и поздно возвращается домой. Мама совсем
молодая, с гладко зачесанными на пробор волосами, с мягкой доброй улыбкой на
лице. Готовая всем и всегда помочь. Щемящие душу воспоминания...
