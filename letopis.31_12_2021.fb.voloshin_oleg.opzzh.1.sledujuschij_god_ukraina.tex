% vim: keymap=russian-jcukenwin
%%beginhead 
 
%%file 31_12_2021.fb.voloshin_oleg.opzzh.1.sledujuschij_god_ukraina
%%parent 31_12_2021
 
%%url https://www.facebook.com/oleg.voloshin.7165/posts/5045714285461592
 
%%author_id voloshin_oleg.opzzh
%%date 
 
%%tags 2022,geopolitika,rossia,ukraina
%%title Следующий год непременно изменит в Украине многое
 
%%endhead 
 
\subsection{Следующий год непременно изменит в Украине многое}
\label{sec:31_12_2021.fb.voloshin_oleg.opzzh.1.sledujuschij_god_ukraina}
 
\Purl{https://www.facebook.com/oleg.voloshin.7165/posts/5045714285461592}
\ifcmt
 author_begin
   author_id voloshin_oleg.opzzh
 author_end
\fi

Следующий год непременно изменит в Украине многое. Восемь лет фактор России в
политических процессах удивительным образом присутствовал во всех спорах и
дискуссиях, но при этом сама соседняя держава будто за игровым столом не
присутствовала. И это при тотальной зависимости от поставок энергоресурсов,
статуса одного из крупнейших торговых партнёров, сотнях тысяч заробитчан и
миллиардах пересылаемых ими денег. Не стоит забывать и о настроениях в
обществе: при честной конкуренции выступающие за нормализацию отношений с
Россией силы имели бы четверть мест в парламенте, а после реинтеграции
Донбасса, вероятно, и больше трети. 

Тем не менее, украинцев упорно убеждали во взаимоисключающих вещах: Кремль
якобы спит и видит, как любыми методами влиять на Украину, но при этом по факту
никакой реальной роли в украинской политике не играет, его интересы можно
безнаказанно игнорировать, все связанное с Россией и просто русским притеснять
и преследовать. Сейчас в Москве четко и громко сказали: «Довольно». Ни одна
великая держава не потерпела бы такую ситуацию у своих границ продолжительное
время. А тесная историческая и культурная связь тем более не позволяет просто
относиться к соседу как к чуждому, отрезанному ломтю. Кстати, не будь этого
фактора, Россия могла бы предпринять гораздо более жесткие действия в
экономической и энергетической сферах - чего чужих-то жалеть. 

2022 год станет непременно временем возврата России в качестве одного из
ключевых игроков на украинской шахматной доске. Переговоры Байдена с Путиным в
принципе это уже зафиксировали. И в Белом доме к этому целиком готовы как в
силу общего переноса фокуса внимания на сдерживание Китая, так и благодаря
пониманию того, что американское влияние в украинских делах (особенно, служащих
источников доходов для западной элиты) в любом случае сохранится. Никто в
Москве не собирается назначать нам прокуроров или формировать наблюдательный
совет Укрзализныци. Проблема для США и ЕС, реально готовых к отказу от
выталкивания России из Украины, только в том, чтобы соблюсти внутренние
приличия с точки зрения их же собственной недавней бескомпромиссной риторики.
Хотя неготовность принять страну в западные институции (особенно, в ЕС) была
всем очевидна ещё до того, как Кремль перешёл во внешнеполитическое
наступление. 

Если же вдруг договориться не удастся, то… В любом случае прежнего формата
существования Украины, клептократические власти которой прикрывают собственную
некомпетентностей постоянным повышением градуса антироссийской истерии. 

Грядущий год будет очень непростым с учётом падения доходов граждан, инфляции,
дефицита энергоресурсов и полной некомпетентности правящего режима во всем,
кроме освоения бюджетных средств. Но он может принести восстановление давно
искусственно нарушенных балансов. Без чего весь корабль украинской
государственности пойдёт на дно, когда ужасный конец для многих станет
предпочтительнее затянувшегося ужаса без конца.

\ii{31_12_2021.fb.voloshin_oleg.opzzh.1.sledujuschij_god_ukraina.cmt}
