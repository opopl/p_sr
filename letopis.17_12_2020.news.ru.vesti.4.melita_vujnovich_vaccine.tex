% vim: keymap=russian-jcukenwin
%%beginhead 
 
%%file 17_12_2020.news.ru.vesti.4.melita_vujnovich_vaccine
%%parent 17_12_2020
 
%%url https://www.vesti.ru/article/2499795
 
%%author 
%%author_id 
%%author_url 
 
%%tags covid_vaccine
%%title Мелита Вуйнович: вся надежда на вакцину от COVID-19
 
%%endhead 
 
\subsection{Мелита Вуйнович: вся надежда на вакцину от COVID-19}
\label{sec:17_12_2020.news.ru.vesti.4.melita_vujnovich_vaccine}
\Purl{https://www.vesti.ru/article/2499795}

\ifcmt
pic https://cdn-st1.rtr-vesti.ru/vh/pictures/xw/308/043/0.jpg
\fi

\index[names.rus]{Вуйнович, Мелита!Представитель Всемирной организации здравоохранения в России, 17.12.2020}
\index[rus]{Коронавирус!Россия!ВОЗ, 17.12.2020}

Представитель Всемирной организации здравоохранения в России Мелита Вуйнович
рассказала в эфире телеканала "Россия 24", как ВОЗ оценивает ситуацию с
коронавирусом в России и остановит ли вакцинация эпидемию.

\textbf{- Сейчас мы видим, что в разных странах ситуация остаётся сложной. В Германии
ужесточается карантин, отменяются новогодние празднества. Почему же не помогают
принимаемые меры предосторожности и попытки остановить пандемию?}

- Надо понять, что человеческое поведение в этой эпидемии, как много раз мы
говорили, практически провоцирует вирус идти дальше. Если меры приняты, но они
не соблюдаются, тогда это значит, что меры неприменимы. Из-за этого мы видим в
разных странах, что даже когда рекомендуется масочный режим, физическая
удалённость, отмена массовых мероприятий – мы наблюдаем ту картину, которую
имеем. Это значит, что до того момента, когда будет массовая вакцинация и
подъём общего иммунитета, эффективность применяемых мер будет зависеть от того,
насколько много людей соблюдают ограничения. Если 60\%, то этого недостаточно.
Нужно минимум 90\%.

\textbf{- Сейчас продолжается много споров среди экспертов по поводу того, кто же
всё-таки является наиболее опасным и наиболее активным переносчиком вируса. Это
люди с легкими симптомами или это бессимптомные носители?}

- Надо учитывать, что число зараженных очень большое. Когда вы имеете такое количество инфицированных, то практически расширение будет большое, и абсолютно неважно, один человек заражает одного или десять. Потому что, если эти следующие в цепочке, которых заразили, будут заражать десять снова – это идёт расширение, как ветки одного дерева. Из-за этого любой человек, который имеет сомнения или который мог быть заражён, должен защищать себя от инфекции, даже если не знает, что он заражен.

\textbf{- Вы упомянули о вакцинах. Сейчас уже появилось несколько вакцин по
всему миру, в том числе несколько российских. Но мы знаем, что практически все
вакцины имеют свои особенности и свои требования по транспортировке, хранению,
проведению вакцинации. Могут ли вакцины первого поколения остановить пандемию?}

- Надо понимать, что нужно достаточное количество вакцин, чтобы пересечь
пандемию. Сейчас главный вопрос – это доступ к вакцине везде в мире. Другой
большой вопрос – это вопрос производительности. Насколько производители вакцины
смогут ответить на спрос? Следующий год окажется ключевым, чтобы мы могли
практически иметь доступ к вакцинам, которые могут помочь пресечь эту большую
эпидемию.

\textbf{- Вы справедливо упомянули о сложностях масштабного производства новых вакцин.
Какую политику будет проводить Всемирная организация здравоохранения для того,
чтобы обеспечить доступность этих вакцин в разных странах? Может быть, будут
выделены дополнительные средства, может быть, будут проведены какие-то
договорённости со странами-производителями, что вы будете делать?}

- Это делается с апреля. Кроме этого, ВОЗ работает с каждой страной, у которой
есть сложности, чтобы разработать план по вакцинации. Ведутся переговоры с
разработчиками. Им предложено подавать документы о своей вакцине в единую
ковакс-систему под эгидой ВОЗ, где после одобрения вакцины могут быть
предоставлены разным странам.

\textbf{- Если какая-то страна проявит интерес к вакцинации российской вакциной,
Всемирная организация здравоохранения может эту страну поддержать финансово или
рекомендациями российской вакцины?}

- Нет. Двусторонние отношения между странами – это двусторонние. Каждое
государство может сделать сертификацию вакцины в своей стране. ВОЗ здесь не
вмешивается. Но если ВОЗ сделала сертификацию, тогда даётся рекомендация, и
этот процесс очень часто учитывается, и делается гораздо быстрее сертификация в
этой стране. Ковакс-система – общий механизм, который обеспечивает закупки
вакцин, но это не отменяет двусторонние договоренности между странами.

\textbf{- В этой системе сейчас уже зарегистрированы какие-то вакцины от ковида? Есть
ли среди них российские?}

- Мы знаем, что российские производители проявили интерес, подтвердив это
письмом в штаб-квартиру ВОЗ. Для этого надо представить нам необходимые
документы для оценочной группы. Помимо российской вакцины, рассматриваются
Pfizer, AstraZeneca, Moderna. Результаты по каждой вакцине будут опубликованы
отдельно. Это значит, что вакцины не сопоставляются между собой. Делается
оценка сопоставимости со стандартами – по качеству, безопасности,
эффективности, холодовой цепи, транспортировке, устойчивости и другим
параметрам.

\textbf{- Да, там очень много условий. Сколько времени может занимать сертификация? И
можете ли вы назвать, какие российские вакцины претендуют сейчас на участие в
этой системе?}

Производитель "Спутника V", например, предоставил письмо, в котором подтвердил
свою заинтересованность. Потом мы услышали на международной конференции
заявление Роспотребнадзора, что последующая вакцина тоже зарегистрирована.

Делается обзор всех документов, самое главное, чтобы включались или
промежуточные, или окончательные результаты третьей клинической фазы
исследований вакцины. Оценка включает в себя переговоры с национальным
регуляторным органом, инспекцию – это очень сложный процесс, но ВОЗ делает всё
возможное, чтобы сократить его. Назвать конкретные сроки не могу, потому что
они зависят от многих факторов.

\textbf{- В последнее время стали говорить о новом британском штамме коронавируса,
поскольку именно там он был обнаружен, якобы он более заразен, чем известный до
сих пор. И в международной классификации болезней в 10-й версии появилось
определение постковидного синдрома. Что это такое и как с этим бороться?}

- В Великобритании выявлено более 1000 случаев – вирус находится под постоянным
наблюдением по всему миру. Учёные могли увидеть изменение генома, сейчас ещё
изучается его влияние. На этот момент учёные Великобритании не увидели
опасность того, что изменение генома будет влиять на эффективность вакцины.
Исследования продолжаются

Что касается посткоронавирусного синдрома, здесь мы услышали от многих стран,
что после того, как человек вылечился, у определённого процента остаются разные
последствия по здоровью (лёгкие, сердечно-сосудистая система). Сейчас
собираются данные со всего мира о таких пациентах, и ВОЗ просил все
министерства здравоохранения и большие университетские и клинические больницы,
чтобы следили за такими людьми.

\textbf{- И есть ещё признаки энцефалопатии, поражений головного мозга. Это тоже бывает
отложенным действием коронавируса.}

- Да.

\textbf{- Как вы оцениваете ситуацию с коронавирусом в России в сравнении с другими
странами? Мы видим, что число жертв по всему миру превышает прогнозы, о которых
мы говорили ещё год назад. Ожидаемо ли такое количество жертв для Всемирной
организации здравоохранения?}

- Всемирная организация здравоохранения не делала такое моделирование по всему
миру. Это очень тяжело прогнозировать. Конечно, масштаб пандемии
беспрецедентный, никто этого не ожидал, хотя мы все знали, что ситуация очень
серьёзна.

Что касается России, конечно, очень хорошо, что все данные пересматриваются,
что делается их анализ. И мне кажется, общий анализ того, что произошло в этом
году, будет в 2021 году. Надеемся, что ситуация будет идти к лучшему. Сейчас
наблюдаем постепенное снижение заболеваемости, но число ежедневных случаев еще
очень высоко. Весной, когда выявлялось 10 тысяч заболевших в день, мы были
очень расстроены, а сейчас мы говорим о более чем 25 тысяч.

Но, как я говорю, ситуация обнадеживает. В Москве идет снижение случаев без
какого-то нового скачка. Очень важно соблюдать меры безопасности, дистанцию,
чтобы не было нового очага заражения, какого-то невиданного подъёма.
