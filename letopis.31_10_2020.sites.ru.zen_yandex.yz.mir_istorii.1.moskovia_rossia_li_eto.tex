% vim: keymap=russian-jcukenwin
%%beginhead 
 
%%file 31_10_2020.sites.ru.zen_yandex.yz.mir_istorii.1.moskovia_rossia_li_eto
%%parent 31_10_2020
 
%%url https://zen.yandex.ru/media/history_world/moskoviia--rossiia-li-eto-5f9d63269ac0705ae4842a0c
 
%%author Мир Истории
%%author_id yz.mir_istorii
%%author_url 
 
%%tags moskovia,russia
%%title Московия – Россия ли это?
 
%%endhead 
 
\subsection{Московия – Россия ли это?}
\label{sec:31_10_2020.sites.ru.zen_yandex.yz.mir_istorii.1.moskovia_rossia_li_eto}
\Purl{https://zen.yandex.ru/media/history_world/moskoviia--rossiia-li-eto-5f9d63269ac0705ae4842a0c}
\ifcmt
	author_begin
   author_id yz.mir_istorii
	author_end
\fi

\index[rus]{Русь!История!Московия, наследие Руси, 31.10.2020}

Сегодня в публицистике, а еще чаще среди политиков можно услышать весьма
странные комментарии относительно наследия Руси. В России говорят о том, что
Русь – исключительно «российское наследие», украинские политики заявляют, что
существовала только «Русь-Украина», а Россия – это «Московия».

В самой России не любят, а иногда даже избегают термина «Московия», хотя это не
решает исторический (и политический) спор, а наоборот создает еще большую
путаницу. Давайте попробуем разобраться, откуда появился термин «Московия»,
какое он имел значение, а главное: Московия – это Россия?

\ifcmt
  pic https://avatars.mds.yandex.net/get-zen_doc/230233/pub_5f9d63269ac0705ae4842a0c_5f9d69059ac0705ae48e0efa/scale_1200
  caption Карта Russia фламандского картографа Герарда Меркатора, 1595 г. Московия (указана стрелкой) обозначена как одна из её местностей.
\fi

\subsubsection{Вопрос исторического наследия}

Русь – это вообще чье наследие? Это вопрос уже больше века волнует историков,
журналистов и политиков. И вот тут кроется самая большая сложность: если мы
хотим получить простой ответ – то нам нужно слушать политиков, причем
желательно «своих», они нам точно правду скажут. То, что звучит в газетных
статьях и высказываниях политиков, часто около научные разговоры. Приведу
примеры нескольких таких полемик:

\begin{itemize}

\item 1. Украинский историк М. Грушевский вводит термин «Украина-Русь», чтобы подчеркнуть факт того, что украинцы формируются в период Руси, а само государство – и их наследие. Историк писал:

		«Мы знаем, что Киевское государство, право и культура, являются творением
		одной народности, а Владимиро-Суздальская держава – другой». 

Другие журналисты (и к сожалению историки) продолжают эту мысль, говоря, что
наследник должен быть один: если это Украина, тогда Россия должна искать другие
корни своей истории.

\item 2. В России выводят начало Руси с 862 года, с Новгорода, тем самым
				заявляя, что это «раньше прихода Олега в Киев». Да и сам Олег (как и
				Рюрик, родоначальник династии) был с Новгорода.

Затем идут дальше, находят в истории киевских князей этапы биографии, когда они
начинали свою «политическую карьеру» с нынешних российских земель: Святослав,
Владимир Великий и Ярослав Мудрый. Чаще всего среди россиян звучат
династические (Рюриковичи правили в Москве до конца 16 века) и церковные
(киевский митрополит перебрался в Москву) аргументы в пользу «монополии на
наследство».

\ifcmt
  pic https://avatars.mds.yandex.net/get-zen_doc/1328418/pub_5f9d63269ac0705ae4842a0c_5f9d6d0a9ac0705ae494cf70/scale_1200
  caption Русь при Ярославичах / Koryakov Yuri / ru.wikipedia.org
\fi

\item 3. В украинском парламенте звучат мысли о том, что «Россия – это
Московия, она украла украинское прошлое». Российские политики
отказываются принять факт того, что украинский этнос
формировался в период Руси (а некоторые и вовсе, что он
существует). Все это примеры шовинизма с обеих сторон.

Украинский историк Ярослав Грицак проводит параллели между Западной и Восточной
Европой: 

\begin{leftbar}
	\begingroup
		\em
			«На Западе похожий спор вызывало наследие Карла Великого и империя
				Франков. Но постепенно сложился компромисс, что итальянцы, немцы,
				французы, а также швейцарцы причастны к истории империи».
	\endgroup
\end{leftbar}

Однако не тех, не тех в тот период не существовало, они лишь формировались.
Похожий компромисс в отношении Руси использовали в советское
время: Древняя Русь считалась «колыбелью трех
восточнославянских» народов.

\ifcmt
  pic https://avatars.mds.yandex.net/get-zen_doc/4004066/pub_5f9d63269ac0705ae4842a0c_5f9d7075b09e797ceb179e13/scale_1200
  caption Русь, Польша и Литва в 1139 году
\fi

\end{itemize}

Если историку отвечать на вопрос, чья же Русь, то ответ будет весьма
противоречивый на первый взгляд: ничья! Попробую объяснить свое
мнение.

Князья Олег, Святослав и другие не были ни украинцами, ни россиянами. Нации –
это часть мировоззрения модерного мира, начиная с 19 века. Само
слово «Русь» пришло из Скандинавии. Государство не было
этническим, это был конгломерат разных племен (славянских,
тюркских, балтийских и финно-угорских) во главе с варяжскими
правителями. Это было типичное государственное образование
раннего средневековья: «мы вас защищаем от окружающих врагов,
вы нам дань, мы ее на продажу».

Этничность вообще не имела значение. Австрийский историк Андреас Каппелер в
книге «Малая история Украины» также затрагивают тему современного спора России
и Украины о наследии: 

\begin{leftbar}
	\begingroup
		\em
			«Он [спор], если подходить трезво, абсолютно лишний, потому что в нем оперируют современными национальными категориями мышления, далекими от Средневековья».
	\endgroup
\end{leftbar}

Именно в период Руси сформировались элементы идентичности, общие для восточных
славян: православное христианство, язык и письменность на основе кириллицы.
Кроме того, именно в Руси зародились названия, которые активно использовались
после ее распада, порождая путаницу.


\ifcmt
  pic https://avatars.mds.yandex.net/get-zen_doc/1537151/pub_5f9d63269ac0705ae4842a0c_5f9d6db23910530e0dea39ae/scale_2400
  caption Пример совместного использования: Russia, в просторечии Moscovia, Космография Блау, Голландия, 1645 г. 
\fi

\subsubsection{История названий}
\index[rus]{Русь!История!Малая Русь, Великая Русь}

После начала феодальной раздробленности каждая из частей Руси использовала
название «Русь»: от Галиции до Новгорода и Рязани. Затем князь Галицкого
княжества Юрий Львович добился для своего государства права на митрополита,
которых теперь в восточной Европе стало два: в Москве и Галиче. Чтобы не
путать, патриарх Константинополя придумал названия: «Малая Русь» и «Великая
Русь».

В западной Европе появлялся термин «Россия» - та же «Русь», но написано на
латинский лад. Потом Польша захватила Галицию и назвала ее «Русское
воеводство», а в Великом княжестве Литовском началась гражданская война: князь
Свидригайло хотел создать свое государство – «Великое княжество Русское», с
центром в Киеве. На востоке усиливается Владимиро-Суздальское княжество,
которое становится Московским, а его правители начинают называть себя «русскими
князьями» и «собирателями земель русских». Все это вносит ужасную путаницу в
работу картографов 15-16 века.

Путешественник и картограф Фра Мауро на своей карте 5 раз указывает название
«Россия». На других картах восточной Европы эту путаницу пытались предотвратить
используя название «Московия». Тогда Московское княжество, а затем царство
стремительно расширялось, победило Орду, вступило в войны с Литвой, Ливонским
орденом, затем Польшей. 

\ifcmt
  pic https://avatars.mds.yandex.net/get-zen_doc/1362253/pub_5f9d63269ac0705ae4842a0c_5f9d6a5349505f6811f23cdf/scale_1200
  caption Фрагмент карты мира Фра Мауро, 1459 год
\fi

\index[writers.rus]{Маржерет, Жак!17 век!Состояние Российской
				державы и Великого княжества Московского, 1607, Париж}

Московское царство было тогда главным «ньюсмейкером» в восточной Европе,
отсюда и такой интерес к его границам. Поэтому постепенно картографы и
путешественники задумались об уместности использования термина
«Московия».

Француз \textbf{Жак Маржерет} в начале 17 века написал книгу \emph{«Состояние Российской
державы и Великого княжества Московского»}, которая в 1607 году была
напечатана в Париже. В ней он отмечал:

\begin{leftbar}
	\begingroup
		\em
«Считают ошибкой именовать жителей московитами, а не русскими, но так делаем не только мы, далекие жители, но и их соседи. Сами люди, на вопрос, какой они нации, отвечают: Russac, что означает – русский». 
 
	\endgroup
\end{leftbar}

Для автора говорить «Московия» - это так же, как называть французов
«парижанами». С начала 18 века Московское царство начинают называть «Россией»,
то есть «Русью» на латинский лад. Окончательно Петр I в 1721 году меняет
официальное название: Московское царство становится Российской империей.


\ifcmt
  pic https://avatars.mds.yandex.net/get-zen_doc/118779/pub_5f9d63269ac0705ae4842a0c_5f9d6f7b9ac0705ae49912b8/scale_1200
  caption Российская империя (1745 год)
\fi

Так мы плавно подобрались к вопросу о том, является ли Московия Россией. Мне
кажет, что и да, и нет. Несомненно, этап Московского княжества и царства –
важная часть истории России. Это период, когда одно княжество захватило все
остальные и на карте Восточной Европы возникло единое государство.

С другой стороны, Россия – это не только Москва и ее история. Сегодня в России
проживают десятки народов, которые имеют свою историю, а сводить все к истории
Московского царства – это не современный путь формирования политической нации.
Россия – это единство в многообразии.

\index[writers.rus]{Паули, Густав-Теодор!«Этнографическое описание народов России», СПб. 1862}

\ifcmt
  pic https://avatars.mds.yandex.net/get-zen_doc/1703615/pub_5f9d63269ac0705ae4842a0c_5f9d6af51f9f737992305ea9/scale_2400
  caption Великороссияне разных губерний (Густав-Теодор Паули. «Этнографическое описание народов России», СПб. 1862)
\fi

Таким образом, можно сказать, что «Московия» - это не Россия, это ее часть, а
также отдельный этап истории формирования российской (а не «русской»)
идентичности. Вместе с тем, не стоит забывать, что «Московия», как и «Россия»,
- это изначально латинские термины, а не самоназвания. В конечном итоге
правители выбрали термин «Россия», так и начался новый этап ее истории.

\begin{itemize}
\item \cusr{Одереж}

Глупо искать. Поясню. Мы все россияне русский татарин калмык чуваш сахалинец.
Наши родные всех кровей. Я гуран жена моя белорусска у дочки
муж украинец а у сына татарка все мы любим друг друга ....

\item \curs{Олег Волков посмотреть можно!}

Одереж, Всё правильно говоришь,только нациков пора проучить!Время пришло!Нужно их на колени поставить перед матерями ДОНБАССА и пусть они их судят!А оставшихся в живых, выдворить в 3 области,где их историческое место,а не засовывать голову в песок и говорить ничего не можем сделать!Можем и ещё как - можем!Сколько ещё детей должно погибнуть?Сколько матерей и не в чём не виновных мирных жителей должно погибнуть из - за бездействия?Нам нужно вводить войска и плевать на крики! Мы быстро помойку захлопнем,если кто - то её откроет!У нас есть чем ответить! В АМЕРИКЕ - АЛЯСКА + 3 штата!У АНГЛИИ часть РОССИЙСКИХ земель! Германии можем напомнить,что,ГДР принадлежала и принадлежит - РОССИИ! Почему мы с такой быстротой и лихостью освободили РЕЙСТАГ?Не потому ли,что,рейхстаг строили наши предки - РУСИЧИ(по теперешнему - РОССИЯНЕ! И мы, его, восстанавливали! А посмотрите на всё остальное!Там прошли раскопки,даже ГИТЛЕР не тронул и не стёр всё,потому,что,он думал,что,всё строили АРИИ,а раз так,то,нужно сохранить(ведь они считали себя - АРИЙЦАМИ!)А строили - предки РОССИЯН,то есть - РУССКИХ!Вы почитайте эти названия и сами всё поймёте! А теперешний РЮГЕН,он во все не РЮГЕН,его названия - БУЯН!И столица у него называлась - АРКОНА!Пора всё вернуть!ВРЕМЯ ПРИШЛО! И справедливость ТРЕБУЕТ!

\item \cusr{Сибиряк Петров}

Одереж, что за национальность сахалинец!?)) Тогда и уралец, и сибиряк
национальности?)) Смешал всё в кучу,"россиянин". Русские -
основа, стержень и создатели России и без них "россияне"
разбегутся по своим чувашиям и татариям.

\item \cusr{елена матвеева}

Сибиряк Петров, В Москву в 30 годы 20 века очень много мордвы переселилось и других финно-угров,а улица АРБАТ татарское название,после войны 1612 года там были татарские жилые кварталы и торговые ряды.Мы русские россияне нация собранная из наций и народностей,в отличии от всяких там хохлов и бульбашей,что от жадности и зависти к нам подселились и понаехали,а теперь стучат себе в грудь что они люсски.

\item \cusr{елена матвеева}

Сибиряк Петров, В Сибирь при советах хохлов много набежало и этнически себя чистыми великороссами считают.

\item \cusr{Граф Галас}

Олег Волков посмотреть можно!, ну какой ты русич? Ты москаль, к Руси никакого отношения не имеешь.

\item \cusr{Матвей Рустов}

Сибиряк Петров, правильно, еще Бисмарк сказал:

\enquote{ Даже самый благополучный исход войны никогда не приведёт к распаду России, которая держится на миллионах верующих русских греческой конфессии. Эти последние, даже если они впоследствии международных договоров будут разъединены, так же быстро вновь соединятся друг с другом, как находят этот путь друг к другу разъединённые капельки ртути.}

И для Украины:

\enquote{ Не надейтесь, что единожды воспользовавшись слабостью России, вы будете получать дивиденды вечно. Русские всегда приходят за своим. И когда они придут — не надейтесь на подписанные вами иезуитские соглашения, якобы вас оправдывающие. Они не стоят той бумаги, на которой написаны. Поэтому с русскими стоит или играть честно, или вообще не играть}

\item \cusr{Георгий}

ИСТОРИЯ РОССИИ- ВОЙНА- ПРИРОСТ ТЕРРИТОРИЙ. ЕСЛИ ЕЕ УРЕЗАЮТ-ЕЩЕ БОЛЬШИЙ ПРИРОСТ.
РОССИЯ- ЗЕМЛЯ МАГНИТ- ХРАНИМЫЙ БОГОМ. ДЕЛО ВРЕМЕНИ.......

\item \cusr{Павел Ряряев}

				"...С начала 18 века Московское царство начинают называть «Россией», то есть «Русью» на латинский лад..."

Поскольку написано несколько невнятно и двусмысленно (кто начал называть страну Россией), уточню:

Как раз на "латинский лад" это и будет в разных вариантах именно "Русь" (Russia, Russie, Russland итд...), а "Россия" - это уже на греческий лад со времён Никона...
\item \cusr{DurDom}

Все разговоры о том, что на самом деле называется Русью будут
бессмысленными до тех пор, пока эти псевдо- историки, которые
слепили свои докторские и кандидатские диссертации на основе
сфальсифицированной версии Истории государства Российского,
написанной придворным литератором Н.Карамзиным по заказу немки
Екатерины2., будут руководствоваться фальшивой летописью
"Повесть временных лет", которая якобы была написана каким-то
безвестным киевским летописцем Нестором, а на самом деле эта
Повесть была написана немецким историком Миллером в середине 18
века, который при этом заявил, что он списал свою копию с
подлинной летописи, но при этом не представил подлинника и не
сказал где его взял. А никакого подлинника просто не было!!
Кроме Миллера сочиняли "древние" русские летописи ещё два немца
- Байер и Шлёцер, которых тоже пригласили в Россию для
сочинения фальшивой типа "древние" русские летописи. А потом
все фальшивки передали Карамзину, который и написал фальшивую
Историю Руси- России., и которой до сих пор засирают мозги
российским школьникам. Московия - это безусловно русская
идентичность. Московское княжество разросталось постепенно в
результате завоеваний московскими князьями соседних удельных
русских княжеств. Главную роль сыграл Иван3 , который за годы
своего правления увеличил территорию Московского княжества в 5
раз , покорив Великий Новгород, и включил его в состав своего
княжества. А если иностранцы называли Московское княжество
Русью, то это их право, Как хотели так и назвали, и на своих
топографических картах нарисовали. Мы же называем немецкий
Дойчлянд Германией, и при этом немцы на нас не обижаются.

\item \cusr{Андрей Гуркин}

Была огромная империя Русов, много было провинций

\item \cusr{Qainar Moldash}

Как логично все расписано: московия и есть. москву, как кормили, так и кормят

\item \cusr{Юрий}

Россия искаженное слово от Русь. Этимология слова Русь раскрыта в
толковом словаре русского языка выдающимся этнографом В.Даль, означает
светлое место. Русь это место размещения деревни (деревянных жилищ )
поселения родовой общины Людей на берегу реки. Согласно выводам В.Даля
население восточной Европы до крещения называли свою этническую
принадлежность Люди Мужской род Люд женский род Людь. Славянами людей
окрестила православная церковь, ибо сказано Все рабы божьи.

\item \cusr{petrovich}

Наше государство и учёные историки не решают вопрос исторического
именования территорий и народностей. Хотя бы конференцию историков
-представителей бывших республик созвали бы. Да принародно поговорили.
Вроде хотели с западными славянами пообщаться. Это хорошо. Время то
идёт. Надо же предлагать "умным" людям способы и формы единения ,жить
то надо. Всё равно объединимся. Или в жизни или в погибели. Нам
выбирать. Амбиции историков, закрепленные в разных вариантах учебников
только прибавляет хаоса в голову школяра. Это плохо. Люди сами
занимаются стихийным и исследованиями . Это хорошо. И судят историю
народа и государства - это плохо. "Академики" Идеологии боятся, а как
без неё, если государство существует. И не одно.

\item \cusr{Serge Mark}

Я живу не в Москве, но естественно русский, но называться "московитом"
("москалем") и пр. не считаю для себя зазорным. Москва создала империю. Петр
столицу в Петербург, но и только. Московский диалект лег в основу русского
разговорного языка ("аканье" ), Москва, как бы ее не хаяли в провинциии ,дорога
сердцу каждого настоящего русского человека... И лишь горечь возникает от
ощущения, что великий русский город Киев, пока находится во власти народа даже
не имеющего по большому счету самоназвания,не имеющего общего менталитета и
общей исторической гордости, народа, продававшего себя всегда и везде по сходной
цене. Конечно это не относится к многочисленным людям, говорящим и думающим на
Украине по-русски и не понимающим в какую западню они попали, согласившись с
антирусской политикой руководителей псевдостраны и ее религиозных деятелей. Если
они думают,что это берет начало после Донбасса и Крыма, то глубоко
ошибаются, Это началось даже гораздо раньше, чем правление подонка Кравчука.

\item \cusr{Аляксандр Гурыновіч}

Ак-Урусь, или Белая Русь – Московия

\url{https://nicolaev.livejournal.com/1499995.html}

"Поскольку монгольский Белый Хан был верховным владыкой всех русских княжеств,
то и после распада Золотой Орды его власть, по московской интерпретации,
перешла к московскому Великому князю, который стал Белым Ханом. Или в переводе
на адаптированный греческий - Царем.

Термин «самодержец Белой Руси» исчезает только у Петра I, и вместо него
появляется другое - «самодержец России». Таким образом, благодаря усилиям Петра
Московия - Белая Русь впервые в истории меняет свое название - на имя
«Россия»."

\item \cusr{Светлана Инусова}

"Украинский историк Ярослав Грицак проводит параллели между Западной и
Восточной Европой: «На Западе похожий спор вызывало наследие Карла Великого и
империя Франков. Но постепенно сложился компромисс, что итальянцы, немцы,
французы, а также швейцарцы причастны к истории империи»."

Вранье украинского националистов. Они всегда приводят этот пример, отрицая
множество фактов. Грицаку надо бы проводить параллель не с империей франков,
так как она распалась в начале 9 века. Когда Русь только зарождалась. А с
феодально раздробленными Польшей, Италией, Германией, Францией. Вот такое
сравнение будет корректным. Древнюю Русь населяли в основном русские. Древняя
Русь была главным образом государством одной русской народности. А империю
Карла Великого населяли галло-римляне, италийцы, франки, германцы, южные
славяне и другие совершенно разные народы. Не случайно Карл Великий принял
титул императора в Риме как наследник прежних римских императоров. Римская
империя была многоэтничным государством. Этнографические же различия
представителей древнерусской народности 12 века не превышали такие у
великорусов 19 века. Россия-Русь и Древняя Русь одна и та же страна и
государство в разное время. У Франции тоже более половины территории веками
входило в состав Англии. Это не повод считать что Франция и французы появились
в 15-16 веках. И не повод считать что немцы появились в 19 веке. Немецкая
народность сложилась в 10 веке. Хотя теперь австрийцы отдельная нация, никто не
говорит что Германии и немцев до 19 века не было.

\item \cusr{КуКу ЛяЛя}
Век они разбираются. Вы лучше скажите, кто теперь вам светлое будущее будет строить? 


\item \cusr{Олег Волков посмотреть можно!}

ДА!все мозги заплёл! Вообщем так,РЮРИК и Олег и дружина -
РУСИЧИ!Никакими СКАНДИНАВАМИ они - никогда не были! не были! Они пришли
с РУСЬКОГО МОРЯ(НЫНЕ - ЧЁРНОЕ МОРЕ!) Так же РУСИЧИ жили на острове -
БУЯН - ныне - остров - РЮГЕН! Страна называлась - АРКОНА! Но,из - за
того,что туда свозились товары и деньги после торговли,АРКОНА стала
богатой и на неё напали АНГЛО - САКСЫ и пожгли её,а людей - истребили!
Теперь,время пришло заплатить по счетам! Украины,как и КИЕВСКОЙ РУСИ -
никогда не было!УКРАИНА (теперь не до УКРАИНА)переводится как ОКРАИНА
РУСИ!(И если можно,не называйте впредь её - УКРАИНОЙ никогда! У них
было всего 3 области и только!Скоро всё встанет на своё место!)Что
касаемо КИЕВСКОЙ РУСИ, КИЕВСКАЯ РУСЬ получила своё название из
того,что,один князь рассказал свой сон на вече и все его тут - же
подхватили,да,дескать мы,все этот сон видели!Вот,так и прижилось это
название!А РЮРИКА с братьями и дружиной пригласили на княжение с
КИЕВА(для того,чтобы,прекратить распри(лад творить,то есть - мир между
князьями!)Никто не хотел брать на себя смелость княжить,по этому и
пригласили их! Это даже не обсуждается,потому,что,это - неоспоримый
ФАКТ! Вы бы,прежде чем ересь писать и подхрюкивать
НОРМАНИСТАМ,настоящую историю РУСИ(РОССИИ) изучили! И
УВЫ,ФРАНЦИЯ,совсем не ФРАНЦИЯ,это - настоящий - ИЗРАИЛЬ!Не верите?А
вы,посмотрите доказательство сначала!(ФРАНЦИЯ - КОЛЫБЕЛЬ
ХРИСТИАНСТВА!)А насчёт МОСКОВИИ,вся РУСЬ(прежде, чем,
объединиться,воевали все княжества между собой,пока князь ОЛЕГ не
объединил все эти княжества в одно целое и прекратили платить дань
ХАЗАРАМ!А в последствии их уничтожил сын РЮРИКА!Почему процветала
ХОЗАРИЯ?Она процветала не вся,а только - ВЕРХУШКА!Почему?потому,что,это
были кочевой народ,степняки,а верхушка были - ЕВРЕЯМИ!И поэтому,их
называли не разумными!ЕВРЕИ богатели и процветали,а простой
народ,нищенствовал!Изучите настоящую историю РУСИ!И никогда не пишите
того,что,не знаете!Теперь,что,касается ПЕТРА - 1!Пётр ПЕРВЫЙ Хотя и
сделал кое - что для РУСИ,он много сделал и против РОССИИ!Он испохабил
и переврал историю!Он говорил о том,что,РУСИЧИ были безграмотными и
ничего не знали и не умели!А теперь,находятся свидетельства,что,это -
ЛОЖЬ!Даже, дети умели и читать, и писать!Пётр -1 все рукописи,все книги
и карты,по которым РУСИЧИ ходили в моря и океаны он отдал - ГЕРМАНИИ!И
пора найти всё и перевести в РОССИЮ, ибо ГДР - часть (РОССИИ!)Почему
мы, РЕЙХСТАГ взяли с такой быстротой и лихостью?Не потому ли,что,его
наши предки строили - РУСИЧИ?

\item \cusr{Георгий}

УКРАИНА. БЕЛОРУССИЯ. КАЗАХСТАН- ЭТО ЧАСТИ РОССИИ- ОТРЕЗАННЫЕ ЛЕНИНЫМ. РАСЧЛЕНЕНИЕ РОССИИ..... НАМ- СОБИРАТЬ ОБРАТНО.


\item \cusr{Баваров Александр}

Вы не сочиняйте не может дунай,волга,днепр иметь по два истока каждая,как у
каждой реки один исток так и у каждого народа один исток один корень.И так
русские это народ который формировался западная
украина,белоруссия,новгородская,псковскская ленинградская
,вологодская,архангельская области. 862г.это начало обьединенной РУСИ.
Эстонцы,Финны,частично Шведы,Поляки ,жили в русском госсударстве. Многовекавое
противо стаяние с ордой многим русским изменил вектор , а долгое проживание
отдельно и рядом с другими народами изменили образ жизни и менталитет на западе
более культурнее, чистоплотнее трудолюбивее,честнее, мастеровитее , на вастоке
также но грязнее,неопрятнее,ленивее,ХИТРОЖОПЕЕ.

\item \cusr{Кадырберген Шаймердинов}

РУСЬ от слова -УЛУС! КИЕВСКИЙ УЛУС- КИЕВСКИЙ РУСЬ!

\item \cusr{Николай Боголюбов}

является ли автор человеком? мне кажется что и да и нет. вроде по форме оно
человек, но вот по мозгам - мимикрирующий кастрюлеголовый укропитек

\item \cusr{елена матвеева}

Николай, Вы хотите сказать,что русские и украинцы один народ?СОВКОВОЕ
МЫШЛЕНИЕ!!!!В СССР И НЕГРЫ С РУССКИМИ БЫЛИ ТОВАРИЩИ И БРАТЬЯ.В СССР готовы были
с кем угодно русский народ братать,потому в школах и вдалбливали в голову
всякую ересь и глупости.Русские и украинцы народы разные с разной
историей,языком и культурой.Даже кухня разная,национальная одежда,обряды.

\item \cusr{Николай Боголюбов}

елена, согласен на 100 Укры это оподляченные русские

\item \cusr{елена матвеева}

Николай, Укры-это посполитые холопы,мешали их все кто хотел,нет государства владеют и имеют.

\item \cusr{Николай Боголюбов}

елена, к моему величайшему сожалению Вы не правы. имел место быть случай
формирования ярой кастрюлеголовой бандеровки во вполне себе русской дворянской
семье. Можете погуглить Галю Мазуренко. Когда она получала железный крест из
рук самого Петлюры, ее отец был зампред Продкомиссии губернского совнаркома.
Далее отмечен госнаградами и умер уже после войны. Брат ее прошел с боями до
Кенигсберга. Вот Вам и этногенез этих небратьев.

\item \cusr{Ирек Губайдуллин}

Россия-это Московия. Не больше,ни меньше,не хуже,не лучше,это Московия.

Киев был там отдельно,Москва отдельно,так было изначально и всегда,так и сейчас
есть. Еслибы так было не так,то и не было бы так как сейчас.

Отдельные разные земли,отдельные разные княжества,отдельные разные
народы,отдельный разный быт,другие соседи,другой климат.

Это отдельные народы и страны,похожие только языком и больше ничем.

\item \cusr{Николай Сергеев}

Для начала нужно знать хотя бы основы лингвистики. РУССИЯ и РОССИЯ - это два
разных слова. Пройдитесь по старинным картам и посмотрите хронологию - когда
появляются два этих названия. Ну а потом все остальное.

А если Вы еще и в историю поглубже проникнете, то прямо в ОФИЦИАЛЬНЫХ
исторических источниках к удивлению увидите, что КИЕВ - это название небольшого
городка на территории нынешнего Петербурга. А если вдобавок почитаете Указы
Петра 1, то к удивлению своему узнаете, что он приказывал звать людей на
строительство Петербурга в УКРАИНУ. Так и написано.

Учите матчасть для начала.

\item \cusr{Сергей Голицын}

Николай, тебя дурака можно сразу же опровергнуть.просто возьми и забей слово
РОССИЯ в разные языки например в англосаксонский и увидишь слово RUSSIA. А
вообще рОссия это чисто греческое слово .В библии есть такой народ апокалипсиса
РОШ князя Гога и Магога .Дак вот греки не сомневались , что этот народ русские
, тем более РОШ по гречески будет РОС .отсюда и РОССИЯ и РОСЫ.Про это много кто
говорит из греческих летописцев.

\item \cusr{Владимир К}

Николай, Да, хотел, даже построил, но назвал его Азов., потому что шведы
одолевать стали, пошёл на север воевать и там решил город поставить. Но самое
интересное, что название Санкт-Петербург город получил не в честь Петра
первого, а в честь Святого Петра, апостола Христа. Ну видно тут сыграло роль,
что царём тоже Пётр был.

\item \cusr{ДОНСКОВ ДМИТРИЙ}

Россия, наследница Хазарии. Такие же вечные междоусобицы, многонациональность,
многоконфециональность, постоянные поборы с населения и иудеи у власти. Не вижу
совершенно ни какой разницы.

\item \cusr{igor moshkin}

Для безграмотных: с греческого языка перевод: Московия-Руссия.

И находилась Московская Русь и русские князья: Владимир и Ярослав на территории
верхней Волги (Ярославская область) и карту прилагаю!

\ifcmt
  pic https://avatars.mds.yandex.net/get-zen_pictures/3403730/1185376109-1604593752716/orig
  caption И карту прилагаю...
	width 0.2
	fig_env wrapfigure
\fi


\item \cusr{Sychev Vitali}

Никакой-такой Руси, как показано на карте № 2, тем более, в виде сплоченного
государства, не существовало. А существовал некий конгломерат княжеств, в то
или иное время ведомый неким, более сильным княжеством, что иногда называют
Киевской Русью. В 12-13 вв. всё это быстро вступает в инерционную фазу, затем в
фазу обскурации, т.е. приходит в упадок, утрачиваются систем. связи, единство,
захвачено, ополячено-олитовчено и т.д. Однако в северо-восточной части, на
Владимиро-Суздальской земле, в том числе под влиянием Орды, случается новый
этногенез, замешанный на славянах, литовцах, тюрках и финно-уграх, и на сцену
выползает новое образование, получившие название Московия (т.н. Московская
Русь), которое в пассионарной фазе постепенно разрослось до России, а затем и
Росс. империи.

Таким образом, свершенного логично можно прийти к заключению, что та Русь
800-1200 гг. и Московия-Россия 1200-1917 гг. - это в большой степени разные
гос.-полит. образования, связанные между собой лишь совсем тонкой ниточкой.


\item \cusr{Егор Попель}

Sychev Vitali, тогда получается, что "та Русь", которая 800-1200 гг., осталась
без наследников, ибо этногенез современных украинцев или белорусов - это та ещё
каша, что-то ополячено, что-то отуречено...

\item \cusr{Мангупский}

С какого перепугу Московия присвоила себе историю Новгорода? В Новгороде все
были грамотными, берестяные грамотви находят десятками тысяч, была демократия,
это было независимое государство. В Московской орле все были безграмотные
холопы, ни одной грамоты не найдено. Московия Новгород захватила, разорила,
уничтожила население, установила свои тоталитарные порядки, присвоила историю и
культуру.

\item \cusr{Дмитрий Осокин}

)) Привести много интересных фактов, чтоб не сделать никакого вывода :)..

И, да, - в современной России очень даже часто используется слово "Московия" -
касательно жителей столицы :)

Собственно, тут возможны два подхода - исторический - тогда кто там впервые сам
себя назвал "всероссийским"? :)

и генетический - но тут черт ногу сломит

\item \cusr{Ку-Ку Ля-Ля}

То, что язык государственный русский, вам это тоже ни о чем не говорит. Читайте и нюхайте портянки.

\index[writers.rus]{Ку-Ку Ля-Ля}

\begin{multicols}{2}
	\obeycr
Пришли лихие времена,
Но русским не нужна война,
А нужен мир, в душе покой,
Чтоб жить надеждой и мечтой.

Но враг не дремлет у ворот,
А будоражит наш народ -
Страну чтоб в щепки разнести,
Мол, так от бедности спасти.

Но это ложь, другая цель.
Страна Россия - цитадель!
И силой русских не сломить,
В любом бою не победить!

Умеем дать врагу отпор,
Живёт тем Русь уж с древних пор
Любовь и веру сберегла
И неприступна здесь для зла.

А враг хитёр и ищет путь
Как можно русских обмануть
И в ход идёт шантаж и лесть
Чтоб подменить деньгами - честь.

Чтоб насаждать здесь; фальшь, цинизм,
Жестокость, зависть, сатанизм...
И либеральный всякий вздор,
Где лишь на выгоду упор.

А нам другое здесь важней
Страна должна быть для людей!
Чтоб процветали тут всегда -
Деревни, сёла, города!

Чтоб мы ценили вновь и вновь
Заботу, верность и любовь!
Чтобы шумели тополя,
И крепла русская земля!
	\restorecr
\end{multicols}

\item \cusr{Русь триединая}

Россия - это не латинское название, а греко-византийское. Автор почему-то
забывает, что русские княжества находились под влиянием православной греческой
культуры, а не латинской. И восприняли именно греческое название Россия,
которое впервые встречается для Руси ещё в X веке. В Западной Европе Русь
называли по разному - Russia, Rossia (под влиянием Византии), Ruthenia и даже
Roxolania. Московия - это латинский термин эпохи феодальной раздробленности
(Москва, Московское княжество), который поляки и литовцы продвигали даже в
период централизованного Русского государства, чтобы не признавать право
русских государей на те земли Руси, которые оставались под польско-литовским
владычеством. Иначе говоря, отрицали за русскими их подлинный этноним, чтобы
делегитимизировать объединение.

\item \cusr{Равшан Строитель}

Впервые термин "Росия" (да, да, именно с одной "с") заграничных источниках
начали использовать в 950-х годах во времена правления великого князя
Святослава Игоревича. Росией называлась Русь в писаниях Императора Византии
(Константина Багрянородного) — на греческий манер "Ρωσία".

Первое на Руси известное упоминание слова «РОССИЯ» кириллицей — надпись на
последней странице Лествицы Иоанна Синайского, переписанной митрополитом
Киприаном: «В лето 6895 [1387], априлиа 24, съвръшишяся сия книгы в Студийской
обители Киприаном смереннымъ митрополитом Кыевскымъ и всея Росия» (РГБ. Ф.
173/1. № 152. Л. 279 об.

Прошу обратить внимание на названия месяцев в древнерусских летописях.
Современные украинские "народные" месяцы были придуманы одновременно с
изобретением украинского языка в 1857 году украинским юмористом Пантелеймоном
Кулишом.

\item \cusr{Александр Т}

Равшан, а что "да, да именно с одной "С""? Называть-называли, да только
ударение было на первом слоге РОсия, а не как в слове РоссИя. Страна называлась
Рось, а когда купца спрашивали, он и говорил: "Я из РОси или РОсии", а так как
у немцев склонений нет, то они так и называли нашу страну - Росия, из-за
невозможности изобразить правильное двойное звучание (ИИ) латинскими буквами
(скорей всего писали "Rosiy", или как-то так). А вот "русами" нас называли
восточные и азиатские племена.

Вообще, я не знаю, откуда взяли, что "у" появилось с Европы? Но вот как раз в
Европе, из-за похода монголов на них, и появилась буква "U" в написании
названия наших соотечественников, так как монголы именно русами, называли тех
роских воинов, каких они востребовали с князей, для своего авангарда.

\item \cusr{Olli Rud}

Что за чушь! Петр Первый не переименовывал никакое Московское царство в
Российскую империю. Со времен Ивана IV официальное название нашей страны на
всех печатях и документах Роусiя, Русiя или реже Росiя. В титуле великих
князей, царей и императоров "Великий князь Владимирский" стояло перед
"Московский". В 1721 году Петр заменил слово "царство" на "империя". Турцию
тоже в Европе долго на картах писали не Турцией, а Natolia. Византия - тоже
чужое название по столице, сами византийцы были уверены, что живут в Ромее. Да
и современные китайцы свою страну называют Чжунхуа, а не Китай или Чайна.
Жители нашей страны тысячу лет знали, что они русские. Данных о том что они
себя называли московитами нет. Московит - даже окончание не славянское а
латинское.

\item \cusr{Старая Ладога-отец городов русских}

Сколько же умников, писать научились, а мыслить и читать нет. Строчат
всякую хрень про «великих и древних укра»-«ляховских австрияко-угорских
холопов, говорящих на жаргоне»(польское), «изблеванных на Руси»(Г.С.
Сковорода), «оскотинившихся русские»(нем.нац.), «рождённых»
католиками-басурманами в Талергофе в 1914 г., Терезине и Думене.

Государственными языками Византии были латинский и греческий языки. На них
«Русь» звучит как «Россия», «Рутения».

По переписи населения России(Руси) 1897 г. основная нация это «Русские» и
состоит она из 3-х субэтносов:

1. Белоруссы-олитовченные Русские;

2. Малороссы-ополяченные Русские;

3. Великороссы.

Запорожские казаки-«Хохлы» и есть часть «Малороссов», «украмовы» они не знали!

Изначально они были один и есть единый Русский народ Руси(России, Рутении) и
говорил он изначально на едином «СлОвЕнском» языке, «Киевском», «Русинском»,
«Рутенском», «Русьском», «Руснаковском», «Русском».

«Русские»(«Русьские»), («Русины»(австрийско-угорское)-это часть нации
«словене»(«соколы») подданные Рюриковича, который был словенином(соколом) с
Померании Германии как и Новгородцы-ильменские словене.

Вот и все!

Вот зелёная это Русь(Россия).

На финском языке «Русский»-это «kieli», вот и «Киев» нашёлся.

\ifcmt
pic https://avatars.mds.yandex.net/get-zen_pictures/3503722/1037768021-1605277663149/orig
\fi

\item \cusr{condorss}

Все Россия, кроме славянизированных тюрок с Украинских степей.

\item \cusr{Alex Ivan}

Литовские феодалы захватили и подчинили себе западные русские земли, тогда как
в северовосточных землях сохранилась прежняя элита, правящая династия и
государственность. Поэтому Россия и только Россия является даже не наследником
древней Руси, поскольку не было какого то перехода от одного государства к
другому, Россия это и есть Русь. Тогда как Украина и Белоруссия, как
государства являются новообразованием.

\item \cusr{Mr. Rusvector}

Потому и поделили - великая, малая, белая Русь.

Но всему виной националисты, которые перетасовали названия на новый лад,
придумав исторические фальшивки в поддержку своей брехни. Вот и идут
кровопролитные споры из-за названий, а народ-то остался весь из той Руси. Это
как люди из одного города, но из разных районов.

\item \cusr{Петров Владимир}

Все так.И все же Киевская Русь более русская.

\item \cusr{Николай Никешин}

Русь,это земля где жили Русичи или Русские.Не важно каким богам они молились и
как назывались их земли-княжества. Важно кем они себя сами считали.Их объединял
Русский язык. Россия ,это многонациональное содружество народов, и всех их
называют россияне, которые общаются друг с другом на Русском языке.

\item \cusr{Ars S}

Все спецы по истории, любой запросто выложит свою трактовку, особенно, если
хвост накручен политикой. С физикой, химией, математикой, как то посложнее.
Факт непролазного дилетантизма будет очевиден сразу.

\item \cusr{АнКор}

Зачем всё усложнять? Какими-то московиями и украинами-русь?

Всё проще.

Русь пошла от Новгорода и Киева (именно в таком порядке).

Русь по гречески Рѡсїѧ

Впервые в кириллической записи слово “Росия” было использовано в 1387 году в
титуловании митрополита Киприана, собственноручно подписавшегося как
“митрополит Кыевскый и всея Росия”.

Греки использовали такие варианты, как “χώρα τῆς Ῥωσίας” – буквально “страна
Росии” и “γῆ Ῥωσίκή” – “Росская земля”. С 14 века в греческих текстах
появляются также формы с двумя “c”.

Было время когда Русь была раздроблена на множество княжеств и все они были
Русью.

И только Московское смогло объединить почти все княжества под одной властью.

Одним словом, Москва - это Русь, Росия, Россия.

А Русь, Росия, Россия - это в том числе и Москва, и Новгород, и Киев, а
впоследствии множество других городов и земель

\item \cusr{Игорь}

Карта Russia 1595 г на Латыни, не может современный переводчик перевести о чём
тут написано , один из индоевропейских языков италийской группы, на котором –
приблизительно с 6 в. до н.э. по 6 в. н.э. – говорили древние римляне и который
был официальным языком Римской империи.

\ifcmt
pic https://avatars.mds.yandex.net/get-zen_pictures/3403730/1008174654-1606845020866/orig
\fi

\item \cusr{Владимир Скептик}

Начну с конца.

Русь - от арабского руус - головы, головные, главные.

-

Москва так же образована от арабского корня маск/моск, что означает выплавка
стали. Ва - это буква в арабском языке означающая союз.

В русском языке соответствует запятой (стоит за пятой буквой в арабском
алфавите).

-

Московия образовалась от Руси Рюриковичей, которая распалась во время Великой
Смуты. Оставив такие крупные осколки как Тартария.

Московия - это Русь Романовых, которая скукожилась до этих размеров.

(см. рис. ниже)

Это хорошо изображено на этой карте. Границы романовской Руси закрашены.

-

Восточнее этой Руси - осколки Руси Рюриковичей. Великая Тартария.

\ifcmt
pic https://avatars.mds.yandex.net/get-zen_pictures/3503722/1040316633-1606656233172/orig
\fi

\item \cusr{Саша Бу}

С Урала и Сибири вид один:сплошная Украина-окраина

\item \cusr{Раис}

Выбрали термин Россия и отсюда появились русские. Появилось Украина и формируется новый народ украинцы.

\item \cusr{Вадим Бонд}

Когда существовала Русь, России не существовало. И существовала Русь там, где
сегодня Украина. Заявление "украинский этнос сформировался в Руси" - верх
необразованности.

\item \cusr{mk moskvich}

украинцы это вообще не славяне,.. Первые украинцы были касогами. Косог или
козак, это одно и тоже слово...

В. Гатцук, украинский исследователь начала XXв.писал: «Города Канев, Черкасы,
Чигирин основаны и населены Касогами, и все сельское, хуторское, поселение по
обоим берегам Днепра к югу от Киева состояло тогда из Касогов. Касоги-Черкесы
слились с местным тюрско-славянским населением и образовали, вместе с ними, так
называемое украинское племя. От них теперешние украинцы получили те свои
особенности, которыя отличают чистый их тип от Великороссиян — темный цвет
волос и глаз. Территория и народ Украины назывались Черкасией вплоть до времен
Екатерины II.

♦Степан Львович Рудницкий, Украинский географ, картограф, публицист, академик
Академии Наук Украинской ССР. Родился: 3 декабря 1877 г., Пшемысль, Королевство
Галиции, Австро-Венгрия:

«Антропологический тип украинцев выражает полную самостоятельность против
польского, белорусского и великоруского типа», — констатирует С. Рудницкий.
«Поляк, белорус и россиянин, антропологически стоят очень близко друг к другу;
украинец, в свою очередь, очень отличается от всех своих соседей и, с
антропологической точки зрения, занимает полностью самостоятельное место.

\item \cusr{Александр Серышев}

Московия - часть Киевской Руси со столицей в Москве . Южные -Киевские земли это
колония Польши .

\item \cusr{Борис Новицкий}

Слово "Русь" не шведское, как утверждает автор. За тысячу лет до появления такого наименования, как "шведы", имя "русы" уже поминалось в исторических документах.

VI век. Сирийский автор Псевдозахарий упоминает народ рос в Причерноморье.

VI век. Историк начала XI века ас-Са'алиби в рассказе о постройке Хосровом I (531 – 579) Дербентской стены называет наряду с турками и хазарами русов.

VI век. Прикаспийский автор XV века Захир-ад-дин Мар'аши упоминает русов в районе Северного Кавказа.

626 год. Византийский поэт Константин Манасси (XII в.) называет русских в числе осаждавших вместе с аварами Константинополь.

643 год. Арабский автор ат-Табари (838 – 923) дважды называет русов как врагов мира, в особенности арабов.

765 (или 773) год. Византийский хронист Феофано (ум. 817) упоминает русские хеландии (корабли). Норманисты греческое «та роусиа» читают как «красные».

773 – 774 годы. Во французской поэме об Ожье Датчанине (XII – XIII вв.) упоминается русский граф Эрно, возглавлявший русский отряд, защищавший Павию – столицу лангобардов – от войска Карла Великого. В Северной Италии русы занимали район Гарды близ Вероны (скандинавы «Гардами» называли Восточную Русь).

И ТАК ДАЛЕЕ.

(Из А.Г. Кузьмина. Начало Руси).

\item \cusr{Александр Дмитриев}

Русские в ещё самом первом упоминании, ещё задолго до кн. Владимира, это
суперэтнос. То есть это не узконационадбное объединение по критерию крови или
религии и племя. А Русь (Россия) место где мы жили, да и живем до сих пор,
несмотря на разные перевороты и революции, где обожали все это переписывать во
имя политических интересов....

\item \cusr{ДЯДЯ ВАСЯ}

Пока видно, что - как страну назовешь, так она и заживет.

- Укра! Естественное желание что нибудь стащить, например историю.

- У краи на! Стоя у края собственной истории обвинять в этом кого угодно, но
только не себя...

\item \cusr{Игорь. Фомин}

Единственный вывод из всей этой исторической белиберды-история Российской империи началась со смуты, смутой и кончится.

\item \cusr{Малюта Скуратов}

Европа и БУФЕРНАЯ МАРИОНЕТОЧНАЯ РОМАНОВСКАЯ МОСКОВИЯ это два огромных
сепаратистских отколовшихся куска , в результате Великой Смуты в 1605 году от
ВСЕМИРНОЙ РУССКОЙ АРИЙСКОЙ ИМПЕРИИ- ВЕЛИКОЙ ТАРТАРИИ( МОНГОЛО-ТАТАРИИ) , нашего
СССР древнего мира..

\item \cusr{Антуан Свиньин-Оноре}

"Поскольку Украина, как независимое государство, юридически никогда не
существовала на карте Европы,

она просто неожиданно всплыла благодаря незаконному перевороту в Москве и
распаду Советского Союза в 1991 году.

Всплыла, как всплывают нечистоты.

Поэтому её исторически законные соседи воспринимают Украину именно, как
зловонные нечистоты." Тема закрыта.

В следующей статье мы вам расскажем о древнем Шумерском царстве, построенном в
IV-III тыс. до н. э украинцами, назло москалям.

\item \cusr{Васильева Валентина}

КУев тоже Россия. Пора его вернуть домой, а хохлов сослать в их Львов.

\item \cusr{Владимир Макаров}

\index[writers.rus]{Макаров, Владимир!К восточным славянам}

К восточным славянам

Как богатырская застава
Стояла Киевская Русь.
И Византия трепетала
Когда гулял Олег-дедусь.
*
И славу предка не роняли
Святой Владимир, Святослав.
И в них соседи уважали
Лихую удаль, добрый нрав.
*
А принцы крови позабыли,
Что надо старших уважать.
Не стали витязей дружины
Друг другу в битвах помогать.
*
Поодиночке их разбили
И на Днепре, и под Москвой.
И Русь святую поделили
Враги между собой.
*
И долгие четыре века
Славяне были в кабале.
Объединившись, били шведа,
Французов на Березине.
*
Фашизм на Волге раздавили -
Европу от чумы спасли,
И государство возродили
Мощнее Киевской Руси.

*

И мир не мог с ним не считаться,
Соседи за защитой шли.
С ним мало кто мог потягаться,
Славяне вместе - так сильны!
*
Заветы предков позабыли
Амбициозные вожди
И Государство разделили-
Врагов исполнились мечты.
*
Славяне, мы по крови братья!
Мы все из Киевской Руси.
В отдельности нам нету Счастья!
И только вместе мы сильны!
\url{https://fabulae.ru/poems_b.php?id=62024}
*
\url{http://fabulae.ru/autors_b.php?id=1561}

\item \cusr{Васильева Валентина}

Ни кто и никогда Киевскую Русь РУСЬЮ не называл. Киевскую Русь придумали
историки. что бы отделить эпоху перехода от центра правления в Киеве в центр
правления Москва. Когда татары захватили Киев, русские построили новую столицу
Москва. Так, что не надо нам тут родство с хохлами приписывать. Хохлоублюдки -
это туда, в Польшу и прочие недостраны.

\item \cusr{Николай Логинов}

Московия - это, изначально, ТАЙный анклав Византии (Bis Antique, т.е. 2-й Рим)
на Руси. Его целью было тайное распространение жидовской ереси - ХРЕСТианства.
Поэтому, на его месте изначально был вырыт ТАЙный - Се-ТАЙ- город (который
романовские историки переименовали в нынешний Китай-город), в котором прятались
привезённые Д.Донским из Ватикана, так наз. "вольные каменщики" - мазуны (лат.
masoni).

\end{itemize}
