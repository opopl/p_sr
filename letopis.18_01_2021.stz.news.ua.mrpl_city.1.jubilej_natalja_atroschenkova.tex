% vim: keymap=russian-jcukenwin
%%beginhead 
 
%%file 18_01_2021.stz.news.ua.mrpl_city.1.jubilej_natalja_atroschenkova
%%parent 18_01_2021
 
%%url https://mrpl.city/blogs/view/do-yuvileyu-natali-atroshhenkovoi-1
 
%%author_id demidko_olga.mariupol,news.ua.mrpl_city
%%date 
 
%%tags 
%%title До ювілею Наталі Атрощенкової
 
%%endhead 
 
\subsection{До ювілею Наталі Атрощенкової}
\label{sec:18_01_2021.stz.news.ua.mrpl_city.1.jubilej_natalja_atroschenkova}
 
\Purl{https://mrpl.city/blogs/view/do-yuvileyu-natali-atroshhenkovoi-1}
\ifcmt
 author_begin
   author_id demidko_olga.mariupol,news.ua.mrpl_city
 author_end
\fi

\ii{18_01_2021.stz.news.ua.mrpl_city.1.jubilej_natalja_atroschenkova.pic.1}

17 січня відзначила своє 70-річчя провідний майстер сцени Донецького
академічного обласного драматичного театр (м. Маріуполь) \emph{\textbf{Наталя Норисівна
Атрощенкова}}. Завдяки чудовим сценічним даним, високій працездатності та
професійній майстерності актриса займає помітне місце у трупі. Віддана театру,
енергійна та темпераментна, вона створила цілу галерею образів різних за своїм
характером, але однаково змістовних та цікавих. Наталя Норисівна закінчила
театральну студію при Київському державному театрі оперети за фахом \enquote{артистка
музкомедії та драми} (1976 рік). У Донецькому академічному обласному
драматичному театрі (м. Маріуполь) працює з 1977 року. Володіючи музичним
слухом і неабиякими вокальними даними, для театру актриса стала справжньою
знахідкою.

Творчі роботи Атрощенкової Наталі відзначені різними дипломами Міністерства
культури СРСР, ВТО, нагородами від НСТДУ, Почесними грамотами Донецької
облдержадміністрації та Управління культури і туризму. У 1979 році за роль
Секретаря у виставі \enquote{Перука із Гонконгу} К. Саконі на фестивалі драматичного
мистецтва Угорської Народної Республіки в Радянському Союзі (м. Москва) актрису
нагороджено Дипломом Міністерства культури СРСР та Всеросійського театрального
товариства.

Режисерка-постановниця театру \emph{\textbf{Добрунова Анжеліка Арганівна}} зазначила, що
\enquote{Наталя Атрощенкова – актриса – поза амплуа і віку, універсальна, стильна,
буває, несхожа на саму себе, але завжди запам'ятовується, адже володіє
неймовірним магнетизмом і безсумнівною чарівністю!}. Сценічні образи, створені
Наталею Норисівною відрізняють яскравість форми і емоційна переконли\hyp{}вість.
Найкраще вдаються актрисі характерні та гострохарактерні ролі. Серед найбільш
яскравих ролей можна виділити наступні: Донья Серафіна (\enquote{Раба свого коханого}
Л. де Вегі), Жозефіна (\enquote{Жозефіна та Наполеон} І. Губача), Фатіма (\enquote{Останнє
кохання Насреддина} В. Костянтинова, Б. Рацера), Шарлотта Гейз (\enquote{Лоліта} за
В. Набоковим та Е. Олбі), Фаіна (\enquote{Корабель кохання} Н. Коляди), Софія Сергіївна
(\enquote{Невільниці} О. Островського), Марлен Дітріх (\enquote{Таємниця Дітріх} К. А. Інднесс),
Олександра Колонтай (\enquote{Чорне та червоне, або Маріупольський скарб Нестора Махна}
В. Сухорукова), Єлизавета ІІ (\enquote{З привітом, твої таргани!} О. Богаєва), Бабуся
(\enquote{Вісім люблячих жінок} Р. Тома) тощо.

\ii{18_01_2021.stz.news.ua.mrpl_city.1.jubilej_natalja_atroschenkova.pic.2}

Проте найособливіше місце в творчій біографії Наталії Атрощенкової зайняла роль
у виставі \enquote{Таємниця Дітріх} К. А. Інднесс, в якій вона блискуче зіграла
велику кіноактрису Марлен Дітріх. Спектакль і зокрема робота Наталії Норисівни
із захватом були сприйняті глядачем і по праву могли називатися бенефісними.

За час роботи в театрі Наталія Норисівна завжди брала участь у суспільному
житті колективу. З 2000 по 2004 роки Атрощенкова Н. Н. вона була членом профкому
театру, з 2003 р. – членом художньої ради театру. На посаді голови первинного
осередку НСТДУ з 2006 р.

Цікаво, що актриса є чудовим модельєром  одягу і дизайнером: у неї гарний смак,
фантазія і до того ж золоті руки. Репетитор з техніки мови \textbf{Наталя Гончарова}
поділилася, що вперше побачила Наталю Норисівну, коли була ученицею молодших
класів. Її завжди захоплював в ній неймовірний альянс сили і жіночності. \enquote{Її
статура, її голос щоразу притягують до неї, наче магнітом. Віднедавна мені б
дуже хотілося побачити її в ролі Елеонори Аквітанської (\enquote{Лев взимку}). Мені
здається, роль незламної королеви повною мірою її!}. Гончарова наголошує, що
Атрощенкова має приголомшливе почуття стилю. Це жінка, яка надихає і приклад
якої хочеться наслідувати.

%\ii{18_01_2021.stz.news.ua.mrpl_city.1.jubilej_natalja_atroschenkova.pic.3}

\ifcmt
  tab_begin cols=2,no_fig,center,separate,no_numbering

  pic https://mrpl.city/uploads/posts/redactor/t2fxqdeisoeqcqte.jpg
  pic https://mrpl.city/uploads/posts/redactor/3wiy4gmgxtrjjs6a.jpg

  tab_end
\fi

Актрису із задоволенням у свої вистави запрошує \emph{\textbf{Анжеліка Добрунова}}: 

\begin{quote}
\em\enquote{з Наталею
Норисівною  мені довелося працювати багато. Вона завжди дуже відповідально
ставиться до ролей. Якщо довірилася вже режисерові, то робить все, щоб виникла
співтворчість. Багато придумує, пропонує, небайдужа до роботи партнерів,
пропонує цікаві рішення і їм. Завжди коректна, толерантна, і завжди в
приголомшливій формі. Цікаво відзначити, що ця красива та чарівна жінка, не
боїться, і, навіть із задоволенням, йде на всякого роду зміни своєї
зовнішності, в характерних ролях. Володіючи прекрасним смаком, часто бере
участь у створенні або покращенні костюмів. І найголовніша, мабуть, якість її
це те, що вона за будь-яких обставин – на сцені, за лаштунками, в спілкуванні –
завжди відповідає цьому поняттю АКТРИСА. Не грає в житті, а несе в собі
мистецтво! Думаю, нам пощастило, що ми маємо можливість з нею працювати!}.
\end{quote}

Колеги підкреслюють, що Наталя Норисівна має вроджену аристократичність,
почуття етики по відношенню до інших.

Акторська чуйність, ретельне розкриття сценічних характерів, висока
працездатність і дисциплінованість жінки – завжди вражали театральний колектив.
І що найцінніше, Наталя Атрощенкова не втратила цих якостей і сьогодні.
Маріупольці знають, люблять і цінують творчість Наталі Норисівни. Вітаю
талановиту актрису з ювілеєм та сподіваюся, що глядачі після локдауну матимуть
можливість бачити її у нових виставах.

%\ii{18_01_2021.stz.news.ua.mrpl_city.1.jubilej_natalja_atroschenkova.pic.4}

