% vim: keymap=russian-jcukenwin
%%beginhead 
 
%%file 23_08_2021.fb.bljuminov_aleksej.1.nezalezhnist_30_let_dela
%%parent 23_08_2021
 
%%url https://www.facebook.com/albluminov/posts/4408516149205542
 
%%author_id bljuminov_aleksej
%%date 
 
%%tags 30_let,nezalezhnist,ukraina
%%title С тридцатилетием славной незалежности такие странные дела творятся
 
%%endhead 
 
\subsection{С тридцатилетием славной незалежности такие странные дела творятся}
\label{sec:23_08_2021.fb.bljuminov_aleksej.1.nezalezhnist_30_let_dela}
 
\Purl{https://www.facebook.com/albluminov/posts/4408516149205542}
\ifcmt
 author_begin
   author_id bljuminov_aleksej
 author_end
\fi

С тридцатилетием славной незалежности такие странные дела творятся, что хоть
бери и снимай новый сезон одноименного сериала. То значит мегафлагшток "впав",
то  вот в Одессе Терри Гиллиам неожиданно вызвал когнитивный диссонанс у
рогулей собравшихся на местечковую ярмарку тщеславия, "пидступно" напомнив им,
кто они и откуда.

 Всего одна фраза Гиллиама: «Спасибо, Россия! Спасибо, Одесса!» сыграла роль
мальчика из сказки Андерсена про голого короля. Потому что как ни натягивай на
Одессу глобус Украины, сколько не кричи дурным голосом "I am not Russia", а
миру Одесса интересна исключительно в разрезе русской и советской культуры, вне
которой никакой Одессы просто не существует, о чем гость из Гамерики и сказал
хуторской общественности, вспомнив про Эйзенштейна и "Броненосец Потемкин". 

Смешно, но у части рефлексирующих рагулей - свой вывод из истории с Гиллиамом.
Мол, это прокол не нынешней культурной политики в целом, а исключительно дятлов
пропаганды. А так -то нужно побольше и почаще "пропагандировать миру украинскую
Одессу" и будет всем щастье, а ворогам - смерть. 

Смешно, потому что "украинская Одесса" - это фикция, миф для внутреннего
потребления, для сельских дурачков. Как, собственнно, и все это нынешнее
"украинское кино", с выдернутым из него позвоночником в виде кино советского. 

Остаются носороги-сенцовы, какие то ваятели саг про "киборгов" и прочих
"титановых джексонов", и сельская суперстар Ольга Сумская - вершина творческой
карьеры которой - роль наложницы турецкого султана в снятом в середине
девяностых убогом малобюджетном "мыле", о котором после "Великолепного века"
вообще вспоминать стыдно. 

Впрочем, не стыдно, вспоминают же. Но - что показательно - как ни высасывай из
пальца, а чтобы хоть как-то зацепить  читательский интерес, в заголовки выносят
все ту же убогую "Роксолану", потому что больше выносить нечего. Не цепляет.
Оставляя у читателя единственный вопрос: а кто это, собственно, такая?

Положа руку на сердце, даже русская Одесса малоинтересна миру за пределами
нишевого ностальгического сантимента потомков местных еврейских эмигрантов, а
одной из культурных столиц она была исключительно в контексте Российской
империи и СССР, причем, даже не всей их истории, а лишь конкретного временного
ее отрезка. 

Но цимес в том, что именно этот отрезок истории нынешние рогули, с их
декоммунизациями и деоккупациями, пытаются всячески заретушировать, замазать,
стереть и делать вид, что его вообще не было. 

То есть, они уничтожают ту единственную фишку которую можно продать на мировом
культурном рынке в качестве местного айдентити. Дюк в вышиванке не продается.

И как ни пыжится современная Проня Прокоповна, а  никак не удается сойти за
городскую. И даже школа танцев Соломона Пляра не помогает. 

\obeycr
Тетя Соня, не вертите задом, 
это не пропеллер, а вы не самолет..
Две шаги налево, Две шаги направо,
Шаг впирод наабарот.
\restorecr
