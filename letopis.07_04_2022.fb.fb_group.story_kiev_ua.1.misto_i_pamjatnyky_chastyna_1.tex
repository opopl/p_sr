% vim: keymap=russian-jcukenwin
%%beginhead 
 
%%file 07_04_2022.fb.fb_group.story_kiev_ua.1.misto_i_pamjatnyky_chastyna_1
%%parent 07_04_2022
 
%%url https://www.facebook.com/groups/story.kiev.ua/posts/1897762543753871
 
%%author_id fb_group.story_kiev_ua,grabar_sergij.kiev
%%date 
 
%%tags 
%%title Місто і пам'ятники. Частина І
 
%%endhead 
 
\subsection{Місто і пам'ятники. Частина І}
\label{sec:07_04_2022.fb.fb_group.story_kiev_ua.1.misto_i_pamjatnyky_chastyna_1}
 
\Purl{https://www.facebook.com/groups/story.kiev.ua/posts/1897762543753871}
\ifcmt
 author_begin
   author_id fb_group.story_kiev_ua,grabar_sergij.kiev
 author_end
\fi

Місто і пам'ятники

Частина І

Київ – місто мого натхнення, міцність, зібрана у пружний кулак. Кожен день
цього смертельного бою за наше з тобою існування підтверджує усталену тезу –
немає жодної сили проти Вічності Добра. Місто-спалах у анналах історії,
відтворювач світла, одвічний вартовий пам'яті.

Пам'яті застиглої у пам'ятниках твоїх. Прийшов час, і вони попросили твоєї
допомоги, твого захисту, багатолітні сторожові, що своїм існуванням підтримують
спадковість історії, незриму могутність чар твоїх. Охорона - це відбувається
сьогодні на наших з тобою очах, нашими з тобою руками, незборимий Києве.

\ifcmt
  ig https://scontent-mxp1-1.xx.fbcdn.net/v/t39.30808-6/277820124_4976760132442635_5761414692457504366_n.jpg?_nc_cat=111&ccb=1-5&_nc_sid=5cd70e&_nc_ohc=4hlQozV8u2MAX_qseMC&_nc_ht=scontent-mxp1-1.xx&oh=00_AT_7IWS2N1QR9rAnIQ27hzyizIdCj63HLoL9PlgwQyobbw&oe=62654502
  @wrap center
  @width 0.8
\fi

Пам'ятники... У самому вже слові закладено відзвук існування міста, що роками
вшановує згадку про події та людей, які є Оберегами слави його.

Вони різні, зведені у різні часи твого життя: від найстаршого, строгого --
тосканської колони повернення Магдебурзького права, встановленої у 1802-1803
роках архітектором Андрієм Меленським на березі сивочолого Дніпра, до одного з
найновіших – пам'ятника Олександру Вертинському, роботи скульптора Бориса
Довганя, відкритому в 2019 році на Андріївському узвозі.

У Києві пам'ятників багато. За моїми підрахунками – 364 монументи, пам'ятники,
пам'ятні знаки, стели, погруддя. Це без тих, що знаходяться на Байковому та
інших кладовищах міста. Які з них збережуться у віках? Це питання історичного
майбутнього, пов'язане з багатьма складовими. Доля того чи іншого пам'ятника,
його тяглість у часі залежить від різних чинників: влади, матеріалу, художньої
та історичної цінності.

Звичайно, у цьому дописі я не буду говорити про всі пам'ятники, бо є такі, що
вже стали Оберегами міста, а є тимчасові, прохідні. Я спробую розповісти про
рідні мені, киянину, про невіддільні від рельєфу і життя Києва.

У кожного з них своя біографія, бо у вирі десятиліть з'являлися й такі, що
прилітали, мов нічні метелики, і зникали навічно – так, що сьогодні й згадки
серед жителів Києва про них немає.

Щось подібне, наприклад сталося з пам'ятником Карлу Марксу, встановленому на
Думській площі на місці, скинутого 1917 року, пам'ятника Петру Столипіну (1913
р., скульптор Етторе Ксіменес). Взагалі, пам'ятників Карлу Марксу на цьому
місці було два. Перший, роботи скульптора Йосипа Чайкова, встановили 20 лютого
1919 року. Це було виготовлене з гіпсу погруддя на дерев'яному постаменті. 31
серпня того ж 1919 року і у перший  день приходу нової влади до Києва, він був
зруйнований Денікінськими військами. Другий – того ж самого скульптора,
відкрили 1 травня 1922 року. Він теж був гіпсовим, хоча вже пофарбований під
бронзу. Постамент цього разу зробили з асфальту і збиралися до кінця травня
фігуру одного з вождів світового пролетаріату змінити на бронзову. Пам'ятник
простояв до 1933 року, а потім одної ночі його таємно знесли. Так що з Карлом
Марксом у Києві на центральній площі та і взагалі не склалося.

Думська площа, що потім мала кілька різних назв, простояла без монументів
досить довго. Вже після Другої світової війни, по відновленню Хрещатика у 70-х
роках ХХ століття на площі з'являється великий монумент Жовтневої революції
(1977р., скульптори В. Бородай, І. Зноба, В. Зноба). Величний,
червоно-гранітний В. Ленін з чотирма бронзовими представниками народу простояли
до 1991 року і були демонтовані.

За десять років, на сьогоднішньому Майдані Незалежності було встановлено
Монумент Незалежності (2001 р., скульптор А. Кущ) – тріумфальну колону,
облицьовану італійським мармуром, розміщену на постаменті у вигляді у вигляді
християнського храму та увінчану фігурою Жінки-Берегині з калиновою гілкою у
руках. За досить невеликий час відновлення нашої державності цей монумент
побачив чимало героїчних та трагічних подій і поза сумнівом став справжнім
символом і самого Києва, і всієї нашої Держави. Він залишається Оберегом і
сьогодні – в дні жорстокої російської навали, приходу нелюдів у людській
подобі.

Одним із символів стійкості і відродження міста Києва є для мене Пам'ятник
княгині Ользі – споруджений, зруйнований і відновлений. Історія цього
пам'ятника цікава багатьма подіями навколо нього.

Повна назва монумента: «Пам'ятник княгині Ользі, святому апостолу Андрію
Первозванному та просвітителям Кирилу і Мефодію». На сьогоднішній день – це
чотири мармурові фігури на гранітному постаменті. Історія спорудження і у
художньому, і в історичному плані визначного витвору мистецтва має свої певні
передумови. Первісно, на площі перед міською реальною школою збиралися
встановити пам'ятник Тарасу Шевченку. Були зібрані кошти на його спорудження,
але втрутився очільник київської шкільної округи, на думку якого місце перед
російським учбовим закладом не відповідало ідеологічній концепції. Також
долучилося Військово-історичне товариство, яке, як компроміс, запропонувало
ідею створення цілої низки пам'ятників від Михайлівського Золотоверхого
монастиря до Софії Київської під назвою «Історичний шлях», де фігурували би
відтворені у бетоні: Кий, Щек, Хорив, князі Олег, Ігор, Ольга, Феофан
Прокопович та інші. Всього двадцять пам'ятників.

Було прийняте рішення розпочати алею з монумента княгині Ользі. У об'явленому
конкурсі взяло участь кілька майстрів, в тому числі керівник скульптурних
майстерень Київського художнього училища Федір Балавенський та його учень Іван
Кавалерідзе. Переміг у конкурсі Іван Кавалерідзе. Саме його проект в подальшому
і було втілено у життя. Молодий скульптор став автором фігур княгині Ольги,
просвітників Кирила та Мефодія. Статую Андрія Первозванного створив скульптор
Петро Сниткін. У втіленні у життя всієї композиції взяв участь архітектор
В'ячеслав Риков. Все відбувалося під безпосереднім доглядом та при участі
Федора Балавенського.

Пам'ятник відкрили 4(17) вересня 1911 року і він простояв до 1919 року, коли
фігуру княгині Ольги, розпиливши на окремі частини, зняли і на її місці
встановили погруддя Тараса Шевченка роботи скульптора Бернарда Кратка. Кирила з
Мефодієм та апостола Андрія зашалювали фанерними листами. 1920 року бюст Т.
Шевченка зняли, а 1923 року демонтували й інші фігури.

Десь наприкінці 80-х років минулого століття доля звела мене з Миколою
Петровичем Візирем визначним археографом, організатором досліджень архівної та
рукописної справи. На той час Микола Петрович був завідувачем відділу рукописів
ЦНБ АН УРСР. Він розповів мені, що існує лист робітників, які демонтували
пам'ятник, з описом, де які частини фігури княгині Ольги закопані. Знав про цей
лист і видатний археолог, дослідник Києва Віктор Харламов.

Коли 1995 року було прийняте рішення про відновлення пам'ятника княгині Ользі,
Архітектурно-археологічна експедиція під керівництвом Віктора Харламова
розпочала пошуки частин пам'ятника. Вдалий археолог, на чиєму рахунку були
просто колосальні знахідки останків Юрія Долгорукого у церкві Спаса на
Берестові, винайдення та атрибуція поховань митрополита Петра Могили, Євпраксії
та інших в Успенському соборі, розкопки кам'яної Трапезної ХІ століття на
території Києво-Печерської лаври, дослідження Житнього торгу на Подолі,
Кловського та Федорівського соборів, Віктор був абсолютно впевнений у знайденні
всіх частин фігури Великої княгині. Пам'ятаю, він  числа 11 лютого 1996 року
зателефонував мені і сказав: «Сергію, я вже точно знаю, де знаходиться голова
від скульптури. От тільки сніг зійде і ми її дістанемо». Не судилося, за два
дні у нього відірвався тромб і він раптово помер... Тож і до сьогодні на
Андріївському узвозі, трохи нижче церкви, біля Музею Івана Кавалерідзе стоїть
дивна бетонна скульптура без голови – той давній монумент роботи славетного
майстра.

Пам'ятник-композицію у мармурі відновили у травні 1996 року.

Складна доля Києва у його символах – старих, і нових, відновлених, як у випадку
з княгинею Ольгою. Складна доля міста у його людях і пам'яті незгасимій.

Неподалік від пам'ятника княгині Ользі, поблизу стіни Михайлівського
Золотоверхого монастиря 12 вересня 1993 року було відкрито невеличку стелу із
сірого граніту з Хрестом всередині, у якому бронзове відтворення Богородиці з
дитиною на грудях. Це – Пам'ятний знак жертвам Голодомору 1932-1933 років в
Україні (скульптор - Василь Перевальський, архітектор – Микола Кислий). Як на
мене, цей невеличкий, але надзвичайно щемливий пам'ятник є найкращим з усіх,
присвячених тим страшним подіям, подіям, що і тепер, в дні нашого з вами життя,
мають своє страшне продовження. Матір-Берегиня зі свого невеличкого гранітного
постаменту намагається захистити кожного з нас, наше місто від дикої навали
нелюдів.

Пам'ятники Києва – це не просто історія, це наші Обереги, це – могутня пам'ять
Вічного Міста.

\ii{07_04_2022.fb.fb_group.story_kiev_ua.1.misto_i_pamjatnyky_chastyna_1.cmt}
