% vim: keymap=russian-jcukenwin
%%beginhead 
 
%%file 21_02_2023.stz.news.ua.donbas24.3.ljudy_zdyvuvaly_za_rik_vijny_dobrym_sercem.2.minin
%%parent 21_02_2023.stz.news.ua.donbas24.3.ljudy_zdyvuvaly_za_rik_vijny_dobrym_sercem
 
%%url 
 
%%author_id 
%%date 
 
%%tags 
%%title 
 
%%endhead 

\subsubsection{Денис Мінін — витягував людей з Маріуполя}

% denys minin
\ii{21_02_2023.stz.news.ua.donbas24.3.ljudy_zdyvuvaly_za_rik_vijny_dobrym_sercem.pic.2}

\textbf{Денис Мінін} — відомий в Маріуполі та Україні телеведучий. Коли почалося
повномасштабне вторгнення, він прагнув врятувати з обложеного міста свою
родину. Він виїхав у Запоріжжя та почав шукати способи порятунку батьків.
Згодом його ініціатива виросла у \href{https://mrpl.city/news/view/denis-minin-o-slozhnom-no-vazhnom-puti-e-vakuatsii-mariupoltsev-iz-osazhdennogo-goroda}{\emph{великий волонтерський проєкт}}.%
\footnote{Денис Минин о сложном, но важном пути эвакуации мариупольцев из осажденного города, Марина Баліоз, mrpl.city, 07.04.2022, \par\url{https://mrpl.city/news/view/denis-minin-o-slozhnom-no-vazhnom-puti-e-vakuatsii-mariupoltsev-iz-osazhdennogo-goroda}}

Перша експедиція, яку йому вдалося здійснити, сталася в середині березня 2022
року. Відтоді волонтери скоординувалися та, ризикуючи життям, вивозили
маріупольців. Їм доводилося терпіти знущання окупантів, перебувати під
обстрілами, дехто потрапив у полон.

Вже в середині червня завдяки ініціативі Дениса Мініна було врятовано тисячі
маріупольців, зокрема й 200 дітей різного віку. Також у Запоріжжі він
організував гуманітарний штаб для допомоги переселенцям. У жовтні 2022 року у
це приміщення влучила російська ракета.

\textbf{Читайте також:} \href{https://donbas24.news/news/nacgvardijec-evakuyuvav-ponad-500-voyiniv-z-garyacix-tocok-donbasu}{\emph{Нацгвардієць евакуював понад 500 воїнів з \enquote{гарячих} точок Донбасу}}%
\footnote{Нацгвардієць евакуював понад 500 воїнів з \enquote{гарячих} точок Донбасу, Наталія Сорокіна, donbas24.news, 06.07.2022, \par%
\url{https://donbas24.news/news/nacgvardijec-evakuyuvav-ponad-500-voyiniv-z-garyacix-tocok-donbasu}%
}
