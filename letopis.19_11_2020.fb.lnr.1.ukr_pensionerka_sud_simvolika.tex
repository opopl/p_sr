% vim: keymap=russian-jcukenwin
%%beginhead 
 
%%file 19_11_2020.fb.lnr.1.ukr_pensionerka_sud_simvolika
%%parent 19_11_2020
 
%%url https://www.facebook.com/groups/LNRGUMO/permalink/3213302452114640/
%%author 
%%author_id 
%%tags 
%%title 
 
%%endhead 

\subsection{Украинскую пенсионерку осудили за фото Брежнева и советский флаг в \enquote{Одноклассниках}}
\label{sec:19_11_2020.fb.lnr.1.ukr_pensionerka_sud_simvolika}
\Purl{https://www.facebook.com/groups/LNRGUMO/permalink/3213302452114640/}

\ifcmt
pic https://external-waw1-1.xx.fbcdn.net/safe_image.php?d=AQAYaJM-Aj6HTzgy&w=500&h=261&url=https%3A%2F%2Fnovorosinform.org%2Fwp-content%2Fuploads%2F2020%2F11%2FRIAN_archive_734809_Members_of_Moscows_Soviets_Communist_and_civic_organisations_attend_International_Womens_Day_meeting.jpg&cfs=1&ext=jpg&_nc_cb=1&_nc_hash=AQAdys8-wqB2Z-KC
\fi

Украинскую пенсионерку осудили за фото Брежнева и советский флаг в \enquote{Одноклассниках}.

Женщину приговорили к 5 годам тюрьмы с заменой наказания на испытательный срок 1 год.

Районный суд Херсонской области признал пенсионерку виновной в распространении
коммунистической символики, пропаганде коммунистического режима и приговорил её
к пяти годам лишения свободы с заменой на испытательный срок в один год. Об
этом в своём телеграм-канале сообщил бывший депутат Верховной рады Алексей
Журавко.

Женщина на своей странице в «Одноклассниках» публиковала фото и видеофайлы с
изображениями символики коммунистического режима.

Так, например, 9 ноября 2016 года на «стене» соцсети женщина разместила коллаж
из фотографий советского государственного и партийного деятеля Леонида
Брежнева.

Сообщается, что пенсионерка полностью признала вину и раскаялась.  Напомним, 8
сентября в центре Львова полиция задержала 21-летнего парня, который шёл по
улице в футболке с коммунистической символикой. 

На ней были нарисованы серп и молот, а также имелась надпись «СССР». 

Согласно статье 436-1 Уголовного кодекса Украины, ему грозит до пяти лет
тюрьмы.

При этом в сентябре 6-й апелляционный суд Киева пришёл к выводу, что нацистская
символика СС «Галичина» не подпадает под запрет символики коммунистических и
национал-социалистических режимов.
