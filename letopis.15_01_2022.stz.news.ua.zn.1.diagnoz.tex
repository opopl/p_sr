% vim: keymap=russian-jcukenwin
%%beginhead 
 
%%file 15_01_2022.stz.news.ua.zn.1.diagnoz
%%parent 15_01_2022
 
%%url https://zn.ua/ukr/SOCIUM/khibnij-diahnoz-.html
 
%%author_id news.ua.zn,druzenko_gennadij
%%date 
 
%%tags ukraina,istoria,diagnoz
%%title Хибний діагноз, або Невивчені уроки минулого
 
%%endhead 
\subsection{Хибний діагноз, або Невивчені уроки минулого}
\label{sec:15_01_2022.stz.news.ua.zn.1.diagnoz}

\Purl{https://zn.ua/ukr/SOCIUM/khibnij-diahnoz-.html}
\ifcmt
 author_begin
   author_id news.ua.zn,druzenko_gennadij
 author_end
\fi

Відомо, що хибний діагноз у медицині чи хибна кваліфікація злочину в
юриспруденції майже завжди призводять до результату, протилежного очікуваному.
Від помилкового діагнозу, а отже й неправильного лікування перебіг хвороби
тільки погіршується. Від хибної кваліфікації переступу шанси злочинця на
виправдувальний вирок значно зростають. Я стверджую, що колоніально-окупаційний
наратив — це хибний діагноз нашому минулому, який заважає вирватися з його
полону, або, за словами \href{https://nashformat.ua/products/podolaty-mynule-globalna-istoriya-ukrainy-925296}{%
нашого вдумливого історика професора Ярослава Грицака,
«Подолати минуле».}

\ii{15_01_2022.stz.news.ua.zn.1.diagnoz.pic.1}

Безліч фактів свідчать, що Україна ніколи не була колонією Росії. Натомість
козацька еліта (українська нація з’явиться на світ набагато пізніше) виступила
співтворцем Російської імперії, а українці — співтворцями СРСР. Як шотландські
еліти під тиском обставин, але цілком свідомо утворили спільно з англійцями
Британську імперію. Досить згадати визначальну роль прокоповичів, розумовських,
безбородьків чи кочубеїв у творенні імперії з центром у штучному
космополітичному Петербурзі, аби переконатися, що козацькі еліти свідомо
розбудовували спільну з єдиновірними московитами імперію (і таким чином через
використання імперської військової потуги воліли вирішити, й таки вирішили
«польське» та «османське» питання). Спробуймо уявити русинів-галичан на
аналогічних посадах в Австрійській, а пізніше — в Австро-Угорській імперії. Не
виходить? У мене теж...

Це не означає, що імперія була раєм для українців. Аж ніяк. Але Російська
імперія точно не була державою великоросів. Вона була космополітичною країною,
в якій етнічна німкеня стала однією з найвидатніших очільниць, у якій еліта в
певні періоди говорила французькою і німецькою краще, ніж російською, на службу
до якої з’їжджалися представники чи не всіх європейських етносів (цікава
історія, як корсиканець Наполеон ледь не підписав контракт на службу Її
величності). Її найвидатніші поети були нащадками африканців (Пушкін) та
шотландців (Лермонтов). А власне чорношкірий прадід Пушкіна Абрам Ганнібал зміг
дослужитися до генерал-аншефа й отримати дворянський титул у Російській
імперії, що свідчить про надзвичайну на той час відкритість і стрімкі соціальні
ліфти тієї держави, якими козацький стан активно користувався (згадаймо
запаморочливу кар’єру братів Кирила то Олексія Розумовських, що є чудовою
ілюстрацією приповідки «з грязі в князі»).

Врешті-решт те, що у свідомості українців головним ворогом виступали поляки, а
не «московити», підтверджує той промовистий факт, що на початку ХХ сторіччя
Правобережна Україна була вкрай монархічною й лояльною до російського
імператора, якого українці завжди підтримували під час польських повстань ХІХ
сторіччя. «Союз русского народа» (СРН), більше відомий як «чорна сотня»,
справедливо вважав українське Правобережжя своїм оплотом: тільки Почаївський
осередок СРН налічував сто тисяч осіб — чверть усіх чорносотенців імперії. Не
дивно, що на виборах до Третьої Державної Думи радикальні монархісти домоглися
найбільшого успіху саме на Правобережній Україні. Чим далі на Схід, тим менше
було депутатів у СРН. У Харківській губернії вони отримали тільки три мандати з
десяти, проте в Київській — тринадцять із тринадцяти! Також повною перемогою
чорносотенців завершилися вибори в Подільській та Волинській губерніях. І це
1907 рік — майже півстоліття після видання Валуєвського циркуляра і 30 років
після оприлюднення Емського указу!!!

Українська модерна нація (як і інші європейські нації) буквально визрівала в
лоні імперій. Спочатку як культурна окремішність, а згодом сформувалася
претензія на політичну автономію у складі Росії та Австро-Угорщини (які були
поліетнічними імперіями, а не національними державами), відповідно
великоруського чи австрійського й угорського народів. Тімоті Снайдер слушно
наголошує, що до ХХ століття існування європейських націй у складі імперій було
правилом, а не винятком. 

Розв’язавши з допомогою імперської потуги свої екзистенційні загрози із Заходу
(католицька Польща) та Півдня (мусульманська Порта), українські еліти стали
обережно подумувати про більшу політичну суб’єктність у складі Російської
імперії. Міхновський зі своїми ідеями незалежної України на початку ХХ сторіччя
видавався українським елітам небезпечним радикалом, а не виразником vox populi
— голосу народу.

Тільки в цьому контексті можна зрозуміти, чому Центральна Рада, що утворилася
ще в березні 1917-го, а в червні того ж року проголосила себе «найвищим
державним органом України», фактично без бою програла Україну більшовикам, які
прийшли до влади на дев’ять місяців пізніше і мали відтворювати державність,
фактично, з хаосу й руїн, на них же й перетворивши російську державну машину.
Мене завжди цікавило, чому, маючи таку фору в часі та ідеальні умови для
українізації військових частин (у травні 1917 року в Києві відбувся
Всеукраїнський військовий з’їзд, який ухвалив рішення про створення української
національної армії, а вже в серпні Верховний головнокомандувач російських
військ генерал Лавр Корнілов віддав Павлу Скоропадському наказ українізувати
34-й корпус і перетворив його на 1-й український корпус), Центральна Рада
опинилася фактично безпорадною перед (увага!!!) сімома (лише сімома!!!)
тисячами муравйовських зарізяк? І здала їм півмільйонний Київ, у якому було
дев’ять тисяч українського війська з начебто лояльних до Центральної Ради 20
тисяч?

Ось цікавий уривок зі спогадів першого очільника українського уряду Володимира
Винниченка «Відродження нації», який частково відповідає на моє майже риторичне
запитання:

\begin{zzquote}
«На наших очах «розкладались» найміцніші, найзавзятіші полки, що прибували з
таким ентузіазмом до Київа. Полки імени ріжних гетьманів, які так свідомо, так
струнко, так рішуче вступали в столицю України для оборони й захисту її, які
так веселили всі національні серця своєю національною свідомостю, щиростю,
жовто-блакитними прапорами й українськими піснями, які так гучно кричали
«славу» українській владі, ці полки через якусь пару тижнів дивним способом
спочатку губили все своє завзяття, потім упадали в апатію, в «нейтралітет» до
большевиків, а потім — повертали разом з тими большевиками свої українські
баґнети проти нас».	
\end{zzquote}

З одного боку, революція 1917 року змусила українців приймати передчасні (!!!)
пологи політичної нації. Ми бачимо, що Центральна Рада до останнього не
наважувалася проголошувати повну незалежність України від Росії. Від Першого
Універсалу, що декларував: «Не одділяючись від всієї Росії, не розриваючи з
державою російською, хай народ український на своїй Землі має право сам
порядкувати своїм життям», до Третього, що проголошував створення УНР, але «не
відділяючись від республики Російськоі і зберігаючи єдність іі» з метою
«помогти всій Росіі, щоб вся республика Російська стала федерацією рівних і
вільних народів». До речі, Третій Універсал був проголошений уже після (!!!)
більшовицького заколоту в Петрограді. До повного розриву з уже червоною Росією
УНР спонукали не внутрішня логіка чи суспільний запит, а тиск Центральних
держав на мирних перемовинах у Бересті, що й дозволило представнику Другого
Райху на тих переговорах, генералові Максу Гофману, пізніше написати у
спогадах: «Я придумав Україну».

З іншого боку, вододіл під час революції та визвольних змагань проходив по
лінії не так національно-культурного, як соціального розмежування. Не
забуваймо, що коли в лютому 1918-го Київ брали загони етнічного росіянина
(«великороса») Муравйова, то через рік Директорію з української столиці вибив
уже етнічний українець Микола Щорс на чолі Другої бригади Першої української
радянської дивізії, до складу якої входили Богунський і Таращанський полки,
сформовані майже повністю з етнічних українців. Він же 1918-го воював на
Чернігівщині проти німецько-австрійських військ, які окупували Україну на
запрошення уряду УНР, а пізніше — гетьмана Скоропадського.

Українці саме тому й вагалися між Центральною Радою, гетьманом та Директорією,
з одного боку, і червоним урядом України — з іншого, що соціальне питання було
для них не менш важливим за національне. Коли всі ці щорси, боженки, пархоменки
обирали між Директорією й УСРР — вони чомусь обирали останню, що позиціювала
себе як «справді соціалістична» українська республіка, що створила власну
Українську радянську (червону) армію на чолі з етнічним українцем Володимиром
(Антоновим-) Овсієнком і наркомом військових та морських справ етнічним
українцем Миколою Подвойським...

Отже, УНР/Гетьманат/Директорія для багатьох українців були недостатньо лівими,
але водночас недостатньо дружніми з Росією, яка після більшовицького перевороту
позиціювала себе як локомотив світової революції та боротьби за соціальну
справедливість. Повернімося до свідчень Винниченка: 

\begin{zzquote}
«Вся причина [поразки] в тому, що українська влада, що вся керуюча, партійна
українська демократія розійшлася з своїми масами, що вона була соціально
непослідовна, нерішуча, невиразна й не соціалістична. [...] Повторяю сказане мною
в початку сієї праці: ми, всі українські керуючі партії, були не соціалістами,
а тільки демократами, республіканцями й національними революціонерами».
\end{zzquote}

Саме тому, коли ми розглядаємо радянський період в історії України як
окупаційно-колоніальний, ми ставимо хибний діагноз хворобі, що погубила нашу
державність століття тому. Українці масово й активно долучилися до
комуністичного експерименту саме тому, що прагнули соціальної справедливості
більше, ніж незалежності. Ідея побудови справедливого суспільства на той час не
мала собі рівних за привабливістю. На «лівизну» хворіли майже всі українські
політики.

Нобелівська лауреатка Свєтлана Алексієвич, можливо найбільш глибока дослідниця
агонії та смерті СРСР, дуже слушно пише у своїй книжці «Время секонд-хенд»: 

\begin{zzquote}
«Из «Записных книжек» Шаламова: «Я был участником великой проигранной битвы за
действительное обновление жизни». Написал это человек, отсидевший семнадцать
лет в сталинских лагерях. Тоска об идеале осталась... Советских людей я бы
разделила на четыре поколения: сталинское, хрущевское, брежневское и
горбачевское. Я — из последнего. Нам было легче принять крах коммунистической
идеи, так как мы не жили в то время, когда идея была молодая, сильная, с
нерастраченной магией гибельного романтизма и утопических надежд».
\end{zzquote}

Українці кінця 1910-х — 1920-х жили в часи, коли комуністична ідея була
«молодою, сильною, з нерозтраченою магією погибельного романтизму та утопічних
надій». Недарма один із найвідоміших творів співця українського комунізму
Миколи Хвильового так і називається «Я (Романтика)». І тому сьогодні сумно й
смішно спостерігати, як адепти націоналізму — іншої тоталітарної спокуси першої
половини ХХ століття — безапеляційно засуджують комунізм. Врешті-решт, навіть у
своєму зеніті — в нацистському Третьому райху — націоналістична ідея спрацювала
тільки тому, що вона була водночас соціалістичною. Як і в Ізраїлі 1948–50-х
років.

В окупаційно-колоніальній парадигмі неможливо зрозуміти, чому сталінський режим
винищив не тільки представників українського «розстріляного відродження», а й
квіт російської культури: Льва Гумільова, Осіпа Мандельштама, Даниїла Хармса,
Бориса Пільняка. Тим, хто вижив (як і українці Борис Антоненко-Давидович чи
Григорій Кочур), жилося теж не солодко: Ахматова, Пастернак, Шаламов... Мовчу
вже про «філософський пароплав»… З іншого боку, співці комунізму точно не
бідували при «совєтах», пишучи що російською, що українською: Довженко, Бажан,
Корнійчук, Рильський, а пізніше — Павличко, Драч, Яворівський (сумнозвісні
своїм листом Андропову на засудження шістдесятників Світличного, Стуса,
Сверстюка, Караванського та інших)... 
 
Чи мав місце в радянські часи імперський шовінізм? Безумовно! І справді, якщо
1972-го Бродського за «антирадянську діяльність» позбавлили громадянства й
вислали з СРСР, то Стуса кинули на п’ять років до в’язниці та Мордовських
таборів плюс два роки каторги... Але бачити в Радянському Союзі тільки
«русифікацію» та «колонізацію» України — означає бачити краплі, та не бачити
дощу. 

Саме об’єктивні успіхи в електрифікації, індустріалізації, урбанізації,
боротьбі з неписьменністю, масовими хворобами, створенні ефективної й доступної
системи охорони здоров’я та загальної освіти, доступності університетської
освіти, масовому забезпеченні громадян житлом і забезпеченні відносної
соціальної справедливості та соціальних ліфтів — створили радянській владі
високу легітимність в очах, зокрема й українських громадян СРСР. Засаднича
проблема Союзу полягала не в правильній інтуїції збільшення ролі держави в
суспільстві як контрбалансу й противаги розгулу ринкових сил (що, залишені
напризволяще, працюють як центрифуга: роблять багатих ще багатшими, а бідних —
ще біднішими), а в нерозумінні меж людських можливостей. В амбіції тотально
змінити світоустрій і «розвернути річки назуспіт». Коли стало зрозуміло, що
людині не під силу побудувати настільки радикально «новий світ», як обіцяли
комуністи, ідея видохлась — і СРСР помер.

Але своїм 70-річним існуванням він змінив весь світ, який під загрозою і
спокусою комунізмом став набагато справедливішим та гуманнішим. Принаймні в
соціальному сенсі. 

Утім, повертаючись до нинішнього панівного в Україні сприйняття нашої історії.
Це зовсім інший наратив. Який говорить, що на українських землях одвічно жили
працьовиті, хоробрі та волелюбні вояки-хлібороби. Яких раз у раз підступно
завойовувала, російщила та нищила Москва. Причому питання, як семитисячна армія
Муравйова могла фактично без бою дійти до Києва й за кілька днів узяти столицю
УНР, чиї війська налічували щонайменше 20 тисяч, вважається провокаційним. 

Це контрпродуктивний наратив. Який тільки підкреслює нашу слабкість. І таїть у
собі тільки нові й нові розчарування в майбутньому. Бо сильний має мужність
дивитися правді у вічі, визнавати власні помилки та брати на себе
відповідальність за скоєні злочини. А українці накоїли їх чимало. Зокрема
знищуючи інших українців під час визвольних змагань, встановлення радянської
влади, страшного Голодомору, Другої світової війни чи задушливих часів
Щербицького-Федорчука (обидва, до речі, теж етнічні українці).

Національна ідея мобілізує. І це ми побачили на власному прикладі у 2014–15
роках, коли українці створювали добробати, масово волонтерили для фронту і
перетворили АТО на справді народну війну. Але в цієї мобілізації є свої межі.
Бо національна державність — це лише оболонка, посудина, яка може бути
наповнена дуже різним змістом. І якщо у формулі державності бракує соціальної
солідарності та справедливості, це дуже нестійка сполука. 

Аби вистояти в наш критичний час і не скотитися до чергової руїни (яка для
українців завжди закінчується поверненням у «русский мир»), нам потрібно
засвоїти уроки імперського та радянського періодів у нашій історії. Відмовитися
від спрощено-хибного наративу жертви. Зрозуміти, навіщо козацькі еліти й
інтелектуали-могилянці допомогли Петру І та його наступникам і наступницям
збудувати Російську імперію. Бо це урок стратегічного і геополітичного (а не
примітивно-догматичного) мислення на століття наперед. А також усвідомити, чому
століття тому українці підтримали радянську владу і масово переходили на бік
червоних. Не розкопуючи небіжчика СРСР, ми маємо успадкувати від нього етос
соціальної справедливості та солідарності, який перетворює маси на спільноту.
Ми маємо не проклинати наших батьків, що жили в радянські часи й реалізували
себе в радянських умовах, як запроданців та колаборантів, а, навпаки, —
подякувати їм за житлові масиви, дороги, електростанції, заводи, школи,
лікарні, університети, аеропорти, що дозволили незалежній Україні протриматися
тридцять років, майже не інвестуючи в базову інфраструктуру, а лише
експлуатуючи радянський спадок.

Якщо ми засвоїмо ці два уроки, є шанс, що цього разу Україна вистоїть. Якщо ні,
і знову в усьому будуть винні злі та підступні москалі, руїна і повернення в
«русский мир» майже неминучі. Бо учня, який не засвоїв програму, знову й знову
залишають на повторний курс...
