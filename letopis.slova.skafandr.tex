% vim: keymap=russian-jcukenwin
%%beginhead 
 
%%file slova.skafandr
%%parent slova
 
%%url 
 
%%author 
%%author_id 
%%author_url 
 
%%tags 
%%title 
 
%%endhead 
\chapter{Скафандр}
\label{sec:slova.skafandr}

%%%cit
%%%cit_head
%%%cit_pic
%%%cit_text
И снова ЧП. После соприкосновения «Союза» с «Салютом» выяснилось, что они не
дотянулись друг до друга на 90 миллиметров. В такой ситуации переходить на
станцию без \emph{скафандров} — смерти подобно. Их же на трехместных кораблях
отменили еще с полета «Восхода-1» в октябре 1964 года.  Существовавшие в то
время \emph{скафандры} «Беркут» и разработанный на его основе «Ястреб» весили
более 20 килограммов каждый, не считая ранцевого оборудования. Это полезное, но
громоздкое оснащение занимало много места, и им решили пожертвовать в пользу
третьего члена экипажа, тем самым сильно опередив американцев, которые
запустили сразу троих астронавтов только в октябре 1968 года. К тому же
считалось, что советские космические корабли настолько надежны в плане
герметичности, что в них можно летать хоть в одних трусах
%%%cit_comment
%%%cit_title
\citTitle{«Они были обречены» 50 лет назад погиб экипаж «Союза-11». Кто
виноват в главной трагедии советской космонавтики?: Космос: Наука и техника:
Lenta.ru}, Сергей Варшавчик, lenta.ru, 30.06.2021
%%%endcit

