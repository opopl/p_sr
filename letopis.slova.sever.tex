% vim: keymap=russian-jcukenwin
%%beginhead 
 
%%file slova.sever
%%parent slova
 
%%url 
 
%%author 
%%author_id 
%%author_url 
 
%%tags 
%%title 
 
%%endhead 
\chapter{Север}
\label{sec:slova.sever}

%%%cit
%%%cit_pic
\ifcmt
  pic https://avatars.mds.yandex.net/get-zen_doc/4389079/pub_60c3819603088a3a1a36d8ea_60c38bfa03088a3a1a574b82/scale_1200
	caption Великороссам приходилось выживать в суровом климате Северной Руси. Зато они были свободными, и умели свою свободу защищать
\fi
%%%cit_text
В \emph{землях северных} переселенцев никаких торговых путей не было, и все
необходимые товары им пришлось научиться изготавливать самостоятельно, проявляя
изрядную сообразительность и смекалку. Этим они обычно занимались зимой, после
короткого ударного лета, во время которого приходилось напрягать все силы,
чтобы успеть провести все полевые работы, пока погода позволяет это сделать. К
тому же \emph{северные русские}, которые позже стали называть себя
великороссами, никогда не жили в больших деревнях, как на юге, и таким образом
привыкли действовать малыми силами, но не менее ударно. Это развило в них
упорство, необходимость поступать практично и как можно более эффективно,
соответственно очень сильно повысилось их военное искусство
%%%cit_title
\citTitle{Почему великороссы стали такими сильными, а малороссы остались такими слабыми?}, 
Исторический Понедельник, zen.yandex.ru, 11.06.2021
%%%endcit

