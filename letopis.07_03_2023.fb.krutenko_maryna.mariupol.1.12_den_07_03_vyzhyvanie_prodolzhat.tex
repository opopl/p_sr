%%beginhead 
 
%%file 07_03_2023.fb.krutenko_maryna.mariupol.1.12_den_07_03_vyzhyvanie_prodolzhat
%%parent 07_03_2023
 
%%url https://www.facebook.com/marinakrytenko/posts/pfbid0Lsr1waPuRDN8ZzUMb6HCnhokPkHnsNLHKtiX8YVxvicSLhGHEEDC8mHhayB4k3kFl
 
%%author_id krutenko_maryna.mariupol
%%date 07_03_2023
 
%%tags 07.03.2022,dnevnik,mariupol,mariupol.war
%%title ДВЕНАДЦАТЫЙ ДЕНЬ ВОЙНЫ. 07.03.22 Так как мы не уехали, нужно продолжать своё выживание
 
%%endhead 

\subsection{ДВЕНАДЦАТЫЙ ДЕНЬ ВОЙНЫ. 07.03.22 Так как мы не уехали, нужно продолжать своё выживание}
\label{sec:07_03_2023.fb.krutenko_maryna.mariupol.1.12_den_07_03_vyzhyvanie_prodolzhat}

\Purl{https://www.facebook.com/marinakrytenko/posts/pfbid0Lsr1waPuRDN8ZzUMb6HCnhokPkHnsNLHKtiX8YVxvicSLhGHEEDC8mHhayB4k3kFl}
\ifcmt
 author_begin
   author_id krutenko_maryna.mariupol
 author_end
\fi

ДВЕНАДЦАТЫЙ ДЕНЬ ВОЙНЫ. 07.03.22

Так как мы не уехали, нужно продолжать своё выживание. 

Проснулись, поели и пошли за водой. На околицах 23 микрорайона шёл бой. 

Очередь была из 1000 человек. Из-за полной информационной изоляции, люди были в
не понимании что им делать и сколько ещё все это будет продолжатся. 

Главным в спортзале-бомбоубежище был тренер Топузов Евгений, который управлял
всем: гуманитарной помощью, волонтерами, этим водовозом который привозил воду.
Мне нужно было найти лекарство для сына, он сказал по рации чтоб я прошла и на
все вопросы, я должна была говорить «Жека тренер мне разрешил».

Он занял место мэра, который должен был заняться вопросами горожан. И таких
«мэров» были в каждом районе и в каждом бомбоубежище. А наш мэр сбежал не
предупредив об опасности горожан. Спасаться и боятся - не стыдно, нужно было
объявить об этом жителям города, а там это их выбор, по крайней мере ты их
предупредил.....

Я наблюдала как на улице возле Терраспорт подъезжали военные и привозили
продукты для беженцев которые находились в спортзале. В этом спортзале жила
девушка или жена одного военного, когда он приезжал она плакала и обнимала его,
когда он уезжал она плакала и не хотела его отпускать. 

На носилках в спортзал занесли парня, краем глаза мне показалось что это мой
сын, я начала паниковать, присмотревшись, поняла, что это не он. 

Тем временем Никита пошёл в сторону улицы Зелинского, чтоб посмотреть на
разрушения (300-400 м от нас). Увидел, что теплотрасса перебита и люди набирают
техническую воду. Эту воду нельзя было пить, в неё добавляют противомикробный
компонент, ещё вода ржавая. Никита прибежал домой взял канистры и принёс домой
20 литров. 

По-моему в этот день на площадь Кирова прилетела ракета С300, мы слышали звук,
потом узнали, что там есть погибшие и раненые. 

Военные (наши украинские) дали мне ящик размороженной красной рыбы и сказал,
чтоб мы взяли себе и раздали соседям. Я расфасовала по пакетикам примерно по
одному или два килограмма и смотря сколько человек в семье, разнесла по
соседям. 

Мы сварили уху и засолили рыбу для бутербродов...

Под привыкший звук боев, мы легли спать. 

Продолжение следует....

%\ii{07_03_2023.fb.krutenko_maryna.mariupol.1.12_den_07_03_vyzhyvanie_prodolzhat.cmt}
