% vim: keymap=russian-jcukenwin
%%beginhead 
 
%%file 06_11_2021.fb.jeremenko_evgenia.1.den_osvobozhdenie_ulica_pushinoj
%%parent 06_11_2021
 
%%url https://www.facebook.com/jenjafr/posts/1552090035130348
 
%%author_id jeremenko_evgenia
%%date 
 
%%tags 06.11.1943,1943.kiev.osvobozhdenie,6_nov,gorod,istoria,kiev,nacizm,osvobozhdenie,sssr,ulica.kiev.pushinoj,vov
%%title 6 ноября 1943 года ‒ ДЕНЬ ОСВОБОЖДЕНИЯ КИЕВА от немецко-фашистских захватчиков
 
%%endhead 
 
\subsection{6 ноября 1943 года ‒ ДЕНЬ ОСВОБОЖДЕНИЯ КИЕВА от немецко-фашистских захватчиков}
\label{sec:06_11_2021.fb.jeremenko_evgenia.1.den_osvobozhdenie_ulica_pushinoj}
 
\Purl{https://www.facebook.com/jenjafr/posts/1552090035130348}
\ifcmt
 author_begin
   author_id jeremenko_evgenia
 author_end
\fi

И ещё раз о ноябрьских днях 1943-го

6 ноября 1943 года ‒ ДЕНЬ ОСВОБОЖДЕНИЯ КИЕВА от немецко-фашистских захватчиков

УЛИЦА ФЕОДОРЫ ПУШИНОЙ

В 1975 году к 40-летию Победы в Великой Отечественной войне 1941-45 гг., улица
Северная в Святошине была переименована и стала носить имя Феодоры Пушиной. 

Вглядитесь в эти старые фотографии. Милая девичья улыбка, по-юношески припухшие
щёчки, полная жизни ладно сбитая фигурка. До своих 20-ти лет эта девочка не
дожила ровно неделю. Отдала жизнь свою за други своя. Девушка из далёкой
предуральской Удмуртии погибла на нашей украинской земле, на киевской улице,
спасая раненых.

\ifcmt
  tab_begin cols=3

     pic https://scontent-frt3-1.xx.fbcdn.net/v/t1.6435-9/254342521_1552088845130467_2430736957911963084_n.jpg?_nc_cat=108&ccb=1-5&_nc_sid=8bfeb9&_nc_ohc=oCvZ3WK8cncAX_Iuq_r&_nc_ht=scontent-frt3-1.xx&oh=a3a7aa8a6762ebba864302bf255b1323&oe=61AACFF8

     pic https://scontent-frx5-1.xx.fbcdn.net/v/t1.6435-9/253371894_1552088895130462_7293224364241855830_n.jpg?_nc_cat=110&ccb=1-5&_nc_sid=8bfeb9&_nc_ohc=cnaiWkYUmwgAX8AxjyO&_nc_ht=scontent-frx5-1.xx&oh=b659e4c62b12476d6ec0ef0062f648d4&oe=61AAC789

     pic https://scontent-frt3-1.xx.fbcdn.net/v/t1.6435-9/253490456_1552088928463792_7767050916975731054_n.jpg?_nc_cat=107&ccb=1-5&_nc_sid=8bfeb9&_nc_ohc=p05tq3lmVnQAX9LouL0&_nc_ht=scontent-frt3-1.xx&oh=14a346142b5e549334eaeb6ac6520966&oe=61ACFC8C

  tab_end
\fi

Эта улыбчивая девушка ‒ лейтенант медицинской службы Феодора Андреевна Пушина,
одна из 17 женщин-медиков, получивших звание Героя Советского Союза.

На сайте «Одна Родина» в рассказе о Феодоре Пушиной опубликованы уникальные
свидетельства: отрывки из переписки Феодоры с родными. В связи с памятной датой
предлагаю этот материал с незначительными сокращениями.

\ifcmt
  tab_begin cols=3

     pic https://scontent-frx5-1.xx.fbcdn.net/v/t1.6435-9/253261155_1552088975130454_2602959387330338164_n.jpg?_nc_cat=100&ccb=1-5&_nc_sid=8bfeb9&_nc_ohc=wJUPNeSIe1MAX-x6Kia&_nc_ht=scontent-frx5-1.xx&oh=db495ed3a0e2a7765680aa7f69f98fba&oe=61AB5B2D

     pic https://scontent-frx5-1.xx.fbcdn.net/v/t1.6435-9/254044899_1552089011797117_3759483173657515988_n.jpg?_nc_cat=111&ccb=1-5&_nc_sid=8bfeb9&_nc_ohc=LhtPszlE8skAX_fyvhz&tn=lCYVFeHcTIAFcAzi&_nc_ht=scontent-frx5-1.xx&oh=1c2089df90cfc9f7fe71353a4aeb5d8d&oe=61ACA803

     pic https://scontent-frx5-1.xx.fbcdn.net/v/t1.6435-9/253639106_1552089051797113_1978218277752779253_n.jpg?_nc_cat=105&ccb=1-5&_nc_sid=8bfeb9&_nc_ohc=ENAeNRAOz7QAX-nTaY6&_nc_ht=scontent-frx5-1.xx&oh=d24ca31937bcf574f15212ab9fbe1e25&oe=61AAAB10

  tab_end
\fi

Феодора (Федора) Пушина родилась 1 ноября 1923 года в селе Иж-Забегалово
Якшур-Бодьинского района Удмуртской АССР. В многодетной крестьянской семье она
была девятым ребёнком. Окончив 7 классов, в 1939 году Феодора поступила в
фельдшерскую школу в Ижевске, где жила её сестра Анна с мужем. Девушке трудно
давалась учеба. Она писала домой, брату Павлу: «Учиться тяжело. Наверно, не
осилить мне. Поеду домой, к родителям. Как ты смотришь?» Но брат мудро и
рассудительно отвечал: «Ты и в детстве была не из пугливых. Так неужели сейчас
решила отступить? Учись — потом спасибо скажешь».

\ifcmt
  tab_begin cols=2

     pic https://scontent-frt3-1.xx.fbcdn.net/v/t1.6435-9/253334691_1552089105130441_4729357331970472274_n.jpg?_nc_cat=107&ccb=1-5&_nc_sid=8bfeb9&_nc_ohc=VGvopAb4ehEAX8-lxtb&_nc_ht=scontent-frt3-1.xx&oh=b8aa7d70c83869c8be47a6f3b280e56f&oe=61AAE416

     pic https://scontent-frt3-1.xx.fbcdn.net/v/t1.6435-9/253565097_1552089141797104_2357392347400317013_n.jpg?_nc_cat=108&ccb=1-5&_nc_sid=8bfeb9&_nc_ohc=VelNDYxMOjQAX_c9A0k&_nc_ht=scontent-frt3-1.xx&oh=3f3db7b556d54e112d06385d6ac3f773&oe=61AD148E

  tab_end
\fi

Сегодня Ижевский медицинский колледж носит имя своей отважной воспитанницы.
Доучивалась девушка уже во время войны. В январе 1942 года была направлена на
работу в фельдшерско-акушерский пункт в селе Кекоран. И сразу на молодую
фельдшерицу было возложено огромное количество обязанностей. В связи с кадровым
дефицитом ей пришлось стать универсальным врачом и выполнять функции хирурга,
терапевта, акушера и педиатра. 18-летняя Феодора сутками принимала больных,
проверяла санитарное состояние школы и фермы, проводила с населением беседы на
темы гигиены и санитарии.

\ifcmt
  tab_begin cols=2

     pic https://scontent-frt3-1.xx.fbcdn.net/v/t1.6435-9/253948429_1552089391797079_8109418674856407320_n.jpg?_nc_cat=102&ccb=1-5&_nc_sid=8bfeb9&_nc_ohc=xl9QP0Fzg9EAX96aib6&_nc_ht=scontent-frt3-1.xx&oh=e4c19a2616150ab6a9f79a481bdf9a2a&oe=61ADF627

     pic https://scontent-frt3-1.xx.fbcdn.net/v/t1.6435-9/252752711_1552089218463763_8118555096261790743_n.jpg?_nc_cat=108&ccb=1-5&_nc_sid=8bfeb9&_nc_ohc=4u3RvnTTMw4AX_WD5Tp&_nc_ht=scontent-frt3-1.xx&oh=f392a04781e32d114ad6847f4c6063dc&oe=61AB1D65

  tab_end
\fi

Через полгода — в апреле 1942 года — Ф. Пушину призвали в ряды Советской армии.

Писатель В. Широбоков в очерке «На рассвете» описывает эпизод прощания Феодоры
с мирной жизнью: «Всюду разлилась талая вода — ни пройти, ни проехать. А Феню
Пушину вызвали в военкомат. Хорошо, что она была не одна. С нею рядом шагала
сестра Аня. Чуть ли не вплавь переправлялись сёстры через бурлящие овраги и
лога. Аня тоскливо посмотрела на сестрёнку. Ей она почему-то показалась
маленькой, одинокой, и захотелось чем-то помочь, хотя бы душевными словами. Но
те слова не пришли, на глазах выступили горячие росинки. Аня шагала молча,
перекидывая тяжелый чемодан из одной руки в другую. Молчала и Феня. 

\ifcmt
  tab_begin cols=3

     pic https://scontent-frx5-1.xx.fbcdn.net/v/t1.6435-9/253565178_1552089281797090_3242114650655402828_n.jpg?_nc_cat=111&ccb=1-5&_nc_sid=8bfeb9&_nc_ohc=L3I19nqR1GEAX_8cel2&_nc_ht=scontent-frx5-1.xx&oh=6fe686cc46bf30ef35bd5e0ddc0bbba8&oe=61AB3E1E

     pic https://scontent-frt3-2.xx.fbcdn.net/v/t1.6435-9/254065250_1552089318463753_5154830453039337592_n.jpg?_nc_cat=101&ccb=1-5&_nc_sid=8bfeb9&_nc_ohc=zjufznxLP88AX9Or_TS&_nc_ht=scontent-frt3-2.xx&oh=ca87e6ad0eb7609bfeff1c4eaaed3544&oe=61AC749A

     pic https://scontent-frx5-2.xx.fbcdn.net/v/t1.6435-9/253932987_1552089351797083_862273057587356227_n.jpg?_nc_cat=109&ccb=1-5&_nc_sid=8bfeb9&_nc_ohc=IkIBG6ivkvMAX-c1vop&tn=lCYVFeHcTIAFcAzi&_nc_ht=scontent-frx5-2.xx&oh=a75b19eb69ba621de2cb30114033f4a9&oe=61AAF6D1

  tab_end
\fi

На станции Узгинка они сели в поезд, вечером приехали в Ижевск, а после
полуночи простились. Аня целовала любимую сестру, жала её руку, шептала
напутственные слова и заплакала. Не выдержала и Феня, слёзы покатились по её
горячим щекам. И вот свисток паровоза, последние заветные слова. ... Поезд
медленно тронулся и увёз девушку далеко-далеко, туда, где шли ожесточённые
бои».

А уже в августе молодая санитарка оказалась на фронте. Она служила
военфельдшером на 1-м Украинском фронте. Бойцы назвали свою благодетельницу
Фаиной.

\ifcmt
  tab_begin cols=2

     pic https://scontent-frt3-1.xx.fbcdn.net/v/t1.6435-9/253731135_1552089531797065_6474889526152682601_n.jpg?_nc_cat=106&ccb=1-5&_nc_sid=8bfeb9&_nc_ohc=wg__lN9_-8gAX-QYRnX&_nc_ht=scontent-frt3-1.xx&oh=4be16ab7fc0f58aa718fa94ec0041e3e&oe=61AC4737

     pic https://scontent-frx5-1.xx.fbcdn.net/v/t1.6435-9/253252960_1552089448463740_3213282472205226177_n.jpg?_nc_cat=110&ccb=1-5&_nc_sid=8bfeb9&_nc_ohc=ULv6HfnDZsEAX_5vU27&_nc_ht=scontent-frx5-1.xx&oh=df76d0352d37f52cdb2d30a27be6cf11&oe=61AC95B5

  tab_end
\fi

Из писем Ф. Пушиной к сестре Анне мы узнаём, что суровая фронтовая жизнь не
очерствила её сердце, на фронтовых дорогах она повстречала свою любовь. Об этом
девушка решалась говорить только намёками и даже не открыла сестре имя
избранника. Известно только, что он тоже был санинструктором. Их встречи были
короткими, беседы задушевными и спасительными. Но этой любви не суждено было
продолжаться — парня перевели в другую часть.

\ifcmt
  tab_begin cols=2

     pic https://scontent-frt3-2.xx.fbcdn.net/v/t1.6435-9/253282497_1552089505130401_5514220140261749324_n.jpg?_nc_cat=103&ccb=1-5&_nc_sid=8bfeb9&_nc_ohc=-ZcUwmecOUwAX-9de7B&tn=lCYVFeHcTIAFcAzi&_nc_ht=scontent-frt3-2.xx&oh=9f2cadad2aeefa4c47a93ea42e7f9c60&oe=61AD9AB0

     pic https://scontent-frt3-1.xx.fbcdn.net/v/t1.6435-9/253575818_1552089581797060_4228401866006901649_n.jpg?_nc_cat=104&ccb=1-5&_nc_sid=8bfeb9&_nc_ohc=9V6ZrAoV3QwAX9RTWoQ&tn=lCYVFeHcTIAFcAzi&_nc_ht=scontent-frt3-1.xx&oh=aa503e3fccc22e560e9a00e408b3dcb6&oe=61AC6E9E

  tab_end
\fi

7 февраля 1943 года Ф.А. Пушина вывела 45 раненых бойцов из-под
артиллерийско-миномётного огня в районе деревни Прилепы Курской области. Спустя
всего 4 дня, 11 февраля, находясь на передовом медицинском пункте в деревне
Пузачи, Феодора оказала медицинскую помощь 57-ми раненым бойцам и командирам. А
при отходе советских частей из этой деревни вынесла все перевязочные материалы
и медикаменты. Стоит только вдуматься в эти цифры. Какое огромное количество
страдающих окровавленных, изувеченных солдат «прошли» через заботливые руки
Фени Пушиной! Какую небывалую душевную стойкость, какую храбрость и
самоотверженность проявляла эта девушка с добрым, широким крестьянским лицом!

«Здесь бои идут круглые сутки, — писала Феодора родственникам. — Всё кругом
свистит и сверкает, а на душе хоть и страх, но делаешь своё, где присядешь, где
ляжешь да опять вперед пробираешься».

За тот эпизод Ф. Пушина была награждена орденом Красной Звезды.

Конечно, как и весь народ, Феня свято верила в скорую победу над фашистами. Она
писала сестре: «Октябрьский праздник мы хотим провести в Киеве. Город
обязательно освободим от немецких захватчиков...».

В ноябре 1943 года наши войска вели напряженные бои за освобождение Киева.
Особенно жестокие бои развернулись в районе Святошино: здесь, по шоссе
Киев-Житомир, проходил последний оборонительный рубеж гитлеровцев, прикрывавший
подступы к Киеву. 5 ноября советские солдаты заняли Святошино и начали
продвигаться к центру города. В Святошине расположились госпиталь (здание
бывшей школы №72 на ул. Ф. Пушиной 54, сейчас Межшкольный
учебно-производственный комбинат Святошинского района ‒ прим. авт.) и
санитарная часть 520-го стрелкового полка 167-й стрелковой дивизии, в котором
служила Феодора Пушина. Госпиталь был переполнен. Федора трудилась, помогая
истекающим кровью бойцам, не смыкая глаз, не покладая рук.

Рано утром 6 ноября над Святошино пронеслась группа вражеских бомбардировщиков,
которые нанесли удар по селу. От прямого попадания бомбы загорелось здание
святошинского полевого госпиталя. Феня Пушина со своим соратником, командиром
санитарной роты Николаем Копытёнковым бросилась спасать раненых, рискуя жизнью.
Пробираясь почти наощупь в дыму и огне, отважные медработники спасали
человеческие жизни одну за другой. Бездвижных они выносили на руках, тем же,
кто хоть как-то мог передвигаться, помогали выйти из пламени. Фаина успела
вывести из горящего здания более тридцати солдат и офицеров. Чудом удалось
спасти всех раненых. Когда молодая медсестра бросилась за последним, здание
стало обрушиваться. На девушке загорелась одежда, и она потеряла сознание. Н.
Копытенков в последний момент, перед самым обрушением здания, вынес её из огня.
На Феодоре не было «живого места». Вскоре Ф. Пушина со сплошными ожогами тела,
с сильным повреждением головы и большими потерями крови, отравленная угарным
газом, скончалась на руках своего боевого товарища Н. Копытёнкова. В этот день
советские войска освободили Киев. 

Похоронили Феодору Пушину на Святошинском кладбище.

Вечная память и Царство Небесное одной из тех, кто отдал свою едва начавшуюся
жизнь за неизвестных друзей – за нас с вами, без раздумий выполнив Христов
завет: «Нет больше той любви, как если кто положит жизнь свою за други своя»
(Иоан., 15:13).
