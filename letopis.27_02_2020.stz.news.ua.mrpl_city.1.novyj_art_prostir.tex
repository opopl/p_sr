% vim: keymap=russian-jcukenwin
%%beginhead 
 
%%file 27_02_2020.stz.news.ua.mrpl_city.1.novyj_art_prostir
%%parent 27_02_2020
 
%%url https://mrpl.city/blogs/view/stvorennya-novogo-art-prostoru
 
%%author_id demidko_olga.mariupol,news.ua.mrpl_city
%%date 
 
%%tags 
%%title Створення нового арт-простору
 
%%endhead 
 
\subsection{Створення нового арт-простору}
\label{sec:27_02_2020.stz.news.ua.mrpl_city.1.novyj_art_prostir}
 
\Purl{https://mrpl.city/blogs/view/stvorennya-novogo-art-prostoru}
\ifcmt
 author_begin
   author_id demidko_olga.mariupol,news.ua.mrpl_city
 author_end
\fi

Дорогі читачі, маю сміливість запропонувати вам відкрити для себе новий
культурний майданчик у нашому місті. За останні роки Маріуполь дуже змінився,
він став стрімко розвиватися, об'єднуючи навколо себе тих, хто готовий до змін
не тільки зовнішніх, але й внутрішніх. Для мене Маріуполь – це мій дім, з яким
я пов'язую своє майбутнє. Можливо, саме тому я хочу продовжити діяльність, що,
сподіваюся, допоможе зберегти та примножити історичну і культурну спадщину
нашого неповторного і унікального міста.

\subsubsection{Передісторія}

У 2017 році в Маріуполі була зареєстрована нова громадська організація \textbf{\enquote{Фонд
збереження культурної спадщини Маріуполя}}. Мета організації – виявлення,
збереження та примноження культурної та історичної спадщини Маріуполя. Голова
організації \textbf{Сабадаш Юлія Сергіївна}, доктор культурології, професор МДУ і мій
науковий керівник, дуже мудра і енергійна людина, вирішила створити
організацію, яка об'єднає небайдужих та ініціативних містян з активними і
творчими студентами. Мене обрали заступником голови організації і деякі напрями
діяльності нашого Фонду ми почали реалізовувати. Ще тоді у мене було багато
ідей і ще більше планів. Але життя внесло свої несподівані корективи. Всі свої
сили я спрямувала на захист дисертації, а потім створення сім'ї... І все ж таки
ідея створити новий арт-простір, який допоможе втілювати всі цілі організації
мене не полишала. А коли у 2019 році я почала писати блоги про яскравих і
непересічних маріупольців, то остаточно переконалася, що в моєму місті стільки
талановитих людей, які потребують у спілкуванні, визнанні, презентації власної
творчості, що створення нового вільного простору не буде зайвим. 

\subsubsection{Що це буде?}

\textbf{\enquote{М-студія} (Повна назва – \enquote{Студія культуротворчого перфомансу Маріуполя})} –
таку назву носитиме новий арт-простір, що відкриється при Маріупольському
державному університеті в аудиторії 108. В студії не тільки відбуватимуться
заходи, головна мета – об'єднати ініціативних та небайдужих маріупольців, які
готові вивчати, популяризувати та зберігати історичні і культурні здобутки
міста. Водночас це вільний простір, який надасть можливість творчим і
креативним маріупольцям заявити про себе. Я сподіваюся, що ця студія стане
майданчиком для проведення творчих зустрічей, майстер-класів, розробки нових
екскурсій, реалізації спільних проєктів, презентації творчості талановитих
містян. Найважливіше для мене – це не шукати контент для заповнення простору.
Головне, щоб ті, кому є чим поділитися, самі приходили. А М-студія стане
гостинним домом для всіх, хто намагається зберегти історичну спадщину Маріуполя
та привнести щось своє, унікальне і самобутнє у майбутній культурний розвиток
міста.

\ii{27_02_2020.stz.news.ua.mrpl_city.1.novyj_art_prostir.pic.1}

\subsubsection{Основні напрями діяльності}

Основні напрями діяльності М-студії будуть перетинатися з напрямами ГО
\enquote{Фонд збереження культурної спадщини Маріуполя}:

1. Виявлення, наукове вивчення та збереження пам'яток історії та культури
Маріуполя. Одним з головних завдань студії буде створення музею просто неба на
території старого цвинтаря Маріуполя, заснованого у 1832 році. Всіх охочих
долучитися до цієї благородної і цікавої справи буду дуже чекати у новому
арт-просторі.

2. Співпраця з міжнародними організаціями та вітчизняними громадськими
організаціями у сфері збереження культурної спадщини Маріуполя. Сподіваюся, що
в студії стануть можливими розробки нових проєктів, які обов'язково знайдуть
фінансову підтримку.

3. Культурно-просвітницька діяльність серед населення. Проведення екскурсій,
історично-культурних квестів серед маріупольців. У студії будуть розроблятися
нові екскурсійні маршрути, які сприятимуть популяризації історичного минулого
міста та підвищить туристичну привабливість Маруполя.

4. Підвищення театральної культури мешканців міста, їхньої обізнаності з питань
історії театрального мистецтва Маріуполя. При М-студії продовжить діяльність
\textbf{арт-центр \enquote{Театр всередині тебе}} (створений у 2017 році). Завдяки активній
діяльності як професійних театрів Маріуполя, так і самодіяльних, є потреба у
створенні платформи для діалогу між глядачами та режисерами і акторами.

5. Формування аналітичних та інформаційних продуктів, що сприятимуть посиленню
інтересу з боку громадян, експертного середовища до охорони культурної спадщини
Маріуполя. Видання книг, брошур, каталогів.

6. Участь у створенні нових експозицій, музеїв, культурних ландшафтів,
реконструкція та реставрація культурних пам'яток Ма\hyp{}ріуполя з метою подальшого
їхнього збереження. При студії працюватиме театральна експозиція, що
висвітлюватиме історію театральних закладів Маріуполя з XIX ст. до сьогодення.
Всі учасники студії зможуть брати безпосередню участь у створенні нових
експозицій та виставок.

7. Творчі зустрічі з митцями Маріуполя (художниками, письменниками, фотографами
тощо).

8. Захист і популяризація української мови. Сподіваюся, що в студії спілкування
українською буде невимушеним, а обговорення правил, чи новвоведь в українській
мові стане ще одним напрямом діяльності Студії культуротворчого перфомансу.

Цифра 8 в нумерології вважається однією з найсильніших. Спо\hyp{}діваюся, що 8
напрямів у роботі студії посприяють її успішній діяльності.

\subsubsection{Коли відкриється?}

М-студія відкриється 5 березня о 14:30 в аудиторії 108 Маріупольського
державного університету (історичний факультет) виставкою робіт Христини Кібець
на тему: \enquote{Оголюючи наші душі}. Роботи цієї художниці відрізняються особливим
авторським баченням та мають сильний посил до сучасників. Маю надію, що Студія
культуротворчого перфомансу Маріуполя стане важливим культурним осередком для
багатьох містян, готових до об'єднання, взаємодії та нових звершень.
