% vim: keymap=russian-jcukenwin
%%beginhead 
 
%%file slova.passport
%%parent slova
 
%%url 
 
%%author_id 
%%date 
 
%%tags 
%%title 
 
%%endhead 
\chapter{Паспорт}

%%%cit
%%%cit_head
%%%cit_pic
\ifcmt
  pic https://avatars.mds.yandex.net/get-zen_doc/1708203/pub_6182e0ea7b739e003bd576af_6182ef6a4598a221ee7e215e/scale_1200
  @width 0.4
\fi
%%%cit_text
Тут я дрогнула, потому что в подкладке, в потайном кармашке незаметном у меня
лежит российский \emph{паспорт}. На одну секунду вывернула подкладку и спрятала
ее обратно. У мента глаза зажглись, почуял добычу.  - Дайте мне ее пощупать, -
говорит.  - Зачем?  - У вас в подкладке что-то зашито.  - Нет.  - У вас
\emph{паспорт России}?  - Нет - А что вы там прячете?  - Не лезьте в мою сумку!
- Что вы там прячете? У вас \emph{паспорт России}!  - Вы не имеете права! –
отталкиваю его руки.  Взглядом в меня впивается, как иглой, пытается под кожу
проникнуть. У ментов глаза льдистые, прожигающие вселенским холодом. Это
международное у них, от страны не зависит.  Я молча складываю вещи обратно,
пальцы мои дрожат.  - Дайте сюда вашу сумку, - повторяет он угрожающе.  - Вы не
имеете права!  - У вас \emph{паспорт России}! – с таким озлоблением бросил, как
будто уличил в преступлении и сейчас защелкнет на мне наручники.  Наступает на
меня, я от него пячусь. Он кладет свои руки на мою сумочку
%%%cit_comment
%%%cit_title
\citTitle{Таможня не дает добро}, 
Юлия Вельбой, zen.yandex.ru, 04.11.2021 
%%%endcit
