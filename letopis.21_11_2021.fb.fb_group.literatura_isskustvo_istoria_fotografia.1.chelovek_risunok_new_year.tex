% vim: keymap=russian-jcukenwin
%%beginhead 
 
%%file 21_11_2021.fb.fb_group.literatura_isskustvo_istoria_fotografia.1.chelovek_risunok_new_year
%%parent 21_11_2021
 
%%url https://www.facebook.com/groups/2600166296972834/posts/3075596226096503
 
%%author_id fb_group.literatura_isskustvo_istoria_fotografia,apelsinka_listopadovna
%%date 
 
%%tags 1925,hudozhnik,isskustvo,istoria,kultura,novyj_god,risunok,sssr,zarubin_vladimir.hudozhnik.sssr
%%title Человек, нарисовавший Новый Год
 
%%endhead 
 
\subsection{Человек, нарисовавший Новый Год}
\label{sec:21_11_2021.fb.fb_group.literatura_isskustvo_istoria_fotografia.1.chelovek_risunok_new_year}
 
\Purl{https://www.facebook.com/groups/2600166296972834/posts/3075596226096503}
\ifcmt
 author_begin
   author_id fb_group.literatura_isskustvo_istoria_fotografia,apelsinka_listopadovna
 author_end
\fi

Человек, нарисовавший Новый Год. 

Автором целого мира забавных зверушек на легендарных советских открытках был
Владимир Иванович Зарубин. За 30 лет работы в свет вышло более 1,5 миллиарда
открыток и конвертов с его рисунками, однако умер художник практически в
нищете. 

\begin{multicols}{2}
\ii{21_11_2021.fb.fb_group.literatura_isskustvo_istoria_fotografia.1.chelovek_risunok_new_year.pic.1}
\end{multicols}

B 1925 году в небольшой деревне Орловской области в семье Зарубиных родился
третий сын. Мальчик рос очень одаренным, и родители в меру сил поощряли его
увлечение рисованием. Так, например, отец подсказал Володе начать собирать
собственную коллекцию открыток. В те годы получить по почте от родных красивую
картинку с небольшим письмом было настоящей радостью. Именно это счастье,
связанное с почтальоном и вестью от далеких друзей, художник сумел сохранить в
памяти и затем воплотить в собственных рисунках. Коллекция у маленького Вовы
собралась, кстати, очень солидная – около пяти тысяч разноцветных карточек.
Такая не у каждого мальчишки была! 

\begin{multicols}{2}
\ii{21_11_2021.fb.fb_group.literatura_isskustvo_istoria_fotografia.1.chelovek_risunok_new_year.pic.2}
\end{multicols}

Bo время войны семью раскидало по свету. Старшие сыновья ушли на фронт, а
младший попал в оккупацию и был вместе с другими односельчанами отправлен в
Германию. Работал на заводе, несколько раз чуть было не попал под расстрел, но
выжил и сумел после победы благополучно вернуться домой. Правда, в родной
деревне он уже не остался. Юношу забрали в армию, а затем он осел в Москве,
пошел работать на завод, учился в вечерней школе. 

Вместе c огромной армией детей, переживших страшные годы, Владимир Зарубин
сумел догнать и получить то, что отняла у него война – часть жизни, старшие
классы школы, студенческие годы. Ему удалось поступить на курсы
мультипликаторов, и много лет затем талантливый художник трудился на студии
«Союзмультфильм». Глядя на его открытки, мало кто догадывался, что этот же
художник был автором образов из сотни любимых советских мультфильмов: «Маугли»,
«Ну, погоди!», «По следам бременских музыкантов», «Раз — горох, два — горох»,
«Тайна третьей планеты», «Жил-был пёс» и множества других. 

Открытки он начал рисовать в 1962-м году. Эпоха соцреализма была очень строга к
любому виду творчества, а тем более – к тому, которое «шло в массы», поэтому
каждую новую картинку должен был одобрить худсовет. Первые образцы ежиков и
зайчиков ставили членов комиссии в тупик: что это – новое слово в советском
искусстве или образчик капиталистического упадничества? 

От многих идей приходилось отказываться, но художник продолжал рисовать в своем
стиле, и скоро миллионы простых людей проголосовали за него, выбирая на
прилавках киосков не смелых пионеров, бодро шагающих под знаменами в светлое
будущее, а мишек на санках, снеговиков, наряжающих елку, и зайчат с цветами,
спешащих поздравить кого-то в сказочном лесу с днем рождения. Так открытки
Владимира Зарубина стали неотъемлемой частью советского быта. Мало кто знал имя
художника, однако все пытались перерисовать его милых зверушек. 

Для художника, рисующего открытки, Владимир Зарубин был достаточно известен. У
него скоро появились поклонники, которые писали мастеру. Современники
вспоминают, что он всегда отвечал на эти письма. Характер этого человека был,
наверное, с первого взгляда виден в его работах: искренний, открытый, очень
добрый – именно таким он был и в жизни, поэтому поклонники его творчества,
получая в ответ письма, полные тепла, не были разочарованы в своем кумире. 

%\ii{21_11_2021.fb.fb_group.literatura_isskustvo_istoria_fotografia.1.chelovek_risunok_new_year.pic.3}
%\ii{21_11_2021.fb.fb_group.literatura_isskustvo_istoria_fotografia.1.chelovek_risunok_new_year.pic.4}

\ifcmt

tab_begin cols=2,no_fig,center

  ig https://scontent-lga3-1.xx.fbcdn.net/v/t39.30808-6/257771821_457091069368529_5529994046319200996_n.jpg?_nc_cat=1&ccb=1-5&_nc_sid=b9115d&_nc_ohc=I3rULyZWGVIAX_CQl1n&_nc_ht=scontent-lga3-1.xx&oh=8d6363b0505ebafd0b2b7bcaa8db8474&oe=61A07873

  ig https://scontent-lga3-1.xx.fbcdn.net/v/t39.30808-6/259702331_457091096035193_3352430136508190144_n.jpg?_nc_cat=1&ccb=1-5&_nc_sid=b9115d&_nc_ohc=a-rsVS8ZeT0AX-Ov5oN&_nc_ht=scontent-lga3-1.xx&oh=aa3e433da95329f9ff73c8e5a54da43d&oe=61A10F55

tab_end
\fi

К сожалению, перестройка выбила художника из колеи. В 90-х годах ему шел уже
седьмой десяток, а в этом возрасте подстраиваться под мир, который рушится на
глазах, очень сложно. Открытки катастрофически теряли актуальность, казалось,
что почтовые перевозки вообще скоро канут в Лету, поэтому художнику пришлось
менять специфику работы. Чтобы выжить, он вынужден был бегать по маленьким
издательствам, пытался получить хоть какие-то деньги за свою работу, но это
выходило все хуже. 

\begin{multicols}{2}
\ii{21_11_2021.fb.fb_group.literatura_isskustvo_istoria_fotografia.1.chelovek_risunok_new_year.pic.5}
\end{multicols}

Однако работать он не переставал, до последних дней из-под его кисти выходили
такие милые и знакомые зверушки, которые вдруг перестали быть нужными. Однако
силы человека не безграничны. После очередного телефонного звонка из
разорившегося издательства, получив известие, что денег за работу последних
недель он не получит, Владимир Зарубин слег с тяжелейшим сердечным приступом.
Он умер от инфаркта, и сын, который был с ним рядом, ничем не смог помочь
70-летнему отцу, а скорая помощь, к сожалению, опоздала. Несмотря на то, что в
период с 60-х по 90-е годы было выпущено огромное количество — более 1,5
миллиарда открыток с рисунками Владимира Зарубина, сегодня они ценятся у
коллекционеров. Некоторые считаются раритетами и стоят очень дорого. В
филокартии даже существует самостоятельное направление — коллекционирование
открыток Владимира Зарубина. 

Кстати, если хорошо поискать, то наверняка у каждого, рожденного в CCCP,
найдется где-нибудь в стопке старых открыток или в альбоме образец творчества
этого прекрасного художника. Его работы настолько узнаваемы, что подписи не
требуют.©️

\#Чеширский\_кот
