% vim: keymap=russian-jcukenwin
%%beginhead 
 
%%file 05_12_2020.sites.ru.zen_yandex.yz.netrivialnaja_istoria.1.jazyk_xviii_vek
%%parent 05_12_2020
 
%%url https://zen.yandex.ru/media/id/5f6a1d4eaddbe610281941b4/imel-ia-paki-pereturku-velikii-moguchii-iazyk-v-xviii-veke-5fcb60c17e300d7cca870671
 
%%author 
%%author_id yz.netrivialnaja_istoria
%%author_url 
 
%%tags 
%%title «Имел я паки перетурку». Великий, могучий язык в XVIII веке
 
%%endhead 
 
\subsection{«Имел я паки перетурку». Великий, могучий язык в XVIII веке}
\label{sec:05_12_2020.sites.ru.zen_yandex.yz.netrivialnaja_istoria.1.jazyk_xviii_vek}
\Purl{https://zen.yandex.ru/media/id/5f6a1d4eaddbe610281941b4/imel-ia-paki-pereturku-velikii-moguchii-iazyk-v-xviii-veke-5fcb60c17e300d7cca870671}
\ifcmt
	author_begin
   author_id yz.netrivialnaja_istoria
	author_end
\fi

\index[rus]{Русский язык!Великий, могучий язык в XVIII веке, 05.12.2020}

В России XVIII век – это один большой переворот. Нет, революция! Великая
языковая революция. То, что приписывают Пушкину, сделали, в основном, его
дедушки и прадедушки. Взгляните на один и тот же текст в двух переводах!

\begin{leftbar}
  \begingroup
    \em\Large\bfseries\color{blue}
«Неприятели же уже к стенам приступать стали, и тогда граждане с молением упрашивали его, чтоб помогл, и как едва уговорила его жена, он вышел и Тестиевых детей осталых побил, и сам бияся умер; по смерти же Мелеагровой Алтея и Клеопатра сами обвесилися, а жены, кои плакали по мертвом, преложилися в птицы»
  \endgroup
\end{leftbar}
Это 1725 год.

\begin{leftbar}
  \begingroup
    \em\Large\bfseries\color{blue}
«Когда же неприятели подошли почти к самым стенам, то граждане начали Мелеагра
с покорностью просить о вспоможении, и он, склонившись наконец с трудом
на прозьбу своей супруги, вышел противу их: и убивши прочих Тестиевых
сынов, умер в сражении. По смерти Мелеагровой Алтея и Клеопатра сами
себя повесили. А те женщины, которые оплакивали его кончину, претворены
были в птиц».
  \endgroup
\end{leftbar}

А это – 1787-й!

Между двумя отрывками – только 60 лет, однако кажется, что изменений в языке
случилось больше, чем за последующие 230. Особенно в синтаксисе. Время летело
вскачь! Русский литературный язык создавался в эти годы и еще чуть позже.

Андрей Болотов – о его великолепных мемуарах я говорил уже в нескольких статьях
– работал в конце XVIII века. Он уже, казалось бы, «на нашей стороне» великого
разрыва. И все же в его речи много архаизмов и разного филологического
гурманства.

В общем, читать – одно удовольствие!

\begin{leftbar}
  \begingroup
		\em\Large\bfseries\color{blue} 
				«Не успело двух недель пройтить после святой
				недели и весна только что вскрыться, как получен был в полк секретный
				ордер, чтоб ему немедленно выступить из своих кантонир-квартир и иттить
				в Ригу, как назначенное для генерального рандеву место».
  \endgroup
\end{leftbar}

Рядом с секретным ордером, кантонир-квартирами и рандеву – «иттить» и
«пройтить». Андрей Тимофеевич всегда так пишет! Однако вместе с этим смешным
«иттить» – «весна вскрывается»… какой точный и прекрасный образ! Сразу видишь
ледоход…

\ifcmt
pic https://avatars.mds.yandex.net/get-zen_doc/1592433/pub_5fcb60c17e300d7cca870671_5fcb6151788eda75c774332d/scale_1200
\fi

Люди в воспоминаниях Болотова постоянно «упражняются». Это означает часто
что-то делать, например «упражняться в разговорах» – часто, помногу
разговаривать. В XVIII веке так все говорили. Если бы слово «упражняться» было
покороче, в русском языке, возможно, появилась бы новая временная форма…

Случается, Андрея Тимофеевича нелегко понять.

\begin{leftbar}
  \begingroup
    \em\Large\bfseries\color{blue}
«Имел я в сию осень паки перетурку».
  \endgroup
\end{leftbar}

В переводе на современный язык – «этой осенью меня опять перегоняли с места на
место».

Другие болотовские пассажи можно понять легко, но неправильно.

\begin{leftbar}
  \begingroup
		\em\Large\bfseries\color{blue} 
				«Обокравшись уже гораздо в канцелярии и имея
				особую в ней комнату, вздумал я однажды испытать, не могу ли я иногда
				между дел в самой канцелярии сколько-нибудь порисоваться»
  \endgroup
\end{leftbar}

Конечно, автор ничего не крал, ни перед кем не рисовался. Болотов хотел
сказать, что он освободился от дел и занялся рисованием.

Или еще один пассаж! На мой взгляд, самый задорный.

\begin{leftbar}
  \begingroup
    \em\Large\bfseries\color{blue}
    «Жена его была старушка самая шлюшечка и человек препростой, но дочь имел он преизрядную и предорогую девицу»
  \endgroup
\end{leftbar}

Ну и семейка… а еще дворяне! На самом деле слово «шлюха» имело тогда смысл
«неряха», а «дорогой» могло значить еще и «замечательный, выдающийся».

\begin{leftbar}
  \begingroup
    \em\Large\bfseries\color{blue}
«Он велел послать к себе своего сына, который несколькими годами был меня
моложе и учился тогда по-немецки и по-французски и был предорогой мальчик»
  \endgroup
\end{leftbar}

Таких слов – ложных друзей переводчика – немало. «Пасть» могло значить
«ловушка», «едкий» – это вкусный, «гонка» – гонение, а «настроение» –
раздражение. «Растрогать» – это разозлить. «Изгваздать» колокольню – просто
нарисовать ее небрежно. Слово «благословение» могло значить и проклятие.

«Танакать» – означало напевать, «тазать» – бранить, «гузать» – «пятиться»,
«жустарить» – жевать, «окончины» – стекла, «шептальцы раков» – раковые клешни.

Вместо «с младых ногтей» в то время говорили «от младых когтей» (коготь – это
ноготь).

Болотов все еще использует слова «взрачный» и «льзя».

Какие-то выражения, кажется, пришли прямо из крестьянской жизни. Вот из
тогдашнего армейского быта:

\begin{leftbar}
  \begingroup
    \em\Large\bfseries\color{blue}
    «Такого покоса трудно было опять дожидаться»
  \endgroup
\end{leftbar}

Покос – это никак не «откос», а шанс продвинуться: удачное время, когда многих
производят в офицерские чины.

И эти старорусские слова смешаны с западными заимствованиями: «шлесарь»
(слесарь), «инвентор» (изобретатель), «термин» (срок), «рантерея»
(казначейство).

Довольно часто Болотов употребляет слово «комиссия». В середине XVIII века это
означало «задание, поручение», но чуть позже приобрело и дополнительный смысл:
«затруднение, хлопоты».

Такая смесь европеизмов с архаизмами делает тексты XVIII века довольно
забавными, будто бы их писал дедушка Козьмы Пруткова.
