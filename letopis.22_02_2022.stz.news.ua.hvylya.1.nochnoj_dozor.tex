% vim: keymap=russian-jcukenwin
%%beginhead 
 
%%file 22_02_2022.stz.news.ua.hvylya.1.nochnoj_dozor
%%parent 22_02_2022
 
%%url https://hvylya.net/analytics/247670-ukraina-eto-nochnoy-dozor-civilizacii-chto-izmenilos-dlya-ukrainy-posle-priznaniya-rossiey-ordlo
 
%%author_id news.ua.hvylya,romanenko_jurij
%%date 
 
%%tags __feb_2022.putin.priznanie,putin_vladimir,ukraina
%%title Украина — это ночной дозор цивилизации: что изменилось для Украины после признания Россией ОРДЛО
 
%%endhead 
 
\subsection{Украина — это ночной дозор цивилизации: что изменилось для Украины после признания Россией ОРДЛО}
\label{sec:22_02_2022.stz.news.ua.hvylya.1.nochnoj_dozor}
 
\Purl{https://hvylya.net/analytics/247670-ukraina-eto-nochnoy-dozor-civilizacii-chto-izmenilos-dlya-ukrainy-posle-priznaniya-rossiey-ordlo}
\ifcmt
 author_begin
   author_id news.ua.hvylya,romanenko_jurij
 author_end
\fi

\begin{zznagolos}
Украине нужно вводить не военное, а осадное положение. Очевидно, для того,
чтобы Украина состоялась, как государство, она должна радикально пересмотреть
отношение к безопасности. Украина должна ощетиниться, как ёж.	
\end{zznagolos}

\ii{22_02_2022.stz.news.ua.hvylya.1.nochnoj_dozor.pic.1}

Украине нужно вводить не военное, а осадное положение. Очевидно, для того,
чтобы Украина состоялась, как государство, она должна радикально пересмотреть
отношение к безопасности. Украина должна ощетиниться, как ёж.

В кратчайшие сроки должна быть проведена реформа безопасности. Проведена чистка
всех правоохранительных органов от российских агентов влияния и резко увеличена
их эффективность.

ВПК должен стать становым хребтом экономики. Армия - становым хребтом
государства.

Иначе государство рухнет из-за того, что люди будут не верить в его
перспективу. Бизнес будет бежать в Польшу и Европу вместе с активной частью
населения, понимая, что Украину в любой момент могут порезать, как салями.

У Украины должна появиться злость в хорошем понимании этого слова. Эта злость
должна покоиться на нашем абсолютном моральном превосходстве, очевидном для
всего мира. Потому что теперь Западу абсолютно ясно, что Украина действительно
фронтир. Фронтир между адекватностью и паранойей.

Необходимо провести быстрые консультации со странами Восточной и Центральной
Европы по поводу взаимопомощи. Очевидно, что все они находятся под угрозой
уничтожения Россией. Особенно страны Балтии. Поэтому необходимо сфокусировать
это понимание на ускоренное перевооружение Украины, используя их возможности и
мотивации.

Стратегический союз с Польшей и Великобританией должен получить
институциональное развитие в кратчайшие сроки. Польша превращается в наш
стратегический тыл, потому отношения с ней требуют соответствующих
договорённостей.

Надо также дать внятный сигнал Турции, потому что заявления Путина несут угрозу
и для неё, поскольку он с таким же энтузиазмом завтра выкатит претензии на
Карс, Ардаган и Эрзурум.

Исторические фантазии Путина превратились в главную угрозу крушения мирового
порядка. Украина должна это эксплуатировать, выстраивая новую систему
экономических отношений. Нам предстоит радикальное изменение энергетического
сектора, поскольку Путин будет зажимать нашу экономику. Потому Украине
необходимо радикально увеличить добычу газа, угля, полностью пересмотреть
импорт нефтепродуктов и ядерного топлива, поскольку угроза над морскими
коммуникациями будет постоянной. Матвиенко вчера четко заявила об ужесточении
политики России в этом направлении.

Поэтому «Велике будівництво» необходимо радикально переформатировать.
Максимально урезать строительство дорог и социальной инфраструктуры и
освободившиеся деньги направить на ВПК, сектор безопасности и энергетический
сектор. Без хороших дорог и ледовых дворцов мы еще как-то протянем, а без ракет
и тепла - нет. Вливание денег в ВПК создаст огромный мультипликационный эффект
для экономики. Нам нужен свой аналог DARPA для развития военных технологий.
Нужно разворачивать на западной Украине большой кластер производства военной
техники, боеприпасов. И переносить туда производство. Весь восток и юг под
постоянной угрозой.

Проще говоря, нам нужна экономика военного времени, которое будет теперь
постоянным.

Также нужно договариваться о реструктуризации долгов с Запада. В идеале он
должен простить их, как Польше в 90-е.

Зеленский также должен радикально пересмотреть внутреннюю экономическую
политику. В условиях мощного давления России нужно облегчить бремя для наших
граждан. Нужно не только отменить РРО ФОПам, но и отменить налоги для ФОПов на
три года. Государство потеряет 50 млрд гривен прямых поступлений, но получит
больще за счет косвенных налогов. Плюс это резко расширит социальную базу
Зеленского, а широкая социальная база принципиально важна для любого
правительства в такие кризисные моменты. Это нужно делать быстро, как и другие
послабления для бизнеса, которые заблокировал Гетманцев. Гетманцева нужно
убрать, как можно быстрее, чтобы бизнес увидел перспективу радикального
улучшения экономической политики. Если такой перспективы не дать прямо сейчас,
то Украина столкнется с чудовищным бегством капитала в условиях перманентной
угрозы уничтожения Путиным. Нужно создать систему мотиваций для бизнеса
оставаться в Украине. Например ввести наконец налог на выведенный капитал.
Самое время.

Короче говоря, угрозы и действия Путина требуют радикальных шагов в экономике.
Иначе экономика рухнет, а с ним и государство, которое не сможет выполнять
базовых функций из-за фискального кризиса.

Мы должны исходить из этого и жить с этим. Угроза со стороны России -фактор
определяющий, долгосрочный и всеохватывающий все уровни жизни. Путин сказал об
этом чётче некуда.

Плюс сегодня стало известно, что Москва признает ЛНР и ДНР в тех границах, в
рамках которых руководство республик осуществляет свои полномочия. То есть
Россия может завести в ЛДНР огромную армию, которая будет действовать от имени
ОРДЛО, если они будут захватывать новые куски территории Украины, то они будут
говорить, что признают ЛДНР в таких границах, которые есть на текущий момент. А
дальше будет все следующий шаг - отхватив новый кусок территории, они назовут
этот кусок Новороссия. Совбез России признает этот кусок точно в такой же
форме, как вчера ОРДЛО. Дальше новоявленная Новороссия предъявит новые
территориальные претензии к Украине и этот сериал будет продолжаться до того
момента, пока или не получат так по зубам, что страшно будет лезть дальше, или
не сожрут всю Украину.

По сути, Россия придумали прокси, чтобы говорить, что "это не мы воюем с
Украиной, а эти якобы "республики". Однако по факту это война между Украиной и
Россией.

Поэтому Украина обречена стать более собранным и жестким государством.
Продление с радикальными изменениями смерти подобно.

Но у меня большой оптимизм по поводу Украины. Путин заставляет нас
объединиться. И мы объединимся. Невозможно принять эту мертвечину. Сегодня
Украина — это ночной дозор цивилизации. Этот высокий статус обязывает нас
покончить с тем, что нас убивает. Как тебе такие мысли, Джон Сноу?
