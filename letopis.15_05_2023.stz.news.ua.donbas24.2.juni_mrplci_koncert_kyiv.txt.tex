% vim: keymap=russian-jcukenwin
%%beginhead 
 
%%file 15_05_2023.stz.news.ua.donbas24.2.juni_mrplci_koncert_kyiv.txt
%%parent 15_05_2023.stz.news.ua.donbas24.2.juni_mrplci_koncert_kyiv
 
%%url 
 
%%author_id 
%%date 
 
%%tags 
%%title 
 
%%endhead 

Ольга Демідко (Маріуполь)
15_05_2023.olga_demidko.donbas24.juni_mrplci_koncert_kyiv
Маріуполь,Україна,Мариуполь,Украина,Mariupol,Ukraine,Concert,Концерт,Дети,Діти,Children,Київ,Киев,Kyiv,Kiev,date.15_05_2023

Юні маріупольці виступили з концертом у Києві (ФОТО)

У столиці діти Маріуполя до Дня сім'ї підготували творчий концерт

13 травня в Києві до Міжнародного дня сім'ї відбувся концерт юних маріупольців.
Діти підготували пісенні, танцювальні і драматичні номери, які не залишили
нікого байдужим і подарували незабутнє свято всім присутнім. В кінці концерту
маріупольці отримали грамоти, подяки і цінні подарунки.

Читайте також: У Києві презентували музичний відеокліп «Кохання сталь» (ВІДЕО)

Заняття з акторської майстерності для діточок і підлітків від 5 до 14 років
проходили в рамках соціальної реабілітації «ЯМаріуполь. Культура». Маленькі та
юні маріупольці створювали сценічні образи, працювали з літературними творами,
тренували пластику тіла і вимову, експериментували та генерували творчі ідеї.

«Майже рік після окупації міста вони продовжували займатися в Києві з
викладачами з міста Марії. І тепер, як справжні артисти, готові показати свою
майстерність», — зазначила директорка Департаменту культурно-громадського
розвитку Маріупольської міської ради Діана Трима.

Юні маріупольці під керівництвом фахівців з «ЯМаріуполь.Культура» підготували
насичену святкову програму, яка приємно вразила всіх присутніх. Номери з
естрадного вокалу запам’яталися своєю емоційною образністю, запальні танці, які
демонстрували маленькі артисти, перенесли всіх у світ ритму та енергії. А
театралізовані етюди, підготовлені молодими талантами, стали справжнім
відкриттям для всіх присутніх.

Читайте також: Коли розпочнуться літні канікули у школах Донеччини — подробиці

Керівники дитячих колективів і одночасно актори Театру авторської п'єси
«Conception» Олександр Пихтін і Вероніка Павлюк дуже задоволені дебютом своїх
вихованців.

«Мені сподобався цей захід. Всі діти дуже кмітливі. Ніхто не забув свій текст,
старалися втілити в життя все, що планувалося, без помилок. Насправді для них
це був перший виступ перед великою аудиторією. Звісно, всі хвилювалися, адже
дітки досить маленькі. Вік акторів від 5 до 12 років. Брав участь і мій
синочок. Я дуже радий, що всі впоралися і залишилися задоволеними від свого
виступу», — зазначив керівник дитячого театру «Тези» Олександр Пихтін.

«Я дуже задоволена виступом свого колективу сучасної хореографії M-dance.
Насправді ми вже досить довгий час готували номери, тому не сильно хвилювались.
Учасниками колективу є діти 7−14 років, але всі вони однаково наполегливі і
старанні. Я дуже задоволена результатом. Все пройшло досить душевно і
зворушливо», — поділилась керівниця колективу сучасної хореографії M-dance
Вероніка Павлюк. 

Читайте також: У Києві презентують проєкт «Портрети Маріуполя» — як потрапити на виставку

Подія зібрала багато маріупольців, для яких після всього пережитого дуже
важливі подібні святкові заходи. Наприкінці заходу всі маленькі та юні артисти
Маріуполя отримали чудові книжкові подарунки «АБЕТКА маленьких переможців»,
виданої за підтримки Міністерства культури та інформаційної політики України,
«Пригоди поні Тоні та його друзів» від Єлизавети Соломатиної, «Казки бабусі
Калинівни» Світлани Луцкової та подарункові книжкові набори «ВАУ» від
видавництва ArtBooks.

Нагадаємо, раніше Донбас24 розповідав, що у столиці захисниця Маріуполя
презентувала нові вірші.

Ще більше новин та найактуальніша інформація про Донецьку та Луганську області
в нашому телеграм-каналі Донбас24.

Фото: з відкритих джерел
