% vim: keymap=russian-jcukenwin
%%beginhead 
 
%%file 27_07_2021.fb.kushnarjev_roman.1.kakzhetak
%%parent 27_07_2021
 
%%url https://www.facebook.com/roman.kushnaryov/posts/4074105805976474
 
%%author Кушнарев, Роман
%%author_id kushnarjev_roman
%%author_url 
 
%%tags grin_grey.gruppa.ukr,jazyk,kultura,mova,muzyka,ukraina,ukrainizacia
%%title Дебільні запитання ПАЧЄМУ і КАҐЖЕТАК у відповідь на прояв української ідентичності
 
%%endhead 
 
\subsection{Дебільні запитання ПАЧЄМУ і КАҐЖЕТАК у відповідь на прояв української ідентичності}
\label{sec:27_07_2021.fb.kushnarjev_roman.1.kakzhetak}
 
\Purl{https://www.facebook.com/roman.kushnaryov/posts/4074105805976474}
\ifcmt
 author_begin
   author_id kushnarjev_roman
 author_end
\fi

Це все мало колись статися. Оці всі горіння срак від того, що в Харкові місцеві
(не приїжджі!) вимагають обслуговування українською. Вибухи застоялого гівна
від того, що припалу пилом музичну группу з двох Г не запросили на концерт на
30-річчя незалежності, бо у них немає пісень українською мовою. Дебільні
запитання ПАЧЄМУ і КАҐЖЕТАК у відповідь на прояв української ідентичності.

Насправді, воно б мало статися ще в 90-ті, але симптоматично, певно, що це все
відбувається під самісінький тридцятий День народження нашої держави.

До біблійних 40 років залишилося всього десятиліття.

\ii{27_07_2021.fb.kushnarjev_roman.1.kakzhetak.cmt}
