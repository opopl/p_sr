% vim: keymap=russian-jcukenwin
%%beginhead 
 
%%file 29_12_2021.stz.news.ua.tyzhden.1.travma_istoria
%%parent 29_12_2021
 
%%url https://tyzhden.ua/Politics/253983
 
%%author_id malko_roman
%%date 
 
%%tags 
%%title Травмовані історією
 
%%endhead 
\subsection{Травмовані історією}
\label{sec:29_12_2021.stz.news.ua.tyzhden.1.travma_istoria}

\Purl{https://tyzhden.ua/Politics/253983}
\ifcmt
 author_begin
   author_id malko_roman
 author_end
\fi

Матеріал друкованого видання № 51 (735) від 22 грудня.

Чому легковажність перетворилася на фактор національної політики

Недовіра суспільства до влади стрімко зростає. Провідні соціологічні агенції
фіксують падіння майже синхронно. І хоча вітчизняна соціологія нерідко буває
доволі маніпулятивною, приховувати очевидне стає дедалі важче. Суспільство
розчароване у своєму виборі дворічної давності та, схоже, серйозно
дезорієнтоване.

2021 рік став переломним для української влади з кількох причин. Ресурс
міцності, накопичений попередниками, нарешті почав вичерпуватися. Систему
покинуло чимало професіоналів, які попри певні корупційні грішки таки
забезпечували її безперебійне функціонування. На їхнє місце призначили
цілковитих випадковостей на зразок заступника міністра МВС Олександра
Гогілашвілі, й усе почало валитися. Помножимо тотальний непрофесіоналізм у всіх
сферах на бажання швидко озолотитися (поки не прогнали) — й матимемо відповіді
на запитання, чому все так погано та чому рівень довіри до влади й далі падає,
а громадян більше не тішать відосики та не подають надій численні обіцянки й
маніпуляції. Спроба відмотати ситуацію назад за допомогою дешевих трюків на
кшталт показових розправ над опонентами вже нічого не змінить. І тут роль
головного запобіжника відіграє саме розчарування й образа. Тепер, за іронією
долі, ті шалені 73\%, на яких Володимир Зеленський в’їхав колись у владу й
завіз за собою цілу ватагу авантюристів, битимуть його нещадно.

Біда лише в тім, що це блискуче падіння жодним чином не трансформується в чиєсь
зростання. На тлі розчарування владою не збільшується довіра до її конкурентів,
яких, здається, сьогодні вдосталь на будь-який смак. Та й місце месії знову
вакантне, і потреб суспільства, які підштовхнули його у 2019-му дати прочухана
старим елітам і привести до влади нікому не відомих авантюристів, так і не було
задоволено. Навпаки, стара система, яка з горем пополам функціонувала й давала
змогу хоч якось у ній жити, нині розвалюється остаточно. Або, як твердить
опальний Андрій Богдан, на чиїй совісті якраз і є тріумф Зеленського, країна
наближається до прірви — й далі лише катастрофа. Чи то суспільство так
втомилося від постійних обломів і йому тепер байдуже, що насправді буде, чи
справді не бачить, хто міг би йому допомогти виборсатися з ями, але живих
ознак, що вказували б, у який бік може хитнутися маятник, наразі не видно. Це
при тому, що й охочих приміряти на себе тогу рятівника серед конкурентів
Зеленського чимало й вони невтомно себе рекламують. Дехто роками та
десятиліттями наближався до цієї мети, не гребуючи нічим. Щоправда, чи
відважаться вони саме сьогодні кинутися в останній бій за свою мрію, ще
питання. Надто непроста заварилася в країні каша.

Те, що так буде, якщо українці легковажно відважаться обрати собі президентом
коміка — людину, яка нічогісінько не розуміє в функціонуванні державних
механізмів і вміє лише клеїти дурня на сцені, Тиждень з номера в номер
попереджав ще задовго до перемоги Володимира Зеленського. Адже такий розвиток
подій цілком закономірний, і не треба бути Нострадамусом, щоб його передбачити.
Упродовж довгої історії людства схожі нещастя траплявся неодноразово. Тому цей
сюр, який українці нині мають нагоду спостерігати на владних пагорбах, у
телеефірах і навіть починають відчувати на власній шкурі, шлунках і гаманцях, є
лише наслідком їхнього безвідповідального вибору кількарічної давності. Така
собі, кажучи мовою військових, «отвєтка». Їхнє бажання радикальних змін, нових
гострих відчуттів і казкових перевтілень, поєднане з небажанням думати й
передбачати можливі наслідки, призвели до класичного «маємо те, що маємо». І
проблема навіть не в тому, що цей черговий експеримент над собою вони можуть не
пережити, бо ситуація справді критична. Багато громадян досі не готові визнати
свою вину й воліють просто абстрагуватися. Щоправда, наступного разу ці люди
знову прийдуть на виборчі дільниці та проголосують — хто по-приколу, хто на
приколі, а хтось, щоб комусь насолити. Такий спосіб поведінки цілком відповідає
стилю їхнього життя, й у цьому, здається, наша найбільша трагедія.

\begin{zznagolos}
ІНОДІ ВИДАЄТЬСЯ, ЩО УКРАЇНСЬКЕ СУСПІЛЬСТВО — ЦЕ ВЗАГАЛІ СУЦІЛЬНА
БЕЗВІДПОВІДАЛЬНІСТЬ, МІНЛИВІСТЬ І ЛЕГКОВАЖНІСТЬ. АЛЕ ЦЕ НАСЛІДКИ ТИСЯЧ ДРІБНИХ
І ВЕЛИКИХ ТРАВМ, ЯКІ ДОВЕЛОСЯ ПЕРЕЖИТИ УКРАЇНЦЯМ ЗА ОСТАННЮ СОТНЮ РОКІВ, КОЛИ
ЖОРНА ІСТОРІЇ МОЛОЛИ ВСІХ БЕЗ РОЗБОРУ, А ЦІННІСТЬ ЛЮДИНИ ЗВОДИЛАСЬ ДО НУЛЯ	
\end{zznagolos}

Адже безвідповідальність в українців тотальна, до того ж спостерігається
повсюдно на всіх рівнях. Спортсмени й артисти впевнені, що мають право бути
поза політикою, коли їхня країна заливається кров’ю. Бізнесмени — що мають
право торгувати з окупантом, адже це лише бізнес (більшість будівельних
матеріалів та побутової хімії в Україну завозять з країни агресора). Чиновники
крадуть гроші з будівництва доріг, а потім розбивають на них свої машини.
Лікарі дають клятву Гіппократа, а згодом продають довідки про вакцинацію та
ставлять діагнози залежно від настрою. Поліцейські шиють справи невинним, щоб
отримати премію, та «кришують» продаж наркотиків. Ласі до життя в кредит
громадяни роблять заручниками своїх рідних і друзів. Рекламодавці та продавці
постійно брешуть, щоб розкрутити клієнта на гроші, а будівельники халтурять і
роблять усе те саме, що й згадані вище персонажі. Навіть батьки народжують
дітей і потім не займаються їхнім вихованням, бо не мають часу, не вміють чи не
хочуть. Et cetera, et cetera, et cetera…

На цьому тлі ставлення до політичних питань і власного вибору не є чимось
особливим та унікальним. Так само легковажно ми часто ставимося до
навколишнього простору, у якому живемо (загиджені під’їзди, понищені ліфти), до
природи, дотримання правил руху на дорозі та вживання алкоголю за кермом, до
власного харчування, здоров’я, репутації і свого майбутнього. Навіть до
накопичення грошей і цінностей. Іноді видається, що українське суспільство — це
взагалі суцільна безвідповідальність, мінливість і легковажність, до того ж
боротися з цим явищем немає жодного сенсу. Проте не боротися також не вдасться.

Корені проблеми тягнуться в глибини підсвідомості. Це наслідки тисяч дрібних і
великих травм, які довелося пережити українцям за останню сотню років, коли
жорна історії мололи всіх без розбору. Коли цінність людини, хоч би ким вона
була і якими статками володіла, зводилась до нуля. І вимірювалась лише
здатністю вижити. За умов, у яких довелося жити більшості наших співгромадян —
батькам, матерям дідам та бабуям, — відповідальність не мала жодного
прикладного значення. Вона аж ніяк не сприяла виживанню, а навпаки, могла
зашкодити. Радянська людина, яка відповідально поставилася б до свого вибору
під час голосування, за законом жанру мала б стати дисидентом. Відповідальний
колгоспник, який відмовив би собі в «задоволенні» тягти додому силос чи зерно,
не зміг би вигодувати порося й курей, а отже, не мав би чим годувати дітей. А
його колега робітник якби не вкрав трохи цвяхів і кілька дощок, нізащо не зміг
би облаштувати своє скромне помешкання, адже купити цю розкіш за совка було
проблемно. Планувати майбутнє, накопичувати статки, зводити будинки чи збирати
гроші, як і берегти здоров’я та відповідально працювати, для радянської людини
не мало сенсу. У кращому разі в майбутньому на неї чекав комунізм, де й так усе
буде, як у казці, до того ж задарма. Щоправда, до цього ще треба дожити.
Будь-якої миті можна було позбутися всього нажитого непосильною працею, тому
єдине, що залишалося, — жити не озираючись і не загадуючи. Адже ніхто не
гарантував, що завтра тебе не вивезуть у Сибір, не заморять голодом чи не
відправлять гинути в Афганістані. Якщо ти раптом виявишся потрібним державі чи
заводу, тебе й так якось урятують, а якщо ні, то (цитуючи бородатий радянський
анекдот) — кому таке життя треба.

Очевидно, що для вільної людини, громадянина вільної держави, таке ставлення до
життя та цінностей неприпустиме. Але нас, українців, зрозуміти можна й навіть
треба — щоправда, це не привід розслаблятися й жаліти себе. Рубці на генах досі
сильні, й затерти їх буде дуже непросто. Тим паче вражені всі, хто більше чи
менше мав нещастя особисто або за посередництва своїх предків пожити в
проклятому совку. І всім нам, звісно, потрібна серйозна реабілітація. Проте до
повного оздоровлення ніколи не дійде, якщо пацієнт сам того не захоче, не
усвідомить власної проблеми та не допоможе сам собі її подолати. Урешті, час і
саме життя також у певному сенсі лікують. Особливо якщо карколомні перевтілення
долі твоєї країни стосуються також тебе безпосередньо, відбуваються на твоїх
очах і, можливо, навіть із твоєї вини. Вибір 2019 року в перспективі не лише
призведе до катастрофи (якщо раптом до цього дійде), а й, безумовно, стане
важливим уроком. Можливо, уже став. І ймовірно, що саме завдяки йому українці
сьогодні вже не такі відважні у вияві своїх дивних симпатій. Обережніше
ставляться до політиків, які намагаються маніпулювати, спостерігають за ними на
відстані й вичікують. Можливо, якраз стрімке піке поверне їх до тями (якщо не
вб’є) та змусить тверезо дивитися на своє місце в історії, ретельніше
аналізувати події та слова, менше вірити в казки й відповідальніше ставитися до
свого вибору. Очевидно, що це не трапиться миттєво, але критичні ситуації
нерідко пришвидшують процеси, які за спокійних обставин могли визрівати роками.
І сьогодні, схоже, саме такий момент — пан або пропав. 
