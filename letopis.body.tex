% vim: keymap=russian-jcukenwin
%%beginhead 
 
%%file body
%%parent 

%%endhead 

\ii{articles}

\ii{kiev}
\ii{wikipedia}
\ii{anekdoty}
\ii{letters}

% расшифровки
\ii{transcripts}

% мое
\ii{moje}

% build
\ii{build}

% переводы
\ii{translations}

% литература
\ii{literatura}

\ii{list}

% города
\ii{goroda}

\ii{stihi}

\ii{citations}
\ii{dvk}
\ii{jumor}
\ii{nenavyst}
\ii{people}
\ii{poetry}
\ii{poslovicy}
\ii{proza}
\ii{saint_russia}
\ii{slovar}
\ii{slova}
\ii{songs}
\ii{topics}
\ii{writers}

\ii{stranaua}
\ii{todo}
\ii{test}

% https://www.facebook.com/marina.sidoniuk/posts/pfbid02jESojKV9QWTtw4pgqTJCnFCkaTU8hcF3rtp6gNxCKT82ifsZYUGERiVvqdbWRz3Bl
% https://www.facebook.com/MAlina.Mykhailova/posts/pfbid0rcSWyZYyHDbptqz2uVJAg1X1nGwUDwP49GVcwqhoj6o3NtXXSsFhwcyQP3ywyiBpl
% https://www.facebook.com/sergey.evsiukov/posts/pfbid02vWiv7W6wGTEGw7LuUBQbANm2MNehpr3xwTzeeaxoKWRAMCqFgWX8ZAaLux7KifRul
% https://www.facebook.com/lepska.kepska/posts/pfbid02vBwThsvoRsEBSUodxGaqd2D6Toa5yxxKMQCB9SkwQw39aYFi9Qm3bJirRwgbQeVXl
% https://www.facebook.com/serjvlk/posts/pfbid02FJyCzRoQHxZpmnbMq2DKsoB5modrg29UayX2cXmLVWQzMfHmP17j7ENyMNnNunhkl
% https://www.facebook.com/havryshko/posts/pfbid0NoxJjeAoajoG6kLKxhAwiPVjWWkRxR1aqb4KD7bTLwthJFXFYBMGEWXd3EE1s43Ul
% https://www.facebook.com/marjana.savka/posts/pfbid02FP1ftsHU17pBXneYntc8sWZP7YzPh2o5wFX1g7Yb4MtHwfWDyqu51vvQ4wbcHh5Jl
% https://www.facebook.com/victor.taran.5/posts/pfbid0Am1dxk4hDUXsdSRboXVoZu2riPzNuFK5vy5sAJovBsWtJaRZC4Ljcm1YzoxzqYLTl
% https://www.facebook.com/vitalii.chepynoga/posts/pfbid02Wg8EyuMfJjB819D8Y7wtYr3MXYn5cF25597dCsJ9vb3QDpuhDyqZam44XbfDNfTLl

% Привіт. Шо там, як ти там, Боже?
% https://www.facebook.com/LiudmylaHorova/posts/pfbid02jSTh1nhrj4sU1CLfz4JhT8Lqra44oS4t48SuFH6mNMvew9HhASFbCwEuZ5pA56pJl

% http://www.barabooka.com.ua/ilyustruyu-knizhki-dlya-ditej-svitlana-baluh

% https://www.facebook.com/dubinskaya999/posts/pfbid0V4KHVWDcwx7Fn8JG6ePxSHvukgRJLARr3ENQe5AmjFAYa6jeGcdYmXMtqJmKeTJol
% https://www.facebook.com/permalink.php?story_fbid=pfbid0LMDg2AZyEuwZcxccC1egBaHUA4UzVnGmHhBpgN7zajJTv7JzhaedWF7vTL3L9uYql&id=100009901380868
% https://www.facebook.com/vladchu/posts/pfbid02EFNpq4gv3tffD6SprTAM8DkB4ohD4ndyCeLczG1LWwYxCXrEHQVuv9Ynn4wen1Dfl

% https://www.facebook.com/groups/853117518526778/posts/926678901170639
% https://www.facebook.com/dmitro.raimov/posts/pfbid0Vsc4YxwLTEYKNDykfp6TjzU71uQt18Aorsi2ux54zf8X21gpjdbivUhdaw6ECoiql

% https://www.facebook.com/AnastasiyaPika/videos/2608610159262568/

%(зараннє вибачайте, нижче речі, за які мене можуть зараз закидати тапками...
%але... ) Ваша Книжка... Ви кажете, чи загине вона чи ні... Я розумію, Ви
%дуже-дуже хвилюєтесь, випускаючи своє творіння у великий світ... Книжка Ваша
%буде жити довго-довго, через покоління ваше Слово буде говорити нашим вже
%онукам і праонукам нащадкам, що і як було... А щодо ваших сумнівів, чи загине
%книга чи ні... Вбити книгу важко... вбити книгу - якщо вже так говорити...
%можуть такі речі - (1) байдужість людська (2) як не дивно, страх її читати. Ось
%(2) мене більш бентежить... оскільки мені дуже сумно бачити коментарі на кшталт
%- моє серце розірветься, якщо я почну читати, або ж.. я куплю, але не знаю, чи
%вистачить сил її читати. Люди добрі, вибачте... але... ми всі так хочемо
%Перемоги... всім серцем хочемо, всі про це кажуть. Але. Перемогу насправді
%здобувають сильні духом. Перемогу нам ніхто на блюдєчке с голубой каємочкой не
%принесе, її треба виборювати важкою працею, кожен день, кожень день... І
%зокрема через духовний розвиток, через переборювання самого себе... Так!
%Перемогу здобувають сильні духом люди, що здатні переборювати свої страхи,
%болі, переборювати в собі відчай, безнадію. Так, сильні духом здобувають
%перемогу! І пані Надія найкращий приклад, оскільки вона дійсно показала всім
%нам приклад, як переборювати свій біль і відчай! Вона пройшла через пекло, як і
%багато інших людей, але ж сил в неї вистачило книжку написати, так! І сил, як
%не дивно, вистачає на багато чого! А щодо страхів... тут лише питання - взяти і
%прочитати книжку, яку ви ж самі і замовили... Якщо вже замовили, то нащо оце
%бідкання, що я не знаю, чи зможу її читати? Якщо вже сказав А... то кажи і Б...
%Може тоді краще було її не замовляти, так, якщо не будете читати? Або ж що,
%книжка прийде і ви її на поличку поставите, чи як? Якщо вже немає сил у вас
%просто взяти і прочитати книжку,  навіть якщо там є важкі речі, то у мене
%просте питання, якщо ви боїтесь взяти в руки всього лиш - книжку - це книжка,
%не ракета і не бомба...   і прочитати, то... як ви збираєтесь перемагати... а
%ви взагалі Перемогу хочете чи ні? 

%Доброго дня! Дуже добре, що Ваша книжка виходить. Це велика подія, особливо з
%тої точки зору, що в Києві, якщо походиш по книгарням - то знайдеш одну-дві
%книги про Маріуполь - і усе... Тому... потрібні книжки про Маріуполь. Так, дуже
%потрібні! Багато-багато книжок. І не лише про жахіття, а про Мирне Місто
%Маріуполь. Про Майбутнє Маріуполя, про Віру в Маріуполь. Багато-багато книжок
%треба! Місту Маріуполю мало однієї книжки чи декількох - як би ці декілька
%книжок не були сильними і правдивими, Місту Маріуполю потрібні сотні, тисячі
%книжок... Так, сотні, тисячі різноманітних книжок... Щоби не провалитись в
%чорну безодню забуття...  І... Книжка називається Надія. Але є ще Віра та
%Любов. Так, Трійця - Надія, Віра, Любов. Тому... побажання 
%(і не тільки Вам, а взагалі всім маріупольцям - 
%бо хто ж краще за маріупольців зможе написати про Маріуполь?). Пишіть нові
%книжки... Не зупиняйтесь!  І... я вчора ходив по центру, Михайлівська площа. Як
%Ви думаєте, що я там побачив... Я недавно писав про фільм Місто 44. До речі...
%Якщо так подумати... 2022 = 2*22 = 44... Так от. Там я... несподівано! Варшаву
%побачив... Так, колись вщент зруйноване місто, а зараз мирне європейське місто
%стоїть поруч з Маріуполем на площі, обіймає його, шле Маріуполю привіт, і
%немовби каже - все буде добре, друже Маріуполь. Мене знищували так само, як і
%тебе. Але я відродилась. Я колись пережила жахіття такого ж масштабу і горя,
%але я відродилась з попелу, і там, де колись палало полум'я і горіли будинки,
%зараз весело граються діти, а замість зруйнованих мостів через Віслу - стоять
%нові, гарні мости... Так, все буде добре, друже Маріуполь, каже тобі Варшава!

%Доброго дня. Щодо російської мови... Я так скажу. Ви напевне знаєте, що зараз
%річниця Майдану, річниця розстрілу Небесної Сотні, річниця Революції
%Гідності... Україна вже назавжди змінилась завдяки Майдану... А я Вам нагадаю,
%з чого почався Майдан. А Майдан, власне кажучи, почався з короткого допису
%російською мовою (!) Мустафи Найєма у фейсбуці - 21.11.2013 - Встречаемся в
%22:30 под монументом Независимости. Одевайтесь тепло, берите зонтики, чай,
%кофе, хорошее настроение и друзей. Перепост всячески приветствуется! Ось такий
%лаконічний запис російською мовою. З нього все і почалось... і небайдужі, так -
%найкращі! люди все дуже добре зрозуміли, що треба робити, одяглись тепло, взяли
%парасольки, каву, хороший настрій і друзів, і все почалось! Як не крути. Ось
%так!

%Ви знаєте... в чомусь Ви праві... а в чомусь я посперечаюсь, щодо доцільності
%витрачання часу. Персонально щодо цією особи я вже все сказав, вона сама завела
%себе в глухий кут, так що дійсно немає сенсу далі з нею спілкуватись. Але...
%Розумієте. Так, таких людей багато... І... подібних людей ще більше в росії...
%мільйони Це поле бою, інформаційний фронт. Це дуже важливий фронт протистояння,
%і тут головна зброя - Слово, Розум. Треба розвитати ці навики, я так думаю, 

%Будь-ласка... Знаєте, це дуже обширна, велика тема насправді. Чи варто
%сперечатись чи ні. Тут ми згадали про Грецію. Маріуполь - грецьке місто...
%Поліс... Знаєте, колись я був у Херсонесі... Дуже-дуже особливе місце! Місце
%сходження цивілізацій та культур, місце - з якого християнство було прищеплено
%на Русь, на Київську Русь... Бо ж Володимир, перш ніж похрестити киян,
%хрестився сам в Херсонесі... Так от. Греція - колиска... колиска Культури,
%Науки, Літератури... У школі вивчають теорему Піфагора, п'ять постулатів
%Евкліда. Дивишся на небо або ідеш у Київський планетарій, бачиш... Як у відомій
%пісні, небо незміряне всипано зорями! Що то за божа краса! Зорі... Сузір'я...
%Велика Медведиця, Мала Медведиця... Давньогрецькі міфи... Все те, що ми маємо
%зараз - технології, наука - колись почало розвиватись у Греції... Греки
%дивились на небо... думали... а чому так, а не інакше? Чому існує щось, а не
%ніщо? Чому? Чому? І... як? Як зробити так, щоби було справедливе суспільство,
%що краще - аристократія чи демократія або ж тиранія? І так далі... і так далі..
%Я до чого. У греків була школа риторики, діалогу (Сократ) - вміння спокійно,
%обгрунтовано перемагати у суперечках, розплутувати щось. Так от. Тут проблема
%не конкретно в цій пані - вона, може, персонально дуже навіть непогана людина,
%а в тому, що часто люди собі вбивають в голову різні ментальні конструкції.
%Вбивають і вірять в це. Конструкція може бути хибна, але люди вперто вірять.
%Тут вже питання до того, як мозок працює на глибинному рівні, чому це так
%стається. І тут потрібні Розум і Слово, так, сильне, влучне Слово, щоби з цим
%боротись. Але звісно, важливо знати, коли і як писати, і коли варто
%зупинитись... А в більш глобальному плані - у нас зараз іде Війна з росією не
%тільки на військовому фронті, а також Війна Ментальна, Ідеологічна,
%Інформаційна, Когнітивна. Глобальна Когнітивна та Інформаційна Війна... І
%персонально в мене є певний досвід систематичного вивчення явища російських
%зомбі (у соцмережі однокласники)... Росіяни дійсно перетворились на стадо диких
%агресивних зомбі... Ох! Ризикую забити все тут своїми думками, вибачте! Хотів
%до речі тут спитати, чи знаєте ви Почепцова Георгія? Він спеціаліст саме по цій
%темі. Я деякий час тому купив його книжки, і він викладав в Маріуполі.
%Наскільки я знаю, він живий,  слава Богу, продовжує займатись тематикою
%інформаційних війн. Важлива тема, у мене тут є багато чим поділитись, але вже
%пізніше якось... Вибачте за стільки слів.

%Дякую! Тоді... ще трохи про Херсонес і Маріуполь... В Херсонесі, місті-державі,
%люди уявляли себе передовсім як Громадяни Херсонесу. Ось навіть клятва у них
%була! Не греки, а перш за все як Громадяни Херсонесу. Так само, як у Київській
%Русі були Кияни, Новгородці, Чернігівці, Галичани (жителі древнього Галича),
%інших міст... Кияни або ж чернігівці перш за все думали про себе як Киян або
%жителів Чернігова, а вже потім як свою приналежність до Русі... Київська Русь -
%скандинави називали її Гардаріка - Країна Міст, Країна Городів! І в цьому
%сенсі... ви писали... ви як маріупольці зараз у розпачі, так, Маріуполь
%зруйновано, вас розкидало по всьому світі... жахливий, трагічний рік! Але...
%історія справжнього, живого, Українського, Маріуполя не закінчилась в 2022
%році. Хоча... зараз на даний момент він залишився у вашій пам'яті, щасливий,
%радісний Український Маріуполь!  І... було питання у Вашому пості. Хто ж ви є?
%А я так думаю. Українці і Громадяни Маріуполя. Так, і Просто і Гордо, Знаєте.
%Незважаючи ні на що! Громадяни Міста Маріуполя, прекрасного найгарнішого Міста
%у Моря! Міста Трамваїв та парка Веселка, Міста Азовсталі та Героїв Полку Азов,
%Міста Щасливого та Трагічного... от колись Олександр Олесь вірша такого
%написав... З Журбою Радість обнялась... а далі ви самі вже продовжите, як
%забажаєте... А щодо Києва... я от киянин. Київ, як і будь яке місто, то живе не
%тільки у фізичному просторі, а у духовному... Півтора тисячі років історії,
%бездонний колодязь історії, з якого можна пити і пити... і який ніколи не
%висохне... і мені от про Київ напевне найбільш подобаються такі рядки із романа
%Павла Загребельного Смерть у Києві ... Київ був повен лагідного Сяйва. Воно
%лилося згори, із спокійного осіннього Неба, високо знесеного над Київськими
%Горами, знизу піднімалося йому навстріч Сяйво зелене, а між зеленим і
%високоголубим тихо плавало Золото Соборів, легко лягала поміж ними перша
%пожовклість листя, і неначе в душу вхлюпувалося оте м'яке світіння, і
%відчувалося, що, входячи в цей Город, мовби стаєш безсмертним... Розумієте...
%Місто, в якому стаєш безсмертним... Немає на Землі іншого Києва, немає більше
%ніде такого Міста! А яка фраза про Місто Маріуполь Вам найбільше до душі?

%о, це знову ти, Дурник! Те, що я написав вище  - це, зокрема, слова з творчості
%(1) Тараса Шевченка, нашого Кобзаря, у нього до речі, скоро День Народження!
%(2) Пісня нашого гурту Океану Ельзи (3) Вічна творчість Григорія Сковороди,
%нашого Філософа. Видно ти в школі замість уроків робив щось інше, не те, що
%треба зазвичай робити на уроках (а я нагадаю, на уроках треба вчитись, а не
%лізти до дівчат за сусідньою партою), то й не знаєш. А дарма не знаєш! Хіба ж
%не про таких як ти Тарас писав - а то залізете на небо, і ви не ви, і я - не я?
%Так що... те що ти просто дурник, мені вже зрозуміло досить добре ))) Просто
%Звичайнісінький Дурник Дурненко... так і буду тебе називати - Мессір Дурник.
%Дурник-Шкурник ))) так що йди  назад в школу вчитись, Дурник!

%https://www.facebook.com/events/719929419509214/?post_id=720187276150095&view=permalink
