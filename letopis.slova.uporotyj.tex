% vim: keymap=russian-jcukenwin
%%beginhead 
 
%%file slova.uporotyj
%%parent slova
 
%%url 
 
%%author_id 
%%date 
 
%%tags 
%%title 
 
%%endhead 
\chapter{Упоротый}
\label{sec:slova.uporotyj}

%%%cit
%%%cit_head
%%%cit_pic
%%%cit_text
\emph{Упоротый} категоричен. Не слышит и не хочет слышать противоположную точку
зрения — она его убивает. Отсюда — пена у рта и порванная на груди рубаха.
\emph{Упоротый} не любопытен и не любознателен. Не любит думать. Новую
информацию подвязывает под давно сложившийся пазл стереотипов. Поэтому
\emph{упоротые} не меняют точку зрения. Любят обвинять других в «переобувании»
\emph{Упоротый} зациклен на себе, ему не интересен окружающий мир.
\emph{Упоротые} — глубоко несчастные люди. Не ведитесь на браваду. Павлиний
хвост самомнения на самом деле —  замаскированный комплекс неполноценности.
\emph{Упоротый} агрессивен. Раньше \emph{упоротые} выплескивали яд и желчь на
лавочках, теперь — в социальных сетях.  \emph{Упоротый} несчастен в любви. У
него проблемы в сексе. Отсюда ханжество и нетерпимость ко всему, что
сексуально. \emph{Упоротые тётки} любят перемывать кости красивым женщинам.
Мужики — обсуждать части тела.  \emph{Упоротый} лишен культуры уважения, в
избытке — неуважение ко всем и ко всему.  Во время революций \emph{упоротые}
первыми бегут бросать брусчатку в оппонентов. Это их бенефис.  Иногда
\emph{упоротые} становятся политиками или лидерами мнений
%%%cit_comment
%%%cit_title
\citTitle{День Гидности и значение слова «упоротый»}, Диана Панченко, facebook, 21.11.2021 
%%%endcit
