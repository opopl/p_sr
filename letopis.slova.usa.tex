% vim: keymap=russian-jcukenwin
%%beginhead 
 
%%file slova.usa
%%parent slova
 
%%url 
 
%%author 
%%author_id 
%%author_url 
 
%%tags 
%%title 
 
%%endhead 
\chapter{США}
\label{sec:slova.usa}

Сбежавшие из приюта в \emph{США} дети нашли в чужом доме автоматы и полчаса отстреливались от полиции,
strana.ua, 03.06.2021

\emph{Посольство США} в Киеве призвало власти Украины привлечь к формированию
Высшего совета правосудия (ВСП), Высшей квалификационной комиссии судей (ВККС)
и руководящих органов Национального антикоррупционного бюро (НАБУ) независимых
экспертов.  Пост на эту тему появился на странице посольства в Фейсбуке 21 мая
по итогам встречи представителей США и ЕС с председателем Верховной рады
Дмитрием Разумковым и заместителем главы Офиса президента Украины Андреем
Смирновым,
\citTitle{Посольство США рассказало, как защитить антикоррупционные органы Украины}, Наталья Полулях, strana.ua, 21.05.2021

Зараз українці, які продовжують вірити в пустопорожні заяви і декларації наших
західних партнерів, чимось нагадують африканських туземців в яких європейці
обмінювали золото та платину на скляні буси. Всі демократичні цінності
номіновані в доларах і євро. Недарма в чудовому фільмі \enquote{Пограбування казино}
було сказано: \enquote{Зв'язковий мафії: Ти цинічна сволота, ти це знаєш? Обама по
телевізору: Утвердитись в фундаментальній істині, що нас багато, але ми єдині!
Зв'язковий мафії: Чув що він сказав? Замислись. Джек Коган: Не сміши мене \enquote{ми
один народ}. Цей міф створив Томас Джефферсон – американський святий, бо він
написав, що всі люди створені рівними. У що він явно не вірив, якщо залишив
своїх дітей підневільними. Заможній сноб, який не бажав платити податки
Британії. Він написав красиві слова і підняв людей, які пішли помирати за ці
слова. А він сидів, пив дороге вино і шпекав свою рабиню. (Вказує на телевізор
з Обамою) Він говорить мені, що ми живемо в суспільстві? Я живу в Америці, а в
\emph{США} кожен сам за себе. \emph{Америка} – це не країна, це лише бізнес. Тож трясця,
плати мені}, 
\citTitle{Про Революцію Гідності та її наслідки}, Стефан Закревський, analytics.hvylya.net, 06.06.2021

То есть \enquote{стратегического партнера} \emph{американцы} даже лично не
уведомили о решении, которое напрямую задевает экономические интересы Украины.
Несмотря на такое отношение, Зеленский все еще считает, что, пусть даже
остается один процент вероятности остановить \enquote{Поток}, \emph{США}, при
желании, могут это сделать, \enquote{чтобы защитить Европу и Украину}. А пока,
по мнению Зеленского, всех защищает Украина. Причем исключительно из
\enquote{веры в цивилизованный мир}. Которой, судя по намеку Зеленского, сам
Запад в отношении нашей страны отнюдь не исповедует. \enquote{Иногда, вы
знаете... иногда эта вера в цивилизованный мир – вот что приносит победу. Разве
можно было представить, что армия 40-миллионной Украины сможет противостоять
одной из империй. Какой шанс был у такой страны, как наша, защитить свою
территорию, народ и суверенитет? Тем не менее я считаю, что мы это сделали и
все еще делаем это каждый день}, – пафосно заключил президент,
\citTitle{Зеленский интервью Axios – зрада по Северному потоку, НАТО и встрече с Путиным}, Максим Минин, strana.ua, 07.06.2021

Олигархи существуют и в \emph{США}. Американские олигархи тоже не платят налоги.
Недавно в СМИ \emph{Штатов} попали данные налогового управления. Из них выяснилось,
что владельцы крупных бизнесов платят в качестве налогов едва ли 1\%
заработанных денег. Самая низкая эффективная ставка оказалась у Баффета (0,1\%
заработанного). На его фоне Илон Маск выглядит просто транжирой (3,7\%).  На
страже состояния Джеффа Безоса (Amazon), Илона Маска (Tesla и SpaceX), Уоррена
Баффета (Berkshire Hathaway), Марка Цукерберга (Facebook), Джорджа Сороса
(понятно) и Майкла Блумберга (Bloomberg) стоят высококлассные схемы налоговой
оптимизации. А теперь - и администрация Байдена, которая начала расследование
\enquote{незаконной} утечки данных.  Понятно, что эта информация произвела неприятное
впечатление на рядовых жителей Штатов. В зависимости от штата, только ставка
подоходного налога колеблется от 4,63\% до 13,3\%. А эффективная налоговая
ставка, по расчетам журналистов, составляет порядка 14\% с семьи. Опираясь на
эти данные, политики-популисты могут с легкостью рассказывать, что в \emph{США}
\enquote{бедные платят за богатых}. Политикам-консерваторам будет сложно объяснить, где
разумная грань между стимулированием бизнеса и уклонением от налогов,
\citTitle{Налоговое управление США выловило злостных неплательщиков налогов}, Сергей Лямец, strana.ua, 09.06.2021

%%%cit
%%%cit_head
%%%cit_pic
\ifcmt
  pic https://img.strana.ua/img/article/3402/amerikanku-nazvali-samoj-96_main.jpeg
	caption Кристина Квин в рейтинге влиятельности обошла Зеленскую и Венедиктову. Фото: Facebook/U.S. Embassy Kyiv Ukraine 
\fi
%%%cit_text
Посол США в Украине Кристина Квин возглавила рейтинг самых влиятельных женщин
Украины по версии журнала \enquote{Фокус}. Любопытно, что она опередила в Топ-100
первую леди страны Елену Зеленскую и генпрокурора Ирину Венедиктову.
Любопытно, что самой влиятельной женщиной Украины признана гражданка США.
Супруге президента  Украины Елене Зеленской отвели лишь вторую ступень.  На
третьей - генпрокурор Ирина Венедиктова, а на четвертой - лидер партии
\enquote{Батькивщина} Юлия Тимошенко. Пятое место досталось также иностранке - послу
Великобритании Мелинде Симмонс
%%%cit_comment
%%%cit_title
 \citTitle{Американку назвали самой влиятельной женщиной Украины}, Юлия Супрун, strana.ua, 24.06.2021
%%%endcit

%%%cit
%%%cit_head
%%%cit_pic
%%%cit_text
Вновь получается некий когнитивный диссонанс в развитии: стратегический
политический партнер - \emph{США}, но с этой страной почти не торгуем и американских
инвестиций у нас кот наплакал.  Плюс \emph{американцы} вводят пошлины на наш металл в
размере 25\%, не позволяя даже делать из него продаваемые нам локомотивы (все
должно быть \emph{американское}, а уровень локализации локомотивов в Украине обеспечим
за счет перекрашивания оных в брендовые цвета УЗ – \enquote{краска уж больно у нас
дорогая})
%%%cit_comment
%%%cit_title
\citTitle{Отрицательное торговое сальдо Украины с РФ - патриотичная дыра / Лента соцсетей / Страна},
Алексей Кущ, strana.ua, 24.06.2021
%%%endcit

%%%cit
%%%cit_head
%%%cit_pic
%%%cit_text
Однако тактика \emph{США} и Евросоюза, которые отстаивали интересы фармацевтических
корпораций, являлась в этом смысле предельно эгоистичной. Конечно, западные
политики постоянно рассуждают на тему сотрудничества в борьбе с пандемией. Но,
по факту, они стремились обеспечить прививками основную часть населения Первого
мира, игнорируя ситуацию на периферии – чтобы потом цинично зарабатывать на
бедных странах, продавая им свои вакцины, или используя их с целью
политического давления.  Для реализации этой схемы фармацевтические корпорации
вытесняли с международного рынка конкурентов, в лице вакцины «Спутник V» и
препаратов из КНР – причем, эту дискриминацию зачастую объясняют причинами
политического характера. В то же время, Европа и \emph{США} отказывались приостановить
действие патентов на свои идеологически правильные вакцины – чтобы их можно
было свободно производить в Третьем мире. Прежде всего, потому, что это лишило
бы корпоративный менеджмент запланированных доходов
%%%cit_comment
%%%cit_title
\citTitle{Для победы над коронавирусом надо было вакцинировать Третий мир / Лента соцсетей / Страна}, 
Андрей Манчук, strana.ua, 24.06.2021
%%%endcit

%%%cit
%%%cit_head
%%%cit_pic
\ifcmt
  pic https://focus.ua/static/storage/thumbs/890x450/4/df/427f316a-f215db70d47f1cc8a4ac9f177f7dfdf4.jpg?v=7256_1
	caption Рейтинг \enquote{ТОП-100 самых влиятельных женщин Украины} по версии журнала Фокус, июнь 2021
\fi
%%%cit_text
В перечень самых влиятельных женщин 2021 года вошли женщины во власти,
предпринимательницы, активистки и многие другие. Номер 1 — американка,
временный поверенный в делах \emph{США} в Украине Кристина Квин.  Об этом
говорится в №25 журнала Фокус, который поступил в продажу 25 июня. Полный
вариант рейтинга с профайлами участниц будет опубликован на сайте Focus.ua во
вторник, 29 июня
%%%cit_comment
%%%cit_title
\citTitle{Топ-100. Кто вошел в рейтинг самых влиятельных женщин Украины по версии Фокуса}, 
Галина Ковальчук, focus.ua, 25.06.2021
%%%endcit

