% vim: keymap=russian-jcukenwin
%%beginhead 
 
%%file slisarenko_oleksa_andrijovich.statti.suspilne_06_08_2020
%%parent slisarenko_oleksa_andrijovich.statti
 
%%endhead 
\subsubsection{Розстріляні в урочищі Сандармох: у Харкові вшанували жертв сталінських репресій}
\url{https://suspilne.media/53342-rozstrilani-v-urocisi-sandarmoh-u-harkovi-vsanuvali-zertv-stalinskih-represij/}

\url{https://youtu.be/DNZ85lG51_0}
  
\vspace{0.5cm}
 {\ifDEBUG\small\LaTeX~section: \verb|slisarenko_oleksa_andrijovich.statti.suspilne_06_08_2020| project: \verb|family| rootid: \verb|p_saintrussia| \fi}
\vspace{0.5cm}

Харків'ян, яких розстріляли за вказівкою радянського керівництва у 1937 році в
Карелії, на майдані Свободи згадали поіменно.

Учасники акції зібралися біля намету "Все для Перемоги" о 17:00. Прийшли
близько десяти людей.

``Ми хочемо показати, яких харків'ян знищили протягом тижня: з початку листопада
і до сьомого числа 1937 року'', --- розповіла співорганізаторка заходу Вікторія
Склярова.

5 серпня, каже Склярова, вважається початком “Великого терору” 1937-1938 років.

На майдані учасники акції-реквієму зачитали 77 прізвищ: це поети, письменники,
режисери, науковці та священники, які жили на Харківщині та були страчені в
урочищі Сандармох.

Почути всі прізвища та дізнатися про долі розстріляних, можна у стрімі ---
Суспільне транслювало акцію наживо:

Серед уродженців Харківщини, які були вбиті у Сандармосі --- український поет
та прозаїк Олекса Слісаренко. Про долю прадіда Олександр Поплавський дізнався
від родичів, коли був школярем:

"Ми усього цього не знали за часів Радянського Союзу. Це вже коли бабуся Наташа
померла, прийшли родичі, принесли всі ті книжки Слісаренко. У мене є чотири
томи, це оригінали".

Олександр працює програмістом у Києві. У вільний час вивчає власний родовід та
збирає архівні документи:

"У мами своєї, у дядька розпитую — хто що знає? Я думаю, мало хто знає. Там
треба копати, копати. Якщо людина не знає про своїх предків, вона як листя —
полетіло, зірвало його, спалило. І все. А якщо це вже рід, розумієте, це вже як
дуб. Його так просто не вирвеш. Тому я саме цим і займаюсь".

За словами Олександра, інший його прадід у 1930-х роках служив в НКВС. Чи був
він причетним до розстрілів, Поплавський не знає.
 
\url{https://www.youtube.com/watch?v=xS2olFxe1Ys&feature=emb_rel_end}

Серед харків'ян, яких стратили у Сандармосі, були драматурги Микола Куліш і
Лесь Курбас. Всього в урочищі розстріляли понад тисячу репресованих. Кожен
четвертий був українцем, розповідає історикиня Марія Тахтаулова.

Історик Юрій Дмітрієв, який віднайшов це поховання на початку 1990-х років,
зараз перебуває у російській в'язниці.

"Світова спільнота і цех істориків та юристів говорять про те, що це було
політичне замовлення за його діяльність саме на тлі досліджень, пов'язаних зі
сталінськими репресіями. Саме він віднайшов це урочище Сандармох і розказав про
нього світу. Тому що ми не знали, що сталося із Соловецьким етапом. І ці 1111
імен, які ми зараз читаємо. І завдяки Юрію Дмітрієву ми зараз ці імена
віднайшли", - каже Тахтаулова.

Авторка: Дар'я Рудаченко
