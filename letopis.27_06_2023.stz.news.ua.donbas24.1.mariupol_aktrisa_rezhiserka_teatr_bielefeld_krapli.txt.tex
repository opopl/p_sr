% vim: keymap=russian-jcukenwin
%%beginhead 
 
%%file 27_06_2023.stz.news.ua.donbas24.1.mariupol_aktrisa_rezhiserka_teatr_bielefeld_krapli.txt
%%parent 27_06_2023.stz.news.ua.donbas24.1.mariupol_aktrisa_rezhiserka_teatr_bielefeld_krapli
 
%%url 
 
%%author_id 
%%date 
 
%%tags 
%%title 
 
%%endhead 

Маріупольська актриса і режисерка очолила новий театр у Німеччині (ФОТО)

У місті Білефельд запрацював новий український театр «Краплі»

Народний театр Маріуполя «Театроманія» після повномасштабного вторгнення рф
продовжує розвивати та популяризувати театральну культуру Маріуполя. Зокрема, у
німецькому місті Гановер працює «Театроманія 2.0», якою керує режисер Антон
Тельбізов, а в м. Білефельд створено філію «Театроманії» театр «Краплі» під
керівництвом актриси і режисерки Ольги Самойлової. Обидва театри працюють з
українською молоддю! Головною метою театрів є поширення української культури,
мови, традицій в країні, яка прихистила їх від війни!

Читайте також: Театр авторської п'єси Conception представив поетичну імпрезу
«Лінії життя»

Історія створення театру «Краплі»

За словами Ольги Самойлової, театр виник випадково. У актриси вже було декілька
проєктів, серед яких один вона реалізовувала з учнями профучилища. За один
тиждень рижисерка поставила з ними виставу. Згодом про це дізналися
представники Українського волонтерського центру, які запросили Ольгу працювати
до себе. Там актриса почала вести театральний гурток. Спочатку учасниками стали
5 дівчаток і 1 хлопчик. Але актриса розмістила оголошення, в якому розповіла
про роботу гуртка. Завдяки цьому на одному із занять Ольги вже було 19 осіб, з
яких троє — хлопці. Назву «Краплі» театру запропонував чоловік Ольги — Валерій
Мережаний.  «Ми, як краплі одного чогось великого, ніби цілого океану. І ці
краплі у різному вигляді: у вигляді дощу, чи у вигляді інших водойм знаходяться
в різних куточках землі. Ми як маленькі шматочки одного чогось великого
цілого», — розповіла Ольга.

Режисерка своїх учнів називає «мої крапельки». Наразі у колективі 15 акторів і
актрис, яких об'єднує одне бажання — розповідати про страшну війну, що
перевернула життя кожного з них. У театрі грають хлопці і дівчата з різних міст
України: Одеси, Вінниці, Харкова, Луганська та Маріуполя.

Читайте також: Нові проєкти «Театроманії 2:0» та життя колективу за кордоном

Перша вистава театру «Краплі»

24 червня театр «Краплі» було запрошено на місцевий телевізійний канал «21»
разом з українськими музикантами і співаками — родиною Катерини та Макса
Тітових, де в програмі були виконані українські пісні та номер, присвячений
подіям в Україні «Незламна». Ольга Самойлова створила музичний номер, в якому
центральне місце посів образ України. Його втілила дівчинка, яка пригощала то
яблуками, то зерном своїх сусідів. Потім сусіди вирішили, що вони можуть і самі
взяти все, що забажають. І тоді вони порушили кордони та зайшли на її
територію. Дівчинка намагалася захистити свою територію, але бій був нерівний,
тому що тих, хто напав, було дуже багато. Фінал номера вийшов досить
символічний: Україна перебуває в огні, а дівчина, що співає, підходить до неї і
вкладає в її руку квітку, які є прообразом життя, після чого дівчинка встає і
протягує руку до сонечка. Ця квітка зцілює та рятує дівчину-Україну.

У Білефельді дуже підтримують театр. Глядачі, які вперше побачили «Незламну»,
наголосили, що це театр справжніх професіоналів, роботи яких вони будуть з
нетерпінням чекати.

«Вони так грають, тому що це болить. А коли болить, людина, яка навіть є
непрофесійним актором, буде віддаватися на всі 100%. І я хочу на цій хвилі
зробити ще одну виставу», — додала Ольга.

Читайте також: «Життя переселенське» — маріупольський театр підготував нову
прем'єру

Подальші плани театру

Наразі театр «Краплі» працює над виставою, присвяченою Маріуполю.

Ользі дуже допомагає її учениця, яка виїхала з Маріуполя — Анастасія Сарбаш.
Близько 5 років вона навчалася в «Театроманії», потім поїхала до Києва, де
вступила до Київського національного університету культури і мистецтв на
акторський факультет. Торік вона його закінчила. Ольга разом з Анастасією пише
сценарій майбутньої вистави. Цікаво, що дівчина з Маріуполя, яка грає в театрі,
написала щоденник про пережите, і коли Ольга його побачила, вирішила, що він
може бути взятий за основу одного з монологів до вистави.

На День Конституції України колектив виступатиме у місцевій синагозі. Там Ольга
Самойлова читатиме вірші Ліни Костенко, а театр знову покаже номер «Незламна».

Раніше Донбас24 розповідав унікальні факти про театральну культуру Маріуполя.

Ще більше новин та найактуальніша інформація про Донецьку та Луганську області
в нашому телеграм-каналі Донбас24.

Фото: з архіву Донбас24
