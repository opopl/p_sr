%%beginhead 
 
%%file 07_03_2023.fb.fb_group.mariupol.pre_war.2.istoriya_shkoli_na_k
%%parent 07_03_2023
 
%%url https://www.facebook.com/groups/1233789547361300/posts/1423149315091988
 
%%author_id fb_group.mariupol.pre_war,elena_mariupolskaja
%%date 07_03_2023
 
%%tags mariupol,mariupol.pre_war,istoria,mariupol.istoria,shkola,obrazovanie,1883
%%title История школы на Карасьевке
 
%%endhead 

\subsection{История школы на Карасьевке}
\label{sec:07_03_2023.fb.fb_group.mariupol.pre_war.2.istoriya_shkoli_na_k}
 
\Purl{https://www.facebook.com/groups/1233789547361300/posts/1423149315091988}
\ifcmt
 author_begin
   author_id fb_group.mariupol.pre_war,elena_mariupolskaja
 author_end
\fi

История школы на Карасьевке

В 1883 году сыновья и дочери карасьевских жителей вошли в одноэтажный,
приземистый дом, где им предстояло обучаться чтению, письму и началам
математики.

В отчете мариупольской уездной земской управы за 1884 год дано описание первой
карасьевской школы. Она состояла из двух комнат, общая площадь которых
составляла 130 квадратных аршин (65.75 кв. метров). В этом же документе
отмечалось, что в школе полы были \enquote{совсем плохи}, что в зимнюю пору из погреба
дуло.

В 1892 году, или чуть раньше, Мариупольская городская дума ходатайствовала
перед вышестоящими властями о присвоении школе имени митрополита Игнатия.
Точную дату удовлетворения ходатайства установить не удалось, но то, что, по
крайней мере, в 1905 году она уже именовалась Игнатьевской, подтверждается
документами Мариупольской уездной земской управы.

В 1905-1906 учебном году в школе обучалось 249 детей: 136 мальчиков и 113
девочек. 

В разные годы учебное заведение на Карасьевке называлось по-разному. В 1910
году : Городское начальное училище имени митрополита Игнатия (Карасьевское
1-е). Как это часто бывает, изменение в названии не обязательно сопровождалось
изменениями по существу. Как и раньше здесь преподавали Закон Божий, русский и
церковно-славянский языки, арифметику, географию, природоведение и пение.

В 1908 году в Карасьевке была открыта еще одна школа – Игнатьевская 2-я. В
одном старинном источнике сказано, что \enquote{образовалась она из вечерней
самостоятельной школы, помещавшейся в здании Игнатьевской 1-й земской школы}.

Игнатьевская (Карасевская 1-я) помещались в одноэтажном здании на
Косьмо-Дамиановской площади в непосредственной близости от церкви Рождества
Пресвятой Богородицы. Здание это, правда, многократно перестроенное,
существовало еще в 1968 году. Позже на его месте ремонтно-строительный трест
выстроил себе двухэтажную контору, которую теперь занимает  Налоговая
администрация. 

После революции, в 1918 году Игнатьевская школа утратила собственное имя и
получила порядковый номер. Теперь она стала одиннадцатой начальной. К тому
времени в ней было пять классных комнат. В 1928 году ее преобразовали в
семилетку.

Через какое-то время после разрушения Карасьевской церкви на пустыре рядом со
старым школьным зданием, где зимой школьники устраивали каток, началось
строительство четырехэтажного корпуса. 1 сентября 1938 года его двери
распахнулись перед учениками, которые занимались в семилетней школе № 11 и
ребятами близлежащих кварталов. Тогда был произведен набор с первых по восьмые
классы. Восьмых классов было два. 

19 июня 1941 года в школе на Карасьевке состоялся выпускной бал
десятиклассников – первый после открытия нового здания. Его участники, дожившие
до наших дней, запомнили его светлым, радостным и веселым... А через три дня
началась война.

С первых дней войны в школе был развернут госпиталь. С первых дней оккупации
гитлеровцы устроили вместо него лагерь советских военнопленных. 

Позже оставшихся в живых обитателей лагеря переместили в еще более ужасное
место – бывший учебный комбинат завода имени Ильича. В освободившееся здание на
какое-то время въехал немецкий штаб. А после в школе устроили склад зерна. Все
классы были засыпаны кукурузой, просом, пшеницей. И так было до отступления
оккупантов в сентябре 1943 года. Перед уходом они подожгли школу. Загорелись
деревянные перекрытия, а потом рухнули под грузом почерневшего от огня зерна».

Вскоре после освобождения Мариуполя занятия в 11-й школе возобновились. Она как
бы вернулась к своим истокам. Теперь это была начальная школа, разместившаяся в
одноэтажном здании, где она находилась до 1938 года. Благо, огонь его пощадил. 

В 1948 году пришло распоряжение передать помещение детскому дому. Привезли из
города Сталино (так тогда называли Донецк) большую группу детей разного
возраста, а учеников школы расселили по другим школам.  Только в 1951 году
отстроенная школа  открыла двери перед детьми  района Карасьевки. До 1953 года
школа была семилеткой. в 1961 году, она стала одиннадцатилеткой и получила
замысловатое название \enquote{трудовой политехнической}.

С 1965 по 1980 годы школу  преобразовали в английскую. Именно в 1965 году
начали принимать в первые классы необычного, по тем временам, учебного
заведения детей со всего города. ... (Из материалов сайта Старый Мариуполь)

%\ii{07_03_2023.fb.fb_group.mariupol.pre_war.2.istoriya_shkoli_na_k.eng}
%\ii{07_03_2023.fb.fb_group.mariupol.pre_war.2.istoriya_shkoli_na_k.cmt}
