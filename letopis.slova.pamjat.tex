% vim: keymap=russian-jcukenwin
%%beginhead 
 
%%file slova.pamjat
%%parent slova
 
%%url 
 
%%author 
%%author_id 
%%author_url 
 
%%tags 
%%title 
 
%%endhead 
\chapter{Память}
\label{sec:slova.pamjat}

%%%cit
%%%cit_head
%%%cit_pic
%%%cit_text
В субботу, 19 июня, в Киеве открыли \emph{памятник}, установленный в честь
медицинских сотрудников, которые скончались во время борьбы с эпидемией
коронавируса.  Монумент установили возле городской клинической больницы №4 в
Соломенском районе столицы.  Об этом сообщается на странице больнице в
Facebook.  \enquote{Врачи - главные герои последних двух лет во всем мире. Весь
2020 год ежедневно врачи боролись с маленьким вирусом, который повлиял на ход
жизни многих людей. И в этой борьбе, к сожалению, у нас были не только победы,
но и поражения - мы теряли жизни... Жизни пациентов и жизни врачей, которые
жертвовали собой в борьбе за жизнь других. Поэтому врачи - реальные герои этого
времен}, - говорится в сообщении.  В церемонии открытия участвовали министр
здравоохранения Украины Виктор Ляшко, депутаты Верховной Рады Украины, депутаты
Киевского городского совета и председатель Соломенской РГА.
\enquote{Медицинским работникам, которые пожертвовали своей жизнью в борьбе с
пандемией Covid-19 в Украине}, - написано на \emph{памятнике}
%%%cit_comment
%%%cit_title
\citTitle{В Киеве появился памятник медикам, которые отдали свои жизни в борьбе с Covid-19}, 
Эллина Либцис, kiev.strana.ua, 19.06.2021
%%%endcit


%%%cit
%%%cit_head
%%%cit_pic
%%%cit_text
Цікава штука, ця глибина \emph{пам'яти} і тяглість. Може, звісно, я всі ці роки дуже
погано дивився, але наша історична \emph{пам'ять} як певний суцільний сюжет
закінчується приблизно на українській революції. Перша світова. Далі в минуле
тяглости ніби немає, є лише якісь фрагменти спогадів: козаки,
політично-культурна буча зламу століть, а між тим - тиша. Дев'ятнадцяте
століття з нашої історії ніби випадає, а все що перед ним - узагалі час
старозавітніх пророків і патріархів, тільки десь посереди первісного бульйону
дев'ятнадцятого століття бродять Шевченко, Франко і Українка
%%%cit_comment
%%%cit_title
\citTitle{У нас досі радянська пам'ять. Світ нібито почався в 1917-му, до того був морок}, 
Остап Українець, gazeta.ua, 25.06.2021
%%%endcit

%%%cit
%%%cit_head
%%%cit_pic
%%%cit_text
Правда, в сети ходит непроверенная информация о том, что в 1963 году, будучи в
Кривом Роге милицейским начальником, дед Зеленского якобы был замешан в
массовом расстреле толпы во время беспорядков, вызванных ментовским беспредом.
После чего якобы пошел на повышение, то есть был такой себе «небольшой
Новочеркасск в Кривом Рогу». Но, еще раз повторим, что эта информация не
проверенная.  Так неужели же Зеленский, внук «деда, который воевал», не в
состоянии «закрыть рот» подчиненному ему напрямую «зооветеринарному СНБО» и
разным там «центрам по дезинформации» при СНБО, чтобы те не топтались по \emph{памяти}
погибших и победивших в той войне?!
%%%cit_comment
%%%cit_title
\citTitle{Краткий словарь грантоедов под редакцией СНБО / Лента соцсетей / Страна}, 
Александр Карпец, strana.news, 26.10.2021
%%%endcit

%%%cit
%%%cit_head
%%%cit_pic
%%%cit_text
У всех очевидцев взрыва взяли подписку о неразглашении. Сам Аюпов впервые
поделился своими \emph{воспоминаниями} с родными только в конце 1990-х. Сейчас ему
хорошо за 80. В \emph{память} о службе на Новой Земле осталась справка о том, что он
«принимал непосредственное участие в подготовке и проведении натурных ядерных
испытаний в атмосфере на Центральном полигоне в 1961-1962 годах, работая
длительное время в условиях высокого уровня радиации».
Впоследствии Аюпов запрашивал информацию о том, какую именно дозу радиации он
получил во время испытаний. Тем не менее в документах указано, что дозовой
нагрузки у участников испытаний из его войсковой части не имеется. Совет по
установлению причинной связи заболеваний, инвалидности и смерти граждан,
подвергшихся воздействию радиационных факторов, в его случае воздержался от
вынесения соответствующего вердикта
%%%cit_comment
%%%cit_title
\citTitle{«Дом будто ножом срезало» 60 лет назад СССР взорвал «Царь-бомбу» — самую мощную в истории. Что помнят о взрыве очевидцы?: Общество: Россия: Lenta.ru}, Дмитрий Окунев, lenta.ru, 30.10.2021
%%%endcit

%%%cit
%%%cit_head
%%%cit_pic
\ifcmt
  pic https://zz.te.ua/wp-content/uploads/2021/11/249868260_1959330300917624_6599417831403188154_n.jpg
  @width 0.4
\fi
%%%cit_text
Всеукраїнський турнір з вільної боротьби серед чоловіків \enquote{Тернове поле}
стартував у Тернополі. Близько сотні учасників з різних міст України та з-за
кордону з’їхалися, аби поборотися за звання \enquote{Майстра спорту}.  Змагання
присвячені \emph{пам'яті} заслуженого тренера України, судді міжнародної категорії
Віталія Назарчука. Поєдинки проходять у спортивному залі
фізкультурно-оздоровчого комплексу.  Як розповів головний суддя Сергій Цимбал,
у цьогорічному турнірі змагається 9 команд з 20-ти міст України, а також дві
команди з-за кордону – з Молдови та Республіки Вірменія
%%%cit_comment
%%%cit_title
\citTitle{У Тернополі проходить всеукраїнський турнір з вільної боротьби \enquote{Тернове поле}}, 
Софія Романська, zz.te.ua, 01.11.2021
%%%endcit
