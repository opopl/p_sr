% vim: keymap=russian-jcukenwin
%%beginhead 
 
%%file moje.kremlevskie_narrativy.jazyk
%%parent moje.kremlevskie_narrativy
 
%%url 
 
%%author_id 
%%date 
 
%%tags 
%%title 
 
%%endhead 

\subsubsection{Положение русского языка в Киеве, в частности, и вообще в Украине}

Тут, к сожалению, ситуация совершенно удручающая с точки зрения поддержки
государством, на содержание которого мы исправно платим налоги. Язык, на
котором писал свои сочинения Григорий Саввич Сковорода, наш великий
философ-просветитель, который в свое время учился в Киевской Академии, и чей
памятник стоит на Подоле, как раз возле Андреевского Спуска, - по факту
систематически изгоняется из сферы образования, науки, культуры. Ну, точнее его
пытаются изгнать, унизить на государственном уровне, хотя априори очевидно, что
это абсолютно бессмысленная и глупая затея, поскольку Украина, - родина Николая
Гоголя и Тараса Шевченка, - как говорила на двух родственных языках, - на
русском и на украинском, так и будет говорить. Украина говорит на русском, - от
Ужгорода до Одессы, от Львова до Харькова, от Чернигова до Днепропетровска. Так
было вчера, так есть сейчас, и так будет всегда. Но... есть хорошая пословица -
научи дурака молиться, и он лоб расшибет об стену, и у нас действительно есть
много дураков, очень энергичных и пассионарных дураков, между прочим, которые
несмотря ни на что, продолжают заниматься этой глупой затеей (вытеснение
русского языка и так называемая украинизация), вместо того чтобы тратить свои
духовные силы и свою бесценную жизнь на что-то более творческое и
созидательное. Плевки на русский язык в Украине, - это примерно как плевки в
потолок, лежа на диване.  Лежишь, плюешься, - и чувствуешь себя хорошо. Хотя по
факту, - просто диванный идиотизм. И потолок-то придется же когда-нибудь
вытирать!

\ifcmt
  tab_begin cols=2,no_fig,center
     pic https://scontent-frt3-1.xx.fbcdn.net/v/t39.30808-6/274088649_1938418763005602_9087982079840120782_n.jpg?_nc_cat=104&ccb=1-5&_nc_sid=730e14&_nc_ohc=LVcOvIp3pa8AX8eUnPc&_nc_ht=scontent-frt3-1.xx&oh=00_AT-nR4qMJHxWtl8z3Qr7sEqtYMlymi_1qaJIwBPkGHq4vA&oe=621D4471

		 pic https://pbs.twimg.com/media/D6Y1aJ0WsAAzlva.jpg
  tab_end
\fi

Так вот, знаете, сейчас как раз уже 300 лет нашему
великому философу, и это поразительно, просто поразительно, что язык, на
котором он писал о самых сокровенных вещах - Человеке, Боге, Душе -
систематически оплевывается и унижается. Некоторые недалекие люди говорят, а
иные даже исступлено кричат, испуская потоки желчи по бескрайнему интернету,
что это язык московских оккупантов, говорят, что это московский (москвинский,
мокшанский) язык. Чушь собачья! Русский язык, раз уж вспомнили о Сковороде, это
- прежде всего язык Сковороды, поскольку на 90 процентов или более язык его
произведений совпадает со современным русским языком. Не верите? Ну так
почитайте сами, откройте в оригинале Сад Божественных Песен, глаза ж вам даны
не только для того, чтобы смотреть передачи ТСН, не так ли? Конечно, в его
произведениях есть примеси украинских (малороссийских) слов, и грамматические
конструкции немного старомодны, все-таки писал он более двух столетий назад; но
по факту, язык его произведений - с нашей (Киевской) точки зрения, это самый настоящий русский язык, это уж
будьте уверены. И значит, если унижают русский язык, то по факту унижают язык
Григория Саввича, а значит, это плевок в самого Сковороду.  Задумайтесь об
этом, господа патриоты. Вот, например, песнь 18-ая, красиво, правда?

\raggedcolumns
\begin{multicols}{2} % {
\setlength{\parindent}{0pt}
\obeycr
Ой ты, птичко жолтобоко,
Не клади гнезда высоко!
Клади на зеленой травке,
На молоденькой муравке.
\smallskip
От ястреб над головою
Висит, хочет ухватить,
Вашею живет он кровью,
От, от! кохти он острит!
\smallskip
Стоит явор над горою,
Все кивает головою.
Буйны ветры повевают,
Руки явору ломают.
\smallskip
А вербочки шумят низко,
Волокут мене до сна.
Тут течет поточок близко;
Видно воду аж до дна.
\smallskip
На что ж мне замышляти,
Что в селе родила мати?
Нехай у тех мозок рвется,
Кто высоко в гору дмется,
\smallskip
А я буду себе тихо
Коротати милый век.
\smallskip
Так минет мене все лихо,
Щастлив буду человек.
\restorecr
\end{multicols} % }

\ifcmt
tab_begin cols=2
  @caption Памятник Григорию Сковороде в городе Киеве

  ig https://avatars.mds.yandex.net/get-altay/2776464/2a00000170f4a99f4b8134bb851eb03ec81d/XXL
	ig https://ic.pics.livejournal.com/s_a_plotnikov/85688553/71932/71932_original.jpg

tab_end
\fi

Далее, что касается русского языка, является ли он своим или чужим, мы скажем,
что русский, - это один из двух основных Киевских языков, да, языков Города
Киева, наравне с украинским, на котором, кстати, написан гимн Киева, Як тебе не
любити, Києве Мій:

\raggedcolumns
\begin{multicols}{3} % {
\setlength{\parindent}{0pt}
\obeycr
Грає море зелене,
Тихий день догора.
Дорогими для мене
Стали схили Дніпра,
Де колишуться віти
Закоханих мрій
Як тебе не любити,
Києве мій!
\smallskip
В очі дивляться канни,
Серце в них переллю.
Хай розкажуть коханій,
Як я вірно люблю.
Буду мріяти й жити
На крилах надій
Як тебе не любити,
Києве мій!
\smallskip
Спить натомлене місто
Мирним, лагідним сном.
Ген вогні, як намисто,
Розцвіли над Дніпром.
Вечорів оксамити,
Мов щастя прибій
Як тебе не любити,
Києве мій!
\restorecr
\end{multicols} % }

\ifcmt
  tab_begin cols=3,no_fig,center
     pic https://i2.paste.pics/3435a4bd2b8d179dd31668aa7dd9c70f.png
		 pic https://i2.paste.pics/78cf109cef08a466679e274394180c4c.png 
		 pic https://i2.paste.pics/04f319315c3b814bc8cdca8acd50b23b.png
  tab_end
\fi

Ничуть не желая унизить или умалить украинский язык (мы про
него напишем в другом месте), мы тем не менее скажем, что русский - это Язык,
великий язык на котором говорят миллионы Киевлян каждый день. В разговорах в
семье, на улицах; в школах и на свиданиях; во время телефонных разговоров и
также во время разговоров по скайпу, во время переписки по вайберу или по
телеграму. На русском языке в Киеве говорят уборщицы и депутаты Верховной Рады;
говорят сотрудники СБУ и врачи скорой; говорят везде, и тут, и там. В
маршрутках говорят и в метро тоже; говорят в институтах Академии Наук, и в
Милицейской Академии тоже; Говорят зимой, и летом; осенью, и весной. На русском
языке матерится Петр Порошенко, когда ему вручают повестку, и на русском же
языке Антон Геращенко призывает россиян остановить Путина. Да, представте себе,
дамы Фарион и Ницой, в Киеве говорят на прекрасном, удивительном русском языке,
и воистину поразительно, что нам нужно вообще об этом писать и напоминать,
поскольку обилие и разнообразие разговорного и письменного русского языка в
Киеве - это очевидный факт. 

Киев - это возможно вообще самый богатый русским
языком Город в мире, и неудивительно, - поскольку в Киеве была же первая библиотека, первые школы
на Руси, и первый свод законов, - Русская Правда... И именно Киевляне - а
Киевлянин это не только тот человек, кто родился здесь, - а в более широком
смысле - это вообще любой человек, кто душевно связан с Киевом, кто любит Киев,
кому Киев дорог, - Булгаков, родившийся здесь, Гоголь, мечтавший занять кафедру
при Киевском Университете, Пушкин, бывавший здесь в юности, Высоцкий, гостивший
в детстве у своей киевской бабушки, - и многие другие - преображали русскую
культуру и язык, каждый по-своему выводя ее на новый уровень.  На русском языке
поют Киевляне Потап и Настя, на русском языке долгие годы выступает Студия
Квартал 95; на русском языке был снят замечательный сериал Слуга Народа - к
сожалению, он был использован Зеленским в итоге не в лучшую сторону... Молодой
талантливый русскоязычный актер, пришедший на волне народных чаяний, так изящно
и смешно пародировавший украинских политиков, и предавший, - увы, -
впоследствии все идеалы, на которых он пришел к власти...    

\ifcmt
  tab_begin cols=3,no_fig,center
     pic https://lh4.googleusercontent.com/-UBWNb92DrJY/ULExscthAxI/AAAAAAAAFNw/s7085Li2Sak/s900/20120617-114000.jpg
		 pic https://lh6.googleusercontent.com/-bB20pabQmw4/ULExURIT2zI/AAAAAAAAFLw/_ypSk3bdu3w/s900/20120616-113915.jpg
		 pic https://lh4.googleusercontent.com/-WZTmPXZnwak/ULExanzAt6I/AAAAAAAAFMQ/f1arqxd0jbU/s700/20120616-114441.jpg
  tab_end
\fi

Русский язык абсолютно Киеву органичен и Киеву совершенно свой. На русском
говорят и пишут Антон Геращенко и Арсен Аваков, Петр Толочко и Олеся Медведева,
Алексей Арестович и Борислав Береза, Олег Волошин и даже Юрий Бутусов. На
русском языке был написан призыв Мустафы Найема, с которого начался Майдан-2, а
значит, русский язык, - это язык Майдана.  На русском языке говорили воины
Красной Армии, которые ценой огромных жертв освободили Киев, и на русском языке
писал свои книги Михаил Булгаков. Это наш - Киевский язык, корни которого идут
прямиком в Киевскую Русь, во времена, когда на стенах Софиевского Собора
простые Киевляне оставляли свои надписи. Вы хотите сказать, что это все
староукраинский язык, а русский язык - это миф путинской пропаганды, что это
все навязанное, не свое, родное. Что была именно Украина-Русь, и именно так, а
не иначе.  Ну а мы тогда ответим, что Украина-Русь - это какой-то непонятный
конструкт, какой-то вообще выдуманный мираж. Был Киев, Стольный Град Киев,
который столько раз грабили и уничтожали, а он все возрождался, Киев,
объективный, реальный, вот он - здесь - прямо под ногами! 

\ifcmt
  tab_begin cols=4,no_fig,center

     pic https://scontent-frt3-1.xx.fbcdn.net/v/t39.30808-6/274298591_5398943100149786_6052439259617304698_n.jpg?_nc_cat=104&ccb=1-5&_nc_sid=730e14&_nc_ohc=fZoy-V3nYC8AX-cLUe7&_nc_ht=scontent-frt3-1.xx&oh=00_AT-_LmgNiwF-JpZVtqveovUTA6BjPpbCG_bUziH_5vFmqQ&oe=621E8523

		 pic https://scontent-frt3-1.xx.fbcdn.net/v/t39.30808-6/274541079_5398687480175348_4058488227587905878_n.jpg?_nc_cat=108&ccb=1-5&_nc_sid=730e14&_nc_ohc=wJ1kHYt2DV4AX_xt40Q&_nc_ht=scontent-frt3-1.xx&oh=00_AT-vRH1Q7BKxNILDVCBEkh-x-KhByEpGZRIKiSvrbaaTjg&oe=621E4628

		 pic https://scontent-frt3-2.xx.fbcdn.net/v/t39.30808-6/274078774_5390065007704262_9192112163879752157_n.jpg?_nc_cat=101&ccb=1-5&_nc_sid=730e14&_nc_ohc=c_5cGMcePbgAX_Yzw5l&_nc_ht=scontent-frt3-2.xx&oh=00_AT-Aey1JU_1W78A7j3JSnhJvISUI4nAoe4VC1rr9hvpfhQ&oe=621CC761

		 pic https://scontent-frx5-2.xx.fbcdn.net/v/t39.30808-6/274154149_5390028534374576_4309070714717242551_n.jpg?_nc_cat=109&ccb=1-5&_nc_sid=730e14&_nc_ohc=d-Y3WzxY3kYAX-nZstG&tn=Sz-7xQ5oxc85gbiv&_nc_ht=scontent-frx5-2.xx&oh=00_AT9M3n4NvN-3nc9i1pDRqy9eYwUwzXUTr9NGSQttuTHRUg&oe=621E70BD

  tab_end
\fi

Град Киев, куда язык
человеческий вел, стоило только спросить...  а как нам добраться до Киева, не
подскажете? И тут же путнику охотно указывали, - сначала пройди вон тот лес, -
ты ж не балбес? - потом поверни направо, потом - налево, потом будет речка с
русалками, - потом будет еще один лес, - только гляди, такой тебе совет, - не
попади ты волкам на обед, - ну аж потом ты наконец-то увидишь Киев...  И он
здесь стоит уже тысячу лет, наш Киев, в котором были киевляне и князья, бояре и
смерды, тиуны и дружинники. И все это было связано со Псковом и Новгородом,
Суздалем и Ярославлем, и другими городами Северо-Восточной Руси.  И все это
вместе в массовом сознании было Русь. Просто Русь. А потом, - постепенно Русь
разделилась, раздробилась, раскололась на части, и каждая часть пошла своей
дорогой, - одна путем вольницы и освободительных войн, а другая - путем
жесткой, централизованной государственности и имперскости, - потом соединившись
вместе, а потом снова в наши времена разойдясь... А пресловутая Украина-Русь?
Вот именно так, Украина-Русь, или же Русь-Украина, через дефис, и никак иначе,
а то Грушевский на пару с Арестовичем обидятся?  Да чепуха это все. Был Киев -
а потом все остальное. И что сейчас важно, по-настоящему важно, это не то,
насколько именно сегодняшний Киевский русский язык подобен языку графити Софии
Киевской, а то, что Киев сейчас является невероятно замусоренным, загрязненным,
разбитым городом, чье нынешнее запущенное состояние (можно еще вспомнить про
Чернобыль рядом) абсолютно не соответствует его истинной значимости, являясь
наглядным примером порочности и примитивности нынешней украинской
государственной идеологии, а также культурного обнищания и духовной деградации
украинского народа в целом.

\ifcmt
  tab_begin cols=3,no_fig,center

     pic https://gordonua.com/img/article/15213/29_big.jpg?v1601729183
		 pic https://avatars.mds.yandex.net/i?id=515efdebec88624056a69b00ecc6e3b0-5886792-images-thumbs&n=13
		 pic https://avatars.mds.yandex.net/i?id=a951e988dcb2b7168b7301d363197d0a-5747194-images-thumbs&n=13

  tab_end
\fi

Далее, касательно притеснений русского языка и в Киеве, и по Украине, это
действительно правда, и Кремль здесь особо ни причем, поскольку же именно ты,
государство Украина, управляющее Городом Киевом последние 30 лет, хорошо
постаралась в этом, закрывая русские школы, вводя глупые украинизаторские
законы, и вообще систематически унижая русский язык - кровный язык Киева! - как
второстепенный, иностранный язык.  Знаешь, Украина, есть такая пословица:
насильно мил не будешь. Это касательно развития украинского языка. К сожалению,
тоталитарная репрессивная политика в отношении статуса русского языка и русской
культуры не привела к настоящему, духовному подъему украинской культуры и
языка, как ты хотела, не привела. Совсем наоборот.

\ifcmt
  tab_begin cols=3,no_fig,center
     pic https://i2.paste.pics/c4004e2585224bc31f681c37e8d849bc.png
		 pic https://i2.paste.pics/5de7ab7812afc20b9788023f41992509.png
		 pic https://i2.paste.pics/b7b038ded8df6885ce858fc8fc4054af.png
  tab_end
\fi

Alas! Украинский язык - певучий, удивительный язык, - проникающий в самую глубь
сердца, озаряющий самые потаенные уголки души, - стоит только открыть томик
Леси Украинки или же Олександра Олеся, или послушать какую-нибудь песню
Русланы, - ну например, старую добрую песню Свитанок, или же, Знаю Я -
превратился в государственный фетиш, в мертвый символ, в канцелярский язык, в
язык, на котором заставляют говорить насильно, в язык, который многие люди
вообще абсолютно искренне, от души ненавидят, поскольку он ассоциируется с
ненавистью, грабежами, убийствами и насилием, творимым незаконными военными
действиями ВСУ, а также с людоедскими заявлениями Фарион, Дроздова, Ницой, и их
многочисленных единомышленников, которые готовы резать москалей и малороссов,
которые повсюду видят предателей, живя во мраке мысленных чудовищ, порожденных
ими самими... Знаешь, Украина, апостол Павл сказал когда-то, что, \enquote{если
буду говорить красиво, но не буду иметь любви, то буду подобен меди звенящей},
- то есть, даже если ты говоришь самым изысканным украинским языком, но при
этом ты зол и глуп, и душа твоя пуста и не имеет сострадания и сочувствия, а
только бесконечное - росия-ворог, путин..., москалей-на-ножи, то какой толк
тогда в твоем украинском? Как ты думаешь, Украина, если бы например Леся Украинка перенеслась
машиной времени в наше время, она бы тоже орала кричалки, которые орут фанаты
на матче Украина-Россия?

\ifcmt
  tab_begin cols=3,no_fig,center
     pic https://avatars.mds.yandex.net/i?id=2a00000179f5e312985d9263bd58639efd1b-4613271-images-thumbs&n=13
		 pic https://ye.ua/images/news/poster_photo_title-1496732268.jpeg
		 pic https://www.megakniga.com.ua/uploads/cache/Products/Product_images_474227/0c4864_w600.jpg
  tab_end
\fi

Украинский язык в итоге стал скучным пропагандистским орудием... Он потерял
живительную силу сердца и души, потерял то, благодаря чему он жил и развивался
столетиями, несмотря на все преграды, превратившись в итоге в средство унижения
и ненависти. Парадоксально, но факт. Лучшие произведения на украинском языке
были созданы в те времена, когда Украина не имела своей государственности,
будучи разделенной между империями. Именно в те времена жили и творили Леся
Украинка, Иван Франко, Тарас Шевченко и многие другие. Они не имели никакого
понятия, что означает выражение Крым Наш, ничего не ведали про страну-агрессора
и про электоральные санкции Запада, но, тем не менее их сочинения будут жить
вечно... Получив же независимость, Украина дала государственный статус
украинскому языку, пнув при этом ногой русский. Пинок вначале был
интелигентный, мягкий, на уровне Конституции только; но постепенно удары
становились все жестче и жестче... все наглее и наглее... 

\ifcmt
  tab_begin cols=2,no_fig,center
     pic https://avatars.mds.yandex.net/get-zen_doc/1708669/pub_5cfdc85eaff15000afe57c46_5cfdc898babd4000b092e106/scale_1200
		 pic https://img5tv.cdnvideo.ru/webp/shared/files/202104/1_1298133.jpg
  tab_end
\fi

И чем больше ты, Украина, пинала русский язык, насмехалась над русской
культурой, своей собственной культурой, тем больше ты, Южная Россия, на самом
деле зверела, тем больше ты теряла свою собственную душу, свое сердце, Украйна,
тем больше ты тупела, превращаясь в итоге просто в раздолбанную лодку без весла
и паруса, которую швыряет куда-сюда по просторам информационного океана, разве
не так? И что в итоге получилось на сегодняшний день, вы и так отлично знаете.
С другой стороны... русский-то язык в Украине стал только еще живее, еще
интереснее, еще богаче, чем был во времена имперские и советские, его бьют,
издеваются, а он только крепнет, и пускает все новые побеги, - одна только
Олеся Медведева уже более тысячи видео наваяла, - а если взять еще добавить
Шария... а вот украинский язык - потускнел, поверь, потемнел, почернел...
Знаете, много книжек нынче издается на украинском, можно зайти например в
Книгарню Є... потом, походу заметить где-то в глубине полочку с русскими
книжечками и интеллигентной маркировкой \enquote{Книги Іноземними Мовами},
потом пройти дальше, ибо то, что надо, украинская таможенная служба уже
отправила обратно прямо в Кремль... повыбирать украинских книжек и журналов,
сколько хочешь, - есть и очень хорошие, правда, но также немало просто
пропагандистского мусора...  Увы!  Ложка дегтя таки портит всю бочку с медом!
Держи-мова и генетический стержень расовых украинцев, едрена мать! Слыша сейчас
периодически звуки взрывов над Киевом и вой сирены на второй день войны, ну что
ж, скажем вам так - ваша теория правильных украинцев и умственно неполноценных
диких московитов, отлично работает!  Посеешь ветер, пожнешь бурю!  Хотели
злобных москалей, так получите! Мордор, орду и орков вызывали тысячами, если не
миллионами, комментов и постов по телеграму и фейсбуку? Так вот они уже тут,
прямо за Оболонью. Российско-украинскую войну вызывали? Вот пожалуйста, они уже
тут как тут, резвятся вовсю по всей песочнице, и танки, и десант, и артиллерия,
и бог знает что еще. Люди-то понятливые, хотя и долго запрягают. Вам нравилось
оскорблять русских, весь русский народ, нравилось \enquote{москаляку на
гилляку}, нравилось массово орать \enquote{путин - хйло}, думали, что русские,
от Москвы до Владивостока, просто такие себе безвольные бараны без сердца,
разума и глаз? Вы ж наверное не думали, что это все возможно видит и смотрит
вся Россия, от Кремля и МИДА и до какой-нибудь бабушки в Иркутске?  И вот все
ваши кричалки и материализовались... Анастасия Дмитрик так интересно читала
стихи про небратьев в свое время, вот и получила Настенька, что хотела... 

\ifcmt
  tab_begin cols=3,no_fig,center
     pic https://i2.paste.pics/803d6ba55e25b3932d422a2784a56d62.png
		 pic https://i2.paste.pics/ba6f8da4cf889225670ad6c0a713e79c.png
		 pic https://i2.paste.pics/03b693c3a796d20869846ac43a6ba7cf.png
  tab_end
\fi

Сорри, отвлеклись... продолжая тему... надо признать, что украинские
интеллектуалы, да и просто обычный народ, особенно его \enquote{патриотическая}
часть в массе своей загнали себя в тупик самого слова \enquote{Украина},
разделив свою реальность на Украину и Россию, проукраинские и пророссийские
нарративы, украинцев, малороссов, какаяразница, украина-понад-усе и тому
подобное... загнав поэтому и сам украинский язык в ловушку, в своего рода
информационное гетто, лишив его живительной силы сердца и искренности, разума и
мудрости, чего Украине так не хватает в наше время! А ведь украинский язык же
не появился просто так, - а напротив, он же ж именно такой и есть певучий и
красивый, поскольку в давние времена люди хотели вольно жить на своей земле,
хотели любить и жить в гармонии с природой, - но им приходилось все время
воевать и сражаться за право быть на своей земле, - и отсюда красота языка,
\enquote{смачность} языка,  и также божественная красота украинских песен, -
как отображение внутренних волевых, чистых порывов души и сердца. Вот,
например, такая есть песня с комментариями под видео... 

\href{https://www.youtube.com/watch?v=--iWRp8Jr6k}{%
Українські пісні онлайн. Чом ти не прийшов, youtube}

\raggedcolumns
\begin{multicols}{4} % {
\setlength{\parindent}{0pt}
\obeycr
Чом ти не прийшов,
Як місяць зійшов?
Я тебе чекала.
Чи коня не мав,
Чи стежки не знав,
Мати не пускала?
\smallskip
І коня я мав,
І стежку я знав,
І мати пускала.
Найменша сестра,
Бодай не зросла,
Сідельце сховала.
\smallskip
А старша сестра
Сідельце знайшла,
Коня осідлала:
\enquote{Поїдь, братику,
До дівчиноньки,
Що тебе чекала}.
\smallskip
Тече річенька
Невеличенька,
Схочу - перескочу.
Віддайте мене,
Моя матінко,
За кого я схочу.
\restorecr
\end{multicols} % }

\ifcmt
  tab_begin cols=2,no_fig,center
     pic https://i2.paste.pics/39b5912a65683fffc41863c68bfc5f26.png
		 pic https://i2.paste.pics/dd441b74a982c23a31f5cd313bc89e3d.png
  tab_end
\fi

\begin{itemize} % {
\item Боже, как я люблю украинские песни. Я сразу возвращаюсь в детство, когда все
еще были живы-здоровы, а когда за столом собирались родственники..., украинские
песни лились рекой одна за другой.... как же было здорово!!!

\item Процветания тебе, Украина  @igg{fbicon.heart.sparkling} ... с Любовью и
уважением из России...

\item Люблю Украину и украинские песни. Они самые лучшие, душу выворачивают. Слова и
музыка завораживает. Класс!!!!!!!!!

\item Песня звучит, как звучат струны души у прекрасных и красивый людей. Хочется
слушая, подпевать, хотя  не  всё слова понимаю.  @igg{fbicon.hands.pray} 

\item Я сама русская. Но очень люблю украинские песни.

\item молодцы только украинцы так поют сказочно

\item Слушаю, подпеваю с удовольствием! У меня русско-украинские корни. Люблю Россию,
люблю Украину!!! Господь все управит, все у наших народов будет хорошо!!!
\end{itemize} % }
