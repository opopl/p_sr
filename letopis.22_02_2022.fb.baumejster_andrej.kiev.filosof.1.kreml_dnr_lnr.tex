% vim: keymap=russian-jcukenwin
%%beginhead 
 
%%file 22_02_2022.fb.baumejster_andrej.kiev.filosof.1.kreml_dnr_lnr
%%parent 22_02_2022
 
%%url https://www.facebook.com/andriibaumeister/posts/4787714368016751
 
%%author_id baumejster_andrej.kiev.filosof
%%date 
 
%%tags __feb_2022.putin.priznanie
%%title Признание Кремлем "ДНР" и "ЛНР"
 
%%endhead 
 
\subsection{Признание Кремлем \enquote{ДНР} и \enquote{ЛНР}}
\label{sec:22_02_2022.fb.baumejster_andrej.kiev.filosof.1.kreml_dnr_lnr}
 
\Purl{https://www.facebook.com/andriibaumeister/posts/4787714368016751}
\ifcmt
 author_begin
   author_id baumejster_andrej.kiev.filosof
 author_end
\fi

Признание Кремлем \enquote{ДНР} и \enquote{ЛНР} вызвало бурную реакцию
украинских социальных сетей и очень сдержанную и даже успокаивающую реакцию
украинского президента. 

Посколько меня просят прокомментировать это событие, скажу кратко, в нескольких
тезисах.

1. Несмотря на эмоциональный всплеск со стороны блогеров и соцсетей, \enquote{акт
признания} выглядит скорее, как ожидаемое событие. Заметьте, я говорю \enquote{выглядит
как}. Более понятными становятся слова Байдена о \enquote{небольшом вторжении} за
которым последуют \enquote{меньшие санкции} (см. мой анализ на моей стене). Пойдет ли
Путин дальше? Это вопрос ближайших недель дипломатических переговоров и
закулисных игр. Будем внимательно наблюдать и анализировать. 

Кстати, в отличие от Мюнхенской речи, в которой (по моему личному мнению) было
много ненужной литературщины и много ненужной игры слов (которая все равно
теряется при переводе), вчерашняя речь президента Зеленского была достойной и
на уровне. Это слова зрелого человека. 

2. Кремль все дальше сам загоняет себя в мышеловку. Одно дело - фактически
контролировать буферную зону напряжения, и совсем другое - ее юридическое
признание. Что это дает Кремлю? Возможность все время держать Украину в
напряжении и в состоянии перманентной угрозы. И выглядеть убедительно внутри
России (конечно, в глазах определенных групп населения). Но в стратегическом
плане, я не думаю, что такие шаги усилят Кремль. Скорее, последствия будут
болезненными и в обозримой перспективе разрушительными. По крайней мере, в
политическом плане. 

3. Для Украины появляются дополнительные возможности для маневра. Если вы
внимательно следите за информацией, для вас не секрет, что выполнение Минских
соглашений в последнее время воспринималось как досадный политический тупик.
Вспомните хотя бы недавнее высказывание Данилова: \enquote{Выполнение Минских
соглашений означает разрушение страны} (1.02.22). Например, \enquote{Голос Америки} (и
другие западные СМИ) поместил это высказывание под заголовком: \enquote{Секретарь СНБО
Украины высказался против Минских соглашений}. Я уже не говорю о прямых
высказываниях многих украинских политиков и политических комментаторов по этому
поводу. 

4. На юридическом и политическом уровне \enquote{акт признания} исключает
многозначность оценок. Это явное нарушение международного права и
государственного суверенитета Украины. Западные партнеры официально
отреагировали так, как должны были отреагировать. Сказаны адекватные слова,
сделаны адекватные заявления. Будет ли такой же адекватной экономическая
реакция? 

А вот здесь я не уверен. США продолжат свою логику санкций. О старой Европе
такого сказать нельзя. Если Кремль не пойдет дальше, это станет некоторым
облегчением для Берлина и Парижа. Продолжится двойная игра: резкие политические
заявления будут идти рука об руку с активным экономическим сотрудничеством с
Москвой. У Кремля много механизмов воздействия и кроме экономической сферы.
Например, геополитические козыри (Африка, Ближний Восток, Латинская Америка и,
конечно, Китай). И если Китай уже в этом году активизируется относительно
Тайваня, западной дипломатии придется очень непросто. 

Но, опять же, все будет зависеть от того, остановится ли Путин на признании
псевдореспублик или попытается сделать еще ряд опасных шагов. 

5. Лично для меня подобный \enquote{акт} - удар по моей стране. Украина без Донецка и
Луганска (пусть и временно) - это потеря в культурном и политическом плане (я
даже не говорю здесь об экономике). Сила Украины - в разнообразии, в культурном
и политическом богатстве. Я искренне надеюсь, что такая Украина - это Украина
завтрашнего дня. И этот \enquote{акт} только усилит солидарность внутри нашей страны. И
запустит движение не в сторону \enquote{малой}, а в сторону \enquote{большой} Украины. 

В связи с этим, очень надеюсь, что многие украинские писатели, политики и
\enquote{лидеры мнений} переморят свои оценки Восточной Украины и Донбасса. Не хочу
сейчас приводит длинный список (и по датам) тех откровенных антигуманных и
контр-культурных высказываний, которые позволяли себе некоторые деятели
культуры и некоторые интеллектуалы по поводу будущего восточных регионов (да и
многие политики этим грешили, чего уж там). Не пришло ли время взять свои слова
обратно и проявить солидарность? И кроме гневных речей в адрес Кремля (тут,
живя в Украине, мужества особого не нужно) признать и свои собственные ошибки.
И осознать, что мы - единая страна у которой (я в этом убежден) большое
будущее. Мы все - \enquote{большая} Украина.
