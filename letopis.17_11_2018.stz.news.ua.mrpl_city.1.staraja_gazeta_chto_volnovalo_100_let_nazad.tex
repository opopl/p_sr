% vim: keymap=russian-jcukenwin
%%beginhead 
 
%%file 17_11_2018.stz.news.ua.mrpl_city.1.staraja_gazeta_chto_volnovalo_100_let_nazad
%%parent 17_11_2018
 
%%url https://mrpl.city/blogs/view/staraya-gazeta-chto-volnovalo-mariupoltsev-bolee-100-let-nazad
 
%%author_id burov_sergij.mariupol,news.ua.mrpl_city
%%date 
 
%%tags 
%%title Старая газета: Что волновало мариупольцев более 100 лет назад?
 
%%endhead 
 
\subsection{Старая газета: Что волновало мариупольцев более 100 лет назад?}
\label{sec:17_11_2018.stz.news.ua.mrpl_city.1.staraja_gazeta_chto_volnovalo_100_let_nazad}
 
\Purl{https://mrpl.city/blogs/view/staraya-gazeta-chto-volnovalo-mariupoltsev-bolee-100-let-nazad}
\ifcmt
 author_begin
   author_id burov_sergij.mariupol,news.ua.mrpl_city
 author_end
\fi

\ii{17_11_2018.stz.news.ua.mrpl_city.1.staraja_gazeta_chto_volnovalo_100_let_nazad.pic.1}

Давным-давно ушли в мир иной люди, встречавшие Новый, 1900 год. Стерты с лица
земли православные храмы и костел, украшавшие некогда Мариуполь. Обветшали
дома, бывшие в те времена новостройками, город перенес пожар, учиненный
гитлеровцами при отступлении в начале сентября 1943 года. А газеты – хрупкие,
пожелтевшие листы бумаги, жизни-то которым было отмерено день-другой, остались.
Конечно, не все: матушка-история тщательно отсевает зерна от плевел. Вот
\textbf{восемьдесят восьмой номер \enquote{Мариупольского справочного листка} за 11 августа
1900 года} дошел до наших дней. Дошел, потому что хранился в семейном архиве
Даниловых сто восемнадцать лет. Сейчас он находится у доцента Приазовского
государственного технического университета, кандидата технических наук Сергея
Сергеевича Данилова.

Дело в том, что на второй странице \enquote{Мариупольского справочного листка}, о
котором здесь идет речь, была напечатана заметка, касающаяся городской
больницы, а в ней лестный и, как свидетельствуют другие источники, вполне
заслуженный отзыв о ее заведующем - докторе И. И. Данилове - дедушке Сергея
Сергеевича. Вот почему так бережно обращались со старой газетой и передавали ее
из поколения в поколение. На рубеже XIX и XX веков в Мариуполе Илью Ильича
Данилова знали и стар и млад. И как прекрасного врача, и как одного из
организаторов городской больницы (сейчас это 3-я горбольница), и как
многолетнего главного врача этого учреждения. Не пожалеем времени и приведем
здесь строки из упомянутой заметки: \emph{\enquote{Доктор И. И. Данилов – образцовый труженик.
Его энергии и добросовестности в исполнении принятых им на себя обязанностей
нет предела. Но непомерный труд может источить энергию, сокрушить идеально
могучие нравственные силы. Городская управа должна понять это и дать
заведующему больницей доктору помощника – другого доктора}}. Дали
самоотверженному врачу помощника или – нет, нам, к сожалению, установить не
удалось.

\textbf{Читайте также:} 

\href{https://mrpl.city/blogs/view/blagodijna-diyalnist-teatralnih-pratsivnikiv-mariupolya}{Благодійна діяльність театральних працівників Маріуполя, Ольга Демідко, mrpl.city, 13.11.2018}

А сейчас давайте полюбопытствуем, о чем еще писала мариупольская газета конца
XIX века. Кстати, сообщим, что ее первый номер вышел в свет 15 января 1899
года. Первая и последняя страницы четырехполосного издания формата нынешнего
\enquote{Приазовского рабочего} были целиком и полностью отданы рекламе. Мариупольцам
предлагалось \enquote{пильзенское пиво} наследников Боте из Екатеринослава. В
мануфактурном магазине Левина - остатки товаров со скидкой 30\%. У Кипмана на
Екатерининской улице - велосипеды, пишущие машинки \enquote{Ремингтон} и швейные машины
\enquote{Зингер}, весы \enquote{Фербенкс}, множительные аппараты Эдисона, столы-бюро \enquote{Дерби},
несгораемые кассы, мимеографы (так в старину называли известные нам ротаторы) и
многое другое, увы, как и сейчас, все больше импортного производства.
Справедливости ради нужно привести объявление и отечественного производителя:
\emph{\enquote{Антон Бабек: земледельческие орудия и машины, грузовые дроги, брички, арбы,
экипажи, чугунное и медное литье, устройство и постановка балконов, памятников,
решеток и прочее}}...

Мариупольская таможня объявляла торги огнеупорного кирпича и стали. Кто-то
пытался провезти контрабанду: не удалось. Анонсировались концерты
симфонического оркестра под управлением М. Я. Свердлова в Александровском парке,
- на месте этого некогда зеленого уголка нашего города с середины 30-х годов XX
века и до наших дней возвышаются 43-й и 45-й дома по проспекту Мира. Оркестр
маэстро Свердлова состоял из тридцати музыкантов, в том числе пять солистов:
скрипач, виолончелист, арфист, гобоист и кларнетист. Между прочим, большинство
оркестрантов были выпускниками консерваторий: Московской или
Санкт-Петербургской. Обещалось, что они выступят в специально отведенном для
этого симфоническом вечере. Кто еще выступал с этим оркестром - неизвестно.

Не пожалело денег на рекламу страховое общество \enquote{Россия}. Предлагалось
\emph{\enquote{страхование отдельных лиц от всякого рода несчастных случаев; страхование
пассажиров на железных дорогах и водяных путях; коллективное страхование от
несчастных случаев служащих в правительственных и общественных учреждениях;
коллективное страхование рабочих на фабриках, заводах и постройках}}. Прошло сто
восемнадцать лет, а виды страховых услуг почти не изменились, как и расходы на
рекламу. Интересно, как выполняло свои обязательства перед клиентами это
страховое общество - так же, как некоторые нынешние?

В магазине \enquote{часов, золотых и серебряных вещей и музыкальных инструментов} А. В.
Гуровича можно было купить всевозможные часы, табакерки, серебряные сервизы и
музыкальные инструменты - от балалайки и мандолины до флейты и тромбона, от
пианино и скрипки до цитры и бубна. В ассортименте магазина его конкурента по
реализации музыкальных инструментов и нот Адольфа Валериановича Гаслинского не
было часов и сервизов, но были картины, рамы, багеты и \enquote{всегда свежие струны},
которые отсутствовали у Гуровича. Вот было раздолье для музыкантов – любителей
и профессионалов. Хуже было тем, кто занимался в Мариуполе живописью: за
материалами приходилось ездить в Ростов-на-Дону в магазин красок Жан-Левина
(его реклама присутствует в газете). Зато какое разнообразие красок: масляные в
трубочках, акварель жидкая и сухая, акварель в ящиках ценой от 12 коп. до 4
рублей, гуашевые в стеклянных банках, прозрачные - для живописи по стеклу.
Впрочем, не так уж и сложно было попасть в Ростов, если любишь искусство: по
воскресеньям, вторникам и четвергам из Мариупольского порта в Ростов отплывал
пароход. Можно было туда добраться и по железной дороге, но с пересадками, как
и сейчас.

\ii{17_11_2018.stz.news.ua.mrpl_city.1.staraja_gazeta_chto_volnovalo_100_let_nazad.pic.2}

Что же волновало мариупольскую общественность в последний год? Почти то же, что
и нас. Плохое освещение улиц и то, что их скверно и нерегулярно метут и
по­ливают, грязь на пляжах и купание там собак. Сообщалось, что ведутся
переговоры об устройстве в Мариуполе элект­рического трамвая. Но, как показало
будущее, на переговорах дело и закончилось. Заметим, что трамвай появился
только в советское время, в начале тридцатых годов.

\textbf{Читайте также:} 

\href{https://archive.org/details/05_05_2018.sergij_burov.mrpl_city.mariupolskij_tramvaj}{Мариупольский трамвай, Сергей Буров, mrpl.city, 05.05.2018}

Газета как газета: объявления (\emph{\enquote{молодой человек за умеренную плату готовит в
средние учебные заведения}}), реклама (о ней говорилось выше), фельетон
(\enquote{Мариупольская цирюльня}), стихи - не очень профессиональные, но зато на злобу
дня. Например, такие:

\begin{center}
{\em
Во дни, когда Восток объят грозою,

Когда пришла тревожная пора,

Во весь свой рост ты встала предо мною,

О, русская подвижница-сестра!..

Я вижу вновь твой образ грустно-бледный

И красный крест я вижу на груди...

И знаешь ты, что ждет тебя твой бедный

Страдалец-брат... Иди к нему, иди!} и т.д.
\end{center}

Но вот на что особенно обращаешь внимание: в справочном отделе - такса на хлеб
и мясо, утвержденная \enquote{господином Начальником губернии}. И это на подъеме
капиталистической эпохи! В разгар самых что ни на есть рыночных отношений
мариупольская городская дума, ссылаясь на статью 63 Городового положения,
оказывается, жестко регламентировала цены на продукты. Каковы же были цены?
Фунт (410 г) белого хлеба стоил 5 копеек, а фунт говядины - 8 копеек. То ли
хлеб был дорог, то ли мясо дешево? Для сравнения: по данным известного
краеведа, доцента Ждановского металлургического института Дмитрия Николаевича
Грушевского в это время неквалифицированные рабочие только что построенных близ
станции Сартана металлургических заводов \enquote{Никополь} и \enquote{Русский Провиданс}
получали в среднем за двенадцатичасовой день 60 - 80 копеек. Во что же
обходилась пища духовная? Например, номер того же \enquote{Мариупольского справочного
листка}? От двух с половиной до четырех копеек: дешевле, если подписаться на
год, дороже, если на три месяца. Вот и выбирайте - или полфунта мяса, или
городские новости.

Как всякая уважающая себя газета, \enquote{Мариупольский справочный листок} имел раздел
криминальной хроники. Два \enquote{леденящих душу} преступления в Мариуполе
зафиксировал репортер тех лет: восьмого августа у обывательницы украли на рынке
кошелек с тремя рублями, а четвертого - некая \enquote{дама в шляпе} пыталась там же
стащить у поселянки... курицу! Вот бы нашей полиции бороться с такими
\enquote{преступлениями}.

Жаль, что под материалами газеты вместо подлинных фамилий стоят псевдонимы
(\enquote{Обыватель}, \enquote{Бес-Арабский}, \enquote{Печорин}, \enquote{Свирель}, \enquote{Р.Л}), а другие вовсе
лишены подписей. Кто эти дотошные репортеры, развязные фельетонисты, всезнающие
обозреватели мариупольской прессы на рубеже двух веков? Для истории осталась
лишь фамилия П. Н. Фалькевича - редактора и издателя листка. Известен также
владелец типографии, в которой печаталось это периодическое издание – это Л. Я.
Шпарбер. Сохранился еще и адрес редакции газеты: в доме адвоката Ильи
Эммануиловича Юрьева напротив часовни - значит, в том здании на углу проспекта
Мира и Греческой улицы. В том самом, где до революции был устроен первый в
Мариуполе кинотеатр... Но правильнее, наверно, будет на этом остановиться в
изложении истории приметного здания и просто назвать его современный адрес –
проспект Мира, 40.

\textbf{Читайте также:} 

\begin{minipage}{0.9\textwidth}
\href{https://mrpl.city/blogs/view/prospekt-miru-podorozh-u-chasi}{%
Маріупольський проспект Миру: подорож у часі, Ольга Демідко, mrpl.city, 25.10.2018}
\end{minipage}

Прошли бурное XX столетие, мировая война, революция, затем война гражданская,
на левом берегу Кальмиуса вырос гигант-завод, огненным смерчем по Мариуполю
прошлась Вторая мировая война. Изменились город и люди, живущие в нем, и
сегодняшние городские газеты приобрели почти такой же вид, если не внешне, то
по своим темам и обилию рекламы, какой имел \enquote{Мариупольский справочный листок} в
последний год XIX века. Колесо истории сделало еще один оборот.
