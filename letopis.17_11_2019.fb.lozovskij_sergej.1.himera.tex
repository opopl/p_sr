% vim: keymap=russian-jcukenwin
%%beginhead 
 
%%file 17_11_2019.fb.lozovskij_sergej.1.himera
%%parent 17_11_2019
 
%%url https://www.facebook.com/permalink.php?story_fbid=2810559962311773&id=100000733907860
 
%%author_id lozovskij_sergej
%%date 
 
%%tags etnogenez,etnos,gumilev_lev,himera,obschestvo
%%title Химера - КАК ЖЕ ЭТО ВСЕ УЗНАВАЕМО
 
%%endhead 
 
\subsection{Химера - КАК ЖЕ ЭТО ВСЕ УЗНАВАЕМО}
\label{sec:17_11_2019.fb.lozovskij_sergej.1.himera}
 
\Purl{https://www.facebook.com/permalink.php?story_fbid=2810559962311773&id=100000733907860}
\ifcmt
 author_begin
   author_id lozovskij_sergej
 author_end
\fi

КАК ЖЕ ЭТО ВСЕ УЗНАВАЕМО

Химе́ра (др.-греч. Χίμαιρα — «Химера») — в пассионарной теории этногенеза
этническая форма и продукт контакта несовместимых (имеющих отрицательную
комплементарность) этносов, принадлежащих к различным суперэтническим системам,
в результате выросшие в химере люди утрачивают этническую традицию и не
принадлежат ни к одному из контактирующих этносов. Химеру можно
охарактеризовать как общность деэтнизированных, выпавших из этносов людей. В
химере господствует бессистемное сочетание несовместимых между собой
поведенческих черт, на место единой ментальности приходит полный хаос царящих в
обществе вкусов, взглядов и представлений, в такой среде расцветают
антисистемные идеологии. Потеря своеобразных для каждого этноса адаптивных
навыков приводит к отрыву населения от кормящего ландшафта. В отличие от этноса
химера не может развиваться, а способна лишь некоторое время существовать,
впоследствии распадаясь — происходит своего рода этническая аннигиляция.
Возникшие в недрах химеры антисистемы выступают, как правило, инициаторами
кровопролитных конфликтов, либо химера становится жертвой соседних этносов.

Химера может существовать в теле здорового этноса, подобно раковой опухоли,
существуя за его счёт и не выполняя никакой конструктивной работы. При этом она
может быть относительно безвредной (пассивной) либо же становиться рассадником
антисистем (агрессивная химера).

Большинство известных в истории этнических химер сложились за счёт вторжения
представителей одного суперэтноса в области проживания другого, после чего
агрессор пытался жить не за счёт использования ландшафта, а за счёт
побеждённых. Результатом в конечном итоге всегда был распад и гибель химеры,
так как победители деградировали не в меньшей степени, чем побеждённые.

Термин введён Л. Н. Гумилёвым.

\ii{17_11_2019.fb.lozovskij_sergej.1.himera.cmt}
