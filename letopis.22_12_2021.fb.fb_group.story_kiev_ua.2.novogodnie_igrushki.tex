% vim: keymap=russian-jcukenwin
%%beginhead 
 
%%file 22_12_2021.fb.fb_group.story_kiev_ua.2.novogodnie_igrushki
%%parent 22_12_2021
 
%%url https://www.facebook.com/groups/story.kiev.ua/posts/1824420474421412
 
%%author_id fb_group.story_kiev_ua,levickaja_natasha.kiev
%%date 
 
%%tags igrushka,novyj_god
%%title НОВОГОДНИЕ ИГРУШКИ
 
%%endhead 
 
\subsection{НОВОГОДНИЕ ИГРУШКИ}
\label{sec:22_12_2021.fb.fb_group.story_kiev_ua.2.novogodnie_igrushki}
 
\Purl{https://www.facebook.com/groups/story.kiev.ua/posts/1824420474421412}
\ifcmt
 author_begin
   author_id fb_group.story_kiev_ua,levickaja_natasha.kiev
 author_end
\fi

НОВОГОДНИЕ ИГРУШКИ

Ностальгия по детству и теплые воспоминания о новогодних семейных праздниках
вызывают у нас желание хотя бы ненадолго вернуться в прошлое, в детство!
Наверное, поэтому у нас  такое трепетное отношение к старым новогодним
игрушкам.

Все они  хранят бесценную память  о чём-то  очень  близком. О тех, кто был
рядом с нами в те далёкие времена - вот наши молодые папа и мама, которые
украшают с нами ёлку, наши бабушки и дедушки, запах хвои и мандарин,
предвкушение и ожидание чуда! 

\begin{multicols}{2} % {
\setlength{\parindent}{0pt}

\ii{22_12_2021.fb.fb_group.story_kiev_ua.2.novogodnie_igrushki.pic.1}

\ii{22_12_2021.fb.fb_group.story_kiev_ua.2.novogodnie_igrushki.pic.2}
\ii{22_12_2021.fb.fb_group.story_kiev_ua.2.novogodnie_igrushki.pic.2.cmt}

\ii{22_12_2021.fb.fb_group.story_kiev_ua.2.novogodnie_igrushki.pic.3}
\ii{22_12_2021.fb.fb_group.story_kiev_ua.2.novogodnie_igrushki.pic.3.cmt}

\ii{22_12_2021.fb.fb_group.story_kiev_ua.2.novogodnie_igrushki.pic.4}
\ii{22_12_2021.fb.fb_group.story_kiev_ua.2.novogodnie_igrushki.pic.4.cmt}

\ii{22_12_2021.fb.fb_group.story_kiev_ua.2.novogodnie_igrushki.pic.5}
\ii{22_12_2021.fb.fb_group.story_kiev_ua.2.novogodnie_igrushki.pic.5.cmt}

\end{multicols} % }

Конечно, времена меняются и  последние годы в украшении наших ёлок стал
преобладать современный стиль - шары, атласные банты...

И безусловно, монохромные шары и атласные банты на ёлке – это красиво! Но это
холодная красота, а в старых игрушках теплая и душевная. Они из прошлого и
хранят дух того далёкого времени, когда мы верили в новогодние чудеса!

Мы давно уже повзрослели, возмужали, огрубели душой, взгляды на жизнь стали
циничнее, однако в каждом из нас взрослом осталось что-то детское, и  где-то в
далеком уголке нашей души все ещё теплится маленькая надежда на чудо, которое
все мы ждём накануне каждого Нового года!

\begin{zznagolos}
\obeycr
\enquote{Старые игрушки... просто раритет.
Облупились где-то, сколько же им лет?
Кукуруза, летчик, часики, луна...
Дед Мороз, хлопушки, бусы, бахрома...
Шишки и фонарик, звездочка внутри.
Вот олень картонный, и рога... целы!
Сказочное царство в доме у меня.
Это - детство... Память... Целая страна!
Есть, конечно, лучше..., но не заменить
То, что было в детстве. А в душе... щемит.}
\restorecr
\end{zznagolos}

(И. Воробьёва)
