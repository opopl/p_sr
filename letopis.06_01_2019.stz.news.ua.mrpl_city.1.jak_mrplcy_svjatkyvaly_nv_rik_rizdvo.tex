% vim: keymap=russian-jcukenwin
%%beginhead 
 
%%file 06_01_2019.stz.news.ua.mrpl_city.1.jak_mrplcy_svjatkyvaly_nv_rik_rizdvo
%%parent 06_01_2019
 
%%url https://mrpl.city/blogs/view/yak-mariupoltsi-ranishe-svyatkuvali-novij-rik-i-rizdvo-hristove
 
%%author_id demidko_olga.mariupol,news.ua.mrpl_city
%%date 
 
%%tags 
%%title Як маріупольці раніше святкували Новий рік і Різдво Христове?
 
%%endhead 
 
\subsection{Як маріупольці раніше святкували Новий рік і Різдво Христове?}
\label{sec:06_01_2019.stz.news.ua.mrpl_city.1.jak_mrplcy_svjatkyvaly_nv_rik_rizdvo}
 
\Purl{https://mrpl.city/blogs/view/yak-mariupoltsi-ranishe-svyatkuvali-novij-rik-i-rizdvo-hristove}
\ifcmt
 author_begin
   author_id demidko_olga.mariupol,news.ua.mrpl_city
 author_end
\fi

З нагоди новорічних свят, коли маріупольці вже поринули в атмосферу
святкування, я вирішила дізнатися, як раніше мешканці Маріуполя зустрічали
Новий рік і святкували Різдво Христове. Моїми респондентами стали п'ятеро
маріупольців: \textbf{\em Добриця Єфросинія Петрівна (88 років), Єрьоменко Антоніна
Петрівна (81 рік), Сильченко Ніна Дмитрівна (79 років), Агарков Віктор Якович
(68 років) та Мінайлов Микола Семенович (73 роки)}. Вони люб'язно розповіли про
те, як готувалися і святкували новорічні свята 30 чи 50 років тому.

\ii{06_01_2019.stz.news.ua.mrpl_city.1.jak_mrplcy_svjatkyvaly_nv_rik_rizdvo.pic.1}

Так, до Нового року та Різдва раніше готувалися доволі довгий час. А все тому,
що було важко дістати продукти, все необхідне купували за кілька місяців і
дбайливо зберігали до потрібного моменту. Сьогодні важко собі це уявити, але
щоб дістати основні інгредієнти, наприклад, салату олів'є, потрібно було сильно
постаратися: у вільному продажу не було майонезу, зеленого горошку – запасатися
починали з жовтня. \emph{Антоніна Петрівна} розповіла, що з великими труднощами
діставали і основний напій свята – \enquote{Радянське} шампанське. Інколи просили
допомоги у знайомих, чи домовлялися з продавцями заздалегідь. Загалом, щоб
\enquote{дістати} продукти, потрібно було вистояти в годинних чергах, отримати в
продуктових замовленнях шпроти, ікру, копчену ковбасу. Обов'язковими стравами
на святковому столі були олів'є, холодець, заливна риба, морквяний і буряковий
салати, оселедець під шубою, мариновані огірки і помідори. Солодкого,
наприклад, тортів, у продажі не було (чи це був справжній дефіцит), тому
здебільшого доводилося пекти самим. Всі маріупольці згадують про шалені черги,
які були складовою підготовки до новорічних свят.

\ii{06_01_2019.stz.news.ua.mrpl_city.1.jak_mrplcy_svjatkyvaly_nv_rik_rizdvo.pic.2}

\textbf{Читайте також:} \emph{Свято наближається...}%
\footnote{Свято наближається..., Ольга Демідко, mrpl.city, 24.12.2018, \par%
\url{https://mrpl.city/blogs/view/svyato-nablizhaetsya}
} %
\footnote{Internet Archive: \url{https://archive.org/details/24_12_2018.olga_demidko.mrpl_city.svjato_nablyzhajetsja}}

\ii{06_01_2019.stz.news.ua.mrpl_city.1.jak_mrplcy_svjatkyvaly_nv_rik_rizdvo.pic.3}

\emph{Микола Семенович} розповів, як виглядали раніше костюми. Так, раніше на
новорічні свята одягалися в національні костюми чи костюми водолазів,
космонавтів, які сьогодні вже не зустрінеш. \emph{Єфросинія Петрівна} згадує, що
знайти модну сукню в магазині теж було завданням складним, тому, як правило,
молоді дівчата собі шили самі. Причому підходили до цього завдання дуже
відповідально, доволі довгий час обирали потрібний фасон, шукали тканину і
необхідний декор, щоб виглядати чарівно. Це ж стосувалося і зачіски, адже
салонів краси не було і все доводилося робити самим, чи звертатися за допомогою
до подруги.

\textbf{Читайте також:} \emph{Украл ли Санта Рождество?}%
\footnote{Украл ли Санта Рождество?, Яна Іванова, mrpl.city, 06.01.2019, \url{https://mrpl.city/blogs/view/ukral-li-santa-rozhdestvo}}

Квитки на Новорічну ялинку для дітей теж було дістати не так легко, як і
подарунки, які також були дефіцитом. \emph{Ніна Дмитрівна} наголосила, що найчастіше
жінкам дарували парфуми, а чоловікам одеколони. Прямо як у всім знайомому
фільмі \enquote{Ирония судьбы, или С легким паром!} 

\ii{06_01_2019.stz.news.ua.mrpl_city.1.jak_mrplcy_svjatkyvaly_nv_rik_rizdvo.pic.4}

\emph{Віктор Якович} згадує, що дуже любили раніше дивитися на новорічні свята
передачу \enquote{Голубой огонек}, фільм \enquote{Карнавальная ночь}, чи мультик \enquote{Морозко}.

Ялинки маріупольці теж прикрашали, але іграшки були іншими. Так, ялинкові
іграшки робили з паперу, вати та інших матеріалів, прикрашали і цукерками.
Іграшки зі скла з'явилися наприкінці 1960-х років, як і пластикові, на яких
можна було зустрінути радянську символіку.

Антоніна Петрівна зазначила, що загалом Новий Рік був найбільш очікуваним
святом і до нього починали готуватися навіть влітку, адже саме у новорічну ніч
можна було всіх вітати, приходити до сусідів в гості, ділитися казковим
настроєм і, незважаючи на безліч труднощів, все ж таки вірити, що дива існують
і наступний рік буде набагато кращим...

\textbf{Читайте також:} \emph{Мариуполец из пластиковых отходов сделал яркое праздничное шоу}%
\footnote{Мариуполец из пластиковых отходов сделал яркое праздничное шоу, Яна Іванова, mrpl.city, 06.01.2019,\par\url{https://mrpl.city/news/view/mariupolets-iz-plastikovyh-othodov-sdelal-yarkoe-prazdnichnoe-shou-foto-plusvideo}}
