%%beginhead 
 
%%file 28_10_2022.fb.galushko_denis.1.mi__dine_do_chogo_ne
%%parent 28_10_2022
 
%%url https://www.facebook.com/HalushkoDenis/posts/pfbid02fSAkcYBitNu63L3qqYBzYj5yPjapDyUaMFScJ2YYe5gp1Qbp71YxaPUGZ6yZkMCEl
 
%%author_id galushko_denis
%%date 28_10_2022
 
%%tags 
%%title Ми єдине до чого не можемо звикнути це діти
 
%%endhead 

\subsection{Ми єдине до чого не можемо звикнути це діти}
\label{sec:28_10_2022.fb.galushko_denis.1.mi__dine_do_chogo_ne}

\Purl{https://www.facebook.com/HalushkoDenis/posts/pfbid02fSAkcYBitNu63L3qqYBzYj5yPjapDyUaMFScJ2YYe5gp1Qbp71YxaPUGZ6yZkMCEl}
\ifcmt
 author_begin
   author_id galushko_denis
 author_end
\fi

\#запискидилетантазсу

Десь 1,5 місяці може більше. Не фіксую, в месенджерах таймери на зникнення
повідомлень.  Бахмут.

Ми постійно рухаємося, змінюємо позицію, шукаємо нові, двіжуємо. 

Ми звикли вже майже до всього, ми бачимо, що відбувається довкола, ми бачимо і
фіксуємо кого кладуть обличчям до низу, щоб не фіксувати руки.  

Наші точки евакуації, це чиїсь кінцеві крапки в чиємусь житті. 

Звикаєш, розумієш, хтось заходить в сектор чистим, хтось приїхав вже з пікапом
у крові з під Харкова. 

Але виходять всі однаково, так непереможені, так незламні, так, але не всі.

Немаю болю, немає страху, немає сил в хлопців, їх нема, фіксувати щось зайве,
тільки робота, яка полягає в тому щоб знищити ворога і вберегти своє життя.

Я бачу їх порожні очі, це виснаження.

Деяким достатньо 2 -годин сну і він на коні, в деяких цей стан просто
хронічний.  

Всі підтримують один одного, жартом, сарказмом, кавою. 

Є довбойоби, що гризуть тебе за твою і свою втому, за свою безхребетність і
слабкість, їх не багато та на них вже одноманітно і байдуже, ні хто не звертає
увагу, а як перейдуть лінію, підуть за кораблем, вони це відчувають, не
переходять.

До чого я це, до суті, це така військова лірика, одного дилетанта на війні.

Але тема є, нам тут немає часу займатись цивільними. 

Нещодавно, я прочитав, ніби, як здається автору, приголомшену статтю, «хлопчик
в Бахмуті, був евакуюваний поліцейськими які ризикували своїм життям рятуючи
його після того як 152- й снаряд вбив його батьків»

«Ризикували» -  мені це слово порізало слух, мій внутрішній слух, до
поліцейських взагалі без притензій, інший контекст.

Тут приходи, кожен день 24/7, місто давно вже примара, нічого майже не
працює, світло, вода, десь, іноді щось, деякі будинки по старій схемі готують
по черзі на вогнищі чи буржуйці у підвалі, щоб не вбило. А сидіти в квартирі
безпечно, а готувати їжу у підвалі, де логіка?

Ми єдине до чого не можемо звикнути це діти.

Я, автомат, 8-маг ак, кевларовий шолом, бронежилет, розгрузка, гранати, такмед,
мультітул, ніж, ліхтар, хавка, турнікети, рація, додаткові бокові пакети броні,
це так треба, я військовий, я на війні і не один, нас мінімум двоє або група.

Поруч зі мною троє дітей просто в штанцях і куртках на велосипедах на краю
Бахмута, біля тої промки яку нещодавно звільнили, катаються.

На іншому краю вулиці, поки ми дійшли до точки нашого слідування 120-мм
влучанням в огород, вбило чоловіка, а його доньку забрали військові медики.

«Місто, привид, в якому гуманітарна катастрофа» - це просто пафосні слова для
чергового депутата довбойоба який приїде фоткатися на в’їзді з табличкою
Бахмут, коли ми просунемося далі, але чи можно буде тоді врятувати цих дітей,
очікуючи цей час, я не впевнений. 

Я військовий, не червоний хрест, не рятівник року, не супермен, не
гуманітарна місія, не юнісеф, не фонд «який продає лате в кафе Київа щоб
зібрати кошти на корм і врятувати собачок із Бучі» - я не проти цієї
ініціативи, але мені здається діти важливіші.

Врятуйте наших дітей, заберіть звідси дітей.

Їх батьки довбодятли, я з ними спілкувався, діагноз встановити легко, а діти їх
заручники.

Діти не в чому не винні не кидайте їх тут на призволяще. 

Сподіваюсь, що хтось дотичний прочитає і перенесе це в правильну дієву голову.

Це відбувається не тільки в Бахмуті, це відбувається по всій лінії зіткнення
і поки ти не тут, ти не розумієш масштаб катастрофи.

Без них в нас немає майбутнього, ми наче їх зрадили. 

А на фото просто школа, а в центрі встояла стіна з написом «Добро...», робіть
добро, шерти, кричить, зверніть увагу на цю проблему, а нам вибачте, воювати
треба.

\#ОксанаЖолнович відкрийте свою сторінку і очі!
\#міністерствосоціальноїполітикиУкраїни
