% vim: keymap=russian-jcukenwin
%%beginhead 
 
%%file 19_01_2022.yz.italomania_i_gastromania.1.etruski_russkie.cmt
%%parent 19_01_2022.yz.italomania_i_gastromania.1.etruski_russkie
 
%%url 
 
%%author_id 
%%date 
 
%%tags 
%%title 
 
%%endhead 
\zzSecCmt

\begin{itemize} % {
\iusr{Кушнаренко Дмитрий}

Русская нация, или великороссы, начала складываться после Батыева нашествия и
окончательно оформилась где-то в 15-16 веках, т.е. через много столетий после
исчезновения этрусков. Так что не надо подражать нашим небратьям и пытаться
выкопать Средиземное море.

У нашей нации достаточно своих, всепризнанных достижений.

Другое дело, возможная связь этрусков с южными славянами... но это,
действительно, должно быть доказано научными методами, может быть и
генетическими исследованиями, если таковые возможны.

Автору - спасибо.

\begin{itemize} % {
\iusr{Андрей Кондратьев}

Кушнаренко, нам ещё предстоит выкопать своё море? @igg{fbicon.beaming.face.smiling.eyes} 
\iusr{Александр Григорьевич}

Кушнаренко, \enquote{Русская нация, или великороссы, начала складываться после Батыева
нашествия и окончательно оформилась где-то в 15-16 веках} - не согласен.

Русская нация начала формироваться в древнем Великом Новгороде и Пскове 9-11
веках.

\iusr{Евгений Б}

Кушнаренко, на самом деле, одно ведь другому не мешает, толькотмух от котлет
отделить надо. Этрусски-рим-греки....тут взаимосвязь и культурные
щаимствования, я думаю, споров вызывать не будут? Кирилицу греки принесли, если
не ошибаюсь, ну во всяком случае, греческте ученые мужи приложили свои руки.
Вполне ВЕРОЯТНО, что люди и впрям были образованные и кое какие заимствования
для нашего языка пришли от них, но врятли наоборот)) а все остальное, как то
что мы, якобы, можем прочесть и все такое....типа родня там далтняя, думаю эту
простоткому то хочется капельку исторического величия)) но это лишь моя думка
такая)

\iusr{Александр Ром}

Кушнаренко, исследуйте названия городов, провинций и рек в Европе, да и в
Африке: Ес-пани-ойла, Барсе-лона, Лон-дон, Сток-хольм (холм), Пар-ис, Сена и
т.д. и т.п....

\iusr{Василий Фролов}
\textbf{Александр Ром},

Хочу добавить ещё один интер.ресный испанский городок.. на севере Африки -
Мелилья,.. т.е. мелованый.. А знаете кАк.. в древности он назывался?
Руссодир!!!? Мол, с глубокой древности основан..-. Феникийцами. Да? А кто
такие эти феники? Да это хВенекийцы, или проще говоря Венеды/Венеты..

Но больше всего мена умиляет это князь Куева - Дир или другое имя.. Диррос!? А
теперь сравните это имя с.. Мелильей в древности.. Соответствия не видите
разве?

\iusr{灰狼}
\textbf{Александр Ром}, 

это римские названия на латинском лондиниум, медиоланум и т. д. искажённые
словоблудами фоменковцами. до нашей эры русского языка не было.

\iusr{алексей смирнов}

Кушнаренко, если привлечь гипотезу Фоменко о лживости термина \enquote{темные века} и
учесть существование государства Русов на Дунае в средневековье то тогда
элемент фантастики исчезнет.

\iusr{Газ Р. Андрей}
\textbf{Александр Григорьевич}, 

Выселок славян в Новгороде ильменьском был уничтожен Иваном 4 под корень.
Русские это Мордва, Меря, Чудь, и т. д. Фино- угры. Руськие это славяне.

\iusr{灰狼}
\textbf{Евгений}, кирилицу создали православные монахи проповедники, болгары по
происхождению, кирилл и мефодий. за основу был взят греческий алфавит. они же
перевели молитвы с латинского на славянский язык (старо болгарский). отче наш
иже еси на небеси - это староболгарский язык.

\iusr{Вячеслав Бо}
\textbf{Александр Ром}, а вслушайтесь в словоа кат-ях? Жо-па? Ну французский же?
\iusr{flusspferd-17}
Василий Фролов, дeбил, финикийцы - это семиты, т.е. родственники современных жыдоевреев и арабов.

\iusr{Евгений Б}
\textbf{灰狼}, 

не спорю, но учение греческое, так что заимствования вероятны с высркой
степенью. Я просто к тому, что этл в наш язык заимствовали, а не из него) я не
лингвист-историк) поверхностное понимание и лошика вот наше все @igg{fbicon.beaming.face.smiling.eyes} 

\iusr{Александр Ром}
\textbf{灰狼}, странно, логику не поймали, как будто и не русский Вы парень... Может быть, и олива тоже римское слово, а если вот так напишем: ойл-ива, дерево ива встречали, есть и шаровидная, и плакучая, и имя Иван от него пошло... Если логику продолжить то: Ойл-имп, Ие-рус-ойл-им, Пале-стена и т.д. и т.п.

\iusr{灰狼}
\textbf{Александр Ром}, 

совсем чтоли на почве клесовщины. иван это еврейское имя иона. известно из
библии. палестина это римская провинция сириа филистима на месте современного
израиля.

\iusr{Александр Ром}
灰狼, так Вы, братец, еврей!!! Откуда Вы взяли, что имена Иван и Иона связаны?
А почему Вы уверены, что Иван произошло от Ионы, а может быть, Иона от Ивана?!
Если название города Ие-рус-ойл-им является древнерусским, то еврейские
историки сильно заблуждаются по поводу истории своего народа... Советую Вам
тренировать аналитический аппарат!!!

\iusr{Alexander kiselev}
\textbf{Андрей Кондратьев}, когда то и Черное море, называлось Русским

\iusr{Alexander kiselev}
\textbf{灰狼}, ну да, обьяснялись берестеными граматами и на пальцах.

Есть языки древние и есть языки новые, вы их легко определите по количеству
слов в языке. Если словарный запас маленький то и язык молодой. Русский язык
самый большой по запасу слов ( это не могло взяться из воздуха) по количеству
букв тоже самый большой. Сравните с греческим, он уступает во всем и при этом
считаетя самым древнем, мне кажется что это абсурд. Если я не прав, растолкуйте

\iusr{Вася Пупкин}
\textbf{Александр Григорьевич}, оба вы ..... хороши!

Нации как явление, понятие начали складываться после наполеоновских войн.

До того были подданные т.е. присягу дал и всё.
\iusr{Вася Пупкин}

Кушнаренко, Нации как явление, понятие начали складываться после наполеоновских войн.

До того были подданные т.е. присягу дал и всё.

\iusr{Mugadin Dudaew}

Кушнаренко, а как же славянство?

\iusr{Моисей Ахметов}
\textbf{Александр Григорьевич}, это была древнерусская общность, так и не ставшая нацией
в постоянных междуусобицах. А русская нация, действительно, стала складываться
после нашествия Батыя. Хотя, разумеется, без древнерусского периода не было бы
последующих.  

\iusr{Александр Григорьевич}
\textbf{Моисей Ахметов}, \enquote{так и не ставшая нацией в постоянных междуусобицах} -
междуусобицы не являются признаком отсутствия нации. Междуусобицы бывают и в
обычной семье между кровными братьями, что не отменяет их кровного родства и
принадлежности к одной семье.

\iusr{Ничевушка с лапами}

Кушнаренко, а как же переселение готов, славян в 300-500гг.?
\iusr{Дмитрий Димас}

Кушнаренко, добавь hdd в голову, а то памяти явно недостаточно чтобы дальше 15 века запомнить

\iusr{灰狼}

Кушнаренко, словяне это прежде всего языковая группа. тирены же говорили на
языке. у которого в европе родственных нет.

\iusr{Николай Логинов}
\textbf{Кушнаренко}, \enquote{нации} были придуманы в 16 веке, во времена реформации. Само слово
nazi, - это искажённое руское слово \enquote{наши}. Nazi, т.е. \enquote{наши} назывались
реформаторами те народы, которые им удавалось отколоть от Великой Руси. Т.наз.
\enquote{руССкая нация} - это остатки от массового и повсеместного, романовского
геноцида РУских в 18-19 вв.

\iusr{Медея Полевской}

灰狼, только не Иона, а Иохан (Иоан, Йоган, Жан, Джоан)
\iusr{灰狼}
\textbf{Медея}, это всё производные от иудейского имени иона, которые переиначили на различные языки
\iusr{灰狼}

\textbf{Газ Р. Андрей}, в 11 веке русь начала заселять земли на северо-востоке, мокшанские болота т. н. залесскую русь, коренным народом там были разные финно-угорские племена. с каждым годом доля славянского населения увеличивалась, постепенно славяне стали большинством. южные княжества постоянно разорялись кочевниками. большая масса народа переселилось с юга на северные земли, тем самым усилив их. залесская русь стала центром силы и наиболее богатой территорией на руси. владимирско-суздальские князья стали считать себя главнее киевских.

\iusr{Татьяна Тонина}
\textbf{Кушнаренко}, да что вы!

Русские были самыми первыми людьми на Земле

\end{itemize} % }

\iusr{Тамара Галкина}

Буквы похожи, потому что в основе этрусского и греческого, а значит, и русского
(пришедшего из Византии) алфавита лежит финикийский. Кстати, писали этруски
справа налево, еще и поэтому их никак не могли расшифровать. Этим занимались
многие, в том числе Вяч.Вс. Иванов.

\end{itemize} % }
