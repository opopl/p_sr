% vim: keymap=russian-jcukenwin
%%beginhead 
 
%%file songs.tin_soncja.pisnja_chugajstera.bel
%%parent songs.tin_soncja.pisnja_chugajstera
 
%%endhead 

\subsubsection{Беларусский}

У спрадвечных пушчах, дзе ня згасны зык балота,
Дзе лягла Батыя Чорна-Залатая шмота,
Часам на сьцяжынку зачарованы выходзіць
Лесавік-Чугайстар. І ціха так заводзіць:

Калі можаш, завітай да мяне ты,
Калі хочаш, паспытай маёй вады,
Калі верыш, дык мне шчыра раскажы,
Пра што ты мроіш, дзеля пазбыцца самоты...

Брамы павуціньня ад сасны і да бярозы –
Варта вось чаканьня, што пільнуе сэнс цьвярозы.
Палымнее рута, жыцьцядайна расквітае,
Ды ніхто ня бачыць! Ніхто таго ня знае!

Стогне хіжым плачам нат дрыгва на ўсім Палесьсі,
Попел пазакратаў лісьце у драўлянскім лесе.
Безнадзейна квола пахілілася таполя,
Дзьме трывожны вецер з атручанага поля.

Што мы прысьнілі, болей не вярнецца –
Казкі і мроі нібы ў дым.
Занадта рана супыніла сэрца,
Шкада памерці маладым.

Мой голас зь неба чорны крук даставіць,
Сплыве па Прыпяці вянок.
Чакайце вестак пад пахмурным Сонцам,
Пад зорак расьсяваны змрок.
