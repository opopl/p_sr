% vim: keymap=russian-jcukenwin
%%beginhead 
 
%%file 22_12_2021.tg.bilchenko_evgenia.1.dlinnaja_noch
%%parent 22_12_2021
 
%%url https://t.me/bilchenkozhenya/5375
 
%%author_id bilchenko_evgenia
%%date 
 
%%tags 
%%title БЖ. Эта самая, самая, длинная ночь
 
%%endhead 
\subsection{БЖ. Эта самая, самая, длинная ночь}
\label{sec:22_12_2021.tg.bilchenko_evgenia.1.dlinnaja_noch}

\Purl{https://t.me/bilchenkozhenya/5375}
\ifcmt
 author_begin
   author_id bilchenko_evgenia
 author_end
\fi

Сегодня мне предложили покаяться за свое прошлое ещё раз. Ничего более пошлого
я не слышала. Каяться в России - это не уважать Россию, не любить ее, пытаться
пристроиться, приспособиться, понравиться. Это же ужасно! Как по мне, от матери
легче нести наказание, пусть незаслуженное или с переусердием, но никак не эти
танцы. 

Я уважаю Россию, я люблю её настолько сильно, что считаю дешевкой к ней
приспосабливаться. Поэтому я покаялась ещё на Украине, где совершенное покаяние
стоило мне потерь, а не приобретений. И покаялась, в первую очередь, перед
Украиной, иначе зачем? Иначе - грош цена такому покаянию. Тогда это выходит
какой-то грех: каяться ради выгоды?

\ii{22_12_2021.tg.bilchenko_evgenia.1.dlinnaja_noch.pic.1}

Ничего не сотру из злого, простите, те, кто меня поддерживает, хочу принять
все, что выпало. Смотрю на свой организм со стороны. Если можете, репостните
украинским, именно украинским, СМИ и Фейсбуку, которые копипастят мою страницу
ВК по грязным помойкам злорадства, что это - далеко не картина России. Люди
сознательно работают на воссоздание облика тьмы. Работают невольно в
коллаборации с украинскими безумцами. Россия - не такая, в ней очень много
света. Это правда. Чистая правда. Мне вы должны верить.

Вчерашней ночью я не спала (перестала спать) и неожиданно поняла, что я не
просто прощаю травящих меня людей, я их понимаю и принимаю.

В 2014 году произошло... Чтобы понять, что произошло с большинством некогда
нормального населения Украины, мгновенно возненавидевшего своего собрата ни за
hren собачий, до Крыма, наконец, до Донбасса, надо сначала почувствовать
изнутри, а потом посмотреть со стороны. Так легче диагностировать. Суть
безумия, любого безумия, в том, что оно воспринимается как откровение Бога. Оно
наполняет пространство таинством, мгновенно отбирающим у человека мысль и
свободу и наполняющим его какой-то сладкой, пьяной и рьяной, разрушительной и
разрушающей идентичностью. Это произошло со многими жителями Украины. Увы. Это,
возможно, необратимо.

Находясь в политически адекватной, нейтральной среде, которой является Россия,
это безумие со стороны выглядит уже не увлекательным и не адски пугающим, а
смешным. Да, именно так, дьявольски смешным, что, конечно, ещё ужаснее.

Когда Украина вошла в пространство ненависти, Россия остановилась в немом
оцепенении. Этого здесь не ждали. Этого невозможно было представить. Ведь
Россия осталась в состоянии до безумия-2014. Для большинства русских людей
внезапно вспыхнувший огонь фантазма, разожженный на дровах травмы, выглядел,
как сын-подросток, который, как-то искривлено и патологически взрослея, вынул
нож и пырнул им мать. Мать удивлённо посмотрела на него...

Сначала. А потом пришел ответ, и он был криком ярости. Волна ненависти породила
ответную волну ненависти, и кровь за кровь, рана за рану, слово за слово,
смерть за смерть, абсурд за абсурд. 

Я понимаю тех, кто от всей души не прощает меня в России, потому что они не
прошли моего пути, для них мои страдания - не существуют или они заслужены. Они
ничего не искупают. Они как бы невидимы и не в счёт, потому что прошлое
перевешивает всё, кроме смерти. Так в их глазах выглядит. Может, и саму мою
смерть.

Я также понимаю тех, кто нападает на меня на Украине, потому что я показала
сыну нож, которым он пырнул мать. Я на него указала напрямую. Сын, находясь в
галлюцинаторном бреду, полагал, что его калечит собственная мать. Эту отнятую
идентичность, попытку отнять порочную картинку и посмотреть в глаза своему
преступлению против себя же, Украина мне не простит. Та Россия, что не прощает,
тоже никогда не простит. Это надо принять как часть покаяния и искупления. Я
все это знала, когда делала то, что делаю.

Люди, в принципе, в быту - везде хорошие. Изо всех сил пишут друзья-украинцы:
мол, возвращайся из тьмы. Приходят подержать за руки друзья-россияне: мол, не
возвращайся во тьму.

Получается, что путь из тьмы во тьму мы придумали себе сами, как самую длинную
ночь столетия. Это путь краха русского мира, русской цивилизации, которую рвут
на части с одной и с другой стороны на радость нашим утонченным заморским
соседям. Нас не надо специально бомбить, расширять НАТО. Мы перережем друг
друга сами.

Так выглядит ад. И мы с ним сейчас стоим лицом друг к другу: ад и моя яростная
упрямая нежность, и я не знаю, кто кого. Это последняя битва. Штрафбат должен
быть искуплен и завершён чем угодно. Смиряюсь перед концом, люблю, прощаю,
прошу прощения.
