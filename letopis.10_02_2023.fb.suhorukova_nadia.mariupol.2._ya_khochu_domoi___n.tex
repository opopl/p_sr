%%beginhead 
 
%%file 10_02_2023.fb.suhorukova_nadia.mariupol.2._ya_khochu_domoi___n
%%parent 10_02_2023
 
%%url https://www.facebook.com/permalink.php?story_fbid=pfbid0dphiNSyjNjy9688CgLdzAHPNNhvKvNbDbNRkPazTFTEtEZy2u5TrC9WuPpH2GAcol&id=100087641497337
 
%%author_id suhorukova_nadia.mariupol
%%date 10_02_2023
 
%%tags mariupol
%%title "Я хочу домой". Нужно такое тату
 
%%endhead 

\subsection{\enquote{Я хочу домой}. Нужно такое тату}
\label{sec:10_02_2023.fb.suhorukova_nadia.mariupol.2._ya_khochu_domoi___n}

\Purl{https://www.facebook.com/permalink.php?story_fbid=pfbid0dphiNSyjNjy9688CgLdzAHPNNhvKvNbDbNRkPazTFTEtEZy2u5TrC9WuPpH2GAcol&id=100087641497337}
\ifcmt
 author_begin
   author_id suhorukova_nadia.mariupol
 author_end
\fi

\enquote{Я хочу домой}. Нужно  такое тату. Всё равно где. Если поможет вернуться, то и
на лбу. 

Одна  женщина, у которой погиб сын, сказала, что мы, выбравшиеся из Мариуполя,
\enquote{покалеченные, изувеченные уроды и не можем жить прежней жизнью}. 

Мы, пока, вообще не можем жить. 

Мы как растения, которые вырвали из земли. 

Корни остались в почве и они ещё выглядят как живые, но каждый день вянут всё
сильнее. 

Мы делаем вид, что живём. 

Объяснить как это  - невозможно. 

Кто не испытал - не напрягайтесь. 

Узнать боль через слова не получится. 

Кажется, ты внутри жизни: смотришь, удивляешься, даже смеёшься. 

Говоришь другим, что тебе легче.  

На самом деле врёшь. Всем. И  себе в том числе. 

Это ненастоящие чувства и жизнь ненастоящая. 

Настоящая осталась в городе, из которого тебя выгнали бомбами, выкурили
пожарами, выдолбили снарядами. 

И даже те, кто сейчас там,  в Мариуполе, они тоже не живут. 

Иначе, откуда  отчаяние и боль до неба? 

"1 вересня пройшло ... 

Вперше, після 35 років, я не працюю... 

Така пустота на серці, бо відчула себе не потрібною в цьому світі. 

23 лютого в мене було все і хотілося жити, працювати...

24 лютого життя обірвалося... 

Клята війна відібрала мою душу, відібрала сенс життя. 

Любий ліцей, любий Маріуполь залишилися тільки у пам'яті"

У нас ностальгия по Мариуполю. Он   остался в другом измерении.  

Тот,  который сейчас - не наш город. 

Нашего в нём больше ничего нет.

Может быть только могилы близких и цветы в палисадниках, если их не разбомбили. 

Кто-то ходит к психологам. И пытается вернуться в жизнь.

Пройти реабилитацию и не засохнуть до смерти. 

У кого-то отрастают новые корни. Они тоненькие, как пух и не могут зацепиться в
земле. 

Мы всё равно  продолжаем вянуть. 

Учим чужой язык и придумываем повод не плакать. 

Ходим по незнакомым  улицам и ищем в них черты своих родных и любимых улочек и
переулков. 

Учимся кататься на велосипеде и представляем, как проедем на нём,  по своему
двору там, в Мариуполе. 

Разглядываем чужие достопримечательности и говорим: \enquote{А у нас гораздо красивее}. 

Потом добавляем: \enquote{Было. До войны}

Ночью накатывает страх. 

Самый ужасный кошмар с начала войны. 

Ночью я боюсь, что больше никогда не вернусь в свой город. 

Мое \enquote{я уехала ненадолго} -  превращается в \enquote{я уехала навсегда}

Этот текст сохранила прекрасная Оксана. И вернула его мне. Спасибо за то, что
мы вместе. Что не теряем друг друга, а значит не потеряем и наш  город. 

А фото Вити  Дедова. У него недавно был День рождения. Он тоже водолей, как и
я. Витя погиб почти год назад. 11 марта.  А фотографии его живут. На них
Мариуполь, которого больше нет.

%\ii{10_02_2023.fb.suhorukova_nadia.mariupol.2._ya_khochu_domoi___n.cmt}
