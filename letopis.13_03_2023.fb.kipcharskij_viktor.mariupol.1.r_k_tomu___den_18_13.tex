%%beginhead 
 
%%file 13_03_2023.fb.kipcharskij_viktor.mariupol.1.r_k_tomu___den_18_13
%%parent 13_03_2023
 
%%url https://www.facebook.com/permalink.php?story_fbid=pfbid0KMbKtUJHdLqpEw7y28cQtRHLsuh35EDSUZ9bVhQpChh5JcSdLohJ4kkc9eL2VtLUl&id=100006830107904
 
%%author_id kipcharskij_viktor.mariupol
%%date 13_03_2023
 
%%tags mariupol,mariupol.war,dnevnik
%%title Рік тому:  День 18-13.03.32.  Неділя
 
%%endhead 

\subsection{Рік тому:  День 18-13.03.32.  Неділя}
\label{sec:13_03_2023.fb.kipcharskij_viktor.mariupol.1.r_k_tomu___den_18_13}

\Purl{https://www.facebook.com/permalink.php?story_fbid=pfbid0KMbKtUJHdLqpEw7y28cQtRHLsuh35EDSUZ9bVhQpChh5JcSdLohJ4kkc9eL2VtLUl&id=100006830107904}
\ifcmt
 author_begin
   author_id kipcharskij_viktor.mariupol
 author_end
\fi

Рік тому: 

День 18-13.03.32.  Неділя

Спав до півночі - з вечора заткнув вуха ватою.

Зараз дуже гучно та інтенсивно гупає. Схоже, є прильоти.

\ii{13_03_2023.fb.kipcharskij_viktor.mariupol.1.r_k_tomu___den_18_13.pic.1}

4:00. Прокинувся під залпи. Гупало довго. Потім червоне сяйво за доменними і
повітряна хвиля так вдарила по будинку, що вікно вигнулося і штовхнуло повітря
у кімнату: на столі зашурхотіли папери. На кухні підстоибнули кришки на
каструлях. Потім настала тиша і лише хвилин за 20 гупнув поодинокий постріл.

4:30. Гупає у звичному режимі.

6:30. Грію чайник та бідон. Воду з чайника попереливав у термоси, наповнив його
наново. Взяв чайник у Едіка, пішов кіп'ятити.

На Гонди за 42-м лежить снаряд, що не вибухнув. Схоже, що його розвернули гілки
дерев і він гупнув у стіну боком, потім відскочив і впав на дорогу. Хтось побіг
у госпіталь і повідомив військових. Згодом під'їхали поліцейські, порадили
позначити снаряд, аби хтось його не "потурбував" і чекати на саперів.

Близько сьомої пройшла чутка, що на Нептуні записують на роздачу їжі. Син побіг
туди. 

До багаття підійшов хлопчик років 10-12. Він продавав сигарети "Стронг" по 150
гривень 4 пачки по 5 сигарет. Наталчин Сергій обміняв кілограм борошна на вісім
таких пачок. Останню велику пачку на 25 сигарет хлопчик віддав за 100 гривень
йому дали 200, він, даючи решту витягнув пачку грошей завтошки сантиметри 3!
Бізнес йде дуже добре: хтось зранку займає на Нептуні та біля адміністрації
чергу, чекає доки привезуть сигарети, купує, а потім хлопчик розпродує їх у
роздріб.


Повернувся син. Йому дали номери 1463-1468 - це наші номери на реєстрацію.
Написали кульковою ручкою на руці, як у чергах 90-х років. Сказали прийти по
12-й.

Занесли чайники і пішли на Нептун. Стали у чергу. Сказали, що будуть щось
давати дітям до 6-ти років. Син пішов за невісткою та онукою. Дали печиво:
сказали взяти з великої коробки, але небагато. Син дістав з файлу А4 свідоцтва
про народження і поклав у той файл печиво. Дають тільки на дитину: чоловік
показував свідоцтво про народження дитини, пояснював, що в дитини вітрянка, але
їсти вона хоче як здорова - печива йому так і не дали.

Я стояв майже до 15-ї. За цей час двічі привозили вантажівкою медичне
обладнання з вцілілих приватних клінік: апарати УЗД, гінекологічні крісла,
тощо. Допомогав розвантажувати. 

Окрім черги за їжею, стояли люди з дітьми на прийом у педіатрів: лікарі
приймають у басейні. 

\ii{13_03_2023.fb.kipcharskij_viktor.mariupol.1.r_k_tomu___den_18_13.pic.2}

Кілька разів починали стріляти міномети: таке враження, що стріляли від парку
Петровського або з кварталів і міни з неприємним свистом літали просто над
головами. Люди з черги розбіглися: один хлопець ліг на асфальт під півметровим
парканчиком, а його дівчина стояла поруч - скоріше за все, не хотіла бруднити
пилюкою пуховика.  Дивлячись на нього люди почали лягати просто на асфальт. Я,
може теж ліг би, але не був певний, що зможу підвестися - від довгого стояння
нога заніміла, кілька разів її хапала судома. 

Здавалося, що міни падають десь поруч: я кілька разів бігав дивитися, чи не у
наш двір - начебто ні, падало десь у приватному секторі ближче до Покришкіна.
Уламок міни поцілив у стіну 46-го будинку з боку 42-го. Хтось схопив "сувенір"
і, оскільки пекло пальці, запхав у кишеню. З кишені пішов дим.

На час реконструкції Нептун огородили парканом з листів металу (зараз той
паркан почали розбирати заради дерев'яних брусків, на яких кріпили листи. З
боку котельні були ворота, зварені з арматури: до Нового року їх прикрасили
ялинковим "дощиком". Хлопець з дівчиною ховалися за цим "захистом".

Нарешті дійшла черга: записали на видачу їжі у суботу, 19-го на 23:25. Дали
паперовий талончик. Хлопці на вході, що слідкували за порядком, зірвали голоси,
викрикуючи номери, та наводячи лад.

Поки стояв, почув багато новин: на підвіконні першого поверху басейну близько
10 діб лежала мертва бабця із будинку на розі: її загорнули у простирадло в
поклали там, аби пси не дістали.

На колишньому футбольному полі за церквою біля адміністрації вже 6 могил.
Комунальні служби прислали допомогати рити могили двох робітників із ...
"дитячою лопаткою". Вони просили лопати у мешканців.

Старший чоловік розповідав, як він поїхав провідати тещу  на Лівому, та
довелося її вивозити, бо там вже були суцільні руїни: вона сиділа у підвалі.
Повернутися через Мухіна не вдалося - там був бій. Поїхав через Пост-міст - на
Набережній пробив колеса - приїхав на дисках.

Дуже змерзл, особливо ноги та аж знайшлися руки хоч і у перчатках.

Вирішив зігрітися алкоголем - вперше з початку "гарячої" війни.

В шафі "Адам" знайшов сливову настоянку; в коридорній шафі - ще одну пляшку з
абрикосовою. Шукав щось міцніше.

Відкрив джин, який у 2015-му році мені подарували на день народження колеги.
Хоч би встигнути випити все це до від'їзду...

17:25. Варили соус: картопля на зажарці з моркви та цибулі. Кип'ятити воду собі
та Едіку - він поклав у свій чайник пару яєць.

Несподівано до вогнища підійшов сусід по гаражах Ігор: ми обійнялися на
радості. Наші гаражі не пограбували і не розбомбило, а от у сусідній ряд
прилетіло (чи не тиждень тому, коли я зливав бензин, бо з того часу я погано
чую на праве вухо?). Гаражі охороняють двоє охоронців, які постійно там живуть.
Ігор та інші сусіди носять сторожам їжу.

Ігор попав під обстріл на Кіровському; сгоріла половина будинку з Орбітою та
повністю сгорів наступний. Розбили школу 28 (де вчився я та вчилися сини) та
дитячий садочок за нею (коли ми їхали, вікна у школі були цілі, крім одного -
мабуть залізли мародери). Садочка не було видно від дороги. Побито будинки по
Блажевича по той бік проспекту Металургів, але будинок сватів цілий.

Після вечері (я вечеряв, бо не обідав) переказував онукам про хлопчика Крака та
Вавельского дракона-смока. Онуки сиділи у мене на дивані, загорнуті у ковдру і
слухали: онук вже надцятий раз - колись це була його улюблена історія на ніч.
Онучка згадала: " А ти хіба не пам'ятаєш (така в неї з'явилася звичка), що
світив і зелене було?" - вона згадала про люмінісцентний лак, яким я фарбував
саморобні приманки для риби.

Вперше онука вислухала розповідь до кінця - напевно не хотіла вилазити з-під
теплої ковдри.

Гумконвой не може виїхати з Бердянська.

Фото сина

1,2 -  снаряд на дорозі.

3 - дрова у під'їзді

%\ii{13_03_2023.fb.kipcharskij_viktor.mariupol.1.r_k_tomu___den_18_13.cmt}
