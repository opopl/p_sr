% vim: keymap=russian-jcukenwin
%%beginhead 
 
%%file 30_11_1867.maksimovich_ma.1.rossia_malorossia
%%parent 30_11_1867
 
%%url 
 
%%author_id 
%%date 
 
%%tags 
%%title 
 
%%endhead 

\subsection{Об употреблении названий Россия и Малороссия в Западной Руси.}
\label{sec:30_11_1867.maksimovich_ma.1.rossia_malorossia}

\url{https://zapadrus.su/bibli/2012-09-28-20-56-09/1365-kl-2.html}

«Не очень давно было толкование о том, будто Киевская и вся западная Русь не
называлась Россией до ее присоединения к Руси восточной; будто и название Малой
России или Малороссии придано Киевской Руси уже по соединении ее с Русью
великою или Московской. Чтобы уничтожить навсегда этот несправедливый и
нерусский толк, надо обратить его в исторический вопрос: когда в Киеве и в
других западно-русских областях своенародные имена: Русь, Русский начали
заменять, по греческому произношению их, именами Россия, Российский?

Ответ: с девяностых годов 16-го века, в правление короля Жигимонта III-го, то
есть, вскоре после того, как земля Киевская и все княжество Литовское были
присоединены к Польше, на Люблинском сейме 1596 года.

Основанием такого ответа служат тоговременные акты, письменные и книги,
печатанныя в разных областях Русских, присоединенных к Польше. 

Приведу свидетельства тех и других.

1) Вот первая книга, напечатанная в Киеве, в типографии Печерской лавры –
Часослов, 1617 года. В предисловии к ней иеродиакона Захария Копыстенскаго
сказано: «Се, правоверный христианине и всяк благоверный читателю, от нарочитых
мест в России Кийовских, сиречь лавры Печерскиа»...

2) В послесловии к Анфологиону, изданному 1619 года, типограф Памво Берында
говорит к читателю о лавре Печерской: «се убо приносит ти... матерь твоя в
России Малой».

3) В Поучениях св. Дорофея, напечатанных 1628 г., лаврский наместник Филофей
Казаревич с братией говорят, что сия книга в общую пользу «великаго и славнаго
народа Российскаго и прочих народов пресладчайшаго языка славенска».

4) Основательница Киевскаго Богоявленскаго братства, Анна Гулевичевна Лозкина,
в своей записи о том 1615 г., говорит, что она учреждает его — «правоверным и
благочестивым христианом народу Российскаго, в поветех воеводств Киевскаго,
Волынскаго и Брацлавскаго будучим».

5) В «реестре Братском», тогда же начатом, сказано: «начинаем сие
душеспасительное, друголюбное братство цер­ковное в Богоспасаемом граде
Киеве..… на утешение и утверждение в благочестии нашему Российскому роду» и т.
д.

6) А вот еще стих из Виршей ректора Киево-братской школы, Кассиана Саковича, на
погребение гетмана Петра Конашевича-Сагайдачнаго, напечатанных 1622 года:
«Зоставивши теды нам святыню в России».

7) Окружная грамота, напечатанная в Киеве 1629 года, начинается так: «Иов
Борецкий, милостию Божиею архиепископ Киевский и Галицкий и всея России, всем
посполито Российскаго рода, так в короне Польской, яко и у великом князстве
Литовском, всякого достоинства, духовнаго и светцкого, высокого шляхетного и
низшого посполитого стану людем»…

Но довольно о Киеве; обратимся к земле Галицкой.

8) Там Львовское братство в своей типографии прежде всего издало Грамматику,
1591 года, в наставление «многоименитому Российскому роду». В ней упоминается о
пришествии патриарха Иеремии «во страны Российския»; митрополит Киевский и
Галицкий Михаил (Рагоза) именуется «архиепископом всея России».

9) Того же 1592 года Львовское братство обращалось в Москву к царю Феодору
Иоанновичу, с просительными посланиями, в которых именуют его «светлым царем
Российским», вспоминают «князя Владимира, крестившаго весь Российский род» и т.
п.

10) Минуя другие акты и книги, назову еще Октоих, изданный - во Львове 1630
года; там сказано в посвящении о братстве – «в граде Леондополи Малыя России».

В земле Волынской находим тоже.

11) В числе книг, напечатанных в Остроге, известна книга Василия великаго (о
постничестве), изданная 1594 года. В ее предисловии встречается такое
выражение: «вы же, о православный Российский народе»!

12) В местечке Рохманове, принадлежавшем княгине Ирине Вишневецкой, Кирилл
Транквилион-Ставровецкий напечатал 1619 года свое Евангелие учительное. В одном
предисловии говорит он: «ревностию поревновах о спасении братии моей, рода
моего Российского». В другом предисловии встречается выражение: «по всей земли
Российской».

Такое же употребление имен «Россия, Российский» было тогда и на северо-западе
Русском.

13) В столичном городе Литовского княжества, Вильне, где была долго и
резиденция митрополитов Киевских, Михаил Рагоза первый из них стал писать в
своем титуле: «всея Росии» или «всея России», – как это видно из подлинных
актов 1590–1599 годов. Так продолжал писать и его преемник, униатский
митрополит Ипатий Потей, в 1600–1608 годах. А прежние митрополиты Киевские,
бывшие до Михаила Рагозы, писали «и всея Руси» или «всея Русии». Так писали в
своем титуле и Московские митрополиты, бывшие до учреждения патриаршества в
Москве. Первый патриарх Московский Иов писал уже «и всея Росии» (в 1586–1589
годах).

14) В предисловии к Евангелию учительному патриарха Каллиста, напечатанному в
1616 г. в Евю, коштом Богдана Огинскаго, сказано, что Русский перевод этой
книги – "выданьем з друку, на все широкии славного и старожитного народу
Российского краины разослан»...

15) Жители города Ратна, в 1614 году писали к преемнику Потея, Иосифу Руцкому,
о непринуждении их к унии: «отче архиепископе всего Российского языка»!

Но возвратимся к Киеву. По возобновлении здесь православной митрополии в 1620
году, митрополиты Иов Борецкий, Исаия Копинский, Петр Могила продолжали писать
в титуле: «и всея России».

Приведу несколько выражений из писаний Петра Могилы:

16) В его предисловии к Анфологиону, изданному 1636 года, читаем: «святого
великого Российского князя Владимира» – «и всей церкви православной Российской»
– «презацный народ Российский».

17) В его огласительном листе к Луцкой братии на Киевском соборе 1640 г.,
сказано: «от немалого часу вся православная церковь Российская, будучи от
апостатов в великом преследованю, не могла прийти до того, абы волков драпежных
отдалити от себе».

18)  В грамоте патриарха Феофана, писанной в генваре 1621 года, при отъезде его
из Киева, сказано: «смирения нашего благословение всем благочестивым христианом
в Малой России….. сыном церкви Российския восточныя». ...
