% vim: keymap=russian-jcukenwin
%%beginhead 
 
%%file 18_07_2021.fb.fb_group.fond_maje_znachennja.1.kamin_v_dome_kupca_goffa
%%parent 18_07_2021
 
%%url https://www.facebook.com/groups/maieznachennia/posts/375939227383899
 
%%author_id fb_group.fond_maje_znachennja,fedorovskij_jaroslav.mariupol
%%date 
 
%%tags 
%%title Камин в доме купца Гоффа. Проспект Мира, 19 (Мариуполь)
 
%%endhead 
 
\subsection{Камин в доме купца Гоффа. Проспект Мира, 19 (Мариуполь)}
\label{sec:18_07_2021.fb.fb_group.fond_maje_znachennja.1.kamin_v_dome_kupca_goffa}
 
\Purl{https://www.facebook.com/groups/maieznachennia/posts/375939227383899}
\ifcmt
 author_begin
   author_id fb_group.fond_maje_znachennja,fedorovskij_jaroslav.mariupol
 author_end
\fi

Камин в доме купца Гоффа. 

Проспект Мира, 19 (Мариуполь) 

\url{https://goo.gl/maps/xu5avBbK9JzwFQFAA}

\ifcmt
  ig https://i2.paste.pics/PJY3J.png?trs=1142e84a8812893e619f828af22a1d084584f26ffb97dd2bb11c85495ee994c5
  @wrap center
  @width 0.9
\fi

Кстати, информация для тех, кто недавно к нам присоединился или ещё не знаком с
инициативой ребят из Архі-Місто \textbackslash Архи-Город.\footnote{\url{https://www.facebook.com/arximisto}}

Ребята второй год исследуют Мариупольский Некрополь и проводят титаническую
работу над поиском памятников людей, которые были причастны к строительству
Мариуполя. 

Группе нужна финансовая и человеческая поддержка, интерес со стороны
общественности и желание помочь сохранить историю для потомков.

\ifcmt
  ig https://i2.paste.pics/PJY5G.png?trs=1142e84a8812893e619f828af22a1d084584f26ffb97dd2bb11c85495ee994c5
  @wrap center
  @width 0.9
\fi

Цель - проводить экскурсии в
\href{https://www.facebook.com/groups/278185963354519}{Мариупольский Некрополь:
убираем, исследуем,
восстанавливаем},\footnote{\url{https://www.facebook.com/groups/278185963354519}}
рассказывать о судьбах людей и учиться современному поколению брать
ответственность за своё прошлое.

P.S. Камин находится в доме семьи, чьи памятники ребята нашли, очистили и
привели в \enquote{божеский вид}. 

По всем вопросам поддержки, экскурсий, обращайтесь к Андрею Марусову.\footnote{\url{https://www.facebook.com/andrei.marusov}}
