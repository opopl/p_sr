% vim: keymap=russian-jcukenwin
%%beginhead 
 
%%file 11_01_2022.fb.fb_group.story_kiev_ua.1.mesta_sily
%%parent 11_01_2022
 
%%url https://www.facebook.com/groups/story.kiev.ua/posts/1837615959768530
 
%%author_id fb_group.story_kiev_ua,bergelson_aleksandr.kiev
%%date 
 
%%tags kiev
%%title МЕСТА СИЛЫ... ЧУВСТВА И ЗНАКИ...
 
%%endhead 
 
\subsection{МЕСТА СИЛЫ... ЧУВСТВА И ЗНАКИ...}
\label{sec:11_01_2022.fb.fb_group.story_kiev_ua.1.mesta_sily}
 
\Purl{https://www.facebook.com/groups/story.kiev.ua/posts/1837615959768530}
\ifcmt
 author_begin
   author_id fb_group.story_kiev_ua,bergelson_aleksandr.kiev
 author_end
\fi

МЕСТА СИЛЫ... ЧУВСТВА И ЗНАКИ...

Знаете... Я таки скажу... За себя...

Я все-таки живу чувствами и верю в знаки... Я точно знаю, что ничего вокруг
просто так не происходит... И если в мою голову таки что-то попало, так это
оттуда не вышибить ничем...

\ii{11_01_2022.fb.fb_group.story_kiev_ua.1.mesta_sily.pic.1}

Я не христианин... И уж тем более не православный... Меня и иудеем-то
правоверным сложно назвать... В синагоге я бываю от силы два-три раза в год... И
один из них тогда, когда нужно купить мацу перед Песахом... Ну вот не близки
мне религиозные темы... Избави Б-г – я этим не горжусь и не кичусь... Просто это
не мое... Как по мне – есть во всем этом какая-то ненатуральность... 

Хотя я и не атеист... Понимаю, осознаю и верю в существование некой
сверхъестественной силы – не могу объяснить какой – следящей и управляющей...
К которой я (говорю только о себе) могу обращаться и с которой возможно
общение на каком-то своем уровне и найденном или понятом  общем языке... 

Это очень свое... Очень глубокое... И совершенно неважно – где и когда это
происходит... Вот возникла потребность... Появилось желание... И происходит...

\ii{11_01_2022.fb.fb_group.story_kiev_ua.1.mesta_sily.pic.2}

Давно хотел побывать в Выдубецком монастыре... Был там однажды... Давно и
недолго... Очень понравилось место... Спокойное и какое-то величественное... Есть
там вот это ощущение Места Силы... Одного из... Бередящего и чистящего душу...
Открывающего чувства и эмоции... И собирающего тебя в кучку... Как пазл...

А тут вот вдруг в один момент все звезды стали как надо... И Рождество позади...
И Шабат закончился... И время есть... И желание... И к разговору готов... И
церквушка в Михайловском на втором этаже, которую – говорят – крайне редко
открывают, оказалась открыта... А ей почти тысяча лет... И даже служба была в
соборе, которой я вообще никогда в жизни не видел... Никакой!.. 

\ii{11_01_2022.fb.fb_group.story_kiev_ua.1.mesta_sily.pic.3}

Ну – о самой службе конечно не мне судить... Не совсем по мне этот пафос и
взгляд в толпу, но мимо всех... Опять же – это исключительно мои наблюдения
и выводы... А вот хор был действительно хорош... 

В общем – все получилось... Как нельзя лучше...

А после – еще и театр добавил чувств... И тут тоже – все одно к одному... И
место – еще одно - Силы... И антураж... И пьеса... И актеры... И люди... И пойманные
моменты, упавшие и запавшие в душу...

\ii{11_01_2022.fb.fb_group.story_kiev_ua.1.mesta_sily.pic.4}
\ii{11_01_2022.fb.fb_group.story_kiev_ua.1.mesta_sily.pic.5}
