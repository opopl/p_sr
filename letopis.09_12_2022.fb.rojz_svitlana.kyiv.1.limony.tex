% vim: keymap=russian-jcukenwin
%%beginhead 
 
%%file 09_12_2022.fb.rojz_svitlana.kyiv.1.limony
%%parent 09_12_2022
 
%%url https://www.facebook.com/svetlanaroyz/posts/pfbid032NTdV4myXPp5CbikRrwB2bA14ga7Lz7WKpzLHXXQQT52Vd1JeYfBTmNeoUvNYcGRl
 
%%author_id rojz_svitlana.kyiv
%%date 
 
%%tags 
%%title Коли життя підкидає тобі лимони - зроби з них лимонад
 
%%endhead 
 
\subsection{Коли життя підкидає тобі лимони - зроби з них лимонад}
\label{sec:09_12_2022.fb.rojz_svitlana.kyiv.1.limony}
 
\Purl{https://www.facebook.com/svetlanaroyz/posts/pfbid032NTdV4myXPp5CbikRrwB2bA14ga7Lz7WKpzLHXXQQT52Vd1JeYfBTmNeoUvNYcGRl}
\ifcmt
 author_begin
   author_id rojz_svitlana.kyiv
 author_end
\fi

Сьогодні думала про фразу: \enquote{коли життя підкидає тобі лимони - зроби з них
лимонад}. Це прекрасно про зовнішні обставини. А якщо ти сам вже, як цей лимон,
що давно пішов на лимонад? І, навіть, шкірка вже нарізана на цукати.

Фаза війни, яку ми проживаємо, має назву - фаза виснаження (чи розчарування). І
багато хто так і каже: я вижатий - вичавлений, я виснажена.

У знайомих та клієнтів я питаю - а хто видавлював \enquote{сік}, чи в чиїх руках ваш
лимон чи апельсин? В ваших, чи комусь \enquote{його} передали. А як будете себе
почувати, якщо уявите, що \enquote{свій лимон} повертаєте собі. А після чого саме
відчуваєте себе вичавленим, а хто - що \enquote{п'є ваші соки}. А що вам зазвичай додає
соковитості, стає вашим \enquote{сонцем} та \enquote{грунтом}. А як ви відчуваєте, що вже
потрібно попіклуватись про цілісність?

Я сьогодні відкрила очі зранку з думкою - я пуста. Наче крапля за краплею,
разом із хвилинами світла - ти цідиш, вижимаєш з себе сили і енергію, щоб жити,
працювати, підтримувати. І хронічний стрес виснажує. А живлення не можеш
постійно шукати всередині. І коли його критично недостатньо - наш мозок вмикає
режим енергозбереження. Схожий чимось на анабіоз. (Якщо такий стан тримається
більше 2 тижнів, важливо звертатись до спеціалістів!)

Я звикла бути діяльною, активною, шукати і створювати можливості відновлення.
Але, коли у тебе немає \enquote{соку}, коли ти вже не відчуваєш тіла - \enquote{шкірки}, коли
звичні опори не працюють (світло, можливість планування - це також зовнішні
опори) - такий стан \enquote{без відчуття форми} може лякати.

Я сьогодні вирішила з ним не боротись. Бо і цей стан важливий. І саме зараз
закономірний. Якщо вже дозволив собі спустошитись - потрібно дати дозвіл і
простір на наповнення.

А далі у мене виникла така метафора: Коли немає соку, і м'якоті, і шкірка
витончена... залишаються зернятка. І їм би надати можливість і створити умови,
щоб вони зростали в новий \enquote{фрукт}.

Я сьогодні себе питала - що може бути моїм \enquote{грунтом}? Сиділа за комп'ютером -
працювала потроху - загорнута в плед. І уявляла, наче я зернятко в долонях
\enquote{Матері Землі}, чи \enquote{Матері Життя}. Я дозволяла собі не поспішати, не створювати
нового, не стимулювала себе на \enquote{подвиги}.

Для когось - такий \enquote{грунт} - фізична активність, рухи, спорт, масажи, їжа -
смаколики, переміщення в просторі, контакти, обійми, творчість, хоббі, молитва,
медитація, навчання, прибирання, мрії, робота, сон, тиша. Для мене сьогодні це
була \enquote{турботлива пауза} - без почуття провини.

Пауза, щоб прорости чимось новим.

Обіймаю, Родино @igg{fbicon.heart.red}  Як хочу для нас сил, відновлення. І як хочу Перемоги.

\ii{09_12_2022.fb.rojz_svitlana.kyiv.1.limony.orig}
\ii{09_12_2022.fb.rojz_svitlana.kyiv.1.limony.cmtx}
