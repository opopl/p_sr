% vim: keymap=russian-jcukenwin
%%beginhead 
 
%%file slova.reklama
%%parent slova
 
%%url 
 
%%author_id 
%%date 
 
%%tags 
%%title 
 
%%endhead 
\chapter{Реклама}

%%%cit
%%%cit_head
%%%cit_pic
%%%cit_text
Cambridge Analytica как раз оперировали подобной информацией о 87 миллионах
пользователей Facebook. Некоторые считают, что этот кейс — прекрасный пример
работы социальной инженерии. Другие уверены, что это просто хорошо настроенный
таргетинг. Или просто старая добрая \emph{реклама}, которая существует десятилетия.
Что вы думаете?  Да, с помощью таргетинга Cambridge Analytica дотянулась до тех
пользователей, которые окончательно не определились со своим кандидатом. Это
достаточно узкая выборка, очень точечный таргетинг.  Но политическая \emph{реклама}
показывалась с учетом психологического портрета пользователя. И это уже
нетипично для старой доброй \emph{рекламы}, это нечто новое.  Вы можете таргетировать
рекламу на конкретную демографическую группу или выбрать людей определенной
профессии. Но когда \emph{реклама} рассчитана на тревожного человека, и вы точно
знаете, что она на него подействует, это уже манипуляция.  Потому что именно
так манипуляция и выглядит: вы используете человеческие уязвимости
%%%cit_comment
%%%cit_title
\citTitle{«У людей должен быть выбор» Как соцсети взламывают мозг человека и кто на самом деле их контролирует?: Coцсети: Интернет и СМИ: Lenta.ru}, 
Федор Тимофеев, lenta.ru, 28.10.2021
%%%endcit
