%%beginhead 
 
%%file 22_06_2022.fb.leuckij_mykyta.mariupol.1.istoria
%%parent 22_06_2022
 
%%url https://www.facebook.com/permalink.php?story_fbid=pfbid0mkqa4xRrHWPEpzYcRnvnjS8pD6cRHGHAXzFXvaFuaQa1Xj2zy97xwR6iZNmz2upcl&id=100028679918596
 
%%author_id leuckij_mykyta.mariupol
%%date 22_06_2022
 
%%tags mariupol,mariupol.war,text.story
%%title Вітаю! Сьогодні хочу поділитися з вами своєю історією. Мене звати Микита, я з Маріуполя
 
%%endhead 

\subsection{Вітаю! Сьогодні хочу поділитися з вами своєю історією. Мене звати Микита, я з Маріуполя}
\label{sec:22_06_2022.fb.leuckij_mykyta.mariupol.1.istoria}

\Purl{https://www.facebook.com/permalink.php?story_fbid=pfbid0mkqa4xRrHWPEpzYcRnvnjS8pD6cRHGHAXzFXvaFuaQa1Xj2zy97xwR6iZNmz2upcl&id=100028679918596}
\ifcmt
 author_begin
   author_id leuckij_mykyta.mariupol
 author_end
\fi

Вітаю! Сьогодні хочу поділитися з вами своєю історією. Мене звати Микита, я з
Маріуполя. У свої 23 роки був одним з власників Изба-Читальня та Jager Point
Bar.

У 19 з партнерами ми почали займатися бізнесом, будували власне майбутнє не
дивлячись ні на що, так би мовити відмовляючись від пропозицій юності,
працювали непокладаючи рук. Але 24 лютого життя змінилося. 

24 лютого - день, коли щасливе життя обірвалося, та розділилося на "до" та
"після". Вдень ми зустрілися з партнером з бізнесу у Ізбі-Читальні, та навіть
не могли уявити, що чекає далі.. Перші ночі від початку війни ми ночували у
підвалі власного закладу з друзями та близькими великої команди працівників. З
кожним днем в укритті нашої історичної будівлі (будинку Іллі Юрьєва, що вже
переживав війну) збиралось все більше та більше людей. Щодня до нас приходили
різні незнайомі з небезпечних районів міста, сподіваючись на "зелений коридор",
якого не існувало - вони прямували до центру, де їх розвертали назад, та
вертатися було вже нікуди, а Ізба зустрічала усіх.. 

До нашого прихистку першою прибула родина з сусіднього будинку - засновники
типографії  та туристичного агентства

Аня, Илля та їхні маленьки діти. 

Нас стало більше, але ми не засмучувалися. Це були перші дні та у мережі
вирував дух швидкої перемоги. Кожен з нас вірив, що все буде добре. 

Одного дня вимкнули світло на дві доби - це був перший дзвіночок. Потім
електрику вимкнули після гучної канонади і це нажаль вже було назавжди. На 7-10
день ми домовилися з Данилом (моїм бізнес партнером) відпустили усе, що було,
бо зрозуміли, що все буде набагато важче, ніж здавалося, та краще одразу
підготуватися морально до найближчого майбутнього. 5-7 березня - дні, коли над
містом з'явилася авіація. Страх у кожного в очах. Приховати неможливо. Немає
світла, вимкнули воду та газ. 

Перший гучний вибух - снаряд влучив у будинок навпроти. Нашу будівлю трясе і ми
швидко йдемо у підвал. Підіймається паніка, намагаєшся її заспокоювати, а
всередині хочеш кричати вголос від почуттів. Зараз неможна бути слабким - для
людей навкруги ти та твоя команда єдина надія на спасіння. Тож сидячи у підвалі
та заспокоюючи усіх, незважаючи на власну тривогу, ми, здається, чекали кінця.
Страх, що це ніколи не скінчиться, починав вивертати душу назовні.. І от знову
тиша. Один з товаришів - Евген, піднявся нагору. Пройшло декілька хвилин і він
повернувся зі словами "вікон немає". Надворі мінусова температура, вікон немає.
Товариш Денис запропонував чергувати, аби до приміщення ніхто сторонній не
проник, адже усередині вже багато дітей та літніх людей. Після цієї пропозиції,
спати ми почати почерзі, змінами: перша зміна з 23.00 до 03.00, друга з 03.00
до 07.00. 

Перша зміна. Я сиджу у верхньому одязі на сходах у підвал. Ми завісили ковдрами
прохід, та це не рятувало, бо навулиці мінус і дме крижаний вітер. Дослухаєшся
кожного звуку і чекаєш коли зміна закінчиться, бо дуже хочеш спати. Куртки не
знімаєш - бо зараз як ніколи боїшся захворіти, адже хвороба дорівнює смерть.
Ліків немає, неймовірно холодно. 

Додамо про чергування. 

Одного дня, коли через чергування мій недосип вже став нормою, але мозок
відмовлявся це приймати, почалися галюцинації. Ти дивишся у пітьмі на обличчя
товариша, а це сидить старий чоловік. Жахливо трясеш головою і потім розумієш,
що ввижається. Так повторюється день у день, майже кожної ночі. Чергування
навчило спати сидячи чи навіть стоячи просто спершись ліктем на край дверного
отвору на виході з підвалу. 

Вранці після чергування усі сплять, ти дуже хочеш лягти - але місця немає.
Будити друзів теж не хочеш і йдеш набирати воду. Літаки у небі, а ти набираєш
воду. Єдині мокрі кросівки покриваються льодом, руки які вже не слухаються і
перестають розгинатися. Після, йдеш за сокирою і рубаєш дрова, щоб закип'ятити
води дітям, літнім людям та друзям які скоро прокинуться. 

Приготування їжі, окремий процес.

Готуєш чи допомогаєш мовчиш і думаєш як хочеться до коханих людей, як
обіймаєш найрідніших. Невгамовне бажання готувати сніданок для коханої або
сидіти зараз в батьківській хаті, яка в цей момент вже зруйнована та вигоріла
до тла (але я про це ще не знаю), дивитися на маму, яка з любов'ю порається
по кухні. Навертаються сльози, але не можна давати слабину.

Приготування їжі займає дуже багато часу - людей чимало. Усі, хто сподівався
скоріше встигнути на неіснуючий "зелений коридор", не взяли з собою зовсім
нічого з продуктів. Їжі небагато, і через деякий час довелося видавати усім
половину, а згодом і третину звичної порції. Солодке здавалося чимось
неймовірним та вже нереальним. Після приготування заходиш всередину приміщення
та чекаєш, коли зігрієшся. Знімаєш взуття та кулаками натираєш стопи, аби трохи
зігріти ноги. 

Одного дня до нас прийшов мій товариш Ескандер (псевдонім ще до війни) та
приніс з собою пів палки ковбаси. Цей шматок здавався тоді чимось неможливим.

Кожного дня ти живеш та чекаєш вечора. Вечір - час, коли усі гості нашого
прихистку починають засинати. А ти з командою ловиш момент та вчишся грати на
гітарі, згадуєш мотиви пісень, співаючи їх гуртом, спілкуєшся з усіма, аби не
піддаватися паніці та суму. 

Сидячи в підвалі, всі знайомі стали друзями, які відкрилися тобі, показали одне
одному справжніх себе. Ось ніби знаєш людину добре, а виявляється зовсім ні.
Тобі розповідають про дитинство, юність, і в цих словах ти знаходиш те, що
збігається з твоєю власною історією або зовсім відрізняється. Дивуєшся. Я для
себе відкрив багато людей і що найголовніше - відкрив з кращого боку. Тепер ці
люди для мене дуже важливі. 

До речі, час даремно ми не втрачали. Щоб відволіктися від думок про рідних і
коханих, з якими давно не було зв'язку (а дістатися до них було практично
неможливо з шансами 50 на 50), ми почалися освоювати нові для нас заняття.
Наприклад, один із товаришів у підвалі навчив мене грати на гітарі. З іншим ми
написали пісню. З третім грали у шахи. 

Мабуть, неможливо передати почуття, коли не можеш спати від вибухів та думок,
що десь там зараз твої кохані та найважливіші люди.. Голова розривається.
Намагаєшся відволіктися, щоб просто відволіктися. Тебе заспокоюють друзі поруч.
Хоча в їхніх очах теж тривога від переживань і такий самий страх за близьких,
але вони все одно підтримують.

І ось ситуація. Підвальне приміщення у 45 квадратів вже вміщує 35 осіб. Стоять
столи - на них лягали спати дівчата з нашого колективу; після чергування наші
хлопці лягали під столи та вже на будь-який вільний кут столу. Далі стояли
дивани - на них "проживали" сусідка з двома дітьми. За ними були з'єднані дві
лавки, на котрих спали дідусь та бабуся Данила, що змогли дійти з Лівого
берегу, але по дорогі їх обікрали та викрали сумку з речами. Вони прийшли з
однією палицею та кнопочним телефоном у руках. Після них стояв стіл, на якому
"жила" дівчина, що прийшла майже пішки до Маріуполя, аби вивезти з пекла власну
дитину. У підвалі жили літні люди, що раніше мали житло у цій будівлі, декілька
сімей з Східного та Лівого берегу. Люди приходять та приїжджають, живуть. Їдуть
далі чи назад додому.  

На 21 день війни до нас, у центр, приїхав мій батько за словами: "нам потрібно
виїжджати зараз". Я умовив команду та більшість жителів прихистку Ізби-Читальні
поїхати з нами. Сідаючи до машини, батько сказав мені: "як приїдемо додому, не
лякайся - у будинку два трупи". Перше моє питання "чи жива мама?". Відповіддю
було "так" і ми поїхали. На шляху у нас вистрілив танк і ми дивом змогли
врятуватися - батько здав назад за декілька секунд "до". Потім помчав крізь
чорний дим, адже дороги назад вже не було, ми почули ще автоматну чергу. Дивом
залишившись живими та діставшись будинку, де були батьки, я став свідком
страшної картини: будинок друзів родини зруйновано, як і сотні навколо. Усе
горить, електричні стовпи лежать на дорозі, дерева, догорають автомобілі та
будинки, усюди дроти та військова техніка. Забігши додому, я бачу кров на
стінах та підлозі. Спустившись до підвалу будинку, я побачив маму..

дітей та бабусю, що жили з онукою у тому будинку. 

У наш будинок потрапило декілька снарядів - уламками скла на місці вбило дитину
2х років та його матір. Решту жильців намаглися врятувати, відвізши до
найближчої лікарні. Після чого він поїхав за мною, аби також вивезти дітей -
адже у будинку знаходитися було вже небезпечно. До речі, власник, повернувшись
до будинку буквально за кілька годин забрати речі перед вивозом постраждалих із
лікарні (оскільки лікарню цього ж дня розформували), розповів, що в будинок
потрапило ще кілька снарядів. 

Що ж, повернемося.  

Ми почали садити дітей до машини, спробували завести одну з розбитих машин
поруч. Повністю прогнувсь, зовсім без вікон звичайний ланос вмістив у себе ще
декілька людей. На Стрільці нам допомогли та замотали вікна плівкою. 

Я любив життя, я любив своє місто, свою родину, свою справу, свою тодішню
реальність. Тепер я не маю нічого. Ні домівки, ні справи, ні близьких.. 

Прибувши до Дніпра та передавши дітей їхнім батькам, першим місцем, куди я
потрапив був First Wave- заклад, що не функціонував з 24 лютого і працював лише
як волонтерський центр (саме сюди мене привели мої товариші, які на той момент
вже мешкали у місті). Тут годували переселенців - людей, які втратили усе.
Вийшли волонтери й так само нагодували нас. Наступного дня ми вирішили приїхати
та допомогти їм: возили та розвантажували продукти, допомагали в усіх
необхідних питаннях. В цей момент Андрій (засновник закладу) був на фронті. Так
минуло декілька тижнів. 

Одного разу за вечерею у First Wave ми познайомилися з людиною, якій розповіли
свою історію, як виїхали з досвідом, але без копійчини у кишені. Після вечері
цей чоловік піднявся та сказав: "Працюйте!". Це був Андрій. 

Це був шанс на життя, на нове життя. Нас підтримали у Дніпрі. І після подальших
розмов я зрозумів, що ми повинні продовжувати волонтерську діяльність. 

Сьогодні у First Wave ми годуємо до 400 осіб в день. Багато закладів, які
підтримували людей, які потребують гарячих обідів - зупиняють свою діяльність.
З кожним днем наша команда волонтерів годує все більше і більше людей. Ми
допомагаємо, як і зовсім нещодавно допомогли нам. Також усіма силами
намагаємось розвивати заклад, аби усі члени колективу також мали гарне майбутнє
та теж отримали шанс на нове життя. 

Тепер нове міста та зовсім інші реалії. Всередині кожного з нас війна розірвала
на мільйони шматочків. Проте не можна здаватися, потрібно йти вперед! Ми
потребуємо вашої підтримки. Пропонуємо кожному з вас зустрітися та
познайомитися з нами! Замовляйте піцу, залишайте правдиві відгуки, відмічайте
нас у соціальних мережах - усе це є неймовірно цінним. Кожен публічний відгук
та репост неодмінно допоможе нашій справі розквітнути. І тому ми завчасно
вдячні усім вам! 

Щодо мене, то мене можна майже щодня зустріти у залі, власне як і усіх членів
нашого колективу. Не соромтеся підходити та знайомитися, ставити запитання та
спілкуватися, ми завжди радіємо новим знайомствам. Давайте ставати однією
великою родиною First Wave!

%\ii{22_06_2022.fb.leuckij_mykyta.mariupol.1.istoria.cmt}
