% vim: keymap=russian-jcukenwin
%%beginhead 
 
%%file 18_01_2022.fb.ermolaev_andrej.1.noch_surka
%%parent 18_01_2022
 
%%url https://www.facebook.com/yermolaievandrey/posts/1001293880482966
 
%%author_id ermolaev_andrej
%%date 
 
%%tags obschestvo,psihologia,strana,ukraina
%%title Ночь сурка (не о политике)
 
%%endhead 
 
\subsection{Ночь сурка (не о политике)}
\label{sec:18_01_2022.fb.ermolaev_andrej.1.noch_surka}
 
\Purl{https://www.facebook.com/yermolaievandrey/posts/1001293880482966}
\ifcmt
 author_begin
   author_id ermolaev_andrej
 author_end
\fi

Ночь сурка (не о политике)

...Иногда мне кажется, что я сплю. Сон громкий, сварливый, без начала и конца.
И с каким-то катастрофичным надрывом \enquote{вот-вот}... Сюжет путается, меняются
герои, образы, маски, кличи и призывы, но вот эта струна надрыва и ожидания
срыва звенит постоянно. И мысль, словно с \enquote{параллельного мира}: вот проснусь, и
сон улетучится. А он не улетучивается. Такой же истеричный и бессвязный,
переходит в новые сюжеты. Чувствую, что уже не способен их толком разобрать,
потому что герои вчерашнего сна становятся анти-героями сегодняшнего, и так без
конца. 

... Вспомнился мне смешной, ироничный и немножко философичный фильм \enquote{День
сурка}. Но отличие в том, что этот комедийный зверек все-таки просыпался, внося
сомнительную ясность в дату грядущего общего просыпания - наступление весны.
Наш же сурок какой-то странный. Просыпается ли он вообще - не ясно, потому что,
по всей видимости, происходит это какой-то неизвестной ночью, когда все еще
спят. А когда все утром просыпаются - спит этот местный сурок. 

Такая вот бесконечная последняя осень-зима...

Так и живем. От сюжета к сюжету. Не просыпаясь.

\ii{18_01_2022.fb.ermolaev_andrej.1.noch_surka.cmt}
