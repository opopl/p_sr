% vim: keymap=russian-jcukenwin
%%beginhead 
 
%%file 02_11_2020.fb.ekaterina_zharkih.1.patriotism
%%parent 02_11_2020
 
%%url https://www.facebook.com/kate.zharkih.5/posts/5208846775808051
%%author 
%%tags 
%%title 
 
%%endhead 

\subsection{Патриотизм - что это такое?}

\Purl{https://www.facebook.com/kate.zharkih.5/posts/5208846775808051}
\Pauthor{Жарких, Екатерина}

\textbf{Патриотизм не должен ослеплять нас. Любовь к отечеству есть действие ясного рассудка, а не слепая страсть.
(с) Карамзин}

Меня, как и многих из вас, бесит искусственность и неискренность. Когда человек
строит из себя то, чем не является. Работает только на внешнее впечатление,
пока суть остаётся.  Вся истерия показушного псевдо-патриотизма в Украине
вскрыла в людях самое худшее. Обществу дали моральное право на ненависть,
гордыню и злость.

Человеку потерянному, не знающему своего предназначения, не понимающему своей
цели, живущему серой жизнью вдруг дают самый доступный способ ощутить свою
важность, особенность и ценность: ненавидеть и презирать других, «неправильных»
граждан. А вместе с этим правом выдают карт-бланш проявлять свою ненависть:
оскорблять, травить и физически преследовать прямо на улицах – все знают,
наказания за это от государства не будет. 

Таких людей поощряют превозносить себя над другими и гордиться не тем, что он
сделал что-то хорошее для этого мира, а тем, что он лучше других. Сознательнее,
патриотичнее, прогрессивные. Другие-то хуже, биомасса или отсталые совки,
смотреть на них противно. А вот ты – молодец. На печеньку, приходи на выборы за
добавкой. 

Отсюда же любовь (как показывают рейтинги) ко всяким омерзительным шоу, где
участники ведут себя неадекватно и асоциально (вроде «кохана, ми вбиваємо
дітей»). 

И вот этот простой человек начинает во всю пользоваться правом вести себя
по-скотстки, при этом прикрываясь «патриотизмом».

В вышиванке – значит он лучше тебя, «несознательного» индивида в простой
рубашке. И плевать, что ты в этой рубашке честно работаешь, платишь налоги и
работаешь на благо страны в самом прямом смысле, а не занимаешься показухой. 

С языком та же история. 

У людей забрали возможность найти своё место в жизни, разрушив экономику. И,
вместо того, чтобы вернуть или создать новые рабочие места, дать перспективы,
верхушка пошла простым путём – поделили нацию на сорта. И пока все дерутся
между собой под «смерть ворогам», никто не замечает, как страна идёт ко дну.
Этим и манипулируют в последние годы. Результат такого пути мы видим – граждане
гибнут, становятся беднее, а перспективы Украины выглядят всё мрачнее. 

Показной патриотизм – показатель небольшого ума. Это желание быть с толпой,
стадный позыв. Обычно такие люди не владеют ни малейшей аргументацией своей
правоты или превосходства: просто потому что я отношусь к этой группе людей, я
красавчик, а ты чмо. С такой же логикой этот человек никогда не сможет принять
мнение другого, он его даже не услышит: зачем таким важным персонам вообще
выслушивать жалких недочеловеков? Закричать кричалками или избить, если они
мыслят по-другому – вот это дело. А раз нам можно, то тем более это нужно. Это
же значит быть «свідомим». 

Настоящий патриот – это тот человек, который думает об интересах всех своих
соотечественников. Взвешивает, что будет лучше всего для его народа в данный
период времени и поступает сообразно этому. И не важно, на каком языке он
говорит. Ходит ли каждый день в вышиванке. Вешает ли национальный флаг в каждый
угол. И как громко он прокричал «Слава Украине!». 


Это показывается в других, реальных действиях. В ежедневном созидании, в
честном труде, в земных делах на общее благо. Даже если он никогда в жизни не
выйдет на митинг или уличную акцию, но просто будет трудится, ходить на выборы
и голосовать за таких же прагматичных людей, настоящих патриотов дела – этого
будет достаточно. 

И поверьте, если бы все украинцы, которые готовы и хотят трудиться, жили своей
жизнью: зарабатывали на здоровую еду, достойный быт и хороший отдых, могли бы
качественно лечиться и давать своим детям качественное образование – в общем,
жить обыкновенной жизнью, то, я думаю, не дошло бы до всей этой грызни,
унижений и агрессии к таким же гражданам, как и ты.

Как по мне, именно это – непатриотично.  Просто за псевдо-патриотизмом спрятали гнилое нутро.

А этим гнилым нутром пользуются все, кто дотянется. И обычно это отнюдь не те
люди, которые хотят процветания своей стране, а которые хотят на ней
заработать. 

Как говорил Бирс: «Патриот – человек, которому интересы части представляются
выше интересов целого. Он игрушка в руках государственных мужей и орудие в
руках завоевателей».

А что для вас настоящий патриотизм? Научились видеть нутро человека под мишурой псевдо-патриотизма?

