% vim: keymap=russian-jcukenwin
%%beginhead 
 
%%file slova.donbass
%%parent slova
 
%%url 
 
%%author 
%%author_id 
%%author_url 
 
%%tags 
%%title 
 
%%endhead 
\chapter{Донбасс}
\label{sec:slova.donbass}

Он также рассказал, что в съёмках хроники медицинских будней им помогал
реальный врач из Сибири, а точнее с Крайнего Севера. Медик болеет за
\emph{Донбасс}, помогает и бывает здесь. Сам о себе он говорит – просто доктор,
просто «Потапыч», \textbf{Премьера песни от группы «Зверобой» – Шанс на
спасение}, miaistok.su, 18.06.2020

Мир и воссоединение с \emph{Донбассом} еще впереди, Андрей Ермолаев, strana.ua,
30.05.2021

\enquote{Почта ЛНР} определила победителей конкурса \enquote{Люблю тебя, мой
край родной, Донбасс!}, lug-info.com, 31.05.2021

Детей и стариков \emph{Донбасса} жаль больше всего.. Гончаренко все равно ответит за
то, что делал и делает.. не верю,что ему не страшно.. Сны наверное снятся..
комментарий, \textbf{Опять Шария достали. Что дальше?} Анатолий Шарий, youtube.com, 31.05.2021

02 июня 2014г. - день, когда был окончательно потерян \emph{Донбасс},
\textbf{Ровно семь лет назад Украина навсегда потеряла Донбасс, - Помните женщин с оторванными ногами у здания Луганской ОГА?} 
Андрей Головачев, strana.ua, 02.06.2021

Можно сколько угодно играть в игру под названием Минские Соглашения, формула
Штайнмайера, план Козака-Ермака и т.д., но приходится признать, что Донбасс
ушел и нет силы, которая могла бы его вернуть,
\textbf{Ровно семь лет назад Украина навсегда потеряла Донбасс, - Помните женщин с оторванными ногами у здания Луганской ОГА?} 
Андрей Головачев, strana.ua, 02.06.2021

\enquote{Точка невозврата для \emph{Донбасса}}. Семь главных вопросов об авиаударе ВСУ по Луганску 7 лет назад,
Максим Минин, strana.ua, 02.06.2021

Так получилось, что почти вся моя жизнь и воспоминания связаны с войной, из-за
чего у меня нет сожалений и печали о прошлом. Я живу настоящим и изредка думаю
о будущем, в котором есть место наивной и глупой мечте, вызывающей улыбку.
Вполне реальная, согревающая и почти осязаемая она позволяет не отчаиваться
даже в самые тяжёлые времена. Я хочу, чтобы в небе \emph{Донбасса} летали пассажирские
самолёты, а не бумажные фонарики. Любая мечта должна превращаться в реальность.
Так должно быть и я верю, что так и будет,
\textbf{Фаина Савенкова. Верить и надеяться}, antifashist.com, 02.06.2021


Гибель людей на \emph{Донбассе} говорит о том, что наше общество больное,
Денис Жарких, strana.ua, 04.06.2021

Гибель людей на \emph{Донбассе} говорит о том, что наше общество больное. И, что
человек, что общество, приступают к лечению, когда чувствуют себя больными.
Именно чувствуют, но для этого надо эти самые чувства иметь. К сожалению, мы
продолжаем развивать бесчувственное общество потребителей, паразитов, не
имеющих чувств, а только желудок.  Вакциной от бесчувствия является творчество,
искусство, красота. И в этом плане творчество маленькой девочки Фаины
Савенковой лучик света в этом огромном хаосе преступных желаний политиков и
безразличия масс. В творчестве Фаины нет пафоса, нет конъюнктуры, нет
романтизации войны - это творчество чистого ребенка повзрослевшего во время
войны более большинства взрослых,
\textbf{Гибель людей на \emph{Донбассе} говорит о том, что наше общество больное},
Денис Жарких, strana.ua, 04.06.2021

Посмотрите внимательно: видите контур «незалежной»? Так там и Крым, и \emph{Донбасс}
на старом месте – как было до госпереворота, пардон, до «революции гидности». А
еще, наябедничаю, там и «кричалки» бандеровские, которые возродил майдан 2014
года и закрепил на государственном уровне, шелком вышиты. Да-да: «Слава
Украине! Героям слава!» - как и положено настоящему арийско-патриотическому
государству (правда, почему-то второй лозунг, очень похожий на лозунг
нацистской Германии, спрятан в тылу, на спине, да еще и с изнанки вышит)),
\citTitle{Форма не главное – главное содержание! И в спорте тоже...}, Мысли Бабы Яги, zen.yandex.ru, 07.06.2021

Слов о \enquote{раковой опухоли} мы там не нашли (хотя именно на дневники Гончара как
на источник этой цитаты ссылался в недавнем интервью Кравчук). Но отыскали два
других упоминания о Донбассе.  В опубликованной версии дневников Гончара есть
запись от 1993 года. Там писатель действительно нелестно отзывается о регионе и
пишет, что \enquote{Донбасс - это люмпенизированный край}.  Написано это было не просто
так, а во время масштабной забастовки в Донецкой и Луганской областях лета 1993
года. Это была крупнейшая (до первого Майдана) акция протеста в истории Украины
и сильно потрясла страну, во многом поменяв политический расклад, приведя во
власть в Киеве донецких промышленников во главе с Ефимом Звягильским.  На этом
фоне в июне 1993 года Гончар полагал, что Донбасс \enquote{спровоцирован и обдурен
шовинистами}. И своей забастовкой не дает угля днепропетровским металлургам,
которые рискуют остаться без доменных печей.  По мнению писателя, шахтеры
шантажируют Украину и у них \enquote{нет чувства рабочей чести},
\citTitle{Донбасс как раковая опухоль. Есть ли это в дневниках Олеся Гончара},
Максим Минин; Екатерина Терехова, strana.ua, 08.06.2021

%%%cit
%%%cit_pic
%%%cit_text
А то получается, как с товарищем Врангелем.  Андрій Бузаров опублікував фото
бордів, які масово розміщені на тій стороні \emph{Донеччини}, яка перебуває під
контролем української влади. Я все розумію, але невже не можна пропагандистські
гасла писати без граматичних помилок?  Чомусь згадуються спогади одного з
діячів мельниківської ОУН: у 1942 році вони виготовили листівку «Бей фашистских
гадов!», бажаючи переконати німецьку поліцію Волині у тому, що це - справа рук
червоних партизанів. І підписали: «Красный отряд имени товарища Врангеля»
%%%cit_title
\citTitle{Неужели нельзя пропаганду на Донбассе исполнять без грамматических ошибок?}, 
Константин Бондаренко, strana.ua, 12.06.2021
%%%endcit


%%%cit
%%%cit_head
%%%cit_pic
%%%cit_text
Да, и завершает свой поносик Блогер беспощадно хлестким упреком – как вы
посмели рассуждать о возобновлении саммитов ЕС-Россия, когда гоблинцы не
вернули нам до сих пор Крым и \emph{Донбасс}? «Дякую за ваши чесни видповиди». Все,
снято! Да не лыбься ты! Тут не в тему. Нет, блядь, ничего не переснимаем, так
пойдет. Сразу просто в конце обрежем.  Напомню, это было двухминутное обращение
Блогера к будущему канцлеру Германии. Пацан с района задал четкие вопросы.
Жаль, про репарации ничего не сказал. Или мог бы предложить снять совместный
сериал «Братья по оружию». Еще и на мордорском СТС чуток роялти поднять. Но в
целом тоже неплохо. Высокое чувство собственного достоинства. Отсутствие
ощущения берегов. Шизоидный эгоцентризм. И инфантильный набор претензий.
Настоящий повелитель нелепости
%%%cit_comment
%%%cit_title
\citTitle{Пацан с района задал четкие вопросы будущему канцлеру / Лента соцсетей / Страна}, 
Игорь Лесев, strana.ua, 29.06.2021
%%%endcit

%%%cit
%%%cit_head
%%%cit_pic
%%%cit_text
Як годиться, з другого боку підпряглися націоналісти. На цьому гріх було не
заробити політичні бонуси – і тут я скажу болюче – таким само політикам, які
також не уявляли Україну незалежною. 
Адже поки наші політики змагаються в метафорах, російськомовний одесит
проводить місяці й роки в окопах на \emph{Донбасі}, б’ючись пліч-о-пліч з
україномовним львів’янином проти росіян та проросійських бойовиків
%%%cit_comment
%%%cit_title
\citTitle{Українці не розуміють одне одного не через мову, а через небажання слухати, чути і сприймати}, 
Юлія Мендель, www.pravda.com.ua, 07.07.2021
%%%endcit

%%%cit
%%%cit_head
%%%cit_pic
\ifcmt
  pic https://img.vz.ru/upimg/m11/m1106740.jpg
  width 0.4
\fi
%%%cit_text
Удивительно, но похожий взгляд на события распространен и в самой РФ. Акценты,
разумеется, совсем другие – из ряда «братской помощи» и «борьбы с фашизмом»,
однако определяющая роль России как организатора и исполнителя под сомнение не
ставится.  В реальности, как это часто бывает, все значительно сложнее. Да,
жизнедеятельность современного \emph{Донбасса} практически полностью зависит от
России. Но считать, что руководство ДНР и ЛНР абсолютно подконтрольно Кремлю и
не имеет никакой автономии, сродни самоупоению.  И в 2014 году, и сейчас многое
в регионе определялось инициативой снизу. Это было проявлением собственной воли
местных элит, а не исполнением чужих приказов.  Как следствие, политическое
устройство непризнанных республик даже сейчас, спустя семь лет после начала
войны, остается совсем не таким, каким бы его хотела видеть Москва. С точки
зрения улучшения управляемости и более эффективного расходования ресурсов
представляется разумным как минимум объединение ДНР и ЛНР в одно
государственное образование
%%%cit_comment
%%%cit_title
\citTitle{Зачем Донбассу феодальная раздробленность}, Дмитрий Бавырин, vz.ru, 01.07.2021
%%%endcit

%%%cit
%%%cit_head
%%%cit_pic
%%%cit_text
Более обычными, хоть и не менее неприятными, стали политическая
ангажированность и несправедливое судейство. Но тут все понятно: Олимпиада -
элемент геополитической борьбы, достижения на соревнованиях демонстрируют
уровень развития и состоятельности стран, ведь в экономически слабых или
политически нестабильных государствах едва ли возможно подготовить
команды-медалисты.  Жители \emph{Донбасса} по понятным причинам сопереживали и болели
за Россию. Порадовал наш земляк, 27-летний уроженец Луганска Александр Бондарь,
который принес российской команде бронзовую медаль в синхронных прыжках в воду.
Напомню, что уехать из \emph{Донбасса} спортсмену пришлось из-за развязанных Киевом
боевых действий
%%%cit_comment
%%%cit_title
\citTitle{НЕДЕЛЯ ГЛАЗАМИ ЭКСПЕРТА: Назначенные убивать, откровения Карамельки и неспортивное поведение}, 
, lug-info.com, 15.08.2021
%%%endcit
