% vim: keymap=russian-jcukenwin
%%beginhead 
 
%%file 12_02_2022.stz.news.ua.strana.1.vojna_nachnetsja_15_fevralja.1.novyj_vitok
%%parent 12_02_2022.stz.news.ua.strana.1.vojna_nachnetsja_15_fevralja
 
%%url 
 
%%author_id 
%%date 
 
%%tags 
%%title 
 
%%endhead 

\subsubsection{Новый виток \enquote{вторжения России}}

Первым шагом стало ночное решение Байдена призвать всех американцев немедленно
покинуть Украину. В течение сегодняшнего дня аналогичное решение приняло
множество других стран -  Нидерланды, Япония, Южная Корея, Израиль, Черногория,
Норвегия, Латвия, Эстония и Великобритания.

Среди них Израиль и Британия начали к тому же сокращать персонал посольств. А к
вечеру своих граждан призвал к выезду Европейский союз, который раньше
отказывался поддерживать эти меры.

Правда, пока неизвестна позиция таких стран как Германия, Италия, Франция и
Австрия, да и ряда других государств ЕС. Пока информации об эвакуации их
посольств нет. То есть еще не все разделяют опасения насчет \enquote{вторжения}.

Но нужно констатировать, что количество таких стран значительно возросло. И это
весьма негативный сигнал для всего мира, даже в случае отсутствия \enquote{вторжения}. 

Затем целое цунами военного нагнетания вокруг Украины началось в американских
СМИ.

\enquote{США считают, что Россия решила вторгнуться на Украину уже на следующей
неделе}, - заявляют журналисты NBC и PBS.

Два представителя американской администрации говорят, что \enquote{ожидают ужасающей
кровавой кампании, которая начнется с двух дней воздушных бомбардировок и
радиоэлектронной борьбы, за которыми последует вторжение с возможной целью
смены режима}.

CNN чуть более осторожна в прогнозах, то тоже говорит о начале вторжения России
еще до окончания Олимпиады. Причем россияне, мол, будут \enquote{брать Киев}. 

\enquote{У США и их союзников есть новая информация, которая предполагает, что Россия
может планировать нападение на Украину до окончания Олимпиады, вопреки
предыдущим оценкам. Новые данные поступают после того, как за последние 24 часа
официальные лица резко повысили актуальность публичных предупреждений,
связанных с Украиной. Киев входит в число целей, определенных в российском
планировании}, - заявил сегодня телеканал.

CNN также сообщил, что Штаты отзывают своих представителей из миссии ОБСЕ в
Украине.

А агентство Bloomberg (то самое, что выпустило фейковую новость о том, что
Россия уже \enquote{вторглась}) назвало конкретную дату нападения - 15 февраля. И
заявило, что все начнется с провокации \enquote{под чужим флагом} (этот сценарий
\enquote{Страна} уже анализировала). Немецкий Der Spiegel называет иную дату - 16
февраля.

Далее масла в огонь начали подливать уже официальные лица. Советник Байдена по
нацбезопасности Джон Салливан заявил, что война может начаться со дня на день.

\enquote{Российские военные действия могли начаться в любой день. И это остается
правдой. Это может начаться в любой день, включая время до окончания Олимпиады.
Не буду раскрывать данные разведки, но расположение войск в России и Беларуси
возле украинской границы позволяет России начать атаку в любой день}, - заявил
чиновник Белого дома.

Он даже описал, какой будет атака. 

\enquote{Российское вторжение на Украину, если оно произойдет, скорее всего, начнется с
воздушных бомбардировок и ракетных ударов, которые могут убить мирных жителей,
независимо от их национальности}, - рассказал Салливан.

Правда, оговорка \enquote{если оно произойдет} как бы множит на ноль все эти
апокалиптические прогнозы. И такая ремарка звучала сегодня не в первый раз. Но
она мало что значит на фоне общего накала военной риторики. 

