% vim: keymap=russian-jcukenwin
%%beginhead 
 
%%file slova.nauka
%%parent slova
 
%%url 
 
%%author 
%%author_id 
%%author_url 
 
%%tags 
%%title 
 
%%endhead 
\chapter{Наука}

%%%cit
%%%cit_pic
\ifcmt
  pic https://gdb.rferl.org/9685EC99-92CA-46AD-B6CF-C5DCA0F34508_w1597_n_r0_st.jpg
	caption Аркуш 271 Остромирового Євангелія, однієї з найвидатніших пам'яток письменства періоду України-Русі. Переписане у 1056–1057 роках
\fi
%%%cit_text
Робота з адаптації старослов'янського тексту Святого Письма відбувалася не лише
у Києві. «Від початку кожен зі слов'янських народів однакові літери
старослов'янської мови вимовляв трохи по-своєму. Образно кажучи, у мертвий
трафарет вкладав свою мовну культуру. Саме це і створювало неповторне обличчя
старослов'янської мови, яке відрізнялося від усіх інших. Ці різновиди
\emph{науковці} назвали редакціями. Редакція сербська, македонська, українська,
білоруська...» – пише Ганна Куземська, автор книги «Якою мовою молилася давня
Україна»
%%%cit_comment
%%%cit_title
\citTitle{Чому написане у Києві Остромирове Євангеліє – «достояние России»?},
Ірина Костенко; Ірина Халупа, radiosvoboda.org, 13.06.2021
%%%endcit

