% vim: keymap=russian-jcukenwin
%%beginhead 
 
%%file 21_07_2021.fb.mischishin_mihail.1.pismo_patriotam
%%parent 21_07_2021
 
%%url https://www.facebook.com/mischishin/posts/4151751251527529
 
%%author_id mischishin_mihail
%%date 
 
%%tags patriotizm,pismo,strana,ukraina
%%title Письмо патриотам. Лупайте сю скалу!
 
%%endhead 
 
\subsection{Письмо патриотам. Лупайте сю скалу!}
\label{sec:21_07_2021.fb.mischishin_mihail.1.pismo_patriotam}
 
\Purl{https://www.facebook.com/mischishin/posts/4151751251527529}
\ifcmt
 author_begin
   author_id mischishin_mihail
 author_end
\fi

Письмо патриотам.
Лупайте сю скалу!

Нормальным людям сегодня приходится жить среди диких, обезумевших людей и
видеть, как они с завидным упорством пилят сук, на котором мы все сидим -
Украину.

\ifcmt
  pic https://scontent-frx5-2.xx.fbcdn.net/v/t1.6435-9/219878542_4151820241520630_5262677270472989240_n.jpg?_nc_cat=109&ccb=1-5&_nc_sid=730e14&_nc_ohc=6RNcDWT88NQAX-mKA80&_nc_ht=scontent-frx5-2.xx&oh=69643e92433015c9af12a56f345afc0c&oe=615B612C
  width 0.9
\fi

С визгом, хохотом, воплями, их глупыми, тошнотворными речевками и вопросами:
чей Крым (это все равно, что спросить у кошки, чья сметана), слава нации,
смерть ворогам.

Если учесть, что в результате умирают, в основном, украинцы, то мы - и есть их настоящие враги.

Здравому человеку, как октябренку Пете, тяжело сегодня жить на свете.

Но интересно.

Многие еще не до конца скурвившиеся люди приходят к тому, что молчание - золото. И они безусловно правы. И все же...

Мы живём в стране, придуманной и созданной не для нас.

А для геополитических и экономических утех Запада.

Мы - это такая резиновая и виртуальная геополитическая кукла.

Разграбить, воспользоваться потенциалом Украины для себя (как Вы думаете, что
сейчас делают украинские заробитчане в строго поддерживающих наши Майданы
Польше и Литве, как не работают на эти страны?), подкузьмить и создать угрозы
для России...

Обычно в Украине в мэйнстримных СМИ нет и не может быть другой повестки дня.

Так на Западе, скорее всего, уверены и думают. То, что все мы здесь идиоты. И
нас можно задурить и заставить разрушать свою страну, превращая ее в лакомство
для США и Европы, а себя - в остарбайтерское дешевое трудовое и пушечное мясо
для Европы и Америки.

Или мы - предатели, с которыми всегда можно договориться.

И украинская знать уже в нескольких поколениях находит себя в этом. В служении
Западу за толику малую за счёт своего народа и страны. Нашего будущего.

И все бы хорошо и ладно. Но в Украине ещё остались думающие люди. Настоящие патриоты.

Сегодня для того, чтобы быть самым настоящим патриотом, как ни странно, нужно
не так уж и много. Просто быть реалистом. И они у нас есть. Есть ещё защитные
силы у обманутого и бесправного народа.

Да, такие люди часто остаются неуслышанными. Это не их проблема.

Украинские пашни и работающие на них селяне настолько отравлены выбросами
гибридной войны, десятки лет успешно ведущейся против Украины со всех сторон,
что достучаться до них практически невозможно.

Знать, продавшая все и вся, привычно сопит в тряпочку. Ей  реализм и знания
патриотов ни к чему. Она все прекрасно понимает и сама. Но хочет денег и не
хочет проблем с Западом, где деньги лежат.

Вот почему украинский патриот сегодня - это глас вопиющего в пустыне. Его в
лучшем случае молча изучат, возьмут на вооружение уже как бы от себя, не
сославшись, или вставят в речевки и программу очередной Слуги народа, которую
не собираются выполнять.

Тем не менее, многие украинские все еще неуслышанные реалисты делают сегодня
великое дело.

Быть реалистом в создаваемом уже для грабежа не отдельных стран, а всей планеты
виртуальном искривлённым и выдуманном идеологическом пространстве - это
смертельно опасное занятие, которое только и является угрозой для сильных мира
сего, погрязших во лжи. Только это может их остановить.

Сегодня в мире идет негласное соревнование. Кто кого опередит. Или сильные мира
сего успеют первыми. И тогда они под видом цифровизации всего сущего, в том
числе - и людей - превращают все человечество в контролируемых ими биороботов и
наши и другие Майданы - полигоны этого  будущего усмиренного и покорного
пастухам стада людей, превращенных, по сути, в животных. Или реалисты сумеют
остановить упрочение власти сильных мира сего и предотвратить это, не вступая в
противоречие с кибернетизацией, роботизацией и научным прогрессом, результаты
которого, как известно, можно использовать и во зло, и во благо.

Это вечная история. В чьи руки, добрые или злые, попадают новые знания. Сегодня
знания о людях и управлении ими находятся в злых и богоборческих руках.

Быстро прогрессирующую и добившуюся впечатляющих результатов науку управления
людьми используют им же во зло, превращая нас в рабов, уверенных, что именно
сейчас мы свободны, как никогда.

Так как тайное все более становится явным и времена и сроки подходят,
властелины мира спешат. Они уже не боятся вскрыть карты на сомнительных и,
скорее всего, подтасованных выборах президента в США (что не мешает им обвинять
Бацьку Лукашенко в том же самом, хотя Тихановской, скажем так, далековато до
Байдена, в чью пользу подтасовывали результаты выборов), в появлении круто
изменившего жизнь на планете вируса Ковид-19.

И этому пока что никто не может противостоять. Все только дивятся, до какой же
степени они незаметно успели загнать в стойло практически всю планету.

Нам, всем людям доброй воли, нужна критическая масса реалистов. И в чем-то -
романтиков, верящих, что победить сильных мира сего - возможно. Не нам - так
Господу. Когда и где она образуется, мы не знаем.

И все равно - лупайте сю скалу. По капле выдавливайте из себя раба. Даже без
внешнего видимого результата - это вдохновляющий пример свободного человека,
столь нужный украинцам сегодня.

Свобода - это не отсутствие запретов. А понимание реалий, в которых живёт
человек. И с этой точки зрения вдохновляющий пример настоящих патриотов важен
для Украины. Это наш путь в лучшее будущее, попасть куда мы можем только
отказавшись от ставшей нашей плотью и кровью лжи.

К счастью, у нее - короткие ноги, которые только кажутся нам привлекательными.

Настанет время - и мы увидим всю низость, духовную нищету и уродство того, что
сегодня кажется нам благородным, великим и прекрасным.

А не эти вот все холивары.

Пишите в охотку, может быть, не так часто, не стараясь образумить, угодить, просветить, объяснить. И заработать.

Не надеясь на быстрый результат, тем более, личный.

Работа - от нас. Но результат - от Господа. А у Него свое понимание меры и
времени. 

P.S.

Цитаты из украинских классиков, на мой взгляд, которые используют
политтехнологи, нами часто понимаются не вполне верно. Как призыв к революции,
борьбе с Россией.

Ивана Франко, как и Лесю Украинку, Тараса Шевченко и других украинских
классиков, наши патриоты часто одевают в противогазы, вручают им в руки
милицейские дубинки и отправляют на драку с Беркутом, в чем они вряд ли бы
приняли участие, будь они живы.

Для меня призыв лупать сю скалу Каменяра Франко - это пожелание разбить скалу
невежества в самом себе, раздолбать свою гордыню, незнание, поверхностность,
тщеславие, лень и т.п. Это не лозунг борьбы с Януковичем и Россией:). И, тем
более - упаси, Боже, другими украинцами.

Использование классиков в политических лозунгах, как раз и свидетельствует о
том, что мы их так и не поняли. И пока что еще не доросли до них.

Любого, украинского и не только, классика нет без любви. Любви ко всему сущему.
Даже к врагам. Ибо и враг наш - от Господа. Потому что наш враг часто
оказывается для нас более полезен и важен, чем друг.

Павел Вернивский

12 ч.  · 

В Голливуде есть одно простое правило. Если фильм пользуется спросом у
зрителей, то почти наверняка будут снимать продолжение. И наоборот. Если фильм
не оправдал себя в прокате, то продолжение никто снимать не будет. 

Все потому, что съемки фильма стоят времени и денег. И потому никто не хочет
лезть в заведомо убыточный проект.

Когда я пишу, то руководствуюсь абсолютно такими же правилами. Да, за свои
посты я деньги не беру, но тем не менее, я трачу время чтобы их написать. Так
как денег за посты не беру, то реакцию могу понять только по лайкам. 

И самое интересное, что учитывая качество информации, которую я даю, лайки это
совсем не цена. Подобную информацию некоторые люди продают за хорошие деньги. И
я сейчас не преувеличиваю. 

И какой смысл мне тратить 2-3 часа своего времени, чтобы найти информацию,
перевести ее в пост, который все равно никто потом читать не будет? По этой же
причине я не хочу писать книгу. По этой причине я часто удаляю свои посты.
Просто не вижу спроса. А писать для себя. Ну зачем? Я и так это знаю)))

\ii{21_07_2021.fb.mischishin_mihail.1.pismo_patriotam.cmt}
