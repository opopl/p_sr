% vim: keymap=russian-jcukenwin
%%beginhead 
 
%%file 03_06_2021.fb.vjatrovich_vladimir.1.borjba_zakon
%%parent 03_06_2021
 
%%url https://www.facebook.com/volodymyr.viatrovych/posts/10219634731198323
 
%%author Вятрович, Владимир
%%author_id vjatrovich_vladimir
%%author_url 
 
%%tags 
%%title То ж боротьба триває
 
%%endhead 
 
\subsection{То ж боротьба триває}
\label{sec:03_06_2021.fb.vjatrovich_vladimir.1.borjba_zakon}
\Purl{https://www.facebook.com/volodymyr.viatrovych/posts/10219634731198323}
\ifcmt
 author_begin
   author_id vjatrovich_vladimir
 author_end
\fi

На жаль, наш парламент замість вирішення реальних проблем витрачає час на
розгляд проєктів, які є не просто порожніми й неякісними, але й написані цілком
у руслі російської пропаганди.

Можна лише поспівчувати людині, якій треба вигадати, що б такого дописати в
закон про засудження тоталітарних режимів, коли там усе, що треба, давно
написано і чудово працює.

Перший варіант цього шедевру законотворчої думки був написаний настільки
недбало, що слово «режим» у ньому мало жіночий рід, Третій Рейх фігурував як
одна з радянських республік. А на додачу в Україні заборонялося виконувати гімн
дружньої нам Німеччини, який автор цього трешу обізвав нацистським, але  який
насправді з 1922 і досі є незмінним. 

Але навіть внесений автором на заміну другий варіант цього проєкту, хоча й
позбавлений найбільш анекдотичних норм, лишається так само недолугим. 

Вже сама його назва наче списана з російських агіток, причому списана з
російської мови настільки буквально, що воєнні злочині у називаються
«військовими» - автор явно не розрізняє ці два зовсім різні поняття.

Втім, назва в цьому законопроєкті живе сама по собі (мовляв, караул! - нацизм
легалізують, злочинців героїзують),  а текст – сам по собі. І  тут можна лише
погодися з численними зауваженнями Головного науково-експертного управління,
яке відзначає, що цей текст лише псує і послаблює чинне законодавство. Яким,
нагадаю, нацистська пропаганда і символіка не просто заборонені, а є
кримінальним злочином, за який передбачено до 10 років ув’язнення. 

Єдине пояснення появи таких шедеврів – допомога державі-агресору в спробах
дискредитації України, підтримка російської брехні про  реабілітацію «осіб, які
брали участь у масових знищеннях євреїв і ромів» у нашій державі. 

За результатом голосування цей законопроєкт провалено. 

Хоча нажаль ухвалили інший з російського порядку денного, в якому йдеться про
потребу запобігання антисемітизму в Україні. То ж боротьба триває.

\emph{Андрей Калашников}

це так схожн на українців)
 · Reply · See Translation · 4h

\emph{Павло Тар}

Тепер єврей може спокійно називати мене "хохлом", чи якось ще. Євреї тепер у нашій державі народ з виключними правами? Конституція, рівність усіх перед законом? "Ой, ше ви говорите..."
 · Reply · See Translation · 9h

\emph{Олександр Сахно}

Мій текс, початок 2020 року.
Уряд України , що це , ми його не обираємо і не призначаємо і не звільняємо ! Уряд України , це ШИРМА від нас для оборудок Президента України з Нардепами України !
І так всі 28 років !
Президент України він же Голова Уряду України , так повинно БУТИ !
Заборонити нардепам вирішення кадрових питань Виконавчої та Судової влади України , виконуючи ст.6 Конструкції України.
Що з цим будемо робити , розділи 4,5,6 і 8 протирічать ст. 6 , а розділи 10,11 протирічать ст.7 та ст. про УНІТАРНІСТЬ ДЕРЖАВИ УКРАЇНА !
Президент України + Нардепи України , яких ми обираємо , ЩО З ЦИМ БУДЕТЕ РОБИТИ , це ваша відповідальність перед виборцями , якщо ЗВІСНО ви це розумієте та можете осягнути своїм розумом !?!
ІМПІЧМЕНТУ ПРЕЗИДЕНТА ТА ВІДЗИВ НАРДЕПІВ НЕ МОЖЛИВО В ДЕРЖАВІ УКРАЇНА !
НАВЧАТИ треба ВИБОРЦІВ виборами кожні 2-3 роки.
Хто економить на виборах той отримує і олігархів і диктатуру і зубожіння населення.
Спочатку треба змінити термін надходження ЇХ в посадочному НАМИ місці !
Знахабніли вже всі , бо термін п'ять років.
П'ЯТЬ РОКІВ ЗАБАГАТО !
По моїм правилам був би вже одинадцятий термін , був би ШАНС для виборців обрати кращого.
Чесно кажучи, дуже про це мрію !
Обов'язково !
Треба вибори депутатів проводити кожні два роки всіх рівнів і голів міст села селищ, а нардепів і президента кожні три роки.
І тільки так , крапка !
А реєстрація кандидатів повинна закінчуватися мін за рік до виборів , з обов' язковим самостійним поясненням джерел походження і майна і коштів , а якщо хтось дав , то хай той пояснить джерело.
На мій погляд, це головне що треба зробити новому президенту , прошу вдумайтесь !
Партія в Україні , це коли вона має 498 сьогоднішніх районих ОСЕРЕДКІВ ! Так повинно були і так буде в Україні !
Заборонити зміну назви партії за рік до виборів !
Заборонити безпартійним бути в списках партії !
Заборонити депутатам партійцям в часи своєї депутатської каденції переходити в інші партії , хочеш перейти здавай мандат !
Досить дурити виборців , ВИБОРЧИЙ ПРОЦЕС для виборців , а не для кандидатів !
Доречі , нарешті в Запорізькій області Чернігівське ОБ'ЄДНАЛИСЯ в одну ТерГромаду ВСІМ районом , а це 4 селища і 36 сіл , тепер вони будуть Соціально , Економічно та Культурно СПРОМОЖНОЮ ТЕРИТОРІАЛЬНОЮ ГРОМАДОЮ !
МОЖЛИВО тепер виконають і Конституцію України і ЗУ про місцеве самоврядування та нарешті розроблять і приймуть основну Програму Соціального Економічного та Культурного Розвитку своєї ТерГромади ( таку програму ніхто в Україні 25 років не приймали і не розробляли ) , а це 1200 км2 з населенням біля 20 000 Громадян України !
І не буде для них ніяких РайРади та Райдержадміністрації !
АдмінТерУстрій закладений в нашій Конституції України заклав підвалини і анексії і окупації наших земель !
ТРЕБА ЗМІНИТИ АДМІНТЕРУСТРІЙ ДЕРЖАВИ УКРАЇНИ , тоді не буде кому надавати особливий статус !
Щоб завадити та вибити грунт для сепаратизму федералізму та колоборанству ТРЕБА :
ОБОВ'ЯЗКОВО ! ЩЕ РАЗ !
ЛІКВІДУВАТИ райРади та райдержадміністрації всі і повністю і назавжди, їх функції та повноваження ПЕРЕДАТИ територіальним громадам , які є центрами сьогоднішніх районів , їх всього в Україні 498 штук !
Території районів об'єднати з центрами районів , назвати це утворення МІСТОМ , це і буде соціально економічно та культурно СПРОМОЖНА громада !
Таких громад в Україні буде 498 шт.!
А після об'єднання в 498 міст ЛІКВІДУВАТИ обов'язково всі ОДА і ОблРади та Верховну Раду АРКрим і Уряд АРКрим !
Таким чином буде Державний бюджет України та 498 бюджетів МІСТ і все !
Той хто не ходить на вибори той або хворий або велика падлюка та ворог Народу України щоб отримати права то спочатку виконай обов'язок прийди на вибори та проголосуй головою !
Вибори треба проводити кожні три роки , а реєстрацію кандидатів всіх рівнів треба закінчувати за рік до виборів !
Виборця потрібно виховувати виборами!
Треба вибори проводити кожні три роки , а реєстрацію кандидатів завершувати за рік до виборів !
Вибори проводяться для виборців , а не для кандидатів !
Реєстрацію кандидатів, на місцеві вибори в жовтні 2020 року , треба завершити хоча б в березні 2020 року ! Вимагаємо це !
