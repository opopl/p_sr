% vim: keymap=russian-jcukenwin
%%beginhead 
 
%%file 06_12_2021.fb.fb_group.story_kiev_ua.2.knigoizdateli_bratja_novickie.pic.2.cmt
%%parent 06_12_2021.fb.fb_group.story_kiev_ua.2.knigoizdateli_bratja_novickie
 
%%url 
 
%%author_id 
%%date 
 
%%tags 
%%title 
 
%%endhead 

\iusr{Юрий Панчук}
Долина речки \enquote{Нивка}

\iusr{Евгения Ерёменко}
\textbf{Юрий Панчук} 

долина речки Борщаговки  @igg{fbicon.face.smiling.eyes.smiling}. Во многие
краеведческие материалы вкралась странная описка. Нивка (местные ещё называли
Святошинкой и, простите, Вонючкой, было за что) протекала местами под землёй,
местами по поверхности через все просеки (справа от Б.-Литовского, если спиной
к центру), сейчас в коллекторе, впадает в пруд на 5-й просеке. На ней стоят
высотки...

\iusr{Юрий Панчук}
\textbf{Евгения Ерёменко} 

Странно, на гугл-карте она \enquote{Нивка}, и в книге \enquote{Энциклопедия
киевских рек} тоже.

\iusr{Евгения Ерёменко}
\textbf{Юрий Панчук} 

а Вы спросите у знатоков Святошина, работавших в архивах: Сергея Вакулишина,
Кирилла Степанца. Такие путаницы бывают иногда. А если подключить логику, то
так и выйдет: по Борщаговкам течёт одноименная речка, от Нивок - Нивка
@igg{fbicon.face.smiling.eyes.smiling} .

Недавно сравнивала советские карты нашей местности: вместо 2 улиц Совхозной и
Ирпенской обозначена одна. Причем на одной карте она Совхозная, на второй
Ирпенская. Наверное, чтобы какого-нибудь врага запутать
@igg{fbicon.face.grinning.smiling.eyes} 


\iusr{Юрий Панчук}
\textbf{Евгения Ерёменко} Уже спросил ))

\iusr{Юрий Панчук}
\textbf{Евгения Ерёменко} 

На карте от ВДНХ она Борщаговка, а потом после какого то пруда в районе Жулян
уже Нивка ))

\iusr{Евгения Ерёменко}
\textbf{Юрий Панчук} 

ага, как раз в районе Борщаговок течёт вдруг Нивка
@igg{fbicon.laugh.rolling.floor} 

\iusr{Юрий Панчук}
\textbf{Евгения Ерёменко} 

Ну вообще то по книге так и есть, от ВДНХ через Жуляны, конечную 1го трамвая и
через Житомирскую трассы. Мы даже с Кириллом ходили на экскурсию по этому
маршруту, вдоль Нивки. За две экскурсии прошли весь маршрут.


\iusr{Евгения Ерёменко}
\textbf{Юрий Панчук}  @igg{fbicon.face.smiling.eyes.smiling} 

\iusr{Юрий Панчук}
\textbf{Евгения Ерёменко}

Кирилл сказал, что \enquote{Борщаговка} все таки правильнее, хотя в народе чаще ее
называют \enquote{Нивка} аж от ВДНХ. А настоящая \enquote{Нивка} действительно правый приток в
15-й пруд. Интересно, что я много раз видел это фото, но впервые узнал пейзаж
на заднем плане ))) Все время фокусировался на трамвае. Наверно этот скан в
лучшем качестве. Узкоколейный трамвай ходил по этой стороне Брест-Литовского аж
до нынешнего Воздухофлотского путепровода, а по противоположной стороне ходил
трамвай от Бессарабки. В начале 20-х годов их объединили и появился переезд
возле Пушкинского парка.

\iusr{Евгения Ерёменко}

Трамвай с правой на левую сторону перемещался под мостом возле м. Шулявская,
шел по стороне КПИ. А рельсы сняли ок. 1980 года (маршруты 23 и 6), 14-й
по-моему позже

\iusr{Юрий Панчук}
\textbf{Евгения Ерёменко} 

Да, к олимпиаде сняли трамвай от центра до метро Большевик и расширили
проспект. А оставшаяся часть линии кажется еше года три работала. Переезд точно
был возле Пушкинского парка, я помню.
