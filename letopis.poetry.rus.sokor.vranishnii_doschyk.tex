% vim: keymap=russian-jcukenwin
%%beginhead 
 
%%file poetry.rus.sokor.vranishnii_doschyk
%%parent poetry.rus.sokor
 
%%url http://maysterni.com/user.php?id=8770
%%author 
%%tags 
%%title 
 
%%endhead 

\subsubsection{Вранішній дощик}
\Purl{http://maysterni.com/user.php?id=8770}

Вранішній дощик
Хто ж це цокає в віконце,
Ранком-вранці на зорі?
Не прокинулось Сонце!
Шумить дощик у дворі.

То сильніший, то тихіший,
Грім по небу гуркотить.
Сон під дощик ще міцніший,
І засне, хто ще не спить.

Сонечко крізь дощ пробилось,
Ніжить промінь по щоці.
В тиші очі пробудились,
Чи від гвалту горобців?

Чутно півня — Ку-Ку-Рі-КУ,
Щебет співу солов’я.
Чутно різні голоси здалеку,
Сповнена в піснях Земля.

Дощем зелень освіжилась,
Квітки розкрили пелюстки,
Щоб насіння їх родилось,
Є комашки, бджілки та пташки.

Небо синє та глибоке,
Вранці вмилося дощем.
Сонечко всесвітнє око,
Землю славить з Новим Днем.
