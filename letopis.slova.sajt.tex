% vim: keymap=russian-jcukenwin
%%beginhead 
 
%%file slova.sajt
%%parent slova
 
%%url 
 
%%author 
%%author_id 
%%author_url 
 
%%tags 
%%title 
 
%%endhead 
\chapter{Сайт}
\label{sec:slova.sajt}

%%%cit
%%%cit_head
%%%cit_pic
%%%cit_text
Глава Венгерского МИД поднял в ПАСЕ вопрос о существовании в Украине печально
известного \emph{сайта} «Миротворец». Так, в частности, министр осудил этот
позорный ресурс, назвав его «списком врагов, которые должны быть наказаны». И
ведь дело говорит иностранец! По европейским меркам, такие списки -
средневековая жуть
%%%cit_comment
%%%cit_title
\citTitle{Глава Венгерского МИД поднял в ПАСЕ вопрос о существовании в Украине сайта Миротворец}, 
Максим Могильницкий, strana.ua, 23.06.2021
%%%endcit

%%%cit
%%%cit_head
%%%cit_pic
\ifcmt
  pic https://coollib.net/i/5/410505/img_5.jpg
  @width 0.4
\fi
%%%cit_text
Четвертый попутчик ездил в Киев к товарищу. Они были в школьные годы «не
разлей-вода» и продолжали дружить после армии. Потом дороги их разошлись, друг
перебрался в Киев, и вот, лет двадцать спустя, они нашлись на \emph{сайте}
«Одноклассники». Стали переписываться в сети, потом друг пригласил его в гости.
Теперь он возвращался домой, перебирая в памяти обрывки впечатлений. Главным
впечатлением было посещение Лавры. Приглашавший и принимавший его друг искренно
удивился, узнав, что школьный товарищ много раз бывал в Киеве, но ни разу не
удосужился спуститься в пещеры и пройтись с молитвой по темным подземным
коридорам. Б один из дней они и поехали в Лавру, спустились в Ближние пещеры и
неторопливо обошли их. Возле каждого гроба останавливались, крестились и
целовали стекло над мощами. Друг кратко рассказывал о каждом святом, и было
видно, что с Лаврой и ее историей его связывает крепкая многолетняя любовь
%%%cit_comment
%%%cit_title
\citTitle{Возвращение в Рай}, Андрей Ткачев
%%%endcit


