% vim: keymap=russian-jcukenwin
%%beginhead 
 
%%file 25_12_2022.fb.fb_group.ukrainska_poezia_i_proza.1.ukr_genij_marchuk
%%parent 25_12_2022
 
%%url https://www.facebook.com/groups/255425469158429/posts/747985896569048
 
%%author_id fb_group.ukrainska_poezia_i_proza,zubchenko_nadia.kyiv
%%date 
 
%%tags marchuk_ivan.hudozhnik
%%title Знайомтесь: український геній
 
%%endhead 
 
\subsection{Знайомтесь: український геній}
\label{sec:25_12_2022.fb.fb_group.ukrainska_poezia_i_proza.1.ukr_genij_marchuk}
 
\Purl{https://www.facebook.com/groups/255425469158429/posts/747985896569048}
\ifcmt
 author_begin
   author_id fb_group.ukrainska_poezia_i_proza,zubchenko_nadia.kyiv
 author_end
\fi

\obeycr
Знайомтесь: український геній.
Живе сьогодні, поруч, серед нас.
Такі митці народжувались у епохи попередні.
Івана Марчука - обрав наш час.
Багато перешкод йому прийшлось здолати.
Бідність в дитинстві. Й олівцю було, радів...
"Інакшість" в юності. Ну як її сховати?..
Осуд за те, що малював так, як хотів.
Не зміг скоритися. Не став одноманітним,
як за радянщини ухвалено було.
Тому і нехтували вільнолюбцем "ненадійним",
та визнання до нього, попри все, прийшло...
Рим. Міжнародна академія мистецтва.
Він членом Золотої гільдії стає.
Висока честь - отримати партнерство,
яке окрилює й наснаги додає.
Ігнорування, утиски, памфлети...
А він творив... І успіх сам його знайшов.
За версією знаної британської газети,
у сотню геніїв сучасних увійшов.
Його роботи повторити неможливо.
Магічну таїну полотен знає лише він...
У вічі дивиться ним виплекане диво
в простій майстерні, зі звичайних стін.
Дивує особливий стиль мистецький.
І це не реалізм чи романтизм...
Ні, народився український, марчуківський,
відомий тепер всюди - пльонтанізм.
Із переплетень тисяч тонких ліній,
неначе в вишуканому шитві,
з'являється об'ємність і світіння...
І дихає життя на полотні...
Чекають майстра кращі виставкові зали.
Зачарували світ його картини...
Важливо, щоб і ми про нього знали:
Іван Марчук - жива легенда України.
пльонтанізм від діалектизму "пльонтати" - переплітати.
\smallskip
Надія Зубченко
\restorecr

\ii{25_12_2022.fb.fb_group.ukrainska_poezia_i_proza.1.ukr_genij_marchuk.orig}
\ii{25_12_2022.fb.fb_group.ukrainska_poezia_i_proza.1.ukr_genij_marchuk.cmtx}
