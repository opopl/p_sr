% vim: keymap=russian-jcukenwin
%%beginhead 
 
%%file 18_06_2020.fb.zharkih_ekaterina.1.mova_jazyk.cmt.ljubska_mova_divcha
%%parent 18_06_2020.fb.zharkih_ekaterina.1.mova_jazyk.cmt
 
%%url 
 
%%author 
%%author_id 
%%author_url 
 
%%tags 
%%title 
 
%%endhead 
\paragraph{Стільки люду знищили, насаждаючи російську мову і тепер якесь дівча}

\emph{Надія Лубська}

Стільки люду знищили, насаждаючи російську мову і тепер якесь дівча, що не знає
історії і не має поняття любові до України, як держави, бо країна без державної
мови, це колонія ... пише таку несінитницю...

\begin{itemize}

\emph{Юлия Юлечка}
\textbf{Надія Лубська} да сколько можно а???
Все нормальные люди знают историю!

И как ограбили Украину и уничтожали русские историю ее ,Голодомор отдельно
вообще боль!И как Польша нас тут в дерьмо смешивала рвала на куски!!!чего ж вы
молчите и претесь туда!?кушать хочется ,ДА??

Так при чем тут язык !!!

Вам мадам лучше будет от этого или как??
 · Reply · 24w
\emph{Olga Golik}
\textbf{Надія Лубська} вы превратились в колонию благодаря вашим любым друзям.
Никогда нас язык не делил на украинцев и неукраинцев. Я живу в Украине и говорю
на русском и это не значит, что не люблю свою страну. Никто меня не заставит
говорить иначе.

\emph{Валентина Зарейчук}
\textbf{Надія Лубська} приведите пример уничтожения людей за мову,интересно очень.
 · Reply · 24w
\emph{Lyusya Khramova}
\textbf{Надія Лубська} ,шановно! Українською буде "нісенітниця". І ваш коментар як раз і є прикладом нісенітниці. Якщо і були свого часу по відношенню до українців репресії,то не через їх мову. На то були зовсім інші причини,які,до речі,торкнулися й представників інших народів і національностей.

\emph{Alenka Bokareva}
\textbf{Надія Лубська} насаждали как раз украинский язык, а не русский
 · Reply · 24w
\emph{Єгор Луговик}
\textbf{Юлия Юлечка} тобі мало? Що ти тут маніпулюєш? Кушать їй хочеться. Не
рефлексуй, випий заспокійливі і лягай спати

\emph{Єгор Луговик}
\textbf{Olga Golik} ольгінський бот, накинулись на людину. Аргументи будуть? Чи тільки пуста демагогія?
 · Reply · See Translation · 24w
\emph{Єгор Луговик}
\textbf{Валентина Зарейчук} 400 років русифікації. Погулять, накінець. Бо ніхто вам нічого доводити не буде. Немає сенсу дурню пояснювати, чому небо голубе.
 · Reply · See Translation · 24w
\emph{Єгор Луговик}
\textbf{Люся Храмова} вас вчили по радянській системі, де не визнавали ніяких
народів, натомість депортували та асимілювали. Білорусів, Українців, Кримських
татар русифікували уже давно. Білоруську у Білорусії і досі вважають мовою
меншини і це не є нормально. Це про інші народи, національності та мовне
питання. Ваше заперечення немає нічого спільного з реальністю.

\emph{Єгор Луговик}
\textbf{Аленка Бокарева} судячи по твоїй сторінці і тому, що репостиш, ти піздобол і сепар
 · Reply · See Translation · 24w
\emph{Валентина Зарейчук}
\textbf{Єгор Луговик} в точку.
 · Reply · 24w
\emph{Єгор Луговик}
\textbf{Валентина Зарейчук} ну хоч розумієте свою помилку.
 · Reply · See Translation · 24w · Edited
\emph{Elena Andrushchenko}
\textbf{Єгор Луговик} Бред......

\emph{Єгор Луговик}
\textbf{Елена Андрущенко} на жаль, у вас в голові.
 · Reply · See Translation · 24w
\emph{Elena Andrushchenko}
\textbf{Єгор Луговик} котик....
 · Reply · 24w
\emph{Єгор Луговик}
\textbf{Елена Андрущенко} мяу

%%%fbauth
%%%fbauth_name
\iusr{Lyusya Khramova}
%%%fbauth_url
\urlFriend{https://www.facebook.com/priglashayu}
%%%fbauth_pic
\ifcmt
  ig https://scontent-cdg2-1.xx.fbcdn.net/v/t1.6435-9/185059798_2879199929026736_3884656571075756264_n.jpg?_nc_cat=102&ccb=1-3&_nc_sid=09cbfe&_nc_ohc=EyCj6bbwxRoAX-GOBAP&_nc_ht=scontent-cdg2-1.xx&oh=305822340cde15ba5e9182f39dd0170d&oe=60EB4AC5
  width 0.2
\fi
%%%endfbauth

\textbf{Єгор Луговик}, я, зазвичай, з котами не розмовляю.)) Це - щодо вашої
аватарки. А на все інше, що ви понаписували в своєму коменті, можу відповісти.
Я дійсно вчилась за часів Радянського Союзу. Вступила до Львівського
університету наприкінці семидесятих. За всі роки навчання тільки один предмет
нам викладали тоді російською мовою - політекономію. Решта - українською!

Спілкувалися між собою мої одногрупники майже всі українською. Російськомовних
в групі на 26 студентів нас було тільки дві - я і ще одна дівчина,яка так само
як і я, закінчувала у Львові російську школу. Російські школи тоді були, але
переважали все одно українські. 

Мій батько, який був народжений у невеликому містечку Львівської області ще за
Польщі, тільки завдяки політиці українізації, а не міфічної русифікації, про
яку ви пишете, зміг отримати і середню, і вищу освіту українською мовою, що за
часів Польщі було абсолютно неможливо. 

Так, при СРСР російську мову вивчали, бо то була мова загального спілкування по
всіх теренах величезної країни. Знаючи російську, ти міг спокійно їхати у
відрядження у Казахстан або Туркменію, а відпочивати у Грузію або Латвію,що ми
і робили. 

І всюди ми могли порозумітися, хоча у всіх республіках була своя мова
спілкування. Пам'ятаю, як мене вразили вивіски в Казахстані. Начебто написані
звичними для нас буквами кирилиці, але читалося все, як повна абракадабра.))

Російську мову ніхто тоді не насаджував, вона просто була необхідна людям
різних республік для простішого спілкування. Російськм мова тоді об'єднувала
всіх в одній величезній країні. До речі, українська мова могла би так само
стати об'єднуючим фактором для багатонаціональної України. Але для цього зовсім
не треба забороняти спеціальним законом використання інших мов, бо це викликає
тільки зустрічний супротив і відштовхує від української.

А вас, з вашим історичним невіглаством, мені щиро шкода. Уявляю, як вам важко
жити із такими знаннями.


%%%fbauth
%%%fbauth_name
\iusr{Єгор Луговик}
%%%fbauth_url
\url{https://www.facebook.com/ehor.luhovik}
%%%fbauth_pic
\ifcmt
  ig https://scontent-cdt1-1.xx.fbcdn.net/v/t1.6435-9/158513178_442809416965398_2148738451334202769_n.jpg?_nc_cat=105&ccb=1-3&_nc_sid=09cbfe&_nc_ohc=2-Y7-KP1AHUAX_ZLT4A&_nc_ht=scontent-cdt1-1.xx&oh=166ea0242cf96a8667eb9d4bc2413880&oe=60EBE671
  width 0.2
\fi
%%%endfbauth
\textbf{Люся Храмова} чиста маніпуляція. Ви брешете і не червонієте. Хто
забороняє використання російської мови? Ви накрутили себе, російськомовні, аж
смішно)))) Хай відштовхує як можна далі, бо коли нашу історію називаєте
невіглаством, уже не хочеться ніякого миру, тим більше руцкого. Ви оперуєте
історичним періодом, який не є прикладом об'єднання народів. Навпаки, тюрма
народів, де депортували та репресували величезну кількість національностей. 

Розстрілювали, засилали в Сибір. Я вам бажаю того, що ви проповідуєте, терору і
репресій. Адже історична пам'ять українця не починається із радянською союзу, а
задовго до того. Тому саме ваше знання історії більше схоже на гарну ширму, яку
та активно пропагували радянські пропагандисти.

\emph{Lyusya Khramova}
\textbf{Єгор Луговик}, відповім спеціально для людини, яка ідентифікую себе з котом і не розуміє найпростіших текстів. Невіглаством я назвала не історію,а ваше знання її. Це - по-перше! По-друге,я можу з достовірністю говорити про те,чому я була абсолютним свідком. Не відкидаю репресій,які свого часу відбувались в Україні через розкулачування та співробітництво деяких представників західноукраїнських земель з фашистами під час війни.Але стверджувала і стверджую,що ви безпідставно брешете,коли говорите,що ці репресії відбувались за мовною ознакою! Не брешіть,або поміняйте своє фото на аватарці на собачку!
Я вам вже сказала,що мені шкода вас. Ви народилися в "тюрмі народів" ,тому,напевно,ваша освіта нагадуює тюремну.)))

\end{itemize}
