% vim: keymap=russian-jcukenwin
%%beginhead 
 
%%file 28_08_2021.fb.denisenko_ljudmila.doneck.1.den_shahtera_doneck
%%parent 28_08_2021
 
%%url https://www.facebook.com/LED65/posts/4548557681861524
 
%%author Денисенко, Людмила (Донецк)
%%author_id denisenko_ljudmila.doneck
%%author_url 
 
%%tags donbass,doneck,gorod,pozdravlenie,prazdnik,shahter
%%title День любимого города и день самой донецкой профессии - ШАХТЁРА
 
%%endhead 
 
\subsection{День любимого города и день самой донецкой профессии - ШАХТЁРА}
\label{sec:28_08_2021.fb.denisenko_ljudmila.doneck.1.den_shahtera_doneck}
 
\Purl{https://www.facebook.com/LED65/posts/4548557681861524}
\ifcmt
 author_begin
   author_id denisenko_ljudmila.doneck
 author_end
\fi

Доброе утро, мои дорогие друзья и читатели!

Последнее воскресенье августа в моём родном Донецке - двойной праздник. День
любимого города и день самой донецкой профессии - ШАХТЁРА, которая не зря
считается одной из самых опасных и экстремальных в мире. Мой дед работал
маркшейдером (для тех, кто не в курсе, маркшейдер - это специалист по
геодезическим съёмкам горных разработок) в ш/у «Петровское». 

\ifcmt
  tab_begin cols=3

     pic https://scontent-cdg2-1.xx.fbcdn.net/v/t39.30808-6/240585621_4548556725194953_977394848358249041_n.jpg?_nc_cat=108&_nc_rgb565=1&ccb=1-5&_nc_sid=730e14&_nc_ohc=msTGZNApH0sAX-rZOOS&_nc_ht=scontent-cdg2-1.xx&oh=a0a19bdfbf50fbbdd9d793b2ae09e987&oe=61304222

     pic https://scontent-cdg2-1.xx.fbcdn.net/v/t39.30808-6/240460105_4548556745194951_6802359967879986114_n.jpg?_nc_cat=102&_nc_rgb565=1&ccb=1-5&_nc_sid=730e14&_nc_ohc=9cTyOTWy_mcAX_bFimQ&_nc_oc=AQkog1jLNs_Wb5R2C0W61uPAhGqZYPK59ha24pd-PQiMG7FmGuiYzviJAXSdype6qVQ&_nc_ht=scontent-cdg2-1.xx&oh=6b688ab5cd44ee3c5adc34ec0075c0ad&oe=612F5ED0

		 pic https://scontent-cdg2-1.xx.fbcdn.net/v/t39.30808-6/240384077_4548556758528283_5824086368145947875_n.jpg?_nc_cat=111&_nc_rgb565=1&ccb=1-5&_nc_sid=730e14&_nc_ohc=l4bL1dI3L2IAX-Agkwa&_nc_ht=scontent-cdg2-1.xx&oh=148ac2cd8093553bdf651fce089d8693&oe=612FF286

  tab_end
\fi


Папа был горным электромехаником, закончил Донецкий политехнический институт и
на заре своей карьеры спускался в шахту, да и на пенсии работал дежурным на
вентиляционном стволе шахты «Куйбышевская», и мама закончила горный техникум по
специальности ГЭМ, тоже несколько лет проработала в шахте. 

И муж мой закончил Донецкий политехнический институт по специальности
«Шахтострой», и начинал свою карьеру горным мастером, затем начальником
участка, главным инженером, начальником ШСУ (шахтостроительного управления). 

\ifcmt
  tab_begin cols=2

     pic https://scontent-cdt1-1.xx.fbcdn.net/v/t39.30808-6/240393496_4548556881861604_623313157550989065_n.jpg?_nc_cat=105&_nc_rgb565=1&ccb=1-5&_nc_sid=730e14&_nc_ohc=nAHuYEVK0CoAX_28Sd3&_nc_ht=scontent-cdt1-1.xx&oh=bd4a49a0257d086b6572609d51fa4329&oe=61313B59

     pic https://scontent-cdt1-1.xx.fbcdn.net/v/t39.30808-6/240466876_4548557565194869_220877602104681195_n.jpg?_nc_cat=103&ccb=1-5&_nc_sid=730e14&_nc_ohc=TJJIiwPBWHcAX__BTX4&_nc_ht=scontent-cdt1-1.xx&oh=649e2ed9037f538f309a921e6476f0f4&oe=613027F6

  tab_end
\fi

В общем, я не понаслышке знаю, что такое шахтёрский труд. И как миллионы
донбассовцев всегда с почтением отношусь к людям, которые каждый день
спускаются в забой добывать «чёрное золото», а их женам и матерям признательна
за то, что хранят очаг, ждут и тепло встречают своих добытчиков.

И как любой житель Донбасса, знаю, что такое «тормозок»! Для москвича или
киевлянина слово «тормозок» может означать что угодно, но жители всех
шахтерских регионов точно знают его настоящий смысл.

Итак, Шахтёрская еда – тормозок.

И да, дорогие мои читатели, реклама на сайте не для красоты - это ваша реальная
помощь нуждающимся людям и животным военного Донецка, потому, потратьте пару
минут - ознакомьтесь с заинтересовавшей картинкой! Спасибо!

\href{https://www.abcslim.ru/news/11289/shahtjorskaja-eda-tormozok/}{%
Шахтёрская еда – тормозок, abcslim.ru, 28.08.2021%
}
