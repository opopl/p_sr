% vim: keymap=russian-jcukenwin
%%beginhead 
 
%%file 18_03_2023.fb.krutenko_maryna.mariupol.1.voina_eto_deistvitel.cmt
%%parent 18_03_2023.fb.krutenko_maryna.mariupol.1.voina_eto_deistvitel
 
%%url 
 
%%author_id 
%%date 
 
%%tags 
%%title 
 
%%endhead 

\qqSecCmt

\iusr{Irina Matushina}

Ты описывала свои переживания, поэтому не надо извиняться!!!!! Люди были разные....

\iusr{Maryna Krutenko}

Чтоб выехать из Мариуполя в Милекино 15.03, водители микроавтобусов просили 20
000 грн. Люди которые ничего не имели, даже не имели шанса уехать...

\begin{itemize} % {
\iusr{Віктор Куліков}
\textbf{Maryna Krutenko} мы одни из них
\end{itemize} % }

\iusr{Ірина Холод}

Главное в поступках людей - это мотив. Если речь идёт о сохранении жизни, своей
или близких, то это одно дело. И совсем другое дело, когда на горе и бедах
сознательно наживаются. Если такой мотив движет людьми, то, я думаю, нажитое
будет не в прок. Я думаю, рано или поздно, правило бумеранга сработает.

\begin{itemize} % {
\iusr{Віктор Куліков}
\textbf{Ірина Холод} 

то, что происходило в Мариуполе.... там не было мотива... люди делали всё что
могли, что бы выжить.... кто как мог.... а совесть, характер, образ жизни и образ
мыслей у каждого по разному... не все лезли в магазины... но и не нам их
судить...

Я видел, что делала мать десятерых детей и видел двух здоровых лбов в
строительном магазине на микроавтобусе... но что двигало всех этих людей я не
знаю.... и не сужу никого.... всем было страшно и все пытались спасти свою
жизнь... только все разными способами...

\iusr{Ірина Холод}
\textbf{Віктор Куліков}, жизнь сама рассудит (или Бог). Нам даже не стоит беспокоится. А человек на то и человек, чтобы быть мотивированным. Иначе далеко можно зайти.
\end{itemize} % }

\iusr{Maryna Krutenko}
\textbf{Ірина Холод} 

я думаю, что в тяжелые времена вылезает настоящая натура человека. Есть
исключения из правил. Когда мне одна женщина сказала: «я не знаю как бы я
поступила», я сказала, что то кем она является когда никто не видит, вылезет в
первую очередь.

Не все же детей своих ели во время голода....

\begin{itemize} % {
\iusr{Ірина Холод}

Да, Марина. Главное - понимать кто ты, на что ты способен, а что не сможешь
сделать ни при каких обстоятельствах. Да, в таких ситуациях ты истинный. Ты
именно понимаешь, какой ты на самом деле. Иногда это очень трудно принять.

\end{itemize} % }

\iusr{Яна Лысенко}

Да назвать людей мародерами которыми нечего не оставалось делать в этой
ситуации тоже конечно надо было удержаться от своего мнения ставя всех под одну
планку 😘😘.не в обиду Маринка ❤️

\iusr{Maryna Krutenko}

Я не обижаюсь. Под одну гребёнку поставила, я не права. Но такие были. Шубы,
техника, гантели, мячи, чехлы для телефонов воровали зачем???

Еда и лекарства, это необходимость...

Ладно даже водка, может народ раны обрабатывали...

Стремянка, какая необходимость было ее воровать?

\begin{itemize} % {
\iusr{Ірина Холод}

Это нужно проговаривать. Я считаю, что Марина права. Тем более, что мы тоже
были в этом аду.

\end{itemize} % }

\iusr{Сергей Будченко}

На Митрополитской возле магазина Планета был Каскад строительный магазин, я
проходил мимо дома в котором он находился из подъезда торчала пачка
гипсокартона, понятно что не купленого, 16 марта проходил мимо с 17 дом сгорел
полностью, я ещё подумал пригодился ли этим людям гипсокартон?

\iusr{Victoria Kharmysheva}

Бог всегда смотрит на мотивы... а не на поступки... поэтому Он и есть любящий
ПАПУСЯ... милости всегда больше осуждения.

Все скоро станет на место.. и уже будет лучше и краще, и чище.. Украина
расцвете.. и люди з нею.. а орки.. здохнуть.. ибо вони орки.. из ада вышли.. туда им
и дорога..

Слава Украине и нашим героям.. людям.. воинам.. волонтерам.. нашим дiтям
теж.. потому что они не должны это проходить.. Слава нашему живому БОГУ
🇺🇦

\iusr{Ruslan Paul}

У моей дочери Насти в подвале церкви начался страшный бронхит переходящий в
пневмонию в самый разгар бомбёжек и ей нужны были антибиотики. Я пошёл по
тыщику, по всему городу шариться по аптекам, мне было очень страшно, везде
стреляли, бомбы падали. Все аптеки уже были вскрыты, я обыскивал все ящики в
аптеках в надежде что хотя бы что то найти. И надеялся что хотя бы одна аптека
будет не вскрыта, что я буду первый. Но уже было пусто ...

И в магазинах продуктовых \enquote{мародёрил} с пацанами, что бы была еда у 300
человек в нашем подвале. Было конечно очень стыдно первое время, но инстинкт
выживания побеждал. Ты не знал сколько этот ад продлится и сколько ещё нужно
что бы можно было выжить и прокормиться самому и хватило людям. Конечно были
братья, которые на много больше заботились. И им весь почёт...я ими восхищаюсь,
потому что они до сих пор в опасности служат людям и Украине. Пусть их Бог всех
бережёт.

\begin{itemize} % {
\iusr{Maryna Krutenko}
\textbf{Руслан Пауль} вы для меня герои!!!!
\end{itemize} % }

\iusr{Андрей Мартынов}

Мародерство когда ты берешь то что тебе хочется, зная что это не принесет тебе
пользы: телевизор, в то время когда нет света, а когда ты берешь что-то чтобы
выжить: медикаменты, еду, средства личной гигиены, то это наврятли можно
осудить. Хотя осудить могут те, кто там не был. Это всегда проще.

\iusr{Валентина Самарина}

Мородерство это когда тебя нет в квартире но ты рядом просто спустился вниз там
безопасней а сосед выносит продукты и вещи это мородерство когда ходят по
квартирам в своем же доме. А когда нужно накормить детей и стариков это нет.

Честно с соседом поступила не по христиански дала в морду и по ....не за себя
за соседей стариков. Даже асвабалители не прпятствовали
