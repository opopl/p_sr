% vim: keymap=russian-jcukenwin
%%beginhead 
 
%%file 19_04_2021.fb.respublikalnr.2.ukr_obrazovanie_soros
%%parent 19_04_2021
 
%%url https://www.facebook.com/groups/respublikalnr/permalink/798912634077863/
 
%%author 
%%author_id 
%%author_url 
 
%%tags 
%%title 
 
%%endhead 
\subsection{Газета \enquote{Республика} (№15, 2021г).  Деградация.  Несознательных оставят без стипендии}
\Purl{https://www.facebook.com/groups/respublikalnr/permalink/798912634077863/}

\ifcmt
  pic https://scontent-iad3-2.xx.fbcdn.net/v/t1.6435-9/175990227_124162549763853_4441645094170544793_n.jpg?_nc_cat=101&ccb=1-3&_nc_sid=825194&_nc_ohc=56RF_P7ahF0AX_SUn0E&_nc_ht=scontent-iad3-2.xx&oh=83d56eaa346e51bc493d1b026149ff0e&oe=60A25D01
\fi

«Соросятина» превращает учебные заведения в идеологические питомники укропатриотов.

Недолго радовалась оппозиция назначению министром образования Сергея Шкарлета –
бывшего члена Партии регионов и соратника Дмитрия Табачника. Почувствовавший
модные «тренды» Шкарлет подружился с парламентским соросятником.

Две якобы непримиримые стороны заключили негласное соглашение: Шкарлет не
пересматривает начатые соросятами реформы, а они, в свою очередь, перестают
атаковать министра в медийном поле. 

А чем же будет заниматься министр Шкарлет на столь значимом посту, если влияние
на образовательную сферу перехватывают структуры Сороса? Да, собственно говоря,
тем же, что и все его предшественники, – «пилить» бюджетные средства и
проворачивать коррупционные схемы. А в это время молодых и перспективных будут
отправлять в мясорубку соросовских «нарративов», как студентов северодонецкого
вуза-беженца, самоназвавшегося «ВНУ им. В. Даля». Вместо того чтобы изучать
технические науки, студентов обязывают принимать участие в мероприятиях,
финансируемых зарубежными некоммерческими организациями. 

Руководство вуза требует заполнение «листа активности», который нужно регулярно
заполнять и каждый месяц сдавать куратору. Если явка на соросовские мероприятия
низкая, студент лишается стипендии. Для особо злостных «нарушителей»
предусмотрено отчисление из университета.

Студенты не особо хотят посещать мероприятия, где коучи по соросовским
методичкам промывают мозги чужой идеологией. Но сопротивляться, увы, тоже не
могут, поскольку никто не хочет быть отчисленным из университета, особенно на
старших курсах. Организаторов, попытавшихся обратиться к ректору с требованием
отменить обязательное участие во внеучебных мероприятиях, припугнули серьёзным
разбирательством. 

Соросятина, оккупировавшая министерство образования, упорно и планомерно
превращает учебные заведения в идеологические питомники укропатриотов. Ну а
старая «гвардия» коррупционеров от образования увидела новые возможности и
головокружительные перспективы для личного обогащения. 

К примеру, в Киеве ректором КНУ стал коррупционер-националист с русской
фамилией Бугров. Укропатриоту добыл место чёрный политтехнолог Андрей
Коваленко, которому поручено в новом информационном центре по борьбе с
дезинформацией Донбасское направление. Коваленко считается хорошим специалистом
в организации ботоферм и распространении фейков. По слухам, нынешний ректор КНУ
заплатил политтехнологу 10 тысяч долларов. Но главное – он дорвался до
бесценной киевской земли, принадлежащей университету. 

Вот и получается: управленцы воруют, соросятина руководит образовательным
процессом, а студенты входят из «выщив» толерантными недорослями. 

Игорь МИРТОВСКИЙ, ГАЗЕТА "РЕСПУБЛИКА" (№15, 2021г).

\verb|#газета #республика #политика_Украины #образование|
