% vim: keymap=russian-jcukenwin
%%beginhead 
 
%%file 02_12_2020.news.ru.lenta_ru.mozzhuhin_andrei.1.pribaltika.podarok_ot_stalina
%%parent 02_12_2020.news.ru.lenta_ru.mozzhuhin_andrei.1.pribaltika
 
%%url 
 
%%author 
%%author_id 
%%author_url 
 
%%tags 
%%title 
 
%%endhead 

\subsubsection{Подарок от Сталина}

\textbf{«Лента.ру»:} В своей книге вы указываете, что существующие в предвоенных
прибалтийских государствах «социально-политические условия в значительной
степени облегчили Советскому Союзу реализацию планов по включению Латвии, Литвы
и Эстонии в свою геополитическую орбиту». Что под этим подразумевается?

\textbf{Юлия Кантор:} 

\ifcmt
pic https://icdn.lenta.ru/images/2019/01/22/12/20190122125338421/preview_a51ce0042251b524973801d8db1f4dc6.jpg
caption Юлия Кантор
fig_env wrapfigure
width 0.2
\fi

Что почти во все годы межвоенного существования независимых прибалтийских
государств там функционировали жесткие авторитарные режимы, дрейфовавшие в
сторону диктатуры. После государственных переворотов в Литве в 1926 году, а
также в Латвии и в Эстонии в 1934 году, там фактически не работали парламенты,
активно действовала свирепая политическая полиция, существовала жесткая цензура
и были запрещены коммунистические партии.

Во всех трех странах накануне 1939-1940 годов не существовало условий для
демократического развития или какого-либо серьезного гражданского
самовыражения. Во многом именно это обусловило сравнительную безболезненность
включения Литвы, Латвии и Эстонии в орбиту Советского Союза. Ну и, безусловно,
податливость тамошних лидеров, которые тихо подчинялись нажиму из Москвы,
сдавали собственные позиции и независимость своих государств. И при этом вели
успокоительную риторику и внутри своих стран, и во внешний мир.

\ifcmt
pic https://icdn.lenta.ru/images/2020/10/20/16/20201020162102743/preview_4cf477194cab6288f216380170849d95.jpg
fig_env wrapfigure
width 0.3
\fi

\lenta{Каким термином, на ваш взгляд, было бы корректно обозначать включение Прибалтики в состав СССР в 1940 году? Сейчас в этих странах говорят об «оккупации» или об «аннексии», а нынешний российский президент Путин недавно употребил слово «инкорпорация».}

Я в книге использую термин «советизация». Под ним подразумевается весь комплекс
идеологических, военных, политических и экономических мер, с помощью которых
Советский Союз в 1939-1940 годах подготавливал присоединение Прибалтики,
формально осуществленное законными властями всех трех стран. А как именно все
это происходило, в книге достаточно подробно описано. Податливость
прибалтийских элит сочеталась с недвусмысленными угрозами применения военной
силы со стороны Москвы и одновременно щедрыми посулами от нее.

\lenta{Говоря про обещанные щедрые посулы, вы имеете в виду передачу Литве польского
города Вильно?}

Это самый яркий пример. В современной Литве не любят вспоминать, что ее
нынешняя столица Вильнюс стала для нее сталинским подарком — в обмен на
беспрепятственное размещение советских военных баз на литовской территории. Это
случилось в октябре 1939 года, тогда Литва еще была независимым государством.
Когда в сегодняшней Литве критикуют пакт Молотова – Риббентропа, как-то
«забывают», что обретение столицы литовского государства — его прямое
следствие.

\lenta{В книге вы рассказываете, как в октябре 1939 года после размещения в Прибалтике
советских военных баз, проведенного после заключения договоров с
правительствами независимых балтийских государств, Молотов с трибуны Верховного
Совета СССР официально заверил мировое сообщество об отсутствии планов их
советизации. Примерно то же самое несколькими днями раньше в личной беседе
Сталин заявил и лидеру Коминтерна Георгию Димитрову. Как вы думаете, насколько
кремлевские руководители в тот момент были искренни или вся их риторика
изначально была лживой?}

Мне трудно сказать, о чем на самом деле думал Сталин. Возможно, он просчитывал
варианты дальнейшего развития событий в Европе, внимательно наблюдая за
поведением прибалтийских государств и отслеживая реакцию Англии и Франции.

\ifcmt
tab_begin cols=2
	caption Книга «Прибалтика. 1939-1945 гг. Война и память», РГАСПИ

	pic https://icdn.lenta.ru/images/2020/10/20/15/20201020155708415/pic_963b14c11160b1511dbd0827cc14e62e.jpg
	caption Представители Красной армии и Краснознаменного Балтийского флота приветствуют депутатов Государственной думы Эстонии. Таллин, 22 июля 1940 г.

	pic https://icdn.lenta.ru/images/2020/10/20/15/20201020155711982/pic_d9a8d92843e3697bc579983044cf7390.jpg
	caption Девушка с острова Муху (Моон). Эстония, 1940 г. 
tab_end
\fi

Но западным демократиям, по-прежнему мечтавшим столкнуть лбами СССР и Германию,
тогда было не до Прибалтики, а правительства Латвии, Литвы и Эстонии после
заключения договоров о размещении у них советских войск выступили с примерно
такими же заявлениями, что и Молотов. Наверняка такое их податливое поведение
побудило Сталина к проведению более активной и брутальной политики.

