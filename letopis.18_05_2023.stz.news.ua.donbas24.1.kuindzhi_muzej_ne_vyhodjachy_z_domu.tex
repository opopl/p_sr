% vim: keymap=russian-jcukenwin
%%beginhead 
 
%%file 18_05_2023.stz.news.ua.donbas24.1.kuindzhi_muzej_ne_vyhodjachy_z_domu
%%parent 18_05_2023
 
%%url https://donbas24.news/news/mariupolskii-muzei-im-ai-kuyindzi-vidteper-mozna-vidvidati-ne-vixodyaci-z-domu-foto
 
%%author_id veremeeva_tetjana.zhurnalist.donbas24.sumy,news.ua.donbas24
%%date 
 
%%tags 
%%title Маріупольський музей ім. А. І. Куїнджі відтепер можна відвідати, не виходячи з дому (ФОТО)
 
%%endhead 
 
\subsection{Маріупольський музей ім. А. І. Куїнджі відтепер можна відвідати, не виходячи з дому (ФОТО)}
\label{sec:18_05_2023.stz.news.ua.donbas24.1.kuindzhi_muzej_ne_vyhodjachy_z_domu}
 
\Purl{https://donbas24.news/news/mariupolskii-muzei-im-ai-kuyindzi-vidteper-mozna-vidvidati-ne-vixodyaci-z-domu-foto}
\ifcmt
 author_begin
   author_id veremeeva_tetjana.zhurnalist.donbas24.sumy,news.ua.donbas24
 author_end
\fi

\begin{qqquote}
Стало відомо, як побачити знищену та розкрадену росіянами експозицію
\end{qqquote}

\ii{18_05_2023.stz.news.ua.donbas24.1.kuindzhi_muzej_ne_vyhodjachy_z_domu.pic.1}

\href{https://donbas24.news/news/dramteatr-mariupolya-simvol-pamyati-boli-i-tugi-za-zagiblimi-mariupolcyami}{Росіяни знищили} 
багато історичних та культурних пам'яток в Україні, зокрема в
Маріуполі. Деякі визначні місця зруйнованого міста Марії продовжують існувати у
віртуальному світі й усі бажаючі можуть їх відвідати. Наприклад, зруйнований
маріупольський художній музей ім. А.І. Куїнджі - у мережі з'явився 3D-тур
експозицією художнього музею.

Яким чином можна побувати у музеї, Донбас24 розповіли у культурно-туристичному
центрі \enquote{Вежа}.

\textbf{Читайте також:}

\href{https://donbas24.news/news/do-tragicnoyi-ricnici-okupaciyi-v-mariupoli-u-kijevi-vidkrijetsya-nova-vistavka}{До
трагічної річниці окупації Маріуполя у Києві відкриється нова виставка}

Інформація, що внаслідок бойових дій музей ім. А. І. Куїнджі зруйнований
з'явилася у квітні 2022 року. Тоді ж стало відомо, що окупанти вивезли усі
експонати до тимчасово захопленого Донецька. На той момент вважалося, що музей
втрачений назавжди.

\ii{18_05_2023.stz.news.ua.donbas24.1.kuindzhi_muzej_ne_vyhodjachy_z_domu.pic.2}

Однак пізніше стало відомо, що засновник студії TravelboxVR на ім'я Сергій у
2021 році почав оцифровувати всі туристичні локації Приазов'я для великого
віртуального 3D-туру. Зйомка у Маріуполі відбулася у жовтні. Тоді вдалося
відзняти художній музей та музей народного побуту.

Відвідати 3D-тур художнім музеєм ім. А. І. Куїнджі можна за
\href{https://mistomariupol.com.ua/uk/tours/kuyindzhi-museum}{посиланням}.

\ii{18_05_2023.stz.news.ua.donbas24.1.kuindzhi_muzej_ne_vyhodjachy_z_domu.pic.3}

Решту знакових для Маріуполя об'єктів Сергій планував відзняти навесні 2022
року, але це, на жаль, не вдалося зробити у зв'язку з початком повномасштабного
вторгнення рф.

\textbf{Читайте також:} 

\href{https://donbas24.news/news/ukrayinska-ilyustratorka-malyuje-kartini-kavoyu-sered-yiyi-robit-arxitektura-mariupolya-foto}{Українська ілюстраторка малює картини кавою: серед її робіт - архітектура Маріуполя (ФОТО)}

\ii{18_05_2023.stz.news.ua.donbas24.1.kuindzhi_muzej_ne_vyhodjachy_z_domu.pic.4}

Раніше \textbf{Донбас24} розповідав, що \href{https://donbas24.news/news/den-muzeyiv-yak-rosiyani-ruinuyut-kulturnu-spadshhinu-donbasu-foto}{\em маріупольський художній музей ім. А. І. Куїнджі}
був розміщений у старовинній будівлі 1902 року. Він був присвячений одному із
найвідоміших маріупольців у світі - художнику Архипу Куїнджі. У трьох залах
музею зберігалися картини майстра.

\ii{18_05_2023.stz.news.ua.donbas24.1.kuindzhi_muzej_ne_vyhodjachy_z_domu.pic.5.kuindzhi_kartiny}

\textbf{Читайте також:} \href{https://donbas24.news/news/film-pro-mariupol-peremig-na-kinofestivali-u-ssa}{Фільм про Маріуполь переміг на кінофестивалі у США}

Також тут можна було побачити роботи відомих на весь світ мариністів та
реалістів: Івана Айвазовського, Олексія Боголюбова, Василя Верещагіна, Миколи
Дубовського, Лева Лагоріо.

Загалом експозиція музею нараховувала близько 2000 експонатів.

Ще більше новин та найактуальніша інформація про Донецьку та Луганську області
в нашому \href{https://t.me/donbas24}{телеграм-каналі Донбас24}.

Фото: Туристичний сайт Маріуполя
