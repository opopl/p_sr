% vim: keymap=russian-jcukenwin
%%beginhead 
 
%%file 24_11_2021.fb.taksjur_jan.1.radikalnoje_sredstvo.cmt
%%parent 24_11_2021.fb.taksjur_jan.1.radikalnoje_sredstvo
 
%%url 
 
%%author_id 
%%date 
 
%%tags 
%%title 
 
%%endhead 
\subsubsection{Коментарі}

\begin{itemize} % {
\iusr{Мирослава Александровна Бердник}

\ifcmt
  ig https://scontent-frx5-1.xx.fbcdn.net/v/t39.1997-6/s180x540/245618490_978248802757934_2347610152518002387_n.png?_nc_cat=110&ccb=1-5&_nc_sid=ac3552&_nc_ohc=2at-sTw1wbEAX8P-Vt3&_nc_ht=scontent-frx5-1.xx&oh=2b24b18a31cf51b411d2cc7f1e79b121&oe=61A2A801
  @width 0.2
\fi

\iusr{Мирослава Александровна Бердник}

Замечательно1

\iusr{Евгений Гороховский}
\textbf{Мирослава Александровна Бердник} согласен!

\iusr{Елена Самойленко}
Вы не просто талантливый писатель, вы провидец!

\iusr{Елена Елена}
Шедевр!

\iusr{Татьяна Херсонская}

За 7 лет многое изменилось. Выросли новые поколения счастливцев. Мистер Муланд
прислал джавелины. Теперь они будут пользоваться ими. Когда прочитают
инструкцию. Шлите же инструкцию мистер Во... Муланд!

\iusr{Serge Kompany}
В стиле Оруэлла.

\iusr{Станислав Шпиль}
Класс! Прямо провидчески. Единственно что изменилось это таблетки заменили уколами...

\iusr{Татьяна Чугаенко}
Ян, это гениально!

\begin{itemize} % {
\iusr{Ян Таксюр}
\textbf{Татьяна Чугаенко} Спасибо, Танюша. Честно говоря, считал проходным текстом, на злобу дня. Вроде, "статьи по поводу". А потом решил попробовать опубликовать. Рад, что тебе понравилось.

\iusr{Татьяна Чугаенко}
\textbf{Ян Таксюр}, мне не понравилось. Мне ОЧЕНЬ понравилось
\end{itemize} % }

\iusr{Marina Scherbakova}
 
\ifcmt
  ig https://scontent-frt3-1.xx.fbcdn.net/v/t39.1997-6/s180x540/259733587_434872271578084_6605020266068955425_n.png?_nc_cat=104&ccb=1-5&_nc_sid=ac3552&_nc_ohc=bJmn_8v5t2AAX_S8tf5&_nc_ht=scontent-frt3-1.xx&oh=36a8a443b53f627b374513b8603d1730&oe=61A263CC
  @width 0.2
\fi

\iusr{Дарья Максименко}
Как всегда прекрасно, Ян Ильич  @igg{fbicon.heart.red}

\iusr{Ирина Черниенко}
Как всегда в точку, тонко, иронично, просто нет слов, чтобы выразить восхищение.

\iusr{Татьяна Луника}
Семь лет....

\iusr{Елена Хатор}
Супер! Спасибо!

\iusr{Борис Кальчев}
Классно.

\iusr{Тамара Логинова}
В десятку !

\iusr{Виктор Щелинский}

И без таблеток хорошо помню: при Кучме дом поставил, при Ющенко, газ провёл ,
при "злочиннiй владi" кондиционер, спутниковое, интернет, ноутбук купил, потому
что пенсия 200 у. е и работал. При при "праведниках", не злодiях газом не
отапливаюсь, кондиционеры не включаю, спутниковое закодировали, интернет не
оплачиваю и не включаю и пенсия стала 100 уе.

\iusr{Alisa Katsen-Globa}

Это лучшее, что читала. Смех сквозь слёзы! И абсолютно точно. Надо бы перевести
мистеру Муланду и сотоварищам.

\ifcmt
  ig https://scontent-frx5-2.xx.fbcdn.net/v/t39.1997-6/p480x480/105941685_953860581742966_1572841152382279834_n.png?_nc_cat=1&ccb=1-5&_nc_sid=0572db&_nc_ohc=DkwTfYKMYwYAX-Jgl55&_nc_ht=scontent-frx5-2.xx&oh=e664bbb1aabeafa6660f78433b6922e0&oe=61A436CB
  @width 0.2
\fi

\iusr{Ольга Савка}
Печально, но такова реальность!
Спаси Бог Вас Ян!!!! Без комментариев!!!!

\iusr{Anton Denikin}
Прекрасно! Написано в стиле Шварца и Горина. Беда в том, что наша реальная жизнь уже восемь лет в пишется стиле Кафки.

\iusr{Евгения Заславская}
Семь лет! И ничего... Только ещё веселее.

\iusr{Лилия Боровских}
Ужасная правда...(((@igg{fbicon.heart.red}

\iusr{Иван Рябов}
Замечательная притча! Напомнило Шварца "Дракон".

\iusr{Александр Жуков}
Браво, Ян!)

\end{itemize} % }
