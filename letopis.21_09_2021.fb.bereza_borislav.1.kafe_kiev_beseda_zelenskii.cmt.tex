% vim: keymap=russian-jcukenwin
%%beginhead 
 
%%file 21_09_2021.fb.bereza_borislav.1.kafe_kiev_beseda_zelenskii.cmt
%%parent 21_09_2021.fb.bereza_borislav.1.kafe_kiev_beseda_zelenskii
 
%%url 
 
%%author_id 
%%date 
 
%%tags 
%%title 
 
%%endhead 
\subsubsection{Коментарі}

\begin{itemize} % {
\iusr{Sonya Ostrovsky}
Deep...

\iusr{Соня Мармеладова}
\textbf{Sonya Ostrovsky} Deeper...

\iusr{Олександр Мудрицький}
Все Так .

\iusr{Вадим Хейлик}
Масштабность Зе? Квадратура круга!

\iusr{Светлана Ледзевирова}
Ставлю лайк, а хочется рыдать

\begin{itemize} % {
\iusr{Виктория Эльцуфина}
\textbf{Светлана Ледзевирова} аналогично

\iusr{Тала З.}
Ставьте рыдающий смайл, если хочется рыдать!
Я именно так и сделала.

\iusr{Светлана Ледзевирова}
\textbf{Тала З.} Он подписывается как "сочувствую". Я конечно сочувствую,но это не выражает всего спектра моих чувств
\end{itemize} % }

\iusr{Dimitri Avaliani}

Простите у меня деликатный вопрос - /при Зеленском Украина действительно
превращается в какое-то квазигосударство с неработающими институциями/ - а до
Зеленского Украина такой не была?

\begin{itemize} % {
\iusr{Nadiya Pavlusiv}
\textbf{Dimitri Avaliani} Ні , не була! Зєля за два роки знищив і руйнує все, що було зроблене!

\iusr{Борислав Береза}
\textbf{Dimitri Avaliani} до него никто и никогда не имел такой власти, такого влияния и таких полномочий в одном лице. И все же так, как он никто бездарно не правил. Даже Янукович.

\iusr{Вячеслав Раєвський}
\textbf{Dimitri Avaliani} інколи по питанню розумієш, що не варто навіть відповідати

\iusr{Mariana Piskova}
\textbf{Dimitri Avaliani} 

пересмотрите тогда Украину 2 года назад и сейчас. За правление моносброда и ко
- уничтожено даже то что хоть немного работало.... @igg{fbicon.face.pleading} 

\iusr{Dimitri Avaliani}
\textbf{Borislav Bereza} Это правда, налицо профуканный шанс и бездарно растраченный кредит доверия. Но разрушить то, чего не было, не мог даже он. На мой дилетантский взгляд со стороны.

\iusr{Богдана Ковальчук}
\textbf{Dimitri Avaliani} 

не була..., але давайте гіпотетично уявимо, що була...питання - ви для чого чи
для цього Зеленського обирали?)

\iusr{David Davidoff}

До Зеленского, с 14-го, наметился напрям на развитие цивилизованной державы.
Который след было бы далее проявлять.

С 19-го - даже не откат, а падение за черту ниже янычарской поры. @igg{fbicon.hand.victory}
@igg{fbicon.face.smiling.eyes.smiling}  @igg{fbicon.sun.with.face} 


\iusr{Евгений Порцев}
\textbf{Борислав Береза} Нет, это не управление, а метод тыка.

\iusr{Светлана Прокопчук}
\textbf{Dimitri Avaliani} Нет,не была!!!

\iusr{Constantine Marakhov}
\textbf{Dimitri Avaliani} , 

Зеля зараз за ступенем упоротості десь на рівні "пізнього" Януковича. Його
поява дійсно є прямим наслідком попереднього цирку в політиці. ... І дійсно,
якби не його повна тупість і не істерика через те, що всім зрозуміла
необхідність його якнайшвидшого зникнення з посту президента, то можна було б
спокійно проаналізувати те, що трапилося за останні 7-8 років і зробити
правильні висновки ))


\iusr{Olga Golovatska}
\textbf{Dimitri Avaliani} 

можно...авторитет страны, до него Украина воспринималась с большим(ударение на
первый слог) уважением !


\iusr{Anton Pfennig}
\textbf{Dimitri Avaliani} Владимир Александрович, это вы?
\end{itemize} % }

\iusr{Olga Denisenko}
Надеюсь, папа не все деньги потратит в Лондоне

\begin{itemize} % {
\iusr{Yulia Basyubina}
\textbf{Olga Denisenko} если папа уже в Лондоне, то он хорошо подумал о будущем своих блудных детей. Беда в том, что не у всех папы в Лондоне.
\end{itemize} % }

\iusr{Анна Наговицына}

мой бывший университетский преподаватель стал астрологом, иногда читаю его
статьи ,так вот, он утверждает, что Зеленскому суждено стать последним
украинским президентом , после него страна или прекратит своё существование,или
кардинально поменяет устройство. Раз Зеленский обещал не идти на второй срок,то
ждать осталось немного

\begin{itemize} % {
\iusr{Борислав Береза}
\textbf{Анна Наговицына} Зеленский спит и видит второй срок. ))

\iusr{Юлія Олександрівна}
\textbf{Borislav Bereza} тюремний)

\iusr{Анна Наговицына}
\textbf{Borislav Bereza},а я думала в последний день первой каденции сядет на велосипед и укатит за горизонт  @igg{fbicon.laugh.rolling.floor}  обещал ведь)

\iusr{Юлія Олександрівна}
\textbf{Анна Наговицына} буде ХУНТА! І ТРИБУНАЛ! ЗЕЛЕНОМУ КОДЛУ!
P.s. Спитайте у свого астролога. (Казав зеля: зуб даю)

\iusr{Борислав Береза}
\textbf{Анна Наговицына} горизонт - это воображаемая линия. Поэтому он укатить может лишь в воображении.

\iusr{Тарас Загородний}

\textbf{Борислав Береза} тю. Там в третий смотрят. Не недооценивай

\iusr{Олег Котенко}
\textbf{Юлія Олександрівна} Простите, а вы уверены, что понимаете, что такое хунта?

\iusr{Юлія Олександрівна}
\textbf{Oleh Kotenko} звичайно)

\iusr{Олег Котенко}
\textbf{Юлія Олександрівна} И кого же вы видите в роли украинского Пиночета?

\iusr{Юлія Олександрівна}
\textbf{Oleh Kotenko} я ж написала під коментом про астролога, а я не астролог) але варіанти є) і це єдиний вірний не демократичний вихід для збереження Незалежності України) гра хуйлом йде не на життя) в демократію вже погралися) іншого шансу може не бути)

\iusr{Олег Котенко}
\textbf{Юлія Олександрівна} Ну, то есть, хунту вы хотите, а предположений насчет генерала-предводителя у вас нет. Ок. Допустим, Ярош подойдет?

\iusr{Constantine Marakhov}
\textbf{Borislav Bereza} "Зеленский спит и видит второй срок"... Зважаючи на те як його ковбасить і те що він і хвилини не може тримати нормальний вираз обличча, то сни там досить екзотичні. І до реальності тамдуже далеко. В кращому випадку це паралельна реальність ))

\iusr{Юлія Олександрівна}

Oleh Kotenko люди читають жопою) я не стверджувала ніде, що я хочу хунту) на мій погляд, це єдиний не демократичний варіант, на жаль в демократичний я не вірю) далі я написала, що варіанти є) але це мої варіанти по кандидатурам, вам їх озвучувати я не збиралася) і моя думка про Яроша: він не лідер)

@igg{fbicon.index.pointing.up} звичайно, все суб'єктивно) гарного дня)

\end{itemize} % }

\iusr{Yulia Basyubina}
Украина переживёт. Главное, чтобы был выбор после Зеленского.

\begin{itemize} % {
\iusr{Виктория Эльцуфина}
\textbf{Yulia Basyubina} выбор всегда есть, и в 19 тоже был

\iusr{Yulia Basyubina}
\textbf{Виктория Эльцуфина} на последних выборах президента Украины был выбор? Вы серьёзно?
Это, кстати, общемировой тренд: публичная политика всё меньше соответствует запросам граждан, которые каждый раз чувствуют себя как в лавке с десятью сортами протухшего сыра. И Зеленского, в частности, выбрали в знак протеста политике Порошенко. Нельзя загонять людей в угол. Ну и люди сами должны своевоеменно смекать, из чего что последует.

\iusr{Виктория Эльцуфина}
\textbf{Yulia Basyubina} я серьезно!! а вы можете протестовать дальше не думая о последствиях! Как-то так... к сожалению

\iusr{Соня Мармеладова}
\textbf{Yulia Basyubina} саботировать выборы это тоже выбор. Если все кандидаты гавно, можно просто не придти на выборы. И если таких людей будет много и выборы не будут считаться легитимными, то весь мир увидит, что в нашем королевстве что-то прогнило.

\iusr{Маргарита Марго Мар}
\textbf{Yulia Basyubina}, открою вам страшную тайну - и в 14-м был выбор. Но народ идет за дудочкой - водинтур, новыелитца и т.д. Думать логически народ не умеет - увы...

\iusr{Yulia Basyubina}
\textbf{Маргарита Марго} Мар не будьте такой злой. Пока не научились вживлять "чип демократии", ей придётся обучаться долго и терпеливо, это долгий процесс. Посмотрите на США, Великобританию. Без ошибок и глупостей даже через 200 лет не обходится.

\iusr{Маргарита Марго Мар}
\textbf{Yulia Basyubina}, при чем здесь злость? вы же не видите альтернативу, как и большинство. А времени у Украины нет на 200 лет обучения, все идет к финишу... Папа прав.

\iusr{Marija Dudene}
\textbf{Yulia Basyubina} , bil vibar iz 39 kandidatov, zabili... No vibrali ZE - raduites... P.s. Mečta sbilas - tolko ne Porošenko.

\iusr{Наталія Віннікова}
\textbf{Соня Мармеладова} открою вам вторую страшную тайну, в законодательстве Украины нет нижнего порога явки, чтобы признать выборы не легитимными

\iusr{Соня Мармеладова}
\textbf{Наталія Віннікова} а жаль

\iusr{Наталія Віннікова}
\textbf{Соня Мармеладова} согласна на все 100\%
\end{itemize} % }

\iusr{Boris Nemirovski}

Ну, насправді, можна стверджувати, що останнім справжнім імператором був
Діоклетіан Август. Це той самий, з капустою.  @igg{fbicon.smile}  Та й те - його правління та
"квадрига" були останнім сплеском порядку: до нього була ціла череда
"солдатських імператорів" (він, зрештою, й сам таким чином імператором став),
після нього - гризня між августами та цезарями...


\iusr{Мадлена Амстел}
Схоже на те, що синок під дією певних речовин.

\begin{itemize} % {
\iusr{Борислав Береза}
\textbf{Мадлена Амстел} не похоже. Они пили вино.
\end{itemize} % }

\iusr{Іра Раткевич}

Переживем, ще не таке проходили, як зелена шмаркля. А от що і як довго
прийдеться відновлювати і виправляти - то інше питання, не веселе, до речі.

Прорвемся! Мусимо, заради дітей, принаймні.

\iusr{Olga Shtepa}
Он хоть в Лондон сможет уехать, к отцу:)

\begin{itemize} % {
\iusr{Yulia Basyubina}
\textbf{Olga Shtepa} а это ключ у пониманию инфантильности и безответственности юных вольных "экспериментаторов"

\iusr{Юлія Олександрівна}
\textbf{Olga Shtepa} так, семидесятитрьохвідсоткові обрали зрадника-членограя, але єслішо, то вони в Лондон) а вигрібати хто мусить?!
 @igg{fbicon.face.angry.horns}{repeat=3}  сцуки!
\end{itemize} % }

\iusr{Dmytro Nohal}
Странное свидание у той пары

\begin{itemize} % {
\iusr{Борислав Береза}
\textbf{Dmytro Nohal} мне это не показалось свиданием. Скорее поздний обед.

\iusr{Dmytro Nohal}
\textbf{Борислав Береза} Но я так понял, они не часто встречаются, поэтому их обсуждение для меня странно, как то так.
\end{itemize} % }

\iusr{Kostyantyn Zorkin}
вроде и сложилось у человека все. и мозги на месте. а сына не смог воспитать

\begin{itemize} % {
\iusr{Максим Образенко}
\textbf{Kostyantyn Zorkin} зря вы так думаете. Как по мне, папа молодец. Не вмешивается в личное мнение сына, и этим его воспитывает.

\iusr{Алла Чебан}
\textbf{Kostyantyn Zorkin} Да, вы правы, я тоже не смогла сына воспитать. Я придерживаюсь принципа, человек, чтобы понял, должен мучиться и кстати эту фразу я первый раз услышала от моего сына. Теперь, я ему говорю тоже самое.
\end{itemize} % }

\iusr{Людмила Кужель}

Приклад Януковоща доводить, що в Україні не кожен президент добуває до кінця
своєї каденції, а Україна виживає навіть зі зруйнованими інституціями.

\begin{itemize} % {
\iusr{Соня Мармеладова}
\textbf{Людмила Кужель} этот не, Янукович. И его шобла отмороженная тем более. Зверье ещё то. Этих, смерть 100 человек не остановит. Скажут все были агенты Кремля, соберут РНБО и введут военное положение.

\iusr{Людмила Кужель}
\textbf{Соня Мармеладова} і такий варіант можливий.

\iusr{Соня Мармеладова}
\textbf{Людмила Кужель} нажаль.
\end{itemize} % }

\iusr{Алексей Нечаев}
Опять папередники виноваты.

\iusr{Alex Jack}

...а и сынок не ел, ибо загружен наркотой был, как и большинство политической
аэлиты. приходят в ресторан и не едят, а только в туалет бегают шморгануть ещё
разок


\iusr{Constantine Marakhov}

"А я возвращаюсь в Лондон. Когда у вас тут все на@бнется прилетай." ... Чувак
по ходу вже взагалі не має відношення до України. Це "англієць українського
походження". І ймовірно залишив батьківщину до того як Зеля став президентом. І
не тому що "правителі" тут тупі, а тому що він в принципі не вірить, що тут
може бути успіх і процвітання. ))) .. Він може бути 1000 разів правий, але
розмовляти на тему майбутнього України з людиною, яка знайшла тепле місце там,
де чужий нарід за нього вже організував цивілізоване середовище , і емігрував
не тому, що йому загрожувала небезпека, а просто тому, що він від народження не
знав нащо він живе на цій землі і єдиний принцип "уби бене ібі патріа" - це
себе не поважати  @igg{fbicon.face.grinning.squinting}{repeat=2} 

\iusr{Вадим Одаренко}
Переживёт. Но какая будет плата за это...

\iusr{David Davidoff}
На то его к булаве и протащили @igg{fbicon.hand.victory}  @igg{fbicon.face.smiling.eyes.smiling}  @igg{fbicon.sun.with.face} 

\iusr{Анжелика Барсукова}
Украина имеет запас прочности больше чем Римская империя... отлично же папаша историю понимает. Пусть летит в Лондон. Там все такие малограмотные.

\iusr{Игорь Клименко}

Наш нарИд к моему глубокому сожалению с таким аппетитом уплетал московитские и
околомосковитские кАнализации из всех труб про "зраду зрадную, наживательство
на войне и прочих вальцманов" что теперь не проймешь. Чисто детские отмазки -
"а вот до, а вот что ваш, а Иловайск" дабы оправдать свою наивную глупость,
прихоть и принципиальное нежелание признать собственную ошибку. 73\% умакнули
всю страну в дерьмо, предали бойцов которые полегли в боях. Ради обещаний
медийного клоуна и амбиций его обиженной бабушки.

ЇжТе!

\iusr{Constantine Marakhov}

Зеля рилом не вийшов, щоб щось будувати або розвалювати. Наш нарід пережив
стільки жахіття, а цей геній каже, що якась зелена шмаркля розвалить країну?
Самим не смішно? .. Чувак бахвалиться - як йому добре живеться в Лондоні і
каркає, що у нас тут все на@бнётся просто, щоб ще більше показати - який він
найрозумніший і успішний. І як йому звідти з Лондона комфортно каркати? Так він
підсвідомо ще й чекає, щоб тут накрилося все мідним тазом, щоб з поважним
виглядом сказати: А я ж казав !!! ))) ... Це розмова ні про що. Не дивно, що у
такого батька виріс такий син. Обидва без коренів і без базового розуміння
сенсу свого життя в Україні  @igg{fbicon.face.grinning.squinting} 

\begin{itemize} % {
\iusr{Костя Курліщук}
\textbf{Constantine Marakhov} без образ, але Ви також слабо знаєте історію.

\iusr{Constantine Marakhov}
\textbf{Костя Курліщук} ".....але Ви також слабо знаєте історію" .... Можна мати загальне уявлення про світову історію. Можна вивчити за книгами історію Української держави. Але в підручниках і монографіях не напишуть, що вся історія України складається зі знакових подій, які здаються на перший погляд другорядними і не допомогли весь час зберігати державність, але в кінцевому підсумку створили всі політичні підстави для існування нашої країни. Це ознака того, що протягом 1000 років все відбувалося не просто так. І нам судилася велика роль в найближчому майбутньому Європи )

\iusr{Костя Курліщук}
\textbf{Constantine Marakhov}
Повністю з вами згоден. Але 1000 років я стяну @igg{fbicon.face.woozy} 
\end{itemize} % }

\iusr{Павло Ігнатенко}

Ми не Римська імперія, розвалитись не можемо, а от заторохкотіти до складу
ерефії, цілком можливо. От тоді буде остаточний, та безповоротний пиздець, про
що і мріє мій тесть.


\iusr{Анжелика Барсукова}

Папаша, видимо, уверен, что Рим распался от того что император не справился. А
ничего что он был ребёнком и Рима вообще на момент коронации не существовало в
том смысле, в каком существует Украина?


\iusr{Pavlo Pyshkin}

Проблема в том, что за Зе голосовали как те, кто видел в нем "новое лицо",
которое сметет совок, коррупцию в т.ч. порошенковскую и приведет нас в Европу,
так голосовали и те, кто видел в нем, "своего парня", который договорится с
Путиным и перестанет украинизировать Украину. Вторые уже полностью
разочаровались, а некоторые из первых еще верят  @igg{fbicon.smile} 

\iusr{Валерій Линчук}

Переживемо, але якщо шмарклю і всіх його дружбанів повісимо на ліхтарях на Алеї
Небесної Сотні...

\begin{itemize} % {
\iusr{Наталія Віннікова}
\textbf{Валерій Линчук} алея замала
\end{itemize} % }

\iusr{Natalia Likhacheva}
Папа прав!

\iusr{Василь Люлява}
Батько вивчав історію, син розумово розвивався на показах 95кв і 'Сватів'.

\iusr{Ирина Панская}
Зато весело

\begin{itemize} % {
\iusr{Людмила Михайленко}
\textbf{Ирина Панская} Дуууже! Так весело, аж плакати хочеться!

\iusr{Ирина Панская}
\textbf{Людмила Михайленко},зеленим по фіг,в який країні жити.

\iusr{Людмила Михайленко}
\textbf{Ирина Панская} Так нам же, хто не позеленів, не по фіг! А доводиться терпіти зелений балаган.

\iusr{Ирина Панская}
\textbf{Людмила Михайленко} ,на жаль
\end{itemize} % }

\iusr{Sergey Neyolov}

Вот не пойму откуда берется ложь, что ему все мешают, а он такой великий, как
айсберг океане?! Может это скрытой оправдание, что человек проголосовал за
дурака, но признать не может и пытается, оправдаться.


\iusr{Денис Тураев}

Будьте осторожны ⚠️ Берёза ходит по ресторанам с подслушивающей аппаратурой!
Приём!  @igg{fbicon.face.tears.of.joy}{repeat=3}  Повторяю, аппаратура при нём!  @igg{fbicon.smile} 

\begin{itemize} % {
\iusr{Людмила Михайленко}
\textbf{Денис Тураев} Так слуги ж самі пропонують доносити за винагороду, то може Береза підробляє? @igg{fbicon.face.tears.of.joy}{repeat=3}  Вибачте, пане Березо, за жарт на Вашу адресу.
\end{itemize} % }

\iusr{Ірина Карасьова}

Згодна з Вами. Мені становиться страшно жити у цій країні. Сліпий не бачить, а
глухий не чує, як гине країна. Я не вірю ,що країною править коЗел, всім
заправляє беня і ко. Чмо ні в чому не може розібратися, тому діє по підказкам.
Я за нього не голосувала, бо зразу сказала, що це буде коломойський, він буде
правити та оббирати країну.

\begin{itemize} % {
\iusr{Constantine Marakhov}
\textbf{Ірина Карасьова} 

"...глухий не чує, як гине країна" .....Та це суцільне диво, що наша країна
взагалі існує)) ... Хоча і "всім заправляє Беня і ко" зрештою ця конкурентна
олігархічна квазідемократія весь цей час і є запорукою демократичних процесів.
.. РФ, Білорусь та інші (здебільшого азіатські) пострадянські країни з їх
забронзовілими авторитарними лідерами мають набагато менше шансів видертись на
шлях еволюції і розвитку )


\iusr{Ірина Карасьова}
\textbf{Constantine Marakhov} 

я здивована,як олігархи можуть підтримувати демократію. Коломойський що казав,
взяти 100мільярдів доларів і нехай Крим буде москальським, і треба повернути до
ерефії а не до западу. Знайдіть пост Дмитра Чекалкіна за 2019 рік і що він пише
про зелю. І тоді все зрозумієте куди веде країну це чмо.


\iusr{Constantine Marakhov}
\textbf{Ірина Карасьова} 

"я здивована,як олігархи можуть підтримувати демократію. " ... Це не зовсім
демократія. Я написав "квазідемократія", яка підтримує процеси. )) Все просто -
справа в тому що їх (олігархів) багато, а не один на всіх, як у сусідів .. Беня
заради грошей може бути сьогодні патріотом, завтра зрадником і він дійсно не є
державником бо йому потрібні слабкі підконтрольні держ. структури. Але він
розуміє, що якби сюди зайшла раша, то він одразу перетворився б з ляльковода на
раба пуйла )

\end{itemize} % }

\iusr{Денис Белый}
Зеленський це свідомий ворог Української Держави.

\iusr{Арсен Звенигора}
када все наипнется-даже поезда ходить не будут...

\iusr{Алла Дмитренко}
+

\iusr{Степан Свиридос}
... проклятое зебильё...

\iusr{Александр Кузьменко}
Папа в Лондоні, у цього хлопця при любому розкладі буде О’кей , а от молодь в Україні постраждає.

\iusr{Виктор Игуминов}

Увы... Это было ясно еще 2.5 года назад... Но лично у меня оставалась надежда
что ему не дадут все развалить союзники Украины... но там оказалось тоже все не
просто...

\iusr{Yevgeniy Nikitenko}

Да. Печально. Хоть научный труд пиши "как дурак может погубить целую страну".

Хотя, наверное, подобное уже давно написано. А в истории появится еще один
прецедент. Про который кто-то кому-то за столиком в кафе упомянет... может
быть.

\iusr{Александр Сгибнев}
болезнь крупных организаций и стран - набирать дураков-подхалимов вместо спорящих специалистов

\iusr{Roman Protov}
Не понятно одно: почему у яблонь получаются такие вишенки? Это реально вырождение или массовая генетическая мутация...

\iusr{Александр Аникевич}

К огромному сожалению Украина начала превращаться в ,,квазигосударство с
неработающими институциями,, намного раньше, правда более профессионально. И
чем дальше, тем быстрей.


\iusr{Nataliia Kachalova}

Я верю, что наебнётся Вова, а Украина выстоит. Слишком мелкое это ничтожество,
чтоб всё разрушить.

\begin{itemize} % {
\iusr{Oksana Svechnikova}
\textbf{Nataliia Kachalova} , "надія вмирає останньою..." І все ж таки дійсно хочеться вірити, що Україна вистоїть.
\end{itemize} % }

\iusr{Vlad Moshenskyi}
У Зеленського таке завдання: зробити так, щоб нічого не працювало. Так що він
свою роботу робить добре. На жаль.

\begin{itemize} % {
\iusr{Анна Воскобойник}
\textbf{Vlad Moshenskyi} це завдання у Єрмака. Зеля лише не дуже розумний наркоман на чолі держави, який взагалі може не розуміти стан речей - там тепла ванна та наркота, з якої він виринає доволі рідко, судячи з його поведінки.
Ха що нам все це...?(((
\end{itemize} % }

\iusr{Сергій Клачко}
В виборі між компетентними керівником і популістом наріт обере популіста - бо обіцянки солодші

\iusr{владислав полторакин}
\textbf{Dmitriy Lashch} наверное это было бы смешно.....если бы не было так грустно☹

\iusr{Виктория Мельченко}
Сліпий сліпого в яму заведе! ( Біблія).

\iusr{Валерий Бондаренко}
Чего ожидать когда страной управляет натуральный Фунт, марионетка, а по факту бородатый аферист.

\iusr{Олег Здесенков}
Масштабность зеленского - надо же такое придумать !

\iusr{Амет Крымский}
"Мудрый народ" именно этого и хотел.

\iusr{Сергей Неганов}
Ну слабых глав государств не смещают они выгодны правда до определенной степени.
Термидорианский заговор явно не грозит Зеленскому.
Военный тоже.
Попытки смещения путём Майдана тоже.
Акции протеста это из серии; поорали и разошлись».

\iusr{Сергей Неганов}
Масштаб Зеленского?
Ну разве что как модель машины скорее а не сама машина 1:12 .

\iusr{Александр Плаха}
а що в нас колись працювали інституції? не смішіть закони понаприймають а виконують тількі ті що владі потрібні.

\iusr{Dex Labs}
К сожалению, ни дедушке с Лондона, ни парню в армани нету и не будет никаких дел к реальной ситуации в Украине. У них другой мир.

\iusr{Andrew Bondar}
Коротко.)

\ifcmt
  ig https://scontent-lga3-1.xx.fbcdn.net/v/t39.30808-6/242634027_4252582348122694_932315023718169876_n.jpg?_nc_cat=103&_nc_rgb565=1&ccb=1-5&_nc_sid=dbeb18&_nc_ohc=ICcx3qx1U7gAX9NIJ_z&_nc_ht=scontent-lga3-1.xx&oh=a7ef4d62b6b3b52b0c3e923fc17ce2d1&oe=615132EF
  @width 0.3
\fi

\begin{itemize} % {
\iusr{Людмила Гусак}
\textbf{Andrew Bondar} Зізнаюся, я завжди думала, шо він якийсь дурачок.
\end{itemize} % }

\iusr{Геннадий Великохатский}
«император Рима в наши дни – это любой, кого послушают легионы.»
В. Пелевин

\iusr{Александр Ковалев}
Люди не состоянии понять масштабность Зеленского? ...

\iusr{Luydmila Bogomiahkova}
да, вариант, когда каждая кухарка может управлять государством реально не работает...

\iusr{Ігор Мазур}

На цьому фоні, виходять назад і питаються дорвалися до влади таки покидьки та
зрадники як Добкін, Мураєв, Лесик. Їх біг-бордами увішан весь Харків. Вони
представляють себе як борці за національні інтереси. Це капець якийсь.

\iusr{Константин Лупал}

Давно думал, что ситуация у нас в стране, что то мне напоминает. Специально
перепроверил свою память. Спросил Гугл. И таки да! В 1983 году на экраны СССР
вышел фильм "Дублёр начинает действовать." Речь идёт об эксперименте,когда на
одном из заводов, молодые, образованные и очень САМОУВЕРЕННЫЕ В СЕБЕ люди,
полностью заменили все руководство.Да, у них было образование и амбиции. Но
полностью отсутствовал жизненный и профессиональный опыт. И что из этого вышло,
можно легко догадаться. Тем кто не видел, очень рекомендую посмотреть этот
фильм. А кто смотрел, можно обновить воспоминания


\iusr{Юлия Манастырная}
Не запас прочности больше, а народ пассивнее. Римлян было проще раскочегарить.

\iusr{Валерий Поддубенко}
Гарна аналогія.

\iusr{Yuriy Tarasovskyy}
Тяжело папе, видимо, с таким сыном.  @igg{fbicon.smile} 
Великодушно философский подход  @igg{fbicon.smile} 

\iusr{Gleb Onski}
Ви перебільшуєте. Зараз трохи не так все рухне. Зараз не той час коли валяться країни і будуються нові імперії.

\iusr{Maksym Bobrov}
 @igg{fbicon.thumb.up.yellow} 

\iusr{Виталий Качуровский}

Ну, если Зе уже сравнили с последним императором римской империи, то видимо
сейчас здесь должен оставить комментарий ПАП : "Если бы вы видели, какую
капусту я вчера купил у бабули на каневской трассе, вы бы не просили меня
вернуться."

\end{itemize} % }
