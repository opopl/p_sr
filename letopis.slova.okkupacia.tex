% vim: keymap=russian-jcukenwin
%%beginhead 
 
%%file slova.okkupacia
%%parent slova
 
%%url 
 
%%author 
%%author_id 
%%author_url 
 
%%tags 
%%title 
 
%%endhead 
\chapter{Оккупация}
\label{sec:slova.okkupacia}

%%%cit
%%%cit_head
%%%cit_pic
%%%cit_text
В \emph{оккупированных} городах Украины издавались газеты на украинском языке,
а помощникам нацистов из местных разрешали забирать имущество своих жертв -
например, евреев. Также они получали льготы по ведению мелкого бизнеса вроде
торговли (часто это были \enquote{отжатые} у тех же евреев или поляков магазины или
хозяйства).  Но любые претензии на политическое самоуправление давились на
корню. И в этом смысле \enquote{украинофильские} воззрения Альфреда Розенберга так и
остались теорией. Хотя, разумеется, никаким украинофилом он не был. Основой его
теории по славянскому вопросу был раскол среди ключевых народов СССР. И главным
\enquote{раскольником} Розенберг видел Украину 
%%%cit_comment
%%%cit_title
  \citTitle{22 июня - 80 лет нападения на СССР. Что немцы готовили для украинцев}, Максим Минин, strana.ua, 22.06.2021
%%%endcit
