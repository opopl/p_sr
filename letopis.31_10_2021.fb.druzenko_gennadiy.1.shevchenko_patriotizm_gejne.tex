% vim: keymap=russian-jcukenwin
%%beginhead 
 
%%file 31_10_2021.fb.druzenko_gennadiy.1.shevchenko_patriotizm_gejne
%%parent 31_10_2021
 
%%url https://www.facebook.com/gennadiy.druzenko/posts/10158588796088412
 
%%author_id druzenko_gennadiy
%%date 
 
%%tags gejne_genrih.poet.germania,patriotizm,shevchenko_taras,ukraina
%%title ВЧІТЬСЯ, БРАТИ МОЇ, ДУМАЙТЕ, ЧИТАЙТЕ...
 
%%endhead 
 
\subsection{ВЧІТЬСЯ, БРАТИ МОЇ, ДУМАЙТЕ, ЧИТАЙТЕ...}
\label{sec:31_10_2021.fb.druzenko_gennadiy.1.shevchenko_patriotizm_gejne}
 
\Purl{https://www.facebook.com/gennadiy.druzenko/posts/10158588796088412}
\ifcmt
 author_begin
   author_id druzenko_gennadiy
 author_end
\fi

ВЧІТЬСЯ, БРАТИ МОЇ, ДУМАЙТЕ, ЧИТАЙТЕ...

Ця класична цитата з Тараса Шевченка якось сама собою спала на думку після
прочитання "Після війни" Тоні Джадта. Я цілковито погоджуюсь з Volodymyr
Yermolenko, який колись сказав: "Якщо у вас є можливість прочитати тільки одну
книжку про повоєнну Європу, прочитайте "Postwar: A History of Europe Since
1945". 

\ifcmt
  ig https://scontent-lhr8-1.xx.fbcdn.net/v/t39.30808-6/250257768_10158588769773412_8608336583632147524_n.jpg?_nc_cat=110&ccb=1-5&_nc_sid=730e14&_nc_ohc=ICDkFjtDn9kAX-NnyQj&_nc_ht=scontent-lhr8-1.xx&oh=3c5cc5146e8d8afc17693820863f173c&oe=6184F0C2
  @width 0.4
  %@wrap \parpic[r]
  @wrap \InsertBoxR{0}
\fi

Насмеперед цей епос про складність та розмаїтість життя. Про відсутність
простих відповідей на складні питання та багатовимірні виклики. Про те, що
багато викликів, перед якими постає сучасна Україна, аж ніяк не унікальні:
європейські держави не так давно експерементували з вільним ринком,
приватизацією, націоналізацією та шукали modus vivendi співіснування з
ментально чужими, але потужними Росією та Америкою. 

Одне слово, аби не винаходити давно винайдений велосипед і не танцювати на
граблях, на яких вже понабивали ґулі інші, українцям варто вивчати історію,
особливо так талановито та масштабно написану, як епічне історичне полотно Тоні
Джадта.

Поки читав, я напевно зробив сотні нотаток на берегах своєї електронної книжки.
Дозволю собі тут нагадати (словами Тоні Джадта) урок, який історія дає
українцям, одержимим боротьбою з корупцією, в якій вони бачать корінь усіх зол:

«В Італії, де по завершенні війни християнські демократи завжди мали зручні та
вигідні відносини з банкірами, бізнесменами, підрядниками, міськими патронами,
державними службовцями та, як часто подейкували, мафією, нове покоління молодих
суддів відважно кинуло виклик десятиліттям, коли громадськість була змушена
мовчати. Іронія полягала в тому, що насамперед це підкосило соціалістів, яких
потягнув донизу скандал tangentopoli («міста хабарів»), після якого почалися
розслідування їхнього управління містом Мілан. Партія втратила репутацію, а її
лідер, колишній прем’єр-міністр Беттіно Краксі, був змушений втекти в еміграцію
по той бік Середземного моря — до Тунісу.

Але справи соціалістів були нерозривно пов’язані зі справами християнських
демократів, їхнього давнього коаліційного партнера. Обидві партії ще більше
дискредитувала подальша хвиля арештів та обвинувачень, а їхній занепад
спричинив ціле плетиво політичних домовленостей та угод, які формували
італійську політику впродовж двох поколінь. На виборах 1994 року всі провідні
політичні партії країни, за винятком колишніх комуністів і колишніх фашистів,
були, по суті, усунені з політичної арени; єдиним, хто стабільно виграв від
цього політичного землетрусу, був колишній лаунж-­співак, одіозний медіа-магнат
Сильвіо Берлусконі, який увійшов у політику не так щоб продовжити
загальнонаціональне очищення, як щоб переконатися, що на його власних
бізнес-­оборудках це не позначиться негативно.»

Але найбільше мені "зайшла" цитата з Гайне, яку автор наводить наприкінці
книжки:

«Сто сімдесят років тому, на світанку доби націоналізму, німецький поет Генріх
Гайне виявив красномовну розбіжність між двома типами колективних почуттів:

"Нам [німцям], — писав він, — наказали бути патріотами, і ми стали патріотами,
тому що ми робимо все, до чого змушують нас наші правила. Однак не варто
думати, що це такий самий патріотизм, як і те почуття, що так само називається
тут, у Франції. 

Патріотизм француза означає, що в нього зігрівається серце, і завдяки цьому
теплу воно розтягується та збільшується, тож його любові вистачає не лише на
найближчого родича, а й на всю Францію, увесь цивілізований світ. 

Патріотизм німця означає, що його серце скорочується і стискається, як шкіра на
холоді, і тоді німець починає ненавидіти все іноземне, не хоче більше бути
громадянином світу, європейцем, а лише провінційним німцем".»

Дуже хочеться, аби обираючи між "(ранньо)німецьким" та
"(ідеалізовано)французьким" типом націоналізму, українці таки обрали другий.
Націоналізм, в якому любов до свого відкриває двері у ширший світ, а не замикає
їх на великий замок ресентименту, ненависті та підозрілості до чужого.

Дякую, Наш Формат за черговий шедевр українською!

Olena Prikhno
