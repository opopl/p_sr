% vim: keymap=russian-jcukenwin
%%beginhead 
 
%%file slova.angel
%%parent slova
 
%%url 
 
%%author 
%%author_id 
%%author_url 
 
%%tags 
%%title 
 
%%endhead 
\chapter{Ангел}
\label{sec:slova.angel}

%%%cit
%%%cit_head
%%%cit_pic
%%%cit_text
Є інша сторона такого примусу. Як писав Григорій Померанц: "Диявол починається
з піни на губах \emph{ангела}, який вступив у бій за праве діло. Усе
перетреться на порох — і люди, й системи. Та вічним є дух ненависті у боротьбі
за праве діло. І завдяки йому, зло на Землі не має кінця".  \emph{Ангели} в
боротьбі за праве діло – за державну мову – на жаль, природньо вмикають
ненависть до своїх опонентів та відчувають насолоду від цієї ненависті.  І нема
на це ради – борючись проти чогось, ми завжди вмикаємо ненависть проти цього, і
нам це подобається, оскільки ненависть дає енергію та відчуття сенсу!  Це саме
те зло, про яке писав Померанц.  Як наслідок маємо роз’єднання людей, розпад
суспільства
%%%cit_comment
%%%cit_title
\citTitle{Чи можна примусити збірну України розмовляти державною мовою?}, 
Євген Лапін, www.pravda.com.ua, 06.07.2021
%%%endcit
