% vim: keymap=russian-jcukenwin
%%beginhead 
 
%%file 02_04_2021.fb.fb_group.story_kiev_ua.1.kreschatik.cmt
%%parent 02_04_2021.fb.fb_group.story_kiev_ua.1.kreschatik
 
%%url 
 
%%author_id 
%%date 
 
%%tags 
%%title 
 
%%endhead 
\subsubsection{Коментарі}

\begin{itemize} % {
\iusr{Vladimir Glikin}

\ifcmt
  ig https://i2.paste.pics/c8614490ab23e57428977ecf0aa332b9.png
  @width 0.2
\fi

\iusr{Игорь Бойко}
Музыка кошмарная

\begin{itemize} % {
\iusr{Мила Подлесная}
\textbf{Игорь Бойко} Отчего же? Типичная для того времени.

\iusr{Игорь Бойко}
\textbf{Мила Подлесная} Для того времени много чего было типично. Это не меняет ее кошмарности.

\iusr{Виктор Петровский}
\textbf{Игорь Бойко} сейчас музыка для души? Что аж срать хочетса извините за грубое слово. Просто душа сейчас поёт и радуетса.

\begin{itemize} % {
\iusr{Игорь Бойко}
\textbf{Виктор Петровский} Я что-то говорил про сейчас? У автора был выбор. Автор выбрал хрень.

\iusr{Виктор Петровский}
\textbf{Игорь Бойко} я понял, хрени сейчас помоему выше крышы

\iusr{Игорь Бойко}
\textbf{Виктор Петровский} Говорите о современной хрени в другом месте, я тут при чем?

\iusr{Елена Червоная}
\textbf{Виктор Петровский} хорошего воспитания Вам музыка не дала(

\iusr{Виктор Петровский}
\textbf{Игорь Бойко} я что к вам с претензиями? Мы наверно не так поняли друг друга. Извините если что.

\iusr{Игорь Бойко}
\textbf{Виктор Петровский} Всего доброго, тоже не имел намерения вас задеть.

\iusr{Виктор Петровский}
\textbf{Елена Червоная} За грубость я извинился.

\iusr{Елена Червоная}
\textbf{Виктор Петровский} ну и хорошо. @igg{fbicon.smile}  @igg{fbicon.hot.beverage} 
\end{itemize} % }

\end{itemize} % }

\iusr{Ната Бузань}

\ifcmt
  ig https://i2.paste.pics/3ea38b7a28ccdc0ed454a0787afa9227.png
  @width 0.2
\fi

\iusr{Мила Подлесная}

Глядя на такие видео, осознаешь, что прожил не одну, а несколько жизней, причем
очень разных.

\begin{itemize} % {
\iusr{наталья колесник}
\textbf{Мила Подлесная} а сколько еще проживем @igg{fbicon.face.grinning.smiling.eyes}  каждый день, как на войне

\iusr{Вася Лейкин}
\textbf{наталья колесник} Теперь и мы можем представить, что творилось в сердцах людей в далёком 1917....

\ifcmt
  ig https://scontent-frx5-2.xx.fbcdn.net/v/t39.1997-6/s168x128/10333122_298592793654253_1393149385_n.png?_nc_cat=1&ccb=1-5&_nc_sid=ac3552&_nc_ohc=mveKXQAb0-EAX9AMwCW&_nc_oc=AQn-l89ij5ICcVqBORP4agU_gNQDcP9wxSAwcAYRkoTSjHFJh6rGL0LHYnWJD4h6Ij4&tn=lCYVFeHcTIAFcAzi&_nc_ht=scontent-frx5-2.xx&oh=00_AT8IQpm553Bc_ETdrztVv9k-OYk8O9ziLln9cQ2UPuRqgQ&oe=61BCE147
  @width 0.1
\fi

\end{itemize} % }

\iusr{Алла Поповская}
Замирает сердце,как хочется вернутся...... @igg{fbicon.heart.red}{repeat=2}

\iusr{Марина Лабунец}
Каштаны!!!!!!!!!!!!

\iusr{Татьяна Задорожная}

Ого немногочисленных, людей дуже багато, немногочисленно авто це да, и ще
невелик1 каштан, бо коли 82 роц1 це вже велетн1

\begin{itemize} % {
\iusr{Леонид Красиловский}
\textbf{Татьяна Задорожная} зачекайте ще пару років і по Хрещатику тільки собаки бездомні будуть бігати((((
\end{itemize} % }

\iusr{Надежда Баторевич}

\ifcmt
  ig https://i2.paste.pics/9b53d25f3d4f3631c17230458a0210ad.png
  @width 0.2
\fi

\iusr{Анатолий Лернер}
Да, это был ещё Киев...

\iusr{Наталья Ершова}
Да это был Киев наш!!! Киевляне меня поймут!!!

\iusr{Людмила Кралина}
Грустно.

\iusr{Полина Микульская}
Ностальгия

\iusr{Юрий Панчук}

У меня тоже есть раннее воспоминание, как мы с бабушкой переходили Проспект
Победы поверху и не по переходу примерно напротив ЗАГСа. Помню один грузовичок
ГАЗ 51 даже остановился, хотя мы могли подождать и бабушка сказала \enquote{Вот хороший
водитель}. Тогда было по две полосы в каждую сторону и подземных переходов еще
не построили, ходил трамвай.


\iusr{Катя Теркина}
Какая зелёная улица

\iusr{Надежда Василенко}
Это город, который имеет лицо. А не лоскутное одеяло, утопленное в бетоне

\iusr{Леонид Красиловский}
А когда мне было два годика то по Крещатика конки запряженные бегали.....)))))
и Сталину ещё год жизни был отведён ....

\iusr{Людмила Иванова}
А музыка из Золотой мины!

\iusr{Людмила Иванова}
Еще были каштаны на Кресте!просто тогда не было Кличко с елками

\begin{itemize} % {
\iusr{Светлана Сергеева}
\textbf{Людмила Иванова} Ели в Киеве тоже были помимо каштанов. Голубые ели тоже были прекрасным украшением города.
\end{itemize} % }

\iusr{Тамара Мельничук}
Це прекрасно але в той час транспорту було на процентів 90 менше

\begin{itemize} % {
\iusr{Ольга Поканевич}
\textbf{Тамара Мельничук} Ага, зігнали з усього Києва. І людей теж.  @igg{fbicon.face.woozy} 

\iusr{Виктор Петровский}
\textbf{Тамара Мельничук} потому что работа была у людей на местах. Извините за грубое слово, какого хрена человеку ехать в Киев, мучитса, снимать квартиру, койко место. Всю страну уничтожили, ветер гуляет по полям и селам. И что делать людям? Все в Киев тут хоть не здохнешь. Так и живём 30 лет, а что дальше?
\end{itemize} % }

\end{itemize} % }
