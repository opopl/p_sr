% vim: keymap=russian-jcukenwin
%%beginhead 
 
%%file 06_01_2022.fb.fb_group.story_kiev_ua.1.vladimirskij_sobor.cmt
%%parent 06_01_2022.fb.fb_group.story_kiev_ua.1.vladimirskij_sobor
 
%%url 
 
%%author_id 
%%date 
 
%%tags 
%%title 
 
%%endhead 
\zzSecCmt

\begin{itemize} % {
\iusr{Мария Самбур}
Впечатлило.

\iusr{Милана Пономаренко}

\ifcmt
  ig https://scontent-frt3-1.xx.fbcdn.net/v/t39.30808-6/271307405_10225615677601811_6163033601611553590_n.jpg?_nc_cat=106&ccb=1-5&_nc_sid=dbeb18&_nc_ohc=1jGUyo0CvG8AX_lx41h&_nc_ht=scontent-frt3-1.xx&oh=00_AT-pJhxeXjfA1pReAZhLGxo6KLgXK0tbN4LSPOKJgDlTOQ&oe=61DE3D69
  @width 0.3
\fi

\iusr{Александра Цымбал}

Этот стих очень красиво декламирует актриса в 1 или 2 серии сериала про Мишку
Япончика) прям пересматривала этот эпизод несколько раз. Трогательный очень.

\begin{itemize} % {
\iusr{Оксана Денисова}
\textbf{Александра Цымбал} Не смотрела, но верю и посмотрю, спасибо!

\iusr{Милана Пономаренко}
\textbf{Александра Цымбал}

\ifcmt
  ig https://scontent-frt3-1.xx.fbcdn.net/v/t39.30808-6/271138010_10225615698042322_2405064161922108616_n.jpg?_nc_cat=107&ccb=1-5&_nc_sid=dbeb18&_nc_ohc=Hg-tFSf4zLQAX9gWQgK&_nc_ht=scontent-frt3-1.xx&oh=00_AT9GpNcZlgsOZzCw6V-1Z2m9ErYaHx6Rt4eWXWL5mvc92w&oe=61DCC3F7
  @width 0.3
\fi

\iusr{Александра Цымбал}
\textbf{Милана Пономаренко} с Рождеством! Здоровья и Счастья!

\iusr{محمد الربيعى}
\textbf{Александра Цымбал} @igg{fbicon.heart.red}

\end{itemize} % }

\iusr{Марина Острополец}
Спасибо @igg{fbicon.heart.exclamation}
Срезонировало...

\iusr{Ирина Химуля}
Меня крестили во Владимирском Соборе.

\begin{itemize} % {
\iusr{Nataliya Andrienko}
\textbf{Ирина Химуля} меня тоже

\iusr{Галина Кошмак}
Улюблений храм моєї доньки, неймовірна енергетика. Вітаю всіх зі святом!


\ifcmt
  tab_begin cols=3,no_fig,resizebox=0.5

     pic https://i2.paste.pics/00164444626fd1590e11c4f28781e184.png
     pic https://i2.paste.pics/c993b145e501368c3dcf08db70ba0568.png
     pic https://i2.paste.pics/63f6aaae27f3186e1c7c683436d123eb.png

  tab_end
\fi

\end{itemize} % }

\iusr{Таня Сидорова}
І моє улюблене.

\iusr{Анна Земко}

Собор, який привів мене до Бога)! Дякую...

\begin{itemize} % {
\iusr{Irina Trisyachna}
\textbf{Анна Земко} І мене Аню, я після інституту часто туди заходила.

\iusr{Анна Земко}
\textbf{Irina Trisyachna} Так само, Іро... Якраз в інституті, а ще у мене бабушка поряд жила - на Чапаєва (це біля Франка), тепер там сестри живуть...
\end{itemize} % }

\iusr{Mykhailo Nykonchuk}

\obeycr
Застиляйте гарно столи,
Та оселі прибирайте.
Ангел вилетить з під неба,
Промовить радісну новину.
\smallskip
Що народиться дитина - Месія.
У Йосипа і Марії.
Величать будуть Ісус,
Прославляти не забудь!
\smallskip
Прилине звістка в кожний дім,
Радість безмежна буде в нім.
Люди святкувати стануть,
День народження Ісуса.
\smallskip
І співати всім колядки,
Що народилося дитятко.
В Віфліємі в печері- стайні,
Прийде в наш світ маленький Ісус.
\smallskip
Різдво - родине свято,
Де вся сім'я збереться за столом.
На ньому страв буде багато,
І в кожного, в душі лише добро.
\smallskip
Так буває тільки на Різдво.
Бажаю весело і багато,
Відсвяткувати Різдвяне свято.
\smallskip
Автор: Михайло Никончук
\restorecr


\iusr{Татьяна Цыбульская}
Спасибо, за стихотворение это чудо.

\iusr{Александр Харченко}
Это работа Врубеля!

\begin{itemize} % {
\iusr{Александра Цымбал}
\textbf{Александр Харченко} а разве не в Кирилловской Врубеля?

\iusr{Александр Харченко}
\textbf{Александра Цымбал} и в Кириловской! Дело в том, что очень высокое сходство! У Васнецова не тот стиль!

\iusr{Светлана Кузнецова}
\textbf{Александра Цымбал} Врубель писал ее с Праховой. После чего отношения усложнились. И это таки работа Врубеля. И Кириловскую тоже он расписывал!

\iusr{Оксана Денисова}
\textbf{Александр Харченко} Нет, это Виктор Васнецов! Врубель почти не расписывал Владимирский собор.

\iusr{Helen Ferrum}
\textbf{Оксана Денисова} И Васнецов написал богоматерь с Лели Праховой  @igg{fbicon.face.flushed} 
Он, видимо, тоже был в нее влюблен, как и Врубель.
Вот только история об этом почему то умолчала  @igg{fbicon.face.grinning.sweat} 

\iusr{Оксана Денисова}
\textbf{Helen Ferrum} Нет, Васнецов писал Богоматерь и младенца со своей жены и сына.

\iusr{Helen Ferrum}
\textbf{Александр Харченко}
Только собралась написать, @igg{fbicon.smile} 
Странно  @igg{fbicon.thinking.face} 
Мне всегда казалось, что Врубель настолько узнаваем  @igg{fbicon.face.flushed} 
Но, возможно, только
для тех, кто считает еоо гением, @igg{fbicon.smile} 

\iusr{Татьяна Гордиенко}
\textbf{Александр Харченко}, нет, автор - Виктор Васнецов. Писал со своих жены и сына

\iusr{Александр Харченко}
\textbf{Татьяна Гордиенко} именно! Я вас тролил по тихому!

\iusr{Татьяна Гордиенко}
\textbf{Александр Харченко}, тю

\iusr{Александр Харченко}
\textbf{Татьяна Гордиенко} извините!

\end{itemize} % }

\iusr{Вадим Горбов}

Этой росписи посвящён яркий рассказ Куприна «Погибшая сила».

\iusr{Оксана Денисова}
\textbf{Вадим Горбов} Да, чудесный рассказ!!!

\iusr{Irina Burmistrova Baulina}
 @igg{fbicon.hands.pray} 

\iusr{Валентина Зинчук}
Мой любимый собор!

\iusr{Alla Stehachova}
Меня и мою дочь в нем окрестили.

\iusr{Татьяна Николаева}
\textbf{Alla Stehachova} и меня

\iusr{Ludmila Kurilova}
Христос народився!!! Славімо Його!!!

\iusr{Валентина Белозуб}
Оксана, большое спасибо за поэтическую страничку и прекрасное фото моей любимой картины..

\iusr{Оксана Денисова}
\textbf{Валентина Белозуб} Спасибо Вам!

\iusr{Liudmila Kabanova}
Мой любимый и судьбоносный храм! Вчера была в нем. Энергетика в нем особенная.

\iusr{Оксана Денисова}
\textbf{Liudmila Kabanova} Согласна, энергетика в нем особенная!

\iusr{Наталия Привалко}

\ifcmt
  ig https://i2.paste.pics/6c50f1163761c0d57cbe92ec9666d342.png
  @width 0.3
\fi

\iusr{Регина Кучеренко}
А я после прочтения стиха и созерцания фото храма расплакалась. Спасибо.

\iusr{Оксана Денисова}
\textbf{Регина Кучеренко} Спасибо Вам за Ваши эмоции @igg{fbicon.heart.red}

\iusr{Оксана Бабич}

Согласна. В Собор нереально не влюбиться. Мне ещё там нравится \enquote{Воскрешение
Лазаря}. Если долго и пристально смотреть, кажется, что картина-живая.

\begin{itemize} % {
\iusr{Оксана Бабич}
\textbf{Оксана Денисова} ,я - человек нерелигиозный, но в Соборе испытываю особое наслаждение

\iusr{Оксана Денисова}
\textbf{Оксана Бабич} Я тоже!

\iusr{Оксана Бабич}
\textbf{Оксана Денисова}, а за Блока особенное Благодарю. Один из любимейших стихов

\iusr{Оксана Денисова}
\textbf{Оксана Бабич} И я его с детства очень люблю!

\iusr{Ольга Морозова}
\textbf{Оксана Бабич} Как знать, может вы более религиозная, чем те, которые себя так называют.
\end{itemize} % }

\iusr{Юлия Романюк}

Я была уверена что это работа Врубеля разве не они с Катарбинским занимались
реставрацион. Работами Собора?!

\begin{itemize} % {
\iusr{Оксана Денисова}
\textbf{Юлия Романюк} 

Вы перепутали с Кирилловской церковью, там была реставрация. А Владимирский
собор просто расписывали, в том числе и Врубель, и Котарбинский. Но это - работа
Васнецова.

\begin{itemize} % {
\iusr{Юлия Романюк}
\textbf{Оксана Денисова} спасибо большое, да, я действительно перепутала !:)

\iusr{Юлия Романюк}
Благодарю!

\iusr{Ирина Гутт}
\textbf{Оксана Денисова} 

Врубелевская работа, если я правильно помню, на входе в собор. Врубель был
нездоров. Лечился в лечебницах и по доброй воле, когда чувствовал приближение
приступа (психическое заболевание). Умер в больнице от крупозного воспаления
лёгких - долго стоял в морозный день перед открытой форточкой.

\end{itemize} % }

\iusr{Тамара Романова}
\textbf{Julia Romanjuk} 

Врубель, действительно, должен был расписывать Владимирский собор, но что-то
произошло, уже не помню, и был приглашен Васнецов с группой художников.

\begin{itemize} % {
\iusr{Юлия Романюк}
\textbf{Тамара Романова} спасибо!

\iusr{Татьяна Бурлачук}
\textbf{Тамара Романова} если мне память не изменяет, Васнецов расписал Владимирский собор по эскизам Врубеля ))
А Дева Мария списана с жены Врубеля (возможно, перепутала))

\iusr{Владислав Киселев}
\textbf{Татьяна Бурлачук} знаменитый Врубелевский иконостас находится в Кирилловской церкви.

\iusr{Владислав Киселев}
\textbf{Татьяна Бурлачук} Если говорить о жене Врубеля - она была актриса и сценические наряды ей придумывал и рисовал сам Врубель.
\end{itemize} % }

\iusr{Надежда Залескова}
\textbf{Юлия Романюк} в одном вы правы, глаза
Очень похожи на врубелевскую Царевну - лебедь)))

\iusr{Владислав Киселев}
\textbf{Юлия Романюк} Врубель во Владимирском соборе написал орнаменты на порталах.

\iusr{Sergyi Grabar}

Так, Врубель виконав тільки орнаменти. Його ескізи до розписів Володимирського
собору знаходяться в Київській картинній галереї (бувший Музей російського
мистецтва).


\iusr{Юлия Романюк}
\textbf{Sergyi Grabar} Спасибо!

\end{itemize} % }

\iusr{Владимир Новицкий}

Спасибо Оксаночка! С Рождеством!!! Вы себе даже представить не можете как ине
хочеться снова побывать в своём любимом Владимирском Соборе. За стихи
отдельгное спасибо.

\iusr{Оксана Денисова}
\textbf{Владимир Новицкий} Спасибо Вам и с Рождеством!!!

\iusr{Andrej Anfalow}
Энергия... Энергетика - это отрасль промышленности...

\iusr{Тома Храповицкая}

Истинно... всегда возникают в Памяти ассоциации, глубокие и дополняющие... С
Рождеством Христовым Вас Оксана и Вашу семью! Радуйтесь красоте, и Нас радуйте!
Вы чувствуете...

\iusr{Оксана Денисова}
\textbf{Тома Храповицкая} Спасибо большое и Вас с Рождеством!!!

\iusr{Марина Тычинская Колодий}
Спасибо, навеяли чудесное настроение!

\iusr{Оксана Денисова}
\textbf{Марина Тычинская Колодий} Я рада!

\iusr{Людмила Скомаровська}
Проникновенный
Образ. Можно долго смотреть не отрываясь******

\iusr{Александр Крвавницкий}
Замечательно! — картина и стихотворение !

\iusr{Богдана Шаповалова}
Це неймовірний собор, мабуть другий після Софії Київської. Обожнюю їх обох  @igg{fbicon.hearts.two}  @igg{fbicon.hearts.revolving}  @igg{fbicon.hearts.two} 

\iusr{Ludmila Kool}
Меня тоже крестили.

\iusr{Ольга Кириченко}

Огроменное спасибо за прекрасную публикацию.

Мой любимый собор всю жизнь: в студенчестве не пускали ( а нам так хотелось!),
а потом когда стало возможно, всякий раз посещала этот собор. Самые прекрасные
воспоминания утренних посещений. Поезд приходил рано и вот я пешком шла от
вокзала до собора. Иногда он был еще закрыт, открывался ровно в 7-00. И вот я
одна в соборе - это невероятное состояние таинства и святости. Спасибо, что мои
\enquote{министерские} работали с 9-00 и еще не очень любили, когда посетители
раньше приходят.

Так что можно было сполна насладиться посещением. Так было почти всегда, когда
приезжала поездом. Это незабываемо!

Спасибо за прекрасные воспоминания!

\iusr{Оксана Денисова}
\textbf{Ольга Кириченко} Спасибо Вам за Ваши чудесные воспоминания!

\iusr{Олег Лях}

Дуже гарний собор, завжди любив бувати тут, як був студентом, але відтоді як
розкольники його привласнили, на жаль не можу туди зайти помолитись.


\iusr{Виктория Логинова}

Хоть и Владимирский собор не является моим любимым, но то что там бесподобно
красиво - не поспоришь. Иконы - великолепные, архитектура храма - прекрасная.
Но вот чего-то для души не хватает. Я о себе. Спасибо, Оксана, за чудесную
публикацию - прекрасная икона и потрясающие стихи!

\begin{itemize} % {
\iusr{Оксана Денисова}
\textbf{Виктория Логинова} Спасибо Вам!

\iusr{Виктория Логинова}
\textbf{Оксана Денисова}

\ifcmt
  ig https://scontent-frx5-1.xx.fbcdn.net/v/t39.30808-6/271549510_3148564705426124_12418555872429437_n.jpg?_nc_cat=110&ccb=1-5&_nc_sid=dbeb18&_nc_ohc=GstypW3uC84AX-aCCsk&_nc_ht=scontent-frx5-1.xx&oh=00_AT_Yug0W8gPg3h6kJjefSVH5nLuAwL89PyNVy6SyBwIdhA&oe=61DD1A9A
  @width 0.3
\fi

\end{itemize} % }

\iusr{Iryna Naidonova}
Благодарю!

\iusr{Татьяна Бурлачук}

Я захожу во Владимирский, чтобы полюбоваться росписями )) провожу много времени возле каждой. Очень красиво.
В Софиевской Оранта превосходная из мозаики )) Там и старые росписи частично открыты ))

\iusr{Елена Рогозина}
Спасибо, Оксана! Так отозвалось... Я живу в Швеции уже полгода, но так скучаю
за родным Киевом!

\begin{itemize} % {
\iusr{Оксана Денисова}
\textbf{Елена Рогозина} Спасибо Вам, рада, что понравилось!

\iusr{Елена Рогозина}
\textbf{Оксана Денисова} мои поздравления с наступающим Рождеством!!

\iusr{Оксана Денисова}
\textbf{Елена Рогозина} Спасибо! И Вас с Рождеством!

\iusr{Елена Рогозина}
\textbf{Оксана Денисова} спасибо!

\iusr{Natalia Bastun}
\textbf{Елена Рогозина} ya zivy 21 god v Amerike, a tak ze skychayu po Kievy!!! Eto ne prohodit....

\iusr{Татьяна Иванова}
\textbf{Елена Рогозина} полгода..., а полжизни в другом мире, то как???

\iusr{Елена Рогозина}
\textbf{Татьяна Иванова} девчонки, мы сами сделали выбор. Только я свой сделала почти в 60 лет! В этом возрасте отрываться от корней очень непросто...

\iusr{Татьяна Иванова}
\textbf{Елена Рогозина} а я в 40+...

\iusr{Galyna Bartko}
\textbf{Елена Рогозина} Всё проходит со временем
И это пройдет, с Рождеством христовым Вас

\iusr{Елена Рогозина}
\textbf{Galyna Bartko} спасибо, взаимно!
\end{itemize} % }

\iusr{Larisa Olesyuk}

А я люблю в соборе Богоматерь Михаила Врубеля. Многие его эскизы для росписи
были отвергнуты, а эта работа, я считаю, достойно вписалась в ансамбль.

\ifcmt
  ig https://scontent-frt3-2.xx.fbcdn.net/v/t39.30808-6/271440977_4734780863302875_5475854373799412141_n.jpg?_nc_cat=103&ccb=1-5&_nc_sid=dbeb18&_nc_ohc=RMZi7grQ1BAAX9DqSXm&_nc_ht=scontent-frt3-2.xx&oh=00_AT9Tb4P6o6UI0aU8wZhP6AKxziVpXiUxfWKrHSO1TRrqDg&oe=61DDF2BB
  @width 0.2
\fi

\end{itemize} % }
