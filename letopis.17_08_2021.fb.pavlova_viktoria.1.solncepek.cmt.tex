% vim: keymap=russian-jcukenwin
%%beginhead 
 
%%file 17_08_2021.fb.pavlova_viktoria.1.solncepek.cmt
%%parent 17_08_2021.fb.pavlova_viktoria.1.solncepek
 
%%url 
 
%%author_id 
%%date 
 
%%tags 
%%title 
 
%%endhead 
\subsubsection{Коментарі}

\begin{itemize} % {
\iusr{Виктория Андриечко}
Викуля, полностью с тобой согласна.

% -------------------------------------
\ii{fbauth.zhemchugova_akulina.doneck.dnr}
% -------------------------------------

Думаю, это потому, что показали лишь начало войны. И ты права, трудно актерам
передать ужас, если они этого в глаза не видели. Мне тяжело было смотреть.

\begin{itemize} % {
\iusr{Виктория Павлова}
\textbf{Акулина Жемчугова} 

Аня, ну ты же знаешь, что у меня специфический вкус на фильмы, театральные
постановки, литературу. Я не хочу тебя обидеть ни в коем случае, но для меня
это не кино о войне. Это набор жестоких и кровавых сцен, надерганных отовсюду и
соединенных воедино. Я не поняла режиссерского замысла. Если бы главными
героями этой картины были врачи скорой, если бы на них был построен сюжет и их
глазами показаны события тех дней, судьбы людей, тогда другое дело. Я этого не
увидела в фильме. Зато увидела и услышала много ненужного - того, что
обесценило сюжет. При всем моем уважении к Кравченко, в этой роли он слаб. О
водителе скорой вообще промолчу - просто безэмоциональная игра актера. Он даже
не понимает, кого он играет. Но это повторюсь, мое личное мнение.


\iusr{Акулина Жемчугова}
\textbf{Виктория Павлова} , 

с чего ты взяла, что ты меня можешь этим обидеть? ) Вика, ты сказала своё
мнение - я своё) Ты же не просила своё мнение оставлять при себе. Для меня
фильм тяжелый. И в этом фильме показана 100\% правда. И многим, не живущим и не
жившим в Донецке в период войны, откроется правда, даже по этому кусочку.

\iusr{Акулина Жемчугова}
\textbf{Виктория Павлова} , и у тебя прекрасный вкус, как по мне. Ну, не понравился фильм - у меня тоже такое бывает, представляешь) это не повод обижаться ни на кого. В этот раз мы по-разному смотрим на фильм - это нормально и вполне допустимо.

\iusr{Виктория Павлова}
\textbf{Акулина Жемчугова} Аня, тебе Женька не звонил еще?

\iusr{Акулина Жемчугова}
\textbf{Виктория Павлова} , не? А должен был?

\iusr{Виктория Павлова}
\textbf{Акулина Жемчугова} да, должен сегодня позвонить. В личку напишу сейчас
\end{itemize} % }

\iusr{Дина Логвинова}

Фильм посмотрела. Чувства и эмоции, которые он у меня вызвал, аналогичны Вашим,
Виктория. Думала написать об этом, но сработал инстинкт самосохранения... Вам -
респект!


\iusr{Марина Лепёшкина}

Виктория, а Донбасс Окраина смотрели? Какие впечатления? Поделитесь?

\begin{itemize} % {
\iusr{Виктория Павлова}
\textbf{Марина Лепёшкина} Не смотрела и вряд ли буду

\iusr{Алена Михайлова}
\textbf{Виктория Павлова} 

Викуль, я ждала, когда напишешь ты! Фильм я не смотрела, пока.... читаю только
форумы на эту тему. Не смотрю, потому как приехала в гости к куме, она уехала из
Донецка, семь лет назад и возвращаться не собирается! Уезжала из Киевского, под
обстрелами, а мы сами такие же сидели в Углегорске! Ты знаешь, что для меня Это
все- открытая рана! Не хочу ее расстраивать, кумуську и сама пока
расстраиваться не хочу, но уже по отзывам думаю, что не все так прекрасно снято!
В последнее время, большинство все видят, как через призму Скобеевой и Попова!
Наверное, когда я доеду до просмотра этой картины, все страсти улягутся!


\iusr{Виктория Павлова}
\textbf{Алена Михайлова} мне не понравилось. Расстроила бездарная трактовка темы

%%fbauth
%%fbauth_name
\iusr{Татьяна Приходченко}
%%fbauth_name_profile
%%fbauth_url
\url{https://www.facebook.com/tatyana.prikhodchenko}\par
%%fbauth_place
%%fbauth_place_from
%%fbauth_id
%%fbauth_front
%%fbauth_desc
%%fbauth_www
%%fbauth_pic
%%fbauth_pic portrait
%%fbauth_pic background
%%fbauth_pic other
%%fbauth_tags
%%fbauth_pubs
%%endfbauth

Восхищаюсь Вашим мужеством такое написать в условиях ежеминутной пропаганды про
укрфашистов и Россию-матушку-освободительницу... Думала, что никто уже в ДНР не
осмеливается говорить правду! Хотя, конечно, у всех правда своя ... А фильм не
смотрела и не собираюсь!

\iusr{Виктория Павлова}
\textbf{Татьяна Приходченко} 

спасибо! Вы знаете, я давно уже не смотрю телевизор, не читаю пропагандистов -
так проще) И фильм бы этот я не посмотрела, если бы не бум в социальных сетях.
Потом думала, писать или не писать этот пост, зная, что реакция на него будет
неоднозначной. Решила, что напишу. Теперь вижу, что не мне одной он не
понравился.

\iusr{Юлия Андриенко}
\textbf{Виктория Павлова} а я наоборот встречаю чаще такую реакцию, как у тебя. Сказать ничего не могу, ибо смотреть не собираюсь. Потому что муж, потому что люблю, потому что нас ждёт путешествие и мне насточертело ставить мою жизнь на паузу в ожидании победы, конца войны, признания, наступления (нужное подчеркнуть). Это моя жизнь и она проходит. Поэтому здравствуй, мир!

\iusr{Виктория Павлова}
\textbf{Юлия Андриенко} Юля, счастливого тебе отдыха! Наслаждайся жизнью.

\end{itemize} % }

\iusr{Виктория Иванова}
Я полностью с тобой согласна. Получилось какое то одеяло из лоскутов

\begin{itemize} % {
\iusr{Виктория Павлова}
\textbf{Виктория Иванова} точно, у меня тоже нет целостной картины

\iusr{Виктория Иванова}
\textbf{Виктория Павлова} Возможно, чуть лучше Окраины и Ополченочки, но все равно, ощущение, что писали сценарий на коленке и в судорогах.
Чтобы снимать фильм о войне, эту войну нужно пережить
\end{itemize} % }

\iusr{Анатолий Дубина}

Девочки, вы слишком многого хотите... Современное кино - просто... Как три
копейки. А-ля Голливуд... А Вы настраивались на "Войну и мир"... Могу понять
некоторое Ваше разочарование, но таковы реалии жизни. Да и рассчитано оно на
аудиторию, которая о 14м-15м слышала из новостей, но не переживала. Имхо как
говорится.

% -------------------------------------
\ii{fbauth.vasilenko_evgenij.doneck.dnr}
% -------------------------------------

Я думал что мне одному фильм бредом показался, уже начал думать что я нне такой
какой-то. @igg{fbicon.frown}  Да, фильм рассчитан на определенную аудиторию, продолжение
программы "60 минут". Донбасс окраина для меня намного интереснее было
смотреть.

\begin{itemize} % {
\iusr{Виктория Павлова}
\textbf{Евгений Василенко} я не смотрела, "Окраину" и вряд ли в ближайшее время посмотрю. С меня хватит пока этих кинематографических игрищ(

\iusr{Акулина Жемчугова}
\textbf{Евгений Василенко} , ты смотришь « 60 минут?»)

\iusr{Евгений Василенко}
\textbf{Акулина Жемчугова} Бывает. Можно какие-то выводы сделать по политике партии и народа @igg{fbicon.laugh.rolling.floor} 
\end{itemize} % }

\iusr{Наталья Позднякова}
Спасибо за качественный отзыв)

\end{itemize} % }
