% vim: keymap=russian-jcukenwin
%%beginhead 
 
%%file letters.leonid
%%parent letters
 
%%url 
 
%%author_id 
%%date 
 
%%tags 
%%title 
 
%%endhead 

%Добрий день, Сергію!
Доброго вечора, пані Олено!

Мене звати Іван, я киянин, програміст і живу і працюю в Києві. Багато зараз
читаю про Маріуполь, у мене є вже також декілька книжок про Ваше неймовірне
Місто. На жаль, до війни не встиг побувати у Маріуполі, так що мене...  із
Маріуполем мене пов'язує тільки пам'ять про те, що я колись сів на потяг
Київ-Маріуполь, і потім зійшов на проміжній станції... Давно вже це було...
Але... Я близько приймаю до серця, те, що сталось із Маріуполем і
маріупольцями. Страшна трагедія... яка забрала життя близьких, друзів, дітей,
батьків, розкидала маріупольців по всьому світу...  Тим не менш, я абсолютно
впевнений, що кращі часи для Маріуполя ще попереду, і впадати у відчай точно не
треба...  Так, як я вже сказав, я програміст. У вільний від роботи час я також
займаюсь систематизацією різних публікацій про війну, у мене є власний проект
Літопису Війни, яким я займаюсь - поки що сам - власне кажучи, я записую та
збегіраю різні важливі пости на фейсбуці, потім розкидаю їх по авторам та
темам. Технологія, яку я використовую, називається LaTeX (нею користуються
фізики та математики для запису своїх наукових публікацій, це є де-факто
стандарт у науковому світі). Так от.  Я зберігаю у печатному вигляді те, що я
читаю, те, що я вважаю, потрібно зберігати...  Про Маріуполь я теж зберігаю
останнім часом, і доволі багато. І про довоєнний мирний Маріуполь, і про
жахіття війни. У мене є телеграм-канал https://t.me/kyiv_fortress_1, і там
багато вже чого викладено, так як також у цьому фейсбук-акаунті, також я
поступово викладаю - звісно - завжди вказуючи авторів та дату публікації
оригіналу - на https://archive.org - це є сайт Машина Часу - Архів Інтернету -
присвячений збереженню взагалі всього в Мережі - посилання на мій акаунт там
https://archive.org/details/@kyiv_chronicler (chronicler = літописець, тобто
значить Київський Літописець).

Нащо все це робиться... знаєте, конкретно щодо Маріуполя... Маріуполь фізично
то вбили...  але Маріуполь не помер. Душа Маріуполя не померла і жива.
Маріуполь залишився в Духовному Просторі - у пам'яті, фото, усілілих книжках,
старих публікаціях, в живих людях - маріупольцях - розкиданих зараз по всьому
світу...  Але... якщо не записувати, якщо систематично не займатись Маріуполем
в духовній та культурній сфері - якщо цим не займатись... то є ризик... що
Маріуполь помре духовно ...  і це буде вже назавжди. А духовна смерть - це ще
страшніше, ніж смерть фізична. І зокрема, як деякі ознаки цього процесу... Я
зараз живу в Києві. І виявляється, дуже мало книжок про Маріуполь!  Обійшов
десятки книгарень - топових в центрі - поїхав на Петрівку - нема практично
книжок про Маріуполь!  А про це я пишу більш докладно у своєму пості від 03/03/2023
https://www.facebook.com/ivan.ivan.kyiv/posts/pfbid02X7PFBDFiFcNY9Ge7mY2XDvykevGa8KWRUyXKK5MPFsH8NyFSstBCeZPb2GoCQ4mkl

Тому... я всім цим і займаюсь. Маріуполь повинен бути звільнений не тільки
воєнним шляхом, але також і культурним, інформаційним, духовним шляхом.
Сподіваюсь, все, що я пишу, буде Вам цікаво. 

%Потрібно дуже багато ще роботи попереду, роботи духовно-культурної, окрім чисто
%військового звільнення, щоби знову вдихнути в зруйноване Місто Життя на повну,
%знову зробити Маріуполь таким же радісним, сяючим, щасливим, яким він був
%всього навсього рік з гаком тому.

З повагою

Іван.

%09:10:01 04-03-23
%Доброе утро, Леонид! Меня зовут Иван, я киевлянин, программист и живу и работаю

Добрый вечер, Алевтина! 

Меня зовут Иван, я киевлянин, программист и живу и работаю в Киеве. Много
сейчас читаю про Мариуполь, у меня есть также несколько книжек про Ваш
замечательный Город. К сожалению, до войны не успел побывать в Мариуполе, так
что меня... с Мариуполем связывает только воспоминание о том, как я когда то
сел на поезд Киев-Мариуполь, и потом сошел на промежуточной станции... Давно
это уже было... Но... я принимаю близко к сердцу, все, что произошло с
Мариуполем и с мариупольцами. Страшная, невообразимая трагедия...  которая
забрала ваших близких, родных, детей, родителей, раскидала вас, мариупольцев,
по всему миру... Тем не менее, я уверен, лучшие времена для Мариуполя еще
впереди, и отчаиваться точно не надо... Потому что Город - это не только
постройки, улицы, парки, сами по себе... Город - это также Духовная Сущность, в
том числе - это книги, память, фото, истории, люди...  И я как раз из того Города,
который как раз много-много раз разрушали, сжигали, разоряли, здесь творились
ужасные трагедии и бедствия, но Киев каждый раз восставал, залечивал свои раны,
и становился только краше! И я уверен, что Мариуполь тоже обязательно
восстанет из пепла, как птица-Феникс, и снова засияет, да, прекрасная
Жемчужина-Город на берегу Азовского Моря!

... Да, как я сказал выше, я программист. В свободное время от работы... я
также занимаюсь систематизацией разных публикаций о войне, у меня есть
отдельный проект Летописи Войны, который я веду - пока что сам - довольно уже
давно, собственно говоря, я сохраняю от забвения посты на фейсбуке, потом
раскидываю их по авторам, темам. Технология, которую я использую, называется
LaTeX (не знаю, имеете ли вы отношение к физике или математике, но это та
технология, которая повсеместно используется учеными в научном мире для записи
и публикации своих статей и исследований ). Так вот.  Я сохраняю в печатном
виде, то, что я читаю, считаю важным для сохранения...  Про Мариуполь я в
последнее время тоже сохраняю, и довольно много. И про довоенный мирный
Мариуполь, и для ужасы войны. У меня есть телеграм канал
https://t.me/kyiv_fortress_1, там уже довольно много выложено, так же как и на
этом фейсбук - аккаунте, также я постепенно выкладываю собранные  материалы - с
уважением к авторам - я всегда указываю, кто и когда что создал - на
https://archive.org - это интернет сайт, посвященный сохранению всего в
Интернете, ссылка на меня там https://archive.org/details/@kyiv_chronicler.
Зачем все это делается... знаете... Мариуполь физически то убили... но
Мариуполь не умер. Мариуполь остался в Духовном Пространстве, и это очень очень
важно... Но если не записывать... если этим не заниматься... есть риск, что
Мариуполь умрет духовно... и это будет уже навсегда. А духовная смерть - это...
еще страшнее, чем смерть физическая... Поэтому я этим и занимаюсь, хотя я и
киевлянин, вообще то говоря. Хочется, знаете, тоже приехать в наш Украинский
Возрожденный Мариуполь, как гость, в свое время ) Надеюсь, все о чем я пишу,
Вам будет интересно. Если Вы не против, я был бы счастлив также опубликовать в
том числе Ваши стихи в том виде - (буква-в-букву, и с указанием Вашего
авторства), в котором я записываю публикации - в таком виде их нельзя удалить
по чьей то прихоти, как это происходит очень часто на фейсбуке (я в свое время
составил небольшой сборник поэзии о войне из различных постов на фб, ссылка на
него на сайте в телеграме выложена - https://archive.org/details/vijna.poezii).

Я послал Вам запрос на дружбу в фейсбуке, не знаю, принимаете ли Вы новые запросы, 
но буду рад подружиться с Вами здесь на фб.

С уважением, Иван.

