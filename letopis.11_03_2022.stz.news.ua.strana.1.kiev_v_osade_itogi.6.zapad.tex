% vim: keymap=russian-jcukenwin
%%beginhead 
 
%%file 11_03_2022.stz.news.ua.strana.1.kiev_v_osade_itogi.6.zapad
%%parent 11_03_2022.stz.news.ua.strana.1.kiev_v_osade_itogi
 
%%url 
 
%%author_id 
%%date 
 
%%tags 
%%title 
 
%%endhead 

\subsubsection{Ни членства в ЕС, ни закрытого неба}
\label{sec:11_03_2022.stz.news.ua.strana.1.kiev_v_osade_itogi.6.zapad}

Запад тем временем продолжал разочаровывать Украину. Несмотря на множество
заявлений о поддержке Украины, саммит Евросоюза отложил в долгий ящик вопрос о
предоставлении нашей стране статуса кандидата в члены ЕС.

Правда, Европа все-таки решила отказаться от российских нефти и газа, но готова
сделать это только к 2027 году (да и то с оговорами по газу, что это будет
сделать очень трудно). Так что, по сути, речь идет о декларации, от которой
Украине никакой пользы нет.

Следует отметить, что на Западе понимают разочарование украинского общества в
его действиях. Генсек НАТО Йенс Столтенберг так и заявил: \enquote{НАТО понимает
разочарование Украины из-за нежелания Альянса закрыть небо, но эскалация
приведёт к жертвам и страданиям за пределами страны}.

С жертвами и страданиями в \enquote{пределах} Украины в Альянсе, похоже, уже смирились.

Впрочем, такая тактика Запада также своего рода попытка подтолкнуть Украину к
компромиссу с Россией. Россию к компромиссу также подталкивают – постоянно
ужесточая санкции.

Принесут ли такие меры свой эффект – увидим уже в ближайшее время.
