% vim: keymap=russian-jcukenwin
%%beginhead 
 
%%file slova.istoria
%%parent slova
 
%%url 
 
%%author 
%%author_id 
%%author_url 
 
%%tags 
%%title 
 
%%endhead 
\chapter{История}

В українській \emph{історії} цей міф про спартанців також проявився. Відомою
стала \emph{історія} з козацьких воєн, розказана Миколою Костомаровим, який
своєю чергою переповів німецького хроніста Йоахіма Пасторія та польського
шляхтича Станіслава Освенцима. Під час битви під Берестечком сталася цікава
подія, яку хроністи описали приблизно так: «Одна ватага козацька, зложена з
двох або трьох сот козаків, зробивши засіку на однім острівці, ставила мужній
опір так відважно, що Потоцький велів їм обіцяти життя, коли піддадуться. Але
вони не схотіли того прийняти. На знак повного відречення викинули з чересів
своїх гроші у воду, і так сильно почали відбиватися від наших, що кінець кінцем
піхота мусила на них гуртом наступати. А хоч розірвала (їх укріпленнє) і
розігнала з острову — але вони уступилися на болота, не хотячи піддатись, так
що там їх кожного доводилось добивати. А оден з них, добившися до човна, на
очах короля і всього війська дав докази мужності не хлопської! Кілька годин
відбивався він з того човна косою, зовсім не зважаючи на стрільбу — котра чи то
припадком його не трапляла — чи такий твердий був, що кулі його не брали. Аж
оден Мазур з Ціхановецького повіту, розібравшися на голо і по шию бредучи,
насамперед косою вдарив, а потім оден солдат пробив посередині пікою або списом
і добив його таким чином — на велику потіху і вдоволеннє короля, що довго
дивився на сю траґедію»,
\citTitle{Як метафори та міфи керують нашою уявою}, Богдан Шумилович, zaxid.net, 09.06.2021

Люди люблять виразні \emph{історії}, особливо якщо вони драматичні. Такою є \emph{історія}
про 300 спартанців, які захистили батьківщину від більшої кількісно армії
персів. Проте якщо ви дивилися фільм Зака Снайдера, то могло виникнути
відчуття, що перед вами радше ідеологічний соцреалізм, історія про боротьбу
американців зі злим Іраном, а не сюжет про давнє минуле.  У сонячний день
серпня 480 року до н. е. грецьке місто-держава Спарта святкувало одне з
найважливіших фестивалів року, тижневе свято «Карнея», яке було даниною поваги
Аполлону Карнейосу, богу зграй та стад домашньої худоби. На час празнику було
заборонено брати участь у будь-якій війні, щоб бог не зіпсував джерела їжі на
наступний рік. Зазвичай це не було проблемою, але тут з'явився перський цар
Ксеркс, що зібрався помститися за свого батька, який 10 років тому безуспішно
намагався захопити Грецію,
\citTitle{Як метафори та міфи керують нашою уявою}, Богдан Шумилович, zaxid.net, 09.06.2021

Люди схильні вірити у все, що заманеться – брехливим каналам Медведчука чи
«чесним» медіа Порошенка, незрозумілим сайтам типу «Страна», які поширюють
фейки та міфи. Чому, запитаєте ви? Дуже просто: реальність не має смислу,
\emph{історія} – це теж набір беззмістовних сюжетів. Смислу потоку \emph{історії} надають
\emph{історики} чи письменники, а в нашому світі – кінематографісти чи навіть блогери.
Хто створить більш переконливу картинку реальності, той і перемагає в битві за
уяву. І ця картинка не мусить бути правдивою. Окрім того, люди схильні
сприймати світ через стереотипні історії, і якщо ми прийняли якийсь міф чи
метафору з раннього віку, то її дуже складно змінити. Іншими словами, коли я
бачу подію чи якийсь феномен у реальності, я його співвідношу з уже наявним у
моїй голові образом. Себто коли польський шляхтич бачив, як б'ється на смерть
ватага козаків, то він побачив \emph{античну історію}, яку засвоїв ще дитиною від
своїх вчителів. Ми розуміємо реальність лише тоді, коли вона має для нас смисл!,
\citTitle{Як метафори та міфи керують нашою уявою}, Богдан Шумилович, zaxid.net, 09.06.2021

Новое время родило когорту историков-самоучек, которые умеют выжать сенсацию
нового знания буквально из одного документа. Даже целые книги пишут. Вот, к
примеру, наткнулся на очередной тг-канал с секретами тайных архивов кейджиби. А
там выложили уникальное свидетельство подлых замыслов Империи Зла - мишени
Осоавиахима. Враг на них носит каску Броди (Mk III), следовательно \enquote{в СССР
учились стрелять по британцам}. Но камон, друзья, такие же каски носили в
армии США. Значит в СССР готовились стрелять по американцам. А форма на парне
похожа на немецкую, так что СССР готовился первым напасть на Германию, а
британцев косить уже потом. И самое главное: в конце 1930-х годов в РККА было
решено создать стальной шлем для частей ПВО и гражданской обороны. За образец
взяли каску Броди и с 1938 года начали делать ее на Ленинградском металлическом
заводе.  Следовательно, в СССР готовились стрелять по своим! Страшная правда
про заградотряды ПВО раскрыта,
\citTitle{Новое время родило когорту историков-самоучек}, Дмитрий Заборин, strana.ua, 10.06.2021

