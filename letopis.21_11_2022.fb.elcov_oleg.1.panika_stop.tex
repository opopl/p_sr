% vim: keymap=russian-jcukenwin
%%beginhead 
 
%%file 21_11_2022.fb.elcov_oleg.1.panika_stop
%%parent 21_11_2022
 
%%url https://www.facebook.com/permalink.php?story_fbid=pfbid0iDt2zgiAJSV4iHKxFQjD9BwxR91uVfzWiQ8mX5zmsenApuMEXJ9Loun1PGkkob5Pl&id=834102485
 
%%author_id elcov_oleg
%%date 
 
%%tags 
%%title Паника - стоп!
 
%%endhead 
 
\subsection{Паника - стоп!}
\label{sec:21_11_2022.fb.elcov_oleg.1.panika_stop}
 
\Purl{https://www.facebook.com/permalink.php?story_fbid=pfbid0iDt2zgiAJSV4iHKxFQjD9BwxR91uVfzWiQ8mX5zmsenApuMEXJ9Loun1PGkkob5Pl&id=834102485}
\ifcmt
 author_begin
   author_id elcov_oleg
 author_end
\fi

Паника!

Подкрадывается момент истины. С усилением ограничений привычного быта
проявляются алармисты-всепропальщики, невротические персоны и представители
прочих психических патологий. Некоторым легче умереть, чем прожить сутки без
света и интернета. Тот, кто взобрался пешком на 15-й этаж ощущает себя титаном
духа и геркулесом. Но второй раз на такое не способен. Трудно тому, кто родился
на асфальте, всю жизнь провел в офисе и не знает как приготовить корм при
отсутствии электрики. 

То тут, то там появляются островки паники. Умонастроения масс легче всего
оценить в соцсетях и группах ОСББ. Отдельные граждане начинают суетиться и
делиться своей паникой с прочими. По постам видно: у многих урожденных горожан
- истерика, и психиатров на всех не хватит. 

Власть безмолвствует, что только будоражит активность алармистов. Ширятся самые
безумные слухи и рекомендации «экспердов». Тема №1: генераторы. Это очень
напоминает панику двухлетней давности с поисками аппаратов искусственного
дыхания. ДТЭК сообщил, что свет могут отключать на 30 часов, но забыл сказать,
как вести себя при этом. Ужас-ужас: сутки без света! Генератор - вот спасение
от всех бед! На кой ляд он нужен, если способен запитать лишь пару чайников и
зарядить мобилки? Все электронные девайсы окажутся бесполезны - интернет
пропадет. Но с заряженной мобилкой как-то спокойнее сидеть в темноте.  Может
вместо генератора купить электронную книгу? Держит заряд пару недель, ей и
подсветить можно. А главное - не будет времени паниковать: открыл книжку - и
попал в другой мир без перебоев с электрикой и без русского мира.

Эти дни показали насколько люди технически безграмотны. Складывается
впечатление, что вокруг - исключительно менеджеры по продажам, риэлторы, юристы
и айтишники. Скажите: есть хоть кто-то с техническим дипломом - настоящим, не
купленным?  А со здравым смыслом? 150 лет назад Гюстав Лебон объяснил, что под
воздействием толпы сознание индивида падает до уровня человек в толпе с самым
низким уровнем интеллекта. А ведь ничего ужасного не произошло. Просто стало
меньше электричества, просто пропадает интернет, просто воздушные тревоги. Что
же тогда «ужасно»? Спросите у тех же айтишников, сельских тружеников и
менеджеров по продажам, которые девятый месяц сидят в окопах, в грязи, под
пулями и снарядами. 

Паника - стоп! Гуглим, пока есть интернет: как готовить еду на открытом огне и
на газовой горелке, покупаем термобелье, сублиматы, фонарики, обследуем села
поблизости на предмет зимней миграции «еслишо». Детей и особо нервных женщин
эвакуируем в Европу - энергию нужно экономить, лишние потребители не нужны. 

Экономика оказалась живучей, чем казалась. Армия не посыпалась - стоит и
побеждает. Гражданским - не истерить.

\ii{21_11_2022.fb.elcov_oleg.1.panika_stop.orig}
\ii{21_11_2022.fb.elcov_oleg.1.panika_stop.cmtx}
