% vim: keymap=russian-jcukenwin
%%beginhead 
 
%%file 20_01_2022.fb.fb_group.story_kiev_ua.1.detstkie_zabavy.cmt
%%parent 20_01_2022.fb.fb_group.story_kiev_ua.1.detstkie_zabavy
 
%%url 
 
%%author_id 
%%date 
 
%%tags 
%%title 
 
%%endhead 
\zzSecCmt

\begin{itemize} % {
\iusr{Михаил Федюничев}
В обруч с ,, вилочкой ,, из проволоки я сам играл в семидесятых годах ))

\iusr{Елена Гнеденко}

Помню, в послевоенные годы, да и в пятидесятых годах прошлого столетия,
мальчишки, из семей победнее, катали такие обручи.Эта игра была распространена
по всему Киеву.

\begin{itemize} % {
\iusr{Вадим Ляшенко}
 @igg{fbicon.thumb.up.yellow} 

\iusr{Сергей Оборин}
\textbf{Елена Гнеденко} 

Я гонял такой обруч в Пермской области в начале 60-х, куда приезжали с
двоюродным братом летом на каникулы. Для этого сходили на станцию за 2 км в
депо, там нам рабочий-сварщик отрубил от бухты проволоки два кольца и заварил
их в кольцо. Затем зачистил на точиле, и вручил нам. Кочерги сделали сами из
более тонкой проволоки, которой тогда везде в депо было немеряно. До конца
каникул развлечение было обеспечено.

\iusr{Вадим Ляшенко}
 @igg{fbicon.thumb.up.yellow} 
\end{itemize} % }

\iusr{Сергій Батурин}
Я ще бачив, як старші пацани обручі ганяли. Ми вже ні.

\iusr{Vira Stepanova}
Так играли уже и в 20-е годы, встречала в литературе эпизод.

\iusr{Нина Кубанова}
Играли
И мы бегали с такими

\iusr{Elena Sobko}
До и в 60- годы гоняли обруч ребята.

\iusr{Ljudmila Luist}

Взглянув на фото я сразу узнала и вспомнила забытый вид забавы послевоенных
лет. Это было в Эстонии в 48-50 годах, куда папа перевез семью из Ленингр. обл. в
1946г. Надо было спасать детей от голода и начинать новую жизнь после
возвращения из Челябинских лагерей.


\iusr{Михайло Наместник}
І я ганяв вниз по Гоголівській.  @igg{fbicon.smile} 

\begin{itemize} % {
\iusr{Олександр Тимченко}
\textbf{Михайло Наместник} и я развлекался и по Гоголевской и по Коцюбинского(дом угловой)

\iusr{Михайло Наместник}
\textbf{Олександр Тимченко} Привет, земляк. Помните ли вы так называемую \enquote{Каховку} на задних дворах Гоголевской 42-44?
\end{itemize} % }

\iusr{Вадим Вересюк}
Палочка на конце загибалась в виде буквы Ч и назавалась подпихалкой!

\iusr{Вадим Ляшенко}
 @igg{fbicon.thumb.up.yellow} 

\iusr{Вадим Вересюк}
Она делалась из крепкой проволоки.

\iusr{Igor Ilinski}
да, ещё застал, в 60-е

\iusr{Ольга Кириченко}

Так, я пам'ятаю таку гру. Другий варіант - кочерга. У нас вона називалася
дротяник. Звичайно, у хлопців це краще виходило, але в мене було також своє
колесо з велосипеда і дротяник( тато зробив). Мені не можна було без цього, на
моїй вулиці хлопчача компанія. По рівній доріжці, або на стадіоні класно
виходило- тільки встигай бігти!

\iusr{Тарас Єрмашов}

Ще на поч. 80-х в сільській місцевості іноді діти обручі від бочок ганяли та
\enquote{кадила} з консервних банок робили. )

\iusr{Александр Сорокин}
У нас в 61 тоже было такое развлечение.

\iusr{Volodymyr Nekrasov}

В моїх 70-х вже так не бігали. Але якось побачив цю гру у фільмі «Дом в котором
я живу», тож мама охоче пояснила, що то за гра.

\begin{itemize} % {
\iusr{Александр Листровой}
\textbf{Volodymyr Nekrasov} 

Я по Отрадному с друганами гонял, но не обод от бочки а старое колесо от
\enquote{лисапета}  @igg{fbicon.face.grinning.smiling.eyes} 

\iusr{Volodymyr Nekrasov}
\textbf{Александр Листровой} 

точно! Було таке. А ще шини пускали з гори, зовсім не думаючи куди воно
прилетить  @igg{fbicon.man.facepalming} 

\iusr{Александр Листровой}
\textbf{Volodymyr Nekrasov} 

А ещё зимой мы воровали металлические ящики у \enquote{молочного} на углу
Городнянской (теперь Билецкого) и Героев Севастополя и ходили в \enquote{сад} -
теперь парк Отрадный- кататься с горки к \enquote{озеру}. Иногда лёд не
выдерживал, но озерцо было мелким, поэтому обходилось мокрой одеждой и синяками
на заднице от ремня родителей, @igg{fbicon.face.grinning.smiling.eyes}  Кстати,
никто это не считал насилием над ребенком, обычный педагогический процесс! Но
понимание - через жопу в мозги - доходило быстро
@igg{fbicon.laugh.rolling.floor} 

\iusr{Volodymyr Nekrasov}
\textbf{Александр Листровой} о, да! Металеві ящики з-під молочки, це було круто  @igg{fbicon.thumb.up.yellow} 
\end{itemize} % }

\iusr{Иван Купетман}
\textbf{Volodymyr Nekrasov} бігаааалиии)) я 66 року народження, бігали аж гай шумів )))

\begin{itemize} % {
\iusr{Вадим Ляшенко}
 @igg{fbicon.thumb.up.yellow} 

\iusr{Volodymyr Nekrasov}
\textbf{Иван Купетман} охоче вірю. Я на Сирці виріс. Мабуть у нас ланлшафт не підходящий. Все з гори, або у гору.

\iusr{Иван Купетман}
\textbf{Volodymyr Nekrasov}

\ifcmt
  ig https://scontent-frx5-1.xx.fbcdn.net/v/t39.1997-6/p480x480/105941685_953860581742966_1572841152382279834_n.png?_nc_cat=1&ccb=1-5&_nc_sid=0572db&_nc_ohc=1anf0aAHI2UAX9akyI-&_nc_ht=scontent-frx5-1.xx&oh=00_AT9gF_HICKpI4CYSjyxBTpJXwYs4FRhS4YxCmI4rVHHnDA&oe=61EF5C4B
  @width 0.3
\fi

\end{itemize} % }

\iusr{Ірина Кравець}

А мой старший брат с ровесниками играли в пекаря. Мне иногда доставалась
почетная роль поднести консервную банку. Хотя это было так давно.

\iusr{Иван Купетман}
\textbf{Ірина Кравець} ооооо це ще та гра, пальці завжди побиті, на ногах синці, але всі щасливі !!!!

\iusr{Надежда Владимир Федько}

Пам'ятаю таку забаву...

\iusr{Игорь Моржецкий}

У меня в детстве (70-е) было более продвинутое колесо. Оно было приделано к
лыжной палке и ехало очень красиво. .

\iusr{Олена Андурова}

Об игре в колесо знаю, но не видела. Зато видела покатушки пацанов на досках с
подшипниками вместо колес с горки по дороге. Шум, грохот, море удовольствия.

\iusr{Volodymyr Nekrasov}
\textbf{Олена Андурова} 

була десь тут моя публікація саме про такі дошк. Саме так! Шум, грохот...
@igg{fbicon.grin} 

\iusr{Владимир Новицкий}

Спасибо, помню и мы так развлекались, другого не было!!

\iusr{Сергій Бєлах}
Хтось з тих, хто на фото, мабуть ще живий...

\iusr{Геннадий Маргулис}

Гонял. Приблизительно 1961 - 1964 год. Потом пошли серьезные тенировки. Обруч и
и кочережку кому то подарил.

\iusr{Ольга Круковская}

МАЛО ОСТАЛОСЬ НАС, КТО БЫЛ СВИДЕТЕЛЕМ И УЧАСТНИКОМ ДЕТСКИХ РАЗВЛЕЧЕНИЙ И
\enquote{ИЗОБРЕТЕНИЙ}. ДЛЯ ОСУЩЕСТВЛЕНИЯ ИХ В ТЕ ВРЕМЕНА. РОДИТЕЛЯМ ПРИШЛОСЬ
НЕМАЛО ПЕРЕЖЕВАТЬ ОТ ЭТОГО-ВСЕ ПОДВАЛЫ, ЧЕРДАКИ И ПРОЧИЕ \enquote{ЛАЗЕЙКИ} БЫЛИ
ЗАДЕЙСТВОВАНЫ В ДЕТСКИХ ИГРАХ. НИЧЕГО-ВЫРОСЛИ И ТОТ, КТО СТРЕМИЛСЯ К
ОСУЩЕСТВЛЕНИЮ МЕЧТЫ, ДОСТИГ ЭТОГО. ВОСПОМИНАНИЯ ВЗЯЛИ В БАГАЖ И
\enquote{ДОСТАЕМ} ПОРОЮ И ДЕЛИМСЯ С НАСЛЕДНИКАМИ. ЗДОРОВЬЯ ВАМ, ДРУЗЬЯ ИЗ
ПРОШЛЫХ НЕПРОСТЫХ ЛЕТ!

\iusr{Владимир Картавенко}
Грохот и пыль на дорогах. Колеса танков и самолетов можно было \enquote{достать} на меткобинатах... Было.

\iusr{Микола М. Демянов}
було таке, було

\iusr{Геннадий Романов}

Я сам гонял обод от бочки, крючком из проволоки в пятидесятые. Крючок мне папа
кажется согнул на з/д МИТОС (угол ул. Володарского), А были счастливчики находили
колёса от танковых фрекционов, это было по сравнением с моим от бочки - \enquote{мерс}.

\iusr{Николай Гребенкин}

У нас в конц 40-х было такое развлечение. Кочергу делали из проволоки сечением
4-5 мм. и катали обручи от бочки. Этто было круто!

\iusr{Валентина Трефилова}

мой младший брат с друзяками гоняли эти обручи по улицам, девочки этим не баловали...

\iusr{Людмила Серая}

Мой младший брат, гоняя эту \enquote{звездочку}, упал на нее и рассек себе
губу. Зашивали и остался на губе шрам. Это было в середине 50-х.


\iusr{Анна Сидоренко}
Даже я успела ещё погонять колесо кочерёжкой которую брала у мамы от грубки.

\iusr{Вадим Ляшенко}
 @igg{fbicon.thumb.up.yellow} 

\iusr{Татьяна Пастухова Чигрина}

Спасибо, напомнили как я тоже катала обруч изогнутой кочергой по Сырецким
улицам Спасибо,

\iusr{Вова Журавский}

Одно из детских развлечений 50-60 годов ! А еще зимой катание на коньках,
держась за борт грузовика!

\end{itemize} % }
