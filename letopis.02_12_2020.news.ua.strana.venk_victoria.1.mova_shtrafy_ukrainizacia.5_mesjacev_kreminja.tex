% vim: keymap=russian-jcukenwin
%%beginhead 
 
%%file 02_12_2020.news.ua.strana.venk_victoria.1.mova_shtrafy_ukrainizacia.5_mesjacev_kreminja
%%parent 02_12_2020.news.ua.strana.venk_victoria.1.mova_shtrafy_ukrainizacia
 
%%url 
 
%%author 
%%author_id 
%%author_url 
 
%%tags 
%%title 
 
%%endhead 

\subsubsection{Пять месяцев Креминя}

Тарас Креминь был назначен на должность мовного омбудсмена 8 июля. Это бывший
депутат от \enquote{Народного фронта}, который к тому же является соавтором закона о
тотальной украинизации.

Кандидатура Креминя была основной\Furl{https://strana.ua/news/220616-taras-kremen-upolnomochennyj-po-ukrainskomu-jazyku-chto-o-nem-izvestno.html} еще изначально (его рекомендовала Кабмину
омбудсмен Денисова, которая также досталась в наследство от партии
Яценюка-Турчинова).

Однако у Зеленского тогда не решились на столь явный плевок в лицо тем
избирателям, которые голосовали не столько за Зеленского, сколько против
националистов и украинизаторов. Поэтому в итоге пост \enquote{шпрехенфюрера} получила
относительно умеренная педагог из Николаева Татьяна Монахова. 

Тогда это вызвало неудовольствие у тех, кто ставил на Креминя - в частности,
Монахову критиковал экс-глава Института нацпамяти Владимир Вятрович. И другие
представители предыдущей власти, ранее топившие за \enquote{армию, мову, виру}. 

Однако позже Татьяна решила уйти - по ее словам, власть так и не обеспечила ей
ни офиса, ни зарплат для сотрудников. И часть расходов она вообще вынуждена
была оплачивать из своего кармана. 

Был ли это сознательный саботаж со стороны Минкульта - непонятно, но факт
остается фактом: пост по итогу занял Креминь. Которого, в отличие от
предшественницы, снабдили всем, что необходимо. Уже через месяц после
назначения офис мовного омбудсмена получил статус юрлица.\Furl{https://strana.ua/news/282741-sekretariat-upolnomochennoho-po-zashchite-jazyka-stal-jurlitsom.html} Монахова не могла
этого добиться почти год. 

