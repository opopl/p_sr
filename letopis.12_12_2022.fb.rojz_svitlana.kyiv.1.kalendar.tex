% vim: keymap=russian-jcukenwin
%%beginhead 
 
%%file 12_12_2022.fb.rojz_svitlana.kyiv.1.kalendar
%%parent 12_12_2022
 
%%url https://www.facebook.com/svetlanaroyz/posts/pfbid028Ts7i7zqA6vZLZGSgtjysq7JYEUL9SuhaBjSruBCbBF9twC5DAD484PpUS27QKGul
 
%%author_id rojz_svitlana.kyiv
%%date 
 
%%tags 
%%title Календар свят на кожен день
 
%%endhead 
 
\subsection{Календар свят на кожен день}
\label{sec:12_12_2022.fb.rojz_svitlana.kyiv.1.kalendar}
 
\Purl{https://www.facebook.com/svetlanaroyz/posts/pfbid028Ts7i7zqA6vZLZGSgtjysq7JYEUL9SuhaBjSruBCbBF9twC5DAD484PpUS27QKGul}
\ifcmt
 author_begin
   author_id rojz_svitlana.kyiv
 author_end
\fi

Календар свят на кожен день. Грудень. У нас в холодильнику лежить лист до
свМиколая – донька давно використовує холодильник, як поштову скриньку. Вперше
вона написала свій лист українською.  І іноді перепитує "а цього року будуть
свята? А мені, як кожного року, прийде лист від Миколая?" (я подумала: яке
щастя, що підготувала його). Від того, що вона пише в листі, хочеться плакати,
вбивати і обіймати її одночасно. Я знаю, що вона не буде такою, як до війни,
всі діти, що зростають під час війни – вже інші, доросліші, вразливіші і
сильніші. Але вона залишається дитиною. Поки у неї є віра в дива, поки вона
дитина  – я буду зберігати її мрії і допомагати створювати відчуття свята. 

А календар я обов'язково колись опублікую (є ж свята на кожен день року)

Давайте збережемо в собі кожен день грудня, ані дня життя не віддамо ворогу.

Грудень

\begin{itemize}
  \item 1. День брязкання монетами в кишені
  \item 2. День теплих поглядів
  \item 3. День розтягування рота в посмішці
  \item 4. День добрих прикмет
  \item 5. День валяння в кучугурі
  \item 6. День хутра та пуху
  \item 7. День віри в єдинорогів
  \item 8. День плутання слідів
  \item 9. День клацання зубами
  \item 10. День ходіння навшпиньки
  \item 11. День вигадування снів
  \item 12. День вдивляння всередину
  \item 13. День сидіння в пледі
  \item 14. День в'язаних шкарпеток
  \item 15. День рахування днів до весни
  \item 16. День барабанного стукання по каструлях
  \item 17. День висовування ноги з-під ковдри
  \item 18. День богатирського здоров'я
  \item 19. День шелесту обгорток від цукерок
  \item 20. День вигадування подарунків
  \item 21. День розуміння мови фей/гномів
  \item 22. День котячих слідів на снігу
  \item 23. День мандаринових фантазій
  \item 24. День обгорткового паперу
  \item 25. День аромату хвої
  \item 26. День скрипіння снігу
  \item 27. День облизування тарілки
  \item 28. День гірлянд та ліхтариків
  \item 29. День прибирання у внутрішньому саду
  \item 30. День викреслення зробленого
  \item 31. День приручення Щастя
\end{itemize}

ілюстрація \href{https://www.facebook.com/diana.dubossarskaart}{Diana Dubossarska}

Нам зараз так потрібні хоч маленькі доторки радості!

Обіймаю, Родино @igg{fbicon.heart.red} Як хочу Перемоги

\ii{12_12_2022.fb.rojz_svitlana.kyiv.1.kalendar.orig}
\ii{12_12_2022.fb.rojz_svitlana.kyiv.1.kalendar.cmtx}
