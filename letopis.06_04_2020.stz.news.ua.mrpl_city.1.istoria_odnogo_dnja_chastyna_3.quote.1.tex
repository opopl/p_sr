% vim: keymap=russian-jcukenwin
%%beginhead 
 
%%file 06_04_2020.stz.news.ua.mrpl_city.1.istoria_odnogo_dnja_chastyna_3.quote.1
%%parent 06_04_2020.stz.news.ua.mrpl_city.1.istoria_odnogo_dnja_chastyna_3
 
%%url 
 
%%author_id 
%%date 
 
%%tags 
%%title 
 
%%endhead 

\begin{quote}
\small
	
%\newgeometry{textwidth=0.8\textwidth}
%\begin{minipage}{0.8\textwidth}
6 апреля писарь, заведывавший канцелярией \enquote{Начальника Мариупольской команды
внутренней стражи}, по заведенному порядку, в 8 часов утра, явился в небольшую
комнату, на дверях которой был прибит полулист бумаги с криво выведенными
большими буквами надписью: \enquote{Канцелярия}. Писарь был из молодых; эту должность
он занимал только с 1862 года. Раньше он входил в состав команды,
сформированной в городе Екатеринослав из 40 рядовых; команда была сформирована
в 1858 году, и в том же году была прислана в город Мариуполь под начальством ее
постоянного начальника Лисенко. (...)

Писарь войсковой команды отличался как образцовый \enquote{службист} и за образцовое
исполнение дисциплины, в 1860 год был произведен в унтер-офицеры. Обязанности
писаря он исполнял с тем же неизменным усердием, с тем же строгим исполнением
дисциплины, как и обязанности рядового.

...Он достал несколько толстых тетрадей в потертом переплете, затем, обмокнув в
пузырек с чернилом гусиное перо, стал выводить буквы на сырой, немного мохнатой
и похожей на войлок, бумаге, из которой состояли книги для входящих и исходящих
бумаг.

\ii{06_04_2020.stz.news.ua.mrpl_city.1.istoria_odnogo_dnja_chastyna_3.pic.1}

В это же время твердым шагом, не спеша, но и не медля входил приличный,
солидный, с бравым видом господин лет тридцяти. Роста он был выше среднего,
блондин, одет в плохенькое пальто светло коричневого цвета, заметно потертое; в
такие пальто обыкновенно одеты приказчики, стоящие за прилавком. Неизвестный,
войдя в канцелярию, поздоровался с писарем; последний встал, вытянулся,
поклонился, молча сел. И стал опять усердно выводить буквы, но душа его была
уже охвачена какой-то неясной тревогой.

Писарь почувствовал, что необычное появление неизвестного господина неспроста.

Между неизвестным и писарем в это время завязался следующий диалог:

\emph{Неизвестный:} Кто начальник?

\emph{Писарь:} Штабс-капитан Лисенко.

\emph{Неизвестный:} Хорошо-ли обращается с солдатами?

\emph{Писарь:} Так точно, все обстоит благополучно, позвольте доложиться начальн...

\emph{Неизвестный} (резко обрывая): Не надо, останься... Есть печатная бумага? 

Уже после второго вопроса писарь понял, что предчувствие его не обмануло; ему
показалось очевидным, что перед ним какой-то большой начальник \enquote{ибо никто иной
как только начальник, станет спрашивать как обращаются с солдатами}: от этой
мысли он заволновался, \enquote{весь затрясся} и дрожащей рукой положил на стол 6
листков белой бумаги. Между тем \enquote{большой начальник} уже отдал сухо приказ:
\enquote{Напиши на этой бумаге, что я скажу...}

Оторопь охватила писаря, в голове мелькнула мысль удрать, убежать, но это была
только мысль без решимости; на самом деле писарь боялся тронуться с места. Он
только смог произнести тихим, умоляющим голосом.

- Позвольте доложить начальнику команды.

- Но на эту мольбу неизвестный еще более сухо, повысив голос, ответил:

- Не надо. Пиши!

Писарь покорно сел перед листом белой бумаги. Неизвестный, поглядывая в
памятную книжку, высунутую из внутреннего кармана свого верхнего пальто, стал
диктовать, а писарь записывать следующее:

\enquote{По данной мне власти Государя Императора Всероссийского Александра
Николаевича, в Царском Селе 18 мая 1862 года, лицом и Именем которого
повеливаю: бывшего заседателя этого суда Николая Логафетова, за грабеж и
смертоубийство}...

Но силы писаря ему изменили. Уже когда он выводил слово: \enquote{Государя Императора},
ему сперло дух и он чувствовал как земля уходит из-под ног. Когда же были
упомянуты преступления Логафетова, о которых шепотом и почему-то со страхом
говорилось в Мариуполе, как будто бы все были соучастниками его преступлений,
писарь всем своим существом постиг, что перед ним – великую власть имущая
персона, и так заволновался, что его руки с гусиным пером среди стиснувшихся
пальцев запрыгали на бумаге:

Начатый лист был испорчен появившейся на нем чернильной кляксой и до
неузнаваемости безобразно выведенными буквами последних слов, отражавших пляску
руки на блумаге.

Неизвестный прекратил диктовать и повелительно сказал писарю:

Встань, оправься, пройдись по комнате, не надо дрейфить!

Пока писарь отправлялся, неизвестный стал около двери, отрезав путь к бегству.

- Ну теперь пиши, – опять приказал неизвестный, и, после переписывания на
чистый лист уже написаного, продолжал диктовать следующее: \enquote{и вообще за все
злоупотребления лишить всех прав состояния с ссылкой на Алтайские заводы в
вечные работники, имение продать с публичного торга и удовлетворить всех
должников и претендентов; а остальное затем должно поступить в казну}.
Просмотрев написанное, неизвестный собственоручно \enquote{подписывает} его так:
\enquote{Уполномеченный Государя моего, верноподанный Григорий Власов Ильяшенко.
Мариуполь, апреля 6 дня. 1863 года}. Подпись эта не произвела впечатления на
писаря, который после перенесенных тревог, перестал временно реагировать и
только пасивно продолжал писать под диктовку еще 2 приказа.

\end{quote}

