%%beginhead 
 
%%file 08_02_2023.fb.aronov_igor.aktor.rezhysser.kyiv.1.v_yacheslav_dolzhenk
%%parent 08_02_2023
 
%%url https://www.facebook.com/aronov.san/posts/pfbid02tuFzSqUVCi1drnLifCrxYrfQYrS5uppV7eScdiCeG8Eew8h6XTQ7uszcEZJA4Y6al
 
%%author_id aronov_igor.aktor.rezhysser.kyiv
%%date 08_02_2023
 
%%tags mariupol,dolzhenko_vjacheslav.mariupol,mariupol.pre_war,mariupol.war,muzei
%%title В'ячеслав Долженко
 
%%endhead 

\subsection{В'ячеслав Долженко}
\label{sec:08_02_2023.fb.aronov_igor.aktor.rezhysser.kyiv.1.v_yacheslav_dolzhenk}

\Purl{https://www.facebook.com/aronov.san/posts/pfbid02tuFzSqUVCi1drnLifCrxYrfQYrS5uppV7eScdiCeG8Eew8h6XTQ7uszcEZJA4Y6al}
\ifcmt
 author_begin
   author_id aronov_igor.aktor.rezhysser.kyiv
 author_end
\fi

З червня місяця підключився до збору свідчень про воєнні злочини кац..пів в
рамах Архіву війни. Багато чого чув, багато з ким зустрічався, їздив по
Київщині і сподіваюсь поїхати ще. Не писав про це тут, бо і так здається
зрозуміло, що відбулось і відбувається.

Якось так сталось, що останні півтора місяці спілкуюсь виключно з
маріупольцями, що вижили. Досвід їхньої трагедії не можливо описати. З цього
спілкування виникли ідеї мистецько-соціальних проєктів, які, сподіваюсь,
найближчим часом почнуть втілюватись.

Але весь цей час я більш-менш тримався - до сьогодні. Поки не зустрівся з
чоловіком на рік старшим від моєї мами, який в палаючій від рус..ого снаряду
кухні тягнув свою 91-річну маму в одній нічній сорочці до люку свого музею, що
був у підвалі його будинку. Музею, який він зробив самотужки і 30 років збирав
туди антикваріат, ікони, книги, старовинні речі з усього міста, і за який йому
згодом пропонували будь-які гроші. Музею, в який він притягнув з сільського
городу справжню скіфську кам'яну бабу, і яка потім перетворилась на попіл від
температури. Як зрештою і вся справа його життя. За кілька днів, він віз маму
на кравчучці по порожньому зруйнованому району і, побачивши відірвану башту
рус..ого танку, все одно мимоволі подумав, що варто було б затягнути артефакт
до себе і зробити з нього флагшток для майбутніх поколінь. 

Там є ще багато історій про нього - про те як він пообіцяв міській владі
безкоштовно висадити 10 тис дерев за рік і безуспішно роками боровся за це, як
вони самотужки виготовляли унікальний кришталь в радянські часи - матеріалу
вистачить не на один фільм.

На щастя, їх з мамою вдалось влітку дивовижним способом вивезти завдяки
небайдужим друзям. Зараз В'ячеслав живе єдиною мрією - повернутись до Маріуполя
і зробити кращий музей, ніж він мав до цього. І багато хто вже пообіцяв свої
предмети до майбутнього музею.

Вважаю, що про цю дивовижну людину мають знати. 

Лінк на фільм і інтерв'ю в коментарях.

P.S. Коли я вийшов у передпокій за речами, то почув як він питав у лежачої мами
чи все добре, а вона відповідала дуже бадьорим голосом, що ми не здамось і
обов'язково переможемо, і помирати не збирається ні в якому разі. Вона його
власне найбільше і підтримує в цій мрії про відновлення музею.

На фото фраза, яку він щойно виписав з книги і показав мені на прощання.
Смартфону він ніколи не мав і користуватись не вміє, аде обов'язково навчиться,
коли збиратиме експонати для відновлення музею в Маріуполі

%\ii{08_02_2023.fb.aronov_igor.aktor.rezhysser.kyiv.1.v_yacheslav_dolzhenk.cmt}
