% vim: keymap=russian-jcukenwin
%%beginhead 
 
%%file books.alternatyvna_evolucija.suma_ljubovi
%%parent books.alternatyvna_evolucija
 
%%endhead 

\subsection{Альтернативна Еволюція СУМА ЛЮБОВІ}

Мандрівники Духу йдуть, летять, прагнуть по Дорогах Всесвіту, їх стрічають
Друзі й Вороги, ті, що люблять, і ті, що ненавидять. Тисячі закликів звучать.
Куди? Для чого?

Хто допоможе зрозуміти — де Оаза Радості, а де пастка Змія?

Релігія, окультизм, містика, наука, теїзм, атеїзм… Усі прагнуть оволодіти
головним Скарбом Буття — живим Серцем Мислячих Істот. Саме тут — головна битва
мільйоноліть і суть Містерії віків.

Друже мій!

Хто ти — не знаю. Який ти — не відаю.

Може, ти юна дівчинка, котра очікує свого небесного принца? Чи мати багатьох
дітей? Захоплений студент, який схиляється над старовинними манускриптами,
розшукуючи там сліди вічнодіючого Духа? А може, виснажений сумнівами учений? Чи
ти дух незримого світу і лише готуєшся до тяжких земних стежок?

Хто б ти не був, де б ти не був, який би ти не був, — вітаю тебе! Коли б ти не
жив — у минулому, сучасному, в прийдешньому (бо для Вічності все в спільноті),
— я люблю тебе! І шлю тобі своє серце, відкрите всім світам та епохам.

Що хочу сказати? Адже так багато вже сказано. Океан ідей. Потоки слів. Гори
книг. Тисячі кілометрів кіноплівок. Що додам до цього?

Щире слово.

Були вже суми теології. Суми технології. Суми інтелектуальних догм. Я не стану
їх повторювати.

Шлю тобі не наукове дослідження, не хитромудрий сюжет, не законодавчу доктрину.

Співаю Пісню Серця. Даю СУМУ ЛЮБОВІ.

Ти не запитаєш, яка в цьому потреба та необхідність. Ти бачиш, що світ
здригається в передсмертній агонії. Бути йому чи не бути — залежить від нас, бо
це наш світ, створений нашим духом і волею…

Друже мій!

Одного разу в моє серце ввійшов Біль Світу. І став моїм болем. І я вже не міг
позбутися його — ані вдень, ані вночі.

Я усвідомив: злочинно заперечувати цей біль, ігнорувати або забувати. Я
збагнув: його взагалі не повинно бути! Все живе (а живе — все!) палко жадає
буття без болю, без страждання, без трагедії. Біль — то розпечений шворінь,
всаджений невідомим катом у Нерв Світу.

Страждає все: мати, яка народжує дитя, пташка, що кладе яйце, зерно, котре
заперечує себе, щоб стати паростком, клітина, що віддає власне буття для
існування багатоклітинного організму; волає про справедливість оленятко,
спливаючи кров’ю в пащі лева, благають від неба рятунку вівці й корови на наших
бойнях, де готується для гурманів субстрат буття із страждання й крові,
мучиться трудівник для блага абстрактного ідола — держави, карається сонце, що
зігріває планети з їхнім безглуздим життям, корчиться в апокаліптичному трепеті
світовий простір, засіяний ранами зірок та галактик, що прагнуть в нікуди…

О, ми знаємо з тобою, друже мій, що вчені міщани та жерці — холопи грізного
небесного деспота знайдуть у своїх катехізисах чи монографіях археологічні
докази необхідності страждання, його корисності, цінності, його неминучості.
Вони притягнуть для ствердження «істини» священні книги, свідоцтва авторитетів,
хитросплетіння діалектики. Проте що нам до цього? Що нам базікання учених або
жерців світових релігій, які твердять, буцімто вони знайшли істину, але якщо
їхня духовна чи інтелектуальна деспотія не принесла людству Радості й Любові?!

Єдине, що варто шукати, в ім’я чого варто змагатися й помирати, — це РАДІСТЬ! Або ЛЮБОВ, бо це одне й те ж! А якщо істина — не Любов, то нам не потрібна істина. Проте відкинемо слова, рушимо до суті!

Я збагнув: у корені цього Світу відсутня Любов і Радість. Бо те, що породжене Любов’ю, не може й не повинне страждати!

Хто стверджує, що Світло може проявитися лише на тлі Пітьми, що Радість
неможлива без Страждання, той перебуває на первісному рівні мислення, котре
ґрунтується на біполярному, чорно-білому розмежуванні Світу, той ще не проник у
багатомірність Світобудови, в її веселкову поліалектику, в її глибинну Свободу.

Всі «закони» детермінізму та необхідності, які вивчаються нами, — всього лиш
покривало для бездуховних сліпців, полонених ілюзією механічного Космосу. Але
облишимо теорію.

Необхідно знайти рішення, що лежить в корені цього світу: Злочин і Зрада — чи
Випадок, який сотворив наш Всесвіт у неосяжній космічній Лотереї? Випадок
відпадає, бо Світ надто постійний у своїй неухильній і нескінченній
жорстокості. Воістину практично всі інформаційні шляхи Ойкумени, всі русла
Софії-Мудрості узурповані хижаками мороку. Тож сподіватися на чудоподібне
зникнення Світу Муки — пусте заняття.

Необхідні шляхи нові, небувалі, парадоксальні. Потрібна радісна Альтернатива,
Альтернатива Волі, Альтернатива руйнування Космічної В’язниці. Ось чому я
вирішив прийняти до своєї душі цей вічний біль і збагнути Світ, тобто —
збагнути себе!

І що ж я узрів, приклавши до Світу мірило Любові, мірило дитячого серця?

ЛАБІРИНТ МІНОТАВРА — ось справжня назва нашого світу.

Елліни залишили мудрий і глибокий міф, який відображає страшну реальність.
Звіролюдина, котра знаходиться в кожній клітинці життя, невпинно вимагає
страхітливої данини — енергії Світла, Духу, перепрацьованого в процесі еволюції
в тріпотливу тканину життя. Всі їдять всіх, щоб жити!

Але навіщо? Який сенс у цьому потворному взаємопожиранні?

Ця реальність катівського Лабіринту давно жахала духовних мандрівників. Вони
шукали виходу з вселенського черева у Світ Радості, тому що передчували його в
своєму серці. Чулися могутні заклики до Звільнення впродовж тисячоліть. І люди
прагнули услід покликам, але… ще темнішою ставала ніч, ще густішим — туман
ілюзій, ще шаленіше гарчав Мінотавр, ще вище мурувалися стіни Лабіринту.

Вселенський крематорій працює на повну потужність. У топці космічного Молоха
горять тіла, метали, душі, ідеї, держави, нації, племена, планети, зірки. Так
звана еволюція, а з нею й цивілізація — то лише шлаки страхітливого
палахкотіння Матерії у багатті Часу й Простору. Містики й окультисти, віруючі
та аскети мріють про Еру Преображення, де страждання й пошуки відшкодуються
осягненням Істини й Царства Любові. Даремні й марні ці очікування! Вони — той
комплекс ідей, що дозволяє утримувати в Лабіринті страждаючі душі людей.

Прагматична наука і скептичні філософи відкинули потойбіччя, містику, аскетизм
і захопилися земною метою. Так, вони створили безліч практичних цілей. Але
яких? Куди їхня мета веде?

Соціальні концепції оновлення людського життя на практиці показали всю марність
побудови гармонійного світу в ілюзорній сфері тління й відносності. Ці
концепції просто замінили потойбічну релігію поцейбічною, сформувавши ієрархію
уже повністю земних божків-ідолів.

Чому люди уміють мужньо захищати себе, своїх близьких, друзів, рідну країну,
чому вони можуть рятувати лахміття з пожежі, — і вони ж покірно й байдуже ждуть
загибелі всього сущого?

Дивовижний та грізний парадокс. Він мусить бути з’ясований. Тому що якщо справа
й далі піде так, то Планета й Людство приречені. Ні контроль над кількістю
населення, ні праця вчених над створенням синтетичної їжі, ні перспективи
переселення в інші світи зоряного Космосу, ні мрії про опанування термоядерної,
анігіляційної чи іншої енергії, ні проекти збереження Біосфери й захисту
вимираючих видів флори й фауни — ніщо не в силі вирішити головної проблеми:
проблеми Єдності Життя, а з нею й проблеми Єдності Сфери Розуму. Досягнення
найдивовижніших результатів в тій чи іншій окремій галузі науки, творчості,
народного господарства чи соціології можуть навіть прискорити руйнацію Життя,
якщо вони переслідують інтереси частини, а не Цілості.

Але хто і де на Землі думає й творить в ім’я Цілості? Ми бачимо лише її
словесні уламки, що носять назви «Бог», «Людство», «Космос», «Мир», «Істина» і
т. д. Але реальні дії здійснюються лише заради ілюзорних частин, тимчасових
псевдореальностей, що зникають у хвилях Хроносу. Історія тисячократно
продемонструвала людям, що всі їхні труди, війни, створення імперій, повстання,
походи, ідеологічні протистояння — даремні, що вони завжди лишаються духовно
оголеними перед оком Вічності й повинні знову й знову братися до безглуздої
Сізіфової праці, піднімаючи на гору камінь абсурдного буття.

Де причини вражаючої інерції? Втома душі? Сон Розуму? Чи гіпноз невідомого
Ворога, зацікавленого в знищенні Життя?

Необхідно знайти отруйний вірус і знищити його, щоб повернути Буття до Радості.
Заради цієї мети можна віддати все, тому що ми здобуваємо все — і навіть
більше.

Рецепт рятунку єдиний — пробудження. Повернення до Першожиття, котре вічне,
незнищенне і світоносне. Пробуджений погляд по-дитячому бачить те, що тяжко
побачити мудрецям, зануреним у вивчення дискретностей. Він може побачити й
шляхи виходу зі Світового Лабіринту.

Відверте Слово Співчуття має звучати безперервно. Чи зможе воно втілитися в
життя, — залежить від глибини сну людей, закоханих у своє хворе ілюзорне буття.

У цей час останніх можливостей, коли кожне відкрите зусилля може бути рятівним,
ми мовчання вважаємо злочином — якщо серце вимагає термінової дії для
пробудження душ, здатних до Єднання. Тільки спільно можна утвердити шлях
збереження свідомості й розуму серед вселенського хаосу, а отже — шлях до
рятунку Психосфери, що сформована в Океані Всесвіту тяжкою віковічною Еволюцією
Духу.

Чи вірить автор цих рядків в успіх?

Безумовно! Космічний переворот Свідомості здійсниться. Бажано, щоби це Дійство
відбулося з мінімальними руйнуваннями, без «архангельських труб», без кривавої
вакханалії революцій і контрреволюцій.

Учитель Серця в «Пісні Надземній» говорить: «Нитка щастя може бути зіткана.
Тому ще раз звернемося до Серця Народного».

Істинно, лише Серце Народне без лукавих мудрувань може зрозуміти велич і
незвичайність грядущого народження людини від Духу — вогняного Народження.

Окультизм і містика, релігія і марновірні течії зучили думати, що народження
згори, заповідане Христом, — це щось трансцендентне й надприродне. «Пісня
Наземна» ж говорить: «Без містики, але як Соратники Матері Світу, пройдемо в
нову Країну завзято, радісно й стрімко. І хай ніщо не зіб’є вашого ритму».

До Нового Народження покликані всі — вчені й хлібороби, будівники та інженери,
шукачі тонких технологій та романтики зоряних доріг, поети й барди, художники й
зодчі, керівники народів і всі, хто прагне Нового Світу.

Нове Небо й Нова Земля суджені. Ця віра зігрівала мільйони сердець подвижників
у всі віки тяжких духовних пошуків. Де основа такого переконання?

У надії на підтримку Духа Цілості, що незламний при будь-яких падіннях окремих
людей, народів та цивілізацій…

