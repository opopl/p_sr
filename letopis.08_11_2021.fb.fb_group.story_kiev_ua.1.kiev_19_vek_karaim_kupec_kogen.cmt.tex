% vim: keymap=russian-jcukenwin
%%beginhead 
 
%%file 08_11_2021.fb.fb_group.story_kiev_ua.1.kiev_19_vek_karaim_kupec_kogen.cmt
%%parent 08_11_2021.fb.fb_group.story_kiev_ua.1.kiev_19_vek_karaim_kupec_kogen
 
%%url 
 
%%author_id 
%%date 
 
%%tags 
%%title 
 
%%endhead 
\subsubsection{Коментарі}

\begin{itemize} % {
\iusr{Lyudmila Edelweiss}
А я знаю про караимов - была на экскурсии в Крыму  @igg{fbicon.smile} 

\begin{itemize} % {
\iusr{Оксана Денисова}
\textbf{Lyudmila Edelweiss} В Евпатории очень красивая Кенасса!

\iusr{Lyudmila Edelweiss}
\textbf{Оксана Денисова} да, именно там, в Евпатории  @igg{fbicon.face.smiling.eyes.smiling} 
\end{itemize} % }

\iusr{Елена Мельникова}
Благодарю за интереснейшую информацию.

\begin{itemize} % {
\iusr{Оксана Денисова}
\textbf{Елена Мельникова} Спасибо Вам!
\end{itemize} % }

\iusr{Елена Сидоренко}

Очень интересно, спасибо. Была на Вашей экскурсии, Оксана, на улице Пушкинской и
всё видела своими глазами. @igg{fbicon.heart.sparkling}  @igg{fbicon.hands.applause.yellow} 

\begin{itemize} % {
\iusr{Оксана Денисова}
\textbf{Елена Сидоренко} Спасибо Вам, что ходите на мои экскурсии, «живьём» увидеть всегда интереснее @igg{fbicon.grin} 
\end{itemize} % }

\iusr{Andrei Kolinichenko}
Про караимов сегодняшних ни разу не слышал. Где они?

\begin{itemize} % {
\iusr{Natali Grebenik}
\textbf{Andrei Kolinichenko} в Крыму живут , только в Евпатории их около 2000 чел

\iusr{Оксана Денисова}
\textbf{Andrei Kolinichenko} Они есть, но их очень мало осталось.

\iusr{Ирина Алексеенко}

Живут караимы и в литовском городе Тракай. Несколько столетий назад один из
литовских князей вывез в Тракай большую группу караимов - искусных строителей и
ремесленников.

\iusr{Юлия Тертица}
Много караимов было в Литве, в Тракай и сейчас большая община

\iusr{Alexander Bronstein}
\textbf{Andrei Kolinichenko} 

Я общался с ккараимами в Литве, в Тракае. Там у них кенесса есть. Много веков
назад местный князь привёз их из Крыма в свою дружину.

\iusr{Вікторія Максимчук}

Приятно познакомиться! Во мне течет караимская кровь. Нас мало. В каждом
областном центре есть небольшое караимское общество. Центр караимской культуры
перенесен после аннексии Крыма в Мелитополь. В моем родном Николаеве караимов
человек 20-25 не больше.

Караимская кенасса в Киеве так и не возвращена караимскому обществу. В ней был
дом актера. Что сейчас - не знаю. Но такое чудесное здание должно быть
однозначно историческим памятником и не в частных руках.

\begin{itemize} % {
\iusr{Julia Ablamska}
\textbf{Вікторія Максимчук} в ней по-прежнему дом актёра.
\end{itemize} % }

\iusr{Alexey Paschenko}
\textbf{Andrei Kolinichenko} В Украине они компактно жили на Волыни, в Луцке


\ifcmt
  tab_begin cols=2,no_fig,width=0.2

     pic https://scontent-lga3-1.xx.fbcdn.net/v/t1.6435-9/255061607_5089103611099765_7920922044466814686_n.jpg?_nc_cat=104&ccb=1-5&_nc_sid=dbeb18&_nc_ohc=gShz-fUPlI8AX9ZEily&_nc_ht=scontent-lga3-1.xx&oh=a6d2b9ad611928b5b07c18404ecf7e12&oe=61AE887D

     pic https://scontent-lga3-1.xx.fbcdn.net/v/t1.6435-9/254795948_5089104114433048_776356214594163573_n.jpg?_nc_cat=105&ccb=1-5&_nc_sid=dbeb18&_nc_ohc=ksjJHBZC3iQAX8p49Yu&_nc_ht=scontent-lga3-1.xx&oh=37c8abd5eddc23080c6678e2afa3458d&oe=61AE8542

  tab_end
\fi

\end{itemize} % }

\iusr{Лилия Момотюк}

Спасибо! Вообще впервые слышу эту информацию к своему стыду

\begin{itemize} % {
\iusr{Оксана Денисова}
\textbf{Лилия Момотюк} Главное, что теперь Вы знаете!

\iusr{Лилия Момотюк}
\textbf{Оксана Денисова} это да. Благодарю Вас
\end{itemize} % }

\iusr{Nina Cygankova}
Спасибо, очень интересно, здание мне знакомо но истории его не знала, спасибо

\begin{itemize} % {
\iusr{Оксана Денисова}
\textbf{Nina Cygankova} Спасибо Вам!
\end{itemize} % }

\iusr{Всеволод Цымбал}

Всё это прекрасно, память для города. Но схемы ухода от налогов работали уже
тогда. Так называемая благотворительность

\begin{itemize} % {
\iusr{Оксана Денисова}
\textbf{Всеволод Цымбал} Вот не согласна с Вами! Не обижайте Киевских меценатов, они очень много для Киева делали. И они Вам не могут ответить!

\begin{itemize} % {
\iusr{Ирине Вильчинская}
\textbf{Оксана Денисова} 

Нынешним нашим согражданам, к сожалению, не всегда понятно различие таких
понятий, как современное спонсорство и тогдашнее -меценатство и
благотворительность...


\iusr{Всеволод Цымбал}
\textbf{Оксана Денисова} 

Я ни в коем случае не хотел обидеть людей, которые создавали историю нашего
Города. Но благотворительность испокон веку была деятельностью с двойным дном.
Нынешние бизнесмены используют опыт коллег из прошлых времён на 200 процентов.
Какие-то храмы аляповатые, которые только лишь портят городские виды, помогают
попавшим в беду людям на копейку, а отмывают на этом миллионы... Меценаты из
прошлого, конечно, тоже "химичили" в этих вопросах (налоговые службы во все
времена пытались успешных людей обобрать по максимуму), но после них
действительно остались архитектурные шедевры, которые украшают Город уже не
один век

\end{itemize} % }

\iusr{Олена Бородатова}
\textbf{Всеволод Цымбал} 

для киевских купцов, промышленников, учёных конца 19-начала 20 века
благотворительность не была "так называемой". Мы до сих пор с благодарностью
пользуемся профинансированными ими зданиями, учреждениями образования, культуры
и здравоохранения.

\begin{itemize} % {
\iusr{Лидия Гончарук}
\textbf{Олена Бородатова} А КПИ? Разве можно сравнивать с сегодняшним днем

\iusr{Всеволод Цымбал}
\textbf{Олена Бородатова} 

Вот только не надо придираться к словам. Купец в прошлые времена, и нынешний
бизнесмен в первую очередь думают о своём обогащении при минимальных затратах.
И меценатство никогда не было главным мотиватором. Рассказывать о
благотворительности красиво умели всегда. Послушаешь, почитаешь - да купцы и
фабриканты просто в убыток себе работали, чтобы только город украсить. Да
ничего подобного. Шедевры архитектуры, церкви, больницы и прочие деяния
сохранились, за это предкам спасибо. Но не всё так просто. Никто в убыток себе
никогда ничего не делал. В каждом акте меценатства на первом месте стоял личный
интерес. И в нынешние времена ничего не изменилось. Только лишь усугубилось,
причём уродливо

\iusr{Оксана Денисова}
\textbf{Всеволод Цымбал} 

В семье Терещенко было правило, которое было установлено первым из них- Николой
Терещенко, 70\% от прибыли отдавать на благотворительность. И никто из его
потомков ни разу это правило не нарушил. И где здесь расчёт ? В чем? Вы такое
сейчас можете представить?

\end{itemize} % }

\iusr{Rimma Turovskaya}
В то время благотворительность была еще и богоугодным делом.

\end{itemize} % }

\iusr{Nina NinaNina}
Реклама табака Когена в Киеве

\ifcmt
  ig https://scontent-lga3-1.xx.fbcdn.net/v/t1.6435-9/254934336_6331160823625920_4916227171599738599_n.jpg?_nc_cat=105&ccb=1-5&_nc_sid=dbeb18&_nc_ohc=nsyrRYv22OkAX_ont3o&_nc_ht=scontent-lga3-1.xx&oh=752406ff576d713afdc500df6f68eeef&oe=61AFAB98
  @width 0.3
\fi

\begin{itemize} % {
\iusr{Оксана Денисова}
\textbf{Nina NinaNina}

\ifcmt
  ig https://scontent-lga3-1.xx.fbcdn.net/v/t39.30808-6/254498595_10158786935792198_9073828387409100192_n.jpg?_nc_cat=101&ccb=1-5&_nc_sid=dbeb18&_nc_ohc=98r59e9IhhkAX8jDUR0&_nc_ht=scontent-lga3-1.xx&oh=b3fee1fdb75d012bc935da2ae2162bc8&oe=618FC2D4
  @width 0.3
\fi

\iusr{Nina NinaNina}
\textbf{Оксана Денисова}

\ifcmt
  ig https://scontent-lga3-1.xx.fbcdn.net/v/t1.6435-9/254553095_6331166640292005_4133227406899968140_n.jpg?_nc_cat=108&ccb=1-5&_nc_sid=dbeb18&_nc_ohc=qEHMZybNolUAX9xgcps&_nc_ht=scontent-lga3-1.xx&oh=ccde9bd0501b5457487e83bc48205747&oe=61B0A434
  @width 0.3
\fi

\end{itemize} % }
\iusr{Olena Babak}

А теперь дом актера..

\begin{itemize} % {
\iusr{Оксана Денисова}
\textbf{Olena Babak} Да, но уже без купола, к сожалению @igg{fbicon.face.unamused} 
\end{itemize} % }

\iusr{Журавель Павло}

Караимы любят говорить, что они тюрки, хотя в анамнезе семиты, как и евреи.
Фамилия "коген" говорит о носителе, что он из потомков коенов. Служителей
Иерусалимского Храма. Наличие коенских фамилий сообщает о еврейском
происхождении этноса.

\begin{itemize} % {
\iusr{Оксана Денисова}
\textbf{Журавель Павло} Спасибо Вам большое, очень интересное разъяснение!

\iusr{Вікторія Максимчук}
\textbf{Журавель Павло} Вы заблуждаетесь относительно происхождения караимов. Историческое место проживания, язык, обычаи, культовые строения - имеют тюркское, а не иудейское происхождение.

\begin{itemize} % {
\iusr{Журавель Павло}
\textbf{Вікторія Максимчук} 

Это долгий разговор. Да, последние сто лет они доказывают тюркское
происхождение. Началась эта история с последнего гахама С. Шапшала. Это он
"тюркизировал" караимов. Есть дореволюционная открытка на которой на
евпаторийской кенассе, на фасаде имеется шестиконечная звезда. В целом все
еврейские субэтносы похожи на народы среди которых они обитают: горские евреи
на кавказцев, сефарды на испанцев и тд.

\iusr{Журавель Павло}
\textbf{Вікторія Максимчук} Впрочем, если Вы мне приведёте серьёзное исследование на эту тему буду Вам благодарен.

\iusr{Edward Koren}
\textbf{Вікторія Максимчук} 

Караимы происходят от еврейской группы, которая в Египте в 10 веке откололась
от талмудического иудаизма ( раббанитов, по терминологии караимов). Всегда и
везде видели себя частью еврейского народа, по израильским законам имеют право
на репатриацию. Лишь в конце 19 века появилась "тюркская" теория, для того
чтобы антиеврейские законы Российской Империи обошли караимов стороной. Уже
упоминаемый Шапшал довел при совке эту теорию до маразма, придумав "священные
дубы" и прочую чушь. Караимы - часть еарейского народа. Хотя т они этого или
нет.

\iusr{Irena Shvartsburd}
\textbf{Журавель Павло} вы правы на 100\%

\end{itemize} % }

\iusr{Ksenia Sereda}
\textbf{Журавель Павло}, \textbf{Edward Koren} 

в Евпатории на экскурсии ещё в 2007-м году рассказывали, что они евреи. И
раньше знала. А сейчас удивилась, что кто-то этого не знает и более того
отрицается их происхождение в пользу тюркского.

\begin{itemize} % {
\iusr{Журавель Павло}
\textbf{Ksenia Sereda} Бытует "хазарская теория". Довольно популярна среди караимов и крымчаков.

\iusr{Ksenia Sereda}
\textbf{Журавель Павло}, интересно. Спасибо, что написали комментарий и возникла дискуссия. Всегда знала, что они "тюркские евреи". А теперь уже почитала чуток))

\iusr{Журавель Павло}
\textbf{Ksenia Sereda} Они потомки переднеазиатских караимов. Но некоторым нравятся романтические теории.

\iusr{Irena Shvartsburd}
\textbf{Журавель Павло} а в чем их отличие от крымчаков? Всегда считала что это один народ. Если вы знаете, буду благодарна за объяснение.

\iusr{Edward Koren}
\textbf{Irena Shvartsburd} 

Крымчаки - это просто евреи, но крымские, использовали в быту
еврейско-татарский язык, но в религии такие же,,как и все евреи, талмудические.
Караимы - отрицают Талмуд и признают только Тору. Но в быту использовали все
тот же еврейско-татарский язык, ныне мертвый. Крымчаков нацисты уничтожали,
караимов - нет.

\iusr{Irena Shvartsburd}
\textbf{Edward Koren} спасибо.

\iusr{Журавель Павло}
\textbf{Edward Koren} 

Уточню. Этнолект крымчаков нельзя назвать полноценным языком в отличие от идиша
или ладино. Это крымско-татарский с некоторым заимствованием из
древнееврейского. Крымчаки, судя по фамилиям (Ламброзо, Пиастро, Измерли,
Лехно, Варшавский и тд.), смешанная группа куда влились сефарды, восточные
евреи, ашкенази и тд. В течение нескольких веков они сформировались, как
отдельный субэтнос переняв татарский язык, одежду, кулинарию, образ жизни.

\end{itemize} % }

\end{itemize} % }

\iusr{Elena Tsirulnik}
Очень интересно, спасибо! Кенаса случайно была не там, где кинотеатр "Заря" когда то ?

\begin{itemize} % {
\iusr{Оксана Денисова}
\textbf{Elena Tsirulnik} Да, одно время там был кинотеатр «Заря», у меня даже фото где- то есть !

\begin{itemize} % {
\iusr{Elena Tsirulnik}
\textbf{Оксана Денисова} Просто я жила на Рейтерской в 70-ых годах, и всегда говорили что там ,где "Заря"-была синагога,а оказалось была кенаса.Бегали в этот кинотеатр повторных фильмов  @igg{fbicon.grin} 

\iusr{Оксана Денисова}
\textbf{Elena Tsirulnik} Хороший, мне рассказывали, был кинотеатр!

\iusr{Татьяна Петрачек}
\textbf{Оксана Денисова} Да,я там была не раз, очень красиво внутри и снаружи здание очень красивое,жалко,что сейчас никому нет дела,реставрировать бы его.

\iusr{Оксана Денисова}
\textbf{Татьяна Петрачек} Там и сейчас внутри красиво, но реставрировать обязательно надо!

\iusr{Elena Tsirulnik}
\textbf{Оксана Денисова} Да, было время @igg{fbicon.face.grinning.smiling.eyes} 
\end{itemize} % }

\iusr{Zoya Pshenichny}
\textbf{Elena Tsirulnik} Да ,Леночка, мой любимый и родной кинотеатр Заря.

\begin{itemize} % {
\iusr{Elena Tsirulnik}
\textbf{Zoya Pshenichny} Ага, бегали туда постоянно в кино. @igg{fbicon.face.zany} 
\end{itemize} % }

\end{itemize} % }

\iusr{Петр Кузьменко}
Очень интересный и великолепно изложенный материал. Благодарю!  @igg{fbicon.hands.applause.yellow} 

\begin{itemize} % {
\iusr{Оксана Денисова}
\textbf{Петр Кузьменко} Спасибо большое!

\iusr{Rozana Kruchinina}
\textbf{Петр Кузьменко} спасибо большое
\end{itemize} % }

\iusr{Раиса Карчевская}
Оксана!
Большое спасибо за очень интересный пост и фотографии

\begin{itemize} % {
\iusr{Оксана Денисова}
\textbf{Раиса Карчевская} Спасибо Вам!
\end{itemize} % }

\iusr{Владимир Ходоровский}

В этом здании одно время был Кукольный театр, а затем кинотеатр "Заря".Если
будете в Крыму (Бахчисарай) - посетите Чуфут-Кале! А если занесёт в Литву, то
познакомьтесь с Тракаем (под Вильнюсом) - столицей караимов. Эти места значительно
расширят познания о караимах!

Несмотря на иудейские корни, Гитлер не трогал караимов(В отличие от евреев).

P. S. Я уже соскучился за нашими встречами на Лиге экскурсоводов Киева(?!)

\begin{itemize} % {
\iusr{Оксана Денисова}
\textbf{Владимир Ходоровский} В Крыму была в Чуфут- Кале, а вот в Тракае нет, надо будет побывать! Спасибо!

\iusr{Roman Sytnyk}
\textbf{Владимир Ходоровский} За Галич забули.

\begin{itemize} % {
\iusr{Сергей Пацкин}
\textbf{Roman Sytnyk} Какой из Галчей, наш или уркаинской?

\iusr{Roman Sytnyk}
\textbf{Сергей Пацкин} Український. Там велика колонія караїмів була у 1920-30-х роках.

\iusr{Вікторія Максимчук}
\textbf{Roman Sytnyk} Насколько мне известно караимов в Галиче не осталось. Но есть музей караимской культуры.
\end{itemize} % }

\end{itemize} % }

\iusr{Aleksandr Mitryaev}

Мало кто знает, что во время наибольшего расцвета польско-литовского
государства, литовский князь Витовт, то примерно 1380 год, вывез из Чуфут-Кале
200 семей караимов и они поселились вблизи замка Тракай в Литве. До сих пор по
Вильнюсу и Каунасу ходят люди, имеющие паспорт Литвы, Но их лица так и не стали
славянскими, литовскими: они круглы и эти люди гордятся своим происхождением,
берегут свою идентичность ....

Во времена СССР, такие факты тщательно замалчивались, так как русские
приписывали себя к славянам, а не к финно-угорским племенам. Но поляки 380 лет
после татаро-монгольского нашествия владели киевскими землями, Киев был
окраинным городом Польско-литовской империи. На Щекавице в Киевестоял замок
польского Киселя и так далее.

Вплоть до того, что земли южнее Полесья заселялись украинскими потомками.
народа Киевской Руси.

Вплоть до фальсификации истории. Во время татаро-монгольского нашествия в 1240
году никакого перетока людей из киевских территорий на север не было, а в Киеве
во время наибольшего расцвета Киевской Руси имел всего 50 000 жителей.

Вплоть до того, что начало Москвы не 1147 год, а на 200 лет позже.

\begin{itemize} % {
\iusr{Оксана Денисова}
\textbf{Aleksandr Mitryaev} спасибо Вам за такой подробный экскурс в историю!

\iusr{Елена Сидоренко}
\textbf{Aleksandr Mitryaev} скажите, это караимов называли "литваки" или литовских евреев за акцент?

\iusr{Aleksandr Mitryaev}

Это, Елена Сидоренко, я не понимаю. Тупой. Караимы не литваки. Караимы, тюркский народ, а по религии одна из веток иудаизма. Не путайте евреев и их религией. В Киеве кеншаса караимов была одна (Кинотеатр Заря, повторного кино, Дом актера), Синагог было центральных три. Это при Советах Ляльковый театр, вторую вспомню, скажу. А третью переделали, надстроив ещё этаж в Дом автомобилиста, там был кинотеатр, Дом детского творчества ...
\end{itemize} % }

\iusr{Анна Сидоренко}

Спасибо вам большое, я не знала про купца и табачный завод, да ещё и на
Крещатике.

\begin{itemize} % {
\iusr{Оксана Денисова}
\textbf{Анна Сидоренко} Крещатик, 17 - это был адрес его фабрики

\iusr{Анна Сидоренко}
Спасибо Оксаночка , не люблю вилять хвостом, если я не знаю то я прямо и пишу об этом.

\begin{itemize} % {
\iusr{Оксана Денисова}
\textbf{Анна Сидоренко} Всего знать невозможно, тем интереснее узнавать что- то новое!!!

\iusr{Анна Сидоренко}
Согласна!
\end{itemize} % }

\end{itemize} % }

\iusr{Нина Гордийчук}
Спасибо за интересную информацию.

\iusr{Наташа Котляренко}

Караимы были верной гвардией литовского князя. И они жили в тракае. Кстати, во
времена СССР, когда в паспорте была графа национальность, в стране было около
100 человек караимов в Литве. Хотя Сталин их выслал в Сибирь.

\iusr{Марианна Смакова}
Огромное спасибо, осень интересно.

\begin{itemize} % {
\iusr{Оксана Денисова}
\textbf{Марианна Смакова} Спасибо Вам!
\end{itemize} % }

\iusr{Владимир Ходоровский}

Караимы, как и теперь швейцарцы, охраняющие папскую резиденцию в Ватикане, были
на службе в охране замка литовский королей!

\begin{itemize} % {
\iusr{Оксана Денисова}
\textbf{Владимир Ходоровский} Интересно!

\iusr{Юрий Панчук}
\textbf{Владимир Ходоровский} И Крымского хана

\iusr{Юрий Панчук}
\textbf{Владимир Ходоровский} Считались самыми неподкупными стражами

\iusr{Юрий Панчук}

Не знаю, писал ли кто то про табачку Когенов на бульваре Шевченко, а после
1985г в начале проспекта Победы. Ещё несколько лет назад она стояла.

\begin{itemize} % {
\iusr{Оксана Денисова}
\textbf{Юрий Панчук} Табачка попала в руки компании Реемтсма- Империал Тобакко, но потом фабрику закрыли. Даже не знаю , стоит ли здание.

\iusr{Татьяна Годынская}
\textbf{Юрий Панчук} нет там уже фабрики... @igg{fbicon.face.rolling.eyes} 

\iusr{Ирина Петрова}
\textbf{Оксана Денисова} нет, здание уже разрушено.
\end{itemize} % }

\iusr{Людмила Дзюбенко}
\textbf{Владимир Ходоровский} была в. Литве в 1987 году на экскурсии, возили на улицу, где проживают караимы. Их осталось очень мало.

\iusr{Аркадий Шухман}
\textbf{Владимир Ходоровский} В городе Тракай в Литве есть музей караимов, был там лет сорок назад. Очень интересно, надеюсь, что он существует и сегодня. Караимы охраняли в свое время Тракайский замок.

\end{itemize} % }

\iusr{Валентина Юшко}
Спасибо. Киевлянка, но не знала об этом факте. Побольше бы таких публикаций. Еще раз спасибо.

\begin{itemize} % {
\iusr{Оксана Денисова}
\textbf{Валентина Юшко} Спасибо Вам!
\end{itemize} % }

\iusr{Светлана Гаврилко}
Спасибо за интересную статью!!!!!

\begin{itemize} % {
\iusr{Оксана Денисова}
\textbf{Светлана Гаврилко} Спасибо Вам!
\end{itemize} % }

\iusr{Irena Visochan}
Спасибо, большое, я впервые слышу эту историю, очень интересно.

\begin{itemize} % {
\iusr{Оксана Денисова}
\textbf{Irena Visochan} Спасибо Вам!
\end{itemize} % }

\iusr{Liudmila Kabanova}

Кенаса прекрасна! Каждый день мимо неё прохожу. Там реставрация аж "плачет"!
Сверху распадаются фрагменты лепки.

\begin{itemize} % {
\iusr{Оксана Денисова}
\textbf{Liudmila Kabanova} Да, реставрировать ее очень нужно! Если бы ещё купол восстановить !

\iusr{Наталия Ковалева}
\textbf{Liudmila Kabanova} А рядом находился "Еврейский театр", несколько раз была на постановках,очень понравилось!
\end{itemize} % }

\iusr{Татьяна Васильевна Зубко Маркина}
В Крыму видела город караимов. Спасибо за интересный рассказ и всем за комментарии

\iusr{Галина Ливанская}
это здание сохранилось?

\begin{itemize} % {
\iusr{Оксана Денисова}
\textbf{Галина Ливанская} Кенасса? Конечно, сохранилась, Ярославов Вал 7. Она теперь только без купола.

\ifcmt
  ig https://scontent-lga3-2.xx.fbcdn.net/v/t39.30808-6/254710291_10158787029927198_6240391643539576128_n.jpg?_nc_cat=102&ccb=1-5&_nc_sid=dbeb18&_nc_ohc=t-UOz2RVkJQAX86ayjx&_nc_ht=scontent-lga3-2.xx&oh=bc63f1852461f4ff93b0b7210139264c&oe=618FAF89
  @width 0.4
\fi

\begin{itemize} % {
\iusr{Анна Слонина}
\textbf{Галина Ливанская} да сохранилось.

\iusr{Галина Ливанская}
\textbf{Анна Слонина} спс

\iusr{Аркадий Израилевский}
\textbf{Галина Ливанская} бывший кинотеатр "Зоря"
\end{itemize} % }

\end{itemize} % }

\iusr{Виктория Логинова}

Про караимов слышала, была в Бахчисарае и Чуфут-Кале. Но, что они были в Киеве
и даже построили кенасу (к сожалению, почему-то совсем её не помню), не знала.
Очень интересно. Спасибо огромное, Оксана!

\begin{itemize} % {
\iusr{Оксана Денисова}
\textbf{Виктория Логинова} Спасибо Вам!

\iusr{Mykhaylo Losytskyy}
Дім актора на Ярославовому Валу, близько до Золотих воріт.

\begin{itemize} % {
\iusr{Виктория Логинова}
\textbf{Mykhaylo Losytskyy} спасибо большое. Дом актёра знаю, но не поняла, что речь об этом здании. Очень красивое, хоть и требует реставрации.

\iusr{Sem Zilman}
\textbf{Mykhaylo Losytskyy} Yes! Я там бывал очень часто, так мой приятель был в Доме Актёра зам.директора...

\iusr{Валентина Осокина}
\textbf{Mykhaylo Losytskyy} Бывший кинотеатр Заря
\end{itemize} % }

\iusr{Ирина Ярмак}
\textbf{Виктория Логинова} есть в Литве город Тракай, недалеко от Вильнюса. Там живут караимы, которые были вывезены князем Витаутосом время его военного похода на Крым.

\begin{itemize} % {
\iusr{Виктория Логинова}
\textbf{Iryna Yarmak} спасибо. К сожалению, ещё там не была.

\iusr{Наталья Корниенко}
\textbf{Виктория Логинова Тракай} - это must see, волшебный мир, замок на острове! Два года назад поезки в Вильнюс были доступны
\end{itemize} % }

\iusr{Roman Sorokhtei}
\textbf{Виктория Логинова} Караїми з Криму на заклик і по запрошенню короля Данила збудували Галич!
\end{itemize} % }

\iusr{Наталия Ковалева}

У мужа бабушка, киевлянка, рассказывала, что девушек, продающих папиросы
называли "табакрошками", прямо провели экскурсию, еще раз пройдусь по этим
местам, очень интересно, спасибо!!! @igg{fbicon.maple.leaf}  @igg{fbicon.fallen.leaf} 🎩

\end{itemize} % }
