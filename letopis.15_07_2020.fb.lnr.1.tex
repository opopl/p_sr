% vim: keymap=russian-jcukenwin
%%beginhead 
 
%%file 15_07_2020.fb.lnr.1
%%parent 15_07_2020
 
%%endhead 
  
\subsection{Девушки-снайперы ВСУ со свастикой вошли в один из посёлков Донбасса}
\url{https://www.facebook.com/groups/LNRGUMO/permalink/2847502238694665/}

\cite{15_07_2020.fb.lnr.1}

\vspace{0.5cm}
{\small\LaTeX section: \verb|15_07_2020.fb.lnr.1| project: \verb|letopis| rootid: \verb|p_saintrussia|}
\vspace{0.5cm}

\ifcmt
fig_begin 
	tex \centering
  img_begin 
    tags 15_07_2020.fb.lnr.1
    tags 15_07_2020,fb,lnr,1
    width 0.5
  img_end
fig_end
\fi

%\rindex{Праздники!Преображения Господнего}
%\rindex{Преображения Господнего}

\index[index.rus]{Преображения Господнего}

19 августа 2014 года --- в великий праздник Преображения Господнего --- я вышла из
храма святого Георгия Победоносца и увидела страшную картину. На центральных
улицах села стояли танки, БТРы, машины, в которых находились солдаты украинской
армии. Больше всего меня потрясли девушки-снайперы с немецкой свастикой... В
тот момент мне было очень страшно, я подумала: «Как хорошо, что мой дедушка,
который воевал в годы Великой Отечественной войны, не дожил до этого дня».

\index[index.cities.rus]{Старобешево, пгт}
24 августа нам были слышны выстрелы в \city{пгт Старобешево}. А в ночь с 26 на 27
августа обстреляли моё село. Были повреждены несколько домов. Мы с детьми
прятались от снарядов в подвале. 28 августа нас обстреляли из «Градов». Два
снаряда попали прямо во двор моей бабушки. Дом был разрушен, но я благодарила
Бога, что остались в живых дети, которые играли неподалёку в песке, и
беременная женщина, сидевшая рядом с ними\ldots К сожалению, в тот страшный день
была убита одна женщина-беженка, ранены первоклассница Лиза Бабаш и
учитель-пенсионер Антонина Попова.

\index[index.names.rus]{Попова, Антонина}
\index[index.names.rus]{Барабаш, Лиза}

В сентябре в село вошли ополченцы. Сначала мы думали, что надо немного
потерпеть, и всё скоро закончится, но зимой обстрелы возобновились\ldots В декабре
2014 г. --- январе 2015 г. село бомбили из «Урагана», «Градов» и танков. На
школьную территорию попало несколько снарядов. Были ранены наши односельчане ---
\name{Григорий Дубина}, \name{Исаак и Неонила Гудиновы}. Тоже наш односельчанин, Василий
Хумуриц, получил тяжёлое ранение. Его доставили в больницу г. Докучаевска, где
он и скончался от большой потери крови. Там же был похоронен, так как из-за
сильных обстрелов его не смогли привезти в родное село...
\index[index.names.rus]{Хомуриц, Василий}
\index[index.names.rus]{Гудинова, Неонила}
\index[index.names.rus]{Гудинов, Исаак}
\index[index.names.rus]{Дубина, Григорий}

Наше село оказалось на линии фронта, тянулись мучительные долгие дни войны\ldots Но
односельчане в этих условиях продолжали жить. Сажали огороды, вели приусадебное
хозяйство, начали отстраивать повреждённые в результате обстрелов дома. Нам
была выделена гуманитарная помощь в виде строительных материалов.

В один из августовских дней 2015 года в 5:30 утра я, как обычно, выгнала
пастись гусей. Стояла тишина, и на её фоне неожиданно раздался взрыв. Снаряд
упал на соседнюю улицу. Я подумала, что это случайно, но потом начался обстрел.
Украинский танк бил прямо по домам на улицах Дзержинского и Горького. Были
ранены \name{Екатерина Шевченко} (она еще спала), \name{Василий Федорец} (осколки его
«догнали» в подвале). Мы сидели в подвале и молились. Дети смотрели на меня, и
я просто не имела права плакать, чтобы им не было страшно. Хотя душа моя
разрывалась от боли. За что нас убивают?\ldots
\index[index.names.rus]{Василий, Федорец}
\index[index.names.rus]{Шевченко, Екатерина}

Несмотря на эти страшные события, жители нашего села не хотели бросать свои
дома. Да и как бросить всё, что нажито собственным тяжёлым трудом, как бросить
свою родную землю? Но и находиться в постоянном страхе было невыносимо. И
потянулись вереницы машин\ldots Люди прятались от бомбёжек в соседних пгт
Старобешево и Новый Свет. На ночь выезжали из села, чтобы хотя бы ночью
отдохнуть, а утром возвращались домой --- управиться по хозяйству, собрать
фрукты-овощи, выкопать картошку.

Как-то в августе я собирала помидоры, и в этот момент начали стрелять. Мне
повезло, я уцелела. А вот сосед \name{Сергей Козлов}, в это же время тоже работавший в
огороде, был тяжело ранен и скончался в больнице. Двое маленьких детей остались
без отца\ldots
\index[index.names.rus]{Козлов, Сергей}

В августе 2016 года также был сильный обстрел нашего села из «Градов» и танков.
Несколько домов были повреждены\ldots Мои земляки --- мужественные люди, и благодаря
им родное село живёт, вопреки всему. Работают школа, детский сад, амбулатория,
Центр греческой культуры, сельская библиотека. Ежегодно, несмотря на военные
действия, у нас проходит национальный греческий праздник «Панаир». Даже в
бомбёжки был открыт храм святого Георгия Победоносца. Были установлены
поклонные кресты на подъезде к селу. Односельчане молились и просили Бога о
мире.

\ldotsСейчас, когда я пишу эти строки, периодически слышны взрывы в г. Докучаевске.
До сих пор люди ложатся спать одетыми, а рядом всегда --- сумка с документами. Мы
постоянно напряжены, ведь никогда не знаешь, в какую минуту и где может
разорваться снаряд. Но мы верим в то, что добро всегда побеждает зло, а после
войны всегда наступает долгожданный мир\ldots

Татьяна Д. пгт. Осыково, Старобешевский район
\index[index.cities.rus]{Осыково, пгт, Старобешевский район}

В проекте «Как я встретил начало войны» каждый житель Донбасса может
рассказать, как именно изменила война его жизнь, что произошло в его судьбе с
началом боевых действий в Донбассе. Необходимо, чтобы весь мир узнал о тех
тревожных днях 2014 года, когда началась гражданская война.

Вы можете отправить свою историю нам на почту: pismo@donbasstoday.ru
