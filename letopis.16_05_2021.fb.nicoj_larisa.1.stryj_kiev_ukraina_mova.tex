% vim: keymap=russian-jcukenwin
%%beginhead 
 
%%file 16_05_2021.fb.nicoj_larisa.1.stryj_kiev_ukraina_mova
%%parent 16_05_2021
 
%%url https://www.facebook.com/larysa.nitsoi/posts/4302342456464520
 
%%author 
%%author_id 
%%author_url 
 
%%tags 
%%title 
 
%%endhead 
\subsection{Стрийщина – земля обітована для кожного українця}
\Purl{https://www.facebook.com/larysa.nitsoi/posts/4302342456464520}

Стрийщина – земля обітована для кожного українця. 

Стрийські діти не вірять, що в Києві учні, вчителі з учнями і люди на вулицях говорять російською.  
- Що, справді?! – перепитували по кілька разів. 
У Стрийських школах висять портрети Бандери, Шухевича та інших націоналістів. Часто в рушниках, як ікони в церкві.
У Стрийській сьомій школі (в школі, люди, не в музеї) на дітей дивиться великий стенд  «Декалог українського націоналіста». Хто не знає, це 10 правил, які знають стрийські діти.
1. Здобудеш Українську Державу, або згинеш у боротьбі за Неї.
2. Не дозволиш нікому плямити слави, ні чести Твоєї Нації.
3. Пам’ятай про великі дні визвольних змагань.
4. Будь гордий з того, що ти є спадкоємцем боротьби за славу володимирового тризуба.
5. Пімсти смерть Великих Лицарів.
6. Про справу не говори з тим, з ким можна, а з тим, з ким треба.
7. Не завагаєшся виконати найнебезпечнішого чину, якщо цього вимагатиме добро справи.
8. Ненавистю і безоглядною боротьбою прийматимеш ворогів твоєї нації.
9. Ні просьби, ні грозьби, ні тортури, ані смерть не приневолять Тебе виявити тайни.
10. Змагатимеш до поширення сили, слави, багатства й простору української держави.
Завмираю перед стендом, гамуючи здивовано-радісне калатання серця. Неймовірно, діти в школі виховуються за цими правилами.
- Як! Як ви таке тут повісили?! – питаю в присутніх представників відділу освіти, дві пані Ірини.
- А що? - запереживали вони. - Ви знайшли помилки?
Помилки?! Регочуся. Уявляю, на які клаптики порвали б директора школи в будь-якому іншому українському місті за такий стенд. А тут департамент освіти видивляється, чи немає в тексті помилок. 
- Мені особливо подобається пункт 6, - каже директор Ярослав Богданович.
- А мені особливо подобається 4 і 10 пункт. – кажу в унісом йому. -  І особливо вражає пункт 8. - До вас треба всю Україну возити на навчання, як дітей треба виховувати. Можна я сфотографую це і виставлю в дописі?
- Можна, а чому Ви питаєте? 
- Переживаю, що після мого допису замість похвали на вас посиплеться… Тому й перепитую, щоб були готові.
- Ой, - сміється директор школи. - Можете від мене всім українцям ще й анекдот передати. 
Який «м-ль» кращий, перший чи другий? Перший. Бо перший прийшов у 1939-му і пішов у 1941-му. А другий прийшов у 1944-му та так і лишився.
Сміюся разом з директором і тишком розглядаю його.  Що за порода така, не боязлива? А він тим часом показує мені на мурал на стіні школи. 
- Це діти захотіли, кажуть, модно. Самі гроші збирали. А я вибивав дозвіл в архітектурі, бо ж ми пам’ятка. З художниками радилися, щоб не псувало стіни. Кажу дітям, усе розумію, легінь, бартка, книжка, а що то за песик такий? А вони сміються і кажуть, що то головна фішка.  
У школі, якою він керує, колись учився Степанко Бандера. В шкільному коридорі висить велике старе люстро в рамі, в яке колись бігав і заглядав підліток Бандера, майбутній лідер нації. Трепетно фотографуюся біля цього дзеркала. 
…Наступною в графіку бібліотека. Робочий час. Хто прийде в таку годину на зустріч зі мною? Подумки розслабляюся. Прийде п’ятеро, буде п’ятеро. Заходжу. Яблуку немає де впасти. 
- Ми повісили оголошення, всі збіглися, - щебече пані Людмила, директор бібліотеки.
Гм. Це ж треба, в Стрию народ читає бібліотечні оголошення. 
- Бачу, у вас тут дружна спільнота. Ви «свиснули» - і всі збіглися.
- Так, дружна, - хвалиться пані Людмила і розповідає про їхній сильний волонтерських рух, який і досі возить хлопцям на передову варенички… - Буває по-різному, як у великій родині, що й погиркаємося... Але як тільки мова йде про Україну – ми всі як один, плече біля плеча…
… Побували в 10-й школі та в Лисятицькій. Дітей годують як у ресторані. В Нежухівській школі директор Анатолій Миколайович з гордістю показує ремонти, хвалиться, що 4 випускників (сільської школи) взяли майже 200 балів на ЗНО і каже, що прийшов на директорське крісло з бізнесу. І знову в мене шок, бо ж, як правило, зі школи тікають. Нащо їм та школа з тими зарплатами, якщо можеш займатися бізнесом? А тут навпаки, бізнесмен приходить з бізнесу і стає директором школи, щоб її підняти. І таки піднімає… 
У школах на Стрийщині часто можна побачити табличку над класом «Кабінет християнської етики». 
Священники на Стрийщині говорять про любов до Бога, до України і що з ворогом треба боротися. Священники Стрийщини говорять, що ми, українці, повинні усвідомити, що північний ворог ніколи не буде нам другом. Запам’ятали? Ніколи не буде нам другом. Не треба тішити себе ілюзіями. (Згадайте друга Макаревича з нашим борщем, який, виявляється, не наш. Каже наш друг.). Але священники Стрийщини говорять, що гіршим ворогом для українців є наше взаємопоборювання, яке призводить до самознищення.
На зустріч, яка відбувається у церкві, приходять атовці на чолі з Назаром Фоміним. Приносять веселий подарунок від ГО «Ветеранське братерство». Футболка з хештегом ні_існуванню_росії. І зображення мапи, де замість тієї частини суші – Теплий океан. Тут же одягаю футболку на себе… 
У Стрию і на Стрийщині ніде не звучить російська мова від слова «взагалі». Не звучить в магазинах. Не звучить в кафешках. Не звучить на вулиці. Не звучить від посадовців. Не звучить від людей. Тут з динаміків не звучать російськомовні пісні від слова «зовсім». І українські російськомовні співаки тут не звучать. Скільки прислухалася – всюди лише українськомовні пісні і зарубіжні (московські – не зарубіжні, то ворожі). «Ну не може цього бути, що ніде не звучить російське», - кажу я собі і прямую до вуличної кав’ярні. Точно з такої в Києві, коли я перед дорогою в Стрий купувала собі каву, на всю вулицю з динаміка волало «Ах какая женщіна»... Іду до схожої кав’ярні в Стрию, з якої теж лунає на вулицю музика. Лунає різна, окрім москвомовної. 
В останній день перебування в Стрию не витримую. В ресторані, де ми вечеряли, і де поруч в інших людей відбувалося веселе гуляння з танцями (не польки, а якісний кавер-гурт почергово з  плейлистом) питаю в офіціанта, вдаючи ображену, ніби мені дуже треба російське: 
- Скажіть, будь ласка, я за цілий вечір не почула жодної російськомовної пісні. Куди ви їх діли?
Друзі, стримуючи сміх, уткнули носи в тарілки.
- Вибачте, - каже офіціант,- але ми не вмикаємо такі пісні.
- Чому? – питаю здивовано.
- Ну, тому що у нас будуть неприємності. 
- Серйозно? Від кого? Від влади? Так ми їм не скажемо.
- Ні. Од відвідувачів, - каже офіціант. – Знаєте, різні люди приходять. Вони цього не потерплять. У нас уже були прецеденти, тому вибачте, російську пісню ми вам не увімкнемо. Ніяк… 
Смакую оте «відвідувачі цього не потерплять». 
- А ми їх попросимо, - вмовляю я офіціанта, випробовуючи його мовну стійкість.
- Я згадав, була заборона від влади, - каже офіціант, у якого з мовною стійкістю все гаразд.
- Яка заборона? Якої влади? Місцевої чи центральної? – перепитую, ніби не знаю про рекомендаційну Ухвалу (а не заборону) Львівської  обласної Ради, яку запропонували свободівці і обласна рада підтримала не «крутити» російську попсу в публічних місцях. 
- Ой, я не знаю, яка це була влада, – каже офіціант, - але щось таке чув. Тому ми не можемо, вибачте…
От для чого місцевим була потрібна та ухвала, щоб відмахуватися нею, як останнім спасінням від приставучих московитян.  Тобто, якщо в Києві чи інших зросійщених містах не те що ухвалу не празнують, а й закон, обов’язковий для всіх, то в Стрию, здається, всі наче чекали хоч якогось рекомендаційного «папірця», щоб прикритися ним від особливих надокучливих. Вибачте, ні, не можемо увімкнути. В очі ухвали не бачили, але чули, що вона є, тому вибачте, жодної російської пісні. За російськими піснями котіться в Київ.
Прекрасні стрияни. Совєти, ще у свій час, щоб розчинити і зросійщити Стрий (як вони зросійщували всі українські міста), збудували в Стрию військове містечко і направили сюди жити 20 тисяч російських військових. Однак, стався безпрецедентний випадок в Україні, не російськомовні розчинили Стрий, а Стрий (скільки того Стрия) асимілював 20 тисяч російськомовних граждан, зробивши їх українськомовними українцями…  
Перший жовто-блакитний прапор в Україні був піднятий саме в Стрию. Зі Стрия родом Блаженнійший Святослав Шевчук. В Стрию вчився в гімназії і пластував Степанко Бандера. Зі Стрийщини відома родина Нижанківських. Це в Стрию створили «Маслосоюз» і стрийським маслом «завалили» Європу. У Стрию народився перший підрозділ Січових Стрільців (СС)… Стрий… Стрий… Зі Стрия… У Стрию… Цих захопливих розповідей стриян не переслухати. Ось воно, 4-те правило українського націоналіста: «Будь гордий з того, що ти є спадкоємцем боротьби за славу Володимирового Тризуба».
Стрияни вважають себе пупами землі - і це прекрасно, і це вражаюче на тлі нашої масової української  маловартісности. Якщо в Кривому Розі місцеві українчики зі скепсисом і надією в голосі питали в мене, приїжджої, що в них гарного в місті (і я розумію, чому питали, бо відчувають шалену нестачу українського і передоз чужорідного совка), а в Чернівцях українські екскурсоводи замовкають у присутності румунських екскурсоводів, бо ті поставили себе головними в Чернівцях і українське місто називають румунським, а в Дніпрі ніяк не можуть врятувати музей Яворницького від московитського переродження – то в Стрию кожен зустрічний годинами мені розповідатиме, чим особливий кожен метр квадратний їхнього УКРАЇНСЬКОГО краю, який внесок зробили стрияни для України і світу, і в чому стрияни перші. А щоб я не сумнівалася, керівник культури, пан Богдан Бойко, який усі дні сипав цитатами з творів українських письменників, вручив мені неймовірний довідник з 20 тисячами прізвищ стриян, які зробили вагомий внесок у розбудову України… Вдумайтеся лише, уклали багатотисячний довідник з іменами стрийців, які зробили вагомий внесок у РОЗБУДОВУ УКРАЇНИ. Ось воно, 10-те правило українського націоналіста в дії, яке прищеплюють стрийським дітям: «Змагатимеш до поширення сили, слави, багатства й простору Української Держави».
У мене склалося стійке враження, що Стрий – то взагалі пуп землі і бандерівська столиця. І це не я столична гостя (толку з тих висоток), а, навпаки, вони - з української столиці, а я приїхала до них у столицю з київської змосковщеної периферії… Я чітко розумію, як повинно бути в Києві, яке наповнення – як у Стрию. Як повинно бути в більшості міст України – як у Стрию. Але так не є, бо ми не стрияни, які «не потерплять…»
Дякую тобі, Боже, за цих упертих (в найкращому розумінні цього слова) переконаних українців і за неймовірну енергетичну підзарядку. Її б та на всю Україну.
------------
Дякую міському Голові Олегові Канівцю та заступникові з гуманітарних питань Грех Христині за запрошення. Думаю, і українці вдячні, які подумки разом зі мною побували у Вас у гостях.
------------
Читайте бігенько, поки за таку крамолу не забанили 


\ifcmt
  pic https://scontent-iad3-1.xx.fbcdn.net/v/t1.6435-9/187102373_4302339016464864_5218327536561529951_n.jpg?_nc_cat=101&ccb=1-3&_nc_sid=730e14&_nc_ohc=k53fOj5-fusAX9FnNei&_nc_ht=scontent-iad3-1.xx&oh=933a21720af1366357257e368d8cfbd5&oe=60CCBA8F

	pic https://scontent-iad3-1.xx.fbcdn.net/v/t1.6435-9/186518716_4302339466464819_8426205700523663396_n.jpg?_nc_cat=111&ccb=1-3&_nc_sid=730e14&_nc_ohc=cxWiPuBAXIYAX_P7sBG&_nc_ht=scontent-iad3-1.xx&oh=26b08dec31cb8b1ad64730bca9d914aa&oe=60CDDCE8

	pic https://scontent-iad3-1.xx.fbcdn.net/v/t1.6435-9/187688362_4302339703131462_8135032732188468558_n.jpg?_nc_cat=108&ccb=1-3&_nc_sid=730e14&_nc_ohc=Q42mWICyS9oAX8pqPZx&_nc_ht=scontent-iad3-1.xx&oh=f52529de1b7c34d46d13cfad769f6bee&oe=60CA6867

	pic https://scontent-iad3-1.xx.fbcdn.net/v/t1.6435-9/187428781_4302340536464712_5241314683700271203_n.jpg?_nc_cat=111&ccb=1-3&_nc_sid=730e14&_nc_ohc=EHwtU1JLLgUAX84l5m2&_nc_ht=scontent-iad3-1.xx&oh=2b97460e60baa024a087973919951df5&oe=60C9F92B

	pic https://scontent-iad3-1.xx.fbcdn.net/v/t1.6435-9/186522484_4302340683131364_3133748464520177100_n.jpg?_nc_cat=110&ccb=1-3&_nc_sid=730e14&_nc_ohc=8gF2uMVWGI4AX_cxoUv&_nc_ht=scontent-iad3-1.xx&oh=d07d05936960e2259e5b64742a6b1adc&oe=60CAC7D3

	pic https://scontent-iad3-1.xx.fbcdn.net/v/t1.6435-9/187955304_4302341343131298_3936257290888245371_n.jpg?_nc_cat=110&ccb=1-3&_nc_sid=730e14&_nc_ohc=b2Nb-eOZXqoAX-xt0g3&_nc_oc=AQnPbcUCL6nRJQvJB35XWhPG6NMCn6lak3nimveZToyA_qcZsT9EsFrHjhDawzFaTuY&_nc_ht=scontent-iad3-1.xx&oh=403081ea63985eca6d623aec8cfc32ed&oe=60CDD25D

	pic https://scontent-iad3-1.xx.fbcdn.net/v/t1.6435-9/186817295_4302341713131261_3047977204972993711_n.jpg?_nc_cat=111&ccb=1-3&_nc_sid=730e14&_nc_ohc=iS-MtBEtzmsAX8MeO2d&_nc_ht=scontent-iad3-1.xx&oh=129c189582678d8abb6c7834a7b99ff9&oe=60CDDF6C

	pic https://scontent-iad3-1.xx.fbcdn.net/v/t1.6435-9/187778740_4302341929797906_3409372384509023389_n.jpg?_nc_cat=107&ccb=1-3&_nc_sid=730e14&_nc_ohc=QnJ8rL8vfpgAX9pxbqL&_nc_ht=scontent-iad3-1.xx&oh=7aed774ab7dce2c346b2b1ce65bfc27c&oe=60CD88AC

	pic https://scontent-iad3-1.xx.fbcdn.net/v/t1.6435-9/186510153_4302342289797870_6383735110400607614_n.jpg?_nc_cat=102&ccb=1-3&_nc_sid=730e14&_nc_ohc=kqqHdhRioAEAX_98Xip&_nc_ht=scontent-iad3-1.xx&oh=1b9dd8ee9227d2721a1d04664599fd4f&oe=60CCC9FB
\fi

Олена Броварська

Слава Україні! І слава Богу, що Стрий був, є і завжди буде саме таким
націоналістичним, з нього починається кровообіг нашої нації! Немає тут суму!
Шана і честь! Це приклад для усієї України!

Володимир Бердянський

А я не вірю, що стрийскі діти не дивляться російських блогерів і не слухають
роспопсу. З усіма наслідками з цього.

Victory Turyanitza

Володимир Бердянський Ви чули передачі «Радіо Стрий - радіо нашого міста»?
«Наші слухачі вітають свого друга з днем народження піснею Віталія Козловського
..., та за умовами нашого радіо пісні російською мовою не звучать в його ефірі.
Тому вітаємо вашого друга іншою піснею улюбленого співака!» Це було ще 2010. І
Дзідзя вперше почула на цьому радіо!

Володимир Бердянський

Вікторія Туряниця тобто діти знають російську попсу не з радіо і не зі школи.
От і все. Третьосортне л-но перемагає, бо "модне". Сучасні мас-медіа
ефективніші у 1000 разів у русифікації, ніж сталінські концтабори.
Запропонувати щось натомість не можу, ідей не маю. Лише констатую.

Victory Turyanitza

Володимир Бердянський Так. Але не з радіо і не зі школи - і це вже неабищо! Та
інет і ТБ-помийка задньощелепні у переважній більшості. І потрібна комплексна
державна політика. А отут ми... у повному лайні... І загроза в ньому
потонути... що дуже страшно...

Таїса Поцілуйко

Які ж молодці там! Якби ж по всій Україні отак... Мрії вголос...

Лиска Лілія

Дякую пані Ларисо, що пишете нам такі бомбезні розповіді!!! Дуже вражена,бо
чомусь думала,що Стрий,звичайне містечко,як всі інші.Але ж молодці мешканці та
влада їхня,що так прекрасно зберегли своє Українське місто!!! Є з кого приклад
брати!!! Як каже мій чоловік:"о,то не все пропало!" І не пропаде! Україна
жила,жиє і буде жити вічно! І,оці стрийські школярики,приклад для українських
діточок!!!

Viktor Marchenko

Браво, пані Ларисо! Давно нічого подібного не читав. Гордий з того, що в нас в
Україні є хоч Стрий. Молодці!

Nina Popel

Як завжди, велика подяка вам пані Лариса! Ваші дописи виховують і дають
насолоду від чудового слова! Наснаги вам і СИЛЬНОГО ДУХУ!

Степан Шніцар

А Ви ще завітайте в Комарно. З комарнівських ткачів походить родина Коновальця.
Завітайте в Малу академію мистецтв, в місцеві церкви. Комарно - перше і єдине
місто, якому вдалося цілком легально здійснити перепоховання енкаведистського
цвинтаря з центру міста на старий польський. Потім комуняки внесли зміни в
закон і ця практика не набула поширення.

Ntina Ntoubrova

Залишилося в Стрию виховати талановитих україномовних макроекономістів та
фінансистів, інженерів та топ-менеджерів - і тоді дійсно щось в країні
зміниться. Бо знаннями народних пісень та православними молитвами країну не
нагодуєш і товари з високою додатковою вартістю не створиш.

А от наші західні українці, отримуючи класну економічну, інженерну, Ай-Ті,
управлінську та фінансову й політологічну освіту в Львівських вишах, чомусь
їдуть піднімати економіку будь-якої іншої країни (росії зокрема)... І їм все
класно вдається.. Там вони, засукавши рукави та показавши свої 32 гострі зуби,
не бояться з самого низу, з рівня імігранта (за замовчуванням - людини НЕ
першого сорту) підніматися чи не до головних міністерських посад. А в нашій
країні чомусь безпомічно пасують свій м,яч - безмізковим нахабним нащадкам та
викормишам комуняк та "братків". І де ота войовничість дівається?

Василь Коружак

Стрий - найкраще місто і це є правда.

Інакше можуть стверджувати лише люди, які не знають життя поза кордонами
Галичини та й в інших наших містах.

Надія Собко-Федорів

Моя дочка проживає у Києві і викладає у школі. Не правда що у Києві такі
русофоби. Різні люди і відношення різне до мови до національних символів. Хоча
є люди провокатори. Але я рада що Україна відновлюється оживає і хоч малими
кроками з бідами війною ми вже відзначаємо 30років незалежності України. І
головне Любити і поважати хоч яка не є але наша Україна.

Daniel Los

Илон Маск основывает первую колонию на Марсе, исследователи СRISPER
приблизились к излечению рака, а в Стрыю ходят в вышиванках, бьют в бубен и
поклоняются лысому дядьке. Ужас. Как хорошо что нам не по пути.

Олексій Устяк

Читав уважно текст , виявив декілька російських слів , які дуже мулять око : -
говорить ( розмовляти ), прекрасно ( чудово) , робочий день ( день праці ,
граждан ( громадян )...  Якого милого в Державній школі находяться класи
християнської етики ,чому так нагло порушується Конституція України , яка
наголошує на тому , що церква відокремлена від держави і не допустимо надавати
преференції одній релігійній організації у збиток іншим !!!!  Прошу свою
корупційну релігію вивчати в приватних закладах , а не вдержавному навчальному
закладі !!!  Ще одне питання ,а за кого голосували ваші священники в першому
турі президенських виборів , часом не за промосковських рошенівських регіоналів
,які рясно засівають усі церкви корупційним баблом вже більше 15 років ??!

Наталія Катерняк Трух

ДЯКУЮ Вам, пані Ларисо, за такий влучний, щемливий і правдивий допис про наш
Стрий. Так, ми такі, і як би тут тролики і ватніки не намагались спаплюжити і
принизити наш край, ми і надалі продовжуємо жити з нашою ПРАВДОЮ !!! Творимо,
працюємо, мислимо з нашим баченням життя,з нашим знанням історії , дуже часто
не з історичних книжок чи досліджень, а з розповідей наших бабусь і дідусів, це
передається з покоління в покоління.

І так, стрияни - унікальні!!! Нас знищувавали у тюрмах, вивозили на Сибір,
щедро годували совковими "цінностями", але Ми вистояли і Ми є, хоч і як би це
не дратувало московитів.

Слава Україні!!!

Микола Степаненко

Стрий без сумніву славне місто.Та я знав Стрий коли української мови почути
було там зась.Лише коли Мирон Лисишин з друзями добре випивали то лунали
українські пісні і мова.Та хвала стрийчанам що вони змогли так поставити це
питання і українська мова стала звучати всюди.Нажаль Стрий під московським
окупантом був лише 78 років а моя Кіровоградщині більше трьохсот років,де
навіть московський матюк вкорінився так що не викорчуєш.

Олег Щербан

Я теж дитиною не знав, що у Києві розмовляють російською, бо в той період
телебачення було українськомовним. І я з Стрийщини якщо що)
