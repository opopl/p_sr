% vim: keymap=russian-jcukenwin
%%beginhead 
 
%%file 26_12_2021.stz.edu.lnr.lgaki.1.prazdnik_deti_respubliki
%%parent 26_12_2021
 
%%url https://lgaki.info/novosti/darim-prazdnik-detyam-respubliki
 
%%author_id 
%%date 
 
%%tags 
%%title Дарим праздник детям Республики!
 
%%endhead 
\subsection{Дарим праздник детям Республики!}
\label{sec:26_12_2021.stz.edu.lnr.lgaki.1.prazdnik_deti_respubliki}

\Purl{https://lgaki.info/novosti/darim-prazdnik-detyam-respubliki}

Студенты Академии Матусовского уже провели целую серию новогодних торжеств для
луганской детворы.

\ii{26_12_2021.stz.edu.lnr.lgaki.1.prazdnik_deti_respubliki.pic.1}

Завершая ежегодную благотворительную акцию «Чародеи», в ходе которой студенты и
педагоги вуза собирают вещи, канцелярию и подарки для детворы, которой особенно
нужна поддержка и тепло любящих людей, представители Студенческого совета
съездили в гости к малышам в Луганскую республиканскую детскую туберкулезную
больницу. Сладкий подарок вручили каждому ребенку, а учреждению передали вещи и
канцелярию. Кроме того, подарили детям подарили фотосессию с Дедом Морозом и
Снегурочкой.

\ii{26_12_2021.stz.edu.lnr.lgaki.1.prazdnik_deti_respubliki.pic.2}

После поехали в Суходольскую специальную коррекционную школу-интернат, где тоже
передали деткам собранные нами вещи, канцелярию и сладкие подарки.

\ii{26_12_2021.stz.edu.lnr.lgaki.1.prazdnik_deti_respubliki.pic.3}

Студенты отделения культуры колледжа Академии Матусовского не отстают! Студенты
групп ОКТ-3, ОКК-3, ОКА-1 под руководством Галины Косачевой подарили настоящую
сказку младшим школьникам Луганского учебно-воспитательного объединения
«Спортивная академия «Заря» и Луганского учебно-воспитательного комплекса № 43
имени Е. Ф. Фролова. Также их выступление стало настоящим подарком для детей
сотрудников Управления ГИБДД МВД ЛНР.

\ii{26_12_2021.stz.edu.lnr.lgaki.1.prazdnik_deti_respubliki.pic.4}

Около шестисот учеников 6-9 классов из 25 школ Республики дистанционно стали
участниками новогодней интерактивной сказки, которую подготовили студенты 2-го
курса специализации «Театральное творчество» под руководством Ольги Куриловой.
А учащимся 8-9 классов луганской школы № 38 имени К. Е. Ворошилова и ребятам из
Общественной организации «Луганское Республиканское Общество Слепых» эту сказку
показали очно. Неповторимые, совершенно невероятные эмоции от общения испытали
и артисты, и их зрители.

Преподаватель Татьяна Ерина со студентами групп ОКТ-2, ОКНВ-2, ОКВН-1, ОКЭВ-2
провели свою праздничную акцию «Новогодний музей» для учеников 2-3 классов
луганской гимназии № 30 имени Н. Т. Фесенко» в Луганском краеведческом музее.

— Мы получили незабываемые эмоции от того, что подарили ребятам сказку, веру в
чудеса, – поделились своими впечатлениями участники мероприятий. – Было очень
весело, но самое приятное – видеть улыбки и счастливые лица наших маленьких
зрителей.

И это еще не конец!

Фото Студсовета и из личного архива студентов.
