% vim: keymap=russian-jcukenwin
%%beginhead 
 
%%file 28_12_2021.stz.edu.dnr.doneck.ntu.2.koncert_prokofjev
%%parent 28_12_2021
 
%%url http://donntu.org/news/id202112281245
 
%%author_id edu.dnr.doneck.ntu
%%date 
 
%%tags donbass,dnr,muzyka,kultura,doneck
%%title В ДонНТУ состоялся концерт симфонической музыки
 
%%endhead 
\subsection{В ДонНТУ состоялся концерт симфонической музыки}
\label{sec:28_12_2021.stz.edu.dnr.doneck.ntu.2.koncert_prokofjev}

\Purl{http://donntu.org/news/id202112281245}
\ifcmt
 author_begin
   author_id edu.dnr.doneck.ntu
 author_end
\fi

В Донецком национальном техническом университете 27 декабря прошел концерт
симфонической музыки, посвящённый 130-летию со дня рождения выдающегося
композитора, пианиста, дирижёра Сергея Прокофьева. В этот вечер слушателям
представилась возможность насладиться творчеством нашего известного земляка в
исполнении артистов Донецкой государственной музыкальной академии имени С. С.
Прокофьева (ДГМА), с которой ДонНТУ накануне своего 100-летия заключил договор
о сотрудничестве.

\ii{28_12_2021.stz.edu.dnr.doneck.ntu.2.koncert_prokofjev.pic.1}

Ведущая мероприятия, кандидат искусствоведения Мария Романец, рассказала об
этапах жизненного и творческого пути композитора. Он писал во всех современных
ему жанрах и создал собственный новаторский стиль. Многие его сочинения вошли в
сокровищницу мировой музыкальной культуры, а сам Прокофьев принадлежит к числу
наиболее значительных и репертуарных композиторов XX века.

\ii{28_12_2021.stz.edu.dnr.doneck.ntu.2.koncert_prokofjev.pic.2}

На мероприятии выступили артисты симфонического оркестра (дирижер – Владимир
Заводиленко), камерного хора (руководитель – Алиме Мурзаева), а также солист,
лауреат международных конкурсов Карен Торосов (фортепиано). В программу
концерта вошли марш из оперы «Любовь к трем апельсинам», первая часть симфонии
№ 1«Классической», концерт для фортепиано с оркестром № 1. Также прозвучали
произведения «Вставайте, люди русские!» и «Мертвое поле» из кантаты «Александр
Невский». В сопровождении симфонического оркестра их исполнили камерный хор и
студенты кафедры академического пения ДГМА, а также солистка Донецкого театра
оперы и балета им. А. Б. Соловьяненко, лауреат международных конкурсов Дарья
Терещенко.

\ii{28_12_2021.stz.edu.dnr.doneck.ntu.2.koncert_prokofjev.pic.3}


В завершение мероприятия ректор ДонНТУ А. Я. Аноприенко поблагодарил артистов
за прекрасный концерт и поздравил всех присутствующих с наступающими
праздниками, пожелав здоровья, добра, мира.
