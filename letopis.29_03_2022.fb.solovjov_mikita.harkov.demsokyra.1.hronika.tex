% vim: keymap=russian-jcukenwin
%%beginhead 
 
%%file 29_03_2022.fb.solovjov_mikita.harkov.demsokyra.1.hronika
%%parent 29_03_2022
 
%%url https://www.facebook.com/Mikita.Solovyov/posts/7324116977658652
 
%%author_id solovjov_mikita.harkov.demsokyra
%%date 
 
%%tags 
%%title Хроника Харькова, 29-е марта
 
%%endhead 
 
\subsection{Хроника Харькова, 29-е марта}
\label{sec:29_03_2022.fb.solovjov_mikita.harkov.demsokyra.1.hronika}
 
\Purl{https://www.facebook.com/Mikita.Solovyov/posts/7324116977658652}
\ifcmt
 author_begin
   author_id solovjov_mikita.harkov.demsokyra
 author_end
\fi

Хроника Харькова, 29-е марта. 

34-й день восьмилетней войны, идущей уже несколько веков. (c) Aleksandr Shulman

Обстрелы продолжаются. Примерно на вчерашнем уровне. Не так много как бывало,
но чувствительно. Опять в основном окраины города. Разрушения и пострадавшие
есть, кажется без жертв. 

Город продолжают активно убирать. Разбирают завалы на Сев. Салтовке, писали что
еще в нескольких районах. В центре просто планомерно чистят улицы. Латают сети
жизнеобеспечения. Ищут и устраняют места утечки из водопровода, надеются
восстановить нормальное давление в сети. 

\ii{29_03_2022.fb.solovjov_mikita.harkov.demsokyra.1.hronika.pic.1}

В магазинах постепенно восстанавливается ассортимент. То есть восстанавливается
ассортимент товаров ежедневного спроса. По другим позициям наоборот, подходят к
концу еще довоенные запасы. За счет открытия новых могазинов, очереди в
супермакетах постепенно становятся меньше. Все еще большие по мирным меркам в
большинстве, но уже не в часах. В большинстве районов при наличии денег покупка
еды перестала быть квестом. Проблема скорее в том, что цены выросли заметно, а
денег у людей все меньше, работы в городе почти нет. 

По области ситуация без особых изменений. То есть очень неровная. Довольно
много обстрелов. в том числе тяжелыми ракетами по Люботину. Стабильно тяжелая
ситуация в оккупированных населенных пунктах. Стабильно жуткая ситуация в
блокадном Изюме. В части оккупированных городов находятся коллаборанты из
местной власти. В этот раз на сотрудничество с оккупантами пошел Голова ОТГ
Балаклеи. К сожалению, не он первый. 

Вокруг Харькова ЗСУ продолжают зачищать мелкие пригороды, которые еще неделю
назад уверенно контролировали русские. Процесс идет последовательно, день ото
дня еще немного добавляется к зоне безопасности города. Но самые плотно занятые
пригороды (Циркуны, Тишки, Русская Лозовая) еще впереди. И важно понимать, что
на их зачистку уйдет не день и не неделя. Самые жаркие бои на территории
области на изюмском направлении, насколько можно судить по открытым источникам.
Там наши силы стараются помешать захвату Изюма, нарушать логистику. И я не
оставляю надежду, что в результате смогут снять блокаду. В ходе боев нашим
удалось захватить еще какое-то количество техники, в том числе еще что-то из
РЭБ.

Да, я вас наверняка уже достал со своими рассказами о кофейнях. так вот, ничто
не остановит идею, час которой настал! Сегодня нашел в городе открывшуюся пару
дней назад кофейню. Маленькую, но настоящую. Можно сесть внутри, можно сесть за
единственным столиком на улице. Вот прям-таки сесть и выпить сидя чашку кофе,
даже с пирожным. Сбоку от Большого гастронома на Пушкинской по Манизера. А так
как мой пунктик уже известен всем знакомым, то мне кидают анонсы.
Завтра-послезавтра открывается еще одна. А на днях даже пиццерия начнет
работать, правда пока в режиме доставки.  @igg{fbicon.smile} 

Вы можете решить, и достаточно обоснованно, что это у меня просто не все дома,
раз я столько внимания уделяю таким вещам. Но для меня это один из самых
наглядных признаков оживающего города. А для меня очень важно, чтобы город
вернулся от выживания к жизни. Пусть трудной, пусть опасной, но жизни. 

На мой взгляд, сейчас для скорейшего возвращения больше всего городу не хватает
работающего транспорта. Харьков слишком большой город, чтобы в нем можно было
как-то работать на расстоянии пешей доступности. И в первую очередь необходимо
запускать метро. Я уже писал сегодня, что метро как убежище на мой взгляд
сейчас работает больше в минус, чем в плюс. А в качестве основного транспорта
оно и по эффективности, и по безопасности на порядок важнее и лучше всего
остального. Если даже больше не будет из транспорта работать ничего, но
заработает метро, то город станет более-менее транспортно связным. И это даст
очень мощный толчок к возвращению жизни в город. 

Когда гулял, сделал сегодня фотку Шевченко. Не знаю почему, но мне она кажется
очень символичной для сегодняшнего Харькова. 

Харьков стоит!

Слава Украине! 

Низкий поклон нашим защитникам!

\textbf{\#ХроникаХарькова}

\ii{29_03_2022.fb.solovjov_mikita.harkov.demsokyra.1.hronika.cmt}
