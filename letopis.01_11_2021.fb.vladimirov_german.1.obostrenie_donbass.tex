% vim: keymap=russian-jcukenwin
%%beginhead 
 
%%file 01_11_2021.fb.vladimirov_german.1.obostrenie_donbass
%%parent 01_11_2021
 
%%url https://www.facebook.com/permalink.php?story_fbid=10223074291319181&id=1637795073
 
%%author_id vladimirov_german
%%date 
 
%%tags donbass,obostrenie,ukraina,vojna
%%title Очередное обострение на Донбассе
 
%%endhead 
 
\subsection{Очередное обострение на Донбассе}
\label{sec:01_11_2021.fb.vladimirov_german.1.obostrenie_donbass}
 
\Purl{https://www.facebook.com/permalink.php?story_fbid=10223074291319181&id=1637795073}
\ifcmt
 author_begin
   author_id vladimirov_german
 author_end
\fi

Вот уже неделю все обсуждают очередное обострение на Донбассе. Людям уже
пришлось свыкнуться с постоянными обстрелами, но занятие серой зоны и
применение Байрактара — события из ряда вон выходящие. Все в очередной раз
убедились, что ни российский паспорт, ни участие в российских выборах не
гарантирует никакой защиты и безопасности. Спрашивается, чему тогда все
радовались, когда начали выдавать паспорта? Погибнуть под обстрелом с паспортом
РФ в кармане приятнее, чем без него? Ради чего были эти пустые обещания, если в
действительности мы наблюдаем полное бездействие?

\ifcmt
  ig https://scontent-frt3-1.xx.fbcdn.net/v/t39.30808-6/250349469_10223074290919171_2272956268341894465_n.jpg?_nc_cat=102&ccb=1-5&_nc_sid=730e14&_nc_ohc=XP0futKdH2IAX8HnyLG&_nc_ht=scontent-frt3-1.xx&oh=b99342b7025037d0b57b7846d6f88d93&oe=619AA41E
  @width 0.4
  %@wrap \parpic[r]
  @wrap \InsertBoxR{0}
\fi

Люди выживают уже 7 лет в катастрофических условиях, при этом наблюдая, как
вражеская армия получает всё новые и новые виды оружия для их уничтожения. Так
может быть пора вспомнить лозунг «своих не бросаем», и дать этим людям хотя бы
уверенность в том, что завтра у них на пороге не окажутся нацисты из ВСУ? Или
Россия не имеет других ресурсов для защиты своего народа, кроме «глубокой
озабоченности»?

Мне стыдно перед жителями Донбасса за то, что они брошены теми, кому
доверились, на произвол судьбы. И пусть я лично не могу защитить их и
остановить войну, но со своей стороны буду делать всё, чтобы хоть как-то
облегчить их жизнь. Потому что это наши, русские люди. И потому что для меня
слова «своих не бросаем» — не лозунг, а принцип по жизни.

Автор: Александра Лыгина \url{https://vk.com/alygina}

\ii{01_11_2021.fb.vladimirov_german.1.obostrenie_donbass.cmt}
