% vim: keymap=russian-jcukenwin
%%beginhead 
 
%%file 25_04_2022.stz.pc.ua.dou.1.geroizm_pogljad_kapellana.3.zustrich_z_synom
%%parent 25_04_2022.stz.pc.ua.dou.1.geroizm_pogljad_kapellana
 
%%url 
 
%%author_id 
%%date 
 
%%tags 
%%title 
 
%%endhead 

\subsubsection{Зустріч з сином}
\label{sec:25_04_2022.stz.pc.ua.dou.1.geroizm_pogljad_kapellana.3.zustrich_z_synom}

Якщо дружина вже приїжджала до мене, то Габріеля я не бачив 50 днів. Утрьох ми
гарно провели час та навіть забрали нашого кота з Тернополя, який жив у наших
друзів з початку війни.

\ifcmt
  ig https://s.dou.ua/storage-files/d57Cqm5_1.jpeg
	@caption Я і Габріель у Львові наступного дня після ракетного удару по Вишневому
  @wrap center
  @width 0.6
\fi

У понеділок, 18 квітня, я відправив свою сім'ю назад до Польщі. А через 30
хвилин рашисти нанесли ракетний удар по Львову.

Я бачив і чув ці ракети та вибухи. Це був жах. Саме тоді загинули люди.

Хочу поділитись декількома історіями, думаю, вам буде цікаво.
