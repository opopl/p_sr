% vim: keymap=russian-jcukenwin
%%beginhead 
 
%%file 04_02_2020.fb.alenkrig.1.sharij_evrej
%%parent 04_02_2020
 
%%url https://www.facebook.com/permalink.php?story_fbid=1275632075979155&id=100005971240357
 
%%author 
%%author_id 
%%author_url 
 
%%tags 
%%title 
 
%%endhead 

\subsection{Евреи - Шарий Анатолий}
\url{https://www.facebook.com/permalink.php?story_fbid=1275632075979155&id=100005971240357}

И кто бы мог подумать, что молодой украинский парень мог написать такое....
Спасибо, Анатолий Шарий.  Шарий Анатолий Анатольевич (р.1978), украинский
журналист. Публицист.

\ifcmt
  pic https://scontent-ams4-1.xx.fbcdn.net/v/t1.6435-9/83918423_1275632045979158_933242010544898048_n.jpg?_nc_cat=107&ccb=1-3&_nc_sid=730e14&_nc_ohc=EQEfEZhH-60AX8GD9_s&_nc_ht=scontent-ams4-1.xx&oh=1ccde4fddc1fe3ca45753dc229dde548&oe=6090A5A8
  width 0.4
\fi

«Еврей». Какие мысли рождаются у вас в голове при этом слове? Никаких? Но
что-то щелкает в мозгу, да? Переключатель какой–то срабатывает. Дискомфортно
становится на секунду. При слове «белорус» ведь не происходит такого? Как и при
слове «молдаванин» и даже «цыган». А при этом волшебном слове девять украинцев
из десяти вздрогнут и поднимут глаза. Обычно обывателями это слово произносится
как-то шепотом, особенно если разговор идет о ком-то знакомом – «а ты знаешь,
ведь он (шепотом) ЕВРЕЙ”. Я вспоминаю свое детство. Не помню, где я первый раз
услышал (подслушал) про эту национальность, но однажды подошел к маме и спросил
ее: „Мама, а Брежнев – еврей?” Помню, как она на меня посмотрела. Шел 1981 год…
Потом мой брат, человек абсолютно не злой, кормил голубей и отгонял от хлебных
крошек воробьев. Отгонял со словами „ну, налетели, ЕВРЕИ”. Я полагаю, что у
многих людей отношение к людям этой национальности формируется еще с детства,
впитывается если не с молоком матери, то с манной кашей – точно.  Написано
очень много книг по „еврейскому вопросу”, из поколения в поколение передаются
рассказы про „хитрых евреев, мешающих нам жить”… Так что же это за нация такая
страшная? Зачем жить мешают, воздух отравляют? Кто они такие? Хотя, повторяю,
написано уже слишком много, а я хочу высказать свое предположение по поводу
того, почему мой любимый украинский народ так не любит прекрасный еврейский
народ. «И сотворил Бог человека». Противники божественного сотворения Земли
часто задают вопрос, способный, по их мнению, выбить почву из-под ног любого
верующего: „А откуда негры взялись, если Адам белым был?”. Ну, во–первых, я не
припоминаю описания цвета кожи Адама, а во–вторых, КТО СКАЗАЛ, ЧТО БОГ СОТВОРИЛ
ВСЕХ ЛЮДЕЙ? Хотите считать своими предками обезьян – считайте! А насчет евреев
в Библии сказано четко – народ Божий. Ну, это для тех, кто не считает Библию
просто занимательной книжицей. 

Откуда взялась эта чудовищная ненависть к евреям у моих братьев – православных?
От неграмотности? Или извечного поиска виновных в своем бедственном положении?
Или от глубоко, напрочь въевшегося ментального „хай і у них все буде погано,
якщо у мене погано?” Да, что есть, то есть – я лично ни одного бедного еврея не
видел. Умеют работать люди. „Та вони ж хитрі які!” А в чем, собственно,
хитрость? В том, чтобы не пропить скопом заработанные деньги? Не упасть,
пьяному, лицом в грязь на улице, а отнести деньги семье? В том ли эта
сакраментальная „хитрость” заключается, что еврей не пройдет мимо заработка,
посчитав, что ему не „по масти” этим заниматься?  В начале прошлого века все
больше евреев, которых страна, носившая в то время название Российская Империя,
ненавидела лютой ненавистью, всячески ограничивала в правах и убивала,
устраивая погромы, решались на то, чтобы плыть в Америку. Зачастую это
„путешествие” занимало 14-17 дней. Чаще всего – в нижнем отделении корабля,
находившемся ниже уровня воды и неотапливаемом. С детьми. В сырости и холоде.
При перезагруженности корабля в 2-3 раза. С узелками, бросив все накопленное на
растерзание героев – казаков, все «геройство» которых сводилось к тому, чтобы
опьяненными безнаказанностью и всячески поддерживаемыми властью на всех уровнях
разорять и убивать ни в чем не повинных беззащитных людей, евреи плыли к мечте.
Те, кто уехали раньше, в редких письмах домой описывали страну, в которой к ним
относились как к людям, а не как к скотам… Так вот, те, кто выдержал плавание и
попал на берег Америки, поселялись в гетто. Жили по три семьи в одной крошечной
комнате. И работали. Много и тяжело. Это поколение подарило Америке множество
врачей, прекрасных юристов, бизнесменов и ученых. Вопреки логике, ВСЕ жители
еврейского гетто впоследствии становились на ноги и находили себя в жизни.
Единственное сравнение, которое приходит на ум – „птица-феникс”… 

Что еще так не нравится моему прекрасному народу в этой нации? То, как у евреев
тесны родственные и вообще семейные связи? Ну да, оставить ребенка в роддоме
для еврейской мамы, как и для папы, представляется невозможным, абсолютно
безумным поступком. В еврейских семьях почитают старших, супруги как ни в какой
еще нации сильно привязаны друг к другу и к детям. И если ребенок – любимое
чадо для любого папы и мамы, в еврейских семьях ребенок – просто неоценимое
сокровище! Здесь следят за каждой “стрункой” этого маленького человечка, за
каждым его шагом. А он – этот человечек – когда вырастет, повзрослеет, создаст
свою семью – будет так же точно любить свое чадо. И это многим непонятно. Ну
да, дикость в наше время… 

Насчет „евреев–обманщиков” и „нечестных евреев”. Во все времена жили евреи,
шедшие путем правды. Вот почему еврей, по словам А. Горького, “всегда был тем
маяком, на котором гордо и высоко разгорался над всем миром неослабный протест
против всего грязного, всего низкого в человеческой жизни, против грубых актов
насилия человека над человеком, против отвратительной пошлости и духовного
невежества”. “И как же, в самом деле, этот могучий голос любви к истине мог не
возбуждать ненависти тех, кто строит себе хоромы из лжи и обмана на фундаменте
насилия и преступления?” – резонно спрашивает А. Горький в статье “Еврейский
вопрос”. Злы ли евреи? Кто так считает, пусть попробует показать им, что он их
друг. Настоящий друг. Результат, уверен, не заставит себя долго ждать. Той,
прежней, осторожности как ни бывало. Оказывается, что евреи – это добрые и
отзывчивые люди. 

А откуда же это недоверие в общении? Таков результат многолетних скитаний по
миру, недоброго, а порой и чудовищно жестокого обращения с ними. Сколько их
обижали (и это ох как мягко сказано!) христиане, которые, согласно Библии,
должны были нести в мир только любовь! И вообще, как можно было вот так
относиться к евреям, ведь первые христиане, и главное, сам Иисус (!) были
евреями?!! Пойду, скажу какому–нибудь «святому отцу – батюшке» в канун
Рождества, так еще и по шее получу… 

В одном классе со мной учились два паренька-еврея. Я их обижал и задевал.
Нормальные были ребята. А я „цеплял” их по любому поводу и без повода. Просто
потому, что евреи. И Максим, и Глебка вот уже как 14 лет живут в Калифорнии. Я
пытался с ними связаться неоднократно. Чтобы извиниться. За свое свинство и
глупость. Кстати, к слову о „жадных евреях”. У Глебки дома жила целая свора
беспородных собак. Он, как и его родители, не мог пройти на улице мимо
скулящего комочка, обреченного на голодную или холодную смерть. Так вот они,
эти самые „жадные евреи”, оплатили далеко не дешевый перелет всех семерых
песиков. Очень дорогой перелет. Мы все тогда посчитали это безумием… „Еврей” –
в переводе с арамейского „освобожденный”. То есть люди, сбросившие с себя иго
чьего–то там гнета (египетского или узбекского – не имеет значения), могут
именоваться евреями. Я сбросил с себя тяжеленное ярмо нелюбви к этой нации.
Таким образом, я – еврей, да еще какой! Я долго копался в себе, очень много
общался с евреями, пытаясь понять их. Я их не понял. Но полюбил всем сердцем! Я
люблю эту нацию. Люблю за неиссякаемую энергию, кипящую в ней, вечную жизненную
силу, способность находить выход даже из самых-самых тупиковых ситуаций. Евреи,
как яблоня: какие бы ни были непогоды, она каждый год дарит нам спелые плоды.
Как солнце, которое согревает нас своими лучами. Как маленький ручеек, который
все бежит и бежит куда-то… 

Я люблю поэзию этого народа, восхищаюсь его остроумием, часами могу слушать его
песни, прекрасные, то безудержно веселые, то бесконечно грустные… Я каюсь в
том, что очень сильно заблуждался, я каюсь за свою ограниченность в набившем
оскомину „еврейском вопросе”, я прошу прощения за притеснения, хамство, а порой
за элементарное жлобство своего народа по отношению к прекрасному народу –

ЕВРЕЯМ. Шалом вам, братья. Мазл тов, дорогие!
