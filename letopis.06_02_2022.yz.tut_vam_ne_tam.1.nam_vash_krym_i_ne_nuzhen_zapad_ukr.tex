% vim: keymap=russian-jcukenwin
%%beginhead 
 
%%file 06_02_2022.yz.tut_vam_ne_tam.1.nam_vash_krym_i_ne_nuzhen_zapad_ukr
%%parent 06_02_2022
 
%%url https://zen.yandex.ru/media/tut_vam_ne_tam/da-nam-vash-krym-i-ne-nujen-poznakomilis-s-parniami-iz-zapadnoi-ukrainy-61ff86decceb253faca2801c
 
%%author_id yz.tut_vam_ne_tam
%%date 
 
%%tags krym,obschenie,rossia,ukraina
%%title «Да нам ваш Крым и не нужен» - познакомились с парнями из Западной Украины
 
%%endhead 
 
\subsection{«Да нам ваш Крым и не нужен» - познакомились с парнями из Западной Украины}
\label{sec:06_02_2022.yz.tut_vam_ne_tam.1.nam_vash_krym_i_ne_nuzhen_zapad_ukr}
 
\Purl{https://zen.yandex.ru/media/tut_vam_ne_tam/da-nam-vash-krym-i-ne-nujen-poznakomilis-s-parniami-iz-zapadnoi-ukrainy-61ff86decceb253faca2801c}
\ifcmt
 author_begin
   author_id yz.tut_vam_ne_tam
 author_end
\fi

Как изменился наш разговор, когда они узнали, что мы с подругой из Крыма?

- Если я поеду в Крым, у меня будут проблемы. Это не патриотично. 

- Почему? Разве вы не считаете его своим? 

- Он наш, но не подконтрольный. А значит, ехать туда опасно. Кроме того, наши
люди такого поступка не поймут. Ехать в Крым? Оставлять там деньги? А потом они
пойдут на нужды русской армии!  Ни за что не поеду туда, пока он российский. Вы
вообще читали историю? Если бы читали, не спорили бы сейчас.

Мы познакомились с этими ребятами на Шри-Ланке, и пока они не узнали, откуда мы
приехали, общались мы вполне нормально. Но разговор определённо стал более
интересным с этого момента.

\ii{06_02_2022.yz.tut_vam_ne_tam.1.nam_vash_krym_i_ne_nuzhen_zapad_ukr.pic.1}

Один из них считал, что Крым непременно вернётся, главное чтобы «было
нормальное руководство, а не Зеленский с его шайкой». Другой заявил, что Крым
не нужен Украине.

«Зачем он нам? Наши люди говорят, что он нам и не нужен. Одни проблемы для
государственного бюджета, Крым же такой неразвитый, столько вкладывать в него
денег надо.

Я сказала ему, что многое изменилось и стало лучше. 

-Это не только мое мнение, многие крымчане считают, что при Украине в
полуостров ничего не вкладывали, не было никаких финансовых вложений, и Крым
был в очень плохом состоянии на момент референдума.

Ребята захотели узнать, что же изменилось с тех времён.

-Постоянно слышим, что у вас теперь одни проблемы. Недавно смешной репортаж
видел, в Севастополе открыли фонтан, а воды в нем нет, из-за нехватки. Вот
глупое руководство! Деньги на фонтан промотали, а воды в Крыму ведь нет! Так и
стоит этот фонтан, ну и позор.

Я сказала украинцу, что о таком не слышала, и что сейчас водохранилища
наполнены.

-Ну, так сейчас зима! А вот летом что у вас творится! И кому нужен такой отдых?
Ещё и дорого все у вас, пусть россияне и отдыхают. А у нас вариантов хватает,
куда полететь отдыхать, после Шри-Ланки едем в Тайланд. 

Мне хотелось объяснить, что в Крыму видны изменения, показала фотографии дорог,
моста, рассказала об открытии больниц и прочих объектов. 

—Я знаю, что у вас показывают, как разрушается Таврида, сыпется мост, как не
хватает людям воды и что никто не едет к нам. 

-Так и у вас всякий бред про нас показывают. Особенно меня разозлил репортаж
Соловьева о том, что мы, украинцы «Чёрное море лопатой вырыли». Ну и бред! А
насчёт того, что у вас стало лучше, так ведь и у нас стало! Приезжайте
посмотрите, как села в Западной Украине изменились! Дороги строят в Украине,
при руководстве Порошенко стали выделяться финансы на развитие поселков. Хоть
он нам и не нравится, выбирать было не из кого, он лучше этого нынешнего
клоуна.

\ii{06_02_2022.yz.tut_vam_ne_tam.1.nam_vash_krym_i_ne_nuzhen_zapad_ukr.pic.2}

Я спросила, верят ли люди обещаниям Зеленского вернуть Крым, на что ребята
ответили, что он должен так говорить, иначе его просто уберут.

«Да как он вернёт ? Надо было раньше об этом думать! Но мы бы очень хотели,
чтобы в Крыму провели честный референдум, а не тот, что был».

Мы старались меньше поднимать тему политики, так как поехали вместе на сафари
на пол дня, и ругаться не входило в наши планы, но ребята сами возвращались к
этой теме вновь и вновь.

«Что? У вас даже Макдональдса нет в Крыму? И эпл пэй не работает? Ну и дикари,
понятное дело, какое развитие может быть на оккупированой территории» - любая
тема о Крыме переходила в политику.

Как выяснилось, они были ещё и ярыми сторонниками использования
«государственного языка», расклеивают плакаты по городу, работают в какой-то
общественной организации добровольцами, за идею.

-Только прошу тебя, не говори на украинском, звучит слишком грубо! Наш язык
милозвучный, нежный!

Мы пытались иногда говорить на украинском, но им это также не нравилось, из-за
частых ошибок.

«Вот был бы Крым нашим, вы бы сейчас все говорили на украинском правильно» -
постоянно говорили украинцы.

