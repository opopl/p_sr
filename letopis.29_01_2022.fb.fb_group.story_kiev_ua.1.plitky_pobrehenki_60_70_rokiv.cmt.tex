% vim: keymap=russian-jcukenwin
%%beginhead 
 
%%file 29_01_2022.fb.fb_group.story_kiev_ua.1.plitky_pobrehenki_60_70_rokiv.cmt
%%parent 29_01_2022.fb.fb_group.story_kiev_ua.1.plitky_pobrehenki_60_70_rokiv
 
%%url 
 
%%author_id 
%%date 
 
%%tags 
%%title 
 
%%endhead 
\zzSecCmt

\begin{itemize} % {
\iusr{Іринка Іринка}

Єдине що не чула, так то про \enquote{мать Молотова} і про Швецію!
@igg{fbicon.beaming.face.smiling.eyes}{repeat=4} 

\iusr{Elena Kravchenko}

О могиле матери Молотова говорила мама, там похоронены ее свекровь и она часто
проходила мимо знаменитой могилы


\iusr{Катя Бекренева}
Из всего слышали только о королеве и Приме)))

\iusr{Максим Коровниченко}

А я в школе искал красную пленку: фотографируя на неё, по заверениям старших
друзей, люди получались на фото голыми. У нас с приятелем был список
одноклассниц, которым нужно было обязательно устроить красную фотосессию)

\iusr{Савченко Жанна}

На початку 80-х обговорювали приїзд Пугачової. Десь в Гідрорарку її ледве не
задушив натовп, а вона зняла свої труси і кинула прихильникам.

А ще роздавала свій одяг і особисті речі зі своєі сумочки, - комусь дісталась
помада ( повезло!). Ну, гола вернулась додому.

Може правда?!


\iusr{Лариса Захарченко}
Кроме Молотова всё слышала

\iusr{Валентин Ореховский}
И японцы давали 10-Лямов за разрешение очистит русло Днепра, но все что
найдут-их.

\begin{itemize} % {
\iusr{Тетяна Слись}
\textbf{Валентин Ореховский} не правда, це були китайці!

\iusr{Валентин Ореховский}
\textbf{Тетяна Слись} тогда китайцы в 60-70-х не были так высоко технологичны,как японцы,по этому в байках фигурировали именно японцы!

\iusr{Тетяна Слись}
\textbf{Валентин Ореховский} а спорим, китайцы!

\iusr{Валентин Ореховский}
\textbf{Тетяна Слись} да хоть удмурты!

\iusr{Таня Сидорова}
\textbf{Валентин Ореховский} Либіді

\iusr{Андрей Акуленко}
\textbf{Валентин Ореховский} И было в 16 веке;))
\end{itemize} % }

\iusr{Надежда Лабик}
А еще, сразу после войны, что ходил мужик с молотком и этим молотком убивал людей.

\iusr{Olena Skrypka}

Яка прєлєсть усі ці анєкдоти). На Нивках у 70-ті це теж емоційно обговорювали:
не тільки бабусі під парадними, а й школярики на перервах між затяжками
Malboro))).  @igg{fbicon.face.upside.down} 


\iusr{Тетяна Слись}

На ковбасному заводі вночі щури підїдають залишки м'яса. Вранці включається
м'ясорубка, і щури, що туди залізли, ідуть у фарш.

\begin{itemize} % {
\iusr{Jarek Yaroslav}
\textbf{Тетяна Слись} ну это к сожалению правда

\iusr{Тетяна Слись}
\textbf{Jarek Yaroslav} я так и подозревала, давно колбасу не ем.

\iusr{Ирина Лазарева}
А я в детстве пересказывала байку про пирожки с человеческим мясом, якобы ноготь где-то нашли
\end{itemize} % }

\iusr{Mak Poznyak}
за 60-70-ті не скажу, але згадалося, як у мої юні 90-ті, ми глузували з дівчат, переконуючи їх, що
помаду роблять з псячих прутнів)

\iusr{Наталия Калинина}
Не думала, що ти такий старий! @igg{fbicon.laugh.rolling.floor} 

\iusr{Елена Мельникова}
Директор м'ясокомбінату з'їдає на сніданок 1кг. сосисок виготовлених за особливим рецептом.
В інши вироби додають туалетний папір.

\iusr{Нина Кубанова}
\textbf{Елена Мельникова} Да, про туалетную бумагу в колбасе точно говорили))

\iusr{Тамара Ар}

На счёт побрехеньок не знаю, была школьницей, но то, что везде по школе, по
домам, в центре были развешаны лозунги \enquote{Хай живе радянська Україна!}, -
запомнилось очень хорошо!


\iusr{Maryna Chemerys}
Про матір Молотова чула з дитинства)

\iusr{Xenya Tazkaya}
Если рассказать политический анекдот, то родителей посадят. Остального не слышала.

\begin{itemize} % {
\iusr{Iryna Mozalevska}
\textbf{Xenya Tazkaya} и из комсомола выгонят

\iusr{Xenya Tazkaya}
\textbf{Iryna Mozalevska} , это само собой!  @igg{fbicon.smile} 
\end{itemize} % }

\iusr{Jarek Yaroslav}
киевская прима делается на фабрике английской компании «имперский табак» так что доля правды там есть )

\iusr{Savelyeva Olena}
А зачем об этом вспоминать?

\begin{itemize} % {
\iusr{Галина Литвиненко}
\textbf{Savelyeva Olena} И вообще, зачем эту ерунду писать и еще на странице нашей группы?

\iusr{Savelyeva Olena}
\textbf{Галина Литвиненко} 

Согласна абсолютно! Ещё странно, что все эти байки и сплетни выложил мужчина.
Это не сексизм, если что  @igg{fbicon.face.wink.tongue} ...

\iusr{Olena Chebaniuk}
\textbf{Savelyeva Olena} 

Є така наукова дисципліна - культурна антропологія, одне з її завдань як в
текстах усної комунікації відбивається картина світу різних прошарків населення
на певному історичному етапі. Культурною антропологією займаються вчені не
залежно від статті в унівеститетах та наукових інститутах в усьому світі. Гугл
вам в поміч.

\iusr{Savelyeva Olena}
\textbf{Olena Chebaniuk} Вот оно что! Ну в научных целях - другое дело...
\end{itemize} % }

\iusr{Алла Титаренко}

Ключи на шее на верёвочке надо прятать под одежду а то воры увидят ( прямо
взглядом сфотографируют) и обворуют квартиру  @igg{fbicon.face.tears.of.joy}  Обувь надо обувать только с
правой ноги а то не вырастешь  @igg{fbicon.face.tears.of.joy} 

\iusr{Оксана Макаренко}

В чорном-прєчорном городє, на чорной прєчорной уліце в чорном-прєчорном домє
єсть чорная-прєчорная комната, в якій стоїть чорний-прєчорний гроб! А в ньому
живе чорна-пречорна рука!..... ну а далі у кого як фантазія
розгулювалась)))))) @igg{fbicon.face.tears.of.joy} 

\iusr{Iryna Mozalevska}
Слышала только про пальчики в котлета, думаю, что это отголоски голодомора

\begin{itemize} % {
\iusr{Нина Кубанова}
\textbf{Iryna Mozalevska} Мясорубка не справилась с детской косточкой? ))

\iusr{Наталия Жминко Сычевска}
\textbf{Нина Кубанова} фу !
\end{itemize} % }

\iusr{Владимир Мареев}
Подтверждаю, про негра сифилитика сам слышал году в 1975

\iusr{Cheslav Kutcher}
были, были ттакие побрехеньки

\iusr{Yurii Kadochnikov}
Про приму та негра теж чув

\iusr{Нина Кубанова}
В случае со стаканом - верю

\iusr{Андрей Махов}
Вспоминается Юз Алешковский с рассказом \enquote{Кенгуру и Руру} ...

\iusr{Антонина Кольвах}
все слышала, кроме английской королевы с Примой ))

\iusr{Юрий Шевченко}
\enquote{Ой, что деется, вчерась траншею рыли,
Так откопали две коньячные струи...}

\iusr{Павел Пауль}

С докторской колбасой - это реально было. Киевская прима в коричневых пачках
ценилась, но чтоб королева... О таксисте слыхали. Ну а к черным студентам из
братских стран отношение особое было на нашей улице после того как парня
зарезали.

\iusr{Евгений Ступацкий}
О!!!!! Про сифілітіка то була дійсно мощна байка. Розповідалася у різних варіаціях! @igg{fbicon.face.tears.of.joy} 
\iusr{Сергей Сирош}
Про киевмеую приму самая старая байка, никто не интересовался курит ли королева...
Это сродни про Черчилля и армянский коньяк, и более того, про коньяк Шустов и его участие в Парижской выставке
Все брехня, но маркетологи работали

\iusr{Alexandr Bouuyhuk}
))) это из тех самых совковых легенд.... еще ссали в бочку с квасом, или кидали трупы

\iusr{Наталия Жминко Сычевска}
Первое - про Приму - правда .

\iusr{Сергій Батурин}
Я колись цілий роман написав про Київ 70-х, там був розділ \enquote{Легенди старого міста}. Переповідати не стану.
 @igg{fbicon.smile} 

\ifcmt
  ig https://scontent-mxp2-1.xx.fbcdn.net/v/t39.30808-6/s851x315/272850322_4796531030400334_708337661579999649_n.jpg?_nc_cat=103&ccb=1-5&_nc_sid=dbeb18&_nc_ohc=afiHgxmwJD0AX83JRmg&tn=lCYVFeHcTIAFcAzi&_nc_ht=scontent-mxp2-1.xx&oh=00_AT8AnT4Vzh40gKcfsiYl-ZAapbfuvKw3fUKSi5uRcs3mtg&oe=61FA7E96
  @width 0.3
\fi

\iusr{Natalie Michael}

Правдива історія від самовидця: Ідуть двоє в тролейбусі з тортом \enquote{Вєтка}. Поруч
сидить бабця з київським тортом і куняє. Молодь тихенько замінила коробки.
Бабця прокидається і плаче: Де ж мій котик, ховати везла!


\iusr{Natalie Michael}

А ще про таксиста і труп. Пасажир сказав: скоро почнеться війна з Китаєм. А
сьогодні у машині буде труп. Труп дійсно з'явився, відповідно. всі чекали на
війну з Китаєм.

\iusr{Владимир Кулеба}

Як не дивно, більшість байок прочитав уперше! Хоч у Києві з 1958 року. І збираю
різні байки, переважно з футбольного життя. Ось дві, почуті ще в студентській
молодості на футбольній \enquote{брехалівці}, котра збиралася давним-давно біля таблиць
на стадіону ім Хрущова (нині \enquote{Олімпійський}, а згодом - на пл. Ленінського
комсомолу, нині Європейської. "Перед війною \enquote{Динамо} очолив ленінградець і
москвич Михайло Бутусов. Він грав у нападі і володів смертельним ударом з
правої ноги. Після того, як виконуючи пенальті, переламав бокову штангу, йому
заборонили бити правою ногою і виконувати одинадцятиметрові". На праву ногу
йому прикріпили чорну пов'язку, щоб не вбив кого."Ще одна: перший повоєнний
міжнародний матч \enquote{Динамо} відбувся в 1956 році зі збірною Індії. Індуси грали
босоніж, тільки капітан в обмотках. Він і забив киянам гол з пенальті. До цього
рахунок уже був 13:0 на користь \enquote{Динамо}. Незбагненно, але в воротах індусів
стояла мавпа, відбиваючи багато ударів. Першим, кому вдалося її \enquote{прошити} був
улюбленець киян Паша Віньковатов. Він був також знаменитий, тим що в будь-якому
стані з'їдав \enquote{Київський} торт".

\iusr{Петро Гарматюк}

Лиш одна поправка: не київську \enquote{Приму} курила англійська королева, а
ПРИЛУЦЬКУ.

\begin{itemize} % {
\iusr{Xenya Tazkaya}
\textbf{Петро Гарматюк}, і це була не королева, а Черчіль, і не курити, а пити, і не приму, а вірменський кон'як.

\iusr{Петро Гарматюк}
\textbf{Xenya Tazkaya}
Нехай по вашому, хоча могло бути й щось опосередковане: сиділи удвох бухали, а \enquote{табачек врозь}.
Так, трішки по паскудному, але ...!
\end{itemize} % }

\iusr{Грег Станлий}
и кто одобряет публиковать такой бред...

\iusr{Андрей Калашников}

Когда пошёл в школу \#25 (а она напротив Андреевской церкви) 81 год, там ходила
байка про попов с монахинями, которые затащили пионер(а)(ку) в подвал, сначала
пытали, а потом и убили @igg{fbicon.face.screaming.in.fear} 😨 Поэтому, не ходите в Андреевскую церковь @igg{fbicon.face.anxious.sweat}  @igg{fbicon.grin}  @igg{fbicon.face.tears.of.joy}  @igg{fbicon.laugh.rolling.floor} 


\iusr{Ольга Васильевна}
Наверное все прошли в детстве:...\enquote{черная, черная рука...} , интерьер зависил от фантазии говорящего)))

\iusr{Марина Даниленко}
Я теж чула про таксиста та негра)))) Років 40 тому

\iusr{Виктория Косцевич}

Помню еще две популярных киевских басни. Случай, как мальчик одел на голову
вазу и не мог снять, и история про пыжиковую шапку.

\iusr{Валентин Трусов}

Про негров было многое, теперешняя кремлевская подстилка актер Ярмольник, родом
изо Львова, когда то рассказывала, как местные \enquote{бендеровцы} повесили
негра-студента на березке... @igg{fbicon.face.tears.of.joy} 

\iusr{Анастасия Чанцова}
Супер! Дякую

\end{itemize} % }
