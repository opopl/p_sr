% vim: keymap=russian-jcukenwin
%%beginhead 
 
%%file 10_06_2021.fb.buzhanskii_max.1.nacizm_unikalnost
%%parent 10_06_2021
 
%%url https://www.facebook.com/permalink.php?story_fbid=1964882877009515&id=100004634650264
 
%%author 
%%author_id buzhanskii_max
%%author_url 
 
%%tags 
%%title А вот нацизм уникален. Уникален в том, что кроме врагов и рабов, у него есть ещё одна категория, те, кто ему не нужен.
 
%%endhead 
 
\subsection{А вот нацизм уникален. Уникален в том, что кроме врагов и рабов, у него есть ещё одна категория, те, кто ему не нужен.}
\label{sec:10_06_2021.fb.buzhanskii_max.1.nacizm_unikalnost}
\Purl{https://www.facebook.com/permalink.php?story_fbid=1964882877009515&id=100004634650264}
\ifcmt
 author_begin
   author_id buzhanskii_max
 author_end
\fi

Есть тысячи, десятки тысяч примеров того, что такое нацизм.

Вот не хочет человек знать, не интересуется, живёт театром, футболом,
инстаграмом, \enquote{не хочет ворошить прошлое}, как иногда говорят.

И никогда про нацизм не слышал.

А потом случайно попадётся на глаза заметка, пара строчек всего, пробежал
глазами и всё понял.

Если захотел.

Не узнал всё, история бесконечна и по горизонтали, охватившей весь земной шар и
по вертикали, спускающейся с уровня лидеров держав до счастья или трагедии
каждой отдельно взятой семьи, всего не узнаешь.

Не узнал, но понял.

Если захотел понять.

Нацизм то, он уникальный, его ни с чем не спутаешь, если один раз видел.

Лидице.

Крошечная деревушка в Чехии, 42й год.

Доблестные диверсанты убили в Праге Гейдриха, немцы не могли не ответить
репрессиями, но это и не шло в разрез с планом, сильней репрессии, больше
надежд на сопротивление чехов, такая чудовищная математика войны.

И немцы ответили.

Окружив 10 июня 1942 деревню Лидице, в связи с предположением о том, что там укрывают диверсантов.

Я не буду рассказывать про всё мужское население деревни старше 15 лет, 178
человек, расстрелянных на месте.

Про всех женщин, 195 человек, отправленных в концлагерь.

А вот из сотни детей, 13 годовалых младенцев оставили в живых для онемечивания,
а остальных, вместе с детьми из соседней деревни Лежаки, отправили в газовую
камеру.

Если вы это прочли, вы знаете про нацизм всё.

Дети не мужчины, они ничем не угрожают Рейху, их не расстреляют на месте.

Дети не взрослые женщины, они не могут работать, они не нужны.

Поэтому, нацисты их просто утилизировали.

Чудовищное слово, в структуре нацизма означающее третий вариант.

\begin{itemize}
  \item 1.  Опасные.
  \item 2.  Рабы.
  \item 3.  Ненужные.
\end{itemize}

История репрессий в истории человечества бесконечна.

Можно начать со сторонников политических партий Суллы и Мария и проскрипций,
продолжить Октавианом и Антонием, прийти к якобинцам Великой Французской
Революции, и коммунистам в Большой Террор.

Можно.

И дальше можно, просто мы ж просвещенные теперь, пролистал новость не глядя,
кофе остывает, зачем портить утро тем, что происходит рядом, но не с тобой?

А вот нацизм уникален.

Уникален в том, что кроме врагов и рабов, у него есть ещё одна категория, те, кто ему не нужен.

Просто занимает место, которое он считает своим.

И он, нацизм, это место освобождает, конвеерно.

Технологично.

Понимаете, глядя на ребёнка, он не видит ребёнка.

Он видит место, занято-свободно.

И всё, что ему нужно, это чтоб ему не мешали.

Освободит и займёт.
