% vim: keymap=russian-jcukenwin
%%beginhead 
 
%%file slova.putin
%%parent slova
 
%%url 
 
%%author 
%%author_id 
%%author_url 
 
%%tags 
%%title 
 
%%endhead 
\chapter{Путин}
\label{sec:slova.putin}

%-------------------
«Стало страшно жить до жути, он везде — \emph{Владимир Путин}», — смеется Lobsang R.,
\citTitle{«Он везде!» — «Никто не знает, где Путин появится завтра»}, Оксана Переможенко, regnum.ru, 06.06.2021
%-------------------

\emph{Путин} трясет \enquote{Ящик Пандоры}, но пока не открывает,
obozrevatel.com, 24.05.2021

Навіть попри те, що \emph{Путін} не почав нову фазу завоювань в Україні –
останній епізод кілька тижнів тому змусив людей у Вашингтоні і Києві
понервувати, radiosvoboda.org, 24.05.2021

В особі Лукашенка ми маємо справу з \emph{Путіним}, radiosvoboda.org, 25.05.2021

За його словами, нинішня ситуація є на руку російському президентові Володимиру
\emph{Путіну}, який має «набагато кращий шанс протиснути свій сценарій зміни
влади у Білорусі», а це несе набагато більші загрози для України,
radiosvoboda.org, 25.05.2021

На вашому YouTube каналі одне із рекордсменів за переглядами – відео \enquote{Чи
матюкались давні українці?}. Давайте розставимо крапки над і щодо лайок в
українській мові. Чи винен в цьому умовний колективний \enquote{путін}?
pravda.com.ua, 25.05.2021

Клутин - \emph{Путин}, вакцинация - эскалация. Байден забыл, как и что
называется, news24ua.com, 16.04.2021

\emph{Путін} застосовує «стратегію божевільного» – Дмитро Кулеба, radiosvoboda.org, 01.05.2021

Единственный пункт, который вызвал панику у придворного журналистского
профсоюза НСЖУ, — это право СБУ блокировать любые сайты в рамках
«информационных операций» против агентов \emph{Сами Знаете Кого}, \textbf{На
границе Ада. Появилась финальная версия законопроекта о реформе СБУ},
ukraina.ru, 26.05.2021

\emph{Путин} разорвал налоговое соглашение с Нидерландами, lenta.ru, 26.05.2021

Портрет \emph{Путіна} з вирваних зубів виставлений на аукціон, radiosvoboda.org,
27.05.2021

Росія – централізована автократія з ручним керуванням \emph{Путіна}, на нього
зав'язано все, radiosvoboda.org, 30.05.2021

Росія страждає від структурних проблем, які виходять за рамки особливостей
режиму \emph{Путіна}, і тому навіть з його смертю або відходом від влади
політичний курс в Росії може не змінитися, radiosvoboda.org, 30.05.2021

Як ще \emph{Путін} при цьому не застрелився – не знаю, це ж ціла система ППО!
Гаразд, годі сарказму. Лише дуже наївна людина може повірити, що в липні чи
серпні воювала якась «нова армія», не та, яка звільняла Красний Лиман. Як і в
те, що \emph{Путін} справді планував захопити Київ,
\textbf{П'ять цікавих фактів – від \emph{УПА} до АТО}, Павло Зуб'юк, zaxid.net, 01.06.2021


Истеричный лукашизм, или Подарок для \emph{Путина}, Во избежание спекуляций как
с той, так и с другой стороны автор сразу же отметит, что, имея белорусские
корни, а также придерживаясь старой доброй советской традиции, с искренней
симпатией относится к сябрам, причем прежде всего именно к Белоруссии и к
близкородственным нам, выросшим с нами из одного древнерусского и советского
корня братьям-белорусам. Поэтому вся эта антибелорусская, совершенно идиотская,
скотская, подонковская, гнусная, ублюдочная (далее уже только нецензурные
выражения!) истерия, которая развернута в Украине в последние дни в исполнении
разного рода \enquote{актывиздов} и Зе-власти, вызывает рвотный рефлекс и классовую
ненависть,
\textbf{Европа впала в иррациональный истеричный лукашизм}, Александр Карпец,
strana.ua, 03.06.2021

Причем наиболее вероятен именно второй вариант, поскольку \emph{Путин} опыт
Украины, надо полагать, учел, и Беларусь просто так не отдаст. А если так, то
наиболее лакомые куски белорусской экономики пойдут под дерибан околопутинским
олигархам, а остальное будет уничтожено за ненадобностью. Этим
\enquote{кремлевский вариант} кончины белорусской государственности отличается
от \enquote{условно демократического}, при котором будет просто уничтожено все,
за очень редким исключением, вроде добычи калия,
\textbf{Европа впала в иррациональный истеричный лукашизм}, Александр Карпец,
strana.ua, 03.06.2021

Посол України в Німеччині: \enquote{Божевільна ціль \emph{Путіна} – знищити Україну},
radiosvoboda.org, 03.06.2021

Його план (\emph{Путіна} - ред.), як і раніше, полягає у знищенні України як
незалежної держави. Це і є його божевільна ціль. Прозахідний та демократичний
розвиток моєї батьківщини для \emph{Путіна} – це наче більмо на оці, тому що це
загрожує його тоталітарному та корупційному режиму..., 
\textbf{Посол України в Німеччині: \enquote{Божевільна ціль \emph{Путіна} – знищити Україну}},
radiosvoboda.org, 03.06.2021

Допускаю, что вскоре мы станем свидетелями душераздирающего флешмоба «не дай
белорусу!» Вероятно, найдутся желающие покидать яйцами и зеленкой в белорусское
посольство.  Возможно, украинские радикалы даже песенку про Лукашенко споют. Ту
самую, которую они поют про \emph{Путина}. Не знаю, станет ли Кулеба (по
примеру Дещицы) подпевать, но если будет — то ничуть не удивлюсь. Потому как
подобные напевы очень повышают концентрацию гидности и положительно влияют на
чувство собственного величия, которое в такие моменты приобретает буквально
вселенские размеры,
\citTitle{Украина как унтер-офицерская вдова - Голос Севастополя - новости
Новороссии, ситуация на Украине сегодня}, , voicesevas.ru, 06.06.2021

В сети оценили заявление бывшего украинского президента, а ныне депутата
украинского парламента Петра Порошенко. По мнению украинского политика, семь
лет назад не Украина развязала гражданскую войну в Донбассе, а против Украины
якобы началась «крайняя волна российской агрессии». «Никто не знает, где \emph{Путин}
появится завтра», — отметил Порошенко,
\citTitle{«Он везде!» — «Никто не знает, где Путин появится завтра»}, Оксана Переможенко, regnum.ru, 06.06.2021

«Зачем он из \emph{Путина} делает суперчеловека? Такое впечатление, что он с \emph{Путиным}
ложиться и встаёт», — написала Natalia N.  «Могу сказать, в России будет
завтра», — отреагировал Александр К.  «А как же украинская разведка? Лучшая
служба в мире, которая ловит шпионов с фотоаппаратами выпуска 70-х годов
прошлого века? Разве они тоже бессильны? Ай да «Темнейший!» — пишет Валерий Р,
\citTitle{«Он везде!» — «Никто не знает, где Путин появится завтра»}, Оксана Переможенко, regnum.ru, 06.06.2021

«А че сразу \emph{Путин}? Он один в России, что ли? Мы все только и ждем приказа,
чтобы пойти и сеять разумное, доброе и вечное среди укронацистов и
общечеловеков! А то что-то подзабыли, кто в мире хозяин!» — подытожил Вадим К,
\citTitle{«Он везде!» — «Никто не знает, где Путин появится завтра»}, Оксана Переможенко, regnum.ru, 06.06.2021

«Comedy Club, видимо, посмотрел, болезный... Там МиГ с Путиным за секунду из
Москвы до Вашингтона долетает...» — добавил Олег П.,
\citTitle{«Он везде!» — «Никто не знает, где \emph{Путин} появится завтра»}, Оксана Переможенко, regnum.ru, 06.06.2021

«Стало страшно жить до жути, он везде — \emph{Владимир Путин}», — смеется Lobsang R.,
\citTitle{«Он везде!» — «Никто не знает, где Путин появится завтра»}, Оксана Переможенко, regnum.ru, 06.06.2021

Даже в пьяном дебоше солдат НАТО они видят руку \emph{Путина}.  В Эстонии крепко
пьяные НАТОвские солдаты устроили дебош и в агрессивной форме унижали местных
жителей.  В ответ на данную ситуацию «британские эксперты» на полном серьезе
заявили, что поют алкоголем НАТОвских вояк агенты Путина для дискредитации
военно-политического блока НАТО))) Видимо их покусали «украинские эксперты»,
ведь симптомы налицо, нужна срочная госпитализация в психдиспансер и курс
лечения от русофобии))),
\citTitle{Британские эксперты обвинили Путина в пьяном дебоше НАТОвских солдат}, Дмитрий Василец, strana.ua, 09.06.2021

%%%cit
%%%cit_pic
%%%cit_text
Перед встречей с Байденом \emph{Путин} обозначил \enquote{красные линии} по
Украине. Их всего две: Украина (не) в НАТО и нацизм в Украине.  Во-первых, НАТО
у своих границ Россия не приемлет. С одной стороны, это сигнал НАТО: не надо
давать надежду Украине, а с другой — Украине: не надо заводить противников
России к её границам и нагнетать конфронтацию.  Во-вторых, это констатация, что
\enquote{война с Россией} в Украине – это, по своей сути, война с русскими:
людьми, культурой, языком. То есть власть Украины воюет не с Россией, а с
собственными гражданами. Суть — нацизм.  Казалось бы, это и без заявления
\emph{Путина} очевидные факты. Но, к сожалению, не для всех. И путь к миру в
Украине и с Россией один: Украина без НАТО и без нацизма. Этого же хотят и
более 50\% граждан Украины
%%%cit_title
\citTitle{Путин обозначил Байдену красные линии по Украине}, Александр Скубченко, strana.ua, 11.06.2021
%%%endcit

%%%cit
%%%cit_pic
%%%cit_text
\enquote{Журналист ему задал вопрос: \enquote{Вы убийца?} Тот самый, на который
Байден (применительно к \emph{Путину}) ответил утвердительно. \emph{Путин} сказал, что
подобные обвинения он уже слышал много раз, \enquote{особенно во время борьбы с
терроризмом на Северном Кавказе}. Но эти ярлыки, по мнению \emph{Путина}, не то, из-за
чего можно \enquote{даже слегка} переживать}, - говорится в сообщении.  При
этом \emph{Путин} утверждает, что все время \enquote{руководствовался интересами
российского государства и русского народа}.  Также журналисты перечислили перед
\emph{Путиным} имена людей, смерти которых связывают с работами российских спецслужб:
Анна Политковская, Александр Литвиненко, Сергей Магницкий, Борис Немцов, Михаил
Лесин.  В ответ на это \emph{президент РФ} заявил, что вопрос похож на
\enquote{несварение желудка. Только словесное}. При этом он уточнил, что
упомянутые люди погибли в разное время, по разным причинам и от рук разных
людей
%%%cit_title
\citTitle{Путин убийца? Президент России ответил на скандальный вопрос в интервью NBC}, Карина Вольтер, strana.ua, 12.06.2021
%%%endcit

%%%cit
%%%cit_pic
%%%cit_text
Москва основана Русскими, киев город Русский, Минск освобожден от немцев и
прочих Русскими. Все кто с нами - друзья. Все, кто против нас - враги.  За нами
- правда, сила и будущее. Только так! Кто против \emph{Путина} - мой враг
%%%cit_comment
Деревенский
%%%cit_title
\citTitle{Кто основал Москву? Вы даже удивитесь - ее основал великий киевский князь в XII веке и заселил ее украинцами}, 
Исторический Понедельник, zen.yandex.ru, 18.02.2021
%%%endcit

%%%cit
%%%cit_pic
\ifcmt
  pic https://avatars.mds.yandex.net/get-zen_doc/4337106/pub_60ab080d706b5c5631d9337f_60ab1efed750f6113afa3544/scale_1200
	caption Главноечтобынепутин
\fi
%%%cit_text
Давайте представим что его мольбы услышаны и пришёл наконец вместо ненавистного
\emph{Путина} Главноечтобынепутин. Жизнь сразу наладилась, у моего знакомого очень
быстро пропал второй подбородок, через год он поймет что надо учиться
выращивать картошку, а то и первый с голодухи отвалится. Но так как он вечно
всем недоволен, то и как раньше на каждом углу и во всех соцсетях визжит как
ему плохо и какая власть вороватая. Так вот, очень быстро за ним приехал ночью
воронок и упаковал Серёгу далеко и на долго, чтобы не орал всем очевидные вещи
что жить стало РЕАЛЬНО голодно. Вот идет он в наручниках и улыбается, ведь
самое что сейчас при власти \emph{Главноечтобынепутин}
%%%cit_title
\citTitle{Главноечтобынепутин}, Мак Сим, zen.yandex.ru, 25.05.2021
%%%endcit

%%%cit
%%%cit_pic
%%%cit_text
Чи може \emph{Путін} відмовитися від своїх поглядів? Звичайно ж, ні. Єдине, від чого
він може відмовитися – так це від втілення цих поглядів у життя. Але статися це
може тільки під тиском цивілізованого світу. Тільки в тому випадку, якщо ціна
агресії проти України виявиться для російського президента непідйомною. І саме
для того, щоб виключити саму можливість російської агресії проти України, є
такою важливою співпраця України зі Сполученими Штатами і країнами
Європейського Союзу та зміцнення обороноздатності країни.  Тому що,
врешті-решт, важливим є зовсім не те, що напише у своїй статті
публіцист-початківець \emph{Володимир Путін}. Важливим є те, які накази віддасть
російським військам і спецслужбам президент \emph{Володимир Путін}
%%%cit_comment
%%%cit_title
\citTitle{Путін знову береться за історію України}, 
Віталій Портников, gazeta.ua, 14.06.2021
%%%endcit

%%%cit
%%%cit_head
%%%cit_pic
%%%cit_text
\emph{Владімір Путін} успішно зіграв на почуттях про велич російського народу, який
ніхто не завойовував і не поставив на коліна. Затуманив голову нісенітницями
про відновлення імперії, про великодержавну місію. І все заради того, щоб
відсунути в часі крах своєї нелюдської і грабіжницької системи. Кинув мільярди
на оболванюючу пропаганду. Розпочав деструктивну боротьбу зі всім нормальним
світом. Анексії, агресії, війни, кібератаки, корумпування світових
топ-політиків, фінансування терористичних режимів і рухів. Після такого
переліку російських «цінностей» стає однозначно зрозуміло, що \emph{Путін} рятує свій
клептократичний заповідник шляхом світової деструкції. Зрозуміло також, що все
це довго тривати не буде. Але попити крові всім нормальним людям навколо він
зможе
%%%cit_comment
%%%cit_title
\citTitle{Подвійний ворог України - ZAXID.NET}, 
Василь Расевич, zaxid.net, 18.06.2021
%%%endcit

%%%cit
%%%cit_head
%%%cit_pic
%%%cit_text
Президент России \emph{Владимир Путин} рассказал о возможной встрече с
украинским коллегой Владимиром Зеленским. Об этом он рассказал во время прямой
линии с гражданами в среду, 30 июня. \emph{Путин} заявил, что не отказывается
от встречи с президентом Украины Владимиром Зеленским, но надо понять, о чем
говорить. По его словам, Зеленский отдал Украину под полное внешнее управление.
\enquote{Что встречаться с Зеленским, если он отдал свою страну под полное
внешнее управление? Судьба Украины решается в Вашингтоне. И отчасти в Берлине и
Париже}, - сказал \emph{Путин}
%%%cit_comment
%%%cit_title
\citTitle{Путин прокомментировал встречу с Зеленским}, Антон Щукин, strana.ua, 30.06.2021
%%%endcit
