% vim: keymap=russian-jcukenwin
%%beginhead 
 
%%file 19_06_2021.fb.savenkova_faina.1.skobcov_vladimir
%%parent 19_06_2021
 
%%url https://www.facebook.com/permalink.php?story_fbid=323954015892204&id=100048328254371
 
%%author 
%%author_id savenkova_faina
%%author_url 
 
%%tags literatura,ljubov,lnr,pismo,poezia,savenkova_faina,skobcov_vladimir.dnr.orfei
%%title Вчера написала Владимиру Леонидовичу Скобцову
 
%%endhead 
 
\subsection{Вчера написала Владимиру Леонидовичу Скобцову}
\label{sec:19_06_2021.fb.savenkova_faina.1.skobcov_vladimir}
\Purl{https://www.facebook.com/permalink.php?story_fbid=323954015892204&id=100048328254371}
\ifcmt
 author_begin
   author_id savenkova_faina
 author_end
\fi

Вчера написала Владимиру Леонидовичу Скобцову. Мы с ним давно не общались и
многие, наверное, не знают, что он был одним из первых, кто начал помогать мне
с творчеством. А история о том, как он учил меня писать стихи когда-нибудь
обязательно появится в рассказах. Но сейчас не об этом. Владимир Леонидович
один из моих любимых поэтов. А его лирика, наверное, самое важное и нужное.
Ведь рассказывать нужно не только о войне, но и о любви.

\ifcmt
  pic https://scontent-frt3-2.xx.fbcdn.net/v/t1.6435-9/201577821_323953499225589_4125046289250173240_n.jpg?_nc_cat=101&ccb=1-3&_nc_sid=730e14&_nc_ohc=2VxPleAzYikAX_7MJ5e&_nc_ht=scontent-frt3-2.xx&oh=37c040cdaf65d4c43bec668561b2e985&oe=60D364F1
\fi

\begin{itemize}
\iusr{Виктория Сем}
Символичное фото у \enquote{Друга} Равиля:) Один из немногих памятников, у которого не надоедает быть.

\iusr{Владимир Скобцов}
Спасибо, солнышко! Пиши и радуй нас.

\iusr{Анна Владимировна}

Он самый классный у нас! Это душа человек. Его произведения трогают душу и погружает в воспоминания. Он пишет про нас. И огромное ему спасибо! Никакого пафоса. Он человечный.

\iusr{Акулина Жемчугова}
Два гения

\iusr{Елена Владимировна}
Это был один из первых поэтов, мной записанных. А то, что Вы рассказали, открывает его с новой стороны.
\end{itemize}

