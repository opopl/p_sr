% vim: keymap=russian-jcukenwin
%%beginhead 
 
%%file 22_12_2021.fb.kyiv_arsenal.1.dveri
%%parent 22_12_2021
 
%%url https://www.facebook.com/Kyiv.Arsenal/posts/385870579957197
 
%%author_id kyiv_arsenal
%%date 
 
%%tags 
%%title Порятунок старовинних дверей входить у моду
 
%%endhead 
 
\subsection{Порятунок старовинних дверей входить у моду}
\label{sec:22_12_2021.fb.kyiv_arsenal.1.dveri}
 
\Purl{https://www.facebook.com/Kyiv.Arsenal/posts/385870579957197}
\ifcmt
 author_begin
   author_id kyiv_arsenal
 author_end
\fi

Порятунок старовинних дверей входить у моду.

Мода на відновлення старовинних дверей поширилася і у Харкові. Ми вже
розповідали, як в Івано-Франківську ініціатива місцевих волонтерів призвела до
популяризації реставрації старовинних брам по всьому місту. У Харкові ж місцеві
архітектори і волонтери вирішили врятувати історичну спадщину першої
української столиці, і запустили проєкт Dveryki.Kharkiv.

\ii{22_12_2021.fb.kyiv_arsenal.1.dveri.pic.1}

Зрозумівши, що у головному місті Слобожанщини збереглися десятки старовинних
дверей, чий вік перевалив за сотню років, вирішили провести аудит, щоб
зрозуміти, скільки ж їх усього залишилося. Виявилося, що більшість із них
взагалі ніколи не реставрували, інші ж відремонтували неправильно, і вони
втратили свій історичний вигляд. А тому ситуація потребувала об’єднання зусиль
багатьох фахівців. 

\ii{22_12_2021.fb.kyiv_arsenal.1.dveri.pic.2}

Відтак, ентузіасти почали збирати каталог старовинних дверей, для чого створили
групи у Facebook і Instagram. Першими в черзі на реставрацію стали двері
будинку доктора Ар'є на Пушкінській, 7, виготовлені в 10-х роках ХХ століття у
стилі модерн. Є й більш старі екземпляри кінця ХІХ століття. Згодом до
ініціативи долучилися небайдужі громадяни. 

\ii{22_12_2021.fb.kyiv_arsenal.1.dveri.pic.3}

З часом волонтери обрахували і приблизний бюджет реставрації одних дверей.
Зрозуміло, що це задоволення не з дешевих, але мова йшла про капітальний
ремонт. Виходило, що середня реставрація дверей івано-франківськими
ентузіастами обходилася у суму 60-150 тисяч гривень. Харківські ж реставратори
почали обраховувати, де можна було зекономити на роботах, щоб зменшити їхню
вартість. 

Виявилося, що цілком можливо відреставрувати брами за суму від 15 до 40 тисяч
гривень. Питання полягає не лише у вартості реставраційних робіт, але й у тому,
чи зніматимуть двері і відвозитимуть в інше місце для проведення робіт, чи
ставитимуть їм тимчасову заміну. Втім, досвід Києва показав, що цілком можна
реставрувати двері, не знімаючи їх з під’їзду. 

Відтак, постала потреба знайти фахівця, який би міг відреставрувати двері на
місці. Це істотно здешевлює кошторис робіт. І у жовтні цього року стартували
ремонтні роботи. Чимало харків’ян здивувалися, коли на місці старих і
занедбаних входів до під’їздів побачили старовинні розкішні вхідні брами. 

Фото: Старовинні двері під’їздів Харкова.
