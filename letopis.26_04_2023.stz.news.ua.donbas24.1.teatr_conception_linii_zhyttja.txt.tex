% vim: keymap=russian-jcukenwin
%%beginhead 
 
%%file 26_04_2023.stz.news.ua.donbas24.1.teatr_conception_linii_zhyttja.txt
%%parent 26_04_2023.stz.news.ua.donbas24.1.teatr_conception_linii_zhyttja
 
%%url 
 
%%author_id 
%%date 
 
%%tags 
%%title 
 
%%endhead 

Театр авторської п'єси Conception представив поетичну імпрезу «Лінії життя»

Маріупольський театр у столиці виступив з новою постановкою

25 квітня у Києві в Національному музеї літератури відбулася чергова прем'єра
Театру авторської п'єси «Conception», який продовжує підтримувати та
популяризувати культуру Маріуполя. Цього разу актори представили театралізовану
поетичну імпрезу авторів міста Марії «Лінії життя».

Читайте також: У Києві відновили маріупольську виставу (ФОТО)

«Це чуттєва подорож у світ поезії та театру, що розкриває глибокі переживання
міста Марії під час війни», — йшлося в анонсі заходу.

Зі сцени лунали вірші відомого поета та письменника, композитора, доктора
соціологічних наук, професора Маріупольського державного університету Богдана
Слющинського, який загинув у Маріуполі. Дуже вразили глядачів і щемливі вірші
актора театру Дмитра Гриценка. Також у програму увійшли вірші маріупольських
поетес Оксани Стоміної, Марії Сладкової, Олени Мирошниченко, Ірини Громової та
Ольги Демідко. Завдяки віршам цих авторів і театралізованому дійству актори
змогли майстерно передати настрій міста, його красу та складність, показавши,
як війна впливає на життя людей та як важливо зберігати людськість у будь-яких
умовах.

«Акторам було легко працювати з текстами маріупольських авторів, дуже легкий
слог, відчувалось, що писали люди з нашого регіону. Загалом для театру це дуже
цікавий досвід, адже це не просте прочитання віршів, була і акторська робота,
тому показана імпреза може вважатися повноцінною виставою», — зазначив режисер
Театру авторської п'єси «Conception» Олексій Гнатюк.

Читайте також: Маріупольські актори зіграли благодійну виставу в Києві (ФОТО)

Актори Театру авторської пєси «Сonception» чутливо, щиро і правдиво читали
вірші маріупольських авторів, які акумулювали свій біль і горе від війни у
римовані рядки. Поетична імпреза «Лінії життя» змогла розкрити всю глибину
трагедії окупованого міста. Ці вірші дійсно чіпляють за живе, адже вони
розповідають правду про цю страшну, болючу і жорстоку війну.

«Дуже вдячні всьому колективу театру за цю важливу і чутливу поетичну виставу.
Ми сиділи і плакали, але усвідомлювали, що завдяки таким віршам і постановкам
наступні покоління зможуть дізнатися правду», — наголосила маріупольчанка Марія
Макарова.

Читайте також: «Життя переселенське» — маріупольський театр підготував нову
прем'єру

Театралізовану поетичну імпрезу «Лінії життя» заплановано показувати й надалі,
тому вона буде введена до репертуару театру.

«Така форма має бути в нашому репертуарі. Насправді, це буде
вистава-трансформер, яку можна буде показувати на різних локаціях», — додав
Олексій Гнатюк. 

Раніше Донбас24 розповідав унікальні факти про театральну культуру Маріуполя.

Ще більше новин та найактуальніша інформація про Донецьку та Луганську області
в нашому телеграм-каналі Донбас24.

Фото: з архіву Донбас24
