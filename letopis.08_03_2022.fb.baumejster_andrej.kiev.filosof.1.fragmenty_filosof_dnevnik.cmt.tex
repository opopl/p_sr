% vim: keymap=russian-jcukenwin
%%beginhead 
 
%%file 08_03_2022.fb.baumejster_andrej.kiev.filosof.1.fragmenty_filosof_dnevnik.cmt
%%parent 08_03_2022.fb.baumejster_andrej.kiev.filosof.1.fragmenty_filosof_dnevnik
 
%%url 
 
%%author_id 
%%date 
 
%%tags 
%%title 
 
%%endhead 
\zzSecCmt

\begin{itemize} % {
\iusr{Mariya Arenz}

Дорогой Андрей Олегович, сегодня пересмотрела Вашу беседу \enquote{Опровержение
манипуляций}, записанную 6! месяцев назад.

Кремлёвский карлик либо посмотрел Ваше видео, либо прочитал рекомендуемую Вами
книгу @igg{fbicon.beaming.face.smiling.eyes}.

Разрешите здесь поделиться ссылкой на эту лекцию - это просто гениально!

\iusr{Mariya Arenz}
\href{https://youtu.be/I46SJGqotJ8}{%
Как обнаруживать и опровергать манипуляции. Практическое руководство, %
Andrii Baumeister, youtube%
}

Продолжение цикла бесед о теории и практике пропаганды и манипуляции. На этот
раз я говорю о методах манипуляции и о том, как их обнаруживать и разрушать
изнутри. В беседе рассматриваются 14 методов манипуляции: 

\begin{itemize}
  \item 1. Нормирование языка. Так \enquote{революция} или \enquote{госпереворот}? \enquote{Автократия} или \enquote{демократия}? 
  \item 2. Использование оценочных понятий. Кто здесь \enquote{левые} (\enquote{правые}) \enquote{радикалы}? Так кто здесь \enquote{популист}? 
  \item 3. Метод сокращения историй. Почему идеологи либерализма сокращают собственную историю? Когда империализм и колониализм стали исчадиями ада? 
  \item 4. Метод замалчивания. Как Франция стала страной-победительницей? Кто жил во Львове в 1939 году? 
  \item 5. Метод монотонных повторений. \enquote{Вода камень точит}. Если на горькое все время говорить \enquote{сладкое}, то оно... таки станет сладким. 
  \item 6. Метод преувеличений. Страха никогда не бывает мало. Если ты не принимаешь идей неолибералов, то строишь концлагеря. Лукавство Хайека и Фридмана. В другой версии: \enquote{сегодня ты играешь джаз, а завтра родину продашь}. 
  \item 7. Известия \enquote{со всех сторон} (сообщения из каждого утюга). Блогеров и экспертов не бывает мало. 
  \item 8. Если все вокруг говорят то же самое. то это правда. Или \enquote{не доверяй собственным глазам}. \enquote{Не выделяйся}! Ты что, самый умный? Тебе это не выгодно. 
  \item 9. Эффект качелей. О злом Трампа и добром Обаме. Если ты борешься со злом, то от этого ты автоматически не становишься добрым. 
  \item 10. Использование соцопросов для формирования общественного мнения. Или ты слышал, что \enquote{люди говорят}? Опять не выделяйся! 
  \item 11. Создание NGO. Тут я лучше промолчу... 
  \item 12. Создание токсических перечней. Трамп, Путин, Эрдоган, Орбан, Сальвини, Качиньский и, конечно, Си. Ты тоже хочешь в эту компанию? 
  \item 13. Влияние на эмоции. Как ты можешь так думать (говорить), когда... Поставьте после многоточия что-то страшное и у вас все получится. 
  \item 14. Инсценировка конфликтов чтобы создавать мнение. Андрей, а ты патриот? А то мы тут начинаем подозревать, что ты... 
\end{itemize}

Тайм-коды: 

\begin{itemize}
  \item 00:00 Введение. Книга Albrecht Müller. Glaube wenig. Hinterfrage alles. Denke selbst. Wie man Manipulationen durchschaut. 
  \item 04:00 1. Нормирование языка. Так "революция" или "госпереворот"? "Автократия" или "демократия"? 
  \item 10:00 2. Использование оценочных понятий. Кто здесь "левые" ("правые") "радикалы"? Так кто здесь "популист"? 
  \item 15:52 3. Метод сокращения историй. Почему идеологи либерализма сокращают собственную историю? Когда империализм и колониализм стали исчадиями ада? 
  \item 42:38 4. Метод замалчивания. Как Франция стала страной-победительницей? Кто жил во Львове в 1939 году? 
  \item 46:41 5. Метод монотонных повторений. "Вода камень точит". Если на горькое все время говорить "сладкое", то оно... таки станет сладким. 
  \item 47:48 6. Метод преувеличений. Страха никогда не бывает мало. Если ты не принимаешь идей неолибералов, то строишь концлагеря. Лукавство Хайека и Фридмана. В другой версии: "сегодня ты играешь джаз, а завтра родину продашь". 
  \item 49:00 7. Известия "со всех сторон" (сообщения из каждого утюга). Блогеров и экспертов не бывает мало. 
  \item 50:48 8. Если все вокруг говорят то же самое. то это правда. Или "не доверяй собственным глазам". "Не выделяйся"! Ты что, самый умный? Тебе это не выгодно. 
  \item 52:40 9. Эффект качелей. О злом Трампа и добром Обаме. Если ты борешься со злом, то от этого ты автоматически не становишься добрым. 
  \item 54:03 10. Использование соцопросов для формирования общественного мнения. Или ты слышал, что "люди говорят"? Опять не выделяйся! 
  \item 57:40 11. Создание NGO. Тут я лучше промолчу... 
  \item 59:52 12. Создание токсических перечней. Трамп, Путин, Эрдоган, Орбан, Сальвини, Качиньский и, конечно, Си. Ты тоже хочешь в эту компанию? 
  \item 1:01:33 13. Влияние на эмоции. Как ты можешь так думать (говорить), когда... Поставьте после многоточия что-то страшное и у вас все получится. 
  \item 1:02:00 14. Инсценировка конфликтов чтобы создавать мнение. Андрей, а ты патриот? А то мы тут начинаем подозревать, что ты...
  \item 1:04:32 Резюме
\end{itemize}

Финансовая поддержка наших усилий позволит сделать наш продукт более
качественным. Поэтому мы будем благодарны за финансовую поддержку проекта. 

Карта Приватбанка 4149 4991 3111 9297

Стать патроном канала: \url{https://www.patreon.com/andriibaumeister}

\begin{itemize}
  \item 4314 1400 0121 8103 RUB
  \item 4314 1400 0121 8053 USD
  \item 5355 2800 0189 5323 EUR
  \item 5355 2800 0528 7451 UAH
\end{itemize}

\iusr{Nadia Leoni}
\enquote{Как могут освобождать ненавистники свободы?} Браво, Андрей Олегович!
 @igg{fbicon.hands.applause.yellow}   @igg{fbicon.hands.applause.yellow}   @igg{fbicon.hands.applause.yellow} 

\iusr{Kira Savy}
Сильнейший программный текст. Да, все так, без божества и вдохновения...Спасибо вам, вы помогаете дышать

\iusr{Alexander Levine}
В строках этого мощного поста неумолимо читается судьба и Украины и России!

\iusr{Юрій Чорноморець}
Столько раз предупреждали, что такое возможно... и опять неожиданно.

\iusr{Gleb Nemo}
Указывает направление откуда ветер дует

\ifcmt
  ig https://scontent-frt3-1.xx.fbcdn.net/v/t39.30808-6/275386657_7158185550889553_3076040086815087301_n.jpg?_nc_cat=106&ccb=1-5&_nc_sid=dbeb18&_nc_ohc=1k1nwMQHVI0AX9nkb6e&_nc_ht=scontent-frt3-1.xx&oh=00_AT-HPac5pPR_8K3FrU6B_KqLVDXGP9zbX-R7kAM4KS-bvQ&oe=622D8520
  @width 0.3
\fi

\iusr{Ната Руст}

кажется, в тексте есть ответ  @igg{fbicon.face.smiling.sunglasses} выползли за своей, только воображаемой,
нужностью.. важная и непреодолимая потребность даже у просто двуногих

\iusr{Наталія Іванова}
Спасибі за СЛОВО. Правдиве і болісне...

\iusr{Анна Яковлева}
Все верно. Только еще везде ненависть, ужасная ненависть

\iusr{Boris Teishev}
\textbf{Vладимир Zеленский}

\iusr{Maria Lypych}
Метафізична війна, ого, які вони езотерики  ☺ ️ 

\iusr{О. Олег Кобель}
Вони так довго вдивлялися в безодню.....

\iusr{Энна Кот}

Да, они вызывают чувство жалости....в своей злобе и ненависти ко всему светлому
и прогрессивному. Хотят казаться \enquote{модными}, но не могут. Украина платит кровью
за духовный прогресс человечества.

\iusr{Irina Zuevic}
У Вас есть догадки, когда закончится война?

\iusr{Александр Анохин}
Ух!

\iusr{Владимир Величко}
Мощно

\iusr{Владимир Калюжный}

26.10.2016

\href{https://www.facebook.com/Vladimir.Kalyuzhniy/posts/943868742385201}{%
Фашизм – иллюзия превосходства, переходящая в страсть насилия, Владимир Калюжный, facebook, 26.10.2016
}

\iusr{Юрій Чорноморець}

\href{https://risu.ua/aleksey-shevchenko-doktor-filosofskih-nauk-fashizm-kak-eshatologicheskaya-religiya-v-rossii_n43135}{%
Алексей Шевченко, доктор философских наук. Фашизм как эсхатологическая религия в России, risu.ua, 07.12.2010%
}

\iusr{Maria Lypych}

\ifcmt
  ig https://scontent-frt3-2.xx.fbcdn.net/v/t39.30808-6/275291654_4918484794867629_2482972238407739686_n.jpg?_nc_cat=103&ccb=1-5&_nc_sid=dbeb18&_nc_ohc=Qr_FsDT9cRkAX-a0RK5&_nc_ht=scontent-frt3-2.xx&oh=00_AT-MaZC90pwKc7o00d2yXonAjKyLlJjAevosB_tIkhlGjw&oe=622E08A2
  @width 0.3
\fi

\iusr{Marina Baranivska}
 @igg{fbicon.hands.applause.yellow}{repeat=5} 

\iusr{Таня Максимчук}

На территории РФ очень много психически травмированных пропагандой людей. Мои
двоюродные брат и сестра, их подростковые дети, они целиком и полностью в
другой извращенной реальности. Хотя столько лет жили в Украине и только восемь
в русском мире. Сейчас находятся в Москве. Им хватило пару дней для оправдания
всего этого циничного и бессмысленного ужаса. Я не уверена в том, можно ли этих
людей вытащить на светлую сторону смысла, смогут ли они мыслить самостоятельно
хотя бы процентов на пять.

\begin{itemize} % {
\iusr{Сергій Мороз}
\textbf{Таня Максимчук} 

Такая же картина. К сожалению, людей, способных мыслить независимо и создавать
целостную картину мира, очень мало. Отсюда и время от времени прорастающие
Гитлеры, в любом обществе (Трамп или Орбан) Рецепта человечество пока не
нашло...

\iusr{Gleb Nemo}
\textbf{Таня Максимчук} 

в Украине, и даже в Киеве, есть такие, которые оправдывают Путина. Даже среди
моих друзей. И, что удивительно, никто их не ограничивал в доступе к информации
мировой паутины интернета.  @igg{fbicon.thinking.face} 

\iusr{Ната Руст}
\textbf{Gleb Nemo} похоже, дело не только в среде проживания тела..

\iusr{Gleb Nemo}
\textbf{Ната Руст} в том-то и дело. Этот феномен надо изучать.

\iusr{Ната Руст}
\textbf{Gleb Nemo} причём, заново  @igg{fbicon.face.smiling.sunglasses}  учение о пропащих душах и их спасении

\iusr{Юлия Чинарева}
\textbf{Gleb Nemo} 

в каком процентном соотношении? Интересно. Способ «освобождения» не отрезвляет?
То, что в Украине есть люди, симпатизирующие Путину, я знала. По моим
предположениям, вряд ли их больше 5 процентов было (и в основном на востоке).
Но удивительно, как можно оправдывать того, кто бросает на твой дом бомбы? Или
они тоже считают, как нам тут впаривают, что все злое творят ВСУ? А российская
армия «творит добро».

\iusr{Юлия Чинарева}
\textbf{Таня Максимчук} 

если бы пропаганда не работала, то ее бы не производили. Активно поддерживает
«специальную военную операцию» не более 10 процентов. Это - ярые поклонники
«особого пути» типа Прилепина и его аудитории. Они именно «Za» и «Вперёд,
гасить нациков». Но большая часть, процентов 60-70, как Вы верно определили,
«оправдывает». Чувство вины угнетает как в индивидуальном, так и в коллективном
преломлении. Отсюда: подмена понятий, перевод стрелок и так далее... Нам 8 лет
внушают, что в Украине сплошные нацики, а само украинское государство -
искусственно созданный антирусский проект. Причём второе исповедуют даже
некоторые противники агрессии: не поддерживают вторжение, но считают Украину
вот этим самым... проектом. Порядка 30-40 процентов не поддерживает вторжение с
разной степенью активности.

\end{itemize} % }

\iusr{Sergiy Yarunsky}
Россия и есть пустота, дыра, мусоросвалка.

\iusr{Gleb Nemo}

\ifcmt
  ig https://scontent-frx5-1.xx.fbcdn.net/v/t39.30808-6/275367614_7158174657557309_7876378366935173351_n.jpg?_nc_cat=110&ccb=1-5&_nc_sid=dbeb18&_nc_ohc=2Gqkp89THKQAX_kBHD2&_nc_ht=scontent-frx5-1.xx&oh=00_AT-brpZn0BjuSgN_lGIQKEbzY7Ft9EA_pk62vme_gmEsXg&oe=622DB25E
  @width 0.3
\fi


\end{itemize} % }
