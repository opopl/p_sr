% vim: keymap=russian-jcukenwin
%%beginhead 
 
%%file 24_11_2021.fb.andrienko_julia.doneck.1.front
%%parent 24_11_2021
 
%%url https://www.facebook.com/permalink.php?story_fbid=4275187849273689&id=100003475479379
 
%%author_id andrienko_julia.doneck
%%date 
 
%%tags dnr,donbass,front,ukraina,vojna
%%title Какие ребята стоят там, на передке. День и ночь стоят. Восьмой год
 
%%endhead 
 
\subsection{Какие ребята стоят там, на передке. День и ночь стоят. Восьмой год}
\label{sec:24_11_2021.fb.andrienko_julia.doneck.1.front}
 
\Purl{https://www.facebook.com/permalink.php?story_fbid=4275187849273689&id=100003475479379}
\ifcmt
 author_begin
   author_id andrienko_julia.doneck
 author_end
\fi

Тяжело писать, когда мало эмоций (да и надо ли?), но когда их так много, что
они накрывают тебя с головой, это еще сложнее. 

Ощущаешь вполне свое несовершенство. Давно у меня не было столь сильных по
контрасту впечатлений, как в поездке с военным волонтером Ирина Ивановна
Полторацкой. Вся она будет на нашем канале \url{https://t.me/donbassr} 

\ifcmt
  ig https://scontent-frx5-1.xx.fbcdn.net/v/t39.30808-6/259476971_4275181032607704_5624943206966722096_n.jpg?_nc_cat=105&ccb=1-5&_nc_sid=730e14&_nc_ohc=6qE5X7MXI-kAX-e6Mzp&_nc_ht=scontent-frx5-1.xx&oh=7b8d6200c6cfb225a30b98e909de465e&oe=61A4E1B0
  @width 0.4
  %@wrap \parpic[r]
  @wrap \InsertBoxR{0}
\fi

А сейчас хочу вам рассказать историю, которая и на новость будто не тянет, но
лучше всего показывает, какие ребята стоят там, на передке. День и ночь стоят.
Восьмой год. Мы живем, жалуемся, требуем, злимся, а они – стоят. Порой мы о них
забываем, но не они о нас. 

Так вот, испив чай в блиндаже и наобщавшись с военными, мы возвращались в
цивилизацию, ехали по Октябрьскому. Спросила, есть ли у ребят минутка, мне бы с
мирными еще хотелось пообщаться. Так точно, отвечают. Не могу передать, как мне
это их «Так точно!» прям по сердцу - ни нытья тебе, ни набивания цены.

Подошла я к двум пожилым женщинам на остановке, разговорились. Одна потом ушла.
А оставшаяся старушка 82-х лет всю жизнь мне свою рассказала – как глаз
потеряла на производстве, как домишко ее пострадал от обстрелов, как сама она
его латает, как дочка умерла, а за ней - муж и как живет она на 7 тысяч пенсии.
Я еще задавала вопросы, а сама уже думала: помогу-ка ей хоть немножко и тут же
соображаю, что сумку-то свою с кошельком я оставила в располаге. Да и Ирина
Полторацкая тоже оставила. Какие там сумки в окопах, когда на плечах тяжелые
броники?! Но желание помочь до того сильное, что я подошла к военным и
попросила занять мне эти деньги. Командир молча протянул тысячу. Бабушка была
изумлена, только и повторяла, глядя уцелевшим глазом: «Да за что мне, дочка? За
что?»

А в располаге только скинув броник, я потянулась за кошельком отдать долг.
«Нет, не надо. Я не возьму», - сказал командир. «Как не возьмете?! Это было мое
желание, почему вы должны за него платить?», - настаиваю я. «Ну, считайте это
было и наше желание», - кратко отвечает. Вот так. Командир этот Дмитрий – отец
троих детей. Небось, тысяча – совсем не лишняя, чтобы платить за прихоти
взбалмошной журналистки. Вот они такие русские витязи, богатыри наши. Я еще
пыталась настаивать, но Ирина, лучше знакомая с нравом военных, сказала, что
это напрасно, если сказали не возьмут, значит, не возьмут. 

Я хочу рассказывать о наших военных. И буду. Когда-то говорили, народ и армия
едины, все для фронта – все для победы. Так почему же эта тема закрыта? Будто и
нет у нас героев. Будто и нет войны. А может кому-то хочется, чтобы мы так
думали?

\ii{24_11_2021.fb.andrienko_julia.doneck.1.front.cmt}
