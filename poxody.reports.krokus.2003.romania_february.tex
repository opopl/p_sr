% vim: keymap=russian-jcukenwin
%%beginhead 
 
%%file reports.krokus.2003.romania_february
%%parent body
 
%%endhead 
\section{Лыжный поход по Восточным Карпатам, Румыния}
\url{http://krokus.org.ua/pohod/rom03z/otchet.html}
  
\vspace{0.5cm}
 {\ifDEBUG\small\LaTeX~section: \verb|krokus.2003.romania_february| project: \verb|poxody| rootid: \verb|p_saintrussia| \fi}
\vspace{0.5cm}

Горы Сухард и Родна.

Т/к ``Крокус'', т/к ``Университет'', Киев.

\subsection{Участники}

\begin{itemize}
	\item Таисия Богдзиевич
	\item Илья Гуз – руководитель
	\item Евгений Котюх – фотограф
	\item Павел Котюх – хронометрист
	\item Ирина Осипова – завхоз и финансист
	\item Александр Поплавский – переводчик
	\item Павел Приймачек – ремонтник
	\item Алексей Таланцев – медик
	\item Владимир Трылис
\end{itemize}

\subsection{Маршрут}

\begin{itemize}
\item 1. С.Якобени (Iacobeni) – пер. между г.Ливада (Livada, 1463) и Якоб
				(Iacob, 1322) – р.Циотина (Ciotina) – с.Валеа Банкулуй – р.Кошница
								(Cosnita) – пер.Сухард (Suhard, 1200) – р.Мария Маре (Maria
								Mare) – р.Сомешу Маре (Somesu Mare) – р.Аринул (Arinul) –
								с.Валеа Маре (Valea Mare).

\item 2. С.Кормая (Cormaia) – р.Кормая – т/б “Фармекул Падури” (“Farmecul
				Padurii”) – г.Детуната (Detunata, 1752, 1А) – р.Изворул Палтинишулуй
								(Izvorul Paltinisului) – р.Изворул Скарицелей (Izvorul
								Scaricelei) – р.Ребра (Rebra) – лесничество “Гура Ребра” (Gura
								Rebra) – р.Ребришоара Маре (Rebrisoara Mare) – седловина Недеи
								(Nedeii, 1450) – вершина 1705 м – вершина 1685 м – вершина 1720
								м (1А) – седловина Батриней (Batrinei) – р.Иза (Iza) – с.Моисей
								(Moisei).
\end{itemize}

\subsection{Время проведения}

3-11 февраля 2003 г.

\subsection{График движения}

3.2.2003 (1-й день).

Село Якобени (Iacobeni) – пер. между г.Ливада (Livada, 1463) и Якоб (Iacob, 1322).

7,1 км, подъем с 900 до 1400 м. Время: 7:40 – 13:30, ЧХВ - 3:55.

Ясно, -10…-15ºС, на хребте (безлесый участок) – ветер.

От шоссе перешли по подвесному мостику через р.Бистрица-Аури (Bistrita-Aurie)
на южной окраине села Якобени (Iacobeni). За мостиком имеется развилка. Дорога
на перевал в хорошем состоянии. Дорога идет вдоль речки Хаю (Haju), в западном
направлении, в верхней части проходит по руслу (на карте в этом месте
обозначена пунктиром). Через 100 м несколько затрудненного продвижения по
руслу, вышли на лесовозную дорогу, которая серпантином вывела на отрог г.Рунку
(D.Runcu, 1122,1372). От края леса до хребта – пологий подъем. Ориентир – на
хребте имеется небольшая скала с растущей под ее укрытием елью. Непосредственно
на хребте имеются постройки для летнего выпаса скота.


