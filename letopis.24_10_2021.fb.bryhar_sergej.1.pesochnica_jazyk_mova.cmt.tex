% vim: keymap=russian-jcukenwin
%%beginhead 
 
%%file 24_10_2021.fb.bryhar_sergej.1.pesochnica_jazyk_mova.cmt
%%parent 24_10_2021.fb.bryhar_sergej.1.pesochnica_jazyk_mova
 
%%url 
 
%%author_id 
%%date 
 
%%tags 
%%title 
 
%%endhead 
\subsubsection{Коментарі}

\begin{itemize} % {
\iusr{Ntina Ntoubrova}

Краща відповідь на "село" - " А ти що, понаїхавший? звідки в тебе такі
комплекси щодо села? Ааа. Ти "каріной адісіт" - ну тоді 100\% понаїхавший
перекати-поле. Бо місцеві знають, що вони українці і живуть в Україні".

\iusr{Vitaliy Chemerys}

Так, звичайно важко, бо боротьба і протистояння цій к\#ц\#пській біомасі забирає
багато сил і нервів, але було би ще важче, як би змирилися, "опустили руки" і
стали "как всє"... Тримайтесь, все буде добре!

Слава Україні!

\begin{itemize} % {
\iusr{Serhii Bryhar}
\textbf{Vitaliy Chemerys} Героям Слава!
\end{itemize} % }

\iusr{Лариса Артем'єва}
Напевно про село почули діти вдома. Так і виховують зневагу до інших людей, в даному випадку до людей, що спілкуються державною мовою.

\iusr{Елена Пака}

Хамство і агресія - це вже у крові запорєбрікових. Результат багаторічного
відбору. Знищувалася інтелігенція, ті, хто мав честь, людяність, совість.
Навряд на росії в найближчі роки з'являться нові сахарови, політковські,
новодворські...

Відірватися подалі і забути як страшний сон...

\iusr{Valik Marushchak}

Оце, про село - це мерзотний тупий кцпо-комуняцький пережиток, який
кцпо-комуняки відтворюють в своїх поколіннях. Коли село - це розвалений колгосп
Червоне Дишло, поламані паркани і грязні кірзові чоботи.

В той час, як в нормальних країнах село - це вільний приватний простір
господарів, розумних, забезпечених, з охайними дворами, причепуреними садами,
чистими асфальтованими вуличками, спокоєм, затишком. Як в Німеччині, Австрії,
вже і в Польщі, Чехії, вже і в Україні до цього трохи доходить.

Тому, так - "я село. Багате заможне українське село, з триповерховим котеджем,
власним басейном і ігровим майданчиком, і двома авто в дворі. А ти сараке убоге
дитя однокімнатної обдертої хрущовки, маршруток і спільного немитого
неосвітленого під'їзду".

Якось так би я сказав.

\begin{itemize} % {
\iusr{Serhii Bryhar}
\textbf{Valik Marushchak} 

Але певну рацію м*с*а*і мають. Погодься, якби не село, українська мова могла би
і не вижити. Релігійні та інші обрядові звичаї, самобутність, фольклор, і
головне - мова. Все це ворогам витравити з села так і не вдалося. Отже, вони
знають, на що зляться, і чого вчать своїх дітей.

\iusr{Михаил Логинов}
\textbf{Valik Marushchak}, або дитя комуни в центрі Одеси з таким туалетом. Зате парускі гаваріт і каренной адесіт.

\ifcmt
  ig https://scontent-frx5-1.xx.fbcdn.net/v/t1.6435-9/248437836_2089259481229544_7049246837596086573_n.jpg?_nc_cat=110&ccb=1-5&_nc_sid=dbeb18&_nc_ohc=0SMILud-VXsAX90_2ro&_nc_ht=scontent-frx5-1.xx&oh=e36dd090ca95910125fc29ec11aa7a54&oe=619CBB8F
  @width 0.4
\fi

\iusr{Valik Marushchak}
\textbf{Serhii Bryhar} 

саме так 100\%! Як доповнення до мого коменту про причини ненависті
мскломерзоти до українського села.

І мені просто сумно, що багато представників town-, чи village- населення якось
встидаються свого походження перед отакими обдертими лапотними пролетарями "с
ґорада".

Бо зневага нормального українського селища до цих завезених має бути
аналогічна, бо селища-нині об'єднані громади були би, як села в Данії, чи
Фінляндії, якби не голодранські комуняки.

А той зародиш лаптя таки реальний рашонацик, як тут казали в коментах, там явно
назріває потреба розмови з лапотними авторами такого екземпляра.

\iusr{Тетяна Лук'янова}
\textbf{Михайло Логінов}, 

ох, із тими комунами бодай не згадувати. Сподіваюсь, той алкаш, сусід мого
дядька по Островидова (на жаль, знову забула, як ця вулиця по старо-новому
називається), який чіплявся до мене, тоді 13-річної, здох від улюбленої звички,
а його діти попри все виросли нормальними людьми.

\end{itemize} % }

\iusr{Маруся Чурай}
Супер

\iusr{Ірина Магрицька}
У подальшому це може травмувати дитину, на жаль.

\begin{itemize} % {
\iusr{Ганна Собко}
\textbf{Ірина Магрицька} яку дитину?
Свою?
Ні.

\iusr{Наталія Толстіхіна}
А коли усі однокласники говорять російською? Травмує

\iusr{Roman Romanow}
\textbf{Ірина Магрицька} так!! треба обов'язково гарний прямий, сайдстеп на робочому слєгну звецця. його освоїти і буде досить)) неважно дитина чи на 15 кг важчий бичок. просто назустріч і вбік)
\end{itemize} % }

\iusr{Yuriy Yastreb}
Співрозмовник вашого Назара - реальний нацист, і треба імігрантам про це казати.

\iusr{Неонила Дрыгуш}
Росте справжній українець! Молодець!

\iusr{Ганна Собко}
Ваша м'якість - ваша помилка....
Ми не можемо толерувати агресивних цапіа...
А той малий- аресивний цап.

\iusr{Yurco AciMov}

Ми навіть у поїздках до Криму на питання звідки відповідали, що місцеві, тобто
українські, а не понаєхавшиє. А вже потім можна уточнювати населений пункт,
адресу тощо))

\begin{itemize} % {
\iusr{Юрко Градовський}
\textbf{Yurco AciMov} Я також. У 2013
\end{itemize} % }

\iusr{Nada Slobodan}
Дякую !

\iusr{Анна Бутова}

Діти росіян однозначно. А вам, як батькові потрібно провести бесіду з батьками
хлопчика . Хамство не можна дозволяти до дітей а тимпаче дорослих .


\iusr{Екатерина Вигилёва}

Ого!!! Я в таких випадках дуже голосно на повні груди починаю принижувати отого
"городского" і саме українською.

Ні, не ображати, не лаятися, а саме принижувати.

Вони такі швидко затикаються.

Як можна бути свідком гноблення і змовчати?!! Як? Як не заступитися за іншу
дитину, яку явно булять?

\begin{itemize} % {
\iusr{Serhii Bryhar}
\textbf{Катерина Вігільова} До речі, це ж неділя. Багато хто був з обома батьками. От у того хлопця, який, за його (мілкого агресора) логікою "тупить", поряд були мама і тато. Стояли собі, та й стояли. Це вже, перепрошую, їхня справа. Що ж до мого, то я спеціально не втручаюся. Бачу, що він уже вчиться і сам не пасувати. Щось уже виходить. То я б і не заважав. Я його потім похвалю)..

\iusr{Екатерина Вигилёва}
\textbf{Serhii Bryhar} за Вашого - я підтримую. Я й сама певний час не втручаюся - теж виховую самостійність. Але стою поруч, щоб вона відчувала підтримку і тил.
А от щодо того хлопчика - то це звісно жах((( Ви праві. Мають батьки захищати. Шкода того хлопця((((
\end{itemize} % }

\iusr{Yaroslav Lytovchenko}

Сергію, я реально захоплений Вашою витримкою і боротьбою в тому ватному
гадючнику на своїй рідній землі. І насолоджуюся Вашими постами. Побільше б
таких родин, як Ваша, і все буде добре в Україні. Дякую Вам!

\iusr{Павло Шубарт}

\ifcmt
  ig https://scontent-frx5-1.xx.fbcdn.net/v/t1.6435-9/248037119_181158840860227_1620278844603679563_n.jpg?_nc_cat=105&ccb=1-5&_nc_sid=dbeb18&_nc_ohc=QlDiVtj50HYAX-A7fT1&_nc_ht=scontent-frx5-1.xx&oh=a77efc55c2533d1a85f277308927cbfc&oe=619AE5A5
  @width 0.4
\fi

\begin{itemize} % {
\iusr{Svitlana Chub-Krywuzka}
\textbf{Павло Шубарт} ну так. До того ж, це в чистому вигляді колхозний язик, бо це ж вони нам свої колхози нав'язали
\end{itemize} % }

\iusr{Павло Шубарт}


\ifcmt
  tab_begin cols=3,no_fig

     pic https://scontent-frx5-1.xx.fbcdn.net/v/t1.6435-9/248401824_181159994193445_127702639534314124_n.jpg?_nc_cat=105&ccb=1-5&_nc_sid=dbeb18&_nc_ohc=mb74d0-QSgoAX9JAFMF&_nc_ht=scontent-frx5-1.xx&oh=2c2ac53143c5d7299b021abd43ac49c1&oe=619D133E

     pic https://scontent-frt3-1.xx.fbcdn.net/v/t1.6435-9/248252998_181160570860054_1237073350209398720_n.jpg?_nc_cat=107&ccb=1-5&_nc_sid=dbeb18&_nc_ohc=crrid6oGiTMAX_wpEEm&_nc_ht=scontent-frt3-1.xx&oh=3b5c2110d3d5abac4b8f2b3fd9b309f8&oe=619C4910

		 pic https://scontent-frt3-2.xx.fbcdn.net/v/t1.6435-9/248017572_181160770860034_3413380032705047239_n.jpg?_nc_cat=103&ccb=1-5&_nc_sid=dbeb18&_nc_ohc=-f2W14wLL6IAX_m5kcf&_nc_ht=scontent-frt3-2.xx&oh=49046cc5dae5804e8921fc966754cc60&oe=619A584A

  tab_end
\fi

\iusr{Юрій Олександренко}

коли це стане звичним з дитинства, то не буде забирати сил. але може привчити
бути весь час ворожим до оточення. треба спостерігати уважно.

\iusr{Микола Савка}
Треба було вуха понакручувати пі.дюку малому!

\iusr{Тетяна Лук'янова}

"Пісочничні хроніки " - 12 балів  @igg{fbicon.thumb.up.yellow}  Назар молодчина
@igg{fbicon.thumb.up.yellow} @igg{fbicon.heart.red} А той хлопчик зовсім береги
поплутав. Хто він такий, що з ним не можна "спорити"? І чому це він має право
не говорити української, а Назар йому ні сіло ні впало має "па рускі"?

\begin{itemize} % {
\iusr{Serhii Bryhar}
\textbf{Тетяна Лук'янова} А він не здасться!.. От видно, що мій малюк))).

\iusr{Yaroslava Zheyko}
\textbf{Тетяна Лук'янова} пздюка малого вдома накачали, діти ж вони як дзеркало

\iusr{Тетяна Лук'янова}
\textbf{Yaroslava Zheyko} , та очевидно. Найімовірніше, бабки з дідами. А може, і батьки, які там стояли, ніби всралися, не поснідввши.
\end{itemize} % }

\iusr{Svitlana Padiy}

Дай Вам Боже ! Це ж що відбувається, щоб тебе, українця , в Україні, за
українську мову ще й принижували якісь манкурти?!

\begin{itemize} % {
\iusr{Serhii Bryhar}
\textbf{Svitlana Padiy} Приниження я тут не бачу. Навіть малий уже розуміє, що таке стати в позу. І в нього більш-менш виходить). Але виразна неадекватність протилежної сторони - це так. М*с*ал*с*ке виховання, що тут скажеш.
\end{itemize} % }

\iusr{Ніна Гук Кукурудза}

Моя доня цьогоріч пішла до школи. І почалось. То пісеньку на перерві вмикали
російськомовну, то мультик замість фізкультури(!) "Маша і мєдвєдь" російською.
Написала вчительці в батьківському чаті(з надією, що хтось з батьків підтримає.
Наївна ). Вчителька ніби поставилась з розумінням. А вчора прийшла зі школи і
каже - на англійській вмикають навчальне відео, а переклад російською. І це
Вінничина, україномовне село.

\begin{itemize} % {
\iusr{Svitlana Chub-Krywuzka}
\textbf{Ніна Гук} Кукурудза продовжуйте довбати ватну вчительку. Може, вона просто трохи пустоголовенька й раніше над такими речами не замислювалася.
Виробіть у неї рефлекс. ''А, тут російська, за це мене пані Ніна сваритиме, краще не буду''

\iusr{Ніна Гук Кукурудза}
\textbf{Svitlana Chub-Krywuzka} та вчителька не ватна, може скоріше "какая разніца". Напередодні 14 жовтня дітям провели цікавий урок до Дня захисників і захисниць. Був випадок, коли вчителька виправляла суржик в дітей. Є ще асистент вчительки, можливо це її відповідальність, не знаю. Розумію одне, раніше всіх все влаштовувало, поки не з'явились ми. Найближча школа в районі за 12 км, і я маю сумніви, що там краща ситуація. Налаштовую дитину, що вона борець, що це нелегко, але дуже необхідно.

\iusr{Svitlana Chub-Krywuzka}
''Какаяразніца''
Ну так я ж і кажу, що пустоголовенька)

\iusr{Олександра Решотко}
\textbf{Ніна Гук Кукурудза} Треба писати і звертатися до директора, адже це зневага до українців, до державної мови і пряме порушення закону.

\iusr{Ніна Гук Кукурудза}
\textbf{Олександра Решотко} 

у директора свій магазин на території школи(тож його більше турбує бізнес, а не
справи школи), за 20 років навіть спортзал не спромігся побудувати, а ще
учасник антимайдану в 2014. Там кругова порука, в районі "всі свої". Наразі, з
вчителькою простіше порозумітись.

\iusr{Олександра Решотко}
\textbf{Ніна Гук Кукурудза} зрозуміло, але все одно треба гуртуватись і наполягати на своєму. @igg{fbicon.heart.blue}  @igg{fbicon.heart.yellow} 

\end{itemize} % }

\iusr{Tetyana Rozumenko}

це щастя, що дитина в такому віці вже може дати відсіч та наполягти на своєму.
хух, дай боже вам здоров'я та наснаги у цій боротьбі. бо це дуже важко.


\iusr{Максим Меркулов}

Я хоч і не українець, але теж розмовляю українською, бо... А кат його відає! Є
в цій країні і мові щось таке, що зачаровує і не відпускає. Серйозно.

\begin{itemize} % {
\iusr{Тетяна Лук'янова}
\textbf{Максим Меркулов} , значить, Ви українець у душі, дай Вам Боже здоров'я!

\iusr{Максим Меркулов}
\textbf{Тетяна Лук'янова} Я б посперечався, але... навзаєм.  @igg{fbicon.smile} 

\iusr{Тетяна Лук'янова}
\textbf{Максим Меркулов} , ну бодай трішки  @igg{fbicon.grin} 

\iusr{Svitlana Chub-Krywuzka}
\textbf{Максим Меркулов} а хто Ви хіба??  @igg{fbicon.face.astonished} 

\iusr{Максим Меркулов}
\textbf{Svitlana Chub-Krywuzka} Щирий космополіт.

\iusr{Svitlana Chub-Krywuzka}
\textbf{Максим Меркулов} нічого не зрозуміла)). Ви особа без громадянства з невідомим національним походженням?)

\iusr{Максим Меркулов}
\textbf{Svitlana Chub-Krywuzka} Я особа без національности. Громадянство маю. Походження змішане, але це не має великого значення.

\iusr{Svitlana Chub-Krywuzka}
\textbf{Максим Меркулов} це трохи дивно, бо якщо людина знає про декілька своїх національних походжень, то може собі з них обрати. Або в кожному разі може обрати бути українцем)

\iusr{Roman Romanow}
\textbf{Максим Меркулов} це свобода. власне коєхто з рускіх людей з-під Єбєнєй в Києві дуже за Україну. і українську вчать. ну просто бо "тут свабода а ані баяцца". це важний фактор))
\end{itemize} % }

\iusr{Володимир Олівець}

Жити в Коцюбієві і не розуміти москвинської- виглядає на казку! На жаль усі казки швидко закінчуються.

Але батькам Назара респектисько!

\begin{itemize} % {
\iusr{Serhii Bryhar}
\textbf{Володимир Олівець} Він, я думаю, вже розуміє. Але не розмовляє. Поки що зовсім. Реально може лише якісь в'їдливе слово впіймати. Або кілька слів. І все.

\iusr{Володимир Олівець}
Пане Bryhar, ваше виховання дуже важливе, але аби син протримався якнайдовше йому треба знайти опору у вигляді иншої націоналістичної дитини

\iusr{Svitlana Chub-Krywuzka}
Бвахахахахах, ''націоналістичної дитини''.
Назар просто українська дитина.
\end{itemize} % }

\iusr{Galia Kruvoruchko}

.... , дежав'ю,, - стільки років пройшло-а мені здалося, що то я,. "совок",
Чернівці, і коли Тобі казали - а Ти шо? с "западнай".... (ця гідра багатоголова
- невмируща??,,,,  @igg{fbicon.face.screaming.in.fear} ), віддаю честь ВАШОМУ святому терпінню, а у хлопчиків
Ваших - бійцівські якості!!!.  @igg{fbicon.thumb.up.yellow} 🔱

\iusr{Мироненко Анатолій}

Шановний пане Сергію, ви - розумна, освічена й свідома людина. Я завжди з
цікавістю читаю ваші дописи. Сам я зростав у зросійщених Сумах і бачив (й не
раз відчував на собі) яким чином у 60-70-тих роках відбувалося насильницьке
зросійщення українців, особливо вихідців з сіл. Бажаю вам і вашій родині
витримки і вього найкращого. Але хочу зауважити, що такі випадки (як з Назаром)
будуть виникати і в майбутньому. Коли, наприклад, декілька таких малолітніх
українофобів будуть вимагати від вашого сина, щоб він був "как всє" (а вас не
буде поруч) - тоді можливий конфлікт і, навіть, душевна травма (якщо не гірше)
дитини. На мій погляд вам, все ж, потрібно, поки діти ще малі, змінити місце
проживання (на жаль) на більш українське щодо мови, культури і звичаїв,
принаймні, допоки діти не підростуть і будуть здатні вже самостійно протистояти
тиску "рускава міра" у різних його проявах. Удачі вам на цьому непростому
шляху. Тримайтесь.

\begin{itemize} % {
\iusr{Тереза Мама}
\textbf{Мироненко Анатолій} Справжня мудрість у ваших порадах

\iusr{Тетяна Гога}
\textbf{Мироненко Анатолій }це ви проганяєте їх зі своєї землі, бо якась мокша не розуміє? В Україні у школах вчать всюди українську. А дитина поки виросте їх добре навчать, що тут українська. Ви їм пропонуєте виїхати.. Або вчити в Україні українську. А хлопчик, який виросте, буде загартованим і дасть відсіч отим вашим мокшам.

\iusr{Мироненко Анатолій}
\textbf{Тетяна Гога} Це я проганяю? Бог з вами. Я просто часто спілкуюся з автором допису у ФБ (ми з ним давно є фб-друзями). Він вже предметно думає над цим питанням. Можливо, переїде у Львів або ще кудись (бо поки розмірковує над цим питанням). А ось звинувачувати когось в тому, що він не казав, є не дуже добре (як і закликати до боротьби з "рускім міром", маючи на своїй сторінці майже всі "москвоязичні" уподобання).

\iusr{Тетяна Гога}
\textbf{Мироненко Анатолій} шановний пане Анатолію, я ніколи не послуговуюсь російською. НІКОЛИ. Я завжди пишу тільки українською. Звідки ви взяли, що у мене мокшанські уподобання. Назвіть хоч один, крім, можливо, поезії і вітань. І то дуже рідко.

\iusr{Мироненко Анатолій}
\textbf{Тетяна Гога} Я жодного слова не казав про ваше послуговування російською. А про ваші ФБ-уподобання я дізнався, переглянувши вашу сторінку. Можу перерахувати все те, що я там побачив (якщо потрібно). Але я вас ні в чому не звинувачував, на відміну від вас, а згадав просто до слова. Всього вам найкращого.

\iusr{Bohdan Kozak}
\textbf{Мироненко Анатолій}

Так, проганяєте. Непрямо, звісно, але... Ви ж не пропонуєте записати Назара на
айкідо (хай і кцпоязичне поки що), далі на карате, а ще краще на Бойовий Гопак
у тому ж Львові, де, упевнений, у пана Сергія стільки друзів, що хтось із них
візьме до себе жити Назара, ще й безоплатно... А Ви пропонуєте більш
капітулянтський варіянт...

\iusr{Мироненко Анатолій}
\textbf{Bohdan Kozak} Ось для чого щось домислювати і вигадувати. Щодо моєї згадки про зміну місця проживання то це не я придумав, а сам пан Сергій вже давно про це писав. Ми з ним давні фб-друзі і тому я про це знаю. Я вже давав відповідь пані Тетяні з цього приводу трохи раніше - почитайте.

\end{itemize} % }

\iusr{Svitlana Chub-Krywuzka}

Та вони насправді жалюгідні, оті московитики.  Я ось читала цю розповідь,
уявляла, і мені стало смішно.  Які ж вони жалюгідненькі. Примітивщина. ''Ти
чьо, сєло, гаварі нармальна'' - ааааахахах. Оце і все, на що ви спроможні,
окупантські правнучата?) Та ми ж їх переможемо на раз-два. Ваші, Сергію, Назар
з Остапом. Одним поглядом.

\begin{itemize} % {
\iusr{Людмила Демянчук}
\textbf{Svitlana Chub-Krywuzka} Незавжди, діти дуже жорстокі, а ще от таке московське потороччя !
Добре, що батько поряд, а так не знати , які б були дії малого москворотого шакала, який навряд чи змінить своє відношення до українців.
Це той варіант, коли презирство і ненависть до українського, це щеня всмоктало з молоком мамаші
\end{itemize} % }

\iusr{Олексій Гаруда}
Гарно

\iusr{Тереза Мама}
Ви мужній чоловік і я певна, що такими ж виховаєте своїх дітей. Надія є на те, що поки діти виростуть ситуація зміниться на краще.

\iusr{Viktor Voznyak}

Цей "узькоязикий рагулізм" не закінчиться доти , допоки на державному рівні не
будуть присікати розповсюдження "общєпанятного лайна" у всіх сферах діяльност і
потрібно діяти саме так, як це колись робили проти української мови ! ТБ і
підприїмців позбавляти ліцензій і штрафувати за іншомовність, освітту, медицину
і військову і держслужби зобов'зати спілкування виключно державною ! За саботаж
і небажання ввести позбавлення місця роботи, навчання, без права поновлення і
притягнення до адмін.відповідальності.( штрафом на користь розвитку мови)


\iusr{Валентин Шкарупа}
Вже дорослий

\iusr{Liudmyla Chayka}

Моєму було років 12, коли хтось заявив йому "чіво ти па-украінскі гаваріш? Міня
ета бєсіт". Якось сміг відповісти і навіть не посваритися, навіть трохи
перевиховати. Наші діти дадуть собі раду

\iusr{Nadiia Obertan}
Я б сказала б цьому мальчіку, щоб він розмовляв нормальною мовою.

\iusr{Артем Шульженко}
Все йде від батьків. Там і батьки такі чорти-українофоби

\begin{itemize} % {
\iusr{Валерій Лапікура}
\textbf{Артем Шульженко} Та ж ясно, що "и не спорь со мной" від батьків почув. Ну і про сєло. Нічого, Назар його зрештою перевиховає. А ще коли Остап підключиться - у хлопа не буде жодного шансу.
\end{itemize} % }

\iusr{Roman Romanow}
ніт. без мішочка з рукавичкаме таки не обійтись ((

\begin{itemize} % {
\iusr{Serhii Bryhar}
\textbf{Roman Romanow} Та він і так має такий характер, що... Вирізати - не проблема. Проблеми потім можуть бути в мене).

\iusr{Roman Romanow}
\textbf{Serhii Bryhar} люде схильні недооцінювати техніку й майстерність. а це зря (
\end{itemize} % }

\iusr{Александр Бакай}
Автору респект!!!

\iusr{Sergiu Devdyk}
Я б не витримав, а б запитав його(направду як в дорослого), що:
Чи він розуміє , що є два народи, українці і росіяни, і що деколи треба вибирати, хто ви молодий чоловіче?
Ну приблизно так би і зформолював.

\begin{itemize} % {
\iusr{Sergiu Devdyk}
І хай дитина думає!

\iusr{Serhii Bryhar}
\textbf{Sergiu Devdyk} Може ще так і зробимо. Я так бачу, одними стежками ходимо).

\iusr{Sergiu Devdyk}
\textbf{Serhii Bryhar} Та то більшість українства так йде! Нажаль, а може на краще.
\end{itemize} % }

\iusr{Інна Коляденко}
пора вже їх називати ОРДЛО або Ростов

\iusr{Bialyk Vasyl}
Ну і це нормально, щоб в українській державі переслідували українське якіст недолюдки кацапські?

\iusr{Yaroslav Datsko}
Що і треба було доказати. Пацан просто жорстко вас тролив. Це така форма булінгу

\iusr{Рианна Дюман}

Який Ви молодець!!! Пишаюся Вами, я як корінна львів‘янка.. і прикро за моіх
родичів, які переіхали в Одесу, і нажаль пєрєшлі на нормальний язик:(((


\iusr{Mashinskaya Anna}

Ой, а дітятко, схоже, йуний соціопат. Від таких взагалі краще триматися подалі.
А як заговорює, відбривати якомога грубіше, їм таке добре заходить.


\iusr{Ksenia Maryniak}

Давно, з самого початку треба було державі розгорнути (разом з рекламною
кампанією підтримки SME "Україні потрібні буржуазні націоналісти!") всебічну
програму "Село — наша культурно- економічна скарбниця".

\iusr{Олексій Сірож}
А той малий вперто й агресивно гне своє "гаварі нармально".

\iusr{Ирина Москаленко}
\textbf{Олексій Сірож} Скоріш за все, батьки- бидло кцпськн.

\iusr{Roman Pohrebennyk}

Сергію, Ви дуже правильно виховали свого сина. А якщо йому доведеться захищати
право на свою ідентичність то можна і "врізати" нахабі по пиці, бо добро може
бути і з кулаками.

\iusr{Jurko Zełenyj}

tak, ja vystojaw!
nu i huli z togo?
projobani ż roky - ne povernesz...
a otoczynė j seredovyżcze w cilomu lipsze naviť j blyźko ne stalo! Nawpaky: zkacaplynė take, jikogo j w 1980h rokah za sovjetïw ne bulo!!  @igg{fbicon.face.confused} 

\iusr{Prymara Makeeva}

А я село протиставляю дярьовнє та кальхозу. Кальхознікі абіжаюцца

\end{itemize} % }
