% vim: keymap=russian-jcukenwin
%%beginhead 
 
%%file 04_05_2019.fb.lesev_igor.1.kolomojskij_slova_civil_war.cmt
%%parent 04_05_2019.fb.lesev_igor.1.kolomojskij_slova_civil_war
 
%%url 
 
%%author_id 
%%date 
 
%%tags 
%%title 
 
%%endhead 
\subsubsection{Коментарі}

\begin{itemize} % {
\iusr{Сергей Киселев}
Есть ещё религиозный аспект. Православные с одной стороны, сектанты и язычники с другой

\begin{itemize} % {
\iusr{Игорь Лесев}
Думаю, преувеличиваете. У нас с удовольствием убивают друг друга и без обременительных споров о религиозных догматах.

\iusr{Сергей Киселев}
\textbf{Игорь Лесев} однако православных капеланнов из ВСУ изгнали, и все храмы в воинских частях отдали сектантам.
\iusr{Елена Кулакевич}
\textbf{Сергей Киселев} не совсем пойму с какой стороны сектанты а с какой православные?!

\iusr{Rustam Isakov}
\textbf{Елена Кулакевич} у товаріща ПГМ

\iusr{Oksana Kudrya}
\textbf{Елена Кулакевич} , с обеих. )

\iusr{Андрей Копейка}
\textbf{Сергей Киселев} подавляющее большинство воюющих с оружием в руках совершенно не разбирается в тонкостях этих религиозных аспектов

\iusr{Елена Кулакевич}
\textbf{Rustam Isakov} у каждого своя точка зрения, не надо оскорблять. Кто-то и про меня такое говорит (даже знаю кто  @igg{fbicon.face.smiling.eyes.smiling} ). Просто интересно разобраться

\iusr{Rustam Isakov}
\textbf{Елена Кулакевич} так а що там розбиратись, для нього визнана Константинополем ПЦУ це сектанти, а самопроголошена УПЦ МП істінниє православниє. У нього на сторінці там взагалі "константінопольскій патріархат отпал от РПЦ" ))))))

\iusr{Marina Grishin}
\textbf{Сергей Киселев}
Этого не было в истоках конфликта, это, как и многое другое привнесено позже... Вначале - язык...
\iusr{Елена Кулакевич}
\textbf{Rustam Isakov} 

вы будете очень удивлены, но для меня, не признанная тремя из четырёх древних,
и 14 из 15 автокефальных Православных церквей, ПЦУ тоже является
раскольническим объединением. Но сектой это назвать нельзя, т.к. это
противоречит самому определению понятия «секта». Поэтому, и хотелось бы
уточнения, что имелось ввиду. Особенно, если принять во внимание, что в ЗСУ
служит достаточно большое количество верных Украинской Православной Церкви. И
это факт, а не пропаганда, жертвой которой стали, в равной мере, к сожалению, и
вы, и автор комментария.


\iusr{Марина Ныкытенко}
\textbf{Rustam Isakov} , константінопольскій патріархат просто отпал, от чего?, дело десятое.

\iusr{Сергей Киселев}
\textbf{Елена Кулакевич} ПЦУ и УКГЦ это скорее не раскольники, а неоязычники. Поклоняются Бандере и Путину
\end{itemize} % }

\iusr{Сергей Сергеев}

Да, ведь претензии там не территориальные, и не из-за ресурсов идет война. А
языковая и мировоззренческая.

\begin{itemize} % {
\iusr{Игорь Лесев}
да

\iusr{Игорь Писаренко}
\textbf{Сергей Сергеев} сложно представить языковую войну

\iusr{Сергей Сергеев}
\textbf{Игорь Писаренко} зачем ее представлять, если она идет.

\iusr{Ирина Слюсарева}

"В Киеве, Москве, Брюсселе и Вашингтоне об этом не хотят говорить". - О чем не
хотят говорить в Москве, что там гражданская война? Только об этом и говорят,
вот уж не надо. Все 5 лет говорят: это гражданская война.

\end{itemize} % }

\iusr{Марина Прохорова}
Насчёт войны с Россией есть простой тест. Калибры на Банковую прилетали? - Нет? Ну вот и не пи...

\begin{itemize} % {
\iusr{Василий Стоякин}
Когда прилетают калибры и томогавки - это не война, а миротворческая операция
\end{itemize} % }

\iusr{Tatyana Meylakhs}

Забавное наблюдение: Украина - это антиРоссия, а есть ещё и АнтиУкраина, но не
Россия - зазеркалье какое-то)). Отличный пост, но я не поняла одного. Что
именно не признают в России? Что конфликт межэтнический? Так вроде кричат на
каждом углу, поэтому и автономию отстаивают. Но может, я вас не поняла

\begin{itemize} % {
\iusr{Игорь Лесев}

У Москвы такая же хитросделанная позиция, как и у Киева. Говорят, "конфликт
внутренний", но тут же называют Порошенко "коллегой". Рассказывают о
"бомбардировках Донбасса", но через Медведчука продают нефтепродукты Украине...
Такие же удобные красавчики, как и в Киеве.


\iusr{Tatyana Meylakhs}
\textbf{Игорь Лесев} 

в том что ведут себя непоследовательно и даже подло по отношению к жителям
Донбасса, согласна полностью. Но боюсь, что последовательность может привести к
трагическим результатам. Но я не государственный деятель. А пост хороший!


\iusr{Игорь Лесев}
\textbf{Tatyana Meylakhs} куда уже трагичнее... а за оценку спасибо

\iusr{Tatyana Meylakhs}
\textbf{Игорь Лесев} 

к сожалению никогда не бывает так плохо, чтобы не могло стать хуже. Эта аксиома
должна помогать людям быть ответственными в своих решениях.

\end{itemize} % }

\end{itemize} % }
