% vim: keymap=russian-jcukenwin
%%beginhead 
 
%%file 05_12_2021.stz.news.ua.tyzhden.1.intellektualni_zmagannja
%%parent 05_12_2021
 
%%url https://tyzhden.ua/Columns/50/253788
 
%%author_id ljutyj_taras
%%date 
 
%%tags kultura,ukraina,intellekt
%%title Інтелектуальні змагання
 
%%endhead 
\subsection{Інтелектуальні змагання}
\label{sec:05_12_2021.stz.news.ua.tyzhden.1.intellektualni_zmagannja}
\Purl{https://tyzhden.ua/Columns/50/253788}

\ifcmt
 author_begin
   author_id ljutyj_taras
 author_end
\fi

Матеріал друкованого видання \textSelect{№ 48 (732) від 1 грудня}

\textSelect{Заснована в 1632 році Київська колегія, згодом Києво-Могилянська академія
(КМА), певний час була чи не єдиним закладом старої України, де трималась
інтелектуальна традиція. Попри перешкоди з боку польської влади щодо намагань
створити університет як осідок православного істеблішменту, малоруську систему
навчання було налагоджено так, щоб ширити нове й наздоганяти згаяне.}

Дослідниця Валерія Нічик, яка простежувала впливи німецької культури на тодішню
нашу освіту, виявила, що навіть в епоху Просвітництва тутешні професори у свої
лекційні програми включали давно засвоєний західними мислителями спадок
схоластики, щоб прищепити в нас раціональні принципи пізнання. Якщо ними не
володіли, то годі було не лише конкурувати в теоретичних диспутах, а й
сподіватися на поважні цивілізаційні зрушення. Звісно, без уміння оперувати
абстракт­ними категоріями ніколи не вдасться нічого звершити в предметній
реальності. Європейські раціоналісти та емпірики ще в зачині індустріальної
доби вдавалися до критики авторитетів і шукали методів, спираючись на які
вдалося б сягнути посутніх інновацій у технологічній та інституційній царинах.
Водночас у КМА впроваджували заледве не перші заходи з синтезу наук і мистецтв.

Читайте також: \href{https://tyzhden.ua/Columns/50/253612}{Розкодовуючи знаки, %
Тарас Лютий, tyzhden.ua, 14.11.2021}

Далебі, виплекане в специфічних умовах знання приносить негадані плоди. Власне,
соціально-політичні реалії української старожитності не дають змоги говорити
про спільність еліти щодо вироблення релігійної чи державницької візій. Дарма,
що складалася вона переважно з випускників КМА. Інтелектуальному клімату,
зорієнтованому на відкритість у виборі стратегії розквіту, врешті забракло
спільного бачення. Нічик окреслила два основні напрями могилянської вченості:
науково-освітній, виразники якого схилялися до логіко-природничої проблематики
(Інокентій Гізель, Теофан Прокопович, Григорій Кониський) та
етико-антропософський з домінуванням поетико-символічного сприйняття й
містичних осяянь (Данило Туптало, Паїсій Величковський, Григорій Сковорода).
Спочатку вони врівноважувалися. Проте що примарніший вигляд мала далі автономія
країни, то дужче переважало друге спрямування, оскільки кожна особа мусила
покладатися на себе, перейматися внутрішнім світом і прагнути знайти незмінну
твердь, якою міг бути тільки Бог. Неконсолідована старшина й охоче до влади
духовенство не завжди могли похвалитися одностайністю. Хоча суспільне життя й
ототожнювали з морально скерованим існуванням у плинних обставинах, дехто з
адептів раціонального міркування, і тут на думку одразу ж спадає Прокопович,
оцінюючи перспективи державотворення, ставали фундаторами імперських структур.
Внутрішні екзистенційні кризи допроваджували до стану, коли питання формування
національної мови чи самобутності культури не стояли на порядку денному, бо
переважало бажання досягти нарешті щастя та заспокоєності.

\begin{zznagolos}
Іноді, коли окреслюють чинники, що визначали риси нашої культури, говорять про
дефіцит раціо. Частково це пояснюють історичні обставини, унаслідок яких і
виникала нагальна необхідність реагувати на зовнішньополітичні виклики
\end{zznagolos}

Невже це призводило до переваги застарілих форм освіти й припинення вивчення
новочасної наукової думки? Нітрохи. Про неї виученики й далі дізнавалися з
викладів і бібліотечних книжок. Рівнобіжно відбувалися спроби надолужувати
ознайомлення з поглядами таких схоластів і гуманістів, як Тома з Аквіни,
Марсілій Падуанський, Бонавентура, Петрарка, Данте, Макіавеллі та інші. Поза
тим іще Петра Могилу вабили ідеї Реформації: педагогіка, переклад і тлумачення
Біблії заради того, щоб «испытывати писания». Не уривалися й академічні
контакти, хоча вони й були спорадичними. Ґізель листувався з істориком Йоганом
Гербінієм, Стефан Яворський із Лейбніцем, Прокопович з гальським пієтистом
Авґустом-Германном Франке, а ректор Самуїл Миславський із Фрідріхом Християном
Баумейстером. 

У середині XVIII століття дехто з могилянців (Яків Козельський, Григорій
Козицький, Володимир Каліграф) тяжів до раціоналізму Християна Вольфа. До того
ж близько 20 українських слухачів відвідували в Кенігсбергу лекції Канта його
«докритичного» періоду, присвячені діалектиці, свободі волі, праву. З
вольфіанством був обізнаним і Сковорода. Та це принципово не визначало спосіб
його мислення. Він скептично поставився до поширення природознавства й віддавав
перевагу навчанню окремого зацікавленого люду. На переконання філософа, знати
про світ іще недостатньо для суспільного прогресу, позаяк освіта починається з
самопізнання. Видима природа лише натякає на приховану істину. Водночас
цілковито відкидати просвітницьку настанову в ученні Сковороди теж не варто.
Замість тотального проєкту розуму він пропонує індивідуальний, де знання про
людину та Всесвіт корелюють між собою. Утім, дійти цього вдається через
розуміння особистості.

Читайте також: \href{https://tyzhden.ua/Columns/50/253562}{Наука вільних %
мулярів, Тарас Лютий, tyzhden.ua, 07.11.2021}

Як бачимо, намір опановувати науки в українській традиції не переводився. Та
всяке живлення впливами мало б колись переростати в генерування самобутності
світового масштабу. Іноді, коли окреслюють чинники, що визначали риси нашої
культури, говорять про дефіцит раціо. Частково це пояснюють історичні
обставини, унаслідок яких і виникала нагальна необхідність реагувати на
зовнішньополітичні виклики. Однак вдаймося до умовного способу й припустімо
бодай на мить, що поява однодумців, а зрештою інституцій, які уможливили б
могутні поштовхи до розвитку світської науки, могли б суттєво змінити абрис
нашого культурного життя та здійснити потужний цивілізаційний крен. 
