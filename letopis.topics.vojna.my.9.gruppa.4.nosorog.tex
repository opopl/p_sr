% vim: keymap=russian-jcukenwin
%%beginhead 
 
%%file topics.vojna.my.9.gruppa.4.nosorog
%%parent topics.vojna.my.9.gruppa
 
%%url 
 
%%author_id 
%%date 
 
%%tags 
%%title 
 
%%endhead 

Бешеный Носорог

... Касательно россии, есть такой образ медведя, россияне сами им гордятся, и
всюду сюда его пихают, дескать, разбудили медведя, так теперь будет ойойой.  А
мне в этой связи вспомнилась другая аналогия, Носорог. Это дикий злобный зверь,
знаете.  На самом деле смертельно опасный зверь, которого не так уже и легко
укротить и повалить с ног, потому что он большой, тяжелый, грузный, очень злой
и дикий. Опасное очень животное, с которым с большим трудом и смертельно
рискуя, могут не справиться даже сотня человек.  Вот, почему то россия мне
сейчас больше напоминает Носорога, чем Медведя. Это не означает, что я не верю
в Победу Украины над россией, наоборот, очень даже верю и знаю, что так и
будет!  Но суть в том, что победа над злобным и диким зверем, как вы сами
понимаете, это нелегкая и тяжелая задача, и требует много усилий и большой
цены. И цена, которую мы все платим за то, чтобы этот Носорог в конце концов
свалился с ног, и был обезврежен, поистине колоссальна и велика. И этот дикий
зверь еще довольно силен, жесток, и пока что лично мне не видно, что он сам по
себе завтра вдруг свалится и умрет, сам по себе.  И да, в связи с этим мне
кажется, что у многих людей есть такое себе немного расслабленное и
легкомысленное в чем то отношение к нашей Победе.  Дескать, Победа будет
обязательно. Ну будет и все! Ну просто будет и все!!! А че, ты сомневаешься, ты
какой то странный, или что? Да будет Победа, и все!

А как она случится? Сколько еще времени потребуется?  Какую цену еще придется
заплатить? И что такое, по сути, есть Победа Украины, в чем она собственно
должна состоять?  В смерти путина? А если путин будет властвовать еще лет
десять? В Гаагском трибунале? А как организовать трибунал, если россия -
ядерная держава, и они сами в открытую говорят, что плевать им на Гаагу. В
прямом военном поражении россии по всем фронтам? А это точно будет? И так
далее, и так далее. Сразу как то возникает много вопросов...  И вопросов
сложных, трудных.  Знаете, верить очень хорошо, очень хорошо, но еще лучше,
когда Вера идет вместе с Разумом, потому что иначе есть риск впасть в такое
себе самоуспокоение, очень опасное самоуспокоение, дескать, считай Победа у нас
уже в кармане, так чего волноваться. А в связи с этим напомню, что пока что
россия до сих пор оккупировала часть Херсонской и Запорожской областей; много
людей живут в оккупации; Мариуполь был разбомблен и захвачен в итоге врагом;
десятки тысяч мирных жителей погибло; миллионы беженцев; постоянные ракетные
обстрелы Харькова, Николаева, и других городов; ВСУ пока еще не освободили даже
Херсон; Запорожская АЭС по прежнему под контролем орков, которые периодически
устраивают провокации и обстрелы, устраивая таким образом наглый и безумный
ядерный шантаж; на Донбассе пока что все сложно и довольно таки тяжело.  Очень,
очень большая, необычайно большая и страшная цена платится каждый день за то,
чтобы Украина была независимой, сильной и свободной державой!


