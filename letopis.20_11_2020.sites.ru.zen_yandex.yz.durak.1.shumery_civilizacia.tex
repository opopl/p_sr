% vim: keymap=russian-jcukenwin
%%beginhead 
 
%%file 20_11_2020.sites.ru.zen_yandex.yz.durak.1.shumery_civilizacia
%%parent 20_11_2020
 
%%url https://zen.yandex.ru/media/durak_pro/civilizaciia-kotoroi-mogli-dostatsia-znaniia-bogov-5fb6777215eeab2863a01c21
 
%%author Дурак (Яндекс Zen)
%%author_id yz.durak
%%author_url 
 
%%tags ancient,istoria
%%title Цивилизация, которой могли достаться знания богов
 
%%endhead 
 
\subsection{Цивилизация, которой могли достаться знания богов}
\label{sec:20_11_2020.sites.ru.zen_yandex.yz.durak.1.shumery_civilizacia}
\Purl{https://zen.yandex.ru/media/durak_pro/civilizaciia-kotoroi-mogli-dostatsia-znaniia-bogov-5fb6777215eeab2863a01c21}
\ifcmt
	author_begin
   author_id yz.durak
	author_end
\fi

\index[rus]{История!Шумер}

\ifcmt
pic https://avatars.mds.yandex.net/get-zen_doc/1921148/pub_5fb6777215eeab2863a01c21_5fb67793b543ac7480ae86ab/scale_1200
caption Цивилизация, которой могли достаться знания богов
\fi

\textbf{Самой ранней известной человеческой цивилизацией на Земле считается цивилизация
древних Шумеров, рассвет которой пришелся на период середины 4-го тысячелетия
(до н. э.), когда народы Месопотамии вступили в эпоху бронзы.}

Словом «нибиру» шумеры называли место откуда верховный бог в Древней
Месопотамии Мардук осуществляют управление миром.

\begin{leftbar}
	\begingroup
		\em «Не-беру его звезда, поставленная им сверкать в небе (Табличка VII,
				124-132) С помощью Не-беру держит он (Мардук) пересечение небес и
				подземного мира.»
	\endgroup
\end{leftbar}

Не зря символами Мардука являются мотыга и лопата – инструменты земледелия.
Потому что именно с земледелия начинается современная цивилизация, как таковая.
Ведь согласно версии официальной исторической науки, до цивилизации Шумер
человечество не имело, ни городов, ни государств, и на протяжении более 40-ка
тысячелетий жило родоплеменными общинами, занимаясь охотой и собирательством.

\subsubsection{Фатальный шаг к цивилизации}

Так продолжалось бы и дальше, если бы 7 тысячелетий назад боги не спустились на
Землю и не научили людей различным знаниям, в первую очередь земледелию.

Освоив земледелие люди стали вести осёдлый образ жизни, и концентрироваться в
поселениях возле плодородных земель.

Постепенно такие поселения росли, превращаясь в города, где людям приходилось
объединяться под управлением верховного лидера – царя, который, скорее всего,
избирался горожанами из числа самых уважаемых и авторитетных вождей.

\begin{leftbar}
	\begingroup
		\em Шумеро-аккадская цивилизация состояла из двух народов, говоривших на
				разных языках – шумеров и аккадцев.
	\endgroup
\end{leftbar}

Название «люди Шумера» было введено историками в середине XIX века для
обозначения древнего несемитского населения Месопотамии, язык которого также
получил название шумерского. 

\ifcmt
pic https://avatars.mds.yandex.net/get-zen_doc/1885164/pub_5fb6777215eeab2863a01c21_5fb67b3b0a790b7b98a8c4cc/scale_2400
caption Вероятные учителя человечества
\fi

Сами шумеры так себя не называли и не отделяли от другого местного населения —
семитоязычных аккадцев. Оба народа имели единое самоназвание — «черноголовые»
(шум. sag-gig-ga, аккад. ṣalmat qaqqadim), что в широком смысле означало
«люди».

\subsubsection{Уровень технологий}

От Шумеро-аккадской цивилизации человечеству не досталось ни одного чуда света.
Знаменитые Висячие сады Семирамиды достижение их наследников – Вавилонян.

Шумеры не умели обрабатывать мегалиты, как например, народы мезоамерики, и
строили свои зиккураты (культовые постройки) из глиняных кирпичей. Шумерский
кирпич был очень плохого качества, поэтому большая часть этих построек не
сохранилась.

Шумеры стали одной из первых цивилизаций, обзавёдшихся письменностью. Писали
они на глиняных табличках клинописью. К тому же, Шумеры разработали особую
технологию оттиска своих клинописных текстов и рисунков с помощью
цилиндрических печатей. Их по праву можно считать первой цивилизацией
придумавшей технологию книжной печати. Только «книги» древних Шумеров были
глиняными.

\subsubsection{Вероятные учителя человечества}

Вероятно, Шумеры приобрели все свои знания от более древних народов, живших в
Южной Месопотамии, до их появления. Это подтверждают ряд заимствованных в
Шумерском языке слов из более древнего, исчезнувшего протоевфратского языка,
таких как названия профессий: baḫar (горшечник), tibira (кузнец) и другие.
Иными словами, до появления в Месопотамии шумеры не обладали знаниями
металлургии и гончарного дела.

Некоторые лингвисты полагают, что загадочным древним народом у которого шумеры
позаимствовали свои знания были праиндоевропейцы.

\begin{leftbar}
	\begingroup
		\em Праиндоевропейцами принято называть некий народ предшественник
				индоевропейцев, от которых, в последствии, произошли более ста народов
				Европы и Азии, в том числе и славяне. 
	\endgroup
\end{leftbar}

Богов с планеты Нибиру шумеры называли аннунаками. Именно их шумеры считали
своими учителями – теми, кто подарил людям знания в готовом виде.

\ifcmt
  pic https://avatars.mds.yandex.net/get-zen_doc/3966998/pub_5fb6777215eeab2863a01c21_5fb67b206a210f1cc01ce12c/scale_2400
	cpx Вероятные учителя человечества
\fi

Поэтому, вполне вероятно, что резкий скачок развития человечества был
спровоцирован не естественными причинами, а кем-то извне.

Иначе откуда древние люди могли получить знания, изменившие их мировоззрение и
образ жизни?

А как вы думаете?
