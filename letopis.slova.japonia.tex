% vim: keymap=russian-jcukenwin
%%beginhead 
 
%%file slova.japonia
%%parent slova
 
%%url 
 
%%author 
%%author_id 
%%author_url 
 
%%tags 
%%title 
 
%%endhead 
\chapter{Япония}
\label{sec:slova.japonia}

%%%cit
%%%cit_head
%%%cit_pic
%%%cit_text
В \emph{Японии} перезапустили АЭС, давно отработавшую свой срок.
\emph{Японские} власти не только собираются слить в океан радиоактивную воду,
которая скопилась на аварийных реакторах «Фукусимы». Они демонстративно
перезапустили старую АЭС «Михама», хотя она отработала свой срок одиннадцать
лет назад.  «Михама» давно имеет печальную репутацию. Во-первых, АЭС
расположена на острове в морской бухте, что представляет собой серьезный риск
при возникновении цунами.  Во-вторых, в 2004 году там произошла крупная авария,
когда раскаленный пар сжег пятерых работников станции, а десятки человек
получили ранения. Причем, этот аварийный инцидент был отнюдь не первым
%%%cit_comment
%%%cit_title
\citTitle{Кино про это не снимут, санкции не наложат, Олимпиаду не заберут / Лента соцсетей / Страна}, 
Андрей Манчук, strana.ua, 24.06.2021
%%%endcit
