% vim: keymap=russian-jcukenwin
%%beginhead 
 
%%file 31_08_2019.stz.news.ua.mrpl_city.1.aviacia_obelisk_gorsad_mariupol
%%parent 31_08_2019
 
%%url https://mrpl.city/blogs/view/ko-dnyu-aviatsii-ukrainy-obelisk-v-gorodskom-sadu-mariupolya
 
%%author_id burov_sergij.mariupol,news.ua.mrpl_city
%%date 
 
%%tags 
%%title Ко Дню авиации Украины: обелиск в Городском саду Мариуполя
 
%%endhead 
 
\subsection{Ко Дню авиации Украины: обелиск в Городском саду Мариуполя}
\label{sec:31_08_2019.stz.news.ua.mrpl_city.1.aviacia_obelisk_gorsad_mariupol}
 
\Purl{https://mrpl.city/blogs/view/ko-dnyu-aviatsii-ukrainy-obelisk-v-gorodskom-sadu-mariupolya}
\ifcmt
 author_begin
   author_id burov_sergij.mariupol,news.ua.mrpl_city
 author_end
\fi

\ii{31_08_2019.stz.news.ua.mrpl_city.1.aviacia_obelisk_gorsad_mariupol.pic.1}

16 апреля 1944 года в воскресенье в Мариуполе было по-весеннему прохладно, но
солнечно. День этот был отмечен двумя событиями, позже вошедшими в историю
нашего города. По прошествии многих лет каждое из них обросло легендами. И как
это бывает, чем отдаленнее по времени публикация от свершившегося факта, тем
больше в ней предположений - и домыслы приобретают вид истины в последней
инстанции. Так, к сожалению, произошло и с событиями в Мариуполе 16 апреля 1944
года.

29 сентября 1943 года в бою над рекой Молочной погиб командир 104-го
гвардейского авиаполка, Герой Советского Союза майор \textbf{Владимир Григорьевич
Семенишин.} Тело отважного летчика было доставлено в Мариуполь и предано земле в
центральном сквере. Скромный обелиск над его могилой находился северо-западнее
фонтана по оси проспекта Республики (теперь – пр. Мира), примерно в 20 метрах
от центра фонтана. Поблизости от этого места во время оккупации немцы хоронили
своих солдат и офицеров. Но в первые же дни после освобождения Мариуполя кресты
выдернули, а могильные холмики сравняли с землей. Возможно, люди, которые
хоронили советского летчика, не знали о таком соседстве.

\textbf{Читайте также:} 

\href{https://mrpl.city/news/view/v-mariupole-zavershili-raskopki-po-e-ksgumatsii-nemetskih-i-rumynskih-soldat}{%
В Мариуполе завершили раскопки по эксгумации немецких и румынских солдат, Ярослав Герасименко, mrpl.city, 26.05.2018}

В феврале 1944 года в районе Гудермеса на Северном Кавказе в авиационной
катастрофе разбился заместитель командира 100-го полка 9-й гвардейской
Мариупольской истребительной авиадивизии капитан, Герой Советского Союза
\textbf{Николай Лавицкий.} Было решено похоронить Лавицкого в Мариуполе - в
Городском саду. А заодно перенести туда и прах Семенишина. Эта печальная
церемония как раз и произошла в воскресенье 16 апреля 1944 года. Проводить в
последний путь героев-летчиков пришли жители города - в основном женщины,
пожилые люди, школьники. Где была большая часть мариупольских мужчин - понятно:
почти три года шла кровопролитная война. Был в тот день в Городском саду и
восьмиклассник \emph{\textbf{Юрий Дзюман-Грек.}} Через два года он окончит с
золотой медалью школу, поступит в Ростовский машиностроительный институт,
окончит его с красным дипломом. Будет заниматься техникой строго засекреченной.
Станет главным конструктором одного из отделов на Тяжмаше. Будет награжден
орденом \enquote{Знак Почета}, медалями. Но это все будет потом.

А пока он стоял среди людей и вспоминал отца, погибшего на фронте. Кстати,
именно от Юрия Николаевича довелось узнать подробности событий, которые
происходили в Мариуполе 16 апреля 1944 года. Прозвучали речи, солдаты произвели
залп холостыми патронами, объявили, что траурный митинг закрыт. Часть
присутствовавших на похоронах отправилась пешком на аэродром, в том числе - и
Юрий. Там должны были передать самолет, купленный на деньги, собранные
учителями и учениками мариупольских школ, летчику 9-й гвардейской дивизии Герою
Советского Союза \textbf{Ивану Бабаку.} 

Где находился аэродром? Да там, где с семидесятых годов прошлого столетия
началось строительство Западного жилого массива. Примерно от современной улицы
Урицкого до улицы Троицкой. От того аэродрома осталось два объекта:
диспетчерская на улице Троицкой, сейчас в ней находится 5-я музыкальная школа
(правда, здание это построено уже после войны), и остатки ангара на улице
Урицкого. Кстати, коль скоро речь пошла о бывшем аэродроме, скажем, что во
время оккупации здесь базировались и взлетали отсюда немецкие Юнкерсы, Хейнкели
и Мессершмитты. Однополчанин Ивана Ильича Бабака - \textbf{Григорий Дольников} - в
воспоминаниях писал, что в составе делегации, которая принимала самолет, были
Бабак, Лавицкий и начальник политотдела дивизии полковник Дмитрий Мачнев.
Однако самого Дольникова в делегации не было. Значит, написано это с чужих
слов. И Лавицкий, и Мачнев приезжали в Мариуполь, но этот визит состоялся 3
января 1944 года, когда коллективу завода имени Ильича было передано почетное
знамя от летчиков прославленной дивизии.

\textbf{Читайте также:} 

\href{https://archive.org/details/17_08_2019.sergij_burov.mrpl_city.vokrug_skvera_istoria_centra_mariupoljaia-centra-mariupolja}{%
Вокруг сквера: история центра Мариуполя, Сергей Буров, mrpl.city, 17.08.2019}

А Иван Бабак? Иван Ильич признался мариупольскому журналисту Семену Гольдбергу,
что он не присутствовал на передаче самолета по той причине, что в то время
находился в госпитале. Кто же из летчиков на самом деле был на мариупольском
аэродроме 16 апреля 1944 года? Об этом довелось узнать от Юрия Николаевича
Дзюман-Грека. Самолет принимал командир дивизии полковник \textbf{Ибрагим Магометович
Дзусов}. После митинга полковник Дзусов сделал на этом самолете несколько кругов
над аэродромом. Мариупольский школьник Юра Дзюман-Грек тщательно зарисовал эту
боевую машину. Будучи уже взрослым, он решил сопоставить свой рисунок с
фотографиями самолетов времен войны. Оказалось, что на аэродроме был
американский истребитель \enquote{Кёртисс}. Как же так? Ведь известно, что именным
самолетом Бабака была \enquote{Аэрокобра}, да и весь полк летал на таких истребителях.
Почему получилась такая подмена - никто не знает. Вот такая история.

Сначала над братской могилой летчиков-героев был установлен скромный памятник с
пятиконечной звездой, окрашенной в красный цвет. А в 1948 году на его месте был
сооружен величественный обелиск. Известны его авторы – \textbf{скульптор Иван
Савельевич Баранников и архитектор Николай Иосифович Никаро-Карпенко.}
Торжественное открытие монумента состоялось 5 мая 1948 года.

Стоит над обрывом в Городском саду обелиск – дань памяти летчикам-побратимам,
Героям Советского Союза Владимиру Семенишину и Николаю Лавицкому, участникам
освобождения Мариуполя от немецко-фашист\hyp{}ских оккупантов.

Почему вспомнилась история с обелиском?

Да потому, что 31 августа – День авиации Украины.

\textbf{Читайте также:} 

\href{https://archive.org/details/11_05_2019.sergij_burov.mrpl_city.junost_kotoroj_ne_bylo_vospominania_vojna}{%
Юность, которой не было: воспоминания мариупольцев о военном времени, Сергей Буров, mrpl.city, 11.05.2019}

