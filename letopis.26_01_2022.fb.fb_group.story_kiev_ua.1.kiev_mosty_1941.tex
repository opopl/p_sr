% vim: keymap=russian-jcukenwin
%%beginhead 
 
%%file 26_01_2022.fb.fb_group.story_kiev_ua.1.kiev_mosty_1941
%%parent 26_01_2022
 
%%url https://www.facebook.com/groups/story.kiev.ua/posts/1848138372049622
 
%%author_id fb_group.story_kiev_ua,ljashenko_vadim
%%date 
 
%%tags 1941,istoria,kiev,most,nkvd,rkka,vov
%%title Взорванные в 1941 году киевские мосты
 
%%endhead 
 
\subsection{Взорванные в 1941 году киевские мосты}
\label{sec:26_01_2022.fb.fb_group.story_kiev_ua.1.kiev_mosty_1941}
 
\Purl{https://www.facebook.com/groups/story.kiev.ua/posts/1848138372049622}
\ifcmt
 author_begin
   author_id fb_group.story_kiev_ua,ljashenko_vadim
 author_end
\fi

Взорванные в 1941 году киевские мосты 

В ночь с 17 на 18 сентября 1941 г.  командира 4-й дивизии войск НКВД полковника
Федора Мажирина вызвал нарком внутренних дел Украинской ССР В. Сергиенко. 

Как только комдив вошел в кабинет, нарком сказал:

- Товарищ Мажирин, по решению Ставки Верховного Главнокомандовання наши войска
оставляют Киев, Вы назначаетесь комендантом города. В ваше распоряжение
переходят все военные части НКВД, милиции, истребительные отряды. На протяжении
ночи все защитники Киева  должны отойти на Левый берег Днепра. С этой минуты Вы
можете действовать в соответствии с обстановкой. Вам поручается взорвать мосты
через Днепр. Помните, за каждый мост вы отвечаете головой!

Вспоминает полковник Федор Мажирин, возглавлявший 4 дивизию НКВД, которой было
поручено взорвать киевские мосты: «День 19 сентября выдался на удивление
солнечным и теплым. Над Днепром чистое голубое небо. Часов в 11 утра фашисты
открыли ураганный огонь по юго-западным окраинам города, а затем осторожно
двинулись вперед, к мостам. Еще немного и их надо взрывать. Сигнал - и мы со
своего командного пункта увидели столбы  огня и дыма над железнодорожным мостом
имени Петровского - центральные фермы рухнули в воду. Разъяренные фашисты
попытались с ходу форсировать реку. Меткий пулеметный огонь с левого берега
отбросил их.

% 1-2
\ii{26_01_2022.fb.fb_group.story_kiev_ua.1.kiev_mosty_1941.pic.1}

Наводницкий деревянный мост был центральным, и основная масса арьергардных
частей выходила на него. Военный инженер 3 ранга А. Финкельштейн, отвечавший за
уничтожение этой переправы, выжидал до последнего момента, стараясь пропустить
последнюю группу отставших солдат. Лишь когда вражеские мотоциклисты вырвались
на берег и открыли ураганный пулеметный огонь, инженер подал сигнал.

\ii{26_01_2022.fb.fb_group.story_kiev_ua.1.kiev_mosty_1941.pic.3_4}

Облитый смолой и бензином, вспыхнул деревянный Наводницкий мост. И вот, прямо
перед нашим КП, на мосту сверкнула вспышка и черный султан взрыва подбросил
вверх искореженные  пролеты. На мгновение обнажилось песчаное дно, затем воды
Днепра сомкнулись, и вокруг  груды обломков посредине реки беспорядочно
заплясали волны. Потом у берега взвилось несколько водяных фонтанов: моряки
Днепровской флотилии уничтожили свои корабли. Над Днепром повисло облако дыма и
пыли.

\ii{26_01_2022.fb.fb_group.story_kiev_ua.1.kiev_mosty_1941.pic.5_6}

Бойцы, охранявшие Наводницкий мост на правом берегу, отходили по уже горевшему
настилу. Вслед за ними кинулись фашистские автоматчики. Саперы, дождавшись,
когда наши бойцы ступили на землю, взорвали толовые шашки, привязанные к сваям,
и пылающий мост обрушился в Днепр, под своими обломками похоронив вражеских
солдат. И вот уже просто перед командным пунктом комдива на мосту имени Евгении
Бош блеснула вспышка, и черный столб взрыва подбросил вверх изуродованные
конструкции. Над Днепром повисло облако дыма и пыли".

\ii{26_01_2022.fb.fb_group.story_kiev_ua.1.kiev_mosty_1941.pic.7_8}

Мажирин связался с командиром 87-й стрелковой дивизии, чтобы согласовать
дальнейшие действия. Частям арьергарда, в которые входили части 87-й стрелковой
дивизии,  железнодорожники 31-го мостового батальона 25-й железнодорожной
бригады, поддерживаемые огнем бронепоездов № 31, 34 и 66 внутренних войск НКВД,
было приказано держаться до наступления темноты, а затем отходить в общем
направлении на Борисполь.

Свидетели утверждают, что приказ о немедленном взрыве моста Евгении Бош 19
сентября стал полной неожиданностью для арьергарда отступающих советских войск
и сотен эвакуировавшихся киевлян. Множество людей погибло при взрыве и
обрушении ферм моста. Из советского учебника \enquote{Инженерные войска} известно, что
при подготовке обороны Киева три взвода спецминирования устанавливали радиомины
на мостах через Днепр.

Из доклада штаба Юго-Западного фронта начальнику штаба главного командования
Юго-Западного направлений о состоянии охраны и обороны переправ через р. Днепр

Переправы №№ 6, 7, 8, 9 и 10 – киевские мосты :

Переправа № 6 у Воскресенская сл. – ж.-д. мост им. Петровского. Мост
подготовлен для пропуска танков и механического транспорта. Охрана 100 чел. 56
ждп. Вооружена: два станковых пулемета, три ручных пулемета, 2000 гранат, 85
активных винтовок. Комендант – капитан Гондельбер.

Переправа № 7 у Никольская Слобода состоит из двух мостов:

а) мост Евгения Бош – охрана 100 чел. 56 ждп. Вооружены: два станковых
пулемета, три ручных пулемета, 85 активных винтовок, 200 гранат;

б) мост Русановский – охрана 60 чел. 56 ждп. Вооружение: два станковых
пулемета, мототанкер и мотолодка. Начальник охраны – командир 2 б-на 56 ждп.
Комендант – полковник Кузмин – 17 понтонного полка.

Переправа № 8 у Предмостная, мощный понтонный мост. Охрана – рота понтонеров 1
понт. полка. Комендант – подполковник Федоров – 17 понт. полка.

Переправа № 9 – наводницкий мост. Охрана моста 80 человек 56 ждп. Вооружение: 2
станк. Пулемета, 3 ручн. пулемета, 60 активных винтовок, 1800 гранат, одна
мотолодка.

Комендант – ст. лейтенант Ивницкий от 17 понт. полка.

Переправа № 10 – дарницкий ж.-д. мост, приспособлен для движения механического
транспорта. Охрана 100 человек 56 ждп. Вооружение: два станк. пулемета, три
ручн. пулемета, 35 активных винтовок, 260 гранат. Начальник охраны моста – нач.
штаба 2 батальона 56 ждп. Комендант – майор Тихонов – Инжуправления.

Районы переправ 6, 7, 8, 9, 10 имеют окопы для круговой обороны частей охраны.

(Ф. 229, оп. 9776с, д. 75, лл. 33-38. Незаверенная машинописная копия)

ФОТО:

01. 19 сентября 1941 года, оставляя Киев, части Красной Армии взорвали мост
имени Бош, который киевляне по привычке называли \enquote{Цепным}, и проложенный за ним
по направлению к Броварскому шоссе Русановский мост. Сегодня примерно на этом
месте находится мост Метро (А. Анисимов).

02. Новый мост на месте старого разрушенного Цепного моста был открыт 10 мая
1925 года. Он был назван в честь революционерки и участницы Гражданской войны
на Украине Евгении Богдановны Бош, покончившей жизнь самоубийством в том же
году, за несколько месяцев до открытия моста.

03. Так выглядел мост Евгении Бош в 1930-е годы.

04. Расчет советской 37-мм зенитной автоматической пушки обр. 1939 г. на боевой
позиции у моста Е. Бош, июль-август 1941 года.

05. Мост защищают катера Днепровской флотилии, рисунок Г. Малакова.

06. Легкий танк Т-26 и пикап ГАЗ-М-415 на мосту Евгении Бош, август 1941 года.

07. Эвакуанты с Киева уходят на левый берег по мосту им. Е. Бош, август 1941
года.

08. Дым над Киевом, вдалеке виден разрушенный мост Е. Бош, сентябрь 1941 года.

\ii{26_01_2022.fb.fb_group.story_kiev_ua.1.kiev_mosty_1941.cmt}
