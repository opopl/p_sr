% vim: keymap=russian-jcukenwin
%%beginhead 
 
%%file 16_03_2023.fb.fb_group.mariupol.pre_war.2.svyato_uspenskii_khr.cmt
%%parent 16_03_2023.fb.fb_group.mariupol.pre_war.2.svyato_uspenskii_khr
 
%%url 
 
%%author_id 
%%date 
 
%%tags 
%%title 
 
%%endhead 

\qqSecCmt

\iusr{Elena Ganzina}

В этом храме мы крестили сына. Отец Вадим служил. Хороший был храм. Печально...
Помню там был маленький домик и начинал служить отец Николай. А теперь ни
чего😭😭😭

\begin{itemize} % {
\iusr{Світлана Федоровська}
\textbf{Elena Ganzina} так спочатку настоятелем був отець Миколай. Потім був отець Олексій. І багато років наш отець Дімітрій. Отець Вадим служить у Свято- Покровському храмі. Стільки щасливих годин на Пасхальне богослужіння,стільки радісних днів в недільній школі з нашими дітьми,стільки подій пов'язано з нашим любим храмом. Але орки знищіли все.

\iusr{Chirkina Tatyana}
\textbf{Елена Ганзина} дай Боже восстановят, надо🙏💕 надеется на лучшее

\begin{itemize} % {
\iusr{Іванова Яна}
\textbf{Chirkina Tatyana} то замість нього зробили коробку і то вже зовсім інша споруда... від колишньої церкви нічого схожого не залишилось. От і "восстановленіє".

\iusr{Світлана Федоровська}
\textbf{Chirkina Tatyana} хто відновить? Там окупація. Та і відновляти немає що,все знищено.

\iusr{Chirkina Tatyana}

Если вы так считаете пусть будет по вашему я не спорю, слава🙏 Богу! Я всё вижу

\iusr{Світлана Федоровська}
\textbf{Chirkina Tatyana} це не я так вважаю. Це болюча реальність,від храму нічого не лишили.

\iusr{Elena Ganzina}
\textbf{Чиркина Татьяна} 

вы считаете, что восстановят -может быть, но кто? Эти орки? Этот храм строили на
подаяния прихожан, этот храм был светлым душевно, а что будет ... Что будет время
покажет...

\iusr{Екатерина Воробьёва}
\textbf{Чиркина Татьяна} всевидящая пророчица Татьяна)))

\end{itemize} % }

\iusr{Лилия Букина}
\textbf{Елена Ганзина} Я ходила в храм, когда это был маленький домик. Служил наш батюшка Николай. Он же и крестил мою внучку.

\begin{itemize} % {
\iusr{Світлана Федоровська}
\textbf{Лилия Букина} на превеликий жаль,немає нашого чудового храму.

\iusr{Лилия Букина}
\textbf{Світлана Федоровська} 

Как жаль! Столько светлых воспоминаний связано с этим храмом! Много лет назад
мы с сыном впервые приняли причастие. И в этот день я приобрела 2 больших
иконы- Матерь Божия с младенцем и икону Иисуса Христа.Эти иконы просуществовали
у меня до 15-го марта 2022 г. Ве

\iusr{Світлана Федоровська}
\textbf{Лилия Букина} 

так це дуже боляче. Наш храм був для мене як другий дім, там я вперше
сповідалася та прийняла святе причастя. Там хрестили мого синочка, та він ходив
у недільну шкоду. Там було стільки доброго що все розповідати можна годинами.

\iusr{Лилия Букина}
\textbf{Світлана Федоровська} Поздно вечером сгорела моя квартира со всеми другими иконами и со всем моим имуществом. Я в это время пряталась в школьном подвале.

\iusr{Світлана Федоровська}
\textbf{Лилия Букина} розумію вас,наше житло теж знищіли і наші ікони. Ви живі,слава Господу!

\iusr{Лилия Букина}

Тогда у окошечка работала т. Люба. Помню и других бабушек. Столько хороших и
светлых воспоминаний! Ну, а потом ходила в ближний храм, возле дома. В
Михайловский Собор. А также часто ездила в храм Марии Магдалины. Там тогда
служил о. Николай. Когда его не стало, то его зять Алексей.

\iusr{Chirkina Tatyana}
\textbf{Лилия Букина} на Огробазе, светлая🙏💕 память😔 о. Николаю Царствие Небесное он крестил моих дочерей, он ещё людей исцелял, быть добру❤😘
\end{itemize} % }

\iusr{Nika Gree}
\textbf{Елена Ганзина} я там крестила дочь.

\end{itemize} % }

\iusr{Татьяна Пахомова}

Храм восстановили, только вход с другой стороны, с улицы Станиславского

\begin{itemize} % {
\iusr{Світлана Федоровська}
\textbf{Татьяна Пахомова} ви мабуть щось плутаєте. На місці де був Свято-Успенський храм поставили якусь собачу будку, зроблену з дощок. На неї заперли вцелілий купол. Це тепер якась не зрозуміла організація.

\begin{itemize} % {
\iusr{Татьяна Пахомова}
\textbf{Світлана Федоровська} не буду спорить, я не заходила, но вижу вход в храм сделан с пластика

\iusr{Світлана Федоровська}
\textbf{Татьяна Пахомова} це не храм. Це якась секта.

\iusr{Татьяна Пахомова}
\textbf{Світлана Федоровська} даже не знаю, что там, не заходила, надо спросить у жителей, кто рядом живут

\iusr{Світлана Федоровська}
\textbf{Татьяна Пахомова} спитайте. Але ті хто ходив навколо Азовсталі щоб \enquote{вигнати демонів}, наврядчи зможуть нормальну відповідь дати. Страшні часи.
\end{itemize} % }

\end{itemize} % }

\iusr{Victoria Rohozhyna}

Це той храм, що був біля 51-ї школи? Місця мого дитинства, мого щасливого
життя. Ми там хрестили сина і я сама там приймала хрещення (до 28 років була
нехрещеною).

\iusr{Світлана Федоровська}
\textbf{Victoria Rohozhyna} так.

\iusr{Nika Gree}

Мамин дом на этой улице
