% vim: keymap=russian-jcukenwin
%%beginhead 
 
%%file 08_05_2022.fb.demsokyra.1.gumdopomoga_gruzia
%%parent 08_05_2022
 
%%url https://www.facebook.com/sokyra.space/posts/1471258540006087
 
%%author_id demsokyra
%%date 
 
%%tags 
%%title Гуманітарна допомога з Грузії
 
%%endhead 
 
\subsection{Гуманітарна допомога з Грузії}
\label{sec:08_05_2022.fb.demsokyra.1.gumdopomoga_gruzia}
 
\Purl{https://www.facebook.com/sokyra.space/posts/1471258540006087}
\ifcmt
 author_begin
   author_id demsokyra
 author_end
\fi

Якби речі могли говорити, ви б дізналися, що

У далекій Грузії чуйні люди зібрали для українців, які потерпають від війни та
путінської орди, гуманітарну допомогу – продукти харчування, ліки, медичні та
гігієнічні засоби, ковдри, матраци та постіль, а також речі для малюків.

%\setotherlanguage{georgian}

Завдяки іншим небайдужим людям та
\href{https://www.facebook.com/gpost.ge}{საქართველოს ფოსტა/Georgian Post} ці
речі приїхали в Україну – до Львова, де наші невтомні волонтери хабу
\href{https://www.facebook.com/sokyra.space.lviv}{Демократична Сокира. Львів}.
Ann  Baru, Татуся Бо, Марія Грицай, Maxim  Parenko та ще багато добрих людей
перевантажили це в іншу вантажівку і за допомогою наших постійних партнерів з
Укрпошта доправили у Київ.

У столиці Ярослав Зінченко, Constantin Zaliznyak, Владислав Гребельник, Єлісей
Ходоловський та інші активні люди з Демократична Сокира. Київ розвантажили,
посортували й доправили на склади цінний гуманітарний вантаж, звідки він
роз'їжджається по різних куточках Київщини та України.

Так, сьогодні частина «гуманітарки» поїхала разом із Олена Почтарьова на
Черкащину, де волонтерський штаб Демократична Сокира. Черкаси допомагає
переселенцям, літнім людям, місцевим лікарням та вихованцям дитячого будинку.
Вже від завтра допомогу зможуть отримати ті, хто її потребує.  

У темні часи добре видно світлих людей. І ми тішимося, що вони гуртуються у та
навколо «Демократичної Сокири». 

Тішимося і не втомлюємося дякувати їм: 

-  Колективу «Укрпошти», завдяки якому вдається доставляти сотні тон
гуманітарних та спеціальних вантажів, продуктів та засобів гігієни з-за кордону
та по Україні.  Особисто Igor Smelyansky, Юлії Павленко, Наталі Прокоповій,
Tanya Shkryum, та Игорь Поликарпов

-  Анастасії Орел, яка шукає і знаходить для нас фури для транспортування
вантажів по Україні.

-  Юра Сиділо, який надав Львівському волонтерському хабу можливість
користуватися власним складом для гуманітарних вантажів

-  Микола Яненко, який виручає на Черкащині й не лише з орендою авто

-  Родині Тертичних та компанії з Хмельницького TextilHill - текстиль для дома
за безкоштовну постільну білизну, турботливо передану через Татусю Бо нашим
підопічним. 

І ще багатьом-багатьом добрим людям, малим підприємцям і великому бізнесу, які
допомагають нам допомагати іншим, і чиї імена ми ще не раз згадаємо у наших
звітах. 

Працюємо. Перемога буде!
