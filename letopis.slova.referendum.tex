% vim: keymap=russian-jcukenwin
%%beginhead 
 
%%file slova.referendum
%%parent slova
 
%%url 
 
%%author 
%%author_id 
%%author_url 
 
%%tags 
%%title 
 
%%endhead 
\chapter{Референдум}
\label{sec:slova.referendum}

%%%cit
%%%cit_head
%%%cit_pic
%%%cit_text
1 декабря 1991 года на \emph{референдуме} граждане Украины поддержали независимость,
живя еще в рамках советских правовых норм, но - с новой принятой и признанной
Конституантой - Декларацией о суверенитете-1990.  Несмотря на весь публичный
пафос, вся последующая политико-правовая работа - от Конституционного договора
до последних правок в Конституцию-1996 - фактическое де-конституирование той
общности-нации, которая учредила Украину-1991. Может, потому весь путь
независимости - путь потерь, бесконечных конфликтов и даже войны. Миллионы - за
пределами гражданского пространства, сотни тысяч - мигранты, внутренние
конфликты все больше приобретают характер классовых конфликтов и \enquote{культурных
расколов} на этнической и религиозной почве. Угрозы новых потерь только
нарастают
%%%cit_comment
%%%cit_title
\citTitle{Весь путь нашей независимости - путь потерь, бесконечных конфликтов и даже войны / Лента соцсетей / Страна}, 
Андрей Ермолаев, strana.ua, 30.06.2021
%%%endcit

%%%cit
%%%cit_head
%%%cit_pic
\ifcmt
  pic https://strana.ua/img/forall/u/0/36/2021-07-05_16h38_53.png
	width 0.4
\fi
%%%cit_text
\enquote{Переименовать могли только придурки, если честно. И те, которые в Верховной
(З)раде проголосовали за переименование, я бы им поотрывал то, что у них есть,
поотрывал бы с ходу. У кого из жителей поселка [спросили]. Что, \emph{референдум}
провели? Спросили мнения?  Это же как-то несерьезно ни разу}, - ответил нам
противник переименования.  Он говорит о том, что большая часть депутатов не
знает, что такое Конституция, а \enquote{президент фактически узурпировал власть}.
\enquote{Ну, какой Нью-Йорк? Я вообще был против всех этих переименований. Это сколько
денег пошло. И по большому счету если что-то делается, то после этого должно
наступать что-то лучшее. Мне кажется, хоть Вашингтон, хоть Нью-Йорк, хоть
Брюссель, хоть Берлин, лучше не станет, потому что хозяина нет}, - добавляет
наш собеседник
%%%cit_comment
%%%cit_title
\citTitle{Гетьман в Нью-Йорке. Как Порошенко устроил черешня-тур на Донбасс}, 
Оксана Малахова, strana.ua, 06.07.2021
%%%endcit
