% vim: keymap=russian-jcukenwin
%%beginhead 
 
%%file 16_03_2018.stz.news.ua.mrpl_city.1.vtracheni_hramy
%%parent 16_03_2018
 
%%url https://mrpl.city/blogs/view/vtracheni-hrami-mariupolya
 
%%author_id demidko_olga.mariupol,news.ua.mrpl_city
%%date 
 
%%tags 
%%title Втрачені храми Маріуполя
 
%%endhead 
 
\subsection{Втрачені храми Маріуполя}
\label{sec:16_03_2018.stz.news.ua.mrpl_city.1.vtracheni_hramy}
 
\Purl{https://mrpl.city/blogs/view/vtracheni-hrami-mariupolya}
\ifcmt
 author_begin
   author_id demidko_olga.mariupol,news.ua.mrpl_city
 author_end
\fi

Поставлю риторичне питання тим читачам, які народилися і виросли в Маріуполі:
чи хотіли б ви побачити рідне місто 100 років тому, відчути атмосферу іншого
століття, познайомитися з його вулицями, архітектурою? Впевнена, що більшість
буде зовсім не проти здійснити подорож у часі, побачити будівлі, які збереглися
лише на світлинах, відчути себе причетним до далекого, сповненого таємниць
минулого.

\ii{16_03_2018.stz.news.ua.mrpl_city.1.vtracheni_hramy.pic.1}

Немає нічого неможливого. Спробуємо повернутися в минуле Маріуполя XIX ст. і
уявити місто, яке нараховувало понад десять великих споруд релігійного
призначення. Серед них – християнські храми, каплиці та собор, католицький
костел і кілька єврейських синагог. На жаль, через антирелігійну кампанію в
СРСР було встановлено, що всі культові споруди не представляють ні художньої,
ні історичної цінності, тому повинні бути використані за іншим призначенням або
знищені. На початку тридцятих років в Маріуполі знесли каплицю Святої Марії
Магдалини на перехресті сучасного проспекту Миру і вулиці Грецькій, що заважала
прокладанню трамвайної колії, а до кінця десятиліття в місті не залишилося
жодного храму. Більше пощастило синагогам, які більшовикам не так заважали, як
християнські храми.

\ii{16_03_2018.stz.news.ua.mrpl_city.1.vtracheni_hramy.pic.2}

Ну що ж, спробуємо відновити в уявленні декілька маріупольських храмів.
Почнемо з найбільшого і головного храму Маріуполя – Харлампієвського собору
(Собору Святого Харлампія). Він був закладений на місці ДОСААФу на Соборній
(Базарній) площі 10 травня 1831 року. Споруджували новий собор понад десять
років на державні та благодійні кошти. У 1845 р. в центрі Маріуполя був
освячений чудовий храм у візантійському стилі з трьох прибудов. Центральний
боковий вівтар був зведений на честь великомученика Харлампія, саме тому весь
собор став називатися Харлампіївським, правий – на честь Георгія-Побідоносця, а
лівий був присвячений святителю Миколаю, в пам'ять про запорізьку
Свято-Миколаївську церкву.

Святителя Миколу православні парафіяни особливо шанували. У 1891–1892 рр.
збудували нову дзвіницю і з'єднали її з церквою, після чого до Харлампієвського
собору могли увійти п'ять тисяч осіб одночасно. Дзвіниця стала найвищою
спорудою міста, орієнтиром для мореплавців і мандрівників. У напрямку всіх
чотирьох сторін світу з висоти на Маріуполь дивилися циферблати баштового
годинника. Діаметр циферблата становив близько двох метрів, довжина
годинникової стрілки – 50 см, хвилинної – 75 см. Загальна вага всього механізму
годинника досягала 800 кілограмів. Це було найбільше приміщення в історії
Маріуполя. Харлампієвський собор став осередком майже всіх святинь і
національних цінностей маріупольських греків. Зокрема, тут зберігалася
чудотворна ікона Георгія Побідоносця, що датується серединою XI століття.
Всередині містилася усипальниця митрополита Ігнатія, Євангелія видання 1671 і
1740 років, хрест з часткою Животворящого Хреста Господнього 1767 року,
плащаниця, великий патріарший хрест, інші реліквії, жалувані грамоти Катерини
II та Олександра I, всі основні святині греків-переселенців. У 1937 році
найбільший собор Маріуполя зруйнували більшовики. Останки митрополита Ігнатія
перемістили в краєзнавчий музей, а коштовності вилучила держава.

\ii{16_03_2018.stz.news.ua.mrpl_city.1.vtracheni_hramy.pic.3}

На місці драматичного театру знаходилася Церква Святої Марії Магдалини на
Олександрівській площі (нині сквер біля драматичного театру) – фундамент
закладений ще в 1862 р., але церква була освячена лише в 1897 р. Тоді ж навколо
храму, що вражав мешканців своєю красою, було закладено міським садівником Г.Г.
Псалті великий красивий сквер, який у наші дні став улюбленим місцем відпочинку
маріупольців. Церква мала три престоли: її головний престол був присвячений
святій рівноапостольній Марії Магдалині, правий – Покрову Пресвятої Богородиці,
лівий – святому пророкові Предтечі й Хрестителя Господнього Іоанну. У
Марії-Магдалинівській церкві серед інших цінностей зберігалася запрестольна
ікона Спасителя часів запорізьких козаків. Храм був знесений в першій половині
1930-х рр.   Можливо, вже 2018 року маріупольці зможуть побачити бронзову
інсталяцію Церкви Святої Марії Магдалини, що посприяє ознайомленню з найбільш
чарівним, але, на жаль, втраченим храмом Маріуполя.

\ii{16_03_2018.stz.news.ua.mrpl_city.1.vtracheni_hramy.pic.4}

На перетині сучасного бульвару Шевченка і вулиці Куїнджі розташовувався більш
скромний храм Різдва Богородиці, що веде свою історію з 1780 року. Його стіни
й дзвіниця були складені з червоної цегли, а купол – з дерева. І цей храм мав
свої чудодійні реліквії. У ньому зберігалися ікони Косьми та Даміана, а також
святої Параскеви, вишита шовком плащаниця, чаша, кадило і два древніх
Євангелія. У стінах цієї церкви прийняв хрещення видатний земляк Маріуполя А.
І. Куїнджі. У 1875 році там же він обвінчався з Вірою Леонтіївною
Кечеджи-Шаповаловою. Після своєї смерті в 1910 році Архип Іванович залишив 10
000 рублів на облаштування школи при церкві. У 1937 році храм був підірваний,
а на його місці побудували школу № 11. Частину предметів цього храму можна
побачити в експозиції музею імені Архипа Куїнджі. Там зберігається Євангеліє і
купіль, в якій проводилися обряди хрещення.

Чи в силах ми сьогодні зберегти архітектурну спадщину Маріуполя? Чи потрібно
буде майбутнім поколінням уявляти те, що існує зараз? Відповіді на ці питання
можна буде дізнатися з часом... Сьогодні ж у Маріуполі планують відремонтувати
фасади кількох будівель навколо Театральної площі, і, якщо вірити словам
головного архітектора Маріуполя Олександра Каверіна, місто повинно заграти
новими фарбами. Ну що ж, будемо сподіватися, що кращі зміни попереду, і
намагатися слідкувати за подіями, що відбуваються навколо архітектурної
спадщини міста, адже її подальша доля залежить від кожного з нас.
