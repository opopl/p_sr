%%beginhead 
 
%%file 16_09_2022.fb.kutnjakova_maria.zhurnalist.vilnjus.mariupol.1.6_misjaciv_tragedia_dramteatr
%%parent 16_09_2022
 
%%url https://www.facebook.com/m.kutnyakova/posts/pfbid02jcEH8CJ96tpq92ZaDdpFKAFFuaHijsyjmvjDP3CDYuxqUycxvxPhrN3wt2YZevFbl
 
%%author_id kutnjakova_maria.zhurnalist.vilnjus.mariupol
%%date 16_09_2022
 
%%tags teatr.mariupol.drama,mariupol.war,mariupol
%%title 6 місяців з трагедії Драматичного театру у Маріуполі
 
%%endhead 

\subsection{6 місяців з трагедії Драматичного театру у Маріуполі}
\label{sec:16_09_2022.fb.kutnjakova_maria.zhurnalist.vilnjus.mariupol.1.6_misjaciv_tragedia_dramteatr}

\Purl{https://www.facebook.com/m.kutnyakova/posts/pfbid02jcEH8CJ96tpq92ZaDdpFKAFFuaHijsyjmvjDP3CDYuxqUycxvxPhrN3wt2YZevFbl}
\ifcmt
 author_begin
   author_id kutnjakova_maria.zhurnalist.vilnjus.mariupol
 author_end
\fi

6 місяців з трагедії Драматичного театру у Маріуполі

Це досі не вкладається у голові. Як це взагалі можливо? Я бачила все це на
власні очі. Але мій мозок не може це прийняти.

16 березня, ранок, стало нарешті трошки тепліше, горів Центральний ринок і
Універмаг, тож над центром було багато диму. А потім я чую літак. Десь дуже
поруч. Він скинув бомби і полетів. "Оце мені пощастило," - подумала я і вийшла
до скверу. Театр вже горів з третього поверху, дві бокові частини завалилися,
дах розлетівся уламками. І найстрашніше це були крики: десятки людей
викрикували імена, просто ридали над тілами, голими руками розбирали завали,
поранені стогнали. Мені знадобилася десь хвилина, щоб у мовчазному замороженні
усього тіла усвідомити: будівля зруйнована, а десь там мої рідні. 

Не хочу розказувати подробиці моїх блукань навколо і в середині будівлі. Це
були найстрашніші хвилини мого життя. Шукати свою маму там і бачити на власні
очі понівечених людей, жінок, які рвали на собі волосся від відчаю і плачущих
чоловіків, які вибивали двері, щоб дати вийти людям з різних приміщень. Але мої
знайшлися. Вони всі отримали незначні ушкодження, але вижили. Тепер я вірю у
дива. 

Я багато про той день думала. Спілкувалася десь з 30 людьми, які теж вижили в
театрі, хтось з них втратив там близьких. І ось що мені не дає спокою.

Того ранку десь на іншому березі Азовського моря в літак сів пілот. Він бачив
завод Азовсталь, стару частину міста, сквер та червоний помітний дах театру. Ні
з чим не можна переплутати. Не знаю чи бачив він 2 велетенських написи "Діти"
біля театру, але в нього точно була конкретна ціль. І він скинув 2 бомби. По
500 кг. Від них зазвичай залишається воронка в кілька десятків метрів, а
будівлі від них розлітаються дуже легко. Він скинув, вбив сотні людей та дітей,
і повернувся. Ввечорі прийшов до дому і поїв страви дружини, подивився новини і
ліг спати. Щоб потім знову летіти через море і скидати бомби.

З путіним, шойгу і т.п. все ясно. А ось ці виконавці — як їм спиться ночами?
Між собою ворог називав льотчиків "маестро", їм напевно дають нагороди, і може
театр, то і не перша трагедія, автором якої стали такі льотчики. 

Як так можна? Я не розумію людей, які вірять, що цю трагедію не влаштувала
росія. Я не розумію людей, які допомагають ворогу після того, що він там
зробив. Я не розумію людей, які захищають колоборантів після цього. Це довгий
ланцюжок мого нерозуміння.

Але я і не хочу усвідомлювати до кінця. Мені інколи здається, що всього цього
не було. Ба більше, що й мого мирного життя не було — я його вигадала. І ще
страшніше — що не було дивовижного спасіння. Що мої рідні загинули, а все
подальше — моя уява. А може і я там загинула. Хоча якась доля правди в цьому є.
Ми всі трошки померли у Маріуполі в ту кляту весну. Старих нас нема. Старим тут
не місце. 

Світла пам'ять всім загиблим, покарання для їх вбивць.
