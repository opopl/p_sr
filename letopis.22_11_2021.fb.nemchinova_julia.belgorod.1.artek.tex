% vim: keymap=russian-jcukenwin
%%beginhead 
 
%%file 22_11_2021.fb.nemchinova_julia.belgorod.1.artek
%%parent 22_11_2021
 
%%url https://www.facebook.com/permalink.php?story_fbid=1035750020330370&id=100016861452041
 
%%author_id nemchinova_julia.belgorod
%%date 
 
%%tags artek.detskij.lager,deti,rossia,zhizn
%%title Что же такое Артек?
 
%%endhead 
 
\subsection{Что же такое Артек?}
\label{sec:22_11_2021.fb.nemchinova_julia.belgorod.1.artek}
 
\Purl{https://www.facebook.com/permalink.php?story_fbid=1035750020330370&id=100016861452041}
\ifcmt
 author_begin
   author_id nemchinova_julia.belgorod
 author_end
\fi

Что же такое Артек? 

Это когда находишь таких же как ты увлечённых, таких же активных и
позитивных...

10 лет Даша учится музыке. С 5 лет был кружок при Дворце детского творчества,
потом в 8  лет  решили все же продолжить обучение  в музыкальной школе, поэтому
пришлось идти с самого начала, в первый класс. Все эти 10 лет не было особого
интереса или желания, как все, были и слезы и желание бросить, убеждали,
уговаривали... Я говорила, что это как твоя работа. Хочется или не хочется, а
надо... 

И вот, в последний год, вдруг Даша поняла, что такое музыка. Поняла, что это
круто, кода ты умеешь играть. Началось все с рояля на Везелке. Раз поиграла,
другой...

- А ты можешь мелодию из того фильма?

Спросила ее сверстница.

-Нет, не могу. Но я выучу. Приходи на следующей неделе. 

Так Даша начала разучивать то, что ей нравится, то, что восхищает ее друзей,
свертников...

\ifcmt
  ig https://scontent-frx5-1.xx.fbcdn.net/v/t39.30808-6/260404583_1035749950330377_2792449873308583425_n.jpg?_nc_cat=110&ccb=1-5&_nc_sid=8bfeb9&_nc_ohc=-6CXPvLhZ08AX9jPmZi&_nc_oc=AQkI3kz_USWdaVMI4r5FGgJhJlyjwJK9MkP-0DC5rkP0PwctyizIv68YT5UKlEzB_UU&_nc_ht=scontent-frx5-1.xx&oh=4dd14d2c613c57fe57986cec97d60be4&oe=61A2AE07
  @width 0.4
  %@wrap \parpic[r]
  @wrap \InsertBoxR{0}
\fi

Сначала разучила мелодию песни Порнофильмов "Я так соскучился", потом
саундтреки известных молодежных фильмов, потом просто красивые мелодии находила
в тик токе и разучивала. 

Перешла в новую школу и там, на перемене, открыла пианино ( благо оно доступно
для детей) и начала играть. Вся школа сбежалась послушать знакомые нотки и
подпевала... Кто-то снимал на видео...

- Как классно ты играешь! А ты давно занимаешься?

Заинтересовались сверстники и признали "своей".

А теперь в Артеке Даша познакомилась с удивительной девочкой, которая тоже с 5
лет занимается вокалом. И они спелись... В любую свободную  минутку бегут к
инструменту и разучивают новый кавер. 

На втором видео литературный вечер, посвященный Бродскому. Разучили за три
часа. 

Классно, когда ты умеешь немного больше, чем другие... Значит 10 лет не зря...

Наконец-то дошло...

Папа, который тоже играл в детстве, смахивает слезу от счастья... Не зря...

\ii{22_11_2021.fb.nemchinova_julia.belgorod.1.artek.cmt}
