%%beginhead 
 
%%file 28_04_2023.fb.mdu.1.ekskursia_kyiv_vulycja_jaroslaviv_val
%%parent 28_04_2023
 
%%url https://www.facebook.com/mdu.mariupol/posts/pfbid0rrocSGrToTwwyftnEqZTqpQXokR4MA2rRx5RZxhPVDrS3rsC8gqiQMPppPa2cm8zl
 
%%author_id mdu
%%date 28_04_2023
 
%%tags 
%%title Екскурсія Києвом - Вулиця Ярославів Вал
 
%%endhead 

\subsection{Екскурсія Києвом - Вулиця Ярославів Вал}
\label{sec:28_04_2023.fb.mdu.1.ekskursia_kyiv_vulycja_jaroslaviv_val}

\Purl{https://www.facebook.com/mdu.mariupol/posts/pfbid0rrocSGrToTwwyftnEqZTqpQXokR4MA2rRx5RZxhPVDrS3rsC8gqiQMPppPa2cm8zl}
\ifcmt
 author_begin
   author_id mdu
 author_end
\fi

Знайомство з культурою міста, яке прихистило після вимушеного переїзду, –
важлива частина адаптації маріупольців 🙌 

У межах діяльності гуманітарного штабу Маріупольського уні\hyp{}верситету жителі
героїчного міста відвідали екскурсію \enquote{Прогулянка вулицями Старого Києва}🕍

Дізнатися таємниці однієї з найстаріших вулиць столиці, що носить назву
Ярославів Вал, і завітати до її унікальних місць, маріупольцям допомогла
доцентка Кафедра Культурології МДУ \href{\urlDemidkoIA}{Olga Demidko}.

Під час екскурсії присутні почули історію найбільш старовинної оборонної та
сакральної пам'ятки архітектури Золоті ворота, побачили Караїмську кенасу,
побудовану у дивовижному мавританському стилі. 

На території одного із дворів маріупольці познайомилися з дивовижним вороном
Крумом, який вразив своїми розмірами 🪶

Свого часу у будинках по вулиці Ярославів Вал мешкали видатні особистості,
серед яких барон Рудольф Штейнгель, лікарі Василь Образцов і Микола Стражеско,
поетеса Леся Українка, авіаконструктор Ігор Сікорський. Маріупольці також
змогли насолодитися красою цих архітектурних перлин і дізнатися про них різні
цікавинки.

✨Наступної п'ятниці о 13:00 планується ще одна екскурсія \enquote{Прогулянка вулицями
Старого Києва}. Записатися на неї зможуть усі охочі маріупольці за телефоном
0971961966.

📍 Місце зустрічі – станція метро Золоті ворота.

\#МДУ \#Маріуполь \#Київ 

\#викладачі \#екскурсія \#культура 

\#історія
