% vim: keymap=russian-jcukenwin
%%beginhead 
 
%%file kiev.pogoda.22_06_2021
%%parent kiev.pogoda
 
%%url 
 
%%author 
%%author_id 
%%author_url 
 
%%tags 
%%title 
 
%%endhead 
\section{22-06-2021}

Солнечная погода - облака. Сижу дома.

Ясною погодою у Києві нас цей день не порадував - хмари закрили небо вже з
самого ранку та протрималися до пізнього вечора. Без опадів.  Народний прогноз
погоди: У цей день вшановується пам'ять святителя Кирила Олександрійського. У
минулі часи день 22 червня мав ще й таку назву - Кирило Кінець Весни Початок
Літу. Це другий день літнього сонцестояння, і це найдовший день у році. З цього
приводу наші пращури говорили, що \enquote{На Кирила сонечко віддає землі усю свою
силу}. По сонцю і про погоду судили: якщо воно світить вдень ніби через
оболонку - потрібно незабаром чекати на дощ.

Схід 4:44 Захід 21:15
макс.: +33.4°C (1891) мін.: +6.3°C (1910)

