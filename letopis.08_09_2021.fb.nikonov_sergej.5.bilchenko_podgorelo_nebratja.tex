% vim: keymap=russian-jcukenwin
%%beginhead 
 
%%file 08_09_2021.fb.nikonov_sergej.5.bilchenko_podgorelo_nebratja
%%parent 08_09_2021
 
%%url https://www.facebook.com/alexelsevier/posts/1576830399328936
 
%%author_id nikonov_sergej
%%date 
 
%%tags bilchenko_evgenia,liberalizm,rossia,ukraina
%%title БЖ. Подгорело у небратьев
 
%%endhead 
 
\subsection{БЖ. Подгорело у небратьев}
\label{sec:08_09_2021.fb.nikonov_sergej.5.bilchenko_podgorelo_nebratja}
 
\Purl{https://www.facebook.com/alexelsevier/posts/1576830399328936}
\ifcmt
 author_begin
   author_id nikonov_sergej
 author_end
\fi

Евгения Бильченко. Скопировано точно.  Чистый копипаст. No comments, likes or dislikes. 

БЖ. Подгорело у небратьев.

Очень произошел печальный ивент для русских либералов. Три дня они горько
плакали в Фейсбуке, ободренные (у меня компьютер меняет на "ободранные", sorry)
душераздирающей поддержкой околонаучной киевской братии. Ну, поплакали,
поплакали и смирились.

\ifcmt
  tab_begin cols=2

     pic https://scontent-frt3-1.xx.fbcdn.net/v/t1.6435-9/240568163_1576824555996187_1485693551588442084_n.jpg?_nc_cat=108&_nc_rgb565=1&ccb=1-5&_nc_sid=730e14&_nc_ohc=XGRthscQN4sAX_MTi52&_nc_ht=scontent-frt3-1.xx&oh=9e45c1c09469d14d6234725ee746f155&oe=615FDD09

     pic https://scontent-frt3-1.xx.fbcdn.net/v/t1.6435-9/240541858_1576825299329446_8772857337202034132_n.jpg?_nc_cat=108&ccb=1-5&_nc_sid=730e14&_nc_ohc=dszqXxthoMQAX96YIZs&_nc_ht=scontent-frt3-1.xx&oh=73ca4b6a8c2be721949b31057e84f1b1&oe=615FAC3B

  tab_end
\fi

"Фаланстер" опубликовал рекламу моего "Сентиментального насилия либерализма".
Продираясь сквозь общий плач, мне написал старый знакомый из США, учёный,
который вознамерился вести со мной диалог о том, что либерализм - очень хорош.
Видимо, отсюда из Украины, самого центра либерального безумия, я должна была
возрадоваться. Я честно возрадовалась, так как человек я открытый, и
предупредила, что будет дискуссия. А ещё попросила убрать в первом абзаце
приевшуюся мысль, что в России все плохо-плохо. Хотя бы потому, что книга - не
о России, а, если уж на то пошло, об Америке.

Предсказуемо оказалось, что диалог в понимании моего западного партнёра - это
только его мнение и ничьё другое. Потому убрать русофобскую вставку нельзя и
особо спорить тоже. Потому что нет никакого либерализма, есть только демократия
и "Мордор" супротив нее. Так как книга о либеральной цензуре, иллюстрация вышла
потрясающая. "Мы за любое свободное мнение, если это мнение - наше".

Тем временем, русская либеральная тусовка рукопожатных интеллектуалов
продолжала стенать. Выяснилось, что скоро Путин даст мне денежную премию (???),
"Фаланстер" - "филиал "Единой России", цветную революцию придумали "русские
нацисты", а Межуев, дорогой Вадим Михайлович Межуев, философ вообще-то
классических либеральных взглядов, - "агент вообще-то классических либеральных
взглядов, - "агент Кремля", да.

В итоге один из сотрудников "Фаланстера" таки послал одного, эээ, "свободного
человека", на три буквы, за что тут же стал "неотёсанным русским медведем". Я б
тоже послала, но я забанена. Учитывая ацкое, сцуко, возрастание активности
кремлёвских агентов, возбудились спящие из Берлина, Праги и ещё Бог весть каких
мест. Всех их очень опечалило тот факт, что база критики у агента Бильченко -
научная и во многом англоязычная, потому что их кураторы хвастаются, а сам
агент - с дредами и весьма "ихний" на вид. Дошло до того, что иные из преданных
сторонников украинского майдана с чисто советской хваткой начали камлать по
поводу моей возмутительно авангардной прически. Либералы, они такие либералы...

Развалинами Рейхстага удовлетворен.
