% vim: keymap=russian-jcukenwin
%%beginhead 
 
%%file 01_09_2023.stz.news.ua.mrpl.0629.1.pro_kozacke_korinnja_mariupolja_1
%%parent 01_09_2023
 
%%url https://www.0629.com.ua/news/3653256/pro-kozacke-korinna-mariupola-castina-1-davni-sprobi-osagnuti-cinnik-ukrainskogo-kozactva-v-istorii-mariupola
 
%%author_id korobka_julia.mariupol,korobka_vadim.mariupol,news.ua.mrpl.0629
%%date 
 
%%tags 
%%title Про козацьке коріння Маріуполя. Частина 1. Давні спроби осягнути чинник українського козацтва в історії Маріуполя
 
%%endhead 
 
\subsection{Про козацьке коріння Маріуполя. Частина 1. Давні спроби осягнути чинник українського козацтва в історії Маріуполя}
\label{sec:01_09_2023.stz.news.ua.mrpl.0629.1.pro_kozacke_korinnja_mariupolja_1}
 
\Purl{https://www.0629.com.ua/news/3653256/pro-kozacke-korinna-mariupola-castina-1-davni-sprobi-osagnuti-cinnik-ukrainskogo-kozactva-v-istorii-mariupola}
\ifcmt
 author_begin
   author_id korobka_julia.mariupol,korobka_vadim.mariupol,news.ua.mrpl.0629
 author_end
\fi

\begin{quote}
Вадим Коробка та Юлія Коробка, доценти кафедри історії та археології
Маріупольського державного університету, підготували великий текст - результат
їхніх досліджень щодо встановлення дати заснування Маріуполя.

Текст великий, тому будемо публікувати його частинами. Слідкуйте за нашими
публікаціями.
\end{quote}

Опановуючи територіальні трофеї війни (1768 – 1774 рр.) із Портою та охоплюючи
цупкими обіймами адміністративно-терито\hyp{}ріального устрою українські землі,
Російська імперія прагнула їх довічної інкорпорації. Ці перетворення в наслідок
ліквідації Запорізької Січі (1775 р.) та ануляції її володінь поширились і в
ук\hyp{}раїнському Надазов'ї (Приазов'ї).

Відразу попередня історія нашого краю почала викривлятися та замовчувалась.
Причому це робилося не тільки на догоду імперським інтересам, а  і з корисливої
вигоди призначених на Південь України з берегів Неви адміністраторів, які
прагнули царських нагород та щедрих пожалувань за старанну службу та імітацію
заснування нових населених пунктів. Але найсуттєвішою ціллю все ж було
висування на перший план начебто цивілізаторської місії Російської імперії за
рахунок обминання мовчанкою непересічної ролі чинника українського козацтва в
історії Південної України, загалом, та нашого краю, зокрема.  У цьому плані
Катерина ІІ маніфестом 1775 р., що побачив світ після зруйнування Січі, явила
настанову для підданих, яка передбачала \enquote{истребление} \enquote{на будущее время и
самого названия Запорожских Козаков},  і яку, слава Богу, не можливо було
здійснити, тому і не було реалізовано.

\ii{01_09_2023.stz.news.ua.mrpl.0629.1.pro_kozacke_korinnja_mariupolja_1.pic.1}

Вагомий внесок у відродження пам'яті про запорожців, їхню останню Січ та
належних їй земель, що охоплювали значну територію Центральної, Південної та
Східної України аж до річки Кальміус, зробив Аполлон Скальковський.

Його монографія \enquote{История Новой Сечи или последнего Коша Запорожского} (1841
р.), містила багатий фактичний матеріал, залучений із документів козацького
архіву, доступного досліднику. Науковому опануванню загальних питань  історії
Нової Січі у подальшому посприяли розвідки таких здібних дослідників, як Дмитро
Яворницький, Володимир Голобуцький та деяких інших. 

Спроби в ХІХ ст. відновити забуту козацьку минувшину  безпосередньо Маріуполя
можна спостерігати у працях архієпископа Гавриїла (Розанова) (1853 р.)

\ii{01_09_2023.stz.news.ua.mrpl.0629.1.pro_kozacke_korinnja_mariupolja_1.pic.2}

та єпископа Феодосія (Макаревського) (1880 р.),

\ii{01_09_2023.stz.news.ua.mrpl.0629.1.pro_kozacke_korinnja_mariupolja_1.pic.3}

які в різні роки очолювали Катеринославську єпархію та мали доступ до її
архіву. Особлива значущість праць цих авторів полягає в тому, що документи,
якими вони користувались, не збереглися до наших днів.

Здобутки в цьому плані істориків-кліриків тлумачили директор маріупольської
чоловічої гімназії Григорій Тимошевський (1892 р.), радянський краєзнавець
Дмитро Грушевський, а також у незалежній Україні – журналіст Микола Руденко та
професор історії Василь Пірко. Ці дослідники свої розповіді про  стале
перебування українського козацтва на теренах Надазов'я будували на уривчастій
джерельній базі. На результатах досліджень позначилася відсутність доступу до
архівних матеріалів – діловодства Кошу (центрального органу управління Січі) й
імперської картографії – та друкованих збірок документів, які побачили світ ще
у ХІХ ст. і стали бібліографічною рідкістю. 

Шлях наукового пізнання був звивистий і триває донині.  У його активі потужний
фактологічний ресурс, який потребує систематизації та узагальнення. Водночас,
науковий пошук від самого початку супроводжувало  помилкове тлумачення деяких
історичних джерел, що породжувало оповідання, які не відповідають історичній
дійсності, але це тема окремого допису.

На початку наших міркувань відразу визначимось, що під козацьким корінням
Маріуполя мається на увазі Кальміуська паланка або Кальміус. Це було стале
селище українських козаків, військово-адміністративний центр однієї з
територіальних округ, з яких складались належні Новій (останній) Запорозькій
Січі землі – Вольності Війська Запорозького Низового або Запорожжя).
Кальміуська паланка (Кальміус) як населений пункт був стабільним місцем
проживання людей, що склалося внаслідок службової, господарської та іншої їх
діяльності. В 1780 р. його простір був цілковито поглинутий (став складовою
частиною) поселення православних переселенців з Кримського ханату, яке отримало
назву Маріуполь та статус міста. А поліетнічна громада новоприбулих мешканців
міської території та заснованих ними навколишніх сіл були наділені імперським
законодавством привілеями станового характеру, що передавались у спадок до їх
скасування, в основному, в 1870-х рр. За ними у законодавстві закріпилась назва
– \enquote{маріупольські греки}.

Підйом інтересу україноцентричних представників маріупольської громадськості до
місцевої спадщини запорозького козацтва був природно зумовлений проголошенням
незалежності України та хвилею національного піднесення, що докотилася й до
Маріуполя. Його виявом було встановлення пам'ятного знаку, присвяченого
500-річчю українського козацтва. Зрозуміло, що без участі та підтримки міського
самоврядування тут не обійшлось. До речі, селище Кальміуська паланка
(Кальміус), адміністративно-службовий центр східних володінь Нової Запорозької
Січі було розташоване в іншому місці, але про це на початку 1990-х рр. ще
нікому не було відомо.

\ii{01_09_2023.stz.news.ua.mrpl.0629.1.pro_kozacke_korinnja_mariupolja_1.pic.4}

Хай там як, але пам'ятний знак був символом, якому не судилося стати місцем
пам'яті* в публічному просторі Маріуполя, бо структури муніципальної влади
ніколи не були готові використовувати його як міський церемоніальний об'єкт. Та
загалом пам'ять про козацьку спадщину нашого міста не набула чіткості й не
знайшла офіційного притулку. На перешкоді її кристалізації було чимало
чинників.

По-перше, тяжів трафарет 1778 року, який довільно, без дискусії визначили датою
заснування міста ще в 1977 р. Не так просто було змінити офіційну розповідь про
початок Маріуполя. Багаторічна традиція маріупольського міського літочислення
уявлялася непорушною. 

По-друге, очільники міста враховували, що будь-який вияв прихильності до
української історичної тематики на теренах нашого краю міг вартувати якихось
виборчих балів.

По-третє, деякі громадські організації нащадків маріупольських греків,
гіперактивно прагнули запровадити офіційний міський наратив, про монопольну
роль їх предків у заснуванні Маріуполя. Оповідь про це на тлі   візитів у наше
місто найвищих посадових осіб Грецької Республіки в певних колах міської
громадськості  ставала аксіомою. Водночас, на теренах міста та околиць
зарясніли місця пам'яті про православних переселенців із Кримського ханату
(маріупольських греків), вочевидь, не в останню чергу й завдяки іноземним
фінансовим ін'єкціям.

По-четверте, уривчасті відомості про козацьке селище, що протягом другої
половини 1740-х років – початку 1780 х рр. існувало на теренах, де простягся
Маріуполь, також не сприяли формуванню усталеної розповіді про козацьке минуле
нашого краю. Дуже негативну роль, на наш погляд, відіграла закорінена, навіть у
наукових працях, розповідь про селище або укріплення українських козаків –
Домаху. Воно ніби було розташовано неподалік від впадіння р. Кальміус в
Азовське море. Зазвичай про таку Домаху повідомлялось без посилання на
історичні джерела.

З іншого боку спостерігалася відсутність наполегливого пошуку з боку офіційних
краєзнавців – керівного ядра Маріупольського краєзнавчого музею. Давно
визначений 1778 рік заснування міста ветерани Маріупольського краєзнавчого
музею прагнули захистити, вчиняючи наукоподібний спротив пропозиціям його
перегляду, навіть вдаючись до логічних хитрощів. Зокрема, було здійснено спробу
продемонструвати темпоральний розрив між козацьким селищем та Маріуполем в
солідному опусі \enquote{Мариуполь и его окрестности: взгляд из ХХI века} (Маріуполь,
2008)., що побачив світ без рецензування вченими, які здійснюють дослідження у
відповідній галузі. Там йшлося про те, що козацького населеного пункту ніби вже
не існувало з 1773 р. в місцевості, де згодом розпочинався Маріуполь. За
свідчення бралося ніщо (!), а саме, відсутність запису про існування селища в
гирлі Кальміусу в щоденнику професора Йоганна Гюльденштедта – очільника
імперської академічної експедиції в Надазов'ї, – який проїздив у нашій
місцевості за 12 верст від впадіння річки в Азовське море! У такий спосіб,
бездоказово, робилася спроба продемонструвати відсутність наступності  між
селищем Кальміус та Маріуполем.

\ii{01_09_2023.stz.news.ua.mrpl.0629.1.pro_kozacke_korinnja_mariupolja_1.pic.5}

Маріупольські музейники-\enquote{корифеї} загалом не ігнорували наявність запорожців в
Надазов'ї, але зображали їх як один з епізодів у калейдоскопі нашої минувшини –
кіммерійці, скіфи, сармати, печеніги, половці, монголи, ногаї... Правда,
українське козацтво на сторінках зазначеної книги зображалось майже натхненно,
але воно не потрапляло в мейнстрім історії міста. Складався нелогічний збіг
обставин – офіційні краєзнавці, які мали у своєму розпорядженні чималий, на
певному етапі унікальний, багаж відомостей про наш край, не прагнули заглибити
історію Маріуполя на кілька десятків років. Це мабуть була виняткова ситуація в
Україні та за її межами.

Через зазначені чинники, попри прагнення окремих ентузіастів різних часів, у
маріупольців відчуття спадкоємства з козацьким часом не формувалося, бо не мало
підтримки з боку владних структур. Тема українського козацтва на теренах
Надазов'я не знаходила належного вжитку в місцевих закладах освіти.

ДАЛІ БУДЕ...

\textbf{ЧИТАЙТЕ ТАКОЖ:}

\href{https://www.0629.com.ua/news/3653011/vkradena-istoria-koli-i-hto-zasnuvav-mariupol-ta-comu-pravda-pro-ce-bagato-rokiv-prihovuvalas-vid-suspilstva}{%
Вкрадена історія. Коли і хто заснував Маріуполь, та чому правда про це багато років приховувалась від суспільства, 0629.com.ua, 31.08.2023}
