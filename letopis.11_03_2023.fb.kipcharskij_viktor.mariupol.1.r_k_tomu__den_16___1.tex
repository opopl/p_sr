%%beginhead 
 
%%file 11_03_2023.fb.kipcharskij_viktor.mariupol.1.r_k_tomu__den_16___1
%%parent 11_03_2023
 
%%url https://www.facebook.com/permalink.php?story_fbid=pfbid02jPfed1PFd8YRvUBJLCF4iaJ84BDbfMA8kXccQBvjKagtNjdnPpfnhZXs4KWPA4g2l&id=100006830107904
 
%%author_id kipcharskij_viktor.mariupol
%%date 11_03_2023
 
%%tags mariupol,mariupol.war,dnevnik,11.03.2022
%%title Рік тому: День 16 - 11.03.22. П'ятниця
 
%%endhead 

\subsection{Рік тому: День 16 - 11.03.22. П'ятниця}
\label{sec:11_03_2023.fb.kipcharskij_viktor.mariupol.1.r_k_tomu__den_16___1}

\Purl{https://www.facebook.com/permalink.php?story_fbid=pfbid02jPfed1PFd8YRvUBJLCF4iaJ84BDbfMA8kXccQBvjKagtNjdnPpfnhZXs4KWPA4g2l&id=100006830107904}
\ifcmt
 author_begin
   author_id kipcharskij_viktor.mariupol
 author_end
\fi

Рік тому:

День 16 - 11.03.22. П'ятниця. 

Близько 1:00. Гул літака. Чотири пуски ракет по ньому. Гул затихає, тобто літак
відлітає.

Оля каже, що перед тим були дивні залпи - ніби бій вже десь поруч.

\ii{11_03_2023.fb.kipcharskij_viktor.mariupol.1.r_k_tomu__den_16___1.pic.1}

По шостій біля вогнища (замість люльки -  тютюн же не знайшов)  запалив Captain
Black Cherry. Хлопці звернули увагу, почали кепкувати: "Палич, який магазин
винесли?". Стас попросив сигарету, я пригостив. Я пожартував: вранці треба
запалювати свічку і молитися. А оскільки свічки нам можуть ще знадобитися для
освітлення та й їжу можна на них гріти, то я буду вранці палити одну сигарету й
молитися. Стас подивився на сигарету так, ніби хотів віддати, якби не запалив.
(Я досі вранці, коли можу, виходжу на вулицю, запалюю сигарету і молюся за
звільнення полонених друзів, за виздоровлення поранених, за збереження тих, хто
воює та за упокій вбитих). 

7:30. Літак над нами. Чотири ракети по ньому. Чотири дуже гучних (бо близько)
розриви.

Стрілковий бій: Перші ворота? Площадка Б?

8:00. В наш під'їзд зайшли чоловік та жінка з собакою. Хтось помітив, що вони
пішли у підвал, почали обурюватись. Хтось пояснив, що чоловіка уламком поранило
у нирку, його зашили у госпіталі і відправили додому через загрозу бомбування. 

Толик вирішив відвезти їх додому, його машина не заводиться, тож його син
Андрій та Наталчин Сергій почали її штовхати. Потім прибіг Сергій, покликав
заштовхали машину назад. Вони встигли заштовхали її за Покришкіна. Сніг,
слизько. Штовхали. Потім зупинили джип (кілька машин не зупинилися), який
затягнув у наш двір. Або "секретка" на сигналізації, або залив свічки.

Пішли в балку по дрова. Хлопці спустилися у балку, я приймаю гілки на дорозі.
На схилі падає огрядний чоловік, кричить від болю, матюкається. Каже,  що не
треба допомогати,  що встане сам, бо в нього пошкоджені спина та коліно.
Підвівся, трохи пройшов та знову впав. Я виламав з паркану штакетину, витягнув
цвяхи та дав йому, аби він на неї спирався. Вони з жінкою йдуть на Жилкопи до
знайомих, бо в їхньому будинку вибило вікна. Їм назустріч також йде багато
людей.

Поранений з нашого підвалу пішов через запах солярки. 

11:00. Приніс чергову порції дров. Втомився, бо слизько, та ще й штовхав
машину. Відпочиваю біля вогнища. Люда поставила на вогонь величезну каструлю,
що викликало незадоволення: "Ти що, митися будеш?", "На це місце можна було 3-4
чайники поставити. Люда спокійно чистить бурак, картоплю, ріже капусту,
смажить  цибулю та моркву - варить борщ. Коли борщ зварено, відливає у меншу
каструлю і носить по квартирах  старшим людям.

Коли я пішов по чайник та бідон, чув як вона вмовляє Ольгу-кришнаїтку з п'ятого
поверху: "Та він же вегетаріанський, в ньому нема ані м'яса , ані жиру".
Принесла й Едіку, вмовила і його. 

Наталка  розповідає, що снаряд прилетів у 219-й будинок у квартиру сусідів її
родичів - у родичів горів балкон. А чим тушити, як води нема?

Прилетіло й у будинок перед "Щирим кумом". Чоловіка на кухні оглушило і в нього
луснули внутрішні органи. У жінки у сусідній кімнаті луснули барабанні
перетинки у вухах і судини у очах: вона оглухла та осліпла. Повзала по
квартирі, порізала руки й коліна уламками скла та посуду, шукала чоловіка, а
він помирав на кухні та кликав її. Вона його не чула, та почули сусіди і вивели
її. Почали піднімати його, а він помер. Отак він її врятував, а йому допомогти
не змогли...

До мами приїхав Віталій з першого під'їзду. Їхав по Куінджі: від Драму на
проїжджій частині три вирви від бомб, скрізь уламки, трамвайні дроти. На Драмі
стоять машини: на холоді без вогнищ люди очікують коридору. Рашисти зайняли
погранчастину на Флотській. Тобто, Цибульських вже "звільнили". Може дітям
поїхати до них? Може там вже не стріляють, та й є підвал. (Пізніше стало
відомо, що там "пінг-понг" - літає то туди, то сюди) 

На автостанції стоять побиті уламками та спалені автобуси, на яких планували
вивезти з міста людей до Запоріжжя. На Правому березі був жорсткий обстріл -
люди довго сиділи у підвалах. Це розповіли люди які приїхали до родичів у дворі
Віталія  - машину геть усю посікло уламками, вікна побиті. Віталій каже, що у
місті дві "разу спокою" його та наш двори 

Міст Талаківка-Сартана підірвано, але рашисти вже зайшли на Волонтерівку та
"рівняють" її минометами. (Рівняють вулицю Рівну).

14:45. Приліт приблизно за Старим мартеном на Ілліча (копровий цех?): два
стовпи чорного диму, потім велика хмара бурого.

17:00. Біля вогнища Стас розповів, що коли вони з жінкою пішли по воду, то повз
них проїхав білий бусик. Щось негучно клацнуло. Огрядний чоловік почав валитися
на паркан. Задрали куртку - на правому боці живота маленька дірочка, з якої
біжить червона кров. Мабуть, стріляли з мелкашкі.

Ще один сусід розповів, що коли його знайомий пішов по воду до гаражів, то
поруч вибухнули дві міни 80 мм. Може десь за будинками, бо ані ми ані інші
сусіди вирв від мін не бачили.

Близько 18-ї. На 42-му будинку у заскленому балконі відбивається червоне
сонце, що сідає: від вибухів та пострілів рами вигинаються і відображення
хитається.

Над головами гуде літак, відстрілюючи теплові пастки, які повільно летять до
землі. По цих пастках можна відновити траєкторію літака: заходить десь від
П'ятого мікрорайону, пролітає на Азовмашем та відлітає за Шлакову гору.
Спокійно та безкарно...

Фото: карта руйнувань Кальміуського району. На схід від нашого будинку теж були
руйнування у приватному секторі, але будинок у більш-менш спокійному місці...

"Оаза спокою"...

%\ii{11_03_2023.fb.kipcharskij_viktor.mariupol.1.r_k_tomu__den_16___1.cmt}
