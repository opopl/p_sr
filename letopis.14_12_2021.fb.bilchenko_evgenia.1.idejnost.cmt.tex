% vim: keymap=russian-jcukenwin
%%beginhead 
 
%%file 14_12_2021.fb.bilchenko_evgenia.1.idejnost.cmt
%%parent 14_12_2021.fb.bilchenko_evgenia.1.idejnost
 
%%url 
 
%%author_id 
%%date 
 
%%tags 
%%title 
 
%%endhead 
\zzSecCmt

\begin{itemize} % {
\iusr{Анатолий Миронов}

\begin{multicols}{2} % {
\obeycr
Вновь дон Кихотвы потомки
ломают с лязгом зубочистки.
Злой событийности по кромке,
в ту бездну бросив мыслей брызги.
\smallskip
Но им едва ли удержаться
на кромке этой, посредине.
С глумливой рожицей паяца,
какой известно в том повинен,
\smallskip
что быть не может в жизни вовсе,
... да и не очень-то хотелось.
Расплаты дата где-то после,
на фоне белом, красным рделась.
\smallskip
В злом ожиданьи окончаний
Санкт-Петербургства Ленинграда.
Они по-Бродски прокричали:
\enquote{Уволь... спасибо, но не надо}.
\smallskip
Уют домашний в прозе ставень
узором белым разольётся.
Как ярок рифмы желчный плавень
Луны, на самом дне колодца
\smallskip
Невы, в чернеющих пределах,
канвы событий лязг по кромке.
Где выживали... так несмело
его бесславные потомки.
\smallskip
для Б.Ж.
\restorecr
\end{multicols} % }

\iusr{Евгения Бильченко}
\textbf{Анатолий Миронов} Мне внятно лишь про Луну, очень зримый образ. Но в целом, да.

\iusr{Алексей Бажан}

А что Вы подразумеваете под \enquote{украинстvoм}? Конвенционально, в русской
консервативной среде, под ним понимают учение об отдельном от русского народе,
не имеет значения, насколько \enquote{братском}. К Мазепе, увы, дело не сводится -
Бузина Мазепу (Петлюру, Бандеру и прочих героев национального пантеона) ругал
почище русских националистов, что нисколько не мешало ему быть убежденным
самостийником. Портрет напомнил душевную шишью песню, Елизарову до такого расти
и расти. 

\href{https://www.youtube.com/watch?v=VSZM26b2YcY}{%
Псой Короленко \& Шиш Брянский Сказка о братьях Гримм, youtube, 16.11.2019%
}

\begin{itemize} % {
\iusr{Ермаков Станислав}
\textbf{Алексей Бажан} Вряд ли Бузину можно назвать убежденным самостийныком в своей книге \enquote{Союз плуга и тризуба} он разносит идею самостийности в пух и прах. Да, в интервью он заявлял, что видит Украину независимой, но нужно понимать, что любой другой ответ привел бы к проблемам с сбу и судом в итоге.

\iusr{Алексей Бажан}
\textbf{Ермаков Станислав} Бузина не очень жаловал русских и Россию, хоть и старался, учитывая настроения своих читателей, свое к ним нерасположение не афишировать.

\href{https://buzina.org/politics/658-dvatsatiletuye-prazdnika-narodnoy-nezavisimosti.html}{%
Двадцатилетие праздника народной независимости, buzina.org, 01.12.2011%
}

\begin{multicols}{2} % {
День Независимости Украины давно следовало бы перенести на Первое декабря.
Во-первых, хорошо запоминаемая дата. Во-вторых, самая подходящая погода для
парадов – никто не грохнется в обморок от жары, как это уже не раз бывало
летом. И, в-третьих, что самое главное – 24 августа 1991 года Верховная Рада
приняла всего лишь Акт провозглашения Независимости! Но провозглашение – это
что-то вроде декларации о намерениях. Легитимность этому акту придал только
всеукраинский референдум 1 декабря того же года. Не группа депутатов и даже не
весь парламент, а народ! По сути, до этого референдума вопрос реального
суверенитета Украины висел в воздухе. И только получив на руки реальную цифру
поддержки (90,32\% избирателей проголосовали за идею «незалежності»),
новоизбранный в тот же день президентом Леонид Кравчук смог ехать в Беловежскую
Пущу на встречу с Ельциным и Шушкевичем и «делить на троих» Советский Союз.
Напомню, что главный вопрос на референдуме звучал проще простого: «Чи
підтверджуєте ви Акт проголошення Незалежності України?» Акт поддержали 28
миллионов человек (в том числе и автор этой статьи), которые теперь с полным
правом могут назвать себя, а не мифического Грушевского, сдавшегося большевикам
за академическую пайку, «отцами» и «матерями» независимости. Нынешняя страна
стала результатом нашего тогдашнего выбора.

Возвращаясь в тот день, я спрашиваю себя: почему я проголосовал за
независимость? Буду абсолютно откровенен. Прежде всего, потому что
неадекватность Горбачева была к тому моменту абсолютно очевидна. Это сегодня
Михаил Сергеевич называет себя «идиотом» в интервью журналу «Шпигель» за то,
что не отправил в свое время Ельцина послом в какую-нибудь экзотическую страну.
Но в 1991 году он до такой самокритичности, с которой я полностью согласен, еще
не дошел. Зато весь (еще советский тогда) народ уже понимал то, до чего
Горбачев дошел только сейчас. Вы хотите, чтобы вами правил идиот? Вот и я не
хотел!

Но Ельцин в качестве вождя (даже на танке) тоже абсолютно не подходил для
украинской ментальности. Это был типичный медведь-дуболом с признаками
прогрессирующего алкоголизма на лице. Он уже падал с какого-то моста в
подмосковную речку, что широко освещала и зарубежная, и советская перестроечная
пресса. Представить его своим президентом я не мог даже в страшном сне. Он мог
быть кумиром Хакамады, Немцова, Гайдара-младшего (того, что, в отличие от деда,
на коня не мог залезть даже со стула), но не моим. Это была не моя сказка про
пьяного царя. Мне хотелось гетмана. Пусть тоже не идеального, но своего. Хитрый
и пронырливый бывший главный идеолог УССР Леонид Кравчук мало походил на
Богдана Хмельницкого. Но, может, нам и нужен был тогда именно такой
псевдо-Богдан. Не от Бога, а от людей, то есть, от нас. На тот момент Кравчук
олицетворял лучшие качества украинского характера – умение договариваться и
быть «своим» во всех противоборствующих лагерях - и в Рухе, и в Компартии.
Одна его половина была желто-синей, а другая – красной, а все это вместе и
являлось символом тогдашнего состояния Украины.
 
Еще одним фактором в пользу голосования за независимость был мой опыт армейской
службы. Если офицеры в Советской Армии были по преимуществу славянами, то мы,
солдаты-срочники, жили в условиях полного интернационала. Это был совершенно не
похожий на официальный «интернационализм», выражавшийся в «землячествах» и
казарменных драках по национальному признаку. Побывав в этом Институте
прикладной этнографии, ты спрашивал себя: почему, собственно, мы должны жить в
одной стране с этими совершенно не похожими на нас людьми с Кавказа и Средней
Азии? Пусть у них будет своя страна с их обычаями, а у нас – своя. И мы, и они
имеем на это право, закрепленное в Конституции СССР. То есть, право на выход из
созданного Лениным государства.

Нам всем надоел на тот момент диктат Москвы, которая была, как мы помнили из истории, всего лишь основанной киевским князем пограничной крепостью в земле племени вятичей. Идеологема древнерусского единства, официально принятая в Советском Союзе, имела ахиллесову пяту – столицу-узурпатора. Центром Киевской Руси мог быть только Киев. Мы были стихийными киевоцентристами! Тех, кто упрекал нас в украинском сепаратизме, мы с таким же основанием могли назвать потомками древнемосковских сепаратистов. Впрочем, в 1991 году они были еще и просто московскими сепаратистами. Ведь сегодня как-то забылось, что ельцинская Россия начиналась именно с русского национализма – откровений Солженицина, писателей-деревенщиков (родных клонов наших «хуторян» из киевской СПУ), требований образования компартии РСФСР и прочей лапотной антиимперской экзотики. Конечно, Ельцин и К°, оседлавшие эту фольклорную струю, усиленную дефицитом водки, в будущем надеялись тоже стать адептами «единой и неделимой». Но тогда Борис Николаевич еще говорил: «Берите суверенитета, сколько сможете». Вот мы и взяли, сколько смогли, в свою хату с краю.

И, наконец, хотелось попробовать что-то новое. Будущее показало, что это был чисто украинский выбор, позволивший нам избежать очень многих потрясений. Расстрел парламента в Москве в 1993 году и две чеченские войны стали уже не нашей историей. Нашим оказался Майдан, закончившийся, что ни говори, в отличие от аналогичных московских «майданов» начала 90-х, абсолютно бескровно.


 
Напрасно говорят, что украинская нация еще не сложилась. Она созрела уже к 1
декабря 1991 года, самовыразившись в результатах референдума. Просто это была
нация, не вписывавшаяся в узкие рамки традиционного украинского национализма.
Эта нация говорила и думала на двух языках. Она строила свое государство,
учитывая традиции и УНР, и УССР, и, может быть, больше всего Украинской Державы
Скоропадского, тяготела поочередно, а часто и одновременно и к Европе, и к
России, но всегда оставалась самой собой – явлением таким же
уникально-противоречивым, как Австрия в Германском мире. Вроде немцы, но особые
– австрийцы. Может быть, и не немцы совсем. Полный аналог нашей хитрой
идентичности!

Оглядываясь назад, я думаю, что это было прекрасное двадцатилетие – украинский
Золотой Век. Каждый взял свободы, сколько захотел. Кто хотел приватизировать –
приватизировал. Кто жаждал спиваться – спился. Главное, никто никому не мешал.
В начале 90-х в Украине произошла тихая мелкобуржуазная революция. Если заводы
достались только избранным, то квартиры и дачи – всем. Страна квартирантов
стала объединением собственников. А главным конфликтом – не идеологический, а
битва за «сите та щасливе життя». В бандитских разборках за дележ «общенародной
собственности» погибли сотни и тысячи. В дискуссиях за государственный язык –
ни одного! Это и есть неоспоримымым доказательством подлинного конфликта внутри
Украины. Главным же конфликтом будущего станет требование поделиться к тем, кто
уже все поделил. Начало этого противостояния мы и видим сегодня на киевских и
донецких улицах. Делитесь, господа, чтобы не потерять все!
\end{multicols} % }

\iusr{Ермаков Станислав}
\textbf{Алексей Бажан} 

Статья десятилетней давности, времен Януковича, когда Бузину устраивал
существующий политический строй и на большее никто не расчитывал.. И это статья
из серии - \enquote{у нас тут юбилей, напиши что нибудь хорошее}. В книгах Бузина
откровеннее. Прикреплю тут пару финальных страниц из \enquote{Союз плуга и тризуба}.

\ifcmt
  ig https://scontent-frx5-2.xx.fbcdn.net/v/t39.30808-6/267669812_2084811175003029_4479478137077578156_n.jpg?_nc_cat=109&ccb=1-5&_nc_sid=dbeb18&_nc_ohc=zFA0DBmzUHoAX_gs50N&_nc_ht=scontent-frx5-2.xx&oh=00_AT_QDXQtJRN1MsQy4El339zMXetfon2PzNvl2CTXwKrBrg&oe=61BEEBEC
  @width 0.4
\fi

\iusr{Ермаков Станислав}
\textbf{Алексей Бажан} И ещё

\ifcmt
  ig https://scontent-frt3-1.xx.fbcdn.net/v/t39.30808-6/267497567_2084811468336333_725002780737505068_n.jpg?_nc_cat=107&ccb=1-5&_nc_sid=dbeb18&_nc_ohc=e06jNZ1oA0IAX9ui8RS&_nc_ht=scontent-frt3-1.xx&oh=00_AT9McCl1Wx7e_48ui8lYGSM-PSwvtjuizfOB-DQkxi_Xhw&oe=61BE5347
  @width 0.4
\fi

\iusr{Vera Romanova}
\href{https://shel-gilbo.livejournal.com/331746.html}{%
Украинский кабинет и будущее Трампа, shel-gilbo.livejournal.com, 05.09.2019%
}

\iusr{Алексей Бажан}
\textbf{Ермаков Станислав} 

Статья примечательна прорвавшейся-таки глухой неприязнью к России и русским,
так ли уж обязательно было среди полагающихся ритуальных формул отпускать
шпильки в наш адрес? Никакого противоречия между статьей и приведенными Вами
цитатами нет - Бузина осуждал \enquote{неправильные} и \enquote{несвоевременные} телодвижения,
но полагал, что в 1991-ом время пришло.


\iusr{Ермаков Станислав}
\textbf{Алексей Бажан} 

Нужно учитывать взросление автора с годами и помнить о его монархических
взглядах. В книге воскрешение Малороссии он бугвально боготворит Российскую
Империю. Ну и нужно ещё понимать, что многие в Украине верили десять лет назад
что возможно строить страну без идеологии, так сказать утопия государства без
идеологии -\enquote{просто адекватной страны}. Некоторые до сих пор в это верят. Но
повторюсь в книгах Бузина откровеннее чем в статьях.

\iusr{Евгения Бильченко}
\textbf{Алексей Бажан} 

\enquote{украинством} в данном случае я называю целостный образ русофобской Украины как
\enquote{отдельности} (разной степени филии), которая в русской либеральной среде
используется как товарный знак. Конвенционально: да, наверное, моя трактовка
ближе к консерватизму. К Мазепе не сводится. Богун and so on, это Сеть. За Псоя
спасибо, тут я могу похвастаться, что знаю. Я так не знаю рок тонко, как вы, в
деталях.

\iusr{Алексей Бажан}
\textbf{Ермаков Станислав} 

Монархист никогда не позволил бы себе те мерзкие шуточки, какие Бузина позволял
себе в отношении последнего Государя, к тому же канонизированного святого. Вот
Сталин для него святое, никаких шуток. Так что совершенно правильно его
аттестовал Дмитрий Евгеньевич в качестве украинского националиста.

\iusr{Евгения Бильченко}
\textbf{Ермаков Станислав}, \textbf{Алексей Бажан}, 

спасибо за деконструкцию психотипа Олеся, она помогает мне понять
малороссийский психотип. Я его не понимаю, этот психотип. Точнее, так. Этот
психотип содержит имплицитную русофобию как некую
травму/обиду/неудовлетворенное прошение (\enquote{Жаль, подмога, не пришла}). Это тоска
по Отцу, мёртвому Отцу, от которого ты желаешь получить погон, но он не даёт
отмашку обозначающего. Это чистой воды Лакан. В Реальном я тоже немного Бузина.
В пространстве Символического (русской поэзии) этот психотип исчезает: я и есть
сам Отец, сама Россия, нуминозное переживание мощи охватывает меня как
цивилизационного субъекта. Та же биполярка у Медведчука, но Олесю и мне проще.
Медведчук может быть только в Реальном как политик и уходит в малороссийство.
Олесь, как и я, может уйти в Текст. Потому в книгах он и есть Отец.

\iusr{Евгения Бильченко}
\textbf{Алексей Бажан} Алексей, ну, и позвольте мне быть гимназисткой рядом с МЮ, право слово)

\iusr{Алексей Бажан}
\textbf{Евгения Бильченко} 

Русский рок и рэп знаю в основном в специфическом ракурсе - музыкально
состоятельные проекты. Всякие Соломенные еноты с, возможно, как мне
рассказывали, замечательной поэзией прошли мимо. Елизаров пробовал петь Шиша,
но не пошло.

\href{https://www.youtube.com/watch?v=imeap-aSHWo}{%
Михаил Елизаров. Кома, youtube, 16.12.2012%
}

\iusr{Евгения Бильченко}
\textbf{Алексей Бажан} 

Елизаров через час недалеко от меня блистает на Григорьевских чтениях в
довольно либеральном арт-центре \enquote{Борей}. А я выполняю черновую пролетарскую
работу, пока там традиционалисты целуются с авангардистами). На этом месте у
меня начинается Реальное и во мне просыпается травма Бузины). Длится это долго,
до ночи обычно. Пока я не напишу стихотворение. Так что, я про Олеся все знаю.

\iusr{Евгения Бильченко}
\textbf{Алексей Бажан} \textbf{Ермаков Станислав} 

я не могу не привести этот текст - это в десятку вашей дискуссии, его у меня
никто не понял, сломанная форма и странный образ...

\headCenter{БЖ. Викинг}

\begin{multicols}{2} % {
\obeycr
БЖ. Викинг
\smallskip
И ты стоишь, такой, среди поля и думаешь:
\enquote{Ё моё}, -
Как этот hренов железнобокий Бьёрн
Из кичового сериала...
\smallskip
"Как мне вас мало, братва, вас мало.
Как я тут лишний, братва, тут лишний.
Тёплый мороз гланды щенком лижет,
И получается смерть - не от ковида.
\smallskip
Без вас я умру, как видно.
Всё, что мне нужно, -
Лишь маленькое подтверждение.
\smallskip
Ни должности и ни деньги, ни деньги,
Ни должности...
До ближайшей деревни дольше идти,
Чем до планеты Марс.
\smallskip
У меня больше нет масок.
Я знаю, что скоро уйду, потому - наглый.
Нет, вытирать мне соплю не надо.
Не надо трогать мою соплю.
\smallskip
Я вас и так люблю,
Не пытаясь жалобить,
Не пытаясь вжаривать"...
\smallskip
И ты стоишь, такой, среди поля,
Затолканным
В рот
\enquote{Пожалуйста}.
\smallskip
13 декабря 2021 г.
\restorecr
\end{multicols} % }


\iusr{Алексей Бажан}
\textbf{Евгения Бильченко} 

Сквозь оптику Жарикова все намного проще. Ключевой вопрос - \enquote{откуда говорил
Бузина}? В терминах Бурдьё, каковы его координаты в социальном и политическом
полях. Реальные, не декларируемые.


\end{itemize} % }

\end{itemize} % }
