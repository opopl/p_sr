% vim: keymap=russian-jcukenwin
%%beginhead 
 
%%file 29_01_2021.stz.news.ua.mrpl_city.1.vasha_lesja
%%parent 29_01_2021
 
%%url https://mrpl.city/blogs/view/vasha-lesya
 
%%author_id demidko_olga.mariupol,news.ua.mrpl_city
%%date 
 
%%tags 
%%title Ваша Леся
 
%%endhead 
 
\subsection{Ваша Леся}
\label{sec:29_01_2021.stz.news.ua.mrpl_city.1.vasha_lesja}
 
\Purl{https://mrpl.city/blogs/view/vasha-lesya}
\ifcmt
 author_begin
   author_id demidko_olga.mariupol,news.ua.mrpl_city
 author_end
\fi

\ii{29_01_2021.stz.news.ua.mrpl_city.1.vasha_lesja.pic.1}

Її вірші вчать змалку, про неї завжди багато говорять, пишуть, дискутують… Ця
жінка самостійно вивчила 11 мов, не лише писала вірші, а й створювала нові
слова. 25 лютого виповниться 150 років від дня народження видатної письменниці,
поетеси, найвидатнішої жінки в історії України \emph{Лариси Петрівни Косач-Квітки}. З
нагоди цієї події Донецький академічний обласний драматичний театр (м.
Маріуполь) та Маріупольське телебачення вирішили підготувати спільний
телевізійно-театральний проєкт \emph{\enquote{Ваша Леся}}. Завдяки цьому фільму глядач
здійснить екскурс головними етапами життя Лесі Українки.

Авторкою задуму стала талановита актриса маріупольського драматичного театру
\emph{\textbf{Дар'я Іванова}} (Жанна д'Арк \enquote{Біла ворона}, Ангеліна \enquote{MAIDAN INFERNO} тощо), яка
підготувала сценарій до фільму. Дар'я поділилася, що ще за місяць до початку
роботи над проєктом читала різні статті і побачила незвичну публікацію про Лесю
Українку, яка відрізнялася від біографії зі шкільної програми. Перечитавши
статтю, актриса дуже захопилася історією поетеси. А коли побачила дату
народження, зрозуміла, що скоро її ювілей, дівчині спало на думку щось створити
до цієї події.

\begin{quote}
\em\enquote{Почала читати про неї ще більше, подивилась фільм \enquote{Іду до тебе}
1971 року про історію відносин Лесі Українки і Сергія Мержинського. Прочитала
декілька п'єс. І мені спала на думку ідея простої розповіді про її дивовижне і
нелегке життя, але щоб це не носило публіцистичний характер, долучити до цього
акторів, які б змогли обіграти ситуацію. Така, своєрідна  реконструкція подій з
історично-пізнавальними елементами}, 
\end{quote}
– поділилася Даша.

\ii{29_01_2021.stz.news.ua.mrpl_city.1.vasha_lesja.pic.2}

Коли Дар'я написала мені про свій задум, я одразу прийшла у захват від ідеї. І
дуже радію що президент ТОВ \enquote{ТРО \enquote{Маріупольське телебачення}} Микола Осиченко
підтримав цей проєкт. Оператори Маріупольського телебачення професійно і
відповідально віднеслися до технічної частини проєкту (зйомки та монтаж), а
актори із задоволенням перевтілилися в історичних персоналій. Назва фільму –
\emph{\enquote{Ваша Леся}} підкреслює простоту письменниці, так вона підписувала свої листи,
такою вона залишилася для свого народу.

\ii{29_01_2021.stz.news.ua.mrpl_city.1.vasha_lesja.pic.3}

Дар'я Іванова стала не тільки сценаристом та режисером, вона ж підбирала
акторів. Також виконала головну роль і зіграла Лесю дорослу. Її головним
помічником став \emph{\textbf{Артур Войцеховський}}, який допомагав кожному актору впоратися зі
своїм завданням. Артур наголосив, що процес зйомок потрібен кожному артисту і є
безцінним досвідом. Він же зіграв роль Сергія Мержинського – кохання усього
життя Лесі Українки. До речі, Артур Войцеховський дуже захоплюється режисурою і
створює власні фільми.

Цей міні-фільм став дебютною роботою для маленької \emph{\textbf{Софії Казакової}} – доньки
хормейстера театру Оксани Казакової. Дівчинка зіграла Лесю маленьку. Для неї це
був хвилюючий, але дуже цінний і цікавий досвід. Для більшості акторів зйомки у
фільмі стали першими. Зокрема, яскрава і талановита актриса маріупольського
драматичного театру \emph{\textbf{Віра Шевцова}}, яка майстерно виконала роль ведучої,
поділилася, що досвіду зйомок у неї ніколи не було, тому для неї це вдвічі
цікавіше.

Матір Лесі Українки, яка була теж письменницею, Олену Пчілку  (справжнє ім'я
Ольга Петрівна Драгоманова-Косач) дуже щиро і чуттєво зіграла \emph{\textbf{Ольга Самойлова}}.
Цікаво, що Олена Пчілка сімейною мовою спілкування встановила українську і
прищеплювала своїм дітям любов до української народної пісні, казок і традицій.
Роль люблячого і турботливого батька – Петра Антоновича Косача,
високоосвіченого поміщика, талановито виконав \emph{\textbf{Валерій Капінус}}. А з роллю лікаря
маленької Лесі майстерно впорався \emph{\textbf{Олексій Панасюк}}.

\ii{29_01_2021.stz.news.ua.mrpl_city.1.vasha_lesja.pic.4}

Попереду у нас ще завершальні зйомки. Перевтіляться у видатних постатей також
Надія Лавриненко (Ольга Кобилянська), Анатолій Шевченко (Михайло Драгоманов) і
Вадим Єрмишин (Климент Квітка).

Найважливіше для проєкту \enquote{Ваша Леся} є відтворення історичної епохи та
перенесення глядача в кінець XIX століття. Необхідно було правильно підібрати
меблі, костюми, реквізит. Театральним цехам – костюмерному, перукарському,
реквізиторському і багатьом співробітникам театру, які для глядачів залишаться
за кадром, довелося постаратися. І все ж окремо хочу відзначити начальницю
перукарського цеху \emph{\textbf{Юріну Елеонору Норисівну}}, яка з особливим натхненням
створювала кожен образ до фільму.

\ii{29_01_2021.stz.news.ua.mrpl_city.1.vasha_lesja.pic.5}

Генеральний директор Донецького академічного обласного драматичного театру (м.
Маріуполь) Кожевніков Володимир Володимирович  дуже зрадів, що з'явилася така
ідея, адже через карантинні заходи театр переживає складні часи. Сподіваюся, що
міні-фільм \enquote{Ваша Леся} знайде свого глядача і стане приємним відкриттям не
тільки для маріупольців, а й для багатьох українців з інших міст.

