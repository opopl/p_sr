% vim: keymap=russian-jcukenwin
%%beginhead 
 
%%file 04_02_2022.yz.istoricheskie_naperstki.1.rashka
%%parent 04_02_2022
 
%%url https://zen.yandex.ru/media/id/5ef8896c0d13dd78e21972de/istoriia-epiteta-rashka-i-kak-normalnyi-chelovek-na-nego-reagiruet-61fb6d78f100ac0b819e6f33
 
%%author_id yz.istoricheskie_naperstki
%%date 
 
%%tags epitet,istoria,rashka.serbia,rossia,serbia,slovo
%%title История эпитета «Рашка» и как нормальный человек на него реагирует...
 
%%endhead 
 
\subsection{История эпитета «Рашка» и как нормальный человек на него реагирует...}
\label{sec:04_02_2022.yz.istoricheskie_naperstki.1.rashka}
 
\Purl{https://zen.yandex.ru/media/id/5ef8896c0d13dd78e21972de/istoriia-epiteta-rashka-i-kak-normalnyi-chelovek-na-nego-reagiruet-61fb6d78f100ac0b819e6f33}
\ifcmt
 author_begin
   author_id yz.istoricheskie_naperstki
 author_end
\fi

Вы когда-нибудь жили в Рашке? Ну... или бывали в ней? Кто-то бросится насадить
автора такого грязного оскорбительного намёка на патриотичные, очень
аутентичные вилы, но поспешу успокоить — речь пойдёт о настоящем средневековом
государстве под названием Рашка.

\ii{04_02_2022.yz.istoricheskie_naperstki.1.rashka.pic.1}

А попутно разберёмся, почему «Рашкой» презрительно дразнят страну Россию
детишки олигархической элиты, хрустобулочные эльфы и космополиты со своих
диссидентских диванов из-за бугра. Даже придумав специальный персонаж для
Тырнета (на обложке), «квадратный ватник Рашка».

\subsubsection{Первое сербское государство}

было образовано 8 сентября 1331 года, называлось Рашка (на тогдашней
межгосударственной латыни — Rasciae, Rassa, Rassia, Raxia, Rascia). Возглавил
его легендарный великий правитель, Стефан Урош IV Душан из Немановичей
(1308-1355 гг.), король Сербии, ставший царём сербо-греческого царства в 1345
году, как только провозгласил сербскую архиепископию — патриархией.

Сербы называли это государство в обиходе по старинке — Raška, Raska zemlja, в
русской литературе встречалось кириллическое написание «Расия», в
дореволюционных исторических изданиях — «Рассия». Даже была фантастическая
версия: предки русских пришли с Балкан на Восточно-Европейскую равнину, эпитет
«Русь / Рашка» могли принести с собой.

Согласно политическим и географическим рассуждениям византийского императора
Константина Багрянородного, после переселения сербов на Балканы «область Рашка»
располагалась между реками Сава, Врбас и Ибар. Позднее сместилась несколько
юго-восточнее, до X века была частью первого Сербского княжества под
собственным именем. А после его распада и гибели Часлава Клонимировича (933-960
гг.) — часть «Рашки» оказалась в составе Византии.

\ii{04_02_2022.yz.istoricheskie_naperstki.1.rashka.pic.2}

Исторически корректным говорить так. Рашка — это государственное сербское
территориально-административное образование (жуп), которое было юридически
оформлено с 1077 года. К середине XII века идёт его усиление, в 1168 году
верховным жупаном Рашки стал Стефан Неманя, основатель династии Неманичей.

В 1190 году Византийская империя признала независимость Сербии, в 1217 году сын
Стефана Немани — Стефан Первовенчанный был коронован королём сербов. На том
настоящая история Рашки закончена, государство стало называться Королевство
Сербия (Краљевина Србија), ставшее в 1346 году царством. Но католики упорно
называли его «Рашкой», а болгары использовали эпитет до самого XV века, если
говорили вообще о Сербии...

Сегодня Рашка — один из округов Сербии (300 тыс. населения, площадью 4 тысячи
кв.км), названный в честь древнего государства со столицей в Кралево. Но
сохранилась и община с главным городом Рашка. В нём проживает около шести
тысяч, а вся община насчитывает не более 25 тысяч человек. После окончательного
раздела Югославии в 2003 году часть Рашки (Санджака) вошла в состав Черногории.

Если будете в сербской Рашке, сразу признавайтесь, что из России. Вас крепко
обнимут и напоят до бессознанки ракией. Потому что здесь уверены: термины Рашка
и Сербия, так же как рашанин / рассиянин /россиянин и серб — это исторические
синонимы.

\subsubsection{Совок из Рашки...}

Кем мы только не были в истории... скифами, тартарами, Ордой, грязными
московитами, «жандармами Европы», тюрьмой народов, большевистским концлагерем,
сегодня — Мордором назначены. Русская заграница всегда помогала внутренней
оппозиции как-нибудь обозвать собственную страну пообиднее.

А с начала Миллениума всё чаще замелькало словечко «Рашка», которое очень
полюбила «золотая молодёжь», с курортов мира обожая блеснуть интеллектом и
воткнуть в английское «Раша» (Russia) ещё одну буковку, с определённым
негативным контекстом: «Везде хорошо, только не в Рашке». А с ростом
политической напряженности, интернет-творцы стали лепить мемы, карикатуры и
демотиваторы самого мерзкого содержания.

На самом деле, эпитет довольно старый, впервые зазвучал в эмигрантской среде
1990-х. Во всяком случае, его в Питере точно фиксировали тогда. А популярным
стало на рубеже тысячелетий. Сначала звучало презрительно, сегодня больше
нейтрально, поскольку телепрограмма «Наша Раша» Светлакова и Галустяна сделала
англоязычное название России милым, по-доброму смешным.

То есть, историзмом слово обладает, явно будет включено в словари. Но потеряло
негативный политический заряд, его просто «заболтали», как говорят лингвисты.
Хотя придумано было с той же целью, как и «совок», которое вряд ли очистится от
отрицательного прочтения и звучания.

Сейчас понятен путь положительного перерождения «Рашки». Уже абсолютно все
знают, что Россия — Russia (Раша). А русификация идёт по давним чертежам
«партнерского слова». Типа Алёша — Алёшка, Серёжа — Серёжка, берёза — берёзка,
дочь — дочка.

Это внутри «Рашки» так работает, но не в среде наших диссидентствующих
соотечественников за рубежом, внутренних агрессивных либералов. Ни один бложек
«критики путинизма» не обходится без резко отрицательного контекста слова.

\ii{04_02_2022.yz.istoricheskie_naperstki.1.rashka.pic.3}

Психологи объясняют ситуацию просто. Некоторым уехавшим из России людям (или
мечтающих ее покинуть) крайне важно верить — ничего хорошего на родине они не
оставили. Говорить «Россия» или «русский» — себя любимого не уважать, слова на
генетическом уровне восприятия дышат ностальгией, великой литературой и
традициями предков.

А вот слово «Рашка» и «совок»... самое то, чтобы пузыриться ненавистью или
злобой, агрессией и раздражением. Применительно к тем глупцам и дремучим
«московитам», которые никак не поймут: ЛГБТ и целовать ботинки чёрным — это
здорово, а Путин и любить свою страну — это плохо. Бедные, несчастные люди.
Пожалеем их...

\subsubsection{Рашка и Раша... это демоны.}

В Средневековье была написана одна книга, называлась «История брата Раша», даже
напечатана в Лондоне Эдвардом Алди. В ней рассказана поучительная история, как
Князь Тьмы нашёл один скатившийся в грешную разудалую жизнь монастырь с
распутными монахами и назначил демону Рашу задание — захватить там власть.

Тот явился в аббатство наниматься на службу, был принят приором и назначен
поварёнком. Потом кучей хитроумных манёвров сам стал настоятелем монастыря.
Произведение было на Западе очень популярным, неоднократно переиздавалось,
считается обязательным для изучения в университетах англосаксов по обе стороны
Атлантики по курсу «Средневековая литература».

Эрудицией и знаниями о «демоне Раше» блеснул ныне покойный Уоррен Кристофер,
госсекретарь США первого срока Клинтона, когда на одной из сессий НАТО
категорически отказался рассматривать возможность вступления России в
Североатлантический блок (представляете, времена были!). Он заявил с высокой
трибуны:

\begin{zznagolos}
«Этот демон Раша пришёл, чтобы потворствовать националистическим и
сепаратистским тенденциям, разрушать единство альянса изнутри».	
\end{zznagolos}

Так что теперь понятно, откуда в англосаксонской политической элите столь
прочны ассоциации России с «демоном».

Хотя... скорее всего, просто рубились в Госдепартаменте середины и конца 90-х в
древнюю компьютерную игрушку «Герои меча и магии» (первый релиз вышел в
1995-м). Там главный злобный Босс — демон по имени Рашка. Судя по
интеллектуальному уровню политической элиты Америки — в это поверю скорее, чем
в университетские воспоминания из средневековой литературы.
