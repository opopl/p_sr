% vim: keymap=russian-jcukenwin
%%beginhead 
 
%%file 01_08_2021.fb.sharij_anatolij.1.mechta
%%parent 01_08_2021
 
%%url https://www.facebook.com/anatolijsharij/posts/10226961748898132
 
%%author Шарий, Анатолий
%%author_id sharij_anatolij
%%author_url 
 
%%tags mechta,obschestvo,strana,ukraina
%%title Раньше у людей была мечта о коммунизме
 
%%endhead 
 
\subsection{Раньше у людей была мечта о коммунизме}
\label{sec:01_08_2021.fb.sharij_anatolij.1.mechta}
 
\Purl{https://www.facebook.com/anatolijsharij/posts/10226961748898132}
\ifcmt
 author_begin
   author_id sharij_anatolij
 author_end
\fi

Раньше у людей была мечта о коммунизме. Мне мама говорила, что большинство
жителей страны относилось к этому стремлению достаточно скептически, потому что
в СССР люди были весьма и весьма грамотные и начитанные. На мякине не
проведешь.

Союз распался. И вот уже независимым украинцам начали время от времени
подбрасывать новые мечты. Я лично хорошо помню "золото Полуботка", мол, оставил
он бочку с золотом в Швейцарии, а там уже такие проценты, что каждому жителю
Украины по килограмму. Вот уж заживем! 

Была мечта о том, что мы сами станем второй Швейцарией, ее запустил Кравчук,
по-моему. В самом начале, потом замолчали. 

Потом - накормим всю Европу. Она, бедная, голодает, мы ее накормим. 

Все изменил майдан. После первого начались мечты о вступлении в ЕС. Вот уж
тогда-то заживем точно.

Пришел второй майдан и опять вступление в ЕС. Вообще, вступление в ЕС заменило
наступление коммунизма. Только на дворе уже был не СССР и слегка люди, мягко
говоря, не такие стали начитанные. Больше веры в них появилось в чепуху, чем
знаний. 

Пришел Зеленский - и вот к мечте о вступлении в ЕС добавилась мечта о
вступлении в НАТО. Вот тогда заживем!

Прошло 30 лет, люди старели, умирали, болели, коротали свои дни в панельках,
выходя на загаженные дворы по выщербленным лестницам зассаных подъездов с
облупившейся штукатуркой. 

Через 30 лет независимости великое европейское государство пришло  к тому, что
часть его населения, в частности полностью пораженное в правах старшее
поколение, не ело говядины годами, а покупка нового телевизора стала
невозможной в принципе. 

19 миллионов человек ОФИЦИАЛЬНО за чертой бедности. 

И тогда появилась новая мечта. Это кажется нелогичным и совершенно
идиотическим, но новая власть подарила всем довольным своей прекрасной жизнью
украинцам новую мечту. Золотую. 

\ifcmt
  pic https://scontent-cdg2-1.xx.fbcdn.net/v/t1.6435-9/228427676_10226961747538098_4303852292267737895_n.jpg?_nc_cat=100&ccb=1-3&_nc_sid=730e14&_nc_ohc=dZe_gQ4t0lgAX8qiEmh&_nc_ht=scontent-cdg2-1.xx&oh=e669cd8e510d04ee6a9ec42c63b6f953&oe=612B2B1A
  width 0.4
\fi

Экстрадиция Шария.

Я не без удовольствия наблюдаю за тем, как жители все тех же обветшалых
панельных нор, получающие свои гигантские зарплаты и поедающие исключительного
качества продукты, которые могут позволить себе купить, ждут этого ЧУДА.

Экстрадиция Шария! 

Я думаю, украинская власть сама не ожидала, что количество "мало читающих" со
времен СССР так катастрофически увеличилось. 

Что оживотнивание прошло такими темпами и что эту чушь можно вбрасывать, как
силос крупному рогатому скоту, раза два в месяц стабильно. 

Нищим. Раздетым. Голодным. 

Люди за пределами европейского государства живут. Путешествуют. Пьют, танцуют,
радуются. Покупают машины, смеются, тратят деньги, забивают багажники авто
продуктами на выходные, жарят барбекю, ходят по музеям. 

И мечтают, конечно. Но о чем-то совсем другом, более нормальном, менее
фантастическом. Нет там чьих-то экстрадиций, вступлений в какие-то союзы, блоки
и прочего гна. 

Часть адекватных, "много читающих" должна выруливать из всего этого идиотизма.

Должна вырваться. 

Потому и не забиваю болт на эту часть суши, не ставлю печать "безнадежно".
Только из-за части адекватных людей с правильными мечтами.

\ii{01_08_2021.fb.sharij_anatolij.1.mechta.cmt}
