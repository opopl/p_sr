% vim: keymap=russian-jcukenwin
%%beginhead 
 
%%file 19_05_2018.stz.news.ua.mrpl_city.1.sudba_inzhenera_chernova
%%parent 19_05_2018
 
%%url https://mrpl.city/blogs/view/sudba-inzhenera-chernova
 
%%author_id burov_sergij.mariupol,news.ua.mrpl_city
%%date 
 
%%tags 
%%title Судьба инженера Чернова
 
%%endhead 
 
\subsection{Судьба инженера Чернова}
\label{sec:19_05_2018.stz.news.ua.mrpl_city.1.sudba_inzhenera_chernova}
 
\Purl{https://mrpl.city/blogs/view/sudba-inzhenera-chernova}
\ifcmt
 author_begin
   author_id burov_sergij.mariupol,news.ua.mrpl_city
 author_end
\fi

О том, что некогда существовал инженер Чернов, свидетельствуют лишь несколько
альбомчиков с аккуратно приклеенными фотографиями, его диплом об окончании
Харьковского технологического института Александра III, разрозненные
воспоминания родственников, доживших до наших дней, да конторская папка с
документами почти семидесятилетней давности, хранящаяся в архиве очень
серьезного учреждения. О содержимом этой папки речь пойдет позже, а пока
попытаемся представить жизнеописание \textbf{Андрея Васильевича Чернова}.

\ii{19_05_2018.stz.news.ua.mrpl_city.1.sudba_inzhenera_chernova.pic.1}

Он родился в Мариуполе в 1883 году на Слободке в семье стекольщика. Еще в
начальной школе обнаружил большие способности в учении, сочетавшиеся с
прилежанием. По рекомендации учителей его приняли в мариупольскую
Александровскую мужскую гимназию. За успехи в учебе, и учитывая бедность семьи,
Андрей Чернов был освобожден от платы за обучение.

Блестяще окончив гимназию, он отправился в Харьков – в то время ближайший к
Мариуполю университетский город. Он мечтал стать инженером. Харьков поразил его
огромными соборами и многоэтажными домами, которых в его родном городе тогда не
было и в помине. Успешно сдав вступительные экзамены и выдержав конкурс, Чернов
поступил в технологический институт. Для того чтобы он мог учиться,
Мариупольская мещанская управа высылала ему по 10 рублей в месяц, сумму
достаточную лишь на очень скромное существование. Поэтом студенту Чернову
приходилось давать уроки детям из состоятельных семей, да и не гнушаться
другими заработками. Тем не менее, это не помешало Андрею Васильевичу окончить
в 1912 году институт с отличием. По действующим тогда правилам диплом
инженера-механика с отличием обеспечивал получение чина X класса,
относительного высокого в чиновничьей иерархии царской России. По окончании
института Чернов некоторое время работал в Харькове у подрядчика Винарского –
владельца частной инженерной конторы, специализирующейся на железобетонных
конструкциях.

Но прервем на время повествование о карьере новоиспеченного инженера и
остановимся на событиях его личной жизни. Еще, будучи студентом, Андрей
Васильевич в 1909 году приезжает в родной город, чтобы сделать предложение
Ефросинье Михайловне Арихбаевой, миловидной девушке из большого старинного
мариупольского рода. Предложение было принято. Через некоторое время в молодой
семье родился сын Всеволод.

В мае 1913 года инженер Чернов отправляется в Архангельскую губернию и
устраивается помощником начальника службы пути на станции Няндома Северной
железной дороги. К слову скажем, что именно к этому периоду времени относятся
многие фотографии в альбомах, о которых говорилось ранее. Железнодорожное
полотно, проложенное сквозь редколесье, крушение поезда, таежный поселок,
одинокая избушка – все это запечатлел Андрей Васильевич на снимках. Судя по
всему, он был страстным фотографом. На Севере он работал ровно год.

В 1914 году инженер Чернов с семьей переезжает в Бахмут (с 1924 г. -
Артемовск), где до 1920 г. работает в земской управе инженером по дорожным
сооружениям. С установлением советской власти Андрей Васильевич продолжает
заниматься инженерной деятельностью. Но если раньше круг его обязанностей
ограничивался пределами города, то теперь, когда Артемовск стал
административным центром Донбасса, он расширился на всю Донецкую губернию.
Инженер Чернов занимает в губернских учреждениях последовательно должности
заведующего дорожным отделом, заведующего стройконторой, заведующего подотделом
благоустройства, промышленного инспектора, инженера по водоснабжению, главного
инженера Артемовского райуправления Донбассводтреста - организации, которая
занималась водоснабжением донецкого края. В 1932 году столицей вновь
образованной области провозглашают город Сталино – нынешний Донецк. Спустя два
года туда переехал Донбассводтрест вместе со всеми сотрудниками, а с ними и
Андрей Васильевич. Теперь он начальник строительно-монтажного сектора треста.
Но независимо от названий должностей, ему приходится заниматься одним и тем же:
строить дороги, мосты, плотины, устраивать водохранилища. К слову сказать, он
непосредственно участвовал в создании Старокрымского водохранилища, которое и
по сей день обеспечивает водой Мариуполь.

Как до революции, так и при советской власти семья жила размеренной
обеспеченной жизнью. Ефросинья Михайловна не работала, занималась домашним
хозяйством и воспитанием сына. Да и где бы она могла работать, ведь у нее не
было никакой специальности. По понятиям старого времени замужней женщине и не
было резона ее приобретать. Андрей Васильевич мог позволить себе съездить на
курорт. Сохранилась фотография, датированная 1932 годом: чета Черновых в
Ессентуках. Но все благополучие рухнуло 2 февраля 1938 года. Именно в этот
злосчастный день инженера Чернова, только что вернувшегося из командировки,
арестовали прямо на железнодорожном вокзале в Сталино.

Настало время обратиться к содержимому папки, упоминавшейся выше. В этой папке
собраны документы следственного дела по обвинению Чернова А.В. в участии в
контрреволюционной вредительской организации. Постановление об аресте было
принято отделом НКВД 27 декабря 1937 года. Его дубликат направили прокурору
области Р. А. Руденко. Тот завизировал постановление, добавив к фразе \emph{\enquote{является
участником контрреволюционной вредительской организации}} еще одно слово –
\emph{\enquote{троцкистской}}, что означало расстрельную статью. Он подписал ордер на арест
Андрея Васильевича и содержание его под стражей. Но арестован он был не сразу,
а примерно через два месяца после утверждения ордера. Первый допрос Чернова
состоялся в день ареста. Его обвиняли во вредительстве при строительстве
водохранилищ, в намеренном перерасходе средств при строительстве
гидротехнических сооружений и затягивании сроков их ввода в эксплуатацию, а
также в вербовке сотрудников \enquote{Донбассводтреста} в преступную организацию.
Андрей Васильевич не признает обвинения, но уже 5 февраля, через три дня после
ареста, как это значится в протоколе допроса, он во всем \enquote{сознается}. Можно
только догадываться, как удалось добиться следователям \enquote{признания} вины у
невиновного человека.

Начались бесконечные допросы. Протоколов этих допросов в деле множество. В
апреле, когда следственное дело было закончено, Чернов подписывает документ,
что он не имеет претензий к следователям. Материалы дела направляют в Киев, там
получают \enquote{добро}. Затем они попадают в Верховный Суд СССР. Выездная коллегия
Верховного Суда приговаривает Чернова к высшей мере наказания – расстрелу.
Приговор был приведен в исполнение 5 сентября 1938 года.

Семье сообщили, что Андрей Васильевич осужден на десять лет лагерей без права
переписки. Его сына Всеволода исключили из числа студентов индустриального
института. Ефросинья Михайловна, потеряв кормильца, перебралась в Мариуполь. Ее
как могли поддерживали родственники. Лишь в 1956 году она отважилась направить
письмо Н.С. Хрущеву, как члену Президиума Верховного Совета СССР. Она просила
освободить мужа, поскольку срок его наказания давно истек. Ее надежда увидеть
мужа живым рухнула, когда из Москвы пришел пакет с документом, в котором
значилось, что А.В. Чернов умер в 1940 году в местах заключения. Как позже
стало известно, что Андрей Васильевич был расстрелян 5 сентября 1938 г.
Ефросинья Михайловна доживала свой век в крайней нужде.

Через десятилетия после расстрела инженер Чернов был реабилитирован. Он был ни
в чем не виноват.

\emph{PS. Этого очерка не было бы, если родственник Андрея Васильевича
инженер-конструктор, изобретатель Георгий Митрофанович Сорока не сохранил
документы инженера Чернова, воспоминания о нем, а известная правозащитница и
писательница Галина Михайловна Захарова не помогла автору этих строк
ознакомиться с \enquote{расстрельным} делом А. В. Чернова.}
