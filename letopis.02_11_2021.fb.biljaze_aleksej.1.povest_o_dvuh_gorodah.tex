% vim: keymap=russian-jcukenwin
%%beginhead 
 
%%file 02_11_2021.fb.biljaze_aleksej.1.povest_o_dvuh_gorodah
%%parent 02_11_2021
 
%%url https://www.facebook.com/aleksey.bilyaze/posts/4487552498030117
 
%%author_id biljaze_aleksej
%%date 
 
%%tags ekonomika,genuja,istoria,italia,srednevekovie,torgovlja,venecia
%%title УРОКИ СРЕДНЕВЕКОВОЙ ТОРГОВЛИ. ПОВЕСТЬ О ДВУХ ГОРОДАХ
 
%%endhead 
 
\subsection{УРОКИ СРЕДНЕВЕКОВОЙ ТОРГОВЛИ. ПОВЕСТЬ О ДВУХ ГОРОДАХ}
\label{sec:02_11_2021.fb.biljaze_aleksej.1.povest_o_dvuh_gorodah}
 
\Purl{https://www.facebook.com/aleksey.bilyaze/posts/4487552498030117}
\ifcmt
 author_begin
   author_id biljaze_aleksej
 author_end
\fi

УРОКИ СРЕДНЕВЕКОВОЙ ТОРГОВЛИ. ПОВЕСТЬ О ДВУХ ГОРОДАХ

Жители поселений вокруг Венецианской лагуны основали Венецию в 697 г.; жители
Генуи организовались в общину около 1096 г. К середине XIV века Венеция и Генуя
стали двумя наиболее успешными в коммерческом отношении городами-государствами
во всей Европе.

\ifcmt
  ig https://scontent-frt3-1.xx.fbcdn.net/v/t39.30808-6/250988661_4487552194696814_5032875347736702053_n.jpg?_nc_cat=108&ccb=1-5&_nc_sid=730e14&_nc_ohc=voLGKZ8_bu4AX-IX_QC&_nc_ht=scontent-frt3-1.xx&oh=0363e699a9edfb41868f9b2d2a5caa60&oe=619C6ACA
  @width 0.45
  %@wrap \parpic[r]
  @wrap \InsertBoxR{0}
\fi

I. ВДАЛИ ОТ ИМПЕРИЙ

Возвышение городов произошло в XII веке на фоне упадка мусульманских и
византийских сил в Средиземноморье. На города претендовали более крупные
политические образования: Византийская империя на Венецию и Священная Римская
империя на Геную. Города формально признавали власть империй, но де факто были
самостоятельными.

Политическая организация и институты в Генуе и Венеции внешне кажутся
одинаковыми. Оба города управлялись олигархами, а политические лидеры – были de
jure обычными гражданами, которые избирались и подчинялись закону. Так, Венеция
управлялась дожем и герцогским советом, а Генуя после 1194 г. –
градоначальником-подеста (podesta) и советом ректоров. Политические институты,
установившиеся в городах, поддерживали межклановую кооперацию, подстегивали
коммерческую экспансию и способствовали политическому порядку.

Несмотря на сходства и подобие траектории развития на первых этапах, в
дальнейшем судьба в корне отличалась, а упомянутые траектории развития –
разошлись. Венеция оказалась способной поддержать политический порядок в
меняющемся мире и мобилизовать ресурсы для экономического роста даже с началом
эпохи Великих географических открытий. Генуя не отличалась такой же
политической стабильностью, и частые внутренние кризисы стали для нее причиной
экономического упадка. Институты Венеции оказались самоподдерживающимися, тогда
как в Генуе – саморазрушающимися.

II. ГЕНУЯ

В течение первого столетия существования Генуи (1096-1194 гг.) выборные консулы
были политическими, административными и военными лидерами. Обычно, консулами
становились представители самых влиятельных кланов, и нередко они использовали
свою власть для создания эксклюзивной ренты для своего клана. Первый клан
состоял из семей Фиески и Гримальди, а второй – из семей Дориа и Спинола.

Крупные семьи при помощи особого института – патронажа-альберги –
концентрировали вокруг себя более мелкие семьи. Эти связи выстраивались при
помощи формальных договоров, на основании которого семья-принципал обязывалась
опекать семью-агента, а семья-агент обязывалась согласовывать свои действия с
семьей-принципалом. Кроме того, агент получал право использовать фамилию
семьи-принципала, вроде Дориа-Памфили или Гримальди-Сева.

Создание двух кланов из двух семей каждый привело к началу гонки вооружений:
каждый из кланов понимал, что как только оппонент ощутит его слабость, жди
беды. В итоге, каждый из кланов сосредоточился на обретении политического и
экономического господства в городе, с тем чтобы через время конвертировать его
в военную мощь. От насилия по факту кланы удерживало лишь наличие сопоставимой
военной силы у клана-соперника, и некоторое время кланы сосуществовали и даже
сотрудничали в условиях вооруженного нейтралитета.

Период политической стабильности поспособствовал экономическому росту. Но по
мере накопления богатства Генуей, власть над городом становилась все более
привлекательной, так что сложившийся пакт о ненападении постепенно исчерпал.
Гонка вооружений ускорилась: кланы скупали недвижимость и превращали ее в
укрепленные районы, некоторые из которых напоминали полноценные форты;
расширялись сети патронажа-альберги; участились заказные убийства и кровная
месть за них.

Мир сохранялся лишь по причине постоянного наличия внешней угрозы. Например, в
1154 г. очередную попытку установить контроль над Генуей сделал Фридрих
Барбаросса. Под влиянием этой угрозы кланы сосредоточили ресурсы на
противодействии ему, но после смерти Барбароссы начался новый виток гонки
вооружений, и на этот раз она вылилась в гражданскую войну.

Коммуна увязла в длительной гражданской войне, в течение которой кланы
попеременно получали верховную власть. Но при любом изменении в геополитике,
второй клан сразу же привлекал новых союзников и бросал вызов клану-правителю.
Заговоры и междоусобицы были особенно ожесточенными в 1189-1194 гг., когда даже
само дальнейшее существование города было поставлено под вопрос.

Но в 1194 г. императору Священной Римской империи Генриху IV для войны с
Сицилией понадобилось содействие генуэзского флота, что было невозможно в
условиях гражданской войны. Обещая награды и угрожая войной, Генрих IV принудил
кланы к перемирию, гарантом которого стал новый для Генуи институт – подеста.

Подеста избирался комитетом представителей окрестностей Генуи, которые были
настолько обширными, что патронажные сети кланов были не в состоянии их
охватить. Нанимаемый на год, подеста выполнял функции военного лидера,
верховного судьи и управленца коммунальным хозяйством. Помимо прочего, именно
подеста подчинялись муниципальная милиция и особый военный контингент
наемников. Этот контингент был равноудален от обоих кланов, и стал особой
гарантией в их отношениях.

Как институт, подеста оказался в недостаточной мере самоподдерживающимся. Как и
консульская система, подеста изначально нес в себе семена краха.

Необходимость в подеста была обусловлена необходимостью поиска баланса сил
между кланами. Проблема была в том, что силы подеста были ограничены и не имели
собственных источников для роста. Из этого следовала недолговечность института:
экономический рост в Генуе приводил к усилению кланов, богатство
конвертировалось в военную мощь, и в определенный момент военные силы,
контролируемые подеста, оказались несущественными относительно сил кланов.

Снова оказавшись на пороге гражданской войны, в 1339 г. Генуя провела
политическую реформу и ввела институт выборных дожей по аналогии с Венецией,
которая к тому времени уже опередила Геную в развитии.

Тем не менее, время было упущено. В 1347-1353 гг. Черная смерть унесла и в
Генуе, и в Венеции по 80 тысяч человек, или по 60% населения. Более
институционально-устойчивая Венеция достаточно восстановилась быстро, тогда как
Генуя, которая только-только провела политическую реформу и не успела
выработать привыкание горожан к новой системе управления городом, была
отброшена в развитии.

III. ВЕНЕЦИЯ

Ранняя история Венеции совпадает с историей Генуи. Кланы боролись за пост дожа:
вначале представителя Византийской империи, а с момента обретения Венецией в
679 г. независимости – выборного монарха, сосредоточившего в своих руках
судебную, законодательную и исполнительную власть. В 1032 г. власть дожей была
ограничена установлением совета консулов, а в 1172 г. венецианцы постановили,
что действия дожа никогда не должны противоречить рекомендациям его советников,
так что Венеция превратилась из монархии в республику.

Но в конце XII века, в момент упадка Византии, перед Венецией открылись новые
перспективы. В частности, появилась возможность установить контроль над
торговыми маршрутами. Чтобы воспользоваться этой возможность, Венеции следовало
объединиться. Обеспечить это объединение должно было соглашение между кланами,
согласно которому все они обязались объявить войну клану-отступнику,
попробовавшему узурпировать власть. Это нововведение не просто остановило гонку
вооружений между кланами, но и привело к появлению особой венецианской
идентичности, нечто вроде локального патриотизма – явления, которое не
наблюдалось в Генуе.

Несмотря на то, что в Венеции тоже существовали патронажные сети, использование
кланами этих сетей во время выборов дожа было затруднено процедурой выборов.
Для избрания дожа при помощи жребия и обсуждений из Большого совета
формировался избирательный комитет. Этот комитет опять же при помощи жребия и
обсуждений составлял список кандидатов на пост дожа. Задача жребия была снизить
роль патронажа, а кроме того, в комитете мог быть только один представитель
клана, и он был обязан брать самоотвод всякий раз, когда шло рассмотрение
кого-то из его родственников.

Подобным, пусть и менее сложным образом, избиралась большая часть чиновников
Венеции. Их число было значительным, а срок пребывания на посту – коротким, так
что в определенный промежуток времени занимать посты могли члены сразу
нескольких кланов. Это служило дополнительным источником баланса сил. А чтобы
не допустить извлечения представителями власти ренты из своих должностей, за
деятельностью каждого чиновника следил комитет, обычно наполненный
представителями других кланов.

Система сдерживания и противовесов в отношениях кланов, помимо прочего,
повышала перспективы обычных граждан Венеции, и вырабатывало у граждан
лояльность городу. В целом, граждане Венеции, не аффилированные с кланами,
управляли своим городом в большей степени, чем граждане Генуи.

IV. РОЛЬ И ВАЖНОСТЬ ИНКЛЮЗИИ

Институты, которые сложились в Венеции, поощряли инклюзию: участие относительно
больших групп населения в экономической активности.

Важные компоненты инклюзии – защита прав собственности, беспристрастная система
правосудия и разные возможности для участия всех граждан в экономической
активности, в Венеции были представлены в большей степени, чем в Генуе. Эта
инклюзия, помимо прочего, еще и  способствовала повышению качества
человеческого капитала за счет более эффективного применения талантов и
навыков. Как следствие, экономика Венеции росла быстрее, чем экономика Генуи.

Инклюзивные институты требуют, чтобы доступом к экономическим процессам и
защитой прав собственности пользовалось как можно большее количество граждан. В
Генуе, где так или иначе правили кланы, пусть и ограниченные подеста, те, кто
принял патронаж этих кланов посредством альберги, имели преимущества перед
прочими горожанами. В Венеции подобный институт развития не получил, и, как
следствие, Венеция была более инклюзивной, нежели Генуя. Это не имело особого
влияния здесь и сейчас, но определило траекторию развития городов.

Примечательно, что даже после открытий Христофора Колумба и Васко да Гама,
когда Венеция и Генуя утратят свою роль эксклюзивных посредников в торговле с
востоком, Генуя придет в упадок, а Венеция – обретет новую специализацию:
выпуск товаров и организация развлечений премиум-класса. Это позволит Венеции
сносно существовать до наших дней, став в конечном итоге одним из наиболее
романтических городов на земле. До эпидемии Венецию посещало около 25 млн.,
тогда как Геную – не более 1 млн.

\ii{02_11_2021.fb.biljaze_aleksej.1.povest_o_dvuh_gorodah.cmt}
