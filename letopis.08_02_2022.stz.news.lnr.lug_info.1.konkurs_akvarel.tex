% vim: keymap=russian-jcukenwin
%%beginhead 
 
%%file 08_02_2022.stz.news.lnr.lug_info.1.konkurs_akvarel
%%parent 08_02_2022
 
%%url https://lug-info.com/news/uchashiesya-iz-dvuh-regionov-lnr-i-rf-napravili-525-rabot-na-konkurs-volshebstvo-akvareli
 
%%author_id news.lnr.lug_info
%%date 
 
%%tags akvarel,donbass,isskustvo,konkurs,lnr,rossia
%%title Учащиеся из двух регионов ЛНР и РФ направили 525 работ на конкурс "Волшебство акварели"
 
%%endhead 
 
\subsection{Учащиеся из двух регионов ЛНР и РФ направили 525 работ на конкурс \enquote{Волшебство акварели}}
\label{sec:08_02_2022.stz.news.lnr.lug_info.1.konkurs_akvarel}
 
\Purl{https://lug-info.com/news/uchashiesya-iz-dvuh-regionov-lnr-i-rf-napravili-525-rabot-na-konkurs-volshebstvo-akvareli}
\ifcmt
 author_begin
   author_id news.lnr.lug_info
 author_end
\fi

%Pupils of educational institutions of Perevalsk district (Lugansk People's Republic) 
%and Ulyanosk region (Russian Federation) have directed 525 artistic works to the competition 
%\enquote{The Magic of Aquarel}. This was reported by the Ministry of Education and Science of 
%Lugansk People's Republic.

Обучающиеся образовательных учреждений Перевальского района и Ульяновской
области направили 525 творческих работ на конкурс \enquote{Волшебство акварели}. Об
этом сообщили в Министерстве образования и науки ЛНР.

\href{https://www.youtube.com/watch?v=M_gB1QB8Zis}{%
Конкурс Волшебство акварели Творческие работы победителей, youtube, 08.02.2022%
}

\ii{08_02_2022.stz.news.lnr.lug_info.1.konkurs_akvarel.scr.1}

\enquote{На базе Бугаевской средней школы № 14 был проведен дистанционный конкурс
\enquote{Волшебство акварели}. Всего на конкурсе было представлено 525 работ из
образовательных учреждений Перевальского района и образовательных учреждений
Ульяновской области РФ}, - говорится в сообщении.

\ii{08_02_2022.stz.news.lnr.lug_info.1.konkurs_akvarel.scr.2}

Творческое состязание в номинациях \enquote{Мир, в котором я живу} и \enquote{Алиса в стране
чудес} проводилось с 10 января по 1 февраля при поддержке Минобразования ЛНР и
Министерства просвещения и воспитания Ульяновской области. Его участниками
стали учащиеся пяти возрастных категорий: 7-8 лет, 9-10 лет, 11-12 лет, 13-14
лет, 15-17 лет.

\ii{08_02_2022.stz.news.lnr.lug_info.1.konkurs_akvarel.scr.3}

\enquote{Поскольку конкурс не ограничивал детское творчество, то рисунки могли быть
выполнены на любом материале (бумага, картон, холст) и исполнены в любой
технике рисования (масло, акварель, тушь, цветные карандаши) в формате А4}, -
рассказали в Минобразования.

\ii{08_02_2022.stz.news.lnr.lug_info.1.konkurs_akvarel.scr.4}

Оценивание работ проводилось на сайте бугаевской школы № 14 путем голосования,
в нем приняли участие 6 324 человека. В каждой номинации и возрастной группе
были определены по три победителя.

\ii{08_02_2022.stz.news.lnr.lug_info.1.konkurs_akvarel.scr.5}
\ii{08_02_2022.stz.news.lnr.lug_info.1.konkurs_akvarel.scr.6}
\ii{08_02_2022.stz.news.lnr.lug_info.1.konkurs_akvarel.scr.7}
\ii{08_02_2022.stz.news.lnr.lug_info.1.konkurs_akvarel.scr.8}
\ii{08_02_2022.stz.news.lnr.lug_info.1.konkurs_akvarel.scr.9}
\ii{08_02_2022.stz.news.lnr.lug_info.1.konkurs_akvarel.scr.10}

%2:38
