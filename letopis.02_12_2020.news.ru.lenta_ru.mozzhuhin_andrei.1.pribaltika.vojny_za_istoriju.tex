% vim: keymap=russian-jcukenwin
%%beginhead 
 
%%file 02_12_2020.news.ru.lenta_ru.mozzhuhin_andrei.1.pribaltika.vojny_za_istoriju
%%parent 02_12_2020.news.ru.lenta_ru.mozzhuhin_andrei.1.pribaltika
 
%%url 
 
%%author 
%%author_id 
%%author_url 
 
%%tags 
%%title 
 
%%endhead 

\subsubsection{Войны за историю}
\label{sec:02_12_2020.news.ru.lenta_ru.mozzhuhin_andrei.1.pribaltika.vojny_za_istoriju}

\lenta{Но ведь во всех этих исторических спорах речь идет не о конкретных фактах, а об
их интерпретациях.}

Не всегда. Мне странно слышать разговоры о «недопустимости переписывания
истории». В этом есть лукавство и отчетливый политический подтекст.

Потому что под «переписыванием» явно подразумевается вскрытие для кого-то
неудобных фактов и документов. Историю никогда невозможно переписать, но
дописать ее на основе ранее неизвестных фактов и документов, порой и
переосмысления и сопоставления, можно и даже необходимо. Если ее пишут раз и
навсегда, а все, что не укладывается в «утвержденный» канон, вычеркивается или
цензурируется, то она из науки неизбежно превращается в идеологию.

\lenta{В Прибалтике, и не только там, нам теперь предлагают «платить и каяться»,
ссылаясь на исторические факты, которые замалчивались в советское время.}

Раскаяние не тождественно извинению, это добровольное осознание и преодоление.

\begin{leftbar}
	\large
Наше прошлое таково, что всем европейским народам есть за что перед друг
другом каяться — и еще неизвестно, кому перед кем больше. Нельзя целые
нации разделять на безгрешных мучеников и отъявленных злодеев
\end{leftbar}

А вот ситуация, когда пособники нацистов, виновные в смерти и бесчестии тысяч
своих сограждан, теперь героизируются как «борцы за независимость», должна
вызвать тревогу самих тамошних элит. И быть в поле внимания Евросоюза, который
сквозь пальцы смотрит на весьма симптоматичные факты.

\lenta{Какие факты?}

Например, на марши сторонников легионеров СС — а ведь вступившие в балтийские
легионы давали присягу на верность лично Гитлеру. На торжественные
перезахоронения бежавших с немцами «идейных марионеток», проводивших
мобилизацию в СС, — как это произошло с генералом Рудольфом Бангерскисом,
перезахороненным на Братском воинском кладбище Риги.

\ifcmt
pic https://icdn.lenta.ru/images/2020/10/21/13/20201021132616411/pic_643a037a7630175c368a887cd633c6d7.jpg
caption Надгробие на могиле группенфюрера СС Рудольфа Бангерскиса на Братском воинском кладбище Риги. Его прах с воинскими почестями перенесли туда из Германии в марте 1995 г.  \href{https://upload.wikimedia.org/wikipedia/commons/0/06/BraluKapi_Bangerskis.jpg}{Фото: Wikipedia}
\fi

\index[names.rus]{Бангерскис, Рудольф!Группенфюрер СС}

Именем Казиса Шкирпы, сотрудничавшего с абвером и возглавившего летом 1941-го
уже упоминавшееся мною пронацистское «Временное правительство Литвы», названа
улица в Каунасе. А эстонские школьные учебники и вовсе пишут о том, что
«Эстония оказалась в числе государств, проигравших Вторую мировую войну». Такие
историко-политические метастазы губительны для массового сознания.

\begin{leftbar}
	\bfseries
	Материалы по теме
00:01 — 22 октября 2019
Память подвела
\href{https://lenta.ru/articles/2019/10/22/free_riga/}{В Латвии придумали новую версию освобождения от нацистов — без Красной армии}
\ifcmt
	ig https://icdn.lenta.ru/images/2019/10/21/17/20191021171430951/tabloid_02f66daf54badae48e9f739e63fc7b5d.png
	width 0.2
\fi
\end{leftbar}

Политические элиты государств Балтии сегодня строят новую идентичность своих
народов на антироссийских (даже не антисоветских) идеологемах. Поэтому все их
претензии никак не зависят от существующего у нас политического контекста: в
1990-е годы они относились к нашей стране, из рук которой получили
независимость — второй раз за ХХ столетие, — почти так же, как теперь.
Предъявлять нам счет за десятилетия «оккупации» абсолютно бесперспективно.

\lenta{Но они пока так не считают.}

Увы. Сейчас в этих государствах предпочитают вспоминать только про депортации и
репрессии (а у нас как раз предпочитают об этом не вспоминать), но совершенно
забыли, что именно Советский Союз спас литовцев, латышей и эстонцев не просто
от онемечивания, а от тотального уничтожения — культурного и физического.

\begin{leftbar}
	\large
Политики воюют за историю, когда им нечего больше предложить своим гражданам,
когда у них нет твердой почвы под ногами. Но войны за историю никогда
ни к чему конструктивному не приводили
\end{leftbar}

Их цель — не поиск истины, а стремление оправдать чьи-то действия в прошлом.
Поэтому пока что мы должны стремиться хотя бы к открытому и честному диалогу со
своими ближайшими соседями в Европе, чтобы иметь возможность спокойно обсуждать
с ними спорные вопросы нашего общего прошлого.

Беседовал \href{https://lenta.ru/parts/authors/mozjukhin/}{Андрей Мозжухин}
