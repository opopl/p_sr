% vim: keymap=russian-jcukenwin
%%beginhead 
 
%%file 16_02_2022.tg.tkachev_jurij.2.delaj_chto_dolzhen
%%parent 16_02_2022
 
%%url https://t.me/dadzibao/5299
 
%%author_id tkachev_jurij
%%date 
 
%%tags oppozicia,opzzh,politika,rossia,ugroza,ukraina
%%title Делай что должен - и будь что будет
 
%%endhead 
 
\subsection{Делай что должен - и будь что будет}
\label{sec:16_02_2022.tg.tkachev_jurij.2.delaj_chto_dolzhen}
 
\Purl{https://t.me/dadzibao/5299}
\ifcmt
 author_begin
   author_id tkachev_jurij
 author_end
\fi

Нардепы от ОПЗЖ, уехавшие из Украины в последние дни, дали свои объяснения
тому, почему уехали. По их словам, уехали они не из страха перед российским
нападением, в которое не верят, а из-за угрозы зачистки со стороны связки
спецслужб и активистов. 

Лично я нисколько не сомневаюсь в том, что акции, о которых идёт реально
готовятся и несомненно пройдут в случае любого обострения ситуации - будь то
\enquote{российская агрессия}, в которую лично я не верю, агрессия Украины в
Донбассе, которой я опасаюсь. Либо просто если украинские власти решат, что
такая зачистка целесообразна. 

Скажу больше: я не сомневаюсь, что это не просто возможно, но и рано или поздно
неизбежно произойдёт в тех или иных масштабах. 

Однако даже с учётом этого я считаю, что отъезд из страны в такой ситуации
является действием, для политика недопустимым.

Любой человек, находящийся сегодня в реальной оппозиции к действующему режиму,
должен отдавать себе отчёт, что борется с реальным злом. И жертвы в этой борьбе
неизбежны. Сбежать за границу при признаках угрозы - значит вынудить зло искать
себе другие жертвы: зачистка, о которой идёт речь, нужна ведь даже не столько
для того, чтобы \enquote{выключить} тех или иных людей, которых считают источником
угрозы (таких уже или выключили, или выдавили из страны уже давно). Речь идёт
об акте террора, призванном запугать других несогласных и сломить их волю к
сопротивлению. То есть, кого-то зачищать будут в любом случае. Убегая, бегущий
автоматически подставляет под зачистку кого-то ещё. 

Я вовсе не хочу обвинить в трусости и предательстве любого, кто в этих условиях
предпочёл покинуть страну и оставаться за её пределами. К примеру, когда уехали
Шарий или главред \enquote{Страны} Гужва, это было разумно и правильно. Эти люди делают
колоссальное дело, мешая монополизации информационного пространства. Эти люди
должны быть живы, на свободе и продолжать свою работу.

Но - уж простите, если кого обижу - какая польза делу от условного нардепа от
ОПЗЖ?  Будучи в оппозиции, они не могут влиять на принимаемые в государстве
решения - и мы видим, что они на них не влияют, по крайней мере в том смысле, в
котором должны были бы. Единственный смысл существования такой категории
сегодня - это быть лицом, знаменем, символом, лидером, тем, на кого
ориентируются, тем, кто подаёт пример.

И более того, в мирное время те же нардепы имеют от своего статуса массу выгод
и преимуществ. Так почему же во время сложное, когда опасность становится
реальной (или чуть более реальной, чем обычно) эти люди, вместо того, чтобы
отрабатывать свой статус лидера и примера для подражания, показывают спину,
причём в том числе и своим избирателям?

Да, я говорю о жертвенности, и не скрываю этого. И я понимаю, что нельзя
требовать ото всех готовности к мученичеству. Но тем, кто к нему не готов,
нечего браться за борьбу со злом и, в частности, становиться публичным
политиком на оппозиционном фланге в полицейском государстве, коим является
Украина. А раз уж назвался груздем - полезай в кузов. 

Многие философии и идеологии, бросавшие вызов установившемуся порядку,
добивались успехов и утверждались именно на крови своих убеждённых
приверженцев. Очень жаль, что мир так устроен, но он устроен именно так. В том
же христианстве высочайше чтят мучеников за веру: именно эти люди составляют
большую часть списка христианских святых. 

Да, в том же христианстве осуждается сознательное стремление к мученичеству.
Однако есть разница между минимизацией угроз и капитуляцией, отказом от своей
миссии как таковой.

И да, всё вышеизложенное я пишу с уверенностью в том, что ко мне лично в Одессе
\enquote{чистильщики} отправятся одним из первых. Более того, я морально и технически
готов к такому развитию событий с лета 2014 года. Но при этом никуда не уезжаю
и уезжать не собираюсь.

\enquote{Делай что должен - и будь что будет} - вот тот единственный принцип, по
которому, на мой взгляд, можно жить сегодня в Украине. Сам так поступаю и всем
советую.
