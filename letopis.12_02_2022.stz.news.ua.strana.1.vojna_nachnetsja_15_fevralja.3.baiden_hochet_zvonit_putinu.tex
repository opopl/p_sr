% vim: keymap=russian-jcukenwin
%%beginhead 
 
%%file 12_02_2022.stz.news.ua.strana.1.vojna_nachnetsja_15_fevralja.3.baiden_hochet_zvonit_putinu
%%parent 12_02_2022.stz.news.ua.strana.1.vojna_nachnetsja_15_fevralja
 
%%url 
 
%%author_id 
%%date 
 
%%tags 
%%title 
 
%%endhead 

\subsubsection{Байден хочет звонить Путину}

При этом на фоне разжигания нового пожара якобы неизбежной войны, появились
отчетливые сигналы, что может ускориться как раз не военное, а дипломатическое
направление.

Американский президент сегодня экстренно созвонился с лидерами Британии,
Италии, Польши, Румынии, Франции, ФРГ, ЕС и НАТО.

Беседа шла почти полтора часа и касалась Украины. Стороны договорились \enquote{удвоить
дипломатические усилия}. При этом британская The Guardian сообщила, что
президент США сообщил союзникам: вторжение России состоится в течение
нескольких дней. 

Неизвестно, действительно ли Байден такое говорил. Но стоит вспомнить, что он
обзванивал европейцев накануне своей встречи с Путиным в начале декабря. Чтобы
сверить позиции. 

И здесь интересно, что также готовится новый контакт с российским президентом.
В Белом доме официально заявили о необходимости срочного телефонного разговора
Байдена с Путиным.

Параллельно CNN сообщила, что сотрудники Госдепа занимаются подготовкой личных
встреч между высокопоставленными представителями Вашингтона и Москвы на разных
уровнях - от президентов до министров иностранных дел.

То есть на фоне военной истерии могут возобновиться высокоуровневые контакты
Путина и Байдена. Которые, не исключено, что закончатся громогласной
\enquote{перемогой} и отменой войны (подготовку к которой, повторимся, Россия
настойчиво отрицает).

Но это лишь один из сценариев дальнейшего развития ситуации.

Разберем четыре основных версии, куда события могут пойти дальше. 

