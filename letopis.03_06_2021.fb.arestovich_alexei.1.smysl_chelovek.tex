% vim: keymap=russian-jcukenwin
%%beginhead 
 
%%file 03_06_2021.fb.arestovich_alexei.1.smysl_chelovek
%%parent 03_06_2021
 
%%url https://www.facebook.com/alexey.arestovich/posts/4386203751443671
 
%%author Арестович, Алексей
%%author_id arestovich_alexei
%%author_url 
 
%%tags 
%%title Познавая самого себя, человек обретает смысл себя
 
%%endhead 
 
\subsection{Познавая самого себя, человек обретает смысл себя}
\label{sec:03_06_2021.fb.arestovich_alexei.1.smysl_chelovek}
\Purl{https://www.facebook.com/alexey.arestovich/posts/4386203751443671}
\ifcmt
 author_begin
   author_id arestovich_alexei
 author_end
\fi

- Что объединяет атеиста и верующего?
Сомнения.
Куда жить? Как жить? Что есть добро и зло? Морален ли я и мое сообщество или аморальны?
Что в будущем? Нужно ли жертвовать сегодня ради завтра? В чем смысл прошлых страданий?
Верующий и атеист объединены проблемой смысла.
Где-то с 1500 года, мысль человеческая пошла путём конструирования.
Мы научились измерять и изменять мир, но разучились измерять и изменять самих себя.
Сомнения возросли. 

\ifcmt
  pic https://scontent-cdt1-1.xx.fbcdn.net/v/t1.6435-9/195793382_4386195604777819_7820907512040054935_n.jpg?_nc_cat=105&ccb=1-3&_nc_sid=8bfeb9&_nc_ohc=UGe8vUX2eVYAX_7STao&_nc_ht=scontent-cdt1-1.xx&oh=df66f3f457df932d600e564d7dd0ba43&oe=60DEC541
\fi

Истинное знание - это знание причины.
Сомнения происходят от незнания смысла мысли, жизни, деятельности. 
Смысл не есть производное от знания.
Смысл не может быть сконструирован.
Смысл может быть только обретён.
Современный человек предпочитает рассчитать и сделать.
А смысл как раз про то, что нельзя рассчитать и сделать.
К счастью, за нас в своё время постарались древние греки.

Они совершили несколько фундаментальных открытий в области понимания мира и нахождения ему смысла.

Одно из важнейших - «сущее есть должное», т.е. предназначение человека
проистекает непосредственно из устройства его бытия, мира-как-он-есть.

А как дан мир человеку? Через его личную картину мира. 

Это значит, что человек, желающий обрести смысл и понять своё предназначение,
должен знать свою картину мира и уметь разобраться в ней.

Понимать свою картину мира и выводить из неё личное предназначение мы с вами
будем на семинаре в субботу, 12 июня, с 17.00 до 21.00, онлайн.

Этот семинар - один из интерактивных, когда после первой части теории, мы
разбираем кейсы участников семинара - помогая им понимать и описывать их личные
картины мира и понимать - что в связи с этим новым знанием им следует сделать
со своим мышлением, образом жизни и деятельности?   

Познавая самого себя, человек обретает смысл себя.

На семинаре Вы получите теорию, алгоритмы и способы описания собственной (и
чужой) картины мира, понимания предназначения, вытекающего из неё, и его
реализации в своей мысли, образе жизни и деятельности.

————————
Записаться: 
рус:
\url{https://go.apeiron.school/kartina-mira-ru-fb}
укр:
\url{https://go.apeiron.school/kartyna-svitu-ua-fb}

Контакти адміністратора/Контакты администратора:

Facebook: \url{https://www.facebook.com/kate.ksiondzyk/} 
Email: \url{kateksiondzyk@apeiron.school} 
Телефон: +38(063)0680165
