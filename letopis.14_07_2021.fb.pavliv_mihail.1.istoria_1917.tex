% vim: keymap=russian-jcukenwin
%%beginhead 
 
%%file 14_07_2021.fb.pavliv_mihail.1.istoria_1917
%%parent 14_07_2021
 
%%url https://www.facebook.com/m.pavliv/posts/3459548810814763
 
%%author Павлив, Михаил
%%author_id pavliv_mihail
%%author_url 
 
%%tags 1917,istoria,rossia,rusmir
%%title И кошмары Майдана, Одессы, Донбасса имеют своих близнецов братьев за сто лет до того в лютом 1917
 
%%endhead 
 
\subsection{И кошмары Майдана, Одессы, Донбасса имеют своих близнецов братьев за сто лет до того в лютом 1917}
\label{sec:14_07_2021.fb.pavliv_mihail.1.istoria_1917}
 
\Purl{https://www.facebook.com/m.pavliv/posts/3459548810814763}
\ifcmt
 author_begin
   author_id pavliv_mihail
 author_end
\fi

Со дня на день будем поминать черные дни 103 года назад когда взял наш народ на
душу великий грех и нарушил клятву данную пред Богом за триста лет до того на
Земском соборе, измарав себя цареубийством и гибелью алапаевских мучеников.

А сколько еще было безвинных душ сгинувших до того и особенно после того?
Сотни, тысячи, миллионы.

Кровавое помешательство, скудоумие и зверства. Русский бунт, бессмысленный и
беспощадный.

У всех этих гекатомб трупов были свои предтечи. Ненависть и людоедство толпы не
враз рождается.

И кошмары Майдана, Одессы, Донбасса имеют своих близнецов братьев за сто лет до
того в лютом 1917.

Просто почитайте строки ниже. Просто почитайте об этих предвестниках великой
крови.

Толпа, чернь, зло. И тонкая линия в погонах. 

(24 февраля/9 марта)

"…Все они знали, что им запрещено применять оружие, а против них — можно. Они
знали своих вчерашних раненых и избитых в нескольких местах столицы. Им стоять
на постах уединённых — мишенями для гаек и камней, когда войска усмехаются
сторонне, а толпа видит, что власти нет.

…Балк объявил им: распоряжением министра внутренних дел тяжело раненные вчера
два чина полиции получат по 500 рублей пособия. (А им жалованья-то в месяц было
42 рубля, многие рабочие больше них получали).

…Ежели на полицейских вот так бы близко часто смотреть вплоть — тоже ведь люди. Тоже подумать — и они на службе, и у них семьи и дети.

— А ваши бабы за хлебом стоят в хвостах?
— А где ж им брать?
— А что ж мы их не видим?
— А что ж им, нашу форму натягивать?

…И остановилась тысяча перед дюжиной. Всё ж таки первым без головы остаться…

Но кто позадей, значит догадался, поднял и кинул — сколотого острого льда кусок
— в городового! Тот схватился, кровью залитый, шибко залитый, и шашку выронил.

А как кровь пролилась — побежали через них. И кто-то по пути из снежной кучи
выдернул — лопата! Она ещё страшней, если размахнуться!

…Заворачивая через площадь, проскакал на вороном коне раненый конный
полицейский — в чёрной шинели, в чёрной шапке-драгунке с чёрным султаном, а с
лицом окровавленным. Он с трудом держался на лошади.

А донцы вослед ему, издеваясь, закричали:

— А что, фараон, получил по морде? Теперь держись за гриву, а то закопаешь редьку!

…На Гороховую стекались полицейские донесения. Были толпы по тысяче, по три
тысячи, сегодня первый день появлялись кой-где и красные флаги. Были ранены
городовые на Литейном проспекте, на Знаменской площади, на Петербургской
стороне, и некоторые тяжело, за эти два дня ранениями и ушибами пострадало 28
полицейских, но ни полиция, ни войска не произвели ни единого выстрела, никого
не ранили холодным оружием, никого не ушибли при разгонах.

(25 февраля/10 марта)

…Всё же пристав Спасской части задержал до полудня человек шестьдесят, заводя
их в замкнутый каменный двор на Невском против Гостиного Двора. Тут по Невскому
от Знаменской площади повалила большая толпа. Пристав послал в Гостиный Двор за
условленной помощью к командиру пехотного караула — и тщетно ожидал с четырьмя
полицейскими, увещевая наседающую разъярённую толпу. Воинская помощь не пришла.
Тогда он сам прорвался в Гостиный Двор и просил помощи от стоявшей там сотни
4-го Донского полка. Сотник ответил, что имеет задачу лишь охранять Гостиный
Двор. Другой казачий офицер согласился помочь, но опоздал: толпа уже смяла
полицейских, освободила арестованных, а надзирателя Тройникова повалили на
землю и били поленом по голове, пока не потерял сознания.

…По Косой линии Васильевского острова шёл городовой с двумя подручными
дворниками. Толпа рабочих решила, что он ведёт арестованных, — накинулись,
отняли шашку, ею же покрестили до крови, зубы выбили.

…На подходах к Литейному мосту с Выборгской стороны и сегодня стягивалось много
тысяч рабочих. Навстречу выехал по Нижегородской улице старик-полицмейстер
полковник Шалфеев с полусотней казаков и десятком полицейских конных
стражников. Поставив из них заслон у Симбирской улицы, Шалфеев один выехал
вперёд к толпе и уговаривал её разойтись. Толпа в ответ хлынула на него,
стащила с лошади, била лежачего кто сапогами, кто палкой, кто железным крюком
для перевода рельсовых стрелок. Раздробили переносицу, иссекли седую голову,
сломали руку.

А казаки — не тронулись на помощь. (Толпа на это и рассчитывала).

Бросились выручать конные городовые, произошла свалка. Здоровый детина
замахнулся большим ломом на вахмистра, тот сбил нападавшего рукояткой
револьвера. Из толпы бросали в конных полицейских льдом, камнями, затем стали
стрелять. Тогда ответили выстрелами и полицейские.

После первых выстрелов казаки (4-й сотни 1-го Донского полка) повернули и
уехали прочь полурысцой, оставляя полицейских и лежащего при смерти на мостовой
Шалфеева.

…В толпе увеличилось молодёжи — интеллигентной и полу-. Разрозненно, по одному,
но во многих местах, стали появляться красные флаги. И когда ораторы
поднимались, то кричали не о хлебе, а: избивать полицию! низвергнуть преступное
правительство, передавшееся на сторону немцев!

…Казак на лету вырвал красный флаг, проскакал с ним два десятка саженей,
оторвал от древка. Знаменосец побежал за казаком, упрашивал вернуть. Казак,
незаметно для начальства, сбросил — и флаг уже подхвачен и в кармане.

Из толпы стали бросать в городовых пустыми бутылками. Потом дали по городовым с
полдюжины револьверных выстрелов — одного ранили в живот, другого в голову, тех
ушибли бутылками.

…На углу Невского и Михайловской толпа остановила извозчика с ехавшим
городовым. А на коленях у него был ребёнок, подкинутый, — вёз его в
воспитательный дом. Револьвер отняли, а самого отпустили — вези.

…Тут же, в кофейной «Пекарь», дежурил полицейский надзиратель. Увидели его — и
стали бросать в кофейню бутылки, камни, разбили три оконных стекла. Добрались
внутрь до полицейского, отняли и поломали шашку. Кафе спустило железные шторы.

…К Казанскому мосту нашла новая толпа — тысяч пять, с красным флагом и песнями.
Разлилась по площади у собора. «Долой самодержавие!» — «Долой фараонов!» И —
«Долой войну!»

Простых баб почти нет в толпе, а много курсисток. Рабочие и студенты менялись
фуражками — братались. Пошловатый мастеровой повёл под ручку курсистку в шубке.
Она поглядывала смущённо счастливо.

…На углу Невского и Пушкинской несколько человек из толпы бросились на
помощника пристава со спины, ударили, отобрали шашку, браунинг — и под общие
возгласы угроз оттащили по Пушкинской, вкинули в подъезд.

…К четырём часам пополудни и позже в разных местах Невского — у Пушкинской, у
Владимирского, у Аничкова моста — толпа обезоруживала городовых и избивала их
тяжело.

…Молодой человек в студенческой фуражке вытащил из-под пальто предмет, стукнул
о свой сапог — и бросил под конных городовых, в середину. Оглушительный треск —
и лошади взорваны, седоки навзничь.

…Уже немало полицейских участков на окраинах было разгромлено и не имело связи с центром.

Пристава полковника Шелькина, 40 лет служившего в одном из выборгских участков,
рабочие — знали его хорошо — переодели в штатское, кожаную куртку, перевязали
голову платком как раненому — и увезли перепрятать, пока полицию громят.

Пристав дальнего Пороховского участка скрылся от толпы в подъезд, там купил у
швейцара лохмотья (швейцар потребовал 300 рублей) и в таком виде ночью, когда
всё успокоилось, пошёл к семье на Невский.

(26 февраля/11 марта)

…В градоначальство явился пороховский пристав, который вчера покупал себе
лохмотья у швейцара, — и доложил, что пороховский участок больше не существует.
И подсчитать убитых и раненых полицейских — некому.

(27 февраля/12 марта)

…А к вечеру подвалили молодые охтенцы назад, да кто Арсенал погромил — те и с винтовками.

И там, сям собирались: да что ж мы у себя-то фараонов не выведем? Ведь их везде покончали, к ним помощь уж никая не приспеет.

…На полицейский участок повалили сами, гурьбой, фонари разбивая. (Как зазвенит да как потухнет — лихо на сердце!)

На углу Георгиевской и Большого подвалили к участку — а те окна раскрыли — да и пальнули.

…Но никого не поранили. (Может, в воздух били).

Завалили подальше, в боковые улицы, стали ждать.

…И — побежали со всех сторон! И — прихватили городовых — не успели те ни
выстрела сделать, а уж вот мы, к стенам прилипли, окна побили им — камнями,
лёдом, и двери высаживаем, чем ни попадя.

И — внутрь толпой! А — чего толпа не сделает? Да у них-то сердце — давно в
пятках, да куда им деться? Никуда не денетесь, ваши все далёко!

Не стреляли.

Схватывали их, одного по пятеро, тут же по морде били для началу, но — лишь для
началу. А потом с руками извёрнутыми, выломанными — да вытаскивали их наружу,
где простор для боя легче. Одни кричали, ругались, другие стонали, третьи
просили.

Нет уж, у нас теперь не упросишься! Нет уж, дорвались! Много вы над нами поцарствовали, а теперь мы над вами!

— Братики!.. Ради Бога!.. Дети остаются…

Бей, кромсай их в мясо, не слушай! Ишь ты, дети! Добивай, чем схватил —
палками, прикладами, штыками, камнями, сапогами в ухо, головы в мостовую, кости
ломай, топчи их да втаптывай, да поплясывай!

Ещё от кого последнее:

— Бра-атики…

А как нас хватали — тогда не братики были? Эй, кто своих добил, дохрипел — иди нам помогай, доплясывать!

А бумаги ихние — на улицу вышвыривай!

Да почто? — поджигай да вместе со стенами!

Эх, вот когда наша жизнь начнётся — только теперь!

Не хотим боле с полицией жить — хотим жить по полной слободе!

…И Васька Каюров и Пашка Чугурин теперь имели по винтовке и по патронной ленте
через плечо. С оружием ходишь, хотя стрелять не умеешь — а совсем другая сила в
тебе, и ноги куда легче ходят.

Ещё ходили, штурмовали и подожгли два полицейских участка, городовиков уложили
несколько, остальных избили, арестовали — и в их же кутузку.

…Семёновцы просидели весь день запертыми в своих казармах за Загородным, пока
вечером не подошла восставшая толпа. Тогда — хлынули к ней. Ругань, крики,
песни. Взяли оркестр и пошли к полицейскому участку. Разбили его, убили
пристава. Подожгли.

Из толпы — увязали труп пристава в пачки бумаг и бросили в огонь.

…А кого больше всего искали бить и убивать — городовых. При беспорядочной и
неумелой стрельбе, когда пули шально отскакивают от стен, — в один голос
решали, что это городовые засели на чердаках и отстреливаются. Но нигде не
находили их. И тем больше на них ярились.

Вот на Пушкинской улице толпа людей что-то мутузит в своём центре. Потом
перестала. Наклонились посмотреть — разбежались. На снегу остался убитый
полицейский.

…Наступило такое, что каждый житель столицы, из двух с половиной миллионов,
оказался предоставлен сам себе: никем не руководим и никем не защищён.
Выпущенные уголовники и городская чернь делают что хотят.

Уголовники помнят камеры мировых судей, где их судили, — и громят их. На 2-й
Рождественской сжигали все дела мирового судьи, ворохи бумаг, а заодно грелись.

С особым озлоблением и ничего не щадя, громят квартиры приставов, всем соседям
известные. Из одной такой с третьего этажа швыряли на мостовую имущество,
мебель, выкинули и пианино. И всё затем сжигали на костре.

…А какой-то человек (позже узналось: освобождённый из тюрьмы неприятельский
агент Карл Гибсон) звал толпу громить «охранку» — и увлёк её громить
контрразведку Петроградского военного округа на Знаменской улице. Служащих
контрразведки отвели в Таврический и посадили как «охранников».

…И весь вечер и ночь Петроград ловил и убивал свою полицию. По ночному времени,
далеко не отводя, убивал на улицах, топил в прорубях Обводного канала.
Снаряжались автомобильные экспедиции за городовыми.

…А мысль массы, освобождённой от полиции, быстро зреет: почему не погромить
частные дома? В квартирах, хоть и не найди офицера, ой-ой-ой сколько добра
можно прихватить. И начали ходить по квартирам: «У вас офицеров нет? Разрешите
проверить». Все ворота и подъезды велят держать открытыми — для поисков и
обысков.

На Знаменской улице дворник не сразу отпер ворота прохожей банде — его убили за
это.

…За день были подожжены кроме Окружного суда: губернское Жандармское
управление, Главное Тюремное управление, Литовский замок, Охранное отделение,
Александро-Невская полицейская часть и много, почти все полицейские участки.
Сожгли и здание полицейского архива у Львиного мостика.

Большой пожар был на Старо-Невском. Уже в темноте, при огне, из окон как будто
прыгали с высокого этажа люди. Большая толпа стояла и глазела. Оказалось: это
чучела одетые выбрасывают, горел полицейский музей.

Говорили: пристава Александро-Невской части подхватили на штыки и бросили в огонь.
(28 февраля/13 марта)
…С утра возобновились поиски городовых. Врывались в дома, в квартиры, искали по доносам и без них. Убегающие по улицам ломились в запертые ворота. Ведут арестованных городовых, околоточных, переодевшихся в штатское, — кто в извозчичьем армяке, кто в каракулевом жилете, кто и вовсе не переодевался, а в чёрной шинели своей, с оранжевым жгутом. Кого привыкли видеть важными, строгими — идут растерянные, испуганные, с кровоподтёками, в царапинах, побитые.
Вот — старый, широкошеий, шинели надеть не дали. Баба кричит: «Нассать ему в глаза!»
Ведут с избытком радостного конвоя, человек по пять на одного, винтовку кто на ремне, кто на плечо, кто на изготовку, а ещё кто-нибудь самый ярый — впереди с обнажённой шашкой, и отводит прохожих. И мальчишки с палками.
Из толпы — враждебные крики.
…Волокли за ноги по снегу связанного городового. Кто-то подскочил и выстрелом кончил его.
…На Васильевском острове везли городового на санях, ничком привязанного, а размозжённая нога его бескостно болталась и кровянила. С двух сторон сидело по солдату, и один из них прикладом долбил городового по шее. Озверелые бабы догнали и стали у привязанного уши отрывать.
…Какие полицейские участки ещё не были сожжены вчера — те горели теперь. В костре перед участком горят стулья, горят бумаги, пламя подхватывает их вверх. Через разбитые окна выбрасывают ещё новые бумаги, а кто-то длинной палкой размешивает их в огне. Из толпы кто глазеет, кто греется, приплясывают мальчишки, хлопая на себе пустыми рукавами материных куртеек, весёлая возня.
Из домов, соседних с пожарами, невольные беженцы с пожитками кочуют в другие дома. Только у таких и беда.
…Ещё кое-где костры — около квартир полицейских приставов сжигают выброшенную утварь, мебель.
На Моховой из окна пристава грохнули на мостовую рояль, а тут доколачивали прикладами.
Оратор, стоя на ящике, просит товарищей военных не бросать в костёр патроны, они ещё понадобятся в борьбе с контрреволюцией. Но уж как начали забаву — оторваться нельзя, и все бросают. Патроны взрываются с треском и заглушают оратора.
Николаевский вокзал немного громили, и он немного загорелся. Вели двух жандармских офицеров, будто бы пойманных при поджоге, — и конвой солдат охранял их от растерзания. Над Знаменской площадью свистят пули, неизвестно откуда и куда. Кассы закрыты, а поезда отходят, можно ехать.
…Одного прохожего арестовали за то, что у него толстая рожа (городовой?). Другого — что слишком быстро шёл по улице (хочет скрыться?).
…По Театральной площади две образины тянули маленькие санки, и к ним привязанный труп городового на спине. Из встречных останавливались и со смехом спрашивали, как «фараон» был убит. А двое мальчишек лет по 14 бежали сзади и старались всадить убитому папиросу в рот.
Трупы убитых городовых сбрасывали и в помойные ямы.
…Вот солдат с ружьём на ремне, а к дулу привязаны две искусственных белых розы (вынес из чайной). Вот студент ведёт за собой сквозь густоту тротуара десяток солдат — какая-то ясная у них цель, дружно идут. Вот солдат трясёт револьвером над головой и выкрикивает угрозы. Вот юноша лет 17 несёт над головой, гордо трясёт, всем показывает — обнажённую офицерскую шашку с георгиевским темляком (отняли у георгиевского кавалера).
У одного из волынцев на штыке болтается трофей — разодранный жандармский мундир. Кричит во весь голос:
— Конец фараонам! Довольно нацарствовали!
…По Лиговке к Знаменской площади валит толпа — много солдат, чёрных штатских, мальчишек — сопровождают захваченного высокого жандарма в форме. И ещё, и ещё со всех сторон к толпе лезут, останавливают. Крики.
Позади жандарма подымается винтовка прикладом вверх и медленно тяжело опускается ему на голову. Шапка с жандарма слетает. И второй раз отмахивается та же винтовка — и опускается второй раз, по голой голове. В кровь. Жандарм оглядывается, что-то говорит и крестится. Его бьют ещё в несколько рук, он падает.
…На Сенной площади броневики разбивают магазины с продуктами. Городового привязали к двум автомобилям и разорвали.
(1/14 марта)
…Толпа подростков, а с ними двое-трое взрослых ведут по улице арестованного городового в форме, саженного роста, вместо лица кровавая маска. Мальчишки на ходу дёргают его, толкают, щиплют, плюют на него. Он, не пошатываясь, идёт.
Завели в какой-то двор и донеслось несколько выстрелов.
…В доме жил и вчера арестован помощник пристава. Но и сегодня время от времени подходят и стреляют по его окнам. А в доме — и другие квартиры.
— На то и слобода: куды хочу, туды стреляю.
…Везут по Фонтанке и так: грузовик-платформа, на ней сидят и стоят избитые чины полиции, окружённые штатскими с красными повязками на рукавах.
Из толпы кричат со злостью:
— Куда их везёте? Давите гадов на месте! Поставить в ряд, да из поганого ружья одной пулей!
…Все аптеки на Невском закрыты. А над каждой аптекой висит, как положено, двуглавый орёл.
И вот какой-то рабочий догадался или надоумили. Сыскал лесенку, приставил и бил орла молотком. На тротуар сыпались осколки.
Мимо шли два иностранца, с очень довольным видом, разговаривали по-английски. Оглянулись, засмеялись, пошли дальше.
…Нигде у ворот уже не стоят дворники, не охраняют порядка. Каждый волен делать, что хочет.
Лазаретные солдаты тоже сбегают в город, ночевать не возвращаются или поздно. Сестры их просят: хоть по телефону сообщать о себе.
На Суворовской улице жгли соломенное чучело, одетое в мундир полицейского. И бороться-то не стало с кем живым!
(2/15 марта)
…К ПОЛИЦИИ — Есть только один способ выйти из ужасного положения — это сдаться! Только таким путём городовые могут получить пощаду.
…Матросы провели арестованного городового. Девочка у подъезда, стоящая вместо швейцара, сказала:
— Ой, как я не люблю фараонов!"
