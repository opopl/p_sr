% vim: keymap=russian-jcukenwin
%%beginhead 
 
%%file 07_06_2021.fb.arestovich_alexei.1.ukraina_razvitie_obschestvo
%%parent 07_06_2021
 
%%url https://www.facebook.com/alexey.arestovich/posts/4397450873652292
 
%%author Арестович, Алексей
%%author_id arestovich_alexei
%%author_url 
 
%%tags 
%%title Почему рациональные советы умных специалистов о том, как нам реорганизовать Рабкрин, спасти мир и Украину, не взлетают?
 
%%endhead 
 
\subsection{Почему рациональные советы умных специалистов о том, как нам реорганизовать Рабкрин, спасти мир и Украину, не взлетают?}
\label{sec:07_06_2021.fb.arestovich_alexei.1.ukraina_razvitie_obschestvo}
\Purl{https://www.facebook.com/alexey.arestovich/posts/4397450873652292}
\ifcmt
 author_begin
   author_id arestovich_alexei
 author_end
\fi

- Почему рациональные советы умных специалистов о том, как нам реорганизовать Рабкрин, спасти мир и Украину, не взлетают?..

Потому, что политика иррациональна в своей основе.

Если, согласно исследованиям, более 50\% граждан Украины ментально принадлежит
к лево-патерналистской картине политики и экономики, эта картина и эта
принадлежность висит на политике и экономике мельничными жерновами, замедляет
реформы и гасит начинания.

\ifcmt
  pic https://scontent-cdg2-1.xx.fbcdn.net/v/t1.6435-9/197445677_4397449606985752_339902895115281683_n.jpg?_nc_cat=104&ccb=1-3&_nc_sid=8bfeb9&_nc_ohc=0MdCKl0TGzoAX_ca254&_nc_ht=scontent-cdg2-1.xx&oh=82a1d7f699cf458ba379769bfbad10e4&oe=60E1EEDB
\fi

Альтернатива - жёсткое навязывание повестки и политики (читай, Пиночет/Саакашвили).

И тогда коллективный народ начинает не устраивать цена.

Специалисты могут предлагать одну программу гениальнее другой, но огромное,
тяжёлое, вязкое море коллективного бессознательного, выраженное через
голосования, выборы, рейтинги, опросы (и непосредственно) не даёт и не пускает.

Отсюда - фрустрация соцсетей. 

Как же так: 
- столько умных советов, а дело идёт так медленно?!..

Соцсети не замечают, что они, со своими вздохами и охами - тот же народ, утонувший во фрактальности:
- тоска по грамотным советам в ситуации своей же тотальной неграмотности.

Не может считаться грамотной мысль, не учитывающая иррациональное. 

Повышайте грамотность, полегчает.

Гипотеза Карла Поланьи предполагает:

- выигрывает тот, кому удалось синхронизировать темп изменений с темпом
приспособления к ним. Если темп изменений низкий, вы отстаете. Если темп
изменений слишком высокий, а темп приспособления низкий, то у вас происходит
революция или иные катаклизмы.

Или репрессии - в петровско-сталинском варианте.

Да и Ли Кван Ю отнюдь не был паинькой, вместе с Ден Сяо Пином.

Так что Вы выбираете?..

Я - за постепенное изменение коллективного бессознательно, в том числе - через
изменение институциональных принципов, т.е. изменения норм и приучение народа
(т.е. всех нас) к необходимым условиям в пользу открытого общества.

Однако, может получится так, что темп изменений, котрые удастся
синхронизировать с темпом приспособленности, не будет адекватен вызовам
сегодняшнего дня, и темпу развития других. Вот и получится (и уже получается по
факту) - вроде и движемся вперед, но при этом все больше и больше отстаем.

Поэтому критически важным является необходимость определить точки роста +
критические факторы, и там темпы изменений должны быть не плановыми, а
экстраординарными.

\ii{07_06_2021.fb.arestovich_alexei.1.ukraina_razvitie_obschestvo.cmt}


