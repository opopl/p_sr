% vim: keymap=russian-jcukenwin
%%beginhead 
 
%%file 24_08_2021.fb.zyden_igor.1.strana_birthday_independence
%%parent 24_08_2021
 
%%url https://www.facebook.com/igor.zyden/posts/4897940440219386
 
%%author Зыден, Игорь
%%author_id zyden_igor
%%author_url 
 
%%tags nezalezhnist,strana,ukraina
%%title ДЕНЬ РОЖДЕНИЯ СТРАНЫ, НО НЕ ДЕНЬ НЕЗАВИСИМОСТИ
 
%%endhead 
 
\subsection{ДЕНЬ РОЖДЕНИЯ СТРАНЫ, НО НЕ ДЕНЬ НЕЗАВИСИМОСТИ}
\label{sec:24_08_2021.fb.zyden_igor.1.strana_birthday_independence}
 
\Purl{https://www.facebook.com/igor.zyden/posts/4897940440219386}
\ifcmt
 author_begin
   author_id zyden_igor
 author_end
\fi

ДЕНЬ РОЖДЕНИЯ СТРАНЫ, НО НЕ ДЕНЬ НЕЗАВИСИМОСТИ.

Подозреваю, что 95\% просто глянут фото и даже не прочтут этот текст.

Как вы уже поняли, пост будет правдивым, но не праздничным. Кто хотят – смогут
и без этого находить поводы для празднования. А правде в глаза следует глянуть.

И я не за себя парюсь – у меня, по сравнению с многими украинцами, дела обстоят
более, чем великолепно. «Мне за державу обидно» (с).

К тому же, пару месяцев назад был день Конституции, которая не выполняется.

\ifcmt
  pic https://scontent-cdt1-1.xx.fbcdn.net/v/t1.6435-9/240413268_4897925030220927_2130638435623405276_n.jpg?_nc_cat=103&ccb=1-5&_nc_sid=730e14&_nc_ohc=BcVEk0mtsAgAX-jjEvp&_nc_ht=scontent-cdt1-1.xx&oh=81a5e2cd475c8672043efb58c3312e0f&oe=614B6744
  width 0.4
\fi

Скажете, что я не прав? Давайте тогда почитаем Конституцию.

В статье 1 говорится, что Украина является суверенной, независимой,
демократической, социальной и правовой. Про суверенность и независимость я чуть
позже скажу. А на уровне отношения к граждам вы точно уверены, что государство
социальное? То есть, когда формируется бюджет страны на следующий год, то в
первую очередь планируется по максимуму финансировать здравоохранение,
образование, затем армию и правоохранительные органы, потом остальные сферы, а
уже из того, что останется, если хватит, то на содержание Президента, Верховной
Рады, Кабинета Министров? Так у нас в стране?

А по избирательной ответственности вы точно уверены, что оно правовое? Об этом я тоже скажу.

В статье 3 говорится, что человек, его жизнь, здоровье, честь, достоинство,
неприкосновенность и безопасность являются наивысшей ценностью. Это поэтому в
период пандемии средства из статьи здравоохранение перенаправляют то на
строительство дорог, то на неокупаемые так называемые патриотические фильмы? И
это из-за ценности жизни, здоровья, чести, достоинства определенные группы
людей нападают, объявляют сафари, а государство делает вид, что все хорошо,
когда сам гарант Конституции говорит, что ему не интересны люди?

В статье 14 говорится, что земля является основным национальным богатством,
находящимся под особой охраной государства. Это поэтому открывается рынок
земли, даже не спросив через референдум у граждан хотят ли они этого? И кстати,
рынок земли ведь открыт в угоду представителей других стран. То где же
независимость? Или независимость подразумевает, что т.н. «власть» Украины в
принятии антигосударственных решений полностью не зависит от украинского
народа?

В статье 21 говорится, что люди равны в своем достоинстве и правах, а в статье 24, что все равны перед законом.

Тогда почему в стране годами действует норма, как в старом сериале, когда за
кражу велосипеда могут посадить, а за кражу вагонами, в том числе из
государственного бюджета, никого не привлекли? Или почему человек, справедливо
обвиняемый по одному эпизоду в похищении и пытках человека, а по другому
эпизоду в убийстве, свободно разгуливает только потому что группа радикальный
представителей устроила шум и испортила здание офиса Президента? И почему
устранение того, что эта группа натворила, происходило за бюджетные средства, а
не из кармана тех, кто все эти бесчинства творил?

И этот перечень можно продолжать долго.

Из политической части лишь скажу, что в Конституции Украины четко указывается,
что депутат Верховной Рады на заседаниях обязан голосовать лично. И возникает
тогда вопрос. За 30 лет сколько человек и каким образом наказаны за
кнопкодавство, которое является грубейшим нарушением Конституции?

Про Конституцию пока хватит.

Думаю не стоит лишний раз приводить в пример факты, что когда распался СССР и
Украина стала отдельным государством, то она по была практически на одном
уровне с Израилем и очень сильно опережала Польшу. А на сегодняшний день, что
от Израиля, что от Польши Украина отстала крайне сильно.

Простой вопрос. Вследствие чего это случилось? Ответ такой же простой.
Вследствие «грамотного» управления государством. Тогда что же мы празднуем?

Среди разных событий, Украине довелось пережить два масштабных, названных
революциями: т.н. «оранжевая революция» и «революция достоинства» - она же
Евромайдан. И определенные группы людей считают, что этим нужно гордиться.

Хотя, давайте на чистоту. Что полезного дали Украине эти события?

В результате оранжевой революции президентом Украины стал Виктор Ющенко. При
чем стал им вместо всё же следующего президента Януковича, которого тот же
Ющенко потом назначил премьер-министром. Но что полезного получила Украина в
результате президентства Ющенко? Многие специалисты в этой сфере твердо
заявляли, что Ющенко как мог профукал все возможности.

С Евромайданом, к сожалению, все еще печальнее, потому что погибли люди.

Но какой конечный результат? Янукович сбежал, к власти пришли другие. Курс
национальной валюты упал в более чем три раза (даже было еще больше, но видимо
решили не увлекаться).

Увеличили долг Украины за счет набранных под шумок кредитов, не принося никакой пользы для государства и граждан.

Зато появился успешный повод, на который можно все спихивать – Россия и Путин.

Да, Россия совершила серьезнейшую агрессию против Украины. Сначала,
воспользовавшись суматохой и политической неразберихой, с легкостью была
произведена аннексия Крыма, в то время как западные страны и международные
организации лишь выражали глубокую обеспокоенность. Кстати, фото в посте для
меня мелкая символизация, что несмотря на нынешнюю ситуацию, для меня Крым –
это Украина.

Затем в нескольких областях при российской поддержке организовались очаги
кипения. Благо, при помощи как влиятельных украинских представителей (в т.ч.
олигархата), так и небезразличных граждан, большинство точек кипения удалось
погасить, но, к сожалению, часть Донецкой и Луганской областей потеряли
контроль Украины.

Плюс, российские пропагандистские ресурсы активно очерняли и демонизировали
Украину, что так же подлило своего масла в огонь.

Но т.н. «новая украинская власть» разве стала совершать все необходимые
действия как для противодействия России, так и для блага украинских граждан?
Или они просто решили думать как можно на этом зарабатывать, прикрываясь
российской агрессией.

Нелегальная добыча янтаря – Путин виноват.

Контрабандный вывоз карпатского леса – Путин виноват.

Коррупция в Украине – Путин виноват.

Необоснованной тарифы поднимают – Путин виноват.

И т.д.

Вот небольшой пример.

Нам стали рассказывать, что не покупаем российский газ, получаем его по реверсу
из Словакии. Так ведь все вменяемые люди знают, что это вранье, а правда звучит
так. Украинский «НАК НАФТОГАЗ» не закупает газ напрямую у российского
«ГАЗПРОМА». Словакия используется как прокладка в схеме и забирая себе газ из
транзитного, украинские чиновники оформляют это как будто газ по транзиту дошел
до Словакии, а оттуда вернулся в Украину. И всё для того, чтобы оправдать
повышение стоимости, которую заплатят граждане Украины. Это как с углем история
про « Роттердам+ ».

Другими словами, используются схемы для обогащения определенных групп людей, но
при этом в ущерб карману украинских граждан. Так это разве повод для
празднования?

Но вернемся к вопросу независимости.

Помним т.н. «пленки Деркача» и разговор предыдущего президента Украины с пока
тогдашним вице-президентом США, где президент Украины то договаривается о смене
неудобного генерального прокурора, то хвастается о перевыполнении плана
повышения тарифов для граждан Украины? Это называется независимостью? Или все
таки это действия в интересах другого государства? Тогда что мы празднуем?

Кстати, о тарифах.

Киевляне прекрасно знают, что в столице проблема с централизованным горячим
водоснабжением. Например, во многих домах минимум с мая этого года не
удосужились жильцам домов возобновить горячую воду. При чем, пользы нет ни от
обращений в КМДА, ни от обращений в Кабинет Министров. И это при регулярном
роста тарифов.

Почему же так происходит?

У подрядчика руки не из того места растут? Так власть вроде и нужна, чтобы
контролировать таких подрядчиков, требовать выполнения работы, а в случае
невыполнения обязанностей подрядчиков – не просто менять подрядчика, но и
требовать выплату неустойки.

А если дело не в подрядчике, то в чем? Денег нет, но вы держитесь?

А как же повышение тарифов? А на строительство дорог в период борьбы с вирусом?
А на ненужные флагштоки? А на парад? На это все деньги нашли, а на
водопроводные коммуникации нет? Тогда большие вопросы к разумности и
компетентности власти. Что же тогда празднуем?

Кстати, еще о деньгах и независимости.

Кто внимательно смотрели нашумевший в свое время сериал «Слуга народа»,
прекрасно помнят, как персонаж Зеленского услышав немыслимые требования от МВФ
посылает последних в жопу и говорит речь, показывающую, что ему важна Украина.

Что же делает реальный Зеленский, находясь на должности Президента Украины в
этой ситуации? Заходит в Верховную Раду и начинает рассказывать какие законы
нужно обязательно принять, аргументируя это тем, что этого требует МВФ.

Ну так где независимость?

А что скажете про зарплаты ненужных наблюдательных советов, куда входят
иностранцы и не приносят для Украины никакой пользы? Я не прав? Тогда по
пунктам перечислите какую пользу Украина получила именно благодаря
наблюдательными советам.

А когда при этом предприятия уходят в убыток, эти самые наблюдательные советы
почему-то не несут никакой ответственности. А зарплаты почему-то хотят иметь на
уровне чуть ли не канцлера Германии и президента США. И некоторые в даже
получают. Получают из государственного бюджета Украины. То есть, из наших с
вами налогов. Это тоже независимость такая?

Еще хотите про деньги? Запросто.

Казалось бы в Украине созданы независимые антикоррупционные органы. Скажете,
что они полезные? Тогда сравните сколько денег из бюджета выделено на их
содержание и посмотрите какие копейки возвращены в государственный бюджет
благодаря деятельности этих органов. Или кто-то наивно верит, что в Украине
побороли коррупцию?

Я вам скажу больше. Человек, по недоразумению и ныне занимающий должность главы
Национального антикоррупционного бюро Украины официально и с наличием решения
суда внесен в единый государственный реестр коррупционеров. А другим судом
признано незаконное назначение данного человека на должность главы НАБУ. Это
так действует правовое и независимое государство?

Есть так же фанаты, готовые кричать, что обалденным достижением государства
Украина является получение безвиза со странами Евросоюза.

Что полезного это дало именно для страны? Не для определенных категорий граждан, а именно для страны. Ровным счетом ничего.

Те, кто ездят в другие страны, могли это делать и раньше. Просто сейчас немного
экономят на этом. А те, кто не могли себе позволить такие поездки – для них
безвиз никакой позитивной роли не сыграл.

А если кто-то называет пользой безвиза поездки на заработки заграницу, то какая
польза от этого для Украины? Налоги с заработанных денег эти граждане не
платят, какие бы сказки они не рассказывали. А их хвастовство, что они лично
себе что-то купили на заработанные – это для страны пользы не несет.

Во-вторых, почему же эти люди вообще едут в другую страну на заработки, а не
зарабатывают в Украине? Работу найти не могут? В другой стране платят больше?
Тогда что же Украине праздновать, если работоспособные граждане с радостью из
нее готовы уезжать в поисках лучшей жизни?

И что забавно, часто именно от них слышим предложение делать что-то для
государства. От людей, которые вместо развития своей страны, едут в чужую ради
собственного блага.

А основной принцип любого взаимодействия между гражданами и государством такой:
граждане ведут культурный и мирный образ жизни, соблюдают законы и платят
налоги – государство обеспечивает необходимое существование в виде
распределения этих налогов на потребности граждан, а так же их защиту, но при
этом привлекает к ответственности за нарушение законов.

Так что лично я делаю все необходимое со своей стороны. И в отличии от разного
быдла веду культурный образ, и не устраиваю шумихи по ночам, и мусор не
сваливаю где-попало. К тому же, по налогам лично я плачу в объемах больше, чем
традиционные наемные работники. А значит и по поступлениям налогов от меня, я
тем более опережаю тех, кто уезжают на заработки. 

А если мы платим налоги, а в ответ возможны случаи когда полиция может не
приехать на вызов или забить на расследование дел или когда за лечение в
государственных медицинских учреждениях мы должны еще платить деньги – значит
государство не выполняет свои функции. И не выполняет скорее всего не потому
что денег не хватает, а потому что на себя любимых и свои хотелки слишком много
выделяет, что на народ не остается.

Поэтому, каждый плательщик налогов имеет полное право требовать отдачи от
государства. А если этой отдачи нет – за что тогда налоги платим? А если же
позиция государства «ты налоги плати, но в ответ ничего не требуй», то что же
празднуем? Независимость власти от собственного народа? А стоит ли это называть
тогда праздником?

Каждый решает для себя. Но на мой взгляд, день рождения у послесссровской
Украины есть, а вот с независимостью сложновато.
