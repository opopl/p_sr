% vim: keymap=russian-jcukenwin
%%beginhead 
 
%%file 24_07_2020.fb.lnr.1
%%parent 24_07_2020
 
%%endhead 
\subsection{Таможенники и пограничники ЛНР изъяли у контрабандиста антиквариат на 500 тыс. руб.}
\label{sec:24_07_2020.fb.lnr.1}
\url{https://www.facebook.com/groups/LNRGUMO/permalink/2874255139352708/}

Сотрудники Государственного таможенного комитета (ГТК) и Службы пограничной
охраны МГБ ЛНР изъяли у контрабандиста на таможенном посту "Изварино" предметы
искусства, представляющие культурную ценность, на сумму около 500 тыс. руб. Об
этом на пресс-конференции в ГТК сообщил председатель ведомства Юрий
Афанасьевский.

"На таможенном посту "Изварино" в ходе работы совместной досмотровой группы из
числа сотрудников Государственного таможенного комитета ЛНР и Службы
пограничной охраны МГБ ЛНР выявлены незадекларированные раритетные печатные
издания, альбомы с марками, картины, медали и монеты. Товары и транспортное
средство были изъяты", - сказал он.

В ходе повторного осмотра изъятого автомобиля таможенники Республики обнаружили
картины, фарфоровые и керамические изделия.

Председатель Гостаможни уточнил, что собственником указанных предметов является
гражданин Российской Федерации, который под видом переезда на постоянное место
жительства в Россию пытался незаконно вывезти культурные ценности за пределы
ЛНР. В ходе доследственной проверки установлено, что контрабандист занимается
розничной торговлей антиквариата.

"По результатам проведенных экспертных оценок установлено, что изъятые предметы
имеют культурную, историческую, художественную и материальную ценность. Общая
среднерыночная стоимость изъятых предметов составляет около полумиллиона
рублей", - отметил Афанасьевский.

Он рассказал, что в действиях правонарушителя усматриваются признаки
преступления, предусмотренного статьей 274 Уголовного кодекса ЛНР "Контрабанда
культурных ценностей".

"Подследственность данного вида преступления относится к компетенции
Министерства государственной безопасности ЛНР, в связи с чем все материалы в
ближайшее время будут переданы в МГБ ЛНР для дальнейшей проверки", - добавил
председатель ГТК Республики.

Начальник управления по надзору за исполнением законодательства в сфере
экономики Генеральной прокуратуры ЛНР Дмитрий Верещага подчеркнул, что проверка
по факту контрабанды культурных ценностей находится и на особом контроле
Генпрокуратуры.

"За совершение действий, связанных с контрабандой культурных ценностей,
предусмотрена как административная ответственность (конфискация товара,
стопроцентный штраф от стоимости изъятых товаров), так и уголовная
ответственность в виде лишения свободы от трех до семи лет", - уточнил он.

  
