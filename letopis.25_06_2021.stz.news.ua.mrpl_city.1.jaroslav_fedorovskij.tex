% vim: keymap=russian-jcukenwin
%%beginhead 
 
%%file 25_06_2021.stz.news.ua.mrpl_city.1.jaroslav_fedorovskij
%%parent 25_06_2021
 
%%url https://mrpl.city/blogs/view/yaroslav-fedorovskij-mayu-vnutrishnyu-potrebu-dopomagati-ridnomu-mistu
 
%%author_id demidko_olga.mariupol,news.ua.mrpl_city
%%date 
 
%%tags 
%%title Ярослав Федоровський: "Маю внутрішню потребу допомагати рідному місту..."
 
%%endhead 
 
\subsection{Ярослав Федоровський: \enquote{Маю внутрішню потребу допомагати рідному місту...}}
\label{sec:25_06_2021.stz.news.ua.mrpl_city.1.jaroslav_fedorovskij}
 
\Purl{https://mrpl.city/blogs/view/yaroslav-fedorovskij-mayu-vnutrishnyu-potrebu-dopomagati-ridnomu-mistu}
\ifcmt
 author_begin
   author_id demidko_olga.mariupol,news.ua.mrpl_city
 author_end
\fi

\ii{25_06_2021.stz.news.ua.mrpl_city.1.jaroslav_fedorovskij.pic.1}

Нещодавно багато активних  маріупольців познайомилися з дуже цікавим і
унікальним рухом  \emph{\textbf{\enquote{Doors Life Matters / Життя Дверей Має Значення}}} та його
керівником – \emph{\textbf{Ярославом Федоровським}}, який за досить невеликий час зміг
згуртувати чимало містян навколо проблеми збереження архітектурної спадщини
міста. Головна мета його проєкту – реставрація старовинних дверей. Для цього
маріупольці на чолі з Ярославом збирають кошти та вмовляють власників будинків
погодитися на реставрацію... Втім крім реставрації дверей у Ярослава є ще один
проєкт, присвячений наданню статусу пам'ятника архітектури будівлі колишній
Маріїнській жіночій гімназії (теперішній час – Гімназія № 1).  Насправді про
сам рух і перші позитивні результати від діяльності Ярослава Федоровського
відомо більше, ніж про нього самого. Хто цей активний і небайдужий чоловік,
який зумів зібрати навколо себе справжню команду ентузіастів, готових діяти
разом на благо рідного міста? Пропоную ближче познайомитися і відкрити для себе
ще одного непересічного маріупольця...

Народився Ярослав у небагатій родині в Маріуполі. Сім'я жила в Новоселівці, а
бабусі – на Слобідці. Мама працювала продавчинею-консультантом в магазині
\enquote{Лазур}, потім – в \enquote{Євробуд}. Батько працює в Порту. Хлопець зростав у любові,
турботі та увазі. Любов до Маріуполя виникла саме з дитинства завдяки
розповідям бабусі та прабабусі, з якими часто гуляв. Дуже багато про минуле
розповідала прабабуся. Цікаво, що хрестили її в \emph{Харлампіївському соборі}
(історично головний і перший  православний храм у місті). Розповідала і про
голод, і про евакуацію.  Всі ці розповіді підштовхнули 13-річного Ярослава до
вивчення краєзнавства та генеалогії. Незважаючи на різне походження пращурів,
він себе вважає корінним маріупольцем.  У школі подобалося вивчати істо\hyp{}рію
через її суперечливості. Завжди критично підходив до думок авторів. Любив
математику, з легкістю розв'язував задачі. Вчителю природознавства вдалося
прищепили допитливому Ярославу  любов до тварин. У 10 років хлопець зацікавився
розведенням кроликів і курей. Він хотів знайти ту галузь, яка зможе розвиватися
в промисловому місті. Переконав бабусь, що тваринництво це прибуткова справа.
Крок за кроком почав створювати власну ферму. Утворилася вона в 2001 році і
згодом почала приносити чималі прибутки. З 2004 по 2008 роки навчався в
індустріальному технікумі та отримав спеціальність електрика.

У 2008 році Ярослав вступив до ПДТУ на 3 курс на спеціальність \enquote{Інформаційні
технології}. Програмуванням маріуполець завжди захоплювався. Також в Маріуполі
навчався в комп'ютерній академії \enquote{Шаг}. Наш герой відрізняється неабиякою
працездатністю. Під час навчання в  університеті (з 2008 по 2011 роки) теж не
зміг сидіти без роботи – починає працювати в таксі. У 2011 році уклав контракт
з декількома шлюбними агентствами і возив іноземців з різних аеропортів
знайомити з міськими красунями. Згодом придбав другий   автомобіль. Після ПДТУ
почав кар'єру менеджера з продажу. У 2014 році через військові дії всі три види
прибутку – і ферма, і шлюбні агентства, і торговельна діяльність – припиняють
існувати. Саме тоді Ярослав приймає рішення переїжджати до Харкова. Там пройшов
курси з програмування. Зараз працює менеджером проєктів. Працював з різними
компаніями Києва та Вінниці.

Головним хобі для чоловіка з дитинства залишається футбол, в який він грає з 12
років. Починалося все з двору, але поступово навіть з'явилося бажання стати
професіоналом. З другом дитинства Євгеном Куриловим вирішивши будувати
футбольну кар'єру, Ярослав намагався стати частиною престижного футбольного
клубу \enquote{Новатор}. Проте тренер клубу сказав хлопцям, що вони  слабкі гравці і
краще їм сидіти на дивані. На щастя, Ярослава це не зупинило, адже він
продовжує й досі грати. Також любить риболовлю. У Харкові чоловік зрозумів, що
потрібно більше подорожувати. На його думку, вивчати культуру одне одного – це
шлях до толерантності. До кінця року Ярослав  хоче відвідати всі найбільші
міста України, в яких ще не був. Насправді міст залишилося не так багато. Крім
подорожей любить смачно і ситно поїсти.

\ii{25_06_2021.stz.news.ua.mrpl_city.1.jaroslav_fedorovskij.pic.2}

Свою громадську діяльність пов'язує з переїздом з Маріуполя, адже в інших
містах думки про малу Батьківщину не залишали Ярослава. Насправді він дуже
любить дізнаватися про міста,  в яких жив. Полюбляв гуляти і в Харкові, і у
Вінниці. Був багато разів в Одесі і у Львові. Порівнюючи розвиток туризму в
інших містах, зрозумів, що в Маріуполі існує проблема комунікації між
громадськістю та владою. У Ярослава є внутрішня потреба допомагати місту, тому
він вирішив почати з проєктів, які можуть згуртувати населення. Одночасно
населення має саме говорити за свої проблеми. Ярослав вважає, що
відповідальність повинна лежати на жителях міста. Коли людина сама захоче взяти
участь в добровільних пожертвах, зробити власний внесок у розвиток міста, тоді
ця діяльність набере обертів. Через збереження малих архітектурних форм Ярослав
хоче достукатися до населення. На думку чоловіка, якщо двері або піддашок
будуть відреставровані силами жителів міста, тоді це буде початком важливого та
великого шляху. Згодом і влада долучиться, адже побачить, що для містян цей
напрям дійсно важливий.  У Ярослава вже були зустрічі з власниками деяких
будинків. Він навіть порахував точну кількість старовинних  будинків, де
необхідно відреставрувати двері. З 21 дверей, тільки одні перебувають у
комунальній власності. Ярослав познайомився з 16 власниками, які зовсім не
проти реставрації. Єдина затримка – це процеси і гроші. До речі, грошей не
потрібно так багато, як наприклад, в Одесі. У Маріуполі реставрація одних
дверей коштуватиме 22 – 23 тисячі. 9 тисяч вже назбирали, залишилося зовсім
трохи. Кошти збирають і завдяки проданим магнітам, у створенні яких Ярослав
особисто брав участь. У багатьох містян виникло питання, а як зберегти
відреставровані двері. На що активіст спокійно відповідає: 

\begin{quote}
\em\enquote{Звісно, є Закон про
збереження малих архітектурних форм, але і самі містяни повинні охороняти свою
архітектурну спадщину і не дозволяти одне одному її знищувати!}. 
\end{quote}

\ii{25_06_2021.stz.news.ua.mrpl_city.1.jaroslav_fedorovskij.pic.3}

\emph{\textbf{Улюблена книга:}} 

\begin{quote}
\enquote{Микола II. Життя і смерть} Едварда Радзинського, \enquote{В тайзі, у Єнісея} Віктора Астаф'єва, \enquote{Доктор Живаго} Бориса Пастернака.
\end{quote}

\emph{\textbf{Улюблений фільм:}} 

\begin{quote}
\enquote{В бій ідуть тільки \enquote{старики}} (1973) \enquote{Доктор Живаго} (2006).
\end{quote}

\emph{\textbf{Порада маріупольцям:}} 

\begin{quote}
\em\enquote{Частіше виходити зі свого життєвого кокона, дивитися ширше, більше подорожувати і сильніше любити себе!}.
\end{quote}
