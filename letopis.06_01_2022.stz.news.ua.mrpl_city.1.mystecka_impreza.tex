% vim: keymap=russian-jcukenwin
%%beginhead 
 
%%file 06_01_2022.stz.news.ua.mrpl_city.1.mystecka_impreza
%%parent 06_01_2022
 
%%url https://mrpl.city/blogs/view/mistetska-impreza
 
%%author_id demidko_olga.mariupol,news.ua.mrpl_city
%%date 
 
%%tags 
%%title Мистецька імпреза
 
%%endhead 
 
\subsection{Мистецька імпреза}
\label{sec:06_01_2022.stz.news.ua.mrpl_city.1.mystecka_impreza}
 
\Purl{https://mrpl.city/blogs/view/mistetska-impreza}
\ifcmt
 author_begin
   author_id demidko_olga.mariupol,news.ua.mrpl_city
 author_end
\fi

\ii{06_01_2022.stz.news.ua.mrpl_city.1.mystecka_impreza.pic.1}

З 16 грудня і на всі новорічні свята у Художньому музеї ім. Куїнджі була
представлена виставка робіт студентів та викладачів Маріупольської філії
Національній академії образотворчого мистецтва і архітектури \enquote{Мистецька
імпреза}.

\ii{06_01_2022.stz.news.ua.mrpl_city.1.mystecka_impreza.pic.2}
\ii{06_01_2022.stz.news.ua.mrpl_city.1.mystecka_impreza.pic.3}

На виставці експонуються картини педагогів з живопису та графіки, зокрема і
заслужених художників України \emph{\textbf{Людмили Ма\hyp{}сcальської}} і \emph{\textbf{Володимира Харакоза}}. Також
представлені роботи й інших викладачів філії, серед яких і картини талановитої
педагогині та художниці \emph{\textbf{Любові Макаренко}}, молодого викладача і самобутнього
художника \emph{\textbf{Даниїла Немировського}}, художниці-постанов\hyp{}ниці Донецького академічного
обласного драматичного театру (м. Маріуполь) \emph{\textbf{Анни Апельт-Матвейчук}} та відомих
маріупольських художниць \emph{\textbf{Марини Черепченко}}, \emph{\textbf{Олени Українцевої}} та \textbf{\emph{Ольги
Пінчукової}}. Виставка також демонструє вміння студентів Маріупольської філії
НАОМА працювати у різних жанрах і техніках образотворчого мистецтва. Серед
студентських робіт особисто мене найбільше вразили картини \emph{\textbf{Заворітньої Марії}}.

\ii{06_01_2022.stz.news.ua.mrpl_city.1.mystecka_impreza.pic.4}

Директорка Маріупольської філії НАОМА \textbf{Галина Батичко} наголосила, що

\begin{quote}
\enquote{ця виставка є дуже важливою, адже вона дозволяє показати, чого саме
навчають у Національній академії образотворчого мистецтва і
архітектури, які успіхи у студентів за перший рік навчання і в якому
напрямі заклад рухатиметься далі.  Водночас художня спільнота може
висловити певні побажання та, можливо, зауваження, які ми обов'язково
врахуємо наступного року}.
\end{quote}

Галина Іванівна дуже сподівається, що така виставка стане традиційною.

\ii{06_01_2022.stz.news.ua.mrpl_city.1.mystecka_impreza.pic.5}

Всього можна побачити 81 роботу викладачів та студентів Ма\hyp{}ріупольської філії
НАОМА. Тематика не була обмеженою. Можна побачити і графічні роботи, і роботи,
що були створені під час вивчення академічного живопису, академічного рисунку,
шрифтів тощо. Загалом картини студентів були створені протягом першого
навчального року, тому ця виставка є своєрідним річним звітом. Директорка
Художнього музею ім. Куїнджі \textbf{Тетяна Булі} зазначила, що подібна
виставка у них проводиться вперше. Вона підкреслила, що така подія була
довгоочікуваною, і, що 

\begin{quote}
\em\enquote{всі представлені роботи свідчать про великий і потужний
потенціал нашої молоді. Ми бачимо, що студенти 1 (а на сьогодні вже 2) курсу
продемонстрували свої можливості працювати у різних жанрах, у різних техніках.
Вони ставлять перед собою серйозні художні задачі і намагаються їх вирішувати}.
\end{quote}

Водночас Тетяна Юріївна зауважила, що студентам треба ще попрацювати, знайти
власний стиль і вже через рік на новій виставці ми побачимо ще більш високий
рівень.

\ii{06_01_2022.stz.news.ua.mrpl_city.1.mystecka_impreza.pic.6}

Якщо в стінах Художнього музею ім. Куїнджі експонується тільки 81 робота, то в
приміщенні Маріупольської філії Національної академії образотворчого мистецтва
і архітектури представлено ще більше унікальних робіт, які були створені
протягом першого навчального семестру. До речі, найкращі студентські роботи
відбираються для подальшого зберігання у методичному фонді. Тут і роботи з
академічного живопису, академічного рисунку, графічних технік, копіювання.

\ii{06_01_2022.stz.news.ua.mrpl_city.1.mystecka_impreza.pic.7}

Директорка Маріупольської філії НАОМА Галина Батичко зазначила, що студентські
роботи можуть мати і практичне спрямування. Зокрема, міські пейзажі можна було
б використовувати і в туристичній діяльності або для ілюстрації книжок,
створення афіш, путівників чи довідників. Насправді студенти пробують себе у
різних жанрах і, судячи з результатів, у них все виходить. Щиро радію, що в
Маріуполі – Батьківщині видатного художника-пейзажиста Архипа Куїнджі –
розкривається стільки нових талантів і раджу всім на власні очі подивитися
унікальні праці як досвідчених, так і молодих художників, що представляють
різноманітність творчої манери та небайдужість до зображеного. Виставка
працюватиме до 8 січня.

\ii{06_01_2022.stz.news.ua.mrpl_city.1.mystecka_impreza.pic.8}
