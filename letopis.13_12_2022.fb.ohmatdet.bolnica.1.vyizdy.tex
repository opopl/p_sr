% vim: keymap=russian-jcukenwin
%%beginhead 
 
%%file 13_12_2022.fb.ohmatdet.bolnica.1.vyizdy
%%parent 13_12_2022
 
%%url https://www.facebook.com/ndslohmatdyt/posts/pfbid02eEoesNuD9VLcPQP2pemQZgGHdWgQUnHwmH5UCST7v54Mzb6bXCuAwhEk9q6kd5Tvl
 
%%author_id ohmatdet.bolnica
%%date 
 
%%tags 
%%title 30 виїздів на деокуповані території: як лікарі Охматдиту допомагають дітям після окупації
 
%%endhead 
 
\subsection{30 виїздів на деокуповані території: як лікарі Охматдиту допомагають дітям після окупації}
\label{sec:13_12_2022.fb.ohmatdet.bolnica.1.vyizdy}
 
\Purl{https://www.facebook.com/ndslohmatdyt/posts/pfbid02eEoesNuD9VLcPQP2pemQZgGHdWgQUnHwmH5UCST7v54Mzb6bXCuAwhEk9q6kd5Tvl}
\ifcmt
 author_begin
   author_id ohmatdet.bolnica
 author_end
\fi

⚡️30 виїздів на деокуповані території: як лікарі Охматдиту допомагають дітям після окупації⚡️

Ще 3 роки тому лікарі КДП НДСЛ «Охматдит» започаткували проєкт по виїзній формі
роботи, аби надати спеціалізовану медичну допомогу дітям, які не мають змоги
приїхати до Києва та пройти належне обстеження. Була створена бригада з
фахівців, серед яких педіатр, ендокринолог, гастроентеролог, офтальмолог,
отоларинголог, кардіолог, хірург, гінеколог, уролог, ортопед, психолог,
невролог, алерголог, гематолог, діагностична служба та лабораторія. З початком
повномасштабного вторгнення лікарі вирішили не тільки продовжити цю діяльність,
а ще більше її активізувати.⚡️

«Коли почалося повномасштабне російське вторгнення, ми зрозуміли, що тепер, як
ніколи, ця форма роботи є потрібною. Діти України мають отримувати належну
медичну допомогу, попри те, де вони мешкають. Багато медичних закладів
знищуються російськими окупантами, тож ми вирішили роботи виїзди на території,
які наші воїни звільнили від окупації», — пояснює Ольга Медведєва, завідувачка
педіатричного відділення КДП НДСЛ «Охматдит».@igg{fbicon.heart.red}

З травня й до сьогодні команда спеціалістів Охматдиту зробила три десятки
виїздів. Зазвичай лікарі приймали дітей у школах, кабінети яких підлаштовували
під свої прийоми. З Києва з собою спеціалісти привозили усе необхідне
обладнання. У середньому за один день оглядали 200 дітей. Перші візити лікарі
здійснили до сіл Синяк та Бабинці, що на Київщині, які ще місяць до цього були
окупованими. Потім були в Бучі, Гостомелі, Бородянці, на Чернігівщині та
Житомирщині. Загалом оглянуто понад 4 тисячі дітей. Лікарі надали понад 17
тисяч консультацій.🙌🏻

Медики зізнаються, що ніхто не міг уявити, що їх чекає на деокупованих
територіях: «Ці люди постійно сиділи в підвалах. Вони зазнали великих
психологічних травм. Були і ті, хто пережив насилля... Ми побачили, що стрес та
холод має неабиякий вплив на здоров’я дітей. З’явилося багато ендокринних та
урологічних проблем»,— додає Ольга Медведєва.

Під час консультацій, лікарі поліклініки Охматдиту виявили понад 15 тисяч
патологій у дітей, яких проконсультували. Найчастіше зустрічались хвороби
органів травлення, опорно-рухового апарату та уражень ендокринної системи. 2
тисячі дітей було направлено на безкоштовне дослідження та подальше лікування
до поліклініки Охматдиту.🏥

Пройти безкоштовне медичне обстеження можна у Консультативно-діагностичній
поліклініці Охматдиту за адресою: м.Київ, вул.Стрітенська, 7-9.

Записатися на прийом до лікаря можна за посиланням: \url{https://ohmatdyt.simplex-med.com/}
