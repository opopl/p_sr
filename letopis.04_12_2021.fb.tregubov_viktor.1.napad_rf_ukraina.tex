% vim: keymap=russian-jcukenwin
%%beginhead 
 
%%file 04_12_2021.fb.tregubov_viktor.1.napad_rf_ukraina
%%parent 04_12_2021
 
%%url https://www.facebook.com/victor.tregubov.5/posts/4804504626236367
 
%%author_id tregubov_viktor
%%date 
 
%%tags napadenie,rossia,ukraina
%%title Так, тепер давайте серйозно - напад РФ на Україну
 
%%endhead 
 
\subsection{Так, тепер давайте серйозно - напад РФ на Україну}
\label{sec:04_12_2021.fb.tregubov_viktor.1.napad_rf_ukraina}
 
\Purl{https://www.facebook.com/victor.tregubov.5/posts/4804504626236367}
\ifcmt
 author_begin
   author_id tregubov_viktor
 author_end
\fi

Так, тепер давайте серйозно.

У низці світових медій зв'явилися ті чи інші плани нападу РФ на Україну. З
посиланням на власні джерела.

Втім, якщо проаналізувати ці плани, стає помітно хоча б те, що вони
відрізняються. А ще те, що в них нічого принципово нового - від 2014-го вони
відрізняються хіба обговоренням можливості атаки з білоруського напрямку.

Себто, якісь "внутрішні джерела" предоставили купі медій, причому класу від
серйозних до таблоїдів, різні плани нападу.

Чому я стверджую, що це не вартує обговорення?

Ви не знаєте, чи є це реальність чи планова дезінформація. Ви не знаєте, чи є
це "власне джерело" розвідкою, а якщо є, то якої країни. А може кротом? А може
просто за гроші поставили - сюрприз, так теж буває і в західних медіа, я
гарантую вам.

- Але ж от цей план виглядає доволі реалістично! - каже мені один дописувач.

Свята правда. І той виглядає реалістично. І отой теж. І план не атакувати, а
шантажувати, виглядає досить реалістично. І план пробити суходольний коридор на
Крим. І план удару через Харків та Азов, щоб оточити групу Об'єднаних сил. І
план анексії Буджака. І план повітряного десанту в... да де завгодно. І ще
багато чого.

Тож який сенс обговорювати взаємовиключні та рівнозначно можливі варіанти?

"Краще перебдіть, ніж недобдіть".

Ні, не краще.

Якщо ви не є військовозобов'язаним, якщо ви по життю чистий мирняк, вам треба
мати план на випадок повномасштабної війни, це правда. Тільки вам його треба
мати з 2014-го року. Якщо ви його за сім років не розробили, то зараз може й не
допомогти, бо ви не дуже розумний і навряд чи зможете тим планом скористатися з
таким-то рівнем смекалки. Вибачте.

А більше вам нічого не треба. Особливо вам не треба бігати по соціальних
мережах з криками "алярм", бо зробите ви не краще, а набагато гірше. 

Істинно кажу, не так я боюсь російського нападу, як повторення 2014-го року,
коли по соцмережах бігали істеричні матрони, рвучи сиве волосся на усіх
частинях тіла та перемикаючись зі "свічечка, хлопчики" на "влада зливає
патріотів в казанах" з частотою у півгодини. Попутно засвічуючи позиції,
доводячи солдатських матерів до серцевих нападів та ховаючи справжні дані у
потоку безперервного трешу.

Сутність російських планів - створити хаос, спровокувати істерію, змусити вас
перебдіти так, щоб від вашого бдєжу очі різало. І щоб ви під кожним кущем
діверсанта бачили, поки діверсант насправді спокійно йде по вулиці.

Вірте в Збройні Сили. Зараз все залежить від них. Хочете допомогти - плітіть
сітки, жертвуйте гроші, поможіть волонтерам. А бігати з "от ще звідки вони
можуть напасти" не варто. Вони можуть звідусіль, і ви не допомагаєте.

\ii{04_12_2021.fb.tregubov_viktor.1.napad_rf_ukraina.cmt}
