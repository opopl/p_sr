% vim: keymap=russian-jcukenwin
%%beginhead 
 
%%file people.turchinov_aleksandr_politik_ukr
%%parent people
 
%%url 
 
%%author 
%%author_id 
%%author_url 
 
%%tags 
%%title 
 
%%endhead 
\section{Турчинов Александр}

%%%cit
%%%cit_head
%%%cit_pic
%%%cit_text
Именно поэтому, кстати, нас пытаются убедить в том, что сразу после майдана
власть являлась учредительной и честно исполняла волю народа.
Правовую оценку действиям парламента в 2014-м, а также прокуроров, которые
расследуют сфабрикованные дела и ставят решение КСУ под сомнение, дает мой
коллега Владимир Богатырь в своей новой статье. Почитайте. Она интересная.
Меня же сегодня занимает вопрос отсутствия у \emph{Турчинова} полномочий для
подписания законов, принятых в период с 23-го февраля по 2-е марта 2014-го.
Тогда ведь Конституция образца 96-го действовала и обязанности президента
должен был исполнять премьер-министр, а не \emph{какой-то Турчинов}
%%%cit_comment
%%%cit_title
\citTitle{Ранее, чтобы выглядеть прилично, власть меняла Конституцию / Лента соцсетей / Страна}, 
Максим Могильницкий, strana.ua, 28.06.2021
%%%endcit
