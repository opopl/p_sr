% vim: keymap=russian-jcukenwin
%%beginhead 
 
%%file 30_11_2021.fb.fb_group.story_kiev_ua.2.aksak
%%parent 30_11_2021
 
%%url https://www.facebook.com/groups/story.kiev.ua/posts/1808977445965715
 
%%author_id fb_group.story_kiev_ua,poluektov_igor
%%date 
 
%%tags gorod,istoria,kiev
%%title Ян Мартинович Аксак
 
%%endhead 
 
\subsection{Ян Мартинович Аксак}
\label{sec:30_11_2021.fb.fb_group.story_kiev_ua.2.aksak}
 
\Purl{https://www.facebook.com/groups/story.kiev.ua/posts/1808977445965715}
\ifcmt
 author_begin
   author_id fb_group.story_kiev_ua,poluektov_igor
 author_end
\fi

Уявіть сценарій голлівудського фільму:

Головний герой - успішний юрист, який багато років вирішував задачі найвищої
складності для наділених владою персон, а також і великих бізнесменів.

\ii{30_11_2021.fb.fb_group.story_kiev_ua.2.aksak.pic.1}

За рахунок неймовірних комбінацій виграє справи, які завідомо здавались
програшними. Справжній \enquote{Адвокат диявола}, якому не знайти рівних.

Особливо успішний у всіх можливих і неможливих схемах з рейдерства нерухомого
майна, цілих сіл і містечок.

Його генієм знані особи входять у вищий світ та обростають такими маєтностями,
що стають ледь не найбагатшими людьми не лише країни.

\ii{30_11_2021.fb.fb_group.story_kiev_ua.2.aksak.pic.2}

Приходить день, коли і сам дорвався до влади. А тоді, не без заздрощів дивиться
на успіхи клієнтів, яких підніс так високо. 

Вирішує вступити у відкриту битву за ласий шматок з найбільш могутнім і
успішним з них, шляхом неймовірної комбінації рейдурнути його бізнес-імперію на
свою користь...

Такий фільм точно став би рекордсменом прокату.

Але все це не чиїсь вигадки, а про реальну київську історію, про реальних
людей, які жили тут.

Я завжди казав, що українська історія цікавіша за всі фільми разом узяті)

.

Як ви вже мали здогадатись, ми знову занурюємось у буремні XV-XVII ст, коли
Київ жив неймовірно цікавими подіями, аферисти всіх гатунків легко ставали
шляхтичами, за короткий період могли багатократно збагачуватись, а іноді і
втрачати все в один день, коли хтось інший провертав ще більш цікаву і
винахідливу комбінацію.

% 3-5
\ii{30_11_2021.fb.fb_group.story_kiev_ua.2.aksak.pic.3}

І ця друга частина розповіді про того неймовірно амбітного авантюриста, який
спочатку допоміг родині Кобизевичів-Ходиків піднестись на саму вершину владного
Олімпу, могутності та багатства, а згодом і задумав прибрати до своїх рук їх
майно.

Характер торгового Києва тоді був визначальним. Але не самою тільки торгівлею
жило місто. Київські ремісники та виготовлені ними товари мали високу
репутацію, поступово ж — і саме з перших десятиліть XVII століття починаючи, —
Київ стає також культурним та освітнім центром.

% 6-7
\ii{30_11_2021.fb.fb_group.story_kiev_ua.2.aksak.pic.4}

Нічого дивного, що такі умови — і чималі шанси, які Київ надавав своїм жителям,
і той підвищений ризик, на який вони наражалися, — притягував до міста
своєрідний елемент: людей енергійних, не без авантюрної жилки, таких, що в разі
чого вміли захистити себе і свої права. 

\ii{30_11_2021.fb.fb_group.story_kiev_ua.2.aksak.pic.8_9}

Міські урядники, хоча і керувались європейським правом, але ідеалізувати їх не
варто, бо не все було так просто. Наприкінці XVI ст. більшість міських
війтівств опинилася в руках магнатів та заможної шляхти. 

\ii{30_11_2021.fb.fb_group.story_kiev_ua.2.aksak.pic.10_11}

Починаючи щонайменше від 1570 року кияни здобули привілей на безперечне право
самим обирати своїх війтів, у відповідному документі значилося, що такою була
«стародавня традиція». Отож, коли їм у 1593 році спробували накинути певного
шляхтича-землевласника Еразма Стравинського, вони повстали згуртованими рядами
і за три тижні зуміли позбутися приблуди. Війтом став київський городянин,
довголітній райця, що вже й раніше виконував війтівські обов’язки, Яцько
Балика. Як ми пам’ятаємо з першої частини, саме його намагалися скинути Ходики,
що мріяли про уряд війта. 

.

І от саме в тому Києві набув великого значення київський підвоєводій, пізніше
київський земський суддя, Ян Мартинович Аксак. 

Ранньомодерна людина не існувала «сама собою», неодмінно бувши частиною своєї
родини і роду взагалі. Скажімо, від яких предків походив? Вельми важливе
питання для тогочасного шляхтича. Вони любили бачити своїх протопластів у
найвідоміших героях античної історії. Аксаки, нітрохи не вагаючись, виводили
своє походження від Александра Македонського і Демосфена. Але вже сучасники
вважали, що такі претензії непомірні, а саме слово «аксак» звучить якось
по-татарськи. Аксаки охоче з цим згодилися і внесли до списку предків ще й
Тамерлана — Тимура Кульгавого, бо «кульгавий», начебто, по-татарськи «аксак». А
то є і ще одне подібне слово — «аксакал».

Аксаки послужили князю Ольгерду і дістали невелику земельну вислугу на Поліссі.
У 1515 році в документах згадується Федір Аксак, схоже, що він був не тільки
першим «документальним» Аксаком, але й представником першого християнського
покоління роду: йому навіть нікого було вписати в дуже престижний Пом’яник
Києво-Печерського монастиря через відсутність предків-християн. Дальша історія
роду досить типова для литовської епохи. У Федора було два сини: Григорій і
Мартин. Згідно із законами князівства, маєток поділу не підлягав. Тому
батьківський спадок дістався старшому, Григорію, а менший, Мартин, вирушив
робити кар’єру до столиці — Вільні. Там він здобув протектора в особі канцлера
Миколая Радзивілла. Тим часом Григорій Аксак помер, Мартин повернувся додому,
відсудив спадок від братової Овдотеї, а племінники, якщо вони були, невідь-де
поділися. У самого Мартина були два сини, менший, Михайло, загинув у сутичці з
татарами, а старший, Ян, — це і є наш Ян Мартинович Аксак.

.

Швидко зрозумівши, що на батьківському спадку далеко не заїдеш, Ян Аксак
подався під крило князя Костянтина-Василя Острозького. І в ході князівської
служби проявився талант Яна — геніальний, незрівнянний юридичний хист. Право
було не дуже легким до використання поєднанням Литовського Статуту, Саксона —
міського магдебурзького і звичаєвого права, однак Ян Аксак умів це все
запам’ятати, узгодити і використати. Його противники з часткою глуму заявляли,
що він найбільш злодійський судовий вирок здатний обставити такими юридичними
штучками, що жодному іншому суду не вдасться його скасувати.

Як відомо з історії Острозьких, князь Костянтин Костянтинович у майнових
суперечках більше покладався на такі старі добрі методи, як загони слуг з
вилами, косами і сокирами (якщо вже не просто військо), а також на глибоку
переконаність, що йому, представнику майже королівської родини, можна все.
Однак часи змінювалися і довелося князю Острозькому оцінити талант свого
підлеглого. З одного боку, він Аксака недолюблював і називав суцігою, з другого
— явно йому сприяв. За протекцією князя Острозького, що певний час справував
службу воєводи київського, Ян Аксак став київським підвоєводою, а пізніше —
київським земським суддею. Одружився з галицькою шляхтянкою Барбарою
Кльонською, народилося в них троє синів — Стефан, Михайло і Марко, словом, все
як у людей.

Мабуть, не без княжої допомоги Аксак у 1595 дістав від короля Сиґізмунда ІІІ
земельне надання на Київщині — село Гуляники, пізніше зване ще й Мотовилівкою.

Власне, з тією землею було не менше клопоту, ніж зиску від неї. Часи були
бурхливі, представники багатьох родів десь щезали, потім виринали, на один і
той же маєток не раз претендувало кілька осіб, причому права всіх були однаково
законними. Отак раптом з’явився потомок начебто вимерлого роду попередніх
власників Гуляників, Михайло Радзимінський, вимагаючи свого. Почався судовий
процес, що загрожував тягнутися десятиліттями...

Але тут у житті Аксака стався крутий поворот, спричинений майже випадковою
зустріччю. Ця зустріч дала початок фантастичній судовій справі, достойній
навіки бути записаною в аннали як не судочинства, то науки про всеможливі афери
й авантюри. Заодно ж інтереси Аксаків зіткнулися з інтересами іншого роду, в
дечому навіть на Аксаків схожого - Ходиками.

.

Тяганина за Басань.

Це щось таке фантасмогорійне, що з цього історичного епізоду запросто вдалося б
зробити наш варіант діккенсового «Холодного дому» і «Золотого теляти» водночас,
а ще залишилося б досить матеріалу для Бальзака і Дюма. Не те, щоб про
«басанську справу» нічого не писали, … але не вадить все впорядкувати і
нагадати. 

Отож: 16 століття було часом колонізації областей, які спустіли від татарських
набігів, у тому числі південної Київщини. Одній такій території особливо не
щастило: рід за родом, почавши на ній господарювати, вигасав. Між тим, земля
була багатою і за спокійних мирних умов не поступалася б удільним князівствам.
Хай там як, але на ній виникли і заселилися містечка Басань та Биків, а при них
— ще дев’ять сіл. Однак черговий власник, залізши в борги, в 1578 р. продав цю
маєтність київському городянину Андрію Кошколдовичу. А далі, шляхом неймовірної
рейдерської оборутки та за явною корупційною допомогою Яна Аксака, переходить
до Ходиків (про що було в попереньому пості).

Аксак оцінивши красу своєї гри на користь клієнта - Василя Ходика, трохи
подумав, а пізніше вирішив, що талант не має пропадати, а тому можна відбити це
майно вже на свою користь. І починається ця сама нова схема, згадкою про яку я
завершив попередній пост.

Він знаходить у остерського старости папери вигаслого вже більш ніж півстоліття
тому князівського роду Половців-Рожиновських. Самі ці князі начебто були
потомками половецького хана Тугорхана, перейшли на службу до Володимира
Мономаха і дістали від нього величезні володіння. Потім ці земельні права були
підтверджені також київськими князями Олельковичами. Резиденцією Половців був
тодішній Сквир (теперішня Сквира), але, оскільки їх маєтки розміщувалися на
обидвох берегах Дніпра, центром задніпровських володінь став замок Рожинів у
Остерському повіті, від назви Рожинова і походить друга частина прізвища
Половців — Рожиновські. Територіально до складу цих маєтностей входив і
Басанський ключ, який Василь Ходика виманив у Кошколдовичів. Син останнього
власника Половщизни потрапив у татарський ясир, де і слід його загинув, а ще
один син, малолітній Семен, востаннє згадувався у документах за 1536 рік.
Тридцять років потому рід остаточно був визнаний вигаслим, земля — виморочним
спадком, що підлягав переходу в королівську скарбницю і розподілу між новими
пожалуваними. 

.

Аксак знаходить такого собі Юрія Семеновича Рожновського. Співзвучність
наймення та імені останнього з Половцями — Семеном Рожиновським, народжує
геніальну ідею великої омани. 

Обидва спільники — Ян Аксак і новоявлений князь Половець-Рожиновський уклали
офіційну угоду. Внеском Аксака у спільну справу був його юридичний талант і
оборотні засоби, Рож(и)новський прислужився своїм іменем і документами
князів-Половців, а гонораром Аксака мала стати половина половецького спадку.
Заодно з перспективою величезної матеріальної вигоди від цієї авантюри, Аксак
здобув ще й можливість притиснути хвіст Василю Ходиці. Вони, правда, колись
були спільниками, але вже встигли всерйоз розсваритися. Як це сталося,
невідомо. Можливо, Аксак справді позаздрив маєткам звичайного начебто
київського міщанина. А, можливо, Василь Ходика, як це трапляється
скоробагатькам, був у приватному спілкуванні далеко не найприємнішим чоловіком.

Зійшлися у спорі за чуже майно два гідних авантюристи. 

І справді, попри всі старання, Ходика не зумів побити козирі супротивників. Як
писав Антонович: «...Аксак одночасно вів позов і проти остерського старости
Ратомського, і проти Василя Ходики, вимагаючи повернення своєму новому
клієнтові спадку князів Половців-Рожиновських; для підтримки позову він
пред’являв грамоти князів Володимира Ольгердовича, Олелька Володимировича,
заповіт князя Яцька Михайловича Половця і документи, які свідчили, що його
клієнт справді є Юрієм Семеновичем Рожиновським», себто, можливим сином
останнього з Половців, Семена. 

Воістину, ідея самозванства витала в повітрі, час був такий на початку XVII
cт.! 

Аксак виграв зліплену на колінці справу у всіх судових інстанціях. 

У Ходики були зв’язані руки: тільки-но він починав піддавати сумніву законність
паперів Рожиновського, як супротивники негайно вимагали ревізії його власних
документів і шляхетських прав. А, оскільки тут і справді справа була дуже
нечистою, довелося Ходиці влаштувати пожежу, під час якої, начебто, згоріли
його шляхетські папери, і надалі показувати копії. А ще довелося йти на поклін
до князя Костянтина Острозького — з якихось власних міркувань той підтримав
Ходику проти Аксака.

Вступати в сварку з могутнім княжим родом Аксаку не хотілося, тому хитрий
сутяга придумав ще один геніальний юридичний хід. 1604 року він, як пише
Н.Яковенко: «не чекаючи фіналу справи, добився листа від короля, за яким
придеснянські маєтки до остаточного вирішення справи умовно передавалися Юрію
(Рожиновському), а той, у свою чергу, на другий же день «продав» їх Аксакові». 

З рештою маєтності Аксак порадив Рожиновському вчинити так: краще синиця в
жмені, ніж журавель у небі, слід би продати права на ці землі якійсь впливовій
особі, що не боялась би Ходики. Кому ж краще, як не сину князя Костянтина,
Янушу Острозькому! Таким чином, одним махом вбивалися два зайці: Рожновський
діставав грошову винагороду, що влаштовувала його далеко більше, ніж землі, в
які ще треба би вкладати власні кошти, а заодно нейтралізувався найпотужніший
покровитель Ходики, князь Острозький. Не буде ж він судитися з власним сином
заради якогось татарського приходня!

Ех, якби то були кращі роки Костянтина Костянтиновича! Тікав би Ходика з Басані
і не знати, чи втримався б у Києві. Але князь вже був у літах похилих,
переймався більше релігійними суперечками, отож князь Януш лише зайняв Басань,
а на скаргу Ходики відповів зустрічним позовом. Хоча Аксак вже не мав власного
інтересу в справі, але охоче допомагав князю Янушу юридичними консультаціями. 

За порадою Аксака, противники Ходики вдарили по його найслабшому місцю:
сумнівному шляхетству (підробка шляхетських документів каралася конфіскацією
всього майна). Як пише Н.Білоус («Київ наприкінці XV — у першій половині XVII
століття. Міська влада і самоврядування»): «15 квітня 1609 року «київському
скарбному» возним генералом був вручений позов до Коронного Трибуналу в Любліні
по звинуваченню «о неслушноє уживанє титулу и прерокгатив шляхетских ку
затлуменю вольностей народу шляхетского». 

Якщо ж би навіть, всупереч усьому, Ходика не сам змайстрував шляхетські папери,
то давно втратив право на шляхетство, оскільки займався лихварством. Ходика ж
запевняв, що давно облишив не тільки лихварство, а й всяку торгівлю взагалі, а
відповідні справи передав брату Федору.

Правда, то вже була фінальна частина процесу. Що там казати, лиха личина не
пропаде, Ходика знову викрутився. 1607 року він запропонував князю Янушу
мирову. За Антоновичем: «Ходика продав князю Янушу свої права на Басанський та
Биківський маєтки, одержав як завдаток 3000 золотих, а решту суми мав одержати
при остаточному здійсненні акту на володіння; тим часом, аж до здійснення цього
акту, князь Острозький повернув йому маєтки для приведення в порядок рухомого
майна».

Князь Януш вчинив нерозсудливо: Ходика й не думав повертати маєтки. 

Тягнучи і зволікаючи, він дочекався смерті батька Януша, могутнього Костянтина
Острозького у 1608 р., а там, скориставшись тим, що Януш був зайнятий то
поділом батьківської спадщини, то політичними й державними справами — він був
сенатором і краківським каштеляном, — зовсім відмовився повертати переяславські
маєтки. Навіть і завдатку не віддав! Єдине, чим зумів ще його дістати князь
Острозький — це черговим оспорюванням ходиківського шляхетства. 

Але і воно вже останньому не зашкодило, бо відійшов від справ, всіх своїх
численних дочок добре повіддавав заміж і наділив багатим приданим. Так і помер
мирно 1616 року, лишивши основне майно сину, що також зажив поганої слави. 

Але на цьому не припинились би лихі справи ні Ходиків, ні Аксаків, ні інших
подібних.

Однак у 1648 році, зі спалахом Хмельниччини, історія розвела давніх
супротивників по різні боки барикад. Більшість киян одразу ж перейшла на бік
козаків. У верхівці міста цей процент був нижчим, а все ж істотним. Як запевняє
Н.Білоус, «аналіз списків «Реєстру Війська Запорізького» 1649 року свідчить про
покозачення ледве не половини всіх родин, що належали до тогочасної правлячої
міської еліти. У «Реєстрі» зафіксовані імена й прізвища нащадків радців та
їхніх родичів…» — а на першому місці Павло і Герасим Балики.

Натомість Андрій Ходика рішуче виступив на боці проурядової, «пропольської»
партії, на початку повстання пацифікуючи двори та господарства неблагонадійних
киян, заодно не жалуючи своїх сусідів, себто супротивників, хоч би вони були
переконаними лоялістами, змушеними через це покинути Київ. Певні шляхтичі
Михайло і Габріель Тиші-Биківські скаржилися на війта Ходику, який «скористався
їхньою відсутністю та загальним безладом у Києві й ущент спустошив їхній двір:
поруйнував і позносив будівлі, а навіть звелів вирубати сад» (Н.Білоус).

«Тим часом події відбувалися надзвичайно швидко і невдовзі стало очевидним, що
цього разу козацький рух здобуде перемогу: київське населення усе сильніше
заявляло про своє співчуття і козаки наблизились до міста. Становище війта і
його партії дуже ускладнилось. Як пізніше скаржився Андрій Ходика, деякі члени
магістрату вступили у змову проти війта, вирішивши не лише позбавити його
уряду, а й відібрати маєток і саме життя. Довгий час війт був змушений
переховуватись і, врешті, на початку 1649 року він підписав зречення з посади
та письмово визнав незаконними усі свої дії впродовж чотирьохлітнього
урядування містом». (В.Антонович).

Що ж, схоже, що за чверть століття кияни значно уцивілізувалися і навчилися
усувати війтів, не вдаючись до драстичних методів, як було з Федором Ходикою в
1625-му році!

У липні 1651 року, після поразки козаків під Берестечком, до Києва увійшли
війська Януша Радзивілла. Чи то намагаючись врятувати місто від спустошення, чи
то у тривозі за власне життя, «…війт з писарем клали князеві під ноги корогви і
ключі городові віддали, і зложили наново присягу». (М.С.Грушевський, «Історія
України-Руси», т.9). 

Коли литовські війська у вересні того ж 1651 року залишили  Київ, пішов із ними
також Андрій Ходика. Розумів, мабуть, що всі мости зірвані і повороту назад не
буде. Невідомо, з яким почуттям прощався він із містом, де народився і прожив
усе життя, де жило кілька поколінь його роду і такі шалені статки. Невідома
також його дальша доля, але більше Ходики в Києві не з’являлися.

Так само цікаво розпорядилась і з родом Аксаків, які також втратили свої статки
в буремні роки революції. І, цікаво, що у «Реєстрі Війська Запорізького» за
1649 рік значаться аж двоє Аксаків. За припущенням Н.Яковенко, це могли бути
потомки тих племінників Мартина Аксака, у яких він колись відібрав батьківську
спадщину. Помстились за несправедливість.

Автори: Igor Pokuektov і Tin-tina Antonina

.

Київ - місто можливостей)

Тут можна як здобути великі статки за кілька років, так і втратити все в один
день.

Так було, так є, так буде!)

На фото замальовки Києва 1651 року.
