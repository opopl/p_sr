% vim: keymap=russian-jcukenwin
%%beginhead 
 
%%file 04_03_2019.stz.news.ua.mrpl_city.1.mrpl_poetesa_oksana_stomina_zhyttja_mizh_rjadkiv
%%parent 04_03_2019
 
%%url https://mrpl.city/blogs/view/mariupolska-poetesa-oksana-stomina-zhittya-mizh-ryadkiv
 
%%author_id demidko_olga.mariupol,news.ua.mrpl_city
%%date 
 
%%tags 
%%title Маріупольська поетеса Оксана Стоміна: життя між рядків
 
%%endhead 
 
\subsection{Маріупольська поетеса Оксана Стоміна: життя між рядків}
\label{sec:04_03_2019.stz.news.ua.mrpl_city.1.mrpl_poetesa_oksana_stomina_zhyttja_mizh_rjadkiv}
 
\Purl{https://mrpl.city/blogs/view/mariupolska-poetesa-oksana-stomina-zhittya-mizh-ryadkiv}
\ifcmt
 author_begin
   author_id demidko_olga.mariupol,news.ua.mrpl_city
 author_end
\fi

\ii{04_03_2019.stz.news.ua.mrpl_city.1.mrpl_poetesa_oksana_stomina_zhyttja_mizh_rjadkiv.pic.1}

Не так багато людей, чия присутність додає легкості та комфорту. Здається, що
такі люди навіть не уявляють, як багато вони можуть зробити для тих, хто поруч.
У них безмежно позитивна енергетика, що має цілющі властивості. Моя наступна
розповідь присвячена саме такій легкій і чудовій Людині, з якою пропоную
познайомитися ближче. Це маріупольська поетеса і громадська діячка \textbf{Оксана
Стоміна}, яка живе віршами, завжди усміхнена і сповнена ідей, надихає вмінням
співпереживати та робити все щиро, по-справжньому...

\textbf{Читайте також:} \emph{Тетяна Живолуга: маріупольський ангел свята}%
\footnote{Тетяна Живолуга: маріупольський ангел свята, Ольга Демідко, mrpl.city, 25.02.2019, \par%
\url{https://mrpl.city/blogs/view/tetyana-zhivoluga-mariupolskij-angel-svyata}
} % 
\footnote{Internet Archive: \url{https://archive.org/details/25_02_2019.olga_demidko.mrpl_city.tetjana_zhivoluga_mrpl_angel_svjata}}

Поетеса любить Маріуполь з дитинства. Вона згадує, як гуляла містом з дідусем і
бабусею. Дідусь Іван найбільше любив Приморський парк, знав у ньому кожну
квіточку і кущ. Маленька Оксана думала, що цей парк – справжнє володіння
дідуся, яке він передав онучці у спадок. Також вона обожнює Азовське море, яке
її надихає. Наша героїня наголошує, що \emph{\enquote{любов до свого міста, шанобливе
ставлення до людей, позитивне ставлення до життя не є чимось особливим. Це
скоріше норма для адекватної й вихованої людини}}. Оксана намагається жити так,
щоб радіти та радувати інших. Вона любить любити й вдячна Богу та батькам за
цей щедрий подарунок – життя!

\begin{quote}
\color{blue}\enquote{Без меры глуп и в меру благороден,

Чуть-чуть романтик, где-то циник, я

Вдруг понял: в мире много чьих-то родин,

Но есть одна особая – моя!}.
\end{quote}

Любов до гармонії римованого слова і літератури прищепили Оксані батьки. Вона
виросла на чудових віршах своєї мами. Цікаво, що перша книжка, яку Оксана
видала з сестрою Юлею, стала саме збірка маминих дитячих віршів, які вони знали
напам’ять і легко відтворили для цієї справи. Батько Оксани пише легку та
іронічну прозу. Зокрема, на полицях магазинів можна знайти дуже цікаву книгу
про його першу подорож за кордон, в Ізраїль.

\ii{04_03_2019.stz.news.ua.mrpl_city.1.mrpl_poetesa_oksana_stomina_zhyttja_mizh_rjadkiv.pic.2}

За освітою Оксана – вчитель початкових класів, математики та психолог. З
дитинства вона мріяла бути вчителем, але \emph{\enquote{людина розраховує, а Бог вирішує...}}.
У результаті, займалася ще й рекламою, і страхуванням.

\textbf{Читайте також:} \emph{Елена Беркова: за внешней хрупкостью скрывается сильный человек}%
\footnote{Елена Беркова: за внешней хрупкостью скрывается сильный человек, ДК Молодежный, mrpl.city, 26.02.2019, \par%
\url{https://mrpl.city/blogs/view/elena-berkova-za-vneshnej-hrupkostyu-skryvaetsya-silnyj-chelovek}%
}

Наша героїня з дитинства любить подорожувати. Цю любов їй теж прищепили батьки,
а з часом її бажання відвідувати нові країни підтримали й чоловік з донькою. На
їхньому рахунку вже 30 країн. Оксана підкреслює, що

\begin{quote}
\em\enquote{подорожі не тільки збагачують інтелектуально. Вони роблять тебе толерантнішим,
уважнішим, вчать шанобливому ставленню до різних людей. І, як це не дивно,
посилюють любов до Батьківщини. Це дійсно велике щастя – повертатися додому!}.
\end{quote}

\ii{04_03_2019.stz.news.ua.mrpl_city.1.mrpl_poetesa_oksana_stomina_zhyttja_mizh_rjadkiv.pic.3}

\medskip
\begin{quote}
\color{blue}\enquote{Есть много разных мечт, больших и праздных,

Есть много правд, натянутей струны...

Есть много судеб самых-самых разных.

Моя – равна судьбе моей страны!}.
\end{quote}
\medskip

Оксана дуже болісно переживала військові дії в країні. Вона бачила трагедії
людей поруч і хотіла дати людям можливість розповісти, що вони переживають. Так
виникла ідея видати збірку \textbf{\enquote{По живому. Околовоенные дневники}}, яка стала
важливою і для авторів (всі автори збірки – мешканці прифронтової або
окупованої території), і для тих, хто вміє читати. Не просто знає букви, а
готовий відчувати прочитане. \emph{\enquote{Це історія, яку потрібно зберігати}}, - наголошує
поетеса. І водночас це її особистий маленький, посильний внесок у перемогу
України, і у перемогу Миру над Війною загалом.

\medskip
\begin{quote}
\color{blue} Кожен із нас у Всесвіті – тільки атом.

Знаю. Але ж ці очі, дві глибини...

Боже, якщо він потрібен тобі солдатом,

Дай йому шанс повернутися із війни.

Боже! Серед шаленої круговерті

Вже не важливі приводи та причини.

Знаєш, якщо він буде за крок від смерті,

Хай вона не помітить цього хлопчини!

Десь там, за бруствером мрії мовчать зозулі,

А під уламками віри бракує сонця.

Але нехай його оминають кулі!

Боже, признач йому Янгола-охоронця!

Якщо це справді для чогось потрібно, може,

Цього не покарає Твоя десниця?

Дай йому першим зробити свій постріл, Боже!

І... відпусти йому це, коли все скінчиться.	
\end{quote}
\medskip

Презентація збірки відбулася по всій Україні: від Одеси до Луцька, від
Краматорська до Львова. І усюди авторів зустрічали небайдужі та щирі слухачі.
\enquote{По живому} – це не комерційний, а соціальний проект. Тому всі партнери
допомагали на волонтерських засадах. Охочих допомогти було і є дуже багато.
Наприклад, переклад книги литовською мовою, а згодом і презентацію в Литві
зробив Союз письменників Литви абсолютно безкоштовно. Цікаво, що англійською
збірку переклав американський перекладач Райлі Костіган, який є перекладачем
книги Сергія Жадана та до якого стоїть неймовірна черга. Він переклав більшу
частину прози. Причому, як всі на волонтерських засадах, ще й подякував за
надану можливість перекладати таку сильну книгу. Але перш за все Оксана дуже
вдячна другу і співавтору проекту \textbf{Олегу Украінцеву}!

\ii{insert.read_also.demidko.viktoria_lisogor}

\ii{04_03_2019.stz.news.ua.mrpl_city.1.mrpl_poetesa_oksana_stomina_zhyttja_mizh_rjadkiv.pic.4}

Цікавою і продуктивною для Оксани була \href{https://mrpl.city/news/view/v-germanii-mariupoltsy-popali-v-istoriyu-foto}{торішня поїздка до Німеччини}.\footnote{В Германии мариупольцы попали в историю (ФОТО), mrpl.city, 25.06.2018, \par\url{https://mrpl.city/news/view/v-germanii-mariupoltsy-popali-v-istoriyu-foto}} 
На запрошення декана університету міста Зіген професора Фолкера Вулфа вона і
троє маріупольських художників - \textbf{Олена Украінцева, Ганна Торкаєнко і
Євген Сенсуаліс}, які ілю\hyp{}стрували книгу \enquote{По живому}, презентували її та
\href{https://mrpl.city/news/view/v-germanii-poyavilsya-mariupolskij-plyazh-tonnel-s-angelom-i-slovo-iz-veshhej-foto}{виставку картин і інсталяцій про життя біля війни}.\footnote{В Германии появился мариупольский пляж, тоннель с ангелом и слово из вещей (ФОТО), mrpl.city, 21.06.2018, \par\url{https://mrpl.city/news/view/v-germanii-poyavilsya-mariupolskij-plyazh-tonnel-s-angelom-i-slovo-iz-veshhej-foto}} Під час поїздки
маріупольська поетеса познайомилася з професором Девидом Ренделом з Британії та
перекладачкою Ритою Гринко з Зігена, в партнерстві з якими вже видала нову
книгу. Ця збірка – це не перекладне видання, а абсолютно нова книга, яка
відразу ж вийшла у світ німецькою мовою. У перекладі її назва звучить так:
\textbf{\enquote{Війна приходить без запрошення. Українські щоденники}}. Центральне
місце в ній посідають світлини, а до них – публіцистичні статті, есе і
щоденникові нариси. Шість авторів, три розділи – Крим, Донбас, Маріуполь.
Розділ про Маріуполь найбільший і містить, крім інших, світлини чудового
маріупольського фотографа \textbf{Євгена Сосновського}.

Наступним проектом Оксани й команди ГО \enquote{Паперові сходи} стане дитячий проект,
розрахований на участь дітей прифронтових територій, де зараз найбільш гостро
відчувається підвищена тривожність і нервова напруга серед дорослих і дітей.
Оксана з Олегом Украінцевим вирішили створити унікальну книгу, написану
підлітками, але адресовану дорослим. Вона так і називається \textbf{\enquote{Лист дорослому}}.
Запропонували дітям написати листа на будь-яку тему з питаннями, побажаннями,
зауваженнями, болем і радістю. Це буде свого роду посібник по вихованню дітей,
написаний ними самими. Поради дорослим, так би мовити, із перших рук. Книжка
буде невимовно зворушливою, обіцяє авторка.

\ii{04_03_2019.stz.news.ua.mrpl_city.1.mrpl_poetesa_oksana_stomina_zhyttja_mizh_rjadkiv.pic.5}

Маріуполь присутній у кожному вірші Оксани: у рядках чи між рядків. Десь
згадується назва вулиці, десь розповідається про когось із симпатичних їй
маріупольців. До речі, один з віршів – \enquote{Останній прапор} – про події 2014 року
в Маріуполі завдяки відомому київському автору і виконавцю В'ячеславові
Купрієнку став піснею. А Валентина Півень зробила на пісню чудовий кліп. Багато
маріупольців побачать в ньому себе.

\ii{insert.read_also.demidko.sosnovskij}

Оксана – дуже різнобічна людина. Вона катається на лижах, грає у великий теніс,
любить велосипед, походи в гори. Також поетеса - унікальний колекціонер.
Наприклад, у неї можна побачити національні головні убори. Її гості дуже
полюбляють приміряти предмети колекції. Збирає також слонів. І найцікавіше –
зізнання в коханні, написані закоханими на асфальті. Вона їх фотографує і
зберігає. Хобі Оксани підтримують і друзі – надсилають світлини з різних міст.
А головні убори привозять звідусіль.

\begin{quote}
\color{blue}\enquote{Ми – немов дві половини одного міста.

Без тебе для мене це місто не має змісту,

Це місто не має сенсу і глибини.

Без мене руйнуються в ньому дахи і стіни,

Без мене воно перетвориться на руїни,

Немов після землетрусу або війни.

Проте, якщо ми удвох, все іде як треба:

Всі сходи за першим бажанням ведуть у небо,

Всі двері – у щастя, всі хвіртки – у ті сади,

Де квітне кохання, й співають про нього птиці,

Де на діамант перетворюються дрібниці,

Де мешкає ніжність і всюди її сліди}.	
\end{quote}

\ii{04_03_2019.stz.news.ua.mrpl_city.1.mrpl_poetesa_oksana_stomina_zhyttja_mizh_rjadkiv.pic.6}

Дорогі маріупольці, читайте вірші Оксани Стоміної, вони є в різних збірках. Є
вірші й у відкритому доступі в інтернеті. Відкрийте для себе унікальну авторку
з рідного міста і можливо знайдете щось і про себе!

\ii{insert.read_also.demidko.chernov}

\textbf{Улюблені книги Оксани Стоміної:} \enquote{Над прірвою в житі} та інші речі Дж. Д.
Селінджера, твори О. Генрі, Г. Горіна, \enquote{Важкий пісок} та інші твори А.
Рибакова, \enquote{Бабин Яр} А. Кузнєцова, книги Дж. Орвелла, звичайно, короткі твори
Януша Вишневського та вірші М. Цвєтаєвої, І. Бродського, В. Маяковського, Б.
Пастернака, В. Полозкової, С. Жадана та багато інших.

\textbf{Курйозний випадок з життя:} \enquote{У мене їх було чимало, але перше, що згадалося, –
це випадок з дитинства. Справа була класі в третьому, напевно. Я займалася
танцями, і одного разу під час виступу просто на сцені у мене відірвався ґудзик
на спідниці. Спідниця почала падати, але її вчасно підхопив мій партнер.
Довелося по ходу змінити танець. Так ми до кінця і дотанцювали: він тримав мене
за талію, а заразом тримав мою спідницю. До речі, ця історія – чудова
ілюстрація до сьогоднішньої розмови. До слова про надійних попутників. Чого вам
усім і бажаю, дорогі мої маріупольці!}.

\textbf{Улюблені фільми:} \enquote{Життя прекрасне} Роберто Беніньї! \enquote{Голова у хмарах},
\enquote{Піаніст}, \enquote{Форест Гамп}, \enquote{Правила виноробів}, \enquote{Термінал}, \enquote{Амелі}, \enquote{Той самий
Мюнхгаузен}, \enquote{Собаче серце} тощо.

\textbf{Порада маріупольцям:} \enquote{Не відмовляти собі в задоволенні радувати інших.
Деякі люди вважають такі потуги марними, мовляв, одній звичайній людині не під
силу змінити світ. А ви не думайте про результат, отримуйте задоволення від
процесу. Знайдіть надійних супутників, однодумців, і насолоджуйтеся дорогою.
Коли робиш щось хороше, все чарівним чином складається саме собою...}.

\textbf{Читайте також:} \emph{Антон Тельбизов: человек за занавесом}%
\footnote{Антон Тельбизов: человек за занавесом, ДК Молодежный, mrpl.city, 20.02.2019, \par%
\url{https://mrpl.city/blogs/view/anton-telbizov-chelovek-za-zanavesom}
}
