% vim: keymap=russian-jcukenwin
%%beginhead 
 
%%file 04_11_2021.fb.fb_group.story_kiev_ua.1.semja_kiev_okupacia.cmt
%%parent 04_11_2021.fb.fb_group.story_kiev_ua.1.semja_kiev_okupacia
 
%%url 
 
%%author_id 
%%date 
 
%%tags 
%%title 
 
%%endhead 
\subsubsection{Коментарі}

\begin{itemize} % {
\iusr{Олег Коваль}

Повторяйте свои публикации без предисловий. Главное, не нарушать правила нашего
сообщества: "Оскорбления, ненормативная лексика, ссылки в постах и комментариях
на сторонние ресурсы категорически запрещены."

\iusr{Ольга Кирьянцева}

Спасибо большое Вам за рассказ. Прочитала, что называется, на одном дыхании,
как и недавно перечитала так же "Бабий Яр" Кузнецова. Обязательно надо знать ,
что пережили люди в те страшные годы. И особенно ценно - это рассказы, если не
самих очевидцев, то их близких родственников. Спасибо, пишите ещё.

\iusr{Валентина Ефимова}
Спасибо за рассказ.

\iusr{Анна Сидоренко}
Очень интересно, не смотря на большой текст читается легко и быстро, спасибо.

\iusr{Natasha Fadeeva}
Спасибо.

\iusr{Тома Храповицкая}

Читала и представляла, но... не нахожу слов... Никакими словами и никаким
значком, вроде "нравится", невозможно ощутить то состояние, состояние Вашей
Семьи... Спасибо Вам Yuriy, спасибо...

\iusr{Татьяна Ткаченко}

Спасибо за рассказ! Все очень интересно и правдиво, судя по воспоминаниям моих
близких, которые тоже пережили оккупацию в Киеве и которых уже нет. Пишите.
Ждем продолжения историй.

\iusr{Людмила Навроцкая}

Спасибо за рассказ. Без слез не могла читать, так как я родилась когда немцы
были уже в Киеве. Жили мы на Шулявке напротив политехнического института. Мама
была одна с моей сестрой и мной. Папа был на фронте. Что ей пришлось пережить-
трудно даже представить. Я же помню только нищие и голодные послевоенные годы.


\iusr{Лариса Кальнооченко}
Спасибо, очень интересно. Пишите и далее.

\iusr{Таня Гур}
Спасибо ,очень интересно! А мои бабушка и тётя погибли в бабьем яру.

\iusr{Наталия Педос}
Очень интересно и трогательно. Спасибо!

\iusr{Kaurkovska Veronika}

Потрясающее! Будто прожила кусок чужий жизни! Продолжайте, это интересно!
Истории в оккупации и я от бабушки слышала. Но подробности описывать так, как
Вы это делаете, не каждому дано, Yuriy Bubnov.

\iusr{Лена Олейник}
Спасибо! Интересные факты и фото!

\iusr{Лариса Захарченко}
Пишите ещё, я плачу, не приведи Господь повторения

\iusr{Vsevolod Tatarenko}

Спасибо. Я с 59-го года жил с родителями на Тургеневской. Поэтому очень хорошо представлял, как и где все это происходило.

Между прочим, уже в шестидесятых мы с пацанами ещё находили в Павловском садике
немецкие солдатские бачки для пищи, похожие на термосы.

\iusr{Владимир Куриленко}

Спасибо! Очень познавательно и интересно. В связи с описанием оккупационного
режима, вызывающего какой-то сюрреалистический ужас, вспоминаются "Бабий яр"
Некрасова (?) и некий балетмейстер - француз Киевского оперного театра (о нём
много писали в связи с вековыми культурными связями французского и украинского
народов) поспешил отправить Гитлеру благодарственную телеграмму за освобождение
Киева от большевицкой тирании. Вот так! Кому оккупация, голод, холод, риск
потерять родных и собственную жизнь, а кому-то долгожданная свобода от красного
террора! P?S? P/S/Этого француза (фамилия вылетела из головы) французские
власти за эту телеграмму привлекли к ответу за коллаборационизм.

\iusr{Ольга Кроншевська}

Мою бабусю чоловік, що пішов на фронт запевнив, ЩО КИЇВ НЕ ЗДАДУТЬ!!! І вона
лишилася з мамою ( мамі було 16 років). Маму таки вивезли примусово до
Німеччини вже перед визволенням Києва, незважаючи на хворе серце.

Бабуся якось знаходила і носила їжу нашим полоненим. І одного разу її навіть
провела ніч в гестапо( вже вирішила навіть повіситися). Але черговий німецький
офіцер пожалів і відпустив додому.

Тож причини залишитися були різні. І поведінка під окупацією різна. Я проти
категоричних висновків!

\begin{itemize} % {
\iusr{Natalia Nikolayenko}
\textbf{Ольга Кроншевська} , 

так багато людей просто не мали можливості евакуюватися. Вони були кинуті
напризволяще. Я читала книжку» Кремлевские жены», і там были спогади, на
хвилиночку, жінки Хрущова. Коли їй стало відомо, що місцевість, де жили іі
батьки, буде ось—ось окупована , приїхала їх забирати. Звісно, у неї була
можливість приїхати машиною. Пильні сусіди побачили і повідомили, куди «
належало». Так прийшли заарештовувати за « розповсюдження панічних слухів і
настроїв» . Правда, побачили, хто « розповсюджує»,і пішли собі. Але не у всіх
чоловік був першим секретарем ЦК КП(б) України. У кого підприємство евакуювали,
той зміг виїхати. Моя бабуся розповідала, що треба було зібратися за три
години. Грудна дитина, валіза, і в дорогу. Це в Харкові.

\end{itemize} % }

\iusr{Лариса Шакалова}
Очень интересно!
А всем диванным экспертам, которые знают кто был или на данный момент есть предателями, остаётся пожелать мозгов.

\iusr{אירינה קומרובסקי}
Читала и плакала. Мои пережили звакуацию, голод...

\iusr{Анна Шустерман}
Спасибо, Юрий! Читаю Ваши воспоминания уже не впервые и каждый раз со слезами, представляю и не могу представить, страшно...

\iusr{Алла Рублюк}
Спасибо

\iusr{Ольга Солдатова}

Спасибо за рассказ... родилась после войны и выросла на рассказах мамы, которая
не успела эвакуироваться и с двумя маленькими детьми пережила в Киеве на
Лукьяновке оккупацию.... отец был в ополчении, рыли окопы и никто не сообщил, что
Киев оставили... немцы всех взяли в плен и, если бы мама не забрала его с
плена, он бы умер там с голода (ей люди передали записку, где он
находится).... перед освобождением Киева немцы на стадионе Старт собрали жителей
для отправки в Германию, вокруг стадиона ходил конвой, папа с мамой и с двумя
сестрами, с вещами на детской коляске были тоже на стадионе, вечером, когда конвой
прошел мимо, они быстро скрылись в темноте и проходными дворами вернулись
домой, но свет не включали и пересидели несколько дней, пока Киев не
освободили... Сколько радости было!... Преклоняюсь перед людьми, пережившими
оккупацию... многим этого не понять...

\end{itemize} % }
