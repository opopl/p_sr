% vim: keymap=russian-jcukenwin
%%beginhead 
 
%%file 16_02_2019.stz.news.ua.mrpl_city.1.k_122_letiu_mmk_im_iljicha_rekordy_makara_mazaja
%%parent 16_02_2019
 
%%url https://mrpl.city/blogs/view/k-122-letiyu-mmk-im-ilicha-kak-gotovilis-rekordy-makara-mazaya
 
%%author_id burov_sergij.mariupol,news.ua.mrpl_city
%%date 
 
%%tags 
%%title К 122-летию ММК им. Ильича: как готовились рекорды Макара Мазая
 
%%endhead 
 
\subsection{К 122-летию ММК им. Ильича: как готовились рекорды Макара Мазая}
\label{sec:16_02_2019.stz.news.ua.mrpl_city.1.k_122_letiu_mmk_im_iljicha_rekordy_makara_mazaja}
 
\Purl{https://mrpl.city/blogs/view/k-122-letiyu-mmk-im-ilicha-kak-gotovilis-rekordy-makara-mazaya}
\ifcmt
 author_begin
   author_id burov_sergij.mariupol,news.ua.mrpl_city
 author_end
\fi

\ii{16_02_2019.stz.news.ua.mrpl_city.1.k_122_letiu_mmk_im_iljicha_rekordy_makara_mazaja.pic.1}

О легендарном сталеваре мартеновского цеха № 2 металлургического завода имени
Ильича Макаре Никитиче Мазае написано много. И в довоенное время, вскоре после
установления им рекордов по выплавке стали, как об одном из инициаторов
стахановского движения, и неизмеримо больше - после того как страна узнала о
его трагической гибели от рук гитлеровцев. Узнала из газетных публикаций, из
очерка известного московского журналиста Бориса Галина \enquote{Песня о Макаре Мазае} в
его сборнике \enquote{В Донбассе} (1946), из стихотворной повести \enquote{Макар Мазай} поэта
Семена Кирсанова (1947), между прочим, отмеченной в 1951 году Сталинской
премией. В этих произведениях немало придуманного авторами, что вполне
допустимо в художественной литературе, к сожалению, перекочевавшего в
последующие годы в публикации, претендующие на документальность.

И все же, Макар Мазай, проведя 28 октября 1936 года плавку за рекордные 6 часов
50 минут и обеспечив съем жидкой стали в 13,4 тонны с одного квадратного метра
мартеновской печи, совершил подвиг, тем самым обеспечив себе место в истории.
Все авторы, пишущие о легендарном сталеваре, утверждают, что подвиг этот
совершен благодаря предложению Макара Мазая увеличить глубину ванны
мартеновской печи за счет так называемых ложных порогов, а также поднять свод
печи. Если это так, то кто поддержал инициативу рабочего с годичным стажем
сталевара, кто материализовал его предложения?

\textbf{Читайте также:} 

\href{https://mrpl.city/news/view/v-mariupole-otprazdnovali-122-letie-metkombinata-imeni-ilicha-video-plusfoto}{%
В Мариуполе отпраздновали 122-летие меткомбината имени Ильича, Олена Онєгіна, mrpl.city, 15.02.2019}

В марте 1997 года автору этих строк довелось встретиться с ныне покойным
Сергеем Георгиевичем Фетисовым, кандидатом технических наук, бывшим в годы
проведения рекордных мазаевских плавок заместителем начальника цеха № 2 по
технологии. Он подарил ксерокопию статьи, напечатанной в сдвоенном номере 4-5
журнала \enquote{Сталь} за 1937 год. Вот ее название – \enquote{Технологическая и
организационная подготовка рекордов сталевара Макара Мазая}. Ее авторы – уже
упомянутый Сергей Георгиевич Фетисов и начальник цеха Яков Аронович Шнееров.

\ii{16_02_2019.stz.news.ua.mrpl_city.1.k_122_letiu_mmk_im_iljicha_rekordy_makara_mazaja.pic.2}

Статья эта состоит 16 страниц убористого текста с таблицами, графиками,
многочисленными ссылками на литературные источники и представляет собой отчет о
научно-технической работе, направленной на повышение производительности
конкретного мартеновского цеха. Итоги этой работы приведены в первом абзаце
статьи: \enquote{Мартеновский цех № 2 завода им. Ильича в октябре 1936 г. первым в
Союзе освоил технические мощности, установленные для него отраслевой
конференцией металлургии. Цех дал за этот месяц 31 741 т стали, при
обязательстве в 29 360 т, достигнув съема с 1 м2 пода 7,33 т, против намеченных
отраслевой конференцией 6,7 т}. Следовательно, инициатива повышения
производительности цеха шла \enquote{сверху}, а не наоборот, не от Макара Мазая, как
принято считать.

\textbf{Читайте также:} 

\href{https://archive.org/details/09_02_2019.sergij_burov.mrpl_city.k_122_letiu_mmk_im_iljicha_dmitrii_grushevskii}{%
К 122-летию ММК им. Ильича: Дмитрий Грушевский, Сергей Буров, mrpl.city, 09.02.2019}

Далее перечисляются конструктивные недостатки цеха: узость литейного пролета,
одноярусное расположение кранов, углубление регенераторов в землю,
недостаточная площадь плацев для обрубки слитков и др. Заметим, что цех был
спроектирован и построен еще в дореволюционные годы. Был сделан анализ
технологических факторов, которые в наибольшей степени могут влиять на
повышение производительности печей. В результате были проведены следующие
мероприятия:

- произведена модернизация кранового хозяйства литейного пролета, которая
выразилась в том, что грузоподъемность двух разливочных кранов была увеличена
до 150 т, третий кран заменен на новый – 150-тонный, построенный силами самого
завода;

- реорганизовано ковшевое хозяйство цеха, имевшиеся 75-тонные сталеразливочные
ковши были заменены 100-тонными, 45-тонные чугуновозные ковши были заменены
60-тонными, все это позволило разливать стотонные плавки (номинальная емкость
каждой из пяти печей составляла до реконструкции 50 т);

- параллельно литейному пролету цеха № 1 был построен отдельный склад с тремя
мостовыми кранами для обработки и хранения слитков обеих цехов, что существенно
облегчило работу в разливочном пролете;

- изготовлены большегрузные платформы для транспортировки слитков весом 60 – 80
т;

- расширены ямы под желобами печей.

Во время текущих и капитальных ремонтов были внесены такие изменения в
конструкцию печей:

- углублена ванна поднятия порогов завалочных окон с уровня 650 мм до уровня в
800 мм;

- подняты своды на всех печах на 250 мм;

- увеличены сечения вертикальных каналов с 1,8 м2 до 2,4 м2 ;

- распыление мазута, в целях увеличения количества сжигаемого топлива, было
переведено на пар давлением 10-12 атм вместо 5 – 6 атм, увеличена также
температура пара до 2 500 вместо 1 500.

\vspace{0.5cm}
\begin{minipage}{0.9\textwidth}
	
\textbf{Читайте также:} 

\href{https://archive.org/details/02_02_2019.sergij_burov.mrpl_city.k_122_letiu_mkk_im_iljicha_vladimir_bojko}{%
К 122-летию ММК им. Ильича: Владимир Бойко, Сергей Буров, mrpl.city, 02.02.2019}
\end{minipage}
\vspace{0.5cm}

Будем надеяться, что даже неспециалисту ясно, что предстояло выполнить огромный
объем конструкторских, а затем строительно-монтажных работ, где увеличение
высоты порогов и подъем свода лишь один из эпизодов намеченной реконструкции.

Все мероприятия проведены были в короткие сроки. Они были начаты в октябре 1935
года и закончены в мае 1936-го, и в июне средняя масса плавки составила уже
99,5 т. Среднесуточная производительность по цеху в календарные сутки в июне
была 840 т против 645 т в январе. Кроме указанных выше технических мероприятий,
было внедрено ряд организационных. Авторы статьи пишет: \enquote{Недостаточно, конечно,
было установить определенные положения и написать инструкции, надо было
обеспечить проведение намеченных мероприятий. Для этого, в частности, была
установлена система многоступенчатого контроля работ в цехе. Были введены
специальные книги приемки и сдачи для сталеваров, в которых каждым сталеваром,
принимающим печь, вносятся данные о ее состоянии по элементам. Аналогичные
книги для занесения данных обо всех печах введены для мастеров}.

\vspace{0.5cm}
\begin{minipage}{0.9\textwidth}
	
\textbf{Читайте также:} 

\href{https://mrpl.city/news/view/na-mmki-prodolzhayut-prazdnovat-den-rozhdeniya-foto}{%
На ММКИ продолжают праздновать день рождения, Ігор Романов, mrpl.city, 14.02.2019}
\end{minipage}
\vspace{0.5cm}

Профессионалы, конечно, должным образом оценят масштабность реконструктивных
работ в мартеновском цехе № 2, выполненных в давних уже 1935-1936 годах. Да и
на читателей, далеких от мартеновского производства, должен произвести
впечатление сам перечень выполненных в кратчайшие сроки работ, которые создали
условия для рекордов нашего прославленного земляка Макара Мазая и его
соратников.

А вот что пишут авторы статьи о секрете рекордов сталеваров мартеновского цеха
№ 2: \enquote{Из числа стахановцев нашего цеха выдвинулся по показателям своей работы
сталевар Макар Мазай, ставший инициатором всесоюзного социалистического
соревнования сталеваров. Он полностью овладел техникой работы с большими
завалками и первый достиг блестящих результатов на этом пути: большая нагрузка
на под, определенный порядок завалки, обеспечение максимальной быстроты ее,
энергичное проведение процесса кипения и сохранение во все периоды плавки
высокого температурного режима печи – это основные факторы, осуществлением
которых тов. Мазай и его товарищи добились своих рекордов}.

\textbf{Читайте также:} 

\href{https://archive.org/details/22_04_2017.sergij_burov.mrpl_city.istoria_stalevar_ivan_katrich}{%
История: сталевар Иван Катрич, Сергей Буров, mrpl.city, 22.04.2017}

В статье Я. Шнеерова и С. Фетисова перечислены фамилии нескольких сталеваров,
повторивших рекорды Макара Мазая, а в некоторых случаях и превзошедших их –
\textbf{Лозин, Шашкин, Неделько, Тюляков, Чайкин, Шкарабура, Усов, Зиновкин, Катрич}.
Быть может, кто-нибудь из читателей узнает среди них своих дедов и прадедов.
Этими людьми можно гордиться.

И еще одна цитата из этой статьи: \enquote{Резюмируя изложенное, следует сказать, что
успех мартеновского цеха № 2 явился результатом усилий всего коллектива цеха и
завода, усилий целеустремленных и планомерных. Надо отметить исключительное
внимание, которое было уделено проведенной коллективом цеха технологической и
организационной перестройке работы цеха со стороны директора завода Н. В. Радина
и технического директора П. Я. Кравцова, непосредственно руководившего
осуществлением целого ряда наших мероприятий}.

Директор завода им. Ильича Николай Викторович Радин и технический директор Петр
Яковлевич Кравцов по ложному обвинению в 1937 году были арестованы и объявлены
\enquote{врагами народа}. Н. В. Радин в 1938 году расстрелян, а П. Я. Кравцов выжил в
лагерях и был освобожден в 1943 году. Их имена были вычеркнуты из истории. Вот
почему замалчивали многие годы имена тех, кто обеспечил рекорды Макара Мазая.

\textbf{Читайте также:}

\href{https://mrpl.city/news/view/vdvoe-bolshe-investitsij-napravil-metinvest-v-aktivy-mariupolyafoto}{%
Вдвое больше инвестиций направил Метинвест в активы Мариуполя, Іоанна Вишневська, mrpl.city, 12.02.2019}
