% vim: keymap=russian-jcukenwin
%%beginhead 
 
%%file 09_12_2021.fb.fb_group.story_kiev_ua.1.pamjati_eldara_rjazanova.cmt
%%parent 09_12_2021.fb.fb_group.story_kiev_ua.1.pamjati_eldara_rjazanova
 
%%url 
 
%%author_id 
%%date 
 
%%tags 
%%title 
 
%%endhead 
\subsubsection{Коментарі}

\begin{itemize} % {
\iusr{Елена Мельникова}
Как поэтично отозвались о режиссере!

\iusr{Анатолий Золотушкин}
\textbf{Елена Мельникова} Спасибо

\iusr{Chyzhyshyn Oksana}
А де саме був цей кінотеатр Дніпро ?
Саме там, де зараз Ляльковий театр?

\begin{itemize} % {
\iusr{Анатолий Золотушкин}
\textbf{Chyzhyshyn Oksana} да

\iusr{Irena Visochan}
\textbf{Chyzhyshyn Oksana} Гуляешь по Крещатику, идёшь к стадиону,, Динамо,,(тогда) и тут слева, вверху на ,,Дніпрових кручах,, возвышается кинотеатр ,,Днепр,, позже он стал к-том ,,Стерео,,

\begin{itemize} % {
\iusr{Chyzhyshyn Oksana}
\textbf{Irena Visochan}
От дякую, а то поспорили десь в коментарях, чи був Дніпро стерео.
Я його запам, ятала якраз по стереофільмах, бо в Києві просто кінотеатрів вдосталь.
А мені доказували, що тільки Орбіта біля Бесарабки.

\iusr{Alexander Kleimenov}
\textbf{Chyzhyshyn Oksana} «Орбіта» теж була стерео до того, як стерео зробили «Дніпро»

\iusr{Chyzhyshyn Oksana}
\textbf{Alexander Kleimenov}
Знаю, була )
Мені доказували, що тільки Орбіта була стерео

\iusr{Irena Visochan}
\textbf{Chyzhyshyn Oksana} 

На Красноармейской был изначально к-т ,,Кинопанорама,, тоже с какими -то
спецэффектами, не помню точно, исправьте кто знает. Вот ,когда уже, его после
ремонта назвали Орбита, по-моему, у меня не получилось там бывать, мимо проходила
каждый день. Может, Орбита тала стерео, а,, ,,Зоря,, Дружба,, стали кинотеатрами
повторного фильма, возможно, и Днепр как-то перепрофелировали. Не уверена.


\iusr{Лариса Павловская}
\textbf{Irena Visochan} 

Кинопанорама была на Шота руставели возле кукольного (ныне синагоги Бродского).
Я там тоже в специальных очках какой-то фильм смотрела. Виноградник запомнился
с крупными гроздями винограда

\iusr{Irena Visochan}
\textbf{Лариса Павловская} 

А позже Кинопанорама стала Орбита, я не ошибаюсь? В Кинопанораму мы ходили всем
классом (культпоход) выдавали очки, зрелище было запоминающееся на всю
жизнь. Причем, там подолгу шёл один фильм, всегда была очередь (много гостей
киевских), а потом на следующий новый фильм-очередной культпоход. Класс!

\end{itemize} % }

\end{itemize} % }

\end{itemize} % }
