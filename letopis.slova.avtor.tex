% vim: keymap=russian-jcukenwin
%%beginhead 
 
%%file slova.avtor
%%parent slova
 
%%url 
 
%%author 
%%author_id 
%%author_url 
 
%%tags 
%%title 
 
%%endhead 
\chapter{Автор}

%%%cit
%%%cit_pic
%%%cit_text
Второе место - «Нич яка мисячна». Еще одна из самых лиричных и красивых песен,
которая тоже считается народной. Песня «Нич яка мисячна» прозвучала в моем
любимом и любимом многими зрителями фильме Л. Быкова «В бой идут одни старики».
И спел ее Н. Кондратюк. Однако, как и у предыдущей песни, у этой тоже есть свои
\emph{авторы}. Стихи написал М. Старецкий, а музыку изначально писал Н. Лысенко, а
впоследствии А. Волощенко и В. Овчинников.  Она полюбилась и вошла в репертуар
многих артистов. Очень трогательно ее исполняет актер и прекрасный исполнитель
Евгения Дятлов. Восторг, да и только. Послушайте:
%%%cit_title
\citTitle{5 задушевных украинских песен, которые пели наши родители, а теперь поем мы}, 
Кино Вояж И Не Только, zen.yandex.ru, 07.06.2021
%%%endcit

%%%cit
%%%cit_pic
%%%cit_text
Вот как могут совершаться чудеса! Абсолютно незнакомый мне \emph{автор} написал
обо мне, моей семье, моей жизни и судьбе - каждое слово! Спасибо Вам, и сердцем
и душой.

"Летять, нiби чайкi i днi, i ночi
в синю даль, в синю даль.
А серце менi шепоче: "Кинь печаль, кинь печаль
Ген сонце на хвилях заграло
i радicть буяе навкруг.
Наче в свiтi не стало, наче в свiтi не стало розлук".
%%%cit_comment
Валентина Чурсина
%%%cit_title
\citTitle{5 задушевных украинских песен, которые пели наши родители, а теперь поем мы}, 
Кино Вояж И Не Только, zen.yandex.ru, 07.06.2021
%%%endcit
