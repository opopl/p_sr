%%beginhead 
 
%%file 13_03_2023.fb.vitko_olena.mariupol.1.tse_stalosya_r_k_tom
%%parent 13_03_2023
 
%%url https://www.facebook.com/olena.vitko.3/posts/pfbid06KDAynsw1fNuZDKV2nKvDQKf51iwjr23g9F6UjcmLs8BLgqJgVA8LKRvfERfbMJGl
 
%%author_id vitko_olena.mariupol,voloshyn_andrij.mariupol
%%date 13_03_2023
 
%%tags text.forward,text.story,mariupol,mariupol.war
%%title Це сталося рік тому. 11 березня 2022 року. Десь о шостій ранку (Andriy Voloshyn)
 
%%endhead 

\subsection{Це сталося рік тому. 11 березня 2022 року. Десь о шостій ранку (Andriy Voloshyn)}
\label{sec:13_03_2023.fb.vitko_olena.mariupol.1.tse_stalosya_r_k_tom}
 
\Purl{https://www.facebook.com/olena.vitko.3/posts/pfbid06KDAynsw1fNuZDKV2nKvDQKf51iwjr23g9F6UjcmLs8BLgqJgVA8LKRvfERfbMJGl}
\ifcmt
 author_begin
   author_id vitko_olena.mariupol,voloshyn_andrij.mariupol
 author_end
\fi

\#Маіуполь\_хроніки\_пекла

Це сталося рік тому. 11 березня 2022 року. Десь о шостій ранку.

\ii{13_03_2023.fb.vitko_olena.mariupol.1.tse_stalosya_r_k_tom.pic.1}

Наше сховище під дев'ятиповерхівкою по проспекту Миру 127 ще спало.

Це був підвал-коридор, що проходив під усім будинком.

Уздовж стін хто на чому спали люди. Чоловіки, жінки, старі, діти.

Ми прийшли сюди десь 5-го числа одними з найостанніших. І саме тому нам
дісталося місце у самому кінці коридора. Поставили два пивних ящики, на них
поклали двері. Двері були вузькі. Сантиметрів 60 в ширину. Для трьох дорослих
людей (нас із дружиною та її 84-річної мами) - аж ніяк. За два дні я приніс з
розбитого і розграбленого АТБ ще два ящики. А у смітнику біля під'їзд знайшов
ще одні двері, зовсім вузькі, мабуть від кладовки.

Наше \enquote{ліжко} стало рази в півтора ширшим. Вночі на ньому могли впритул спати
двоє людей. Третя людина дрімала сидячи в ногах. Впродовж ночі мінялися.

По обидві сторони коридору були  двері, що вели у бокові кімнати. Як у
справжньому гуртожитку.

Ці двері були з нефарбованого алюмінію із скляним верхом.

У більшості кімнат також були люди.

Двері, біля яких була наша лежанка - були зачинені.

Під ранок 11 березня стало зимно. І щоб було тепліше, я попросив дружину вкрити
мене курткою із головою.

Скільки пройшло часу - не знаю. Може 10 хвилин, може півгодини.

Сильний різкий поштовх. І мене засипає дрібними осколками скла із дверей. Вони
падають на куртку, саме туди, де під нею обличчя.

Потім крики, дитячий плач... Все це у пітьмі.

Хтось із чоловіків вмикає світло - світлодіодні стрічки із тютюнових кіосків,
що живилися від автобусних акумуляторів.

Протилежна від нашої частина коридору була вся в диму. Це був бетонний пил.

Стало зрозуміло, що сталося щось дуже погане. Але що саме - ми не знали.

У наш бік із того боку підвала, із димової хмари стали рухатися люди. 

Через короткий час кума (саме до них ми переїхали з Лівобережжя) принесла на
собі молодшу дочку, що була поранена. (Набагато пізніше ми дізналися - у
дівчинки стався перелом стегна). Ще троє дітей прийшли самостійно, хто в чому.
Без курток, без взуття. 

Дякувати Богу, в нашому кінці сховища був другий вихід. Адже перший, через який
усі заходили у приміщення, повністю завалило.

Саме через цей другий вихід люди почали виходити на вулицю. 

Батько дітей, голова сім'ї наших друзів, залишався на тому кінці коридору. Він
допомагав вибратися людям, декількох буквально витягнув з-під завалів. 

У найбільшій боковій кімнаті, де було мабуть до сотні людей - бетонні стельові
плити просто впали на підлогу. 

Коли ми вибралися на вулицю, було вже світло. Випав сніг. Білий-білий. 

Кругом постійно греміло. 

Мы вирішили йти до району 1000 дрібниць. Щоб було безпечніше - перейшли на
другий бік проспекту Миру. Саме звідти і побачили, що у нашої дев'ятиповерхівки
просто відсутній цілий під'їзд. Там де були квартири - ми бачили небо. З даху
до підвалу. 

То була авіабомба.

Наші друзі - батько із трьома дітьми - відірвалися від нас, вони йшли швидше. А
ми - із старою людиною. 

Кума із пораненою дочкою залишилася чекати на наших військових, які мали
доправити дитину до лікарні. 

Отак йшовши із безперервною молитвою ми дісталися до підвалу новобудови у дворі
за ЦНАПом, як його називали сбушного будинку. 

То була проміжна зупинка...

\href{https://www.facebook.com/profile.php?id=100045301456680}{Andriy Voloshyn}

%\ii{13_03_2023.fb.vitko_olena.mariupol.1.tse_stalosya_r_k_tom.cmt}
