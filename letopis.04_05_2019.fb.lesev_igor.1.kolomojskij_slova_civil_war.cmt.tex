% vim: keymap=russian-jcukenwin
%%beginhead 
 
%%file 04_05_2019.fb.lesev_igor.1.kolomojskij_slova_civil_war.cmt
%%parent 04_05_2019.fb.lesev_igor.1.kolomojskij_slova_civil_war
 
%%url 
 
%%author_id 
%%date 
 
%%tags 
%%title 
 
%%endhead 
\subsubsection{Коментарі}
\label{sec:04_05_2019.fb.lesev_igor.1.kolomojskij_slova_civil_war.cmt}

\begin{itemize} % {
\iusr{Сергей Киселев}
Есть ещё религиозный аспект. Православные с одной стороны, сектанты и язычники с другой

\begin{itemize} % {
\iusr{Игорь Лесев}
Думаю, преувеличиваете. У нас с удовольствием убивают друг друга и без обременительных споров о религиозных догматах.

\iusr{Сергей Киселев}
\textbf{Игорь Лесев} однако православных капеланнов из ВСУ изгнали, и все храмы в воинских частях отдали сектантам.
\iusr{Елена Кулакевич}
\textbf{Сергей Киселев} не совсем пойму с какой стороны сектанты а с какой православные?!

\iusr{Rustam Isakov}
\textbf{Елена Кулакевич} у товаріща ПГМ

\iusr{Oksana Kudrya}
\textbf{Елена Кулакевич} , с обеих. )

\iusr{Андрей Копейка}
\textbf{Сергей Киселев} подавляющее большинство воюющих с оружием в руках совершенно не разбирается в тонкостях этих религиозных аспектов

\iusr{Елена Кулакевич}
\textbf{Rustam Isakov} у каждого своя точка зрения, не надо оскорблять. Кто-то и про меня такое говорит (даже знаю кто  @igg{fbicon.face.smiling.eyes.smiling} ). Просто интересно разобраться

\iusr{Rustam Isakov}
\textbf{Елена Кулакевич} так а що там розбиратись, для нього визнана Константинополем ПЦУ це сектанти, а самопроголошена УПЦ МП істінниє православниє. У нього на сторінці там взагалі "константінопольскій патріархат отпал от РПЦ" ))))))

\iusr{Marina Grishin}
\textbf{Сергей Киселев}
Этого не было в истоках конфликта, это, как и многое другое привнесено позже... Вначале - язык...
\iusr{Елена Кулакевич}
\textbf{Rustam Isakov} 

вы будете очень удивлены, но для меня, не признанная тремя из четырёх древних,
и 14 из 15 автокефальных Православных церквей, ПЦУ тоже является
раскольническим объединением. Но сектой это назвать нельзя, т.к. это
противоречит самому определению понятия «секта». Поэтому, и хотелось бы
уточнения, что имелось ввиду. Особенно, если принять во внимание, что в ЗСУ
служит достаточно большое количество верных Украинской Православной Церкви. И
это факт, а не пропаганда, жертвой которой стали, в равной мере, к сожалению, и
вы, и автор комментария.


\iusr{Марина Ныкытенко}
\textbf{Rustam Isakov} , константінопольскій патріархат просто отпал, от чего?, дело десятое.

\iusr{Сергей Киселев}
\textbf{Елена Кулакевич} ПЦУ и УКГЦ это скорее не раскольники, а неоязычники. Поклоняются Бандере и Путину
\end{itemize} % }

\iusr{Сергей Сергеев}

Да, ведь претензии там не территориальные, и не из-за ресурсов идет война. А
языковая и мировоззренческая.

\begin{itemize} % {
\iusr{Игорь Лесев}
да

\iusr{Игорь Писаренко}
\textbf{Сергей Сергеев} сложно представить языковую войну

\iusr{Сергей Сергеев}
\textbf{Игорь Писаренко} зачем ее представлять, если она идет.

\iusr{Ирина Слюсарева}

"В Киеве, Москве, Брюсселе и Вашингтоне об этом не хотят говорить". - О чем не
хотят говорить в Москве, что там гражданская война? Только об этом и говорят,
вот уж не надо. Все 5 лет говорят: это гражданская война.

\end{itemize} % }

\iusr{Марина Прохорова}
Насчёт войны с Россией есть простой тест. Калибры на Банковую прилетали? - Нет? Ну вот и не пи...

\begin{itemize} % {
\iusr{Василий Стоякин}
Когда прилетают калибры и томогавки - это не война, а миротворческая операция
\end{itemize} % }

\iusr{Tatyana Meylakhs}

Забавное наблюдение: Украина - это антиРоссия, а есть ещё и АнтиУкраина, но не
Россия - зазеркалье какое-то)). Отличный пост, но я не поняла одного. Что
именно не признают в России? Что конфликт межэтнический? Так вроде кричат на
каждом углу, поэтому и автономию отстаивают. Но может, я вас не поняла

\begin{itemize} % {
\iusr{Игорь Лесев}

У Москвы такая же хитросделанная позиция, как и у Киева. Говорят, "конфликт
внутренний", но тут же называют Порошенко "коллегой". Рассказывают о
"бомбардировках Донбасса", но через Медведчука продают нефтепродукты Украине...
Такие же удобные красавчики, как и в Киеве.


\iusr{Tatyana Meylakhs}
\textbf{Игорь Лесев} 

в том что ведут себя непоследовательно и даже подло по отношению к жителям
Донбасса, согласна полностью. Но боюсь, что последовательность может привести к
трагическим результатам. Но я не государственный деятель. А пост хороший!

\iusr{Игорь Лесев}
\textbf{Tatyana Meylakhs} куда уже трагичнее... а за оценку спасибо

\iusr{Tatyana Meylakhs}
\textbf{Игорь Лесев} 

к сожалению никогда не бывает так плохо, чтобы не могло стать хуже. Эта аксиома
должна помогать людям быть ответственными в своих решениях.

\end{itemize} % }

\iusr{Дана Крон}
\textbf{Игорь Лесев}.

Вы заблуждаетесь насчет этнического конфликта. Очень поверхностное рассуждение. С
двух сторон практически одинаковый процент русских и украинцев.

Вот послушайте Алексея Мозгового.

\href{https://youtu.be/hEjCho4jz4g}{%
Новости 2015. Песню "Марцинiвка" поет автор, командир бригады "Призрак" Алексей Мозговой, youtube, %
08.06.2015%
}

\begin{itemize} % {
\iusr{Игорь Лесев}

Этничность определяется не по графе в паспорте, а по внутренней
самоидентификации. Спросите у 10 жителей Донецка, кем они себя СЕЙЧАС считают -
русскими или украинцами. Хотя по переписи 2001 года, более 60\% населения
отметили себя как украинцы. Это раз. И Мозговой, если не ошибаюсь, убит. У нас
пока только Кличко умеет с мертвыми милиционерами разговаривать.

\iusr{Natalia Delgyado}
\textbf{Игорь Лесев} думаю, сейчас большинство жителей Донецка готовы признать себя даже гражданами Сомали, только чтобы закончилось вот это все..

\iusr{Дана Крон}
\textbf{Игорь Лесев} 

Зачем мне спрашивать? Я живу в Донецке и вижу. Я Вам предложила послушать как
пел \enquote{русский} Мозговой украинские песни. Ваша ирония неуместна.

Что касается этничности, то здесь существуют разные подходы. Вот, полюбопытствуйте

\url{http://cito-web.yspu.org/link1/metod/met119/tema6/Ex6.1.html}

\iusr{Сергей Сергеев}
\textbf{Дана Крон} вы путаете украинцев и украинство в его националистическом проявлении.

\iusr{Дана Крон}
\textbf{Сергей Сергеев} 

Я ничего не путаю и разделяю эти понятия. До 2004 года, пока выгодополучатели не
начали манипулировать темой мови и истории в Багдаде было все спокойно.

Это не межэтнический конфликт, как уверяет нас автор поста, а конфликт интересов кукловодов.

\iusr{Natalia Delgyado}
\textbf{Дана Крон} 

а почему, когда всплыла тема "мовы" Донецкий регион начал себя
противопоставлять всему украинскому? Ведь понятно, же, что несмотря на все
законы о "мове" здесь не Северная Корея, и за инакомыслие или инакоязычие никто
никакой ответственности не несет.

\iusr{Дана Крон}
\textbf{Natalia Delgyado} 

С чего Вы это взяли? На таком уровне разговор теряет смысл. Вы оперируете
пропагандистскими штампами. Такими же как "донецкое быдло"  @igg{fbicon.face.smiling.eyes.smiling} 

И насчёт "никто не несет ответственности" - Вы заблуждаетесь.

\iusr{Natalia Delgyado}
\textbf{Дана Крон} 

ну война-то началась почему-то в донецком и луганском регионе, а не в сумском
или харьковском, хотя эти области тоже граничат с Россией. Меня интересует
почему именно эти регионы (или кукловоды этих регионов) начали активно
сопротивляться украинизации. И здесь речь не о штампах и критике жителей.

\iusr{Василий Стоякин}

На Кубани живут русские, которые поют украинские песни. А такоего фрикативного г как в Воронеже вна нет в принципе.

\iusr{Дана Крон}
\textbf{Natalia Delgyado} 

может потому что им кто-то в 91 ом году обещал что насильственной украинизации
не будет.. может быть кто-то был против нацизма и переписывания
истории.... задумайтесь. Кому вообще не нужен был электорат Донбасса? Кто не
голосовал бы за политиков захвативших власть антиконституционным
способом? Ответьте сами на эти вопросы, пожалуйста.

\iusr{Natalia Delgyado}
\textbf{Дана Крон} 

я всю жизнь разговариваю на русском языке, и даже во времена работы в гос
органах, и в поездках в западные области с местными жителями: они со мной на
украинском, я с ними на русском. И ни разу не получила ни от кого ни единого
"нацистского" замечания по поводу "москальской мовы". Поэтому совершенно не
понимаю проблем "насильственной украинизации". Кто кого и когда насиловал
"мовой" непонятно.

\iusr{Natalia Delgyado}
\textbf{Дана Крон} и почему "насильственную украинизацию" остро воспринимает именно Донецк а не Харьков, по-прежнему неясно.

\iusr{Андрей Гриценко}
\textbf{Natalia Delgyado}, мне нравится этот классический подход: "если я не вижу суслика, то его нет". А он есть.

\iusr{Владимир Михайловский}
\textbf{Natalia Delgyado} 

Если вы искренни в своем утверждении "совершенно не понимаю проблем
"насильственной украинизации",значит любые попытки объяснить вам это
бессмысленны. Такое бывает. Например, люди с двумя дипломами и вполне себе
интеллигентные так и не могут уразуметь бином Ньютона. Лично знаком с
выпускницей физфака, которая искренне не может понять, почему самолет не падает.
Она прекрасно знает как рассчитать подъемную силу крыла и пр., но почему не
падает - не понимает. Впрочем, что-то мне подсказывает, что в вашем случае с
искренностью есть проблемы.

\iusr{Natalia Delgyado}

Да уж... так и вижу как в дома Донецка врываются бандеровцы с факелами и травят
бедолашек-донетчан: "шпрехен укр мовою, москалi клятi"...ужас ужас

\iusr{Марина Ныкытенко}
\textbf{Natalia Delgyado} , 

именно этим ваши хэрои там занимаются. А ты закончик про мову прочти, радой
проголосованный, может поймешь почему самолёт не падает, а может и нет.

\end{itemize} % }

\iusr{Марина Фокина}

Мы здесь все рассчитывали на крымский сценарий. Значит проблема в
самоидентификации. Даже украинцы считали себя больше русскими. А раз
русские, значит пора домой.

\begin{itemize} % {
\iusr{Игорь Лесев}

Знаете, бывает иногда так, что когда приглашают девушку на кино, а она хочет
смотреть только кино. Если вас куда-то не берут, может вы где-то там не очень
нужны? Возможно, стоит начать разговаривать - хотя бы пытаться - там, где вы с
каким-то условностями нужны.

\iusr{Марина Фокина}
\textbf{Игорь Лесев} 

А мы и на Украине не сильно нужны. И со стороны Украины условий нет? На Украине
никто не предлагает никакого компромисса, только и слышны зачистки, поражение в
правах, мрази. А то, что нас сегодня куда-то не берут, мы к этому относимся с
пониманием. Очередь за паспортами этому пример. За ними идут даже те, кому они
не особо нужны. Когда тебя обстреливают градами и ты не знаешь, куда бежать и
где прятаться и при этом понимаешь, что это не какие-то мифические враги, а
собственное государство, то желание вернуться нет. А если при этом в семье есть
убитые, а таких много, то как простить? Вчера Тука на ток шоу сказал, что здесь у
тех, кто имеет тарелки, всех переписывают и им блокируют укр. каналы. Это не
правда. Всё мы видим и нам не нравится то, что мы видим.

\iusr{Игорь Лесев}
\textbf{Марина Фокина} вы правы, в Украине вы тоже особо никому не нужны. И здесь особо не видно подвижек в попытках услышать оппонента.

\iusr{Арина Родионова}

Я бы хотела возвращения Донбасса.. потому что остаться в стране, населенной
практически нацюками даже думать грустно... Но могу понять нежелание Донбасса
вернуться к тем, кто методично их уничтожал, как физически, так и морально.
Лично я бы не смогла простить... Как будет у них - не знаю

\iusr{Марина Фокина}
\textbf{Арина Родионова} 

Той Украины, которая была до этих событий уже нет. Возвращаться некуда. У меня
сын только выехал с двумя грудными детьми, как к нему в квартиру прилетел
снаряд. Только поставили окна, через месяц опять вылетели от взрывной волны. И
это в центре Донецка. Всё это пережить и вернуться? А зачем тогда столько
жертв? Я бы не хотела, но нас не спрашивают. Может ещё и «реализуют минские
соглашения».))

\iusr{Ирина Слюсарева}
\textbf{Марина Фокина}, 

большая Россия очень даже хотела бы, чтобы Донбасс к нам вернулся. Многие мои
знакомые регулярно помогают Донбассу, перечисляют деньги на разную гуманитарку.
Но ведь не отпускает вас ненька. По-доброму не отпустит. Но и автономию не
даст. Так что пока - паспорта вот. В России очень радуются, что хоть какие-то
шаги начались в помощь вам.

\end{itemize} % }

\iusr{Владимир Волков}
Пост хороший и правильный.
Сейчас набегут люди "от предыдущего президента" и забросают тапками.

\begin{itemize} % {
\iusr{Игорь Лесев}
не, там уже финансовый аут, не набегают. Уже больше месяца пишу и до сих пор нет бана. Привыкаю жить по-новому

\iusr{Владимир Волков}
\textbf{Игорь Лесев} меняется концепция? Такими темпами глядишь и народ обратно на Родину потянется..
А то пока приятней/безопасней наблюдать за событиями со стороны!

\iusr{Игорь Лесев}
\textbf{Владимир Волков} ну не то, чтобы меняется. Аналогия как при Хрущеве. Теперь просто можно Солженицына читать не только в самиздате, но и официально публиковать. Вот и все изменения.

\iusr{Елена Ланская}

Всегда смеюсь, когда на политШоу с БПП или НФ между очередными лозунгами и
патетикой, вставляют.."Российский окуппант-агрессор должен уйти с Донбасса.."!
Может, поэтому и непускают украинских журналистов на нашу сторону, чтобы они не
рассказали правду?! Буквально сегодня слушала, как Диана Панченко говорила, что
с удовольствием бы журналисты НьюсВан поехали на территорию ЛДНР, пообщаться с
людьми. Но, понимают, что по возвращению, сразу под белы ручки их загребет СБУ.
Нет здесь никаких российских войск. Все.. местные пацаны! РосСоветники
посещают., это да. Так что, все это страшилки свидомитов ..

\end{itemize} % }

\iusr{Александр Таламанчук}

Сложная тема. Не все готовы её обсуждать, а если верить интересным
исследованием, то адекватно воспринимать окружающую действительность и строить
причинно-следственные связь в состоянии где то около 20процентов населения
планеты.)

\begin{itemize} % {
\iusr{Игорь Лесев}
ну жить ведь все равно нужно учиться в мире реальных вещей

\iusr{Александр Таламанчук}
Это 100 процентов.
Просто констатирую, что это важная и сложная задача. Очень хорошо, что за неё взялись но она может быть даже опасна.
\end{itemize} % }

\iusr{Дмитрий Александров}

Всё в этом мире уже когда-то было:

"Единственный пропагандистский трюк, который мог удасться нацистам и фашистам,
заключался в том, чтобы изобразить себя христианами и патриотами, спасающими
Испанию от диктатуры русских. Чтобы этому поверили, надо было изображать жизнь
в контролируемых правительством областях как непрерывную кровавую бойню
(взгляните, как пишут "Католик хералд" и "Дейли мейл" -правда, все это кажется
детски невинным по сравнению с измышлениями фашистской печати в Европе), а
кроме того, до крайности преувеличивать масштабы вмешательства русских. Из
всего нагромождения лжи, которая отличала католическую и реакционную прессу, я
коснусь лишь одного пункта -присутствия в Испании русских войск. Об этом
трубили все преданные приверженцы Франко, причем говорилось, что численность
советских частей чуть ли не полмиллиона. А на самом -деле никакой русской армии
в Испании не было. Были летчики и другие специалисты-техники, может быть,
несколько сот человек, но не было армии. Это могут подтвердить тысячи
сражавшихся в Испании иностранцев, не говоря уже о миллионах местных жителей.
Но такие свидетельства не значили ровным счетом ничего для франкистских
пропагандистов, из которых ни один не побывал на нашей стороне фронта. Зато
этим пропагандистам хватало наглости отрицать факт немецкой и итальянской
интервенции, хотя итальянские и немецкие газеты открыто воспевали подвиги своих
"легионеров". Упоминаю только об этом, но ведь в таком стиле велась вся
фашистская военная пропаганда."

Дж. Оруэлл, "Вспоминая войну в Испании".

\begin{itemize} % {
\iusr{Игорь Лесев}

Военспецов и летчиков в Испании из Союза было несколько тысяч. И не только из
Союза. Равно как и франкистов массово поддерживали немцы с итальянцами.
Герника, например, ставшая символом немецкой интервенции. И все внешние силы
играли значительную роль в военном конфликте. Но война все равно была
гражданской, потому что основные массы противоборствующих сторон состояли из
граждан Испании.


\iusr{Дмитрий Александров}
\textbf{Игорь Лесев} Так и я об этом.

\end{itemize} % }

\iusr{Арина Родионова}

Наверное, таки надо посмотреть интервью, а то только "перлы" ИВ по интернету
гуляют) да и на реакцию Бмгуса тоже)

Никому не известно что он думает на самом деле, но то, что так говорит, уже
неплохо. А вот теперь послушаем что будет говорить новоизбранный и оценим
степень влияния на него ИгорьВалерьича)

\begin{itemize} % {
\iusr{Игорь Лесев}

думаю, Игоря Валерьевича отодвигают от тела новоизбранного, потому он и задает
сейчас такой тренд в информационном потоке... Коломойский просто так в штаны не
пукнет, а не то, чтобы случайно поднимать резонирующие темы


\iusr{Арина Родионова}
\textbf{Игорь Лесев} , на счёт того, что "просто так не пукнет" категорически согласна) но кто может отодвигать от тела официального владельца этого новоизбранного тела ? сша? Они так же поступили с Фирташем, отодвинув его от тела порошенка
\end{itemize} % }

\iusr{Юлия Гигиберия}

...и тут как солнце из за туч...(с) слов нет, одни маты.. ну, ок, доползли до
признания, хотя бы номинально, гражданского конфликта, даст бог и сам конфликт
исчерпаем...

\begin{itemize} % {
\iusr{Игорь Лесев}
никто еще ничего не признал, проходит только аккуратное прощупывание почвы

\iusr{Елена Максюченко}
\textbf{Игорь Лесев} полностью согласна, Игорь, как чета там появится, так ор. А я считаю, что идет скрытый опрос, для выработки тактики, ведь со стратегией давно все ясно, где то со второй половины 18 года
\end{itemize} % }

\iusr{Юлия Трестер}

Бенефициары и выгодополучатели, организаторы и использователи войн всегда
договариваются и имеют профиты для себя. Людей жалко очень. На этом этапе нужно
людям уже определиться на моей малой родине Гражданин какой ты страны. Кто в РФ
пожалуйста паспорт и вперёд кто Украина остаётся. И да страна не Швейцария, но
она твоя. Донбасс вернут но желательно чтоб там были те кто идентифицирует себя
как Гражданин Украины. 

Мне может многое не нравится в политике проукраинской с перегибами, НО страна и
госмашина имеют право на это защищать свои территории и своих граждан. Но вот
последнее хромает. Я как меня называют «донбассятина», и я глотаю и не спорю.
Спорю когда про «шахтеры быдло». И я не позволю использовать своё прошлое
против государства Украина какое б оно не было. Я понимаю, что есть оценка
«нравится- не нравится», а есть Конституция и паспорт который не просто бумага.
Это я родилась в СССР, росла на Донбассе, а живу в Украине. Но вот дети мои
рождены в Украине и обязаны знать язык, уважать свою страну и язык тоже. Если
бы русскоговорящие объединялись как Граждане Украины, создавали свои ГО и
цивилизованно защищали себя и от РФии кстати тоже, вернее от навязчивых попыток
спасти, то норм, но когда это используют чтоб пролезть в парламент, чтоб
побряцать электоральной поддержкой на переговорах про схемы и должности, вот
уже не то. 

Люди потерялись, но пора взрослеть и определиться. И этап выдачи паспортов даёт
возможность определиться и уехать если невмоготу.  Насильно ж мил не будешь. 

Но я надеюсь что когда нибудь те политики и чиновники, кто бросил своих граждан
там и особенно маленьких, кто не организовал эффективную эвакуацию, не дал
денег и жилья и работы... вот ответят за это предательство.

\begin{itemize} % {
\iusr{Владимир Волков}
\textbf{Юлия Трестер} 

вот тут только единственный вопрос - как потом будут определять оставшихся
граждан Украины? Только лишь по паспорту или будут рыться в прошлом? Условная
бабушка продававшая пирожки в том числе военным на территории ЛДНР будет ли
считаться военной преступницей? А то Аваков как то речи толкал на эту тему.

\iusr{Игорь Лесев}

Зря ты так разбрасываешься людьми. Уедут. Насильно мил не будешь. Люди - это в
прямом смысле главное богатство страны. Источник создания добавленной
стоимости. И про "обязан знать/любить/уважать" тоже спорно. Гражданин обязан
платить налоги и соблюдать законы. Больше никто никому ничего не обязан. Да, и
думаю, никто никого глобально наказывать не будет. Потому что наказание - это
путь в пустоту. Все хотят справедливости и отмщения. А значит и будем
находиться там, где стоим сейчас. Может, другое поколение уже наших детей будет
более конструктивным.

\iusr{Юлия Трестер}
\textbf{Владимир Волков} 

тяжелый вопрос но есть горькая никому не нравящаяся правда о том что могло и не
сделало государство Украина для своих граждан, у которых там активы, квартиры,
родственники. Да можно было в начале аннексии и окупации придумать формат
прийди в СБУ напиши какие там активы, кто родственники, и подпиши бумагу что
именно ты действуешь в интересах Украины. И я б пришла. Потому что я не отвечаю
за выбор мамы например которая осталась в Крыму, но жаль отца квартиру которого
пока он тут в днрии пытались отжать несколько раз. А мне б после такого
заявления выдали б обязательство от имени Государства Украина вернуть мне мои
активы, рассказали бы как мы компенсации получим не вообще Украина, а мой папа
выбравший Украину и потерявший там, или бизнес, который там отжали. Ведь
великие порешали, а не такие великие теряли. Граждане Украины которые оттуда
живут в раскоряке один на один с этими проблемами. И никто не поймёт тех, кто
потерял свой дом и место где похоронены предки. Бабушка не смогла сделать выбор
потому что ей не предложили переехать в Украину на хорошую пенсию в пансионат
прекрасный. А вот как наказать тех, кто не организовал эвакуацию детей и даже
не думал об этом? Эти дети самое главное, а бабушка взрослый человек и сделала
выбор где торговать

\iusr{Юлия Трестер}
\textbf{Игорь Лесев} 

конструктив это выбор если люди выбрали страну получили паспорт пусть едут. А
вот разобрать что могло, должно было и не сделало государство для своих граждан
на территории, которую окупировали . Уверена рано или поздно этот вопрос
поднимется. Ибо соблюдают законы и платят налоги именно для этого, чтоб имея
всю полноту власти, ресурсы и силу обеспечить своим гражданам безопасность и
защиту и реализацию полноценную своих прав. В Конституции ж не просто слова


\iusr{Елена Максюченко}
\textbf{Юлия Трестер} какая вы наивная, прямо завидую

\iusr{Юлия Трестер}
\textbf{Елена Максюченко} я не наивная я верю в то, что люди важны, особенно дети

\iusr{Елена Максюченко}
\textbf{Юлия Трестер} верить, что люди важны и быть наивной, разные вещи. Я тоже в это верю, но в остальном отдаю отчет в происходящем. Вы пребываете в виртуальности, сделала бы, сделали бы, в жизни не бывает сослагательного наклонения, все много жестче

\iusr{Юрий Отцович}
Много букв , а в итоге мы разные !!!

\iusr{Юлия Трестер}
\textbf{Елена Максюченко} ну да до тех пор пока мы не требуем исполнения главного закона от высших чиновников то все жёстче потому что на людей можно плевать. Статьи 3 недостаточно что ли

\ifcmt
  ig https://scontent-frx5-1.xx.fbcdn.net/v/t1.6435-9/59623160_2396186503746311_8509510367309725696_n.jpg?_nc_cat=100&ccb=1-5&_nc_sid=dbeb18&_nc_ohc=O7X0NHK-suUAX8TY1t2&_nc_ht=scontent-frx5-1.xx&oh=3ca79d7d60c68e06c6830ba44c1b3dd3&oe=61BACF65
  @width 0.3
\fi

\iusr{Фролов Руслан}
\textbf{Юлия Трестер} и много вас таких на Донбассе?

\iusr{Юлия Трестер}
\textbf{Фролов Руслан} каких?

\iusr{Фролов Руслан}
\textbf{Юлия Трестер} заукраинских обязанномовных.

\iusr{Юлия Трестер}
\textbf{Фролов Руслан} говорить на языке страны гражданином которой ты есть нормально и правильно

\iusr{Фролов Руслан}
\textbf{Юлия Трестер} так много вас таких?

\iusr{Марина Ныкытенко}

а есть Конституция (с) - да ты шьо, тестер, вспомнила на 5ом году коллективной
стрыбаныны по общественному договору, память внезапно прорезалась, прям чудо

\iusr{Марина Ныкытенко}
\textbf{Юлия Трестер}, вы наведите европейский порядок в понятии "язык страны", потом умничайте пустопорожними нафталиновыми лозунгами, нет у вас украиноговорящей страны, нету, запиши, выучи

\iusr{Фролов Руслан}
\textbf{Марина Ныкытенко} провокатор по ходу.

\end{itemize} % }

\iusr{Yura Serdyuk}
Если бы люди с такой лекгкостью не желали убивать друг - друга. Заповедь: " Не убий!" не появилась бы.

\iusr{Сергей Сысоев}
Ох@ительно написано!!! @igg{fbicon.thumb.up.yellow} 

\iusr{Елена Фёдорова}

Тут такая заковыка. Сразу обозначу. Я - из Днепропетровска. Семья проживала в
нем с конца 1944 ода, сразу после освобождения города. Их отправили в теплушках
восстанавливать город. Конкретно - Горный институт и Криворожские ГОКи. Корни -
дореволюционный Санкт-Петербург, профессор Санкт-Петербургской Горной
Академии,. Дед же был проректором Днепропетровского Горного института. Второе -
город изначально звался Екатеринослав и историю его основания Вы, вероятно,
знаете, ну или легко найдете в Википедии. Как мы, выросшие в русской культуре
восприняли события 13-14 г? как посягательство на свою идентичность. культуру,
этническую принадлежность. Кто посягнул - неонацисты. При полной поддержке США
и ЕС. Вот мы и поднялись защищать страну от распада, упадка и уничтожения. На
кого мы могли рассчитывать? На своих же по культуре и идентичности - на
русских, где бы они ни находились. Защищать Украину? Да, готовы от внешнего
агрессора - поляков, американцев и прочих натовцев, или кому там теоретически
может взбрести в голову. И от сил, возжелавших устроить геноцид. Я уверена,
пройдет время, будут названы причины, пружины, заинтересанты,
выгодоприобретатели госпереворота, все будет разложено по полочкам. Состоятся
суды. И вот тогда выяснится, что именно русские Украины, которых сейчас
убивают, выдавливают из страны, над которыми глумятся и унижают, именно они
спасли Украину и сохранили ее государственность.

\begin{itemize} % {
\iusr{Игорь Лесев}
спасибо за биографическую справку

\iusr{Ирина Слюсарева}
\textbf{Игорь Лесев}, 

вам нравится не вся правда. А только та ее часть, которая вас устраивает. Грубо
говоря, украинская часть. Потому что Елена Федорова вам не справку написала.
Она вам написала, что на Донбассе - русские живут. И конфликт не они начали. И
не по этническим мотивам.

\iusr{Екатерина Асатурова}
\textbf{Ирина Слюсарева}, Вы абсолютно правы.
\end{itemize} % 

\iusr{Ни Соо}

Сам целый министр обороны Полторак крайне удивлен словами Коломойского о
гражданском конфликте! Ведь он ДУМАЛ, что в Украине нет ни одного человека,
который бы так думал!!! То есть по его авторитетному мнению ВСЁ население
Украины считает, что идёт война с РФ. Интересно было бы даже в ФБ провести
интерактивный опрос - чем граждане Украины считают конфликт на Донбассе,
гражданской войной или российско-украинской войной, как это трактует
официальная пропаганда. То что пропаганда не всегда действенна подтверждают
результаты последних выборов.

\iusr{Дмитрий Коломийченко}

1. Для начала нужно говорить правду, называя вещи своими именами, а не удобными
определениями. Покупаем российский газ, российский уголь, бензин, пропан, ведём
бизнес в России и тд.

2. \textbackslash\textbackslash Это самый банальный этнический
конфликт.\textbackslash\textbackslash Не так. Пару недель назад прямо тут одна
барышня отказывала мне украинцу в праве жить в Украине за неудобные вопросы о
политике властей. Да и известные мне случаи вызовов для бесед в СБУ по поводу
подозрения в сепаратизме опять же касались этнических украинцев, а не русских.

3. Большего сепаратиста чем ПАП мне сложно представить, вся его политика была
нацелена на разделение. Наши - не наши. Все эти патриоты при любом неудобном
случае стучащие и отправляющие оппонентов в Москву. Они разве не сепаратисты?

Так что конфликт намного сложнее чем межэтнический.

А Коломойский прекрасно понимает откуда 73\%. Поэтому и так говорит.

\iusr{Maksim Kammerer}

и умные делают ляпы - этнический конфликт, русские убивают украинцев, укры -
орков - эту мысль да в уста палубиёв фетровичей и иже с ними - вот она радость!
- но умный отличается от ряженного - умением признавать свои ляпы, так что пан
чы господин ЛЕсев чы ЛесЕв время у вас есть, не проспать бы, как то вот так:)

\iusr{Василий Стоякин}

Текст отличный, но я вот совсем не уверен, что эти теоретические построения
ведут к миру. К миру ведет победа. Украина победила, но оказалась не в
состоянии воспользоваться плодами победы. Вашингтон победа как таковая не
интересует, его интересует безусловная капитуляция Москвы. До этого еще далеко.

\begin{itemize} % {
\iusr{Елена Фёдорова}
и никогда не будет

\iusr{Василий Стоякин}
\textbf{Елена Фёдорова} никогда не говорите никогда. Только вот недавно, четверть века назад было, а сейчас вдруг не будет?

\iusr{Елена Фёдорова}

русские - государственники, они могут что-то проиграть,ю но не сдаются и
возрождаются. Это, как говорится, в крови. Чего, не в обиду, не хватает
"маленькому украинцу" с хуторяким мышлением. В критический момент они "умывают
руки" и, вполне разумно с бытовой точки зрения, предоставляют право решать свою
судьбу более мощным игрокам

\iusr{Игорь Лесев}

В каком месте Украина победила? Ну ладно, парад в Севастополе откладываем.
Гелетей в отпуске. Но что-то в Донецке правильные парады тоже не проходят.
Поэтому, такая себе победа. Со слезами на глазах.


\iusr{Василий Стоякин}
\textbf{Игорь Лесев} Минск-2

\iusr{Елена Фёдорова}

победы видимой, с плененным врагом, флагами и парадами быть, увы! не может.
Победа лишь в том, чтобы остановить убийства и снизить уровень ненависти. И
договориться, как жить дальше без кровопролитий. И, может, как-то по-тихоньку
начать восстанавливать экономику.Людям дать работу. и проч. социалку

\iusr{Игорь Лесев}
\textbf{Василий Стоякин} если не ошибаюсь, после Минск-2 отминусовалось еще и Дебальцево. Потрясающая перемога.

\iusr{Василий Стоякин}
\textbf{Игорь Лесев} после Минска-2 можно было и Мариуполь отминусовать. Делов-то

\iusr{Елена Фёдорова}

Без идеологической мишуры и фантиков. В сухом остатке имеем: Россия сразу
заявила, Донбасс не планируем брать. Донбасс не хочет, чтоб его сжигали, как
Одессу, как бы это понять пересичному украинцу? Или обязаны покорно лечь на
плаху? К тому же на Донбассе есть знак равенства между фашизмом и Украиной. Это
заслуга украинской государственной машины и действий ВСУ. Украина не хочет
принимать Донбасс назад, по крайней мере с людьми. Бюджет не выдержит выплатить
долги по пенсиям,,компенсации и затраты на восстановление инфраструктуры.

\iusr{Игорь Лесев}
\textbf{Василий Стоякин} не, ну если в этом направлении размышлять, то это несомненная победа. Сколько у нас еще городов остается? 350 точно наскребем. Вот тебе и 350 побед. Все-таки Петр Алексеевич научил нас всех позитивно фантазировать

\iusr{Василий Стоякин}
\textbf{Игорь Лесев} 

это направление не обязательное. Главное то, что Минску-2 никаких ЛДНР не
существует - это территория Украины, которая должна быть возвращена и зачищена.
Плюс-минус один населенный пункт тут роли не играет.

\iusr{Maksim Kammerer}

ну, ну- а мальчик вася одел памперсы и взял москву, как-то так куется победа:)

\iusr{Ирина Слюсарева}
\textbf{Василий Стоякин}, 

вот из-за таких, как вы, сложно договориться. Минские договоренности является
"победой" Украины, атож. И на бумаге ЛНДР не существует. Какой-то мечтатель
прямо. Ну если ЛНДР не существует, если это

\iusr{Василий Стоякин}
\textbf{Ирина Слюсарева} 

Я тут причем? Минские соглашения опубликованы, никто не мешает взять их и
прочитать. Поискать там ссылку на народные республики. Попытаться понять как
это - федерализация, если с одной стороны - Украина, с другой - Новоазовский
район Донецкой области. Допустим это не победа Украины. Но целью Украины
является ликвидация республик, а именно она-то и предполагается минском.

\iusr{Ирина Слюсарева}
\textbf{Василий Стоякин}, 

об чем и речь. "Предполагается" - не тождественно "является". Республики-то
есть. И очень мало шансов, что они будут зачищены на условиях Украины.

\iusr{Василий Стоякин}
\textbf{Ирина Слюсарева} Республики не будут зачищены только в одном случае - если Минск выполнен не будет.

\iusr{Ян Пругло}
\textbf{Игорь Лесев} 

страна не может заявлять о победе по определению, если война идет на ее
территории. Это УЖЕ удар. А победа одних или других решается на более
глобальных фронтах, здесь просто захолустье. Это как внезапное просветление
власти в Румынии в 44-м году.

\end{itemize} % }

\iusr{Елена Фёдорова}

Почему никто не говорит,что на Украине воюют русские с русскими, судя по языку
общения. Да и последние выборы показали мощь смешанного русско-украинского
населения. На Украине конфликт цивилизационных проектов пытаются перевести в
плоскость этнического. Я имею в виду хуторянское и государственное мышление и
идентичность


\iusr{Rustam Isakov}

тут є іще суто практична сторона питання. Якщо громадянського конфлікту-війни
немає, то треба погоджуватись на реінтеграцію Донбасу в рамках МУ. Там же наші
люди, котрі мріють повернутись в Україну. Але зараз в рейтингах партія
Медведчука на другому місці. Після повернення Донбасу вона логічно буде на
першому і цілком імовірно збере більшість у Раді. А там і до зміни курсу не
далеко. І тоді ініціювати громадянський конфлікт доведеться уже тим, хто його
зараз не хоче визнавати. Точно так, як у 2014 захоплювали ОДА і відмовлялись
підкорятись Києву спочатку майданівці, якщо хтось забув. І якби Янукович
переміг, то центр сепратизму був би у Львові, а не Донецьку. Але життя нічому
не вчить.

\begin{itemize} % {
\iusr{Игорь Лесев}

Вы правы. У нас же удобнее строит колхоз с родоплеменными отношениями, нежели
подняться чуть выше в цепочке и выстраивать комфортное общество, где никому
никто ничего не должен, а главный скрепляющий винтик - это интерес и комфорт

\end{itemize} % }

\iusr{Сергей Бухтияров}

Вы бы видели, что сегодня устроили на блокпосту в Мариуполе на выезде в сторону
Мангуша. Тупо остановили проезд. Более двух часов в очереди. Это делают
специально, т. к. в этот день был на двух других блокпостах, где не было никаких
очередей. И что должны люди думать о них. Маты громко! Половина разворачивалась и
уезжала. Достали!


\iusr{Владимир Михайловский}

Гражданская война - да. Этнический конфликт - категорически нет. Вы безусловно
правы в том,что успешное лечение без точной диагностики невозможно. А если
присвоивший себе права лекаря боится диагноза, но продолжает "лечить", значит он
шарлатан.

\begin{itemize} % {
\iusr{Марина Ныкытенко}
конфликт не этнический, а этнографический
\end{itemize} % }

\iusr{Владимир Волков}

называйте, как угодно, но суть от этого не изменить: одни украинцы убивают
других. то, что - при участии граждан других государств - сама суть не
меняется, т. к. "пригласительные билеты" поступают с обеих сторон. гости помогают
убивать - ни больше, ни меньше.

\iusr{Матвей Кублицкий}
стадии отрицания и гнева прошли. наступает стадия торга. 20-22 года - депрессия и лишь после принятие....

\iusr{Марина Ныкытенко}
хочу видеть список признаков, по которым автор отличает русских от украинцев

\iusr{Тетяна Кузьмішкіна}
Цікава точка зору ...

\iusr{Ольга Ремигайло}
Есть и Одесса - Москва, и Киев - Москва, и масса автобусов и BlaBlaCar, которые заполнены )))). Самолеты через Минск и Ригу также )

\iusr{Ольга Ремигайло}
Войны между нашими странами нет (и ни дай Бог), а есть противостояние между государствами

\iusr{Виталий Резниченко}

Наконец-то пошло понимание... Замечательно, браво, пост заслуживающий внимания
и уважения. Пора бы уже проснуться братьям вильным козакам, а то шляхта нэ
дримлэ...

\end{itemize} % }
