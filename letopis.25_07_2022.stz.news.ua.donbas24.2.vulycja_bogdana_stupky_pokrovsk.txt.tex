% vim: keymap=russian-jcukenwin
%%beginhead 
 
%%file 25_07_2022.stz.news.ua.donbas24.2.vulycja_bogdana_stupky_pokrovsk.txt
%%parent 25_07_2022.stz.news.ua.donbas24.2.vulycja_bogdana_stupky_pokrovsk
 
%%url 
 
%%author_id 
%%date 
 
%%tags 
%%title 
 
%%endhead 

Вулиця Богдана Ступки в Покровську: яким був життєвий шлях українського митця

Завдяки декомунізації 2016 року в Україні безліч вулиць, проспектів, провулків
і навіть міст змінили свої назви. В той самий час у Покровську вулиця Ватутіна
перетворилася на вулицю Богдана Ступки. Така трансформація якомога краще
підкреслює прояв української ідентичності в міському просторі.

Богдан Ступка — видатний театральний діяч і кіноактор, володар численних звань
та премій за внесок у розвиток мистецтва України, до останніх днів життя
відданий йому. Творча біографія Богдана Сильвестровича налічує понад сто
фільмів та серіалів та півсотні ролей у театрі, які завжди мали величезний
успіх у глядачів. 

Богдан Сильвестрович Ступка народився 27 серпня 1941 року неподалік Львова.
Через кілька років після Перемоги, коли хлопчикові було сім, родина переїхала
до Львова, у цьому місті пройшли дитинство та юність великого актора.

Богдан ріс в оточенні музики: батько був хористом Львівського оперного театру,
там же співав дядько, а тітка працювала концертмейстером. Мати в молодості
працювала продавцем у кооперативі, після заміжжя господарювала, але також мала
артистичний талант — добре співала і грала в колективі художньої
самодіяльності.

Достаток у сім'ї артиста оперного театру був вкрай скромним, але грошей на
дороге захоплення сина літературою не шкодували. Батьки були проти того, щоб
Ступка пов'язував життя з мистецтвом.

Після десятирічки хлопець вирішив вступити до політехнічного інституту. Але
успіх не посміхнувся випускнику, він не пройшов конкурс.

Перш ніж прийти до акторства, Богдан Ступка змінив кілька професій: у молодості
він працював учнем у слюсарній майстерні при виші, робив знімки у студентській
обсерваторії та працював у молодіжному джазовому ансамблі «Медікус». Виступи
Ступки мали шалену популярність у публіки, і зрештою молодий львів'янин
зрозумів, що настав час починати серйозно розвиватися в акторському напрямку.

Згодом Богдан Ступка вступив до драматичної студії при Театрі імені Марії
Заньковецької та закінчив її у 1961 році. Після цього він залишився в трупі
театру, де пропрацював сімнадцять років. .

У 1963 році артиста призвали до армії, і він протягом трьох років поєднував
службу з професійною діяльністю в Ансамблі пісні та танцю військового округу
Прикарпаття. Паралельно заочно навчався на філологічному факультеті Львівського
державного університету імені Івана Франка.

У 1978 році переїхав до Києва і почав грати в Київському українському
драматичному театрі імені Івана Франка.

Широку популярність артисту в світі кіно принесла драма «Білий птах із чорною
ознакою» (1971), де він зіграв непросту роль Ореста Звонаря. У цій картині
Богдану Сильвестровичу довелося зіграти першу постільну сцену. Його партнерка
Лариса Кадочникова була одружена з постановником, і артисти ніяк не могли
розслабитися перед камерою. Щоб все ж таки зняти епізод, який Ступка пізніше
назвав «стриптизом по-радянському», акторам довелося випити коньяку.

Після виходу стрічки на екрани режисери буквально атакували Ступку новими
сценаріями. Його фільмографія складається з таких кіношедеврів, як

«Найостанніший день» (1972), «Право на любов» (1977), «Дударики» (1980),
«Таємниці святого Юра» (1982), «Камінна душа» (1988), «Вогнем і мечем» (1999)
та інших.

У 1999 році Богдан Ступка отримав посаду міністра культури та мистецтв України.
На цій посаді Богдан Сильвестрович прослужив лише два роки, пізніше
відмовившись від неї заради ближчої йому посади художнього керівника Театру
імені Івана Франка, який став йому за минулі роки другим домом.

За час, поки Ступка був міністром, зроблено чимало позитивних змін. Він
розпорядився, щоб за наявності вільних місць у залі на спектаклі пускали
студентів та пенсіонерів безкоштовно. На жаль, з його звільненням з посади цю
практику було скасовано.

Своє особисте життя актор пов'язав із Ларисою Корнієнко. У шлюбі подружжя
прожило сорок п'ять років, трохи не доживши до золотого весілля. Дружина
залишалася з Богданом Сильвестровичем аж до його смерті.

У лютому 2012 року в пресі з'явилися повідомлення про те, що Богдан
Сильвестрович серйозно хворий. Обстеження показало злоякісну пухлину. Протягом
двох років актор мужньо боровся із хворобою.

У ніч із 21 на 22 липня, не доживши місяць до 71-річчя, видатний драматичний
актор Богдан Ступка помер в українській столиці, у спеціалізованій клініці
«Феофанія». Причиною смерті стала серцева недостатність. В останній шлях
Богдана Сильвестровича проводжали тисячі відданих шанувальників його
багаторічної творчості. 

Богдан Ступка у своїх висловлюваннях завжди дуже підтримував Україну та
українців. «Головне, що слід розуміти українцям, — це те, що вони унікальні, бо
вільні!» — одна з його найпопулярніших цитат.

Читайте також:

— Як морпіхи захищали Маріуполь: історія мужнього воїна

— Маріупольці взяли участь в благодійному концерті в англійському місті

Найсвіжіші новини та найактуальнішу інформацію про Донецьку й Луганську області
також читайте в нашому телеграм-каналі Донбас24.

ФОТО: з відкритих джерел
