% vim: keymap=russian-jcukenwin
%%beginhead 
 
%%file 18_08_2020.fb.ukc_ukrainianculturalcenter.1.nkvd_vov_dneproges
%%parent 18_08_2020
 
%%url https://www.facebook.com/UKC.UkrainianCulturalCenter/posts/2642488112683448
 
%%author 
%%author_id 
%%author_url 
 
%%tags 
%%title 
 
%%endhead 

\subsection{18 серпня 1941 року близько 20:00 співробітники НКВД без попередження підірвали греблю Дніпрогесу}
\Purl{https://www.facebook.com/UKC.UkrainianCulturalCenter/posts/2642488112683448}


\ifcmt
  pic https://scontent-mad1-1.xx.fbcdn.net/v/t1.6435-9/118084846_2642487889350137_8824438141754383117_n.jpg?_nc_cat=108&ccb=1-3&_nc_sid=8bfeb9&_nc_ohc=cvrRVT72bHMAX88URK4&_nc_ht=scontent-mad1-1.xx&oh=5c0802f24b97c1222dc15546f672802d&oe=60958ABB

	pic https://scontent-mad1-1.xx.fbcdn.net/v/t1.6435-9/118027508_2642487919350134_2257713098012213760_n.jpg?_nc_cat=111&ccb=1-3&_nc_sid=8bfeb9&_nc_ohc=-zySavbYf0EAX-1o_ti&_nc_ht=scontent-mad1-1.xx&oh=11ef96f99990251724e631711e43fc75&oe=60944BDB

	pic https://scontent-mad1-1.xx.fbcdn.net/v/t1.6435-9/118118174_2642487956016797_568403737493348379_n.jpg?_nc_cat=100&ccb=1-3&_nc_sid=8bfeb9&_nc_ohc=GzHQQdCLqwoAX-5iDEc&_nc_ht=scontent-mad1-1.xx&oh=a11f4d76ef9e15a81f0106d34392e2dc&oe=6093E8AA

	pic https://scontent-mad1-1.xx.fbcdn.net/v/t1.6435-9/118038520_2642487986016794_2566356838811102482_n.jpg?_nc_cat=101&ccb=1-3&_nc_sid=8bfeb9&_nc_ohc=4A2CZwaG8BsAX99rBmB&_nc_ht=scontent-mad1-1.xx&oh=b541e747ccbf4d3bdeeaf146bba13007&oe=6096C308

	pic https://scontent-mad1-1.xx.fbcdn.net/v/t1.6435-9/118144846_2642488009350125_2382634755920725096_n.jpg?_nc_cat=102&ccb=1-3&_nc_sid=8bfeb9&_nc_ohc=dD_NYwPU4u0AX82vSS0&_nc_oc=AQntOMoVda6FVtalYOxjSGkzDabSC_rxdizQz-Fif1f95R1QU3BQIGvv3RaOC2Y0zhw&_nc_ht=scontent-mad1-1.xx&oh=675b275d495363a1ef6d69680f565ed5&oe=60939BE2

	pic https://scontent-mad1-1.xx.fbcdn.net/v/t1.6435-9/117987246_2642488046016788_589845723085274165_n.jpg?_nc_cat=107&ccb=1-3&_nc_sid=8bfeb9&_nc_ohc=mss97CmlRcUAX9yPjCI&_nc_ht=scontent-mad1-1.xx&oh=ec4c4ad9fca9eda52053d011cfda80d6&oe=6095C010

	pic https://scontent-mad1-1.xx.fbcdn.net/v/t1.6435-9/118089121_2642488082683451_90597442796391950_n.jpg?_nc_cat=107&ccb=1-3&_nc_sid=8bfeb9&_nc_ohc=uub0Aq5WZ7YAX9ot30T&_nc_ht=scontent-mad1-1.xx&oh=478d035d47bc048fd38bfbeca5f6f8d0&oe=60938C73
\fi


18 серпня 1941 року близько 20:00 співробітники НКВД без попередження підірвали
греблю Дніпрогесу. Вибух 20 тонн толу зруйнував частину греблі довжиною 165
метрів, внаслідок чого 20-метрова водяна хвиля змила прибережну міську смугу,
плавні Хортиці і дійшла до міст Марганця та Нікополя, розміщених майже за 80
кілометрів вниз за течією Дніпра.

Кількість жертв трагедії достеменно невідома, оскільки їх ніхто не рахував.
Історики та старожили кажуть, що хвиля у 30 метрів заввишки горнула все на
своєму шляху.  

Найчастіше дослідники говорять про 100 тисяч загиблих: 80 тисяч жителів
Запоріжжя та його околиць, біженців із сусідніх регіонів та близько 20 тисяч
радянських солдатів, які не встигли покинути місто. Німецьке командування
оцінювало свої втрати в живій силі у 1 500 осіб.

На той час радянські спецслужби проводили масове мінування міст, які здавали
німцям, за даними Інституту національної пам'яті. А потім підривали їх, аби за
будь-яку ціну завдати ворогу максимальної шкоди у відповідності із «тактикою
випаленої землі», проголошеної Сталіним 3 липня 1941 року.

У промові Сталін закликав знищувати все цінне, що неможливо було вивезти з
територій, які опинилися під загрозою ворожої окупації. 

Попри те, що ці заклики стосувалися переважно таких ресурсів як транспортні
засоби, пальне, збіжжя, худоба та запаси продуктів харчування, під час відступу
не рахувалися зі знищенням пам’яток архітектури світового значення та важливих
інженерних споруд.

Промова Сталіна спиралася на положення Директиви Раднаркому СРСР та ЦК ВКП(б)
для партійних та радянських організацій прифронтових областей про мобілізацію
всіх сил і засобів на розгром фашистських загарбників від 29 червня 1941 року.

Під час втілення «тактики випаленої землі» радянським військовим командуванням
та НКВД загинули десятки тисяч цивільних і радянських військових. Вивезення і
знищення запасів продуктів харчування вже восени та взимку спричинило масовий
голод на окупованих нацистами територіях.

Найвідомішими проявами такої тактики під час відступу Червоної армії у 1941
році стали знищення старовинних архітектурних пам’яток центральної частини
Києва та греблі Дніпрогесу у Запоріжжі.

Так у Києві перед здачею міста німцям комуністична влада почала нищити запаси
харчів, зруйнувала водогін, електростанцію, висадила у повітря чотири мости. За
свідченням очевидців, мостами під час вибуху все ще відступали радянські
війська.

Саме тоді радянські спецслужби замінували майже всю центральну частину міста.
Точно відомо про замінування будинків на Хрещатику, будинків Держбанку,
Оперного театру, Центрального комітету партії на Михайлівській площі та будівлі
НКВД на Володимирській, Музею Леніна — колишнього педагогічного музею, де свого
часу засідала Українська Центральна Рада. Існують також свідчення про спроби
замінувати Софійський собор.

На території Лаври, окрім Успенського собору, замінували й інші будівлі.

Німці не поспішали брати місто, практично не обстрілювали його і лише посилили
оточення — вони планували зберегти Київ для розміщення гарнізонів.

Німецькі війська увійшли в Київ 19 вересня 1941 року.

Цікаво, що Совінформбюро не повідомляло про здачу української столиці аж до 21
вересня. У Києві більше доби хазяйнували окупанти, а радянське радіо та газети
стверджували й надалі, що «протягом 20 вересня наші війська вели наполегливі
бої із супротивником на всьому фронті й особливо запеклі під Києвом». Вочевидь,
і тут Совінформбюро збрехало. Саме з таких «дрібничок» і формувалося геть
викривлене радянське бачення цієї війни.

Вибухи на Хрещатику сталися 24 вересня — почалися величезні пожежі. Німці
намагалися їх гасити, протягнувши шланги до Дніпра і качаючи воду. Однак
радянські підпільники їх перерізали. Радянську пожежну техніку вивезли з Києва
раніше, німецьких пожежних машин не вистачало. Пожежа тривала майже два тижні.

Всього в районі Хрещатика зруйнували 324 будівлі, тисячі киян залишилися без
даху над головою.  Вибух в Успенському соборі Лаври стався 3 листопада 1941
року, за дві години по тому, як його відвідав словацький президент Йозеф Тісо. 

\verb|#УКЦ|
