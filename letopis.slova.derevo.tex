% vim: keymap=russian-jcukenwin
%%beginhead 
 
%%file slova.derevo
%%parent slova
 
%%url 
 
%%author 
%%author_id 
%%author_url 
 
%%tags 
%%title 
 
%%endhead 
\chapter{Дерево}
\label{sec:slova.derevo}

%%%cit
%%%cit_pic
%%%cit_text
Вспоминаю 14-й год, вся страна, вернее, та ее часть, которой не нужен скипидар
на жопу, закрашивала движимое и недвижимое имущество в желто-синие цвета.
Заборы, столбы, лавочки, \emph{деревья}, вот все. У меня у подъезда перекрасили даже
урну. Феерическое зрелище. Помойка в цвете национального флага. Правда, через
несколько дней на ней же дописали черным мелком «Росiя». Но это еще только
больше подчеркнуло ебанариум головного мозга желто-голубых покрасчиков
%%%cit_comment
%%%cit_title
\citTitle{Бандеровцы при власти создали в Украине уютненькую Уганду}, 
Игорь Лесев, strana.ua, 13.06.2021
%%%endcit

%%%cit
%%%cit_head
%%%cit_pic
%%%cit_text
Жребий был брошен, когда Енот орлом набросился на голову человека,
поднимавшегося по лестнице. Обдумывая случившееся, можно было прийти к выводу,
что они вышли из этой переделки гораздо благополучнее, чем могли. Пострадал
лишь Марк: удар копытом в живот и царапина на плече, по которому его плашмя
ударили мечом.  Хол сидел за \emph{деревом} и смотрел на волков. Их поведение
свидетельствовало, что поблизости никого нет. Хол встал и зашуршал листвой.
Волки повернули к нему головы и вскочили. Он снова зашуршал, и волки, как серые
тени, исчезли в лесу.  Хол спустился по холму, огибая две группы углей. Они все
еще излучали жар, такой приятный в это холодное утро. Он постоял немного,
впитывая тепло.  Хол отыскал на влажной земле следы лошадей и людей, и подумал
о судьбе хозяина и его жены. Он вспомнил слова девушки о том, что хозяйка
пряталась в погребе от своего пьяного мужа. Была ли она там, когда гостиница
запылала? Если так, то ее сгоревшее тело должно было находиться там, среди
углей: деревянная гостиница вспыхнула как спичка, а выбраться оттуда было
невозможно
%%%cit_comment
%%%cit_title
\citTitle{Зачарованное паломничество}, Клиффорд Саймак
%%%endcit
