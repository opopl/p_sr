% vim: keymap=russian-jcukenwin
%%beginhead 
 
%%file 10_06_2019.stz.news.ua.mrpl.vidsebe.1.atmosferni_kutochky_mariupolja
%%parent 10_06_2019
 
%%url https://vidsebe.media/blogs/34032/atmosferni-kutochki-mariupolya-abo-kudi-podatisya-yakscho-vpershe-u-misti-
 
%%author_id news.ua.mrpl.vidsebe,mysyk_bogdana.kyiv
%%date 
 
%%tags kultura,mariupol,mariupol.pre_war,mariupol.vezha
%%title Атмосферні куточки Маріуполя, або куди податися, якщо вперше у місті
 
%%endhead 
 
\subsection{Атмосферні куточки Маріуполя, або куди податися, якщо вперше у місті}
\label{sec:10_06_2019.stz.news.ua.mrpl.vidsebe.1.atmosferni_kutochky_mariupolja}
 
\Purl{https://vidsebe.media/blogs/34032/atmosferni-kutochki-mariupolya-abo-kudi-podatisya-yakscho-vpershe-u-misti-}
\ifcmt
 author_begin
   author_id news.ua.mrpl.vidsebe,mysyk_bogdana.kyiv
 author_end
\fi

\ifcmt
  ig https://i2.paste.pics/d0fcef53bd966a735f2da619a783cce3.png
  @wrap center
  @width 0.8
\fi


10 червня 2019 08:55

Богдана Мисик

Учасниця проекту GOxChange

\textbf{Перед поїздкою до Маріуполя в голові крутилось єдине питання: «Наскільки там
безпечно?». Повернувшись до Києва, я можу дати відповідь: Маріуполь — це мирне
та спокійне місто.}

Зараз Маріуполь дуже стрімко змінюється та розвивається завдяки зусиллям
активних жителів. Саме з такими людьми ми зустрічалися під час програми
\href{https://vidsebe.media/read/ljudy/33574/mi-rizni-ale-ridni-yak-molod-zi-vsiei-ukraini-doslidzhuvala-mariupol-ta-borolasya-zi-stereotipami}{GOxChange}. Нам вдалося відвідати безліч крутих та атмосферних місць, до яких я
б хотіла радо запросити і вас.

\begin{center}
\Large\em Топ-3 куточки Маріуполя, які мені найбільше припали до душі: 
\end{center}

\begin{center}
{\Large\href{https://www.facebook.com/vezhacreativespace}{Vezha Creative Space}}
\end{center}

\ii{10_06_2019.stz.news.ua.mrpl.vidsebe.1.atmosferni_kutochky_mariupolja.pic.1}

Сюди можна прийти, щоб відпочити, насолодитися прекрасним видом на місто та
дізнатися нові подробиці про Маріуполь. Колись, це була водонапірна вежа, але з
часом вона перестала виконувати свою початкову функцію та була закинута. Не
буду вдаватися в подробиці та розповідати про всі перетворення, які з нею
відбулися (включаючи часи, коли вона слугувала відділенням банку). Зазначу лиш,
що зараз вежі дали друге життя та перетворили її у справжній креативний
простір.

\begin{center}
	\begin{fminipage}{0.9\textwidth}
		\href{http://vidsebe.media/read/gajdy/29413/ne-buterbrod-s-ikroj-interesnye-suveniry-kotorye-nestydno-privezti-iz-mariupolya}{%
Не бутерброд с икрой: интересные сувениры, которые нестыдно привезти из Мариуполя, %
vidsebe.media, 11.04.2019}
	\end{fminipage}
\end{center}

Тут постійно проводяться різноманітні івенти (безкоштовні або за символічну
плату): перегляди кінострічок, зустрічі з цікавими людьми, лекції тощо. На
одному з поверхів ви знайдете фортепіано — на ньому можна навіть пограти, якщо
це не заважатиме іншим відвідувачам. І обов'язково, якщо будете у Вежі,
піднімайтеся до самого верху — не пошкодуєте.

\begin{center}
{\Large\href{https://www.facebook.com/halabudavp}{Вільний простір Halabuda}}
\end{center}

\ii{10_06_2019.stz.news.ua.mrpl.vidsebe.1.atmosferni_kutochky_mariupolja.pic.2.halabuda}

Якщо ви любите тренінги, курси та просто неформальну освіту — вам сюди! Це
місце розпочинало свій шлях, як пуста коробка без нічого. Але Дмитро Чичера,
Галя Балабанова та інші натхненні волонтери, здавалося би, зробили неможливе,
перетворивши її на вільний простір з креативною назвою Halabuda. Саме тут ми,
учасники проекту, мали змогу послухати та познайомитися з багатьма громадськими
діячами, активістами, свідомими жителями міста.

\begin{center}
{\Large\href{https://www.facebook.com/laziness.coffee.mrpl}{Кав'ярня LENЬ}}
\end{center}

\ii{10_06_2019.stz.news.ua.mrpl.vidsebe.1.atmosferni_kutochky_mariupolja.pic.3.kavjarnja_len}

Тут ви відчуєте себе як вдома, або навіть краще! Книги на полицях, намальований
олень на стіні, альтернативні методи заварювання кави, смачні десерти та
привітна посмішка власниці Жені Стецької — все це робить кав'ярню особливою.
Справжній райський куточок для тих, хто любить спокій та затишок. Наш останній
святковий сніданок в LENЬ неможливо забути — прекрасне завершення чудової
програми.

\begin{center}
	\begin{fminipage}{0.9\textwidth}
		\href{https://vidsebe.media/articles/28930/vse-nachalos-s-rogov-pravdivaya-istoriya-otkrytiya-kofejni-len}{%
Все началось с рогов: правдивая история открытия кофейни LENЬ, Тетяна Мацкова, vidsebe.media, 05.04.2019%
}
	\end{fminipage}
\end{center}

Ще хочу згадати платформу \enquote{ТЮ}, яхт-клуб та море — але їх словами не опишеш, це треба бачити!

\ii{10_06_2019.stz.news.ua.mrpl.vidsebe.1.atmosferni_kutochky_mariupolja.pic.4.platforma_tju}

\begin{center}
\Large\bfseries І, звісно, завод.
\end{center}

Цей пункт слід вказати окремо від мого топу. Непідготовленим туди краще не йти
— подих перехоплює від побаченого. Це саме той досвід, який змінює твоє
світосприйняття та показує, що насправді важливо і які проблеми нам треба
вирішувати якнайшвидше.

\ii{10_06_2019.stz.news.ua.mrpl.vidsebe.1.atmosferni_kutochky_mariupolja.pic.5.zavod}

Хтось на проекті сказав: \enquote{Бути еко-активістом в Києві та говорити про
сортування сміття набагато легше, ніж бути еко-активістом в Маріуполі}. Чи
можна отруєння повітря та води вважати злочином проти населення? Відповідь стає
очевидною, коли ти дуже хочеш, але не можеш вдихнути на повні груди.

На цій оптимістичній ноті хотілося би долучитися до вимог еко-активістів
Маріуполя та закликати керівників заводу встановити якнайшвидше очисні фільтри.  

\emph{У матеріалі використані фото з архіву програми GOxChange.}
