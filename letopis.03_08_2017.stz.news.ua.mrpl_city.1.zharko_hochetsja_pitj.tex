% vim: keymap=russian-jcukenwin
%%beginhead 
 
%%file 03_08_2017.stz.news.ua.mrpl_city.1.zharko_hochetsja_pitj
%%parent 03_08_2017
 
%%url https://mrpl.city/blogs/view/zharko-hochetsya-pit
 
%%author_id burov_sergij.mariupol,news.ua.mrpl_city
%%date 
 
%%tags 
%%title Жарко, хочется пить
 
%%endhead 
 
\subsection{Жарко, хочется пить}
\label{sec:03_08_2017.stz.news.ua.mrpl_city.1.zharko_hochetsja_pitj}
 
\Purl{https://mrpl.city/blogs/view/zharko-hochetsya-pit}
\ifcmt
 author_begin
   author_id burov_sergij.mariupol,news.ua.mrpl_city
 author_end
\fi

\ii{03_08_2017.stz.news.ua.mrpl_city.1.zharko_hochetsja_pitj.pic.1.gazvoda}

О чем писать в жару? Конечно, о мариупольской  июльско-августовской  погоде.
То, что сейчас происходит, наши современники сами знают. А раньше как было?
Первые послевоенные годы. Базарная площадь. Палящее солнце, среди продавцов и
по­купателей шныряет босоногий, вечно голодный долговязый подросток с помятым
алюминиевым бидоном, напол­ненным холодной водой из кринички – колодца на
Малофонтанной улице, к ручке бидона приторочена метровой длины бечевкой
видавшая виды эмалированная кружка. Подросток голосом, ломающимся с детского
дисканта на будущий баритон, истошно кричит: \enquote{Кому воды налить холодной? Есть
холодная вода. Рупь кружка, рупь кружка}. К концу дня, устав, он \enquote{меняет
пластинку} и нараспев заводит: \enquote{Колбаса, сметана, сыр, табак и спички,
часовщик, ювелир, а кому налить водички...} Этот нескладные вирши, нужно
думать, он придумал сам.

\ii{03_08_2017.stz.news.ua.mrpl_city.1.zharko_hochetsja_pitj.pic.2.krinichka}

Будущий \enquote{баритон}, естественно, был не единственным на базаре \enquote{водоносом}.
Промышляли здесь  и другие мальчишки, которые сбегали по каменным ступеням с
бидончиками к колодцу. Набирали спасительную влагу, и мчались стремглав вверх
по лестнице к месту торга своего товара.  Товар их - вода - была хоть и
жестковата, зато  прохладна и чиста.  Это была та же вода, которую пили  казаки
из крепостицы – центра Кальмиусской паланки, а затем греки, переселенцы из
Крыма. Эту же воду гражданский  инженер  Виктор Александрович Нильсен
использовал в построенном им водопроводе.

* * *

Начало 50-х годов. Воду, пронизанную искрящимися на солнце пузырь­ками,
мариупольцы называли зельтерской. Не все, а только те, кто в сознательном
возрасте захватил \enquote{мирное время} и НЭП. На языке стариков и старушек
двух послевоенных десятилетий обозначался исторический период до начала Первой
мировой войны, совпавший с их молодостью. Для пред­ставителей более молодых
поколений, привыкших ко всякого рода словесным сокращениям, этот
прохладительный напиток был просто газводой.

Достижения современной технологии загнали ее в пластиковые бутылки,
различающиеся названиями на голубых, светло-синих или синих наклейках, но,
кажется, содержа­щие один и тот же продукт. Но перенесемся в прошлое
Мариуполя. На проспекте Республики расставлены удобные скамейки для отдыха
досужего люда, а у редакции \enquote{Приазовского рабочего} - витрина со свежим номером
единственной городской газеты, \enquote{лишний билетик} на вечерний сеанс нового
фильма в кинотеатре ­\enquote{Победа} спрашивают уже на углу улицы Карла Маркса, у
магази­на \enquote{Культтовары}. Летними вечерами по проспекту, причем только по одной
его стороне, фланирует публика.  На сравнительно небольшом отрезке проспекта от
Базар­ной площади до сквера можно увидеть по меньшей мере пять-шесть голубого
цвета тележек, у которых идет торговля газводой, особенно бойкая в жару. Стоят
они близ Торговой у треста, рядом с Дворцом культуры завода \enquote{Азовсталь}, чуть
ниже кинотеатра \enquote{Победа}, напротив 16-й аптеки, расположенной в первом этаже
45-го дома, откуда рукой подать до сквера.

Нехитрое оборудование тележки состоит из устройства для мытья стаканов (чтобы
стакан вымыть, нужно поставить на круг, поворот ручки - и струи воды омывают
его изнутри и снаружи), источающего газводу крана, носок которого обернут в
несколько слоев марлей, венчает все стойка с двумя стеклянными
сосудами-трубками, оснащенными со дна краниками. Один из сосудов заполнен
мандариновым или грушевым сиропом, содержимое другого - тоже сироп, но
малиновый или, скажем, вишневый. Само собой разумеется, ни в ман­дариновом, ни
в малиновом, ни в грушевом сиропе сока этих плодов нет и в помине, их роль
исполняют синтетические эссенции, окрашенные в соответствующие цвета.

Уровень сиропа в сосудах постепенно снижается, но все­гда остается прикрытым от
взоров жаждущих хлебнуть стаканчик - другой газводы наклейкой-ценником, на
котором обозначены стоимость стакана воды без сиропа - ребятня называет ее
\enquote{чистой}, с одинарной порцией сиропа и порцией двойной. Когда-то стакан
наполнялся до краев, теперь - не более чем на три четверти. Но и за то спасибо.
В конце концов, можно купить два или даже три стакана. Цена-то плевая.  Все
знающие, все слышащие и все разумеющие люди говаривали, мол, \enquote{газировщицы}, то
есть продавщицы газводы, за летний сезон сколачивали приличные состояния, а
потому место у голубой тележки покупалось за солидные деньги. Так ли это?
Правда ли? И раньше никто точно не знал, а теперь - тем более. И спросить-то не
у кого. А и надо ли?

Но это сторона потаенная, а память вновь возвращает в прошлое. Тележка. Рядом -
баллон с углекислым газом, обтянутый светлым чехлом, чтобы не перегрелся.
Женщина в белом накрахмаленном кокошнике, прикрепленном закол­ками к прическе,
в белом же непромокаемом фартуке. Руки у нее от постоянного соприкосновения с
холодной водой красные, воспаленные. У женщины уже в середине дня усталый,
отсутствующий взгляд. Как автомат, она раздает стаканы с пузырящейся жидкостью,
окрашенной то в оранжевый, то в ядовито-красный цвет, а то и с неокрашенной. О
дно блюдца звякают монеты, мокрая \enquote{сдача} передается покупателям, от тележки
струится тонкий ручеек, газвода с шипением вырывается из крана. \enquote{Мне - чистой}.
\enquote{Два стакана с малиновым}. \enquote{Двойной  грушевый}... Жарко, хочется пить.
