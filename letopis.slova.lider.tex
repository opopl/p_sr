% vim: keymap=russian-jcukenwin
%%beginhead 
 
%%file slova.lider
%%parent slova
 
%%url 
 
%%author 
%%author_id 
%%author_url 
 
%%tags 
%%title 
 
%%endhead 
\chapter{Лидер}
\label{sec:slova.lider}

%%%cit
%%%cit_head
%%%cit_pic
%%%cit_text

Вы знаете признаки \emph{настоящего лидера}?  Украины постоянно пытаются его
найти, но каждый раз обжигаются.  Может, если бы знали, было бы легче?  Сегодня
в СЕО Club было крутое обсуждение темы. Каждый, конечно, взял с него своё.  Мои
тезисы. Делюсь:

\begin{itemize}
	\item 1. \emph{Лидер} - не может быть \emph{лидером} без большой идеи
	\item 2. \emph{Лидер} это тот кого выбирают другие что бы идти за ним
	\item 3. \emph{Лидер} - целостная личность
  \item 4. \emph{Лидер} может и должен иметь эффективную коммуникацию и с рабочими и с интеллигенцией и с дипломатами и с военными, и пользоваться их доверием.
  \item 5. \emph{Лидер} притягивает к себе, его харизма завораживает, «соблазняет»
  \item 6. \emph{Лидер} умеет принимать сродные решения (в сложных ситуациях,особенно когда любое решение имеет свои риски)
  \item 7. \emph{Лидер} эффективен в критической ситуации (когда остальные склонны впадать в ступор, панику)
  \item 8. \emph{Лидер} умеет правильно использовать ситуацию
  \item 9. \emph{Лидер} умеет не только выигрывать, но и проигрывать. (Вы помните, проигрывать тоже модно достойно)
\end{itemize}
И важно помнить (\emph{лидеру} и избирателям, что \emph{лидер} это не навсегда))
%%%cit_comment
%%%cit_title
\citTitle{Украина постоянно пытается найти лидера, но каждый раз обжигается}, 
Анатолий Амелин, strana.ua, 01.08.2021
%%%endcit

%%%cit
%%%cit_head
%%%cit_pic
%%%cit_text
Нынешний локдаун помог понять, почему людям нравятся нелепые решения СНБО a la
Daniloff. А заодно стало понятно, почему идея гетьманата - не такая уж
бесплотная.  Пересічним нравится жесткая рука. По нескольким причинам:
- сильная рука дает чувство стабильности, потому что будущее становится якобы
более предсказуемым. Будущее беспокоит всех, главным образом потому, что оно
неизвестно. А сильная рука создают иллюзию, что \emph{лидер} знает, куда идти.
Конечно, это иллюзия - но она работает.  - сильная рука вносит в жизнь смысл.
Раньше я жил просто не пришей рукав, а теперь у меня есть четкий образ врага в
виде сильной руки. Есть враг - есть я.  - сильная рука объединяет. Теперь нам
есть что обсудить с другими пересічними. Нас всех возмущает сильная рука, что
она вытворяет. Мы возмущены таким (незаслуженным) обращением с людьми (с нами).
Мы даже выйдем с плакатами постоять и искренне попротестовать
%%%cit_comment
%%%cit_title
\citTitle{Население Украины по факту подвержено идее гетьманата / Лента соцсетей / Страна}, 
Сергей Лямец, strana.news, 04.11.2021
%%%endcit

%%%cit
%%%cit_head
%%%cit_pic
\ifcmt
  tab_begin cols=4
     pic https://strana.news/img/forall/u/0/92/%D0%B3%D0%B5%D1%80%D0%BC%D0%B0%D0%BD(1).png
  tab_end
\fi
%%%cit_text
Но более интересные вещи она рассказала о переговорах \emph{лидеров} оппозиции
- Яценюка, Кличко и Тягнибока - с Януковичем.  "Они постоянно бывали в
Межигорье и там договаривались. И Порошенко там был, что ни для кого не было
секретом. Был один случай, когда приехал ко мне один из \emph{лидеров}
оппозиции с одним очень богатым человеком. И мы говорили вместе, как размыть
все это противостояние", - делится Анна Герман.  По ее словам, разговаривали с
президентом они поначалу "спокойно и нормально".  И не только с президентом.
"Андрей Петрович Клюев (глава администрации Януковича - Ред.) постоянно был в
контакте и с "Правым сектором", и со "Свободой". И они приходили. Не было
никакого сигнала, что может пролиться кровь".  Но когда первая кровь пролилась,
для Януковича, по словам Герман, это был очень сильный удар. После смерти
майдановца Сергея Нигояна (кто его убил - не установлено до сих пор), Герман и
глава АП Клюев пришли в кабинет к президенту, но его там не было - хотя, если
из приемной приглашали, то Янукович обычно всегда был на месте
%%%cit_comment
%%%cit_title
\citTitle{Как убивали на Майдане и что он дал стране. Воспоминания участников событий 8 лет спустя}, 
Юлия Колтак, strana.news, 21.11.2021
%%%endcit
