%%beginhead 
 
%%file 22_02_2023.fb.fb_group.story_kiev_ua.1.chervona_kalina_v_ri
%%parent 22_02_2023
 
%%url https://www.facebook.com/groups/story.kiev.ua/posts/2145076119022511
 
%%author_id fb_group.story_kiev_ua,zagrebelna_iryna.kyiv
%%date 22_02_2023
 
%%tags isskustvo,kartina,hudozhnik
%%title Червона калина Віри Павленко
 
%%endhead 

\subsection{Червона калина Віри Павленко}
\label{sec:22_02_2023.fb.fb_group.story_kiev_ua.1.chervona_kalina_v_ri}
 
\Purl{https://www.facebook.com/groups/story.kiev.ua/posts/2145076119022511}
\ifcmt
 author_begin
   author_id fb_group.story_kiev_ua,zagrebelna_iryna.kyiv
 author_end
\fi

Відвідали в національному музеї українського народного декоративного мистецтва
виставку «Червона калина Віри Павленко». Вирішила поділитись враженнями.

Знаменита Петриківка...

Це селище міського типу в Дніпропетровській області, відоме своїм народним
мистецтвом - «Петриківським розписом», або просто «петриківкою». 5 грудня
2013 року петриківський розпис було внесено до Репрезентативного списку
нематеріальної культурної спадщини людства ЮНЕСКО. Вважається, що цей вид
мистецтва походить від селянського хатнього стінопису, який був характерний
для багатьох селищ Дніпропетровщини, а також і інших областей України кінця
ХІХ - початку ХХ ст., проте основним осередком  петриківського розпису стала
саме Петриківка.

Роком першої офіційної фіксації цього виду народного мистецтва вважають 1913
рік, коли на виставці в Петербурзі були представлені зразки народної
творчості Катеринославщини, зібрані в експедиціях 1911 і 1913 років. Це
сталось завдяки дослідженням етнографа Дмитра Яворницького. Тоді ж стали
широко відомі імена петриківських майстринь початку ХХ ст.- Тетяни Пати,
Надії Білокінь, Ярини Пилипенко та Параски Павленко. Доробки цих майстринь
народного розпису добре відомі і збережені в кількох музеях нашої країни.

Промислове використання петриківського розпису було вперше налагоджено на
Київській сувенірній фабриці їм.Т.Г.Шевченка. Це сталось внаслідок того, що
протягом 1935-44 років до Києва переїхало кілька петриківських майстринь. Це
сестри Павленко - Ганна і Віра, Пелагея Глущенко, Віра Клименко, Марфа
Тимченко. В Києві ж вперше почали застосовувати петриківський розпис на
порцеляні. Перші такі роботи були створені в 1936 році Ганною Павленко під
час навчання в Київській школі майстрів народного мистецтва. Пізніше ця школа
стала Киівським художньо-промисловим технікумом. Тепер це Київська державна
академія декоративно-прикладного мистецтва і дизайну їм. М.Бойчука. На
порцеляні петриківський розпис отримав значного розвитку, коли в 1944 році
його було широко запроваджено на Киівському експериментальному
кераміко-художньому заводі(КЕКХЗ). Завод існував з 1924 по 2007 рік спочатку
в Києві на вул. Червоноармійській, а з 1973 року переїхав в місто Вишневе. Це
було єдине підприємство з налагодженим масовим виробництвом порцеляни,
прикрашеної петриківським розписом. Товари експортували по всьому світу. Але
в 2007 році, нажаль, припинив своє існування. 

Мистецтво краси справжнього петриківського розпису передавалось у спадок з
покоління в покоління в деяких творчих родинах. Саме про таку родину я хочу
розповісти.

Українська майстриня народного мистецтва Параска Миколаївна Павленко - одна з
найстаріших відомих петриківчанок і одна з найвідоміших майстринь цього
напрямку початку ХХ століття. Маленькою шестирічною дівчинкою Параска
навчилась малюванню від своєї мами Оксани Четверик. Її уява малювала гарні,
пишні, живі квіти, якими декорувались стіни, печі в оселі. Вона оздоблювала
не тільки свою оселю, а й на продаж робила «мальовки», що допомагало годувати
велику сім'ю під час важких 20-30х років. В її доробку безліч колоритних
панно, створених умілими руками майстрині, гармонійні композиції яких
поєднують в собі багатство квіткового різнобарв'я. Починаючи з 1935 року
Параска Миколаівна брала участь в республіканських, а з 1938 року - в
міжнародних виставках. Довгі, але непрості, 102 роки прожила Параска
Павленко. Все було на цьому шляху - війни, голодомор, який вона пережила,
маючи шестеро дітей. Під кінець свого життя переїхала до дочок в Київ. Окрім
того, що майже все життя не випускала з рук пензлика, вона шила, вишивала,
гарно співала, мала гострий розум, ясний зір, добру пам'ять.

Саме Параска Миколаївна навчила двох своїх донечок, Віру і Ганну, малювати.
Мамине навчання і було тією першою школою малювання, тим чистим чарівним
джерелом, з якого дівчата черпали цю чудодійну красу, перетворивши протягом
років простенькі народні мальовки на прекрасні твори мистецтва.

Першою до Києва в 1935 році приїхала п'ятнадцятирічна Ганна. На той час в
Києво-Печерській Лаврі були організовані майстерні, згодом була відкрита
школа майстрів народної творчості. А Ганна Павленко стала однією з перших її
учениць. А у вільний час і сама навчала петриківському розпису інших учнів.
Як розказувала сама Ганна Іванівна Павленко(пізніше за чоловіком Черніченко),
вона плакала, не хотіла сама жити в чужому місті, далеко від мами, від сім'ї.
Але в той час, коли вона вже зібралась їхати додому, тодішній директор музею
українського мистецтва Пімен Михайлович Рудяков не пустив, мовляв, здібна -
треба вчитись, розвиватись. Ця культурна, освічена людина саме в цьому
зіграла неабияку роль в житті юної дівчини. (Хоча самого Пімена Михайловича
спіткала важка доля - сам єврей за національністю, він був пізніше засуджений
за український націоналізм). Довелось залишитись.

В 1937 році до Києва приїжджає старша сестра Ганни - Віра Іванівна Павленко.
Вона вступає до школи народних майстрів, де на той час вже навчалась Ганна
Іванівна. 

Закінчивши школу, сестри з 1944 року почали працювати на КЕКХЗ, вступили до
Спілки художників України. Віра та Ганна працювали в художній лабораторії при
КЕКХЗ, яка стала своєрідним центром національного стилю розпису на фарфорі.
Майстрині оздоблювали сервізи, декоративні тарелі, сувеніри, унікальні вази і
блюда для представницьких подарунків і виставок. Свідченням високої
майстерності художниць є витончена кольорова гама їхніх робіт, відточений
малюнок, віртуозне виконання. Роботи сестер експонувались, окрім вітчизняних,
на виставках далеко за межами України. Це Німеччина, Японія, Італія, Польща,
Франція, Болгарія, Угорщина, Чехословаччина, Бельгія, Канада. Віра і Ганна
плідно працювали в галузі книжкової графіки, створювали розписи для
текстильних виробів. Віра виконувала альфрейно-оформлювальні роботи на
замовлення Управління у справах архітектури УРСР. Ганна розписувала головні
фасади кафе «Літо» і «Весна», що знаходились на території ВДНГ, бібліотеку
національного музею російського мистецтва в Києві. Сестрами розписані вікна в
музеї УНДМ. А велика інтер'єрна ваза заслуженого майстра народної творчості
Ганни Іванівни Черніченко до цього часу прикрашає представництво України в
ООН.

Гідним продовжувачем сімейної традиції в мистецтві стала донька Ганни Іванівни
- Наталія Іванівна Черніченко-Лампека, член Спілки майстрів народного мистецтва
України та Київської організації Спілки художників України. Закінчивши разом з
чоловіком, Лампекою Миколою Геронтійовичем, Львівський інститут прикладного та
декоративного мистецтва, Наталка, як і Микола, працювала на КЕКХЗ. Працюючи
разом, ця пара створила велику кількість прекрасних творів з порцеляни і
кераміки. З-під їхнього «пера» виходили шикарні вази, сервізи та інші вироби,
оздоблені петриківським розписом. В цьому дуеті форма, в-основному, належала
Миколі, а вже естетичний бік, вся краса кінцевого результату в гармонії з цією
формою була створена Наталею, яка з великим натхненням винаходила нові цікаві
елементи розпису, підбирала нові сполучення кольорів.

Наталія Іванівна є авторкою багатьох творів, учасницею всеукраїнських та
міжнародних виставок. Її роботи зберігаються в Національному музеї УНДМ в
Києві, в Чорноморському музеї фарфору, Запорізькому художньому музеї.

Член Спілки дизайнерів України і Спілки художників України, Микола
Геронтійович Лампека, окрім того, що є художником декоративної кераміки, ще є
скульптором і дизайнером. Він бере участь у всеукраїнських і міжнародних
художніх виставках і симпозіумах кам'яної та дерев'яної скульптури. Його
монументально-скульптурні роботи є в багатьох українських містах, а також в
Казахстані, Туреччині, Франції. Зокрема, в Києві цей автор створив моделі
керамічних розеток для Михайлівського Золотоверхого собору, кахлі для печей у
митрополичих палатах на території Софійського собору.

Крім творчої діяльності подружжя Черніченко-Лампеки є успішними в галузі
педагогіки. Наталія - викладач кафедри художньої кераміки Київської державної
академіі декоративно-прикладного мистецтва і дизайну їм. М. Бойчука. Там же
викладав і Микола. А ще він працював в Киівській дитячій академії мистецтв,
Інституті дизайну і ландшафтного мистецтва. Тепер викладає в Національному
транспортному університеті.

Не зрадили сімейної традиції і Іван з Романом - сини подружжя Лампек.
Закінчивши академію їм.М.Бойчука, вони успішно працюють в галузі графічного
дизайну. Може колись продовжить сімейну справу і п‘ятирічний онук Василько.

Зараз в національному музеї УНДМ проходить ретроспективна виставка «Червона
калина Віри Павленко», організована до 110-річчя художниці. Виставкова
експозиція презентує понад 50 творів декоративного розпису та фарфору. Кияни
та гості нашого славного міста можуть до 26 лютого долучитись до споглядання
цієї краси, що є унікальним явищем українського і світового народного
мистецтва. 

Чарівні петриківські візерунки квітнуть на порцелянових сервізах, вазах,
тарелях, панно, рідними українськими орнаментами лягають на шовкові тканини.
На них квітне сама Украіна, її світла, тепла, лагідна, щира різнобарвна душа.

P.S. Навздогін написаному - суттєве уточнення: виставку Віри Павленко
подовжено до 12 березня. Телефонувала до музею. Підтвердили.
