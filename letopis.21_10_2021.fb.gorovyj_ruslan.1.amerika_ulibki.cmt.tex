% vim: keymap=russian-jcukenwin
%%beginhead 
 
%%file 21_10_2021.fb.gorovyj_ruslan.1.amerika_ulibki.cmt
%%parent 21_10_2021.fb.gorovyj_ruslan.1.amerika_ulibki
 
%%url 
 
%%author_id 
%%date 
 
%%tags 
%%title 
 
%%endhead 
\subsubsection{Коментарі}

\begin{itemize} % {
\iusr{Юлія Ткаченко}
З Індії вертаєш - як на похорон. Так в Індії просто в тряпочку завернута людина іде і посміхається, гола-боса. А у нас - всі в пєчалі...

\iusr{Надія Кухар}
Теж це зауважила, коли були з дочкою у Чікаго, це був 2003 рік.Абсолютно незнайомі люди вітаються і бажають гарного дня!!! Це чудово!

\iusr{Соня Поліщук}
Схуд...

\iusr{Roman Nagornyi}

 @igg{fbicon.beaming.face.smiling.eyes} 

\iusr{Валерий Сазонов}
Ой, куме, не кажіть!  @igg{fbicon.frown} 

% -------------------------------------
\ii{fbauth.polozhinskij_aleksandr.ukraina.muzykant}
% -------------------------------------

Заїдь в резервацію - згадай рідну Неньку.

\begin{itemize} % {
\iusr{Олександр Положинський}

\textbf{Nien Thyen} мені норм.

\iusr{Андрей Коваленко}
\textbf{Nien Thyen} дядя,тебе з дурки випустили чи втік?
\end{itemize} % }

\iusr{Tetiana Halych}
я думала мені здається, ці "похмурі ліца" плід уяви... а от і ні

\iusr{Руслан Дацюк}
Це велика і суттєва різниця , так і живем!!!

\iusr{Victoria Konvaliya}

100\%. Причому ладно б іще ту ношу сумно тягли, - та хоч не перетворюювали б цей
край на якийсь безпросвітний бидлятник, де без осатанілих хамів і дня не
перейдеш(


\iusr{Roman Ivanyshyn}
Є таке.
А дома сумують, бо (!)такую страну праcралі і шаряться зараз на залишках совка. А нового нічого не будують.

\iusr{Алла Лебедєва}
То ви мене мало знаєте просто.  @igg{fbicon.face.rolling.eyes} 

\iusr{Людмила Лещук}
Та забань ти вже цього писучого

\begin{itemize} % {
\iusr{Александр Иванов}
\textbf{Liudmyla Gorova} як блоха стрибає, зараза

\iusr{Ruslan Gorovyi}
\textbf{Liudmyla Gorova} вже

\iusr{Ruslan Gorovyi}
\textbf{Liudmyla Gorova} йобнутий якийсь
\end{itemize} % }

\iusr{Demianiw Oksana}

Мені трохи шкода, що все не так райдужно насправді.
Тим не менш, гарного насиченого часу ще в гамериці @igg{fbicon.wink} 

\begin{itemize} % {
\iusr{Ruslan Gorovyi}
\textbf{Demianiw Oksana} я тут 5 раз. Був в 14 штатах. Тож якісь висновки робити можу

\iusr{Demianiw Oksana}
\textbf{Ruslan Gorovyi} я рада, що тобі подобається, справді
\end{itemize} % }

\iusr{Любов Мовлянова}

мене це почало харити з першого відвідування Європи... Приїхала зі стажування у
Франції в 92-му - зайшла в метро і ох4їла від смурних кольорів одягу і
перекошених смурних облич. Все на порівнянні працює, так...


\iusr{Maria Korshunovych}
Це в точку!

\iusr{Роман Гуржій}
Привітність і Приватність

\iusr{Maria Korshunovych}
До речі, Європа в порівнянні зі Штатами, також відчувається «сумною». Особливо східна іі частина.

\begin{itemize} % {
\iusr{Любов Мовлянова}
\textbf{Maria Korshunovych} 

Нагадали... якось в році так... 2005-му, мабуть, їдемо автобусом до Чехії.
Перетинаємо кордон Польща-Чехія, до салону заходить пані з паспортного контролю
збирати паспорти. Я подаю свій паспорт і за звичкою вже посміхаюся зі словами
"прошу, пані!". У відповідь чую російською від суворої чеської митниці: "Что
смешного???" і розумію, що мене, здається, зараз можуть просто вивести з
автобуса... Бо я дуже відрізняюся від загальної маси:(


\iusr{Maria Korshunovych}
\textbf{Любов Мовлянова} 

ото ж.. я нещодавно чере Франкфурт-на-Майні летіла. Там точно не західні німці.
Совок соціаліалістичний ще довго життя паскудити буде.

\end{itemize} % }

\iusr{Лиля Ромакина}

Після річного річного перебування у США по науковій програмі, ми повернулися в
Україну. Поїхали на дачу, йдемо там нашою вулицею. За новою звичкою я йду і
посміхаюся до чоловіка, який порається на своїй ділянці. І тут чую від нього:
"Чого либішся?"  @igg{fbicon.frown} 

Як кажуть - коментів немає


\iusr{Валентина Прокопюк}
В Італії так само - всі посміхаються.

Коли італійці вперше запросили в гості мого чоловіка - питалися, чого це він
завжди нахмурений ходить по вулицях. Що я йому сказала?


\iusr{Deanna Polk}
This makes me happy after all of the ugly stuff we have going on. Thank you my dear friend.

\iusr{Володимир Маслечкін}

Якось читав схожу думку росіянки про Україну (саме Київ). Вона жила в центрі і
була здивована: привітність, посмішки і відкритість людей. Чудово було б, якби
вся Україна так виглядала. На це треба час і пам'ять про те, хто хуйло.)

\iusr{Олександр Чех}

Так, я це помічаю на гейті рейсу назад в Україну - оточення раптом стає
суворішим, напруженішим, з,являються люди, вдягнені па-багатаму) при цьому я
сам напружуюсь, бо чого вони всі такі навколо?))


\iusr{Olha Tsvyntarna}

Ага, а ще компліменти роблять не тому, що хочуть познайомитись, а просто бо так
прийнято робити, коли бачиш у когось щось гарне. Я пробувала так в Україні
робити - люди губляться і не знають що я хочу)))


\iusr{Святослав Воробченко}
Та то ти в мордорі не був. В порівнянні з ним Україна - це Америка.

\iusr{Roman Ivanyshyn}
Вони з садочків вчать посміхатись і вітатись

\iusr{Igor Alksnin}
Ще дякують за все, а не мені всі і все винні.

\iusr{Victor Verba Motanka}
Все у наших руках. Будемо це змінювати!

\begin{itemize} % {
\iusr{Маріна Омелянюк-Сахарук}
\textbf{Victor Verba Motanka}  @igg{fbicon.thumb.up.yellow} 
\end{itemize} % }

\iusr{Оксана Гілета}

Це правда. В Канаді виходиш з автобуса, а водій може тобі сказати "Have a nice
day". Або якось я сіла в автобус, який їхав не по маршруту, а на станцію. Це
була зима і неділя, а в неділю громадський транспорт ходить з великими
перервами. То водій запропонував завезти мене трохи далі, до зупинки, біля якої
є кафе, щоб я не замерзла, чекаючи автобус.

\iusr{Bob Mead}

То ви ж по селах тусуєтесь, тут у людей все добре, то й не напружуються. По
Чікаго прогуляйтеся, та хоч по даунтану - відчуйте різницю. Тіко, від гріха
подалі, не контактуйте очами з бомжами.

\begin{itemize} % {
\iusr{Ruslan Gorovyi}
\textbf{Bob Mead} я по Квінсу ходив, і. Чикаго і в ЛА...
\end{itemize} % }

\iusr{Раїса Бабич}

То ви про совок забули. "Смех без прічіни прізнак дурачіни". Син вернувся із
школи і прямо з порогу: мамо я бачив в трамваї сліпого,що стояв і усміхався!


\iusr{Олександр Книга}
 @igg{fbicon.face.grinning.smiling.eyes}{repeat=3} 

\iusr{Леся Герасименко}
Так, саме це і є велика відмінність між нами, я себе ловлю час від часу на переборі серйозності до цього часу

\iusr{Олександр Кондратюк}
"Мала авіація"? — Кайф!

\iusr{Ната Ш.}
Вєковая скорбь, ага((

\iusr{Наталка Черненко}

Цікаво... приїхала пару днів тому зі Слов'янська... при тому, що в мене в
Слов'янську багато кльових, позитивних людей... реально ми зробили дуже крутий
Фест і є купа результатів... добрих історій... але... є енергетика міста, в якому
була війна, є відчуття, що не всі раді, коли з ними інакші люди... було
відчуття, що ми інакші... коли на приколи, на які в нас реагують ржачем і
влаштовують флешмоби, там обіцяли вивести на фронт і лишити там (на мем про
Бандеру, до прикладу)...коли, як і в моєму рідному Кривому Розі, на
україномовних людей дивляться не завжди норм, часом ржуть: "гиги, цукор..." або
оці вічні: "я не понимаю на украинском"...на щастя, це не серед моїх знайомих
слов'янців, але крім них були ж і інші люди... я приїхала з бажанням з'їздити на
західну, відпочити і прийти до тями... та й таке... може ці всі історії даються
знати, тому в нас так... хз


\iusr{Alena VrediNa}

Понимаю, о чём вы) я предпочитаю Азию для путешествий, и всегда по возвращении
сначала улыбаюсь всем людям, с которыми есть хотя бы визуальный контакт. Потом
через раз, потому что косо смотрят или не правильно истолковывают.


\iusr{Оксана Левченко}

Зате, тих, що подорожують ( і не лише до Єгипту @igg{fbicon.face.grinning.sweat} )у нас одразу видно. Бо усміхнені
Значить свої @igg{fbicon.face.upside.down} 

\iusr{Prymara Makeeva}
Носити посмішку серед наших людей - ще та пригода @igg{fbicon.wink} . Утім, цікаво @igg{fbicon.face.upside.down} 

\iusr{Ирина Устим}
Что ты мелешь, " "амэрыканэць" улыбчивый?
Люди везде разные.
Кто тебе виноват, что ты в своём болоте, видишь только дерьмо.Но чужим ты гордишься, как своим.
Стыдно за таких " амэрыканцив" @igg{fbicon.laugh.rolling.floor}{repeat=5} 

\begin{itemize} % {
\iusr{Ruslan Gorovyi}
\textbf{Ирина Устим} Боже, звідки тут така вонючка? Причом пише з Ізраїля. Ви хоч би шось почитали про людину пост якої коментуєте...
\end{itemize} % }

\iusr{Зоя Четверикова}
Маю надію, що ті, хто ходить зараз до дитсадка вже будуть посміхатись на вулиці.

\iusr{Юля Толоконникова}
Тут ви праві на 100%

\iusr{Михайло Думяк}
В західні Європі так само, якщо ваші погляди пересіклися, обов'язково усміхнуться і привітаються!)

\iusr{Микола Мудроха}

так само кажуть білоруси коли приїхають в Україну!
Таке ж враження від України після Європи.
Але все буде добре!  @igg{fbicon.wink} 

\iusr{Оленка Краснокутська}
У нас президент Зеленський. Нам не до жартів, він жартує за всіх @igg{fbicon.face.rolling.eyes}  @igg{fbicon.face.tears.of.joy} 

\iusr{Юлия Панюта}
Аж за душу взяло! Це перші добрі слова в адресу сучасної молоді, за останні роки!!!

\iusr{Viktoriia Berezka}
Тому ж є об'єктивна причина. У нас люди, переважно, замучені життям та безнадією.

\begin{itemize} % {
\iusr{Олена Грінцова}
\textbf{Viktoriia Berezka} вы думаете ТАМ все живут без тех же проблем? Это менталитет наш! Чтобы не происходило в жизни украинец всегда не доволен

\iusr{Ruslan Gorovyi}
\textbf{Viktoriia Berezka} немає ніякої об’єктивної причини... це ментальне відлуння совка - сидіти тихо, мовчати і чекати коли за тобою прийдуть

\iusr{Viktoriia Berezka}
\textbf{Олена Грінцова} не згодна категорично.

\iusr{Viktoriia Berezka}
\textbf{Ruslan Gorovyi} звичайно, вам не зрозуміти, як це бути хворим, зневіреним і існувати на три тисячі на місяць.
\end{itemize} % }

\iusr{Олена Грінцова}

Как я тебя понимаю. И начинаешь это чувствовать уже внутри самолёта, а особо
когда проходишь паспортный контроль... и лица у всех пограничников: "всем
бояться". Ни тебе приветствия, ни драсте

\iusr{Павел Больба}
Видатні досягнення совкової стоматології призвели до зникнення посмішки. Бо нащо ото лякати перехожих.

\iusr{Oleksandr Niushko}
 @igg{fbicon.100.percent} 

\iusr{Ольга Кущенко}
Прочитала коментарі. Люди, а до вас варто посміхатися? До переважної більшості? Ви не совок?

\iusr{Жадана Солодка}
Точно так і є. Була, бачила.

\iusr{Bodnar Olena}
Знаючи хто хуйло, не усміхнешся


\end{itemize} % }
