% vim: keymap=russian-jcukenwin
%%beginhead 
 
%%file 06_12_2021.fb.lesev_igor.1.a_esli_russkie_taki_napadut.cmt
%%parent 06_12_2021.fb.lesev_igor.1.a_esli_russkie_taki_napadut
 
%%url 
 
%%author_id 
%%date 
 
%%tags 
%%title 
 
%%endhead 
\subsubsection{Коментарі}
\label{sec:06_12_2021.fb.lesev_igor.1.a_esli_russkie_taki_napadut.cmt}

\begin{itemize} % {
\iusr{Oleg Voloshin}

Если честно, как-то очень путано вышло. И причём тут Крым? Никого он реально не
волнует. Тех, кто ненавидел Россию до 2014, года не удовлетворил бы его
возврат, а остальным плевать на самом деле.

И вот то, что под боком у великой державы уже 8 лет существует слабая страна,
единственной целью которой, судя по действиям ее властей, является нанесение
всяческого вреда этой великой державе, то вот и единственный логичный мотив
разрубить Гордиев узел. Военное освоение территории Украины - причина ударить
на упреждение. Никто не станет ждать, пока здесь реально появятся американские
и британские базы. Если гарантировать России отказ от таких планов, у Москвы
исчезнет окончательно и без того весьма слабое желание трогать это болото. Все
равно Украина обречена на угасание.

\begin{itemize} % {
\iusr{Людмила Левченко}
\textbf{Oleg Voloshin} 

А зачем отказываться? Чем они мешают \enquote{великой державе} с бедным народом?
Нападут? Не смешите мои тапки, сама Россия в это не верит!

\iusr{Oleg Voloshin}

\textbf{Людмила Левченко} 

тянут сюда военных НАТО, все время грозят начать наступление на ДНР,
провоцируют учения возле Крыма, разместили здесь всю возможную разведывательную
инфраструктуру США и Британии и тд. Этого мало? В России угрозу использования
территории Украины для создания давления на РФ рассматривают более чем
серьезно. Поставить ракеты с подлётным временем до Кремля 8 мин и потом можно
прямо диктовать России условия. Военные мыслят не намерениями, а потенциалом.
Если Вы не враги, то зачем тащите под наши границы свою военную инфраструктуру?
Вот их вопрос американцам. Представьте российские ракеты и базы на Кубе. Уже
ведь был Карибский кризис, который едва не завершился ядерной войной.

\end{itemize} % }

\iusr{Алексей Корецкий}

Пока есть реальная надежда и перспектива развалить Украину изнутри, используя
любые способы внешнего и внутреннего давления, никакой "большой войны" не
будет. Это намного дешевле, а главное в разы менее рискованно и предсказуемо,
чем масштабные военные действия.

Для этого нужно элементарно понимать хотя-бы основные цели РФ. А то не
экономические или территориальные интересы, а прежде всего контроль и влияние,
война за идентичность.

Плюс любая крупная военная операция всегда рискует затянуться на многие месяцы
или годы, став вместо блицкрига вялотекущим перманентным конфликтом размазанным
во времени и по территории. Чему гарантировано будут всячески способствовать
другие внешние интересанты. Поскольку любые проблемы гуманитарного и
экономического характера, как и военные потери и соответственно ослабление РФ,
тоже выгодны для целого ряда 3-х государств. Начиная от непосредственно соседей
вроде Китая или Турции и далее по списку ...

\begin{itemize} % {
\iusr{Владимир Очеретный}
\textbf{Алексей Корецкий} 

России в принципе не нужен развал Украины. Другое дело, что в нынешних границах
она неминуемо превращается в Бандеростан - кто бы ни приходил к власти. Думаю,
РФ было бы выгодно, если бы Польша, Румыния и Венгрия забрали те территории,
которые считают исторически своими. Тогда из остатков можно было бы делать,
если не пророссийскую, то хотя бы не враждебную Украину.

\end{itemize} % }

\iusr{Arina Levina}

Если Россия перекроет газ в ЕС, то примерно 40\% европейской индустрии и
домохозяйств ожидает коллапс. Замены нет и в ближайшем будущем не будет.
Разразится кризис какого ещё не было в истории Европы

\begin{itemize} % {
\iusr{Алексей Корецкий}
\textbf{Arina Levina}

Вы хоть представляете о чем вы пишите, понимаете глобальность проблем которые
хотите охватить?

Масштабный военный конфликт между РФ и Украиной это сотни тысяч комбатантов и
резервистов ввлечённые в боевые действия, сотни, а возможно тысячи километров
линии боевых действий. Массированные удары артиллерии, РСЗО, авиации по
множеству целей ... Сотни тысяч, как минимум беженцев в сопредельные Украине
страны Европы ... И не пойми какие ещё гуманитарные последствия масштабов 35
миллионный страны. Плюс экономический коллапс и падение на многих мировых
финансовых площадках.

Тут газ для европейских домхозяйств это даже не номер 5 проблема, с которыми
Европе придётся столкнуться.


\iusr{Ольга Циброва}
\textbf{Arina Levina} 

У 2009-му вже пробували перекрить, але коли Тимошенко обрисувала Путіну
масштаби техногенної катастрофи, яка накриє не лише Україну, а ще і Європу з
Росією, то негайно були відкриті вентилі і Путін пішов на перемовини.

\emph{Владимир Очеретный}
\textbf{Алексей Корецкий} 

Войны проходят по разным сценариям. Достаточно отправить руководство ВСУ на
бессрочную консультацию с Мазепой (ракетой по Генштабу), подавить ПВО и систему
связи, и комбатанты остаются без управления. После чего одна фраза "Россия
начала вторжение" оставят украинские позиции пустыми. А вот дальше события
могут развиваться по-разному. Не исключено, что в Харькове и Одессе начнут
резать бандеровцев и объявлять о своей независимости. Но этого может и не
произойти. Короче, куча вариантов, но для полной оккупации Украины России нужна
полноценная мобилизация, что вызовет большое неудовольствие граждан РФ и
приведёт к резкому ухудшению экономической ситуации. Поэтому в случае
серьёзного обострения ситуации на Донбассе Россия, имхо, снова займётся
организацией котлов.

\iusr{Ирина Папакина}
\textbf{Ольга Циброва}
Боже, вот она - параллельная реальность!

\iusr{Ольга Циброва}
\textbf{Ирина Папакина} Віднайдіть в інеті статтю Михайла Саратова на цю тему і ви зрозумієте, про яку реальність тоді йшла мова.

\iusr{Ирина Папакина}
\textbf{Ольга Циброва} От того, что в параллельной вселенной не один человек, а несколько, она не перестает быть параллельной)
\end{itemize} % }

\iusr{Mila Bondar}

Ну да. Если есть что то, что можно присунуть, то рано или поздно это и
случится. Во всяком случае, планы такие всегда будут. И причина тупая и
простая, восстановление того, что было до 91-го. Потому что обманули.... @igg{fbicon.face.downcast.sweat} . Зря
что ли, про вкусный пломбир столько пишется?

\iusr{Serge Pavlovsky}

Вся эта истерия вокруг какойтотам маштабной воойны на фоне коллапса экономики,
медицины, энергетики и прчего- закончится банальным - дайте денег. На надувные
лодки, ледоколы, айфоны, джавелины и прочий мотлох.

\iusr{Михаил Омский}
С возвращением в ФБ  @igg{fbicon.smile} 

\iusr{Любовь Чуб}
Ну, а варианты того, зачем им нужна джинса?

\begin{itemize} % {
\iusr{Матвей Кублицкий}

Тут как раз просто. Для НАТО данная угроза с востока - условие выживания. Не
будь этой мифической угрозы и у многих возникнет закономерный вопрос: а зачем
тогда вообще нужно это НАТО? Украина - лишь одна из серий по общему обоснованию
существования НАТО)))))

\iusr{Геннадий Шевчук}

Посеять тревогу у граждан дабы не задавали вопросов президенту о
экономике - \enquote{Враг у ворот} - закручиваем гайки, затягиваем пояса и т.д

\iusr{Ирина Папакина}
Думаю, прочтение «1984» даст ответ на этот вопрос)

\iusr{Дмитрий Воронин}
Для обоснования \enquote{борьбы}

\iusr{Любовь Чуб}
Вы все по-своему правы.
Но я спрашивала именно о конкретной ситуации: каковы составляющие этой истерики, льющейся из всех утюгов? Что за этим стоит? Что последует? Или не последует...

\iusr{Дмитрий Воронин}
\enquote{Война нервов}, до ошибки оппонента
\end{itemize} % }

\iusr{Юрий Зыбин}

Солидарен с Олегом Волошиным что немного путано изображено Таки Чернигов
немного вскружил голову Игорю и он немного расслабился Ну я полагаю Игорёк в
скором будущем восстановит форму и в дальнейшем будет таким же зубастым и
искромётным как был раньше Теперь по теме Для нашего северного соседа Украина
пока не представляет никакой угрозы если только с нашей стороны не будет каких
либо провокаций и реального появления инобаз с ракетами В моём комментарии на
предыдущую статью Игоря я прогнозировал что сильным ходом путина могло быть
создание блока россии китая и латиноамериканских стран в пику AURUSа с
последующими учениями в Мексиканском заливе Думаю что это еще впереди А
настоящим сюрпризом и сильнейшим ходом путина стало сегодняшнее соглашение с
Индией Это может означать полный отрыв её от США А это полностью меняет мировую
расстановку сил Так что можем расслабиться Никаких нападений не предвидится А
завтра за нас всё решат И предполагаю к сожалению всё будет не в нашу пользу С
уважением к Игорю Юрий

\iusr{сергей зиновьев}
А как насчет Тайваня ? Угроза нарадения для прикрытия операции Китая ?

\iusr{Alex Meleshko}

\ifcmt
  ig https://i2.paste.pics/a2d8612aaf8b5aaca1cc2fac2620426e.png
  @width 0.4
\fi

\iusr{Natalia Mieshkova}
До 2013года мечтали о ,, маленькiй i красивой Украiне...,,

\iusr{Геннадий Шевчук}

Все проще- кому нужны бедные родственники, завоюешь нас, а потом еще и кормить
надо, долги выплачивать. Пусть теперь содержат те кто загнал нас в эти долги @igg{fbicon.face.wink.tongue} 

\iusr{Евгений Отовчиц}

Я думаю, что вторжение в Украину напрямую зависит от СП-2. Если его таки
стопорнут или, ещё лучше(хуже) порежут на металлолом, то аннексия Украины и её
ГТС - совершенно логичный ответ. А если СП-2 сертифицируют и запустят - на
Украину станет наплевать всем, включая наших западных "партнёров".

Санкции никого не пугают. Европа может отказаться от русского газа и покупать
словацкий, как Украина. Торговая блокада тоже не может быть единой и полной,
Германия не откажется от России, что уж говорить о всяких Грециях и Италиях.
Турция тоже будет торговать. Ну и т.д.

А вот позитивное решение "украинского вопроса" (прекращение войны и
стабилизация региона) станет сильным козырем со стороны РФ.  Вопрос же только в
деньгах и энергетике. По какой трубе качаем и как ограничить Газпром?

\begin{itemize} % {
\iusr{Матвей Кублицкий}

никакие плюшки отвладения укрогтс не перекроют финансовые потери от этой войны.
СП-2 - если что - не проект одного газпрома. В нем заинтересованы все ведущие
газовые игроки ЕС. На самом деле борьба за СП-2 - это борьба Германии за свою
независимость от США


\iusr{Евгений Отовчиц}
\textbf{Матвей Кублицкий} 

совершенно верно. Но ведь никто не мешает РФ кулуарно объяснять западным
партнёрам, что транзита через Украину не будет, финансировать за счёт РФ
украинский режим никто не будет, СП-2 - решение проблем, зарежете СП-2 -
зарежем "проевропейский" режим в Украине?

Это торги. Создание сильной переговорной позиции. В конце концов немцы победят,
я думаю.


\iusr{Матвей Кублицкий}
\textbf{Евгений Отовчиц} 

я лишь о том что сертификация СП-2 вообще никак не влияет на возможность войны
на Украине. Она - война - просто не нужна России от слова "вообще". В ней нет
никакого смысла

\iusr{Евгений Отовчиц}
\textbf{Матвей Кублицкий} 

о войне речь не идёт. Речь о силовом вмешательстве в управление территорией.

Запад (США) хочет продолжения транзита газа через Украину и, таким образом,
сохранение влияния на ЕС в энергетике. Германия и Россия хотят зарабатывать без
оглядки на США. Поэтому если СП-2 грохнут - вторжение в Украину и подчинение
ГТС в управление РФ - правильный и логичный ответ с точки зрения геополитики.
США не нужно вторжение РФ в Украину. США нужен антироссийский режим в Украине и
блокировка СП-2. Россия же такой возможности не допускает - либо запуск СП-2,
либо конец проекта \enquote{салорейх}. Аргументация в диалоге: зажмёте СП-2 - салорейху
крышка. Возьмём под контроль украинскую ГТС.

Потому что с запуском СП-2 в украинской ГТС нет никакого смысла. Никто её не
будет содержать, эксплуатировать и модернизировать. США же хочет управлять
украинской трубой, но чтобы затраты на украинскую трубу и власти несла Россия,
а с СП-2 этого не будет.

Но так или иначе, вопрос скоро разрешится. Если СП-2 запустят - любое вторжение
в Украину потеряет смысл. Потому что потеряет смысл украинская труба. Всем
будет плевать, что здесь происходит. А денег от транзита на вооружение ВСУ
больше не будет. Украина превратится в оболочку государства, имитацию, без
какого-либо наполнения.

\iusr{Геннадий Гаврилов}
\textbf{Евгений Отовчиц} зеленая плесень-вот и все наполнение

\iusr{Евгений Отовчиц}
\textbf{Геннадий Гаврилов} ...и пара провинциальных, хотя и талантливых, олигархов)

\iusr{Ник Викторов}
Мнение (по поводу \enquote{адских} санкций):
\url{https://cont.ws/@waif17/2151250}

\iusr{Владимир Очеретный}
\textbf{Евгений Отовчиц} 

Зачем СП-2 резать на металлолом? Лежит в море, никому не мешает. По деньгам он
уже окупился. Отказаться от российского газа Европа не может - иначе получит
настолько дорогой килоВатт, что её экономика станет полностью
неконкурентноспособной.

\end{itemize} % }

\iusr{Павел Науменко}
Как всегда, блестяще!

\iusr{Александр Белинский}
Саддам влез в Кувейт!!!

\begin{itemize} % {
\iusr{Игорь Лесев}
\textbf{Александр Белинский} да, опечатался, спс

\iusr{Александр Белинский}
\textbf{Игорь Лесев} 

А визг про вторжение может иметь две причины:

1. Приближение свержения Зеленского. После предыдущего свержения президента
отвалился Крым.

2. Накануне пика кризиса 2008 года разжигали войну 08.08.08. Накануне
следующего кризиса разожгут тщательнее...

\end{itemize} % }

\iusr{Oleksandr Novokhatskyi}

Мне всё понравилось. Отличный ракурс подачи, "неэкспертно-жизненный", без
сложностей в бытовом русле логики течения жизни. Спасибо.


\iusr{Ирена Березник}

Кстати, кое-кто считает это дымовой завесой для принятия закона о двойном
гражданстве и скупке земель теми же мистерами украинского происхождения.


\iusr{Ирена Березник}

Да по-моему, в этот бред не верят даже упоротые. Просто еще один водевиль от
известного квартала. Кстати, срежиссированный из-за колобани.


\iusr{Нафаня Озеров}
Бжезинский когда то сформулировал. \enquote{Россия без Украины никогда не будет супердержавой}.
Но тех несколько сот тысяч, хватит чтоб разгромить ВСУ, но точно не хватит для оккупации/контроля за такими территориями.
Так что это превентивная мера

\begin{itemize} % {
\iusr{Владимир Очеретный}
\textbf{Нафаня Озеров} А почему следует считать мнение Бжезинского правильным? Его прогнозы из \enquote{Великой шахматной доски}, мягко говоря, не сбылись. Россия уже супердержава - без Украины.

\iusr{Нафаня Озеров}
\textbf{Владимир Очеретный} нет. Не тянет. Ни по экономике, ни по людям. Только что ЯО и осталось

\iusr{Владимир Очеретный}
\textbf{Нафаня Озеров} А с Украиной и экономика станет ого-го и количество населения взлетит до небес?

\iusr{Нафаня Озеров}
Владимир Очеретный да

\iusr{Олег Новиков}
\textbf{Нафаня Озеров} а можно обрисовать механику этого превращения? Как оно будет с Украиной-то?

\iusr{Нафаня Озеров}
\textbf{Олег Новиков} да скорее всего никак)
В общем с автором согласен.
Мог бы быть Союз. Тот же ОДКБ, ЕАЭС
Но не задалось

\end{itemize} % }




\end{itemize} % }
