% vim: keymap=russian-jcukenwin
%%beginhead 
 
%%file 11_05_2021.fb.fb_group.story_kiev_ua.1.hochu_baobabov.cmt
%%parent 11_05_2021.fb.fb_group.story_kiev_ua.1.hochu_baobabov
 
%%url 
 
%%author_id 
%%date 
 
%%tags 
%%title 
 
%%endhead 
\zzSecCmt

\begin{itemize} % {
\iusr{Лианна Каушанская}

по-киевски??? не далее чем вот всю неделю прошедшую, в Одессе обкорнали
Соборную площадь - вот те самые люди с мотокосами, бухие, в говняно-зеленых
робах и с надписью Зелентрест на спинах... будь они трижды прокляты)

\begin{itemize} % {
\iusr{Виктория Угрюмова}
\textbf{Лианна Каушанская} как жаль, такое впечатление, что зелень реально мешает дышать каким-то космитам. Сегодня вышли - воздух уже пыльный и весной не пахнет

\iusr{Лианна Каушанская}
\textbf{Виктория Угрюмова} вот, вот.
\end{itemize} % }

\iusr{Игорь П.}
Да, похоже озеленители взяли на вооружение тактику выжженой земли(

\iusr{Людмила Полевичек}
БРАВО!!!

\iusr{Ирина Жадан}

Да-с! Было бы смешно, если бы не было так грустно... Вот сейчас у нас под
окнами орет эта мотокоса и сделать ничего нельзя, ответ один-нас заставляют. А
то, что мы платим за придомовую территорию, никого не волнует. Я сама говорила
в ЖЕКе, что это наш двор, не трогайте, а лучше убирайте подъезды, но никто
никого не слышит!

\begin{itemize} % {
\iusr{Виктория Угрюмова}
\textbf{Ирина Жадан} 

спрашивала сто раз устно и письменно - вот вы все скосили - а где посеять
траву? А у нас нет денег. А за что мы платим? за покос, а можно не косить - и
не платить? нельзя. А можно платить, чтобы не косили? нельзя.Бред сивой кобылы.
Мама переписывалась с ними - на 17 кнг несколько папок переписки. Завершилась
эта история с маминой смертью


\iusr{Ирина Жадан}
Да, это ежегодный кошмар. Деревья, цветы, траву никто поливать не хочет, только уничтожать.

\iusr{Виктория Угрюмова}
Ирина Жадан увы
\end{itemize} % }

\iusr{Валентин Стадник}

Виктория єто ужасно. Научитесь таки любить людей. Ви знаете у меня тоже есть
мотокоса. Живу в селе. Тоже вот покашиваю, как ви пописиваете. А Киев засрали не
озеленители, а жлоби с баблом. А у вас мужик с мотокосой источник всех
бед. Охренеть.

\begin{itemize} % {
\iusr{Ирина Жадан}
Да! Потому, что выкашивают до земли и трава потом даже расти не может! Вы же, наверняка, так не косите у себя во дворе.

\begin{itemize} % {
\iusr{Валентин Стадник}
\textbf{Ирина Жадан} 

Кошу. За сезон 3 раза. Не будишь косить может случится пожар. Трава сразу красивая
и зеленая, а потом она сохнет и становится причиной пожара, хати горят
леса. Девчата ви с какой планети?


\iusr{Валентин Стадник}
\textbf{Ирина Жадан} Просто нужно поливать. Как вопрос другой.

\iusr{Виктория Угрюмова}
\textbf{Валентин Стадник} чтобы поливать цветы, нуджно, чтобы этот прекрасный озеленитель, которого вы защищаете, не срезал незабудки и нарциссы

\iusr{Валентин Стадник}
\textbf{Виктория Угрюмова} Кто єтот озеленитель? Слишком обще. У нас бардак. Єто знают все. Б ете, бейте в 10-ку.
\end{itemize} % }

\iusr{Ирина Жадан}

Валентин, у нас была дача и был прекрасный газон. Так мы никогда не косили
траву до самой земли, всегда немного оставляли и бык красивый, зеленый двор. И

\begin{itemize} % {
\iusr{Валентин Стадник}
\textbf{Ирина Жадан} 

Ирина, єто дача, 6 соток. Не Киев и даже не мои 60 соток. Там проще, на даче. Просто
я в шоке, когда говорят что мужик с мотокосой символ проблем. Косить и поливать.

\end{itemize} % }

\iusr{Марина Лабунец}
\textbf{Валентин Стадник}! 

Киев ... (испортили) власти, которые поставили непомерно высокие цены на
отопление и воду, и, понимая, что люди не в состоянии больше платить, исключили
из тарифов на обслуживание другие, как им кажется, второстепенные вещи: полив,
уход за зелеными насаждениями и т. д. У ЖЭКов остался только вывоз мусора (что,
к счастью, еще делается) и уборка двора и парадного (на бумаге, но не на деле).

\begin{itemize} % {
\iusr{Валентин Стадник}
\textbf{Марина Лабунец} 

Власти, они же и есть жлоби з баблом. Застройщик, ему на Киев
наплевать. Прибиль. Маркса учили? Он все популярно об яснил. 300\% прибили и мать
родную продадут.

\end{itemize} % }

\iusr{Ирина Жадан}
Да нет! Мужик-исполнитель, но ведь можно же хоть немного зелени оставлять. Тех
же одуванчиков, пока они цветут, красиво ведь!


\iusr{Валентин Стадник}
Переезжайте в село у нас одуванчиков и всей прочей красоти полно.
.

\iusr{Ирина Жадан}

Так это ведь никто и не делает! Даже те цветочки, которые мы же сами и
посадили! А ведь высадка цветов возле домов и уход за ними входит в оплату
придомовой территории.

\iusr{Валентин Стадник}
\textbf{Ирина Жадан} А где ви видите порядок? у нас все так. Бардак везде.

\iusr{Ирина Жадан}
Таки да!

\iusr{Валентин Стадник}
А киевлянину?

\iusr{Александр Митряев}
Из одуванчиков можно делать салат .... В СССР жрать нечего было, вот и делали ....

\iusr{Валентин Стадник}
\textbf{Aleksandr Mitryaev} Кстати, говорят полезная штука єтот салат.

\iusr{Виктория Угрюмова}
\textbf{Валентин Стадник} 

Скажите, вы цветущие вишни у себя в саду тоже пилите каждую весну? И яблони? И
абрикосы срубаете - молодые, не старые? Липы уничтожаете? А жасмин вырубили? А
сирень? нет - это не мы охренели. Посмотрите, что творится на наших глинобитных
утоптанных газонах. Мужчжик с мотокосой в середине фераля - не хотите? А в
ноябре - в 6 утра, под окнами, когда о траве уже никто не вспоминает.юЛистья
желтые над городоом кружатся

\begin{itemize} % {
\iusr{Валентин Стадник}
\textbf{Виктория Угрюмова} 

Виктория, я вообще то о другом. Проблема есть, не спорю. Но Ви как то не туда
бьете. И не по тем. И что єто не очень люблю людей. Не любите тихо.


\iusr{Виктория Угрюмова}
\textbf{Валентин Стадник} спасибо, постараюсь

\iusr{Валентин Стадник}
\textbf{Виктория Угрюмова} Вам спасибо за понимание.
\end{itemize} % }

\iusr{Александр Митряев}
Полезно было пиво со сметаной, а сейчас уже ничего не помогает ....

\iusr{Валентин Стадник}
\textbf{Aleksandr Mitryaev} Не поспоришь тут.

\iusr{Александр Митряев}
Тюльпагы полезно срезать, тогда они становятся красными и стоят долго ...

\end{itemize} % }

\iusr{Оксана Ходорич}

Мужик с мотокосой - это и есть символ уничтожения самозабвенно прекрасного в
весенней зелени Киева. Вика права, я начинаю тоже ненавидеть Маленького принца,
который выдирал ростки баобабов, хотя у него были причины хотя бы: его планета
чересчур маленькая, и корни деревьев могли её разорвать. Но цветы, кусты,
деревья точно не разорвут Землю, скорее рачительное хозяйствование, приводящее
к выжжженой земле, образованию всё большего количества пустынь, уничтожит её. А
город, конечно, должен быть зеленым

\begin{itemize} % {
\iusr{Виктория Угрюмова}
\textbf{Оксана Ходорич} 

я не ненавижу маленького принца, и даже людей с мотокосой. Но любить их
решительно отказываюсь, ибо не вижу в них ни разума, ницели, ни совести
\end{itemize} % }

\iusr{Леся Фандралюк}

У меня за окном деревья, с кронами -пеньками((( кому они мешали???! (( И это со
стороны двора. А со стороны площади и проспекта - почти все уничтожено((


\iusr{Ирина Жадан}
И это действительно страшно!

\iusr{Марина Лабунец}

Замечательно, Виктория! И главное, разумная идея: засадить все баобабами, и
делу конец. А так: сажай, ухаживай за зелеными насаждениями - долго и моторшно.
Вспомните, какие раньше были ухоженные дворы: дворники их ежедневно их
поливали, стригли кусты (не врачу же или учителю это делать вместо своей
работы!). А что касается уборки, наша дворничка Люба как поставила совок с
метлой между вторым и третьим этажами год назад в начале карантина, так они и
стоят там до сих пор: самообслуживание.

\begin{itemize} % {
\iusr{Валентин Стадник}
\textbf{Марина Лабунец} А кто будет платить за воду на полив. А раньше и медицина била бесплатной и не только. И ви забили что машин во дворах раньше не било, когда дворники поливали кусти. У нас кто любит свой город поливают кусти, деревья возле дома из своих кранов.

\begin{itemize} % {
\iusr{Виктория Угрюмова}
\textbf{Валентин Стадник} я поливала, носила с 9 этажа, пока было что поливать

\iusr{Валентин Стадник}
\textbf{Виктория Угрюмова} 

Не забивайте, уровень грунтових вод упал. Колодец два раза углублял. Все
углубляли. Осадков мало. Полив хорош когда перидически идут дожди. Дефицит води в
криму и Донбассе бил при Союзе еще. Радуемся что в кранах есть вода А войни за
воду уже ведутся и далі буде.

\end{itemize} % }

\iusr{Ирина Жадан}
В каждом ЖЕКе заложена определенная сумма именно на полив дворов. Но заставить дворника делать ее работу невозможно!

\iusr{Марина Лабунец}
\textbf{Валентин Стадник}! Ну, так давайте, ни за что не платить: ни за полив, ни за вывоз мусора и т.д. На то и власти, чтобы все продумать. А насчет машин: разве им место во дворах? Вы где-нибудь видели машины, хаотично стоящие во дворах Европы, в которую мы так стремимся?

\begin{itemize} % {
\iusr{Валентин Стадник}
\textbf{Марина Лабунец} Я о том что ситуция изменилась. А не о том что ви мне вменяете. Вообще я живу в селе чего и вам желаю.
\end{itemize} % }

\iusr{Галина Гурьева}

У нас дворник выходил в 7. 00 и в 10. 00 его уже не было(((Зато стояли совок и
веник, тележка для мусора в разных концах двора. Она же работает!!! Непонятно
только где?!


\iusr{Марина Лабунец}
\textbf{Галина Гурьева}! Вспомнила, что у нас на работе был один начальник. Он приходил
утром (всегда вовремя!), вешал на стул свой пиджак и ставил портфель рядом со
столом. И дальше его днем с огнем было не найти. Когда спрашивали, где он,
всегда было понятно, что на работе. Но где он - никто не знал; и придраться
было не к чему: рабочее место не пустое. В самом конце рабочего дня он
появлялся и забирал свои вещи И так - изо дня в день.


\iusr{Гриша Григорий Панаид}
\textbf{Марина Лабунец} наш человек!!!

\end{itemize} % }

\iusr{Anežka Travnitsa}

Поражаюсь энтузиазму дворников, но от них требуют косить газон в связи законом,
который гласит, что нужно косить в целях улучшения санитарной ситуации, так как
в траве размножаются опасные клещи и богзнаетчто. И начинать что то менять
нужно с требования от депутатов отменить этот закон.

\begin{itemize} % {
\iusr{Anežka Travnitsa}
Или по крайней мере косить только раз в год осенью

\iusr{Марина Лабунец}

Если не будут косить газоны, то при нашем менталитете и качестве уборки все
будет завалено мусором, который будет какое-то время невидимым из-за высокой
травы.


\iusr{Виктория Угрюмова}
\textbf{Марина Лабунец} понимаете, можно косить и поливать - в Киеве никогда не было заросших газонов, а можно выкашивать. Зачем передергивать?
\end{itemize} % }

\iusr{Марина Прокопенко}
Рассказ просто блеск!!!

Очень весело...

@igg{fbicon.laugh.rolling.floor}{repeat=3} 

И по существу...

@igg{fbicon.face.happy.two.hands}{repeat=3} 

А про баобабы...

Несколько месяцев назад пыталась вырастить из подаренной косточки...

Мучалась-мучалась...

Не вышло...

@igg{fbicon.face.rolling.eyes}{repeat=3} 

Очень уж хотелось баобабов...

@igg{fbicon.face.smiling.eyes.smiling}{repeat=3} 

\iusr{Vadim Vadim}
Шо дети малые))
В мотокосу нужно заливать бензин, а от простой косы начальнику ЖЭКА надоя нету))!

\iusr{Татьяна Длигач}

Ещё один ужас ужасный-обрезка деревьев. У меня закрадываетс- сомнение-не хотят
ли работники зеленстороя просто уничтожить деревья, чтоб было поменьше работы
дворникам. @igg{fbicon.anger} 


\iusr{Ирина Жадан}
Даже не сомневайтесь, так и есть!

\iusr{Vadim Basovskiy}
Секрет английского газона: триста лет стричь и поливать. Ничего, со временем
всё наладится, всего двести семьдесят лет осталось.


\iusr{Ирина Жадан}
Согласна! Мы подождем!

\iusr{Ирина Иванченко}
Немедленно провести флешмоб\#Даешь киевлянке по баобабу!" как средство
озеленения озеленителей!

\begin{itemize} % {
\iusr{Игорь П.}
\textbf{Ирина Иванченко} 

А чего только киевлянке?( Я псу своему вечно обещаю - купим тебе баобаб,
посадим посреди хаты, чтоб ты хозяев в дурную погоду на улицу не вытаскивал.


\iusr{Ирина Иванченко}
\textbf{Игорь П.}, в таком случае, ДВА флешмоба, собак обижать нельзя!

\iusr{Игорь П.}
\textbf{Ирина Иванченко} Немецкая морда передает Спасибо!
\end{itemize} % }

\iusr{Oksana Romanova}

Прекрасный текст и позиция, спасибо. Киев стал обездоленным, город с такой
огромной и полноводной рекой - гибнет от жары и идиотизма каждый год. Сейчас
просматриваются виды, который раньше никто и не знал- нет деревьев, газонов,
скверов. Все забито в Фэм Полностью уничтожена система поливов и вообще мытья
города( ни поливальных машин, ни людей со шлангами. Даже старые деревья
оконтовавыют дурноватыми бровками торчащими на 20см и дождевая вода мчится мимо
создавая реки и потопы. А газоны - да , это верх жлобства. Не бывает пожаров на
3 день тепла в 12 градусов, какой смысл уничтожения? - веселее раздать косилки
чем полить город водой.

\iusr{Мирослава Кизилова}
Рассказ - блеск!
Великолепно!
Завтра прочту своим студентам - будущим ландшафтным дизайнерам!))

\begin{itemize} % {
\iusr{Валентин Стадник}
\textbf{Miroslava Kizilova} Прикольно. У нас леса гибнут, а ландшафтние дизайнери процветают, наверное.

\begin{itemize} % {
\iusr{Мирослава Кизилова}
\textbf{Валентин Стадник} этот вопрос к лесничествам и их управленцам

\iusr{Валентин Стадник}
\textbf{Miroslava Kizilova} Я знаю. Нархоз всетаки закончил. Вообще єто уже проблема общая. Єкология називаєтся.

\iusr{Мирослава Кизилова}
\textbf{Валентин Стадник} есть понятие об экологии, а есть понятие об экологической совести - а это уже касается и всего человечества, и отдельно каждого. В том числе - хозяйственника любого масштаба, лесничего, ландшафтника, да и работника с косой.

\iusr{Валентин Стадник}
\textbf{Miroslava Kizilova} А ви лектор неплохой. Не обижайтесь, говорю без ехидства.

\iusr{Мирослава Кизилова}
\textbf{Валентин Стадник} не обижаюсь)
У меня первое образование биолог, а второе ландшафтный архитектор. Экология, совесть по отношению к Природе, ландшафт с позиции «не навреди», а не шкурного интереса - все это наболевшее, за что борьба наша во всех видах работы. И у нас с этим сложно, увы((

\iusr{Валентин Стадник}
\textbf{Miroslava Kizilova} 

Ви в лесу давно били? Сейчас добавится бед и сельхозугодьями. Распашут и отравят
все. Поєтому я остро реагирую на \enquote{проблеми} города. К сожелению есть вещи
пострашнее.


\iusr{Мирослава Кизилова}
\textbf{Валентин Стадник} все нормальные люди должны реагировать остро. И все же Должен работать закон . ... дальше можно говорить до бесконечности

\iusr{Валентин Стадник}
\textbf{Miroslava Kizilova} Да, молодежь обязана верить. Даже в закон. Надобраніч.

\iusr{Мирослава Кизилова}
\textbf{Валентин Стадник}

\ifcmt
  ig https://i2.paste.pics/c0ca8381ea0b5b1e6d63e04149ac2db6.png
  @width 0.1
\fi

\end{itemize} % }

\end{itemize} % }

\iusr{Примаченко Віктор}
Ода київзеленстрою та пирію  @igg{fbicon.thumb.up.yellow}  @igg{fbicon.laugh.rolling.floor} 
Пирій взагалі то геніальна знахідка. Зелений та можна не садити та не поливати взагалі)
Певне сортовий.
Доречі, бюджет зеленстрою краще не знати)

\iusr{Татьяна Гурьева}
Наверное, всех в целом любить невозможно. Но некоторых, я надеюсь, Вы всё-таки
любите и цените. @igg{fbicon.face.kissing.closed.eyes} 

\begin{itemize} % {
\iusr{Виктория Угрюмова}
\textbf{Татьяна Гурьева} 

да)))))) но коглда я впервые опубликовала этот текст в большорй центральной
газете, я ведь еще не знала многих любимых мною нынче людей - ))


\iusr{Татьяна Гурьева}

\ifcmt
  ig https://scontent-frt3-1.xx.fbcdn.net/v/t39.1997-6/p240x240/13144937_1034553033294350_1183101131_n.png?_nc_cat=107&ccb=1-5&_nc_sid=0572db&_nc_ohc=-oGZC5ANrQ0AX8tOxel&_nc_ht=scontent-frt3-1.xx&oh=00_AT8MUhRaoUd8tky6UJckrOJX5MiOo4Omrwmk7nie-zcW3A&oe=61C1724A
  @width 0.2
\fi

\end{itemize} % }

\iusr{Елена Романская}

Вот уж наболело так наболело! И у нас та же история. Флажков, правда, нету, но
есть колышки. И даже целый огород вдоль Русановского канала. Там вам и
клубника, и смородиновые кусты, даже тыквы с кабачками. Вот насчёт картошки не
уверена. И ведь как удобно устроились - канал рядом, с поливом проблем нет. А
скажи этим садоводам-огородникам, что сие тут не к месту, искиенне обидятся.  @igg{fbicon.face.confused} 


\iusr{Евгения Бочковская}
Попала по случаю на Крещатик... Первый раз со времени появления ковида !
Шарахаюсь от метро...
И ЧТО ...???
Слепящее солнце, пыль, серость сиротливо брошенных во времени домов, но очень престижных..., снующие по тротуару самокаты и велики..., дорогущие кафешки..., островки разрекламированных и уже отцветающих тюльпанов...
И все ! ГДЕ МОЙ , ТАКОЙ РОДНОЙ, ЗЕЛЕНЫЙ И ТЕНИСТЫЙ КРЕЩАТИК
С ЦВЕТУЩИМИ В ЭТО ВРЕМЯ КАШТАНАМИ?!  @igg{fbicon.face.crying.loudly} 
К нам гости приезжали в это время на каштаны...
И на сирень в Ботаническом над Днепром.

\iusr{Tetiana Petrovska}
\textbf{Евгения Бочковская} , НАШ Крещатик с НАМИ

\iusr{Анна Шустерман}
Один из немногих островков нетронутый мотокосой.

\ifcmt
  ig https://scontent-frt3-1.xx.fbcdn.net/v/t1.6435-9/185959387_4018118908246783_9134453463535922366_n.jpg?_nc_cat=104&ccb=1-5&_nc_sid=dbeb18&_nc_ohc=eH7Q5P9J3OYAX_xXWqY&_nc_oc=AQlOsSyX-6gU4gcZmuyVeb_dNVmh0g6fRQpWYpS938Aq6SrvQfXT4qnJzRK8564r2yE&_nc_ht=scontent-frt3-1.xx&oh=00_AT9Sun_XLBeh54GKAw2Ws5BYnO_0dcB2OO3lO8KpXk8uTQ&oe=61E3321C
  @width 0.3
\fi

\iusr{Елена Иваненко}

Cегодня утром ехали через Оболонь на Берковцы. И вот они эти дяденьки-жжужалки.
Там где они поработали сегодня - ещё местами что-то зеленеет, но там где пару
дней назад - выженная земля - как в конце знойного жаркого лета. Сразу
вспомнила, что нигде за рубежом не встречала таких страшных дяденек. И,
соответственно, газоны везде наблюдались зеленные и ухоженные. Но траву везде
косили либо косилки-тележки, либо \enquote{новинка сезона} - минитрактор-косилка (им
даже горки и пригорки типа нашего Владимирского спуска не помеха). Подумала -
может нам петицию в Горадминистрацию написать - \enquote{Долой страшных грозных
мотокосов с наших улиц} ???☺

\iusr{Светлана Веник}

Позавчера видела как спилили большое здоровое дерево на Европейской площади ( в
сквере где был памятник Петровскому) Почему в Праге в скверах деревья достигают
таких размеров, как на фото??!.... Вопросик!

\ifcmt
  ig https://scontent-frx5-1.xx.fbcdn.net/v/t1.6435-9/186537395_1367858453571813_3072161202905506620_n.jpg?_nc_cat=110&ccb=1-5&_nc_sid=dbeb18&_nc_ohc=E7WX5z22TlcAX_E8e4f&_nc_ht=scontent-frx5-1.xx&oh=00_AT_dLX-rJJdldV3ESGb2GR2QMD35ZfOo2M6jJqlVJSb7fg&oe=61E0C5BD
  @width 0.25
\fi

\end{itemize} % }
