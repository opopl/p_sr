% vim: keymap=russian-jcukenwin
%%beginhead 
 
%%file 16_11_2020.news.ua.radio_svoboda.1.dnr.doneck_koncert_ukr_muzyka
%%parent 16_11_2020
 
%%url https://www.radiosvoboda.org/a/30952696.html
%%author 
%%tags 
%%title 
 
%%endhead 
 
\subsection{Через погрози у Донецьку скасували концерт української музики}
\Purl{https://www.radiosvoboda.org/a/30952696.html}

{\bfseries
«Музика поза політикою», «скасуйте «ДНР»», «блокадний Ленінград» та
«укропи». Така риторика зараз переважає у коментарях під повідомленням
про заборону української рок-музики у донецькому клубі «Gung'Ю'бazz».
Невже в окупованому Донецьку досі відбуваються українські концерти, хто на
них ходить, і чому скасували черговий? Розповість Радіо Донбасс.Реалії.
}

Вранці у неділю 15 листопада 2020 року на Вк-сторінці події «Хиты украинского
рока», що мала відбутися у легендарному Донецькому клубі «Gung'Ю'бazz»,
з’явилося повідомлення про скасування цього заходу. Як пояснили організатори,
відбулося це «у зв’язку з великою кількістю негативних відгуків». Цікаво, що не
меншу кількість негативних відгуків викликало і саме повідомлення про
скасування концерту.

\ifcmt
pic https://gdb.rferl.org/7db893ed-b67c-411c-8b2c-e8adf61ffa00_w650_r0_s.png
\fi

\subsubsection{Невже у Донецьку досі відбуваються заходи з українською музикою?}

Так, відбуваються. Найбільш відомі з них – якраз такі вечори українського
року у «Gung'Ю'бazz». І проводить їх заклад не підпільно: активно запрошує
всіх бажаючих.

\ifcmt
pic https://gdb.rferl.org/613129b7-d3e7-4279-bf2c-e8a3d5074b52_w650_r0_s.jpg
caption Реалії сучасного Донецька: вартість входу на концерт каверів українських гуртів у рублях
\fi

Ось, наприклад, афіша подібного концерту в листопаді 2019 року. З того часу
вхід подорожчав на 50 російських рублів.

На одному з них пару років тому інкогніто побувало і Радіо Донбас.Реалії.
Ось яка там атмосфера.

Але українську музику можна інколи почути і в інших клубах та закладах.
Але зазвичай це або вже давно відомі хіти, або нові відомі композиції, що
потрапляють у загальний плейлист. Про тематичні заходи там не йдеться.

%Вулиця Донецька, квітень 2018 року Вулиця Донецька, квітень 2018 року
%Ось про ту вечірку
%Вечірка української музики викликала ажіотаж в окупованому Донецьку
\url{https://www.radiosvoboda.org/a/29202555.html}

\subsubsection{Хто і як реагує на концерт і його скасування?}

У коментарі на сторінку заходу після його анонсу незабаром прийшла велика
кількість антиукраїнських коментаторів. Один з них – колишній так званий
«міністр закордонних справ» угруповання «ДНР».

\ifcmt
pic https://gdb.rferl.org/472a60ec-d190-4d3c-8530-c343546bab83_w650_r0_s.png
\fi

Або представник Донецького відділення російської консервативної організації
«Ізборський клуб» Олександр Дмитрієвський. Чи очільник запорізького
«Антимайдану», що втік в ОРДЛО, Артем Тимченко.

\ifcmt
pic https://gdb.rferl.org/b72d476b-1f7d-4925-b390-23dff493007e_w650_r0_s.png
\fi

У коментарях їм активно апелюють люди, які були підписані на спільноту клубу у
Вконтакті, або на сам захід, та збиралися на нього прийти. У тому числі,
говорять, що з погрозами прийшли переважно ті, хто ніколи у «Gung'Ю'бazz» не
бував.\Furl{https://t.me/infosotka_dn/2268}

\subsubsection{А які концерти взагалі зараз проходять у Донецьку?}

Радіо Донбас.Реалії багато розповідали про культурне життя у окупації. З
концертами там ситуація не найкраща. Переважно у Донецьк, як і у Луганськ,
приїжджають російські артисти 3-го ешелону, давно забуті гурти та
виконавці.

%Культурне життя в окупованих містах: назад у 80-і роки (рос.)
%https://www.radiosvoboda.org/a/donbass-realii/30117656.html

Концерт Григорія Лепса на площі Леніна у Донецьку 8 вересня 2020 став,
мабуть, наймаштабнішим подібним заходом з часів початку війни на Донбасі.
Його активно використовували у пропаганді підконтрольні угрупованню «ДНР»
та особисто його ватажку Денису Пушиліну ЗМІ.

Навіть буває так, що під виглядом відомого гурту виступають його двійники.
Як от на цьому відео донецького блогера з концерту на День міста:

Дещо краще працюють невеликі заклади, де відбуваються локальні концерти з
місцевими, маловідомими чи кавер-гуртами. Ось, наприклад, такі:

В популярном Chicago Music Hall до войны был ночной клуб и «Байкерз бар» В
популярном Chicago Music Hall до войны был ночной клуб и «Байкерз бар»

ДИВІТЬСЯ ТАКОЖ:

«Ганджубас» під «русским миром»: де тусуються в окупації (рос.)
\url{https://www.radiosvoboda.org/a/30117474.html}

