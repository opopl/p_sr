% vim: keymap=russian-jcukenwin
%%beginhead 
 
%%file slova.propaganda
%%parent slova
 
%%url 
 
%%author 
%%author_id 
%%author_url 
 
%%tags 
%%title 
 
%%endhead 
\chapter{Пропаганда}
\label{sec:slova.propaganda}

%%%cit
%%%cit_pic
%%%cit_text
С Украины может начаться обнуление суверенитетов, полученных в 1991 году. И это
процесс экономический в первую очередь. Но скажи это \emph{украинским пропагандистам},
они уши заткнут. Они не хотят, не желают видеть объективные процессы
экономического поражения Украины. Главная опасность суверенитету Украины вовсе
не Россия, а глупое руководство, которое свою страну грабит, и управлять ей не
умеет. Но \emph{пропаганда} трубит, что это руководство очень умное. Наш президент
строит страну дураков, которых объявил умными. Не смешно, если честно
%%%cit_title
\citTitle{Зеленский продолжает раскалывать Украину по целому ряду признаков}, 
Денис Жарких, strana.ua, 13.06.2021
%%%endcit

%%%cit
%%%cit_head
%%%cit_pic
%%%cit_text
Так, рассуждая о путях \enquote{подрыва биологической силы народа} и уменьшения
рождаемости славян, чиновник из ведомства Розенберга пишет, что в первую
очередь эта политика будет направлена на русских. Но в будущем такая практика
будет применяться \enquote{частично и к Украине}.  \enquote{Пока мы заинтересованы в том, чтобы
увеличить численность украинского населения в противовес русским. Но это не
должно привести к тому, что место русских займут со временем украинцы}, - пишет
Ветцель.  Далее он перечисляет весь инструментарий, который планируется для
этого применять: пропаганда абортов и \emph{пропаганда} \enquote{счастья} бездетной жизни
(мол, дети - это расходы и хлопоты), а также содействие добровольной
стерилизации. Должен быть обеспечен полный доступ населения и к
противозачаточным средствам
%%%cit_comment
%%%cit_title
  \citTitle{22 июня - 80 лет нападения на СССР. Что немцы готовили для украинцев}, Максим Минин, strana.ua, 22.06.2021
%%%endcit

%%%cit
%%%cit_head
%%%cit_pic
%%%cit_text
– Жертвы, потери, разрушение – не то, чем можно в сознании широких народных
масс перечеркнуть слово \enquote{великая}. Особенно если в ваших руках все механизмы
\emph{пропаганды}.  Финляндия в 1939-1944 годах три раза по-крупному воевала против
СССР. Теряла в масштабах этой малочисленной страны большое количество людей и
значительную часть территорий, причем наиболее обжитых и промышленно значимых.
И тем не менее для финнов – это великая победа. Они сохранили свою страну и не
превратились в российское Нечерноземье
%%%cit_comment
%%%cit_title
  \citTitle{Мовчання Сталіна, помилки Гітлера, окупація Києва і провалений бліцкриг – Марк Солонін про війну СРСР і Німеччини}, 
	Євген Руденко, www.pravda.com.ua, 22.06.2021
%%%endcit

%%%cit
%%%cit_head
%%%cit_pic
%%%cit_text
Это - ответ нашим политическим интриганам.  Суровая реальность для небольшой
части украинцев, называющих братский русский народ по своему скудоумию
\enquote{москалями} желающими поработить Украину.  Живой опрос на Питерской фан-зоне
разбивает вдребезги оголтелую \emph{пропаганду ненависти и розни}.  Убежден, что
невежество и политические интриги очень скоро развеяться и наконец пройдет
урегулирование конфликта на Донбассе и ситуации с Крымом.  Ведь чем сильнее
Украина, тем сильнее и Россия
%%%cit_comment
%%%cit_title
\citTitle{Футбольного поражения Украине не желал практически никто из россиян / Лента соцсетей / Страна}, Дмитрий Василец, strana.ua, 24.06.2021
%%%endcit

%%%cit
%%%cit_head
%%%cit_pic
%%%cit_text
Обман \emph{пропаганды} о «правах человека», «свободе личности», «гендерном
равенстве», «бесконфликтном обществе» то есть этических норм, провозглашенных
высшими ценностями в западном обществе уже в достаточной мере разоблачен. На
сегодняшний день ни в Европе, ни в США свободы личности не наблюдается, а
очевиден диктат меньшинства над большинством, бесправие человека перед системой
и неспособность защитить свою семью, свою систему этических ценностей, полная
экономическая зависимость индивидуума от государства и надгосударственных
структур
%%%cit_comment
%%%cit_title
\citTitle{Революция Духа – единственный путь спасения России}, 
Юрий Барбашов, voskhodinfo.su, 30.06.2021
%%%endcit

%%%cit
%%%cit_head
%%%cit_pic
%%%cit_text
Факты не имеют для цензоров никакого значения, потому что они считают
пророссийской \emph{пропагандой} любую критику современной украинской
действительности. И если вы пишите об ультраправом насилии и космических
тарифах на коммуналку, о провале вакцинации и нехватке кислорода в украинских
больницах, о позорной внешней зависимости и тотальной коррупции, о
милитаристской истерии и патриотической показухе, которая скрывает за собой
безграничное воровство – значит вы работаете на \emph{пропаганду} агрессора.
Украинскому блогеру нельзя выступать за мир и разрядку, нельзя разоблачать
лицемерие президента и напоминать о его обещаниях обманутым избирателям. С
точки зрения чиновника-патриота, добропорядочные граждане вообще не должны
высказывать публичное недовольство властью. Украинцам разрешается критиковать
ее только справа – с позиции порошенковской Партии Войны. Все прочие претензии
по адресу Зеленского или Шмыгаля представляются проплаченной Кремлем заказухой.
Потому что на Банковой до сих пор не верят, что украинцы могут не любить их
совершенно бескорыстно, не ради денег
%%%cit_comment
%%%cit_title
\citTitle{Цензоры считают пророссийской пропагандой любую критику украинских реалий / Лента соцсетей / Страна}, 
Андрей Манчук, strana.news, 27.10.2021
%%%endcit

%%%cit
%%%cit_head
%%%cit_pic
%%%cit_text
Люблю наглядные примеры того, как отличается «русский мир» от того, что делает
Украина. Особенно в контексте того, что российская \emph{пропаганда} не с 2014 года, а
гораздо раньше очень мощные ресурсы бросала на продвижение тезиса, что Украина
не состоялась и не может состояться, как отдельное независимое государство. В
ход шли все аргументы, логику из-за ее неудобства \emph{российская пропаганда} заперла
в клетке и брала громкостью. Тут и Вторая Мировая Война шла в ход (освободили –
как будто мало украинцев воевало и отдали жизни), и построенные в советское
время предприятия (снова – как будто без участия Украины), и многое другое 
%%%cit_comment
%%%cit_title
\citTitle{И какая страна теперь 404?}, Кирилл Сазонов, censor.net, 28.10.2021%
%%%endcit

%%%cit
%%%cit_head
%%%cit_pic
%%%cit_text
Американские ученые ожидали значительного выпадения радиоактивных осадков. Но
провели анализ проб продуктов «Царь-бомбы» и установили, что она была заключена
в свинцовую оболочку, и что на реакцию деления пришлось менее двух процентов
энергии взрыва, а остальная энергия — на реакцию синтеза. Как резюмировал
атомщик Лэпп, именно тогда специалисты США поняли: реально создать термоядерную
бомбу любых размеров и при сравнительно небольших дополнительных затратах.
«В значительной степени реакцию США следует рассматривать в контексте дебатов
по договору о запрещении ядерных испытаний, — заметил в разговоре с «Лентой.ру»
американский военный аналитик, руководитель экспертного портала Global Security
Джон Пайк. — В целом я считаю, что для Москвы это была \emph{пропагандистская}
неудача, поскольку никто не верил в мирные цели испытаний. Само испытание не
стало неожиданностью для американских властей, потому что Хрущев постоянно
говорил о таких планах в течение 1961 года. А ранее в том же году КГБ
посоветовал ЦРУ не беспокоиться, поскольку испытания были нацелены не на
Америку, а на Китай. Много спорили о том, нужна ли США такая бомба («ну, я бы
хотел, чтобы у нас была одна из подобных "машин Судного дня"»). Эти споры не
прекращались в течение нескольких лет — пока не спала напряженность в мире»
%%%cit_comment
%%%cit_title
\citTitle{«Дом будто ножом срезало» 60 лет назад СССР взорвал «Царь-бомбу» — самую мощную в истории. Что помнят о взрыве очевидцы?: Общество: Россия: Lenta.ru}, Дмитрий Окунев, lenta.ru, 30.10.2021
%%%endcit

%%%cit
%%%cit_head
%%%cit_pic
\ifcmt
  tab_begin cols=2

     pic https://avatars.mds.yandex.net/get-zen_doc/5324345/pub_60f29964cf9db26dda7a8641_60f2a190116f4228d0472a93/scale_1200

     pic https://avatars.mds.yandex.net/get-zen_doc/916951/pub_60f29964cf9db26dda7a8641_60f2a42b4b5d8b6f292d6bc5/scale_1200

  tab_end
\fi
%%%cit_text
Естественно поляки именуя русских "московитами" тем самым пытались лишить
русское государство права на наследство русских земель и развернули активную
\emph{пропаганду}. Которая впрочем им не помогла.  А любознательный читатель
может просто изучить карты ДО того как началась эта пропаганда и обнаружить,
что например еще в XIV веке в западных странах рисовали карты где русское
государство именовалось Россией. Например вот страница из Каталонского атласа
%%%cit_comment
%%%cit_title
\citTitle{300-летию \enquote{московии} посвящается}, Илья Duke, zen.yandex.ru, 19.07.2021
%%%endcit

%%%cit
%%%cit_head
%%%cit_pic
%%%cit_text
Кстати, о кино. В «тотальной войне» украинские \emph{пропагандисты} не брезгуют
не только топтать памятники мёртвым, но и перевирать саму память о них.
«Зловонная яма под названием "Обозреватель" забулькала и вынесла на
поверхность, что ушедший к праотцам Армен Джигарханян якобы говорил, что
русского языка нет, — сообщал ТГ-канал «Зрада чи Перемога», — Эти дебилоиды
вытащили цитаты из фильма, в котором снимался актёр, и выдали их за его личное
мнение». А зачем выбирать средства, если они за майданную родину сражаются?
Потому человеческая подлость множится и торжествует.  Нельзя говорить, что все
жители Украины сошли с ума — даже в тяжелейших условиях дискриминации и
этноцида многие остаются по-настоящему русскими людьми на русской земле. И
летний опрос социологической группы «Рейтинг» показал, что 41\% украинцев
согласны с тезисом Владимира Путина о том, что украинцы и русские — один народ
%%%cit_comment
%%%cit_title
\citTitle{Геть от цивилизации! Одна нация, одна мова, один гетман}, 
Константин Кеворкян, ukraina.ru, 11.11.2021
%%%endcit

%%%cit
%%%cit_head
%%%cit_pic
%%%cit_text
Вдова погибшего майдановца Тамара Швец сначала не захотела отвечать на вопрос,
какие достижения у Майдана. Но потом сказала, что "мы изменились духовно и
душевно. Молодежь изменилась к лучшему. Но действия власти направлены на
деградацию общества".  Экс-советник Януковича Анна Герман говорит, что после
Майдана у его сторонников забрали мечту о том, что они будут жить как в Европе.
"Потому что соглашение об Ассоциации оказалось действительно невыгодным для
Украины. И как мы теперь живем? Намного хуже".  "Украинцы поверили
\emph{пропаганде} оппозиции на Майдане, что мы будем жить как в Европе. Так же,
как они поверили фильму Зеленского, что президент будет ездить на велосипеде и
ходить в таких же штанах, как простой народ. Это были две фикции в истории
Украины: выборы Зеленского и вот эта европейская мечта", - считает Герман
%%%cit_comment
%%%cit_title
\citTitle{Как убивали на Майдане и что он дал стране. Воспоминания участников событий 8 лет спустя}, 
Юлия Колтак, strana.news, 21.11.2021
%%%endcit

%%%cit
%%%cit_head
%%%cit_pic
%%%cit_text
А теперь представьте, что нам, кроме перечисленных неологизмов постоянно
вливают в уши \emph{пропагандистские} штампы про \enquote{Наша страна понад усэ}, \enquote{Смэрть
ворогам}, \enquote{москоляку на гиляку} и самое отвратительное \enquote{Путин - ...
-ла-ла-ла-ла}... Если я, пардон, говорила в детстве слово \enquote{сволочь} или
\enquote{дурак}, бабушка заставляла меня идти чистить зубы и полоскать рот. Ибо для
ребенка это гадкие слова. А тут вся страна (и это мы уже перешли к Украине) -
от младенца до старика смакует во рту матерное, отвратительное словесное месиво
- и только улыбки шире. А ведь \enquote{лайфхаки} у них там тоже никто не отменял.
представляете, какая в голове каша? Украинцы так дорожат мовой, что даже пишут
её с заглавной буквы - Мова, не иначе. И что в этой Мове сегодня в сухом
остатке - заимствованные американизмы, да мат, да проклятия с ненавистью. Ведь
сепаров с колорадами и ватниками придумали не мы...
%%%cit_comment
%%%cit_title
\citTitle{О русском, украинском языках и о смыслах...}, 
Красная Книга Человечества..., zen.yandex.ru, 22.09.2021
%%%endcit
