% vim: keymap=russian-jcukenwin
%%beginhead 
 
%%file 19_02_2023.fb.sosnovskij_evgenij.mariupol.1.50_v_ddt_nk_v_mar_up.cmt
%%parent 19_02_2023.fb.sosnovskij_evgenij.mariupol.1.50_v_ddt_nk_v_mar_up
 
%%url 
 
%%author_id 
%%date 
 
%%tags 
%%title 
 
%%endhead 

\qqSecCmt

\iusr{Evgeny Sosnovsky}

Посилання на онлайн-крамницю \url{https://www.instagram.com/mariupol.souvenir}

\iusr{Оксана Стомина}

Клас! Мммм...

\iusr{Виктория Артюшкевич}

Супер!

\iusr{Valeriya Conacher}

Скажiть в Юрмалу можна замовити?)

\begin{itemize} % {
\iusr{Evgeny Sosnovsky}
\textbf{Valeriya Conacher} Чесно, кажучи, не знаю. Напишіть, будь ласка, в онлайн-крамницю. Думаю, що міжнародною поштою теж можна.

\iusr{Valeriya Conacher}
\textbf{Evgeny Sosnovsky} дякую.
\end{itemize} % }

\iusr{Alex Volodarskiy}

Скільки ти мені цієї краси встиг показати.

\begin{itemize} % {
\iusr{Evgeny Sosnovsky}
Так, ключове - встиг... Але які були плани і скільки ще не встигли...
\end{itemize} % }

\iusr{Olga Demidko}

Яка краса!!! 😍🫶🥰

\iusr{Анна Головко}

Трохи сумно, що так... вчасно 😥

Де можна придбати?

\begin{itemize} % {
\iusr{Evgeny Sosnovsky}
\textbf{Анна Головко} Якщо хтось є у Кривому Розі, то в центрі \enquote{ЯМаріуполь}. Або в онлайн-крамниці 
\url{https://www.instagram.com/mariupol.souvenir}
\end{itemize} % }

\iusr{Tatyana Grebenyuk}

Які гарні😍

\iusr{Алина Иванова}

Можна зробити кахлі, були б красиві.

\iusr{Ольга Самойлова}

Супер!) Вітаю, дуже хочу замовити!

\iusr{Diana Kostyak}

Маю ваші 3 листівки подруга незадовго до початку війни бігала по Маріуполі
шукала мені листівки з видами міста, знайшла, відправила, встигли дійти. Тепер
перелистую альбом з листівками і завжди на ваших зупиняюсь зі сльозами на очах
😥

\begin{center}
\begin{minipage}{\textwidth}
\iusr{Diana Kostyak}

\ifcmt
  tab_begin cols=2,no_fig,center,separate

     pic https://scontent-fra3-1.xx.fbcdn.net/v/t39.30808-6/332094039_931935317962657_135950934644325107_n.jpg?_nc_cat=104&ccb=1-7&_nc_sid=dbeb18&_nc_ohc=Dyp2Gi5MOkAAX9ckVGj&_nc_ht=scontent-fra3-1.xx&oh=00_AfCM55groEQ5TWS3ACcIaFXq6wTfQz0BxXAxUxQY_-R-0g&oe=641A233E

     pic https://scontent-frt3-2.xx.fbcdn.net/v/t39.30808-6/332121615_593450135591144_9116810457828348892_n.jpg?_nc_cat=108&ccb=1-7&_nc_sid=dbeb18&_nc_ohc=InnyDFYTpcUAX830WzV&_nc_ht=scontent-frt3-2.xx&oh=00_AfD8ET8CHUoFgBTXJMEixk6TVzccwjPWXYB1hkb2rGIMJw&oe=641ABC2B

  tab_end
\fi
\end{minipage}
\end{center}

\iusr{Elena Nikishova}

❤️❤️❤️

\iusr{Олександр Ірванець}

Маріуполь прекрасний!

Я закохався в це місто ще в 2015-му!

після Перемоги маєте найперш відбудувати Башту в центрі міста і будинки з
башточками на Центральній Площі, там, де жила Оксанка Стоміна з Дімою!

\begin{itemize} % {
\iusr{Оксана Стомина}
\textbf{Олександр Ірванець} будемо відбудовувати, Сашко. Ще зберемося у нас. Але дуже сумно, що не всі. Ось щойно дізналася про ще одного нашого чудового хлопця. Щойно загинув. Саш, скільки чудових людей ми втратили! Який біль!

\iusr{Evgeny Sosnovsky}
\textbf{Олександр Ірванець} Башта вціліла. Будинки за баштами теж більш-менш. Відновляться, відбудуються. А ось людей вже не повернеш...

\iusr{Alena Pakhomova}
\textbf{Олександр Ірванець} Вежа. 🙂
\end{itemize} % }

\iusr{Lyudmila Osadchuk}

Пане Євген, дякую Вам за Вашу любов до міста, вкладену в кожен Ваш знімок, за
любов до України. За те, що вижили, вберегли, і донесли нам.

\iusr{Людмила Мікуліна}

\ifcmt
  igc https://scontent-fra5-2.xx.fbcdn.net/v/t39.30808-6/332684475_633330181889809_8756845427678892139_n.jpg?_nc_cat=109&ccb=1-7&_nc_sid=dbeb18&_nc_ohc=esxaQsjI9Z4AX_gSC1k&_nc_ht=scontent-fra5-2.xx&oh=00_AfBljk1WmDLFBQzJOQRCihYt1HmmWWkHiKJ-5YwyjzocLA&oe=6419F927
	@width 0.5
\fi

\iusr{Galyna Van Den Bril}

Ваші фото неоціненний скарб, який допоможе відтворити найкращі будови Маріуполя!

\begin{itemize} % {
\iusr{Evgeny Sosnovsky}
\textbf{Galyna Van Den Bril} Хай Буде Так...
\end{itemize} % }

\iusr{Татьяна Дубинина}

Интересуюсь ценой.

\begin{itemize} % {
\iusr{Evgeny Sosnovsky}
\textbf{Татьяна Дубинина} Напишіть повідомлення за вказаним у першому коментарі посиланням. Я не знаю ціну
\end{itemize} % }

\iusr{Olesia Rodina}

Була в Маріуполі восени 2014 року. Ніколи не забуду гарне, творче, привітне місто.
