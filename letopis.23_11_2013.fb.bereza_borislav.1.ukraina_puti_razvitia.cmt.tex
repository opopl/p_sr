% vim: keymap=russian-jcukenwin
%%beginhead 
 
%%file 23_11_2013.fb.bereza_borislav.1.ukraina_puti_razvitia.cmt
%%parent 23_11_2013.fb.bereza_borislav.1.ukraina_puti_razvitia
 
%%url 
 
%%author_id 
%%date 
 
%%tags 
%%title 
 
%%endhead 
\subsubsection{Коментарі}

\begin{itemize} % {
\iusr{Fima Khayutin}
всё верно!!

\iusr{Roman Shuta}

Есть еще два партнера Канада и Кирибати... в Канаде можно стать еще одной
провинцией. А с Кирибати - самой большой налоговой халявой на континенте )))


\iusr{Борислав Береза}

Границы. В них вся фишка. Грузия выбрала партнером Штаты. Но когда российские
танки вошли в Грузию Штаты были далеко. А Европа рядом. А крымскую карту
Кремль, рано или поздно, захочет разыграть.

\iusr{Fima Khayutin}
если бы грузины успели вступить в НАТО, то кремль бы 1000 раз подумал... дело не только в границах...

\iusr{Борислав Береза}
О! А это, Фима, вторая фаза.  @igg{fbicon.smile} 
И прощаемся с российским ЧФ.

\iusr{Fima Khayutin}
будет сложно, Севастополь очень пророссийский... часто бываю там..

\iusr{Борислав Береза}

Сами уйдут при создании определенных условий. Как политических так и
коммунальных. А то что Севастополь пророссийский это не заслуга Кремля, это
недоработка Киева. Стратегическая.

\iusr{Борислав Береза}
Кирилл, а Кенигсберг немецкий, а Курилы японские. А Сибирь якуто-бурятская. Не стоит ревизионизмом увлекаться.

\iusr{Борислав Береза}

Кирилл, пойми мы не против русских, мы против имперской политики, которая на
нас распространяется и за хорошую жизнь. Я процитирую слова моей подруги Анна
Балент: "Я НЕ пошла и не даже не собиралась на Майдан. И я НЕ против России в
принципе, поскольку с рождения говорю по-русски и там вся моя семья. Но я
против трусости идти вперед и признавать очевидные вещи."

Вот и мы не хотим застрять в совке. Мы хотим идти вперед, а Кремль тянет нас с
собой, примитивно покупая наших воров-правителей.

\iusr{Борислав Береза}

Кирилл, а с VII и по XIII век Крым был под хазарами и Византией. Затем череда
из генуэзцев, татар и османцев владели им. Потом 19 апреля 1783 Катька
присоединила его к России, а после 19 февраля 1954 Крым передали УССР. Украина,
как наследник и правопреемник получила права на Крым. Это признали все. И
Россия тоже. Все. Поставить пора бы точку. Но империи неймется. Я бы на месте
правителей не о Крыме думал, а о Чечне, Дагестане, Ингушетии. О Кавказе. Ведь
рано или поздно эта бомба рванет. И не просто оторвется, тогда, Кавказ от
России, а ударной волной может и саму империю развалить. Проведи параллели с 91
годом и СССРом. По этому пусть Москва занимается своими территориями. И не
трогайте наше. И будет благодать!

\iusr{Борислав Береза}

Друже, а Япония тоже позволяла нелепые телодвижения? А Аляску забирать будете?  @igg{fbicon.smile} 

Жить надо сегодняшним днем. И завтрашним. Но не прошлым. А иначе старые обиды
будут мешать строить новую жизнь. Помнить хорошее и стараться не вспоминать
плохое. Но и не забывать ничего.


\iusr{Борислав Береза}
Каким образом?

\iusr{Борислав Береза}
А Италия? Дуче тоже начудил. Чего же их, за компанию, не наказали?

\iusr{Борислав Береза}
А кто спорит?  @igg{fbicon.smile} 

\iusr{Fima Khayutin}
\textbf{Кирилл Потемин} 

а австралия, америка, мексика, израиль, европа... тоже станем выяснять, кто кому
должен? тогда россия это московское княжество, чуть дальше московской области...

\iusr{Fima Khayutin}
вообще хорошо, что вы Кирилл Потемин, верите в будующее...)))

\iusr{Fima Khayutin}

я плохо по русски, 25 лет живу в Израиле.... учить русский вроде незачем.., врядли
пригодиться... вам вот лучше языки учить....))


\iusr{Сергей Данченко}
Если это приватная дискуссия, то общайтесь с Березой в приватном чате. Тем более с таким пафосом.

\iusr{Fima Khayutin}
\textbf{Кирилл Потемин} я вроде не на вашей стенке высказываю своё мнение...

\iusr{Fima Khayutin}

меня всегда забавляют люди из россии... сами в междусобойчиках льют грязь на
свою страну так, что от брызг не увернуться..! но стоит инородцу что то сказать
про великую империю, тут же начинается массовый разрыв тельников за родину...

\iusr{Екатерина Гольцберг}

А России придется учить китайский... а русский они не выучили до сих пор...

\iusr{Fima Khayutin}
Екатерина до урала английский, за уралом китайский....

\iusr{Aleks Dro}

Кирил я понимаю что вам и мое мнение не интресно но у вас на странице фото с
дагестанцами - фашистами вам бы не крым а москву вернуть обратно

\iusr{Борислав Береза}
Ого! Я уехал, а тут только развернулись дебаты...  @igg{fbicon.smile} 

\iusr{Fima Khayutin}
это разве дебаты...?)) игра в одни ворота))

\iusr{Борислав Береза}
А тем временем, до Вильнюса, времени все меньше и меньше...

\iusr{Вадим Ермоленко}

Слава. Может пора просто навести порядок? без сильных соседей? Вон Фины два раза
лупили своего сильного соседа... Один раз получили серьезных люлей. нол
выжили... И не имея ни истории *триста лет колоннии - далее теменьи
"Калевалла"), харизмы, ни ископаемых, за сто лет независимости производят ВСЕ -
от "Тикуриллы" до "Нокии"... При населении в 4. 5 млн. У них во всей Лапландии 200
тыс человек и 200 тыс. оленей... Но "сухо и тепло"...


\iusr{Борислав Береза}

Вадим, менталитет другой. Тут же часть генофонда уничтожили, часть
эмигрировала, а оставшиеся, действительно, жили под пресингом идеологии. Моисей
не зря водил народ свой 40 лет по пустыне. Ждал пока не только помнющие рабство
вымрут, но и когда появится новое поколение - рабства не знающее. Для того, что
бы такое поколение появилось в Украине нужно для начало отойти от
совково-российского влияния. И первым этапом в этом процессе будет попытка
ассоциации с ЕС.

\end{itemize} % }
