% vim: keymap=russian-jcukenwin
%%beginhead 
 
%%file 06_07_2023.stz.news.ua.mrpl.0629.1.mariupolchanka_zaproshuje_na_vystavku.txt
%%parent 06_07_2023.stz.news.ua.mrpl.0629.1.mariupolchanka_zaproshuje_na_vystavku
 
%%url 
 
%%author_id 
%%date 
 
%%tags 
%%title 
 
%%endhead 

Вивезла з блокади унікальні зацифровані документи та фото. Маріупольчанка запрошує на виставку

Ольга Демідко - активістка, волонтерка, журналістка, доцентка кафедри
культурології Маріупольського державного університету. Вона змогла вивезти з
окупованого міста унікальні зацифровані документи та фото, і тепер організовує
виставки та екскурсії, щоб нагадати всьому світу про знищення української
культурної спадщини через агресію росії. 

СМС з погрозами 

Виїхати з Маріуполя Ольга з маленьким сином змогла лише 20 березня: в неї не
було власної автівки, тому родина змушена була переживати страшні обстріли у
будинку в приватному секторі. На їхній вулиці, зізнається Ольга, були тільки
дві родини, які мали патріотичні переконання. Всі інші, на жаль, вважали
"руський мір" гарним варіантом для себе. Тому шукати порятунок вирішили разом з
однодумцями, і щоб вивезти родини, навіть намагалися викупити одну з автівок в
іншого сусіда, який відмовляв, бо не вважав російську армію загрозою. 

- В мене хвора мама, лежача бабуся, батько після операції та трирічна дитина.
При відсутності машини вивезти усіх майже нереально. З кожним днем ситуація
погіршувалася: попри відсутність стабільного зв'язку на мій телефон інколи
"проривалися" смс з погрозами, як патріотичній активістці та волонтерці. А 20
березня почали горіти будинки на нашій вулиці. У нас обвалилася стеля,
зачепивши бабусю. Мама кричала, щоб я їхала та рятувала сина, - розповідає
Ольга. -  Тому коли той сусід, що мав кілька автівок, але відмовлявся нам
продати одну, зник безвісті, ми вирішили взяти транспорт без його згоди. Але
виїхати з сусідами змогла тільки я з малюком.

У дворі вони залишили повідомлення, що зобов'язуються повернути або сплатити її
вартість. Без документів на машину було складно проходити через блок пости.
Врятувало, що власник автівки був колишнім чоловіком сусідки, з якої вони
виїжджали, і спільно прізвище переконало окупантів пропустити трьох дорослих та
трьох дітей. Три місяці Ольга з сином була у Празі. Увесь цей час шукала
зв'язок з батьками. Згодом знайома вивезла їх до Мелекіного, де було світло,
вода, опалення. А у серпні родини з'єдналася у Києві.

Бабуся, на жаль, не пережила цих жахіть: через стрес вона перестала їсти, а 7
квітня померла. Але про цю родинну трагедію Ольга дізналася тільки наприкінці
квітня, коли вперше змогла почути голос рідних. 

Відновлення університету

До повномасштабного вторгнення Ольга реалізовувала багато проєктів у рідному
місті: з 2014 року проводила екскурсії старою частиною Маріуполя, робила
тематичні зустрічі, на яких можна було дізнатися про таємниці старовинного
цвинтаря, археологічну спадщину міста, легенди Азовського моря та різні
цікавинки, що пов'язані з Маріуполем. А ще на міському телебаченні Ольга вела
програму "Маріуполь театральний". 

- Звісно, коли я переїхала до Києва, мені хотілося відновити діяльність. Тому я
цікавилась різними проєктами, писала гранти. Водночас я повернулася до
викладацької справи — наш Маріупольський державний університет релокувався до
столиці, і вже 18 квітня ми почали працювати. 

Пані Ольга каже, що більшість викладачів університету виїхали з окупованого
міста, а керівництво навчального закладу робило усе, щоб об'єднати спільноту
вже на новому місці. Звісно, було немало важких психологічних моментів  — у
багатьох викладачів і студентів загинули рідні та близькі. На жаль, у самому
колективі вишу теж є загиблі — як серед викладацької, так і серед студентської
спільноти. 

- Все це дуже важко згадувати... Але ми намагалися підтримувати один одного та
рідний університет, - каже пані Ольга. - У червні їздили до Буковеля на
стратегічні сесії в межах проєкту "Відродження переміщених університетів:
посилення конкурентоспроможності, підтримка громад" за фінансової підтримки
Європейського Союзу.  Зараз університет грантовими коштами відбудовує
навчальний корпус, щоб студенти мали можливість зустрічатися та займатися
наживо. Бо живі зустрічі - важливі.

На стратегічні сесії не всі змогли приїхати: викладачі евакуювалися у різні
країни Європі та області України. Але більшість змогли побачити один одного
після важких випробувань, і це дало натхнення. 

- Протягом чотирьох днів ми розв'язували важливі питання університету, будували
плани, обговорювати складні моменти, думали про взаємодію з громадою, де тепер
будемо жити та працювати, - пояснює Ольга. -  На останнє націлені й проєкти,
які ми розробляємо зараз на кафедрі культурології. Особисто я відчула потужний
поштовх, повірила у наш потенціал, побачила очі колег, переконалася, що рідний
університет незламний.

Здавалося б, зізнається Ольга, зараз доводиться робити неможливе. Але так
приємно чути, як студенти обговорюють майбутнє приміщення для коворкінгу чи
тішаться новим меблям. Сили додає причетність до роботи Штабу університету, де
допомагають не тільки викладачам та студентам, а всім. 

- Бо важливо не тільки дати набір продуктів, але й запропонувати культурне
перезавантаження. Після таких зустрічей люди не хочуть розходитися, їм важливо
говорити, розповідати про свій досвід чи плани на майбутнє.  Київ@Маріуполь 

Київ@Маріуполь 

Перша київська екскурсія, яку провела Ольга Демідко, була присвячена паралелям
між пам'ятками Києва та Маріуполя. Це був цікавий та корисний досвід, але Ольга
вирішила далі розробляти проєкти виключно по Києву. До столиці переїхало багато
маріупольців, яким важливо відчувати підтримку спільноти земляків, але інколи
дуже важко чути про зруйноване рідне місто. 

- Багато хто зізнавався, що відчуває біль, коли чує про те, що вже знищено
вдома. І люди просили створити просто екскурсії з історії Києва. Я вивчила
старовинну частину столиці, де багато вулиць з унікальною історією. Почала з
Ярославового валу — там що не будинок, то цікавинка. До речі, якою мовою
проводити екскурсії, у Києві, на відміну від Маріуполя, вже не питаю.
Маріупольці охоче говорять українською, помітно, як вони стараються, -
зізнається Ольга.

А 7 липня, в День народження народної артистки України, почесної громадянки
Маріуполя Світлани Отченашенко, в приміщенні університету відкривається
пересувна фотовиставка "Літопис театрального життя Маріуполя".  Ця ініціатива,
автором якої стала Ольга Демідко, перемогла у Конкурсі наукових проєктів за
напрямом "Наука та культура: збереження і популяризація історико-культурної
спадщини", що проводив Київський університет імені Бориса Грінченка в рамках
Всеукраїнської науково-практичної конференції "Дослідження молодих вчених: від
ідеї до реалізації". 

- Саме Світлана Отченашенко радила мені зацифрувати безціні документи та фото.
Я зберегла понад тисячу світлин, серед яких архівні матеріали Донецького
академічного обласного драматичного театру (м.Маріуполь), Народного театру
"Азовсталь", особистого архіву Світлани Отченашенко, фондів Маріупольського
краєзнавчого музею. На них можна побачити приміщення, безліч афішок, програмок,
портрети акторів - це те, що знищено, потримати в руках вже не вдасться, але ще
можна побачити на світлинах, - розповідає Ольга.

В рамках проєкту заплановані творчі зустрічі з театральними працівниками:
акторами, режисерами Маріуполя та проведення лекцій та круглих столів,
присвячених історії театрального мистецтва міста. Загалом проєкт сприятиме
руйнуванню радянської "парадигми закритості" музеїв та архівів. Виставка, що
створена за оновленими концепціями та підходами, стане чудовою платформою для
презентації та популяризації історико-культурної спадщини міста-героя як в
Україні, так і за кордоном. 

Авторка проєкту вважає, що дуже важливо долати стереотипи навколо театрального
життя Маріуполя, бо багато хто досі просуває ідеї про виключно російську
площину культурного життя Приазов'я. Вона згадує багатьох митців та вистави,
які були пов'язані з українською театральною традицією, і навіть про годинник,
який подарували Марку Кропивницькому вдячні театрали Маріуполя. 

- Чи, наприклад, свідчення того, як ОУН у Маріуполі сприяло розвитку театру.
Розумієте, треба доводити - наша історія варта того, щоб про неї розповідати та
її зберігати. Навіть коли місто окуповано, а артефакти знищені, - пояснює
Ольга. - Зараз мені допомагають студентки кафедри культурології, і відкриття
відбудеться саме у нашому університеті. Це символічно, а далі, думаю, будемо
експонувати в інших музеях та навчальних закладах, можливо, за кордоном.  Як
жити з колаборантами

Останнім часом в інформаційному просторі з'являються повідомлення про намагання
окупаційної влади "налагодити" культурне життя зруйнованого ними Маріуполя.
Зокрема з'явилася інформація, що талановиту молодь з захоплених ними регіонів
готові приймати на навчання у театральні виші Москви — на спеціальну програму.
Як людина, котра багато сил та часу присвятила вивченню театрального життя
Маріуполя, пані Ольга болісно реагує на подібні цинічні заяви. Але філософськи
пояснює ці тенденції: 

- Я розумію, що там у всіх сферах є колаборанти, можливо, хтось захоче
поповнити й ці ряди, - каже вона.-  Багато хто піддався впливу
пропагандистської машини, ми є свідками того, що багато людей "перевзулися".
Навіть ті, від кого ніколи б цього не очікував... Але було й по-іншому.
Наприклад, Сергій Забугонський разом з дружиною багато років працював у нашому
драмтеатрі, й вони ніколи не приховували, що російська мова для них ближче, а
вчити тексти українською — незручно. Я спілкувалася з ним, брала інтерв'ю, і
вважала його таким, що першим буде вітати росію. Його дружина була комендантом
в драмтеатрі, коли там переховувалися люди під час бомбардувань. І вони обидва
бачили той літак, який скинув бомбу на приміщення з надписом "Діти". Після
цього виїхали до Європи, хоча мали родичів у рф — після того, що побачили, вже
не могли навіть обговорювати щось про "братів" та дружніх "сусідів". Але, на
жаль, багато хто, переживши це на власному досвіді, так і не переконався, на
чиєму боці правда... Сподіваюсь, що після деокупації їх там вже не буде. Добре,
що ми побачили, хто є хто. Мені не хотілося б з ними стикатися та
співпрацювати. 

Повернення до рідного Маріуполя - складне питання для Ольги. Вона зізнається,
що зруйноване місто може стати тригером тих емоцій, які точно не будуть на
користь їй та її дитині. Бо навіть зараз чотирирічний син, дивлячись на фото,
каже - зруйноване...  

- Думаю, я туди поїду, готова допомогти у відновленні. Але зараз не розумію, як
та де там жити, - з сумом зізнається Ольга. 
