% vim: keymap=russian-jcukenwin
%%beginhead 
 
%%file 18_10_2020.fb.vasiliy.volga.1.moskva
%%parent 18_10_2020
%%url https://www.facebook.com/Vasiliy.volga/posts/2724821547835260
 
%%endhead 

\subsection{«МОСКВА СЛЕЗАМ НЕ ВЕРИТ»}
\label{sec:18_10_2020.fb.vasiliy.volga.1.moskva}

\url{https://www.facebook.com/Vasiliy.volga/posts/2724821547835260}

Не знаю, как вы смотрите этот фильм и сколько раз вы его смотрели, я же смотрю
его уже даже не глазами, а как-то словно всем собой, своей каждой клеткой души
и даже в кончиках пальцев я ощущаю покалывание, когда слышу слова из песни Юрия
Визбора:

«Не сразу все устроилось,
Москва не сразу строилась,
Москва слезам не верила,
А верила любви».

Сегодня мы смотрели этот фильм с моей младшей дочерью.

Ей 18 лет. Когда она мне сказала сегодня утром, что она посмотрела первую серию
и ей очень понравилось, и она хочет, чтобы вторую серию мы посмотрели вместе, я
вдруг как-то понял, что это пожалуй один из самых важных моментов в нашей с ней
истории. Ведь что получается - что она моя, что вся гадость последних тридцати
лет, творящаяся на Украине, её не задела совершенно, что с того момента, как
она первый раз два года назад побывала в Москве и влюбилась в Арбат, в ширь, в
храмы, парки, в людей, что она по новому начала гордиться своей фамилией, что
она чувствует так же, как чувствую я.

Какая-нибудь мерзость обязательно сегодня напишет под этим постом «Чемодан –
вокзал - Россия» и еще что-нибудь. Напишет это полный кретин, не понимающий,
как моя дочь и я, как мы любим Киев. Но так же этот кретин не поймет, как у нас
изболелось сердце от того, что сделали с нашим Киевом его новые хозяева --- новое
старое рагулье, которое увы, надо признать, своим жлобством, оказалось сильнее
нас, а нас оказалось мало, чтобы им противостоять. Да и силища за ними встала
огромная, за нами никто не встал. Самые лучшие из нас либо погибли, либо
вынуждены были бежать из страны.

Но не хочу о плохом. Я стал сентиментален. Когда я видел, как моя дочь
радовалась, смеялась, утирала слезу именно в тех местах фильма, где я
радовался, смеялся и утирал слезу, то даже до сих пор у меня, когда я пишу об
этом, просится слеза. Добрая, умилительная, счастливая.

Вот такой он этот фильм «Москва слезам не верит». Двухсерийная академия добра.

И даже незаметно для самого себя я напеваю Визбора:

«Александра, Александра,
Этот город --- наш с тобою,
Стали мы его судьбою —
Ты вглядись в его лицо.
Чтобы ни было в начале,
Утолит он все печали.
Вот и стало обручальным
Нам Садовое Кольцо».

Добра вам друзья, и много любви вашим семьям.
