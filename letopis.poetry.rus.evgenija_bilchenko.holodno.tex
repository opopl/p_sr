% vim: keymap=russian-jcukenwin
%%beginhead 
 
%%file poetry.rus.evgenija_bilchenko.holodno
%%parent poetry.rus.evgenija_bilchenko
 
%%url https://t.me/bilchenko_z/190
%%author 
%%tags 
%%title 
 
%%endhead 

\subsubsection{БЖ. Холодно}
\label{sec:poetry.rus.evgenija_bilchenko.holodno}
\Pauthor{Бильченко, Евгения}
\Purl{https://t.me/bilchenko_z/190}

Сижу в квартире, в панике, в болезни.
Язык, страной обложенный, как спамом,
Пытается быть чище и полезней
Для общества. Запив гидозепамом

Бессильных строк таинственную силу,
Я всё ещё пытаюсь быть пророком
Двойным - для Украины и России:
Избит двумя, зато не одиноко.

О, как меня (плохое междометье)
Невинно (даже жаль их всех) хлестали!
Я делал микс, мешая жизнь со смертью,
Мешая воду с раскалённой сталью.

Напиток получался слишком сладким
И слишком горьким - вкус не сочетаем
Ни с нервным поэтическим припадком,
Ни с дамой, пощадившей горностая

Во имя европейских брендов эко:
Зато она приобрела в Берлине
Красивое пальто из человека -
Ручную вязку из венозных линий.

А мне прислали из Москвы жилетку:
Всамделишную, с запахом Сибири.
Сижу в квартире и молюсь на клетку.
Спасибо, Бог, за то, что не убили.

2 ноября 2020 г. 

Илл.: Эль Греко (или: Алонсо Санчес Коэльо). "Дама в мехах".
