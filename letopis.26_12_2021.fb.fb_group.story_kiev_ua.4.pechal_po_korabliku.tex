% vim: keymap=russian-jcukenwin
%%beginhead 
 
%%file 26_12_2021.fb.fb_group.story_kiev_ua.4.pechal_po_korabliku
%%parent 26_12_2021
 
%%url https://www.facebook.com/groups/story.kiev.ua/posts/1827021630827963
 
%%author_id fb_group.story_kiev_ua,poljakova_galina.kiev
%%date 
 
%%tags kiev
%%title Печаль по Кораблику
 
%%endhead 
 
\subsection{Печаль по Кораблику}
\label{sec:26_12_2021.fb.fb_group.story_kiev_ua.4.pechal_po_korabliku}
 
\Purl{https://www.facebook.com/groups/story.kiev.ua/posts/1827021630827963}
\ifcmt
 author_begin
   author_id fb_group.story_kiev_ua,poljakova_galina.kiev
 author_end
\fi

По поводу моих миниатюр на тематику историческую возникло много споров. Это
хорошо. Хотя высказывания оппонентов были не всегда тактичными.  Досталось и
мне, и героям миниатюр. Ну и ладно. Сделаем передышку. Вот миниатюра о наших
днях. 

«ПЛЫЛ КОРАБЛИК ГОЛУБОЙ, 
А НА НЕМ И МЫ С ТОБОЙ ...»

Только тот, кто знает и любит Киев, поймет мою печаль по Кораблику. 

Вы, конечно же, знаете Киев. А знаете ли вы о таком скромном микрорайоне –
Соцгород? Это далеко не Липки. Соцгород расположен в Дарнице и представляет
собой совершенно уникальное явление. Его называют «осколком Советского Союза».
Но он, скорее, «привет из 50-х». Был спланирован еще до войны, но построен
сразу же после ее окончания. Для начала были возведены домики немецкие. Строили
их пленные немцы. Строили так, как умели. Эти дома сейчас сносят – а зря! Они
уютные, 2-3 этажные, как в старых книгах со сказками. В Соцгороде преобладают
«хрущевки», немного «сталинок». И кое-где во дворах еще сохранились фонтаны,
теперь уже сухие и полуразрушенные. 

Я о Соцгороде хоть и слыхала, но ничего, к своему стыду, не знала. Но потом
купила там квартирку под офис – и влюбилась в этот район со всеми его сиренями,
акациями, тихими улочками, маленькими сквериками и широкими бульварами. Любовь
пришла потом, а сначала был трезвый расчет: метро в пяти минутах, а цена совсем
никакая. 

Очень скоро я заметила, что для того, чтобы маршрутка остановилась возле нашего
офиса надо произнести короткое заклинание: «Водитель! Возле Кораблика,
пожалуйста!» Маршрутка как вкопанная останавливалась на углу улицы Бажова и
бульвара Верховного Совета, или как именуют его киевляне – БВС. 

Все было здорово, но я хотела увидеть и Кораблик. А Кораблика не было нигде. Я
обошла все окрестности, заглядывала во дворы, но все напрасно. Не было ни
скульптур хотя бы отдаленно напоминающих очертания судна, ни рисунков на
стенах. Муралы появились у нас позже. Но для них нужны голые стены, а все стены
в Соцгороде «глазастые» – полно окошек. А вот корабликов нет. 

Купленный офис – квартирку в «хрущевке» – я очень полюбила. С утра солнце
светит в одну комнату, а после обеда – заливает светом другую через широкие
трехстворчатые окна. А что еще очень удачно – через дорогу располагался
гастроном. Обычный советский гастроном. В конце 90-х они еще были в Киеве. Три
торговых зала, грязно-белые кафельные стены и серые плиты на полу, маленький
кафетерий на четыре столика. А там салатики, жареная рыбка еще теплая, и
пирожные, дивные лимонные пирожные, тоже еще теплые, потому что все там же и
готовилось, так как при гастрономе была еще и кухня. Продавщицы были милые и
приветливые тетки, охотно подсказывали, что сегодня стоит покупать, готовы были
выбрать булочку порумянее и нарезать колбаску хоть поперек, хоть вдоль. Тут же
у окна прилепились прилавки с газетами, золотыми изделиями Киевской Ювелирки и
что-то еще, не помню, что именно, но очень полезное. Я и гастроном этот
полюбила. В нем было уютно. И покупатели, и продавцы знали друг друга в лицо, а
самых давних и самых почтенных – еще и по именам. За какой-то год я тоже была
причислена к сонму избранных.

В тот день у меня было настроение сентиментальное, я разболталась с
продавщицей. Она оказалось старожилкой Соцгорода. Какая находка! 

- Танечка, голубушка! Пожалуйста, расскажи мне про Кораблик. Вот этот участок
суши на углу Бажова и БВС все называют Корабликом. Что это за кораблик такой?
Где он? Что он? Я его уже второй год ищу. 

- Как это ты не знаешь, Галочка? А ну, посмотри налево.

Я послушно развернулась в указанном направлении и уставилась на соседний зал и
на стойку кафетерия в глубине третьего зала. Кораблика там не было. 

- Ты не туда смотришь. Вот ОН, на стене. 

Я глянула на стену, облицованную лет сорок назад грязно-белыми кафельными
плитками, подняла глаза … Вот он! На высоте пары метров, на стене, облицованной
грязно-белыми кафельными плитками, из кафельных осколков был выложен маленький
Кораблик. Узкий корпус, двойной парус, трепещущий вымпел на верхушке мачты.

Эта мозаика явно не вписывалась ни в какие СНиПы и ГОСТы. Кто выложил эту
картинку из осколков кафеля? 

Кораблика больше нет. Вместо гастронома – безликий супермаркет сети Фора.

А ведь Кораблик был.

\ii{26_12_2021.fb.fb_group.story_kiev_ua.4.pechal_po_korabliku.cmt}
