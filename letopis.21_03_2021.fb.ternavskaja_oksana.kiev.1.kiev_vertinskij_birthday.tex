% vim: keymap=russian-jcukenwin
%%beginhead 
 
%%file 21_03_2021.fb.ternavskaja_oksana.kiev.1.kiev_vertinskij_birthday
%%parent 21_03_2021
 
%%url https://www.facebook.com/venta.avia/posts/4052508811437462
 
%%author_id ternavskaja_oksana.kiev
%%date 
 
%%tags 1889,21_mar,birthday,gorod,kiev,vertinskij_aleksandr
%%title Киев — Родина нежная - день рождения Александра Вертинского
 
%%endhead 
 
\subsection{Киев — Родина нежная - день рождения Александра Вертинского}
\label{sec:21_03_2021.fb.ternavskaja_oksana.kiev.1.kiev_vertinskij_birthday}
 
\Purl{https://www.facebook.com/venta.avia/posts/4052508811437462}
\ifcmt
 author_begin
   author_id ternavskaja_oksana.kiev
 author_end
\fi

21 марта 1889 года в Киеве родился  Александр Николаевич Вертинский - русский и
советский эстрадный артист, киноактёр, композитор, поэт и певец, кумир эстрады
первой половины XX века, отец актрис Марианны и Анастасии Вертинских.

\begin{multicols}{2}
\obeycr
Киев — Родина нежная,
Звучавшая мне во сне,
Юность моя мятежная,
Наконец ты вернулась мне!
\smallskip
Я готов целовать твои улицы,
Прижиматься к твоим площадям,
Я уже постарел, ссутулился,
Потерял уже счет годам...
\smallskip
А твои каштаны дремучие,
Паникадила весны —
Все цветут, как и прежде могучие,
Берегут мои детские сны.
\smallskip
Я хожу по родному городу,
Как по кладбищу юных дней,
Каждый камень я помню смолоду,
Каждый куст — вырастал при мне.
\smallskip
Здесь тогда торговали мороженым,
А налево — была каланча...
Пожалей меня, Господи, Боже мой!...
Догорает моя свеча!...
\restorecr
\end{multicols}
