% vim: keymap=russian-jcukenwin
%%beginhead 
 
%%file 11_12_2020.news.ru.vesti.2.krym_ukraina_sila
%%parent 11_12_2020
 
%%url https://www.vesti.ru/article/2497719
 
%%author 
%%author_id 
%%author_url 
 
%%tags 
%%title Бессмертный захотел силой вернуть Крым Украине
 
%%endhead 
 
\subsection{Бессмертный захотел силой вернуть Крым Украине}
\label{sec:11_12_2020.news.ru.vesti.2.krym_ukraina_sila}
\Purl{https://www.vesti.ru/article/2497719}

\ifcmt
pic https://cdn-st1.rtr-vesti.ru/vh/pictures/xw/307/778/1.jpg
\fi

Вернуть Крым силой призвал Киев бывший украинский представитель на переговорах
по Донбассу, экс-посол Украины в Белоруссии Роман Бессмертный. Дипломат
подчеркнул, что в первые пять лет после потери Крымского полуострова Киев еще
мог восстановить над ним контроль мирной реинтеграцией. По его словам, сейчас
остался лишь единственный способ возвращения утраченных территорий – силой.

Сначала Киеву необходимо отселить с полуострова всех, кто лоялен, заявил в
эфире крымско-татарского телеканала ATR украинский дипломат. Далее Роман
Бессмертный предлагает возвращать Крым на Украину построением армии, оборонного
комплекса и планирования военных операций. Ранее Роман Бессмертный планировал
вернуть Крым через Минск, задействовав проживающих в Белоруссии этнических
украинцев.

Возвращение Крыма и действия по осуществлению этого плана – популярная тема для
украинских политиков, отмечает РИА Новости. Действующий министр иностранных дел
Украины Дмитрий Кулеба считает, что Крымский полуостров сам "выскользнет" из
рук России и станет украинским.

Евгений Яранцев, вступивший в 2014 году в батальон "Айдар" (организация
запрещена в России как террористическая), с возвращением Крыма решил не
связываться, а просто продать крымские земли американцам. Видимо, он не знаком
с определением понятия мошенничество.

О планах "присоединения" Крыма к Украине с помощью троллейбусного маршрута до
Керчи размышлял в недалеком прошлом глава Генического района Херсонской области
Александр Воробьев.

Более сложные комбинации выстраивает украинский журналист Айдер Муждабаев. Он
предложил воссоздать Великое княжество Литовское, присоединив к нему Польшу,
Украину и часть российской территории.
