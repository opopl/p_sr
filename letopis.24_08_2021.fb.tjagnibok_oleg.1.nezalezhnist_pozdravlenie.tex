% vim: keymap=russian-jcukenwin
%%beginhead 
 
%%file 24_08_2021.fb.tjagnibok_oleg.1.nezalezhnist_pozdravlenie
%%parent 24_08_2021
 
%%url https://www.facebook.com/oleh.tiahnybok/posts/4253232851391616
 
%%author Тягнибок, Олег
%%author_id tjagnibok_oleg
%%author_url 
 
%%tags nezalezhnist,partia_svoboda,pozdravlenie,ukraina
%%title З 30-ю річницею відновлення Незалежності України!
 
%%endhead 
 
\subsection{З 30-ю річницею відновлення Незалежності України!}
\label{sec:24_08_2021.fb.tjagnibok_oleg.1.nezalezhnist_pozdravlenie}
 
\Purl{https://www.facebook.com/oleh.tiahnybok/posts/4253232851391616}
\ifcmt
 author_begin
   author_id tjagnibok_oleg
 author_end
\fi

З 30-ю річницею відновлення Незалежності України!

Миру і щастя на СВОїй, БОгом ДАній землі!

Саме так – відновлення нашої Незалежності. Бо Україна не вчора народилася.
Покоління й покоління українців мріяли та боролися за цю Незалежність. Сотні і
сотні тисяч українських родин. Серед них і моя родина.

Візьмімо книжку «Самостійна Україна». Її автор – ключовий ідеолог і практик
українського націоналізму Микола Міхновський. Книга написана у 1900 році, а
вперше видає її один з моїх предків – Льонгин Цегельський, один з
батьків-засновників Західноукраїнської Народної республіки. А ще він автор Акта
Злуки 22 січня 1919 року.

\ifcmt
  pic https://scontent-cdt1-1.xx.fbcdn.net/v/t1.6435-9/238356924_4253229311391970_2022811924912117283_n.jpg?_nc_cat=101&_nc_rgb565=1&ccb=1-5&_nc_sid=8bfeb9&_nc_ohc=VEc5ENrj_MsAX9zE2Qo&_nc_ht=scontent-cdt1-1.xx&oh=8522c6e209faf5707a63ea3e3a705fa3&oe=614C7CF6
  width 0.7
\fi

Цитую: «Державна самостійність єсть головна умова існування нації...», далі
інші актуальні цитати Міхновського по пам’яті: «Без сили нема права, щоб
здобути право, треба мати силу», і найважливіше сьогодні: «Головна причина
нещасть нашої нації – брак націоналізму серед ширшого загалу її».

Хіба не є ці слова надзвичайно актуальні і сьогодні, через 120 років? І я вже
не кажу про крилату цитату Міхновського: «Україна для українців». І на 30-му
році відновлення Незалежності хочу спитати: а якщо Україна не для українців, то
для кого?

А хіба наше покоління не боролося за Україну? Так. Голодували студентами на
граніті… Агітували за Незалежність перед референдумом 1991-го…  Виходили на
Майдани… А багато хто за той час і життя поклав… Це типова історія майже кожної
української родини. І саме тому наші діти і внуки, а в когось і правнуки,
ростуть уже в незалежній Україні. На своїй, Богом даній землі.  

І це має тривати вічно – Україна незалежна, вільна і сильна.

Що для цього потрібно? Насамперед чітко сказати усьому світові, ким ми з вами
є. Ми – не стара совєтська колонія з умовними українськими ознаками. Ми –
національна держава українського народу, що постала як результат тисячолітньої
традиції української державності. Ми – правонаступники Української Народної
Республіки, а не УССР, яка була окупаційною адміністрацією. І це необхідно
НЕГАЙНО закріпити у законодавстві.

Що нам це дасть? По-перше, розірвемо всякі «зв’язки» із Совєтським Союзом та
«рускім міром» в головах усіх українців. По-друге, повністю відмежуємось від
тих злочинів, які комуністичний режим чинив у нас на території в час окупації –
з 1921 по 1991 рік.

А по-третє... Мабуть, здивуєтесь – гроші. Ми зможемо висунути Московії вимоги
компенсацій за злочини, скоєні тоталітарним комуністичним режимом проти
українців під час окупації 1921–1991 років.

А ще зможемо змусити Московію відшкодувати втрачені заощадження громадян
України, передати належну нам частку Алмазного фонду, золотих та валютних
запасів і закордонного майна колишнього Совєтського Союзу. Законопроект, який
«Свобода» давно підготувала і зареєструвала у Верховній Раді, дозволить усе це
зробити. Залишилось тільки взяти та проголосувати.

Ми завершимо цю боротьбу. Завершимо остаточною перемогою. Щоб наші діти, і
внуки, усі наступні покоління українців уже не мусили боротися за Україну, а
щасливо жили і самі творили своє майбутнє. 

На СВОїй, БОгом ДАній землі. 

Ще раз зі святом, українці!

\ii{24_08_2021.fb.tjagnibok_oleg.1.nezalezhnist_pozdravlenie.cmt}
