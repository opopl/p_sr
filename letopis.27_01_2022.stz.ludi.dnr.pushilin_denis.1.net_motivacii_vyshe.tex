% vim: keymap=russian-jcukenwin
%%beginhead 
 
%%file 27_01_2022.stz.ludi.dnr.pushilin_denis.1.net_motivacii_vyshe
%%parent 27_01_2022
 
%%url https://denis-pushilin.ru/press/denis-pushilin-net-motivatsii-vyshe-chem-u-teh-kto-zashhishhaet-svoyu-zemlyu-svoi-semi-i-svoi-doma
 
%%author_id 
%%date 
 
%%tags donbass,ukraina,vojna,pushilin_denis
%%title Денис Пушилин: Нет мотивации выше, чем у тех, кто защищает свою землю, свои семьи и свои дома
 
%%endhead 
 
\subsection{Денис Пушилин: Нет мотивации выше, чем у тех, кто защищает свою землю, свои семьи и свои дома}
\label{sec:27_01_2022.stz.ludi.dnr.pushilin_denis.1.net_motivacii_vyshe}
 
\Purl{https://denis-pushilin.ru/press/denis-pushilin-net-motivatsii-vyshe-chem-u-teh-kto-zashhishhaet-svoyu-zemlyu-svoi-semi-i-svoi-doma}

\begin{zznagolos}
Глава Донецкой Народной Республики Денис Пушилин в интервью
информационно-аналитическому изданию Украина.ру рассказал о ситуации на линии
соприкосновения в Донбассе, последствиях украинских провокаций в медийном
пространстве и настроении в рядах Народной милиции ДНР. 	
\end{zznagolos}

– Денис Владимирович, какова в настоящий момент военная ситуация на линии
соприкосновения?

– Мы видим, что к линии соприкосновения продолжается стягивание сил и средств
противника. Вооруженные формирования Украины наращивают численность личного
состава, а также вооружения.

\ii{27_01_2022.stz.ludi.dnr.pushilin_denis.1.net_motivacii_vyshe.pic.1}

Украинское командование размещает военную технику, в том числе
бронетранспортеры, БМП, зенитные ракетные комплексы, разведывательно-дозорные
машины, гаубицы вблизи жилых домов в Николаевке, Новоигнатовке, Майорске,
Староигнатовке, Павлополе, Кирилловке и других населенных пунктах,
подконтрольных киевской власти. И эти факты подтверждаются отчетами СММ ОБСЕ.

В район населенного пункта Красноармейск переброшены реактивные системы
залпового огня «Смерч» и РСЗО «Ураган». Сотни тонн летального вооружения
Украина получила от стран НАТО, прибывают группы инструкторов из США и
Великобритании. Показательно также, что на территорию линии соприкосновения
перебрасывают украинских военных, до этого прошедших подготовку у инструкторов
из Великобритании.

Все это происходит на фоне того, что в переговорном процессе по-прежнему
наблюдается тупиковая ситуация.

Если говорить о количестве обстрелов, то на данный момент ситуация относительно
спокойная, мы не фиксируем их увеличение.

– Насколько вероятно украинское наступления? Есть ли данные разведки на этот
счет?

– Действия Украины на линии соприкосновения говорят о том, что ВФУ готовятся к
наступательным действиям. Ситуация такая, что даже наблюдатели ОБСЕ
подтверждают перемещения украинской техники.

У нас есть данные о готовящихся украинских провокациях, цель которых –
поставить под удар системы жизнеобеспечения прифронтовых территорий и наших
предприятий с опасным производством. Надеемся, что огласка этих планов
остановит диверсантов.

В подразделениях противника приостановлены отпуска для личного состава, из
госпиталей досрочно выписывают механиков-водителей и наводчиков.

К линии соприкосновения со стороны Украины стягиваются самоходные реактивные
установки разминирования УР-77, идет снабжение переносными установками
разминирования УР-83П. Это говорит о том, противник готовится к проделыванию
проходов в минно-взрывных заграждениях для наступательных действий.

Мы наблюдаем прибытие на железнодорожные станции Красноармейска, Дружковки,
Зачатовки эшелонов с горюче-смазочными материалами и боеприпасами. При этом для
маскировки перемещение производится в ночное время суток.

Большая часть украинских провокаций сегодня перенеслась в медийное пространство
и связана с заявлениями о якобы российской агрессии. Искусственно создаваемый
поток дезинформации привел к панике среди украинского населения, проблемам
инвестиционного характера. Отзыв дипмиссий ряда стран подлил масла в огонь. В
связи с чем украинским властям пришлось дезавуировать собственные утверждения,
заявлять, что подтверждений возможного нападения на Украину не нашли.

Будет Украина наступать или нет – зависит от решений, принимаемых не в этой
стране, а теми, кто ею управляет и играет ключевую роль в сегодняшних событиях.

– Какое настроение в рядах Народной милиции ДНР?

– Сегодня подразделения Народной милиции – это эффективные, организованные,
слаженно действующие части. Мы постоянно проводим тактико-строевые,
тактико-огневые занятия, различные учения для личного состава, прорабатываем
варианты развития событий, и всегда готовы адекватно ответить противнику. Кроме
того, к оперативному реагированию готовы наши экстренные службы и профильные
ведомства.

Что касается настроения, тут можно сказать следующее: нет мотивации выше, чем у
тех, кто защищает свою землю, свои семьи и свои дома
