% vim: keymap=russian-jcukenwin
%%beginhead 
 
%%file 13_10_2021.fb.menendes_enrike.1.donbass_identichnost.cmt
%%parent 13_10_2021.fb.menendes_enrike.1.donbass_identichnost
 
%%url 
 
%%author_id 
%%date 
 
%%tags 
%%title 
 
%%endhead 
\subsubsection{Коментарі}

\begin{itemize} % {
\iusr{Энрике Менендес}
Картинка вам, чтобы не быть голословным

\ifcmt
  ig https://scontent-frt3-1.xx.fbcdn.net/v/t39.30808-6/244752603_6278243732218000_2500055396869254737_n.jpg?_nc_cat=104&ccb=1-5&_nc_sid=dbeb18&_nc_ohc=Cz4rBIg737gAX8d6tOA&_nc_ht=scontent-frt3-1.xx&oh=2add01d9cb111c3fb665e4d888927dc8&oe=61A47260
  @width 0.4
\fi

\begin{itemize} % {
\iusr{Иван Шашлов}
Красивое...

\iusr{Энрике Менендес}
\textbf{Иван Шашлов} Да, но НЕ НАШЕ!

\iusr{Энрике Менендес}
\textbf{Иван Шашлов} Я за то, чтобы в Ивано-Франковске художник из Авдеевки нарисовал трубы кокосхима, только врядли такое произойдёт.

\iusr{Stanislav Tsykalovskyi}
Красивый мурал. Побольше бы таких.

\iusr{Иван Шашлов}
\textbf{Энрике Менендес} ну, трубы - это б-гомерзко же и не лепо.
Их и здесь же остаться не дОлжно, только кавьярни и барбершопы.

\iusr{Rostyslav Pelekhovych}

Энрике, а яке ваше? як виглядали селянки на "Донбасі"?
> Я за то, чтобы в Ивано-Франковске художник из Авдеевки нарисовал трубы кокосхима
це було б релевантно, якби в ІФ ходили в селянському одязі. Але ні

\iusr{Ism Vitaly}
\textbf{Энрике Менендес} Ну да. Коксохим то он аутентичнее будет.
Пусть он отравляет не только воздух но и радует глаз на мурале.
И написать "Товарищ! Твой завод - твоя гордость"
 · Ответить · 6 нед.
\iusr{Olha Vesnianka}
\textbf{Ism Vitaly} Жесть. Або відстійник зобразити

\iusr{Анатолий Наседкин}
\textbf{Rostyslav Pelekhovych} як даунбаській "іспанєць" тільки з "длінимі валасамі" )))

\iusr{Лида Радкевич}
\textbf{Энрике Менендес} что красивого в трубах коксохима?

\iusr{Анатолий Наседкин}
\textbf{Энрике Анатольевич Менендес} звісно ж не ваше. Ваше це якийсь "моторолєр" з баяном чи "ґіві" з пляшкою піва-Дєсант ))).


\iusr{Энрике Менендес}
\textbf{Stanislav Tsykalovskyi} Да и терриконы во Франковске тоже было бы круто нарисовать. Как думаешь, будет красиво?

\iusr{Stanislav Tsykalovskyi}
\textbf{Энрике Менендес} я не люблю терриконы.
Степь люблю, байбаков люблю. Если байбака зафигачить - это будет наш, донбасский стиль?

\iusr{Тая Дунаева}
\textbf{Энрике Менендес} красиво.. но я не уверенна даже о слобожаночке.. это разве их узоры?

\iusr{Rostyslav Pelekhovych}
Энрике, ну так що саме на муралі, окрім слова «слобожанка», не відповідає ідентичності мешканців науколишніх сел ?

\iusr{Евгений Афендиков}
Там некоторые и не знают что такое террикон ы, были приезжие спрашивали - а что за горы у вас.

\iusr{Vitalii Ovcharenko}
\textbf{Rostyslav Pelekhovych} В робе - фуфайке с номером на спине надо було нарисовать, вот тогда для Енріке було б хорошо))))

\iusr{Игорь Рязнов}
да какая это слобожанка))) на узбечку больше похожа))))

\iusr{Владимир Кабасин}
\textbf{Энрике Анатольевич Менендес}, ВАШЕ! Ты в Украине живёшь! Прекрати её делить!

\iusr{Aleksej Vasiliev}
\textbf{Энрике Менендес} Девушка таки лучше труб кокосхима.

\iusr{Aleksej Vasiliev}
\textbf{Евгений Афендиков} Терриконы не тянут на предмет гордости.

\iusr{Aleksej Vasiliev}
\textbf{Энрике Менендес} Вы трубу за девушку? Странный обмен любезностями.))

\iusr{Городецкая Марина}
\textbf{Евгений Афендиков},спи*дел? а так хочется, чтобы выглядело, как взаправду?

\iusr{Городецкая Марина}
Мандес, ты ватный ублюдок всячески стараешься, чтобы пуйлу понравиться?

\iusr{Евгений Афендиков}
\textbf{Aleksej Vasiliev} 

терриконы это часть нашего окружения, у когото гималаи и карпаты, а у нас
терреконы, кстати у некоторых есть названия. и как люди лазят на говерлу, мы
пацанами лазили на терриконы.


\iusr{Александр Балдынюк}
А тысячи солдат росармии и рос танки на Донбассе - это ваше?

\iusr{Сергей Бондаренко}
\textbf{Энрике Анатольевич Менендес} - автор явно на подтанцовках у россиянских пропагандонов. "народ дамбаса", "дамбасянская идентичность"..
Явно подзабыл сей "испанец", как голодом выморили украинцев на этой территории, и навезли туда эту "идентичность" из северных краёв.
\end{itemize} % }


\iusr{Энрике Менендес}
Делитесь новостью. Пишите тэг \textbf{\#Я\_это\_Донбасс}

\begin{itemize} % {
\iusr{Ievgen Nefediev}
\textbf{Энрике Анатольевич Менендес} \textbf{\#ПішовНахуйМенендес}

\iusr{Илья Петиченко}
\textbf{Энрике Анатольевич Менендес} 

тегнули два человека. Дончане не думают о своей идентичности. Те дончане
которые поддержали псевдореферендум думали не о создании новой республики, а
том чтобы с Донбассом сделали как с Крымом - примазали к нефтяной экономике в
качестве младшего родственника. Вы занимаетесь ерундой.

\end{itemize} % }

\iusr{Энрике Менендес}
\textbf{Денис Казанский} 

давно тебя хотел спросить - почему ты так ненавидишь Донбасс? Это же явная
атака на нашу региональную идентичность, в которой нет ничего плохого. Она есть
в любом другом регионе страны - где-то больше, где-то меньше

\begin{itemize} % {
\iusr{Денис Казанский}
\textbf{Энрике Менендес} не могу понять, это стеб, или у тебя крыша потекла?

\iusr{Юрий Лукшиц}

Казанский просто хочет украинизироваться. Он не понимает, что для националистов
и "патриотов" он никогда своим не станет. И служить он будет любой власти.

\iusr{Serg Kosinov}

Ленин называл Горького "наш путаник". Энрике милый путаник. Но когда количество
этих милых путаников превышает некую критическую величину, начинается война.

\iusr{Dima Schors}

Энрике, Казанский проявил себя еще в самом начале - в тексте про мертвого бойца
ЗСУ на проводах. К чему эти ваши, Энрике, экивоки?)))

\iusr{Анатолий Наседкин}
\textbf{Юрий Лукшиц} 

ДБЙБ, просто будь-яка притомна людина розуміє що "труби коксохіма" не можуть
бути будь-якою ідентичністю. Може ви себе й ідентифікуєте як "мєталурґі" чи
"шахтьори", але навіть притирене керівництво донбабве намагається замінити
українську ідентичність на "раіскую" ))). Балалайка і водка зараз в тренді на
даунбасі, звісно коли "герої" вийдуть з цехів якогось "коксохіму". Ферштейн?

\iusr{Энрике Менендес}
\textbf{Анатолий Наседкин} А что может быть идентичностью, любезный Вы наш? Борщ?

\iusr{Энрике Менендес}
\textbf{Денис Казанский} Это прямой вопрос, от которого ты ушёл. С крышей у меня всё в порядке во всех смыслах.
Кстати, я заметил твой милый стук в СБУ. Скриншоты твоих показаний мне прислали )

\iusr{Сергей Шабашов}
\textbf{Энрике Менендес} странно, войска РФ вас не смущают, а мурал в Авдеевке вас смущает.

\iusr{Энрике Менендес}
\textbf{Анатолий Наседкин} Ещё раз говоришь слово "даунбас" и уходишь в бан. ДБЙБ.

\iusr{Анатолий Наседкин}
\textbf{Энрике Анатольевич Менендес} Вам як майже "европецю" та справжньому донбасцю, скажу - якби ж ви вчились так як треба то не писали б дурні. Весь ваш "допис" то суцільна демонстрація невігластва.

\iusr{Анатолий Наседкин}
\textbf{Энрике Анатольевич Менендес} звісно, будь яка національна кухня є ознакою та складовою ідентичності. Що ви пишете? Не грузіть ні себе ні людей, зупиніться, не робіть з себе посміховисько.

\iusr{Александр Лапин}
\textbf{Энрике Менендес}, есть люди, у которых место сидения определяет точку зрения. Было бы странно требовать от блоггера на содержании власти критики политики этой самой власти...

\iusr{Венди Белова}
\textbf{Денис Казанский} а раньше не текла?

\iusr{Игорь Рязнов}
у казанского просто работа такая))) деньги он на этом зарабатывает!!!!

\iusr{Константин Исиков}
\textbf{Денис Казанский} это у тебя крыша потекла, ЧМ0

\iusr{Олег Платонов}
\textbf{Энрике Анатольевич Менендес} 

нет никакой региональной идентичности!, Особенно Донбасса... Донбасс это
географический регион и не нужно сочинять. Большинство людей свезли с разных
регионов для проведения индустриализации, а вы про идентичность... Не знаю под
чем вы, но из-за таких как вы в нашей стране война и каждый день гибнут люди..

\iusr{Галина Борисова}
\textbf{Олег Платонов} если вы не из Донбасса, то вы просто не знаете, о чём говорите. Как сказал мне один человек: "Донецкий - это бренд! "

\iusr{Олег Платонов}
\textbf{Галина Борисова} ну-ну, и где ваш Донбасс? Доидентичелись? Считаете себя особенными... Так каждый человек может сказать, что он особенный - и будет прав, только у него всё так же две руки, две ноги и кушать хочется..

\iusr{Natasha Semergey}
\textbf{Энрике Менендес} 37 й год форева ! Времена меняются, а сущность людей...

\iusr{Любовь Головко}
\textbf{Энрике Менендес}  @igg{fbicon.thinking.face}{repeat=3}  а в чем проявляется лично ваша любовь к Донбассу, начиная с довоенного периода?

\iusr{Игорь Афендиков}
\textbf{Галина Борисова} 

Вы правы. Сама фраза несёт глубокий смысл. "Донецкий" отвечает на вопрос чей?
Донецкий уголь, донецкий коксохим, донецкое пиво или донецкая мафия. В данном
выражении имеется ввиду донецкий раб? Или холоп? Принадлежность чего-то, не
одушевлённого, кому-то. Дончанин это более верный вариант.


\iusr{Rostyslav Pelekhovych}
> Как сказал мне один человек: "Донецкий - это бренд! "
Це був Янукович, Галина?

\iusr{Сергей Голован}

да не ненавидят они Донбасс, бесполезно таким обьяснять про любовь и ненависть,
люди за деньги работают, завтра заплатят будут горячо любить

\iusr{Энрике Менендес}
\textbf{Любовь Головко} 

Мне легко ответить на этот вопрос - хотя я много раз уже рассказывал.

1) Вы наверное знаете, что в Донецке у меня был бизнес - агентство по
интернет-рекламе. В её миссии ещё с 2010 года было записано "развивать
маркетинг в Донецке". Меня цепляло, что в таком крупном городе нет ни одного
партнёра Гугла или Яндекса. Я задался целью и в 2012 году моё агентство стало
первым в Донецке сертифицированным партнёром Гугл и Яндекс.

2) Я платил налоги. Моя фирма работала в белую.

3) Я развивал своих сотрудников. Обучал их за счёт компании. Сейчас почти все они переселенцы и благодарят меня за полученные знания, что помогло им найти хорошую работу.

4) Я дважды выступал спонсором и соорганизатором донецкого TEDx. Один раз выступал как спикер - без гонорара.

5) Я вёл для студентов курс по саморазвитию в ДонНУ. Бесплатно. Правда успел выступить только раз - это была зима 13-го ((

6) Выступал как лектор в донецком клубе предпринимателей. Бесплатно.

7) Моя жена была активным членом сообщества мам kroha.dn.ua. Мы постоянно занимались благотворительностью.

\iusr{Городецкая Марина}
\textbf{Serg Kosinov}, он не "путаник", он ватник и коллаборант. Работает на пуйла, однозначно...

\iusr{Галина Борисова}
\textbf{Rostyslav Pelekhovych} нет, это был человек, который под обстрелами ездил по линии разграничения и кормил стариков.

\iusr{Любовь Головко}
\textbf{Энрике Анатольевич Менендес} 

не могу принять этот список, как доказательство вашей особой любви к Донбассу.
Многие из живущих в Донецке (как и в других городах, поселках, селах Донбасса)
могут написать аналогичный список. Денис Казанский в том числе....

Вы еще и компьютерной техникой занимались, не так ли?  @igg{fbicon.wink} 

Да и некоторые другие вехи вашей деятельности вы в перечень "любви к Донбассу"
не включили... Понимаю ваше желание дистанцироваться от некоторых моментов...

А Донбасс, он разный, у его жителей разные взгляды, мнения и приоритеты и...
своя любовь малой родине... И никто не может претендовать на исключительность
своей любви...

p.s. относительно ваших возмущений красивым муралом...

Почитайте книгу "История Харьковского Слободского козачьего полка 1651-1765" Е.
А. Альбовского. В сети есть сканы издания 1895 или 93 года...

К слову, Юзовка (и значительная часть Донецкого кряжа) долгое время входила в
состав Екатеринославской губернии )))

\iusr{Olga Alunci}
\textbf{Олег Платонов} есть региональная идентичность.

\iusr{Вадим Щетинин}
\textbf{Денис Казанский}, разрешите Энрике выложить в чат скрины Вашего "стука" в СБУ.
"Кстати, я заметил твой милый стук в СБУ. Скриншоты твоих показаний мне прислали )"

\iusr{Mykola Malukha}
\textbf{Энрике Анатольевич Менендес}, 

так чего слился с ответа по поводу идентичности Донбасса? В чем идентичность
Донбасса? Нытье "мы кормим всю Украину", говорить на безграмотном русском
языке, работать шахтером, пьянствовать после смен, молиться на завод и не
мыслить своей жизни вне завода.... что составляет идентичность Донбасса?

\iusr{Олег Платонов}
\textbf{Olga Alunci} 

где-то есть(Волынь, закарпатье, но не на Донбассе.. менталитет - согласен, но
идентичность??? Нет такой народности, нет языка, даже диалекта( не считая
фени), нет обычаев и нет традиций... Так о какой идентичности идёт речь? О том
что все выросли у терриконов и у всех есть в семье шахтеры? Я понимаю, что с
Украиной вы себя не ассоциируете и судя по всему с Россией тоже, не смотря даже
на пропаганду.. я понимаю, что вы ищите идентичность, но ее нет, скажите
спасибо СССР и рос пропаганде, ну и конечно же властям в Украине за всё время
независимости, вас так и не интегрировали ментально...

\iusr{Любовь Головко}
\textbf{Oleg Platonov} 

не во всех в семьях есть шахтеры. @igg{fbicon.smile}  И терриконы есть не везде... И жители
Донбасса очень разные по национальности, вере, ментальности, культурным
предпочтениям, уровню образования и т.д.

Огромная вина за то, что произошло - на ПР. Сейчас большинство из регионалов,
благодаря которым все и случилось, - в подконтрольной Украине, и очень
комфортно себя там чувствуют. У них есть деньги, связи... Их дети-внуки
спокойно учатся в украинских школах и вузах... И плевать им на войну, на
проблемы региона, на семьи, которые оказались по разные стороны линии раздела,
на горе и боль своих земляков...

Вот и вся идентификация... @igg{fbicon.frown} 

\iusr{Олег Платонов}
\textbf{Lyubov Golovko} согласен с вами.. у меня тоже есть корни с Донбасса и семья жены - переселенцы

\iusr{Валерий Антонов}
\textbf{Олег Платонов} вы пытаетесь говорить о том, чего не знаете. Вы вообще были в тех краях о которых идет разговор ? Думаю что, нет..

\iusr{Сергий Федорынчык}

Енрике, на питання - в чому саме виявляється ідентичність? - Ви мусите дати
власну відповідь, бо підняли цю тему, а не запитувати у інших.

\iusr{Сергий Федорынчык}
\textbf{Энрике Анатольевич Менендес} 

Прочитати це цікаво і корисно для кращого розуміння Вашої особистості, бо під
час нашої короткої зустрічі Ви дуже мало про це згадували. Разом з тим, Ваші
позитивні якості можуть виявитись надто недостатніми для неймовірно складного
завдання - досягнення миру на Донбасі. І спроба створення політичної партії
скоріше віддалить Вас від мети, ніж наблизить. Я ще раз нагадую про Вашу
обіцянку - написати цілісний виклад Ваших поглядів на проблеми Донбасу та шляхи
їх розв'язання. Це абсолютно необхідно для розвитку громадського руху за
порозуміння та мир на Донбасі. Без порозуміння не буде й миру, навіть якщо
війна набридне жителям ОРДЛО. Бо Кремль від цього лише перестане прикриватись
думкою цих людей у своїх дальніх намірах відновити Російську Імперію. Ви
стільки разів дорікали Україні, що вона не чує Донбасу (зокрема, у Вашому
озвученні). Але що саме Ви хочете сказати Україні - досі так і неясно, якщо
написати цього Ви не в змозі. Цікаво, чи чують Вас у Кремлі? - принаймні,
кремлянам Ви не дорікаєте! Ще простіше запитання - чи чують Вас на Донбасі,
зокрема в ОРДЛО, крім Ваших друзів, знайомих та читачів Вашої сторінки у
Фейсбуці?

\iusr{Валерий Харламов}
\textbf{Энрике Анатольевич Менендес}
Ну, а теперь зачем занимаешься херней?

\iusr{Борис Зленко}
\textbf{Oleg Platonov} 

язвк русский. Диалект своеобразный - надо быть родом из Луганска, чтобы понять.
У нас есть русские особые, слова которые не употребляются в той же Полтаве
например. Словарь русского языка Владимир Даль написал уроженец Донбасса.
Традиции у нас есть. Отлтчающиеся от Львовских.

\end{itemize} % }

\iusr{Энрике Менендес}
\textbf{Сергей Сивохо} 

а ты как к такому относишься? Мы долго ещё будем молчать, пока убивают нашу
идентичность?

\begin{itemize} % {
\iusr{Владимир Кабасин}
\textbf{Энрике Анатольевич Менендес}, ты, я вижу, такой борец за мир и единство в Украине, что тебе иззавидуется любой «ополченец».
КТО?
УБИВАЕТ?
ТВОЮ?
ИДЕНТИЧНОСТЬ?
Кроме тебя самого.

\iusr{Марк Зоряний}
\textbf{Владимир Кабасин} он фашист на зарплате РФ


\iusr{Владимир Кабасин}
\textbf{Марк Зоряний}, 

вот с этим я не согласен. То, что полезный для целей Кремля дурачок —
соглашусь. То, что заблудившийся в реальности, склонный выгораживать
землячество человек — соглашусь. Тем, кстати, Кремлю и полезен, потому что так
удобнее же делить государство — Донбасс отдельно, а Слобожанщина — отдельно.
Львов — вообще Польша, не наше это, типа.

Энрике, когда про самобытность «народа Донбасса» говорят, то что конкретно
имеют ввиду? Есть национальная кухня Донбасса? Национальная одежда? Может быть
есть народные песни Донбасса (пару советских песен из репертуара Кобзона и
Бернеса не предлагать)? Или танцы?

Какую именно идентичность можно на Донбассе «убить»?

\iusr{Сергей Перевозчиков}
\textbf{Энрике Менендес} вашу идентичность убили в 2014 году.

\iusr{Андрій Винниченко}
\textbf{Володимир Кабасін} Он стопроцентный сепаратист, у него сепаратистская идентичность. Потому его и в ДНР считают предателем, поскольку деятели ДНР отоджествляют себя с Россией, считают себя русскими патриотами, а ДНР частью Великой России.

\iusr{Olga Alunci}
\textbf{Сергей Перевозчиков} и ее продолжают убивать в Украине. Мендес все правильно написал.

\iusr{Андрій Винниченко}
\textbf{Марк Зоряний} Фашист не может быть на зарплате. Фашисты были идейными как и коммунисты.

\iusr{Марк Зоряний}
\textbf{Андрій Винниченко} а он фашист за гроші, які не пахнуть. Завтра заплатять він буде прихильником визвольного руху ОУН -УПА. Він по духу поліцай

\iusr{Светлана Перепелкина}
Сергей, они убили свою идентичность в 14году, призвав путина в Украину.
\end{itemize} % }

\iusr{Энрике Менендес}
\textbf{Андрій Єрмолаєв} ты только сегодня писал про Донбасс. Полюбуйся. Может я с ума сошёл, как некоторые пишут? И нет никаких поводов для возмущения?

\begin{itemize} % {
\iusr{Андрій Єрмолаєв}

Одна из целей когнитивной войны, как составляющей т.н. "многосферной стратегии"
или "операции", - смещение и трансформация смыслов, привычных и значимых для
социального уклада. Донбасс - это уклад жизни, сложившийся в регионе в
определенном историческом процессе и контексте. А не территория с терриконами и
не "масса населения". Разрушение Донбасса, как и его "рождение" в связи с
возникшим "сопротивлением" - связано с этим укладом и реакцией на попытки его
разрушить. Помнишь попытки переучить "шахтеров - в предприниматели?" А
использование реформы угольной отрасли как способа "закрыть неэффективное" и
"перепрофилировать"? А бандитские времена, которые более десятилетия как бы "не
замечали", и которые оставили неиздадимый след на истории, имидже и стереотипах
о Донбассе? Думаешь, случайно? А я думаю, нет. Уже тогда болялись. Особенно
после событий 1989-1993гг. Вот тогда еще все началось. А сейчас - это уже
попытки закрыть тему окончательно. Донбасс, превращенный просто в обеднеывшее и
дезориентированное население, - это и есть цель войны. А вот население как
ресурс можно делить, "мигрировать", переобустраивать и т.д. Так что тут все три
стороны в игре (Киев, Москва, Большой Запад). Никому из них не нужен
амбициозный промышленный регион, с пролетарскими замашками, с запросом на
индустриальный уклад, образование и т.д. и т.п. Ну а вся эта сбежавшая,
набившая карманы на Донбассе "рег-элита" - случайные люди, игроки в казино,
просто проходимцы. И как видно теперь, без корней (настоящих корней) и без
совести. Но с самим Донбассом еще страница не перевернута...

\iusr{Алексей Зубов}
\textbf{Андрій Єрмолаєв} , о да-да: мы такие ох@енные, а нас никто не любит и «не слышит». А откуда все эти «рег-элиты», януковичи, бандиты? А хрен знает, «случайные люди». Мы за них, конечно голосовали килотоннами, но вообще их рептилоиды подбросили, да. «Стереотипы» и вот это вот всё.
То есть «донбасская идентичность» никаких рефлексий не предполагает. Мы услышали наконец.
Спасибо жителям Донбасса, так сказать, дуже Дякую

\iusr{Олег Платонов}
\textbf{Андрій Єрмолаєв} поплава


\iusr{Энрике Менендес}
\textbf{Андрій Єрмолаєв} Шикарный комментарий, Андрей. Спасибо!

\iusr{Ольга Сидорова Сидорова}
\textbf{Андрій Єрмолаєв} ты примитивный прокремлевский пропагандист расскажи про хунту и про фашистов и про мураевский пропагандонский украинофобский канал НАШ который спокойновоняет русским миром и никто этих Беляшей не трогает

\iusr{Сергий Федорынчык}

Ай-яй-яй, Андрію! Цілей не буває без цілепокладання та іх носіів. Хто, де і коли
ставив цілі руйнаціі ідентичності Донбасу?

\iusr{Владимир Владимиров}
\textbf{Андрій Єрмолаєв} "Рождение Донбасса в связи с возникшим сопротивлением" - получается, что это внебрачный, да ещё и брошенный ребенок Бородая и Стрелкова?

\iusr{Оксана Возова}
\textbf{Энрике Анатольевич Менендес} два "українолюба".

\end{itemize} % }

\iusr{Энрике Менендес}
\textbf{Илья Кононов} - Вы ведь тоже об этом недавно писали

\begin{itemize} % {
\iusr{Илья Кононов}

Конечно. У части нашего правящего класса и обслуживающей его интеллигенции есть
желание изменить идентичность нашего региона. Донбассу навязывают Слобожанщину
как замену его самого. Они исторически связаны, но по многим характеристикам
отличаются. Слобожанщина в культуре не дала украинского-русского синтеза, а для
Донбасса - это основа его культурной идентичности. Само самоназвание нашего
региона то пытались заменить на безликое и никого не затрагивающее Подонцовье,
то объявляют вообще "нарративом врага".

\iusr{Илья Кононов}

\href{http://www.ostrovok.lg.ua/knigohranilishhe/nauka/kononov-if-etnos-cinnosti-komunikaciya-donbas-v-etnokulturnih-koordinatah-ukrayini-monografiya}{%
Кононов І.Ф. Етнос. Цінності. Комунікація (Донбас в етнокультурних координатах України). Монографія, %
ostrovok.lg.ua%
}

\iusr{Илья Кононов}

\href{http://geopolitika.am/dir/wp-content/blogs.dir/1/files/2021_3_21_33.pdf}{%
Донбасс в региональной системе Украины, geopolitika.am%
}

\iusr{Станислав Печенкин}

Та нема там ніякого російсько-українського синтезу. Від слова "зовсім". Це
звичайний совок. Густий совок.

\iusr{Андрій Винниченко}
\textbf{Stanislav Pechenkin} 

Совок це люди з радянськими поглядами? Якщо ви маєте на увазі радянських людей,
то у радянських людей не було ненависті до українців та України, Україну вони
бачили УРСР. А цей синтез з Пушиліними, Плотницькими, Захарченками, Моторолами,
Гіві на чолі ненавидить українців і все українське.


\iusr{Halyna Pry}
\textbf{Илья Кононов} йдіть до біса зі Слобожанщини з вашою "культурою' та синтезом. Відповідь з Харкова.

\end{itemize} % }

\iusr{Энрике Менендес}
Людям, у которых нет идентичности, видимо, уже ничем не помочь (

\begin{itemize} % {
\iusr{Anatoly Vlasyuk}
\textbf{Энрике Анатольевич Менендес} Чому б вам не написати про те, що твориться у вас під носом? Я маю на увазі так звану "Ізоляцію" в Донецьку. Оце тема для з'ясування ідентичності на тлі убивств і катувань! Але ж ви не посмієте про це написати, бо самі станете в'язнем "Ізоляції". А всі ваші пости - це для того, щоб розхитувати Україну задля російської окупації.

\iusr{Serg Kosinov}
\textbf{Энрике Менендес} Автор сам не понял, насколько он прав.

\iusr{Алексей Зубов}
\textbf{Энрике Менендес} , вы на вопрос не ответили: условный янукович - это часть вашей идентичности, или, как изволит выражаться Ермолаев, «случайные люди»?

\iusr{Сергей Перевозчиков}
\textbf{Энрике Менендес} да, Вам уже ничем не помочь.

\iusr{Rostyslav Pelekhovych}
Ви пропустили слово, має бути «Людям у которых нет донбасской идентичности ничем не помочь»

\iusr{Владимир Кабасин}
\textbf{Энрике Анатольевич Менендес}, 

погоди, это ты сейчас отстаиваешь вот прямо в этом посте, что у тебя нет
украинской идентичности. Что Украина отдельно, со всякими своими
Слобожанщинами, но Авдеевка — уже нет. Там уже не Украина.

И что это необходимо отстаивать. Сивохо даже на помощь позвал.

Мы все видели «идентичность Донбасса». Итогом правления твоих земляков был
Майдан. Дважды. А потом, не устану повторять, Донбасс выстрелил себе в ноги,
типа, не услышали его. Услышали, в том и дело. Зачем же себе самострел делать,
вот что непонятно. И как теперь выкарабкиваться из этой ситуации, когда часть
Донбасса агрессор использует в собственных интересах и ни при каких условиях,
кроме полного подчинения, добровольно его не отдаст.

Энрике, не будь глупцом.

\iusr{Сергей Основин}
\textbf{Энрике Менендес} суть в том, что людям, у которых нет идентичности, помощь не нужна. Тем более, помощь от тех, у кого идентичность есть.  @igg{fbicon.wink} 

\iusr{Вадим Щетинин}
\textbf{Энрике Менендес} ее, этой идентичности и до 2014 не было, результат на лицо.

\iusr{Вадим Щетинин}
\textbf{Энрике Менендес}, эта тема по муралу просто никакашка.
Попробуйте, что-то боле серьезно, если потяните.
Например такие вопросы, которые поднимались на "Студресрублике".
Вот о чем нужно сподвигнуть думать молодежь.

\href{https://studrespublika.com/buty-razom/}{%
Навіщо нам, таким різним, бути в Україні разом?, studrespublika.com, 06.10.2021%
}

\iusr{Андрей Колосов}
\textbf{Владимир Кабасин} 

Ну, ни о каком полном подчинении никого никому речи нет. Минские соглашения
закладывают некие гарантии субъектам ОРДЛО, поскольку уже было беспрецедентное
решение - АТО.

\iusr{Андрей Колосов}

Энрике, все же о региональной идентичности - нельзя так категорично: хочешь,
имей ее, не хочешь - не имей. Сейчас она обострена нашим положением "все в
одной лодке". А до событий 2014? Насколько она была значима? Ее значение было
ничтожным. Обычный региональный патриотизм: "Шахтер", луганские тепловозы,
донецкие розы и т.п. Такое же во всех регионах имеет место....

\iusr{Марк Зоряний}
\textbf{Anatoly Vlasyuk} 

Ви праві, він за свою ідентичність уявну збирається воювати, але скоріше
підбурювати інших, бо саме чамороше тільки подряпати нігтіками може

\iusr{Владимир Кабасин}
\textbf{Андрей Колосов}, 

ОРДиЛО не субъекты, а объекты, начнём с этого. И, конечно, некая особость этих
частей двух областей Украины будет закреплена законодательно. На период
нормализации. А после будет снята.

Никакого, абсолютно, повода для автономизации у Донбасса нет. Вообще.

\end{itemize} % }

\iusr{Tatyana Malyarenko}

имхо, слобожвнская идентичность еще слабее, чем донбасская  @igg{fbicon.smile}  я бы начинала
беспокоиться, когда девушку в половецком костюме на муралах будут рисовать:)

\begin{itemize} % {
\iusr{Энрике Менендес}
\textbf{Tatyana Malyarenko} Слобожанской идентичности нет вообще. Я ни разу не видел человека с татуировкой "Слобожанщина". А с татуировкой "Донбасс" видел сотни )

\iusr{А. Полтавцев}
\textbf{Энрике Анатольевич Менендес} Так чего ж тогда так переживаете-то?  @igg{fbicon.smile}  Татуировка же круче мурала  @igg{fbicon.smile} 

\iusr{А. Полтавцев}
\textbf{Tatyana Malyarenko} 

Скажите, а что точно не приемлемо для Донбасса: девушка в половецком костюме
или та же самая девушка вообще без костюма? Социологическое исследование
провожу, о хрупкости донбасской идентичности через разрушительное воздействие
изобразительного искусства. А если будет девушка с веслом? А если будет шахтёр
в женском половецком платье? В чём истинная опасность: гендер или одежда, как
считаете?

\iusr{Ольга Сидорова Сидорова}
\textbf{Энрике Анатольевич Менендес} а разве православные ходят с наколками? Скрепненько так)))))

\iusr{Александр Лапин}
\textbf{Энрике Менендес} , насчёт отсутствия слобожанской идентичности, это ты зря. Харьковчане обидятся  @igg{fbicon.wink} 

\iusr{Городецкая Марина}
\textbf{Энрике Менендес},так называемый народ - "донбассята"? это донецкий этнос?

\iusr{Serhiy Yaryhin}
\textbf{Энрике Анатольевич Менендес}
Татуювання - це сильна ознака ідентичності  @igg{fbicon.wink} 
Може ще купола згадаємо?

\iusr{Сергий Федорынчык}

Ах, Енрике! Для людей справді культурних, а такі є у всіх краях Украіни, і на
Донбасі також, татуювання є проявом кримінальноі субкультури.

\iusr{Игорь Рязнов}
\textbf{Сергий Федорынчык} тото сейчас все в США в наколках ходят)))

\iusr{Игорь Рязнов}
\textbf{Городецкая Марина} 

ровно 100 лет назад так же смеялись над украинским и белорусским этносом. А
коммунисты создали Союз включили в состав УССР и БССР в паспорта написали
национальности и уже никто не смеётся, а считают себя щирими украинцами. Так
что история развивается по спирали

\iusr{Aleksej Vasiliev}
\textbf{Энрике Менендес} А я помню "Все будет Донбасс". И умиления это не вызывало.

\iusr{Dmytro Rybakov}
\textbf{Энрике Анатольевич Менендес} тобто дівчину треба було щ татухою в качалці намалювати і все було б "ідентично"?

\iusr{Halyna Pry}
\textbf{Энрике Анатольевич Менендес} а, так і запишемо: наколки на тілі - це ідентичність. Слава Богу, в Слобожанщині дійсно такої ідентичності ніколи не було, бгг.

\iusr{Ievgen Nefediev}
\textbf{Энрике Анатольевич Менендес} 

єбать ти баклан. Якщо вся твоя донбаська ідентичність - це татуювання у зеків,
то твій діагноз зрозумілий. І для довідки щодо ідентичності слобожан -
харків'яни і досі пам'ятають, що Харків був столицею України, а їхній
університет було засновано на початку 19 століття, коли Юз ще не народився,
тому завали. Який пиздець у твоїй голові.

\end{itemize} % }

\iusr{Сергей Прокопенко}
На самом деле мурал должен называться: "Гуцулка Ксеня шукає 10 відмінностей териконів від лисих Карпат..."

\begin{itemize} % {
\iusr{Сергей Сидоров}
\textbf{Сергей Прокопенко} тож і є саме терікони від ч. моря певно.

\iusr{Svitlana Kovalova}
\textbf{Сергей Прокопенко} та где там Западной до авдеевских красот

\iusr{Сергей Прокопенко}
\textbf{Svitlana Kovalova} та конечно... особенно сердце кровью обливается, когда вспоминаешь, как паньство в щахту спускается...

\iusr{Сергий Федорынчык}
В Червоноградв та Нововолинську панство в шахти спускається, в Бориславі та Дашаві нафту та газ видобуває.

\iusr{Сергей Прокопенко}
\textbf{Сергий Федорынчык} і шо там, є мурали донбаських жіночок?) чи 'носіі етнокультури' так справляються?
\end{itemize} % }

\iusr{Сергій Борис}

Яка крихка, однако, ця донбаська ідентичність, якщо мурал "Слобожаночка" здатен
її знищити.

\begin{itemize} % {
\iusr{Энрике Менендес}
\textbf{Сергій Борис} Какая хрупкая украинская идентичность, что признание русского вторым государственным способно её уничтожить

\iusr{Сергій Борис}
\textbf{Энрике Анатольевич Менендес} Давайте розглянемо таку можливість, чому ні. Одразу після звільнення ерефією українських територій, сплати репарацій, публічного каяття за геноцид, репресії і окупацію та офіційного засудження й відмови від злочинної імперської ідеології "руського міра".

\iusr{А. Полтавцев}
\textbf{Энрике Анатольевич Менендес} т.е. донбасская идентичность на самом деле рузкая?  @igg{fbicon.smile}  Труба коксохима это более правильный символ чем медведь с балалайкой? А что Вы ещё знаете о шизофрении?

\iusr{Ольга Сидорова Сидорова}
\textbf{Энрике Анатольевич Менендес} о украинофобиия поперла

\iusr{Станислав Печенкин}

Крихка? Так крихка. А чому крихка? Хто доклав до цього руку? Та той ( чи та
) хто і окупував частину Донбасу. Тобто Росія. Тому нашу мову та культуру
треба відроджувати і всіляко підтримувати. А визнання другою державною
російської мови цьому процесу сприяти аж ніяк не буде.


\iusr{Ольга Сидорова Сидорова}
\textbf{Andriy Poltavtsev} налейте ещё \textbf{Энрике Анатольевич Менендес} стакан водки и он раскроется по полной, начнёт рассказывать как в австрийском генштабе придумали украинский язык, как граф Потоцкий придумал украинцев.

\iusr{Сергий Федорынчык}

Так, Енрике, усі культури та мови в усіх частинах колишньоі Російськоі Імперіі
крихкі та вразливі перед натиском брутального "русского мира" з матюками. Для
жителів Бельгіі дві державних мови означають обовязковість володіння іншою, ніж
своя. Валони володіють фламандською, фламандці - французькою. Практично всі
бельгійці також володіють німецькою та англйською, а культурна людина в Бельгіі
починається з 5-оі мови. Для носіів "русского мира" вимога російськоі як другоі
державноі потрібна лише для того, щоб не вчити інших мов і тихо вважати іх
носіів унтерменшами -"чурками", "хохлами", "бульбашами", "татарвою",
"нациками". На жаль, більшість жителів Донбасу, особливо тих, що залишились на
непідконтрольних територіях, недостатньо знайомі ні з украінською мовою та
культурою, ні з російською. Натомість надзвичайно поширені татуювання та
матюки. Взагалі, Енрике, прошу вибачити за різкість, поки Ви не оволоділи
іспанською, Вам, на мою думку, було б краще не зачіпати делікатну тему
ідентичності. Вважаю, що Вам зараз варто згорнути обговорення, яке найбільше
розбурхує саме тих людей, які найменше до нього готові.

\iusr{Aleksej Vasiliev}
\textbf{Энрике Менендес} Украинская идентичность и правда хрупкая при русском втором. Это трудно не понимать. Вы неудачно ввязались в тему, явно, на мой взгляд, для вас проигрышную. И вам же вредную.

\end{itemize} % }

\iusr{Михайлов Олег}
Энрике, Вы умничка!

\begin{itemize} % {
\iusr{Михайлов Олег}
Сначала люди смеются, потом говорят - может быть, а потом - так это общеизвестно)

\iusr{Марк Зоряний}
\textbf{Михайлов Олег} хуюмничка он, разжигатель розни в обществе. Такие как этот глист в обмороке и стали причиной войны , предавши Украину и звавшие войска Хуйла

\iusr{Михайлов Олег}
\textbf{Марк Зоряний}
Ваш мат вызывает рознь между интеллигентными людьми и ...

\iusr{Городецкая Марина}
\textbf{Михайлов Олег},а ты недоумок

\iusr{Михайлов Олег}
\textbf{Городецкая Марина} Да уж, достойный уровень общения(
\end{itemize} % }

\iusr{Евгений Николаевич}

Это называется "долбаебам нет покоя!!!", в городе эпидемия, нет врачей... Очень
жаль мой город...

\begin{itemize} % {
\iusr{Svitlana Kovalova}
\textbf{Евгений Николаевич} а местная власть есть?

\iusr{Ievgen Nefediev}
\textbf{Евгений Николаевич} Правильно! А замість того, щоб писати про це, Менендес питає у Магомедова, чого мурал цей намалювали.
\end{itemize} % }

\iusr{Дмитрий Золотухин}

Вы бы так свой родной регион от захватчиков защищали в 2014-м  @igg{fbicon.smile} 
Цены б вам не было

\begin{itemize} % {
\iusr{Энрике Менендес}
\textbf{Дмитрий Золотухин} Я защищал. Где Вы были в этот момент не ясно

\iusr{Olga Kosse}
\textbf{Дмитрий Золотухин}, о, диванные защитники подъехали. Шо там по перепостам? Всех победили?)

\iusr{Vladislav Baturskiy}
\textbf{Olga Kosse}, сами бы попробовали всех победить с этих диванов! Только критиковать горазды...

\iusr{Viktor Taran}
\textbf{Энрике Анатольевич Менендес} як конкретно ви захищали? ви пішли добровольцем в армію? якщо так - в яку?

\iusr{Алексей Павлов}
Петух Димочка Золотухин умасливал интимные отверстия Пороха и ко. Ему некогда было. Стыдно, Энрике, такое спрашивать!

\iusr{Olga Kosse}
\textbf{Viktor Taran}, а защищают только через армию? И что у вас по защите - такой гладенький ГОшник. Добробат? Контракт? Пицца для военных раз в месяц и откупились?))

\iusr{Андрей Колосов}
\textbf{Viktor Taran} 

Можно ответить за коллегу, поскольку я был в точно таком же положении в
Луганске. Мы вели проукраинский майдан с 30 ноября 2013 по конец апреля 2014.
Но уже с марта руководство страны пошли на сдачу наших городов, 6 апреля - СБУ
Луганска с оружием, потом управление милиции Славянска и аж...до 5 июля- сдача
Донецка. Сдавали и офицеры, как вы пишите, армии, не выполнявшие требований
Уставов несения военной службы - без боя сдавали военные городки.


\iusr{Olga Kosse}
\textbf{Vladislav Baturskiy}, я безусловно критикую всех, кто рассказывает о "защите родины" с точки зрения армии и войны. Все эти адепты войны откупаются постиками поддержки военным в Facebook. Ну, пиццу им оплачивают раз в месяц, поздравляют в праздники. Все лишь для того, чтобы им было комфортно каждый день возвращаться в свои тёплые кроватки, да по ресторанам ходить, а не в акопах жить.

\iusr{Юрий Лукшиц}
Не нужно было сносить неконституционно власть в Киеве. Глядишь, и Крым с Донбассом не нужно было бы защищать тогда.

\iusr{Сергій Борис}
Ще чого, тільки від муралів і художників)

\iusr{Michail Vidiassov}
\textbf{Энрике Менендес} 

Так разве не ровно за это Вы стояли с майдановцами ? Были бы Вы с местными
регионалами, пытавшимися фрондировать и торговаться с Киевом, то была бы
понятна позиция: боролись, проиграли, теперь критикуете установленные
победителями порядки. Но Вы же сделали сознательный и сравнительно
информированный выбор когда пришёл "момент истины" и что-то действительно
решалось кулаками. Не логично ли Вам "лупить себя по затылку"?

\iusr{коршунова татьяна}
\textbf{Дмитрий Золотухин} а вы, надеюсь, защищали? или как во времена Киевской Руси-каждый только свой родной регион должен защищать?

\iusr{Энрике Менендес}
\textbf{Viktor Taran} не вижу Ваш комментарий (

\iusr{Olga Kosse}
\textbf{Viktor Taran}, пропал вместе со своим комментарием. Видимо, донор не согласовал месседж.

\iusr{Viktor Taran}

\textbf{Энрике Анатольевич Менендес} тобто? він є

\iusr{Viktor Taran}
\textbf{Энрике Анатольевич Менендес}

\ifcmt
  ig https://scontent-frt3-1.xx.fbcdn.net/v/t39.30808-6/245038643_10158670196646859_1663325792435579103_n.jpg?_nc_cat=108&ccb=1-5&_nc_sid=dbeb18&_nc_ohc=sjkbVOPpoSMAX9g3mv3&_nc_ht=scontent-frt3-1.xx&oh=13f06b75e72bd1cd88cb7ea0e7bfbcd1&oe=61A52A50
  @width 0.3
\fi

\iusr{Настя Степанова}
\textbf{Дмитро Золотухін} , это очень смищно, прям очень))) Безотносительно Энрике. Семь лет прошло, уже можно было не только начальную школу-то закончить, а после седьмого класса уже можно быть способным и на элементарную аналитику. Но это я, конечно, по своим детям сужу)))

\iusr{Сергей Перевозчиков}
\textbf{Энрике Менендес} Да, где Вы были с 7 до 11?

\iusr{Марк Зоряний}
\textbf{Энрике Менендес} да ты там вонял на оккупированной территории , но тебя на подвал посадили, где фсб и завербовало

\iusr{Андрей Николаев}
\textbf{Дмитро Золотухін} Они их туда, звали

\iusr{Городецкая Марина}
\textbf{Viktor Taran},ответа не будет, не ждите...

\iusr{Вадим Щетинин}
\textbf{коршунова татьяна}, типа сейчас не так. Ну прям увесь Бандерстан встал под ружжо.

\iusr{Андрій Винниченко}
\textbf{Viktor Taran} Если он считает, что Украина на него напала, чтобы лишить его идентичности, то очевидно, что он в 2014 году защищал свою страну Донбасс, от злых украинских захватчиков. Как и сегодня защищает.

\iusr{Елена Кальненко}
\textbf{Настя Степанова} За 7 лет можно грамоте научиться, чтобы не было ,,смищно,, и безотносительно.

\iusr{Настя Степанова}
\textbf{Елена Кальненко} , ахахахах)))))))

\iusr{Halyna Pry}
\textbf{Энрике Анатольевич Менендес} ви не захищали. ви сиділи в ахметівських структурах та "гуділи", а потім корчили з себе "відповідальних громадян", якими ніколи не були. Давайте назад, гудіть в Донецьку, не в Україні.
Якщо у вас клепки не вистачає навіть на те, аби розумітися в ментальностях людей, що поряд за сотню км (Харків, Суми, північ Луганщини, та сама "слобожанська ідентичність", якої за вами нема), то що ви взагалі знаєте про Україну?

\end{itemize} % }

\iusr{Роман Вовченко}
"Слобожанка" на Рианну очень похожа

\iusr{Иван Шашлов}

У нас еще в городе полно "политических украинцев" и "приазовских сепаратистов",
которые любят подрассказать, что "Мариуполь - не Донбасс". Ну в некоторых
случаях еще пытаются приазовских греков на свою сторону привлечь. Это тогда,
когда они с греками за Мариуполь не воюют, рассказывая, что "вас тут не стояло,
вы пришли на готовый город, крепость Домаха, Кальмиусская паланка".

\begin{itemize} % {
\iusr{Николай Анфалов}
\textbf{Иван Шашлов} просто некоторые "товарищи" пытаются оправдать ту линию разграничения, которая есть. Причём пытались это сделать задолго (!) до войны. А потом и в непризнанных республиках начались идеи о том, чтоб оправдать эту же линию разграничения, что, мол, в Краматорск и Мариуполь - это" свидомая бандеровская Украина"... Как же эти господа работают по обе стороны на раскол, чтоб только удержать сегодняшнее положение дел в головах людей ((((

Иван Шашлов
Николай Анфалов ну это же просто очередные следы работы по вытравливанию донбасса и донбасской идентичности, откалывание по куску и создание множественных линий раздела и резмежевания, антагонизмов. Разделяй и властвуй.

Николай Анфалов
Иван Шашлов ну вот пару часов назад в одном из обсуждений россиянин упрекнул меня в том, что я живу в "свидомом краматорске"

\iusr{Иван Шашлов}
\textbf{Николай Анфалов} а он бы хотел, чтобы вы где жили?
Один мой действительно свидомый знакомый-порохобот мечтал жить в Веллингтоне, например. Как сейчас - не знаю, вроде по прежнему в ватном Мариуполе живёт.

\iusr{Николай Анфалов}
\textbf{Иван Шашлов} ну так он посчитал меня предателем, что не поверил в счастливую жизнь в ДНР и не терплю все тяготы и невзгоды ради России. А все, кому не нравится ДНР и кто не молится на Путина, для него - "свидомые".

\iusr{Иван Шашлов}
\textbf{Николай Анфалов} честных поехавших и людей не зарплате хватает везде, к сожалению

\iusr{Николай Анфалов}
\textbf{Иван Шашлов} этот из поехавших. Он сказал, что все знает о ситуации на Донбассе из форумов и начал рассказывать что и как у нас происходит. Хотя сам дале географию Донбасса не знает. Но герой и "патриот"

\iusr{Иван Шашлов}
\textbf{Николай Анфалов} нуштош. Вообще я верю, что кое-что можно реально знать и из форумов. Как и в то, что на форумах могут быть специально обученные люди, как раз чтобы таких вот правдой накачивать  @igg{fbicon.smile} 

\iusr{Николай Анфалов}
\textbf{Иван Шашлов} просто есть люди, сильно подверженые пропаганде. Они, сидя в РФ, даже не могут себе представить, как можно уехать из Донецка в Краматорск жить и работать.

\iusr{Иван Шашлов}
\textbf{Николай Анфалов} ну глупые, что сказать. Таких хватает.

\iusr{Николай Анфалов}
\textbf{Иван Шашлов} они не глупые, они просто реальности ситуации сете не представляют вообще.

\iusr{Анатолий Наседкин}
\textbf{Ivan Shashlov} взагалі то "приазовські греки, історик ви наш, це депортовані з Криму християни, депортовані ще навіть ДО анексії Криму Катькою-2. І це історичний факт, що так їх переселили саме на землі Запоріжжя.

\iusr{Евгений Афендиков}
Так верните крым грекам, чтоб не кому не достался

\iusr{Halyna Pry}
А відколи Маріуполь став Донбасом, там є шахти і вугілля? Донецька область - так, але це не вугільний Донбас.
\end{itemize} % }

\iusr{Марія Подибайло}

ну так, і якийсь Данілов, і Стяжкіна, і Казанський і багато інших не мають
відношення до регіону, мабуть. А от якби на муралі була галичаночка, то це
означало б, що хтось намагається "насадити галицьку ідентичність"?

і так, нашу Донеччину реально "руйнують з двох сторін"? Енріке, ви нічого не
плутаєте? Чому тоді неокуповані території краще розвиваються, ніж до війни?
Думаю, ви чітко розумієте, що руйнує ЛИШЕ РОСІЯ З ЇЇ КОЛАБОРАНТАМИ, РОЗВ'язавши
війну! А ЗСУ обороняють свою територію від російської агресії!

Впевнена, що ви точно розумієте й те, що якби не було російської агресії, ЗСУ
не було б необхідності бути присутніми там, де вони є.

Ми ж, загалом, всі прекрасно розуміємо куди ви хилите своїми дописами та
інтерв'ю. Вас дурнем не назвеш. Тут багато інших визначень доречні...

\begin{itemize} % {
\iusr{Сергей Ваганов}
\textbf{Марія Подибайло} целиком и полностью поддерживаю!

\iusr{Энрике Менендес}
\textbf{Марія Подибайло} Не имеют. Они уехали и это их выбор. А мы остались

\iusr{Марія Подибайло}
\textbf{Энрике Анатольевич Менендес} цікаве, власне, питання. Вони поїхали чому? Де би зараз були, якби не поїхали?

\iusr{Vitalii Ovcharenko}
\textbf{Энрике Анатольевич Менендес} в донецке?)))

\iusr{Наталия Билык}
\textbf{Марія Подибайло} хорошо сказано, вот только непонятний месседж : « чому тоді неокуповані території краще розвиваються , ніж до війни». Це ви про що, шановна? Про закриті шахти? Про те, що на металобрухт вирізається все вщент? Про стовідсотково дотаційну економіку? Чи я чогось не знаю? То поділіться, будьласка)))

\iusr{Марія Подибайло}
\textbf{Natalia Bilyk}, якщо ви уважно прочитаєте, то мова йшла про руйнування від бойових дій "з обох сторін". А не про економіку чи політику. Це окрема дискусія. І не коротка. Але якщо дуже хочеться, давайте порозмишляємо і над цим. Найпростіші з питань: хто вивіз з Донецька заводи і випиляв їх на металобрухт. Ну і ще для розуміння: чи працюють заводи "градообразующіє" в Донецьку? Бо економіка - складна тема. А я писала про очевидні зовнішні, швидше, зміни.

\iusr{Наталия Билык}
\textbf{Марія Подибайло} ви відійшли від теми питання. Я вказала конкретну фразу з вашого коменту, яка мене здивувала і попросила іменно її прокоментувати, про розвиток підконтрольних територій, а не взагалі про Донецькі заводи)))

\iusr{Марія Подибайло}
\textbf{Natalia Bilyk} ? )

\iusr{Александр Ковалев}
\textbf{Марія Подибайло} Руйнування від бойових дій? Так. Не підконтролна територія зазнала більше руйнувань від бойових дій ніж підконтрольна. І як висновки можна зробити? Хто більше обстрілював кого?

\iusr{Наталия Билык}
\textbf{Марія Подибайло} що вам не зрозуміло? Ви написали ( дослівно) : « чому тоді неокуповані території краще розвиваються , ніж до війни». Аргументуйте, будьласка. Лише цей посил.

\iusr{Сергей Сидоров}
\textbf{Марія Подибайло} де ЖЗТМ/Азовмаш?

\iusr{Марія Подибайло}
\textbf{Сергей Сидоров} у їх власників варто запитати, мабуть. якщо об'єктивно, Азовмаш- не питання війни, правда?)
\end{itemize} % }

\iusr{Олексій Чупа}

Тут питали про те, в чому полягає ідентичність донбаська?

Свою відповідь я дав херзнаколи в книзі "Бомжі Донбасу": основне у ідентичності
Донбасу - відмова від етнічної складової, і схильність до ідентифікації за
класовою( в соціалістичному розумінні) ознакою, або за фахом.

По суті - це частково успішний експеримент зі створення "радянської людини", і
донецький кам'яновугільний басейн одне з небагатьох місць, де це пройшло.

Але одночасно є ще Донеччина - вона на сотні років старша, її ідентичність у
національному в тому числі.

І зараз триває боротьба ідентичностей - і поки у Донбасу в союзниках буде РФ,
він завжди буде програвати.

\begin{itemize} % {
\iusr{Евгений Лукожа}
\textbf{Олексій Чупа} и много вы выиграли с западными союзниками?  @igg{fbicon.face.grinning.sweat} 

\iusr{Яна Дрюченко}
\textbf{Евгений Лукожа} 

набагато більше, ніж мати «в родичах» скрепних представників фейкового етносу
«росіяне». Які спромоглися (історія вам в поміч) лише будувати людожерські
режими, красти будь що і будь в кого (після крадіжки завсякчас наполягаючи на
тому, що вкрадене «свае, исконнае» та кошмарити сусідів.

\iusr{Олексій Чупа}
\textbf{Евгений Лукожа} вони заводи не пиздять з покерфейсом, і українців не вбивають, not a worse choice

\iusr{Евгений Лукожа}
\textbf{Олексій Чупа} 

намного худший выбор. Кто кого на Донбассе убивает это вопрос спорный, а вот
то, что у вас теперь на троих умерших один рождённый, это вполне себе заслуга
ваших партнёров и последствия вашего выбора. Украина каждый год теряет 300.000+
своего населения. Это вы убийством государства не считаете? Или вы и на это
готовы лишь бы назло России? Вот такие вы патриоты.

\iusr{Яна Дрюченко}
\textbf{Евгений Лукожа} 

ви краще в своїй країні поратися розпочинайте. Переймайтеся власною
народженістю (спойлер - в вас по багатьом регіонам за винятком великих міст є
відток населення).

Не від хорошого життя ви лізете до сусідів. І примусово депортуєте тих же
українців з донбасу в свої заполярні ебеня. Оскільки тамтешні мешканці вже не
взмозі виконувати найпростіші завдання. Бо ви їх споіли вщент.

Мабуть, від вмсркодуховного життя та від захвату ваші співвітчизники глушать
бояру з денотуратом.

\iusr{Яна Дрюченко}
\textbf{Евгений Лукожа} 

взагалі, ваші коментарі як завжди демонструють неабиякий рівень захмарної вашої
збоченості. Оскільки ви весь час лізете повчати сусідів - як їм в себе
поратися. А (між тим) власна срака виглядає з-під обриганих брудних штанів
часів покорення (і не вашими дідами) очакова.

\iusr{Олексій Чупа}
\textbf{Евгений Лукожа} а, ты из Самары) о чем разговаривать, иди на хуй:)

\iusr{Сергий Федорынчык}
\textbf{Евгений Лукожа} 

А справді, чому Вас, самарця, цікавлять проблеми на Донбасі? В. Путін стверджує,
що Росія не є стороною цього конфлікту.

\iusr{Энрике Менендес}
\textbf{Олексій Чупа} Я читал, кстати, твою книгу. Форма понравилась, а смысл нет

\iusr{Евгений Лукожа}
\textbf{Яна Дрюченко} у нас рождаемость примерно на уровне смертности. Даже в Европе многие страны таким похвастаться не могут.
Что касается боярышника, то посмотрите уровень его потребления на Украине. Очень удивитесь.

\iusr{Евгений Лукожа}
\textbf{Сергий Федорынчык} стороной нет. Но это один из поводов для санкций. Так что, нас это тоже касается.

\iusr{Яна Дрюченко}
\textbf{Евгений Лукожа} 

справді, здивуюся. Оскільки «настойка баярьішника» в нас відсутня ))) і цьому
декілька пояснень: в нас п’ють горілку. Сортів якої вистачає на будь які смаки
и гаманець.

А крім того, досхочу будь якого іншого алкоголю, на всі смаки і гаманці. А крім
того, існують дорооовгі традиції домашніх настоянок і домашнього жеж самогону.
На бруньках, на кісточках, і так да-ді. А «боярьішник» в нас не п‘ють )))
додають до чаю глід - як і потрібно при лікуванні застуди )))

\iusr{Сергий Федорынчык}

Ці санкціі - "як слону дробина" - через колосальність російських ресурсів та
безмежне терпіння російського населення. Злочинні намагання Кремля у будь-який
спосіб відновити Російську Імперію та поглинути або підпорядкувати Украіну,
Молдову, Бєларусь, Грузію і т. д. не засуджуються більшістю населення Росіі,
навіть сприймаються з радістю - "Крим - наш!". 

Що ж, вам ще доведеться
зрозуміти глибину мудрості слів Антуана Буве де ля Мьорта, які стосуються і цих
намагань: "Це набагато гірше, ніж злочин! Це - помилка!" Зазнавши ніщивних і
принизливих поразок, Німеччина та Японія глибоко переосмислили себе, зрозуміли
згубність шляху здобувати місце у світі страхом, брутальною силою та війною і
зосередились на мирному розвитку. Нині вони - серед найбільш шанованих та
впливових краін світу. 

А Росія так і не переосмислила свою "Великую Победу",
втратиши в 5 разів більше солдатів, ніж Німеччина, у війні, яку Гітлер та
Сталін готували і починали разом, а потім готувались напасти один на одного,
але Гітлер випередив на два тижні. Ні США, ні НАТО за Украіну з Росією воювати
не будуть, і в Кремлі це розуміють. Там готують повномасштабний напад на
Украіну неспішно та заздалегідь - як анексія Криму готувалась ще з 2006 р. 

Але ця війна, попри можливість успіху завдяки великим воєнним перевагам РФ,
стане початком загибелі Росіі - вона втратить Сибир та Далекий Схід. Якщо Крим
можна загарбати через те, що він колись був російським, так і Сибир на давніх
китайських картах позначалась як "своя" територія. Та й Сахалін колись був
японським. А Чукотка набагато ближча до Аляски, ніж до Урала, який напевно
залишиться російським. Китай теж неспішно готується, і мільйон китайців в РФ не
тільки торгують.

\iusr{Сергий Федорынчык}

Шановна Яно! Дозвольте уточнити: "настоянка глоду" ("бояришник" російською)
коштує в украінських аптеках біля 10 гривень і активно вживається багатьма
місцевими алканавтами.

\iusr{Ievgen Nefediev}
\textbf{Энрике Анатольевич Менендес} бо ти не зрозумів змісту, все просто.

\end{itemize} % }

\iusr{Анатолий Евтушенко}

Вообще не пойму причин возмущения. С чего ты вообще решил, что он слобожанку
нарисовал типа как местную жительницу. Там ещё, например вот такой мурал есть.
Должны ли мы агриться, что художник размывает человечность жителей Донбасса,
изображая нас бездушными роботами?

\ifcmt
  ig https://scontent-frt3-1.xx.fbcdn.net/v/t39.30808-6/245405580_5042503962430761_7606769103839410231_n.jpg?_nc_cat=108&ccb=1-5&_nc_sid=dbeb18&_nc_ohc=JhLIFLCQNcYAX_O9aUi&_nc_ht=scontent-frt3-1.xx&oh=11b2b3ec5598357dfb7e54f35f178d14&oe=61A4515D
  @width 0.4
\fi

\begin{itemize} % {
\iusr{Анатолий Евтушенко}

«Роспись является продолжением моей серии муралов с украинками в этническом
наряде. В Авдеевке изображена девушка в традиционном стиле Слобожанщины на фоне
абстрактного восхода солнца над терриконами», — пояснил художник. Ну то есть
тут могла вместо Слобожанки появиться легко волынянка или гуцулка. Должны ли мы
были тогда делать выводы, что художник считает местных гуцулами?

\iusr{Энрике Менендес}
\textbf{Анатолий Евтушенко}

\url{https://www.facebook.com/jimmy.iurii/posts/4591772240885716}{%
"Слобожанка" - Новий мурал на Донеччині, Авдіївка, Yuriy Pitchuk, facebook, 13.10.2021%
}

\iusr{Энрике Менендес}
\textbf{Анатолий Евтушенко} Тебе норм? Мне нет

\iusr{Анатолий Евтушенко}
\textbf{Энрике Менендес} ну так и шо? Там ровно та же цитата, что и в моем каменте. Он нигде не пишет, что считает Авдеевку Слобожанщиной. Легко могла там появиться и гуцулка.

\iusr{Olga Kosse}
\textbf{Анатолий Евтушенко}, если бы там была не "слобожанка", вопросов бы не было. Но это целенаправленная программная работа. Например, в Интерсити "Константиновна- Киев" каждые 15 минут показывают ролики о путешествиях по Украине, и там весь Донбасс обозначен Слобожанщиной.

\iusr{Энрике Менендес}
\textbf{Анатолий Евтушенко} 

Но появилась именно Слобожанка. А в Интерсити из региона в Киев Донецкая и
Луганская область относятся в рекламе на экранах к Слобожанщине. А Стяжкина
написала статью на "Радио Свобода", что Донбасса не просто нет, но это ещё и
вредно.

А Данилов сказал, что нет никакого Донбасса. А Резников, что Донбасс это
раковая опухлоль.

Но ничего страшного. Мы ж гибкие ребята. Нас хоть в лоб, хоть по лбу. Лишь бы
не обвинили в сепаратизме.

\iusr{Анатолий Евтушенко}
\textbf{Olga Kosse} а некоторые жители остальной
Украины считают жителей Донбасса покорными и бездушными. Если бы на мурале был
не робот...

\iusr{Vladislav Baturskiy}
\textbf{Анатолий Евтушенко} пусть в Киеве рисуют муралы с неграми и китайцами. Прямо напротив Администрации президента. А Вы поясните недоумевающим, что это ну вот так вот. Негр и китаец на фоне Лавры с какого-то рожна. А почему нет, в самом деле?

\iusr{Анатолий Евтушенко}
\textbf{Энрике Менендес} да срать мне на Данилова с Резниковым. Я тебе говорю, что конкретно в этом случае твоё возмущение неоправданно. Даже если они туда сдобожанку намеренно притянули (хотя у меня большие сомнения), в Авдеевке просто появился ещё один клёвый мурал. И ни один житель Авдеевки от этого себя не перестанет определять как жителя Донбасса.

\iusr{Анатолий Евтушенко}
\textbf{Vladislav Baturskiy} не знаю, о чем вы, но если хоть один человек перестанет называть африканцев так, как вы - всё не зря!

\iusr{Vladislav Baturskiy}
\textbf{Анатолий Евтушенко}, Указ 7/8 шьёшь, начальник? Не надо. Это по-американски "негр" - это оскорбление. А я на русском общаюсь.

\iusr{Mariia Marchenko}
\textbf{Anatoliy Yevtushenko} прям согласна с тобой. Стоит, например, в Канаде писанка неописуемых размеров ни к селу не к городу, и никто ещё не пожаловался, что она посягает на идентичность канадцев.

\iusr{Энрике Менендес}
\textbf{Анатолий Евтушенко} Уж не знаю. Фукуяма целую книгу про идентичность написал. Читал? )

\iusr{Vladislav Baturskiy}
\textbf{Mariia Marchenko}, 

а с чего канадцам жаловаться на посягательства, если там процентов пять
населения — этнические украинцы? А там, где писанка неописуемых размеров, так
там и процент может быть повыше.

На Донбассе слобожаночек нету. При всём уважении. Потому что Донбасс — это не
Слобожанщина.

\iusr{Victor Proskurnikov}
\textbf{Энрике Менендес} 

Прекрасная книга (Это об Идентичности Фукуямы). Прочитал месяц назад. Уж
гораздо актуальнее его эссе Конец истории. Жду рецензии. До сих пор под
впечатлением от Свободы слуг (Маурицио Вироли) и Воображаемых сообществ
(Бенедикт Андерсон). Достойны быть в библиотеке каждого.

\iusr{Rostyslav Pelekhovych}
> \url{https://www.facebook.com/jimmy.iurii/posts/4591772240885716}
ну от сам атор муралу пише "Новий мурал на Донеччині". що не так?

\iusr{Любовь Головко}
\textbf{Vladislav Baturskiy} по-русски арап.  @igg{fbicon.smile}  Слово негр - иностранного происхождения - niger, позднелат. negrus, negra («тёмный», «чёрный») и греч. Νέγρος, испанское negro «чёрный (цвет)».

\iusr{Vladislav Baturskiy}
\textbf{Любовь Головко}, это по-древнерусски "арап". Сейчас не все поймут, что это за слово такое. В лучшем случае, спутают с "арабом".

\iusr{Anton Savidi}
\textbf{Lyubov Golovko} αραπης (арап-ис) - это греческое слово, которое заимствовали на Руси, как и куча других в то время.
\end{itemize} % }

\iusr{Stanislav Tsykalovskyi}

Я все таки не понял, в чем ПРОБЛЕМАТИКА?
Есть мурал девушки в народном костюме Слобощанщины - СЛОБОЖАНКА.
ПРОБЛЕМА В ЧЕМ???????????????????????????

\begin{itemize} % {
\iusr{Энрике Менендес}
\textbf{Stanislav Tsykalovskyi} У тебя нет проблемы. Вот у меня с Россией тоже нет проблем. А у тебя есть

\iusr{Stanislav Tsykalovskyi}
Я не спрашивал про твои проблемы или мои проблемы.
Мы же обсуждаем МУРАЛ. С ним что не так? Красивый, яркий, позитивный.
Где тут зарыто?

\iusr{Валентина Титова}
\textbf{Stanislav Tsykalovskyi} а что она (СЛОБОЖАНКА) в Авдеевки делает? В гостях?

\iusr{Stanislav Tsykalovskyi}
\textbf{Валентина Титова} наконец-то!!! Да!!! В гостях, как дома. Потому что мы одинаковые и разные одновременно. Да, потому что Донбасс гостепримный! Мы рады всем. А еще возможность узнать себя, нас, их. Здорово же!

\iusr{Валентина Титова}
\textbf{Stanislav Tsykalovskyi} 

Если бы это было именно так! Вы же сами знаете, что всё далеко не так позитивно
и радужно. Например, ЗНАЮ ТОЧНО, в Авдеевке никогда и никто не разговаривал на
украинском. Ничего не имею против украинского языка (закончила Львовский
институт). Сейчас ПРИНУЖДАЮТ! Всё остальное - в том же духе. Да и в гости идут по
приглашению обычно. Поэтому, согласна не с Вами, а с Энрико- не место ей в
Авдеевке. Уверена, что так думает всё местное население Авдеевки. Молчат,
конечно((( куда деваться((

% -------------------------------------
\ii{fbauth.cikalovskij_stanislav.kiev.ukraina}
% -------------------------------------

\textbf{Валентина Титова} гость в доме - представитель Бога.
Мы сами отгораживаемся от всего, а потом еще ходим кричим - ЛЮБИТЕ НАС ЛЮБИТЕ!!!

\iusr{Валентина Титова}
\textbf{Stanislav Tsykalovskyi} 

до того лета 14-года, когда нас начали \enquote{любить} ,я абсолютно не разделяла людей
по национальностям - да и как разделить, если я русская, муж наполовину белорус
(по отцу), наполовину украинец (по матери), наши дети кто? А мне и неважно
было. И друзей много с Западной Украины. А вот после 14-го, когда нас так
\enquote{залюбили} многих до смерти, моё отношение изменилось в корне. Не нужна мне
такая любовь и даром.

\iusr{Stanislav Tsykalovskyi}
\textbf{Валентина Титова} спасибо России.

\iusr{Валентина Титова}
\textbf{Stanislav Tsykalovskyi} 

ааа... так это Россия на Донбасс с танками пошла? Российские танки? Я просто не
узнала, уж простите.

\ifcmt
  ig https://scontent-frt3-1.xx.fbcdn.net/v/t39.30808-6/244932251_10225634358956962_6753943566293033687_n.jpg?_nc_cat=106&ccb=1-5&_nc_sid=dbeb18&_nc_ohc=T1CAT7hbGg8AX-ZUDyg&_nc_ht=scontent-frt3-1.xx&oh=9f2a02d10c5d52a9fb3a42d09a79ace1&oe=61A5B341
  @width 0.4
\fi

\iusr{Stanislav Tsykalovskyi}
\textbf{Валентина Титова} 

на меня танки не шли. Шли на террористов и бандитов.
Вы еще скажите русских танков нет.
И в Крыму не было до Донбасса

\iusr{Александр Малюк}
\textbf{Валентина}, Энрике же Вам написал: Слобожанщина очень близко к Авдеевке. Во время индустриализации огромное количество слобожанцев и слобожанок переехали работать в Авдеевку. Но при этом Энрике почему-то думает, что мы, донбасяне (или донбассовцы?), должны этого стесняться. Странная позиция. Кокошники его не смущают, а слобожаночки смущают.  @igg{fbicon.smile} 

\iusr{Валентина Титова}
\textbf{Stanislav Tsykalovskyi} 

ага... на фото видно, что останавливают танки именно террористы. Да и на Луганск
авианалёта украинского самолёта, когда погибли гражданские люди, тоже не было.
Я помню - там кондиционер взорвался.

\iusr{коршунова татьяна}
\textbf{Stanislav Tsykalovskyi} айда все в Донецк и Луганск!здорово же!многие у нас и не были никогда. а мы рады всем.

\iusr{Олег Платонов}
\textbf{Энрике Анатольевич Менендес} У тебя с Россией нет проблем, потому что ты коллаборант... или просто идиот.. я читаю твои посты и просто в шоке от всей этой каши в твоей голове ..

\iusr{Сергий Федорынчык}

Енрике, тут Ви помиляєтесь. Проблеми з "русским миром" Кремля, а по суті -
намаганнями відновити Російську Імперію є у цілого світу, і у жителів Росіі
також. Не було б цих намагань - не було б окупаціі Криму та війни на Донбасі.
Усіх розбіжностей та протиріч між Донбасом та рештою Украіни аж ніяк не
вистачило б на війну, якби не зусилля Кремля з іі організаціі.

\iusr{Stanislav Tsykalovskyi}
\textbf{коршунова татьяна} и у нас и мы у них. только мы прем свою донбасскую самобытность, а люди пожимают плечами над нами

\iusr{Stanislav Tsykalovskyi}
\textbf{Валентина Титова} а этих чего не останавливали?
ПРОШУ! Не отвечайте.

ВИДЕО

\iusr{Валентина Титова}
\textbf{Stanislav Tsykalovskyi} они по НАМ не стреляли

\iusr{Stanislav Tsykalovskyi}
А этих?

ВИДЕО

\iusr{Валентина Титова}
\textbf{Stanislav Tsykalovskyi} Прошу, не отвечайте. У вас правда какая-то неправильная. Вы не ответили, КТО убил людей при авианалёте на Луганск

\iusr{Stanislav Tsykalovskyi}
\textbf{Валентина Титова} по мне не стреляли. Почему по вам стреляли?

\iusr{коршунова татьяна}
\textbf{Stanislav Tsykalovskyi} 

а люди, что пожимают плечами над нами, гордятся своей украинской
самобытностью. я когда то сказала с что на Донбассе больше 100 национальностей
уживались мирно, так мне в ответ презрительно пожали плечами и бросили сквозь
зубы-потому у вас такое и творится, что всякого сброду туда понавезли..

Ага-хто -то по над усе, а нас понавезли. какая самобытность у сброда?

\iusr{Ольга Сидорова Сидорова}
\textbf{Энрике Анатольевич Менендес} если с Россией нет проблем так зачем все эти Донбасс - Донбас?

\iusr{Сергий Федорынчык}
\textbf{Валентина Титова} 

1) Чи брали Ви участь у "референдумі" щодо незалежності ДНР? 

2) Чи розумієте звязок "референдумів" з появою на Донбасі загонів
Стрелкова-Гіркіна та російських козаків Козіцина, а потім регулрних підрозділів
російської армії ? 

3) Як, по Вашому, мали діяти частини Збройних Сил України, якщо по ним стріляли
з позицій, розташованих поруч з мирним населенням, в т.ч.  Вами?

\end{itemize} % }

\iusr{Vitalii Ovcharenko}

Жесть. В то время когда в донецке умирают сотни, а то и тьісячи дончан, Енріке
обвиняет в уничтожении Донбасса художника нарисовавшего картину...

Енріке - настоящий защитник Донбасса)))))

\begin{itemize} % {
\iusr{Roman Eridan}
Енріко завжди намагається обвиняти лише тих, кого, як він вважає, безопасно обвиняти.

\iusr{Vitalii Ovcharenko}
\textbf{Roman Eridan} або тих проти кого він системно діє)))

\iusr{Андрій Винниченко}
\textbf{Vitalii Ovcharenko} Тих кого він ненавидить.
\end{itemize} % }

\iusr{Антон Морозов}
Ок. Допустим.
Какая идентичность у Донбасса, поясните?

\begin{itemize} % {
\iusr{Энрике Менендес}
\textbf{Антон Морозов} Своя. Как у любого другого исторически сложившегося региона. А у Буковины какая? А у Баварии? А у Галичины? А у Каталонии?
Вы вообще хоть когда-то читали что-то о региональной идентичности?

\iusr{А. Полтавцев}
\textbf{Энрике Анатольевич Менендес} Какая? Чем отличается? Почему эта идентичность сама пригласила тех кто её уничтожает - "узкий мир"? Идентичность подсознательно хочет умереть?

\iusr{А. Полтавцев}
\textbf{Энрике Анатольевич Менендес} 

охренеть, значит муралы уничтожают идентичность, а разрушение промышленности
узким миром - очень даже норм. Культура - угроза идентичности, а
физ. уничтожение экономики, доведение населения до нищеты с целью вывоза
населения в условный Ханты-Мансийск и на дальний восток - не угроза? Вам не
кажется, что Вы не на тех направляете свою агрессию и "праведный гнев"?

\iusr{Лариса Дорганова}
\textbf{Энрике Менендес} таким образом ваш ряд можно продолжить Новороссией и Малороссией, ведь это тоже исторически сложившиеся регионы, значит они тоже имеют свою идентичность и тут мы оказываемся на поле роспропаганды и всей росполитики (путинская речь 18 марта 2014 года про Новороссию из 8 областей)

\iusr{Юрий Лукшиц}

Идентичность Донбасса очень простая. Антинационалистическая, поликультурная,
индустриальная, урбанизированная.

\iusr{Юрий Троян}
\textbf{Юрий Лукшиц} 

а какая она в США... политкультурная, антинационалистичееская, индустриальная и
тд. Америка - это Донбасс! Или наоборот.

\iusr{Юрий Лукшиц}
Соединённые Штаты являются децентрализованной федерацией. Сами-то как думаете?

\iusr{Сергей Еременко}
\textbf{Антон Морозов} Национальная кухня.

\iusr{Энрике Менендес}
\textbf{Лариса Дорганова} Нет, потому что это регионы Российской Империи. А Донбасс - это уже современное понятие

\iusr{Юрий Троян}
\textbf{Юрий Лукшиц} завидую только

\iusr{Настя Степанова}
\textbf{Anton Morozov} , 

короче, поехал сотрудник, молодой пацан, в родные края, в Краматорск. И на
вокзале перед уездом получил хорошенько по лицу, т.к. не всем это лицо
понравилось. А ты говоришь, в чем идентичность?))) Вот тебе, на тебе)))))

Я очень люблю родной край издалека. И чем дальше, тем больше люблю)

Взяла сегодня на работу тезку, Настю, из Мариуполя. Но тут должен рассудить нас
глас донбасского народа - Мариуполь еще Донбасс или уже нет))) пойдет ли мне
плюсик в карму))))))

\iusr{коршунова татьяна}
\textbf{Настя Степанова} а в Киеве никто по лицу не получает? а во Львове? прикиньте, и в Париже парижанам прилетает. куда податься? @igg{fbicon.thinking.face} 

\iusr{Настя Степанова}
\textbf{Tatiana Korshunova} , ой, я точно не же суи Донбасс при всей своей донецкости, поэтому перекручивайте, как хотите))))

\iusr{Антон Морозов}
\textbf{Сергей Еременко} можно подробнее?

\iusr{Сергей Еременко}
\textbf{Антон Морозов} национальные песни и сказки.

\iusr{Марк Зоряний}
\textbf{Энрике Менендес} ты гнус фашистский, у тебя идентичность, как у полицая служить своему хозяину Хуйлу

\iusr{Сергий Федорынчык}

Енрике, не знаючи іспанськоі мови, Вам буде важко в це повірити, але є іспанці
по-за межами Каталоніі, Галісіі та Басконіі, які знають ці мови, бо вважають іх
частиною іспанськоі ідентичності.

\iusr{Оксана Гусейнова}
\textbf{Настя Степанова} ні Маріуполь це Приазов'є, вони даано хотіли відділитися в окремий район від "самобитного донбасу".

\iusr{Олександр Ходорівський}
\textbf{Антон Морозов} донбасянська мова. нє?

\end{itemize} % }

\iusr{Александр Лапин}
\textbf{Энрике Менендес}, 

я вообще противник акционных муралов. Особенно за бюджетный кошт. Другое дело,
когда сами совладельцы многоквартирного дома нанимают художника и заказывают
ему сюжет мурала. Но сомневаюсь, что в Авдеевке кто-то спросил жильцов. Что
касается изображения, то почему бы не взять что-то из истории самой Авдеевки -
первого поселенца Авдея, какую-нибудь гагайскую крестьянку...

\begin{itemize} % {
\iusr{Татьяна Переверзева}
\textbf{Александр Лапин} цей мурал створено не за бюджетні кошти

\iusr{Александр Лапин}
\textbf{Татьяна Переверзева} , а жильцы дома выбирали его сюжет?

\iusr{Валерий Антонов}
\textbf{Татьяна Переверзева} этот мурал сделал художник из Франковска.. не одевались у нас так и перья павлина к головному убору не пристраивали. При первой же возможности удалим этот непонятный мурал со стены дома

\iusr{Татьяна Переверзева}
\textbf{Валерий Антонов} Ви мали на увазі такий одяг як традиційний?

\ifcmt
  ig https://scontent-frt3-1.xx.fbcdn.net/v/t39.30808-6/245473900_1574196406251987_5712722811262889745_n.jpg?_nc_cat=102&ccb=1-5&_nc_sid=dbeb18&_nc_ohc=UmYsK6pAak4AX_mwHoI&_nc_ht=scontent-frt3-1.xx&oh=e62451576073d936623d817a02b3a7ca&oe=61A3DEF6
  @width 0.4
\fi

\iusr{Валерий Антонов}
\textbf{Татьяна Переверзева} а что вам не нравится в этом фото.. Все мужики на фронте, почти наверняка, женщины работают в тылу.. Тяжелое время было, как я понимаю

\end{itemize} % }

\iusr{Maria Zolkina}

Энрике, а ты уверен, что посыл этого мурала был именно таким: Авдеевка –
Слобожанщина? А если бы там девушку в буковинской вышиванке изобразили, это
тоже бы означало, что Донбасс -это не Донбасс вовсе, а Буковина? В политическом
смысле послы про то, что Донбасса как отдельной единицы не существуют, скорее,
связаны с тем, чтобы как раз не идти на поводу у российских посылов про народ
Донбасса, его особенности и так далее. Да и то, что сейчас называют Донбассом –
всю подконтрольную территорию Донеччины и Луганщины, это тоже не Донбасс в
этнографическом и историческом смысле. Этот регион неоднороден, идентичность
местная либо региональная, безусловно, есть у части населения, но на подобные
ситуации я бы точно не реагировала как на наступление на идентичность.

\begin{itemize} % {
\iusr{Сергей Бондаренко}
\textbf{Maria Zolkina} - автор явно на подтанцовках у россиянских пропагандонов. "народ дамбаса", "дамбасянская идентичность"..
Явно подзабыл сей "испанец", как голодом выморили украинцев на этой территории, и навезли туда эту "идентичность" из северных краёв.
\end{itemize} % }

\iusr{Stanislav Tsykalovskyi}
ПОВНА ВЕРСІЯ МУРАЛУ /чомусь не пройшла конкурсну комісію/
«Слобожанка»
Новий мурал на Донеччині, Авдіївка, 2021.
Розпис є продовженням моєї серії муралів з українками в етнічному вбранні. В Авдіївці забражена дівчина в традиційному строю Слобожанщини на фоні абстрактного сходу сонця над териконами.

\ifcmt
  ig https://scontent-frt3-1.xx.fbcdn.net/v/t39.30808-6/245320810_4745375335496759_933990269027543180_n.jpg?_nc_cat=108&ccb=1-5&_nc_sid=dbeb18&_nc_ohc=M8A2ui1tkNAAX-7_nVW&_nc_ht=scontent-frt3-1.xx&oh=4a4a977c3d39c26bac826d6ce0058830&oe=61A469CC
  @width 0.4
\fi

\iusr{Stanislav Tsykalovskyi}

А ты бы променял вот такие бы муралы на мир на Донбассе? Чтобы люди не умирали
без воды, без лекарств, замкнутые какими-то русскими в большом концлагере?

Или все таки по зубам?

\begin{itemize} % {
\iusr{Энрике Менендес}
\textbf{Stanislav Tsykalovskyi} А что, других опций нет?

\iusr{Stanislav Tsykalovskyi}
Опций много.
Но они появляются, когда первый шаг сделаешь.
Вот ты можешь его сделать?
Мурал или убийство людей?

\iusr{Rostyslav Pelekhovych}
)) Забавно, що погодитись на умови Москви заради миру Енріке готовий. а на умови умовного ІФ - ні

\iusr{Rostyslav Pelekhovych}
нє палімся
\iusr{Олег Дьяченко}
нацик наоборот

\iusr{Татьяна Шевченко}
\textbf{Stanislav Tsykalovskyi} Без какой воды, и без каких лекарств, здесь умирают люди? Если люди гибнут здесь, то только от агрессии небратьев...

\iusr{Stanislav Tsykalovskyi}
\textbf{Татьяна Шевченко} поехайте в наши Ровеньки или Красный луч. Там уже больше месяца нет воды.

\iusr{Александр Лапин}
\textbf{Stanislav Tsykalovskyi} , какой-то у Вас выбор без выбора. Прямо как в тюремных загадках: «Есть два стула, на одном пики точены, на другом х..и дрочены, на какой сядешь, на какой мать посадишь?».

\iusr{Stanislav Tsykalovskyi}
\textbf{Александр Лапин} и что сложного?

\iusr{Татьяна Шевченко}
\textbf{Stanislav Tsykalovskyi} Ну ведутся ремонтно-восстановительные работы...

\iusr{Александр Лапин}
\textbf{Stanislav Tsykalovskyi} , для Вас, видимо, ничего)))

\iusr{Stanislav Tsykalovskyi}
\textbf{Татьяна Шевченко} восьмой год.

\iusr{Stanislav Tsykalovskyi}
\textbf{Александр Лапин} окей.
\end{itemize} % }

\iusr{Дмитрий Баевский}

Да, исторически к Слобожанщине относятся Славянск, Краматорск и Бахмут. Как по
мне, не краеешек, а значительная часть области. Но да, меньшая. И не Авдеевка,
согласен.

Но пусть лучше жители Донецкой области живут в свободной и мирной Украинской
Слобожанщине, чем уничтожаются экономически и физически Россией, пусть даже
уничтожаются под названием Донбасс.

\begin{itemize} % {
\iusr{Энрике Менендес}
\textbf{Дмитрий Баевский} Поглядите карту

\iusr{Александр Лапин}
\textbf{Энрике Менендес} , начнём с того, что Бахмут - это Екатеринославщина, чьим уездным центром он был до 1920х...
\end{itemize} % }

\iusr{Liubov Aresu}

Хм. ..... как профессия может стать национальной идентичностью ?!

Меня вот жизнь из одного «Донбасса» забросила в другой «Донбасс». За несколько
тысяч км. Но такие же шахтерские городки и посёлки , только шахты тут закрыли
ещё в 50-60х годах.

Конечно, возмущались и тогда и сейчас. И много посёлков умерло буквально и
продолжают умирать особенно те, где кроме шахт ничего не было. Только никто не
тянет одеяло на себя про идентичность особенную и ее уничтожение потому что
были шахтерами.

Я вообще коренная. У меня в Бахмуте половина кладбища Мариупольского - родня
все. А баба с дедом вообще жили на Нижней Мариупольской, кладбище из окон дома
видно. Так вот нет никакой донбасской идентичности, потому что шахтёр или
металлург. Не несите чепуху !

\begin{itemize} % {
\iusr{коршунова татьяна}
\textbf{Liubov Aresu} а за что нас, еще при Юле, призывали проволокой колючей обносить? @igg{fbicon.thinking.face} по какому признаку?

\iusr{Марина Иванищенко}
\textbf{Liubov Aresu} ,а у меня есть такой местечковый патриотизм. мне дышать без этого воздуха больно было.

\iusr{Liubov Aresu}
\textbf{Марина Иванищенко} вы знаете, мне и зори в небе в родном месте другие. И дышится да, по другому. Потому что - дом. Но я украинка. С Донбасса. С востока Украины. Украинка!

\iusr{Ольга Сидорова Сидорова}
\textbf{Tatiana Korshunova} по признаку Моторол , Бесов и Гиви

\iusr{коршунова татьяна}
\textbf{Ольга Сидорова Сидорова} а вам ближе бесы из Торнадо? начните во Львове кричать что тут вокруг уроды и бандиты, и найдутся те, кто заткнет вам рот обломком кирпича. И это будет доказательством того, что вы были правы - во Львове вокруг одни уроды и бандиты.
Как говорится - не вымогай, а то выпросишь.
\end{itemize} % }

\iusr{Mykola Malukha}

"7 лет наш регион физически уничтожают с разных сторон." ну чего юлить, так
и пиши прямо "Украина уничтожает Донбасс"

\begin{itemize} % {
\iusr{Энрике Менендес}
\textbf{Mykola Malukha} И Украина тоже (

\iusr{Mykola Malukha}
\textbf{Энрике Анатольевич Менендес}, спасибо за откровенность
\end{itemize} % }

\iusr{Stanislav Tsykalovskyi}

Донецкая область Конец XIX - начало XX в.

Женский костюм.

Рубашка с нашитой вдоль выреза горловины полочкой, украшенная вышивкой
красно-черного цвета и мережкой. Кофта расшита цветной тесьмой. Полы обшиты
кружевом. Широкая юбка из яркой фабричной ткани по подолу обшита плисом и
подшита «щеточкой», чтобы не протиралась на перегибе. Фартук-передник из
фабричной ткани, украшенный мережкой и подшит по подолу рюшей. Кашемировый
платок повязывался на «сеточку», которая одевалась на «гуглю» - пучок туго
скрученных волос. Высокие кожаные ботинки с красными шнурками.

Мужской костюм.

Туниковидная рубашка с широким стоячим воротником и широкой манишкой. Воротник,
манишка и низ рукавов вышиты. Обязательным атрибутом мужского костюма был
короткий жилет - «жильотка» темного цвета. Брюки узкие, городского покроя.
Сапоги - «бутылки» с твердыми блестящими голенищами. Головной убор - фуражка
(картуз).

Современные художники-модельеры часто обращаются к лучшим образцам народной
одежды. Орнамент, колористика, вышивка, кружево, конструктивные формы народной
одежды находят отражение в современном искусстве моделирования костюма. И это
не простое отображение, повтор прошлых форм, его пропорций и цветовых
сочетаний, а творческая наследственность традиций.

\ifcmt
  ig https://scontent-frx5-1.xx.fbcdn.net/v/t39.30808-6/245463901_4745386508828975_1280685415640270858_n.jpg?_nc_cat=105&ccb=1-5&_nc_sid=dbeb18&_nc_ohc=qXbCjRHIhCoAX_0hu89&_nc_ht=scontent-frx5-1.xx&oh=bc6d8b8362b8a66fb05254845d87eebe&oe=61A4426C
  @width 0.4
\fi

\begin{itemize} % {
\iusr{Rostyslav Pelekhovych}
\textbf{Stanislav}, а труби коксохіму де? Мабуть теж галичани малювали

\iusr{Stanislav Tsykalovskyi}
Може ще паравоз вишити від Луганського паравозобудівного?

\iusr{Сергей Сидоров}
\textbf{Stanislav Tsykalovskyi} донецкая область конца ХІХ века это "пэрл!на"

\iusr{Иван Наконечный}
\textbf{Сергей Сидоров} чему Вы удивляетесь ? Многие из «фейсбучных историков», не знают, что Донецк ранее назывался Сталино, а до этого Юзовка .

\iusr{Сергей Сидоров}
\textbf{Иван Наконечный} 

я думал, после 1991 нечему будет удивляться, но 2014 все-таки смог. Вернее,
предчуствие, когда в декабре 2013 из Мариуполя резко эвакуировались рошен и
оби... стало смутно понятно, что зреет шухер... но, думал экономический, а-ля
90-е, но что вот так, с ракетами и снарядами... хотя еще в 91м дед сказал, что
просто так от россии куски не отрываются, будет война. Не верилось до
последнего.

\iusr{Сергей Сидоров}
А вот эти реквизиторы лучше бы что ли Серафимовича почитали, Куприна... и не мололи бы про "белые штаны и носки без дырочек"

\iusr{Stanislav Tsykalovskyi}
\textbf{Сергей Сидоров} вы как-то прям к словам цепляетесь. Ну если написать, что "Екатеринославская губерния, Конец XIX - начало XX в. " - вообще 99\% зумеров нынешних "русских" донецких взрыв мозга получат
\end{itemize} % }

\iusr{Vladislav Maistrouk}

Назву «Донецький басейн», скорочено «Донбас» дав Ковалевський Євграф Петрович,
гірничий інженер із Харкова, який виконав перше стратиграфічне і геологічне
дослідження Донбасу і подав першу карту залягання вугільних пластів Донбасу
(початок ХІХ ст.).

До этого на картах писали - Дикое поле.

Когда Юзовку переименовывали на Донецк, жители потеряли прежнюю идентичность?
Стоит ли восстановить исконное имя, как с Нью-Йорком?

Что вообще вкладывается в донбасскую идентичность? Я консолидированного мнения
не слышал. Только противоборствующие версии того, что такое Донбас и его роль
среди других украинских регионов.

Наверное нет правильной трактовки. Но есть опасная реакция на чужое мнение. Вот
ты готов дать в зубы...зачем? Что это изменит?

Доведи, культурным способом, исторически основанную концепцию края где ты
родился и рос. Потому что идентичность Донбасса построенная партией регионов и
Ахметовым, это микс народного фольклора (как украинского, так и переселенцев из
России, которые в итоге стыдили местных из-за их «мови»), советской пропаганды,
и политической выгоды местных бизнесменов.

\iusr{Artem Mykhailychenko}

было бы из-за чего рубашку на себе рвать

\begin{itemize} % {
\iusr{Энрике Менендес}
\textbf{Artem Mykhailychenko} А что нет? Если вас назвать не украинцем а малороссом, будете рвать рубаху?

\iusr{Artem Mykhailychenko}
\textbf{Энрике Анатольевич Менендес} Донецкая и Луганская область такая же часть Украины, как и Черкасская, Херсонская или моя Днепропетровская. Попуститесь уже со своим местечковым псевдопатриотизмом.

\iusr{Сергей Сидоров}
\textbf{Artem Mykhailychenko} а 150 лет назад вы бы сказали: такая же часть РИ, как Варшава или Гельсингфорс, а 50 лет назад - такая же часть СССР, как Кузбасс или Магнитка... а что вы скажете еще через 50х лет?


\iusr{Энрике Менендес}
\textbf{Artem Mykhailychenko} Это вы попуститесь с вашими тщетными попытками уничтожить нашу идентичность. Буду бить в зубы - клянусь

\iusr{Artem Mykhailychenko}
\textbf{Энрике Анатольевич Менендес} вы пьяны?

\iusr{Artem Mykhailychenko}
любить свой город, свой край - это нормально. но быть местечковым "патриотом"... и бить в зубы за это. окей. это ваши проблемы.

\iusr{Энрике Менендес}
\textbf{Artem Mykhailychenko} Я всегда пьян. Это ж всем ясно )


\iusr{Энрике Менендес}
\textbf{Artem Mykhailychenko} Не перекручивайте - я сказал, что буду бить в зубы, если будут пытаться уничтожить мою идентичность
\end{itemize} % }

\iusr{Stanislav Tsykalovskyi}
Донбасс до войны (Луганск)

\ifcmt
  ig https://scontent-frt3-1.xx.fbcdn.net/v/t39.30808-6/245432756_4745397858827840_8067124190521038732_n.jpg?_nc_cat=102&ccb=1-5&_nc_sid=dbeb18&_nc_ohc=qRXTetNyqO0AX-WSajG&_nc_ht=scontent-frt3-1.xx&oh=eb8144aa8de2604871a2c99d13045dfa&oe=61A49A38
  @width 0.4
\fi

\begin{itemize} % {
\iusr{Stanislav Tsykalovskyi}
Мир

\ifcmt
  ig https://scontent-frt3-2.xx.fbcdn.net/v/t39.30808-6/245040952_4745400812160878_8074872046861773895_n.jpg?_nc_cat=103&ccb=1-5&_nc_sid=dbeb18&_nc_ohc=QUfw1v38H-8AX9LZBvT&_nc_ht=scontent-frt3-2.xx&oh=a65564e23664e0e6d939e8196f650be3&oe=61A501EF
  @width 0.4
\fi

\iusr{Stanislav Tsykalovskyi}
Даже косоворотных не били. Жили спокойно, мирно.

\ifcmt
  ig https://scontent-frx5-1.xx.fbcdn.net/v/t39.30808-6/245004998_4745419122159047_8878403336795055812_n.jpg?_nc_cat=105&ccb=1-5&_nc_sid=dbeb18&_nc_ohc=Rv-2a5VPFT8AX9YwzRs&_nc_ht=scontent-frx5-1.xx&oh=01d4d378da56670921a38534a1b585eb&oe=61A5AC95
  @width 0.4
\fi

\iusr{Александр Ковалев}
\textbf{Stanislav Tsykalovskyi} Згоден. Таким був Луганськ до войни, до грьобаного майдауна.

\iusr{Stanislav Tsykalovskyi}
\textbf{Александр Ковалев} да, когда русские начали шатать Луганск то всё

\iusr{Александр Ковалев}
\textbf{Stanislav Tsykalovskyi} Може і так, тільки чомусь після авіаударів по Луганську та Станиці Луганській, загинули лише громадяни України і жодного російського окупанта. Не розумію тільки, чому всій Україні збрехали про кондиціонер?

\iusr{Stanislav Tsykalovskyi}
\textbf{Александр Ковалев} война дело грязное. А у бандитов и террористов нет национальности.

\iusr{коршунова татьяна}
\textbf{Stanislav Tsykalovskyi} ой, какие фото! а говорили гнобили украинство, уничтожали прямо. а ти диви..

% -------------------------------------
\ii{fbauth.kovalev_aleksandr.rovenjki}
% -------------------------------------

\textbf{коршунова татьяна} У нас на луганщине до майданового шабаша проходили парады вишиванок, олімпіади з української мови та літератури, всі зупинки, стволи шахт, труби котельних були пофарбовані у колір прапора України. А коли на них поїхали танки, то люди , спочатку, виходили на них без зброї, без балоклав, без біт, без коктейлів Молотова. З іконами.

\iusr{Stanislav Tsykalovskyi}
Танки не ехали на людей. Танки ехали по дорогам. Танки ехали по своей стране, защищать свою страну, защищать от русских агрессоров.
Не надо теплое с мягким смешивать.

\iusr{Stanislav Tsykalovskyi}
\textbf{коршунова татьяна} кто говорил? Я не говорил.
Говорили, что гнобят русскоЕ. Я и этого тоже не говорил.

\iusr{Ольга Сидорова Сидорова}
\textbf{Александр Ковалев} слышали слышали как умалишённая колаборантка Поклонская рассказывала про иконы которые останавливали фашисткие танки ((( твари лугандонские

\iusr{Ольга Сидорова Сидорова}
\textbf{Александр Ковалев} кто орал Путин приди принеси нам кровавый русский мир ?)))) мирные граждане ????

\iusr{Александр Ковалев}
\textbf{Stanislav Tsykalovskyi} 

Захищати від російських агресорів, то зрозуміло. А нагадайте будь ласка, якого
числа Росія вероломно напала багатотисячною армією на Донбасі? Нагадаю вам, що
перші озброєнні добровольчі батальйони для відправки на Донбас почали
створювати ще за місяць до Гіркіна з купкою росіян у одному містечку Донецької
області. ЯК розуміти слова Антона Геращенко, який казав, що, цитую-" батальоны
ехали для УСМИРЕНИЯ Донбасса"?

\iusr{Stanislav Tsykalovskyi}
\textbf{Александр Ковалев} не отделяйте Донбасс от всей Украины. Вот вам ответ.

\ifcmt
  ig https://scontent-frt3-1.xx.fbcdn.net/v/t39.30808-6/244990286_4747661348601491_8801527435956769189_n.jpg?_nc_cat=108&ccb=1-5&_nc_sid=dbeb18&_nc_ohc=9CF3yD1_fpAAX_SZV0p&_nc_ht=scontent-frt3-1.xx&oh=7df72b15fb7efe4f7968d66634dacfd2&oe=61A4FE81
  @width 0.4
\fi

\iusr{Александр Ковалев}
\textbf{Stanislav Tsykalovskyi} 

У Криму на НЕ ЗАКОННОМУ референдумі мешканці Автономної Республіки Крим сказали
свою думку. Я це не підтримував. ЗСУ просто пішли звідти. Де у Криму були самі
жорстокі бої? Де ворог поніс великі втрати? Які міста Криму постраждали від
обстрилів і авіаударів окупантів??? А що по Донбасу? Коли Росія вероломно
напала на Донбасі? Чому казали про вгамування Донбасу? Нагадаю, що у Києві
гинули громадяни України, в Одесі в домі профсоюзів загинули грамадяни України,
в Маріуполі 9 травня загинули лише громадяни України, після авіаударів по
Луганську та Станиці Луганській, Сніжному загинули лише громадяни України.

\iusr{Stanislav Tsykalovskyi}
Это ваше мнение.
Уже все случилось. Что дальше? Как вы видите выход?

\iusr{Александр Ковалев}
\textbf{Stanislav Tsykalovskyi} Я розумію, що трапилось, то трапилось. Просто цікаво знати вашу думку. Якого числа Росія вероломно напала на Донбасі?

\iusr{Stanislav Tsykalovskyi}
Думаю она там была задолго до активных боевых действий

\iusr{Александр Ковалев}
\textbf{Stanislav Tsykalovskyi} 

Вы ДУМАЕТЕ?????????? Прикольно. А я, наприклад, вважаю, що як що не Майдан з
підтримкою іноземців, то і Крим, і Донбас, як і раніше дотримувалися
українських законів. Нагадаю вам, що ПЕРШІ захоплення міліції, ОДА,
прокуратури, СБУ, військових частин, підпалювання казарми почалися НЕ на
Донбасі і Криму. Ви згодні????????

\end{itemize} % }

\iusr{Stanislav Tsykalovskyi}
Наш Донбасс не обошли стороной! У нас есть своё.

\ifcmt
  ig https://scontent-frx5-1.xx.fbcdn.net/v/t39.30808-6/245191422_4745412285493064_1188791288766969754_n.jpg?_nc_cat=111&ccb=1-5&_nc_sid=dbeb18&_nc_ohc=ieiuDNvc3h4AX8nVC4v&_nc_ht=scontent-frx5-1.xx&oh=7dcbc30e7a1d84a3f84cdc1cc50e0a61&oe=61A541EA
  @width 0.4
\fi

\iusr{Stanislav Tsykalovskyi}

А вообще мы это все уже прошли в 2004. Вроде на эту тему все было сказано и
разобрано еще тогда. Кто, как и почему нас тогда делил на сорта.

Через двадцать лет что опять?

\ifcmt
  ig https://scontent-frt3-1.xx.fbcdn.net/v/t39.30808-6/245155968_4745316825502610_9135258332530149168_n.jpg?_nc_cat=104&ccb=1-5&_nc_sid=dbeb18&_nc_ohc=TzPv-WFTJPAAX92Oyuf&_nc_ht=scontent-frt3-1.xx&oh=410c7ae3ae940fa67eab92e5ae175765&oe=61A598E3
  @width 0.4
\fi

\iusr{Serhiy Yaryhin}

Не зважаючи на те, що у пості не зустрічається словосполучення "Услыште
Донбас", але сам пост саме про це.

Вже не перший раз звертаюся до Вас - так що Донбас хоче сказати?

І ще питаннячко - а чи чує Донбас, чи має бажання він когось почути?

\begin{itemize} % {
\iusr{Андрій Винниченко}
\textbf{Сергій Яригін} Ну так там в ДНР і ЛНР є люди, які слухають Путіна і його кураторів, воюють проти України, і які заявляють, що вони це Донбасс.

\iusr{Serhiy Yaryhin}
\textbf{Андрій Винниченко}
Заявляти та бути - дві великі різниці  @igg{fbicon.smile} 
\end{itemize} % }

\iusr{Serhiy Yaryhin}
Мені здається, що Ви трошки маніпулюєте.  @igg{fbicon.wink} 
По-перше, ніхто не намагається знищити Донбас.
По-друге, може у Вас лише комплекс меншовартості та хуторянства?

\begin{itemize} % {
\iusr{Андрій Винниченко}
\textbf{Сергій Яригін} Навпаки він себе вважає вищим за українців. Це в нього ксенофобія, а не хуторянство, він же пояснює, що українці для нього чужі, і як він до них ставиться. До росіян у нього ненависті немає, росіян він же не лає, росіяни для нього свої.

\iusr{Ольга Сидорова Сидорова}
\textbf{Андрій Винниченко} так він і написав « в мене претензій до Росії немає»
\end{itemize} % }

\iusr{Misha Stepanets}

Энрике, вы немного исказили причинно следственные связи. Может быть ваши
замечания по поводу костюма рациональны, но это в плоскости дискуссии
краеведов, искусствоведов и т.д. и влияние изображения костюма на идентичность
не так значительно как вы это представили в статье. То что реально влияет на
жизнь оккупированного региона (и возле) так это российская агрессия. На
оккупированных территориях реально идёт ломание тех кто остался, насильная
паспортизация, «российфикация», милитаризация и тд. Вот это проблема о которой
вы даже не слова не сказали...

\iusr{Olha Vesnianka}

Як на мене, мурал не вартий того, аби розпалювати ворожнечу. Різне малюють,
моря в Києві, мексиканські кактуси деінде, жодних проблем з тим

\iusr{Владимир Кабасин}
\textbf{Энрике Анатольевич Менендес}, 

вот удивительное дело. Вчера или позавчера встрял в спор у известного не только
в России священника Андрея Кураева, что, дескать, «укронацики» не различают и
НЕ ОСОЗНАЮТ многогранности различий у собственного народонаселения. При этом у
него, как у всякого русского демократа, есть решение вопроса: «федерализация
(империя)». Это я дословно. Вот так вот, да, два слова одно за другим.

Теперь вот ты, рассказывая про «идентичность Донбасса» внезапно встал на дыбы
от мурала «Слобожаночка» в Авдеевке. То есть ты как раз против, чтобы регионы
Украины культурно обменивались ментальными различиями основываясь на общности
интересов.

Скажи пожалуйста, а почему ему там не быть? Как по мне, то весь Донбасс, за тот
громкий выстрел себе в колени, стоит разрисовать всей Украиной. Потому что это
именно Донбасс внезапно, но достаточно давно, «словил звезду» и обозначил себя
каким-то сверх-уникальным регионом, недостойным любой другой точки на карте
Украины.

Вот погляди на себя, Энрике. В тебе сейчас говорит ксенофоб: «как можно в
Авдеевке рисовать что-то из Слобожанщины! Позор! Катастрофа!»

В своём ли ты уме?

Ну и с другой стороны. У вас там на Донбассе что, своих художников нет
разрисовать собственные серые стены вам приятными муралами? Нет. Живёте среди
серых стен промышленных, и без того не сильно эстетичных, районов, но и
возмущаетесь что рисунок не тот!

Пальмы нужно было нарисовать. Ты бы тоже взвился. Особенно на Новый год.

\begin{itemize} % {
\iusr{Андрій Мокров}
\textbf{Володимир Кабасін} "Словил звезду". Гарно сказано. Зараз шахтарі на непідконтрольній працюють за тарілку супу. То вже нова меньальність, чи нє? А ще там надрачують всіх проти України там, що і в Авдіївці таке можна побачити. Це нова ментальність?
\end{itemize} % }

\iusr{Валерий Прявинов}

Донбасская идентичность. Серьезно? А в чем именно она выражается? Мне случилось
подолгу, месяцами, жить в разных городах Донецкой и Луганской областей в
промежутке 2003-2009. Как и других регионах Украины. И не только Украины. Так в
чем заключается эта самая идентичность, которую разрушает мурал в Авдеевке?
Экспертом себя не считаю, но готов аргументированно полемизировать.

\begin{itemize} % {
\iusr{Венди Белова}
\textbf{Валерий Прявинов} а вам не ответят. Это пост для людей со спИЦальнымм мировоззрением.

\iusr{Александр Лапин}
\textbf{Валерий Прявинов} , в том, что Донбасс порожняк не гонит  @igg{fbicon.wink}  Если что-то пообещал дончанин, то в 99 процентов случаев он расшибётся в лепёшку, но выполнит обещанное. 1 процент - это шлак, который присутствует в любом регионе. Если пообещал киевлянин - ближе к дате сам перезвони ему и напомни, скорее всего, он уже и забыл, что обещал...

\iusr{Алексей Иванов}
\textbf{Валерий Прявинов} Региональная идентичность Донбасса на фоне Украины - это язык, традиционная религия, восприятие истории.

\iusr{Игорь Рязнов}
\textbf{Валерий Прявинов} а в чем заключается украинская идентичность? Вот только не надо говорить про "що не з’їм то надкушу")))

\iusr{Валерий Прявинов}
\textbf{Александр Лапин} 

Вы знаете, у меня был длительный опыт взаимодействия с местными строительными
компаниями. Материалы поставлял в указанный период. И по всей Украине.
2004-2009. Цифры - вещь неумолимая. Дебиторскую задолженность чем восточнее -
тем проблемнее было получать. И бандиты приезжали разговаривать, что негоже у
правильных пацанов требовать свои же деньги не один раз. Киевляне - часто
просто падали на мороз, западенцы - когда понимали, что не управляются в сроки
- заранее просили подписать допсоглашения. Не, на бытовом уровне "лопни, но
держи фасон" - было, как впрочем и везде... Варварская роскошь на центральных
улицах и разруха, грязь и нищета на окраинах. Квалифицированный персонал хрен
найдешь, что компенсируется рекордно низкой заработной платой среди
городов-миллионников. Изнуренные черные лица и артритные пальцы окраин резко
контрастируют с розами в центре. РиЛе - бох, или полубох, а про смерть Грека
просто боятся говорить. Единственное красивое здание СБУ, возле которого даже
остановится поглазеть нельзя бо сразу подходит мент, с вопросом, чего встал и
предложением ехать нахуй (дословно). Золоченые сусальным золотом уличные фонари
на частном ипподроме по 1,5 долл квадратный сантиметр и спивающееся население
окраин,медленно растаскивающее на металл умершее предприятие. Ребят, вам
было-таки чем гордится, но еще больше было чему учится у остальных регионов
Украины. Как по мне, уж простите, "донбасская идентичность сильно сродни
советской идентичности, когда гордимся "потомушто потому". И еще - Донбасс -
это не только Донецк. Как и Киев - не только урядовый квартал. Про Луганскую
область вообще молчу. А что насчет трупов из обрушившихся копанок, которые "по
пацански" вытаскивают и кладут у дороги в посадке? Это если будут разрабатывать
копанку дальше, иначе "умер Ефим - да и хер с ним". И при всех этих прелестях -
Донбасс - это Украина.

\iusr{Валерий Прявинов}
\textbf{Игорь Рязнов} нет у меня ответа. Так я ж и не качаю про "украинскую идентичность" и посягательства на нее. Хотя, как совет - сьездите на север Донбасса, который Луганская область, да поглядите на своих же земляков. Вот и увидите "украинскую идентичность" с вашим, региональным оттенком. Очень поучительно.

\iusr{Валерий Прявинов}
\textbf{Алексей Иванов} а чем отличается язык, традиционнная религия и восприятие истории Харьковщины от Донеччины и обеих от Николаевщины?

\iusr{Александр Лапин}
\textbf{Валерий Прявинов} , 

ну да, наш довоенный Донецк был городом контрастов. Кто искал в нём бандитов -
находил в нём ребят покойного Грека или, на тот момент живых братьев Немсадзе.
Что касается Ахметова, то его сильно демонизировали (скорее всего, с его же
ведома и одобрения), но его реальный вес в Донецке показали события 2014. А что
сказать о смерти Брагина, кроме красивой мемориальной мечети на Октябрьском?
По-моему, её ещё не расхерачили освободители-защитники. Не знаю, в каком году
Вас посылал на Щорса мент возле печально известного после 2014 года здания СБУ,
но, как минимум, с 2010 я там бывал достаточно регулярно как адвокат, и никто
меня никуда не посылал. Да и ментов я там не видел. А кто ехал в Донецк не за
бандитской экзотикой, приезжал на в Областной краеведческий музей, на
«Донбасс-арену» и в «Дружбу», в библиотеку имени Крупской, в академический
драматический театр, или на худой конец, в цирк «Космос» и погулять в парке
Щербакова. Насчёт окраин, так я скажу, что довоенные донецкие задворки были
поухоженней и обустроенней, чем современные столичные локации вроде Водогона
или ДВРЗ. Да что там говорить, локальные клоповники в Киеве встречаются даже в
центре - давно были в зассанном и провонявшем бомжами переходе под Майданом?
Или срачёвнике на кольце скоростного трамвая возле вокзала? После довоенного
Донецка единственный более чистый город , который я пока встретил, это столица
Катара Доха. Там до сих пор за граффити или мусор в неположенном месте могут
руку отрубить. Так чему нам надо было учиться у других регионов? Как мародёрить
во время АТО даст в Песках и высылать добычу домой в родное село? Чем в
остальных, особенно западных регионах, можно похвастаться, созданным людьми, а
не природой?

\iusr{Валерий Прявинов}
\textbf{Александр Лапин} 

спасибо за ответ. Донбасс-Арена тогда строился, когда дела заносили тудой.
Бандитская экзотика сама меня находила, а не я ее искал - идентичность. Насчет
созданного людьми на западной части Украины - да вы шутите. Именно поэтому
туристические маршруты конечно пролегают через Донецк. В любые времена, но в
кавычках. А на западной, кроме природы очень есть, на что поглядеть,
рекомендую. Насчет "самого-самого" это опять же к Союзу. Это когда дать флажок
и похлопать в ладошки куда выгоднее, чем заплатить справедливо заслуженную
оплату. И конечно же, это западенцы растащили Метро при парочке трупов. Еслишо
- трупы тоже носители идентичности. Были. Мне, знаете ли, очень хватило
пронаблюдать "приднестровскую идентичность", причины, развитие и промежуточные
результаты. Тут от меня недалеко, видно хорошо. Насчет мародерства спорить не
могу, потому-как любая война сопровождается этим и еще более ужасными
явлениями. И еще, Донбасс - это не только Донецк.

\iusr{Александр Лапин}
\textbf{Валерий Прявинов} , 

что на Западной Украине есть интересного, кроме природы? Даже во Львове все
архитектурные и кулинарные изыски с лихвой перекрываются фекальным запахом из
несчастной Полтвы, загнанной под землю, не очищаемой полностью и несущих
фекалии батяров и панов аж в Балтийское море. Причём они ещё об этом с
гордостью говорят. И это в самом центре типа европейского города! Или можно
вспомнить закарпатские горные речки, обильно заспанные местным населением, на
что потом жалуются соседи по границе после каждого половодья. Да, чуть не
забыл, как насчёт копателей янтаря и дровосеков древесины для ЕС? Чем это лучше
наших копанок? А ещё можем поговорить о ксенофобии...

\iusr{Валерий Прявинов}

Безусловно, ксенофоб из Львова вежливо игнорирующий речь на неприятном для него
языке куда опаснее типичного камбродского пацыка. Правда, без кастета. Или
Каменный Брод, район Луганска, таки-да старый - не Донбасс?

\iusr{Александр Лапин}
\textbf{Валерий Прявинов} , представьте в любом городе, и даже в любой стране, есть районы, куда не рекомендуется забредать нормальным людям во избежание трагических последствий...

\iusr{Валерий Прявинов}

И да, мне невозможно представить, что незаконные лесорубы как и копачі янтаря
бросят своего погибшего товарища по промыслу в посадке возле дороги, как
непотребную вещь. Понятно, что конкуренция - другое дело. Я ж про товарищей,
которые носители идентичности. Про людей, а не про незаконность бизнеса.

\iusr{Алексей Иванов}
\textbf{Валерий Прявинов} 

ну зачем вы претворяетесь? Донбасс разговаривает на русском, в тот момент как
основная часть населения Украины в говорит на уже ядрёность суржике и
украинском. Я уже молчу про полное игнорирование русского языка в а официальном
уровне, даже во времена Януковича. К 2014 году в четырёх районах Донецкой
области не было не то, что русскоязычных школ, а даже русскоязычных классов. На
Украине действовало около 200 законов и законодательных актов ограничивающих
использование русского языка. Религия, мы сейчас говорим про традиционное
позиционирование населения, а не про фактическое участие население в
религиозной жизни. Так вот. 80\% населения Донбасса соотносило себя с церковью
Московского Патриархата. Такие проценты были разве, что в Крыму. На остальной
Украине это были или греко- католики, либо разные раскольничьи группировки
УАПЦ, Киевский Патриархат. Этнический компонент даже по значительно
сфальсифицированный переписи 2001 года в Донецке и Макеевке, Енакиево русских
больше, чем других этносов. Что в реальности значит, что русских в РЕАЛЬНОСТИ
просто математически больше в регионе. То есть их около 60\%. Даже тут у
большинства кто топит за Украину в большинстве своём русские фамилии. И топят
за Украину на русском языке. ЭТОГО, что я перечислил уже достаточно, что бы
НИКОГДА не заловить глупый вопрос почему в Донбассе произошёл 2014 год. Тут всё
ясно и очевидно.

\iusr{Венди Белова}
\textbf{Александр Лапин} порожняк - это состав пустых вагонов.

\iusr{Александр Лапин}
\textbf{Валерий Прявинов} , 

а что думаете, умерших на нелегальной добыче янтаря провожают на лафете с
фанфарами товарищи?

Скорее всего, всё так же - оттащили от места добычи подальше, а ближе к ночи
вызвали копов анонимно...

Цитата: «Отака в нас робота. Це тяжка робота. І у хлопців, які з помпами,
тяжка. От сьогодні чоловік молодий ще, впав і вмер біля підстанції (на одному з
Клесівських родовищ. -Ред.). Впав і вмер, кажуть тромб відірвався. Не витримав.
Зараз багато молодих вмирає», - додає вона».

\href{https://www.bbc.com/ukrainian/news-39151322}{%
Полісся: бурштин, життя і смерть, bbc.com, 03.03.2017%
}

\iusr{Александр Лапин}
\textbf{Венди Белова}, 

Вы якобы жили в Донецке и не знали, что это слово имеет, как минимум, два
значения? Попробуйте догадаться, в каком из них его употребил я выше. Уверен,
что с помощью словаря справитесь  @igg{fbicon.face.tears.of.joy} 

\url{https://ru.wiktionary.org/wiki/порожняк}

\iusr{Игорь Рязнов}
\textbf{Валерий Прявинов} ну так значит нет украинской идентичности! Значит и Украину нужно отменить. Нет идентичности - нет Украины.

\iusr{Венди Белова}
\textbf{Александр Лапин} это тюремный жаргон. Вы его считаете идентичностью? Ну, я свой город не ассоциирую с таким примитивом.

\iusr{Александр Лапин}
\textbf{Венди Белова} , это не тюремный жаргон, а донецкий. Такой же как «тормозок». Не только ведь шахтёры его берут с собой на работу, правда?

\iusr{Валерий Прявинов}
\textbf{Александр Лапин} я не думаю - я знаю. Криминальные вуйки не боятся ответственности, приносят погибшего в родные дворы, оплачивают похороны, заказывают ритуальные службы и помогают семье погибшего. А не оставляют как ненужную вещь в посадке.

\iusr{Александр Лапин}
\textbf{Валерий Прявинов} , 

тогда Вам виднее. Лично я склоняюсь, что в любой маргинальной среде рано или
поздно устанавливаются примерно одни и те же правила игры - «человек человеку -
волк». Да и не в маргинальной тоже. Киевские бомжи трусы своих товарищей,
например не только в посадке оставляют, но ещё и раздевают.

Вы тыкаете нам копанками, но копанки до войны были уделом маргиналов без роду
без племени: Алкоголиков, бомжей и т.д. Нормальный донецкий мужик с руками не
из задницы и головой на плечах, и не п..здабол (а это довольно быстро
выяснялось) всегда мог найти работу.

\href{https://24tv.ua/kyivnews/ru/kieve-teligi-nashli-mertvuju-goluju-kriminalnye-novosti-ukrainy_n1655189}{%
"Лежит человек, а возле летают мухи": в Киеве в кустах обнаружили мертвую голую женщину – видео, 24tv.ua, %
13.06.2021%
}

\iusr{Валерий Прявинов}
\textbf{Алексей Иванов} 

я еще раз повторяю вопрос - с которого начал - каким боком мурал "Слобожаночка"
на стене в Авдеевке разрушает "донбасскую идентичность". Какого милого донецкие
расписываются за весь Донбасс, украинскую его часть? Это можно было бы
игнорировать, когда бы подобная позиция не была бы питательной средой для куда
более опасных сепаратистских движений. В питательный бульончик подсыпали
истеричной пропаганды, добавили оружия и микробов - всевеликих кизяков,
гиркиных и прочей сволочи. И получилась опухоль, которая мучает и убивает не
только весь организм, но и первичную питательную среду, которая идентичные.

Далее: Московский Патриархат является доминирующей ветвью православной религии
на территории ВСЕЙ Украины. К сожалению.

Русский язык также является наиболее распространенным в употреблении на
восточных и южных территорииях Украины. Особой ценностью не является, мое
мнение. А вот якорем, на который прекрасно цепляется русская шовинистическая
пропаганда-является.

Кстати, вас не смущает, что топит за донбасскую идентичность человек Энрике
Менендес? Нет?

В наших краях водится такой борцун за русско-одесскую идентичность, Морис
Ибрагим, уроженец города Лебанон. Так тот топит за кокошники и Пушкина с милым
ливанским акцентом. Ну, тут Одесса, пафосу предпочитают юмор. Потому и
"защитничек" такой забавный и колоритный.

Еще раз повторюсь - носителям "донбасской идентичности" было и есть чем
гордится, равно как и учится у других регионов. И это касается всяких других
идентичностей. Уникалов с Закарпатья, Бессарабии, Буковины.

\iusr{Александр Лапин}
\textbf{Валерий Прявинов} , 

пристёбываться к фамилиям и личностям оппонентов, вместо того, чтобы
оппонировать по сути - вот это и есть проявление ксенофобии несвойственной
Донбассу. У нас там до 2014 года мирно жили и вместе трудились разные этносы, и
никто никого не клевал за и происхождение. Вы ещё нам с Энрике уши и носы
поменяйте, благо расчеты от ваших исторических учителей наверняка остались. А
потом мантры «в Украине нет нацизма». А это что? Или «это другое»?

\iusr{Валерий Прявинов}

Игорь Рязнов вы меня агрессивно воспринимаете. Возможно, я недостаточно ясно
выражаю свою мысль. Донбасс, какой он был - это плавильный котел. Не остывший,
да еще серьезно разогретый в 2014. И про идентичность говорить не получается.
Обосновано, имею ввиду. Когда и если бы подобную риторику не использовали
пропогандисты для фактического ухудшения ситуации - да и бог бы с ним. Насчет
украинской идентичности - а вот именно, что ее и можно найти в ассортименте на
западной и центральной части Украины. Потому что веселейшие времена военного
коммунизма и массового террора их коснулись на поколение позже и уже на
затухании. Там - неурожай, тут - Голодомор. Там ущемления, тут репрессии. Там
Бандера в тюрьме за убийство Перацкого - тут враги
революции/классовые/бельгийские шпионы/антисоветчики/заговорщики.

Соответственно, там где меньше выпалывали население и сохранились носители
идентичности. Видел я и идентичность региона Донбасс, украинской части. Найти
несложно - это взрослые люди, которые могут поправить могилку своих прабабушек
не выходя за пределы региона, а тем более страны. Вам не понравится.

Что касается переселенцев в 1-2 поколении - извините, так конь не ходит. Пассажиры корабля, которые решили, что они его владельцы? Серьезно?

\iusr{Александр Лапин}
\textbf{Валерий Прявинов} , 

вы, одессит, постоянно пытаетесь делить нас, донецких, на какие-то угодные
истинным украинцам сорта: то по именам, то по цензу осёдлости, то ещё как-то.
Не припомню, чтобы кто-то из донетчан, родившихся и выросших в нашем регионе в
этом делении нуждался. Это в Галиции до сих пор слово на букву «ж» - не
оскорбление (по мнению «титульной нации»). Представьте, что Голодомор был
везде, где проживали крестьяне, которых коммунисты коллективизировали.
Погуглите голод в Воронежской, Липецкой областях. Даже на обычно сытой Кубани
он был. Ну и так для справки могилы моих обоих бабушек и одного из дедушек все
в Донецкой области, от Мангуша до Доброполья. Так что нехер мне тут
рассказывать про «пассажиров». Я же не рассказываю вам, в каком поколении
считаться одесситом...

\iusr{Валерий Прявинов}
\textbf{Александр Лапин} 

нет, уважаемый оппонент. Никто уши с носами вам измерять (вы же это имели
ввиду) не собирается. Это методология северных братушек, скорее. И я не
пристебываюсь к имени и фамилии, а задаю абсолютно резонный вопрос - каким
боком человек с испанскими/латиноамериканскими родителями не отстаивает
сохранение кастильских, арагонских или картахенских идентичностей? Имеет ли он
на это моральное право и для чего он это делает?

А также: я потомок одесских идентичностей, "каратель 2 мая", просто "каратель".
И я могу показать не выезжая с родного города, с какого дворика выводили моих
предков в гетто в 1941, в каком подвале в центре Одессы умерли от голода и
испанки мои прямые предки в 1918-1920 гг и пишу вам из дому, который стоит на
месте старого дома, который стоял на месте старого дома, который купил мой
предок-казак, переселенец с Дона. И таки-да, имею право говорить за одеську
ідентичність украінського міста Одеса. И, конечно же, не давал, не даю и не
собираюсь давать возможностей нести горе в любимую Одессу. Самым прямым и
безусловно негуманным образом, если нет других вариантов.Без штангенциркуля,
зато исключительно за дело. Если вы действительно болеете за родной регион, то
для начала неплохо бы обьективно оценить результаты и проанализировать на
причинно-следственные связи! Было-стало и почему.

\iusr{Александр Лапин}
\textbf{Валерий Прявинов} , 

вот и занимайтесь своей родной одесской идентичностью. Зачем лезете с
непрошеными советами в нашу донецкую? Мы тут со своими как-то сами разберёмся.
А Энрике Менендес, если вдруг вы на знали, потомок испанца, вывезенного в СССР
в ходе Гражданской войны в СССР. Недавно он подробный пост писал о своём деде и
прочей родне. Мы с ним учились в одной школе в Артёмовске. Так что не вам
рассказывать о Донецкой идентичности ни ему, ни мне. Энрике - дончанин. Им
родился, им и живёт. А подъёбки по поводу его испанских корней оставьте себе.
На этом в продолжении дискуссии не вижу смысла.

\iusr{Валерий Прявинов}
\textbf{Александр Лапин} 

извините. Но внимательно посматривать на идентичность все же придется. Я даже
наглядно покажу, почему. Про указывать или более - так по ситуации. На снимках
ниже примечательный микроавтобус до и после 02.05.2014. Покажете аналогичное
транспортное средство, которое возило боевичков с оружием до начала войны?
Типа, вот любимым донбассцам для защиты идентичностеи от укронацистов. Вот в
этом проблема. Ранее проблема была в том, что это не одесские бандиты палили
донецкие бизнесы и убивали людей при полном попустительстве (точнее, в доле)
одесского генерала а ровно наоборот. Проиллюстрировать, или поверите на слово?

\ifcmt
  ig https://scontent-frx5-1.xx.fbcdn.net/v/t39.30808-6/245827914_3063408520598780_6276887334770277535_n.jpg?_nc_cat=110&ccb=1-5&_nc_sid=dbeb18&_nc_ohc=dgmZBRgjlyMAX9HzNAt&_nc_ht=scontent-frx5-1.xx&oh=10bd34f62f9611fd9426bbc0e1e8556a&oe=61A41DBB
  @width 0.4

	ig https://scontent-frt3-1.xx.fbcdn.net/v/t39.30808-6/245903807_3063408947265404_6657737209785863006_n.jpg?_nc_cat=107&ccb=1-5&_nc_sid=dbeb18&_nc_ohc=EnMbKhR6VYUAX9VDf6V&_nc_ht=scontent-frt3-1.xx&oh=bd8628c2f6af22592c109280d9053629&oe=61A483A7
  @width 0.4
\fi

% -------------------------------------
\ii{fbauth.lapin_aleksandr.transhumanism_inc}
% -------------------------------------

\textbf{Валерий Прявинов} , и что? Спросите у того, кто его так разрисовал.
Вполне возможно, что это был какой-то «капитан какао» или ещё какой-то из ваших
упоротых? То, что вы сейчас делаете называется попыткой коллективной
ответственности. Попробуйте представить, что вас заставляют отвечать на вопросы
о преступлениях Стерненко, к которым вы (я надеюсь), не имеете никакого
отношения. Только потому что вы тоже из Одессы. Приятно было бы?

\iusr{Игорь Рязнов}
\textbf{Валерий Прявинов} так ты определись уже сам написал нет украинской идентичности @igg{fbicon.face.confused} . Если есть то в чем выражается? А то пишешь, пишешь и нечего конкретного, случайно не депутат какой то)))
И кто такие пассажиры корабля? Украинцы что ли? Ну как не крути эти земли завоевывала российская царица, расселяла здесь сербов, венгров то же она.

% -------------------------------------
\ii{fbauth.prjavinov_valerij.odessa.ukraina}
% -------------------------------------

\textbf{Александр Лапин} 

дискуссия не всегда бывает приятной. Я готов дать свое оценочное мнение
действиям Стерненко, тем более, был неплохо знаком с Сергеем. И, как нормальный
укронацист, готов отвечать за сопричастность к движению. Про землячество речь
не идет - Стерненко родился и вырос... в Бессарабии. Которая регионально -
Одесская область, а вот идентично - нет. Разные субкультуры. Да, здесь учился и
попал в движение. Наперед скажу - действия Стерненко считаю оправданными, как и
действия любого защищающегося от физического воздействия бандитов. Не обязан
худощавый одноглазый хлопец оценивать заказанную двум быкам степень воздействия
на себя. Но это в любом случае есть и будет его индивидуальная ответственность
и, учитывая заполитизированность процесса - Сергей, к сожалению, сядет.
Наиболее вероятно. Была проделана масса оперативной работы по горячему, потому
картина произошедшего мне в достаточной степени ясна. То, что это дополнительно
демонизирует правые движения также очевидно. Одессит Грациотов также убил
ударом ножа в живот одного из двух пьяных быков, которые напали на него с
девушкой и затаптывали парня. Норм, по итогу был оправдан.

А бусик именно в таком виде приехал в город весной 2014 в помощь боевикам
Антимайдана. Понимаете? Это из одесского санатория Куяльник выбрасывали
донбасскую гопоту, свезенную калечить здоровье и имущество евромайдановцев. И
это то, которое идентичное и порожняк не гонит, наряду с интеллектуалами вроде
Менендеса.

Знаете, как тяжело жить в черноморском городе, бывшем Кочубееве, ныне Одесса?
Город сплошь изуродован образчиками чуждой архитектуры, которая разрушает уже
столетия местные степные идентичности?

\iusr{Александр Лапин}
\textbf{Валерий Прявинов} , 

сочувствую, что вам тяжело жить в Одессе. Надеюсь, как-то справитесь. Но не
надо, пожалуйста, ставить знак равенства между Донецкой идентичностью и
преступниками. В Одессе криминалитета не меньше. А в отдельные периоды истории
было и побольше. Мне кажется, это очевидно, но вдруг непонятно: люди одной
Донецкой идентичности могут иметь разное этическое происхождение, разное
вероисповедание и тем более различные политические взгляды. Если у вас до сих
пор есть какие-то претензии к пророссийски встроенным дончанам, им их,
пожалуйста, и предъявляйте.

\iusr{Валерий Прявинов}
\textbf{Игорь Рязнов} 

вы полагаете, что до походов Екатерины была безлюдная пустыня? Нет, уважаемый,
было большое количество таки идентичного украинского населения, впоследствии
размытого и уничтоженного (далеко не полностью) имперской политикой Российской
империи, Советского Союза, войнами, индустриализацией. И она, идентичность,
сохранилась в регионе. Потому не стоит Донецку расписываться за весь Донбасс.
Результаты можно оценить без микроскопа.

\iusr{Валерий Прявинов}
\textbf{Александр Лапин} 

вот видите. Дискуссия выходит вновь в нормальное русло и это прекрасно. Но мне
нужно на капельницу ехать, потому с интересом продолжу немного позже.

\iusr{Александр Лапин}
\textbf{Валерий Прявинов} , 

так а что тут продолжать. Вы, одессит, отказываете нам с Энрике в праве на
свою, донецкую, идентичность. О чём дальше говорить. К счастью, лично мне от
вашего мнения по этому поводу, как и от любого другого, ни холодно, ни жарко.
Никто не отберёт у меня ни место моего рождения, ни родной язык, ни право
считать себя дончанином.

\iusr{Игорь Рязнов}
\textbf{Валерий Прявинов} 

откуда могло взяться украинское население, если Украины в те времена не было?
Да и про Советский Союз, про размытие, смешно. Случайно не руководство СССР
создало УССР, наделило территориями, сделало участником ООН. Записало людям в
паспорта национальность - украинец!!! Ведь если б этой записи не было сейчас
нельзя было и говорить о такой национальности!!! А то что при СССР заставляли в
школах учить украинский язык. Так о каком размытии ты говоришь?

И ведь как не крути ЛНР и ДНР - это примерно тоже самое, что 100 лет назад УНР
и ДНР. Даже название похожи. Их тогда тоже никто не признал. А теперь благодаря
СССР есть Украина и Белоруссия. И если ты признаешь Украину то придется со
временем признать и ЛНР и ДНР. И говорить уже о новых национальностях))). Как
не крути, но история развивается по спирали. Между прочим ЛНР и ДНР уже сейчас
просуществовали намного дольше чем УНР и БНР 100 лет назад!!!

\iusr{Валерий Прявинов}
\textbf{Игорь Рязнов} 

существование Украины это и есть доказательство существования украинской
идентичности. А это наличие культуры, обычаев, традиций и языка, помноженное на
время. Другой вопрос, что украинский народ длительное время был лишен
государственности, максимум - автономия оккупированных областей. Оккупированых
чаще всего Польшей и предшественниками, Россией и ее предшественниками, Турцией
и ее предшественниками. Это были империи, две из которых распались до
есстественных унитарных государств. И то, проблема сепаратизма в Турции стоит
остро, ибо большАя часть территории и населения инородны (исторический
Курдистан). Россия - взагали неоимперия, которая удерживается в своих нынешних
границах только силой, а не лояльностью лоскутов. Прекращаешь проявлять силу -
и все разлезается. Куда более сильные и богатые империи расползлись даже в 20
веке: например Британская и Советский Союз. Неполностью, пока что.

Далее: теперешняя ситуация Донбасса это последствия незаконченно эксперимента
по созданию усредненного советского человека, чем занимался Союз все время,
когда не воевал. Потому я и сравнил Донбасс с неостывшим плавильным котлом.
Котел остывает столетиями и из населения появляется народ, еще спустя столетия
- появляется нация. Понимаете? Население-народ-нация. Население можно
переселить, народ - покорить, а нацию - толтко уничтожить. Нынешний виток
российско-украинской войны лишь ускорил переход народа Украины в состояние
нации. А истребить нацию в количестве десятков миллионов человек не рискнет
даже взбесившаяся бензоколонка. Частично разрушить Украину России сил
технически хватает, а вот покорить и тем более истребить - нет. Это не вышло
даже у немцев, безусловно лучшей и сильнейшей армии мира своего времени. Это я
про Вторую Мировую. Уже тогда происходило развитие украинского народа в
состояние нация. Это я про УПА и поддержку населением своих воинов. Вообще,
западной части Украины здорово повезло пересидеть под оккупацией Польши по
сравнению с несравненно более жестокой оккупацией восточной Украины Советским
Союзом. Это когда истреблялись по классовому признаку десятки миллионов
населения, а народы угнетались и переселялись. Потому в Тернопольской области -
неурожай, а в Хмельницкой - Голодомор. Это как авария по сравнению с
катастрофой.

Далее: ЛНР и ДНР - это искусственные образования. Потому как речь идет только о
населении этих формаций. Не о народе. А народов там основного населения два -
украинский и русский.

Теперь смотрим нынешнее положение: происходит деиндустриализация оккупированной
части Донбасса, происходит стабильная убыль населения как естесственная так и
эмиграция. О международном признании государственности речь не идет и не может
идти. Далее: в войне задействовано не менее 3 процентов населения непризнанных
республик, то есть куча трудоспособного народу не производит а потребляет. Это
примерно эквивалентно, если бы Украина ебанулась и вдруг слепила армиб примерно
800 тыс. человек, а Россия - 4 млн. Это не выдержит никакая экономика, потому
перспективы позитивного развития самостоятельных республик равны 0.0\%. Эти
непризнанные республики закончатся ровно тогда, когда перестанут быть нужны
России как инструмент давления на Украину. Крыма это не касается - россияне
готовили его давно и уйти к Украине Крым сможет только при факте распада
России.

Мое мнение: война еще долго будет продолжаться в таком вялотекущем режиме, это
тормозит Украину на пути в Европу и делает невозможной признание
государственности непризнанных республик.

Да даже смотрящий за ДНР - Антюфеев, представитель старой, еще советской гбшной
сволочи. Который по странному стечению обстоятельств занимал аналогичный пост
лет 20 в Приднестровье. Совпадение, да? Приднестровье никто не признал уже 30
лет, экономика умерла, стабильная убыль населения. В среднем у мужчин 3-4
паспорта: приднестровский, молдавский, украинский, российский. Ничего не
напоминает?


\end{itemize} % }

\iusr{Санта Иванова}

Донбасс это всё-таки сплав нардов-языков-национальностей. И самоидентификация не
касалась этого сплава. Больше характеризовался характером который из этого всего
получился по региону.


\iusr{Alla IIvanova}

Кто такой Казанский стоит ли обращать внимание на его алкоголические бредни?

Я не знаю, какой у него высер получился на этот мурал в его говнобложике,
может, давайте, посмотрим на это, как на неплохое полотно на стене.

Надеюсь скоро все эти казанские, гармаши, мацуки уйдут из политического поля
жителей на Донбассе.

Предателей своих земляков никто не уважает! Их презирают даже те, кто их
пользует!

Не обращайте внимание на алкоголический бред этих человекоподобных, иначе их
назвать нельзя.


\end{itemize} % }
