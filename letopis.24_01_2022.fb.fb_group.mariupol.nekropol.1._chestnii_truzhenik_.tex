%%beginhead 
 
%%file 24_01_2022.fb.fb_group.mariupol.nekropol.1._chestnii_truzhenik_
%%parent 24_01_2022
 
%%url https://www.facebook.com/groups/278185963354519/posts/681125753060536
 
%%author_id fb_group.mariupol.nekropol,marusov_andrij.mariupol
%%date 24_01_2022
 
%%tags mariupol.nekropol,mariupol.istoria,istoria
%%title "Честный труженик" Парфений Малышкин: загадка разгадана
 
%%endhead 

\subsection{\enquote{Честный труженик} Парфений Малышкин: загадка разгадана}
\label{sec:24_01_2022.fb.fb_group.mariupol.nekropol.1._chestnii_truzhenik_}
 
\Purl{https://www.facebook.com/groups/278185963354519/posts/681125753060536}
\ifcmt
 author_begin
   author_id fb_group.mariupol.nekropol,marusov_andrij.mariupol
 author_end
\fi

\textbf{\enquote{Честный труженик} Парфений Малышкин: загадка разгадана}

Чудом сохранившийся, этот памятник одиноко стоит на окраине Некрополя. Парфений
Яковлевич Малышкин, умер 19 апреля 1906 года, 62 лет от роду. Необычным его
делает эпитафия: \enquote{\em Честному труженику от Мариупольской городской думы}. Кем был
Парфений Малышкин? За какие заслуги дума поставила ему памятник?

\ii{24_01_2022.fb.fb_group.mariupol.nekropol.1._chestnii_truzhenik_.pic.1}

\#люди\_мариупольского\_некрополя

Парфений Малышкин родился в селе Троицком Павлоградского уезда в семье
государственных крестьян. Женился на односельчанке Анне Макаровне. Переехал в
Мариуполь – в поисках лучшей доли. У него родились дочери. В 1871 году
крестником одной из них, Александры, стал Авксентий Дьяченко-Белый, полицейский
надзиратель, который через пару десятков лет выбьется в люди и станет гласным
думы, председателем Сиротского суда (его надгробие сохранилось)...

К сожалению, супруга Малышкина заболела какой-то психической болезнью. В 1887
ее лечили в Екатеринославской губернской земской больнице \enquote{от
умопомешательства}. Из-за бедности ее муж так и не смог оплатить лечение (7
рублей 50 копеек).

Эти факты краеведы
\href{https://www.facebook.com/profile.php?id=100014684651188}{Сергій Катрич} и
\href{https://www.facebook.com/helga.buzlami}{Helga Buzlami} установили в
течение года после обнаружения памятника. Но тайна эпитафии оставалась
нераскрытой.

К поискам подключились Юлия и \href{https://www.facebook.com/vadikor}{Вадим
Коробка}, доценты кафедры исторических дисциплин Мариупольского
государственного университета. Благодаря их статье \enquote{Таємницю епітафії
розкрито} - загадку можно считать разгаданной! Итак, цитирую их находки:

В \enquote{Отчете Мариупольской городской управы о приходе и расходе городских сумм} за
1906 год указывается, что Парфений Малышкин был служащим канцелярии управы. На
его похороны из городской казны было потрачено 47 рублей и 23 копейки.

В посмертной служебной характеристике Парфения Яковлевича говорится, что 

\begin{leftbar}
\noindent\em он прослужил в канцелярии управы около 35 лет, был одним из самых
добросовестных, аккуратных и трудолюбивых служащих, выполнял свои обязанности
самым безупречным образом, приходил в канцелярию всегда первым и уходил
последним.... вообще можно сказать, что он работал не покладая рук.	
\end{leftbar}

После его смерти остался только домик, в котором он жил с больной женой.
Замужние дочери обратились в управу с просьбой выделить средства на пожизненное
содержание матери (35 рублей в месяц).

На заседании в мае 1906 года гласные городской думы удовлетворили их просьбу и
решили ежегодно выделять дочерям из городского бюджета 300 рублей на содержание
Анны Малышкиной.

Кроме того, дума выделила 100 рублей на установку памятника самому Парфению Малышкину.

Как выяснили мариупольские историки, до 1911 года сумма в 300 рублей выдавалась
регулярно, а вот в 1913 году ее почему-то сократили до 275 рублей.

Интересно, что в 1913 году всего 18 мариупольчанок получали денежную помощь из
городского бюджета. Вероятно, это были вдовы служащих городского самоуправления
(например, семейство умершего регистратора П.Пономарева получило 80 рублей)...

Статью \enquote{Таємницю епітафії розкрито} см. здесь \enquote{АКТУАЛЬНІ ПРОБЛЕМИ НАУКИ ТА
ОСВІТИ: Збірник матеріалів XXIІІ підсумкової науково-практичної конференції
викладачів МДУ / За заг. ред. М.В. Трофименка. – Маріуполь: МДУ, 2021}
(страницы 105-106) \url{https://cutt.ly/8IMXOvM}

P.S. Спасибо \href{https://www.facebook.com/profile.php?id=100013986611400}{Дейниченко Елена} за помощь в подготовке этого материала! 🙂 

%\ii{24_01_2022.fb.fb_group.mariupol.nekropol.1._chestnii_truzhenik_.cmt}
