% vim: keymap=russian-jcukenwin
%%beginhead 
 
%%file 26_08_2021.fb.rjurikov_andrej.1.ludi_neludi_odessa
%%parent 26_08_2021
 
%%url https://www.facebook.com/andry.rurikov/posts/4193869194065339
 
%%author Рюриков, Андрей
%%author_id rjurikov_andrej
%%author_url 
 
%%tags chelovek,odessa,ukraina,zhizn
%%title Есть люди, а есть нелюди
 
%%endhead 
 
\subsection{Есть люди, а есть нелюди}
\label{sec:26_08_2021.fb.rjurikov_andrej.1.ludi_neludi_odessa}
 
\Purl{https://www.facebook.com/andry.rurikov/posts/4193869194065339}
\ifcmt
 author_begin
   author_id rjurikov_andrej
 author_end
\fi

Есть люди, а есть нелюди. Жил некоторое время назад  на Молдаванке один такой
нелюдь. Жил, правда, недолго. Но зла успел натворить. Даже родным и близким
людям успел испортить жизнь. Фамилия, имя и отчество подлеца изменены, так как
его дочь  живет на Молдаванке , назло всем превратностям судьбы растет
приличным человеком и учится в медицинском университете....

Одесские зарисовки.

Холера нашего двора.

Родился Петр Алексеевич Лившиц в те августовские  дни 1970-го года, когда
Одесса, изолированная от всего внешнего мира, корчилась от холеры.

-Холера ясна,- сказала мать-полька новоявленного миру и была таки права, видать
предчувствуя материнским сердцем, что её сын принесёт родным, близким и соседям
немало горя.

Опустим тот нежный возраст мальчика Пети, когда  битые стекла у пенсионерки
Лидии Йосифовны в лютый февраль и выливание с балкона кипятка на головы
прохожих казались злодейством только самим потерпевшим, а все остальные
воспринимали происходящее только как причуды молодого растущего организма. Но
организм по имени Петр рос и вырос в матёрого уголовника Петра Алексеевича, не
очень удачно для его пятой точки отсидевшего два раза за изнасилования. После
второй отсидки страдающая от издевательств от  приличных уголовников пятая
точка подсказала Петру Алексеевичу, что справлять свои естественные мужские
потребности надо законным способом и привела её обладателя в ЗАГС по улице
Степовой на роспись со жгучей  красавицей с черными счастливыми глазами Фаиной
Рашидовной Билялетдиновой, после чего молодожёны поселились в двухкомнатной
квартире Фаины в нашем дворе. Петр Алекеевич был человеком не стеснительным и
уже на четвёртый день медового месяца, выйдя из свадебного запоя показал всю
широту души, избив  жену, прозванную им "законной грушей" и пропив  дорогой,
подареный им на свадьбу сервиз. Где-то на четвёртый месяц супружества сервизы и
счастье в глазах Фаины закончились, но появилась беременность, которая
счастливо разрешилась девочкой в том-же роддоме, где тридцать три года назад
мать Петра судьбоносно выругалась. "Теперь у меня будет две законные груши"-
сказал новоявленный отец и стал в пьяном угаре с ещё  большей жестокостью
избивать жену до потери молока у последней. 

Время шло, переломам и лежаниям в травматологии  Еврейской больницы, где Фаина
стала  родным человеком, уже пропал счет, Петра Алексеевича несколько раз
соседи увещевали ( и даже ногами) изменить образ жизни и отношение к
окружающим, но всё было тщетно. 

И вот Петр Алексеевич Лившиц наконец-то сделал в своей жизни первое доброе
дело! Оно оказалось и последним. Он умер. Умер от того, что на украденные у
дочки-студентки деньги, которые ей оставила   в очередной раз лежащая в
травматологии мать на начало учебного года в медуниверситете, он выпил спирта.
Спирт оказался не только "паленым", но и древесным, со всеми вытекающими для
организма "Холеры ясной" последствиями.

....На похоронах Лившица сопровождали гроб две одинокие фигуры дальних
родственников из  Херсона.

"Сдох, холера"- сказали соседи......
