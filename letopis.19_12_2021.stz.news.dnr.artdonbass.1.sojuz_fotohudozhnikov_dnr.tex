% vim: keymap=russian-jcukenwin
%%beginhead 
 
%%file 19_12_2021.stz.news.dnr.artdonbass.1.sojuz_fotohudozhnikov_dnr
%%parent 19_12_2021
 
%%url http://www.artdonbass.ru/ru/news/ezhegodnoe-sobranie-ts-soyuz-fotohudozhnikov-dnr-proshlo-v-hm-art-donbass.html
 
%%author_id 
%%date 
 
%%tags 
%%title Ежегодное собрание членов Творческого союза «Союз фотохудожников ДНР»
 
%%endhead 
\subsection{Ежегодное собрание членов Творческого союза «Союз фотохудожников ДНР»}
\label{sec:19_12_2021.stz.news.dnr.artdonbass.1.sojuz_fotohudozhnikov_dnr}

\Purl{http://www.artdonbass.ru/ru/news/ezhegodnoe-sobranie-ts-soyuz-fotohudozhnikov-dnr-proshlo-v-hm-art-donbass.html}

Вчера, 18 декабря, в стенах художественного музея «АРТ-Донбасс» прошло
ежегодное собрание членов Творческого союза «Союз фотохудожников ДНР».

\ii{19_12_2021.stz.news.dnr.artdonbass.1.sojuz_fotohudozhnikov_dnr.pic.1}

На встрече были подведены итоги работы Союза за 2021 год.  Председатель
Творческого союза Артем Поваров представил подробный отчёт, который очень
впечатлил собравшихся. Проведено 7 персональных выставок в Художественном музее
«Арт-Донбасс», Макеевском художественно-краеведческом музее, Музее истории
города Харцызска, Горловском художественном музее. Передвижную выставку
«Рукотворные святыни», в которую вошли более 50 фоторабот, посмотрели жители
Макеевки и Харцызска. 

\ii{19_12_2021.stz.news.dnr.artdonbass.1.sojuz_fotohudozhnikov_dnr.pic.2}


Фотографы Республики приняли активное участие в
республиканских и всероссийских конкурсах, одержав победу. Среди них ХI-й
открытый фотоконкурс имени В.А.Собровина, г. Белград, VI Российский
фотографический конкурс «Светопись-2021» в номинации «Художественная
фотография» в Русском музее фотографии, г. Нижний Новгород, межрегиональный
фотоконкурс «Народы Сибири: между прошлым и будущим» г. Иркутск,
республиканские конкурсы фотографии: «Зимний пейзаж моего города» и «Осенняя
пора». Так же члены союза активно взаимодействуют с общеобразовательными
учреждения города и Республики принимая участия в работе и жюри проводя занятия
мастер-класс. Радиопередачи и телевизионные сюжеты республиканских каналов
освещали деятельность Союза.

В ходе встречи обсудили планы мероприятий на будущий год, а это персональные
юбилейные выставки в музеях Республики, новый передвижной проект «Мосты –
изящные творенья» совместно с художественным музеем «Арт-Донбасс» в музеях
Республики, тематические выставки в музеях России, в рамках интеграционной
программы «Россия-Донбасс», проведение занятий мастер-класс, организация III
открытого городского фотоконкурса, посвященного Дню города и Дню шахтера.
Также познакомились с кандидатами для вступления в Союз и их творчеством. Общим
голосованием были приняты новые члены ТС «Союз фотохудожников ДНР», которыми
стали Алина Боднар, Евгения Карпачева, Наталья Афанасьева, Андрей Жданов.
Художественный музей благодарит ТС «Союз фотохудожников ДНР» за активную
жизненную позицию и творческую работу «без штампов»! Впереди еще один
творческий год, который подарит много интересного.

Репортаж: Виктор Черемисин.
