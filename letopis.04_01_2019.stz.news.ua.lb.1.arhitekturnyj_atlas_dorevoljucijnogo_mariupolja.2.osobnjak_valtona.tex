% vim: keymap=russian-jcukenwin
%%beginhead 
 
%%file 04_01_2019.stz.news.ua.lb.1.arhitekturnyj_atlas_dorevoljucijnogo_mariupolja.2.osobnjak_valtona
%%parent 04_01_2019.stz.news.ua.lb.1.arhitekturnyj_atlas_dorevoljucijnogo_mariupolja
 
%%url 
 
%%author_id 
%%date 
 
%%tags 
%%title 
 
%%endhead 

\subsubsection{Особняк британского вице-консула Вильяма Вальтона}

Не менее солидный двухэтажный особняк находится почти по соседству с домами
Трегубова. Его владельцем был подданный английской Королевы - вице-консул
Великобритании в Мариуполе Вильям Фомич Вальтон. Особняк вырос на улице
Георгиевской где-то в начале ХХ века, в его облике достаточно хорошо
просматривается влияние модного тогда модерна. Но насколько сильно внешний вид
дома изменился за прошедшие годы, сказать трудно. В сентябре 1943 отступающие
немецкие войска сожгли практически весь центр Мариуполя, в том числе и дом
Вальтона.

\ii{04_01_2019.stz.news.ua.lb.1.arhitekturnyj_atlas_dorevoljucijnogo_mariupolja.2.osobnjak_valtona.pic.1}

Восстановить его взялся трест \enquote{Ждановжилстрой} только в 1962 году. Об этом
свидетельствует табличка на фасаде, с формулировкой: здание \enquote{построено}. Видимо
разрушения были настолько сильными, что \enquote{восстановлением} эти работы назвать не
решились. С тех пор в покоях британского подданного функционирует
противотуберкулезный диспансер. Состояние особняка с натяжкой можно назвать
удовлетворительным. Само здание вот уже полвека не видело капитального ремонта,
а на задний двор лучше не заглядывать. Любопытных там ожидают развалины фонтана
и брошенные старые корпуса тубдиспансера. Включить особняк Вальтона в
туристический маршрут никто не решится, хотя он того стоит. Хороший повод
ходатайствовать перед Её Величеством о назначении в Мариуполь вице-консула.

\ii{04_01_2019.stz.news.ua.lb.1.arhitekturnyj_atlas_dorevoljucijnogo_mariupolja.2.osobnjak_valtona.pic.2}
\ii{04_01_2019.stz.news.ua.lb.1.arhitekturnyj_atlas_dorevoljucijnogo_mariupolja.2.osobnjak_valtona.pic.3}
\ii{04_01_2019.stz.news.ua.lb.1.arhitekturnyj_atlas_dorevoljucijnogo_mariupolja.2.osobnjak_valtona.pic.4}
