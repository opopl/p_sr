% vim: keymap=russian-jcukenwin
%%beginhead 
 
%%file 07_12_2021.stz.news.lnr.lug_info.1.vystavka_filatelia
%%parent 07_12_2021
 
%%url https://lug-info.com/news/pocta-lnr-priglasaet-na-filatelisticeskuu-vystavku-k-220-letiu-vladimira-dala
 
%%author_id 
%%date 
 
%%tags 
%%title "Почта ЛНР" приглашает на филателистическую выставку к 220-летию Владимира Даля
 
%%endhead 
\subsection{\enquote{Почта ЛНР} приглашает на филателистическую выставку к 220-летию Владимира Даля}
\label{sec:07_12_2021.stz.news.lnr.lug_info.1.vystavka_filatelia}

\Purl{https://lug-info.com/news/pocta-lnr-priglasaet-na-filatelisticeskuu-vystavku-k-220-letiu-vladimira-dala}

Государственное унитарное предприятие (ГУП) "Почта ЛНР" приглашает на
филателистическую выставку, посвященную 220-летию Владимира Даля. Об этом
сообщили на предприятии.

\ifcmt
  ig https://storage.lug-info.com/cache/9/6/a357eb9f-f9a9-43ec-bc4c-aef62840cf0d.jpg/w700h474
  @width 0.4
  %@wrap \parpic[r]
  @wrap \InsertBoxR{0}
\fi

"Посетители выставки познакомятся с уникальной частной коллекцией "Даль
Владимир Иванович" луганского филателиста Сергея Коваленко. Она включает в себя
почтовые блоки, марки, открытки и конверты, посвященные нашему выдающемуся
земляку, его творчеству, жизненному пути, современникам и эпохе", – говорится в
сообщении.

На предприятии уточнили, что на выставке представлена филателистическая
продукция СССР, России, Украины и Луганской Народной Республики.

"Особую историческую ценность представляет художественный маркированный
конверт, с памятным штемпелем первого дня, выпущенный к 175-летию Казака
Луганского", – отметили в ГУП "Почта ЛНР".

Выставка продолжит работу до конца декабря 2021 года.

Экспозиция расположена в отделении почтовой связи Луганск-1 по адресу: Луганск,
улица Почтовая, 22. Посетить выставку можно с понедельника по пятницу с 8:00 до
17:00, а также в субботу с 8:00 до 16:00.

Напомним, глава ЛНР Леонид Пасечник объявил 2021 год в Республике Годом
Владимира Даля.

Владимир Иванович Даль (1801 - 1872) – писатель, врач, лексикограф, создатель
Толкового словаря живого великорусского языка. Родился в поселке Луганский
Завод (ныне Луганск) 10 (22) ноября 1801 года, с детства был очень привязан к
родному краю, позже даже взял себе псевдоним Казак Луганский.

В 1819 году закончил Петербургский Морской кадетский корпус, служил на флоте,
затем изучал медицину в Дерптском университете (сейчас – Тартусский
университет). В 1828-1829 году принимал участие в русско-турецкой войне,
участвовал в сражениях, оперировал в условиях полевых госпиталей, отмечен
наградами. Работал ординатором в военно-сухопутном госпитале Петербурга.

Вскоре Даль всерьез занялся литературой. В 1832 году были опубликованы его
"Русские сказки. Пяток первый". Был дружен с известными писателями и поэтами:
Гоголем, Пушкиным, Крыловым, Жуковским. Вместе с Пушкиным путешествовал по
России. Даль лечил Пушкина после роковой дуэли, присутствовал при кончине
великого русского поэта.

Даль написал более ста очерков, в которых рассказывал о русской жизни. Он много
путешествовал, поэтому отлично знал русский быт. Также Даль составил учебники
"Ботаника", "Зоология", а в 1838 году стал членом Петербургской академии наук.
Но самой значительной и объемной работой остается "Толковый словарь",
содержащий примерно 200 тысяч слов.

С 1849 по 1859 год Даль проживал в Нижнем Новгороде, где служил управляющим
удельной конторой, после переехал в Москву. За это время напечатал множество
статей, работ. Первый том "Толкового словаря" вышел в 1861-м. А через год были
опубликованы "Пословицы русского народа". Деятельность Даля была отмечена
Ломоносовской премией. Часть лексикографического материала, который он собрал
для своего "Толкового словаря", была передана им Борису Гринченко, составителю
первого русско-украинского переводного словаря.

Скончался Даль 22 сентября 1872 года.
