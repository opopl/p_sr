% vim: keymap=russian-jcukenwin
%%beginhead 
 
%%file 13_11_2022.fb.fb_group.story_kiev_ua.1.svet
%%parent 13_11_2022
 
%%url https://www.facebook.com/groups/story.kiev.ua/posts/2065932296936894
 
%%author_id fb_group.story_kiev_ua
%%date 
 
%%tags 
%%title Вчера выключили свет на полтора часа раньше
 
%%endhead 
 
\subsection{Вчера выключили свет на полтора часа раньше}
\label{sec:13_11_2022.fb.fb_group.story_kiev_ua.1.svet}
 
\Purl{https://www.facebook.com/groups/story.kiev.ua/posts/2065932296936894}
\ifcmt
 author_begin
   author_id fb_group.story_kiev_ua
 author_end
\fi

Вчера выключили свет на полтора часа раньше.

Хаотичное отключение, такой себе график не по графику, выбивает из уже почти
привыкшего  @igg{fbicon.man.facepalming}  образа жизни за последний месяц. 

\ii{13_11_2022.fb.fb_group.story_kiev_ua.1.svet.pic.1}

\obeycr
Никак нельзя подгадать рассчитанные до минут дела. Твой собственный график неуспеваемости - это твоя зона ответственности.
Не рассчитывая на благосклонность электрических сетей, каждый день начинается с ритуала: кофе и просмотр новостей. Пока есть связь. Пока я тоненькой серебряной ниточкой на связи с миром. Серебряной, потому что в нашей квартире отключение света сопровождается тоненьким стеклянным звуком.
Брови - это мои эмоции. В зависимости от прочитанного, эти два черных крючка то взлетают на затылок, то обнимаются на переносице.
В последнее время они мирно сосуществуют каждая на своем месте. 
В последние дни новости хорошие.
Каждое утро - это обязательный звонок родителям, дочке, и, если повезет, брату, каждому в отдельности:
- Мам, ты как?
- Пап, ты как?
- Доць, вы как?
- Брат, ты как?
Если мама долго не берет трубку, мое сердцебиение учащается, в голове мысли одна впереди другой. За полминуты я могу написать в воображении сценарий самых ужасных событий, что позавидует сам Энтони Хопкинс.
Ну наконец! Услышала голос и в голове тараканы сменили гопак на вальс.
— Мам, чего ж ты так долго трубку не брала?, - удивляюсь я, прогоняя тахикардию. 
- Да телефон на кухне, а я пока дошла... 
Длинные минуты пасмурного дня перерастают в длинные часы ожидания электричества. 
В блокноте список неотложных дел на вечер в отведенные часы пишутся не в столбик, а через запятую. 
Связь то глушится, то совсем пропадает. 
..Иду с работы домой. 
Вечерний ритуал разговора с мамой. Идут длинные гудки. 
- А, ты знаешь, на каком бульваре мы сейчас живем?, - вместо обычного «привет» говорит мама.
Чувствую, снова учащается пульс, в голове тараканы подготовились к джайву.
- Мам? Ты хорошо себя чувствуешь?
Мама игнорируя мой вопрос: 
- На бульваре Жюля Верна, представляешь! Ой, у нас снова свет отключили, а у меня батарея садится, я тебе потом перезвонююююю....
Хм, Жюль Верн.. Откапываю из чердака памяти школьные произведения. Ну, конечно, «Дети капитана Гранта», «Пятнадцатилетний капитан», «Двадцать тысяч лье под водой»...
Наверное, лучшее название было бы Французский бульвар. Раз уж у нас один французский писатель сменяет другого, то пусть будет маленькая Франция в спальном районе Киева.
Как по команде отключается электричество во всем районе. Город, по крайней мере моя Оболонь, погружается во тьму. 
После яркого освещения - хоть глаз выколи. Проезжающие машины с мощными фарами ослепляют глаза. Останавливаюсь просто на дороге и пытаюсь привыкнуть. 
В голове пульсирует «Двадцать тысяч лье под водой». Под водой, во тьме...
Нет. Не возьмешь. Мы оттолкнемся от этого сумрачного дна и возродимся как птица Феникс. Как бы ты не бомбил, как бы ты ни пытался погрузить нас в беспросветную тьму. 
Мы не сдадимся. 
Никогда. 
Тому що ми сміливі нести світло, чуєш, виродок?
\restorecr

\ii{13_11_2022.fb.fb_group.story_kiev_ua.1.svet.orig}
\ii{13_11_2022.fb.fb_group.story_kiev_ua.1.svet.cmtx}
