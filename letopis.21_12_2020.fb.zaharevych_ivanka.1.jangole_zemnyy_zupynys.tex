% vim: keymap=russian-jcukenwin
%%beginhead 
 
%%file 21_12_2020.fb.zaharevych_ivanka.1.jangole_zemnyy_zupynys
%%parent 21_12_2020
 
%%url https://www.facebook.com/zakharevych/posts/3203402783098318
 
%%author Zakharevych, Ivanka
%%author_id zaharevych_ivanka
%%author_url 
 
%%tags skochyljas_igor
%%title Янголе земний, зупинись…
 
%%endhead 
 
\subsection{Янголе земний, зупинись…}
\label{sec:21_12_2020.fb.zaharevych_ivanka.1.jangole_zemnyy_zupynys}
\Purl{https://www.facebook.com/zakharevych/posts/3203402783098318}
\ifcmt
	author_begin
   author_id zaharevych_ivanka
	author_end
\fi

\index[names.rus]{Скочиляс, Ігор!декан Гуманітарного Факультету УКУ, 20.12.2020}

Янголе земний, зупинись…

Такі тексти завжди даються найтяжче. Коли сльози здушують горло, душа стоїть на
колінах, схиливши голову, а розум не може сприйняти факт про невимовну втрату.

Пан Ігор Скочиляс був Людиною науки, світла й посмішок. Вмів і любив жартувати,
любив писати і креслити мої тексти, а точніше майже їх переписувати. 

\ifcmt
tab_begin cols=4
	caption Ігор Скочиляс, декан Українського Католицького Університету, 20.12.2020

pic https://scontent-frt3-1.xx.fbcdn.net/v/t1.0-9/132245618_3203398099765453_6152579475953451964_o.jpg?_nc_cat=102&ccb=2&_nc_sid=730e14&_nc_ohc=IBO_Vn4hxfoAX_FgJMB&_nc_ht=scontent-frt3-1.xx&oh=427eaf39d2672e2b20c4e09f8cb9a47e&oe=600EEFED

pic https://scontent-frt3-1.xx.fbcdn.net/v/t1.0-9/132269130_3203397959765467_3912079683256944175_o.jpg?_nc_cat=104&ccb=2&_nc_sid=730e14&_nc_ohc=JL2s5V5mnwEAX9jqDEW&_nc_ht=scontent-frt3-1.xx&oh=d40db4061eddb741bd2e4511a9f05c77&oe=600F98A1

pic https://scontent-frx5-1.xx.fbcdn.net/v/t1.0-9/132037962_3203397919765471_7957438982651719188_o.jpg?_nc_cat=105&ccb=2&_nc_sid=730e14&_nc_ohc=kNrVjGo6XXwAX_47LV7&_nc_ht=scontent-frx5-1.xx&oh=b689f30b12303582543852e2f82c4fac&oe=601111EE

pic https://scontent-frt3-2.xx.fbcdn.net/v/t1.0-9/131952879_3203397933098803_8188288839701548694_o.jpg?_nc_cat=101&ccb=2&_nc_sid=730e14&_nc_ohc=WsFCAi1xkXkAX-VqKQG&_nc_ht=scontent-frt3-2.xx&oh=573ad5265405ecd88a5ae47ab612fc99&oe=600D932E

pic https://scontent-frt3-1.xx.fbcdn.net/v/t1.0-9/132252938_3203398179765445_4407654136970416248_o.jpg?_nc_cat=106&ccb=2&_nc_sid=730e14&_nc_ohc=M6a2BwzsmuYAX8rp2Q4&_nc_ht=scontent-frt3-1.xx&oh=e8250789422241ce45cd847b6aba636c&oe=600DAA03

pic https://scontent-frt3-1.xx.fbcdn.net/v/t1.0-9/132554686_3203398156432114_2654068712633232637_o.jpg?_nc_cat=107&ccb=2&_nc_sid=730e14&_nc_ohc=YvhYa9laRHUAX_e-3Lt&_nc_ht=scontent-frt3-1.xx&oh=6b3ccf8db829975e2b606ad099270ee9&oe=600EF028

pic https://scontent-frt3-1.xx.fbcdn.net/v/t1.0-9/131339418_3203398159765447_6841321882605610970_o.jpg?_nc_cat=108&ccb=2&_nc_sid=730e14&_nc_ohc=SwUUXB-0pdoAX_23lT1&_nc_ht=scontent-frt3-1.xx&oh=b2754d2e409ae57aaca7aca7264e0210&oe=600EAE22

pic https://scontent-frt3-1.xx.fbcdn.net/v/t1.0-9/132144900_3203398219765441_5851529001198902365_o.jpg?_nc_cat=106&ccb=2&_nc_sid=730e14&_nc_ohc=tj9i-sIwPTEAX83X44c&_nc_ht=scontent-frt3-1.xx&oh=29fe9942fc10f786c7abc51a952f959c&oe=600FDC2F
tab_end
\fi

Він став деканом Гуманітарного Факультету УКУ, коли ми вступили до
університету. 

Я була старостою курсу, і пан Ігор був моїм науковим наставником протягом усіх
років навчання. Він був моїм Добрим Вчителем, Усміхненим Приятелем, Вимогливим
Науковим Керівником. 

Він мало спав, і часто відписував мені о четвертій ранку, завжди хотів знати
мою думку стосовно різних питань. Він, декан гуманітарного факультету радився
зі мною, тоді ще студенткою першого курсу. Він мав до мене величезний кредит
довіри.

Він бачив у мені Людину. Його людяність, повага і невичерпна доброта просто
обеззброювали. 

Я ніколи не бачила пана Ігора злим, знервованим, агресивним. Попри свою
зайнятість, він завжди мав для мене час.  І говорили ми здебільшого, не тільки
про мою наукову роботу, а про життя. 

Він навчив мене читати і писати по-новому. Писати і переписувати. Не боятися
креслити. Він вболівав за кожен мій успіх, щиро тішився, коли бачив мій прогрес
у текстах, всіляко допомагав і сприяв. Завдяки ініціативі пана Ігора, я
побувала у різних архівах, і розпочала своє навчання тут, у Мюнхені. Коли я
вагалась до останнього їхати чи ні, пан Ігор завжди вірив у мене більше, ніж я
сама. Він засвідчував свою підтримку. Він молився за мене.

Він любив ділитися, розповідати смішні історії. Ігорку (Медвідь), пам’ятаєш
нашу улюблену, яку пан Ігор часто любив розказувати?  Про те, як німці прийшли
забирати в діда корову, а дідусь вийшов на подвір’я, витягнув вперед праву руку
і сказав воякам: «Мінуточку!». Я досі чую у вухах сміх пана Ігора, і те як він
з витягнутою рукою показував того дідуся. Або якось, коли після не дуже
успішного захисту курсової роботи одного із студентів, пана Ігора попросили
зробити коментар. Пригадую, він тоді зітхнув, посміхнувся, встав, лагідно
сказав: «Всякоє диханіє хай хвалить Господа», і спокійно мовчки сів на своє
місце.

Пан Ігор реалізував так багато академічних проектів, написав так багато книг,
але моєю улюбленою завжди залишиться ініціатива «Гуманітарних бесід». Саме там,
я вперше почула про динаміку трьох «С» від Владики Бориса Ґудзяка, і відчула
комунікацію й дух спільноти гуманітарного факультету.

Пан Ігор любив свого кота, архіви, свою роботу, людей, а найбільше – студентів.
Він стояв горою за кожного з нас. Він хотів знати, що відбувається у нашому
житті, і в УКУ і поза університетом. Був особливо уважним до нашого, тоді
першого курсу. Це був найкращий декан, про якого можна було тільки мріяти. Він
вчився разом з нами, обідав за одним столом, писав нам рекомендаційні листи,
їздив у історичні експедиції та поїздки. Він переживав, коли ми отримували
талони на іспитах, і коли забували приходити на академічну літургію в середу до
університетської каплиці, що на Свенціцького, він з розумінням продовжував нам
дедлайни і закривав очі, коли ми не завжди встигали здавати роботу вчасно. Він
дбав, про наші розуми й наші душі. Він випромінював добро і людяність. 

Він був найповноціннішим живим втіленням заповіту Блаженнішого Любомира Гузара
«Бути Людиною». Він, насправді був Людиною.

Він завжди обіймав при зустрічі, і при прощанні, був лагідним і вимогливим,
добрим і відданим своїй Церкві, своїй родині, своїй роботі, а найбільше, нам,
своїм студентам. Він варив з нами куліш у селі Лавочне, реставрував могили на
цвинтарі у Ґорайці, стояв у холодні ночі на Майдані. Його завжди було багато,
він підтримував і надихав нас, він йшов і дивився з нами в одному напрямку. Він
не працював в УКУ, а служив цій спільноті.

Якось пан Ігор, прийшов до нашої 318 аудиторії, і розповів історію про одну
дивовижну  зустріч, яка за його словами, дуже вплинула на його життя. Якось
вранці, він як завжди у швидкому темпі ішов на роботу, був дуже заклопотаним, і
не звертав уваги, на людей, що проходили повз нього, бо хотів якнайшвидше
добратись до УКУ, щоб розпочати робочий день. Якась бабуся, намагалась перейти
дорогу, і шукала когось, хто б міг їй допомогти. У ранкову пору, кожен, як і
пан Ігор, поспішав на роботу чи навчання, чи мав купу інших справ, і зовсім не
мав часу, аби перевести бабцю через дорогу. Пан Ігор, також, спочатку її не
помітив, і у швидкому темпі, пройшов повз неї. Тоді він почув, як хтось сильним
голосом кликав : « Ангеле земний, зупинись!» Пан Ігор сказав, що не звернув на
це уваги, і мовчки, не оглядаючись, йшов далі. Тоді голос зробився сильнішим і
хтось знову промовив йому в спину : «Ангеле земний, зупинись!». Тоді пан Ігор
спантеличено повернувся, і побачив бабусю, яка втретє сказала : «Ангеле земний,
зупинись!». Пан Ігор підійшов до бабусі, і запитав, чи вона звертається саме до
нього. Бабця нічого не відповіла, лишень мовчки показала палицею на пішохідний
перехід. Пан Ігор, допоміг перейти бабусі дорогу, і у спокійному темпі пішов на
роботу. Він сказав нам, що після того випадку, він завжди старався буди уважним
і знаходити час для кожного, кого він зустрічав на своєму шляху.

Згадуючи цю історію, хочеться лишень розпачливо крикнути у  простір цього
світу, що вмить осиротів , коли ми усі дізнались про втрату нашого Мудрого
Вчителя, Дорогого Приятеля, Доброго Християнина : «Пане Ігоре, Ангеле земний,
зупинись!...» 

Від себе особисто, і від імені нашої групи усіх студентів-істориків випуску
2016 року УКУ, складаємо найглибші співчуття дружині Ірині, дочці Соломії та
всій спільноті УКУ. Дякуємо Богові за дар зустрічі з паном Ігорем у житті
кожного з нас, огортаємо його у своїх молитвах. Світла і вічна пам'ять.

20.12.2020. A.D.

Львів-Мюнхен

Щиро,

Іванка Захаревич

Фото з особистого архіву та архіву УКУ.
