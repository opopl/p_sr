% vim: keymap=russian-jcukenwin
%%beginhead 
 
%%file 24_08_2020.fb.fb_group.story_kiev_ua.1.boris_paton_kievljanin.cmt
%%parent 24_08_2020.fb.fb_group.story_kiev_ua.1.boris_paton_kievljanin
 
%%url 
 
%%author_id 
%%date 
 
%%tags 
%%title 
 
%%endhead 
\zzSecCmt

\begin{itemize} % {
\iusr{Irina Benderovska}
Серёжа, как написал ! @igg{fbicon.hands.applause.yellow}{repeat=3} 

\iusr{Полина Любченко}
Великое государство определяют великие люди и их великие дела,

Но, благодаря вятровичам нет никакой инновационной украины и быть не может, тк
нет никакого развития реального, сплошные лозунги и агитки на фоне тотального
регресса и деградации

\begin{itemize} % {
\iusr{Людмила Човнюк}
Якось вас жаль...

\iusr{Полина Любченко}
\textbf{Людмила Човнюк} а вас не жаль, так вам и надо

\iusr{Maryna Chemerys}
\textbf{Полина Любченко} Що \enquote{надо}?

\iusr{Аркадий Шухман}

Мост назван в честь Евгения Патона, отца Бориса Евгеньевича. А еще - содержать
Академию Наук могут себе позволить только богатые страны. Я так думаю...

\end{itemize} % }

\iusr{Сергій Боровик}

Дурак ты, Басистый. Когда-то А Вознесенский написал:\enquote{ ...не надо улиц
переименовывать, постройте новые и назовите}. Но Витя кроме инструкций из мокши
ни хрена не читает. Жаль.


\iusr{Полина Любченко}
\textbf{Сергій Боровик} если бы хоть что-нибудь читал-может, больше толку было бы

\iusr{Андрій Чмир}
Лист 40ка академіків проти Сахарова він всеж підписав.

\begin{itemize} % {
\iusr{Полина Любченко}
\textbf{Андрій Чмир} не ошибается тот, кто ничего не делает, да и заслуги и достоинства перевешивают

\begin{itemize} % {
\iusr{Андрій Чмир}
\textbf{Полина Любченко} а якщо вам таке лікар скаже, коли ви дізнаєтеся, що він діагнози спутав?

\iusr{Андрій Чмир}
\textbf{Полина Любченко} це називається \enquote{кон'юнктурщина}.

\iusr{Полина Любченко}
\textbf{Андрій Чмир} ЛЮБОЙ ИМЕЕТ ПРАВО НА ОШИБКУ

\iusr{Андрій Чмир}
\textbf{Полина Любченко} бажаю отримати свій бумеранг

\iusr{Полина Любченко}
\textbf{Андрій Чмир} уууу, угроза?!
Уже боюсь

\iusr{Андрій Чмир}
\textbf{Полина Любченко} ви все прекрасно зрозуміли, не клейте дурепу.
\end{itemize} % }

\iusr{Сергей Пацкин}

И сидел чугуниевой жопой до последнего, вечный функционер и конформист. Ученый
посредственный.

\end{itemize} % }

\iusr{Igor Kapiy}

Знаходжусь в глибокому смутку, бо велика людина - Борис Євгенович Патон,
відійшла в кращій світ.

Та не треба приплетати сюди шановного історика Володимира Михайловича
В'ятровича. Кожен має свою місію на цей планеті, та й в нашій державі.

Дивіться краще на себе, і не заважайте будувати нашу країну, після
багатолітньої епохи ментальної деградації.

\begin{itemize} % {
\iusr{Сергей Удовик}
\textbf{Igor Kapiy} Вятровича как историка на западе не признают, он не отвечает западной методологии, это агитатор в стиле необолшевизма

\begin{itemize} % {
\iusr{Igor Kapiy}
\textbf{Сергей Удовик} Ну такое...
Причем здесь признание на Западе. Ведь мы идём у нас в Украине своим путём. Да, иногда тернистым, так и Рим по по легенде не сразу строился.
Важно @igg{fbicon.backhand.index.pointing.up}  что это наш путь.
И мне, и моему огромному окружению, людей разного возраста и так называемых \enquote{социальных сословий} это очень нравиться.
Как минимум переименование улиц очищает наш родной Киев от темных энергий тех людей, которые участвовали в приведении к общему минусовому знаменателю Украинской идентичности.

\iusr{Сергей Удовик}
\textbf{Igor Kapiy} по части переименований я только за, часть, как Кудри, спорна, а вот на кого переименовывают, большой вопрос. Поэтому в Киеве есть улицы, которые по 5-6 раз переименовывали))

\iusr{Igor Kapiy}
\textbf{Сергей Удовик} Про ул. Ивана Кудри, согласен, жил рядом 4 года.

\iusr{Maryna Chemerys}
\textbf{Сергей Удовик} Так і до революції постійно перейменовували, і в центпі, і на околицях, і ніхто не ридав

\iusr{Сергей Удовик}
\textbf{Maryna Chemerys} и этих переименуют нечего печалится
\end{itemize} % }

\iusr{Полина Любченко}
\textbf{Igor Kapiy} плохим танцорам ноги мешают, а историческая правда и
надуманная история - очень разные вещи

\iusr{Igor Kapiy}
\textbf{Полина Любченко} 

Причем здесь надуманная история, вот у меня, человека имеющего доступы в бывшие
архивы КГБ, не напрямую, пусть через посредника, как раз всплывает история,
какая была в реальности, а не придуманная, и надуманная. Ежедневно читаю
определенное количество документов, раннее бывших под грифом \enquote{секретно}


\iusr{Игорь Марьяш}
Вятрович не историк, Патон служил любой власти и угробил науку

\end{itemize} % }

\iusr{Олег Мороз}
Аренда превыше всего

\iusr{Наталия Трач}

году в 1992-93 в Киеве, в здании КИМО на Мельникова заседал Межгосударственный
координационный совет (МКC) СНГ по науке собрались главы НАН со всего СНГ,
тогда это было модно из всего запомнила только слова Патона о том, что сейчас
финансирование системы НАНУ со всеми академическими институтами едва покрывает
расходы на зарплату, что (практически прямое цитирование) превращает НАНУ в
собес...  я не уверена, что в нашей группе стоило выступать с таким постом,
уверена, что он приведет к многодневной и очень жесткой \enquote{дискуссии} я хотела бы
о другом дорогие мои ученые-киевляне, не могли бы вы по возможности объективно
рассказать о состоянии в НАНУ и академических институтах за последние 20-25 лет

PS. достижения Евгения Оскаровича и Бориса Евгеньевича в развитии
электросварки, мировой уровень их трудов, их вклад в мировую науку несомненны,
неоспоримы и являются гордостью украинской науки

этот вопрос не обсуждается ну и чтоб два раза не писать: на мой взгляд Вятрович
ваще тут никаким боком

\begin{itemize} % {
\iusr{Полина Любченко}
\textbf{Наталия Трач} это образно, имя нарицательное, хорошо, что не при чем

\begin{itemize} % {
\iusr{Наталия Трач}
\textbf{Полина Любченко} и как нарицательное тягать всуе не стоит

\iusr{Полина Любченко}
\textbf{Наталия Трач} не создавай себе кумира!

\iusr{Наталия Трач}
\textbf{Полина Любченко} и никого не втаптывай в грязь )

\iusr{Наталия Трач}
ребяты, а пост не о том
такшо не отвлекайтесь, о Патоне, так о Патоне
\end{itemize} % }

\iusr{Полина Любченко}
Тот, кто перевирает историю, отлично знает, что и зачем делает

\begin{itemize} % {
\iusr{Igor Kapiy}
\textbf{Полина Любченко} Как переврать ложь?

\iusr{Полина Любченко}
\textbf{Igor Kapiy} очередной, еще более кощунственной и нелепой ложью

\iusr{Igor Kapiy}
\textbf{Полина Любченко} 

Историю переписывать смысла нет, ведь думающие люди понимают, что \enquote{История не
имеет сослагательного наклонения}!

Если бы да кабы... В реальности мы находим в архивах много крайне чудовищных
манипуляций ранее. Сейчас? Да может кто-нибудь играется в альтернативное
развитие истории как сказочники.

Но все же, я счастлив жить в эти годы, когда то что можно действительно узнать
правдивое о прошлом столетии...

Вот такого рода документы читаю регулярно: Их наверное миллионы страниц, ещё не
всё доступны, отсканированы, но ужас то....

\ifcmt
  ig https://scontent-frt3-1.xx.fbcdn.net/v/t1.6435-9/118401498_3225967834147056_7256441273550561242_n.jpg?_nc_cat=108&ccb=1-5&_nc_sid=dbeb18&_nc_ohc=YIswx-3khYgAX94xAeX&_nc_ht=scontent-frt3-1.xx&oh=00_AT-39jpnRTTaCABJhwbZ4u7i1ynWPszIMeZOovS4k0wqTw&oe=6200025B
  @width 0.2
\fi

\iusr{Виктория Довгая}
\textbf{Igor Kapiy} и где ужас @igg{fbicon.beaming.face.smiling.eyes}  в этом документе то

\iusr{Наталия Трач}
\textbf{Виктория Довгая} та действительно
работали люди

\iusr{Igor Kapiy}
\textbf{Виктория Довгая} просто пример.

\iusr{Maryna Chemerys}
\textbf{Полина Любченко} Підіть до архіву і почитайте документи, може, очі відкриються, і не будете плутати правду з брехнею

\iusr{Ольга Кирьянцева}
\textbf{Maryna Chemerys} зачем тратить жизнь на \enquote{раскопки} прошлого? Жить надо Сегодня, думая о Будущем. То, что Было, уже ушло безвозвратно...

\iusr{Виктория Довгая}
\textbf{Igor Kapiy} странный пример в контексте вашего диалога выше

\end{itemize} % }

\iusr{Наташа Гресько}
\textbf{Наталия Трач} ага. и интересно, что будет с недвижимостью? с той же Наукой (которую они с Фокиным еще в перестройку примутили)? что было хорошего в сове - академическую науку хоть как-то финансировали.

\iusr{Наталия Трач}
\textbf{Наташа Гресько} ну, собственно, поэтому и прошу компетентніх комментариев
тока шота их нету (

\iusr{Наташа Гресько}
\textbf{Наталия Трач} может, тут нет никого из академии? узнаю все свежие сплетни - отпишусь) но - чисто визуально - в Академе новострои уже наступают на институты, и на спорткомплекс.

\end{itemize} % }

\iusr{Владимир Новицкий}

\ifcmt
  ig https://scontent-frt3-1.xx.fbcdn.net/v/t39.1997-6/p240x240/15945720_178517522626602_2398857140258209792_n.png?_nc_cat=106&ccb=1-5&_nc_sid=0572db&_nc_ohc=8GBK2ND2ERkAX-h1sDC&_nc_ht=scontent-frt3-1.xx&oh=00_AT8nmLWygONSCdt3uHpCKJ_uuT3oF2elTN3tRM1sa656mQ&oe=61DF6960
  @width 0.2
\fi

\iusr{Игорь Марьяш}
Как тебе такое, Элон Маск?

\begin{itemize} % {
\iusr{Наталия Трач}
та абзац, подумал Маск )

\iusr{Игорь Марьяш}
\textbf{Наталия Трач} сварку на марс

\iusr{Сергей Удовик}
\textbf{Игорь Марьяш} пока будут славить вятровичей, масков не будет, а в пику маскам будет самый большой в мире флаг из соломы на 54 метра, пусть Маск обзавидуется

\ifcmt
  ig https://scontent-frx5-1.xx.fbcdn.net/v/t1.6435-9/118297078_1233759566958257_7029924332198508459_n.jpg?_nc_cat=110&ccb=1-5&_nc_sid=dbeb18&_nc_ohc=XgVMRuMk9rYAX_2toZ7&_nc_ht=scontent-frx5-1.xx&oh=00_AT-nwOlPwnXzN2EgdHKvwG5K5IiQ--xEXW9O1RXQFhAnXg&oe=620299A4
  @width 0.3
\fi

\begin{itemize} % {
\iusr{Игорь Марьяш}
\textbf{Сергей Удовик} Вятрович уже не при должности, но украинской науки не стало намного раньше

\iusr{Сергей Удовик}
\textbf{Игорь Марьяш} во-первых, я использовал как имя нарицательное, во вторых, рулят там его последователи, которые стремятся его превзойти

\iusr{Игорь Марьяш}
\textbf{Сергей Удовик} да вроде поутихли.

\iusr{Сергей Удовик}
\textbf{Игорь Марьяш} да нет, разрабатывают, как с Родины Матери герб убрать, мост Патона декомунизировать да и тачанку в Каховке
\end{itemize} % }

\end{itemize} % }

\iusr{Светлана Комина}

Ця людина потрібна була людству. В різних сферах науки, медицині його
винаходи. Працював до останніх днів. Може тому, Господь дав йому довге життя, щоб
допомагати людям. Імя Бориса Патона уже увійшло в історію, назван міст його
ім"ям, орден за вклад у науку імені Бориса Патона, може і вулицю
назвуть. Шануємо і пам"ятаємо.

\begin{itemize} % {
\iusr{Наталия Трач}
ну, міст, все ж таки, ім'ям батька, світлої пам'яті Євгена Оскаровича

\iusr{Светлана Комина}
Його батько не встиг завершити міст, бо помер, то завершував будівництво Борис Патон.

\iusr{Наталия Трач}
\textbf{Светлана Комина} 

його батько не дожив кілька місяців до відкриття мосту, працювали вони звісно
разом, а міст таки імені Євгена Оскаровича

а завершувати там вже було хіба що асфальт покласти та рейки може трамвайні

\iusr{Светлана Комина}

Ну нехай буде так, всерівно Патон багато чого зробив. Вам аби посперечатися.

\iusr{Виталий Войцеховский}
\textbf{Светлана Комина} 

да, приходится постоянно спорить, что бы сквозь невежество людям знания
донести. А вы и дальше соглашайтесь со всем подряд

\iusr{Maryna Chemerys}
\textbf{Светлана Комина} Йому пам'ятник вже давно стоїть на вулиці Богдана Хмельницького
\end{itemize} % }

\iusr{Zosya Rozanova}
\textbf{Светлана Комина}, мост назван в честь его отца, Евгения Оскаровича Патона!

\iusr{Константин Апыхтин}

Украина - была одной из самых мощных, самых развитых в научном и
технологическом плане республик СССР. И наука, и медицина, - были у нас одними
из лучших в Союзе. Есть чем гордиться. А не разрухой, репрессиями и руиной, как
это стало быть принятым вбивать в голову молодым людям. Без Патона, без его
Института - не было бы корпусов межконтинентальных баллистических ракет и
топливных баков к ним. И, неизвестно чем могло бы закончиться
военно-политическое противостояние с США. Скорее всего, ядерной бомбардировкой
СССР, планы которой строились начиная с 1945 года. А так, ответный ход, военный
паритет - мощные ракеты SS-19, SS-20, и ни один враг не пикнет, ни в Европе, ни
в Азии, ни в Америке! Вот страна была!

\begin{itemize} % {
\iusr{Igor Kapiy}
\textbf{Константин Апыхтин} Зачем Вам противостояние? Боитесь кого-то?

\begin{itemize} % {
\iusr{Константин Апыхтин}
\textbf{Igor Kapiy} , противостояние было, холодная война, не слышали такого, не помните? А я помню.

\iusr{Igor Kapiy}
\textbf{Константин Апыхтин} 

История был мой любимый предмет в школе, вместе с некоторыми другими. Но
вероятно Слава Богам, что Зира Леонтиевна, божий одуванчик , примерно 1920 года
рождения, закрываясь в классе, нам рассказывала, отвлекаясь от школьной
программы, историю казачества, УНР и УПА, за что ей великий поклон. А по теме,
то время холодной войны характеризовалось Западным миром и Восточным,
совместными страхами, оттуда и гонка вооружений.

\iusr{Константин Апыхтин}
\textbf{Igor Kapiy} , социальная справедливость была. Экономическая стабильность. Жильё строили для простых людей, и давали его бесплатно. Образование, медицина - бесплатными были.

\iusr{Константин Апыхтин}
\textbf{Igor Kapiy} , история распорядилась иначе, ни казачество, ни УНР, ни УПА - не смогли заручиться широкой народной поддержкой и долго быть при власти. Потому что у них не было программы защиты трудового населения, небогатых людей. И только советская власть смогла построить средний класс. Ну, и да, в результате революции, как всегда это бывает, к власти пришли негодяи. Но это уже совершенно другая история...

\iusr{Igor Kapiy}
\textbf{Константин Апыхтин} Как киевлянин в 3ем поколении, частично согласиться могу, бабушка, давно покойная так и получила хрущевку в своё время, но сколько труда было потрачено (участник войны, герой труда). Сейчас, если брать её возраст на то время, проще при средней зарплате взять ипотеку, и так не напрягаться.
За сим откланяюсь. Всем доброй ночи.

\iusr{Vadym Kozhushko}
\textbf{Konstantin Apykhtin} лучше бы деньги нормальные платили, а люди сами бы решали, что им с ними делать. А вот это всё : « этому дала, этому не дала, этому три раза дала.» И кто решал кому что давать? Неэффективная экономическая модель, поэтому и накрылась медным тазом.

\iusr{Pavlo Bondarenko}
\textbf{Константин Апыхтин},

вот не надо социальной справедливости, квартиры бесплатно, ха ха, платили
копейки, и ща счет жтого строили бесплатные квартиры
\end{itemize} % }

\iusr{Андрій Чмир}
Тунельний папір з'явився після ракет.

\end{itemize} % }

\end{itemize} % }
