% vim: keymap=russian-jcukenwin
%%beginhead 
 
%%file 12_05_2021.kudrjac_evgenij.1.boris_barskij_smeh_lechit
%%parent 12_05_2021
 
%%url https://kudryats2019.wordpress.com/2021/05/12/борис-барский-всё-таки-смех-лечит
 
%%author_id kudrjac_evgenij
%%date 
 
%%tags 
%%title Борис Барский: «Всё-таки смех лечит!»
 
%%endhead 

\subsection{Борис Барский: «Всё-таки смех лечит!»}
\label{sec:12_05_2021.kudrjac_evgenij.1.boris_barskij_smeh_lechit}

\Purl{https://kudryats2019.wordpress.com/2021/05/12/борис-барский-всё-таки-смех-лечит}

\ifcmt
 author_begin
   author_id kudrjac_evgenij
 author_end
\fi

\ifcmt
  ig https://kudryats2019.files.wordpress.com/2021/05/d091d0b0d180d181d0bad0b8d0b9.png?w=1278&h=768&crop=1
  @width 0.4
  @wrap \parpic[r]
\fi

\begingroup
\bfseries

Читатели среднего поколения и старше должны хорошо помнить одесскую
комик-труппу «Маски-шоу» во главе с Георгием Делиевым. Один из ярких персонажей
этого коллектива – Борис Барский, который запомнился по роли дедушки.

Он – поистине народный артист, и это высокое звание он заслужил задолго до
того, как оно было присвоено этому замечательному человеку.  Кроме того, Борис
– поэт и драматург. Наш корреспондент побеседовал с легендарным «дедушкой»,
чтобы получше узнать основные вехи его биографии.
\endgroup

\textSelect{- Борис, я прочитал, что Вы в детстве отнюдь не были «белым и пушистым», однако
Вы не создаёте впечатления хулигана и двоечника... За что Вас ругали, – за
какие-то неблаговидные поступки?}

— Я любил экспериментировать и легко поддавался дурному влиянию улицы. А все
мои друзья были хулиганами, поэтому, когда они мне рассказывали о каком-либо
научном эксперименте, я тут же его воплощал в действие. Например, я узнал, что
если мокрой промокашкой обернуть цоколь лампочки и вкрутить лампочку в патрон,
она некоторое время будет давать свет. Затем промокашка высыхает и свет гаснет.
Вот я не поленился и прошелся по всем лампочкам в классе. Приходит учитель,
спрашивает: «Почему вы сидите в темноте?» Включает свет, и свет, таки горит. Но
не долго. А дальше электрик возится с розетками, включателями и выключателями и
ничего не понимает. Или ещё мне рассказали: если циркулем легонько шлёпнуть по
лампочке, то в ней появится дырочка, из которой постепенно улетучится весь газ,
при этом лампочка цела, но, опять же — не горит.

А еще на школьных каникулах родители меня отправляли в пионерские лагеря. В
одном из них у меня даже появилось прозвище «Помазанник Божий» Зубной пастой
ночью я перемазал весь наш отряд. Если по правде – всех девчонок. Потом паста
кончилась, но под руку попался флакончик с зелёнкой. Я нагрёл её подмышкой до
температуры человеческого тела. Потом, уже на улице увидел, что окно у
директора лагеря открыто. И спит он прямо под окном. Ну, я ему легонечко
помазал лысину зелёнкой. А, в те годы фильм был популярный – «Фантомас» На
следующее утро была линейка, и директору пришлось надеть шляпу, – он очень
долго оттирал голову, но так до конца не оттёр! Но меня никто из друзей не
выдал, а сам я тоже не признался!

\textSelect{— Но в более зрелом возрасте вам это всё пригодилось! Вы окончили факультет
атомной энергетики Одесского политехнического института, после чего три года
работали в Центре стандартизации и метрологии. Честно говоря, я плохо
представляю Вас в этой роли. Как Вы из атомщика-энергетика стали
артистом-комиком?}

— На первом курсе института я пошёл в студию пантомимы, потому что там было
много стройных девушек и репетировали они все в лосинах. Это такие непрозрачные
колготки. А для молодого человека это зрелище было очень увлекательным! Как
известно, девушки – двигатель прогресса, и тогда в группе занималось 13-14
девочек и 2-3 пацана, поэтому я там был, как розовый пони на радужной лужайке.
Короче, увлекся я пантомимой сильно. Да и после института старался не
пропускать ни одной репетиции. Кстати, тогда меня заметил и пригласил в свой
коллектив Александр Жеромский (Александр Жеромский – известный советский мим и
режиссёр. Был руководителем ансамбля пантомимы, который затем трансформировался
в «Театр эксцентрики». Прим. автора)

В то время мы много и плотно общались со всеми мимами Советского Союза: Игорь
Малахов ездил к Мацкявичусу (Гедрюс Мацкявичус – литовский театральный
режиссёр. Основоположник направления сценического искусства – пластической
драмы. Создатель «Театра пластической драмы». Прим. автора) Георгий Делиев
окончил Лицедей- лицей в Питере, тогда еще Ленинграде, у Славы Полунина. А
меня, как я уже сказал, пригласит Жеромский. Я приехал в Москву. В Россконцерте
меня отсмотрели и утвердили актером оригинального жанра с окладом в 80 рублей и
местом в общежитии. Но я-то, как в анекдоте, рассчитывал на трёхкомнатную
квартиру на Тверской и зарплату 400 рублей.

\textSelect{— Напомните его, пожалуйста, для наших читателей..}

— Старый, хорошо потрёпанный жизнью, столичный актёр решил переехать в
провинцию, чтобы служить в местном театре. Он выдвинул требования: 5000 рублей
жалования, апартаменты и извозчика в театр и обратно. Вместо этого ему
предложили 1000 рублей, «клоповник» и чтобы он ходил в театр и обратно пешком.
«Согласен!», – вальяжно сказал актёр.

А мне тут предлагают – всего 80 рублей! Короче, как говорил герой одного
популярного фильма: «Для Атоса это было слишком много, а для графа де Ляфер –
очень мало!». Да у меня, на тот момент, было уже двое детей, поэтому я, в
отличие от актёра из анекдота, отказался.

Но в 1984 году у нашей одесской «Фабрики звёзд» в лице Валентины Михайловны
Прокопенко – методиста – организатора культурно-просветительской работы высшей
категории, театрального режиссёра, на тот момент возглавлявшей концертный отдел
Одесской областной филармонии, появилась идея создать коллектив мимов. Вот с
этого всё и началось: мы пришли в филармонию. Вначале должны были работать с
музыкальным ансамблем «Скифы», но буквально через две недели ансамбль
разругался и разбежался. А через месяц мы уже сделали свою концертную
программу, которую показали в Доме актёра. Директором филармонии тогда был
Анатолий Александрович Зубовский, он посмотрел и спросил: А зачем вам ансамбль?
Вы можете и сами, так сказать без ансамбля... Кстати, когда мы были на гастролях
в Израиле, по-моему, в Ашдоде, он приходил к нам на концерт. Тогда, на том
концерте мы увидели его в зале, пригласили на сцену и объяснили зрителям, что
он – наш «крёстный отец» в искусстве. Зал аплодировал стоя ...

\textSelect{- У Вас есть и режиссёрское образование: в 1992 году Вы окончили ГИТИС по
специальности режиссура эстрады и массовых представлений. Что оно Вам дало,
ведь Вы там учились в довольно зрелом возрасте?}

— Говорят, что умный всю жизнь учится, дурак – учит. Мы вместе с Делиевым
учились в мастерской Андрея Николаевича Николаева – великого клоуна Андрюши,
профессора ГИТИСа, у которого в своё время училась Алла Пугачёва. А вместе с
нами занимался Саша Буйнов. Это была хорошая школа. На первых экзаменах мы
сдавали пантомиму и клоунаду, как актёрское мастерство и сразу же получили по
пятёрке, потому что мы – талантливые. На следующей сессии Андрей Николаевич нам
сказал: «То, что вы умеете делать, от вас никуда не уйдет. Теперь вы будете
опереточными артистами!». Дальше пошло и поехало. Вот мы уже солисты оперы,
потом драматические артисты. Каждую сессию он нам открывал всё новые и новые
«горизонты». Всё это помогло нам раздвинуть рамки. Отсюда и разноплановость
нашего творчества. На сегодняшний день в нашем репертуаре порядка десяти
авторских спектаклей. Абсолютно разные: от абсурдистской и психоделической
комедии «Ночная симфония» – до моих поэтических комедий и буффонадных
спектаклей «Ромео и Джульетта», «Отелло», «Орфей и Эвридика», «Моцарт и
Сальери»...

\ifcmt
  ig https://kudryats2019.files.wordpress.com/2021/05/d091d0bed180d0b8d181d091d0b0d180d181d0bad0b8d0b9.png
  @width 0.4
  %@wrap \parpic[r]
  @wrap \InsertBoxR{0}
\fi

\textSelect{— Как же Вы решили «замахнуться на Вильяма нашего – Шекспира»? Это,
согласитесь, – смелый шаг!}

— Согласитесь и вы, что «Ромео и Джульетту» написал не Уильям Шекспир, т.к.
этот сюжет был известен ещё за 200 лет до него. У меня очень большая
библиотека, я всегда увлекался книгами. Помните, как в советские времена мы
собирали макулатуру и получали за это хорошие книжки. Во время гастролей, куда
бы мы ни приехали – первым делом бежали в книжные магазины. Перестройка,
Гласность и для нас открылись такие имена, как Андрей Платонов и Леонид
Андреев, стихи ОБЭРИУТИТов. Каждые гастроли приносили 10-12 книг. А у нас
вечные перевесы в самолётах, но не из-за реквизита, из-за книг.

И вот, глядя на эти аккуратно собранные на полках книги, я подумал: «Почему
\enquote{Ромео и Джульетта} – трагедия? Ведь любовь убить невозможно и нереально: она
всегда существовала за тысячелетия до нас. И существовать она будет вечно!». Я
решил сделать из этого комедию. Если мой вариант «Ромео и Джульетты» проживёт
после меня ещё лет 400-500, как у Шекспира, то значит и я – гений!

\textSelect{-Кроме этого, Вы ещё являетесь и автором восьми поэтических сборников. Вы
начали писать стихи в юношеском возрасте? Как вообще Вы стали поэтом?}

— Я им не становился. Почему-то все мои друзья, клоуны говорят мне, что я –
поэт. Я отшучиваюсь, говорю им, что я – не поэт, я – клоун. Они тоже шутят: да,
какой из тебя клоун? Ты –  поэт! Зато поэтов переубеждать не надо. Все они
хором твердят, что я никакой не поэт, что я –  клоун. Наверное, это – хобби.
Первые рифмы родились во время первой влюбленности. Девочки в школе всегда
делали подарки мальчикам к 23-му февраля. Я мечтал, что в тот день, девочка,
которая мне очень – очень нравилась, подарит мне пистолет. А она подарила мне
маленький макет крейсера «Аврора», который на фиг мне не был нужен! Мне стало
так обидно. Как она могла так подло со мной поступить? Тогда я написал какое-то
ироничное стихотворение. С этого всё и пошло...

\textSelect{-Как Вы думаете, сейчас стало сложнее смешить публику, чем, например, 30 лет
назад? Юмор как-то изменился за это время?}

— Темы, на которые мы шутим, – вечные, мы не пытаемся шутить на темы политики,
потому что это – сиюминутно: политик пришёл, затем – ушёл, и уже через полгода
о нём никто и не вспомнит! На вечные темы шутить несложно, а сейчас нам стало
даже немного легче, ведь при всём негативе, который сегодня льется с экрана
телевизора и происходит в стране, люди вдруг потянулись в театр. Самый большой
комплимент для актёра, когда к тебе сзади подходит зритель и незаметно
дотрагивается. Определяя, таким образом, живой ты или инопланетянин, какой-то!
(Смеётся). А сейчас стало много подходить людей – беженцев или тех, кто воевал.
Они говорят: «Большое спасибо! За последние полтора-два месяца мы впервые на
вашем спектакле улыбнулись!». И это – чудо, которое происходит! Наверно,
всё-таки смех лечит...

\textSelect{— Большое спасибо, Борис, за этот интересный разговор!}

«Немецко-русский курьер», февраль-март 2021 года

Беседовал Евгений Кудряц
