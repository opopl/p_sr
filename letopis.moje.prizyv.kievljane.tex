% vim: keymap=russian-jcukenwin
%%beginhead 
 
%%file moje.prizyv.kievljane
%%parent moje.prizyv
 
%%url 
 
%%author_id 
%%date 
 
%%tags 
%%title 
 
%%endhead 

\subsubsection{Киевляне}
\label{sec:moje.prizyv.kievljane}

Позвольте представиться. Мы - Киевляне, жители тысячелетнего города Киева,
Матери Городам Русским, великого Города, имеющего фундаментальное,
цивилизационное значение для Украины, России и Беларуси, в котором находятся
наши тысячелетние святыни - Киево-Печерская Лавра с пещерами и церквями, а
также музеем украинского народного исскуства, где выставлены картины Катерины
Билокур, потом, София Киевская с Орантой и графити, Выдубицкий монастырь...
Кириловская церковь... и мы сейчас сидим взаперти в одной из киевских
квартир... в одном из районов Киева. 

\ifcmt
  tab_begin cols=3,no_fig,center

     pic https://imgprx.livejournal.net/7e1ce8e6027e9679df611b32075c84e3335b582d/unetDuy7QiQvFEm0gXgp96CCXHBfuvbWLNFWKdmaTByU3REWqdE9cZ6UTbzrIzMo7dryyOB7DReZwNFAe1xZZ7AZaTh7ExCDyLMFoBLM-SNzdHmaT96O9s1pw--OMYz-
		 @caption Катерина Билокур, украинская художница, 7 декабря 1900 - 10 июня 1961 

		 pic https://downloader.disk.yandex.ru/preview/84ca10b737473047b94ae17dbfaf9e1ecf45c6745d5354e73f261a0542f309d9/621e02a3/UP_XBMrfNr2Ly-GOzifT3Fk9UjX9rTew9ay3k59jmrCmalkUeiwWg22CvCeHH91mq_J6bpmRyOJonT3VoXnDag%3D%3D?uid=0&filename=0%20%D0%9D%D0%B0%D1%82%D1%8E%D1%80%D0%BC%D0%BE%D1%80%D1%82%20%D1%81%20%D0%BA%D0%BE%D0%BB%D0%BE%D1%81%D1%8C%D1%8F%D0%BC%D0%B8%20%D0%B8%20%D0%BA%D1%83%D0%B2%D1%88%D0%B8%D0%BD%D0%BE%D0%BC%20%281958-1959%29....jpg&disposition=inline&hash=&limit=0&content_type=image%2Fjpeg&owner_uid=0&tknv=v2&size=1905x884
		 @caption Натюрморт с колосьями и кувшином, 1958-1959, Катерина Билокур

		 pic https://imgprx.livejournal.net/b77ab6f3bb0b2e785df13e89c915a31cde782ea3/4FP7ppgT-YtNhEwdLGd1_0aokx1X9JZ0Pp-EFpRoUTQSrtAeaaX-pxxt5jloFNH-lH71cCv3tqf7Aj2Tzvy5Y3KBDmNoycRdF18PFOIfsEYtBw99CI0JLdYn9xALPVpLDEWmJTciY_AKl231J4YjsBgAcRi2GUferBmiQedYl-p_hqx_0Hy4qNKXNaShBjAZWi-zf3i0DrbBeCSxUgC_STYAGCs0Xx_dPBNSGHyXqXI
		 @caption «Щастя», 1950, Катерина Билокур

  tab_end
\fi
