% vim: keymap=russian-jcukenwin
%%beginhead 
 
%%file 24_11_2022.fb.filatov_boris.1.situacia_dnepr
%%parent 24_11_2022
 
%%url https://www.facebook.com/permalink.php?story_fbid=pfbid02YWU2p5LCcHUP4VeaahDX9pcPf9ZqXWyMAzQUFf2V9NXxsNEPxsyg3m4KiSQp6fN8l&id=100002157183088
 
%%author_id filatov_boris
%%date 
 
%%tags 
%%title І про ситуацію у Дніпрі станом на зараз
 
%%endhead 
 
\subsection{І про ситуацію у Дніпрі станом на зараз}
\label{sec:24_11_2022.fb.filatov_boris.1.situacia_dnepr}
 
\Purl{https://www.facebook.com/permalink.php?story_fbid=pfbid02YWU2p5LCcHUP4VeaahDX9pcPf9ZqXWyMAzQUFf2V9NXxsNEPxsyg3m4KiSQp6fN8l&id=100002157183088}
\ifcmt
 author_begin
   author_id filatov_boris
 author_end
\fi

І про ситуацію у Дніпрі станом на зараз. 
...

Відучора всі комунальні служби в повному складі безперервно працюють спільно з
енергетиками. І продовжать це робити - до повного відновлення міських систем. 

Зараз більше половини мешканців Дніпра вже зі світлом. Та пам’ятайте, що на це
впливають також аварійні і планові відключення електрики. 

Лікарні зберігають стабільне живлення (та можливість перепідключитися на
генератори).

...

Місто запустило всі насосні станції. Мешканці лівого берегу вже з водою. На
правому очікуємо її подачу десь у найближчі години. Зрозумійте: зараз водоканал
подає воду за регламентом - так, щоб уникнути гідроударів в мережі.

...

Тепло. Всі котельні правобережжя працюють і подають тепло в будинки. На лівому
березі опалення поки отримують ті, хто заживлений від Придніпровської ТЕС. 

Стабільні тепло- та водопостачання можна очікувати десь після 20:00.  ...  А
поки нагадаю, що в райадміністраціях у всіх районах і соццентрах відкриваються
пункти незламності. 

Там тепло, є вода і електрика.

Адреси:

\begin{itemize}
  \item - Лівобережна адміністрація (вул.20-річчя Перемоги, 51);
  \item - Адміністрація Амур-Нижньодніпровського району (просп Мануйлівський, 31);
  \item - Адміністрація Соборного району (площа Шевченка, 7);
  \item - Адміністрація Новокодацького району (просп. Сергія Нігояна, 77);
  \item - Центральна адміністрація (просп. Лесі Українки, 65);
  \item - Адміністрація Шевченківського району (вул. Михайла Грушевського, 70);
  \item - Центр соціальної підтримки дітей на сімей «Барвінок» (вул. Янтарна, 45);
  \item - Центр соціальної підтримки дітей та сімей «Обійми» (вул. Тверська, 41);
  \item - Центр соціальної підтримки дітей та сімей «Довіра» (вул. Тополина, 33);
  \item - Центр соціальних служб по Новокодацькому району (вул. Новоорловська, 2/92);
  \item - Відділ територіального центру в Чечелівському районі (вул. Романа Шухевича, 28/32);
  \item - Відділ територіального центру в Самарському районі (вул. Іларіонівська, 2);
  \item - Міський центр соціальної допомоги (вул. Лоцманська).
\end{itemize}
...

Ми все витримаємо і все відновимо. Почекайте ще трохи.

\ii{24_11_2022.fb.filatov_boris.1.situacia_dnepr.orig}
\ii{24_11_2022.fb.filatov_boris.1.situacia_dnepr.cmtx}
