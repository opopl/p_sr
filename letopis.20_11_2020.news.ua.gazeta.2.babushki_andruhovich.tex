% vim: keymap=russian-jcukenwin
%%beginhead 
 
%%file 20_11_2020.news.ua.gazeta.2.babushki_andruhovich
%%parent 20_11_2020
 
%%url https://gazeta.ua/blog/53935/babushki-sterezhut-minule-molod-nablizhaye-majbutnye-sansi-na-peremogu-rivni
 
%%author 
%%author_id 
%%author_url 
 
%%tags 
%%title Бабушки стережуть минуле. Молодь наближає майбутнє. Шанси на перемогу рівні 
 
%%endhead 

\subsection{Бабушки стережуть минуле. Молодь наближає майбутнє. Шанси на перемогу рівні}
\label{sec:20_11_2020.news.ua.gazeta.2.babushki_andruhovich}

\Purl{https://gazeta.ua/blog/53935/babushki-sterezhut-minule-molod-nablizhaye-majbutnye-sansi-na-peremogu-rivni}

\begin{leftbar}
	\bfseries
Полтавські битви насправді зовсім не місцевого значення. В них відбивається вся
наша теперішня ідентичнісна розірваність.
\end{leftbar}

Не цілком зрозуміло, чому Карл XII замість відійти від Полтави лишався там,
вибравши битву, в якій --- і він про це знав --- сумарні сили супротивника
перевищували його сили приблизно втричі. Про нищівну перевагу москвинів у
кінноті й артилерії взагалі краще не згадувати.

З такими розкладами то не битва мала бути, а суцільне побоїсько.

Карл міг його уникнути, своєчасно зняти облогу і відступити від фатального
міста. Тим самим зберегти військо, дати своїм вірним, але смертельно виснаженим
каролінерам перепочити, ще раз набратися славного бойового духу, підстерегти
воєнну фортуну деінде --- під Харковом чи Курськом, лиш не під Полтавою.

Проте Карл, Carolus Rex, був не кимось іншим, а собою. Він, як і його
каролінери, ніколи не відступав. У тактичному статуті його армії такого
маневру, як відхід, узагалі не передбачалося.

Що ж, не скористався можливістю організованого тактичного відходу --- й довелося
вже не просто відступати, а мчати стрімголов (у товаристві Мазепи) за турецький
кордон.

Пьотр тим часом насолодився вікторією по саме нікуди --- заливаючи нутрощі свої
та генералітету алкоголем, а полтавські поля --- кров'ю невідкладно страчуваних
на місці козаков изменников. Їм відмовлялося навіть у привілеї стати, як шведи,
полоненими. Бо горе переможеним --- й особливо тим, які не суб'єкти історії.

\begin{leftbar}
	\bfseries
Горе переможеним --- й особливо тим, які не суб'єкти історії
\end{leftbar}

Наступного дня Пьотр в'їхав тріумфатором у Полтаву і, втомлений вже не так
битвою, як тяжким вікторіанським похміллям, зупинився на спочинок у домі
певного козака, де нібито стаціонував гарнізонний комендант. Є всі підстави
припускати, що царська пиятика, бессмысленная и беспощадная, знайшла в тому
місці друге дихання. Наслідком цього бурхливого --- назовім його так --- нічлігу
140 років по тому став пам'ятник із написом \enquote{Петръ I покоился здесь после
подвиговъ своихъ 27 Июня 1709 года}.

До цього \enquote{монумента царевому відпочинку}, зведеному в центрі Полтави на розі
тодішньої Дворянської з тодішньою й теперішньою Спаською, безпосередньо
причетний петербурзький Брюллов (не ще один знаний нам Карл, а Карла брат,
архітектор). Пам'ятник, як слушно описує Вікіпедія, \enquote{являє собою прямокутний
триярусний чавунний обеліск} на гранітному цоколі. Вгорі його вінчають атрибути
царського відпочинку --- горизонтально покладені щит і меч та шолом над ними. На
нижньому ярусі зображений лев, чи то сплячий, чи вражений і повержений. Якщо
друге, то він шведський лев. Посередині ж обеліска паношиться герб Російської
імперії, орел о двох головах.

Імперія дуже дбала про ту, свою Полтаву, а точніше, пам'ять про неї, такий собі
аналог нинішнього \enquote{деды воевали}. Крім Пам'ятника Слави (зведений 1811 р.),
увінчаного тим-таки двоголовцем, та щойно описаного нічліжного пам'ятника
(1849), є ще пам'ятник коменданту Полтави Келіну (1909), статуя Петра в
натуральний зріст перед входом до Музею Полтавської битви (1915), а також ряд
інших пам'яток пізнішого часу, що ними вже СССР як зразковий спадкоємець
матушки России донесхочу наситив простір українського серця.

\begin{leftbar}
	\bfseries
Імперія дуже дбала про ту, свою Полтаву, а точніше, пам'ять про неї, такий собі
аналог нинішнього \enquote{деды воевали}
\end{leftbar}

І щойно останніми роками цей простір захвилювався.

Тож тепер лише два полтавські пам'ятники свиням (один --- \enquote{м'ясної породи}, а
другий --- \enquote{одвічній годувальниці}), як і пам'ятник полтавській галушці, можуть
почуватися в безпеці й певності. За жодного повороту історичних обставин цих
пам'ятників валити не будуть. На противагу деяким іншим.

Шість років тому почалося щось таке, що можна б назвати… Як? Пробудженням?
Опритомненням? У будь-якому разі, той початок надзвичайно виразний: 21 лютого
2014 року в Полтаві повалили Леніна, чи не першого на Лівобережній Україні. А
далі вже і власними своїми руками Путін став робити все, щоб двоголові орли
імперії відчули на собі, як Україна від неї йде.

Так над орлом, що вінчав 13-метровий заввишки Пам'ятник Слави, з'явилися
синьо-жовтий і червоно-чорний прапори навперехрест. Орла ж на пам'ятнику
царевого спочинку молодь і самого розмалювала в синє й жовте. У такому вигляді
цьому реліктові петербурзького ніколаєвського класицизму й можна було б дати
спокій (Брюллов-брат же!), якби не умовні бабушки. Вони знову помалювали
чавунного орла на чорно. Кажуть, ніби в такий спосіб унаочнилася метафора з
конфліктом поколінь --- синьо-жовта юнь vs чорні бабушки. Станом на минулу суботу
я міг бачити, що бабушкінське зверху, але не думаю, що то останнє слово. М'яч
на полі молодих.

Загалом же з бабушками непросто. Це їхніми немічними тілами прикрився
пройдоха-мер, коли 2013-го заблокував перейменування вулиці Паризької Комуни на
Героїв Крут. Вулиця Паризької Комуни --- це та сама Дворянська, на якій той самий
пам'ятник. Парадокс бабушок у тому, що їм однаково, що захищати, --- чи Паризьку
Комуну, а чи царського двоголового орла. Аби лиш минуле.

\begin{leftbar}
	\bfseries
Парадокс бабушок у тому, що їм однаково, що захищати, --- чи Паризьку Комуну, а
чи царського двоголового орла. Аби лиш минуле
\end{leftbar}

Мер натомість уляпався в історію тим, що, за свідченням Вікіпедії, всіляко
саботував декомунізацію, похвалявся, що \enquote{поки є міським головою, жодна вулиця в
Полтаві перейменована не буде}, і провокував громадян тепер уже крилатою
фразою: \enquote{Якщо я впевнений, що можу обійти закон і мені за це нічого не буде, –
я буду це робити постійно}.

Про всяк випадок запам'ятаймо йому слова \enquote{і мені за це нічого не буде}. Кажуть,
ніби його шанси на близьких уже виборах знову надзвичайно високі.

Проте склалося так, що 2016 року й цей мастодонт капітулював. І Паризька Комуна
з міста пішла, а вулицю Дворянську перейменували на Пилипа Орлика. При цьому
все вийшло навіть на краще. Бо хто небіжчикові Петрові якісь Герої Крут? А от
Пилип Орлик, генеральний секретар Мазепи і перший законник майбутньої України,
йому найлютіший ворог. Та він його з притаманною собі насолодою викінченого
садиста залюбки закатував би в якій-небудь Петропавловці! А тепер він, Пьотр,
вимушено спочиває на вулиці з його іменем --- і в цьому якийсь дивний компроміс,
чи радше оксюморон.

Полтавські битви насправді зовсім не місцевого значення. В них відбивається вся
наша теперішня ідентичнісна розірваність.

Бабушки стережуть минуле, в якому їм було жахливо, але за цей жах вони його й
люблять.

Молоді активісти наближають майбутнє, десакралізуючи російського орла, бо під
ним ніяке майбутнє неможливе.

Шанси на перемогу в обох сторін знову приблизно рівні, але суб'єкт уже ні за що
не погодиться знову перейти в об'єкти.

Публікується з дозволу видання \enquote{Збруч}
