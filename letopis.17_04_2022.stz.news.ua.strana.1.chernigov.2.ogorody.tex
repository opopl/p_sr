% vim: keymap=russian-jcukenwin
%%beginhead 
 
%%file 17_04_2022.stz.news.ua.strana.1.chernigov.2.ogorody
%%parent 17_04_2022.stz.news.ua.strana.1.chernigov
 
%%url 
 
%%author_id 
%%date 
 
%%tags 
%%title 
 
%%endhead 

\subsubsection{Погибших хоронили прямо в огородах}

После жесточайших обстрелов армии РФ в Новоселовке не осталось почти ни одного
уцелевшего дома. Кому-то, как мальчику Степану, снаряд всего лишь проделал
дырку в крыше или стене — сами селяне считают, что им еще повезло. Большинство
домов разрушено полностью: кое-где остались ошметки стен, но крыша обвалилась
до фундамента. Многие жилища дотла сгорели. 

До вторжения в Новоселовке проживало около 700 человек, но теперь едва ли
наберется сотня. Все остальные либо погибли, либо уехали в другие регионы
Украины или за границу, так как здесь им просто негде больше жить.

— Сколько людей без домов, без крышы над головой... Тут легче пересчитать тех, у
кого после войны еще хоть что-то осталось. А так куда не пойди — везде руины, —
рассказывает местный житель Вячеслав.

\ii{17_04_2022.stz.news.ua.strana.1.chernigov.pic.3}

Некоторых селян завалило в подвалах, где они прятались от бомбежек, а доставать
их было некому.

Погибших хоронили прямо в огородах.

— На огородах закапывали людей. Через пять дней (после того, как перестали
бомбить — Ред.) вышли хоронить, так мертвые уже черные были, разлагались. А так
даже выйти нельзя было — постоянно стреляли вот с той горы. Прямо по нам. Со
двора выйдешь — стреляют. Открывали дверь и выходили пригнувшись, потому что
пули летали над головой, как только мы вылезали из погребов, — рассказывает
местная жительница, пенсионерка. 

В Новоселовке сейчас не найти человека, который не потерял бы в этой войне
друзей или родных. 

— Нашего Валика раскопали из завалов только на четвертый день, похоронили. Все
в своих дворах. На кладбище это нереально, — рассказывает еще одна жительница
села.

На вопросы о том, сколько людей здесь погибло, ни у кого нет ответа. Под
завалами до сих пор могут находиться люди. До момента, пока российские войска
не отошли от Чернигова, некому было вытаскивать мертвых — живые пытались
выжить. Да и под постоянными обстрелами разгребать развалины было безумием. 

— Я точно знаю о троих погибших — это только те, с кем я был знаком. Сосед
напротив, взрывной волной его убило. Там же, в огороде и похоронен. Второго —
под сараем завалило, его нашли не сразу. Третьего на улице убило, — говорит
местный житель Вячеслав. 

Самому ему чудом удалось уцелеть, когда в его дом попал снаряд. 

— Я в этом доме родился, там еще моя мама жила. Сам дом у меня деревянный,
обложенный кирпичом. Прилетел снаряд… Я захожу в спальню — улицу вижу:
светится. Дырка в стене. Стол письменный, за которым я еще в школьные годы
уроки учил, разнесло в дребезги и отбросило взрывной волной от окна к двери,
так что я даже дверь не мог открыть. Если б я был в этот момент в этой комнате
или в погребе, завалило бы и убило бы взрывной волной, — рассказывает Вячеслав.

\ii{17_04_2022.stz.news.ua.strana.1.chernigov.pic.4}

Другим повезло меньше. Вот пенсионерка стоит в очереди за гуманитарной помощью
— в селе волонтеры раздают хлеб, крупы. Ее дом полностью сгорел, теперь она
живет у соседей. О войне ей говорить трудно — все время срывается на плач. 

Местные рассказывают, что по домам селян российские войска били со всей
жестокостью. Враги в первый день войны пыталась наскоком взять Чернигов, но
украинские войска их отбросили. 

— Тогда русские закрепились на вон той горе и от отчаяния били по селу из
всего, что у них было: танки, артиллерия, самолеты, — рассказывают селяне. 

На центральной улице стоит дом. Точнее — то, что от него осталось. Крыши нет.
Только фундамент и остатки стен со следами черной копоти на белом кирпиче.
Единственное, что уцелело — кусок забора и зеленая входная дверь. На ней —
предупреждающая табличка из прошлой жизни: \enquote{Во дворе — злая собака}. 

Сейчас ни двора, ни собаки, ни ее хозяина тут нет. Зато по развалинам
по-хозяйски прогуливается бездомная кошка. Ластится к прохожим, ищет тепла
человеческих рук, и в обмен на поглаживание за ухом даже позирует для фото на
фоне руин. Жизнь — на фоне смерти. 

Мужчины лезут на крыши, чтобы залатать шифером дыры в кровле. Старушки с
ведрами и тряпками моют заборы, пытаясь стереть с них угольно-черные следы
гари, а вместе с ними — свои страшные воспоминания.

— Тяжело мы это все пережили. Сидели в подвале. Пули над головой свистят,
ракеты, бомбы… И из минометов, и из пушек, и из танков стреляли. Отсюда
буквально в 100 метрах в последний день (перед тем, как российские войска
отошли — Ред.) из танка женщину на куски разорвало, а хату спалило. Обстрелы
постоянные. Было такое, что трое суток подряд били — и день, и ночь, —
вспоминает женщина.

В Новоселовке до сих пор нет ни воды, ни света, ни газа. Местные живут без всех
благ цивилизации уже больше месяца. С едой тоже проблематично: селяне выживали
в основном благодаря запасам консервации, а сейчас — с помощью волонтеров.

— Сидели в погребе с утра до вечера. Перестали стрелять — мы бегом выскочили,
картошки сварили, с консервацией поели. И так где-то 20 дней жили. Мы оттуда и
вылазить не хотели. Потому что так бомбили, что все вокруг горело. Слава богу,
наша хата осталась. А потом волонтеры приехали — так мы уже поели, хлеба
привезли, — рассказывает местная пенсионерка.

После того, как Новоселовку освободили, в село приехал Тимур — врач из Ирпеня,
который теперь помогает своим родителям. Больше месяца они выживали здесь под
обстрелами.

— Когда я приехал и увидел все это, у меня был шок. Родители были здесь все
время. Дом сильно не пострадал, но окна выбиты, и для проживания он больше не
пригоден. Ни света, ни газа — ничего нет. Месяц с родителями не было связи. У
них не было возможности уехать, машина на газу, а газа нет. Они застряли тут.
Им было очень страшно. Мама каждый раз молилась, чтобы не прилетело. В соседний
дом попал авиаснаряд — дома больше нет. Но слава богу, мои родители целы, —
говорит медик.

Местные жители своими глазами видели российских военных, сталкивались лицом к
лицу с их танками, некоторые оказывались на волосок от смерти. 

— Я как-то говорил с одним из русских. Он контрактник. Так вот он мне сказал:
\enquote{Я не знаю, что я здесь делаю. Но у меня приказ. Что я могу?}.. А как-то я шел
на вон ту гору позвонить — связь только там более-менее ловила. Иду и вижу —
навстречу мне едет русский \enquote{тигр}. Я руки из кармана вытащил, держу на виду.
Проехали мимо. Повезло, — вспоминает местный житель Виталий. 

По его словам, при уходе российские войска грабили дома местных жителей. Факты
мародерства есть, но таких зверств и пыток над мирными жителями, как в Буче, в
Новоселовке, к счастью, не было. 

— После Бучи русским нет прощения. Я не знаю, как к ним относиться теперь. У
меня сосед — каменщик, он строил дома в Ирпене и Буче. Там много переселенцев с
Донбасса, которые бежали от войны в 2014-м. От той войны они сбежали, а
\enquote{русский мир} их догнал там. Может, они им так отомстили. Вот это, — Виталий
оглядывается по сторонам, указывая на разрушенные хаты, — и есть \enquote{русский мир}.
Это у них норма бытия. А зачем нам такое? Теперь все с нуля надо начинать. Я
одного хочу: чтобы мир был. Все восстановить и жить дальше.
