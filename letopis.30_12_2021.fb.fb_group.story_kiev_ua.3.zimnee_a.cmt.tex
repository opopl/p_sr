% vim: keymap=russian-jcukenwin
%%beginhead 
 
%%file 30_12_2021.fb.fb_group.story_kiev_ua.3.zimnee_a.cmt
%%parent 30_12_2021.fb.fb_group.story_kiev_ua.3.zimnee_a
 
%%url 
 
%%author_id 
%%date 
 
%%tags 
%%title 
 
%%endhead 
\zzSecCmt

\begin{itemize} % {
\iusr{Татьяна Кохановская}
Двоечка и сейчас очень редко ходит, а вы молодец, нормальный километраж нахаживали.

\iusr{Arkadi Romansky}

а за 71ю школу - отдельное человеческое спасибо - я на 10 лет позже (начиная с
73 года) в нее тоже хаживал с Тупикова (нижней - караваевской части). тогда еще
и \enquote{девятка} ходила мимо гастронома на Большевике но влезть ни в нее, ни в 21й
или 22й троллейбус - было mission impossible.... когда начали строить
Индустриальный мост и 21й стал разворачиваться на Тупикова - появился шанс
доезжать до будущего сквера на Большевике и топать по Гули Королевой....

\iusr{Віка Виктория}

Рассказ про родные места, но я позжего года рождения, 1967..... Но я еще помню
тот дом с гастрономом впритык к мосту) \enquote{двойка} и сейчас так же ходит,
иногда.....  @igg{fbicon.smile} 


\iusr{Arkadi Romansky}
\textbf{Віка Виктория} я тоже 1967 - если Вы тоже учились в 71й - то мы знакомы...

\iusr{Valentina Kuzmitskaya-Svitich}
Пишіть ще. Пишіть, вас цікаво читати.

\iusr{Андрей Воробей}

Спасибо!

\iusr{Виктор Марченко}

Жалко скрипочку... А вдруг стал бы известным скрипачом!!
@igg{fbicon.face.grinning.smiling.eyes} 

\begin{itemize} % {
\iusr{Arkadi Romansky}
\textbf{Виктор Марченко} надо было в 5ю музыкальную ходить - и по Брест-Литовскому можно было и на 23м (трамвае) и на 5е (троллейбусе)  @igg{fbicon.smile} 

\iusr{Виктор Марченко}
Надо было... @igg{fbicon.grin} 

\iusr{Людмила Мозговая}
\textbf{Виктор Марченко} вдруг не произошло...
\end{itemize} % }

\iusr{Катя Бекренева}
Это угловой дом с Выборгской?

\begin{itemize} % {
\iusr{Анатолий Каганович}
\textbf{Катя Бекренева} Угловой \#16, а наш за ним, глубже во двор

\iusr{Катя Бекренева}
\textbf{Анатолий Каганович} понятно) В угловом жила мамина подруга, мы часто в 60-70х годах были там в гостях.
\end{itemize} % }

\iusr{Defektolog Logoped}

Посмотрела видео. Нахлынули воспоминания. Шапочка у меня была такая, как у
девочки на видео. Шапочка голубого цвета. Я носила её с удовольствием.  @igg{fbicon.smile} 

\iusr{Irina Kalinkovitsky}

Спасибо! Бесценные кадры нашего быта: и лыжи, и собака, и ранний подъём, и
кормление внука, и раскладушка. Сама на такой проспала много лет лицом к
балкону. А за спиной родители, а справа - бабушка. И всё это на 18-и метрах,
что считалось тогда просторным помещением.

Завод \enquote{Большевик}, ул.Выборгская и Индустриальная, где мы жили уже в отдельной
квартире, двухэтажный гастроном под мостом, 71 школа, метро \enquote{Большевик} были
уже позднее, в 70-е.

\iusr{Надежда Грабовская}
Скрипочка всё таки запомнилась...

\iusr{Ната Ли}
Здорово написано!!

\iusr{Λάρισα Λάρισα}
Мой отец тоже учился в 71 школе, в 1965 как раз закончил.

\iusr{Нататлья Бешун}

Огромное спасибо за теплые воспоминания о нашем районе. Я выросла на Гарматной
29, в сталинке где был хлебный магазин. А в 71 школе училась моя бабушка Белла
до войны. На Гарматной работал дедушка. \enquote{Двойка}- это отдельная тема! Ее ждали
все, знакомились на остановках, обсуждали последние новости.


\iusr{Igor Popell}

Учитель музыки говорит ученику:

— Предупреждаю, если ты не будешь вести себя как следует, я скажу твоим
родителям, что у тебя есть талант.


\iusr{Нататлья Бешун}
Видео кадры потрясающие и раритетные!

\iusr{דני כהן}
Нататлья Бешун Это мой Папа ז"ל снимал ...

\iusr{Таня Сидорова}
Цікаво та зворушливо

\iusr{Раиса Карчевская}
Прекрасный рассказ и фотографии

\iusr{Vladimir Tkachev}
Спасибо! С наступающим!!!

\iusr{Елена Рачина}

В 65 г.я тоже была первоклассницей, но на скрипке играл мой старший брат, а я
играла на нервах у родителей.

\iusr{Marina Lavrow}
\textbf{Елена Рачина} а я в 67м .

\iusr{Олег Бабич}

Очень трогательно! Фильм напомнил и мое детство. Я недалеко жил. А что
случилось с 500 -летним дубом, в каком месте он стоял?

\begin{itemize} % {
\iusr{Анатолий Каганович}
\textbf{Олег Бабич} Дуб так и стоит. Под мостом возле 5го цеха.

\iusr{Олег Бабич}
\textbf{Анатолий Каганович} Ок!
\end{itemize} % }

\iusr{Лариса Горовая}
Чудесный рассказ! И чудесные кадры. Как сейчас говоряь \enquote{Світлини}.

\iusr{Tatiyana Serbina}
сберкасса на Гарматной 10, у ПТУ № 7 ?

\ifcmt
  ig https://scontent-frx5-2.xx.fbcdn.net/v/t39.30808-6/270894345_661931391477581_7486813382853649224_n.jpg?_nc_cat=109&ccb=1-5&_nc_sid=dbeb18&_nc_ohc=KDsX5ZBtmnsAX8aHUOP&_nc_ht=scontent-frx5-2.xx&oh=00_AT8-AWI4xmAMA1PqXlgDK5KNdROlFJegIWIVzldEMDrFNA&oe=61D70A89
  @width 0.4
\fi

\begin{itemize} % {
\iusr{Анатолий Каганович}
\textbf{Tatiyana Serbina} Именно эта сберкасса.

\iusr{Tatiyana Serbina}
\textbf{Анатолий Каганович} в фильме инспектор не туда приехал @igg{fbicon.laugh.rolling.floor} 

\ifcmt
  ig https://scontent-frt3-2.xx.fbcdn.net/v/t39.30808-6/271080219_661980684805985_4797540544967484284_n.jpg?_nc_cat=103&ccb=1-5&_nc_sid=dbeb18&_nc_ohc=Ayzj2VRk-IsAX8HTFHt&_nc_ht=scontent-frt3-2.xx&oh=00_AT-Fj9X6xe6ZSiBJJzmp2Fx98kkhHmZ9XdwLAIhC7R50qg&oe=61D7EDBB
  @width 0.4
\fi

\end{itemize} % }

\iusr{Tatiyana Serbina}

В фильме \enquote{Инспектор уголовного розыска}, говорят по рации прибыть на ограбление
сберкассы по адр. Гарматная 12, а прибывает инспектор совсем не на
Гарматную, сберкасса тода была в маленькой хатке около ПТУ-7!

\ifcmt
  ig https://scontent-frx5-1.xx.fbcdn.net/v/t39.30808-6/270824456_661951204808933_8294283884338621264_n.jpg?_nc_cat=111&ccb=1-5&_nc_sid=dbeb18&_nc_ohc=yxcAZNYKKtMAX8DxOyc&_nc_ht=scontent-frx5-1.xx&oh=00_AT_ZelWqH6vB6tWIz_zxgttkobf5jZIFmvfIYzmom56SeQ&oe=61D71E73
  @width 0.4
\fi

\iusr{Victoria Novikov}

Спасибо! Замечательный рассказ! Я точно решила в детстве, что если у меня будут
дети, то они НИКОГДА не будут играть на фортепиано. Даже если захотят!
@igg{fbicon.face.grinning.smiling.eyes}  @igg{fbicon.laugh.rolling.floor} 


\iusr{Catherine Murashchyk}

Такой симпатичный рассказ, такой красивый мальчик, но господа! Пожалуйста, не
пишите «со школы»! Это неграмотно! У киевлян прямо болезнь какая-то!

Правильно «из школы»! А то начинаешь сомневаться в качестве советского
образования  @igg{fbicon.grin} 


\iusr{Алла Васильева}

Неймовірне відео, згадується все, круті схили Голосіївського лісу.... . Дякую.
Розповідь чудова і за змістом, і за стилем  @igg{fbicon.hand.waving} 

\iusr{Тетяна Табачна}
папа красавец был

\end{itemize} % }
