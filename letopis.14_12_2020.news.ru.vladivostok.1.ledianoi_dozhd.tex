% vim: keymap=russian-jcukenwin
%%beginhead 
 
%%file 14_12_2020.news.ru.vladivostok.1.ledianoi_dozhd
%%parent 14_12_2020
 
%%url https://www.newsvl.ru/vlad/2020/12/14/195422/
 
%%author 
%%author_id 
%%author_url 
 
%%tags vladivostok,russia
%%title После ледяного дождя во Владивостоке началась массовая вырубка деревьев (ФОТО)
 
%%endhead 
 
\subsection{После ледяного дождя во Владивостоке началась массовая вырубка деревьев (ФОТО)}
\label{sec:14_12_2020.news.ru.vladivostok.1.ledianoi_dozhd}
\Purl{https://www.newsvl.ru/vlad/2020/12/14/195422/}

\index[cities.rus]{Владивосток!Россия!Ледяной дождь, 14.12.2020}

\ifcmt
  pic https://static.vl.ru/news/1607923933854_default
\fi

Во Владивостоке продолжают пилить повреждённые ледяным дождём деревья. Где-то
убирают сломанные ветви, а где-то рубят «под самый корешок», как у медицинского
университета на Острякова.

Всего в городе пострадало около 80\% деревьев. В администрации Владивостока
объясняют, что только весной будет понятно, какие из них выжили, каким
потребуется обрезка, а какие погибли и их придётся спилить совсем.

Сейчас полностью распиливать можно только те деревья, которые упали, вырваны с
корнем или переломились у корня. Обломанные ветки убирают чуть ниже места
слома. Целые деревья рубить под корень нельзя.

На территории ТГМУ (Острякова, 2) после ледяного дождя остались 40 ровненьких
пеньков. Сложилось впечатление, что у рабочих просто не было стремянки, чтобы
подпилить сломанные ветки, поэтому пилили по стволу. Однако в университете
объяснили, что рубили только те деревья, которые сломались.

«Деревья были повреждены льдом. Мы просто боялись, что они повалятся на людей
или на машины. Пришлось ликвидировать последствия. Всё делали своими силами», –
рассказала заведующая хозяйством административно-хозяйственного управления.

\ifcmt
tab_begin cols=2
	caption После ледяного дождя во Владивостоке началась массовая вырубка деревьев

	pic https://static.vl.ru/news/1607924236637_default
	pic https://static.vl.ru/news/1607924241375_default
	pic https://static.vl.ru/news/1607924243399_default
	pic https://static.vl.ru/news/1607924245389_default
tab_end
\fi

«Всё сделано в рамках закона. Были сильные повреждения. Эти повреждения
представляли угрозу для пребывания в непосредственной от них близости людей», –
подтверждает один из проректоров.

Где бы это ни происходило, если вы видите, что кто-то рубит дерево, вместо
того, чтобы аккуратно отпилить сломанные ветки, сфотографируйте или снимите
видео – у вас останутся доказательства, что дерево не было сломано пополам.

Обратитесь к рабочим. Среди них должен быть старший участка, который сможет
предъявить документы на свою организацию и при необходимости связаться с
сотрудником мэрии.

Если люди с пилами не могут объяснить, что они делают и с чьего позволения,
звоните в полицию, управление охраны окружающей среды и управление дорог и
благоустройства администрации Владивостока.

Статья полностью: \url{https://www.newsvl.ru/vlad/2020/12/14/195422}
Новости Владивостока на VL.RU


