% vim: keymap=russian-jcukenwin
%%beginhead 
 
%%file 21_07_2020.fb.lnr.19
%%parent 21_07_2020
 
%%endhead 
  
\subsection{Правозащитник о проблемах ополченцев, хранящих боеприпасы}
\url{https://www.facebook.com/groups/LNRGUMO/permalink/2863796020398620/}

\vspace{0.5cm}
{\small\LaTeX~section: \verb|21_07_2020.fb.lnr.19| project: \verb|letopis| rootid: \verb|p_saintrussia|}
\vspace{0.5cm}

Руководитель КПЦ «Война и мир» Андрей Седлов предупредил военнослужащих и
бывших ополченцев об уголовной ответственности за хранение боеприпасов.

«К нам стали обращаться ребята, которые ранее проходили службу в ополчении.
Хочу отметить, что мы всем готовы помогать, но порой, случаи, с которыми к нам
обращаются бывшие ополченцы бывают довольно безрассудные. Ребята после службы
уезжают, оставляя свои вещи на хранение друзьям. Не занимайтесь глупостями!
Если вы берете на хранение вещи, проверяйте, что там находится. Кто-то случайно
забыл патрон, или какую-то часть вооружения. Будьте внимательнее. Вы же и сами
подводите своих друзей. Не создавайте лишние проблемы», — рассказал он.

Авторская съемка военкора команды News Front Катерины Катиной.
  
