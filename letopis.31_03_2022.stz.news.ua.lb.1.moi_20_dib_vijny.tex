% vim: keymap=russian-jcukenwin
%%beginhead 
 
%%file 31_03_2022.stz.news.ua.lb.1.moi_20_dib_vijny
%%parent 31_03_2022
 
%%url https://lb.ua/society/2022/03/31/511668_mariupol_tochka_nepovernennya_moi.html
 
%%author_id news.ua.lb,stanislavskyj_ivan.mariupol.fotograf.zhurnalist
%%date 
 
%%tags 
%%title Маріуполь, точка неповернення. Мої 20 діб війни
 
%%endhead 
 
\subsection{Маріуполь, точка неповернення. Мої 20 діб війни}
\label{sec:31_03_2022.stz.news.ua.lb.1.moi_20_dib_vijny}
 
\Purl{https://lb.ua/society/2022/03/31/511668_mariupol_tochka_nepovernennya_moi.html}
\ifcmt
 author_begin
   author_id news.ua.lb,stanislavskyj_ivan.mariupol.fotograf.zhurnalist
 author_end
\fi

\begin{quote}
\em\bfseries	
Від LB.ua. Ми розпочинаємо цикл публікацій щоденників маріупольського
журналіста Івана Станіславського, який 20 днів прожив у заблокованому
російськими військами місті і записував, що відбувалося. 
\end{quote}

\begingroup
\bfseries
– Я народився і все життя прожив у Маріуполі. Так склалося, що з цим містом
була пов'язана чи не вся моя діяльність. Я працював у футбольному клубі під
назвою \enquote{Маріуполь}, публікував краєзнавчі нариси, робив фотографії, досліджував
мистецьку спадщину, проводив тематичні екскурсії. Я дуже добре знав своє місто
і намагався примножувати знання.

Але найголовнішим завданням для мене завжди було зацікавити Маріуполем інших,
бо Маріуполь того вартий. Я вважав так раніше і вважаю так зараз. Це дивовижне
місто зі складним минулим і ще більш складним сьогоденням. Природно, що я
пов'язував своє майбутнє з тим місцем, де народився, а інших варіантів не
уявляв.

24 лютого Росія напала на Україну та почала безглузду, варварську війну,
спричинивши для нашого міста катастрофу, яка не порівнянна навіть з наслідками
Другої світової війни. Місто, яке я так любив, тотально зруйноване, зруйновано
тими, хто називає себе нашими братами.

\ii{31_03_2022.stz.news.ua.lb.1.moi_20_dib_vijny.pic.1}

Через 20 діб війни я евакуювався з розумінням, що мене \enquote{звільнили} не тільки
від 35 років минулого. Можливо, що будь-яке майбутнє, пов'язане з Маріуполем,
більше неможливе. На згарищі можна звести нові житлові будинки, лікарні,
фонтани і театр, але відбудувати можна не все. У ментальному сенсі є межа, за
якою зв'язок минулого з сучасним переривається, між ними утворюється порожнеча,
заповнити яку ніколи не вдасться. Багато маріупольців уже відчувають це. Я щиро
сподіваюся, що Маріуполь постане з попелу, як Фенікс, і мені колись пощастить
повернутися в рідні краї. Проте той колишній Маріуполь – здається, він уже
пішов у небуття разом з тисячами загиблих містян, про яких лишилися тільки
спогади.
\endgroup

\subsubsection{24 лютого. День 1}

Початок війни проспав. Прокинувся о сьомій від дзвінка дружини. \textbf{Повна
дезорієнтація, завис на телефоні, намагався зрозуміти, що коїться в країні та
місті. Рано-вранці ворог обстріляв аеропорт Маріуполя та низку військових
об'єктів по всій Україні.} Дістав сумку, з якою їжджу у відрядження. Якісь речі
в ній уже лежали, щось покидав, машинально, без усвідомлення, потрібно це чи
ні. Зібрав документи. Думка: може, поїхати в село, далі від міста? 

І. подалася на роботу, як зазвичай, а вже в дорозі їх завернули по домівках.
Невже це не можна було зробити раніше, атаки почалися ще о п'ятій ранку?! Побіг
у гараж по машину, щоб забрати І. з роботи, бо регулярного транспорту звідти
немає. \textbf{В інтернеті пишуть, що з моря висадився російський десант. Це означає,
що якісь автошляхи, можливо, перерізано.} Поки доїхав з гаража додому, І.
повідомила, що вже на півдорозі. Чекатиму вдома. І. повернулася. Каже, що їм
сказали працювати з дому - підтягнути хвости. Яка маячня, кому це потрібно?
Посварилися. Усі збуджені. У село вирішили не їхати. Поки що.

\begin{quote}
\em\large
З новин зрозуміло, що розпочалася повномасштабна війна. Загарбники наступають
одразу на кількох напрямках, здійснилися найгірші прогнози. Отакої! Увесь
попередній місяць іноземні ЗМІ приїжджали робити репортажі про те, як Маріуполь
готується до війни, і дивувались, майже не знайшовши приготувань. Я також
вважав, що загроза перебільшена
\end{quote}

О 8:30 щось дуже гучно вибухнуло в районі аеропорту. Метінвест повідомив про
консервацію виробництва на підприємствах міста. Якось занадто оперативно.
Вочевидь, вони готувалися до цього заздалегідь, рішення вже було прийняте.
\textbf{Чутки про десант з моря виявилися брехнею. Якось відлягло. Подивилися звернення
мера, як завжди пафосно та майже беззмістовно.}
