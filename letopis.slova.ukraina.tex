% vim: keymap=russian-jcukenwin
%%beginhead 
 
%%file slova.ukraina
%%parent slova
 
%%url 
 
%%author 
%%author_id 
%%author_url 
 
%%tags 
%%title 
 
%%endhead 
\chapter{Украина}

Так и живем в \emph{Украине}: удочек нет у власти для простых \emph{украинцев},
и рыбы нет для страждущих, \textbf{Еще никому из президентов не удалось объединить
Украину}, Александр Гончаров, strana.ua, 24.05.2021

Євробачення 2021: \emph{Україна} посіла п'яте місце, beztabu.net, 23.05.2021

\emph{Украину} нужно лишить права оценивать выступления участников от
страны-агрессора: это аморально, Игорь Кондратюк, obozrevatel.com, 23.05.2021

Уберите российский сапог и \emph{украинская} культура побьет все мировые
рекорды, Елена Кудренко, obozrevatel.com, 23.05.2021

\emph{Украинские} туристы подрались в самолете: мама младенца озвучила свою
версию конфликта, Марина Петик, obozrevatel.com, 24.05.2021

Не просто \emph{Україна}, а \emph{Україна}-Русь. Як Михайло Грушевський захищає
державні кордони?  radiosvoboda.org, 23.05.2021

Навіть попри те, що Путін не почав нову фазу завоювань в \emph{Україні} –
останній епізод кілька тижнів тому змусив людей у Вашингтоні і Києві
понервувати,
radiosvoboda.org, 24.05.2021

«Дайте нам грошей!»: большая \emph{украинская} мечта о великой халяве, from-ua.com, 24.05.2021

«Евровидение» в Европрессе: Италия — победитель, Германия и Британия  «пасут
задних», а \emph{Украину} назвали «кошмарной танцевальной угрозой», sharij.net,
24.05.2021

Зато выступление GoA от \emph{Украины} с песней «Шум» назвали одним из лучших
выступлений вечера,
sharij.net, 24.05.2021

Паралимпийская сборная \emph{Украины} заняла первое место на чемпионате Европы,
завоевав 37 золотых медалей, sharij.net, 23.05.2021

Кладбище империй: Чему научила США и \emph{Украину} война в Афганистане,
sharij.net, 24.05.2021

Для \emph{Украины}, где миллионы людей с разных сторон, Холодная война долго не сможет быть холодной,
\textbf{Дело Протасевича - еще один повод раздуть Холодную войну}, Денис Жарких, strana.ua, 24.05.2021

Вот я читаю посты некоторых российских патриотов, и не нахожу особой разницы
между ними и патриотами \emph{украинскими},
\textbf{Дело Протасевича - еще один повод раздуть Холодную войну}, Денис Жарких, strana.ua, 24.05.2021

Тобто. вже рік тому можна було бачити, що Північний потік 2 для \emph{України}
- програна справа, \textbf{Наш МИД живет в каком-то выдуманном мире}, Владимир
Воля, strana.ua, 24.05.2021

У келіях Печерського монастиря запалав світильник \emph{української} культури. Саме
тут бере свій першопочаток давня \emph{українська} література, художнє мистецтво,
медицина. Нестор-літописець – перший історик \emph{України}-Руси, автор Повісті, що є
головним джерелом вивчення \emph{української} історії, Агапій – перший відомий лікар,
Аліпій – перший живописець...
radiosvoboda.org, 06.05.2017

Шановні туристи, ми не заперечуємо вагомості внеску цього київського духовного
центру до скарбниці всіх східнослов'янських культур, але наголошуємо, що Лавра
– це передусім феномен \emph{української} культури».
radiosvoboda.org, 06.05.2017

Відвідувачі можуть запитати, чому у тексті Повісті Нестора Літописця (мова
твору церковнослов'янська) древня обитель іменується лише Печерською. Один із
найкращих знавців давньоруських літописів професор Василь Яременко стверджує –
це лише доказ того, що розмовною мовою автора була \emph{українська}. Саме тому
«\emph{українська} лексика ллється суцільним потоком у Повісті: жито, колодязь,
подружжя, туга, печера…» – аргументує професор.
radiosvoboda.org, 06.05.2017

\emph{Украина} была одной из главных составных частей Советского Союза — чрезвычайно
развитой в экономическом отношении, многолюдной и комфортной республикой.
\emph{Украинцы} по сути являлись второй имперской нацией, а выходцы из республики со
своими многочисленными группами поддержки порою оказывались на самых вершинах
власти, \textbf{Надоела Украина? Ежедневное «иди и смотри»}, Константин Кеворкян, ukraina.ru, 24.05.2021

Скоро в \emph{Украину} ворвется лето: синоптики заявили о \enquote{шикарной} погоде и сделали детальный прогноз, obozrevatel.com,
25.05.2021

На реках и водоемах с начала года утонуло более 300 \emph{украинцев}: причины жуткой
статистики, obozrevatel.com, 25.05.2021

\emph{Україна} і ЄС готують рішення про припинення польотів до і над Білоруссю.
Якими можуть бути інші санкції? radiosvoboda.org, 25.05.2021

«Це незграбно і підло. Винуватці зриву спецоперації \emph{українських}
спецслужб мають бути встановлені і покарані. А спроби їх «відмазати» марні», –
написав у твітері колишній міністр, radiosvoboda.org, 25.05.2021

\emph{Украина} может стать главным карателем Белоруссии, vz.ru, 25.05.2021

Сама \emph{Украина} попадет в патовую ситуацию, ведь для нее полеты над Россией
уже и так закрыты, а теперь закроется небо еще и над Белоруссией. Облетать
\emph{украинцам} придется много и долго, vz.ru, 25.05.2021

Важной является позиция \emph{Украины}, которая очень часто действует в русле западных политиков,
vz.ru, 25.05.2021

Уж если борщ не \emph{украинский}, Царит не наша правота – Расколот мир! В нём нет
единства, Повсюду мрак и темнота, odnarodyna.org, 25.05.2021

Бойцы, хранитель святыни, Над нами реет вечный стяг!  Неспевший славу \emph{Украине}
Державе нашей – лютый враг, odnarodyna.org, 25.05.2021

Эй, Макаревич, дорогуша, Ты лоб, пожалуйста, не морщь!  Ты \emph{украинцам} плюнул в
душу, А заодно попал нам в борщ!
odnarodyna.org, 25.05.2021

Но независимой державой \emph{Украйне} быть уже пора: И знамя вольности
кровавой, Я подымаю на Петра, Александр Пушкин, Полтава

Під впливом «Історії Русів» писали свої твори на \emph{українську} тематику
Кіндрат Рилєєв, Микола Гоголь, Тарас Шевченко. Використовував «Історію Русів» і
Пушкін, пишучи «Полтаву». Саме з цього \emph{українського} твору з'явилися і в
творі Пушкіна мотиви героїчної \emph{української} історії, мотиви боротьби
\emph{України} за незалежність, radiosvoboda.org, 25.05.2021

Храня суровость обычайну, Спокойно ведал он \emph{Украйну}, Молве, казалось, не
внимал, И равнодушно пировал, Александр Пушкин, Полтава

На вашому YouTube каналі одне із рекордсменів за переглядами – відео
\enquote{Чи матюкались давні \emph{українці}?}. Давайте розставимо крапки над і
щодо лайок в \emph{українській} мові. Чи винен в цьому умовний колективний
путін?  pravda.com.ua, 25.05.2021

Остап \emph{Українець}: Твоя індивідуальна \emph{українська} мова не має бути такою, як
вимагає неіснуючий \enquote{золотий стандарт}, pravda.com.ua, 25.05.2021

Часом ви дозволяєте собі говорити страшні речі. Про те, що \emph{українська}
мова не наймилозвучніша в світі, що вона не походить безпосередньо від
фінікійського письма, що лайки прийшли в неї не виключно як запозичення з
російської, а \emph{українською} не лише з мальвами розмовляють. Чому ви досі
не в списку ворогів нації? pravda.com.ua, 25.05.2021

Якщо розбирати, яке слово де-факто прийшло з \emph{української}, а яке
запозичене, вийдуть дуже прикольні речі. Вийде, що слово \enquote{щур} в
\emph{українській} мові скоріш за все з російської, а російське слово
\enquote{крыса} скоріш за все з \emph{української}, просто свого часу ми ними
помінялися, pravda.com.ua, 25.05.2021

Верховный суд \emph{Украины} принял решение о том, что в Ровенской области настоятели
двух захваченных храмов УПЦ останутся проживать вместе со своими семьями в
церковных домах, fraza.com, 25.05.2021

Сладкие парочки: Россия и \emph{Украина}, mikle1.livejournal.com, 21.11.2009

Даже я понял, что он лично виноват в том, что Россия до сих пор не развалилась
на демократические Чечни и Запопинские республики Окрайны, Китайны и прочая.
Еще он виноват в \emph{украинском} голодоморе 1932 года, монголо-татарском
нашествии и проигрыше Мазепы в 1709 году. Змею, укусившую вещего Олега тоже
тренировал он, mikle1.livejournal.com, 21.11.2009

Такие вот культы личности с обратными знаками. 2 страны, 2 личности, 2 системы
ценностей и 2 результата. Все смелые, прогрессивные, правильные, модные,
продвинутые, революционные и прочая говорят, что лучше \emph{украинский}
вариант, mikle1.livejournal.com, 21.11.2009

Не возвращайся в \emph{Украину}!  Угроза жизни дышит в спину, В затылок дышит
подлецо – Фашизма подлое лицо! Юнна Мориц, www.owl.ru

Василь Шкляр: «\emph{Українською} заговорять навіть в Африці, якщо без неї неможливо
буде повноцінно жити», globlvillage.com, 15.07.2020

Тим часом, \emph{Україна} вже не вперше у своїй історії опинилася сам на сам з
найкровожерливішим агресором. Соромно бачити, як плазує перед Росією Франція,
як стримано поводиться Німеччина, наче вони досі не оговталися від давніх
поразок у війнах з цим монстром, Василь Шкляр, globlvillage.com, 15.07.2020

Зеленский уже и не скрывает, что он является марионеткой в руках иностранцев и
все делает в их интересах, а не интересах \emph{Украины} и \emph{украинцев}.
Вот честно и открыто рассказывают как будут продавать землю иностранцам о чем
ранее молчал, Ростислав Кравец, strana.ua, 25.05.2021

\enquote{Руслан и Людмила} в вольном пересказе первого премьер-министр
\emph{Украины} Витольда Фокина. Цікаво. Чудово, www.skadtalk.cc, 21.06.2020

\enquote{Господи милосердний, \emph{Україна} – це не Раша}, Ірина Фаріон,
obozrevatel.com, 16.06.2019

Недогромадяни. Як колаборантки викидали \emph{українські} паспорти і нарвалися
на неприємності, \url{https://www.youtube.com/watch?v=yMFZEZLmnec} 21.05.2021,

Викинула \emph{український} паспорт – отримала проблеми. Чому колаборантів
варто позбавити права голосу? - Сергей Стерненко,
\url{https://www.youtube.com/watch?v=DR-IeF3SKA0}, 20.05.2021

Цей пам'ятний день було запроваджено Організацією \emph{українських} націоналістів на
честь наших Героїв, щоб ми, \emph{українці}, про них не забували. Щоб від покоління до
покоління, до наших дітей і внуків передавалася \emph{українська} традиція, віра,
патріотизм, виховання, національна пам'ять та національні Герої, – наголосив
Олег Тягнибок, zz.te.ua, 25.05.2021

\enquote{Ты на \emph{Украине} сейчас находишься! Ты понял? Ты понял меня? Ты на
\emph{Украине} находишься!}, Менти настільки \enquote{переатестувались}, що
підзабули уроки Майдану, Євген Дикий, gazeta.ua, 23.05.2021

Але ні, ми тут, в \emph{Україні}, для чогось робили Революцію гідності - зокрема для
того, щоб подібні нащадки НКВД не мали жодного шансу безкарно вбивати нас.
Здається, за 7 років вони настільки \enquote{переатестувались}, що підзабули уроки
Майдану, Євген Дикий, gazeta.ua, 23.05.2021

Заявляю, что я нахожусь в \emph{Украине} и не собираюсь прятаться от
правосудия. Я и дальше буду участвовать в законных следственных действиях и
добиваться справедливости как для себя, так и для всех \emph{украинских}
избирателей, доверивших мне высокое звание народного депутата \emph{Украины}»,
— подчеркнул оппозиционный политик, fraza.com, 25.05.2021

Єдине «тішить» у цій ситуації, що й \emph{українці} дали «достойну відповідь»
слов'янам. \emph{Українські} глядачі лише проголосували за виступ від однієї
слов'янської країни – Росії. Хоча Росію (що показово!) представляла співачка,
яка не є слов'янкою. Але ж \emph{українці} без «братньої Росії» жити не можуть – хоча
й воюють із нею. Тому й дали співачці Маніжі від Росії 4 бали. Добре, що не 12, 
Про слов'янську солідарність? Петро Кралюк, day.kiev.ua, 24.05.2021

Участник кампании по вакцинации публичных людей от коронавируса в \emph{Украине} не
может найти вторую дозу препарата, strana.ua, 25.05.2021

В половине регионов \emph{Украины} приостановлена вакцинация от Covid-19,
strana.ua, 24.05.2021

Ковидные качели. В \emph{Украине} снова пошел рост числа новых зараженных. За
сутки плюс 3 395 человек, strana.ua, 26.05.2021

... и в целом, в \emph{Украине} властям можно всё, \textbf{Украина легализовала
сомалийские действия государства в воздухе}, Елена Лукаш, strana.ua, 26.05.2021

- Александр Гарриевич, не \enquote{в \emph{Украине}}, а \enquote{на
\emph{Украине}}. Это важно. \enquote{В \emph{Украине}} говорят
\emph{украинские} сепаратисты, а сторонники триединого русского мира говорят
\enquote{на \emph{Украине}}, Мак Сим, zen.yandex.ru, 08.04.2021

Ржу и плачу. У меня, в Беларуси, 15 суток за это. У немцев, наверное, не
меньше. На \emph{Украине} какая то собственная Европа?
Игорь Правлоцкий, комментарий к видео, \textbf{Посадки
самолётов от Порошенко до Лукашенко. Что нужно понимать по скандалу с
задержанием Протасевича}, Олеся Медведева, strana.ua, 24.05.2021

Предположительно, президент Беларуси под территорией, \enquote{где угробили 300
человек} имеет в виду \emph{Украину} - 17 июля 2014 года, в небе над Донбассом был
сбит самолет, который следовал из Амстердама в Куала-Лумпур, strana.ua, 26.05.2021

\enquote{Кто платил подонку, который убивал людей в братской Украине?} -
отметил Лукашенко, \textbf{\enquote{Убивал людей в братской Украине}. Лукашенко
впервые прокомментировал задержание Протасевича}, strana.ua, 26.05.2021

Зеленский готовит закон о поражении \emph{Украины} в войне России, Владимир
Воля, strana.ua, 26.05.2021

В \emph{УССР} не было школы и садика, где дети не учили бы \emph{украинский}
фольклор, Андрей Манчук, strana.ua, 26.05.2021

Богдана була носієм дуже важливих цінностей. Не тільки журналістських, але
цінностей ідентифікації людини до \emph{українського}, до \emph{українських}
тем, до \emph{української} позиції, до \emph{українського} вільнодумства, з
точки зору несення своїх персональних цінностей гідности, \textbf{У Києві
презентували книгу спогадів про журналістку Радіо Свобода Богдану Костюк},
radiosvoboda.org, 26.05.2021

\emph{Украинец} установил рекорд, внедрив с свое тело восемь чипов. С их помощью он
заводит авто и открывает квартиру, strana.ua, 26.05.2021

Іван Франко – титан \emph{українського} слова і \emph{української} думки,
Михайло Цимбалюк, censor.net.ua, 26.05.2021

Ми мусимо навчитися чути себе \emph{українцями} — не галицькими, не
буковинськими \emph{українцями}, а \emph{українцями} без офіціальних кордонів.
І се почуття не повинно у нас бути голою фразою, а мусить вести за собою
практичні консеквенції. Ми повинні — всі без виїмка — поперед усього пізнати ту
свою \emph{Україну}, всю в її етнографічних межах, у її теперішнім культурнім
стані, познайомитися з її природними засобами та громадськими болячками і
засвоїти собі те знання твердо, до тої міри, щоб ми боліли кождим її частковим,
локальним болем і радувалися кождим хоч і як дрібним та частковим її успіхом, а
головно, щоб ми розуміли всі прояви її життя, щоб почували себе справді,
практично частиною його, \textbf{Одвертий лист до галицької молодежі}, Іван
Франко, 1905

Не забувайте, що ми досі в Галичині жили з національного погляду крайнє
ненормальним життям. Велика більшість нашої нації лежала безсильна,
закнебльована, а ми, маленька частина, мали свободу рухів і слова. І нам іноді
здавалося, що ми – вся \emph{українська} нація, що ми її чільні ряди, її
репрезентанти перед світом. Тепер, коли на російській \emph{Україні} не
сьогодні то завтра повстануть десятки таких центрів, якими тепер являються
Львів та Чернівці, ся наша передова роля скінчилася, \textbf{Одвертий лист до
галицької молодежі}, Іван Франко, 1905

Якщо це Русь, то Русь, а не \emph{Україна}. Якщо це Московія, то Московія, а не
Росія, \textbf{Життя нації. Використання топонімів Русь, Московія, Україна,
Росія}, zrada.org, 12.03.2021

\emph{Украина} в топ-5 стран мира по количеству сомнительных бот-сетей в Фейсбуке, 
strana.ua, 26.05.2021

Число смертей от коронавируса в \emph{Украине} перевалило за 50 000 человек с начала
эпидемии, strana.ua, 27.05.2021

Соцопросы регулярно показывают, что у нас процентов 70 считают - \emph{Украина}
движется в неправильном направлении. Причем, в какой-то момент этот процент
социологи фиксируют при всех президентах \emph{Украины}, при всех Кабминах и Верховных
Радах. То есть, \emph{украинцы} из раза в раз выбирают во власть людей, которыми потом
недовольны, \textbf{Протасевич нам важнее, чем состояние собственной экономики}, Владислав Михеев, 
strana.ua, 27.05.2021

Возмущаться бардаком на соседнем участке, подглядывая за соседом в дыру забора,
для \emph{украинцев} почему-то намного органичнее, чем навести порядок на своем
огороде, \textbf{Протасевич нам важнее, чем состояние собственной экономики}, Владислав Михеев, 
strana.ua, 27.05.2021

\emph{Україна} – не США, проте ми країна, яка воює і відстоює свою
незалежність. Це те, що притягує людей в армію, \textbf{Чому українські
військові йдуть з армії},  pravda.com.ua, 27.05.2021

Падение капитальных инвестиций в \emph{Украине} - просто беспрецедентное,
Александр Гончаров, strana.ua, 27.05.2021

14 травня в \emph{Україні} вперше відзначили День пам'яті \emph{українців}, які
рятували євреїв під час Другої світової війни, \textbf{«Через страх перед
німецькою владою»}, Андрій Усач, zaxid.net, 26.05.2021

\emph{Украина} не смогла нажиться на авиационной блокаде Беларуси, Михаил
Чаплыга, strana.ua, 27.05.2021

\enquote{Мне в голову приходит госпожа Меркель и ее походы за продуктами. Мне
хочется, чтобы \emph{Украина} была как Германия}, - добавила она, \textbf{\enquote{Не
может пройти 30 метров}. На вокзале Киева ради удобства Порошенко перегнали
поезд на другую колею}, strana.ua, 27.05.2021

Первые 224 \emph{украинца} получили уколы еще незарегистрированной в стране
вакцины Johnson \& Johnson, strana.ua, 27.05.2021

\emph{Пенсионер} получает в Украине в среднем 3500 грн в месяц, а распорядители
пенсий по всей \emph{Украине} по 1800 баксов.  Я уверен, что ни в Польше, ни в
Литве государственный клерк не получает таких зарплат. А сколько же тогда
получает директор департамента? А сколько трудяги центрального аппарата?, -
удивляется киевский предприниматель Павел Себастьянович, который на своей
странице в Фейсбуке поделился скрином декларации чиновницы, \textbf{Было 12
тысяч, стало 30. Как при Зеленском пошли в рост зарплаты чиновников, опережая
частный сектор}, strana.ua, 26.05.2021

Госдума России призвала мир осудить тотальную \emph{украинизацию}, strana.ua,
20.01.2021

Терористи вважають Лукашенка негідним. А 30\% \emph{українців} - найкращим
президентом, Сергій Фурса, gazeta.ua, 25.05.2021

На Донбассе снайпер террористов убил \emph{украинского} воина, obozrevatel.com,
27.05.2021

Папа Римский обеспокоен эскалацией насилия на востоке \emph{Украины},
strana.ua, 18.04.2021,

В этом законе сегодня вы закладываете очень серьезную бомбу против командиров
\emph{украинских} Вооруженных Сил. С большим трудом мы вытащили сержанта Маркива,
которого схватили, задержали за рубежом. По этому закону, который вы сейчас
проголосуете, могут задержать его командира роты, его командира батальона и
командующего Нацгвардии, \textbf{До 15 лет за участие в боевых действиях на Донбассе. Кого будут судить по закону о военных преступлениях}, strana.ua, 27.05.2021

Запад говорит \emph{Украине}: \enquote{Ребята, может, вы, наконец, приберетесь
дома}, hvylya.net, 27.05.2021

Сегодня Петр Порошенко напомнил, что он чемпион \emph{Украины} по неявке на
допросы, \textbf{Навстречу солнцу и прочь от допроса. Как Порошенко уехал от
СБУ в клубничный тур на Донбасс}, strana.ua, 27.05.2021

\emph{Украинский} политический пиар – бессмысленный, беспощадный и обязательно
неадекватный, - пишет журналист Вячеслав Чечило, \textbf{Навстречу солнцу и
прочь от допроса. Как Порошенко уехал от СБУ в клубничный тур на Донбасс},
strana.ua, 27.05.2021

Оно ведь как? Все «патриоты» \emph{Украины} жаждут призвать к ответу режим
Лукашенко, но чужими руками и за чужой счет. На что у союзного интуриста
бюджетов не предусмотрено, то с радостью оплатит \emph{украинский}
налогоплательщик. Хочет он того или нет.  Хорошо устроились, дорогие борцы!
\textbf{\emph{Украинский} патриот жаждет бороться с режимом Лукашенко - но чужими
руками}, Максим Могильницкий, strana.ua, 27.05.2021

\emph{Украина} - как феодальная Англия времен шерифа Ноттингема, только без Робин
Гуда, Алексей Кущ, strana.ua, 27.05.2021

Ще у 2017 році Верховна Рада звернулась до Конгресу США з проханням надати
\emph{Україні} статус основного союзника поза НАТО, але, як відомо, наша країна
цей статус так і не отримала. По – перше, США не схильні підписувати угоду, яка
би вимагала від них прийти на допомогу \emph{Україні}. По – друге, американська
еліта не хоче надавши цей статус \emph{Україні}, підштовхнути Росію до Китаю,
\textbf{Битва за \emph{Україну}. Хто винен, та що робити?}, Стефан Закревський,
hvylya.net, 27.05.2021

Битва за \emph{Україну}. Хто винен, та що робити? hvylya.net, 27.05.2021

Співробітники компанії з країни-агресора працюватимуть в \emph{Україні} «за
обміном», \textbf{«А як \emph{українською} «дружба?»: агенція, що розробила бренд
Ukraine NOW, привезла до Києва російських рекламників}, kyiv.media, 27.05.2021

В США расследуют вмешательство \emph{Украины} в президентские выборы-2020 на
стороне Трампа, strana.ua, 28.05.2021

\emph{Украинские} дипломаты сейчас пытаются помочь с возвращением \emph{украинцам}, которые
лечатся в Беларуси и \enquote{застряли} там из-за прекращения авиасообщения,
\textbf{Кулеба объяснил, почему Киев не отзывает посла из Минска после ареста
Протасевича}, strana.ua, 28.05.2021

Сытник заявил, что директора НАБУ должны выбирать иностранцы. «Для сохранения
институциональной независимости Национального антикоррупционного бюро
\emph{Украины}», \textbf{Послушать Сытника, так \emph{украинскому} народу
нельзя доверять государство}, Вячеслав Чечило, strana.ua, 28.05.2021

Олег Скрипка назвал \enquote{не \emph{украинцами}} тех \emph{украинских}
артистов, которые поют песни на русском, strana.ua, 28.05.2021

А есть те, что ездят в Россию. Пусть зарабатывают, но для меня они не являются
\emph{украинцами}, они просто резиденты \emph{Украины}, они не разговаривают на \emph{украинском},
они не находятся в мировоззрении \emph{украинском}, они не живут в \emph{украинском} мире.
Они живут в паттернах, в постсовках. Я иногда, может, неправильно говорил на
этот счет или не так интерпретировали мои слова, что таких людей нужно
изолировать, - сказал Скрипка, strana.ua, 28.05.2021

\enquote{Это резиденты \emph{Украины}, это не \emph{украинские} артисты. Это
люди, которые живут на этой территории, но они не \emph{украинцы}. Это чисто
моя личная точка зрения. Для меня этот вопрос не является вопросом. Люди тут
живут, но они являются частью русского мира}, - сказал Олег Скрипка, strana.ua,
28.05.2021

И этим заявлением Костржевский, как говорится, испортил всю малину премьеру.
Тот ведь рассказывал, какие прибыли \emph{Украина} получит после того, как весь мир
обложит авиасанкциями Белоруссию, а оказалось, что вместо гипотетических
прибылей мы уже имеем реальные убытки. То есть санкциями \emph{Украина} выстрелила
себе в ногу, strana.ua, 28.05.2021

Як може \emph{Україна} зупинити терористів, які захопили Раду ООН з прав
людини, lb.ua, 26.05.2021

\enquote{Зеленский имитирует Путина. Он считает Путина успешным и хочет быть таким же...
Мы видим параллели. Мы понимаем, куда идет наша страна и куда идет
\emph{Украина}: туда же}, - заявил российский главред, dialog.ua, 28.05.2021

\enquote{Медведчук - человек плоть от плоти} ру***ого мира... Свои три канала
(ZIK, NewsOne и \enquote{112 Украина} - ред.) он превратил в рупор Путина. Они
с утра до вечера занимались тем, что подрывали \emph{украинскую} власть,
dialog.ua, 28.05.2021

За перші чотири місяці 2021-го в \emph{Україні} сталося 126 підліткових
самогубств або спроб – Денісова, radiosvoboda.org, 28.05.2021

Як хочеться потрапити у світ без Русской общины \emph{Украины}, Русского блока
та їм подібних московських шавок.  Як хочеться почути живих Драгоманова та
Ключевського; тих людей, які займались історією як наукою, а не як повією,
\textbf{Забыть все = Забить на всё. Як хочеться усе забути}, zrada.org,
24.07.2010

Я не даремно говорив про те, що націонал-патріоти перетворилися на
\emph{український} різновид гаїтянських тонтон-макутів - секретну службу, що
тероризувала населення, сіючи жах і беззаконня (при цьому прості люди вірили,
що тонтон-макути - живі мерці, зомбі).  Схоже на те, що \emph{українські}
націоналісти почали активно переймати практику вуду (найбільше розповсюджену
саме на Гаїті).  Ляльку Путіна голками ще не тикають (покищо), але макет Кремля
вже спалили))), \emph{Украинские} националисты начали активно перенимать
практику вуду, Константин Бондаренко, strana.ua, 28.05.2021

Беларусь в ответ на введение запрета на авиасообщение, запретила поставки 95
бензина в \emph{Украину}. Теперь за тупость нашей власти как обычно заплатят
простые \emph{украинцы}, ведь повышение цен ударит именно по простым людям,
strana.ua, 28.05.2021

100 гривен в день за проезд в \emph{киевском} городском транспорте?  Это на
наших глазах становится суровой реальностью, Александр Карпец, strana.ua,
28.05.2021

\emph{Украина} - страна кривых зеркал и фискального каннибализма, Вячеслав
Черкашин, strana.ua, 28.05.2021

В \emph{Украине} явно готовятся схемы по массовой скупке земель, strana.ua,
29.05.2021

В \emph{Украину} возвращаются даже не девяностые, а Средневековье, Андрей
Манчук, strana.ua, 29.05.2021

Нет, это не девяностые - это гораздо глубже и хуже. \emph{Украинское} село
возвращается к архаическим корням, со всеми вытекающими отсюда последствиями,
которые будут все чаще напоминать о Средневековье, \textbf{В \emph{Украину}
возвращаются даже не девяностые, а Средневековье}, Андрей Манчук, strana.ua,
29.05.2021

Смелость, проявляемая в слабоумии и отваге, ведет к разрушению, Именно она
характерна для \emph{украинской} власти, Виктор Скаршевский, strana.ua, 29.05.2021

Зеленский в этой борьбе проиграет, потому что он абсолютно одинок в этом
эпическом противостоянии. Его никто не поддерживает в ТАКОЙ борьбе: ни Запад,
ни политические игроки, ни \emph{украинское} общество, \textbf{Порошенко -
живое олицетворение коррупции}, Андрей Головачев, strana.ua, 29.05.2021

Разложение силовой и правоохранительной системы ставит крест на каких либо
попытках реформировать \emph{Украину} в правовом и демократическом направлении, так
как приводит к узурпации власти Президентом, strana.ua, 29.05.2021

Позже \emph{украинца} обнаружили. Ему потребовалась медицинская помощь. Выяснилось,
что он долгое время провел на глубине 40 метров, а потом слишком резко всплыл
на поверхность. Это вызвало азотное отравление крови - Кессонную болезнь,
\textbf{Черный день на Красном море. Как в Египте погиб бывший глава
\emph{украинской} разведки и что о нем известно}, strana.ua, 28.05.2021

Сначала спасатели забили тревогу, когда \emph{украинец} вовремя не поднялся на
поверхность, а когда он появился, оказалось, что ему нужна медицинская помощь.
Спасти его не удалось.  Собрали подробности гибели экс-главы \emph{украинской}
разведки, а также факты из его биографии, strana.ua, 28.05.2021

Если посмотреть на карту \emph{Украины}, то, найдя на ней Днепр, можно увидетьи
один из его притоков — реку Десну.  Любой человек, хоть сколько-нибудь знакомый
с азами географии, скажет, что Десна — левый приток Днепра. Но слово «десна» на
древнерусском означает «правая». А такое выражение, как «сесть одесную
хозяина», означало буквально сесть справа от хозяина.  В современном
сербохорватском языке — одном из близких родственников русского — «десна» также
означает «правая». Так почему же тогда Десна является левым притоком Днепра?
Дело в том, что восточные славяне, осваивая и называя эту реку, двигались вверх
по Днепру, а при этом Десна действительно располагалась справа, \textbf{Когда
лево находится справа?}, vogrugsveta.ru

Усі вже, здається, помітили скандал щодо «\emph{української} російської мови»,
а дехто навіть встиг взяти в ньому участь. Багатьох обурила сама постановка
питання. Мовляв, якась \emph{українська} російська мова? В \emph{Україні} не
може бути нічого російського. Ці люди роблять вигляд, що в природі існують так
звані «\emph{українські} етнічні землі» – чітко окреслена плюс-мінус до
кілометра територія, де здавна мешкає автохтонне \emph{україномовне} й
\emph{українокультурне} населення, тобто етнічні \emph{українці}. Всі інші на
цій території є чужинцями або навіть ще гірше – чужинцями-колонізаторами,
\textbf{\emph{Українська} російська цибуля} Павло Зуб'юк, zaxid.net, 15.04.2021

\emph{Украина} напоминает фильм про Ивана Васильевича, когда Жорж Милославский
угнал народ на войну, чтобы те не поняли, что у власти самозванцы,
\textbf{Анекдот дня (за \emph{Украину})}, Крымский кот, zen.yandex.ru,
27.05.2021

\emph{Украина} объявила, что откроет Чернобыль для туристов. Говорят, это как
Диснейленд, только двухметровая мышь - настоящая!  \textbf{Анекдот дня (за
\emph{Украину})}, Крымский кот, zen.yandex.ru, 27.05.2021

Мне всегда нравилось \emph{украинское} народное творчество. Нежный, ласковый
язык.  Песни на \emph{украинском} - сама нежность. Звучат всегда душевно и с
особым трепетом.  Ну, конечно же, зависит от исполнителя.  Пелагея, исполняющая
\enquote{Цвiте терен} - лучшее, что могло произойти.  Хочу поблагодарить за
такой потрясающий дуэт: очень рада за Веронику Сыромля. Молодец, девочка,
ангельский голос. В тандеме с ее наставницей - звучит до глубины души.
Пересматриваю который раз.  Эту же песню они вдвоем исполняли на открытии
Крымского моста в 2018 году. Дуэт Пелагеи и Вероники просто великолепен, очень
гармоничен, голоса красиво перекликаются, одинаковы в исполнении по силе и
чистоте, \textbf{\emph{Украинская} песня в исполнении Пелагеи}, КИНОТЕРАПИЯ,
zen.yandex.ru, 27.05.2021

В Киеве считали, что самоопределятся крымчане обособленно не имеют права и
точка. Именно тогда лидер УНСовцев Корчинский, изрёк ту саму фразу: «Крым будет
\emph{украинским} или безлюдным» — устраивая марши устрашения в Севастополе и
других городах, \textbf{Аннексия на аннексию?}, Дмитрий Жук, zen.yandex.ru,
05.05.2021

Аналогічно й у випадку СРСР. Фактично після закінчення експеременту з
коренізацією в СРСР утворилася негласна ієрархія народів, у якій українці були
«другі серед рівних». Якщо точніше – \emph{українці} були майже росіянами. Звісно,
\emph{українська} мова і культура розвивалися суто в межах дозволеного державою
невидимого ґетта, але пересічний етнічний \emph{українець} мав кар'єрні перспективи не
набагато гірші, ніж етнічний росіянин, 
\textbf{\emph{Українська} російська цибуля} Павло Зуб'юк, zaxid.net, 15.04.2021

\enquote{Проверяли канал для \enquote{взрослого} оружия}. Кто и зачем вез в
\emph{Украину} через Румынию тысячи револьверов, strana.ua, 29.05.2021

А по поводу заворота самолёта \emph{украинскими} покровителями \enquote{Азова}
и Протасевичей, закрытия ими целых изданий и каналов, убийства Бузины и угроз
писавшим об ультраправых журналистах -- все молчат, \textbf{История с
Протасевичем объемно показывает, кто чего стоит}, Роман Подолян, strana.ua,
29.05.2021

Современная \emph{Украина} — вполне состоявшееся государство, причем
состоявшееся на антироссийской основе. Оно сумело внести раскол в народ
\emph{Украины} и ещё попортит нам много крови и по Донбассу, и по Крыму, и по
многим другим вопросам. И за всё это \emph{Украине} надо будет не только
предъявлять претензии – за всё это её надо будет наказывать, \textbf{«Грехи»
1990-х и борьба за \emph{Украину}: Киев не должен уйти от ответственности},
regnum.ru, 19.05.2021

У России на \emph{украинском} направлении есть два разных, хотя и
взаимосвязанных дела: защита интересов страны перед лицом враждебной политики
нынешнего \emph{украинского} государства и «борьба за \emph{Украину}», о
которой тоже сказал депутат Затулин, \textbf{«Грехи» 1990-х и борьба за
\emph{Украину}: Киев не должен уйти от ответственности}, regnum.ru, 19.05.2021

Что же до современной \emph{Украины}, так это вполне состоявшееся государство,
оно состоялось на антироссийской основе, оно сумело внести нужный ему раскол в
народ \emph{Украины}, в среду православных на \emph{Украине}, оно ещё очень
много попортит нам крови и по Донбассу, и по Крыму, и по отношениям с Европой,
США, Турцией, государствами на пространстве бывшего СССР, \textbf{«Грехи»
1990-х и борьба за \emph{Украину}: Киев не должен уйти от ответственности},
regnum.ru, 19.05.2021

По поводу такого традиционного блюда, как борщ, власти \emph{Украины} сломали
немало копий, пытаясь представить его исключительно своим национальным
достоянием.  \emph{Украинский} МИД положил немало сил и энергии на протесты,
чтобы заклеймить позором издания или зарубежные рестораны, назвавшие его
русским. Тем не менее рецептов этого блюда сотни, и они даже в традиционной
\emph{украинской} кухне существенно отличаются в зависимости от региона. Есть
борщ полтавский и черниговский, полесский и винницкий.

С этой мыслью парни мечутся уже добрые два года. Как сделать так, чтобы СМИ
писали о власти или хорошо, или ничего? Сама формулировка намекает, что власть
в \emph{Украине} не совсем жива, но вопрос стоит именно так, \textbf{Проект
антиолигархического закона нацелен на удушение СМИ}, Сергей Лямец, strana.ua,
29.05.2021


Все происходящее в \emph{украино}-российских и \emph{украино}-белорусских
отношениях часто не имеет смысла. Ну, то есть, рационально объяснить эту череду
выстрелов себе в ногу невозможно. Однако, на удивление, все становится на свои
места, если принять за аксиому, что цель – максимальный отрыв \emph{Украины} от
«русского мира», \textbf{\emph{Украину} хотят любой ценой оторвать от русского
мира}, Вячеслав Чечило, strana.ua, 29.05.2021

В \emph{Украине} идет бесконечная борьба с ветряными мельницами, Павел Себастьянович,
strana.ua, 29.05.2021

Клімкін: Американський батальйон біля Одеси був би гарантією безпеки \emph{України}, 
radiosvoboda.org, 29.05.2021

Столица отмечает день рождения. За свою тысячелетнюю историю город повидал
немало. В разное время Киевом правили Рюриковичи, татары, литовцы, поляки,
советы. Но все равно город остался истинно \emph{украинским}. Шарм улиц, уют
двориков и даже царь-балконы делают Киев неповторимым. Богатая история оставила
для нас многочисленные памятники архитектуры, культуры и природы, obozrevatel.com, 29.05.2021

Але чому в нашому суспільстві немає культу успішного бізнесмена, який створює
геніальний продукт? Доблесного військового, який героїчно захищає
\emph{Україну}?  Просунутого хіміка, який робить приголомшливі відкриття у
найсучаснішій лабораторії? Актора, кумира мільйонів, який заробляє небосяжні
суми? \textbf{Запитання від дитини}, Андрій Любка, day.kiev.ua, 28.05.2021

Брехню про участь Протасевича у боях на Донбасі як «бойовика-найманця» одним із
перших почав поширювати \emph{український} блогер Анатолій Шарій, надавши як
доказ обкладинку часопису «Чорне сонце» зі світлиною молодого усміхненого
чоловіка у формі батальйону «Азов» з автоматом, \textbf{Тріумф негідників},
day.kiev.ua, 27.05.2021

Сегодняшние реалии российско-\emph{украинских} отношений совсем не похожи на
содружество, а современная \emph{Украина} не совсем похожа на независимое
государство и внутренне расколота. В таких обстоятельствах договориться о
содружестве практически невозможно. \enquote{Переформатирование} \emph{Украины}
и её населения продолжается с большим усердием. Каковы же будут результаты,
каким будет наше будущее? - \textbf{Россия и \emph{Украина} - утраченное
Содружество}, Igor Novikov, zen.yandex.ru, 29.05.2021

И новая киевская власть, и местные \emph{украинские} силовики заняли тогда
странную выжидательную позицию. Чего же они ждали? Фактически местным и
приезжим революционерам предоставлялась свобода действий по принуждению всех
несогласных с ними крымчан к безоговорочному повиновению, любыми способами.
Военнослужащие \emph{Украины}, обязанные защищать \emph{украинских} граждан,
расслабленно наблюдали за происходящим из-за ограды своих расположений. Вот в
таком состоянии их и застали внезапно появившиеся россияне... \textbf{Как
\emph{Украина} свой Крым не защитила}, Igor Novikov, zen.yandex.ru, 18.03.2021

И почему-то именно жители современной \emph{Украины} умудряются преподносить
сюрпризы.  Помните такую группу как «Вопли Видоплясова»? Должны бы помнить. Как
минимум у них был один большой хит, который крутили везде. Называлась та песня
«Весна» и в своё время она изрядно пошумела. Хотя поколение постарше может
вспомнить и песню «Танцы». Собственно, на мой взгляд «ВВ» первая группа,
которая популяризировала \emph{украинский} язык на просторах России. Хотя
играть они и вовсе начинали ещё до того как \emph{Украина} с Россией отделились
границами, \textbf{Олег Скрипка. Человек, забывший родину}, soullaway
soullaway, zen.yandex.ru, 24.05.2021

«\emph{Украинцы} ждут неизбежного — когда \emph{Украина} распадется», Аннотация: В сети
обсудили заявление депутата \emph{украинского} парламента Нестора Шуфрича о том, что
за все санкции \emph{украинской} власти в отношении Белоруссии поплатится простой
\emph{украинский} народ, regnum.ru, 29.05.2021

\emph{Український} Стоунхендж. На Дніпропетровщині поряд із майбутньою забудовою
знайшли стародавній кромлех: що з ним буде? radiosvoboda.org, 29.05.2021

Так что \emph{украинцам} просто необходимо учиться быть гражданами одной
страны, \textbf{Восстановить единство \emph{украинской} нации будет непросто},
Олесь Доний, glavred.info, 23.11.2019

\emph{Украинское} общество взбудоражили слова Сивохо о том, что нужно просить
прощение у жителей Донбасса. Данное утверждение вполне могло понравиться части
людей, живущих на оккупированных территориях, а также части тех, кто покинул
Донбасс, как вынужденный переселенец, но не получил ожидаемого тепла и внимания
со стороны государства, \textbf{Восстановить единство \emph{украинской} нации
будет непросто}, Олесь Доний, glavred.info, 23.11.2019

Эта истина, которая для кого-то звучит отвлечённо, для нас, живущих на
\emph{Украине}, составляет часть нашей жизни. \emph{Украинский} церковный
раскол, углублённый и благословлённый патриархом Варфоломеем в 2018 году
продолжает приносить горькие, преступные плоды, \textbf{Канонический бунт
Фанара: время мягких решений церкви заканчивается}, odnarodyna.org, 26.05.2021

Поёшь на русском? Ты не \emph{украинец}! \emph{Украинский} музыкант, фронтмен
известной группы «Вопли Видоплясова» Олег Скрипка назвал артистов
\emph{Украины}, которые поют песни на русском языке «не \emph{украинцами}».
Настоящие \emph{украинские} певцы – это те, кто, во-первых, говорят, пишут
песни и поют на мове, а, во-вторых, не ездят в Россию. Ты можешь сколько угодно
любить свою страну \emph{Украину}, но Олег Скрипка не будет тебя считать
\emph{украинским} певцом. О музыкальных талантах, кстати, речь не идёт,
\textbf{Олег Скрипка: \emph{украинский} певец не может петь по-русски},
odnarodyna.org, 29.05.2021

Голая девушка попала в кадр в прямом эфире \emph{украинского} канала во время
включения редактора \enquote{Дождя}. Видео, strana.ua, 29.05.2021

«Кремль спотворює історію регіону, щоб легітимізувати ідею про те, що
\emph{Україна} і Білорусь є частиною «природної» сфери впливу Росії, –
розвінчують путінський міф в Chatham House. – Історично неправильно
стверджувати, що Росія, \emph{Україна} і Білорусь коли-небудь складали єдине
національне утворення. Останні дві країни насправді мають політичні та
культурні корені в європейських за своєю суттю структурах, таких як Велике
князівство Литовське», radiosvoboda.org, 30.05.2021

Верю, что миро-творческий и воссоединительный процессы в \emph{Украине} еще
впереди. И нынешний дурдом, замешанный на травмах войны, страхах, эгоизме,
близорукости и безответственности, мы переживем достойно, \textbf{Мир и
воссоединение с Донбассом еще впереди}, Андрей Ермолаев, strana.ua, 30.05.2021

Биткоин потребляет в год больше электроэнергии, чем \emph{Украина}, Анатолий
Амелин, strana.ua, 30.05.2021

У добу Русі-\emph{України}, за середньовіччя графіті сприймали як безпосереднє
звернення до святих і до Всевишнього по допомогу, radiosvoboda.org

Вот такая сегодня у нас тяжелейшая ситуация не только на долговом рынке, а и в
целом в экономике \emph{Украины}. И надо честно признать: благополучие рядовых
налогоплательщиков тает буквально на глазах, Александр Гончаров, strana.ua,
30.05.2021

Михайло Слабошпицький – член Національної спілки письменників \emph{України},
лауреат національної премії імені Тараса Шевченка (2005), за роман-біографію
«Поет із пекла». Серед його робіт – документальна, публіцистична та біографічна
проза, а також численні твори для дітей. Його син Мирослав є відомим
\emph{українським} кінорежисером, дочка Іванна – журналісткою,
radiosvoboda.org, 30.05.2021

\emph{Украинский} сериал: почему у них каша в голове, Сериал \enquote{Папик}
наглядно показывает, почему у \emph{украинцев} каша в голове. И это не вопрос -
это утверждение, ЗВЕЗДУЛЬКИ, zen.yandex.ru, 01.05.2021

Однако, слов из песни не выкинешь и мы, русские, глядя на \emph{Украину} и
\emph{украинцев} всё чаще приходим к выводу, что там люди словно с ума
посходили, ЗВЕЗДУЛЬКИ, zen.yandex.ru, 01.05.2021

Некоторые в счастливом экстазе ежегодно празднуют день независимости
\emph{Украины}, другие воспринимают этот день как личную и общественную
трагедию, лишившую их очень многого, \textbf{О независимости \emph{Украины}:
как не обмануться в мире обмана}, ukraina.ru, 29.05.2021

Це европа, безвиз, безгаз, безбензин, безвакцин, беззавод, безземель, безнадёг,
бездонбасс, безкрым, безбожие...\footnote{Украина} Sergey Stus, комментарий,
\textbf{Потерять яйца в сражении с Беларусью}, Анатолий Шарий, youtube,
30.05.2021 

Живу в \emph{Украине}, рад искренне за Бацьку который не зассал и не дал
заднюю, и за санкции в нашу сторону, тем больше \emph{украинцев} прозреет от
ничтожности нашей конченной власти, и быстрее мы их сметем, Дмитрий СМ,
комментарий, \textbf{Потерять яйца в сражении с Беларусью}, Анатолий Шарий,
youtube, 30.05.2021

Я видел много идиотов, и слышал разных чудаков. Но столько сколько \enquote{в}
\emph{Украине}, ещё не видел дураков. В долгах погрязнув как в болоте,
обворовав самих себя. Вопят, орут всё о свободе, в своих грехах других виня!
комментарий, \textbf{Потерять яйца в сражении с Беларусью}, Анатолий Шарий,
youtube, 30.05.2021

Белорусы, если вы читаете мой коммент знайте: мы адекватные \emph{украинцы}
против этого гребанного маразма который творит наша вонючая, наркоманская
власть!  комментарий, \textbf{Потерять яйца в сражении с Беларусью}, Анатолий
Шарий, youtube, 30.05.2021

Только великие мастера школы \emph{Майданлинь} способны ударить сами себя по
яйцам пяткой, комментарий, \textbf{Потерять яйца в сражении с Беларусью},
Анатолий Шарий, youtube, 30.05.2021

\emph{Украинская} власть давно не имеет яиц, у нее только рабочая дупа,
комментарий, \textbf{Потерять яйца в сражении с Беларусью}, Анатолий Шарий,
youtube, 30.05.2021

Что такое жизнь под \emph{украинскими} обстрелами? Это когда вечером 1 июня, в
День защиты детей, у мемориалов погибшим детям Донбасса в небо взмывают сотни
бумажных фонарей, чтобы осветить путь ангелам. Ведь малышам, запускающим
фонарики, сложно объяснить, почему у этих ангелов отняли их короткую жизнь,
лишив возможности повзрослеть и увидеть мир на нашей Родине. Теперь они могут
только наблюдать с небес и плакать вместе со взрослыми, утешая их, Фаина
Савенкова, facebook.com, 30.05.2021

На \enquote{двери} расположен QR-код, который при переходе ведет на фото с
обрисованным ранее Офисом президента и надписью \enquote{Президент
\emph{Украины} лох}, strana.ua, 30.05.2021

\emph{Украине} необходима деконструкция национального мифа – чтобы строить свое
будущее не на архаической полотняной сорочке и не на «исконно
\emph{украинском}» борще – а на развитии общедоступного образования, науки и
высокотехничного производства, \textbf{Косоворотка и вышиванка должны сближать,
а не разделять народы}, Андрей Манчук, strana.ua, 30.05.2021

«Страна»: популярне і проросійське медіа в \emph{Україні}, radiosvoboda.org,
30.05.2021

Новинний вебсайт «Страна.ua» – серед найпопулярніших медіа в \emph{Україні},
radiosvoboda.org, 30.05.2021

Співаючий далекобійник. Як \emph{український} хорист став противником Путіна,
зіркою YouTube та потрапив в Канни, pravda.com.ua, 28.05.2021

В'ятрович попереджає про наміри влади скасувати норми про \emph{українську} мову
фільмів і преси, umoloda.kyiv.ua, 29.05.2021

\enquote{А от для Зеленського і проросійських олігархів – власників телеканалів
– це замах на \enquote{святе}. Адже серіальчики і фільми в телевізорі мають
бути на общєпонятном язикє, а \emph{українська} мова, цитуючи скандальну
продюсерку \enquote{1+1} годится только для комедий}, - додає обранець,
umoloda.kyiv.ua, 29.05.2021

На його думку, \enquote{Зеленський і його слуги вирішили реалізувати одну з
головних цілей Москви у гібридній війні проти \emph{України} – забетонувати
русифікацію \emph{українського} медіапростору}.  \enquote{Кожен, хто голосуватиме за
ці анти\emph{українські} закони, відверто працює на Кремль, якими б байками про
\enquote{збитки від \emph{української} мови} це не прикривалося. Наш обов'язок –
зупинити цей нахабний російський реванш}, – завершив В'ятрович,
umoloda.kyiv.ua, 29.05.2021

\emph{Украине} снова спокойно не спится. Подавай им теперь изменения в правилах
голосования на Евровидении. Естественно дело в России. На \emph{Украине}
вообще нет других причин как наша страна и еë граждане.  В этот раз их тонкая
натура не может смириться с тем фактом, что на международном конкурсе они
вынуждены оценивать участницу из страны-агрессора. Им это кажется странным,
\textbf{Евровидение для \emph{Украины} не просто конкурс}, Мила Белая,
zen.yandex.ru, 24.05.2021

Между прочим, политика политикой, а выступление \emph{Украины} мне понравилось.
Уж точно больше Манижи. Интересная яркая исполнительница, хороший голос,
постановка номера интересная, музыка запоминающаяся.  \enquote{Я считаю
аморальным оценивать что-либо и кого-либо, кто представляет страну-агрессора!
Для меня лично агрессор всегда на последнем месте независимо от вида
соревнования!} – член \emph{украинского} жюри Игорь Кондратюк. А как же
искусство и спорт вне политики? Разве такие мероприятия не призваны
продемонстрировать единство и перемирие. Для чего они ещё созданы?
\textbf{Евровидение для \emph{Украины} не просто конкурс}, Мила Белая,
zen.yandex.ru, 24.05.2021

Є \emph{українці}, які чекають \enquote{Спутник V}. Що з ними робити – ребус
для Ляшка, Сергій Грабовський, gazeta.ua, 27.05.2021

А є ж іще у YouTube т.зв. «Перший козацький»; як слушно зауважив один
журналіст, «Телеканал «Інтер» у порівнянні з «Першим козацьким» – це просто
вісник \emph{українського} націоналіста»... Звичайно, чинна влада не господарює у
YouTube. Але виникають запитання. Скажімо: чи звертався Офіс президента чи
якась інша інстанція до адміністрації YouTube з проханням прикрити «Перший
незалежний» чи бодай припинити його трансляцію під марками трьох уже
заблокованих для \emph{українського} користувача телеканалів? Адже, як на мене,
йдеться про пряме знущання не лише з президента Зеленського та РНБО, а й із
\emph{Української} держави. Чи, може, зверталися, проте якось ніяково, без
наполегливості? \textbf{«Заблоковані» телеканали та антидержавна пропаганда}, day.kiev.ua, 18.05.2021

Широкий і вольний був їм шлях на \emph{Україну}. Літня спека застелила його на
долоню курявою. Сонце пекло з гарячого неба. Курява посіла на семінаристів,
обліпила їм лиця так, що вони не впізнавали один одного. Піт котився з їх
потьоками і, помочивши чорну куряву, пописав їх лиця довгими смужками,
\textbf{Хмари}, Іван Нечуй-Левицький

Сніг за шиворот, в Карпати ми летим, На літній резині і майже без бензини, І
весело, бо я тут не один — На лижі їде ціла \emph{Україна}!  \textbf{Буковель},
Кузьма Скрябін, Повне зібрання творів, Харків, 2019

Але, як і революціонери, що використовували нову форму Інтернету для об'єднання
та усунення ворога, Росія тепер використовувала мережі, щоб розірвати
\emph{Україну} на частини, \textbf{Війна лайків. Зброя в руках соціальних
мереж}, П. В.  Сінґер, Емерсон Т. Брукінґ, Харків, 2019

Важливим прецедентом цього стала \emph{Україна}. Кількість негативних
російськомовних новин про \emph{Україну} зросла вдвічі, а потім утричі. Етнічні
росіяни всередині \emph{України} невдовзі збурилися проти активістів, що
скинули проросійський уряд, \textbf{Війна лайків. Зброя в руках соціальних
мереж}, П. В. Сінґер, Емерсон Т. Брукінґ, Харків, 2019

Зеленский создает в \emph{Украине} \enquote{университет будущего} для \enquote{людей
будущего}, strana.ua, 01.06.2021

Зеленский предложил превратить \emph{украинские} киностудии в музеи, strana.ua,
28.05.2021

Рост ВВП в \emph{Украине} за последние десятилетия унизительно низок,
strana.ua, 31.05.2021

В \emph{украинском} магазине продают книгу участника ОУН о том, почему евреев следует
называть ж@дами, strana.ua, 07.02.2021

А мне вот даже чуточку жаль авторов подобной макулатуры. Ведь чушь про «дело
своего народа» они выдают на-гора искренне. И сами рады бы послужить, да только
народа у них нет. Немецкий этим недугом давно переболел и на «расово верных»
бумагомарателей из \emph{Украины} глядит с отвращением. Как на прокаженных
глядит и ничего общего с такими иметь не желает, \textbf{Несчастные мерзавцы,
эти авторы книжек во славу героев СС}, Максим Могильницкий, strana.ua,
31.05.2021

Но \emph{украинский} пример говорит об обратном. Страна, в которой господствует
идеология \emph{украинского} национализма (идеология же), катится в пропасть с
той же скоростью, с которой Россия рвётся к звёздам, \textbf{Непойманный
украинский крокодил}, Ростислав Ищенко, ukraina.ru, 31.05.2021

Часть, даже очень важная, не может жить и развиваться без целого. А целое, даже
без очень важной части, может функционировать, зачастую не менее эффективно,
чем с ней. \emph{Украина} была очень важной частью России, не менее, а
возможно, и более важной, чем ноги для лётчика-истребителя. И на этом основании
решила, что она способна одна не просто выжить, но жить лучше, чем вместе,
\textbf{Непойманный украинский крокодил}, Ростислав Ищенко, ukraina.ru,
31.05.2021

Вы, звезда, Анатолий. Главный человек в \emph{Украине}. Пиарят, вас, столько говорят,
не забывают. Это здорово. Мы с вами. Шарий супер!
комментарий, \textbf{Опять Шария достали. Что дальше?} Анатолий Шарий, youtube.com, 31.05.2021

Гончаренку снова пошел в одно место. Удачи Шарийцам, потому что с этим
беспределом нужно заканчивать.  Я переживаю за нормальных \emph{украинцев}, они ни в
чем не виноваты,
комментарий, \textbf{Опять Шария достали. Что дальше?} Анатолий Шарий, youtube.com, 31.05.2021

Мне вот интересно: а здоровые в \emph{украинском} политикуме остались?
комментарий, \textbf{Опять Шария достали. Что дальше?} Анатолий Шарий, youtube.com, 31.05.2021

\emph{Украина} - равняйся на Шария! Анатолий, ты ЗНАМЕНИТ на Родине,
комментарий, \textbf{Опять Шария достали. Что дальше?} Анатолий Шарий, youtube.com, 31.05.2021

\begin{itemize}
\item Мы отключили Крыму свет и воду, чтобы вернуть доверие крымчан. 
\item Мы обстреливаем Донбасс, чтобы они увидели, что \emph{Украина} за мир. 
\item Мы закрываем телеканалы, чтобы все знали,что у нас свобода слова. 
\item Мы проводим факельные шествия со свастикой на знаменах, чтобы все увидели, что фашистов у нас нет.
\item Мы не сажаем убийц, чтобы все знали, что на \emph{Украине} есть правосудие.
\item Мы запретили русский язык и школы, чтобы все увидели, что \emph{Украина} едина.
\item Мы узурпировали власть, чтобы все увидели что Янукович узурпатор.
\end{itemize}
комментарий,  Olga Vpalto, \textbf{Опять Шария достали. Что дальше?} Анатолий Шарий, youtube.com, 31.05.2021

а еще - голодомор - это геноцид \emph{украинцев}, а 400 трупов в день из-за \enquote{ми ведемо
перемовини} - это норма. Ты, сука, должен любить \emph{Украину}! Иначе, чемодан,
вокзал... Польша,
комментарий, Александр, \textbf{Опять Шария достали. Что дальше?} Анатолий Шарий, youtube.com, 31.05.2021

\emph{Украинцы} вас всех устраивает происходящие в стране?? Может пора импичмент?
комментарий, \textbf{Опять Шария достали. Что дальше?} Анатолий Шарий, youtube.com, 31.05.2021

Представьте, что Анатолий, нелегально проехался по \emph{Украине}, наснимал роликов и все это выложил в сети. Кондрашка хватила бы многих:))),
комментарий, \textbf{Опять Шария достали. Что дальше?} Анатолий Шарий, youtube.com, 31.05.2021

Та поезда похоже \emph{Украине}, если действительно за зе 20\% проголосовали бы...... Я
думаю за 3года зелька не то еще придумает, как собственный народ обобрать,
Тимошенко Юлька уже что-то там проорала, про то шо пенсии хотят отобрать....
Нууу терпите дальше, молодцы \emph{Украины}, в правильном направлении движитесь,
комментарий, \textbf{Опять Шария достали. Что дальше?} Анатолий Шарий, youtube.com, 31.05.2021

Толик, мы ждем тебя дома, в \emph{Украине}! Хватить любить Родину в Европе,
комментарий, \textbf{Опять Шария достали. Что дальше?} Анатолий Шарий, youtube.com, 31.05.2021

Практичний досвід \emph{України} у боротьбі із систематичними російськими
пропагандистськими та дезінформаційними наративами бере свій початок у 2013
році, і за цей час \emph{українські} експерти, неурядові та державні установи
розробили низку інструментів та стратегій для подолання цих викликів,
radiosvoboda.org, \textbf{Російська підривна діяльність, Білорусь та Крим –
серед тем другої зустрічі \emph{Українсько}–чеського форуму}, 31.05.2021

Чеські та \emph{українські} історики працюють разом, щоб зробити доступними
\emph{українські} архіви. Наприклад, чеські історики отримали доступ до
\emph{українських} архівів КҐБ (на відміну від подібних архівів у Росії), що
дозволило виявити нові та невідомі факти про те, що сталося в Чехословаччині
після 1945 року, наприклад, з громадянами Чехословаччини, викраденими в СРСР, 
radiosvoboda.org, \textbf{Російська підривна діяльність, Білорусь та Крим –
серед тем другої зустрічі \emph{Українсько}–чеського форуму}, 31.05.2021

Пенсионерка в Первомайске подорвалась на \emph{украинском} снаряде во время работы на огороде,
miaistok.su, 31.05.2021

Опять рост после снижения. Новых зараженных в \emph{Украине} за сутки стало
больше на 2 137 человек, strana.ua, 01.06.2021

Нас плавно превращают в сырьевую колонию без науки, технологий и индустрии, Так
происходит мягкий захват Украины, Александр Гончаров, strana.ua, 01.06.2021

\enquote{Бандеровец} - \emph{украинский} нацистский коллаборационист. Как
Netflix перевёл цитаты из \enquote{Брата} и \enquote{Брата-2}, strana.ua,
01.06.2021

Выкладываю кусок стенограммы нашей беседы с Вадим Аристов. Он дал яркую и
доступную картинку о том, что врачи уже поняли о ковиде и почему \emph{Украине} грозят
локдауны до 2023 года, \textbf{\emph{Украине} нужно понять, что ей грозят локдауны до 2023 года}, strana.ua, 01.06.2021

Интервью Зеленского немецкому изданию - жалобная просьба о помощи, У него все
виноваты, и только \emph{Украина} стеной стоит за Европу! ... все в мире должны
защищать \emph{Украину}, но они какие-то хрупкие (ЕС, НАТО), не вооружают
\emph{Украину}, не отказываются от Северного потока-2 и даже не признают Россию
стороной конфликта ... \emph{Украина} защищает, \emph{Украина} – стена
безопасности этой философии существования, европейской цивилизации, прав и
свобод человека;,  Виктор Скаршевский, strana.ua, 01.06.2021

Третьи в мире по детской порнографии и рост изнасилований. Как в \emph{Украине}
\enquote{защищают детей}, strana.ua, 01.06.2021

Сотни тысяч для главы набсовета, 8 тысяч - для инженера. Какие зарплаты платят
на \enquote{\emph{Укр}зализныце}, strana.ua, 01.06.2021

Нажали на кулёк. Как в \emph{Украине} запретили пластиковые пакеты и кого за
них оштрафуют, strana.ua, 01.06.2021

За предыдущие несколько лет команда Нафтогаза добилась почти невозможного -
значительной отсрочки достройки \enquote{Северного потока-2}. Однако до победы еще
далеко и бой \emph{Украины} против российской геополитической трубы не завершен,
\textbf{Галантерейщик и кардинал. Сцена третья.  Апофеоз абсурда}, strana.ua, Валентин Землянский, 01.06.2021

В \emph{Україні}, станом на 2019 рік, діяли 282 вищих навчальних заклади, де здобували освіту 1 300 000 студентів, 
\textbf{Щодо вчорашнього указу про \enquote{Президентський університет}}, Костянтин Матвієнко, pravda.com.ua, 01.06.2021

Адже молодь залишає \emph{Україну} не лише тому, що їй нерідко пропонують за кордоном
якісніше і безкоштовне навчання, але і кращі життєві перспективи, що
ґрунтуються на ліпших за \emph{українські} суспільно-державних взаєминах.  Наразі, ані
Академія Наук, ані окремі визначні \emph{українські} вчені не наважилися прямо вказати
на волюнтаристський та безперспективний підхід у питанні заснування нового
університету. Може б президенту було б правильним, перш, ніж підписувати явно
сирий Указ, зустрітися з науковцями задля консультацій?
\textbf{Щодо вчорашнього указу про \enquote{Президентський університет}}, Костянтин Матвієнко, pravda.com.ua, 01.06.2021

Ось вже два роки Зелеенський є Президентом \emph{України}...і пробуде ще принаймі 3
роки...Бо вірогідно ще виберуть його ЛЮДИ на 5 років... Погані, скажете,
люди... А Ви що краще? Хто з критиків має суспільний високий рейтинг довіри в
поваги в суспільстві. Ніхто в т.ч і К Матвієнко...Чому критика? Якби то були
комуністи сказав, що ідеологія...А так або жаба даве або втратили гроші або
заробляють гроші, як Маітвієнко... Щодо РІШЕННЯ Президента... Має право...Хто
хоче теж мати право, то йдіть на вибори - хай Вас виберуть...І будете свій ВИШ
влаштовувати...або виливати золотий батон...
коментар, Віктор Совщак, \textbf{Щодо вчорашнього указу про \enquote{Президентський університет}}, 
Костянтин Матвієнко, pravda.com.ua, 01.06.2021

\enquote{Революція вихідного дня}, як іронічно називають численні протести у Російській
Федерації, позатим викликала великий інтерес в \emph{українському} експертному
середовищі, та й у суспільстві загалом. Це перед усім зумовлено війною, яку
росіяни розв'язали проти \emph{України} 7 років тому. Зміна влади у країні-агресорі
теоретично може мати своїм наслідком припинення війни і відновлення
територіальної цілісності нашої країни у її законних кордонах, визнаних
світовим співтовариством. Однак, зараз є цілком доречним згадати вираз
Володимира Винниченка про те, що \enquote{вся російська демократія, включно з
опозицією, закінчується на \emph{українському} питанні}. Тому за територіальну
цілісність України та невтручання росіян в \emph{українські} справи ще доведеться
поборотися, як з нинішньою, так і з наступною російською владою,
\textbf{Знову \enquote{Україна не \emph{Росія}}, проте багатьом дуже хочеться стерти цю відмінність},
Костянтин Матвієнко, pravda.com.ua, 02.02.2021

Докорінною відмінністю \emph{українських} революцій від, того, що нині бачимо у
сусідів, полягає у неодмінній наявності парламентської опозиції, яка мала
високий рівень політичної та організаційної спроможності, доступ до ЗМІ, та, що
дуже важливо, була наділена парламентським імунітетом, без чого ніякі намети на
Майдані Незалежності встановити було б неможливо,
\textbf{Знову \enquote{Україна не \emph{Росія}}, проте багатьом дуже хочеться стерти цю відмінність},
Костянтин Матвієнко, pravda.com.ua, 02.02.2021

Хотите разобраться с этим вопросом?:) Тогда сначала нужно определение война
кого с кем? Война России с \emph{Украиной}? нет, ее нет. И опять же, нужно
определиться с тем, что вкладывается в \enquote{\emph{Украина}} и
\enquote{Россия}. Если под \enquote{\emph{Украина}} понимать текущую и
предыдущую власть (полностью управляемую из-за океана), то война есть. А если
страну, людей, и в моем понимании будущее этих людей, то ее не может быть в
принципе.  Схематично так – власть захвачена иностранной державой, которая
развязала конфликт с Россией (текущей властью) с целью ее ослабления и
перемещения поближе границ НАТО.  Какова должна быть позиция гражданина
\emph{Украины}? углублять конфликт, положившись на дядю Сэма? или подумать а во имя
чего граждан власть посылает умирать? вспомните Вьетнам. И нашу олигархию:)
коментар, Mike\_Kharkov, \textbf{Знову \enquote{Україна не \emph{Росія}}, проте багатьом дуже хочеться стерти цю відмінність},
Костянтин Матвієнко, pravda.com.ua, 02.02.2021

\begin{itemize}
\item Как-то жалко такое читать, с обязательными \enquote{решениями съезда} в
        предисловии (\enquote{зумовлено війною, яку росіяни розв'язали проти України 7
        років тому}), клеймлением империалистических агрессоров и их
        пособников, о роли партии и лично (ЮВТ), о великом народе (\enquote{бачимо
        кардинально вищий рівень зрілості тодішнього українського суспільства у
        порівнянні з нинішнім російським}) и не прикрытое ничем тут видим, а
        тут не видим.
\item Можно и покороче – \enquote{Відіграв свою важливу роль доброзичливий до
        революціонерів нейтралітет олігархів – мобільні оператори не вимикали
        телефонного зв'язку та інтернету в київському середмісті в ході
        революцій, так само працювали банкомати та інші сервіси.} Этого
        достаточно, и для полной честности бы добавить про сговор олигархии с
        иностр. державами для взаимной пользы от доения населения, названный
        революциями. И опять ассоциации с пафосом 1917. Не было штурма Зимнего,
        ребята, не было. Но опять пишут учебники.
\end{itemize}
коментар, Mike\_Kharkov, \textbf{Знову \enquote{Україна не \emph{Росія}}, проте багатьом дуже хочеться стерти цю відмінність},
Костянтин Матвієнко, pravda.com.ua, 02.02.2021

Що чекає на охорону здоров'я в \emph{Україні} у 2018 році, або як з'їсти слона? 
Тимофій Бадіков, pravda.com.ua, 11.01.2018

Із початком нового року в \emph{Україні} офіційно розпочалася медична реформа.
Напередодні законодавчий старт змінам в охороні здоров'я дали Президент і Уряд.
При цьому, всі ми добре розуміємо, що \emph{українська} медицина не зміниться за одну
мить, тільки за підписами посадових осіб. Зміни в системі охорони здоров'я
\emph{України} – це поетапний комплекс дій та завдань для всіх сторін. Від
злагодженості їхньої роботи, від наявності стратегії та вміння реагувати на
виклики, та, насамперед, від нас самих – громадськості, залежить, чи досягне
\emph{українська} медицина європейського рівня, чи продовжить падіння у прірву,
Що чекає на охорону здоров'я в \emph{Україні} у 2018 році, або як з'їсти слона? 
Тимофій Бадіков, pravda.com.ua, 11.01.2018

Для \emph{української} системи охорони здоров'я 2018 рік обіцяє бути цікавим та насиченим подіями,
\textbf{Що чекає на охорону здоров'я в \emph{Україні} у 2018 році, або як з'їсти слона?} 
Тимофій Бадіков, pravda.com.ua, 11.01.2018

Цілком у тренді явного тепер союзу політиків з колишньої Партії регіонів та
Блоку Петра Порошенка і Народного фронту є вчорашнє рішення Шевченківського
районного суду Києва, яким відмовлено юридично встановити факт агресії Росії
проти \emph{України},
\textbf{Роби те, що слід, і нехай станеться те, що статися має!}
Костянтин Матвієнко, pravda.com.ua, 13.05.2016

\begin{itemize}
\item Минуло понад 30 років відтоді, як \emph{українська} стала державною, та
        тиждень від запровадження обов'язкової \emph{української} мови у сфері
        послуг. Ресторани, спортзали та магазини по всій країні навіть у
        соцмережах починають писати рекламні тексти \emph{українською}. Вата ж
        – іще більше горлати по Медведчуківським каналам про
        \enquote{нелюдські} утиски \enquote{корінного} російськомовного
        населення.
\item Які проблеми сьогодні існують у російської мови? Ніяких. Взагалі. Мені їх навіть вигадати складно, щоб звучало правдоподібно.
\item Які проблеми сьогодні існують в \emph{української} мови? Чимало, як на
державну. Досі у нас немає якісного культурного продукту, який збирав
би більше аудиторії, ніж російський.
\end{itemize}
\textbf{Чи є різниця, якою мовою говорити?} Андрій Білецький, pravda.com.ua, 22.01.2021

Вата постійно говорить, що захищати \emph{українську} не потрібно тому, що російська
мова з'явилася внаслідок \enquote{конкуренції}.  Ніби \emph{українці} самі
\enquote{обрали} її як свою рідну, як президента-клоуна у 2019 році. А тепер їй
для чогось потрібен \enquote{захист}.  Та пригадаймо шкільний курс історії –
російську на \emph{Українських} землях захищали найбільше, знищуючи
\enquote{конкурентів}. Взяти хоча б Валуєвський Циркуляр та Емський Указ, які
забороняли \emph{українську} мову та дозволяли знищувати тих, хто її
популяризував. Перераховувати усі 134 задокументовані факти утисків
\emph{української} мови немає потреби. Їх легко можна знайти у шкільних
підручниках з літератури та історії \emph{України},
\textbf{Чи є різниця, якою мовою говорити?} Андрій Білецький, pravda.com.ua, 22.01.2021

Що ж робити тим, хто любить \emph{Україну}, але все життя говорить російською? Все
просто – вчіть державну мову. За 30 років Незалежності це можна було зробити
без проблеми. Звичайна людина може за рік навчитися вільно говорити будь-якою
мовою. За півроку – якщо живе у мовному середовищі.  Чи є різниця, якою мовою
говорити? Так, є. Мова – це природній та культурний кордон. Він окреслює ареал
проживання народу. Він єднає нас із попередніми поколіннями \emph{українців}. Це – те,
що треба захищати завжди,
\textbf{Чи є різниця, якою мовою говорити?} Андрій Білецький, pravda.com.ua, 22.01.2021

Похоже, что приходит конец \emph{рАсейской} космонавтике. Дырка в МКС,
кончилась жратва, а потом и туалет сломался. Все признаки, что пипец не за
горами.  А в то время, пока \emph{Раша} заклеивает скотчем дырки на МКС и
просит у американцев еду, Украина развивает космические технологии:
\enquote{Американская компания-разработчик ракетно-космической техники Firefly
Aerospace Inc., собственником которой является украинский бизнесмен Максим
Поляков, успешно провела предполетные испытания своей первой ракеты Alpha.} (С)
коментар, \textbf{Чи є різниця, якою мовою говорити?} Андрій Білецький, pravda.com.ua, 22.01.2021

Чим більше людей буде розмовляти \emph{українською}, тим швидше ми проженемо окупанта,
тому хто не згоден нехай їдуть геть з країни, а якщо хочуть жити та розвиватись
тут, то будьте ласкаві спількуйтеся державною мовою,
коментар, \textbf{Чи є різниця, якою мовою говорити?} Андрій Білецький, pravda.com.ua, 22.01.2021

Ви для початку поясніть це своїм активістам Нацкорпусу, які хіба що крім
західної \emph{України} переважно російськомовні. А то якісь подвійні стандарти...
коментар, \textbf{Чи є різниця, якою мовою говорити?} Андрій Білецький, pravda.com.ua, 22.01.2021

Кто убил 17 миллионов \emph{украинцев}?, Юрий Гуленок, hvylya.net, 01.06.2021

Кто виноват в том, что миллионы \emph{украинцев} умерли преждевременно, миллионы не
родились, миллионы батрачат вдали от семей, надеясь уехать из этой несчастной
страны навсегда?  Ответ общеизвестен: во всех наших бедах виноваты те
избиратели, которые каждый раз избирают во власть воров, негодяев, невежд и
клоунов,
\textbf{Кто убил 17 миллионов \emph{украинцев}?}, Юрий Гуленок, hvylya.net, 01.06.2021

\emph{Украина} вчера стала третьей в Европе по смертям от коронавируса, Елена Вьюн, strana.ua, 02.06.2021

Ровно семь лет назад \emph{Украина} навсегда потеряла Донбасс, - Помните женщин с оторванными ногами у здания Луганской ОГА? 
Андрей Головачев, strana.ua, 02.06.2021

Когда я посмотрел кадры женщин с оторванными ногами, ползающих в лужах своей
крови, перед входом в здание Луганской ОДА после авиационного удара \emph{украинских}
ВВС я написал следующие слова: \enquote{Все! Донбасс ушел. Забудьте! Сейчас уже
нет силы, которая могла бы возвратить Донбасс} К сожалению, по прошествии 7 лет
я вынужден признать, что мой прогноз оказался верным,
\textbf{Ровно семь лет назад \emph{Украина} навсегда потеряла Донбасс, - Помните женщин с оторванными ногами у здания Луганской ОГА?} 
Андрей Головачев, strana.ua, 02.06.2021

Через 10 лет население \emph{Украины} составит не больше одной сотой населения Китая,
Зато все мы будем в вышиванках, Юрий Касьянов, strana.ua, 02.06.2021

Население Китая сегодня - 1,4 млрд человек, население Земли - 7,9 млрд человек.
Население \emph{Украины} - 0,04 млрд человек, \textbf{Через 10 лет население
\emph{Украины} составит не больше одной сотой населения Китая, Зато все мы
будем в вышиванках}, Юрий Касьянов, strana.ua, 02.06.2021

Якщо ти живеш і працюєш в \emph{Україні}, є \emph{українцем}, поважаєш традиції та інших
людей, то ти за замовчуванням маєш бути патріотом. І говорити про виховання
патріотів – це нонсенс. Але в \emph{Україні} ситуація склалась трішки по-іншому.
Оскільки є багато людей, які ненавидять все \emph{українське}, але живуть і працюють
тут. Це проблема, але не потрібно, реагуючи на це, починати виховувати їхніх
антиподів. Це призводить до конфліктів та нерозуміння процесів. Тоді боротьба
за щось стає самоціллю, а результатом ніколи не буде розвиток, 
\textbf{Союз – мертвий, комсомол – живий, Тиск замість розвитку! Що хочуть від \emph{української} молоді?},
Станіслав Безушко, zaxid.net, 31.05.2021

Передовсім проблема лежить у площині цінностей. У більшості \emph{українців} вони
насправді відсутні. Цінності заміняються поведінковими моделями. Ми сприймаємо
за цінність набуту модель поведінки і вважаємо її головним маяком у житті. І це
стосується не тільки бабусь та батьків, а й молоді. Молодь має модель, але не
розуміє її суті. Старше покоління (якщо рівень освіти та критичного мислення
дозволяє) здатне оцінити всі ризики й загрози. Молодь, не розуміючи звідки
взялась така модель поведінки, просто копіює її та не заглиблюється в суть,
\textbf{Союз – мертвий, комсомол – живий, Тиск замість розвитку! Що хочуть від \emph{української} молоді?},
Станіслав Безушко, zaxid.net, 31.05.2021

Термін «передова радянська молодь» поступово трансформувався у «найкращі
представники \emph{української} молоді». Замість «виховувати молодь в дусі комунізму»
тепер виховують «патріотичну молодь». І такі порівняння можна продовжувати дуже
довго,
\textbf{Союз – мертвий, комсомол – живий, Тиск замість розвитку! Що хочуть від \emph{української} молоді?},
Станіслав Безушко, zaxid.net, 31.05.2021

Перший. Ні, \emph{Україна} не стала незалежною завдяки Бандері та \emph{УПА}.
Це може не подобатися, але це факт. Ідеологія інтеґрального націоналізму
бандерівців робила ставку на один спосіб звільнення \emph{України} –
загальнонаціональне визвольне повстання. Насправді незалежність \emph{України}
стала можливою передусім через шкурне, еґоїстичне бажання еліт \emph{УРСР}
роздерибанити майно республіки. Тобто держава \emph{Україна} створена не
патріотичними повстанцями, а радянськими пристосуванцями. Але так, наявність в
історії Бандери та \emph{УПА} дещо вплинула на особливості суспільної думки та
радянської гуманітарної політики в Галичині і, меншою мірою, на Волині,
\textbf{П'ять цікавих фактів – від \emph{УПА} до АТО}, Павло Зуб'юк, zaxid.net, 01.06.2021

Так виглядає, що час «рожевих окулярів» давно закінчився. Більшість \emph{українців}
знають хто є ворогом, який напав на півдні і сході \emph{України}. Є навіть бачення і
певні кроки в побудові сучасних збройних сил, які зможуть дати відсіч зовнішній
агресії. Проте ми повинні усвідомити, що акцент на відбитті такої агресії і
навіть на поверненні захоплених територій є лише частиною правильної зовнішньої
політики. Навіть побудова успішної економічної країни не вирішить усіх проблем
із РФ. Ми повинні усвідомити і змиритися з тим, що нам і в майбутньому
прийдеться жити із нашим екзистенційним якщо навіть не ворогом то точно
екзистенційним опонентом. І можливості позбутися цього у нас не має,
\textbf{\emph{Росія} – вічний супротивник України}, Роман Кізима, zaxid.net, 28.05.2021

Маючи постійного сильного опонента і навіть ворога під боком нам потрібно
враховувати той факт, що зовнішні фактори безумовно будуть впливати і на
внутрішні. І чим більш успішною буде \emph{Україна}, тим більшою буде протидія зі
Сходу. Нам не вдасться ігнорувати цивілізаційну зовнішню загрозу, закрившись у
власній шкаралупі. І приклад Ізраїлю у постійному ворожому оточенні лише
підтверджує цей факт. Будьмо сильні, успішні та найголовніше – проінформовані.
А проінформовані – значить озброєні...
\textbf{\emph{Росія} – вічний супротивник України}, Роман Кізима, zaxid.net, 28.05.2021

Останні декілька днів український політично ангажований facebook-сегмент рясніє
фотографіями депутатів однієї з політичних сил (як мінімум п'ятеро), які по
черзі несуть на мужніх чоловічих і жіночих плечах автомобільну шину. Все це
відбувається на фоні мітингу підтримки колишнього президента від кримінального
переслідування. Як відомо проти колишнього гаранта Конституції відкрито близько
10 (!) кримінальних справ. Починаючи з доволі серйозних, типу можливих
корупційних призначень і завершуючи доволі екзотичними, на кшталт «справи
Томосу»,
\textbf{Автомобільна шина як символ \emph{українського} бумерангу}, Роман Кізима, zaxid.net, 20.06.2020

Три місяці щеплень позаду. Чи стає \emph{Україна} ближчою до колективного
імунітету від коронавірусу?, Ольга Кириленко, pravda.com.ua, 02.06.2021

Про те, наскільки далека від цього \emph{Україна} говорить її місце у \enquote{загальному
заліку} країн за відсотковим охопленням щепленням – 132 місце зі 178, між
Нікарагуа та Соломоновими островами. Дуже далеко навіть після сусідніх Росії та
Білорусі, які на початку пандемії відверто ігнорували карантинні заходи.  УП
розповідає, що говорить наука про колективний імунітет від ковіду та чи
досяжний він для \emph{України} з нинішніми темпами вакцинації,
\textbf{Три місяці щеплень позаду. Чи стає \emph{Україна} ближчою до колективного
імунітету від коронавірусу?}, Ольга Кириленко, pravda.com.ua, 02.06.2021

Почему разваливается \emph{украинская} семья?, Иван Пургин, from-ua.com, 02.06.2021

На этот вопрос дают ответ психологи, считающие, что в \emph{Украине} материальные
проблемы просто доминируют над остальными, зачастую скрывая их. Они изначально
закладываются во многих браках, когда к своей будущей «половинке»
предъявляются, в первую очередь, именно материальные требования. Особенно это
касается женщин, мечтающих о принце на белом «мерседесе». Но они забывают, что
на чисто денежных отношениях семья не строится. «Если человек продал себя в
брак, то однажды его могут выбросить и купить другого», - гласит современная
народная мудрость,
\textbf{Почему разваливается \emph{украинская} семья?}, Иван Пургин, from-ua.com, 02.06.2021

Поднять уровень жизни \emph{украинцев} хотя бы втрое, до нижнего европейского, уже
непосильная задача для современной \emph{Украины}. Для этого потребовались бы не
просто колоссальные реформы, но и полная смена всей власти и политической
концепции – а этого просто не позволят сделать. Но если произойдет какое-то
чудо, форс-мажор, и однажды \emph{украинцы} проснутся в стране со средним доходом
1000-1200 евро на человека, то их радость будет недолгой. Потому что они просто
перейдут на более высокий потребительский уровень (выше, чем сейчас в \emph{Украине},
но еще ниже, чем в Германии или Франции), где люди испытывают такую же нехватку
денег. Только их будет не хватать уже не на детское питание или планшет, а на
что-то более существенное,
\textbf{Почему разваливается \emph{украинская} семья?}, Иван Пургин, from-ua.com, 02.06.2021

Впрочем, многие \emph{украинцы} об этом даже не догадываются. Потому что если наша
власть в чем и преуспела, так это в умении «замыливать» глаза и «запудривать»
мозги иллюзиями «перемог». И если «загуглить» статистику браков и разводов в
\emph{Украине}, то 95\% полученных результатов будут ссылаться на данные Министерства
юстиции. А они выглядят не просто ободряюще, а действительно «переможно»! Так,
например, согласно Минюсту, в 2019 году на 237 тысяч зарегистрированных браков
пришлось всего 38 тысяч разводов, то есть около 16\%. О таких показателях в
Европе и не мечтают! Кажется, что можно смело развалиться в кресле и начать
витийствовать о семейных ценностях \emph{уникальной нации},
\textbf{Почему разваливается \emph{украинская} семья?}, Иван Пургин, from-ua.com, 02.06.2021

Так от, пригадався мені цей апокриф після ознайомлення зі змістом пояснювальної
записки нещодавно опублікованого законопроєкту під номером 5554, в якій
стверджується, що обов'язкове використання \emph{української} мови при
демонстрації на \emph{українському} телебаченні російськомовних серіалів
\enquote{призведе до руйнування кінематографічної та телевізійної галузі від
чого, найбільших втрат зазнає \emph{український} глядач.},
\textbf{Микола Лукаш - \emph{Київ} - Черга - Дівчата - СРСР - Легенда}, Філіп Іллєнко, facebook.com, 31.05.2021

Воно там трохи колись пострибало на \emph{українському} телебаченні, а потім
його забули.  То треба ж про себе нагадати хоч якось, от воно й вищить.
Звичайний представник «гонимого» народу без країни, роду та племені, що за
гроші і в церкві пердне,
коментар, \textbf{Микола Лукаш - \emph{Київ} - Черга - Дівчата - СРСР - Легенда}, Філіп Іллєнко, facebook.com, 31.05.2021

Сегодня исполняется семь лет авиаудару по Луганску, во время которого погибли
восемь человек. 2 июня 2014 года ракеты \emph{украинского} штурмовика Су-25
обстреляли здание областной администрации в самом центре города. Это случилось
неподалеку от детского садика, сквера и жилых многоэтажек.  \emph{Украина} до сих пор
официально не признала своей роли в этой трагедии. Никаких дел не заведено и
расследований не ведется. Хотя другой авиации, кроме \emph{украинской}, над Луганском
не было. На седьмую годовщину авиаудара его снова обсуждают в \emph{Украине}. Звучат
мнения, что случившееся стало точкой невозврата в конфликте на Донбассе,
\textbf{\enquote{Точка невозврата для Донбасса}. Семь главных вопросов об авиаударе \emph{ВСУ} по Луганску 7 лет назад},
Максим Минин, strana.ua, 02.06.2021

5 марта в Луганске выбрали первого \enquote{народного губернатора} Александра
Харитонова. А через месяц противники Майдана взяли штурмом здание областного
СБУ, которое превратили в свой штаб. Это произошло 6 апреля - в день, когда
началась Антитеррористическая операция, объявленная исполняющим обязанности
президента \emph{Украины} Турчиновым. Через две недели \enquote{губернатором} объявили
бывшего десантника Валерия Болотова, а 27 апреля провозгласили \enquote{ЛНР}. Далее под
контроль сепаратистов стали переходить как объекты в Луганске, так и другие
города области. А 12 мая прошел \enquote{референдум} об отделении от \emph{Украины}, 
\textbf{\enquote{Точка невозврата для Донбасса}. Семь главных вопросов об авиаударе \emph{ВСУ} по Луганску 7 лет назад},
Максим Минин, strana.ua, 02.06.2021

Далее \emph{украинские} власти запустили версию о взорвавшемся кондиционере. По словам
замгенпрокурора Николая Голомши, сторонники \enquote{ЛНР} пытались сбить
украинский самолет из переносного зенитно-ракетного комплекса. Но попали в
здание ОГА - а именно, в кондиционер. Правда, судя по фото здания до и после
обстрела, на месте попадания ракеты не было кондиционера. Он находился двумя
окнами левее и был целым после взрыва,
\textbf{\enquote{Точка невозврата для Донбасса}. Семь главных вопросов об авиаударе \emph{ВСУ} по Луганску 7 лет назад},
Максим Минин, strana.ua, 02.06.2021

Университет Зеленского закончится очередным полетом в космос. В стране нет тех,
кто учил бы молодых \emph{украинцев} по-новому, Дмитрий Раимов, strana.ua, 02.06.2021

Эти новости - \enquote{первые ласточки} того безрадостного будущего, в котором правящие
элиты разных стран попытаются \enquote{размыть} ответственность за принятие
управленческих, правосудных, и иных решений, от которых может зависеть жизнь и
смерть, \enquote{перепоручив} её всевозможным роботам и \enquote{искусственному интеллекту}.
Конечно же, в реальности ни о какой \enquote{диктатуре машин} и \enquote{технологической
сингулярности} речи не идёт. Управлять всё равно будут люди. Но управляющий
слой надёжно прикроет себя от возмущения масс, переложив в их сознании
ответственность за принятие всех важных решений на машины. Кстати, именно
поэтому в \emph{Украине} так любят говорить о цифровизации,
\textbf{В Ливии квадрокоптер самостоятельно принял решение убить человека}, Даниил Богатырев, 02.06.2021

\enquote{Дальше Ткаченко обложит налогами PornHub}. Сеть обсуждает идею \enquote{слуг народа} \emph{украинизировать} Netflix и YouTube,
Виктория Венк, strana.ua, 14.11.2019

Закрытие телеканалов и ущемления по языку. Как в \emph{Украине} нарушают права человека. Анализ доклада ООН,
Максим Минин, strana.ua, 02.06.2021

2 июня 2014 года многие луганчане считают «точкой невозврата»: в этот день
гибридные формирования России штурмовали Луганский погранотряд. Среди горожан
всю весну распространяли слухи о якобы едущем в город «Правом секторе» с
Майдана – тот так и не прибыл, зато \emph{украинские} военный части начали штурмовать
вооруженные отряды, подконтрольные России. Ко второму июня все административные
здания в городе уже были захвачены, и вокруг погранзаставы началась настоящая
война с выстрелами и взрывами. Однако, в интерпретации пропагандистов 2 июня
войну, наоборот, развязали «\emph{украинские} каратели», нанеся авиаудар по зданию
Областной госадминистрации,
\textbf{«Никто нам не вернет украденные годы»: как в \emph{Луганске} вспоминают захват города},
Донбас.Реалії, Записки з окупації, radiosvoboda.org, 02.06.2021

\underline{Валентина, пенсионерка: Неужели в Луганске не было милиции?}
– Уже в мае повсюду по городу стояли вооруженные люди. В окнах государственных
зданий были мешки с песком. На зданиях висели не \emph{украинские} флаги. Вывозилась
документация. Вопрос – почему был такой бардак? Да потому что всё было
проплачено. Неужели в Луганске не было милиции? Почему тогда люди ходили с
оружием? Откуда взялись эти казаки, причем сразу и везде? Еще тогда были
отключены все \emph{украинские} каналы на телевидении. Вместо них показывали
\emph{аниукраинские} ролики, пропаганда – жесточайшая. Людям попросту промывали мозги.
А правду узнать было неоткуда. Поэтому такая поддержка России,
\textbf{«Никто нам не вернет украденные годы»: как в \emph{Луганске} вспоминают захват города},
Донбас.Реалії, Записки з окупації, radiosvoboda.org, 02.06.2021


\underline{Виктор, пенсионер: страну пишу только с маленькой буквы},
– Точка невозврата наступила, когда жгли коктейлями и били различными
предметами ментов, вплоть до убийства. Когда боевики Парубия вели отстрел с
гостиницы по «сотне» и ментам. И когда людей сожгли заживо в Доме профсоюзов в
Киеве. Затем был Дом профсоюзов в Одессе, горотдел милиции в Мариуполе... Думаю
этого достаточно, чтобы не любить \emph{«украину»}. Возможно, у вас возник вопрос,
почему я пишу с маленькой буквы? Я \emph{эту страну} с 2014 года пишу только с
маленькой буквы,
\textbf{«Никто нам не вернет украденные годы»: как в \emph{Луганске} вспоминают захват города},
Донбас.Реалії, Записки з окупації, radiosvoboda.org, 02.06.2021

\underline{Кристина, работник культуры: мы все в анабиозном сне} – До сих пор
помню события тех дней.  Мы понимали и видели, как работал отлаженный механизм
захвата города.  Осознала, что началось страшное, когда поздно вечером услышала
автоматные очереди. Тогда на городке завода ОР захватывали военную часть. Было
очень страшно, всю ночь рыдала, понимая, что жизнь теперь изменится, и это
война. Все, что происходило в городе после этого, похоже на сон. Мы понимали и
видели, как работал отлаженный механизм захвата города. Но до последнего
верили, что всё скоро закончится. Во время обстрелов было особенно страшно,
погибали люди, а сбитый самолет... Я слышала, как он взорвался! После этого мы
на время выскочили из города и вернулись только в октябре. Вернулись жить
дальше в свою квартиру, смириться было сложно с обстоятельствами. Мы понимаем,
что ничего хорошего в оккупации не может быть, но живем дальше. А сколько нас
таких! Даже многие соседи в доме, рьяно радующиеся русскому миру, сегодня хотят
возврата \emph{Украины}. Город жив, но он как в анабиозном сне,
\textbf{«Никто нам не вернет украденные годы»: как в \emph{Луганске} вспоминают захват города},
Донбас.Реалії, Записки з окупації, radiosvoboda.org, 02.06.2021

Колишній очільник УІНП, народний депутат \emph{України} від «Європейської
солідарності» Володимир В'ятрович вважає, що «декомунізувати» герб заважає
відсутність політичної волі. «Було б бажання, вже давно гроші б знайшли і
зробили», – сказав В'ятрович у коментарі Радіо Свобода. Володимир Бірчак,
керівник академічних програм у Центрі досліджень визвольного руху, один із
авторів пакету законів про «декомунізацію», нагадав у коментарі Радіо Свобода,
що «як символіка тоталітарного режиму герб має бути демонтованим», а уряд має
виконати норму закону,
\textbf{«Перекрити тризубом»: що заважає декомунізувати герб СРСР на монументі «Батьківщини-матері» у Києві}?
Ірина Штогрін, radiosvoboda.org, 01.06.2021

Заграничные белорсские патриоты составили революционный план Перемога, Но опыт
\emph{Украины} показывает, что за Перемогой всегда идет Зрада, Дмитрий Василец,
strana.ua, 03.06.2021

Интервью Зеленского отражает слабость \emph{украинской} внешней политики,
Никому не интересна страна, которая всегда попрошайничает, Виктор Суслов,
strana.ua, 03.06.2021

И главный вопрос: кому адресованы эти жалобы и какой в них смысл? Ну разве что
в очередной раз попугать Европу, что \enquote{Путин обязательно нападет. Ну до 12-16
сентября - точно}.  Так все равно денег не дадут и в \enquote{буржуинство} не возьмут.
А продолжат дальше разбирать Украину \enquote{на запчасти}.  На повестке дня -
\enquote{раздача} украинской земли и природных ресурсов,
\textbf{Интервью Зеленского отражает слабость \emph{украинской} внешней политики,
Никому не интересна страна, которая всегда попрошайничает}, Виктор Суслов,
strana.ua, 03.06.2021

Для граждан \emph{Украины} и России смысл этого диалога объяснять не нужно. Но для
иностранцев его решили разъяснить. И в разъяснении аналогом слова
\enquote{бандеровец} стала фраза \enquote{\emph{Украинский} нацистский коллаборант}. В
результате поднятого вокруг такого \enquote{перевода} скандала новые субтитры
стали политически нейтральными и непонятными для абсолютного большинства
иностранцев: \enquote{бандеровец} стал \enquote{banderite}. И некоторые
политики записали это себе как \enquote{перемогу}. При этом главной
\enquote{зрады} никто не заметил.  Как известно, фильм \enquote{Брат-2} в
\emph{Украине} запрещен. И тем не менее, фильм появился на \enquote{Нетфликсе}, а
\emph{украинские} политики не только ничего не сделали для снятия фильма с платформы,
но и прорекламировали \enquote{новинку} для максимально широкой отечественной
аудитории,
\textbf{Операция \enquote{Антиолигарх}, антибандеровский \enquote{Нетфликс},
МИД \emph{Украины} против \enquote{БелАвиа}. Итоги \enquote{Страны}},
strana.ua, 03.06.2021

Я президент цієї федерації. У нас є осередки в багатьох містах \emph{України}: Київ,
Харків, Дніпро, Одеса, Миколаїв, Запоріжжя, Івано-Франківськ, Львів і в містах
по областях. Змагання ми проводимо кілька разів на рік, але, наприклад,
минулого року їх взагалі не було через ковід і карантин. Ще не змагалися в 14
році, бо тоді було не до цього. Зараз ми збираємо близько сотні учасників - це
для аматорського спорту дуже великий показник. Не завжди навіть олімпійські
види спорту збирають стільки людей,
\textbf{Доброволець Олександр Воробєй: \enquote{Війна не змінює людину, вона
просто розкриває її сутність. Якщо людина була гандоном, то після фронту вона
стане махровим п\#дарасом}},
Віка Ясинська, censor.net.ua, 02.06.2021

Зараз ми міняємо приміщення – і я хочу, щоб ми робили 400 ножів на місяць. Для
\emph{України} це нормальна кількість. Бо якщо в принципі говорити про партії
ножів, які у нас є, – в основному це Китай. А наші вироби майже повністю
зроблені з український матеріалів: наприклад, сталь, робота теж українська.
Називаються наші ножі \enquote{Blade Brothers Knives}, бо мій клуб називається
\enquote{Blade Brothers}.  Дизайн лого нам зробив дизайнер Сергій Снурник,
\textbf{Доброволець Олександр Воробєй: \enquote{Війна не змінює людину, вона
просто розкриває її сутність. Якщо людина була гандоном, то після фронту вона
стане махровим п\#дарасом}},
Віка Ясинська, censor.net.ua, 02.06.2021

Я пройшов метрів десять і полегшено зітхнув. Голову не повертав, боявся проклять...
Іду далі, бачу чоловіка, одягнутий в хороше пальто, розмовляє по дорогому
телефону. Поряд з ним ідуть два \enquote{тітушки}, молоді хлопці, в тактичному одязі.
Один з них так подивився на мене, що моя рука автоматично полізла в сумку за
пістолетом. На щастя, не довелося його використовувати, тим більше, поблизу
стояли кілька хлопців з Національної гвардії \emph{України}.  Я дійшов до площі, де
стояла імровізована сцена, звідки кричали пригноблені вожді опзж, серед яких
\enquote{ошивався} комуніст симоненко,
\textbf{Чому партія \enquote{опзж}, морить своїх людей голодом?}, 
Михайло Ухман, censor.net.ua, 03.06.2021

Вони говорили кілька хвилин, у співрозмовника рабіновича час від часу світився телефон, вартістю 30 тисяч гривень.
А поряд, під парканом - сиділи прості люди, яких збирали по всій \emph{Україні}.
- Я жрать хочу, б..ть, - кричала одна старша жіночка. - Уже трі чіса нас дєржат.
- Подожді, сейчас наорутса із сцени і покормлят нас. Обіщалі же - промовив
чоловік, який стояв біля неї в дешевому одязі і розбирав телефон - жабку, яка:
\enquote{Заглючила, бля..ь}!
Що і хто кому обіцяв, я не знаю. Довелося іти по своїм справам. Але те, що
керівництво опзж використовує простих громадян в своїх цілях - це сто
відсотків. А вони, як стало баранів ідуть за тими, хто навіть не може їх
нагодувати...
Так і здохните рабами,
\textbf{Чому партія \enquote{опзж}, морить своїх людей голодом?}, 
Михайло Ухман, censor.net.ua, 03.06.2021

Якщо почуєте від когось, що в \emph{Україні} \enquote{громадянський конфлікт},
а не війна з Росією - розкажіть і покажіть цю історію. Якщо це не допоможе -
бийте в голову, стріляйте в коліна, адже перед вами може бути, як мінімум
сепаратист,
\textbf{Російські офіцери в \emph{українському} полоні},
Михайло Ухман, censor.net.ua, 17.05.2021

Група ворожих розвідників висунулася в напрямку наших позиції, не зустрівши
опору, зуміла підійти впритул і була виявлена \emph{українським} воїном - Вадимом
Пугачовим, який отримав важкі поранення в ході перестрілки і згодом помер.
Наші бійці зуміли затримати Єрофиєва і Олександрова, які потім дали важливі для
нас свідчення.  У 2016 році відбувся суд, росіян визнали винними і дали
тюремний термін. Проте через деякий час, їх обміняли на так звану політичну
бранку - Надію Савченко. Яка згодом відкрито почала працювати з ворогами нашого
народу, адже її полон і перебування в \emph{Україні} після визволення - це суцільна
трагікомедія, через яку загинули десятки \emph{українських} воїнів, 
\textbf{Російські офіцери в \emph{українському} полоні},
Михайло Ухман, censor.net.ua, 17.05.2021

О скандалах с изгнанием из Киевского оперного театра им. Шевченко руководителя
балетной труппы Дениса Матвиенко и с изгнанием из Полтавского
музычно-драматычного им. Гоголя главрежа Алексея Коломийцева писалось,
говорилось и показывалось много – истории известные и абсолютно архетипичные
для \emph{Украины}. Можно еще вспомнить инновационный театр Андрея Жолдака – он тоже
где-то в творческих \enquote{бегах}.  Иначе в \emph{Украине} НЕ БЫВАЕТ,
\textbf{Мариинский-2 и украинская культура: Почему \emph{Украина} – это фабрика человеческого мяса? (обн.)},
Ірина Славінська, pravda.com.ua, 03.05.2013

А вот в \emph{Украине} нет ни западного представления об \enquote{общем благе}, ни российского
имперского дискурса. Поэтому \emph{Украина} – это питомник талантов, которые могут
реализоваться только где-то далеко. \emph{Украина} – это ферма уникального
человеческого мяса,
\textbf{Мариинский-2 и украинская культура: Почему \emph{Украина} – это фабрика человеческого мяса? (обн.)},
Ірина Славінська, pravda.com.ua, 03.05.2013

\emph{Украинские} олигархи вкладываются в футбольные стадионы и футбольные клубы. Или
в высокохудожественные выставки с намотанными кишками и заспиртованными
коровьими головами. Но не в театры. А зачем? Ведь функции театра, а особенно
цирка, в \emph{Украине} неплохо исполняет Рада, телеканалы и т.д.  Едва ли не главная
борьба в XXI веке – это борьба с свино-человечеством, с обыдливанием человека,
с превращением его в тупое животное-потребителя. \emph{Украина}, кажется, с этого
фронта уже давно дезертировала,
\textbf{Мариинский-2 и украинская культура: Почему \emph{Украина} – это фабрика человеческого мяса? (обн.)},
Ірина Славінська, pravda.com.ua, 03.05.2013

\emph{Украина} как всегда – производитель ценнейшего человеческого материала (в плане
оперных голосов – не нужны дополнительные рекомендации). Однако ни одного
бренда, ни одного успешного проекта. А советские институции и форматы –
одряхлели и обветшали.  Попытка превратить Одесскую оперу в театр мирового
класса, которой несколько лет назад занимался режиссер Сергей Проскурня,
разбилась о тупизну и коррупционность одесских жлобов и киевских чиновников,
\textbf{Мариинский-2 и украинская культура: Почему \emph{Украина} – это фабрика человеческого мяса? (обн.)},
Ірина Славінська, pravda.com.ua, 03.05.2013

Власти \emph{Украины} начали силовую операцию против Донбасса в апреле 2014 года.
Урегулирование конфликта базируется на Комплексе мер по выполнению Минских
соглашений, подписанном 12 февраля 2015 года в белорусской столице участниками
Контактной группы и согласованном с главами стран - участниц \enquote{нормандской
четверки} (Россия, Германия, Франция и \emph{Украина}). Документ, в частности,
предусматривает прекращение огня и отвод тяжелых вооружений от линии
соприкосновения,
\textbf{Киев препятствует попыткам ЛНР выстроить отношения 
с государствами мира – Дейнего}, lug-info.com, 03.06.2021

\emph{Украинская} русофобия становится все более радикальной. Марков поделился инструментом для борьбы с ней,
ukraina.ru, 03.06.2021

З давніх-давен українців відзначало особливе ставлення до жінок, і це в часі
сягає не століть, а тисячоліть. Так стверджує історія та археологія – наші
праматері були берегинями родини, роду, племені й усього народу.  Виважене і
шанобливе ставлення українців до жіноцтва було поширене серед усіх соціальних
станів і за Київської Русі, і після її Хрещення. Права жіноцтва були вписані
до \enquote{Руської Правди} – зведення тогочасних законів Київської Русі, виконаного
Ярославом Мудрим на початку XІ століття,
\textbf{Шанобливе ставлення до жіноцтва в \emph{Україні}},
Марія Гуцол, slovoprosvity.org, 17.05.2021

Архітектурний проєкт у стилі модного тоді конструктивізму мав і \emph{українські}
риси, якими відомий зодчий Василь Кричевський завжди доповнював і прикрашав
свої споруди. Щоправда, авторам довелося змінювати початковий задум через брак
коштів, і їхнє творіння значно відрізнялося від первісного проекту, тож
архітектори були незадоволені,
\textbf{Знакові будівлі центру столиці}, Надія Наумова, slovoprosvity.org, 17.05.2021

Але в час політичних репресій деяким мешканцям знаменитої споруди як ворогам
народу довелося переходити на інші, некомфортні квартири. За багато років
будинок відбив складний шлях \emph{української} літератури та її творців, про нього
можна написати роман, як це зробив Юрій Тріфонов (\enquote{Дом на набережной}),
\textbf{Знакові будівлі центру столиці}, Надія Наумова, slovoprosvity.org, 17.05.2021

Уночі не спали, бо саме після півночі чи вдосвіта з під’їздів виводили ворогів
народу, тих, хто не хотів іти проти своєї совісті. А ті, хто уникнув арешту,
упокорилися, рішуче стали на шлях оспівування соціалістичної дійсності,
продовжували жити і писати тут свої твори. У деяких згадано і цей будинок:
\enquote{Коли у червні почалась війна…} Л. Первомайського, \enquote{У Києві, де
вгору гнеться вулиця...} С. Голованівського, \enquote{Наш добрий дім} В.
Малишка.  Тут були написані визначні, різні за тематикою твори \emph{української}
літератури: \enquote{Мир хатам, війна палацам} Юрія Смолича, \enquote{Собор}
Олеся Гончара, переклад \emph{українською мовою} поеми Шота Руставелі \enquote{Витязь
у барсовій шкірі} Миколи Бажана, \enquote{Гуси-лебеді летять} Михайла
Стельмаха, поезії \enquote{Знову цвітуть каштани... (Київський вальс)},
\enquote{Пісня про рушник} Андрія Малишка,
\textbf{Знакові будівлі центру столиці}, Надія Наумова, slovoprosvity.org, 17.05.2021

Маємо зворушливі спогади про Євгена Плужника – талановитого поета, який вірив у
ленінські ідеї та Жовтневу революцію. Він не мав власного кутка і жив у приймах
у своєї дружини в кімнаті реквізованого \enquote{прибуткового} будинку на вулиці
Прорізній. Плужник одним із перших членів Спілки письменників \emph{України} вступив
до кооперативу і заплатив гроші за житло. Його так хвилювала думка про те, чи
увійдуть улюблені меблі в кімнати, що він, вимірявши їх розміри, побіг у свою
ще неофіційну квартиру. Повернувся веселий і повідомив дружині: \enquote{Стане, стане!}
Мав туди заселитися 11 грудня. Але вже дещо відчував і наготував про всяк
випадок вузлик з усім необхідним. Євгена \enquote{взяли} в кімнаті на Прорізній 4
грудня 1934 р. на очах у дружини та її сестри,
\textbf{Знакові будівлі центру столиці}, Надія Наумова, slovoprosvity.org, 17.05.2021

У дворі в невеликому будинку, що поволі розвалюється, можна було б створити
філію Музею літератури, або віддати приміщення Музею історичного центру міста
Києва. Або ж створити літературну кав’ярню, де відвідувачі могли б згадати курс
\emph{так званої української радянської літератури},
\textbf{Знакові будівлі центру столиці}, Надія Наумова, slovoprosvity.org, 17.05.2021

Не важно, что именно для каждого из нас сегодня является олицетворением этого
страха, кого именно каждый из нас сегодня видит в качестве угрозы - Америку,
Кремль, \emph{Украину}, гомосексуалистов или турков, \enquote{развратную} Европу,
пятую колонну или просто начальника на работе или полицейского у входа в метро.
Важно - осознаем ли мы, до какой степени наши сегодняшние личные страхи, личное
ощущение внешней угрозы - в реальности являются лишь призраками прошлого,
существование которого мы так боимся признать,
\textbf{Бред сумасшедшего! Как с этим жить?}, Владимир Яковлев, news24ru.net, 03.06.2021

Объекты Северного потока-2 уже переводят в тестовый режим. Как-то не помогают
протесты \emph{украинской} патриотической общественности..., Александр
Рябоконь, strana.ua, 03.06.2021

Словом, \emph{Украина} стреляет себе не просто в ногу, а в голову из-за какой-то
недоросли... по фамилии... как его?.. а, Протасевич!  И кстати, обратите
внимание: несмотря ни на что, Лукашенко, как бы к нему не относиться,
продолжает называть \emph{Украину} братской страной...,
\textbf{Европа впала в иррациональный истеричный лукашизм}, Александр Карпец,
strana.ua, 03.06.2021
