% vim: keymap=russian-jcukenwin
%%beginhead 
 
%%file 22_05_2022.fb.fb_group.ohtyrka_operatyvna.1.pozyvnyj_italiec
%%parent 22_05_2022
 
%%url https://www.facebook.com/groups/498652738278706/posts/560386075438705
 
%%author_id fb_group.ohtyrka_operatyvna
%%date 
 
%%tags 
%%title Позивний Італієць - розповідь
 
%%endhead 
 
\subsection{Позивний Італієць - розповідь}
\label{sec:22_05_2022.fb.fb_group.ohtyrka_operatyvna.1.pozyvnyj_italiec}
 
\Purl{https://www.facebook.com/groups/498652738278706/posts/560386075438705}
\ifcmt
 author_begin
   author_id fb_group.ohtyrka_operatyvna
 author_end
\fi

\ii{22_05_2022.fb.fb_group.ohtyrka_operatyvna.1.pozyvnyj_italiec.pic.1}

31-річний Сергій (на фото зліва) - боєць окремого підрозділу ЗСУ - має
екзотичний позивний: «Італієць». Таким іменем давно нагородили побратими, бо
хлопець з юних років жив в Італії, але повернувся захищати батьківщину ще
2015-го. Так і воює за свою землю з російськими окупантами. Зараз - на
Харківщині, а перед тим рятував Охтирку разом із 93 бригадою. Ексклюзивно для
«Охтирки оперативної» Сергій розповів про той період війни, і сам відверто
здивувався словам мера Павла Кузьменка про масове дезертирство.

\subsubsection{Початок боїв}

- Перша колона русні, яка прорвалася в Охтирку, була атакована хлопцями з 93-ї
бригади і частково розгромлена біля Успенської площі. Там 24 лютого відбулася
перша серйозна сутичка, яка продовжилася в мікрорайоні «Дачний», а згодом у
Кардашівці наші атакували другу колону ворожої бронетехніки (танки,
бронемашини...). На наступний день хлопці вступили в бій з ворогом на
перехресті біля заправки «Маршал», де сконцентрувалася найбільша колона орків.
І дуже багато техніки сунулося на Охтирку з Тростянецького напрямку – про що
інформували мобільні групи та авіарозвідка. Саме тоді наше командування
ухвалило єдине тоді можливе рішення - підірвати міст через Ворсклу в
Климентовому.

Я сам заїхав в Охтирку 25 лютого – моїм завданням було, як кажуть на
військовому жаргоні, «висікати» русню. Вели розвідку, знаходили зручні місця і
стежили за силами противника - передавали координати нашій артилерії, а там
одразу розбирали к...цапів по шматочках.

\subsubsection{Підтримка охтирчан}

- Російські солдати реально не думали, що перед ними стоїть армія, яка готова
дати серйозну відсіч. І зовсім не очікували героїчного спротиву від населення –
це був найбільший сюрприз для них. Мешканці швидко пережили перший шок, а потім
навіть почали формувати маленькі партизанські загони. Російська бронетехніка
часто губилася, бо місцеві прибрали всі дорожні знаки. А ще погода була на
нашому боці: сніг танув, танки загрузали. Ще й пальне у орків закінчувалося,
тож екіпажі кидали техніку і тікали.

Я давав інтерв'ю італійському Інтернет-виданню (я ж італійською розмовляю - не
даремно там стільки прожив) - так я їм говорив, що в нас би не вийшло так
зупинити загарбників, якби не допомога населення. Охтирчани активно включилися
з перших днів, навіть з перших годин війни - в усьому: допомагали евакуйовувати
поранених, робили барикади з мішків з піском, привозили нам продукти, воду,
медикаменти... Працівники шкіл готували обіди, возили хлопцям на позиції...
Окрема історія - наші машини: вони часто потрапляли під обстріли, їх треба було
ремонтувати. А хто схоче і зможе?... Це ж в умовах війни. Охтирський механік
Сашко зібрав місцевих хлопців,і вони разом, під обстрілами, повертали в стрій
нашу техніку.

Навіть дехто з місцевих депутатів з перших днів узяв зброю і пішов в армію – на
правах звичайних бійців. Не називатиму їхні прізвища, щоб не звинувачували в
політичній рекламі (сміється). Волонтери місцеві одразу долучилися. Хтось
робив, що може і як може. Дехто підійшов серйозніше - перш ніж щось везти,
розпитували: які потрібні дрони, які приціли, які машини?... Підтримуємо
зв'язок і досі, отримуємо від них передачі вже на Харківщині.

\subsubsection{Чи було масове дезертирство?}

- До нас, навіть на передову, доходять чутки, що мер Охтирки розповідає про
масове дезертирство під час перших атак...

(прим: Охтирський міський голова Павло Кузьменко на зустрічі з охтирчанами
сказав, цитата: «Коли гахнули перші три бомби, у нас було дезертирів 2000
чоловік. Залишилось із бійців сто. Мені удалось зібрати 50 чоловік в місті на
наступний день, а ще дві машини зброї, гранат, гранатометів. Оце те ЗСУ! А
вилізло воно на восьмий день війни. І ви мені будете розповідати хто тут і що
сохранив?!».)

Так от... Коли колони русні тільки-но почали заходити, то їхня кількість значно
переважала ті сили, які були в Охтирці - пояснює Сергій. – У прикордонників,
нацгвардійців та поліції не було засобів, щоб розбити ворога. З одними
автоматами в руках танки не дуже зупиниш... Тому вони отримали наказ
відступити. Але це зовсім не дезертирство, так що ваш мер, як кажуть,
\enquote{трохи} перебільшує.

Навіщо він це робить, мені важко припускати... Кого він там збирав?... Бійці 93
бригади точно нікуди не тікали й захищали місто, як своє рідне. 

\subsubsection{Місто-герой}

- Бої за Охтирку - один із ключових етапів одразу на трьох напрямках: на
Харківському, Сумському та Полтавському. Успіх українців в маленькому містечку
не дав русні остаточно захопити ці обласні центри та прорватися далі, на Київ.
Повторюся – здійснити це вдалося виключно завдяки єднанню охтирчан та
військових. На мою особисту думку, Охтирка стала містом-героєм завдяки нашій
єдності.

\subsubsection{Бої тривають}

- Ми зупинили їх з боку Ізюма і обов'язково перемога буде за нами. Основна наша
проблема (та й усього світу) – к...цапів дуже багато. Ми їх б'ємо, а вони
сунуть! Мені дуже подобається вислів: російська армія не сильна - вона просто
дуже довга. Беруть кількістю живої сили - завалюють своїми трупами. Власних
солдатів не шкодують і навіть не забирають тіла загиблих.

\textbf{Тому наші хлопці апріорі сильніші, бо вони вмотивовані! Так що все буде
Україна!}
