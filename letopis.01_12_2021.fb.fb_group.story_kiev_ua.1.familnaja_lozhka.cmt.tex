% vim: keymap=russian-jcukenwin
%%beginhead 
 
%%file 01_12_2021.fb.fb_group.story_kiev_ua.1.familnaja_lozhka.cmt
%%parent 01_12_2021.fb.fb_group.story_kiev_ua.1.familnaja_lozhka
 
%%url 
 
%%author_id 
%%date 
 
%%tags 
%%title 
 
%%endhead 
\zzSecCmt

\begin{itemize} % {
\iusr{Оксана Дубинина}

Георгий, ничего себе! Каждый абзац Вашей публикации тянет на отдельный роман о
судьбе человека. Очень интересно, а еще чудо, как Вы сохраняете историю своего
рода. Спасибо!


\iusr{Георгий Майоренко}

Благодарю. Ещё очень важно встретить добрых людей на пути, которые подскажут и
помогут в поиске. И некоторые из этих добрых людей в группе \enquote{КИ}. Здесь я даже
родственников нашел. Так-что отдельная благодарность создателям группы и
модераторам. И с первым днём зимы!

\iusr{Татьяна Гурьева}

Блеск) как интересно читать о таких людях, поднимающих империю. Вот именно род
укрепляет человека, многое было завязано на этом. Славная история)

\begin{itemize} % {
\iusr{Георгий Майоренко}
\textbf{Татьяна Гурьева} 

Спасибо. Сейчас читаю воспоминания, переписку некоторых из тех, кого упомянул.
И таки да, интеллектуалы были высшей пробы! Причем, без айфонов, интернета.
Сейчас это сложно даже представить!

\iusr{Татьяна Гурьева}
\textbf{Георгий Майоренко} это называется порода)

\begin{itemize} % {
\iusr{Георгий Майоренко}
\textbf{Татьяна Гурьева} И колоссальный труд. Ведь многие аристократы жизнь \enquote{прожигали}. И такое было.
\end{itemize} % }

\iusr{Татьяна Гурьева}
\textbf{Георгий Майоренко} да всяко было. Главное, что в сухом остатке)

\iusr{Кретов Андрей}

Спасибо огромное за такой интереснейший материал!!!

О В. Шульгине знал, читал его мемуары и книгу, написанную после его
насильственного возвращения в Союз. Знал и что его племянник был в Раде, а вот
остальное для меня откровение. О богатейших Оболонских знал, а вот о
Жлобах... Спасибо!

\begin{itemize} % {
\iusr{Георгий Майоренко}
\textbf{Кретов Андрей} Да, слово жлобы звучит грубо. Но из песни слов не выкинешь! ))

\iusr{Rimma Turovskaya}
\textbf{Георгий Майоренко} Может быть, это сейчас звучит грубо, а в те времена это слово имело совсем другой смысл.

\iusr{Георгий Майоренко}
\textbf{Rimma Turovskaya} 

Есть версия, что в польском варианте это звучало не Жлобы - Погорельские, а
Злобы - Погоржецкие. Так что, возможно, эти люди были либо прижимистыми, либо
злобными. Хотя иногда употребляют слово жлоб по отношению к здоровяку - амбалу.

Выходили из избы
здоровенные жлобы...

В. Высоцкий

\iusr{Rimma Turovskaya}
\textbf{Георгий Майоренко} Да, вариантов много.

\iusr{Rimma Turovskaya}
\textbf{Георгий Майоренко} 

Георгий, я хотела спросить, неужели все эти материалы о своих предках Вы нашли
в архивах? Наверно, это потому, что среди них много выдающихся личностей?

\iusr{Георгий Майоренко}
\textbf{Rimma Turovskaya} 

Дело в том, что мне достался по наследству неплохой домашний архив. Да, так
совпало, что некоторые родственники были людьми незаурядными. Но эту тему я
начал подробно исследовать два года назад. До этого - все не было времени, не
доходили руки.


\iusr{Георгий Майоренко}
\textbf{Rimma Turovskaya} 

А с архивами (Киева и не только) сотрудничаю достаточно активно. А ещё в сборе
материала помогают многие замечательные люди. В том числе и из группы КИ. Под
Новый год подниму за них бокал.

\end{itemize} % }

\end{itemize} % }

\iusr{Natasha Levitskaya}

Георгий! Вам браво и восхищение! Какой огромный труд найти вот такую историю
своего рода! Интереснейшая история!

Спасибо вам за ваши публикации, за такие настоящие Киевские истории о людях,
которые эту историю делали. Такие публикации украшение нашей группы! @igg{fbicon.hands.applause.yellow} 

\begin{itemize} % {
\iusr{Георгий Майоренко}
\textbf{Natasha Levitskaya} Благодарю за высокую оценку. Ищем, стараемся. Находим - делимся. С первым зимним днём!

\iusr{Natasha Levitskaya}
\textbf{Георгий Майоренко}
Спасибо, взаимно! @igg{fbicon.snowflake}

\iusr{Светлана Манилова}
\textbf{Георгий}, после Наташи уже добавить нечего. Спасибо!

\iusr{Георгий Майоренко}
\textbf{Светлана Манилова} Весьма признателен, Светлана! Удачи и добра!
\end{itemize} % }

\iusr{Вадим Перегуда}

Дуже вдячний за публікацію, ви приклад для наслідування. Цікаво, що мій власний
рід пов'язаний також з Погорельськими, тільки Перегуда - Погорельськими гербу
Сас.

\begin{itemize} % {
\iusr{Георгий Майоренко}
\textbf{Вадим Перегуда} Спасибо! Очень интересно! А где ваши Перегуды - Погорельские проживали? Жлобы (Бублики) - Погорельские с Черниговщины.

\iusr{Георгий Майоренко}
\textbf{Вадим Перегуда} Вадим, возможно Погорельские где-то в прошлом и пересекались? Вот схема наших.

\ifcmt
  ig https://scontent-mxp2-1.xx.fbcdn.net/v/t39.30808-6/263155342_2785142971784926_6980182949990922893_n.jpg?_nc_cat=102&ccb=1-5&_nc_sid=dbeb18&_nc_ohc=OcS7JqWGYhUAX-COsip&_nc_ht=scontent-mxp2-1.xx&oh=00_AT-br_kfiOibYeyfjyw6weqiWSXz3bZPuZeXb4FFp5wWEA&oe=61DA07DD
  @width 0.6
\fi

\end{itemize} % }

\iusr{Наталья Писная}

Интерес к историям семьи возникает с нашим взрослением, когда задумываешься, а
что после нас? Но не каждому дано открыть тайны прошедших лет и столетий..

Вам удалось ! Спасибо!

\begin{itemize} % {
\iusr{Георгий Майоренко}
\textbf{Наталья Писная} 

Благодарю. Думаю, никогда не поздно заняться поисками. Просто нужно время и
знание источников. Хотя и в домашних фотальбомах можно найти ключи от многих
тайн прошлого.

\iusr{Наталья Писная}
\textbf{Георгий Майоренко} 

Спасибо за подсказку. Мои корни ( бабушка и дед) жили в Киеве, учились и
работали, но годы репресий не прошли мимо - семью рассорили, все потерялись на
долгие годы... Где только не оказались родные... А мне вот суждено было вернуться
в Киев! И очень этому рада.

\iusr{Георгий Майоренко}
\textbf{Наталья Писная} 

Надо выяснить, где предки жили до революции. И тогда уже выяснять детали.
Хорошо, если они обитали в одном районе и были прихожанами одной церкви. Тогда
в церковных книгах о них вся информация. Тогда церковные книги заменяли архивы
ЗАГСов и программу \enquote{Дiя}.

\iusr{Наталья Писная}
\textbf{Георгий Майоренко} Спасибо.
\end{itemize} % }

\iusr{Ірина Кравець}
Потрясающе!!! Как впечатляет. Вот уж история. А как написано, живо, легко...

\iusr{Георгий Майоренко}
\textbf{Ірина Кравець} Спасибо на добром слове. Удачи и добра!

\iusr{Лора Лора}

Браво, Георгий! Это огромный труд - по крупицам собирать историю своего рода!

\begin{itemize} % {
\iusr{Георгий Майоренко}
\textbf{Лора Лора} 

Спасибо, дорогая сестричка! Ты ведь тоже к этой линии имеешь некоторое
отношение! Кстати, всем нашим привет от тети Нины из Питера. Там сейчас
отмечается 30-летие степ-театра Винниченко. Вот это уже твоя близкая родня.
Удачи и здоровья!


\iusr{Alyona Artobalevska}
\textbf{Георгий Майоренко} я прочитала с огромным удовольствием, как буд-то бы побывала там, в той эпохе! Спасибо за увлекательные открытия нашего родового древа!

\iusr{Георгий Майоренко}
\textbf{Alyona Artobalevska} О, родня подтянулась! Всех наших обнимаю!
\end{itemize} % }

\iusr{Ольга Писанко}
Ооочень интересно! Особенно о Василии Шульгине!

\begin{itemize} % {
\iusr{Георгий Майоренко}
\textbf{Ольга Писанко} 

О Василии Шульгине - интереснейшая тема! Но она требует отдельного рассказа. У
него ведь и отец был известнейшим человеком в Киеве, и отчим...

\end{itemize} % }

\iusr{Раиса Карчевская}

Георгий!

Спасибо большое за Ваши потрясающе интересные воспоминания

\begin{itemize} % {
\iusr{Георгий Майоренко}
\textbf{Раиса Карчевская} Благодарю, Раиса. Успехов и благополучия!
\end{itemize} % }

\iusr{роман чудновский}
Знатно @igg{fbicon.hands.applause.yellow}  @igg{fbicon.sun.with.face}  @igg{fbicon.smile} 

\iusr{Георгий Майоренко}
\textbf{роман чудновский} Та да @igg{fbicon.face.tears.of.joy} 

\iusr{Svitlana Plieshch}

Спасибо вам за историю. Так много событий, что можно книгу писать. Вдохновляете
на изучение истории своего рода.

\begin{itemize} % {
\iusr{Георгий Майоренко}
\textbf{Svitlana Plieshch} 

Благодарю. Жаль, что мало документов передается от поколения к поколению. Вот
потом и ломай голову, что это были за люди? Но иногда удается по крупицам
собрать материал.

\iusr{Svitlana Plieshch}
\textbf{Георгий Майоренко} ощущение было такое, что вы о них знаете почти все.

\iusr{Георгий Майоренко}
\textbf{Svitlana Plieshch} 

О некоторых - немало. Но белых пятен - уйма! Понятно, что нельзя объять
необъятное. Стараюсь ориентироваться на минимум. Главное, что есть на свете
добрые люди, которые могут помочь полезными советами.

\iusr{Svitlana Plieshch}
\textbf{Георгий Майоренко} удачи в поисках информации, вдохновения и новых, интересных историй. Ждем
\end{itemize} % }

\iusr{Марина Соболевская}

Пишете замечательно! Биографии - таким легким и увлекательным слогом!
Изумительно.)

\begin{itemize} % {
\iusr{Георгий Майоренко}
\textbf{Марина Соболевская} 

Спасибо, Марина за добрые слова! Это краткое изложение. У некоторых личностей
из сюжета интереснейшие судьбы, рассказ о них потянет на целую книгу. А о
некоторых известна самая малость...

\begin{itemize} % {
\iusr{Марина Соболевская}
\textbf{Георгий Майоренко}, 

написать родословные книги увлекательно- непростая задача.)) У самой такая
цель. Но еще не уложились мысли - как совместить цифры, биографии с семейными
историями. У вас это чудесно получилось.)

\iusr{Георгий Майоренко}
\textbf{Марина Соболевская} 

Цель благородная! Удачи! Если будут вопросы в техническом плане - обращайтесь.
Мне самому помогают люди, которые в этих вопросах - \enquote{собаку съели}!

\iusr{Марина Соболевская}
\textbf{Георгий Майоренко} , спасибо!) Напишу вам в личку.)
\end{itemize} % }

\end{itemize} % }

\end{itemize} % }
