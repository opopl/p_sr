% vim: keymap=russian-jcukenwin
%%beginhead 
 
%%file slova.gora
%%parent slova
 
%%url 
 
%%author_id 
%%date 
 
%%tags 
%%title 
 
%%endhead 
\chapter{Гора}

%%%cit
%%%cit_head
%%%cit_pic
%%%cit_text
Було се 1241 року. Весна стояла в тухольських \emph{горах}.  Одної прегарної
днини лунали лісисті пригірки Зелеменя голосами стрілецьких рогів і криками
численних стрільців.  Се новий тухольський боярин, Тугар Вовк, справляв великі
лови на грубу звірину. Він святкував почин свого нового життя,— бо недавно
князь Данило дарував йому в Тухольщині величезні полонини і ціле одно пригір'я
Зелеменя; недавно він появився в тих \emph{горах} і побудував собі гарну хату і
оце першу учту справляє, знайомиться з довколичними боярами. По учті рушили на
лови в тухольські ліси
%%%cit_comment
%%%cit_title
\citTitle{Захар Беркут}, Іван Франко
%%%endcit
