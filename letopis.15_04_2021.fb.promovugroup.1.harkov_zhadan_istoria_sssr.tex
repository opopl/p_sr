% vim: keymap=russian-jcukenwin
%%beginhead 
 
%%file 15_04_2021.fb.promovugroup.1.harkov_zhadan_istoria_sssr
%%parent 15_04_2021
 
%%url https://www.facebook.com/groups/promovugroup/permalink/953551241885404/
 
%%author 
%%author_id 
%%author_url 
 
%%tags 
%%title 
 
%%endhead 
\subsection{Чому Харків не відчуває себе загалом українскьким - Жадан - Савчук}
\label{sec:15_04_2021.fb.promovugroup.1.harkov_zhadan_istoria_sssr}
\Purl{https://www.facebook.com/groups/promovugroup/permalink/953551241885404/}

Подивилася відео Жадана, де він розмовляє з Олександром Савчуком про Шевельова
і Харків. Вони там розмірковують чому Харків не відчуває себе загалом
українскьким, хоча в центрі майже кожна будівля це про українську історію — ось
будинок, в якому жив Шевельов, а ось в цьому Микола Хвильовий, а ось театр
"Березіль", а в цьому будинку проходив суд по справі СВУ.  

В принципі, у мене є
власна версія. Я писала про це. СРСР робив все, щоб вбити локальну історію. Не
просто українську, а взагалі локальну. В Харкові не було нічого. Ленін сюди не
приїздив, з'їзд РСДРП не проходив. А більше нічого видатного і не відбувалося.
Якась історія була в Москві і Санкт-Петербурзі. Ясно, що там Кремль, Зимовий
палац, а також Васнєцови-Брюлови. Рєпін? Ось хороший приклад. В Харкові в музеї
висить один з варіантів \enquote{Козаків}, оригінал, на хвилиночку. І? Нам про це
навіть не розповідали в школі. Тобто, те, що Рєпін народився в Чугуєві начебто
і не було секретом, але екскурсію в Чугуїв не робили, до музею клас не водили
подивитися на "Козаків", ніяк цю тему не педалювали. Натомість, піонерська
дружина носила ім'я Зої Космодем'янської. 

А центральний парк імені Горького. А
не Макаренка, наприклад. Тобто, знову ж таки, не те, щоб це був секркт, але про
Макаренко і ФЕД я дізналася з книги, а не в школі. (я наводжу Макаренка не як
важливу для мене постать, а як суто радянську фігур, цілком з радянського
пантеону). Книжка \enquote{вулиці і площі Харкова} була максимально знеособленою. Про
ОЛешу і Катаєва я теж дізналася з \enquote{Алмазный мой венец}, хоча теж вони були
цілком радянськими письменниками. Тобто, місту заборонялося мати власне
обличчя. НА канікулах в університеті подруга запросила мене до себе в Черкаси,
і я була вражена, майже шокована тим, що під Черкасами є дом-музей
Чайковського, бо всім відомо, що Чайковський жив в Кліну. Розумієте? От так це
працювало. Не те, щоб в школі не розповідали про античні грецькі міста
Причорномор'я, але робили це так само, як і про Рєпіна. Греки жили в Греції, шо
не ясно? Як і Рєпін в Санкт-Петербурзі, і Чайковський так само. У вас не було
нічого цікавого, а хто народився, той поїхав в Москву, бо там центр Всесвіту. 

В Харкові вперше в СРСР розщепили атом, тут жив Курчатов, Синельніков і
Вальтер. Про Вальтера я дізналася від батька, про Синельникова коли вперше
зайшла в аудиторію Синельникова. Знову ж таки, жодного шкільного утрєнніка про
мирний атом, хоча портрет Курчатова висів у кабінеті фізики. Атом, діти,
розщепили в СРСР, а не в Харкові.

Це настільки глибока прошивка, що 70\% харків'ян досі не можуть її позбутися.
