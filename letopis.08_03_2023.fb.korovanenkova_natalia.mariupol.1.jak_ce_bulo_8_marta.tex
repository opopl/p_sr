%%beginhead 
 
%%file 08_03_2023.fb.korovanenkova_natalia.mariupol.1.jak_ce_bulo_8_marta
%%parent 08_03_2023
 
%%url https://www.facebook.com/natali.korovanenkova/posts/pfbid0QKv6dTER9pts224jUDjvSEc57SG3V9DiTLN1Yw1xcXbLN5wHoT4ziLG2nfEqy3MTl
 
%%author_id korovanenkova_natalia.mariupol
%%date 08_03_2023
 
%%tags mariupol,mariupol.war,dnevnik,08.03.2022,album.korovanenkova.jak_ce_bulo
%%title 8 марта 2022 - Утро, у костра уже собрался народ
 
%%endhead 

\subsection{8 марта 2022 - Утро, у костра уже собрался народ}
\label{sec:08_03_2023.fb.korovanenkova_natalia.mariupol.1.jak_ce_bulo_8_marta}

\Purl{https://www.facebook.com/natali.korovanenkova/posts/pfbid0QKv6dTER9pts224jUDjvSEc57SG3V9DiTLN1Yw1xcXbLN5wHoT4ziLG2nfEqy3MTl}
\ifcmt
 author_begin
   author_id korovanenkova_natalia.mariupol
 author_end
\fi

\#маріуполь\_2022 \#як\_це\_було

8 марта 2022

Утро, у костра уже собрался народ, подьехала полицейская машина,
разговаривает с нашими мужчинами.. я только выхожу с подьезда и начинаю
задавать вопроси, на которые у них нет ответа.. но сказали, что можно
составить список коллективный и отнести в красный крест, который находится
возле ценрального ровд и получить продукты и медикаменты.

Быстро пишу список. Но идти никто не хочет, у некоторых еще есть продукты,
некоторые собираются выехать, другие говорят, "завтра может пойдем"...

По району вроде не сильно лупят, идти пеше от нас 3-4 остановки по проспекту, 

Нас четверо, четыре женщины,  перебежками  по поспекту ...  через мост, на
Кирова и дальше к ровд.

\ii{08_03_2023.fb.korovanenkova_natalia.mariupol.1.jak_ce_bulo_8_marta.pic.1}

Вид улицы привел в шок, все магазины, аптеки, табачные киоски были разбиты,
варварски разграблены, люди тачками везли всякую хрень вместе со стелажами и
прилавками

На заборе "Гусей" была надпись большими буквами - "Охраняется, стреляю без
предупреждения"

Обувной магазин "Мида", сиротливо в разбитые окна смотрели остатки обуви,

Дальше  несколько мужиков пытаются вскрыть магазин.

Везде хаос, со всех сторон стреляет и летает..

Несколько раз приходилось падать на асфальт или под лавку, как получалось(

Наконец мы добрались до ровд,  там еще стояли машины и были полицейские, на
наше обращение к ним,  что за углом, в двух шагах вскрывают магазин, они
просто отмахнулись.

Возле красного креста очень много людей, все просят, никому ничего не дают,
отсылают писать списки..

А мы со списком, огромным на весь дом и у нас его принимаю )

Наследующий день надо прийти  забрать гуманитарку.

У нас получилось ! и мы неспешно возвращаемся домой.

Возле ровд стоит полицейская машина и полицейский смачно жует шоколадку
"милкивей", мы при виде шоколада чуть слюной не подавились, коллективно..

"Я пойду попрошу", - соседки хватают за руку и шикают на меня:"Не
надо! Стыдно! Пойдем..."

Но сегодня же 8 марта   и шоколадка перед глазами, я уже остановиться не могу.

Подхожу, поздравляю с праздником, предлагаю поделиться. В ответ поздравляет
нас и открывает заднюю дверцу маши, машина доверху завалена шоколадками.... и
дает нам одну "милкивей"...

П.с. я думала какую картинку поставить к 8 марта, но утром пришло сообщение с
Мариуполя  от соседки, с которой мы ели "милкивей"🍫

%\ii{08_03_2023.fb.korovanenkova_natalia.mariupol.1.jak_ce_bulo_8_marta.eng}
%\ii{08_03_2023.fb.korovanenkova_natalia.mariupol.1.jak_ce_bulo_8_marta.cmt}
