% vim: keymap=russian-jcukenwin
%%beginhead 
 
%%file 18_10_2021.fb.bilchenko_evgenia.5.pesenka_pro_kota
%%parent 18_10_2021
 
%%url https://www.facebook.com/yevzhik/posts/4353432778025132
 
%%author_id bilchenko_evgenia
%%date 
 
%%tags bilchenko_evgenia,poezia
%%title БЖ. Песенка про кота и ещё одного кота
 
%%endhead 
 
\subsection{БЖ. Песенка про кота и ещё одного кота}
\label{sec:18_10_2021.fb.bilchenko_evgenia.5.pesenka_pro_kota}
 
\Purl{https://www.facebook.com/yevzhik/posts/4353432778025132}
\ifcmt
 author_begin
   author_id bilchenko_evgenia
 author_end
\fi

БЖ. Песенка про кота и ещё одного кота

\ifcmt
  ig https://scontent-lga3-1.xx.fbcdn.net/v/t1.6435-9/246267797_4353432634691813_5882621707905214237_n.jpg?_nc_cat=108&ccb=1-5&_nc_sid=8bfeb9&_nc_ohc=7-NnNYkTrhkAX-p3yN8&_nc_ht=scontent-lga3-1.xx&oh=4ade4c95f92a93c03694daa9ff667a99&oe=6192EAE6
  @width 0.4
  %@wrap \parpic[r]
  @wrap \InsertBoxR{0}
\fi

В свитере салатовым,
С томиком Ахматовой,
В шерстяной панаме, -
Страж при Мандельштаме.
Рыженький, как Лёва, -
Паж при Гумилёве.
Среди арок Росси
Ветер его носит.
В разноцветных тиграх
Гетров, будто титры,
Лапушки мелькают.
Жизнь его - такая
Нежная подруга,
Барыня, прислуга.
Гордая царица,
Дворовая птица.
Голубей гоняя,
Бдит он, словно няня,
За торговлей бойкой,
За музейной Мойкой.
За туристом-дурнем,
За бычками в урне.
За идеей фикс у
Всех пощупать Сфинкса.
Что ему Египет,
Если млечный выпит
Путь его котовий
Из небесной крови?
Здравствуй, шёрстка дыбом.
Жаль, что нет квартиры.
Мы под небом дымным 
Два ваяем мира
Строгими резцами
Ласкового Клодта.
Принимай же, царе,
Знатную работу.
Зыркнул на задумку:
- Это же лошадки!
Расплескала рюмку.
Уронила шапку.
В свитере малиновом
С томиком Марининым,
Знай, стою мечтаю
Ублажить кота я.
Только дома свой есть:
Страсть моя и совесть.
Злится целый вечер
Кот мой человечий.
- Хочешь Монферрана?
- Хочешь Моргенштерна?
Трагика и рана.
Юмор и цистерна.
- Хочешь руку в лапу?
- Хочешь "вместо лампы"
Из решёток сосен
Солнечную осень?
В худи цвета олова
С томиком Лимонова
Вслед его судьбе
Тороплюсь я, но  гуляет,
Огнь над градом распыляя,
Сам он по себе.
Сам он по себе.
18 октября 2021 г.
