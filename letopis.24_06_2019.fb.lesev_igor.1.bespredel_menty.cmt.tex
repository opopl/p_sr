% vim: keymap=russian-jcukenwin
%%beginhead 
 
%%file 24_06_2019.fb.lesev_igor.1.bespredel_menty.cmt
%%parent 24_06_2019.fb.lesev_igor.1.bespredel_menty
 
%%url 
 
%%author_id 
%%date 
 
%%tags 
%%title 
 
%%endhead 
\subsubsection{Коментарі}
\label{sec:24_06_2019.fb.lesev_igor.1.bespredel_menty.cmt}

\begin{itemize} % {
\iusr{Павло Бубенко}
Ща АБА прочтет - и фсе...Пиши - пропало...

\iusr{Марина Прохорова}
В Украине нельзя только оспаривать принадлежность Крыма и войну с Россией. Всё остальное - можно (

\iusr{Виктор Данченко}
да, люди озлоблены, и как обычно это выливается на более слабых, а этим еще дали власть и оружие

\iusr{Дмитрий Коломийченко}
У нас всё чаще наблюдаю как частные охранники приезжают разбираться. Люди тревожные кнопки жмут.

\begin{itemize} % {
\iusr{Игорь Лесев}
от охраны иногда больше толку, те хоть с понтами побазарить могут, а эти как неандертальцы с оружием

\iusr{Дмитрий Коломийченко}
\textbf{Игорь Лесев} Так там практически все бывшие менты. По крайней мере у нас в Харькове так. В общем частная полиция тебе бережёт. Купи тревожную кнопку и спи спокойно.
\end{itemize} % }

\iusr{Оксана Супрунюк}

Я вспоминаю старого участкового. Гад был еще тот. Но краденное находил как
породистая ищейка. Знал всех неблагополучных на районе, все малины,
притоны, точки по продаже наркоты и самогона. Его боялись как огня. Им пугали
всю шпану. Одно его имя наводило ужас на домашних дебоширов. Вычистили. Не
прошел переаттестацию. Теперь чуть ли не каждый месяц у нас обносят квартиры.

\begin{itemize} % {
\iusr{Фролов Руслан}
\textbf{Оксана Супрунюк} а гад-то почему?

\iusr{Оксана Супрунюк}
\textbf{Фролов Руслан} Он всем тыкал, невзирая ни на возраст, ни на пол, матерщинник был жуткий. Грубый и хамоватый. Ну и поборами мелкими не брезговал. Ментяра в общем, цепной пес. Таких всегда недолюбливают)

\iusr{Роман Соломашко}

Вот такими по сути и были старые менты. Но... Старая гвардия знала Закон. И
если мент понимал, что он его преступает, то делал это так, что комар носа не
подточит. Эти, новоряженные, усвоили со всего только один закон, подчиняйся,
потом оспаривай. Мол, в суде разберутся. А суд, для простого украинца, это
нечто, как собрание богов Олимпа. Да и пошлины такие, что не каждый и в
состоянии оплатить.

\end{itemize} % }

\iusr{German Gorozhanski}

Кажись у Салтыкова-Щедрина есть: Чтоб бояться судебного пристава достаточно
одного только страха, совсем не обязательно быть в чем-то виноватым.

\iusr{Полина Дьяченко}

Читала что этим новым полициянтам просто давали ключи от новых машин. Если
разбил даже выговоров не делали. Они как-бы не совсем в законе или нет на них
законов. Они не несут должной ответственности. Такая потешная полиция

\iusr{Marina Garbuzova}
Браво!!!

\iusr{Матвей Кублицкий}

всё логично. старых зачистили. пришли новые. передача опыта не произошла. новые
знают о работе в полиции только из фильмов и телепередач. впитали всё самое
плохое. Так сказать - отразили в полиции все современные тенденции украинского
общества

\iusr{Елена Скачко}
Ужас. ..ребенок один в машине. . Они вообще адекватные? Неужели никто не был наказан? Об остальном вообще молчу ...

\begin{itemize} % {
\iusr{Olha Pokotylo}
ребенок остался с дядей - знакомым или родственником пары, который парковался во втором ряду и с которого все началось, на целых минут 40.
\end{itemize} % }

\iusr{Якимец Евгений}
Да расстреливать этих мразей или публичное гражданское задержание с пожизненной посадкой.

\iusr{Игорь Чечко}
Вот поэтому мы отказались от идеи поехать в гости в Одессу. У меня осталась одна жизнь и совсем нет запасных детей.

\iusr{Oleksandr Novokhatskyi}

Полиция, это инструмент. Какую функцию закладывают для этого инструмента так он
и работает. От функции, этот инструмент и наполняется кадрами. Какая цель
существования сегодняшней, "реформированной" полиции? Ответ очевиден - селфи
делать.

Чего заложили "на входе", то и получили "на выходе".

Цель, функция, назначение, - только эти "пустые слова" определяют то, что мы
получим в результате.

Так что, "хлебайте это счастье ситичком" ...

\iusr{Walentin Melnikow}
Хороший тэзис, случайные люди в погонах.

\iusr{Elena Godzynskaya}
у нас полицайки ходят в мини, в ботфортах и с дамскими сумочками) ролевые игры бл***

\iusr{Сергей Побережний}
Вместо ожидаемой реформы мы получили стон Общества ! от того, что должно быть вершиной правосудия.

\iusr{Лебідь Михайло}
Так вам же ... екс-менти були погані! Чого ж плачетесь? Насолоджуйтесь "реформою", затіяною бандою при владі та ... селфуйте!

\iusr{Sergey Labuz}
Беззаконие и беспредел встречается везде

\iusr{Леонид Якимов}
Последствия революции ... достоинства. Ключевое слово - революция.

\end{itemize} % }
