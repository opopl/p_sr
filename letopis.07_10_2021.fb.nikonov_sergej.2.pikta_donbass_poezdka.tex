% vim: keymap=russian-jcukenwin
%%beginhead 
 
%%file 07_10_2021.fb.nikonov_sergej.2.pikta_donbass_poezdka
%%parent 07_10_2021
 
%%url https://www.facebook.com/alexelsevier/posts/1597526033926039
 
%%author_id nikonov_sergej,pikta_svetlana
%%date 
 
%%tags donbass,pikta_svetlana,poezdka,reportazh,vojna
%%title Светлана Пикта - поездка на Донбасс - Улица Стратонавтов. Аэропорт. Иверский
 
%%endhead 
 
\subsection{Светлана Пикта - поездка на Донбасс - Улица Стратонавтов. Аэропорт. Иверский}
\label{sec:07_10_2021.fb.nikonov_sergej.2.pikta_donbass_poezdka}
 
\Purl{https://www.facebook.com/alexelsevier/posts/1597526033926039}
\ifcmt
 author_begin
   author_id nikonov_sergej,pikta_svetlana
 author_end
\fi

Прочитал и снова зарепостил ещё один день поездки Светланы по Донбассу. Аналогичные поездки, насколько я знаю, только Татьяна Монтян делала и делают люди Анатолия Шария. Снова без комментариев и на чужие комментарии не реагирую. 

\ifcmt
  tab_begin cols=3

     pic https://scontent-lga3-2.xx.fbcdn.net/v/t1.6435-9/244662686_1597585023920140_4793344893389490118_n.jpg?_nc_cat=101&ccb=1-5&_nc_sid=730e14&_nc_ohc=B6VKifamaPIAX8OqOXq&_nc_ht=scontent-lga3-2.xx&oh=d33192eb4ac9f4910e76c27b0757b29a&oe=6185D679

     pic https://scontent-lga3-2.xx.fbcdn.net/v/t1.6435-9/244783554_1597585080586801_5719100106409123975_n.jpg?_nc_cat=103&_nc_rgb565=1&ccb=1-5&_nc_sid=730e14&_nc_ohc=gU4KV_cUSBUAX8PgzBP&_nc_ht=scontent-lga3-2.xx&oh=bce749ee658748e02a0777e302c35142&oe=6184F83E

		 pic https://scontent-lga3-2.xx.fbcdn.net/v/t1.6435-9/244661710_1597585233920119_4641247243115651384_n.jpg?_nc_cat=100&_nc_rgb565=1&ccb=1-5&_nc_sid=730e14&_nc_ohc=bh3DAqwT2bEAX_1eUxX&_nc_ht=scontent-lga3-2.xx&oh=8cd8e193aecca4cd749b0a215ca90c6c&oe=6185B760

  tab_end
\fi

Улица Стратонавтов. Аэропорт. Иверский.
- Что-то мы поздно едем, в это время все сматываются отсюда, - задумчиво произнёс Андрей, наблюдая, как навстречу нам выезжают редкие машины.
Асфальт весь в дырах от снарядов, старые дырки подлатаны, свежие - приходится объезжать.
Путиловский мост разрушен и закрыт, проезжая по другой ветке автомоста, Андрей вдруг открывает окно и орёт с высоты куда-то через посадку:
- Ukr\#\#py p\#\#sy. - и оборачиваясь ко мне , как ни в чём ни бывало, поясняет - Традиция у меня такая.
Энергии у этого человека, как у атомной станции. Я провела в поездках с ним всего два дня, но мне кажется, что я уже вечность с ним езжу "по адресам". Меня абсолютно не напрягает в нём то, что напрягало бы в любом другом: ни то что он курит, ни то что может иногда по-простому, матом. У него есть главное, какой-то несгораемый вечный двигатель внутри, и я смотрю на этого человека с открытым ртом. "Если есть в этом мире святость, то она выглядит именно так" - размышляю я.
Между тем мы въезжаем на Стратонавтов, впереди огромное солнце, оно слепит нас. "Едем на солнце, вдоль рядов кукурузы".
- Что ты там снимаешь, ничего же не видно!
- Солнце. Красиво! - поясняю я.
- На обратном пути лучше будет видно, здесь были очень зажиточные дома, а сейчас это "улица, которой нет".
- Вообще никто не живёт?
- Нет, живут. Вот видишь, даже автобус.
Мы увидели, как со Стратонавтов уезжает последний автобус. 18.00, чётко по расписанию. Да. В Красной зоне по "улице, которой нет" ходит автобус, ходит по расписанию, несмотря на постоянные обстрелы, горы развалин и выбоины от снарядов. Медики, соцработники, коммунальщики, водители - в критический момент эти незаметные люди стали героями, причём совершающими свой подвиг не однократно, а ежедневно.
И снова я размышляю о святости. Здесь приходит очень много мыслей о смысле жизни. Не в том ли он, чтобы стать святым?
Громадное закатное солнце приводит нас прямо к Иверскому монастырю. Множество раз я видела эту русскую святыню на фотографиях, но реальность не имеет ничего общего с изображением. Когда ты идёшь вдоль аллеи роз, подворачивая ноги на выбоинах от снарядов, твоё паломничество к этому знаменитому израненному храму, к дверям, закрытым на крест, становится главным событием твоей жизни, после которого ты уже не станешь прежней.
На обратном пути захожу на кладбище. Мраморные плиты, памятники, ограды - всё разбито, искрошено. И мёртвых не пощадили.
Останавливаюсь возле памятника со сбитой головой. бюст стоит наверху, а голова внизу, она лежит прямо под надписью "гори, гори, моя звезда"...
Возвращаюсь к машине.
Андрей рассказывает, как попал здесь под мощный артиллерийский огонь.
- Поехали назад, здесь очень опасно после захода солнца находиться.
Едем по Стратонавтов. Теперь видно всё: дыра в стене с торчащим диваном, обугленные крыши, горы развалин и мусора возле каждого дома. Попало везде и всем. Не через раз - каждому.
- А вот знаменитая "девятка". По ней выпустили 26 пакетов града.
- Что такое девятка? Улица?
- Нет, девятиэтажка. Дом.
- По одному дому 26 пакетов??
- Да. Но она, кстати, выстояла.
Смотрю на зияющую дырами девятиэтажку. "Значит не так уж и плохо строили", - думаю, вспоминая Спитак и Ленинакан...
Выезжаем наконец-то в город. Красивый, просторный, чистый, освещенный фонарями. Контраст поражает. Ведь всего-то каких-то пять минут отделяют ад от цивилизации. Какая тонкая грань отделяет благополучие от беды, жизнь от смерти! Это же просто паутина между ними, не более!
Через сутки, стоя посреди равнодушной толпы на Павелецком вокзале в Москве, я ощутила, что в Донецке с меня будто стёрли огромный слой пыли. И даже местами вместе с кожей.
