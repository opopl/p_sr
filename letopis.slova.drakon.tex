% vim: keymap=russian-jcukenwin
%%beginhead 
 
%%file slova.drakon
%%parent slova
 
%%url 
 
%%author 
%%author_id 
%%author_url 
 
%%tags 
%%title 
 
%%endhead 
\chapter{Дракон}
\label{sec:slova.drakon}

%%%cit
%%%cit_head
%%%cit_pic
%%%cit_text
«Единственный способ избавиться от \emph{дракона}, это иметь своего собственного» -
этот высказывание из пьесы Евгения Шварца «Дракон» представляется обличением
образа мысли людей, воспитанных диктатурой, людьми свободномыслящими, власти
\emph{«дракона»} оппонирующими. Между тем написана была аллегорическая пьеса в 1944
году. Когда стало понятно, что противостоять \emph{немецкому дракону} оказался
способен лишь не менее страшный для европейцев \emph{дракон советский}.  Сегодня
Россия оказалась лицом к лицу с \emph{драконом} западной тоталитарной диктатуры,
основанной на гуманистических, но абсолютно античеловеческих, на поверку
«ценностей», стремящегося к ее уничтожению и поглощению. Имеет в тылу \emph{дракона
китайского} с его не менее тотальным контролем над личностью и собственной
национал-коллективистской этикой. Не столь агрессивный в отношении России, но
вряд ли откажется поживится богатыми ресурсами остатками туши. В подбрюшье у
России вскармливается тем же Западом \emph{дракон} Великого Турана, с элементами
исламской этической диктатуры. Причем выращивается он с целью стравливания его
с Россией, так же как выращивался для этой цели национал-социалистический
гитлеровский режим в Германии
%%%cit_comment
%%%cit_title
\citTitle{Революция Духа – единственный путь спасения России}, 
Юрий Барбашов, voskhodinfo.su, 30.06.2021
%%%endcit

