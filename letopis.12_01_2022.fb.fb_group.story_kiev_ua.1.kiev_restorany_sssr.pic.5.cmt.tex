% vim: keymap=russian-jcukenwin
%%beginhead 
 
%%file 12_01_2022.fb.fb_group.story_kiev_ua.1.kiev_restorany_sssr.pic.5.cmt
%%parent 12_01_2022.fb.fb_group.story_kiev_ua.1.kiev_restorany_sssr
 
%%url 
 
%%author_id 
%%date 
 
%%tags 
%%title 
 
%%endhead 

\iusr{Виктор Марченко}
Интересно, а сколько сейчас стоит заливная осетринка?? @igg{fbicon.face.wink.tongue} 

\iusr{Дмитро Кальной}
\textbf{Виктор Марченко} 

цена почти не изменилась. Серьёзно.

Как и тогда, около 1/111...1/125 от средней зарплаты в городе. 185 грн. Чтоб не
рекламировать, можно погуглить \enquote{Осетр заливний 185.00})

\iusr{Виктор Марченко}
\textbf{Дмитро Кальной} 185 грн. за 100 грамм. А каков был вес той порции осетрины в ресторанном меню? @igg{fbicon.grin} 

\iusr{Alexander Galperin}
\textbf{Виктор Марченко} Каким был реальный вес нужно у повара и официанта спросить. @igg{fbicon.face.smirking} 

\iusr{Дмитро Кальной}
\textbf{Виктор Марченко} надо признать, скорее всего 200 г, но ведь в те времена за сам вход в ресторан обычно надо было заплатить от рубля, а то и дороже.

\iusr{Надежда Владимир Федько}
\textbf{Виктор Марченко} Тоді зарплата молодого спеціаліста-інженера була 95-100 рублів мінус податок з доходу і мінус податок за малосімейність.

\iusr{Виктор Марченко}
Я в 1971 році, як молодий лікар, отримував 90 руб. Потім 120 руб. @igg{fbicon.grin} 
