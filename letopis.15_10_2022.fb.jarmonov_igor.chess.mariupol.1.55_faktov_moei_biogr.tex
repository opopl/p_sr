%%beginhead 
 
%%file 15_10_2022.fb.jarmonov_igor.chess.mariupol.1.55_faktov_moei_biogr
%%parent 15_10_2022
 
%%url https://www.facebook.com/yarmonovip/posts/pfbid0HKsSkXHkSEtgeoUud4MY9LFpJCxgbMPgHrJZaM4x95aSfMmFgiEWdBfkbYdBR5PYl
 
%%author_id jarmonov_igor.chess.mariupol
%%date 15_10_2022
 
%%tags mariupol,chess,jubilej
%%title 55 фактов моей биографии (к своему юбилею)
 
%%endhead 

\subsection{55 фактов моей биографии (к своему юбилею)}
\label{sec:15_10_2022.fb.jarmonov_igor.chess.mariupol.1.55_faktov_moei_biogr}

\Purl{https://www.facebook.com/yarmonovip/posts/pfbid0HKsSkXHkSEtgeoUud4MY9LFpJCxgbMPgHrJZaM4x95aSfMmFgiEWdBfkbYdBR5PYl}
\ifcmt
 author_begin
   author_id jarmonov_igor.chess.mariupol
 author_end
\fi

\ii{15_10_2022.fb.jarmonov_igor.chess.mariupol.1.55_faktov_moei_biogr.pic.1}

55 фактов моей биографии (к своему юбилею): 

1.Свой юбилей я встречаю на Святой земле.

2.Родился в полярном поселке Никель на Кольском полуострове.

3.Первенец в семье.

4.Появился на свет с тяжелой родовой травмой. Врачи предрекали медленное
угасание, в лучшем случае жизнь в вегетативном состоянии. 

5.В ответ на страшный врачебный диагноз и предложение медиков написать отказ от
новорожденного ребенка, мои родители поклялись сделать все, чтобы поставить
меня на ноги, и вырастить успешным человеком. 

6.Назвали меня в честь князя Игоря.

7.Мои родители Петр Михайлович и Валентина Павловна, ветераны труда
металлургического комбината \enquote{Азовсталь} г. Мариуполя. 

8.Первую букву \enquote{О} произнес в 4 года.

9.Первая \enquote{серьезная} книга – \enquote{Робинзон Крузо}. Папа читал ее мне перед сном,
когда мне было 4-5 лет. 

10.Любимая моя книга в детстве \enquote{Необычайное путешествие Нильса с дикими
гусями}. После чтения таких книг у меня развилась любовь к путешествиям. 

11.В школьные годы находился на домашнем обучении. Был \enquote{хорошистом}.  

12.Любимый школьный предмет в старших классах: литература.

13.В подростковом возрасте (в 6-8 классах) перенес несколько хирургических
операций, после чего начал ходить с тростью.

14.В шахматы научился играть в 6 лет.

15.Первый тренер по шахматам – папа Петр Михайлович.

16.Первая шахматная книга: \enquote{Книга начинающего шахматиста} В.Левенфиш.

17.2-й разряд по шахматам мне присвоили заочно. В начале 90-х гг. была
телевизионная программа \enquote{Шахматная школа}, которую вел Ю.Авербах. Я решил
правильно 9 из 10 заданий. Просто по почте прислали справку о присвоении 2
разряда.

18.Первая моя шахматная задача Мат в 2 хода опубликована в городской газете
\enquote{Приазовский Рабочий} 1983 г., было мне тогда лет 15.

19.В 1986 г. занял 3 место в Чемпионате шахматного клуба Мариуполя (Жданов) и
выполнил норму 1 разряда.

20.В 1988 г. выполнил норму К/МС. 

21.В 1990 г. впервые стал чемпионом города Мариуполя (Жданов).

22.Являюсь 4-х кратным чемпионом города Мариуполя (Жданов).

23.В 2000 г. первый раз стал чемпионом Украины среди шахматистов с поражением
опорно-двигательного аппарата. 

24.Являюсь 10-кратным чемпионом Украины среди шахматистов с поражением
опорно-двигательного аппарата (последний раз – в 2019 г.).

25.Первый раз стал чемпион мира в 2002 г. (будучи кандидатом в мастера спорта
дебютировал на чемпионате мира в г. Грабине, Чехия).

26.Являюсь 5-кратным чемпионом мира IPCA (2002, 2013, 2016, 2018, 2019 гг.).

27.В 2002 г. на Конгрессе во время Всемирной шахматной Олимпиады в г. Блед
(Словения) мне присвоено звание \enquote{международный мастер}.  

28.В 2004 г. - присвоено звание \enquote{мастер спорта Украины}.

29.В 2006 г. - присвоено звание \enquote{Заслуженный мастер спорта Украины}.

30.В 2002 г. впервые дебютировал на Всемирной шахматной Олимпиаде  (г.Блед,
Словения) в составе международной команды шахматистов-опорников (IPCA).

31.Всего принял участие в 8 (!) Всемирных шахматных Олимпиадах (2002, 2006.
2008, 2010, 2012, 2014, 2016, 2018 гг.).

32.Увлечение шахматной композицией началось в подростковом возрасте, с участия
в конкурсах решения в местных городских газетах. 

33.Всего опубликовано более 350 шахматных задач и этюдов в различных изданиях
(в том числе и зарубежных).

34.В Альбомах ФИДЕ помещены мои 20 задач и 1 этюд (на данный момент). 

35.В 2011 г. мне присвоено звание \enquote{мастер ФИДЕ} по шахматной композиции. 

36.В 2012 г. в Киеве состоялась самая памятная встреча – с Юрием Львовичем
Авербахом, мудрейшим и интереснейшим человеком.

37.В 2012 г. я выиграл конкурс по решению. На решение 5 шахматных задач мат в 2
хода отводилось 30 мин. Мне, единственному из участников, удалось решить все
задания (за 15 мин.).

38.Памятный специальный Кубок, диплом и книгу Ю.Авербаха \enquote{О чем молчат фигуры}
с авторской подписью я получил из рук самого Юрия Львовича! 

39.В 2013 г. я встретил свою жену Галину. Познакомились мы, как волонтеры, в
Мариуполе. 

40.Памятная поездка в Одессу в сентябре 2013 г. с женой Галиной. Встреча с
известными одесскими шахматными композиторами: С.Н.Ткаченко, Ю.Гордиааном,
В.Черноусом. 

41.В 2014 г. я впервые поехал с женой за границу, на свою 6-ю Всемирную
шахматную Олимпиаду в г. Тромсе, Норвегия. Для Галины это была первая Всемирная
шахматная Олимпиада! 

42.В 2014-15 гг. мы с женой оказались в г. Лютеж, во Всеукраинском
профессионально-реабилитационном Центре. И до сих пор общаемся с участниками
нашей учебной группы ОКН-96. 

43.В 2020 г. вышла в печать книга \enquote{Шахи – можливості без меж}. Мне довелось
закончить грандиозный труд безвременно ушедшей из жизни мариупольской
писательницы Г.Гаевской, книгу об истории зарождения и развития шахматного
движения среди людей с поражением опорно-двигательного аппарата в Украине.  

44.Любимое лакомство – черный твердый шоколад (без добавок). 

45.В жизни ценю стабильность, комфорт.

46.Не люблю суету и хаос. 

47.Терпеть не могу хамство и насилие.

48.Вдохновляюсь старинной архитектурой, классической музыкой. Очень люблю
оперетту.

49.В 2019 г. мне присвоено звание \enquote{Почетный Гражданин г. Мариуполя}.

50.В 2020 г. награжден Президентом Украины орденом \enquote{За Заслуги} ІІІ степени. 

51.В 2021 г. выиграл Чемпионат Украины по шахматам среди ветеранов.

52.В июле 2022 г. я принял участие в грандиозном спортивном празднике Израиля –
Маккабиаде! Этим фактом я очень горжусь и счастлив!

53.В 2020 г. в составе команды УКРАИНА-3 выиграл бронзовую медаль 1-й Всемирной
шахматной онлайн-паралимпиады FIDE!

54. В составе сборной Украины по шахматной композиции стал серебряным призером
командного Чемпионата мира в 2022 г.!

55. МЫ выжили! В расстреливаемом, разрушенном Мариуполе, полтора месяца в
тяжелейших условиях: без связи, света, воды, газа, отопления. И теперь мы на
земле Обетованной!

%\ii{15_10_2022.fb.jarmonov_igor.chess.mariupol.1.55_faktov_moei_biogr.cmt}
