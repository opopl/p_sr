% vim: keymap=russian-jcukenwin
%%beginhead 
 
%%file 04_01_2019.stz.news.ua.lb.1.arhitekturnyj_atlas_dorevoljucijnogo_mariupolja.7.kinoteatr_pobeda
%%parent 04_01_2019.stz.news.ua.lb.1.arhitekturnyj_atlas_dorevoljucijnogo_mariupolja
 
%%url 
 
%%author_id 
%%date 
 
%%tags 
%%title 
 
%%endhead 

\subsubsection{Кинотеатр \enquote{Победа}}

\enquote{Победа} - единственный мариупольский кинотеатр, продолжающий работу с
дореволюционных времен. Кинематограф \enquote{Гигант} - прямой предок
\enquote{Победы} - открылся 23 октября 1913 в нижнем этаже двухэтажного здания
на центральном проспекте. В 30-е годы ХХ века зал получил новое,
соответствующее своей эпохе имя - \enquote{КИМ}, в честь Коммунистического
Интернационала Молодёжи. После того как немцы захватили город, киноаппараты
простаивали совсем недолго, и уже с декабря 1941 года кинотеатр функционировал
под названием \enquote{Soldatenkino}. В октябре 1943 года во время отступления вермахта
двухэтажное здание сильно пострадало от пожара, и возобновить киносеансы
удалось только в феврале 1946. Тогда кинотеатр и получил своё нынешнее название
– \enquote{Победа}.

\ii{04_01_2019.stz.news.ua.lb.1.arhitekturnyj_atlas_dorevoljucijnogo_mariupolja.7.kinoteatr_pobeda.pic.1}

В 60-е годы фасад дореволюционного здания облицевали вездесущей бежевой
плиткой, начисто лишив индивидуальности. Работы по возвращению кинотеатру
первозданного облика начались в 1988 году и продлились четыре года. В наше
время, наряду с оригинальным историческим обликом \enquote{Победа} интересна
тем, что является единственным в городе оплотом некассового альтернативного
кинематографа. Но уже несколько лет \enquote{Победа} балансирует на грани
выживания. В январе 2018 года историческое здание выставили на продажу, и
судьба его остается в подвешенном состоянии.

\ii{04_01_2019.stz.news.ua.lb.1.arhitekturnyj_atlas_dorevoljucijnogo_mariupolja.7.kinoteatr_pobeda.pic.2}

\emph{Если говорить о личном вкладе в архитектурное достояние Мариуполя, то, нельзя
не упомянуть имя Виктора Александровича Нильсена – городского архитектора. За
годы своей работы (предположительно 1900-1917 годы) он внёс огромный вклад в
формирование облика Мариуполя. Кроме символа города - водонапорной башни - по
его проектам построены жилые дома и учебные заведения, церкви и здания
технического назначения. Многие постройки Нильсена продолжают украшать
Мариуполь и в наше время. Справедливым и уместным стало недавнее переименование
в честь архитектора улицы Энгельса. Но очень жаль, что дома, в которых проживал
сам зодчий, украшением города назвать нельзя даже с натяжкой.}

\ii{04_01_2019.stz.news.ua.lb.1.arhitekturnyj_atlas_dorevoljucijnogo_mariupolja.7.kinoteatr_pobeda.pic.3}
\ii{04_01_2019.stz.news.ua.lb.1.arhitekturnyj_atlas_dorevoljucijnogo_mariupolja.7.kinoteatr_pobeda.pic.4}
\ii{04_01_2019.stz.news.ua.lb.1.arhitekturnyj_atlas_dorevoljucijnogo_mariupolja.7.kinoteatr_pobeda.pic.5}
\ii{04_01_2019.stz.news.ua.lb.1.arhitekturnyj_atlas_dorevoljucijnogo_mariupolja.7.kinoteatr_pobeda.pic.6}
\ii{04_01_2019.stz.news.ua.lb.1.arhitekturnyj_atlas_dorevoljucijnogo_mariupolja.7.kinoteatr_pobeda.pic.7}
