% vim: keymap=russian-jcukenwin
%%beginhead 
 
%%file 19_07_2020.fb.lnr.13
%%parent 19_07_2020
 
%%endhead 
  
\subsection{Во Львове заявили, что поляки паразитируют на теме Волынской резни}
\url{https://www.facebook.com/groups/LNRGUMO/permalink/2858326937612195/}

\vspace{0.5cm}
{\small\LaTeX~section: \verb|19_07_2020.fb.lnr.13| project: \verb|letopis| rootid: \verb|p_saintrussia|}
\vspace{0.5cm}

Официальная Варшава создаёт целую «индустрию геноцида» на теме истребления
польского населения украинскими националистами во время Волынской резни 1943
года. Впоследствии поляки рассчитывают требовать от Украины компенсаций.

Об этом в эфире своего видеоблога, заявил экс-советник главы СБУ, бывший
депутат Верховной рады от ВО «Свобода», львовский неонацист Юрий Михальчишин,
передает корреспондент «ПолитНавигатора».

«Речь идет о совместном продукте польской семьи историков Семашко, отца и
дочери, которые в своих подсчетах (довольно-таки спекулятивных) завышали
количество жертв в три раза, выйдя на сакральную цифру в 100 тысяч польских
жертв УПА на Волыни. Плюс, предыдущие публикации, в которых сказано 35 тысяч
жертв – это было трехкратное завышение, и на этом не остановились.

Пользуясь читательским успехом, базируясь на поддержке среды так называемых
кресовьяков – это бывшие польские жители земель, которые оказались в составе
УССР после 1945 года, и были отправлены в Польшу, сейчас их потомки также очень
поддерживают праворадикальные силы в Польше, и они требуют постоянного внимания
к проблематике событий на Волыни 1934 года с расчетом на обвинение украинцев в
этнических чистках, военных преступлениях и спланированном геноциде», – говорит
Михальчишин.

«Эти события, которые в 2003 году начинают набирать масштабных цифровых
параметров – более 100 тысячи жертв, начались с того, что в польском политикуме
произошла значимая ассиметричная переоценка этих событий и, соответственно к
смене политических настроений был проложен путь – к признанию на официальном
уровне этих событий геноцидом, и обвиняют в его создании ОУН. Отрицание этих
событий в Польше является криминальным преступлением – так же, как и
использование красно-чёрной символики, которая является «бандеровской». Но и на
этом решили не останавливаться, в произведениях некоторых современных польских
историков речь идет уже про 200 тысяч жертв.

С большим уважением и болью отношусь ко всем невинно погибшим в частности на
Волыни поляков летом 1943 года, но нельзя закрывать глаза на явную
фальсификацию истории и на то, что такая цифровая инфляция имеет ярко
выраженный политический подтекст. Это создание целой индустрии геноцида для
того, чтобы требовать компенсации (неизвестно, только финансовые или нет) от
нынешнего Украинского государства. Это всё может продолжиться и затянуться на
неопределенный срок и принять совершенно иные обороты», – опасается
националист.

Тарас Стрельцов 
