% vim: keymap=russian-jcukenwin
%%beginhead 
 
%%file 30_11_2021.stz.edu.lnr.lgaki.1.diktant_konkurs_rossia
%%parent 30_11_2021
 
%%url https://lgaki.info/novosti/vpervye-uchastvovali-vo-vserossijskom-izobrazitelnom-diktante
 
%%author_id 
%%date 
 
%%tags isskustvo,studenty,lnr,lgaki,rossia,konkurs,kultura,diktant
%%title Впервые участвовали во Всероссийском изобразительном диктанте
 
%%endhead 
\subsection{Впервые участвовали во Всероссийском изобразительном диктанте}
\label{sec:30_11_2021.stz.edu.lnr.lgaki.1.diktant_konkurs_rossia}
\Purl{https://lgaki.info/novosti/vpervye-uchastvovali-vo-vserossijskom-izobrazitelnom-diktante}

29 и 30 ноября 2021 года студенты 1-го и 2-го курсов специальности «Живопись»
отделения изобразительного искусства колледжа Академии Матусовского впервые
приняли участие во Всероссийском изобразительном диктанте-2021, который
проводится в рамках Международного благотворительного конкурса «Каждый народ —
художник». Организатор конкурса — Международный союз педагогов-художников.

\ii{30_11_2021.stz.edu.lnr.lgaki.1.diktant_konkurs_rossia.pic.1}

Общая тема конкурса в этом году: «Моя страна – моя история». Нашим участникам
организаторы предложили самим выбрать темы, связанные с историей города и края,
историческими личностями и местными достопримечательностями.

Студенты работали в течение трех часов, каждый по-своему воплощая неповторимый
образ Донбасса, его героев, историю и культуру родной земли.

— Этот конкурс помог каждому участнику оценить свои способности, закрепить
знания и умения, полученные на занятиях, а также почувствовать себя частью
большого дружного художественного сообщества, — рассказали наши участники
диктанта.

Фото – из личного архива студентов.

