%%beginhead 
 
%%file 26_08_2021.fb.fb_group.mariupol.nekropol.1.na_platanovoj_allee_5_derevjev
%%parent 26_08_2021
 
%%url https://www.facebook.com/groups/278185963354519/posts/587073059132473
 
%%author_id fb_group.mariupol.nekropol,arximisto
%%date 26_08_2021
 
%%tags 
%%title На Платановой аллее в Некрополе высадили еще пять деревьев
 
%%endhead 

\subsection{На Платановой аллее в Некрополе высадили еще пять деревьев}
\label{sec:26_08_2021.fb.fb_group.mariupol.nekropol.1.na_platanovoj_allee_5_derevjev}
 
\Purl{https://www.facebook.com/groups/278185963354519/posts/587073059132473}
\ifcmt
 author_begin
   author_id fb_group.mariupol.nekropol,arximisto
 author_end
\fi

\vspace{0.5cm}
\textbf{На Платановой аллее в Некрополе высадили еще пять деревьев}

На прошедших выходных волонтеры высадили еще пять деревьев на Платановой аллее
в Мариупольском Некрополе. В результате аллея насчитывает уже пятнадцать
платанов.

\#новости\_архи\_города

Саженцы-однолетки платана кленолистного были закуплены благодаря пожертвованиям
мариупольцев, неравнодушных к судьбе Старого городского кладбища, как сообщает
Андрей Марусов, директор общественной организации \enquote{Архи-Город}, координирующей
волонтерское движение. Их приобретение обошлось в 471 грн., вместе с доставкой.

Аллея была заложена в апреле этого года вдоль самой древней аллеи Некрополя.
Когда-то она вела от Марьинки к алтарю Всехсвятской церкви.

Платан был выбран не только из-за его красоты, быстрого роста и
засухоустойчивости, что важно ввиду климатических изменений.

Он имеет символическое значение. Это едва ли не самое древнее дерево из ныне
живущих на Земле (зародился около 100 млн. лет назад).

Платан ассоциируется с Грецией. Создание Платановой аллеи - это дань памяти и
уважения к грекам-первопоселенцам Мариуполя. Самое древнее датированное
захоронение Некрополя находится на этой аллее. Оно принадлежит Гавриилу
Сахаджи, греку-первопоселенцу (1833-34 гг.).

Платановая аллея имеет и экологическое значение. Мариупольский Некрополь – это
потенциальные \enquote{зеленые легкие} центра города. Но на сегодня 16 гектаров
его территории заполнены сухостоем, дикой сиренью и выродившимися деревьями.

Мы высаживаем саженцы только на очевидно пустых местах, чтобы не нарушать права
родственников мариупольцев, похороненных в Некрополе, как подчеркивает Андрей
Марусов. Например, один платан была посажен на месте мусорника, который
расчистили волонтеры. Другой – на месте куста дикой сирени. Кстати, все
саженцы, посаженные этой весной, хорошо прижились; некоторые уже больше метра
высотой!

Во время работы мы обнаружили две древние плиты (более подробную информацию об
этих открытиях опубликуем чуть позже).

Для создания полноценной Платановой аллеи требуется не менее 40 деревьев (длина
аллеи - около 200 метров).

Мы будем рады любым пожертвованиям на дальнейшее продолжение аллеи этой осенью!
Средства можно перечислить на карточку Привата 5168 7554 5285 3746. Обязательно
укажите в назначении - \enquote{для Платановой аллеи}. Контактный телефон 096 463 69
88.

Мы искренне благодарим волонтеров, которые участвовали в нынешней посадке и
благоустройстве аллеи – Илью Луковенко (Илья Луковенко), Valentina Evseenko,
Александра и Наталия Шпотаковская. Мы также признательны всем нашим
благотворителям!
