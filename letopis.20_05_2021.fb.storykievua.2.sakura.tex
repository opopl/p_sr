% vim: keymap=russian-jcukenwin
%%beginhead 
 
%%file 20_05_2021.fb.storykievua.2.sakura
%%parent 20_05_2021
 
%%url https://www.facebook.com/groups/story.kiev.ua/permalink/1665937196936408/
 
%%author 
%%author_id 
%%author_url 
 
%%tags 
%%title 
 
%%endhead 
\subsection{Киев - Сакура - Весна - Андреевский Спуск}
\label{sec:20_05_2021.fb.storykievua.2.sakura}
\Purl{https://www.facebook.com/groups/story.kiev.ua/permalink/1665937196936408/}

Как известно, японские граждане любят сакуру -  ну просто "тащатся" от нее!
Специально приезжают издалека и любуются её сказочным цветением. Что они в этом
находят? Во-первых, это просто красиво. А ещё рядом с сакурой японцы предаются
воспоминаниям, светлым и грустным, и, наверно, осознают, как мимолетна жизнь,
молодость, красота... Что ж, молодцы!

Но чем мы хуже? В нашем замечательном городе тоже кое-где цветет сакура, но в
мае особенно пышно цветет сирень - кстати, мой любимый цветок... Когда она
расцветает, самое время где-то найти местечко для созерцания ее красоты и для
размышлений о смысле жизни... Сиреневый куст можно обнаружить у любой
"хрущёвки". И возле моего дома сирени предостаточно, но шум большого города
мешает насладиться её ароматом в полной мере, ощутить её нежную магию... В
ботсаду на Печерске она - самых разных оттенков, она там прекрасна, но сколько
же народу рыщет и "селфится" возле кустов - даже становится боязно за неё...

\ifcmt
  pic https://scontent-iad3-1.xx.fbcdn.net/v/t1.6435-9/187803299_4248702338481296_3076905735206814058_n.jpg?_nc_cat=110&ccb=1-3&_nc_sid=825194&_nc_ohc=a8Hwc9JBe_wAX8wA4ks&_nc_ht=scontent-iad3-1.xx&oh=59752e3156f9fa8350f84cf3071a161f&oe=60C929D6
\fi

Но чудесное местечко я всё же нашел - на одном из холмов, что окружают
Андреевский. Если не полениться и одним майским днем (лучше утром, когда
молодежь, приходящая туда под вечер с пивом, ещё спит) вскарабкаться по крутой
лесенке, то сразу оказываетесь рядом с цветущей "стеной" сирени. Если усесться
на подстилку перед ней, лицом к Андреевскому спуску, то будет виднеться только
шпиль "Замка Ричарда" и - чуть поодаль - церковь... Замрите на несколько минут.
И случится чудо.  Время как будто остановит свой бег. Тонкий аромат, источаемый
сиреневыми кустами, словно вырвет вас из сегодняшнего дня и унесёт в
сиреневатые дали.

... Где-то люди станут вновь воевать, исходя ненавистью; где-то будут голодать;
кто-то вновь начнёт безумно стремиться к власти, обыватели же - обрастать
барахлом, а наскучившие друг другу супруги - устраивать бури в стакане воды...
А здесь "шум времени" отступит, аромат сирени вскружит голову, фантазия начнет
рисовать самые прекрасные свои картины - и вы несомненно ощутите, как хороша и
хрупка эта жизнь... Где-то внизу,  явственно услышите вы, в доме номер 13,
вдруг хлопнет  дверь - и юный Миша Булгаков побежит по древней брусчатке в
Александровскую гимназию, а под Андреевской церковью раздастся  "Канареечка" -
это Голохвастов, в приличном "пинжаке" и с цветком в петлице, будет спускаться
на Подол со своим друзьями-оболтусами, обольщая местных барышень...

Захочется сказать: остановись, мгновенье, ты прекрасно!

Ведь сирень в Киеве - пусть даже майский дождь щедро поливает наш город - это счастье!

И размышляя обо всём этом: о японцах с их сакурой, о Киеве, украшенном
красавицей сиренью, о ее магии и о мимолетности всего прекрасного, что осеняет
нашу жизнь, - я однажды сложил эти строки...

Как японец у сакуры в тихом восторге
Замирает и шепчет ей строчку Басё,
Так и я у сирени почувствую столько,
Что вульгарным пред этим покажется всё:
Всякий трон и карьеры кормушка любая,
Хадж к неону витрин и туфте позолот,
И лощёных вождей трескотня площадная,
И готовый за пайку бесчинствовать сброд…
Сколько б я ни встречал подбиравших объедки,
Сколько б грязи ни видел, считая версты,
Вновь лицом окунувшись в лиловые ветки,
Я поверю, что мир этот можно спасти!
Сам не раз к ней, расцветшей, приду я с работы,
Чтоб спастись её свежестью в серые дни:
В ней одной лишь найду несказанное что-то,
Может быть, благородству людскому сродни!..
И не грозди – отряды сиреневых конниц,
Что на помощь нам скачут, увидит душа;
И родным вдруг предстанет тот самый японец,
Что у сакуры дивной застыл чуть дыша.
