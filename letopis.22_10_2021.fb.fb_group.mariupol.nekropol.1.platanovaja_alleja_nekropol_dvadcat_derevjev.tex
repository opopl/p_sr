%%beginhead 
 
%%file 22_10_2021.fb.fb_group.mariupol.nekropol.1.platanovaja_alleja_nekropol_dvadcat_derevjev
%%parent 22_10_2021
 
%%url https://www.facebook.com/groups/278185963354519/posts/624194972086948
 
%%author_id fb_group.mariupol.nekropol,arximisto
%%date 22_10_2021
 
%%tags 
%%title На Платановой аллее Мариупольского Некрополя – уже двадцать деревьев!
 
%%endhead 

\subsection{На Платановой аллее Мариупольского Некрополя – уже двадцать деревьев!}
\label{sec:22_10_2021.fb.fb_group.mariupol.nekropol.1.platanovaja_alleja_nekropol_dvadcat_derevjev}
 
\Purl{https://www.facebook.com/groups/278185963354519/posts/624194972086948}
\ifcmt
 author_begin
   author_id fb_group.mariupol.nekropol,arximisto
 author_end
\fi

\vspace{0.5cm}
\textbf{На Платановой аллее Мариупольского Некрополя – уже двадцать деревьев!}

В первой половине октября волонтеры высадили пять саженцев платана восточного
на Платановой аллее Некрополя; общее количество деревьев на аллее достигло
двадцати.

\#новости\_архи\_города

Благодаря пожертвованиям мариупольцев волонтеры сумели закупить и посадить еще
пять саженцев-однолеток платана восточного (Platanus orientalis) на Платановой
аллее, самой древней аллее Некрополя.

Саженцы были куплены в онлайновом магазине lizgard по цене 100 грн. за штуку.
Вместе с доставкой на закупку было потрачено 575 грн.

Таким образом, на сегодня Платановая аллея насчитывает 15 саженцев платана
кленолистного (Platanus acerifolia), самого распространенного, засухо- и
мороустойчивого вида, а также пять саженцев платана восточного
(\enquote{чинары}), самого известного вида.

Первые посадки были сделаны еще в апреле этого года (см.
\href{https://archive.org/details/26_04_2021.fb.arximisto.volontery_allei_platanovaja_dubovaja_nekropol}{Волонтеры
заложили Платановую и высадили Дубовую аллеи в Некрополе, Архі-Місто,
26.04.2021}\footnote{\url{https://archive.org/details/26_04_2021.fb.arximisto.volontery_allei_platanovaja_dubovaja_nekropol}}).
На закупку и доставку 20 саженцев было потрачено 2 333 грн., пожертвованных
мариупольцами. Чтобы минимизировать риск поставки некачественных саженцев, мы
закупали у разных компаний, которые существуют на рынке уже несколько лет. Это
дендроферма семьи Бережнюк, Львовская область, онлайновые магазины
\enquote{Листик} и \enquote{БаТаня} (Мелитополь), а также Lizgard, Киевская
область.

Из-за ограниченности средств мы закупали одно- или двухлетки. Более взрослые
саженцы стоят на порядок больше.

Чтобы предупредить обмерзание саженцев, мы их мульчируем (то есть, выкладываем
слой из древесной коры и листьев вокруг шейки дерева).

Чтобы предупредить их уничтожение посетителями Старого городского кладбища, мы
огородили каждый саженец палками и синей лентой. Чуть позже мы также прикрепим
бирки с указанием рода дерева, временем посадки и контактами волонтеров. 

\textbf{Зачем мы это делаем?}

Чтобы преобразовать Некрополь в \emph{\enquote{зеленые легкие} центра Мариуполя}. На сегодня
Старое городское кладбище – это беспорядочные заросли айланта, дикой сирени,
выродившихся кленов и осины. Но потенциально – это мощный щит на пути
восточного ветра с выбросами \enquote{Азовстали} и завода им. Ильича...

Чтоб \emph{сделать Старое городское кладбище – ухоженным Некрополем}, с красивыми
долговечными аллеями из породистых деревьев, цветниками и т.п. К сожалению,
сегодня это – клоака, в десяти минутах ходьбы как от традиционного центра
города (Театральный сквер), так и от нового, вокруг площади Свободы (и будущего
университетского кампуса \enquote{Метинвест Политехники} и МарГУ...)

Чтобы превратить древнейшую аллею Некрополя - \emph{в символ уважения к вкладу
приазовских греков в создание и развитие Мариуполя}. Именно на ней находится
самое древнее датированное захоронение греков-первопоселенцев как Мариуполя,
так и Приазовья (Гавриил Сахаджи, 1833-35 гг.).

Чтобы в городе появился новый \enquote{историко-культурный магнит} – Платановая
аллея Мариупольского Некрополя, аллея всех исторических эпох Мариуполя.

\textbf{Почему выбран платан?}

Платан - необычно красивое дерево. Его мощный высокий ствол временами обретает
почти мраморный оттенок. Это единственное дерево, которое регулярно сбрасывает
свою кору (\enquote{бесстыдница} – его народное название). Многие города Крыма и
Причерноморья гордятся своими старинными платановыми аллеями.

Платан – долговечное, засухо- и достаточно морозо-устойчивое дерево. Из-за
очевидного изменения климата Приазовья он может стать оптимальным кандидатом
для высадки в регионе.

Платан – быстро растущее и доминирующее дерево. В будущем его большая крона
будет подавлять дикие заросли других деревьев.

Платан устойчив к промышленному загрязнению и природным вредителям.

Платан имеет символическое значение. У многих он ассоциируется со
Средиземноморьем и Грецией. Кроме того, платан – одно из древнейших ныне
живущих растений. Он возник около 90-100 миллионов лет назад, во времена
динозавров...

\textbf{Как мы это делаем?}

Мы сажаем деревья на очевидно пустых местах, чтобы не нарушать права и интересы
родственников мариупольцев, похороненных на Старом городском кладбище: на
мусорниках после их расчистки; рядом с пнями (очевидно, что эти места заброшены
уже очень давно); в свободном пространстве между могилами и т.д.

Например, в самом начале аллеи с правой стороны есть большой участок, на
котором мы сознательно не садили платаны. На нем много сохранившихся могил, за
некоторыми явно ухаживают, поэтому мы отказались от посадки – до получения
согласия родственников.

\textbf{Что дальше?}

На рынке появились недорогие саженцы еще одного вида – платана западного
(Platanus occidentalis), но мы отказались от его приобретения. Наши средства на
озеленение Некрополя полностью исчерпаны.

Для завершения аллеи необходимо не менее 20 саженцев. Очевидно, что это задача
весны 2022 года. Этой весной мы подавали заявку на участие в программе
\enquote{Зеленого центра Метинвеста}, но никакого ответа не получили. Попробуем
в следующем году...

Огромное спасибо всем благотворителям за пожертвования и волонтерам – за
работу!

Мы будем рады любым предложениям, замечаниям и рекомендациям!

Контакты: arximisto@gmail.com, +38096 463 69 88.

\#платановая\_аллея

P.S. И - держим кулаки, чтоб все саженцы успешно перезимовали! 🙂 Самые первые
- уже выше одного метра (а были - 30-50 см. при покупке). Мы очень надеемся,
что через год на аллее будут уже двухметровые красавцы...
