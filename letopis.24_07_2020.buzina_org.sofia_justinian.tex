% vim: keymap=russian-jcukenwin
%%beginhead 
 
%%file 24_07_2020.buzina_org.sofia_justinian
%%parent 24_07_2020
 
%%endhead 
\subsection{Профукали!!! Прости нас св.Юстиниан!!!}
\url{https://buzina.org/golos-naroda/3715-sofia.html}
  
\vspace{0.5cm}
 {\ifDEBUG\small\LaTeX~section: \verb|24_07_2020.buzina_org.sofia_justinian| project: \verb|letopis| rootid: \verb|p_saintrussia| \fi}
\vspace{0.5cm}

\index{Церковь!Собор Святой Софии в Константинополе}
\index{История!Византия!Юстиниан}

Строительство Юстиниана

Много государственных средств тратил Юстиниан на монументальное строительство.
И в этом плане обессмертил себя созданием Св. Софии Цареградской.

Константиновская Св. София сгорела во время народного бунта при Иоанне
Златоусте. Перестроенная вновь, сгорела в 415 г. при Феодосии II;
восстановленная, скромная по материалам и размерам, опять сгорела в 532 г. при
самом Юстиниане во время бунта νίκα. Спасенный мудростью Феодоры, благодарный
Юстиниан задумал и осуществил строительство в своем роде уникальное. С 532 по
537 г. в рекордный срок он создал храм — чудо истории — с его архитектурно
смелым, висящим прямо над стенами куполом. Архитекторами его были Анфемий
Тралльский и Исидор Милетский. Кельнский собор строился 500 лет. Римский Св.
Петр — 350 лет. Да и наши соборы: Исаакий — 90 лет, московский Христа Спасителя
— 50 лет. A Св. София — в одну пятилетку, когда нужны были долгие морские тяги
для своза разноцветных колонн из разных развалин и концов империи: из Сирии —
Баальбека, Египта.

Еще до турок латинскими варварами IV крестового похода был изрублен на куски и
растащен престол храма, составленный из сплавов золота и серебра, с богатейшими
самоцветами. Златобуквенная молитвенная надпись, украшавшая его перед, была
такова:

«Твоя от Твоих приносим Тебе Твои, Христе, рабы Юстиниан и Феодора. Милостиво
прими сие, Сыне и Слове Божий, за нас воплотившийся и распятый. И сохрани нас в
вере Твоей православной и государство, которое Ты нам вверил, умножи во славу
Свою и сохрани предстательством св. Богородицы и Приснодевы Марии!»

Но ... не конец еще соблазнам и не конец моей апологии Великого Юстиниана.

12 лет прожил Юстиниан после V Вселенского собора и был свидетелем того, как
все его героические предприятия ради возврата в лоно кафолической церкви
еретичествующих окраин империи не увенчались никаким сколько-нибудь
значительным успехом. Между тем слепые в своем заносчивом упорстве еретические
партии монофизитских диссидентов с увлечением продолжали свои шумные
литературные споры. И надо признать, не без таланта. Крайний монофизит, епископ
Галикарнасский Юлиан развивал свою доктрину об одной лишь видимости
человеческой природы Христа. Плоть облекала, как привидение, одну божественную
природу. Это приводило в восторг коптскую толпу. И положение умеренного, умного
Севира Антиохийского, утверждавшего реализм и тленность плоти Христа, было
крайне непопулярно. Его вульгарно бранили «фтартолатром,» т.е. поклонником
тления, за то, что он точно клеймил юлианистов «афтартодокетами» т.е.
проповедниками нетленной и мнимой, призрачной человечности; еще острее —
называл их «фантазиастами.» Но эти «фантазеры» были соблазнительны для черни,
как соблазнителен всякий для низов коммунизм. Бесполезно состязаться с
соблазненными подобным социализмом низами. Их уже не превзойти никакими
крайностями. To, что мы элементарно знаем теперь, неясно еще было «наивному»
Юстиниану.

Он, до маниакальности занятый спасением империи через компромиссы и демагогию,
в своем грешном отрыве от общецерковного соборного мнения соблазнился впасть в
старый свой грех — навязал церкви безумную демагогическую доктрину юлианистов
путем принудительных подписей под автократическим декретом василевса. Это было
актом безумия (Юстиниану исполнилось уже 82 года). «Кого Господь захочет
наказать, то прежде всего отымет разум.» B отрыве от соборности Юстиниан
оказался во власти искушения. И подверг искушению иерархию. Это было в
последние месяцы перед смертью Юстиниана (в 565 г.).

Почти современный ему историк Евагрий повествует так: «...издал эдикт, в
котором тело Господа назвал не подлежащим тлению и не причастным естественным и
невинным страстям и говорил, что Господь вкушал пищу точно так же и до
страдания, как потом по воскресении, т.е. что будто бы всесвятое тело Его ни в
вольных, ни в невольных «страстях» (эмоциях) не переживало никаких превращений
или перемен с момента образования его в утробе и даже после воскресения.
Василевс принуждал согласиться с этим учением всех иерархов» (Церковная
история. IV, 39).

Первым воспротивился ставленник Юстиниана, в свое время так угодивший ему
Константинопольский патриарх Евтихий. Его арестовали в храме во время
богослужения и подобрали группу епископов для кривосудия над ним. На суд он не
пошел, был судим заочно за выдуманные мелочи, осужден и сослан в свой старый
монастырь в Амасию (в Понте). На место Евтихия был назначен один из
антиохийских пресвитеров, знаменитый канонист Иоанн Схоластик. Ο ереси
Юстиниана пришли слухи и на Запад. Из Галлии епископ Низиерий писал императору
увещание — отстать от ереси и, по меньшей мере, не преследовать православных.
Патриархи Александрийский и Иерусалимский послали свои отказы от подписи
еретического указа. Патриарх Антиохийский Анастасий успел даже собрать местный
собор, епископы которого в числе 195 солидарно заявили, что они все покинут
свои кафедры, но не примут учения «фантазиастов.» Ο решении собора были
извещены все многочисленные сирские монастыри. На это последовал указ
самодержца ο смещении Анастасия с Антиохийской кафедры. Анастасий уже заготовил
прощальное послание к своей пастве, как вдруг пришла облегчающая весть ο
внезапной смерти Юстиниана. Тот же Евагрий (IV, 41) записал: «Потому что Бог
предусмотрел ο нас нечто лучшее» (Евр. 11:40), то Юстиниан, в то время как
диктовал постановление ο ссылке Антиохийского патриарха Анастасия и
единомышленных с ним иерархов, был поражен невидимо ударом и отошел из сей
жизни. Таким образом, возбудив везде смуту и тревоги и при конце жизни получив
достойное этих дел возмездие, Юстиниан перешел «в преисподние судилища.»

Новый василевс, племянник скончавшегося — Юстин II, сбросил камень,
навалившийся на совесть иерархов, уже собиравшихся в столицу для подписи
безумного указа. Указ Юстина отсылал епископов по домам, отложив всякие новости
в делах веры.

Прав Евагрий, что Юстиниан поражен был стрелой смерти провиденциально. Но он
неправ в своем пристрастном суде современника, поторопившегося в раздражении
послать Юстиниана в преисподнюю. «Бог предусмотрел... нечто лучшее,» — скажем
теми же словами послания к Евреям (Евр. 11: 40). Богу утодно было внезапной
смертью избавить великого василевса от великого искушения и развенчания в
глазах церкви. A церковь, возвышаясь над преходящими страстями времени, оценила
иначе общий итог заслуг пред нею идейного императора. Она эту выдающуюся чету
на троне христолюбивых василевсов — Юстиниана и Феодору — вскоре же
канонизовала. Память их в нашем календаре 14 ноября.

Как и в случае с Константином Великим и с нашим крестителем князем Владимиром,
эта канонизация не суеверное приравнивание их к «Единому Безгрешному» («да не
будет!»), a только благодарная признательность за великую ревность ο славе и
единстве церкви, и до наших дней еще ощутимую и как бы осязаемую в знаменитом
законодательном «Кодексе» Юстиниана и его чудесном, как бы вечном цареградском
храме св. Софии.
  
