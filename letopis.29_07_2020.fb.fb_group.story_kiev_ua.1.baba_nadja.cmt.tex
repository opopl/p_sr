% vim: keymap=russian-jcukenwin
%%beginhead 
 
%%file 29_07_2020.fb.fb_group.story_kiev_ua.1.baba_nadja.cmt
%%parent 29_07_2020.fb.fb_group.story_kiev_ua.1.baba_nadja
 
%%url 
 
%%author_id 
%%date 
 
%%tags 
%%title 
 
%%endhead 
\zzSecCmt

\begin{itemize} % {
\iusr{Наталья Рыбкина}
злобно

\begin{itemize} % {
\iusr{Андрей Журавлев}
я бы это назвал скорее сарказмом.

\iusr{Андрей Шиян}
\textbf{Наталья Рыбкина} не согласен! Очень человечно и с юмором!

\iusr{Володимир Шуневич}
\textbf{Наталья Рыбкина} Вы правы. Я бы сказал резче - говнистенько как-то...

\iusr{Андрей Журавлев}
\textbf{Володимир Шуневич} огромное спасибо за ваш отзыв. Я серьезно. Кто то в комментах рассказ хвалит, а кто то ругает. И это нормально. Значит рассказ получился!)

\iusr{Наталья Рыбкина}
\textbf{Володимир Шуневич} Следующая публикация этого автора подтверждает Ваше наблюдение.

\iusr{Андрей Журавлев}

Человек без чувства юмора - это тяжёлый случай. Жизнь видать не сильно балует.
А насчёт юмора в рассказах - я заранее именно с ВАМИ согласен, что там его и
близко нет)))

\end{itemize} % }

\iusr{Олег Московко}

Школа \#66 возле Индустриального моста - тоже английский с первого класса, а в
десятом - до десяти уроков в неделю... те же \enquote{допросы военнопленных} и изучение
структуры американской армии, правда директора были \enquote{нормальными}...но доставало
по 6-7 человек в подгуппе на уроках английского, приходилось \enquote{готовиться}. Хотя
со знанием английского выше \enquote{Ландон - из зе кепитал...} оканчивали не многие. @igg{fbicon.smile} 

\begin{itemize} % {
\iusr{Андрей Журавлев}

я вам скажу так - подобные школы давали очень хорошую БАЗУ языка. Я закончил
школу 40 лет назад, по английскому у меня было 4. Язык мне понадобился в начале
2000-х (я - инженер). Буквально за месяц-другой практики я вспомнил почти все.
Я сам был очень удивлен. В итоге спикаю, пишу и т.д. очень даже гут.


\iusr{Олег Московко}
\textbf{Андрей Журавлев} 

ключевое - \enquote{понадобился}, я закончил в 77, нженерил, керувал... но английским
пользовался только дабы узнать сколько ликера надо для коктейля.


\iusr{Андрей Журавлев}
\textbf{Олег Московко} все правильно. Но без хорошей языковой базы я бы \enquote{учил} все по новой неизвестно сколько времени. А так хватило пару месяцев.
\end{itemize} % }

\iusr{Alla Alla}

Прочла с мыслью, что наша директриса была точный клон бабы-Цеткин, значит их
делали под копирку и садили на директорские места, чтобы школьное время
запомнилось, как ужас всей жизни @igg{fbicon.laugh.rolling.floor}{repeat=3} 

\begin{itemize} % {
\iusr{Володимир Шуневич}
\textbf{Alla Alla} 

Очень Вам соболезную. Для нас школьные годы были лучшим, самым счастливым
временем жизни. Обьездили с учителями и директором на школьном автобусе
полСоюза от Ленинграда до Сухуми и от Бреста до Волгограда. Спортом занимались
весьма успешно, учителя были интересные, кружки. Директор тоже был по-своему
оригинал, но все любили его как отца. И школа была одной из лучших в
Житомирской области... Многие киевские - хлевы по сравнению с ней. Кстати, у
нас английский преподавали со 2-го класса...

\iusr{Маргарита Мышанская}

Вспоминаю школу с содроганием. Бесконечное унижение личности детей и
подростков. Бесконечные вычитывания по любому поводу (как в анекдоте \enquote{почему в
шапке, почему без шапки?}), бестактность и наглость учителей. С облегчением
вздохнула, когда все закончилось. Институт (КПИ, ТЭФ) показался раем. Стараюсь не
думать и не вспоминать школу вообще... Знакомимся-школа №4, Соломенка. @igg{fbicon.face.pensive} 

\end{itemize} % }

\iusr{Ольга Даненко}

Все бы ничего, но украинский язык в исполнении такой личности как вы ее описали
звучит весьма неправдоподобно.

\begin{itemize} % {
\iusr{Олег Московко}
\textbf{Ольга Даненко} все \enquote{спецшколы} были україномовными.

\iusr{Hardashnyk Irene}
\textbf{Олег Московко} Неправда! 

Я окончила 32-ю, т. н. английску. ю базовую школу ин яза, на ул. Федорова, школа
была русская, а вот недалеко находящиеся 56-я и 130-я, действительно были
украинскими.

\iusr{Андрей Журавлев}
\textbf{Hardashnyk Irene} Все вы правильно говорите, но 32-ю вряд ли можно назвать спецшколой. Программа по инглиш там была хорошая, но менее насыщенная, чем, например, у нас в 87-й. У вас не было техн. перевода и военного перевода.

\iusr{Олег Московко}
\textbf{Hardashnyk Irene} исключения только подтверждают правила! @igg{fbicon.smile} 

\iusr{Ольга Даненко}
\textbf{Andriy Zhuravlyov} 

я не знаю сколько вам лет и настолько ли вы старый как я, но я прекрасно помню
отношение к украинской мове в советские годы и тем более в Киеве. Так что
извините, но я вам не верю.

\iusr{Ольга Сахарова}
\textbf{Ольга Даненко} 

Были украинские школы, были говорящие на украинском языке учителя, дети,
семьи... \enquote{Бабу Надю} помню плохо, но была свидетелем, как говорили между собой
другие учителя по-украински. Отношение было чаще к конкретным людям.

\iusr{Андрей Журавлев}
\textbf{Ольга Даненко} 

это ваше личное дело. Кстати, в комментах выше есть 2 человека, которые учились
в моей школе. Они почему то поверили. А я в этой школе отучился с 1969 по 1979
годы. все было именно так. Другое дело - отношение к укр языку в быту. тут я с
вами соглашусь.

\iusr{Андрей Журавлев}

Ольга, у нас была украинская школа и почти все учителя говорили по украински
даже на переменках. В том числе и баба Надя. Как они и она общались в быту - я
не знаю.

\iusr{Олег Московко}
56, 130, 87, 66, 80 ...судя по комментам - україномовні.

\begin{itemize} % {
\iusr{Андрей Журавлев}
\textbf{Олег Московко} 

56, 130, 87 - точно украинские. Про остальные не знаю. Вообще в центре (тогда -
Ленинский р-н) было 12 школ. В 70-х из них русских было только 3 или 4 (33, 21,
57 и вроде все). Когда я вернулся из армии в 1983, зашел в школу, общался с
учителями. Узнал, что из нашей 87-й и еще некоторых убрали английский язык
(начали учить как везде, с 5 класса) и несколько школ района перевели на
русский язык обучения. Украинских школ осталось меньшинство. Вот такая была
\enquote{русификация по тихому}.

\iusr{Ирина Петрова}
\textbf{Andriy Zhuravlyov} 

русскоязычные - 94-я, 86-я, 49-я, 90-я...это навскидку. А вот украинскую помню
из нашего района Ленинского только 117-ю на Энгельса. С директрисой, конечно,
не повезло... Нашу 94-ю Бог миловал, ни об одном из учителей не могу вспомнить
негативно)

\end{itemize} % }

\iusr{Белла Кушнир}
ошибаетесь такие личности говорили на шдиш и украинском

\iusr{Alik Perlov}
\textbf{Белла Кушнир} почему вы так решили??))

\iusr{Андрей Журавлев}

До 1980 в Ленинском районе укр школ было больше чем русских. Потом все
поменялось. Да и 49 и 90 школы - это другие районы


\iusr{Маргарита Мышанская}

На Соломенке - шк. №115 украинская с усиленным английским. Я возле нее жила. И мои
соседи в нее ходили. Там говорили по-украински. И это было в 70-х. Так что не надо
тра-ля-ля об отношении к украинскому языку при СССР. Просто у родителей был
выбор куда отдавать ребенка учится. А в школах с обучением на русском языке было
2 урока украинской мовы и 2 урока украинской литературы в неделю. Расказывать об
ущемлении украинского языка будите рассказывать молодому поколению (дурить их), а
нас не надо... Мы знаем все по своей жизни в Киеве.

\begin{itemize} % {
\iusr{Андрей Журавлев}
\textbf{Маргарита Мышанская} 

а вы знаете, что например, в 1980 году 3 или 4 киевские школы в тогдашнем
Ленинском районе, в котором всего было 12 школ, перевели на русский язык
обучения? Просто взяли - и перевели. Никого не спрашивая. Это в самом маленьком
районе города. И нормальную карьеру в СССР мог сделать только человек
русскоязычный. Украинские и другие языки считались второстепенными, для
\enquote{селюков}. Так что русификация была. Естественно, не только в Украине. Она в те
времена была уже даже не насильственная. Но целенаправленная.

\end{itemize} % }

\iusr{Маргарита Мышанская}

А Вы знаете, что в 90-х, подбирая для дочери школу в Святошинском районе я не
нашла ни одной с русскоязычным обучением. Еле нашла школу, где было 2 урока в
неделю русского. Это что? Мое мнение-должен быть выбор и карьера не должна
зависеть от мовы. Вы учились в КПИ. Кто-то кого-то притеснял за разговоры на
украинском языке? У нас в группе полно было ребят говоривших на украинском. Это
никак не влияло на их успеваемость и получение диплома. Хватит уже этого
всего. Бесконечная ложь и профанация. И, кстати, именно мои однокурсники из сел и
маленьких городков преуспевали в обучении и сейчас занимают высокие посты на
энергетических предприятиях.

\begin{itemize} % {
\iusr{Андрей Журавлев}
\textbf{Маргарита Мышанская} 

На тему языка можно спорить до бесконечности. И я бы наверное согласился с вами
лет 9-10 назад. Но в свете военной агрессии РФ против Украины считаю, что чем
быстрее наши дети и внуки забудут русский язык, тем лучше. Тем более есть
гораздо лучшая и перспективная альтернатива - английский. И да, в Украине один
государственный язык - украинский. Все. Точка.

\end{itemize} % }

\iusr{Маргарита Мышанская}
Можете начинать забывать сию же секунду уже! @igg{fbicon.laugh.rolling.floor}{repeat=4} 

\iusr{Маргарита Мышанская}
...И точка. @igg{fbicon.face.upside.down} 

\iusr{Маргарита Мышанская}

А тогда почему Вы свои опусы пишете на русском, а не на украинском или
английском? Как-то лицемерно получается! @igg{fbicon.face.tears.of.joy}{repeat=5} 

\begin{itemize} % {
\iusr{Андрей Журавлев}
\textbf{Маргарита Мышанская} 

все просто - родной язык для меня русский. Благодаря школе знаю украинский на
5+ и английский 4+. Мог бы и на этих языках написать. Но мне кажется, что
получилось бы хуже. Да и читателей было бы меньше. Не понимаю при чем тут
лицемерие. Да, я считаю что в Украине должны обязательно изучаться укр и англ
языки. За государственный счет. Остальные языки - только по желанию, в том
числе и русский.

\end{itemize} % }

\iusr{Маргарита Мышанская}

Вы пишете \enquote{чем скорее дети и внуки забудут русский язык-тем лучше}. Вы имеете
русскую фамилию и родной язык для Вас русский. Вы хотите, чтобы дети и внуки
предали свой род, своих предков и мыслили на английском или украинском? О чем
Вы? \enquote{Да и читателей было бы меньше..} А Вы задумайтесь почему их было бы меньше?

\begin{itemize} % {
\iusr{Андрей Журавлев}
\textbf{Маргарита Мышанская} 

так не о чем тут задумываться - понятно, что в Киеве пока еще из читающей
публики большинство читает на русском. Я не буду вдаваться в детали почему так
сложилось, но эту ситуацию надо менять. Естественно, не насильственно. А менять
примерно так, как русификация делалась в СССР в 80-е. Потихоньку. И при чем тут
язык предков? Мои родители - русскоязычные, родились в РФ. Но сознательно
отдали меня в укр школу. Потому что считали, что я должен знать укр язык. Тем
более, что вы прекрасно знаете, что в быту у нас можно говорить на любом языке.
Но государственным и обязательным для изучения должен быть один язык. А русский
язык... Его изучение в ближайшей перспективе не принесет практической пользы
нынешним первоклассникам. Так что лучше пусть учат английский.

\end{itemize} % }

\iusr{Маргарита Мышанская}

Страны с несколькими государственными языками: Канада-англ. и франц.,
Индия-англ. и хинди, Швейцария-немецкий, франц. итал, ретороманский(0. 1\%
населения), Финляндия-фин., шведский(шведы 5\% населения), Мальта-мальтийский,
англ., Кипр-греческий, турецкий, Люксембург-франц., немецкий, люксембургский
(франсик мозепан). Список можно продолжить... Откуда, из какой тьмы и
преисподней вырвались идеологи того, о чем Вы говорите? Каким же узколобым
мышлением надо обладать, чтобы поддерживать то, о чем Вы пишете?  @igg{fbicon.face.worried}  Мне жаль
людей, которые думают так как Вы... Убожество это все, ущербность

\begin{itemize} % {
\iusr{Андрей Журавлев}
\textbf{Маргарита Мышанская} 

узколобое мышление и ущербность - это как раз непонимание того, что наш сосед
РФ - это враг, желающий уничтожить Украину как государство. Как минимум, враг
на нашем отрезке времени. Язык - это как говориться \enquote{одно из}. Крепко в вас
засела мнимая советская \enquote{дружба народов}. Вы не обратили внимание, что
например, Франция почему то не отправляет свои войска в Канаду или Люксембург
для защиты франкоязычного населения? Или вы не в курсе, что в США живут даже,
наверное, марсиане и люди говорят там на миллионе разных языков, но
государственный язык почему то один.

\end{itemize} % }

\iusr{Маргарита Мышанская}

Отто фон Бисмарк: \enquote{Могущество России может быть подорвано только отделением от
нее Украины... необходимо не только оторвать, но и противопоставить Украину
России. Для этого нужно лишь найти и взрастить предателей среди элиты и с их
помощью изменить самосознание одной части великого народа до такой степени, что
он будет ненавидеть все русское, ненавидеть свой род, не осознавая этого. Все
остальное-дело времени...}

\begin{itemize} % {
\iusr{Андрей Журавлев}
\textbf{Маргарита Мышанская} 

вряд ли это сказал Бисмарк, но со смыслом цитаты я согласен. Только маленькая
поправка - почему надо взрастить \enquote{предателей}? Кого и что они предают?
Вас к Соловьеву, в Москву, на эфир не приглашали?)))

\end{itemize} % }

\iusr{Маргарита Мышанская}

А Вас на эфир \enquote{Великий Львів} не приглашали?
@igg{fbicon.face.grinning.squinting}  Если нет, то есть над чем работать....
Хотя нет, не пригласят. Фамилия москальская. С такой фамилией в новом
украинском мире Вам карьеры не видать. Как впрочем, и денег. Каким бы
\enquote{свідомим} Вы не бы @igg{fbicon.beaming.face.smiling.eyes}
@igg{fbicon.face.grinning.squinting}  @igg{fbicon.face.tears.of.joy} ли.

\begin{itemize} % {
\iusr{Андрей Журавлев}
\textbf{Маргарита Мышанская} 

карьеру я давно сделал, денег тоже хватает, так что все - мимо))) Тем более,
что я не сталкивался с дискриминацией по поводу фамилии или русского языка
(хотя свободно владею украинским). А вот \enquote{дискриминацию} по поводу отсутствия
мозгов наблюдал неоднократно. Так что все таки так называемый \enquote{укр национализм}
в московском понимании и в Киеве, и во Львове отсутствует. Украинцы -
здравомыслящий и практичный народ. если что, я имею ввиду не этнических
украинцев (к которым я не отношусь), а украинцев по гражданству (к которым я
без сомнения принадлежу). Есть такое понятие - полезные идиоты. Это те, которые
за все хорошее против всего плохого. Им \enquote{какаяразница} на каком языке говорить,
они считают что в Украине пылает гражданская война, они считают, что надо
оставить все как было до 2014 года... Только так, как до 2014 уже не будет.
Есть всего лишь 2 варианта: или Украина отстоит свою независимость, или сюда
придет Россия. И тогда они не спросят, на каком языке вы говорите. Вы и вам
подобные, наравне с такими, как я, будете для них всего лишь \enquote{продажными
хохлами}. Вот тогда вы и объясните им то, что впариваете тут мне.

\end{itemize} % }

\iusr{Маргарита Мышанская}

Конечно! Так как до 2014 года уже не будет! Потому что совершен кровавый
госпереворот, от которого выиграли только олигархи ( многократно увеличили свои
состояния), а \enquote{здравомыслящий и практичный народ} стал пятикратно
беднее, еще и кровь пролил за толстосумов, которые обрели власть в результате
кровавой бойни в центре Киева, и продолжает этот народ детей своих посылать на
бойню...

\begin{itemize} % {
\iusr{Андрей Журавлев}
\textbf{Маргарита Мышанская} с цифрами в руках очень просто опровергнуть ваш бред. Но скорее всего дискутировать с вами бесполезно. И очень плохо, что такие как вы имеют право голоса. И гореть вам в аду за ваши мысли и слова о своих соотечественниках.
\end{itemize} % }

\iusr{Маргарита Мышанская}

Я никогда не голосую-не за кого... Я всегда работаю в комиссиях-секретарем. А
насчет ада -не горячитесь. Пути Господни неисповедимы... @igg{fbicon.smile} 

\iusr{Андрей Журавлев}

я не горячусь, я просто назвал вещи своими именами. Человек, ненавидящий свою
родину добром не кончит.

\iusr{Маргарита Мышанская}

Дело не в моей любви к Родине, дело в оценке того, что происходит на моей
Родине... А происходят события, которые отбрасывают ее развития на десятилетия
назад. И маячит весьма реально колониальное будущее..

\iusr{Маргарита Мышанская}

Как говорится: \enquote{Не рой яму другому-сам в нее попадешь!} За сим
разрешите откланяться...

\end{itemize} % }

\iusr{Раиса Карчевская}

Мне очень понравился ваш рассказ с юмором и очень правдивый. Я проходила
практику английского языка в 32 школе рядом с институтом иностранных. языков, в
котором я училась и мне было очень комфортно с детьми. Школа была очень
сильная, английский учили с 1го класса, а в группе было 10 человек, но дети были
хорошего уровня. Англичанка в классе была замечательная и она всегда оставляла
меня студентку'-практикантку абсолютно спокойно с детьми. Мне очень нравилась
практика и я 2 года подходила в одном и том же классе. Я переводчик, но учу
своих родственников и их детей, и друзей английскому

\begin{itemize} % {
\iusr{Андрей Журавлев}

32 школу отлично знаю, там много друзей детства училось. Я живу там рядом. Но в
32-й программа была пожиже нашей 87-й. Там не было техперевода и военки.


\iusr{Раиса Карчевская}
\textbf{Андрей Журавлев} Да там не было тех.перевода и военки

\iusr{Hardashnyk Irene}

Да, тех перевода и военки у нас не было, но были очень сильные учителя. Я без
репетиторов и блата в 77-м поступила в ин яз после 32-й. У нас каждый год туда
поступали по 2-3 выпускника.

\iusr{Алексей Усенко}
\textbf{Раиса Карчевская} 

а дериктриса - гоффно..... меня не взяли в 4 -й клас именно из-за не желания
моих родителей дать взятку... мой отец сделал глупость ...приехал на \enquote{Волге}..

\iusr{Алексей Усенко}

Меня отдали в самую \enquote{хулиганскую} школу микрорайона... 37..... о чем я ни
сколько не жалею.... единственная школа где директор был Мужик... Николай
Александрович Витьман.. Да Бог ему - здоровья!!

\iusr{Андрей Журавлев}
\textbf{Алексей Усенко} знаю 37-ю. У меня сын там учился. Директор тоже был Витьман.

\iusr{Алексей Усенко}
\textbf{Андрей Журавлев} это он и есть)
Преподаватель математики..

\iusr{Андрей Журавлев}
\textbf{Алексей Усенко} да, он самый
\end{itemize} % }

\iusr{Violeta Tetiana Stetsenko}

Я проучилась в 87 школе с 1961 по 1971 год. Это на моей памяти Надія Сергіївна
стала директриссой и именно наш класс дал ей прозвще Баба Надя. Она преподавала
у нас Обществоведение, но тогда она не была такой вредной. А про курцов в
туалете. это точно, дверь нараспашку \enquote{Ану, виходь!} Мы её абсолютно не боялись,
относились с юмором. И Ивана Захаровича прекрасно помню, тогда еще заливали
каток на спортивной площадке, а онк по радио из спортзала руководил секцией
фигурного катания. Типа \enquote{Ира, Наташа, еще разок повторить поворот} и при этом
было слышно булькание разливаемого в стаканы.

\begin{itemize} % {
\iusr{Андрей Журавлев}
да, насчет бульканья в стаканы Захарыч был чемпионом)))

\iusr{Natasha Levitskaya}
\textbf{Violeta Topy}
Вот другой взгляд на одну и ту же школу и на одного и того же человека!
\end{itemize} % }

\iusr{Наталя Піпаш}
Згадав....

\iusr{Ольга Морева}

Мда, такую страну пооср@ли, с мороженым, с карательной педагогикой, с линейками
и политинформациями. Вот сразу тоской повеяло от воспоминаний о
пионэрско-комсомольском детстве. Теперь кошмары будут сниться.

\begin{itemize} % {
\iusr{Ольга Сахарова}
Феномен \enquote{бабы Нади} очень популярен и ныне. Просто несколько слов в лозунгах поменяли...

\iusr{Maxim Kostenko}
\textbf{Ольга Морева} нууу... не знаю...
Сейчас, наверное, процветаем!)))

\begin{itemize} % {
\iusr{Андрей Журавлев}
\textbf{Maxim Kostenko} так ведь сейчас у каждого есть выбор - процветать или нет. Остальное - от лукавого...

\iusr{Maxim Kostenko}
\textbf{Andriy Zhuravlyov} ну да, ну да...
Самая бедная страна в Европе...
Видать выбрать никак не могут!

\iusr{Ольга Морева}
\textbf{Maxim Kostenko} кто именно выбрать не может?

\iusr{Андрей Журавлев}
\textbf{Maxim Kostenko} ну вы может и не можете выбрать. Я лично свой выбор еще в конце 80-х сделал.

\iusr{Maxim Kostenko}
\textbf{Ольга Морева} украинский народ!

\iusr{Maxim Kostenko}
\textbf{Andriy Zhuravlyov} )))
В Украине больше 40М жителей!!!

\iusr{Андрей Журавлев}
\textbf{Maxim Kostenko} и что из этого следует? То что 40 млн человек могут сделать свой выбор. Выбор как им жить. Но кто то этот выбор может и не делать, а плыть по течению и считать, что ему все должны)

\iusr{Maxim Kostenko}
\textbf{Andriy Zhuravlyov} вот видите, не все жк украинцы такие гении чедлвечесьва как Вы!)))

\iusr{Julia Ablamska}
\textbf{Maxim Kostenko} вы не ощущаете разницы между \enquote{бедная страна Европы} И \enquote{республика в Совке}? Если так, примите мои соболезнования

\iusr{Maxim Kostenko}
\textbf{Julia Ablamska} ну как сказать...
ВВП УРСР был где-то 20ке мира.

\iusr{Julia Ablamska}
\textbf{Maxim Kostenko} по мнению советской статистики? :)) ну-ну

\iusr{Андрей Журавлев}
\textbf{Maxim Kostenko} конечно не все. Особенно безграмотные с потугами на чувство юмора

\iusr{Андрей Журавлев}
\textbf{Julia Ablamska} не, ну танков на заводе Малышева точно больше почти всех делали. За исключением танковых заводов в РСФСР)))

\iusr{Maxim Kostenko}
\textbf{Julia Ablamska} нет, конечно!

Мировой банк для Вас авторитет???

http://en.classora.com/.../ranking-of-the-worlds-richest...

Украина - на 30м месте...
Где сейчас напомнить???

\iusr{Maxim Kostenko}
\textbf{Andriy Zhuravlyov} та вот... все 40 миллионов почти!

\iusr{Maxim Kostenko}
\textbf{Andriy Zhuravlyov} да не только танков...
Хотя, конечно, определенный дефецит был.

\end{itemize} % }

\iusr{Aleksandr Vasyliev}
\textbf{Ольга Морева} , сочувствую

\iusr{Андрей Журавлев}

В 1989 возможно Украина и была в тридцатке. По валовому продукту. Если оценить
реальную стоимость выпускаемого то показатели сильно рухнут вниз. Пример -
автор Запорожец ценою в районе 3000 рублей. По искусственному курсу 1 долл=65
копеек. Это советская цена. Реальная цена такого автор долларов 150 по
реальному курсу 1 долл=2 рубля. И так по всем видам продукции

\begin{itemize} % {
\iusr{Maxim Kostenko}
\textbf{Andriy Zhuravlyov} Чего????
Это Вы сами посчитали?)))

\iusr{Андрей Журавлев}
\textbf{Maxim Kostenko} любой человек, читающий в советское время газету Известия с курсом валют Госбанка СССР, и при этом знающий курс тогдашнего черного рынка, может это посчитать с достаточной степенью точности. Ну а госцена Запорожца никогда секретом не была.
\end{itemize} % }

\end{itemize} % }

\iusr{Светлана Алексеееко}
\textbf{Иван Захарович} был настоящий физрук. Он старался для детей-каток-был лучший !А школа была украинская.

\iusr{Татиана Корнейко}

Моя дочка училась в этой школе! Сейчас там самый лучший директор - Любовь
Ивановна! Каждого ученика знает по имени!

\begin{itemize} % {
\iusr{Андрей Журавлев}
к сожалению, с 1980 года, наша школа уже не английская(((

\iusr{Елена Царовская}
\textbf{Татиана Корнейко} 

Категорически не согласна. Знать имена учеников в маленькой школе- не велика
заслуга. А вот то что при ней школа совсем \enquote{скатилась} - это точно. Держалась
наша 87 только благодаря стараниям Наталии Витальевны Галкиной. Она ещё как-то
традиции держалась, но, увы, в минувшем году и она ушла

\end{itemize} % }

\iusr{Эльвира Земблевич}

Я училась в 30 школе с \enquote{преподаванием ряда предметов на английском языке} в
60-70 годы и даже обидно, ничего отвратительного об учителях припомнить не
могу. Была противная вахтерша баба Вера. Учителя физики Соломона Борисовича я
разыскала через ФБ в Израиле и на встрече одноклассников устроили видио
встречу. До слез. Школа дала все и инглиш, и физику, и математику, и
литературу, и спорт. Лучшее время!

\begin{itemize} % {
\iusr{Андрей Журавлев}
вы считаете, что я написал об учителях что то отвратительное? И в мыслях такого не было.

\begin{itemize} % {
\iusr{Эльвира Земблевич}
\textbf{Андрей Журавлев} я не вопреки Вам с любовью вспомнила о своей школе. Но даже при желании не смогла бы вспомнить ничего плохого. Спустя почти 50 лет мы с одноклассниками родные люди!

\iusr{Андрей Журавлев}
\textbf{Elvira Zemblevich} так и мы до сих пор с одноклассниками встречаемся. И по моему, то что написал я - совсем не плохое. А скорее просто смешное)

\iusr{Natasha Levitskaya}
\textbf{Андрей Журавлев}
Вот смешного точно нет в вашем рассказе...

\iusr{Андрей Журавлев}
\textbf{Natasha Levitskaya} ну так на всех не угодишь) Для кого то смешное, для кого то - нет. Кто то мой рассказ ругает, кто то хвалит, почитайте комменты. И это нормально. Значит рассказ удался. Да, и спасибо за ваш отзыв. Серьезно.

\iusr{Эльвира Земблевич}
\textbf{Андрей Журавлев} определенное настроение Вашего рассказа задала картинка, как дополнение образа учителей
\end{itemize} % }

\iusr{Maxim Kostenko}
\textbf{Elvira Zemblevich} Вы врете!
В СРСР была рабская система!!!
Да и комсомолочек насиловали!!!

\begin{itemize} % {
\iusr{Ольга Морозова}
\textbf{Maxim Kostenko} В какой школе, кто, в каком году насиловал? Попридержите свое воображение.

\iusr{Ирина Петрова}
\textbf{Ольга Морозова} скорее всего - это потуга выдать шЮтку юмора...

\iusr{Maxim Kostenko}
\textbf{Ольга Морозова} это же совок!!!

\iusr{Ольга Морозова}
\textbf{Maxim Kostenko} Совок - это ваши прадеды, родители, вы, ваши дети. И внуки тоже совок - генетика.

\iusr{Maxim Kostenko}
\textbf{Ольга Морозова} А Вы, значит, не из совка?)))

\iusr{Ольга Морозова}

В СССР были разные времена хорошие и трагичные. Многие люди жили достойно и
честно.Не надо всех и все поливать грязью. Критикуйте современные проблемы.

\end{itemize} % }

\end{itemize} % }

\iusr{Maryna Chemerys}

Я в цій школі проходила педагогічну практику на 5 курсі на початку 80-х - цілий
місяць викладала англійську учням 5-го класу. Їхнія вчителька хворіла, і мене
особливо ніхто не контролював. Діти були дуже хороші, вчились з цікавістю, ми
навіть невеличку виставу інглійською поставили. На щастя, з директоркою справи
не мала)

\begin{itemize} % {
\iusr{Андрей Журавлев}
Баба Надя пішла зі школи ще у 1976 чи 77 році на підвищення кудись у міністерство чи щось таке.

\iusr{Maryna Chemerys}
\textbf{Andriy Zhuravlyov} Значить, мені іпощастило)
\end{itemize} % }

\iusr{Андрей Гулый}
Наверное в каждой школе есть такая старая и абсолютно невменяемая личность.
У меня химичка такая была, вспомню, аж глаз дергается. Возрастом лет под сто и со старческим маразмом.

\iusr{Світлана Савдерова}

Красиво написано. Спасибо. Я училась в 80 школе на дружбе народов. И тех
перевод и военный перевод были. И школа украинская. И много учителей
ангязыка- это да! В те годы кажись в 90 школе на Печерске директор тоже баба
Надя была. Другая, но с такой же кличкой.

\begin{itemize} % {
\iusr{Ірина Загребельна}
\textbf{Светлана Савдерова} а у нас в 80-й с 1962 по 1972 был тех.перевод и английская литература. Военного перевода не было.

\iusr{Світлана Савдерова}
\textbf{Ірина Загребельна} я училась с 67по 77. Видимо изменили программу .

\iusr{Людмила Романова}
\textbf{Ірина Загребельна} в эти годы моя мама - Суворова В.В. преподавала в 80-й историю @igg{fbicon.face.grinning.big.eyes} 

\iusr{Ірина Загребельна}
\textbf{Людмила Романова} помню Валентину Васильевну! Благодаря ей я стала готовиться к каждому уроку истории, не сачкуя.

\iusr{Людмила Романова}
\textbf{Светлана Савдерова} баба Надя - это химичка была в 90 школе. Прекрасный человек и учитель. А директора звали Прасковья Потаповна.
\end{itemize} % }

\iusr{Dubizhansky Ludmila}

Да, музыка навеяла! Прекрасно помню бабу Надю! Но я закончила 87 школу в 66
году, и директором у нас тогда была Софья Германовна Ковалик, а завучем - Лидия
Трофимовна Чайка. Баба Надя тогда еще не свирепствовала. А самый яркий эпизод
нашей школьной жизни связан с директрисой. На уроке истории, средь шумного
бала, с нее упала юбка. Прямо в натуре, соскользнула и упала на пол в мгновение
ока. Она моментально ее подняла и выскочила из класса. Больше она у нас не
преподавала историю. Моя подруга Ленка, решившая прикинуться больной и
пропустившая школу в этот день, очень сокрушалась, что лишила себя такой
радости.

\begin{itemize} % {
\iusr{Андрей Журавлев}
\textbf{Лидия Трофимовна} Чайка на пенсию если и выходила, то не надолго. Она в 70-х преподавала у нас биологию. Потом в 83-м, после службы в армии, я заходил в школу, она еще работала.

\iusr{Violeta Tetiana Stetsenko}
\textbf{Andriy Zhuravlyov} Лидия Трофимовна Чайка до выхода на пенсию жила прямо в школе на первом этаже, где потом была столовая.

\iusr{Dubizhansky Ludmila}
Биологию преподавала другая Лидия Трофимовна, Дмитренко! Мы ее обожали, хотя она была очень строгой и требовательной.

\begin{itemize} % {
\iusr{Андрей Журавлев}
\textbf{Dubizhansky Ludmila} может я и перепутал насчет биологички... Но Дмитренко - фамилия бабы Нади. Это точно.

\iusr{Violeta Tetiana Stetsenko}
\textbf{Dubizhansky Ludmila} Она была потрясающая учительница! На ее уроке мы даже рыбу препарировали.

\iusr{Violeta Tetiana Stetsenko}
\textbf{Andriy Zhuravlyov} да, их было две, баба Надя и биологиня.

\iusr{Dubizhansky Ludmila}
\textbf{Violeta Topy} Такого литературного богатого и красивого украинского языка не было больше ни у кого. С ужасом узнала, что через несколько лет после нашего выпуска ее единственный сын утонул в Днепре во время экзаменов. Судьба была очень жестока к ней.
\end{itemize} % }

\iusr{Ната Шинкевич}

я училась в 87-й школе с 58 по 66 год. Директором сначала была Софья Германовна
(она жила в здании школы на первом этаже), а потом Дмитренко Надежда Сергеевна.
Завучем была Чайка Лидия Трофимовна, у нас она преподавала украинский язык и
литературу, а биологию преподавала Дмитренко Лидия Трофимовна. Английский язык
у нас преподавала замечательный педагог Слепян Нетта Абрамовна, она
ориентировала нас на перевод (поскольку мы учили англ.яз. с 5 класса), что
весьма пригодилось в дальнейшем.

\begin{itemize} % {
\iusr{Андрей Журавлев}
\textbf{Ната Шинкевич} про Нетту Абрамовну что то слышал, она по моему моего дядю учила (он 87-ю закончил в 1969)

\iusr{Ната Шинкевич}
\textbf{Andriy Zhuravlyov} с Неттой Абрамовной можно было поговорить об искусстве, музыке, литературе. Свой первый урок в новом учебном году она начинала с вопросов: кто где отдыхал и, главное, что прочитал за лето

\iusr{Dubizhansky Ludmila}
\textbf{Ната Шинкевич} 

Похоже, мы учились в одно время, но я Вас не помню, может, под другой фамилией?
Нетеу помню прекрасно. Интеллигентная, элегантная, с прекрасными манерами, одно
слово - аристократка! А у нас английский преподавала Анна Ефимовна, она потом
уехала в Москву, и много лет до ее отъезда в Америку мы виделись и
перезванивались


\iusr{Dubizhansky Ludmila}
\textbf{Ната Шинкевич} Видимо, Вы учились в одном классе с Наташей Загорской и Скуридиными! Поэтому мы в один год окончили школу! В английских классах Нета не преподавала, а наш выпуск был первым английским

\iusr{Елена Царовская}
\textbf{Dubizhansky Ludmila} 

Фамилия Анны Ефимовны - Блох. Она блестяще знала предмет. А вот Нета Абрамовна
в моих. воспоминаниях всегда с сигаретой и музыцирующая

\end{itemize} % }

\iusr{Елена Царовская}
\textbf{Dubizhansky Ludmila} я окончила школу в 70 ом, но эту историю и у нас пересказывали

\begin{itemize} % {
\iusr{Андрей Журавлев}
\textbf{Елена Царовская} 

про Клару Цыцкин? Неет... Это моя бабушка такое озвучила! Когда я был в 6 или 7
классе. Кто такая Клара Цеткин я тогда не знал, но Клара Цыцкин мне очень
понравилась)))

\iusr{Елена Царовская}
\textbf{Андрей Журавлев} 

Андрей, обратите внимание кому ответ адресован. Это по поводу юбки я написала. А
по поводу Вашего основного текста тоже имею что сказать. шас пишу

\end{itemize} % }

\iusr{Dubizhansky Ludmila}
Легенда!

\iusr{Андрей Журавлев}

Про погибшего сына слышал... Я и Лидию Трофимовну запомнил в трауре. У нас она
всегда ходила в строгом чёрном платье. Ну и ещё запомнилось, как она
периодически взрывалась из за нас, балбесов. Нам было лет по 12-13, ну и бывало
всякое на уроках. Типа плеваться из трубочки жеваной бумагой... НегIдник,
невiглас, покидьок - это малая часть того что она могла сказать в адрес
хулигана. А где бы мы ещё украинские соленые словечки выучили?)

\end{itemize} % }

\iusr{Валентина Козачук}

В моей школе №162 директора тоже были строгие, школа была украинская, все в
школе разговаривали по- украински.. Были разные учителя.. А вот боялись все
учительницу географии, она очень долго преподавала .Как урок географии, так на
всех переменах мы стояли вдоль стеночек на переменах с учебниками географии в
руках.. И тишина... Последние годы учебы 9 и 10 класс директором у нас была
учительница химии и биологии Мария Адамовна.. Я была всегда прилежная ученица,
так что проблем с нарушением дисциплины не наблюдалось.. Тогда каждую пятницу
мы старшекласники драили школу... мыли с мылом панели и резиночкой вытирали
черные полосы на линолиуме. Захотелось нам старшекласникам устроить вечеринку в
честь 8- Марта.. Так мы ходили к директору и уговаривали ее нам разрешить в
школе погулять, а за это мы всю школу вымоем до блеска.. Правда школа у нас была
маленькая, двухэтажная. Разрешила гулять в ее классе...

\begin{itemize} % {
\iusr{Lena Pavlikova}
\textbf{Валентина Козачук} 

Это не в частном секторе в старых Ново-Беличах? На ее стройке мы проходили
практику, а потом нас из 72-й перевели в эту. Директором была Нина Васильевна,
историк. 1959-1960 г. Школа украинская, но лучший преподаватель - Гращенко
Макар Иванович, - русязык,литература. Его не любили за принципиальность в
оценках. Отставной офицер. Математик Марфа Леонтьевна замечательная. Проходила
как-то возле школы, наша двухэтажка заброшена, а много чего достроили.

\iusr{Валентина Козачук}
\textbf{Lena Pavlikova} 

да это она, но я училась в 70-80 -х годах и этих преподавателей уже не
было.. Математику, рисование и черчение преподавал Нименький Владимир
Михайлович, был у нас класным руководителем.. Русский язык вела Берестова, и при
мне поменялось несколько директоров..

\end{itemize} % }

\iusr{Oleksa Vovk}

Хорошо описана система английских уроков. Именно так у нас в 51й школе и было.

\begin{itemize} % {
\iusr{Андрей Журавлев}

ну в 51-й в те времена учились супер мажоры - отпрыски Щербицкого, Ляшко (не
Скотыняки!), Ватченко и прочих столпов коммунизма в Украине.

\begin{itemize} % {
\iusr{Oleksa Vovk}
\textbf{Andriy Zhuravlyov} Внук Щербицкого, внучка Ляшко. Дети футболистов Динамо. И я.

\iusr{Андрей Журавлев}
\textbf{Oleksa Vovk} я понял, вы позже учились.

\iusr{Oleksa Vovk}
\textbf{Andriy Zhuravlyov} Да нет, именно в то время.

\iusr{Андрей Журавлев}
\textbf{Oleksa Vovk} значит я ошибся)
\end{itemize} % }

\iusr{Людмила Мозговая}
\textbf{Oleksa Vovk} да! Училась в ней. И директор Тамара Ивановна!

\iusr{Oleksa Vovk}
\textbf{Людмила Мозговая} Кищук

\end{itemize} % }

\iusr{Pavel Krasutsky}

Все совковые директора производились в одном Гороновском инкубаторе. Они играли
роль церберов-вертухаев, с детства приучая нас к концлагерю.

\begin{itemize} % {
\iusr{Людмила Мозговая}
\textbf{Pavel Krasutsky} 

помню великолепную директор школы 134 Альбину Ивановну. Рада что судьба
подарила мне встречу с этой изумительной женщиной.

\iusr{Ирина Петрова}
\textbf{Pavel Krasutsky} 

не обобщайте... не все... если не повезло в 38-й, то это не значит, что все
директора были такими...

\end{itemize} % }

\iusr{Татьяна Шиверская}

\ifcmt
  ig https://i2.paste.pics/ad57d2d5f30eea89c755a4a786bbbc5c.png
  @width 0.3
\fi

\iusr{Елена Царовская}

Да-а, Андрей. Залила Вам баба Надя \enquote{сала за шкуру} @igg{fbicon.grin}  коль скоро Вы до
сих пор ее ТАК помните. Прочла, невольно улыбнулась. Школа была действитльно
замечательная Тут недавно уже был пост о \enquote{бульварном} квартале ул.
Горького(нынче Антоновича).  Конечно и про школу нашу, и про учителей, и про
выпускников многое было сказано. Надо бы вспомнить и о том что школа
\enquote{була пiд патронатом Спiлки письменниуiв}. Дети и внуки многих
классиков окончили нашу школу. Преподавание велось на укр. языке, но на
переменах ( в том числе и учителя англ. языка) говорили на русском. 

Баба Надя действительно была достаточно ограниченной особой, но страха при этом
не внушала. Из туалета. гоняла регулярно ( именно так как Вы нарисовали) Но вот
писем на производство родителям ( да ещё и с последствиями) не было ни одного.
Её воспитательные методы особенно доставали Евгена Антоненко-Давидовича. Их
диалоги были достойны пера Жванецкого. 

А вот имена Вы и правда перепутали. Надежда Сергеевна Дмитренко- директор,
Лидия Трофимовна Чайка- завуч. Она действительно укр. язык преподавала, И
\enquote{командиром} над англичанами была не она, а Тамара Ивановна Луценко
(жена знаменитого поэта). К слову, их сын Сергей окончил нашу школу в 69 ом
году. 

Лидия Трофтмовна Дмитренко - преподавала биологию. Её трагически погибший
единственный сын Гарик учился со мной в параллельном классе. После Вашего
рассказа у читателей сложилось мрачное впечатление о школьных директорах. Не
хотите уравновесить рассказом об Ольге Андреевне Приймак.? Вы ведь учились в
период ее директорства? Или с ней тоже связан негатив?

\begin{itemize} % {
\iusr{Андрей Журавлев}

Елена, спасибо за ваш отзыв. Да, я действительно перепутал (а вы напомнили)
Лидию Трофимовну Чайку с Тамарой Ивановной Луценко. Именно Луценко была завучем
по английскому. Сергея Луценко я хорошо помню, он - одноклассник моего
покойного дяди, часто заходил к нам в гости, когда мы жили рядом со школой на
Горького. Не знал, что его отец - поэт. 

Хорошо, что вы напомнили про писателей.  Тараса Рыльского хорошо помню, он
учился на 2 или 3 класса старше меня. В моем классе училась дочь писателя П.
Дрофаня. 

Теперь про то, в чем вы ошибаетесь или просто не в курсе (вы учились гораздо
раньше меня, я учился 1969-1979). Кляузы на работу - баба Надя сама
рассказывала об этом на нескольких школьных линейках (у нас такие проводились
раз в неделю в воспитательных целях). Рассказывала типа для острастки, чтобы
достучаться до сознания раздолбаев. На переменах учителя английского (в мое
время) не всегда, но часто общались на английском и многие учителя чаще
общались на украинском, чем на русском. Тоже не все, но большинство. 

Баба Надя на моей памяти общалась только на украинском. Кстати, в мое время,
скорее всего, уже не было ваших учителей, те фамилии, что называли вы и ваши
ровесники, мне не знакомы. 

Ну и баба Надя абсолютно не заливала мне под шкуру сала. Она меня очень любила
и называла \enquote{гордістю школи} и даже иногда прощала мелкие шалости на ее
уроках. Баба Надя ушла из школы, когда я закончил 8 класс (как раз тогда, когда
я научился пить вино и курить). И 9 и 10 классы я уже учился при директорстве
Ольги Андреевны. Действительно, классная была тетка.

И, судя по другим комментам, у большинства читателей не сложилось мрачное
впечатление о нашей с вами школе. Да и рассказ я писал конкретно про бабу Надю,
и никак не про школу. Может напишу еще и про нее (школу). Если вам будет
интересно, то я буду здесь публиковать рассказы про Политех, где я отучился,
как и в школе тоже 10 лет (почему так - станет понятно из рассказов), рассказы
про всякие забавные случаи из жизни киевлян в 70-80 годы и т.д. Читайте на
здоровье.

\begin{itemize} % {
\iusr{Yuri Ginsburg}
\textbf{Andriy Zhuravlyov} Я тоже учился в одном классе с Сергеем Луценко. Как имя Вашего покойного дяди?

\iusr{Андрей Журавлев}
\textbf{Yuri Ginsburg} Миша Финкельштейн

\iusr{Yuri Ginsburg}
\textbf{Andriy Zhuravlyov} Мишу помню ... Это параллельные классы, Сергей перешел в класс \enquote{А} после шестого класса, кажется.
Цаество небесное Вашему дяде и моему соученику.

\iusr{Елена Царовская}
\textbf{Андрей Журавлев} 

Господи! \enquote{Как причудливо тасуется колода}! Вашего дядю знала ещё до школы. Я
уже обещала написать пост о парке Шевченко Точнее о том что в конце 50-х,
начале 60-х в парке гуляли группы дошкольнят. Эдакая альтернатива детского
сада. С Мишей мы были в такой группе. ( в таких случаях говорят - \enquote{горшочные
друзья}). Вы жили в 24-ом номере( дом геологич управления)? Вот в эту же
детскую группу ходили и Ваши соседи - Ершов, Дима Кубышкин. С Вами же мы
разминулись почти на десять лет (мое время 1960-1970гг). Но общие учителя, кроме
упомянутых директоров, завучей, все же были: бессменная химичка Елена Ильинична,
англ. язык-Пушко Алла Петровна, Рыбчинская Роза Борисовна. Думаю, что Вы помните и
Аллу Моргулян ( в девич Гальперина) - мою одноклассницу, которая в нашей школе
начинала старшей вожатой, а затем преподавала англ. язык.

\iusr{Dubizhansky Ludmila}
\textbf{Елена Царовская} 

Об Елене Ильиничне и Розе Борисовне остались самые теплые воспоминания. С
дочкой Розы Борисовны, Тамарой, мы встречалиси и после школы. Она очень хороший
врач.

\iusr{Елена Царовская}
\textbf{Dubizhansky Ludmila} 

Тамара Рыбчинская и её муж Игорь - прекрасные врачи. Кажется я уже писала, что
Рыбчинские года три живут в Австралии. Розе Борисовне в прошлом месяце
исполнилось 95!

\end{itemize} % }

\iusr{Олеся Розмариновская}

Мы, с подружками, знали телефон Антоненко-Давидовича и звонили, заранее зная, что
Женьки нету дома. И, когда, поднимал трубку его папаня, просили позвать \enquote{Женю к
телефону !!!} Ярий украинец, орал в трубку: \enquote{Тут, нема ниякого ЖенИ, тут йе, тильки
- ЕвгЭн !!!}

\begin{itemize} % {
\iusr{Violeta Tetiana Stetsenko}
\textbf{Олеся Розмариновская} Этот ярый украинец отсидел в лагерях и мама Евгена тоже, он сам родился на поселении, оттуда, вероятно, и его сильное заикание. Тяжелая судьба семьи да и Евгена.
\end{itemize} % }

\iusr{Violeta Tetiana Stetsenko}

Елена, в какие годы Вы учились? Мне кажется, мы гдето с Вами должны были
пересекаться, раз Вы так хорошо помните о диалогах с Евгеном
Антоненко-Давидовича. Я училась с ним в одном классе.

\begin{itemize} % {
\iusr{Елена Царовская}
\textbf{Violeta Topy} Конечно пересекались. Вы учились в классе Пушко Аллы Петровны. Окончила школу годом раньше(я училась в одном классе с Толей Хостикоевым). Из Вашего выпуска иногда вижу Наташу Усову
\end{itemize} % }

\iusr{Андрей Журавлев}

Елена, да мир тесен. Насчёт 24 номера вы не ошиблись. Кубышкина помню он к нам
часто заходил. Но лучше помню его отца. Он обучил меня азам игры в преферанс.
Вместе с мом дедом. Ершов - помню только фамилию, были такие соседи. Учителя,
тех кого вы перечислили, помню, но преподавала у нас только химичка. Тоже была
забавная тетечка. Особенно в свете эмиграции когда в 70-х начали уезжать
ученики школы. Надо будет про неё написать. Спасибо что напомнили. Ну и Аллу
Гальперину помню очень хорошо. Заловила нас когда то за распитием, но никому не
настучала)))

\iusr{Maryna Chemerys}

Тамара Іванівна вела у мене педагогічну практику. Я е знала, що вона дружина
Дмитра Луценка! Потім вона перейшла працювати до СШ 130.

\begin{itemize} % {
\iusr{Елена Царовская}
\textbf{Maryna Chemerys} 

Так, вона дружина автора слiв пiснi \enquote{Як тебе не любит Киеве мiй} Дуже
богато робила для вшанування його пам'ятi

\end{itemize} % }

\end{itemize} % }

\iusr{Елена Губаревич}

\ifcmt
  ig https://scontent-mxp1-1.xx.fbcdn.net/v/t1.6435-9/116335468_10220858321239697_4312399486041037377_n.jpg?_nc_cat=100&ccb=1-5&_nc_sid=dbeb18&_nc_ohc=pzDFvf-WEUIAX8oKcyu&_nc_ht=scontent-mxp1-1.xx&oh=00_AT_P3dq84jptX3Nu59b9sgm4_79_CZrwO7KnMT_7ORMd-Q&oe=62039287
  @width 0.2
\fi

\begin{itemize} % {
\iusr{Людмила Черненко}
\textbf{Елена Губаревич}, и это лицо мне знакомо. Ленинский район был маленький и мы наглядно знали всех.
\end{itemize} % }

\iusr{Елена Губаревич}

\ifcmt
  ig https://scontent-mxp1-1.xx.fbcdn.net/v/t1.6435-9/116293995_10220858326279823_5559528018568573590_n.jpg?_nc_cat=102&ccb=1-5&_nc_sid=dbeb18&_nc_ohc=anU9XLHKEiYAX_tqIVS&_nc_ht=scontent-mxp1-1.xx&oh=00_AT8MqXaJetrZiGww26ftxzh6J_6FlAS26_lim46f_lnHTQ&oe=6205E442
  @width 0.2
\fi

\begin{itemize} % {
\iusr{Violeta Tetiana Stetsenko}
Русская литература и особая любовь к Маяковскому.

\begin{itemize} % {
\iusr{Елена Царовская}
\textbf{Violeta Topy} 

не только к Маяковскому. Ещё Маленький принц, например. Вообще очень
талантливый, яркий, разносторонний человек. Преподавала рус. язык и ли-ру., уроки
интересные и совершенно не стандартные в рамках программы, была завучем по
воспитательной работе,. только потом - директором. Именно ее стараниями появилась
Агитбригада ( о ней уже писали). Только вот судьба печальная. После смерти в
минувшем году единственного сына актера Андрея Хоритонова сильно
сдала.... Грустно

\iusr{Андрей Журавлев}
\textbf{Елена Царовская} Елена, а где почитать про агитбригаду? Я был одним из первых ее участников в 1977(78?) - 1979 гг.

\iusr{Елена Царовская}
\textbf{Андрей Журавлев} 

Андрей, вынуждена Вас разочаровать. Агитбригада начиналась с нашего
выпуска, т.е. в учебном году 1969-70. Тогда же \enquote{оторвали} призовое место на
городском конкурсе. Прекрасно помню как с криком Р-е - в-ол-юция! выскакивал на
сцену Оля Матвеевская ( ныне проф. биологии). Вы уже были только достойными
продолжателями традиций. А упоминали этот коллектив здесь же в Киевских
историях ( когда речь шла о \enquote{бульварном} квартале) буквально несколько недель
тому назад. Правда Агитбригаду кто-то назвал театром. Не знаю как в ваше
время, но нашей - той первой, занимался Владимир Андреевич Приймак - младший брат
Ольги Андреевны). Вся семья талантливая, артистичная

\end{itemize} % }

\iusr{Людмила Черненко}
\textbf{Елена Губаревич}, 

очень хорошо помню. Я тогда работала старшей вожатой в 79. По-моему, мама
артиста, снимавшегося в роли Овода.

\begin{itemize} % {
\iusr{Елена Губаревич}
\textbf{Людмила Черненко} Да,

\url{https://uk.m.wikipedia.org/wiki/Харитонов_Андрій_Ігорович}

\iusr{Людмила Черненко}
\textbf{Елена Губаревич}, да!!! На минутку забыла!!!!

\iusr{Nadiya M Shana}
\textbf{Elena Gubarevich} Андрей учился и окончил 71 школу.

\end{itemize} % }

\iusr{Андрей Журавлев}

При мне скорее всего Ольга Андреевна возродила агитбригаду. Потому что до моего
9 класса я агитбригады не припомню. Да было весело, типа репетиции иногда
вместо уроков. У нас постановка была явно профессиональная, возможно та, что
была и у вас. Были много раз на всяких конкурсах, призы брали. К нам даже в
школу, на черных Волгах, приезжали пару раз какие то важные дяди, все школьное
начальство перед ними на задних лапках бегало. Показывали мы им свое
выступление. Дяди важно хлопали, потом державным тоном говорили, мол верной
дорогой идете, товарищи, и уезжали. Не знаю, кто это был. Примака Вову отлично
знал (это я про племянника О. А.), получается, что он сын В. А. Примака, он был
мой одноклассник, тоже был в агитбригаде. Уникальный был тип... Его все время
на \enquote{подвиги} тянуло... О. А. не знала, что с ним делать... В 1980 он сел
надолго за изнасилование не знаю жив ли.

\end{itemize} % }

\iusr{Pavel Chekh}
Лютый рассказ но интересный

\iusr{Andrey Kitchnyaev}

Учился в этой школе в 8 классе в 1970 году. С директором ни разу не пересекался
и абсолютно не интересовался межучительскими отношениями. Автор должен
обозначать временные рамки своего рассказа.


\iusr{Татьяна Шиверская}
Получила кайф, как говорит мой внук. Огромное спасибо!

\iusr{Олена Медведева- Прицкер}

Андрей Журавлев! Большое спасибо за прекрасный рассказ о школе и учителях. Я
внимательно прочитала все комментарии, но не нашла ни одного знакомого, хотя
многие мои друзья и одноклассники жили в те времена на Горького и учились в
знаменитой 145 школе.

\iusr{Марина Ковешникова}

Здорово. Хочу предложить рассказ о нашей директрисе. Мы называли ее ЕБ. Это от
имени и отчества. У нас было строгое правло писать чернилами и только
фиолетовыми. Я так понимаю, что она имела какое-то медицинское образование и
считала этот цвет успокивающим. Кроме того преподавала у нас биологию
человека.(могу ошибиться с названием предмета). Ну мы получили учебники. Какой
самый интересмный раздел. Да. Моче-половая система. Она нам была отдана на
самостоятельное изучение. Срез знаний был в виде контрольной. И молчек. А еще
хочеться сказать, что в моем фотоальбоме и моего брата, кот. старше меня на 12
лет, была одна и таже фотография ЕБ.

\iusr{Любовь Бодня}
А про бабу Параску и Майдана не хотите написать???!!!

\begin{itemize} % {
\iusr{Олег Кельнский}
\textbf{Любовь Бодня} самая лучшая английская и «элитная» была наша 57я)

\iusr{Андрей Журавлев}

Пока не знаю. А вот про киевских ватников, ненавидящих свою родину, напишу
обязательно. Откуда вы такие только повылазили?

\begin{itemize} % {
\iusr{Олег Кельнский}
\textbf{Андрей Журавлев} ты лучше про своих жлобов напиши, которые засрали Киев и сделали переворот)

\iusr{Олег Кельнский}
\textbf{Андрей Журавлев} 

у тебя с головой проблема. Что бы это написать, нужно жить в этой среде) А ты
сначала попробуй выехать «патриотичный» ты наш) И почему тебя должны волновать
социальщики?) Жлобская зависть ?))


\iusr{Олег Кельнский}
\textbf{Андрей Журавлев} вот видишь, ты даже не заметил, как стал считать чужие деньги, а это и есть жлобство)

\iusr{Олег Кельнский}
\textbf{Андрей Журавлев} хамло и жлоб он и в Африке ....)
\end{itemize} % }

\iusr{Марина Немезида}
\textbf{Светлана Савдерова} 

Мой дед в детстве и юности жил на Институтской у дяди ( он его и его сестру
забрал, когда умер их отец в Самаре). Но в 30-е семью репрессировали и всех
выслали. Дед жил и работал в Магнитогорске, Уфе, после выселения из Крыма татар
был направлен туда. Его мечта была вернуться в Киев, на Крещатик... Он завещал нам
это сделать. К сожалению, на Крещатик не удалось вернуться... Живем на Артема, а в
Крыму остались могилы и дом...

\end{itemize} % }

\iusr{Irena Visochan}
Прочитала с интересом! Какие все все-таки были разные школы в нашем городе! @igg{fbicon.hearts.two} 

\iusr{Надежда Кобко}

А. Журавель, благодаря этой школе и её директору вы получили неплохое
образование, судя по вашей грамотности. Но, написав такое о директоре, вы
поступили, извините, как никчемный человек. Такие, скорее всего, были
\enquote{стукачами} в советское время.

\begin{itemize} % {
\iusr{Ирина Петрова}

«О мёртвых либо хорошо, либо ничего, кроме правды»...? Так? Тысячам учителей
давали прозвища, они, кстати, их прекрасно знали, иногда виду не подавали,
иногда сами шутили. Автор вспомнил свои детские годы, таких учителей тогда
было, увы, очень много... кстати, даже в знаменитом фильме \enquote{Доживём до
понедельника} есть подобный образ... некоторые из таких учителей просто портили
жизнь ученикам, кстати, в этом рассказе о таком не написано. Наверное,
печальных последствий пед. работы такого преподавателя не было, что
замечательно.

\iusr{Світлана Савдерова}
\textbf{Надежда Кобко} 

Журавлев написал свои воспоминания. Так он помнит и так воспринимает. А вы на
личности и на оскорбления перешли. Кто тут не прав ? Скорее вы.

\begin{itemize} % {
\iusr{Надежда Кобко}
\textbf{Светлана Савдерова} 

Это он оскорбил человека, который не может ему ответить. Я не знакома ни с
одним, ни с другим. А со стороны мне кажется оскорблением именно директора
школы. Тем более, посмотрите, какую фотографию он разместил. Им, скорее всего,
движут личные обиды, это же очевидно. Почитайте другие его воспоминания, может
быть, посмотрите с другой стороны. Благодарю за Ваш ответ и за Ваше мнение


\iusr{Андрей Журавлев}
\textbf{Надежда Кобко} 

Вот личных обид у меня нет. Я, к вашему сведению, был одним из самых любимых
учеников бабы Нади. А если вы так воспринимаете мои воспоминания с высоты моих
же прожитых лет, то могу лишь вам посочувствовать. Как то так. Кстати, если вы
читали предыдущие комментарии, большинство людей, даже ученики моей же школы,
восприняли рассказ положительно. Но есть и отрицательные отзывы. Это означает,
что рассказ удался)))

\end{itemize} % }

\iusr{Надежда Кобко}
\textbf{Андрей Журавлев} Тогда Вы просто неблагодарный ученик, что очень печально. Как-то так

\begin{itemize} % {
\iusr{Андрей Журавлев}
\textbf{Надежда Кобко} вы имеете право думать как вам угодно.

\iusr{Андрей Журавлев}
\textbf{Надежда Кобко} 

кстати, насчет фото вы наверное правы. Баба Надя хоть и была \enquote{настоящим
коммунистом}, но в наличии ума ей не откажешь. Так что поищу другую
иллюстрацию.

\iusr{Violeta Tetiana Stetsenko}
\textbf{Andriy Zhuravlyov} Кстати, в мое время она всегда очень элегантно одевалась.

\iusr{Андрей Журавлев}
\textbf{Violeta Topy} ну такие детали я помню смутно...
\end{itemize} % }

\end{itemize} % }

\iusr{Agnessa Skuridina}

Я тоже закончила эту школу в 1966 году. У нас действительно директором была
Надежда Сергеевна, но то, чтобы кто-то называл ее бабой Надей, тем более Кларой
Цеткин не помню. И никто не замирал при ее появлении. У нас вообще были очень
демократичные отношения с учителями. Мы с ними ходили в походы (Якименко
Т.И.), на зимние каникулы в 11 классе ездили в Ленинград с Ламазовой Е.Е. Обо
всем этом сохранились самые тёплые воспоминания. А сын Ольги Андреевны Приймак
Андрей Харитонов стал известным артистом и сыграл главную роль в
фильме \enquote{Овод}. По-видимому, все о чем написал автор происходило уже значительно
позже.

\begin{itemize} % {
\iusr{Андрей Журавлев}

со временем много что меняется... В 1966 баба Надя еще, скорее всего, не вошла
во вкус власти. Кстати, бабой Надей ее уже называли, когда я поступил в школу
(1969 год). Сын Ольги Андреевны стал хорошим актером... А вот ее племянник Вова
(мой одноклассник), в 1980 сел за изнасилование...

\begin{itemize} % {
\iusr{Елена Царовская}
\textbf{Андрей Журавлев} 

Ни Вовы, ни его отца Владимира Андреевича, увы, уже нет в живых. Минувшим летом
после длительной болезни похоронили и Андрея Харитонова... Если для кого-то это
важно- урна с его прахом на Байковом. Там семейное захоронение.


\iusr{Андрей Журавлев}
\textbf{Елена Царовская} 

В школе мы с Вовой-младшим дружили. У него уже тогда проскакивали уголовные
наклонности. После школы мы еще поддерживали отношения, но я уже начал
сторониться его, так как \enquote{на подвиги} его тянуло все сильнее... Итог - весной
(или летом 1980) случилось групповое изнасилование, кстати было это на
цокольном этаже дома по Горького, 24... Так он сел, говорили на 8 лет. Больше я
его не видел. Возможно вы знаете, как его судьба сложилась после тюрьмы?

\end{itemize} % }

\iusr{Елена Царовская}
\textbf{Agnessa Skuridina} 

Вот я как бы промежуточное \enquote{временное звено} между Вами и автором этого поста.
Надежду Сергеевну,как только на посту директора она сменила Софию Германовну,
по моему сразу стали называть бабой Надей. Во всяком случае и я, и мои
одноклассники помнят это прозвище. Но уже писала о том что в наше время она
никакого страха не внушала. А прозвище звучало беззлобно и как-то даже по
домашнему.

\iusr{Dubizhansky Ludmila}

Конечно, согласна. Мы в одном году заканчивали школу, только вы 11-летку, а мы
-10. Эта школа оставила самые теплые воспоминания. Мы с Еленой Ефимовной
Ломазовой после 8 класса плавали на теплоходе по Днепру, маршрут назывался
\enquote{Дорогами 7-летки}. Надежда Сергеевна у нас не преподавала, и ее, безусловно,
никто не называл Бабой Надей. Уровень учителей был очень высокий. Объектом
насмешек скорее была историчка, даже имени ее не помню, единственная
безграмотная училка за все годы учебы. И в меньшей степени математичка Домникия
Евдокимовна, просто бесцветная, особенно в сравнении с нашей блестящей Лилией
Яковлевной Садыковой. Зато чертежник по кличке Чижик внушал трепет. Мой
одноклассник Карен ловил перед уроком трех мух, складывал из бумаги карету и
эту тройку гнедых выводил на парад по столу в начале урока. Хохоту было! Чижик
серьезно повлиял на характер моих романтических отношений на всю оставшуюся
жизнь. Из-за стойкого отвращения к черчению я обращала внимание только на
мальчиков с техническим уклоном со старших классов школы до первых курсов
института. Черчение и начерталка остались позади, а тенденция закрепилась
прочно. Поэтому из-за узости взглядов и кругозора даже замуж пришлось выйти за
инженера!

\begin{itemize} % {
\iusr{Андрей Журавлев}
\textbf{Dubizhansky Ludmila} 

Чижика я застал. В 8 классе он вел у нас черчение. Но к тому времени он уже
сильно сдал, ходил очень медленно, в общем возраст... Трепета уже не внушал.

\iusr{Андрей Журавлев}

а вот те, кто учился до меня, должны помнить старика Хоттабыча. Он жил в
полуподвальном помещении дома №4 по Горького, выход из него был как раз на
сторону школы. У него там даже маленький огородик был. У него была длинная
седая борода, ходил он с палочкой, одет был в черную фуфайку и сапоги, его
всегда сопровождала мелкая дворняжка. Мне он запомнился тем, что очень
неоднозначно реагировал на пионерские галстуки. Увидев стайку ребят в
галстуках, он громко начинал ругаться, самое безобидное было что то типа \enquote{у,
комсомольці, млять...} Один раз баба Надя, став свидетелем такой картины,
пробовала его урезонить, мол \enquote{дідусю, що трапилось?} В ответ Хоттабыч
замахнулся на нее палочкой с возгласом: \enquote{У-у, курва!}

\iusr{Violeta Tetiana Stetsenko}
\textbf{Dubizhansky Ludmila} 

Ой, Вы мне напомнили. Доминикия Евдокимовна Кодий преподавала у нас математику.
После её уроков у меня сложилось абсолютное непонимание алгебры. Из её перлов
запомнилось обращение к девочкам: \enquote{Серьожки наділа, а математику не вчиш!}.
Когда ты стоял у доски и молчал, а она к тебе подходила, была только одна
мысль, ударит или нет. А потом ей удалили аппендицит и она вернулась в класс
абсолютнго другим человеком, подобрела. Уже гораздо позже она была классным
руководителем у моего кузена, так он вспоминал ее только восторженно. Вот что
удаленный аппендикс делает!

\iusr{Dubizhansky Ludmila}
\textbf{Violeta Topy} 

Слава богу, преображение личности обошлось малой кровью и незначительными
потерями плоти! По-моему, Таисия Саввишна тоже математику преподавала, но не у
нас. Помню как сейчас ее мелкие черные кудряшки, широченную белозубую улыбку и
диссонанс почти негритянской внешности и такого теплого домашнего украинского
акцента.

По прошествии более чем 50 лет после выпуска с грустью констатирую, что таких
трепетных воспоминаний о школе, как у большинства из нашего поколения, у детей
наших уже нет. Мой сын окончил 126 школу в Дарнице почти 30 лет назад, с
друзьями-одноклассниками дружит и встречается, вместе ездит в отпуск, но об
учителях и школе практически не вспоминает. Может, такая школа..

\iusr{Ольга Свиткова}
\textbf{Андрей Журавлев} 

Дом 4/6 - это наш. Фамилия деда Калайда, был вредный, но на детей из \enquote{своего}
дома вредность не распространялась!

\end{itemize} % }

\iusr{Андрей Журавлев}

Давно это было, но мне сейчас кажется, что дед больше злился на красные
галстуки. К детям он агрессии не проявлял

\end{itemize} % }

\iusr{Алина Билык}
Иа фотка есть. Там 33 помню. Или где это. Что такое Антоновича

\iusr{Эльвира Земблевич}
антоновича - горького

\iusr{Марина Немезида}
А что касается школы - мы верны 57-ой, хотя есть и другие хорошие школы, в т ч
и 87

\begin{itemize} % {
\iusr{Андрей Журавлев}
у меня в 57-й двоюродная сестра училась - Зоя Финкельштейн. Не знаете ее?

\iusr{Марина Немезида}
Знакомая фамилия... Но не знаю
\end{itemize} % }

\iusr{Agnessa Skuridina}

Я тоже прекрасно помню чертежника Чижика. В мою бытность он был уже добродушным
и незлобливым пожилым человеком. Мы его не боялись и даже по-своему любили, не
отказывая себе в удовольствии пострелять в него из трубочек жеваной бумажкой,
когда он поворачивался лицом к доске. Он слегка отмахивался и тогда мы всем
классом вполголоса заводили популярную тогда песню «Сиреневый туман над нами
проплывает...». На это Чижик вообще не обращал внимания.

Нашему классу вообще многое прощалось. Мы были единственным русским классом в
украинской школе, переведённым из 33 школы. Каждый год нас собирались
расформировать, но с учётом того, что практически половина класса была
отличниками, нас оставляли в покое, хотя примерной дисциплиной мы не отличались.
Помню, как мы всем классом, наплевав на уроки, отправились на вокзал встречать
Фиделя Кастро. Он, тогда чуть ли не самый популярный мировой политический
деятель, приехал в Киев в сопровождении Н. С. Хрущева. Такое событие мы, конечно,
не могли пропустить.

К нашему удивлению, нас особенно не наказали. Отделались какими-то выговорами, а
учителя потом подходили и тихонько спрашивали: »Ну, как Фидель вживую? Похож на
того, что в телевизоре?». Мы заносчиво отвечали: »Надо было с нами идти».

\begin{itemize} % {
\iusr{Dubizhansky Ludmila}

Я помню почти весь ваш класс! Мне кажется, он был очень дружным и как-то более
социально однородным, в отличие от нашего, обычного класса обычной украинской
школы, которая через год после реорганизации престижной 92-й тоже стала
английской. Параллельный класс, переведенный из 92, был переполнен, и наш
пролетарский 2-А пополнился детьми интеллигенции.

\end{itemize} % }

\iusr{Agnessa Skuridina}
\textbf{Dubizhansky Ludmila}

У нас тоже математику преподавала замечательная Таисия Саввична. В младших
классах учились ее сыновья, с такими же негритянскими кудряшками. Мы их
называли Таисятами. Через несколько лет я, уже врач, встретила Таисию Саввичну в
Октябрьской больнице почему-то в инвалидном кресле. Мы искренне обрадовались
друг другу и обнялись, как родные. В нашей школе преподавали не только
прекрасные учителя, но и учились незаурядные личности. Стоит только назвать
народную артистку УССР Раису Недашковскую, сыгравшую в 10 классе Мавку в фильме
«Лiсова пiсня», народного артиста УССР Анатолия Хостикоева; Максима
Рыльского(внука знаменитого поэта); сына Юрия Збанацкого; заслуженного артиста Украины Тараса Денисенко, сына
известной актрисы Наталии Наум; Лукьяна Галкина, сына учительницы нашей школы
Наталии Витальевны, в настоящее время известного украинского критика,
генерального продюсера канала Культура.

\begin{itemize} % {
\iusr{Dubizhansky Ludmila}

И еще покойный, к сожалению, Богдан Жолдак, вы с ним учились в параллельных
классах. Конечно, помню их всех. Рая Недашковская была такая красавица, мы ею
всегда восхищались. А ее сестра Надя была нашей пионервожатой...


\iusr{Yuri Ginsburg}

Да, Таисися Саввична Прима. Прекрасный педагог и очень справедливый человек.

А с совершенно замечательным Лукьяном Галкиным я познакомился в позапрошлом
году, но не знал, что он тоже из 87ой.

\end{itemize} % }

\iusr{Agnessa Skuridina}

Да, наш класс был действительно очень дружным. К удивлению многих окружающих, мы
стали самыми близкими друзьями на всю жизнь, часто встречались хотя учились в
разных вузах. Создавали семейные пары, выступали друг у друга свидетелями,
вместе ходили в горы и занимались водным туризмом на байдарках, побывав на
многих реках от Карелии до Южного Урала. Конечно, мы - это не весь класс, но
сформировавшаяся ещё в школе группа, обросшая мужьями, жёнами, детьми и
внуками, поддерживает самые тесные отношения до сих пор.

А познакомила и объединила нас много-много лет назад школа №87. Ура!!!

\iusr{Agnessa Skuridina}
\textbf{Андрей Журавлев}

Так Вы жили в 24 доме на Горького? Как интересно. Я с1950г до сих пор живу в
этом доме. Недавно была публикация о «бульварном» квартале Горького, упоминался
и наш дом, его знаменитые жители. Мы с братом близнецом Сашей, тоже полжизни
прожившим в этом доме, долго вспоминали, кого мы знаем из семьи Зеркалей.
Получалось, что кроме Вовки-никого. А потом сообразили, что Вовка и есть отец
видного украинского дипломата Ланы Зеркаль. О Викторе Некрасове, Евгении
Мирошниченко, ранее проживавшими в нашем доме, мы, конечно, знали давно. Хорошо
помню ещё мальчиком Мишу Финкильштейна, Диму Кубышкина, семью Ершовых,
упоминавшихся в комментах.

Почему то никто не вспоминает замечательную учительницу физики Веру
Ивановну. Благодаря ей многие ребята из нашего класса выбрали техническую
профессию. Ее муж С. С. Дьяченко был зав. кафедрой эпидемиологии в Киевском
мединституте. Я слушала его лекции, которые он принципиально читал виключно
украiнською мовою, хотя абсолютно все преподавание велось тогда на русском
языке. Их сын Сергей Дьяченко стал знаменитым писателем-фантастом, лауреатом
многих престижных международных премий по литературе.

\begin{itemize} % {
\iusr{Андрей Журавлев}

Вову Зеркаля помню смутно он старше меня. А вот его отца дядю Леню помню хорошо
я даже про него реферат писал как о ветеране войны. Он мне ещё фото
презентовал. Баба Надя реферат одобрила). А вы не помните моего деда Михаила
Моисеевича? Мы жили в 17 квартире.

\end{itemize} % }

\iusr{Natasha Zagorskaya}

Спасибо, ребята - разбередили@igg{fbicon.heart.red}...

Огромный пласт нашей жизни, вернее - ее основа - это наше школьное время (по
крайней мере, для меня). С радостью увидела в комментах моих друзей, с которыми
дружим (страшно сказать!) с первого класса. Это была именно 87-я школа...

Школьные годы наши были чудесные! И не только потому, что мы были юными и
полными сил и надежд - мы чудесным образом совпали... И у нас действительно был
( и есть) очень дружный класс, и были особенные отношения с учителями, в
которых НИКОГДА не было страха и унижения, о котором многие написали. Мы были
просто младшими людьми...

Учителя наши были очень личностными людьми, с разными характерами, но весь
педагогический коллектив был ярким и запомнился на всю жизнь, со многими из них
у нас сложились прочные человеческие отношения на долгие годы. Нам повезло еще
в одном - возможно, это был один из периодов расцвета этого коллектива, как
бывает в театре. И тут как раз мы...) Другого объяснения я не нахожу.

Должна сказать, что знания, которые мы получили в нашей школе, оказались
прочными и дали возможность в год \enquote{двойного} выпуска (10-ти и последний
- 11-летки) и соответственно двойного конкурса в вузы многим из нас, без всяких
репетиторов, поступить в престижные украинские вузы... Я поступила в политех, и
физика с химией были сданы на \enquote{отлично}. Это продолжалось и на первых
двух курсах - \enquote{пятерки} в зачетке не переводились, даже однажды Юрий
Андреевич Сикорский поставил мне \enquote{5 с плюсом}.

Спасибо Вере Ивановне (физика) и Елене Ильиничне(химия), Лидии Трофимовне Чайке
( укр. язык и литература), и всем нашим учителям!

Приветствую всех, кто участвует в этих школьных зарисовках, желаю всем здоровья
и процветания! Моим любимым друзьям - моя любовь и пожелания скорой нашей
встречи@igg{fbicon.heart.red}!

\begin{itemize} % {
\iusr{Dubizhansky Ludmila}

Наташа, полностью разделяю твое отношение к школе и учителям. Действительно,
всех вспоминаю с огромным теплом и уважением. Не было ни одного случая, когда
кто-то из детей был несправедливо наказан или чувствовал предвзятое отношениек
себе. Нам действительно повезло на таких учителей, как Елена Ильинична, Вера
Ивановна. У нас преподавала украинский замечательная Анастасия Тихоновна,
прекрасный человек. А вот Алевтину Степановну, учительницу географии, я тоже с
благодарностью вспоминаю. У нас был очень сильный выпуск, и несмотря на двойной
конкурс, в том году многие поступили в ВУЗы. А я еще помню тебя председателем
совета дружины!!!

\iusr{Natasha Zagorskaya}
\textbf{Dubizhansky Ludmila} Милочка, я тебя вспомнила - как приятно встретить \enquote{своих} людей! Теплейшие воспоминания о нашей школе и людях...

\iusr{Natasha Zagorskaya}

\ifcmt
  ig https://scontent-frx5-2.xx.fbcdn.net/v/t39.1997-6/s480x480/69112831_916954355314613_6921504127247712256_n.png?_nc_cat=1&ccb=1-5&_nc_sid=0572db&_nc_ohc=a8fLgW0nsOAAX8TZww6&_nc_ht=scontent-frx5-2.xx&oh=00_AT9FDbJEhZcYO1ozf5uGxA7uDOGI07JBo9lFBGlxU44CYw&oe=61E38E8D
  @width 0.2
\fi

\end{itemize} % }

\iusr{Лидия Загребельная}

Английский язык в нашей школе был с первого класса! Нас готовили к языку с
детского сада, чтобы идти в школу в первый класс с базовыми знаниями. Вот как!

\iusr{Maryna Chemerys}

У нас в дитсадку теж була англійська для бажаючих, була дуже хороша вчителька.
Це в Києві була поширена практика. І я потім до англійської школи пішла і
багато чого вже знала)


\end{itemize} % }
