% vim: keymap=russian-jcukenwin
%%beginhead 
 
%%file 15_02_2017.stz.news.ua.mrpl_city.1.istoria_direktor_mmk_im_iljicha_aleksandr_garmashev
%%parent 15_02_2017
 
%%url https://mrpl.city/blogs/view/aleksandr-garmashev
 
%%author_id burov_sergij.mariupol,news.ua.mrpl_city
%%date 
 
%%tags 
%%title История: директор ММК им. Ильича Александр Гармашев
 
%%endhead 
 
\subsection{История: директор ММК им. Ильича Александр Гармашев}
\label{sec:15_02_2017.stz.news.ua.mrpl_city.1.istoria_direktor_mmk_im_iljicha_aleksandr_garmashev}
 
\Purl{https://mrpl.city/blogs/view/aleksandr-garmashev}
\ifcmt
 author_begin
   author_id burov_sergij.mariupol,news.ua.mrpl_city
 author_end
\fi

\ii{15_02_2017.stz.news.ua.mrpl_city.1.istoria_direktor_mmk_im_iljicha_aleksandr_garmashev.pic.1}

В краеведческой литературе Мариуполя личность бывшего директора завода имени
Ильича Александра Фомича Гармашева освещена скупо. А было время, когда его имя
в нашем городе гремело. Много легенд ходило о его жесткости, умении выполнять
самые сложные задания, правда, действуя более кнутом, нежели пряником. Что ж,
таково было время. Светлана Антоновна Исакова, будучи руководителем музея
комбината им. Ильича, собрала кое-какие документы о Гармашеве: несколько
страничек воспоминаний вдовы Александра Фомича, материалы предвыборных кампаний
в советы, от Ильчевского районного до Верховного Совета УССР, куда
баллотировался директор завода. Сведения о некоторых этапах  его служебной
карьеры удалось найти в интернете. Помогли и книги Д.Н. Грушевского \enquote{Имени
Ильича}, Ю.Я. Некрасовского \enquote{Огненное столетие}, Г.М. Захаровой \enquote{Кто он –
директор Радин?}.

Александр Фомич Гармашев родился в августе 1907 года в поселке станции
Дебальцево Екатерининской железной дороги. В двенадцать лет лишился отца. Фома
Федорович Гармашев был убит белогвардейцами в числе семидесяти трех шахтеров
шахты №4/5 \enquote{Провиданс}. Александр оставил учебу в школе и поступил на шахту,
где совсем недавно работал его отец. Только двадцатилетним он вновь сел за
парту – рабфаковскую. В 1935 году недавний шахтный рабочий окончил
Ленинградский индустриальный институт и с дипломом инженера-металлурга по
обработке металлов давлением приехал по направлению в Мариуполь на завод имени
Ильича. 

Молодой специалист был принят мастером в сортопрокатный цех, который достался в
наследство от \enquote{Русского Провиданса}, цех, где царил тяжелейший ручной труд. Там
Александр за три года прошел должностные ступеньки мастера, обер-мастера,
начальника смены. Потом он был переведен заместителем начальника, а затем стал
начальником цеха №8. При всей своей занятости он находил время для футбола – не
зрителя, а активного игрока заводской команды \enquote{Сталь}. Ветеран завода Игорь
Семенович Смык рассказывал: \enquote{Среди нас, мальчишек, ходила байка, что, мол,
полузащитнику Гармашеву разрешают бить мяч только левой ногой, потому что если
он ударит правой, то может убить подвернувшегося под мяч футболиста противника...}

7 июня 1938 года в Москве был арестован директор завода имени Ильича Николай
Викторович Радин. На следующий день, 8 июня, в его квартире в Мариуполе был
проведен обыск, и в этот же день приказом № 173 Народного Комиссара оборонной
промышленности был назначен новый директор завода – Александр Фомич Гармашев,
ему тогда было всего тридцать лет. Одна из газет того времени писала, будто
завод до прихода нового руководителя не выполнял плановых заданий, и только
благодаря тому, что под руководством Гармашева было развернуто социалистическое
соревнование, дела пошли на лад. Соцсоревнование помогло или волевые качества
Александра Фомича, но производственные показатели предприятия были столь
успешными, что он и передовой прокатчик Л.Т. Мирошниченко были избраны
делегатами  XVΙΙΙ съезда ВКП(б), который состоялся в марте 1939 года.

С первых дней нападения фашистской Германии на Советский Союз завод имени
Ильича перешел на номенклатуру и объемы продукции военного времени. Сверх
программы были изготовлены двенадцать бронепоездов. В таких условиях особенно
ярко проявились организаторские способности и неукротимая воля Гармашева. С
приближением фронта к Донбассу началась эвакуация оборудования в восточные
районы  страны. К утру 8 октября 1941 года она практически была завершена. На
совещании в заводоуправлении подводили ее итоги. И тут в директорский кабинет
ворвался диспетчер завода: \enquote{Немецкие мотоциклисты по улице Левченко движутся к
Первым воротам!} Руководство завода во главе с директором, военпреды,
специалисты, всего числом до восьмидесяти человек, двинулись через черный ход
на территорию завода, лавируя между цехами, в конце концов, покинули ее и
отправились в сторону Таганрога.

В эвакуации Александр Фомич трудился на оборонных предприятиях. В Сталинграде -
директором завода №264 (Сталинградская судоверфь), в Горьком – главным
инженером завода №112 (завод \enquote{Красное Сормово}), в Нижнем Тагиле  -
заместителем директора завода №183 (Харьковский паровозостроительный завод). За
умелую организацию работ по восстановлению танков и другой бронетехники А. Ф.
Гармашев был награжден боевым орденом Красной Звезды и медалью \enquote{За оборону
Сталинграда}.  

10 сентября 1943 года немецко-фашистские оккупанты были изгнаны из нашего
города. 11 сентября в Мариуполь прибыли А.Ф. Гармашев и ряд руководящих
работников завода имени Ильича, находившихся в эвакуации. Цеха и сооружения
завода лежали в руинах. 12 сентября начались восстановительные работы, на
следующий день в них приняли участие десять тысяч человек. Хронологию тех
незабываемых дней читаешь как сводку с поля боя: 20 сентября - дала ток
восстановленная турбина, 22 сентября – пущены цех водоснабжения и хлебозавод,
30 сентября – завод и Ильичевский район получили электроэнергию, 10 октября –
дала первую плавку стали после восстановления мартеновская печь №5 и т.д. Этот
бешеный  темп задал и поддерживал директор А.Ф. Гармашев. 29 сентября он сделал
доклад активу ильичевцев, в котором были определены мероприятия по
восстановлению завода и график пуска восстановленных объектов. Мероприятия и
график были утверждены. Еще шло восстановление цехов, а завод уже начал
выдавать продукцию. К концу 1943 года  было произведено 25 тысяч тонн стали,
более тысячи тонн проката, для  действующей армии изготовлено 1380 автодеталей,
сделан ремонт двадцати танков…

Настали мирные дни. Осваивались новые виды продукции, началось серийное
изготовление железнодорожных цистерн, была внедрена технология автоматической
электросварки труб, что позволило в сжатые сроки  изготовить 12 тысяч тонн труб
большого диаметра для газопровода Дашава – Киев. Во всех этих делах
непосредственное участие принимал Александр Фомич не только как администратор,
но и как инженер. В обстановке строжайшей секретности на заводе имени Ильича
были спроектированы и изготовлены первые в Советском Союзе устройства для
заправки ракет. За эту работу инженерам А.Ф. Гармашеву, П.М. Ходосу, А.В.
Лопатину, Л.С. Сигину в 1949 году было присвоено звание лауреатов Сталинской
премии ΙΙΙ степени. 

В 1950 году Александр Фомич был назначен директором Ижорского завода в городе
Колпино Ленинградской области. Но поработал там всего год. 26 февраля 1950 года
было издано постановление Совета Министров СССР, подписанное Сталиным, о
строительстве на Енисее, в пятидесяти километрах от Красноярска, подземного
комбината № 815 (будущего Горно-химического комбината), комплекса по
производству оружейного плутония. В 1951-1953 годах этим комбинатом руководил
Александр Фомич Гармашев. В 1953—1956 гг. он - директор Брянского
машиностроительного завода. Все предприятия, которые возглавлял герой этого
очерка, были частью военно-промышленного комплекса страны. В 1956 году было
утверждено Положение о Комитете по делам изобретений при Совете Министров СССР,
первым его председателем был назначен А.Ф. Гармашев. На этом посту он находился
с 1956-го по 1961 год.

Кавалер двух орденов Ленина, ордена Красной Звезды, лауреат двух Сталинских
премий, депутат Верховного Совета СССР, кандидат технических наук Александр
Фомич Гармашев скончался 5 марта 1973 года.
