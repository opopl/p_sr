% vim: keymap=russian-jcukenwin
%%beginhead 
 
%%file proza.rus.beresnev_alendr.1.ukrajna_pokajatys
%%parent proza
 
%%url https://proza.ru/2017/12/28/1392?fbclid=IwAR2M96aBFXRRDCnL08zMYkwiGqajiTCXoYXqsgQlp4Qb_ON7HIl1h8dcCxM
 
%%author 
%%author_id 
%%author_url 
 
%%tags 
%%title 
 
%%endhead 

\section{Мати УкРайНа, - чи не пора покаятись?}
\label{sec:proza.rus.beresnev_alendr.1.ukrajna_pokajatys}
\Purl{https://proza.ru/2017/12/28/1392?fbclid=IwAR2M96aBFXRRDCnL08zMYkwiGqajiTCXoYXqsgQlp4Qb_ON7HIl1h8dcCxM}

Алендр Береснев
(12.12.1999 р.)

«Так говорит Господь:  остановитесь на путях ваших и рассмотрите, и расспросите
о путях древних, где путь добрый, и идите по нему, и найдете покой душам вашим.

Но они сказали:  не пойдем.

К кому мне говорить и кого увещавать, чтобы слушали? Вот, ухо у них
необрезанное, и они не могут слушать;       вот, слово Господне у них в посмеянии:
оно неприятно им…   Ибо от малого до большого, каждый из них предан корысти,
и от пророка до священника --- все действуют лживо…»  (Иеремия, гл.6).
«Гвалт! Рятуйте, хто в Бога вірує!»  (Т.Г.Шевченко).

І виповнився світу  ЧАС.*

Та Україна* міцно спала.
З своїх кайданів не вставала.
На сни із золота прикрас
хтиво* надію покладала…
     Якби прокинутись, та згодом
     всім панством й трудовим народом
     від гордування* відійти
     і до покаяння піти!..
Та лиха ж доля!   Бо не знала
ні  Бога,  ні законів суть.
Тож в демократію криваво
і сліпо протинала путь…
Щоб голоса Христа* не чуть.
     Самозакоханність --- то горе.
     Від нього люд сліпим стає.
     …Так Перший світ прибрало море.
     Бо зло рятунку не дає.
Те зло --- є пиха* виховання,
що нині школа надає.
Бо в ній жіноче викладання,
що життю сенсу не дає.
     І та  с е р е д н я я  освіта
     породжувала дуресвіта
     від плодів древа  с е р е д  раю –
     --- його, як коноплю* я знаю.
Цей первісний наркотик Єви --
така пітьма, --- що Бог прости!
То ж на питання Бога: «де ви?»,
ніхто не зміг відповісти…
     А шлях до гідності людської –
     відомо з вікових начал!
     --- Та  звістка з уст Христа звучала!
     в зворотній стороні лежав …
              Закон суворий так казав:
              народ, якщо втрачає віру
              в Єдиного Отця  Творця –
              навалом* зло в серця стягає,
              що й наближає дні кінця...

Демократична Україна
не гірше за блудного сина
вже й самостійність прокляла –
за тридцять … центів продала…
              На перепуттях вік проводить,
              лихі знайомства там заводить.
              Та св‘ято бога-сатани
              св‘яткує посеред чуми…
              Ганьбить братів, чужих честує –
              як важкохвора репетує…
              Стовбичить на обох ногах
              Перед воротами Європи –
              --- та марно:  бо в її очах
              ми всі нікчемнії холопи.
___________________________________

Отож, брати, ЧАС*  вибирати:
чи повернутись до Життя,
чи…  доларові* догоджати,
змінивши гідність на сміття.
___________________________________

              О Господі!  Настав мій  ЧАС
              сполошним дзвоном  прозвучати.
              Прости нас,  Отче!  --- і всіх нас
              помилуй!!!
                Я ж пішов волати:
              --- Не спи!  Прокинься!  Св‘ята доля –
              --- то шлях до  Господа з неволі!
              Небесної не нехтуй школи!
              Прокинься ж нині !!!
                … чи ніколи…
___________________________________


               ПОСЛЕСЛОВИЕ С ПРИМЕЧАНИЯМИ:

     До предела сжато повествуя, я должен слегка приоткрыть тайны Слова, как могущественнейшего инструмента воздействия во всех сферах созидательного творчества, которое, увы, людям не вверено на пространстве Начального периода Бытия, ярко характеризующегося упорным нежеланием людей отстать от младенческого нечестия («ро»), и по покаянию осознанно устремиться по Пути духовного совершенства к исполнению своего ДОЛГА земного существования. Ибо только оно приносит угодный Богу плод, всецело надеясь всем сердцем на милость Небесного Учителя. Лишь при нашей благоговейной неотступности свершится записанное у Луки (24:45): «Тогда отверз им  УМЪ к уразумению Писаний».
То есть, с явлением на землю Сына Божия, чрез Которого произошли истина и благодать (Ин.1:17), у людей действительно появилась полнота надежды обрести  к л ю ч  разумения (Откр.3:7) Нового завета чрез покаяние, уверование и научение Христом всех учеников Своих, сколько бы их ни было за период Его двухтысячелетнего царствования времен Бытия до скончания века (Мат.28:20). «Потому что конец закона --- Христос, к праведности всякого верующего» (Рим.10:4).  Но здесь ---  ВНИМАНИЕ! --- слово «христос» несет в себе отвлеченное, отстраненное от  п р и в ы ч н о й  конкретики значение , изначально несущее в себе явление преобразующей силы осуществления замысла Творца по возделанию духовного плода совершенства, звучащее в полноте своей как: ОСВЯЩЕНИЕ  ПОМАЗАНИЕМ.  Это и есть название, или, вернее, ИМЯ той миссии, с которой Бог Вседержитель послал в мир людей Сына Своего под знаком имени этой миссии. Ибо «ОСВЯЩЕНИЕ ПОМАЗАНИЕМ» привычно принято нарекать в российском наречии языка греческими словами, поданными в латинизированной форме, как:  «ИИСУС ХРИСТОС» ! ---  Аминь!

     А это и есть ТАЙНА  ИМЕНИ  Божиего Помазанника. Которую ныне прозевали «иудеи» (Рим.2:28,29).
Ибо помазание --- это благословенное освящение МИРОМ земли (Евр.6:7), малая мера которой --- КОСТЬ {ТСВРЖЯ, В.И.Даль, М., 1881, т.2,с.177}– притчево обозначала сосуды плоти тел земных людей, в которых т.н. среднее образование (от плодов запретного древа СРЕДИ рая) производило терния и волчцы, как итог невежества действия заблуждения (2Фес.2:10,11).  Об этой земле сказано, что она  «негодна и близка к проклятию, которого конец --- сожжение» (Евр.6:8), так как такое «нечестивое производство» накликает на их головы войну (Ис.27:4; 13:11-13;  Иез.7:2-9!), как возмездие Господа Бога Саваофа (Рим.12:19; Иоиль,3:4,16). Это хорошо знал даже Уинстон Черчилль: «Если страна между нечестием и войной выбирает нечестие, то получает войну». Так что помазание --- это утверждение учением Иисуса Христа высшей доблести достоинства мужской храбрости --- ВОЗДЕРЖАНИЯ !

    --- Итак, все уверовавшие твердо знают и ясно разумеют, что Христос --- конец религии храмового, ознакомительного, --- но не преобразующего спасением! --- служения, служители которого «не разумея праведности Божией и усиливаясь поставить собственную праведность, не покорились праведности Божией» (Рим.10:3). Но религиозное служение с его Моисеевым законом не дает подвизавшимся  н а с л е д и е  спасения, «ибо не законом даровано Аврааму, или семени его, обетование --- быть наследником мира, но праведностью веры» (Рим.4:13).  Однако, почему же Закон, данный чрез Моисея, и известный ещё под именем ТОРА, в делах которого народ Израиля искал праведности (Рим.9:30-33), не доставил им праведности?

      Ответ, облаченный в тайну притчи, находим в самой ТОРЕ: «Не вари козленка в молоке матери его» (Исх.23:19; 34:26; Втор.14:21). Тайна притчи указывает на созвучие слов «тора» и «торакс», которые семантически однокоренные, где греч. torax --- женская грудь, указывает на источник душевного, плотяного питания похоронным хлебом (Осия,2:2! и 9:4), каковым является молоко матери, ибо «всякий, питаемый молоком, несведущ в слове правды, потому что он младенец; твердая же пища свойственна совершенным, у которых чувства навыком приучены к различению добра и зла» (Евр.5:13,14). В созвездие этих славных слов входит также красноречивое лат. torus --- выпуклость, и греч. tragos --- козел, вырастающий из козленка, вскормленного молоком матери, что и составляет суть смысла такого планетарного явления, как  ТРАГЕДИЯ;  и лат. mamma --- женская грудь, известное в российском наречии как «мама» --- и это слова, действительно несущие в себе тайны.   Из добрых ли?

      Но Господь сказал ученикам Своим, --- сколько бы их ни было и откуда бы они не явились пред лице Его: «вам дано знать тайны Царствия Божия, а прочим в притчах, так-что они видя не видят и слыша не разумеют» (Лк.8:10). Так притчей о козлике Господь возвещал всех нас о безумии эмансипации, о выведении сына из-под опеки ОТЦА. Первый пример такого безумия явил нам Адам своим ослушанием к словам Божиего запрета, и сполна испил «сеничкин яд» этого молока от древа  СРЕДИ  рая...
      К чему же тогда изумляться трагической истории бытия планеты Земля, если за грех неверия и пренебрежительного ослушания Бог сказал Адаму: «за то, что ты послушал голоса жены твоей и ел от дерева, о котором Я заповедал тебе, сказав:  «не ешь от него»,  проклята земля за тебя... Терние и волчцы произрастит она тебе...» (Быт.3:17,18).  Вот почему звучат две тысячи лет суровые слова Божиего предостережения, записанные чрез ап. Павла: «жена да учится в безмолвии, со всякою покорностью; а учить жене не позволяю, ни властвовать над мужем, но быть в безмолвии. Ибо прежде создан Адам, а потом Ева; и не Адам прельщен, но жена, прельстившись, впала в преступление; впрочем спасется чрез чадородие, если пребудет в вере и любви и в святости с целомудрием»(1Тим.2:11-15).

     И есть великая тайна в словах:  «И сказал Господь Бог: не хорошо быть человеку одному;  сотворим ему помощника, соответственного ему» (Быт2:18). Но помощника В ЧЕМ?  Судя по характерному портрету женщины, обрисованному чрез Соломона и ап.Павла --- явно не для помощи в исполнении долга временного существования, требующего колоссального преимущества причины над следствием (а Ева была ПРОИЗВОДНОЕ от Адама, т.е. следствие) и неизмеримого превосходства Творца над тварью: «женщина безрассудная, шумливая, глупая и ничего не знающая», «всегда учащаяся и никогда не могущая дойти до познания истины» (Пр.9:13; 2Тим.3:7). --- Замечу, что ГЛУПОСТЬ --- это мера удаления от Бога, от Света истины (Ис.9:2; 49:6; Ин.8:12), по стезе которой пошел первый мир за Евой и погиб от вод потопа, явив тем самым катастрофическую участь всякого противления помыслам Бога (Быт.7:21,22; и Ис.14:24).  Ибо «в начале словом Божиим небеса и земля составлены из воды и водою: потому тогдашний мир погиб, быв потоплен водою. А нынешние небеса и земля, содержимые тем же Словом, сберегаются огню на день суда и погибели нечестивых человеков» (2Пет.3:5-7).
     Для сохранения в памяти этого назидательного события и существует слово МОРЕ , которое в свете Нового языка раскрывается как «наследие противящихся» --- мо-ре. Так и слово «Моав» раскрывается как «потомство, наследие отца» --- «мо Ав» (архимандрит Никифор, «Библ. Энц-я», М.,1891, с.479). И так как от потопа спаслось всего восемь душ (1Пет.3:20), то тогдашний мир позволительно охарактеризовать этим словом «мо-ре», а нынешний, так далеко ушедший во тьму демократического атеизма, в церковность храмового служения (!NB:  Чис.16:40; 3Цар.12:31; 4Цар.17:32; 2Пар.13:9; Иез.22:26; 34:2-10; 44:9-13, ибо Бог предстоятелей святилища лишил священства --- Ис.43:28; Осия,4:6), --- воспринял обличающий символ пустоты от др. арамейского слова «МА» --- пустота, что удвоением при неизбежности осуществления сего (Быт.41:32) превращается в «МАМА» --- пустыня. И это результат смертоносного питания (Быт.2:17) «козлят» цивилизации молоком матери их (Исх.23:19). Все ли видят бремя этой духовной трагедии, разродившейся проказой духа? (патриарх Тихон, «Росія въ проказе»; читанное 14 января 1918 г. в Николо-Воробьинской церкви Москвы на литургии).
     Вот почему так странно сказал пророк Исаия об Иоанне Крестителе: «глас вопиющего в ПУСТЫНЕ...». Ибо Предтеча проповедовал в ИУДЕЙСКОЙ ПУСТЫНЕ, говоря: покайтесь, ибо приблизилось Царство Небесное. Об этой современной «иудейской пустыне», бесчисленно умножившей ныне свои культовые храмы (ибо у христиан храмов НЕТ ! --- архмдт Никифор, «Библ. Энц-я», М., 1891, с. 544; Мат.6:6; Лк.17:20-23), Иоанн Богослов предрекал в Откровении, называя их «сатанинскими сборищами, из тех, которые говорят о себе, что они Иудеи, но не суть таковы, а лгут» (Откр.3:9), ибо не желают покаяться.

      Но это еще не все. Бог, «Который в прошедших родах попустил всем народам ходить своими путями, хотя и не переставал свидетельствовать о Себе благодеяниями», «в последнии дни сии говорил нам в Сыне, Которого поставил наследником всего, чрез Которого и веки сотворил. Сей, будучи сияние славы и образ ипостаси Его и держа все словом силы Своей, совершив Собою очищение грехов наших, воссел одесную (престола) величия на высоте, будучи столько превосходнее Ангелов, сколько славнейшее пред ними наследовал имя» (Деян.14:16; Евр.1:2-4). Вот в тайне этого не познанного людьми слова «имя» скрываются многие неожиданности, ибо Ева, уведя вселенную от славнейшего имени, дарующего благоприятный исход из Времени достойным исполнением долга людей на Пути «СПАСЕНИЯ ПОМАЗАНИЕМ», увлекла нас соблазнами (Мат.18:7) на земных пажитях, произращающих терние и волчцы. «Имея пажити, они были сыты; а когда насыщались, то превозносилось сердце их, и потому они забывали Меня» (Осия,13:6; 7:14;   Пр.30:9,21-23).  Об этой «забывчивости» и напоминают  н ы н е  известные САНКЦИИ в последний раз. Разумеют ли это люди? Хоть кто-нибудь им ЭТО объясняет?
     «Трубит ли в городе труба (англ. TRUMP --- знамение времени о Втором пришествии: 1Фес.4:16,17; --- А.Б.), --- и народ не испугался бы?  Бывает ли в городе бедствие, которое не Господь попустил бы?  Ибо Господь Бог ничего не делает, не открыв Своей тайны рабам Своим, пророкам» (Амос,3:6,7).
«Вот, предсказанное прежде сбылось, и НОВОЕ  Я возвещу;  прежде нежели оно произойдет, Я возвещу вам» (Ис.42:9).

                «Сказал безумец в сердце своем: «нет Бога». Развратились они, и совершили гнусные преступления;  нет делающего добро.   Бог с небес призрел на сынов человеческих, чтобы видеть, есть ли разумеющий, ищущий Бога.   Все уклонились, сделались равно непотребными; нет делающего добро, нет ни одного.   Неужели не вразумятся делающие беззаконие, съедающие народ Мой, как едят хлеб, и не призывающие Бога?» (Пс.52:2-5). Как здесь не вспомнить слова царя царей Навуходоносора: «И все, живущие на земле, ничего не значат; по воле Своей Он действует как в небесном воинстве, так и у живущих на земле, и нет никого, кто мог бы противиться руке Его и сказать Ему: «что Ты сделал?»...
Ныне я, Навуходоносор, славлю, превозношу и величаю Царя Небесного, Которого все дела истинны и пути праведны, и Который силен смирить ходящих гордо» (Дан.4:32-34).
                Неужели так худо «среднее образование» от Евы, что народы столь сильно заблудились?..
     «И сказал Гоподь: не вечно Духу Моему быть пренебрегаемым человеками; потому что они плоть... что велико развращение человеков на земле, и что все мысли и помышления сердца их были зло во всякое время» (Быт.гл.6).  «Гнева нет во Мне. Но если бы кто противопоставил Мне волчцы и терны, Я войною пойду против него, выжгу его совсем.   Разве прибегнет к защите Моей, и заключит мир со Мною?  тогда пусть заключит мир со Мною» (Ис.27:4,5). Не единственный ли это путь к миру на земле?

     И в подобие последнего парламентера «пред наступлением дня Господня, великого и страшного», Бог послал в мир пророка исповедания  и м е н и  Своего, то есть человека, разоблачающего светом учения Истины тьму невежества «жителей земли», душевное достоинство которых (Иак.3:15) в Библии наречено словом «евреи», как начальной меры образования общего (лат. communis) достоинства, этого ярого поборника плотского противления младенчества детей Евы, «ибо плоть желает противного духу, а дух --- противного плоти: они друг другу противятся, так что вы не то делаете, что хотели бы» (Гал.5:17).
И обличающим фактором этой духовной невозрожденности, этого упорного невежества в стремлении к стабилизации в конце века, выступает острая актуальность «Послания к Евреям» апостола Павла, которое по своему назидательному накалу обличающего характера должно бы называться «ПОСЛАНИЕ К ЦИВИЛИЗАЦИИ», доныне пребывающей по причине отступления (2Фес.2:3) в «еврействе» с его тщетными поисками закона праведности (Рим.9:30-33), и не зрящей необходимости «вникнуть в закон совершенной свободы» (Иак.1:25), поистине освобождающей от «лжи действия заблуждения, посланного им Богом за то, что они не приняли любви Истины для своего спасения» (2Фес.2:10-12). То есть, не приняли  СПАСЕНИЕ  ПОМАЗАНИЕМ. Разве они не читали: «И познаете истину, и истина сделает вас свободными ... Итак, если Сын освободит вас, то истинно свободны будете» (Ин.8:32-36)?

      Итак, чем же их так прельстили «терние и волчцы»?
      И чем же тогда вразумительно объяснить тот факт необходимости посылания Богом в мир «пророка пред наступлением дня Господня, великого и страшного», то есть НЫНЕ? Не тем ли, что Бог проклял священников земли за лицеприятие и вероломство (Малахия, гл.2), и неотменимо лишил их священства за отступление (Исаия,43:27,28)?     Неужели в этой тьме нет ни одного духовного, который «судит о всем, а о нем судить никто не может» (1Кор.2:15), который бы донес слова Божиего ультиматума  «ПОКАЙСЯ  ИЛИ  ПОГИБНИ»  из судьбоносного знамения Ионы пророка до слуха нынешних руководителей стран мирового сообщества, о которых сказано: «род лукавый и прелюбодейный ищет знамения; и знамение не дастся ему, кроме знамения Ионы пророка... Ниневитяне восстанут на суд с родом сим и осудят его, ибо они  ПОКАЯЛИСЬ  от проповеди Иониной; и вот, здесь больше Ионы» (Мат.12:39-41)?
     И ныне многие обозреватели и эксперты узрели необычное явление концентрации власти в одних руках среди руководящего аппарата каждой из ведущих стран мирового сообщества, что смутило их неутвержденные души ожиданием чего-то необычного, даже грозного. Те немногие, которые «обратились от тьмы к свету и от власти сатаны к Богу» (Деян.26:18), увидели в этом явлении прямое указание на побуждение к принятию архитрудного решения по этому последнему ультиматуму-заповеди Саваофа.
И все видят, что  КОМПРОМИСС  ЗДЕСЬ  НЕВОЗМОЖЕН.

Смысл же нашего временного существования возглашает (Иер.7:23) насущную необходимость исполнения  д о л г а  --- рождения свыше (Ин.3:7,3) путем совершенства, «доколе все придем в единство веры и познания Сына Божия, в мужа совершенного, в меру полного возраста Христова;  дабы мы не были более младенцами, колеблющимися и увлекающимися всяким ветром учения, по лукавству человеков, по хитрому искусству обольщения» (Ефес.4:13,14).  «Ибо будет время, когда здравого учения принимать не будут, но по своим прихотям будут избирать себе учителей, которые льстили бы слуху; и от истины отвратят слух и обратятся к басням» (2Тим.4:3,4).
    А совершенство «в меру полного возраста Христова» упомянуто ап. Павлом, именуемым прежде Савл Тарсянин, как процесс неотступного духовного движения по пути преобразования, внутренне содержащего как бы четыре неразрывных этапа нарастающего духовного возрождения: «Они Евреи? --- и я. Израильтяне? --- и я. Семя Авраамово? --- и я. Христовы служители? --- в безумии говорю: я больше» (2Кор.11:22,23).  И эта «мера полного возраста Христова» вполне удовлетворяет смыслу нашего существования, к завершению которого мы должны были обрести через страдания оставления опустошающего самолюбия бесценный плод духовного совершенства, притчево называемый «чело века»: человек, пророк, исповедник Христа, наследник богоусыновления... --- это все одна суть.

 --- То не унтерменши ли те, кто противится долгу духовного совершенства, упорно цепляясь за «сытые пажити питания похоронным хлебом» для насыщения едино только плоти, а смысл существования на пути исполнения долга не только не воплощают, но даже представления о нем не имеют?
Можно ли считать ученика первого класса школы-десятилетки зрелым выпускником? Или студента первого курса --- дипломированным специалистом? И где те ВУЗы, что профессионально и достойно готовят для жизнеутверждающего служения нынешних президентов? Почему так происходит?
     «Это оттого, что народ Мой глуп --- не знает Меня; неразумные они дети, И  НЕТ  У  НИХ  СМЫСЛА;  они умны на зло, но добра делать не умеют... Смотрю --- и вот, нет человека...» (Иер.4:22,25).

     «Так говорит Господь: проклят человек, который надеется на человека и плоть делает своею опорою, и которого сердце удаляется от Господа. Он будет --- как вереск в пустыне и не увидит, когда придет доброе... Лукаво сердце более всего и крайне испорчено; кто узнает его? --- Я, Господь, проникаю сердце и испытываю внутренности, чтобы воздать каждому по пути его и по плодам дел его» (Иер.17:5-10). «Перестаньте вы надеяться на человека, которого дыхание в ноздрях его: ибо что он значит?» (Ис.2:22).
     Здесь в тайне Небесной притчи говорится о том, как мы «сканируем» информационное поле вселенной, «считывая» посредством дыхания, как процесса прокачки попеременными вдох-выдохами воздуха –возрождающего духа, наполненного помимо света, газов, энергии уже известных полей воздействия, еще и невыявленными и непознанными субстанциями энергоинформационного воздействия, транслируемых для всеобъемлющего обеспечения воплощения замысла творения из Центра мироздания, называемого по общему согласию Богом Творцом Вседержителем.  Трансляция этой неподвластной уразумению людей СИЛЫ (см. 1Кор.4:20) является фактором своевременного воплощения в действительность воли Творца, в полноте абсолютной достаточности выраженной  д у х о м  Писаний, которым наполнен Новый язык. Этому многотрудному языку научается всякий, уверовавший на пути совершенства (Мк.16:17) в обетование Бога. «Обетование же, которое Он обещал нам, есть жизнь вечная» (1Ин.2:25).
     Об этой ТРАНСЛЯЦИИ написано, что «душевный человек  НЕ  ПРИНИМАЕТ  того, что от Духа Божия, потому что он почитает это  б е з у м и е м ; и не может разуметь, потому что о сем надобно судить духовно. Но духовный судит о всем, а о нем судить никто не может» (1Кор.2:14,15).
                Так что же все-таки «принимают» душевные, земные люди?
Есть-таки и злой дух от Господа (1Цар.16:14-16,23), который порабощал в отсутствие дел совершенства «и вас, мертвых по преступлениям и грехам вашим, в которых вы некогда жили, по обычаю мира сего, по воле князя, господствующего в воздухе, духа, действующего ныне в сынах противления, между которыми и мы все жили некогда по нашим плотским похотям, исполняя желания плоти и помыслов, и были по природе чадами гнева, как и прочие...» (Ефес.2:1-3) --- здесь ап. Павел свидетельствует о принявших дар благодати от Бога, которые благодатию были спасены чрез веру, но прежде были мертвы по преступлениям, будучи поражены злым духом среднего образования, господствующего в сем мире.
     Как здесь искренне не пожалеть тех, кто отверг своим упорством  «СПАСЕНИЕ ПОМАЗАНИЕМ»?.. То есть, отвергли Иисуса Христа? --- Ведь должны были стать сынами БОЖИИМИ, а не религиозными...
   
      Вот на этом пути совершенства Божией любовью, Савл Тарсянин однажды услышал слова, ставшие  р о л е в о й  сущностью его последующего служения. Однажды, «идя в Дамаск со властью и поручением от первосвященников, среди дня на дороге я увидел, государь, с неба свет, превосходящий солнечное сияние (говоря языком современных понятий --- как бы световое излучение ядерного взрыва, --- А.Б.), осиявший меня и шедших со мною. Все мы упали на землю (не от ударной ли волны? --- А.Б.), и я услышал голос, говоривший мне на Еврейском языке: Савл, Савл! что ты гонишь Меня?  трудно тебе идти против рожна.  Я сказал: кто Ты, Господи?  Он сказал: Я Иисус, Которого ты гонишь; но встань и стань на ноги твои; ибо Я для того явился тебе, чтобы поставить тебя служителем и свидетелем того, что ты видел и что Я открою тебе, избавляя тебя от народа Иудейского и от язычников, к которым Я теперь посылаю тебя,  открыть глаза им, чтобы они обратились от тьмы к свету и от власти сатаны к Богу, и верою в Меня получили прощение грехов и жребий с освященными» (Деян.26:12-18).
     Воистину верны слова пр. Исаии: «если вы не верите, то потому, что вы не удостоверены» (Ис.7:9).

      «Увы, народ грешный, народ обремененный беззакониями, племя злодеев, сыны погибельные!
Оставили Господа, презрели Святого Израилева, --- повернулись назад. Во что вас бить еще, продолжающие свое упорство? Вся голова в язвах, и все сердце исчахло... Земля ваша опустошена; города ваши сожжены огнем; поля ваши в глазах ваших съедают чужие; все опустело, как после разорения чужими...  И когда вы простираете руки ваши, Я закрываю от вас очи Мои; и когда вы умножаете моления ваши, Я не слышу: ваши руки полны крови.
     Омойтесь, очиститесь; удалите злые деяния ваши от очей Моих; перестаньте делать зло; научитесь делать добро; ищите правды; спасайте угнетенного; защищайте сироту; вступайтесь за вдову.
     Тогда придите, и рассудим, говорит Господь. Если будут грехи ваши, как багряное, --- как снег убелю;  если будут красны, как пурпур, --- как вОлну убелю.      Если захотите и послушаетесь, то будете вкушать блага земли.    Если же отречетесь и будете упорствовать, ТО МЕЧ ПОЖРЕТ ВАС: ибо уста Господни говорят...  И сильный будет отрепьем, и дело его --- искрою;  и будут гореть вместе, --- и никто не потушит» (Исаия, гл.1).
«Посему бодрствуйте, памятуя, что я три года день и ночь, непрестанно со слезами учил каждого из вас. И  н ы н е   предаю вас, братия, Богу и Слову благодати Его, могущему назидать вас более и дать вам  НАСЛЕДИЕ  со всеми освященными» (Деян.20:31,32).    Это слова ап. Павла, но и Иисус прежде говорил: «У пророков написано: «и будут все научены Богом». Всякий, слышавший от Отца и научившийся, приходит ко Мне» (Ин.6:45) --- «Итак выйдем к Нему за стан, нося Его поругание; ибо не имеем здесь постоянного града, но ищем будущего, и так будем чрез Него непрестанно приносить Богу жертву хвалы, то есть, плод уст, прославляющих имя Его» (Евр.13:13-15).
     Но скажет кто-нибудь: а что значит «прославлять имя Его»? и что такое --- ИМЯ?

          Научившиеся у Господа по Его приглашению: «Приидите ко Мне, все труждающиеся и обремененные... и научитесь от Меня... и найдете покой душам вашим...» (Мат.11:28-30; +1Кор.2:12) получили духовное удостоверение (см.Ис.7:9) в животворной силе истины и благодати (Ефес.2:8), словом правды пасущих всяких учеников Христа, и обрели на пути духовного совершенства ключи Царства Небесного (Откр.3:7).
          Этими ключами и явился Новый язык. Он появился животворящей духовной отраслью воплощения неотменимой воли Творца, коей принадлежит сокрытое тайной несокрушимое могущество и которое непостижимо выше всякого разумения людей, неотвратимым движением в вечности и во времени осуществляющее замысел Вседержителя: «С клятвою говорит Господь Саваоф: как Я помыслил, так и будет; как Я определил, так и состоится… Ибо Господь Саваоф определил, и кто может отменить это?  рука Его простерта, --- и кто отвратит ее?» (Ис.14:24,27). «Кто укажет Ему путь Его, кто может сказать: «Ты поступаешь несправедливо»? (Иов,36:23).  Не упустить бы еще при этом, что «видимое временно, а невидимое вечно» (2Кор.4:18):  «сокрытое принадлежит Господу, Богу нашему, а открытое нам и сынам нашим до века» (Втор.29:29). Поэтому все храмы земли не Божии, а людские, из определения: «Не можете служить Богу и маммоне», ибо «отдавайте кесарево кесарю (деньги: пошлин, налогов, «на храм» и т. п.—А.Б.), а Божие Богу», т. е. исповедание учения Христа --- «Спасение Помазанием» (см. Мат.6:5-7; Мат.10:32; Евр.13:15; Пс.49:14; Иов,36:24). И еще свидетельство архимандрита Никифора, «Библейская Энциклопедія», М., 1891г., с.544:
                «Іудеи внушали обратившимся ко Христу соотечественникамъ, что законъ Моисеевъ данъ при посредстве ангеловъ, что Моисей выше Іисуса, преданнаго на безславную смерть, что Богоучрежденное при Моисее богослуженіе само по себе великолепно и достойно Божества, а у Христіанъ, напротивъ, НЕТЪ  НИ  ХРАМА, ни священства, ни алтарей, ни жертвъ». Но!! --- Ис.43:28!
    «Так говорит Господь: небо --- престол Мой, а земля --- подножие ног Моих; где же построите вы дом для Меня, и где место покоя Моего?» (Ис.66:1).  «Бог, сотворивший мир и все, что в нем, Он, будучи Господом неба и земли, не в рукотворенных храмах живет и не требует служения рук человеческих, как-бы имеющий в чем-либо нужду, Сам дая всему жизнь и дыхание и все…» (Деян.17:24-31).
    И возникает насущный вопрос: а есть ли ИМЯ у смысла нашего существования? а у служения, с которым Господь Бог послал Сына Своего к «погибшим овцам дома Израилева» (Мат.15:24)??!!
        ИМЯ --- это программное обеспечение духовной миссии руководства;   это отображение силы могущества роли деятельности, служения, существования, назначенное Богом для оповещения знамений грядущих событий на кратковременном пространстве Бытия.


     Новый язык появился из сокровищницы умершего «великорускаго живаго языка», в котором люди не отыскали пользы, ибо посеян был «оживляющий» язык на каменистую почву сердец нечестивых. Исходя из слов Христа: «не здоровые имеют нужду во враче, но больные» (Мат.9:12), Господь Бог подал племени нечестивых могущественнейшее «лекарство» «живаго языка», в духе которого была создана  РУСКАЯ  БИБЛІЯ. Но с подачи митрополита Серафима (Глаголевский) --- «Не поймет народ!..», царь Николай 1 запретил в 1826 г. («до высочайшего изволения») деятельность Библейского Общества по печатанию Библии по низкой цене, которое создал в 1813 году князь А.Н.Голицын, министр просвещения на то время. Такое положение дел не способствовало духовному возрождению народов империи.
Протестантизм же, исповедовавший научение народных масс Новому завету по Учению Иисуса Христа без поповства и храмового служения, робко проник на короткое время в узкие круги петербургского света чрез немногочисленные выступления лорда Редстока и общество Василия Пашкова лишь спустя 400 лет после своего возникновения. В советское время Руская Библия, печатавшаяся в Америке и Англии, просачивалась в страну посредством туристов. Закон Божий и поныне доминирует на пространствах России, незримо противясь животворящему Духу Учения Христа (Ин.6:63), этому  СПАСЕНИЮ  ПОМАЗАНИЕМ. Последовательность исповедания воли Бога была повреждена: «ибо закон дан чрез Моисея,
БЛАГОДАТЬ же и истина произошли чрез Иисуса Христа» (Ин.1:17).  Однако «отменение же прежде бывшей заповеди бывает по причине ее немощи и бесполезности, ибо закон ничего не довел до совершенства; но вводится лучшая надежда, посредством которой мы приближаемся к Богу»(Евр.7:18,19)

    «Бог, богатый милостью, по Своей великой любви, которою возлюбил нас, и нас, мертвых по преступлениям, оживотворил со Христом, --- БЛАГОДАТИЮ ВЫ СПАСЕНЫ, --- и воскресил с Ним, и посадил на небесах во Христе Иисусе, дабы явить в грядущих веках преизобильное богатство благодати Своей в благости к нам во Христе Иисусе. Ибо  БЛАГОДАТИЮ  ВЫ  СПАСЕНЫ  ЧРЕЗ  ВЕРУ,  и сие не от вас, Божий дар:  не от дел, чтобы никто не хвалился» (Ефес.2:4-9).
              Это звучит Новый завет Господа нашего Иисуса Христа.    Его имя в текстах Писаний несет животворящий смысл долга нашего существования:  СПАСЕНИЕ  ПОМАЗАНИЕМ.  Вочеловечивание.
«Говоря «новый», показал ветхость первого; а ветшающее и стареющее близко к уничтожению»(Евр.8:13)

      Почему же не отыскали смысл всего сущего, хотя и читали: «просите, и дано будет вам; ищите, и найдете»? (Мат.7:7). «Почему?  потому что искали не в ВЕРЕ, а в делах закона;  ибо преткнулись о камень преткновения, как написано: «вот, полагаю в Сионе камень преткновения и камень соблазна;  но всякий, верующий в Него, не постыдится» (Рим.9:30-33). Поэтому «не принесло им пользы слово слышанное, не растворенное верой слышавших» (Евр.4:2). Окончательно же похоронили «рускій языкъ» в начале ХХ-го века, истребив педаГОГической реформой 1918 года его таинственную азбуку, таившую в себе духовный ключ к уразумению Нового языка, постигаемого на пути совершенства посредством УМА, отверзаемого едино только Господом Иисусом Христом. И есть странное выражение сути одного явления: «функциональная роль Мыслете --- утверждение», что в Новом языке кратко записывается --- УМЪ.
Вспомнить бы еще, что: «И назвал Бог твердь небом» (Быт.1:8). Отсюда глагол «утверждать» стал тем действенным фактором, что воссоздает бессмертное «небесное тело» (1Кор.15:40-49).
      Итак, «если пшеничное зерно, падши в землю, не умрет, то останется одно; а если умрет, то принесет много плода» (Ин.12:24). В результате воплощения этой замечательной притчи многие увидели смертельное разделение «живаго» языка, как одно из свойств руководящего совершенства, в силу которого (см.1Кор.4:20) диалекты (наречия, говоры и т.п.) устремились к обособлению от др. диалектов, дабы сделаться самостоятельными языками. Но это были всего лишь производные от функции начального языка, только призрачные тени гадательного характера описания явлений, а не сам свет сути вещей. Но не все уразумели его тайную дифференциацию, в основе которой сокрыты как самолюбие, так и связанное с ним духовными нитями возмездия (Рим.11:36; 12:19)  всепожирающее наползание деградации, вырождения, упадка...

     Косвенное указание на эту смерть языка мы находим у В.И.Даля (ТСВРЖЯ, М., 1882, т.4, с.216) в словарной статье «СЛАВЯНСТВО», ибо смерти предшествует знамение разделения (Мат.12:25):
«Часть задунайской славянщины отуречилась, а полабская онемечилась».
Онемеченное наречие «великорускаго живаго» языка со временем стало называться «украинской мовой». Слово «онемеченное» произошло от имени богини воздаяния Немезиды (Немециды, Немесиды), неизбежно карающей за преступление, как ослушание исполнения воли Вседержителя.
     Таким образом дошли до нас смутные, земные наречия белорусского, югославского, польского, российского, украинского и др. диалектов первой волны славянства, как прославляющее «служение смертоносным буквам, начертанное на камнях» (2Кор.3:7-11), т.е. служение осуждения, служение религиозному закону, исполнение которого должно было остановить безумное бегство вселенной (всех народов и племен земли) в пропасть проклятия (Быт.3:17,18). Потому закон мертвит, прославляясь в преходящем служении осуждения (2Кор.3:9).  Это храмовое служение.
     Была назначена Богом и вторая волна славянства: «служение Нового завета, не буквы, но духа», последующее служение оправдания. «Доныне, когда они читают Моисея, который полагал покрывало на лице свое, чтобы сыны Израилевы не взирали на конец преходящего, умы их ослеплены: ибо то же самое покрывало доныне остается неснятым при чтении Ветхого Завета, потому что оно снимается Христом... 
     Господь есть Дух; а где Дух Господень, там свобода» (2Кор. гл.3).

     Библия содержит в себе ответы на все вопросы, какие только могут возникнуть на земле.
Благоговейно и многотерпеливо, с высочайшим воздержанием в поступках, можно в продолжение своей суетной жизни достичь на пути совершенства плода благополучного исхода, как некой чистой одежды нравственности, приготовленной на брачный пир. Руководящим напутствием на этом пути был пример кротких жителей Верии, которые «были благомысленнее Фессалоникских: они приняли слово со всем усердием, ежедневно разбирая Писания, точно ли это так...» (Деяния, 17:11). Это методика.
     Время (бремя, беремя, беременность) завершилось... Истекает день НЫНЕ (Евр.4:7). Всему свое время.

ПРИМЕЧАНИЯ:
1. Украйна, --- общее душевное имя вселенной по удалению «На край» от Бога, как Центра мироздания, а также страны, для знамения миру в конце века о начале «очищения» поражением «у края стаНа» (Чис.11:1) при явлении разделения на республики;  res  publica --- общественные дела противления (Богу).
2. Час (часъ), --- знамение грядущего «утверждающего истребления червя» --- «Час воли Божией», а также бренд появления ныне десяти руководителей стран мирового сообщества, которые «примут власть со зверем, как цари, на один ЧАС; они имеют одни мысли...» (Откр.17:12). В России, например, знаковое появление герба царя-императора; царского торгового флага, как обличителя торговой сделки Иакова и Исава (Быт.25:30-34! и Малахии,1:2-6); появление TV программы «Шестьдесят минут», что == ЧАС...
         ЧЕРВЬ --- нарицательное прозвище душевного статуса невозрожденных людей: «в земле, слепой и ненасытный», своим нечестием притягивающий на свою голову САНКЦИИ в преддверии ВОЙНЫ.
3. хтиво, --- похотливо, с самонадеянной спесью.
4. гордування, --- пренебрежение.
5. Христа, --- здесь: «христос», как имя акта помазания при духовном преобразовании совершенством;
    «СПАСЕНИЕ ПОМАЗАНИЕМ» = «Иисус Христос» --- имя Божиего плана по сотворению достойного плода в сосудах плоти людей для благополучного исхода  ЧЕЛОВЕКОВ в бессмертие, Великая Победа (1Кор.15:54).
6. пиха, --- глупость гордого надмения.
7. конопля, --- рай-дерево (Быт.2:9,17; 3:3), Ricinus communis; (ТСВРЖЯ, В.И.Даль, 1881, т.2, с.152).
8. навал, --- Библейское «Навал» --- безумный (1Цар.25:25).
9. долар, --- лат. dolor --- боль, страдание, скорбь... В величайшей тайне сокрыта суть этого явления. Иисус Христос не зря, ох не зря, осознанно и действенно воспринял его.  VIA DOLOROSA --- путь скорби --- есмь величайший подвиг из всех возможных. Он подвигает к глубочайшему переосмыслению происходящего, к мобилизации поиска смысла при молчаливом созерцании действительности, когда в тишине и благоговейном покое, трансляция духа «МЫСЛЕТЕ» обретает приоритет над истерическим визгом «глушилок» СМИ, как полпреда бесовского князя, «господствующего в воздухе, духе, действующего ныне в сынах противления» (Ефес.2:2).
    Молчание --- есмь могущественнейшее «слово», дарующее по нарастающей тот незримый приоритет веры, который пробуждает «функциональную, действенную роль Мыслете --- утверждение», одним словом «УМЪ».
    Почему же мир это слово "доллар" воспринимает иначе? Под действием атеистического заблуждения (2Фес.2:10-12) произошла коммунистическая (от лат. communis - общеобразовательный) переориентация сознания, породившая термин "перевернутый мир".

10. Роль, --- «нечестие людского разделения»; покаяние --- оставление роли, отвержение самолюбия, самости, устремление к миру с Богом,  у-ничИжение, а не  у-ничТОжение (Иона,3:6;  Дан.4:14).
11. Родина, --- нечестивая (земная, душевная) сила, ибо «дина» --- единица силы в системе СГС; а в Библ. Энциклопедии архимандрита Никифора,  «Дина» --- судъ (жестокій, женскій, т.е. юный, из-за истории, связанной с Диной, дочерью Иакова --- см. Быт. гл.34, где поступили вопреки Закону: Втор.22:28,29).
12. Майдан, --- у Никифора: Данъ --- судія, ма --- пустота;  «ма-й-Данъ» --- в предупреждающем прочтении (справа-налево), читается: «утверждение Судией освящения пустоты»; Судия --- (Иаков,4:12).
13. Пророк, --- ПРО --- покой нечестия, погибель;  РОКЪ --- нечестие «земного» утверждения, проказа духа;
      пророк --- раб Божий (Амос,3:7), исповедник истины (Ис.49:3; Ин.14:6), человек (Иов,36:24,25).
14. Рош, --- у Никифора: «Рошъ» --- глава (Иез.гл.38;39); ныне: лидер атеизма (от др.евр. Ат --- мать);
      «утверждение смертоносным неблагоприятным нечестием» --- секира (Мат.3:10), занесенная над неплодной смоковницей, и «величающаяся» пред Тем, Кто рубит ею (Ис.10:15).
15. Террор, --- рус. ТЕРОРЪ --- абсолютное и всеобъемлющее убеление чистотой непорочности;  святость Господа Бога Саваофа. См. Иез.7:9.
16. Смоква, --- разрушение «земного» наследия Словом.   Москва --- «разрушение «земным» словом наследия» (Рим.4:13).
17. Сирия, --- рус. Сирія (Сирііа), --- « Слово бодрствующего Божиего возделания (образования)» --- «пробуждение»;  «асадъ» --- «утверждение земного (АДового) истребления» (Соф.1:18;  Иез.21:30-32)
18. ЕВРО, --- нечестивое ведение силой, энергия нечестия...(Иез.7:19;  1Тим.6:9,10).
19. ЕВРОПА, --- образование покоя нечестивого ведения силой...   
20. Трамп (англ. TRUMP), --- 1. труба;  2. книжн. трубный глас; 3.the last TRUMP, the TRUMP of doom ---  (архангельская) труба, которая зазвучит в день Страшного суда --- к чему бы это ?!... (1Фес.4:16,17)
          и т.д.

                ЗАКЛЮЧЕНИЕ:
«Душевный человек не принимает того, что от Духа Божия, потому что он почитает это безумием; и не может разуметь, потому что о сем надобно судить духовно.  Но духовный судит о всем, а о нем судить никто не может» (1Кор.2:14,15).

     Я мирно исполняю на планете свое посольство исповедания Истины, о чем сказано: «Так и вы, когда исполните все повеленное вам, говорите: «мы рабы ничего нестоющие, потому что сделали, что должны были сделать» (Лк.17:10)».  И так как многие пренебрегают живым общением со мной, тщеславясь своим «высоким» (Лк.16:15) положением в мире, то я исполняю волю Пославшего меня: «посему надлежало тебе отдать серебро Мое торгующим, и Я пришед получил бы Мое с прибылью» (Мат.25:27).

    
      Благослови вас Господь Бог! 
