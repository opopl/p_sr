% vim: keymap=russian-jcukenwin
%%beginhead 
 
%%file 28_10_2021.fb.dudkin_jurij.3.osvoboditeljam
%%parent 28_10_2021
 
%%url https://www.facebook.com/y.dudkin/posts/4835143983202491
 
%%author_id dudkin_jurij
%%date 
 
%%tags 1944,mirvojna2,nacizm,rkka,sssr,ukraina,vov
%%title ОСВОБОДИТЕЛЯМ
 
%%endhead 
 
\subsection{ОСВОБОДИТЕЛЯМ}
\label{sec:28_10_2021.fb.dudkin_jurij.3.osvoboditeljam}
 
\Purl{https://www.facebook.com/y.dudkin/posts/4835143983202491}
\ifcmt
 author_begin
   author_id dudkin_jurij
 author_end
\fi

ОСВОБОДИТЕЛЯМ

Ежегодно, 28 октября в Украине отмечается День освобождения Красной Армией от
фашистских захватчиков. Этот государственный праздник посвящен окончанию
Восточно-Карпатской операции во время Великой Отечественной войны, когда
немецкие войска были изгнаны с территории Советской Украины.

\ifcmt
  pic https://scontent-lhr8-2.xx.fbcdn.net/v/t1.6435-9/249680709_4835143909869165_2958709630208332370_n.jpg?_nc_cat=104&ccb=1-5&_nc_sid=8bfeb9&_nc_ohc=G8PAjpcwSDsAX9PIKbi&_nc_ht=scontent-lhr8-2.xx&oh=71bed60790ce7ed911ad57c9468189b6&oe=61A204CD
  @width 0.8
  %@wrap \parpic[r]
\fi

С января 1943 года по осень 1944 года велись бои за освобождение всей
территории Украины. В этот период на украинской земле были сосредоточены до
половины всех советских войск.

На территории Украины Красная армия провела в общей сложности 15 наступательных
операций.

Прочитав несколько постов мною уважаемых людей замечаю, как невольно исчезают
уточнения, что освобождение было совершено победоносной Красной Армией, а
определение Великая Отечественная война - вообще не упоминается. Пропаганда
современной Украины делает свое дело. Даже окрепшие умы уже свыклись с тем,
чтобы называть войну просто "второй мировой", не говоря уже о неокрепших.

Не стесняйтесь, друзья мои, называть вещи своими именами. А подвиги советских
солдат-освободителей Красной Армии произносить громко и гордо!

С ПРАЗДНИКОМ, УКРАИНА!

СЛАВА ВОИНАМ КРАСНОЙ АРМИИ И

СОВЕТСКОМУ НАРОДУ!

\ii{28_10_2021.fb.dudkin_jurij.3.osvoboditeljam.cmt}
