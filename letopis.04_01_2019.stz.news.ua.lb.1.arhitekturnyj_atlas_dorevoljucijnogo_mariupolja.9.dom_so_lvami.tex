% vim: keymap=russian-jcukenwin
%%beginhead 
 
%%file 04_01_2019.stz.news.ua.lb.1.arhitekturnyj_atlas_dorevoljucijnogo_mariupolja.9.dom_so_lvami
%%parent 04_01_2019.stz.news.ua.lb.1.arhitekturnyj_atlas_dorevoljucijnogo_mariupolja
 
%%url 
 
%%author_id 
%%date 
 
%%tags 
%%title 
 
%%endhead 

\subsubsection{Дом со львами}

В том же стиле, но несколько ранее, по собственному проекту Нильсен возвёл ещё
один дом. Среди горожан он известен как \enquote{дом со львами}. Ещё до революции
архитектор продал эту недвижимость казачьему офицеру, а сам переехал в
новостройку у Городского сада. В 30-х годах, после того как имущество \enquote{буржуев}
отдали народу, в \enquote{доме со львами} проживал директор завода \enquote{Азовсталь} Яков
Гугель. В 1937 году выяснили, что он, как член контрреволюционной организации,
проводил активную работу против политики ВКПб. Директора и строителя
\enquote{Азовстали} расстреляли, а семью репрессировали. Вероятно, после этого дом был
передан в коммунальную собственность и поделен на квартиры. Так и есть до сих
пор. У \enquote{дома со львами} несколько хозяев, и вместе с тем, ни одного. Зданию не
оказывается достойный уход, и его состояние постепенно ухудшается.

\ii{04_01_2019.stz.news.ua.lb.1.arhitekturnyj_atlas_dorevoljucijnogo_mariupolja.9.dom_so_lvami.pic.1}

\emph{Второй большой пласт мариупольских архитектурных древностей составляют корпуса
учебных заведений. Исходя из особенностей их функционального назначения,
подобные строения отличались масштабом и основательностью. К проектированию и
строительству таких сооружений предъявлялись особые требования, а сам процесс
всегда сопровождался общественным вниманием. Благодаря этим обстоятельствам
учебные заведения старого Мариуполя стали относительно хорошо изученными
фрагментами старины.}

\ii{04_01_2019.stz.news.ua.lb.1.arhitekturnyj_atlas_dorevoljucijnogo_mariupolja.9.dom_so_lvami.pic.2}
\ii{04_01_2019.stz.news.ua.lb.1.arhitekturnyj_atlas_dorevoljucijnogo_mariupolja.9.dom_so_lvami.pic.3}
\ii{04_01_2019.stz.news.ua.lb.1.arhitekturnyj_atlas_dorevoljucijnogo_mariupolja.9.dom_so_lvami.pic.4}
