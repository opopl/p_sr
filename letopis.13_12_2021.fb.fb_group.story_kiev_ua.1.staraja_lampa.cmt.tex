% vim: keymap=russian-jcukenwin
%%beginhead 
 
%%file 13_12_2021.fb.fb_group.story_kiev_ua.1.staraja_lampa.cmt
%%parent 13_12_2021.fb.fb_group.story_kiev_ua.1.staraja_lampa
 
%%url 
 
%%author_id 
%%date 
 
%%tags 
%%title 
 
%%endhead 
\zzSecCmt


\begin{itemize} % {
\iusr{Татьяна Сухарева}

\ifcmt
  ig https://i2.paste.pics/8fdc4fd8530d8cf8fe41afcd8672fa53.png
  @width 0.2
\fi

\iusr{Vitali Andrievski}
Точно такой солонкой с крошечной ложечкой мы пользуемся до сих пор

\iusr{Tatyana Mazerati}
u nas toje bila takaya..

\iusr{Диана Тараненко}

\ifcmt
  ig https://scontent-frt3-1.xx.fbcdn.net/v/t39.30808-6/265821322_1379535735834938_3248021953733934623_n.jpg?_nc_cat=107&ccb=1-5&_nc_sid=dbeb18&_nc_ohc=uMAFbkcyYAcAX8GsdID&_nc_ht=scontent-frt3-1.xx&oh=00_AT_REOdPmED8d1OnNCUkii_HoOv0MAolQsT-hibIERY8cA&oe=61C1C20C
  @width 0.2
\fi

\iusr{Irena Visochan}
\textbf{Диана Тараненко} красота...

\iusr{Диана Тараненко}

\ifcmt
  ig https://scontent-frx5-2.xx.fbcdn.net/v/t39.30808-6/267500423_1379535832501595_4092148761539192916_n.jpg?_nc_cat=109&ccb=1-5&_nc_sid=dbeb18&_nc_ohc=U0eYRhKK2J8AX-Bn9u9&_nc_ht=scontent-frx5-2.xx&oh=00_AT-e6GQy0ecOPvuSIZ6j-s-W_uptaUNqYFM7r1Xku-KFQQ&oe=61C1B8CC
  @width 0.2
\fi

\begin{itemize} % {
\iusr{Olga Borodina}
\textbf{Диана Тараненко}. А чашечка и поднос из Греции! не уверена на счет кувшина...

\iusr{Диана Тараненко}
\textbf{Olga Borodina}

\ifcmt
  ig https://scontent-frx5-1.xx.fbcdn.net/v/t39.30808-6/265996537_1379992242455954_4121498991306701210_n.jpg?_nc_cat=110&ccb=1-5&_nc_sid=dbeb18&_nc_ohc=h98cxRQt-4cAX-iaHHy&_nc_ht=scontent-frx5-1.xx&oh=00_AT-W96el6Mk0sPSTCE4FAuV2pvUVw8UnFM84o9p6RcGzxA&oe=61C074EC
  @width 0.2
\fi

\iusr{Диана Тараненко}
Чашечки с подносиком с Греции, а кувшин еще бабушкин. Покупали на ярмарке.

\end{itemize} % }

\iusr{Диана Тараненко}

\ifcmt
  ig https://scontent-frx5-1.xx.fbcdn.net/v/t39.30808-6/265964926_1379536245834887_3273959977200348292_n.jpg?_nc_cat=110&ccb=1-5&_nc_sid=dbeb18&_nc_ohc=xMlHUaeHOdAAX-GsU7R&_nc_oc=AQkI7W8NXrMMUkwFaLf9fw-7FI6Tupr6V1bEQZT8fcS3tTE8iug9lgYawINYMbs-Spw&_nc_ht=scontent-frx5-1.xx&oh=00_AT-puR6vQgsNgVVOaRStIacOegewG0YcbYJ-SDlS31dF7A&oe=61C11003
  @width 0.2
\fi

\begin{itemize} % {
\iusr{Irena Visochan}
\textbf{Диана Тараненко} ....и ложечки знакомые, ..
\end{itemize} % }

\iusr{Диана Тараненко}

\ifcmt
  ig https://scontent-frt3-2.xx.fbcdn.net/v/t39.30808-6/267490437_1379536332501545_6556117343089149884_n.jpg?_nc_cat=101&ccb=1-5&_nc_sid=dbeb18&_nc_ohc=LXglWQPWs2IAX-Ueos4&_nc_ht=scontent-frt3-2.xx&oh=00_AT9rAN8MHePAe0hy9e56Z2cPlAXMtpo6WHGhDj6k3NSY_A&oe=61C1BCB7
  @width 0.2
\fi

\begin{itemize} % {
\iusr{Victoria Novikov}
\textbf{Диана Тараненко} Я очень жалею, что мы ничего из этого не взяли с собой, когда уезжали.

\iusr{Диана Тараненко}
\textbf{Victoria Novikov} я берегу эти вещи и дорожу ими....
Я это просто люблю.

\iusr{Кулішова Ольга}
\textbf{Диана Тараненко} Магазин назывался \enquote{Культтовары}. Очень любили
в него заходить после уроков. Училась в школе номер 32.
\end{itemize} % }

\iusr{Tatyana Smirnova}
Спасибо за уют истории и акварели, просто кожей ощутила тепло и благость.

\iusr{Victoria Novikov}
\textbf{Tatyana Smirnova}  @igg{fbicon.heart.eyes} 

\iusr{Анна Сидоренко}

И у нас была такая лампа, стаканы имели свойство разбиваться поэтому чаще
пользовались эмалированными кружками, ну, а рецепты это святое. Спасибо вам.

\begin{itemize} % {
\iusr{Victoria Novikov}
\textbf{Анна Сидоренко}  @igg{fbicon.heart.eyes} 

\iusr{Анна Сидоренко}
Забыла написать, мы с мамой очень любили пить чай в прикуску именно с кусковым сахаром.

\iusr{Victoria Novikov}
\textbf{Анна Сидоренко} Я никогда не пила сладкий кофе или чай. Только с «чем-то» сладким. Все мы родом из детства!

\iusr{Анна Сидоренко}
Это точно.

\iusr{Irena Visochan}
\textbf{Анна Сидоренко} . ..а были специальные личики я любила раскатывать кусочки.,
\end{itemize} % }

\iusr{Antonina Chepiga}
Тепло...

\iusr{Ирэн Маркевич}
Огромное спасибо

\ifcmt
  ig https://i2.paste.pics/af76b0a70c4960bb8bca4964e6dea4fc.png
  @width 0.2
\fi

\iusr{Victoria Novikov}
\textbf{Ирэн Маркевич}  @igg{fbicon.heart.eyes} 

\iusr{Evgen Zachepylenko}
Душевная история замечательная акварель на акварельной бумаге.

\iusr{Victoria Novikov}
\textbf{Evgen Zachepylenko}  @igg{fbicon.heart.eyes} 

\iusr{Ирина Иванченко}
Чудесно просто!

\iusr{Victoria Novikov}
\textbf{Ирина Иванченко}  @igg{fbicon.heart.eyes} 

\iusr{Татьяна Сидорук}
Как мило!

\ifcmt
  ig https://i2.paste.pics/af76b0a70c4960bb8bca4964e6dea4fc.png
  @width 0.2
\fi

\iusr{Петр Кузьменко}

Прекрасно! Истинно киевские воспоминания... @igg{fbicon.hands.applause.yellow} 

\iusr{Victoria Novikov}
\textbf{Петр Кузьменко}  @igg{fbicon.heart.eyes} 

\iusr{Irena Visochan}

И у нас была такая же лампа, подстаканники, солонка, и чуть побольше серебряная
сахарница. У многих киевлян была одинакова посуда.


\iusr{Елена Мельникова}

\ifcmt
  ig https://i2.paste.pics/7c791234521f6dd932200359e09b033d.png
  @width 0.2
\fi

\iusr{Inna Sarkisova}

У нас тоже была такая лампа. Хранили мы ее в подвале нашего дома. Когда выключали
электричество, а это было большой редкостью, кто-то из взрослых спускался наощупь
вниз и приносил лампу. Мы ее зажигали. Она была всегда заправлена керосином. Часто
мы не успевали даже насладиться ее светом, теплом и специфическим запахом. Свет
включали быстро. Но процедура была очень радостная, все
бегали, суетились, сбегались из других комнат за круглый стол с керосиновой
лампой, рассказывали всякие истории. И, конечно, очень знаком и любим стакан с
подстаканником и ложкой. И сахарница.

Спасибо за ностальгический рассказ с натюрмортом.

\iusr{Victoria Novikov}
\textbf{Inna Sarkisova}  @igg{fbicon.heart.eyes} 

\iusr{Татьяна Бригинец}
Замечательный текст! Прекрасная акварель. Интересно, кто автор?

\iusr{Victoria Novikov}
\textbf{Tetiana Brihinets} Я указала.
Художник ЛюбовьТитова

\iusr{Татьяна Гняздо}
И лампа была, и подстаканник, и солонка такая точь-в-точь!

\iusr{Александр Фурманов}
Подстаканник и солонка сохранились до сих пор.

\iusr{Татьяна Степанова}
Душевно и очень теплые воспоминания @igg{fbicon.person.gesturing.ok} 

\iusr{Татьяна Николаева}
Как же тепло Вы написали все. Как в детство нырнула. Только без дедушки. И без мамы (

\iusr{Татьяна Ховрич}

Viktoria, а не могли бы Вы напомнить рецепт кисло-сладкого Вашей бабушки? Если
это - не семейная тайна, конечно. Заранее благодарю!

\begin{itemize} % {
\iusr{Victoria Novikov}
\textbf{Tatyana Hovrych} 

К сожалению я совершенно не умею готовить. И не знаю ни одного
рецепта приличного. Мой предел - это помазать курицу майонезом
и поставить в духовку.  @igg{fbicon.face.grinning.smiling.eyes} 

\begin{itemize} % {

\iusr{Татьяна Ховрич}
\textbf{Victoria Novikov}, спасибо!

\iusr{Елена Сидоренко}
\textbf{Victoria Novikov} браво! Уважаю! @igg{fbicon.heart.beating} Если умеешь и любишь готовить - половину жизни проходит на кухне @igg{fbicon.cry} 
\end{itemize} % }

\iusr{Natasha Levitskaya}
\textbf{Татьяна Ховрич}

Моя бабушка готовила кисло-сладкое и называется это эсик флэйш. Я готовлю по
памяти, как бабушка, рецептов не осталось. Говядина должна быть с жирком -
грудинка . Мясо обычно с луком слегка поджаривается, лук должен стать
прозрачным, но не зажареным, потом тушится с водичкой и специми - соль, перец,
как для жаркого до готовности. Когда мясо уже мягкое, добавляете мякоть черного
хлеба, томатную пасту и вишневое варенье. Хлеб, чтобы появилась немного
густота. Всё добавляете по вкусу - я никаких пропорций никогда не придерживаюсь
- только все по вкусу. Оно должно быть кисло - сладкое и красивого цвета, а
мясо, как написала Виктория, можно есть губами

\begin{itemize} % {
\iusr{Елена Сидоренко}
\textbf{Natasha Levitskaya} моя бабушка добавляла раскрошенный медовый пряник и томатную пасту. @igg{fbicon.heart.beating} 

\iusr{Татьяна Ховрич}
\textbf{Natasha Levitskaya} , огромное спасибо! Сохранила. На выходных,
Обязательно, приготовлю. Ела в гостях тридцать лет назад. Запомнилось!!!!

\iusr{Natasha Levitskaya}
\textbf{Елена Сидоренко}
Леночка, медовый пряник нало было пойти купить, а хлеб и варенье всегда под рукой. И варенье вишнёвое придавало особый вкус.

\iusr{Natasha Levitskaya}
\textbf{Татьяна Ховрич}
Варенье, в смысле сироп, без вишен. @igg{fbicon.grin} 

\iusr{Елена Сидоренко}
\textbf{Natasha Levitskaya} и ещё ржаной хлеб, сладковатый, тоже очень хорошо! @igg{fbicon.thumb.up.yellow} 

\iusr{Татьяна Ховрич}
Спасибо большое за рекомендации-подсказки, хозяюшки!
\end{itemize} % }

\end{itemize} % }

\iusr{Юрий Панчук}

В Одессе есть памятник Вере Холодной, а в Киеве институт ботаники имени Николая
Холодного, ее родственника.

\begin{itemize} % {
\iusr{Nina Tkachenko}
\textbf{Yury Panchuk} 

Вона навіть і в Київ приїжджала! Тому що вийшла заміж за Володимира Холодного
(брата Миколи Холодного) Віра Левченко познайомилася з юнаком — Володимиром
Холодним. Він був правником за фахом й захоплювався автоспортом, брав участь в
автоперегонах, навіть видавав газету «Авто». Згодом, вони покохали одне одного,
а вже незабаром одружилися, незважаючи на те, що члени обох сімей були проти
раннього шлюбу.

\iusr{Victoria Novikov}
\textbf{Yury Panchuk} Как интересно!
\end{itemize} % }

\iusr{Юрий Панчук}

Мои дедушка с бабушкой тоже пили чай максимально горячим. Если чай не горячий,
это не чай, а помои, говорила бабушка ))). Красивый натюрморт.


\iusr{Dubizhansky Ludmila}

Спасибо за рассказ и за эту прекрасную, живую и нежную акварель!

\iusr{Victoria Novikov}
\textbf{Dubizhansky Ludmila}  @igg{fbicon.heart.eyes} 

\iusr{Людмила Новікова}

Гречно дякую, за талановито написане оповідання. Воно, ще й просвітницьке,
заперечує тезу, що гаряче - шкідливе. Дідусь, очевидно, прожив довго, попиваючи
гарячий напій


\iusr{Victoria Novikov}
\textbf{Людмила Новікова} Кстати, да! Дедушка прожил очень долго @igg{fbicon.face.tears.of.joy} 

\iusr{Татьяна Зубко Маркина}

\ifcmt
  ig https://scontent-frx5-2.xx.fbcdn.net/v/t39.1997-6/s168x128/93118771_222645645734606_1705715084438798336_n.png?_nc_cat=1&ccb=1-5&_nc_sid=ac3552&_nc_ohc=r5H2JVtNyrYAX-kHZyn&tn=lCYVFeHcTIAFcAzi&_nc_ht=scontent-frx5-2.xx&oh=00_AT_LszzKKJTx3XbfHFVxsQd1Wk87UMH0AXSPpS4cuVfsMg&oe=61C0D13F
  @width 0.1
\fi

\iusr{Tatyana Smirnova}

А мой папа говаривал, что чай должен быть крепким и горячим, как женский
поцелуй

\iusr{Victoria Novikov}
\textbf{Tatyana Smirnova}  @igg{fbicon.face.grinning.smiling.eyes}  @igg{fbicon.heart.eyes} 

\iusr{Людмила Мозговая}
И у бабушки была такая лампа...

\iusr{Евгения Бочковская}

У нас была такая лампа... У меня сейчас только верхняя часть, без фитиля. Используем для свечи...
И для церковной свечи, чтобы не гасла, в Пасхальные Дни.

\iusr{Maryna Bilous}
От Вашего рассказа повеяло теплом!

\begin{itemize} % {
\iusr{Victoria Novikov}
\textbf{Maryna Bilous} Благодарю!

\iusr{Maryna Bilous}
\textbf{Victoria Novikov} это я Вам благодарна за воспоминания. Много общих моментов в жизненном укладе наших бабушек/ дедушек.
\end{itemize} % }


\iusr{Polina Feldman}

я помню когда у нас небыло зимой света бабушка включала керасиновую да так жили
мы и все было прекрастно и весело @igg{fbicon.face.smiling.hearts} 

\iusr{Наталия Петрушевская}
Здорово!

\iusr{Irina Kaminsky}

Все детали акварели с такой любовью! Текст замечательный, а последняя фраза —
супер. Напомнила мне фразу из поста о Басилашвили: «Оглянись, посмотри. Этого
никогда больше не будет».

\iusr{Victoria Novikov}
\textbf{Irina Kaminsky} Спасибо !  @igg{fbicon.heart.eyes} 

\iusr{Галина Гордеева}
Так приятно, окунуться в теплые воспоминания детства,

\iusr{Tanya Orlov}
У моих родителей была. Такая солонка.

\iusr{Natasha Levitskaya}

Спасибо, Виктория! Так душевно, такие теплые воспоминания!

Чудесный рассказ.

Так окунули в детство... Вот дедушкина серебряная стопочка, она всегда стояла
на столе - он из нее пил всегда. И водичкой запивал таблетки, и водочку, и пару
глоточков вина - всегда из неё. И дедушки нет уже 37 лет, а этот стаканчик я
храню . И самое вкусное кисло-сладкое готовила моя бабушка. И вроде я делаю
всё так же, но это не совсем, как вкус бабушкиного в моем детстве.


\iusr{Victoria Novikov}
\textbf{Natasha Levitskaya}  @igg{fbicon.heart.eyes} 

\iusr{Наталья Писная}

И я вспомнила сразу свою бабушку и ее комнатку и старинный буфет.. тогда в
детстве нам казалось, что там всегда хранилист вкусности и подарочки для нас,
ее внуков...


\iusr{Victoria Novikov}
\textbf{Наталья Писная}  @igg{fbicon.heart.eyes} 

\iusr{Оксана Яновская}
Очень душевно, спасибо, светлые воспоминания

\iusr{Мила Димитриева}

\ifcmt
  ig https://scontent-frx5-2.xx.fbcdn.net/v/t39.1997-6/s168x128/93118771_222645645734606_1705715084438798336_n.png?_nc_cat=1&ccb=1-5&_nc_sid=ac3552&_nc_ohc=r5H2JVtNyrYAX-kHZyn&tn=lCYVFeHcTIAFcAzi&_nc_ht=scontent-frx5-2.xx&oh=00_AT_LszzKKJTx3XbfHFVxsQd1Wk87UMH0AXSPpS4cuVfsMg&oe=61C0D13F
  @width 0.1
\fi

\iusr{Ольга Глиняная}
Керосинку не застала. Я несколько моложе. Солонка такая у меня есть до сих пор. И подстаканник помню

\iusr{Светлана Александренко}
Замечательный рассказ! Спасибо за настроение, за теплоту!

\iusr{Victoria Novikov}
\textbf{Светлана Александренко}  @igg{fbicon.heart.eyes} 

\iusr{Изольда Арбенина}

У меня такая солонка и керосиновая лампа- очень красивая, в полной исправности,
в запасе фитили- осталось все это от бабушки!

\begin{itemize} % {
\iusr{Наталья Костюченко}
\textbf{Изольда Арбенина} и у нас точно такая же солонка с ложечкой)

\iusr{Ljudmila Luist}
\textbf{Изольда Арбенина} Такую солонку я купила на знаменитом блошином рынке в
Х-ки. Увлекло, что предмет из бывшей страны проживания.
\end{itemize} % }

\iusr{Алим Федоринский}
Лампа скоро пригодится, запасайтесь керосином.

\begin{itemize} % {
\iusr{Victoria Novikov}
\textbf{Алим Федоринский} оптимист? @igg{fbicon.face.grinning.smiling.eyes} 

\iusr{Алим Федоринский}
\textbf{Victoria Novikov} Реалист


\iusr{Victoria Novikov}
\textbf{Алим Федоринский}  @igg{fbicon.face.grinning.smiling.eyes} 
\end{itemize} % }

\iusr{Татьяна Данилова}

\ifcmt
  ig https://scontent-frx5-2.xx.fbcdn.net/v/t39.1997-6/s480x480/47472862_1846683478763869_5271603212267290624_n.png?_nc_cat=1&ccb=1-5&_nc_sid=0572db&_nc_ohc=CbmrU4AU_XkAX_8leoP&_nc_ht=scontent-frx5-2.xx&oh=00_AT-jbYeC_qyRSVPPaqj3f9NaX7A9OTLkD4s2ZaGy9596IQ&oe=61C07827
  @width 0.2
\fi

\iusr{Леся Лагуна}
Спасибо большое за тёплые воспоминания.

\iusr{Victoria Novikov}
\textbf{Елена Лагуна}  @igg{fbicon.heart.eyes} 

\iusr{Rimma Yeliseykina}

\ifcmt
  ig https://scontent-frt3-2.xx.fbcdn.net/v/t39.30808-6/266956251_10219339598751113_5416823408653154620_n.jpg?_nc_cat=103&ccb=1-5&_nc_sid=dbeb18&_nc_ohc=SJaD8rBKJTkAX_HWFnz&_nc_ht=scontent-frt3-2.xx&oh=00_AT8jQpZgnr4cQGEwNRbqsu7J8-rHTgqED1-gt2J-dFjiyQ&oe=61C230C4
  @width 0.2
\fi


\iusr{Світлана Куликова}

\ifcmt
  ig https://scontent-frx5-2.xx.fbcdn.net/v/t39.1997-6/s168x128/47270791_937342239796388_4222599360510164992_n.png?_nc_cat=1&ccb=1-5&_nc_sid=ac3552&_nc_ohc=hSdFnkRMAt8AX-KHr1I&_nc_ht=scontent-frx5-2.xx&oh=00_AT9Hi990n2fJpOowUyslNnf9EHL_TPwuQMYi7MTmRBAKxQ&oe=61C105F6
  @width 0.1
\fi


\iusr{Ольга Писанко}

А у меня до сих пор живы щипчики для колки сахара. Бабушка пила тоже только
кипяток! Из эмалированной кружки. Как не обжигалась?!

\iusr{Victoria Novikov}
\textbf{Ольга Писанко} Точно! @igg{fbicon.100.percent} 

\iusr{Аркадий Израилевский}

Точно такая солонка с ложечкой стоит у меня в серванте и подстаканник с
памятником Богдану Хмельницкого.

\begin{itemize} % {
\iusr{Dina Bitel}
\textbf{Аркадий Израилевский} и у меня

\iusr{Sasha Jampolsky}
\textbf{Аркадий Израилевский} и у меня такая есть

\iusr{Ирина Саркисова}
\textbf{Sasha Jampolsky} и у меня солонка такая, а подстаканник живет у сестры в Израиле, мой дед тоже не пил чай из чашки
\end{itemize} % }

\iusr{Ирина Ковальская}
У меня такая солонка и лампа, а у моей мамы подстаканник.

\iusr{Олена Медведева- Прицкер}

Понравилась киевская история ! Дом за Костелом - значит рядом Полицейский садик
- чудом сохранившийся уголок детства. Я редко, но обязательно захожу туда,
сажусь на лавочку и вспоминаю бабушку, к которой здесь бывала в пятидесятые
годы.

\begin{itemize} % {
\iusr{Victoria Novikov}
\textbf{Olena Medvedyeva-Pritsker} Полицейский садик сохранился? Как здорово!  @igg{fbicon.thumb.up.yellow} 

\iusr{Valentina Urban}
\textbf{Olena Medvedyeva-Pritsker} я училась в школе \# 32 ( 1961-1971 гг.) и прекрасно помню вышеупомянутые и Костёл, и Полицейский садик, и Райисполком Московского района, и завод П/Я \#24, и ИнЯз- все родное и близкое.

\iusr{Victoria Novikov}

И я.

\iusr{Ирина Хоняк}
Я училась тоже в 32 школе (1952-1963), сначала в старом здании, за Инязом,
напротив была козачья часть, году в 1956 построили новую
\end{itemize} % }

\iusr{Ольга Душакова}

Спасибо! @igg{fbicon.heart.red}. Мне, как киевлянке, очень было интересно
прочесть ваш рассказ-воспоминание!

\begin{itemize} % {
\iusr{Victoria Novikov}
\textbf{Olga Dushakova}  @igg{fbicon.heart.eyes} 

\iusr{Валентина Трефилова}
ой и я, как киевлянка, очень хорошо помню такую лампу, она была у нас дома, а куда потом делась не знаю
\end{itemize} % }

\iusr{Tatyana Makovetskaya}
Какое все родное

\iusr{Victoria Novikov}
\textbf{Tatyana Makovetskaya}  @igg{fbicon.heart.eyes} 

\iusr{Владимир Новицкий}
Спасибо, чудесный рассказ. Это наша молодость!!

\iusr{Victoria Novikov}
\textbf{Vladimir Novitsky}  @igg{fbicon.heart.eyes} 

\iusr{Ирина Клименко}
Солонка стоит на столе и сейчас, переехала с Красноармейской 25 @igg{fbicon.face.smiling.eyes.smiling} 

\iusr{Ирина Клименко}
А вот дедушкин любимый подстаканник куда-то потерялся  @igg{fbicon.cry} 

\iusr{Татьяна Чута}
Очень теплый, приятный рассказ, спасибо @igg{fbicon.rose} 

\iusr{Victoria Novikov}
\textbf{Татьяна Чута}  @igg{fbicon.heart.eyes} 

\iusr{Наталия Платонова}
Чудо ПАМЯТИ и ЛЮБВИ! СПАСИБО!

\iusr{Victoria Novikov}
\textbf{Наталия Платонова}  @igg{fbicon.heart.eyes} 

\iusr{Valentina Alexandrovna Guseva}
очень доброе воспоминание, вызывает чувства любви и нежности к прошлому!

\iusr{Victoria Novikov}
\textbf{Valentina Alexandrovna Guseva}  @igg{fbicon.heart.eyes} 

\iusr{Инна Валентиновна}
Да, очень проникновенно и схоже со многими из нас. Я тоже помню такую лампу и
серебряный подстаканник. Mamory...

\iusr{Viktoria Terpylo}
А теперь эти вещицы называются винтажом в интерьере или даже антикваром ..одним
словом воспоминания

\iusr{Ніна Нагорська}

У мене зберігається така лампа, і більша, з ручкою. З нею ходили у хлів до
худоби взимку

\iusr{Ирина Саркисова}

Очень трогательно, особенно место, где упоминается про магазин канцтоваров
напротив Владимирского рынка, мы туда сбегали с уроков, там был целый мир...



\end{itemize} % }


