% vim: keymap=russian-jcukenwin
%%beginhead 
 
%%file articles.06_08_2020.news.strana_ua.hirosima
%%parent 06_08_2020
 
%%endhead 
\subsection{Шоу Трумэна. Зачем американцы 75 лет назад сбросили атомную бомбу на Хиросиму}
\url{https://strana.ua/news/282721-bombardirovka-khirosimy-hlavnye-fakty-v-hodovshchinu-ataki.html}
  
\vspace{0.5cm}
 {\ifDEBUG\small\LaTeX~section: \verb|articles.06_08_2020.news.strana_ua.hirosima| project: \verb|letopis| rootid: \verb|p_saintrussia| \fi}
\vspace{0.5cm}

Сегодня исполняется 75 лет со дня ядерного взрыва в Хиросиме.

75 лет назад, 6 августа 1945 года, американцы по приказу своего президента
Гарри Трумэна сбросили ядерную бомбу "Малыш" на японский город Хиросима. Он
стал целью, поскольку в 1945 году там находился второй Генштаб Японии и генштаб
Военно-Морских сил. 

А еще там жили сотни тысяч гражданских, четверть которых погибла. 

Мощность взрыва составила примерно 20 тысяч тонн тротила. Для сравнения силу
взрыва в Бейруте оценивают в 50 тонн тротила. Спустя три дня еще более мощная
ядерная бомба "Толстяк" была сброшена на город Нагасаки.

Сегодня во всем мире отмечают день запрета ядерного оружия. В мемориальном
парке мира в Хиросиме проводится церемония памяти. Каждый год 6 августа в 8:15
там проходит минута молчания. Это время, когда произошел взрыв.

Отвечаем на десять вопросов о первой ядерной бомбардировке в истории
человечества --- зачем американцы сбросили бомбу, было ли это необходимо и что
поменялось в мире.

\subsubsection{Почему сбросили бомбу?}

Бомбу на Хиросиму сбросили уже на завершающем этапе Второй мировой войны.

26 июля 1945 года в рамках Потсдамской конференции была обнародована
совместная декларация от имени правительств Великобритании, США и Китая (СССР
присоединился к Потсдамской декларации 8 августа- Ред.). Они потребовали от
Японии безоговорочной капитуляции на условиях, предложенных союзниками. В
случае отказа Японии пригрозили быстрым и полным уничтожением.

В Японии отклонили требования Потсдамской декларации. Примерно через неделю
последовала атомная бомбардировка Хиросимы. 

Сбросив бомбу американцы продемонстрировали всему миру - а особенно СССР - свою
силу, показав, что они сверхдержава с ядерным козырем в кармане. Кроме того,
это оправдало гигантские затраты на Манхэттенский проект по разработке атомной
бомбы. 

Правда, эффект оказался скорее обратным - Советский Союз лишь ускорил свои
работы по атомной бомбе и уже через четыре года после атаки на Хиросиму успешно
испытал ее. 

\subsubsection{Как бомбили?}

Приказ о бомбардировке Хиросимы и Нагасаки был отдан американским президентом
Гарри Трумэном. Их решено было осуществить с 3 августа, как только позволят
погодные условия. В день бомбардировки над Хиросимой было чистое небо.

Бомбу нес бомбардировщик под командованием одного из лучших пилотов США -
30-летнего Пола Тиббетса. Он дал борту имя Enola Gay  - в честь матери. Самолет
вылетел с базы на острове Тиниан Марианские острова).

Японские радары раннего предупреждения зафиксировали приближение трех
американских самолетов в южной части Японии. Но поскольку их было мало, в
японской армии отменили воздушную тревогу. Небольшие группы чужих самолетов не
перехватывали в целях экономии горючего.

"Малыш" летел над городом около 45 секунд, бомба разорвалась на высоте около
600 метров над землей. Ее вес составлял 4,4 тонны, длина - 3 метра , диаметр -
71 см. Бомба вмещала 64 кг урана.

По воспоминаниям Тиббетса, сбросив бомбу, бомбардировщик тут же начал
удаляться, чтобы не попасть во взрывную волну. "В какой-то момент машину словно
огрели огромным хлыстом. Нас стало трясти как грушу, но самолет выдержал,
вторая волна оказалась уже слабее" - рассказывал он.

Бомбу разработали в рамках Манхэттенского проекта. Это программа США по
разработке ядерного оружия, в которую вложили более \$2 млрд. Всего было
изготовлено три бомбы - Штучка, Малыш и Толстяк. Первую взорвали над пустыней в
штате Нью-Мексико, остальные - в Японии.

\subsubsection{Сколько человек погибло?}

Население Хиросимы тогда составляло около 400 тыс. человек. От взрыва погибли
от 70 до 80 тысяч человек. Те, кто был в эпицентре - погибли мгновенно, их тела
сразу же обуглились. Многие погибли под обломками зданий или в огненном смерче
пожаров. Температур была такой, что в двух километра от эпицентр взрыва в домах
загорелась бумага.

Общее число погибших, включая тех, кто был ранен и пострадал от лучевой
болезни, достигло примерно 90-166 тысяч человек. Если же учитывать умерших от
рака и других долгосрочных воздействий взрыва, то количество погибших могло
превысить 200 тысяч человек.

Выживших после атомной бомбардировки Хиросимы, а также их детей называли
"хибакуся" -  переводится с японского как "люди, подвергшиеся воздействию
взрыва". Им полагалась медпомощь и выплаты от правительства. При этом в
обществе отношение к ним было настороженное, нередко с такими людьми не желали
вступать в брак, поскольку был риск рождения больных детей, не принимали на
работу.

\subsubsection{Что было разрушено?}

Бомба взорвалась в 600 метрах над землей. Радиус полного разрушения составил
примерно 1,6 километра. Большую часть города охватили пожары. Почти 90\% зданий
в городе были повреждены либо полностью уничтожены.

При этом некоторые железобетонные выстояли, поскольку создавались с учетом
землетрясений. К примеру остался целым каркас здания Промышленной палаты
Хиросимы, известный как "купол Гэмбаку". Хотя он находился в 160 метрах от
эпицентра взрыва.

Сейчас его руины --- самый известный экспонат атомного взрыва в Хиросиме. Он
вошел в список всемирного наследия ЮНЕСКО.

Купол Гэмбаку

Город отстроили к началу 60-х.

\subsubsection{Какой была радиация после взрыва?}

О радиационном загрязнении тогда никто не подозревал, поэтому эвакуация не
проводилась. А высокую смертность в последующие годы и генетические отклонения
у детей, рожденных после бомбардировки, поначалу со взрывом не связывали.

Точно оценить масштаб радиационного загрязнения сейчас нельзя из-за нехватки
информации. Но технически бомба "Малыш" по современным меркам несла маломощный
ядерный заряд. Она содержала 64 кг урана, но реакция и деление происходили в
700 граммах ядерной начинки. Для сравнения на ЧАЭС в момент аварии находилось
несколько тонн продуктов деления.

\subsubsection{Что произошло после бомбардировки?}

В США выступили с заявлением спустя 16 часов после атаки. Президент Трумэн объявил об успешной операции и потребовал, чтобы Япония признала безоговорочную капитуляцию.

"Мы полностью уничтожим способность Японии вести войну. Именно с целью предотвратить разрушение Японии был выпущен ультиматум от 26 июля в Потсдаме. Их руководство немедленно отвергло его условия. Если они не примут сейчас наши условия, пусть ожидают дождь разрушений с воздуха, подобного которому еще не было на этой планете" - сказал американский президент.

В Японии о том, что что-то произошло в Хиросиме стали догадываться спустя 20 минут, когда выяснилось, что там не отвечает телеграфная сеть, а военная база замолчала. Первое время не было понятно, что произошло, и местные газеты вышли с заголовками, что десятки американских самолетов сбросили зажигательные бомбы на Хиросиму. Тогда еще не могли представить, что это последствия взрыва одной бомбы, но ядерной.

После того, как в Японии осознали, что произошло, в стране стали обсуждать возможность  капитуляции. Император выступал за мирные переговоры, военные советовали потянуть время, чтобы нанести урон противнику и тогда можно будет поторговаться. 

В это время СССР, на посредничество которого в переговорах надеялись в Японии,
объявил стране восходящего солнца войну и начал Манчжурскую операцию. А США 9
августа нанесли второй сокрушительный ядерный удар - по городу Нагасаки. В этот
раз бомба была еще мощнее, но последствия были не такими разрушительными, как в
Хиросиме из-за ряда факторов.

В результате голоса за и против капитуляции разделились поровну и решающее слово было за императором Хирохито.

"Если мы продолжим сражаться, это не только приведет к коллапсу и уничтожению японской нации, но и к полному исчезновению человеческой цивилизации. В такой ситуации, как мы можем спасти миллионы наших подданных или оправдать себя перед священным духом наших предков?" - сказал он.

Япония капитулировала 15 августа 1945 года, приняв условия Потсдамской
декларации с одной оговоркой о неприкосновенности императора.

\subsubsection{Что произошло после объявления капитуляции}

15 августа капитулировала японская армия в Китае, численностью 1,2 млн человек.
16 августа - Квантунская армия в Маньчжурии, численностью 700 тысяч человек
(накануне она была полностью разбита советскими войсками). К ноябрю 1945 года
сдались и другие части японской армии. 

С 1945 по 1952 год Япония была оккупирована. Туда прибыли и разместились на
базах около 40 тысяч американских военных. Она не обладала государственным
суверенитетом, правительство и император подчинялись Верховному Командующему
Союзными войсками. Была принята новая Конституция. Император отрекся от своего
божественного происхождения, женщины получили избирательные права, синтоистскую
религию отделили от государства.

Оккупация завершилась после вступления в силу Сан-Франциского мирного договора.
Американские военные базы функционируют на территории Японии о сих пор.

\subsubsection{Бомба была спасением или военным преступлением?}

После ядерных атак на Хиросиму и Нагасаки ведутся дискуссии о том, были ли
оправданы эти бомбардировки. Главный вопрос, были ли они необходимы для победы
США в Тихом океане.

Сторонники считают, что бомба на Хиросиму предотвратила дальнейшие
многотысячные потери, ведь Япония планировала продолжать боевые действия, и
поставила окончательную точку во Второй мировой войне.

Противники говорят, что в американских атаках не было необходимости: вступление
в войну СССР поставило бы точку, и японцы все равно бы сдались.

Кроме того, они задаются вопросом, почему нельзя было сбросить бомбу не на
многотысячный город, где жило мирное население. Ядерную атаку они называют
аморальным военным преступлением и проявлением государственного терроризма. 

\subsubsection{Жалеют ли в США?}

Командир бомбардировщика, который сбросил бомбу, Пол Тиббетс позже рассказывал,
что не испытывает сожалений.

"Я горд, что был способен, начав с ничего, распланировать операцию и привести
ее в исполнение так безукоризненно, как я сделал… Я сплю спокойно каждую ночь"
- заявил он в оном из интервью.

"Если вы поставите меня в такую же ситуацию, то да, черт побери, я сделаю это
снова" - сказал он за несколько лет до смерти. 

После смерти Тиббетс завещал развеять свой прах над Ла-Маншем и не ставить
мемориал или надгробную плиту, чтобы демонстранты, выступающие против ядерного
оружия, не превратили это в место протестов.

Остальные члены экипажа также считали, что совершили благое дело и спасли США и
мир от войны. Последний из членов экипажа Enola Gay умер в 2014 году. Теодор
Ван Кирк умер в доме престарелых в возрасте 93 лет. Он был штурманом и навел
самолет на цель. 

Внук Гарри Трумэна вспоминал, что дед всю жизнь считал, что сбросить бомбу на
Хиросиму и Нагасаки было правильным, и США никогда не попросят прощения.

При этом не все в США были согласны сбросить бомбу на Хиросиму. Американский
командующий Дуайт Эйзенхауэр был против этого, считая, что атомное оружие
должно оставаться психологическим средством давления.

Действующие американские президенты долгое время не посещали мемориалы, жертвам
атомных бомбардировок. Первым стал Барак Обама. Правда, он не принес извинений
потомкам пострадавших. В свою очередь премьер-министр Японии не извинился за
Перл-Харбор.
