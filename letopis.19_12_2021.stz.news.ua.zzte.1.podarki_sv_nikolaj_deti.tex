% vim: keymap=russian-jcukenwin
%%beginhead 
 
%%file 19_12_2021.stz.news.ua.zzte.1.podarki_sv_nikolaj_deti
%%parent 19_12_2021
 
%%url https://zz.te.ua/volontery-buchachahrokhlibpromu-do-dnia-sviatoho-mykolaia-podaruvaly-20-tysiach-sviatkovykh-podarunkiv-foto
 
%%author_id 
%%date 
 
%%tags 
%%title Волонтери "Бучачагрохлібпрому" до Дня Святого Миколая подарували 20 тисяч святкових подарунків. ФОТО
 
%%endhead 
\subsection{Волонтери \enquote{Бучачагрохлібпрому} до Дня Святого Миколая подарували 20 тисяч святкових подарунків. ФОТО}
\label{sec:19_12_2021.stz.news.ua.zzte.1.podarki_sv_nikolaj_deti}

\Purl{https://zz.te.ua/volontery-buchachahrokhlibpromu-do-dnia-sviatoho-mykolaia-podaruvaly-20-tysiach-sviatkovykh-podarunkiv-foto}

Одне із найпотужніше підприємств Тернопіля ТзОВ «Бучачагрохлібпром» до Дня
Святого Миколая роздало майже 20 тисяч святкових подарунків.

\ii{19_12_2021.stz.news.ua.zzte.1.podarki_sv_nikolaj_deti.pic.1}

Як повідомили на підприємстві святкові пакунки отримали:

1048 – працівникам підприємства;

12462 подарунків власникам земельних ділянок (орендодавцям);

5519 – Школи та дошкільні навчальні заклади;

266 – одинокі та діти сироти в тому числі які перебувають в спеціальних центрах
в с. Бариш, дитячих будинках м. Тернопіль та Бережанах;

300 – Релігійним громадам;

150 – Бучацькому пласту.

Загалом – 19745 солодких подарунків.

\ii{19_12_2021.stz.news.ua.zzte.1.podarki_sv_nikolaj_deti.pic.2}

Зокрема, у Трибухівській школі відбулося театралізоване дійство зі Святими
Миколаєм. З щедрими подарунками завітав Святий Миколай до учнів школи від
Голови спостережної ради товариства «Бучачагрохлібпром», Героя України Петра
Іваровича Гадза та заступника голови спостережної ради Лілії Прокіпчук. Усі
школярі (427), окрім добротного подарунка з солодощами отримали і смачні
яблука.

\ii{19_12_2021.stz.news.ua.zzte.1.podarki_sv_nikolaj_deti.pic.3}

«День Святого Миколая Чудотворця – це світле і радісне свято, яке дарує нам
віру в здійснення мрій, вселяє в душу почуття надії, любові і відчуття дива,
надихає на добрі справи, – йдеться у повідомленні школи. – У Трибухівській ЗОШ
І-ІІІ ст. 17 грудня відбулося свято \enquote{Іде Святий Миколай}, яке підготували учні
3-А та 3-Б класів. Ось і настала мить, коли Св. Миколай завітав до нашого
навчального закладу. Чудотворець був зі своїми помічниками-ангелами. Усі учні
початкових класів отримали подаруночки, а у відповідь діти підготували для
Миколая вірші та пісні. Адміністрація школи, вчительський та учнівський
колективи висловлюють подяку за щедрі подарунки голові Трибухівської ОТГ
Ковдрину О. І., начальнику відділу освіти, культури, молоді та спорту
Трибухівської сільської ради Пошелюжній І. А., депутатам та керівному складу
Трибухівської ОТГ. Нехай Господь Бог обдарує Вас своїми ласками та щедротами,
пошле Вам та Вашим родинам здоров’я, радість, достаток і любов!»

\ii{19_12_2021.stz.news.ua.zzte.1.podarki_sv_nikolaj_deti.pic.4}

У свою чергу, діти по всій Трибухівській громаді отримали продурунки.
Традиційно, вже третій рік поспіль, 16-17 грудня, напередодні цього великого
свята, до вихованців дитячих садочків та учнів загальноосвітніх шкіл
Трибухівської гпромади з солодкими подарунками від голови наглядової ради
Бучачагрохлібпром – Петра Івановича Гадза та Трибухівської сільської ради,
завітав Святий Миколай зі своїми ангелятами, адже саме діти для нього –
найрадісніші і найсвятіші створіння, які завжди потребують захисту, уваги та
лагідного ставлення.

\ii{19_12_2021.stz.news.ua.zzte.1.podarki_sv_nikolaj_deti.pic.5}

Діти на якусь мить поринули у чарівний казковий світ. Позитивні емоції та
гарний настрій отримали всі присутні у закладах освіти.

Смаколики отримали всі вихованці та школярі нашої територіальної громади.

Керівництво та колектив ТзОВ «Бучачагрохлібпром» вітає всіх з цим світлим
святом. Бажає радості і щедрих подарунків. Бажає знайти під подушкою смачну
шоколадку і добре побажання, розділити веселощі та радість з коханими і
близькими людьми, наповнити свій будинок святковим затишком і променистим
щастям!

\ii{19_12_2021.stz.news.ua.zzte.1.podarki_sv_nikolaj_deti.pic.6}
\ii{19_12_2021.stz.news.ua.zzte.1.podarki_sv_nikolaj_deti.pic.7}
\ii{19_12_2021.stz.news.ua.zzte.1.podarki_sv_nikolaj_deti.pic.8}
\ii{19_12_2021.stz.news.ua.zzte.1.podarki_sv_nikolaj_deti.pic.9}
\ii{19_12_2021.stz.news.ua.zzte.1.podarki_sv_nikolaj_deti.pic.10}
%\ii{19_12_2021.stz.news.ua.zzte.1.podarki_sv_nikolaj_deti.pic.11}
