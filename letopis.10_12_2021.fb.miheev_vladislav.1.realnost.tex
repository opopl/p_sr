% vim: keymap=russian-jcukenwin
%%beginhead 
 
%%file 10_12_2021.fb.miheev_vladislav.1.realnost
%%parent 10_12_2021
 
%%url https://www.facebook.com/vladislav.mikheev.5/posts/4724926080907463
 
%%author_id miheev_vladislav
%%date 
 
%%tags informacia,obschestvo,realnost'
%%title Мы живём в непроверяемой реальности
 
%%endhead 
 
\subsection{Мы живём в непроверяемой реальности}
\label{sec:10_12_2021.fb.miheev_vladislav.1.realnost}
 
\Purl{https://www.facebook.com/vladislav.mikheev.5/posts/4724926080907463}
\ifcmt
 author_begin
   author_id miheev_vladislav
 author_end
\fi

Мы живём в непроверяемой реальности. Например, мы думаем,  что социологи меряют
общественное мнение. На самом деле они меряют формирующее воздействие
коллективного мифотворца - телевизора и интернета. Возможно, пора усомниться в
существовании у представителей современного общества какого-либо мнения в
принципе. 

Любое коллективное знание - это на самом деле вопрос не знания, а веры. Так
повелось издревле, телевизор и интернет стали лишь усилителями сигнала.
Мнение, высказанное коллективным рассказчиком по поводу трёх китов, на которых
покоится плоская земля, неоспоримо. До тех пор, пока коллективный рассказчик не
предпочтет другую систему аргументации.

Нечто похожее происходит и с образом врага. Абсолютное большинство людей
считает врагами тех, с кем они физически ни разу не контактировали и в реальной
жизни никогда не встречались. Именно поэтому знаменитые телемосты между СССР и
США были беспрецедентной попыткой двух коллективных рассказчиков
синхронизировать дискурсы.  Сейчас  технологических возможностей в сфере
информации намного больше, но ни одной попытки построить \enquote{мост} вместо
\enquote{забора} не наблюдается. 

Цифровая эпоха добавила к страху перед неопределенным будущим возможность
коллективного рассказчика безнаказанно и бездоказательно сказать про любого
любую гадость. Или превратить в сакральную жертву профессионального подонка.
Или обосновать необоснованное ожидание.

Социологическая реальность  принципиально непроверяема.

И, пожалуй, наиболее адекватным ситуации может быть только \enquote{движение
неприсоединения} - к любому мнению, которое считается общественным.

\ii{10_12_2021.fb.miheev_vladislav.1.realnost.cmt}
