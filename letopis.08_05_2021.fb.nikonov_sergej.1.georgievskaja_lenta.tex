% vim: keymap=russian-jcukenwin
%%beginhead 
 
%%file 08_05_2021.fb.nikonov_sergej.1.georgievskaja_lenta
%%parent 08_05_2021
 
%%url https://www.facebook.com/alexelsevier/posts/1488939261451384
 
%%author 
%%author_id 
%%author_url 
 
%%tags 
%%title 
 
%%endhead 
\subsection{Не сила права, а право силы}
\Purl{https://www.facebook.com/alexelsevier/posts/1488939261451384}


\ifcmt
  pic https://external-frt3-2.xx.fbcdn.net/safe_image.php?d=AQGlrKpQzOsLeRVK&w=500&h=261&url=https%3A%2F%2Fsharij.net%2Fwp-content%2Fuploads%2F2021%2F05%2Flenta.jpg&cfs=1&ext=jpg&ccb=3-5&_nc_hash=AQFesi_hQONE8lE0
\fi


Не сила права, а право силы. Не сила ума, а власть организованных и вооруженных
карьеристов или  безумцев, нападающих на умных и порядочных людей, помнящих
свое прошлое. Они решили сделать напоминалку о запрете ношения георгиевских
лент.  Спасибо, что нам напоминают о том, что следует действовать грамотно и
свято уважать ордена, медали и символы великих предков подальше от разной
шелухи, часть из которой в погонах. Но они же просто великолепным способом
дефакто запретили наш украинский флаг!!! Ведь, что он теперь часто
символизирует? Правильно, тоталитарный режим. Нет, формально у нас демократия.

Ибо формально нет абсолютной власти тирана. Но у нас диктатура или господство
государства и деловой верхушки, а не людей. Плюрализма мнений нет. Если человек
высказывает мнение по ключевым политическим вопросам, которое  радикально
отличается от официальной позиции, то рискует быть битым, убитым или
посаженным. Его право на достойную жизнь тоже попрано.  Нищета, деградация
образования и науки. Так вот проклятый тоталитарный СССР уважал все это куда
получше. Уровень жизни был выше и стабильность, и образование и
здравоохранение. В общем, молодцы полицейские, Лучше б шли в Верховный Совет за
отменой этого закона, и на банковскую б зайти не помешало. 

А да конечно, законы
конкретизируют символику и флаг не соответствует по признакам. Но сути
соответствует. И нацизм развивается. В общем в перечень забыли добавить.  Как я
смею об этом говорить? Я знаю, что у многих этот флаг искренне ассоциируется с
совершенно иными  вещами. Но так и с флагом СССР. Тут и сталинский террор и
космонавтика, и дефициты и нормальная зарплата, и закрытость и гордость за
страну, и экономический потенциал  и недостаток потребительских благ, и
реальная дружба народов и попытка соединить несовместимые культуры и
мировоззрения. Контрасты, а ваши запреты лишь свидетельство однобокости. 

Кстати они исполнены в духе советской бюрократии.
