% vim: keymap=russian-jcukenwin
%%beginhead 
 
%%file 26_02_2022.fb.fb_group.story_kiev_ua.1.hroniki_vojny
%%parent 26_02_2022
 
%%url https://www.facebook.com/groups/story.kiev.ua/posts/1869610273235765
 
%%author_id fb_group.story_kiev_ua
%%date 
 
%%tags __feb_2022.vtorzhenie
%%title ХРОНИКИ НЕОБЪЯВЛЕННОЙ ВОЙНЫ
 
%%endhead 
 
\subsection{ХРОНИКИ НЕОБЪЯВЛЕННОЙ ВОЙНЫ}
\label{sec:26_02_2022.fb.fb_group.story_kiev_ua.1.hroniki_vojny}
 
\Purl{https://www.facebook.com/groups/story.kiev.ua/posts/1869610273235765}
\ifcmt
 author_begin
   author_id fb_group.story_kiev_ua
 author_end
\fi


ХРОНИКИ НЕОБЪЯВЛЕННОЙ ВОЙНЫ

Или: день третий. Будем жить!

00:00

Вторая ночь в метро.

Поражаюсь терпению наших людей. Жёстко, неуютно, некомфортно. Но все вежливы,
готовы друг другу помочь и поддержать.

00:30

Человеческий муравейник отходит ко сну. Всхлипывают дети, скулят собаки, о
чём-то мяукают коты.

Подгоняют состав. Всё же не на полу. Вдоль платформы проходит сотрудник метро:
"Кому кипяток?". В глазах усталость, на лице - улыбка.

01:00

Большинство спит, вжашись друг в друга телами, взявшись за руки в поисках тепла
и поддержки. У тех, кто не спит, один вопрос: \enquote{Что там наверху?}. Ответ ищут в
новостной ленте и соцсетях.

03:00

А наверху неспокойно. Идут бои за аэродром Василькова, за ТЭС-6 и в районе
Зоопарка.

03:50

Принимаем решение дожидаться рассвета.

PS:

А прошлой ночью, в самый разгар обстрела Киева, в метро родилась девочка.
Назвали Миа.

Будем жить!

(фото\#1 девочки не моё)
