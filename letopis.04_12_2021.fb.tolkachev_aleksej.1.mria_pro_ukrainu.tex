% vim: keymap=russian-jcukenwin
%%beginhead 
 
%%file 04_12_2021.fb.tolkachev_aleksej.1.mria_pro_ukrainu
%%parent 04_12_2021
 
%%url https://www.facebook.com/oleksiy.tolkachov/posts/4881738148512525
 
%%author_id tolkachev_aleksej
%%date 
 
%%tags mechta,ukraina
%%title Мрія про Україну
 
%%endhead 
 
\subsection{Мрія про Україну}
\label{sec:04_12_2021.fb.tolkachev_aleksej.1.mria_pro_ukrainu}
 
\Purl{https://www.facebook.com/oleksiy.tolkachov/posts/4881738148512525}
\ifcmt
 author_begin
   author_id tolkachev_aleksej
 author_end
\fi

\enquote{Ех, бл*дська Данія, п*здєц всім сподіванням} (З п'єси \enquote{Гамлет} Подерв'янського)

13 років тому, 4 грудня в \enquote{Українській Правді} вийшла моя стаття, яка
докорінно змінила подальше життя.

Тоді я вперше озвучив, що нам і всьому суспільству потрібна масшатбна мрія - це
саме та сила, яка може згуртувати, наповнити смислом наше буття в Україні. 

\ifcmt
  ig https://scontent-frx5-1.xx.fbcdn.net/v/t39.30808-6/262685411_4881734308512909_2253814928633886383_n.jpg?_nc_cat=105&ccb=1-5&_nc_sid=730e14&_nc_ohc=wgFp4pu_3McAX_xgcbL&_nc_ht=scontent-frx5-1.xx&oh=28f2906c44c4cff7a219055a9ae3be83&oe=61B1810B
  @width 0.4
  %@wrap \parpic[r]
  @wrap \InsertBoxR{0}
\fi

Пізніше ця та подальші праці стали основою для книги \enquote{Омріяна Україна.
Ключ до майбутнього}.

Мрія і бачення майбутньго здатні створити диво в Україні. Я був вражений, коли
у 2016 році під час мого всеукраїнського туру \enquote{Революція Світогляду}
ідею \enquote{Омріяної України} сприймали \enquote{на ура} 90-95\% людей, з
якими був контакт. Тисячі активістів по Україні гуртувалися...

Було відчуття, що нарешті я знайшов інструмент масової трансформації, відтак,
довгоочікуваний прорив - не за горами. Лише дайте точку опертя, і ми би
перевернули не лише Україну, але й весь світ.  

Однак пройшло 13 років. 

\enquote{Омріяна Україна} розрослася купою проектів, планами будівництва Міста
Майбутнього в м. Орбіті Черкаської області, концепцією Безумовного освноного
доходу, моделлю цифрової самоорганізації, яка зараз втілюється у вигляді
соціальної мережі, концепцією нової школи тощо. Однак, в Україні все рухається
дуже повільно, тому що у правильні й потрібні ідеї не вкладають гроші ті, хто
їх має! 

Зерна не можуть прорости без поживних речовин. Це стало моїм головним
розчаруванням за 13 років, що минули -  в Україні не інвестують у перспективні,
потужні, яскраві ідеї. Пасіонарії та спражні самостійні лідери нікому не
потрібні. Моральні авторитети - не популярні. Гроші знаходяться на усіляку
х@\%ню, одноразові політичні проеки, блискучі фантики, всередині яких порожнеча. 

Ті, хто має змогу і ресурси, нічого не хочуть змінювати у нашому українською
болоті. \enquote{Тільки п*издЯт і п*здят!}, як казав наш Слуга Народу в образі учителя
Голобородька. 

Але я не покладаю надії побачити втілення усіх ідей ще за свого життя. 

Хоча вже й розумію, що для цього доведеться жити дуже довго!

А сьогдні для багатьох ця стаття, як і \enquote{Мрія про Україну} стала ще
актуальнішою:

\ii{link.04_12_2008.stz.news.ua.pravda.tolkachev_aleksej.1.mria_pro_ukrainu}
