% vim: keymap=russian-jcukenwin
%%beginhead 
 
%%file 31_12_2020.news.ru.vesti.1.putin_new_year
%%parent 31_12_2020
 
%%url https://www.vesti.ru/article/2505932
 
%%author 
%%author_id 
%%author_url 
 
%%tags 
%%title Новогоднее обращение: Путин поблагодарил россиян за чувство локтя
 
%%endhead 
 
\subsection{Новогоднее обращение: Путин поблагодарил россиян за чувство локтя}
\label{sec:31_12_2020.news.ru.vesti.1.putin_new_year}
\Purl{https://www.vesti.ru/article/2505932}
\index[names.rus]{Путин, Владимир!Президент России, новогоднее поздравление, 31.12.2020}


\ifcmt
  pic https://cdn-st1.rtr-vesti.ru/vh/pictures/xw/309/000/1.jpg
  width 0.6
\fi

Жители Петропавловска-Камчатского и Анадыря первыми в России отмечают
наступление Нового года. По традиции россиян с праздником поздравил президент
страны Владимир Путин.

Президент отметил в своем поздравлении, что никто год назад не мог представить,
через какие испытания нам придется пройти.

"И сейчас кажется, что уходящий год вместил в себя груз нескольких лет. Он был
трудным для каждого из нас, с тревогами и большими материальными сложностями, с
переживаниями, а для кого-то — с горькими утратами близких, любимых людей", –
отметил Путин.

"Но уходящий год был связан и с надеждами на преодоление невзгод, – уверен
президент, – с гордостью за тех, кто проявил свои лучшие человеческие и
профессиональные качества, с осознанием того, как много значат надежные,
искренние, настоящие отношения между людьми, дружба и доверие между нами".

"Этот год мы прошли вместе, с достоинством, как и подобает единому народу,
который почитает традиции своих предков", – сказал Путин, отметив при этом, что
эти ценности – мужество, отзывчивость и милосердие — "в наших сердцах, в нашем
характере и поступках".

\begin{leftbar}
  \begingroup
    \em\Large\bfseries\color{blue}
Мы равняемся на наших дорогих ветеранов, на доблестное поколение, победившее
заразу нацизма. Несмотря ни на что, мы исполнили наш священный сыновний долг —
с благодарностью и признательностью отметили 75-летие Великой Победы.
  \endgroup
\end{leftbar}

"Да, новый опасный вирус изменил, перевернул привычный образ жизни, работы,
учебы, заставил пересмотреть, скорректировать многие планы. Но так устроен мир,
что в нем неизбежны испытания", – напомнил глава государства.

\begin{leftbar}
  \begingroup
    \em\Large\bfseries\color{blue}
Они побуждают нас более внимательно вглядеться в жизнь, прислушаться к своей
совести, отбросить мелкое, суетное и по-настоящему оценить самое важное. А это
дар человеческой жизни, это семья, наши мамы и папы, дедушки и бабушки, это
наши дети — еще малыши и уже повзрослевшие, это наши друзья и коллеги,
бескорыстная помощь и общая энергия добрых дел — масштабных, на всю страну, и
небольших, в пределах района, улицы, дома, но от этого не менее значимых.
  \endgroup
\end{leftbar}

"Испытания и беды обязательно проходят. Так было всегда. А остается с нами
главное, все, что делает нас благородными и сильными: любовь, взаимопонимание,
доверие и поддержка. Поэтому хочу пожелать, чтобы трудности уходящего года
поскорее канули в Лету. А все, что мы обрели, все лучшее, что раскрылось в
каждом человеке, обязательно осталось с нами навсегда. Сегодня очень важно
верить в себя, не отступать перед трудностями, беречь нашу сплоченность, это
основа наших общих успехов в будущем", – говорится в обращении президента.

\begin{leftbar}
  \begingroup
    \em\Large\bfseries\color{blue}
Убежден, вместе мы все преодолеем, наладим и восстановим нормальную жизнь и с
новой энергией продолжим решать задачи, стоящие перед Россией в наступающем,
третьем десятилетии ХХI века.
  \endgroup
\end{leftbar}

Владимир Путин напомнил, что не все сейчас находятся за новогодним столом: "Еще
много людей в больницах, и, уверен, все они чувствуют, как их поддерживают
родные и друзья. От всей души желаю вам, дорогие мои, одолеть болезнь и
вернуться домой как можно быстрее".

\begin{leftbar}
  \begingroup
    \em\Large\bfseries\color{blue}
К сожалению, эпидемию еще не удалось полностью остановить. Борьба с ней не
прекращается ни на минуту. Врачи и медицинские сестры, экипажи скорой помощи
продолжают мужественно трудиться. Многие из них в эту праздничную ночь на
дежурстве.
  \endgroup
\end{leftbar}

"Так же неустанно, с высокой ответственностью выполняют сложнейшие задачи
сотрудники экстренных служб, наши военнослужащие в горячих точках за рубежами
России, наши миротворцы и боевые расчеты армии и флота, – сказал Путин. –
Благодаря всем, кто несет свое служение днем и ночью, при любых обстоятельствах
находится на посту, граждане России могут спокойно собраться сегодня у родных
очагов со своими близкими; с надеждой на лучшее и с планами на будущее
встретить Новый год, загадать сокровенные желания".

"Дорогие друзья! Давайте в эти мгновения помечтаем о самом светлом, о мире и
благополучии, о счастье и радости для всех, кто рядом, кто нам дорог, для всей
нашей страны. Хочу сказать спасибо каждому из вас, потому что мы вместе. А
когда мы чувствуем надежный локоть людей, стоящих рядом, Россия становится
одной большой семьей. Я искренне всем вам желаю крепкого здоровья, веры,
надежды и любви, как близким и дорогим людям. Счастья вам в новом, наступающем,
2021 году! С праздником, дорогие друзья!", – заключил свое новогоднее обращение
президент России.


