% vim: keymap=russian-jcukenwin
%%beginhead 
 
%%file 17_09_2019.stz.news.ua.mrpl_city.1.danylo_chufarov
%%parent 17_09_2019
 
%%url https://mrpl.city/blogs/view/mariupolets-danilo-chufarov-stante-gidnimi-zhitelyami-chudovogo-mista
 
%%author_id demidko_olga.mariupol,news.ua.mrpl_city
%%date 
 
%%tags 
%%title Маріуполець Данило Чуфаров: "Станьте гідними жителями чудового міста!"
 
%%endhead 
 
\subsection{Маріуполець Данило Чуфаров: \enquote{Станьте гідними жителями чудового міста!}}
\label{sec:17_09_2019.stz.news.ua.mrpl_city.1.danylo_chufarov}
 
\Purl{https://mrpl.city/blogs/view/mariupolets-danilo-chufarov-stante-gidnimi-zhitelyami-chudovogo-mista}
\ifcmt
 author_begin
   author_id demidko_olga.mariupol,news.ua.mrpl_city
 author_end
\fi

Наступний нарис присвячено відомому маріупольському спортсмену зі світовим
ім'ям, провідному громадському діячу та просто небайдужому маріупольцю \textbf{Данилу
Володимировичу Чуфарову}. Через 7 років після Паралімпіади він повернувся до
Лондону, де сьогодні проходить чемпіонат світу з плавання. Всі маріупольці
вболівають за свого унікального співмешканця і бажають йому лише перемог та
позитивних емоцій. А доки Данило змагається, пропоную познайомитися з ним
ближче.

\ii{17_09_2019.stz.news.ua.mrpl_city.1.danylo_chufarov.pic.1}

Наш герой завжди був націлений на успіх, і завдяки підтримці рідних, постійній
роботі над собою та прагненню до самовдосконалення він отримував лише високі
результати як в навчанні, змаганнях, так і в роботі. Його енергійна діяльність
надихає, а наполеглива праця вражає багатьох українців.

У басейн хлопця привело нещастя. У восьмирічному віці він став втрачати зір, і
лікарі рекомендували батькам відвести дитину на плавання – м'язи підтягнути,
поліпшити кровообіг головного мозку і, як наслідок, підтримати загальний стан у
непосидючого хлопчика. Незважаючи на те, що хлопець спочатку не прагнув стати
плавцем, але природжена вдача далася взнаки, адже хлопець звик боротися за
перемогу. Змалечку Чуфаров намагався все встигати: і тренування, і заняття в
школі. До речі, маріуполець закінчив школу із золотою медаллю.

\ii{17_09_2019.stz.news.ua.mrpl_city.1.danylo_chufarov.pic.2}

У ДЮСШ № 5, яку тоді очолював Євген Червяков, чотири роки Данило тренувався у
\textbf{Надії Новоселецької}. У школі йому доводилося багато тренуватися. З 6 ранку -
тренування, потім шкільні заняття і знову басейн. Коли група розпалася,
перейшов до відомого тренера \textbf{Віктора Олександровича Вангельєва}, заслуженого
тренера України, заслуженого діяча фізичної культури України. Програми
маститого маріупольського фахівця спортсмен виконує неухильно, аж до останніх
стартів на Паралімпійських іграх і чемпіонатах світу та Європи.

\textbf{Читайте також:} \emph{\enquote{Мариуполь 240+1. Перезагрузка}: горожан приглашают на масштабный концерт ко Дню города}%
\footnote{\enquote{Мариуполь 240+1. Перезагрузка}: горожан приглашают на масштабный концерт ко Дню города, mrpl.city, 16.09.2019, \par%
\url{https://mrpl.city/news/view/mariupol-240-plus1-perezagruzka-gorozhan-priglashayut-na-masshtabnyj-kontsert-ko-dnyu-goroda-1}
}

Слава прийшла до маріупольця вже на змаганнях Паралімпіади в Пекіні в 2008-му,
де він здобув срібну медаль на дистанції 400 метрів вільним стилем і бронзову в
цій же спортивній дисципліні на 100 метрів.

Наступний успіх на Паралімпійських іграх в Лондоні в 2012-му був уже
закономірний. Спортсмен став відомим не тільки україн\hyp{}ським фахівцям, але і
суперникам з усіх країн світу. Результат був високим. У підсумку – \enquote{срібло} у
змаганнях на 400 метрів вільним стилем і \enquote{бронза} в комплексному плаванні на
200 метрів. Спортсмен пробував свої сили на інших дистанціях вільним стилем,
брасом, батерфляєм. Найважчою Паралімпіадою виявилася в Ріо-де-Жанейро в 2016
році. Тренувався Данило наполегливо з потроєною енергією і виграв в Ріо
\enquote{бронзу} на дистанції 200 метрів комплексним плаванням. Змагався і на інших
дистанціях. Медаль дісталася нелегко. Жорсткі тренування підсадили здоров'я. На
тлі навантажень і спеки плавець скинув 10 кг ваги. Але він не здався і, як
наслідок, привіз з Бразилії нову олімпійську медаль.

Данило має дві вищі освіти. Закінчив економіко-правовий факультет
Маріупольського державного університету за спеціальніс\hyp{}тю \enquote{Менеджмент
організацій} (2011 рік) та факультет \enquote{Фізичне виховання} Донбаського державного
педагогічного університету. Спеціальність \enquote{Теорія і методика фізичного
виховання} (2013 рік).

\ii{17_09_2019.stz.news.ua.mrpl_city.1.danylo_chufarov.pic.3}

Щоб перерахувати всі здобутки і нагороди нашого героя - не вистачить однієї
статті. Він, заслужений майстер спорту, є п'яти\hyp{}разовим призером Літніх
Паралімпійських ігор (по дві медалі у 2008 та 2012 роках, одна медаль у
2016-му), багаторазовим чемпіоном і рекордсменом Європи та світу. Став
переможцем таких конкурсів: \enquote{Майбутнє Маріуполя} (2008 р.) та \enquote{Маріуполець
року} (2012 р.), а також став лауреатом обласного конкурсу \enquote{Кришталеве серце} в
номінації \enquote{Життя, як подвиг}. За перемогу в літніх Паралімпійських іграх (2008,
2012 та 2016 рр.), виявлені мужність, самовідданість та волю до перемоги,
утвердження міжнародного авторитету України був нагороджений Орденом \enquote{За
мужність} I, II та III ступенів. У 2008 році йому вручили грамоту Верховної
Ради, в 2008 і 2012 роках – Кабміну України. А у 2017 році міський голова Вадим
Бойченко вручив рекордсмену диплом почесного громадянина Маріуполя.

\textbf{Читайте також:} \emph{\enquote{Почему я выбираю... плавание}, - Даниил Чуфаров}%
\footnote{\enquote{Почему я выбираю... плавание}, - Даниил Чуфаров, Георгий Федоренко, mrpl.city, 17.09.2017, \par%
\url{https://mrpl.city/blogs/view/pochemu-ya-vybirayu-plavanie-daniil-chufarov}
}

Данило любить своє місто, йому комфортно просто знаходитися в Маріуполі! Не
часто у громадського діяча виходить гуляти по місту, але з однаковим
задоволенням він гуляє і центральними вулицями, і в сквері, і в Міському саду.
Особливо любить місця, де відчувається близькість моря!

\ii{17_09_2019.stz.news.ua.mrpl_city.1.danylo_chufarov.pic.4}

Наразі чоловік працює спортсменом-інструктором штатної збірної команди України
з плавання Укрцентру \enquote{Інваспорт}. Деякий час (до січня 2016 року) займав посаду
доцента кафедри \enquote{Фізичне виховання, спорт та здоров'я людини} Маріупольського
державного університету. Сьогодні маріуполець не уявляє свого життя без
громадської діяльності. Він активно проводить пропаганду здорового способу
життя, захисту навколишнього середовища та допомоги людям з вадами здоров'я, а
також є прихильником рішучих змін для всебічного поліпшення рівня життя всіх
жителів нашого міста. У 2013 році став співзасновником Благодійного Фонду
розвитку громади \enquote{Наш дім – Маріуполь}. Загалом завдяки його діяльності багато
юних маріупольців повірили в себе та власні сили. Водночас Данило є автором
ідеї та головним організатором дитячо-юнацького турніру з плавання \emph{\enquote{на кубок
Заслуженого майстра спорту Данила Чуфарова}} у місті Маріуполі. У лютому 2016
року разом з активістами Маріуполя організував на площі Свободи вечір пам'яті
Андрія Кузьменка (Кузьми \enquote{Скрябіна}).

\ii{17_09_2019.stz.news.ua.mrpl_city.1.danylo_chufarov.pic.5}

Як громадський працівник і співзасновник благодійного фонду Чуфаров активно
займається створенням транспорту для людей з обмеженими можливостями. У тому,
що в нових тролейбусах і автобусах муніципального транспорту встановлено
пристрій, що дозволяє почути на вулиці номер маршруту та інші дані, – заслуга
саме Данила. Перед світлофорами встановлені розпізнавальні стрічки, в
підземному переході нижня сходинка пофарбована в жовтий колір – в цьому теж
заслуга Чуфарова.

У маріупольця залишається не так багато вільного часу, але він все ж таки
встигає і зустрітися з друзями, і відпочити разом з коханою дружиною, яка
повністю розділяє інтереси чоловіка. Загалом Данило і його дружина \textbf{Ярина} – є
зразком щасливого і гармонійного українського подружжя. І справа не тільки в
тому, що їх єднають спортивні інтереси. Це дійсно той унікальний випадок, коли
дві сильні особистості створили міцний союз і по-справжньому кохають, поважають
і підтримують один одного.

\ii{17_09_2019.stz.news.ua.mrpl_city.1.danylo_chufarov.pic.6}

З жовтня 2013 року по теперішній час Данило займає посаду Доцента кафедри
\enquote{Фізичне виховання, спорт та здоров'я людини} Маріупольського
державного університету.

\textbf{Читайте також:} \emph{Мариупольский пловец стал призером на чемпионате мира по плаванию в Лондоне}%
\footnote{Мариупольский пловец стал призером на чемпионате мира по плаванию в Лондоне, mrpl.city, 11.09.2019, \par%
\url{https://mrpl.city/news/view/mariupolskij-plovets-stal-prizerom-na-chempionate-mira-po-plavaniyu-v-londone-foto}
}

\textbf{Улюблений фільм:} 

\begin{quote}
\em\enquote{Улюблений фільм виділити не зможу, але виділю акторів – це
Том Хенкс і Едді Мерфі}.
\end{quote}

\textbf{Улюблена книга:} 

\begin{quote}
\em\enquote{\enquote{Собаче серце} М. Булгакова. Також перечитав усю творчість і з
нетерпінням чекаю щось нового від Бориса Акуніна}.
\end{quote}

\textbf{Порада маріупольцям:} 

\begin{quote}
\em\enquote{Любіть себе, любіть своє місто і країну – зробіть своє
місто кращим, а самі станьте його гідними жителями!}.
\end{quote}

Фото з архіву Данила Чуфарова.
