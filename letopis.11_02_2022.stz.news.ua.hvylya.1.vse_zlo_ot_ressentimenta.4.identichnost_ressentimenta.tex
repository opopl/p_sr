% vim: keymap=russian-jcukenwin
%%beginhead 
 
%%file 11_02_2022.stz.news.ua.hvylya.1.vse_zlo_ot_ressentimenta.4.identichnost_ressentimenta
%%parent 11_02_2022.stz.news.ua.hvylya.1.vse_zlo_ot_ressentimenta
 
%%url 
 
%%author_id 
%%date 
 
%%tags 
%%title 
 
%%endhead 

\subsubsection{Идентичность Ресентимента}
\label{sec:11_02_2022.stz.news.ua.hvylya.1.vse_zlo_ot_ressentimenta.4.identichnost_ressentimenta}

Именно благодаря отторжению понимания и многолетнему умолчанию этой темы
политиками, социальными психологами и интеллектуалами вообще Ресентимент проник
не просто на уровень мышления, воли, веры, патии и глобального психического
состояния человечества, но и на уровень идентичности, политики, исторических
нарративов, геополитических представлений, межцивилизационных отношений и даже
в гуманитарные концепции.

Ресентимент сегодня не просто заразен в некоторых отдельных современных средах:
он отражается на другие среды, распространяется на другие коллективы и
сообщества, перенимается от предков к потомкам, а также подхватывается через
идейные доктрины как мыслевирусы и ментальные закладки.

Психическое заражение Ресентиментом, о котором говорит Шелер, происходит на
идентичностном уровне через индоктринацию. Ресентиментальная индоктринация суть
искажение массового сознания под ресентиментальную доктрину. Более того, также
возможно ресентиментально-доктринальное замещение. То есть все
ресентиментальные теории и концепции хорошо замещают, а иногда даже
поддерживают друг друга: теории классового Ресентимента (марксизм, в меньшей
степени неомарксизм), теории националистического Ресентимента, теории
гендерного Ресентимента (в том числе радикальный феминизм и ЛГБТ-ресентимент),
теории расового Реентимента (белый и черный расизмы), теории эйджистского
Ресентимента (ювенальная юстиция), теории компенсаторной справедливости, теория
привилегий и т.д.

Когда Ресентимент добирается до уровня идентичностей, то ресентиментальные
отношения устанавливаются между ресентиментальными идентичностями, то есть они
борются друг с другом и вытесняют друг друга. Ресентиментальная идентичность
всегда оправдывает себя тем, что это не за себя подозрение, зависть,
злопамятность, месть, ненависть и агрессия, это же за оскорбление всего
сообщества. Групповой Ресентимент он всегда с блеском в глазах, с упорством и
кажущимся якобы достоинством, потому что как бы ради добра. Именно поэтому
Ресентимент групповой не эгоистичен, но он всегда является трайбалистским, он —
за своих.

Ресентиментал — не мыслитель и не исследователь. Ресентиментал — не независимый
эксперт. Ресентиментал — пропагандист, контролер, обвинитель, мститель и
абъюзер. Ресентиментал разрушает любую договорную коммуникацию, превращая ее в
манипуляцию, обвиняя в манипуляции других, разрушая ее через дискоммуникацию
(срач) и раскол аудитории.

Мышление ресентиментала обычно непонятно для нересентименталов. Уникальность
ресентиментального мышления в том, что его связности выстраиваются не логически
связанными теориями и концепциями, а единым Ресентиментом. Некоторые
ресентименталы довольно удачно мимикрируют под концептуальное мышление.

Можно говорить на любые темы с националистом, сексистом, расистом, эйджистом
или феминисткой, пока речь не заходит о ресентиментальном характере их
взглядов. Ресентименталы не просто отрицают свой Ресентимент. Ресентименталы не
просто сознательно устанавливают мыслительно-рефлексивные лакуны, то есть темы,
относительно которых рефлексии и мышления избегают, рационализируя это самыми
разными способами. Ресентименталы отрицательно-агрессивно воспринимают саму
постановку вопроса, отторгая разговоры о Ресентименте, подавая его как
справедливость.

Цель ресентиментала — достичь справедливости для своих за счет чужих или, как
это делается в западных университетах, — обратить Ресентимент против себя,
признавая свои привилегии и свою вину. Обратный Ресентимент не стоит путать с
зеркальным. Зеркальный Ресентимент — это когда Ресентимент чужих против своих
отражается в Ресентименте своих против чужих. А обратный Ресентимент — это
принятие чужого Ресентимента против себя, подкрепляя тем самым чужой
Ресентимент (например, в установке «проверь свои привилегии» или в установке
«пробуждения»). Еще один пример концептуализации обратного Ресентимента: теория
«пределы роста» Римского клуба.

Если попадание в состояние Ресентимента произошло не так давно, его можно
отследить через радикальное изменение своего круга общения и появление
враждебного отношения к бывшим друзьям или знакомым. Но когда Ресентимент
достаточно старый, круг общения становится устоявшимся и все рассматривается
уже как обычное состояние.

Более того, старые ресетиментальные состояния получают самооценку важности.
Таким образом, возникает акцентуированный Ресентимент. Лучше всего акцентуацию
создают экстраординарные состояния кризиса, войны, революции и т.п. Самооценка
Ресентиментала — сознательный или пробужденный, окружающие — несознательны.

Акцентуированный ресентиментал мало способен меняться: он знает все, что нужно.
Достучаться до ресентиментала, назвав его ресентименталом, невозможно, то есть
такие рефлексивные определения не становятся маркерами «не так» для
ресентиментала.

Высокий уровень образования и ученая степень кандидата наук или доктора наук не
уберегают от состояния Ресентимента. Даже знание о Ресентименте не предохраняет
от попадания в это состояние. Очевидно, психо-социальные состояния
Ресентимента, особенности их проявления и противодействия нужно изучать в
школе. Да и студентам вместо лекций и семинаров по толерантности, взятию на
себя вины или признанию своих привилегий нужно преподавать практики избегания
широкомасштабного Ресентимента и давать навыки психического суверенитета. Но
государственные или даже финансируемые олигархией университеты будут против
этого.

Ресентимент в сегодняшнем мире становится не просто субкультурой, а глобальной
культурой. Это «культура отмены», культурные установки «проверь свои
привилегии» (check your privilege), пробужденчество (wokeism), новая
нормальность и т.д.

Перспектива расширения глобального Ресентимента: генетический,
трансгуманистический, социальных сетей, космический, климатический, цифровой,
уже проявившийся роботический, и допустимо виртуалистский и пришельский.

Особенно социально опасным является воспитание ресентиментальных предубеждений
молодежи. Они происходят из индоктринированного искажения нормативного
мышления, которое приучается видеть только ресентиментальные факты и не видеть
факты любви, дружбы, справедливости, гармонии и равновесия.

Ресентиментальная компенсация у молодежи суть вырожденная в эмоциональный
интеллект патия, использующая комплекс неполноценности, превращающая его в
комплекс преимущества и превосходства, то есть зависть, ревность и месть с
позиции гипоидентичности переводятся на позицию гиперидентичности. Скажем
установка «проверь свои привилегии» вполне уживается с превосходством над
токсичными взрослыми, а толерантность уживается с «культурой отмены». Причем,
все это на фоне запрета на ненормативное мышление и рефлексию.

Современная молодежь отчасти собирает на себе все возможные Ресентименты:
расовый, гендерный, националистический, эйджистский, феминистский и даже
ЛГБТ-ресентимент. В кризисные времена молодые люди это столь сложные сочетания
Ресентимента, что Ресентимент на Ресентименте сидит и Ресентиментом погоняет.

Для любого зеркального Ресентимента или для его передачи из поколения в
поколение нужны так называемые Генераторы Ресентимента, то есть люди, группы,
институты пропаганды, партии, общественные группы, ресентиментальные политики
памяти (политики злопамятности) и даже стратегии долгосрочного чествования
героев и реагирования на унижения и оскорбления, включая холокосты и геноциды.
Генераторы тотального Ресентимента — это глобальные СМИ и социальные сети под
влиянием государств и олигархов.

Генераторами политического Ресентимента являются партии и общественные группы.
Их организации построены на жестком трайбализме «свой-чужой», а их дискурс
построен через оппозицию «мы и они».

В последнее время возникли глобальные генераторы Ресентимента, например
Глобальный страх коронавирусной эпидемии, порождающий инфодемию и продуцирующий
массовые Ресентименты (масочники и антимасочники, ваксеры и антиваксеры).

В ресентиментальном обществе Ресентимент хорошо монетизируется. Поэтому многие
люди делают генерацию Ресентимента своей профессией, когда осуществляют
продвижение и даже продажу простых смыслов и перспектив в обмен на лояльность к
Ресентименту. Собственно поэтому Ресентимент в обществе невозможно искоренить,
пока существуют генераторы Ресентимента.
