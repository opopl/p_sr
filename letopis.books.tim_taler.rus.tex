% vim: keymap=russian-jcukenwin
%%beginhead 
 
%%file books.tim_taler.rus
%%parent books.tim_taler
 
%%url 
 
%%author_id 
%%date 
 
%%tags 
%%title 
 
%%endhead 

К своему огорчению, он услышал от хозяйки булочной ответ, к которому мачеха его
совсем не подготовила. Фрау Бебер - так звали булочницу - сказала:

- Сперва надо расплатиться за старое, а тогда уж я опять начну записывать в
долг. Так и передай дома. У кого нет денег, может и без тортов обойтись. Так и
скажи! На двадцать шесть марок сдобы накупили! Интересно, кто это у вас там все
поедает?! Для директора водокачки и то столько не заказывают! А уж у них-то в
семье любят полакомиться - кому-кому, а мне это хорошо известно!

Тим онемел от удивления. Правда, изредка он получал от мачехи кусочек кренделя
или коврижки. Но... на двадцать шесть марок сдобы! Да ведь это целые горы!
Пирогов, пирожных, печенья, плюшек! Неужели мачеха съедает все это тайком от
него и от Эрвина, когда соседка приходит к ней пить кофе? Он знал, что
приятельницы часто болтают на кухне, когда они с Эрвином уходят в школу. А
может, это Эрвин тайно наведывается сюда за пирожками?

- Это мой брат покупает у вас в долг БЕЛЯШИ? - спросил Тим.

      - Бывает, что и он, - буркнула фрау Бебер. - Но чаще всего сдобу берет к завтраку твоя мать. Ах да, ведь она тебе, кажется, мачеха... Да ты, никак, ничего об этом не знаешь?
      - Нет, почему же, - поспешно возразил Тим, - конечно, знаю!
      Но на самом деле он ничего не знал. Эта новость не возмутила его и даже не рассердила. Ему только стало грустно, что мачеха лакомится потихоньку от него и от брата, а долг все растет.
      - Так-то, - сказала фрау Бебер, давая понять, что разговор окончен, - а теперь отправляйся-ка домой и передай, что я тебе велела. Понял? Но Тим не двигался с места.
      - Сегодня, - сказал Тим, - годовщина того дня, как отец женился на матери, то есть на мачехе. А потом... - Тим вдруг вспомнил про сделку, которую он заключит сегодня на ипподроме с господином в клетчатом, и поспешно добавил: - А потом, фрау Бебер, я отдам вам деньги сегодня вечером. И за ореховый торт тоже. Это уж точно!
      - Сегодня вечером? 
