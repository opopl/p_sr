% vim: keymap=russian-jcukenwin
%%beginhead 
 
%%file 24_08_2021.fb.mahin_aleksandr.1.tablica_nezalezhnist.cmt
%%parent 24_08_2021.fb.mahin_aleksandr.1.tablica_nezalezhnist
 
%%url 
 
%%author_id 
%%date 
 
%%tags 
%%title 
 
%%endhead 
\subsubsection{Коментарі}

\begin{itemize}
%%%fbauth
%%%fbauth_name
\iusr{Stas Cherkassky}
%%%fbauth_url
%%%fbauth_place
%%%fbauth_id
%%%fbauth_front
%%%fbauth_desc
%%%fbauth_www
%%%fbauth_pic
%%%fbauth_pic portrait
%%%fbauth_pic background
%%%fbauth_pic other
%%%fbauth_tags
%%%fbauth_pubs
%%%endfbauth

А откуда числа? Табличка в Вики совсем другие числа показывает. Например, в
Америке на 2019 год за \$60к, а вовсе не 20. Или это какие-то другие единицы?

\ifcmt
  ig https://scontent-frx5-2.xx.fbcdn.net/v/t39.30808-6/239770892_10159838217002847_8836127132476138474_n.jpg?_nc_cat=109&_nc_rgb565=1&ccb=1-5&_nc_sid=dbeb18&_nc_ohc=YMeLIAi1TJQAX_MFDDp&_nc_ht=scontent-frx5-2.xx&oh=e66034eed54c5a849ad04a3af059a43e&oe=6138BE07
  width 0.4
\fi

\begin{itemize}
%%%fbauth
%%%fbauth_name
\iusr{Александр Махин}
%%%fbauth_url
%%%fbauth_place
%%%fbauth_id
%%%fbauth_front
%%%fbauth_desc
%%%fbauth_www
%%%fbauth_pic
%%%fbauth_pic portrait
%%%fbauth_pic background
%%%fbauth_pic other
%%%fbauth_tags
%%%fbauth_pubs
%%%endfbauth
 
\textbf{Stas Cherkassky} у тебя на душу населения
\end{itemize}

%%%fbauth
%%%fbauth_name
\iusr{Stas Cherkassky}
%%%fbauth_url
%%%fbauth_place
%%%fbauth_id
%%%fbauth_front
%%%fbauth_desc
%%%fbauth_www
%%%fbauth_pic
%%%fbauth_pic portrait
%%%fbauth_pic background
%%%fbauth_pic other
%%%fbauth_tags
%%%fbauth_pubs
%%%endfbauth
 

А, понял, это не на душу населения а глобальный ВВП. Но тогда зачем смотреть на
ppp? Обычно на этот паритет смотрят, когда хотят оценить уровень жизни
населения, потому что они покупают товары и услуги на местном рынке. А если
смотреть на целую страну, на размер экономики, то логичнее сравнивать просто
доллар с долларом: все покупают ту же нефть, те же самолёты, телефоны,
автомобили и тд.

Например, по ррр Россия оказывается "несправедливо" богаче, так как она продаёт
нефть и газ за доллары на внешнем рынке, а своему населению (в среднем) платит
дешевеющими рублями.

\begin{itemize}
%%%fbauth
%%%fbauth_name
\iusr{Александр Махин}
%%%fbauth_url
%%%fbauth_place
%%%fbauth_id
%%%fbauth_front
%%%fbauth_desc
%%%fbauth_www
%%%fbauth_pic
%%%fbauth_pic portrait
%%%fbauth_pic background
%%%fbauth_pic other
%%%fbauth_tags
%%%fbauth_pubs
%%%endfbauth
 
\textbf{Stas Cherkassky} 

Во-первых, я почти не сравнивал страны со странами, а сравнивал динамику роста,
а тут не сильно важно какой показатель брать.

Во-вторых, у стран может очень сильно измениться структура импорта-экспорта.
Например, Украина в начале пути экспортировала в основном машиностроение, а
сейчас продукцию первого, второго переделов: древесину, руду, продукты питания,
максимум металлопрокат. Хотелось от этого избавиться. Та же фигня с Россией, с
одной стороны она получает около трети доходов от продажи углеводородов, а с
другой сейчас большая часть получаемой от экспорта выручки помещается в
"кубышку" (зарубежные казначейские облигации, валюту, драгметаллы), и её лучше
исключить - внутри страны она не производит никакого серьезного экономического
эффекта, а играет роль стабилизатора.

Есть еще одно важное отличие: в расчетах этого показателя используется критерий
национального владения. То есть продукция завода, которым владеют, скажем,
немцы, на территории России будет включаться в Российский ВВП, но не в ВНД.
Пристальный взгляд на Центральную Европу, например, показывает, что в последние
10-20 лет ВВП рос гораздо быстрее, чем ВНД - производства работают, но не
принадлежат "стране".

Подводя итоги в выигрыше будут страны с отрицательным торговым сальдо, типа
Турции, США, Латвии или Греции, а страны типа России, Норвегии, Германии или
Китая - в проигрыше, но дело такое. Всё равно сравнить было интересно.

%%%fbauth
%%%fbauth_name
\iusr{Stas Cherkassky}
%%%fbauth_url
%%%fbauth_place
%%%fbauth_id
%%%fbauth_front
%%%fbauth_desc
%%%fbauth_www
%%%fbauth_pic
%%%fbauth_pic portrait
%%%fbauth_pic background
%%%fbauth_pic other
%%%fbauth_tags
%%%fbauth_pubs
%%%endfbauth
 
\textbf{Alexandr Makhin} 

спасибо, интересные соображения! (кстати, что такое ВНД? Это на душу или ещё что-то?)

Мне кажется, разница в этих показателях всё же, есть: некоторые выводы о
динамике роста не совсем верны для номинального ВВП. Например, Польша - всё ещё
порядка 1/5 от Франции, Индия - далеко не обогнала Германию и Японию, Россия -
не в первой десятке и тд.

Ещё поблема с РРР (я уже упоминал): если в какой-то стране, вдруг обвалилась
местная валюта, то её РРР может вырасти без роста производства вообще - просто
потому, что местные товары услуги (та их часть, что меньше зависит от импорта)
- подешевеют в долларах.

%%%fbauth
%%%fbauth_name
\iusr{Александр Махин}
%%%fbauth_url
%%%fbauth_place
%%%fbauth_id
%%%fbauth_front
%%%fbauth_desc
%%%fbauth_www
%%%fbauth_pic
%%%fbauth_pic portrait
%%%fbauth_pic background
%%%fbauth_pic other
%%%fbauth_tags
%%%fbauth_pubs
%%%endfbauth
 
\textbf{Stas Cherkassky} ВНД - Валовой национальный доход. The gross national income (GNI), previously known as gross national product (GNP).

%%%fbauth
%%%fbauth_name
\iusr{Роман Белкин}
%%%fbauth_url
%%%fbauth_place
%%%fbauth_id
%%%fbauth_front
%%%fbauth_desc
%%%fbauth_www
%%%fbauth_pic
%%%fbauth_pic portrait
%%%fbauth_pic background
%%%fbauth_pic other
%%%fbauth_tags
%%%fbauth_pubs
%%%endfbauth
 
\textbf{Alexandr Makhin} А почему страны с отрицательным сально будут в выигрыше?

%%%fbauth
%%%fbauth_name
\iusr{Stas Cherkassky}
%%%fbauth_url
%%%fbauth_place
%%%fbauth_id
%%%fbauth_front
%%%fbauth_desc
%%%fbauth_www
%%%fbauth_pic
%%%fbauth_pic portrait
%%%fbauth_pic background
%%%fbauth_pic other
%%%fbauth_tags
%%%fbauth_pubs
%%%endfbauth
 
\textbf{Alexandr Makhin} понятно (насчет ВНД). Не знаю, кстати, как расценивать это отличие (завод производит внутри, но принадлежит зарубежным владельцам). Слобныу вопрос, наверное, стоит смотреть на оба показателя.

\end{itemize}

%%%fbauth
%%%fbauth_name
\iusr{Irina Kaystra}
%%%fbauth_url
%%%fbauth_place
%%%fbauth_id
%%%fbauth_front
%%%fbauth_desc
%%%fbauth_www
%%%fbauth_pic
%%%fbauth_pic portrait
%%%fbauth_pic background
%%%fbauth_pic other
%%%fbauth_tags
%%%fbauth_pubs
%%%endfbauth
 

\obeycr
Саша, сравнивать цифры с цифрами бессмысленно!
Нужно брать сектора, регионы, время!!!
Это всего лишь статистика!
Узкая!
Прости, что ты хотел сравнить?
В Польше лучше, чем в Украине?!?
Тебе - да!
Мне нет!
Россия прогрессирует!?!?
Для моего брата - да!
В экономике? Не знаю.
В политике - для меня - нет!
\restorecr

\begin{itemize}
%%%fbauth
%%%fbauth_name
\iusr{Александр Махин}
%%%fbauth_url
%%%fbauth_place
%%%fbauth_id
%%%fbauth_front
%%%fbauth_desc
%%%fbauth_www
%%%fbauth_pic
%%%fbauth_pic portrait
%%%fbauth_pic background
%%%fbauth_pic other
%%%fbauth_tags
%%%fbauth_pubs
%%%endfbauth
 
\textbf{Irina Kaystra}

Взрослые очень любят цифры. Когда рассказываешь им, что у тебя появился новый
друг, они никогда не спросят о самом главном. Никогда они не скажут: «А какой у
него голос? В какие игры он любит играть? Ловит ли он бабочек?» Они спрашивают:
«Сколько ему лет? Сколько у него братьев? Сколько он весит? Сколько
зарабатывает его отец?» И после этого воображают, что узнали человека.

Когда говоришь взрослым: «Я видел красивый дом из розового кирпича, в окнах у
него герань, а на крыше голуби», — они никак не могут представить себе этот
дом. Им надо сказать: «Я видел дом за сто тысяч франков», — и тогда они
восклицают: «Какая красота!»


%%%fbauth
%%%fbauth_name
\iusr{Irina Kaystra}
%%%fbauth_url
%%%fbauth_place
%%%fbauth_id
%%%fbauth_front
%%%fbauth_desc
%%%fbauth_www
%%%fbauth_pic
%%%fbauth_pic portrait
%%%fbauth_pic background
%%%fbauth_pic other
%%%fbauth_tags
%%%fbauth_pubs
%%%endfbauth
 
\textbf{Александр Махин}
Да.
Примерно.
\end{itemize}

%%%fbauth
%%%fbauth_name
\iusr{Роман Белкин}
%%%fbauth_url
%%%fbauth_place
%%%fbauth_id
%%%fbauth_front
%%%fbauth_desc
%%%fbauth_www
%%%fbauth_pic
%%%fbauth_pic portrait
%%%fbauth_pic background
%%%fbauth_pic other
%%%fbauth_tags
%%%fbauth_pubs
%%%endfbauth
 

Почему Чехии нет?)

Чехия по ППП на душу догнала Японии и Италию. Но японскими корпорациями владеют
сами японцы, а чешскими в основном немцы, так что в Японии думаю остается
больше денег. То же самое справедливо и для Польши

ППП хороший показатель, не идеальный, но лучшего у нас нет. ППП не учитывает
основные товары- мобильные телефоны, аренда жилья и качество товаров думаю не
сильно учитывается.

\begin{itemize}
%%%fbauth
%%%fbauth_name
\iusr{Александр Махин}
%%%fbauth_url
%%%fbauth_place
%%%fbauth_id
%%%fbauth_front
%%%fbauth_desc
%%%fbauth_www
%%%fbauth_pic
%%%fbauth_pic portrait
%%%fbauth_pic background
%%%fbauth_pic other
%%%fbauth_tags
%%%fbauth_pubs
%%%endfbauth
 
\textbf{Roman Belkin}
Всё есть, просто я не стал прикреплять огромную простыню и скрыл большую часть строк. Если интересно могу дать ссылку на весь документ. А пока Чехия:
Нет описания фото.

%%%fbauth
%%%fbauth_name
\iusr{Роман Белкин}
%%%fbauth_url
%%%fbauth_place
%%%fbauth_id
%%%fbauth_front
%%%fbauth_desc
%%%fbauth_www
%%%fbauth_pic
%%%fbauth_pic portrait
%%%fbauth_pic background
%%%fbauth_pic other
%%%fbauth_tags
%%%fbauth_pubs
%%%endfbauth
 
\textbf{Alexandr Makhin} интересно, что РФ выросла больше Чехии.
Видимо это эффект низкой базы РФ
\end{itemize}

%%%fbauth
%%%fbauth_name
\iusr{Анна Воронцова}
%%%fbauth_url
%%%fbauth_place
%%%fbauth_id
%%%fbauth_front
%%%fbauth_desc
%%%fbauth_www
%%%fbauth_pic
%%%fbauth_pic portrait
%%%fbauth_pic background
%%%fbauth_pic other
%%%fbauth_tags
%%%fbauth_pubs
%%%endfbauth
 

Внимательно прочитала твой анализ( и участников дискуссии). Рада что такой
титанический труд не для работы.а для души.понимания и анализа. Боюсь спросить
про политиков.аналитиков и руководителей ненькi-они думают что творят и куда
ведет в прошлом великую державу? Ада не боятся?


\end{itemize}

