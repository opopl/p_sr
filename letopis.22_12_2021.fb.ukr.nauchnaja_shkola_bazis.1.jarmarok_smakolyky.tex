% vim: keymap=russian-jcukenwin
%%beginhead 
 
%%file 22_12_2021.fb.ukr.nauchnaja_shkola_bazis.1.jarmarok_smakolyky
%%parent 22_12_2021
 
%%url https://www.facebook.com/sciencebasis/posts/1239588796536221
 
%%author_id ukr.nauchnaja_shkola_bazis
%%date 
 
%%tags jarmarka,novyj_god,prazdnik,shkola,ukraina
%%title Передноворічний ярмарок смаколиків від учнів 5-6 класів!
 
%%endhead 
 
\subsection{Передноворічний ярмарок смаколиків від учнів 5-6 класів!}
\label{sec:22_12_2021.fb.ukr.nauchnaja_shkola_bazis.1.jarmarok_smakolyky}
 
\Purl{https://www.facebook.com/sciencebasis/posts/1239588796536221}
\ifcmt
 author_begin
   author_id ukr.nauchnaja_shkola_bazis
 author_end
\fi

Передноворічний ярмарок смаколиків від учнів 5-6 класів!

Здається у нас з’явилася нова традиція. Сьогодні шкільна їдальня «Базису» на
кілька перерв перетворилася на святковий ярмарок! Учні 5-6 класів з гордістю
запрошували усіх спробувати та придбати солодощі, більшість з яких вони
напередодні старанно готували з батьками, а також напої, новорічні декорації:

- Скуштуйте безалкогольний гарячий глінтвейн! Він зроблений з виноградного
соку, який мені бабуся з села передала!

- Ці іграшки з ялинки ми з мамою вчора розмальовували!

Використовували й маркетингові підходи, розраховані на азарт покупців:

- Купуйте рогалики по 10 гривень! Серед рогаликів у цій коробці є 4 з начинкою.
Якщо ви витягнете один з них – 5 гривень повертається вам на мобільний рахунок.

- Вигідна пропозиція! Одне шоколадне печиво – 10 гривень, а два – 15!

Зібрані під час ярмарку кошти, класи, які брали участь у ярмарку, витратить на
спільні передноворічні розваги: похід в кіно чи на каток, відвідування музею, а
можливо просто на піцу.
