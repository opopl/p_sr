% vim: keymap=russian-jcukenwin
%%beginhead 
 
%%file 20_07_2020.fb.lnr.2
%%parent 20_07_2020
 
%%endhead 
\subsection{Группа украинцев набрала в Чернобыле 200 кг радиоактивной черники на продажу}
\label{sec:20_07_2020.fb.lnr.2}
\url{https://www.facebook.com/groups/LNRGUMO/permalink/2861135340664688/}
  
\vspace{0.5cm}
{\small\LaTeX section: \verb|20_07_2020.fb.lnr.2| project: \verb|letopis| rootid: \verb|p_saintrussia|}
\vspace{0.5cm}
  
19.07.20. Сообщение от издания «Русская весна».

«Группа украинцев набрала в Чернобыле 200 кг радиоактивной черники на продажу.
Трое жителей Ровенской области насобирали в Чернобыльской зоне отчуждения около
200 килограммов радиоактивной черники на продажу.

Об этом сообщают представители Государственной пограничной службы Украины:
«Радиационный фон собранных ягод, которые злоумышленники намеревались продать,
почти вдвое превышал допустимую норму.  Вчера вечером пограничный наряд отдела
„Млачевка” совместно с сотрудниками Нацполиции в зоне ЧАЭС обнаружили
нарушителей радиационного режима», --- указано в сообщении.  Так, трое жителей
Ровенской области пробрались на территорию зоны отчуждения и собрали около 200
килограммов черники, которую позднее собирались продать.  Завышенные значения
радиации собранных ягод «бизнесменов» не смутили.

Отмечается, что при допустимом радиационном фоне до $0,3 \mu S\nu/h$ фон ягод
показывал $0,5 \mu S\nu/h$.

Полиция составила протоколы о нарушении требований радиационной безопасности на
загрязненных местах, а также о нарушении требований режима радиационной
безопасности, однако опасность, которой дельцы могли подвергнуть своих
клиентов, правоохранители квалифицировать не смогли.

Поэтому нарушителям грозит штраф от 25 до 50 тысяч гривен или ограничение
свободы от трех до семи лет.» См.ВИДЕО:
