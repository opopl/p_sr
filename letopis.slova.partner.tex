% vim: keymap=russian-jcukenwin
%%beginhead 
 
%%file slova.partner
%%parent slova
 
%%url 
 
%%author_id 
%%date 
 
%%tags 
%%title 
 
%%endhead 
\chapter{Партнер}

%%%cit
%%%cit_head
%%%cit_pic
%%%cit_text
Про американо-украинское \emph{"партнёрство"}.
Говорить о \emph{партнёрстве} между США и Украиной, конечно, смешно и нелепо.
Речь может идти только о беспрекословном подчинении Украины интересам США в
ответ на их поддержку в противостоянии с Россией. А это не \emph{партнёрство},
а , скорее какой то симбиоз, когда маленькой рыбке большая акула позволяет
найти убежище в себя под брюхом в обмен на множество мелких унизительных
обязанностей. Но тем не менее..  США, по словам Кулебы, предложили Украине
срочно, буквально в течении 2- 3 дней, подписать новую Хартию о
\emph{стратегическом партнёрстве}. Срочность обосновывается американской
стороной настоятельной необходимостью учесть" некоторые новые вызовы
современности".  Что это за новые вызовы? Тут и к бабке не ходи! Это подписание
29 интеграционных соглашений между Путиным и Лукашенко, которые, по сути,
означают аншлюс Беларуси. С военной точки зрения это означает, что Батька
согласился предоставить свою территорию для дислокации российских ВС. А это уже
существенно меняет систему безопасности Европе. Практически все уже готово для
начала войны за Украину, о неизбежности которой я написал сразу же после того,
как Украина попала в зависимость от США
%%%cit_comment
%%%cit_title
\citTitle{Отношения США и Украины партнерством никак не назовешь / Лента соцсетей / Страна}, 
Андрей Головачев, strana.news, 09.11.2021
%%%endcit
