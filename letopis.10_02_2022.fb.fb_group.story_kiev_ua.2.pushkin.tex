% vim: keymap=russian-jcukenwin
%%beginhead 
 
%%file 10_02_2022.fb.fb_group.story_kiev_ua.2.pushkin
%%parent 10_02_2022
 
%%url https://www.facebook.com/groups/story.kiev.ua/posts/1858523957677730
 
%%author_id fb_group.story_kiev_ua
%%date 
 
%%tags kiev,pushkin_aleksandr
%%title Ему был 21 год и он приехал в Киев. Звали его Александр Пушкин
 
%%endhead 
 
\subsection{Ему был 21 год и он приехал в Киев. Звали его Александр Пушкин}
\label{sec:10_02_2022.fb.fb_group.story_kiev_ua.2.pushkin}
 
\Purl{https://www.facebook.com/groups/story.kiev.ua/posts/1858523957677730}
\ifcmt
 author_begin
   author_id fb_group.story_kiev_ua
 author_end
\fi

Ему был 21 год и он приехал в Киев. Звали его Александр Пушкин, его отправили
из Петербурга в Екатеринослав к генералу Инзову, по сути это была высылка из
столицы, где в столь юные года он уже успел «наводнить Россию возмутительными
стихами» - так выразился император Александр I. Пушкин тогда пробыл в Киеве
всего один день и Киев ему очень понравился.

\ifcmt
  ig https://scontent-mxp1-1.xx.fbcdn.net/v/t39.30808-6/273514698_10158951211172198_1574442845696188910_n.jpg?_nc_cat=103&ccb=1-5&_nc_sid=5cd70e&_nc_ohc=BXKk-AAlCdwAX_5pRi7&_nc_ht=scontent-mxp1-1.xx&oh=00_AT_k3QIBBjZMjPyP7WiEoh2BsnsbWDFyoliT6cTAoFjtvQ&oe=620C4197
  @wrap center
  @width 0.8
\fi

Ровно через год в январе 1821 года он вернулся и пробыл в Киеве долго. Приехал
он сюда из Каменки с компанией друзей, в Киеве как раз шла ежегодная
Контрактовая ярмарка и время они проводили весело: играли вместе с поэтом
Денисом Давыдовым в карты в Контрактовом доме, гуляли по Печерску, были балы в
доме генерала Раевского. Пушкин ухаживал за дочкой Раевского Машей, ей тогда
шёл 17-й год, она была юна, очаровательна и не знала, какая тяжелая судьба жены
декабриста Волконского у неё впереди. Пушкин побывал в Лавре, увидел могилы
Искры и Кочубея, много позже все это он вспомнит, когда будет писать свою
«Полтаву», которую он посвятит Марии Раевской.

Но особенно Пушкину тогда понравились Киевские горы - он побывал на Щекавице,
где по преданию был похоронен князь Олег, и в Киеве уже начал писать первые
строки «Песни о вещем Олеге». И его очень заинтересовали истории про киевских
ведьм и шабаши- он побывал на нескольких Лысых горах, и все жалел, что ни одной
ведьмы так и не увидел @igg{fbicon.grin}. И он написал стихотворение «Гусар», которое редко кто
читал. А ведь прочитав это стихотворение, можно просто самостоятельно проводить
экскурсию «Мистический Киев» @igg{fbicon.grin}  и какая прелесть строки:

\begin{zznagolos}
\obeycr
«То ль дело Киев! Что за край!
Валятся сами в рот галушки,
Вином — хоть пару поддавай,
А молодицы-молодушки!» ! 
\restorecr
\end{zznagolos}

Все, больше в Киеве Пушкин ни разу не был. А жаль, хотя и лет впереди у него
оставалось не так-то и много. И меня всегда поражает, сколько он успел за свою
столь недолгую жизнь! 

А потом было 8 февраля 1837 года, дуэль и 10 февраля, после которого уже не
было ничего...
