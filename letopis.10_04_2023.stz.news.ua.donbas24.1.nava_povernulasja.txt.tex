% vim: keymap=russian-jcukenwin
%%beginhead 
 
%%file 10_04_2023.stz.news.ua.donbas24.1.nava_povernulasja.txt
%%parent 10_04_2023.stz.news.ua.donbas24.1.nava_povernulasja
 
%%url 
 
%%author_id 
%%date 
 
%%tags 
%%title 
 
%%endhead 

Еліна Прокопчук (Маріуполь)
Маріуполь,Україна,Мариуполь,Украина,Mariupol,Ukraine,Полон,Азовсталь,Валерія Суботіна,Андрій Суботін,date.10_04_2023
10_04_2023.elina_prokopchuk.donbas24.nava_povernulasja

Nava повернулася — з полону звільнили захисницю Маріуполя Валерію Суботіну
(ФОТО)

Військослужбовиця полку «Азов» втратила чоловіка на «Азовсталі» та майже рік
провела у неволі

Відбувся черговий обмін полоненими — додому повернули 100 українців. Серед них
— маріупольчанка, військовослужбовиця полку «Азов», поетеса Валерія Суботіна
(Карпиленко), відома під позивним Nava.

Після початку повномасштабного вторгнення військова захищала Маріуполь, а
навесні разом з побратимами вийшла з «Азовсталі» та весь цей час перебувала у
російському полоні. Валерія відома своєю творчістю та трагічною історією
кохання — на «Азовсталі» вона одружилася з чоловіком, а через три дні стала
вдовою.

Читайте також: «Заборонив втрачати себе»: воїн ЗСУ родом з Донецька про те, як
вдалося пережити 321 день полону (ФОТО)

Нині військова, жінка раніше викладала у Маріупольському державному
університеті, є кандидаткою наук із соціальних комунікацій та поетесою. До
повномасштабної війни Валерія встигла видати власну збірку поезій «Квіти та
зброя».

Зараз її віршами захоплюються тисячі українців — навіть під час перебування у
«Азовсталі» поетеса писала нові проникливі рядки. А ще 5 травня, в день
народження полку «Азов» Валерія та її коханий, прикордонник Андрій Суботін з
позивним Борода, побралися. Пара зробила кільця з фольги.

Читайте також: Вбили маму та вивезли в Донецьк: що довелося пережити двом
маленьким сестрам з Маріуполя

Усього через три дні жінка стала вдовою — Андрій Суботін героїчно загинув,
захищаючи Україну.

Ще на «Азовсталі» Валерія пообіцяла вибратися та жити за двох. Нарешті вона у
вільній Україні.

Нагадаємо, фільм про Маріуполь переміг на кінофестивалі у США.

Ще більше новин та найактуальніша інформація про Донецьку та Луганську області
в нашому телеграм-каналі Донбас24

ФОТО: з відкритих джерел
