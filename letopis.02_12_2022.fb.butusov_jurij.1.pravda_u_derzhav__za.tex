%%beginhead 
 
%%file 02_12_2022.fb.butusov_jurij.1.pravda_u_derzhav__za
%%parent 02_12_2022
 
%%url https://www.facebook.com/butusov.yuriy/posts/pfbid07fnzRp9XN5kiGfmwp3iE8FuAtpHrjc7xz6wjH1A34x2KJipcEMkcGdfKLqMf1Hvul
 
%%author_id butusov_jurij
%%date 02_12_2022
 
%%tags 
%%title Правда у державі завжди на часі
 
%%endhead 

\subsection{Правда у державі завжди на часі}
\label{sec:02_12_2022.fb.butusov_jurij.1.pravda_u_derzhav__za}

\Purl{https://www.facebook.com/butusov.yuriy/posts/pfbid07fnzRp9XN5kiGfmwp3iE8FuAtpHrjc7xz6wjH1A34x2KJipcEMkcGdfKLqMf1Hvul}
\ifcmt
 author_begin
   author_id butusov_jurij
 author_end
\fi

Правда у державі завжди на часі. 

Правда у державі це не чиєсь особисте - вона належить усім.

Правда під час війни розхитує тільки слабких та брехливих. 

Правда робить сильними тих, хто прагне змін.

Зміни під час війни потрібні як ніколи, і зміни починаються саме з  визначення
того, що відбувається з нами. 

Бо загальне визнання правильних рішень починається з загального визнання
правильних слів.  

В розвинених країнах у державах та сіспільствах працюють сталі установи,
експертні центри,  вони можуть корегувати свої дії через систему стримувань та
противаг, як зараз у США. Правду змушені обговорювати всередині влади через
сильну політичну конкуренцію. Так, правда у кожного своя, але факти, які
формують нашу реальність, це об'єктивна річ. Тому наявність дискусії щодо
фактів та правди, яка на них будується, дозволяє виробляти найбільш адекватне
рішення.  

У нас всього цього нема. Тому у нас неможливо усвідомити та запровадити зміни
мовчки, бо у нашій державі нема самостійних інституцій, і все працює виключно
на мотивації та зусиллях громадянського суспільства. Публічність в Україні -
єдиний каталізатор змін.

Якби у нас рахувалось скільки людей ми втрачаємо через брехню...

Якби у нас рахувалось скільки можливостей втрачалось та втрачається, тому що
замість змін пусті слова...

Казати, що правда не на часі, значить позбавляти себе майбутнього.

