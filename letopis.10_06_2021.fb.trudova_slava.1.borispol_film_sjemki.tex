% vim: keymap=russian-jcukenwin
%%beginhead 
 
%%file 10_06_2021.fb.trudova_slava.1.borispol_film_sjemki
%%parent 10_06_2021
 
%%url https://www.facebook.com/tsmediacenter/posts/1768483416667593
 
%%author_id trudova_slava
%%date 
 
%%tags borispol,donbass,film,ukraina,vojna
%%title «Небесний координатор»: під Борисполем знімають фільм про волонтера-розвідника «Чубаку»
 
%%endhead 
 
\subsection{«Небесний координатор»: під Борисполем знімають фільм про волонтера-розвідника «Чубаку»}
\label{sec:10_06_2021.fb.trudova_slava.1.borispol_film_sjemki}
 
\Purl{https://www.facebook.com/tsmediacenter/posts/1768483416667593}
\ifcmt
 author_begin
   author_id trudova_slava
 author_end
\fi

«Небесний координатор»: під Борисполем знімають фільм про волонтера-розвідника
«Чубаку»

Під Борисполем знімають документальний фільм про видатного волонтера-розвідника
Володимира Кочеткова-Сукача, а точніше – епізоди до стрічки «Небесний
координатор». Під час одного з них події відбуваються на блок-посту, над яким
майорять прапори бойовиків. Один з них – трофейний, каже режисер Євген Ярило.

\ifcmt
  tab_begin cols=2

     pic https://scontent-yyz1-1.xx.fbcdn.net/v/t1.6435-9/197795530_1768483336667601_3204557349758304322_n.jpg?_nc_cat=101&ccb=1-5&_nc_sid=730e14&_nc_ohc=z92FDAj14F0AX__aVau&_nc_ht=scontent-yyz1-1.xx&oh=b493e81dd16dfa20a4108bc56ae87147&oe=6178E153

     pic https://scontent-yyz1-1.xx.fbcdn.net/v/t1.6435-9/196542687_1768505316665403_1265365208479254904_n.jpg?_nc_cat=100&ccb=1-5&_nc_sid=730e14&_nc_ohc=oKGpY1gTz14AX8bY-VV&_nc_ht=scontent-yyz1-1.xx&oh=88eb770fcff728e83c13c26826804baa&oe=61779BCD

  tab_end
\fi

Більше про «Чубаку» та зйомки картини він розповів журналістам «ТС».

«Це буде стрічка про видатну людину сучасної російсько-української війни -
Володимира Кочеткова-Сукача, позивний «Чубака», який загинув у 2015 році.

Він заснував підрозділ аеророзвідки, що дало можливість нашій армії стати більш
сучасною», - каже режисер.

До зйомок залучили побратимів Володимира, а кошти на картину виділив
Український культурний фонд.

«Зараз знімаємо один з епізодів тих подій на блок-посту, коли Володимир зробив
вчинок, який його характеризує», - зазначає Євген Ярило.

Загалом стрічка буде складатись із документальних та ігрових моментів.

\ifcmt
  tab_begin cols=2

		 pic https://scontent-yyz1-1.xx.fbcdn.net/v/t1.6435-9/198761543_1768505359998732_4301160407461021114_n.jpg?_nc_cat=100&ccb=1-5&_nc_sid=730e14&_nc_ohc=efUgaUMaeU4AX8-Bzte&_nc_ht=scontent-yyz1-1.xx&oh=fb15caf612b41ffb8da2198099c8de50&oe=617950F0

		 pic https://scontent-yyz1-1.xx.fbcdn.net/v/t1.6435-9/198156689_1768505323332069_3150495556609241996_n.jpg?_nc_cat=104&ccb=1-5&_nc_sid=730e14&_nc_ohc=uvYb9swBkkcAX85-GSE&_nc_ht=scontent-yyz1-1.xx&oh=b52f739b4c5dfd4fce7044f153839ba3&oe=6179B3C8

  tab_end
\fi

«Документальні – це будуть спогади побратимів та людей, які його знали, а
ігрові – постановка тих подій», - додає режисер.

Хронометраж стрічки - 45-50 хвилин. Побачити «Небесного координатора» можна
буде вже зовсім скоро, адже картину планують здавати у серпні.

«Ми вкладаємось у графік, хоч ковід і намагався наламати нам всю історію. Під
час роботи над фільмом нам постійно допомагали: коли потрібно було авто - його
підігнали хлопці з «Правого сектора», коли знадобились гільзи та інша військова
атрибутика – відгукнулись на стрільбищі у Броварах. Знаєте у всьому цьому
відчувається рука самого Володі, який десь зверху керує цим фільмом, його
підтримка», - додає Євген Ярило.

Режисер має надію, що з часом в Україні будуть знімати більше фільмів про
полеглих українських героїв і з’явиться новий кінопантеон.

«Павліки Морозови у головах мають відживати, нам потрібно плекати своїх героїв,
щоб було з кого брати приклад», - резюмував чоловік.

Зі сторінки проєкту «Небесний координатор» у Facebook:

Володимир Кочетков-Сукач був великою людиною - з великим серцем і просто
великим.

У АТО його називали «очима», через густу бороду дали позивний «Чубака», та він
носив шеврон з написом «Dron» і не звертав уваги на військові звання.

З перших днів російської агресії Володимир, який ніколи до цього не служив в
армії, перебував на передовій.

Разом з однодумцями він створив волонтерську групу «аеророзвідка», яка за
допомогою дронів проводила складні розвідувальні операції для наших розвідників
і артилеристів.

Завдяки створеній ним «Аеророзвідка» в дні запеклих боїв за ДАП у наших
військових щодня з'являлися важливі дані про переміщення сил супротивника. По
знімках з дронів Кочеткова бійці АТО виходили з оточення в районі Донецького
аеропорту.

«Чубака» загинув 15 березня 2015 року, підірвавшись на розтяжці недалеко від
Красногорівки Донецької області. У той день він повинен був встановити на
передовій спеціальні пристрої для дистанційної розвідки.

Документальний фільм «Небесний координатор» відтворює момент заснування
«Аеророзвідки» України.

У фільмі ви зможете побачити справжнє людське обличчя війни та зрозуміти тих,
хто опинився віч-на-віч з цим найлютішим винаходом людства.
