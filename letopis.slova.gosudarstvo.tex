% vim: keymap=russian-jcukenwin
%%beginhead 
 
%%file slova.gosudarstvo
%%parent slova
 
%%url 
 
%%author 
%%author_id 
%%author_url 
 
%%tags 
%%title 
 
%%endhead 
\chapter{Государство}

%%%cit
%%%cit_pic
%%%cit_text
И вот, в начале XX века, когда в результате Первой мировой войны, и
Австрийская, и Российская империи развалились, у западных русских, потерявших
свои корни и ставших «украинцами», появилась идея создания собственного
\emph{государства}. Ну, идеи создания собственных независимых \emph{государств} имеются не
только у национальных меньшинств, но даже и у однородных в этническом плане
социальных групп. Достаточно вспомнить Дальневосточную республику, образованную
в 1920 году и населенную русскими. Самый яркий пример – образование ряда
испаноязычных латиноамериканских \emph{государств}, образовавшихся на руинах Испанский
Америки. Можно вспомнить и прецедент образования США – ведь отделились от
Англии не индейцы и не негры, а те же самые англичане
%%%cit_title
\citTitle{Почему современную Украину назвали «окраиной», а не более престижно - Киевской Русью?}, 
Исторический Понедельник, zen.yandex.ru, 22.02.2021 
%%%endcit

