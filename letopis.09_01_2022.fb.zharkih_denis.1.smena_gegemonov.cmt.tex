% vim: keymap=russian-jcukenwin
%%beginhead 
 
%%file 09_01_2022.fb.zharkih_denis.1.smena_gegemonov.cmt
%%parent 09_01_2022.fb.zharkih_denis.1.smena_gegemonov
 
%%url 
 
%%author_id 
%%date 
 
%%tags 
%%title 
 
%%endhead 
\zzSecCmt

\begin{itemize} % {
\iusr{Marina Golovakha}
Надо этот класс как-то обозначать.
Электронные сми, блогеры, ЛОМы, пиарщики, политтехнологи, агенты влияния, лоббисты.
Виртуалы?
Информационщики?
...?

\begin{itemize} % {
\iusr{Денис Жарких}
\textbf{Marina Golovakha} не только они. На майданы бегают офисные хомячки, всевозможные продавцы воздуха и прочее

\iusr{Ира Гаврилова}
либерасы

\iusr{Marina Golovakha}
\textbf{Денис Жарких} эти миньоны: не генераторы информации, а зараженные, одержимые информационными идеями распространители спама/рассылок/репостеры/пехота акций протеста

\iusr{Юрий Лукашин}
\textbf{Marina Golovakha} креаклиат, креативеый класс. Уже лет 15 про них много говорится

\iusr{Marina Golovakha}
\textbf{Юрий Лукашин} точно же.
Креаклы!!!
\textbf{Денис Жарких}, что думаешь?
Клеаклы в значении паразитирующих и как то еще нормальных людей, создающих смыслы назвать.

\iusr{Вячеслав Викторович Еля}
Есть хороший термин на западе Демократура. А это все пропагандисты Демократуры.

\iusr{Вячеслав Викторович Еля}
У Демократуры ценности не такие как у фашизма но методы примерно такие же.

\end{itemize} % }

\iusr{Marina Golovakha}

Демократии больше нет. Капитализм показал свое совершенно тоталитарное лицо.

1) Набрать аудиторию и отрезать/блокировать/банить несогласных в глобальных
сервисах;

2) набрать массу владельцев карт и блокировать счета, арестовывать и проч;

3) набрать массу пользователнй смартфонов через удобства и следить за ними
через приложения, блокировать и т.п подобие тюрьмы с электронным браслетом.

Это три основных механизма

\iusr{Вячеслав Викторович Еля}
\textbf{Marina Golovakha} Цифровая Демократура.

\iusr{Sergey Raykov}

Демократия - власть народа по- гречески... но в Древней Греции... права имели
не все: рабы, иноземцы, другие неграждане...

"Не прикасайтесь ко мне я римский гражданин!- говорили в Риме, но было 10 млн
рабов в Империи и миллионы в провинциях неграждан Была такая прослойка в
Риме... их называли агенты... люди, оказывающие услуги сильным мира...
патрициям, сенаторам, всадникам...  Этих людей, которые паразитируют на
создании информационных полей, я бы тоже назвал агентами..

В сериале "Зимняя вишня" экс- советский инженер (Виталий Соломин) попадает в
приморскую деревушку в Бельгии. Надо чем то заниматься. Знакомится с
бельгийцем. Тот спрашивает: Что умеешь делать, кем работал? Наш: Был инженер в
НИИ, искал узкие места, решал проблемы.. Бельгиец ни черта не понял. Спрашивает:
Машину можешь водить? Наш: Могу и люблю.

Бельгиец: Ок, мы будем партнёры.. будем водить большегруз по Европе..
Человек занялся реальным делом...
Нам казалось, что в сложный переходный период, система очищалась, самоптимизировалась..
Но как только жить стали лучше... класс "агентов" воспрянул... как птица Феникс из пепла...

\emph{Marina Golovakha}
\textbf{Sergey Raykov} агенты и спящие агенты )
 · 18 ч.
\emph{Даниил Богатырёв}

Не кажется ли Вам, что пора отказаться от концепции \enquote{классов}, как в их
либеральной, так и в коммунистической трактовке? Они попросту не
репрезентативны. Определять то, кем человек является, исходя из его достатка,
или обладания частной собственностью и/или средствами производства - всё равно,
что тыкать пальцем в небо. Попасть в точку можно, разве что, случайно.
Количественный подход не способен определить суть той или иной социальной
группы, её устремлений, и т. п. И тем более - суть конкретной личности.

В этом плане, гораздо более репрезентативным мне представляется качественный
подход, в котором люди объединяются в группы для классификации исходя из их
врождённых склонностей, задатков и качеств. Можете называть это \enquote{внутренними
сословиями}, или \enquote{внутренними кастами}. Не уверен, что эти термины будут поняты
корректно. Но, во всяком случае, если отбросить модернистское количественное
понимание справедливости (в частности - социальной) и возвратиться к описанному
в "Государстве" Платона качественному пониманию (справедливость - это когда
каждый занимается тем, к чему наиболее склонен; каждый на своём месте), мой
посыл станет понятен.

При такой трактовке станет совершенно очевидно, что в любую эпоху процент
условного "духовного и мыслящего сословия", "воинского и политического
сословия", "торгового сословия", "ремесленного сословия" и внесословных изгоев
примерно одинаков. И не важно, какие занятия "в моде" в конкретную эпоху. Важно
то, что среди социальной группы, условно обозначенной Вами как "класс
интеллигенции" присутствуют как интеллигенты и интеллектуалы (профессора,
философы, и т. п.), так и ремесленники, видящие в симулировании
интеллектуального труда синекуру. А значит, эта социальная группа (или "класс")
не репрезентативна.


\end{itemize} % }
