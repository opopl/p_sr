% vim: keymap=russian-jcukenwin
%%beginhead 
 
%%file 16_07_2017.stz.news.ua.mrpl_city.1.istoria_den_metallurga
%%parent 16_07_2017
 
%%url https://mrpl.city/blogs/view/istoriya-den-metallurga
 
%%author_id burov_sergij.mariupol,news.ua.mrpl_city
%%date 
 
%%tags 
%%title История: День металлурга
 
%%endhead 
 
\subsection{История: День металлурга}
\label{sec:16_07_2017.stz.news.ua.mrpl_city.1.istoria_den_metallurga}
 
\Purl{https://mrpl.city/blogs/view/istoriya-den-metallurga}
\ifcmt
 author_begin
   author_id burov_sergij.mariupol,news.ua.mrpl_city
 author_end
\fi

В той, уже ушедшей в историю эпохе поначалу праздников было маловато. Ну, День
Парижской коммуны, 8 Марта, 1 Мая, очередные годовщины Октябрьской революции,
Красной Армии. Список их рос медленно. В 1922 году появился День печати, с
1933-го свой профессиональный праздник отмечали авиаторы, с 1936-го –
железнодорожники. В 1936-м учреждено празднование Дня Конституции. Перед
развалом Советского Союза, в 1991 году, существовало более 30 праздничных дней.
28 сентября 1957 года был учрежден День металлурга.

Впервые этот праздник отметили в воскресенье, 20 июля 1958 года, в том числе и
в нашем городе. За день до этого в Москве был подписан Указ о присвоении
звания Героя Социалистического Труда за выдающиеся успехи в деле развития
черной металлургии ряду работников этой отрасли. В том числе азовстальцам -
старшему мастеру доменного цеха М. М. Ивченко, старшему мастеру рельсобалочного
цеха П. А. Чеховскому, старшему горновому Г. Л. Чуйко, сталевару В. А. Шкуропату.
Высокой награды были удостоены ильичевцы - старший мастер мартеновского цеха
И.А. Лут, старший вальцовщик листопрокатного цеха Е. Т. Страхов, старший мастер
листопрокатного цеха А. Л. Деревянко, старший  мастер мартеновского цеха П. А.
Васильев. 21 июля 1961 года был подписан Указ о присвоении почетного звания
\enquote{Заслуженный металлург УССР}. Среди отмеченных были и представители нашего
города - старший мастер мартеновского цеха завода \enquote{Азовсталь} Виталий  Аврамов
и сталевар завода имени Ильича Виктор Ясиновой.

Каким был профессиональный праздник металлургов в нашем городе в разное время?
Об этом могут дать представление наугад взятые выдержки из номеров газеты
\enquote{Приазовский рабочий}. 

14 июля 1962 года. Славными трудовыми делами отмечает на заводе им. Ильича
коллектив второго трубосварочного цеха. На счету трубосварщиков сотни тонн
сверхплановых труб. Их продукция идет на международный нефтепровод \enquote{Дружба}, на
многие нефтегазовые магистрали страны. В этом году строители введут в
эксплуатацию на заводе им. Ильича доменную печь, три сверхмощные мартеновские
печи и цех холодного проката. Азовстальцы получат в третьем квартале первую в
стране кислородную установку БР-2. Это позволит намного увеличить выпуск
чугуна, стали и проката. На заводе им. Ильича есть на кого равняться. Это
сталевары Лев Горбачев, Владимир Слободяник, Михаил Гонда, Иван Копылов,
доменщик Виктор Попенко, прокатчики Павел Таран, Георгий Максимча и сотни
других.

16 июля 1967 года. Сегодня в кинотеатре \enquote{Победа} демонстрируются фильмы,
посвященные людям огненной профессии – творцам металла. Среди кинолент –
\enquote{Спокойная сталь}, \enquote{Гордое дыхание} и \enquote{Для тебя литейщик}. А вечером на летней
эстраде Центрального парка культуры и отдыха состоится  большой концерт
симфонической музыки. Артисты Донецкой филармонии исполнят по заявкам
металлургов произведения советских, русских и зарубежных композиторов.

17 июля 1976 года. В канун Дня металлурга на заводе \enquote{Азовсталь} достигнута
новая трудовая победа – выплавлена 100-миллионная тонна азовстальского чугуна.

17 июля 1992 года. Уважаемые мариупольцы и гости нашего города! 19 июля, в
воскресенье, приглашаем вас и ваши семьи на  большой культурно-спортивный
праздник, посвященный Дню металлурга, в парк культуры и отдыха комбината
\enquote{Азовсталь}. В программе: концерт детской художественной самодеятельности
Дворца культуры Коксохимического завода, для малышей работает городок
аттракционов, на танцевальный марафон приглашают ВИА \enquote{Эпизод} и народный
вокально-инструментальный ансамбль \enquote{Чайка}, концерт самодеятельных коллективов
Дворца культуры \enquote{Азовсталь}, праздничный фейерверк.
