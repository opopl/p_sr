% vim: keymap=russian-jcukenwin
%%beginhead 
 
%%file 21_11_2021.fb.nitsoi_larysa.1.revgidnosti_selo_bazar
%%parent 21_11_2021
 
%%url https://www.facebook.com/larysa.nitsoi/posts/4885436018155158
 
%%author_id nitsoi_larysa
%%date 
 
%%tags 1921,godovschina,istoria,jazyk,maidan2,moskovia,mova,revgidnosti,rossia,ukraina,unr
%%title Восьма річниця Революції Гідности. Я стою в селі Базар...
 
%%endhead 
 
\subsection{Восьма річниця Революції Гідности. Я стою в селі Базар...}
\label{sec:21_11_2021.fb.nitsoi_larysa.1.revgidnosti_selo_bazar}
 
\Purl{https://www.facebook.com/larysa.nitsoi/posts/4885436018155158}
\ifcmt
 author_begin
   author_id nitsoi_larysa
 author_end
\fi

Восьма річниця Революції Гідности. Я стою в селі Базар... Рівно сто років тому
1921 року Українські бійці армії УНР, які потрапили в полон, всі як один,
відхилили пропозицію перейти в лави червоної армії котовського. Їх розстріляли.
Усіх. 359  унеерівців. Полонені бійці УНР відмовилися співпрацювати з
Московією. А ще, вони відмовилися перейти з ворогами на московську мову. Хтось
із них перед розстрілом вигукнув «Слава Україні!» Хтось перед кулеметною чергою
встиг заспівати «Ще не вмерла України ні слава, ні воля, ще нам
браття-українці»...  ГІДНІ сини України.

\ifcmt
  ig https://scontent-frx5-1.xx.fbcdn.net/v/t39.30808-6/258166930_4885435864821840_1290740144116196950_n.jpg?_nc_cat=110&ccb=1-5&_nc_sid=8bfeb9&_nc_ohc=3czfTvlx4OcAX-cSNWl&_nc_ht=scontent-frx5-1.xx&oh=4d53f1b0a5e94e83682969d8e5d64073&oe=619F6902
  @width 0.4
  %@wrap \parpic[r]
  @wrap \InsertBoxR{0}
\fi

Сьогодні, 100 років з тієї дати. Так співпало, що на ці дні припадає Річниця
новітньої революційної гідности. Восьма… Що в нас на сьогодні зі співпрацею з
ворогом? З тим самим ворогом, щоб був у героїв Базару. Спортсмени мовчать, що
увесь спорт в Україні, олімпійські збірні, тренери – змосковщені. Україномовних
дітей, юнаків і дівчат, які приходять в спорт, перемелює тренерська змосковщена
команда, і юні спортсмени стають московитами, бо робоча мова спорту в Україні –
московська. Українські спортсмени єдині, хто на міжнародних змаганнях дає
прес-конференції не державною мовою, а московською, демонструючи на весь світ,
що теза путіна «ми адін нарот» – достовірна. Я кажу українському спорту і
спортсменам: «Це ганьба». У відповідь проти мене піднімається частина
україномовної патріотичної ЕЛІТИ на захист цих спортсменів. Україномовна
українська ЕЛІТА мені каже: «Ларисо, не чіпай спортсменів, це не на часі. Вони
патріоти. Вони гарні спортсмени». А хіба я обговорюю їхній патріотизм? Хіба я
обговорюю їхню фаховість? Я обговорюю гідність. Бо спортсмени, які мають
гідність, не говорили б на весь світ мовою ворога, коли мають ВЛАСНУ мову, мову
свого народу. Українські спортсмени, які утверджують путінську тезу про «адін
нарот» – ллють воду на млин ворога, працюють на руку ворогу. А, отже,
співпрацюють з ворогом. 

Восьмий рік гідности... Частина українських співаків співають в Україні мовою
ворога, а інші музиканти мовчать, вважаючи не гідно виступати проти своїх.
Частина українських співаків створюють свої нові альбоми мовою ворога і
демонструють всій Україні: «Це нормально, мати свою мову, але помножити її на
нуль і творити українське мистецтво не українською, а  мовою Московії.
Дивіться, підростаюче покоління українців, які ми успішні, які ми доглянуті,
які в нас будинки і відпочинки. Ми, ваші кумири, зразок для наслідування. Ми,
маючи ВЛАСНУ мову свого народу, співаємо мовою Московії, розмовляємо мовою
Московії, і, оскільки ми зразок для наслідування - робіть як ми!». Московія
саме цього й добивалася усю свою історію – утвердити московську мову в Україні.
І частина українських музикантів, утверджуючи мову ворога у своїй країні – ллє
воду на млин ворога. Утверджує і поширює мову ворога у своїй країні на шкоду
українській мові – хочете чи не ні, але, працює на ворога. Українські співаки,
які їздять на Московію на гастролі – співпрацюють з ворогом.  Я кажу: «Це
ганьба». Але частина українськомовної еліти повстає проти мене й каже: «Ларисо,
не чіпай музикантів. Просто їм тісно в межах нашої країни. Їм потрібні ширші
горизонти. Але вони патріоти». Я питаю в тих, хто співпрацює, і в тих, хто
мовчки споглядає на цю шкоду: «Де ваша гідність? Чи задумуєтеся ви, що формуєте
гідність у всього народу? Яка ця, сформована вами, гідність?» 

Восьмий рік гідности... Українське кіно знімається в Україні мовою Московії.
Нехай би ви продавали те кіно в Московію московською мовою, але ЧОМУ в Україні
для українців на всіх каналах ТБ це кіно йде мовою Московії?  Чому не
українською? Чому не державною мовою, як це в інших країнах? Українські актори
їдуть в Московію і знімаються в московських фільмах, граючи московських
військових з московською символікою, тоді, коли ці військові в житті вбивають
мій народ на сході! Сміються: «А ми ж не у військовому фільмі знімаємося, це
московський військовий детектив». Я кажу: «Це ганьба!» У відповідь частина
українських українськомовних патріотичних акторів, частина україномовної
патріотичної еліти повстає на захист цих акторів! Кажуть мені: « Не чіпай
акторів! Вони патріоти! Українське кіно не розвивається, куди їм дітися? Їм
треба десь заробляти!»

Це ви про заробітки плачетеся мені, українській письменниці НЕЧИТАЮЧОГО НАРОДУ?
Це ви кажете мені, українській письменниці, країна якої не популяризує книги,
письменників і читання? Країна якої займає найнижчі щаблі в читацьких
міжнародних рейтингах? Ви говорите про заробляння грошей і горизонти мені,
українській письменниці, яка живе з книжок, але нація якої замість прочитати
книжку, сидить у телевізорах, а в тих телевізорах ВИ! Це ви МЕНІ кажете? 

Розкажіть про свої горизонти і заробляння отим героям Базару, що в мене за
спиною. Розкажіть їм, що таке «немає виходу» і що таке ваш патріотизм. Заодно
розкажіть, що таке гідність. Усі розкажіть, хто співпрацює, хто мовчить, хто
стає на захист їхньої співпраці. 

Я прощаюся з Вами. Бо що означає для мене цей допис? Він означає бан на місяць.
А що для мене означатиме цей бан? Це означатиме, що я протягом місяця не зможу
нікого загітувати (серед і так нечитаючих) придбати мою хоча б одну-дві якісь
книжки. А я живу з цих книжок і в мене є лише цей майданчик, який мені
заблокують. А йдуть свята, під які мої книги гарно підходять. Це і 6 грудня
(День Збройних сил України), і 18 грудня (День Святого Миколая), Але я сидітиму
забанена. Я знаю, на що йду, але не можу змовчати. Так, це такий мій дрібний
героїзм. А для багатьох ще й дурість...

Для порівняння, розповім, як пройде цей місяць мого бану для вас, тих, кого я
згадала в дописі, кому малі горизонти в нашій країні. Актори й далі
зніматимуться в москвомовних фільмах і гратимуть в москвомовних виставах – і їм
за них заплатять. Інша україномовна частина акторів спостерігатиме і мовчатиме
або підтримуватиме. Співаки й далі співатимуть москвомовні пісні, їх і далі
показуватимуть по телевізору та різних шоу, інша частина співаків,
україномовних, заздрісно чи без заздрості, мовчки спостерігатиме. Актори
театрів з московськими виставами, якщо й не гратимуть через карантин, отримають
зарплату з бюджету, або гратимуть, тому все одно отримають, так і не повставши
проти вистав, які змосковщують українців-глядачів. Так само заплатять
змосковщеним спортсменам... І всі ці актори, співаки й спортсмени знову розкажуть
на різних ток-шоу, що в них немає горизонтів і іншого виходу. І частина
української еліти (україномовної, патріотичної) знову їх публічно підтримає. А
мій народ сидітиме і це все жертиме з телевізора, який це все популяризує і
популяризуватиме. 

З восьмим Днем гідности тебе, Україно і українська еліто. Скільки б Ви не
захищали своїх друзів-акторів, друзів-співаків, друзів-спортсменів, чи й не
друзів, а просто так із симпатії, ЇХНЯ СПІВПРАЦЯ З ВОРОГОМ – це співпраця з
ворогом. І хоч скільки не прикривайте вашу підтримку патріотизмом чи
прекрасними особистими якостями – ви захищаєте співпрацю з ворогом!  Не питай
мене, українська еліто, і не стогни, звідки в нас так багато вати. Бо це твоя
відповідальність. 

З восьмим Днем Гідности... А за Восьмим буде Дев’ятий... А за ним Десятий... День
Гідности?

\ii{21_11_2021.fb.nitsoi_larysa.1.revgidnosti_selo_bazar.cmt}
