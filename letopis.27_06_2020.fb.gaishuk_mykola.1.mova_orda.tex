% vim: keymap=russian-jcukenwin
%%beginhead 
 
%%file 27_06_2020.fb.gaishuk_mykola.1.mova_orda
%%parent 27_06_2020
 
%%url https://www.facebook.com/mykola.gaishuk.3/posts/977503956013794
 
%%author 
%%author_id 
%%author_url 
 
%%tags 
%%title 
 
%%endhead 

\subsection{Нагадаю, як вбивали ординці Мову...}
\Purl{https://www.facebook.com/mykola.gaishuk.3/posts/977503956013794}

1622 — наказ царя Михайла з подання Московського патріарха Філарета спалити в державі всі примірники надрукованого в Україні "Учительного Євангелія" К. Ставровецького.
1696 — ухвала польського сейму про запровадження польської мови в судах і установах Правобережної України.
1690 — засудження й анафема Собору РПЦ на "кіевскія новыя книги" П. Могили, К. Ставровецького, С. Полоцького, Л. Барановича, А. Радзивиловського та інших.
1720 — указ Петра І про заборону книгодрукування українською мовою і вилучення українських текстів з церковних книг.
Диякон УПЦ МП про анафему Мазепі. Чи існує вона досі?
1729 — наказ Петра ІІ переписати з української мови на російську всі державні постанови і розпорядження.
1763 — указ Катерини II про заборону викладати українською мовою в Києво-Могилянській академії.
1769 — заборона Синоду РПЦ друкувати та використовувати український буквар.
1775 — зруйнування Запорізької Січі та закриття українських шкіл при полкових козацьких канцеляріях.
1789 — розпорядження Едукаційної комісії польського сейму про закриття всіх українських шкіл.
1817 — запровадження польської мови в усіх народних школах Західної України.
1832 — реорганізація освіти на Правобережній Україні на загальноімперських засадах із переведенням на російську мову навчання.
1847 — розгром Кирило-Мефодієвського товариства й посилення жорстокого переслідування української мови та культури, заборона найкращих творів Шевченка, Куліша, Костомарова та інших.
1859 — міністерством віросповідань та наук Австро-Угорщини в Східній Галичині та Буковині здійснено спробу замінити українську кириличну азбуку латинською.
1862 — закриття безоплатних недільних українських шкіл для дорослих в підросійській Україні.
1863 — Валуєвський циркуляр про заборону давати цензурний дозвіл на друкування україномовної духовної і популярної освітньої літератури: "жодної окремої малоросійської мови не було і бути не може".
Консул Росії: Україну придумали австріяки, а українську мову - комуністи
1864 — прийняття Статуту про початкову школу, за яким навчання має проводитись лише російською мовою.
1869 — запровадження польської мови в якості офіційної мови освіти й адміністрації Східної Галичини.
1870 — роз'яснення міністра освіти Росії Д.Толстого про те, що "кінцевою метою освіти всіх інородців незаперечно повинно бути обрусіння".
1876 — Емський указ Олександра  про заборону друкування та ввозу з-за кордону будь-якої україномовної літератури, а також про заборону українських сценічних вистав і друкування українських текстів під нотами, тобто народних пісень.
1881 — заборона викладання у народних школах та виголошення церковних проповідей українською мовою.
1884 — заборона Олександром IIІ українських театральних вистав у всіх малоросійських губерніях.
1888 — указ Олександра IIІ про заборону вживання української мови в офіційних установах і хрещення українськими іменами.
1892 — заборона перекладати книжки з російської мови на українську.
1895 — заборона Головного управління в справах друку видавати українські книжки для дітей.
1911 — постанова VII-го дворянського з'їзду в Москві про виключно російськомовну освіту й неприпустимість вживання інших мов у школах Росії.
1914 — заборона відзначати 100-літній ювілей Тараса Шевченка; указ Миколи ІІ про скасування української преси.
"Жидо-мазепинцы". Як чорносотенці агітували проти ювілею Шевченка
1914, 1916 — кампанії русифікації на Західній Україні; заборона українського слова, освіти, церкви.
1922 — проголошення частиною керівництва ЦК РКП(б) і ЦК КП(б)У "теорії" боротьби в Україні двох культур — міської (російської) та селянської (української), в якій перемогти повинна перша.
1924 — закон Польської республіки про обмеження вживання української мови в адміністративних органах, суді, освіті на підвладних полякам українських землях.
1924 — закон Румунського королівства про зобов'язання всіх "румун", котрі "загубили матірну мову", давати освіту дітям лише в румунських школах.
1925 — остаточне закриття українського "таємного" університету у Львові
1926 — лист Сталіна "Тов. Кагановичу та іншим членам ПБ ЦК КП(б)У" з санкцією на боротьбу проти "національного ухилу", початок переслідування діячів "українізації".
1933 — телеграма Сталіна про припинення "українізації".
1933 — скасування в Румунії міністерського розпорядження від 31 грудня 1929 p., котрим дозволялися кілька годин української мови на тиждень у школах з більшістю учнів-українців.
1934 — спеціальне розпорядження міністерства виховання Румунії про звільнення з роботи "за вороже ставлення до держави і румунського народу" всіх українських вчителів, які вимагали повернення до школи української мови.
1938 — постанова РНК СРСР і ЦК ВКП(б) "Про обов'язкове вивчення російської мови в школах національних республік і областей", відповідна постанова РНК УРСР і ЦК КП(б)У.
1947 — операція "Вісла"; розселення частини українців з етнічних українських земель "урозсип" між поляками у Західній Польщі для прискорення їхньої полонізації.
Акція "Вісла" - останній акт польсько-української трагедії
1958 — закріплення у ст. 20 Основ Законодавства СРСР і союзних республік про народну освіту положення про вільний вибір мови навчання; вивчення усіх мов, крім російської, за бажанням батьків учнів.
1960-1980 — масове закриття українських шкіл у Польщі та Румунії.
1970 — наказ про захист дисертацій тільки російською мовою.
1972 — заборона партійними органами відзначати ювілей музею І.Котляревського в Полтаві.
1973 — заборона відзначати ювілей твору І. Котляревського "Енеїда".
1974 — постанова ЦК КПРС "Про підготовку до 50-річчя створення Союзу Радянських Соціалістичних Республік", де вперше проголошується створення "нової історичної спільноти — радянського народу", офіційний курс на денаціоналізацію.
1978 — постанова ЦК КПРС і Ради Міністрів СРСР "Про заходи щодо подальшого вдосконалення вивчення і викладення російської мови в союзних республіках" ("Брежнєвський циркуляр").
1983 — постанова ЦК КПРС і Ради Міністрів СРСР "Про додаткові заходи з поліпшення вивчення російської мови в загальноосвітніх школах та інших навчальних закладах союзних республік" ("Андроповський указ").
Книговидання в УРСР: скільки російською і скільки українською
1984 — постанова ЦК КПРС і Ради Міністрів СРСР "Про дальше вдосконалення загальної середньої освіти молоді і поліпшення умов роботи загальноосвітньої школи".
1984 — початок в УРСР виплат підвищеної на 15\% зарплатні вчителям російської мови порівняно з вчителями мови української.
1984 — наказ Міністерства культури СРСР про переведення діловодства в усіх музеях Радянського Союзу на російську мову.
1989 — постанова ЦК КПРС про "законодавче закріплення російської мови як загальнодержавної".
1990 — прийняття Верховною Радою СРСР Закону про мови народів СРСР, де російській мові надавався статус офіційної.
2012 - прийняття Верховною Радою України проекту Закону "Про основи державної мовної політики", який загрожує значним звуженням сфери використання української мови у ключових сферах життя в більшості регіонів України.
