% vim: keymap=russian-jcukenwin
%%beginhead 
 
%%file 31_03_2021.fb.bilchenko_evgenia.1.vne_sistemy_gimn_pridurkov
%%parent 31_03_2021
 
%%url https://www.facebook.com/yevzhik/posts/3764944123540670
 
%%author 
%%author_id 
%%author_url 
 
%%tags 
%%title 
 
%%endhead 

\subsection{Мы – те, кого Бог позовёт в разведку, Когда нападёт на мир}
\Purl{https://www.facebook.com/yevzhik/posts/3764944123540670}


\ifcmt
  pic https://external-bos3-1.xx.fbcdn.net/safe_image.php?d=AQG9Tv9H6KNLFdQQ&w=500&h=261&url=https%3A%2F%2Fi.ytimg.com%2Fvi%2FbWcZjovwong%2Fmaxresdefault.jpg&cfs=1&ext=jpg&ccb=3-4&_nc_hash=AQGL-MZ5N-Odx2hv
\fi

Спасибо, мой товарищ Екатерина Жарких
 , за включение меня в передаче в топ любимых женщин. Не уверена, что заслуживаю. Но мне радостно и приятно, дорогая и прекрасная, умная и смелая, красивая и честная моя Катя.
https://m.youtube.com/watch?v=bWcZjovwong&feature=youtu.be
Я в ответ - могу лишь стихи:
БЖ. Гимн придурков
Моим друзьям, которые - вне Системы, посвящается...
Мы спим на чужих воровских квартирах,
Мы служим мишенью в богатых тирах;
Мы, пятясь, ползём по болотным тинам
И к солнцу бежим рысцой;
Мы ноги ломаем на ровном месте;
Мы бешенство делаем долгом чести;
Мы ломимся в хату к чужой невесте…
Когда же нас бьют в лицо, –
Мы, сплюнув, бросаем в ответ: «Мещане!» –
Мы жизнь обретаем в смертельной ране.
Мы мячик находим для бедной Тани,
Чтоб гвоздик в него вколоть;
Сияя античной улыбкой мима,
Идём мы сквозь время – вперед и мимо.
Мы – не-вы-но-си-мы… Неизлечимы
Ни дух наш, ни стать, ни плоть.
О нас продолжают сниматься фильмы.
Наш дым пропускают сквозь сотни фильтров.
Нас чтут как пророков. Как в неофитов
В нас камни метает сброд;
Мы видим себя на большом экране
Как явный просчёт во вселенском плане:
Мы можем казаться нежнее лани,
Но в каждом живёт койот.
Мы рвано обуты. Легко одеты.
Мы курим вонючие сигареты.
Мы вами, должно быть, давно отпеты
Во тьме воровских квартир,
Куда мы в окошко стучимся веткой,
Но стук этот слышится крайне редко…
Мы – те, кого Бог позовёт в разведку,
Когда нападёт на мир.
