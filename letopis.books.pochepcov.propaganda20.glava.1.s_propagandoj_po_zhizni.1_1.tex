% vim: keymap=russian-jcukenwin
%%beginhead 
 
%%file books.pochepcov.propaganda20.glava.1.s_propagandoj_po_zhizni.1_1
%%parent books.pochepcov.propaganda20.glava.1.s_propagandoj_po_zhizni
 
%%url 
 
%%author_id 
%%date 
 
%%tags 
%%title 
 
%%endhead 

Во все века люди находятся в рамках действий определенных систем социального
управления. Наложенные на них рамки всегда носили внешний характер. Наиболее
известны религиозные требования к правильному поведению. Но и просто
человеческие объединения начались с достаточно тоталитарных условий. Это так
называемые дворцы-государства (см. о них [1–5]). Это был достаточно жесткий мир
бедных людей. Поэтому справедливы слова исследователей того периода [3]:
«Бедность и молчание могут лучше отражать в любую эпоху человеческие условия,
чем богатство и великолепие».

Информация в тот период имела другой статус. С одной стороны, ее было мало,
поскольку мир был сужен пределами видимости, знания не передавались, поскольку
не было эффективной системы их фиксации. С другой стороны, мир был постоянным и
стабильным, изменения проходили мимо него, по этой причине нечего было
фиксировать. С третьей – был резко занижен статус человека в низу иерархии, его
мнения и интересы были абсолютно несущественными для тех, кто находился наверху
иерархии. Это было традиционное общество, где информация не играла большой
роли.

Государства-дворцы начались как способ хранения дополнительной
сельхозпродукции. Потом произошел их рост, усложнение бюрократии, которая
вскоре перестала справляться с управлением большими территориями, людьми и
запасами. Как видим, аграрные цивилизации хранят сельхозпродукцию, как
информационные – информацию и знания, видя степень своей выживаемости именно в
своем типе продукта. Кстати, отсюда следует, что и идеологические цивилизации,
примерами которых были тоталитарные государства, не зря повсюду расставляют
памятники своих вождей. Они тоже «хранят их от порчи» таким своеобразным
способом.

О дворцах-государствах говорят следующее [5]: «На сегодня это самые ранние
государства в истории человечества. Устройство их поначалу казалось странным:
центр всего – большое сооружение, целый лабиринт каких-то помещений. Постепенно
выяснилось, что это разного рода склады – «закрома родины». Некоторые из
государств обладали письменностью, плоды которой заполняют часть помещений
дворца, – это архивы. Содержание текстов не оставляет сомнения: это инструкции
– что, где, когда сеять, жать, доить, сколько чего поставить в закрома и когда,
кому, какие строительные и транспортные работы произвести. А также кому сколько
из запасов выдать на пропитание, посев, строительство».

Дворцы-государства создавали дворцы для одних и хижины для других. И это
распределение труда должно было удерживаться, что делается чаще с помощью меча,
чем слова.

Но меч – это тактический вариант управления. Религия же или пропаганда являются
стратегическими вариантами. Они не столько заняты фактами, как правилами. И это
более важно, поскольку факты являются результатом применения правил. Создав
правило, можно получать новые факты. Создав новый факт, все равно не возникнет
нового правила.

Примером пропаганды 1.0 являются социальные протесты. Здесь есть случайность,
импровизация, несущественность эстетической составляющей. И самое главное – эта
пропаганда направлена на дестабилизацию социосистемы. Это по сути своей
любители, которые иногда могут побеждать. И тогда звучат слова из перeвода
Маршака: «Когда мятеж кончается удачей, зовется он, как правило, иначе».

Открытой формой пропаганды 1.0 занимаются и государства, но не отрицательной, а
позитивно-ориентированной, в чем особенно преуспели тоталитарные государства,
которые вовсю хвалят себя и ругают других. Они не хотят оставлять человека вне
себя ни на работе, ни дома. Тоталитарное государство не боится писать СЛАВА
КПСС на каждой крыше, поскольку функционирует вне конкуренции. Это как
рекламный лозунг ЛЕТАЙТЕ САМОЛЕТАМИ АЭРОФЛОТА в СССР, когда других компаний
просто не было.

Пропаганда 2.0 – это пропаганда профессионалов. В этой систематике появляется
важная эстетическая составляющая. Это голос Левитана в одной ситуации,
приведший к тому, что он стал личным врагом Гитлера, но это и очарование улыбки
Татьяны Самойловой или Одри Хепберн в другой. Это телесериалы, где «свои»
всегда побеждают «чужих», что, конечно, более приятно, чем если бы все было
наоборот. Пропаганда 2.0 никогда не будет проповедовать пессимизм.

Это более рекламный подход, пришедший с развитием рекламы, когда реагирование
является более эмоциональным, чем рациональным. А такое реагирование по сути
своей автоматическое, в нем нет места разуму. На очарование мы всегда будет
реагировать прогнозируемо.

Если в пропаганде 1.0 работало только содержание, то в пропаганде 2.0 работает
и эстетическая форма. Именно по этой причине мы и сегодня можем смотреть
советские фильмы как сильные с художественной стороны, оставляя возможные
идеологические вкрапления без внимания. Да и не все фильмы могли их иметь.
Фильм «Офицеры» – да, фильм «Золушка» – нет. Произведения искусства более
далекого прошлого тоже имеют идеологическую составляющую, но она еще сильнее
закрыта от нас временем.

Государство строит свои системы предупреждения против негативно-ориентированной
пропаганды 1.0, по сути пытаясь уловить переход от индивидуального к массовому
протесту. Большие массивы людей не могут возникнуть сами по себе без
определенной организационной активности, которую и пытается отслеживать
государство. Но оно также занято и оценкой развития протестности по социальным
сетям (см., например, в России [6–8], в Британии [9—12], см. также презентацию
из «утечки» Сноудена [13]).

Пропаганда 2.0 направлена на удержание картины мира, что делается достаточно
часто с помощью эстетически ориентированных методов типа телесериалов и кино,
поскольку главная эстетика там – визуальная. Если на вербальную эстетику можно
еще отвечать рационально, то визуальная эстетика проходит без таких
возможностей, она более эмоциональна с точки зрения ее инструментария
воздействия.

Такая пропаганда демонстрирует необходимость существования государства и его
институтов (армия, полиция, суд) как гарантов стабильности. Практически все
«долгоиграющие» символизации (название улиц и площадей, памятники и пр.)
направлены на демонстрацию вечности данной власти. Иногда для этого приходится
отображать нестабильность на телеэкране, чтобы показать нужность государства
как защитника. О государстве как о торговце страхами говорят Рансьер и
Павловский.

Пропаганда 1.0 имеет эстетическую составляющую как факультативную, доминирующим
сообщением в ней является идеологическая составляющая. Совершенно наоборот
построена Пропаганда 2.0, для нее идеологемы уходят на второй план, хотя и
присутствуют. Но будучи фоновыми, они уже не воспринимаются так активно, не
требуют реагирования на себя. Получатель информации реагирует на первый план,
оставляя второй вне реагирования.

Пропаганда 1.0 сталкивается с Пропагандой 2.0 в случае революций. Это
столкновение двух нарративов: доминирующего, который удерживается Пропагандой
2.0, и контрнарратива, привнесенного Пропагандой 1.0. В случае победы революции
Пропаганда 1.0 постепенно превращается в вариант Пропаганды 2.0. И население
снова получает рассказы о правильности теперь уже новой власти.

Пропаганда 1.0 имеет еще одну важную особенность – она строится на понятии
Врага, выстраивая если не все, то очень многое вокруг него. Можно предположить,
что это отголосок более древних подходов, когда главным противопоставлением
было «мы» и «они». Не зря Красная шапочка построена на запрете не разговаривать
с чужими. Тем более в прошлом «чужие» были гораздо опаснее.

Сегодня условная партия «Красной шапочки» выходит на борьбу против партии
«Волка». В партию-конкурента вписываются все варианты грехов, в то же время
партия «Красной шапочки» чиста и идеальна. Но на следующем этапе все опять
меняется.

Враг системно очень интересен, ведь даже когда его нет, его системно все равно
вписывают, поскольку без него распадется наше привычное бинарное видение мира.
Враг прямо и косвенно помогает становлению Героя. Четкость врага помогает в
создании такой же четкости героя.

Вот как о враге отзываются исследователи [14]: «Дискурсы о Враге основаны на
серии бинарных оппозиций, таких как добрый/ лой, справедливый/несправедливый,
виновный/невиновный, рациональный/иррациональный,
цивилизованный/нецивилизованный, – что может быть названо плавающими
означающими. Как плавающие означающие эти дихотомии не имеют фиксированного
значения, но они артикулируются до, во время и после конфликта. Более того,
конструирование Врага сопровождается конструированием идентичности самого Себя,
которое делается в антагонистическом отношении к идентичности Врага» (см.
подробнее о плавающих означающих [15–19]).

Э. Лаклау (см. о нем [20]) в соавторстве с Ш. Муфф воспользовались не только
термином «плавающего означающего», но и известным термином А. Грамши
«гегемония» (см. взгляд на гегемонию с современных позиций [21]). Ш. Муфф (см.
ее био [22]) считает задачей медиа создание полемических публичных пространств,
где есть возможность высказывания несовпадающих альтернатив [23]. Правда, при
этом такое понимание несколько не совпадает с ее же представлением о медиа как
о проводнике гегемонии.

В другом своем интервью она говорит [24]: «В демократии никто не может тотально
оккупировать место власти. Тоталитаризм, наоборот, всегда является попыткой
сделать это и стараться снова занимать это место. Конечно, пустое место власти
всегда кем-то занято, в противном случае у вас не будет политического порядка.
Но реальная разница между демократическим и тоталитарным режимами лежит в том,
что при демократии эта власть всегда временна и может быть оспорена».
