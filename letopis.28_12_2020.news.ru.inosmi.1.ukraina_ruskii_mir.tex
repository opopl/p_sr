% vim: keymap=russian-jcukenwin
%%beginhead 
 
%%file 28_12_2020.news.ru.inosmi.1.ukraina_ruskii_mir
%%parent 28_12_2020
 
%%url https://inosmi.ru/politic/20201228/248847471.html
 
%%author 
%%author_id 
%%author_url 
 
%%tags 
%%title Гордон (Украина): Украина никогда не откажется от оккупированных Крыма и Донбасса — Кулеба
 
%%endhead 
 
\subsection{Гордон (Украина): Украина никогда не откажется от оккупированных Крыма и Донбасса — Кулеба}
\label{sec:28_12_2020.news.ru.inosmi.1.ukraina_ruskii_mir}
\Purl{https://inosmi.ru/politic/20201228/248847471.html}


\ifcmt
  pic https://cdn2.img.inosmi.ru/images/24307/25/243072523.jpg
\fi

\begin{leftbar}
  \begingroup
    \em\Large\bfseries\color{blue}
В интервью немецкому изданию Internationale Politik министр иностранных дел
Украины Дмитрий Кулеба заявил, что разрыв его страны с «русским миром»
состоялся, и что он окончателен. При обсуждении ситуации на Донбассе он был так
же категоричен в формулировках.
  \endgroup
\end{leftbar}

От оккупированных Крыма и части Донбасса Украина никогда не откажется. Об этом
министр иностранных дел Украины Дмитрий Кулеба заявил немецкому изданию
Internationale Politik в интервью, которое было опубликовано 28 декабря.

Речь об этом зашла при обсуждении ситуации на Донбассе и вероятности «мирного
соглашения между Россией и Украиной».

Кулеба объяснил, что в данном вопросе Украина руководствуется двумя
основополагающими принципами: «никогда не откажется от Крыма или Донбасса» и
«продолжит свой путь как часть Запада». Объясняя второй принцип, дипломат
подчеркнул, что разрыв Украины с «русским миром» окончателен.

По словам главы МИД, «статус регионов, которые сейчас временно находятся под
оккупацией, мыслим только в рамках Украины».

В 2014 году Россия оккупировала Крым и начала вооруженную агрессию на востоке
Украины. Боевые действия ведутся между Вооруженными силами Украины (ВСУ) с
одной стороны и российской армией и поддерживаемыми Россией боевиками, которые
контролируют часть Донецкой и Луганской областей, с другой. Официально РФ не
признает своего вторжения в Украину, несмотря на предъявляемые Украиной факты и
доказательства.
