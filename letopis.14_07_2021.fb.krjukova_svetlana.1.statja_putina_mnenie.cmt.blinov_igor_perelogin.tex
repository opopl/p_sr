% vim: keymap=russian-jcukenwin
%%beginhead 
 
%%file 14_07_2021.fb.krjukova_svetlana.1.statja_putina_mnenie.cmt.blinov_igor_perelogin
%%parent 14_07_2021.fb.krjukova_svetlana.1.statja_putina_mnenie.cmt
 
%%url 
 
%%author 
%%author_id 
%%author_url 
 
%%tags 
%%title 
 
%%endhead 
\paragraph{Блинов - Игорь, перелогинься))))}
\label{sec:14_07_2021.fb.krjukova_svetlana.1.statja_putina_mnenie.cmt.blinov_igor_perelogin}

\begin{itemize}
%%%fbauth
%%%fbauth_id
%%%fbauth_tags
%%%fbauth_place
%%%fbauth_name
\iusr{Андрей Блинов}
%%%fbauth_front
%%%fbauth_desc
%%%fbauth_url
%%%fbauth_pic
%%%fbauth_pic portrait
%%%fbauth_pic background
%%%fbauth_pic other
%%%endfbauth
 

Игорь, перелогинься))))

Предлагаю наедине внимательно почитать текст Договоров и Постановлений и Прав
Войска Запорожского Орлика 1710 года (ту что называют конституцией в широких
кругах). И найти там и про народ малороссийский, и про потомков козарских, и
язык изложения изучить, и имя автора тоже. Много выводов неожиданных можно
сделать, если читать исторические документы в оригинале

\begin{itemize}
%%%fbauth
%%%fbauth_id
%%%fbauth_tags
%%%fbauth_place
%%%fbauth_name
\iusr{Алексей Беликов}
%%%fbauth_front
%%%fbauth_desc
%%%fbauth_url
%%%fbauth_pic
%%%fbauth_pic portrait
%%%fbauth_pic background
%%%fbauth_pic other
%%%endfbauth
 
\textbf{Андрей Блинов} они же в переводе на современную мову читают...а там это жестко вычищено)

%%%fbauth
%%%fbauth_id
%%%fbauth_tags
%%%fbauth_place
%%%fbauth_name
\iusr{Natasha Semergey}
%%%fbauth_front
%%%fbauth_desc
%%%fbauth_url
%%%fbauth_pic
%%%fbauth_pic portrait
%%%fbauth_pic background
%%%fbauth_pic other
%%%endfbauth
 
\textbf{Андрей Блинов} что-то мне подсказывает что Игорь думает иначе чем Светлана )))

%%%fbauth
%%%fbauth_id
%%%fbauth_tags
%%%fbauth_place
%%%fbauth_name
\iusr{Роман Шевченко}
%%%fbauth_front
%%%fbauth_desc
%%%fbauth_url
%%%fbauth_pic
%%%fbauth_pic portrait
%%%fbauth_pic background
%%%fbauth_pic other
%%%endfbauth
 
Света писать только может, читать не её конёк.

%%%fbauth
%%%fbauth_id
%%%fbauth_tags
%%%fbauth_place
%%%fbauth_name
\iusr{Ольга Попова}
%%%fbauth_front
%%%fbauth_desc
%%%fbauth_url
%%%fbauth_pic
%%%fbauth_pic portrait
%%%fbauth_pic background
%%%fbauth_pic other
%%%endfbauth
 
\textbf{Роман Шевченко} чукча не читатель - чукча писатель!

%%%fbauth
%%%fbauth_id
%%%fbauth_tags
%%%fbauth_place
%%%fbauth_name
\iusr{Нюра Берг}
%%%fbauth_front
%%%fbauth_desc
%%%fbauth_url
%%%fbauth_pic
%%%fbauth_pic portrait
%%%fbauth_pic background
%%%fbauth_pic other
%%%endfbauth
 

\textbf{Роман Шевченко} так она и пишет с кучей грамматических ошибок

%%%fbauth
%%%fbauth_id
%%%fbauth_tags
%%%fbauth_place
%%%fbauth_name
\iusr{Кирилл Семенов}
%%%fbauth_front
%%%fbauth_desc
%%%fbauth_url
%%%fbauth_pic
%%%fbauth_pic portrait
%%%fbauth_pic background
%%%fbauth_pic other
%%%endfbauth
 
\textbf{Алексей Беликов} Неа, в том-то и дело, что "Малая Россия" и "народ малороссийский" даже на оф. сайте украинского министерства.

%%%fbauth
%%%fbauth_id
%%%fbauth_tags
%%%fbauth_place
%%%fbauth_name
\iusr{Алекс Єрмоленко}
%%%fbauth_front
%%%fbauth_desc
%%%fbauth_url
%%%fbauth_pic
%%%fbauth_pic portrait
%%%fbauth_pic background
%%%fbauth_pic other
%%%endfbauth
 
\textbf{Андрей Блинов} Ну, это же не значит, что в Конституциях Филлипа (а не Пилипа) Орлика российский язык. Это язык Острожской академии 16 века, куда Российская империя пришла только через 200 лет и явно не могла принести этот язык. Это "украинский язык" с той разницей, что тогда не было слова "украинский", а язык был

%%%fbauth
%%%fbauth_id
%%%fbauth_tags
%%%fbauth_place
%%%fbauth_name
\iusr{Сергей Перевозчиков}
%%%fbauth_front
%%%fbauth_desc
%%%fbauth_url
%%%fbauth_pic
%%%fbauth_pic portrait
%%%fbauth_pic background
%%%fbauth_pic other
%%%endfbauth
 
\textbf{Андрей Блинов} и про государство московское, Вы пропустили. Про Россию там ничего не написано.

%%%fbauth
%%%fbauth_id
%%%fbauth_tags
%%%fbauth_place
%%%fbauth_name
\iusr{Євген Лихошерстов}
%%%fbauth_front
%%%fbauth_desc
%%%fbauth_url
%%%fbauth_pic
%%%fbauth_pic portrait
%%%fbauth_pic background
%%%fbauth_pic other
%%%endfbauth
 
\textbf{Андрей Блинов} Язык изложения - староукраинский. Но не только на это нужно обратить внимание, но и на суть исторического описания, где сказано о том, что Московское государство нарушило свой договор с Хмельницким, которым гарантировала козацкому народу вольности, а вместо этого пыталась надеть ярмо на казаков и малопоссийский народ, записывая козацкие города в московские области.

%%%fbauth
%%%fbauth_id
%%%fbauth_tags
%%%fbauth_place
%%%fbauth_name
\iusr{Андрей Блинов}
%%%fbauth_front
%%%fbauth_desc
%%%fbauth_url
%%%fbauth_pic
%%%fbauth_pic portrait
%%%fbauth_pic background
%%%fbauth_pic other
%%%endfbauth
 

\textbf{Алекс Єрмоленко} фраза "российский язык" обычно выдаёт автора с головой. Российская империя, которая появилась в 1721 году, никак не могла придумать язык, это очень дешёвое передёргивание. Дискуссия о том, какой это язык - староукраинский, старорусский, новохозарский, руська мова - очень дискуссионный, как назвать одно и то же явление. О формировании разных языков на землях Руси после татаро-монгольского нашествия написано куча трактатов. Изучив эпистолярный труд 300-летней давности, можно увидеть, насколько и в чём тот "украинский" отличен от современного украинского и современного русского

%%%fbauth
%%%fbauth_id
%%%fbauth_tags
%%%fbauth_place
%%%fbauth_name
\iusr{Алекс Єрмоленко}
%%%fbauth_front
%%%fbauth_desc
%%%fbauth_url
%%%fbauth_pic
%%%fbauth_pic portrait
%%%fbauth_pic background
%%%fbauth_pic other
%%%endfbauth
 
\textbf{Андрей Блинов} 

Ну, вы же возразили автору, что в Конституции Орлика не украинский язык. Мое
возражение, что это язык, который был в ходу именно в Украине.

Вот еще один любопытный документ.

\enquote{Книжка Собраниє вещей нужнейших}, Острог, 1580 год. Написано ну совсем как
по-русски, говоря по современному. Но какой это язык?

Этот язык в ходу в Украине. То есть допустимо называть его "украинским", ну,
или староукраинским, чтобы не путать с современным значением этого термина

%%%fbauth
%%%fbauth_id
%%%fbauth_tags
%%%fbauth_place
%%%fbauth_name
\iusr{Петр Сидоров}
%%%fbauth_front
%%%fbauth_desc
%%%fbauth_url
%%%fbauth_pic
%%%fbauth_pic portrait
%%%fbauth_pic background
%%%fbauth_pic other
%%%endfbauth
 
\textbf{Алекс Єрмоленко}

Русь под колоссальным влиянием Византии стала называться на византийский манер
– Россия.

Украина – это окраина России и Польши, это лоскутное одеяло из осколков России,
Польши, Венгрии и Румынии, сшитое воедино при большевиках. Украина – это
Малороссия и Новороссия. 80\% Украины занимала половецкая степь Дешт-и-Кипчак.

Русь стала мощным Царством в 1547 и великой Империей в 1721. Russia – это Русь
на латыни. В Германии Россию называют «страна руссов» – Russland. Тогда как в
Латвии мы Krievija (по славянскому племени кривичи).

Россия оформилась как государство в 862 году с призванием варягов. И с тех пор
никогда не теряла своей государственности и не была частью чужих государств.

С 862 по 1598 Россией правила древнерусская династия Рюриковичей, начавшаяся в
Новгороде и пресекшаяся в Москве.

Кочевники ордынцы никогда не жили в лесистой Московской Руси, Орда была очень
далеко от Москвы в степях к югу в районе современной Астрахани, Калмыкии,
Казахстана и Дикого Поля (Украина). Потомки Золотой Орды это тюркские и
мусульманские народы Средней Азии.

*В 1862 году в Новгороде был торжественно установлен грандиозный памятник
«Тысячелетие России», популярная городская достопримечательность.*

Новгородцы гордятся своими истоками Руси и русской государственности, и
являются неотъемлемой частью русского народа.

%%%fbauth
%%%fbauth_id
%%%fbauth_tags
%%%fbauth_place
%%%fbauth_name
\iusr{Алекс Єрмоленко}
%%%fbauth_front
%%%fbauth_desc
%%%fbauth_url
%%%fbauth_pic
%%%fbauth_pic portrait
%%%fbauth_pic background
%%%fbauth_pic other
%%%endfbauth
 
\textbf{Петр Сидоров} 

Это все байки. В русском языке невозможен переход ни Русь в Россию, ни
наоборот. Про "мощное Царство" 1547 года расскажите королю третьеразрядной
Швеции Юхану, который с 5 тысячным войском похоронил это "мощное Царство",
отняв у нее Нарву и все балтийское побережье, которое новгородцы удерживали 600
лет.

Но если вы претендуете на эрудита, ответьте на вопрос, из какого языка слово
"Русь"


%%%fbauth
%%%fbauth_id
%%%fbauth_tags
%%%fbauth_place
%%%fbauth_name
\iusr{Денис Бонд}
%%%fbauth_front
%%%fbauth_desc
%%%fbauth_url
%%%fbauth_pic
%%%fbauth_pic portrait
%%%fbauth_pic background
%%%fbauth_pic other
%%%endfbauth
 
\textbf{Євген Лихошерстов} Садись. Два!

%%%fbauth
%%%fbauth_id
%%%fbauth_tags
%%%fbauth_place
%%%fbauth_name
\iusr{Андрей Блинов}
%%%fbauth_front
%%%fbauth_desc
%%%fbauth_url
%%%fbauth_pic
%%%fbauth_pic portrait
%%%fbauth_pic background
%%%fbauth_pic other
%%%endfbauth
 

\textbf{Алекс Єрмоленко} 

"Ну, вы же возразили автору, что в Конституции Орлика не украинский язык" - вы
же журналист, укажите цитату, где именно я возразил. Или признайте, что
предпочитаете дофантазировать.

Язык книг и других письменных источников 15-18 вв. показывает эволюцию от
старославянского к языковым диалектам, которые стали основой литературного
русского и украинского языков, что было закреплено во времена Пушкина и
Котляревского-Шевченко

%%%fbauth
%%%fbauth_id
%%%fbauth_tags
%%%fbauth_place
%%%fbauth_name
\iusr{Алекс Єрмоленко}
%%%fbauth_front
%%%fbauth_desc
%%%fbauth_url
%%%fbauth_pic
%%%fbauth_pic portrait
%%%fbauth_pic background
%%%fbauth_pic other
%%%endfbauth
 
\textbf{Андрей Блинов} 

Я додумал. Сейчас перечитал еще раз, вижу, что интерпретировал неверно. Вопрос
снимается.

Правильно говорить о влиянии на эволюцию современных рус. и укр. языков
старославянского и древнерусского языка. Хотя последние многие путают, в том
числе, что удивительно, некоторые украинские историки


%%%fbauth
%%%fbauth_id
%%%fbauth_tags
%%%fbauth_place
%%%fbauth_name
\iusr{Андрей Блинов}
%%%fbauth_front
%%%fbauth_desc
%%%fbauth_url
%%%fbauth_pic
%%%fbauth_pic portrait
%%%fbauth_pic background
%%%fbauth_pic other
%%%endfbauth
 

\textbf{Алекс Єрмоленко} 

воот, благодарю за смелость признать собственное заблуждение.

И вот как получается: русини, руськи люди, русичи - это в значительной мере о
тех, кто в современном понимании является украинцами. Так почти всех и
литвинами можно считать. Хотя происхождение слова "Русь", вероятнее всего,
скандинавское.

Но в некоторый исторический момент отказались называть себя росами, русами,
предпочтя этноним "украинцы". Хотя исторически Уокраина по Ипатьевской летописи
- это действительно было южная окраина Руси, к которой можно отнести лишь
северные и северо-западные территории нынешней Украины (достаточно сказать, что
славяне южнее р.Рось не жили, продвижение на юг стало возможно лишь с
имперскими завоевательными походами, об этом забывают, включая то, как Кубань
стала с украинским, а не ногайским населением). Получается, общего таки очень
много, включая чванство, гордыню и невежество. Вот и путаемся и спорим: с какой
стороны кошерно разбивать яйцо. Вечная история про остро- и тупоконечников


\end{itemize}

\end{itemize}

