% vim: keymap=russian-jcukenwin
%%beginhead 
 
%%file slova.brat
%%parent slova
 
%%url 
 
%%author 
%%author_id 
%%author_url 
 
%%tags 
%%title 
 
%%endhead 
\chapter{Брат}
\label{sec:slova.brat}

%%%cit
%%%cit_head
%%%cit_pic
%%%cit_text
Данные заявления были не единичными. Мнение о \enquote{\emph{братских народах}} проходит
красной линией в других интервью Зеленского. Он также был уверен: говорить
по-русски и любить Украину можно. 
\enquote{Если на востоке и в Крыму люди хотят говорить по-русски, отцепитесь, отстаньте
от них. На законном основании дайте им говорить по-русски. Язык никогда не
будет делить нашу родную страну. У меня еврейская кровь, я говорю по-русски, но
я гражданин Украины и люблю эту страну. Россия и Украина – действительно
\emph{братские народы}}, - говорил Зе в 2014-м в эфире \enquote{1+1}
%%%cit_comment
%%%cit_title
\citTitle{Зеленский о русских и русском языке. Что будущий президент говорил в 2014 году}, 
Екатерина Терехова, strana.ua, 03.07.2021
%%%endcit

%%%cit
%%%cit_head
%%%cit_pic
%%%cit_text
Еще одному мужчине также без разницы, на каком языке отвечают на
пресс-конференциях спортсмены.  \enquote{Хоть бы на английском и китайском, лишь бы
друг друга понимали и друг другу желали добра}, - говорит он.  И добавляет: \enquote{Я
к россиянам вообще отношусь спокойно - как к нашим \emph{братьям}. Они для меня как
были близкие люди по крови, так и остались}
%%%cit_comment
%%%cit_title
\citTitle{Что говорят украинцы о пресс-конференциях футболистов на русском языке. Опрос Страны}, 
Антонина Белоглазова, strana.ua, 03.07.2021
%%%endcit


%%%cit
%%%cit_head
%%%cit_pic
%%%cit_text
А хорошо Владимир Александрович про двоюродных \emph{братьев}-то сказал! Как
припечатал: вы-де за тыщи километров от нашей великой истории находились и не
зазихайте тут. Если конкретнее, находились они в Великом Новгороде, который и
был первым центром власти на Руси. Именно оттуда прибыл «наш украинский князь»
Владимир, захватив киевский престол. Мокшанский оккупант он практически. Как
упустили этот момент??  Ведь и основанную им церковь, доныне здравствующую,
теперь принято называть «кремлевской» и вражеской, чуждой истинному патриоту. И
рассматривать единоверцев Владимира Святого, вышедших на Крестный ход, как
объект для резни бензопилой и вообще порицания.  Знаете, я даже испытываю
радость от непринадлежания к этому блядскому цирку. И только укрепляюсь в
нежелании иметь хоть что-то общее Вот С Этим
%%%cit_comment
%%%cit_title
\citTitle{А ведь князь Владимир, получается, по сути - мокшанский оккупант}, 
Дмитрий Заборин, strana.ua, 30.07.2021
%%%endcit


%%%cit
%%%cit_head
%%%cit_pic
%%%cit_text
Гей, \emph{брате мій, брате}! Чи ж приймуть твою душу степи, ліси й пасовиська, а чи
байдуже відторгнуть, як немилосердні заводи, вбиті хімією колишні ріки і
сповнені затаєного жаху атомні електростанції, що висять на сумному дереві
України, мов червиві груші? Жертва безтямної індустріалізації, ти лежиш тепер
мертвий у столиці нашого народу, бо ті люди, які оточували тебе, коли й
спроможні ще безтямно зачинати і продовжувати собі подібних, то вже давно
нездатні ховати своїх небіжчиків, а тільки вибурмочують над ними нікчемні,
безсоромні й цинічні чужі слова.  \emph{Брат} лежав мертвий і непохований, а мені
згадувався він тільки живий
%%%cit_comment
%%%cit_title
\citTitle{Тисячолітній Миколай}, Павло Загребельний 
%%%endcit

