% vim: keymap=russian-jcukenwin
%%beginhead 
 
%%file slova.zapad
%%parent slova
 
%%url 
 
%%author 
%%author_id 
%%author_url 
 
%%tags 
%%title 
 
%%endhead 
\chapter{Запад}

%%%cit
%%%cit_pic
%%%cit_text
А вот \emph{западные} русские, которые остались под властью Польши, долго искали свое
уникальное самоназвание, но так и не нашли. За неимением лучшего они приняли
польское определение этой части русской нации, так как, по сути, русскими уже
не являлись – ассимиляция русского населения в польской среде происходила
достаточно интенсивно. И в то время как русский язык в землях, принадлежавших
Польше, никак не развивался с древних времен и очень сильно пополнился
полонизмами и германизмами («папир», «файно», «накшталт», «кошт», «страва» и
прочее), менялось и самосознание \emph{западных русских}. Эти русские, с одной
стороны, уже забыли, что они русские, а с другой – все же не хотели становиться
поляками до конца
%%%cit_title
\citTitle{Почему современную Украину назвали «окраиной», а не более престижно - Киевской Русью?}, 
Исторический Понедельник, zen.yandex.ru, 22.02.2021 
%%%endcit

