% vim: keymap=russian-jcukenwin
%%beginhead 
 
%%file 14_11_2022.stz.news.ua.donbas24.1.ukr_mista_tematychni_proguljanky_prysv_mrpl
%%parent 14_11_2022
 
%%url https://donbas24.news/news/v-ukrayinskix-mistax-proxodyat-tematicni-progulyanki-prisvyaceni-mariupolyu
 
%%author_id demidko_olga.mariupol,news.ua.donbas24
%%date 
 
%%tags 
%%title В українських містах проходять тематичні прогулянки, присвячені Маріуполю
 
%%endhead 
 
\subsection{В українських містах проходять тематичні прогулянки, присвячені Маріуполю}
\label{sec:14_11_2022.stz.news.ua.donbas24.1.ukr_mista_tematychni_proguljanky_prysv_mrpl}
 
\Purl{https://donbas24.news/news/v-ukrayinskix-mistax-proxodyat-tematicni-progulyanki-prisvyaceni-mariupolyu}
\ifcmt
 author_begin
   author_id demidko_olga.mariupol,news.ua.donbas24
 author_end
\fi


\ii{14_11_2022.stz.news.ua.donbas24.1.ukr_mista_tematychni_proguljanky_prysv_mrpl.pic.front}
\begin{center}
  \em\color{blue}\bfseries\Large
Українці зможуть дізнатися більше про історію Маріуполя завдяки проєкту \enquote{2:40. Паралелі}
\end{center}

Трагедію, яка сталася з Маріуполем, і досі багатьом переселенцям та
переселенкам міста важко прийняти. Адже маріупольці звикли сприймати свою малу
Батьківщину, як дуже квітуче місто з низкою перспектив і можливостей. Його
\href{https://donbas24.news/news/top-5-vcililix-znakovix-istoricnix-budivel-mariupolya-foto}{\emph{унікальна історична та культурна спадщина}}%
\footnote{ТОП-5 вцілілих знакових історичних будівель Маріуполя, Яна Іванова, donbas24.news, 12.07.2022, \par\url{https://donbas24.news/news/top-5-vcililix-znakovix-istoricnix-budivel-mariupolya-foto}}
 приваблювала як місцевих мешканців, так і тих, хто вперше мав змогу завітати до міста Марії. Саме тому наразі
викликають чималий інтерес прогулянки в різних містах України з розповідями про
історичний та культурний розвиток Маріуполя.

\begin{leftbar}
\emph{\enquote{Говорити про історію Маріуполя зараз актуально, як ніколи. В першу чергу,
це дозволяє актуалізувати трагедію знищеного росіянами міста, розповідати про
нього та нагадувати про невимовний біль, який окупанти завдали всім нам}}, —
наголосила громадська діячка і маріупольська журналістка Алевтина Швецова, яка
проводила екскурсію у Львові.
\end{leftbar}

\ii{14_11_2022.stz.news.ua.donbas24.1.ukr_mista_tematychni_proguljanky_prysv_mrpl.pic.1}
\ii{insert.read_also.demidko.donbas24.kartyny_oleksandr_lukjanov}

\subsubsection{Як називається проєкт та хто виступає ініціаторами?}

У назву проєкту — \textbf{\emph{\enquote{2:40. Паралелі}}} — закладений певний символізм, адже
прогулянки-екскурсії одночасно проходять у декількох містах і починаються рівно
о 2:40 після полудня. Цифри відповідають 240-ій річниці заснування міста Марії,
яку відзначали у 2018 році.

\begin{leftbar}
\emph{\enquote{Багато хто вважає, що це був початок нового розквіту та важливих перетворень
міста — відкриття оновлених парків, створення сучасних просторів чи
започаткування нових яскравих фестивалів}}, — наголосила Алевтина Швецова.  
\end{leftbar}

\ii{14_11_2022.stz.news.ua.donbas24.1.ukr_mista_tematychni_proguljanky_prysv_mrpl.pic.2}
\ii{insert.read_also.demidko.donbas24.nimecchyna_tvorchi_vechory_oksana_stomina}

Тоді ж відбулося й офіційне відкриття культурно-туристичного центру \enquote{Вежа} у
ревіталізованій будівлі старої водонапірної вежі, яка стала візитівкою міста та
одним з найвпізнаваніших його символів.

\ii{14_11_2022.stz.news.ua.donbas24.1.ukr_mista_tematychni_proguljanky_prysv_mrpl.pic.3}

Організаторами тематичних прогулянок виступила \emph{\textbf{ГО \enquote{Вежа} та спільнота
\enquote{Маріуполь — туристичне місто}}}. Ініціатори заходу вирішили згуртувати навколо
заходу маріупольців і мешканців інших міст, які готові провести історичні
паралелі між українськими містами та знайти тотожності.

\ii{14_11_2022.stz.news.ua.donbas24.1.ukr_mista_tematychni_proguljanky_prysv_mrpl.pic.4}

\textbf{Читайте також:} \href{https://donbas24.news/news/studentka-mdu-rozpovidaje-pro-mariupol-v-centralno-jevropeiskomu-universiteti}{\emph{Студентка МДУ розповідає про Маріуполь в Центрально-Європейському університеті}}%
\footnote{Студентка МДУ розповідає про Маріуполь в Центрально-Європейському університеті, Ольга Демідко, donbas24.news, 07.11.2022, \par%
\url{https://donbas24.news/news/studentka-mdu-rozpovidaje-pro-mariupol-v-centralno-jevropeiskomu-universiteti}%
}

\subsubsection{Де вже відбулися прогулянки та які подальші плани?}

Прогулянки \enquote{2:40. Паралелі} вже пройшли у \emph{\textbf{Львові}} (гідом виступила Алевтина
Швецова), \emph{\textbf{Одесі}} (екскурсовод — куратор проєкту \enquote{ЯМаріуполь.Культура} в Одесі —
Ісидор Бадасен) та \emph{\textbf{Чернівцях}} (гід — куратор проєкту \enquote{ЯМаріуполь.Культура} в
Одесі - Наталія Арусланова). Цікаво, що до заходів долучалися як маріупольці,
так і місцеві мешканці. Під час екскурсій вони дізнавалися цікавинки про
українські міста, згадували історію Маріуполя та проводили паралелі, які
стосуються архітектурної та культурної спадщини.

Найближчим часом планується продовжити цикл екскурсій не лише в межах України,
а й за кордоном, де наразі перебувають переселенці з міста Марії.

\ii{14_11_2022.stz.news.ua.donbas24.1.ukr_mista_tematychni_proguljanky_prysv_mrpl.pic.5}

Нагадаємо, раніше Донбас24 розповідав, коли \href{https://donbas24.news/news/koli-ukrayina-poverne-mariupol-general-ssa-ben-godzes-podilivsya-prognozami}{Україна поверне Маріуполь}.%
\footnote{Коли Україна поверне Маріуполь — генерал США Бен Годжес поділився прогнозами, Еліна Прокопчук, donbas24.news, 08.11.2022, \par\url{https://donbas24.news/news/koli-ukrayina-poverne-mariupol-general-ssa-ben-godzes-podilivsya-prognozami}}

Ще більше новин та найактуальніша інформація про Донецьку та Луганську області
в нашому телеграм-каналі Донбас24.

ФОТО: з відкритих джерел

\ii{insert.author.demidko_olga}
%\ii{14_11_2022.stz.news.ua.donbas24.1.ukr_mista_tematychni_proguljanky_prysv_mrpl.txt}
