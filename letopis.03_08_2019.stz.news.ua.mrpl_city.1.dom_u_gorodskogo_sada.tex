% vim: keymap=russian-jcukenwin
%%beginhead 
 
%%file 03_08_2019.stz.news.ua.mrpl_city.1.dom_u_gorodskogo_sada
%%parent 03_08_2019
 
%%url https://mrpl.city/blogs/view/dom-u-gorodskogo-sada-starejshaya-pyatie-tazhka-v-mariupole
 
%%author_id burov_sergij.mariupol,news.ua.mrpl_city
%%date 
 
%%tags 
%%title Дом у Городского сада: старейшая пятиэтажка в Мариуполе
 
%%endhead 
 
\subsection{Дом у Городского сада: старейшая пятиэтажка в Мариуполе}
\label{sec:03_08_2019.stz.news.ua.mrpl_city.1.dom_u_gorodskogo_sada}
 
\Purl{https://mrpl.city/blogs/view/dom-u-gorodskogo-sada-starejshaya-pyatie-tazhka-v-mariupole}
\ifcmt
 author_begin
   author_id burov_sergij.mariupol,news.ua.mrpl_city
 author_end
\fi

\ii{03_08_2019.stz.news.ua.mrpl_city.1.dom_u_gorodskogo_sada.pic.1}

Если современному мариупольцу сказать, что кто-то из общих знакомых живет на
Энской улице в пятиэтажном доме и только это, то его собеседник в ответ
покрутит пальцем у виска. Действительно, в наши дни в Мариуполе имеется тысячи
пятиэтажек, так называемых \enquote{хрущевок}, и не только \enquote{хрущевок}
во всех районах города. Иное дело - до и в первое десятилетие после окончания
Отечественной войны. Достаточно было сказать: \emph{\enquote{Он живет в пятиэтажке
около сквера}}. Значит – это дом №45 на проспекте Республики. Если говорили
\emph{\enquote{Ее квартира в пятиэтажке у Горсада}}, было ясно, что некую
обладательницу квартиры можно найти в доме №1 по улице Энгельса. Дом,
построенный рядом с Городским садом, сразу бросается в глаза. И не удивительно
– ведь его окружают одноэтажные строения. Приходилось слышать, что одним из
авторов его проекта был молодой архитектор Петр Теслер.

\textbf{Читайте также:} 

\href{https://mrpl.city/news/view/v-serdtse-mariupolya-raspolozhilsya-tsentr-sovremennogo-iskusstva-foto-1}{%
В \enquote{сердце} Мариуполя расположился центр современного искусства, Олена Онєгіна, mrpl.city, 29.07.2019}

Многое рассказал об этом заметном здании \textbf{Георгий Сергеевич Котельников}. Его
детство и молодые годы прошли в нем. По его воспоминаниям, дом был заселен в
1940 году. Именно в этом году он пошел в первый класс школы, и в этом же году
его родители и, естественно и он с ними, вселились в одну из квартир этого
дома. Георгий Сергеевич рассказал, что дом был построен за счет
коксохимического завода. Неудивительно, что среди первых жильцов был директор
завода Александр Андреевич Сталев и главный инженер предприятия Сергей
Борисович Котельников с семьями. Нетрудно догадаться, что главный инженер -
отец Георгия Сергеевича, поделившегося историей дома. Все квартиры этой
пятиэтажки были заселены семьями работников Коксохима, кроме нескольких,
предоставленных сотрудникам НКВД.

Некоторые особенности дома. Он отапливался кочегаркой, находящейся в подвале.
Топливо – каменный уголь. Газ к дому не был подведен. Поэтому в квартирах на
кухнях были сложены печки для приготовления пищи и подогрева воды. Уголь для
печек брали в кочегарке. Газ к дому подвели в послевоенные годы. А в опустевшей
кочегарке, естественно после ремонта, устроили детскую игровую комнату. Дом был
спроектирован и построен в духе нового социалистического быта. В нем было
немало многокомнатных квартир, рассчитанных на проживание нескольких семей.
Общими были кухни, сантехническое оборудование.

Дом у Городского сада представлял собой автономное образование. Он имел свои
магазины. Располагал даже собственной телефонной станцией.

В нем был клуб, где находилась библиотека, комната с бильярдным столом,
комнатушкой для художника, а также зал для проведения общих собраний. Георгий
Сергеевич рассказал, что в этом зале в 44 году проходил суд над пособниками
оккупантов. За домом построили двухэтажный детский сад для детей работников
Коксохимического завода. Сейчас это здание отделено от дома коксохимиков
забором. Да и детского сада давно нет.

\textbf{Читайте также:} 

\href{https://archive.org/details/27_07_2019.sergij_burov.mrpl_city.letnii_teatr_v_mariupole}{%
Летний театр в Мариуполе, Сергей Буров, mrpl.city, 27.07.2019}

В первом подъезде есть лифт. Один из краеведов утверждал, что именно в этом
доме поселили Гитлера во время его пребывания в Мариуполе, поскольку это был
единственный в городе дом с лифтом. Георгий Сергеевич этот миф развеял – до
войны лифта в доме не было. Он был установлен не ранее 1956 года, скорее всего
в 60-х годах. О том, что во время оккупации в этом доме жили немцы, нет
сомнений. Люди, поселившиеся после освобождения города в квартиры дома,
обнаружили следы пребывания немецкой солдатни. Это были обрывки газет на
немецком языке, элементы радиоустройств и тому подобное. Но кто именно там жил?

Ответ на этот вопрос нашла историк, авторитетная исследовательница \textbf{Валентина
Михайловна Зиновьева}, работая в архивах, она установила следующее. В доме №1
квартировал летный состав истребителей, бомбардировщиков и морской авиации. В
разное время здесь находились штаб 1-й моторизованной дивизии СС Йозефа (Зеппа)
Дитриха, части, которой захватили Мариуполь 8 октября 1941 года. А также штаб
командующего 1 танковой армии генерал-полковника Эвальда фон Клейста.

С тех пор много воды утекло в Кальмиусе. Дряхлеет дом. Впрочем, чему
удивляться? Ведь в будущем году ему исполнится восемьдесят лет. Сейчас
большинство балконов превращено в дополнительные комнатушки.

Помнится, в 40-50-е годы прошлого века балконы дома становились дополнительные
\enquote{трибунами} стадиона. На них толпились жильцы дома. И часто оттуда неслось:
\enquote{Судью на мыло!}, \enquote{Мазила!}, и при удаче родной команды дружное:
\enquote{Го-о-о-о-о-л!}

\textbf{Читайте также:} 

\href{https://mrpl.city/blogs/view/rekonstruktsiya-istorichnoi-budivli-dityachoi-hirurgii-v-mariupoli-vihodit-na-novij-riven}{%
Реконструкція історичної будівлі дитячої хірургії в Маріуполі виходить на новий рівень, Павло Кириленко, mrpl.city, 24.07.2019}

\clearpage
