% vim: keymap=russian-jcukenwin
%%beginhead 
 
%%file 26_12_2021.fb.fb_group.story_kiev_ua.2.istoria_kievljanina_glava_12_ljonja.cmt
%%parent 26_12_2021.fb.fb_group.story_kiev_ua.2.istoria_kievljanina_glava_12_ljonja
 
%%url 
 
%%author_id 
%%date 
 
%%tags 
%%title 
 
%%endhead 
\zzSecCmt

\begin{itemize} % {
\iusr{Володимир Цибульський}
Надо было таки дать по морде фулиганам!

\begin{itemize} % {
\iusr{Сергей Кабыш}
\textbf{Володимир Цибульський} Что прошло, то прошло.

\iusr{Tatiana Pani}
\textbf{Володимир Цибульський} На все воля Божа.

\iusr{Володимир Цибульський}
\textbf{Tatiana Pani} А я досі жалію...

\iusr{Tatiana Pani}
\textbf{Володимир Цибульський} Знову ж таки, на все воля Божа.
\end{itemize} % }


\iusr{Chyzhyshyn Oksana}
Ох, ці скульптури зі сторінок шкільних підручників, які потім зустрічаєш в Луврі, в історичному центрі Риму.

Вдивляєшся, впізнаєш, посміхаєшся:

- Так от які ви, вже й не надіялась зустрітись )

\iusr{Раиса Карчевская}

Очень интересно. Такая встреча с другом детства по прошествии такого
времени. Написано очень хорошо и читается на одном дыхании. Спасибо большое


\iusr{Светлана Гурневич}
Дружбаны @igg{fbicon.wink} 

\ifcmt
  ig https://i2.paste.pics/b39519a13f2938cb8d4d100dd6a5404d.png
  @width 0.1
\fi

\iusr{Naumova Svitlana}

Не зачепило, вибачте. Просто історія, яких тисячі, без чогось особливого і без
художнього смаку. Немає чому, навіть, ні повчитися, ні позаздрити. Я прожила
нелегалом в Берліні 4 місяці, боялась лишній раз на очі поліцейському
потрапити, і потім... 23 роки легально. Я знаю, що це таке. Про яку медицину
він розповідає? Вона йому недоступна! А про податки, то, взагалі атас! Знайшов
чим здивувати, вся Україна так живе і зараз, все роблять, щоб неплатити. В цій
групі Київські історії є, по справжньому достойні художньо цікаві розповіді.
Вибачте.

\begin{itemize} % {
\iusr{Татьяна Оксаненко}
\textbf{Naumova Svitlana} Извините, но Вы очень категоричны и резки.. У Сергея целый цикл автобиографичных рассказов. Есть, очень удачные, ироничные. Как говорят, на вкус и цвет....

\iusr{Naumova Svitlana}
\textbf{Татьяна Оксаненко} Можливо...інші не читала. Я вибачилась до і після.

\iusr{Сергей Кабыш}
\textbf{Naumova Svitlana} Дякую, нема за що вибачатися. У цій повісті я нічого не вигадую, а кожен сприймає на свій смак.
\end{itemize} % }

\iusr{Татьяна Оксаненко}

Мне интересны Ваши воспоминания, Сергей. Тем более что, в те, годы очень многие
так поднимались(в плане достатка), так и слетали вниз. А, то, что все скучают
по Киеву, это чистейшая правда. Мы все родом из детства и чем старше
становимся, тем больше сентиментальности


\iusr{Леонид Климчук}
Мой тебе совет.....

Пообщайся ( проконсультируйся) с Нашим общим другом Рамусем Сергеем, он
заслужанный журналист, один из первых редакторов и ведущих телепрограммы
милицейской хроники \enquote{Именем закона} и \enquote{Надзвичайнi новини}

Он бы тебе многое объяснил и помог в твоём творчестве....

\iusr{Леонид Климчук}

Серёга ну ты и \enquote{писсака}....

Ведь писать о других, на добно учтиво и объективно...

А ты к сожелению, пишешь крайне субъективно... С тонким налётом неготивщины...
в адрес тех, о ком повествуешь...

Это как то некорректно, писать о старом друге (Леониде), предворительно не
согласовав с ним отдельные моменты твоего повествования.

Не хорошо....

Как-то не по дружеский....

\begin{itemize} % {
\iusr{Сергей Кабыш}
\textbf{Леонид Климчук} Эта глава была написана ранее и согласована с Леонидом. Его дословный коммент: Не всё так, но пойдёт. После этого опубликована, а сейчас - повтор без изменений, но с фото.
\end{itemize} % }


\iusr{Леонид Климчук}
А вцелом, написано как то пресновато и слишком наресованно....
Себя ты выделяешь \enquote{белым и пушистым} а всех остальных в тёмных тонах...
И тем самым ты себя выделяешь как яркий индивид....
Для собстенного настольного дневника, где описываешь летопись своей личной жизни. Это нормально....
А для публикации....
Увы, не неочень то....
Ведь можешь и лучше....

\iusr{Пётр Долгих}

\ifcmt
  ig https://i2.paste.pics/71bc51a26eb14fb415f1ae58d0d007a7.png
  @width 0.2
\fi

\iusr{Lara Ilich}

мне нравятся такие личные заметки, не отредактированные и неотцензурированные
для печати. Они больше дают для понимания времени и людей, чем блестящие эссе
журналистов.

Приехал в Киев из северной страны друг детства, юности. Вспоминаем нашу
развесёлую компанию, которая заселила весь белый свет от Канады до Австралии
(через Европу и Африку). Я с грустью замечаю, что одна из всех тут осталась,
непонятно что сторожу. А он мне \enquote{зато ты ДОМА}... И с такой печалью он
это сказал. Причём, все люди способные, творческие, очень успешные
(материально), реализовавшиеся. НО молодость въелась, наверное, навсегда.


\iusr{Елизавета Бушеленко}
Спасибо!

\iusr{Тамара Ар}
Интересно, а техникум какой был в начале поста? Не политехникум ли связи?

\begin{itemize} % {
\iusr{Сергей Кабыш}
\textbf{Тамара Ар} Именно он, на Леонтовича.

\begin{itemize} % {
\iusr{Тамара Ар}
\textbf{Сергей Кабыш} интересно, а группа какая?

\iusr{Сергей Кабыш}
\textbf{Тамара Ар} 8-ЛКСС-64 б

\iusr{Тамара Ар}
\textbf{Сергей Кабыш} это после 8 класса,,,,,,, тогда Вас приветствует сейчас группа РС! ,,, В моем лице

\iusr{Тамара Ар}

\ifcmt
  ig https://scontent-frx5-2.xx.fbcdn.net/v/t39.1997-6/s168x128/70051830_2383482495080781_6192462346766516224_n.png?_nc_cat=1&ccb=1-5&_nc_sid=ac3552&_nc_ohc=Os76yTWyi_8AX84Atyx&tn=lCYVFeHcTIAFcAzi&_nc_ht=scontent-frx5-2.xx&oh=00_AT-BvGM-L4ZRRq7Jy4NpIL0aSrePDTR71vp6FT_Pz6iDXg&oe=61CF4627
  @width 0.1
\fi

\iusr{Сергей Кабыш}
\textbf{Тамара Ар} Взаимно. Про Политехникум предыдущая глава (11). Удачи!

\end{itemize} % }

\iusr{Тамара Ар}
\textbf{Сергей Кабыш} до встречи! Приятно вспомнить классные годы учебы!

\end{itemize} % }

\iusr{Людмила Охтень}
А где почитать главу 11. Я тоже там училась

\iusr{Светлана Манилова}
\textbf{Людмила Охтень} 

\url{https://www.facebook.com/groups/story.kiev.ua/posts/1822140554649404}

\end{itemize} % }
