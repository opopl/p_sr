% vim: keymap=russian-jcukenwin
%%beginhead 
 
%%file 22_11_2020.fb.volga_vasilii.1.v_chem_sila_brat
%%parent 22_11_2020
 
%%url https://www.facebook.com/Vasiliy.volga/posts/2758945254422889
 
%%author Волга, Василий Александрович
%%author_id volga_vasilii
%%author_url 
 
%%tags 
%%title В ЧЕМ СИЛА, БРАТ?
 
%%endhead 
 
\subsection{В чем Сила, Брат?}
\label{sec:22_11_2020.fb.volga_vasilii.1.v_chem_sila_brat}
\Purl{https://www.facebook.com/Vasiliy.volga/posts/2758945254422889}
\ifcmt
	author_begin
   author_id volga_vasilii
	author_end
\fi

\ifcmt
pic https://scontent.fiev6-1.fna.fbcdn.net/v/t1.0-9/126438156_2758961437754604_7233665846828496724_n.jpg?_nc_cat=108&ccb=2&_nc_sid=8bfeb9&_nc_ohc=QjjuQNRuHpMAX-DiMU8&_nc_ht=scontent.fiev6-1.fna&oh=7a56f46eec8a36ed6fa2d3ec93bbd9b4&oe=5FEBCB40
caption В чем Сила, Брат?
\fi

\obeycr
Написал я этот рассказ четыре года назад. Мне напомнил его сегодня один из моих друзей-подписчиков. Тогда это было еще очень больно. Боль была острая, но была надежда. 
Сегодня боль стала тише, притупилась.
Но и надежда ослабла, почти исчезла.
И все же я хочу напомнить о той боли. Об остром её ощущении. 
Ещё я хочу напомнить о героях, которые погибли в самой настоящей войне за Родину.
Рассказ называется

В ЧЕМ СИЛА, БРАТ?

 «До сих пор я нахожусь под впечатлением того разговора. Уже вторые сутки. Не второй день, а именно вторые сутки. Я думаю, вспоминаю, слышу интонации, ударения, глубокие паузы того разговора. Даже снится мне, что я задаю своему собеседнику вопросы, которые я так и не задал, но очень хотел задать. 
Знал я Петра Васильевича шапочно. Т.е. раньше мы встречались либо в рабочей обстановке, либо за праздничным столом, когда его, как и меня, приглашали в гости к нашим совместным знакомым. Но ни он меня, ни я его на Дни Рождения никогда не звали, и даже телефонами мы обменялись случайно. 
Петру Васильевичу 50. Родом он из Донецка. Был он когда-то военным, потом бизнесменом, потом Народным Депутатом, потом министром, сегодня он люстрированный безработный. Живет в Киеве. Причем, в Киеве живет давно. С 1994 года. 
Был у Петра Васильевича брат. Он не был так успешен, как Петр Васильевич. Всю свою жизнь от рождения и до своей смерти он прожил в Донецке. Работал врачом. Звезд с неба не хватал, но хирургом был настоящим. Звали его Иваном. Я его видел всего один только раз. По случаю. 
Было это в 2004-м году в Донецке. Вернее в Ясиноватой. Случилось мне с ним играть в бильярд. Сильно он играл. С азартом. Широко. Предложил на деньги. Я отказался. Он пошутил, но как-то классно пошутил. По-доброму. Мы крепко тогда выпили. Именно вовремя застолья выяснилось, что он брат Петра Васильевича. Был он на пять  лет младше его, и когда мы с ним познакомились, было ему 33 года.
Больше я его не видел, но в памяти осталось его открытое лицо, широкая улыбка, громкий ясный голос и редкое ощущение настоящести. Настоящий человек! Так говорят о таких людях. 
Две недели назад он погиб. Он ушел добровольцем в ополчение еще в Славянск. Сначала был просто рядовым бойцом. Хотел воевать, а не лечить. Но война оказалась кровавой, и хирурги в ней стали на вес золота. Да и стрелок из него был никудышный. Зрение. Определили Ивана в полевой госпиталь, так он и остался по медицинской части. 
Обстоятельств его гибели я не знаю. Родители сообщили Петру Васильевичу только то, что Иван погиб. Отец ему позвонил. Впервые за два года. Просто сказал, что Ивана убили, и положил трубку. Они уже два года не общались. После того, как братья – Иван с Петром – сильно поругались, родители приняли сторону Ивана и отец, резкий и решительный, хоть уже и старый, человек, бывший шахтер, запретил Петру звонить и приезжать. Петр Васильевич рассказал мне, что о новостях он узнавал только от матери, когда та, уходя к соседке, звонила с ее телефона, чтобы отец не знал. Отец, как и Иван, считал Петра предателем. Шкурой, предавшей Донбасс. 
- Василий, скажи мне, ты ведь тоже с Донбасса родом, неужели мы и вправду с тобой сволочи? – спросил меня Петр, когда мы, по его просьбе, встретились попить кофе, и когда пять минут пустого разговора прошли. 
Причем, на «ты» мы с ним не были никогда.
- Петр Васильевич, что-то случилось? – ответил я, не говоря ни «вы», ни «ты». 
- Случилось. Иван погиб…- и Петр рассказал мене о звонке своего отца. Мать так и не позвонила ему за эти две недели. Он звонил соседке почти каждый день, но та, ответив ему два раза, сказала, что обязательно передаст матери о его звонке и перестала поднимать трубку. Петр собрался ехать в Донецк. Перед поездкой решил встретиться со мной. 
- Погиб Иван, - продолжал Петр Васильевич. Говорил он сдержано, по-мужски, без дрожи в голосе и очень основательно. – Знаешь, Василий, никак вот не могу взять в толк – а как это все произошло? Как случилось так, что все мы, обычные простые люди, встали по разные стороны фронта. И как это вообще стало возможно, чтобы фронт? А? Ведь фронт! Ведь убиваем мы друг друга. Ведь Ваньку-то убил кто-то из Киева, или из Житомира, или из Днепра. Василий! Как это возможно?! Как это стало возможно?
Он помолчал. Говорил он тихо, но сильно и пронзительно, так, что люди в кафе стали оборачиваться на нас. Петр оглянулся вокруг, опустил голову, зашатал ею из стороны в сторону и оттуда, из глубины своих тяжелых мыслей, снизу вверх, посмотрел на меня.
- Ведь врагами мы расстались, Василий, - продолжил Петр. – Ванька сказал мне тогда, что я Родину свою променял на бабки и на домашний уют, сказал, что сволочь я и баба. Причем при жене своей сказал мне это. При сыне своем. И сын-то его, Димка, так ведь тогда смотрел на меня. Жестко. С презрением смотрел…
Я молчал. Петр тоже замолчал. Так прошло несколько минут. Они не были тяжелыми, они были настоящими – эти минуты. Очень и очень сильно я чувствовал то, что было на душе у Петра. 
Через несколько минут Петр Васильевич продолжил:
- Я не верю в эту войну. Ванька верил. И вот в чем дело, Василий. Ванька сказал мне, что я трус и именно поэтому я придумал себе сказочку про то, что война эта не имеет смысла и, что только дураки идут воевать. Слышишь? Вот ведь как. Вот ведь сволочь какая. Ведь зачем он так сказал? Ведь не надо было так. А?
Он опять замолчал. 
- Петр Васильевич, зачем ты ко мне-то? – спросил я у него.
Он посмотрел на меня, отпил свой кофе и ответил:
- Скажи мне, Василий, неужели Ванька прав?
- Не знаю, - ответил я ему. – Не к тому ты пришел, Петр Васильевич, с таким вопросом. Одно я могу сказать тебе совершенно точно: Ванька твой – герой. Вот это истина, не требующая доказательств. 
Я еще хотел сказать что-то, и спросить что-то хотел, но Петр Васильевич вдруг закрыл глаза ладонью и крепко, так что желваки вспухли, стиснул зубы. 
Когда он убрал ладонь, глаза были совершенно сухие, но мне показалось, что обозлились они как-то. 
- Спасибо, Василий, - сказал он. Поднялся, одел свою куртку, пожал мне руку и ушел. 
До сих пор в душе моей тот разговор колобродит. И вопрос его я уже сотни раз за эти двое суток себе задавал: "Ведь Ваньку-то убил кто-то из Киева, или из Житомира, или из Днепра. Как это возможно?! Как это стало возможно?".
Не вернется Петро обратно в Украину. В Донецке останется».

ПС: Петр вернулся в Киев. Отец не принял. Сказал ему, что если хочет остаться
на Донбассе, то может идти в ополчение. Петр не захотел.
\restorecr

\paragraph{Комментарии}

\begin{itemize}
\item \textbf{Езерская Анна}

Я тоже напишу в тему небольшой рассказ. У моей дочери есть подруга. Ее отец
родом из Горловки. Майданутый на всю голову(был года 4 назад). Мать братья все
в Горловки. Решил он к ним с подарками съездить. Пробирался какими-то окольными
путями. Приехал, а мать даже не посмотрела в его сторону, а братья выгнали
взашей. Он приехал в Киев, напился вусмерть и плакал весь вечер. Но так и не
понял, за что они с ним так. Пытался потом звонить, так они просто трубку не
берут.

\item \textbf{Svetlana Kuka}

Сильный рассказ, второй раз его читаю у вас, В.А. Не дай Бог такого раскола в
семье, в стране. Мне кажется, что отец отказал сыну Петру в возвращении, по-
мужски отрезал ему все концы, а может стоило попробовать и уж тогда, видя его
слабость и предательство, вычеркнуть из семьи, хотя Тарас Бульба тоже ведь не
рассуждал.

\item \textbf{Ульяна Комракова}
Вот рассказ Елены Лавровой, до слез просто. Перекликается с Вашим рассказом.\Furl{https://proza.ru/2016/10/18/750?fbclid=IwAR1PvvhZrUxqpiK2qbA5oOR0BK_VB4I-HBlbm9UXgNeCSlhzKiOgY8CgPB4}

\item \textbf{Vera Akopova}
Это даже представить невозможно. Как будто Москва с Челябинском воевала. Но на
Украине это случилось. Западенцы конечно другие. Я 20 лет там
прожила. Но убийцами своих стали в итоге люди, не отличающиеся
ничем от них. Вот в чем ужас(

\item \textbf{Мила Биожо}
Для меня тоже дикость, когда если бы Краснодар, Краснодарский край, воевал со
Ставрополем, Москвой, Саратовым, Омском...Один и тот же народ.
Это как возможно такое? Убивать их только за то, что они были
против «революции»?

А все проще.

Нацистам, которых на Украине много, сказали, что на Донбассе - русские. Вот их
и надо убивать. За это платили, награждали, обещали земли.

Вообще, с развалом Союза, вытянули на свет всю антирусскую гнусь. Их
сознательно отбросили в их холопское мышление, в котором они
прибывали столетиями: вечные батраки поляков, австрияк...

Большую роль играют евреи, бывшие пионеры, комсомольцы и прочие вожаки, которые
также активно меняли окрас, политику, если речь заходила о
русских.

Все наши большевики - евреи. Кто приехал из Англии, Америки, Франции, из
местечек...только с одной целью - уничтожение славян,
гражданская братоубийственная война, а проще, геноцид.

\end{itemize}
