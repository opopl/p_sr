% vim: keymap=russian-jcukenwin
%%beginhead 
 
%%file link.29_11_2021.bbc.1.golodna_tusa
%%parent 01_12_2021.fb.uljanov_anatolij.1.istoria_pro_zhabu_i_gadjuku.cmt
 
%%url 
 
%%author_id 
%%date 
 
%%tags 
%%title 
 
%%endhead 

\href{https://www.bbc.com/ukrainian/news-59461911}{%
"Голодна туса". Соцмережі розлютила вечірка блогерів у День пам'яті Голодомору, bbc.com, 29.11.2021%
}

\begin{multicols}{2}
\ifcmt
  ig https://ichef.bbci.co.uk/news/800/cpsprodpb/C794/production/_121829015_a58fe3ab-8cb4-4845-9a3f-0d3dce4c57aa.jpg
  @width 0.4
\fi

Соцмережі обурила гучна вечірка з нагоди дня народження блогера Олександра
Заліска, яку він влаштував у Львові 27 листопада, коли в Україні відзначали
День пам'яті жертв Голодоморів.

Сама вечірка відбулась у стилі, що передбачає відверті конкурси, еротичні танці
від шоу-балету, а численні гості виголошували тости імениннику, стоячи на
столах.

Чернівецький блогер Олександр Заліско, за чиїм життям стежать мільйони
українців у Instagram, запросив до себе на свято багато інших популярних
блогерів, серед яких і мільйонники.

Ті, хто отримав запрошення, називали цю вечірку "тусовкою року" і ділилися
враженнями у сторіз. Оскільки сторіз - це відео, які існують лише добу, всі
вони вже зникли з Instagram, однак їх встигли заскрінити журналісти.

Розважав гостей стендап-комік та популярний телеведучий Володимир Остапчук. В
одній зі своїх сторіз він опублікував фото святкового столу, яке підписав
"голодна блогерська туса".

\ifcmt
  ig https://ichef.bbci.co.uk/news/549/cpsprodpb/7974/production/_121829013_089a3f74-195e-4374-bd4f-9995705501f4.jpg
  @width 0.3
\fi

\uzr{\enquote{Сьогодні головна тусовка року}}

Олександр Заліско, чий день народження припадає на 27 листопада, зібрав
популярних блогерів у Будинку вчених, архітектурній пам'ятці в центрі Львова.

"Один зі змістів (мова оригіналу - Ред.) мого життя - це туса. Я радий, що ви
прийшли на тусу. Мою тусу. І я сподіваюся, що вона справді буде грандіозною", -
сказав Заліско у сторіз на початку вечірки.

Журналісти помітили, що на вечірку прийшли багато Instagram-зірок: Наталі
Литвин, Таня Самбурська, Валерія Юрченко, Саша Бо. У з цих блогерок - від 1 до
2 мільйонів підписників.

"Сьогодні головна тусовка року, - написала Литвин. - Олександр Заліско,
обожнюємо твої вечірки!"

Блогерка опублікувала сторіз із конкурсами на святі. На одному з них гостей
шмагали батогом.

\uzr{\enquote{Срамота}}

У соцмережах вечірку відразу ж почали критикувати. Ось кілька коментарів із
дописів блогерів:

"Жах! Як у день пам'яті жертв Голодомору можна влаштовувати таке "дійство"? -
написала користувачка Олеся Данилишин - Невже абсолютно немає людських
цінностей".

"Срамота, стид і позор веселитися в таку чорну дату, що ви несете в маси? Від
вас такого не очікувала, західноукраїнські патріоти, хіба так вас виховували
ваші батьки та як вони на це дивляться?", - написала інша користувачка, Ірина
Дрогомирецька.

Однак більшість із цих коментарів уже не знайти в Instagram, оскільки сторіз -
це відео, які існують в Instagram лише добу, а можливість коментувати фото з
вечірки запрошені блогери і Заліско закрили.

Деякі користувачі запитували, як сталося так, що в таку жалобну дату стало
можливим орендувати зал під день народження.

Українська письменниця та видавець Юлія Сливка помітила, що лідери думок мало
згадували про Голодомор у день скорботи, але натомість не посоромились
викладати подібний контент.

"Здається, ви не розумієте, про що йдеться, - емоційно відреагувала Сливка. -
Ви могли просто не народитись. Не було б розваг! Нас з вами могло не бути".

Було і чимало людей, які підтримували Заліска, вітали його зі святом, та
відзначали хорошу організацію вечірки.

\uzr{\enquote{Я не обирав день, коли мені народитись}}

Попри гостру реакцію, наступного дня іменинник Заліско залишився задоволений
вечіркою, а критику в свій бік вважає лицемірством.

"Настільки швидко піднялась ця тема в інтернеті та наскільки швидко всі стали
правильними", - написав блогер у себе в сторіз.

Він відзначив, що у скорботний день працювали ресторани, телебачення, люди
святкували весілля та слухали музику.

Заліско пише, що поважає людей, які "справді" вшановують пам'ять жертв
Голодомору, та не хизуючись цим у соцмережах, ставлять свічку.

"Але таких меншість", - вважає блогер.

"27 листопада - це мій день народження! І я святкую свій день! Я не обирав
день, коли мені народитись. І я святкую його так, як захочу!" - емоційно
відреагував Заліско.

За його словами, він щороку святкує свій день народження в останню суботу
листопада, і досі це нікого не хвилювало.

\uzr{\enquote{Немає контрактів на рекламу - немає грошей}}

\ifcmt
  ig https://ichef.bbci.co.uk/news/720/cpsprodpb/CC02/production/_121862225_d45f112f-2a5f-40e7-bf8b-afcd05b57521.jpg
  @width 0.4
\fi

Така реакція ще дужче розпалила соцмережі. На інцидент у своєму фейсбуці уже
відреагувала Анна Герич, головна спеціалістка з охорони культурної спадщини
Львівської облради.

На її думку, популярні блогери повинні відповідальніше поводитись у соцмережах,
бо "з їхніх думок народжуються тренди", молодь вірить у їхні поради та
висновки.

"От чим відрізняються ті наші сусіди з-за поребрика, що заперечують Голодомор,
і ті, що глумляться над пам'яттю своїх предків, мільйонів дітей померлих від
голоду?" - запитує Герич.

У чиновниці також є питання до поліції, яка не завадила вечірці відбутися у
Львові, який досі належить до"червоної зони", а тому проводити масштабні події
там заборонено.

"(Поліція - Ред.)і далі не бачитиме? До тих, хто здав Будинок вчених, а завтра
скаже, що "ну, ми ж не перевіряємо, що там будуть робити", - написала Герич.

Після вечірки, у понеділок 29 листопада з'явилось повідомлення, що через
порушення правил карантину Будинок вчених тимчасово закрили.

Однак журналіст Олександр Рудоманов вважає, що блогери "розуміють лише
мотивацію грошей, а не слів".

"Немає контрактів на рекламу - немає грошей", - додав він у своєму фейсбуці,
чим започаткував новий флешмоб.

У списку роботодавців блогерів є багато відомих брендів.

Після скандалу з "голодною тусою" користувачі соцмереж йдуть на сторінки цих
брендів і вимагають у них відповіді: чи будуть компанії надалі співпрацювати з
Заліско та його гостями, і чи поділяють їхню позицію щодо святкувань у День
пам'яті жертв Голодоморів?

Наразі жодна з компаній офіційно свою позицію не повідомила. Але у коментарях
Pepsi Co Ukraine зазначила, що офіційно закінчила співпрацю з блогером Заліско
ще у жовтні 2021 року.

\end{multicols}

