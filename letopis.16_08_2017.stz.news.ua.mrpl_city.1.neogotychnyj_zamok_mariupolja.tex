% vim: keymap=russian-jcukenwin
%%beginhead 
 
%%file 16_08_2017.stz.news.ua.mrpl_city.1.neogotychnyj_zamok_mariupolja
%%parent 16_08_2017
 
%%url https://mrpl.city/blogs/view/neogotichnij-zamok-mariupolya
 
%%author_id demidko_olga.mariupol,news.ua.mrpl_city
%%date 
 
%%tags 
%%title Неоготичний замок Маріуполя
 
%%endhead 
 
\subsection{Неоготичний замок Маріуполя}
\label{sec:16_08_2017.stz.news.ua.mrpl_city.1.neogotychnyj_zamok_mariupolja}
 
\Purl{https://mrpl.city/blogs/view/neogotichnij-zamok-mariupolya}
\ifcmt
 author_begin
   author_id demidko_olga.mariupol,news.ua.mrpl_city
 author_end
\fi

\ii{16_08_2017.stz.news.ua.mrpl_city.1.neogotychnyj_zamok_mariupolja.pic.1}

Якщо міста порівнювати з людьми, то Харків – це друг, з яким хочеться
спілкуватися і проводити час. Полтава – це симпатична дівчина, яку хочеться
любити і плекати. А Маріуполь – це такий знайомий, який потребує уваги,
спілкування, якому потрібні любов і розуміння... Місто, як і людина, має ім'я,
вік, свою історію виникнення, літопис подій, культуру, статус, свою сім'ю, в
яку входять не тільки розташовані в ньому будинки, вулиці, а й ми, його
мешканці...

Як багато ви знаєте про місто, в якому живете? Чи є у вас улюблені місця, з
якими пов'язано багато спогадів? Переконана, що є! Але, що саме вам відомо про
історію Маріуполя, його вулиці, будинки, які зберігають свої таємниці і
легенди?

Кожне місто, безумовно, унікальне і має власні здобутки. Від того, як  ми
ставимося до своєї історії, залежить і подальший розвиток нашого міста.
Водночас не кожне місто може похвалитися настільки багатим минулим, яке має
Маріуполь.

\ii{16_08_2017.stz.news.ua.mrpl_city.1.neogotychnyj_zamok_mariupolja.pic.2}

Уявімо, що вас попросили назвати найефектнішу споруду, розташовану в Маріуполі.
Яку будівлю ви би назвали? Думаю, що справедливим було б відзначити будинок
Ґампера – яскраву архітектурну пам'ятку міста, що була побудована в 1898 р.
Невеликий будинок у стилі неоготики, створений з червоної цегли, розташований в
районі Слобідки. Відмінні особливості архітектури: стрілчасті вікна, цегляні
візерунки і прибудова у вигляді вежі середньовічного замку. Чи всі маріупольці
знають, що на рубежі XIX – XX ст. тут жив і працював відомий маріупольський
лікар Сергій Федорович Ґампер, який користувався великою популярністю і серед
мешканців міста?!

\ii{16_08_2017.stz.news.ua.mrpl_city.1.neogotychnyj_zamok_mariupolja.pic.3}

Сергій Федорович народився в 1859 році в Маріуполі в родині лікаря, який займав
посаду судово-медичного експерта, потім був лікарем і викладачем Маріупольської
чоловічої гімназії. Він був людиною цілеспрямованою і старанною, наполегливо
йшов до своєї мети – продовжити справу батька – працювати лікарем. Середню і
вищу освіту Сергій Ґампер здобув у Харкові: після закінчення гімназії він
вступив на медичний факультет Харківського університету. Завершивши навчання в
університеті, тільки через два роки з дипломом лікаря приїхав на Батьківщину –
до Маріуполя і зайняв посаду свого батька.

Однак неабиякі розумові здібності, амбітність не втримали Сергія Федоровича в
Маріуполі. Через невеликий проміжок часу молодий лікар відправляється в
Петербург на кафедру терапії Військово-медичної академії. Там протягом двох
років виконує велику науково-дослідницьку роботу, яка стала основою його
докторської дисертації. До речі, маріупольці можуть пишатися, що їхньому
земляку на захисті дисертації опонував майбутній великий фізіолог, Нобелівський
лауреат і академік Іван Петрович Павлов. Для нас є важливим той факт, що,
незважаючи на запрошення залишитися в столиці і зайняти професорську посаду,
Сергій Федорович повертається до рідного міста. За своє відносно коротке життя
Сергій Ґампер дуже багато зробив для Маріуполя і як високоосвічений лікар, і як
громадський діяч. Доктор Ґампер працював два десятиліття з невеликим, аж до
своєї смерті, яка настала на початку 1911 року. До цього сумного моменту він
був головним лікарем Маріупольської повітової земської лікарні.

\ii{16_08_2017.stz.news.ua.mrpl_city.1.neogotychnyj_zamok_mariupolja.pic.4}

Вдячні маріупольці поважали та любили Сергія Федоровича за його енергійність,
щирість та небайдужість. Спуск, що веде до його будинку, мешканці міста назвали
докторським або Ґамперовським, який і сьогодні зберіг цю назву.

У радянські часи в будинку Ґампера розміщувалися комунальні квартири.

Будівля нагадує замок, чим і привертає увагу до себе. За розповідями
старожилів, що передавалися з покоління в покоління, у минулому перед будинком
був фонтан, у якому плавали лебеді, і дві дорожки вели до берега моря, а поруч
з фонтаном росла освячена верба, посаджена самим Ґампером.

\ii{16_08_2017.stz.news.ua.mrpl_city.1.neogotychnyj_zamok_mariupolja.pic.5}

Пізніше було добудовано південну частину будинку, а до нашого часу збереглися
лише старі дерев'яні сходи та металевий бак для води на горищі, під башенкою.
Навпроти знаходиться старий дерев'яний будинок, якому також більше 100 років
(вул. Земська, 34). Купець і відомий у нашому місті домовласник І. Найдьонов
подарував його для Гоголівського початкового училища. Встановлена мешканцями
невеличка табличка свідчить про те, що попечителем училища був теж доктор
Сергій Федорович Ґампер, який жив навпроти.

Сьогодні будинок Ґампера – дійсно унікальна, рідкісна та красива споруда, де
мешкає декілька сімей, – потребує в захисті, реконструкції та збереженні. Сліди
руйнування і занедбаності помітні всюди. Від того наскільки серйозно ми
поставимося до збереження нашої історичної спадщини, безумовно, залежить і
подальший розвиток міста.

Сподіваюся, що зі мною погодяться, якщо я скажу, що Маріуполь – молоде і
самобутнє місто в Україні, зі своєю унікальною історією та неповторною життєвою
атмосферою, може здивувати власними здобутками не тільки гостей, але й власних
мешканців. І ми вже давно повинні почати пишатися, навчитися цінувати, а
головне – любити рідний Маріуполь.

\emph{Наступного разу ви дізнаєтеся більше про будинок Регіра, який є найбільшим
особняком Маріуполя.}
