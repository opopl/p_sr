% vim: keymap=russian-jcukenwin
%%beginhead 
 
%%file 28_01_2022.fb.fb_group.story_kiev_ua.1.kiev_visim_sekretiv.6.kamjanycja_kiev_vijta
%%parent 28_01_2022.fb.fb_group.story_kiev_ua.1.kiev_visim_sekretiv
 
%%url 
 
%%author_id 
%%date 
 
%%tags 
%%title 
 
%%endhead 
\subsubsection{6. Музей «КАМ’ЯНИЦЯ КИЇВСЬКОГО ВІЙТА»}

Знаходиться по вул. Костянтинівська 6/8, виконаний в стилі українського бароко,
є прекрасним прикладом житлової архітектури Києва межі XVII-XVIII ст.

Належав київській купецькій та козацькій родині Биковських, двоє з яких
залишились в історії міста як війти. 

\ii{28_01_2022.fb.fb_group.story_kiev_ua.1.kiev_visim_sekretiv.6.kamjanycja_kiev_vijta.scr.1}

Війти - це очільники місцевого самоврядування, що пов’язані із тогочасною
традицією України, Литви, Польщі і Німеччини. Виходила ця традиція із
магдебурзького права, що з кінця XII ст. виникло в Німеччині, а з кінця XIII
ст. активно поширювалося нашими землями.

\ii{28_01_2022.fb.fb_group.story_kiev_ua.1.kiev_visim_sekretiv.6.kamjanycja_kiev_vijta.scr.2}

Щоб якось забути і замовчати цю багатовікову києво-європейську традицію
місцевого самоврядування, яка уславлювала самоорганізацію і незалежність від
королів та царів, за часів імперії був вигаданий міф про тз. “будинок ПєтраІ”.
Хоча цей міф давно спростований, жодного джерела про це не знайдено, а ПєтроІ
ніколи в ньому і не бував, але лише зараз вдалося нарешті позбутись цього
неправдивого кліше. І ми тут маємо яскравий приклад як імперія і совєти
намагались переписувати історію. Проте, правда завжди поверне своє. Як і
відбулось з історією цієї садиби і типової подільської історії війтів. 

\ii{28_01_2022.fb.fb_group.story_kiev_ua.1.kiev_visim_sekretiv.6.kamjanycja_kiev_vijta.scr.3}

Повернемось краще до неї: Київський магістрат викупив будівлю у Биковських з
метою використання у громадських потребах. Значної шкоди і запустіння споруда
зазнала після пожежі 1811 року, яка охопила більшу частину Києво-Подолу.  

Зараз тут розміщено чудовий музей історії міста. 

\ii{28_01_2022.fb.fb_group.story_kiev_ua.1.kiev_visim_sekretiv.6.kamjanycja_kiev_vijta.scr.4}

Цікаві музейні предмети, які доповюють києвознавчий і культурний досвід новими
фактами. Музей обіцяє, що: \enquote{Ви відчуєте, як голосом однієї споруди промовляє
ціле місто, що завжди прагло європейських стандартів життя на рівні освіти,
доброхотства, міського врядування і побуту}. 

\ii{28_01_2022.fb.fb_group.story_kiev_ua.1.kiev_visim_sekretiv.6.kamjanycja_kiev_vijta.scr.5}

Найбільш цінне тут те, що цінують справжніх київських героїв і меценатів, які
зберегли для нас старий добрий Київ: 

Остафія Дашковича (в 2021 році нарешті в столиці з’явилась вулиця імені цього
видатного героя нашої історії), Костянтина і Василя Острозьких (про яких я
чимало писав статей, які відновили багато київських святинь своїм коштом), мого
улюбленого Савви Туптала (портрет якого я повернув в Кирилівську церкву), а
також і легендарного гетьмана Івана Мазепи (який, певно, найбільше вклався у
відновлення і реставрацію київських будівель і збереження обличчя нашого
міста)...

\ii{28_01_2022.fb.fb_group.story_kiev_ua.1.kiev_visim_sekretiv.6.kamjanycja_kiev_vijta.scr.6}

Я зайшов в музей з дитиною, а тому не мав можливості належно вивчити всі
експонати. Але оцінив рівень гостинності, коли з Марком активно спілкувались
працівники, займали його, доки я вичитував тексти і намагався роздивитися як
найбільше (але точно зрозумів, що ще повернусь і не раз). 

А коли ми
разговорились і наглядачки дізнались, що ми з Марком самі фарбували та
доглядали за пам’ятником Магдебурзькому праву (велике фото якого при виході з
музею), то одна з них кудись зникла, за кілька секунд з’явилась у супроводі
директриси, вказала на нас: \enquote{оце вони!}  @igg{fbicon.smile}  Пані Оксана, попросила нас
розповісти більше. Коротко це зробили. Запросили описати нашу історію і
заповнити анкету. Було незручно, але ми дуже поспішали і відмовились. Я лишив
всі свої дані, щоб зі мною могли зв’язатись пізніше. Тоді вона побігла на
кухню, дістала велику шоколадку і подякувала Марку, сказавши, що вони дуже
цінують таких київських меценатів, яким не все одно до нашої історії і
пам’яток. Така була мі-мі-мішна сцена  @igg{fbicon.smile}  Марк не зрозумів, чому за цікаве
фарбування пам’ятника ще щось і дають, але був радий  @igg{fbicon.smile}  

