% vim: keymap=russian-jcukenwin
%%beginhead 
 
%%file 16_11_2020.fb.tanya_naumchuk.1.pchelka_pritcha
%%parent 16_11_2020
 
%%url https://www.facebook.com/groups/1057886447663671/permalink/3467954729990152/
%%author Наумчук, Таня
%%author_id tanya_naumchuk
%%tags 
%%title 
 
%%endhead 

\subsection{Пчелка (Притча)}
\Purl{https://www.facebook.com/groups/1057886447663671/permalink/3467954729990152/}
\Pauthor{Наумчук, Таня}
\index[writers.rus]{Наумчук, Таня!Пчелка (Притча)}

\ifcmt
pic https://scontent-waw1-1.xx.fbcdn.net/v/t1.0-9/125777819_788472078677682_194066816861703071_o.jpg?_nc_cat=103&ccb=2&_nc_sid=825194&_nc_ohc=ihU3Fw1vAbcAX8gOg69&_nc_ht=scontent-waw1-1.xx&oh=e415ba7e7275ace48fb78fcb4fcb411a&oe=5FDF531D
caption Пчелка (Притча)
\fi

Когда Он создал ее, то увидел, что это одно самое полезное из Своих меньших творений. Господь сказал ей:

--- Я наделил тебя самыми лучшими качествами. Ты будешь очень трудолюбивая. С
самой ранней весны до поздней осени неустанно будешь трудиться, а плоды твоих
трудов будут ценными и необходимыми для Человека. Мед будет питать и
услаждать его, а так же лечить его недуги. Я покажу тебе цветок, нектар
которого добавлений в мед, сделает его неуязвимым для порчи. Тебя всегда
будет привлекать все только прекрасное и благоухающее. Всякая нечистота и
мерзость будут для  тебя отвратительны. Даже отходы твоего труда будут
применяться Человеком на пользу. Из воска будут делать свечи. В Моих храмах
их будут возжигать и их огонёк Я увижу на Верху. Это будет знак, что ты также
трудишься для Меня. Ты, как и все, будешь смертна. Но и после смерти тебя
будут использовать Люди для лечения. Даже твой яд Я наделил целебной силой.
Ты сама --- одна ЛЮБОВЬ! Нет в тебе изъяна.

Услышав такое о себе от Творца, пчёлка так обрадовалась, что  от гордости
за себя вздернула высоко головку и принялась танцевать в счастливом ритме.
Увидев это, Господь опечалился: \enquote{И тут нет смирения...}

--- Но, чтобы ты не возомнила о себе лишнего, от ныне твоя голова всегда будет
опущена, а танцевать ты будешь только на цветкак во время сбора пыльцы и
нектара. И, что бы ты была в смирении, Я наделил тебя жалом, но не как других
для защиты, а только к смерти! Если же ты кого обидешь (ужалишь) --- УМРЕШЬ!

С тех пор пчёлка всегда была в смирении и послушании пред Творцом. И, даже,
если случалось ей кого-либо обидеть, она умирала во смирении...

12.11.2020 г. (Наумчук Татьяна).
