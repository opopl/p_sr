%%beginhead 
 
%%file 04_07_2023.fb.mariupol.muzej.mkm.1.skorlupin_evgen_petrovich
%%parent 04_07_2023
 
%%url https://www.facebook.com/100093184796939/posts/pfbid02J4QNqbP6jGYNDydfR4hXojSQqKrz1d9QxtUAcTzjG8Y3hiaaLMMzh1VGMk4zf7BAl
 
%%author_id mariupol.muzej.mkm
%%date 04_07_2023
 
%%tags 
%%title Скорлупін Євген Петрович
 
%%endhead 

\subsection{Скорлупін Євген Петрович}
\label{sec:04_07_2023.fb.mariupol.muzej.mkm.1.skorlupin_evgen_petrovich}

\Purl{https://www.facebook.com/100093184796939/posts/pfbid02J4QNqbP6jGYNDydfR4hXojSQqKrz1d9QxtUAcTzjG8Y3hiaaLMMzh1VGMk4zf7BAl}
\ifcmt
 author_begin
   author_id mariupol.muzej.mkm
 author_end
\fi

Цим дописом маріупольські музеї починають цикл розповідей про долю
маріупольських художників у часи широкомасштабної російської війни проти
України. І перший матеріал  - це трагічна історія майстра станкового і
монументально-декоративного мистецтва, заслуженого художника України Скорлупіна
Євгена Петровича.

\ifcmt
  ig https://scontent-fra3-1.xx.fbcdn.net/v/t39.30808-6/347567636_135799809536178_2250425298738796654_n.jpg?_nc_cat=110&ccb=1-7&_nc_sid=730e14&_nc_ohc=6-IKjO-hpP8AX-SEv71&_nc_ht=scontent-fra3-1.xx&oh=00_AfCTFfunEL7fKXd1uZAnOm8ihLRVWCAbz4ChqayObvjwbg&oe=64A9FECC
  @wrap center
  @width 0.9
\fi

Все своє творче життя  Євген Петрович провів у Маріуполі. Працював у
виробничо-художніх майстернях міста, приймав участь у виставках і пленерах,
створював скульптури для зон відпочинку і дитячих оздоровчих центрів. В історії
культури Маріуполя відомий ще й тим, що був одним з ініціаторів  проведення у
місті виставок пейзажного живопису \enquote{Меморіали Куїнджі}.  Шанувальники
мистецтва любили  його картини за доступність художньої мови, цікаві
колористичні рішення, особливу емоційність його пейзажів. Безумовно, він був
яскравим представником реалістичного живопису. В колекції Маріупольського
краєзнавчого музею зберігалися картини, подаровані Євгеном Петровичем. Вони
мали не тільки художню, але і історичну цінність, тому що зображали
маріупольський порт, підприємства міста у  70-ті, 80-ті, 90-ті роки XX
сторіччя. В останні роки життя Євген Петрович переніс інсульт, але продовжував
працювати, був частим гостем у художньому музеї. Все змінилося 24 лютого 2022
року. 

\ifcmt
  ig https://scontent-fra3-1.xx.fbcdn.net/v/t39.30808-6/347565290_135799939536165_7083465836806583212_n.jpg?_nc_cat=110&ccb=1-7&_nc_sid=730e14&_nc_ohc=6_-viTpzhQcAX9Srs73&_nc_ht=scontent-fra3-1.xx&oh=00_AfBJFuXsK-gNmY16cg0h8mERewBMIXZgOabmAyGUvxWFzQ&oe=64AA00D0
  @wrap center
  @width 0.9
\fi

Маріупольці пам'ятають, якими були для всіх кінець лютого -  березень  2022
року.  У холодній, тому що внаслідок обстрілів у квартирі вже не було вцілілих
вікон,  без достатньої їжі і води, без можливості  здобуття  медичної допомоги,
Євген Петрович помер.  Це сталося у час найстрашніших авіобомбодувань нашого
міста 16 березня 2022 року. Його вдова Ольга Дмитрієвна Скорлупіна (багато
маріупольців добре пам'ятають їі як прекрасного педагога маріупольської
художньої школи ім. А. І. Куїнджі) прийняла рішення залишитися з чоловіком у
квартирі. Можливості захоронити  Євгена Петровича не було, у під'їзді Ольга
Дмитрієвна  залишилося майже одна. Через два дні окупанти (у той час ця частина
міста  вже була занята ворогом) вигнали ї ї з квартири.  Тіло Євгена Петровича
залишилося там. А ще через деякий час, внаслідок прямого попадання, у будинку
сталася страшна пожежа, згоріла і квартира. 

\ifcmt
  ig https://scontent-fra3-1.xx.fbcdn.net/v/t39.30808-6/347584705_135800029536156_3576263666264704444_n.jpg?_nc_cat=103&ccb=1-7&_nc_sid=730e14&_nc_ohc=8VX5qclGNF8AX9cGS28&_nc_oc=AQnEYjsbFX8TQF_hZea0Q9xegDYNhOSSYR5Laks2ev8ILb1vgjxlslofFBmk6IcSIYk&_nc_ht=scontent-fra3-1.xx&oh=00_AfDWsejvAp1lUEkeAQRPc9enPbnpRHi1XQYiw7TXTJInMw&oe=64A93E0C
  @wrap center
  @width 0.9
\fi

Про це дуже важко писати. Останки Євгена Петровича ще довгий час пролежали у
квартирі і тільки восени були захороненні родичами біля могили матері у селищі
Пещане біля Маріуполя.  У вогні загинули і картини Євгена Петровича, які
зберігалися у зібранні музею.  Можна сказати, що митець розділив долю своїх
картин....

\ifcmt
  ig https://scontent-fra5-2.xx.fbcdn.net/v/t39.30808-6/347419368_135800372869455_6438382689634695977_n.jpg?_nc_cat=106&ccb=1-7&_nc_sid=730e14&_nc_ohc=1RNUIikxLV4AX8YEn6L&_nc_ht=scontent-fra5-2.xx&oh=00_AfD1RwwUWUq34mF8V9PE3sbIwL3Zm7iisE6l5LG_HWNJzw&oe=64A8E925
	@caption_here Скорлупін Євген Петрович
  @wrap center
  @width 0.5
\fi
