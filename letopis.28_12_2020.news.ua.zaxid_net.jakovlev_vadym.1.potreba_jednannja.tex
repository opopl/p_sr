% vim: keymap=russian-jcukenwin
%%beginhead 
 
%%file 28_12_2020.news.ua.zaxid_net.jakovlev_vadym.1.potreba_jednannja
%%parent 28_12_2020
 
%%url https://zaxid.net/konkurentniy_avtoritarizm_ta_potreba_yednannya_n1512454
 
%%author 
%%author_id jakovlev_vadym
%%author_url 
 
%%tags 
%%title Конкурентний авторитаризм та потреба єднання
 
%%endhead 
 
\subsection{Конкурентний авторитаризм та потреба єднання}
\label{sec:28_12_2020.news.ua.zaxid_net.jakovlev_vadym.1.potreba_jednannja}
\Purl{https://zaxid.net/konkurentniy_avtoritarizm_ta_potreba_yednannya_n1512454}
\ifcmt
	author_begin
   author_id jakovlev_vadym
	author_end
\fi
\index[rus]{Русь!Потреба єднання, 28.12.2020}

\begin{leftbar}
  \begingroup
    \em\Large\bfseries\color{blue}
    Головна проблема українців не в роз’єднаності, а в невмінні з нею жити
  \endgroup
\end{leftbar}


\ifcmt
  pic https://zaxid.net/resources/photos/news/202012/1512454.jpg
\fi


Один український журналіст дуже обурювався гаслом передвиборчої програми Петра
Порошенка 2014 року «Єдина країна – Единая страна». Муляла йому «единая
страна», оскільки в єдиній країні він жити ладен, а в «единой стране» – ні.
Минуло відтоді багато часу. Порошенко встиг відмовитися від своєї передвиборчої
риторики і раптово в останній рік президентства перетворився на войовничого
націоналіста, а відкинуте ним «Єдина країна – Единая страна» переробив на свій
копил Зеленський і завдяки цьому переміг на виборах.

Журналіст тим часом здобув популярність, послідовно борючись з «единой
страной», зокрема зчинивши скандал, коли прогнав з ефіру власної програми
російськомовного політолога. Але і «єдина країна», і «единая страна» зависли в
повітрі як красива риторична фігура, не ставши втіленою ідеєю та послідовною
політичною програмою. І головне питання тут не так у тому, як правильно
говорити про єдність чи роз’єднаність українців – українською чи російською, як
у тому, що дискусії, в який же спосіб нам об'єднуватись, у суспільстві майже
немає. Уже два президенти прийшли до влади на об'єднавчих гаслах, але,
відмовившись від них або не змігши послідовно втілити в життя, розгубили
популярність. Хоча падіння рейтингів, поза сумнівом, має й інші причини. А
«хайпуючі» на роз'єднавчій риториці «лідери громадської думки» так і залишились
на інформаційній периферії, обслуговуючи своїми вправляннями у розпалюванні
ворожнечі абсолютну меншість українського суспільства.

Тож чи дійсно українці прагнуть об'єднання і єдності, обираючи політиків, які
пропонують «єдину країну»? Чи це просто красиве, з естетичного погляду,
популістське гасло, що заспокоює нас, змушуючи почуватися кращими в теорії, ніж
ми є направду, і яке не віддзеркалює справжніх прагнень українців?

Американський дослідник країн колишнього СРСР Лукан Вей винайшов цікавий термін
«змагальний авторитаризм» на позначення поширеного в посткомуністичних країнах
політичного режиму. Змагальний, конкурентний авторитаризм – це і не
авторитаризм, і не демократія. Він передбачає наявність працюючих демократичних
інститутів (таких як вільні вибори, плюралізм ЗМІ, легальна опозиція), які,
щоправда, використовують олігархи, права людини в країні змагального
авторитаризму не дотримуються, а політики, перемігши на виборах завдяки
демократичним процедурам, діють зазвичай цілком в авторитарному дусі, борючись
з опозицією і намагаючись взяти все під свій контроль. До типових держав
конкурентного авторитаризму належать Грузія, Молдова, Вірменія, Росія часів
Єльцина та Україна.

Однією з властивостей конкурентного авторитаризму є жорсткий поділ суспільства
на агресивно налаштовані одна до одної групи, які у своїй підтримці тих чи тих
політичних сил доходять до фанатизму та неспроможні на консенсус поміж собою.
Цей поділ підживлюють заради власної вигоди політики, використовуючи кишенькові
ЗМІ, дружніх до себе журналістів, культурних діячів та представників
громадянського суспільства.

Авжеж, теорію про змагальний авторитаризм можна залишити в рамці винятково
політичної системи, але важко не замислитися в контексті цієї ідеї про вплив
такого устрою на загальні суспільні тенденції.

За останній місяць українське суспільство (принаймні активну його частину, яка
прагне впливати на процеси в державі) сколихнуло дві теми. Перша – це новина
про популярність російського контенту серед українців на ютубі. Друга – це
скандал з головною ялинкою країни, яку спочатку прикрасили казковим капелюхом,
а потім традиційною зіркою. Ці дві теми відразу розділили посполитих на два
непримиренні табори, в кожного з яких були безкомпромісна позиція та
обов'язкова зневага до опонента.

Щойно виникає якась дискусія, українці відразу пристають на ту чи ту сторону і
вважають її позицію істиною в останній інстанції, не терплячи критики. Опонент
майже завжди демонізується, його позиція висміюється і відкидається не просто
як неправильна, а як кричущий прояв його особистих, інтелектуальних, моральних
недоліків. Чи сильно відрізняється модель поведінки політиків у режимі
змагального авторитаризму від того, як поводиться суспільство? Схоже, що не
дуже.

Цьогоріч Міністерство культури організувало серію круглих столів, покликаних
розпочати дискусію на такі болючі і гострі теми, як мова, релігія, історія та
культура. Спонукою організації цих заходів було бажання показати приклад
толерантного діалогу між представниками іноді радикально протилежних поглядів.
Треба зазначати, що один з ініціаторів та ідеологів круглих столів Олесь Доній
рік тому зазнав нападу з боку ультраправих за участь у лекції, присвяченій
толерантності. Ультраправі патріоти прозвали Донія, активного учасника всіх без
винятку демократичних революцій в Україні та борця за Незалежність, «ворогом
нації» за його бажання жити в толерантнішій країні. Між іншим, особливою
прикметою України як держави конкурентного авторитаризму є традиція величати
чомусь «активістами» та «радикально налаштованими патріотами» тих, кого в
демократичних державах за їхні дії і погляди називали б «неофашистами».
Використання цих евфемізмів розмиває межу між справжніми активістами,
прихильниками громадянського суспільства, та неофашистами. Це недопустима для
демократичних країн практика, оскільки відкритий діалог для демократії є
безумовною цінністю, а всі, хто прагне замість діалогу вчиняти насилля над
опонентами, обов’язково маргіналізуються та зацитьковуються як вороги вільного
та стабільного суспільства.

Національні круглі столи Міністерства культури не викликали активної
зацікавленості в суспільстві. Чесно кажучи, сумніваюся, що цей захід якось
сильно вплинув і на владу. А втім, час покаже. Інтелектуали сприйняли круглі
столи здебільшого критично – подекуди виправдано критично, але більшою мірою
упереджено. На моїй же пам'яті це перша за багато років масштабна дискусія,
ініційована державою, до якої долучили різних за візіями та переконаннями
науковців, митців, блогерів та громадських діячів. В одному круглому столі
зголосився взяти участь і я, оскільки такий формат спілкування вважав не те що
надпотрібним, а навіть вже перезріло важливим для України. Хоча на круглих
столах не було озвучено ворожих до української державності думок, багато хто
сприйняв їх як «малоросійську диверсію». Це вже стандартна реакція частини
суспільства, яка перебуває у своїй інформаційно-ідеологічній бульбашці, на
будь-які прояви думок «не в тренді», що загрожують монополії цієї агресивної
фанатичної меншості на «українську ідею».

Єдність українців – це наша національна ілюзорна надія, яку ми плекаємо на
кожних виборах. Єдності не може бути без культури діалогу. Режим змагального
авторитаризму, в якому країна існує майже 30 років, не навчив і не може навчити
нас культури діалогу. Та ми цього і не прагнемо. Можливо, є сенс визнати, що
українці ненавидять одне одного. Патріоти не знайдуть спільної мови з
«малоросами», російськомовні з україномовними, «бандерівці» зі «совколюбами», а
«прогресивні ліберали» з «традиціоналістами». Що ж нам тоді залишається, якщо
ми не хочемо скотитися до граничної сегрегації всередині суспільства, яка може
призвести до запиту на повзучу федералізацію? Треба визнати той факт, що ті,
кого ми ненавидимо, мають право на свою істину. Якщо для когось капелюх на
ялинці – це важливий прояв секулярних цінностей та ознака прогресу, а для
іншого – наруга над традицією, то, може, є сенс щороку ставити не одну, а дві
головні ялинки в столиці? Одну – з капелюхом, а іншу – із зіркою. Головна
проблема українців не в нашій роз’єднаності, а в нашому невмінні з нею жити.

