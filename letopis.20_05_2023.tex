%%beginhead 
 
%%file 20_05_2023
%%parent may_2023
 
%%url 
 
%%author_id 
%%date 
 
%%tags 
%%title 20-05-2023
 
%%endhead 

% topics.vojna
% topics.vojna.day.451
% topics.vojna.week.65
% week.65.3

\section{20-05-2023}
\label{sec:20_05_2023}

