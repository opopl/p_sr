% vim: keymap=russian-jcukenwin
%%beginhead 
 
%%file 14_01_2022.fb.fb_group.story_kiev_ua.3.detstvo.cmt
%%parent 14_01_2022.fb.fb_group.story_kiev_ua.3.detstvo
 
%%url 
 
%%author_id 
%%date 
 
%%tags 
%%title 
 
%%endhead 
\zzSecCmt

\begin{itemize} % {
\iusr{Rima Ternovski}
Такой душевный пост, тепло и хорошо на душе. Благодарю Вас  @igg{fbicon.hands.pray} 

\iusr{Ірина Бондарєва}
Спасибо Вам! Мне очень приятно... @igg{fbicon.smile} 

\iusr{Анна Шустерман}
Ирина, замечательно! Очень перекликается с моей, когда-то, любовью поболеть.
Спасибо за пост! Пишите ещё, у Вас хорошо получается.@igg{fbicon.heart.red}

\iusr{Ірина Бондарєва}
Благодарю! Радостно и приятно! @igg{fbicon.face.smiling.eyes.smiling}  @igg{fbicon.heart.beating} 

\iusr{Igor Popell}

Я в далеком детстве любил листать фото-альбом у бабки с дедом на Подоле. У меня
даже название для него было copyrighted \enquote{дашле}. Потому что слово \enquote{дальше} не
выговаривывываловось! Поэтому было просто \enquote{давай дашле!} )))

\iusr{Ірина Бондарєва}
Здорово!  @igg{fbicon.smile} 

\iusr{Ludmila Teslenko-ponomarenko}
так, бабусіни альбоми - то любов назавжди! Чудова розповідь, пречудова! Хворіти я теж любила

\iusr{Ірина Бондарєва}
Щиро Вам дякую! Дуже приємно! @igg{fbicon.smile} 

\iusr{Юрій Федосєєв}
Яка тепла історія

\iusr{Ірина Бондарєва}
Щиро дякую! @igg{fbicon.smile} 

\iusr{Оксана Ходорич}
Дуже ніжно і смачно про дитинство

\iusr{Ірина Бондарєва}
Дякую! @igg{fbicon.smile} 

\iusr{Bella Tarapakina}
Очень хорошо написано, с удовольствием читала. Спасибо ⚘

\iusr{Ірина Бондарєва}
Вам спасибо большое! @igg{fbicon.smile} 

\iusr{Alla Yevstratova}
Очень душевно. Но что это за дикий текст. Никто не заметил?

\iusr{Ляля Пышная}
\textbf{Alla Yevstratova} , прочитайте в оригинале. У меня тоже бывает такой шок. А
потом понимаю, что читаю автоматический перевод.

\iusr{Alla Yevstratova}
Я имею ввиду неправильный, исковерканный русский язык и дикую стилистику.

\begin{itemize} % {
\iusr{Светлана Манилова}
\textbf{Alla}, пост в группе на украинском. Видимо, Вы читали на русском.

\iusr{Павел Гурнік}
\textbf{Alla Yevstratova} Не знаєте української мови ?
А що так ?

\iusr{Наталья Владимировна Ярмоленко}
\textbf{Alla Yevstratova} Аби шо!!
\end{itemize} % }

\iusr{Алина Шиндировская}
И мне мама рассказывала, как она любила с книжечками болеть

\iusr{Мария Горбатенко}
Із великим задоволенням читаю такі щирі і добрі пости @igg{fbicon.hands.pray} 

\iusr{Ірина Бондарєва}
Щиро Вам вдячна! @igg{fbicon.smile} 

\iusr{Ivan Krakovsky}
Душевно и очень интересно!

\iusr{Ірина Бондарєва}
Благодарю! @igg{fbicon.smile} 

\iusr{Klara Mezhebovsky}
Я тоже часто болела в детстве-очень похоже, все хорошо описано

\iusr{Ірина Бондарєва}
Спасибо! @igg{fbicon.smile} 

\iusr{Ольга Писанко}
Дуже теплий нарис! Теж згадала, як під час дитячих хвороб залюбки переглядала
сімейні фото, а бабуся розповідала де і хто...

\iusr{Ірина Бондарєва}
Дуже Вам дякую! Так, всі ми, чи майже всі, дітьми так любили роздивлятися фото
рідних... затамувавши подих... @igg{fbicon.smile}  @igg{fbicon.heart.beating} 


\iusr{Ольга Скуцкая}

Я теж народилася в лікарні Калиніна на вулиці Зоологічній. Пам'ятаю жартувала,
що народилася в зоопарку)))

\begin{itemize} % {
\iusr{Ірина Бондарєва}

Чудово! Ми, бувало, ходили до лікаря, вже коли я підросла, через Пушкінський
парк... любила гуляти в ньому... ну і зоопарк, зрозуміло, теж... @igg{fbicon.smile} 


\iusr{Ольга Скуцкая}
\textbf{Ірина Бондарєва},

а ми з бабусею в зоопарк завжди йшли через кіностудію Довженка, там у садочку
паслись коні і ми їх годували печівом)


\iusr{Ірина Бондарєва}
От про коней я не знала... на жаль...
\end{itemize} % }

\iusr{Наталья Рубанская}

І я народилась в Калініна, живу все життя на Борщагівській, а педіатр схожа у
мене була, Горбатенко Ріта Антонівна ( приймала на Комінтерну). Варню не варю, а
гриби мариную

\begin{itemize} % {
\iusr{Ірина Бондарєва}
Як чудово! А в якому номері живете?

\iusr{Наталья Рубанская}
\textbf{Ірина Бондарєва} в 12

\iusr{Ірина Бондарєва}
Це ближче до проспекту Перемоги, раніше Брест-Литовський... я у 21 номері жила. Навпроти нас був 16

\iusr{Наталья Рубанская}
\textbf{Ірина Бондарєва} це над трамваєм, 16 поверховий

\iusr{Ірина Бондарєва}

Ближче до \enquote{мостіка} - так дорослі казали про будинки, що були розташовані на
початку вулиці... \enquote{Він, вона живуть біля мостіка} Там ще фабрика була,
здається \enquote{Промткач} називалася

\end{itemize} % }

\iusr{Ирина Слободенюк}

Так уютно, тепло и нежно. Плед, варенье и даже гарчее молоко, которое пью до
сих пор, когда болею. Вот и сейчас с чашкой молока (горячего с маслом, мёдом и
содой) прочла Ваш душевный рассказ. Спасибо!

\begin{itemize} % {
\iusr{Ірина Бондарєва}

Вам большое спасибо! @igg{fbicon.smile} 
Как Вы его пьете?! Мне страшно вспоминать! Выздоравливайте!

\iusr{Ирина Слободенюк}
\textbf{Ірина Бондарєва} спасибо!

\end{itemize} % }

\iusr{Tetiana Desiatka}
Щиро дякую за ваш душевний допис. Згадала і своє дитинство. @igg{fbicon.hands.pray} 

\iusr{Ірина Бондарєва}
Дякую Вам! @igg{fbicon.smile} 

\iusr{Defektolog Logoped}

Душевно. Я тоже часто болела. Бабушка натирала козьим жиром. И меня также поили
молоком с содой и мёдом. Иногда вместо коровьего молока заставляли пить козье.
Это было ужасно (козье молоко) для меня. Фото храню в семейном архиве. Подобный
врач был у моих детей. Вчера ей исполнилось 85лет. Дети мои взрослые. А я
каждый день ей желаю здоровья. И таких,как я,очень много. Общаемся до сих пор.
Спасибо за пост.

\begin{itemize} % {
\iusr{Ірина Бондарєва}

Спасибо Вам большое! Да, козье молоко это ещё страшней! Козьим жиром меня тоже
натирали и ещё скипидарной мазью  @igg{fbicon.eyes} 

\iusr{Defektolog Logoped}
\textbf{Ірина Бондарєва} 

И скипидарной мазью также меня натирали. И смесь алоэ вера с медом я пила по
столовой ложке 3раза в день.. У нас всегда рос цветок. Почему-то бабушка
называла его агава.

\iusr{Ірина Бондарєва}

У нас тоже всегда рос этот цветок, но бабушка называла его столетник. Его соком
мне закапывали нос при насморке

\iusr{Нина Светличная}
\textbf{Ірина Бондарєва} 

Так, алое тоді душе широко використовували. Це вже пізніше я дізналась, що з
ним треба бути дуже обережним - різко підскакує тиск і, взагалі, багато
\enquote{побочок}(

\end{itemize} % }

\iusr{Марина Яловенко}
Теплые воспоминания!

\begin{itemize} % {
\iusr{Ірина Бондарєва}
Спасибо!

\iusr{Марина Яловенко}
\textbf{Ірина Бондарєва} ,особенно душевно о педиатре-редко сейчас таких врачей встретишь!!! потому, что раньше шли по-призванию

\iusr{Ірина Бондарєва}
Да, правда! Это действительно было служение, а не работа...
\end{itemize} % }

\iusr{Ассоль Грей}
Як Чудово і комфортно написано! Аж тепло на душі!

\iusr{Ірина Бондарєва}
Сердечно Вам дякую! @igg{fbicon.smile} 

\iusr{Ассоль Грей}
Гарненька дівчинка на фото!)

\iusr{Ірина Бондарєва}
Дякую! @igg{fbicon.smile} 

\iusr{Татьяна Иванова}
Спасибо!

\iusr{Ірина Бондарєва}
И Вам спасибо!

\iusr{Людмила Билык}

Яка гарна дівчинка... Які чудові і розумні очі... Мене так само мама
лікувала...  @igg{fbicon.grin} ... І я, так само, коли хворіла, присвячувала
час своїм улюбленим справам - читанню і перегляданню сімейних альбомів...
@igg{fbicon.wink}  Не було телевізора, інтернета, телефона врешті - решт...
Зате було багато часу для розвитку своєї уяви і фантазування... Дякую за тепло
спогадів, яке подарували...  @igg{fbicon.heart.sparkling}  А в лікарні ім.
Калініна я народила доньку...  @igg{fbicon.smile} 

\begin{itemize} % {
\iusr{Ірина Бондарєва}

Щиро Вам дякую за добрі слова! Так, правда, для розвитку уяви були всі умови і
все тому сприяло! Телевізор був, як з'явився, дуже якийсь обмежений час на
день.

\end{itemize} % }

\iusr{Валентина Самойленко}

Які приємні і душевні спогади, все, що з дитинства, чомусь саме особливе, чи є час
споглядати, а потім життя пробігло одна метушня. Я теж і зараз люблю переглядати
світлини, то було життя, у кожного своє, а все швидко минає тільки залишаються
спогади.

\iusr{Ірина Бондарєва}
Щиро Вам дякую!
Так, дорогі серцю спогади... дорогі обличчя на фотознімках...

\iusr{Елена Соколова}

И моё детство прошло на улице Борщаговской. И родилась я в больнице Калинина. И
болеть я любила. Но не из-за опенек и вишнёвого варенья....

\begin{itemize} % {
\iusr{Ірина Бондарєва}
Замечательно!  @igg{fbicon.smile}  В каком номере Борщаговской Вы жили?

\iusr{Елена Соколова}
\textbf{Ірина Бондарєва} Борщаговская 113, кв.3.

\iusr{Ірина Бондарєва}
О, это очень далеко от нашего дома... мы жили в 21
\end{itemize} % }

\iusr{Павел Гурнік}
так, дитинство, \enquote{хвороби}, вдосталь часу на книжки, ілюстрації і альбоми музеїв
світу ... полуничне, грушкове варення ...

\iusr{Наталія Лук'янова}
дякую за розповідь

\iusr{Ірина Бондарєва}
Дякую Вам! @igg{fbicon.smile} 

\iusr{Ardzinba Eliso}

Читаешь, и сразу тепло и уютно  @igg{fbicon.hearts.two}  прекрасный рассказ,
спасибо вам  @igg{fbicon.hibiscus} 

\iusr{Ірина Бондарєва}
И Вам большое спасибо! @igg{fbicon.smile}  @igg{fbicon.heart.beating} 

\iusr{Ludmila Karpuk}
Одразу видно мрійливого художника..

\iusr{Ірина Бондарєва}
Щиро Вам дякую! @igg{fbicon.smile} 

\iusr{Олена Гаврилюк}

І моє дитинство пройшло на Борщагівській. І народилась я в лікарні Калініна.
Борщагівська мого дитинства дуже відрізняється від нинішньої. А хворіти,
мабуть, всі в дитинстві любили. Пам'ятаю, що наша дитяча поліклініка була поряд
із заводом \enquote{Більшовик}. А розглядати товсті сімейні альбоми - улюблене
заняття @igg{fbicon.face.smiling.halo} 

\begin{itemize} % {
\iusr{Лора Гончарова}
\textbf{Олена Гаврилюк} И выросла в этой поликлинике. Жила на Выборгской.

\iusr{Ірина Бондарєва}
\textbf{Olena Gavryliuk} 

Так, та Борщагівська дитинства немає нічого спільного із сьогоднішньою...то
були одноповерхові будинки, двори із садами, городами, по нашій стороні у
дворах протікала Либідь, поїзди по насипі бігли...Я жила у 21 номері, а Ви в
якому?


\iusr{Олена Гаврилюк}
\textbf{Ірина Бондарєва}, я набагато далі, у 152, біля редукторного заводу, майже на перетині з Індустріальною.

\iusr{Ірина Бондарєва}
А, зрозуміло...так, це дуже далеко
\end{itemize} % }

\iusr{Tina Marsagishvili}

Какой хороший теплый рассказ! Благодаря автора я тоже окозалась волшебный мыр
своего детство. Все было по другому и далеко отсюда но с темже волшебним мигом
детства, спасибо автору так живо все описали и заставили меня тоже окунутся в
детство, когда живи били родители, брат и сестри.


\iusr{Ірина Бондарєва}
Спасибо большое!

\iusr{Наталия Хохлова}
Дякую за теплу розповідь, і я побувала, завдяки вам, у своєму дитинстві, бвльше таких розповідей

\iusr{Ірина Бондарєва}
Щиро дякую!

\iusr{Неоніла Ганюк}
Чудові спогади!
Пишіть ще!
Чекаємо!

\ifcmt
  ig https://scontent-frx5-2.xx.fbcdn.net/v/t39.1997-6/s168x128/66699203_620859688421494_8793453631660621824_n.png?_nc_cat=1&ccb=1-5&_nc_sid=ac3552&_nc_ohc=8uAcnUPCBzkAX-OtblF&_nc_ht=scontent-frx5-2.xx&oh=00_AT8PQJgroaQ1Qo5JvPkd9IoRuX-VZ0YDRIr4QIY_EEvRYA&oe=61E8319F
  @width 0.1
\fi

\iusr{Ірина Бондарєва}
Автор
Дуже дякую! @igg{fbicon.smile} 

\iusr{Елена Ясь-Лебединская}

\ifcmt
  ig https://scontent-frx5-2.xx.fbcdn.net/v/t39.1997-6/s168x128/12532987_1062180607159155_4812529_n.png?_nc_cat=1&ccb=1-5&_nc_sid=ac3552&_nc_ohc=LuoWykyh9acAX850h9i&_nc_ht=scontent-frx5-2.xx&oh=00_AT_TxvYXNlwIXZQKyXVM6_8TCdVXvmNaWcsoKbyHSrjrMA&oe=61E7C257
  @width 0.1
\fi

\iusr{Ірина Бондарєва}

\ifcmt
  ig https://scontent-frx5-2.xx.fbcdn.net/v/t39.1997-6/s168x128/47189048_1074621272699305_1540067752134311936_n.png?_nc_cat=1&ccb=1-5&_nc_sid=ac3552&_nc_ohc=I8b2gdXLuJUAX8KMaUI&_nc_ht=scontent-frx5-2.xx&oh=00_AT8jUN4IPS-VAwoptfmN6Kii6WoelGjjKNezm2UNKKRAKQ&oe=61E73A06
  @width 0.1
\fi

\iusr{Ирина Лукьянова}

Прекрасные воспоминания, спасибо, что поделились! Но, какой ужасный перевод на
русский язык, я такого ещё не встречала.

\begin{itemize} % {
\iusr{Ірина Бондарєва}
Спасибо большое! А разве без перевода нет возможности прочитать?

\iusr{Лариса Коновалова}
\textbf{Ирина Лукьянова} ???

\iusr{Ірина Бондарєва}
Мой текст опубликован на украинском

\begin{itemize} % {
\iusr{Ирина Лукьянова}
\textbf{Ірина Бондарєва} 

Я не знаю украинского. Более того, белорусский и польский понимаю, как
оказалось, а украинский, не понимаю.

\iusr{Ірина Бондарєва}

Понятно... @igg{fbicon.smile}  Видимо, тот перевод, который предложен, действительно
ужасен... Сожалею...  В белорусском и польском очень много похожего с
украинским... да и с русским много общего... попробуйте прочесть без перевода...

\iusr{Irena Visochan}
\textbf{Ірина Бондарєва} Не просто на украинском, а еще и на Очень красивом, литературном украинском. Спасибо!

\iusr{Ірина Бондарєва}
\textbf{Irena Visochan} 

Ой, ну что Вы! Я с детства русскоязычная была многие годы... с 2014 года
постепенно перешла на рідну мову... но до красоты и грамотности ещё ооочень
далеко!

Но спасибо Вам большое! Принимаю авансом... @igg{fbicon.smile}  Очень приятно!
\end{itemize} % }

\end{itemize} % }

\iusr{Людмила Орловская}

Який теплий та душевний спомин! Дякую за приємне оповідання зранку,наче в
своєму дитинстві побувала. Будьте здорові!


\iusr{Ірина Бондарєва}
Щиро Вам дякую! І Вам здоров'ячка міцного!

\iusr{Олена Спицька}

Розповідь ця мене теж наштовхнула на спогади!

І я хворіла в дитинстві дуже часто! І мене також намагалися напоїти тим бридким
молоком із содою, маслом і медом! Але і я більше двох ложок проковтнути не
могла!

Досі згадую із здриганням!

\iusr{Ірина Бондарєва}

\ifcmt
  ig https://scontent-frt3-1.xx.fbcdn.net/v/t39.1997-6/s168x128/10574683_1520973608132736_2140492843_n.png?_nc_cat=104&ccb=1-5&_nc_sid=ac3552&_nc_ohc=blJe77QQ2AwAX_Tyi3q&_nc_ht=scontent-frt3-1.xx&oh=00_AT92cAVE4o8mypdIx7PJESg4g4xTciEPyBE0P4OfI6-h9w&oe=61E87427
  @width 0.1
\fi

\iusr{Світлана Наговіцина}
Що ж ви така красуня у мамочки. Дуже гарна дівчинка

\iusr{Ірина Бондарєва}
Ой, дякую Вам красненько! Дуже приємно  @igg{fbicon.face.smiling.eyes.smiling} 

\iusr{Lyudmila Nikitina}
Дуже гарно і затишно від Вашої статті!

\iusr{Ірина Бондарєва}
Щиро дякую і дуже рада!

\iusr{Маргарита Дуда}
Неймовірно теплий спогад

\iusr{Ірина Бондарєва}
Дякую дуже!

\iusr{Алла Конопацька}
Ви - справжеій майстер слова. Я прочитала вашу імторію з таким задоволееням!!! Пишіть ще +!!

\iusr{Ірина Бондарєва}
Щиро Вам дякую! Мені дуже приємно!

\iusr{Людмила Таркан}
Дуже цікава розповідь ! Нагадала і моє дитинство!

\iusr{Ірина Бондарєва}
Дякую Вам! @igg{fbicon.smile} 

\iusr{Регина Бочковская}
Які ж ми всі були схожі в нашому дитинстві. Щиро дякую за розповідь.

\iusr{Ірина Бондарєва}
Це правда... Дякую Вам!


\end{itemize} % }
