% vim: keymap=russian-jcukenwin
%%beginhead 
 
%%file 01_12_2021.stz.edu.lnr.lgau.1.studenty_protiv_spid
%%parent 01_12_2021
 
%%url http://lnau.su/novosti/studenty-lgau-protiv-vich-spid-my-zdorovoe-pokoleniya-nashej-respubliki
 
%%author_id 
%%date 
 
%%tags 
%%title Студенты ЛГАУ против ВИЧ/СПИД! Мы — здоровое поколения нашей Республики!
 
%%endhead 
\subsection{Студенты ЛГАУ против ВИЧ/СПИД! Мы — здоровое поколения нашей Республики!}
\label{sec:01_12_2021.stz.edu.lnr.lgau.1.studenty_protiv_spid}

\Purl{http://lnau.su/novosti/studenty-lgau-protiv-vich-spid-my-zdorovoe-pokoleniya-nashej-respubliki/}

\ii{01_12_2021.stz.edu.lnr.lgau.1.studenty_protiv_spid.pic.1}

\textSelect{Студенты Луганского государственного аграрного университета (ЛГАУ) 1 декабря
провели социальную акцию «АнтиВИЧ», приуроченную к Всемирному дню борьбы со
СПИДом. Участие в мероприятии приняли более 50 человек.}

\ii{01_12_2021.stz.edu.lnr.lgau.1.studenty_protiv_spid.pic.2}

Всемирный день борьбы с ВИЧ/СПИД отмечается ежегодно 1 декабря. В этот день во
всем мире говорят о СПИДе, о том какую угрозу существованию человечества несет
эта глобальная эпидемия, о масштабах этой трагедии.

Этот день отмечен своим символом – красной лентой, которая обозначает
поддержку, сострадание и надежды на будущее без СПИДа.

Студенты ЛГАУ обеспокоены ростом ВИЧ-инфицированных среди молодежи. В связи с
этим в университете была проведена акция, которая была направлена на повышение
уровня знаний о СПИДе.

В рамках данного мероприятия обучающиеся раздали информационные буклеты
студентам и сотрудникам университета, в которых рассказывается о всемирной
проблеме общества, что требует внимания.

Организаторами акции выступил отдел по воспитательной и социальной работе ЛГАУ
совместно со студенческим самоуправлением вуза.

\begin{zzquote}
— Идея проведения такого проекта у нас возникла не сегодня. Наш студенческий
актив  ежегодно проводит в рамках этой даты различные мероприятия, —  рассказал
студент инженерного факультета \textSelect{Артем Редькин.}
\end{zzquote}

Своими впечатлениями поделилась обучающаяся агрономического факультета \textSelect{Алина
Серебряк:}

\begin{zzquote}
— Я рада, что приняла участие в таком мероприятии. Спасибо университету,
что проводят  полезные и познавательные акции, ведь это очень важно для каждого
человека.	
\end{zzquote}

Вся акция прошла под девизом: «Студенты ЛГАУ — здоровое поколение нашей
Республики!»

Пресс-центр университета, фото Игоря Одинцова
