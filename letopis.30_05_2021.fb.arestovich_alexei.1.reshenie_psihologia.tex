% vim: keymap=russian-jcukenwin
%%beginhead 
 
%%file 30_05_2021.fb.arestovich_alexei.1.reshenie_psihologia
%%parent 30_05_2021
 
%%url https://www.facebook.com/alexey.arestovich/posts/4378241895573190
 
%%author Арестович, Алексей
%%author_id arestovich_alexei
%%author_url 
 
%%tags 
%%title Тайна «решения» заключается в том, что принятие решения - это насилие над собой
 
%%endhead 
 
\subsection{Тайна «решения» заключается в том, что принятие решения - это насилие над собой}
\label{sec:30_05_2021.fb.arestovich_alexei.1.reshenie_psihologia}
\Purl{https://www.facebook.com/alexey.arestovich/posts/4378241895573190}
\ifcmt
 author_begin
   author_id arestovich_alexei
 author_end
\fi

- Тайна «решения» заключается в том, что принятие решения - это насилие над собой.
Принимающий настоящее, сильное решение никогда не имеет полноты необходимых данных для уверенного действия. 
Значит, необходимо действовать против собственных страхов, неуверенности, сомнений, альтернатив, никогда не зная до конца, даже спустя годы, был ли ты прав или нет?..
Черчиль называл это дорогой от одной неудачи к другой, не теряя энтузиазма. 
В этом смысле люди резко делятся на три отряда:
- категорически избегающие сильных решений,
- принимающие, но не охотно, что называется, когда жизнь приперла к стенке,
- охотно принимающих решения.
Модуль поможет представителям всех троих групп:
- первые научатся тому, насколько в кайф уметь принимать решения, вести свою жизнь собственной рукой, 
- вторые научатся припирать к стенке жизнь своими сильными ходами, 
- третьи смогут получить в распоряжение более результативные алгоритмы.
Подробности в видео:
\url{https://youtu.be/u16Vb22la_g}
—————
Записаться: 
рус:
\url{http://go.apeiron.school/modul-IPR-ru-fb}
укр:
\url{http://go.apeiron.school/modul-MPR-ua-fb}
