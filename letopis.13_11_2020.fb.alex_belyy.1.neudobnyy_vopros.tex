% vim: keymap=russian-jcukenwin
%%beginhead 
 
%%file 13_11_2020.fb.alex_belyy.1.neudobnyy_vopros
%%parent 13_11_2020
 
%%url https://www.facebook.com/groups/1418036601769427/permalink/2801022390137501/
%%author 
%%tags 
%%title 
 
%%endhead 

\subsection{Неудобный вопрос}
\label{sec:13_11_2020.fb.alex_belyy.1.neudobnyy_vopros}
\Pauthor{Белый, Алекс}
\Purl{https://www.facebook.com/groups/1418036601769427/permalink/2801022390137501/}

\ifcmt
pic https://scontent.flwo4-1.fna.fbcdn.net/v/t1.0-9/125231720_181341860262327_264921840133978726_n.jpg?_nc_cat=101&ccb=2&_nc_sid=8bfeb9&_nc_ohc=-R9Lu_NNfskAX_aDIUK&_nc_ht=scontent.flwo4-1.fna&oh=7dbb9da69c610a923f7b2f355953529c&oe=5FD4CABA
\fi

Как-то не везёт Украине на президентов. Все, кто был на этом посту, не
выполнили свои обещания, и страна из промышленной державы превращается в
сырьевой придаток для Запада. Но Зеленский и здесь переплюнул всех своих
предшественников. Он не просто не выполнил ни одного своего обещания, он сделал
все наоборот. Так, как он, ещё не делал ни один президент до него. Я задумался,
почему же нам так не везет? 

И понял, мы с вами выбрали из всех президентов только одного.  Это был Леонид
Кучма.  Кравчук достался нам от КПСС, и свои выборы проиграл.  

Ющенко нам подарили американцы, Януковича нам подарил Ахметов. Порошенко -
опять подарок американцев, и, наконец, Зеленский. Ну, этого все знают, кто нам
подарил, господин Коломойский. Подарки один лучше другого. Вот и живем мы все
хуже и хуже. 

Нас было в 1991 году 52 миллиона, а теперь, говорят, около 32 миллионов. Да и
то, сколько живет в стране, а сколько за ее пределами ищет «своё счастье», мы
тоже не знаем. Перепись населения никто делать не хочет. Будет очень
неутешительная картина незалежности. Мы по всем показателям скатились на
последние место в Европе и по уровню жизни, и по её продолжительности, и по ВВП
на душу населения. Но есть, где мы первые, это по уровню преступности в Европе. 

\subsubsection{Большие достижения}

Многие ищут причины не в нас, а пытаются переложить вину,
кто - на Россию, кто - на американцев. Согласен, внешний фактор очень сильно
повлиял на нашу жизнь. Но есть вопрос, а где мы с вами? Что же с нами сделали,
если мы зависим от всех, кого попало. Обидно, что мы так и не смогли построить
хоть что-то своё, а только и делаем, что разрушаем. На сколько ещё хватит того,
что было сделано нашими дедами, не знаю. Но боюсь, запас прочности уже на
исходе. 

А что потом? Полный развал? 

А что мы с вами оставим нашим детям и внукам? Как бы не пришлось от них
выслушать пару-тройку неприятных слов. Скажут, как же вы так могли просрать
такую страну? Что мы им ответим?

