
\ifDOCSTYLErevtex%
  \onecolumngrid
\fi%

\isec{tex4ht}{Translating\ \LaTeX\ to HTML using \texht}

\texht is a system that offers a way to create configurable hypertext
versions of \ \LaTeX\  documents; it is an extension of TEX, in general,
and of \ \LaTeX, in particular.  In this chapter we concentrate on its
default setup that produces HTML output.  The system consists of two
parts: style files to supplement \LaTeX 's existing ones with HTML
features and a postprocessor for extracting HTML files from the output of
the \ \TeX\  program. \texht leaves the major task of processing the source
documents into DVI code to TEX itself. Consequently, \texht has access
to the full power of TEX for handling fonts, macros, variables, category
codes, and other important features that are quite commonly hidden from
the users but are essential for style files. Most applications need to
worry about only a few major features of the system. In the case of HTML
output from \ \LaTeX\  sources, the user may hardly notice any intrusion
from the system. 

Despite being a flexible authoring system rich in capabilities, \texht
keeps its features transparent for most applications. Typically,
compiling and viewing documents that employ \texht is as simple as doing
the same without using the system. The first three sections of this
chapter are geared toward such applicationsexplaining how the system can
be employed by casual users who do not need special configurations. We
list the most commonly used package options, show a complete example
with its input and output, and briefly describe how the process works.
The fourth section describes the new commands that are provided in the
package for authors who wish to add explicit links, achieve HTML effects
manually, or add information to the CSS style sheet.  The fifth section
explains how the system can be extensively configured to produce
different effects from the same \LaTeX\ markup. 

The final part of the chapter goes into more detail of how the various
parts of the \texht process work and how they can be changed. This will
be especially useful for system adminstrators installing \texht for use
by a group of authors. 

\texht is available from either \TeX\ archives or its own Web page [],
where there are also many examples. Currently, \texht is still a work in
progress. A few of the features described here might change in time, but
the changes are expected to be minor, as most features are completely
stable. 

\iii{4_1_Using_texht}
\iii{4_2_a_complete_example}
\iii{4_3_manual_creation}
\iii{4_4_How_texht_works}
\iii{4_5_Extended_customization_of_texht}
\iii{4_6_The_inner_workings_of_texht}
\iii{texht_Summary}
