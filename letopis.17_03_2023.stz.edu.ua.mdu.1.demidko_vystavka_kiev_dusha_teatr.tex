% vim: keymap=russian-jcukenwin
%%beginhead 
 
%%file 17_03_2023.stz.edu.ua.mdu.1.demidko_vystavka_kiev_dusha_teatr
%%parent 17_03_2023
 
%%url https://history.mdu.in.ua/news/docentka_kafedri_kulturologiji_olga_demidko_vidkrila_u_kievi_fotovistavku_teatr_dusha_mariupolja/2023-03-17-456
 
%%author_id edu.ua.mdu
%%date 
 
%%tags 
%%title Доцентка кафедри культурології Ольга Демідко відкрила у Києві фотовиставку «Театр. Душа Маріуполя»
 
%%endhead 
 
\subsection{Доцентка кафедри культурології Ольга Демідко відкрила у Києві фотовиставку \enquote{Театр. Душа Маріуполя}}
\label{sec:17_03_2023.stz.edu.ua.mdu.1.demidko_vystavka_kiev_dusha_teatr}
 
\Purl{https://history.mdu.in.ua/news/docentka_kafedri_kulturologiji_olga_demidko_vidkrila_u_kievi_fotovistavku_teatr_dusha_mariupolja/2023-03-17-456}
\ifcmt
 author_begin
   author_id edu.ua.mdu
 author_end
\fi

Доцентка кафедри культурології, кандидат історичних наук \href{https://archive.org/search?query=creator\%3A"Ольга+Демідко+(Маріуполь)"}{Ольга Демідко} 15 та 16
березня взяла участь у низці важливих заходів. Зокрема, 15 березня вона
долучилася до \href{https://archive.org/details/video.15_03_2023.ukrainskij_kryzovyj_media_centr}{публічного обговорення \enquote{Театр. Культурний код Маріуполя}}.

А 16 березня
\href{https://archive.org/search?query=creator\%3A"Ольга+Демідко+(Маріуполь)"}{Ольга
Демідко} відкрила у Києві фотовиставку \enquote{Театр. Душа Маріуполя}, на якій
розповіла і про історію будівництва приміщення драмтеатру і про жахливу
трагедію, яка сталася рік тому. Організаторами заходу виступили \href{\urlmkulturaIA}{Департамент
культурно-громадського розвитку Маріупольської міської ради} та \href{\urlKievDramTeatrFrankaIA}{Національний
академічний драматичний театр ім. Франка}. На фотовиставці були представлені
світлини з архівів \href{\urlMariupolDramTeatrIA}{Донецького академічного
обласного драматичного театру (м. Маріуполь)}, з фондів Маріупольського
краєзнавчого музею та особистого архіву
\href{\urlSvitlanaIvanivnaOtchenashenkoIA}{народної артистки України Світлани
Отченашенко}, які дослідниця збирала з 2014 року і змогла вивезти з Маріуполя.

\href{https://archive.org/search?query=creator\%3A"Ольга+Демідко+(Маріуполь)"}{Ольга
Демідко} наголосила, що \enquote{попри численні намагання окупантів знищити
культурну спадщину Маріуполя, ми мусимо зберегти нашу пам'ять, культуру та
історію, яка створювалася не одне століття}. Виставка працюватиме протягом
місяця у \href{\urlKievDramTeatrFrankaIA}{Національному академічному драматичному театрі ім. Франка (Київ, площа
Івана Франка, 3)}. Також доцентка виступила перед Національним академічним
театром опери та балету України ім. Тараса Шевченка на мирній акції вшанування
всіх загиблих маріупольців у драмтеатрі.


