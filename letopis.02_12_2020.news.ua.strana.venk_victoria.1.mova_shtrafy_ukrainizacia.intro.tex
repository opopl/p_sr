% vim: keymap=russian-jcukenwin
%%beginhead 
 
%%file 02_12_2020.news.ua.strana.venk_victoria.1.mova_shtrafy_ukrainizacia.intro
%%parent 02_12_2020.news.ua.strana.venk_victoria.1.mova_shtrafy_ukrainizacia
 
%%url 
 
%%author 
%%author_id 
%%author_url 
 
%%tags 
%%title 
 
%%endhead 
\ifcmt
pic https://strana.ua/img/article/3042/34_main.jpeg
caption Тарас Креминь. Фото из Facebook уполномоченного
\fi

Украинский мовный омбудсмен вчера напомнил, что с 16 января вся сфера услуг
должна перейти на украинский язык. 

Этот переход - одна из самых жестких норм языкового закона, принятого на излете
правления Порошенко и не отмененного Зеленским. 

При этом по факту придерживаться этой нормы не обязательно до 2022 года -
поскольку только тогда власть получит право взимать штрафы за нарушения. 

В связи с этим многие русскоязычные украинцы надеются, что к тому времени вся
эта история сойдет на нет, и штрафовать никого не будут. Однако, глядя на
кипучую деятельность Офиса по защите украинского языка, в это верится крайне
мало. 

Разобрались в новом витке украинизации, который наступит после новогодних
праздников, и проанализировали, что сейчас делает украинский \enquote{шпрехенфюрер}
Тарас Креминь. 

