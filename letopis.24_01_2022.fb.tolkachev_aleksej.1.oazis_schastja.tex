% vim: keymap=russian-jcukenwin
%%beginhead 
 
%%file 24_01_2022.fb.tolkachev_aleksej.1.oazis_schastja
%%parent 24_01_2022
 
%%url https://www.facebook.com/oleksiy.tolkachov/posts/5076713815681623
 
%%author_id tolkachev_aleksej
%%date 
 
%%tags future,mechta,obschestvo,schastje,ukraina
%%title Як Україні стати оазою щастя?
 
%%endhead 
 
\subsection{Як Україні стати оазою щастя?}
\label{sec:24_01_2022.fb.tolkachev_aleksej.1.oazis_schastja}
 
\Purl{https://www.facebook.com/oleksiy.tolkachov/posts/5076713815681623}
\ifcmt
 author_begin
   author_id tolkachev_aleksej
 author_end
\fi

Як Україні стати оазою щастя? 

Продовжу публікувати уривки з мого допису в книгу \enquote{Культура Діалогу}. 

Світ стрімко змінюється, і вже зараз базові потреби людини відрізняються від
того переліку, який враховував знаменитий Абрахам Маслоу у своїй піраміді
потреб. Кожній людині в прожитковому мінімумі відтепер потрібні не лише їжа та
одяг, але ще й інтернет з мобільним зв’язком. Водночас, і конкретні функції
держави змінюються, трансформуються, а вимоги до державних сервісів стають все
вищими. 

\ii{24_01_2022.fb.tolkachev_aleksej.1.oazis_schastja.pic.1}

І поки Україна в своїх реформах намагається наздогнати Європу зразка ХХ
століття, розвинені країни пішли вже в далеке майбутнє, конкуруючи одна з одною
за людей. Саме так, передові країни намагаються привабити до себе людей, які є
основною цінністю та рушієм прогресу, відповідно, створюють з цією метою все
більшу інфраструктуру щастя. Допоки в Таллінні та Люксембурзі вводять
безкоштовний проїзд в громадському транспорті, Швеція та Фінляндія змагається в
соціальному захисті та стандартах освіти, а Китай будує зелені міста-сади, що
не викидатимуть CO2, і життя в яких буде казкою.

Єдиний шанс для України потрапити в майбутнє – це вже сьогодні проектувати та
конструювати інфраструктуру щастя для людей ІІІ тисячоліття. 

Нам необхідно зазирнути в майбутнє і запропонувати людям те, чого немає більше
ніде в світі – стати унікальними в тому щасті, яке пропонує Україна. 

На мій погляд, Україна може здійснити декілька трансформацій, що будуть
революційними та творитимуть передумови для щастя:

1. Виплата Безумовного Основного Доходу (БОД) кожній людині. Дуже складно бути
щасливою людині, яка живе в страху перед завтрашнім днем, перед скрутою та
бідністю. Водночас, Четверта технологічна революція вже найближчими роками
замінить мільйони людей на робочих місцях штучним інтелектом та робототехнікою.
Людська праця стане менш затребуваною, що поставить людей на межу виживання.
Саме в цьому контексті в світі обговорюють необхідність виплачувати БОД, однак
в Україні готується експеримент з виплати БОД, що буде поєднаний з масовою
психологічною роботою для максимально проявлення здібностей кожної людини,
вивільненої від примусу праці та матеріальної залежності. Виплата БОД разом із
відкриттям можливостей для розвитку людського потенціалу може створити
неймовірну інфраструктуру для щастя та якісно нове суспільство. 

2. Реформа грошей. В сучасному світі гроші є абсолютно ліквідним активом, якого
не вистачає на всіх. Це найдефіцитніший актив, за який ведеться боротьба: «люди
гинуть за метал» - це правило досі актуальне хоч гроші вже відійшли від
металевої форми в електронну. Для того, щоб знизити рівень конкуренції в
суспільстві, гроші повинні перестати бути дефіцитом та стати доступними кожному
в більш-менш достатній кількості. З одного боку, це буде досягнуто за рахунок
виплати кожному Безумовного основного доходу. З іншої сторони, є рецепти
реформи грошової системи, що призведе до підвищення швидкості обертання грошей
в економіці, щоб усі люди та учасники економічних процесів мали кращі шанси на
доступ до фінансового ресурсу. 

3. Повна трансформація освіти. В Україні можна створити першу в світі
віртуальну освітню систему, що навчатиме дітей зі всього світу у віртуальній
реальності. Це дасть змогу якісно підвищити рівень освіти, адже діти
перестануть отримувати знання, натомість проживатимуть досвід. Скоротиться
тривалість шкільної освіти, натомість буде створена система безперервного
навчання й розвитку людини протягом всього життя. Хіба не будуть люди щасливими
в країні, де школа відійде від своєї дисциплінарної сутності та стане місцем
для проведення дозвілля дітьми? Адже в давньогрецькому початковому значенні
слово «школа» означає саме «дозвілля». 

4. Творення екосимбіотичних соціальних систем – міст майбутнього Футурополісів.
Чимдалі, разом із духовним розвитком людини, поставатиме потреба гармонійної
взаємодії та співжиття з живою природою та навколишнім світом. Саме в цьому
питанні Україна може запропонувати якісно новий спосіб організації життя людини
в Футурополісах – мережевих містах чисельністю до 25-30 тисяч мешканців, де
буде безпосередній контакт з природою, високий рівень життя, водночас людина не
буде затиснута надмірною урбанізацією та не відчуватиме психологічного тиску
великого міста. 

В Україні є можливість стати «країною щастя і добра», однак для цього необхідно
мислити стратегічно, мріяти амбітно, мати правильні ціннісні орієнтири та
керуватися любов’ю до людини. 

Далі буде...

З книги \enquote{Культура діалогу}. Олексій Толкачов - Чи буде щастя в Україні? -
Харків, \enquote{Фоліо}, 2021 рік.

Photo by Vitaliy Mytnik
