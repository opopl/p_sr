% vim: keymap=russian-jcukenwin
%%beginhead 
 
%%file 02_12_2020.news.ua.strana.venk_victoria.1.mova_shtrafy_ukrainizacia
%%parent 02_12_2020
 
%%url https://strana.ua/news/304234-ukrainizatsija-s-16-janvarja-cheho-zhdat-sfere-usluh-i-chem-zanimaetsja-movnyj-ombudsmen-kremen.html
 
%%author Венк, Виктория
%%author_id venk_victoria
%%author_url 
 
%%tags 
%%title Штрафы, угрозы, стукачество. Как власти собираются заставить говорить на украинском учителей и сферу услуг
 
%%endhead 
 
\subsection{Штрафы, угрозы, стукачество. Как власти собираются заставить говорить на украинском учителей и сферу услуг}
\label{sec:02_12_2020.news.ua.strana.venk_victoria.1.mova_shtrafy_ukrainizacia}
\Purl{https://strana.ua/news/304234-ukrainizatsija-s-16-janvarja-cheho-zhdat-sfere-usluh-i-chem-zanimaetsja-movnyj-ombudsmen-kremen.html}
\ifcmt
	author_begin
   author_id venk_victoria
	author_end
\fi

\ifcmt
pic https://strana.ua/img/article/3042/34_main.jpeg
caption Тарас Креминь. Фото из Facebook уполномоченного
\fi

Украинский мовный омбудсмен вчера напомнил, что с 16 января вся сфера услуг
должна перейти на украинский язык. 

Этот переход - одна из самых жестких норм языкового закона, принятого на излете
правления Порошенко и не отмененного Зеленским. 

При этом по факту придерживаться этой нормы не обязательно до 2022 года -
поскольку только тогда власть получит право взимать штрафы за нарушения. 

В связи с этим многие русскоязычные украинцы надеются, что к тому времени вся
эта история сойдет на нет, и штрафовать никого не будут. Однако, глядя на
кипучую деятельность Офиса по защите украинского языка, в это верится крайне
мало. 

Разобрались в новом витке украинизации, который наступит после новогодних
праздников, и проанализировали, что сейчас делает украинский \zqq{шпрехенфюрер}
Тарас Креминь. 

\subsubsection{Что будет со сферой услуг в январе}

Тарас Креминь у себя в Facebook написал, что с 16 января 2021 года сфера
обслуживания переходит на украинский язык - в соответствии со статьей 30 закона
\zqq{Об обеспечении функционирования украинского языка как государственного}. 

\zqq{Все поставщики услуг, независимо от формы собственности, обязаны
обслуживать потребителей и предоставлять информацию о товарах и услугах на
государственном языке. В супермаркете и в интернет-магазине, в кафе, банка, на
АЗС, в аптеке или библиотеке, где бы ты ни был - обслуживание должно быть
украинским. Только по просьбе клиента его персональное обслуживание может
осуществляться на другом языке}, - сообщил уполномоченный. 

Он призвал жаловаться на нарушителей и оставил e-mail.

\ifcmt
pic https://strana.ua/img/forall/u/0/34/%D0%A1%D0%BD%D0%B8%D0%BC%D0%BE%D0%BA(427).JPG
caption Скриншот из Facebook
\fi

В комментариях под постом Креминя пользователи интересуются, что предусмотрено
для нарушителей этой нормы. И какие будут штрафы. 

\ifcmt
pic https://strana.ua/img/forall/u/0/34/%D0%A1%D0%BD%D0%B8%D0%BC%D0%BE%D0%BA(428).JPG
caption Скриншот из Facebook
\fi

При этом пока что беспокоиться предпринимателям рано. Штрафы заработают лишь
через полтора года - с 16 июля 2022 года. До этого времени, по сути, никаких
санкций для тех, кто обслуживал покупателей на не государственном языке, быть
не может, хотя механизм наказания нарушителей уже прописан. 

В нем говорится, что при выявлении нарушения - читай: общения с клиентом на
русском языке - уполномоченный или его представитель составляет акт и объявляет
фирме предупреждение. И требование устранить нарушения в течение 30 дней. 

В случае повторного нарушения в течение года составляется протокол. И
выписывается постановление о штрафе в размере от 300 до 400 необлагаемых
налогом минимумов (5100-6800 грн). Его нужно будет оплатить в течение 15 дней.

Повторное админнарушение в течение года - влечет штраф в размере от 8500 до 11
900 грн. При этом постановление о штрафе можно попробовать обжаловать в суде. 

\subsubsection{Пять месяцев Креминя}

Тарас Креминь был назначен на должность мовного омбудсмена 8 июля. Это бывший
депутат от \zqq{Народного фронта}, который к тому же является соавтором закона о
тотальной украинизации.

Кандидатура Креминя была основной\Furl{https://strana.ua/news/220616-taras-kremen-upolnomochennyj-po-ukrainskomu-jazyku-chto-o-nem-izvestno.html} еще изначально (его рекомендовала Кабмину
омбудсмен Денисова, которая также досталась в наследство от партии
Яценюка-Турчинова).

Однако у Зеленского тогда не решились на столь явный плевок в лицо тем
избирателям, которые голосовали не столько за Зеленского, сколько против
националистов и украинизаторов. Поэтому в итоге пост \zqq{шпрехенфюрера} получила
относительно умеренная педагог из Николаева Татьяна Монахова. 

Тогда это вызвало неудовольствие у тех, кто ставил на Креминя - в частности,
Монахову критиковал экс-глава Института нацпамяти Владимир Вятрович. И другие
представители предыдущей власти, ранее топившие за \zqq{армию, мову, виру}. 

Однако позже Татьяна решила уйти - по ее словам, власть так и не обеспечила ей
ни офиса, ни зарплат для сотрудников. И часть расходов она вообще вынуждена
была оплачивать из своего кармана. 

Был ли это сознательный саботаж со стороны Минкульта - непонятно, но факт
остается фактом: пост по итогу занял Креминь. Которого, в отличие от
предшественницы, снабдили всем, что необходимо. Уже через месяц после
назначения офис мовного омбудсмена получил статус юрлица.\Furl{https://strana.ua/news/282741-sekretariat-upolnomochennoho-po-zashchite-jazyka-stal-jurlitsom.html} Монахова не могла
этого добиться почти год. 

\subsubsection{Авакову можно, другим - нельзя}

В первые же дни своей работы Креминь отметился двойными стандартами: он заявил,
что не против русскоязычных выступлений Арсена Авакова. Поскольку этот министр
- слишком масштабная личность и обременен весьма тяжелыми государственными
заботами. 

Впрочем, разгадка скорее в том, что Креминь - однопартиец главы МВД и еще
раньше пел ему хвалебные оды. 

Что же до остальных русскоязычных украинцев, то здесь позиции \zqq{шпрехенфюрера}
совершенно непримиримые. С введением этой осенью украинизации русских школ
Креминь обещал разобраться с учителями, которые ведут уроки на русском.

В начале ноября он заявил, что получает десятки жалоб на то, что педагоги в
столице переходят с детьми на русский язык. А внешкольное образование - кружки,
секции - \zqq{грешит} этим вообще повсеместно. 

Конечно, такие претензии в XXІ веке звучат максимально дико. Но украинизатора
из \zqq{Народного фронта}, а главное - тех, кто его назначил, это совершенно не
беспокоит. 

Интересно, что Креминь в этом смысле даже не скрывает духовного родства с
Порошенко, который летом собирал акции под Радой против законопроекта \zqq{слуги
народа} Максима Бужанского. Документ дает отсрочку русским школам на три года
по украинизации.

Креминь тогда выступил с экс-президентом на одной сцене и призвал не принимать
законопроект. В итоге \zqq{слуги} даже не вынесли его на рассмотрение. 

А в последнее время \zqq{шпрехенфюрер} просто фонтанирует инициативами.

Так, после встречи с коллегой из Латвии, он загорелся идеей мобильного
приложения, через которое можно будет жаловаться на нарушителей мовного закона.
И выступил за \zqq{цифровой мониторинг} использования государственного языка. А
также, по примеру Латвии, введение института \zqq{старших инспекторов}, которые
будут проверять все и вся. 

У Эстонии, где тоже есть подобная должность, украинский \zqq{шпрехенфюрер}
планирует позаимствовать систему проверок заведений торговли. Эта маленькая
страна делает по три тысячи проверок в год - видимо, в более масштабной Украине
хотят проводить гораздо больше рейдов. 

А буквально пару дней назад Креминь обеспокоился, что преподаватели для
слабослышащих детей все еще используют русский язык жестов. То есть
деятельность чиновника доходит уже до полного абсурда. 

Судя по заявлениям Тараса Креминя, он хочет контролировать буквально всё. Так,
он заявил недавно, что дистанционное обучение во время карантина несет угрозы.
Но не в падении качества образования - этот вопрос, видимо, не так важен
омбудсмену - а в том, что облегчает переход учителей на русский язык. Поскольку
по время онлайн-конференций их никто не контролирует.

Чтобы это пресечь, Креминь предложил фотографировать, снимать на видео и
записывать аудио нарушений.

Правда, здесь возникает сразу два вопроса. Первый - об адекватности самого
омбудсмена. Каким образом фотография может передать, на каком языке говорил
педагог? Второй - о юридической стороне вопроса. Будут ли суды принимать во
внимание любительские видео и тем более аудио? При этом нет сомнений, что таких
доказательств самому омбудсмену вполне хватит.

Также показателен сам факт, что учеников открыто призывают \zqq{стучать} на
учителей. 

\ifcmt
pic https://strana.ua/img/forall/u/0/92/%D0%BA%D1%80%D0%B5%D0%BC%D0%B8%D0%BD%D1%8C(1).png
\fi
