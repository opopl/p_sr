% vim: keymap=russian-jcukenwin
%%beginhead 
 
%%file slova.sssr
%%parent slova
 
%%url 
 
%%author 
%%author_id 
%%author_url 
 
%%tags 
%%title 
 
%%endhead 
\chapter{СССР}

%%%cit
%%%cit_pic
%%%cit_text
\enquote{Червону руту} играли и пели в концертных залах, ресторанах, на танцплощадках
по всему \emph{СССР} и за его пределами. Белорусский композитор Эдуард Ханок в одном
интервью признался, что под влиянием \enquote{Червоной руты} он написал другую
популярную песню на украинском языке - \enquote{Верба} на стихи Юрия Рыбчинского
%%%cit_comment
%%%cit_title
\citTitle{Какая легенда вдохновила Владимира Ивасюка написать \enquote{Червону руту} 
(украинскому шлягеру - 50)}, 
Татьяна Кроп, zen.yandex.ru, 07.12.2020
%%%endcit

%%%cit
%%%cit_pic
%%%cit_text
Вскоре на телеэкраны вышел первый в \emph{советской Украине} музыкальный фильм
\enquote{Червона рута}, в котором в главных ролях вместе с названными Зинкевичем и
Яремчуком снялась еще одна восходящая звезда - молодая преподавательница
культпросветучилища, очаровательная София Ротару.  На долгие годы украсила ее
репертуар \enquote{Червона рута}, которую София Михайловна любит исполнять в своих
сольных концертах. Во многом благодаря именно этой песне сформировался
сценический образ Софии Ротару - романтической, загадочной, умеющей любить и
влюблять в себя красавицы
%%%cit_comment
%%%cit_title
\citTitle{Какая легенда вдохновила Владимира Ивасюка написать \enquote{Червону руту} 
(украинскому шлягеру - 50)}, 
Татьяна Кроп, zen.yandex.ru, 07.12.2020
%%%endcit

