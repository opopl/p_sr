% vim: keymap=russian-jcukenwin
%%beginhead 
 
%%file slova.religia
%%parent slova
 
%%url 
 
%%author 
%%author_id 
%%author_url 
 
%%tags 
%%title 
 
%%endhead 
\chapter{Религия}
\label{sec:slova.religia}

%%%cit
%%%cit_head
Война за Веру Христову - Страна Агностиков и Атеистов
%%%cit_pic
%%%cit_text
– У збірці є рядок: \enquote{Відважно воює за віру Христову країна агностиків та
атеїстів}. Це поетичний троп чи діагноз Україні?  – Це просто образ. Не брався
би ставити діагнози, занадто це марнославно й претензійно. Якщо ви питаєте про
\emph{релігію}, то крім агностиків та атеїстів маю велику кількість знайомих
вірян різних церков. До їхніх духовних пошуків та розчарувань ставлюсь із
повагою та розумінням. Якось так сталося, що ще з юнацького віку трапилося
бути свідком постання різних українських \emph{релігійних} громад. Тому це все
моє середовище, незалежно від моїх власних \emph{релігійних} переконань
%%%cit_comment
%%%cit_title
\citTitle{Письменник Сергій Жадан: Мені цікаво говорити про речі прості й
очевидні}, Олег Поляков, life.pravda.com.ua, 13.06.2021
%%%endcit

%%%cit
%%%cit_head
%%%cit_pic
%%%cit_text
Перше: ми з матерії зробили якийсь грубий, однотипний «конструктивний» набір.
Відкривши певні «закони», вирішили, що вони обов’язкові й універсальні для
всесвіту, для всієї безмірності. Друге: руйнуючи містичні, \emph{релігійні} картини
світу, ми залишаємося в полоні антропоцентризму і, якщо можна так висловитися,
«очевидізму» — тобто трафаретного, повсякденного бачення… Третє: відкриваючи
нові можливості енергетики, генетики, техніки, пізнання, ми для себе робимо
виняток: еволюція людини у більшості наукових висновків вважаються завершеною.
Отже, ми потрапляємо в парадоксальну ситуацію: творячи експоненту ментальної,
розумової, гностичної еволюції, ми самі залишаємося своєрідними рептиліями,
реліктами, минулих природнотворчих епох. Це породжує ножиці, що неодмінно
розріжуть наш зв’язок із власним породженням, — з техноеволюцією, котра
блискавично розвивається, скоріше — саморозвивається, бо вже часто незалежна
від волі людини. Такий шлях — самогубство!
%%%cit_comment
%%%cit_title
\citTitle{Вогнесміх}, Олесь Бердник
%%%endcit
