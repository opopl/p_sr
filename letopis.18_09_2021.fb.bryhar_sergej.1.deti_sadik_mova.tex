% vim: keymap=russian-jcukenwin
%%beginhead 
 
%%file 18_09_2021.fb.bryhar_sergej.1.deti_sadik_mova
%%parent 18_09_2021
 
%%url https://www.facebook.com/serhiibryhar/posts/1790390954494182
 
%%author_id bryhar_sergej
%%date 
 
%%tags deti,detsad,jazyk,mova,ukraina,ukrainizacia
%%title Як на мене, все в мовному питанні починається з садочка
 
%%endhead 
 
\subsection{Як на мене, все в мовному питанні починається з садочка}
\label{sec:18_09_2021.fb.bryhar_sergej.1.deti_sadik_mova}
 
\Purl{https://www.facebook.com/serhiibryhar/posts/1790390954494182}
\ifcmt
 author_begin
   author_id bryhar_sergej
 author_end
\fi

Останнім часом я регулярно перетинаюся з дітьми з одеського приватного садочка.
Гуляємо на одному майданчику. Так от вони навіть не розуміють українську
(сподіваюся, що не всі, але якщо розмовляти вибірково - не розуміють). Весь
виховний процес ведеться лише однією мовою - отією, яку в нашому регіоні
нещодавно позбавлено статусу регіональної.

\ifcmt
  ig https://scontent-frx5-2.xx.fbcdn.net/v/t39.30808-6/242263725_1790390924494185_3562217767311413368_n.jpg?_nc_cat=109&_nc_rgb565=1&ccb=1-5&_nc_sid=8bfeb9&_nc_ohc=ErLNgkg_zAEAX9tW3H4&_nc_ht=scontent-frx5-2.xx&oh=8c8ead84ec7da933725126ade3a840bc&oe=614BF1F8
  @width 0.4
  %@wrap \parpic[r]
  @wrap \InsertBoxR{0}
\fi

Нерідко доводиться чути: "російський садочок - це не біда, бо діти підуть до
школи, там навчаться, і говоритимуть". 

Ну що ж, частково я згоден. Але то лише частково. Хіба достатньо просто
навчитися? Так, це, звісно, добре, але ми це мали й раніше. Тепер уже необхідно
виносити мову за межі освітніх закладів. Білоруські діти теж навчаються
білоруської. І що? Від цього вона змінює статус "майже мертва" на статус
"жива"? На жаль, ні. Це вже практично "латина"...

Як на мене, все в мовному питанні починається з садочка. От як там, так людина
й вважатиме за правильне далі. На жаль, поки що наша держава навіть не
намагається пояснювати дітям, що розмовляти в публічному просторі російською -
це неправильно. Навпаки - вперто не бачить у цьому проблеми. Звідси й
результати...

Я тут нещодавно розмовляв з чоловіком із Києва, який розповів, що його син
ходить у гарний україномовний садочок. Ну й говорить: "більшість родичів,
сусідів, знайомих розмовляють російською, діти у дворі - російською, а він нє -
українською, і все". От я й кажу: "правильний садочок - це дуже важливо".

\ii{18_09_2021.fb.bryhar_sergej.1.deti_sadik_mova.cmt}
