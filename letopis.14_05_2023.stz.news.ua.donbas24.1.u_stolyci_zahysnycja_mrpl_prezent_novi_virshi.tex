% vim: keymap=russian-jcukenwin
%%beginhead 
 
%%file 14_05_2023.stz.news.ua.donbas24.1.u_stolyci_zahysnycja_mrpl_prezent_novi_virshi
%%parent 14_05_2023
 
%%url https://donbas24.news/news/u-stolici-zaxisnicya-mariupolya-prezentuvala-novi-virsi
 
%%author_id demidko_olga.mariupol,news.ua.donbas24
%%date 
 
%%tags 
%%title У столиці захисниця Маріуполя презентувала нові вірші
 
%%endhead 
 
\subsection{У столиці захисниця Маріуполя презентувала нові вірші}
\label{sec:14_05_2023.stz.news.ua.donbas24.1.u_stolyci_zahysnycja_mrpl_prezent_novi_virshi}
 
\Purl{https://donbas24.news/news/u-stolici-zaxisnicya-mariupolya-prezentuvala-novi-virsi}
\ifcmt
 author_begin
   author_id demidko_olga.mariupol,news.ua.donbas24
 author_end
\fi

\ii{14_05_2023.stz.news.ua.donbas24.1.u_stolyci_zahysnycja_mrpl_prezent_novi_virshi.pic.front}

\begin{center}
  \em\color{blue}\bfseries\Large
  У Києві відбувся творчий вечір, присвячений маріупольчанці Валерії Суботіної
\end{center}

13 травня у Будинку кіно в Києві всі охочі нарешті \href{https://donbas24.news/news/polonyanka-z-azovstali-valeriya-subotina-rozkaze-pro-odruzennya-pid-obstrilami-ta-polon-podrobici}{\emph{зустрілися з
маріупольчанкою}},
\footnote{Полонянка з Азовсталі Валерія Суботіна розкаже про одруження під обстрілами та полон — подробиці, Ольга Демідко, donbas24.news, 10.05.2023, \par\url{https://donbas24.news/news/polonyanka-z-azovstali-valeriya-subotina-rozkaze-pro-odruzennya-pid-obstrilami-ta-polon-podrobici}}
поетесою та захисницею Маріуполя Валерією (Навою) Суботіною
(Карпиленко), яка тільки 10 квітня повернулася з полону. Дівчина 10 місяців
перебувала в оточенні ворогів і, оговтавшись, зустрілася зі своїми земляками.
Зустріч організувала мати її загиблого чоловіка Людмила Суботіна. Це був
зворушливий творчий вечір, який зібрав у Червоній залі Будинку кіно друзів,
знайомих, посестер та побратимів Нави.

\textbf{Читайте також:} \href{https://donbas24.news/news/istoriya-stalevoyi-navi-zaxisnicya-mariupolya-rozpovila-pro-vesillya-na-azovstali-ta-rosiiskii-polon-video}{\emph{Історія сталевої Нави — захисниця Маріуполя розповіла про весілля на Азовсталі та російський полон}}%
\footnote{Історія сталевої Нави — захисниця Маріуполя розповіла про весілля на Азовсталі та російський полон, Еліна Прокопчук, donbas24.news, 05.05.2023, \par%
\url{https://donbas24.news/news/istoriya-stalevoyi-navi-zaxisnicya-mariupolya-rozpovila-pro-vesillya-na-azovstali-ta-rosiiskii-polon-video}%
}

На заході друзі Валерії і всі, кого вона надихала, виходили на сцену, аби
поділитися життєвими історіями, які їх поєднують з військовослужбовицею, та
почитати її вірші зі збірки \enquote{Квіти і зброя}. Водночас Валерія презентувала три
власні поезії. Перша була присвячена Маріуполю та маріупольцям, які не могли
говорити, тому що у них не було зв'язку.

\begin{leftbar}
\emph{\enquote{Росія робила, все, аби велика земля, весь світ не міг почути цих людей. і в
цьому був найбільший цинізм. Вони не просто вбивці, не просто вбивці людей,
вони ще робили все і роблять все, аби світ не знав правду. Аби світ не почув
і не побачив того, що вони роблять. Але світ побачив і почув, в тому числі,
завдяки пресслужбі Азову}}, — наголосила Валерія.
\end{leftbar}

\href{https://archive.org/details/video.14_05_2023.donbas_novyny.novi_virshi_valerii_subotunoi}{%
Відео: Нові вірші Валерії Суботіної, Донбас Новини, 14.05.2023}%
\footnote{\url{https://www.youtube.com/watch?v=wFUxcaDsduM}} %
\footnote{\url{https://archive.org/details/video.14_05_2023.donbas_novyny.novi_virshi_valerii_subotunoi}}

\ifcmt
  ig https://i2.paste.pics/PRU36.png?trs=1142e84a8812893e619f828af22a1d084584f26ffb97dd2bb11c85495ee994c5
  @wrap center
  @width 0.9
\fi

Другий вірш поетеса присвятила військовим, які наразі перебувають у полоні -
цей вірш вона написала декілька днів назад.

\begin{leftbar}
\emph{\enquote{Ми знали, що нас тут чекають, ми розуміли, що інакше бути не може. Але
майже рік... Майже рік знаходитися серед ворогів, серед людей яких ти
ненавидиш і геть нічого не можеш зробити... Це важко. І тому нам треба
робити все, аби поскоріше їх звідти забрати... Будь ласка, друзі,
пам'ятайте про героїв Азовсталі. Будь ласка, представники ЗМІ, не
забувайте про них, продовжуйте про них говорити}}, — наголосила Валерія
Суботіна.
\end{leftbar}

\textbf{Читайте також:} \href{https://donbas24.news/news/u-kijevi-vidbuvsya-zaxid-na-pidtrimku-viiskovopolonenoyi-z-azovstali-foto-video}{\emph{У Києві відбувся захід на підтримку військовополоненої з Азовсталі}}%
\footnote{У Києві відбувся захід на підтримку військовополоненої з Азовсталі, Ольга Демідко, donbas24.news, 17.12.2022, \par%
\url{https://donbas24.news/news/u-kijevi-vidbuvsya-zaxid-na-pidtrimku-viiskovopolonenoyi-z-azovstali-foto-video}%
}

\ii{14_05_2023.stz.news.ua.donbas24.1.u_stolyci_zahysnycja_mrpl_prezent_novi_virshi.pic.1}

\begin{leftbar}
\emph{\enquote{Нас усіх об'єднує війна... Серед нас є сім'ї, які дочекалися рідних з полону.
Ось і наша сталева донечка разом із нами, дякувати Богу. Але ніхто із нас не
може спокійно жити, бо дуже багато наших рідних і близьких зараз у полоні. Ми
маємо разом боротися, щоб якомога швидше наші воїни повернулися з полону. І ми
продовжуватимемо боротьбу, бо материнські серця не можуть по-іншому}}, —
зауважила мати загиблого захисника Маріуполя Людмила Суботіна.
\end{leftbar}

Останній вірш, який представила Валерія, є найбільш важливим для неї, адже він
про коханого чоловіка Андрія, з яким Валерія пробула у шлюбі три дні: 5 травня
2022 року було зареєстровано шлюб, а 7 травня він не повернувся з бойового
завдання. На творчому вечорі жінці подарували картину, присвячену їй та Андрію... 

\begin{leftbar}
\emph{\enquote{Не просто пам'ять про нього живе. Він живе. Ті, хто знали мого Андрія
знають, що він завжди робив неймовірні речі для мене. Ця друкарська
машинка, яку мені сьогодні подарували, всі квіти... Я впевнена, що все це
робить він вашими руками. І він зараз поруч. І все те, що зараз
відбувається, я впевнена, йому дуже сильно подобається}}, — зазначила
поетеса.
\end{leftbar}

\textbf{Читайте також:} \href{https://donbas24.news/news/u-kijevi-prezentuyut-projekt-portreti-mariupolya-yak-potrapiti-na-vistavku}{\emph{У Києві презентують проєкт \enquote{Портрети Маріуполя} — як потрапити на виставку}}%
\footnote{У Києві презентують проєкт \enquote{Портрети Маріуполя} — як потрапити на виставку, Еліна Прокопчук, donbas24.news, 12.05.2023, \par%
\url{https://donbas24.news/news/u-kijevi-prezentuyut-projekt-portreti-mariupolya-yak-potrapiti-na-vistavku}%
}

\ii{14_05_2023.stz.news.ua.donbas24.1.u_stolyci_zahysnycja_mrpl_prezent_novi_virshi.pic.2}
\ii{14_05_2023.stz.news.ua.donbas24.1.u_stolyci_zahysnycja_mrpl_prezent_novi_virshi.pic.3}

На зустрічі всі присутні могли підійти до Валерії, висловити їй слова подяки за
захист, підтримати дівчину, отримати автограф та обійняти її.

\ii{14_05_2023.stz.news.ua.donbas24.1.u_stolyci_zahysnycja_mrpl_prezent_novi_virshi.pic.4}
\ii{14_05_2023.stz.news.ua.donbas24.1.u_stolyci_zahysnycja_mrpl_prezent_novi_virshi.pic.5}

Нагадаємо, раніше Донбас24 публікував історії \href{https://archive.org/details/07_03_2023.olga_demidko.donbas24.nadyhajuchi_istorii_zhinky_priazovja}{\emph{п'яти жінок з Маріуполя}},
\footnote{Надихаючі історії про непересічних жінок Приазов'я, Ольга Демідко, donbas24.news, 07.03.2023, %
\par\url{https://donbas24.news/news/nadixayuci-istoriyi-pro-neperesicnix-zinok-priazovya}, \par%
Internet Archive: \url{https://archive.org/details/07_03_2023.olga_demidko.donbas24.nadyhajuchi_istorii_zhinky_priazovja}%
} чия непересічність, самовіддана діяльність і винятковий талант надихають.

Ще більше новин та найактуальніша інформація про Донецьку та Луганську області
в нашому телеграм-каналі Донбас24.

Фото: Наталі Дєдової та з архіву Донбас24

\ii{insert.author.demidko_olga}
%\ii{14_05_2023.stz.news.ua.donbas24.1.u_stolyci_zahysnycja_mrpl_prezent_novi_virshi.txt}
