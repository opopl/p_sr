% vim: keymap=russian-jcukenwin
%%beginhead 
 
%%file 09_01_2022.fb.krjukova_svetlana.1.tost_vsegda_za_ljubov.cmt
%%parent 09_01_2022.fb.krjukova_svetlana.1.tost_vsegda_za_ljubov
 
%%url 
 
%%author_id 
%%date 
 
%%tags 
%%title 
 
%%endhead 
\zzSecCmt

\begin{itemize} % {
\iusr{Jeka Zender}
Как душевненько @igg{fbicon.face.savoring.food}{repeat=2}  @igg{fbicon.clinking.glasses}  @igg{fbicon.bottle.popping.cork}  @igg{fbicon.beaming.face.smiling.eyes} 


\iusr{Svetlana Kryukova}
\textbf{Jeka Zender} спасибо)

\iusr{Наталья Крамаренко}
Пост жизненный!!!


\iusr{Svetlana Kryukova}
\textbf{Наталья Крамаренко}  @igg{fbicon.face.smiling.sunglasses} 

\iusr{Александр Петров}

Да уж Светлана, ваша способность в маленькой короткой заметке передать не
только вкус, запах и аромат, но и теплоту \enquote{заброшенных внутрь дров}, вызывает
искренний восторг!

\iusr{Svetlana Kryukova}
\textbf{Александр Петров} спасибо)

\iusr{Oleksandr Novokhatskyi}

Слова о простом и ежедневном рождаются не у каждого художника )
Замечательная зарисовочка высокой вязью слова ))) спасибо

\iusr{Svetlana Kryukova}
\textbf{Oleksandr Novokhatskyi}  @igg{fbicon.face.smiling.eyes.smiling} 

\iusr{Vladimirs Sokolovs}
Как ярко и сочно! Передан аромат и место времени. Браво!


\iusr{Svetlana Kryukova}
\textbf{Vladimirs Sokolovs}  @igg{fbicon.face.eyes.star} 

\iusr{Nikolay Vassilkov}
У вас талант


\iusr{Svetlana Kryukova}
\textbf{Nikolay Vassilkov} спасибо  @igg{fbicon.face.relieved} 

\iusr{Віталій Черепаха}
Зощенко перевернулся от фельетона \textbf{Svetlana Kryukova}

\iusr{Svetlana Kryukova}
\textbf{Віталій Черепаха} вы мне льстите

\iusr{Віталій Черепаха}
\textbf{Svetlana Kryukova}, не без этого, но навеяло, не скрыть...

\iusr{Валерий Мудрак}
Здорово!... Надо продолжать...

\iusr{Svetlana Kryukova}
\textbf{Валерий Мудрак} спасибо!

\iusr{Александр Витальевич Герольских}

Первое предложение заслуживает отдельного комента.

На нём как на стержне держится вся пирамидка. Разновеликие только кругляши идут
затем подряд. Не получилось произведения, а оно уже и виделось, но очерк хорош!

ПишИте - это Ваше.

\iusr{Адвокат Валерий Кострюков}
Неплохой слог.

\begin{itemize} % {
\iusr{Svetlana Kryukova}
\textbf{Адвокат Валерий Кострюков} неплохой комментарий )

\iusr{Адвокат Валерий Кострюков}
\textbf{Svetlana Kryukova} Слог — это минимальная фонетико-фонологическая единица, характеризующаяся наибольшей акустико-артикуляционной слитностью своих компонентов, то есть входящих в него звуков. Слог не имеет связи с формированием и выражением смысловых отношений. Это чисто произносительная единица.)))

\iusr{Владимир Малый}
\textbf{Адвокат Валерий Кострюков} Я всегда задавался вопросом - кто эти люди, создающие Википедию @igg{fbicon.thinking.face} ???

\ifcmt
  ig https://scontent-frx5-1.xx.fbcdn.net/v/t39.30808-6/271588658_4692232444187493_3223469768158948410_n.jpg?_nc_cat=105&ccb=1-5&_nc_sid=dbeb18&_nc_ohc=pPCaaVEQmdEAX86dhjD&_nc_ht=scontent-frx5-1.xx&oh=00_AT_OXgwKtFYWbeUY40DtqrLi_PWdsGNCpH08UKHROcIBSQ&oe=61E1FF49
  @width 0.2
\fi

\iusr{Анна Кифенко}
\textbf{Адвокат Валерий Кострюков} , омонимы - это одинаковые по написанию и звучанию, но разные по значению слова и другие единицы языка. Это про слог, если что.

\iusr{Кирилл Кузьмицкий}
\textbf{Svetlana Kryukova} неплохой диалог)

\iusr{Tana Istama}
\textbf{Svetlana Kryukova} что значит талант!!!

\iusr{Константин Колпаков}
\textbf{Svetlana Kryukova} а где это кафе? Сам сейчас в Москве, но очень захотелось 100 грамм  @igg{fbicon.face.grinning.sweat}{repeat=4} 

\end{itemize} % }

\iusr{Alexander Berman}

Светлана, обожаю ваши зарисовки. Они, как фрагменты фильмов Феллини.
@igg{fbicon.face.smiling.eyes.smiling} 

\iusr{Svetlana Kryukova}
\textbf{Alexander Berman} неплохо)

\iusr{Nadiya Tkachova}
Светлана! Короткий тост: за умение писать...

\iusr{Svetlana Kryukova}
\textbf{Nadiya Tkachova}  @igg{fbicon.face.smiling.sunglasses} 

\iusr{Сергей Шабовта}

Ваши литературные зарисовки выверенно лаконичны, тонко детализированны и всегда
таят в себе добрую улыбку.


\iusr{Svetlana Kryukova}
\textbf{Сергей Шабовта} спасибо)

\iusr{Yana Andreieva}
Классно написано!
Еврейская семья просто не знает, что лучший в мире мальчик - мой сын  @igg{fbicon.smile}  )))

\begin{itemize} % {
\iusr{Валерий Мудрак}
Вы еврейка?...

\iusr{Yana Andreieva}
\textbf{Валерий Мудрак} , а шо такое?
\end{itemize} % }

\iusr{Антонина Подрезенко}
Супер!

\iusr{Эдуард Натаров}
Вот об этом и пишите, у Вас здорово получается!

\iusr{Ірина Оснач}
Прекрасна замальовка!

\iusr{Владимир Воротнюк}

всё на любви замешано. Их, любвЕй, оченно много, начиная от любви к борщу и
Родине, и, завершая оной к бабам и рыбалке)

\iusr{Oleg Golokhvastov}
В захолустном ресторане,
Где с пятеркой на «ура»
Громыхают стопарями –
Кто не допили с утра. (с)

\iusr{Константин Марфлюк}

Вы талантливы... Умение передавать словом вкус, настроение, запахи в одной
зарисовке не каждому дано. Мне показалось микс Жванецкого и Булгакова. Хотя я не
критик и ни на что не претендую.

\iusr{Алексей Иванов}
Безнадёга.

\iusr{Мила Ющенко}
Чувствуется автор читает много хорошей литературы.

\iusr{Vitaliy Pechkurov}
 @igg{fbicon.hands.applause.yellow}{repeat=5} @igg{fbicon.heart.red} @igg{fbicon.bouquet} 

\iusr{Алекс Щеглов}
Шо це було?

\iusr{Александр Алексеевич}
Шарман. Правильно сделал, что забрал бряцки. Подлости женской нет предела, всё раздадут, если не детям-дуракам, то первым встречным.

\iusr{Станіслав Нагурний}
Гарно. тепло.
Дякую.

\iusr{Vio Cherri}
И шо там за этими бАрдовыми шторами - кофе эКспрессо подают?))

\iusr{Лариса Осипенко}
\textbf{Vio Cherri} , киевляне и понаехавшие вроде как дрова, заброшенные внутрь) Сейчас тоже исправит. Хотя это сложнее, чем бАрдо на бордо)

\iusr{Анатолий Чабанюк}
Светлана, более современных посетителей не заметили?

\iusr{Лариса Полякова}
да. всегда ! до тех пор, пока смерть не разлучит нас.  @igg{fbicon.face.tears.of.joy}  есть три -четыре классических тоста. они , как отче наш. на все времена.

\iusr{Георги Радов}
И, всё таки, лучше, чем картины писать.
Захватывающе...

\iusr{Виктор Цымбал}
Так-то оно лучше. Пока Святки.

\iusr{Yartsev Anatoliy}
Романтично. Киев и киевляне.

\iusr{Адвокат Олег Несинов}
...C такими литературными и художественными талантами и зачем Вам политика...это болото...?

\iusr{Вячеслав Кедровский}

А ведь да от печи в отличии от батареи тепло совершенно другое. Оно
проникающее. Подтверждаю как имеющий и то и другое. Прекрасно написано вроде
как бы и из ничего. Вот такая она Светлана Крюкова. Тэнкс.


\iusr{Александр Садовский}
Такую информацию можно было получить, переодевшись и маскируясь под
\enquote{завсегдатайку}... Или - \enquote{завсегдатая}... )

\iusr{Эрл Батори}
про \enquote{напугали петарды} - спИздила

\iusr{Олег Сластенов}
Хорошо пишете, прямо как в живую)
Вы там что пили, куда ехали не спрашиваю?)

\iusr{Jorge Antonio}
Алілуя  @igg{fbicon.hands.pray}{repeat=3} 

\iusr{Anatoliy Veda}
Похож на записки врача, Светлана пора писать книги

\iusr{Александр Маханько}
Это и есть Украина.

\iusr{Валерий Мудрак}
Хм, как много комментариев. Проснулись.... Просыпались бы еще, когда выбираете. Что? Кого? Не имеет значения. Главное вовремя глазки протереть...

\iusr{Евгений Колесников}
Так а где Зе, уехал на троллейбусе \#43??? @igg{fbicon.thinking.face} 

\iusr{Всеволод Дорошенко}
Жизнь продолжается даже в лагерях, чего говорить о кафе!

\iusr{Дмитрий Марунич}
Хде Рэндж Ровер?)

\iusr{Александр Будник}
на 3й платформе...

\iusr{Анастасия Товт}
\textbf{Дмитрий Марунич} на платформе 9¾

\iusr{Александр Шибанов}
Ещё и не такое услышишь на Автостанции «Пiвденна»  @igg{fbicon.face.wink.tongue} 

\iusr{Oleg Arzt}
Это классика

\iusr{Dmytro Nohal}
последний раз на том вокзале был ровно 20 лет назад)))

\iusr{Александр Синельников}
Жизненно

\iusr{Дмитрий Каптель}
Долго думал о вениках, наконец-то дошло, что это букеты цветов
Старею

\iusr{Андрей Рубрук}
Хороший рассказ

\iusr{Melnik Vadim}
Раннее ? Из пережитого. Меню на стенке.0.33 .100гр. Х/закуски.

\iusr{Александр Харченко}
Пришла подруга поплакатся в жолетку, ржали до утра.

\iusr{Игорь Дорник}
Ну за здоровье пить глупо, медицина платная. Любовь пока традиционная, без
гендерного равенства и без брачного договора.

\iusr{Борис Дробот}
Кафе 4 сезона.

\iusr{Владимир Вахрушев}
smm для забегаловки  @igg{fbicon.thumb.up.yellow}  @igg{fbicon.face.grinning.squinting} 

\iusr{Ярослав Шрамко}
дебилы, жЫвотные...




\end{itemize} % }
