% vim: keymap=russian-jcukenwin
%%beginhead 
 
%%file slova.missia
%%parent slova
 
%%url 
 
%%author 
%%author_id 
%%author_url 
 
%%tags 
%%title 
 
%%endhead 
\chapter{Миссия}

%%%cit
%%%cit_pic
%%%cit_text
Однако надо понимать, что эти воззрения у Розенберга появились еще когда он в
нацистском госаппарате не работал, а был только функционером НСДАП. То есть
тотальная война еще не была в повестке дня. Когда же он возглавил министерство
по делам восточных территорий нацистской Германии, эти планы даже близко не
подошли к своей реализации.  Причина была в том, что сам Адольф Гитлер никаких
марионеточных государств на территориях Украины создавать не собирался. Он
видел свою \emph{миссию} в прямом завоевании украинских территорий и заселении их
немцами. 
%%%cit_comment
%%%cit_title
\citTitle{22 июня - 80 лет нападения на СССР. Что немцы готовили для украинцев}, 
Максим Минин, strana.ua, 22.06.2021
%%%endcit
