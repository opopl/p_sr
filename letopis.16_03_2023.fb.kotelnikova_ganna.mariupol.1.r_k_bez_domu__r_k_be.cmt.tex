% vim: keymap=russian-jcukenwin
%%beginhead 
 
%%file 16_03_2023.fb.kotelnikova_ganna.mariupol.1.r_k_bez_domu__r_k_be.cmt
%%parent 16_03_2023.fb.kotelnikova_ganna.mariupol.1.r_k_bez_domu__r_k_be
 
%%url 
 
%%author_id 
%%date 
 
%%tags 
%%title 
 
%%endhead 

\qqSecCmt

\iusr{Елена Попова}

Нет слов....

\iusr{Александр Соловчук}

послезавтра я выеду и моя дата печальная(

\begin{itemize} % {
\iusr{Ganna Kotelnikova}
\textbf{Александр Соловчук} , головне, що ти зміг виїхати!
\end{itemize} % }

\iusr{Vira Pavliuk}

Жахливі спогади, що тут скажеш....

\iusr{Anzhela Grebenyuk}

Рік тому спали в коридорі, це врятувало. Дві міни під балкони та вікно кухні.

\iusr{Angelina Agvanyan}

А ми тільки 22го вийдемо, рік тому я ще в підвалі😖😖😖😖😖 мозок дуже багато
зтер з пам'яті, багато страшних спогадів вже не пам'ятаю, захисна реакція,
мабуть

\begin{itemize} % {
\iusr{Ganna Kotelnikova}
\textbf{Angelina Agvanyan}, да, в мене теж все злилося в одну сіру смугу ...

\iusr{Oksana Tereshchenko}
\textbf{Angelina Agvanyan} воно все всередині і нікуди не стерлось нажаль...

\iusr{Angelina Agvanyan}
\textbf{Oksana Tereshchenko} а те, що в середині, то назавжди😪😢

\end{itemize} % }

\iusr{Аня Сліпченко}

Анютка що ти пережила, читаю і мурахи по шкірі ../ як шкода. Обіймаю тебе .. я
в сусідній області. Якщо що

\begin{itemize} % {
\iusr{Ganna Kotelnikova}
\textbf{Аня Сліпченко} , дякую. Гума чекає)) її треба комусь віддати!
\end{itemize} % }

\iusr{Ольга Ниякая}

Ми були неподалік, у підвалі дома навпроти церкви (по Соборной)

Завтра руZzня скине бомбу на драм. (((( ми через 10 хвилин після цього поїдемо
на залишках палива через палаюче місто на Портовське, приєднаємось до колони і
потім 20 годин в холодній машині (економили бензин) до Бердянська.

\iusr{Натали Кина}

15 багато хто їхав навмання, спроби до були, але поодинокі. Я не сподівалась
виїхати, але 14 зі мною заговорила про це сусідка (дім не мій, сусіди по суті
незнайомці). Домовились на ранок 15, потім вона засумнівалась, бігали шукати
зв'язок, дзвонили, але майже нічого не чути, смс надсилали до своїх, щоб
дізнатись хоч щось. А там так само \enquote{чекають}. Чутка про коридор була непевною.
Тож вважаю, що колонна тоді зібралась із самих рішучих, хто мав бодай якийсь
транспорт. А 16 вже на вільній стороні отримали страшну новину про Драму ..

\begin{itemize} % {
\iusr{Ganna Kotelnikova}
\textbf{Натали Кина}, так само. Пам'ятаю, як прочитала про те, що на театр скинули бомбу, не могла повірити, це був шок. Як таке може бути? Напис ДІТИ не могли не бачити....досі в голові не вкладається 💔

\iusr{Oksana Tereshchenko}
\textbf{Ganna Kotelnikova} в мене є люди, які там були. Та встигли виїхати. Десь півтори тисячі не встигло...

\iusr{Натали Кина}
\textbf{Ganna Kotelnikova} ми ще всю ніч писали Діти на машинах...
\end{itemize} % }

\iusr{Alla Yali}

Слава богу, приняла решение и выехала ❤️С родителями и собаками.

Жизнь продолжается и очень приятно видеть твою улыбку, твои большие дела💪 и позитивный настрой 🫠

\begin{itemize} % {
\iusr{Ganna Kotelnikova}
\textbf{Alla Yali}, да довелося приймати рішення за всіх, брати відповідальність за родину, і це було вірне рішення.

\iusr{Александр Соловчук}
\textbf{Alla Yali} собаку забрали 👍
\end{itemize} % }

\iusr{Юлія Кальницька}

З днем народження

\begin{itemize} % {
\iusr{Ganna Kotelnikova}
\textbf{Юлія Кальницька}, дякую.
\end{itemize} % }

\iusr{Наталія Беспалова}

15 наш останній день у Маріуполі. 16 вранці ми поїхали у нікуди.. хоча, ще напередодні не думали про це..

\iusr{Ангеліна Шостак}

Як добре, що вдалось🙏

Твій біль за місто можу, мабуть, уявити.

\iusr{Татьяна Боднарь}

Я працювала в театрі 15 років..

Сиділа в підвалі свого дому (на вул. Зелінського, біля літака)

І, коли нас нещадно бомбили, не припиняючи, з 9-го березня, дуже переймалася тим, що не пішла в драмтеатр..

Нерви настільки здавали, що навіть думала бігти туди..

Не судилося. Врятувалася.

Вийшла 24 березня..

Мій рідний театр😪

\begin{itemize} % {
\iusr{Ganna Kotelnikova}
\textbf{Татьяна Боднарь}, добре, що не пішли.
\end{itemize} % }

\iusr{Петренко Тимофій}

Ми виїхали 17 березня — семеро людей, з них двоє дітей, дві кішки, мінімум речей на легковому авто.

\iusr{Anya Miller}

Какое счастье что вы приняли это решение и уехали. Вы живы и это главное!

\begin{itemize} % {
\iusr{Ganna Kotelnikova}
\textbf{Anya Miller}, да. Дякую вам за допомогу ❤️ Не знаю, що б ми без вас робили.

\iusr{Olena Mladenova}

Мои дорогие Анечки...

\iusr{Ganna Kotelnikova}
\textbf{Olena Mladenova}, ти наша дорога Оленочка! ❤️
\end{itemize} % }

\iusr{Lora Koposova}

Добре, що вибрались з пекла!
