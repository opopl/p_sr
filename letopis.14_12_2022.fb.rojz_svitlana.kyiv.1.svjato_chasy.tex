% vim: keymap=russian-jcukenwin
%%beginhead 
 
%%file 14_12_2022.fb.rojz_svitlana.kyiv.1.svjato_chasy
%%parent 14_12_2022
 
%%url https://www.facebook.com/svetlanaroyz/posts/pfbid0paoi2pg5CWXDEW3onHFVDuKQP2DQ2b7qTYUEiU5gXAqtfTS1nEaAE5K3azY9zcQGl
 
%%author_id rojz_svitlana.kyiv
%%date 
 
%%tags 
%%title Свято має бути в будь-які часи!
 
%%endhead 
 
\subsection{Свято має бути в будь-які часи!}
\label{sec:14_12_2022.fb.rojz_svitlana.kyiv.1.svjato_chasy}
 
\Purl{https://www.facebook.com/svetlanaroyz/posts/pfbid0paoi2pg5CWXDEW3onHFVDuKQP2DQ2b7qTYUEiU5gXAqtfTS1nEaAE5K3azY9zcQGl}
\ifcmt
 author_begin
   author_id rojz_svitlana.kyiv
 author_end
\fi

Свято має бути в будь-які часи! Ми з
\href{https://www.facebook.com/jenia.bogachyk}{Evgenia Matsibora}  думали, як
втілити наш проект \enquote{дзвінок свМиколаю} цього року. Більшість наших  \enquote{Миколаїв},
з якими ми разом з
\href{https://www.facebook.com/biosphere.corporation}{Biosphere Corporation}
влаштовували свято для дітей два роки поспіль, зараз  створюють справжні дива
заради нас на Фронті. А дітям та дорослим, навіть більше, ніж в минулі роки,
потрібен дозвіл на відчуття радості. І традиції неможна переривати - контакт,
близькість, надія зараз терапевтичні. 

\ii{14_12_2022.fb.rojz_svitlana.kyiv.1.svjato_chasy.pic.1}

Ми вигадали невеличкий Чат-бот. В тому числі, щоб і діти, які в Україні, і
діти, які зараз за кордоном, могли доторкнутись до дива.

\url{https://t.me/StNicholasbot}

Як побудовано спілкування в чат-боті:

1. Ми пояснюємо  дітям, чому цього року телефонний зв'язок неможливий, записали
привітання від нашого героя (герой, що виступає в ролі Миколая - одягнутий і
знятий так, щоб у дітей не було прив'язки к конкретному образу), далі ми
скеровуємо на текст в листівці, що може допомогти і дорослим знайти слова для
розмови з дитиною.

І обережно надаємо підтримку емоціям в короткому спілкуванні

2. Важливо попередити дитину, що це буде дуже короткий контакт і у неї не буде
можливості в цьому чат-боті написати про своє бажання - на цьому важливо
наголосити, щоб уникнути розчарування. 

3. Спілкування завершується пропозицією активності - будь ласка, запропонуйте
дитині зробити щось із того, що там прочитаєте. 

В найскладніші часи ми маємо зберігати контакт з життєвістю та радістю.
Розуміємо, що чат-бот не досконалий, він не замінить спілкування, як в минулі
роки. Але він створений із турботою та любов'ю, і сподіваємося, додасть дітям
та дорослим радості. 

Спасибі Євгенії - ви б бачили, як ми по ночах ловили мережу, щоб синхронізувати
дії і скільки всього Женя зробила!, Саша Марк  - він уособлює Миколая, і Andriy
Zdesenko  за цю можливість. 

Перші відгуки від дітей та дорослих (вони теж приєднуються) - прекрасні
@igg{fbicon.heart.red}

Посилання на Чат-бот тут \url{https://t.me/StNicholasbot}

Можна зайти по кюар коду, що на картинці. 

Обіймаю, Родино @igg{fbicon.heart.red} як вірю в Чудеса. І як впевнена в
Перемозі.

\ii{14_12_2022.fb.rojz_svitlana.kyiv.1.svjato_chasy.orig}
\ii{14_12_2022.fb.rojz_svitlana.kyiv.1.svjato_chasy.cmtx}

