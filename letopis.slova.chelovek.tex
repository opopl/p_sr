% vim: keymap=russian-jcukenwin
%%beginhead 
 
%%file slova.chelovek
%%parent slova
 
%%url 
 
%%author 
%%author_id 
%%author_url 
 
%%tags 
%%title 
 
%%endhead 
\chapter{Человек}
\label{sec:slova.chelovek}

%%%cit
%%%cit_pic
%%%cit_text
Самое дорогое - это \emph{Люди}
%%%cit_comment
%%%cit_title
\citTitle{Желая быть ангелом, превращаемся в демона}, 
Протоиерей Андрей Ткачёв, youtube.com, 04.06.2021
%%%endcit

У диплоїдних організмів, таких, як \emph{Людина}, присутні копії майже всієї
ДНК.  Зокрема, є пари хромосом по одній від кожного з батьків, однак для
дослідників цінність представляє сума тільки однієї копії пар хромосом.
Фактично, розмір геному \emph{Людини} являє собою приблизно 6 мільярдів пар
основ, однак вчені використовують тільки половину в якості
\enquote{репрезентативної} копії.  Вчені зазначили, що 8\% геному пропустили у
2003 році не через відсутність в них важливої інформації, а через технологічні
обмеження. Дослідники наголосили, що ці обмеження вдалося подолати завдяки
секвенуванню, яке розробили вчені з Pacific Biosciences і Oxford Nanopore,
\textbf{Вчені повністю розшифрували геном людини}, day.kiev.ua, 03.06.2021

В жизни любого \emph{человека} есть, как хорошие, так и отрицательные моменты.
Существует заблуждение, что психотерапия направлена на то, чтобы сделать ее
протекание счастливым и беззаботным. Но на самом деле, специалисты в этой сфере
работают над тем, чтобы последствия испытанного стресса, страха, неудач не
мешали вам в дальнейшем. Например, если вы не можете устроиться на хорошую
работу – психотерапевт не поможет. Но, если вы боитесь ходить на собеседования,
испытываете трудности в коммуникации с новыми людьми, рушите связи из-за
конфликтного характера – вполне,
\textbf{Для чего нужна психотерапия личности?}, beztabu.net, 02.06.2021

Фаина ищет и находит простые слова, образы, сюжеты. У нее, в отличие от
современных потребителей, болит душа. Не слышать ее может лишь тот, у кого души
нет. Для кого патриотизм это не жертвенность во имя страны, а способ
обогащения, для кого война только прибыльный бизнес, для кого своя рубашка
ближе к телу, а хата с краю.  Послушайте ее обращение в ООН. Никакой
театральности, надрыва, пафоса. Она, скорее, робеет, чем кого-то обвиняет,
требует. Ей как-то неудобно, ведь она понимает, что дети не помнят мира, а
большие дяди в политике не понимают. Точнее, все эти дяди понимают, но просто
дяди негодяи, и заняты другим. А ей, Фаине, за этих дядей неловко и стыдно.
Хотя стреляют не по им, а по ней. Вот это и есть крест настоящего
\emph{Человека}, испытывать стыд за чужую подлость. Маленький \emph{Человек},
но большая душа,
\textbf{Гибель людей на Донбассе говорит о том, что наше общество больное},
Денис Жарких, strana.ua, 04.06.2021

Ведь важно не то, что плетут политики, а важно, чтобы не стреляли, чтобы \emph{Люди}
не погибали, чтобы мы опять стали \emph{Людьми}, а не чинами и капиталами. Общество
больно, его покидает \emph{Человечность}. Очевидно, прохудилась общественная совесть.
Как вы думаете?,
\textbf{Гибель людей на Донбассе говорит о том, что наше общество больное},
Денис Жарких, strana.ua, 04.06.2021

Как жаль, как ужасно жаль, что не существует больше такого \emph{Человека}... Он
остался в прошлом. Он продолжает жить лишь в книгах, фильмах, рассказах наших
дедов...  \emph{Человек} утратил свое величие, и произошло это не в одно мгновение.
Ныне мы живем при капитализме, и нет больше тех великих идеалов - да вообще
никаких идеалов нет, а есть лишь личный интерес, не выходящий за рамки
материального: тепла, сытости и комфорта. Мы превратились в потребителей. Идея
низвергнута. У нас - тех, кто был когда-то велик как единый народ - ныне
раздор, вражда и отчужденность. Мы думаем по-другому, поступаем по-другому. Мы
извратили многие понятия. Мы упрекаем и обвиняем друг друга, завидуем и злимся,
предъявляем взаимные счеты. Мы больше не \enquote{Величество}...  Мы больше не братья.
Мы живем в разных государствах, между нами пролегли границы. Мы отдаляемся друг
от друга с огромной скоростью... ,

Каждый \emph{человек} — это целый огромный мир в себе, это сложная загадка с долгим
прошлым и бесконечным будущим, 

То що ж сталося у жовтні 2018 року? Можна з впевненістю стверджувати, що
«Портрет Едмунда Беламі» не має «фіксованого» автора, але має власників. Поміж
численними співавторами, які створили це зображення, має бути введена на рівних
нежива субстанція, якою є алгоритм. Тобто 2018 року у світ «високого мистецтва»
увійшла нова гібридна авторська сутність – \emph{людино-машина}. Це сталося
цілком очікувано. Технології, які значною мірою змінили навколишній світ, тепер
сфокусовані на \emph{Людині} і на всьому живому на планеті. Ми стоїмо перед
питаннями: Чи дійсно \emph{Людина} звільнилася від Бога і сама може впливати на
все живе? Чи може \emph{Людина} потрохи зникне на корить \emph{людино-машини}?
Чи не стане \emph{людське} життя лише виконанням програми, яку написали десь, у
науковій лабораторії? Яким буде поняття/розуміння \emph{Людини} в майбутньому?,
\citTitle{Мистецтво в добу «привидів у машинах»}, Богдан Шумилович, zaxid.net, 02.06.2020

Облишимо культурно зумовлені факти. Маркес, Борхес і Льйоса змусили мене
задуматися над іншим. Власне, над тим моментом зрілості, коли сам обираєш, як
ти називаєшся. Навіть прийнявши заготовлені родом варіанти, \emph{людина} має свободу
вибору. Хтось живе як Толік, хтось — як Анатолій Васильович, хтось взагалі стає
Анатоль. Кілька літер, додаткових звуків і відразу зовсім інший портрет. Навіть
те, як людина вимовляє власне ім'я, повільно й поважно, чи квапливо і голосно,
скаже чимало про власника і його ментально-душевний стан.  Напевне, в
дорослості, ми всі розуміємо — немає красивих і некрасивих імен. Є особистості,
що своїм досвідом і характером перетворили власне ім’я на символ. Символ сили
чи слабкості, перемоги чи підлості,
\citTitle{Як ти називаєшся?}, Анна Данильчук, day.kyiv.ua, 10.06.2021

%%%cit
%%%cit_head
%%%cit_pic
%%%cit_text
С минимальным наборов навыков уже можно идти во власть.  Очень обидно, что мы в
современном мире так быстро дошли до точки, когда \emph{человеку}, чтобы выжить,
вообще ничего не обязательно.  Ну, разве что держать ложку и уметь застегивать
ширинку. С этим набором умений уже можно придти к власти.  Арифметика - да
нахер ее арифметику, литература - сложно, гуманитарные знания - о чем это,
права \emph{человека} - заумно.  Прямохождение - ок навык, социальная лестница тебя
ждет, \emph{человек 21-го столетия}. Моргай, маши руками, топай, скоро даже кашлять
будет можно, но не всем. Победа будет за нами, одомашненные зверьки
%%%cit_comment
%%%cit_title
\citTitle{Мы дошли до точки, когда человеку, чтобы выжить, вообще ничего не обязательно / Лента соцсетей / Страна}, 
Евгений Ихельзон, strana.ua, 24.06.2021
%%%endcit

%%%cit
%%%cit_head
%%%cit_pic
%%%cit_text
Ну а лидером списка стал \emph{человек}, не имеющий ни украинского гражданства, ни
украинских корней. Это Кристина Квин – даже не посол, а временный поверенный в
делах США на Украине. Ее лидерство – констатация, что страна находится под
внешним управлением. Только в таких условиях дипломат может быть влиятельнее
генпрокурора
%%%cit_comment
%%%cit_title
\citTitle{Список самых влиятельных женщин стал позором Украины}, 
Станислав Борзяков, vz.ru, 26.06.2021
%%%endcit

%%%cit
%%%cit_head
%%%cit_pic
%%%cit_text
Ну, а если и примет, то только лишь для нашего общего блага. Потому что мы не
только правовое, но и социальное государство, где \emph{человек} есть высшая
ценность. ( статья 3). И ради этого \emph{человека} можно немного нарушить
Конституцию или даже узурпировать власть Всем, конечно, нельзя, но Президенту
можно. Он ведь всенародно избранный! А народ наш мудр! С днем Конституции!
%%%cit_comment
%%%cit_title
\citTitle{Конституция Украины содержит целый ряд замечательных положений / Лента соцсетей / Страна}, 
Андрей Головачев, strana.ua, 28.06.2021
%%%endcit

%%%cit
%%%cit_head
%%%cit_pic
%%%cit_text
Также он рассказал, что сегодня на предприятии числится 1580 \emph{человек}, но
реально работает около сотни специалистов.  \enquote{Сейчас мы выводим около
400 \emph{человек} работающих на заводе. Это \emph{люди} разного возраста. Это
обслуживающий персонал, охрана, работники жилищно–коммунальной сферы и около
ста производственников, работающих над задачей}, - сообщил Александр Кривоконь.
\enquote{А другая часть - более тысячи \emph{человек} сегодня находятся на
простое.  Завод у нас находится на простое. Мы работаем официально 4 часа в
неделю}, — разъяснил он
%%%cit_comment
%%%cit_title
\citTitle{ХГАПП больше не может производить самолеты}, Антон Щукин, strana.ua, 27.06.2021
%%%endcit

%%%cit
%%%cit_head
%%%cit_pic
%%%cit_text
Говорим на украинском, но находимся все там же.  За темой русского наследия,
русской культуры, которые якобы транслируются через русский язык независимо от
контента, ускользает вопрос прав \emph{человека}.  Закон про функционирование
государственного языка противоречит Всеобщей декларации прав \emph{человека} и нашей
Конституции. И главное - он ничего не меняет в общественно-политических, в
экономических отношениях. Т.е. нельзя сказать - вот до этого закона было так, а
после принятия закона стало вот так. Ничего не стало. Ничего не изменилось
%%%cit_comment
%%%cit_title
\citTitle{В плане развития страны увеличение числа украиноговорящих ничего не дает / Лента соцсетей / Страна}, 
Павел Себастьянович, strana.ua, 27.06.2021
%%%endcit

%%%cit
%%%cit_head
%%%cit_pic
%%%cit_text
Какое отношение Зеленский к дорогами имеет? А к земле? Никакого абсолютно! Но
он там, потому что требует запад и потому что там деньги. А языки и тарифы
\emph{людям} — причём тут Зеленский, да?  Не надо на людей перекладывать.
\emph{Люди} вас избрали в надежде, что исполните свои обещания. Но вы
исполняете желания олигархов, МВФ и внешнего управления, а на \emph{людей} вам
плевать. Но \emph{люди} у вас не такие, конечно же. А вы хорошие там все
%%%cit_comment
%%%cit_title
\citTitle{Самое простое для власти - переложить ответственность на рядовых людей / Лента соцсетей / Страна}, 
Александр Скубченко, strana.ua, 02.07.2021
%%%endcit

%%%cit
%%%cit_head
%%%cit_pic
%%%cit_text
Французский суд осудил вчера Брижит Бардо - потому что она назвала оппонента
\enquote{\emph{недочеловеком}}. А Олег Семинский, депутат украинского парламента от
партии президента Зеленского, открытым текстом пишет такое:
\enquote{Русские на законодательном уровне - не коренной народ, поэтому они не будут
иметь возможность в полной мере обладать всеми правами \emph{человека} и
основополагающими свободами, определенными нормами международного права, а
также предусмотренными в Конституции и законах Украины. В отличие от украинцев,
крымских татар, караимов и крымчаков}.
В принципе, он может сразу призывать к геноциду - никакого серьезного наказания
за это все равно не последует
%%%cit_comment
%%%cit_title
\citTitle{Семинский может открыто призывать к геноциду - все равно не накажут / Лента соцсетей / Страна}, Андрей Манчук, strana.ua, 03.07.2021
%%%endcit

%%%cit
%%%cit_head
%%%cit_pic
%%%cit_text
Построение \emph{человечеством} ужасающей антиутопии, от рабства в которой людей
спасет только второе пришествие Господа – один из центральных моментов
христианского учения. Знаки и приближение, неминуемое наступление апокалипсиса
христианское общество наблюдало и предсказывало на протяжении всех двух
тысячелетий со времени распятия Христа. И тем не менее каждый кризис, падение
христианской цивилизации оказывались лишь началом очередного пути ее развития.
В настоящий момент мирового кризиса христианства примеров которому трудно и
вспомнить, у России еще есть возможность реализовать себя в избранной и
провозглашенной миссионерской роли Третьего Рима, и столицы Империи Духа.  А
удастся ли ей это, даже не в руках и воле \emph{Человеческой}, но в Божиих
%%%cit_comment
%%%cit_title
\citTitle{Революция Духа – единственный путь спасения России}, 
Юрий Барбашов, voskhodinfo.su, 30.06.2021
%%%endcit

%%%cit
%%%cit_head
%%%cit_pic
%%%cit_text
Основной привлекательной чертой западного пути развития является предлагаемое
упрощение \emph{человека}, избавление его от тягот, труда по саморазвитию,
самосовершенствованию, перипетий социальной борьбы и даже трудностей реализации
в гендерной роли. Для любого живого существа привлекателен путь получения
минимального количества ресурсов и благ с минимизацией усилий. В конце концов,
кратчайшим путем достижения неограниченного удовлетворения и счастья отдельного
\emph{человека} является вживление электродов в мозг и раздражение непосредственно
центров удовольствия или центров выработки гормонов. С большой вероятностью,
что именно к такому способу «вознаграждения» и придет в скором времени западное
общество, вместо грубой замены тех же процессов доступным сексуальным
удовлетворением, алкоголем и наркотиками
%%%cit_comment
%%%cit_title
\citTitle{Революция Духа – единственный путь спасения России}, 
Юрий Барбашов, voskhodinfo.su, 30.06.2021
%%%endcit
