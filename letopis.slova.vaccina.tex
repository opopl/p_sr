% vim: keymap=russian-jcukenwin
%%beginhead 
 
%%file slova.vaccina
%%parent slova
 
%%url 
 
%%author 
%%author_id 
%%author_url 
 
%%tags 
%%title 
 
%%endhead 
\chapter{Вакцина}
\label{sec:slova.vaccina}

%%%cit
%%%cit_head
%%%cit_pic
%%%cit_text
Результаты такого подхода возвращаются сейчас бумерангом. Нищая перенаселенная
Индия стала идеальным инкубатором для формирования новых, более заразных
штаммов, которые уже пришли в Европу и США, провоцируя новую волну пандемии.
Очевидно, что такая же угроза потенциально исходит с территории других бедных
стран, с высоким уровнем населения и низкой доступностью медицинского
обслуживания – как в государствах Африки, Юго-Восточной Азии, Ближнего Востока
и Латинской Америки. Для победы над коронавирусом необходимо делать упор на
\emph{вакцинировании} жителей Третьего мира, предоставив им свободный доступ к
качественным \emph{вакцинам} – вместо того, чтобы делать из этого привилегию для
европейцев, американцев и их ближайших союзников
%%%cit_comment
%%%cit_title
\citTitle{Для победы над коронавирусом надо было вакцинировать Третий мир / Лента соцсетей / Страна}, 
Андрей Манчук, strana.ua, 24.06.2021
%%%endcit

%%%cit
%%%cit_head
%%%cit_pic
%%%cit_text
Для реализации этой схемы фармацевтические корпорации вытесняли с
международного рынка конкурентов, в лице \emph{вакцины} «Спутник V» и препаратов из
КНР – причем, эту дискриминацию зачастую объясняют причинами политического
характера. В то же время, Европа и США отказывались приостановить действие
патентов на свои \emph{идеологически правильные вакцины} – чтобы их можно было
свободно производить в Третьем мире. Прежде всего, потому, что это лишило бы
корпоративный менеджмент запланированных доходов
%%%cit_comment
%%%cit_title
\citTitle{Для победы над коронавирусом надо было вакцинировать Третий мир / Лента соцсетей / Страна}, 
Андрей Манчук, strana.ua, 24.06.2021
%%%endcit

