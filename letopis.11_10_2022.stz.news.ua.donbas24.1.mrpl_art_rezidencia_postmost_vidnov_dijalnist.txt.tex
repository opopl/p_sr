% vim: keymap=russian-jcukenwin
%%beginhead 
 
%%file 11_10_2022.stz.news.ua.donbas24.1.mrpl_art_rezidencia_postmost_vidnov_dijalnist.txt
%%parent 11_10_2022.stz.news.ua.donbas24.1.mrpl_art_rezidencia_postmost_vidnov_dijalnist
 
%%url 
 
%%author_id 
%%date 
 
%%tags 
%%title 
 
%%endhead 

Ольга Демідко (Маріуполь)
11_10_2022.olga_demidko.donbas24.mrpl_art_rezidencia_postmost_vidnov_dijalnist
Маріуполь,Україна,Мариуполь,Украина,Mariupol,Ukraine,Художник,Artist,date.11_10_2022

Маріупольська арт-резиденція PostMost відновлює свою діяльність (ФОТО)

Маріупольські художники продовжують свою діяльність в Німеччині

У День художника, 9 жовтня, відновила свою діяльність арт-резиденція Маріуполя
PostMost. Про це на своїй сторінці у Facebook розповіла маріупольська
художниця, яка наразі перебуває у Німеччині, Олена Украінцева.

«PostMost живий! І ми починаємо. Художниці з Маріуполя, які втекли від війни,
знайшли сили і можливість зібратися, щоб продовжити творити про Маріуполь, наш
Маріуполь, про нас. Ми живі та ми продовжуємо малювати і розповідати про наше
чудове місто, яке було і яке буде», — розповіла Олена.

Читайте також: Маріупольські митці у столиці покажуть силу та красу міста
Марії: де подивитися

Кому належить ідея відновити проєкт?

Ідея про продовження проєкту належить Олені Украінцевій. У травні вона стала
учасницею резиденції ArToll в Бедбург-Хау (землі Північна Рейн-Вестфалія) разом
з німецькими художниками. Історії про Маріуполь, його людей та художників, які
любили, берегли та покращували своє місто, дуже зворушили їх. Саме тоді Олені
запропонували організувати резиденцію із українськими художниками. Так і
з’явилася можливість реалізації ідеї.

Нагадаємо, що перша резиденція PostMost «Маріуполь. Місто морське» відбулася у
липні 2021 році і була присвячена Азовському морю. До Маріуполя завітали 9
художників з Києва, Львова та Одеси. Перед художниками стояла одна мета — за 9
днів перебування тут ознайомитися з містом та показати його у своїй фінальній
роботі. Виставка пройшла в арт-просторі Маріуполя «Гастроном» Центру сучасного
мистецтва «Континенталь». 9 резидентів презентували близько 70 робіт. Картини,
фотографії, інсталяції, графічні роботи та навіть мозаїка — і всі вони були
присвячені Маріуполю.

Читайте також: Маріупольський щоденник: прем'єра докфільму та картина на
американському аукціоні Bonhams

Хто тепер працюватиме в арт-резиденції?

До проєкту долучилися художниці, які, рятуючись від війни, виїхали до ЄС. До
війни вони вже були учасницями проєкту PostMost. Це українські художниці з
Маріуполя:

Олена Украінцева, Антоніна Дурнєва, Hilmena, Simona, Марина Черепченко,
Зугайрат Новікова; Інна Абрамова (художниця з Донецька) та художниця з
Німеччини — Іветта Бекер, яка дуже підтримує українських біженців.

Темою проєкту залишається Маріуполь (який є малою Батьківшиною художниць) та
загалом вся Україна.

«Звичайно, всі роботи пронизані війною, кожна художниця пише свій особистий
життєвий досвід за останні 7 місяців», — підкреслила кураторка резиденції Олена
Украінцева.

Читайте також: Художник Денис Метелін долучився до благодійного проєкту зі
збору коштів для ЗСУ (ВІДЕО)

Де можна буде побачити роботи художниць?

Роботи презентуватимуть безпосередньо в резиденції ArToll в Бедбург-Хау 22
жовтня. Потім виставка буде перевезена до Дортмунду у шахту-музей промисловості
Zeche Zollern, де 28 жовтня відбудеться презентація експозиції. Там роботи
пробудуть до весни. Також у Дортмунді до виставки додадуться принти робіт
Анастасії Шишкіної та Олександра Малаховського, які перебувають наразі в
Україні та не мають можливості особисто взяти участь у проєкті.

Раніше Донбас24 розповідав, як у Німеччині створюються футболки, присвячені
Маріуполю.

Ще більше новин та найактуальніша інформація про Донецьку та Луганську області
в нашому телеграм-каналі Донбас24.

ФОТО: з особистого архіву Олени Украінцевої.
