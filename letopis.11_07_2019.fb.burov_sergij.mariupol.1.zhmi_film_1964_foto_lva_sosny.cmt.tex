% vim: keymap=russian-jcukenwin
%%beginhead 
 
%%file 11_07_2019.fb.burov_sergij.mariupol.1.zhmi_film_1964_foto_lva_sosny.cmt
%%parent 11_07_2019.fb.burov_sergij.mariupol.1.zhmi_film_1964_foto_lva_sosny
 
%%url 
 
%%author_id 
%%date 
 
%%tags 
%%title 
 
%%endhead 

\qqSecCmt

\iusr{Елена Аврамова}

Сергей Давыдович, это Вы в центре за столом? Или я ошибаюсь?

\begin{itemize} % {
\iusr{Сергей Буров}

Нет. Это -я. Сейчас чуть изменился.

\iusr{Елена Аврамова}

Совсем немного, я же узнала!

\iusr{Павнл Людин}
Сергей Давыдович, чуть изменились спереди или сзади?!
\end{itemize} % }

\iusr{Мария Долгая}

Вот интересно было бы пофамильно, кто где?

\begin{itemize} % {
\iusr{Мария Долгая}

Спасибо большое.

Фотопортрет делали, скорее всего, в фотографии на проспекте, ниже к/т Победа?
Если не ошибаюсь Лев Сосна в 64-м там работал, а потом перешел в \enquote{Сказку}.

А сама киностудия ЖдМИ фильм, располагалась в цокольном помещении студклуба на Казанцева или в 60-е другом месте?

\iusr{Сергей Буров}

Киностудия располагалась в подвале общежития, в котором с 60-го устроили
учебный корпус. Если не изменяет память. там был иняз. термообработка и еще
какие-то кафедры

\iusr{Мария Долгая}

Да, это 2-ой корпус. Там кафедры иняза, термообработки, электроснабжения, охраны труда, машиностроения.

А в 70-х студия уже была в цокольном помещении студклуба на Казанцева, на противоположной стороне (от 2-го корпуса).

\end{itemize} % }

\iusr{Ирина Курбатова}

ага, мне тоже интересно!

\iusr{Natalya Dedova}

Шикарное фото! И лица! ❤️

\iusr{Irina Papirova}

Such a beautiful picture! Everybody is so good looking! Great time!

\iusr{Janina Skaskiewicz}

Сергей Давыдович, вы ни капельки не изменились, только цвет волос.

\iusr{Lydmila Zhukovets}

Да Сосна был хорошим фотографом он мне свадебное фото делал

\iusr{Tatiana Grechina}

какие все молодые........
