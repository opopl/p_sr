% vim: keymap=russian-jcukenwin
%%beginhead 
 
%%file 27_01_2023.stz.news.ua.donbas24.1.gannover_premjera_teatromania_20
%%parent 27_01_2023
 
%%url https://donbas24.news/news/u-gannoveri-vidbulasya-premjera-dokumentalnoyi-vistavi-teatromaniyi-20
 
%%author_id demidko_olga.mariupol,news.ua.donbas24
%%date 
 
%%tags 
%%title У Ганновері відбулася прем'єра документальної вистави "Театроманії 2:0"
 
%%endhead 
 
\subsection{У Ганновері відбулася прем'єра документальної вистави \enquote{Театроманії 2:0}}
\label{sec:27_01_2023.stz.news.ua.donbas24.1.gannover_premjera_teatromania_20}
 
\Purl{https://donbas24.news/news/u-gannoveri-vidbulasya-premjera-dokumentalnoyi-vistavi-teatromaniyi-20}
\ifcmt
 author_begin
   author_id demidko_olga.mariupol,news.ua.donbas24
 author_end
\fi

\ii{27_01_2023.stz.news.ua.donbas24.1.gannover_premjera_teatromania_20.pic.front}

\begin{center}
  \em\color{blue}\bfseries\Large
22 січня на сцені ListerTurm у Ганновері Театральна
студія \enquote{Театроманія 2:0} представила прем'єру вистави \enquote{Крила}
\end{center}

\href{https://archive.org/details/06_09_2022.olga_demidko.donbas24.teatromania_prodovzhue_rozv_pidtrym_kulturu_mrpl}{\emph{Театральна студія \enquote{Театроманія 2:0}}}%
\footnote{\enquote{Театроманія} продовжує розвивати та підтримувати культуру Маріуполя, Ольга Демідко, donbas24.news, 06.09.2022, %
\par\url{https://donbas24.news/news/teatromaniya-prodovzuje-rozvivati-ta-pidtrimuvati-kulturu-mariupolya}, \par%
Internet Archive: \url{https://archive.org/details/06_09_2022.olga_demidko.donbas24.teatromania_prodovzhue_rozv_pidtrym_kulturu_mrpl}%
} — сучасний молодий театр на чолі з
маріупольським режисером \emph{\textbf{Антоном Тельбізовим}}, який завжди знаходиться у пошуку
нових форм та засобів діалогу з глядачем. Нещодавно \enquote{Театроманія 2:0}
підготувала нову прем'єру спектаклю \enquote{Крила}. Ця документальна вистава
розповідає про любов кожного актора до свого рідного міста і відповідає на таке
важливе життєве питання сьогодення, як, наприклад, суспільство має ставитись до
війни. Водночас ця вистава покликана дати відповідь на те, як молоді біженці
справляються із травмою, спричиненою війною, і стають сильнішими після неї.

\ii{insert.read_also.demidko.donbas24.vid_papirusu_do_gadzhetu_sladkovy}

\subsubsection{Про Театральну студію \enquote{Театроманія 2:0}}

Театральна студія \enquote{Театроманія 2:0} почала працювати після повномасштабного
вторгнення рф в Україну. Це творчий осередок української молоді від 12 до 20
років, які опинилися в Німеччині під час військових дій в Україні. Студія
працює на базі Української Спілки Нижньої Саксонії у місті Ганновер. Керівником
студії є Тельбізов Антон. Протягом 11 років він є режисером Народного театру
\enquote{Театроманія}, який було засновано у 2011 році у Маріуполі. Основний напрям
роботи — опрацювання сучасних проблем у театральних постановках та розробка
соціальних проєктів. У репертуарі театру камерні, музичні та драматичні
вистави.

\ii{27_01_2023.stz.news.ua.donbas24.1.gannover_premjera_teatromania_20.pic.1}

\subsubsection{Про документальну виставу \enquote{Крила}}

Ідея поставити цю виставу прийшла до режисера спонтанно. Антон Тельбізов почав
працювати з молоддю від 12 до 18 років і зрозумів, що дехто з них ніколи не був
на сцені.

\begin{leftbar}
\emph{\enquote{Я вирішив дати їм тему для самостійної роботи — \enquote{Любов до Батьківщини}. Був
вражений, наскільки щирі та чутливі монологи їм вдалося створити. Саме тоді я і
зрозумів, що ця тема стане нашою новою виставою}}, — розповів керівник студії та
режисер Антон Тельбізов.
\end{leftbar}

\ii{insert.read_also.demidko.donbas24.jaki_teatr_proekty_mrpl_zakordon_mytci}
\ii{27_01_2023.stz.news.ua.donbas24.1.gannover_premjera_teatromania_20.pic.2}

Над виставою колектив почав працювати у жовтні 2022 року. Поставити її вдалося
всього за 64 години. Зустрічалися 4 місяці, двічі на тиждень.

\begin{leftbar}
\emph{\enquote{Бажання молодих акторів було настільки великим, що вони легко впорались з поставленою метою}}, — підкреслив режисер.
\end{leftbar}

Дехто з акторів вперше вийшов на сцену, тому ця прем'єра була дуже хвилюючою
для них. Загалом виступило 18 акторів з різних міст України (Київ, Одеса,
Житомир, Луганськ тощо). Назва вистави — \enquote{Крила} — пов'язана з домом, адже,
коли думаєш про свій дім і розумієш, наскільки він далеко, хочеться миттєво
опинитися в рідних стінах, та відчути себе вдома.

\begin{leftbar}
\emph{\enquote{Спогади про дім набувають нового сенсу. Ми згадуємо про найдрібніші речі зі
свого життя: і про улюблену квітку, і про вулицю, і про світло у
сусідньому будинку. Все це зараз сприймається інакше. Нам так хочеться
додому, що ми нібито на КРИЛАХ летимо зі швидкістю світла, щоб вже
ніколи не відпускати своє мирне життя}}, — розповів Антон Тельбізов.
\end{leftbar}

\ii{insert.read_also.demidko.donbas24.mariupolci_vrazyly_kyjan}
\ii{27_01_2023.stz.news.ua.donbas24.1.gannover_premjera_teatromania_20.pic.3}
\ii{27_01_2023.stz.news.ua.donbas24.1.gannover_premjera_teatromania_20.pic.4}

Вистава включає різноманітні акторські вправи театральної сту\hyp{}дії \enquote{Театроманія
2:0}. У ній молоді актори змогли показати все, чому вони навчились протягом 7
місяців. Йдеться і про акторську майстерність, і сценічну мову, сценічний рух,
ритміку та співи. Але головна мета — це показати ставлення людини до своєї
Батьківщини та свого міста, в якому кожен зростав та розвивався. Місто стає
відображенням людського характеру та звичок. Воно несе в собі сліди наших
предків, які жили в ньому, стає пам'яткою історії нашої родини. Наш рідний дім
є невід'ємною частиною нашого життя. Вистава йшла українською та німецькою
мовами. Коли актори говорили українською, з'являлися німецькі титри, коли
німецькою — навпаки, українські.

\ii{27_01_2023.stz.news.ua.donbas24.1.gannover_premjera_teatromania_20.pic.5}

\subsubsection{Враження глядачів}

На цю виставу приїхали люди не тільки з Німеччини, а навіть з інших країн
Європи. Всі вони намагалися своєчасно зареєструватися, щоб побачити нову
виставу \enquote{Театроманії}. Приїжджали навіть, коли вже не було місць. Насправді,
квитки розібрали дуже швидко. За тиждень вже була закрита реєстрація. \emph{\textbf{Ольга
Самойлова}}, маріупольська актриса та режисерка Народного театру \enquote{Театроманія}
теж не могла не приїхати. Вона дуже хотіла підтримати і свого колегу Антона
Тельбізова, і весь колектив, який підняв таку важливу тему.

\begin{leftbar}
\emph{\enquote{На виставу приїхало дуже багато глядачів, наших маріупольців, наших
знайомих з інших міст України. Не тільки з інших міст Німеччини, але й
з Нідерландів, з Польщі. Було видно, що ця тема дуже болить. Глядачі
плакали. Це дуже важка вистава для всіх: і для тих, хто був на сцені, і
для тих, хто був у залі. Більш за все вразило те, що не тільки українці
так реагують, а й самі німці, які дуже близько все сприйняли до серця.
Такі спектаклі дуже важливі, адже вони нагадують про війну в Україні,
про яку за кордоном не завжди пам'ятають. Саме тому про це дуже важливо
говорити, зокрема, і зі сцени}}, — поділилася думками Ольга.
\end{leftbar}

\ii{insert.read_also.demidko.donbas24.u_kyevi_vidnovyly_mariupolsku_vystavu}
\ii{27_01_2023.stz.news.ua.donbas24.1.gannover_premjera_teatromania_20.pic.6}

У залі було 250 глядачів. Після вистави і режисера, і акторів німці не хотіли
відпускати протягом години. Вони були вражені, що діти змогли все це настільки
щиро прожити. Одразу ж вони почали шукати інформацію про Маріуполь та те, що
сталося з містом.

\begin{leftbar}
\emph{\enquote{Насправді для мене неважливо, що відбувається на сцені, важливо, що
відбувається саме в суспільстві. Тому вважаю, що моє надзавдання, яке я
собі ставив у цій виставі, було виконане повною мірою}}, — зазначив
Тельбізов.
\end{leftbar}

Цікаво, що після вистави і актори, і режисер відчули справжній спокій, адже
спектакль дає надію, що ти не один. Водночас актори навіть не зрозуміли до
кінця, який вплив вони мали на глядача, наскільки самовіддано грали. Думали, що
дехто з глядачів нездужав, тому чули нежить, а коли включили світло, вони
зрозуміли, що це був не нежить, а сльози глядачів.

\ii{insert.read_also.demidko.donbas24.chy_ljubyv_marko_kropyvnyckyj_priazovja_stvoren}
\ii{27_01_2023.stz.news.ua.donbas24.1.gannover_premjera_teatromania_20.pic.7}

\begin{leftbar}
\emph{\enquote{Сьогодні в Ганновері відбулася прем'єра документальної вистави \enquote{Крила}! Я хочу
подякувати режисеру Антону Тельбізову за його талант, за те що він робить для
наших дітей і для Перемоги!!! За те що він зробив неймовірну виставу, якою
можно пишатися і яка є справжньою зброєю в інформаційному просторі! За те, що
не було жодної людини в залі, яка б не прожила все, що відбувалося на сцені! Ми
всі крізь сльози і біль народилися знову і це неймовірне відчуття вдячності і
гордості за те, що ми Українці Цю виставу повинен побачити світ!}}, — наголосила
Надія Носовицька, одна з глядачок, яка приїхала до Німеччини з Дніпра.
\end{leftbar}

З документальною виставою \enquote{Крила} театральна студія \enquote{Театроманія 2:0} планує
гастролювати містами Німеччини вже найближчим часом.

\ii{27_01_2023.stz.news.ua.donbas24.1.gannover_premjera_teatromania_20.pic.8}

Раніше Донбас24 розповідав, що військовослужбовець ЗСУ \href{https://donbas24.news/news/viiskovosluzbovec-zsu-vidav-fotoknigu-prisvyacenu-mariupolyu}{\emph{видав фотокнигу, присвячену Маріуполю}}.%
\footnote{Військовослужбовець ЗСУ видав фотокнигу, присвячену Маріуполю, Ольга Демідко, donbas24.news, 11.01.2023, \par\url{https://donbas24.news/news/viiskovosluzbovec-zsu-vidav-fotoknigu-prisvyacenu-mariupolyu}}

Ще більше новин та найактуальніша інформація про Донецьку та Луганську області
в нашому телеграм-каналі Донбас24.

ФОТО: з особистого архіву Антона Тельбізова

\ii{insert.author.demidko_olga}
%\ii{27_01_2023.stz.news.ua.donbas24.1.gannover_premjera_teatromania_20.txt}
