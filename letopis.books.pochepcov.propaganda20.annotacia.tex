% vim: keymap=russian-jcukenwin
%%beginhead 
 
%%file books.pochepcov.propaganda20.annotacia
%%parent books.pochepcov.propaganda20
 
%%url 
 
%%author_id 
%%date 
 
%%tags 
%%title 
 
%%endhead 

\section{Аннотация}

\begin{qqquote}
Пропаганда присутствует в любом обществе и во все времена. Она может быть
политической, а может продвигать здоровый образ жизни, правильное питание или
моду. В разные исторические периоды пропаганда приходит вместе с религией или
идеологией.

Чаще всего мы сталкиваемся с политической пропагандой, например, внутри СССР
или во времена «холодной войны», когда пропаганда становится основным оружием.
Информационные войны, о которых сегодня заговорил весь мир, также используют
инструментарий пропаганды. Она присутствует и в избирательных технологиях, то
есть всюду, где большие массы людей подвергаются влиянию. Информационные
операции, психологические, операции влияния – все это входит в арсенал действий
современных государств, организующих собственную атаку или защиту от чужой
атаки.

Об этом и многом другом рассказывается в нашей книге, которая предназначена для
студентов и преподавателей гуманитарных дисциплин, также ее можно использовать
при обучении медиаграмотности в средней школе.
\end{qqquote}

Георгий Почепцов
Пропаганда 2.0
