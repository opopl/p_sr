% vim: keymap=russian-jcukenwin
%%beginhead 
 
%%file 19_12_2021.fb.fb_group.story_kiev_ua.1.detstvo_shkola.cmt
%%parent 19_12_2021.fb.fb_group.story_kiev_ua.1.detstvo_shkola
 
%%url 
 
%%author_id 
%%date 
 
%%tags 
%%title 
 
%%endhead 
\zzSecCmt

\begin{itemize} % {
\iusr{Kovalenko Alexander}

Форма російських шкіл! В Україні були кітелі.

\begin{itemize} % {
\iusr{Надежда Владимир Федько}
\textbf{Kovalenko Alexander} \enquote{Единая советская школьная форма в нашей стране была введена в эпоху Сталина...}

\iusr{Надежда Владимир Федько}
\textbf{Kovalenko Alexander} Кітелі і брюки сірого кольору з напівшерстяної тканини були введені у 1962 році.

\ifcmt
  ig https://scontent-frx5-1.xx.fbcdn.net/v/t39.30808-6/268839306_4818154371577316_4381893464317755_n.jpg?_nc_cat=100&ccb=1-5&_nc_sid=dbeb18&_nc_ohc=kKPqdmdE0ycAX9XyBTW&_nc_ht=scontent-frx5-1.xx&oh=00_AT_9aq5Z4lyyfJHhh-xI1Gsq6X7npaWt-hPhcawkQyVwiQ&oe=61C652DE
  @width 0.4
\fi

\iusr{Ирина Бельская}

Дедушка вспомнил детство. Это прекрасно особенно с этими художественными
домыслами. А этот дедушка еще был и комсомольцем и членом партии иначе не до
жил бы до сегодняшних лет спокойно.

\begin{itemize} % {
\iusr{Надежда Владимир Федько}
\textbf{Ирина Бельская} 

Я був жовтенятком, піонером, комсомольцем... Пройшов \enquote{учебку} в Острі і службу
на Кубі. Працював у спецпідрозділі МВС; супроводжував літаки (після угону
Бразінскасами); маю спеціальну юридичну освіту; працював інспектором карного
розшуку...

Членом КПСС не був (прийняв принципове рішення ще під час служби на Кубі).

Брав участь у національно-визвольних змаганнях. У 1992 і в 1995 рр. очолював
українські делегації на переговорах в НАТО.

Був помічником-консультантом народного депутата Кушнірова. У 2006-му
балотувався у ВР України від партії \enquote{Нова Сила} Юрія Збітнєва. Старший офіцер.
Дотримуюся правих поглядів. І багато чого ще за своє життя вчинив \enquote{дєдушка}.

Дожив до сьогодення і спокійно продовжую жити далі)

Чого і Вам, пані Ірино, бажаю!

\iusr{Ирина Молибог}
\textbf{Надежда Владимир Федько} 

Я с Лукьяновки, всегда пешком на Крещатик. Горжусь, что ходила по улицам, там где
ходили Вы уважаемый автор Спасибо, родной человек.


\iusr{Надежда Владимир Федько}
\textbf{Ирина Молибог} Дякую Вам за добрі слова.
\end{itemize} % }

\iusr{Владимир Каледин}
\textbf{Kovalenko Alexander} 

Не надо этой \enquote{патриотческой} чуши! Форма была едина по всему СССР! А в
Ростове-на-Дону в 1963 она была такая и в Киеве с Харьковом! Не знаю, где вы там
носили \enquote{кители}!

\iusr{Надежда Владимир Федько}
\textbf{Владимир Каледин} 

Так. Уніформа зразка 1948-1962 рр. була єдиною для всього СРСР!

На фотографіях зустрічаються гімнастерки двох видів:

а) без кишень, на трьох ґудзиках із відкладним коміром - для школярів молодших
класів; 

б) армійського зразка, зі стоячим коміром на п'яти ґудзиках, із
кишенями з клапанами на ґудзиках - для школярів старших класів.

Учні молодших класів поверх коміра гімнастерки першого типу в урочистих
випадках носили білий комірець.

До гімнастерок покладався ремінь чорної шкіри з пряжкою жовтого металу.

\end{itemize} % }

\iusr{Evgen Zachepylenko}
КИЕВ.

\ifcmt
  ig https://scontent-frx5-2.xx.fbcdn.net/v/t39.30808-6/269604510_1053075048599928_8017809321685616508_n.jpg?_nc_cat=109&ccb=1-5&_nc_sid=dbeb18&_nc_ohc=OPxxZjVFeJoAX_70hN-&_nc_ht=scontent-frx5-2.xx&oh=00_AT-t3ify-SosMpdK4cXlITylSGJjnUgevEZUut5nlU8g-w&oe=61C6BB53
  @width 0.5
\fi

\iusr{Evgen Zachepylenko}
Киев женский класс

\ifcmt
  ig https://scontent-frt3-1.xx.fbcdn.net/v/t39.30808-6/268822245_1053077578599675_6266613558038583403_n.jpg?_nc_cat=107&ccb=1-5&_nc_sid=dbeb18&_nc_ohc=3LtBRGfjmlYAX9xjMoH&_nc_ht=scontent-frt3-1.xx&oh=00_AT9gTmqWpwnOE4dvuV24kldoxugr37vlxRTDS5GmNlzH-w&oe=61C64738
  @width 0.5
\fi

\begin{itemize} % {
\iusr{Надежда Владимир Федько}
\textbf{Evgen Zachepylenko} Загальношкільна форма була введена у 1948 році.

\iusr{Evgen Zachepylenko}
1958 десять лет прошло как ввели форму.

\ifcmt
  ig https://scontent-frt3-1.xx.fbcdn.net/v/t39.30808-6/267897889_1053215001919266_5226961007033343415_n.jpg?_nc_cat=106&ccb=1-5&_nc_sid=dbeb18&_nc_ohc=dP67ieXo6ugAX-q42NX&_nc_ht=scontent-frt3-1.xx&oh=00_AT9BGvPcUWwTeds1DdcNvs_btvBUCllBSEa9FXUCtBsAMQ&oe=61C67B41
  @width 0.4
\fi

\iusr{Ирина Молибог}

Сердце разрывается, сколько пережито этими девочками с раннего детства,
учительницы страдание в глазах. Не дай Боже войны.

\iusr{Надежда Владимир Федько}
\textbf{Ирина Молибог} Ми вже 7 років на війні. Правда, деякі заплющують на це очі!

\end{itemize} % }

\iusr{Gennady Henry Sergienko}
Чудова хвилююча розповідь. Дякую

\iusr{Владимир Каледин}
Носил я такую форму в первом классе! \enquote{Мышиного} цвета!

\begin{itemize} % {
\iusr{Надежда Владимир Федько}
\textbf{Владимир Каледин} 

Я написав \enquote{коричневого}, тому, що спочатку форма була темно-сірого, а потім,
після декількох прань, вона стала сіро-бурого кольору. Наступного року мені вже
купили форму більш якісну, сірого кольору.

\end{itemize} % }

\iusr{Александр Петренко}
Я пошел в школу в 1961 году именно в такой форме.

\iusr{Владимир Каледин}
\textbf{Александр Петренко} Я в 1963-м и тоже в такой же!

\iusr{Ольга Круковская}

С болью в сердце и нежностью к былому вспоминаю всю пройденную мной дорогу.
Подруги рядом и друзья! Забыть это никак нельзя. Я помню каждого, Их лица не
раз могли присниться.

\iusr{Анна Изряднова}
Идеология есть в любом государстве

\iusr{Надежда Владимир Федько}

Можна багато знати, можна перетворити свою голову на справжню енциклопедію
всіляких знань, але якщо ти не знаєш, ХТО ТИ, навіть якщо при цьому знаєш все
на світі — ти поверхневий!

Якщо ти не запитуєш себе постійно:

\obeycr
- «Хто я?»
- «Звідки я прийшов?»
- «Навіщо Господь мене створив?»
- «Навіщо я існую?»
- «У чому призначення мого життя?»
- «Якщо я вірю в Бога, то чого Він чекає від мене?»
- «Куди я іду?» — ти поверхневий!..
\restorecr

\iusr{Валентин Трусов}

Пані Надія, для Вас віршик - "Я сижу на вишенке, не могу накушаться, дядя ленин
говорит, надо мама слушаться.. @igg{fbicon.face.tears.of.joy} 

\begin{itemize} % {
\iusr{Надежда Владимир Федько}
\textbf{Валентин Трусов} Пані Надія вже перебуває в іншому вимірі. А за віршик дякую) Я його теж пам'ятаю.

\begin{itemize} % {
\emph{Лариса Павловская}
\textbf{Nadegda Volodymyr Fedko} \enquote{Когда был Ленин маленький с кудрявой головой, он тоже бегал в валенках по горке снеговой}
 · Ответить · Поделиться · 1 д.
\emph{Валентин Трусов}
\textbf{Лариса Павловская} Интересно, где в дальнейшем делись его кудри? Уж не приложила ли к этому руку надежда константиновна? 😟
 · Ответить · Поделиться · 1 д.
\emph{Валентин Трусов}
\textbf{Лариса Павловская} Дополню еще одним стишком - "Я маленькая девочка, я в школу не хожу, я ленина не видела, но я его люблю"... 😉
 · Ответить · Поделиться · 1 д.
\emph{Лариса Павловская}
\textbf{Валентин Трусов} Автор в начале публикации этот стишок цитировал 🙂
 · Ответить · Поделиться · 1 д.
\emph{Лариса Павловская}
\textbf{Валентин Трусов} Сифилис, батенька 🙂
 · Ответить · Поделиться · 1 д.
\emph{Надежда Владимир Федько}

Несамовите вивчення марксизму, титанічна робота над ленінізмом та надмірне
споживання м'яса баранів, телят і кроликів довели Володимира Ілліча до втрати
кудрів!

 · Ответить · Поделиться · 1 д.
\emph{Валентин Трусов}
\textbf{Лариса Павловская} За удовольствия надо платить..🤓
 · Ответить · Поделиться · 1 д.

\emph{Надежда Владимир Федько}
\textbf{Лариса Павловская} До речі, в борделі \enquote{вождь мирового пролетариата} розплачувався партійними коштами)
\end{itemize} % }

\end{itemize} % }

\end{itemize} % }
