% vim: keymap=russian-jcukenwin
%%beginhead 
 
%%file topics.ukraina.poezia.beresnev_alendr.prichinna
%%parent topics.ukraina.poezia
 
%%url 
 
%%author_id 
%%date 
 
%%tags 
%%title 
 
%%endhead 

\url{https://stihi.ru/2012/02/04/5654}

Причинна. - помешанная -
Алендр Береснев
                «Нема кому розкувати,
                Одностайне стати
                За євангеліє правди,
                За темнії люде!
                Нема кому! Боже! Боже!
                Чи то ж і не буде?
                Ні, настане час великий
                Небесної кари.
                Розпадуться три корони
                На гордій тіарі!»…

                Т.Г. Шевченко, ЄРЕТИК, 10 октября 1845 (ВИБРАНІ ТВОРИ, КИЇВ, «Дніпро», 1978)

   1.
Помирає Україна
повільно та тихо…
Бо одна тому причина:
покохала лихо.
               Покохала, та не знала,
               ЩО таке кохання.
               Через те біду зазнала
               в ніч перед світанням.
Бо знехтувала святим Богом
у своїй гордині,
то й живе поза порогом,
наче в домовині.
              Не живе, а доживає
              останні деньочки,
              бо у спадок судьбу має
              жінки-одиночки;
яка глупо веселилась,
та й втратила Мужа…
Яка Богу не молилась,
сміючись:   я – дужа !
              Тож сміялась та кохалась,
               звалась “самостійна”.
               Й тихо Бога оцуралась,
               наче божевільна.
   2.
…Край-христопродавець нині
церкву скрізь мурує.
Бо вона дурній дитині
в пекло шлях торує.
               Наче той Ісав біблійський
               зрікся благодаті,-
               цей Народ, сатані свійський,
               виріс в її хаті.
Тож пошли їм, Христе Боже,
наснагу, прозріти!
Бо попівська рать не може
спасення творити.
               Бо в церквах тих нема Бога.
               Та й бути не може:
               грішми встелена дорога
               дасть буття негоже…
Тож лупає на всі боки
країНА – примара…
Й прогавила слушні роки
«чорнобильска мара»…
   3.
В демократію упала,
наче в страшну пастку,
наче вона і не знала,
що то – смерті частка…
              Що антихристове лоно
              вщент демократичне.
              Тож обожнюють мамону
              й зло атеїстичне.
   4.
А я голосу не маю,
щоб почули люде,
що від Бога – твердо знаю ! –
горе тобі буде.
              Тож вмирає Україна…
              й люде гірко плачуть…
              “До-сконалості” причина.
              Та ніхто не бачить.

А я тихо хреста несу,
та те гірко бачу:
зневірився бідний люде
й нічого не бачить…

   5.
Так невігласи в час скрути
сміються з Вкраїни.
Та ллють нишком вщерть отрути
через ЗМІ  «Новини»…
              …Світ цей пам’яті не має.
              Ні батьків, ні волі…
              Тож лихо їх і карає
               скрутою в неволі.
Нащо їм той шлях широкий,
людський, а не з Неба?
Нащо срібла зиск злоокий,
що Життю не треба?..
               Благодаті оцурались –
               не стало Любові…
               В глупоті перестарались…
               Й зажадалось крові!..
Тож покайтесь одностайнє!
Бо час суєт сплинув.
Благодать зове востаннє,
щоб світ не загинув…
    
              ( украинская   мова  -  онемеченное  наречие  “великорускаго  живаго  языка”, памятным знАком   
     посещения которого запечатлелось имя «Русь»: нов. «исповедание слова разделяющего 
    (размягчающего)» - «посев плевел нечестия» (Мат.13:25-30), почему и стольный град назван «Кий Евъ».
              Это стихотворение было послано в письме президенту Украины В. Ющенко с надеждой на диалог о возможных предстоящих трагических событиях планетарного масштаба по отвергнутому цивилизацией неотменимому Плану существования вселенной в свете Замысла Творца.


         Алендр Береснев.
             5  сентября  2008 г.

         http://www.stihi.ru/2012/02/04/5654
