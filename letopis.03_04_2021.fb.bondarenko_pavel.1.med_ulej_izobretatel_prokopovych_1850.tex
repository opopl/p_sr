% vim: keymap=russian-jcukenwin
%%beginhead 
 
%%file 03_04_2021.fb.bondarenko_pavel.1.med_ulej_izobretatel_prokopovych_1850
%%parent 03_04_2021
 
%%url https://www.facebook.com/groups/kozactvo/permalink/2776919369192221/
 
%%author 
%%author_id 
%%author_url 
 
%%tags 
%%title 
 
%%endhead 

\subsection{Засновник першої у Європі школи для бджолярів - Петро Іванович Прокопович - 1850}
\Purl{https://www.facebook.com/groups/kozactvo/permalink/2776919369192221/}

3 квітня 1850 року відлетів у вирій видатний українець, засновник першої у Європі школи для бджолярів, автор 16 праць, написаних українською мовою, серед яких "Грамота бджоляра" та "Школа бджолярства" Петро Іванович Прокопович.

\ifcmt
  pic https://scontent-ams4-1.xx.fbcdn.net/v/t1.6435-9/168381940_3794710513969653_8881198484717849364_n.jpg?_nc_cat=110&ccb=1-3&_nc_sid=825194&_nc_ohc=kQFWEr2IWYsAX_cEF3j&_nc_ht=scontent-ams4-1.xx&oh=0c1bdcb37096b4837d8b7b58a3ce7e32&oe=608EA993
  width 0.4
\fi


Він створив перший у світі розбірний рамковий вулик. Колосальне досягнення, адже такий вулик дозволяє вилучати мед, не винищуючи бджіл димом, як це практикувалося. Саме вулики Прокоповича нині використовують на культурних пасіках усього світу. 
Він винайшов дерев`яну перегородку з отворами, яка дозволяла проходити у рамки лише робочим бджолам, що дає можливість отримувати чистий мед.
Через школу Прокоповича пройшли понад 700 бджолярів.
Пасіка Прокоповича була найбільшою у світі - понад 10 тисяч бджолосімей (вуликів) і давала дохід у 20 тисяч тодішніх рублів. Саме Прокопович був постачальником меду до імператорського двору.
На жаль, 1879 року раптово помер син великого бджоляра - Степан Петрович Прокопович, який був продовжувачем справи батька. Царський уряд наклав арешт на майно, не попіклувавшись про  його збереження. 
Через це школу було пограбовано. Величезні липи навколо маєтку вирубано, вулики розікрадено. Зі школи зник рукопис останньої невиданої праці Петра Прокоповича і його портрети. Немає впевненості, що Покопович виглядав саме так, як його зображено на відомому портреті та пам`ятнику.
До речі, постаралися свої, українці, які прибрали до рук майно геніального бджоляра та розгромили його школу. З ненависті та заздрості щодо успішної заможної людини? "Лишбинепрокопович"? Чи просто від бажання поживитися "на халяву"?
І коли вас який "русскомірец" єхидно питатиме "аштодалиукраинцимиру" - згадайте поміж іншим і колишнього підпоручника Переяславського кінно-єгерського полку; "Колумба світового бджільництва", як його називають спеціалісти; нащадка давнього козацького роду Петра Івановича Прокоповича.
P.S.
Народ, який забув або не знає свого минулого, не має майбутнього" (Платон)
Якщо вас цікавлять таємниці минулого нашого народу; якщо бажаєте знати уроки історії - пропоную невеличку допомогу. Історичні розслідування, автором яких маю честь бути. Для широкого кола читачів.
Класичні паперові книги (автограф гарантовано) ви можете замовити за посиланням: http://pavlopraviy.blogspot.com/2019/03/blog-post_16.html
Електронні версії за символічну платню тут: https://pavlopraviy.blogspot.com/2018/10/blog-post.html...
