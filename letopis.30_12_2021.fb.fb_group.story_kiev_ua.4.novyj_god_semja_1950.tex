% vim: keymap=russian-jcukenwin
%%beginhead 
 
%%file 30_12_2021.fb.fb_group.story_kiev_ua.4.novyj_god_semja_1950
%%parent 30_12_2021
 
%%url https://www.facebook.com/groups/story.kiev.ua/posts/1829572283906231
 
%%author_id fb_group.story_kiev_ua,zolotushkin_anatolij.hajfa
%%date 
 
%%tags 1950,kiev,novyj_god,prazdnik,semja
%%title Новый год - как отмечали его в нашей семье в 50-е
 
%%endhead 
 
\subsection{Новый год - как отмечали его в нашей семье в 50-е}
\label{sec:30_12_2021.fb.fb_group.story_kiev_ua.4.novyj_god_semja_1950}
 
\Purl{https://www.facebook.com/groups/story.kiev.ua/posts/1829572283906231}
\ifcmt
 author_begin
   author_id fb_group.story_kiev_ua,zolotushkin_anatolij.hajfa
 author_end
\fi

Вот и опять Новый год. Захотелось вспомнить, как отмечали его в нашей семье в
50-е. Тогда я ничего не знал о Хануке и Рождестве, поэтому праздник был только
один. 

\ii{30_12_2021.fb.fb_group.story_kiev_ua.4.novyj_god_semja_1950.pic.1}

Недалеко от нашего дома, на Владимирской, было два ёлочных базара. Один возле
Оперного театра, а второй на площади возле Софиевского собора. Вокруг дощатой
ограды роились толпы людей, и стояла очередь только чтобы туда зайти. Наконец и
нас с мамой допустили вовнутрь. Тщетно мы ковыряемся в ёлочных залежах. Всех
красавиц уже разобрали. Остались какие-то кривобокие или безголовые. Но
всё-таки елка найдена и мы тащим ее  домой. Ей уготован деревянный крест. Чтобы
она туда влезла, папа обтесывает ствол топором. 

Со шкафа спускаются две фанерные коробки с нашими новогодними сокровищами.
Когда-то они работали посылочными ящиками, а теперь, на пенсии, обречены вечно
стоять на шкафу с игрушками. В одном из них лежит ватный Дед Мороз в халате из
красной папиросной бумаги, а в другом - ёлочные украшения. Как тщательно их не
укладывай, все равно несколько разбитых и я предвкушаю поход в игрушечный
магазин, за пополнением. Там все полки устланы ватой, продавщицы в бумажных
венцах, а куклы и машинки сброшены в угол. До них никому нет дела. 

Бабушке ёлочные страсти чужды. Она готовит праздничный стол. Ещё не изобретены
салат имени Оливье и селёдка под шубой, а куплен гусь. С него вытоплен  смалец,
сварено холодное,  налеплены пирожки. Пух заготовлен для подушек, а из перьев
изготовлена щётка\hyp сметка. Ничего не пропало. Одно перо досталось мне, я
пытаюсь им писать, как Пушкин. Кроме клякс ничего не получается. 

Завтра Новый год, потом каникулы, снег, санки и так будет всегда.

Всех читающих эти строки поздравляю с весёлым праздником Нового года и пусть
прошлое останется в прошлом, а будущее будет светло и прекрасно!

\ii{30_12_2021.fb.fb_group.story_kiev_ua.4.novyj_god_semja_1950.cmt}
