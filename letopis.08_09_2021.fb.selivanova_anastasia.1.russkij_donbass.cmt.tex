% vim: keymap=russian-jcukenwin
%%beginhead 
 
%%file 08_09_2021.fb.selivanova_anastasia.1.russkij_donbass.cmt
%%parent 08_09_2021.fb.selivanova_anastasia.1.russkij_donbass
 
%%url 
 
%%author_id 
%%date 
 
%%tags 
%%title 
 
%%endhead 
\subsubsection{Коментарі}

\begin{itemize} % {
\iusr{Юлия Андриенко}
Да, красиво и концептуально. Особенно, как оформлено слово МЫ. На тот обветшалый баннер было уже больно смотреть.

\iusr{Владислав Русанов}
Старый снимали вчера утром. Я как раз шёл на лекцию.

\iusr{Алёна Касёнкина}

\ifcmt
  ig https://scontent-lga3-1.xx.fbcdn.net/v/t39.1997-6/s168x128/17633073_1652591054767295_6333333619058147328_n.png?_nc_cat=1&ccb=1-5&_nc_sid=ac3552&_nc_ohc=9l4KPN4DSSMAX8ShVEj&_nc_ht=scontent-lga3-1.xx&oh=d9d1bac685b0f4de800e9770d89dfdd5&oe=615478BC
  @width 0.1
\fi

\iusr{Андрей Соболев}

\obeycr
Нас когда-то уже разлучили,
Пуповину порвать не смогли.
Языку нас другому учили,
От своей отучали земли.
Мы не Киева против восстали,
Против киевской власти стоим.
Это наш край, угля край и стали,
Что поджечь так не терпится им.
Пуповину порвать их задача,
И залить кровью "русский вопрос"
А Донбасс превратить наш в край плача,
Не дождутся они наших слез.
Той вовек не порвать пуповины,
Власть какую там ни выбирай.
Мы не Юго-Восток Украины,
Юго-Запад России наш край!
08.09.18.
\restorecr

\iusr{Кристина Роянова}
А мне больше нравится- Россия - это Я! Россия - это Ты! Россия - это Мы!

\iusr{Артем Виницкий}
Только вряд ли нам эти надписи помогут  @igg{fbicon.smile} 

\begin{itemize} % {
\iusr{Анастасия Селиванова}
\textbf{Артем Виницкий} а какие надписи вам вообще по жизни хоть раз помогли?

\iusr{Артем Виницкий}
\textbf{Анастасия Селиванова} так в том то и дело , никакие , особенно в экономике .

\iusr{Артем Виницкий}
Я имею ввиду стать , частью РФ

\iusr{Анастасия Селиванова}

Вообще это прекрасный пост для выявления людей с негативными установками.
Приходить вот такое ко мне в позитивный пост и начинает свое внутреннее гавно
здесь разбрасывать... А я просто не готова тратить свое время и энергию на
таких токсичных... Вот и отправляю их в бан... Без сожаления

\iusr{Анастасия Селиванова}
\textbf{Артем Виницкий} вас отправить?

\iusr{Артем Виницкий}
\textbf{Анастасия Селиванова} ооо Анастасия , я реалист , и вижу к чему идёт , вы Приднестровью разюскажите о своих фантазиях  @igg{fbicon.smile} 

\iusr{Жанетта Каравацкая}
\textbf{Артем Виницкий} да оставьте вы этого угрюмого пессимиста.. Он не нашего поля ягода, это-сорняк. Известно, что написанная мечта имеет свойство осуществляться. Вспомните плакаты ВОВ. При всех недостатках-ПОБЕДИЛИ! Потому что хотели и верили.
\end{itemize} % }

\iusr{Андрей Северный}
Хорошо!

% -------------------------------------
\ii{fbauth.jalova_julia.dnr.poet.idealist}
% -------------------------------------

Сегодня у меня спросили об этой надписи - Значит есть Донбасс не русский?))

\begin{itemize} % {
\iusr{Андрей Лимарев}
\textbf{Юлия Ялова} у кого это возник такой вопрос?  @igg{fbicon.face.eyebrow.raised} 

\iusr{Кристина Роянова}
\textbf{Юлия Ялова} Донбасс это не только ЛДНР. Донбасс это и подконтрольная часть .

\iusr{Юлия Ялова}
\textbf{Андрей Лимарев}, 

при чем здесь у кого? Я тоже подписываюсь под этим вопросом, если тебе так
важна персоналия))

\iusr{Андрей Лимарев}
\textbf{Юлия Ялова} я так и понял))

\iusr{Юлия Ялова}
\textbf{Андрей Лимарев}, ты очень "содержателен".

\iusr{Юлия Ялова}
\textbf{Кристина Роянова}, 

и? Каким образом подконтрольность территории какому-либо государству влияете на
самоопределение народа?.. если это народ, а не стадо овец. Кроме того надпись
говорит о русском, а не российском Донбассе, что не противоречит украинской
государственности как таковой и не является посягательством на нее.

Но я представила как бы выглядела надпись, например, "Русский Калининград". То
есть возникает вопрос - Кому и что пытается доказать Донбасс или кого и в чем
убедить? Экономикой нужно заниматься, социальной политикой, работой с
населением и на этих поприщах строить мотивационные плакаты и агитки, а не на
сомнительном патриотизме и противоречивых ура-лозунгах.

\end{itemize} % }

\iusr{Гайкарам Вартанян}

Национальный состав Донецкой области: 50,0\% - украинцы, 45,2\% - русские. С хера
Донбасс русский? Российский в перспективе? Да, но никак иначе. И написать бы
правду, что это всего лишь часть Донбасса. А так получается на заборе
написано... Предыдущий слоган был лучше, во всем. Идеологически правильнее. А
это пустые слова. Потраченное бабло в никуда. Даже ветхий баннер и убитый фасад
- не оправдывают его местонахождение. Лучше бы Портрет Ташкента повесили, а то
жители подзабыли героя ДНР.  @igg{fbicon.face.grinning.big.eyes}{repeat=3} 

\begin{itemize} % {
\iusr{Юлия Ялова}
\textbf{Гайкарам Вартанян}, а что там перед этим был за слоган?

\iusr{Гайкарам Вартанян}
Наш выбор - Россия!

\iusr{Юлия Ялова}
\textbf{Гайкарам Вартанян}, м-да... действительно было лучше.

% -------------------------------------
\ii{fbauth.pavlova_viktoria.doneck.dnr.redaktor.novosti.freelance.writer}
% -------------------------------------

\textbf{Гайкарам Вартанян} 

А откуда эта статистика? Неужели из "Википедии"? Может, вспомните тех, кто в
СССР придумал давать национальность по папе? Это тоже правда. Я русская мозгом
и душой, а записана была всю жизнь украинкой, потому что папа украинец, а мама
русская.

Теперь историческая справедливость восстановлена. Какая я украинка, если толком
не знаю ни языка, ни литературы, ни истории Украины? Не потому, что не учила, а
потому что не хочу - душа не лежала и не лежит. Зато в русский язык,
литературу, культуру я влюблена. У меня русский менталитет. И таких, как я,
здесь сотни тысяч.

\iusr{Юлия Ялова}
\textbf{Виктория Павлова}, донецкие русские вдруг вспомнили, что они русские и при том большие русские, чем кто-либо другой. Бывает))

\iusr{Виктория Павлова}
\textbf{Юлия Ялова} если это камень в мой огород, то мимо. Я никогда не забывала о том, что русская.

\iusr{Юлия Ялова}
\textbf{Виктория Павлова}, это не камень и касается любого, кто говорит об угнетении его несчастного Украиной в период до 2014-го.

\iusr{Виктория Павлова}
\textbf{Юлия Ялова} свежо предание, но развивать тему желания нет. Я уж точно на угнетенную и несчастную не тяну.

\iusr{Юлия Ялова}
\textbf{Виктория Павлова}, но и комментирование было не о вас лично, а по ситуации в ДНР в отношении бездарной агитации.

\iusr{Виктория Павлова}
\textbf{Юлия Ялова}\textbf{Юлия Ялова} 

в таком случае нечего меня упоминать в комментариях. Я к вам не обращалась,
ваше мнение мне не особо интересно. Мне интересны цифры и обобщение в
комментарии товарища с армянской фамилией.


\iusr{Юлия Ялова}
\textbf{Виктория Павлова}, 

не комментируйте там, где комментирование общедоступно и никто к вам не будет
обращаться от слова вообще. Мне, например, неинтересно ваше мнение обо мне...
как и любого другого человека. Мы говорим на конкретную тему и в комментарии
Гайкарама приведены статистические цифры, нравятся они вам или нет. Но вы их
зачем-то решили разбавить своим единичным примером, на который резонно указать,
что никто вас не лишал права уточнить свою национальность и не позволять без
вашего ведома записывать вас украинкой/латышкой/еврейкой.

Почему, начиная с нулевых, Украина (кстати, как и Россия) обходит вопрос
национального состава страны - это уже другая тема. Но то, что банер этот
абсурден и его надпись весьма нелепа, это факт.

\iusr{Виктория Павлова}
\textbf{Юлия Ялова} 

серьезно? Детей до совершеннолетия спрашивали? Не смешите меня. И да, у любой
статистики есть официальный источник. Его я так и не увидела. А Украина не с
нулевых обходит национальность - вы, видно, слишком молоды. В 1994 году у меня
уже был паспорт без национальности. Дальше дискутировать с вами у меня желания
нет.

\iusr{Юлия Ялова}
\textbf{Виктория Павлова}, 

а при чем тут кем вас записали в документах школы? (с молчаливого согласия
ваших родителей). Став совершеннолетней вы уже сами заполняли документы и
идентифицировали себя самостоятельно, не так ли?

Официальный источник любой комментатор обязан предоставлять не по хотению
собеседника. Названные здесь цифры это данные переписи населения. Официальные.
Что же касается паспорта, то отсутствие там графы "национальность" тогда
явилось лишь копированием европейской практики и ее наличие считалось
пережитком советской эпохи... что в общем-то справедливо.

Вы уже дискутируете. Но стандартное "Дальше дискутировать с вами у меня желания
нет" неизбежно для тех, кому нечего сказать))

% -------------------------------------
\ii{fbauth.limarev_andrej}
% -------------------------------------

\textbf{Vartanyan Gaykaram} 

"украинцы" и русские - один народ. Русских на Донбассе - 95\%, тобишь абсолютное
большинство. Оставшиеся 5\% вполне себе ассимилированы. Донбасс русский. Не
настолько русский, насколько армянской является Армения с 98\% армянского
населения, но всё же.


\iusr{Юлия Ялова}
\textbf{Андрей Лимарев}, главное, что сам в это веришь))

\iusr{Гайкарам Вартанян}

Друзья, Вы потеряли суть сказанного мной. Бездарность и абсурдность баннера
зашкаливает. По сути я тоже русский, потому что я родился в Ростове-на-Дону,
жил в Донецке, ну а потом вот это вот все... Русский - для меня родной язык. Я
на нем говорю и думаю. Ну, лично для меня, как в принципе, Вы все славяне
одинаковые.  @igg{fbicon.face.grinning.big.eyes}  Но согласитесть, если бы баннер повесили в Карабахе, где
этнических армян 99\% и там было бы написано "Карабах - армянский", то было бы
логично, разумно, как хотите, но правильнее, что ли. Надеюсь мои мысли вам
понятны. Даже моя русская жена, сказала что баннер из разряда " бабка сказала".


\iusr{Роман Головин}
\textbf{Гайкарам Вартанян} 

Гайкарам Арамович, вот мне бы ваши проблемы со слоганом. Мне есть чем
поделиться по баннеру, но все эти мысли вообще никак не связаны с
изображением))).

\iusr{Юлия Ялова}
\textbf{Роман Головин}, вы считаете он закрывает недостаточно площади здания?))

\iusr{Андрей Лимарев}
\textbf{Гайкарам Вартанян} да знаю я)) Карабах и Армения могут быть армянскими, а Донбасс с Россией русскими - не могут. Ибо Армения и Арцах - для армян, а Россия и Донбасс - для всех желающих  @igg{fbicon.face.upside.down} 

\iusr{Юлия Ялова}
\textbf{Андрей Лимарев}, ты опять не слышишь что тебе говорят.

\iusr{Андрей Лимарев}
\textbf{Юлия Ялова} 

мы с собеседником не в том положении, чтоб я "слушал, что мне говорят". Дети
слушают, что им говорят родители, ученики в школе - учителя, подчинённые -
своего начальника и тд.

По сути же, подобное видение встречается сплошь и рядом, и слышал я это всё
неоднократно))


\iusr{Юлия Ялова}
\textbf{Андрей Лимарев}, 

ты не в меру странный. Свое мнение ты считаешь нормальным высказывать, мнение
других ты считаешь навязыванием и поучением при том, что в данном случае я тебе
всего лишь говорю о том, что ты выдаешь желаемое за действительное.

\iusr{Андрей Лимарев}
\textbf{Юлия Ялова} 

мне ничего не навязывают и не поучают - мы обмениваемся мнениями  @igg{fbicon.face.smiling.eyes.smiling} 

Ты же считаешь, что именно мне надобно принять другую точку зрения, ведь это я
"не слышу, что мне говорят", а не наоборот. Вот, что странно)

Мы с собеседником прекрасно друг друга слышим и понимаем, но у нас разные
мнения на этот счёт. Для меня русский Донбасс это данность не потому что я "не
слышу, что мне говорят", а потому что то, что я слышу (и неоднократно слышал до
этого) не звучит хоть сколько-нибудь убедительно.

\iusr{Юлия Ялова}
\textbf{Андрей Лимарев}, 

ты не заметил, что в диалогах со мной ты совершенно не говоришь на тему заметки
или моего высказывания? Ты судишь о том, что я, по твоему мнению, хочу сказать.

Гайкарам и я говорим об агрессивности надписи и несоответствии его
национальному составу, исходя из самоидентификации граждан, согласно последней
переписи населения. Ты же всего лишь высказываешь свои хотелки, а не данность.
Поэтому говорить о навязывании и поучении, как минимум, глупо. Как и
утверждать, что кто-то пытается здесь кого-то в чем-то убедить.


\iusr{Андрей Лимарев}
\textbf{Юлия Ялова} 

так по теме заметки выше уже было сказано, почему ссылка на проведённую
украинскими органами власти перепись для обоснования нерусскости Донбасса
несостоятельна - в "украинцы" русских людей записывали без их ведома и
согласия. Во-вторых, украинцы и русские это один народ. Поэтому, рассматривая
"украинцев" как другой, самостоятельный, отдельный от русских народ и на этом
основании делая выводы о не\_русскости самого Донбасса, вы ошибаетесь дважды.
Согласно концепции триединого русского народа, от которой в своём утверждении
русскости Донбасса отталкиваюсь я, даже если бы "украинцев" согласно украинской
переписи населения было бы 90\%, русских 5\% и 5\% остальных, Донбасс всё равно
был бы русским.

Пусть вас не угнетает этот факт - это просто данность, а не мои хотелки.
Смиритесь и живите с этим  @igg{fbicon.face.relieved} 

Тем более, вы сами русские.

\iusr{Людмила Денисенко}
\textbf{Андрей Лимарев} 

Правильно сказано! Тем более, что спорящие, видимо, не помнят ту «перепись»,
которую проводили при Украине. К нам, например, не пришёл никто. Я сама звонила
в СШ17, где был один из участков, и диктовала свои данные. При этом трижды
упомянула, что я - русская, родной язык - русский. А как записали - кто знает?
Если была цель показать «украинскость» Донбасса, записали, как им надо. Тем
более, фамилия «подходящая»!


\iusr{Андрей Лимарев}
\textbf{Людмила Денисенко} 

да-да, как и у первого главы ДНР, который согласно той переписи вполне мог
числиться "украинцем" @igg{fbicon.wink} 

Сфальсифицированная перепись населения является крайне хрупким основанием для
отрицания нашей русскости, но ведь от чего-то же надо плясать тем, кому само
имя русское, как кость в горле.

\iusr{Юлия Ялова}
\textbf{Людмила Денисенко}, что ж крымчан в украинцы не записали?)) Люди в большей степени ассоциировали себя тогда со страной, а понятие национальность как было в те годы, так и остается спекулятивным и политическим.

\iusr{Ксения Костина}
\textbf{Андрей Лимарев} , давно перестала спорить с людьми из параллельной реальности... Они очень странные, только время терять.

\iusr{Юлия Ялова}
\textbf{Андрей Лимарев}, Захарченко и был украинец. А вот когда говорят про фальсификации переписи в Украине, то, хохотнув, спрошу - А как там в ДНР с недавней переписью, честные вы наши?

\iusr{Юлия Ялова}
\textbf{Андрей Лимарев}, 

хоть выше, хоть ниже - мнения нескольких людей. Что мешало Донецкой области или
иной другой обжаловать тогдашние результаты, находящиеся в открытом доступе?
Ответ, кстати, на поверхности и как бы не ненавидели Прилепина за мнение про
Донбасс, но в большинстве своих оценочных суждений он прав. Узость взглядов,
зацикленность на себе и восприятие мира через призму "моя хата с краю" присуща
именно местным.

Можно сколько угодно подгонять Донбасс под русскость ровно также как власти в
Киеве подгоняют украинцев под эвропейскисть, но что там, что здесь в
большинстве своем xoxлы. Та небольшая часть, кто сохранил в себе русский дух и
оказался способен объять необъятное поднял Донбасс на бунт и выдерживает все
удары нынешнего политического режима Украины только благодаря русским из России
и всего мира, как добровольцам, так и политическим и общественным силам.

Так что мне не с чем примиряться. Я здесь живу и вижу весь местный xoxлизм в
его красе (нынешние выборы в ГД вообще искрят в донецких xoxлах всеми
красками). И еще раз. Не нужно мне приписывать не только то, что мною не
вкладывается в сказанное, но и то, кто я есть.

\iusr{Андрей Лимарев}
\textbf{Юлия Ялова} 

ополченец Виталий Африка в своём романе "Записки террориста" писал, что
"шокающие", "гэкающие", с фамилией на "ко" и болтающие на суржике ополченцы
крайне резко реагировали на то, что их могли называли "украинцами". В той среде
это считалось хуже 3,14дораса. Не думаю, что корректно называть Захарченко
"украинцем".

\ifcmt
  ig https://scontent-lga3-1.xx.fbcdn.net/v/t39.30808-6/242164664_5012898282058499_6124576567270683069_n.jpg?_nc_cat=107&ccb=1-5&_nc_sid=dbeb18&_nc_ohc=0igCpvbRHCUAX_1mCO-&_nc_ht=scontent-lga3-1.xx&oh=d8b106b6acee5131a25e3b69e35d79d6&oe=61555A83
  @width 0.3
\fi

\iusr{Юлия Ялова}
\textbf{Андрей Лимарев}, 

и опять-таки... думай - не думай, а следовало бы внимательнее слушать самого
Захарченко. Он нечасто говорил что-то стоящее, но, к счастью, он вообще мало
говорил. И тем не менее на эту тему он высказывался не раз и весьма четко.

Но так то ты, конечно, можешь и дальше продолжать оставаться в рядах
спекулирующих национальностью.

Относительно книги - сколько воюющих, столько и мнений о войне и о ее солдатах.
Тем более, когда речь идет о гражданской войне.


\iusr{Андрей Лимарев}
\textbf{Юлия Ялова} 

спекуляция национальностью, это когда при 95\% русского населения и 5\%
нерусского русскость региона ставится под сомнение.

\iusr{Oksana Lazareva}
\textbf{Гайкарам Вартанян} , 

украинцы в Донбассе, в основной массе те, что родились после 1971 года и были
записаны все, как "украинцы". Нас в семье двое: я и сестра, у нас одни
родители, разница в возрасте 2 года и по паспорту я - русская, а сестра -
украинка. А все потому, что кто-то решил, что все дети рождённые после 71 года,
будут украинцами, по месту рождения.


\iusr{Юлия Ялова}
\textbf{Андрей Лимарев}, 

ставится под сомнение кем? Донецкие сами отказались от своей якобы русскости и
вспомнили о ней, только когда оказались не в состоянии противостоять всему
тому, что с их же помощью и получило власть в Украине.

И да, в отличие от Крыма, который встал за возвращение в состав РФ, Донецкая и
Луганская область встали против войны. А сегодня и того хуже, как власть, так и
часть населения с легкостью отрекаются от своих земель, от своих людей и готовы
спокойно, а то и радуясь, наблюдать за обрушением Украины не столько как
государства, сколько погружения ее населения в мрак.

\iusr{Андрей Лимарев}
\textbf{Юлия Ялова} 

многонациональная Российская Федерация также отреклась от русских Украины и до
определённого момента не видела в них своих сограждан и соотечественников.
Именно поэтому у Прилепина Кадыров - это "неизъяснимое русское", а на Донбассе
живут хохлы.

Нет у русских Украины Родины, нет своего национального очага. Даже сейчас, по
признанию Захаровой, российские паспорта на Донбассе выдаются "из гуманитарных
соображений", а не потому что там живут государствообразующие РФ русские. Так
что не в чем винить население, внезапно (в 1991 году) оказавшееся в чуждой и
потенциально враждебной среде независимой от России и СССР Украины и
вынужденное к ней приспосабливаться.

И то, что сейчас этот порочный порядок начинает пересматриваться и что
проговариваются такие естественные, но не для всех очевидные вещи (русский
Донбасс), нельзя не приветствовать!


\iusr{Людмила Денисенко}
\textbf{Oksana Lazareva} У меня у мужа так! Он- 64-го года - русский, брат- 69-го- украинец.  @igg{fbicon.face.flushed} 

\iusr{Юлия Ялова}
\textbf{Андрей Лимарев}, 

"национальный очаг" есть только в головах нацистов и больных людей.

Российская Федерация дала возможность жителям Украины строить свою страну и
отстаивать свою идентичность по тем принципам, которые они считали для себя
важными. Неспособность украинцев, как граждан, сохранить и приумножить
полученное - это не вина России, а беда жителей Украины. И перекладывание вины
на других - еще одна черта xоxлa. Ну, а приветствовать подогрев национализма
это для тебя, мы то знаем)) нормально.


\iusr{Oksana Lazareva}
\textbf{Людмила Денисенко} , украинизация в чистом виде!

\iusr{Андрей Лимарев}
\textbf{Юлия Ялова} ты невменяемая  @igg{fbicon.face.disappointed} 

\iusr{Юлия Ялова}
\textbf{Андрей Лимарев}, зато ты, пытаясь отвечать в духе своего кумира Владимира Вольфовича, скатываешься, как всегда, в никуда))

\iusr{Андрей Лимарев}
\textbf{Юлия Ялова} Донбасс русский. Всего доброго)

\iusr{Юлия Ялова}
\textbf{Андрей Лимарев}, ахаха. Насколько нужно быть самому неуверенным в этом, чтобы повторять сие как мантру)) Впрочем, xоxлонаследие в Донбассе еще долго будет аукаться нам всем.

\iusr{Елена Лимарева}
\textbf{Андрей Лимарев} , слушай, я давно эту недоучку послала открытым текстом, она явно больна и, к тому же, неравнодушна к тебе.Всегда – ноль аргументов, только дешёвая демагогия в расчёте на не требовательную, мягко говоря, публику.

\iusr{Андрей Лимарев}
\textbf{Olena Limareva} про публику так вообще в яблочко  @igg{fbicon.direct.hit} 

\iusr{Сергей Шамов}
\textbf{Гайкарам Вартанян} поверь, все Русские

\iusr{Юлия Ялова}
\textbf{Андрей Лимарев}, 

твоя мама́ снова так разнервничалась, что бросилась в защиту сыночка? Ярая
такая)) То уже в годах мужчины ей чудятся врагами, то молодые женщины бесят при
том, что диалоги до появления ее базарных фраз проходят вполне мирно да и мы не
на твоей странице и не под тобою созданным комментарием)) Ну, русские "Olenы"
они, как видно, такие @igg{fbicon.laugh.rolling.floor} 

Кстати, про публику (так тебе понравившуюся реплику) - это ведь про всех
присутствующих, включая тебя. Исключение то мама́ твоя не прописала @igg{fbicon.laugh.rolling.floor}  Это не
говоря о том, что мои слова она не видит, в отличие от меня... если, конечно,
ты ей в своей манере не отправил скрины.

\iusr{Владимир Карасёв}
\textbf{Гайкарам Вартанян} брат - просто поверь - последние 300 лет у нас 100\% русские!

\iusr{Елена Коринева}
\textbf{Людмила Денисенко} я 1969, но русская ) вот папу мамы, родившегося по месту службы своего отца, в Купянске, записали украинцем, а его сестру, родившуюся уже в Белгороде, - русской )) но это политика коренизации 20х.

\iusr{Александр Минаев}
\textbf{Oksana Lazareva} в точку  @igg{fbicon.smile}  у меня и мать и отец русские, но когда пришли меня регистрировать в ЗАГС, там категорично настаивали что запишут украинцем на том основании, что у них не выполняется план по украинцам  @igg{fbicon.smile}  лишь принципиальная позиция матери избавила меня от участи стать украинцем  @igg{fbicon.smile}  1976 год


\end{itemize} % }

\iusr{Yulia Petrova}

Вот действительно - как шторка за которой спрятали неприглядный фасад общаги -
супер! А вот что делать брюнетисто-бородатой половине обитателей крытого
рынка?... А бурятам, казахам и таджикам?... Татарам в конце концов?... Каким-то
национализмом попахивает... Кто утверждал слоган - хорошо подумал?...

\begin{itemize} % {
\iusr{Роман Головин}
им - быть русскими, то есть уважать местный уклад и не навязывать свой.

\iusr{Игорь Томилин}
\textbf{Yulia Petrova} Многонационалы или ассимилируются, или едут домой, как во всём мире.

\iusr{Елена Коринева}
\textbf{Yulia Petrova} Я знаю и татар, и армян, и греков, которым вполне по душе этот плакат.

\iusr{Иван Ревяков}
\textbf{Yulia Petrova} а вам в голову не приходило, почему, например, НЕМЕЦ Д. И. фон Визин, АРАП А. С. Пушкин и ШОТЛАНДЕЦ М. Ю. Лермонтов считали себя РУССКИМИ, несмотря на своё этническое происхождение?

\iusr{Игорь Томилин}
\textbf{Иван Ревяков} Да там список не заканчивающийся. Барклаи, Беллинсгаузены...

\iusr{Иван Ревяков}
\textbf{Игорь Томилин} я в курсе. А грузин Багратион... И т. д. Молчу о датчанине В. И. Дале, великом лингвисте И. А. Бодуэне де Куртенэ и т. д.

\iusr{Елена Коринева}
\textbf{Иван Ревяков} 

а немец и русский генерал фон Келлер остался верен Государю, один из немногих,
и отказался надеть немецкую форму, чтобы выехать из Киева. Был убит
петлюновцами. Адмирал фон Эссен блестяще воевал с немцами же. Кажется, его
кузен был на противоположной стороне. Они считали себя русскими.

\iusr{Иван Ревяков}
\textbf{Елена Коринева} 

так примеров можно подобрать столько, что получится, как в стихотворении
Мандельштама: "Я список кораблей прочел до середины... ".

\end{itemize} % }

\end{itemize} % }
