% vim: keymap=russian-jcukenwin
%%beginhead 
 
%%file 25_06_2021.fb.bilchenko_evgenia.1.nejroanatomia
%%parent 25_06_2021
 
%%url https://www.facebook.com/yevzhik/posts/4007813849253695
 
%%author Бильченко, Евгения
%%author_id bilchenko_evgenia
%%author_url 
 
%%tags bilchenko_evgenia,poezia,ukraina
%%title БЖ. Нейроанатомия
 
%%endhead 
 
\subsection{БЖ. Нейроанатомия}
\label{sec:25_06_2021.fb.bilchenko_evgenia.1.nejroanatomia}
\Purl{https://www.facebook.com/yevzhik/posts/4007813849253695}
\ifcmt
 author_begin
   author_id bilchenko_evgenia
 author_end
\fi

БЖ. Нейроанатомия.

\begin{multicols}{2}
\obeycr
Страшней всего - не обида, не жажда мщения - равнодушие:
В распущенном пухе тополя, в колтуне, жарой распушённых
Крашеных пасм с проседью... Так дитё, хоть пляши, не кушает,
Хоть греми над ним тарахтелкой, хоть меняй на нём распашонку.
\smallskip
Я не хочу вас кушать. Не звоните мне с оправданиями,
Почему мои взгляды переменились и стали антифашистскими.
Я же не спрашиваю вас: где, когда мое мироздание, 
Рушилось, были вы и каким машинам лизали шины?
Я просто кушать вас не хочу: ни стихи ваши, ни рассказики.
\smallskip
Неужели вы сами не замечаете, что объятия мои - роботу
Стали подобны? - Чисто из вежливости, объятия по оказии,
А мысли мои и чувства мои - далеко-далеко, за ободом,
Всего, что можете вы помыслить, когда узнаёте в городе,
Салоне, театре, библиотеке - рожу мою прорусскую.
\smallskip
Я не хочу вас кушать, спасибо, бывшие, я не голоден:
И это - не наказание за ваш конформизм с обструкцией.
Просто мне - искренно вас не надо. Бывает же, что разлюбливают
Жён и мужей, предавших друзей, смолчавших коллег (что хуже ещё).
\restorecr
\end{multicols}
\begin{multicols}{2}
\obeycr
Я могу улыбаться графу, но сердце отдам лишь люмпену,
Потому что русские - это мира чернорабочие, мои суженые.
Мне до сих пор болит только дурь аспирантов моих, студентов моих,
У которых забрали детства Gaudeāmus-овскую неистовость.
Мне стыдно смотреть, как они бодрятся при мне: им некуда деться.
\smallskip
Я - солдат, я выживу, я готова на смерть за искренность и за истинность.
А они - не готовы. Я дико злюсь,  подползаю к церковной паперти,
Где учусь прощать, но не за себя (деликатность этики
В деле сем - очевидна). Как взаперти память нести
За наших дедов и прадедов, за погоны их, эполетики?
\smallskip
Потому скандалов больше не будет: мадам и мсье, открывайте уши.
Я ухожу незаметно быстро, как в Питере летом - ноченька.
Я просто вас не хочу кушать. Я просто вас не хочу кушать.
Я научилась прямоходячести. У меня теперь -
Позвоночник.
\smallskip
24 июня 2021 г.
\restorecr
\end{multicols}

\ifcmt
  pic https://scontent-lga3-2.xx.fbcdn.net/v/t1.6435-0/p526x296/207176477_4007813809253699_5204818269662346063_n.jpg?_nc_cat=103&ccb=1-3&_nc_sid=8bfeb9&_nc_ohc=YusglngfuI8AX_vS4Zl&_nc_oc=AQkrQ3JSHTGrRA5aYw70A4ta8Zoa7oPdwECSul5V4_YO398w5KwwYpb8jt-Pvt6CI4I&_nc_ht=scontent-lga3-2.xx&tp=6&oh=882acf796f8508abd7dc2efb431f1625&oe=60DB145A
  width 0.3
\fi

\begin{itemize}
\iusr{Настя Бузиашвили}

Начинается новый этап и внутреннее обновление? Это очень, очень радует... За
тёмной ночью следует рассвет... А мировоззрение часто принимают то, которое
удобнее для совести человека. Вопрос в том, какая совесть, на что она
ориентирована, мне кажется... Но не мне судить. Ценности - другие, другое
ценят. Странно и непонятно, почему происходит такая смена ценностей

\iusr{Евгения Бильченко} 
\textbf{Настя Бузиашвили} Кант сказал: люди часто
гипотетический императив (добро с пользой) принимают за категорический (добро
ради абсурда). Поступок в себе. Или у Лакана: травмы всегда оправдываются
удобными значениями культуры. Если твои друзья ценят тебя за славу, взгляды,
совместную выпивку или привычки - это липа. На Донбассе говорят: дружба -
понятие круглосуточное.

\iusr{Галина Маркелова}
ОЧЕНЬ ГОРЬКИЕ СТИХИ...

\end{itemize}
