% vim: keymap=russian-jcukenwin
%%beginhead 
 
%%file 12_08_2022.stz.news.ua.donbas24.1.azovstal_fortecja_mrpl_narys_metalurg_kombinat.1.pochatok
%%parent 12_08_2022.stz.news.ua.donbas24.1.azovstal_fortecja_mrpl_narys_metalurg_kombinat
 
%%url 
 
%%author_id 
%%date 
 
%%tags 
%%title 
 
%%endhead 

\subsubsection{З чого починався комбінат \enquote{Азовсталь}}

Дуже часто металургійний гігант \enquote{Азовсталь} називають містом в місті (через те,
що площа \enquote{Азовсталі} складає близько 11 квадратних кілометрів), але майже сто
років тому, коли побудова заводу лише планувалася, ці землі знаходилися поза
межами тодішнього Маріуполя.

\enquote{Азовсталь} розташовується на лівому березі річки Кальміус і водночас на березі
Азовського моря. Будівництво заводу тісно пов'язане з курсом на
індустріалізацію, який почав проводитися урядом СРСР з кінця 20-х років. У
лютому 1930 року Президія вищої ради народного господарства СРСР прийняла
рішення про будівництво нового металургійного заводу в Маріуполі, оскільки саме
це місто замикало найкоротший шлях морем руди з Камиш-Бурунського родовища на
Керченському півострові.

7 листопада 1930 року почалося укладання бетону в фундамент першої доменної
печі заводу \enquote{Азовсталь} на лівому березі річки Кальміус. На будівництво
\enquote{Азовсталі} було виділено 292 млн радянських рублів, в роботу були залучені
найкращі спеціалісти того часу. Навіть Серго Орджонікідзе особисто приїжджав до
Маріуполя 2 лютого 1933 року, аби контролювати процес будівництва. Згодом район
Маріуполя поблизу \enquote{Азовсталі} буде названо Орджонікідзевським, а біля
заводоуправління \enquote{Азовсталі} довго стоятиме металевий монумент Орджонікідзе.
Але все це буде десятиліття потому.

\ii{12_08_2022.stz.news.ua.donbas24.1.azovstal_fortecja_mrpl_narys_metalurg_kombinat.pic.1}

\begin{center}
  \em\color{blue}\bfseries\Large
12 серпня 1933 року доменна піч № 1 видала перший чавун.

Цей день вважається днем народження \enquote{Азовсталі}.
\end{center}

\ii{12_08_2022.stz.news.ua.donbas24.1.azovstal_fortecja_mrpl_narys_metalurg_kombinat.pic.2.jakiv_gugel}

Першим керівником майбутнього металургійного гіганта був Яків Гугель.

Він зробив вагомий внесок в будівництво та перші визначні роки роботи
\enquote{Азовсталі}, але попри це, доля його склалася трагічно — у 1937 році Якова
Гугеля розстріляли в застінках НКВС.

Але повернемося до визначних дат в історії заводу:

\begin{itemize} % {
\item 1934 рік — в експлуатацію введено доменну піч № 2;

\item 1935 рік — почало діяти сталеплавильне виробництво, вступила в дію перша в
СРСР 250-тонна мартенівська піч, що гойдається;

\item 1939 рік — на \enquote{Азовсталі} встановлено світовий рекорд із продуктивності
доменної печі на добу, домна № 3 видала 1614 тонн чавуну за один день;

\item 1941 рік — на заводі діяли 4 домни і 6 мартенівських печей, будувалися
прокатні цехи.
\end{itemize} % }

% 3 - «Азовсталь», 1938 рік. Автор фотографії Борис Ігнатович.
\ii{12_08_2022.stz.news.ua.donbas24.1.azovstal_fortecja_mrpl_narys_metalurg_kombinat.pic.3}

Під час Другої світової війни у зв'язку з наближенням до Маріуполя лінії
фронту, у період з 12 вересня по 5 жовтня 1941 року колектив працівників провів
демонтаж, відвантаження та вивіз на Урал обладнання, що зайняло 650 вагонів.

У жовтні 1941 року Маріуполь було окуповано. Завод \enquote{Азовсталь} став називатися
\enquote{Азовським заводом № 1} і був переданий в управління концерну Альфріда Круппа.
Окупанти змогли відновити електростанцію, механічний, монтажний,
електроремонтний, кисневий цехи \enquote{Азовсталі}, але відновити мартенівські печі та
ввести підприємство в повноцінну експлуатацію німцям було не під силу.

7 вересня 1943 року, перед відступом німецьких військ із міста, \enquote{Азовсталь} був
практично повністю знищений. Загальні збитки, завдані заводу, становили 204 млн
рублів. Вже після 1945 року \enquote{Азовсталь} відновили, розширили та провели
модернізацію виробництва.

% 4 - Електростанція заводу «Азовсталь», зруйнована окупантами. Фото 1943 року.
\ii{12_08_2022.stz.news.ua.donbas24.1.azovstal_fortecja_mrpl_narys_metalurg_kombinat.pic.4.elektrostancia}

\begin{itemize} % {
\item 19 листопада 1945 року — видала свою першу плавку відновлена мартенівська піч № 1.
\item 1946 рік — відновлення роботи домни № 4.
\item 1948 рік — введення в експлуатацію блюмінга та рейкобалкового цеху (завод став підприємством з повним металургійним циклом).
\item 1949 рік — введення в експлуатацію ще двох домен і аглофабрики.
\end{itemize} % }

% 5 - Заливка чавуну в мартенівську піч заводу «Азовсталь», 1948 рік.
\ii{12_08_2022.stz.news.ua.donbas24.1.azovstal_fortecja_mrpl_narys_metalurg_kombinat.pic.5}

У 50-ті роки:

\begin{itemize} % {
\item запрацювали ще дві домни,
\item почав роботу цех рейкових скріплень,
\item аглофабрика заводу \enquote{Азовсталь} виготовила перший власний агломерат,
\item введений в експлуатацію великосортний прокатний цех,
\item вперше в СРСР на заводі \enquote{Азовсталь} було освоєно виробництво залізничних рейок завдовжки 25 метрів.
\end{itemize} % }

На початку 1960-х років шлаки підприємства почали переробляти для отримання
будівельних матеріалів. Почала зростати на березі Азовського моря відома
азовстальська шлакова гора.

60-ті та 70-ті роки запам'яталися розбудовою заводу, проведенням на базі
\enquote{Азовсталі} наукових конференцій і конкурсів молодих спеціалістів з багатьох
країн, нагородженням заводу почесними відзнаками за досягнення високих
показників у виробництві. Також були введені в експлуатацію доменні печі № 5 і
6, стартувала робота унікального листопрокатного комплексу стану \enquote{3600},
запрацювала перша черга киснево-конвертерного цеху.

% 6 - Центральнi прохіднi комбiната «Азовсталь», 1976 рік.
\ii{12_08_2022.stz.news.ua.donbas24.1.azovstal_fortecja_mrpl_narys_metalurg_kombinat.pic.6.prohidni}

\begin{leftbar}
\em\bfseries Під час цієї великої розбудови підприємства з'явилися й бомбосховища на
\enquote{Азовсталі}, це обумовлювалося постійною готовністю СРСР до війни. А
бомбосховища б могли забезпечити безперервне функціонування металургійного
виробництва навіть під час воєнних дій.
\end{leftbar}

У травні 1984 року завод \enquote{Азовсталь} було перетворено на металургійний
комбінат.

Після проголошення незалежності України комбінат \enquote{Азовсталь} продовжував
потужно працювати та розвиватися: завдяки сертифікації продукції у 90-х роках
минулого століття азовстальські прокат і сталь вийшли на світовий ринок. У
нульових роках нового тисячоліття були впроваджені нові технології та
модернізація виробництва, що в майбутньому підвищило конкурентоспроможність
продукції.

Металургійний комбінат \enquote{Азовсталь} був єдиним виробником рейок в Україні. До
речі, приготуйтеся до приголомшливого факту — у 2001 році виторг \enquote{Азовсталі}
становив понад 500 млн доларів США!

У 2005 році відбулося злиття комбінату з маріупольським коксохімічним заводом
\enquote{Маркохім}.

У 2006 році комбінат \enquote{Азовсталь} увійшов до \enquote{Дивізіону сталі та прокату групи
\enquote{Метінвест}}, акціонером якої є фінансово-еконо\hyp{}мічна група System Capital
Management (власник Рінат Ахметов).

\href{https://archive.org/details/video.03_02_2022.azovstal_metinvest.mrpl_metallurg_kombinat}{%
Відео: \#AZOVSTAL / Мариупольский металлургический комбинат, Азовсталь Метинвест, 03.02.2022}%
\footnote{\url{https://archive.org/details/video.03_02_2022.azovstal_metinvest.mrpl_metallurg_kombinat}} %
\footnote{\url{https://www.youtube.com/watch?v=ogM0dPO8d94}}

\ifcmt
  ig https://i2.paste.pics/PT0H3.png?trs=1142e84a8812893e619f828af22a1d084584f26ffb97dd2bb11c85495ee994c5
  @wrap center
  @width 0.9
\fi

\ifcmt
  ig https://i2.paste.pics/PT0HM.png?trs=1142e84a8812893e619f828af22a1d084584f26ffb97dd2bb11c85495ee994c5
  @wrap center
  @width 0.9
\fi

