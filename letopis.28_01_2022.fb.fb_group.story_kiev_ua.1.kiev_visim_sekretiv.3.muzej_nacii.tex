% vim: keymap=russian-jcukenwin
%%beginhead 
 
%%file 28_01_2022.fb.fb_group.story_kiev_ua.1.kiev_visim_sekretiv.3.muzej_nacii
%%parent 28_01_2022.fb.fb_group.story_kiev_ua.1.kiev_visim_sekretiv
 
%%url 
 
%%author_id 
%%date 
 
%%tags 
%%title 
 
%%endhead 
\subsubsection{3. МУЗЕЙ СТАНОВЛЕННЯ УКРАЇНСЬКОЇ НАЦІЇ}

3. Якщо вас втомлюють музеї, але цікаво скласти загальні уявлення про історію
України, зрозуміти звідки і як з’явились українці. Чи, наприклад, є лише день,
щоб весело і цікаво розповісти іноземцю про нашу країну, то зробити це можна в
МУЗЕЇ СТАНОВЛЕННЯ УКРАЇНСЬКОЇ НАЦІЇ (знаходиться в приміщеннях монументу
\enquote{Батьківщина-мати}).

Тут інформація подана в ігровій формі, можна осягнути базові знання завдяки
інтерактиву. А історичні персонажі оживають завдяки макетам в повний зріст,
яким позадрить знаменитий Музей мадам Тюссо. 

Проте, це не просто виглядає як такий собі історичний Діснейленд, красиво,
жваво і цікаво, але і буквально кожна довідка та інформаційний стенд підписані
певним фаховим істориком, які своїми іменами відповідають за надану інформацію.
Тож, все це ще і інформативно та корисно.

\ii{28_01_2022.fb.fb_group.story_kiev_ua.1.kiev_visim_sekretiv.pic.3}

Як я писав після першого відвідання музею: це найкраще, що зі мною сталося
останнім часом - заглиблення в атмосферу максимальне, а провести там можна і
дві і три години, які пролетять непомітно. Але найбільше враження справили два
зали, які розповідають про сучасну історію України, містять документальні
свідчення російсько-української війни.

