% vim: keymap=russian-jcukenwin
%%beginhead 
 
%%file slova.bogatstvo
%%parent slova
 
%%url 
 
%%author 
%%author_id 
%%author_url 
 
%%tags 
%%title 
 
%%endhead 
\chapter{Богатство}
\label{sec:slova.bogatstvo}

%%%cit
%%%cit_head
%%%cit_pic
%%%cit_text
Мовне питання тому завжди глибоке, що мова – один із найбільш живучих і живих
елементів національної ідентичності. Вона вбирає події, трансформується залежно
від реалій, вона гнучка, справжня, природна. 
Усю енергію борців проти російської мови і майстрів лінгвістичних зомбі-форм
спрямувати б на навчання молодих поколінь звичайної граматики – і питання було
б вичерпано. Суть у тому, що українська мова, як і ідентичність, – це значно
більше явище, ніж тільки протистояння Росії. І в цьому наше, наразі ще не
усвідомлене \emph{багатство}
%%%cit_comment
%%%cit_title
\citTitle{Українці не розуміють одне одного не через мову, а через небажання слухати, чути і сприймати}, 
Юлія Мендель, www.pravda.com.ua, 07.07.2021
%%%endcit

%%%cit
%%%cit_head
%%%cit_pic
%%%cit_text
Долаючи простори Сиротиного кабінету, я наштовхнувся на пласку скляну скриню,
надто знайому мені, щоб поминути без уваги, бо в Марковому кабінеті була
точнісінько така, і я не раз і не двічі не тільки розглядав її, а й вивчав те,
що всередині. На тій, Марковій, скрині було написано: "Ґрунти Української РСР",
і в довгастих ящичках — все наше \emph{багатство}: від славетних, рахманних, як пишно
висловлювались поети, чорноземів, від родючих каштанистих і аж до пісних
поліських пісочків, з погляду агрономії мовби й геть негодящих, але для людей,
які споконвіку на них жили, таких милих і дорогих, що після Чорнобильської
катастрофи ніхто з поліщуків не хотів міняти своєї непоказної земельки на
розкоші степових ланів
%%%cit_comment
%%%cit_title
\citTitle{Тисячолітній Миколай}, Павло Загребельний 
%%%endcit
