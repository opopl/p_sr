% vim: keymap=russian-jcukenwin
%%beginhead 
 
%%file proza.fantastika
%%parent proza
 
%%url 
 
%%author 
%%author_id 
%%author_url 
 
%%tags 
%%title 
 
%%endhead 

\url{https://t.me/authors_anonymous/1200}
ЧИ ТАКЕ СТРАШНЕ ФЕНТЕЗІ, ЯК ЙОГО МАЛЮЮТЬ? 🤔

Розпочинається новий тиждень, тож саме час розпочати нову тему.
Говоримо про фантастичну літературу.
 
Як же її писати? Нижче знайдете кілька порад, які допоможуть створити захопливий твір у жанрі фентезі.

✨ Читайте і перечитуйте. 
Ви можете писати так добре, як читаєте. Вивчіть класику жанру фентезі, звертаючи увагу на те, що захоплює вас у підході кожного автора. Наприклад, побудова світу, розвиток персонажа чи повороти сюжету. Зауважте, як оповідач орієнтується в тих аспектах, які вам здаються найбільш страшними. Ви можете перечитати також свої улюблені книги: імовірно, що на них ви хочете рівнятися.

✨ Знайте, хто ваш читач. 
Для тих, хто вперше пише фантастику, особливо важливо враховувати те, хто ваш читач. Ви пишете для дітей, молоді чи більш дорослих читачів? До якого з багатьох піджанрів фентезі потрапила б ваша історія: героїчне, високе (епічне), ігрове, історичне, гумористичне фентезі? Визначення вашого читача допоможе приймати вдалі творчі рішення.

✨ Почніть із малого. 
Створення вигаданого всесвіту – величезна справа. Пізнайте свій світ фантазій, пишучи короткі оповідання за участю вашого головного героя чи інших людей. Їх навіть не обов'язково публікувати: це письменницьке тренування перед великим забігом. Це дозволить вам формувати фантастичну реальність без тиску. Так робив Толкін перед створенням "Хоббіта".

✨ Продовжуте великим.
Написання фентезі часто передбачає створення нового світу. Відведіть час на те, щоб уявити собі не лише географію місця, а й звичаї, культуру та історію. У гарному фентезі всі ці деталі переплітаються, створюючи неповторний сюжет. 

✨ Виберіть точку зору. 
Фантастична історія може розповідатися від третьої особи через всезнаючого оповідача або від першої особи очима одного персонажа чи багатьох. Хоча перший підхід дозволяє вам викладати деталі як завгодно, та дозволяючи вашим героям вести, ви дасте читачам побачити світ їхніми очима, глибше відчувати напругу та здивування.

✨ Зустріньте своїх героїв. 
Якщо ви можете буквально намалювати персонажів – зробіть це. Якщо ж ні, то проведіть із ними детальне інтерв'ю, розпитавши про їхні емоції, мотиви, звички та, звісно ж, про їхню передісторію. 

✨ Окресліть свою історію. 
Написання фентезі дещо складніше, ніж робота з іншими жанрами. Тому навіть професіонали-фантасти використовують так звані контури історії. Завдяки їм можна відстежувати часові межі, сюжет і трансформації персонажів. Це дозволяє не випустити деталі з-поза уваги.

✨ Створюйте і дотримуйтесь правил. 
Навіть найепічніше фентезі має ґрунтуватися на власній реальності, щоб бути правдоподібним. Якщо це ваша перша книга про вигадані світи, розгляньте деякі соціальні основи, такі як політика чи економіка. Поставте очевидні запитання, наприклад, "звідки беруться річки?" Навіть магічні системи можуть і повинні мати своє правдоподібне обґрунтування.

✨ Пишіть живі діалоги. 
Відповідні стилі мовлення ваших персонажів можуть говорити про настрої та спонукання, а також про їхнє культурне походження у створеній вами цивілізації. Використовуйте розмови для просування сюжету. Дайте читачам через діалоги краще зрозуміти, хто ваші персонажі.

✨ Не поспішайте. 
Створивши унікальний світ і наповнивши його багатими персонажами, ви можете відчути спокусу пояснити все і представити всіх на перших кількох сторінках. Але це може переповнити читача. Натомість розкривайте свій ретельно придуманий світ потроху. Використовуйте всі п’ять відчуттів, щоб оживити світ, оскільки розповідь залучає вашу аудиторію глибше в казку.

\url{https://t.me/authors_anonymous/1201}

Я НЕ ВМІЮ СТВОРЮВАТИ ФЕНТЕЗІ-ГЕРОЇВ!

Відкрию вам маленький секрет: за своєю суттю вони всі подібні.

Ось 8 найпоширеніших типів фантастичних персонажів.

🦸‍♂️ Герой. 
Герой є найважливішим персонажем будь-якої фантастичної історії. Він має пройти квест і перемогти лиходія. Герої можуть набувати різних образів. Іноді герой – боєць, готовий з майстерністю та ентузіазмом стати на боротьбу із зомбі, чаклунами або полководцями. 

Прикладами героїв є Фродо Беггінс ("Володар перснів" Дж. Р. Р. Толкіна), Більбо Беггінс ("Хоббіт", також Толкін), Роланд Дешайн ("Темна вежа" Стівена Кінга).

🧛‍♂️ Лиходій. 
У фантастичних романах лиходій виконує роль головного антагоніста героя. Автори фентезі часто створюють цих персонажів для безпосередньо уособлення сили зла. Представники цього типу персонажів часто є магічними володарями, які керують величезними арміями. Часто виявляється, що лиходій не завжди був чистим злом, а його попередня історія пояснює, як вів став поганим. 

Прикладами лиходіїв є Волдеморт (серія про Гаррі Поттера Дж. К. Ролінг), Біла відьма ("Хроніки Нарнії" К. С. Льюїса), Урсула ("Русалонька").

🧙‍♂️ Наставник. 
Наставник – один із найважливіших та найпам’ятніших персонажів жанру фентезі. Наставником часто є мудра, літня постать (наприклад, старий чаклун чи шаман), який виховує головного героя і надає йому інформацію й підготовку, необхідну для порятунку світу та тріумфу в битві добра проти зла. Наставники допомагають герою вперше зрозуміти його справжні сили. 

Прикладами наставників є Гендальф ("Володар перснів"), Аслан ("Хроніки Нарнії") та Обі-Ван Кенобі ("Зоряні війни").

🤝 Помічник. 
У фантастичній літературі помічник є надійною довіреною особою та непохитним прихильником головного героя. Цей фантастичний персонаж часто є найкращим другом головного героя, і його безсмертна вірність відіграє важливу роль у виконанні місії героя. Помічники часто відчувають себе справжніми людьми з реального світу, навіть якщо вони існують у фантастичному світі магів, чаклунів та магічних сил. Коли головний герой переживає важкі часи, помічник нагадує йому про головні цілі та місію. 

Прикладами помічника є сер Кей ("Легенди короля Артура"), Герміона Грейнджер ("Гаррі Поттер").

👤 Прислужник.
Прислужники існують для того, щоб робити брудну роботу головного лиходія. Вони функціонально є помічниками головного лиходія, і хоча їм, як правило, не вистачає інтелекту лиходія, вони компенсують це своїми жорстокими діями.

Прикладами прихильників є Червохвіст ("Гаррі Поттер"), орки та Урук-Хай ("Володар перснів").

🧝‍♀️ Альтернативний герой. 
Під час написання фантастики альтернативний герой займає простір десь між головним героєм та помічником. Хоча він не є головним акцентом історії, але також орієнтований на перемогу над негідником та вирішення конфлікту, як і головний герой. Альтернативний герой має власну історію та роль у центральному драматичному питанні, а тому стає переконливими персонажем сам по собі. 

Прикладами альтернативних героїв є Арагорн ("Володар кілець"), професорка Макґонеґел ("Гаррі Поттер").

💞 Любовний інтерес.
Любовний інтерес – це звичайний троп при написанні фантастичних історій, який використовується, щоб допомогти показати людську сторону головного героя. Створюючи такі персонажі, не забудьте про багату передісторію та надайте їм переконливих емоцій і бажань. Якщо вони існуватимуть лише як сюжетний елемент для вашого головного героя, то аудиторія, швидше за все, вважатиме таких персонажів неглибокими і нудними. 

Прикладами любовних інтересів є Жовтець ("Принцеса-наречена") Вільяма Голдмана), Джеймі Фрейзер ("Чужинець" Діани Гебелдон).

🧟‍♀️ Чудовисько.
Чудовисько або зле створіння – потойбічна істота (часто щось неживе або фантастичний звір), основною місією якого є знищення та поширення зла. З цими створіннями не можна зволікати. Це такі собі нестримні машини для вбивства, які часто є найстрашнішою перешкодою у подорожі головного героя. 

Прикладами монстрів є Темний ("Колесо часу" Роберта Джордана), Порожняк ("Дім дивних дітей" Ренсом Ріггз).

ЩО РАДИТЬ ПОЧАТКІВЦЯМ ДЖ. Р. Р. ТОЛКІН?
\url{https://t.me/authors_anonymous/1203}

На зв'язку "Поради класиків". 
Що ж радить авторам-початківцям творець "Хоббіта" й "Володаря перснів"?

☝️ Хвилювання марні.
Справді, Токін писав свої книги, щоб догодити собі і письменникові всередині нього. Він очікував, що вони підуть "у макулатуру", коли покинуть його стіл. Насправді ж вони набули нечуваної популярності. Тож якщо ви розважаєте себе, ви вже знаєте одну людину, яка насолоджується вашою книгою.

☝️ Продовжуйте писати, долаючи труднощі.
Письменникові знадобилося СІМ років, щоб написати «Хоббіта». Він збалансував вимогливу денну роботу, хворобу та хвилювання за свого сина, який був далеко від Королівського флоту. 

☝️ Слухайте критиків, яким довіряєте.
Коли його редактор сказав: «Зробіть це краще», Толкін не відкинув поради. Він читав і перечитував, і намагався з усіх сил.
Автор послухав обізнаних відгуків та працював над тим, щоб покращити твір. Хто ж був редактором, якого він слухав? С. С. Льюїс, творець "Хронік Нарнії".

☝️ Нехай ваші інтереси керують вашим письмом.
Спочатку Толкін цікавився мовами. Він скористався цим і створив нові мови, а потім і цілу культуру навколо цього. Чим цікавитесь ви? Перетворіть це на фантастичну історію.

☝️ Поезія може привести до великої прози.
Коли Толкін не міг висловити свої думки бажаною прозою, він писав багато у віршах. Наступного разу, коли застрягнете, можете спробувати фокус Толкіна: спершу напишіть сцену у формі вірша.

☝️ Щасливі аварії.
Скільки б ви не планували, на сторінках кожної книги трапляються щасливі аварії. І до цього треба ставитися як до справжнього скарбу для вашої історії.

☝️ Мрії дають нам натхнення.
У всіх нас є сильні мрії. Але як бути з буквальними мріями?
Коли Толкін мріяв втопитися, він перетворив цей досвід на мотиви та прозу для своїх оповідань. У його "листах" це описується, як сон, а згодом це перетворюється на відчуття вторгнення Мордора в Середзем’я та потоплення Ізенгарда.

☝️ Реальні люди стають чудовими персонажами.
Толкін спирався на реальних людей як на прототипи для населення Середзем’я. Ви також можете орієнтуватися на своїх знайомих для створення історій. Реальні люди роблять дивовижні речі, як великі, так і малі, а потім рідко впізнають себе на сторінці. Це безпрограшний варіант для авторів.

☝️ Ви можете стати наступним автором бестселерів.
Толкін не очікував визнання, яке він отримав від своєї першої книги "Хоббіт". Він відчував, що це "щаслива аварія". Ви не дізнаєтесь, чи матимете успіх, поки не спробуєте. Можливо, наступним бестселером стануть ваші супергерої-однороги (як варіант).

☝️ Книги, які ви пишете, можуть здатися банальними.
Ми не можемо оцінити власну роботу. Сцена, яку ми вважаємо мелодраматичною, для читача може видатися зворушливою. Толкін вірив, що якщо ви навчитесь якогось ремеслу і виллєте своє серце та фантазію на сторінку, робота отримає резонанс. Я теж у це вірю.

