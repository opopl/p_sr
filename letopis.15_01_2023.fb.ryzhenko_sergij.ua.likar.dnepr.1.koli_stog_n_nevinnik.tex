%%beginhead 
 
%%file 15_01_2023.fb.ryzhenko_sergij.ua.likar.dnepr.1.koli_stog_n_nevinnik
%%parent 15_01_2023
 
%%url https://www.facebook.com/rsa010963/posts/pfbid0LuKshHBWW8uRfKfz6qevfi1XiEBUFxxVjieQjMnvEg8KPnYXXxwSbRtcXxgKzWTcl
 
%%author_id ryzhenko_sergij.ua.likar.dnepr
%%date 15_01_2023
 
%%tags 
%%title Коли стогін невинних розриває душу…
 
%%endhead 

\subsection{Коли стогін невинних розриває душу...}
\label{sec:15_01_2023.fb.ryzhenko_sergij.ua.likar.dnepr.1.koli_stog_n_nevinnik}

\Purl{https://www.facebook.com/rsa010963/posts/pfbid0LuKshHBWW8uRfKfz6qevfi1XiEBUFxxVjieQjMnvEg8KPnYXXxwSbRtcXxgKzWTcl}
\ifcmt
 author_begin
   author_id ryzhenko_sergij.ua.likar.dnepr
 author_end
\fi

Коли стогін невинних розриває душу...

Життя - воно буває раз і не назавжди.

У всіх, що надійшли після вибуху в Дніпрі, обличчя були залиті кров'ю.

Одяг разом із плоттю вирваний зі шматками по всьому тілу.

Всі доставлені до Мечникова присипані будівельним пилом.

Дізнатися та визначити вік майже неможливо.

У голову, груди, живіт, кінцівки позалітали металеві уламки, бетон та каміння.

Навколо постраждалих стояв нелюдський крик близьких.

Стогін поранених і рев тих, хто впізнав своїх рідних.

Лікарі Мечникова, які були вихідні, за лічені хвилини були у приймальному
відділенні та рятували поранених, їхній досвід був просто безцінний.

Дуже вдячний: Сергію Косульникову, Vitaliy Peleh, Тамарі Шевченко, Олег Топка,
Тарас Гаценко, Maxim  Makarenkov, Валерий Томилин, Вячеслав Гришин, Владимир
Дубина, Анатолий Галущак, Александр Машин, всім лікарям, медсестрам та
волонтерам, які були разом з нами.

P.S.: Восьмирічній Златі пощастило, вона була поруч із вибухом. Відламками
посікло її обличчя, залишивши тяжку пам'ять про війну у рідному Дніпрі біля
свого будинку.

Всі надавали їй допомогу з любов'ю, як своїй дитині, розуміючи душевну травму.

When the heart-wrenching cries of the innocent can be heard, it tears your soul
apart...

No-one lives forever. 

Everyone who arrived to the hospital after the missile attack in Dnipro had
their faces covered with blood. 

The flesh and clothes were torn all over  their bodies. 

All people sent to the Mechnikov hospital were covered with building dust. 

It was practically impossible to define their age.

Metal fractions, concrete, and stones were embedded into their heads, chests,
abdomens and limbs.

There is an inhuman screaming of relatives around the victims. 

There is the painful cry of those who recognise their close loved ones. 

During their weekend, Mechnikov's Doctors immediately arrived to the emergency
department and were rescuing the injured, their experience was absolutely
invaluable. 

I am extremely grateful to: Сергію Косульникову, Vitaliy Peleh, Тамарі
Шевченко, Олег Топка, Тарас Гаценко, Maxim  Makarenkov, Валерий Томилин,
Вячеслав Гришин, Владимир Дубина, Анатолий Галущак, Александр Машин, and also
to all the doctors, nurses and volunteers, who were with us. 

P.S.: Eight year old Zlata was lucky, she was near the explosion. The fractions
wounded her face, having left drastic memories about the war in her native town
of Dnipro near her own house. 

Everybody helped her with love, as if she were their own child, understanding
the horrific trauma.
