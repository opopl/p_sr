% vim: keymap=russian-jcukenwin
%%beginhead 
 
%%file 12_02_2022.stz.news.ua.zbruc.1.prosto_gogol
%%parent 12_02_2022
 
%%url https://zbruc.eu/node/110533
 
%%author_id mahno_vasyl
%%date 
 
%%tags gogol_nikolai,kultura,rossia,ukraina
%%title Просто Gogol
 
%%endhead 
 
\subsection{Просто Gogol}
\label{sec:12_02_2022.stz.news.ua.zbruc.1.prosto_gogol}
 
\Purl{https://zbruc.eu/node/110533}
\ifcmt
 author_begin
   author_id mahno_vasyl
 author_end
\fi

Гоголь в одному зі своїх паризьких листів, адресованому Миколі Прокоповичу,
приятелеві юності з ніжинського періоду, зауважує: «Пиши моє прізвище
правильно, інаше на пошті відбуваються непорозуміння. Пиши просто, як
вимовляється Gogol». 

\ifcmt
  ig https://zbruc.eu/sites/default/files/hohol_0.png
  @wrap center
  @width 0.8
\fi

У нашому літературному пантеоні Микола Васильович Гоголь
займає суперечливе місце. У нашому, тобто українському. Цю суперечливість
зосереджено у мові та просторі. Спробую пояснити: нам би так хотілося, щоби
Гоголь писав українською, і тоді ніяких непорозумінь довкола нього не виникало
б, а він натомість вибрав мову імперії, залишивши нам у спадок можливість для
нарікань та образ. Сам образ Гоголя – довгоносого, схожого, справді, на птаху з
родини качиних, в обрамленні чорного, розчесаного з проділом, волосся; з
неодмінними вусами, що скрашують довжину носа, в чорному манто, – це образ
аскета й одночасно якогось середньовічного алхіміка. Звісно, походження Гоголя
українське, перші повісті перенасичено українським мелосом та реаліями, а далі,
ну далі ми знаємо. Не дивлячись, що Гоголь всього на п’ять років старший за
Тараса Шевченка і що Петербурґ – частково спільне місто для обох письменників,
їхнє бачення України посутньо відрізнялося. У Гоголя Україна, з часом, взагалі
відступає на другий план. Ні листування з Максимовичем, ні спроби написати
історію Малоросії чи отримати посаду професора в Київському університеті св.
Володимира, ні відвідини родини – матері та сестер – ніщо не спонукало його
стати письменником української мови. У 1840–50-х він усе більше віддаляється
від України, захоплюючись Італією, містичними пошуками, подорожжю до Святих
місць. Він поринає у те, що його постійно супроводжує і мучить, – у
накопичування нарікань про свої хвороби та болісний процес писання другого тому
«Мертвих душ». Достатньо прочитати його «Взгляд на составление Малороссии», яку
він замислював як кількатомову працю. І хоча з того замислу нічого не вийшло й
письменник охолонув до писання такої історії, все ж це промовисте свідчення, що
тема України його переслідувала. Тобто він не міг викреслити її зі своєї
пам’яті, забути чи викинути як використаний носовичок. Охолонув, та не зовсім:
з того цілого прочитаного масиву історичних документів та літописів й зародився
задум написати «Тараса Бульбу». Найголовніше питання Гоголевої літературної
постаті полягає в гамлетівській дихотомії – український чи російський? Якщо
порівняти Гоголя з іншим довголітнім (з перервами) мешканцем столиці Російської
імперії Шевченком, то знаємо, що немає писань Гоголя українською. Він перерізав
пуповину своєї мови як письменник ще, очевидно, в Ніжинському ліцеї. Столиця
імперії приваблювала й заманювала, усі предмети викладали російською, та й
майже усе ліцеїстичне оточення Гоголя також прагло опинитися в Петербурзі. Не
зважаючи на це, оцінки багатьох російських письменників, філософів чи
літературознавців – сучасників Гоголя чи в пізніших часах – кишма кишіли
прямими звинувачуваннями в його нелюбові до Росії та знущанням з російської
мови. Це їхнє інстинктивне відчуття базувалося, як на мене, на інакшості
Гоголя. Процеси такого перетікання інтелектуальних сил з околиць до столиць
імперії – ніяка не новина, а для українського суспільства й поготів – Феофан
Прокопович чи Яків Головацький. Зрештою, були зворотні, так би мовити,
приклади, це Марко Вовчок. Коли етнічна росіянка зуміла стати класиком
національної літератури колоніального народу. Так-от, віддати Гоголя Україні
для імперії завелика розкіш: занадто великий талант, занадто проникливе око,
занадто загадкове письмо. Але висловити певні претензії вустами численних
критиків – а чому б ні? Недарма філософ Бердяєв назвав Гоголя
письменником-фантастом, що змальовував не реальних людей, а духів чи демонів
зла, додавши, що в його духовному типі було щось не зовсім російське. Однак, що
стосується української тематики чи то в «Хуторі поблизу Диканьки», чи в
«Тарасові Бульбі», чи «Пропалій грамоті», – в цих текстах вибухає інша енергія,
прямо протилежна запущеності російської провінції чи петербурзькому столичному
морокові. Після 1844 року Гоголь переживає непросту драму, пов’язану з процесом
писання другого тому «Мертвих душ». Він то пише, то оголошує, що уже закінчив.
Коли Щепкін після чергового спектаклю «Ревізора» сповістив публіці, що Микола
Васильович завершив роман, то Гоголь начебто відповів: «А хто вам це
повідомив?». Хоча у переддень щиро зізнався про це акторові. Роздумування про
українсько-російське пограниччя у Гоголя, тобто Гоголівський синдром, постає не
тепер чи у двадцятому столітті, воно виникло ще за життя письменника, і на те
були свої причини. В одному листів він сам не може собі пояснити, яка у нього
душа – «хохлацька чи російська». Мабуть, такі роздуми спричинені отими
численними закидами в підозрі його нещирості щодо великодержавного й
імперіального самовідчуття. Можливо, навіть сентиментом малої батьківщини, який
природно посилюється з віком. Однак Гоголь усе більше занурюється в російський
побут, петербурзькі та московське вулиці, поміщицькі садиби, в’язке російське
болото розкислих від дощів доріг. Він переживає релігійні духовні шукання,
впереміш із депресіями, скрашаючи їх поїздками за кордон, де йому пишеться, де
він лікується і де насолоджується європейським життям в Німеччині, Франції,
Швейцарії та Італії. У моїй бібліотеці зберігається четвертий том зі зібраних
творів Гоголя 1898 року. Видавець – А. Ф. Маркс. У цьому томі поміщено «Мертві
душі» – перший та другий томи. Другий, звісно, у двох редакціях – первинній та
поправленій. Придбав я цей том на блошиному ринку в Нью-Йорку. На форзаці
красується наліпка якоїсь Radical Library, на якій засторога для читачів, що
книжку можна перетримувати чотирнадцять днів, за кожен додатковий день оплата –
один цент. Підозрюю, що десь в ранніх двадцятих минулого століття цей
багатотомник Гоголя потрапив до Америки. А з плином часу окремими томами
розійшовся різними блошиними ринками. Можливо навіть, що хтось позичив для
читання і вчасно не повернув. Минуло чимало років – і, не розуміючи
слов’янської мови, нащадки того читача викинули книжку на вулицю або продали за
безцінь першому-ліпшому букіністові. Але це тільки мій здогад. Усе могло бути й
інакше. Поміщений у цьому томі портрет Гоголя вигравіруваний на сталі Ф. А.
Брокгаузом в Лейпціґу. Так, тим німецьким видавцем, засновником фірми
«Брокгауз», який нам відомий енциклопедичним словником Брокгауза-Ефрона. Гоголь
ніколи не розкошував, гонорари йшли на покриття боргів та допомогу сестрам. По
смерті, коли описували майно покійного, то у письменника на маєтку залишалося
43 рублі 88 копійок та найкоштовніший предмет небагатого майна – золотий
кишеньковий годинник, власником якого колись був Василь Жуковський. Але,
незважаючи на це, Гоголь часто подорожував, відвідуючи Баден-Баден, Париж, Рим
тощо. Взагалі, подорожі Гоголя, або втікання, виринали з внутрішніх та творчих
спонук. Не треба нікого переконувати, що мандрівка – це поживний хліб для
письменника. До того ж у ті часи петербурзькі богема та знать спокушалися модою
перебути зиму, до прикладу, в теплій Італії чи попопити мінеральні води на
німецьких курортах. Але не Франція з Німеччиною, а саме Італія викликала
захоплення. Чому Гоголь любив Італію? В одному листі до Погодіна з приводу
смерті Пушкіна він відкидає заклик свого адресата повернутися в Росію, кажучи,
що не хоче повторювати типову долю поетів на батьківщині. Він має на увазі, що
така батьківщина вбиває своїх поетів. Натяк цілком зрозумілий, але для Гоголя
дивний, бо відомо, що він був переконаним монархістом та твердив, що у нас,
українців та росіян, повинна бути одна мова – мова Пушкіна. Мила, постійна
пісенька й теперішніх українофобів. Я з особливою прискіпливістю вчитувався в
незакінчений уривок "Рим", який мав би бути романом. Справді, італійський
період Гоголя тривав, в сукупності, десять років. Зважаючи на першу половину
дев’ятнадцятого століття, тобто його неспішність та плинність, гадаю, цілком
достатньо, аби прив’язатися серцем та душею. У Римі перебувала ціла колонія
мистців та багатіїв з Російської імперії, там Гоголь познайомився з Олександром
Івановим, автором "Явлення Христа народові". Там диктував Аксакову деякі глави
«Мертвих душ». Уважав, що Європа існує для того, щоби її бачити, а Італія –
щоби в ній жити. Україна ставала периферією на Гоголевій мапі життя. Важливою
подорожжю була подорож 1848 року до Палестини. У січні 1848-го Гоголь в
Неаполі. Знаємо про це, бо у листах до Жуковського та Шеремєтєвої додає
складену ним молитву перед поїздкою до Гробу Господнього. Далі він у Неаполі,
22 січня – на Мальті, ув лютому – в Єрусалимі. Багатьом своїм адресатам пише,
що при Гробі Господньому згадував їхні імена. Єрусалим в лютому, незалежно від
погоди, все-таки зима. Здається, що це дещо вплинуло на загальний настрій
Гоголя. Митрополит Петрас Мелетій подарує уламок камінчика від Гробу
Господнього та уламок від дверей храму Воскресіння, що згорів був під час
пожежі 1808 року. 

\ifcmt
  ig https://zbruc.eu/sites/default/files/images/2022/02/z1644665658a14i.jpg
  @width 0.4
  @minipage 0.4
  @wrap \parpic[r]
\fi


Гоголь про палестинську подорож та перебування залишає скупі
нотатки. У листі до Василя Жуковського він пише, що десь в Самарії зірвав
польову квітку, а десь в Галилеї другу; й потрапивши під дощ, просидів два дні,
забувши, що сидить в Назареті, так ніби на станції в Росії. Подорож, не
забуваймо, відбувається тисяча вісімсот сорок восьмого року – Європою
прокочується хвиля революційних змін, а в Санкт- Петербузі лютує епідемія,
жертвами якої стають понад одинадцять тисяч містян. Та все ж Гоголь
повертається в Росію, як свідчать його сучасники, до певної міри переродженим.
Одним видавався при піднесеному духові, іншим – геть пригаслим. Та й сам
письменник у листах до друзів свідчить, що перебування в Палестині – важливий
етап його духовної біографії. Відчув, мовляв, свою ницість та черствість серця.
Далі почнеться фізичне згасання Гоголя. Назагал у світогляді письменника можна
віднайти чимало суперечливих й консервативних ідей. Він жив у суперечливі часи
– ми також. От і приходить на думку Микола Гоголь з його українсько-російською
дихотомією, коли жерла російських танків біля наших кордонів націлено на наші
міста і села. Мабуть, у часи найвищої небезпеки з боку Росії Гоголь буде
найкращим, на жаль, прикладом нашого національного роздвоєння, з яким він не
впорався, бо не міг. А ми можемо.


