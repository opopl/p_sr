% vim: keymap=russian-jcukenwin
%%beginhead 
 
%%file 17_02_2022.stz.news.ua.strana.1.politico_vtorzhenie_ramki
%%parent 17_02_2022
 
%%url https://strana.news/news/377282-politico-nazvala-novye-vremennye-ramki-rossijskoho-vtorzhenija-v-ukrainu.html
 
%%author_id volter_karina
%%date 
 
%%tags rossia,ugroza,ukraina,vtorzhenie
%%title Politico назвала новые временные рамки "российского вторжения в Украину"
 
%%endhead 
 
\subsection{Politico назвала новые временные рамки \enquote{российского вторжения в Украину}}
\label{sec:17_02_2022.stz.news.ua.strana.1.politico_vtorzhenie_ramki}
 
\Purl{https://strana.news/news/377282-politico-nazvala-novye-vremennye-ramki-rossijskoho-vtorzhenija-v-ukrainu.html}
\ifcmt
 author_begin
   author_id volter_karina
 author_end
\fi

Издание Politico, которое ранее писало о \enquote{вторжении России} 16 февраля,
теперь называет новые \enquote{временные рамки, за которыми действительно стоит
следить}. Речь идет о времени после 20 февраля.

\ii{17_02_2022.stz.news.ua.strana.1.politico_vtorzhenie_ramki.pic.1}

\ifcmt
  @wrap center
  @width 0.8
\fi

Об этом идет речь в статье Politico от 16 февраля.

\enquote{Важно, что высокопоставленные помощники Байдена никогда публично не заявляли,
что 16 февраля будет определенным днем вторжения — только то, что наращивание
войск президентом России Владимиром Путиным означало, что нападение может
произойти \enquote{в любое время} и, возможно, до завершения Олимпиады}, - уверяет
Politico.

Издание, ссылаясь на \enquote{аналитиков} пишет, что \enquote{одержимость 16 февраля отвлекла
внимание от реальных ключевых дат}.

\enquote{Время после 20 февраля всегда было более важным. Мы посмотрим, что тогда
сделают российские силы}, - цитирует издание \enquote{эксперта по российским
вооруженным силам} Центра военно-морского анализа Майкла Кофмана.

Именно тогда должны завершиться крупнейшие военные учения, после которых лидеры
в Москве и Минске пообещали, что российские войска отправятся домой, пишет
Politico.

Кофман заявил, что действия российских войск после 20 февраля помогут понять
истинные намерения Путина.

\enquote{Есть еще одна причина, по которой 20 февраля имеет большое значение: в этот
день завершается Мюнхенская конференция по безопасности}, - пишет издание.

Кроме того, 20 февраля также является последним днем зимних Олимпийских игр в
Пекине, что по версии Politico, \enquote{наводит некоторых на мысль, что Путин не
начнет вторжение до церемонии закрытия, чтобы осчастливить китайского лидера Си
Цзиньпина}.

Напомним, западные СМИ анонсировали день \enquote{вторжения} на 16 февраля, однако оно
не произошло. После этого в Госдепе США заявили, что Россия может начать
\enquote{вторжение в Украину} в любой момент.

Также мы писали, что в Белом доме заявили, что Россия увеличила военное
присутствие вдоль границы с Украиной на 7 000 человек.

Олеся Медведева в блоге \enquote{Ясно. Понятно} рассказывала, как в Украине оправдывают
фейки Запада о \enquote{вторжении России}. 


