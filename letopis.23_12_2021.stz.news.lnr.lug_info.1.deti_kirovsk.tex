% vim: keymap=russian-jcukenwin
%%beginhead 
 
%%file 23_12_2021.stz.news.lnr.lug_info.1.deti_kirovsk
%%parent 23_12_2021
 
%%url https://lug-info.com/news/obsestvenniki-vrucili-novogodnie-podarki-120-detam-iz-kirovska-i-prifrontovyh-poselkov
 
%%author_id 
%%date 
 
%%tags 
%%title Общественники вручили новогодние подарки 120 детям из Кировска и прифронтовых поселков
 
%%endhead 
\subsection{Общественники вручили новогодние подарки 120 детям из Кировска и прифронтовых поселков}
\label{sec:23_12_2021.stz.news.lnr.lug_info.1.deti_kirovsk}

\Purl{https://lug-info.com/news/obsestvenniki-vrucili-novogodnie-podarki-120-detam-iz-kirovska-i-prifrontovyh-poselkov}

Представители Общественной палаты (ОП) ЛНР и общественной организации \enquote{Союз
женщин Донбасса} (СЖД) вручили новогодние подарки 120 детям из Кировска и
входящих в его состав прифронтовых поселков. Об этом ЛИЦ сообщила пресс-служба
общественной организации.

\ii{23_12_2021.stz.news.lnr.lug_info.1.deti_kirovsk.pic.1}

\enquote{120 детей военнослужащих, погибших при исполнении служебных обязанностей, и
детей из многодетных семей Кировска, прифронтовых поселков городского типа
Донецкий, Червоногвардейское, Голубовское и села Березовское получили
новогодние подарки от ОП ЛНР и СЖД}, - говорится в сообщении.

\ii{23_12_2021.stz.news.lnr.lug_info.1.deti_kirovsk.pic.2}

Вручение подарков состоялось в ходе новогоднего театрализованного концерта \enquote{Как
спасти Новый год} Республиканского центра художественно-эстетического
творчества, прошедшего в Кировском городском Дворце культуры при поддержке
Министерства образования и науки ЛНР и Министерства культуры, спорта и молодежи
ЛНР.

\ii{23_12_2021.stz.news.lnr.lug_info.1.deti_kirovsk.pic.3}

\enquote{Новый год – это всегда вера в чудо, ожидание сюрпризов и приятных впечатлений.
Наши дети по-особому ждут этот праздник. А мы, взрослые, обязаны поддержать им
праздничное настроение и подарить яркие впечатления}, - сказала, открывая
праздничное мероприятие, глава городской администрации Кировска Виктория
Сергеева.

Председатель ОП ЛНР Алексей Карякин пожелал детям, чтобы их будущее было \enquote{таким
же чистым и светлым, как этот снег, выпавший за окном}.

\enquote{Берегите себя и своих близких, любите свою Родину и растите достойными
людьми}, - обратился он к детям.

Первый заместитель министра образования и науки ЛНР Ольга Долженко напомнила,
что приближаются новогодние праздники, \enquote{время волшебства и чудес для каждого
ребенка}.

\enquote{Главное чудо для детей прифронтовых территорий нашей Республики – мирное небо
над головой. И мы верим, что оно наступит}, - сказала представитель
Минобразования.

Она отметила, что \enquote{в театрализованной постановке участвуют такие же дети, с
нетерпением ожидавшие праздничных выступлений}, и пожелала маленьким зрителям
\enquote{в предновогодние дни волноваться только за судьбу героев театрализованной
сказки}.

Член ОП ЛНР, председатель СЖД Наталья Приймак подчеркнула, что в новогодние
праздники дети прифронтовых территорий заслуживают особого внимания и заботы.

\enquote{Хочется окутать их теплом и спокойствием, чтобы в их глазах сияла только
радость и непоколебимая вера в прекрасную жизнь}, - сказала она.

В ходе мероприятия восемь ребят поделились с общественницей своими новогодними
мечтами.

\enquote{Родители детей нашего города и поселков ежедневно совершают подвиг, пытаясь
сохранить нормальную жизнь для своего ребенка в непростой психологической
атмосфере. Очень важно было сегодня подарить им частичку новогоднего настроения
и положительные эмоции}, - рассказала председатель совета женщин при
администрации Кировска Юлия Зибарева.

После новогодней сказки детям вручили сладкие подарки.

Напомним, социальное движение \enquote{Союз женщин Республики}, созданное на базе
Общественной палаты ЛНР, начало свою работу в марте 2021 года. В декабре
движение приобрело статус общественной организации и сменило название на \enquote{Союз
женщин Донбасса}
