% vim: keymap=russian-jcukenwin
%%beginhead 
 
%%file 30_05_2021.fb.zabuzhko_oksana.1.den_kieva_stih
%%parent 30_05_2021
 
%%url https://www.facebook.com/oksana.zabuzhko/posts/10159373226653953
 
%%author Забужко, Оксана
%%author_id zabuzhko_oksana
%%author_url 
 
%%tags 
%%title До Дня Києва - дуже "київський" вірш (із недрукованої книжки "Вірші темних років")
 
%%endhead 
 
\subsection{До Дня Києва - дуже "київський" вірш (із недрукованої книжки "Вірші темних років")}
\label{sec:30_05_2021.fb.zabuzhko_oksana.1.den_kieva_stih}
\Purl{https://www.facebook.com/oksana.zabuzhko/posts/10159373226653953}
\ifcmt
 author_begin
   author_id zabuzhko_oksana
 author_end
\fi

До Дня Києва - дуже \enquote{київський} вірш (із недрукованої книжки \enquote{Вірші темних років}).

ПО МОСТУ МЕТРО (ДОРОГОЮ НА ЄВГЕНА СВЕРСТЮКА).

З цих печерських мортир
Не стрелить ніхто й ніде.
В жодній з твоїх квартир
Ніхто на тебе не жде.

Вік перейшов - як схлип, 
Як із лиця вода:
Ті - на два метри вглиб, 
Тих - не знайти й сліда...

Тільки той самий маршрут
(Вже без адреси в кінці!)
Знає: люди - течуть, 
Снігом у кулаці.

Й ти - те саме дівча,
Що й тридцять тому літ, -
Вже не знайдеш плеча
Виревітись як слід...

В тому самому склі
Того, що й тоді, метро
Твоє обличчя в імлі
Втискується в Дніпро -

Наче мільйони калік,
Вийнятих з рук і ніг,
Творять Господній лик
З твоїм лицем поміж них, -

І над синцем ріки,
Де вогнів фотосток, -
Град попід облаки,
Як осіянний стовп!

Всі твої мертві - в нім,
Мов на стінах церков.
Більша за всі земні,
Горлом іде любов:

Плинуча з-під повік -
І, як земля, тривка,
Непроминальна ввік,
Боже, Твоя Ріка...
