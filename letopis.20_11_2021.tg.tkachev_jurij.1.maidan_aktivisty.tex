% vim: keymap=russian-jcukenwin
%%beginhead 
 
%%file 20_11_2021.tg.tkachev_jurij.1.maidan_aktivisty
%%parent 20_11_2021
 
%%url https://t.me/dadzibao/4715
 
%%author_id tkachev_jurij
%%date 
 
%%tags 
%%title 
 
%%endhead 
\subsection{...ещё одно следствие Евромайдана - активисты}

\Purl{https://t.me/dadzibao/4715}

\ifcmt
 author_begin
   author_id tkachev_jurij
 author_end
\fi

...ещё одно следствие Евромайдана - появление в Украине отдельного класса людей
- т.н. активистов. 

Человеку, в Украине не жившему, значение этого термина понять сложно - как и
то, почему он вызывает такую негативную реакцию в обществе.

Наиболее близкая аналогия - китайские хунвейбины: та же цель (поиск и борьба с
инакомыслящими), те же средства (публичное насилие), та же неподсудность, та же
склонность представать представителями народных масс ("активное меньшинство",
ага), на деле являясь по сути агентами государства и его отдельных органов,
делающих за них наиболее грязную работу или просто то, что власти от своего
имени делать стесняются. 

Однако Украина во всё вносит свои коррективы: и "украинские
хунвейбины"-активисты, во-первых, обслуживают не интересы государства в целом,
а интересы отдельных акторов этого государства: тех или иных спецслужб (сбу,
прокуратура, полиция), органов власти и местного самоуправления (мэрий, ОГА и
т.п.) или даже отдельных игроков - олигархов, политиков и прочих.

Ключевая и характерная черта активистов заключалась в том, что, будучи
формально обычными гражданами, на самом деле они отличались от них совершенно
иным правовым статусом: так, если сам активист может, к примеру, безнаказанно
избить кого-то, уничтожить его собственность или даже убить, то любые
преступления в отношении активистов расследуются с совершенно нехарактерной для
правоохранителей прытью, причём действия эти трактуются на языке УК куда
серьёзнее, чем если бы объектом нападения стал простой смертный: вместо, к
примеру, нанесения телесных повреждений средней тяжести запросто можно было
получить покушение на убийство. Это делает активистов не только неподсудными,
но и физически неприкасаемыми.

Деятельность активистов имеет как бы три слоя. На внешнем они, действительно,
кошмарят неблагонадёжных или приравненных к ним. Такая деятельность призвана
как бы оправдывать само существование активистов, и поэтому они должны её
показывать, даже если не видят внятного повода для неё - отсюда докапывание к
недостаточно патриотичным артистам, к поведшим себя "как-то не так" (по мнению
активистов) коммерческим структурам и т.п.

Второй слой - та самая отработка интересов покровителей: неугодный или
неудобный уже не с точки зрения господствующей идеологии, а мешающий
конкретному актору персонаж объявляется "зрадой", "ватой", "сепаром" или
"соратником Януковича" и подвергается травле. Зачастую травля является всего
лишь способом склонить объекта к неким переговорам с бенефициарами травли, и
когда соглашение достигается, травля заканчивается.

Ну и третий слой - это, собственно, то, чем активисты зарабатывают на жизнь:
откровенное вымогательство, мелкий рэкет и т.п., классическое "создание
проблем, которые можно решить за умеренную сумму". Получив эту сумму, активисты
моментально забывают о том, что когда-либо что-либо имели против. 

В отдельных случаях, пользуясь своей неподсудностью, активисты разворачивают
уже и откровенно криминальные схемы, такие как торговля наркотиками и оружием.
В 2014-2015 году процветали схемы по обналичке и отмыванию денег через
"волонтёров", всерьёз потеснившие на рынке классические "конверты" времён того
же Януковича. А ещё говорят, что ничего не изменилось в стране.

Подобно своим китайским собратьям, активисты постепенно сходят со сцены:
наиболее успешные из них выросли во вполне формальных политиков или чиновников,
другие, сколотив активизмом стартовый капитал, подались в бизнес, третьи
спились или скололись, четвёртые сидят или ломают голову над тем, как избежать
этого, пятые кормят рыб и червей. Нужда в активистах отпала в том числе и
потому, что официальные органы власти уже благополучно освободились от
раздражающих оков процессуальных норм и могут беспределить сами, без
привлечения непонятных пассажиров, зачастую ещё и очень тупых и потому слабо
управляемых.

Однако активизм за эти годы успел сделать немалый вклад в демонтаж Украины как
государства со свойственными ему функциями, практиками и институтами.
