% vim: keymap=russian-jcukenwin
%%beginhead 
 
%%file 21_11_2020.fb.nitsoi_larysa.1.mova_habar
%%parent 21_11_2020
 
%%url https://www.facebook.com/larysa.nitsoi/posts/3797063926992378
 
%%author 
%%author_id nitsoi_larysa
%%author_url 
 
%%tags 
%%title Моїй доні (вона завуч у школі) дали "хабаря".
 
%%endhead 
 
\subsection{Моїй доні (вона завуч у школі) дали \enquote{хабаря}.}
\label{sec:21_11_2020.fb.nitsoi_larysa.1.mova_habar}
\Purl{https://www.facebook.com/larysa.nitsoi/posts/3797063926992378}
\Pauthor{Ніцой, Лариса}

Моїй доні (вона завуч у школі) дали \enquote{хабаря}.

Камери відеоспостереження зафіксували, як вона постійно спонукає україномовних
дітей не переходити на російську. Не секрет, що в україномовних дітей по всій
Україні - проблема з мовною стійкістю. Потрапляючи в коло російськомовних
дітей, україномовні переходять на російську, і байдуже, звідки ці діти, зі
Львова чи Дніпра. Слабка мовна стійкість в українців - це наслідок геноциду. Цю
нашу ваду нам наші колонізатори подали, як нашу перевагу, мовляв, чемні люди
переходять на мову співрозмовника. Насправді, це не так. І слабка стійкість, це
не чемність, а наша слабкість. (Іноземних туристів до уваги не беремо).

Так от. Моя доня виховує в україномовних дітей мовну стійкість.

Батьки це побачили з камер і їм це сподобалося, що моя доня зберігає їхнім
дітям україномовність. За це вони подарували їй вели-и-и-ку шоколадку. Таку, що
всьому колективу на перерві вистачило до чаю.  

Тобто, це був не хабар, а вдячність.  

Якби всі вчителі в Україні так робили, як моя доня, ми б давно поклали край
мовному питанню. Чому це так важливо? Бо збудувати міцну країну зможе лише
народ, об'єднаний спільним світоглядом. А спільний світогляд формує спільна
мова.  

Не розуміє цього лише тупезний, або ворог, який косить під тупезного. 

З Днем української гідності всіх і свободи!
