% vim: keymap=russian-jcukenwin
%%beginhead 
 
%%file books.istoria_rusiv.preface
%%parent books.istoria_rusiv
 
%%url 
 
%%author_id 
%%date 
 
%%tags 
%%title 
 
%%endhead 

\section{ПЕРЕДМОВА}

\ii{books.istoria_rusiv.preface.1}

Істориків та Літописців сеї доби було в Малій Росії задосить. Але як ця країна,
начеб створена або приречена на руїну од частих навал чужинців, а ще частіших
наскоків та січей од народів сусідніх і, зрештою, од ненастанних міжусобиць і
побоїщ зазнала всіляких плюндрувань, згуби та всеспалення і, так би мовити,
залита і напоєна кров'ю людською і посипана попелом, то в такій нещасній землі
чи можливо було зберегти будь-що цілим? А з тої причини взято цю Історію із
літописів і записок Білоруських, як із країни одноплемінної, сусідньої і од
руїн Малоросійських віддаленої.

Вченістю відомий і знатністю славний Депутат Шляхетства Малоросійського пан
Полетика, коли виряджався у справах Депутатства до тої великої Імперської
Комісії для створення проекту нового укладу, то мав конечну потребу роздобути
вітчизняну Історію. Він удався з приводу цього до первісного навчителя свого,
Архієпископа Білоруського Георгія Кониського, котрий був питомим Малоросіянином
і впродовж значного часу перебував у Київській Академії Префектом і Ректором.

І сей-бо Архірей передав Панові Полетиці Літопис, або ж Історію цю, запевнюючи
архіпастирськи, що вона ведена з давніх літ в кафедральному Могильовскому
монастирі тямущими людьми, які здобували потрібні відомості від учених мужів
Київської Академії і різних найповажніших Малоросійских монастирів, а найбільше
від тих, де перебував ченцем Юрій Хмельницький, колишній гетьман
Малоросійський, що полишив у них чимало записок і паперів батька свойого,
гетьмана Зіновія Хмельницького, і самі журнали достопам'ятностей і діянь
національних, та й до всього вона знову ним переглянута і виправлена.

Пан Полетика, звіривши її з багатьма іншими літописами Малоросійськими і
знайшовши її од тих найліпшою, завше дотримувався її у довідках І писаннях по
Комісії. І так ся Історія, пройшовши "стільки умів видатних, здається, мусить
бути достовірною. Лише воєнні дії видадуться, можливо, декому сумнівними, бо ж
занадто численні. Та, міркуючи про становище землі сеї з-поміж народами сливе
непримиренними, судячи про часи та обставини, в яких народ сей завжди майже був
у вогні та плавав у крові, варто зробити висновок, що сього народу все ремесло
й управа полягали у війні та убивствах. Одна Польща доказом всьому тому. Вона
лише тоді була могутньою і страшною, коли мала у себе війська Малоросійські; а
лише скоро їх позбулася, відразу занепадати почала, а наслідки ті відомі.
Історики Польські та Литовські, справедливо запідозрювані у вигадках та
самохвальстві, описуючи діяння народу Руського, що начебто у підданстві
польському пробував, затьмарювали всіляко великі подвиги його, учинюванї на
користь спільної вітчизни своєї і Польської. Навіть самі постанови та привілеї
їхні у сій вітчизні затаювали, наближаючи якомога народ сей до рабського стану
й нікчемства. 

А коли дійшла повість їхня до часів гонінь і тиранства Польських,
на народ Руський учинений з приводу вигаданої від них Унії, а саме як дійшло до
визволення народу свого з кормиги Польської власною його мужністю і майже
безприкладною хоробрістю, то тут виригнули письменники тії всі свої лайки і
всілякого роду неправди І наклепи на сей народ і на його вождів та начальників,
називаючи їх непостійним і бунтівливим хлопством, що по сваволі і буйнощах
своїх бунти і заколоти вчиняло. 

Але діла Гетьманів Руських Косинського, Наливайка, ОстряницІ і, нарешті, великі
діла Хмельницького, листування їхні і декларації доводять вельми тому противне,
і всіляка людина здорового глузду добачить в них істину несумнівну і подвиги
шляхетні і справедливі; побачить при тому і визнає розумний, що всіляке
творіння має право буття своє боронити, власність І свободу і що для того воно
споряджено самою природою, або Творцем своїм достатніми знаряддями чи способом.

Про мужність і заповзятливість народу Руського даємо пораду творцям байок та
критикам заглянути в Історії Грецькі, Римські та інші іноземні; і вони їм
покажуть Кагана, Кия, Оскольда, Святослава, Володимира, Ярослава та інших
великих Володарів, або Князів Руських, що воювали славно з воїнством Руським в
Європі, Азії, Греції і на самі столиці їхні Константинополь і Рим нападали. І
хіба такий народ, який пожив дещо в поєднанні з Поляками і Литовцями у
повсякчасних майже війнах за їхню і за свою вітчизну, чи ж міг він загубити
природну свою хоробрість, яка згодом і над самими Поляками і Литовцями зрештою
доволі себе показала?

Але, незважаючи на все те, варто з жалем сказати, що занесені деякі безглузді
речі і наклепи в самі літописи Малоросійські, на нещастя, творцями їхніми,
питомими Русами, що необачно наслідували безсоромних і злосливих Польських і
Литовських байкотворців. 

Так, приміром, в одній шкільній історійці виводиться
на сцену зі Стародавньої Русі, або нинішньої Малоросії, нова якась земля над
Дніпром, названа тут Україною, а в ній зводяться Польськими Королями нові
поселення і засновуються Українські козаки; а до того ся земля була пустельна і
безлюдна, і Козаків на Русі не бувало. 

Але, видно, пан письменник такої нікчемної історійки не бував ніде, окрім своєї
школи, і не бачив у тій стороні, що її називає він Україною, Руських міст
найдавніших або принаймні далеко давніших од його Королів Польських, себто:
Черкаса, Крилова, Мишурина та старого Кодака над рікою Дніпром, Чигирина над
Тясмином, Умані над Россю, Ладижина і Чагарлика над Бугом, Могилева, Рашкова й
Дубосар над Дністром, Кам'яного Затону і Білозерська у гирлі Лиману. З тих міст
були деякі провІнціальними та обласними Руськими містами впродовж багатьох
віків. Але в нього все те пустеля, і Князі Руські, що виводили великі флотилії
свої в Чорне море із ріки Дніпра, себто з тих самих країн, які воювали на
Грецію, Синоп, Трапезонт і на самий Царгород з військами саме тих областей, ним
у непам'ять пущено; рівно як і сама Малоросія повернута кимось з Польського
володіння без зусилля і по добрій волі, а тридцять чотири кривавих герці, що
були при тому од військ Руських супроти Поляків та Королів їхніх і посполитого
рушення, не заслуговують на те, щоб визнати за народом сим та його вождями за
подвиги їхні і геройство належну справедливість. Одначе хто що не кажи, а
кінець діло вінчає завжди. Прийди і виждь!
