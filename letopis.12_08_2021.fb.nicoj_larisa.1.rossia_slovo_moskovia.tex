% vim: keymap=russian-jcukenwin
%%beginhead 
 
%%file 12_08_2021.fb.nicoj_larisa.1.rossia_slovo_moskovia
%%parent 12_08_2021
 
%%url https://www.facebook.com/nitsoi.larysa/posts/969175950315444
 
%%author Ницой, Лариса
%%author_id nicoj_larisa
%%author_url 
 
%%tags istoria,moskovia,rossia,slovo,slovo.rossia,ukraina
%%title Ми - українці і наша країна Україна (колишня Русь, РОсія) . Вони - московити
 
%%endhead 
 
\subsection{Ми - українці і наша країна Україна (колишня Русь, РОсія) . Вони - московити}
\label{sec:12_08_2021.fb.nicoj_larisa.1.rossia_slovo_moskovia}
 
\Purl{https://www.facebook.com/nitsoi.larysa/posts/969175950315444}
\ifcmt
 author_begin
   author_id nicoj_larisa
 author_end
\fi

Чому кожен, хто вміє докупи стулити хоча б два українських слова - одразу
вважає нормою вставити свою ратичку в українські мовні/мовознавчі процеси і
вивалити в публічному просторі свої "умовиводи", наче видатний мовознавець. 

Може ми, які вчили таблицю Менделєєва, даємо поради хімікам  як будувати
хімічні процеси на виробництві?

Хіба ми, які знають, що таке цегла і цемент, стоїмо на будівництві натовпом і
даємо поради будівельникам, які їм правильно класти ту цеглу і цемент?

\begin{itemize}
  \item Може ми, які знають, що таке щебінь і смола, стоїмо на узбіччі дороги і даємо поради дорожникам, як їх правильно змішувати між собою? 
  \item Може ми, які знають, що таке скальпель, публікуємо в ЗМІ заяви до хірургів з порадами, як правильно різати?
  \item Може ми, які знаємо, що таке гради, даємо поради військовим, як з них стріляти? 
\end{itemize}

Чому НЕФАХІВЦІ з української мови, які знають українські слова, вважають, що
мають моральне право щось там заявляти про українську мову та ще й публічно? 

Арестович заявив, що треба на законодавчому рівні розділити "рускій" і
"російський". Серйозно? Тоді треба на законодавчому рівні розділити
"український" і "юкрейніен". 

Український і юкрейніен - це одне й теж. Але "український" кажуть українці, а
"юкрейніен" кажуть іноземці на англійський лад. 

Руський і російський - це так само одне й те ж. Це все ми - українці доби Русі.

Але ми на себе казали Русь. А іноземці на нас (на грецький лад) казали РОсія. 

Якщо хочете щось розділити (і це похвально), то розділяйте правильно. Ми -
українці і наша країна Україна (колишня Русь, РОсія). Вони - московити.
Московія. Ось їхня назва.

\ii{12_08_2021.fb.nicoj_larisa.1.rossia_slovo_moskovia.cmt}
