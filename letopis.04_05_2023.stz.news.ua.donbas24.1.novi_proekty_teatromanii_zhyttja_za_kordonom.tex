% vim: keymap=russian-jcukenwin
%%beginhead 
 
%%file 04_05_2023.stz.news.ua.donbas24.1.novi_proekty_teatromanii_zhyttja_za_kordonom
%%parent 04_05_2023
 
%%url https://donbas24.news/news/novi-projekti-teatromaniyi-20-ta-zittya-kolektivu-za-kordonom
 
%%author_id demidko_olga.mariupol,news.ua.donbas24
%%date 
 
%%tags 
%%title Нові проєкти "Театроманії 2:0" та життя колективу за кордоном
 
%%endhead 
 
\subsection{Нові проєкти \enquote{Театроманії 2:0} та життя колективу за кордоном}
\label{sec:04_05_2023.stz.news.ua.donbas24.1.novi_proekty_teatromanii_zhyttja_za_kordonom}
 
\Purl{https://donbas24.news/news/novi-projekti-teatromaniyi-20-ta-zittya-kolektivu-za-kordonom}
\ifcmt
 author_begin
   author_id demidko_olga.mariupol,news.ua.donbas24
 author_end
\fi

\ii{04_05_2023.stz.news.ua.donbas24.1.novi_proekty_teatromanii_zhyttja_za_kordonom.pic.front}

\textbf{Колектив \enquote{Театроманії 2:0} готує нові постановки, привячені актуальним і наболілим проблемам.}

Попри низку труднощів, пов'язаних із життям за кордоном, колектив \href{https://archive.org/details/06_09_2022.olga_demidko.donbas24.teatromania_prodovzhue_rozv_pidtrym_kulturu_mrpl}{\emph{Театроманія 2:0}}%
\footnote{\enquote{Театроманія} продовжує розвивати та підтримувати культуру Маріуполя, Ольга Демідко, donbas24.news, 06.09.2022, %
\par\url{https://donbas24.news/news/teatromaniya-prodovzuje-rozvivati-ta-pidtrimuvati-kulturu-mariupolya}, \par%
Internet Archive: \url{https://archive.org/details/06_09_2022.olga_demidko.donbas24.teatromania_prodovzhue_rozv_pidtrym_kulturu_mrpl}%
}

продовжує свою діяльність у Ганновері — місті в Північній Німеччині,
столиці і найбільшому місті Нижньої Саксонії. Художній керівник і режисер
колективу Антон Тельбізов розповів, які вистави були показані у квітні 2023
року та що заплановано представити найближчим часом.

\ii{insert.read_also.demidko.donbas24.jaki_teatr_proekty_mrpl_zakordon_mytci}

Нагадаємо, театральна студія \enquote{Театроманія 2:0} почала працювати після
повномасштабного вторгнення рф в Україну. Це творчий осередок української
молоді від 12 до 20 років, які опинилися в Німеччині під час військових дій в
Україні. Студія працює на базі Української Спілки Нижньої Саксонії у місті
Ганновер. Керівником студії є Тельбізов Антон. Протягом 11 років він є
режисером Народного театру \enquote{Театроманія}, який було засновано у 2011 році у
Маріуполі. Головний напрям роботи — опрацювання сучасних проблем у театральних
постановках та розробка соціальних проєктів. У репертуарі театру — камерні,
музичні та драматичні вистави. Антон Тельбізов зауважив, що у Німеччині дуже
потужна соціальна політика, але до деяких моментів й досі важко звикнути.

\begin{leftbar}
\emph{\enquote{За рік життя за кордоном я відчув неймовірну турботу від Німеччини. Приємно
вражає дуже якісна соціальна допомога як елемент системи соціального
захисту населення. Завдяки наданому житлу, медичному страхуванню і
загальній соціальній політиці відчуваю себе захищеним. Але все ж не так
легко звикнути до іншої культури та іншої мови}}, — зазначив режисер
\enquote{Театроманії 2:0}.
\end{leftbar}

\textbf{Читайте також:} \emph{Маріупольський театр \enquote{Conception} показав у Тернополі свою виставу}%
\footnote{Маріупольський театр \enquote{Conception} показав у Тернополі свою виставу, Ольга Демідко, donbas24.news, 12.04.2023, \par%
\url{https://donbas24.news/news/mariupolskii-teatr-conception-pokazav-u-ternopoli-svoyu-vistavu}%
}

\ii{04_05_2023.stz.news.ua.donbas24.1.novi_proekty_teatromanii_zhyttja_za_kordonom.pic.1}

Та попри наявні труднощі, колектив \enquote{Театроманії 2:0} продовжує працювати на
німецьких сценах. Зокрема, 15 квітня театр успішно представив лялькову казку
\enquote{Сусідка}. Вистава відбулася українською мовою. Ця казка, що пройшла з повним
аншлагом, була поставлена для діточок з України, які бачили війну, втратили
домівки, але все ще вірять в дива. Заплановано виступити з казкою і в інших
містах Німеччини.

\begin{leftbar}
\emph{\enquote{Те, що наші діти виступають в Німеччині, дуже надихає і їх самих, і їхніх
батьків. Адже коли дитина щаслива, щасливі і батьки}}, — наголосив Антон
Тельбізов.
\end{leftbar}

\ii{04_05_2023.stz.news.ua.donbas24.1.novi_proekty_teatromanii_zhyttja_za_kordonom.pic.2}

Крім того, 27 березня \enquote{Театроманія} запустила унікальний\par\noindent проєкт, який колектив
спільно з друзями з Берліну, театром \enquote{Ogalala Kreuz\hyp{}berg}, почали готувати ще у
2021. Це фільм про те, що таке віра і чому вона має значення, набув особливої
актуальності сьогодні. Окремі частини, присвячені духовному пошуку, свободі,
старінню та хворобі, були зняті у відомих місцях Маріуполя, зокрема у
бібліотеці ім. В. Г. Короленка, Центрі сучасного мистецтва \enquote{Готель
\enquote{Континеталь}} та на морі. Ці відео, преставлені на сторінці театру у \href{https://www.facebook.com/groups/1440935349304538}{Facebook},%
\footnote{\url{https://www.facebook.com/groups/1440935349304538}}
нагадують маріупольцям про безтурботні і мирні часи.

\textbf{Читайте також:} \emph{Маріупольські актори зіграли благодійну виставу в Києві}%
\footnote{Маріупольські актори зіграли благодійну виставу в Києві, Артем Батечко, donbas24.news, 17.03.2023, \par%
\url{https://donbas24.news/news/mariupolski-aktori-zigrali-blagodiinu-vistavu-v-kijevi-foto}%
}

\ii{04_05_2023.stz.news.ua.donbas24.1.novi_proekty_teatromanii_zhyttja_za_kordonom.pic.3}

Наприкінці травня відбудеться ще одна прем'єра, яку режисер Антон Тельбізов
готує спільно з драматичним театром Ганновера.

\begin{leftbar}
\emph{\enquote{Я ще не визначився з жанром. Але ця постановка розкаже не тільки про
українців, але й людей з інших країн, які опинилися в Ганновері з
різних причин. У кожного свої проблеми, свої страхи та відчуття. Та
всіх нас об'єднує те, що ми є носіями своєї культури та своєї мови. Дім
знаходиться всередині нас. Це те, що зберігає нас і нашу ідентичність}},
— поділився режисер театру.
\end{leftbar}

Унікальна інтернаціональна вистава про важливі і актуальні проблеми, яка
об'єднає українських, німецьких та артистів інших національностей на одній
сцені, відбудеться 24 травня у Ганновері. А 13 травня 2023 року о 17:00 на
сцені Hölderlin Eins\par\noindent	 (Hölderlinstr.1, 30 625 Hannover) відбудеться
документальна \href{https://donbas24.news/news/u-gannoveri-vidbulasya-premjera-dokumentalnoyi-vistavi-teatromaniyi-20}{\emph{вистава \enquote{Крила}}},%
\footnote{У Ганновері відбулася прем'єра документальної вистави \enquote{Театроманії 2:0}, Ольга Демідко, donbas24.news, 27.01.2023, \par\url{https://donbas24.news/news/u-gannoveri-vidbulasya-premjera-dokumentalnoyi-vistavi-teatromaniyi-20}}
у якій розповідається про любов кожного актора
\enquote{Театроманії} до свого рідного міста.

\ii{04_05_2023.stz.news.ua.donbas24.1.novi_proekty_teatromanii_zhyttja_za_kordonom.pic.4}

Раніше Донбас24 розповідав про поетичну \href{https://donbas24.news/news/teatr-avtorskoyi-pjesi-conception-predstaviv-poeticnu-imprezu-liniyi-zittya}{\emph{імпрезу \enquote{Лінії життя}}},%
\footnote{Театр авторської п'єси Conception представив поетичну імпрезу \enquote{Лінії життя}, Ольга Демідко, donbas24.news, 26.04.2023, \par\url{https://donbas24.news/news/teatr-avtorskoyi-pjesi-conception-predstaviv-poeticnu-imprezu-liniyi-zittya}}
яку підготував Театр авторської п'єси Conception.

Ще більше новин та найактуальніша інформація про Донецьку та Луганську області
в нашому телеграм-каналі Донбас24.

Фото: з архіву Донбас24

\ii{insert.author.demidko_olga}
%\ii{04_05_2023.stz.news.ua.donbas24.1.novi_proekty_teatromanii_zhyttja_za_kordonom.txt}
