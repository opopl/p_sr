% vim: keymap=russian-jcukenwin
%%beginhead 
 
%%file 02_04_2021.fb.fb_group.respublika_lnr.2.lgaki_konkurs_internet
%%parent 02_04_2021
 
%%url https://www.facebook.com/groups/respublikalnr/posts/789529485016178/
 
%%author_id fb_group.respublika_lnr
%%date 
 
%%tags donbass,internet,konkurs,lgaki,lnr
%%title В ЛГАКИ ПОДВЕЛИ ИТОГИ КОНКУРСА ИНТЕРНЕТ-ПРОЕКТОВ
 
%%endhead 
 
\subsection{В ЛГАКИ ПОДВЕЛИ ИТОГИ КОНКУРСА ИНТЕРНЕТ-ПРОЕКТОВ}
\label{sec:02_04_2021.fb.fb_group.respublika_lnr.2.lgaki_konkurs_internet}
 
\Purl{https://www.facebook.com/groups/respublikalnr/posts/789529485016178/}
\ifcmt
 author_begin
   author_id fb_group.respublika_lnr
 author_end
\fi

В ЛГАКИ ПОДВЕЛИ ИТОГИ КОНКУРСА ИНТЕРНЕТ-ПРОЕКТОВ

Учащиеся из ЛНР и России стали победителями конкурса ученических
интернет-проектов "http://xn--90aaubot.com/". Об этом сообщили в Министерстве
культуры, спорта и молодежи ЛНР.

«Конкурс проводился с 20 февраля по 27 марта 2021 года. Его участниками стали
студенты учреждений среднего профессионального образования и учащиеся 8-11-х
классов общеобразовательных и специализированных школ», – говорится в
сообщении.

\ifcmt
  ig https://scontent-mxp1-1.xx.fbcdn.net/v/t1.6435-9/167631948_480718776294249_6397738178084482202_n.jpg?_nc_cat=105&ccb=1-5&_nc_sid=825194&_nc_ohc=y7_Wv6q7z78AX-9RPul&_nc_ht=scontent-mxp1-1.xx&oh=28c21f706c6f037b3a9639ca851beba0&oe=61B62FB2
  @width 0.4
  %@wrap \parpic[r]
  @wrap \InsertBoxR{0}
\fi

Организатором данного проекта выступила кафедры библиотечно-информационной
деятельности и электронных коммуникаций Луганской государственной академии
культуры и искусств (ЛГАКИ) имени Михаила Матусовского.

В вузе рассказали, что на открытый Республиканский конкурс ученических
интернет-проектов \url{http://xn--90aaubot.com/} были представлены авторские работы
и интернет-проекты учеников, раскрывающие современные взгляды молодежи на
проблемы и возможности развития информационного пространства и информационной
культуры.

«Жюри оценивало конкурсантов в четырех номинациях: "Must read: стоит прочесть
каждому", "Моя biblioистория", "Литературный мейнстрим" и "Homo informaticus –
человек информационной эпохи"», – добавили в учреждении образования.

По итогам конкурса в номинации "Must read: стоит прочесть каждому" одержала
победу брянковчанка Анжелика Острик. Второе место было присуждено Ксении
Жилиной из Ровеньков. Третье – Кристине Каркач из Алчевска .

В номинация "Моя biblioистория" первое и второе место достались
представительницам из Красного Луча. Работа Ангелины Чапала одержала победу.
Второй результат достался Кристине Малявке. Третье место занял луганчанин Давид
Козярский.

Победа в номинации "Литературный мейнстрим" была присуждена участнице проекта
из Ярославля Дарье Басовой. Второе заняла Софья Быковская из Алчевска. На
третьем месте – брянковчанка Ирина Темникова.

В номинация "Homo informaticus – человек информационной эпохи" по решению жюри
первое место было присуждено Илье Чулаеву из Алмазной, второе заняла жительница
Молодогвардейска Елена Цинкевич. Третье место – Марк Ляшенко из Стаханова.

Пресс-служба Правительства Луганской Народной Республики

\url{https://sovminlnr.ru/novosti/24023-v-lgaki-podveli-itogi-konkursa-internet-proektov.html}
