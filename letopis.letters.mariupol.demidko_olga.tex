% vim: keymap=russian-jcukenwin
%%beginhead 
 
%%file letters.mariupol.demidko_olga
%%parent letters
 
%%url 
 
%%author_id 
%%date 
 
%%tags 
%%title 
 
%%endhead 

дякую, буду щиро вдячний! Збираюсь робити (і вже роблю) тематичні збірки
(альбоми) по різним темам та подіям... Як-от,
Маріуполь-Травень-Великдень-Писанки-Кролі-2019, або Шахове Життя Маріуполя,
короче багато цікавих тем!  На жаль, є така проблема (тема). Тема називається
Цифрова Спадщина Маріуполя та Важливість її запису та збереження, та її
важливість у справі Культурно-Духовного Відродження Маріуполя.  І Цифрова Спадщина ця, -   
з однієї  сторони, просто величезна - на даний момент, це тисячі, десятки
тисяч, мільйони фото, публікацій в фб, і т.д. А з іншої сторони, все це зараз
перебуває у дуже підвішеному розпорошеному незв'язаному стані, під загрозою
зникнення... Ось наприклад, оте фото зверху Капсули Часу. Я от ніколи не був в
Маріуполі, і взагалі із Маріуполем не був ніяк раніше пов'язаний.  То як я її
дістав? А ось так. Поставив точку в гугл-мапах, заліз, потім в результаті
вишукування по Кролям та Писанкам таки дібрався до цого фото і зберіг.  А поки
я її не зберіг - вона може зникнути в будь-який момент. Сьогодні фото є - а
завтра його вже немає, і назавжди немає... А гугл або фейсбук потім скаже -
вибачте, а оцього фото (сайту, посту, альбому тощо) вже нема, сорян!  І як ти
потім доведеш комусь, що саме оцей офігезний Кроль (я вже визначив його автора,
до речі) таки був у Маріуполі у травні 2019 року? І це насправді вже
відбувається. Маріуполь зникає поступово інформаційно. Як отой Чеширський Кіт -
щасливе безтурботне Місто посміхається, співає, кличе, у ньому кожен день в
минулому відбувається безліч цікавих подій, зустрічей, тощо але... Посмішка
того щасливого Міста кожен Божий день стає все меншою і меншою...  Процес цей
доволі повільний і непомітний для багатьох, але це факт, мирне довоєнне Місто
зникає інформаційно...  Величезний обсяг, просто неймовірно величезний,  -
тобто, все те, чим жило Місто до 24 лютого, поступово зникає з Інтернету.  Є
вже купа сайтів (як-от той vidsebe.media або сайт Маріупольські Новини або ще
десяток - я вже цілий список склав в процесі своїх досліджень), які вже
просто-напросто не існують (публікація Тетяни, яку я виклав недавно, - це вже
мій результат години роботи по ретельному витягуванню змісту з Інтернет-Архіву,
і потім склеювання докупи із максимально можливим збереженням початкового
вигляду і наповнення). А навіть коли залазиш в Інтернет-Архів і щось читаєш,
буває так, що є новина, але фото вже немає.  Наприклад. Є новина в
Маріупольських Новинах, що писанки в Маріуполі було повалені вітром, але
встояли(!) - в Києві на Софійській Площі до речі було теж саме, але писанки
повалились ))) Новина є - а фото вже немає, як оті писанки та кролі встояли
проти вітру... І нічого вже не зробиш, окрім як напевне питати вже конкретних людей,
слухай, а ти пам'ятаєш і може маєш фотки, - як воно саме було, що писанки
встояли проти вітру під час тої травневої виставки?  Ну ок...  тема дуже
важлива, але складно тут в коментарях все тут описати, як треба, а писати
детальний пост про все це, я ще з думками не зібрався...


%13:07:43 24-04-23
дякую! До речі, щодо пошуку... як це виглядає у мене тута на компі -
траверсування вниз в минуле по точці або сторінці місця. Ставиш точку і
починаєш гортати, гортати, гортати... І поки знайдеш хоч щось що тебе цікавить
зараз (тобто Кролі або Писанки), побачиш безліч різних цікавинок, як-то
Пряники Маріуполя тощо. І до речі! деякі Кролики (або навіть Кроликині! От яке
слово походу само собі вилізло), які (точно!) були в Маріуполі, зараз
знаходяться в музеї міста Києва, Богдана Хмельницького 7 (+ там ще писанки в
холі на першому поверсі стоять). Два кролика (по моєму, один з них точно був в
Маріуполі, але треба звірити по відповідним фото з Маріуполя...) стоять на
вході. І купа кроликів (два або три з них точно були в Маріуполі) зараз
одиноко стоять на четвертому поверсі музею. Там нікого немає, немає виставок.
Тільки ліфт, кролики, ну і вікна, і все. Самі собі кролики стоять, сумують...
Чекають... може хто прийде і побалакає з ними... І не знаю ще скільки вони там
будуть... Може завтра їх вже запихнуть в тісну темну кімнату на зберігання і
вони будуть цілий рік чекати аж до наступного Великодня, поки їх знову
дістануть, хто зна?....
