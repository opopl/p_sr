% vim: keymap=russian-jcukenwin
%%beginhead 
 
%%file 08_11_2021.fb.fb_group.story_kiev_ua.1.kiev_19_vek_karaim_kupec_kogen
%%parent 08_11_2021
 
%%url https://www.facebook.com/groups/story.kiev.ua/posts/1793285097534950
 
%%author_id fb_group.story_kiev_ua,denisova_oksana.kiev.ukraina.gid
%%date 
 
%%tags gorod,istoria,kiev,kievljane,kogen_solomon.kiev.kupec.19_vek,xix_vek
%%title Жил в Киеве в конце 19 века богатый караимский купец Соломон Коген
 
%%endhead 
 
\subsection{Жил в Киеве в конце 19 века богатый караимский купец Соломон Коген}
\label{sec:08_11_2021.fb.fb_group.story_kiev_ua.1.kiev_19_vek_karaim_kupec_kogen}
 
\Purl{https://www.facebook.com/groups/story.kiev.ua/posts/1793285097534950}
\ifcmt
 author_begin
   author_id fb_group.story_kiev_ua,denisova_oksana.kiev.ukraina.gid
 author_end
\fi

Жил в Киеве в конце 19 века богатый караимский купец Соломон Коген. Кто такие
караимы? Это тюркская народность, жили они в основном в Крыму . По религии
караимы близки к иудаизму, но не признают Талмуд, поэтому не могут посещать
синагогу и у них собственные молельные дома – кенассы.  Интересно, что в
19-начале 20 века караимы контролировали всю табачную промышленность Российской
империи, все табачные фабрики в Киеве, Москве и Петербурге были в руках
караимов.

\ifcmt
  tab_begin cols=3

     pic https://scontent-lga3-1.xx.fbcdn.net/v/t39.30808-6/254644686_10158786901057198_3441325107305237885_n.jpg?_nc_cat=109&ccb=1-5&_nc_sid=b9115d&_nc_ohc=O27PouH_LL8AX8-teUd&_nc_ht=scontent-lga3-1.xx&oh=f4a74017ff98b8e4f19d1d12a44225f0&oe=61901A31

     pic https://scontent-lga3-1.xx.fbcdn.net/v/t39.30808-6/254630294_10158786901317198_8438258304461909473_n.jpg?_nc_cat=100&ccb=1-5&_nc_sid=b9115d&_nc_ohc=bYECnhuY01cAX_ASKl1&_nc_ht=scontent-lga3-1.xx&oh=7e68ce418cc6dc48f02dbc87de3eb94b&oe=618F7653

		 pic https://scontent-lga3-1.xx.fbcdn.net/v/t39.30808-6/254588534_10158786901767198_251304004969193176_n.jpg?_nc_cat=111&ccb=1-5&_nc_sid=b9115d&_nc_ohc=eJU44Gg3kBsAX8uAsll&_nc_ht=scontent-lga3-1.xx&oh=ddb5ce3f361c0525ca90c5f1e6021fa9&oe=618F3E3F

  tab_end
\fi

Соломон Коген родился в Евпатории, в 1860 году приехал в Киев и первую свою
табачную фабрику открыл – вы не поверите- на Крещатике. Он решил завоевать
рынок новым товаром — «папиросами», а также приучить киевлян к мысли, что за
удовольствие нужно платить. Дело в том, что в середине XIX века Киев был еще
тихим провинциальным городом, где в садах рос крепчайший табак-самосад. Его
киевляне не курили, а нюхали, и всегда носили с собой табакерки. А курили чаще
трубки, особенно часто трубки курили военные.  Именно благодаря стараниям
Когена трубки и табакерки начинают активно вытесняться папиросами. Папиросы
раскручивались как современный модный продукт в отличие от старосветской и
мещанской  «понюшки табаку» . Давайте вспомним пьесу Старицкого «За двумя
зайцами», вот он- конфликт поколений- «шкворчащая папироска» модного Свирида
Петровича Голохвастова и «нюхательный табачок» провинциального отца Прони
Прокоповны .

Позже к Соломону Когену приехал младший брат Моисей и фирма начала называться
«Братья Когены». Они строят корпус своей фабрики на улице Ново-Елизаветинской (
сейчас улица Пушкинская 7, здание в перестроенном виде сохранилось). Усадьба
была сквозной, через двор можно было выйти к дому Когена  на Крещатике, На
первом этаже дома был табачный магазин, т.е. готовые папиросы носили с фабрики
просто через двор,  а на 2-м – квартира, где Соломон Коген жил с семьей. Чуть
позже появился еще один корпус фабрики на улице Лютеранской 4 ( тогда она
называлась Анненковская). Интересно, что девушек, которые работали на фабрике и
измельчали табак, киевляне называли «табакрошки».  

Как все богачи тогдашнего Киева, Соломон Коген занимался благотворительностью,
именно на его деньги по проекту Владислава Городецкого была построена
знаменитая Кенасса – молельный дом караимов на нынешней улице Ярославов Вал. И,
конечно, к курению можно относиться по-разному, но спасибо Когену – Кенасса –
одно из самых красивых и оригинальных зданий Киева.

\ii{08_11_2021.fb.fb_group.story_kiev_ua.1.kiev_19_vek_karaim_kupec_kogen.cmt}
