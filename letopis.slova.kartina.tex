% vim: keymap=russian-jcukenwin
%%beginhead 
 
%%file slova.kartina
%%parent slova
 
%%url 
 
%%author_id 
%%date 
 
%%tags 
%%title 
 
%%endhead 
\chapter{Картина}

%%%cit
%%%cit_head
%%%cit_pic
%%%cit_text
...Малыш лежал на полу в своей комнате и читал книгу, когда снова услышал за
окном какое-то жужжание, и, словно гигантский шмель, в комнату влетел Карлсон.
Он сделал несколько кругов под потолком, напевая вполголоса какую-то весёлую
песенку. Пролетая мимо висящих на стенах \emph{картин}, он всякий раз сбавлял
скорость, чтобы лучше их рассмотреть. При этом он склонял набок голову и
прищуривал глазки.  — Красивые \emph{картины}, — сказал он наконец. — Необычайно
красивые \emph{картины}! Хотя, конечно, не такие красивые, как мои.  Малыш вскочил на
ноги и стоял, не помня себя от восторга: так он был рад, что Карлсон вернулся.
— А у тебя там на крыше много \emph{картин}? — спросил он.  — Несколько тысяч. Ведь я
сам рисую в свободное время. Я рисую маленьких петухов и птиц и другие красивые
вещи. Я лучший в мире рисовальщик петухов, — сказал Карлсон и, сделав изящный
разворот, приземлился на пол рядом с Малышом
%%%cit_comment
%%%cit_title
\citTitle{Карлсон, который живёт на крыше}, Астрид Линдгрен
%%%endcit
