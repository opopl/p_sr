% vim: keymap=russian-jcukenwin
%%beginhead 
 
%%file 28_12_2021.stz.news.lnr.lug_info.2.lgaki_kalendar
%%parent 28_12_2021
 
%%url 
 
%%author_id 
%%date 
 
%%tags 
%%title ЛГАКИ посвятила календарь на 2022 год педагогам-легендам и предстоящему юбилею вуза
 
%%endhead 
\subsection{ЛГАКИ посвятила календарь на 2022 год педагогам-легендам и предстоящему юбилею вуза}
\label{sec:28_12_2021.stz.news.lnr.lug_info.2.lgaki_kalendar}

Луганская государственная академия культуры и искусств (ЛГАКИ) имени Михаила
Матусовского издала календарь на 2022 год, который посвятила педагогам-легендам
и предстоящему 20-летию вуза. Об этом сообщила пресс-служба Министерства
культуры, спорта и молодежи ЛНР со ссылкой на вуз.

\ii{28_12_2021.stz.news.lnr.lug_info.2.lgaki_kalendar.pic.1}

\enquote{На страницах памятного календаря \enquote{Легенды ЛГАКИ: 20 лет
вместе} – лица преподавателей, которые составляют славу вуза и Республики,
основателей творческих школ, мастеров сцены, музыкантов, художников, творцов,
чьи имена знают все ценители искусства, а созданное ими навсегда останется
неотъемлемой частью культуры Луганщины}, – говорится в сообщении.

\ii{28_12_2021.stz.news.lnr.lug_info.2.lgaki_kalendar.pic.2}

В академии рассказали, что героев, изображенных на страницах календаря,
\enquote{выбирали всем ректоратом из числа тех, кто, к огромному счастью, продолжает
творить рядом, и в память о тех, кто уже ушел из жизни, но всю ее отдал
служению делу и вузу}.

\enquote{В издание включены фотографии и биографические справки о педагогах Ольге
Потемкиной, Василии Дунине, Вере Андрияненко, Вере Евдокимовой, Юрии Дерском,
Иване Комиссаренко, Александре Харютченко, Евгении Михалевой, Александре
Коденко, Альберте Левченко, Александре Редькине, Светлане и Дмитрии Витченко},
– уточнили в ЛГАКИ.

Там добавили, что, помимо этого, календарь посвящен 20-летию основания
академии, которое будет отмечаться в 2022 году.

Ранее ЛГАКИ имени Матусовского объявила о проведении конкурса эскизов почтовых
миниатюр к 20-летию академии.

Вуз был основан 8 апреля 2002 года. Сейчас обучение студентов ЛГАКИ ведется на
четырех факультетах и кафедре социально-гуманитарных дисциплин, в структуру
вуза также входят колледж, учебно-научный центр и Детская академия искусств.
