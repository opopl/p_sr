% vim: keymap=russian-jcukenwin
%%beginhead 
 
%%file 29_04_2020.stz.news.ua.mrpl_city.1.vira_shevcova
%%parent 29_04_2020
 
%%url https://mrpl.city/blogs/view/mariupolska-aktrisa-vira-shevtsova-virte-u-vlasnij-uspih-ta-peremogu
 
%%author_id demidko_olga.mariupol,news.ua.mrpl_city
%%date 
 
%%tags 
%%title Маріупольська актриса Віра Шевцова: "Вірте у власний успіх та перемогу!"
 
%%endhead 
 
\subsection{Маріупольська актриса Віра Шевцова: \enquote{Вірте у власний успіх та перемогу!}}
\label{sec:29_04_2020.stz.news.ua.mrpl_city.1.vira_shevcova}
 
\Purl{https://mrpl.city/blogs/view/mariupolska-aktrisa-vira-shevtsova-virte-u-vlasnij-uspih-ta-peremogu}
\ifcmt
 author_begin
   author_id demidko_olga.mariupol,news.ua.mrpl_city
 author_end
\fi

Поки карантин триває, справжнім театралам доводиться запастися терпінням і
чекати на зустріч з улюбленими акторами. Втім маріупольські служителі храму
Мельпомени намагаються проводити прямі трансляції, спілкуватися зі своїм
глядачем, ділитися планами на майбутнє. Думаю, всім причетним до театральної
культури Маріуполя буде цікаво ближче познайомитися з однією з найбільш
яскравих і харизматичних актрис драматичного театру, талановитою і
різноплановою \emph{\textbf{Вірою Шевцовою}}. З перших ролей, зіграних на маріупольській сцені,
вона з легкістю закохувала в себе глядачів. І це не дивно! Актриса гармонійна і
виразна в кожній ролі, дуже цікаво і своєрідно розкриває образи своїх героїнь.

\ii{29_04_2020.stz.news.ua.mrpl_city.1.vira_shevcova.pic.1}

Віра народилася у місті Кривий Ріг Дніпропетровської області в сім'ї економіста
– мами, \emph{Алли Іванівни} та електрика – батька, \emph{Володимира Володимировича}. Мати з
дитинства привчала дівчинку до мистецтва. Вона сама грає на фортепіано і згодом
Вірі також захотілося вступити до музичної школи. Там вона навчилася грі на
фортепіано та бандурі і потрапила до фольклорного дитячого колективу \emph{\textbf{\enquote{КРОК}}} під
керівництвом Самойлової Світлани Вікторівни та до театральної студії під
керівництвом Лупеко Олени Іванівни. Відтоді в її життя прийшло захоплення
музикою і театром. З \enquote{КРОКОМ} вона вивчала календарно-обрядові свята та звичаї
українців, які разом з іншими учасниками відтворювала на сцені, потім
відвідувала міжнародні фестивалі в містах України та Польщі. З дитинства Віра
любила українські народні пісні, співала в хорі, навчилася азам народного
вокалу. Їй дуже подобалося бути причетною до чогось творчого та цікавого, тому
вже тоді вона зрозуміла, що її професія буде пов'язана саме з мистецтвом.

\ii{29_04_2020.stz.news.ua.mrpl_city.1.vira_shevcova.pic.2}

В дитинстві майбутня актриса захоплювалась багатьма професіями. Вдома
намагалася приміряти їх на себе, тепер вже розуміє, що підсвідомо вибирала
акторський шлях. Віра вважає, що актор повинен розбиратися в усіх професіях,
хоча б поверхнево, бо ніколи не знаєш, яку роль доведеться тобі грати завтра
лікаря чи рок-зірку. Коли їй виповнилося 17 років і потрібно було вирішувати
куди вступати, вона обирала між співачкою та актрисою. Спочатку поїхала на
прослуховування до Студії ім.  Григорія Верьовки. Їм потрібні були дівчата –
альти і її прийняли. Але в неї ще лишався час і наполеглива Віра поїхала до
Харкова на запрошення її друга, який вже навчався на актора. Приїхавши на
консультації до Харківського Національного Університету мистецтв ім. І. П.
Котляревського, на акторський факультет, відчувши атмосферу, побачивши таку
купу талановитої молоді, вона зрозуміла що хоче навчатися саме тут і ніде
більше.

\ii{29_04_2020.stz.news.ua.mrpl_city.1.vira_shevcova.pic.3}

Сьогодні дівчина з теплом згадує навчання в університеті та дуже вдячна своїм
викладачам: художньому керівнику курсу, заслуженому діячу мистецтв, режисеру –
\emph{\textbf{Садовському Леоніду Вікторовичу}} і педагогам курсу заслуженому артисту України –
\emph{\textbf{Юрію Степановичу Євсюкову}} та актрисі \emph{\textbf{Марії Олександрівні Бораковській}}.
Отримавши вищу освіту за спеціальністю \enquote{актор драми і кіно}, курс \enquote{Майстерня
55}, вона поїхала пробувати свої сили на маріупольській сцені.

До кожної ролі Віра ставиться з любов'ю і повагою, інакше, на її думку, образ
неможливо створити. Є ролі, які більш близькі її темпераменту і свідомості, але
взагалі вона любить працювати над різними ролями, більш за все подобаються
гостро-характерні, вони набагато об'ємніші. В них широке поле для пошуку
маленьких нюансів і деталей.

У майбутньому актриса хотіла б зіграти у виставі за п'єсою \enquote{Фрекен Жюлі}
шведського драматурга Августа Стріндберга, написаної у жанрі натуралізму.
Згодом автор захопився символізмом і помітив, що його п'єса наповнена
метафорами та символами, які повною мірою відображають внутрішній, емоційний
стан героїв. Віра вважає, що ця п'єса є дуже актуальною, вона розповідає про
відсутність любові та довіри між рідними та сторонніми людьми. І про згубні
наслідки відсутності цієї любові та довіри.

\ii{29_04_2020.stz.news.ua.mrpl_city.1.vira_shevcova.pic.4}

Сьогодні у актриси безліч різнопланових ролей. Вона дуже вдячна за довіру
режисерів, які пропонують їй ролі. Віра завжди намагається робити свою справу
чесно. Їй подобається сам процес створення своєї ролі та вистави загалом. Якщо
глядачі знаходять відгук у роботі театру, це велика перемога. Найголовніша
порада успіху для актриси – \emph{\enquote{любити свою професію, намагатися вдосконалювати
себе, працювати віддано і не звертати уваги на плітки. Ну і, звичайно, ніколи
не зупинятися у своїх досягненнях}}.

Найважливіше, що є в житті кожної людини – це її родина. Мама Віри, \emph{Алла
Іванівна Швецова}, завжди підтримує і вислуховує доньку, намагається бути в
курсі всіх подій її життя. Коли в театрі розпочинається створення нової
вистави, мама завжди просить у Віри п'єсу, щоб прочитати і проаналізувати
побачене з прочитаним. Нашій героїні пощастило в житті знайти відданих і щирих
друзів в університеті і в рідному театрі також. Вона з друзями завжди
підтримують одне одного, аналізують разом роботи, допомагають і підказують як
зробити краще.

Віра може знайти натхнення у будь-чому, наприклад, може надихнутися від
прогулянки до моря, у парку. Любить спостерігати за навколишнім середовищем і
людьми. Актриса розповіла, що в акторських колах є такий вираз, коли актор
бачить цікаву особистість він каже: \enquote{Це справжній персонаж}. Ось коли
гуляєш і бачиш \enquote{персонажа}, намагаєшся зрозуміти цю людину, перейняти
її зовнішні особливості, манеру, може навіть ходу – ось це надихає і Віру на
створення нових персонажів. Ще вона шукає натхнення у музиці, книжках. Читає
різну літературу: саморозвивальну, художню, професійну – акторську та наукову.
Іноді дивиться фільми, зокрема американські чи італійські. А ще дуже
надихається під час розмов зі своїми друзями, вони дуже часто навіть в гремерці
рефлексують з приводу якоїсь ідеї, або тієї чи іншої картини при створенні
вистави і тоді народжуються цікаві придумки, які згодом можуть потрапити до
акторської роботи.

Маріуполь наша героїня полюбила далеко не одразу. Спочатку він їй здавався
сірим та похмурим. Але потім вона почала для себе знаходити улюблені місця.
Найбільше Вірі подобається проводити час біля моря: 

\begin{quote}
\em\enquote{Азовське море дуже
особливе, не таке як Чорне, і має зовсім інший вплив на людей. Як на мене
Азовське море дуже добре \enquote{чистить думки}. Якщо голова забита усякими
проблемами і ти не знаходиш рішення, моя порада, йдіть до моря. Мені
подобається читати там книжки, коли не сильний вітер звичайно, а  просто
прогулюватись...}. 
\end{quote}

Також дівчина любить гуляти у приморському районі восени та
взимку, найбільше навесні і влітку гуляє на Лівому березі. В районі парку
\enquote{Веселка} (до речі, теж одне з улюблених місць актриси, особливо
ввечері) любить спуск до моря, дуже крутий.  Гуляючи там можна поєднати спорт з
прогулянкою. Зазвичай там набагато менше людей, тому можна просто нікуди не
поспішати, а просто гуляти вздовж моря – така прогулянка наповнює Віру енергією
та зарядом бадьорості на усю неділю.

Незважаючи на молодість та велику кількість інтересів, актриса дуже
цілеспрямована, старанна і наполеглива. Вона вважає, що завжди і у всьому
повинен буди сильний мотиватор: \emph{\enquote{чим вищі цілі ми собі ставимо, тим більшого
зможемо досягти в житті}}.

\begingroup
\em

\textbf{Улюблена книга:} 

\begin{quote}
\enquote{Я не можу назвати одну, тому назву декілька.  Мейсон Каррі
\enquote{Режим гения. Распорядок дня великих людей}, Джордж Оруелл \enquote{1984}, \enquote{Скотний
двор}, Метью Уокер \enquote{Зачем мы спим}, Юрій Альшиц \enquote{Акторський тренінг}}.
\end{quote}

\textbf{Улюблений фільм:} 

\begin{quote}
\enquote{Найкраща пропозиція} з Джефрі Рашем, режисер Джузеппе
Торнаторе.
\end{quote}

\textbf{Хобі:} 

\begin{quote}
\enquote{Не можу назвати щось одне своїм хобі, я люблю співати, малювати,
займатись спортом, розмальовувати, аналізувати, колись любила вишивати. Я дуже
швидко захоплююсь, така в мене натура, якщо щось подобається я одразу починаю
цим займатися. Ще є у мене така особливість я збираю цукор з різних міст,
країн. Коли я подорожую, або хтось з моїх друзів то я прошу їх привезти стік
цукру}.
\end{quote}

\textbf{Порада маріупольцям:} 

\begin{quote}
\enquote{Я б порадила йти вперед і не зупинятися. Якщо у вас є
мрія, зробіть усе можливе, щоб вона стала реальністю. Раджу не слухати поради
усіх людей які вам оточують, прислухатися лише до рідних та близьких людей і
вірте у власний успіх та перемогу. І у вас все вийде! І звичайно я раджу усім
маріупольцям частіше відвідувати театр, ділитись своїми враженнями, думати,
міркувати, веселитися. Адже просто перегляд однієї вистави іноді може надихнути
людину і змінити її життя}.
\end{quote}

\textbf{Курйозний випадок з життя:} 

\begin{quote}
\enquote{У мене було дуже багато курйозних випадків як у
житті так і на сцені... Різне траплялося. Одного разу я поверталася з Кривого
Рогу до Маріуполя, і цей шлях видався дуже непростим. Здавалося що усе
складається так, щоб я не встигла приїхати до Маріуполя. На жаль немає прямого
потягу, тому мені доводилося їхати через Дніпропетровськ. Спочатку я
запізнилася на маршрутку до Дніпра, потім запізнилася на потяг і мої речі
буквально виштовхнули з вагону, потім я зрозуміла, що забула паспорт і чекала
на наступний потяг, не знаючи зможу я на нього сісти без документів... Було таке
відчуття, що я нервувала 12 годин, але у такі хвилини твій мозок починає
працювати трохи інакше і ти звертаєш увагу на якісь дрібниці. Було цікаво
спостерігати за зростом власного відчаю, для актора це корисно, можна додати в
емоційну скарбничку...}.
\end{quote}

\endgroup
