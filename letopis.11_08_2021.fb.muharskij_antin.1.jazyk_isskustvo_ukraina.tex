% vim: keymap=russian-jcukenwin
%%beginhead 
 
%%file 11_08_2021.fb.muharskij_antin.1.jazyk_isskustvo_ukraina
%%parent 11_08_2021
 
%%url https://www.facebook.com/AntinMykharskyi/posts/1678509629204564
 
%%author Мухарский, Антин
%%author_id muharskij_antin
%%author_url 
 
%%tags isskustvo,jazyk,kultura,mova
%%title РОСІЙСЬКОМОВНОГО УКРАЇНСЬКОГО МИСТЕЦТВА НЕ ІСНУЄ
 
%%endhead 
 
\subsection{РОСІЙСЬКОМОВНОГО УКРАЇНСЬКОГО МИСТЕЦТВА НЕ ІСНУЄ}
\label{sec:11_08_2021.fb.muharskij_antin.1.jazyk_isskustvo_ukraina}
 
\Purl{https://www.facebook.com/AntinMykharskyi/posts/1678509629204564}
\ifcmt
 author_begin
   author_id muharskij_antin
 author_end
\fi

РОСІЙСЬКОМОВНОГО УКРАЇНСЬКОГО МИСТЕЦТВА НЕ ІСНУЄ. 

Мистецтво — найвища форма політики, часом потужніша за гармати.

Коли говорять гармати — музи мовчать. Однак, коли гармати замовкають, музи знов
починають говорити й дуже важливо, якою мовою вони це роблять.

Під час Майдану багато хто перейшов на українську, відчувши силу справедливості
та народної стихії, але потім схибив, повернувшись у ментальний егрегор нашого
природного супротивника через власну непоборну належність до культурного поля
«южнорусской волни», до якого імперія автоматично зараховує всіх, хто говорить,
співає, або творить в Україні російською.

\ifcmt
  pic https://scontent-cdg2-1.xx.fbcdn.net/v/t1.6435-9/235001553_1678507439204783_6467686155569185463_n.jpg?_nc_cat=107&ccb=1-5&_nc_sid=730e14&_nc_ohc=_twNzHPieSgAX8a3xMI&_nc_ht=scontent-cdg2-1.xx&oh=b1f29bcd863e77faafe6c7957ac5d80b&oe=614CADE3
  width 0.4
	fig_env wrapfigure
\fi

Для тих, хто не в курсі — «южнорусская волна» — термін, введений в обіг
російським арткуратором та галеристом Маратом Гельманом (саме він придумав
славнозвісний плакат поділу України на три сорти) на початку 90-х, відколи в
Москві відбулася групова виставка «Вавилон», у якій брали участь художники
умовного півдня СРСР. Серед них нинішні класики (живі й мертві) Олександр
Ройтбурд, Олег Тістол, Олег Голосій, Олександр Гнилицький. Згодом до них
доєдналися Арсен Савадов, Георгій Сенченко, Юрій Соломко.

Ось чому російські мистецтвознавці завжди матимуть повне право стверджувати,
буцім не було ніякої України, а була «Южная Русь со своім смєшним нарєчієм,
кумєднимі аборігєнамі в красних штанах, што умєлі танцевать гопак, пєть
грустниє пєстні, рісовать прімітівних животних і устраівать майдани, в то врємя
как прогрєсівниє художнікі билі нашимі русскімі, ібо говорілі па русскі, пісалі
па русскі, абщалісь у фісбукє на русском язикє».

«Мистецтво поза політикою» — формулювання для недолугих простаків, бо все в
цьому житті — політика. Росія завжди зналася на термоядерному ефекті формування
культурно-історичної ідентичності народів на основі потужних мистецьких творів.
Шануєш Булгакова, співаєш Висоцького, цитуєш Гайдая — наш, русскій! 

Прив’язаність до російського історичного та культурного дискурсу, звичайно, не
робить із громадян України росіян, але тримає їх у колі ментального тяжіння,
даючи всі підстави кремлівським пропагандистам підживлювати тезу про «адін
народ», бо мислять стратегічно, як мантру повторюючи тезу: чия мова — того й
культура, чия культура — того й територія.

«А зачєм учіть ету мову? Я вєдь тожє люблю Украину. По своєму… І я за нєйо
воєвал. А Сєнцов сідєл. Украінскіє спортсмєни, прославляющіє страну в мірє —
поголовно всє русскоговорящіє, нам что іх в мусорку вибросіть? І вообщє какая
разніца, на каком язикє говоріт чєловєк, єслі он кровь готов пролівать за
родную зємлю ілі виступать под украінскім флагом?»

Та якщо «какая разніца», то говори українською? Але ж ні. Бо «какаяразніца» —
це завжди означає: «пєрєході на русскій».

Розбещення та аморальність — синоніми зросійщення. «Будь прощє, і люді к тєбє
потянутся» — улюблена приповідка російських зубоскалів, бо спрощення та
примітивізація — основна парадигма російського буття. Сприйняття власної
аморальності як норми — є єдиною державною політикою Росії, життєвою філософією
російського народу, у якій найбільш аморальні та злочинні персони досягають
найвищих щаблів у соціально-політичній, економічній та державотворчій
ієрархіях.

Я б не був зараз так стурбований цим питанням, якби не мав власного досвіду
початку 90-х, коли Свято Воскресаючого Духу (за Юрієм Андруховичем) перших
років української Незалежності, з настанням епохи Кучми-Табачника-Медведчука
було закатане в асфальт подібними, уже добре відпрацьованими КДБ-ФСБ методами,
коли роль суспільних моральних авторитетів у публічному просторі відводилась
виключно російськомовним особам. Тут йдеться передусім про популярних естрадних
виконавців, телеведучих, зірок спорту. Це робиться з єдиною метою —
деморалізувати українців у їхній боротьбі проти російської окупації. Бо мова
для людей публічних аж ніяк не засіб комунікації, а, в першу чергу, —
декларація власної належності до того чи іншого культурного поля та
національної ідентичності. Бо якщо твої земляки знаходять із ворогом спільну
мову краще ніж із тобою, то може і ворог не ворог і ти не ти?

«Чорнила поета дієвіші за кров тисяч солдатів» — промовляє давньосманська
приповідка. Публічна персона культурного простору часом вартує цілої дивізії
або армійського корпусу.

Росія це прекрасно розуміє, підтримуючи російськомовну культуру на території
України в усіх формах її існування. Цього й досі, на жаль, не можна сказати про
Україну, де боротьба за національну ідентичність є справою (за визначенням
проросійських ЗМІ та їхніх адептів) «жалкой кучкі азабочєнних гарадскіх
сумасшедшіх», до складу якої входить і ваш покірний слуга.

Моя оптика на ти чи інші події, оцінки діяльності тих чи інших персон
налаштована під кутом запитання: «чи приносить це користь українській Україні»?
Якщо ні, то навіть найталановитіший російськомовний член суспільства, збираючи
мед у стільники інших світоглядних концептів не являє для мене жодної цінності,
але якщо поводиться нейтрально, то й моє ставлення до нього є нейтральним.
Проте воно різко змінюється в разі  вживання ним антиукраїнської риторики, до
якої (без виключень) завжди в той чи інший спосіб  удаються в представники так
званої категорії «русскоговорящіх украінцєв». Їхнє  ставлення до мовного
питання є визначальним фактором моєї особистісної до них лояльності. Бо мова –
це Альфа і Омега української ідентичності. З неї українство починається, з її
невживанням воно й закінчується. 

Російська мова в Україні — мова радянських рабів та їхніх нащадків, які міцно
застрягли в минулому. Нині українською твориться переважна більшість модерного
і прогресивного контенту. Сьогодні практично неможливо уявити собі дійсно
вартий уваги мистецький твір (фільм, книгу), створений в Україні російською.

Все російськомовне приречене на стагнацію та повільне вмирання, оскільки не
підживлюється новими сенсами, концептами, енергіями, чіпляючись за імперські
міфи, або низькопробний кітч масової культури. Російська в побуті дедалі більше
перетворюється на ознаку інтелектуальної імпотенції. А її знання для
культурного освіченого українця цілком може обмежуватися фразою: «Пашол на
ххx…» на адресу представників \enquote{русскаго міра}.

Ось чому для всіх українських митців критично важливо зрозуміти:
російськомовного українського мистецтва не існує! Адже все, що нині твориться в
Україні російською мовою маркує її саме як «южнорусскую волну», що її треба
залишити в історії як яскраву сторінку постколоніального періоду, коли Україна
тільки шукала власну самість.

«Іди за мною й залиш мертвим ховати своїх мерців», — словами Христа промовляє
Україна, яку нам тільки належить збудувати з власних мрій.

Це був уривок із нашої з Elizabeth Bielska спільної книжки "Як перейти на
українську", котра побачить світ у перших числах листопада. 

Якщо ви маєте можливість і бажання підтримати видання, то зробити це можна у
два способи:

1. Передзамовити книжку за 300 грн зараз, щоб отримати з автографом серед
перших.

2. Підтримати сумою від 1000 грн, тоді ваше ім'я буде вказане на спеціальній
сторінці подяк у книзі, яку отримаєте, щойно вона вийде друком, разом з
автографом, дипломом-подякою та браслетом "Шляхетні люди говорять українською".

Вичерпна інформація на сайті UKRIDEABOOK.

PS Поширення допису теж є великою допомогою! 🙂

Дякую за світлину київському фотографу Руслан Михайлович.

\ii{11_08_2021.fb.muharskij_antin.1.jazyk_isskustvo_ukraina.cmt}
