% vim: keymap=russian-jcukenwin
%%beginhead 
 
%%file 27_05_2020.stz.news.ua.mrpl_city.1.oleg_shuhman
%%parent 27_05_2020
 
%%url https://mrpl.city/blogs/view/oleg-shuhman-najgolovnishetse-lyubiti-svoyu-simyu
 
%%author_id demidko_olga.mariupol,news.ua.mrpl_city
%%date 
 
%%tags 
%%title Олег Шухман: "Найголовніше – це любити свою сім'ю!"
 
%%endhead 
 
\subsection{Олег Шухман: \enquote{Найголовніше – це любити свою сім'ю!}}
\label{sec:27_05_2020.stz.news.ua.mrpl_city.1.oleg_shuhman}
 
\Purl{https://mrpl.city/blogs/view/oleg-shuhman-najgolovnishetse-lyubiti-svoyu-simyu}
\ifcmt
 author_begin
   author_id demidko_olga.mariupol,news.ua.mrpl_city
 author_end
\fi

У травні весь світ святкував \emph{Міжнародний день сім'ї}, а 1 червня традиційно
відзначатиметься \emph{День дітей (Міжнародний день захисту дітей)}. З нагоди цих свят
хочеться розповісти про людину, чия діяльність допомагає маленьким маріупольцям
відчути себе більш захищеними, а найголовнішою цінністю в житті є сім'я. \emph{\textbf{Олег
Еріксонович Шухман}} – президент \emph{Регіональної Асоціації Бойових мистецтв і спорту
Донецької області (RAMAS)}, видатний тренер чемпіонів Європи і тато чемпіонів
світу з карате.

\ii{27_05_2020.stz.news.ua.mrpl_city.1.oleg_shuhman.pic.1}

Народився Олег в Кіровограді, але коли йому виповнився рік, батьки переїхали до
Маріуполя. З 15 років почав займатися самбою і дзюдо. Тренер Олега – \emph{Голінський
Вадим Іванович} – зміг надихнути хлопця і заразити бажанням й самому тренувати.
Втім спочатку для хлопця тренерство було лише хобі. За освітою Олег
технік-електрик та вчитель фізичної культури. У 1978 році здобув звання
кандидата у майстри спорту СРСР. Сьогодні чоловік є заслуженим тренером
України, офіційним представником Міжнародної Асоціації Кобудо в Україні,
Головою суддівського корпусу УАК \enquote{Госоку Рю Будо}. 

Постійна підтримка дружини надихнула чоловіка і посприяла створенню спільного
тандему. Щодня він займається карате з дітьми-сиротами з Лівобережного
інтернату, де працює педагогом-організатором його дружина \emph{Людмила}. Вони
намагаються зробити життя сиріт легшим і яскравішим. Проводять спільні свята.
Щороку в інтернаті проводиться конкурс \emph{Містер школи} і \emph{Міс Весна}. Олег
Еріксонович дарує переможцям солодощі та подарунковий грошовий сертифікат, на
який діти можуть собі щось купити. Чоловік тренує і дітей з інвалідністю, що,
безперечно, має позитивний вплив на розширення функціональних можливостей, а
також на їхню соціальну адаптацію.

Олег Еріксонович дуже любить Маріуполь зі всіма його плюсами і мінусами. Він не
втомлюється повторювати, що місто, в якому живеш, потрібно не тільки цінувати,
а й прославляти, чим він невпинно і займається.

\ii{27_05_2020.stz.news.ua.mrpl_city.1.oleg_shuhman.pic.2}

Його вихованці неодноразово брали участь в чемпіонатах Європи та світу і
привозили до Малої Батьківщини безліч нагород. Зокрема, на чемпіонаті України з
пара-карате, який відбувся у Львові, вихованці Олега Еріксоновича та його синів
Віктора Олеговича та Олега Олеговича завоювали 14 золотих, 14 срібних і 10
бронзових медалей, посівши 1 загальнокомандне місце.  Багаторічна праця тренера
принесла свої плоди. Олег Еріксонович у спортивному світі користується
неабияким авторитетом, що підкреслюють присвоєні йому звання судді міжнародної
категорії та інструктора-екзаменатора міжнародної категорії.

Сім'я у маріупольця велика, дружня, і, звичайно, спортивна. Два сина і одна
донька – чемпіони світу. Донька – \emph{Данилова Владислава Володимирівна} – майстер
спорту України, чемпіонка світу з карате. \emph{Шухман Віктор Олегович} та \emph{Шухман Олег
Олегович} – неодноразові чемпіони світу та Європи з карате та кобудо, майстри
спорту міжнародного класу. Дві онучки вже також є чемпіонками світу з карате і
кобудо. Найменшому онуку 5 років, але в сім'ї впевнені, що і він продовжить
родинну традицію.

\ii{27_05_2020.stz.news.ua.mrpl_city.1.oleg_shuhman.pic.3}

Планів на майбутнє багато. Цього року Маріупольській школі шотокан карате і
кобудо, яку започаткував Олег Еріксонович, виповнюється 25 років. Однак, на
жаль, пандемія завадила багатьом планам. Якщо б не карантин, його вихованці
взяли б участь і в чемпіонатах світу.

\ii{27_05_2020.stz.news.ua.mrpl_city.1.oleg_shuhman.pic.4}

\emph{\textbf{Хобі:}} робота для нашого героя і є хобі, але разом з тим він полюбляє займатися японською каліграфією та сумі-е – японським живописом.

\emph{\textbf{Улюблені книги:}} фантастика.

\emph{\textbf{Фільми:}} фантастичні, про космос. Зокрема, \enquote{Зоряні війни} (1986 р.), \enquote{Аватар} (2009 рік), \enquote{Володар кілець} (2001 рік)

\emph{\textbf{Порада маріупольцям:}} 

\begin{quote}
\em \enquote{Любіть свою сім'ю і у вас буде все в порядку. Запам'ятайте: Любов – це Бог, а
Бог – це любов, тому необхідно, перш за все, дарувати любов одне одному. В
цьому проявляється найвища духовність. І найголовніше – бережіть своє
здоров'я!}.
\end{quote}
