% vim: keymap=russian-jcukenwin
%%beginhead 
 
%%file 05_12_2021.fb.baumejster_andrej.kiev.filosof.1.skazka_andersen_snezhnaja_koroleva.cmt
%%parent 05_12_2021.fb.baumejster_andrej.kiev.filosof.1.skazka_andersen_snezhnaja_koroleva
 
%%url 
 
%%author_id 
%%date 
 
%%tags 
%%title 
 
%%endhead 
\subsubsection{Коментарі}

\begin{itemize} % {
\iusr{Андрей Баумейстер}

Прекрасный советский мультфильм \enquote{Снежная королева} (1957). Но из него выброшены
все религиозные сюжеты: сила молитв, Рождественский псалом как главная сюжетная
линия и, конечно, евангельская цитата \enquote{Если не будете как дети...}. Без
религиозной линии сама история. становится одномерной и во многом непонятной:

\href{https://youtu.be/1ajPJI5UyS0}{%
Снежная королева, youtube, 03.10.2013%
}

\ifcmt
  tab_begin cols=3,no_fig,center

     pic https://i2.paste.pics/1a42732a213c666df9f8bf3a10a2e7d2.png
		 pic https://i2.paste.pics/ac93135163c20cca871a4d48a506a2f1.png
		 pic https://i2.paste.pics/96d45b719d5c2b8fd71e08c9b423b129.png

  tab_end

  tab_begin cols=4,no_fig,center

		 pic https://i2.paste.pics/6160894283bfc97974bae133a9c99f88.png
		 pic https://i2.paste.pics/5df55df95d62b15db983380bcad54c39.png
		 pic https://i2.paste.pics/0d42ca1015f8d6fe58799e84d01abf7f.png
		 pic https://i2.paste.pics/8b8737634ca32b88c4daad1f1166d2be.png

  tab_end
\fi

\begin{itemize} % {
\iusr{Irina Zuevic}
\textbf{Andrii Baumeister} да, наверное, именно потому что не читала, не поняла как Вы. Многие его сказки читала, но эту смотрела.
\end{itemize} % }


\iusr{Андрей Баумейстер}

Советский фильм 1966 года (премьера 6.11.1967) по сценарию Шварца (1938). Шварц
вступает в творческий диалог с Андерсоном. У детей появляется взрослый
наставник, \enquote{духовный отец} (прообраз самого Андерсена). Он внезапно возникает в
самые сложные моменты путешествия Герды, чтобы ей помочь. Главный мотив у
Шварца сила духа и самопожертвования. Вместо зеркала Тролля магической силой
обладают поцелуй и дыхание Снежной Королевы:

\href{https://youtu.be/_XiBoSZwKUw}{%
Снежная королева (реж. Г. Казанский, 1966 г.), youtube, 02.01.2021%
}

\headTwo{Киностудия Ленфильм}

Фильм-сказка повествует о необыкновенном путешествии скромной маленькой девочки
Герды. Она разыскивает своего друга Кея, которого похитила и унесла в свое
царство Снежная королева - могущественная злая волшебница.

В поисках любимого друга Герда попадает в замок к хитрому, коварному и в то же
время смешному королю, знакомится с лесными разбойниками. На пути у девочки
будет много преград до решающей битвы со Снежной королевой. Но верное сердце
Герды одолеет все невзгоды...

\obeycr
Режиссёр: Геннадий Казанский
Сценарист: Евгений Шварц
Композитор: Надежда Симонян
Операторы: Вадим Грамматиков, Сергей Иванов
Художники-постановщики: Борис Бурмистров, Е. Яковлева, А. Вагин 
В ролях: Елена Проклова, Слава Цюпа, Наталья Климова, Евгения Мельникова, Николай Боярский, Евгений Леонов, Ирина ГубановаГеоргий Корольчук, Ольга Викланд, Эра Зиганшина
\restorecr

\ifcmt
  tab_begin cols=3,no_fig,center
		pic https://i2.paste.pics/2bc298d3dcd0db933fb0d05a52364256.png
		pic https://i2.paste.pics/2bc298d3dcd0db933fb0d05a52364256.png
		pic https://i2.paste.pics/dd4401dd75a76b4fc905d7855b0a054b.png

  tab_end

  tab_begin cols=4,no_fig,center
		pic https://i2.paste.pics/265046d41e42e4e055df0390719d35c4.png
		pic https://i2.paste.pics/094f2cd0a71c1c0ba47d7afb672e40b3.png
		pic https://i2.paste.pics/2100ab1a3824a7c09cb53667ed0d81b7.png
		pic https://i2.paste.pics/3d99f7ea52adc27d69d77e4a7ae73879.png

  tab_end

  tab_begin cols=3,no_fig,center
		pic https://i2.paste.pics/c1890533fcbbf65b5d4796f248bacbd2.png
		pic https://i2.paste.pics/71aaa245ac37844b8125855cbad05659.png
		pic https://i2.paste.pics/7110dcf23e8005b118906e7ae39f2732.png

  tab_end
\fi

\headTwo{Комментарии к видео}

\begin{multicols}{2}
\iusr{Светлана Щ}

как я детстве всё время ждала передачу \enquote{В гостях у сказки}, ждёшь когда
наступит пятница и бежишь из школы быстрей домой, чтобы успеть. Говорят,она
выходила в разные годы по-разному, но вот я запомнила пятницу. Поделки помню
как показывали, потом дома я сама что-то мастерила после передачи. И какие
чистые и искрение сказки были, и я бы даже сказала интеллигентные, т.к. не было
даже намёка на пошлости и мерзость, которую сейчас целыми днями показывают по
телевидению и смотрят в интернете.

\iusr{Наталья Проплетенина}

В детстве обязательно в доме на Новый год ёлка, мандарины и сказка Снежная
королева, именно этот фильм) Чудесный фильм, спасибо)

\iusr{Кахримонджон Махмудов}

Воспоминаю зимнее каникулы. У нас было настоящее детство. Спасибо тебе СССР.....

\iusr{Richard Mueller}

\enquote{Что враги с нами сделают, пока сердца наши горячи? Да ничего!} - ответил
Сказочник в конце фильма. И он прав!

\iusr{Richard Mueller}

Спасибо  моей любимой Родине СССР за непревзойдённое, детско-юношеское кино! За
воспитание  Человека с большой буквы, низкий поклон!

\begin{itemize} % {

\iusr{Александр Иванов}

Когда вышел этот фильм, все мерзавцы, развалившие и грабящие нашу страну как
раз были детьми, ровесниками Герды и Кая. Фильм они наверняка смотрели, а
теперь заявляют, что СССР кроме калош ничего не выпускал! Вот как так
получилось?

\iusr{Michael Chernyak}

и какое воспитание нам показывает данный фильм? Чудная сцена: принц и принцесса
играют во дворце... за ними носится дюжина пожилых людей со свечами, исполняя
малейший каприз. Высокие отношения! Потрясающий пример для советского
школьника!... типа-знай своё место, если ты не принцесса.

\iusr{Александр Иванов}

 @Michael Chernyak  Тут как раз хотели показать обратный эффект что быть
 принцессами- это стыдно.

\iusr{Michael Chernyak}
 @Александр Иванов  чота по фильму не видно. Наоборот, до финала принцесса
 развлекается и гоняет всех по дворцу.

\end{itemize} % }

\iusr{Richard Mueller}

Самые лучшие, самые человечные, самые добрые - это СОВЕТСКИЕ экранизации сказок!

\iusr{Игорь Литвиненко}

Какие прекрасные были сказки, союз мультфильм показывал мультики которые можно
было смотреть по 5-10 раз, а самое главное они были настоящими! Помню в детстве
собирались брали белое покрывало доставали фильмоскоп и смотрели красочные
мультфильмы! Сейчас же дети все в интернете. Жаль, нельзя вернуть время
Юппи, Зуко, жвачки турбо!

\iusr{Марина Марина}

Ребята! Спасибо Вам огромное, что выставляете Советские фильмы и мультики!!!
Они очень нам всем нужны!!!!!!!!! Счастья вам и здоровья!!!!!!!! Одесса

\iusr{Ирина Богданова}

\enquote{детей надо баловАть, тогда из них вырастут  настоящие разбойники!}  крылатые
слова!!!

\iusr{Vik Sud}
Когда  в детстве увидел заглядывающую в окно Королеву, спать не мог несколько ночей.

\begin{itemize} % {
\iusr{Mako Mako}
 @igg{fbicon.face.grinning.sweat}  @igg{fbicon.face.tears.of.joy}{repeat=5} так жизненно

\iusr{Ксюшечка-Оксаночка Обожаю петь! Это счастье!!!}
Бедняжка.

\iusr{Новый Аккаунт}
Снежная Королева виновата в том, что мужики стали боятся красивых женщин!
\end{itemize} % }

\iusr{Игорь Ануфриев}

Сказка и фильм и наш мультфильм изумительны. Проклова ещё чистая девочка, и
сейчас скотилась к грязным склокам (к большому сожалению). Спасибо всей
киносьемочной группе и знаменитой киностудии Ленфильм. Помним. Золотой фонд.

\end{multicols}

\zzrule

\iusr{Ната Руст}

кажется, сходный случай описан Бажовым в \enquote{Малахитовой шкатулке}: о любви
не:одушевлённого (или мёртвого) к живому в полной мере. такие случаи в
фольклорном периоде развития психологии подразумевали ТОСКУ как ведущую эмоцию
того живого, кого так несчастливо возлюбили. избегать, хоть на время,
губительного эффекта тоски можно за занятиями, требующими предельного внимания
для их выполнения. внимания на тоску тогда просто нет. а настоящее исцеление,
действительно, замешано на горячих слезах, проливаемых кем-то

\begin{itemize} % {
\iusr{Андрей Баумейстер}
\textbf{Ната Руст} да, было бы интересно проследить этот мотив
\end{itemize} % }

\iusr{Oleksandra Korchevska}

Чудовий вибір чим проілюструвати пост. @igg{fbicon.heart.red} До речі, це найпопулярніша книга в
українському ілюструванні Владислава Єрка, постійно бє рекорди продажів в
Америці і Британії в передріздвяний період. А ще якісні ілюстрації Єрка дали
привід почати співпрацю з Єрком для Джон Ролінг і Стівена Кінга, які визнали
його обкладинки найкращими. Є чим пишатись @igg{fbicon.heart.red}

\begin{itemize} % {
\iusr{Андрей Баумейстер}
\textbf{Oleksandra Korchevska} так, я ще розповiм про це видання

\iusr{Oleksandra Korchevska}
\textbf{Andrii Baumeister} очікую з нетерпінням  @igg{fbicon.face.smiling.hearts} 
\end{itemize} % }

\iusr{Lilia Pecheniuk}

Дякую за такий чудовий коментар до книги! Любимо і читаємо тепер вже й дітям  @igg{fbicon.face.happy.two.hands} 
А я ще досі маю конспекти з Ваших лекцій, яким вже скоро 20 років @igg{fbicon.wink} 

\iusr{Андрей Баумейстер}
\textbf{Lilia Pecheniuk} ого! Довго ж я читаю)))!

\iusr{Kirill Molodyko}
Да, с этой книги около 15 лет назад началось мое знакомство с творчеством Владислава Ерко. Это действительно Мастер.


\iusr{Андрей Баумейстер}
\textbf{Kirill Molodyko} это правда. Есть электронная версия издания, я о ней расскажу

\iusr{Юлия Чинарева}

А как же Галилей, якобы говоривший, что «математика - язык, на котором Бог
написал Вселенную?» Спасибо за такой текстуальной анализ!

\iusr{Андрей Баумейстер}
\textbf{Юлия Чинарева} я не согласен с Андерсеном в оценке математики  @igg{fbicon.smile} 

\iusr{Victor Magomet}
Андрей Олегович, а в какой версии посоветуете прочитать сказку? Благодарю.

\begin{itemize} % {
\iusr{Андрей Баумейстер}
\textbf{Victor Magomet} есть академическое издание в серии Литературные памятники. У меня оттуда перепечатка.

\iusr{Victor Magomet}
\textbf{Andrii Baumeister} Спасибо
\end{itemize} % }

\iusr{Michael Baskin}
Утро. Восьмой день Хануки совпадает в этом году с Вторым Воскресеньем, символично?


\iusr{Андрей Баумейстер}
\textbf{Michael Baskin} наверное

\iusr{Veronika Romanova}
Спасибо за разбор этой замечательной сказки! Как-то неожиданно, но по-новогоднему.., надо перечитать.
Да, и иллюстрации издания Абабагаламага очень люблю!


\iusr{Андрей Баумейстер}
\textbf{Veronika Romanova} да, это прекрасное издание! Я хочу рассказать о нем отдельно. В видео как раз его упомянул. Это наша гордость.

\iusr{Irina Zuevic}
А как знать, не слишком ли много интерпретируется, чего на самом деле в тексте может и нет?

\begin{itemize} % {
\iusr{Андрей Баумейстер}
\textbf{Irina Zuevic} это моя интерпретация. Я не претендую на роль литературоведа

\iusr{Мартин Тихолаз}
\textbf{Irina Zuevic} я люблю эту сказку, и, тк люблю ещё и детскую иллюстрацию, держу дома не одну снежную королеву. Попалась как-то и история переводов - супруги Ганзен правили сами себя много, шлифуя тексты и, после 17 года, делая их более подходящим для советских детей. Вот та Снежная, что была издана с рисунками Линча, видимо, содержит дореволюционную версию перевода, и там действительно очевидны христианские мотивы, да и сам текст наименее детский (а, впрочем, действительно ли детские сказки у Великой 4-ки (Андерсен, Гримм, Перро и Гауф?)

\iusr{Наталія Іщук}
Согласна с Вами. Большинство сказок Андерсена - христианские по своей сути. В этом он не одинок. Вспомним хотя бы Клайва Льюиса. Но никто так и не смог его превзойти.
\end{itemize} % }

\iusr{Анна Яковлева}
Совсем новая сказка!! Спасибо за разъяснение. Обязательно перечитаю. В мультфильмах ничего этого нет))

\iusr{Ганна Путова}

Когда я была маленькая, мама читала мне \enquote{Снежную королеву} и доходила до слов:
\enquote{Жил злой тролль, который всех ненавидел}, я спрашивала: \enquote{И нас?} А ведь это
так и есть.


\iusr{Світлана Філоненко}
 @igg{fbicon.face.smiling.halo}  Благодарю, Андрей Олегович @igg{fbicon.face.happy.two.hands} 

\iusr{Olena Ilnytska}
Моя улюблена казка! Я її періодично перечитую  @igg{fbicon.smile} 

\iusr{Наталя Вернидуб}
Дякую! Ніби ж з дитинства знаєш казку напам'ять, але ніколи не задумувалась над релігійністю сюжету.

\iusr{Елена Постольник}
Прямо возвращение в беззаботное детство, сугробы, плотный шарф на шее, мокрые ноги, красные щёчки и ещё большая вера в чудо!

\iusr{Олександр Мартиненко}
Красиво написано... Вам бы самому сказку написать) Уверен, это была бы весьма занятная история))

\iusr{Zhanna Dufina}

Есть ещё чудесная пластинка к фильму «Тайна Снежной Королевы». Там очень
проникновенные и печальные песни, а центральная тема - грусть Герды по
уходящему детству. Так выглядит мир взрослого, если удалить из него Чудо.

\obeycr
Когда повзрослеешь, становится страшно:
Сегодняшний день, он уже не вчерашний.
Хоть небо опять за окном голубое,
Но куклы уже говорят не с тобою.
Солдатики вдруг из коробки не встали,
Солдатики вдруг оловянными стали.
Вчера ты скакал в незнакомые страны,
И вдруг оказалось, что конь деревянный.
Всё стало вчерашним, всё стало иначе,
Ты взрослый и, значит, уже не заплачешь.
Сегодняшний день, он уже не вчерашний,
Когда повзрослеешь, становится страшно.
\restorecr

\iusr{Алла Ванецьянц}
Глубоко и красиво

\iusr{Tatiana Khoptynska}
Спасибо большое! тут же захотелось перечитать!

\iusr{Zrazhevska Nina}

\ifcmt
  ig https://scontent-mxp1-1.xx.fbcdn.net/v/t39.1997-6/s180x540/122199916_349573896375357_4873511739831421242_n.png?_nc_cat=102&ccb=1-5&_nc_sid=ac3552&_nc_ohc=P4im0xUD7TQAX_t5uPa&_nc_ht=scontent-mxp1-1.xx&oh=8538cd2ee870e49a9a335a7f535da6f0&oe=61B5008C
  @width 0.4
\fi

\iusr{Валерий Лема}
\href{https://www.facebook.com/100004146750457/posts/946445258837024/}{%
Сказка, Валерий Лема, facebook, 14.12.2017%
}

\iusr{Петро Розумний}

Одна из удивительнейших сказок прошлого.  @igg{fbicon.flame}{repeat=3}  Как
дивно ее грани раскрываются сегодня  @igg{fbicon.thinking.face} 

\end{itemize} % }
