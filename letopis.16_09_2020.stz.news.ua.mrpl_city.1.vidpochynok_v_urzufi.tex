% vim: keymap=russian-jcukenwin
%%beginhead 
 
%%file 16_09_2020.stz.news.ua.mrpl_city.1.vidpochynok_v_urzufi
%%parent 16_09_2020
 
%%url https://mrpl.city/blogs/view/vidpochinok-v-urzufi
 
%%author_id demidko_olga.mariupol,news.ua.mrpl_city
%%date 
 
%%tags 
%%title Відпочинок в Урзуфі
 
%%endhead 
 
\subsection{Відпочинок в Урзуфі}
\label{sec:16_09_2020.stz.news.ua.mrpl_city.1.vidpochynok_v_urzufi}
 
\Purl{https://mrpl.city/blogs/view/vidpochinok-v-urzufi}
\ifcmt
 author_begin
   author_id demidko_olga.mariupol,news.ua.mrpl_city
 author_end
\fi

Оскільки вересень обіцяє бути спекотним і комфортним для відпочинку, пропоную
ознайомитися з одним з найкращих курортів на узбережжі Азовського моря, який
відрізняється своєю чистотою, інфраструктурою та багатою історією. Це курортне
\emph{\textbf{село Урзуф}}, що розташоване в Мангушському районі.

\ii{16_09_2020.stz.news.ua.mrpl_city.1.vidpochynok_v_urzufi.pic.1}

Село засноване в 1779 році грецькими переселенцями з кримських поселень Урзуф
(Гурзуф), Кизилташ (Краснокам'янка). Тоді в Урзуф прибуло 185 осіб (96
чоловіків і 84 жінки). Отримавши, згідно з грамотою Катерини II, відповідні
права і привілеї, переселенці почали займатися землеробством, городництвом,
вівчарством. Село розвивалося, збільшувалася його чисельність. Вже в 1902 році
в Урзуфі проживало 2380 осіб (з них чоловіків – 1207, жінок – тисяча сто
сімдесят три). Наразі в найбільшому селі Празов'я живе три тисячі чоловік,
проте влітку це число зростає за рахунок туристів. Цікаво, що назва населеного
пункту прийшла від кримського \emph{Гурзуфа}. За однією з версій, ця назва не грецька,
а латинська. \emph{\enquote{Ursus}} у перекладі з латинської означає \emph{\enquote{ведмідь}}. І це пояснення
звучить досить логічно, адже кримський Гурзуф розташований неподалік від
Ведмідь-гори (Аю-Даг). За іншою версією, назва пішла від слова \enquote{горзаннум}. Так
ще до приходу татар до Криму у генуезців називалися фортеці.

Після невеликого історичного екскурсу хочу розповісти про найголовніші, на мою
думку, переваги Урзуфа, як курортної зони. По-перше, це прозора, як скло, вода,
дуже чисті пляжі та м'яке повітря. По-друге, повна відсутність комарів, що
відрізняє Урзуф від інших курортів. Загалом це пов'язано з тим, що поруч із
селом немає лиманів, які крім лікувальних грязей постачають з собою величезну
кількість комах. І, по-третє, приємно вражає розвинута інфраструктура села.
Пляжі Урзуфа впорядковані – є навіси, роздягальні, смітники, туалети. На березі
– атракціони на будь-який смак: гірки, \enquote{банани}, скутери, водні велосипеди. У
центрі селища – картинги, \enquote{Луна-парк}, дискоклуб, кінозали, інтернет-кафе.
Насправді цього літа мені пощастило відпочити на Білосарайці, в Мелекіне та
Уруфі. І, враховуючи всі переваги кожного з місць відпочинку, я можу точно
сказати, що найкраще було саме в Урзуфі. Думаю, що все перелічене є запорукою
для гарного відпочинку в селі і ще більшим зростанням його популярності з
кожним роком. До того ж, важливим є і великий вибір житла в пансіонатах та
базах відпочинку на будь-який смак і гаманець.

% koleso
\ii{16_09_2020.stz.news.ua.mrpl_city.1.vidpochynok_v_urzufi.pic.2}

Сучасний Урзуф дійсно вважається одним з найбільш популярних курортів на
Азовському морі. Сьогодні тут розташовано понад 40 баз відпочинку, включаючи 4
дитячих оздоровчих центри. За рік в село приїжджає близько 200 тисяч
відпочивальників з різних областей України, країн СНД і Східної Європи.
Територія села містить поклади молодої глини, що використовується для
профілактики низки захворювань. Цікаво, що в Урзуфі працює і наметове містечко.
Кемпінг \enquote{Скеля} розташований в самому селищі і добре обладнаний. У кемпінгу є
електрика, питна вода, душові, туалети, місця для шашлику і барбекю.

\ii{16_09_2020.stz.news.ua.mrpl_city.1.vidpochynok_v_urzufi.pic.3}

Для любителів історії цікаво буде дізнатися, що в Урзуфі є\par\noindent\enquote{Шкільний музей},
який був створений в 2003 році за підтримки селищної адміністрації. Фонд музею
налічує понад 100 експонатів, які розкажуть про цікаві факти з історії селища і
життя його мешканців. У музеї зібрані предмети побуту, одяг, господарські речі
та знаряддя праці, які використовували предки.

\ii{16_09_2020.stz.news.ua.mrpl_city.1.vidpochynok_v_urzufi.pic.4}

А справжньою гордістю жителів є \textbf{Храм Архістратига Божого Михаїла}, який
побудований в \emph{1876 році}, і є \emph{найстарішим храмом у всьому Приазов'ї}. Храм був
закладений на місці дерев'яної церкви, побудованої в Урзуфі грецькими
переселенцями. Висота храму становить 33 метри, а будівництво тривало протягом
15 років за участю архітекторів з Італії. Приміщення храму прикрашають понад 50
ікон у візантійському стилі. Храм пережив німецьку окупацію, а під час
радянської влади зовсім був закритий. Сьогодні Михайлівська церква – це
визначна пам'ятка культури та історії.

Ще однією місцевою визначною пам'яткою є колодязь \textbf{\emph{\enquote{Урзуфська казка}}}, який був
побудований в 2005 році. Всі знають, що на курорті питна вода здебільшого
привізна, а в цьому колодязі будь-який охочий може втамувати свою спрагу.

Втім, найголовнішим для всіх відпочивальників Урзуфа є не історичні пам'ятки,
парки чи розважальні центри, а насичене йодом морське повітря, прогріта сонцем
і прозора вода та широкі чисті пляжі, які щороку створюють комфортні умови для
відпочинку і оздоровлення.

\textbf{Читайте також:} \emph{Стало известно, чем удивят мариупольцев на День города}%
\footnote{Стало известно, чем удивят мариупольцев на День города, Ганна Хіжнікова, mrpl.city, 15.09.2020, \par%
\url{https://mrpl.city/news/view/otprazdnuem-masshtabno-stalo-izvestno-raspisanie-meropriyatij-na-den-goroda-mariupolya}
}
