% vim: keymap=russian-jcukenwin
%%beginhead 
 
%%file 25_11_2021.fb.valetov_jan.dnepropetrovsk.1.nuvorish
%%parent 25_11_2021
 
%%url https://www.facebook.com/ian.valietov/posts/4626878327347960
 
%%author_id valetov_jan.dnepropetrovsk
%%date 
 
%%tags bogatstvo,ekonomika,nuvorish,obschestvo,strana,ukraina
%%title Есть такое устаревшее слово и нестареющее понятие - нувориш
 
%%endhead 
 
\subsection{Есть такое устаревшее слово и нестареющее понятие - нувориш}
\label{sec:25_11_2021.fb.valetov_jan.dnepropetrovsk.1.nuvorish}
 
\Purl{https://www.facebook.com/ian.valietov/posts/4626878327347960}
\ifcmt
 author_begin
   author_id valetov_jan.dnepropetrovsk
 author_end
\fi

Есть такое устаревшее слово и нестареющее понятие - нувориш.

Нуово риш. Богатей. 

Новые деньги, недавно разбогатевший, если кто не знает суть термина.

В далеком 1990 году, когда выходцы из СССР были еще относительной диковинкой
для цивилизованного мира, будучи в славном городе Сиэттле я попал в дом
представителя старых денег. Ну, как старых? По американским понятиям -
состоянию семьи лет под 200. Не то, чтобы приплыли в Штаты на "Мэйфлауэре", но
весьма близко по годам.

И бизнес такой стремный, не пацанский по нашим понятиям - не нефть, газ,
металлы или оружие, а фабрика изготавливающая кабинетные рояли. 

Поместье красивое до боли в глазах, несчетное число акров, изумрудного цвета
трава, вдалеке пасутся лошади, просторный дом, стекло-дерево-камень без лепнины
и ангелочков, обставленный с невероятным вкусом... 

Хозяин в джинсах и банальной клетчатой рубеже поверх базовой футболки -
улыбчивый, терпимый к нашему пещерному инглишу, к разинутым в удивление ртам.
Мы привезли в США футбольный клуб Днепр, футболисты - народ простой, не всегда
воспитанный.

Я увидел стоящий изящный рояль с открытой крышкой, сел и начал что-то
наигрывать... Не Мурку, конечно, но и не Брамса -  что-то из эстрады и
корявенько. Оглянулся, сзади стоит хозяин. Я с извинениями встал. 

- Продолжайте, - успокоил меня он. 

И хитро улыбнулся.

- Вам будет приятно, наверное... на этом рояле играл Ростропович...

Не помню его фамилии и марку роялей не помню. Но очень хорошо врезалась в
память манера поведения: отсутствие спеси, и показной роскоши.

В 90-е годы мне довелось увидеть много быстро разбогатевших соотечественников.
Очень быстро разбогатевших. Иногда настолько, что обладатель поместья в Сиэттле
им и в подметки не годился бы.

Пройдя уровень сауны, паленного бухла, блядей и ворованных БМВ, они быстро
выходили на просторы длинных кокаиновых дорожек, дорогих эскортниц и заказных
мерседесов последних моделей, заказанных прямо с завода.

И при этом оставались жадными, ограниченными хамами, презирающими окружающих и
вытирающими ноги о всех, кто ниже по рангу и состоянию.

Быстрые деньги, ничего не поделаешь.

Вертолеты, самолеты, дни рождения с обедами и на царской посуде в средневековых
замках, ванны из Вдовы Клико или Кристалла, официанты, наряженные лакеями,
лакействующее окружение, гнущее спины при подношение подарков...

А попробуют не согнуть - барин затравит!

Потрясающая самоуверенность, плохое образование, недостаток воспитания,
жадность и презрение к окружающим в сочетании с большими деньгами - гремучая
смесь.

Вседозволенность вещь вкусная. Одно дело говорить: эскорты - нахуй! А другое
дело реально ездить без эскортов, когда нет прямой необходимости. 

И не летать на вертолетах МЧСа по своим личным делам, даже если очень надо на
юбилей.

И понимать, когда можно гулять юбилей с медведями, а когда не стоит. Ну, не ко
времени. Можно выпить с друзьями, но без купеческого размаха. Тем более, когда
у половины страны на тебя крапивница. И даты получились неподходящими - или
скорбеть официально, или фейерверки в горах запускать - совместить не
получится.

Нынешняя власть наработала себе на цугцванг. В чем-то она виновата сама, что-то
ей организовали недоброжелатели, но, если брать в целом, ничего особенного не
происходит. Каждую власть в Украине атаковали с момента окончания выборов и до
самого ухода. Это у нас национальный вид спорта.

Надо уметь лавировать, договариваться,  искать компромиссные решения. Это
политика, а не эстрадная тусовка, хотя обстановка чем-то схожа.

Ринат Леонидович - не ангел, родился без золотой ложки во рту и имеет некоторые
особенности в биографии, оказавшие большое влияние на формирование характера и
методов работы. Он договороспособен, если перед ним не стучать тем самым местом
по роялю. А если начинаешь стучать, то не обессудь. Он тоже нувориш, но
выживший, сохранивший и преумноживший свое состояние, а это дорогого стоит. В
сравнении с нынешними,  Ахметов - это уже старые деньги.  

Делать ему "козу" - стратегический просчет. Брать его на испуг худшая тактика
из всех.

Любая ошибка теперь будет выставлена всеобщее обозрение, прокомментирована,
увеличена до подходящих размеров и отлита в бетоне .

Ты этого хотел, Жорж Данден? Накося-выкуси!

Хуже всего, что страну будет лихорадить. Страдать о  этого будет не Ринат
Леонидович и не Зеленский с юбилярами, а экономика, курс гривни и наша
репутация.

С Путиным у ворот - это как раз то, что доктор прописал, чтобы окочуриться всем
миром.

Но людям, которые не выросли из вертолетов, замков и эскортов, на это
наплевать.

Боюсь, что для превращения из нуово риш в олд мани у них не останется ни
времени, ни возможности.

\ii{25_11_2021.fb.valetov_jan.dnepropetrovsk.1.nuvorish.cmt}
