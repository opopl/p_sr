% vim: keymap=russian-jcukenwin
%%beginhead 
 
%%file slova.myshlenie
%%parent slova
 
%%url 
 
%%author 
%%author_id 
%%author_url 
 
%%tags 
%%title 
 
%%endhead 
\chapter{Мышление}
\label{sec:slova.myshlenie}

Мышление, смысл, мыслить, осмыслить

Давайте скажем прямо: управленческий успех в мировосприятии
среднестатистического украинца совершенно не отличим от зрады.  Мы давно
поставили между ними постмодернистский знак равенства.  Когда Украине пророчат
будущее \enquote{серой зоны}, надо понимать, что прежде всего это -
\enquote{серая зона смысла}.  Успех Украины - экономический, геополитический,
социальный - может состояться. Лично для меня это предмет не веры (в данном
случае, вера - антипод \emph{мышления}), но надежды.  Я надеюсь, что в Украине рано
или поздно сформируется волевой центр - субьект, а не обьект истории.  Однако,
без смены \enquote{регистра \emph{мышления}} этого никогда (!) не произойдет,
\citTitle{Что имеют в виду, когда пророчат Украине будущее серой зоны?}, Владислав Михеев, strana.ua, 10.06.2021

Прежде всего, что это будет серая зона \emph{смысла}.  Успех? А как он туда
попал?  Знаете, что лично меня реально забавляет в современном дискурсе?  Две
вещи.  Во-первых, ежесекундное ожидание Апокалипсиса.  Да, не заслужили вы его
еще пока.  Не заслужили! Хотя... ) Во-вторых, репортажи о невиданных успехах:
от строительства лучших за всю историю Украины дорог до пластики лучших в
Украине губ, нереально счастливых семей и непревзойденных карьерных
достижений.  Агонизирующее благополучие... ,
\citTitle{Что имеют в виду, когда пророчат Украине будущее серой зоны?}, Владислав Михеев, strana.ua, 10.06.2021

