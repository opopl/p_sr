% vim: keymap=russian-jcukenwin
%%beginhead 
 
%%file 29_09_2021.fb.jermolenko_vladimir.1.svit_novyn
%%parent 29_09_2021
 
%%url https://www.facebook.com/volodymyr.yermolenko/posts/10158429056003358
 
%%author_id jermolenko_vladimir
%%date 
 
%%tags chelovek,illjuzia,novosti,obschestvo,prekrasnoje,solnce,zhizn
%%title Світ Новин
 
%%endhead 
 
\subsection{Світ Новин}
\label{sec:29_09_2021.fb.jermolenko_vladimir.1.svit_novyn}
 
\Purl{https://www.facebook.com/volodymyr.yermolenko/posts/10158429056003358}
\ifcmt
 author_begin
   author_id jermolenko_vladimir
 author_end
\fi

ми якось не усвідомили того, як опинилися у світі новин. У новинному хронотопі.
Світ новин - це світ, у якому щось постійно стається. Або надто погане, або
надто хороше. 

Медіа здебільшого нам розповідають про погане: bad news is good news. Соцмережі
здебільшого нам розповідають про (чиєсь) хороше: люди діляться своїми успіхами,
радостями, селфі-нарцисизмом і тд. 

Погані новини у медіа дають нам постійне розчарування ("світ котиться в
прірву"), хороші новини у соцмережах дають нам фрустрацію ("у всіх так добре, а
в мене ж як?").  але обидва світи - це ілюзія. Тому що у житті "новина" - річ
дуже рідкісна.  Здебільшого наш час наповнений "неновинами". Або рутиною, або
звичністю, або повторенням, або процесами з невідомим завершенням. Але в цьому
"неновинному" світі безліч класного і цікавого. Коли сонце сходить і заходить -
це не новина, але це прекрасно. Коли змінюються пори року, коли росте дерево,
яке ти посадив, коли твоя дитина з тобою жартує чи коли тобі кайфово від
розумної книжки чи неймовірної мелодії - це "неновина". Але це кайфово. І це є
життя.  

Новинний хронотоп - це важливо, бо події - це важливо. Але не треба плутати
його з життям) і він не має від нас затуляти прекрасної поетичної прози
звичності

\ii{29_09_2021.fb.jermolenko_vladimir.1.svit_novyn.cmt}
