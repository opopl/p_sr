% vim: keymap=russian-jcukenwin
%%beginhead 
 
%%file 25_10_2019.news.pravda_com_ua.SPARTA.PORTOS_gorbachev_i_elcin
%%parent 25_10_2019.news.pravda_com_ua.SPARTA
 
%%url 
%%author 
%%tags 
%%title 
 
%%endhead 

\subsubsection{П.О.Р.Т.О.С., Горбачев и Ельцин}
\label{sec:25_10_2019.news.pravda_com_ua.SPARTA.PORTOS_gorbachev_i_elcin}

Пока Тамара много и долго говорит, периодически ныряя в лирические отступления,
ее коллега Ольга Широкая молча макает кисточку в банку с краской и скрупулезно
выводит на плакате буквы для очередного мотивационного плаката. 

Лозунги в духе марксизма-ленинизма --- неотъемлемая составляющая идеологии
"С.П.А.Р.Т.Ы.", равно как и сокращения в лучших-худших традициях СССР.

\emph{Отдающими советской пропагандой лозунгами в коммуне увешано все, вплоть до туалета}

Ольга представляется "временно исполняющей обязанности гвардейца", проще
говоря, она --- заместитель Тамары.

"Я в "Б.К.Н.Л." с 1991 года, --- вспоминает Ольга. --- Собиралась поступать в УЗПИ
(сейчас --- Украинская инженерно-педагогическая академия) в Харькове, шла на
консультацию по математике. 

Возле входа в метро "Университет" ко мне подошла девушка: "Мы проводим
небольшой социологический опрос, есть несколько минут?" Я глянула на часы –
16:40. Ехать до института 10 минут, значит, несколько минут есть. Она спросила:
"Что такое счастье?" Так мы проговорили четыре часа".

С 1993-го "Б.К.Н.Л." стало называться "Б.К.Н.Л. --- П.О.Р.Т.О.С."
("Поэтизированное Объединение Разработки Теории Общенародного Счастья") имени
офицера группы "Альфа" Геннадия Сергеева. 

Название организации видоизменилось после кровопролития возле Белого дома в
Москве 3-4 октября. В разгар политического кризиса в РФ президент Борис Ельцин
отдал приказ штурмовать здание правительства и стрелять по своим
оппонентам-коммунистам из танков. Юрий Давыдов и другие участники "Б.К.Н.Л."
вышли защищать Белый дом.

"Альфу" Ельцин послал на расстрел, чтобы они постреляли всех защитников, –
рассказывает Ольга. --- Когда "альфовцы" зашли в Белый дом, Юра начал спрашивать,
что такое счастье у одного из командиров Геннадия Сергеева. Командир принял
решение не подчиняться приказу и выводить из здания беззащитных людей. 

И они в своих бронированных автобусах и танках начали вывозить защитников.
Благодаря этому наши вышли живыми. Сергеева убил снайпер, и с тех пор мы носим
его имя. А звание Героя России ему дали через бог знает сколько лет, когда
ФСБшники увидели, что общественность все знает и замять (дело) не удалось".

Через три года участники "Б.К.Н.Л. --- П.О.Р.Т.О.С." вновь принимали участие в
политических процессах в России --- поддерживали в предвыборной кампании Михаила
Горбачева.

"Мы не хотели лишаться своей страны, но нам не хватило опыта, --- ностальгирует
по СССР Тамара. --- На метро "Пушкинской" в Москве стояли с пикетиком, непрерывно
собирали подписи в поддержку Горбачева, разговаривали с населением, музыку 60-х
крутили. 

Был момент, когда оргкомитет собирался в гостинице "Россия", 27 января 1996-го
я там выступила и сказала, что у нас есть девять козочек и три свинки и мы
поднимем сельское хозяйство. В тот год несколько раз встречались с Михаилом
Сергеевичем и Раисой Максимовной (Горбачевыми)".

На выборах 1996 года в РФ снова победил Борис Ельцин, а Горбачев получил всего
0,5\%. Тем временем, участники организации обустроили свое хозяйство в деревне
Караван за Харьковом. Так появился учхоз "С.П.А.Р.Т.А.".

Сегодня территория коммуны включает пять домов и занимает площадь в 1,4
гектара. В разное время в "С.П.А.Р.Т.Е." держали лошадей, свиней, коз, овец,
уток, разводили пчел и выращивали грибы. На данный момент остались только 9
коров --- участники коммуны продают надоенное молоко и иногда делают из него
адыгейский сыр.

{\em
Из всех животных в "С.П.А.Р.Т.Е." осталось только девять коров, да и тех собираются продавать
\/}

О бывшем президенте РФ и других исторических персонажах напоминает нейминг
всего имущества --- как движимого, так и недвижимого. Именем Ельцина, Сталина и
Иуды названы туалеты, в то время как Толстого, Пушкина, Есенина, Маяковского
увековечили в именах сараев, в честь Ярослава Мудрого и Наполеона назвали
погреба, а имена Ликурга и Кобзаря присвоили коровникам.

"В честь людей, на которых хотелось бы быть похожими в первую очередь, названы
конструктивные постройки, --- аргументирует Ольга.

Туалет называется Сталин. Сколько у Сталина, 67 млн погибших? Есть еще туалет
Ельцин. По неофициальным данным, за два дня возле Белого дома 5 тысяч человек
было убито" (по разным оценкам, в противостоянии возле Белого дома погибло от
124 до 350 человек --- УП).

\emph{Все помещения в коммуне названы именами известных исторических персонажей}

Лингвистические эксперименты не заканчиваются на туалетах и сараях. Помимо
Ленина на флаге "С.П.А.Р.Т.Ы." можно обнаружить пятиконечную звезду --- символ
языка эсперанто, который также входит в Теорию Счастья, а также надпись La
Paco, что в переводе с эсперанто означает "мир". Значок с такой же символикой
носит и Тамара.

В коммуне считают, что глобальная "эсперантизация" поможет улучшить
коммуникацию между народами, а эсперанто как язык международного общения "самый
простой и изящный".

