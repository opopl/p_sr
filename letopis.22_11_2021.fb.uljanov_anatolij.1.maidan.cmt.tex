% vim: keymap=russian-jcukenwin
%%beginhead 
 
%%file 22_11_2021.fb.uljanov_anatolij.1.maidan.cmt
%%parent 22_11_2021.fb.uljanov_anatolij.1.maidan
 
%%url 
 
%%author_id 
%%date 
 
%%tags 
%%title 
 
%%endhead 
\subsubsection{Коментарі}

\begin{itemize} % {
\iusr{Анатолий Ульянов}
\href{https://dadakinder.com/txt/2021/11/maidan-2021}{%
ВЗГЛЯДЫ СТРАХА, dadakinder.com, 22.11.2021%
}

\iusr{Lara Putilina}

Все верно, Анатолий! Злятся тут некоторые комментаторы, потому что знают, что
правы вы! А некоторые, видимо, на работе )) Иначе как объяснить их присутствие
на вашей странице! Это прямо смердяковщина какая-то с их стороны

\iusr{Александр Кравченко}
Не для каждого, а для представителей определенного класса.

\iusr{Владимир Могилевский}
Революция ненависти и вражды. Иллюзии были только в самом начале, да и то, с допусками и закрыванием глаз

\iusr{Dima Privalov}
Анатолий а эта однокурсница вас читает? Неплохо было бы ее сюда в комментарии вытащить

\iusr{Tania Czy}

В одном моем любимом сериале был сюжет про пришельцев, что живут в двумерной
реальности, и для нас выглядят как плоские картинки на поверхностях.

С удивлением узнаю, что оказывается это не фантазия, и что бывают пришельцы из
одномерного мира, каковых можно, например, обнаружить в комментариях к данному
посту.


\iusr{Віталій Кекух}
Поїхати - це правильне рішення, нехай щастить

\iusr{Даша Жукова}
Типичный Толик. Майдан -зло и в таком роде. А вот БЛМ - это другое, это о ценностях))

\begin{itemize} % {
\iusr{Евгений Колис}
\textbf{Даша Жукова} ну блм левый в значительной мере. Майдан правый. Одно поэтому добро, другое зло. Вполне логично.

\iusr{Кирилл Линецкий}
\textbf{Даша Жукова} 

это ваш молодой человек так агрессивно бросился оскорблять пользователей facebook? 

\url{https://www.facebook.com/profile.php?id=100014848338120}

Не боитесь связываться с такими угрожающими и опасными для окружающих нацистами
и психопатами?))

\iusr{Даша Жукова}
\textbf{Кирилл Линецкий} вы о чем? Ви - типичный жополиз, цель которого заговорить неудобные для вашего кумира факты из его биографии)

\iusr{Даша Жукова}
\textbf{Кирилл Линецкий} ваш профиль недавно создан, понятно. У Толика подгорело и он решил уже со свежесделанного фейка бомбить)
\end{itemize} % }

\iusr{Man Meloman}

Наверное, для меня написано и правда моя. Толя, может переедешь в рашу на
парашу вонять со своим унылым говном? Может страданий и/или возмущения
поуменшится? По крайней мере, гармоничнее будешь выглядеть.

\begin{itemize} % {
\iusr{Кирилл Линецкий}
\textbf{Man Meloman} улюбленного корча наслухался?))

\iusr{Man Meloman}
\textbf{Кирилл Линецкий} мочу сплюнь потом вякай, уебан

\iusr{Кирилл Линецкий}
\textbf{Man Meloman} типичное "развитие", типичного нацика

\iusr{Man Meloman}
\textbf{Кирилл Линецкий}, чавкай молча

\iusr{Zenek Jewgieniusz}
\textbf{Man Meloman} а может тогда ты поедешь в забитое село, поднимать его с колен, патриот комнатный?

\iusr{Man Meloman}
\textbf{Zenek Jewgieniusz} , гля ещё одно вылезло уебище, притрушенное. Да иди ты на хуй к доктору со своей мочой, уебень, бля)))
\end{itemize} % }

\iusr{Ivan Box}

якщо в людини срачка, наприклад, то не обов’язково це проблеми державного рівня
(якість питної води, продуктів, медицини...). можливо, треба просто руки мити...

\iusr{Віктор Матвєєв}
Я думаю с блиндажа на вопе писать будет легче и проще как-то. Попробуйте.

\iusr{Даша Жукова}

"В результате майданов, и инспирированных ими политических процессов, я не могу
жить и работать в собственной стране"

Ты не можешь работать в стране, в том числе и потому, что выбрал эпатажный путь
раскрутки своей персоны. Когда приходил на эфиры и высокопарно чмырил людей,
иногда явно перегибая палку. Ты однажды жаловался, что там типа какой-то
художник тебе угрожает, если ты прилетишь в Киев, дескать - вот смотрите, у
меня еще одна причина боятся вернуться в Украину. Якобы ты его критиковал. Если
почитать твою "критику" в адрес того персонажа, не защищая его, но по факту -
там не было критики, ты весь текст хуесосил его, называя жопой. Жопа, жопа,
жопа. И таких примеров, когда ты перегибал палку - полно. Не строй из себя
великомученика, мажорчик, легко улетевший в комфортные штаты

\begin{itemize} % {
\iusr{Кирилл Линецкий}
\textbf{Даша Жукова} 

Человек, пришедший на эфир, не имеет права сказать о том, что ему, к примеру,
не по душе "творчество" плюгавенького и картавенького Вакхарчука или теперь на
украинском пространстве стало опасно заявлять даже о своих предпочтениях и
вкусах?


\iusr{Даша Жукова}
\textbf{Кирилл Линецкий} 

вы утрируете, преувеличиваете и приводите левый пример, свой, чтобы заговорить
тему. А я выше написала. Пример его "критики", когда он человека дерьмом
поливает. Таких примеров с Толиком хватает, на ТВ, когда он не "критиковал", а
откровенно хамил. Отчасти за этот эпатаж и хамство ему начали угрожать. Я не
оправдываю тех, кто ему угрожает, я о том, что он принцесса, мажорная, которая
очень выгодно прикинулась великомученицей

\iusr{Даша Жукова}
\textbf{Кирилл Линецкий} 

"на украинском пространстве стало опасно заявлять даже о своих предпочтениях и вкусах?"

Это в духе роспропаганды, прийти, высраться о запретах на укртв. Толику не
запрещали критиковать на укрТВ. Наоброт, он говорил то, за что его бы в США
могли наказать по закону, или бы он стал нерукопожатным. Толика раскручивали
еще с 2004 года, он появлялся на каналах ТЕТ, 5 канал, М1 и не только. Такой
бедный Толик

\iusr{Кирилл Линецкий}
\textbf{Даша Жукова} 

Эпатаж и хамство, если кто еще не заметил, уже давно превратились в
естественные сопроводители любых теле-радио программ и современных шоу. Если вы
так болезненно реагируете на словестные перепалки и игру в демократию, то не
смотрите СМИ и не заходите в т.ч. и сюда. Зачем себя насиловать, верно? Творец
имеет право на выражение своей думки. Зачем угрожать ему за это? В противном
случае Украина превратится (если еще не превратилась) в страну двойных
стандартов, где можно выражать свою точку зрения только если она совпадает с
линией "правящей партии".

\iusr{Vlad Ru}
Дело не в личности автора поста, а в описании ситуации. Лично мне оно кажется довольно точным.

\iusr{Даша Жукова}
\textbf{Кирилл Линецкий} великолепная логика)) Раз творец и все можно - хорошо, договорились))

\end{itemize} % }

\iusr{Блюминов Алексей}
Все верно, Анатолий.

\iusr{Роман Местный}

Во время этих ваших майданов, где-то в провинциальном украинском городе
запросто можно было пизды получить, если правильно не ответить на возглас
"слава Украине", так что те, кто вещают, что Майдан – это «ценность жизни
каждого человека, свободы высказывания и личной свободы, плюрализма в
противовес моноидеологии и единоправильности; ценность справедливости и
солидарности, уважения к правам человека, а ещё – принятия другого» это нахуй
упоротые в ноль люди.

\begin{itemize} % {
\iusr{Кирилл Линецкий}
\textbf{Роман Местный} это лишний раз доказывает правоту утверждения о том, что главной силой переворотов является недовольное своей же необразованностью провинциальное быдло

\iusr{Роман Местный}
\textbf{Кирилл Линецкий} невежество это лишь почва для манипуляции, причина же очевидно иная.

\iusr{Михайло Лебедь}
\textbf{Роман Местный} да и от ментов можно было за этот возглас тогда пизды получить. Да и без повода тоже. Но таким как ты ментовский сапог же ж - как поцелуй матери))) Ты не поймёшь.
\end{itemize} % }

\iusr{Николай Ленивцын}
Эк понабежало-то к тебе поцреотни

\iusr{Anatoly Vlasyuk}
Ну, вы страдаете потому, что ретранслируете взгляды Путина на Украину. К свободе слова это не имеет никакого отношения.

\begin{itemize} % {
\iusr{Анатолий Слободянюк}
\textbf{Anatoly Vlasyuk} свобода слова это, в том числе, и право ретранслировать взгляды Самого Путина. Получая за это вменяемые контраргументы, а не п...ли от уличной фашни под покровительством полиции.

\iusr{Anatoly Vlasyuk}
\textbf{Анатолий Слободянюк} Согласен. Но все происходит на фоне гибридной войны с Россией. Сдохнет Путин, распадется империя - пиздите что хотите, извините за мой французкий. А если вы увидели фашистов в Киеве, а не увидели фашизм в Путине, - тамбовский волк вам товарищ и смихуёчки барселонских коммунистов в спину. Удачи! Сеанс общения окончен.

\iusr{Анатолий Слободянюк}
\textbf{Anatoly Vlasyuk} генно-модифицированной войны йододефицитных патриотов с крипторептилоидами на самом деле

\iusr{Rostyslav Pelekhovych}

> вы страдаете потому, что ретранслируете взгляды Путина на Украину
\textbf{Anatoly Vlasyuk} , та якби ж то! Він страждає [якщо під цим мається на увазі еміграція] геть не за політичні погляди. А на політиці він спекулює, будуючи образ ледь не дисидента для своє тусовочки

\iusr{Кирилл Линецкий}
\textbf{Anatoly Vlasyuk} и не надейтесь. Достойные преемники уже появляются)

\ifcmt
  ig https://i.ytimg.com/vi/HRPslukGOpU/mqdefault.jpg?fbclid=IwAR0MTlH_ixcofNObOcknZW6Aod2tIrMDHUGNNNTiDfZ5Esm65K5k52jBNUU
  @width 0.4
\fi

\end{itemize} % }

\iusr{Михайло Лебедь}

Ну ты ж, Толик, в 2013, наверно, эмигрировал? Да, мученик?  @igg{fbicon.smile} 

И семья твоя, наверно ж, на подвале в пыточной? Что, кстати, соответствует
культуре российских оккупантов. Потому если б это было правдой, то было б
странно, что ты не доволен.

Тебе самому не надоело быть клоуном для клоунов, болтун?  @igg{fbicon.smile} 

\iusr{Кирилл Линецкий}

Все фашики и нацики всегда пользовались одной и той же схемой управления, а
именно- убеждали самые нищие и "обиженные" слои провинциального населения в
том, что те смогут отнять у "зажравшегося" горожанина кусок его сытой и
"непыльной" жизни, так несправедливо доставшийся ему, когда докажут всему миру
свою "избранность" и "титульность". А затем натравливали всю эту массу
озверевших провинциалов на остатки думающей интеллигенции и с помощью
переворотов приводили к власти тоталитарно-нацистскую хунту, к чему так
усиленно стремятся большинство постсоветских, управляемых мафиозными
"князьками" и по сути ничем не отличающихся друг от друга режимчиков.

\end{itemize} % }
