% vim: keymap=russian-jcukenwin
%%beginhead 
 
%%file 15_01_2022.fb.fb_group.story_kiev_ua.2.ulica_grafskaja
%%parent 15_01_2022
 
%%url https://www.facebook.com/groups/story.kiev.ua/posts/1840799389450187
 
%%author_id fb_group.story_kiev_ua,pestun_maksim.kiev.ukraina.pisatel
%%date 
 
%%tags kiev
%%title УЛИЦА ГРАФСКАЯ
 
%%endhead 
 
\subsection{УЛИЦА ГРАФСКАЯ}
\label{sec:15_01_2022.fb.fb_group.story_kiev_ua.2.ulica_grafskaja}
 
\Purl{https://www.facebook.com/groups/story.kiev.ua/posts/1840799389450187}
\ifcmt
 author_begin
   author_id fb_group.story_kiev_ua,pestun_maksim.kiev.ukraina.pisatel
 author_end
\fi

УЛИЦА ГРАФСКАЯ

Все мое детство было связано с классиками марксизма-ленинизма. Кроме того, что
мой дедушка скульптор занимался созданием монументов вождям, я умудрился
родиться на улице Ленина, а когда мне было пять лет, мы переехали на Энгельса
(Лютеранскую). Потом, уже в зрелом возрасте, следуя традиции, я чуть было не
поселился на улице Карла Маркса (Городецкого), но не сложилось...

Наш дом на Энгельса находился как раз в месте излома, где ровная улица
заканчивается и неожиданно круто сбегает к Крещатику. Как сказали бы классики,
«стремительным домкратом».

Во дворе у нас росло огромное, выше дома, старое дерево. Оно уютно закрывало
окна от солнца, и поутру там пели птички.

Однажды его срубили.

Это было одно из первых крупных переживаний моего детства. Я никак не мог
понять, зачем убивать такое красивое и величественное дерево, а когда мне
сказали, что оно заболело, я долго спрашивал у взрослых, почему же его не
лечили, оно же живое...

Такая же история повторилась через полвека с огромной липой во дворе дома, где
я сейчас живу на Большой Житомирской. Думаю, дереву было больше лет, чем самому
дому, а это первый каменный дом в детинце...

На Энгельса во дворе был мой детский садик с заросшими деревьями и кустами
террасами и горкой, спускающейся к театру Франко.

Напротив дома на Лютеранской была школа, куда меня отдали в первый класс.
Поскольку наши окна выходили как раз на ее фасад, мама могла из квартиры
наблюдать, как я заходил в класс и садился за парту, как раньше наблюдала за
мной в детском садике, который находился во дворе, тоже прямо напротив окон. А
на следующий год школу «репрессировали» и забрали под «сахарный институт», а
меня перевели в 147 школу во дворе «пожарки», где сегодня лицей. А потом и ее
переименовали в 132-ю.

Наша школа в народе называлась «Уриловка», по имени директора. Его сын учился в
нашем классе.

Немного ниже нашего дома на другой стороне и сегодня стоит необычное старинное
здание Сулимовка – бывший приют для женщин и девочек. Он получил свое название
от имени хозяина дома Сулимы, родственника гетмана. Его вдова передала
роскошное по тем временам строение и усадьбу городу под бесплатный приют. В
здании была красивая домовая церковь Александра Невского, но до наших дней она
не дожила, как и первоначальный вид дома, строившегося как дворец. Основанный
здесь приют имел долгую историю. Его опекой занимались жены всех губернаторов
Киева, а в нижнем ярусе позднее была устроена известная в городе пекарня. Еще
рассказывали, что в окнах второго этажа видели привидение, что вполне возможно,
учитывая необычную историю здания. С приходом советской власти приют прекратил
свое существование.

Я часто бывал в этом странном доме, где в коммуналках жили мои одноклассники. 

А выше стояло заброшенное тогда здание Лютеранской кирхи. Мы с товарищами часто
пробирались туда поиграть. Рассказывали, что в нем снимали фильм «Вий», но это,
конечно же, выдумки. Там всегда было прохладно и очень интересно, хотя и
немного жутковато, и, если мы заигрывались до темноты, казалось, что
мистическая панночка в любой момент может вспорхнуть с огромного балкона
второго яруса. Так что обратно мы обычно выбирались бегом.

Еще, как я слышал, туда в 1941 году сносили радиоприемники со всего Киева по
приказу немецкой администрации…

Каждое утро меня будил натужный звук мотора старенького хлебного фургона, с
трудом ползущего по крутому подъему. Вообще, машины тогда на улице были большой
редкостью – несколько в час.

Но самое любимое время для нас, детей, была зима. Улицу основательно заносило
снегом не в конце марта, как сейчас, а с самого начала зимы, и он так и лежал
до весны.

Мы почти каждый день катались на санках и иногда, не успев затормозить,
вылетали в арку, которая выходит на Крещатик. Эти ощущения и запах морозного
воздуха я буду помнить всегда.

А весной по улице бежали ручьи, по сегодняшнему дню почти реки. По ним мы
пускали кораблики из коры деревьев с парусами из школьных тетрадей и бежали
вниз посмотреть, чей заплывет дальше.

Ручьи катили по горке маленькие камушки и кусочки битых кирпичей, превращая их
в коричневые, почти идеально круглые шарики, которые я почему-то часто
вспоминаю. Больше я таких в городе нигде не встречал.

Тогда я не знал, что живу на бывшей Графской улице, где начинал свой киевский
период Константин Паустовский, жил философ Николай Бердяев и собиратель
украинской старины Василий Тарновский. Учились сестры Михаила Булгакова, жили
Максим Горький и актриса Мария Андреева, часто бывали Александр Куприн и Михаил
Старицкий. История многих известных далеко за пределами Киева людей
переплеталась с этой улицей.

Но мне Лютеранская дорога тем, что с ней связаны прекрасные воспоминания моего
детства. А что может быть лучше детства в любимом Городе и на любимой Улице?!..
