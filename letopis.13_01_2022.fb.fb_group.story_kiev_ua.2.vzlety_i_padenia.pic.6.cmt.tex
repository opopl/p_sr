% vim: keymap=russian-jcukenwin
%%beginhead 
 
%%file 13_01_2022.fb.fb_group.story_kiev_ua.2.vzlety_i_padenia.pic.6.cmt
%%parent 13_01_2022.fb.fb_group.story_kiev_ua.2.vzlety_i_padenia
 
%%url 
 
%%author_id 
%%date 
 
%%tags 
%%title 
 
%%endhead 

\iusr{Ирина Молибог}
Ах, как хочется ворваться в этот двор!!!!

\iusr{Владимир Новицкий}
\textbf{Ирина Молибог} Представляю Вам эту возможность Ирочка

\ifcmt
  ig https://scontent-frx5-2.xx.fbcdn.net/v/t39.30808-6/270278315_4713293552051227_5035898628162573628_n.jpg?_nc_cat=109&ccb=1-5&_nc_sid=dbeb18&_nc_ohc=_vRZMNy0oe4AX8UzWYq&_nc_ht=scontent-frx5-2.xx&oh=00_AT_f7-eqc2TWlL2nnP0MVlftx9Wo51cr44IRCHHIessnDg&oe=61E6882E
\fi

\iusr{Ирина Молибог}

Спасибо, возвращаюсь в детство, но с совсем другим настроением. 1955 год мы с
родителями вернулись в Киев мне 5, 5 лет,отец военный, служили за границей и на
Дальнем Востоке. Бабушка коренная киевлянка с 1850 года семья её прапра...
приехали в Киев из Чехии по приглашению отстраивать город Киев. Бабушка жила на
Подоле. ,окончила 1 женскую гимназию,курсы сестер милосердия его императорских
... при бол. Зайцевых.. Посещаем её друзей и родню по всему городу. Жили у
бабушки до 1959 года, получил папа квартиру как отставной офицер на Лукьяновке
,в сталинке, отдельная квартира. Я во время походов с бабушкой до 10-12 лет,
скукоживалась вся внутри от страха: подъезды с запахами... обшарпанные стены
,двери, перекошенные домики, веранды -,киевское название, с перекошенными
ветхими перилами. Всё это подкреплялось рассказами взрослых о печальных днях
начиная, с их детства до современных дней, .не любила и боялась Подола.
Лукьяновка -любовь, пешком на Крещатик. И вот чудо - взрослеешь и понимаешь,
нет роднее детских улиц, каждый дом помнит моих дедушек, прапрабабушек. Ваши
фото - общее чувство щемящего сердца. с уважением Ира Молибог

\iusr{Люся Киевская}
Выжил ли Зямка после войны ?!

\iusr{Владимир Новицкий}
Да выжил, этот снимок сделан уже после освобождения Киева. У него был ещё драт близнец Ромка.

\iusr{Лена Шклярська}
Все ли живы, кто на этом фото?
