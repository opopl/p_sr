%%beginhead 
 
%%file 28_02_2023.fb.suhorukova_nadia.mariupol.1.zvuk_bombarduvalnik_
%%parent 28_02_2023
 
%%url https://www.facebook.com/permalink.php?story_fbid=pfbid02HdykF32t8bkwp2Ufpf3uqMMwzDxLc87Dj8ASdz5L3UwpsSyYD6HZjJnmsXnzpRZ5l&id=100087641497337
 
%%author_id suhorukova_nadia.mariupol
%%date 28_02_2023
 
%%tags mariupol,mariupol.war,samolet,samolet.bombardirovschik,vojna.2022
%%title Звук бомбардувальників був таким потужним, що я боялася - літак зачепить крилом будинок і впаде на нас
 
%%endhead 

\subsection{Звук бомбардувальників був таким потужним, що я боялася - літак зачепить крилом будинок і впаде на нас}
\label{sec:28_02_2023.fb.suhorukova_nadia.mariupol.1.zvuk_bombarduvalnik_}

\Purl{https://www.facebook.com/permalink.php?story_fbid=pfbid02HdykF32t8bkwp2Ufpf3uqMMwzDxLc87Dj8ASdz5L3UwpsSyYD6HZjJnmsXnzpRZ5l&id=100087641497337}
\ifcmt
 author_begin
   author_id suhorukova_nadia.mariupol
 author_end
\fi

Звук бомбардувальників був таким потужним, що я боялася - літак зачепить крилом
будинок і впаде на нас. 

Тринадцятирічний Сашко повторював скоромовкою: 

\enquote{Щоб тебе збили над морем, щоб тебе збили над морем...}

Моя мама запитала:

\enquote{Чому саме над морем?}

\enquote{А щоб якнайдалі від міста. Щоб усі залишилися живими}. 

Але літак не збивали. 

Спочатку він був один, потім їх стало два, потім три. 

Згодом ми перестали рахувати. 

Ми змирилися: якийсь із них присудить нам смерть.

\enquote{ГРАДи} чутно здалеку, таке враження, що наближається поїзд. 

Земля починає вібрувати і тремтіти. 

Потім звук дедалі ближче, підлога ходить ходором, усередині все замерзає. 

Відчуття, що лежиш на рейках і зараз тебе переїде чорне смердюче колесо.

Розчавить разом із будинком, із містом, із цілим світом.

Звук пекла - це гудіння літака, яке невдовзі перетворюється на ревище. 

Літак наближається, починає кружляти, і ти майже вмираєш від цього звуку. 

У літаках  замість чортів - роzіяни. 

Волохатою лапою з копитом вони тиснуть на гашетку. 

А потім від потужного удару земля прогинається так, наче влучили зовсім
близько. 

Стіна брижиться, здається, що будинок рухається: робить крок то в один, то в
інший бік, балансує, аби не скластися віялом. 

Щойно набираєш повітря, щоб закричати, літак іде на друге коло.

І ти німієш. 

З горла вилітає придушений писк.

Він збігається з новим ударом та розгойдуванням будинку.

Наш підвал був схожий на храм.

У ньому постійно молилися.

У ньому горіли свічки. 

Тут щохвилини згадували про Бога. 

Говорили те, що думали.

Не соромилися, нічого не вдавали. 

Затуляли рота долонями, коли скрикували, бо не хотіли лякати дітей.

Діти мали очі дорослих і стали напрочуд слухняними.

Маленькі відважні маріупольці.

Раніше для них смерті не існувало. 

Вони просто не знали, що це таке.

Потім зрозуміли, що мама чи тато можуть не повернутися, якщо вийшли з підвалу бодай на хвилинку. 

Як не поверталися батьки їхніх друзів по війні. 

Якось моя семирічна племінниця запитала:

\enquote{Якщо ми помремо - ми все одно залишимося разом?}

\enquote{Звісно, разом. Тільки разом}

І в цьому підвалі.
