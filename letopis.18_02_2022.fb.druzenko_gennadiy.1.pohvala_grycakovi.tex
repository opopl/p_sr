% vim: keymap=russian-jcukenwin
%%beginhead 
 
%%file 18_02_2022.fb.druzenko_gennadiy.1.pohvala_grycakovi
%%parent 18_02_2022
 
%%url https://www.facebook.com/gennadiy.druzenko/posts/10158768471668412
 
%%author_id druzenko_gennadiy
%%date 
 
%%tags grycak_jaroslav,istoria,nacia,ukraina
%%title ПОХВАЛА ГРИЦАКОВІ
 
%%endhead 
 
\subsection{ПОХВАЛА ГРИЦАКОВІ}
\label{sec:18_02_2022.fb.druzenko_gennadiy.1.pohvala_grycakovi}
 
\Purl{https://www.facebook.com/gennadiy.druzenko/posts/10158768471668412}
\ifcmt
 author_begin
   author_id druzenko_gennadiy
 author_end
\fi

ПОХВАЛА ГРИЦАКОВІ.

В цей тривожний час я знаходив втіху, поринаючи в українську історію.
Намагаючись зрозуміти її уроки. І не повторити помилок, яких наші предки
припустились на цій землі.

\ii{18_02_2022.fb.druzenko_gennadiy.1.pohvala_grycakovi.pic.1}

Для мене читання українського наративу, як його бачить Ярослав Грицак у своїй
\enquote{глобальній історії України} дуже близьке. Дещо я для себе відкрив. Дещо
пригадав. Щодо інших подій та інтрепретацій звірив погляди та навів різкість.

Головний висновок для мене: аби подолати минуле, його потрібно прийняти. В нас
справді трагічна історія. Наш край – це справді \enquote{криваві землі}, за влучним
висловом Тімоті Снайдера. Але це не дає нам права покласти відповідальність за
наші невдачі та ріки крові, які текли цією землею, ВИКЛЮЧНО на наших ворогів.
Їх ми навряд чи змінимо. А от засвоїти уроки наших поразок та змінити себе –
можемо спробувати.

Тим, хто прагне використовувати історію як зброю, як пропаганду, читати книжку
не раджу. Реакція буде аля Вятрович. А от тим, хто вірить, що покаяння
відкриває шлях у майбутнє, вона – must read. Врешті-решт покаяння грецькою
μετάνοια – дослівно \enquote{переосмислення}. Чого українці наразі нагально
потребують.  Бо тут я абсолютно погоджуюсь з автором: зашморг минулого не
пускає нас у майбутнє. Бо Україну не можна зробити великою знову (make great
again, як прагнув зробити з Америкою Трамп і прагне зробити з Росією Путін) –
її можна лише зробити великою у майбутньому...

І останнє. Кілька цитати з цієї чудової праці:

\enquote{По-суті йдеться про парадигмальну зміну: перестати дивитись на націю як на
самоціль і сприймати її як платформу для модернізації}.

\enquote{Ризикну припустити, що шанси на український успіх залежать від того, чи зможе
ця третя Україна оформитись в політичний проект і сама прийти до влади. Певною
мірою перемогу Володимира Зеленського та його партії \enquote{Слуга народу} на виборах
2019 року можна вважати кроком до реалізації цього проекту. Однак,
спостерігаючи за діями цієї партії та її лідера, не можна позбутися враження,
що це радше фальстарт, а справжній старт ще попереду}.

І остання. Як на мене, пророча: \enquote{Для побудови нової української нації, окрім
героїв, готових віддати своє життя заради ідеалів, потрібні герої, які би
демонстрували просту людську порядність та жертвували життям заради інших}.
