% vim: keymap=russian-jcukenwin
%%beginhead 
 
%%file 13_10_2021.fb.medvedeva_olesja.1.igra_v_kalmara_ukraina.cmt
%%parent 13_10_2021.fb.medvedeva_olesja.1.igra_v_kalmara_ukraina
 
%%url 
 
%%author_id 
%%date 
 
%%tags 
%%title 
 
%%endhead 
\subsubsection{Коментарі}

\begin{itemize} % {
\iusr{Сергей Юрьевич}

Да уж... выскажу сугубо личное мнение….честно говоря уже надоело смотреть и
слушать как всё плохо в Украине как государстве. От передачи к передачи, от
канала к каналу, от одного обозревателя до другого политолога…одно и то же.
Надоело! Раньше было как то интересно, была надежда, что вся эта критика как то
подействует и на тех на кого она направлена, и на тех, кто её молча
поддерживает. Но увы, с каждым днём, каждым месяцем и годом, эта «правда»
только усугубляла обстановку в Украине, и создалось стойкое ощущение, что
власть специально всё делает наоборот и всеми силами усугубляет всё то, что
подвергается критике. Надоело слушать про оппозицию, которая как бы есть, но
каждая за себя. Надоело слушать про воров и коррупционеров, которых никак не
убавляется, а наоборот, растут и крепнут. Неужели так трудно понять, что жить
«по украински», а именно «вовремя предать значит предвидеть», вовремя
«переобуться», украсть с убытка, «обещать, не значит жениться», «все равны, но
есть равнее», «мой друг – это святое…и на «святом» месте»…и т.д., есть понятия,
которые возведены в государственный ранг, стала нормой самой сущности
государства, и никакая критика, тем более с низов, никогда их непоколебит.
Пройдёт два с половиной года, и очередной «оппозиционер», говорящий правильные
слова, и не дай Бог умеющий играть на баяне и петь матерные частушки (яко бы
близок с народом), примет булаву, положит ладошку на «конституцию»…и уже через
пару месяце положит и на конституцию, и на законы, и на народ, и на государство
в целом, «инструмент» более интимный. Государство, двадцати миллионное
государство, о котором нельзя привести хоть один малюсенький пример чего то
удачного, не есть государство, а является лишь территорией. Я уж не говорю об
успехах в какой то ветви экономики, не об открытии комбината или завода, о
прорыве в науке…но хотя бы об постройке и открытии нового детского сада или
школы, о проекте который принёс хоть какую то прибыль…да хотя бы о том, как
школьник перевёл бабусю через дорогу не попросив при этом финансовой
благодарности. Нет ребята и девчата, не изменить вам Украину, ибо за 30 лет она
так вросла в весь этот негатив, что только рубить… Но увы, у вас всё через
жопу, …одни щепки и остаются.

\iusr{Олег Сюмко}
А в чем фишка?

\begin{itemize} % {
\iusr{Олеся Медведева}
\textbf{Олег Сюмко} надо смотреть сериал. Долго объяснять

\iusr{Олег Сюмко}
\textbf{Олеся Медведева} , черт, дико не хочется((

\iusr{Катя Седых}
\textbf{Олег Сюмко} лизать надо!

\iusr{Виктория Коваль}
\textbf{Катя Седых}  @igg{fbicon.laugh.rolling.floor}{repeat=3} 
С такой фигурой даже если лизать, всё равно приговор "выбыл"

\iusr{Алексей Пикаев}
\textbf{Олег Сюмко} 

Ну там игра была по выковыриванию фигурки из печеного сахара. А по поводу "не
хочется" - ничего не потеряете. Осилил 2 серии. Сериал с тупой фабулой,
высосанными из пальца второстепенными линиями и трогательными диалогами для
домохозяек. Смотрят его в основном из-за эффекта толпы - типа как же сто
миллионов посмотрело, а я еще нет  @igg{fbicon.smile} 


\iusr{Олег Сюмко}
\textbf{Алексей Пикаев} , спасибо что подсказали, а в чем смысл этого процесса?

\emph{Алексей Пикаев}
\textbf{Олег Сюмко} 

Печеный сахар очень хрупкий, а если фигурка треснет в процессе выковыривания,
то игроку пиздец, пуля в затылок не отходя от кассы, не как с Бабченкой или
Шефиром )

\iusr{Олег Сюмко}
\textbf{Алексей Пикаев} , не ну Бабченко то бессмертный он бы мог кучу фигурок переломать
\end{itemize} % }

\iusr{Даниил Мирошниченко}

И смешно, и плакать хочется. До чего довели страну... Вместо всяких красивых
лиц, нам нужны достойные политики в следующем созыве Рады. Те, которые будут
делать то, что обещали. Как в Нашем Крае, к примеру.

\iusr{Awb Andrey}

Мне больше понравился "демократический процесс голосования" в сериале. Вот это
точно было про нас.

\iusr{Владимир Шах}
Что я пропустил? Ничего не понятно

\iusr{Игорь Сидоренко}
Не совсем так.
В "Кальмаре" были правила и не было привелегий.
Если искать киношную аналогию, то это скорее "Безумный Макс"... @igg{fbicon.face.confused} 

\iusr{Руслан Лобанов}
Не люблю больше сериалы), из последних Дикий Ангел и Элен и ребята)

\iusr{Дмитрий Кокорев}
На Украине ...Игра в рапана...)))

\iusr{Виктор Сретенский}

\ifcmt
  ig https://scontent-frx5-1.xx.fbcdn.net/v/t1.6435-9/244744023_603238564041663_1257425168473877281_n.jpg?_nc_cat=105&ccb=1-5&_nc_sid=dbeb18&_nc_ohc=ap6lJJTMYNsAX-lclJo&_nc_ht=scontent-frx5-1.xx&oh=538f2632fa7f1a208e26b5528a2f0257&oe=618CAB0E
  @width 0.4
\fi

\iusr{Henry Soltana}
Капитализм самое отвратительное устройство на планете Земля, которое уничтожает Украину.



\end{itemize} % }
