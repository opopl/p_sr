% vim: keymap=russian-jcukenwin
%%beginhead 
 
%%file slova.most
%%parent slova
 
%%url 
 
%%author 
%%author_id 
%%author_url 
 
%%tags 
%%title 
 
%%endhead 
\chapter{Мост}
\label{sec:slova.most}

%%%cit
%%%cit_pic
%%%cit_text
Жизнь князя Владимира, крестившего Русь в 988 году, была бурной, неспокойной и
неоднозначной. И в мир оной он тоже отошел отнюдь не с улыбкой смирения на
устах. До последнего момента в нем кипела энергия. Тело уже не слушалась, но
мозг продолжал свою кипучую деятельность.  Летопись сохранила для нас
предсмертную фразу, произнесенную князем. Он не молился, не каялся, он отдавал
команду: \enquote{Требите путь и \emph{мостите мост}.  Куда собирался
отправляться Владимир? Для чего ему нужно было расчищать дорогу и строить
\emph{мосты}? Все просто. Он ни много ни мало собирался в очередной раз отправиться в
военный поход
%%%cit_comment
%%%cit_title
\citTitle{Мостите мост. Что означали последние слова князя Владимира?}, 
Русичи, zen.yandex.ru, 09.06.2021
%%%endcit

%%%cit
%%%cit_head
%%%cit_pic
%%%cit_text
Очень прикольно с «крымским \emph{мостом}», вместо которого надо говорить
«незаконно построенный \emph{мостовой} переход между временно оккупированной АР
Крым и РФ через Керченский пролив». Что-то подсказывает, что говорить все равно
будут «крымский \emph{мост}».  Вместо «особого статуса Донбасса» — «особый
порядок местного самоуправления ОРДЛО», что, очевидно, навеяно недавними
юродивыми высказываниями одного отечественного неадекватного деятеля о том, что
никакого Донбасса, дескать, вообще не существует в природе, и это название
якобы выдумала кремлевская пропаганда
%%%cit_comment
%%%cit_title
\citTitle{Краткий словарь грантоедов под редакцией СНБО / Лента соцсетей / Страна}, 
Александр Карпец, strana.news, 26.10.2021
%%%endcit
