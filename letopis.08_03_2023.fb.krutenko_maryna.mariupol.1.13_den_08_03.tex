%%beginhead 
 
%%file 08_03_2023.fb.krutenko_maryna.mariupol.1.13_den_08_03
%%parent 08_03_2023
 
%%url https://www.facebook.com/marinakrytenko/posts/pfbid0jCy7BnA77WStTaVgzSEdMgVhyRtoWYWVHDYmccPb8EssRUCw7JSn2rXFTWZh7mC7l
 
%%author_id krutenko_maryna.mariupol
%%date 08_03_2023
 
%%tags 08.03.2022,dnevnik,8_marta,mariupol,mariupol.war
%%title ТРИНАДЦАТЫЙ ДЕНЬ ВОЙНЫ 08.03.22
 
%%endhead 

\subsection{ТРИНАДЦАТЫЙ ДЕНЬ ВОЙНЫ 08.03.22}
\label{sec:08_03_2023.fb.krutenko_maryna.mariupol.1.13_den_08_03}

\Purl{https://www.facebook.com/marinakrytenko/posts/pfbid0jCy7BnA77WStTaVgzSEdMgVhyRtoWYWVHDYmccPb8EssRUCw7JSn2rXFTWZh7mC7l}
\ifcmt
 author_begin
   author_id krutenko_maryna.mariupol
 author_end
\fi

ТРИНАДЦАТЫЙ ДЕНЬ ВОЙНЫ 08.03.22

Утро как обычно.... Под фоновый звук обстрелов мы позавтракали и \enquote{приняли душ}. 

За ночь выпал снег!!! Ура, мы не должны стоять под обстрелами за водой.... Уже по
отработанной схеме, ведра под сливную трубу и вода в доме))))

Пошли на разведку за едой. Оле дали два килограмма не чищеных кальмаров. 

Пока она их помыла, почистила и начала варить, перестал поступать газ.
Доваривать их пришлось на костре во дворе. 

На улице Женя встретил людей с левого берега Мариуполя которым чудом получилось
доехать в центр города. Они попросили одеяло и топор, чтобы рубать деревья и
готовить еду на костре. Женя прибежал домой и отдал то, что им нужно было. 

С этого дня наша и без того тяжелая жизнь, превратилась в \enquote{каменный век}
Нарубать дрова, развести костёр, сварить еду под открытым небом, а иногда и под
обстрелом....

8 Марти на постсоветском пространстве считается Международным женским днём и
мужчины обычно дарят цветы, конфеты и какие-то подарочки. Нам мужья подарили по
апельсинки, коробку конфет. МАЛЬЧИКИ НАГРЕЛИ НА КОСТРЕ БОЛЬШУЮ КАСТРЮЛЮ ВОДЫ 10
литров И МЫ ЗА ВСЕ ЭТО ВРЕМЯ ПОМЫЛИ ГОЛОВУ И БОЛЕЕ МЕНЕЕ ПОКУПАЛИСЬ В ТАЗИКЕ.
С 02.03 мы толком не мылись. В квартире было холодно, мы не могли посушить
волосы, просто на голову одели шапки. Термобелье, лыжные носки и шапка стали
повседневной одеждой. От шапки болели уши, потому-то она вообще не снималась.
Во-первых было холодно, во-вторых, под шапкой прятались грязные волосы.

Мальчиков за это мы отблагодарили вкусным ужином при свечах (романтические
ужины при свечах у нас были каждый вечер)))) света же не было) На ужин у нас
была уха из красной рыбы и салат с кальмаров. 

Легли спать в холодную постель. Дома было + 10 +15 С

Продолжение следует.....

%\ii{08_03_2023.fb.krutenko_maryna.mariupol.1.13_den_08_03.cmt}
