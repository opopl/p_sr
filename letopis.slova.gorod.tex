% vim: keymap=russian-jcukenwin
%%beginhead 
 
%%file slova.gorod
%%parent slova
 
%%url 
 
%%author 
%%author_id 
%%author_url 
 
%%tags 
%%title 
 
%%endhead 
\chapter{Город}
\label{sec:slova.gorod}

%%%cit
%%%cit_pic
%%%cit_text
\emph{Город} стоит на болотах. Он постоянно подтоплялся, и раз за разом настилалась
новая бревенчатая мостовая. До 28 слоев потом археологи насчитали! Нижние
подмокшие слои, лишенные, как и во всяком болоте, доступа кислорода,
превращались в идеальные капсулы, надежно сохраняющие органику. А береста, как
вы понимаете, это и есть органика.  Вспомните, сколько историй вам в школе
рассказывали о доисторических животных, попавших в трясину. Именно они лучше
всего сохранились. Для палеонтолога болота - это настоящее благословение.  Так
что в Новгороде найдено больше грамот не потому, что это был единственный оазис
просвещения на Руси, а потому что там просто-напросто были наиболее подходящие
условия для их сохранности. Московская береста, а также береста других русских
\emph{Городов} по большей части просто истлела
%%%cit_comment
%%%cit_title
\citTitle{Почему в Москве нашли всего четыре берестяные грамоты, а в Новгороде - свыше тысячи?}, 
Русичи, zen.yandex.ru, 18.05.2021
%%%endcit

%%%cit
%%%cit_head
%%%cit_pic
%%%cit_text
Тут можно долго беседовать со своим бессознательным, выясняя причины подобного
сюрреализма. Но даже не будучи фугасом, превратившимся в траву, за эти два года
я тоже пришел к простому выводу: Я - Человек, рожденный жить. И \emph{Город}, в
котором я живу, не создан для войны. Не создан подсчитывать \enquote{полученные
повреждения в ходе артобстрела}. Не создан считать \enquote{200}, \enquote{300} и \enquote{без вести
пропавших}. Не создан быть полигоном для утилизации боеприпасов и
тестостероновых \enquote{меряний} мужчинами, кто из них лучше умеет убивать. Этот \emph{Город}
рожден жить, любить, вдохновлять, производить и желать доброй ночи
%%%cit_comment
%%%cit_title
\citTitle{Gorlovka.ua: Город, рожденный жить - Блоги}, 
Егор Воронов, gorlovka.ua, 29.07.2016
%%%endcit

%%%cit
%%%cit_head
%%%cit_pic
%%%cit_text
Как часто вы разговариваете с \emph{Городом}, в котором живете? Звоните ли ему
время от времени по стационарному телефону? Справляетесь ли о его самочувствии?
Вижу, что вы смеетесь над моими вопросами.  Чудненько. Вы, наверняка, уверены,
что \emph{Город} - безмолвен и не умеет говорить. На самом же деле он никогда
не молчит.  И слова его - это поступки и чувства живущих в нем людей.
\emph{Город} декламирует сонеты и штампует некрологи, чеканит заметки и
сплетает романы, упражняется в рассказах и плюется афоризмами. Детективы,
любовная \enquote{бульварщина}, мистический реализм, сатира, эпосы, военная
лирика...  Все это он проговаривает своими \enquote{-ами} (французы бы
сказали \enquote{mon ami}). Вот так и Горловка. Хочешь услышать ее - наблюдай
за горловчанами, каждый день составляя тезаурус печалей и радостей этого
\emph{Города Вечной Осени}
%%%cit_comment
%%%cit_title
\citTitle{Gorlovka.ua: Говорить с Городом - Блоги}, 
Егор Воронов, gorlovka.ua, 06.11.2016
%%%endcit

%%%cit
%%%cit_head
%%%cit_pic
%%%cit_text
Впрочем, люди помогают расчищать не только свои дворы, но и общие улицы. В Ялте
жители организовываются через соцсети, приезжают волонтеры из Симферополя и
других мест. Также в \emph{город} прислали военных. Он практически закрыт на
въезд и поэтому на трассе, которая ведет в Ялту огромные пробки.  В южной
столице Крыма проблемы не только с мусором и затоплениями. Потоки местных рек,
потерявших берега, сносят мосты и заливают \emph{город}. Военные отправляют на
место их истока в горах беспилотники, чтобы выяснить перспективы бедствия, а
спасатели тем временем возводят временные дамбы
%%%cit_comment
%%%cit_title
\citTitle{Ялта - наводнение в Крыму. Что происходит 21 июня и как ликвидируют катастрофу}, 
Максим Минин, strana.ua, 21.06.2021
%%%endcit

%%%cit
%%%cit_head
%%%cit_pic
\ifcmt
  pic https://strana.ua/img/forall/u/0/36/2021-07-05_15h50_26.png
	width 0.4
\fi
%%%cit_text
\enquote{А кто это решил вообще?}.  Инфраструктура донецкого Нью-Йорка не сравнится с
его американским тезкой - повсеместная разруха, свалки, нуждающиеся в
капремонте дороги, советские трехэтажки, жилые дома из шлакоблока и с мозаикой
1917 года на стенах, школы, которые отапливаются дровами и углем, поросший
сорняком местный парк.  \emph{Город}, мягко говоря, не соответствует названию
Нью-Йорк, считают жители. Он затухает, наблюдается замирание жизни, а не ее
развитие.  \enquote{Раньше здесь было больше различных предприятий. В настоящий
момент осталось одно предприятие - вот так \enquote{развивается} поселок. Чтобы
соответствовать тому названию, которое сейчас есть, он должен быть лучше, чем в
то время, когда был Новгородским. Нужно много вкладывать средств и труда и
всего остального}, - отвечает нам мужчина на вопрос о переименовании родного
поселка
%%%cit_comment
%%%cit_title
\citTitle{Появление Нью-Йорка на Донбассе - как к этому относятся местные жители. Опрос Страны}, 
Антонина Белоглазова, strana.ua, 06.07.2021
%%%endcit

%%%cit
%%%cit_head
%%%cit_pic
%%%cit_text
Всюду было множество мух. Их выгоняли из комнат полотенцами, закрывали
предварительно ставни. Не удивительно — по дворам стояли мусорные ящики,
содержимое которых вывозилось по мере их переполнения.  По \emph{городу} ходила своего
рода знаменитость — София Александровна. Это была психически больная женщина,
бездомная, грязная, со старой горжеткой на шее, в шляпе с перьями и большими
полями. Одежда ее — грязные лохмотья. Ее всегда сопровождала такая же грязная
собака», — вспоминала одна из старейших жительниц города, врач Мария Плахотина.
И только сатирик Дон-Аминадо (местный уроженец Аминад Пейсахович Шполянский)
язвил: «Держался \emph{город} на трех китах. Вокзал. Тюрьма. Женская гимназия»
%%%cit_comment
%%%cit_title
\citTitle{Игорь Тамм. Нобелевский лауреат из Елизаветграда}, Дмитрий Губин, ukraina.ru, 08.07.2021
%%%endcit

%%%cit
%%%cit_head
%%%cit_pic
%%%cit_text
Ювелирная мастерская талантов.
История же, однако, показала, что не женская гимназия прославила \emph{город}, а
мужская, Александровская. И в ней не учились ни Гриша Радомысльский (он же тов.
Зиновьев, уровень знаний и происхождение мешали поступить), ни Серёжа Костриков
(тов. Киров, он же мальчик из Уржума, жил совсем в других местах), ни Марко
Кропивницкий (родился и рос до её открытия, а учился в соседнем Бобринце), чьи
имена носил этот \emph{город} в разные годы. Кстати, ныне это учебное заведение носит
имя Тараса Шевченко, который тоже не был гимназистом, как и все три «тотема»
этого \emph{города}
%%%cit_comment
%%%cit_title
\citTitle{Игорь Тамм. Нобелевский лауреат из Елизаветграда}, Дмитрий Губин, ukraina.ru, 08.07.2021
%%%endcit

