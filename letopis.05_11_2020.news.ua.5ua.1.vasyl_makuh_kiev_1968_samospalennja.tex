% vim: keymap=russian-jcukenwin
%%beginhead 
 
%%file 05_11_2020.news.ua.5ua.1.vasyl_makuh_kiev_1968_samospalennja
%%parent 05_11_2020
 
%%url 
%%author 
%%tags 
%%title 
 
%%endhead 

\subsection{Всюди культ \enquote{великоросійського преобладанія}: гнівний лист до ЦК КПУ --- чому його автор спалив себе на Хрещатику}
\label{sec:05_11_2020.news.ua.5ua.1.vasyl_makuh_kiev_1968_samospalennja}
\Purl{https://www.5.ua/suspilstvo/vsiudy-kult-velykorosiiskoho-preobladaniia-hnivnyi-lyst-do-tsk-kpu-chomu-ioho-avtor-spalyv-sebe-na-khreshchatyku-228361.html}

05.11.2020 08:00 \href{https://www.5.ua/spetstemy/154/}{Видатні українці}

\index[names.rus]{Радіо Свобода}
\index[names.rus]{Макух, Василь!Самоспалення 1968}

\ifcmt
pic https://www.5.ua/media/pictures/1140x641/199903.jpg
caption Фрагмент листа Василя Макуха Радіо Свобода
\fi

\lettrine[lines=3]{5}{листопада 1968 року} поряд із будинком на вулиці Хрещатик, 27-а в Києві
з'явився охоплений полум'ям чоловік. Він кричав: \enquote{Геть комуністичних
колонізаторів!} і \enquote{Хай живе вільна Україна!}

Пробігши кілька метрів у бік нинішнього Майдану Незалежності, Василь Макух
упав. Свідки і міліція спробували загасити вогонь, але не змогли зробити це
швидко. Наступного дня він помер у лікарні від опіків 70\% тіла.

У радянських газетах про його вчинок не було жодного повідомлення. Факт
самоспалення прагнули приховати, але за кордоном і в самвидаві про
Макуха писали. Друзі згадували про його плани вийти на несанкціоновану
демонстрацію на знак протесту проти окупації Чехословаччини, знайомі –
про його ув'язнення в Сибіру і участь в УПА.

\ifcmt
pic https://www.5.ua/media/pictures/820x546/199909.jpg
caption Хрещатик, 27а - будинок, біля якого Василь Макух вчинив акт самоспалення 
\fi

Досі про самоспалення Макуха відомо мало. Кажуть, що перед тим, як запалити
себе, він розкидав листівки, що в його кишені була записка, яку забрав КДБ, що
напередодні він написав листа в ЦК КПУ. Спочатку в архівах виявили тільки листи
в компартію. Дослідник Віктор Тупілко багато років вивчав життя Василя Макуха і
створив у Донецьку (нині знищений російськими окупантами) музей \enquote{Смолоскип}.
Каже: серед речей, переданих родині Макуха, жодних записок не було. Немає їх і
в архівній справі.

\subsubsection{\enquote{Пийте ще й мою кров...}}

Та, як виявилося, не всюди шукали.

{\bfseries
Десять рукописних аркушів адресовані особисто першому секретареві ЦК КПУ
Петру Шелесту. Лист із підписом \enquote{Макух В. З Дніпропетровська} без дати.
На штемпелі одного з київських поштових відділень вказано --- 5 листопада
1968 р. Це той фатальний день, коли Василь Макух, обливши себе бензином
у під'їзді одного зі столичних будинків, вибіг на Хрещатик...
}

\ifcmt
pic https://www.5.ua/media/pictures/820x546/199930.jpg
caption Василь Макух - відкриті джерела
\fi


\enquote{...КПУ, як та собачка, виконує добросовісно всі примхи московського
господаря, а Ви, шановний секретар ЦК КПУ, говорите, що її утворення у
складі єдиної РКП(б) знаменувало собою новий етап у будівництві нашої…
Так, це був новий етап ворожих підступів після Переяслава, за що
український народ поплатився мільйонами своїх найкращих людей...}.

\enquote{
...А чи не пора з укр. шкільних підручників вимести це сміття
русифікації? Всюди культ \enquote{великоросійського преобладанія}!..
}.

\enquote{...Чому, як тільки хтось з передових людей скаже слово національної
правди, не лизне лакейські п'ятки Кремля, відразу він 
\enquote{націоналіст український}? 
І як тільки не зневажають гідність української людини.
Звичайно, ворог хоче очорнити наших славних предків і сьогоднішніх
передових людей, вирвати з нашого народу…}.

\enquote{...Чому нас називають молодшими братами тоді, коли Київська держава, а
ми, українці, --- її спадкоємці, існувала на два століття раніше, ніж
Московська?.. Ви самі пишете, що кожен п'ятий воїн Червоної армії на
фронті --- це були українці. А славу пожинають росіяни? І так весь час вони
обкрадають нас --- в науці, в літературі, в мистецтві…}.

\enquote{...Протестуючи протів кривд, заподіяних російськими окупантами --- пийте ще
й мою кров (я спалю себе в жертву не Вам, а нашому народові, щоб молоде
покоління сміло, відважно продовжувало святу справу боротьби)...}.

\ifcmt
pic https://www.5.ua/media/pictures/1140x641/199908.jpg
pic https://www.5.ua/media/pictures/1140x641/199907.jpg
pic https://www.5.ua/media/pictures/1140x641/199906.jpg
\fi

У звітах КДБ мовилося про листа Макуха до ЦК КПУ, однак текст його ще
геть недавно був невідомим. Допомогли декомунізація і розсекречення
архівів радянських спецслужб. Конверт --- із зображенням санаторію
"Кришталевий палац", що в Трускавці. Тобто Макух писав листа на своїй
батьківщині, на Львівщині. Лист довгий, і написати його за вечір було
неможливо.

У листі Макух не акцентує увагу на введенні радянських військ до
Чехословаччини, проти якого протестував. Водночас багато говорить про
проблему русифікації в Україні, згадує про зросійщення української
культури й освіти, утиски й переслідування українців.

\ifcmt
pic https://www.5.ua/media/pictures/820x546/199910.jpg
\fi


Лист свідчить, що самоспалення було вчинене не під впливом моменту --- це був
спланований, зважений і усвідомлений акт самопожертви.

То був не перший лист до служителів комунізму. Попри те, що за ним стежили, він
вів підпільну боротьбу, зустрічався з побратимами в різних містах. Жив у
Дніпропетровську --- місті русифікованому, а повернутися в західну Україну,
звідки був родом, не мав права. Тому він із листами постійно писав у компартію,
говорив, що йде тотальна русифікація України, що Україна --- невільна держава, і
проти цього боролися він і його побратими, розповідав племінник Василя Макуха
Юрій.

\subsubsection{Уся сім'я брала участь у визвольній боротьбі}

\enquote{Його боротьба не давала бажаних результатів, багато побратимів
відмовлялися від подальшого опору: у когось були проблеми зі здоров'ям,
хтось --- через страх за свою сім'ю. Він із цим змиритися не міг, але знайти
вихід із ситуації теж не зміг. Говорили, що готувалася акція протесту, але
чи то хтось повідомив про неї, то чи побоялися і не захотіли виходити --- та
він не хотів змінювати свого рішення. Розповів про свої плани близькому
другові, Григорію Ментуху. Ментух намагався його відговорити, але не
вийшло. Він йому сказав, що не бачить інших можливостей для боротьби. Про
плани знав і духівник --- Макух висповідався перед тим, як піти на такий
крок. Той його відмовляв, про що написав у своїх спогадах. Писав, що
розмірковував, як цьому запобігти, але зрозумів, що якби розповів комусь,
це означало би здати його в руки радянському правосуддю}, --- пригадує Юрій
Макух.

Він пригадує, що після трагічної загибелі Василя Макуха на Майдані його
маму два дні тримали в приймальні місцевого міліціонера, а тітка Параска –
сестра Василя Макуха --- незабаром після тортур на допитах померла:

{\bfseries
"
Уся наша сім'я брала участь у визвольній боротьбі, тому ми завжди були
готові до такого розвитку подій. Хоча фактично це було для нас
несподіванкою. Ми про все дізналися від співробітників КДБ, які почали
викликати на бесіди всіх членів нашої сім'ї... Вони сподівалися, що нам
відома якась інформація про те, чим Василь Макух займався, з ким
спілкувався. Але ніхто нічого не знав. Йому було відомо, що за ним
постійно стежать, і він нікого не хотів підставляти. Ці допити були
безглуздими, хоча співробітники КДБ його сестру Параску позбавили життя,
її катували. Моя мама, коли повернулася з допитів, кілька днів плакала,
а потім сказала: \enquote{Це --- трагедія нашої сім'ї, і ми самі повинні впоратися
з цим завданням. Про це нікому не треба розповідати}.
"
}

\ifcmt
pic https://www.5.ua/media/pictures/820x546/199905.jpg
caption Василь Макух із дружиною і донькою, Радіо Свобода
\fi

Раніше, того ж таки 1968 р., Василя Макуха намагався завербувати КДБ. Він
постійно був під контролем. І дуже гостро переживав вторгнення в
Чехословаччину. Це стало додатковим аргументом у боротьбі, почасти
спонукало його на акт самоспалення.

Після трагедії дисидент Євген Пронюк написав статтю \enquote{Пам'яті героя}: її
разом зі світлиною Макуха поширили в самвидаві, оформили як листівку, щоб
про самоспалення дізналося якнайбільше українців. Інший активіст, Степан
Бедрило, за поширення інформації про Макуха був засуджений на 2 роки.
Протестувальника Богдана Чабана за поширення листівок теж спіткала
в'язниця. Згаданий вище Пронюк 1972 р. був звинувачений в антирадянській
пропаганді і засуджений на 7 років тюрем суворого режиму.

Пізніше КДБ почав активно дискредитувати як Василя Макуха, так і його
вчинок. Шукали підтвердження того, що він був психічно неврівноваженим,
але в це ніхто не повірив. Сучасники кажуть: він не вживав алкоголь, був
освіченим, хоча здобути освіту у ВНЗ йому не дали: коли в КДБ дізналися,
що Макух став студентом, його виключили. Формальною причиною стало буцімто
приховування судимості. Довести, що акт самоспалення був несвідомим,
спецслужбам теж не вдалося.

\subsubsection{Ким насправді був Василь Макух}

Народився 1927 р. на Львівщині у селянській родині. Юнаком підтримував
зв'язок із членами \enquote{Просвіти} та ОУН, а під час Другої світової воював у
лавах УПА.

{\bfseries
В архівах КДБ є дві справи на Василя Макуха: перша --- про встановлення
контролю над ним, а друга --- кримінальна, 1946 р., коли він був поранений
у перестрілці з НКВС --- йому тоді прострелили ногу (він кульгав усе
життя), також ключицю, а одна куля зачепила хребет. Макух тоді втратив
свідомість --- і його схопили. Катували --- це видно за матеріалами справи.
Але він нікого не видав. 18-річний хлопець виніс усе --- і моральні
тортури, коли йому погрожували розстрілом, і фізичні. У справі є фраза
слідчого: \enquote{Ну, як це так, ви нічого не знаєте --- ні командира, взагалі
нічого не знаєте! Такого просто не може бути!} 
Відповідь: \enquote{Не знаю}.
}

\ifcmt
pic https://www.5.ua/media/pictures/820x546/185133.jpg
caption ГУЛАГ, Фото з архіву
\fi

Василь Макух у 18-річному віці отримав 10 років таборів з обмеженням прав
на 5 років за участь в УПА, куди вступив 1944 р. У Дубравлазі, у Мордовії,
він познайомився зі своєю майбутньою дружиною Лідією, яка відбувала
покарання, бо ще дитиною її вивезли з окупованого Донецька до Німеччини.
Кажуть, що Макух довго не наважувався пов'язувати з нею життя: він завжди
говорив, що в будь-який момент може померти за Україну. Проте весілля
відбулося, у Макухів народилися дочка Ольга та син Володимир. Сім'я жила в
Дніпропетровську.

\textbf{Уся родина Макухів воювала в УПА.}

\enquote{Ми жили недалеко від кордону, там були болота, і батько (брат Василя
Макуха --- ред.) переправляв розвідників і звичайних людей у Польщу. Багато
втікали через Польщу, щоб не опинитися в Сибіру. Його не заарештували лише
тому, що він сильно захворів. Він одного разу при переході кордону
натрапив на НКВС і прикордонників. Справа була восени, довелося ховатися в
болоті, в воді --- і він захворів. Після цього переводити людей через кордон
почала сестра моєї мами, вона була маленькою дівчинкою. Її вистежили і
відправили в Пермський край на лісозаготівлі, хоча вона була
неповнолітньою}, --- розповідає племінник дисидента Юрій Макух.

2017 року в Києві на будинку, поруч із яким Василь Макух вчинив
самоспалення, встановили меморіальну табличку, а цього року в Празі його
іменем назвали пішохідний міст через річку Ботич.

2015 р. міська влада Дніпра в рамках декомунізації перейменувала вулицю
Краснопартизанську на честь Василя Макуха.

За матеріалами \enquote{Радіо Свобода}

Читайте також: Що спільного між гуфницями запорожців і гаубицями ЗСУ --- з
чого стріляли козаки-артилеристи

\paragraph{Читайте за темою}

\begin{itemize}
\item Батуринська різанина: як російське військо за 48 годин знищило 20-тисячне
місто

\item 19-й кілометр карельської дороги: віднайдені імена українців, страчених в
урочищі Сандармох

\item У Кім Чен Ина істерика: ТОП-5 емоційних відео \enquote{5 каналу}, якими запам'ятався
жовтень
\end{itemize}

