% vim: keymap=russian-jcukenwin
%%beginhead 
 
%%file 12_04_2022.fb.kazanskij_denis.1.rosarmia_doma_ukraincev.cmt
%%parent 12_04_2022.fb.kazanskij_denis.1.rosarmia_doma_ukraincev
 
%%url 
 
%%author_id 
%%date 
 
%%tags 
%%title 
 
%%endhead 
\zzSecCmt

\begin{itemize} % {
\iusr{Исмаилбек Беккужин}

Даже в советских фильмах помню, на оккупированной территории фашисты жили в
чистеньких избах.

\begin{itemize} % {
\iusr{Iryna Bezugla}
\textbf{Исмаилбек Беккужин} ви пам'ятаєте що в них були столові прибори, тарілки для першого і другого.А це свині

\iusr{Лена Ковальова}
\textbf{Ismailbek Bekkuzhin} совєцькіх фільмах!!!!!!!!  @igg{fbicon.face.tears.of.joy}{repeat=8} 

\iusr{Вероника Резниченко}
\textbf{Исмаилбек Беккужин} 

так они и жили в домах, все чисто и культурно. Как пришли так и ушли, но были и
исключения, но бабушка ни разу нк рассказывала, что бы кто - то нагадил в
доме. Может не знала, не было такого случая или пожалела нашу детскую психику.

\iusr{Наталья Матвеева}
\textbf{Ismailbek Bekkuzhin} 

Це так. Я пам'ятаю наша сусiдка розповiдала як в неi жив нiмець офiцер з
денщиком. Коли вiн до неi заселився то привiз столовий посуд не тiльки для себе
але i для неi та ii дiтей. Це не поодинокi випадки. Хоча сам фашизм вiдбiлити
неможливо.

\iusr{Alla Krasilna}

Так моя прабабуся, мама розповідала, коли до неї постоялець німець приніс свої
штани, що б вона виросла, то бабуся схватила їх і до коменданта- не буду я
засрані штани стирати. Комендант пристроїв всіх солдат і на плацу бив, тими
штаньми солдата, що б інші не сміли своє споднє заставляти стирати.

А мама розповідала, що той що у них жив, пригощав їх шоколадками і розповідав,
що в нього вдома \enquote{цвай кіндер}, і війна то погано.

І бабуся казала, що вони купували і молоко і яйця, але селяни не хотіли брати
їхніх грошей, то вони міняли на шоколадки та сгущьонку.

Але були різні німці, на сусідній вулиці палили будинки, бо там знайшли
партизанів, йв центрі села переїхали танками активістів. Війна то страхіття,
нічого доброго.

Бажаю нам всім якнайшвидшої Перемоги, та мирного неба!!!

\iusr{Владимир Ронес}
\url{https://fb.watch/cn-m9nWt8l}

\iusr{Oksana Lysa}
\textbf{Ismailbek Bekkuzhin} німецькі фашисти - то ще були культурні люди в порівнянні з кацапами.

\iusr{Elena Shapoval}
\textbf{Исмаилбек Беккужин} бабуся розповідала про період окупації меньше страхіть ніж про період колективізації

\iusr{Nina Bernak}
\textbf{Наталья Матвеева} 

Так фашизм( ідеологію) не можна відбілити і не треба. Нащадки повинні знати.
Але і ви, і інші коментатори говорять про німців, як націю з їх культурою і
вихованням. І тут ви праві- з культурою і вихованням рашистів не порівняєш. Та
і порівнювати нема чого, бо в них нуль культури і виховання.

\iusr{Людмила Радченко}
\textbf{Alla Krasilna} 

в моїй сім'ї німці вбили дев'ятьох. Не знаю, які вони були, але розстріляли
після катувань діда і його 15-річного сина. В Ржищеві, де Дніпро і кручі.
Прямо в кручу і скидали..

\iusr{Виктория Крот}
\textbf{Исмаилбек Беккужин} сравнили.... немцы культурные цивилизованные люди, а эти - быдло

\iusr{Маша Микитенко}
\textbf{Исмаилбек Беккужин} ну тож фільм, а це правда війни, росеяни вони свині, люблять сміття і гімно, щоб спати було на чому

\iusr{Alla Krasilna}
\textbf{Людмила Радченко}, 

так були різні команди, по різному себе вели, співчуваю вашій родині, у мене
теж один дідусь загинув, а другий був в плєну, тричі тікав, дуже били, але з
другом зуміли втекти до друга в село в Вінницькій обл. Перепливали пізньою
осінню річку, ноги зводили, але втекли. Дідусь ніколи нічого не розповідав про
війну, казав хвала Господу що всі живі, а там був жах.

Війна нічого доброго нікому не приносить.

Бажаю нам мирного неба та Перемоги, молюсь за наших Захисників.

\iusr{Ксения Полякова}
Вероника Резниченко. Исмаилбек Беккужин. 

А вот другая бабушка рассказывала, как на глазах у ее дочери фашисты, жившие у
них в чистеньких избах, привязали ребенка за ноги и за руки к двум танкам и
поехали в разные стороны, разорвав его на глазах у матери, которая тут же
лишилась рассудка. А сосед наш рассказывал, как на его глазах, он был
маленьким, фашист, который жил у них в чистенькой избе, выстрелил в горло его
матери и убил ее, конечно. . Просто так. Но избы были чистенькими и немцы тоже.
Как вам всем тут не стыдно приводить в пример фашистов.!

\iusr{Лариса Штырь}

Німці несли смерть, але були і вийнятки! мій дідусь був родом із хмельниччини,
то в них на постої був німець врач, їх було троє дітей, німець приносив їм їжу і
врятував сестру і брата від дефтирії, а потім ще півсела дітей, він теж говорив
,що не хотів воювати, але двоє його дітей..., в разі відмови могли постраждати.
Та зовсім інше було, коли до села ввійшли сс.... Навіть на війні, як це дико не
звучить, можно лишатися людиною!

\iusr{Зера Менасанова}
\textbf{Вероника Резниченко} 

А моя бабушка рассказывала что когда у них жили немцы, Кто-то мочился прям у
входа. Так вот немцы вычислили и прогнали ссыкуна и он долго не мог найти
пристанище!

\iusr{Lilia Claudel}
\textbf{Ismailbek Bekkuzhin} 

так то же немцы. У них в самых маленьких деревнях цивилизация дошла, в отличие
от огромной по территории России, где цивилизация, да и то сомнительная, только
в Москве. Правда говорят что, половина её уже утонула...  @igg{fbicon.face.screaming.in.fear}  @igg{fbicon.laugh.rolling.floor} 

\iusr{Alla Krasilna}
\textbf{Nina Bernak} , 

бабуся розповідала, що гвалтувань при німцях не було, а коли зайшли
асвабадітєлі то декілька жінок згвалтували, бо вони ж кляті в окупації жили. І
чим далі на захід тим більше гвалтувань, а що вони в Германії творили, їх не
могли зупинити.

\iusr{Nina Bernak}
\textbf{Alla Krasilna} історія повторюється

\iusr{Жанна Васильчук}
Среди всех наций есть свиньи, Но в РФ по моему это большинство!!!

\iusr{Jane Shpott}
\textbf{Людмила Радченко} 

Да так. Знущались але. Були ті що не робили такого але це. Зовсім без
@igg{fbicon.brain}. І вони нав'язують що ми брати то хіба. Ці кати які
ґвалтують от дітей до стареньких до смерті та розрізали вагітних це станісти
які від крові Кайф овали. Так. Ми балакаємо на одній мові. То хто це? До речі.
Німці. Коли напали?  Та. Ця нація Зараз всі роки відчувала біль та сплачувала.
Рашистам але. В цій війні. Украінців більше загинуло. Але  @igg{fbicon.droplet}
@igg{fbicon.glaz} зараз 21 вік. А прийшли недоробки вбити Цивілізовану. Націю
@igg{fbicon.glaz}  @igg{fbicon.droplet}  @igg{fbicon.drop.blood} 

\iusr{Игорь Кальной}
\textbf{Исмаилбек Беккужин} 

и сами убирали и заставляли, если засранцы. Хотя у засранцев они не
квартировались. Детвору брили на лысо, подкармливали. Это про обычную пехоту!
СС лютовало!. Невзовора послушайте как наши Германию брали. Кровь стынет.. в
любой войне должна быть мотивация. Многое зависит от командира да и просто
Человека. Если ты мразь....то это навсегда. А война вскроет любую язву. Кто ты,
как ты...

\iusr{Игорь Кальной}
\textbf{Alla Krasilna} правда(((( бабця те саме казала.
Чорнопогонники були страшні, сс...Колотилівка, зараз Белгородская обл. Була раніше Українською.


\end{itemize} % }

\iusr{Tarnawska Jana}
Ubogie twari

\iusr{Наталя Нагорна}

Саме так і в Ірпіні, і в інших містах і селах Київщини та інших областей
України. Чого чого, а лайна в \enquote{товаришів} окупантів - тонни, просто
тужитися-не-перетужитися

\begin{itemize} % {
\iusr{Маша Микитенко}
\textbf{Наталя Нагорна} ну росеяни люблять гімно і сміття, чистота це не їхня риса

\iusr{Наталя Нагорна}
\textbf{Маша Микитенко} Це факт! Вже весь світ це зрозумів

\iusr{Марія Івкіна}
рос.армії АНАЛоГОВНЕт
\end{itemize} % }

\iusr{Таня Святая}

Дикуни! Просто огидно що таке лазило в людських будинках! Слова \enquote{рос.солдат}-
викликають нудоту і відразу, уявляю в якому багні вони живуть і як від них
тхне. Бидломаса!

\begin{itemize} % {
\iusr{Гриша Гаврилюк}
\textbf{Таня Святая} сто п'ятдесят км від Москви срут у відра і впливають гімно на вулицю. В глибині росії теж саме . Дикуни

\iusr{Таня Святая}
\textbf{Гриша Гаврилюк} та там і за 30 км лише в закритих селищах є цивілізація.

\iusr{Valentyna Valentyna}
Аби виздихали до ноги, дикуни.

\iusr{Светлана Суддя}
\textbf{Таня Святая} и дети у них больные рождаются из-за муссорника, в котором они живут.

\iusr{Наталя Даценко}
\textbf{Гриша Гаврилюк} 30 км від Москви топлять дровами, газу нема.

\iusr{Людмила Горбачевська}

Яка бридота лазила по хатах! Нищота задріпана гидотна, та після них ці хати
невідмити! Свині і то чистіше живуть ніж ці бидляки! Щоб вони виздихали всі в
один день до останьього скота!

\iusr{Таня Святая}
\textbf{Людмила Горбачевська} 

от я теж про це думаю, що після них меблі вже не придатні будуть, бридкі. А в
домах кілька дезінфекцій треба провести, бо в них мабуть купа зарази.

\iusr{Наталія Швець}
\textbf{Таня Святая} ЇХ ЩЕ ІВАН ГРОЗНИЙ НАЗИВАВ СМЕРДАМИ

\iusr{Sofiya Kanas}
\textbf{Таня Святая} Вони такі й були. Під час ІІСВ німок, угорок гвалтували. Це пишуть в спогадах і їхні ветерани.

\iusr{Svitlana Stoliarchuk}
\textbf{Таня Святая}, 

бачила на власні очі в Псковській та Новгородській губерніях в 1991р. Як казала
одна директорша магазина з Пскова - \enquote{будет, что вспомнить}.

\iusr{Пётр Дацюк}
\textbf{Таня Святая} 

Культура людини приходить з малку, з материнським молоком. Яка духовність
людини, такі і вчинки...

\end{itemize} % }

\iusr{Наташа Старовойт}

Они привыкли к гавну. Сами из него состоят.

\begin{itemize} % {
\iusr{Наталья Матвеева}
\textbf{Наташа Старовойт} Точно. Тому залишають одне лайно пiсля себе.

\iusr{Света Мараховская}
\textbf{Наташа Старовойт} розтрощити, що не можуть вкрасти, насрати, це в них вже в крові, бо століття змінюються, а поведінка раши- ні.
\end{itemize} % }

\iusr{Oleg Mykhailov}
Спеціальна особлива хромосома...

\begin{itemize} % {
\iusr{Тетяна Бережна}
\textbf{Oleg Mykhailov} хромосома - ГІМНОсома

\iusr{Людмила Бульба}
\textbf{Oleg Mykhailov} виявляється, лишня хромосома у кацапів своя особиста..., в тісних стосунках з лайном... мразота..

\iusr{Літо Літо}
\textbf{Oleg Mykhailov} дійсно, у більшості з них ЗАЙВІ ХРОМОСОМИ є
\end{itemize} % }

\iusr{Svetlana Oppitz}

Boни навіть не розуміють, що цим свинством, позорять свою країну та марнують
солдатську честь. Офіцери дозволяють це все. Вони не здатні нікого визволяти
,лише грабувати. У мене таке враження, що це воюють злочинці, визволені з
в'язниць. Війна не виправдовує таку никчемну поведінку.

\iusr{Gina Vinu}

Я часто мечтаю о том, чтобы мой дом в Донецке разбомбили бы до тла!!!!! Это
морально легче, чем знать, что Кто-то там хозяйничает сейчас, выбрасывает на
помойку дорогие мне вещи, хранящие воспоминания!!!!!

\iusr{Волкова Элена}
Из чего состоит человек такой по себе и оставляет след. Это в крови, с этим они родились.

\iusr{Светлана Загребельная}
Культура так и прёт..

\iusr{Бондаренко Виктория}
А если б ещё месяц прибыли в этом доме, то пришлось бы откапывать

\iusr{Леся Носуль}

Ничего страшного, после победы уберем, отремонтируем, перестроим, перекрасим,
МЫ СИЛЬНЫЕ, СПРАВИМСЯ!!!!! А вот это г...но вернётся к себе в расею и будет
продолжать так срать у себя регулярно, вот это я считаю конец нации......
ФУУУУ!

\begin{itemize} % {
\iusr{Alexander Golubka}
\textbf{Леся Носуль} Якщо не добити, то вони повернуться через 50 років. @igg{fbicon.face.frowning}{repeat=3}

\iusr{Olga Pimonenko}
\textbf{Леся Носуль} 

так вони так і живуть. Замість унитазу -відро. Як повне - вікно відкрив і вилив.
Все... Жалоби пишуть до путіна, щоб ім шо-небудь зробили, допомогли з лайна
вилізти. А самі тільки бухають та ниють. Забула! Ще ЗАЗДРЯТЬ! Дуууже!

\iusr{Людмила Тренбач}
\textbf{Alexander Golubka} Ой, мабуть, раніше. Що їм поміщає повертатись кожні 5-10 років?

\iusr{Василий Лудин}
\textbf{Леся Носуль} конец это когда их не будет со всем, это орда всегда так жила.

\iusr{Сергей Савчук}
\textbf{Леся Носуль} Золотые слова
\end{itemize} % }

\iusr{Ірина Єгорова}

Згадався давній шок подібного змісту про солдатів і матросів ВОСР (не збіг,
факт: \enquote{вєлікая октябрьская соціалістическая рєвалюція, а необхадимасті каторой
всьо врємя гаварили бальшивікі} - майже цитую вєлікава лєніна), котрі з
особливою пристрастю робили той масовий ВОСР в імператорський посуд, вази,
люстри... Світ змінюється. Раша - ні.

\iusr{Людмила Русин}

За лампу.. в моем доме в Буче аналогично - ещё и проткнули её ножом

\ifcmt
  ig https://scontent-mxp1-1.xx.fbcdn.net/v/t39.30808-6/278422021_10159747807444509_3709118462846300806_n.jpg?_nc_cat=100&ccb=1-5&_nc_sid=dbeb18&_nc_ohc=WO6M7_Cd4xgAX8_FDux&_nc_oc=AQnwwkYTzGNWhi7n_2Vkc_snGA1aYWgu45_uXNCYGc52TZeNpzkT8qzbevp-jJj74FI&_nc_ht=scontent-mxp1-1.xx&oh=00_AT9C9VDvd-DcsZqL5uWbY1wXZI6E3qbpwfB6iEw3wLAqsw&oe=627033B9
  @width 0.3
\fi

\begin{itemize} % {
\iusr{Алена Елена}
\textbf{Людмила Русин} наверное они читали финского писателя...

\iusr{Лариса Степаненко}
\textbf{Людмила Русин} Людочка, як я вам співчуваю. Дай вам сили, щоб все це вимити і далі добре жити.
\end{itemize} % }

\iusr{Оля Сирота}
Хто чим багатий, той тим і ділиться

\iusr{Anna Zhugylina}

Читала Н. Никулина \enquote{Записки о войне}, так там красноармейцы тоже испражнялись в
немецкий фарфоровый сервиз, это национальная игра-забава!? срать там, где жрешь

\begin{itemize} % {
\iusr{Anatol Barchishak}
\textbf{Anna Zhugylina} 

Да, читав Н. Н. Нікуліна, просто вражає страшна правда про війну, а ще більше
вражають русаки які критикують, та звинувачують автора, ніби це не правда.

\iusr{Andrii Pollik}
\textbf{Anna Zhugylina} хороша книга)

\iusr{Олеся Кундухова}
\textbf{Anatol Barchishak} то німецькі фейкі та \enquote{вывсёврёти}.

\iusr{Inna Hriaznykh}
\textbf{Anna Zhugylina} как были быдлом, так и остались

\iusr{Alexandra Khandogina}
\textbf{Анатоль Иосифов} \enquote{вифсьоврьоті}!  @igg{fbicon.beaming.face.smiling.eyes} 

\iusr{Татьяна Васылева}
\textbf{Anna Zhugylina}, жрать из того, во что срут...

\iusr{Алексей Матвиенко}
\textbf{Anna Zhugylina} 

Николай Никулин, «Воспоминания о войне»:

Открыв дверь, я обнаружил гвардии ефрейтора Кукушкина, отправляющего надобность
в севрское блюдо. Салфетки рядом были изгажены...

-- Что ж ты делаешь, сволочь, -- заорал я.

-- А что? -- кротко сказал Кукушкин.

Он был небольшого роста, круглый, улыбчивый и очень добрый. Со всеми у него
были хорошие отношения. Всем он был симпатичен. Звали его обычно не Кукушкин, а
ласково, Кукиш. И вдруг такое! Для меня это было посягательством на Высшие
Ценности. Для меня это было покушением на идею Доброго, Прекрасного! Я был в
бешенстве, а Кукушкин в недоумении. Он натянул галифе и спокойно отправился
досыпать. Я же оставшуюся часть ночи лихорадочно думал, что же предпринять. И
надумал -- однако ничего более идиотского я выдумать не мог.

Утром, когда все проснулись, я велел команде построиться. Видимо, было на лице
моем что-то, удивившее всех. Обычно я никогда не практиковал официальных
построений, поверок и т. п., которые предписывал армейский устав. Шла война, и
мы чихали на всю подобную дребедень. А тут вдруг -- "Рав-няйсь! Смирррна!"...
Все подчиняются, хотя в строю есть многие званием выше меня. Я приказываю
Кукушкину выйти вперед и произношу пламенную речь.

Кажется, я никогда в жизни не был так красноречив и не говорил так вдохновенно.
Я взывал к совести, говорил о Прекрасном, о Человеке, о Высших Ценностях. Голос
мой звенел и переливался выразительнейшими модуляциями. И что же?

Я вдруг заметил, что весь строй улыбается до ушей и ласково на меня смотрит.
Закончил я выражением презрения и порицания гвардии ефрейтору Кукушкину и
распустил всех. Я сделал все, что мог. Через два часа весь севрский сервиз и
вообще вся посуда были загажены. Умудрились нагадить даже в книжные шкафы. С
тех пор я больше не борюсь ни за Справедливость, ни за Высшие Ценности.

\iusr{Anatol Barchishak}
\textbf{Alexandra Khandogina} точно, методички незмінні.

\iusr{Александр Зиновой}
В шухляду письмового столу і вставити в стіл....

\ifcmt
  ig https://scontent-mxp1-1.xx.fbcdn.net/v/t39.30808-6/277816349_2768003400161556_198320573301680048_n.jpg?_nc_cat=100&ccb=1-5&_nc_sid=dbeb18&_nc_ohc=rgMxoGW7_sYAX9oCYUc&_nc_ht=scontent-mxp1-1.xx&oh=00_AT8VSS_lFPnTs1CeHYUERBMTVsjs82Ei0Oo45wl2Ja32Tg&oe=62703D6D
  @width 0.3
\fi

\iusr{Anna Zhugylina}
\textbf{Татьяна Васылева} так тоже можно  @igg{fbicon.face.crying.loudly}{repeat=2} 

\iusr{Irina Ponedelnik}
\textbf{Anna Zhugylina} Это национальная суть, увы.

\iusr{Ирина Шевченко}
\textbf{Anna Zhugylina} Уроды ((((Генетический мусор ((((

\iusr{Олег Петренко}

Хто не читав, ознайомтесь. Це коротеньке оповідання відображає всю сутність
росіян:

Помню, лет пять тому назад мне пришлось с писателями Буниным и Федоровым
приехать на один день на Иматру. Назад мы возвращались поздно ночью. Около
одиннадцати часов поезд остановился на станции Антреа, и мы вышли закусить.

Длинный стол был уставлен горячими кушаньями и холодными закусками. Тут была
свежая лососина, жареная форель, холодный ростбиф, какая-то дичь, маленькие,
очень вкусные биточки и тому подобное. Все это было необычайно чисто, аппетитно
и нарядно. И тут же по краям стола возвышались горками маленькие тарелки,
лежали грудами ножи и вилки и стояли корзиночки с хлебом.

Каждый подходил, выбирал, что ему нравилось, закусывал, сколько ему хотелось,
затем подходил к буфету и по собственной доброй воле платил за ужин ровно одну
марку (тридцать семь копеек). Никакого надзора, никакого недоверия.

Наши русские сердца, так глубоко привыкшие к паспорту, участку, принудительному
попечению старшего дворника, ко всеобщему мошенничеству и подозрительности,
были совершенно подавлены этой широкой взаимной верой.

Но когда мы возвратились в вагон, то нас ждала прелестная картина в истинно
русском жанре. Дело в том, что с нами ехали два подрядчика по каменным работам.

Всем известен этот тип кулака из Мещовского уезда Калужской губернии: широкая,
лоснящаяся, скуластая красная морда, рыжие волосы, вьющиеся из-под картуза,
реденькая бороденка, плутоватый взгляд, набожность на пятиалтынный, горячий
патриотизм и презрение ко всему нерусскому — словом, хорошо знакомое истинно
русское лицо. Надо было послушать, как они издевались над бедными финнами.

— Вот дурачье так дурачье. Ведь этакие болваны, черт их знает! Да ведь я, ежели
подсчитать, на три рубля на семь гривен съел у них, у подлецов... Эх, сволочь!
Мало их бьют, сукиных сынов! Одно слово — чухонцы.

А другой подхватил, давясь от смеха:

— А я... нарочно стакан кокнул, а потом взял в рыбину и плюнул.

— Так их и надо, сволочей! Распустили анафем! Их надо во как держать!

А. И. Куприн

Гастрономический финал

\iusr{Олег Петренко}
\textbf{Анатоль Иосифов} 

Був час, я жив і працював в Бурятії. Так от, народ там поважає тільки грубу
силу (не подумайте, я не бойовик якийсь). Найменший признак слабості - все, ти
там не виживеш (для інформації, якщо ти не був на зоні, то ти недолюдина).

\iusr{Anatol Barchishak}
\textbf{Олег Петренко} мені це теж знайомо, жив та працював колись в Магаданській обл.

\iusr{Anna Zhugylina}
\textbf{Олег Петренко} напомнило \enquote{Тагил рулит}

\iusr{Anna Zhugylina}
\textbf{Алексей Матвиенко} да,да,оно

\iusr{Ilona Erst}
\textbf{Anna Zhugylina} это скрепы!

\iusr{Ирина Ирка}
\textbf{Anna Zhugylina} 

помоему у Никулина (не помню точно чьи записки) еще есть и про то, как рашкин
герой насилует женщин и девочек. Это тоже традиция. Ничено не изменилось у этих
варваров.

\iusr{Алёна Александровна}
\textbf{Anna Zhugylina} 

знакомые рассказывали о кучах говна на кроватях...

думала, что немного преувеличивают.

Но, по-ходу, нет

\ifcmt
  ig https://scontent-mxp1-1.xx.fbcdn.net/v/t39.30808-6/278264421_5113474545412228_7643149225118697104_n.jpg?_nc_cat=104&ccb=1-5&_nc_sid=dbeb18&_nc_ohc=Cnou6f3jpnEAX_G4htj&_nc_ht=scontent-mxp1-1.xx&oh=00_AT9UVXt2V-jHNHgBLrMfWWoOvUxNFlxeKOaHbDbpoGoYEA&oe=6270950A
  @width 0.3
\fi

\iusr{Angel Ray}
\textbf{Anna Zhugylina} та нет, это привычка такая у них. Где ем, там и сру, называется((((

\iusr{Екатерина Чернецкая}
\textbf{Anna Zhugylina} это не игра-забава, это от зависти, сами такого никогда не имели и не видели.

\iusr{Anna Zhugylina}
\textbf{Алёна Александровна} уже не удивляюсь ничему  @igg{fbicon.face.crying.loudly} 

\iusr{Захар Кругляк}
\textbf{Anna Zhugylina} Это бедность фантазии.

\iusr{Tatjana Tananykina}
\textbf{Anna Zhugylina} нет не игра, это сущность россиянина, его натура.

\iusr{Валентина Ковганич-Грищук}
\textbf{Екатерина Чернецкая} І ніколи не будуть мати.

\iusr{Igor Vasilyev}
\textbf{Anna Zhugylina} \enquote{воспоминания}

\iusr{Зера Менасанова}
\textbf{Anna Zhugylina} Срать - в вазы, жрать - из горшков!

\iusr{Айдар Чукмаитов}
\textbf{Anna Zhugylina} 

у Никулина не только в фарфор испражняшись, но кажется ещё и бельгийскими
кружевами и занавесками жопы вытирали.

\iusr{Ольга Прокопченко}

В всём этом есть что-то звериное... Так животные метят свой ареал обитания. Но
люди... Не все, видно эволюционируют. Или это результат отрицательной
селлекции?..

\iusr{Anna Zhugylina}
\textbf{Ольга Прокопченко} 

я не знаю селекция или деградация, мне не понятно ЗАЧЕМ им чужие трусы и
унитазы, КАК можно насиловать детей, что произошло в их головах

\iusr{Ольга Прокопченко}
\textbf{Anna Zhugylina} 

Может быть Стругацкие были правы на счёт \enquote{лучей из башни}... Мрак в
головы сеют российские телебашни. Из чего состоит их вечерний эфир? Шоу
\enquote{разглядываем чужое грязное белье} (сегодня людей по-проще, а затра!
\enquote{звездное} бельишко подвезут) Затем псевдонародные песни ряженых
скоморохов с окраин. Потом высокомерная \enquote{аналитика} событий, вывернутые
наизнанку \enquote{новости}. А высокоинтеллектуальные политические шоу
экспертов... Где они берут этих болванов, которые визжат - мы всё знаем лучше
всех и все кроме нас дураки -\enquote{Это же и так все знают, но бояться
признать}..?. Это какой-то кошмар. А \enquote{Поле чудес} - угадай 3 буквы -
получи мясорубку... КАК это можно смотреть 30 лет??? Бесконечные сериалы, о том
как живут \enquote{среднестатистические} состоятельные люди... только в
телевизоре и в кино. Они уже и согласны с тем, что хорошо может быть только в
телевизоре. А когда видят, что может быть по другому - я не представляю, что у
них в голове происходит? Кто виноват?... Все, кроме россиян. Кто вкладывает это
им в голову?.. Ответ знаете сами.

\iusr{Anna Zhugylina}
\textbf{Ольга Прокопченко}  @igg{fbicon.thumb.up.yellow} 


\end{itemize} % }

\end{itemize} % }
