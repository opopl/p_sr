% vim: keymap=russian-jcukenwin
%%beginhead 
 
%%file 18_07_2020.fb.gorodnenko_yuri
%%parent 18_07_2020
 
%%endhead 
  
\subsection{Собственное наблюдение: если человек кричит, что он "патриот", то непременно от него следует ждать какого-то идиотизма.}
\url{https://www.facebook.com/gorodnenko.yuri/posts/4131334230274947}

\vspace{0.5cm}
{\small\LaTeX~section: \verb|18_07_2020.fb.gorodnenko_yuri| project: \verb|letopis| rootid: \verb|p_saintrussia|}
\vspace{0.5cm}

В Луганске какая-то "Патриотическая ассоциация Донбасса" добилась
переименования местного пединститута, носившего имя Тараса Шевченко. Ну
переименовали и переименовали. Но нет, Шевченко объявили "украинским нацистом",
а его имя решили заменить именем Екатерины II. В центре Луганска планируют
установить ей памятник. Немецкую принцессу Ангальт-Цербстскую объявили
"основательницей" Луганска.  Историческая справка: Луганск появился 3 сентября
1882 года. В этот день император АЛЕКСАНДР III утвердил положение Комитета
министров Российской империи об основании на базе посёлка Луганский завод с
присоединением селения Каменный Брод города Луганск. Деятельность Александра
III до сих пор мало изучена и недооценена историками.

Поселок Луганский завод появился в 1797 году при Павле I. Каменный Брод
известен как минимум с начала XVIII века.

Шевченко был славянофилом и последовательно выступал за объединение всех славян
в рамках одного государства - России. Он НИКОГДА не был сторонником отделения
Малороссии от Великороссии и точно не был нацистом.  Сколько помню в советское
время в Луганске (тогда был Ворошиловград) это хорошо знали. Так откуда взялись
эти "патриоты" и какое право они имеют переписывать историю города?
  
