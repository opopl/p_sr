% vim: keymap=russian-jcukenwin
%%beginhead 
 
%%file slova.tragedia
%%parent slova
 
%%url 
 
%%author 
%%author_id 
%%author_url 
 
%%tags 
%%title 
 
%%endhead 
\chapter{Трагедия}

%%%cit
%%%cit_head
%%%cit_pic
%%%cit_text
30 июня 1971 года произошла крупнейшая \emph{трагедия} в истории советской
космонавтики. При возвращении из полета в полном составе погиб экипаж
космического корабля «Союз-11»: Георгий Добровольский, Владислав Волков и
Виктор Пацаев. Они приняли чужую смерть, поскольку в тот раз не должны были
лететь. Почему именно они были отправлены в полет? Что стало причиной гибели —
ошибка экипажа или отказ техники? «Лента.ру» вспоминает ход тех событий
%%%cit_comment
%%%cit_title
\citTitle{«Они были обречены» 50 лет назад погиб экипаж «Союза-11». Кто виноват
в главной трагедии советской космонавтики?: Космос: Наука и техника: Lenta.ru},
Сергей Варшавчик, lenta.ru, 30.06.2021
%%%endcit
