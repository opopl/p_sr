% vim: keymap=russian-jcukenwin
%%beginhead 
 
%%file 15_10_2020.fb.buzina.1.zvezdochka_moja_jasnaja
%%parent 15_10_2020
%%url https://www.facebook.com/groups/buzinagroup/permalink/546336786181436/?__cft__[0]=AZUFdgT-QZtwXjTRS4ZTWSneluen2FuwTyYNNmyWV4aZCY1LV87TnZJg2A9eviYWFeBTE4zfLsuHPmdNWSv5lpH8HakXArsq_NelPPQdn34t4ie-I2lOIoocAaYXfWZR5LeEpyED6gJBY-Y3xJk4zE4pQFokU4TZ2iBjAD35ZjfcNIiprvwomdCS94EViVcUWOIojR6SB-1BxW3cepxNW2rw&__tn__=%2CO%2CP-R
 
%%endhead 

\subsection{«ЗВЕЗДОЧКА МОЯ ЯСНАЯ»... }
\label{sec:15_10_2020.fb.buzina.1.zvezdochka_moja_jasnaja}
\url{https://www.facebook.com/groups/buzinagroup/permalink/546336786181436/?__cft__[0]=AZUFdgT-QZtwXjTRS4ZTWSneluen2FuwTyYNNmyWV4aZCY1LV87TnZJg2A9eviYWFeBTE4zfLsuHPmdNWSv5lpH8HakXArsq_NelPPQdn34t4ie-I2lOIoocAaYXfWZR5LeEpyED6gJBY-Y3xJk4zE4pQFokU4TZ2iBjAD35ZjfcNIiprvwomdCS94EViVcUWOIojR6SB-1BxW3cepxNW2rw&__tn__=%2CO%2CP-R}

Наверное, все помнят песню «Звездочка моя ясная», но мало кто знает, что эта
популярная песня посвящена юной 19-летней девушке, убитой террористами всего за
3 месяца до ее свадьбы…

15 октября 1970 года, взлетев из батумского аэропорта, самолет АН-24 (рейс 244)
с 46 пассажирами на борту должен был приземлиться в Краснодаре.

Через несколько минут после взлета, на высоте 800 метров, двое пассажиров —
отец и сын Бразинскасы вызвали бортпроводницу Надежду Курченко и передали
записку для пилотов с требованием изменить маршрут и лететь в Турцию. Она
бросилась в кабину и закричала: ‘Нападение!’

Преступники кинулись за ней и, в попытке прорваться в кабину пилотов, начали
стрелять. Позже в обшивке насчитают 18 пробоин.

Несколько пуль были выпущены в сторону салона; никто из пассажиров не
пострадал. Первому пилоту Георгию Чахракия пуля попала в позвоночник, и у него
отнялись ноги.  Превозмогая боль, он обернулся и увидел страшную картину: Надя
без движения лежала в дверях пилотской кабины и истекала кровью.

Штурману Валерию Фадееву прострелили легкое, а бортмеханик Оганес Бабаян был
ранен в грудь. Больше всех повезло второму пилоту Сулико Шавидзе --- пуля
застряла в стальной трубе в спинке его сиденья.  Старший Бразинскас достал
гранату и, угрожая взорвать ее, потребовал от пилотов подчиниться и лететь в
сторону Турции…

В октябре 1970-го, СССР потребовал от Турции незамедлительно выдать
преступников, но данное требование выполнено не было. Турки решили сами судить
угонщиков и приговорили 45-летнего Пранаса Бразинскаса к восьми годам тюрьмы, а
его 13-летнего сына Альгирдаса --- к двум.

В 1974 году в этой стране случилась всеобщая амнистия, и, тюремное заключение
Бразинскасу-старшему заменили на… домашний арест на роскошной вилле в Стамбуле,
а оттуда американские спецслужбы их вывезли в США.

Это был первый случай в общемировой практике воздушного терроризма с убийством
члена экипажа, угоном самолета в соседнюю страну и невозвращением
преступников,чему способствовала явная двойная западная мораль.

В 1980 году Пранас заявил в интервью «The Los Angeles Times», что был
активистом движения за освобождение Литвы и бежал за границу, поскольку на
родине ему грозила смертная казнь. Однако он почему-то забыл рассказать, что
сидел на родине не за патриотизм, а получил два срока за воровство и
злоупотребление служебным положением.

В Америке Альгирдас официально стал Альбертом-Виктором Уайтом, а Пранас —
Фрэнком Уайтом. Они поселились в городке Санта-Моника в Калифорнии, где
работали малярами.

Казалось бы, сбылась американская мечта двух ублюдков, однако Фемида их не
оставила безнаказанными. Под старость характер Альгирдаса стал невыносимым и
они с сыном часто ссорились.

Во время одного из таких конфликтов 45-летний сынок насмерть забил своего
77-летнего папашу бейсбольной битой.

В ноябре 2002 года жюри присяжных в суде Санта-Моники признало Альберта
виновным в преднамеренном убийстве и он был приговорён к 16 годам тюрьмы.

P.S. В память о смелой девушке поэтесса Ольга Фокина, написала стихотворение
под названием «Песни у людей разные» о погибшей бортпроводнице от имени её
молодого человека.


Стихотворение Ольги Фокиной попалось на глаза начинающему тогда композитору
Владимиру Семенову. Он и написал в 1971 году песню «Звездочка моя ясная»,
ставшую хитом на века.

А первым и популярнейшим исполнителем ее стала группа Стаса Намина «Цветы»

Песни у людей разные,
А моя одна на века.
Звездочка моя ясная,
Как ты от меня далека.
Поздно мы с тобой поняли,
Что вдвоём вдвойне веселей
Даже проплывать по небу,
А не то, что жить на земле.
Облако тебя трогает,
Хочет от меня закрыть.
Чистая моя, строгая,
Как же я хочу рядом быть.
Поздно мы с тобой поняли,
Что вдвоём вдвойне веселей
Даже проплывать по небу,
А не то, что жить на земле.
Знаю, для тебя я не бог,
Крылья, говорят, не те.
Мне нельзя к тебе на небо
А-а-а прилететь.
Поздно мы с тобой поняли,
Что вдвоём вдвойне веселей
Даже проплывать по небу,
А не то, что жить на земле.
А не то, что жить на земле.
