%%beginhead 
 
%%file 06_04_2023.fb.suhorukova_nadia.mariupol.1.golovne_scho_zhyvyj
%%parent 06_04_2023
 
%%url https://www.facebook.com/permalink.php?story_fbid=pfbid0uQzog9VfKfSMmFNAiWuP6dqWZCw9aEF4r6p94cmuQBRFdT1RUjzDcQt72xahiVEvl&id=100087641497337
 
%%author_id suhorukova_nadia.mariupol
%%date 06_04_2023
 
%%tags mariupol.war,mariupol
%%title У Маріуполі було все одно, який у тебе вигляд
 
%%endhead 

\subsection{У Маріуполі було все одно, який у тебе вигляд}
\label{sec:06_04_2023.fb.suhorukova_nadia.mariupol.1.golovne_scho_zhyvyj}

\Purl{https://www.facebook.com/permalink.php?story_fbid=pfbid0uQzog9VfKfSMmFNAiWuP6dqWZCw9aEF4r6p94cmuQBRFdT1RUjzDcQt72xahiVEvl&id=100087641497337}
\ifcmt
 author_begin
   author_id suhorukova_nadia.mariupol
 author_end
\fi

У Маріуполі було все одно, який у тебе вигляд.

Головне, що живий. 

Прийшов у шапочці до брів, із протертими серветкою очима та зубами, почищеними
трьома краплями води, і вже красень чи красуня.

Нам було байдуже, яка куртка чи штани на тобі. 

Головне, що ти пробіг цілий квартал і тебе не вбило. 

Усі так раділи, коли хтось приходив! 

Усім було пофіг, які на тобі туфлі.

Ти прийшов, і тебе одразу пригостять супом з простроченими пельменями й оцтом.

Юлька, мама Микитки, коли сталося пряме влучання в будинок, вискочила на
березневий мороз у чоловічих гумових капцях на босу ногу.

Вона бігла з тижневою дитиною на руках і тільки у своїй квартирі на сьомому
поверсі помітила, що взула чуже. 

Там, у місті нашого випробування, всі були одягнені в душу.

Її було найвидніше.

Олена зверху на джинси вдягала піжамні штани.

Щоб було тепліше.

Так у них і приїхала до Мангуша, а потім блукала серед розбитих машин і їла
хліб, який ми купили в першому ж  кіоску окупованого селища.

Дві Наталки ходили по дому у світлих куртках.

І спали в них, бо було дуже холодно.

У напівтемряві Наталки скидалися на Білосніжок.

Коли ми виїхали з Маріуполя, то побачили, що куртки сіро-чорні.

Від кіптяви ґнотів, бруду підвалу та попелу вогнищ. 

Всім було байдуже. 

У всіх життя летіло під укіс зі швидкістю поїзда "Інтерсіті". 

Ми трималися разом і не знали, що робити. 

Ми майже нічого із собою не взяли, не було часу. 

Життя могло закінчитися будь-якої миті, тому головне, щоб одяг був зручний і
теплий. 

Усе зручне й тепле ми одягнули на себе.

У мене на животі ще висіла сумка. Я так і спала у підвалі - обіймаючи сумку. 

Якби в будинок влетів снаряд і мене розірвало б, то родичі, які залишилися,
ідентифікували б мене за сумкою.

У ній лежали паспорт, шапка, серцеві пігулки, залишки корму для собаки й кота,
шматок сала, який нашій сім’ї подарували сусіди по підвалу, рукавички, поламана
пудрениця (там лишилося люстерко).

Я іноді дивилася в нього і здригалася: моє обличчя в темряві підвалу здавалося
незрозумілою плямою. 

Я  дивувалася, що досі відображаюся. 

Роздивитися щось було важко, але насправді я лише хотіла переконатися, що жива.

В кишеньці на блискавці в мене була готівка.

Зняла, а потім забула про неї.

Мені здавалося, що гроші тепер уже не потрібні. 

У блокаді  ми ними не користувалися. 

Узагалі. 

Ні води, ні їжі, ні ліків там не продавали. 

Усе можна було тільки обміняти.

 Чи просто поділитися тим, що маєш. 

Ніхто жодного разу не попросив у мене грошей. 

І я ні в кого їх не просила. 

Це були всього-на-всього папірці.

Коли в Мангуші тамтешній мужик зажадав за шматочок бринзи і сиру п’ятсот
гривень, я віддала їх не вагаючись.

Я не знала, може, за час сидіння в підвалах усе змінилося й тепер воно все так
і коштує? 

Та й узагалі, мені видавалося дивом, що справжню їжу міняють на нікому не
потрібні папірці. 

Ми стояли біля машин і їли хліб із бринзою та сиром. Відщипували і їли. 

Я клала сир у брудну долоню й запихала його в рот. 

Мені було начхати, який це має вигляд збоку. 

Я вперше їла із задоволенням, і мені було все одно, яка ціна. 

Я рахувала:  скільки в мене ще  залишилося папірців, щоб   обміняти їх на цю
чудову їжу? 

На фото бульвар Меотиди. Так він виглядав  30 березня 2023 року. 

Тиждень  тому. 

Ці фото мені надіслала моя знайома. Вона їх теж взяла з сайту.

%\ii{06_04_2023.fb.suhorukova_nadia.mariupol.1.golovne_scho_zhyvyj.cmt}
