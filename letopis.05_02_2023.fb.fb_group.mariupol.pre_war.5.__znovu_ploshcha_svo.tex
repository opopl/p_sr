%%beginhead 
 
%%file 05_02_2023.fb.fb_group.mariupol.pre_war.5.__znovu_ploshcha_svo
%%parent 05_02_2023
 
%%url https://www.facebook.com/groups/1233789547361300/posts/1404013470338906
 
%%author_id fb_group.mariupol.pre_war,avramova_olena.mariupol
%%date 05_02_2023
 
%%tags mariupol,mariupol.pre_war,mariupol.ploscha.svobody,tulpan
%%title І знову Площа Свободи, з її дивовижним, найекзотичнішим тюльпановим деревом
 
%%endhead 

\subsection{І знову Площа Свободи, з її дивовижним, найекзотичнішим тюльпановим деревом}
\label{sec:05_02_2023.fb.fb_group.mariupol.pre_war.5.__znovu_ploshcha_svo}
 
\Purl{https://www.facebook.com/groups/1233789547361300/posts/1404013470338906}
\ifcmt
 author_begin
   author_id fb_group.mariupol.pre_war,avramova_olena.mariupol
 author_end
\fi

І знову Площа Свободи, з її дивовижним, найекзотичнішим тюльпановим деревом. 

Де ще в наших краях було таке диво!? У найближчих околицях - ніде! І не тільки
у ближчих...  Неймовірна краса, як малахітова кам'яна квітка зі сказки. Я за
ним спостерігала з моменту відкриття Площи після реконструкції, коли тільки
побачила табличку з назвою \enquote{Тюльпанове дерево}.

Спостерігала і взимку, і восени, і навесні, і в холод, і сніг, і дощі.
Підходила, розмовляла, питала: \enquote{Ну, як ти? Ростєш?}  А воно відповідало
легеньким похитуванням гілок. Дуже хотілося його дочекатися, квапила час, ніби
боялася не встигнути... ніби відчувала лихо... Раділа, коли з'явилися великі
бутони і воно нарешті розцвіло! 

Цікаво, як воно? Вижило? Вціліло після обстрілів? Квітло без мене?
