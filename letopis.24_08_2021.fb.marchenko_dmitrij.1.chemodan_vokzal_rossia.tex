% vim: keymap=russian-jcukenwin
%%beginhead 
 
%%file 24_08_2021.fb.marchenko_dmitrij.1.chemodan_vokzal_rossia
%%parent 24_08_2021
 
%%url https://www.facebook.com/bizantinum/posts/4305339456210481
 
%%author Марченко, Дмитрий
%%author_id marchenko_dmitrij
%%author_url 
 
%%tags chemodan_vokzal_rossia,kiev,oskorblenie
%%title "Не нравится? Вали! Чемодан, вокзал, Россия!"
 
%%endhead 
 
\subsection{\enquote{Не нравится? Вали! Чемодан, вокзал, Россия!}}
\label{sec:24_08_2021.fb.marchenko_dmitrij.1.chemodan_vokzal_rossia}
 
\Purl{https://www.facebook.com/bizantinum/posts/4305339456210481}
\ifcmt
 author_begin
   author_id marchenko_dmitrij
 author_end
\fi

Мне всегда странно было слышать в свой адрес фразы в стиле: "Не нравится? Вали!
Чемодан, вокзал, Россия!". Так уж сложилось, что я родился, вырос и продолжаю
жить на Куреневке (есть такой киевский райончик). И все мои ближайшие предки,
жили здесь же (примерно с конца 1850-х гг.). Прабабушка похоронена на
Куреневском же кладбище (а теперь и мама). За это время наша семья пережила
всё: революцию, Директорию, немцев, гетьмана Скоропадского, Петлюру,
большевиков, снова немцев... А теперь вот это вот всё. Не без потерь для семьи,
конечно. Но пережили. И что-то мне подсказывает, переживем и всех тех, кто
указывает нам на вокзал.
