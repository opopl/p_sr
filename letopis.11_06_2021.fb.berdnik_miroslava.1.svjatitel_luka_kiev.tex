% vim: keymap=russian-jcukenwin
%%beginhead 
 
%%file 11_06_2021.fb.berdnik_miroslava.1.svjatitel_luka_kiev
%%parent 11_06_2021
 
%%url https://www.facebook.com/permalink.php?story_fbid=1223599894758925&id=100013267772922
 
%%author 
%%author_id berdnik_miroslava
%%author_url 
 
%%tags 11_jun,cerkov,kiev,pravoslavie,rusmir,svjatitel_luka.vojno_jaseneckij,vera
%%title Сегодня день памяти святителя Луки (Войно-Ясенецкого), архиепископа Симферопольского и Крымского
 
%%endhead 
 
\subsection{Сегодня день памяти святителя Луки (Войно-Ясенецкого), архиепископа Симферопольского и Крымского}
\label{sec:11_06_2021.fb.berdnik_miroslava.1.svjatitel_luka_kiev}
\Purl{https://www.facebook.com/permalink.php?story_fbid=1223599894758925&id=100013267772922}
\ifcmt
 author_begin
   author_id berdnik_miroslava
 author_end
\fi

Сегодня день памяти святителя Луки (Войно-Ясенецкого), архиепископа Симферопольского и Крымского.
Русский Киев может гордиться, что его представителем был будущий святитель Лука Войно-Ясенецкий.
Семья Войно-Ясенецких жила в Киеве с 1889 года. И здесь он закончил гимназию и Киевскую художественную школу.
Именно в Киеве он почувствовал, что Господь призвал его делателем на ниву Свою.
Он вспоминал: "В это время впервые проявилась моя религиозность. Я каждый день, а иногда и дважы в день ездил в Киево-Печерскую лавру, часто бывал в киевских храмах и, возвращаясь оттуда, делал зарисовки молящихся людей, лаврских богомольцев, приходивших туда за тысячи верст".
Батюшка О. Владимир Тютенко пишет:   "Для нас всех будет интересным такой пример, что наш святой Лука, профессор, доктор медицины, Войно-Ясенецкий, именно за противодействие обновленческому расколу в 1923 году, был осужден и отправлен в первую ссылку.
Как метод борьбы с Церковью, Советская власть учинила в России обновленческий раскол, а в Украине -автокефальный. После этого, Патриарх Тихон оказался в заключении, епископов и священников изгоняли с их мест, расстреливая и сажая в тюрмы, а на их места ставили обновленцев и автокефалистов. Поэтому, о. Валентин Войно-Ясенецкий принимает монашество и епископский сан, чтобы своим авторитетом защитить Ташкентскую паству от раскольников, так как действующий епископ убежал из-за угроз.
Прослужил ровно неделю, за это время союз студентов обратился к руководству Университета уволить профессора Войно-Ясенецкого за его веру, а местная газета опубликовала статью "Воровской епископ Лука", где с упоминанием церковных канонов указывалось на неправильность Его рукоположения. Т.е., несмотря на ученый и медицинский авторитет, несмотря на Его участие в создании Университета, что служило авторитету Советской власти в этом крае, Его, только за активную церковную позицию и ПРОТИВОСТОЯНИЕ ЦЕРКОВНОМУ РАСКОЛУ, учиненному советской властью, за неделю посадили в тюрьму.
Для меня, абсолютная тождественность настоящего и прошлого, как по действию власти, так и по защите Церкви от учиненного раскола, который губителен для народа, так как власть создает свою "Церковь" только с целью контроля умов человеческих, а не для спасения душ!
А вот Его, святителя Луки, Завещание в 20-е годы:
"16 июня 1923 года Лука написал завещание, в котором призывал мирян оставаться верными Патриарху Тихону, противостоять церковным движениям, выступающим за сотрудничество с большевиками (оно было передано на волю через верующих сотрудников тюрьмы):
«… Завещаю вам: непоколебимо стоять на том пути, на который я наставил вас. …Идти в храмы, где служат достойные иереи, вепрю не подчинившиеся. Если и всеми храмами завладеет вепрь, считать себя отлучённым Богом от храмов и ввергнутым в голод слышания слова Божьего.
…Против власти, поставленной нам Богом по грехам нашим, никак нимало не восставать и во всём ей смиренно повиноваться».
"Вепром" Он называл раскольнического епископа, такого же раскольника, как и наш "Филарет"!
Святителю отче Луко, моли Бога о нас!

\ifcmt
  tab_begin cols=2

     pic https://scontent-lga3-2.xx.fbcdn.net/v/t1.6435-0/s600x600/103958974_964142300704687_2759061747140556395_n.jpg?_nc_cat=109&ccb=1-3&_nc_sid=730e14&_nc_ohc=UXmPADGJFqUAX9BzT8s&_nc_ht=scontent-lga3-2.xx&tp=7&oh=832713c85a81d58cfea624b4ecfd3241&oe=60C81D35

     pic https://scontent-lga3-2.xx.fbcdn.net/v/t1.6435-9/103574958_964142287371355_9201905659579938981_n.jpg?_nc_cat=105&ccb=1-3&_nc_sid=730e14&_nc_ohc=S4S78uiDtgwAX9PNwd_&_nc_ht=scontent-lga3-2.xx&oh=83dd1758e0b45ffb88b0bc9a2a040393&oe=60C87146

  tab_end
\fi

