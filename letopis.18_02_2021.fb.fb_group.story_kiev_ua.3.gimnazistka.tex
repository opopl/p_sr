% vim: keymap=russian-jcukenwin
%%beginhead 
 
%%file 18_02_2021.fb.fb_group.story_kiev_ua.3.gimnazistka
%%parent 18_02_2021
 
%%url https://www.facebook.com/groups/story.kiev.ua/posts/1600856123444516
 
%%author_id fb_group.story_kiev_ua,denisova_oksana.kiev.ukraina.gid
%%date 
 
%%tags kiev,kievljane
%%title История киевской гимназистки
 
%%endhead 
 
\subsection{История киевской гимназистки}
\label{sec:18_02_2021.fb.fb_group.story_kiev_ua.3.gimnazistka}
 
\Purl{https://www.facebook.com/groups/story.kiev.ua/posts/1600856123444516}
\ifcmt
 author_begin
   author_id fb_group.story_kiev_ua,denisova_oksana.kiev.ukraina.gid
 author_end
\fi

История киевской гимназистки.

Киевская гимназистка Надя Хазина. Она была дочерью известного киевского
адвоката Якова Хазина, жила с родителями в большой квартире на улице Рейтарской
25. Вот она на фотографии, 14-летняя, с косами, такой она бегала в гимназию на
соседней Большой Подвальной улице.  Иногда, выходя из парадного подъезда своего
дома, она видела молодых и явно влюбленных друг в друга юношу и девушку. Она
раскланивалась с ними, не подозревая, что встречает она Михаила Булгакова и его
жену Тасю, которые первые несколько месяцев после венчания жили в этом же доме
на Рейтарской 25. 

\ii{18_02_2021.fb.fb_group.story_kiev_ua.3.gimnazistka.pic.1}

Окончив гимназию, Надя Хазина по настоянию отца-юриста поступила на юридический
факультет Киевского университета, но проучилась там недолго, скучной ей
показалась юриспруденция. А вот рисовать ей очень нравилось,  и она брала уроки
в рисовальной студии художницы Александры Экстер. Она вращалась в кругу
киевской богемы,  увлекалась поэзией и в 1919 году в самом модном тогда кафе
«ХЛАМ » (Художники-Литераторы-Актеры-Музыканты) познакомилась с молодым, но уже
известным поэтом Осипом Мандельштамом. Он читал стихи о белых лилиях и на его
концерты она стала приходить с лилиями в руках, их познакомили,  и они как-то
сразу влюбились друг в друга. Они бродили по Киеву, не замечая холода и
разрухи, заходили в маленькие кафе погреться и говорили обо всем на свете.
Особенно уютным им казалась небольшая «Греческая кофейня на паях» в самом конце
Софиевской улицы, там однажды они поклялись быть вместе до конца жизни.

Они оба сдержали свою клятву, только у него жизнь оказалась очень короткой, он
умер в 1938 году от голода в пересыльной тюрьме под Владивостоком. А у нее –
очень длинной, она пережила его на 42 года и вела жизнь вечной скиталицы.
Опасаясь арестов, она будет все время менять адреса, переезжать с места на
место и, опасаясь  обысков, она заучит всю  прозу и   стихи  Мандельштама
наизусть. И в течение долгих десятилетий будет ежедневно кусками все это
повторять. И она добилась того, что все это было опубликовано.  И то, что мы
сейчас можем читать прозу и стихи Осипа Мандельштама, это благодаря
удивительной памяти и удивительной преданности нашей киевской девочки Нади
Хазиной. 

\ii{18_02_2021.fb.fb_group.story_kiev_ua.3.gimnazistka.pic.2}

А на доме №3б на улице Софиевской, где была когда-то греческая кофейня,  есть
вот такая памятная плита. На ней два голубя и как не вспомнить киевские стихи
Мандельштама: 

«Язык булыжника мне голубя понятней, 

Здесь камни — голуби, дома — как голубятни, 

И светлым ручейком течёт рассказ подков

По звучным мостовым прабабки городов».

\ii{18_02_2021.fb.fb_group.story_kiev_ua.3.gimnazistka.cmt}
