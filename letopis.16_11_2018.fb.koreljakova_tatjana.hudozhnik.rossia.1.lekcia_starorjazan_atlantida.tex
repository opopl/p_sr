% vim: keymap=russian-jcukenwin
%%beginhead 
 
%%file 16_11_2018.fb.koreljakova_tatjana.hudozhnik.rossia.1.lekcia_starorjazan_atlantida
%%parent 16_11_2018
 
%%url https://www.facebook.com/permalink.php?story_fbid=2151343008460442&id=100007542902629
 
%%author_id koreljakova_tatjana.hudozhnik.rossia
%%date 
 
%%tags atlantida,isskustvo,lekcia,rjazan.gorod,rossia
%%title ЛЕКЦИЯ "В ПОИСКАХ СТАРОРЯЗАНСКОЙ АТЛАНТИДЫ"
 
%%endhead 
 
\subsection{ЛЕКЦИЯ \enquote{В ПОИСКАХ СТАРОРЯЗАНСКОЙ АТЛАНТИДЫ}}
\label{sec:16_11_2018.fb.koreljakova_tatjana.hudozhnik.rossia.1.lekcia_starorjazan_atlantida}
 
\Purl{https://www.facebook.com/permalink.php?story_fbid=2151343008460442&id=100007542902629}
\ifcmt
 author_begin
   author_id koreljakova_tatjana.hudozhnik.rossia
 author_end
\fi

ЛЕКЦИЯ "В ПОИСКАХ СТАРОРЯЗАНСКОЙ АТЛАНТИДЫ". Сложно называть этот материал
сухим иноземным словом «лекция», потому что это очень живая,    наполненная
озарениями история!

\ii{16_11_2018.fb.koreljakova_tatjana.hudozhnik.rossia.1.lekcia_starorjazan_atlantida.pic.1}

- мой рассказ о Старой Рязани, месте, где на основе найденных артефактов 13
века, в тишине, спрятанное от суеты в туманах и утренних росах, создается новое
стилевое направление, продолжающее историю погибшего в огне древнерусского
города.

\ii{16_11_2018.fb.koreljakova_tatjana.hudozhnik.rossia.1.lekcia_starorjazan_atlantida.pic.2}

-это большой и реальный опыт стилевой работы, добытый подчас в абсолютно
полевых условиях, им я хочу поделиться с теми, кому интересны векторы развития
русского стиля и с теми, кто хочет делать русский стиль, но пока не решается.

-это много нового о забытом старом и можно воочию увидеть превращение «толстых
птичек» 13 века в комоды, светильники, ограды и скамейки.

\ii{16_11_2018.fb.koreljakova_tatjana.hudozhnik.rossia.1.lekcia_starorjazan_atlantida.pic.3}

-это обстоятельный, серьезный разговор, без «продающих текстов», из сердца в
сердце... более 250 файлов визуального ряда, который даже я сама видела пока
только по частям ...если вас интересует такой формат, вы его любите и не
боитесь, то УЖЕ ЗАВТРА БУДУ ЖДАТЬ ВАС 17 НОЯБРЯ В 17-00 В ГАЛЛЕРЕЕ АЛЬПЕРТ.
площадка выставки" Трын-трава".

\ii{16_11_2018.fb.koreljakova_tatjana.hudozhnik.rossia.1.lekcia_starorjazan_atlantida.cmt}
