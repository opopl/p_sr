% vim: keymap=russian-jcukenwin
%%beginhead 
 
%%file 06_09_2021.fb.fb_group.story_kiev_ua.1.dom_lapinskogo
%%parent 06_09_2021
 
%%url https://www.facebook.com/groups/story.kiev.ua/posts/1746744348855692
 
%%author_id fb_group.story_kiev_ua,denisova_oksana.kiev.ukraina.gid
%%date 
 
%%tags gorod,kiev,kievljane
%%title Дом Лапинского
 
%%endhead 
 
\subsection{Дом Лапинского}
\label{sec:06_09_2021.fb.fb_group.story_kiev_ua.1.dom_lapinskogo}
 
\Purl{https://www.facebook.com/groups/story.kiev.ua/posts/1746744348855692}
\ifcmt
 author_begin
   author_id fb_group.story_kiev_ua,denisova_oksana.kiev.ukraina.gid
 author_end
\fi

Есть в Киеве дома, которые просто притягивают к себе своей загадочностью. Таким
необычным и загадочным я считаю дом на улице Гончара, дом, который старые
киевляне до сих пор называют дом Лапинского.

\ii{06_09_2021.fb.fb_group.story_kiev_ua.1.dom_lapinskogo.pic.1}

Михаил Лапинский – известный киевский психиатр, профессор Киевского
университета, купил огромную усадьбу, которая располагалась между улицами
Бульварно\hyp Кудрявской и Мало\hyp Владимирской (так тогда называлась улица Гончара).
Купил у наследников барона Рудольфа Штейнгеля и в огромном доме, стоявшем на
этом участке и выходившем фасадом на Бульварно\hyp Кудрявскую, открыл ставшую сразу
популярной водолечебницу с модными водными массажами и лечебными ваннами.
Водолечебницу он назвал «Бульварно\hyp Кудрявский санаторий»,  и располагалась она
в роскошном доме в готическом стиле. Лечебница приносила хороший доход и
Михаил Лапинский решил застроить вторую часть усадьбы, выходившую на улицу
Мало\hyp Владимирскую. И в 1908 году появился очень красивый доходный дом тоже в
готическом стиле. В доме  было два электрических лифта, роскошные 5-ти и 7-ми
комнатные квартиры, в которых были  паркетные полы, керамическая плитка в
ванной, подоконники из натурального камня. Двор дома был спроектирован как
настоящий двор в замке – с мостом над сухим рвом, с дворовыми пристройками, где
для каждой квартиры были предусмотрены ледник и прачечная. В самой большой
квартире этого дома поселился сам Михаил Лапинский с семьей. Дом выглядел как
настоящий дом\hyp замок, но самая большая тайна этого дома в том, что до сих пор
никто не знает имени его архитектора.

Жил Михаил Лапинский в этом доме до 1918 года, а потом уехал, думал, что на
несколько месяцев, пока все образуется, а оказалось, что навсегда. Он добрался
до Загреба, открыл там частную клинику и стал профессором Загребского
университета. И его судьба оказалась тесно связанной  с судьбой семьи
Булгаковых. Сначала Михаил Лапинский принимал экзамен по неврологии у
студента-медика Михаила Булгакова. И именно клиника доктора Лапинского описана
в романе «Мастер и Маргарита» как пристанище Мастера и поэта Бездомного. Потом
судьба столкнула его с младшим братом Булгакова – Николаем. Дело в том, что
двое младших братьев Михаила Булгакова, Коля и Ваня, в 1919 году эмигрировали,
причем Коля успел окончить гимназию,  а Ваня не успел. Николая Булгакова судьба
занесла в Загреб, где поступить на естественный факультет ему помог именно
профессор Лапинский и позже  все время помогал Коле в тяжелой эмигрантской
жизни. В итоге, Николай Булгаков окончил Загребский университет, стал всемирно
известным ученым\hyp бактериологом.  А вот младший – Ваня, гимназию в Киеве
окончить не успел и всю жизнь работал балалаечником в Париже в русском
ресторане. 

А имя Михаила Лапинского сохранилось в Киеве, потому что старые киевляне до сих
пор называют этот дом домом Лапинского. И дом по-прежнему хранит свою главную
тайну – имя своего архитектора...

\ii{06_09_2021.fb.fb_group.story_kiev_ua.1.dom_lapinskogo.cmt}
