% vim: keymap=russian-jcukenwin
%%beginhead 
 
%%file 20_11_2021.yz.tkachev_andrej.protoierej.1.chto_takoje_chelovek
%%parent 20_11_2021
 
%%url https://zen.yandex.ru/media/andreytkachev__official/chto-takoe--chelovek-61979f0c065aec7d40b5da8d
 
%%author_id tkachev_andrej.protoierej
%%date 
 
%%tags chelovek
%%title «Что такое» – человек?
 
%%endhead 

\subsection{«Что такое» – человек?}
\label{sec:20_11_2021.yz.tkachev_andrej.protoierej.1.chto_takoje_chelovek}

\Purl{https://zen.yandex.ru/media/andreytkachev__official/chto-takoe--chelovek-61979f0c065aec7d40b5da8d}
\ifcmt
 author_begin
   author_id tkachev_andrej.protoierej
 author_end
\fi

Человек – это Бог в грязи. Все «лувры» мира, собранные вместе – это кусочек
грязи по сравнению с душой одного человека. Неважно, какой душой. Любой.

Человек любим Богом. Хотя существует огромное мироздание. Настолько сложное и
красивое, что человек по сравнению с ним меньше вошки, меньше блохи, меньше
молекулы. Человек – это ничто. Вроде бы. Но нет. За него кровь Христова
пролита. А значит, он – бесценный и дорог Богу. Но когда Бог любит человека, Он
не любит то, что в нём сейчас есть. Потому что иногда в этом всём любить-то,
собственно, и нечего. Он любит ту скрытую красоту, которая есть в человеке.

\ii{20_11_2021.yz.tkachev_andrej.protoierej.1.chto_takoje_chelovek.pic.1}

Человек – (каждый) – как непрочитанная книга. Вполне написанная, но не
прочитанная. И тем более, не экранизированная. Великий от невеликого отличается
тем, что великий – «экранизированный», а невеликий – он канул в эту лету
человеческой памяти.

«Человек – это то, что нужно преодолеть!» Слышали вы такую философскую
сентенцию? Самая главная задача наша – это быть выше, чем человек. Человек
постоянно преодолевает себя. Самое главное в преодолении себя – это борьба с
внутренним несовершенством, любовь к Богу и стремление в другую жизнь.

Человек – как матрешка. Внешне – грубая. А там, внутри – поменьше, поменьше.
Потоньше. Самое миниатюрное сокровенное – оно вообще очень маленькое. Наша
личность – «Я», все самое драгоценное, какой-то алмазик внутренний. То, до чего
и добраться тяжело. Сокровенное – «Я». Человеческое.

Человек – всегда – это сосуд. Человек создан как чаша. Как кувшин. В чашу нужно
что-то налить. Вот что в тебя налили, для того ты и живешь. Современный человек
часто имеет такую тенденцию – быть снаружи красивым, а внутри себя хранить
всякую нечисть, всякое стыдное. И тогда человек похож на золотую чашу,
наполненную каким-то свинячим пойлом. А иногда посмотришь на человека –
старушечка, например: ножки еле ходят, на палочку опирается, глазки плохо
видят. «Горшок» старый уже, такой уже потрескавшийся. Сосуд очень-очень не
изящный, а в нем – святыня. В нем – Христос. И это видно.

Человек – несовершенный человек. Вот она, трагедия человеческой жизни. Человеку
легко сделать больно. Нежный организм человеческий. У него столько всяких
нежных мест; и можно столько всякого железа и огня к нему применить, что любой
кремень задрожит. Слабый человек! Даже в числе лучших своих представителей.
Самые лучшие люди оказываются не всегда самые лучшие. А других больше нету. Как
Исаак Сирин говорит: «Совершенство совершенных – воистину несовершенно». Полный
страстями человек. Слабостями и страхами полный. Вот он какой. Каждый хочет
себе поблажки какой-то, какого-то ослабления трудов, чего бы это не касалось.
«А можно мне здесь полегче?» «Или вот здесь?» Это наш такой общий портрет. В
нем есть правда, в нем есть печальная правда, потому что все мы изрядно
ослабели. Каждый в отдельности. И все вместе.

Человек – существо динамическое. Он – движется. Он постоянно находится в
движении: он либо движется вперед, либо откатывается назад, либо стремительно
падает. Но пока жив человек, он может поменяться. Когда не сможет поменяться,
Бог его заберет отсюда. Потому что ему здесь делать больше нечего.

Человек может поступать трояко. Он может поступать «по естеству», может
поступать – «ниже естества» и может поступать – «выше естества». Например, я
взял у Вас взаймы пять тысяч рублей. И вернул Вам пять тысяч рублей. Я поступил
– по естеству. И Вы поступили по естеству. Я взял у Вас пять тысяч рублей, а Вы
сказали: «Не возвращай. Не надо!» Вы поступили выше естества. Вы поступили «по
благодати». Как Христос. И, наконец, я взял у Вас пять тысяч рублей, а Вы
сказали мне: «Через неделю вернешь – семь!» Вы поступили как дьявол. В идеале
христианин должен быть выше естества. Он должен жить – «по благодати». И в
благодати совершать невозможное. Но сегодняшняя жизнь, к сожалению, такова, что
даже естественные, природные, простые, понятные вещи становятся редкостью. В
этом специфика нашей жизни.

Человек что хочет в обычной жизни? Чтобы его не трогали, чтобы он был
безопасен. Чтобы у него был достаток некий, чтобы он не боялся завтрашнего дня.
Чтобы он был здоров. И чтобы те, кого он любит тоже были бы защищены, здоровы и
богаты. Что еще нужно человеку? В принципе, больше ничего. Больше, хоть тресни,
ничего не придумаешь. Но в режиме комфорта, в режиме расслабленности, человек
своего Я не раскрывает. В режиме комфорта мы – свинья свиньей. Вы знаете это по
нашей отечественной истории, как мобилизуется народ в беде. Как он хамеет,
дуреет, жиреет в период благоденствия. Превращается в какую-то карикатуру. А
как приходит беда, люди мобилизуются и становятся вот теми самыми, которыми мы
гордимся. Мы гордимся отцами и дедами, и историей нашей только благодаря
жестоким событиям; которые высветили в людях наших прежних поколений самые
глубинные правильные черты.

Человек не может быть счастлив, только отдыхая, кушая и веселясь. Это –
недостойно человека. Человек, если имеет возможность ничего не делать, он все
равно должен что-то делать. Помните – советский мультик? Мальчик попал в страну
сказок, а там царь забор красит. Он спрашивает: «А зачем ты красишь забор? Ты
же царь». Царь зовет: «Эй, стража. Отрубите ему голову. Он – лентяй!» Лентяй
попался в королевстве. Цари тоже могут заборы красить. Нормальный человек хочет
работать. Труды бывают разными, но трудиться обязательно нужно. Человек до
смерти должен трудиться. И если маленький человек будет делать маленькое доброе
дело каждый день, из его маленьких камушков сложится большая мозаика к концу
его жизни. Большая красивая мозаика добра.

Человек с трудом видит что-то великое, а потом с этим великим всю жизнь живет.
Если человек узнаёт про тех, кто лучше его и радуется, это значит, что в нём
здоровая, совершенно здоровая христианская душа. А если человек узнаёт про тех,
кто лучше его и завидует, и унывает оттого, что он таким быть не может, значит
у него больная гордая паршивая душонка, которую ещё только лечить и лечить.
Проверяйте себя. Есть радость – значит, вы нормальные люди. Нет радости, есть
уныние, значит – душа больная и нездоровая значит, в тебе ещё много такого, не
очищенного, не исправленного.

Человек раскисает от похвалы. Там, где нас хвалят, там мы расплываемся в глупую
улыбку. Правда? Мы – люди, мы – такие. А когда тебе говорят некоторые
правильные вещи, ты можешь раздражаться. Да, люди раздражаются. Раздражаются
именно на то, что задевает самую болячку сердца. В нас столько гнили всякой.

Человек – это гнилая трость. На человека опереться нельзя. Обопрешься – трость
треснет и войдет тебе в руку. Так пишется у пророка. Нужно, главным образом,
надеяться на Бога. На самого хорошего человека тоже нужно надеяться с опаской.
Потому что он сегодня очень хороший. А завтра, глядишь, и порвался по швам.
Бывает такое. С любым человеком бывает. Ну, в первую очередь себе нельзя
верить, потому что мы тоже это «странное существо». И, если мы в себе уверены,
то мы на краю беды. Себе тоже нельзя верить.

Человек с трудом сдерживается в скорби и радости. В скорби человеку тяжело не
показывать свою скорбь. А в радости очень трудно не показывать, что у тебя
радость. Но, видимо, бывают такие ситуации, когда твоя радость оскорбительна
для окружающих. Например, всем – плохо, а тебе хорошо. Ты в лотерею машину
выиграл, а у соседа, например, мать умерла. Ты что будешь петь и плясать под
его забором? Не будешь.

Человек – самое слабое из всех животных. Помните? Лошаденок родился – и уже на
ножки встает. Любое животное, только-только его мамка родила из чрева, оно уже
на слабых ногах, но стоит. А человек как? Пока пойдет, пока заходит – сколько
месяцев пройдет. Он самый слабый. Он самый ничтожный. А вот одарил же Господь
этого слабого человека царственным умом и способностями. Он и полетел по небу,
и преодолел себя. Поплыл под водой – преодолел себя. Научился всякому-всякому –
преодолевает каждый раз себя. Это тоже такая интересная вещь. И какой же смелый
человек, если Бог с ним! Если Бог с человеком, он может идти один против сотни.
И не испугается. А если Бог отступит от человека – он будет бояться всего.
Будет бояться маленького ребенка. Будет бояться падающего листочка. Будет
бояться скрипа автомобильных шин. Будет бояться любой тени, любого лая
собачьего. Всего будет бояться.

Блез Паскаль говорил, что, если бы человек знал, как он велик – он бы
чрезвычайно возгордился. Был бы новый диавол возможно. Была бы опасность нового
грехопадения. Для этого ему дала эта смиренная оболочка: то он спать хочет, то
у него что-то болит, то – чешется. И вот в этом униженном состоянии он
пребывает, чтобы не возгордиться. Потому что по замыслу он очень велик.
Чрезвычайно велик. Человек!
