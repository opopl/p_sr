% vim: keymap=russian-jcukenwin
%%beginhead 
 
%%file 18_12_2021.fb.bilchenko_evgenia.1.eto_konec
%%parent 18_12_2021
 
%%url https://www.facebook.com/yevzhik/posts/4554966111205130
 
%%author_id bilchenko_evgenia
%%date 
 
%%tags bilchenko_evgenia
%%title БЖ. Это конец
 
%%endhead 
 
\subsection{БЖ. Это конец}
\label{sec:18_12_2021.fb.bilchenko_evgenia.1.eto_konec}
 
\Purl{https://www.facebook.com/yevzhik/posts/4554966111205130}
\ifcmt
 author_begin
   author_id bilchenko_evgenia
 author_end
\fi

БЖ. Это конец

Я знала, что это страшно, но я не знала, до какой степени это страшно и
невыносимо. Я не выдержу второй волны травли: я просто не хочу и не могу её
выдерживать. Мне, наверное, надо сделать то, что они от меня хотят. Они от меня
хотят послания: \enquote{Украинцы, которые в 2014 году поддерживали госпереворот, а
потом покаялись: независимо от того, что вы сделали для русского мира, вас
никогда и никто не простит. Не приезжайте сюда никогда}. А я не могу это все
написать на радость моих украинским оппонентам, потому что я очень люблю
Русский Мир. Украинские оппоненты просто захлебываются от радости, что меня
здесь травят. \enquote{Россия уничтожает своих людей}.  Это настолько не так, что я не
могу так предать мир, который люблю. Я не знаю, что делать с этими адскими
статьями, мемами, вирусами, которые льются со всех щелей сети.

\ii{18_12_2021.fb.bilchenko_evgenia.1.eto_konec.pic.1}

Это мой единственный грех. Русский Мир. Я люблю его не так, \enquote{как надо}. Мой
крик о нем не вписывается в рамки структуры, а любить надо исключительно
структурно. Здесь все расписано: каждому сверчку - свой шесток, если местные
русофобы работают на западные гранты посреди столицы, значит, так и надо, и не
дело это - такого ничтожества, как я, возмущаться этому. Вообще надо молчать.
Просто не отсвечивать, сдохнуть, подавиться.

Плохо, что я не могу принять поддержку либералов. Если бы я смирилась и заняла
бы их нишу, я была бы предсказуемые в глазах терзающей меня толпы, я была бы
понятнее.  Я не могу, потому что в моих глазах это и есть предать Русский Мир.
Терзающая меня толпа патриотов как бы вымогает, чтобы я была с либералами: они
им нравятся, откровенно уничтожающие Россию \enquote{свои} нравятся им больше
покаявшихся \enquote{чужих}.

У меня появился \enquote{страшный грех} - самолюбие. Я должна быть тише воды и ниже
травы. Меня приучают к самобичеванию и называют это \enquote{смирением}. Я уже не знаю,
как ещё смириться, как склониться, куда ниже. Мои студенты из Киева и других
городов Украины, которые идеализируют Россию, говорят: \enquote{Евгения Витальевна, ну,
вы же уже преподаете?} Мне очень стыдно отвечать: \enquote{Нет и не буду}. Я не хочу
разуверивать их в России как в месте. Но реальность прёт из всех дыр.

Я не могу им сказать, что меня не только преподавать не берут, мне запрещают
дышать. Дышать - наказуемо. Наказуемо напоминать, что я была доктором наук. Что
я вообще-то - профессор, учёный, у меня хоть что-то есть. Это уничтожается
грязным черным смехом, как будто меня сжигают на костре, и вокруг все хохочут и
юродствуют. Я смирилась с этим. Я разослала везде резюме. 

Когда я поняла, что преподавать - нельзя, и меня начали запугивать моей семьёй
(чего не делали даже в/на Украине), я имела ещё крохи достоинства вспомнить,
что я - поэт. Писатель. Блогер, наконец, и могу фрилансить в сети, выступая со
стихами и продавая рукописи за жалкие донаты: мне же кидают мелочь за тонны
годами вынашиваемой информации, потому что поставить таксу или попросить
гонорар я стесняюсь. Я стесняюсь делать то, что делают писатели во всем мире,
сведя все к минимальной форме труда за посильную сумму. Оказывается, и это
нельзя.  Я стала очень низко себя ценить. Мне запретили делать всё. Нельзя
выступать, писать, говорить. Каблук в рот.

Выход: голодная смерть или работать на грубой физической работе. Разве я ее
презираю? Я готова забыть, что я доктор наук, поэт и все это. Из меня просто
ногами выдавливают представление о том, кто я. Хорошо, я никто (мне кажется, в
этом тоже какой-то грех, я ничего не имею против физического труда. Я просто не
умею. Я умею головой, но мне наступают сапогом на голову. В/на Украине мне
предлагали идти в уборщицы, да. Я бы пошла, но у меня нет здоровья, просто нет
на работу уборщицей. Ну, в гробу везти меня можно, но я не смогу: это
концлагерь, пытка.

Остались два последних киевских заказа: огромные чужие диссертации за 15 тысяч
рублей. При их виде меня уже мутит. Я не могу видеть дисплей монитора с ними. Я
же тоже человек! За что, а... \enquote{Тварь, это в тебе говорит самолюбие, ты
тварь, дерьмо, ничтожество, ты грязная бывшая звезда, заглохни}. Как закрыть
ВК-страницу? Научите, я искала и не нашла, я хочу и боюсь выброситься в окно,
мне просто надо закрыть это. Закрыть и поддаться на иллюзии близких, которые
ничего не понимают, что \enquote{все хорошо}.

Моя кровать стоит так, чтобы дотянуться одной рукой дот чайника и второй до
компьютера. Я с нее практически не встаю. Я не хотела ныть, не хотела говорить,
что зарегистрирована в госуслугах по диагнозу подозрения на рак шейки матки, и
первое, что стоит в списке: \enquote{Биопсия}. Я себя плохо чувствую. Очень плохо. Я не
знаю, что делать в ответ на эту диагностику ценой в золото без полиса. Я не
знаю, зачем я доживаю. Когда врач делал мне спутник, он сказал, что, если я не
буду лечить еще хронический сепсис, дальше будет хуже, я постепенно умру. У
моей болезни - два крыла, делать с этим нечего, это огромные деньги, которые я
не заработаю и не прошу, потому что мне сказали, что уничтожат меня, если я
попрошу в обмен на книги хоть что-то.  

Если мне нельзя ничего - если смерть близка, и я получаю письма с угрозами: \enquote{Не
вздумайте жаловаться!} (чтобы создать картину благополучия, в элементарных
условиях которого мне отказано: умирай тихо), я могу хотя бы показать людям
лучшее, чем светит Русский Мир. Меня муж за руку водил на эти концерты.
Подвозили, чтобы я увидела. Я совершила страшный грех: не будучи в состоянии
работать, писать, выступать в Русском Мире, творить свое, я показала, как
прекрасны его поэты и музыканты, я похвалила его. За это меня не просто смешали
с грязью - меня убили. Я уже оказалась проституткой, которая кутит по кабакам.
Я показывала лучшее, что есть, но оказалось - оно, хотя и не либеральное, но не
в системе, его нельзя. 

В общем, я не знаю, кого просить про быструю смерть. Жить с запретом писать,
что думаешь, работать и лечиться в перспективе медленного мучительного умирания
я не хочу. Я устала от бесконечно длящейся болезни, мое тело больше не в
состоянии. Оно не может. На/в Украине я очень ждала, что люди Медведчука придут
ко мне. Но мне сказали: я - слишком явный Русский Мир, мне надо валить.

Я не могу вернуться: меня на/в Украине не убивают. Ни один из парней в камуфле,
который наматывал круги вокруг моего дома или толкал меня, не врезал мне до
смерти. Это был бы выход! Мне сказали, что мне отказано в убийстве, потому что
это - на руку России. Теперь и вернуться некуда: дом продан, мосты сожжены.
Дома близкий человек считает, что все решит бан моих гонителей. Он не видел,
как в России срывали мне все выступления, он не понимает, каково это: прятаться
постоянно, потеряв предназначение учителя и поэта, за которое он меня и полюбил
отчасти. Бан, и и проблемы нет. Проблема приходит в реальности. Она множится по
сайтам, она бежит за мной, она убивает меня.

В реальности я отказываюсь быть. Я не захотела ехать в Европу, я отвергла
поддержку либералов, которые могли бы меня поддержать, но я сама не могу. И за
это разъяренной патриотической толпой я растерзана до крови, до мяса, до грязи,
моя страница в ВК стала тем же, что ФБ-страница во время украинской травли,
украинские оппоненты радостно приветствуют свое продолжение в моих русских
оппонентов. Я никогда не видела столько уничтожительной людской ненависти, мне
это чуждо, я могу в ответ только умереть.

Простите, но я не хочу жить. Я не могу жить так. Я пыталась. Честно. Человек не
может терпеть так долго такую сильную физическую и моральную муку. Есть
пределы. Я не могу жить в страхе, что делать, если у меня грипп, куда бежать,
если не греет батарея. Это инфантильно, смейтесь, но у меня нет средств на
коммерческую поддержку своей физической жизни, а в иной мне отказано.

Наверно, они хотят, чтобы я сказала: \enquote{Украинцы, не кайтесь и не возвращайтесь.
Россия не прощает своих детей}. А я не могу этого сказать

Это, как пытка сном, пытка в карцере, пытка, которую производит со мной толпа.
Меня всегда ненавидела именно толпа. Именно с подачи толпы потом ко мне
приходили чиновники и спецслужбы. Обычно власти не было никакого до меня дела,
а мне - до власти. Но толпа не хочет моего бытия. Толпа не прощает ничего
вообще. Ты должен искупить смертью все: и ошибки, и грехи, и подвиги

Тола не прощает не украинского прошлого, а то, что мое настоящее \enquote{слишком}
прорусское. 

Я и отказываюсь быть. Простите, но я должна была сказать. Иначе я умру с
сомкнутыми устами. Я не могу жить. Не могу и не хочу

Так жить просто нельзя. Не приезжайте сюда, да. Вас  Россия простит, вас
уничтожит толпа. Как и в/на Украине.

Меня больше не будет в сети долго, возможно, никогда, я пытаюсь взять курс на
самоуничтожение: пусть толпа пишет стихи, раз некому за меня заступиться. Я
ДЕЙСТВИТЕЛЬНО больше не могу ТАК.

%\ifcmt
  %ig https://i2.paste.pics/ca31a2c427ac29f7c0bd9cb4f12562f8.png
  %@width 0.6

	%ig https://i2.paste.pics/349f75af14a99557ac96f61f1bdd7dde.png
  %@width 0.6
%\fi

\ii{18_12_2021.fb.bilchenko_evgenia.1.eto_konec.cmt}
