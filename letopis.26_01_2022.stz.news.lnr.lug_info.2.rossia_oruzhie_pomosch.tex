% vim: keymap=russian-jcukenwin
%%beginhead 
 
%%file 26_01_2022.stz.news.lnr.lug_info.2.rossia_oruzhie_pomosch
%%parent 26_01_2022
 
%%url https://lug-info.com/news/donbass-budet-priznatelen-rf-esli-ona-pomozhet-oruzhiem-v-sluchae-agressii-kieva-pasechnik
 
%%author_id news.lnr.lug_info
%%date 
 
%%tags donbass,lnr,nm_lnr,oruzhie,pasechnik_leonid,pomosch,rossia,ukraina,vojna
%%title Донбасс будет признателен РФ, если она поможет оружием в случае агрессии Киева - Пасечник
 
%%endhead 
 
\subsection{Донбасс будет признателен РФ, если она поможет оружием в случае агрессии Киева - Пасечник}
\label{sec:26_01_2022.stz.news.lnr.lug_info.2.rossia_oruzhie_pomosch}
 
\Purl{https://lug-info.com/news/donbass-budet-priznatelen-rf-esli-ona-pomozhet-oruzhiem-v-sluchae-agressii-kieva-pasechnik}
\ifcmt
 author_begin
   author_id news.lnr.lug_info
 author_end
\fi

Республики Донбасса будут признательны России, если она в случае агрессии Киева
окажет помощь поставками отдельных видов вооружений. Об этом заявил глава ЛНР
Леонид Пасечник.

Ранее сегодня первый вице-спикер Совета Федерации РФ, секретарь Генерального
совета \enquote{Единой России} Андрей Турчак заявил, что Российской Федерации следует
оказать помощь Народным Республикам в виде поставок отдельных видов вооружений.

\ii{26_01_2022.stz.news.lnr.lug_info.2.rossia_oruzhie_pomosch.pic.1}

\enquote{В случае, если здравый смысл у руководства Украины все-таки не проснется, нам,
разумеется, потребуется поддержка. Если Россия нам ее окажет – мы будем
признательны братской стране!} – сказал Пасечник.

Он отметил, что в тревожное для Донбасса время – \enquote{это и обострение на линии
фронта, и похищение наших жителей, и постоянное стягивание Украиной военных и
техники в прифронтовые территории, и участившиеся случаи отправки западных
инструкторов в зону проведения так называемой операции объединенных сил} –
Россия вновь не оставляет жителей Донбасса наедине с врагом.

\enquote{За почти восемь лет войны на Донбассе военные ЛНР и ДНР, усовершенствовав
уровень боевой подготовки, безусловно, научились безукоризненно отражать
агрессию противника, умело отвечая на его провокации, - отметил глава
Республики. - Но не стоит забывать о том, что любимые Киевом \enquote{заокеанские
партнеры} в огромных объемах снабжают \enquote{незалежную} военной помощью, в то время
как львиная доля техники, стоящей на вооружении армий Луганска и Донецка, –
трофейная, полученная в боях с ВСУ за наши города и села в 2014-2015 годах}.

Пасечник подчеркнул, что жители Республики \enquote{должны быть готовы к любому
сценарию развития событий}.

\enquote{Но вместе с тем не теряем надежды, что Киев, наконец, поймет, что ключ от всех
проблем – выполнение Минских соглашений. Война на Донбассе не нужна никому: ни
жителям ЛНР и ДНР, ни самим украинцам, уровень жизни которых стремительно
падает с каждым новым годом вооруженного конфликта на юго-востоке страны}, -
сказал глава ЛНР.

Власти Украины начали силовую операцию против Донбасса в апреле 2014 года.
Урегулирование конфликта базируется на Комплексе мер по выполнению Минских
соглашений, подписанном 12 февраля 2015 года в белорусской столице участниками
Контактной группы и согласованном с главами стран - участниц \enquote{нормандской
четверки} (Россия, Германия, Франция и Украина). Обеспечить выполнение
положений документа стороны конфликта призвал Совет безопасности ООН, который
одобрил документ резолюцией № 2202 от 17 февраля 2015 года.

Комплекс мер предусматривает прекращение огня, отвод тяжелых вооружений от
линии соприкосновения, начало диалога о восстановлении социально-экономических
связей Киева и Донбасса, а также реформу конституции Украины с целью
децентрализации и закрепления \enquote{особого статуса отдельных районов Донецкой и
Луганской областей}.
