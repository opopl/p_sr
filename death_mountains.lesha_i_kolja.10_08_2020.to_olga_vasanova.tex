% vim: keymap=russian-jcukenwin
%%beginhead 
 
%%file lesha_i_kolja.10_08_2020.to_olga_vasanova
%%parent letters
 
%%endhead 
\subsection{Olga Vasanova}
\url{https://www.facebook.com/olga.vasanova}
  
\vspace{0.5cm}
 {\ifDEBUG\small\LaTeX~section: \verb|lesha_i_kolja.10_08_2020.to_olga_vasanova| project: \verb|death_mountains| rootid: \verb|p_saintrussia| \fi}
\vspace{0.5cm}
  
Доброго ранку, Оля! Вчора був міжнародний день альпінізму, щиро поздоровляю
тебе... І ти напевне памятаєш, як ти свого часу летіла 300 метрів з Петроса...
як я тебе возив до лікарні... а нога твоя вся в зеленці була... ти може вже і
забула, оскільки я бачу, ти мене видалила з друзів на фейсбуці... а я то все
памятаю... так от.

Хотів би тут сказати... а нащо ми
ходимо в гори? Різні люди відповідають по різному... а я персонально - заради
людського спілкування, заради Друзів, Товаришів... І тут є зараз в нашому
турклубі Університет трохи дивна ситуація, коли є двоє загиблих товаришів - 20
років загибли в лижному поході в 2001 році... але... про них всі забули... їх
начебто не існує. Люди продовжують ходити в класні походи, лізуть на скелі і
тому подібне... А при тому... про своїх загиблих друзів забули. Ось я навіть
недавно поздоровляв керівника нашого турклуба Володимира Триліса із Міжнародним
Днем Дружби - питаю його, чи ти знаєш, де в Києві поховані Льоша і Коля, це ж
твої загиблі друзі з твого походу в 2001 році, в якому ти був керівником... так
уявляєте, навіть він не знає... питаю різних членів нашого турклубу... де
поховані, хто їх батьки... або мовчать, або кажуть, всі телефони
порозгублювали... хм... щось дуже дивне... Забули своїх загиблих Друзів, і
думають, начебто все ок... Ні, товариші з турклубу Університет, не все ок. НЕ
ВСЕ ОК. Вічна память Льоші і Колі, загиблим у поході 2 категорії складності під
керівництвом Володимира Триліса, відомого також як Інтелігент. НЕ ВСЕ ОК...

https://crime.fakty.ua/98545-brosivshis-na-pomocsh-soskolznuvshim-s-obledenelogo-sklona-tovaricsham-dvoe-kievskih-turistov-razbilis-nasmert
