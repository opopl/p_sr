%%beginhead 
 
%%file 30_01_2023.fb.rojz_svitlana.kyiv.1.pam_yat___stres
%%parent 30_01_2023
 
%%url https://www.facebook.com/svetlanaroyz/posts/pfbid02wPX1JKxxU4NjnErkBTg8BiGKwLWAAL2QxAVyEhe8ztfzbh2gJ39RuY9tSrbH3XGkl
 
%%author_id rojz_svitlana.kyiv
%%date 30_01_2023
 
%%tags chelovek,psihika,psihologia,pamjat,stress
%%title Пам'ять і стрес
 
%%endhead 

\subsection{Пам'ять і стрес}
\label{sec:30_01_2023.fb.rojz_svitlana.kyiv.1.pam_yat___stres}

\Purl{https://www.facebook.com/svetlanaroyz/posts/pfbid02wPX1JKxxU4NjnErkBTg8BiGKwLWAAL2QxAVyEhe8ztfzbh2gJ39RuY9tSrbH3XGkl}
\ifcmt
 author_begin
   author_id rojz_svitlana.kyiv
 author_end
\fi

Пам'ять і стрес. Присвячується всім \enquote{Рибкам Дорі}. Сподіваюсь, буде практично.
Багато, але хочу підійти комплексно. 

Я спостерігаю за собою і проводжу різні опитування, щоб розуміти нашу \enquote{середню
температуру по палаті}.

Зрозуміла - що мені особисто зараз потрібно 3 дні після нового стресору
(обстрілів, загострення небезпеки), щоб відновилась концентрація уваги,
повернулась швидкість мовлення та звична швидкість обробки інформації та
запам'ятовування. Раніше на це йшов 1 день. Після кожного обстрілу на наступний
день йде закономірний емоційний та фізичний відкат. Сонливість, застигання чи
гіперзбудженість.

Під дією гормонів стресу звужуються судини - і, на жаль, закономірні зміни в
тиску, головокруття та головний біль.

Ми знаємо, що довготривалий стрес - пригнічує синтез білка. (Білок - це блок в
будівництві імунної системи) - ми можемо частіше хворіти, це може призвести до
пошкодження клітин мозку, особливо, гіпокампу (він пов'язаний із процесами
довгострокової пам'яті) - він особливо чутливий до дії кортизолу - а це у свою
чергу, призводить до порушення пам'яті і депресії. Стрес знижує активність
префронтальної кори, яка відповідає за контроль, частково за концентрацію,
запам’ятовування і вилучення інформації.

Що для нас тут критично - руйнування зв'язків між нейронами. Якщо ми
спостерігаємо, що концентрація та пам'ять не відновлюється, нам потрібні
неврологи, і можливо, допомога медикаментозна.

Що важливо знати, щоб покращити стан:

☀ Гіпокамп чутливий до кисневої недостатності - нам буквально потрібен кисень,
дихальні вправи, прогулянки.

☀ В гіпокампі, в тому числі здійснюється процес нейрогенезу - утворення нових
нейронів.

А стимулюємо цей процес, коли ми рухаємося.

Нам обов'язково потрібен рух

☀ Проблема може бути і не в самій пам'яті, а і в загальній тривозі,
концентрації уваги, багатозадачності.

Нам потрібні практики для стабілізації емоційного стану (робота з психологами)

Практики уважності (майндфулнес) прості вправи я запропоную в кінці матеріалу.

Нам важливо уникати багатозадачності. Зараз нашому мозку це може бути не по
силах.

☀ Ключовий метод регуляції страху - зміцнення прив'язаності, близькості. Нам
потрібна близькість. Кола підтримки

☀ Процес консолідації пам'яті відбувається під час сну. Нам важливо врегулювати
сон.

Зараз розповім те, що наче напряму не пов'язано із пам'яттю - пов'язано з
реакцією на стрес і його наслідками.

Після травматичної події буде корисним навантажити ділянки мозку, що пов'язані
із просторовою орієнтацію - навантажити просторову та зорову пам'ять - можливо,
тетріс, ігри із словами та літерами (що переставляються).

Після травматичного досвіду до першого сну важливо втручання: потрібно
відновити цілісну картину того, що відбулося, відчути підтримку, розповісти
(\enquote{вигрузити}, вибачте, в когось) ідеально - відпрацювати із спеціалістом
спеціально розроблені протоколи,що допомагають \enquote{розрядити} вплив стресу.

Читаю дослідження про те, що першу ніч після травматичної події краще не
вживати снодійне, щоб сон був поверхневим. добре, якщо він в першу ніч після
події не глибокий. (Процес консолідації пам'яті - записування в довгострокову
пам'ять - відбувається під час сну). Хочу розібратись і про це більше
поговорити з нейропсихологами. Тут дослідження:

\url{https://pubmed.ncbi.nlm.nih.gov/?term=sleep%20deprivation%20%20ptsd%20prevention&sort=date}

\ifcmt
  ig https://i.paste.pics/4821aed7bcab089db54a0d21c29d37ab.png
  @wrap center
  @width 0.8
\fi

Для відновлення роботи мозку нам важливі мікроелементи та вітаміни (які призначає лікар)

\enquote{Цинк бере участь у регулюванні мозкової діяльності, зміцнює нервові волокна.
Вироблення серотоніну неможливе в умовах дефіциту цинку. Вітаміни групи В і D
допомагають засвоювати білок і стимулюють ріст клітин мозку. Магній тонізує
нервову систему й нервові закінчення. Вітамін К бере участь у синтезі жирів,
важливих для розбудови та роботи клітин, а тому вважається одним із найкращих
для пам'яті. Вітаміни А та С підвищують імунітет і підтримують активність
мозку. Лютеїн і зеаксантин корисні для зору, а отже, для зорової пам'яті}.

☀ Практики уважності для дітей та дорослих:

1. Взяти кілька дзвіночків, кожен може мати свій звук. Прикласти до вуха та
прислухатися до свого звуку. Розрізняти інші звуки, але концентруватись на
своєму.

2. Спостерігати за рибками в акваріумі

3. утримати на голові предмет

4. Розплутати вузол (на міцній нитці, шнурку або канаті) дорослі зав'язують
нескладний вузол, дитина розплутує

5. Утримати пір'їну чи кульку над головою диханням

6. Грати в \enquote{замри - розтань} Утримуватися в одній позі якомога довше

7. Спостерігати за падінням снігу, як стікають дощі по склу, як акварель
розтікається серветкою або вологим папером, або фарба розчиняється у воді

8. Нести воду в ложці або склянці, намагаючись не розлити

9. Перебирати намистини чи крупи

10. Дивитися на вогник свічки

11. Пускати бульбашки (не тільки мильні, а й через трубочку видмухувати у
склянку з водою або сік)

12. Зібрати пальці в кулак - на кожному видиху розгинати один палець

13. Уявити, що вдихаєш хмару, а потім, з довгим видихом і з будь-яким звуком
видути його

14. Взяти знайомий предмет і дослідити його – постаратися побачити нові якості.
Раптом ми помітимо те, що не бачили раніше. Колір, форма, матеріал, прозорість,
звук, температура, можливо, смак та ін.

15. Спостерігати за пісочним годинником

16. Уявити навколо себе кулю, що сяє – намагатися утримувати у фокусі уваги

17. Повторювати зациклено одне й те саме слово в такт серцебиття

18. Перебирати намистини на намисті або чотках

19. Говорити слово на кожен крок (щастя, радість, здоров'я...)

20. Насипати в пляшку з водою глітери. Збовтати і спостерігати за тим, як вони
кружляють і осідають

21. Покласти руки на ребра та спостерігати за диханням – як рухаються ребра

(Я б дуже радила в книзі \enquote{нейробіологія психотерапії} Луіс Козоліно главу 15
Наляканий та тривожний мозок. Стор 345 Стрес та гіпокамп. (Автор посилається на
багато робот Роберта Сапольські), якщо потрібно, зроблю фото сторінок - вже
додала в коментарях)

Обіймаю, Родино ❤ як хочу Перемоги
