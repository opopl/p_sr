% vim: keymap=russian-jcukenwin
%%beginhead 
 
%%file 01_05_2021.fb.muraev_evgenij.1.pervomaj
%%parent 01_05_2021
 
%%url https://www.facebook.com/e.murayev/posts/841731383355397
 
%%author 
%%author_id 
%%author_url 
 
%%tags 
%%title 
 
%%endhead 
\subsection{Мир, труд и май Вам, друзья!}
\Purl{https://www.facebook.com/e.murayev/posts/841731383355397}

\ifcmt
  pic https://scontent-bos3-1.xx.fbcdn.net/v/t1.6435-9/180208613_841731313355404_911703014988425716_n.jpg?_nc_cat=110&ccb=1-3&_nc_sid=8bfeb9&_nc_ohc=wnXwgpG6XMgAX-9s34w&_nc_ht=scontent-bos3-1.xx&oh=3d22ebfe0720a04d21ebea6783e5df88&oe=60BAB188
\fi

Первомай - это совсем не пережиток «тоталитарного прошлого», а страшный сон и
«убытки» для всего транснационального Запада, который привык к дешевой рабочей
силе рабов, нелегалов, мигрантов к числу которых, к сожалению, относятся и наши
с вами сограждане, подавшиеся в соседние страны на неблагодарную работу, потому
что в родной стране им кроме «армовира» ничего предложить до сих пор не могут.

Нашим «западным партнерам» неприятно вспоминать, что день солидарности
трудящихся возник в Австралии больше века назад, тогда его поддержали в США,
Канаде и только затем в Российской империи. 

Тогда отстояли единую систему ценностей и во главу угла поставили право
человека на труд, безопасность, гарантированную медицинскую защиту,
санитарно-курортное оздоровление, детские лагеря и оплачиваемые отпуска и
больничные.

Спустя 100 лет  у нас зашкаливает безборотица, самые нищенские зарплаты в
Европе, неуважение к людям трудовых профессий и как закономерный итог - 10
миллионов заробитчан в РФ и ЕС.

Поэтому, сегодня говорить о «солидарности трудящихся» не моветон, а самое
время. Борьба за свои права как никогда актуальна. А главное – борьба за то,
чтобы жить в своём правовом государстве с нормальными, нашими ценностями, в
стране с позицией которой считаются на мировой арене, не грабят и не выкачивают
все ресурсы, держа на кредитной игле.

Так что, когда эти простые вещи дойдут до многих, обязательно врежем
первомайской демонстрацией по первоянварской ходе и покажем что в жизни на
самом деле важно и что объединяет, а не поляризирует наш народ. 

Мир, труд и май Вам, друзья!
