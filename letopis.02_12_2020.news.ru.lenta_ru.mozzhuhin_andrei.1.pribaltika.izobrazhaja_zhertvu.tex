% vim: keymap=russian-jcukenwin
%%beginhead 
 
%%file 02_12_2020.news.ru.lenta_ru.mozzhuhin_andrei.1.pribaltika.izobrazhaja_zhertvu
%%parent 02_12_2020.news.ru.lenta_ru.mozzhuhin_andrei.1.pribaltika
 
%%url 
 
%%author 
%%author_id 
%%author_url 
 
%%tags 
%%title 
 
%%endhead 

\subsubsection{Изображая жертву}

\lenta{Ваши коллеги-историки не так давно сформулировали три главные версии Второй
мировой войны. Первая, которая доминировала много лет: коалиция западных
демократий и коммунистического СССР победила абсолютное зло — гитлеровскую
Германию. Вторая: два жестоких тоталитарных режима боролись между собой; один
рухнул, а другой потом еще несколько десятков лет господствовал в Восточной
Европе. Третья: главное событие Второй мировой — холокост, осуществленный
нацистской Германией при активном содействии части населения оккупированных
стран Восточной Европы. Какая концепция вам ближе?}

Безусловно, ключевую роль в Победе сыграл СССР, что никак не умаляет роли его
союзников. Совершенно ясно, что условия для неизбежного столкновения Германии и
СССР создал «Мюнхенский сговор» 1938 года.

Давайте не будем забывать, что в течение нескольких месяцев 1939 года именно
западные демократии не хотели вступать в коалицию с Советским Союзом против
Германии. Как говорили тогда в Лондоне, между СССР и Британией есть подобие
Великой китайской стены. Поэтому Сталин в августе 1939 года пошел на договор с
Гитлером только после провала переговоров с англичанами и французами, и широкую
коалицию против нацистов с участием СССР удалось создать лишь в 1941 году.

Теперь что касается холокоста. Безусловно, он стал одним из важнейших и главных
событий Второй мировой войны. Одним из главных, но, конечно, не единственным.

Вторая мировая война содержит в себе переплетение самых разных военных,
политических, культурных и этических проблем, многие из которых до сих пор не
исследованы.

\ifcmt
pic https://icdn.lenta.ru/images/2018/01/31/21/20180131212821995/pic_0fb4b35975c7411d078558749518093c.jpg
caption Рабочие британского завода на изготовленных ими танках «Валентайн» для Красной армии с надписью «Вся помощь для России прямо сейчас». 22 сентября 1941г.  Фото: wio.ru
\fi

\lenta{Например?}

Преступления против человечности и тесно связанная с этим тема
коллаборационизма на территориях, оккупированных нацистской Германией. Это то,
что предпочитают замалчивать наши соседи по Восточной Европе, которым очень
нравится представлять себя жертвами двух одинаково бесчеловечных тоталитарных
режимов.

Неудивительно, что в истеблишменте балтийских стран сейчас очень популярен
такой подход к истории, когда собственные страдания (реальные или мнимые)
представляются как результат чьей-то чужой злой воли. Когда себя позиционируют
исключительно как жертву, а в соседе хотят видеть только палача. Это очень
удобно, поскольку подобный прием удовлетворяет вполне понятную потребность
забыть неприятные и позорные страницы своего прошлого, взвалив всю вину и
ответственность за них на кого-то другого.

\lenta{В книге вы это называете классическим советским подходом к истории, когда «в
массовое сознание привносится хорошо отретушированная, \enquote{идеологически
правильная} версия событий».}

Так оно и есть, потому что среди руководителей прибалтийских стран много людей
с советским партийным бэкграундом. Взять, например, предыдущего президента
Литвы Далю Грибаускайте.

Она сначала училась в Ленинградском университете, а потом преподавала в Высшей
партийной школе в Вильнюсе, диссертацию защитила в Академии народного хозяйства
при ЦК КПСС. Таких политиков, поначалу сделавших успешную карьеру в СССР, а
потом легко расставшихся и с прошлым и, очевидно, с прежними убеждениями,
немало в руководящих кругах балтийских государств.

\begin{leftbar}
	\bfseries
	Материалы по теме
00:08 — 18 июля 2016
Железная Даля
Две жизни президента Литвы Грибаускайте
\url{https://lenta.ru/articles/2016/07/18/dalya/}
\ifcmt
	ig https://icdn.lenta.ru/images/2016/07/17/15/20160717151143507/tabloid_095902ed0a3901c63406298a2859c235.jpg
	width 0.2
\fi
\end{leftbar}

И в определенном смысле политические установки довлеют над историческим
сознанием. Но я должна заметить, что в среде профессиональных историков Балтии
такая ситуация не в почете, и у них есть возможность от нее абстрагироваться.
Другой вопрос, что голос историка всегда менее слышим, чем голос правящего
политика.
