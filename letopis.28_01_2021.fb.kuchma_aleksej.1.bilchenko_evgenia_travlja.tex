% vim: keymap=russian-jcukenwin
%%beginhead 
 
%%file 28_01_2021.fb.kuchma_aleksej.1.bilchenko_evgenia_travlja
%%parent 28_01_2021
 
%%url https://www.facebook.com/permalink.php?story_fbid=878387439660374&id=100024673794166
 
%%author 
%%author_id kuchma_aleksej
%%author_url 
 
%%tags bilchenko_evgenia,kultura,kulturologia,obschestvo,protest,travlja,ukraina
%%title Кучма Олексій Андрійович. Виступаю проти цькування - професора Є.В.Більченко
 
%%endhead 
 
\subsection{Кучма Олексій Андрійович. Виступаю проти цькування - професора Є. В. Більченко}
\label{sec:28_01_2021.fb.kuchma_aleksej.1.bilchenko_evgenia_travlja}
\Purl{https://www.facebook.com/permalink.php?story_fbid=878387439660374&id=100024673794166}
\ifcmt
 author_begin
   author_id kuchma_aleksej
 author_end
\fi

Я – аспірант кафедри культурології та філософської антропології НПУ
ім.Драгоманова - Кучма Олексій Андрійович. Виступаю проти цькування - професора
Є.В.Більченко. Диспутом займатися не маю наміру, тому як не бачу гідних, що
хочуть побачити істину і вірять науковій термінології  та реалізації її моделей
- так що ніяких авторитетних джерел - нижче не буде. Розмова буде -
прямолінійна, від народного серця - як не дивно. Тому що наболіло від гніву, що
я ввібрав за останні 5-8 днів, в надії побачити те світло, що завжди знаходив в
віршах Євгенії, але знаходив там тільки загрози та образи, що вже дійшли до
сусідніх сторінок фб - родичів і друзів Євгенії. 

\subsubsection{Громадянська позиція}

Найжахливіший саме цей аспект - адже це особисте життя. Лізти в нього, як
мінімум - низько. Так що, Кесарю кесареве .. Наскільки мені відомо, для того що
б бути залученим до відповідальності, у злочині повинен бути склад з
об'єктивних і суб'єктивних ознак. Розбираючи контекст висунутих вимог до Є.В.,
ви не знайдете їх в розрізі її діяльності. Так їх і не може бути, адже немає
злочину, ніякого. Однак, є точна громадянська позиція. І точна позиція, від її
опонентів. Полеміку, в цьому форматі, я б і сам продовжив, але ж її хочуть
вигнати з посади професора, а не вирішують подискутувати у вільній чи
конкретній формі або намагаються підвести до кримінальної відповідальності. Бо
ніяких, антиправових дій, самою Є. В. Більченко, не було здійснено.

\subsubsection{Патріотизм}

В її підтримку скажу - Є. В. Більченко, ніколи не була судимою за антиукраїнські
погляди. Навпаки, що стосується питань української культури, то це один з
небагатьох, сучасних вчених, що має сміливість відстоювати незалежну,
українську позицію, серед американських глобалістів, що тримають на поводку,
гавкаючий С14 і їм подібних. Логіку вкл. Якщо вона (Є.В) дійсно підтримує
Росію, то навіщо тоді бути патріотом тут? Якщо вона продалася Медведчуку, то
чому у неї практично відсутнє особисте майно? Де ви бачили запроданців, без
грошей? Банальна логіка не дасть збрехати – Є.Б не купається в розкоші, вона
виживає в ЦІЙ країні. Як і багато хто з нас. Що це за країна, де професор
повинен просити гроші на лікування? Або змушений відректися від своїх
політичних і національних поглядів, тремтячи за посаду, стати пристосуванцем.
Євгенія не з таких. Це вкрай вольова і продуктивна людина, з якою краще не
воювати, нікому. Ви платите, податки за її «богемне» життя? - Ну, так
подивіться, своїми очами, чим ця людина живе. Робота. Трудоголізм до втрати
пульсу. У неї навіть дітей немає - діти це книги, що написані, разом з
вимотуючим графіком .. і вічною сесією, активною громадянською думкою, стійким
поетичним словом.

\subsubsection{Народ пригнічений і дивиться в підлогу}

Стерненко і КО, недарма вказують на її психічну неадекватність (в здоровому
тілі здоровий дух?), частиною правди є те, що вона хворіє своєю справою, до
божевілля. Але, читачу - подивіться в дзеркало. Кого ви там побачите? За вашим
міцним здоров'ям, все той же пустельний дух відплати. Око за око і весь світ
осліпне. За її хворобливою статурою, потужна постать, що зараз на межі
вимирання. Я все ще вірю, що кожен з нас робить помилки і Кожен несе за них
відповідальність – на мою думку, помилкою Є. В. - була та віра в народ, чиї серця
та тіла горіли на барикадах. Барикад більше немає, народ пригнічений і дивиться
в підлогу, залишилося тільки солодке томління закритих трун, наших співгромадян
і безсовісна поведінка тих, хто вижив. Війна - це театр абсурду, але також
війна - це конфлікт дрібних людських інтересів. Інтереси України і Росії.
Заходу і Сходу. Їх об'єднання або відторгнення. Діалог культур, що балансує на
межі існування. Об'єктивна сторона зрозуміла. Далі йде пропаганда. Мені шкода,
що Університет змушений відмиватися від того бруду, що несе пропаганда. Мені
соромно за тих, хто зараз роздмухує вогонь ненависті, направлений на своїх
братів – УКРАЇНЦІВ. Чуєте ви цього чи ні, але Є.В ваша сестра, яку ви хочете
знищити, чи одне це, не є доказом розгортання громадянського конфлікту ? 

\subsubsection{Громадяни ДНР і ЛНР}

Суб'єктивна сторона, каже мені наступне - інтелектуала, закидають лайном. Цю
частину, я пишу вже не як аспірант або дослідник, якоїсь певної теми. Тут немає
науки, тільки тикання пальцем і люди, що пишуть загрози і виливають відра
ненависті, порушують честь і гідність, обіцяні  третьою статтею  Конституції
України. І я вже не кажу про – 1,8,10,161 статті. \enquote{Конституція}, для них
можливо потрібна лише, що б підтерти рани на трупах в ДНР і ЛНР, більша
кількість громадян котрих, до певних подій, мали в 2013 році українське
громадянство. Громадяни Донецької і Луганської - в 2013ому, який паспорт мали?
Напевно, коли вмирали, у них перед смертю, за кілька секунд, Даждьбог знімав
баф на українське громадянство, і вони давали клятву пірнути в пекло, де до сих
пір вакантне місце Путіна? У бік цю ліберальну іронію.Я поняття не маю, що
діється в головах людей. Які клепки повернулися не туди і знати не хочу - я не
вірю своєму народу. І я істино боюсь писати правду, яку сам бачив, коли в 2013
році тендітність та яскравість кольорової революції зіткнулася з людськими
жертвами. Мені боляче, було дійсно боляче, за людей, що гинули та продовжують
гинути від рук один одного. Я не хочу більше війни. Ніякої. Але я точно
впевнений, що на цій землі, ніколи не було довгострокового миру. Буферна зона,
зустріч міжнародних інтересів. Це факт. На жаль.  

\subsubsection{Поклик Істини}

Для мене особисто, якщо в «незалежній» унітарній державі, де цензура заборонена
на рівні конституційного права, політичні гоніння на Більченко, за її власні
погляди, вирішаться на користь праворадикалів, стане тільки підтвердженням всіх
тих «шизоїдних» поглядів, що викладалися професором, на в протязі років. Ви
самі, шановні громадяни, підтверджуєте кожне її слово. І якщо хочете знати, то
враховуйте, що за нею стоїть не жалюгідне конспірологічне знання, а певне коло
авторів, визнаних в улюбленій вами Європі та США. Євгенія ніколи би не назвала
Україну американською колонією, без аргументації, але ви не хочете слухати її
аргументацію, бо чхати хотіли на Істину, ви в неї вже давно не вірите.  Так що,
їй це навіть лестить. Ми всі, йдемо на поклики пропаганди. 

\subsubsection{Будьте Людьми!}

Я буквально благаю всіх, хто писав люту дичину в комментарях, будьте ЛЮДЬМИ! Я
розумію, що багато з вас втратили близьких на сході, але не втрачайте здатність
критично мислити, спростовуйте опонентів, але через науку, через конфлікт
Слова, проте без вбивства та крові. Встановлюйте свої порядки, але не радійте
крові ваших ворогів, бо це та біль та страждання, що обов’язково повернуться до
нас самих. І не тільки за дхармічних принципів існування людства. Така наша
банальна біологія – ми смертні. І всі будемо в могилі.  І кожен рік, буде
повільною хворобою та смертю. Так не пришвидшуйте, наш смертельний марш.
Поважайте один одного. Є. В. Більченко заслуговує людської поваги, як і кожен із
людського роду, хоч часто і негідний по діянням своїм. 

\subsubsection{Прірва, що відділяє Людей від Звірів}

P.S. Накриває хвиля відчаю, за ту прірву, що віддаляє людей, від звірів. В такі
хвилини, я не знаходжу жодної причини, вірити тій сволоті, що розпалює
ворожнечу, із лав С14, чи то з агентів Еремля або ще якісь маркерні слова.. Я
просто вам всім НЕ вірю. Але якщо ви не хочете помічати, що МИ тут існуємо.
Люди, які не вірять офіційній українській пропаганді, які вільні самі обирати
свою ідентичність, які сумніваються, як заповідав їм Декарт і ці люди, для вас
нічого не значать, окрім слів ВАТА та Сепарюги, то для мене особисто - це буде
закликом, вважати, що правове поле, більш не актуальне для вирішення будь-яких
спільних питань. І повірите чи ні, але революція це саме ласкаве, що можна буде
зустріти в очах людей, у яких забирають Свободу самоідентифікації! Це не
погроза, це фатальна дійсність – як казав Оруєлл – «Якщо свобода, взагалі щось
означає, то це право казати іншим те, що вони не хочуть чути».
\verb|##наукабезцензури| та \verb|#стопцькуваннюБЖ|.

\ii{28_01_2021.fb.kuchma_aleksej.1.bilchenko_evgenia_travlja.cmt}


