% vim: keymap=russian-jcukenwin
%%beginhead 
 
%%file 01_01_2022.fb.fb_group.story_kiev_ua.2.zimnee.cmt
%%parent 01_01_2022.fb.fb_group.story_kiev_ua.2.zimnee
 
%%url 
 
%%author_id 
%%date 
 
%%tags 
%%title 
 
%%endhead 
\zzSecCmt

\begin{itemize} % {
\iusr{Inna Maistrouk}

Это воспоминания о беззаботном творческом и полном надежд детстве ..когда
веришь в чудеса.. во все времена были беды войны ..каждому поколению приходится
пройти свой путь ..но всегда была и радость и любовь и счастье ..даже в самые
тяжёлые времена ..иначе ьы жизнь на планете остановилась .как и сейчас ...мы
празднуем ..а кто -то защищает наш праздник ...и кого -то не дождутся уже в
новом 2022 году... один солдат уже присоединился к небесной сотне... в новом
году.....

\iusr{Елена Сидоренко}

Спасибо за воспоминания @igg{fbicon.heart.beating}. Вашей красавице маме
огромный привет и наилучшие пожелания. @igg{fbicon.heart.beating} 

\iusr{Irena Gorodetska}

Анатолий, очень люблю Ваши истории! И да, я совок-то почти не застала и, что
логично, не ностальгирую. Но очень люблю читать о своем прекрасном Городе.
Особенно когда меня еще и в проекции не было) С новым годом)

\begin{itemize} % {
\iusr{Анатолий Каганович}
\textbf{Irena Gorodetska} Спасибо! Все собираюсь написать о знаменитом еврейском тамаде - Городецком. Он не Ваш родственник?

\begin{itemize} % {
\iusr{Irena Gorodetska}
\textbf{Anatolii Kaganovych} , 

ооооо, а все может быть! Я помню Андреевский, откуда родом отец, по его
рассказам. Но не за всех родственников он знает. Есть там какие-то тайны. А
рассказать мне о них уже некому. Папа тогда еще мал был для всяких там
дворцовых тайн. Поздний ребенок после смерти старгей сестры в младенчестве.

\iusr{Анатолий Каганович}
Нет, я о другом Городецком - знаменитом тамаде на еврейских свадьбах 70-80х годов.

\iusr{Irena Gorodetska}
\textbf{Anatolii Kaganovych} , и все равно мне будет интересно) Спасибо, что пишете!

\iusr{Светлана Манилова}
\textbf{Анатолий}, Городецкий был тамадой на нашей свадьбе в 1985 году.

\iusr{Анатолий Каганович}
\textbf{Светлана Манилова} 

А свадьба была, наверное, в \enquote{Праге}? И 7-40 во всю на балконе в направлении
КГБ? Был на десятке таких свадеб. И почти всегда с Городецким. Надо заняться и
раскрыть эту тему. Имя его не вспомните?

\iusr{Светлана Манилова}
\textbf{Анатолий}. 

нет, это было в ресторане \enquote{Лісова пісня} на пл. Шевченко. Имя уточню у
мужа, забыла. @igg{fbicon.smile} 

\iusr{Светлана Манилова}
\textbf{Анатолий}, к сожалению, и он не помнит.

\end{itemize} % }

\iusr{Анатолий Каганович}
Жаль.

\end{itemize} % }

\iusr{Олександра Бондар}
Чудові спогади. Є що розповісти онукам.

\iusr{Виктория Киселева}

Октябрьский дворец это и мои шкльные годы, незабываемые воспоминания. Я
занималась в Хореографической студии, которой руководили В. Н Шехтман - Павлова и
З. М Лурье приблизительно с 62 года. Какие замечательные танцы они ставили. В те
времена нам все необходимое для занятий выдавали бесплатно хитоны, потом
купальники, пуанты и т.д. А какие восхитительные шили костюмы. А разве можно
сейчас представить, как все коллективы поплыли на корабле по Днепру и давали,
останавливаясь в городах, концерты! Ну и, конечно, Новогодние капустники для
участников коллективов ! Это все незабываемо!

\iusr{Erena Sevohena}

Мы с сестрой ходили на всё ёлки с атракционами. И подарки были прекрасными

\iusr{Раиса Карчевская}
\textbf{Erena Sevohena} я Я тоже ходила на все елки. Незабываемо

\iusr{Татьяна Ховрич}
Класс! Супер!

\iusr{Тамара Ар}

А не нужно так скрывать своих чувств! В \enquote{совке} было много хороших моментов!
Творчество, хорошее настроение, классных преподавателей и образование и многое
другое, никто не отменял! Хорошего Нового года в нашей славной стране!


\iusr{Сергій Криськов}

Согласен. Полностью отвергать \enquote{совок} невозможно так же, как стереть свою
память. Мы там росли и жили, там прошла наша молодость. Там было и хорошее и
плохое, забывать ничего нельзя, хотя бы потому, чтобы не было повторений жутких
моментов \enquote{совка}.

\begin{itemize} % {
\iusr{Тамара Ар}
\textbf{Сергій Криськов} 

если Вы, 1905 года рождения, то наверное, должны помнить, жуткие моменты,
организованные грузином Джугашвили- Сталиным,,,,,,, я родилась гораздо позже,
поэтому, не знаю жутких моментов

\iusr{Сергій Криськов}

Я 1948 г.р. и не намерен засорять тему спорами о политике.
\end{itemize} % }

\iusr{Gennady Henry Sergienko}

Спасибо за публикацию! Всегда приятно про Октябрьский прочесть, где мне
довелось работать, в том числе принимая участие в организации ёлок.


\iusr{Tatiyana Serbina}

В одном из коридоров Октябрского, стоял *робот*-отвечал на все заданнные ему
вопросы? Ну пришлось его озадачить... не ответил! @igg{fbicon.face.nerd} 

\begin{itemize} % {
\iusr{Tatiana Viktorova}
\textbf{Tatiyana Serbina} точно был!

\iusr{Анатолий Каганович}
\textbf{Tatiyana Serbina} Я пишу, как раз про этого Знайку.

\iusr{Tatiyana Serbina}
\textbf{Анатолий Каганович} не ответил какой Пролив который объединяет 2 моря 2 океана и разделяет 2 континента!

\iusr{Michael Yurovsky}
\textbf{Tatiyana Serbina}
Его потом и во Дворце пионеров выставляли.
\end{itemize} % }

\iusr{Леся Сагайдачная}
А я в этом же дворце уже в 80-х на ёлках трудилась)

\iusr{Владимир Новицкий}

Спасибо, Анатолий!! Хорошие, душевные воспоминания!! Всю Вашу семью поздравляем
с Новым Годом!!! Здоровья, улыбок, радости!!!

\iusr{Victoria Novikov}

Как же это здорово! Я была не из тех детей, которые любили елки, школу , дома
пионеров и тд, но мне так интересно читать Ваши рассказы. Это совсем другая
жизнь, другая сторона жизни. Вы потрясающе пишите. Просто наслаждение.
@igg{fbicon.hands.applause.yellow} 

\end{itemize} % }
