% vim: keymap=russian-jcukenwin
%%beginhead 
 
%%file 19_10_2021.fb.bilchenko_evgenia.3.moris_drjuon_kniga_musorka.cmt
%%parent 19_10_2021.fb.bilchenko_evgenia.3.moris_drjuon_kniga_musorka
 
%%url 
 
%%author_id 
%%date 
 
%%tags 
%%title 
 
%%endhead 
\subsubsection{Коментарі}

\begin{itemize} % {
\iusr{Оксана Пушкина}
может надо было спасти книгу?

\begin{itemize} % {
\iusr{Евгения Бильченко}
\textbf{Оксана Пушкина} может.

\iusr{Евгения Бильченко}
\textbf{Оксана Пушкина} 

но у меня на кону выбор: или заниматься спасением кошек, собак и макулатуры,
или тягаться с нацизмом. Нет, с малого действия здесь уже ничего не начнется, а
то, чем занимаюсь я, сделало меня полутрупом. Надеюсь, книгу выловила бабуля и
купила за нее молоко, это нужнее.

\end{itemize} % }

\iusr{Гамельнский Крысолов}
Книгу можно было спасти. А стране уже ничто не поможет.

\begin{itemize} % {
\iusr{Евгения Бильченко}
\textbf{Гамельнский Крысолов} В спасении бумаги нет смысла. Книга есть в ста версиях. Смысл есть в спасении Донбасса. Донбасс дороже бумаги. Логос дороже сшитых листов. Книга лично у меня есть, больше никому не нужна.

\iusr{Гамельнский Крысолов}
\textbf{Евгения Бильченко} Не успел изменить сообщение, ответили).
Следом картинка мелькнула в сознании - это наша страна в урну брошена, мимо проходят "сильные мира сего" и.. никому уже не нужна. Лишь "жучкам-точильщикам", утилизаторам.
\end{itemize} % }

\iusr{Таня Пономарева}

У нас тут некоторые находят в помойке семейные фотоархивы начала прошлого века,
рукописи научных трудов, наградные документы и кузнецовский фарфор с вензелями.
Так что Морису большой привет. Мир дал трещину, и вода хлещет в трюмы...

\iusr{Сергей Никонов}

Вопрос, кто заменит нынешних сильных, поменяется ли мир? Это по концепту. Да
многим нынешним место там.


\iusr{Андрей Богатырев}
"Они выбросили наши права в мусорное ведро!")))

\iusr{Vogel Paul}

Эх, я бы забрал. Возьму в хорошие руки книги любые, присылайте.) Дрюон это тот
что Проклятые короли? Тогда вообще странно что выкинули это же Игра Престолов
только без магии.


\end{itemize} % }
