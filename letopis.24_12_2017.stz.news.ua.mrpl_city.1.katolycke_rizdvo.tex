% vim: keymap=russian-jcukenwin
%%beginhead 
 
%%file 24_12_2017.stz.news.ua.mrpl_city.1.katolycke_rizdvo
%%parent 24_12_2017
 
%%url https://mrpl.city/blogs/view/rizdvo-hristove
 
%%author_id demidko_olga.mariupol,news.ua.mrpl_city
%%date 
 
%%tags 
%%title Католицьке Різдво
 
%%endhead 
 
\subsection{Католицьке Різдво}
\label{sec:24_12_2017.stz.news.ua.mrpl_city.1.katolycke_rizdvo}
 
\Purl{https://mrpl.city/blogs/view/rizdvo-hristove}
\ifcmt
 author_begin
   author_id demidko_olga.mariupol,news.ua.mrpl_city
 author_end
\fi

\textbf{Різдво Христове} належить до великих християнських свят, які церква
відзначає особливо урочисто. За біблійними свідченнями, цього дня народився Син
Божий – Ісус Христос, якому люди поклоняються вже два тисячоліття. У X столітті
християнство було запроваджено на Русі, й відтоді свято Різдва Христового стало
невіддільною частиною нашої культури. Разом із вірою Христовою в Україні та за
її межами набули поширення такі народні традиції, як колядування та щедрування.
Коляда – це гімн Христу, оспівування його народження. У щедрівках славлять
природу, закликають весну. У день Народження Сина Божого весь світ славить
свого спасителя: люди прикрашають свої будинки святковими ялинками, які
символізують євангельське древо, запалюють свічки, ходять на церковні
богослужіння.

Українці святкують два Різдва: 25 грудня за григоріанським календарем святкують
католицьке Різдво, а в ніч з 6 на 7 січня відзначають православне Різдво за
юліанським календарем. Депутати Верховної Ради підтримали рішення зробити
католицьке Різдво вихідним днем.

За підрахунками деяких експертів, 75\% православних церков у світі відзначає
народження Христа саме 25 грудня і лише 25\% продовжує відзначати 7 січня. На
думку багатьох вчених, юліанський календар довів свою недосконалість. Так,
менше ніж через 100 років (а точніше, 2101 року) Різдво за юліанським
календарем будуть відзначати вже 8 січня, а не 7 січня. Тож календар від Юлія
Цезаря, яким Європа користувалася кілька століть, не є точним та потребує змін.

Папа Римський Григорій ХIII у жовтні 1582 року запровадив новий календар,
названий на його честь григоріанським (згідно з ним, після 4 жовтня одразу
настало 15, тобто була усунена різниця в десять днів). Цей календар не був
оригінальним, він лише виправляв, тобто уточнював старий календар і в такий
спосіб наближав календарний рік до року астрономічного.

Проте, враховуючи, що церква в Україні розділена і що календарне питання
викликає болісні дискусії в церкві, православні українці й надалі будуть
святкувати Різдво 7 січня, що вже стало споконвічною народною традицією.

Цікаво, як маріупольці святкували Різдво Христове у XIX – на початку XX ст.
Мешканці міста вирізнялися своєю релігійністю, тому до церковних свят
готувалися заздалегідь. Різдво вважалося одним з найбільш очікуваних церковних
свят. Йому передував 40-денний різдвяний піст. Переддень, або \enquote{день надвечір'я}
Різдва, проводився в суворому пості й називався святвечором, оскільки в цей
день за церковним статутом вживали в їжу сушені хлібні зерна, змочені водою.
Пост тривав до вечірньої зорі. А потім починалася церковна служба в
православних храмах міста. Парафіяни поспішали кожен у свою шановану церкву.
Найбільш урочисто проходило Різдвяне богослужіння у величному Харлампіївському
соборі Маріуполя XIX століття. Можна уявити, що вхід у храм був усіяний
натовпами метушливих і ошатних маріупольців. Бідні селянські жінки в простих
ситцевих хустках стояли поруч із заможними міщанками та купчихами, обвішаними
золотом. Голосні міської думи та інші чиновники в натовпі майстрових і візників
нічим не видавали свого занепокоєння від такого сусідства. У церкві, як і перед
смертю, всі завжди рівні.

Маріупольці всіх станів і вікових груп збиралися в єдиний натовп на святкове
богослужіння. На устах у них посмішки, в руках – свічки; розум і почуття
поглинені тим, що відбувається. Всі сміються і плачуть від радості. Свято
Різдва у всьому світі заведено відзначати у родинному колі. Не були винятком і
маріупольці, адже після богослужіння містяни поспішали додому.

У роки війни маріупольці не забували про військових і намагалися приготувати
для них подарунки та відправити на фронт. Так, наприклад, у грудні 1915 р.
Маріупольський міський комітет Всеросійського союзу міст закликав надіслати
подарунки фронтовикам до свята Різдва Христового. Зробити посильний внесок
містяни могли коштами або принісши подарунок в окремому пакеті, куди
дарувальнику пропонувалося вкласти листівку для відповіді. Що могло порадувати
бійця на фронті напередодні свят? Чоботи, панчохи, тепла білизна, рукавички,
жилети, шарфи, тютюн, цигарки, сірники, мило, свічки, цукор, чай, поштовий
папір, олівці, конверти та ін. Такий перелік допустимих речей пропонував
комітет. Подарунки приймалися до 10 грудня (адже вони повинні були бути
доставлені до Різдва). У день Різдва в місті та на заводах були відкриті
кінематографи. З дозволу Катеринославського губернатора і Катеринославської
духовної консисторії половину всього валового збору кінематографи перерахували
на користь рухомого госпіталю від населення всієї Катеринославської губернії.

Різдво Христове – довгоочікуване сімейне свято, коли за одним столом збирається
велика родина, щоб привітати найбільш близьких людей.

Я також вітаю вас і бажаю від усього серця щасливих, довгих років в оточенні
любові й удачі! Нехай здійсняться всі ваші заповітні мрії, а найскладніші
завдання легко піддаються вашій чарівності та працьовитості! Зі святим Різдвом!
