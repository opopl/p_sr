% vim: keymap=russian-jcukenwin
%%beginhead 
 
%%file 24_08_2021.fb.mahin_aleksandr.1.tablica_nezalezhnist
%%parent 24_08_2021
 
%%url https://www.facebook.com/amakhin/posts/10224060613815481
 
%%author_id mahin_aleksandr
%%date 
 
%%tags nezalezhnist,statistics,tablica,ukraina
%%title Решил к юбилею запилить табличку
 
%%endhead 
 
\subsection{Решил к юбилею запилить табличку}
\label{sec:24_08_2021.fb.mahin_aleksandr.1.tablica_nezalezhnist}
 
\Purl{https://www.facebook.com/amakhin/posts/10224060613815481}
\ifcmt
 author_begin
   author_id mahin_aleksandr
 author_end
\fi

Решил к юбилею запилить табличку. В качестве главного сравнительного показателя
стран мною был взят ВВП - валовый внутренний продукт, рассчитанный по паритету
покупательной способности. Английский эквивалент термина - gross domestic
product at purchasing power parity.

Период мною взят 1993-2019 – с промежуточными итогами в 2003 и 2013 годах. Не
от самого 1991 года, и не до 2021, который ещё не закончился. Даже по 2020 нет
всех данных, но он был и не показателен из-за коронавируса. 

2003 год знаменателен тем, что это ещё до Оранжевой революции, а 2013 – это
последний год Украины образца 1991 года. 

\begin{itemize}
  \item Колонка 1 – название страны
  \item Колонка 2 – величина ВВП по ППП в соответствующем году
  \item Колонка 3 – место страны в общемировом рейтинге по этому году
\end{itemize}

Последняя колонка – Прирост ВВП по ППП в 2019 году по отношению к 1993 году

\ifcmt
  pic https://scontent-frt3-1.xx.fbcdn.net/v/t39.30808-6/240212785_10224060612575450_7910470858660736862_n.jpg?_nc_cat=102&_nc_rgb565=1&ccb=1-5&_nc_sid=730e14&_nc_ohc=CC1a4fhHxiMAX9C2SyK&_nc_ht=scontent-frt3-1.xx&oh=c53dbb2191152f9f954f0232b2bdac8c&oe=6139A3CA
	width 0.9
\fi

Некоторые выводы:

Китай – чемпион.

Россия к 2013 году смогла выйти на 5 место, но сейчас под санкциями съехала на
шестое, однако если сравнивать её с соседями, она их неплохо догнала. Например,
в 2003 году разрыв с Японией был почти три раза, а в 2019 она отстает всего на
четверть. 

Самый динамичный игрок в Европе - это Польша. Она, как и Россия, передвинулась
на 5 мест вверх (с 25-го на 20-е). 28 лет назад Польша была величиной с одну
шестую Франции, а сейчас – менее чем треть. Это большой рост.

Страны "Старой Европы" (Британия, Франция, Италия) постепенно ухудшают своё положение.

Индия обогнала Японию и Германию.

"Диктаторская" Белоруссия под железной пятой Батьки, напротив, своё положение
не сдала - несмотря на все вывихи, девальвации и проблемы в области финансов. 

И наконец про падение роли и значения Украины за 28 прошедших лет. Она съехала
в мировой табели о рангах по ВВП на 15 мест вниз (с 23-го на 38-е),
относительно России тоже - была больше, чем 1/3 своей соседки, а сейчас едва
переваливает за 13\% от неё - то есть меньше 1/7. То же самое - и относительно
Польши. Если 28 лет назад Украина экономически была больше Польши на 14\%, 291
против 251 млрд.\$, то уже в 2003-м - Польша больше неё в 1,7 раза. А сейчас - в
2,3 раза.

\ii{24_08_2021.fb.mahin_aleksandr.1.tablica_nezalezhnist.cmt}
