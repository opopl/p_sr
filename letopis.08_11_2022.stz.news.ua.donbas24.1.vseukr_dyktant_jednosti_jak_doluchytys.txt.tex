% vim: keymap=russian-jcukenwin
%%beginhead 
 
%%file 08_11_2022.stz.news.ua.donbas24.1.vseukr_dyktant_jednosti_jak_doluchytys.txt
%%parent 08_11_2022.stz.news.ua.donbas24.1.vseukr_dyktant_jednosti_jak_doluchytys
 
%%url 
 
%%author_id 
%%date 
 
%%tags 
%%title 
 
%%endhead 

Ольга Демідко (Маріуполь)
Україна,Мова,Українська мова,День української писемності та мови,Радіодиктант національної єдності,Свято,date.08_11_2022
08_11_2022.olga_demidko.donbas24.vseukr_dyktant_jednosti_jak_doluchytys

Всеукраїнський диктант національної єдності — як долучитися

9 листопада всі охочі українці зможуть взяти участь у традиційному
всеукраїнському Радіодиктанті

Цієї середи, 9 листопада, у День української писемності та мови, в Україні
відбудеться традиційний Радіодиктант національної єдності-2022. Свято розвитку
державної мови було започатковане 9 листопада 1997 року Указом Президента
України Леоніда Кучми. Дата обрана не просто так — 9 листопада українці також
відзначають День вшанування Нестора Літописця — послідовника творців
слов'янської писемності Кирила і Мефодія. Про те, як приєднатися до
всеукраїнського диктанту та перевірити його, розповіло Міністертво культури та
інформаційної політики України.

«Мета радіодиктанту — об'єднати людей навколо України в усьому світі, а не
перевіряти грамотність учасників всеукраїнського флешмобу», — наголосили в
Міністерстві.

Читайте також: Мовні норми вступають в силу — яких правил треба дотримуватися

Авторка цьогорічного тексту — українська письменниця і режисерка, членкиня
Українського ПЕНу Ірина Цілик. Читатиме текст народна артистка, Герой України
Ада Роговцева.

Як взяти участь?

Долучитися до написання всеукраїнського диктанту можуть усі охочі. Для цього потрібно 9 листопада об 11:00 за київським часом увімкнути:

Українське Радіо або Радіо Культура;
телеканал Суспільне Культура;
діджитал-платформи Суспільного: фейсбук-сторінку Суспільного та ютуб-канали Українського Радіо та Суспільне Новини;
додаток suspilne.radio;
додаток «Дія».

"Написання Радіодиктанту-2022 адаптовано до умов воєнного часу. Також у різних
країнах світу будуть облаштовані студії для написання диктанту для вимушених
переселенців та українців, які живуть за кордоном", — додали у Верховній Раді.

Читайте також: Школи, виші та садочки України переходять на дистанційку: деталі

Як можна перевірити Радіодиктант національної єдності-2022?

10 листопада буде оприлюднено текст диктанту, тому якщо ви хочете, щоб ваш диктант перевірили фахівці, ви можете:

Надіслати паперового листа за адресою: 1 001, м. Київ, вул. Хрещатик, 26.  Важливо, щоб гриф дати надсилання був не пізніше ніж 10 листопада.
Сфотографувати/відсканувати написаний текст (у форматі .jpg, .png, .jpeg, .tiff, .pdf) та надіслати його до 11:00 10 листопада на адресу: rd@suspilne.media або через електронну форму для отримання робіт. Її оприлюднять на сайті ukr. radio 9 листопада.

Раніше Донбас24 розповідав, що в Україні запускають Google Знання.

Ще більше новин та найактуальніша інформація про Донецьку та Луганську області в нашому телеграм-каналі Донбас24.

ФОТО: з відкритих джерел.
