% vim: keymap=russian-jcukenwin
%%beginhead 
 
%%file slova.soldat
%%parent slova
 
%%url 
 
%%author 
%%author_id 
%%author_url 
 
%%tags 
%%title 
 
%%endhead 
\chapter{Солдат}

%%%cit
%%%cit_head
%%%cit_pic
\ifcmt
  pic https://gdb.rferl.org/18F77942-07F8-4AB7-BD20-0692B591AF5F_w1023_r0_s.jpg
	width 0.4
	caption Радянські військовополонені під Харковом. Літо 1942 року. Німецьке фото
\fi
%%%cit_text
Залишився лейтенант Чорниш в оточенні, зі своїми \emph{солдатами}. До вечора \emph{солдати}
слухали свого командира, зайняли кругову оборону і відстрілювалися від німців,
а як настала ніч, зникла й дисципліна, бо кожний зрозумів, що ніщо його не
врятує, тільки він сам. \emph{Солдати} перестали слухатися командирських наказів, і
кожний рятував себе як міг. З ранком Чорниш побачив, що залишився він сам.
Сподіваючись вийти з оточення, дійшов він до Дону, але на біду – плавати не
вмів, довелося йти берегом, усі мости були зруйновані, і скрізь уже були німці
%%%cit_comment
%%%cit_title
\citTitle{«Заборонені» спогади з фронтів війни Німеччини і СРСР 1941–1945 років}, 
Ігор Роздобудько, www.radiosvoboda.org, 08.07.2021
%%%endcit
