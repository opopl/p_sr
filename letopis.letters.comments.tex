% vim: keymap=russian-jcukenwin
%%beginhead 
 
%%file letters.comments
%%parent letters
 
%%url 
 
%%author_id 
%%date 
 
%%tags 
%%title 
 
%%endhead 

%https://www.facebook.com/urtab/posts/pfbid024GtGyWgTauBg4vAk56QR65J5RHCqancaYFYfhHNEaFfpVNeHbsWwVL6uBV77X4Xl

Дуже цікаво пишете, дякую. Знаєте, що тут згадалось. Свого часу був знятий
серіал Геній - американці знімали (National Geographic), так от, перша частина
цього серіалу про Альберта Айнштайна - фізика-теоретика, творця спеціальної та
загальної теорії відносності та також автора листа президенту США Франкліну
Рузвельту, з якого почалось створення атомної бомби в Лос-Аламосі, що в
результаті призвело до атомних бомбардувань Хіросіми і Нагасакі... ну а друга
частина - про Пікассо. Кожна частина присвячена якійсь одній людині, і в кожній
частині багато серій. Короче, суть у чому. Творці цього фільму явно не
намагались зробити із Айнштайна святого, а показали його життя у всіх його
суперечностях, де Айнштайн - крім того, що показані всі його творчі натхненні
пошуки від школяра та студента аж до створення спеціальної та загальної теорії
відносності - його найвищого творчого та наукового досягнення, - виходить як і
дуже добра, неймовірна чуйна людина, так і людина, досить таки неоднозначна з
точки зору моралі та його відносин із власною сім'єю (особливо з точки зору
його відносин з Мілевою Маріч) Короче, вийшов насправді дуже життєвий фільм,
який я із задоволенням передивлявся багато раз. І сподіваюсь, колись подібний
фільм знімуть і про нашого Генія Тараса, і краще буде, якщо це зроблять самі
українці, врешті решт, не чекаючи, поки хтось інший не зробить це за них.


%14:13:17 21-03-23
а, ну добре... LaTeX (TeX) - це просто технологія програмування документів,
де-факто стандарт у науковому світі, яку використовують фізики та математики та
вчені інших спеціальностей для запису та публікації своїх робіт. Насправді це
дуже потужна штука, за допомогою якої можна підготувати публікації або книжки
будь-якого рівня складності... А я її використовую для архівації та збереження
різноманітних публікацій, як про війну, так і не тільки. Останнім часом активно
займаюсь Маріуполем, хоча я сам киянин. Як це виглядає, можете подивитись ось
тут https://t.me/kyiv_fortress_1 або
https://archive.org/details/@kyiv_chronicler

%11:14:24 22-03-23
%unsent
а навпаки. Треба щоби ті, що зараз в тилу, надихались енергією Миру, Світла, і
передавали її потім тим, хто потребує. хто дійсно на війні - воїнам,
постраждалим, тим, хто втратив житло і близьких... Я так вважаю. Бо ж є такий
парадокс, що ті хто на війні і хто кожний день ризикує життям, дуже часто духом
сильніші, ніж ті, хто в тилу, і кому нічого безпосередньо не загрожує
(звичайно, окрім ракети з чергової ракетної атаки), але весь час живе новинами
з війни... 



%11:18:51 04-04-23
% https://www.facebook.com/groups/kyiv.in.photo/permalink/5987520858010339/?comment_id=5992460377516387&notif_id=1680596268113248&notif_t=feedback_reaction_generic&ref=notif
Так, Ваші, та фото та історії і спогади інших маріупольців то є неоціненний,
найдорожчий скарб!! І не тільки для маріупольців, а і для всієї України! І на
основі цього скарбу у свій час постане Новий, Відроджений Маріуполь! Так само,
як наприклад, на основі фото та інших історичних даних був відтворений заново
Михайлівський Золотоверхий Монастир в Києві, знищений в 1937 році радянською
владою, так і буде відтворено все, что було зруйновано, я в цьому навіть не
сумніваюсь! ...Так, в цих фото, є велика Сила, Духовна Сила! А посилання (яке
Ви звісно бачили) я просто додав, щоби інші люди також заходили в группу )

% https://www.facebook.com/photoUkraine/posts/pfbid0W6YZgNAzJmr3ZtJoecjmWMiuFUcpNNgNF3MdPsQtVhDYXRsyfUgpPF3D8FoAjpzXl
а я післявчора в метро знаєте, що.. Історія маленька трапилась... їду в вагоні,
а тут хлопець за метр чи два метри біля мене стоїть, а в телефоні у нього
шахи!! Якась задачка напевне була або може книжку шахову читав, хто його зна. А
я в шахи теж граю, дуже цю гру люблю, і досить непогано граю, до речі... Так
от. Поки їхав від Либідської до Поштової, все намагався очима вгледіти, а що ж
він там розгадує ))) Може це і виглядає трохи негарно, але... Ніяк не міг
втриматись, бо цікаво було... але на жаль, не зміг побачити, бо багато людей і
важко побачити було... Так і залишилось це загадкою для мене, а що ж там за
позиція була. Але нічого страшного, в принципі! )))
