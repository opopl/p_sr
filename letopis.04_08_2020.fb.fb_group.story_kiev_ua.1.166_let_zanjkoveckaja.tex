% vim: keymap=russian-jcukenwin
%%beginhead 
 
%%file 04_08_2020.fb.fb_group.story_kiev_ua.1.166_let_zanjkoveckaja
%%parent 04_08_2020
 
%%url https://www.facebook.com/groups/story.kiev.ua/posts/1420158604847603/
 
%%author_id fb_group.story_kiev_ua,petrova_irina.kiev
%%date 
 
%%tags kultura,teatr,ukraina,zanjkoveckaja_maria.ukr.aktrisa
%%title 166 років від дня народження Марії Заньковецької
 
%%endhead 
 
\subsection{166 років від дня народження Марії Заньковецької}
\label{sec:04_08_2020.fb.fb_group.story_kiev_ua.1.166_let_zanjkoveckaja}
 
\Purl{https://www.facebook.com/groups/story.kiev.ua/posts/1420158604847603/}
\ifcmt
 author_begin
   author_id fb_group.story_kiev_ua,petrova_irina.kiev
 author_end
\fi

Четвертого серпня 2020 року минає 166 років від дня народження славетної
української актриси, корифея української сцени, першої Народної артистки
України, однієї з найяскравіших драматичних актрис у театральному просторі
кінця XIX – початку XX століть – Марії Заньковецької.

\ifcmt
  tab_begin cols=3

     pic https://scontent-frx5-1.xx.fbcdn.net/v/t1.6435-9/117133009_3475716295795286_2751342003241032985_n.jpg?_nc_cat=111&ccb=1-5&_nc_sid=b9115d&_nc_ohc=tRI486eb37cAX8WaRwH&_nc_ht=scontent-frx5-1.xx&oh=6e679c89d87fbae1660ea7bf6417f03d&oe=61B6DA2A

     pic https://scontent-frx5-1.xx.fbcdn.net/v/t1.6435-9/117159019_3475716449128604_8162107353984899385_n.jpg?_nc_cat=100&ccb=1-5&_nc_sid=b9115d&_nc_ohc=g3V8E7f4YEsAX8lmVdN&_nc_ht=scontent-frx5-1.xx&oh=a62cc96b68b7c827f3566dcca9e39173&oe=61B5B887

		 pic https://scontent-frx5-2.xx.fbcdn.net/v/t1.6435-9/116712616_3475718985795017_2295892834618807562_n.jpg?_nc_cat=109&ccb=1-5&_nc_sid=b9115d&_nc_ohc=Sw7S21JV_BgAX-eFgkX&_nc_oc=AQnQY38GB_joIhUtqmTLZbGJSXQ900rtXV2T_489_QIHgVx549jWbONmtovKrAciHMU&_nc_ht=scontent-frx5-2.xx&oh=2a5e1ec1d56b91217f0f7066e19d1e36&oe=61B7BF0D

  tab_end
\fi

Сучасники вважали, що її могутній талант зробив би честь найкращій європейській
сцені. Їй пропонували служити в імператорських і приватних театрах Петербурга й
Москви, але актриса залишилася вірною рідній українській сцені.

«Феномен Заньковецької» полягає в тому, що гучну славу здобула актриса
українського театру, якому в Російській імперії XIX століття відмовлялося у
праві на існування, а коли й дозволялись вистави українською мовою, то з
величезними цензурними обмеженнями.

\ifcmt
  tab_begin cols=2

     pic https://scontent-frx5-1.xx.fbcdn.net/v/t1.6435-9/116789671_3475719425794973_819463484781173083_n.jpg?_nc_cat=110&ccb=1-5&_nc_sid=b9115d&_nc_ohc=TwTHMHNwy6kAX_C1zCV&_nc_ht=scontent-frx5-1.xx&oh=18504b22daed41191a8c7fddcafcedb5&oe=61B75621

     pic https://scontent-frt3-1.xx.fbcdn.net/v/t1.6435-9/117255585_3475719162461666_8576861019331141575_n.jpg?_nc_cat=107&ccb=1-5&_nc_sid=b9115d&_nc_ohc=sVBmaq5VGhQAX_qDVxf&_nc_ht=scontent-frt3-1.xx&oh=fd30862e1c8684a60ffd81db726d2b84&oe=61B521A6

  tab_end
\fi

Творчості Марії Заньковецької були притаманні поетичність і гострота почуттів,
той особливий український національний код, який мали Тарас Шевченко, Микола
Гоголь.

Історія її життя – це історія любові, історія єдиного на все життя драматичного
кохання до Миколи Садовського.

Марія Костянтинівна на сцені, в образах своїх героїнь, жила Любов’ю і
«закликала у суфлери почуття». Це допомагало їй відчувати своїх героїнь
емоційно сильно, сприймати їхні біль, страждання, радощі, як власні.
Заньковецька грала на сцені українських жінок. Образи, які вона створила,
порівнювали з найкращими образами світової класики. У цій жінці з безоднею в
очах, з мелодійним голосом, надзвичайною пластичністю рухів, легкою ходою,
сучасники відчували величезну силу душевних переживань. Вона прагнула до
самореалізації і заряджала своїми емоціями всіх навколо.

Її друзями були відомі українські письменники і діячі культури того часу, як
старшого покоління, так і молодшого. Драматурги писали для неї п’єси, поети
присвячували вірші. Талант Заньковецької, її яскрава особистість вражали
А.Чехова, М. Грушевського, С. Петлюру, Л. Толстого, П. Чайковського, Лесю
Українку, М. Рильського, І. Рєпіна, І. Буніна та багатьох інших.

Сорок років життя віддала артистка українській сцені, відродженню й утвердженню
національного театру й українського слова.

Її називали «зорею українського театру», ювілеї її творчості перетворювались на
великі свята культури і театру.

Даниною пам’яті актриси є Музей Марії Заньковецької у Києві, який існує понад
сорок років. Даниною її пам’яті став і Фестиваль «Марія», що народився у Києві
чотирнадцять років тому, ініціатором якого є народна артистка України, лауреат
Національної премії України імені Тараса Шевченка, народна артистка України,
лауреат Національної премії України імені Тараса Шевченка, актриса
Національного театру ім. Івана Франка Лариса Кадирова, яка понад 600 разів
зіграла на театральному кону роль Марії Заньковецької.

З Днем народження, Маріє!

У самому серці Києва є вулиця імені Марії Заньковецької. (до речі - моя рідна)

\ii{04_08_2020.fb.fb_group.story_kiev_ua.1.166_let_zanjkoveckaja.cmt}
