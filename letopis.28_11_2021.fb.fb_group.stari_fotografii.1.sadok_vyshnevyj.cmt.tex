% vim: keymap=russian-jcukenwin
%%beginhead 
 
%%file 28_11_2021.fb.fb_group.stari_fotografii.1.sadok_vyshnevyj.cmt
%%parent 28_11_2021.fb.fb_group.stari_fotografii.1.sadok_vyshnevyj
 
%%url 
 
%%author_id 
%%date 
 
%%tags 
%%title 
 
%%endhead 
\subsubsection{Коментарі}

\begin{itemize} % {
\iusr{Nadiya M Shana}

Все знакомо! Спасибо!

\iusr{Елена Сидоренко}
Да, это хорошо, когда есть родственники или друзья в деревне! @igg{fbicon.heart.beating} 

\iusr{Вера Шейка}
Дійсно безцінні фото.

\iusr{Марат Вазлин}

До войны в Киеве асфальта небыло. Булыжные мостовые а где и вообще грунтовка. А
воздух тогда что в сёлах что в Киеве был практически одинаковый. Единственное
что шуму было много.

\begin{itemize} % {
\iusr{Алексей Никитин}
\textbf{Марат Вазлин} В Киеве начали асфальтировать улицы с конца 1870-х.

\iusr{Марат Вазлин}
\textbf{Алексей Никитин} да да и через 100лет заасфальтировали Крещатик  @igg{fbicon.face.tears.of.joy}. Я родился в конце 50х, у Нас асфальт был только на тротуарах а в 60х начали ложить асфальт на проезжие части.

\iusr{Алексей Никитин}
\textbf{Марат Вазлин} Крещатик заасфальтировали в 1936, когда убрали трамвай.

\iusr{Марат Вазлин}
\textbf{Алексей Никитин} а потом снова положили булыжник  @igg{fbicon.face.grinning.squinting} . Я помню в 60х как убирали булыжник на Крещатике и дожили асфальт, начиная с пл ВЛКСМ до ул Саксаганского. Остальная часть улицы Красноармейская оставалась булыжник.
\end{itemize} % }

\iusr{Мария Иванова}

Какие чудесные фотографии!

У меня село Келеберда, в которую меня с бабушкой отправили на лето, осталась
лишь в памяти. Дом с печкой, дедушка-хозяин со сковородкой, на которой шипят
шкварки, а мы их хватаем, обжигаясь,колодец, закрытый досками, чтобы мы, дети
туда не свалились, второй колодец - журавль, вечерние рассказы о волках,
которые зимой подходят к селу, поля с горохом, речка...

\begin{itemize} % {
\iusr{Галя Твердохлебова}
\textbf{Мария Иванова} 

Спасибо за напоминание! Да, ещё был колодец с \enquote{журавлём} прямо на улице. Я
сейчас на пенсии живу в селе у Десны. У нас во дворе есть колодец, но ведро
опускается на цепи коловоротом. А \enquote{журавля} я больше нигде не видела.

\end{itemize} % }

\iusr{Наталия Бондарева}
Я вам завидую... т. е. вашему детству...

\iusr{Мигловец Людмила}
Какие чудесные фото,

\iusr{Виктор Тимченко}

Кийлов — село, входит в Бориспольский район Киевской области Украины.
Расположено в центральной части Бориспольского района на берегу Днепра. На
севере граничит с селом Воронков. Население по переписи 2001 года составляло
556 человек.

\begin{itemize} % {
\iusr{Галя Твердохлебова}
\textbf{Виктор Тимченко} Да, наша хозяйка ездила на базар в Вороньков. Но тогда до Воронькова было далековато. Ехали на грузовике в кузове. Я там лет 20 не была.
\end{itemize} % }

\iusr{Людмила Гриб-Морванюк}

Яке багатство Ви маєте! Щиро заздрю і про такі світлини можу лише мріяти.

\begin{itemize} % {
\iusr{Галя Твердохлебова}
\textbf{Людмила Гриб-Морванюк} 

Я зараз, як пішла на пенсію, живу з чоловіком в селі. В Киів їду рази два на
місяць. Все життя в столиці прожила, а тепер місто не люблю. Мені тут
комфортно. І подруга моя купила будинок поруч. Ми дружимо родинами.

Село наше на березі Десни. Поряд ліс. Охоронна зона Залісся. У нас дуже гарно
тут!


\iusr{Людмила Гриб-Морванюк}
\textbf{Galya Tverdohlebova} Щиро радію за Вас!!!
\end{itemize} % }

\iusr{Лидия Гончарук}

Уникальный архив. У Вас было прекрасное детство. У меня то же остались
воспоминания отдыха в селе Прохоровка

\iusr{Ирина Иванченко}
Мне очень понравились ваши фотографии и рассказ, спасибо.

\iusr{Алина Маркина Сыченко}
Ваши воспоминания так же бесценны, как и фотографии!

\iusr{Галя Твердохлебова}
\textbf{Алина Маркина} Сыченко спасибо @igg{fbicon.hearts.revolving} 

\iusr{Vlada Marek}
Чудесные фото @igg{fbicon.heart.eyes}  @igg{fbicon.thumb.up.yellow} 


\end{itemize} % }
