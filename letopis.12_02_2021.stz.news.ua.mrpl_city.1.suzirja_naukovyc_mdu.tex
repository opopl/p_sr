% vim: keymap=russian-jcukenwin
%%beginhead 
 
%%file 12_02_2021.stz.news.ua.mrpl_city.1.suzirja_naukovyc_mdu
%%parent 12_02_2021
 
%%url https://mrpl.city/blogs/view/suzirya-naukovits-mdu-z-nagodi-mizhnarodnogo-dnya-zhinok-i-divchat-u-nautsi
 
%%author_id demidko_olga.mariupol,news.ua.mrpl_city
%%date 
 
%%tags 
%%title Сузір'я науковиць МДУ: з нагоди міжнародного Дня жінок і дівчат у науці
 
%%endhead 
 
\subsection{Сузір'я науковиць МДУ: з нагоди міжнародного Дня жінок і дівчат у науці}
\label{sec:12_02_2021.stz.news.ua.mrpl_city.1.suzirja_naukovyc_mdu}
 
\Purl{https://mrpl.city/blogs/view/suzirya-naukovits-mdu-z-nagodi-mizhnarodnogo-dnya-zhinok-i-divchat-u-nautsi}
\ifcmt
 author_begin
   author_id demidko_olga.mariupol,news.ua.mrpl_city
 author_end
\fi

Щороку, починаючи з 2016-го, 11 лютого, жінки-науковці відзначають своє
професійне свято. Особисто для мене цей день дуже особливий, адже протягом
навчання в університеті та аспірантурі я дуже щільно пов'язала своє життя з
наукою. І зараз, вже маючи науковий ступінь, продовжую займатися цією цікавою
справою. 

Міжнародний день жінок і дівчат в науці встановлений резолюцією Генеральної
Асамблеї ООН від 15 грудня 2015 року. Свято засновано \enquote{для того, щоб досягти
повного і рівного доступу жінок і дівчаток до науки, а також забезпечити
гендерну рівність і розширення прав і можливостей жінок і дівчат}. За даними
Інституту статистики ЮНЕСКО лише 28\% науковців у світі – жінки. Загальна
чисельність жінок у вітчизняній науці складає більше 46\%. У Маріуполі працює
багато чарівних науковиць, які вражають своїми фундаментальними дослідженнями у
різних галузях знань. Цей нарис я вирішила присвятити вченим Маріупольського
державного університету, де жінки грають виключну роль у формуванні наукового
потенціалу. 159 жінок МДУ – кандидати і доктори наук (а це 77\%) створюють нові
напрями, свої наукові школи, активно співпрацюють із зарубіжними вченими.
Тільки за минулий рік з 25 науково-дослідних тем 70\% очолювали безпосередньо
жінки-керівниці цих наукових розробок.

\ii{12_02_2021.stz.news.ua.mrpl_city.1.suzirja_naukovyc_mdu.pic.1}

Перший проектор Маріупольського державного університету, доктор економічних
наук, професорка Олена Валеріївна Булатова наголосила, що \enquote{сьогодні здобутки
викладачок університету активно поширюються в Україні, і закордоном}. Перший
проректор МДУ – керівниця наукової школи з фундаментальних досліджень у сфері
глобальних та регіональних інтеграційних процесів, здійснює наукове керівництво
аспірантів і наукове консультування докторантів, підготувала сім кандидатів і
двох докторів наук. Має понад 250 наукових і науково-методичних публікацій.
Вона залучена до розробки регіональних програм розвитку, зокрема \enquote{Стратегії
Маріуполя 2030}. Особливу увагу Олена Валеріївна приділяє розвитку наукового
потенціалу студентства. Щороку в МДУ організовуються Декада студентської науки,
Форуми молодих вчених МДУ, забезпечується проведення Всеукраїнських олімпіад і
конкурсів студентських наукових робіт. В університеті створено розвинуту
систему 39 студентських наукових товариств, у роботі яких беруть участь понад
1300 студентів. Олена Валеріївна надихає своєю відданістю науковій справі,
цілеспрямованістю, щирістю та вимогливістю до себе і колег.

\ii{12_02_2021.stz.news.ua.mrpl_city.1.suzirja_naukovyc_mdu.pic.2}

У Маріупольському державному університеті працює вчена, загальний стаж
науково-педагогічної діяльності якої складає 52 роки. Це Катерина Йосипівна
Щербакова – кандидат педагогічних наук, авторка більш ніж 250
науково-методичних праць, фундаторка наукової школи з дошкільної освіти, яка
розробляє питання пізнавального розвитку особистості. Це єдина в регіоні школа
цього напряму, що дозволило створити єдину в Донецькій області кафедру,
викладачі якої залучені до розробки державного стандарту з дошкільної освіти.
Катерина Йосипівна підготувала 11 кандидатів педагогічних наук за спеціальністю
\enquote{Дошкільна педагогіка}. Спілкування з науковицею породжує у людей
впевненість у своїх силах, створює піднесений настрій, атмосферу діяльної
доброзичливості. Багато в чому завдяки її праці, протягом декількох десятиліть
очолювана нею наукова школа стала центром досліджень в галузі пізнавального
розвитку особистості.

\ii{12_02_2021.stz.news.ua.mrpl_city.1.suzirja_naukovyc_mdu.pic.3}

Дослідження вчених Маріупольського державного університету сьогодні
впроваджуються і в державну освітню політику. Так, доктор педагогічних наук,
професорка, завідувачка кафедри дошкільної освіти Маріупольського державного
університету Олена Геннадіївна Брежнєва стала однією з розробниць державного
стандарту з дошкільної освіти. Водночас вчена є авторкою власної
\enquote{Технології інтегрованих дидактичних модулів} з навчання дошкільників
елементам математики, яку активно пропагує і впроваджує у заклади  дошкільної
освіти України.

\ii{12_02_2021.stz.news.ua.mrpl_city.1.suzirja_naukovyc_mdu.pic.4}

Найбільше мене вражає, що сучасні науковиці вміють успішно поєднувати науку з
реалізацією багатьох суспільно-значущих проєктів. Зокрема, декан факультету
філології та масових комунікацій, доктор філологічних наук, професорка Світлана
Володимиріна Бечотнікова – авторка більше 100 наукових праць, що були
надруковані в Україні та за кордоном – поєднує викладацьку роботу з посадою
директорки телерадіокомпанії \enquote{Євростудія}, яка забезпечує інформаційний та
рекламний супровід для восьми радіостанцій міста Маріуполя. Світлана
Володимирівна успішно реалізує інноваційні науково-освітні проєкти у
партнерстві з Маріупольською міською радою, є переможницею багатьох грантів,
завдяки яким вдалося активізувати співробітництво із зарубіжними спеціалістами.

Є одна науковиця, яка особисто мене надихнула стати викладачем. Це Леніна
Вікторівна Задорожна-Княгницька – доктор педагогічних наук, завідувачка кафедри
педагогіки та освіти,  авторка понад 200 наукових та навчально-методичних
праць, з яких 6 монографій. Кожне її заняття для мене та моїх одногрупників
було справжнім майстер-класом, яке я й досі зберігаю у своїй пам'яті. Саме
завдяки Леніні Вікторівні забезпечується наукове обґрунтування запровадження
нової української школи, розвиток нової спеціальності – \enquote{Початкова освіта}, –
вкрай потрібної для Маріуполя.

\ii{12_02_2021.stz.news.ua.mrpl_city.1.suzirja_naukovyc_mdu.pic.5}

Павленко Олена Георгіївна, доктор філологічних наук, професорка, декан
факультету іноземних мов, авторка понад 150 наукових та навчально-методичних
праць, доклала дуже багато зусиль для інтернаціоналізації вищої освіти Наразі
факультет іноземних мов тісно співпрацює з університетами Греції: університет
Фракії імені Демокрита, університет Західної Македонії (місто Флорина),
університет Чехії, (університет Масарика), це Вільний університет в Німеччині,
інститут ім. Ґете, співпраця з Британською радою тощо. Талановиту Олену
Георгіївну неодноразово залучали до проведення занять з міжкультурної
комунікації і методики викладання іноземної мови для студентів університету
Фракії імені Демокрита (Грецька республіка) та Університету Західної Македонії.

\ii{12_02_2021.stz.news.ua.mrpl_city.1.suzirja_naukovyc_mdu.pic.6}

На дуже високому науково-педагогічному рівні викладає свої дисципліни та
постійно вдосконалює рівень професійної майстерності декан історичного
факультету, доктор історичних наук, професорка Лисак Вікторія Феофанівна. За
роки наукової діяльності вченою було підготовлено та видано понад 100 наукових
праць (зокрема й іноземною мовою). Під керівництвом Вікторії Феофанівни були
захищені 4 кандидатські дисертації.

% sabadash
\ii{12_02_2021.stz.news.ua.mrpl_city.1.suzirja_naukovyc_mdu.pic.7}

Завідувачка кафедри культурології та інформаційної діяльності, доктор
культурології, професорка, авторка більше 100 наукових праць і мій науковий
керівник Сабадаш Юлія Сергіївна протягом своєї наукової діяльності намагається
прищепити студентам не тільки любов до науки, а й до культури рідного міста.
Заснувавши ГО \enquote{Фонд збереження культурної спадщини Маріуполя}, вона надихнула
багатьох науковиць досліджувати проблемні питання культурного розвитку
Маріуполя. Юлія Сергіївна підготувала трьох кандидатів наук.

% anastasia trofimenko
\ii{12_02_2021.stz.news.ua.mrpl_city.1.suzirja_naukovyc_mdu.pic.8}

І держава, і місцеві органи влади, і університет докладають чимало зусиль, щоб
підтримати і мотивувати своїх науковиць. Зокрема, кандидат політичних наук,
доцент кафедри міжнародних відносин та зовнішньої політики МДУ Анастасія
Вікторівна Трофименко, як молода і талановита вчена за вагомі наукові здобутки
стала стипендіатом Кабінету Міністрів України. Вона підкреслила, що
\enquote{наукова діяльність потребує багато зусиль, багато енергії, натхнення,
творчості і авжеж мають бути сприятливі умови для того, щоб визріла ця любов,
це захоплення наукою}.

Путівку у наукове життя молодим науковицям Маріупольського державного
університету дає завідувачка відділу аспірантури Чередніченко Інна Віталіївна,
яка тримає руку на пульсі всіх подій в науковому житті України і завжди
допоможе корисною порадою.

\ii{12_02_2021.stz.news.ua.mrpl_city.1.suzirja_naukovyc_mdu.pic.9}

Щоб висвітлити здобутки науковиць МДУ, мабуть, не вистачить декількох книг, але
всіх їх об'єднує справжня любов до наукової діяльності. Насправді жінки в науці
сьогодні – це невтомні ентузіасти, які щодня успішно доводять, що талант і хист
до науки не мають ні статевих, ні національних  обмежень.
