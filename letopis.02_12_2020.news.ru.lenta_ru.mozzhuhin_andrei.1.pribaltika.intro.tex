% vim: keymap=russian-jcukenwin
%%beginhead 
 
%%file 02_12_2020.news.ru.lenta_ru.mozzhuhin_andrei.1.pribaltika.intro
%%parent 02_12_2020.news.ru.lenta_ru.mozzhuhin_andrei.1.pribaltika
 
%%url 
 
%%author 
%%author_id 
%%author_url 
 
%%tags 
%%title 
 
%%endhead 

\begin{center}
	\em\bfseries\color{red}\huge 
	Зачем Сталин завладел Прибалтикой\par и что с ней хотел сделать Гитлер
\end{center}

\ifcmt
pic https://icdn.lenta.ru/images/2020/10/23/11/20201023112738140/detail_e260fe16c4d6cf1f5b82485b01fd4c55.jpg
caption Немецкие солдаты смотрят на человеческий череп на броне огнеметного танка ОТ-34. Латвия, 1944 г. Фото: waralbum.ru
\fi

\def\lenta#1{\textbf{\em\color{blue} #1}}

\begin{leftbar}
	\bfseries
В издательстве РОССПЭН вышла монография «Прибалтика. 1939-1945 гг. Война и
память», посвященная одной из самых сложных и противоречивых тем европейской
истории XX века: включению Прибалтики в состав СССР после пакта Молотова —
Риббентропа, ее жизни под немецкой оккупацией в годы Великой Отечественной
войны и после освобождения Красной армией. Имели ли шанс прибалтийские страны
сохранить свою независимость во время Второй мировой войны? Почему к июню 1941
года их население возненавидело советскую власть? Какую участь Прибалтике
готовил Гитлер? Кто и почему во время нацистской оккупации уничтожил почти всех
евреев в Литве? Как СССР усмирял Прибалтику после войны и как теперь с ней быть
России? Обо всем этом «Ленте.ру» рассказала автор книги, доктор исторических
наук, главный научный сотрудник Санкт-Петербургского института истории РАН,
профессор РГПУ им. А.И. Герцена Юлия Кантор.
\end{leftbar}

