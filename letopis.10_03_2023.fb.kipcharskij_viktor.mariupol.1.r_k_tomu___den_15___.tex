%%beginhead 
 
%%file 10_03_2023.fb.kipcharskij_viktor.mariupol.1.r_k_tomu___den_15___
%%parent 10_03_2023
 
%%url https://www.facebook.com/permalink.php?story_fbid=pfbid0CgfoK84oCkzgjCzmJeTq5AyhWTeqLc8TTGCZkgrLZk9tV49DLXNFbHBkbTWiw9Qgl&id=100006830107904
 
%%author_id kipcharskij_viktor.mariupol
%%date 10_03_2023
 
%%tags mariupol.war,dnevnik,mariupol,10.03.2022
%%title Рік тому:  День 15 - 10.03.22. Четвер
 
%%endhead 

\subsection{Рік тому:  День 15 - 10.03.22. Четвер}
\label{sec:10_03_2023.fb.kipcharskij_viktor.mariupol.1.r_k_tomu___den_15___}

\Purl{https://www.facebook.com/permalink.php?story_fbid=pfbid0CgfoK84oCkzgjCzmJeTq5AyhWTeqLc8TTGCZkgrLZk9tV49DLXNFbHBkbTWiw9Qgl&id=100006830107904}
\ifcmt
 author_begin
   author_id kipcharskij_viktor.mariupol
 author_end
\fi

Рік тому: 

День 15 - 10.03.22. Четвер. 

Почався третій тиждень "гарячої" війни.

0:50. Час від часу гупає. У вухах на ніч,  як звичайно, вата - у хорошому
сенсі. 

2:30. У кімнаті 15,8 градуси.

5:40. У кімнаті вже 15,1 градуси.

Заллю теплою водою з термосу пару ложок вівсяних пластівців додам пару ложок
бурака, зроблю "намаз" та піду по дрова.

Після шостої біля вогнища почали збиратися люди: кип'ятять воду, готують їжу.

Сашко виніс свій чайник. Очікуючи, поки той закипить, сів біля вогнища на
цурпалок. Напередодні він звернувся до мене: "Як молоде життя?". З'ясували що
він на три місяці молодший від мене. Після цього він звертається до мене
"Старий". 

Сашку стало погано - він чорнобилець. Проблема з легенями, набрякають ноги.
Закінчуються необхідні ліки.  Двоє хлопців взяли його під руки і майже понесли
додому...

Пішли по дрова у балку на Рельєфній, дорогою до міліції на Жилкопи. Знайшли там
багато сухих гілок: люди підрізали дерева (у тому числі й фруктові) та скидали
у балку. До речі, балкою завжди біг струмок з джерел майже від проспекту
Металургів. Тепер струмка не видно - мабуть його засипали сміттям. Він
з'являється набагато нижче і саме з нього ми беремо "технічну" воду. Про те, що
скидали у балку окрім гілок - краще не думати ..

Я залишився на дорозі:  молоді хлопці тягали гілки з низу, я збирав у купу та
тягнув у двір. Зробив три ходки. Потім хлопці тягали, а я почав рубати та
пиляти: дрібні у вогнище, більші - у під'їзд. Добре, що у машині лежали сокира
та гілка, які возив на риболовлю. 

Шукав у машині тютюн - має ж бути, завжди возив з собою запасну люльку та трохи
тютюну. В бардачку та багажнику не знайшов. 

Сусідка з четвертого поверху пішла жити до знайомих на Правий берег - в них
приватний будинок, є пічка та підвал.

Весь час над головою ревуть літаки. В новинах сказали, що скидали бомби на
проспект Будівельників. Ремонтував сусіду з 28-ї ліхтарики.

Пішов кип'ятити чайник. Постукав до Едіка, аби взяти і його чайник. Він ледве
ходить - набрякли ноги, знову мало не впав у коридорі. Ми знову запропонували
йому зварити на курячому бульоні суп: "Я буду вмирати - їсти не хочу, от тільки
чаю...". Не вмовили.

Олексій своєю " Таврією"  поїхав на "Штуку" (магазин "Тисяча дрібниць"). Там в
головному офісі Київстару час від часу запускають генератор і з'являється
зв'язок. Затримався з виїздом хвилин на п'ять - чекав друзів. За мостом їх
зупинила жінка: "Куди ви їдете - там же бомблять!". Бомбили ринок на
Кіровському. Перечекали. Якось доїхали до Штуки, зателефонували, дізналися
новини. Фотографували руйнування у Третій лікарні: пологовий, дитяче
відділення... За годину до лікарніскинули бомбу на ПГТУ - сильно постраждав
третій корпус - " Стекляшка".

Хотіли заїхати на оптовий ринок (Каліфорнію) - там горять машини. Навпроти
Іллічівського суду працюють деякі дрібні магазини - там продають намазку
(паштети). Мабуть, з космічного пайку, якщо судити по цінах 

Лавров сказав, що він нічого не вирішує і повіз вимоги Кулеби - припинення
вогню на 24 години та зелені коридори -  у Москву.

Черговий гумконвой на Маріуполь повернувся у Запоріжжя.

Приходили свати, переказували жахи з рашаФМ.

17:00. На велосипеді від 99-го ПТУ приїхав хлопець: шукав тітку Надію - дружину
Сашка. Розповів новини: в них постійний автоматний вогонь. Росіяни захопили
Аглофабрику, біля бурси заходять у порожні будинки, ставлять у городах
миномети. Розбили 45-ту школу (там вчилася Оля - школа з 1938-го року), дитячий
садочок. Люди ховаються від обстрілів у пішоходному тунелі під залізницею, якою
возили шлак. (Пізніше у тому тунелі складали мертвих).

Розбили Волонтерівку.

Втомився, тож о 18:00 (комендантська година) пішов до дому. Трохи відпочив і
почав шукати тютюн. Знайшов давно забуту пачку цигарок "Кептен Блек", яку
колись купив для Андрія, аби він менше палив погані. Андрій запалив одну і
повернув пачку мені, мовляв для нього надто міцні. Тютюн не знайшов.

Фото: люди несуть воду від гаражів.

З того самого струмка у балці.

\ii{10_03_2023.fb.kipcharskij_viktor.mariupol.1.r_k_tomu___den_15___.pic.1}

%\ii{10_03_2023.fb.kipcharskij_viktor.mariupol.1.r_k_tomu___den_15___.cmt}
