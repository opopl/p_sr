% vim: keymap=russian-jcukenwin
%%beginhead 
 
%%file 20_02_2022.tg.mosijchuk_igor.1.mjunchen_promova_zelja
%%parent 20_02_2022
 
%%url https://t.me/mosiychuk72/11122
 
%%author_id mosijchuk_igor
%%date 
 
%%tags ukraina,zapad,zelenskii_vladimir
%%title МЮНХЕНСЬКА ПРОМОВА ЗЕЛЕНСЬКОГО
 
%%endhead 
 
\subsection{МЮНХЕНСЬКА ПРОМОВА ЗЕЛЕНСЬКОГО}
\label{sec:20_02_2022.tg.mosijchuk_igor.1.mjunchen_promova_zelja}
 
\Purl{https://t.me/mosiychuk72/11122}
\ifcmt
 author_begin
   author_id mosijchuk_igor
 author_end
\fi

МЮНХЕНСЬКА ПРОМОВА ЗЕЛЕНСЬКОГО.

Переосмисливши вчорашню Мюнхенську промову Зеленського мушу сказати...

Вісім років війни ми українці очікували й вимагали від наших українських
керівників держави сказати світові і зокрема нашим західним партнерам те, що
вчора їм в Мюнхені сказав президент України Володимир Зеленський. 

П’ять років в тому числі і я з парламентської трибуни вимагали сказати це
світові від тодішнього президента Петра Порошенка. 

Не зміг, не сказав... боявся... був на скільки залежний... 

Зеленський продемонстрував світу, що має сталеві фаберже а Україна суверенна
держава з власними національними інтересами. 

Водночас, очікувати, що зараз після цієї мюнхенської промови світ зміниться,
повернеться до України обличчям а не філейною частиною тіла і зупинить війну на
Сході нашої держави, не варто. 

Не варто мати завищені очікування, щоб потім не мати величезних розачарувань! 

В Мюнхенській промові Зеленського багато хто з нас почув те, що хотів почути і
чого там насправді не було сказано. Її треба сприймати буквально. Не більше, й
не менше. 

Український політикум має нарешті вибратися з дитячих штанців, відкинути ілюзії
й стати нарешті не балаболами а державними діячами. 

Вчора в Мюнхені на трибуні я нарешті побачив не артиста розважального жанру
який випадково увірвався в політику, а справжнього державного мужа який
переймається національними інтересами та майбутнім своєї держави, свого народу.
В ці темні для нашої держави часи хотілося й хочеться біля нього стати і його
підтримати! І зараз тільки від самого Зеленського залежить чи виконає він
покладену на нього Господом місію, в часі коли війна на порозі а ворог вже
виламує двері нашого дому згуртувати навколо себе народ і вести його до
перемоги, до миру. 

Українські політики, українська опозиція має чітко усвідомити, що ми можемо і
маємо безліч претензій до Зеленського і його «слуг» у внутрішньополітичному
кейсі але якщо хочемо перемогти і зберегти державу то мусимо залишити його на
потім, на після війни... 

Наразі на питання огульної критики всіх і вся, звинувачень у всіх смертних
гріхах, корупції, некомпетенції, жадібності, обливанні один одного брудом та
інші подібні МАЄ БУТИ НАКЛАДЕНО МОРАТОРІЙ! Накладено і опозицією, і владою, і
самим Зеленським! 

Переможемо у війні, повернемо мир на нашу землю і тоді пред’явимо один одному
все, що забажаємо, підемо на вибори й люди оцінять хто й, що зробив... 

Вчора після повернення з Мюнхена президента Зеленського в аеропорту мали б
зустрічати не лише українські силовики а й з протягненою для рукостискання
рукою всі колишні включно з Петром Порошенком президенти України, лідери
української опозиції, святі отці та представники всіх релігійних конфесій... мали
б стати поруч з своїм президентом єдиною монолітною стіною й захистити Україну
від зовнішнього ворога! А президент Зеленський зустрічно простягнувши їм руку
заявити про формування Кабміну національної єдності запросивши до нього
фахівців в незалежності від партійності та політичних поглядів... 

Не приїхали, не простягли руки, не стали поруч, не об'єдналися... але час для
виправлення помилок усіх сторін ще є... хоч і не багато. 

Сьогодні з обіду на моєму ютуб-каналі
\url{https://youtube.com/channel/UCFUrRCG93FBzDiZR1DhpLiw}
з доктором історичних
наук Олегом Однороженко детально розберемо й проаналізуємо
\#мюнхенська\_промова\_зеленського. 

Залітай, підписуйся і коментуй! @igg{fbicon.index.pointing.right} 
