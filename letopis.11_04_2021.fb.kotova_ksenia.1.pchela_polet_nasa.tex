% vim: keymap=russian-jcukenwin
%%beginhead 
 
%%file 11_04_2021.fb.kotova_ksenia.1.pchela_polet_nasa
%%parent 11_04_2021
 
%%url https://www.facebook.com/xeniya.kotova/posts/4298722403520716
 
%%author 
%%author_id 
%%author_url 
 
%%tags 
%%title 
 
%%endhead 
\subsection{Будьте пчелами, независимо от размера крыльев, летайте и наслаждайтесь жизнью}
\label{sec:11_04_2021.fb.kotova_ksenia.1.pchela_polet_nasa}
\Purl{https://www.facebook.com/xeniya.kotova/posts/4298722403520716}

В космическом научном центре НАСА висит плакат с пчелами, на котором написано:

\enquote{Аэродинамическое тело пчелы не приспособлено летать, но хорошо, что пчела об этом не знает}.

\ifcmt
  pic https://scontent-mxp1-2.xx.fbcdn.net/v/t1.6435-9/172184625_4298722246854065_6687652084257462492_n.jpg?_nc_cat=106&ccb=1-3&_nc_sid=8bfeb9&_nc_ohc=WXX2F8LOa8wAX_kg7ka&_nc_ht=scontent-mxp1-2.xx&oh=ae4bd9c41125567d01cbd5f3241f1426&oe=609F1CF7
\fi

Закон физики, аэродинамический принцип гласит, что ширина крыльев слишком
мала, чтобы удержать ее огромное тело в полете, но пчела не знает, она ничего
не знает о физике или ее логике и все равно летит.

Это то, что мы все можем делать, летать и побеждать в любой момент перед
любыми трудностями и при любых обстоятельствах, несмотря на то, что они
говорят.

Будьте пчелами, независимо от размера крыльев, летайте и наслаждайтесь жизнью.
