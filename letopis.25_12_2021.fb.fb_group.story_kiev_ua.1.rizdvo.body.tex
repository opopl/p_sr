% vim: keymap=russian-jcukenwin
%%beginhead 
 
%%file 25_12_2021.fb.fb_group.story_kiev_ua.1.rizdvo.body
%%parent 25_12_2021.fb.fb_group.story_kiev_ua.1.rizdvo
 
%%url 
 
%%author_id 
%%date 
 
%%tags 
%%title 
 
%%endhead 
Сьогодні Різдво Христове... Вперше за 47 років я відзначаю його один, без дружини
Надійки. Традиція – вічна категорія. І я, як завжди було у минулі роки,
розкладаю на простирадлі усі наявні ялинкові прикраси, щоб вибрати з них ті, що
хочу повісити на ялинку. Розкладаю іграшки, а пам’ять відносить мене у минуле,
в далекі щасливі періоди життя – дитинство, юність і зрілість.

***

З часів коли я себе пам'ятаю, у нашій сім'ї була традиція: ялинку ставили і
прикрашали іграшками у другів половині дня 24 грудня. Але про наступаюче Різдво
Христове ніхто нічого не говорив! Чому саме ялинку ставили напередодні Різдва
(яке на теренах СРСР вважається католицьким), а не 31-го грудня, напередодні
Нового року, я не знаю. В дитинстві сприймалося, що так потрібно і це питання
мене не переймало. А коли я зацікавився, то запитати було ні в кого. Усі родичі
вже відійшли за завісу. А загадка залишилася. 

\ii{25_12_2021.fb.fb_group.story_kiev_ua.1.rizdvo.body.pic.a}

%\begin{multicols}{2} % {
%\setlength{\parindent}{0pt}

%\ii{25_12_2021.fb.fb_group.story_kiev_ua.1.rizdvo.pic.1}
%\ii{25_12_2021.fb.fb_group.story_kiev_ua.1.rizdvo.pic.1.cmt}

%\ii{25_12_2021.fb.fb_group.story_kiev_ua.1.rizdvo.pic.4}
%\ii{25_12_2021.fb.fb_group.story_kiev_ua.1.rizdvo.pic.4.cmt}

%\end{multicols} % }

Ялинка у нас була натуральна, завжди під три метри висотою (стелі в квартирі
були під 4.5 метри), приємно пахла хвоєю, і приносила в квартиру запах дикого
лісу. 

Прикрашати ялинку довіряли мені і двоюрідній сестрі Лідії, яка була старша за
мене на 10 років. На самому верху ялинки іграшки вішала Лідія, стоячи на
стільці. Ну, а з середини і нижче було моєю парафією.

%\ii{25_12_2021.fb.fb_group.story_kiev_ua.1.rizdvo.pic.2}
%\ii{25_12_2021.fb.fb_group.story_kiev_ua.1.rizdvo.pic.2.cmt}

Перш за все усі іграшки діставалися з коробок і розкладалися на простирадлі,
яке стелилося на дивані. 

Іграшки були радянські і китайські. Пам’ятаю, що були великі кулі (приблизно
10-12 см в діаметрі) зі східними пагодами і фантастичними рибами та драконами,
які світилися в темряві. Фігурки Діда Мороза і Снігурочки, кілі різного
розміру, шишки, гриби і гриби-мухомори, білочки, горішки, рибки, годинники...
Гірлянди, буси і золотий та срібний дощ, який хвилями спадав донизу... 

З сестрою ми також робили іграшки самостійно. Перш за все сніжинки. Тоді був
папір (його називали «фосфорний»), який світився в темряві після того як
побуває під світлом. З нього ми і вирізали сніжинки. 

Грецькі горіхи загортали у «золоту» та «срібну» фольгу; у верхівку акуратно
забивався маленький гвіздочок, до якого прив’язувалася нитка, і горішки
вішалися на ялинку. Також на ялинку вішалися справжні цукерки. Правда, вони
рідко доживали до розбору ялинки. Грішив я. грішила і сестричка. Ми акуратно
розгортали упаковку знизу, витягали цукерку і знову акуратно загортали
упаковку. Зовні не можна було помітити, що упаковка вже пуста. «Момент істини»
наступав при розборі ялинки... Горішки не можна було витягти, тому вони
благополучно доживали до кінця свят і після розбору ялинки ділилися між мною і
сестрою.

\ii{25_12_2021.fb.fb_group.story_kiev_ua.1.rizdvo.pic.3}

На защіпках до гілок чіплялися маленькі свічники, у які ставилися мініатюрні
свічки. Їх запалювали і ялинку не залишали без нагляду, щоб не трапилося
пожежі.

\ii{25_12_2021.fb.fb_group.story_kiev_ua.1.rizdvo.pic.5}

Верхівки на ялинці були різні. У вигляді шпилю, з зіркою або без зірки. Під
ялинкою ставили великі фігури Діда Мороза і Снігурочки, виготовлені з картону,
пап’є-маше та вати. А через усю кімнату по діагоналям протягувалися нитки, на
яких були нанизані різнокольорові прапорці.

\ii{25_12_2021.fb.fb_group.story_kiev_ua.1.rizdvo.pic.6}

Ввечері 25-го грудня вперше запалювалися свічечки. Після чого вимикалося
електричне світло... Це було справжнє Чудо! Ми сиділи з сестрою і милувалися
ялинкою. Але ніякої святкової вечері не було.

\ii{25_12_2021.fb.fb_group.story_kiev_ua.1.rizdvo.pic.7}

***

За Різдвом наступав Новий Рік. Це було сімейне свято! На ялинці горять
свічечки... Десь о пів-на-дванадцяту сідаємо за святковий стіл... Дорослі п’ють
вино, а ми з сестрою – лимонад. Новорічні страви, перш за все холодець. Десь
хвилин за п’ять до Нового року розкорковується шампанське і розливається по
келихам... Телевізора у нас не було (з’явився тільки  у 1963 році), тому дивитися
новорічне привітання «вождів» не було потреби. Включали радіоприймач, щоб
почути, як опівночі б’ють кремлівські куранти. 

\ii{25_12_2021.fb.fb_group.story_kiev_ua.1.rizdvo.pic.8}

Я і Лідія тримаємо в руках новорічні хлопавки, щоб одночасно з останнім боєм
курантів салютнути. Дванадцятий удар курантів! Ми салютуємо... Бах! Бах! Конфетті
розлітається по кімнаті! 

Дзвін бокалів! З Новим роком! 

Вечеря продовжується... Ми запалюємо бенгальські вогні... Салютуємо новорічними
хлопавками... Годині о другій ночі ми лягаємо спати. Дорослі ще сидять і
розмовляють.

\ii{25_12_2021.fb.fb_group.story_kiev_ua.1.rizdvo.pic.9}

***

Наступає православне Різдво (7 січня), потім мій день народження (10 січня) і
Старий Новий Рік. А 15-го січня, вранці, ялинку розбираємо від іграшок… Іграшки
повертаються у ящики і ховаються на шафу в очікуванні наступних свят. Ялинку
виносять у двір.

Свята закінчилися! Весь святковий цикл продовжувався 21 день!

\ii{25_12_2021.fb.fb_group.story_kiev_ua.1.rizdvo.pic.10}

***

25 грудня 1974 року ми вперше відзначали Різдво і Новий рік разом з Надійкою. 

В Надійчиній сім’ї традиційно ставили ялинку 31 грудня. Оскільки ми були
молодою самостійною сім’єю, то стали дотримуватися моїх традицій. 24 грудня ми
вийшли прогулятися по нашому району… Вже було декілька ялинкових базарів. Разом
вибрали велику ялинку висотою в 3 метри (у нашій квартирі стелі на висоті 3
метри). 

\ii{25_12_2021.fb.fb_group.story_kiev_ua.1.rizdvo.pic.11}

Приносимо додому, розв’язуємо вірьовки, яким пакували ялинку. Вона величезна і
пухнаста! Батьки в стані шоку! Ми заспокоюємо їх тим, що поставимо ялинку у
своїй кімнаті.

\ii{25_12_2021.fb.fb_group.story_kiev_ua.1.rizdvo.pic.12}

Дерев’яна хрестовина від попередніх ялинок для нашої ялинки замала. Знову йду
на ялинковий базар і вибираю потрібну хрестовину. 

Тепер треба «вписати» ялинку в інтер’єр кімнати. Перш за все одягаємо на ялинку
верхівку і я заміряю загальну довжину. Відпилюю зайвий шматок стовбура,
вставляю ялинку у хрестовину і закріплюю забитими гвіздками. Ялинка повинна
стояти міцно, бо треба враховувати, що у нас є маленький член сім’ї – котик
Кешка. Йому може прийти в голову ідея залізти на ялинку або покачатися на
гілці. 

\ii{25_12_2021.fb.fb_group.story_kiev_ua.1.rizdvo.pic.13}

Ялинка встановлена! На розкладному дивані стелимо простирадло і викладаємо на
ньому усі новорічні іграшки і прикраси. 

Процес вбрання ялинки зайняв декілька годин. Але це були години насолоди!

Ялинка простояла до 15 січня.

\ii{25_12_2021.fb.fb_group.story_kiev_ua.1.rizdvo.pic.14}

***

Різдво 2020-го року...

Десь у 2002 році ми з Надійкою придбали китайську штучну світловолоконну
ялинку. Керувалися ми суто екологічними міркуваннями. Ялинка фантастично
світилася, змінюючи кольори і при денному світлі, і в умовах темряви чи
напівтемряви. 

\ii{25_12_2021.fb.fb_group.story_kiev_ua.1.rizdvo.pic.15}

Надійка вже не виходила на вулицю, але по кімнаті ще ходити могла. 

24 грудня... Ми прикрашали ялинку і кожен з нас знав, що ця ялинка остання у
Надійчиному житті. Дружина була мужньою людиною і ми ніколи з нею не говорили
про смерть, хоча обидва знали, що смерть вже стоїть за її спиною. І нічого
змінити ми не в силах!

\ii{25_12_2021.fb.fb_group.story_kiev_ua.1.rizdvo.pic.16}

***

Життя продовжується... 

Я прикрашаю ялинку і впевнений, що Надійці це сподобається. Шлюб вічний!

***

\ii{25_12_2021.fb.fb_group.story_kiev_ua.1.rizdvo.pic.17}
\ii{25_12_2021.fb.fb_group.story_kiev_ua.1.rizdvo.pic.18}
\ii{25_12_2021.fb.fb_group.story_kiev_ua.1.rizdvo.pic.19}
\ii{25_12_2021.fb.fb_group.story_kiev_ua.1.rizdvo.pic.20}

