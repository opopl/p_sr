% vim: keymap=russian-jcukenwin
%%beginhead 
 
%%file slova.demokratia
%%parent slova
 
%%url 
 
%%author 
%%author_id 
%%author_url 
 
%%tags 
%%title 
 
%%endhead 
\chapter{Демократия}
\label{sec:slova.demokratia}

По сути, нет ни единого аргумента, почему должно существовать ограничение по
возрасту для голосования. Это стало бы просто эпохальным прецедентом для
мировой \emph{демократии}, если бы парламент Украины внес изменения в
Конституцию страны и внедрил принцип реального всеобщего избирательного права,
без всяких ограничений по возрасту.  Просто с рождения и до смерти,
\textbf{Давайте распространим право выбирать на детей!},
Игорь Ляшенко, strana.ua, 06.05.2021

%%%cit
%%%cit_head
%%%cit_pic
%%%cit_text
А вірніше – з прадавніх. Історія нашої конституанти відраховується від початку
ІІІ століття до нашої ери! Тоді на місці сучасного Севастополя розташовувався
еллінський поліс Херсонес. Він унікальний тим, що єдиний у всьому Надчорномор'ї
зберіг \emph{демократичний лад}. Не все було гладко, і ймовірно у цей період
там відбулася олігархічна змова або спроба встановлення тиранії, але
народоправство врешті перемогло. На честь події та для запобігання рецидивам
батьки міста уклали для громадян урочисту присягу, текст якої викарбували на
стелі. Так, це ще не зовсім Основний закон. Але у ньому вже є два базові
елементи, без яких неможлива жодна Конституція: окреслення території держави та
зазначення політичного і державного ладу.  Зауважте – 23 століття минуло, а ми
рухаємося у вказаному Херсонесом напрямку – \emph{демократичної республіки}.
Збіг? Не думаю
%%%cit_comment
%%%cit_title
\citTitle{Українській Конституції – 2300 років}, 
Сергій Громенко, gazeta.ua, 26.06.2021
%%%endcit

