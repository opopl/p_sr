% vim: keymap=russian-jcukenwin
%%beginhead 
 
%%file 07_04_2023.stz.news.ua.zaborona.1.mariupol_vajb_still_ugly
%%parent 07_04_2023
 
%%url https://zaborona.com/mariupolskyj-vajb-do-okupacziyi-vid-strit-fotografa-stil-agli
 
%%author_id still_ugly.fotograf.mariupol,news.ua.zaborona
%%date 
 
%%tags kultura,mariupol,mariupol.pre_war
%%title Маріупольський вайб до окупації від стріт-фотографа Still Ugly
 
%%endhead 
 
\subsection{Маріупольський вайб до окупації від стріт-фотографа Still Ugly}
\label{sec:07_04_2023.stz.news.ua.zaborona.1.mariupol_vajb_still_ugly}
 
\Purl{https://zaborona.com/mariupolskyj-vajb-do-okupacziyi-vid-strit-fotografa-stil-agli}
\ifcmt
 author_begin
   author_id still_ugly.fotograf.mariupol,news.ua.zaborona
 author_end
\fi

\ii{07_04_2023.stz.news.ua.zaborona.1.mariupol_vajb_still_ugly.pic.1}

\begin{qqquote}
%\Large
П'ять років тому маріупольський стріт-фотограф з мистецьким псевдонімом Still
Ugly \footnote{\textbf{Ugly}\par в перекладі з англійської - бридкий} почав документувати рідне місто. Сьогодні його кадри з візуального
дослідження простих мешканців Маріуполя, архітектури міста та його
індустріальних ландшафтів сприймаються вже зовсім по-іншому. Після початку
повномасштабної війни Маріуполь спочатку опинився в блокаді, а згодом і в
окупації. Наразі місто майже порожнє через масову евакуацію місцевих, а колись
багатий архітектурний спадок (Заборона \href{https://zaborona.com/shho-vy-znayete-pro-arhitekturu-promyslovyh-mist-vona-prekrasna-os-yak-vyglyadayut-najczikavishi-budivli-mariupolya/}{детально писала про нього тут}) суттєво
постраждав через масовані обстріли окупантів. Заборона ділиться роздумами Still
Ugly про Маріуполь та його мешканців, а також публікує знімки фотографа,
зроблені до окупації міста.
\end{qqquote}

\subsubsection{Два шляхи}

Фотокамера була в моєму житті завжди. У дитинстві я знімав сімейні застілля або
поїздки. Брав потайки сімейну мильницю зі щойно зарядженою плівкою і
фотографував все навколо.

В часи мого юнацького максималізму я ненавидів місто і хотів якомога швидше
виїхати до столиці. Пам'ятаю, як спостерігав не найприємніші картини:
гоп-культура, неповноцінні сім'ї, злидні та кримінал. Дивлячись на все це, мені
хотілося не опинитися на дні, тому в дитинстві завжди доводилося щось робити.
Спочатку я займався плаванням, потім футболом, згодом захопився музикою. 

\ii{07_04_2023.stz.news.ua.zaborona.1.mariupol_vajb_still_ugly.pic.2}

Проте в один момент я зв'язався з поганою компанією. Через це просіло навчання,
я закинув спорт і сприймав майбутнє з повною байдужістю. В цих нескінченних
панельних дворах для юнака було два шляхи: або ти п'єш пиво на лавочці з
кєнтами і куриш траву в під'їзді, або займаєш себе чимось сам, але паралельно
все одно п'єш пиво. Тому що без нього в Маріуполі взагалі важке життя. В якийсь
момент в мою компанію почали приходити люди, які вже в 16 років сиділи на
важких наркотиках. Грошей у них не було. Тому вони крали мобільні телефони на
вулицях чи зламували магазини й кіоски. Це було зовсім не те, чого мені
хотілося. Тож я відразу припинив це спілкування і повернувся до попереднього
способу життя.

\ii{07_04_2023.stz.news.ua.zaborona.1.mariupol_vajb_still_ugly.pic.3}

Якось я виграв фотоконкурс, і мене запросили у фотошколу на навчання. Але тоді
я відмовився: було лінь витрачати на дорогу дві години часу. Згодом фотографія
відійшла на другий план. Потрібно було приділяти увагу навчанню, дівчині та
першій роботі. Я сезон пропрацював офіціантом, після чого став допомагати
батькові з меблями. Це була збірка кухонь і спалень. Завдяки цій роботі я зміг
побувати у багатьох людей вдома, побачити їхній побут. Інколи було страшно від
того, що у всіх клієнтів мого батька, здебільшого мешканців панельок, були
однакові антресолі, шпалери і навіть плитка у ванній.

Сьогодні я шкодую, що тоді не знімав цього. В мене було багато ідей, думав —
встигну. Шкодую, що не розповів більше історій про Маріуполь. 

\ii{07_04_2023.stz.news.ua.zaborona.1.mariupol_vajb_still_ugly.pic.4}

Але доросле життя мені відразу ж набридло. Я завжди хотів знімати кіно, тож
вирішив, що пора з чогось починати.

Одного дня я побачив унікальну ситуацію на автобусній зупинці. Чоловік жадібно
їв шаурму, жінка говорила по телефону, її дитина плакала, а зовсім поруч
валявся бездомний. Все це відбувалося на фоні жахливих рекламних банерів. Люди
вишикувалися ідеально композиційно, на всю площину фотокадру. Світло падало,
ніби його намалював сам Караваджо, і все це нагадувало картину епохи Ренесансу.
Тоді у мене з'явилося усвідомлення, що я маю це все зафіксувати.

\subsubsection{Найфотогенічніше місто}

Багатьом здається, що я знімаю тільки погане, але моя естетика здається мені
досить гармонійною та милою. Все, що я знімаю, присвячене простій людині.
Просто у мене специфічний смак.

З усіх міст, які я знімав, Маріуполь найфотогенічніший. Я можу годинами блукати
одними й тими ж локаціям, а все одно щодня знаходити в них нове. В Маріуполі
живуть одні з найкумедніших людей, яких я зустрічав. Про них хочеться говорити.

\ii{07_04_2023.stz.news.ua.zaborona.1.mariupol_vajb_still_ugly.pic.5}

Його мешканці хочуть спокою і щоб їх менше вчили життя. Їх все влаштовує,
оскільки багато з них звикли до такого життя і не хочуть нічого змінювати. Коли
я їх знімав, мені здавалося, що люди усвідомили свою розгубленість в цьому
світі, але теж часто бачив в них нескінченну кількість щирості й любові до
самої ідеї життя.

Мені цікаво це знімати тому, що я люблю щось незвичне. І маргінали — це люди, в
яких є історії, незвичні хоча б в чомусь. Документація життя тих, на кого всім
плювати — це виклик мені як художнику.

\ii{07_04_2023.stz.news.ua.zaborona.1.mariupol_vajb_still_ugly.pic.6}

Одного разу у дворі я знімав всякий власноруч виготовлений крінж типу тварин з
пляшок і автомобільних шин. І до мене причепився чоловік. Він просив продати
йому наркотиків, думав, що я закладчик, і знімаю тут для дилера. Через дві
години він знову знайшов мене у сусідньому мікрорайоні, слідкував за мною.
Загалом, диваків вистачало. 

Мені подобається працювати з реальністю і збирати з цього різні серії. Підбір
серії картинок, зроблених в одній тематиці, допомагає мені краще пізнати буття
навколо.

\ii{07_04_2023.stz.news.ua.zaborona.1.mariupol_vajb_still_ugly.pic.7}

Але мені не подобається майже все, що я роблю. Я свій найсуворіший критик, і
нічого не можу з цим вдіяти. Та коли бачу це навколо себе, то починаю знімати.
Все, що я сьогодні знімаю, присвячене простій людині і наповнене естетикою
повсякденності. Я бачу красу навіть в якихось потворних для більшості речах.
Ugly-дизайн або прості мужички з заводу з пляшкою пива в руках. Вони чесні й
щирі. Ніколи не знаєш, що від них очікувати.

Так, одного разу я був понятим у кримінальній справі. Тоді один чоловік зарізав
свого друга під легендарним мостом у Маріуполі, що з'єднує два береги. Вони не
поділили якусь дрібницю — типовий п'яний побут. Моя їзда з поліцейськими
тривала пів дня, і вони, не соромлячись мене, жартували про розчленування і
готування їжі з цього. Мені тоді стало максимально огидно, я відчував себе
героєм якогось чорнушного фільму на кшталт \enquote{Вантаж 200}.\footnote{
\textbf{Вантаж 200}\par Cтрічка 2007 року російського режисера Олексія
Балабанова, яку вважають прикладом \enquote{чорнухи} через відверте зображення
зґвалтувань (зокрема, з використанням пляшки чи поруч із тілом померлого
\enquote{афганця}), вбивств та смертної кари.}

\ii{07_04_2023.stz.news.ua.zaborona.1.mariupol_vajb_still_ugly.pic.8}

\subsubsection{Маріуполь до окупації}

Це зараз Маріуполь — відомий на весь світ символ боротьби українського народу.
А до 2014 року Маріуполь був містом, в якому зупинився час. Мені тоді
здавалося, що Радянський Союз все ще існує, нічого нового не будували,
суспільство перебувало в стагнації. Всі гроші, які надходили в бюджет міста,
розкрадалися місцевими депутатами. Місто розвивалося тільки завдяки небайдужим
жителям і приватним підприємцям. Але це було як крапля в морі.

\ii{07_04_2023.stz.news.ua.zaborona.1.mariupol_vajb_still_ugly.pic.9}

Коли у 2014 році почалася війна, вся влада Донецької області переїхала до
Маріуполя. В місто поступово стали приходити великі інвестиції. За кілька років
до повномасштабного вторгнення місто переживало нове народження. Облагородили
кожен двір, побудували дороги, відкрили нові заклади, сучасні медичні та
освітні центри. А найважливіше — кошти дійшли й до культурних центрів.
З'явилося чимало культурних заходів, фестивалів і творчих людей.

Всі кошти йшли на розвиток інфраструктури. Хоча, може, варто було приділити
увагу побудові більш міцної оборони міста. Тому що ніхто, крім військових і
деяких людей, не вірив, що Маріуполь знову опиниться в епіцентрі бойових дій. 

\subsubsection{Маріупольський вайб}

Маріуполь — важке індустріальне місто, яке, попри величезний стрибок в останні
роки, залишиться сірим назавжди. Пострадянська ментальність і сірість міста
залишили серйозний шрам на людях. Більшість з них злі і депресивні. Зараз я
говорю і за маргіналів, і за середній клас суспільства.

Для мене це місто контрастів. Водночас і найбільш проукраїнське місто, де я
був, і місто любителів \enquote{русского мира}. Місто, яке одночасно мене надихає і
вбиває. Місто, з якого хочеш виїхати, але завжди прагнеш повернутися. 

\ii{07_04_2023.stz.news.ua.zaborona.1.mariupol_vajb_still_ugly.pic.10}

В Маріуполі є кілька місць, куди я обожнював ходити знімати. Перше — це
промзони і місця біля \enquote{Азовсталі}. Бувало, просто встану, закурю і хвилини дві
можу залипати на масштаби й силу цього заводу, хоча я бачив його майже щодня.
\enquote{Азовсталь} ділить місто навпіл, і в мене завжди складалося враження, що на
кожному березі [Маріуполя] свій вайб. 

Проте більше часу я проводив на лівому березі. Там є такий пагорб біля однієї з
найкрасивіших церков, які я бачив, де відкривається панорамний вид на море. Там
ідеально все: церква з чорними куполами, зелень, море, гора зі шлаку з димовими
трубами \enquote{Азовсталі}. Це Маріупольський вайб. Мені подобалося, що там ніколи не
було людей, і ми з друзями могли робити все що завгодно.

\ii{07_04_2023.stz.news.ua.zaborona.1.mariupol_vajb_still_ugly.pic.11}

\subsubsection{Місто в окупації}

Я застав війну в іншій півкулі світу. Через свою основну роботу я багато
подорожую світом. Мені потрібно було летіти до Японії, і я поїхав якраз в січні
2022 року, за місяць до повномасштабного вторгнення. Але останні два роки я жив
на два міста — Київ та Маріуполь.

Сьогодні Маріуполь майже стертий з лиця землі, залишки міста віддали чеченцям,
загнали купу мігрантів. Тепер лезгинка у дворі — це норма і в порядку речей.

Це була жахлива зима для тих, хто залишився, я бачив замість вікон написи на
картонках, що немає опалення, що діти замерзають і сидять без світла.

\ii{07_04_2023.stz.news.ua.zaborona.1.mariupol_vajb_still_ugly.pic.12}

Всі мої близькі і друзі виїхали за першої ж можливості. У місті з моїх знайомих
нікого не лишилося. Кілька разів туди їздили друзі з Варшави, щоб залагодити
справи і забрати речі. Міста немає: залишилося пару мікрорайонів вцілілих.
Моєму мікрорайону якраз пощастило найбільше. Прямих бойових зіткнень там
особливо не було.

Останній рік в мене була повна апатія. Здавалося, що мої роботи неважливі. Не
було бажання навіть публікувати щось. Тільки зараз приходить усвідомлення, що
без творчості я просто загинаюся, фотографія — як терапія. В планах зараз
тільки більше знімати, знімати все, що попадеться — треш, кумедних незнайомців,
балкони. Неважливо що — головне знімати.

\ii{07_04_2023.stz.news.ua.zaborona.1.mariupol_vajb_still_ugly.pic.13.pro_avtora}
