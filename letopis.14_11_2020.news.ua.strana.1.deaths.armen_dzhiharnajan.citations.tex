% vim: keymap=russian-jcukenwin
%%beginhead 
 
%%file 14_11_2020.news.ua.strana.1.deaths.armen_dzhiharnajan.citations
%%parent 14_11_2020.news.ua.strana.1.deaths.armen_dzhiharnajan
 
%%url 
%%author 
%%tags 
%%title 
 
%%endhead 

\subsubsection{Цитаты и политическая позиция}
\label{sec:14_11_2020.news.ua.strana.1.deaths.armen_dzhiharnajan.citations}

\begin{itemize}
	
	\item \textbf{Об отношениях Украины и России}. 
		\begin{itemize}
			\item \enquote{Я честно не понимаю: неужели Украина и
			Россия \dshM это действительно разные государства?}
			
			\item \enquote{Это страшно, если россияне и украинцы станут врагами на уровне
			генетической памяти}
			
			\item \enquote{Один дурак бросил камень в воду, 100 умных вытянуть не могут}.
		\end{itemize}

	\item \textbf{О Путине.} В 2012 году был доверенным лицом Владимира Путина на
   президентских выборах. Но в марте 2014 года не стал подписывать
   коллективное обращение российских деятелей культуры, в котором они заявили
   о поддержке позиции Путина по событиям в Украине и Крыму.

   \enquote{Мы все, простите за грубость, стали раком и повторяем: }Пусть Путин будет
   хорошим\enquote{. Но он не может, для этого ему в лицо надо правду сказать...}

	\item \textbf{О сносе памятников Ленину в Украине.} \enquote{Те, кто снес \dshM глупые люди. Сколько
   времени прошло, как Вашингтон возглавлял США? Больше 200 лет, но никому в
   голову не пришло разрушить его памятник. Памятники всегда строили. Пусть
   все останутся, это же красивая вещь...}

	\item \textbf{О женщинах.} \enquote{С женщинами знаете, что самое страшное? С одной и той же
   женщиной в разных ситуациях чувствуешь себя по-разному. Сегодня \dshM хорошо,
   завтра \dshM противно, послезавтра \dshM опять хорошо. А через две недели: кто это
   такая? зачем мне все это?}

	\item 	\textbf{О детях.} \enquote{Дети запоминают не слова, а поступки}.
	\item 	\textbf{О любви.} \enquote{Самое главное \dshM любить людей. Вы думаете это легко? А
   попробуйте. У вас не получится}.

\end{itemize}

