% vim: keymap=russian-jcukenwin
%%beginhead 
 
%%file 25_01_2022.fb.nitsoi_larysa.1.knigi
%%parent 25_01_2022
 
%%url https://www.facebook.com/larysa.nitsoi/posts/5129245930440831
 
%%author_id nitsoi_larysa
%%date 
 
%%tags deti,kniga,literatura,mova,ukraina,verhovna_rada
%%title Сторінки усіх книжок цупкі і на кожній сторінці малюнки!
 
%%endhead 
 
\subsection{Сторінки усіх книжок цупкі і на кожній сторінці малюнки!}
\label{sec:25_01_2022.fb.nitsoi_larysa.1.knigi}
 
\Purl{https://www.facebook.com/larysa.nitsoi/posts/5129245930440831}
\ifcmt
 author_begin
   author_id nitsoi_larysa
 author_end
\fi

Ура! Сьогодні Верховна Рада ухвалила законопроект про книгодрукування і
книгорозповсюдження. Законотворці кажуть, що мета закону - підтримати читання
книг УКРАЇНСЬКОЮ мовою; зменшити вплив країни-агресора на книжковий ринок
України; сприяти розвитку книгарень. 

\ifcmt
  ig https://scontent-lhr8-1.xx.fbcdn.net/v/t39.30808-6/272295363_5129244650440959_4939745835992559946_n.jpg?_nc_cat=109&ccb=1-5&_nc_sid=730e14&_nc_ohc=DyUXHkmN3uwAX-xikOm&_nc_ht=scontent-lhr8-1.xx&oh=00_AT-cT52tFF0kR4kbgmy_qkxb7nH9uqZB-XbhZ-526clNnw&oe=61FBF48D
  @wrap center
  @width 0.8
\fi

Поки я сідаю вичитувати, чи є там підводні камені, бо буде ще друге
голосування, а ви тим часом замовляйте мої книжечки своїм діткам, і про
сусідських не забувайте.  Даруйте просто так, для настрою
@igg{fbicon.gift.wrapped}  і для підтримки @igg{fbicon.heart.with.ribbon}. Усім
підпишу і щось гарненьке побажаю.

Замовляйте тут, пишіть у приватні повідомлення. Гляньте, якої з книг у вас ще
немає?

@igg{fbicon.hibiscus}  Книжка «Бджілка» - казка про бджолу, яка збирала мед.
Це казковий урок природознавства про взаємозв’язок явищ природи. До книги йде
смаковий додаток та доповнена реальність у форматі 3D. Смаковий додаток – стік
в декоративній обкладці. Читаючи книгу одразу можна посмакувати медом від
Бджілки. Доповнена реальність у форматі 3D - якщо навести телефон на кюар-код
у книзі, Бджола з книги оживає і літає перед читачем. 32 сторінки, цупкі,
малюнки на кожній сторінці. Вартість 250 грн. (Вік садочковий, та початкової
школи - для всіх, хто вчиться читати, у книзі великі літери). 

@igg{fbicon.blossom}  Книжка «Неслухи і вередулі» - слухняним дітям читати
заборонено! У книжці зібрано п’ять повчальних казок про Лесюню – вередуню,
Мар’янчика-грубіянчика, Наталочку-забувалочку, Настуню-хитруню і
Сашуню-ледарюню. Усіх їх чекають пригоди і висновки, як робити не варто. А
також кіт, який розмовляє людською мовою, та потоптався своїми лапами по всі
книжці. 56 сторінок, цупкі, малюнки на кожній сторінці. Вартість 200 грн. (Вік
старший садок і початкова школа).

@igg{fbicon.sunflower} Книжка «Незламні мураші» - казка про мурашник і мирне
мурашине життя. Мураш закохується в Мурасю і поки вони з друзями гулятимуть,
на мурашник нападуть сусідські жуки. Мурашам, які не вчилися воювати,
прийдеться захищати свій мурашник. У цій круговерті Мурась загубить Мурасю і
думатиме, що назавжди... але книга, як і всі інші книжки Лариси Ніцой
закінчується щасливо. Книга увійшла в «Книгу рекордів України з наймасовішого
читання дітьми». 48 сторінок, цупкі, малюнки на кожній сторінці. Вартість 200
грн. (Вік початкова школа).

@igg{fbicon.cherry.blossom}  Книжка «Дві бабуськи в незвичайній школі або
скарб у візку» - смішна пригодницька повість про учнів на інклюзивну тематику.
Друзі Василинка (на яку всі кажуть Вася), Любомирчик (на якого всі кажуть
Люба), і їхній друг Олежка, який не може ходити, тому їздить у візку,
потрапляють у шкільне підземелля і шукають скарб. Друзі мріють збудувати
інклюзивну школу, щоб вони могли вчитися в ній разом з Олежкою. Книгу часто
задають у школі на позакласне читання. 192 сторінки, малюнки на кожній
сторінці. Вартість 250 грн. (Вік початкова і середня школа).

Сторінки усіх книжок цупкі і на кожній сторінці малюнки!
