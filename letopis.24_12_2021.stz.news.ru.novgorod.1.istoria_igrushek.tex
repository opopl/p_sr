% vim: keymap=russian-jcukenwin
%%beginhead 
 
%%file 24_12_2021.stz.news.ru.novgorod.1.istoria_igrushek
%%parent 24_12_2021
 
%%url https://news.novgorod.ru/articles/read/1154.html
 
%%author_id lihanova_olga
%%date 
 
%%tags igrushka,prazdnik,ukrashenie,jolka
%%title История игрушек: как украшали ёлку в разные эпохи
 
%%endhead 
\subsection{История игрушек: как украшали ёлку в разные эпохи}
\label{sec:24_12_2021.stz.news.ru.novgorod.1.istoria_igrushek}

\Purl{https://news.novgorod.ru/articles/read/1154.html}
\ifcmt
 author_begin
   author_id lihanova_olga
 author_end
\fi

\ii{24_12_2021.stz.news.ru.novgorod.1.istoria_igrushek.pic.1}

\begin{zznagolos}
Шарики, звёзды, гирлянды, танцоры в национальных костюмах, герои сказок,
космонавты... Коллекция ёлочных игрушек, которая хранится в фондах
Новгородского музея-заповедника, позволяет проследить, как менялась жизнь в
нашей большой стране. Что обсуждали люди, о чём говорили в новостях, какие
книги читали детям — на все эти вопросы можно получить ответ, если
присмотреться к старым символам праздника.
\end{zznagolos}

Игрушки попадают в музей разными путями. Некоторые были куплены в антикварных
магазинах, некоторые привезли из этнографических экспедиций. Бывало, люди сами
приносили игрушки в дар или на продажу. А в 2016 году Новгородский
музей-заповедник приобрёл коллекцию игрушек у Санкт-Петербургского музея
игрушки, которая насчитывает около 16 тысяч экземпляров. Хранители до сих пор
работают над её систематизацией. Каждую вещь надо описать, поставить на учёт.
Работы ещё очень много.

Среди игрушек, хранящихся в Новгородском музее-заповеднике, немало елочных
украшений и различных атрибутов, связанных с празднованием Нового года и
Рождества. Среди них есть как зарубежные, так и отечественные. Сейчас хотелось
бы рассказать именно о последних. 

Если расставлять предметы из коллекции музея по старшинству, в первую очередь
стоит упомянуть ватные ёлочные игрушки. Они появились в России в XIX веке. Их
делали в семьях вместе с детьми, а в начале XX века такие игрушки начали
выпускать артели и кустарные мастерские. Женские журналы по домоводству,
рукоделию и искусству публиковали подробные инструкции о том, как сделать
игрушку из ваты, а в магазинах продавались вырубные листы с лицами ангелов,
детей, Дедов Морозов. Эти картинки можно было вырезать и наклеить на
самодельную игрушку. 

\ii{24_12_2021.stz.news.ru.novgorod.1.istoria_igrushek.pic.2}

Массовое производство ватных ёлочных игрушек началось в 1930-е годы. Для того
чтобы сделать такое украшение, металлический каркас оборачивали ватой, а затем
покрывали крахмальным клейстером со слюдой и посыпали стеклянным снегом. Его
делали так: выдували шар с тончайшими стенками и сбрасывали в специальный ящик,
где он рассыпался на мелкие осколки. Лица игрушек изготавливали из мастики,
папье-маше или воска, а затем грунтовали, окрашивали и вручную рисовали глаза,
брови, губы, румянили ватным тампоном щёки.

\begin{zznagolos}
В сентябре 1929 года Совет Народных Комиссаров СССР принял постановление, по которому Новый год и религиозные праздники, которые до этого были выходными, стали обычными рабочими днями. Их не запретили, а просто превратили в трудовые будни. Праздник начал постепенно забываться, но затем вернулся в новом качестве, уже не связанный с Рождеством. 28 декабря 1935 года в газете «Правда» было опубликовано письмо первого секретаря Киевского обкома П. Постышева «Давайте организуем к Новому году детям хорошую ёлку». Всего через два дня наряженные ёлки стояли уже по всей стране.	
\end{zznagolos}

Светлана Николаева — старший научный сотрудник Новгородского музея-заповедника,
один из хранителей фонда игрушки. Она показывает коробку с фигурками,
сделанными из ваты. Вот девочка с воздушным шаром, вот пионер, вот ребёнок на
санках...

\begin{zznagolos}
— Ёлочные игрушки делали для детей, поэтому здесь в первую очередь то, что было
им близко и понятно. Среди игрушек много детских фигурок. Также есть фигурки
танцоров в национальных костюмах. Есть представители разных профессий:
почтальон, ткачиха, дворник, — рассказывает Светлана Николаева.	
\end{zznagolos}

В 1935-1936 году фабрики начали выпускать ёлочные игрушки с советской
тематикой: с серпом и молотом, с пятиконечными звёздами. В 1950-е стали
популярны монтажные украшения из бусин, полых трубочек, шариков, а в 1960-е —
игрушки на прищепках-подставках. Каждая эпоха несла с собой новые символы. С
развитием сельского хозяйства на советских ёлках появились стеклянные овощи и
фрукты. Самой популярной была «царица полей» — кукуруза. А после первого в мире
пилотируемого полёта в космос начался выпуск игрушечных ракет и космонавтов.
Есть такие и в коллекции Новгородского музея-заповедника. Крохотная фигурка
космонавта входит в набор для украшения ёлочки-малютки. 

\begin{zznagolos} 
Когда в 1935 году Новый год вернулся как праздник, в одних детских учреждениях
его организация не вызвала никаких вопросов, другие же просили подробных
разъяснений о том, как организовать ёлку. В 1936 году вышел сборник «Ёлка»,
который учитывал опыт московских детских садов, наблюдения и рассказы 200
педагогов. В нём рассматривались разные темы, в том числе вопрос о том, как
украшать ёлку. «Основной принцип украшения — легкость, ажурность, блеск и свет.
Цели декоративные здесь основные. Украшая ёлку, следует более крупные игрушки
вешать ниже и вглубь ветвей, более мелкие и лёгкие вверх и по концам веток. Это
создаёт лёгкость, ажурность», — говорилось в рекомендациях.
\end{zznagolos}

Эту ёлочку, к слову, можно назвать ещё одной приметой времени. В 1951 году
Московская фабрика стеклянных ёлочных украшений представила набор для детей
«Малютка» на конкурс, где рассматривались лучшие образцы игрушек со всей
страны. Маленькая ёлочка так понравилась жюри, что коллектив фабрики получил
поощрительную премию — тысячу рублей. Набор рекомендовали внедрить в
производство.

\ii{24_12_2021.stz.news.ru.novgorod.1.istoria_igrushek.pic.3}

Выпуск ёлок-малюток наладили в 1953 году. Стоил такой набор 15 рублей.
Постепенно он стал популярным. Было время, когда без подобной ёлочки не
обходилась ни одна международная выставка. Стенд с украшенной ёлочкой вполне
мог соседствовать с огромными турбинами. Такие же ёлочки дарили гостям
выставок. Появились они и в кино: в фильме Эльдара Рязанова «Карнавальная ночь»
миниатюрные ёлочки украшают столы в зале, где герои празднуют Новый год.

\begin{zznagolos}
Сборник «Ёлка» 1935 года рекомендовал «давать на ёлке возможно больше игрушек
новой тематики, увлекательные для ребят образы строительства, героики». Однако
при этом авторы советовали не увлекаться и не наряжать ёлку только «классово
выдержанными» игрушками, забывая о традиционных. «Совсем недавно один автор
принёс в Комитет по игрушке набор ёлочных украшений. Набор неплохой. В
декоративном и световом оформлении даёт близкие современные образы: дирижабль,
фабрика, Днепрогэс, военное судно, метро и т. д. Для ребёнка будет приятно
найти на ёлке эти новые предметы. Неверно в идее автора лишь то, что в
методической записке он противопоставляет свою тематику «старой» и предлагает
дать ёлку с одной классовыдержанной тематикой», — написано было в сборнике. 	
\end{zznagolos}

Находилось место на ёлках и персонажам детских книг, например, Красной шапочке,
Гадкому утёнку, попугаю из сказки Корнея Чуковского «Доктор Айболит». В фондах
Новгородского музея-заповедника есть целый набор с героями сказочной повести
«Приключения Чиполлино» итальянского писателя Джанни Родари. 

\ii{24_12_2021.stz.news.ru.novgorod.1.istoria_igrushek.pic.4}

Так как коллекция ёлочных игрушек, хранящаяся в фондах музея, ещё не до конца
разобрана, её пока не демонстрировали на отдельной выставке. Однако часть
дореволюционных игрушек побывала на выставке в Коломенском, а некоторые
предметы из коллекции можно было увидеть в Детском музейном центре. 

К слову, на следующий год запланировано большое и важное событие. В церкви
Жён-Мироносиц на Ярославовом дворище должен открыться Музей игрушки. Часть
экспозиционного оборудования уже закуплена. Со временем в здании должна
появиться постоянно действующая выставка. 
