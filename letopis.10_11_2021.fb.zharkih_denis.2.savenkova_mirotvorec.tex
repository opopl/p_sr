% vim: keymap=russian-jcukenwin
%%beginhead 
 
%%file 10_11_2021.fb.zharkih_denis.2.savenkova_mirotvorec
%%parent 10_11_2021
 
%%url https://www.facebook.com/permalink.php?story_fbid=3119565354923527&id=100006102787780
 
%%author_id zharkih_denis,savenkova_faina
%%date 
 
%%tags donbass,lnr,mirotvorec_sajt,savenkova_faina,ukraina
%%title Фаина Савенкова: «Миротворец» просто сайт, которым руководят преступники, разжигающие вражду между людьми
 
%%endhead 
 
\subsection{Фаина Савенкова: «Миротворец» просто сайт, которым руководят преступники, разжигающие вражду между людьми}
\label{sec:10_11_2021.fb.zharkih_denis.2.savenkova_mirotvorec}
 
\Purl{https://www.facebook.com/permalink.php?story_fbid=3119565354923527&id=100006102787780}
\ifcmt
 author_begin
   author_id zharkih_denis,savenkova_faina
 author_end
\fi

Фаина Савенкова: «Для меня «Миротворец» просто сайт, которым руководят
преступники, разжигающие вражду между людьми».

Внесение на сайт «Миротворец» 12-летнего ребенка Фаину Савенкову вызвало
большой резонанс во всем мире. Высказалось очень много людей, но интересно, что
думает сама девочка писатель и драматург, у которой совсем недавно был
тринадцатый День рождения. Сразу скажем, что Фаина выражалась намного
корректнее своих великовозрастных оппонентов. Впрочем, судите сами.  

\ifcmt
  ig https://scontent-frx5-1.xx.fbcdn.net/v/t39.30808-6/256476550_3119565311590198_4538178516532834257_n.jpg?_nc_cat=100&ccb=1-5&_nc_sid=730e14&_nc_ohc=kBoWoxt2WZYAX9cQm9K&_nc_ht=scontent-frx5-1.xx&oh=5499c784ba085cdbc22794d5cff7ca77&oe=619349A3
  @width 0.4
  %@wrap \parpic[r]
  @wrap \InsertBoxR{0}
\fi

Фаина, занесение тебя на «Миротворец» привело к международному скандалу. Как ты
сама отнеслась к тому, что твое имя оказалось там? Что почувствовала в первый
момент?

Привет, Денис. Чувства немного противоречивые. Я знаю, что данные многих моих
друзей есть на этом сайте. Да и само по себе забавно, когда тебя воспринимают
как того, кто угрожает «безопасности человечества». Чувствую себя Дартом
Вейдером каким-то. В остальном – это, конечно, верх бесчеловечности и
беззакония. Я сейчас учусь в 8 классе, и уже не первый год нам преподают
«Обществознание», где мы изучаем общество и, в том числе, основные права людей.
К тому же мне с самого детства объясняли, что никогда нельзя говорить
посторонним людям свой адрес или номер телефона. Это основа безопасности любого
ребенка и, думаю, все родители говорят своим детям то же самое. По крайней мере
должны говорить. И в это же время существуют сайты, выкладывающие твои данные и
данные твоих близких. Это просто ужасно. Нас в школе учили, что вину должен
доказывать суд, но оказалось, что не везде это так. Есть люди, которые
почему-то решили, что они выше закона своего государства. А также что все, кто
думают не так, как они – являются врагами. Сначала я, конечно, очень удивилась,
но потом все-таки огорчилась. Сейчас уже отношусь к этому спокойно и
единственное желание – сказать, что «Миротворец» обычные обманщики,
использующие чувства и убеждения людей в своих целях. Знаете, это как в Средние
века было: не понравился кто-то – один намек инквизиции на связи с дьяволом и
все, домой этот человек уже не вернется. Разбираться никто не будет, на костре
все одинаково горят.

Меня больше всего злит не сам факт внесения в эту базу данных, а то, что они
нарушили несколько законов своей страны.

Что тебя поразило более всего в характеристике «Миротворца»?

Наверное, меня поразило больше всего то, что работающие на этом сайте люди,
придумывают фейки и вообще не знают, о чем пишут. Ведь нормальный человек,
прочитав все эти гадости, покопается в интернете, немного узнает о человеке и
поймет, что все написанное на «Миротворце» - ложь. Я в своем посте в начале
этой истории все рассказала и разложила по полкам. Там даже обвинения
феерические. Не будем касаться слов о пропагандисте и жертве в одном лице, это
еще пустяки. По версии сайта меня внесли за то, что я участвовала фестивале
фантастики в Донецке и написала книгу с Александром Сергеевичем Конторовичем
(об этом говорилось в статье сайта Главком). Пусть и не очень адекватное
обвинение, но вроде всё правильно и честно. Вот только есть у меня тут два
вопроса к болванам с «Миротворца»: если вы внесли меня на свой сайт за это, то
почему Вы сделали это 9 августа 2021? На тот момент книга «Стоящие за твоим
плечом» еще не вышла, а фестиваль был в конце сентября 2021. Наверное, там
работают провидцы. Ну, или обычные вруны. А если за участие в том же фестивале
в 2019 и 2020 годах, то еще удивительнее обвинения. Почему сразу не внесли? В
2019 и 2020 году претензий не было? Так, а-а-а? Простите? На тот момент,
повторюсь, фест 2021 года еще не проводился и тогда при чем он здесь вообще? 

Что можно сказать о тех людях, которые это писали? Зачем они это сделали?

Я думаю, они просто не подумали. Когда информацию о людях вносят на
«Миротворец», то обычно кто-то гордится, а кто-то молчит, чтобы не было
неприятностей. Каждых можно понять. В моем случае всё куда проще: я не
воспринимаю «Миротворец» сам по себе как что-то устрашающее. А гордость… Ну, не
знаю. Для меня это просто сайт, которым руководят преступники, разжигающие
вражду между людьми. А с преступниками нужно бороться. Всё. Поэтому я не
промолчала и написала письмо-обращение к ООН, ЮНИСЕФ и президенту Украины.
Сайт, который заменяет суд и государство, обманывает и призывает к насилию –
надоел всем. Поэтому меня поддерживают и на Украине.

Как развивались события потом, твое к ним отношение?

События развивались стремительно. Многих удивило и шокировало, что меня в 12
лет вписали во враги Украины. Откликнулись ЮНИСЕФ, СМИ, политики в Европе,
России и Украине, за что им большое спасибо. Ведь там не только я, но и
множество других детей. Кстати, пусть и с помощью этого интервью, но хотела бы
попросить не называть меня «жертвой пропаганды», потому что я не призываю к
войне, а всего лишь пишу о необходимости прекращения войны. А в целом, конечно,
я рада, что на проблему все-таки обратили внимание. Повторюсь: с преступниками
нужно бороться. Многие говорят, что всё это политика и видят только «врага
Украины и человечества», кого-то смешит мой кот, оказавшийся на одном из фото
(надеюсь, его уже убрали), кто-то считает, что это пиар и ничего страшного во
внесении на «Миротворец» нет. Но все забывают о самом главном, но неочевидном:
персональные данные ребенка в открытом доступе. Не страшно, говорите? Кто
согласится предоставить всем, совершившим преступления против детей,
фотографии, адрес, номер телефона уже своих детей и внуков? Мало желающих,
правда? А предоставляя данные «Миротворец» занимается именно помощью в
совершении подобных преступлений. Поэтому на самом деле я очень рада, что
отреагировали именно в ЮНИСЕФ.

Тебя неоднократно обвинили в ненависти к Украине. Правда ли это?

Тут можно спорить долго, но отвечу так. Если бы я была «янычаркой Кремля», как
меня назвал Зайцев в своем ответе от имени «Миротворца», то у меня бы не было
так много друзей на Украине. Не было бы упоминаний и интервью в конце концов.
Насколько вы помните, мое эссе «Детский смех победы» появился на украинских
каналах еще летом 2020, даже есть перевод на украинский. Да что там, не было бы
такой поддержки из-за ситуации с сайтом. Сложно ведь поддерживать того, кто
тебя ненавидит.

Как считаешь, любят ли Украину те, кто создал и поддерживает «Миротворец»?

Я не знаю, любят они Украину или нет. Для меня любовь – это когда делаешь
что-то хорошее для того, кого любишь. А «Миротворец» только провоцирует
скандалы и показывает Украину всему миру государством, в котором закона не
существует, где ненавидят всех несогласных. И точно нельзя заменять суд такими
сайтами.

Как думаешь, почему «Миротворец», несмотря на международный скандал, закрыть
нельзя, а общенациональные каналы можно без решения суда, и даже внятного
обвинения? Насколько это соответствует европейским стандартам?

Я не политик и мне сложно говорить об этом. Думаю, это просто страх. Мне
кажется, они больше даже боятся не того, что люди узнают правду, они сами себе
боятся признаться в том, что неправы. Поэтому и не хотят лишних напоминаний об
этом.

Чем должна закончиться твоя история с «Миротворцем»?

Что ты хочешь сказать и пожелать украинцам по ту сторону разграничения?

Я не знаю, чем закончится история с «Миротворцем», знаю только, что права. У
меня нет ненависти к Украине и кому бы то ни было еще. У меня много друзей и в
России, и на Украине, и Европе. Надеюсь, так и будет. Конечно, начались попытки
представить меня злом во плоти, орудием Кремля, но всё это так нелепо. Но этим
людям, наверное, так удобней. Всегда сложно признавать свои ошибки, проще
обвинить во всех бедах кого-нибудь другого. Пусть и ребенка.

А пожелать я бы хотела в первую очередь мира и здоровья. Всем!

\ii{10_11_2021.fb.zharkih_denis.2.savenkova_mirotvorec.cmt}
