% vim: keymap=russian-jcukenwin
%%beginhead 
 
%%file 27_07_2020.stz.news.ua.mrpl_city.1.rodzynka_mrpl_teatru_monovystava_virochka
%%parent 27_07_2020
 
%%url https://mrpl.city/blogs/view/rodzinka-repertuaru-mariupolskogo-teatrumonovistava-virochka
 
%%author_id demidko_olga.mariupol,news.ua.mrpl_city
%%date 
 
%%tags 
%%title Родзинка репертуару маріупольського театру - моновистава "Вірочка"
 
%%endhead 
 
\subsection{Родзинка репертуару маріупольського театру - моновистава \enquote{Вірочка}}
\label{sec:27_07_2020.stz.news.ua.mrpl_city.1.rodzynka_mrpl_teatru_monovystava_virochka}
 
\Purl{https://mrpl.city/blogs/view/rodzinka-repertuaru-mariupolskogo-teatrumonovistava-virochka}
\ifcmt
 author_begin
   author_id demidko_olga.mariupol,news.ua.mrpl_city
 author_end
\fi

\ii{27_07_2020.stz.news.ua.mrpl_city.1.rodzynka_mrpl_teatru_monovystava_virochka.pic.6.feliks_kvytok}

Театральний сезон нашого драматичного театру завершився 24 липня, але глядачі
вже чекають з нетерпінням на новий. Особливо, щоб потрапити на унікальну
прем'єру, яка стала справжньою родзинкою репертуару. Йдеться про моновиставу
\enquote{Вірочка} за оповіданням класика світової літератури А. П. Чехова. Не всім
маріупольцям пощастило потрапити на прем'єру, адже кількість місць була
обмеженою через карантин. Однак найбільші поціновувачі театрального мистецтва
встигли придбати квитки, а хто не встиг, намагався потрапити на виставу всіма
можливими способами. Наприклад, маріупольський поет і театрал Фелікс стояв під
театром з написом: \emph{\enquote{Віддам душу за квиток на \enquote{Вірочку}}!!!}. І все ж завдяки
небайдужим маріупольчанкам, \emph{\textbf{Оксані Стоміній та Наталі Дєдовій}}, чоловік потрапив
на найочікуванішу театральну прем'єру цього літа. Феліксу пощастило, що
директор театру – \emph{\textbf{Володимир Володимирович Кожевніков}} – відрізняється не тільки
мудрістю, а ще й розумінням та людяністю. Всі ж, хто не потрапив на виставу,
сподіваються побачити її восени і дуже вірять, що карантин їхнім планам не
завадить. Я теж не встигла придбати квиток на прем'єру, хоча подивилася
спектакль на генеральному прогоні, за що безмежно вдячна керівництву і
адміністрації театру.

\ii{27_07_2020.stz.news.ua.mrpl_city.1.rodzynka_mrpl_teatru_monovystava_virochka.pic.7}

Моновистава \enquote{Вірочка} недарма створила такий ажіотаж. По-перше, це
класика, яка завжди залишається актуальною. По-друге, це не просто неймовірний
Чехов, майстер художньої деталі, це Антон Павлович українською мовою,
оповідання якого переклала і адаптувала заслужена артистка України, відома в
усій країні режисерка, яка приїхала ставити спектакль за запрошенням
керівництва театру – \emph{\textbf{Людмила Леонідівна Колосович}}. По-третє, це моновистава
молодої і яскравої актриси \emph{\textbf{Віри Шевцової}}, талант якої вже давно став приємним
відкриттям для всього міста. Все вищенаведене ще до прем'єри пророкувало
виставі успіх. Робота проходила динамічно і цікаво як для режисерки, так і для
актриси. Сама Віра Шевцова підкреслила, що зі своєю героїнею вона не дуже
схожа, адже актриса не настільки наївна і романтична натура. А от Людмила
Леонідівна не розуміє, чому героїню вистави не покохав Огнєв, що ж з ним було
не так. За словами режисерки, це біда сучасної молоді. Дівчата знають чого
хочуть, а у хлопців в голові одні комп'ютери...

\ii{27_07_2020.stz.news.ua.mrpl_city.1.rodzynka_mrpl_teatru_monovystava_virochka.pic.8}

У виставі гармонійно поєднані гумор та драма людських стосунків, тому за час її
перегляду глядач і сміється, і співпереживає героїні. Цікаво, що оповідання
\enquote{Вірочка} сам Чехов дуже любив. Свого часу російський прозаїк В. Білібін
зауважив у листі до Антона Павловича: 

\begin{quote}
\em\enquote{Мені здається, що Ви хотіли зобразити не
приватний випадок, описаний вище, а особливий розряд сучасних людей, в яких
через життєві обставини, серце засохло, як квітка в... лексиконі. Вони жадають
кохати, але не можуть...}. 
\end{quote}

І дійсно наприкінці вистави я не розуміла кого шкода
більше, відторгнуту Вірочку або Івана Олексійовича. Все ж більше співчуваю
Огнєву, а не сміливій і щирій головній героїні, яка обов'язково стане щасливою.
Вираз Івана Огнєва: \emph{\enquote{Навіщо вона пішла зі мною, тепер її ще й проводжали треба}}
викликав не тільки шалений сміх у публіки, а ще й багато роздумів. Начебто в
цих думках є і благородство, але таке вимушене і небажане, що здається людина у
свої 29 років втратила смак життя. Протягом вистави мене турбувало лише одне
питання, як така романтична і смілива дівчина могла закохатися в чоловіка зі
спустошеною душею... 

\ii{27_07_2020.stz.news.ua.mrpl_city.1.rodzynka_mrpl_teatru_monovystava_virochka.pic.9}

Для Віри Шевцової це перша моновистава, яку вона готувала з новим для себе
режисером, але актриса впоралася більше, ніж, мабуть, сама сподівалася. Вона
виконувала одразу три ролі: оповідачки, Вірочки і Огнєва і змогла розкрити
характери героїв, з гумором і трагізмом передати цю ліричну і драматичну
історію нездійсненного кохання. Дуже талановита, тонка, наповнена сенсом гра!
Ще більш яскравою виставу зробила поява \emph{\textbf{Ярослава Прикопа}}, не тільки його гра на
гітарі, а ще й кумедні доповнення до Віриних слів. Режисура Колосович вразила –
повна органіка з почуттям стилю і міри. Створені яскраві образи, талановита
гра, старовинні романси, і дивовижно продумані елементи-вкраплення, що додали
моновиставі гумору і продемонстрували всю глибину почуттів, які може відчувати
людська натура, залишили приємний післясмак і можливість повертатися до
побаченого та аналізувати це дійство знову. Моновистава \enquote{Вірочка} стала не
тільки достойним завершенням театрального сезону, а ще й новою, значною
сходинкою Донецького академічного обласного драматичного театру (м. Маріуполь).
Залишається побажати не зупинятися на досягнутому і продовжувати приємно
вражати своїх вдячних глядачів.

\textbf{Читайте також:} \emph{Мариупольский драмтеатр завершил сезон: главные достижения и планы на будущее}%
\footnote{Мариупольский драмтеатр завершил сезон: главные достижения и планы на будущее, Богдан Коваленко, mrpl.city, 25.07.2020, \par%
\url{https://mrpl.city/news/view/mariupolskij-dramteatr-zavershil-sezon-glavnye-dostizheniya-i-plany-na-budushhee}
}

\emph{Фото Оксани Стоміної (Фелікс) та Лева Сандалова}

% 1-5
\ii{27_07_2020.stz.news.ua.mrpl_city.1.rodzynka_mrpl_teatru_monovystava_virochka.pic.1}
