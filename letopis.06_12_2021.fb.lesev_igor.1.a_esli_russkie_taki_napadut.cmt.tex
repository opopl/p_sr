% vim: keymap=russian-jcukenwin
%%beginhead 
 
%%file 06_12_2021.fb.lesev_igor.1.a_esli_russkie_taki_napadut.cmt
%%parent 06_12_2021.fb.lesev_igor.1.a_esli_russkie_taki_napadut
 
%%url 
 
%%author_id 
%%date 
 
%%tags 
%%title 
 
%%endhead 
\subsubsection{Коментарі}

\begin{itemize} % {
\iusr{Oleg Voloshin}

Если честно, как-то очень путано вышло. И причём тут Крым? Никого он реально не
волнует. Тех, кто ненавидел Россию до 2014, года не удовлетворил бы его
возврат, а остальным плевать на самом деле.

И вот то, что под боком у великой державы уже 8 лет существует слабая страна,
единственной целью которой, судя по действиям ее властей, является нанесение
всяческого вреда этой великой державе, то вот и единственный логичный мотив
разрубить Гордиев узел. Военное освоение территории Украины - причина ударить
на упреждение. Никто не станет ждать, пока здесь реально появятся американские
и британские базы. Если гарантировать России отказ от таких планов, у Москвы
исчезнет окончательно и без того весьма слабое желание трогать это болото. Все
равно Украина обречена на угасание.

\begin{itemize} % {
\iusr{Людмила Левченко}
\textbf{Oleg Voloshin} 

А зачем отказываться? Чем они мешают \enquote{великой державе} с бедным народом?
Нападут? Не смешите мои тапки, сама Россия в это не верит!

\iusr{Oleg Voloshin}

\textbf{Людмила Левченко} 

тянут сюда военных НАТО, все время грозят начать наступление на ДНР,
провоцируют учения возле Крыма, разместили здесь всю возможную разведывательную
инфраструктуру США и Британии и тд. Этого мало? В России угрозу использования
территории Украины для создания давления на РФ рассматривают более чем
серьезно. Поставить ракеты с подлётным временем до Кремля 8 мин и потом можно
прямо диктовать России условия. Военные мыслят не намерениями, а потенциалом.
Если Вы не враги, то зачем тащите под наши границы свою военную инфраструктуру?
Вот их вопрос американцам. Представьте российские ракеты и базы на Кубе. Уже
ведь был Карибский кризис, который едва не завершился ядерной войной.

\end{itemize} % }

\end{itemize} % }
