%%beginhead 
 
%%file 20_01_2019.fb.mariupol.kulturna_toloka.1.art_vstrechi_vezha_creative_space
%%parent 20_01_2019
 
%%url https://www.facebook.com/100029107552223/posts/pfbid02JyJkEwLTFskosyLPvGGEHSMhEVoYQycAb9c8LV33UCioXArnZVYLqxnfBRChdFH4l
 
%%author_id mariupol.kulturna_toloka
%%date 20_01_2019
 
%%tags 
%%title Арт-встречи в Vezha Creative Space
 
%%endhead 

\subsection{Арт-встречи в Vezha Creative Space}
\label{sec:20_01_2019.fb.mariupol.kulturna_toloka.1.art_vstrechi_vezha_creative_space}

\Purl{https://www.facebook.com/100029107552223/posts/pfbid02JyJkEwLTFskosyLPvGGEHSMhEVoYQycAb9c8LV33UCioXArnZVYLqxnfBRChdFH4l}
\ifcmt
 author_begin
   author_id mariupol.kulturna_toloka
 author_end
\fi

В Vezha Creative Space стартовал еще один интересный проект, а именно
проведение арт-встреч, посвященных различным направлениям в искусстве. Такие
мероприятия интересны довольно узкому кругу мариупольских эстетов и ценителей
прекрасного, желающих расширить свой кругозор. 

Первая встреча, которая состоялась в прошлую субботу, была посвящена искусству
прерафаэлитов. Об английских поэтах и художниках второй половины ХІХ века
рассказала преподаватель Мариупольской школы искусств Наталья Головенкина. 

Название \enquote{прерафаэлиты} отражало духовное родство с художниками эпохи
Возрождения, которые творили задолго до Рафаэля и Микеланджело.
