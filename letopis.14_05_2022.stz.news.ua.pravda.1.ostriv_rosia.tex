% vim: keymap=russian-jcukenwin
%%beginhead 
 
%%file 14_05_2022.stz.news.ua.pravda.1.ostriv_rosia
%%parent 14_05_2022
 
%%url https://www.pravda.com.ua/articles/2022/05/14/7345856
 
%%author_id dubinjans'kij_mihajlo,news.ua.pravda
%%date 
 
%%tags 
%%title Острів Росія
 
%%endhead 
 
\subsection{Острів Росія}
\label{sec:14_05_2022.stz.news.ua.pravda.1.ostriv_rosia}
 
\Purl{https://www.pravda.com.ua/articles/2022/05/14/7345856}
\ifcmt
 author_begin
   author_id dubinjans'kij_mihajlo,news.ua.pravda
 author_end
\fi

\ifcmt
  ig https://img.pravda.com/images/doc/d/e/de8c965-dubi-ostrov-ros.jpg
  @wrap center
  @width 0.8
\fi

Пушкин, Толстой, Достоевский, Булгаков. Чаще всего украинская дискуссия о
русской литературе, ее значении и ее стигматизации после 24 февраля
ограничивается именно этими четырьмя фамилиями.

Но если уж на то пошло, самое актуальное российское сочинение в 2022 году – это
не \enquote{Война и мир}, \enquote{Преступление и наказание} или
\enquote{Мастер и Маргарита}. Это \enquote{Остров Крым} Василия Аксенова.

О нашумевшей антисоветской книге, написанной в конце 1970-х, вновь вспомнили
восемь лет назад, во время российской аннексии Крыма. Тогда журналистов
привлекла география романа и совпадение фамилии автора с фамилией
новоиспеченного крымского гауляйтера.

Однако в действительности речь в романе идет вовсе не об украинском
полуострове, который авторская фантазия превратила в остров, населила
белоэмигрантами и сделала неким аналогом Тайваня. Нет, это книга о российском
государстве и его потенциальной способности быть другим.

\begin{aanagolos}
  
\end{aanagolos}
