% vim: keymap=russian-jcukenwin
%%beginhead 
 
%%file 08_09_2018.stz.news.ua.mrpl_city.1.k_240_letiu_mariupolja_perekrestok_na_torgovoj
%%parent 08_09_2018
 
%%url https://mrpl.city/blogs/view/k-240-letiyu-mariupolya-perekrestok-na-torgovoj-1
 
%%author_id burov_sergij.mariupol,news.ua.mrpl_city
%%date 
 
%%tags 
%%title К 240-летию Мариуполя: перекресток на Торговой
 
%%endhead 
 
\subsection{К 240-летию Мариуполя: перекресток на Торговой}
\label{sec:08_09_2018.stz.news.ua.mrpl_city.1.k_240_letiu_mariupolja_perekrestok_na_torgovoj}
 
\Purl{https://mrpl.city/blogs/view/k-240-letiyu-mariupolya-perekrestok-na-torgovoj-1}
\ifcmt
 author_begin
   author_id burov_sergij.mariupol,news.ua.mrpl_city
 author_end
\fi

Места, знакомые с раннего детства. Здесь промчались первые тридцать лет жизни...
Детство. Налево – путь в школу. Вверх - в кинотеатр \enquote{Родина}. Направо – к
новому рынку на Сенной площади. Вниз - домой. Помнится Первое Мая сорок пятого
года. Единственное украшение улицы – гирлянда флажков из обложек старых
тетрадей, протянутая от столба до столба. По радио сводки Совинформбюро.
Вот-вот закончится война. И вот Победа!!!

Каждый дом будит воспоминания. Торговая улица, дом №25. На втором этаже жил
\textbf{Сергей Михайлович Дудицкий}, требовательный санврач, а в остальном - простой и
обаятельный человек. Из разговоров взрослых. С первых дней войны Сергей
Михайлович был призван в армию. Служил в одном из мариупольских госпиталей. 8
октября 1941 года, у переправы через Кальмиус, автобус, в котором военврач III
ранга Дудицкий сопровождал раненых и больных бойцов Красной Армии, обстреляли
немецкие автоматчики. Ходячие бойцы разбежались кто куда, на ходу срывая знаки
различия. Дудицкий остался с теми, кто не мог не то что бежать, но и ходить.
Так доктор попал во вражеский плен. Филиал ада – лагерь военнопленных близ
шлаковой горы. Оккупанты потребовали от предателей, которых они поставили
руководить городской управой, немедленно организовать санитарную службу. Они
боялись эпидемий. И тогда вспомнили довоенного главного врача
санитарно-бактериологической станции Дудицкого, и... вызволили его из концлагеря.

\textbf{Читайте также:} 

\href{https://mrpl.city/news/view/v-mariupole-zapustili-blagotvoritelnyj-chelendzh-dlya-spaseniya-zhizni-rebenka}{%
В Мариуполе запустили благотворительный челендж для спасения жизни ребенка, Олена Онєгіна, mrpl.city, 07.09.2018}

Улица Торговая, дом №23. В одной из квартир жил с семьей \textbf{Петр Никитич
Слепканев}. Сын рабочего, прошедший путь от рабочего, до главного
сталеплавильщика завода \enquote{Азовсталь}, а между этими должностями - вечерний
рабфак, затем Мариупольский металлургический институт, мастер, начальник смены,
начальник мартеновского цеха. Он принимал непосредственное участие в разработке
технологии выплавки стали с использованием чугунов с высоким содержанием
фосфора, перевод мартеновских печей на природный газ. В годы Великой
отечественной войны он работал на одном из металлургических заводов Урала, где
изготавливали военную продукцию.

\ii{08_09_2018.stz.news.ua.mrpl_city.1.k_240_letiu_mariupolja_perekrestok_na_torgovoj.pic.1}

Будто знакомые и вместе с тем неузнаваемые виды. Печальные знаки дряхления
улицы. Окна, выходящие на улицу, не просто заложены кирпичом, а еще
заштукатурены. Остатки номерного знака образца 30-х годов прошлого века. Там,
где стояли створчатые ворота, - пустота. Стены с обвалившейся штукатуркой. Что
еще отягощает впечатление заброшенности?

Конечно же, окраска фасадов домов рудной пылью. Эта дизайнерская находка имеет
свою историю. В Мариуполе готовились к молодежным футбольным состязаниям
\enquote{ЕВРО-2009}. Поступила команда \enquote{сверху} привести в порядок улицы, по которым
будут следовать автобусы с командами футболистов. Мол, чтобы не опозориться,
нужно покрасить дома. Какая самая дешевая краска у нас? Рудная пыль! Ею и
покрасили фасады.

\ii{08_09_2018.stz.news.ua.mrpl_city.1.k_240_letiu_mariupolja_perekrestok_na_torgovoj.pic.2}

Светлые воспоминания. Торговая, 30. Здесь находился хлебный магазинчик, в
котором торговала милая девушка по имени \textbf{Мила}. Ее все любили за то, что она
продавала нуждающимся людям свой товар в долг. Там, где сейчас обосновался
ларек по приему вторсырья, в 50-е годы размещался овощной павильон. В конце
августа и в начале сентября привозили помидоры с колхозных и совхозных полей.
Их покупали ящиками. Через час-другой вся округа пропитывалась запахом томатов.
Это рачительные мариупольчанки заготавливали морс для борщей, консервировали
\enquote{синенькие} в томатном соусе и иные изыски местных кулинарок...

\ii{08_09_2018.stz.news.ua.mrpl_city.1.k_240_letiu_mariupolja_perekrestok_na_torgovoj.pic.3}

Увидев человека с фотоаппаратом или телекамерой рядом с домом №27 на улице
Торговой, местные старожилы обязательно скажут, что это дом митрополита
Игнатия. То, что это именно так, один из добровольных экскурсоводов поведал,
что лидер Общества имени Хартахая Игорь Андреевич Налчаджи привозил суда мэра
города, просил его устроить здесь музей митрополита. Осуществить это благое
дело помешало, то обстоятельство, что нужно было дать несколько квартир семьям,
живущим в доме. Что именно в стенах этого здания жил предводитель
греков-переселенцев утверждали и некоторые краеведы-любители. Они ссылались на
довоенную фотографию, на обороте которой по ошибке было написано \enquote{Дом
митрополита}.

Где был дом Митрополита Игнатия по прибытии в наши края? Вспомнилось
прочитанное в старинной книге, что митрополит жил, как рассказывала еврейка
Огуз, в ее доме, а грек Мина Бык утверждал, что большую часть времени
пресвященный находился на даче. Прикинув дату выхода в свет фолианта и время
прибытия митрополита с паствой в Приазовье, стало ясно, что и грек Мина Бык, и
еврейка Огуз родились через 20 -30 лет после кончины митрополита. А то, о чем
они поведали – не более, чем пересказ легенды. Авторитетный историк Раиса
Петровна Божко, на основании документов (!), опровергла версию, что митрополит
Игнатий обитал в доме №27 по Митрополитской улице. И серьезные исследователи
родного края с выводами с Р. П. Божко согласились.

\textbf{Читайте также:} 

\href{https://mrpl.city/news/view/na-vyezde-iz-mariupolya-rozhdayutsya-novye-art-obekty-foto}{%
На выезде из Мариуполя \enquote{рождаются} новые арт-объекты, Ганна Хіжнікова, mrpl.city, 07.09.2018}

Зато никто сейчас не сомневается, что в этом доме жил \textbf{Георгий Антонович
Костоправ}, поэт, драматург, журналист, основоположник литературы греков
Приазовья. Здесь он 23 декабря 1937 года был арестован \enquote{энкаведистами}, а 14
февраля следующего года его расстреляли. Об этом узнано, когда за плечами было
шестьдесят непростых прожитых лет.

Воспоминания сменяют друг друга, мешают друг другу. Перекресток, где было много
народу, сейчас пуст. Прошмыгнула легковушка вниз к \enquote{Азовстали}. Ассоциация:
лязгая по брусчатке траками, движется вниз по Торговой танк, у которого вместо
башни огромный ящик. Стоп. Нужно спешить к остановке, ехать домой. Но где мой
дом? Там или здесь?
