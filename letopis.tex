% vim: keymap=russian-jcukenwin
%%beginhead 
 
%%file _main_
%%parent 
 
%%endhead 
 
%%file f_main

\newif\ifDEBUG
\DEBUGfalse

\newif\ifINDEX
\INDEXtrue

\newif\ifcmt
\cmtfalse

\def\PROJ{letopis}
\def\ii#1{\InputIfFileExists{\PROJ.#1.tex}{}{}}
 
\ii{preamble}
\ii{defs}
 
\begin{document}

%\ii{titlepage}
 
\ii{listfigs}
\ii{listtabs}
\ii{tabcont}
 
\ii{body}
\ii{index}
%\ii{bib}

\end{document}

%ombaranovskyi (at) gmail dot com
%https://www.facebook.com/yaremenko.sergiy/posts/pfbid0rm7p8Qsi41hgemJrkwn7rgMGXSLtLW6ZXAegKGFaFwvhfa5Zx7GLigabcC4rVjJbl
%https://www.facebook.com/desn.rda/posts/pfbid028j7NvqGmhQF9fULm5xmT6GLCvtkuD3vC5V5okompQK8EvLsEx9fqVSskAjwGWvqtl

Доброго вечора. Слава Україні! 

Сьогодні 37-річниця Чорнобильської аварії. Для мене як для киянина... короче,
ціла величезна тема, і я про це напишу окремо. А зараз я хотів би трохи
подумати над таким, оскільки я програміст (про все це я теж напишу більш
докладно). Справа в тому, что ЧАЕС свого часу вважалась найсучаснішим
реактором. По-моєму, академік Александров, чи хтось інший, казав, що
ймовірність аварії на ЧАЕС така, що ... не памьятаю точно його вираз... ну
скажімо така, що ви йдете Хрещатиком і Вам на голову падає астероїд. Короче,
було таке собі шапкозакідательство, що мирний атом ляляля, абсолютно надійно,
товариші фізики все знають, все чьотко. Як воно вийшло насправді (також
враховуючи пізнішу атомну катастрофу на Фукусімі), всі все це знають. А я хотів
тут дещо сказати в розрізі моєї професії як програміста. Ну ок. Тут такий трохи
потік думок... але.. ЧАЕС був продукт свого часу, і всі вважали її абсолютно
надійною, як і інші атомні станції. Але станція вибухнула і спричинила
неймовірно кількість нещасть по всій планеті. От воно як виходить. Вчені щось
створили начебто абсолютно надійне, а воно взяло і вибухнуло. Так от. Я хотів
би сказати пару слів про соцмережі та Інтернет взагалі. Справа в тому, что
зараз у нас іде війна із московією, і однією із суттєвих особливостей цієй
війни, навіть я би так сказав, такою особливістю, яка докорінно відрізняє її
від минулих воєн, як-от друга світова війна, або фінська кампанія, або В'єтнам,
або якась інша війна, є те, що вона також відбувається у кіберпросторі, і також
у ментальному та метафізичному просторах. Кіберпростір - ну тобто є наші
канали, сторінки, і т.д. і є орківські, там, бухі медвєдєви, машки-алкашки і
т.д. Воно може і смішно звучить, аде з їх сторони все це лайно потім влізає в
голови простих росіян, і в результаті отримується така собі армія зомбі, яка
підтримала і продовжує підтримувати вже всі злочини росіян в реальному
просторі. Тобто, смерть, зло, дурість, безумство, руйнування у віртуальному
світі потім призводить до руйнувать і смертей реальних.

І в принципі всі це і так знають, але можливо не всі задумуються над тим, що
Інтернет в глобальному сенсі, і фейбук або телеграм зокрема, є таким собі
досить особливим світом. Окрема Віртуальна Реальність, у якої свої закони,
якісь закономірності і т.д. І насправді, серйозні вчені, навіть фізики та
математики, зараз активно займаються дослідженням Інтернету та соцмереж як
окремого явища,... І т.д. Тут у мене зараз немає багато часу писати. Але.
короче. Внаслідок цієї війни є дуже багато дуже цінної інформації, як-от 

(1) щоденники блокадного Маріуполя, у вигляді фейсбук-сторінок, які самі по
собі є доказом злочинів росіян

(2) всі ті фейсбук-сторінки Маріуполя. до 24.02.2022, є частиною Цифрової
Спадщини Маріуполя, є Конденсованим Згустком Духовної Енергії цього дуже
цікавого, енергійного, та красивого міста, у якому, я на жаль, ніколи не був.
І надійне збереження всієї цієї цифрової спадщини в максимально близькому до
оригіналу форматі є дуже важливою передумовою подальшого духовно-культурного
відродження Міста Марії. Це дійсно так.

(3) інші фейсбук сторінки, альбоми і т.д. і т.д. - тисячі, мільйони записів
різних авторів і т.д. в Україні за різні роки - все це разом є сукупним
Цифровим Образом України, і має величезну цінність саму по собі. 

Ну а тепер. Все це до чого. Справа в тому (обмежимось фб) що все це може бути
зруйновано за мить, внаслідок якоїсь наприклад технічної помилки з боку
сервера. Або. Якийсь топ-модератор десь або програміст там в Каліфорнії напише
програмку і вирішить у свій вільний час трохи... поексперементувати... Ну от
прямо як робітники на ЧАЕС, які проводили (начебто контрольований) експеримент,
який в результаті призвів до глобальної ядерної катастрофи.

А щодо Інтернету та Фейсбуку... короче, почну з простого прикладу. На моєму
ноуті стоїть операційна система Убунту. Там є можлива практично така цікава команда,
називається так

sudo rm -rf /

Пояснення:

sudo - права адміністратора
rm - remove - видалити
/  - кореневий каталог, тобто в ньому міститься все інше
-rf - два ключі - ( -r recurse рекурсивно по директоріям -f force насильно ) 

Короче. Достатньо мені по дурі або по пьяні сісти за ноут, зайти як адмін і
набрати цю команду і натиснути Enter, як все, в принципі все, що важливе для
мене міститься у мене,  на моєму ноуті зникне. Ну просто зникне і все. І усьо.
Але тут в принципі трагедії вселенської немає ніякої. Ну, дурний, витер собі
все на ноуті. Жалко фоточок, там, якихось файлів і усьо. Сам винуватий, пішов
собі, можливо важкими зусиллями відновив частину файлів і т.д. Але. Справа в
тому, что на глобальному рівні - тобто на рівні наприклад фб, а це реально
мільйони - мільярди користувачів по всьому світу - глобальний збій, чиїсь дурні
експерименти на рівні адміністрування всього фейсбуку або чиїсь спеціально
створений вірус може запросто зробити недоступними на тривалий час, або знищити
назавжди, абсолютно безцінні цифрові дані, як от ті ж самі щоденники, які,
звісно, буде неможливо потім відтворити. Також, дуже багато всього управляється
через інтернет, там, АЕС, транспорт і т.д.

Так, тему я задав, а розпишу якось пізніше. Висновок такий.  Чорнобиль - ядерна
катастрофа у фізичному просторі вже стався 37 років назад, хоча в масовій
свідомості атомна енергії було синонімом абсолютно безпечної енергії. Зараз у
нас Цифрова Ера. Ми всі користуємось смартфонами, ноутами, оплачуємо,
спілкуємось, репортажі з фронту, ролики, що роблять самі військові і т.д.
Інтернет, фб, телеграм стали невідьємною частиною нашого життя до такого рівня,
всі вже настільки звикли до цього, що в принципі до-цифрова епоха уявляться
трохи вже як така собі епоха печерних людей. Як це так, а що, дійсно колись
люди не мали фб або навіть смартфона? аааааааааааааа, що ж це таке, щоби
домовитись з дівчиною про побачення, треба було іти до таксофона на вулиці і
дзвонити у гуртожиток і просити, щоби та конкретна дівчина підійшла до
слухавки?!  А якщо раптом телефон в общазі змамався, що ж тоді робити?  ото то
дикі темні часи були!... А колись і того не було, і телефону і телеграфу,
тільки поштові станції між містами, так то напевне взагалі часи динозаврів
були! Також, лист міг йти до адресата цілий місяць, замість того, щоби просто
зробити скайп-дзвінок, - це взагалі неможливо собі уявити, в яких жахливі умови
люди були поставлені! (а за часів Д'артаньяна та трьох мушкетерів, якщо читали,
- Двадцять років опісля, - Д'артаньян взагалі спочатку не міг знайти своїх
старих друзів, тому що він загубив листа від Араміса на полі бою, і все -
зв'язок був утрачений, де той Араміс - в якомусь монастирі... їх тисячі по всій
Франції! - а якшо він у жіночому монастирі стирчить, то що робити тоді?)  Але
Слава Богу, зараз вже все по іншому! - якось хтось десь мені так сказав,
турботливо з посмішкою поглажуючи пальчиками, як якесь маленьке кошеня, свій
найновіший Айфон... Так от. Цифровий Чорнобиль - тобто, глобальний збій Мережі,
фб, телеграму, АБСОЛЮТНО МОЖЛИВИЙ. (в 2020 вже був такий відносно малий збій в
роботі фб та телеграму  - то чому не може статися чогось набагато більш
серйознішого - ну звичайно може!!) І це буде мати абсолютно катастрофічні
наслідки для всього людства. В меншому масштабі, в масштабі України і контексті
нашої перемоги над московією і шаого прагнення покарати росію за всі її
злочини, це призведе до втрати назавжди величезної кількості неймовірно
важливих даних, як от тих самих Маріупольських блокадних щоденників, або ж
памьять про наших загиблих Воїнів, як от Дениса Галушко, який загинув в січні
під Бахмутом, вічна йому Пам'ять!  ( маю на увазі його фейсбук-сторінку - якщо
ця фейсбук сторінка буде втрачена внаслідок якогось технічного збою, ну... це
буде жахливо. Смерть людини це завжди трагедія, але смерть пам'яті - можливо,
ще страшніша річ... - АЛЕ. На щастя, майже 40 постів із сторінки Дениса я
зберіг, і цифровий Чорнобиль конкретно щодо пам'яти цього Героя вже не так
страшний).

А як це все може виглядати в реалі (глобальна цифрова катастрофа, глобальний
Цифровий Апокаліпсис), можете подивитись фільм Міцний Горішок-4 

to be continued

%21:57:12 27-04-23
Андрій Жованик. Світ як воля та уява фільм


Горщик, Горщик і ще Раз Горщик!

Древні кияни вміли досить таки культурно сказати певні речі, так! Короче, я
думаю, от я назбирав собі друзів з Маріуполя, а зараз вони всі побіжуть, бо
мені зараз дуже боляче, аж заплакав, і я скажу те що думаю!

А Маріуполь - це точно Україна???

ОТ ТАКЕ ОТ ПИТАННЯ

Чи це такий собі невеличкий клюб по інтересам, така собі мафія, от цікаво, так!
Всі нам от допомагають, співчувають, допомогають робити виставки в топових
музеях Києва, як от згадаю тут Olga Demidko або ж Оксана Стомина також... КІСІ
(КНУБА) дуже щедро віддав одне із своїх приміщень Маріупольському Університету
на Преображенській, де 1 квітня цього року був проведений день відкритих
дверей... також, маріупольські фотографи із задоволенням відкривають для себе
та інших Київ, його памьятки, вулиці, архітектуру, природу, озера тощо - з
різних абсолютно неймовірних ракурсів, як от Олена Сугак або Evgeny  Sosnovsky
і я впевнений, також дуже багато людей по всій Україні допомагали і допомагають
маріупольціям у всіляких ситуаціях...  і це дуже-дуже добре! але... от...
пустити інших прогосувати за якусь простеньку гарнесеньку чашечку з фотками
Віті Дєдова (яких я назбирав вже сотню напевне в себе... - Царство Небесне та
Вічна Пам'ять Віктору!), то нєєєєєєєєєєєєєєєєєєє

Нікому чужому (боронь Боже, навіть одеситам або ж миколаївським або донецьким
не можна, не кажучи вже про киян, львів'ян або ужгородців) за чашку з нашим
Вітєй право проголосувати не віддамо, так! Вітя наш, наш і тільки наш (с)

А хто вам в 2019 році для Виставки на Театральній Площі намалював отого
неймовірного Зайчика, якого я недавно знайшов в Музеї Історії Міста Києва, а
зараз він знаходиться в Добропарку разом із іншими його друзяками (серед них
також є інші Маріупольскі кролики), ви не цікавились, так??? А звідки взагалі
оті Кролі та Писанки попали в Маріуполь, Ви теж не цікавились, так??? Або ж
куди свого часу Светлана Мешкова-Давиденко повезла свого неймовірного Зайсю на
Виставку, а повезла вона його на Виставку Кролів та Писанок в Києві, де реально
мільйони (!) людей подивились на отого Маріупольського Зайсю, - серед інших
сотен кролів та писанок, - звідки пізніше, всі ті Кролі та Писанки у кількості
30 писанок та 30 кролів помандрували в Маріуполь в 2019 році, - так, Ви теж тим
не цікавились, так? 

хм.......................

ну ГОРЩИК ГОРЩИК ГОРЩИК І ЩЕ РАЗ ГОРЩИК

Оце от знаєте. Недавно Олена Сугак написала Київ як київ (з маленької літери!
так, я розумію, що палець на клавіатурі помилився, але все ж таки... - ... я от
недавно був у Музеї Чорнобиля, так от. Там була вікторина в кінці. Питання було
- на що радіація найбільше впливає - на печінку чи на кістний мозок. Я
насправді то знав, що кістний мозок... але... палець помилився... Я нажав
печінка - помилка. Правильно - кістний мозок. Розумієте, одна помилочка,
пальчик не туди нажав... а стільки трагічного смислу!!! ). Так от... я трохи
так собі вибухнув... так, від отого києва, якогось собі маленького непримітного
києва, а не КИЄВА, МАТЕРІ ГОРОДАМ РУСЬКИМ, СТОЛИЦІ УКРАЇНИ... Місця, де
знаходяться Лавра, Софія, Кирилівська Церква, де Музей Чорнобиля, Ярославів
Вал, Золоті Ворота, метро Хрещатик та інше інше інше.... де був Майдан-2014,
Революція Гідності, де є Верховна Рада, Президент, міністри, і також... Музей
Історії Києва, Музей Історії України, навіть от Музей Туалету є! Київ - то
дійсно неймовірне місто, я вам скажу, бо де ще в Україні є таке місто із цілим
музеєм туалету, де тебе зустрічають якісь столітні дуПи! Так, приходиш в музей,
а тобі привітно так посміхають дуПи на вході!... Так! столітні дуПи якихось
давно померлих людей часів ще першої світової війни, а не столітні кремезні
дуБи, у нас є деякі такі дуБи, що ще часи Богдана Хмельницького і Максима
Кривоноса застали, так!! І от... Розумієте... от в моїй свідомості немає
ніякого києва, а є КИЇЇЇЇЇЇЇЇЇЇЇЇЇЇЇЇЇЇЇЇВВВВВВВВВВВВВВВВВВВВВВВВІ Отак! Перша
літера - К, друга И - третя Ї (від якої заболотні звірі сходять з розуму у себе
на болотах, бо вони все хочуть КИЕВ, все слюнки пускають, все лізуть і
лізуть... їх вбивають, вже 100 000 вбили, а вони все лізуть і лізуть... але нє,
ніфіга ви не отримаєте, бо поїзд пішов! - KYIV NOT KIEV!), четверта В. КИЇВ.
КИЇВ. КИЇВ. І .... Я від цього скажімо трохи вибухнув, і написав цілий пост з
цього приводу, що не треба писати Київ як київ, бо це КИЇВ, розумієте, так? Ну
не пишіть Київ як київ, що ж тут важко то зрозуміти... Пишіть Київ як Київ, бо
це просто КИЇВ. КИЇВ. КИЇВ.

І... ну добре... я в селі... тут дуже гарно... квіточки, тюльпанчики, поля,
озера, гаї... У мене немає натхнення зараз писати довжелезні пости... Піду собі
спочатку яєчню зроблю з кавою... Але питання моє лишається...

А МАРІУПОЛЬ ЦЕ ТОЧНО УКРАЇНА?

%https://www.facebook.com/permalink.php?story_fbid=pfbid02oYGkbx6P3cDk9yNxF56dkf59oKABfwK4hgBRnvLjGv12fPTgBi1X4LUpYF9osDZDl&id=100087626946702

%https://www.facebook.com/GazetaSvit/posts/pfbid0QVsNh8JS1ySUWQ8NujdxCQtPqcNEHwtw1qbvuqicUN4r9TMoAdnQLmt7gmU38Lojl

% https://www.facebook.com/valentina.zhitanska/posts/pfbid02vN3sKetjJBsEsWWRmAqJmRfDDm1fJUNPgNFi3nYSoxhzyXQ8JWRkmfehwPqCwr2rl
... тут за легендою були яке джерело чи то фонтан, з якого била вода, чи то
часів Київської Русі, чи то ще часів Андрія Первозванного... мені от так якось
сказали, коли я так блукав в цих місцях, трохи далі, де довгий прохід до
фунікулера і зазвичай стоять художники... зараз шукаю ту легенду, не можу
знайти... може підкажете...

Цифровий Пересувний Маріуполь 

У мене є ще знаєте, така ідея... Короче, записувати на DVD. Власне, у мене є
вже купа всього про Маріуполь. От вже напевне Гігабайт 5-10 назбиралось...
короче, я планую собі створювати образи дисків, потім туди заливати,
структурувати по темам, і буде так... Наприклад, DVD-1, - це Блокадний
Маріуполь. DVD-2 - це фото, наприклад на один DVD-2 - про море, та про порт,
історія, фото, події.  DVD-3 - щось інше, наприклад, видатні люди Маріуполя,
весь Нільсен, весь Буров, весь той, весь цей... Он один Сергій Буров скільки
всього зробив і написав! Про лише Сергія Бурова, можна окремий DVD записати! І
буде так... DVD-3-Нільсен, DVD-4-Буров, DVD-5-Хтось-Ще Насправді інформації про
Маріуполь просто безліч, але... поки воно все на фб, по різним людям, які не
синхронізувались в тому, які у кого фотки, щоденники, історії, відео... і де...
то існує велика небезпека, що все це з часом буде просто втрачено. Тому що...
фб ненадійний не тільки тому, що у нього погана цензурна політика, а тому що фб
завжди може впасти (у 2020 році фб і телеграм впали на день чи два глобально,
якщо памьятаєте), і навіть обанкротитись, бо це насправді всього навсього
приватна компанія, яка діє за американськими законами, і яка заробляє гроші і
має певну капіталізацію на біржі. І ера фб колись почалась, і колись
закінчиться... Да..  Ось у Olga Demidko є ідея пересувної виставки ФотоЛітопису
Маріуполя... вона навіть перше місце отримала на конкурсі проєктів, дуже круто!
... а у мене... є ідея... просто пересувного Цифрового Маріуполя. Ну тобто,
папочка з DVD-дисками, на яких скажімо, ну, 25 дисків... записано 100
Гігабайт... короче, весь Маріуполь ось тут, в кармані... (звичайно, навіть 100
Гігабайт мало для такого міста, як Маріуполь... але... певний пристойний
масштаб збереження вже буде)... На фото, - оце от - взяв в группі Маріуполь
Довоєнний щойно, домалював зверху - весь Нільсен в кармані (ідея поки що
тільки, але тут потрібні лише натхнення цим займатись, ... також, час на
збирання, записування, ну і власне все...).

... У Карела Чапека є такий цикл, Розповіді з Одного Карману, Розповіді з
Іншого Карману...

А це значить буде Маріуполь в Кармані... От треба щось друзям показати про
Маріуполь ...  нема проблем, відшукав потрібний диск, пішов на зустріч в
кафешку, потім, дістаєш оті диски, все показуєш, розповідаєш, а друзі всі так і
впали - ого! оце у вас така краса в Маріуполі, а я й не знав!


% КИЇВСЬКІ ПЕРЕМОЖЦІ.

%А ви вже чули, що днями на найстаршому і найпрестижнішому міжнародному
%фестивалі короткометражного кіно в Обергаузені — головний приз отримав
%документальний фільм, створений знімальною командою киян? 

%https://www.facebook.com/groups/story.kiev.ua/posts/2190464541150335/

%17:08:43 08-05-23
% https://www.pravda.com.ua/articles/2023/05/8/7401082/

% friend suggestions
% https://www.facebook.com/friends/suggestions/?profile_id=100001349936974
% https://www.facebook.com/profile.php?id=100007146064665
% https://www.facebook.com/yana.gryniv/posts/pfbid02tJ9H6R9bhgdFUzU6oX7LqopdDcwNukh2CbHH6NZYboz2V5pgxFagHArpLiUHCPEWl

Один вагончик вверх
Один вагончик вниз
Проносяться літа

Один вагончик вверх
А інший - їде вниз

Ну а я... пішов у Ліс...

%https://www.facebook.com/groups/713320479592198/posts/1297485821175658

% Кіпчарський, Віктор (Маріуполь)
% Маріуполь Блокадний
%  1,2,3,


%https://www.facebook.com/permalink.php?story_fbid=pfbid02dXmDFhTtDcDf6JcCsWa5zZMbf4p49e15ipVekh5scT1C3icUetAUPbcrnm1ad4jQl&id=1427894275 
%https://www.facebook.com/1427894275/posts/pfbid02dXmDFhTtDcDf6JcCsWa5zZMbf4p49e15ipVekh5scT1C3icUetAUPbcrnm1ad4jQl

% Ух яке знакове фото прислали: перші вибухи на Хрещатику цього дня 1941 р.!
% https://www.facebook.com/oksana.zabuzhko/posts/pfbid08TJYJV2xQXtDYsrXCBnTEr2VAjnZb54u7K9HZrdA3wH7uBNqSekJdR6MiGch2NJ2l

% Выкрик
Как  хочется  домой   вернуться, По  своей  улице  пройти, И  в  родной   город  окунуться, Друзей  встречая  на  пути.
% https://www.facebook.com/groups/2850411311922693/posts/3160549840908837

% Я не раз вже писала - "ми живемо в підручниках з історії".
% https://www.facebook.com/groups/kyiv.in.photo/posts/6167678153327941

% https://www.facebook.com/groups/story.kiev.ua/posts/2205724472957675
% https://www.facebook.com/groups/2141597379210449/posts/6215587028478110
% https://www.facebook.com/groups/ARTtereveni/posts/2579204052220286

#місце_сили: музей загублених речей / Музей Вячеслава Долженко / Место силы МТВ 

Алевтина Швецова

\href{https://www.facebook.com/lizakrestiankina}{Елизавета Крестьянкина}
\href{https://www.facebook.com/groups/1013559558679300}{750 художників}

ЗБІРНИК НАУКОВИХ ПРАЦЬ

Український письменник Олекса Слісаренко
https://www.calameo.com/read/0006329459710de380aa5

Вогнесвіт Олеся Бердника
https://www.calameo.com/books/004917753c17173956ca5

https://archives.gov.ua/wp-content/uploads/2020/02/pres-reliz_20200226.pdf

https://www.facebook.com/groups/vrajenniya.ua/posts/1903991089959321
https://fakty.ua/421900-naslednica-harlan-prinesla-ukraine-12-e-zoloto-evropejskih-igr

https://violity.com/ua/113464383-mariupol-azovmash-strategiya-uspeha-1958-2008

https://www.facebook.com/vitalij.tymchyshyn
https://www.facebook.com/groups/314070194112/user/100007021411459

https://www.facebook.com/permalink.php?story_fbid=pfbid0EW8CTrTDjL56h48EDXxhvhQgdxPU78fZQ86R2526dwv97Xgm4wTEnhCJb72DRcGXl&id=100004775745586
http://fil11.ucoz.ru/index/knizhnye_vystavki/0-32
http://biography.nbuv.gov.ua/data/data/bibliogr/3054.pdf

Secrets of the Academy of Arts🔥
https://www.facebook.com/groups/story.kiev.ua/posts/2259717277558394

Школа до 1 вересня готова!!!  Щиро дякую всім,  хто долучився до підготовки школи до нового 2019/2020
навчального року! !!!  Вона - НАЙКРАЩА і чекає на своїх вихованців! !!
https://www.facebook.com/groups/1955762347979769/posts/2441522292737103

С Днём Рождения Мариуполь!
https://www.youtube.com/watch?v=w5m_Twxt63o

Греческую площадь с золотым символом в Мариуполе открыли танцами и песнями
https://www.youtube.com/watch?v=jG__HUesIKo

Маріуполь,Україна,Київ,Виставка,Художник,Мистецтво,
Mariupol,Ukraine,Kyiv,Kiev,Exhibition,Artist,Art,
Мариуполь,Украина,Киев,Выставка,Художник,Исскуство


мова болота
https://www.facebook.com/permalink.php?story_fbid=pfbid02AMSLM1JoMnh93gSF7esLZS8uRzZJrnHR6APxQqDGfqCdS1rW9Bt3vyDxr7LfmiuBl&id=1796864000

petrushkina
https://www.facebook.com/Photobooks.by.Tatiana/
https://agency-abo.medium.com/%D0%BC%D0%B0%D1%82%D0%B5%D1%80%D1%96%D0%B0%D0%BB-%D0%BC%D0%BE%D0%B3%D0%BE-%D1%96%D0%BD%D1%82%D0%B5%D1%80%D0%BD%D0%B5%D1%82-%D0%B2%D0%B8%D0%B4%D0%B0%D0%BD%D0%BD%D1%8F-%D0%B2%D0%BA%D1%80%D0%B0%D0%BB%D0%B8-%D1%89%D0%BE-%D1%80%D0%BE%D0%B1%D0%B8%D1%82%D0%B8-5e050f52a81a

https://www.space.com/super-blue-moon-august-2023-one-week-away

Центр сучасного мистецтва і культури ім. А. І. Куїнджі
https://www.facebook.com/watercolorsukrainian/posts/3179817192253520

https://www.facebook.com/100023161160311/posts/pfbid02ogkRFwv324fvfCZQFa5eVeHvNNqQgpxsLJpRX8v9qsV2yKaXeofZgKLvuQtaCyxTl

https://www.facebook.com/nakipelovo/posts/pfbid037DH6i5uZoVNEeF4dX5w9LMDWFjYZJyKZyTbb49oFNo2kEQfdurcQTFL16uxcVbzHl
https://www.facebook.com/halabudavp/posts/pfbid02CpBSKx4YeqkycyBjc4CguK3H92JmCtHAsSPn9Zdxw4PBBfbMomnfjQzSXYTgz7kZl

А ми раді повідомити, що ми розпочинаємо навчання офлайн у Львові і набрали вже першачків
https://www.facebook.com/Mariupolartschool/posts/pfbid0t8VUoXJdKWWXQfywdCsMZbukzBJezm4beJuijA1bEd5gibk5pm2zuzqv7zirBK74l

https://ogolosha.ua/uk/prod/katalog-otkrytki-1922-2004-mariupol-72s-av-142-234-XbGP.html
https://ogolosha.ua/uk/prod/katalog-otkrytki-1898-1917-mariupol-100s-a-142-234-XbGL.html

Очерк о творчестве фоторепортёра газеты "Приазовский рабочий" Павла
Мечиславовича Кашкеля, который создал фотолетопись 40 – 50-х годов нашего
города.

https://web.archive.org/web/20210917140431/https://sigmatv.com.ua/ru/kak-v-mariupole-razvivalas-sfera-mediciny_n39589
https://web.archive.org/web/20210921145646/https://sigmatv.com.ua/ru/kak-izmenilsya-mariupol-za-gody-nezavisimosti_n39760

Мариуполь Былое Chereztynnestrybaylo Pavlo
https://www.youtube.com/playlist?list=PLhw9HrsggmwUdh295L64tHv4Z6_vE9NSm

Маріуполь,Україна,Мариуполь,Украина,Mariupol,Ukraine
History,Історія,История
КиївФестивальКролівПисанок2018,Великдень,Easter,Пасха,Київ,Україна,Киев,Украина,Kiev,Kyiv,Ukraine

Маріуполь,Україна,Мариуполь,Украина,Mariupol,Ukraine,History,Історія,История
Маріуполь,Україна,Мариуполь,Украина,Mariupol,Ukraine,History,Історія,История,Architecture,Архітектура,Архитектура

evgenij kozij
https://www.facebook.com/profile.php?id=100014127811690
https://www.facebook.com/il.ko.79/posts/pfbid02TSKeHtrndyVM4pgfA1RLeZdKfP9bd5XZkC6qzQhrkcw2jaLvmeYvpyEos5iF46RZl

https://v-variant.com.ua/article/voin-z-mariupoli-podybaylo

https://mistomariupol.com.ua/uk/vpershe-turystychna-navigacziya-zyavylasya-u-mariupoli



