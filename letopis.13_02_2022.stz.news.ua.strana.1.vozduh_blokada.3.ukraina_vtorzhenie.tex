% vim: keymap=russian-jcukenwin
%%beginhead 
 
%%file 13_02_2022.stz.news.ua.strana.1.vozduh_blokada.3.ukraina_vtorzhenie
%%parent 13_02_2022.stz.news.ua.strana.1.vozduh_blokada
 
%%url 
 
%%author_id 
%%date 
 
%%tags 
%%title 
 
%%endhead 

\subsubsection{Украина и \enquote{вторжение}}

В Украине ситуация развивается двояко. 

С одной стороны, Зеленский продолжает говорить, что \enquote{вторжения} не
будет и что информации о скором нападении у него нет. По поводу распускаемых
западными СМИ слухов о вторжении 16 февраля он публично потребовал
доказательств.

Также принадлежащие Банковой информационные площадки отчаянно пытаются
показать, что далеко не весь мир паникует по этому поводу. 

С другой - армия подает неоднозначные сигналы, звучат заявления о том, что
только \enquote{агрессор} сунется, как получит отпор от \enquote{мощнейшей
армии Европы} (цитата главкома ВСУ Залужного). 

Из чего следует, что такие намерения у \enquote{агрессора}, по версии военных,
есть. Что противоречит заявлениям Зеленского. 

Но в целом видно, что украинская власть очень не хочет нагнетать ситуацию. И,
видимо, не имеет достаточных оснований, чтоб верить в угрозу вторжения. Ведь,
судя по всему, США не делятся данными своей разведки даже со страной, на
которую, по их мнению, планируют вот-вот напасть (иначе бы Зе не говорил, что у
него нет информации о \enquote{вторжении}). 

Если бы американцы предъявили Украине доказательства скорого нападения, мы бы
скорее всего увидели мобилизацию или хотя бы сборы резервистов. Вместо этого
пока только идут учения теробороны с деревянными автоматами. 

То есть, либо доказательств у США вообще нет, либо они руководство Украины не
устроили. 

При этом некие опасения у Киева все же имеются. Так, сегодня на границе
ужесточили пропускной режим для россиян, которых стали пускать лишь на похороны
близких родственников.

Правда, сделали это без помпы и громких объявлений. И можно в целом списать это
на меры перестраховки (аналогичные шаги Украина делала неоднократно при
Порошенко во время обострений с Россией). 
