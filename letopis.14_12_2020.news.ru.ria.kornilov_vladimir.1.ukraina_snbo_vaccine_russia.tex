% vim: keymap=russian-jcukenwin
%%beginhead 
 
%%file 14_12_2020.news.ru.ria.kornilov_vladimir.1.ukraina_snbo_vaccine_russia
%%parent 14_12_2020
 
%%url https://ria.ru/20201214/vaktsina-1589017384.html?in=t
 
%%author Корнилов, Владимир
%%author_id kornilov_vladimir
%%author_url 
 
%%tags covid_vaccine,russia,ukraine,sputnik_v
%%title Украинский СНБО: Россия атаковала Украину своей вакциной
 
%%endhead 
 
\subsection{Украинский СНБО: Россия атаковала Украину своей вакциной}
\label{sec:14_12_2020.news.ru.ria.kornilov_vladimir.1.ukraina_snbo_vaccine_russia}
\Purl{https://ria.ru/20201214/vaktsina-1589017384.html?in=t}
\ifcmt
	author_begin
   author_id kornilov_vladimir
	author_end
\fi

\ifcmt
pic https://cdn25.img.ria.ru/images/07e4/0c/0b/1588782800_0:9:3072:1737_1920x0_80_0_0_de4f203b185dab2f339fc74ea63bb7fc.jpg.webp
\fi

Новость о том, что британская фирма AstraZeneca официально начала
сотрудничество с российскими разработчиками вакцины от COVID-19 "Спутник V",
стала сенсацией для многих на Западе, где доселе наши исследования лишь
усиленно критиковали. Но есть одна страна, где эту новость стараются все еще не
замечать. И этой страной, конечно же, является Украина.

Напомним: на протяжении последних месяцев киевские власти и СМИ делали все от
них зависящее, доказывая, что российской вакцины просто не существует в
природе. И это не преувеличение. Министр здравоохранения Украины Максим
Степанов неоднократно прямо это заявлял. Причем отследить резкое изменение
риторики министра очень легко. Еще вечером 13 октября в своем интервью он
допустил возможность приобретения российской вакцины после одобрения ее на
Западе. А уже 15 октября был вызван на ковер к "начальству" — в посольство США.
После чего американские дипломаты посредством фейсбука уведомили жителей
подконтрольной территории: "Украина НЕ (выделено в оригинале. — Прим. авт.)
будет покупать российскую вакцину от COVID". Выглядело как прямое указание.
Судя по резкому изменению тона Степанова, сразу переставшего вообще замечать
"Спутник", так оно и было воспринято Киевом.

Тотчас же по всем украинским эфирам пошла волна уничижительных комментариев о
российской вакцине — как будто среди журналистов был объявлен негласный конкурс
на то, кто сильнее оскорбит наших ученых. Один из телеведущих, назвав эту
вакцину "мутью", даже заявил, что скорее пустит себе по венам бензин. Но особо
старался, конечно, Дмитрий Гордон, который в различных ток-шоу регулярно
высмеивал саму возможность появления в России вакцины. Обоснование простое: в
российском государстве не могут произвести своих автомобилей, стиральных машин
и брюк. Какие же еще нужны доказательства? Особенно забавно слышать подобное из
уст человека, долгие годы промышлявшего рекламой "золотых пирамидок" для
излечения всех болезней. Странно, что он еще не додумался до изобретения
пирамидки для профилактики COVID.

\href{https://ria.ru/20200812/1575689573.html?in=t}{Спутник V: первая зарегистрированная вакцина от COVID-19}

Но вот незадача: пока украинские деятели всех мастей отрицали наличие
российской вакцины, сами украинцы начали задавать назойливые и очень неудобные
вопросы о том, чем же им вакцинироваться и когда ждать этой возможности. Вполне
логично было бы спросить у американцев, запретивших Украине использовать
российский препарат: что они предлагают взамен? Министерство здравоохранения
отчиталась на днях о новом разговоре Степанова с временной поверенной в делах
США на Украине Кристиной Квин. Министр умолял заокеанских дипломатов начать
поставки американской вакцины в экстренном порядке, "поскольку Украина
находится в чрезвычайно сложных условиях в связи с российской агрессией". Квин
пообещала сделать все возможное для этого.

Однако уже на следующий день после сего торжественного обещания президент США
Дональд Трамп издал указ о том, что доступ иностранцев к американским вакцинам
будет разрешен лишь после окончания полной вакцинации всех желающих
американцев. А это значит, что раньше лета следующего года (в лучшем случае)
украинцам рассчитывать на выполнение обещания посольства США не приходится.

Америка оказалась не одинокой в проявлении своего вакцинного национализма.
Западные спонсоры Украины также делают все возможное для отсечения бедных стран
от скорого доступа к спасительным препаратам. По данным различных СМИ, богатые
страны уже "на корню" закупили будущие вакцины, лишив доступа к ним девять из
десяти стран мира на ближайший год или даже годы.

\href{https://radiosputnik.ria.ru/20201213/koronavirus-1588976058.html?in=t}{\enquote{Невзирая на политику}. Эксперт обозначил лидеров в гонке вакцин от COVID}

Киевский министр здравоохранения в ответ на надоедливые вопросы начал всюду
заявлять о том, что Украина попала в международную программу COVAX, которую
Всемирная организация здравоохранения инициировала для поддержки беднейших
стран мира. Поражает то, с какой гордостью Степанов рассказывает о своем
невероятном успехе в этой связи. То есть, по сути, он хвастает тем, что Украина
на официальном уровне признана нищей. Программа COVAX учреждена в основном для
стран Африки и Азии. Из европейских государств туда включены лишь Украина и
Молдавия. Да уж, невероятный успех.

Но даже учитывая такое "достижение", Киев признает, что сможет получить в
рамках этого проекта максимум восемь миллионов доз, что хватит на четыре
миллиона человек (учитывая, что каждому нужно вколоть по две дозы). На основе
каких-то невероятных умозаключений Центр общественного здоровья при
Министерстве здравоохранения подсчитал, что этого достаточно для вакцинации 20
процентов населения. Из чего можно сделать вывод, что в ведомстве Степанова уже
"сократили" население Украины до 20 миллионов. Никто толком сказать, сколько же
сейчас людей проживает в этом государстве, не может. Официальная статистика все
еще указывает цифры в районе 40 миллионов, "перепись по смартфонам",
инициированная прошлым правительством и раскритикованная всеми специалистами,
выдала около 37 миллионов, спикер Верховной рады Дмитрий Разумков полагает, что
теперь на Украине проживает порядка 30 миллионов человек. А настоящая перепись
там проводилась последний раз 19 лет назад. На прошлой неделе украинское
правительство снова перенесло подсчет населения, теперь — на 2023 год. Можно
посчитать по пальцам одной руки государства, которые не проводили переписи
дольше, да и то к таковым относятся несостоявшиеся или воюющие страны — вроде
Сомали и Йемена.

\href{https://ria.ru/20201212/koronavirus-1588871130.html?in=t}{Вакцина от коронавируса и алкоголь \enquote{разогрели} россиян}

Судя по политике нынешнего Киева, результаты следующей Всеукраинской переписи
будут очень печальными.

Для того чтобы начать массовую вакцинацию от коронавируса и тем самым спасти
своих граждан, у Украины остается лишь одна возможность — получение вакцины из
России. И Москва ведь, как это всегда бывало в истории взаимоотношений наших
народов, снова протягивает руку помощи. Глава Фонда прямых инвестиций Кирилл
Дмитриев после встречи с оппозиционным украинским деятелем Виктором Медведчуком
официально подтвердил готовность производить вакцину на Украине, в частности —
на мощностях харьковской фармацевтической компании "Биолек". И та уверила, что
имеет такую техническую возможность.

Казалось бы, вот оно — спасение для тысяч украинцев, которые не смогут дожить
до появления вакцины даже по программе COVAX (там открыто признают, что в
лучшем случае начнут поставлять препараты к началу второго квартала следующего
года). Но мы же слышали грозный окрик из американского посольства. И поэтому на
Медведчука дружно набросились украинские СМИ, обвиняя его в "пропаганде
российской вакцины".

Скандально известный телеведущий Савик Шустер (гражданин Италии и Канады) в
эфире украинского телевидения гневно заявляет: "Я не могу представить лидера
оппозиции любой страны, который едет в страну-агрессор и с этой страной якобы
договаривается о производстве вакцины". В общем-то, исходя из этого тезиса,
логично было бы задаться вопросом: а может, страна, которая предлагает спасение
жителей соседнего государства путем вакцинации, не является агрессором? Но
такие вопросы в украинских эфирах нельзя задавать. Тем самым Шустер прямо
признал исключительно политический мотив отказа Киева от вакцины. А секретарь
украинского СНБО Алексей Данилов, выступая в том же эфире, только подтвердил
это, заявив, что российское предложение помощи — "это психологическая и
информационная атака на Украину". И очень возмутился тем, что представитель
оппозиции волнуется за жизнь украинских пенсионеров. Видимо, их не жалко.

\href{https://radiosputnik.ria.ru/20201211/retsessiya-1588841440.html?in=t}{Вакцина не сможет исправить ущерб. В ООН предрекли глобальную рецессию}

А ведь, судя по ажиотажу среди украинцев, они-то сами только за то, чтобы
получить помощь из "страны-агрессора". Чем, похоже, активно пользуются
мошенники. Так, бывший министр экологии Украины Остап Семерак утверждает, что
на западе страны уже начали тайно прививаться российской вакциной. Учитывая
строгие правила хранения "Спутника", доставки и самого прививания, можно
допустить предположение, высказанное другим Остапом по поводу средства
"Титаник": "Всю контрабанду делают в Одессе, на Малой Арнаутской улице". Но
факт остается фактом: украинцы хотят прививаться, пытаются всеми правдами и
неправдами получить вакцину из России, но им это не позволяют посольство США и
Савик Шустер.

Украинский журналист Юрий Ткачев подсчитал, что с нынешними темпами смертности
от коронавируса отсрочка вакцинации даже на полгода обойдется в 30 тысяч
умерших украинцев (в два раза больше, чем официально признаны погибшими на
войне в Донбассе). "Иностранные хозяева Украины полагают эти потери
допустимыми", — считает он. Такой себе сопутствующий ущерб во время войны
против "агрессора". Что лишний раз подтверждает: Америка готова воевать с
Россией до последнего украинца. И лишь Россия в очередной раз готова спасать
жизни украинцев.
