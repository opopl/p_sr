% vim: keymap=russian-jcukenwin
%%beginhead 
 
%%file 17_03_2023.stz.news.ua.donbas24.1.mrpl_aktory_zigraly_blagodijnu_vystavu_v_kyevi.txt
%%parent 17_03_2023.stz.news.ua.donbas24.1.mrpl_aktory_zigraly_blagodijnu_vystavu_v_kyevi
 
%%url 
 
%%author_id 
%%date 
 
%%tags 
%%title 
 
%%endhead 

Маріупольські актори зіграли благодійну виставу в Києві (ФОТО)

Вистава пройшла в пам’ятну дату

Маріупольські актори з театру авторської п'єси «Концепція» 16 березня, в
річницю трагічних подій, коли росіяни скинули авіабомбу на драматичний театр в
Маріуполі, де загинули сотні українців, показали благодійну виставу «Рентген
усміхнених сердець». Зібрані під час вистави кошти передадуть на потреби ЗСУ.

Про це повідомила заступниця голови Київської міської адміністрації Ганна
Старостенко, передає Донбас24.

Читайте також: Для маріупольців до Дня закоханих провели концерт у Києві
(ВІДЕО)

«Ми вражені до глибини душі силою духу всіх українців, котрі пережили справжнє
пекло під час окупації, постійні ракетні обстріли та відсутність елементарних
умов для існування, і котрі тепер за допомогою своєї творчості збирають кошти
на допомогу Збройним Силам України», — зазначила посадовець.

Театр авторської п'єси «Концепція» став відкриттям для поціновувачів сучасних
вистав. Майже рік тому акторам разом із режисером Олексієм Гнатюком довелося
покинути рідний Маріуполь та шукати прихисток в Києві. За цей час маріупольці
встигли підкорити найбільші театральні сцени столиці, закохавши в себе
глядачів.

«У такий важкий для кожного маріупольця час ми знайшли підтримку в Києві, що
став другою домівкою для нас. Саме тут, на Троєщині, у дитячій бібліотеці № 115
Деснянського району ми вперше зустрілися колективом після трагічних подій та
розпочали відбудовувати наш театр», — зазначила представниця маріупольської
міської ради Діана Трима.

Раніше Донбас24 повідомляв, що маріупольський «Ренесанс» виступив з концертом у
Києві.

Ще більше новин та найактуальніша інформація про Донецьку та Луганську області
в нашому телеграм-каналі Донбас24.

Фото: з відкритих джерел
