% vim: keymap=russian-jcukenwin
%%beginhead 
 
%%file 11_04_2021.fb.myronenko_petr.1.vneshneje_upravlenie_usa_rossia
%%parent 11_04_2021
 
%%url https://www.facebook.com/petro.myronenko/posts/3104495826444724
 
%%author 
%%author_id 
%%author_url 
 
%%tags 
%%title 
 
%%endhead 

\subsection{Державотворення в Україні}
\label{sec:11_04_2021.fb.myronenko_petr.1.vneshneje_upravlenie_usa_rossia}
\Purl{https://www.facebook.com/petro.myronenko/posts/3104495826444724}

Багато століть під гнітом російської і не лише російської імперії перетворили
Український народ у народонаселення, яке змушене було терпіти насилля заради
виживання нащадків. Іншими словами: інстинкт самозбереження подавив волю до
громадянського самоутвердження.

За цих умов ми втратили навики до державотворення і опинились на межі втрати
національної ідентичності.

Сьогодні, на 30-му році незалежності, Український народ робить перші невпевнені
кроки на шляху утворення національної держави і ці кроки виявиляються дуже
важкими, з російськими кайданами на ногах і на самосвідомості.

Державотворення ускладнюється ще і тим, що Росія переселила велику кількість
свого населення задля агресивного засилля російської мови і процесу подавлення
української і це, треба чесно визнати, їй вдалося.

Російськозалежне населення України роз'їдає тіло України через слабкість
громадянського суспільства і безкінечно бездарну владу, що ми з вами обираємо
то за гречку, то за ілюзорні блага.

Який рецепт можна дати нам всім, Українському народові? 

Кожному наполегливо будувати Україну в собі і навколо себе.

На етапі, коли Україні загрожує держава-агресор Росія, доцільно ввести зовніщнє
управління з боку США, про що я стверджує вже більше десяти років поспіль.

Закрити державний кордон з Російською Федерацією і адміністративний кордон з
окупованими Росією українськими територіями.

Переравати дипломатичні відносини з державою-агресором Росією.

Корупцію і опозицію до держави прирівняти до державної зради.

Лише сильне громадянське суспільство може збудувати сильну державу для своїх нащадків.

З повагою П.В. Мироненко
