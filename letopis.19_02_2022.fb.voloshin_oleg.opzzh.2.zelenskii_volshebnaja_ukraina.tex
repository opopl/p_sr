% vim: keymap=russian-jcukenwin
%%beginhead 
 
%%file 19_02_2022.fb.voloshin_oleg.opzzh.2.zelenskii_volshebnaja_ukraina
%%parent 19_02_2022
 
%%url https://www.facebook.com/oleg.voloshin.7165/posts/5227659847267034
 
%%author_id voloshin_oleg.opzzh
%%date 
 
%%tags strana,ukraina,zelenskii_vladimir
%%title Зеленский и волшебная Украина
 
%%endhead 
 
\subsection{Зеленский и волшебная Украина}
\label{sec:19_02_2022.fb.voloshin_oleg.opzzh.2.zelenskii_volshebnaja_ukraina}
 
\Purl{https://www.facebook.com/oleg.voloshin.7165/posts/5227659847267034}
\ifcmt
 author_begin
   author_id voloshin_oleg.opzzh
 author_end
\fi

Зеленский и волшебная Украина. 

Ещё раз понял, за что широкая масса политического болота увлеклась Зеленским:
такой же инфантильный, на всех обиженный, считающий, что все ему должны,
видящий корень своих проблем в соседях (у рядовых украинцев другой набор:
олигархи, депутаты, иногда евреи, в последние годы Москва). Отсюда же и
восторги по поводу абсурдной идеи восстановить ядерный арсенал. Не буду даже о
международном аспекте. Эти люди вообще знают, в каком состоянии отечественная
фундаментальная наука, инженерная школа, промышленность, тот же Южмаш?! Украина
в своём нынешнем состоянии катера строить не может. Ноль новых предприятий
тяжёлой промышленности за 30 лет. Советское наследие с подачи западных
кураторов распилено и разворовано. Но разве людям, живущим мечтами и сказками
это понять. Ещё Гоголь описал национальный идеал: «Чтоб вареник сам в сметану и
- в рот». 

Но для феномена Зеленского лучше подходит одна из любимых цитат из «Белой
гвардии», которая для меня водораздел между пониманием и принятием: «И в этих
же городишках народные учителя, фельдшера, однодворцы, украинские семинаристы,
волею судеб ставшие прапорщиками, здоровенные сыны пчеловодов, штабс-капитаны с
украинскими фамилиями... все говорят на украинском языке, все любят Украину
волшебную, воображаемую, без панов, без офицеров-москалей, — и тысячи бывших
пленных украинцев, вернувшихся из Галиции. ...Турок, земгусар, Симон. Да не было
его. Не было. Так, чепуха, легенда, мираж. Просто слово, в котором слились и
неутолимая ярость, и жажда мужицкой мести и чаяния тех верных сынов своей
подсолнечной, жаркой Украины... ненавидящих Москву, какая бы она ни была —
большевистская ли, царская или еще какая».
