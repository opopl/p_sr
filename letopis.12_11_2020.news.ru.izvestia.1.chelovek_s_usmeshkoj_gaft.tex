% vim: keymap=russian-jcukenwin
%%beginhead 
 
%%file 12_11_2020.news.ru.izvestia.1.chelovek_s_usmeshkoj_gaft
%%parent 12_11_2020
 
%%url https://iz.ru/1099138/vladislav-krylov-natalia-vasileva-zoia-igumnova-sergei-sychev/chelovek-s-usmeshkoi-umer-valentin-gaft
 
%%author 
%%author_id 
%%author_url 
 
%%tags gaft_valentin
%%title Человек с усмешкой: умер Валентин Гафт
 
%%endhead 
 
\subsection{Человек с усмешкой: умер Валентин Гафт}
\label{sec:12_11_2020.news.ru.izvestia.1.chelovek_s_usmeshkoj_gaft}
\Purl{https://iz.ru/1099138/vladislav-krylov-natalia-vasileva-zoia-igumnova-sergei-sychev/chelovek-s-usmeshkoi-umer-valentin-gaft}

\begingroup
\em\centering\bfseries\large\color{blue}
Каким мы запомним великого русского артиста
\endgroup

\index[names.rus]{Гафт, Валентин!Человек с усмешкой, Известия, 12.12.2020}

{\em Владислав Крылов, Наталья Васильева, Зоя Игумнова, Сергей Сычев}

Совсем недавно он отпраздновал 85-летие. Увы, сегодня, 12 декабря, мы прощаемся
с Валентином Иосифовичем Гафтом. Он был благословлен великим талантом во всех
областях, к которым прикасался. «Известия» вспоминают большого актера, поэта,
драматурга, режиссера — и, прежде всего, удивительного человека.

Он родился 2 сентября 1935 года в семье юриста и домохозяйки, выходцев с
Украины. Одно из первых воспоминаний шестилетнего мальчишки — прощание с
уходящим на фронт отцом, навсегда врезавшееся в память. Иосиф Рувимович прошел
всю войну и вернулся живым, стал адвокатом, а семья обосновалась в самом
московском районе Москвы — в тенистых Сокольниках, на Матросской Тишине. По
соседству с двором, где играл с друзьями юный Валя, — печально знаменитая
тюрьма, шумное студенческое общежитие, полный народного говора рынок — «весь
мир в миниатюре», как шутил впоследствии Гафт.

Семья жила небогато — отец после фронтового ранения часто чувствовал себя
плохо. Как вспоминал впоследствии Гафт, носил он перешитые отцовские брюки и
пальто, даже деньги на суп в буфете Школы-студии МХАТ ему давала сердобольная
тетка-продавщица.

 \def\styleMP{\em\large\bfseries\color{blue}}

\textbf{Светлана Немоляева, народная артистка России:}

\begingroup
\styleMP
Это большая потеря, и для меня она не только творческая, но и личная. Мы с ним
с юности знакомы, со студенческих лет. Можно сказать, что наша жизнь прошла
друг у друга на виду.

Я знала, что он болеет, передвигается в инвалидной коляске. И он меня потряс,
когда в таком состоянии приехал на программу, которую телевидение сделало в мою
честь. Я бы сто раз поняла, если бы он остался дома. Но он такой человек
настоящий, ему было важно приехать, сказать хорошие слова, поздравить.

Он — уникальный, его ни с кем нельзя сравнить. Актер, поэт, гражданин, глубоко,
по-философски мыслящий. Его оценки жизни, того, что с нами происходило, всегда
были острые, неожиданные. Он во всем был Личностью.

Пусть земля ему будет пухом.
\endgroup


Мальчик из интеллигентной столичной еврейской семьи должен был, наверно, пойти
по стопам отца-правоведа. В крайнем случае, стать известным врачом или
музыкантом. Но Гафта захватил театр — причем из-за того, что раннее детство его
пришлось на военные годы, впервые он увидел огни рампы уже довольно большим
мальчиком. 

\ifcmt
  pic https://cdn.iz.ru/sites/default/files/styles/900x506/public/photo_item-2020-12/TASS_1899576_1.jpg?itok=Rwlv-dZI
  caption Валентин Гафт в роли Генриха IV в одноименном спектакле по пьесе итальянского писателя Луиджи Пиранделло на сцене театра «Современник». 1978 год
  width 0.5
\fi

Одиннадцатилетний Валентин пришел на постановку «Особого задания» Сергея
Михалкова в Центральном детском театре — и был поражен тем, что происходило на
сцене. Потом Гафт признавался, что не сразу понял даже, что всё это — лишь игра
актеров. Но эта чудная игра захватила его на всю жизнь. Вскоре он начал
заниматься в школьном кружке самодеятельности, а в старших классах уже твердо
знал, какую дорогу выберет в жизни.

\textbf{Валерий Баринов, народный артист России:}
  
\begingroup
\styleMP
Гафт был великолепным артистом, удивительно добрым человеком, потрясающим,
по-настоящему большим поэтом. Мы с ним однажды провели в поезде вместе две
ночи: одну по пути в Петербург, другую — обратно. Он читал мне свои стихи
две этих ночи подряд. Это было фантастическое время, которое я не забуду
никогда.Он болел, и в таких случаях обычно говорят, что Господь забрал,
прекратил его муки. Но для нас для всех эта потеря очень большая. Он был
действительно человеком Возрождения, талантливым во всем. В общении с
друзьями, в своих появлениях на сцене, в кино. Как поэт он пока недооценён.
Я думаю, потом поймут, что кроме эпиграмм в его наследии есть большие,
серьёзные, потрясающие стихи. Сколько раз я сталкивался с ним в жизни, на
съёмках, в концертах — и всегда получал от общения с ним удовольствие.
\endgroup

В «Щуку» Валентин не прошел, зато был с первого раза принят в Школу-студию МХАТ
(мастерская В.О. Топоркова). Вскоре после выпуска в 1957 году дебютировал на
сцене Театра имени Моссовета (Вторым сыщиком в «Лиззи Маккей» великого Сартра).
Потом служил в Московском драматическом театре (ныне Театр на Малой Бронной).
Но по-настоящему его дарование открылось лишь у Анатолия Эфроса, взявшего
молодого актера в Театр имени Ленинского комсомола. 

\ifcmt
  pic https://cdn.iz.ru/sites/default/files/styles/900x506/public/photo_item-2020-12/TASS_95038_1.jpg?itok=-gGek9aL
  caption Актер Валентин Гафт. 1979 год, Фото: ТАСС/Валерий Христофоро
  width 0.5
\fi

А потом был «Современник», с которым Гафт не расставался до самого конца жизни.
Там он вышел на сцену в последний раз, в 2018 году, в спектакле «Пока
существует пространство», поставленном по его собственному тексту к 90-летнему
юбилею Олега Ефремова.

\textbf{Валерий Фокин, худрук Александринского театра, народный артист России:}

\begingroup
\styleMP
Ушел один из мэтров того «Современника», которого уже нет. Гафт был в одной
компании с Ефремовым, Волчек, Квашой, Дорошиной... Конечно, это очень печально,
потому что эпоха того театра закончилась окончательно. Гафт был очень яркой
актерской личностью. Очень одаренной. Его эпиграммы знала и рассказывала вся
Москва. Его юмор был тонкий и иногда злой, он мог и обидеть какого-то коллегу.
Но когда он кого-то критиковал, иногда несправедливо, эта критика основывалась
на том, что он хотел возбудить своих коллег, добиться, чтобы театр был не
мертвым. Он ненавидел театр аморфный, пресный. Ему нужен был возбудительный,
яркий, аффективный театр. Он играл в первом моем спектакле «Валентин и
Валентина» в «Современнике». И я понимал, что ему нужен не статичный ровный
образ, а взрывчатый. Он сочетал в себе всё лучшее и всё худшее, что есть в этой
актерской профессии. Он был эгоистичен, хотел играть определенные роли, но он
имел на это право. Если же режиссер понимал его исключительные актерские
способности, доверял ему, то в какой-то момент Гафт становился, фактически,
соавтором, выходил на какую-то иную, высокую точку искусства, и начиналась
великая импровизация, идущая при этом от режиссерского замысла. Не секрет, что
многие даже очень хорошие артисты — лишь исполнители. А Гафт — это крупнейшая
личность, которая окрашивала собой любую роль. И, конечно, я никогда не забуду
репетиций Городничего в «Ревизоре», потому что, на мой взгляд, это была вообще
его лучшая роль. Он играл именно то, чего я хотел: умного Городничего. Конечно,
грандиозный артист.
	
\endgroup

Дебют Гафта в кино состоялся еще во время учебы в школе-студии, в 1956 году. Он
появился в маленькой роли убийцы и негодяя Марселя Руже в фильме Михаила Ромма
«Убийство на улице Данте». В дальнейшем из-за характерной внешности Гафту часто
доставались роли отрицательных персонажей — он играл и царских жандармов, и
советских уголовников, и американских церэушников. Во всех этих ипостасях Гафт
просто излучал нешуточную угрозу и невероятную, почти звериную маскулинность.

\ifcmt
  pic https://cdn.iz.ru/sites/default/files/styles/900x506/public/photo_item-2020-12/TASS_2470433_1.jpg?itok=UioeebdB
  caption Актер Валентин Гафт, получивший «Золотого орла» «За вклад в киноискусство», на церемонии награждения премии «Золотой орел – 2011» Фото: ТАСС/Марина Лысцева
  width 0.5
\fi

По иронии судьбы в школьном драмкружке долговязому и худому парнишке всегда
отдавали роли девочек. Впрочем, среди лучших, пожалуй, работ Гафта на экране
были вовсе не злодеи, а безупречный в своем благородстве полковник Покровский
из «О бедном гусаре замолвите слово», комический пройдоха-дворецкий Брассетт в
«Здравствуйте, я ваша тетя!» и циничный (но никак не преступный) председатель
правления кооператива, ветеринар Сидорин в «Гараже».

\textbf{Владимир Бортко, кинорежиссер, народный артист России:}
  
\begingroup
\styleMP
Мы все его знаем и видели очень много раз в
самых непохожих друг на друга ролях и работах. Но бессмысленно
говорить о нем как об актере. Он был Гафт! И этим всё сказано. Я имел
удовольствие снимать Валентина Иосифовича, причем сразу в двух ролях,
и могу говорить о нем как о человеке. Он был чрезвычайно
требователен. Прежде всего, к себе. Но также и к коллегам по работе,
ко всему, что его окружало. И это притом что он был чрезвычайно
веселым человеком, сочинял свои великолепные эпиграммы. Мы на съемках
сидели как-то вечером, он читал эти эпиграммы, я хохотал, как
безумный. Но несмотря на этот хохот, я понимал, с кем я имею дело. С
крупной человеческой личностью, неповторимой, как линии на ладони. Он
оставил нам много. Будем смотреть и радоваться.
\endgroup

Гафт всегда был склонен к иронии и самоиронии — вспомним хотя бы его
легендарные эпиграммы. В советское время их тайком перепечатывали на «эриках» и
«любавах» чаще, чем любой запретный «самиздат» — впрочем, и едкие стихи Гафта
официальными властями явно не поощрялись (хотя смеялись над ними, по слухам,
даже генсеки). Острый на язык, с саркастической улыбкой, он воплощал собой дух
сомнения — и на сцене, и на экране. За это, наверно, он был особенно любим
зрителями — вдвойне и втройне в советские времена, когда сомнения во многих
сферах жизни отнюдь не приветствовались.

\ifcmt
  pic https://cdn.iz.ru/sites/default/files/styles/900x506/public/photo_item-2020-12/TASS_14782517_1.jpg?itok=EtSaODwg
  width 0.5
	caption Президент России Владимир Путин и актер Валентин Гафт, награжденный орденом «За заслуги перед Отечеством» IV степени, во время церемонии вручения государственных наград в Кремле. 2016 год Фото: ТАСС/Михаил Метцель 
\fi

И еще он был удивительно, неистово витален. Болел с 10 лет за «Спартак» («Два
«Спартака» — один в Большом, / Другой в Большом футболе, / Но я бы в театр не
пошел, / Когда «Спартак» на поле», — писал он в одной из эпиграмм). Нежно любил
животных, выступал за создание поста уполномоченного по их правам. Был отмечен
государственными наградами: кавалер ордена «За заслуги перед Отечеством» трех
степеней, народный артист РСФСР.

\textbf{Иосиф Райхельгауз, народный артист РФ, худрук театра «Школа современной пьесы»:}
  
\begingroup
\styleMP
Валентин Иосифович Гафт — человек
разносторонних дарований. Он очень талантливый артист, очень
талантливый, интереснейший поэт. Он себя в этом даже как-то
ограничивал, иронизировал. Одни из сильнейших качеств человека —
ирония и самоирония. Вот он был именно таким. Он сам о себе говорил:
«Когда говоришь Гафт, все думают, что это какое-то издательство». Я
горжусь тем, что, когда впервые напечатали 20 эпиграмм Гафта, написал
вступительную статью о нем. Это было в газете еще в Советской
Эстонии. Я вам это говорю, а осознать не могу. Не верю, что не будет
ночного звонка в 2–3 часа ночи. И ты сквозь сон берешь трубку и
отвечаешь: «Да, дорогой Валентин Иосифович». А там: «Старик, хорошо,
что ты не спишь. Послушай, у меня тут есть что тебе почитать». И
читает стихи. Теперь этого не будет. Всё понятно, возраст, болезнь.
Уходят один за другим великие. Это мощнейшее поколение. Они еще и
войну застали. Они называли себя «дети XX съезда». В их жизнь
вместилось всё! И они это всё пережили и художественно выразили. И
Гафт, конечно, один из самых главных, самых ярких на этом звездном
небосводе русской культуры. Он на меня написал много эпиграмм. Как-то
была ситуация, что я руководил Театром Станиславского, а меня
уволили. Он тут же написал обо мне. Я родился в Одессе, и он это
обыграл: «Одесский пляж на время бросив, / в Москву пожаловал Иосиф.
/ Но наступила пауза / в карьере Райхельгауза. / Не съездить ли для
интересу / тебе назад в свою Одессу?» Мне это льстило и никогда не
было обидно. Если Гафт не писал эпиграмму, значит, человек не достоин
этого.
\endgroup

Он действительно был народным артистом, любимым многие десятилетия, при всех
режимах, во всех слоях общества, но относился к себе беспощадно-критически. «Да
какое там творчество... Мне оно не очень нравится — это главное... Много в кино
я сыграл того, чего играть не следовало бы», — говорил он в интервью
«Известиям»\Furl{https://iz.ru/news/305678} в канун семидесятилетия, уже в никем не оспариваемом звездном
статусе.
 
\ifcmt
	tab_begin cols=3
		caption Актер, режиссер, литератор Валентин Гафт умер в субботу в возрасте 85 лет. Самые яркие образы на сцене и в жизни – в фотогалерее «Известий»
		pic https://cdn.iz.ru/sites/default/files/styles/640x360/public/photo_item-2020-12/1607762410.jpg?itok=KJ_RoLgA

		pic https://cdn.iz.ru/sites/default/files/styles/640x360/public/photo_item-2020-12/1607762771.jpg?itok=d4lsUInQ
		caption Валентин Гафт родился 2 сентября 1935 года в Москве. Школу-студию МХАТ окончил в 1957-м

		pic https://cdn.iz.ru/sites/default/files/styles/640x360/public/photo_item-2020-12/1607763513.jpg?itok=yzIUePRx
		caption В разное время работал в Московском драматическом театре, Театре имени Моссовета, Ленкоме, Театре сатиры

		pic https://cdn.iz.ru/sites/default/files/styles/640x360/public/photo_item-2020-12/1607763482.jpg?itok=iJUCXBGx
		caption С 1969 г. служил в труппе театра «Современник»

		pic https://cdn.iz.ru/sites/default/files/styles/640x360/public/photo_item-2020-12/1607763483.jpg?itok=fpPLqIKM
		caption На театральной сцене исполнил десятки ролей

		pic https://cdn.iz.ru/sites/default/files/styles/640x360/public/photo_item-2020-12/1607763590.jpg?itok=-FBOd69j
		caption Всенародную славу получил после исполнения ролей в популярных советских фильмах – «Семнадцать мгновений весны», «Здравствуйте, я ваша тетя!», «Чародеи» и десятках других
tab_end 
\fi

В этом не было и грана кокетства — Гафт каждый вечер выходил на сцену как в
последний раз и на съемочной площадке выкладывался «до полной гибели всерьез».
Наверно, именно поэтому в его богатой творческой истории просто невозможно
найти проходных ролей. Он был верен призванию до последнего — лишь резкое
ухудшение здоровья в прошлом заставило его окончательно покинуть сцену. 
