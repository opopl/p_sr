% vim: keymap=russian-jcukenwin
%%beginhead 
 
%%file 18_08_2019.stz.news.ua.mrpl_city.1.valentyna_abalmasova
%%parent 18_08_2019
 
%%url https://mrpl.city/blogs/view/valentina-abalmasova-golovnebuti-nebajduzhimi-i-brati-vidpovidalnist-na-sebe
 
%%author_id demidko_olga.mariupol,news.ua.mrpl_city
%%date 
 
%%tags 
%%title Валентина Абалмасова: "Головне – бути небайдужими і брати відповідальність на себе!"
 
%%endhead 
 
\subsection{Валентина Абалмасова: \enquote{Головне – бути небайдужими і брати відповідальність на себе!}}
\label{sec:18_08_2019.stz.news.ua.mrpl_city.1.valentyna_abalmasova}
 
\Purl{https://mrpl.city/blogs/view/valentina-abalmasova-golovnebuti-nebajduzhimi-i-brati-vidpovidalnist-na-sebe}
\ifcmt
 author_begin
   author_id demidko_olga.mariupol,news.ua.mrpl_city
 author_end
\fi

\ii{18_08_2019.stz.news.ua.mrpl_city.1.valentyna_abalmasova.pic.1}

Допоки в нашому місті є небайдужі та ініціативні мешканці, Маріуполь буде
продовжувати стрімко і динамічно розвиватися. Наступна наша героїня – \textbf{Валентина
Абалмасова} – по-справжньо\hyp{}му небайдужа, активна та віддана своїй справі
громадська діячка. І, незважаючи на те, що народилася жінка в іншому місті,
сьогодні вона докладає багато зусиль для змін вже рідного для неї Маріуполя.
Допомагають їй в цьому невичерпний ентузіазм, віра в людей, багато креативних
ідей і готовність брати відповідальність на себе.

\ii{18_08_2019.stz.news.ua.mrpl_city.1.valentyna_abalmasova.pic.2}

Народилася Валентина в місті Балаклії (Харківська область). Найбільше Валентину
надихала мама, яка завжди підтримувала і допомагала доньці. Наразі вона живе в
Полтаві, успішна бізнесвумен, завжди заряджає доньку на нові звершення.
Валентина закінчила Донецький національний університет економіки і торгівлі
імені Михайла Туган-Барановського, спеціальність маркетолог-економіст. Друзі
запросили її до Маріуполя, який Валі сподобався набагато більше, ніж Донецьк.
Особливо сподобалося море. У Маріуполі спочатку влаштувалася на роботу
товарознавцем. У 2012 році Валя познайомилася з майбутнім чоловіком – \textbf{Андрієм}.
Сьогодні у Валентини двоє синів – \textbf{Артем і Арсеній Абалмасови}. Всі імена
чоловіків у сім'ї Абалмасових починаються з \enquote{А}. Мабуть, це не просто так, адже
саме ця літера є першою в алфавіті. Впевнена, що і найголовніші чоловіки нашої
героїні будуть в усьому першими. Валя з родиною любить гуляти на Лівобережному
пляжі, в парку \enquote{Веселка}, який після реконструкції став найулюбленішим місцем
відпочинку.

\ii{insert.read_also.demidko.djuzhok}
\ii{18_08_2019.stz.news.ua.mrpl_city.1.valentyna_abalmasova.pic.3}

Громадською діяльністю почала займатися в декреті. Валентина живе у
Лівобережному районі, з якого не так легко вибратися з коляскою. Вона
переконалася в повній відсутності інфраструктури для мам. Це стало точкою
відліку. Жінка розуміла, що зміни потрібні і починати необхідно з себе. У 2014
році написала оголошення для жителів свого будинку, в якому запропонувала
зайнятися створенням дитячого майданчика. Підписалася як \enquote{мама Арсенія}.
Спочатку її почуло лише 10 осіб. Почали з наступного тижня. Провели суботник.
Долучилося вже 15 сусідів. Двір почав пожвавлюватися. На той момент Валентина
ще не знала повного механізму, як і куди можна звертатися. Тому, за порадою
сусідів, написала звернення народному депутату Матвієнкову С. А. з проханням
допомогти у соціальній мережі. Однак він швидко відгукнувся і приїхав через
тиждень. Встановив 2 гойдалки, лавочки, гірку. Всі не очікували, що вдасться
досягти такого результату. Разом з дітьми на новому майданчику організували
концерт та дитячу дискотеку.

\ii{18_08_2019.stz.news.ua.mrpl_city.1.valentyna_abalmasova.pic.4}

Завдяки тренінгу Active Citizens (Активні Громадяни) та тренерці \href{https://archive.org/details/19_06_2019.olga_demidko.mrpl_city.ilona_arhangelska_ne_bijtes_buty_inshymy}{Ілоні
Архангельській}\footnote{%
Ілона Архангельська: \enquote{Не бійтеся бути \enquote{іншими}}, Ольга Демідко, mrpl.city, 19.06.2019, \par%
\url{https://mrpl.city/blogs/view/ilona-arhangelska-ne-bijtesya-buti-inshimi}, \par%
Internet Archive: \url{https://archive.org/details/19_06_2019.olga_demidko.mrpl_city.ilona_arhangelska_ne_bijtes_buty_inshymy}
} громадська діячка навчилася писати проекти, зрозуміла -
необхідно діяти покроково, щоб досягти поставленої мети. Один з проектів, в
якому вона дотепер продовжує брати активну участь – \textbf{\enquote{Фінансова грамотність}}.
Згадує, як навчала дітей фінансовій грамотності на березі моря (2016 рік).

Протягом 2016–2017 років працювала кураторкою і тренером у \textbf{ГО \enquote{Маріупольська
Спілка Молоді}}. Це був чудовий досвід гарної роботи в команді.

У 2017 році Валентина взяла участь в написанні заявки на грант від
Європейського Союзу. Кошти вдалося отримати завдяки підтримці \textbf{ГО \enquote{Агентство
підтримки місцевих ініціатив}} та співпраці з Лівобережною районною
адміністрацією. Завдяки реалізації проекту \textbf{\enquote{СПОРТ ПОРУЧ}} був створений
універсальний спортивний майданчик для гри у футбол, баскетбол та волейбол.
Зараз вона та інші мешканці її двору чекають завершення будівельних робіт за
програмою \textbf{\enquote{Затишний двір}} у їхньому подвір'ї, територія біля будинку продовжує
облаштовуватися.

\textbf{Читайте також:} \emph{Мариупольские ОСМД: путь к развитию своих домов или бизнес для посторонних?}%
\footnote{Мариупольские ОСМД: путь к развитию своих домов или бизнес для посторонних?, Константин Карцев, mrpl.city, 16.08.2019, \par%
\url{https://mrpl.city/blogs/view/mariupolskie-osmd-put-k-razvitiyu-svoih-domov-ili-biznes-dlya-postoronnih}
}

У 2018 році Валентина взяла участь у конкурсі \textbf{\enquote{Громадський бюджет}} для
покращення покриття на спортивному майданчику, набравши необхідну кількість
голосів – перемогли. Реалізація проекту планується до кінця 2019 року.

\ii{18_08_2019.stz.news.ua.mrpl_city.1.valentyna_abalmasova.pic.5}

У вересні виповниться вже 3 роки, як сім'я Абалмасових сортує свої побутові
відходи. Ініціатором був чоловік. Жінці нічого не залишалось, як тільки
долучитися до цієї корисної і важливої справи. А з початку цього року вже
Валентина залучила до сортування своїх сусідів. Дякуючи участі та перемозі у
конкурсі \textbf{\enquote{Зелена скарбничка}} отримала додаткові знання та мотивацію. 

Разом з групою екоактивістів \textbf{\#Маріуполь\_Сортує} навесні 2019 року провели у
кожному районі міста прибирання, де не тільки збирали сміття, а й розповідали
про необхідність сортування твердих побутових відходів та займалися йогою.

\ii{18_08_2019.stz.news.ua.mrpl_city.1.valentyna_abalmasova.pic.6}

Валентина настільки захопилась темою захисту навколишнього середовища, що зараз
очолює \textbf{Зелену Раду} при виконкомі Маріупольської міської ради. Проектів дуже
багато. Зокрема, працює з \textbf{Анною Чебан (керівник ГО \enquote{ЕКО ГУРТ})} над проектом,
спрямованим на підвищення ефективності взаємодії влади та громади задля
вирішення загальних місцевих екологічних проблем, підвищення культури
поводження з твердими побутовими відходами, створення умов для сортування. 

\textbf{Улюблена книга:} \enquote{Багатий тато, бідний тато}, автор Роберт Кійосакі.

\textbf{Улюблений фільм:} \enquote{Стажер} (2015 рік).

\textbf{Порада маріупольцям:} 

\begin{quote}
\em\enquote{Будьте небайдужими до свого міста, беріть
відповідальність на себе! Задавайте собі частіше питання: а що зробив я, щоб
місто стало кращим?}
\end{quote}

\textbf{Читайте також:} \emph{Здорове тіло і здорова планета: як маріупольці поєднали пробіжку і прибирання}%
\footnote{Здорове тіло і здорова планета: як маріупольці поєднали пробіжку і прибирання, Олександр Северін, mrpl.city, 12.08.2019, \par%
\url{https://mrpl.city/blogs/view/zdorove-tilo-i-zdorova-planeta-yak-mariupoltsi-poednali-probizhku-i-pribirannya}
}
