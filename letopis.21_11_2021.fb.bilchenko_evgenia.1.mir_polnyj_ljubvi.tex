% vim: keymap=russian-jcukenwin
%%beginhead 
 
%%file 21_11_2021.fb.bilchenko_evgenia.1.mir_polnyj_ljubvi
%%parent 21_11_2021
 
%%url https://www.facebook.com/yevzhik/posts/4460580380643704
 
%%author_id bilchenko_evgenia
%%date 
 
%%tags bilchenko_evgenia,ljubov,poezia
%%title Мир, полный любви. Люб-ви
 
%%endhead 
 
\subsection{Мир, полный любви. Люб-ви}
\label{sec:21_11_2021.fb.bilchenko_evgenia.1.mir_polnyj_ljubvi}
 
\Purl{https://www.facebook.com/yevzhik/posts/4460580380643704}
\ifcmt
 author_begin
   author_id bilchenko_evgenia
 author_end
\fi

Постепенно собираю вокруг себя родной и ранний, хотя и умерший, уже не
рокешный, почти уже рэперский мир традиции: без всего того, чего я сильно
боюсь: без позднего Макаревича, без позднего Шевчука, без позднего Гройса, без
новых левых, без неонацистов, без хипстеров... 

\ii{21_11_2021.fb.bilchenko_evgenia.1.mir_polnyj_ljubvi.pic.1}

Мир, полный любви. Люб-ви. 

Спасибо за существование этого мира: Михаилу Владимирскому и Варваре
Добромеловой, Михаилу Елизарову и тем пацанам, друзьям, Миши, что подошли
подать мне руку в рок-клубе, а я - им, и тому мальчику, в форме балтийского
матроса... Это всё очень-очень-очень важно. Вот, когда мое давнее стихотворение
догнало само себя: вас я и искала, родня после того, как в Киеве умерла
контркультура, когда она легла под евро (да, слово скоммунизжено у Захара, я
точнее не придумаю, ибо недоросль). 

\ii{21_11_2021.fb.bilchenko_evgenia.1.mir_polnyj_ljubvi.pic.2}

***

\begin{multicols}{2}
\obeycr
Осень... Славянство... С русым Борисоглебом
Мы говорим про Гегеля и Христа.
Наши подвалы пахнут подпольным небом:
Рядом – друзья, и совесть у них – чиста.
\smallskip
Все они здесь: Есенин, Бодлер, Высоцкий –
Вышли из рая водку в мешке толочь.
Наши подвалы пахнут подпольным солнцем,
А наверху царит мировая ночь.
\smallskip
Право остаться левым вылазит боком.
Вечность горчит вином под голландский сыр.
Наши подвалы пахнут подпольным Богом,
А наверху дуреет безбожный мир.
\smallskip
Где ты, цветочек аленькой, семицветик?
Где вы, крутые красные фонари?
В наших подвалах нам ничего не светит…
Вот почему мы светимся изнутри.
\restorecr
\end{multicols}

PS. Михаил Юрьевич, Лермонтов 21 века, не стеби, пожалуйста, Есенина моего,
Бранимира, очень люблю я его глубинку русскую, его боевитую духовность.

\#МихаилЕлизаров \#Бранимир

\ii{21_11_2021.fb.bilchenko_evgenia.1.mir_polnyj_ljubvi.pic.3}

\ii{21_11_2021.fb.bilchenko_evgenia.1.mir_polnyj_ljubvi.cmt}
