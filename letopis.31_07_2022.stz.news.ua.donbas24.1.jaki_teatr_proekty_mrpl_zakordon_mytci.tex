% vim: keymap=russian-jcukenwin
%%beginhead 
 
%%file 31_07_2022.stz.news.ua.donbas24.1.jaki_teatr_proekty_mrpl_zakordon_mytci
%%parent 31_07_2022
 
%%url https://donbas24.news/news/yaki-teatralni-projekti-ta-kulturni-zaxodi-prisvyaceni-mariupolyu-realizuyutsya-zakordonnimi-mitcyami
 
%%author_id demidko_olga.mariupol,news.ua.donbas24
%%date 
 
%%tags 
%%title Які театральні проєкти та культурні заходи, присвячені Маріуполю, реалізуються закордонними митцями
 
%%endhead 
 
\subsection{Які театральні проєкти та культурні заходи, присвячені Маріуполю, реалізуються закордонними митцями}
\label{sec:31_07_2022.stz.news.ua.donbas24.1.jaki_teatr_proekty_mrpl_zakordon_mytci}
 
\Purl{https://donbas24.news/news/yaki-teatralni-projekti-ta-kulturni-zaxodi-prisvyaceni-mariupolyu-realizuyutsya-zakordonnimi-mitcyami}
\ifcmt
 author_begin
   author_id demidko_olga.mariupol,news.ua.donbas24
 author_end
\fi

\ii{31_07_2022.stz.news.ua.donbas24.1.jaki_teatr_proekty_mrpl_zakordon_mytci.pic.front}

\begin{center}
  \em\color{blue}\bfseries\Large
Андрій Палатний, куратор численних театральних проєктів розповів, як
європейські режисери підтримують Маріуполь
\end{center}

У 2021 році в рамках фестивалю \textbf{iStage 2021} та \textbf{Марафону
міжнародних резиденцій} до Маріуполя приїжджали європейські режисери і разом з
\href{https://www.facebook.com/theatromaniamrpl}{Народним театром \enquote{Театроманія}},
\footnote{\url{https://www.facebook.com/theatromaniamrpl}}
 \textbf{Першою театральною школою-студією} та \textbf{Театром ляльок} створювали унікальні вистави. З
початком повномасштабного вторгнення росії в Україну \textbf{Крістін Дісмен},
\textbf{Мадлен Бонгард} та \textbf{Евангелос Космідіс} вирішили підтримати
місто, яке вони встигли побачити на власні очі розвиненим та квітучим. За
декілька місяців війни від Маріуполя залишилися руїни і про цю трагедію тепер
розповідають режисери у своїх виставах. \href{https://www.facebook.com/andriipalatnyi}{Андрій Палатний},
\footnote{\url{https://www.facebook.com/andriipalatnyi}}
куратор театральної,
перфомативної та цифрової програм \enquote{Гогольfest dream}, фестивалю iStage
2021 та Марафону міжнародних резиденцій підкреслив, що через війну багато
планів не вдалося реалізувати.

\ii{31_07_2022.stz.news.ua.donbas24.1.jaki_teatr_proekty_mrpl_zakordon_mytci.pic.1}

\begin{leftbar}
  \begingroup
{\em\enquote{24 лютого я зустрів в потязі. Їхав до Маріуполя. Потяг зупинився серед
поля. Було незрозуміло, що далі. Ввечері я все ж доїхав до міста, але
всіх пасажирів повезли одразу ж назад до Києва. Наприкінці лютого вже
мав відкриватися Центр сучасного мистецтва \enquote{Готель Континенталь} і я
планував зустрітися з керівництвом ЦСМу, щоб обговорити подальші плани,
адже всі вистави Марафону міжнародних резиденцій мали стати репертуаром
\enquote{Готелю Континенталь}. В перший день війни я побачив востаннє Маріуполь
цілим і неушкодженим}}, — розповів Андрій Палатний.
   \endgroup
\end{leftbar}

Через два тижні Андрій організував зустріч з режисерами, які приїздили до
Маріуполя, і було вирішено створити власну команду. Спочатку вони зробили сайт
mariupol.life англійською мовою, який має на меті познайомити європейських
жителів з Маріуполем, його культурою до війни. Зокрема, на сайті можна побачити
події, що відбувалися в рамках Марафону міжнародних резиденцій 2021. Також на
сайті висвітлені нищівні руйнування Маріуполя.

\begin{leftbar}
\emph{\enquote{Маріуполь бомбили з перших днів війни. Там зруйновано близько 90\%
міста. Це місце гуманітарної катастрофи}}, — зазначено на сайті.
\end{leftbar}

% 2 - Актриси та актор Народного театру «Театроманія» та Першої театральної школи-студії
\ii{31_07_2022.stz.news.ua.donbas24.1.jaki_teatr_proekty_mrpl_zakordon_mytci.pic.2}

Найголовнішою задачею команди — було допомогти маріупольській молоді,
талановитим акторам та актрисам вибратися з облоги міста. Далі кожен з
режисерів почав проводити заходи на підтримку Маріуполя в різних містах та за
власні кошти. Перша масштабна культурна подія відбулася на легендарній сцені
Deutsches Theater в Берліні. Там вразив всіх своїм виступом режисер Народного
театру \enquote{Театроманія} \href{https://www.facebook.com/profile.php?id=100001584004558}{Антон Тельбізов}%
\footnote{\url{https://www.facebook.com/profile.php?id=100001584004558}}
та свою історію через відео-зв'язок
розповіла актриса Марія Бойко. Директорка Junges DT Birgit Lengers підтримала
ініціативу проведення заходу, адже сама з виставою приїздила до Маріуполя в
2019 році, яку представляли в \enquote{Колегіум-школі} № 1.

% 3 - Антон Тельбізов під час виступу на сцені Deutsches Theater в Берліні.
\ii{31_07_2022.stz.news.ua.donbas24.1.jaki_teatr_proekty_mrpl_zakordon_mytci.pic.3}

\begin{leftbar}
\emph{\enquote{Через культурні події, вистави ми почали розповідати світу про Маріуполь, про
його культурні події та повне знищення російськими військами}}, — додав
Палатний.
\end{leftbar}

Кульмінацією стала подія, що відбулася 27 березня, Deutsches Theater Berlin
організував вечір під назвою \enquote{Українські голоси для Маріуполя}, присвячений
портовому місту, де тиждень тому бомбами було зруйновано театр, який служив
укриттям для тисячі маріупольців. 

% 4 - Евангелос Космідіс — режисер-постановник Національного театру Грецї в Афінах та Муніципального театру Пірея
\ii{31_07_2022.stz.news.ua.donbas24.1.jaki_teatr_proekty_mrpl_zakordon_mytci.pic.4}

21 травня в Національному театрі в Празі під час конференції Європейської
театральної конвенції була представлена вистава \enquote{Аляска} режисера-постановника
Національного театру Грецї в Афінах та Муніципального театру Пірея \emph{Евангелоса
Космідіса}, яку він вперше представив у серпні 2021 року в Маріуполі. Виставу
було перекладено англійською мовою. Спектакль залишається актуальним, адже
включає реальні думки та коментарі маріупольських дітей, які зіткнулися з
війною. Наразі виставу перекладено різними мовами, щоб її змогло побачати
якомога більше глядачів.

% 5 - Репетиція в ПК "Молодіжний" з акторами Театру ляльок та Першої театральної школи-студії в Маріуполі
\ii{31_07_2022.stz.news.ua.donbas24.1.jaki_teatr_proekty_mrpl_zakordon_mytci.pic.5}

А 17 червня було відкрито центр Hotel Continental — Art Spase in Exile, в якому
була представлена виставка, присвячена Маріуполю з 2014 до 2022 років. Центр
було створено німецькою режисеркою Крістін Діссман спільно з Андрієм Палатним.
Він повинен реалізувати декілька функцій: стати домом культури для українських
митців та платформою для діалогу між українськими та європейськими діячами.

\begin{leftbar}
\emph{\enquote{Кожен проєкт має свою стратегію розвитку і дає можливості рухатися далі}}, — пояснив Андрій Палатний.
\end{leftbar}

% 6 - Мадлен Бонгард — швейцарська режисерка, акторка та педагогиня
\ii{31_07_2022.stz.news.ua.donbas24.1.jaki_teatr_proekty_mrpl_zakordon_mytci.pic.6}

В підтримку Маріуполя і загалом України проводить зустрічі в zoom \emph{\textbf{Мадлен
Бонгард}} — швейцарська режисерка, акторка та педагогиня. Також вона випустила
\href{https://radiobascule.ch/podcast/65}{подкаст},%
\footnote{\url{https://radiobascule.ch/podcast/65}}
 присвячений місту. Влітку 2021 року в Маріуполі вона провела серію
акторських практик, що досліджували усвідомлення присутності.

\ifcmt
  ig https://i2.paste.pics/PQXIR.png?trs=1142e84a8812893e619f828af22a1d084584f26ffb97dd2bb11c85495ee994c5
  @wrap center
  @width 0.9
\fi

% 7 - Крістін Діссманн — німецька режисерка
\ii{31_07_2022.stz.news.ua.donbas24.1.jaki_teatr_proekty_mrpl_zakordon_mytci.pic.7}

У жовтні 2021 року режисерка з Німеччини \emph{\textbf{Крістін Діссманн}} представила в
Маріуполі виставу-інсталяцію \enquote{Між часами} що поєднала театр, музику та
візуальне мистецтво з експериментальним мистецьким досвідом у новому вимірі і
була присвячена Маріуполю. Режисерка продовжує ставити цю виставу за участі
українських, білоруських та німецьких акторів.

Європейські митці намагаються підтримути маріупольську молодь. Зокрема, завдяки
Europen theatre convention акторка Народного театру \enquote{Театроманія} \emph{\textbf{Марія Бойко}}
протягом 2 тижнів стажувалася з грецькими режисерами.

% 8 - Акторка Марія Бойко під час стажування в Греції
\ii{31_07_2022.stz.news.ua.donbas24.1.jaki_teatr_proekty_mrpl_zakordon_mytci.pic.8}

\begin{leftbar}
\emph{\enquote{Ще дуже багато планів. Зокрема, хочемо поставити виставу, присвячену
художній керівниці Театру ляльок Ірині Руденко та працюємо над
мультфільмом, який присвятимо Маріуполю. Весь матеріал, який у нас був
зібраний з \enquote{Гогольfest dream}, фестивалю iStage 2021 та Марафону
міжнародних резиденцій ми змонтували, протитрували англійською та
французькою і наразі представляємо на європейських медіаплатформах}}, —
додав Палатний.
\end{leftbar}

Також відбуваються перемовини з Шекспіровським фестивалем у Гданську —
містом-побратимом Маріуполя, для подальшого представлення там театральної
програми Маріуполя.

% 9 - Андрій Палатний
\ii{31_07_2022.stz.news.ua.donbas24.1.jaki_teatr_proekty_mrpl_zakordon_mytci.pic.9}

Нагадаємо, раніше Донбас24 розповідав \href{https://donbas24.news/news/vulicya-nilsena-v-mariupoli-shho-vam-vidomo-pro-znamenitogo-arxitektora}{про знаменитого маріупольського
архітектора Віктора Нільсена}.%
\footnote{Вулиця Нільсена в Маріуполі: що вам відомо про знаменитого архітектора, Алевтина Швецова, %
donbas24.news, 31.07.2022, \par%
\url{https://donbas24.news/news/vulicya-nilsena-v-mariupoli-shho-vam-vidomo-pro-znamenitogo-arxitektora}}

ФОТО: з відкритих джерел та особистих архівів Андрія Палатного і Ольги Демідко.

\ii{insert.author.demidko_olga}
%\ii{31_07_2022.stz.news.ua.donbas24.1.jaki_teatr_proekty_mrpl_zakordon_mytci.txt}
