% vim: keymap=russian-jcukenwin
%%beginhead 
 
%%file 18_07_2021.fb.guzhva_igor.1.svetlana
%%parent 18_07_2021
 
%%url https://www.facebook.com/veprwork/posts/4259029567489221
 
%%author Гужва, Игорь
%%author_id guzhva_igor
%%author_url 
 
%%tags __jul_2021.putin.statja,guzhva_igor,krjukova_svetlana,strana_ua,svoboda_slova,trebovanie,ukraina,uvolnenie
%%title И еще по Светлане Крюковой
 
%%endhead 
 
\subsection{И еще по Светлане Крюковой}
\label{sec:18_07_2021.fb.guzhva_igor.1.svetlana}
 
\Purl{https://www.facebook.com/veprwork/posts/4259029567489221}
\ifcmt
 author_begin
   author_id guzhva_igor
 author_end
\fi

И еще по Светлане Крюковой.

Кампания против нее и требования ко мне ее уволить в точности воспроизводит
технологию, которую уже неоднократно отработали националисты. А именно -
закошмарить работодателя, чтоб тот уволил человека, который высказал
"неправильные", с точки зрения националистов, мысли.

Таких случаев масса. Из последнего - увольнение работника Фоззи за то, что тот
похвалил Басту.

Так вот. Хочу еще раз сказать, чтоб все поняли - я не буду увольнять людей за
высказанные ими мнения на их страницах в соцсетях. Ни за критику статьи Путина,
ни за ее позитивные оценки.

Для меня ценность любого члена нашей команды это прежде всего человеческие
качества и профессионализм. Политические убеждения - это его личное дело (при
условии, если они не влияют на качество его работы).  

Светлана Крюкова - очень крутой профессионал и человек, который в самое тяжёлое
для нас время проявил исключительные качества, стойкость и выдержку. И это
является для меня ценностью.

Светлана - человек, который свободно высказывает свои мысли. И говорит то, что
думает. В студии Шустера она одна вместе с Дианой Панченко встала на защиту
закрытых оппозиционных телеканалов. В то время как вся остальная студия горячо
закрытие поддерживала. Но она не побоялась тогда пойти против всех. 

Итого. Попытки заставить меня сделать то, что обычно делают (к сожалению)
украинские работодатели, которые после окриков стерненок увольняют сотрудников,
ни к чему не приведут. 

Я очень не люблю когда начинается травля. Я это видел во времена Майдана. И мы
все помним к чему это привело. И к чему приводит до сих пор.

Не нужно превращаться в стерненок.

P.S. По поводу вопросов о редакционной политике «Страны» ответ простой - она не
меняется.

\ii{18_07_2021.fb.guzhva_igor.1.svetlana.cmt}
