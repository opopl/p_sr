% vim: keymap=russian-jcukenwin
%%beginhead 
 
%%file 21_11_2020.fb.nitsoi_larysa.1.mova_habar
%%parent 21_11_2020
 
%%url https://www.facebook.com/larysa.nitsoi/posts/3797063926992378
 
%%author 
%%author_id nitsoi_larysa
%%author_url 
 
%%tags 
%%title Моїй доні (вона завуч у школі) дали "хабаря".
 
%%endhead 
 
\subsection{Моїй доні (вона завуч у школі) дали \enquote{хабаря}.}
\label{sec:21_11_2020.fb.nitsoi_larysa.1.mova_habar}
\Purl{https://www.facebook.com/larysa.nitsoi/posts/3797063926992378}
\Pauthor{Ніцой, Лариса}

Моїй доні (вона завуч у школі) дали \enquote{хабаря}.

Камери відеоспостереження зафіксували, як вона постійно спонукає україномовних
дітей не переходити на російську. Не секрет, що в україномовних дітей по всій
Україні - проблема з мовною стійкістю. Потрапляючи в коло російськомовних
дітей, україномовні переходять на російську, і байдуже, звідки ці діти, зі
Львова чи Дніпра. Слабка мовна стійкість в українців - це наслідок геноциду. Цю
нашу ваду нам наші колонізатори подали, як нашу перевагу, мовляв, чемні люди
переходять на мову співрозмовника. Насправді, це не так. І слабка стійкість, це
не чемність, а наша слабкість. (Іноземних туристів до уваги не беремо).

Так от. Моя доня виховує в україномовних дітей мовну стійкість.

Батьки це побачили з камер і їм це сподобалося, що моя доня зберігає їхнім
дітям україномовність. За це вони подарували їй вели-и-и-ку шоколадку. Таку, що
всьому колективу на перерві вистачило до чаю.  

Тобто, це був не хабар, а вдячність.  

Якби всі вчителі в Україні так робили, як моя доня, ми б давно поклали край
мовному питанню. Чому це так важливо? Бо збудувати міцну країну зможе лише
народ, об'єднаний спільним світоглядом. А спільний світогляд формує спільна
мова.  

Не розуміє цього лише тупезний, або ворог, який косить під тупезного. 

З Днем української гідності всіх і свободи!

\paragraph{Наталія Музира}

Я в школі зовсім не вчитель, але спонукаю дітей до мовної стійкості. І вони вже
знають, що з Наталією Іванівною, можна розмовляти тільки українською...А як я
радію, коли вони, малі прекрасні пташенята, зросійщені батьками і оточенням,
намагаються зі мною говорити українською, а потім ще й зізнаються мені у своїй
любові за те, що я вчу їх УКРАЇНСЬКОЇМОВИ! У мене тоді виростають крила!..

\paragraph{Руслана Курах}

А в нас одразу вмикають задню, або в самих недостатньо міцні хребти, тому браво
доньці!

\ifcmt
pic https://scontent.fiev6-1.fna.fbcdn.net/v/t1.0-0/p75x225/126805569_3856678021033142_285002965218617078_n.jpg?_nc_cat=104&ccb=2&_nc_sid=dbeb18&_nc_ohc=LD_07VAakoEAX_12mqG&_nc_ht=scontent.fiev6-1.fna&tp=6&oh=93cd6318327d723a7b726d094545abb5&oe=5FE5827A
\fi

\paragraph{Олександр Сушко}
\index[names.rus]{Сушко, Олександр!Лупцював і
примовляв: будете битими, доки не вивчите українську мову}

А я сусідів довів до сказу. Під під'їздом сказав, що будете жити в бідності,
доки ваші нащадки не сформують для себе національну ідею. І її основа -
українська мова. Кинулися битися. Довелося згадати уроки карате. Лупцював і
примовляв: будете битими, доки не вивчите українську мову.

\paragraph{Галина Савченюк}

А наша онука вчить мовної стійкості деяких вчителів. Наприклад, не реагує на
звернення \enquote{Маша}, тому що вона від народження Марічка, а деяким вчителям навіть
ім'я старшокласниці важко запам'ятати.

\ifcmt
pic https://scontent.fiev6-1.fna.fbcdn.net/v/t1.0-9/126793032_175097717660813_4964405533787089779_n.jpg?_nc_cat=110&ccb=2&_nc_sid=dbeb18&_nc_ohc=5GFEheSJ6U8AX_1px9Y&_nc_ht=scontent.fiev6-1.fna&oh=f01577b2f2631c88bd98dbd097e6151f&oe=5FE2B85D
\fi

\paragraph{Іра Літ}

Не мій текст, але повністю згідна!

\ifcmt
pic https://scontent.fiev6-1.fna.fbcdn.net/v/t1.0-0/p75x225/126793032_175097717660813_4964405533787089779_n.jpg?_nc_cat=110&ccb=2&_nc_sid=dbeb18&_nc_ohc=5GFEheSJ6U8AX_1px9Y&_nc_ht=scontent.fiev6-1.fna&tp=6&oh=6a20b02799f768510bc1cafec5b7de76&oe=5FE384D1
\fi

\paragraph{Анатолій Назарук}
Ніколи не переходжу на мокшанську. Ніколи!!!

\paragraph{Иван Ищенко}

Лариса Ніцой, ви знову про спільний світогляд і спільну мову... Це повернення в
\enquote{совок}, з вишиванками й жовтоблакитним, але совок... Де європейські
цінності, мультикультуралізм? Всі, як один, стрункими колонами... В наших
націоналістів комуністично-більшовицька свідомість.

\paragraph{Плотніков Руслан}

Якщо народ країни вперто послуговуються мовою окупанта, то це вказує або на
брак національної свідомості або ця країна окупована

\ifcmt
pic https://scontent.fiev6-1.fna.fbcdn.net/v/t1.0-9/126942171_2683561691954117_5864450420777452587_n.jpg?_nc_cat=106&ccb=2&_nc_sid=dbeb18&_nc_ohc=6ZmopD-q6zQAX884M4-&_nc_ht=scontent.fiev6-1.fna&oh=9544af1e54ab076acf8e2ea9e65de6b6&oe=5FE28391
\fi

\paragraph{Svitoslava Osinnya}

Пані Ларисо! Похвастаюсь! Мій внучок ніколи не переходить:). В нього не
виходить навіть:), якщо попросити щось сказати. Минулої осені, на Кіпрі, на
пляжі до нього підбігла дівчинка і радісно так: прівєт, я Варя! Він дивиться і
мовчить, вона знову : прівєт, я Варя, давай іграть! Остап рішуче: я говорю
тільки українською! Варя була з за парєбріка, але хотілося гратися і вона
вже скоро почала повторювати за ним українські слова:), і розуміла все
прекрасно

\paragraph{Лілія Загребельна}

Отаку позицію русифіковані українці сприймають як агресивну. І постійно
спекулюють на доброті, терпимості та рівності. Мовляв, ми обома мовами
володіємо і обидві рідні і потрібно бути добрими один до одного. 

\paragraph{Лілія Загребельна}

\textbf{Иван Ищенко}

тоді звідки проблема говорити українською мовою? Якщо російську ніхто не
насаджував, чому більшість українців говорить саме нею? Ми ж не в росії живемо!
Люди усвідомлюють себе українцями, а говорити своєю національною мовою не
вміють, бо прийняли рішення з дитинства говорити російською. Вам не здається,
що це нелогічно
