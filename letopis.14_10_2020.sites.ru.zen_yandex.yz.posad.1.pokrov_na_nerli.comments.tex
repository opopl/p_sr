% vim: keymap=russian-jcukenwin
%%beginhead 
 
%%file 14_10_2020.sites.ru.zen_yandex.yz.posad.1.pokrov_na_nerli.comments
%%parent 14_10_2020.sites.ru.zen_yandex.yz.posad.1.pokrov_na_nerli
 
%%url 
 
%%author 
%%author_id 
%%author_url 
 
%%tags 
%%title 
 
%%endhead 

\clearpage
\subsubsection{Комментарии}
\label{sec:14_10_2020.sites.ru.zen_yandex.yz.posad.1.pokrov_na_nerli.comments}

\begin{itemize}
\iusr{Евгений Артёмов}

Был в храме дважды, первый раз зимой, второй во время разлива. Впечатления
незабываемые и о храме и о природе.

\iusr{Kotsa Marsik}

Живу в Сибири,в октябре целенаправленно поехал во Владимир.И в Боголюбово конечно тоже.Но погода пасмурная была

\ifcmt
  ig https://avatars.mds.yandex.net/get-zen_pictures/3403730/1077502758-1604311442294/orig
\fi

\iusr{Синильга}

Красивый храм. В красивом месте. Очень удачное расположение для воздействия на
психику

\iusr{КалистаА Александрова}

Синильга, Мне тоже давно впечатляет это храм, что то тайное чувствуется,
великое, и сердцу милое,так хочется побывать ,помолиться ,
Господи сподоби .

Господи спаси и созрани этот храм на многая ,и многая лета, для наших пра-пра
внуков

Спасибо , что показали нам, дай Бог Вам и вашим близким здоровья, и многая и
благая лета вам и храму.

В каждом храме есть Ангел-хранитель храма,он для нас грешных невидим, но по
повелению Божьему они охраняют храм и невидимо доя нас несут
стражу.

Кто обслуживает этот храми наводят порядок и ведут службы священники счастливые
люди и благословенные Господом, многая и добрая лета им и
ихнему роду. НИЗКИЙ ПОКЛОН ВАМ ВСЕМ,от меня и от моих детей и
внуков.Слава Богу, храните веру православную, в нем же нам
утверждение есть!

\iusr{Виктор Saahov}

Нет церкви изящнее и гармоничнее, чем церковь Покрова на Нерли.

Фотография никогда не передаст этой красоты.

И в России очень мало церквей, переживших Орду.

Если вы хотите увидеть чудо, приезжайте во Владимир. Всё равно в какое время
года.

Она прекрасна всегда!

\iusr{Александр Чехонин}

Даже страшно представить, скольких людей разных эпох помнят эти стены. А
резьба, по моему мнению, имеет азиатские мотивы. Напоминает
скифский "звериный стиль". А может быть, мастера были ещё
недавно язычниками.

\iusr{Николай Поляк-Брагинский}

Этот храм построен методом заливки известковым раствором пространства внутри
двойной стены. В результате он стал единым камнем. Лет через 200-300 после
постройки его пытались сломать - у местного князя был зуд перестройки, но не
смогли. Это рассказывала экскурсовод в 2005 году, там тогда шла реставрация, а
наша группа ездила по "Золотому кольцу".

\iusr{Ирина Замятина}

Какие там прекрасные места - Владимиро-Суздальское княжество.

Я бы сказала, что это наш Ватикан. Сколько там всего интересного!

\iusr{Лариса Тодорчук}

Были много лет назад зимой.

Странно, почему автор не сказала, что никогда, даже в самое сильное половодье,
вода рек не поднимается выше первой ступени!

Храм необыкновенный!

Из путешествия по Золотому кольцу запомнилось, что исконный цвет русских -
красный, а национальное животное - пардус ( похоже на современного леопарда!)

Если есть возможность, очень советую поехать по ЗК., лучше зимой, красивее...

\iusr{Денис}

Внутри были фрески самого Андрея Рублева. Был там и вот что заметил, по дороге
к храму ехал чёрный внедорожник, метрах в ста остановился, из него вышел
батюшка. Из храма попросили всех выйти, а батюшка вошёл и сним люди с младенцем
(таинство крещения). Вот и вопрос - сколько это стоит, если есть внедорожник.

\iusr{Сириус}
отредактировано

В 1887 году под видом реставрации были уничтожены свидетельства того, что храм был славянский в прямом пднимании этого слова. В храме не было привычной картины святых апостолов, а на стенах был орнамент, свастики. Так назаваемые батюшками, языческие" персонажи снаружи храма подтверждают это. Их христиане не решились срубить. Тогда храм потерял бы свой облик и стал не интересен.Храм всегда стоял в стороне и не очень досаждал масонам. Однако после строительства железной дороги, паломников стало много, и правительство решило уничтожить артифакты. Как и на всех славянских, по-настоящему русских храмах на

храме изначально не было креста. P. S. Христианство странная религия. Говорят
одно, а делают прямо противоположное. Не в пример славянской вере, Христианами,
хотя они и плачат, что большевики разрушили часть их церквей были уничтожены
вообще все русские храмы, капища застроены церквями. Аминь, или, вернее Амон!

\iusr{Елизавета Квардицкая}

Да, один из самых любимых владимирцами храмов (еще Дмитриевский собор). Река
поменяла русло, но все равно церковь - чудо чудесное. Умели строить наши
предки.

Сам Владимир очень интересно построен: когда едешь со стороны Суздаля, город
показывается два раза, как будто приветствует, а потом только въезжаешь в
город. И со стороны Мурома тоже: сначала въезжаешь на гору у Байгуш - и видишь
великолепие, а потом город раскрывается, когда подъезжаешь к дамбе. Владимир
строился как столичный город, центр княжества.

\iusr{Андрей Чуйков}

\ifcmt
  ig https://avatars.mds.yandex.net/get-zen_pictures/3403730/34322447-1605811630609/orig
  width 0.4
\fi

\iusr{Алла Рубзова}

Храм действительно один из немногих сохранившихся памятников средневековой
Владимирской архитектуры и резьбы, в настоящее время производит совершенно
другое впечатление, чем при возведении, более камерное, так как изначально
вокруг храма была высокая галерея с колоннами (хоры), оставшаяся дверь как раз
выходила на нее. Ещё один замечательный храм этого времени и стиля Дмитриевский
собор во Владимире, резьбы на его стенах больше

\iusr{Ваня Мухин}

Там лучше всего побывать в июне, когда заливные луга войдут в силу, зацветут
разнотравьем. Запах стоит такой медовый одуряющий, птички в траве свиристят.. И
вот это все..

\end{itemize}
