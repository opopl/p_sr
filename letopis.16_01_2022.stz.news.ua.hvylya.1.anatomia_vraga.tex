% vim: keymap=russian-jcukenwin
%%beginhead 
 
%%file 16_01_2022.stz.news.ua.hvylya.1.anatomia_vraga
%%parent 16_01_2022
 
%%url https://hvylya.net/analytics/245483-anatomiya-vraga-chto-ukraina-nedoocenivaet-v-rossii
 
%%author_id savruckij_bogdan,news.ua.hvylya
%%date 
 
%%tags 
%%title Анатомия врага. Что Украина недооценивает в России
 
%%endhead 
\subsection{Анатомия врага. Что Украина недооценивает в России}
\label{sec:16_01_2022.stz.news.ua.hvylya.1.anatomia_vraga}

\Purl{https://hvylya.net/analytics/245483-anatomiya-vraga-chto-ukraina-nedoocenivaet-v-rossii}
\ifcmt
 author_begin
   author_id savruckij_bogdan,news.ua.hvylya
 author_end
\fi

\begin{zznagolos}
Россия является системным врагом именно украинской государственности. Причем,
врагом во всех предыдущих исторических периодах, включая современный.
\end{zznagolos}

\ii{16_01_2022.stz.news.ua.hvylya.1.anatomia_vraga.pic.1}

Современная Россия, наряду с нашей внутренней украинской, \enquote{аутентичной}
неконсолидированностью, общественным инфантилизмом, антисистемностью и
неорганизованностью, как государственная модель, безусловно, является нашим
настоящим внешним врагом. Прежде всего, Россия является системным врагом именно
украинской государственности. Причем, врагом во всех предыдущих исторических
периодах, включая современный.

Исходя из исторических фактов, имевших место в нашей истории, я считаю, что
важнейшим элементом в долгосрочной стратегии противостояния с РФ, является
\textbf{правильное понимание мотивации и особенностей противника}, с которым Украина
имеет дело. Кстати, в своей войне с Украиной, русские используют такое же
китайское правило - \enquote{для того, чтобы проблему понять, ее сперва нужно
правильно назвать}.

\textbf{О очень упрощенном понимании характера нашей гуманитарной войны с РФ за
идентичность рекомендую прочесть в моей статье на \enquote{Хвиле}, ссылку на
которую я обязательно оставлю для Вас в конце этого текста.}

(ССЫЛКА) 

\href{https://hvylya.net/analytics/242912-kakuyu-voynu-vedet-rossiya-s-ukrainoy-chto-nuzhno-znat-o-vtorzhenii-v-yanvare-2022}{%
Какую войну ведет Россия с Украиной: что нужно знать о «вторжении» в январе 2022, Богдан Савруцкий, hvylya.net, 04.12.2021%
}

В массовом сознании современных украинцев, бытует искаженное и примитивное
представление о нашем извечном противнике, которое мешает нам правильно понять
суть нашего противостояния с Россией и которое в самом же Кремле успешно
используют против нас.

В Украине, в силу общего государственного и общественного инфантилизма,
сознание социума перегружено \enquote{политической метафизикой} в отношении совершенно
осязаемых и грозных факторов. Все чаще дежурные \enquote{мольфары} разного калибра и
разного же уровня примитивизма, являются \enquote{спикерами} наших политических
программ и \enquote{политических прогнозов} о том, что, дескать, \enquote{Путин скоро помрет},
Россия - \enquote{это тьма}, это \enquote{дикая монгольская орда необразованных дикарей, а мы
тут все европейцы с магдебургским правом}, поэтому украинцы, как \enquote{абсолютный
свет}, в любом случае как то там победят \enquote{абсолютную тьму}. Иными словами, наши
извечные враги обязательно, сами собой растают, \enquote{як роса на сонці}. По другому
быть просто не может.

Ну, во-первых, в истории и реальной политике не бывает \enquote{абсолютного света}, как
и \enquote{абсолютной тьмы}, бывает лишь наша эффективная стратегия или наше поражение,
являющимися следствием чужой эффективной стратегии. Фейл эпохи Руїни конца XVII
века, глупое поражение проекта Мазепы в начале XVIII века и крах УНР в первой
половине XX века - это, прежде всего, более эффективная стратегия Москвы, а не
происки метафизических \enquote{темных} сил.

Наш детский миф о \enquote{свете}, который должен победить \enquote{тьму}, безусловно, является
наивным упрощением проблемы и инфантильным зашучиванием серьезности опасности,
которую на самом деле представляет собой Россия и ее внешняя политика

\ii{16_01_2022.stz.news.ua.hvylya.1.anatomia_vraga.1.substrat}
\ii{16_01_2022.stz.news.ua.hvylya.1.anatomia_vraga.2.triggery_starta}
\ii{16_01_2022.stz.news.ua.hvylya.1.anatomia_vraga.3.stratagema_vyzhyvania}
\ii{16_01_2022.stz.news.ua.hvylya.1.anatomia_vraga.4.sakralnost_vlasti}
\ii{16_01_2022.stz.news.ua.hvylya.1.anatomia_vraga.5.votchinnoje_soznanie}



\subsubsection{Эффективность стратегемы выживания}

Эффективность государственной стратагемы в России, во многом, определяется
\textbf{степенью жестокости} властной вертикали. Все периоды, когда в Кремле появлялись
условные \enquote{либералы} такие, как Борис Годунов, самозванец Гришка Отрепьев, Петр
III, Павел I, Александр II, Николай II, Хрущев или Горбачев, приводили к
падению легитимности вертикали и, как следствие - к смуте или \enquote{перестройке},
угрожающей мировоззренческой дезорганизацией, территориальными и ресурсными
потерями.

Смута или \enquote{оттепель} в России - это постоянно невыученный урок истории, после
которого в Кремль, отражая коллективное бессознательное и отвергая
\enquote{классический} вариант развития государства, как создание комфортных условий
жизни для подданных, всегда возвращался \textbf{смиритель смуты}, топором, лагерной
баландой и пулей укрепляющий государственную вертикаль, направляющий ее по
\enquote{иррациональной}, но традиционной стратегеме развития \enquote{защиты от хаоса}. То
есть в русло экспансии.

\subsubsection{Эндогенные силы развития стратегии}

В процессе экспансии Москвы на север, юг и восток, сработал исторический
парадокс - колоссальное увеличение \enquote{фронтира} и своей суверенной территории,
даже после \enquote{величайшей геополитической катастрофы} - краха Золотой (или
Большой) Орды в 1480 году, \enquote{не решило проблему выживания}.

Хотя непосредственная внешняя угроза столице стала существенно меньше, огромные
пространства \enquote{территории хаоса} на востоке, севере и юго-востоке продолжали
существовать и это вынуждало Москву, по примеру античного Рима продолжать
экспансию \textbf{в своей стратегеме \enquote{защиты от хаоса} через завоевание новых
территорий}, покорение и мировоззренческую ассимиляцию \enquote{стабильностью} этнически
чуждых народов - от взятия Казани и \enquote{замирения} черемисов до покорения Чукотки
и от завоевания Кавказа до многочисленных кровавых войн с Османской империей и
Персией.

Таким образом, московский экспансионизм, как \textbf{непрерывный экспорт мировоззрения
\enquote{единоначального порядка} и постоянная борьба с \enquote{территорией хаоса}} стали
самоцелью. Последующее появление уже имперских геополитических амбиций и
создание сфер своего влияния методом \enquote{защиты} через нападение стали эндогенной
силой российской государственной стратегии. Причем, этот \enquote{грубый}, на первый
взгляд, метод работает, вне всякой зависимости от исторического периода и формы
идеологемы, будь то воинствующее православие, марксизм, превращенный русскими
большевиками в суррогат религии или же современный путинский неоимперский
конструкт, построенный на ресентименте по отношению к странам запада и культе
\enquote{недовоеванной} войны.

\subsubsection{Пределы экспансии}

Исторические факты свидетельствуют о том, что московская экспансия в их
своеобразной борьбе с \enquote{хаосом} останавливалась лишь там, где ее ограничивали
обстоятельства непреодолимой силы в виде географических препятствий либо же \textbf{в
виде государственных моделей с высоким уровнем внутренней организации и
возможностями своей политической инициативы.} Именно поэтому расширение на запад
было для Москвы не таким успешным, как это происходило на севере, востоке и
юге, где в большинстве случаев, за исключением, пожалуй, Китая, Москва имела
дело с десятками этносов, не имевшими устойчивой государственности.

Западная территория многоуровневой и \enquote{выборной} свободы всегда была не менее
опасна для московской формы \enquote{единоначальной стабильности} чем восточный
\enquote{фронтир хаоса}, однако, западные государственные модели имели более высокий
уровень общественной организации, чем Астраханское или Казанское ханства,
якуты, чукчи, узбеки или племена северного Кавказа, поэтому экспансия Москвы на
запад шла, по объективным причинам, гораздо медленнее.

В тех случаях, когда соседняя государственная модель или объединение государств
показывали высокий уровень общественной организации и субъектности, Москва не
могла проводить эффективную политику экспансии. Ослабленная Ливония, на которую
напал Иван IV (Грозный) в 1558 году, была почти завоевана Москвой, однако,
столкнувшись с хорошо организованной силой в виде Речи Посполитой и королем
Стефаном Баторием, экспансию пришлось свернуть и вскоре уже московские войска с
трудом сдерживали контрэкспансию объединенного польско-литовского государства,
проиграв в итоге двадцатипятилетнюю Ливонскую войну и потеряв на длительный
исторический период уже завоеванные было территории.

Территория Московского царства, как и территория современной Украины, долгое
время было открыта для регулярных и опустошительных походов татар Крымского
ханства - этого осколка Золотой Орды, который нес \enquote{хаос} с юго-запада.
Столетиями московиты прятались за \enquote{засечными чертами}, платили дань крымским
ханам, а хан Девлет Гирей еще в XVI веке доходил до Москвы и жег ее. Крымское
ханство было небольшим государством но его армия и вассальные отношения с
мощным экспансионистом Османской империей политически усиливали позиции
Бахчисарая. Очень долгое время именно крымские татары были важным фактором
геополитического воздействия Стамбула на Московское царство. Однако, как только
Османская империя организационно ослабла, Российская империя Екатерины II в
результате двух кровопролитных войн, смогла осуществить масштабную экспансию на
юг, оккупировав Крым, уничтожив государственность крымских татар, отобрав
северное Причерноморье и выйдя на Кубань, заложила основы для последующей
кавказской экспансии Москвы.

После войны с Швецией 1808-1809 годов, русские относительно легко, почти на сто
с лишним лет отобрали у ослабленного Стокгольма всю Финляндию, включив ее в
свое политическое пространство, отправив Швецию в \enquote{зрительный зал мировой
геополитики}, однако в 1939-1940 годах XX века высокая государственная и
общественная организация финского общества не позволили сталинскому СССР
навязать Финляндии свое понимание \enquote{единоначальной стабильности}. Финны, военным
методом, уложив в мерзлую землю 150 000 советских солдат, ярко и очень доступно
показали Москве величину материально-ресурсной цены, которую придется заплатить
за борьбу с \enquote{хаосом} на их территории. Соприкоснувшись с реальностью, прагматик
Сталин решил на время остановить экспансию, сконцентрировавшись на подготовке к
\enquote{освободительному походу} в Европу. Последующая \enquote{финляндизация}, имевшая место
после Второй Мировой Войны, так же показывает правильное понимание реальности
прагматичной финской элитой, ставившей в приоритет политический суверенитет и
современная экономически успешная Финляндия является подтверждением этой
\enquote{позорной} стратегии.

Речь Посполитая, в разгар внутриполитического кризиса в середине XVII века,
вызванного шляхетским латифундизмом, который у нас принято называть «повстанням
Хмельницького», утратила внутреннюю организацию. Как следствие, ослабла и
потерпела поражение от Москвы. По результатам войны (1654-1667) Речь Посполитая
потеряла Левобережную Украину и Смоленск. Более того, чуть позже, по Вечному
миру 1686 года, за 146 тысяч рублей, в качестве компенсации, Московское царство
выкупило у поляков Киев, который надолго - до 1991 года XX века года оставался
в культурном, политическом и идентичностном пространстве Москвы, как \enquote{мать
городов Русских}.

Сталинский СССР в 50 - е годы XX века, после Второй Мировой Войны продолжал
расширять горизонт экспансии своего влияния в Восточной и даже Центральной
Европе, пока не уперся в жесткие ответные инициативы стран коллективного
запада. После Фултонской речи Черчилля, в Европе от Киля до Триеста вырос
\enquote{железный занавес} и воплощением его стала Берлинская стена,
разделявшая два мира. Москва, увеличившая свои \enquote{мировоззренческие}
границы от Северной Кореи и коммунистического Китая до Чехословакии и Германии,
остановилась лишь тогда, когда США продемонстрировали порог ядерной цены
невосполнимого ущерба, которую придется уплатить Кремлю и \enquote{Карибский
кризис} стал тем \enquote{моментом истины}, после которого, на время, Москва
сбавила свой экспансионистский напор \enquote{единоначальной стабильности}.

Украинские столпы идентичности

С момента своей успешной экспансии на левобережную Украину в середине XVII века, Московское государство добавило в свою стратегию экспансии "единоначальной стабильности" идею "воссоединения русских земель временно оторванных от пуповины" западными моделями "хаоса свободы". Эту идею Кремль реализовывал последовательно начиная от разделов Польши Екатериной Второй в конце XVIII века, вплоть до раздела польского государства, осуществленного Сталиным и одним "австрийским художником" в конце 30-х годов XX века, когда в политическое пространство СССР вошли Западная Украина и Западная Беларусь.

Для нас сегодня крайне важно понимать, что с точки зрения врага, территория и население Украины - это "законно завоеванный", и "временно утерянный", "объединительный" смысл России, который является для Москвы сакральным в контексте самого бытия государства. Киев является одним из столпов, наряду с Москвой, Владимиром, Новгородом, Минском и Питером, которые легитимизируют "истоки идентичности", "общую историю" и цельность основы восточнославянской имперской конструкции. То есть - "один народ", как любит утверждать дедушка Путин. Пока Киев является столицей суверенной Украины, Кремль продолжит войну всеми доступными мерами и не остановится ни перед какими потерями в процессе экспансии вплоть до угроз "ограниченного" применения ядерного оружия. Пусть у Вас не будет в этом даже тени сомнения. Поэтому Лавров после раундов переговоров с США, НАТО И ОБСЕ акцентировал месседж о том, что:

"Россия никогда не угрожала украинскому народу, в отличие от Зеленского и его соратников" (с)

Месседж Лаврова - стандартный имперский прием, когда русские пытаются стравливать некий "страдающий народ" с "преступным режимом". Этой фразой МИД РФ, по сути, выделяет "украинский народ" в отдельный объект, который необходимо "защитить от хаоса", лишая его возможности иметь свою "неправильную" государственность. Угрожать же Кремль, может, якобы, лишь "Зеленскому и соратникам", то есть "нелегитимной", "не сакральной", "неприродной" организации власти.

Осуществляя комплексный процесс уничтожения украинской субъектности, применяя все доступные меры в современной войне, Путин, последовательно, воюет не с украинцами, как с этносом, он ведет войну с даже гипотетической возможностью украинцев стать (!) политической нацией (!), имеющей свою, отличную от России, общественно-политическую организацию. Путин продолжает реализовывать старую стратагему выживания Москвы - он воюет на территории Украины с "хаосом свободы", продвигая на запад экспансию "единоначальной стабильности".

Давайте научимся с Вами говорить правду хотя бы самим себе, даже, если она нам неприятна. Современная Украина, как модель, объективно, является постколониальным государством, не имеющим своих долгосрочных традиций державности, с крайне низкой, деградировавшей общественной культурой и, как следствие - катастрофически примитивной формой политической и государственной организации, основанной на перераспределении оставшихся ресурсов. Именно по этой причине в Украине отсутствует стратагема классического государственного развития и своей государственной защиты. Поэтому, в настоящий исторический момент, Кремль имеет не только теоретическую, но и организационно-практическую возможность осуществлять десубъективизацию Украины для последующей территориальной экспансии с очень высокой вероятностью успеха.

Таким образом - пределами территориальной экспансии Москвы или экспансии ее геополитического влияния - являются:

1. Уровень общественно-политической организации государственных моделей противников

2. Способность государственных элит противников проявлять инициативу и выстраивать долгосрочные стратегии защиты

3. Осознанная обществом противников необходимость платить ресурсную цену за суверенитет. В том числе, в измерении человеческих жизней своих граждан.

Нынешние страны коллективного Запада, несмотря на относительно высокий уровень общественно-политической организации, утратили мотивацию к экспансии смысловых идей, являющихся основой их общественно-политического договора. Отсутствие желания к экспансии не дает возможности выработать эффективную стратегию защиты, и как следствие порождает утрату инициативы. Кремль пользуется этим, в агрессивной форме навязывает свои инициативы "реальной политики" и заставляет своих "партнеров" в Вашингтоне действовать в режиме реакции, блефуя и шантажируя западных политиков "военно-техническими мерами". Реактивная "тактика" имитации "защиты" и боязнь военных мер, которую сегодня демонстрируют в Вашингтоне, стратегически является контрпродуктивной и, с большой степенью вероятности, может привести к существенной сдаче позиций США, как минимум, в Европе.
