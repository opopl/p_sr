% vim: keymap=russian-jcukenwin
%%beginhead 
 
%%file moje.prizyv.my
%%parent moje.prizyv
 
%%url 
 
%%author_id 
%%date 
 
%%tags 
%%title 
 
%%endhead 

\subsection{Обращение жителей города Киева относительно нынешней российско-украинской войны}

\begin{zzquote}
Я отказываюсь принять конец человека. Легко сказать, что человек бессмертен
просто потому, что он выстоит; что когда с последней ненужной твердыни, одиноко
возвышающегося в лучах последнего багрового и умирающего вечера, прозвучит
последний затихающий звук проклятия, что даже и тогда останется еще одно
колебание — колебание его слабого неизбывного голоса. Я отказываюсь это
принять. Я верю в то, что человек не только выстоит — он победит. Он бессмертен
не потому, что только он один среди живых существ обладает неизбывным голосом,
но потому, что обладает душой, духом, способным к состраданию, жертвенности и
терпению.

Уильям Фолкнер, американский писатель (1897-1962)

В довгу, темную нічку невидну, Не стулю ні на хвильку очей,
Все шукатиму зірку провідну, Ясну владарку темних ночей.
Так! я буду крізь сльози сміятись, Серед лиха співати пісні,
Без надії таки сподіватись, Буду жити! Геть думи сумні! 

Contra spem spero, Леся Українка

Борітеся – поборете, Вам бог помагає! За вас правда, за вас сила, І воля святая!

Тарас Шевченко
\end{zzquote}

\ii{moje.prizyv.vstuplenie}
\ii{moje.prizyv.pesnja.zhyt}
\ii{moje.prizyv.idet_vojna}
\ii{moje.prizyv.kievljane}
\ii{moje.prizyv.obstanovka_na_rajone}
\ii{moje.prizyv.nado_chto_to_delat}
\ii{moje.prizyv.rasstanovka_sil}
\ii{moje.prizyv.chto_delat}

