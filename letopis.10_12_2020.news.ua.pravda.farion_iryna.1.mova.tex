% vim: keymap=russian-jcukenwin
%%beginhead 
 
%%file 10_12_2020.news.ua.pravda.farion_iryna.1.mova
%%parent 10_12_2020
 
%%url https://blogs.pravda.com.ua/authors/farion/5fd2181ff3555/
 
%%author Фаріон, Ірина
%%author_id farion_iryna
%%author_url 
 
%%tags mova,jazyk
%%title ЧЕРГОВА МОВНА ОМАНА У СФЕРІ ОБСЛУГОВУВАННЯ: МОВА, ПРИЙНЯТНА ДЛЯ СТОРІН
 
%%endhead 
 
\subsection{Чергова мовна омана у сфері обслуговування: мова, прийнятна для сторін}
\label{sec:10_12_2020.news.ua.pravda.farion_iryna.1.mova}
\Purl{https://blogs.pravda.com.ua/authors/farion/5fd2181ff3555/}
\ifcmt
	author_begin
   author_id farion_iryna
	author_end
\fi

З різних інформджерел лиються реляції, що від 16 січня 2021 року сфера
обслуговування нарешті возговорить природною українською мовою, а не мовою
окупантів чи чужинців. Мовляв, таке праведне благо несе нам демоліберальний
закон "Про забезпечення функціонування української мови як державної" (5670д),
що набув чинности 16 липня 2019 року. Звісно, рускоязичні знову начебто
матимуть підстави поремствувати, як їх пригнічують свідомі українці в Україні,
де державна мова українська.

Нагадаю, що одним із каменів спотикання цього ліберально-гібридного мовного
закону є його прикінцеві та перехідні положення, де йдеться про відтермінування
дії 16-х базових статей закону та міру покарання за їхнє невиконання (читайте
більше про це мій розділ у монографії "Українська реальність крізь призму
терміна". Львів: Видавництво Львівської політехніки, 2019, с. 47-93).

Серед цих статей є і обговорювана нині ст. 30 "Державна мова у сфері
обслуговування споживачів".

То чи справді від 16 січня 2021 року, себто за півтора року від дня ухвалення,
як це зазначено в прикінцевих положеннях закону, у сфері обслуговування –
найвідкритішій і щоденній царині суспільного життя – запанує державна мова?

На жаль, ні, і то з двох основних причин:

по-перше, цього не передбачає ст. 30; по-друге, за порушення мовних приписів у
сфері обслуговування передбачено штраф аж за три роки від часу ухвалення
закону, а не з 16 січня 2021 року.

Якщо п.1 ст. 30 декларативно зазначає, що "мовою обслуговування споживачів в
Україні є державна мова", то всі наступні шість пунктів передбачають
використання інших мов, серед яких п. 3 для нашої ситуації визначальний: "на
прохання клієнта його персональне обслуговування може здійснюватися також іншою
мовою, прийнятною для сторін".

Таким способом показову оману для довірливих у п.1 про державну мову у сфері
обслуговування розбито вщент, бо торжествує припис "мова, прийнятна для
сторін", узятий із радянського мовного закону 1989 року ("Про мови в
Українській РСР", ст. 4, 5, 15, 16, 17, 18).

Основна політико-правнича максима та новостворений неетнічний лінгвономен
"мова, прийнята для всього населення" чи "мова, прийнята для сторін" не просто
нівелювали декларативну ст. 2 про "державну мову українську в УРСР", але й
цілком заперечували державний статус української мови. Цей закон утверджував
колоніяльну модель розвитку мовного питання в Україні, застосовуючи облудну
ширму про державність української мови, що насправді була позбавлена свого
панівного функціювання в абсолютно всіх сферах суспільного життя: від
державних, освітньо-наукових до сфер споживання та обслуговування. Цей закон
виявився останнім подихом радянської моделі квазідержави УРСР та виявом
суспільної свідомости тодішнього панівного класу в Україні – російськомовної
комуністичної партії радянського зразка. І ось тепер цей термін як виплід
останнього подиху радянської імперії оживає в чинному мовному законі і,
зокрема, у ст. 30 про мову у сфері облуговування...То хто ми сьогодні?

Саме ця лукава словосполука "мова, прийнятна для сторін" є досі основним
колоніяльним маркером українського суспільства. Попри те, що українська мова у
сфері спілкування є мовою використання за так званою дивною процедурою "за
замовчуванням", то це аж ніяк не виключає конфлікту на зумисне створеній стадії
"мови, прийнятної для сторін".

Яким способом, наприклад, касир і покупець обиратимуть мову, "прийнятну для
сторін"?

За яким таким приписом у нашій сфері обслуговування мають працювати поліглоти і
знання яких мов мають бути в їхньому асортименті?

Не важко здогадатися, що "мову, прийнятну для сторін", запроваджено для
москвоязичних, що мають поважні інтелектуальні та політичні проблеми з
опануванням мови Шевченка та Бандери. На жаль, убогі пострадянські законодавці
не виявилися таким свідомими, щоб зруйнувати зону комфорту рускоязичних та ще й
вказати їм, що вони тут не просто зайди, а окупанти!

Врешті який існує спосіб розв'язання цієї проблеми віднайдення "мови,
прийнятної для сторін" для органів правопорядку, якщо закон передбачає
покарання аж за півтора року, тобто в середині 2022-го року?

Про це свідчить Кодекс України про адміністративні правопорушення (Стаття 188
(52)), де передбачено відтерміновані на три роки штрафи в різних сферах
функціонування української мови, а сферу обслуговування можна віднайти хіба в
формулюванні "інші, ніж визначені частинами першою – третьою цієї статті":

"Інші, ніж визначені частинами першою – третьою цієї статті, порушення Закону
України "Про забезпечення функціонування української мови як державної" щодо
порядку застосування державної мови -

тягнуть за собою накладення штрафу від двохсот до трьохсот неоподатковуваних
мінімумів доходів громадян або попередження, якщо порушення вчинене вперше.

Повторне протягом року вчинення порушення з числа зазначених у частинах першій
– четвертій цієї статті, за яке особу вже було піддано адміністративному
стягненню, -

тягне за собою накладення штрафу від п'ятисот до семисот неоподатковуваних
мінімумів доходів громадян".

Отже, трубадурні реляції про українську мову у сфері облуговування від 16 січня
2021 року такі ж оманливі й проблемні, як сам радянський термін "мова,
прийнятна для сторін". Цю статтю неможливо втілити в життя хоча б тому, що
зазначене формулювання неминуче породжуватиме конфлікти, розв'язати які також
неможливо, бо штрафи (чи попередження) в цій ситуації зможуть діяти аж за
півтора року. Та чи почнуть вони діяти взагалі, коли корона всього "мова,
прийнятна для сторін". Говорили, балакали – та й сіли заплакали. Направду,
проблема української мови у власній державі, як і 100 років тому: на ній можна
заснути – і прокинутись у тому ж мовному рабстві. Нема національної свідомости
– не буде й мови.
