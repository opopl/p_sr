% vim: keymap=russian-jcukenwin
%%beginhead 
 
%%file 04_12_2021.fb.zharkih_denis.1.politika_tendencii
%%parent 04_12_2021
 
%%url https://www.facebook.com/permalink.php?story_fbid=3136091559937573&id=100006102787780
 
%%author_id zharkih_denis
%%date 
 
%%tags antirossia,geopolitika,gosudarstvo,ideologia,medvedchuk_viktor,obschestvo,politika,rossia,ukraina
%%title Давайте поговорим о хороших тенденциях в политике, а они есть
 
%%endhead 
 
\subsection{Давайте поговорим о хороших тенденциях в политике, а они есть}
\label{sec:04_12_2021.fb.zharkih_denis.1.politika_tendencii}
 
\Purl{https://www.facebook.com/permalink.php?story_fbid=3136091559937573&id=100006102787780}
\ifcmt
 author_begin
   author_id zharkih_denis
 author_end
\fi

Давайте поговорим о хороших тенденциях в политике, а они есть. Сейчас в
ненационалистическом секторе украинской политики пошла гонка за альтернативный
путь развития Украины. Пока еще обыватель не спросил: "А что так можно было?",
ему, обывателю, внушают, что бандеровская Анти-Россия и есть Украина, и никакой
другой Украины быть не может. Все другое, это уже Россия, все другое - убийство
Украины.  На этот миф уже пошли миллиарды долларов, и поток денег продолжается. 

\ifcmt
  ig https://scontent-frx5-1.xx.fbcdn.net/v/t39.30808-6/264419828_3136071879939541_7268581073775197472_n.jpg?_nc_cat=100&ccb=1-5&_nc_sid=730e14&_nc_ohc=ZwZ6umy-JUEAX83QY9S&_nc_ht=scontent-frx5-1.xx&oh=1837c9dba09caff4ee89800115ee5cec&oe=61AFD55C
  @width 0.4
  %@wrap \parpic[r]
  @wrap \InsertBoxR{0}
\fi

Но реальность не то, что не такая, она противоположная - бандеровская
Анти-Россия не может быть страной ни демократической, ни богатой. За 30 лет
Украина не стала богатым и сильным государством, более того, когда она брала
курс Анти-России, то резко летела вниз в экономическом и политическом плане. И
проблема тут не в России или Западе, это ложный выбор. РФ вовсе не СССР, это
совершенно другая страна, я бы сказал, принципиально другая, а Запад в целом, и
ЕС, США и НАТО имеют не только парадную, но и весьма неприглядные стороны. У
России есть трудности с Западом, но у Запада большие трудности с Китаем, да и
внутри самого Запада есть трудности Франции и Британии, Германии и США, всех не
перечислишь. Короче, система не биполярная, а многополярная. То есть, как любит
говорить Ренат Кузьмин "Выход есть!". Но вот украинская политика пока четко не
сказала в чем же он именно. 

Сегодня прозвучал рецепт от Виктора Медведчука:  "Наша команда ОППОЗИЦИОННАЯ
ПЛАТФОРМА – ЗА ЖИЗНЬ знает, как выйти из этой ситуации. Она знает, что надо
делать. Она считает, что в первую очередь мы должны восстановить наше
государство как правовое, обеспечить выполнение законных прав, обязанностей и
свобод. Мы должны строить прагматическую экономику, восстанавливать
экономические отношения с Российской Федерацией, странами СНГ, странами
Азиатско-Тихоокеанского региона. Мы должны восстановить социальную
справедливость и наполнить реальным содержанием социальные программы. Мы знаем,
как это сделать. Мы знаем, где взять деньги, что надо сделать для экономики. Мы
готовы к этому. И мы одновременно констатируем, что власть не имеет ни
политической воли, ни профессиональной подготовки, ни способности для того,
чтобы исправить вот эту социально-экономическую критическую ситуацию в стране".
Это цитата из его последнего интервью. Что же предлагает Виктор Медведчук?

1. Восстановить законность, права граждан, государственные институты в целом
(их серьезно уничтожили псевдореформами). 

2. Вернутся к прагматизму, и уйти из политиканства в экономике. В это входит
экономическое сотрудничество с Россией и странами СНГ. 

3. Социальная справедливость и социальные программы. 

Но дело не только в лозунгах. Медведчук в этом интервью четко говорит, что
нужно для этих трех пунктов.  

1. Для восстановления законности нужна политическая воля. Власть не должна
делать все, что хочет, все, что ей придет в голову. Медведчук иронизирует над
украинскими политическими силами, которые радовались закрытию оппозиционных
каналов и политическим репрессиям. Сегодня машина репрессий пошла на них (немой
вопрос "А нас за что?", так система, батенька, система). Не проявит украинская
политика сегодня волю соблюдать закон и бороться с беззаконием, не будет
никакой украинской политики, недолго ждать осталось. Для закона необходима
политическая воля, а не истеричные хотелки. 

2. Для экономического прагматизма нужен профессионализм. Тут уж Медведчук не
скрывает сарказма относительно умения новой власти торговать и производить
товары. Тут можно быть солидарным с Виктором Владимировичем - Украина 30 лет
назад осталась без дешевой нефти и газа, но последнее время ухитрилась
проср.../потерять свой последний энергетический козырь - уголь.  Это не от
большого ума. Но прагматизм это, в первую очередь профессионализм. Дайте
управлять умным людям, а не крикунам. 

3. Где взять деньги на социальный программы? Медведчук говорит: "Мы знаем где".
Буд-то мы не знаем  @igg{fbicon.wink}  Большую часть интервью занимают вопросы энергорынка,
наведение порядка там не просто сэкономит миллиарды стране, которые нужны
пенсионерам, инвалидам, медикам и учителям, но и даст толчок к
конкурентоспособности украинской экономики. Ведь дешевая энергия конкурентное
преимущество, низкие цены на товары. Возьмите все эти энергетические прокладки
и отдайте их производителям и на социальные нужды и будет вам счастье.  

Конечно, только тем, что сказал Медведчук страну не вернуть в нормальное русло,
аспектов много. Но он говорит серьезно, и по делу, задает тон. Вот когда
большинство украинских политиков будут говорить серьезно и по делу, Украина
начнет набирать экономический и политический вес. А пока мы в плену собственных
иллюзий, увы...

\ii{04_12_2021.fb.zharkih_denis.1.politika_tendencii.cmt}
