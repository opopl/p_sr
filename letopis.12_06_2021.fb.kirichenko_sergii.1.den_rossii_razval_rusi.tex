% vim: keymap=russian-jcukenwin
%%beginhead 
 
%%file 12_06_2021.fb.kirichenko_sergii.1.den_rossii_razval_rusi
%%parent 12_06_2021
 
%%url https://www.facebook.com/permalink.php?story_fbid=3966721396779919&id=100003262938234
 
%%author 
%%author_id kirichenko_sergii
%%author_url 
 
%%tags den_rossii,rossia,rus',rusmir
%%title «ДЕНЬ РОССИИ» – ДЕНЬ РАЗВАЛА МНОГОТЫСЯЧЕЛЕТНЕЙ РУСИ
 
%%endhead 
 
\subsection{«ДЕНЬ РОССИИ» – ДЕНЬ РАЗВАЛА МНОГОТЫСЯЧЕЛЕТНЕЙ РУСИ}
\label{sec:12_06_2021.fb.kirichenko_sergii.1.den_rossii_razval_rusi}
\Purl{https://www.facebook.com/permalink.php?story_fbid=3966721396779919&id=100003262938234}
\ifcmt
 author_begin
   author_id kirichenko_sergii
 author_end
\fi

«ДЕНЬ РОССИИ» – ДЕНЬ РАЗВАЛА МНОГОТЫСЯЧЕЛЕТНЕЙ РУСИ, ДЕНЬ СЕПАРАТИЗМА, ПРЕДАТЕЛЬСТВА И ИЗМЕНЫ ВСЕХ, ЖИВШИХ ЕДИНО НАРОДОВ!

Беда человечеству исходит от тупых, жестоких и жадных человекоподобных существ.

Главное – их вовремя определить и убрать от власти И изолировать от общества!

Будущее человечества в том, что человеку не будет позволено эксплуатировать
человека.

Счастье для людей возможно только среди равных и в коллективе во взаимном
уважении.

ЯРОБОР.

ДЕНЬ РОССИИ (до 2002 года — День принятия Декларации о государственном
суверенитете РСФСР) — 12 июня — государственный праздник Российской Федерации.
Чтобы лучше его осознать и прочувствовать, нужно окунуться в историю.

Мало кто знает, что 2021-й год летоисчисления от Рождества, на самом деле это
7529 лето от Сотворения Мира в Звездном Храме, исчисляемого от последней и
великой битвы наших предков, установивших мир для своих родов и начавших
строить свою новую государственность. Смену этого летоисчисления, никого не
спросив и наплевав на традиции и обычаи народа Руси, среди прочих других добрых
дел, совершил правитель-западник царь Петр I в 1700 году. Хотя даже в архивах
начала ХХ века мы находим множество папок и документов, датированных прежним
летоисчислением наших предков. Если бы не этот легкомысленный поступок молодого
царя Петра, то Россия имела бы более солидную историю и родословную, чем такие
государства как Китай, Япония, Иран, Израиль, которые не стесняются
демонстрировать исторический возраст своих народов. А чтобы стереть саму память
об этих временах истории, в России были уничтожены множество письменных и
вещественных артефактов в угоду «слияния и соития с цивилизацией Великого
Запада».

У Руси всегда было много врагов, от которых русичи  храбро защищали свое
Отечество. Начиная с «Велесовой книги», завершенной в Х веке и через
последующие Повести временных лет многих монахов-летописцев, начиная с XI века,
любой человек может удостовериться, насколько жестоко стремились завоеватели
поглотить Русь, ее богатства, ресурсы и расправиться с ее свободолюбивым
народом. Ведь чем злила Русь деспотов Запада и Востока? СВОИМ НАРОДОВЛАСТИЕМ и
НАРОДОПРАВИЕМ! Тем, что на Руси не было рабства, феодализма и капитализма в тех
формах, какие пережили большинство стран мира, и в которых и на сегодня
продолжают существовать. 

\ifcmt
  pic https://scontent-lga3-2.xx.fbcdn.net/v/t1.6435-9/200512967_3966725870112805_6305265828275665979_n.jpg?_nc_cat=103&ccb=1-3&_nc_sid=730e14&_nc_ohc=dinmELPKE88AX_HG_Xe&tn=ntrKbsW_7ChXu3v-&_nc_ht=scontent-lga3-2.xx&oh=cb0fd89ecfcc28427dfba49f1b7e8e61&oe=60C9B880
\fi

Русь и ее народ, из-за суровости климата и постоянных набегов завоевателей,
выработала КОЛЛЕКТИВНУЮ СИЛУ ОБЩИННОСТИ, где нет места наследственной власти
паразитов, где все хозяйство и все в хозяйстве, в том числе и средства
производства (земля, плуг, лошади, постройки и др.) были общими, так как отцы и
деды гибли часто, и в любой непредсказуемый день. И тут не было условий
разбираться, где и что кому принадлежит. А вся община русичей выживала от
результатов дружности и храбрости всех и каждого! В ЭТОМ ИСТИННАЯ ПРИРОДА
РУССКОГО ЧЕЛОВЕКА – В РЕАЛЬНОМ РАВНОПРАВИИ ВСЕХ И КАЖДОГО! И именно эта ОБЩИНА
на ВЕЧЕ, КОЛЕ, СХОДЕ СТАВИЛА В УПРАВЛЕНИЕ ТУ ИСПОЛНИТЕЛЬНУЮ ВЛАСТЬ, КАК
ПРАВИЛО, КНЯЗЯ, КОТОРЫЕ ДОЛЖНЫ БЫЛИ ПО ВОЛЕ ОБЩИНЫ РЕШАТЬ ТЕ ВОПРОСЫ, КОТОРЫЕ
СОХРАНЯЛИ ЖИЗНЬ ВСЕХ СОРОДИЧЕЙ В ОБЩИНЕ. Назначение было путем выборов, как
правило, на год, и ни в коем случае не предполагало наследственность в
преемственности власти. На уровне княжеского правления это было вплоть до Х
века, а на уровне старост – до XIX-XX  вв.

Да и СМЫСЛ ОТНОШЕНИЙ СОБСТВЕННОСТИ в большинстве стран состоял в ПАРАЗИТАРНОЙ
ПРИРОДЕ, когда наследственные правители гнобили управляемых ради сытой жизни
для семьи и рода правителей, а не для благополучия всего народа.

Как правило, вооруженные вторжения на Русь, рано или поздно, завершались
поражением врагов и победой русов. Но вот невидимые для большинства глаз люда
внутренние интриги и перевороты, гнусности предателей и изменников наносили
Руси больший урон, чем орды вооруженных иноземных завоевателей.

Но, все по порядку. 

Из официальных данных 12 июня стало праздничной датой с 11 июня 1992 года, по
постановлению Верховного Совета Российской Федерации как «День принятия
Декларации о государственном суверенитете Российской Федерации». Отмечается
ежегодно с 1992 (нерабочий день с 1991 года) в день принятия на первом съезде
народных депутатов РСФСР ДЕКЛАРАЦИИ О ГОСУДАРСТВЕННОМ СУВЕРЕНИТЕТЕ РСФСР 12
ИЮНЯ 1990, утвержденной тогда под председательством только что назначенного
Председателя Верховного Совета РСФСР Б.Ельцина. 25 сентября того же 1992 года
были внесены соответствующие изменения в Кодекс законов о труде. 12 ИЮНЯ 1998
ГОДА Б. Н. Ельцин в своём телевизионном обращении предложил переименовать
праздник в «ДЕНЬ РОССИИ». Официально это название было присвоено с принятием
нового Трудового кодекса в 2002 году.

Еще день 12 июня, в 1991 года известен тем, что тогда прошли выборы президента
РСФСР, на которых победил Борис Ельцин.

А теперь нужно разобраться, а что же это за день – 12 июня 1990 года? В чем его
суть и его значение?

Формально, РОССИЯ В КОНКРЕТНЫЙ ДЕНЬ, 12 ИЮНЯ 1990 ГОДА, под руководством
команды Б.Ельцина и прикрытием команды М.Горбачева ОТДЕЛИЛАСЬ ОТ УКРАИНЫ,
БЕЛОРУССИИ, УЗБЕКИСТАНА, КИРГИЗСТАНА, ТАДЖИКИСТАНА, АРМЕНИИ, МОЛДАВИИ,
КАЗАХСТАНА, ТУРКМЕНИСТАНА, еще живших в тот день ЕДИНОЙ СТРАНОЙ. Их объединял
Союз на основании договора от 30 декабря 1922 года Это уж потом, в последующие
после 12 июня 1990 дни, под давлением и «чутким руководством» команды
М.Горбачева, пошел процесс «парада суверенитетов», (после известного призыва
Б.Ельцина «брать всем желающим столько суверенитета, сколько пожелают»). Но,
как видно, НЕ  БЕЛОРУССИЯ, НЕ УКРАИНА, НЕ УЗБЕКИСТАН, НЕ КИРГИЗСТАН, НЕ
ТАДЖИКИСТАН, НЕ АРМЕНИЯ, НЕ МОЛДАВИЯ, НЕ КАЗАХСТАН, НЕ ТУРКМЕНИСТАН ОТДЕЛИЛИСЬ
ОТ РОССИИ, а с точностью, до наоборот. Именно этот факт, СОБЫТИЕ 12.06.1990 г.
с принятием «Декларации о государственном суверенитете РСФСР» и стали ОСНОВОЙ И
ГЛАВНОЙ ПРИЧИНОЙ РАЗРУШЕНИЯ СССР, КАК ЕДИНОГО ГОСУДАРСТВЕННОГО ОБРАЗОВАНИЯ.
Сепаратизм, предательство СОВЕТСКОГО НАРОДА И ИЗМЕНА СОВЕТСКОМУ СОЮЗУ со
стороны Б.Ельцина и М.Горбачева свое сделали, использовав для этого Россию и
тогдашнюю ее элиту.

В этом и есть ОСНОВНАЯ ДРАМА РОССИИ ПОСЛЕДНЕГО ВРЕМЕНИ, когда из собирательницы
и защитницы союзнических земель братских народов, именно России два
высокопоставленных подонка (М.Горбачев и Б.Ельцин) со своим окружением отвели
роль разрушительницы того, что собиралось воедино веками огромными жертвами и
усилиями миллионов людей.

Не праздновать «День России» нужно, на обломках великого и утраченного, а
скорее все делать для того, чтобы воссоединить порушенное в единый, более
могучий союз, преодолеть те уродливые изменения, которые произошли в обществах
каждой из республик на постсоветском пространстве. Опять нужно освобождать
людей из того состояния, когда из человека труда снова сделали раба, гнущего
спину на господ, но уже по новым, гибридным технологиям, когда эти господа
вроде и невидимы, но их иго явно и непосильно. 

Эксперимент с возвратом капитализма и буржуазии не удался! Вновь нужно
народовластие и народоправие. Но уже не под началом Компартии и не в виде
социализма, а виде нового самоуправляющегося общинного строя, где каждому
отводится не роль «винтика», а роль лидера в том, в чем он горазд и умел.

На сегодня Россия и все постсоветские республики живут по западным шаблонам
толпо-элитарной модели управления, в основе которой капитализм с олигархами,
частная собственность эксплуатация человека человеком, при которых средства
производства оторваны от трудящихся. А сами трудящиеся механизмом мошеннических
партийных выборов отделены от возможности формировать истинно народную власть.

На сегодня в России надежды возлагают на сильного лидера-президента. Да, он
имеет высокий уровень поддержки и смог подчинить своеволие олигархов правилам
государственного капитализма . Но это очень временно, максимум в пределах его
жизни. История подтвердила такой факт в подобных ситуациях не раз. Например, в
Х веке со смертью Святослава Хороброго был воссоздан разгромленный им же
паразитарный Хазарский каганат. После смерти в конце XVI века Ивана Грозного –
правление Годуновых и Смутного Времени с нашествием на Русь польских ляхов и
захват земель Руси западниками. После смерти Иосифа Сталина в 1953 году быстро
последовал упадок  и распад СССР с захватом его обломков теми, кто и планировал
его разрушение в ходе Второй Мировой. Так что если нет сильного и эффективного
коллективного управления, то все достижения державы очень быстро низвергаются
врагами, а народ получает жестокий геноцид и очередной виток жестокого инферно
до следующего выхода из тупика и очередного за ним падения в пропасть
жестокости и отсталости.

Сегодня спираль времени и событий вновь вывела умы лучших сынов и дочерей Руси
к осознанию того, что нужно не медля возродить Державу народовластия трудящихся
на Поконе Рода Всевышнего, на Прави, на Совести, на Общинности, на
Коллективизме, на Товариществе. Державу Духа и Идеи! Иначе сирые и убогие
богатеи, вершащие власть над людом, и далее будут держать в отсталости и
примитивном потребительстве народные массы, сдерживая могучую силу единения
народа в его высоком потенциале к Прогрессу и Творчеству.

Дерзайте смелые и отважные! Не бойтесь бороться и побеждать! Бойтесь бесславно умереть в тихости и смирной покорности!

Русь не умерла! Она живет в Природе, в Святой Земле-Матушке, в Духе Рода, в
крови потомков славных предков, строивших Русь и делавших ее могучей и
непобедимой. Нужно только усилие, чтобы ее возродить до высот могущества и
процветания.

Сергей Кириченко
