% vim: keymap=russian-jcukenwin
%%beginhead 
 
%%file 21_08_2021.fb.fb_group.story_kiev_ua.1.zamkovaja_gora
%%parent 21_08_2021
 
%%url https://www.facebook.com/groups/story.kiev.ua/posts/1734551180075009
 
%%author_id fb_group.story_kiev_ua,sirota_tatjana.kiev
%%date 
 
%%tags gorod,kiev,kiev.gora.zamkovaja
%%title Замковая гора
 
%%endhead 
 
\subsection{Замковая гора}
\label{sec:21_08_2021.fb.fb_group.story_kiev_ua.1.zamkovaja_gora}
 
\Purl{https://www.facebook.com/groups/story.kiev.ua/posts/1734551180075009}
\ifcmt
 author_begin
   author_id fb_group.story_kiev_ua,sirota_tatjana.kiev
 author_end
\fi

%\setlength{\parindent}{0pt}
%\zzBukva{Г}оворят, что \enquote{Умный в гору не пойдёт, умный гору обойдет!}

\lettrine[lraise=0.1, nindent=0em, slope=-.5em, lines=2]{\textbf{Г}}{оворят},
что \enquote{Умный в гору не пойдёт, умный гору обойдет!}

Сколько раз я обходила эту гору ( не сосчитать ) по урочищу
Гончары-Кожемяки ( нынешней Воздвиженке ), Андреевскому и Притисско-Никольской!

\begin{multicols}{2} % {
\setlength{\parindent}{0pt}

\ii{21_08_2021.fb.fb_group.story_kiev_ua.1.zamkovaja_gora.pic.1.gogol}
\ii{21_08_2021.fb.fb_group.story_kiev_ua.1.zamkovaja_gora.pic.1.cmt}
\end{multicols} % }

\lettrine[lraise=0.1, nindent=0em, slope=-.5em, lines=2]{\textbf{И}}{}
вот сегодня я, наконец-то, решила на неё подняться... Но не одна (одна бы не
рискнула, кто его знает, что там наверху))) ), а в компании своего постоянного
спутника моих летних прогулок, шестилетним внуком.

Почти, как птичка))), я взлетела (но, если честно, то вскарабкалась) на эту гору, под
названием Замковая, по крутой лестнице, что ведёт наверх с Андреевского спуска.

Много слышала и читала, что отсюда открываются красивые виды на Город, но 
увиденное превзошло мои ожидания - виды с этого места ВЕЛИКОЛЕПНЫ!!!

\ii{21_08_2021.fb.fb_group.story_kiev_ua.1.zamkovaja_gora.pic.2}

Побродив по горе, полюбовавшись красотой, что предстала перед нашими глазами, мы
по тропе через заброшенное кладбище, по полуразрушенной старой лестнице
спустились на Фроловскую. А тут уже и Контрактовая рядом...

\ii{21_08_2021.fb.fb_group.story_kiev_ua.1.zamkovaja_gora.pic.2.zamkovaja}

А теперь немного про то место, где мы сегодня побывали...

\lettrine[lraise=0.1, nindent=0em, slope=-.5em, lines=2]{\textbf{З}}{амковая}
гора – одна из знаменитых мест силы и, так называемых, \enquote{лысых гор}
Киева. Этот исполинский холм возвышается над Днепром на целых 80 метров.

Её также называют Фроловской или Флоровской горой, Киселёвкой, Киевицей и
Хоривицей.

Говорят, что перед возведением Киева, на Замковой горе располагалась резиденция
князя Кия. 

Во времена Киевской Руси Замковая гора, очевидно, носила имя Хорива – третьего
из легендарных братьев-основателей Киева. Однако эпохи сменились, Киев был
захвачен Батыем, а в XIV столетии на склонах Днепра обосновался литовский князь
Владимир Ольгердович. По его приказу здесь построили деревянный замок, который
на протяжении нескольких веков представлял собой неприступную крепость. Позже
на его месте возвели замок воеводы Адама Киселя, а еще позднее — мощную
фортификацию. Вот и стала гора называться Замковой. Очередной виток истории
потряс эту местность в 1651 году, когда княжеский замок оказался полностью
уничтожен восставшими казаками. Новые власти восстанавливать сооружение не
стали, а о самой Замковой горе вспомнили только в 1816 году  - гору отдали
Флоровскому монастырю и на ней обустроили кладбище.

И сейчас здесь можно найти старинные могилы, датированные 19-тым веком.

В послевоенные времена, гора не принадлежала никому, хотя монастырь и пытался
ее забрать под свое попечительство. На сегодняшний день Замковая гора
принадлежит городу.

20 августа 2021 года.
