% vim: keymap=russian-jcukenwin
%%beginhead 
 
%%file 05_12_2021.fb.baumejster_andrej.kiev.filosof.1.skazka_andersen_snezhnaja_koroleva.cmt
%%parent 05_12_2021.fb.baumejster_andrej.kiev.filosof.1.skazka_andersen_snezhnaja_koroleva
 
%%url 
 
%%author_id 
%%date 
 
%%tags 
%%title 
 
%%endhead 
\subsubsection{Коментарі}

\begin{itemize} % {
\iusr{Андрей Баумейстер}

Прекрасный советский мультфильм \enquote{Снежная королева} (1957). Но из него выброшены
все религиозные сюжеты: сила молитв, Рождественский псалом как главная сюжетная
линия и, конечно, евангельская цитата \enquote{Если не будете как дети...}. Без
религиозной линии сама история. становится одномерной и во многом непонятной:

\href{https://youtu.be/1ajPJI5UyS0}{%
Снежная королева, youtube, 03.10.2013%
}

\ifcmt
  tab_begin cols=3,no_fig,center

     pic https://i2.paste.pics/1a42732a213c666df9f8bf3a10a2e7d2.png
		 pic https://i2.paste.pics/ac93135163c20cca871a4d48a506a2f1.png
		 pic https://i2.paste.pics/96d45b719d5c2b8fd71e08c9b423b129.png

  tab_end

  tab_begin cols=4,no_fig,center

		 pic https://i2.paste.pics/6160894283bfc97974bae133a9c99f88.png
		 pic https://i2.paste.pics/5df55df95d62b15db983380bcad54c39.png
		 pic https://i2.paste.pics/0d42ca1015f8d6fe58799e84d01abf7f.png
		 pic https://i2.paste.pics/8b8737634ca32b88c4daad1f1166d2be.png

  tab_end
\fi

\begin{itemize} % {
\iusr{Irina Zuevic}
\textbf{Andrii Baumeister} да, наверное, именно потому что не читала, не поняла как Вы. Многие его сказки читала, но эту смотрела.
\end{itemize} % }


\iusr{Андрей Баумейстер}

Советский фильм 1966 года (премьера 6.11.1967) по сценарию Шварца (1938). Шварц
вступает в творческий диалог с Андерсоном. У детей появляется взрослый
наставник, \enquote{духовный отец} (прообраз самого Андерсена). Он внезапно возникает в
самые сложные моменты путешествия Герды, чтобы ей помочь. Главный мотив у
Шварца сила духа и самопожертвования. Вместо зеркала Тролля магической силой
обладают поцелуй и дыхание Снежной Королевы:

\href{https://youtu.be/_XiBoSZwKUw}{%
Снежная королева (реж. Г. Казанский, 1966 г.), youtube, 02.01.2021%
}

\headTwo{Киностудия Ленфильм}

Фильм-сказка повествует о необыкновенном путешествии скромной маленькой девочки
Герды. Она разыскивает своего друга Кея, которого похитила и унесла в свое
царство Снежная королева - могущественная злая волшебница.

В поисках любимого друга Герда попадает в замок к хитрому, коварному и в то же
время смешному королю, знакомится с лесными разбойниками. На пути у девочки
будет много преград до решающей битвы со Снежной королевой. Но верное сердце
Герды одолеет все невзгоды...

\obeycr
Режиссёр: Геннадий Казанский
Сценарист: Евгений Шварц
Композитор: Надежда Симонян
Операторы: Вадим Грамматиков, Сергей Иванов
Художники-постановщики: Борис Бурмистров, Е. Яковлева, А. Вагин 
В ролях: Елена Проклова, Слава Цюпа, Наталья Климова, Евгения Мельникова, Николай Боярский, Евгений Леонов, Ирина ГубановаГеоргий Корольчук, Ольга Викланд, Эра Зиганшина
\restorecr

\ifcmt
  tab_begin cols=3,no_fig,center
		pic https://i2.paste.pics/2bc298d3dcd0db933fb0d05a52364256.png
		pic https://i2.paste.pics/2bc298d3dcd0db933fb0d05a52364256.png
		pic https://i2.paste.pics/dd4401dd75a76b4fc905d7855b0a054b.png

  tab_end

  tab_begin cols=4,no_fig,center
		pic https://i2.paste.pics/265046d41e42e4e055df0390719d35c4.png
		pic https://i2.paste.pics/094f2cd0a71c1c0ba47d7afb672e40b3.png
		pic https://i2.paste.pics/2100ab1a3824a7c09cb53667ed0d81b7.png
		pic https://i2.paste.pics/3d99f7ea52adc27d69d77e4a7ae73879.png

  tab_end

  tab_begin cols=3,no_fig,center
		pic https://i2.paste.pics/c1890533fcbbf65b5d4796f248bacbd2.png
		pic https://i2.paste.pics/71aaa245ac37844b8125855cbad05659.png
		pic https://i2.paste.pics/7110dcf23e8005b118906e7ae39f2732.png

  tab_end
\fi

\headTwo{Комментарии к видео}

\begin{multicols}{2}
\iusr{Светлана Щ}

как я детстве всё время ждала передачу \enquote{В гостях у сказки}, ждёшь когда
наступит пятница и бежишь из школы быстрей домой, чтобы успеть. Говорят,она
выходила в разные годы по-разному, но вот я запомнила пятницу. Поделки помню
как показывали, потом дома я сама что-то мастерила после передачи. И какие
чистые и искрение сказки были, и я бы даже сказала интеллигентные, т.к. не было
даже намёка на пошлости и мерзость, которую сейчас целыми днями показывают по
телевидению и смотрят в интернете.

\iusr{Наталья Проплетенина}

В детстве обязательно в доме на Новый год ёлка, мандарины и сказка Снежная
королева, именно этот фильм) Чудесный фильм, спасибо)

\iusr{Кахримонджон Махмудов}

Воспоминаю зимнее каникулы. У нас было настоящее детство. Спасибо тебе СССР.....

\iusr{Richard Mueller}

\enquote{Что враги с нами сделают, пока сердца наши горячи? Да ничего!} - ответил
Сказочник в конце фильма. И он прав!

\iusr{Richard Mueller}

Спасибо  моей любимой Родине СССР за непревзойдённое, детско-юношеское кино! За
воспитание  Человека с большой буквы, низкий поклон!

\begin{itemize} % {

\iusr{Александр Иванов}

Когда вышел этот фильм, все мерзавцы, развалившие и грабящие нашу страну как
раз были детьми, ровесниками Герды и Кая. Фильм они наверняка смотрели, а
теперь заявляют, что СССР кроме калош ничего не выпускал! Вот как так
получилось?

\iusr{Michael Chernyak}

и какое воспитание нам показывает данный фильм? Чудная сцена: принц и принцесса
играют во дворце... за ними носится дюжина пожилых людей со свечами, исполняя
малейший каприз. Высокие отношения! Потрясающий пример для советского
школьника!... типа-знай своё место, если ты не принцесса.

\iusr{Александр Иванов}

 @Michael Chernyak  Тут как раз хотели показать обратный эффект что быть
 принцессами- это стыдно.

\iusr{Michael Chernyak}
 @Александр Иванов  чота по фильму не видно. Наоборот, до финала принцесса
 развлекается и гоняет всех по дворцу.

\end{itemize} % }

\iusr{Richard Mueller}

Самые лучшие, самые человечные, самые добрые - это СОВЕТСКИЕ экранизации сказок!

\iusr{Игорь Литвиненко}

Какие прекрасные были сказки, союз мультфильм показывал мультики которые можно
было смотреть по 5-10 раз, а самое главное они были настоящими! Помню в детстве
собирались брали белое покрывало доставали фильмоскоп и смотрели красочные
мультфильмы! Сейчас же дети все в интернете. Жаль, нельзя вернуть время
Юппи, Зуко, жвачки турбо!

\iusr{Марина Марина}

Ребята! Спасибо Вам огромное, что выставляете Советские фильмы и мультики!!!
Они очень нам всем нужны!!!!!!!!! Счастья вам и здоровья!!!!!!!! Одесса

\iusr{Ирина Богданова}

\enquote{детей надо баловАть, тогда из них вырастут  настоящие разбойники!}  крылатые
слова!!!

\iusr{Vik Sud}
Когда  в детстве увидел заглядывающую в окно Королеву, спать не мог несколько ночей.

\begin{itemize} % {
\iusr{Mako Mako}
 @igg{fbicon.face.grinning.sweat}  @igg{fbicon.face.tears.of.joy}{repeat=5} так жизненно

\iusr{Ксюшечка-Оксаночка Обожаю петь! Это счастье!!!}
Бедняжка.

\iusr{Новый Аккаунт}
Снежная Королева виновата в том, что мужики стали боятся красивых женщин!
\end{itemize} % }

\iusr{Игорь Ануфриев}

Сказка и фильм и наш мультфильм изумительны. Проклова ещё чистая девочка, и
сейчас скотилась к грязным склокам (к большому сожалению). Спасибо всей
киносьемочной группе и знаменитой киностудии Ленфильм. Помним. Золотой фонд.

\end{multicols}

\end{itemize} % }
