% vim: keymap=russian-jcukenwin
%%beginhead 
 
%%file 19_01_2021.fb.helgiv_oleg.1.mova
%%parent 19_01_2021
 
%%url https://www.facebook.com/oleg.helgiv/posts/411031396834679
 
%%author Хелгив, Олег
%%author_id helgiv_oleg
%%author_url 
 
%%tags mova,ukraina,ukrainizacia,zakon
%%title Пройшли перші дні обов'язкового обслуговування на українській мові
 
%%endhead 
 
\subsection{Пройшли перші дні обов'язкового обслуговування на українській мові}
\label{sec:19_01_2021.fb.helgiv_oleg.1.mova}
 
\Purl{https://www.facebook.com/oleg.helgiv/posts/411031396834679}
\ifcmt
 author_begin
   author_id helgiv_oleg
 author_end
\fi

Пройшли перші дні обов'язкового обслуговування на українській мові. Як бачимо,
ніхто щелепу не зламав, а для багатьох це стало поштовхом. Бо в русифікованих
містах може люди і хотіли перейти на рідну мову, але звичка, інерція мислення,
агресивні ватники клієнти. А зараз все. Закон.

Буде так, як з свого часу було з обов'язковим дублежем фільмів на українській,
та квотами на радіо і ТБ. Спочатку воплі та соплі, про те що це неможливо,
через пів року звикли, через рік сподобалося, через три роки всі вже вважали,
що так було завжди.

Людина живе звичками. Тому гарні треба виховувати

\ii{19_01_2021.fb.helgiv_oleg.1.mova.cmt}
