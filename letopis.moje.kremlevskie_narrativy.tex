% vim: keymap=russian-jcukenwin
%%beginhead 
 
%%file moje.kremlevskie_narrativy
%%parent moje
 
%%url 
 
%%author_id 
%%date 
 
%%tags 
%%title 
 
%%endhead 

\subsection{Кремлевские нарративы}

Кремлевские нарративы. Здесь - кремлевский нарратив, там - кремлевский
нарратив. Ущемление прав русскоязычных - Кремль. Геноцид на Донбассе -
кремлевский нарратив. Обнищание населения и гражданская война - Кремль. Высокие
тарифы - Кремль снова. Бунтуют против вакцинации - снова Кремль.  Все вокруг -
агенты Кремля. Вот этот - агент. И этот тоже - агент. И вот эта милая
светловолосая девушка, что сейчас в наушниках у себя слушает Газманова, уже
пол-часа тщетно ожидая маршрутку на Троещину - конечно, тоже, агент Кремля.  И
даже более того, как сказала где-то Лариса Ницой, в Киеве Украины нет, тут
вообще повсюду Путин. Вы думали, Путин в Кремле? Да нет, он давно уже в Киеве,
ведь Лариса Ницой ошибаться не может, и ведь действительно тут всюду говорят на
русском. И так далее.

\ifcmt
  tab_begin cols=2,no_fig,center
     pic https://avatars.mds.yandex.net/i?id=31ae2249ba1e06400bddba166a204810-5874640-images-thumbs&n=13
		 pic https://s00.yaplakal.com/pics/pics_original/0/7/4/4208470.jpg
  tab_end
\fi

Слушайте. Ну надоело уже. Все время несется изо всех утюгов... Кремль, Кремль,
Кремль. Кремль сделал то, Кремль сделал это. Путин мечтает о том, Путин хочет
это. Скажите вот по-честному, Вы правда влюблены в Кремль, в Путина, или что? В
каких-то кремлевских рабов превратились на ментальном уровне. Ни дня не можете
прожить без Кремля и Путина. Путин - просто какой-то бог у вас, причем бог
противоречивый. С одной стороны, это кремлевский карлик, какой-то просто
сумасшедший, а с другой стороны, вы все его всегда внимательно слушаете и
боитесь. 

\ifcmt
  tab_begin cols=3,no_fig,center
     pic http://photos.wikimapia.org/p/00/03/86/83/55_full.jpg
		 pic https://upload.wikimedia.org/wikipedia/commons/5/5e/%D0%A2%D1%80%D1%91%D1%85%D1%81%D0%B2%D1%8F%D1%82%D0%B8%D1%82%D0%B5%D0%BB%D1%8C%D1%81%D0%BA%D0%B0%D1%8F_%D1%83%D0%BB%D0%B8%D1%86%D0%B0_%D0%9A%D0%B8%D0%B5%D0%B2_2012_02.JPG
		 pic https://avatars.mds.yandex.net/i?id=3e72a0b484b3f8631780ad2d1e399331-4077532-images-thumbs&n=13 
  tab_end
\fi

А как же Святая София, Киев, Днепр, университет Шевченко и институт
философии имени Сковороды? Где ваши собственные Киевские нарративы? Где ваше
Киевское слово, твердое, уверенное и решительное, основанное на тысячелетней
мудрости и истории Стольного Града Киева? Разве ж Киев не старше Москвы, не
мудрее, не умнее Москвы?  Почему так много Москвы в информационном пространства
и так удручающе мало Киева? Везде Москва, Москва... также везде Украина,
Украина и еще раз Украина... НАТО, ЕС, лучшие в мире западные партнеры...  а
бедный Киев... Спрятался где-то в красивых альбомах и фейсбук-группах по
истории Киева... Как же так можно, елки-палки, обидно же! Вы же каждый день
ездите не в Московском метро, а в Киевском. В красивейшем киевском метро вы
едете от Арсенальной до Гидропарка, а может быть - от Золотых Ворот до
Выдубичей. Берете Киевское такси, например, Уклон или Болт, если надо быстро
куда-то доехать, а не московское такси - как оно называется, извините, - мы не
в курсе, наверное, надо позвонить в Кремль и узнать у Пескова; кладете ребенка
в Киевскую больницу, например, Охматдет, а не в институт Склифосовского (или
какие там еще есть больницы?). Почему так мало гордости за Киев, за Столицу,
почему такое безумное преклонение перед Москвой.  Киев же уже как тридцать лет
имеет собственное политическое значение, времена, когда весь центр принятия
важных решений по Украинской ССР находился в Москве, давно уже в прошлом, так
почему же вы не пускаете в ход наш родной Киев? Слово Киева, ценности Киева,
мудрость Киева? 

\ifcmt
  tab_begin cols=2,no_fig,center
		 pic https://ic.pics.livejournal.com/tov_tob/21010671/2766179/2766179_original.jpg
		 pic https://avatars.mds.yandex.net/i?id=39c3b676331a509accd7cdfa13574214-5870057-images-thumbs&n=13
  tab_end
\fi
