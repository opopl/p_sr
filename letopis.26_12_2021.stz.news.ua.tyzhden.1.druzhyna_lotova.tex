% vim: keymap=russian-jcukenwin
%%beginhead 
 
%%file 26_12_2021.stz.news.ua.tyzhden.1.druzhyna_lotova
%%parent 26_12_2021
 
%%url https://tyzhden.ua/Columns/50/253988
 
%%author_id ljutyj_taras
%%date 
 
%%tags 
%%title Згадуйте дружину Лотову
 
%%endhead 
\subsection{Згадуйте дружину Лотову}
\label{sec:26_12_2021.stz.news.ua.tyzhden.1.druzhyna_lotova}

\Purl{https://tyzhden.ua/Columns/50/253988}
\ifcmt
 author_begin
   author_id ljutyj_taras
 author_end
\fi

Матеріал друкованого видання № 51 (735) від 22 грудня.

\begin{zznagolos}
Навіщо читати? Таке питання цікавить багатьох людей. Невже для того, щоб
відірватися від поточних подій, а не для осмислення світу? Та чи не зумовлена
нині криза читання переважно тим, що занурення в тексти потребує неабиякого
інтелектуального зусилля? Григорій Сковорода так говорив про причини написання
трактату «Жена Лотова»: це була потреба прогнати нудьгу. Проте читається цей
твір не для розваги, а заради потрапляння в осідок вічності.
\end{zznagolos}

Усе, що лежить на видноколі, — минуще. Проте Біблія здатна навчити, як помічати
невидиме. Не за допомогою прямих указівок. Вони — лише натяк. Як дим указує на
вогонь, так і світ є лише запоною, за якою не згасає полум’я вічності.
Нетерплячий читач усіляко чіпляється за знаки, бо за ними не добачає нічого
істотного. Ніби напівсонний блукає він стежинами текстового мороку. Але час
прокидатися колись настане. Ілюструючи труднощі такого читання, Сковорода
вдається до відомої історії з Книги Буття. Коли Бог вирішує покарати Содом,
праведнику Лотові з двома його доньками та дружиною він наказує вшитися з міста
не озираючись. Однак недовірлива жінка не змогла дотриматися перестороги, таки
глянула назад і перетворилася на соляний стовп. У традиції старого українського
письменства цей сюжет нерідко тлумачили буквально. Та для Сковороди він означає
таємницю.

Як правильно відчитати притчу? Щойно перед читачем постає таке питання, він
потребуватиме наставника. Серед своїх провідників у символічний світ філософ
називає імена богословів, які пропонували переважно моральні сенси Писання:
Василія Великого, Івана Златоустого, Григорія Назіанзина, Авґустина та інших.
Услід за апостолом Павлом отці церкви наполягали, що ненависні помисли нуртують
у зовнішньому світі, а добросерді — в Божому. Звідси Сковорода висновує: смак
до правильного прочитання виникає з любові, коли людина наважиться на
оновлення. Здається, він жартує, коли каже, що тут не обійтися без очищувальних
пігулок. Не вживши їх, кульгатимеш на обидві ноги — матеріальну й духовну.
Такий вигляд має підготовча школа розуміння. Зважмо на ще одну особливість.
Біблія не позбавлена дияволових тенет, проваль і принад, якщо сприймати її
буквально. Мусиш обережно поводитися з óбразами, щоб не зробитися плотяною
скотиною. Читати абияк, пожадливо захлинаючись текстом, негоже. Ковтання без
дозування несмачне: «Багато жерти, а мало жувати — то дурне… Учений забагато
жере. Мудрий їсть мало й зі смаком». Те, що не йде з думки Сковороди, актуальне
донині: «Якщо в наш час або в нашому краї мудреців поменшало, порівняно з
іншими часами чи краями, то причиною є те, що тиняємося незчисленними та
різноманітними отарами книжок — без міри, без добору, без гавані». Цієї міри не
здобудеш, допоки не народишся згори.

\begin{zznagolos}
Усе, що лежить на видноколі, — минуще. Проте Біблія здатна навчити, як помічати
невидиме. Не за допомогою прямих указівок. Вони — лише натяк. Як дим указує на
вогонь, так і світ є лише запоною, за якою не згасає полум’я вічності 	
\end{zznagolos}

У своєму трактаті Сковорода раз у раз повторює фразу з Євангелія від Луки:
«Згадуйте дружину Лотову». Що це означає? Передовсім розбирати, розгадувати,
тлумачити óбрази, себто розбивати ідола буквальності. Відділяти тінь привида
від духа бачення. Біблійне море складається з двох частин: людської та Божої.
Філософ не скупиться на символічні фігури, вказуючи на переродження: Новий
Світ, Люд Божий, Земля живих, Країна й Царство Любові, Республіка, Горішній
Єрусалим тощо. Його утопічна Країна Любові, як і безначальна істина, замкнена
на самій собі. Це кільце вічності, схоже на прадавню доктрину про вічне
повернення. Власне, потрапляння в коловорот і потребує другого неземного
народження. Знаю Людину! — вигукує Сковорода словами з Другого послання
апостола Павла до коринтян. Земну содомську людину створено з плоті та крові.
Вона вихваляється, як мавпа; перемінлива, наче місяць; марнославна як сатана;
лукава, ніби змія; схожа на сиромудрого кумира, що мацає всюдибіч, як сліпий
кріт. Вона як жінка Лотова: реа­гує на зовнішні подразники, бачить усе
безпосередньо, а отже, помиляється. Відтак перетворюється на скелю, стає
жупелом сластолюбства. Лише відкриття простору свободи дасть людині змогу
літати без меж, роззиратися довкола голубиним оком, провіщати сокровенне,
бачити минуле й майбутнє. Вона володітиме вмінням зазирати поза символи.

Сковорода дає практичні настанови з тлумачення Біблії. Щой­но натрапиш на
подразливі сюжети (пияцтво, наложництво, кровоміство, амурні пригоди тощо) — не
варто залишатися на цих «содомських вулицях». Не вони є метою. Щоб досягти
кінцевого призначення, мусиш здолати чимало зваб і притичин. Ох, і невже це
якась глибока істина? Сковорода й сам вважає, що ні, бо зазначає: про це
торочать бабські плітки і знає перша-ліпша дурепа. Підказка філософа зводиться
до того, що Біблія є мостом, який треба щоразу зводити наново. Сюди можна
залучити ідею герменевтичного кола. Справді, розуміння тексту передбачає
спрямування від частин до цілого й навпаки, адже символічні образи Святого
Письма (частини) відсилають до премудрості Бога (ціле). «Біблія є правдива
ліщина. Вирви з цієї ліщини один горіх… Розкуси його та розжуй. Тоді й розжував
усю Біблію». Або: «Частини розбитого дзеркала єдиним увесь Лик відображають».
Так і прокладається шлях до власного серця. Та сягнувши його, непевність і далі
манитиме тебе до себе, щоб знову штовхати до вічності. 
