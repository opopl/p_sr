%%beginhead 
 
%%file 05_05_2023.fb.kafedra.kulturologii.mdu.1.ekskursia_jaroslaviv_val_zoloti_vorota
%%parent 05_05_2023
 
%%url https://www.facebook.com/100090026247557/posts/pfbid041NL8bmvDirSsMgrUPAE5SuTfmEkJAVNCdLZAuCajNLJ4dYqHmtZ5D3rL1b96Fcpl
 
%%author_id kafedra.kulturologii.mdu,demidko_olga.mariupol
%%date 05_05_2023
 
%%tags 
%%title Екскурсія - Ярославів Вал - Золоті Ворота
 
%%endhead 

\subsection{Екскурсія - Ярославів Вал - Золоті Ворота}
\label{sec:05_05_2023.fb.kafedra.kulturologii.mdu.1.ekskursia_jaroslaviv_val_zoloti_vorota}
 
\Purl{https://www.facebook.com/100090026247557/posts/pfbid041NL8bmvDirSsMgrUPAE5SuTfmEkJAVNCdLZAuCajNLJ4dYqHmtZ5D3rL1b96Fcpl}
\ifcmt
 author_begin
   author_id kafedra.kulturologii.mdu,demidko_olga.mariupol
 author_end
\fi

5 травня у рамках діяльності Гуманітарного штабу Маріупольського державного
університету відбулася ще одна екскурсія для маріупольців. Доцентка кафедри
культурології \href{\urlDemidkoIA}{Ольга Олександрівна Демідко} познайомила екскурсантів з найбільш знаковими
будівлями вулиці Ярославів Вал, і 🏰🏘🏫розповіла про  старовинну пам'ятку
оборонної архітектури Київської Русі, одну з найдавніших датованих споруд
Східної Європи, символ Києва - Золоті ворота.🤎 Також маріупольці дізналися,
які видатні письменники, вчені, театральні діячі і музиканти у різний час жили
на вулиці Ярославів Вал.👨🔬👨🎨👩⚕️ 

Найбільше  екскурсантів зацікавила  вражаюча історія будинку Родзянків та
унікальна архітектура караїмської кенаси.

🏯  \#РодинаМДУ  \#Україна \#Маріуполь \#Київ

\#futurestartswithyou  \#mdu\_cultural\_studies \#МДУ  \#нашівикладачі \#екскурсія
