% vim: keymap=russian-jcukenwin
%%beginhead 
 
%%file 03_12_2020.news.ua.gazeta.shmakov_valerii.1.franko_kirillica
%%parent 03_12_2020
 
%%url https://gazeta.ua/articles/culture/_perehid-ukrayinskoyi-movi-na-latinku-ce-civilizacijnij-vibir-oleksandr-frazefrazenko/998369
 
%%author Шмаков, Валерий
%%author_id shmakov_valerii
%%author_url 
 
%%tags 
%%title Перехід української мови на латинку - це цивілізаційний вибір - Олександр Фразе-Фразенко
 
%%endhead 
 
\subsection{Перехід української мови на латинку - це цивілізаційний вибір - Олександр Фразе-Фразенко}
\label{sec:03_12_2020.news.ua.gazeta.shmakov_valerii.1.franko_kirillica}
\Purl{https://gazeta.ua/articles/culture/_perehid-ukrayinskoyi-movi-na-latinku-ce-civilizacijnij-vibir-oleksandr-frazefrazenko/998369}
\ifcmt
	author_begin
   author_id shmakov_valerii
	author_end
\fi

\ifcmt
pic https://static2.gazeta.ua/img2/cache/gallery/998/998369_1_w_1200.jpg?v=0
caption Кирилиця - це штучно розроблений всесвіт - вважає Олександр Фразе-Фразенко 
\fi

Українська мова має перейти на латинку, щоб зробити цивілізаційний вибір між
Азією і Заходом на користь Заходу - переконаний кінорежисер і музикант
Олександр Фразе-Фразенко.

\enquote{Дивіться, Казахстан зараз переходить на латинку. Нурсултан Назарбаєв оголосив
десятилітній перехід на латинку. От нащо Казахстану це? А справа в тому, що
Казахстан – багата держава. Вони мають нафту і нормально розвинену культуру.
Тож у них є змога йти в ногу зі світом. Тому вони роблять цивілізаційний вибір
– з ким вони: з Азією чи із Заходом. Усе просто}.

\enquote{Хоча в наших реаліях це поки що здається непростим, однак такий крок має
змінити наше мислення}, - каже він.

\begin{leftbar}
	\bfseries
Свій \enquote{Вічний революціонер} Іван Франко писав латинкою
\end{leftbar}

\enquote{Ми б перестали думати, що росіяни – наші брати. Тому то вони нам кирилицю і привили, щоб ми думали, що ми частина їхнього світу. Кирилиця – це повністю штучно розроблений всесвіт. Це було пару сотень років тому, не так давно. 

А так ми писали латинкою. Свій \enquote{Вічний революціонер} Іван Франко писав латинкою. І він, до речі, був не ФранкО, а ФрАнко. 

Але совки хотіли його загнати в етнографію, і змінили на ФранкО. Це прізвище
латинського походження, а не кириличне. 

Франко – це система координат Європи. І Нагуєвичі – це не забите село, а нормальне галицьке містечко. Так само, як Турка, де Франц Йосиф любив відпочивати і грати в теніс. Чи Самбір, у якому 2 органних зали, бо одного замало було. А нам через штучний фольклор і гумор специфічний імперський навалювали, що ці всі речі – це периферія і примітивщина.

Радянські анекдоти – усі про нацменшини: про хохлів, про тупих чурок, про грузинів. Тобто висміювалася будь-яка інакшість від 
\enquote{советского чєловєка}. Не було анекдотів про радянську людину. Переважно хто смішний в анекдотах? Грузин, казах, хохол, естонець. Уся різноманітність просто вбивалася і висміювалася
}.

%\paragraph{Во брехун із брехунів! Рукописний оригінал Гімна Франка тута. Из Латиниці тілько назва DE PROFUNDIS\ldots}

\ifcmt
pic https://www.i-franko.name/files/IFranko/autographs/vol01-023.jpg
caption Рукописный вариант Гимна Ивана Франко - Вечный Революционер (1880)
\fi

\begin{leftbar}
	\bfseries
2 роки тому Міністр закордонних справ Павло Клімкін запропонував обговорити
ймовірне запровадження в Україні латиниці поряд із кириличною абеткою.
Про це він написав на своїй сторінці у мережі Facebook, вказавши, що до
такої ідеї його спонукала пропозиція історика й журналіста з Польщі
Зємовіта Щерека.

\enquote{Зємовіт Щерек запитав, чому б Україні не запровадити поряд з кирилицею
латиницю. Наша мета – формування української політичної нації, тому
маємо працювати на те, що нас об'єднує, а не роз'єднує. З іншого боку,
чому б не подискутувати?} – запропонував Клімкін.
	
\end{leftbar}
