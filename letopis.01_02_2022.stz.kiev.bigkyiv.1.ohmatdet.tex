% vim: keymap=russian-jcukenwin
%%beginhead 
 
%%file 01_02_2022.stz.kiev.bigkyiv.1.ohmatdet
%%parent 01_02_2022
 
%%url https://bigkyiv.com.ua/tysyachi-nejmovirnyh-istorij-povernennya-z-togo-svitu-ohmatdytu-sogodni-128-rokiv
 
%%author_id kiev.bigkyiv,lisichkіna_ljubava
%%date 
 
%%tags bolnica,deti,istoria,kiev,medicina,ohmatdet.bolnica
%%title Тисячі неймовірних історій повернення з того світу! Охматдиту сьогодні 128 років
 
%%endhead 
 
\subsection{Тисячі неймовірних історій повернення з того світу! Охматдиту сьогодні 128 років}
\label{sec:01_02_2022.stz.kiev.bigkyiv.1.ohmatdet}
 
\Purl{https://bigkyiv.com.ua/tysyachi-nejmovirnyh-istorij-povernennya-z-togo-svitu-ohmatdytu-sogodni-128-rokiv}
\ifcmt
 author_begin
   author_id kiev.bigkyiv,lisichkіna_ljubava
 author_end
\fi

\begin{zznagolos}
У різні роки й століття, в часи війни та різноманітних криз ця лікарня
трималась, розвивалась і жила завдяки лиш одному — людям. Відданим,
професійним, сміливим та чесним. Саме вони день за днем створювали Охматдит
таким, яким він є сьогодні.
\end{zznagolos}

Як зазначається, чимало хороших  моментів було в  історії лікарні.
 
\textbf{Як все починалося}
 
1 лютого 1894 відкрила свої двері Київська безоплатна Цесаревича Миколая
лікарня для чорноробів. Збудував її відомий український підприємець та меценат
Нікола Терещенко. Сьогодні на території Охматдиту стоїть пам’ятник своєму
засновнику.

\ii{01_02_2022.stz.kiev.bigkyiv.1.ohmatdet.pic.1}

Від 1927 року лікарня була клінічною базою для Інституту охорони материнства та
дитинства. А у 1957 їй надали статус спеціалізованої дитячої лікарні.
Керівництво новоствореного закладу взяла на себе Тетяна Петрівна Новікова. Вона
зробила усе, щоб створити спеціалізовану лікарню для дітей. Вже за рік після
нового статусу відкрито відділення виходжування недоношених дітей.
 
У 1947 році до закладу приєднали поліклініку №11. Нині
консультативно-діагностична поліклініка — це наш передовий підрозділ: вони
консультують пацієнтів, скеровують їх на лікування у стаціонарі та роками
ведуть пацієнтів до здорового життя. Щороку у поліклініці фіксують понад 300
000 відвідувань!

\ii{01_02_2022.stz.kiev.bigkyiv.1.ohmatdet.pic.2}
\ii{01_02_2022.stz.kiev.bigkyiv.1.ohmatdet.pic.3}

У 1975 році відкрито хірургічний та неонатологічний корпуси. Фінансували їх
будівництво за рішенням міської влади з коштів киян, зароблених на громадських
суботниках.
 
\textbf{Перші в неонатології та поява служби дитячої гематології}
 
Почала роботу в 1977 році кафедра неонатології. Вона була першою в Україні та
другою в Радянському Союзі.
 
З 1989 року вперше в Києві в дитячій практиці роблять ультразвукову діагностику
мозку у новонароджених, а також УЗД легенів, плевральної порожнини, шлунку,
вилочкової залози, м‘яких тканин та кістково-суглобної системи.

\ii{01_02_2022.stz.kiev.bigkyiv.1.ohmatdet.pic.4}

Професор Бебешко Володимир Григорович з Центру дитячої онкогематології та
трансплантації кісткового мозку НДСЛ «Охматдит» був першим Головним дитячим
гематологом МОЗ України та засновником служби дитячої гематології.
 
3 липня 1996 року Указом Президента України створено українську дитячу
спеціалізовану лікарню «Охматдит» — тепер вже національного значення.

\ii{01_02_2022.stz.kiev.bigkyiv.1.ohmatdet.pic.5}

«Щоб розповісти про всі наші досягнення, довелося б писати книгу в кількох
томах. Та у нашої команди нема на це часу. Бо кожну мить ми віддаємо, щоб
рятувати життя та здоров’я маленьких українців», – зазначають в  Охматдиті.
 
Сьогодні час подякувати кожному, хто працював та працює на благо лікарні. Разом
з вами та завдяки вам ми робимо справжні дива, які неможливо переказати у
кількох реченнях  З Днем народження, Охматдит!
 
До слова, у Святвечір в Охматдиті \href{https://bigkyiv.com.ua/u-svyatvechir-v-ohmatdyti-vryatuvaly-2-richnu-dytynu-zrobyvshy-transplantacziyu-kistkovogo-mozku}{врятували 2-річну дитину}, зробивши
трансплантацію кісткового мозку.
