% vim: keymap=russian-jcukenwin
%%beginhead 
 
%%file 25_09_2021.stz.news.lnr.lug_info.1.shokolad
%%parent 25_09_2021
 
%%url https://lug-info.com/comments/luganskij-konditer-shokolat-e-kristina-samonina-shokolad-neset-neveroyatnoe-umirotvorenie-pri-rabote-s-nim
 
%%author_id news.lnr.lug_info
%%date 
 
%%tags donbass,isskustvo,konditer,lnr,lugansk,shokolad
%%title Луганский кондитер-шоколатье Кристина Самонина: "Шоколад несет невероятное умиротворение при работе с ним"
 
%%endhead 
 
\subsection{Луганский кондитер-шоколатье Кристина Самонина: \enquote{Шоколад несет невероятное умиротворение при работе с ним}}
\label{sec:25_09_2021.stz.news.lnr.lug_info.1.shokolad}
 
\Purl{https://lug-info.com/comments/luganskij-konditer-shokolat-e-kristina-samonina-shokolad-neset-neveroyatnoe-umirotvorenie-pri-rabote-s-nim}
\ifcmt
 author_begin
   author_id news.lnr.lug_info
 author_end
\fi

\begin{zznagolos}
Луганский кондитер-шоколатье, основатель бренда Macao Chocolate Кристина
Самонина рассказала корреспонденту ЛИЦ об особенностях своей профессии и
процессе создания шоколадных изделий. 
\end{zznagolos}

ПЕРВЫЙ ТОРТ

В кондитерском деле я недавно, с лета 2020 года. Тогда я испекла свой первый
торт на годик сыну, я была в восторге от того, какой он вкусный получился, так
как до этого я никогда и ничего не пекла, да и вообще не любила готовить.

\ii{25_09_2021.stz.news.lnr.lug_info.1.shokolad.pic.1}

Я решила идти в этом направлении профессионально и получить высшее образование.
Осенью (2020 года) поступила в (Луганский государственный) педагогический
университет на специальность \enquote{Технология и организация общественного питания}.
Начала печь торты для родных и близких, проходила мастер-классы онлайн по
выпечке, кремам, начинкам.

СЛУЧАЙНЫЙ МАСТЕР-КЛАСС

Однажды мне попался мастер-класс \enquote{Шоколадный декор}. С этого момента все и
началось: конфеты, плитки, медианты (французская сладость, состоящая из
темного, молочного или белого шоколада), шоколадные пасты, роксы (десерт из
шоколада и хрустящих компонентов). Мне нравилось все, что связано с шоколадом.

\ii{25_09_2021.stz.news.lnr.lug_info.1.shokolad.pic.2}

Конфеты – это моя страсть, мой холст, где я могу творить, создавать что-то
новое, уникальное. Все начинается с идеи вкуса, какой бы мне хотелось именно
сегодня, я создаю дизайн, начинку. Я работаю только на своих рецептах, всегда
ищу новые сочетания вкусов.

\ii{25_09_2021.stz.news.lnr.lug_info.1.shokolad.pic.3}

РОЖДЕНИЕ БРЕНДА

Одной из составляющих шоколада является тертое какао и какао-масло. Я решила
заменить заглавную букву в слове \enquote{какао} на заглавную букву имени своего сына –
Матвея, ведь именно с его появлением все и началось. Так родился бренд Macao
Chocolate и моя деятельность как шоколатье.

\ii{25_09_2021.stz.news.lnr.lug_info.1.shokolad.pic.4}

КТО ТАКОЙ ШОКОЛАТЬЕ?

Шоколатье – это кондитер, специализирующийся на производстве шоколада и изделий
из него. Профессию шоколатье многие считают исключительно творческой. В
действительности же она в значительной степени техническая. Рецептурные и
дизайнерские особенности изделий подчиняются свойствам используемых продуктов.
Чтобы создать оригинальный сладкий десерт шоколатье должен знать о пластических
свойствах шоколада и начинок, о скорости застывания того или иного продукта, о
множестве других нюансов. Без технических знаний воплотить творческие задумки в
жизнь невозможно. Именно поэтому я 2-3 раза в год прохожу обучение у опытных
шоколатье с мировым именем.

СОЗДАНИЕ КОНФЕТ

Что касается времени приготовления: изготовление, например, корпусных конфет
занимает около суток. До начала я созданию дизайн какао-маслом, которое нужно
предварительно темперировать (процесс плавления и повторной кристаллизации
какао-масла в шоколад), даю стабилизироваться. Далее заливаю корпус
темперированным шоколадом. Следом создается начинка, на ее стабилизацию уходит
от 10 до 16 часов. Обычно я готовлю вечером, на следующий день закрываю корпус
и конфеты готовы.

Самое главное в работе шоколатье – это качественные продукты в работе: шоколад,
какао-масло, ваниль. Необходимо иметь формы для изделий: плиток, конфет, фигур
из шоколада. Формы нужно брать только из хорошего поликарбоната: Chocolate
World (Бельгия), Pavoni (Италия), Martellato (Италия), Cacao Barry (Франция).

ШОКОЛАД = ЛЮБОВЬ

Чтобы начать заниматься шоколадом, его нужно полюбить неистово и страстно,
тогда у вас все получится. Шоколад несет невероятное умиротворение при работе с
ним. В нашем городе профессия шоколатье абсолютно не развита. Если вы захотите
ее освоить, придется выезжать как минимум в Российскую Федерацию, чтобы
получить базовые знания по работе с шоколадом. Если есть возможность – во
Францию, Испанию, Италию, ведь именно там преподают мировые шефы: Рамон Морато,
Доминик Персоне, Амори Гишон.

