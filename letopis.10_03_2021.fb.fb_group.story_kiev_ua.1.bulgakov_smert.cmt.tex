% vim: keymap=russian-jcukenwin
%%beginhead 
 
%%file 10_03_2021.fb.fb_group.story_kiev_ua.1.bulgakov_smert.cmt
%%parent 10_03_2021.fb.fb_group.story_kiev_ua.1.bulgakov_smert
 
%%url 
 
%%author_id 
%%date 
 
%%tags 
%%title 
 
%%endhead 
\subsubsection{Коментарі}

\begin{itemize} % {
\iusr{Елена Никитчук}
Браво, за жизнь, за любовь браво за поэзию! Мастер БРАВО! ПОМНИМ И ЛЮБИМ!

\ifcmt
  ig https://i2.paste.pics/10fc8a934f8d1e40b5bdd90db0c5401f.png
  @width 0.2
\fi

\iusr{Виктор Михайлюк}
Нет, нельзя его читать, он же на русском языке писал!...

\begin{itemize} % {
\iusr{Володимир Володимир}
\textbf{Виктор Михайлюк} ваш сарказм не доречний і межує з українофобством.,

\begin{itemize} % {
\iusr{Виктор Михайлюк}
\textbf{Володимир Володимир} Добирайте слова, шановний!!! Я зовсiм не шуткую, Я вже пiднiмаю питания про знесення памятника Пушкiну у Киiвi!!!

\iusr{Неля Архипова}
\textbf{Виктор Михайлюк} Но мы- то с вами точно знаем, что шутите, я сама так люблю, особенно в подобных случаях.

\iusr{Alex Aleksandrov}
\textbf{Виктор Михайлюк} ты тяжелее рюмки водки сроду ничего не подымал

\iusr{Alex Aleksandrov}
\textbf{Valentyna Venglovska} ваши пролетарии как власть захватили первым делом памятники и начали сносить а вторым неугодных расстреливать, не понимаю к чему сейчас ваше нытье

\iusr{Ирина Сологуб}
\textbf{Виктор Михайлюк} ну вообще с ума сошел
\end{itemize} % }

\iusr{Елена Семененко}
\textbf{Виктор Михайлюк} а пусть переведут
А Ви знаете что он бил врачом и тогда существовала методика борьби с дифтиритом при помощи трубочек через которие
Витягивали дифтирийние пленки
Бабушка говорила если в селе заболевал ребенок умирали все
До 10 лет
Она била сельской учительницей
Я вам скажу что если би он етого не делал то и памятника и доброй памяти на Украине о нем би не било

\begin{itemize} % {
\iusr{Люба Микицька}
\textbf{Елена Семененко} ОН в КИЕВЕ бьіл венерологом. ЕГо соседи так и запомнили как \enquote{Миша венеролог.}
\end{itemize} % }

\iusr{Володимир Володимир}
\textbf{Виктор Михайлюк} ви, ще про пам'ятник М. Магомаєву питання підніміть...

\iusr{Ольга Гуменюк}
\textbf{Володимир Володимир}
Ну Магомаев же не москвич, он Заслуженный Азербайджанской Республики. Хотя, нашим нацикам, все, кто не Бандера - враг.

\iusr{Олена Мельникова}
Неужели не шутите?

\iusr{Володимир Володимир}
\textbf{Olga Gumeniyk} не несіть дурню!

\iusr{Люба Микицька}
\textbf{Володимир Володимир} чому? АДМИНЬІ одобряют подобньіе вещи.

\iusr{Володимир Володимир}
\textbf{Lyuba Mukutska} нажаль...(((

\iusr{Руслан Трушков}

Господи, сколько же у нас идиотов... одни с памятниками воюют, другим Булгаков
как кость в горле... главное, чтобы своей же блевотой не удавились... или, не
загнили, как те ананасы из рассказа Коровьева...

\end{itemize} % }

\iusr{Маргарита Мышанская}
Думаю, он реально вечен! Любил повторять: \enquote{Главное-не терять
достоинства}! @igg{fbicon.face.smiling.eyes.smiling} 

\begin{itemize} % {
\iusr{Slava Kogan}
\textbf{Маргарита Мышанская} но пьесу в честь Сталина написал \enquote{Батум}

\iusr{Катерина Мазур}
\textbf{Маргарита Мышанская} но при этом, как русский шовенист мог унижать Украину.

\begin{itemize} % {
\iusr{Юрий Блохин}
\textbf{Катерина Мазур} 

Я думаю, что у Вас было бы другое мнение, если бы Вы были коренной киевлянкой,
родившейся в семье интеллигентов .... в конце 19 ст. А делать вывод о том что
было более 100 лет назад с сегодняшнего сумасшедшего времени, по -меньшей мере
неправильно.

\iusr{Неля Архипова}
\textbf{Катерина Мазур} 

шовинист, если позволите. Он лишь родился в Киеве и безмерно и навсегда полюбил
свой Город и, кстати, прославил его на века, чем сделал для Украины больше, чем
все лжепатриоты и националисты вместе взятые. Тем более, что они в основном
только позорили страну своей ненавистью и ограниченностью. А был он, конечно,
русским, как бы это не было вам противно.

\iusr{Катерина Мазур}
\textbf{Неля Архипова} 

он полюбил этот город, как все русские любят, только без украинцев, считая их
недоравитым плебсом. Умными, достойными могут быть только великороссы, ну еще
русскоговорящие украинцы на вторых ролях, которые забыли свои корни. А все
остальные украинцы это рабы недоразвитые.

\iusr{Alex Aleksandrov}
\textbf{Юрий Блохин} 

те \enquote{коренные киевляне} с московитскими корнями имели очень короткую память и
очень быстро забыли как здесь появились. В 18м году Петлюра немного поправил
ситуацию но им на смену быстро пришла новая волна \enquote{коренных киевлян} с красными
тряпками на перевес. Причём начали они свое киевлянство с тредневного обстрела
города и последующей недельной резни имени муравьёва. Сейчас сново выровняли
погнав их вместе с овощем домой в мск

\iusr{Юрий Панчук}
\textbf{Юрий Блохин} Совершенно согласен, человек другой эпохи, выросший в русской семье, глупо его винить. Родиться в такой семье как Леся Украинка скорее исключение. Родись он в другой семье - возможно не стать бы ему Мастером

\iusr{Alexandr Vlasenko}
\textbf{Катерина Мазур} Это Вы серьезно или иронизируете ? )Я не очень-то рефлексирую. Но Киев без украинцев - это как ? Как пасека без пчел ? @igg{fbicon.face.smiling.eyes.smiling} 

\iusr{Ольга Гринь}
\textbf{Катерина Мазур} 

Госпожа, это не Вы-ли потеряли переписку Энгельса с Каутским? Бегите, пока в
печке не сожгли @igg{fbicon.face.savoring.food}  Вы элементарной грамматике не обучены, не стыдитесь писать
такую чушь беспросветную? @igg{fbicon.face.grinning.squinting}  кто такой шовенист, пилять?!?! И это в присутствии
людей с университетским образованием!


\iusr{Alex Aleksandrov}
\textbf{Ольга Гринь} а вы опровергните сказанное) без козыряний не пойми каким образованием и общих оптекаемых фраз))

\iusr{Ольга Гринь}
\textbf{Alex Aleksandrov} а зачем? Мы ужинаем на дне рождения моего обожаемого папочки в компании любимых, мне не до Вас @igg{fbicon.face.blowing.kiss}  Вы и дальше обижайтесь на весь мир, это конструктив @igg{fbicon.face.savoring.food} 
\end{itemize} % }

\iusr{Маргарита Мышанская}

Знаете, что я думаю.... Да, М. Булгаков -русский монархист. В его биографии
написано, что он собирался, пребывая на Кавказе, эмигрировать с белогвардейцами
за границу, но тиф помешал. Я его понимаю.... Он не жил, а мучился при
советах. Скажите, пожалуйста, как ему было продвигать свое творчество? Он великий
писатель, писатель, не изменивший своему духу

\begin{itemize} % {
\iusr{Alex Aleksandrov}
\textbf{Маргарита Мышанская} внимательно перечитайте \enquote{хождение по мукам} потом нам расскажете чему он там не изменил)

\iusr{Igor Popell}
\textbf{Alex Aleksandrov} \enquote{войну и мир} еще можно перечитать внимательно.

\iusr{Alex Aleksandrov}
\textbf{Igor Popell} перечитайте я не препятствую)
\end{itemize} % }

\iusr{Михаил Фролов}
\textbf{Slava Kogan} что значит \enquote{пьесу в честь Сталина} написал? Когда и где состоялась её премьера? Или премьеры вообще не было?
Расскажите историю создания пьесы?

\begin{itemize} % {
\iusr{Slava Kogan}
\textbf{Michael Frolov} зачем расказывать, для этого Google есть. Не поленитесь и найдите инфо сами. Хотя я guy отнюдь не осуждаю. Время такое было.
\end{itemize} % }

\iusr{Маргарита Мышанская}

Если бы да кабы... Он родился так как родился, как каждый из нас. Я русская по
национальности и не могу мыслить как китаец, японец и даже поляк. И напишу о
любом событии так, как мне дано моими, национальными пониманиями.

\begin{itemize} % {
\iusr{Alex Aleksandrov}
\textbf{Маргарита Мышанская} точно, у вас же руских скрепы и лишняя хоромосома, ещё в 14м году выявили и научно обосновали) потому и не можете мыслить как китаец или поляк))
\end{itemize} % }

\iusr{Маргарита Мышанская}
А вы, которые без скрепов и лишней хромосомы, можете мыслить как китаец или поляк? @igg{fbicon.laugh.rolling.floor} 

\iusr{Михаил Фролов}
\textbf{Алекс Александров} а зачем умному человеку опровергать сказанную кем-то глупость?

\iusr{Igor Popell}
то ли про аптеку, то ли про оптику?

\ifcmt
  ig https://scontent-frt3-1.xx.fbcdn.net/v/t1.6435-9/159245382_10159030637246136_355845014517356467_n.jpg?_nc_cat=102&ccb=1-5&_nc_sid=dbeb18&_nc_ohc=_fFHrnJUvwUAX9hqi1W&_nc_ht=scontent-frt3-1.xx&oh=00_AT_VQYrF0EvxxTsfOiOWtUM1wUN_tSAwiaIcnqOsC25iyw&oe=61E10805
  @width 0.4
\fi

\iusr{Михаил Фролов}
\textbf{Slava Kogan} 

я это знаю, когда ещё гугла не было. И какая же это пьеса \enquote{в честь Сталина}? Вы
сначала написали, а на уточняющие вопросы не отвечаете. Булгаков написал пьесу
по своей инициативе? Или инициатором был кто-то другой?

\begin{itemize} % {
\iusr{Slava Kogan}
\textbf{Michael Frolov} это пьеса о батумской о демонстрации 1902 года и начала рев.деятедьности некоего Джугашвили. Булгаков написал её ,когда его пьесы снимали с репертуара и он был в опале.
\end{itemize} % }

\iusr{Людмила Губианури}

Посмотрите, какие интересные материалы мы опубликовали на стр. нашего музея к
этому дню. Bugakov museum in Kyiv


\iusr{Михаил Фролов}
\textbf{Slava Kogan} 

действительно, пьеса об этом и не только об этом. Но написал пьесу Булгаков
совершенно по другому поводу, а не потому что его пьесы снимали с репертуара.
Вы в курсе истории написания пьесы?

\begin{itemize} % {
\iusr{Slava Kogan}
\textbf{Michael Frolov} насколько я знаю из разных источников, он начал писать пьесу в 1929 году, как раз ,когда в прессе началась травля самого Булгакова и его работ. Может в пьесе и есть какие-то ещё сюжетные линии, но в основе молодой Сталин. Кстати пьесу запретили всё равно к постановке
\end{itemize} % }

\iusr{Михаил Фролов}
\textbf{Slava Kogan} 

пьесу он начал писать и написал в 1939 году. Это год 60-летия Сталина. Тогда
многие театры хотели поставить пьесу, приуроченные к этому событию. Руководство
МХАТа тоже решило поставить пьесу. Зная, что в стране нет драматурга лучше, чем
Михаил Афанасьевич, и что он напишет лучшую пьесу, МХАТовцы обратились к нему.
Михаил Афанасьевич понимал, что если пьеса \enquote{наверху} не понравится, это может
иметь для него негативные последствия. И он сказал, что напишет пьесу о молодом
Сталине (уход из Духовной семинарии, организация Батумской стачки, ссылка в
Сибирь и т.д.). Руководство театра согласилось. Была проведена генеральная
репетиция, которая прошла успешно. Главную роль исполнял Николай Хмелев. После
генеральной репетиции вся труппа выехала на поезде в Батуми в место юношеской
деятельности героя пьесы. На станции Серпухов Булгакову в поезде вручили
телеграмму : надобность в поездке отпала.  Булгаков с супругой вернулись в
Москву, впрочем как и вся труппа.

\begin{itemize} % {
\iusr{Slava Kogan}
\textbf{Michael Frolov} ну так и я о том.
\end{itemize} % }

\iusr{Михаил Фролов}
\textbf{Slava Kogan} 

не в 1929 он начал писать пьесу. А в 1939. Не по своей инициативе, а по просьбе
руководства лучшего в СССР Драматического театра.

\begin{itemize} % {
\iusr{Slava Kogan}
\textbf{Michael Frolov} я читал, что задумал он пьесу в 29, писать начал в 36 и работа шла медленно и закончил в 39. Ну а по просьбе или нет ,получилось то что получилось. Не думаю, что он этим сильно гордился. И мы его почитаем и любим не of эту пьесу.
\end{itemize} % }

\iusr{Михаил Фролов}
\textbf{Slava Kogan} 

\enquote{получилось, то что получилось}. И где же в пьесе восхваление Сталина? Вы
пьесу, вообще-то читали? Почему же её запретили? Объяснения, которые дошли до
Булгакова якобы \enquote{с самого верха} лично мне трудно назвать убедительными. У меня
своя версия причины запрета пьесы. И, уверен, эта пьеса должна быть зачтена в
творчество Булгакова со знаком \enquote{+}. Здесь нет никакого восхваления, лозунгов,
клятв в верности. Это о юноше, выбравшем свой путь в жизни.

\end{itemize} % }

\iusr{Ольга Яворовская}


\iusr{Vitaliy Tretiakov}
Талант

\iusr{Ludmila Debourdeau}
Один из моих любимых писателей @igg{fbicon.heart.red}

\iusr{Ludmila Teslenko-ponomarenko}
Да, рукописи не горят, а самые любимые всегда с нами!

\end{itemize} % }
