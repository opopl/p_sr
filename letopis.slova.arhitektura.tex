% vim: keymap=russian-jcukenwin
%%beginhead 
 
%%file slova.arhitektura
%%parent slova
 
%%url 
 
%%author 
%%author_id 
%%author_url 
 
%%tags 
%%title 
 
%%endhead 
\chapter{Архитектура}

%%%cit
%%%cit_head
%%%cit_pic
%%%cit_text
Андрій Меленський – той, хто зробив Київ цілісним.  Роки життя: 1766-1833.
Стиль: ампір.
Нажаль сьогодні про \emph{архітектора} Андрія Меленського мало хто знає, хоч ми і
послуговуємося його творчою спадщиною. Справа в тому, що саме Андрій Іванович
зробив Київ таким, яким сьогодні ми його знаємо і любимо.
У 1799 році Меленський приїздить до столиці України та обіймає новостворену
посаду \emph{головного архітектора}. В цей час Київ нагадує не стільки цілісне місто,
скільки набір шматків землі, розділених рівчаками. Ні про який цілісний
\emph{архітектурний} ансамбль тут і мови немає
%%%cit_comment
%%%cit_title
  <++>
%%%endcit
