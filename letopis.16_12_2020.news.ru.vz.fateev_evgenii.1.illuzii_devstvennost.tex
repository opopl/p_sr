% vim: keymap=russian-jcukenwin
%%beginhead 
 
%%file 16_12_2020.news.ru.vz.fateev_evgenii.1.illuzii_devstvennost
%%parent 16_12_2020
 
%%url https://m.vz.ru/opinions/2020/12/16/1075853.html
 
%%author Фатеев, Евгений
%%author_id fateev_evgenii
%%author_url 
 
%%tags mysli,russia
%%title Многие из нас лишились иллюзий раньше, чем девственности
 
%%endhead 
 
\subsection{Многие из нас лишились иллюзий раньше, чем девственности}
\label{sec:16_12_2020.news.ru.vz.fateev_evgenii.1.illuzii_devstvennost}
\Purl{https://m.vz.ru/opinions/2020/12/16/1075853.html}
\ifcmt
	author_begin
   author_id fateev_evgenii
	author_end
\fi

\textbf{\em\large Обязательно должно сложиться какое-то новое ополчение из еще живых «жертв»
советского гуманитарного образования. Мы еще остались. Мы еще в форме. Мы еще в
себе. Нам необходимо на время войны превратиться в рыцарей здравого смысла. Мы
еще имеем шанс вернуть управление вечно ускользающим будущим.}

\ii{pic.author.fateev_evgenii}

\ifcmt
pic https://img.vz.ru/upimg/soc/soc_1075853.jpg
\fi

Вокруг нас очень много суеты. Причем вся она мимо. Вроде бы есть чего бояться,
но страх у массовой аудитории вызывает совершенно нестрашное. Или так: страх
вызывает то, что недостойно нашего страха, не заслужило чести быть страшным, не
заслужило нашего усилия быть испуганными. Многое требует нашей заботы и нашего
внимания, но мы обрушиваем свою любовь и опеку на тех, кто в них не нуждается.

Мы понимаем, что с миром, нас окружающим, что-то не так, но причинами назначаем
абсолютно постороннее, а, следовательно, прописываем негодные средства
исцеления. Мы вроде бы вполне справедливо осознаем, что государству требуется
критика. Жесткая, нелицеприятная. Но критикуем мы наше государство совершенно
не по делу. Бьем не за то. И мы, и оно – постоянно обоюдно раздражаемся.
Какое-то плохо скрываемое раздражение проистекает от странной
некоммуникабельности, параллелизма сосуществований, какого-то фатального
непересечения. 

Вообще, любой, кто хоть чем-то в своей жизни поуправлял, поймет меня. И его
претензии к государству будут совсем другого свойства. И как бы нам избавиться
от всяческих водевильных таких объяснений самым разным и очень сложным
процессам? Наверное, это невозможно. Но, может, нам, которые, как я льщу себе,
неводевильные, как-то погромче говорить, как-то обозначить свое присутствие в
медийном поле?

Нам, дядькам, которые к своим 40+ подошли с каким-то опытом, грузом прекрасных
и не очень ошибок, надо как-то погромче высказываться. Может, это пафосно
прозвучит, но нам, может, стоит наконец ощутить поколенческую ответственность?
Ведь мы еще по-советски, по-настоящему по-советски, неплохо образованы. Мы люди
цивилизационного шва: на наших глазах рухнула империя, потом на нас, как на
кроликах, постсоветские элиты учились управлять и очень дорого избавлялись от
собственных иллюзий. Многие из нас состоялись, а если не состоялись, то многое
поняли. И про себя, и про страну. Поняли, скорее всего, что-то свое, но это как
раз и интересно. Нам как-то надо этим опытом поделиться. Желательно,
неангажированно. Или осознанно ангажированно.

Меня очень раздражают все эти бесконечные камлания на тему «свободы». Иногда
кажется, что люди, поющие осанну этому слову, не совсем честны и что-то
недоговаривают. Огромные стада яйцеголовых осмартфоненных поклоняются ему.
Этому слову. Именно слову. Любая критика, любая попытка приглашения к дискуссии
на тему «а что такое свобода» маркируется, как приверженность «рабству» и
вообще.

Сегодня «свобода» – это сложно устроенный идейный дериватив, то есть
обязательство по отношению к базовому активу. Причем посложнее финансовых
деривативов. Его создатели откровенно мухлюют, передергивают и манипулируют.
Этот дериватив ничем не обеспечен. Его необходимо разобрать, разложить,
размотать… Философы давно этим занимаются. Но кто их сегодня читает? 

Вообще, многое в нашей жизни правильно описывать в терминах фондового рынка.
Сегодняшний человек уже обрел свою личную социальную капитализацию. Компании же
и целые страны нагуливают подчас странную капитализацию, которая буквально
парит в воздухе. Причем совершенно непонятно сегодня, по каким законам
осуществляется сборка, накачивание всех этих разновидностей капитализации.
Скорее, все это напоминает запутанностью своих «алгоритмов» казино. Меня не
оставляет ощущение того, что мы играем в игру, в которой совершенно невозможно
обыграть казино, ведомое какими-то неведомыми хозяевами.

Мутный и бесстыдный брендинг вытеснил старую добрую репутацию. Совершенно не
стало качественных вещей. Возникла какая-то бесконечная законническая тяжба.
Сегодня от многого плохого нас хранит хрупкий и подвижный закон. От сущего ада
нас теперь охраняет только писаное право и некоторое количество людей в форме,
которые все еще пока готовы защищать этот закон. А всего остального уже нет.
Нет Бога, нет чести, нет совести… Нет всего разного, что гарантировало нам
что-то главное. Гарантировало нам человечность. Нам вдруг стало возможным
представить себе возможность не быть человеком.

Нам, сегодняшним 40+, нужно постоянно говорить. Таков наш долг. Причем я
понимаю, что многим из нас кажется, что мы говорим в никуда, говорим с самими
собой. Все равно нам необходимо говорить. Мне кажется, самое ценное в нашем
поколении – мы помним человеческий мир, человеческую цивилизацию на самом ее
пике, а пиком ее безусловно была холодная война. Мы еще помним, что такое
хорошее кино. Нам могут сколько угодно говорить, каким великим является
несомненно хороший, но и только, фильм «Джокер», но мы видели скорсезовского
«Таксиста». И мы помним те времена, в которые еще сочинялись оригинальные или
около того мелодии. Мы помним времена, когда еще читали книги, и было
совершенно безопасно иметь собственное мнение.

Из наших голов уже выветрилась вся эта русско-интеллигентская дурь. Нам выпало
сомнительное счастье наблюдать сбычу мечт русской интеллигенции. Всю эту
продажную кодлу однажды лет на десять подпустили к барскому столу, с которого
некоторым упало несколько крошек и обглоданных уже костей. И на что же они
оказались способны? Лучше не вспоминать, хотя развидеть уже невозможно…

Многие из нас очень рано лишились иллюзий. Подчас раньше, чем девственности.
Нас кинула страна очень рано, когда еще не истек наш комсомольский срок. Нам
удалось выжить в 90-е годы, но самое сложное и ценное – многие, очень многие из
нас, нашли в себе душевные силы на то, чтобы еще раз поверить своей собственной
стране. Поверить не без опаски, не без страха. Лично мне выдать такой кредит
доверия своей ветреной Родине было очень непросто. Если Родина нас не подведет,
мы сможем научить этому других. 

Не без нескромности нужно признать, что мы будем в XXI веке последними из
могикан. Мы будем последними доцифровыми людьми ручной выделки. И этот наш
доцифровой опыт сегодня изо всех сил пытаются обесценить и дискредитировать.

Нашу критику, нашу экспертизу, стараются представить в качестве старческого
брюзжания. Чем дальше, тем все больше мы будем утопать со своими словесами в
эйджистском болоте «отцов и детей». Нам придется постоянно доказывать, что мы
не про «в наше время трава была зеленее, а солнце светило ярче», а про что-то
другое.

На наших глазах рождался постмодернистский треш и глум. На наших глазах эти
треш и глум стремительно уже подыхают, зато рождается новый пафос, а мы знаем
толк в пафосе. Мы знаем, каков он на слух, мы пробовали, каков он на вкус.
Самое, пожалуй, главное – мы обладаем бесценной способностью к транзистенции,
постоянству перехода из одного состояния в другое. Нашему поколению так и не
удалось пожить толком в спокойные и мирные времена. Даже путинское
благоденствие – лишь немного затянувшийся и растянувшийся миг для нас. Но все
равно миг. Все равно в нас лежит глубокое, глубинное, уже непобедимое
подспудное недоверие ко всему надежному, постоянному, хорошему. Мы видели, как
быстро все это может разрушиться.

Мы уже не задаем вопросы «неужели так можно было?». Мы уже знаем, что можно уже
все, или почти все, или совсем скоро можно будет все. Боже, вы, молодые, даже
не представляете себе, насколько многое с вами делать уже можно! Почти все!
Вопрос совсем скорого времени, когда это случится. Да какое там! Это грозное
«все» случается уже сейчас.

Пожалуй, самым грустным для нашего поколения будет помнить о тех мечтах, о том,
каким мир мог бы быть. Мы видели в лицо возможный, но так и не случившийся мир.
И нам будет особенно жалко молодых, которые уже не знали другой жизни. Нам не
нужно жадничать. Нам просто необходимо поделиться. Хотя бы на словах и в
словах. Быть может кто-то это все-таки когда-нибудь прочитает.

Мне кажется, настали времена каких-то новых народных ополчений – с клавиатурой
наперевес. И обязательно должно сложиться какое-то новое ополчение из еще живых
«жертв» советского среднего и высшего гуманитарного образования. Мы еще
остались. Мы еще в форме. Мы еще в себе. Нам необходимо на время войны
превратиться в рыцарей здравого смысла. Мы еще имеем шанс вернуть управление
вечно ускользающим будущим. 
