% vim: keymap=russian-jcukenwin
%%beginhead 
 
%%file 28_10_2021.fb.zaborin_dmitrij.1.den_osvobozhdenia_ukrainy
%%parent 28_10_2021
 
%%url https://www.facebook.com/d.zaborin/posts/4742406842446608
 
%%author_id zaborin_dmitrij
%%date 
 
%%tags 1944,1944.ukraina.osvobozhdenie.vov.28_oct,28_oct,istoria,nacizm,pozdravlenie,prazdnik,sssr,ukraina,vov
%%title День освобождения Украины от фашистских захватчиков
 
%%endhead 
 
\subsection{День освобождения Украины от фашистских захватчиков}
\label{sec:28_10_2021.fb.zaborin_dmitrij.1.den_osvobozhdenia_ukrainy}
 
\Purl{https://www.facebook.com/d.zaborin/posts/4742406842446608}
\ifcmt
 author_begin
   author_id zaborin_dmitrij
 author_end
\fi

Если сложить вместе поздравления первых лиц и методички Института нацпамяти, то
получается, что 28 октября 1944 г. наши соотечественники в составе войск
антигитлеровской коалиции в ходе немецко-советской войны завершили изгнание
нацистов с территории Украины, чем и завершили ее повторную оккупацию. При этом
неназванные оккупанты "стали примером для нового поколения украинских
защитников и защитниц, которые обороняют Украину на востоке". 

\ifcmt
  pic https://scontent-lhr8-2.xx.fbcdn.net/v/t1.6435-9/250311181_4742406052446687_6726230728274958268_n.jpg?_nc_cat=104&ccb=1-5&_nc_sid=730e14&_nc_ohc=IE5AJmAcAuwAX9O4K-E&_nc_ht=scontent-lhr8-2.xx&oh=a142d66b4cc5db5137598c1355ae0589&oe=619F4C92
  @width 0.8
\fi

Так что у всех, кто не хочет тоже сойти с ума, пробираясь наощупь в этом
безобразном мире, День освобождения Украины от фашистских захватчиков
по-прежнему случился в ходе Великой Отечественной войны усилиями 1-го и 4-го
Украинских фронтов РККА.

С праздником!

Автор фото: Аркадий Шайхет. 1944 г., окрестности Львова.

\ii{28_10_2021.fb.zaborin_dmitrij.1.den_osvobozhdenia_ukrainy.cmt}
