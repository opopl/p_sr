% vim: keymap=russian-jcukenwin
%%beginhead

%%file man.opts_args.definitions
%%parent man.opts_args

%%endhead
\subsection{Definitions of options and arguments}

\vspace{0.5cm}
 {\ifDEBUG\small\LaTeX~section: \verb|man.opts_args.definitions| project: \verb|latexmk| rootid: \verb|p_saintrussia| \fi}
\vspace{0.5cm}

\subsubsection{file}

One or more files can be specified.  If no files are  specified, latexmk  will,
by default, run on all files in the current working directory with a ".tex"
extension.   This  behavior  can  be changed: see the description concerning
the \verb|@default_files| variable in the section "List of configuration
variables  usable  in initialization files".

If a file is specified without an extension, then the ".tex" extension is
automatically added, just as LaTeX  does.   Thus,  if you specify:

\begin{verbatim}
  latexmk foo
\end{verbatim}

then latexmk will operate on the file "foo.tex".

There  are  certain  restrictions on what characters can be in a filename;
certain characters are either prohibited  or  problematic  for  the  latex
etc programs.  These characters are: \verb|"$"|, \verb|"%"|, \verb|"\"|,
\verb|"~"|, the double quote character, and the control characters  null,
tab,  form  feed, carriage return, line feed, and delete.  In addition
\verb|"&"| is prohibited  when  it  is  the  first character of a filename.

Latexmk  gives  a  fatal  error when it detects any of the above characters in
the TeX filename(s) specified on the command line.  However  before  testing
for illegal characters, latexmk removes matching pairs of double quotes from a
filename.   This  matches the  behavior  of  latex  etc,  and  deals  with
problems  that occasionally result from filenames that  have  been  incorrectly
quoted  on the command line.  In addition, under Microsoft Windows, the
forward slash character \verb|"\"| is a directory  separator, so  latexmk replaces it
by a backward slash \verb|"/"|, which is also a legal directory separator in Windows,
and is accepted  by  latex etc.

\subsubsection{-auxdir=FOO or -aux-directory=FOO}

Sets  the  directory  for  auxiliary  output files of (pdf)latex
(.aux, .log etc).  This achieves its effect by  the
\verb|-aux-directory|  option  of (pdf)latex, which currently is only implemented
on the MiKTeX version of (pdf)latex.

See  also  the  -outdir/-output-directory   options,   and   the
\verb|$aux_dir|,  \verb|$out_dir|,  and  \verb|$search_path_separator|
configuration variables of latexmk.  In particular, see the  documentation  of
\verb|$out_dir|  for  some  complications  on  what directory names are suitable.

If you also use the \verb|-cd| option, and the specified auxiliary output
directory  is a relative path, then the path is interpreted relative to the
document directory.


\subsubsection{-bibtex}

When the source file uses bbl files for bibliography, run bibtex or biber as
needed to regenerate the bbl files.

This  property can also be configured by setting the \verb|$bibtex_use| variable to 2
in a configuration file.

\subsubsection{-bibtex-}

Never run bibtex or biber.  Also, always  treat  .bbl  files  as
precious, i.e., do not delete them in a cleanup operation.

A  common  use  for this option is when a document comes from an external
source, complete with its bbl  file(s),  and  the  user does  not  have  the
corresponding bib files available.  In this situation use of the -bibtex-
option will prevent  latexmk  from trying to run bibtex or biber, which would
result in overwriting of the bbl files.

This property can also be configured by setting the  \verb|$bibtex_use| variable to 0
in a configuration file.

\subsubsection{-bibtex-cond}

When  the source file uses bbl file(s) for the bibliography, run
bibtex or biber as needed to regenerate the bbl files, but  only
if  the relevant bib file(s) exist.  Thus when the bib files are
not available, bibtex or biber  is  not  run,  thereby  avoiding
overwriting  of  the bbl file(s).  Also, always treat .bbl files

as precious, i.e., do not delete them in a cleanup operation.

This is the default setting.  It can also be configured by  setting the
\verb|$bibtex_use| variable to 1 in a configuration file.

The  reason for using this setting is that sometimes a .bbl file is available
containing the bibliography for a document, but the .bib  file  is  not
available.  An example would be for a scien- tific journal where authors submit
.tex and .bbl files, but  not the  original  .bib file.  In that case, running
bibtex or biber would not work, and the .bbl file should be treated  as  a
user source  file,  and  not as a file that can be regenerated on demand.

(Note that it is possible for latexmk to  decide  that  the  bib file  does
not  exist,  even though the bib file does exist and bibtex or biber finds it.
The problem is that the bib file  may not  be  in  the  current directory but
in some search path; the places latexmk and bibtex or biber cause to be
searched need not be  identical.   On modern installations of TeX and related
pro- grams this problem should not  arise,  since  latexmk  uses  the kpsewhich
program to do the search, and kpsewhich should use the same search path as
bibtex and biber.  If this  problem  arises, use the \verb|-bibtex| option when
invoking latexmk.)

Note that this value does not work properly if the document uses biber instead
of bibtex.  (There's a long story why not.)


\subsubsection{-bibtex-cond1}

The same as  -bibtex-cond1  except  that  .bbl  files  are  only
treated as precious if one or more bibfiles fails to exist.

Thus  if all the bib files exist, bibtex or biber is run to generate .bbl
files as needed, and then it is appropriate to delete the bbl files in a
cleanup operation since they can be re-generated.

This property can also be configured by setting the  \verb|$bibtex_use|
variable to 1.5 in a configuration file.


\subsubsection{-bm <message>}

A  banner message to print diagonally across each page when converting the dvi
file to postscript.  The message must be a  single argument on the command line
so be careful with quoting spaces and such.

Note that if the -bm option is specified, the -ps option is  assumed.


\subsubsection{-bi <intensity>}

How  dark to print the banner message.  A decimal number between
0 and 1.  0 is black and 1 is white.  The default is 0.95, which
is OK unless your toner cartridge is getting low.

\subsubsection{-bs <scale>}

A  decimal  number  that  specifies how large the banner message will be
printed.  Experimentation is necessary to get the  right scale  for  your
message, as a rule of thumb the scale should be about equal to 1100 divided by
the number of characters  in  the message.  The default is 220.0 which is just
right for 5 character messages.


\subsubsection{-commands}

List the commands used by latexmk for processing files, and then
exit.

\subsubsection{-c}

Clean up (remove) all regeneratable files generated by latex and bibtex or
biber except dvi, postscript and pdf.  These files are a  combination of log
files, aux files, latexmk's database file of source file information, and those
with extensions  specified in  the  \verb|@generated_exts|  configuration
variable.  In addition, files specified by the \verb|$clean_ext| and
\verb|@generated_exts| configuration variables are removed.

This  cleanup  is instead of a regular make.  See the -gg option if you want to
do a cleanup then a make.

Treatment of .bbl files:  If \verb|$bibtex_use| is set to 0 or  1, bbl files  are
always treated as non-regeneratable.  If \verb|$bibtex_use| is set to 1.5, bbl files
are counted as non-regeneratable conditionally:  If the bib file exists, then
bbl files are regeneratable, and are deleted in a clean up.  But if
\verb|$bibtex_use| is  1.5 and  a bib file doesn't exist, then the bbl files are
treated as non-regeneratable and hence are not deleted.

In contrast, if \verb|$bibtex_use| is set to 2, bbl  files  are  always
treated as regeneratable, and are deleted in a cleanup.

Treatment   of   files  generated  by  custom  dependencies:  If
\verb|$cleanup_includes_cusdep_generated|  is  nonzero,   regeneratable files  are
considered as including those generated by custom dependencies and are also
deleted.  Otherwise these files are  not deleted.


\subsubsection{-C}

Clean up (remove) all regeneratable files generated by latex and bibtex or
biber.  This is the same as the \verb|-c| option with the addition  of dvi,
postscript and pdf files, and those specified in the \verb|$clean_full_ext|
configuration variable.

This cleanup is instead of a regular make.  See the  \verb|-gg|  option
if you want to do a cleanup than a make.

See  the  -c option for the specification of whether or not .bbl files are
treated as non-regeneratable or regeneratable.

If \verb|$cleanup_includes_cusdep_generated| is nonzero,  regeneratable files
are considered as including those generated by custom dependencies and are also
deleted.  Otherwise these files are  not deleted.


\subsubsection{-CA}

(Obsolete).   Now  equivalent to the -C option.  See that option for details.


\subsubsection{-cd}

Change to the directory containing the main source  file  before processing
it.  Then all the generated files (.aux, .log, .dvi, .pdf, etc) will be
relative to the source file.

This option is particularly useful when latexmk is invoked  from
a  GUI configured to invoke latexmk with a full pathname for the
source file.

This option works by setting the \verb|$do_cd|  configuration  variable to
one;  you can set that variable if you want to configure latexmk to have the
effect of the \verb|-cd| option without specifying it on the command line.  See the
documentation for that variable.

\subsubsection{-cd-}

Do  NOT  change to the directory containing the main source file before
processing it.  Then all the generated files (.aux, .log, .dvi,  .pdf,  etc)
will  be  relative  to the current directory rather than the source file.

This is the default behavior and corresponds to the behavior  of
the  latex  and pdflatex programs.  However, it is not desirable
behavior when latexmk is invoked by a GUI configured  to  invoke
latexmk  with  a full pathname for the source file.  See the \verb|-cd|
option.

This option works by setting the \verb|$do_cd|  configuration  variable to
zero.   See the documentation for that variable for more information.

\subsubsection{-CF}

Remove the file containing the database of source file  information,
before doing the other actions requested.


\subsubsection{-d}

Set  draft  mode.  This prints the banner message "DRAFT" across
your page when converting the dvi file to postscript.  Size  and
intensity can be modified with the -bs and -bi options.  The -bm
option will override this option as this is really just a  short
way of specifying:

\begin{verbatim}
  latexmk -bm DRAFT
\end{verbatim}

Note that if the \verb|-d| option is specified, the \verb|-ps| option is assumed.

\subsubsection{-deps}

Show a list of dependent files after processing.  This is in the form  of a
dependency list of the form used by the make program, and it is therefore
suitable for use in a Makefile.  It gives an overall view of the files without
listing intermediate files, as well as latexmk can determine them.

By default the list of dependent files is sent to stdout  (i.e., normally  to
the screen unless you've redirected latexmk's output). But you can set the
filename where the list is sent by the \verb|-deps-out=| option.

See  the section "USING latexmk WITH make" for an example of how
to use a dependency list with make.

Users familiar with GNU automake and  gcc  will  find  that  the \verb|-deps|
option is very similar in its purpose and results to the \verb|-M| option to
gcc.  (In fact, latexmk also has options  \verb|-M|,  \verb|-MF|, and
\verb|-MP| options that behave like those of gcc.)

\subsubsection{-dependents}

Equivalent to \verb|-deps|.

\subsubsection{-deps-}

Do  not  show a list of dependent files after processing.  (This is the
default.)

\subsubsection{-dependents-}

Equivalent to \verb|-deps-|.


\subsubsection{-deps-out=FILENAME}

Set the filename to which the list of dependent files  is  written.   If  the
FILENAME argument is omitted or set to \verb|"-"|, then the output is sent to
stdout.

Use of this option also turns on the output of the list  of  dependent files
after processing.

\subsubsection{-dF}

Dvi  file  filtering.   The  argument to this option is a filter which will
generate a  filtered  dvi  file  with  the  extension ".dviF".   All  extra
processing (e.g. conversion to postscript, preview, printing) will then be
performed on this  filtered  dvi file.

\paragraph{Example usage}

To use dviselect to select only the even pages of the dvi file:


\begin{verbatim}
  latexmk -dF "dviselect even" foo.tex
\end{verbatim}


\subsubsection{-diagnostics}

Print detailed diagnostics during a run.  This may help for  debugging problems
or to understand latexmk's behavior in difficult situations.

\subsubsection{-dvi}

Generate dvi version of document.

\subsubsection{-dvi-}

Turn off generation of dvi version of document.  (This  may  get
overridden,  if  some other file is made (e.g., a .ps file) that
is generated from the dvi file, or if no generated file  at  all
is requested.)


\subsubsection{-e <code>}

Execute  the  specified  initialization  code before processing.  The code is
Perl code of the same form as is used  in  latexmk's initialization  files.
For more details, see the information on the -r option, and the section about
"Configuration/initialization  (RC)  files".  The code is typically a
sequence of assignment statements separated by semicolons.

The code is executed when the -e option  is  encountered  during latexmk's
parsing of its command line.  See the -r option for a way of executing
initialization code from a file.  An error  results  in latexmk stopping.
Multiple instances of the -r and -e options can be used, and they are executed
in the order they appear on the command line.

Some care is needed to deal with proper quoting of special characters in the
code on the command line.   For  example,  suppose you  want  to set the latex
command to use its \verb|-shell-escape| option, then under UNIX/Linux you could use
the line

\begin{verbatim}
  latexmk -e '$latex=q/latex %O -shell-escape %S/' file.tex
\end{verbatim}

Note that the single  quotes  block  normal  UNIX/Linux  command shells  from
treating  the characters inside the quotes as spe- cial.  (In this example, the
q/.../ construct is  a  Perl  idiom equivalent  to  using  single quotes.  This
avoids the complica- tions of getting a quote  character  inside  an  already
quoted string  in  a  way that is independent of both the shell and the
operating-system.)

The above command line  will  NOT  work  under  MS-Windows  with
cmd.exe  or  command.com  or 4nt.exe.  For MS-Windows with these
command shells you could use

\begin{verbatim}
  latexmk -e "$latex=q/latex %O -shell-escape %S/" file.tex
\end{verbatim}

or

\begin{verbatim}
  latexmk -e "$latex='latex %O -shell-escape %S'" file.tex
\end{verbatim}

The last two examples will  NOT  work  with  UNIX/Linux  command
shells.

(Note:  the above examples show are to show how to use the -e to specify
initialization code to be executed.  But the  particular effect can be achieved
also by the use of the -latex option with less problems in dealing with
quoting.)

\subsubsection{-f}

Force latexmk to continue document  processing  despite  errors.  Normally,
when latexmk detects that LaTeX or another program has found an error which
will not be resolved by further processing, no further processing is carried
out.

Note:  "Further  processing" means the running of other programs or the
rerunning of latex (etc) that would be done if no  errors had  occurred.   If
instead, or additionally, you want the latex (etc) program not to pause for
user input after  an  error,  you should  arrange this by an option that is
passed to the program, e.g., by latexmk's option -interaction=nonstopmode.


\subsubsection{-f-}

Turn off the forced processing-past-errors such as is set by the -f  option.
This could be used to override a setting in a configuration file.


\subsubsection{-g}

Force latexmk to process document fully, even  under  situations
where  latexmk  would  normally  decide  that  no changes in the
source files have occurred since the previous run.  This  option
is  useful,  for example, if you change some options and wish to
reprocess the files.

\subsubsection{-g-}

Turn off -g.

\subsubsection{-gg}

"Super go mode" or "clean make": clean out generated files as if -C had been
given, and then do a regular make.

\subsubsection{-h, -help}

Print help information.

\subsubsection{-jobname=STRING}

Set  the  basename  of output files(s) to STRING, instead of the
default, which is the basename of the specified TeX  file.   (At
present, STRING should not contain spaces.)

This  is like the same option for current implementations of the
latex, pdflatex, etc, and the passing of this  option  to  these
programs is part of latexmk's implementation of -jobname.

There is one enhancement, that the STRING may contain the place- holder
\verb|'%A'|. This will be substituted by the basename of the TeX file.   The
primary purpose is when multiple files are specified on the command line to
latexmk, and you wish to  use  a  jobname with  a different file-dependent
value for each file.  For exam- ple, suppose you had .tex files test1.tex and
test2.tex, and you wished  to  compare  the  results of compilation by pdflatex
and those with xelatex.  Then under a unix-type operating system you could use
the command line

\begin{verbatim}
	latexmk -pdf -jobname=%A-pdflatex *.tex
	latexmk -pdfxe -jobname=%A-xelatex *.tex
\end{verbatim}

Then  the  .aux,  .log,  and .pdf files from the use of pdflatex
would have basenames test1-pdflatex  and  test2-pdflatex,  while
from xelatex, the basenames would be test1-xelatex and test2-xe-
latex.

Under MS-Windows with cmd.exe, you would need to double the per-
cent  sign,  so  that the percent character is passed to latexmk
rather than being used to substitute an environment variable:

\begin{verbatim}
	latexmk -pdf -jobname=%%A-pdflatex *.tex
	latexmk -pdfxe -jobname=%%A-xelatex *.tex
\end{verbatim}


\subsubsection{-l}

Run in landscape mode, using the landscape mode for the previewers  and  the
dvi to postscript converters.  This option is not normally needed nowadays,
since current previewers normally  determine this information automatically.

\subsubsection{-l-}

Turn off -l.


\subsubsection{-latex}

This  sets  the  generation of dvi files by latex, and turns off
the generation of pdf and ps files.

Note: to set the command used when latex is specified,  see  the
-latex="COMMAND" option.


\subsubsection{-latex=COMMAND}

This sets the string specifying the command to run latex, and is typically used
to add desired options.  Since  the  string  normally contains spaces, it
should be quoted, e.g.,

\begin{verbatim}
	latexmk -latex="latex --shell-escape %O %S"  foo.tex
\end{verbatim}

The  specification of the contents of the string are the same as
for the \verb|$latex| configuration variable.  Depending on your  oper-
ating  system  and the command-line shell you are using, you may
need to change the single quotes to double quotes (or  something
else).

Note:  This  option when provided with the COMMAND argument only
sets the command for invoking latex; it does not turn on the use
of  latex. That is done by other options or in an initialization
file.

To set the command for running pdflatex (rather than the command
for latex) see the \verb|-pdflatex| option.


\subsubsection{-logfilewarninglist}
-logfilewarnings

After a run of (pdf)latex, give a list of warnings about undefined citations
and  references  (unless  silent mode is on).

See also the \verb|$silence_logfile_warnings| configuration variable.

\subsubsection{-logfilewarninglist-}

-logfilewarnings-  After a run of (pdf)latex, do not give a list
of warnings about undefined citations and references.  (Default)

See also the \verb|$silence_logfile_warnings configuration| variable.


\subsubsection{-lualatex}

Use lualatex.  That is,  use  lualatex  to  process  the  source
file(s)  to  pdf.  The generation of dvi and postscript files is
turned off.

This option is equivalent to using the following set of options

\begin{verbatim}
	-pdflua -dvi- -ps-
\end{verbatim}

(Note: Note that the method of implementation  of  this  option,
but  not  its intended effect, differ from some earlier versions
of latexmk.)


\subsubsection{-lualatex=COMMAND}

This sets the string specifying the command to run lualatex.  It
behaves like the -pdflatex option, but sets the variable
\verb|$lualatex|.

\paragraph{Note:}

This option when provided with the COMMAND  argument  only sets  the command
for invoking lualatex; it does not turn on the use of lualatex. That is done by
other options or in an initialization file.

\subsubsection{-M}

Show  list of dependent files after processing.  This is equivalent to
the -deps option.


\subsubsection{-MF file}

If a list of dependents is made, the -MF specifies the  file  to
write it to.


\subsubsection{-MP}

If a list of dependents is made, include a phony target for each
source file.  If you use the dependents list in a Makefile,  the
dummy rules work around errors the program make gives if you re-
move header files without updating the Makefile to match.

\subsubsection{-MSWinBackSlash}

This option only has an effect when latexmk is running under MS-Windows.  This
is that when latexmk runs a command under MS-Windows, the Windows standard
directory separator \verb|"\"|  is  used  to separate  directory  components in a file
name.  Internally, latexmk uses \verb|"/"| for the directory separator character,
which  is the character used by Unix-like systems.

This is the default behavior.  However the default may have been overridden by
a configuration file (latexmkrc file)  which  sets \verb|$MSWin_back_slash=0|.


\subsubsection{-MSWinBackSlash-}

This option only has an effect when latexmk is running under MS-Windows.  This
is that when latexmk runs a command under MS-Windows,  the  substitution  of
\verb|"\"| for the separator character between directory components of a file
name is not  done. Instead the forward slash \verb|"/"| is used, the same as
on Unix-like systems.  This is acceptable in most situations under MS-Windows,
provided that filenames are properly quoted, as latexmk does by default.

See   the   documentation   for   the   configuration   variable
\verb|$MSWin_back_slash| for more details.


\subsubsection{-new-viewer}

When in continuous-preview mode, always start a  new  viewer  to view  the
generated file.  By default, latexmk will, in continuous-preview mode, test for
a previously  running  previewer  for the same file and not start a new one if
a previous previewer is running.  However, its test sometimes fails (notably if
there is an  already-running previewer that is viewing a file of the same name
as the current file, but in a different  directory).   This option turns off
the default behavior.


\subsubsection{-new-viewer-}

The  inverse  of the -new-viewer option.  It puts latexmk in its
normal behavior that in preview-continuous mode it checks for an
already-running previewer.

\subsubsection{-nobibtex}

Never run bibtex or biber.  Equivalent to the -bibtex- option.

\subsubsection{-norc}

Turn off the automatic reading of initialization (rc) files.

N.B.  Normally the initialization files are read and obeyed, and then command
line options are obeyed in the order they  are  encountered.   But \verb|-norc|
is an exception to this rule: it is acted on first, no matter where it occurs
on the command line.

\subsubsection{-outdir=FOO or -output-directory=FOO}

Sets the directory for the output  files  of  (pdf)latex.   This achieves  its
effect by the \verb|-output-directory| option of (pdf)latex, which currently
(Dec. 2011 and later) is implemented on the common versions of (pdf)latex,
i.e., MiKTeX and TeXLive.  It may not be present in other versions.

See also the -auxdir/-aux-directory options, and  the  \verb|$aux_dir|,
\verb|$out_dir|,  and \verb|$search_path_separator| configuration variables of
latexmk.  In particular, see the documentation of  \verb|$out_dir|  for
some complications on what directory names are suitable.

If  you also use the \verb|-cd| option, and the specified output directory is a
relative path, then the path is  interpreted  relative to the document
directory.


\subsubsection{-p}

Print  out  the  document.  By default the file to be printed is the first in
the list postscript, pdf, dvi that is  being  made.  But you can use the
-print=... option to change the type of file to be printed, and you can
configure this in a start up file (by setting the \verb|$print_type| variable).

However,  printing  is  enabled by default only under UNIX/Linux
systems, where the default is to use the lpr command and only on
postscript files.  In general, the correct behavior for printing very much
depends on your system's software.  In particular, under  MS-Windows you must
have suitable program(s) available, and you must have configured the print
commands  used  by  latexmk.  This  can  be  non-trivial.   See the
documentation on the \verb|$lpr|, \verb|$lpr_dvi|, and \verb|$lpr_pdf|
configuration variables to see how to set
the commands for printing.

This option is incompatible with the -pv and -pvc options, so it
turns them off.

\subsubsection{-pdf}

Generate pdf version of document using pdflatex.  (If  you  wish to use
lualatex or xelatex, you can use whichever of the options \verb|-pdflua|, \verb|-pdfxe|,
\verb|-lualatex| or \verb|-xelatex| applies.)   To  configure latexmk  to  have  such
behavior by default, see the section on "Configuration/initialization (rc)
files".

\subsubsection{-pdfdvi}

Generate pdf version of document from the dvi file,  by  default
using dvipdf.

\subsubsection{-pdflua}
Generate pdf version of document using lualatex.

\subsubsection{-pdfps}

Generate  pdf  version of document from the .ps file, by default
using ps2pdf.


\subsubsection{-pdfxe}

Generate pdf version of document using xelatex.   Note  that  to optimize
processing  time,  latexmk uses xelatex to generate an .xdv file rather than a
pdf file directly.  Only after  possibly multiple  runs to generate a fully
up-to-date .xdv file does latexmk then call xdvipdfmx to generate the final
.pdf file.

(Note: The reason why latexmk arranges for xelatex  to  make  an .xdv  file
instead of the xelatex's default of a .pdf file is as follows: When the
document includes large graphics files,  especially  .png  files,  the
production of a .pdf file can be quite time consuming, even when the creation
of the .xdv file by xelatex  is  fast.  So the use of the intermediate .xdv
file can result in substantial gains in procesing time, since the .pdf file
is produced once rather than on every run of xelatex.)


\subsubsection{-pdf-}

Turn  off  generation  of pdf version of document.  (This can be used to
override a setting in a configuration file.  It may  get overridden if some
other option requires the generation of a pdf file.)

If after all options have  been  processed,  pdf  generation  is still  turned
off, then generation of a dvi file will be turned on, and then the program used
to compiled a document will be latex  (or,  more  precisely, whatever program
is configured to be used in the \verb|$latex| configuration variable).


\subsubsection{-pdflatex}

This sets the generation of pdf files by pdflatex, and turns off
the generation of dvi and ps files.

Note:  to  set  the command used when pdflatex is specified, see
the -pdflatex="COMMAND" option.

\subsubsection{-pdflatex=``COMMAND''}

This sets the string specifying the command to run pdflatex, and
is typically used to add desired options.  Since the string nor-
mally contains spaces, it should be quoted, e.g.,

\begin{verbatim}
	latexmk  -pdf  -pdflatex="pdflatex  --shell-escape  %O  %S" foo.tex
\end{verbatim}

The  specification of the contents of the string are the same as for the
\verb|$pdflatex| configuration variable.  (The option -pdflatex in fact sets
the variable \verb|$pdflatex|.)  Depending on your operating system and the
command-line shell you  are  using,  you  may need  to change the single quotes
to double quotes (or something else).

Note: This option when provided with the COMMAND  argument  only
sets  the command for invoking pdflatex; it does not turn on the
use of pdflatex. That is done by other options or in an initial-
ization file.

To  set  the  command for running latex (rather than the command
for pdflatex) see the -latex option.


\subsubsection{-pdflualatex=``COMMAND''}

Equivalent to -lualatex="COMMAND".


\subsubsection{-pdfxelatex=``COMMAND''}

Equivalent to -xelatex=``COMMAND''.


\subsubsection{-pretex=CODE}

Given that CODE is some TeX code, this options sets that code to
be  executed  before  inputting source file.  This only works if the command
for invoking the relevant *latex is suitably configured.   See the
documentation of the variable \verb|$pre_tex_code|, and the substitution
strings \verb|%P| and \verb|%U| for more details.   This  option works by
setting the variable \verb|$pre_tex_code|.

See also the \verb|-usepretex| option.

An example:

\begin{verbatim}
	latexmk  -pretex='\AtBeginDocument{Message\par}'  -usepretex foo.tex
\end{verbatim}

But this is better written

\begin{verbatim}
	latexmk -usepretex='\AtBeginDocument{Message\par}' foo.tex
\end{verbatim}

If you already have a suitable command configured, you only need

\begin{verbatim}
	latexmk -pretex='\AtBeginDocument{Message\par}' foo.tex
\end{verbatim}

\subsubsection{-print=dvi, -print=ps, -print=pdf, -print=auto,}

Define which kind of file is printed.  This option also  ensures that the
requisite file is made, and turns on printing.

The (default) case -print=auto determines the kind of print file automatically
from the set of files that  is  being  made.   The first  in  the list
postscript, pdf, dvi that is among the files to be made is the one used for
print out.


\subsubsection{-ps}

Generate postscript version of document.

\subsubsection{-ps-}

Turn off generation of postscript version of document.  This can
be  used to override a setting in a configuration file.  (It may
get overridden by some other option that requires  a  postscript
file, for example a request for printing.)

-pF

Postscript  file  filtering.   The  argument to this option is a
filter which will generate a filtered postscript file  with  the
extension ".psF".  All extra processing (e.g. preview, printing)
will then be performed on this filtered postscript file.

Example of usage: Use psnup to print two pages on the one page:

latexmk -ps -pF 'psnup -2' foo.tex

or

latexmk -ps -pF "psnup -2" foo.tex

Whether to use single or double quotes round the "psnup -2" will
depend  on  your  command interpreter, as used by the particular
version of perl and the operating system on your computer.


       -pv    Run file previewer.  If the -view option is used, this will  se-
              lect the kind of file to be previewed (.dvi, .ps or .pdf).  Oth-
              erwise the viewer views the "highest" kind of file selected,  by
              the  -dvi,  -ps,  -pdf,  -pdfps options, in the order .dvi, .ps,
              .pdf (low to high).  If no file type has been selected, the  dvi
              previewer will be used.  This option is incompatible with the -p
              and -pvc options, so it turns them off.


       -pv-   Turn off -pv.


-pvc

Run a file previewer  and  continually  update  the  .dvi,  .ps,
and/or .pdf files whenever changes are made to source files (see
the Description above).  Which of these files is  generated  and
which  is  viewed  is  governed by the other options, and is the
same as for the -pv option.  The preview-continuous option  -pvc
can  only work with one file.  So in this case you will normally
only specify one filename on the command line.  It is  also  in-
compatible  with  the  -p and -pv options, so it turns these op-
tions off.

The -pvc option also turns off force mode (-f), as  is  normally
best  for  continuous  preview  mode.   If you really want force
mode, use the options in the order -pvc -f.

With a good previewer the display will be automatically updated.
(Under  some but not all versions of UNIX/Linux "gv -watch" does
this for postscript files; this can be set  by  a  configuration
variable.   This would also work for pdf files except for an ap-
parent bug in gv that causes an error when the newly updated pdf
file is read.)  Many other previewers will need a manual update.

Important note: the acroread program on MS-Windows locks the pdf
file, and prevents new versions being written, so it  is  a  bad
idea  to  use  acroread  to view pdf files in preview-continuous
mode.  It is better to use a different  viewer:  SumatraPDF  and
gsview are good possibilities.

There  are  some  other methods for arranging an update, notably
useful for many versions of xdvi and xpdf.  These are  best  set
in latexmk's configuration; see below.

Note  that  if  latexmk  dies  or  is  stopped  by the user, the
"forked" previewer will continue to run.  Successive invocations
with  the  -pvc option will not fork new previewers, but latexmk
will normally use the existing previewer.  (At least  this  will
happen  when  latexmk is running under an operating system where
it knows how to determine whether an existing previewer is  run-
ning.)


-pvc-  Turn off -pvc.


\subsubsection{-pvctimeout}

Do  timeout  in pvc mode after period of inactivity, which is 30
min. by default.  Inactivity means a period when latexmk has de-
tected  no file changes and hence has not taken any actions like
compiling the document.


\subsubsection{-pvctimeout-}

Don't do timeout in pvc mode after inactivity.

\subsubsection{-pvctimeoutmins=<time>}

Set period of inactivity in minutes for pvc timeout.

\subsubsection{-quiet}

Same as -silent

\subsubsection{-r <rcfile>}

Read the specified initialization file ("RC file")  before  processing.

Be careful about the ordering: 

\begin{itemize}
	

\item (1) Standard initialization files --- see the section below on
				``Configuration/initialization  (RC) files''  ---  are read first.  

\item (2) Then the options on the command line are acted on in the order they
				are given.  Therefore if  an initialization  file  is  specified by the
				-r option, it is read during this second step.  Thus an initialization
				file specified with the -r option can override both the standard
				initialization files and previously specified options.  But all of
				these can be overridden by later options.
\end{itemize}

The contents of the RC file just comprise a piece of code in the Perl
programming language (typically a  sequence  of  assignment statements); they
are executed when the -r option is encountered during latexmk's parsing of its
command line.  See the -e option for  a  way  of giving initialization code
directly on latexmk's command line.  An error results in latexmk  stopping.
Multiple instances of the -r and -e options can be used, and they are executed
in the order they appear on the command line.


\subsubsection{-recorder}

Give the \verb|-recorder| option with latex and  pdflatex.   In  (most) modern
versions of these programs, this results in a file of extension .fls
containing a list of the files that these  programs have  read  and written.
Latexmk will then use this file to improve its detection of source files and
generated files after  a run  of  latex  or pdflatex.  This is the default
setting of latexmk, unless overridden in an initialization file.

For further information, see the documentation for the \verb|$recorder| configuration
variable.


\subsubsection{-recorder-}

Do not supply the -recorder option with latex and pdflatex.

\subsubsection{-rules}

Show a list of latemk's rules and dependencies after processing.

\subsubsection{-rules-}

Do  not  show  a  list of latexmk's rules and dependencies after
processing.  (This is the default.)


\subsubsection{-showextraoptions}

Show the list of extra latex and pdflatex options  that  latexmk
recognizes,  but  that  it simply passes through to the programs
latex, pdflatex, etc  when they  are  run.   These  options  are
(currently)  a  combination  of those allowed by the TeXLive and
MiKTeX implementations.  (If a particular option is given to la-
texmk but is not handled by the particular implementation of la-
tex or pdflatex that is being used, that program  will  probably
give  an  error  message.)  These options are very numerous, but
are not listed in this documentation because they have no effect
on latexmk's actions.

There  are  a  few  options (-includedirectory=dir, -initialize,
-ini) that are not recognized, either  because  they  don't  fit
with latexmk's intended operations, or because they need special
processing by latexmk that  isn't  implemented  (at  least,  not
yet).

There  are  also options that are accepted by latex etc, but in-
stead trigger actions by latexmk: -help, -version.

Finally, there are certain options for latex and pdflatex (e.g.,
-recorder)  that  trigger special actions or behavior by latexmk
itself as well as being passed in some form to the called  latex
and  pdflatex  program,  or  that affect other programs as well.
These options do have entries in this documentation.  These  op-
tions  are:  -jobname=STRING, -aux-directory=dir, -output-direc-
tory=DIR, -quiet, and -recorder.


\subsubsection{-silent}

Run commands silently, i.e., with options that reduce the amount of
diagnostics  generated.   For example, with the default set- tings, the command
"latex -interaction=batchmode"  is  used  for latex, and similarly for its
friends.

See  also  the  -logfilewarninglist and -logfilewarninglist- options.

Also reduce the number of informational  messages  that  latexmk
itself generates.

To  change  the  options used to make the commands run silently, you need to
configure latexmk with changed values of its configuration    variables, the
relevant    ones   being   \verb|$bibtex_silent_switch|,
\verb|$biber_silent_switch|,  \verb|$dvipdf_silent_switch|,
\verb|$dvips_silent_switch|, \verb|$latex_silent_switch|,
\verb|$lualatex_silent_switch| \verb|$makeindex_silent_switch|,
\verb|$pdflatex_silent_switch|, and \verb|$xelatex_silent_switch|


\subsubsection{-stdtexcmds}

Sets  the commands for latex, etc, so that they are the standard
ones. This is useful to override special configurations.

The result is that \verb|$latex = 'latex %O  %S'|,  and  similarly  for
\verb|$pdflatex|,  \verb|$lualatex|, and \verb|$xelatex|.  (The option \verb|-no-pdf| needed
for \verb|$xelatex| is provided automatically, given that \verb|%O| appears in
the definition.)


\subsubsection{-time}

Show  CPU  time  used.   See  also  the  configuration  variable
\verb|$show_time|.


\subsubsection{-time-}

Do not show CPU time used.  See also the configuration  variable
\verb|$show_time|.


\subsubsection{-use-make}

When  after a run of latex or pdflatex, there are warnings about missing files
(e.g., as requested by the LaTeX \verb|\input|, \verb|\include|, and  \verb|\includgraphics|
commands), latexmk tries to make them by a custom dependency. If no relevant
custom dependency with an  appropriate  source  file is found, and if the
-use-make option is set, then as a last resort latexmk will try to use the make
pro- gram to try to make the missing files.

Note  that  the  filename may be specified without an extension, e.g., by
\verb|\includegraphics{drawing}| in a  LaTeX  file.   In  that case,  latexmk  will
try making drawing.ext with ext set in turn to the possible extensions that are
relevant for  latex  (or  as appropriate pdflatex).

See  also  the documentation for the \verb|$use_make_for_missing_files| configuration
variable.


\subsubsection{-use-make-}

Do not use the make program to try to make missing files.   (Default.)


\subsubsection{-usepretex}

Sets the command lines for latex, etc, so that they use the code that is
defined by the variable \verb|$pre_tex_code| or that is set  by the option
\verb|-pretex=CODE| to execute the specified TeX code before the source file is read.
This  option  overrides  any  previous definition of the command lines.

The  result  is  that  \verb|$latex = 'latex %O %P'|, and similarly for
\verb|$pdflatex|, \verb|$lualatex|, and \verb|$xelatex|.  (The option \verb|-no-pdf|  needed
for \verb|$xelatex| is provided automatically, given that \verb|%O| appears in
the definition.)


\subsubsection{-usepretex=CODE}

Equivalent to \verb|-pretex=CODE -usepretex|.  Example

\begin{verbatim}
	latexmk -usepretex='\AtBeginDocument{Message\par}' foo.tex
\end{verbatim}

\subsubsection{-v, -version}

Print version number of latexmk.

\subsubsection{-verbose}

Opposite of -silent.  This is the default setting.

\subsubsection{-view=default, -view=dvi, -view=ps, -view=pdf, -view=none}

Set the kind of file used when previewing is requested (e.g., by the -pv or
-pvc switches).  The default is to view the "highest" kind of requested file
(in  the  low-to-high  order  .dvi,  .ps, .pdf).

Note  the  possibility  \verb|-view=none|  where no viewer is opened at all.
One example of is use is in conjunction with the \verb|-pvc|  option,  when  you
want latexmk to do a compilation automatically whenever source file(s) change,
but do not want a  previewer  to be opened.

\subsubsection{-Werror}

This  causes  latexmk to return a non-zero status code if any of the files
processed gives a warning about  problems  with  citations  or references
(i.e., undefined citations or references or about multiply defined references).
This is after  latexmk  has completed  all  the  runs it needs to try and
resolve references and citations.  Thus \verb|-Werror| causes latexmk to treat
such  warnings  as  errors,  but  only  when they occur on the last run of
(pdf)latex and only after processing is complete.  Also  can  be set by the
configuration variable \verb|$warnings_as_errors|.

\subsubsection{-xelatex}

Use xelatex.  That is, use xelatex to process the source file(s) to pdf.  The
generation of dvi and postscript  files  is  turned off.

This option is equivalent to using the following set of options

\begin{verbatim}
	-pdfxe -dvi- -ps-
\end{verbatim}

[Note:  Note  that  the method of implementation of this option,
but not its intended primary effect, differ  from  some  earlier
versions  of latexmk. Latexmk first uses xelatex to make an .xdv
file, and does all the extra runs  needed  (including  those  of
bibtex,  etc).   Only  after that does it make the pdf file from
the .xdv file, using xdvipdfmx.  See the documentation  for  the
-pdfxe for why this is done.]


\subsubsection{-xelatex=``COMMAND''}

This  sets the string specifying the command to run xelatex.  It sets the
variable \verb|$xelatex|.  

\paragraph{Warning:}

It is important to ensure that the -no-pdf is used
when xelatex  is invoked, since latexmk expects xelatex to produce an .xdv
file, not a .pdf file. If you provide  \verb|%O|  in  the  command specification,
this  will be done automatically.  See the documentation for the -pdfxe
option for why  latexmk  makes  a  .xdv file rather than a .pdf file when
xelatex is used.

An example of the use of the \verb|-pdfxelatex| option:

\begin{verbatim}
	latexmk  -pdfxe  -pdfxelatex="xelatex --shell-escape %O %S" foo.tex
\end{verbatim}

Note: This option when provided with the COMMAND  argument  only sets  the
command for invoking lualatex; it does not turn on the use of lualatex. That is
done by other options or in an initialization file.

Compatibility between options

The  preview-continuous option -pvc can only work with one file.  So in
this case you will normally only specify one filename  on  the  command
line.

Options  -p, -pv and -pvc are mutually exclusive.  So each of these op- tions
turns the others off.



