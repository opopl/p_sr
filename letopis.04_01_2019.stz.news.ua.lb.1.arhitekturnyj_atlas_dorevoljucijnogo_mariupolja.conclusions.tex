% vim: keymap=russian-jcukenwin
%%beginhead 
 
%%file 04_01_2019.stz.news.ua.lb.1.arhitekturnyj_atlas_dorevoljucijnogo_mariupolja.conclusions
%%parent 04_01_2019.stz.news.ua.lb.1.arhitekturnyj_atlas_dorevoljucijnogo_mariupolja
 
%%url 
 
%%author_id 
%%date 
 
%%tags 
%%title 
 
%%endhead 

***

Нельзя сказать, что мариупольская общественность равнодушна к судьбе
архитектурного достояния. Местные краеведы уделили достаточно много времени
изучению истории старых зданий. Также в городе существует актив небезразличных
граждан. Но пока что обстоятельства сильнее. Общая тенденция печальна – старый
Мариуполь постепенно исчезает.

В качестве показательного примера можно привести дом дореволюционного
градоначальника Попова, который тихонько разваливался у центрального сквера на
глазах всего города целых пять лет. Немногочисленные прецеденты восстановления
исторических сооружений видятся скорее случайностями, чем стратегией. Всего
пять зданий в Мариуполе имеют статус архитектурной памятки местного значения, и
только три относятся к рассматриваемому нами периоду. Но благодаря бюрократии и
эти объекты могут лишиться своей защиты. Кроме отсутствия официального статуса
существует ещё один фактор способствующий упадку исторического центра Мариуполя
– это география.

Практически вся архитектурная старина находится в зоне накрываемой выбросами
металлургического завода \enquote{Азовсталь}. Из исторического центра, под давлением
экологической обстановки переселяется всё больше людей, бросая древние дома на
произвол. Кое-где старый Мариуполь напоминает город-призрак. Район становится
всё менее привлекательным как для жилья, так и для любой хозяйственной
деятельности. А значит, историческая энтропия неизбежно разложит многие
архитектурные памятки на элементарные частицы. Наступит время и останется
только смотреть на фотографии, и вздыхать о потерянной красоте Мариуполя.

Іван Станіславський, журналіст

%\ifcmt
  %ig https://i2.paste.pics/81a7dae62fd6266f17ecbb6f00ff0f18.png
  %@wrap center
  %@width 0.9
%\fi
