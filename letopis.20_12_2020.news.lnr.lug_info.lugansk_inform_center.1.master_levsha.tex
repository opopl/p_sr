% vim: keymap=russian-jcukenwin
%%beginhead 
 
%%file 20_12_2020.news.lnr.lug_info.lugansk_inform_center.1.master_levsha
%%parent 20_12_2020
 
%%url http://lug-info.com/comments/one/master-kluba-levsha-inna-chekareva-ukrashennyi-svoimi-rukami-dom-stanovitsya-osobenno-teplym-i-uyutnym-809
 
%%author 
%%author_id lugansk_inform_center
%%author_url 
 
%%tags 
%%title Мастер клуба "Левша" Инна Чекарева: "Украшенный своими руками дом становится особенно теплым и уютным"
 
%%endhead 
 
\subsection{Мастер клуба \enquote{Левша} Инна Чекарева: \enquote{Украшенный своими руками дом становится особенно теплым и уютным}}
\label{sec:20_12_2020.news.lnr.lug_info.lugansk_inform_center.1.master_levsha}
\Purl{http://lug-info.com/comments/one/master-kluba-levsha-inna-chekareva-ukrashennyi-svoimi-rukami-dom-stanovitsya-osobenno-teplym-i-uyutnym-809}
\ifcmt
	author_begin
   author_id lugansk_inform_center
	author_end
\fi

\index[names.rus]{Чекарева, Инна!Мастер клуба "Левша", ЛНР, 20.12.2020}

\ifcmt
tab_begin cols=3
	caption О том, как изготовить украшения для дома и новогодней елки своими руками ЛИЦ рассказала мастер декоративно-прикладного и изобразительного искусства народного клуба "Левша" Луганского центра народного творчества (ЛЦНТ) Инна Чекарева.

  pic http://img.lug-info.com/cache/a/a/IMG_2963_exposure.JPG/w620h420.jpg
	width 0.4

	pic http://img.lug-info.com/cache/e/e/kruzhok7.jpg/w620h420.jpg
	pic http://img.lug-info.com/cache/1/c/kot.jpg/w620h420.jpg

	pic http://img.lug-info.com/cache/3/3/izobrazhenie.png/w620h420.png
	width 0.4

	pic http://img.lug-info.com/cache/c/f/izobrazhenie_(4).png/w620h420.png
	pic http://img.lug-info.com/cache/7/e/izobrazhenie_(2).png/w620h420.png
tab_end
\fi

\subsubsection{ЛЕКАРСТВО ОТ СТРЕССА}

\textbf{– Инна Михайловна, как давно вы занимаетесь творчеством, в каких техниках
работаете?}

– По образованию я психолог, поэтому очень хорошо понимаю важность арт-терапии
для снятия стресса, восстановления внутреннего спокойствия и гармонии. Начала
заниматься декоративно-прикладным искусством после участия в ежегодных
фестивалях развития личности "Жизнь как чудо", где старались раскрыть различные
возможности каждого человека. В ходе одного из фестивалей я попала на
мастер-класс по валянию из шерсти, и после этого моя жизнь переменилась.

Сейчас работаю в разных техниках. Одна из них – мандалоплетение (мандала в
переводе с санскрита – круг). Это создание в круге различных узоров, выражающих
внутренний мир человека, прекрасный релакс, потому что сам круг завораживает.
Эта техника позволяет раскрашивать картины в круге либо плести их из ниток на
палочках, простор для импровизации безграничен.

Кроме того, занимаюсь различными видами валяния из шерсти. Этот материал дает
особенно много возможностей, позволяет создавать эксклюзивную одежду, украшая
ее неповторимыми узорами, уникальные картины (это так называемая шерстяная
акварель и гобелены), делать различные украшения – броши, бусы, серьги. 

Есть и много других направлений, я не боюсь осваивать новое. Раньше в луганском
Дворце творчества детей и молодежи вела кружок по вязанию из резиночек. Сейчас
провожу для детей занятия по валянию из шерсти в Луганском Дворце культуры,
работаю и с учащимися нескольких столичных школ.   

К примеру, для воспитанников Луганской специальной (коррекционной)
общеобразовательной школы-интерната для глухих детей организовала кружок по
изготовлению изделий из бумаги (квиллингу). Пришла туда сама, предложила свои
услуги – и ни разу не пожалела. С учащимися младших классов этой школы мы
скатываем шарики или полоски из разноцветной мягкой бумаги и выкладываем из них
различные узоры, создавая маленькие индивидуальные открытки и большие общие
полотна. Работаем с бумагой и в технике папертоль, создавая объемные картины.

В ходе занятий дети не только развивают мелкую моторику, они учатся общаться,
работать в команде, находить общий язык. А самое главное – учатся творить,
потому что каждый создает свое авторское произведение, со своими нюансами, а
значит, может проявить себя.

Я прихожу к этим детям не для того, чтобы зарабатывать деньги. С ними интересно
работать. До начала занятия я продумываю все действия пошагово, чтобы дети
справились. Показываю, объясняю, они очень хорошо воспринимают все глазами,
быстро запоминают. На всякий случай на занятиях присутствует воспитатель,
которая хорошо владеет языком жестов. Но обычно мы друг друга понимаем без ее
помощи.  

Словом, я считаю себя счастливым человеком, потому что мое любимое хобби, то,
чем мне очень нравится заниматься, переросло в профессию.

\subsubsection{ТВОРЧЕСКАЯ ДИНАСТИЯ}

\textbf{– Разделяют ли ваше увлечение близкие?}

– Вместе со мной творчеством занимается моя дочь Полина. Точнее, многим
методикам меня обучила именно она, "подсмотрев" их в детском лагере или
санатории. Так было, например, с квиллингом. Сейчас Полина учится на третьем
курсе Института психологии Луганского государственного педагогического
университета. И мы продолжаем работать вместе в разных техниках.

Мне запомнилась история, которая произошла в день празднования Пасхи в 2015
году. Тогда еще в Республике была сложная ситуация, денег часто не хватало. У
нас с довоенных времен оставался запас шерсти, и мы с дочерью сделали
сувенирные пасхальные яйца, выбив шерстью узоры на пенопласте. В праздник пошли
с Полиной в церковь, чтобы освятить свое весьма скудное праздничное угощение. А
потом остановились возле церкви, чтобы попробовать продать свои работы. Сначала
стеснялись, но там стояли и другие люди, торговали кто чем. Мы не верили, что
кого-то заинтересуют наши изделия, но в итоге распродали все яйца. И на эти
деньги смогли достойно отметить праздник. Кстати, именно тогда возле церкви
наши работы увидела одна из мастериц клуба "Левша" и предложила нам вступить в
него.

Буквально на днях состоялась наша первая с дочерью персональная выставка в
ЛЦНТ, и наши работы из шерсти вызвали большой интерес у ее посетителей.

Работы свои я люблю дарить друзьям и знакомым, иногда продаю. Но главное – учу
других тому, что умею сама. Когда ты что-то имеешь, нужно обязательно делиться
с другими. Если делиться добром, в твоей душе его только прибавляется. Это все
работает, я убедилась на собственном опыте. Я могу что-то делать руками –
значит, я должна научить этому других.

\subsubsection{ПРЕДНОВОГОДНИЕ СОВЕТЫ ДЛЯ "ЧАЙНИКОВ"}

\textbf{– На пороге Новый год, и многие, наверное, хотели бы сделать украшения для дома
и новогодней елки своими руками, но не знают, с чего начать. Что вы посоветуете
тем, кто только мечтает заняться творчеством?}

– Главное – хотеть и не бояться пробовать. Ведь высшая математика всегда
начинается с решения простых уравнений.

Самое простое – это плетение на палочках. Нужно приготовить палочки, в
зависимости от размеров работы это могут быть палочки для шашлыков, шпажки или
зубочистки, а также нитки для вязания (толщина тоже варьируется в соответствии
с размерами будущего изделия). Очень легко сделать, например, снежинку. Три
палочки связываем вместе, потом раскрываем их, получается шесть лучиков – это
основа для плетения. И начинаем обматывать. Сначала укрепляем серединку,
обматывая углы. А потом на эту шестиугольную заготовку наматываем нитки: если
мотать по кругу, получится плотно, а если обмотать каждую палочку, снежинка
будет полупрозрачной. Украсить снежинку можно "дождиком", бусинками или
пайетками.    

По такому же принципу можно сделать елочку. Подключайте фантазию, если
использовать нитки разных цветов, у елки получатся контрастные веточки. Пайетки
можно приклеить на готовую работу или вставить нитку в бусину перед началом
плетения – на елочке получатся новогодние шары. А вверху прикрепите звездочку,
к примеру, из фольги. Вариантов – множество, было бы желание.

Можно сделать новогодние украшения и из шерсти. Скажем, мы с детьми делали
фигурку ангела. Нужно купить немного мягкой шерсти для валяния (так называемый
топс). Выкладываем ее на столе, перевязываем, формируя голову. Отдельно делаем
нимб, из другой пасмы (мера длины пряжи, составляющая часть мотка) – руки, они
вкладываются внутрь, из третьей формируем крылья. Все укрепляется нитками, по
принципу куклы-мотанки. Дальше – опять же работает ваша фантазия. Я называю
такую фигурку "теплый ангел", она действительно греет и руки, и душу. 

Несложно своими руками изготовить и шары для новогодней елки. Необходимо купить
готовые шарики, елочки или фигурки животных из пенопласта и шерсть (кардочес).
А затем шерсть прикладывается к фигурке и специальной иголкой для валяния
вбивается в нее. Узор можно делать любой по вашему выбору, потому что шерсть
разноцветная. У вас получится эксклюзивное елочное украшение.

Легко сделать и новогоднюю открытку в технике квиллинга. Разрезаем разноцветные
салфетки на небольшие одинаковые кусочки и двумя руками скатываем шарики. Эти
разноцветные шарики приклеиваем на любую основу по вашему выбору. Получаются
яркие объемные картинки или открытки.

\textbf{– Какие традиции встречи Нового года существуют в вашей семье?}

– На нашей елке всегда много самодельных игрушек. Любим украшать окна и
зеркала, вырезаем маникюрными ножницами различные фигурки из бумаги – снежинки,
свечки, надписи с пожеланиями. Снежинки из ниток подвешиваем и к потолку. А еще
смешиваем зубную пасту с акварельными красками и раскрашиваем стекла в
межкомнатных дверях. Например, в этом году на двери у нас будет красоваться
большой снеговик.

Мы давно убедились, что украшенный своими руками дом становится особенно
теплым, уютным и праздничным.

Обязательно украшаем своими работами и подъезд. Соседи говорят, что это
поднимает им настроение, они уже ждут наших праздничных фигурок.

\subsubsection{СЧАСТЬЕ – В НАС САМИХ}

\textbf{– Что хотелось бы в преддверии Нового года пожелать жителям Республики?}

– У всякого на пути случаются ямы, провалы, но нужно подниматься и идти дальше.
И какие-то увлечения, безусловно, помогают делать это. Поэтому советую всем:
найдите свое хобби, и вы поймете, насколько оно украшает и облегчает жизнь.

Я желаю всем не отчаиваться, не унывать. Пока мы живем, нужно радоваться тому,
что имеем, тому, что нас окружает.     

И еще: где родился – там и сгодился. Не нужно ехать за тридевять земель в
поисках счастья. Оно – в нас самих. Давайте жить в своих домах, любить людей и
делиться с ними добром. И все у нас будет хорошо! 

\textbf{ЛуганскИнформЦентр — 20 декабря — Луганск}


