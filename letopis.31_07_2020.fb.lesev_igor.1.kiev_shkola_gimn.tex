% vim: keymap=russian-jcukenwin
%%beginhead 
 
%%file 31_07_2020.fb.lesev_igor.1.kiev_shkola_gimn
%%parent 31_07_2020
 
%%url https://www.facebook.com/permalink.php?story_fbid=3411660008865078&id=100000633379839
 
%%author_id lesev_igor
%%date 
 
%%tags gimn,kiev,nacionalizm,obschestvo,patriotizm,shkola,strana,ukraina
%%title О гимнопениях в киевских школах
 
%%endhead 
 
\subsection{О гимнопениях в киевских школах}
\label{sec:31_07_2020.fb.lesev_igor.1.kiev_shkola_gimn}
 
\Purl{https://www.facebook.com/permalink.php?story_fbid=3411660008865078&id=100000633379839}
\ifcmt
 author_begin
   author_id lesev_igor
 author_end
\fi

О гимнопениях в киевских школах

Вроде бы, ну что нас еще может удивить от поклонников «революции достоинства»,
которые свое достоинство свели к наличию вышиванки и уточняющего вопроса «чей
Крым»? Ну еще спивати гимн. Вроде бы, ну а шо тут такова? Каждый рагуль со
стажем скажет, что именно с этого начинается патриотизм.

\ifcmt
  pic https://scontent-frx5-1.xx.fbcdn.net/v/t1.6435-9/116696648_3411659765531769_3155076999246387167_n.jpg?_nc_cat=111&ccb=1-5&_nc_sid=730e14&_nc_ohc=ujgfNOfmQrQAX9UvIc5&_nc_ht=scontent-frx5-1.xx&oh=8e1f901ba11697f34f5b3a91b716dc51&oe=61A44925
  @width 0.8
\fi

В 14-м я с интересом наблюдал, как патриотизм распространялся у меня во дворе.
Начали красить кованый забор в желто-синий цвет. Это было очень уродливо и не в
расцветке дело. Просто люди незараженные достоинством в принципе кованые заборы
не красят. Но на середине краска закончилась и начали красить только шпили
забора. Типа дометили территорию. Ползабора сине-желтая, а вторая половина
черная, но с сине-желтыми верхушками. А ведь еще были прохожие с достоинством,
которые подбадривали красителей.

Затем были перекрашены все лавочки. Тоже в желто-синий цвет. Сидела затем
пацанва с пивом на таких лавочках и проникалась достоинством.

А затем начали выкрашивать в желто-синий цвет мусорные урны. Вот эти круглые из
бетона. Тогда ошалели даже проникнутые достоинством. Потому что скоро на одной
из таких урн я увидел надпись черным мелком «Росiя». Мусорная урна выкрашена в
цвет украинского флага, а поверх нее корявая надпись «Россия». Это было время
открытого портала в инферно.

Из того же портала у нас последние месяцы добывает текущий состав Киевсовета.
БПП-шники, расисты и приезжие из Пьемонта для которых Киев – это такой очень
большой Тернополь. При этом Киевсовете появились проспекты Бандеры и Шухевича.
И при этом же Киевсовете задолго «до того как» в столице были закрыты последние
5 русскоязычных школ.

Поэтому, появление еще и обязательного гимнопения в школах – это не их ноу-хау,
а их недоработка. Эти к 20-му году могли бы уже отдельный экзамен ввести по
выразительному пению гимна, обязать учеников носить красно-черный галстук, а
первоклашек принимать в бандернята. Ну могли же? А не сделали. Мерзавцы.

Но давайте смотреть в завтрашний день вместе с Виталием Кличко. Главный рейдер
нашей столицы и на следующие 5 лет будет избран мэром города. Ну потому что
есть договорняк между «слугами», самим Кличко и порошенковской «Солидарностью».
Они же на троих сколотят в следующем составе Киевсовета коалицию.

Соответственно, и все эти гуманитарные извращения будут продолжены и в
следующей пятилетке. Вот даже этот вопрос с гимном. Постоянное его пение – это
девальвация значимости госсимволов, которые по своей задумке должны быть
торжественно-волнительными. А ежедневное исполнение превращает любое начинание
в рутину, да еще и за счет детей. Введите кретины в госучреждениях обязательное
спивание гимна. В 8.55 все собираются в кабинете у директора – посписочная
проверка – и в добрый путь с песней! Но нет же, всегда трахают наших детей.

Но есть ведь и другой момент. О нем не принято говорить, а он все равно есть.
Украинский гимн давно уже граждан государства Украина не объединяет, а
разъединяет. Невероятно гигантская куча людей выключает радио или переключает
телек, когда его слышат. Это не гимн всех сограждан. Этот гимн давно уже
приватизировали особи вроде тех, которые сидят в Киевсовете. Они пропитали гимн
совершенно другим смысловым наполнением.

Да и как можно петь всем вместе гимн в городе, где одновременно стоит памятник
Ватутину и находится проспект Шухевича? Эта власть гнилая во всех своих
начинаниях. В их устах невинно-логичное «мойте руки» будет по итогу сведено до
лютого срача и взаимной ненависти.

\ii{31_07_2020.fb.lesev_igor.1.kiev_shkola_gimn.cmt}
