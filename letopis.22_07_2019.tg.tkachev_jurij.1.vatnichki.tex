% vim: keymap=russian-jcukenwin
%%beginhead 
 
%%file 22_07_2019.tg.tkachev_jurij.1.vatnichki
%%parent 22_07_2019
 
%%url https://t.me/dadzibao/549
 
%%author_id tkachev_jurij
%%date 
 
%%tags 
%%title Давайте уже воспринимать друг друга так, как положено - не как врагов, а как конкурентов
 
%%endhead 
\subsection{Давайте уже воспринимать друг друга так, как положено - не как врагов, а как конкурентов}
\label{sec:22_07_2019.tg.tkachev_jurij.1.vatnichki}

\Purl{https://t.me/dadzibao/549}
\ifcmt
 author_begin
   author_id tkachev_jurij
 author_end
\fi

Тем временем, любимые мои ватнички продолжают наступать ровно на одни и те же
грабли уже фиг его знает какой год подряд: вместо того, чтобы воевать с врагом,
воюют с единомышленниками. Я, к слову, тоже наступил на эти грабли. Постараюсь
не делать этого в будущем. 

Посмотрите на наших маленьких пушистых друзей. Вон, одна одесская соратница
Порошенко пишет: ну, хорошо хоть \enquote{Голос} прошёл. А ведь \enquote{Голос} - прямой
электоральный конкурент её любимого Порошенко! Но она понимает: если \enquote{Голос}
прошёл, то голоса N процентов симпатиков этой партии не слились в унитаз, а
будут голосовать в Верховной Раде, и голосовать, скорее всего, будут хотя бы в
идеологическом плане плюс-минус за то за что надо. Ей надо.

А у нас? ОПЗЖ говорит, что \enquote{Оппоблок} - проект свинофермы чтобы размыть голоса
и откровенно злорадствует по поводу того, что те не прошли. А в \enquote{Оппоблоке}
крысятся на Шария, который, мол, создан специально для того, чтобы опустить их
ниже проходного плинтуса. Шарий в ответ поливает всех, причём с некоторыми
людьми, насколько я знаю, испортил лично хорошие отношения из-за того, что,
мол, принадлежащие или контролируемые ими СМИ называли его земляным червяком. И
кстати называли.

А в итоге что? В итоге и Шарий, и Оппоблок мимо Рады пролетели. А ОПЗЖ хоть и
прошла, но будет сидеть там в глубокой оппозиции. Если бы ОПЗЖ, в свою очередь,
подарила (я очень грубо говорю) 1\% своего рейтинга Оппо и 2\% своего рейтинга
шариям, то у нас в Раде была бы \enquote{ватная} группировка под 20-25\% голосов. А
конкурировать надо не друг с другом, а с явными недругами - да хоть с тем же
Зеленским, из 40 с гаком процентов которого процентов 10 - всё ещё очарованная
им вата. Вот за эти 10\% и боролись бы. А не спорили о том, кто у кого пытается
стырить голоса.

Вы ж поймите, в политике нет слова \enquote{стырили}. В политике есть слова
\enquote{провтыкал}. И если треть вашего рейтинга можно вот так вот запросто стырить,
то проблемы надо искать прежде всего у себя. И вероятно эти люди не
проголосовали бы за вас даже если бы \enquote{воров рейтинга} не было: они бы или
остались дома, или проголосовали бы за того же Зеленского. Просто вы им не
нравитесь, понимаете? И, вероятно, уже не понравитесь.  Так помогите этим людям
найти того, кто ему понравится - в той же идеологической нише!

Так нет же. Да ещё и по избирателям недругов начинают проходиться: тупые, мол,
примитивные, доверчивые, хайпожоры, быдло... И причём на ровном месте зачастую,
вообще без всяких оснований. Зачем? Чего хотите добиться?

Критиковать друг друга и спорить друг с другом можно. Но - только в своём
уютном кругу. Мы в кругу моих близких знакомых (а там есть сторонники всех трёх
вышеперечисленных политсил) спорим о политике в весьма непарламентских
выражениях. Но нам можно, потому что мы понимаем: как дойдёт до дела, мы всё
равно станем плечом к плечу. \enquote{И за всё, что мы делаем, отвечать будем вместе}.
Так и им бы поступать. Но нет, опять срачи какие-то... 

Вон хоть последняя драма: холвар Шарий-Медведева (Страна). Зачем? Какой смысл?
И Шарий, и коллектив Страны - хорошие, годные журналисты. Делающие, вообще
говоря, одно и то же дело - разоблачающие свиноферму и партию войны как
таковую. Так чего сраться? Та же Медведева в своём блоге могла бы сказать, что,
мол, партия моего коллеги Анатолия Шария пока, к сожалению, судя по опросам
набирает столько и столько, но, глядишь, ещё усилит свои позиции, ведь партия
только на взлёте, да и опросам верить - себя не уважать. А Анатолий, даже и не
скажи она так, мог бы ответить, что, мол, я очень уважаю издание Страна и всё,
что они делают, и потому меня вдвойне расстроило, что они вот так вот обо мне
отзываются. Но я понимаю почему это происходит и прошу соратников не обижаться,
ведь генерально мы всё равно на одной стороне, а это всё - дела временные.
Уверяю вас, имиджево выиграли бы от этого и те и те. А так что? Ничего кроме
смехуёчков для вышиватников.

Давайте уже воспринимать друг друга так, как положено - не как врагов, а как
конкурентов, благодаря честному и этичному соперничеству с которыми мы сами
можем стать лучше.
