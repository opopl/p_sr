% vim: keymap=russian-jcukenwin
%%beginhead 
 
%%file 22_12_2021.fb.fb_group.story_kiev_ua.2.novogodnie_igrushki.cmt
%%parent 22_12_2021.fb.fb_group.story_kiev_ua.2.novogodnie_igrushki
 
%%url 
 
%%author_id 
%%date 
 
%%tags 
%%title 
 
%%endhead 
\zzSecCmt

\begin{itemize} % {
\iusr{Валентина Валентина}

\ifcmt
  ig https://scontent-frx5-2.xx.fbcdn.net/v/t39.1997-6/p370x247/47501240_208073730068277_8669225005453672448_n.png?_nc_cat=1&ccb=1-5&_nc_sid=0572db&_nc_ohc=z06OTD6vq34AX8O59t0&tn=lCYVFeHcTIAFcAzi&_nc_ht=scontent-frx5-2.xx&oh=00_AT_QDCZBHNNECLfftXhbr1JPAh4qoj_9U9Uf8gbSrjZRyg&oe=61CB2625
  @width 0.2
\fi

\iusr{Наталия Озерова}
Пахнет детством! Пирожками и мандаринами!

\iusr{Тамара Ар}

Действительно!, Новые игрушки это модерн, креатив и холодная красота Снежной
королевы! Старые игрушки это теплота, доброта, сказка, волшебство!


\iusr{Катя Бекренева}
Елочные бусы еще были)

\begin{itemize} % {
\iusr{Светлана Манилова}
\textbf{Катя}, у нас есть. @igg{fbicon.smile} 

\iusr{Катя Бекренева}
\textbf{Светлана Манилова} У меня дома, в родительской квартире остались, что со всем этим сделала невестка (жена брата), я не знаю. Я еле выцарапывала кое-какие вещи из своей квартиры, и это в 18 лет) Это отдельная грустная история.

\iusr{Светлана Манилова}
\textbf{Катя}, это бусы еще со времен детства моей свекрови. Ей уже 82. @igg{fbicon.smile} 

\iusr{Катя Бекренева}
\textbf{Светлана Манилова} У нас были ещё дореволюционные украшения.

\iusr{Ирине Вильчинская}
\textbf{Светлана Манилова} у меня уже не сохранились. Они крупные были,стеклянные, дутые. Я, в память о них, купила меленькие стеклянные и набрасываю на ветки...

\iusr{Ольга Писанко}
\textbf{Катя Бекренева} и у нас есть !
\end{itemize} % }

\iusr{Юлия Тертица}
На Подоле сейчас выставка старых игрушек

\begin{itemize} % {
\iusr{Irina Boiko}
\textbf{Юлия Тертица} где, очень интересно

\iusr{Юлия Тертица}
\textbf{Irina Boiko} в Центр української культури і мистецтва, вул. Хорива, 19В

\iusr{Irina Boiko}
Спасибо

\end{itemize} % }

\iusr{Наталья Шевченко}
І в мене є такі іграшки!

\iusr{Светлана Манилова}

Моя первая елочная игрушка- \enquote{колокольчик}, будучи уже старенькой и
облупившейся, всегда занимала почетное место на моей елке. Недавно она совсем
рассыпалась, и без нее стало немножко грустно. Когда ее выбрасывала (тяжело
писать это слово...), аккуратно завернув в салфетку, я будто прощалась со своим
старым другом... \enquote{Мне б туда, где ёлка в вате... Где едва за тридцать папе...
Мама шьёт на праздник платье... скоро Новый год... Где намеренья не лживы и
пока ещё... ВСЕ ЖИВЫ... И чисты души порывы... И она поёт...} @igg{fbicon.heart.red}

\begin{itemize} % {
\iusr{Ирине Вильчинская}
\textbf{Светлана Манилова} 

да, Каждый Новый год - это как встреча с прошлым. Перебираешь игрушки, мишуру,
вспоминаешь, КАК это было ТОГДА, когда вся семья еще была полной, с родителями,
бабушками... Как с мамой наряжали сосенку (большую, под потолок!), как
оборачивали в фольгу орехи с нашего ореха-гиганта, как развешивали конфеты и
мандаринки (самые вкусные потом потихоньку, после праздника, исчезали, а бумажки
еще висели @igg{fbicon.face.wink.tongue} ), как мама добавляла последний штрих - на кончики сосновых лап
цепляла кусочки ваты (СНЕГ!). И как, просыпаясь по утрам, уже после праздника,
увидев нарядную елочку, ты ощущал ПРАЗДНИК! А в последнее время ТОЙ жизни стали
покупать смереки. Теперь не покупаю деревца - только ветки. И игрушки,
соответственно, изменили свой \enquote{формат}. Исчезла огромная малиновая шишка,
которая была впору сосне, но уже совсем не по размеру изящным смерековым
веткам, исчезли большущие часы... Но шары все еще лежат в коробках и каждый раз
я их с любовью перебираю... Они ведь ОТТУДА...

\begin{itemize} % {
\iusr{Светлана Манилова}
\textbf{Ирине}, до слез! Очень созвучно! Спасибо!

\iusr{Ирине Вильчинская}
\textbf{Светлана Манилова} 

А еще я помню ТЕ гирлянды. Самодельные с лампочками двух цветов - зеленого и
малинового. Их сделал мой дядя, мамин брат. Он работал на \enquote{Коммунисте},
радиотехник. Зеленые были выкрашены зеленкой, а малиновые даже не знаю ,чем.
Каждая была \enquote{одета} в пластиковую \enquote{рубашечку} молочного
цвета.... Но гирлянда была старенькая и каждый год выяснялось, что она не
работает уже. И тогда начинался целый ритуал. Я бежала к дяде с просьбой
починить... Он, бурчал (предпраздничная суета, хлопоты - да и своя гирлянда
была в той же кондиции!) - доставал волшебный аппарат (я с благоговением
произносила - \enquote{тестер}!) и начинал \enquote{прозванивать} цепь.
\enquote{Виновница} аварии была вычислена и заменена, наша елка снова
загоралась яркими огоньками, подмигивавшими сквозь блестящую мишуру уютно и
тепло.

\iusr{Светлана Манилова}
\textbf{Ирине}, как все знакомо!

\iusr{Наталья Иванова}
\textbf{Ирине Вильчинская}
Как чудесно Вы написали про ТУ, ДОРОГУЮ СЕРДЦУ, ЖИЗНЬ!!! Все ещё живы и праздник настоящий!!!
 @igg{fbicon.face.blowing.kiss}  @igg{fbicon.evergreen.tree}  @igg{fbicon.snowman.without.snow} 

\iusr{Natasha Levitskaya}
\textbf{Ирине Вильчинская}
Спасибо, Ирина! Прекрасные воспоминания, трогательные!

\end{itemize} % }

\iusr{Наталья Иванова}
\textbf{Светлана Манилова}

Готова подписаться под каждым Вашим словом, особенно это стихотворение - до
слез. А с игрушками у меня то же самое происходит. Когда бьются старые, любимые
игрушки, то не могу выбросить, заворачиваю в мягкую бумагу и оставляю в
коробке. @igg{fbicon.heart.red}

\end{itemize} % }

\iusr{Петр Кузьменко}

\ifcmt
  ig https://scontent-frx5-1.xx.fbcdn.net/v/t39.30808-6/269600985_4851628501556062_4650071641437513348_n.jpg?_nc_cat=111&ccb=1-5&_nc_sid=dbeb18&_nc_ohc=FoYPRsGnvYQAX-yl8TB&_nc_ht=scontent-frx5-1.xx&oh=00_AT8WYDTz1j2FBQlMkJXOJhe1NWRDmYaW-AZxGm9VVtjmZw&oe=61CB3BFD
  @width 0.4
\fi

\iusr{Тома Храповицкая}

Как здорово! Всё вспоминается, таинство украшения именно сосны не ёлки, где
какие шары и сосульки, котики в домиках, мальчик с дудочкой и много других
игрушек ёлочных, сохранившихся из Детства... Они помнят, и каждый год ждут
своего выхода! Каждый год, кто-то из Них бьётся и ощущение как потерял Друга,
оттуда из Нашего Детства...


\iusr{Ольга Писанко}

У нас с братом была игра- мы друг другу загадывали отыскать где какая игрушка
висит...

\begin{itemize} % {
\iusr{Ольга Фролова}
\textbf{Ольга Писанко} Я сейчас так играю со своей внучкой))))))
\end{itemize} % }

\iusr{Катя Бекренева}

А мне вспомнилась елка с 81го на 82ой) Сыну было 8 месяцев, муж купил красивую
сосну, сказал не трогать, а то она плохо укреплена. Приходил он поздно,
вечерний КПИ заканчивал, а мне было невтерпёж, я попробовала вроде нормально,
устойчиво. Я нарядила. Среди ночи мы с мужем проснулись от грохота, спросонья
не можем сообразить в чем дело, потом дошло - елка грохнулась. Самое хорошее,
что ребенок не проснулся), а мы с мужем пылесосили палас среди ночи. На
следующий день сгоняла в Минский универмаг, игрушки покупала)


\iusr{Светлана Александренко}
Замечательные стихи и прекрасный, теплый пост!

\ifcmt
  ig https://scontent-frx5-2.xx.fbcdn.net/v/t39.1997-6/p370x247/47501240_208073730068277_8669225005453672448_n.png?_nc_cat=1&ccb=1-5&_nc_sid=0572db&_nc_ohc=z06OTD6vq34AX8O59t0&tn=lCYVFeHcTIAFcAzi&_nc_ht=scontent-frx5-2.xx&oh=00_AT_QDCZBHNNECLfftXhbr1JPAh4qoj_9U9Uf8gbSrjZRyg&oe=61CB2625
  @width 0.2
\fi

\iusr{Irina Boiko}
Обожаю старые игрушки

\iusr{Лидия Скрипченко}

\ifcmt
  ig https://i2.paste.pics/2470c3b4ac6da09b474c352bef97f5ee.png
  @width 0.2
\fi

\iusr{Катя Бекренева}

Самое интересное, что по игрушкам можно проследить историю) Были снежинки, были
красноармейчики, кукуруза, ракета, велосипедики из бус, колокольчики и они
звенели)


\iusr{Анна Шустерман}

У меня целый ящик старых игрушек, но уже года три я его даже не достаю, ставим
малюсенькую ёлочку и несколько красивых игрушек. А раньше всегда ставили ёлку
до потолка, именно ёлку, папа был эстет и сосна для него была просто деревом, и
запах сосны его тоже не устраивал, и я тоже так привыкла. Украшением всегда
руководил папа, он стоял на стремянке и говорил мне какого размера шарик
подать, потом спускался вниз, оглядывал ёлку со всех сторон и опять рассказывал
мне куда что вешать, чтобы было красиво, и даже когда мы уже вместе не жили, я
старалась украсить именно так, потому что придёт папа и обязательно оценит. А
сейчас уже некому оценить, скучаю... @igg{fbicon.snowflake}

\begin{itemize} % {
\iusr{Irena Visochan}
\textbf{Анна Шустерман} Как я тебя понимаю, дорогая! @igg{fbicon.face.downcast.sweat} 

\iusr{Валентина Валентина}
\textbf{Анна Шустерман} Скучай... Игрушки для детей береги!

\begin{itemize} % {
\iusr{Анна Шустерман}
\textbf{Валентина Валентина} , им не надо, у них свои.

\iusr{Валентина Валентина}
Никогда ничего не знаешь...

\iusr{Natasha Levitskaya}
\textbf{Анна Шустерман}
Вы правы, у них свои... А для нас, это память, ассоциации с счастливыми моментами.
\end{itemize} % }

\iusr{Olena Ivanenko}
\textbf{Анна Шустерман} 

Анечка, у меня тоже самое... не для кого.... 2 коробки старых игрушек, Дед
Мороз - почти мой ровестник, а последние пару лет в ходу несколько ярких
шариков ... так, для надо...

\end{itemize} % }

\iusr{Alla Kenya}
До сих пор берегу, которые не разбились

\iusr{Татьяна Сирота}

Как хорошо, как тепло Вы, Наташа, написали @igg{fbicon.hands.applause.yellow}{repeat=3} 
Мы с мужем уже забыли, когда покупали ёлочные игрушки... Ах, да!

В этом году он купил серебряный шар. Очень ему захотелось. А остальные игрушки
на нашей ёлке все старые. Самые \enquote{молодые} из них, куплены примерно 30-35 лет
назад. А остальным лет по 60 - 65. Можно сказать, наши ровесники...


\end{itemize} % }
