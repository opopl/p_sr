% vim: keymap=russian-jcukenwin
%%beginhead 
 
%%file 24_10_2021.fb.fb_group.obukrainjuvannja.1.google_mova.cmt
%%parent 24_10_2021.fb.fb_group.obukrainjuvannja.1.google_mova
 
%%url 
 
%%author_id 
%%date 
 
%%tags 
%%title 
 
%%endhead 
\subsubsection{Коментарі}

\begin{itemize} % {
\iusr{Пані Тетяна}

Росія загарбує все що бачить, особливо те, що погано лежить. Україна свого часу
програла, вірніше навіть не боролась, війну за цю частину мережі. Ґуґл вважає
Україну двомовною. Це на 80 відсотків вина тих, хто ласий на дурничку, на
відкритий код, на неліцензійний продукт. І -и це одно зі старих пристосувань
пошуку. І це не таємниця уже років десять.

\begin{itemize} % {
\iusr{Бажан Козаченко}

Пані Тетяна: заперечую ваше ствердження про "ласий не дурничку, на відкритий
код". Воно аж ніяк НЕ пов'язано з українськістю, навіть навпаки. Я використовую
в повсякденні і для праці дієвий уклад Лінух, вільні застосунки, використовую і
створюю сирці для вільного використання і воно все є українським. Принаймні
змісцевлено воно аж ніяк НЕ гірше обмеженого закладницького. А ось саме
прихильники обмежень і заробітку, такі як Дрібном'який, Ґуґля з її Титрубом,
Амазон, Яблука поширюють різного роду "відмінити", "задати питання" або взагалі
створюють лише москвамовні оболонки. Згадайте ту ж сварку навколо Ніке та
Нестлі.

Саме що націленість на заробіток штовхає купців до можливих обмежень, бо що
менше мов, то менше витрати на змісцевленння, то більше охоплення покупців
реклямою, НЕ потребують вигадувати місцеві назви. Назвали "Машенькою" для
Московії і продають де лише можуть продати. В Польшу і Чехію вони свою
"Машеньку" НЕ продадуть і це збільшує їхні витрати. Тож саме ознакове, з
обмеженими можливостями поширення є джерелом лиха для нас.

\begin{itemize} % {
\iusr{Пані Тетяна}
\textbf{Бажан Козаченко} 

По-перше, я не про вас. По-друге, говорю проте, що бачила власними очима. Коли
українізувала свою Віндовс, побачила всю ту біду. Не тільки Ґуґл, а й
Майкрософт був наскрізь русифікований. Не буду описувати всю мою боротьбу, але
я добилась свого. Тому не кажіть мені, що все не так.

\iusr{Бажан Козаченко}

Пані Тетяна: то ви хибно висловили свою думку, дарма згадавши про вільні
застосунки і відкриті сирці.

\iusr{Пані Тетяна}
\textbf{Бажан Козаченко} 

Ні, не хибно. Всі ті, хто не знав англійської або лінувався її вчити, були у
рунеті. Я й зараз у деяких щирих патріотів бачу у закладках яндекс або мейлру.

\iusr{Бажан Козаченко}

Пані Тетяна: я НЕ маю подовжувати цю суперечку тут, бо воно є поза навієм цієї
спільноти, але НЕ бачу зв'язку з "ласий не дурничку, на відкритий код" і
"яндекс або мейлру".

\end{itemize} % }

\end{itemize} % }

\iusr{Ron Lov}

нажаль це правда, більше того не показує матеріали польською, румунською і
іншими мовами, а саме московською. цю битву ми програЄмо поки що.

\begin{itemize} % {
\iusr{Бажан Козаченко}
\textbf{Ron Lov}: влучно помітив.
\end{itemize} % }

\iusr{Назар Чорношличник}
Так, це правда. Приниження що подане як подарунок

\iusr{Roman Tymtsunyk}
Не робе

\end{itemize} % }
