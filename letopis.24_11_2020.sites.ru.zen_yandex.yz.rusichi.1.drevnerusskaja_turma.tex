% vim: keymap=russian-jcukenwin
%%beginhead 
 
%%file 24_11_2020.sites.ru.zen_yandex.yz.rusichi.1.drevnerusskaja_turma
%%parent 24_11_2020
 
%%url https://zen.yandex.ru/media/politinteres/kakoi-byla-drevnerusskaia-tiurma-5fb6885f62945b1f7ee9316a
 
%%author Русичи (Яндекс Zen)
%%author_id yz.rusichi
%%author_url 
 
%%tags russia,ancient,jail
%%title Какой была древнерусская тюрьма?
 
%%endhead 
 
\subsection{Какой была древнерусская тюрьма?}
\label{sec:24_11_2020.sites.ru.zen_yandex.yz.rusichi.1.drevnerusskaja_turma}
\Purl{https://zen.yandex.ru/media/politinteres/kakoi-byla-drevnerusskaia-tiurma-5fb6885f62945b1f7ee9316a}
\ifcmt
	author_begin
   author_id yz.rusichi
	author_end
\fi

{\bfseries 
Сбежать из нее было очень трудно
}

\index[rus]{Русь!Слова!поруб}

\ifcmt
  pic https://avatars.mds.yandex.net/get-zen_doc/2783222/pub_5fb6885f62945b1f7ee9316a_5fb78a250a790b7b98f792ba/scale_1200
\fi

Когда мы читаем летописи, то нам периодически встречаются фраза о том, что
кого-то посадили в \textbf{поруб}. Или, наоборот, кого-то оттуда извлекли.

В \textbf{порубе} сидел легендарный Всеслав Полоцкий, когда попал в плен к правящим
Русью братьям Ярославичам - Изяславу, Святославу и Всеволоду. Из поруба же его
достали киевляне во время бунта и усадили на княжеский стол. В порубе сидел
брат Ярослава Мудрого Судислав. Киевский князь заточил его туда по какому-то
неизвестному нам обвинению.

И вообще, \textbf{поруб} фигурирует во многих эпизодах истории Древней руси. Понятно,
что так называлась тюрьма. Но вот как она выглядела, многие себе представляют
совершенно превратно.

В русском языке есть слово \textbf{сруб}, означающее бревенчатое строение. Вроде избы.
Соответственно, проводится аналогия: поруб - это такая изба с решетками на
окнах. В общем, какое-то прочно запертое здание.

Современные тюрьмы, если все упростить, так и устроены - стоит какое-то
сооружение, а в нем помещения для заключенных.

Но древнерусский \textbf{поруб} - это нечто совсем другое.

\ifcmt
  pic https://avatars.mds.yandex.net/get-zen_doc/3892121/pub_5fb6885f62945b1f7ee9316a_5fb78ad40a790b7b98f8d0ea/scale_1200
\fi

В одной из телепередач историк Юрий Артамонов для наглядности приводит цитату
из житийного памятника:

\enquote{Святополк [имеется в виду Святополк Изяславич] всадил бяшеть в поруб двух
мужей, заключив их в тяжкие оковы, за некую вину, не проверив ее, но послушав
ложных обвинителей. Они же, находясь в такой беде, много молили святых
страстотерпцев и каждое воскресенье давали денег стражнику, чтобы он покупал
просфору и носил в церковь святых Бориса и Глеба.

Много миновало времени, а они все пребывали в тоске и печали, молясь
беспрестанно святым и призывая их на помощь. Однажды тюремные двери были
заперты, лестница была вынута и лежала снаружи. Они же, и многие другие
заключенные, спали внутри. И вот один из них внезапно очутился ночью снаружи
темницы, продолжая спать. Когда проснулся он, то увидел себя освобожденным от
оков и, оглядевшись, увидел оковы изломанными и валявшимися около него}.

Из этого описания можно понять, что в поруб можно было попасть только при
помощи лестницы, которую куда-то опускали и откуда-то вынимали. А это значит,
что тюрьма представляла собой яму.

\ifcmt
  pic https://avatars.mds.yandex.net/get-zen_doc/3714606/pub_5fb6885f62945b1f7ee9316a_5fb78ae462945b1f7e1eaa82/scale_1200
\fi

Почему же она тогда называлась \textbf{порубом}? Очевидно, потому что во избежание
оплывания земли, эта яма внутри обшивалась бревнами. Как любой нормальный
колодец, например. Сверху очевидно приделывались горизонтальные двери, которые
можно было еще дополнительно запирать. Либо могло надстраиваться еще какое-то
помещение для стражников. Узники тогда сидели прямо под ними в подземной части.

\begin{leftbar}
	\begingroup
		\em
\enquote{В 1030-е годы вблизи Волосовой улицы было построено довольно сложное
сооружение, которое можно соотнести с порубом (тюрьмой, темницей).
Сохранилась его подземная часть в виде ямы диаметром пять метров и
глубиной три метра}.

Из отчета отряда Новгородской экспедиции Института археологии РАН
	\endgroup
\end{leftbar}

Можно содрогнуться, представив, как люди годами сидят в такой яме. Сбежать из
такой конструкции было очень затруднительно. Тот же Судислав, например, провел
в порубе целых 23 года!
