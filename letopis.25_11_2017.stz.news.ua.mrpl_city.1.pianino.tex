% vim: keymap=russian-jcukenwin
%%beginhead 
 
%%file 25_11_2017.stz.news.ua.mrpl_city.1.pianino
%%parent 25_11_2017
 
%%url https://mrpl.city/blogs/view/pianino
 
%%author_id burov_sergij.mariupol,news.ua.mrpl_city
%%date 
 
%%tags 
%%title Пианино
 
%%endhead 
 
\subsection{Пианино}
\label{sec:25_11_2017.stz.news.ua.mrpl_city.1.pianino}
 
\Purl{https://mrpl.city/blogs/view/pianino}
\ifcmt
 author_begin
   author_id burov_sergij.mariupol,news.ua.mrpl_city
 author_end
\fi

Пианино было старинным, без особых украшений, если не считать двух латунных
подсвечников, выполненных в стиле модерн. По семейным преданиям, будто бы
дедушка выиграл его по лотерейному билету, приобретенному по случаю. Что это
была за лотерея, так и осталось загадкой. Правда, бабушка говаривала, что
инструмент куплен за наличные и заплачено за него сполна, а история с лотереей
придумана, чтобы она не ворчала, поскольку деду было известно, что для нее
такая вещь в хозяйстве пригодиться явно не может, стало быть, и не нужна.
Покупка или выигрыш - это случилось еще до революции. Пианино стояло без
применения все годы мировой, а затем гражданской войны. Лишь трем дочерям,
тогда еще малолетним девочкам, было вменено в обязанность по очереди стирать с
его полированных частей фланелевой сухой тряпкой пыль. Когда кто-нибудь из них
забывал это делать, сразу же ее имя появлялось на пыльной поверхности,
начертанное тонким детским пальчиком.

Полезность пианино обнаружилась в годы НЭПа. Его наличие в съемной квартире для
артистов-гастролеров было существенным преимуществом. Поэтому комната,
сдаваемая квартирантам, редко пустовала в летние месяцы. Вот почему бабушку
«мебель», сверкающая черным лаком, больше  не раздражала. Но у дедушки – сына
крестьянина Полтавской губернии, добившегося тяжелым упорным трудом и природной
сметкой некоего благополучия для своей семьи, были свои планы при обретении
пианино. Он мечтал дать своим дочерям образование и воспитание, чтобы они стали
барышнями не хуже, чем дочери мариупольских купцов и чиновников. Даже успел
отдать старшую дочку в частную женскую гимназию Валентины Епифановны
Остославской. Но заниматься ей там довелось лишь один учебный год. Началась
гражданская война, здание гимназии было сожжено, ее владелица и начальница
выехала из Мариуполя. И продолжить учебу пришлось уже в 3-й трудовой школе,
занимавшей здание бывшего Реального училища В. И. Гиацинтова.

А вот что касается обучения музыке своих чад, то мечта дедушки сбылась, он
нанял для двух младших дочерей учительницу, которая и стала преподавать игру на
фортепиано. А старшая? К тому времени она стала помощницей отца в его
портняжном ремесле, ей было не до разучивания гамм и этюдов Черни, да и пальцы
были исколоты иглой. Две младшие оказались способными ученицами. И довольно
скоро стали демонстрировать гостям и своим друзьям несложные для исполнения
вальсы Шопена, пьесу неизвестного автора под названием \enquote{Мадонна}, а также
модные тогда танго и фокстроты, подобранные на слух. Время шло, дочери одна за
другой вышли замуж, и старое пианино вновь умолкло...

В один из первых дней оккупации нашего города войсками вермахта в дом вошел
пожилой немецкий офицер. Он был невысокого роста, его тучный торс плотно
обтягивал мундир, из-под лакированного козырька его фуражки с высокой тульей
поблескивало пенсне без оправы. Еще в коридоре, увидав дедушку, он, не снимая
черных перчаток, достал из кармана бумажку и прочел, напирая на последний слог
«ко», фамилию хозяина дома, а потом вопросительно посмотрел на дедушку. Тот
кивнул головой. Офицер жестом, не терпящим возражения, показал старику, чтобы
ему уступили дорогу, двинулся в комнаты. Деловито обойдя их, он остановился в
той, где стояло пианино. Приблизившись к нему, немец, прежде всего, начал
листать ноты, вшитые в самодельную темно-сиреневую папку. Вдруг его взгляд
задержался на одной из страниц. Установив папку и сдернув с рук перчатки, он
проиграл найденную им вещь (а это был вальс И. Дунаевского из кинофильма \enquote{Цирк})
до конца. Покидая дом, он несколько раз повторил вполголоса себе под нос:
\enquote{Дунаевски? Дунаевски?} - словно проверяя свою память и не находя ответа.

Через короткий промежуток времени  появились три солдата в мундирах мышиного
цвета, верх рукавов которых был обшит полосками серебристого галуна. Двое из
них, вероятно, еще не достигли двадцатилетия, а третий был постарше. Те, что
помоложе, притащили с собой складные кровати с притороченными к ним свернутыми
постелями и футляры с инструментами. Старший был отягощен лишь футляром с
трубой и своими постельными принадлежностями. Они двинулись в чужое жилье,
никого не спрашивая, видимо, их командир рассказал, куда нужно идти, - прямо в
комнату с пианино. Без остановки пришельцы споро стали готовить места для сна.
Молодые разложили походные лежаки, а старший сбросил со стоявшей в комнате
кровати покрывало, подушку и все остальное на пол, побрызгал матрас какой-то
пахучей жидкостью, расстелил свою постель. После этого крикнул: \enquote{Казяйка,
убери!} - эту фразу он, очевидно, заучил из походного немецко-русского
разговорника.

Пока квартировали незваные постояльцы (а это были, по всем признакам, музыканты
полкового оркестра), жизнь в доме превратилась в звуковой ад. Они с утра до
вечера дули в свои трубы, одновременно разучивая каждый свою партию. Какофония
прекращалась только на короткое время, когда эта троица со своими котелками
уходила за едой и когда они, возвратившись, уписывали свой суп и гуляш с кашей,
обильно сдобренные пряностями. Кстати, эти запахи раздражали полуголодных
обитателей дома не менее, чем трубные звуки. Вечером двое молодых резались в
карты, а старший бренчал на пианино. Покой воцарялся лишь раз в неделю, правда,
не на весь день. В субботу музыканты натирали свои трубы до блеска, с особой
тщательностью чистили свое обмундирование и сапоги. К концу дня у них была
репетиция -  теперь они играли все вместе одни и те же мелодии, повторяя их по
нескольку раз. На следующий день их можно было увидеть в Городском саду.
Оркестр, в котором они служили, становился в круг и играл военные марши,
изредка перебивая их вальсами и польками...  Постояльцы исчезли внезапно. Еще
утром разучивали свои партии, а вернувшись с обеда, быстро скатали в тючки свои
постели, сложили кровати и ушли, оставив на память несколько прерывистых
нестираемых кружков на полированной крышке пианино – это были следы их
бритвенных приборов...

Однажды (это было вскоре после освобождения города от оккупантов), средняя дочь
привела в дом свою бывшую учительницу и ее сестру, также учительницу. Они
остались без крыши над головой. Их дом на Итальянской улице – наследство отца –
священника, высланного еще в начале 30-х годов на Соловки, где он и сгинул, -
сожгли немецкие факельщики. Старушки  неприкаянно стояли, не выпуская из рук
узлы со спасенными от огня пожитками. \enquote{Мы будем платить...} - полушепотом
произнесла старшая из них. В ответ дедушка бросил: \enquote{Живите так}, - и пошел к
своему \enquote{Зингеру}. Постоялицы постепенно освоились и стали как бы членами семьи.
Одна из них устроилась преподавать в кулинарное училище. Другая занялась
репетиторством -  подготавливала отстающих детей к переэкзаменовкам по русскому
и французскому языкам, а учеников младших классов – и по арифметике. Как-то
вечером, когда после ужина все сидели за столом, репетиторша промолвила,
обращаясь к дедушке: \enquote{Мы вам очень признательны за ваши благодеяния, чтобы
как-то отблагодарить вас, я готова учить вашего внука музыке...  Только пианино
нужно настроить}.  Дедушка, не поднимая глаз, произнес: \enquote{Пусть учится. Меньше
будет болтаться без дела}.

Пришел настройщик. Он был олицетворением черного цвета. Все у него было черным:
и длинные лоснящиеся волосы, и черточка усов под носом, и потрепанный
двубортный пиджак с завернутым наружу лацканами, и штаны с оттянутыми на
коленях брючинами, и измятый портфель с приспособлениями его ремесла. Лишь
желтоватый цвет худощавого лица, профиль которого напоминал хищную птицу,
выпадал из общей гармонии его образа. При всей мрачности внешнего облика, он
оказался человеком веселым, даже озорным. Он открыл крышку пианино, несколько
раз прочел вслух надпись на фирменной пластинке, прикрепленной над клавиатурой:
\enquote{Adolf Horn}, \enquote{Adolf Horn}, \enquote{Adolf Horn}, и, повернувшись к внуку, показал
язык. Затем, придвинув дубовый стул, присел, нажал правой ногой педаль, с
остервенением бросил пальцы на клавиши и с оглушительным грохотом пробежался по
ним, извлекая несколько бравурных пассажей. Только после этого была снята
передняя панель, обнажившая струны, колки, молоточки и прочие внутренности. И
началась работа. Настройщик извлек из своего камертона звук \enquote{ля} и стал
подтягивать или отпускать струну за струной... 

После настройки пианино превратилось для внука в орудие пытки.
Преподавательница отнеслась к исполнению добровольно взятых на себя
обязанностей со всей ответственностью. Два дня в неделю по два часа  ученик под
ее руководством оттачивал исполнение гамм в разных тональностях, сходящиеся и
расходящиеся, разучивал этюды из \enquote{Школы игры на фортепиано} Берио и
потрепанного сборника \enquote{Petit pianist}. Над теми  нотами, где не была обозначена
аппликатура, т.е. порядок чередования пальцев при игре, учительница сама
надписывала красным карандашом номера пальцев. Если ученик нарушал этот
порядок, наставница пребольно била по \enquote{провинившемуся} пальцу. Вероятно, она
считала, что такой метод является кратчайшим путем в воспитании будущего
виртуоза. Весной парнишка заболел скарлатиной, на том и прекратилось его
музыкальное образование. Итогом усилий педагога и мучений ее питомца стал
выученный на память простенький вальс из оперы Ш. Гуно \enquote{Ромео и Джульетта}. А
пианино вновь замолчало на многие годы...
