%%beginhead 
 
%%file 30_04_2020.fb.prisjazhnjuk_oleg.mariupol.dnipro.1.shedevry_pivo_mozaika
%%parent 30_04_2020
 
%%url https://www.facebook.com/oupris/posts/pfbid04s83hcrGtF5syUwDt8QTCVRc4eozf1tNtTGpZSc6Tic7y76d3NJQLLpVhjbn4pvSl
 
%%author_id prisjazhnjuk_oleg.mariupol.dnipro
%%date 30_04_2020
 
%%tags mariupol,mariupol.pre_war,beer,mozaika
%%title Для того щоб створювати такі шедеври треба понад усе любити не тільки рідне місто, але ще й пиво
 
%%endhead 

\subsection{Для того щоб створювати такі шедеври треба понад усе любити не тільки рідне місто, але ще й пиво}
\label{sec:30_04_2020.fb.prisjazhnjuk_oleg.mariupol.dnipro.1.shedevry_pivo_mozaika}

\Purl{https://www.facebook.com/oupris/posts/pfbid04s83hcrGtF5syUwDt8QTCVRc4eozf1tNtTGpZSc6Tic7y76d3NJQLLpVhjbn4pvSl}
\ifcmt
 author_begin
   author_id prisjazhnjuk_oleg.mariupol.dnipro
 author_end
\fi

Для того щоб створювати такі шедеври треба понад усе любити не тільки рідне
місто, але ще й пиво.  З метою виготовлення лише тільки одного фрагменту
мозаїчного панно, яке розташоване на передньому плані: МАРИУПОЛЬ 1778 +
ЯКОРЬОК, невідомі митці-патріоти міста Марії були змушені відкоркувати та вжити
преорально, за моїм приблизним підрахунком, щонайменше 550 пляшок світлого,
темного та нефільтрованого пива 😲 А на задньому плані ще ж є штурвал, вежа та
сонечко :))))))

%\ii{30_04_2020.fb.prisjazhnjuk_oleg.mariupol.dnipro.1.shedevry_pivo_mozaika.cmt}
