% vim: keymap=russian-jcukenwin
%%beginhead 
 
%%file 23_12_2021.fb.fb_group.story_kiev_ua.2.general_sinkler.cmt
%%parent 23_12_2021.fb.fb_group.story_kiev_ua.2.general_sinkler
 
%%url 
 
%%author_id 
%%date 
 
%%tags 
%%title 
 
%%endhead 
\zzSecCmt

\begin{itemize} % {
\iusr{Петр Кузьменко}
Справжній чоловік, офіцер, патріот, приклад для наслідування!

\iusr{Ольга Хоботня}
Відповідь на питання сьогодення....
Звідки візьмуться Патріоти?
Винищили всіх під коріння..

\begin{itemize} % {
\iusr{Lyubov Pakholchenko}
\textbf{Genady Goldstein} Чтобы быть патриотом, нужно изначально, от рождения иметь такие черты характера, как высоко нравственность, самоотверженность и честность.

\iusr{Галина Полякова}
\textbf{Ольга Хоботня} На щастя патріоти є. І їх в нас багато.
\end{itemize} % }

\iusr{Світлана Куликова}

\ifcmt
  ig https://i2.paste.pics/725517d7ca89ff7e24797d8347a77cd5.png
  @width 0.2
\fi

\iusr{лариса погосова}
А чего было ещё ждать от этой власти?
Сдали эмигрантов, живущих в Чехословакии...
Союзнички...

\iusr{Maksym Oleynikov}

В. С. Сінклер помер у в'язничній лікарні на руках у колишнього підполковника
Армії УНР В. Проходи. Василь Прохода, як і Сінклер, був арештований СМЕРШем у
1945р. в Польщі, пробув 11 років у таборах Воркути і Комі АРСР...

\begin{itemize} % {
\iusr{Владимир Каледин}
\textbf{Maksym Oleynikov} А чего вы хотели! Они были врагами советской власти!

\iusr{Maksym Oleynikov}
\textbf{Владимир Каледин} 

Я, шановний, не хотів НІЧОГО. Просто додав в коментарях факт про цю людину.
Певне, щось хотіли Ви.... @igg{fbicon.face.savoring.food} 

\iusr{Maksym Oleynikov}
\textbf{Юрий Спичак} Звичка. Недоліки виховання... @igg{fbicon.face.savoring.food} 
\end{itemize} % }

\iusr{Вадим Горбов}
Про Тютюнника будет пост?

\begin{itemize} % {
\iusr{Галина Полякова}
\textbf{Вадим Горбов} 

Есть миниатюра о Василе Тютюннике. Но он, как и Отмарштайн, не был киевлянином
и потому такой материал не отвечает условиям группы.

\begin{itemize} % {
\iusr{Вадим Горбов}
\textbf{Галина Полякова} не герой?

\iusr{Галина Полякова}
\textbf{Вадим Горбов} И герой, и умница. Но Вы, как модератор, можете принять решение. Не я. Отмарштайн тоже крайне интересен, но ведь не была одобрена миниатюра о нем.

\iusr{Вадим Горбов}
\textbf{Галина Полякова} Была не моя смена. Я дежурю здесь сутки через трое.

\iusr{Галина Полякова}
\textbf{Вадим Горбов} Мне бы не хотелось сеять смуту в вашей команде. Примите решение и дайте знать. Я с удовольствием скину тексты и фото.

\iusr{Галина Полякова}

Однозначно и Василь Тютюнник и Отмарштайн жили в Киеве, когда служили в
генштабе. Тютюнник ругался с Петлюрой именно в Киеве. а Отмарштайн создавал
контраразведку тоже в Киеве.

\end{itemize} % }

\iusr{Олег Курилов}
\textbf{Вадим Горбов} 

Василия или Юрия хотите?  @igg{fbicon.smile}  (Мне как то Рагоза и Келлер ближе
@igg{fbicon.smile}, из тех кто связан с Киевом, это действительно толковые
генералы и личности )

\end{itemize} % }

\iusr{Андрей Чекховской}
Прислужник Петлюры получил по заслугам.

\iusr{Галина Полякова}
\textbf{Андрей Чекховской} Синклер служил не Петлюре, а Украине

\iusr{Volodymyr Nekrasov}
Дякую за публікацію!

\iusr{Наталія Соколик}
Дякую за публікацію @igg{fbicon.face.happy.two.hands} 

\iusr{Anatolii Motin}
Возможно напишу не
То, но считаю что Офицер принимает присягу лишь Раз, или второй раз когда с него сняли присягу...

\iusr{Anatoliye Anatoliy}
Full respect

\iusr{Вячеслав Анохин}
Достойный человек и выдающийся военный .

\iusr{Юрий Спичак}
Как много всё таки в группе мерзкой ваты!

\begin{itemize} % {
\iusr{Галина Полякова}
\textbf{Юрий Спичак} И это очень печально!

\iusr{Denis Kostjuk}
полностью согласен
\end{itemize} % }

\iusr{Олег Курилов}

\enquote{Смертний вирок.} Это точно? Он же не дожил до приговора суда! Вы не
ошибаетесь? То есть у него не было решения суда! Он умер раньше. И военая
прокуратура КВО требовала для него 10 лет, а не \enquote{высшую меру}! Вот информация:

\enquote{Предлагалось применить высшую меру наказания. Военная прокуратура КВО
\enquote{смилостивилась} до 10 лет лагерей.
Ожидавшего суда 67-летнего генерала 12 марта 1946 года перевели в тюремный
лазарет \enquote{по поводу частых приступов грудной жабы}. Увы, силы оставляли больного
старика... Смерть наступила от инфаркта 16 марта в 7 часов 30 минут.}

\begin{itemize} % {
\iusr{Denis Kostjuk}
\textbf{Олег Курилов} пруф есть?

\begin{itemize} % {
\iusr{Олег Курилов}
\textbf{Denis Kostjuk} 

Да, я выложил цитату нашего исследователя. Кроме того, нет в его биографии
этого факта про какую то высшую меру не на одном сайте про него.

\iusr{Denis Kostjuk}
\textbf{Олег Курилов} Хотелось бы подробнее почитать от первоисточника

\iusr{Олег Курилов}
\textbf{Denis Kostjuk} 

Ссылки сдесь нельзя бросать. Найдите по поису по этой цитате. А что вас
смущает? Источник явно очень симпатизирует этому персонажу. Какие могут быть
вопросы. Кроме того в те годы за участие в гражданке Вышку не давали.
Посмотрите подобные приговоры его \enquote{коллег}.

 · Ответить · Поделиться · 2 д. · Отредактировано
\emph{Denis Kostjuk}
\textbf{Олег Курилов} Я просто хочу почитать про этого человека, а там сделаю выводы, честно я не читал сам пост полностью.
 · Ответить · Поделиться · 2 д.
\emph{Олег Курилов}
\textbf{Denis Kostjuk} Если хотите и вам интересен персонаж могу вам бросить в личку.
 · Ответить · Поделиться · 2 д.
\emph{Denis Kostjuk}
\textbf{Олег Курилов} Спасибо 🙂 хочу
 · Ответить · Поделиться · 2 д.
\emph{Олег Курилов}
\textbf{Denis Kostjuk} Хорошо, сейчас поищу
 · Ответить · Поделиться · 2 д.
\emph{Олег Курилов}
\textbf{Denis Kostjuk} ??? увидели? всё нормально. афтор статьи довольно серьёзный исследователь доктор исторических наук и профессор 🙂.
 · Ответить · Поделиться · 2 д.
\emph{Denis Kostjuk}
\textbf{Олег Курилов} окончание автор поста отписал по своему, везде пшется что умер по болезни сердца. Хотя в то время мусора могли и пытать, с националстами особенно Украинцами чнили жестко и таки да грозило 10 лет тюрьмы судя з друнх источников.
 · Ответить · Поделиться · 2 д.
Олег Курилов
Denis Kostjuk Я вам специально указал на большую симпатию исследователя к националистам! "Одиссея Василия Кука. Военно-политический портрет последнего командующего УПА" Он автор подобных исследований. По этому заподозрить автора в симпатии к прокуратуре КВО очень тяжело.
 · Ответить · Поделиться · 2 д.
Denis Kostjuk
Олег Курилов В принципе Щас много источников просмотрел, везде одно и тоже. Автора не осуждаю, но факты бографи лучше не исправлять.
\end{itemize} % }

\end{itemize} % }

\end{itemize} % }
