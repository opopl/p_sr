% vim: keymap=russian-jcukenwin
%%beginhead 
 
%%file 22_07_2017.stz.news.ua.mrpl_city.1.shalosti_prikazchikov
%%parent 22_07_2017
 
%%url https://mrpl.city/blogs/view/shalosti-prikazchikov
 
%%author_id burov_sergij.mariupol,news.ua.mrpl_city
%%date 
 
%%tags 
%%title "Шалости" приказчиков
 
%%endhead 
 
\subsection{\enquote{Шалости} приказчиков}
\label{sec:22_07_2017.stz.news.ua.mrpl_city.1.shalosti_prikazchikov}
 
\Purl{https://mrpl.city/blogs/view/shalosti-prikazchikov}
\ifcmt
 author_begin
   author_id burov_sergij.mariupol,news.ua.mrpl_city
 author_end
\fi

В эпоху, отдаленную от  нашего времени более чем столетием, в Мариуполе было
немало довольно крупных магазинов, в которых торговали одеждой: дамской,
мужской и детской. Располагались они главным образом на Екатерининской улице
(теперь это проспект Мира) и частично на Торговой, ближе к главной  магистрали
города. В этих магазинах покупателями были люди состоятельные. Для них костюмы,
пальто, сюртуки, меховые дамские манто, платья по последней моде завозились из
Москвы, Петербурга, Одессы, Варшавы, Лодзи и Вильно. А по особым заказам -  и
из Парижа, но это для уж очень богатых людей.

\ii{22_07_2017.stz.news.ua.mrpl_city.1.shalosti_prikazchikov.pic.1}

Владельцы таких торговых заведений дорожили своей репутацией, а потому не
подсовывали своим клиентам товар сомнительного качества: траченный молью,
сшитый гнилыми нитками и, упаси боже, с пятнами и прорехами, даже если
находились они и не на видном месте. Весть о скандале по такому случаю,
учиненном какой-нибудь дамой или господином, мгновенно разнеслась бы по всему
Мариуполю — городок-то был небольшой. А это, естественно, грозило неминуемым
оттоком покупателей к конкурентам, что было равносильно финансовой катастрофе.

\ii{22_07_2017.stz.news.ua.mrpl_city.1.shalosti_prikazchikov.pic.2}

Но кроме крупных и, повторим, дорогих магазинов были небольшие лавчонки, где
можно было купить одежду попроще и подешевле. Находились они на базаре. Для
представителей поколений, родившихся в пятидесятые и последующие годы минувшего
столетия, поясним, что базар занимал площадь, где нынче возвышается здание
бывшего ДОСААФ и территорию за ним - аж до Энергосбыта. Посетителями и
покупателями лавчонок были рабочий люд, чиновная и торговая мелкота, крестьяне
из близлежащих сел и хуторов.  Товар для этих торговых заведений шили местные
портные из хозяйского материала. В погоне за барышами хозяева для пошива
покупали мануфактуру с брачком. Обмануть покупателя было чуть ли не шиком.
Случалось, что уже на готовой одежде обнаруживалась на самом неподходящем месте
дыра. Скажем, на воротнике или где-нибудь на локте. Тогда предприимчивый
лавочник пришивал пуговицу так, чтобы она скрывала дефект.

\ii{22_07_2017.stz.news.ua.mrpl_city.1.shalosti_prikazchikov.pic.3}

Позже, уже во время торга, можно было увидеть такую сценку:

- А шо це за гудзик? — спрашивал удивленный потенциальный обладатель пальто
или, может быть, сюртука.

- Этот? Вот негодные дети. Подумайте только! Уже успели пришить пуговицу. Ну,
ничего, — заискивающим тоном говорил приказчик, — домой, в село, вернетесь —
отрежете лишнюю пуговицу, про запас будет. Селянин из Новоспасовки,
Стародубовки или Сартаны, а может быть, из другого поселения успокаивался,
доставал из укромного места деньги, только что вырученные за привезенную
пшеницу или живность, расплачивался, а ему торжественно вручали тщательно
упакованную вещь. Ее обладатель сопровождался до самой двери приказчиком. Был
тот -  сама любезность...

При покупках правилом хорошего тона было принято торговаться. В этом
заключалась не столько возможная выгода, сколько своеобразный спорт. Кто кого.
Некоторые оборотистые торговцы обращали эту традицию в свою пользу. Чтобы
объегорить покупателя, они разыгрывали примерно такую сценку. В разгар торга в
лавку входил некий господин. Прислушавшись к разговору, он извлекал из кармана
портмоне и принимался рыться в нем, обращаясь к приказчику:

- Я беру эту вещь за ту цену, с которой не согласен ваш покупатель...

Несогласный покупатель спохватывался:

- Накидываю еще рубль, и товар – за мной. Приказчик покорно заворачивал
покупку. \enquote{Победитель} торга покидал лавку. Заметим, что тогда за рубль на этом
же базаре можно было купить увесистую курицу.

На самом деле соперник обладателя покупки был соучастником приказчика. И
действо такое называлось \enquote{подгон}.

Для этой аферы годились как крестьяне, так и горожане. Во время примерки
облаченного в новый наряд покупателя приказчики вертели туда и сюда перед
зеркалом, осыпая похвалами, как великолепно сидит одежда на нем. Незаметно в
карман брюк или жилета подсовывали царской чеканки золотую \enquote{пятерку}. Когда
ошарашенный от повышенного внимания к его персоне покупатель доходил до нужной
кондиции, предлагалось засунуть руки в карманы:

- Вы попробуйте, какие глубокие карманы, уверяю вас, из них ничего не выпадет,
а холстина какая крепкая — выдержит десять лет и не порвется, туда тысячу
рублей золотом засыпь — и карман выдержит, дай бог, чтобы с вами такое
случилось пораньше. Наивный человек исследовал карман за карманом и вдруг...
нащупывал монету.

- И сколько стоит ваш товар?

Тут цена заламывалась более чем приличная. Покупатель, покрывшись холодным
потом, начинал считать в уме. Выходило, что даже при такой трате, с учетом
золотого, костюм достанется почти даром.

- Беру.  

Пока покупатель раздевался и расплачивался, ему заговаривали зубы. Наконец,
вручалась покупка, завернутая в прочную бумагу. Естественно, что монета во
время упаковки извлекалась ловкими пальцами приказчика. Лишь дома
обнаруживалась хитрость. Но кому жаловаться? \enquote{Пятерка} то принадлежала, что там
ни говори, лавочнику...
