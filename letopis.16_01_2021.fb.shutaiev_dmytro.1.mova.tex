% vim: keymap=russian-jcukenwin
%%beginhead 
 
%%file 16_01_2021.fb.shutaiev_dmytro.1.mova
%%parent 16_01_2021
 
%%url https://www.facebook.com/DmytroShutaiev/posts/1302344916808520
 
%%author 
%%author_id 
%%author_url 
 
%%tags 
%%title 
 
%%endhead 

\subsection{Різних офіціанток, барменів, касирів можна і треба задовбувати, щоб вони спілкувалися українською}
\Purl{https://www.facebook.com/DmytroShutaiev/posts/1302344916808520}
\ifcmt
  author_begin
   author_id shutaiev_dmytro
  author_end
\fi

З сьогоднішнього дня вступають в силу законні норми, по яких працівники сфери
обслуговування зобов’язані перейти на українську мову при спілкуванні з
клієнтами. І це чудово, такі правила мали б діяти ще з перших днів
незалежності. Різних офіціанток, барменів, касирів можна і треба задовбувати,
щоб вони спілкувалися українською. Десь в інтернеті є інструкція від мовного
омбудсмена як це правильно і законно робити. Право говорити українською мовою в
українські державі – це святе.  

\ifcmt
  pic https://scontent-mad1-1.xx.fbcdn.net/v/t1.0-9/139592264_1302344770141868_6327215527631075389_n.jpg?_nc_cat=105&ccb=2&_nc_sid=730e14&_nc_ohc=MoKkFAuiTTUAX-eajIZ&_nc_ht=scontent-mad1-1.xx&oh=d7c6616ce9b1748f52bbd1aaaf777eef&oe=6027FDF6
  width 0.4
\fi

Але питання в іншому. Як бути з мігрантами, які ледь спілкуються навіть
російською? Наприклад в таксі чи в службах доставки працює чимало іноземців,
які часто окрім «дякую» і «до побачення» українською нічого не знають. З одного
боку – вони теж сфера обслуговування і зобов’язані знати мову держави, в якій
працюють. З іншого – ці люди приїхали в нашу країну чесно працювати і заробляти
гроші, вони народилися і жили десь в Африці чи на Близькому Сході та вочевидь
не мали можливості вивчити українську.  

Може компаніям типу Bolt або Glovo варто проводити курси української мови для
своїх працівників-іноземців?
