% vim: keymap=russian-jcukenwin
%%beginhead 
 
%%file 02_08_2021.fb.bryhar_sergej.1.lvov_jazyk
%%parent 02_08_2021
 
%%url https://www.facebook.com/serhiibryhar/posts/1755251424674802
 
%%author Бригар, Сергей
%%author_id bryhar_sergej
%%author_url 
 
%%tags jazyk,lvov,mova,ukraina,ukrainizacia
%%title Вони тримають свої рубежі! Вони хочуть перевести нас на російську.
 
%%endhead 
 
\subsection{Вони тримають свої рубежі! Вони хочуть перевести нас на російську.}
\label{sec:02_08_2021.fb.bryhar_sergej.1.lvov_jazyk}
 
\Purl{https://www.facebook.com/serhiibryhar/posts/1755251424674802}
\ifcmt
 author_begin
   author_id bryhar_sergej
 author_end
\fi

Усім, хто не впізнає, розповідаю: це торговий комплекс "Шувар", місто Львів. 

Я не був у цих місцях майже два роки. І от, нещодавно мав там зустріч. Прийшов
раніше, і загалом провів у тій локації близько години. 

І... Апелювати відсотковими показниками мені складно, до того ж, я б не хотів
спрямовувати дискусію в цю площину, тому скажу так: мене вразила кількість
людей - більшості з яких на вигляд до 40 років, - що розмовляли російською. Це
не була якась група. Просто собі люди. Приходили, йшли. Розмовляли між собою,
по телефону, з дівчиною, що продає морозиво. Деякі - з малими дітьми. Я все
розумію, але це був явний перебір.

\ifcmt
  pic https://scontent-cdg2-1.xx.fbcdn.net/v/t39.30808-6/226681472_1755247361341875_1597990764605499206_n.jpg?_nc_cat=100&ccb=1-3&_nc_sid=8bfeb9&_nc_ohc=i4vp0G827v4AX-G8qwM&_nc_ht=scontent-cdg2-1.xx&oh=d10f02c0fda4dc492fc8b3e155c655a4&oe=610FE774
  width 0.4
\fi

Які там, в біса, на Сихові туристи (принаймні як масове явище)? Поблизу
працюють чи мешкають багато айтішників з "востока і юга" (чи й Білорусі - теж)?
Можливо. Але то вже таке... Говорю лише про факт.

Найпомітніше в цьому контексті те, про що я вже неодноразово розповідав:
україномовна більшість це явище толерує. Пересічні львів'яни сприймають
російську як абсолютну норму. І це не про туристичний центр - так усюди.

В моєму місті - хто не в темі, спробуйте повірити мені на слово, але за першої
ж нагоди поспостерігайте самі - не так. Тут русофони особливою толерантністю не
відрізняються. "Ну ладно єщьо турісти - шо с етой западенщіни взять", - думають
вони. А от якщо ти мешкаєш тут, в Одесі, постійно, і продовжуєш розмовляти
українською, наперекір обставинам, жертвуючи власною зручністю, а часто й
практичними перспективами, матимеш зверху хоч і переважно неагресивний, але
постійний тиск: здивування, поради на зразок "учітє русскій - бєз нєго нікак",
часом і якісь нападки та навіть співчуття.

Вони тримають свої рубежі! Вони хочуть перевести нас на російську. 

(Я говорю не про всіх людей, які розмовляють російською (серед них багато
українців, які просто підлаштуватися... які все розуміють, але зробили вибір на
користь зручності та комфорту), а скоріше про загальну атмосферу)

Перепрошую за зовсім не нову думку, але вже як є: "Нам, українофонам,
україномовним українцям, що мешкають на своїй землі, необхідно тримати рубежі!
Особливо стійко - там, де нас більшість"!

\ii{02_08_2021.fb.bryhar_sergej.1.lvov_jazyk.cmt}
