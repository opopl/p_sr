% vim: keymap=russian-jcukenwin
%%beginhead 
 
%%file 28_11_2020.news.world.cbc_ca.brown_chris.1.russia_vaccine_effectiveness
%%parent 28_11_2020
 
%%url https://www.cbc.ca/news/world/russia-vaccine-covid-19-coronavirus-chris-brown-1.5819331
 
%%author Brown, Chris
%%author_id brown_chris
%%author_url 
 
%%tags 
%%title Russia says its COVID vaccine is 95% effective. So why is there still Western resistance to it?
 
%%endhead 
 
\subsection{Russia says its COVID vaccine is 95\% effective. So why is there still Western resistance to it?}
\label{sec:28_11_2020.news.world.cbc_ca.brown_chris.1.russia_vaccine_effectiveness}
\Purl{https://www.cbc.ca/news/world/russia-vaccine-covid-19-coronavirus-chris-brown-1.5819331}
\ifcmt
  author_begin
   author_id brown_chris
  author_end
\fi

\index[rus]{Коронавирус!Вакцина!Спутник V, эффективность, 28.11.2020}

\begin{leftbar}
\bfseries
Despite issues about transparency, Russia's vaccine appears to work well, say experts
\end{leftbar}

\ifcmt
pic https://i.cbc.ca/1.5819349.1606505140!/fileImage/httpImage/image.JPG_gen/derivatives/16x9_780/health-coronavirus-russia-vaccine.JPG
caption A health worker receives Russia's Sputnik V vaccine in Tver, Russia, on Oct. 12. The vaccine has been met with skepticism by much of the Western world, but experts increasingly say it probably works. (Tatyana Makeyeva/Reuters)
\fi

When Russia announced this week that its much-hyped COVID-19 vaccine was up to
95 per cent effective, the news was met with a predictable cheer in Russia and
uncertainty throughout much of Europe and North America.

\enquote{This is great news for Russia and great news for the world,} gushed Kirill
Dmitriev, head of the Russian Direct Investment Fund, which is pouring
countless millions of Russian tax dollars into developing the vaccine it
labelled early on as Sputnik V.

Whether the V stands for \enquote{five} or simply the letter \enquote{V} has
never been fully explained, but either way the association is obvious: the
original Sputnik satellite won Russia the space race more than 60 years ago,
and this new Sputnik will make Russia first in this new race to defeat the
pandemic.

With its hyperbolic announcements and an ambitious — some would say
unattainable — timetable, the Putin government has attempted to demonstrate the
development of Sputnik V has made Russia a vaccine superpower.

Already, among the reputed \enquote{firsts} Russia is claiming: the first COVID vaccine
in the world to be registered; the first vaccine anywhere to be announced as
part of a national vaccination campaign; and trial results that rank it first
in terms of effectiveness.

Not to be outdone, when Pfizer and BioNTech became the first Western vaccine
maker to announce promising results, with 90 per cent efficacy, days later
Sputnik V's makers said their vaccine was even better — by two percentage
points.    

Some Western experts have felt the lack of transparency about the approval
process, combined with a rush to get it registered even before trials started,
damaged the vaccine's credibility from the outset.

Russia licensed it based on early trials involving only 76 people, whereas
usually most approvals come after Phase 3 studies involving tens of thousands
of subjects.     

Even after those early results were published in the reputable medical journal
The Lancet, a group of 37 scientists from 12 countries wrote the publication
questioning the data.

Russian officials and state media pundits have decried the skepticism as
evidence of an inherent \enquote{Western bias} against anything Russian and have
accused U.S. and U.K. media of staging a smear campaign to steal away potential
international customers. 

\subsubsection{Positive results}

Fast forward to this week and news from Sputnik's developer, the Gamaleya
Research Institute of Epidemiology and Microbiology, of very positive results
from a much larger data sample.   

Russia's vaccine has 91.4 per cent efficacy from an analysis of more than
18,000 people, said a release on the Sputnik V website. The vaccine's efficacy
rose to 95 per cent after 42 days.     

Plus, at roughly \$20 US per person, Gamalyea says the Russian vaccine is one of
the cheapest on the market, making it an attractive option for poorer countries
with large populations.

Like the vaccine developed by Oxford University and its partner AstraZeneca,
the Russian vaccine uses human adenovirus vectors, or common cold genes, to
trigger an immune response in the body. An initial shot is followed by a
booster three weeks later.

\ifcmt
pic https://i.cbc.ca/1.5819350.1606505308!/fileImage/httpImage/image.JPG_gen/derivatives/original_1180/health-coronavirus-eu-hungary.JPG
caption Russia's Sputnik V vaccine arrives at Ferenc Liszt International Airport in Budapest, Hungary, on Nov. 19. (Matyas Borsos/Hungarian Foreign Ministry/Reuters)
\fi

The news about the results prompted a change in tone from many Western vaccine
experts. 

\enquote{The data [is] compatible with the vaccine being reasonably effective,} said
Stephen Evans, professor of pharmacoepidemiology at the London School of
Hygiene and Tropical Medicine.

\enquote{These results are consistent with what we see with other vaccines, because the
really big message for global health scientists is that this disease [COVID-19]
is able to be addressed by vaccines.}

Ian Jones, a professor of virology at the University of Reading, concurs.

\enquote{I see no reason to doubt it [the results],} Jones told CBC News in an
interview.    

"I agree that their initial results caused consternation, but I don't think
it's because they weren't valid. They were released a bit soon.

\enquote{I think it's going to be a useful vaccine.}

\ifcmt
pic https://i.cbc.ca/1.5819352.1606505500!/fileImage/httpImage/image.JPG_gen/derivatives/original_1180/health-coronavirus-eu-hungary.JPG
caption A laboratory assistant holds a tube with Russia's Sputnik-V vaccine at the National Institute of Pharmacy and Nutrition in Budapest on Nov. 19. (Matyas Borsos/Hungarian Foreign Ministry/Reuters)
\fi

Their positive assessments are based on the knowledge that the adenovirus
delivery method behind the Gamaleya-made vaccine has proven successful over and
over again.

What was unclear was whether the COVID-19 virus would be resistant, but Evans
says the other drug companies' positive results strongly suggest the Russian
vaccine will likely perform well, too.

\enquote{We now have four vaccines that have some efficacy [on COVID-19], which is way
beyond what we have ever had for an HIV or a malaria vaccine,} said Evans.

\subsubsection{Question of trust}

Ultimately, he says whether a country chooses to buy the Russian-made vaccine
comes down to a question of whether they have confidence in the science behind
it and trust the regulators who approved it.

The Russian vaccine gets treated more skeptically, said Evans, because the
processes in the United States and Europe are far more open and transparent
than they are in Russia.

\enquote{We do not know how carefully their trials are monitored and how carefully they
are reported. We do not know that,} he said.  

\enquote{But the countries that are buying it are buying it on trust that the Russians
have produced something.}
