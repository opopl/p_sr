% vim: keymap=russian-jcukenwin
%%beginhead 
 
%%file 15_05_2021.fb.bilchenko_evgenia.3.filosofia_skovoroda
%%parent 15_05_2021
 
%%url https://www.facebook.com/yevzhik/posts/3893096494058765
 
%%author 
%%author_id 
%%author_url 
 
%%tags 
%%title 
 
%%endhead 
\subsection{Сковорода - в граде Мира, Отца Нашего, Урания}
\label{sec:15_05_2021.fb.bilchenko_evgenia.3.filosofia_skovoroda}
\Purl{https://www.facebook.com/yevzhik/posts/3893096494058765}

Какое это удивительное состояние: услышать любимых философов Владимира Возняка
и Веру Лимонченко на международной конференции, посвященной любимому философу
Руси, - Григорию Саввичу Сковороде. И как это ужасно беспомощно: не мочь в 15
минут изложить все, что я думаю о \enquote{Разговоре пяти путников об истинном счастье
в жизни}, ибо в 18 веке человек сказал все так просто и ясно на дореформенном,
но кристально сочном, русском языке за 300 лет до Жижека и Питерсона. Вот что
такое - хронометраж в духовной философии... Как будто драгоценности просыпал.
Сергей Возняк , и ты там был, мед-пиво пил, \enquote{в граде Мира, Отца Нашего, Урания}
(ГС).


\ifcmt
  pic https://scontent-frt3-1.xx.fbcdn.net/v/t1.6435-9/187122757_3893096130725468_2341211093682005409_n.jpg?_nc_cat=107&ccb=1-3&_nc_sid=8bfeb9&_nc_ohc=6MrA9pJmArYAX88rDoK&_nc_ht=scontent-frt3-1.xx&oh=0a586ca5fc5ecbbd98aa4e5fcb9b64d3&oe=60C5CB67

  pic https://scontent-frt3-1.xx.fbcdn.net/v/t1.6435-9/187182406_3893096237392124_3347695463322035380_n.jpg?_nc_cat=108&ccb=1-3&_nc_sid=8bfeb9&_nc_ohc=9w5e6NHoPHoAX-CfTCi&_nc_oc=AQlLv_LbantnQVGDqJ2AP3KS7qCS8RE4BdjQp_0GI7SdzEs7ONsdtaADUgmOvZuEw1M&_nc_ht=scontent-frt3-1.xx&oh=4b02b05f3e1a97881318f12c071a3f6f&oe=60C6AD79

  pic https://scontent-frt3-1.xx.fbcdn.net/v/t1.6435-9/186450514_3893096397392108_4253025424340605343_n.jpg?_nc_cat=109&ccb=1-3&_nc_sid=8bfeb9&_nc_ohc=acu-zmnYbyUAX-I2iZs&_nc_ht=scontent-frt3-1.xx&oh=07309e068eb06095cfd68db41183ebda&oe=60C6AF79
\fi


Сергей Возняк

Было супер. Я все слашал, даже в дороге в наухах...

Алексей Бажан

Сколько мне известно, \enquote{философию} Сковороды придумал Эрн, из сугубо
националистического побуждения явить миру русского Канта; впрочем, в своем
амплуа \enquote{учителя жизни} Григорий Савиич оказывается во вполне достойном обществе
Льва Толстого и учителя Куна.

Евгения Бильченко

Алексей Бажан Ещё говорят, что Сиддхартха Гаутама Шакьямуни Будда был
внебрачным сыном Лао Цзы, ибо слишком схожи идеи шуньяты и Дао, да.

Сергей Никонов

Эту речь не слушал. Но у Евгении Витальевны плохих речей не бывает. Бывают
неудачные. Но не плохие.

Евгения Бильченко

Сергей Никонов Зато камера приказала долго жить: она у меня самая дешёвая была,
и нынче голос периодически пускается в дальнее плаванье по извивам микрофона,
но я знаю, где купить и что. Вообще я сплошная неудача, о чем тут беседовать?)

Сергей Возняк

Дочекайтеся запису в восьмого столу, а тоді судіть

Евгения Бильченко

Сергей Возняк так они Эрна и Сковороду судят, при чем тут такая мелочь, как я?

Евгения Бильченко

Сергей Возняк Сережа, они не на столы сбегаются, а посмотреть на Бильченко и
что-то такое сказать, что типа ее поразит.
