% vim: keymap=russian-jcukenwin
%%beginhead 
 
%%file 29_11_2020.sites.ru.zen_yandex.asov_alexandr.1.surozhskaja_rus.comments
%%parent 29_11_2020.sites.ru.zen_yandex.asov_alexandr.1.surozhskaja_rus
 
%%url 
 
%%author 
%%author_id 
%%author_url 
 
%%tags 
%%title 
 
%%endhead 
\subsubsection{Комментарии}

\begin{itemize}
\iusr{Алексей П.}

Александр, в Феодосии в "карантине" (остатках крепости) есть надгробия, на
которых древние руны и свастики. По-видимому, они более древние, чем Кафа.
Камни отполированные и глифы хорошо видны. Так же на остатках таможенных врат.

Крым ранее называли Солхат (дом Солнца). В путеводителе по Крыму чётко
написано, что турки после прибытия 30 лет жили в горах, пока им разрешили
спуститься и построить поселение в Бахчисарае в 1553г. н.э.

В Крыму много мест, связанных с древней Върой.

Танаис был построен по архитектуре древних славян. Это очевидно при
внимательном рассмотрении (капище, курганы, менгиры, глифы). Танаис дважды
сжигали, а на третий просто стёрли с лица земли.

В Таганроге есть вход в старый город, вход в который запечатан камнем с рунами,
вход не смогли открыть даже немецкие корабельные пушки. Там же в Таганроге есть
огромное древнее захоронение воинское. На нем построен завод в советское время.
В захоронение все славянские доспехи, много воинов похоронено. Если есть
интерес, дам информацию

\iusr{Александр Асов}
автор

Алексей П., крайне интересно... В Феодосии и я кое-что нашел из старой
русколанской архитектуры... Но вот надгробий с рунами не видел... Как бы их там
отыскать?

\iusr{Алексей П.}

Александр Асов, если итти по тропе от крепости к карантину вдоль моря, то на
подъёме, подле жилых домов лежат плиты с глифами. По-видимому, их перенесли для
"повторного пользования". Были фотки. К сожалению утрачены.

В Бахчисарае, подле колодца у города караимов в известняке "врощены"
металлические эксцентрики. По-видимому, очень древние.

Примеров много. Моя почта \verb|alexey7478sm@yandex.ru|

В теме. Так же занимаюсь исследованиями

\iusr{Lada Vittoria}

Не так давно тоже искала информацию по таврам, потому что попалось на глаза
утверждение, что тавры- это твари, захватчики. Ничего интересного тогда не
нашла, а вот после прочтения Вашей статьи, перед глазами отчётливо встал образ
Минотавра и того самого лабиринта. Не знаю, что это значит, информация от Ваших
статей идёт образами, а не буквами. Поэтому, не понимаю пока, что вижу.

И ещё из вики:

"Есть гипотеза об астроморфном происхождении названия от Taurus (латинский
вариант названия созвездия Тельца, с которым в древности отождествлялся Крым и
прилегающие к нему области)"

\iusr{Сергей Козырев}

Lada Vittoria, тавр и есть телец, вол, бык... Тавры поклонялись быку, тельцу,
поэтому и звались таврами... Были ариями..

\iusr{Lada Vittoria}

Сергей Козырев, да я знаю, что тавр бык, просто увидела какую-то связь между
Кипром и Крымом, и этот лабиринт.... Есть информация, что последняя битва,
которую мы проиграли, проходила в Крыму, поэтому и хочу понять, что я
почувствовала.

\iusr{Max Ten}

Lada Vittoria, прочтите мой коментарий, про Тавров, прймете, чуть ниде или выше

\iusr{Михаил Светлов}

Насчет Быка - Тавра, есть описания словянских богов Грамматиком и Гельмондом в
Ретре, в Венете и Арконе, что характерно, на щите бога было изображение бычьей
головы, хотя тотемным (священным) животным был жеребец.

\iusr{Бутов Евгений}

А сов ! Как вы могли читать ВЕЛЕСОВУ КНИГУ , КОТОРАЯ НАПИСАНА НА ПРА - ЯЗЫКЕ ,
о котором не имеете ни малейшего представления ? ЭХ ЛЬГУНИШКА !

\iusr{Михаил Светлов}

\ifcmt
  ig https://avatars.mds.yandex.net/get-zen_pictures/3435364/236505296-1608112491965/orig
\fi

\iusr{Михаил Светлов}

\ifcmt
  ig https://avatars.mds.yandex.net/get-zen_pictures/3503722/236505296-1608112465953/orig
\fi

\verb|Михаил Светлов|

(«В лето 6576 (1068) индикта 6 Глеб князь мерил море по льду от Тмутаракани до
Корчева — десять тысяч и четыре тысячи сажен»).Тмутараканский камень и надпись
на нём являются доказательством вхождения восточного Крыма в состав княжества.

\ifcmt
  ig https://avatars.mds.yandex.net/get-zen_pictures/3006583/236505296-1608112292330/orig
\fi

\iusr{Антон}
отредактировано

Здорово!!!

Ох уж эти тчурки...

\iusr{Николай В}

О Руси на Крымском полуострове писали еще греки и византийцы во втором веке до
нашей эры, и море называли Русским. Вот только наши грантоеды боятся об этом
писать. Как так, нам же сказали, что небыло русских, пока их Рюрик не вытащил в
цивилизацию, значит их и небыло. А то, что на этих землях еще 26 тысяч лет
назад уже были люди, знакомые с керамикой и изготавливающие фигурки из бивней
мамонта, а это нельзя сделать каменными скребками, мы будим не замечать. Люди
ведь появились 12 тысяч лет назад, да и то в Африке. А все, что с этим не
совпадает подделка.

\iusr{Max Ten}

Небольшое утрчнение
.
Тау это гора АлТай
.
ТауРус, это переводится как Бык, или рогоносец, Македонский , Зулькарнайн
.
Также говорят ТавРус или Тавр
.
Но это в е Тау Русы, горные русы
.
Орда это .... даже не буду обьяснять
.
Казаки какие, Ордынские
.
Тау Орда, это горная Орда
.
Она же Таврда или Таврида
.
В которой жили Тавры или Тау Русы

\iusr{Paul Elisarov}

Спасибо! Большие связные и интересные статьи - редкость на Дзене. И
раскапывание русской истории, с поиском доказательств, а не основываясь на
выдумках - тоже редкость. Я подписался.
\end{itemize}
