% vim: keymap=russian-jcukenwin
%%beginhead 
 
%%file 25_01_2022.tg.lesev_igor.1.javlenie_blogera_narodu
%%parent 25_01_2022
 
%%url https://t.me/Lesev_Igor/284
 
%%author_id lesev_igor
%%date 
 
%%tags birthday,rossia,ugroza,ukraina,zelenskii_vladimir
%%title Явление Блогера народу
 
%%endhead 
 
\subsection{Явление Блогера народу}
\label{sec:25_01_2022.tg.lesev_igor.1.javlenie_blogera_narodu}
 
\Purl{https://t.me/Lesev_Igor/284}
\ifcmt
 author_begin
   author_id lesev_igor
 author_end
\fi

Явление Блогера народу

Еще одна запредельно тревожная ночь позади. Долго не спал, размышлял о наших
слабых точках на границе. Лично меня смущает конотопское направление. А что
если первый удар произойдет там? Утром мои опасения подтвердили коллеги из
британского The Sun. Ребята опубликовали сверхподробную карту вторжения. Путин
таки будет бить через Конотоп и дальше на Киев. Теперь дело только в холодах.
Нужно, чтобы земля промерзла. Русские танки, оказывается, предпочитают ходить
по замороженной земле, а не чтобы слякоть и болотце. И тут все логично. Грязные
танки, штурмующие Киев – это выглядит отвратительно. Поэтому, нужны холода.

А вчера случилось-таки заседание СНБО. Лучшие умы страны во главе с главным
ветеринаром современности успокаивали украинцев. Мне очень понравился Данилов.
Мне вообще нравится его типаж. У такого никогда не будет глупой дилеммы о
стакане, который то ли наполовину полон, то ли наполовину пуст. Данилов всегда
найдет правду на дне стакана. И в этот тревожный для всей страны час ветеринар
был в особом ударе. Дрожащим голосом он сообщил, что ситуация «абсолютно
контролируемая». И потом добавил, что страна у нас одна, отдавать мы ее никому
не собираемся и победа будет точно за нами. Поэтому, суки, не нагнетайте.

Просто напомню, что пару дней назад некто В.А. Зеленский тер в интервью о
«сотнях тысячах погибших» и «неминуемом захвате Путиным Харькова». Есть ли
здесь какое-то противоречие? Конечно же, нет. Не забывайте, что у Данилова не
бывает полупустых стаканов.

Потом появился человек-деменция Шмыгаль. Мало кто знает, что у нас в стране
вообще есть премьер-министр, и еще меньше, чем он у нас тут занимается. Я вот
всё никак не могу запомнить, как правильно ставить ударение в его фамилии –
ШмЫгаль или ШмыгАль. Впрочем, от перемены ударения Шмыгаль более узнаваемым не
становится. Мне кажется, его и в морге хрен бы на раз-два обнаружишь. Вот
представьте – лежат на столах трупы, а Шмыгаль стоит где-то рядышком, а ты на
него все равно не обращаешь внимания.

И вот Шмыгаль сразу же успокоил. «Угроз функционирования национальной экономики
нет». Золотовалютных запасов дофига. Курс стабильный. Курс доллара 28,3 на
сегодня, и это не крах гривны, ничего подобного. Это, сука, «валютный коридор»
и это «один из ключевых макроэкономических показателей здоровой экономики». Я
вообще вижу особую синергию между Даниловым и Шмыгалем. Первый мог бы зверушкам
отрезать лапки, а второй бы читал им лекции о макроэкономической стабильности.

А итог подбил в своем традиционном видосике хорошо так раздобревший Блогер.
Ему, кстати, сегодня 44. Ермак, наверное, уже прогревает вертолеты. Но ближе к
видосику. «Все под контролем. Причин для паники нет». Хорошее такое и чертовски
обнадеживающее начало. Приблизительно так начинают врачи, когда объявляют
пациенту смертельный диагноз. Все-таки, Зеленский уникален сам по себе, и
наркотики тут не при чем. Вчера у него «русские захватят Харьков», а сегодня
«причин для паники нет». И все это умещается одновременно в одной голове.

Дальше блогер перешел к энергетике. «Мы можем констатировать, что спокойно
пройдем весь этот отопительный сезон». Газа – достаточно. Угля – достаточно.
Одним словом, причин для оптимизма – достаточно. А вот причин для паники –
недостаточно. И смотрит на тебя такой красавчик, что сомнений уже никаких –
что-то по итогу таки @бнется.

Не преминул Блогер зачесать и о «Вовиной тысяче». Если тебе уже 60 лет – можешь
купить даже лекарство. А с февраля пенсионеры смогут «эти кошти» использовать
на оплату коммуналки. А дальше еще эту тысячу на «образование» сможет тратить
школота возрастом от 14 до 18 лет.

Нет, я понимаю, что пацанва обитает в параллельной реальности. У них и скумбрия
по 8, и говядина по 20, и новый город на Черном море... Но должен же быть
какой-то предел кокаиновым фантазиям. Это единоразовая тысяча. Тысяча гривен, а
не тысяча евро или тысяча долларов. Сегодня тысяча гривен – это один большой
пакет продуктов в маркете. А они лепят под нее целые программы, куда эти ах
какие деньги можно будет еще потратить.
