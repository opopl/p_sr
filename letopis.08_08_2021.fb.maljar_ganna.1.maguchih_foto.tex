% vim: keymap=russian-jcukenwin
%%beginhead 
 
%%file 08_08_2021.fb.maljar_ganna.1.maguchih_foto
%%parent 08_08_2021
 
%%url https://www.facebook.com/ganna.maliar/posts/1945731408919225
 
%%author Маляр, Анна
%%author_id maljar_ganna
%%author_url 
 
%%tags foto,maguchih_jaroslava,minoborony.ukraina,olimpiada,olimpiada.tokio,rossia,ukraina
%%title Ярослава Магучіх - Фото
 
%%endhead 
 
\subsection{Ярослава Магучіх - Фото}
\label{sec:08_08_2021.fb.maljar_ganna.1.maguchih_foto}
 
\Purl{https://www.facebook.com/ganna.maliar/posts/1945731408919225}
\ifcmt
 author_begin
   author_id maljar_ganna
 author_end
\fi

Наша бронзова призерка Олімпійських ігор в Токіо Ярослава Магучіх дійсно є
молодшим лейтенантом ЗСУ. До складу української збірної ввійшли 60 армійських
спортсменів. 

Фото Ярослави поряд зі спортсменкою Марією Ласіцкене, яка виступала від команди
Олімпійського комітету Росії(збірна  РФ не брала участь в Олімпіаді на
виконання рішення спортивного арбітражного суду (CAS)), 

викликало обурливу реакцію в суспільстві. 

Спортсмени, які представляють Україну на міжнародних змаганнях, повинні
розуміти, що в Україні триває російсько-українська війна і це накладає  певні
обмеження та відповідальність. 

Вірне слідування спортивним традиціям, приміром - фото обіймів зі
спортсменами-громадянами РФ, коли вони теж стають призерами, одразу
використовує ворог як  інформаційно-психологічну гармату, яка демонструє світу,
що "їх тут немає".  

Тобто, необережна поведінка наших спортсменів, яких ми любимо і за яких щиро
вболіваємо, може ставати об"єктом інформаційних спецоперацій ворога.  

Разом з тим, прошу припинити цькування Ярослави, допоки ми не почуємо її власне
пояснення. 

Ярослава Магучіх на даний час знаходиться в Токіо. Одразу після її повернення в
мене запланована з нею зустріч. Про результати розмови повідомлю.
