% vim: keymap=russian-jcukenwin
%%beginhead 
 
%%file 30_07_2021.fb.fb_group.story_kiev_ua.1.velosiped_shuljavka_i_vojna.cmt
%%parent 30_07_2021.fb.fb_group.story_kiev_ua.1.velosiped_shuljavka_i_vojna
 
%%url 
 
%%author_id 
%%date 
 
%%tags 
%%title 
 
%%endhead 
\zzSecCmt

\begin{itemize} % {
\iusr{Алла Гершкович}
Какая страшная судьба...

\iusr{Георгий Майоренко}
\textbf{Алла Гершкович} Да, это большая боль семьи. До конца войны надеялись, что он вернётся. Думали, где-то в партизанах...

\iusr{Irina Bentaleb}

\ifcmt
  ig https://i2.paste.pics/d036862164dc7b75adefa725383e69a4.png
  @width 0.1
	@name scr.eye.cry.yellow
\fi

\iusr{Елена Немерицкая}
Печально... Спасибо, что поделились снимками и за рассказанную историю.

\begin{itemize} % {
\iusr{Георгий Майоренко}
\textbf{Елена Немерицкая} Просто такие судьбы это в какой-то мере боль всех киевлян....

\iusr{Елена Немерицкая}
\textbf{Георгий Майоренко} И не только киевлян... Спасибо, Георгий, что храните память.
\end{itemize} % }

\iusr{Вахтанг Кварелашвили}

Мои прабабушка и прадедушка ( по материнской линии, естественно) жили до войны
на Шулявке, именно на Борщаговской, там где сейчас 19-й корпус КПИ.

\begin{itemize} % {
\iusr{Георгий Майоренко}
\textbf{Вахтанг Кварелашвили} Наши жили на Борщаговской 30. После того, как снесли старую Шулявку, там сейчас Борщаговская 10.

\iusr{Ирина Панченко}
\textbf{Вахтанг Кварелашвили} Мы жили на Борщаговской 56А, там построены корпуса КПИ.Наши дома снесли последними.

\begin{itemize} % {
\iusr{Георгий Майоренко}
\textbf{Ирина Панченко} О, наши дореволюционные соседи! Янковские - фамилия на Шулявке уважаемая!

\iusr{Ирина Панченко}
\textbf{Георгий Майоренко} 

Но, Янковские жили не там, сейчас, даже не представляю где, а спросить негде. А это
была квартира маминого папы Фартушного Иллариона Митрофановича и его мамы
Дарьи.

\end{itemize} % }

\end{itemize} % }

\iusr{Оксана Тернавская}

Георгий Цуканов, помощник Л. И. Брежнева до самой его смерти - это и есть брат
Жорж? Называли его «серым кардиналом». Очень интересно! Расскажите обязательно.

\begin{itemize} % {
\iusr{Георгий Майоренко}
\textbf{Оксана Тернавская} 

\ifcmt
  ig https://scontent-mxp2-1.xx.fbcdn.net/v/t39.30808-6/223819524_2698055967160294_8920150328446410144_n.jpg?_nc_cat=100&ccb=1-5&_nc_sid=dbeb18&_nc_ohc=UijpQIt7CSMAX9RExAy&_nc_ht=scontent-mxp2-1.xx&oh=00_AT_TyAFv9Vz91nT39yC49oTtPU-z-nqnJdZGeG2NrO7Dkw&oe=61DB20CF
  @width 0.2
	%@wrap \parpic[r]
\fi

Да, это брат Володи - Георгий Эммануилович Цуканов. Я сейчас веду переговоры с
родственниками, чтобы не пропали его мемуары. Он не давал интервью, а внуку
своему говорил: \enquote{-Потом прочитаешь мои воспоминания.}

Георгий был старше Володи на 3 года. Накануне войны был инженером на
Металлургическом заводе в Днепродзержинске. Завод эвакуировали в Челябинск, и
Георгий руководил цехом, \enquote{ковал} победу. Вот фото Георгия Цуканова. Ему здесь
17 лет. Конечно же, сделаю о нем отдельный сюжет.


\begin{itemize} % {
\iusr{Оксана Тернавская}
\textbf{Георгий Майоренко} спасибо! Ждём!

\iusr{Irena Visochan}
\textbf{Оксана Тернавская} Да, очень интересно. @igg{fbicon.hearts.two} 

\iusr{Olena Klymenko}
\textbf{Георгий Майоренко} Спасибо, ждем.

\iusr{Георгий Майоренко}
Стараемся хранить память
\end{itemize} % }

\iusr{Ирина Нищимная}
Война самое страшное,,, спасибо за воспоминания,,,,

\iusr{Валентина Ободянская}
Как больно и страшно. Какие судьбы!

\end{itemize} % }

\iusr{Valentina Okladnaya}

И как бы ни было понятно, что этот мальчик всего лишь один из тысяч с подобной
судьбой, как больно читать и осознавать страшную трагедию жестоко загубленной
жизни...

\begin{itemize} % {
\iusr{Георгий Майоренко}
\textbf{Valentina Okladnaya} Да, подобные истории были во многих семьях. Светлая память всем киевлянам, не вернувшимся с той страшной войны...
\end{itemize} % }

\iusr{Всеволод Шевчук}
Цікаво прочитати оповідь про Жоржа сподіваюсь вона не буде такою сумною?

\begin{itemize} % {
\iusr{Георгий Майоренко}
\textbf{Всеволод Шевчук} У Жоржа, кроме потери брата (они с Володей были почти ровесники), была ещё одна трагедия. У него в автокатастрофе погиб сын Михаил.

\iusr{Всеволод Шевчук}
\textbf{Георгий Майоренко} сумно, нажаль в житті бувають неприємності я в 16 років втратив старшого брата!
\end{itemize} % }

\iusr{Oksana Romanova}
Светлая память.

\iusr{Георгий Майоренко}
\textbf{Oksana Romanova} Светлая память всем не вернувшимся...

\iusr{Виктор Михайлюк}
Тихо, стоп!!! Как говорят в. Адессе и на Шулявке, это две разные улицы!!!
Адесс, это одно, а Шулява - это совсем другое!!!

\begin{itemize} % {
\iusr{Виктор Михайлюк}
Кто знает за Шулявку, заходите. поговорим!!!

\begin{itemize} % {
\iusr{Георгий Майоренко}
\textbf{Виктор Михайлюк} Приятно встретить брата по Шулявке!

\iusr{Виктор Михайлюк}
Из каких мест на Шулявке будете? Хотелось бы повспоминать, если Вы из нашенских?

\iusr{Георгий Майоренко}
\textbf{Виктор Михайлюк} Мы с начала Шулявки. Дом наш был до революции и в довоенные годы под номером Борщаговская 30. А на перестроенной Шулявке мы из домов, прилегающих к Бермудскому треугольнику. А школа наша 41!

\iusr{Виктор Михайлюк}
\textbf{Георгий Майоренко} 

41 школа это та что за кривыми домами во дворе? А до бермудского это там где
улица Керосинная была? Но ведь это так далеко от реальной Шулявки! И там ближе
65-я школа была, что ближе к Старту.


\iusr{Георгий Майоренко}
\textbf{Виктор Михайлюк} Нет. Керосинная на другой стороне. Наша - Борщаговская. 41 школа это за Воздухофлотским мостом справа.

\iusr{Георгий Майоренко}
\textbf{Виктор Михайлюк} Отец мой учился в школе, что за Кинотеатром Довженко (светлая память). А я ходил пешком в 41.

\iusr{Виктор Михайлюк}
\textbf{Георгий Майоренко} 

За кинотеатром была 102-я украинская школа, там все мои кореша учились, а я ходил
в 142-Ю что прямо в метро КПИ с1968 по 1976 год!


\iusr{Виктор Михайлюк}
\textbf{Георгий Майоренко} ну да, за кривыми домами, где ещё спортивный магазин был.

\iusr{Георгий Майоренко}
\textbf{Виктор Михайлюк} Классные, кстати, были Спорттовары!

\iusr{Оксана Раєвська}
\textbf{Виктор Михайлюк} я тоже ходила в эту школу, кл рук. Изабела Александовна.
\end{itemize} % }

\iusr{Георгий Майоренко}
\textbf{Виктор Михайлюк} 

Адрес был Борщаговская 30. Кирпичный двухэтажный дом, построенный до революции.
Сейчас там девятиэтажка Борщаговская 10. Вдоль скоростного.

\begin{itemize} % {
\iusr{Оксана Раєвська}
\textbf{Георгий Майоренко} Борщаговская 50, прошло все детство. а во время войны там было подполье, которое предали.

\iusr{Виктор Михайлюк}
\textbf{Георгий Майоренко} Я звиняюсь, но это уже, по моему Квадрат а не Шулявка!

\iusr{Георгий Майоренко}
\textbf{Виктор Михайлюк} 

Квадрат появился в 60-х, когда снесли старую Шулявку, и построили квартал от
Политеха до Военторга, который приобрел квадратный вид. А Шулявка на этом месте
существовала ещё с времён царя Гороха. И улица Борщаговская, кстати, имела
другую конфигурации. Она начиналась от теперешнего красного дома Пр. Победы 15.

\iusr{Георгий Майоренко}
\textbf{Оксана Раєвська} 

Если у вас есть информация по Шулявскому подполью, давайте спишемся. У бабушки
сестру Анну Владимировну Балтушевич расстреляли в Бабьем Яру за связь с
подпольем. Я подавал запрос в архив СБУ, мне, как родственнику погибшей,
прислали материалы по Бабьему Яру, но конкретно по бабушкиной сестре нет
информации. Видимо, немцы спалили архивы перед отступлением.


\iusr{Виктор Михайлюк}
\textbf{Георгий Майоренко} 

извините, но такой информации не имею. Мы переехали на улицу Шулявскую, потом
Ванды Васильевской в 1966 году с Нивок. Батя получил 3-х комнатную квартиру от
завода им Петровского!

\iusr{Георгий Майоренко}
\textbf{Виктор Михайлюк} Соседи!!!

\iusr{Оксана Раєвська}
\textbf{Георгий Майоренко} 

информации ее много. Знаю только, что прабабушка узнавала от родственицы о
положении на фронте, а потом у нее была запад и в доме, куда моя прабабушка чуть
не попала. предупредила соседка. А соседей много осталось в Бабьем яру. немцы
тоже были разные. один показывал фотографии семьи и плакал, другой маму немец
спас, вылечил.

\iusr{Георгий Майоренко}
\textbf{Оксана Раєвська} 

Я чуть позже попробую рассказать историю о нашей тете Ане - родной сестре моей
бабушки. Она работала воспитательницей в Детском садике на Мельникова. Была
коммунисткой, осталась в Киеве для подпольной работы. Ее выдал мужик из своих.
Она жила тоже на Борщаговской 30.


\end{itemize} % }

\end{itemize} % }

\iusr{Николай Овдеенко}
Светлая память

\iusr{Arutyun Avetisyan}
Грустно, но Память важна!

\iusr{Георгий Майоренко}
\textbf{Arutyun Avetisyan} Несомненно... Удачи вам и добра!

\iusr{Valeriy Rukhman}

Я жил На любимой Шулявке с 1967г по 1995г. до Шулявке жил с 1944г на ул Горького
18 кв 15., где прошло моё детство

\iusr{Лариса Сергієнко}

\ifcmt
  ig https://scontent-mxp2-1.xx.fbcdn.net/v/t39.30808-6/228288149_322903156228005_4912860351891453830_n.jpg?_nc_cat=102&ccb=1-5&_nc_sid=dbeb18&_nc_ohc=9w50ljlQ-XIAX-F0Qh7&_nc_ht=scontent-mxp2-1.xx&oh=00_AT8cXj4K-AmTA0F4ACpNn0cH5CIqPXcT1EswdAexysTPjA&oe=61D9F6D9
  @width 0.3
\fi

\iusr{Almira Yusupov}
давайте помнить и детей учить. Это наша общая история. настоящая. не придуманная и не политизированная.

\iusr{Julia Panchul}

с 1946г я гуляла по Борщаговской улице, т.к. жила на территории КПИ, в 1947
пошла в школу 65 на Фабричной улице. Я с одноклассницами ходила по дворам этого
района, в которых были сады и много цветов. Потом в 50 - 60-х Студенты КПИ
снимали углы в усадьбах Борщаговской улицы. Помню как прокладывали трамвай,
достраивался КПИ и сама улица. Это история города.

\iusr{Ольга Круковская}
Таких судеб много и не дай, Бог повториться.

\iusr{Irina Kaminsky}
Какой взгляд и лицо неординарное

\iusr{Юрий Махиня}
Щиро дякую за таку цікаву і дуже зворушливу оповідь!

\iusr{Владимир Новицкий}

Спасибо за интересный рассказ. Дай Бог чтобы подобное не повтлорилось никогда.

\begin{itemize} % {
\iusr{Георгий Майоренко}
\textbf{Владимир Новицкий} Благодарю. А ведь жили наши недалеко от ваших. Начало Борщаговской, до района Евбаза рукой подать. Светлая память всем ушедшим.
\end{itemize} % }

\iusr{Римма Моржевская}
И, сколько же таких судеб

\iusr{Kateryna Masliechkina}
Интересно, спасибо!

\iusr{Елена Никитчук}
Браво за память!

\iusr{Светлана Наливайчук}
Да..

\iusr{Viktoria Terpylo}
Какая грустная история..... а если бы не война, все могло сложиться по другому.....

\begin{itemize} % {
\iusr{Георгий Майоренко}
\textbf{Viktoria Terpylo} 

Да, Володю Цуканова могло ждать прекрасное будущее. Я думаю, он был толковым
парнем. Отец его был замечательным человеком. Рано потерял папу, его отдали в
интернат, он окончил с отличием благотворительную гимназию, поступил в Киевский
университет, затем в КПИ, в Днепродзержинске был завкафедрой. Конечно же и
перед сыном его Володей открылись бы широчайшие возможности.


\iusr{Viktoria Terpylo}
\textbf{Георгий Майоренко} однозначно!... но все происходит по обстоятельствам и иногда не справедливо....
\end{itemize} % }

\iusr{Елена Новикова}

Георигий, с интересом прочитала Ваше сообщение. Как важно знать историю своего
рода!!! Вы очень интересно пишете. Я тоже выросла на Шулявке)), но, гораздо,
позже))) Божьего Благословения Вашему роду!!!

\begin{itemize} % {
\iusr{Георгий Майоренко}
\textbf{Елена Новикова} Спасибо за добрые слова. Хочется сохранить память об этих людях. Ведь они ходили по тем же улицам, что и мы. И мечтали о чём-то светлом.
\end{itemize} % }

\iusr{Natasha Levitskaya}

Спасибо, Георгий! Каждый раз восхищаюсь, как много вы знаете о своих предках и
членах вашей семьи - интересуетесь, разыскиваете. Судьба этого мальчика, как и
судьбы многих пропавших и погибших в той страшной войне. Очень жаль... Ждём
продолжения истории и судьбы его брата.

\begin{itemize} % {
\iusr{Георгий Майоренко}
\textbf{Natasha Levitskaya} 

Благодарю. Спасибо предшественникам, которые сохранили память, фотографии. А я
всегда бережно относился к истории, занимался нумизматикой, коллекционировал
музыкальные записи. Сейчас активно работаю в архивах. Многое ещё предстоит
выяснить.

\begin{itemize} % {
\iusr{Natasha Levitskaya}
\textbf{Георгий Майоренко}
Успехов вам, а нам новых историй!

\iusr{Георгий Майоренко}
Благодарю. Будем искать. Вот, кстати фото могилы отца погибшего Володи - Эммануила Федоровича Цуканова.

\ifcmt
  ig https://scontent-mxp2-1.xx.fbcdn.net/v/t39.30808-6/227442483_2698536910445533_5060046459517060183_n.jpg?_nc_cat=104&ccb=1-5&_nc_sid=dbeb18&_nc_ohc=jbhbxXp9ENQAX-hCoOv&_nc_ht=scontent-mxp2-1.xx&oh=00_AT_T9YxuXATltfl8T3vvAM_N1kqlprIzNccBB8lJx4g_0Q&oe=61DB1D5E
  @width 0.2
\fi

\end{itemize} % }

\end{itemize} % }

\iusr{Нина Опольская}

Память. Как прекрасно, что человек наделён этим. Так кратко и такая судьба у
мальчика. Спасибо за Память, душевный рассказ.

\begin{itemize} % {
\iusr{Георгий Майоренко}
\textbf{Нина Опольская} Мне, наверное, из родственников больше всего жаль Володю Цуканова. Погиб в 19 лет!

\iusr{Нина Опольская}
\textbf{Георгий Майоренко} Очень жаль... Вся жизнь была впереди... Вы схожи с ним

\iusr{Георгий Майоренко}
\textbf{Нина Опольская} Возможно, но больше я похож на его брата Георгия. Меня ведь в честь его так назвали.
\end{itemize} % }

\iusr{Елена Сидоренко}
Такой славный мальчик! Если б не эта война! @igg{fbicon.cry} 

\iusr{Георгий Майоренко}
\textbf{Елена Сидоренко} Очень жаль его... Хоть фотки сохранились.

\iusr{Анна Сидоренко}
Судьба от которой не уйдёшь.

\begin{itemize} % {
\iusr{Георгий Майоренко}
\textbf{Анна Сидоренко} 

Да, эту семью сопровождали беды. У Володиной мамы Антонины первый муж погиб в
Киеве в гражданскую, от него остался сын Георгий. А Володя родился от второго
брака. Получается, Антонина сначала мужа потеряла, потом сына...

\end{itemize} % }

\iusr{Любовь Козинская}

Спасибо! Как хорошо, что вы бережно храните память о своих близких. С
нетерпением жду продолжение вашего интересного повествования

\begin{itemize} % {
\iusr{Георгий Майоренко}
\textbf{Любовь Козинская} Спасибо, Любовь, за добрые слова. Обязательно будет продолжение. Удачи и добра!

\iusr{Любовь Козинская}
\textbf{Георгий Майоренко} И вам всего наилучшего. Жду продолжение
\end{itemize} % }

\iusr{Сингаевская Марина}
Спасибо за память!

\iusr{Георгий Майоренко}
\textbf{Сингаевская Марина} Благодарю! Быть добру!

\iusr{Ольга Зорина}
Очень интересный рассказ ! Спасибо!!!

\iusr{Ніна Бойко}
Дуже гарно написано

\begin{itemize} % {
\iusr{Георгий Майоренко}
\textbf{Ніна Бойко} Спасибо, Нина. Писал и плакал. Так жалко Володю! Погиб в 19 лет...

\iusr{Ніна Бойко}
\textbf{Георгий Майоренко} 

а я читала і думала, ким би він був, чим займався б, дуже шкода. Я люблю
рослини, займаюсь городом і завжди саджаю всі, хоч і кволі, даю шанс і право на
життя, радію коли росте, ... а тут людина

\iusr{Георгий Майоренко}
\textbf{Ніна Бойко} 

Несомненно, толковый был парень. Думаю, был бы прежде всего порядочным
человеком. Отец его заведовал кафедрой в Днепродзержинском металлургическом
институте. Этот институт окончил брат Володи Жора. Может, и Володя поступил бы
туда же?

\end{itemize} % }

\iusr{Татьяна Зубко Маркина}
Спасибо за рассказ, за память вашу, значит и за нашу.

\iusr{Георгий Майоренко}
\textbf{Татьяна Зубко} Благодарю. Светлая память всем ушедшим и не дожившим...

\iusr{Наталия Вигерина}
Больно до слез............

\iusr{Анна Мамрай}
Читаю-, и мурашки по коже. Спасибо.

\iusr{Наталия Харченко}
Плачу, спасибо за память!

\iusr{Наталія Кудря-Маршал}

Как грустно... Мой дедушка Федор, отец мамы, тоже пропал без вести... Так и не
смогли выяснить обстоятельств его гибели... А отец папы прошел концлагеря и
дошел с пехотой до Праги...

\iusr{Георгий Майоренко}
\textbf{Наталія Кудря-Маршал} Светлая память тем, кто шел в атаку за нас, за мирное небо...

\iusr{Раиса Карчевская}
Спасибо за память. Так хорошо, что Вы так много знаете о своих родственниках.

\iusr{Георгий Майоренко}
\textbf{Раиса Карчевская} Благодарю. Стараюсь делиться информацией. Быть добру!

\iusr{Игорь Мезецкий}
Судьба... @igg{fbicon.cry} 

\begin{itemize} % {
\iusr{Георгий Майоренко}
\textbf{Игорь Мезецкий} Печальная судьба.

\iusr{Игорь Мезецкий}
\textbf{Георгий Майоренко} Да! @igg{fbicon.cry} 
\end{itemize} % }

\iusr{Светлана Хоменко}
Какой взгляд у мальчишки с книгой....

\iusr{Георгий Майоренко}
\textbf{Светлана Хоменко} Да, я без слез не могу смотреть на эти фотографии... Так жаль парня...

\iusr{Сергей Черняк}
Привет Киевляне, дорогие земляки, дай Вам всем бог...

\iusr{Георгий Майоренко}
\textbf{Сергей Черняк} Удачи. Светлая память ушедшим.

\iusr{Судак Лариса}
 @igg{fbicon.face.sleepy} 

\iusr{Татьяна Зарицкая}
Оч жаль! Светлая память!

\iusr{Svetlana Naulko}

Георгий, какие безценные истории из памяти своей благодарной - дарите нам!!!
Прочла на одном вдохе - сколько Судеб Настоящих, Блистательных по сути... !!!!
Разве же можно нам жить так, как сейчас ... предавая, не помня, не ровняясь....  @igg{fbicon.heart.with.ribbon} 

\begin{itemize} % {
\iusr{Георгий Майоренко}
\textbf{Svetlana Naulko} 

Спасибо. Будем вместе хранить память об ушедших.

\enquote{Наши мертвые нас не оставят в беде,
наши павшие, как часовые}

В. Высоцкий.

\iusr{Svetlana Naulko}
\textbf{Георгий Майоренко} С благодарностью, Гергий @igg{fbicon.hands.pray} 
\end{itemize} % }

\iusr{Elena Dubovenko}
Эх люди... Сколько молодых жизней пропало, не перечесть! Из-за амбиций власть имущих.
Спасибо за память...

\iusr{Георгий Майоренко}
\textbf{Elena Dubovenko} Спасибо. Светлая память погибшим.

\iusr{Ольга Балаба}
 @igg{fbicon.cry} 

\iusr{Ли Бесс}
Недетское лицо у мальчика, как будто проступает печать нелегкой судьбы

\end{itemize} % }
