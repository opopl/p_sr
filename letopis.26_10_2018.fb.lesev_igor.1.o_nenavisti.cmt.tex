% vim: keymap=russian-jcukenwin
%%beginhead 
 
%%file 26_10_2018.fb.lesev_igor.1.o_nenavisti.cmt
%%parent 26_10_2018.fb.lesev_igor.1.o_nenavisti
 
%%url 
 
%%author_id 
%%date 
 
%%tags 
%%title 
 
%%endhead 
\subsubsection{Коментарі}
\label{sec:26_10_2018.fb.lesev_igor.1.o_nenavisti.cmt}

\begin{itemize} % {
\iusr{Дмитро Петренко}
 @igg{fbicon.100.percent} 

\iusr{Ян Пругло}

Дегуманизация все же произошла раньше, возможно в середине "нулевых". Да, она
была в полушутливом тоне, люди не готовы были убивать, но уже тогда словил себя
на мысли, что гусеница рано или поздно станет бабочкой.

\iusr{Матвей Кублицкий}

Вы четко идете по стопам России 90-х. Общая ненависть в обществе. Помню себя
тогда. бил людей даже не задумываясь. если что не по мне - лови... и вокруг
были такие же злые.... какой-то апофеоз злобы..... щас всё это же наблюдаю на
Украине

\iusr{Михаил Подоляк}

))) ты - умный и интеллектуальный... а значит, тебя не терпят, а наслаждаются отличным общением)...

\iusr{Дмитрий Коломийченко}
Настоящая ненависть началась позже, когда на неё перешли "утюги".

\iusr{Василий Январев}

Не согласен. Захват Киеврады - это в принципе мелочь. Водороздел - это события
19-22 января, когда пролилась первая кровь. Кровь и разделила.

\iusr{Ярослав Козачок}

Отлично написано. Масштаб трагедии 13/14 еще предстоит осмыслить. Только через
5 лет я, например, начинаю выстраивать все произошедшее в систему. Где есть
причины, следствия, действующие лица и исполнители. Один из выводов: проиграли
все. А искусственно навязанная взаимная ненависть стала обязательным условием
сказочного обогащения ограниченного круга лиц. Не могу призывать к взаимному
прощению, поскольку сам еще к этому не готов. Но, наверное, другого пути нет.

\iusr{Светлана Соколова}
вот же ж блин. Игорь, я не женщина)))

\iusr{Андрей Сергеевич}
Про общение в ФБ просто как обо мне написал. Интересно, материться уже нельзя? Правда?

\iusr{Михайло Бойченко}

Если б не мрачная тема, сказал бьі, что пост о любви - потерянной... Держись,
Игорек, мьі еще заставим замирать не одно сердце @igg{fbicon.wink} 

% -------------------------------------
\ii{fbauth.procenko_aleksandr.lugansk.lnr.obschestvennaja_palata_lnr}
% -------------------------------------

Некоторые полагают, что кто-то откуда-то нас всех дистанционно расчеловечивает,
что люди в этом плане исключительно обьекты. Отчасти да, но на самом деле люди
сами себя вполне эффективно могут расчеловечить. Главное создавать правильные
условия, вызывающие правильные эмоции и не мешать, а наоборот поощрять
неправильное поведение, потворствовать проявлениям жестокости и культивировать
безнаказанность.

Гоббсовское состояние догосударственного общества в формате "войны всех против
всех" возвращает свои позиции пропорционально тому, как теряет позиции
государственый порядок.

\end{itemize} % }
