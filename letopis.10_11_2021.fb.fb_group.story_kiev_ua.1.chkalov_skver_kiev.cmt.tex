% vim: keymap=russian-jcukenwin
%%beginhead 
 
%%file 10_11_2021.fb.fb_group.story_kiev_ua.1.chkalov_skver_kiev.cmt
%%parent 10_11_2021.fb.fb_group.story_kiev_ua.1.chkalov_skver_kiev
 
%%url 
 
%%author_id 
%%date 
 
%%tags 
%%title 
 
%%endhead 
\subsubsection{Коментарі}
\label{sec:10_11_2021.fb.fb_group.story_kiev_ua.1.chkalov_skver_kiev.cmt}

\begin{itemize} % {
\iusr{Ольга Стебливец}

Может уже дать этой улице имя человека вне политики, но великого актера,
который жил и работал в этом городе, чтобы не было поводов для следующих
переименований?

\begin{itemize} % {
\iusr{Журавель Павло}
\textbf{Ольга Стебливец} Кому специальность человека, когда мешала. Думаю, неокомиссары, что переименовывают сквер Чкалова особо в подробности не вдаются.

\iusr{Muravov Dmitriy}

Так Чкалов, летчик-технарь, занимался "внеполитикой", как и Богдан Ступка. Но
по нынешним временам Ступка был "внеполитичнее" в правильном направлении  @igg{fbicon.wink} 

\begin{itemize} % {
\iusr{Ольга Стебливец}
\textbf{Muravov Dmitriy} когда Вы слышите имя Чкалова, что первое приходит на ум - знаменитый советский лётчик.
А когда слышите имя Богдана Ступки - великий украинский актер. Есть разница?
К тому же какое отношение Чкалов имеет к Украине и Киеву?

\iusr{Muravov Dmitriy}
\textbf{Ольга Стебливец} 

Ну Вы ж понимаете, что тут проблема субъективного восприятия. Для меня Чкалов
прежде всего летчик, любивший и знавший свое дело, установивший ряд рекордов,
дерзкий, храбрый. Для Вас Ступка - значимый украинский актер. Хотя 30 лет из 50
лет своей активной работы он работал в советские времена, т.е. был советским
актером. Этот "советский стаж" у Ступки даже больше, чем у Чкалова  @igg{fbicon.wink}  Ступка
был министром культуры Украины лет 20 назад. И даже, о кошмар, он сыграл Тараса
Бульбу у режиссёра Бортко, последний, как известно, под нашими санкциями. На
бортовский фильм наши тоже плюются. Во "внеполитику" это никак не подходит. Мы
ведь изначально именно про аполитичность говорили.

Но да ладно, то таке, просто хотел постебаться над этой волной переименований,
ибо ничего кроме бессмысленной траты денег из бюджета и трат для бизнеса
переименование улицы не означает.

С Вами абсолютно согласен - надо переименовать в Маловладимирскую и дело с
концом. Всего доброго  @igg{fbicon.smile} 

\end{itemize} % }

\iusr{Вадим Горбов}
\textbf{Ольга Стебливец} Маловладимирская - самое то.

\begin{itemize} % {
\iusr{Ольга Стебливец}
\textbf{Вадим Горбов} и это тоже вариант, но он, по всей вероятности, не рассматривался.

\iusr{Вадим Горбов}
\textbf{Ольга Стебливец} Так улицу не будут переименовывать. Гончар на ней жил и творил. Дискуссия граждан про сквер.

\iusr{Ольга Стебливец}
\textbf{Вадим Горбов} Вы упомянули улицу, я Вам про улицу и ответила. Лично для меня милее старые названия улиц, исконные, так сказать.
\end{itemize} % }

\iusr{Ілона Луценко}
\textbf{Ольга Стебливец} 

А может хватит переименовывать. А может хватить отмывать на этом наши налоги.
Может лучше новый сквер возвести.

\begin{itemize} % {
\iusr{Ольга Стебливец}
\textbf{Ілона Луценко} это было идеально, но речь- то шла о переименовании.
А я двумя руками за новые скверы, пусть даже миниатюрные, во всех районах Киева.
\end{itemize} % }

\end{itemize} % }

\iusr{Kolesnykov Dimitry}
Вернуть первое историческое и прекратить эту бессмысленную череду героев эпох.

\iusr{Alexander Bezruk}
Вулиця яка? ЧкалОва!
Через 20 лет забудут, кто такой ЧкАлов, людям без разницы, как улица называется, лишь бы комфортно жилось, мне кажется.
Переименование стоит денег.

\begin{itemize} % {
\iusr{Вадим Горбов}
\textbf{Alexander Bezruk} 

та не. В паспортах модно штамп не менять. А таблички - то такое. У нас на
Кирилловской на каждом втором доме табличка улица Фрунзе а переименовали
двадцать лет назад.

\begin{itemize} % {
\iusr{Светлана Манилова}

Вадим, да, обращала на это внимание, а почему 20, а не 6 (скоро 7)?

\iusr{Alexander Bezruk}
\textbf{Вадим Горбов} затраты не на таблички, а на информационные системы, адресные классификаторы, в которых нужно поменять объект.
Ну и таблички тоже чего-то да стоят.

\iusr{Александр Владимирович}
\textbf{Вадим Горбов} они у вас не успевают поменять очень длинная улиц пока закончат опять переименуют

\iusr{Вадим Горбов}
\textbf{Светлана Манилова} я образно. Это Д самая длинная улица. Ваш участок назывался Ежова.)
\end{itemize} % }

\iusr{Irina Somova}
\textbf{Alexander Bezruk} не знал, не знал и забыл...

\end{itemize} % }

% -------------------------------------
\ii{fbauth.gavrilov_igor.kiev.ukraina}
% -------------------------------------

Чкалов закопаний в Кремлівську стіну, треба туди і пам'ятник віддати, до України
при всіх звитягах він ніякого відношення не має. Сквер має бути українським, най
буде в пам'ять Богдана Ступки.

\begin{itemize} % {
\iusr{Natalia Nikolayenko}
\textbf{Игорь Гаврилов}, а нащо це Богдану Ступці? Щоб
прикривались його ім‘ям заради зовсім іншіх цілей?

\begin{itemize} % {
\iusr{Игорь Гаврилов}
\textbf{Natalia Nikolayenko} 

Ступці вже звичайно ніщо. Але ми живемо тут, для нас Богдан Ступка моральний
авторитет, а Чкалов ніхто, він до нашої країни не дотичний. Він частина чужої
історії. До речі які цілі, цілей ніяких нема.


\iusr{Валерий Зиновьев}
\textbf{Игорь Гаврилов} Вибачте, а чим "дотичний" до нашоі краіни Джон Маккейн, що його ім,ям назвали вулицю у Києві?

\iusr{Tatiana Tokarenko Priluchnaya}
\textbf{Валерий Зиновьев} Ви дійсно не знаєте, що Маккєйн друг України, якій до останнього її відстоював та бував тут?

\iusr{Natalia Nikolayenko}
\textbf{Игорь Гаврилов} , це Ви так думаєте щодо цілей.

\iusr{Валерий Зиновьев}
\textbf{Tatiana Tokarenko Priluchnaya} 

а хіба в Україні не було і немає своїх «друзів», обов'язково треба того, кто
воював у В'єтнамі і, напевно же, як війсковий летчик застосовував і напалм
проти народу В'єтнама.

Свої герої вже закінчились ? Залишилися тільки американські?

\iusr{Tatiana Tokarenko Priluchnaya}
\textbf{Валерий Зиновьев} 

у В'єтнамі була війна СРСР із США. В мене чимало друзів в'єтнамців, що там
було, знаю не лише з газет. Багато в'єтнамців зараз мешкають в США.

\iusr{Валерий Зиновьев}
\textbf{Tatiana Tokarenko Priluchnaya} тобто, те, що американці зробили у Сонгмі, це вони так воювали з СРСР?

\iusr{Игорь Гаврилов}
\textbf{Валерий Зиновьев} За неоціненний вклад в боротьбу з радянщиною і російською агресією проти України. Ви певно з Місяця спустились на землю?

\iusr{Игорь Гаврилов}
\textbf{Oleg Reznikov} Ви часом не з Рязані?

\iusr{Oleg Reznikov}
\textbf{Игорь Гаврилов} Киев

\iusr{Игорь Гаврилов}
\textbf{Oleg Reznikov} Дивно, пишете як з Рязані.
\end{itemize} % }

\iusr{Вадим Горбов}
\textbf{Игорь Гаврилов} 

Я за Григория Гершуни если что. А кто где закопан - это не критерий истины.
Юрий Долгорукий в Берестове закопан и что теперь? И чего вы молчали, когда
Кличко прошлым летом Мюнхенский сквер открыл на Дегтяревской? Какое Мюнхен
имеет отношение?

\begin{itemize} % {
\iusr{Игорь Гаврилов}
\textbf{Вадим Горбов} 

Мюнхен був осередком українських емігрантів-патріотів. Юрій Довгорукий, при всіх
його вадах, князь Київський, тому тут і похований. Чкалов похований в Кремлі.

\iusr{Natalia Nikolayenko}
\textbf{Игорь Гаврилов}, 

в Мюнхені багато чого відбувалося і доброго, і поганого з початку його
заснування. І при бажанні можна знайти аргументи і за, і проти.


\iusr{Rimma Turovskaya}
\textbf{Игорь Гаврилов} А еще Мюнхен-город, где зародился фашизм.

\iusr{Вадим Горбов}
\textbf{Игорь Гаврилов} 

я и те кто здесь вас и вашу биографию и гражданскую позицию знают, уважают вас
как патриота, но вы не истина в последней инстанции. Умение отличать добро и
зло - это дар Божий и он даётся далеко не всем.

\end{itemize} % }

\iusr{Юлия Тертица}
\textbf{Игорь Гаврилов} 

по вашей логике все памятники Т. Г. Шевченко из стран «к которым он не имеет
отношения» надо перенести в Канев?

\ifcmt
  ig https://scontent-lga3-1.xx.fbcdn.net/v/t39.30808-6/255363436_10216472711649333_8264627496962983143_n.jpg?_nc_cat=107&ccb=1-5&_nc_sid=dbeb18&_nc_ohc=e-URCkCdA9sAX8zuIVp&_nc_ht=scontent-lga3-1.xx&oh=f4079fa27920b127ad264af2c6c73ff7&oe=618FFC1C
  @width 0.4
\fi

\begin{itemize} % {
\iusr{Игорь Гаврилов}
\textbf{Юлия Тертица} Не смішіть недолугими порівняннями.
\end{itemize} % }

\iusr{Irina Somova}
\textbf{Ігор Гаврилов} в стену не закапывают....

\begin{itemize} % {
\iusr{Игорь Гаврилов}
\textbf{Irina Somova} Засовують, так вам прийнятніше?

\iusr{Irina Somova}
\textbf{Ігор Гаврилов} и опять безграмотно...
\end{itemize} % }

\iusr{Vadim V. Bandourin}
\textbf{Ігор Гаврилов} Судячи з таких креативних думок, чому не "Гавриленко" ніяк не зрозумію?!

\iusr{Igor Namchuk}
\textbf{Ігор Гаврилов},

тоді і пам'ятник Б. Хмельницькому знести треба.. Адже він союз із Московією
уклав... @igg{fbicon.laugh.rolling.floor} Я особисто проти такого перейменування... Краще посадити новий сквер
і назвати на честь Б. Ступки... А якщо народ забуває свою історію, то у нього
немає майбутнього... @igg{fbicon.face.smiling.sunglasses} 

\begin{itemize} % {
\iusr{Виталий Перещ}
\textbf{Igor Namchuk}

Ніякого "союзу" Б. Хмельницький з Московією не укладав. Ніяких документів про
це нема. Все було значно простіше.  Це сторінка з "Русской Истории" 1932ґ
издания.

\ifcmt
  ig https://scontent-lga3-1.xx.fbcdn.net/v/t1.6435-9/254882194_591378551980704_4975350586352260838_n.jpg?_nc_cat=105&ccb=1-5&_nc_sid=dbeb18&_nc_ohc=6YeLRkEuzkoAX8ImON3&_nc_ht=scontent-lga3-1.xx&oh=89f3778026f018948a68255707a720fd&oe=61B13A12
  @width 0.5
\fi

\iusr{Игорь Гаврилов}
\textbf{Igor Namchuk} Свою історію треба пам'ятати, а не культивувати чужу. Посадіть сквер.

\iusr{Natalia Nikolayenko}
\textbf{Игорь Гаврилов} , 

люди сплачують податки і для того, щоб сквер висаджували спеціально навчені
люди. І щоб керівництво міста виділяло на це кошти і територію. До речі, Клічко
робить це по можливості.

\iusr{Татьяна Шиверская}
\textbf{Igor Namchuk} Ваше предложение поддерживаю!
\end{itemize} % }

\iusr{Тапуа Nesterenko}
\textbf{Tatiana Tokarenko Priluchnaya} 

вони вам про напалм, певно, розповіли також. І що американці пішли з В'єтнаму
після того, як в США розгорнулось антивоєнна кампанія?

\end{itemize} % }

\iusr{Микола Веселий}

Президент США не стал бы организовывать торжественные приемы и банкеты для
коммунистического пропагандиста. Чкалов был летчик. Коммунистические
пропагандисты потом перекрашиваются в других пропагандистов при сменах власти.
А летчик - он всегда летчик.

\iusr{Максим Мельничук}

Чкалов ніякого відношення до Києва не мав. Богдан Ступка мав - жив, творив,
виступав тут десятки років. Я взагалі не дуже бачу, що власне тут обговорювати

\begin{itemize} % {
\iusr{Ілона Луценко}
\textbf{Максим Мельничук} Так не обговаривайте! Оставьте сквер в покое. Обговаривайте другие насущные проблемы города! Сквер не трогайте!

\iusr{Максим Мельничук}
\textbf{Ілона Луценко} геніально  ☺ ️  то чому Ви самі доєднуєтесь до обговорення?))

\iusr{Татьяна Гурьева}
\textbf{Максим Мельничук} зробити для Ступки новий сквер, не порушувати те, що побудовано

\begin{itemize} % {
\iusr{Максим Мельничук}
\textbf{Татьяна Гурьева} а хто що збирається там порушувати? Наявність там скверу чи його відсутність не обумовлена назвою.

\iusr{Татьяна Гурьева}
\textbf{Максим Мельничук} памятник приберуть, взагали це память

\iusr{Максим Мельничук}
\textbf{Татьяна Гурьева} Пам‘ять кому-чому? Тут той Чкалов в кращому випадку автопробігом на півдня заїздив

\iusr{Татьяна Гурьева}
\textbf{Максим Мельничук} вважайте як хочете. Багато людей шанують його
\end{itemize} % }

\iusr{Вадим Горбов}
\textbf{Максим Мельничук} шановний я обговорюю книгу а не Ступку

\begin{itemize} % {
\iusr{Максим Мельничук}
\textbf{Вадим Горбов} Ви хоч запам‘ятовуйте, що самі пишите. Там з книги тільки фото та 2 речення. Все інше про постать Чкалова та історію найменувань вулиці

\iusr{Константин Кравченко}
\textbf{Вадим Горбов} ...растолочь Чкалова - в ступке истории )
\end{itemize} % }

\iusr{Валентина Горковенко-Спицына}
\textbf{Максим Мельничук} 

Еще какое отношение имел! Сколько мальчишек, по примеру В.Чкалова, мечтали
летать и стали летчиками. Он был для них героем! Масштаб личности не зависит от
географии и всегда вызывает уважение! У людей, которые это понимают. А для
памяти нашего выдающегося актера Б. Ступки всегда можно и нужно найти
подходящее место, в городе есть для этого большие возможности.

\iusr{Татьяна Гурьева}
\textbf{Валентина Горковенко-Спицына} спасибо) хорошо сказали

\iusr{Rimma Turovskaya}
\textbf{Максим Мельничук} Не надо сталкивать лбами Чкалова и Ступку. Оба они заслужили свое место в истории.
\end{itemize} % }

\iusr{Dmytro Nohal}
На фото вылитый кличко

\iusr{Aleksandr Mitryaev}
Улица "Лысого" из квартала - еще три года продержится название ...

\begin{itemize} % {
\iusr{Вадим Горбов}
\textbf{Aleksandr Mitryaev} лучше Лысого Игоря Кононенко из ЕС

\iusr{Виктория Руденко}
\textbf{Aleksandr Mitryaev} Бред! Сами поняли, что написали
\end{itemize} % }

\iusr{Dmytro Nohal}

Все эти переименования не стоят того, поколение, что родилось в 90 в основной
массе знает ко такой чкалов на столько же, как и ступка

\begin{itemize} % {
\iusr{Вадим Горбов}
\textbf{Dmytro Nohal} возможно, но в этом виноваты и мы с вами и наше поколение

\iusr{Dmytro Nohal}
\textbf{Вадим Горбов} да, абсолютно согласен

\iusr{Oksana Khorozova}
\textbf{Dmytro Nohal} Не забувайте, що цей парк під загрозою знесення!! А ми тут назву обговорюємо. Може статися, що там виросте новий будинок і парку взагалі не буде.

\iusr{Dmytro Nohal}
\textbf{Oksana Khorozova} обязательно, обязательно вырастет!!!

\iusr{Наталия Калинина}
\textbf{Dmytro Nohal} А это же поколение напр. Во Франции, знает, кто такой Робиспьер, Марат, Ейфель, Кокто и др?)))) Но памятники им не сносят и улицы не переименовывают, однако...
\end{itemize} % }

\iusr{Евгений Бабенко}
кадр с того митинга  @igg{fbicon.face.grinning.squinting} 

\ifcmt
  ig https://scontent-lga3-1.xx.fbcdn.net/v/t39.30808-6/255723603_1333479790418470_6837140370238856721_n.jpg?_nc_cat=110&ccb=1-5&_nc_sid=dbeb18&_nc_ohc=wVHzAyZV6T4AX-zT1q5&_nc_ht=scontent-lga3-1.xx&oh=a33e9a24ddc825d4d475c10c2768100c&oe=61902AF2
  @width 0.4
\fi

\begin{itemize} % {
\iusr{Максим Мельничук}
це просто 5. Тонко для декого навіть - не кожен ватнік навіть зрозуміє
\end{itemize} % }

\iusr{Оксана Тукалевская}
Ступка тут и рядом не стоял. При всем моем уважении.

\begin{itemize} % {
\iusr{Наталия Калинина}
Нельзя сравнивать. Никого и ничто.

\iusr{Оксана Тукалевская}
\textbf{Наталия} 

я не сравниваю. Я констатирую факт. Сквер имеет отношение к писателям (разным)
- тут кругом дома литераторов, но, увы, не к Ступке. Искусственное предложение.
\end{itemize} % }

\iusr{Тетяна Іванченко}
Ну да: переименуют, смогут тырить из бюджета на табличках, бланках... и Ступка тут ни при чём.

\begin{itemize} % {
\iusr{Максим Мельничук}

Тирять вони на бруківці, мостах, метро мільярди. На канцелярії не особливо
"натириш". Тут навіть адресу змінювати нікому не доведеться. Просто на мапах та
пару таблиць

\end{itemize} % }

\iusr{Oksana Khorozova}

Олеся Гончара - не той!! Не плутати, бо я там виросла. Сквер імені Олеся
Гончара раніше називався імені Зої Космодем'янської!! Він знаходиться навпроти
гастроному!! Там і пам'ятник їй ...

А цей ( колишній Чкалова) трохи нижче...

\begin{itemize} % {
\iusr{Oksana Khorozova}

Пам'ятник Зої Космодем'янській — пам'ятник герою Радянського Союзу Зої
Космодем'янській, розташований у Києві на розі вул. Олеся Гончара та вул.
Богдана Хмельницького.

Тепер там пам'ятник ОЛЕСЮ ГОНЧАРУ.

\ifcmt
  ig https://scontent-lga3-1.xx.fbcdn.net/v/t1.6435-9/255347709_1065651050868294_507431070532383373_n.jpg?_nc_cat=108&ccb=1-5&_nc_sid=dbeb18&_nc_ohc=fcmakHwPezsAX-Z4Zmc&_nc_ht=scontent-lga3-1.xx&oh=c8f905472a47cf233d5d505ba00debbe&oe=61B0EEE3
  @width 0.4
\fi

\begin{itemize} % {
\iusr{Зоя Луценко}
\textbf{Oksana Khorozova} А был ещё памятник на Подоле, рядом с Красной площадь в скверике

\iusr{Oksana Khorozova}
\textbf{Зоя Луценко} Правда? Не знаю... Красную площадь на Подоле тоже не помню... Только в Москве есть такая)))))

\iusr{Зоя Луценко}
\textbf{Oksana Khorozova} сейчас Контрактова, стоял памятник до 90-х

\iusr{Svetlana Batrak}
\textbf{Oksana Khorozova} 

вы очень ошибаетесь!!!!! Памятник Зое Космодемьянской и ее скверик по прежнему
существуют и никто их не переименовывает. Здесь никогда не было гастронома. А
памятник Олесю Гончару находится в сквере напротив Чкаловского садика и
напротив гастронома.

\iusr{Oksana Khorozova}
\textbf{Svetlana Batrak} 

Так і я так написала! Маємо 2 парки. Перший, ( льотчика), а другий - Олеся
Гончара. Там 2 памятника: О. Гончару та Зої Космодем'янської, який навпроти
гастроному, що знаходиться в письменницькому будинку ( Коцюбинського, 2). Щось
ми одна одну не зрозуміли...))

Знести хочуть парк імені льотчика. Правильно? Оцей на фото.

\ifcmt
  ig https://scontent-lga3-1.xx.fbcdn.net/v/t1.6435-9/254756051_1065665214200211_4702448280431523286_n.jpg?_nc_cat=109&ccb=1-5&_nc_sid=dbeb18&_nc_ohc=bF8z8MQ2-1EAX_WrMKM&_nc_ht=scontent-lga3-1.xx&oh=7d0c8378b505e8e1d8c0c02686e0a96a&oe=61B0073E
  @width 0.4
\fi

\iusr{Svetlana Batrak}
\textbf{Oksana Khorozova} 

нет) там, где памятник Гончару нет памятника Зое). Памятник Зое в другом
скверике - на углу Гончара и Хмельницкого) Сносить скверики пока никто не
собирается. Речь шла о переименовании Чкаловского садика/сквера (там, где
памятник Чкалову) в сквер им. Ступки. Но голосование по этому поводу ещё
продолжается

\iusr{Oksana Khorozova}
\textbf{Svetlana Batrak} 

Видимо понятие "летчик" Вам, Светлана, не понятно... Или еще что-то еще....

\iusr{Svetlana Batrak}

Вы пишите об одном сквере, а фото с Зоей из другого). Действительно, раньше в
сквере, где сейчас Гончар, стоял памятник Зое, давно это было, я там тоже
выросла). Но его перенесли в маленький скверик чуть ниже, откуда ваша фото.
Т.е. сначала сквер гончара, он же писательский, напротив гастронома. Чуть ниже
- Чкаловский, и чуть ниже - Зои. С этими вечными переименованиями
киевлянам-старожилам, чтобы понять друг друга, приходится теперь докапываться
до первоначальных названий и что там было)

\iusr{Oleg Chorny}
\textbf{Oksana Khorozova} 

Контрактова площа на Подолі таки дійсно колись називалася "Червона" (Красная).
В усякому разі в 1970х точно. А от коли їй повернули назву Контрактова не
пам'ятаю, здається в перші роки Незалежності.

\iusr{Svetlana Batrak}
\textbf{Oksana Khorozova} 

пожалуйста, прежде, чем переходить на уничижительный тон, прочтите мой
комментарий. Культура умения вести дискуссию (а здесь речь даже не о дискуссии,
а о расположении объектов), на мой взгляд, должна присутствовать.

\iusr{Oksana Khorozova}
\textbf{Oleg Chorny} 

А я не знала... Виявляється була і не одна! Знайшла в Гуглі: Колишні назви: Олександрівський майдан;
Червоний майдан;
Запорізька площа.

\iusr{Oksana Khorozova}
\textbf{Svetlana Batrak} Разве Зою К. перенесли куда-то?? Но єто уже детали...

\ifcmt
  ig https://scontent-lga3-1.xx.fbcdn.net/v/t1.6435-9/255575380_1065677050865694_8470067698526115792_n.jpg?_nc_cat=103&ccb=1-5&_nc_sid=dbeb18&_nc_ohc=a-d4xEoZz7oAX--7407&_nc_ht=scontent-lga3-1.xx&oh=e89f034486cdac2cf3660290c8ee4a9d&oe=61B089F5
  @width 0.4
\fi

\iusr{Svetlana Batrak}
\textbf{Oksana Khorozova} ооооо)))) Оксана! Зоя здесь. Но здесь нет гастронома! Выше Чкаловский, ещё выше - Гончара, он же писательский, возле гастронома

\iusr{Oksana Khorozova}
\textbf{Svetlana Batrak} 

ПАРК имени ОЛЕСЯ ГОНЧАРА находится на перекрестке ул. Чапаева и М.
Коцюбинского. ФОТО: памятник ОЛЕСЮ ГОНЧАРУ... Если стоять на сступеньках, лицом
к памятнику, то гастроном - справа.

\ifcmt
  ig https://scontent-lga3-1.xx.fbcdn.net/v/t1.6435-9/255347712_1065702140863185_6694746647180867904_n.jpg?_nc_cat=102&ccb=1-5&_nc_sid=dbeb18&_nc_ohc=F4ZZdHT-81gAX8D4EX3&_nc_ht=scontent-lga3-1.xx&oh=6ecc666082cb22e945af348a5747f4fc&oe=61AFE070
  @width 0.4
\fi

\iusr{Svetlana Batrak}
\textbf{Oksana Khorozova} я знаю) об этом вам и писала

\iusr{Oksana Garnets}
\textbf{Oksana Khorozova} Гастроном напротив!

\end{itemize} % }

\end{itemize} % }

\iusr{Ines Kyiv}

Принципіальне значення має одне - чи назва увіковічує історію України, чи
історію срср. В якій країні людина ментально живе, ті цінності вона і
зберігатиме. Ще є прошарок людей в суспільстві який дуже турбується про кошти.
Їх можно заспокоїти. Ціна на таблички закладена в суму "обслуговування
территорії".

\begin{itemize} % {
\iusr{Larysa Karbysheva}
\textbf{Ines Kyiv} Увековечивем память о герое!! А при чем тогда Маккейн ? Или была мысль Воздухофлотский в Немцова...

\begin{itemize} % {
\iusr{Ines Kyiv}

Не знаю как объяснить взрослому человеку, если он киевлянин и патриот конечно,
какой вклад в историю Украины внёс Маккейн и почему именем правозащитника
Немцова хотели назвать Воздухофлотский.

\iusr{Анна Матулевская Финкус}
\textbf{Ines Kyiv} Маккейн столько сделал.... "неоценимый вклад". И Немцов, кстати, тоже.

\iusr{Максим Мельничук}
\textbf{Ines Kyiv} 

Взагалі то Джон Маккейн був затятим нашим союзником в США і просував підтримку
та допомогу Україні у війні з Росією. Борис Нємцов - адекватний політичний діяч
на Росії, який був проти війни з нами. Маккейн - більше заслуговує, Нємцов -
фігура значно меншого масштабу, щоб цілий проспект перейменовувати

\iusr{Oksana Khorozova}
\textbf{Larysa Karbysheva} Не меняем тему, а то забудем о чем речь была...

\iusr{Ines Kyiv}
\textbf{Maksym Melnychuk} 

Згодна. Але На Воздухофлотському знаходиться посольство Росії,саме через це і
з'явилась така пропозиція. Зробили сквер Немцова принаймні. Сплюндрували плиту
з його ім'ям ((( Але адмінінстрація на звернення оперативно відреагувала і
плиту одразу почистили.

\end{itemize} % }

\iusr{Елена Корниенко}
\textbf{Ines Kyiv} 

А в паспортах не надо изменения вносить? А СССР тоже часть истории. Я не
защитница совка и не против переименования, но так как вы называть людей
прошарком, как минимум невоспитаность.

\begin{itemize} % {
\iusr{Ines Kyiv}
\textbf{Елена Корниенко} ))) Это комментарии в сети. Они краткие. Если кого-то обидело слово "прошарок", я изменю на "прошарок людей в суспільстві ". Это корректно и нейтрально.

\iusr{Ines Kyiv}
\textbf{Елена Корниенко} В паспотах название сквера вносить не будут. Что касается названий улиц, то эти изменения вносят только при замене паспорта. Вся улица не должна в очереди стоять , что паспорта поменять.
\end{itemize} % }

\iusr{Оксана Волкова}
\textbf{Ines Kyiv} 

может лучше всё-таки обслуживать территорию, а не таблички менять? И как бы вам
это не было противно, но СССР - это часть истории Украины

\begin{itemize} % {
\iusr{Ines Kyiv}
\textbf{Oksana Volkova} 

Есть люди "рождённые в ссср", а есть "рождённые в Украине, которую захватили
большевики". Эти большевики устроили красный террор, Голодомор, планомерно
уничтожали украинскую интеллигенцию и украинский язык. Сделали до начала 70-х
колхозников невыездными из их сёл и разграбили и уничтожили сотни ! древних
церквей и соборов по всей Украине.

\iusr{Елена Корниенко}
\textbf{Ines Kyiv} 

Интересно вы вообще ДЕЛИТЕ людей. Уничтожали церкви везде по всей
стране, уничтожали татар, русских, украинцев, немцев, которые жили в СССРе, миллионы
людей пострадали от нквд, и кстати одно из самых зверских нквдшных начальств
были в Киеве. И дело не в Ссср, как таковом, а в идеологии и тех, кто эту идеологию
огнём и мечом в людей вгонял. То, что КПСС у нас запрещена и осуждена это как раз
правильно. Намешано много чего. Одних стреляли, другие строчили доносы, третьи
стреляли. Кого вы сейчас готовы судить? или карать?

А были великие люди, которые двигали историю и в то время не играла роли
национальность и ваше деление. Это делали, потому что верили в чистые и светлые
идеалы. И не они их замарали чужой кровью. Скорее своей закрасили.

\iusr{Ines Kyiv}
\textbf{Елена Корниенко} 

ссср как раз создавалась, была построена и поддерживалась идеалогией террора.
Все как будто бы и "братья" )) во в это "братство" заганяли репрессиями. Тех,
кто не желал - в лагеря или на расстел. Или иначе было ? ... Нет отдельно
"хорошего ссср" и " зверств нквдистов".

\end{itemize} % }

\iusr{Жанна Сочивец}
\textbf{Ines Kyiv} 

є назви, які увіковічують повагу до людей, яким вдячний весь світ або якими
весь світ пишається незалежно від країни їх походження, її політичного режиму
та пропаганди

\begin{itemize} % {
\iusr{Ines Kyiv}
\textbf{Zhanna Sochivets} Чим вдячний весь світ Чкалову ?

\iusr{Анна Матулевская Финкус}
\textbf{Ines Kyiv} он сделал то, что до него никто не делал первооткрыватель!!!!!!

\iusr{Natalia Nikolayenko}
\textbf{Ines Kyiv} , 

Чкалов вніс значний вклад у розвиток авіації. А коли народитися , він не
вибирав. І те, що зараз відбувається, спекуляція на імені Ступки. Знайшли
привід стерти пам‘ять і від застройки частини скверу відволікти. Краще б
відновили будинок Сікорських і зробили там музей.

\iusr{Жанна Сочивец}
\textbf{Ines Kyiv} углубитесь в историю вопроса

\iusr{Ines Kyiv}
\textbf{Zhanna Sochivets} Я в связи с серией постов на тему переименования сквера, биографию Чкалова скоро наизусть выучу )

\iusr{Елена Корниенко}
\textbf{Ines Kyiv} Тем же ,чем Амелии Эрхарт.
На то время такое считалось подвигом и эти люди были знамениты.

\iusr{Natalia Nikolayenko}
\textbf{Елена Корниенко} ,да, они были первооткрывателями.
\end{itemize} % }

\iusr{Георгий Горбенко}
\textbf{Ines Kyiv} ПРИНЦИПОВО проти. Будуйте нові.

\begin{itemize} % {
\iusr{Ines Kyiv}
\textbf{Георгий Горбенко} Ваша позиция понятна, товарищ )

\iusr{Оксана Волкова}
\textbf{Ines Kyiv} а строительство нового в корне противоречит вашей теории? Удобнее, конечно, пользоваться тем, что построено при «проклятом совке», только и заботы, что переименовывать.
\end{itemize} % }

\end{itemize} % }

\iusr{Владислав Зимовский}
Верните первое название улице.

\begin{itemize} % {
\iusr{Oksana Khorozova}
\textbf{Владислав Зимовский} 

Какое названий улице имеете ввиду?( Хотя речь о парке!!)

Из истории:

"Вперше згадується у 1834—1836 роках під назвою Володимирська вулиця, на той
час пролягала між сучасними вулицями Великою Житомирською та Рейтарською. У
1840-х — першій половині 1850-х років набула назву Мала Володимирська
(Маловолодимирська) вулиця, на відміну від сучасної Володимирської вулиці. У
1850-х роках вулицю було продовжено через яр, що тривалий час був місцем
звалища побутових відходів (цим спричинена виняткова повільність забудови
вулиці, аж до кінця XIX століття). У 1871 році вулиця так і залишалася не
замощеною до кінця. Оскільки вона різко спускалася вниз, за декілька років від
стоку води посеред неї утворився яр.

Назву Столипінська вулиця отримала 1911 року, на честь російського політичного
діяча Петра Столипіна, який після смертельного поранення помер у клініці,
розташованій на цій вулиці. З 1919 року — вулиця Гершуні на честь російського
революціонера есера Григорія Гершуні.

З 1937 року — вулиця Ладо Кецховелі на честь російського революціонера
більшовика Ладо Кецховелі. Назву вулиця Чкалова, на честь радянського льотчика
Валерія Чкалова вулиця набула 1939 року підтверджена 1944 року.

Під час німецької окупації міста у 1942—1943 роках мала назву вулиця
Антоновича[8], на честь українського історика, професора Київського
університету святого Володимира Володимира Антоновича.

Сучасна назва на честь українського письменника, громадського і культурного
діяча Олеся Гончара — з 1996 року.

Вулиця Олеся Гончара розташована у буферній зоні Софії Київської. Однак 2008
року тут розпочали незаконне зведення ЖК «Фреско Софія». Боротьба активістів
проти забудови тривала одинадцять років, поки 2019 року Верховний Суд остаточно
не заборонив будівництво на охоронній ділянці!!"

\iusr{Maksim Pestun}
\textbf{Владислав Зимовский} строительство запретили( и правильно) только вышло, как и обычно, ее хуже. Теперь там стоит ещё один разваливающийся монстр, перекрыв пол улицы...
\end{itemize} % }

\iusr{Андрей Крамаренко}
Так Ладо Кецхвели это Воздвиженская.

\iusr{Жанна Сочивец}
Мы не умеем ценить замечательных людей @igg{fbicon.face.sad.but.relieved}  Важнее отмывать бабло на новых табличках

\iusr{Oleh Domashevskyi}

А хто взагалі запустив інфу про те, що сквер ніби-то перейменували на Чикаленка
чи Ступки?! Це тільки-но винесли на обговорення і голосів проти саме такого
перейменування вже мало не вдвічі більше

\begin{itemize} % {
\iusr{Bereza Belaya}
\textbf{Oleh Domashevskyi} Ви ж начебто освічена, розумна людина. А пишете якусь фігню, викручуваючи прізвище Чкалова. Недостойно
\end{itemize} % }

\iusr{Алла Вайнерман}
Через 20 лет мало кто вспомнит, кто такой Ступка. А Чкалов уже доказал свое право на память.

\begin{itemize} % {
\iusr{Євген Суходуб}
\textbf{Алла Вайнерман}, зачем нести чушь?

\iusr{Max Gopencko}
\textbf{Алла Вайнерман} культурні люди точно пам'ятатимуть)

\iusr{Лидия Чуватина}
\textbf{Алла Вайнерман} , 

да дело уже не в Ступке и Чкалове, а в Киевлянах, которые родились и выросли на
этих улицах и парках. Приехали в Киев люди и начали все рушить и
переименовывать, много есть новых мест- стройте , называйте, история и есть
история....

\begin{itemize} % {
\iusr{Max Gopencko}
\textbf{Лидия Чуватина} вы какого поколения киевлян имеете ввиду, если улица переименовывалась минимум 8 раз и с не такого далёкого 96-го её должны помнить как улицу Гончара?

\iusr{Валентина Крутова}
\textbf{Лидия Чуватина} а знаете где в Киеве огороды Берлиозовы? того Берлиоза что в мастере и маргарите?

\iusr{Лидия Чуватина}
\textbf{Max Gopencko} , так вот в этом и проблема...

\iusr{Max Gopencko}
\textbf{Лидия Чуватина} 

нет, корень проблемы не в этом, а в том, что улицы называют именами людей,
которые теряют актуальность для времени или места - отсюда и постоянные
переименования. В этом контексте Ступка подходит для названия киевской улицы
явно больше. А вообще, я сторонник того, дабы переименовать все или большинство
улиц в числа или дать им названия соответственно важных объектов, которые на
них находятся. Тогда в будущем потеряется смысл менять названия. Называть улицы
именами военных или политиков - это плохая идея в долгосрочной перспективе.

\iusr{Валентина Крутова}
\textbf{Лидия Чуватина} 

Киевлянах, которые родились и выросли на этих улицах строили страну СССР,
переименовывали своими героями улицы, рушили церкви, декламировали стихи я
русский бы выучил только за то, что им разговаривал Ленин.  Когда Октябрь
орудийных бурь по улицам кровью ли́лся, я знаю, в Москве решали судьбу и Киевов
и Тифлисов.

Кстати, откуда и когда ваши предки в Киев приехали?

\iusr{Алла Вайнерман}
\textbf{Лидия Чуватина} Согласна целяком и полностью.

\iusr{Алла Вайнерман}
\textbf{Max Gopencko}. Коренных

\end{itemize} % }

\iusr{Сергей Мироненко}
\textbf{Алла Вайнерман} 

Театральные люди будут помнить Ступку долго, летающие, серьёзные, люди Чкалова будут знать очень долго

\begin{itemize} % {
\iusr{Max Gopencko}
\textbf{Сергей Мироненко} Ступка зіграв стільки ролей в кіно, що його будуть пам'ятати навіть ті, хто жодного разу не був у театрі.
\end{itemize} % }

\iusr{Тоня Олешко}
\textbf{Алла Вайнерман} 

через 20 лет наши дети Ступку будут помнить, как прекрасного актёра, так же как
помним мы актёров, которых уже нет с нами, но творчество их живет. А про
Чкалова дети наши уже ничего не знают, пройдёт с десяток лет, пенсионеров
нынешних не будет, и о этом человеке никто и не вспомнит.

\begin{itemize} % {
\iusr{Валентина Крутова}
\textbf{Тоня Олешко} ну почему же, в школе будут учить и о Чкалове, и о Амундсене, Беллинсгаузене и о Колумбе...

\iusr{Тоня Олешко}
\textbf{Валентина Крутова} о Чкалове может и будут упоминать в связи с перелетом, но для Киева эта личность ничего не значит. Ступка более известен и близок для киевлян

\iusr{Алла Вайнерман}
\textbf{Тоня Олешко}. Вопрос?

\iusr{Алла Вайнерман}
Странно, что никого не смущает, что Ступка много снимался в Москве. Как всегда политика двойных стандартов.

\iusr{Валентина Крутова}
\textbf{Тоня Олешко} Ступка - важен для Киева, Чкалов - история

\iusr{Алла Вайнерман}
\textbf{Тоня Олешко}. 

Объясните, зачем переименовать улицы? Стройте новые и называйте, как хотите.
Руки прочь от истории. Мы превращаемся в Иванов, не знающих родства. Позор!

\iusr{Тоня Олешко}
\textbf{Валентина Крутова} Ступка важен для Киева и для всей Украины. Чкалов - история несуществующей страны. При чем здесь сквер в центре нашей столицы?
\end{itemize} % }

\iusr{Елена Макухина}
\textbf{Алла Вайнерман} если через 20 лет мало кто вспомнит, кто такой Ступка - это признак вымирания нации...

\iusr{Алла Вайнерман}
\textbf{Елена Макухина}. Или политика двойных стандартов.

\iusr{Max Gopencko}
\textbf{Янина Ромова}, поделитесь, что смешного в моих комментариях нашли?

\iusr{Gennadiy Smertenko}

Масштабы этих личностей прямо скажем отличаются, все дело в коньюктуре рынка.

Надо было бы, назвали скверик имени Ким Чен Ира.

Давайте будем честными.

\begin{itemize} % {
\iusr{Max Gopencko}
\textbf{Gennadiy Smertenko} вы правы. Для нынешней Украины Богдан Сильвестрович - личность намного более масштабная.
\end{itemize} % }

\end{itemize} % }

\iusr{Svetlana Batrak}

Вот я думаю, зачем было запрещать комментарии на мою публикацию, которая
вызвала такой ажиотаж, чтобы опять вернуться к этой теме? Кстати, голосование
по поводу переименования сквера продолжаются в открытом доступе (я этого не
знала на момент публикации). Ссылку не могу дать, т.к. запрещено правилами
группы.

\iusr{Микола Веселий}

Я подробно читал про перелет Чкалова через Северный полюс - тогда это было по
мужеству сравнимо с первым полетом в космос. И мужество это Рузвельт оценил. А
вот некоторых других - президенты США не принимают, (фамилий по понятным
причинам называть здесь не стану), ибо цену им знают. Бывает, когда президенты
вынуждены встречаться с равными себе по статусу главами государств,, исходя из
государственных интересов. С отвращением, но встречаться. А встретившись со
Чкаловым, Рузвельт действительно отдал дань его мужеству

\iusr{Maksim Pestun}
\textbf{Валерий Чкалов} , 

один из Сталинских Соколов, в которых так нуждалась официальная пропаганда
ссср. Ничего плохого для Киева не сделал, впрочем, как и хорошего.
Переименовывать или нет?..Вопрос действительно непростой. Для многих киевлян,
как для моего отца, который вырос в соседнем доме и с детства облазил там
каждый метр и даже нашёл оружие на пригорке за садиком, это ностальгические
воспоминания. Для новых поколений это имя ни о чем не говорит. Ступка, конечно,
намного более знаковое имя для города, но он заслужил и более соответствующее
его истории место...

\begin{itemize} % {
\iusr{Валерий Винарский}
\textbf{Maksim Pestun} мЕНЯ НАЗВАЛИ В ЧЕСТЬ вАЛЕРИЯ чКАЛОВА, Я ЖИЛ НА чКАЛОВА 28 и когда улицу переименовали я понял - ничего хорошего от будущего в этой стране мне ждать не приходится
\end{itemize} % }

\iusr{Виктор Киркевич}
У меня есть открытка 1934 года, где изображен Чкалов на соревнованиях в Киеве.

\iusr{Sergey Chapaev}

да кого трогает. переимнуют, как кому захочется.  комарова переименовали в
гузара. второй к киеву не оносится, у первого тут предки похоронены на
берковцах. пофиг.

\iusr{Люала Пилипейко}

Деньги отмываются на переименовании только так., Причем не одной табличке с
наименованием улиц нигде нет. РАНЬШЕ они были на домах и номер дома там
указывался.

\iusr{Вера-Вета Колесник}

И сквер имени Ступки пусть будет , но зачем ? отменять названия в честь
Чкалова, что за время? мрак полный.

\iusr{Ирина Козина}

во всех наших переименованиях последних лет суть не в конкретных именах, а в
делении истории на "нашу, хорошую" и "чужую, плохую и враждебную". век назад
это уже было. идем шаг в шаг за теми, кого стремимся "забыть". почему-то никто
не желает понимать, что переменить историю невозможно - она уже состоялась

\iusr{Геннадий Бабарика}

Тот, редкий случай, когда можно вернуть историческое название улицы
Маловладимирская, и без всякого наименования зелёного островка! И прекратить
дрязги, и спекуляции! А то уже дошло абсурда всякие мизерные зелёные зоны
называть именными парками и пр. Это в лучшем случае мини сквер... Памятник
пусть себе стоит, хотя, по большому счёту, ему там не место... для Киева он ни
кто! И не надо притягивать за уши... сформированного пропагандистского идола
коммунистической эпохи...

\begin{itemize} % {
\iusr{Boris Shulman}
\textbf{Геннадий Бабарика} 

Не стоит так примитивно оценивать значение того или иного персонажа для нашего
города, мол "для Киева он ничего не сделал!". А святой Андрей - сделал? Святая
София - сделала? Лев Толстой? Пушкин? Киев - часть Мира, и те, кто улучшал Мир,
тот делал добро и Киеву. Но иногда, даже назвав какое-то место в честь
действительно достойного человека, о нём всё равно никто ничего не пишет, не
говорит, и люди о нём ничего не знают. Например, на Нивках назвали площадь
именем Валерия Марченко. И что, много о нём знают жители улиц Стеценко,
Щербаковской и т.д.? А ведь это был настоящий герой Украинской культуры! P.S.
В.Чкалов был образцом мужества и стал примером для сотен тысяч молодых людей (в
том числе и киевлян), которые освободили Киев, Украину и другие страны от
фашизма. Но это, конечно, никак не уменьшает значения замечательного актёра и
человека Богдана Ступки.

\iusr{Bereza Belaya}
\textbf{Геннадий Бабарика} 

ну так вы для Киева и киевлян тоже никто (пиши слитно!), и нечего вдувать в уши
свое предвзято мнение.

\iusr{Bereza Belaya}

Для названия улицы именем Богдана Ступки может быть подобрать другой объект. И
улицу и сквер (или площадь). Имя очень знаковое и для киевлян, и для Украины.

\end{itemize} % }

\iusr{Клим Форманчук}

Чкалов - легендарная личность, и достоин хотя бы того, чтобы не порочили и
унижали его имя никак не запятнанное, от слова "совсем", преступными действиями
против украинского народа, и памятник, как и сквер в том месте должен быть ! Но
и Богдан Ступка - Народный артист не только Украины, но и СССР ! И т.к. Кличко
не только застраивает всевозможными ТРЦ и ЖК зелёные зоны, но и открывает новые
скверы и парки, то и его, Богдана, имя можно "увековечить" в одном из них. Да
и улиц, и площадей "безымянных" в Киеве хватает. Главное что бы "комиссия по
декоммунизации" при киевской мэрии, состоящая почти вся из "западенской
наброды" и откровенного дурачья, не вспомнила, что Богдан Ступка ещё и Народный
Артист СССР и не лишила его этого звания. Вот будет "радость" его родным!

\iusr{Олена Потильчак}

Как на Кличка похож.

\begin{itemize} % {
\iusr{Клим Форманчук}
\textbf{Олена Потильчак} , никому больше не говорите об этом, а то ещё перед тем как "стать на лыжи" в сторону Гамбурга может и памятник под себя любимого "декоммунизировать" приделав к рукам боксёрки.
\end{itemize} % }

\iusr{Gennadiy Smertenko}
Масштабы этих личностей прямо скажем отличаются, все дело в коньюктуре рынка.
Надо было бы, назвали скверик имени Ким Чен Ира.
Давайте будем честными.

\begin{itemize} % {
\iusr{Max Gopencko}
\textbf{Gennadiy Smertenko} вы правы. Для нынешней Украины Богдан Сильвестрович - личность намного более масштабная. При чём тут к Ступке Ким Чен Ин - я если честно, совершенно не понял.

\begin{itemize} % {
\iusr{Олена Харченко}
\textbf{Max Gopencko} пройдуть десятиліття. Чкалов ввійшов в історію, а актори не всі

\iusr{Max Gopencko}
\textbf{Олена Харченко} пройдіться вулицями міста та попитайте людей, хто такий Чкалов і хто такий Ступка. Закладаюся, Богдана Сільвестровича знають майже всі, а от про Чкалова з молодого покоління вам навряд хтось відповість.

\iusr{Татьяна Гурьева}
\textbf{Max Gopencko} не факт

\iusr{Max Gopencko}
\textbf{Татьяна Гурьева} пройдіться, і зрозумієте, що факт.
\end{itemize} % }

\iusr{Татьяна Гурьева}
\textbf{Max Gopencko} в мене е дити, та в моих знайомих теж. Ти, яким розповидали - знають. А Тим, що не знають, то вони взагали мало знають. Бо книга не стала для них справжним другом

\begin{itemize} % {
\iusr{Max Gopencko}
\textbf{Татьяна Гурьева} ваші діти та діти ваших знайомих сильно у меньшості. Розповідати можна про різних особистостей і Чкалов явно не на часі.

\iusr{Irina Kolomiec}
\textbf{Татьяна Гурьева} на жаль так іє
\end{itemize} % }

\iusr{Татьяна Гурьева}
\textbf{Irina Kolomiec} для Вас) жалкую

\iusr{Татьяна Гурьева}
\textbf{Max Gopencko} саме у ций групи досить багато. Чкалов не на часи, Гагарин, велики люди та особистости не на часи?

\begin{itemize} % {
\iusr{Max Gopencko}
\textbf{Татьяна Гурьева} 

не маю нічого проти російсько-радянського Чкалова, лише констатую тенденції в
суспільстві, яке зараз вариться в котлі самостійної України, а не СРСР. А ця
група з купою радянських іммігрантів - не показник поточних столичних настроїв.
Особисто я, розповідаючи про когось дітям, почну точно не з Чкалова, а з
представників української культури, яка знищувалась, витіснялась і
замовчувалась, а зараз потребує відродження, так само як і перекручена
українська історія.

\end{itemize} % }

\iusr{Татьяна Гурьева}
\textbf{Max Gopencko} 

зараз у котли багато шо вариться. Не буду уточнювати. Кожний свое виймае. Герои
ризни. Але це вже не киивська история, вибачайте

\begin{itemize} % {
\iusr{Max Gopencko}
\textbf{Татьяна Гурьева} те що коїться у Києві - це саме київська історія. Навіть історія із Чкаловим, який жодного прямого відношення до столиці не має.
\end{itemize} % }

\end{itemize} % }

\iusr{Владимир Каледин}

Да какая разница, был ли в Киеве когда-то или нет! Было единое государство -СССР
! И он был достойным гражданином и героем этого государства! Может Бандера или
Шухевич что-то полезное сделали для Киева!? Или, тем более," контуженный на всю
голову" американец Маккейн!?

\begin{itemize} % {
\iusr{Ганна Лапа}
\textbf{Владимир Каледин} у Вас вдома на чільному місці стоїть портрет Чкалова.

\iusr{Максим Мельничук}
\textbf{Ганна Лапа} у нього там сталін з брєжнєвим та леніним в придачу

\iusr{Раиса Глывук}
\textbf{Ганна Лапа} а у вас иконостас дома из Бандеры, Шухевича и Шептицкого?

\iusr{Alex Silence}
Тєбя дараґой і юний чкаловєц, да в "Дальстрой", с кіркой, паработать. Єдіноє ґасударство у нєґо било...

\iusr{Раиса Глывук}
\textbf{Alex Silence} было государство. И ты выходец из него. Не забывай.

\iusr{Игорь Скоробогатов}
\textbf{Владимир Каледин} ,есть ли жизнь на марсе, нет ли жизни на марсе - не известно!
Да и какая разница!
\end{itemize} % }

\iusr{Оксана Куліченко}
Чкалова додому. Я "за" повернення історичної назви.

\iusr{Елена Царицына}

Вы когда-либо слышали о переименовании улиц и парков в Европе? Думаю нет. А
переименование несём за собой большие затраты из бюджета. Это лишний повод
стырить городские средства и пополнить карманы депутатов киеврады. Может пора
остановиться и прикрутить это механизм.

\begin{itemize} % {
\iusr{Игорь Гаврилов}
\textbf{Елена Царицына} В Європе нет необходимости в декомунизации.

\iusr{Елена Царицына}
\textbf{Игорь Гаврилов} 

Ещё дебильная тема декомунтзация! Мой дед был коммунистом , который прошёл всю
войну и потом сделал много хорошего для страны, так, что я должна от него
отречься? Чкалов тоже получается никто и ничто, кто не помнит и не уважает
прошлое этой хренью и занимаеться!

\iusr{Oleg Ivanenko}
\textbf{Елена Царицына} 

В Германии (это Европа) БЫЛО массовое переименование улиц в 1945-47. При этом
переименовывали не только улицы с именем Гитлера, но и улицы с именем
заслуженного летчика Геринга (ну стал он главным в люфтваффе, а Чкалов не успел
в ВВС). И это делалось, когда Германия была в руинах...

\iusr{Наталия Вольская}
\textbf{Oleg Ivanenko} , вы великого летчика с фашистом гитлером сравниваете?

\iusr{Alex Yam}
\textbf{Елена Царицына} ,умница

\iusr{Oleksiy Boldyriev}
Так, регулярно перейменовують. Не тільки вулиці та будівлі, але й клітини й органи людського організму.

\iusr{Oleg Ivanenko}

Не с Гитлером, а с Герингом. Тоже прославленный ас Первой мировой, только в
карьере пошёл дальше Чкалова. Для страны сделал много, его люфтваффе помогли
присоединить к рейху Чехию и половину Польши

\iusr{Игорь Гаврилов}
\textbf{Елена Царицына} 

Коммунисты и компартия в Украине запрещены как преступная организация. Страна
которой служил ваш дед и много пользы ей принес была враждебной к украинцам и
нашей стране в целом. Пусть все они остаются в прошлом, а настоящее должно
принадлежать новым поколениям не зараженным вирусом зла.

\begin{itemize} % {
\iusr{Елена Царицына}
\textbf{Игорь Гаврилов} 

Ну за то новая власть пришедшая к рулю была чиста и не порочна!!! Потому у нас
пропали сбережения с сберегательны банков и живём мы так классно и не
враждебно!!! И конечно навык вожди ни Кравчук ни Кучма и остальная шваль не
были членами компартии Украины!! Так в прошлом как Вы говорите мы росли и
учились, так хоть что-то знаем и умеем. А нынешнее поколение будет знать, что
можно постоянно менять названия улиц и писать маразматические диктанты всей
страной, текст которых соответствует уровню сочинения 3 класса Советской школы!

\end{itemize} % }

\end{itemize} % }

\iusr{Natalia Pigulevska}

На этой улице прошло мое детство, у моей бабушки - там в 5 комнатной квартире в
24 доме, во дворе, она жила со своей сестрой, генеральшей, бывшей актрисой
театра Франко. В огромной квартире мы встречали новый год и я просыпалась под
хруст снега на морозных ветках. У нас это так и называлось "пойти на Чкалова" и
другого названия, конечно, не будет для нас. Однако Столыпин, умерший в
госпитале чуть выше бабушкиного дома, был тоже очень достойный человек, и
название улицы своим именем бы украсил. Но только это невозможно в сегодняшней
Украине, пытающейся изо всех сил забыть своё прошлое...

\begin{itemize} % {
\iusr{Татьяна Гурьева}
\textbf{Natalia Pigulevska} как красиво Вы написали) напишите свою Киевскую историю

\iusr{Natalia Pigulevska}
\textbf{Татьяна Гурьева} спасибо  @igg{fbicon.sun.with.face} 

\iusr{Oleg Ivanenko}
\textbf{Natalia Pigulevska} Это Чкалов и Столыпин - "своё прошлое" Украины???

\begin{itemize} % {
\iusr{Роксолана Рудяка}
\textbf{Oleg Ivanenko} именно. А что смущает?

\iusr{Oleg Ivanenko}
\textbf{Роксолана Рудяка} Да Чкалов вообще в Киеве не был, а памятник Столыпину ещё в марте 1917 (ДО комуняк) снес возмущенный киевский народ, выразив свою любовь

\iusr{Natalia Pigulevska}
\textbf{Oleg Ivanenko} народ и революцию устроил, а потом чуть с голоду весь не передох... Народ много чего натворил вообще-то

\iusr{Natalia Pigulevska}

А что касается Столыпина - его убили в киевском оперном театре, поэтому,
конечно, он часть истории Киева (а Киев это вроде как Украина последние лет
200). Ну и про Столыпина вообще почитайте - большой был человек, жив был бы -
не было бы революции и всего последовавшего террора. Поэтому его и убили.
Сильно кому-то это было нужно, увы.

\iusr{Данил Бенатов}
\textbf{Natalia Pigulevska} це все масони.

\iusr{Люба Микицька}
\textbf{Roxolana Rudyaka} 21 століття наче.

\iusr{Дмитрий Артюк}
\textbf{Oleg Ivanenko} Бандеры в Киеве тоже не было, а памятник на Куреневке поставили.

\iusr{Oleg Ivanenko}
\textbf{Natalia Pigulevska} С таким подходом и выборы не стоит проводить, а малейшее инакомыслие - карать, как это делал Столыпин

\iusr{Роксолана Рудяка}
\textbf{Oleg Ivanenko} это противоречит истории нашеи страны?!

\iusr{Роксолана Рудяка}
\textbf{Люба Микицька} і?!

\iusr{Oleg Ivanenko}
\textbf{Дмитрий Артюк} ГДЕ памятник? Фото! Или хоть адрес назовите!

\iusr{Oleg Ivanenko}
\textbf{Дмитрий Артюк} Где памятник Бандере в Киеве? Фото ! Или хотя бы адрес !

\iusr{Oleg Ivanenko}
\textbf{Natalia Pigulevska} То-то петлю виселицы называли тогда "столыпинским галстуком"... Большой был человек...
\end{itemize} % }

\iusr{Вадим Горбов}
\textbf{Natalia Pigulevska} 

Очень противоречивая, хотя бесспорно яркая, историческая фигура, особенно в
нынешних реалиях. Хоть могилу его не трогают - и то уже хорошо.

\end{itemize} % }

\iusr{Вадим Сандино}
Очень жаль, что Киевские власти цепляются за фамилии... Мало кто владеет языком украинским или русским!

\iusr{Lena Pavlikova}
В Гостомелі під Києвом є пам'ятник В. Чкалову біля прохідної "Авіаліній Антонова". Доречний.

\iusr{Светлана Блаус}

Дело даже не в том, чьё имя достойно увековечить в названии улицы, просто
каждое переименование перечеркивает предыдущее, автоматом отбрасывая на свалку
истории достойные имена. Почему не Гончар? Уж не говорю о Чкалове и Столыпине?
Я должна как то обьяснить своим детям, почему нужно зубрить новый адрес?

\begin{itemize} % {
\iusr{Margaryta Kuznietsova}
\textbf{Светлана Блаус} улицу не собираются переименовывать. Речь идёт о сквере в Шевченковском районе.

\iusr{Светлана Блаус}
\textbf{Margaryta Kuznietsova} 

Я понимаю, но там стоит памятник Чкалову, я рассказывала детям ( да и нам в
школе, по адресу этой улицы) рассказывали о его героизме. Почему я должна
объяснять, что чудесный актер Ступка достойнее героя- летчика Чкалова. И
вообще, в чкаловский сквер гулять мы не идем, идем гулять в ступкинский садик,
или сквер ступки?

\end{itemize} % }

\iusr{Наташа Котляренко}

Ступка жил на улице Станиславского..... Кто-то знает, какое отношение
Станиславский имеет к Киеву? А Ступка каждый день, по нескольку раз проходил по
этой улице.... На этой улице выросли его внуки..... и он с ними гулял....

\begin{itemize} % {
\iusr{Світлана Святюк}
\textbf{Наташа Котляренко} 

то давайте вже Станіславського зразу перейменуємо в імена внуків.. У Вас дуже
своєрідні поняття Хто такий Станіславський?? Для Києва?? А хто такий Гавел, чи
Маккейн, чи з сотню інших прізвищ по відношенню саме до Києва??


\iusr{Наташа Котляренко}
\textbf{Світлана Святюк} а каким боком Мастер к улице Чкалова- Гончара? К скверику на Гончара?

\iusr{Наташа Котляренко}
\textbf{Світлана Святюк} и кто знает, может быть когда нибудь будут искать улицу или сквер для имени внука. Гены гениальности истребить нельзя.....
\end{itemize} % }

\iusr{Maksim Pestun}

Почему не вернуть всем старинным киевским улицам первоначальные названия? Чем
плоха Малая Владимирская?...

\begin{itemize} % {
\iusr{Полина Карпенко}
\textbf{Maksim Pestun} а я мечтаю о Фундуклеевской (чтоб вернули)

\iusr{Maksim Pestun}
\textbf{Полина Карпенко} это было бы более чем справедливо!
\end{itemize} % }

\iusr{Изольда Арбенина}
Воздушный хулиган! Сорви- голова!

\iusr{Олег Сошевский}
\textbf{Изольда Арбенина} ,под мостом пролетел и никто не смог повторить!  @igg{fbicon.smile} 

\iusr{Світлана Святюк}

Треба взагалі вулицям давати нейтральні назви А увіковічувати пам'ять про
когось в установах, де вони працювали, в музеях.

\iusr{Алла Волонтырец}
\textbf{Світлана Святюк}
Маловолодимирська!

\iusr{Valeriy Levitskiy}

Запомнилось: 1991 на работе, в пятницу, обсуждали с сотрудниками будущее
Украины. Я самый молодой, остальные прожившие, повидавшие много мужики тогда
45-60 лет. Говорили разное, но смысл сходился на одном. Нужна смена поколений,
придут новые с чистыми мозгами, не забитыми советскими догмами, и все будет ок.
Ну вот пришли... Читаю комменты ,те же споры что и тогда, те же аргументы, чуть
ли не передовицы из газеты Известия цитируются, или наоборот, дикая злоба к
оппонентам прорывается, агрессия оскорбительные намеки. Ничего нового,
конструктивного. Риторические ответы, такие же вопросы. А прошло то 30 лет!!! И
сколько событий, нет: Чкалов, не Чкалов, Речкалов ,Кожедуб (украинец 3 раза
Герой ), Ковпак, Кирпонос, Грушевский, Хмельницкий, Тарас Бульба, итд итп итд
итп итд итп.............

\begin{itemize} % {
\iusr{Максим Мельничук}
\textbf{Valeriy Levitskiy} так було скрізь в усі часи в усіх країнах

\iusr{Dmytro Nohal}
\textbf{Valeriy Levitskiy} изначально ничего не могло стать иначе, кравчук, марчук, звягильский, на манеже все теже. Хотеть от яблони груш)))))

\iusr{Ольга Яворовская}
\textbf{Dmytro Nohal} На осинке не родятся апельсинки!!!!

\iusr{Наташа Котляренко}
\textbf{Valeriy Levitskiy} все возвращается на круги своя..... Спираль жизни.....
\end{itemize} % }

\iusr{Андрій Котлярчук}
Да і бог з ним. У нас свої герої

\begin{itemize} % {
\iusr{Анжелика Ангел}
\textbf{Андрій Котлярчук} У вас- может быть.)))
\end{itemize} % }

\iusr{Надежда Лабик}

Ни улица, ни сквер им. Чкалова никому не мешал. Чкалов - достойный человек, летчик.
Это - факт. А именем Б. Ступки можно назвать улицу, где он жил или ту, где он
работал.

\begin{itemize} % {
\iusr{Олена Потильчак}
А что сделал для Киева Чкалов? Он, вообще здесь когда-нибудь был?
\end{itemize} % }

\iusr{Ирина Голодович}
Я за Чкалова

\iusr{Vadim Vadim}
Пусть построят новые улицы и назовут именами новых героев. Это будет логично.

\iusr{Алексей Чкалов}

Скоро совсем забудут Валерия Павловича. Жаль, личность, безусловно, героическая.

Но хоть мне перестанут задавать тот самый дурацкий вопрос. Нет! Не родственник!  @igg{fbicon.wink} 

\iusr{Олег Сошевский}

Все это делается для того чтобы народ отвлёкся от насущных проблем.

Скоро со следующего года, за газ будут брать деньги по другой схеме. Более
воровской схемы ранее не придумали. Все кивают на Европу. Кубические метры со
счётчика переведут в киловатт/ часы. Естественно поставщик будет сам определять
энергоэффективность, на сколько разбодяжен газ и все это только на основании
количества потребленного голубого топлива в кубах! Нобелевскую премию нужно
вручить авторам!

\begin{itemize} % {
\iusr{Kira Dva}
\textbf{Олег Сошевский} +++
\end{itemize} % }

\iusr{Виталина Киев}
Мне больше всего обидно за Ванду Василевскую. Нравится она мне как писатель

\iusr{Николай Каплунов}
На Кличко смахивает.

\iusr{Marina Ester}
А улица продолжает называться Чкалова?

\end{itemize} % }
