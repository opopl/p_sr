% vim: keymap=russian-jcukenwin
%%beginhead 
 
%%file 10_02_2021.stz.news.ua.mrpl_city.1.grecka_teatralna_studia
%%parent 10_02_2021
 
%%url https://mrpl.city/blogs/view/gretska-teatralna-studiya
 
%%author_id demidko_olga.mariupol,news.ua.mrpl_city
%%date 
 
%%tags 
%%title Грецька театральна студія
 
%%endhead 
 
\subsection{Грецька театральна студія}
\label{sec:10_02_2021.stz.news.ua.mrpl_city.1.grecka_teatralna_studia}
 
\Purl{https://mrpl.city/blogs/view/gretska-teatralna-studiya}
\ifcmt
 author_begin
   author_id demidko_olga.mariupol,news.ua.mrpl_city
 author_end
\fi

\ii{10_02_2021.stz.news.ua.mrpl_city.1.grecka_teatralna_studia.pic.1}

Далеко не всім жителям Приазов'я відомо, що в Сартані вже 21 рік існує
самобутній та колоритний колектив – \emph{\textbf{Грецька театральна студія}}. Вона була
створена у \emph{2000 році} завдяки зусиллям та ентузіазму відомої активістки
національного руху, однієї з ініціаторок відродження грецької культури в
Приазов'ї, членкині Президії ГО \enquote{Союз греків України} \emph{\textbf{Гайтан Марії Георгіївни}}.
Сьогодні театр – єдиний з країн колишнього СНД, який входить в каталог театрів
грецької діаспори світу, єдиний із зарубіжних театрів, який у вересні 2019 року
брав участь в грецькому театральному фестивалі в Афінах, і єдиний, що ставить
вистави давньогрецьким румейським діалектом. Театр є наступником першого
\emph{Державного грецького театру в СРСР} який був створений у 1932 році в Маріуполі і
проіснував до 1937 року.

Спочатку колектив відчував голод як в репертуарі, так і в акторах. Акторській
майстерності потрібно було вчитися прямо на сцені, без вчителів і досвіду.
П'єса \enquote{Вовчиця}, (переклав \emph{\textbf{Дмитро Папуш}}), була першою роботою Грецької
театральної студії. Починався театр з трьох акторів. Не загубитися на сцені,
якоюсь мірою, допомагав артистам досвід виступів в Народному фольклорному
грецькому ансамблі пісні і танцю \emph{\textbf{\enquote{Сартанські Самоцвіти}}}. Грецька театральна
студія представляє собою синтез різних мистецтв: літератури, музики,
хореографії, вокалу, образотворчого мистецтва, відображення дійсності,
конфліктів, характерів. Затвердження тих чи інших ідей на сцені відбувається за
допомогою драматичної дії, головним глядачем якого є сам актор.

У червні 2002 року з виставою \enquote{Жертва Авраама} колектив брав участь у
Міжнародному фестивалі античної драми \enquote{Боспорські агони} у Керчі. Грецька
театральна студія тоді виступала на одній сцені з відомими театрами Москви,
Петербурга, Києва, Вірменії, Німеччини, Польщі, Греції. За успішний виступ
сартанців нагородили дипломом, а на ім'я міського голови Маріуполя фонд
\enquote{Босфор} передав лист подяки. Але після такого успіху в театрі настала криза,
пов'язана з нестачею коштів. Проте завдяки небайдужим учасникам колективу театр
вдалося зберегти. Всі актори віддають себе повністю сцені, намагаючись передати
глядачеві глибину грецької культури.

\ii{10_02_2021.stz.news.ua.mrpl_city.1.grecka_teatralna_studia.pic.2}

З 2012 року художнім керівником і режисером театру є \emph{\textbf{Корона Анатолій
Вікторович}}. Того ж року колективу було присвоєно звання \emph{\enquote{Народний аматорський
колектив Грецька театральна студія}}. За двадцять один  рік існування театру з
першого складу тільки шестеро продовжують свою діяльність і сьогодні. Наразі в
театрі 22 учасників, з яких 15 – представники молоді. У 2019 році за підтримки
голови Сартанського товариства греків \textbf{\emph{\enquote{Еліни Приазов'я} Наталії Петрівни
Папакіци}}, і Маріупольської міської ради, колектив виступив з успіхом в місті
Патри (Греція). Сьогодні репертуар складається переважно з грецьких вистав, але
до свят колектив може поставити і спектаклі російською мовою. Звичайно,
карантин не обйшов стороною театральний колектив, проте Грецька театральна
студія сподівається, що вже найближчим часом зможе приємно здивувати своїх
глядачів новими виставами. 
