% vim: keymap=russian-jcukenwin
%%beginhead 
 
%%file 05_03_2022.fb.forostjak_ruslan.odessa.1.dumky
%%parent 05_03_2022
 
%%url https://www.facebook.com/ruslan.forostyak/posts/10221172675052817
 
%%author_id forostjak_ruslan.odessa
%%date 
 
%%tags __feb_2022.vtorzhenie
%%title Думки, пости та розмови що завдають значної шкоди та діють на користь ворога
 
%%endhead 
 
\subsection{Думки, пости та розмови що завдають значної шкоди та діють на користь ворога}
\label{sec:05_03_2022.fb.forostjak_ruslan.odessa.1.dumky}
 
\Purl{https://www.facebook.com/ruslan.forostyak/posts/10221172675052817}
\ifcmt
 author_begin
   author_id forostjak_ruslan.odessa
 author_end
\fi

Думки, пости та розмови що завдають значної шкоди та діють на користь ворога.

1. \enquote{Достовірна інфа, яку сказали знаючі люди від сина маминої подруги: сьогодні
в Києві/Харкові/Львові буде хімічна атака/наступ/місто заллють ракетами.
Виїжджайте терміново}. 

Бо такої інфи було вже багато, вона з'являється щодня, працює на паніку, і не
несе нічого більше. Хто міг/хотів - той вже виїхав/виїжджає. Ніхто не почне
виїжджати через ваш панічний пост. Але моральну напругу та паніку це посилює. А
отже грає на руку ворогу. Не треба так. 

2. \enquote{Ці козли переселенці/біженці втекли, і ще й просять
допомоги/житла/інформації/волонтерської допомоги/моральної допомоги. Зрадники,
хай радіють, що не під обстрілами}. 

У кожного свій рівень психологічної стійкості, свої причини щоб виїхати,
хвороби, про які ви можете не знати, і про які вам не зобов'язані казати, тощо.
Опинитися без грошей, речей, близьких, їжі, житла у чужому місті/країні - це
дуже страшно, особливо з дитиною. Давайте не мірятися горем і не бити своїх.
Горе у нас спільне. 

3. *Десь там дуже погано, терміново потрібна допомога, прочитав про це пост,
скопіював дослівно без вказання джерела, розбирайтеся* 

Це видимість допомоги, яка лише шкодить і заплутує. Розумно постити тільки ті
прохання про допомогу, де ви особисто берете участь в організації / на зв'язку
з людьми / тощо. Бо чужа інформація може бути застарілою/не актуальною /
фейковою. Ви тільки заберете час і засмічуєте інформпростір. 

4. \enquote{Ті, хто виїхав у безпеку - ви взагалі робите щось? Робіть}. 

Ви не знаєте, хто що робить, і ніхто не мусить вам звітувати. Але такими
постами ви узагальнюєте і демотивуєте тих, хто допомагає. Багато з тих, хто
виїхав - цілодобово збирають гуманітарку, організовують мітинги, працюють на
інформаційному фронті, тощо. 

5. Пости з хейтом Зеленського. 

Бо це наразі наш Президент, головнокомандувач нашої армії. Своє ставлення
приберіть подалі до перемоги. 

6. \enquote{Все дуже погано, ми програємо, але нам про це не говорять}. 

Такі пости запускає в наш інформпростір ворог, щоб демотивувати. Це неправда, і
ніякої користі такі пости не несуть апріорі. 

7. *Копіює новину з якогось ЗМІ або ТГ-каналу, без вказання джерела. Коли
питають, звідки інфа, і чи це не фейк - людина каже, що всі питання до цього
ЗМІ*. 

Ні, коли ви будь-що постите, усі питання до вас, і джерело - це ви. Особливо,
якщо не вказуєте конкретне місце, де ви це взяли. Відповідальність за будь-яку
інформацію, яку ви поширюєте на своїй сторінці - виключно на вас. 

8. Будь-яка інфа з подачею, що \enquote{це приховують}, \enquote{це нікому не кажуть}, \enquote{джерело
у вищих колах стопудово по секрету повідомило}, \enquote{ми всі не здогадуємося, а
насправді отак}. 

Бо це маніпуляція і дурні теорії змови. У такій формі ворогу найпростіше
вкинути будь-який фейк розгубленим людям, яких завалило інформацією, і які не
володіють мінімальною інформаційною грамотністю. Не ведіться. 

Словом, перш ніж будь-що запостити, глибоко вдихніть, і спробуйте зрозуміти,
якого результату ви можете досягнути цим постом. 

Основні принципи: 

- Не поливати своїх, зараз ворог відомий, і він у всіх спільний, між собою
після війни розберемося;

- Не сіяти паніку, страх і зневіру;

- Не засмічувати стрічку видимістю допомоги, бо люди мають бачити пости про
справжню актуальну потребу в допомозі; 

- Перевіряти інформацію, бо будь-які фейки може запускати ворог. 

Примітка: писати про свої почуття, де б ви не знаходилися, і що б не відчували
- можна і треба. Фейсбук наразі - це не тільки місце обміну інформацією, а й
місце для взаємопідтримки та психологічної допомоги.
