% vim: keymap=russian-jcukenwin
%%beginhead 
 
%%file topics.vojna.my.3.gaaga.voprosy.strong.kiev
%%parent topics.vojna.my.3.gaaga.voprosy.strong
 
%%url 
 
%%author_id 
%%date 
 
%%tags 
%%title 
 
%%endhead 

ivan demidov mozgreplied to Владимир
14:54
а Киев вообще то под охраной Господа Бога

    Like
    Reply

ivan demidov mozgreplied to Владимир
и Архангела Михаила

ivan demidov mozgreplied to Владимир
так что ничего у вас не получится

ivan demidov mozgreplied to Владимир
Киев как стоял, так и будет стоять

ivan demidov mozgreplied to Владимир

москва падет, новосиб исчезнет, а Киев будет стоять во веки веков, Вечный Город
Киев, Самый Лучший Город на Земле


ну конечно знаю. Вот смотри... если ты хочешь пройти на Подол... скажем к памятнику Григорию Сковороде... нашему великому философу... то у тебя есть много разных вариантов. Ну например ты можешь начать с метро Университет. Выходишь из метро, идешь мимо Владимирского Собора, выходишь на улицу Богдана Хмельницкого, там кстати когда то был магазин АкадемКнига. Потом, идешь по улице Владимирской. Идешь идешь себе, доходишь до метро Золотые Ворота, там еще памятник Ярославу Мудрому стоит, князь держит в руках макет Собора Святой Софии

    Класс!
    Ответить

vasja petrenkoответил Танюшке
вот. Поворачиваешь на улицу Ярославов Вал, идешь, идешь себе. Проходишь мимо Университета Карпенко Карого, это университет Телевидения и Театра - там классные студенческие спектакли есть - в учебном театре, доходишь до Львовской Площади... далее идешь по улице Большой Житомирской ... до Михайловской Площади... там, где стоят памятники Андрею Первозванному, княгине Ольге, и Кириллу и Мефодию

vasja petrenkoответил Коле вч 04003
вот... дошел ты до Михайловской площади... идешь мимо здания Министерства Иностранных Дел... до фуникулера...

Танюшка Яответила vasja
Херой, на каком фронте родину защищаешь? 😆 😆 😆 😆

vasja petrenkoответил Коле вч 04003
покупаешь билет на фуникулер... заходишь в вагончик, садишься, и медленно медленно едешь вниз, до Почтовой Площади. Вот и все. Ты на Подоле.

понимаешь, да. Я вот не смог бы так рассказать о москве или же о таганроге, а
вот о Киеве - могу. Потому что я здесь уже город исходил пешком, потому что я
здесь вырос, учился, и живу. Поэтому твои визги о том, что я якобы не знаю
Киев, вообще бессмысленны.
