% vim: keymap=russian-jcukenwin
%%beginhead 
 
%%file 14_04_2018.stz.news.ua.mrpl_city.1.kogda_byl_osnovan_mariupol
%%parent 14_04_2018
 
%%url https://mrpl.city/blogs/view/kogda-byl-osnovan-mariupol
 
%%author_id burov_sergij.mariupol,news.ua.mrpl_city
%%date 
 
%%tags 
%%title Когда был основан Мариуполь?
 
%%endhead 
 
\subsection{Когда был основан Мариуполь?}
\label{sec:14_04_2018.stz.news.ua.mrpl_city.1.kogda_byl_osnovan_mariupol}
 
\Purl{https://mrpl.city/blogs/view/kogda-byl-osnovan-mariupol}
\ifcmt
 author_begin
   author_id burov_sergij.mariupol,news.ua.mrpl_city
 author_end
\fi

\ii{14_04_2018.stz.news.ua.mrpl_city.1.kogda_byl_osnovan_mariupol.pic.1.harlampievskij_sobor}

Существуют вечные темы для дискуссий. Для нашего города – это год основания
Мариуполя. Одни утверждают, что возраст его нужно исчислять от образования
сторожевого укрепления – административного центра Кальмиусской паланки
Запорожской Сечи. Другие считают, что дата его рождения – 1780 год, когда в
Северное Приазовье прибыли переселенцы из Крыма. Третьи, опираясь на
документальные источники, определили, что сначала был город Павловск,
основанный в 1778 году. От этой даты ведут городскую историю. У каждого свои
веские аргументы и каждый по-своему прав.

% Мариимагдалиновская церковь
\ii{14_04_2018.stz.news.ua.mrpl_city.1.kogda_byl_osnovan_mariupol.pic.2}

Беглое знакомство с историей некоторых городов Украины позволило сделать вывод,
что для определения их \enquote{возраста} принимают:

- обнаружение при археологических раскопках следов поселений, пребывания и
деятельности человека на территории данного города (Львов, Полтава и др.);

- первое упоминание населенного пункта в летописях или других письменных
источниках (Киев, Чернигов, Ужгород и др.);

- от начала строительства населенного пункта (Горловка, Краматорск, Донецк и
др.);

- от получения населенным пунктом статуса города (Мариуполь, Херсон, Одесса и
др.).

Воспользовавшись статьей \textbf{Г. И. Тимошевского} из известного сборника \textbf{\enquote{Мариуполь и
его окрестности (1892 г. изд.)}}, можно сделать следующие выводы, касающиеся
даты основания нашего города:

- место, занятое в наши дни исторической частью Мариуполя, имело выгодное
географическое положение как с точки зрения защиты от нападения врагов, так и
для развития торговли;

- существовали устные предания, а также сведения из Камерального описания
уезда, что здесь был город Домаха, или Адомаха, без указания времени и точном
месте его расположения;

- во второй половине XVIII века на территории нынешней площади Освобождения и
близ нее находилась казацкое укрепление – центр Кальмиусской паланки;

- основание, если не крепости, то поселка запорожцев, относят к XVI столетию, в
то время, когда правил Предслав Лянцкоронский – первый гетман Войска
Запорожского;

- в 1756 году в крепости находилось 674 казака, в 1714 году там было 200
казаков. Цитата из вышеупомянутого труда Г. И. Тимошевского: \enquote{В 1775 году
Запорожское войско было уничтожено. И Азовский губернатор Чертков область
переименовал в Кальмиусский уезд, а крепость избрал для постройки города
Павловска. Новый город Павловск, очевидно, разрастался, - в нем уже была
соборная церковь, куда 3 декабря 1779 г. назначен был даже соборный протоиерей
Роман Кошевский}.

В 1997 году вышла из печати небольшая по объему, но очень содержательная книжка
исследовательницы истории нашего края \textbf{Рены Ильиничны Саенко \enquote{Из истории
основания города Мариуполя}}. Значение этого труда состоит в том, что ранее
выдвигавшиеся гипотезы по этой теме ею подтверждены или опровергнуты на
основании документов, хранящихся в государственных архивах. В этом труде Р. И.
Саенко цитирует историка запорожского казачества А. А. Скальковского: \emph{\enquote{Для
защиты границы и покровительства рыболовству у реки Кальмиус со стороны
Азовского моря был поставлен сторожевой пост}}. Автор отнес это событие к 1734
году.

\ii{14_04_2018.stz.news.ua.mrpl_city.1.kogda_byl_osnovan_mariupol.pic.3.ekaterininskaja_35_37}

Что же касается существования города Павловска. В приснопамятные времена
многое, что было написано в сборнике \enquote{Мариуполь и его окрестности},
подвергалось сомнению. Мол, гимназические педагоги в угоду царской власти
искажали историческую правду. Да и в годы уже существования независимой Украины
высказывалась мысль, что Павловск существовал лишь на бумаге. Саенко, работая в
Российском архиве древних актов, нашла интересный документ, составленный в
сентябре 1778 года, с названием \enquote{Карта, представляющая Азовскую и
Новороссийскую губернии, с указанием смежным ко оным земель}. На этой карте
обозначен город Павловск в том самом месте, где находится Мариуполь. Да, в
новом городе народа было маловато. По ведомости 1778 года в Павловске на 1
января 1779 года было приписано 57 человек, в том числе 40 мужчин и 17 женщин.
Среди них было 4 купца и 22 цеховых, то есть ремесленников. Как же эти
достойные люди зарабатывали себе на жизнь при таком малолюдье? Так ведь рядом
была слобода Кальмиусская, остаток паланки такого же названия. Там проживало по
разным оценкам от 77 до 79 человек. Нужно присоединить сюда и жителей
зимовников, расположенных по берегам Кальмиуса. Значит, что и для купцов, и для
цеховых было чем и с кем заниматься своими делами. Кроме того, впереди у них
была надежда, что город станет многолюднее, поскольку при них была
\href{https://archive.org/details/07_04_2018.sergij_burov.mrpl_city.cerkov_sv_marii_magdaliny}{заложена
церковь во имя св. Марии Магдалины} и даже начала строиться. Но мечтам их не
суждено было стать реальностью.

Известный государственный деятель \textbf{Григорий Потемкин} в сентябре 1779 года
распорядился поселить греков, выведенных из Крыма на территорию Павловского
уезда. Уезд переименовать, назвав его Мариупольским. Жителям Павловска было
предложено выехать из города. У владельцев было выкуплено 55 домов и построек.
Греки-выходцы из Крыма достигли бывшего Павловска, получившего новое имя –
Мариуполь, 26 июля 1780 года. Во главе их был \textbf{митрополит Игнатий}. Это были
горожане Кафы (Феодосии), Гезлева (Евпатории), Карасубазара, Бельбека,
Бахчисарая и его пригорода, название которого интерпретируется, как Маурум,
Маирум, Мариам и даже Марьино. Первопоселенцев Мариуполя было без малого 3
тысячи душ обоего пола. Здесь будет уместно привести цитату из одного труда Р.
И. Саенко: \emph{\enquote{Роль греков не в основании города, а в другом. Мариупольские греки
внесли огромный вклад в заселение и хозяйственное освоение Приазовья, в
дальнейшую застройку города Мариуполя, превращение его в крупный торговый, а
позже – в крупный промышленный центр}}.

Каждый может по своему усмотрению принять ту или иную точку отсчета для
определения \enquote{возраста} города. Главное, чтобы не доводить дело до абсурда. А
иначе можно договориться до того, что Мариуполь основан в III тысячелетии до
новой эры. А что? В начале 30-х годов на левом берегу Кальмиуса при
строительстве \enquote{Азовстали} рабочий Г. Ф. Кравец обнаружил древнее захоронение
людей. Профессор Н. Е. Макаренко в ходе археологических раскопок установил, что
захоронение принадлежит к стоянке людей и относится к III тысячелетию до новой
эры.
