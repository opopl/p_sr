% vim: keymap=russian-jcukenwin
%%beginhead 
 
%%file topics.ukraina.rossia.ugroza_napadenia.preface
%%parent topics.ukraina.rossia.ugroza_napadenia
 
%%url 
 
%%author_id 
%%date 
 
%%tags 
%%title 
 
%%endhead 
\subsection{Передмова}

Тут ми зібрали деякі матеріали щодо так званого вторгнення Росії в Україну.
Зараз лютий 2022 року, і ми пишемо ці рядки, сидячи в затишній Київській
квартирі десь на Оболоні. Вечоріє. Сонце опускається все нижче і нижче над
Вічним Городом Ярослава і Володимира - Містом-Героєм Києвом, запалюються
вечірні ліхтарі і потихеньку падає температура надворі.

Добре, тут слід зауважити таке. Якщо Україна претендує на спадщину Київської
Русі, то вона також претендує на назву Русь, логічно ж, еге? А якщо так, то
Україна = Русь = Росія = Россия = Раша = Russia, тому що Росія - це всього
навсього грецький (візантійський) варіант слова Русь. А англійською Русь =
Russia (Руссія). А тому найперше треба зрозуміти, про що власне іде мова. Це як
треба розуміти - Україна, тобто Мала (Київська) Росія збирається напасти на
саму себе, чи як?  Чернігів збирається вторгнутись в Одесу, а Львів у Харків?
Ви тут спитаєте, про що тут пишуть ці божевільні?  А ми скажемо, що важливо
точно формулювати свої твердження. Бо інакше можна опинитись в болоті, і
копирсатись в ньому все своє свідоме життя, без надії коли-небудь з нього
вибратись. На даний момент ми таки в цьому болоті все еще знаходимось, але дай
Боже, ми все ж з нього колись виберемось. А для цього необхідно більш чітке та
послідовне мислення. Бо, як казав ще Декарт - Cogito, Ergo Sum - тобто, мислю,
значить, існую.

Так що, панове, формулюйте точніше свої твердження. А далі...

А далі Ви знайдете різноманітні думки від різних людей щодо цього самого
вторгнення.

І знаєте... є три речі, які можна робити нескінченно:

\begin{itemize} % {
\item Проводжати поглядом гарних дівчат на вулицях Києва, аж поки шия не звернеться стрічкою Мьобіуса;
\item Вивчати історію Києва;
\item Спостерігати за вторгненням Росії в Україну - найдовшим по часу в історії людства.
\end{itemize} % }

