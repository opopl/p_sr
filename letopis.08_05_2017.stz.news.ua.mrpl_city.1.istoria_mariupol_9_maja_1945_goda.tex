% vim: keymap=russian-jcukenwin
%%beginhead 
 
%%file 08_05_2017.stz.news.ua.mrpl_city.1.istoria_mariupol_9_maja_1945_goda
%%parent 08_05_2017
 
%%url https://mrpl.city/blogs/view/istoriya-mariupol-9-maya-1945-goda
 
%%author_id burov_sergij.mariupol,news.ua.mrpl_city
%%date 
 
%%tags 
%%title История. Мариуполь: 9 мая 1945 года
 
%%endhead 
 
\subsection{История. Мариуполь: 9 мая 1945 года}
\label{sec:08_05_2017.stz.news.ua.mrpl_city.1.istoria_mariupol_9_maja_1945_goda}
 
\Purl{https://mrpl.city/blogs/view/istoriya-mariupol-9-maya-1945-goda}
\ifcmt
 author_begin
   author_id burov_sergij.mariupol,news.ua.mrpl_city
 author_end
\fi

Репродуктор не выключали ни на минуту. Утро начиналось с боя курантов на
Спасской башне, через некоторое время после позывных радиостанции звучал голос
диктора: \enquote{От Советского информбюро} ... Восьмилетний мальчишка вслушивался в
перечисление немецких городов, взятых Красной Армией, загибал по пальцу на
руке, после каждого названия; потом считал их. От взрослых он слышал: \enquote{Раньше
наши брали города по одному в день, а сейчас – пачками. Видно скоро войне –
конец}. А ему так хотелось, чтобы война закончилась...

8 мая 1945 года к нам в дом нагрянули дедушкин племянник Иван и два его друга.
Эта троица после излечения в ростовском госпитале через Мариуполь добирались в
свои фронтовые части. Все были обрадованы встрече. Потом было застолье.
Выпивали за победу, желали здоровья воинам, желали, чтобы вражья пуля минула
их. Дядя Ваня шутил, рассказывал о боях, как и где его ранили. Засиделись за
полночь. Мама и бабушка стали готовить постели. Кроватей на всех не хватило –
своих шестеро, да гостей – трое. Мальцу постелили на полу под старым пианино.
Взрослые о чем-то перешептывались. Утром сон был прерван гомоном возбужденно
говорящих людей. Кто-то смеялся, кто-то плакал. Прорывались слова:
\enquote{Капитуляция... Победа... Скоро домой...}. Прибежали соседи, поздравляли обитателей
дома, поздравляли их гостей, и опять слезы, прерываемые неожиданным смехом.
Таким запомнился на всю жизнь первый день без войны. Он был объявлен нерабочим.
А вечером из репродуктора раздались залпы победного салюта...

Запомнился еще один день – 9 мая 1946 года. Ясный солнечный день. Тепло.
Городской сад, запах начинающей цвести сирени, скрип под ногами промытого
дождем мелкого желтого гравия на аллеях. Торжественное открытие памятника на
могиле Героев Советского Союза Владимира Семенишина и Николая Лавицкого.
Сердечные встречи со старыми знакомыми, одни из них только что
демобилизовались, другие – недавно вернулись из эвакуации. Тележка с мороженым,
сладкая белая масса между двумя круглыми вафельками, деревенеющий от холода
мороженого язык. День Победы 1947 года был тоже праздничным. Но для родителей,
как и для многих мариупольцев, он прошел в труде на огороде – клочке земли
далеко за городом, выделенном отцу, как и многим другим работникам завода имени
Ильича... 

Прошел семьдесят один год  с того времени, когда ранним утром 9 мая 1945 года
из всегда включенного репродуктора в доме дедушки прозвучал голос Юрия Левитана
с долгожданным сообщением, и мальчишка  узнал, что войне конец. Как все было?
Пользуясь опубликованными источниками, можно воспроизвести историю этого
Великого Дня.  8 мая 1945 года в пригороде Берлина Карлсхорсте в 22 часа 43
минуты по центрально-европейскому времени (9 мая в 0 часов 43 минуты по
московскому времени) был подписан Акт о безоговорочной капитуляции фашистской
Германии. Церемония подписания акта происходила под председательством маршала
Георгия Жукова. Но еще до того, во второй половине 8 мая, - к этому времени
было ясно, что в столице повергнутого рейха Берлине все готово к подписанию
акта, - Сталин подписал Указ Президиума Верховного Совета СССР, в котором день
9 мая был объявлен всенародным праздником в ознаменование победы советского
народа над фашистской Германией. В 6 часов утра по московскому времени этот
Указ по радио был зачитан диктором Левитаном.  Из формулировки указа следовало,
что этот день – выходной. Так оно и было с 1945 по 1947 год включительно.
Указом Президиума Верховного Совета СССР от 23 декабря 1947 года выходной день
был отменён: вместо Дня Победы нерабочим был объявлен Новый год.

Но праздником День Победы остался. И не замалчивался, как можно иногда прочесть
в Интернете. Для этого достаточно просмотреть подшивки газет 40 – 50-х годов и
позже. Каждый год по этому поводу публиковались приказы военных министров,
министров обороны. В них обязательно были строки: \enquote{Сегодня, 9 мая, произвести
салют в столице нашей Родины – Москве, в столицах союзных республик, а также в
городах-героях: Ленинграде, Сталинграде, Севастополе и Одессе}. В некоторых
подобных приказах к указанному перечислению городов добавлялся Калининград и
Львов. В этот или в ближайший с ним день все центральные газеты начинались с
передовиц на тему о Победе. Вот заголовки некоторых из них: \enquote{Праздник Победы}
(Литературная газета, 1948 г.), \enquote{Историческая победа советского народа} (Труд,
1952 г.). \enquote{Завтра – День Победы} (Учительская газета, 1962 г.) и т.д.

По инициативе Л. И. Брежнева в год двадцатилетия Победы указом Президиума
Верховного Совета от 26 апреля 1965г. день 9 мая снова был объявлен нерабочим.
Была учреждена медаль \enquote{Двадцать лет Победы в Великой Отечественной
войне 1941—1945 гг.}.  9 мая 1965 года на Красной площади в Москве состоялся
военный парад. Впереди войсковых колонн двигалось Знамя Победы, нес его Герой
Советского Союза полковник Константин Самсонов, его ассистентами были Герои
Советского Союза сержант Михаил Егоров и младший сержант Мелитон Кантария,
воины, водрузившие красное полотнище на Рейхстаг. Торжественные парады на
Красной площади в День Победы до 1995 года проводились лишь в юбилейные годы —
в 1965, 1985 и 1990 годах. 

Конечно, и в нашем городе День Победы отмечался.  И в те годы, когда он был
выходным, и тогда, когда был – рабочим. Организовывались встречи с ветеранами и
их чествование, в краеведческом музее с давних пор существует посвященная ему
экспозиция. В местных газетах ко дню 9 мая всегда публиковались соответствующие
материалы. Отметим несколько знаменательных событий в истории Мариуполя,
связанных исторической датой.

8 мая 1971 года улица Портовая в Приморском районе получила новое название –
проспект имени Лунина, Героя Советского Союза, нашего земляка, прославленного
моряка-подводника Северного флота.  

9 мая 1975 года на ул. Карпинского в честь 30-летия Великой Победы совершилась
церемония открытия памятного знака – танка Т-34, воздвигнутого на высокий
постамент. Это дань памяти танкистам, которые участвовали в сентябре 1943 года
в освобождении Мариуполя от немецко-фашистских оккупантов, а также
мариупольцам, активно участвовавшим в создании легендарной боевой машины.

8 мая 1985 года состоялся митинг, посвященный открытию памятника ильичевцам,
погибшим в годы Великой Отечественной войны 1941 -  1945 гг.

9 мая 2000 года на территории комбината имени Ильича был открыт памятный знак
\enquote{Ильичевцам – воинам, труженикам, создателям боевого щита Родины}.

7 мая 2011 года народ собрался на открытие нового монумента. Когда спал покров,
все увидели на постаменте бюст контр-адмирала Лунина.

В Мариуполе сложилась традиция собираться 9 мая в день Победы у монумента
\enquote{Воин-освободитель} в Приморском парке. И когда бываешь там, вспоминается то
раннее утро, когда Юрий Левитан сообщил, что войне конец.
