% vim: keymap=russian-jcukenwin
%%beginhead 
 
%%file slova.imperia
%%parent slova
 
%%url 
 
%%author 
%%author_id 
%%author_url 
 
%%tags 
%%title 
 
%%endhead 
\chapter{Империя}
\label{sec:slova.imperia}

%%%cit
%%%cit_head
%%%cit_pic
%%%cit_text
Может, стоит больше внимания обращать на примеры нашего величия?
\emph{Имперскости} нам не хватает. Тут в связи с мовосрачами, поднятыми
Ларисой и Вахтангом, вот что хочу сказать.  \emph{Имперскойсти} нам не хватает.
Величие собственное не осознаём. Всё нам кажется, что кто-то нас использовал.
Что не мы строили СССР, а какие-то марсиане строили его, а Украину
использовали. Что не мы были \emph{Российской империей}, Речью Посполитою и
Княжеством Литовским, а эти \emph{империи} строили какие-то чужинци, а
украинцев порабощали и пили кровь ихнюю.  А как зайдёшь в какой-то музей
польский или литовский - то там вся украинская знать в одном ряду с польской и
литовской знатью. Как-то не похожи князья наши стародавние на подчинённых и
порабощенных... А первыхе лица \emph{Российской империи} и СССР - ну украинцев
больше, чем литовцев, казахов и таджиков
%%%cit_comment
%%%cit_title
\citTitle{Почему нам все время кажется, что нас кто-то использовал? / Лента соцсетей / Страна}, 
Павел Себастьянович, strana.ua, 01.07.2021
%%%endcit
