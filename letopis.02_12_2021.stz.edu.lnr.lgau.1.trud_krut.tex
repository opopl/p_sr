% vim: keymap=russian-jcukenwin
%%beginhead 
 
%%file 02_12_2021.stz.edu.lnr.lgau.1.trud_krut
%%parent 02_12_2021
 
%%url http://lnau.su/novosti/trud-krut-smot-prometej-lgau-prinyal-uchastie-v-konferenczii-priurochennoj-k-25-letiyu-molodezhnogo-trudovogo-dvizheniya-na-luganshhine
 
%%author_id 
%%date 
 
%%tags lnr,donbass,trud,molodezh,lgau
%%title Труд – крут!
 
%%endhead 
\subsection{Труд – крут!}
\label{sec:02_12_2021.stz.edu.lnr.lgau.1.trud_krut}

\Purl{http://lnau.su/novosti/trud-krut-smot-prometej-lgau-prinyal-uchastie-v-konferenczii-priurochennoj-k-25-letiyu-molodezhnogo-trudovogo-dvizheniya-na-luganshhine}

\textSelect{Студенческий молодежный трудовой отряд (СМТО) «Прометей» Луганского
государственного аграрного университета (ЛГАУ) 1 и 2 декабря принял участие в
конференции, приуроченной к 25-летию молодежного трудового движения на
Луганщине, организованной Центральным штабом молодежных трудовых отрядов при
поддержке Министерства культуры, спорта и молодежи Луганской Народной
Республики.}

\ii{02_12_2021.stz.edu.lnr.lgau.1.trud_krut.pic.1}

Луганский государственный аграрный университет является одним из старейших и до
недавнего времени единственным аграрным вузом на Донбассе.

Наряду с обучением студентов, преподавательский коллектив считает своей
важнейшей задачей формирование многогранной личности молодого человека,
раскрытие его организаторского потенциала, самореализации в обществе. Особое
место отводится трудовому воспитанию молодежи, бережному сохранению его
традиций.

Отметим, что первый студенческий строительный отряд «Прометей» был сформирован
в ЛГАУ в 1970 году. В основном, он состоял из студентов факультета механизации
сельского хозяйства, а девушки с агрономического факультета работали поварами.

Бойцы Прометея трудились на строительных объектах Луганска и области, выезжали
в Российскую Федерацию и Республику Крым, где выполняли ремонтно-строительные
работы, занимались благоустройством территорий, приобретали трудовые навыки и
первый опыт работы в коллективе.

Работа конференции стартовала 1 декабря, которая проходила в очном режиме и в
прямой трансляции на онлайн-платформе ZOOM на четырех образовательных
площадках: педагогическое направление МТО ЛНР, строительное направление МТО
ЛНР, сервисное направление МТО ЛНР и онлайн-презентация МТО ЛНР, РШ МТО.

\ii{02_12_2021.stz.edu.lnr.lgau.1.trud_krut.pic.2}

Поприветствовала всех заместитель председателя Правительства ЛНР \textSelect{Анна Тодорова}
и рассказала, что в 2021 году в отрядах работали сотни молодых людей, и с
каждым годом количество желающих трудиться на благо Республики растет.

\begin{zzquote}
«Работа молодежных трудовых отрядов – это возможность не только повысить свои
профессиональные навыки, но пройти настоящую школу жизни, утвердить лучшие
качества, такие как ответственность, самоотверженность, взаимопомощь,
почувствовать, как это, когда твой труд приносит пользу людям», − сказала она.	
\end{zzquote}

\textSelect{Анна Тодорова} пожелала участникам и ветеранам движения новых
проектов и сил на их реализацию.

\begin{zzquote}
«Вы молодцы, вы та молодежь, на которую стоит равняться всем. Студенческие
отряды, трудовые отряды – это определенная группа, в которой всегда есть, о чем
поговорить, это всегда какие-то добрые интересы. Празднуя 25-летие трудовых
отрядов Луганщины, хочу сказать всем вам огромное спасибо за то, что мы эту
работу продолжаем», −  сказал министр культуры, спорта и молодежи ЛНР \textSelect{Дмитрий
Сидоров}, обращаясь к присутствующим в зале.	
\end{zzquote}

Он вручил ведомственные благодарности и памятные знаки «За отработанную целину
(работу)» лучшим бойцам МТО.

\begin{zzquote}
«Сегодня молодежные трудовые отряды – это наш кадровый потенциал, это
будущее Республики, это хорошая школа будущих профессионалов, а также
совершенно особая романтика первых трудовых начинаний в кругу друзей и
единомышленников», −  отметила командир Центрального штаба МТО ЛНР \textSelect{Анна
Сычева.}
\end{zzquote}

Она поблагодарила участников отрядов за их труд, призвав быть полезными
обществу и творить будущее Республики.

Бойцы Студенческого молодежного трудового отряда «Прометей» были награждены
памятной продукцией от Центрального штаба молодежных трудовых отрядов ЛНР и
Министерства культуры, спорта и молодежи Луганской Народной Республики.

\ii{02_12_2021.stz.edu.lnr.lgau.1.trud_krut.pic.3}

Напомним, в 2018 году тогда еще Совет министров ЛНР принял постановление «Об
утверждении Порядка формирования и деятельности молодежных трудовых отрядов на
территории Луганской Народной Республики». В том же году при Министерстве
культуры, спорта и молодежи ЛНР был создан Центральный штаб МТО, членами
которого стали представители региональных штабов молодежных трудовых отрядов, а
также представители образовательных организаций высшего профессионального
образования.

История молодежного трудового движения Луганщины начинается с 1996 года. Тогда
по инициативе студенческой молодежи на базе учебных заведений было возрождено
студотрядовское движение, было сформировано девять отрядов, численность которых
составила 353 человека.

\textSelect{Поздравляем всех бойцов с 25-летие молодежного трудового движения
на Луганщине! Труд – крут!}

Пресс-центр университета, фото Игоря Одинцова и Алексея Волобуева
