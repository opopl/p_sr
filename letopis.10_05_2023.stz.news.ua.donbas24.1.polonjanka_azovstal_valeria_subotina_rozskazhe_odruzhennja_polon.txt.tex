% vim: keymap=russian-jcukenwin
%%beginhead 
 
%%file 10_05_2023.stz.news.ua.donbas24.1.polonjanka_azovstal_valeria_subotina_rozskazhe_odruzhennja_polon.txt
%%parent 10_05_2023.stz.news.ua.donbas24.1.polonjanka_azovstal_valeria_subotina_rozskazhe_odruzhennja_polon
 
%%url 
 
%%author_id 
%%date 
 
%%tags 
%%title 
 
%%endhead 

Ольга Демідко (Маріуполь)
Маріуполь,Україна,Мариуполь,Украина,Mariupol,Ukraine,Полон,Азовсталь,Валерія Суботіна,date.10_05_2023
10_05_2023.olga_demidko.donbas24.polonjanka_azovstal_valeria_subotina_rozskazhe_odruzhennja_polon

omen_3_nightshifterz_radio

Полонянка з Азовсталі Валерія Суботіна розкаже про одруження під обстрілами та полон — подробиці

13 травня у Києві Валерія Суботіна зустрінеться з усіма охочими

Вже 13 травня, цієї суботи, у київському Будинку кіно пройде зустріч із
захисницею Маріуполя, поетесою Валерією Суботіною, яка повернулася з полону 10
квітня. На заході дівчина розповість про все пережите та почитає власні вірші.

Читайте також: Історія сталевої Нави — захисниця Маріуполя розповіла про
весілля на Азовсталі та російський полон (ВІДЕО)

«Так багато вас — прекрасних, сильних, неймовірних українців хотіло зустрітися
зі мною, а я не встигаю... Бо ще обстеження, реабілітація... Але. Мама Людмила
Суботіна, яка так боролася за моє повернення, влаштує нам зустріч», — зауважила
на свої сторінці у Facebook Валерія Суботіна.

Нагадаємо, Валерія Суботіна (Карпиленко) на позивний Nava — маріупольська
поетеса, кандидатка наук із соціальних комунікацій, у минулому випускниця й
викладачка Маріупольського державного університету, яка разом з «Азовом»
захищала рідний Маріуполь. Дівчина завжди відрізнялася неабиякою силою волі,
оптимізмом та винятковим талантом. Для багатьох маріупольців і загалом всіх
українців вона є справжнім взірцем для наслідування.

24 лютого Валерія Карпиленко відчувала — на порозі повномасштабна війна. Жінка
пішла до військкомату і її мобілізували до полку «Азов». Разом зі своїм коханим
— Андрієм Суботіним (позивний Борода) вони опинилися заблокованими на
«Азовсталі». Історія кохання Валерії та прикордонника Андрія стала відомою не
тільки в Україні — вона облетіла весь світ. Пара познайомилась, коли Валерія
працювала в пресслужбі Донецького прикордонного загону. В облозі на «Азовсталі»
вирішили одружитись. Датою обрали 5 травня, бо це день народження полку «Азов».
Проте 7 травня Андрій загинув... Валерія перебувала в полоні 10 довгих місяців і
10 квітня 2023 року її нарешті звільнили!

Читайте також: У Києві відбувся захід на підтримку військовополоненої з
Азовсталі (ВІДЕО)

На зустрічі всі присутні зможуть поспілкуватися з Валерією. Це буде перша
масштабна публічна зустріч з дівчиною після її звільнення. Крім того, на заході
відбудеться презентація музичного відеокліпу Марії Бурмаки «Кохання — сталь».

«Та, головне, ми поспілкуємось і привернемо увагу до наших військовополонених,
що досі чекають свого повернення в рідну Україну», — наголосила Валерія.

Зустріч відбудеться у Будинку кіно (Київ, вул. Саксаганського, 6, Синя зала, 2
поверх) о 16.00.

Нагадаємо, раніше Донбас24 публікував історії п'яти жінок з Маріуполя, чия
непересічність, самовіддана діяльність і винятковий талант надихають.

Ще більше новин та найактуальніша інформація про Донецьку та Луганську області
в нашому телеграм-каналі Донбас24.

ФОТО: з відкритих джерел
