% vim: keymap=russian-jcukenwin
%%beginhead 
 
%%file 16_05_2021.fb.bilchenko_evgenia.1.poezia_piter_konkurs
%%parent 16_05_2021
 
%%url https://www.facebook.com/yevzhik/posts/3894514793916935
 
%%author 
%%author_id 
%%author_url 
 
%%tags 
%%title 
 
%%endhead 
\subsection{БЖ. Питерской конкурсной поэзии славы пост}
\label{sec:16_05_2021.fb.bilchenko_evgenia.1.poezia_piter_konkurs}
\Purl{https://www.facebook.com/yevzhik/posts/3894514793916935}

Ура! Я отжюрила международный фестиваль мелодекламаций "Петербургский Ангел".
Каждый год на этот фестиваль я езжу в сам Город. Но сейчас - карантин. Все -
онлайн. Вот так примерно выглядит честный член жюри после прослушки всех
возрастов и стилей поэтического Ленинграда-Петербурга - 50 авторов - и
практически ни одного слабого. Плюс клипы, плюс мелос, плюс оценка подачи. 

\ifcmt
  pic https://scontent-bos3-1.xx.fbcdn.net/v/t1.6435-9/186466605_3894514647250283_8393033315461406022_n.jpg?_nc_cat=100&ccb=1-3&_nc_sid=8bfeb9&_nc_ohc=vKbQzoWEAEkAX_M1u8g&_nc_ht=scontent-bos3-1.xx&oh=0e7e659959399393bd87931e1fb02c97&oe=60C6AA57
\fi

Впервые, реально, в истории Феста - все герои мощные. Глаза разбегаются. Уши
сворачивает в дудку от корневитой сочной вкусности языка.

Это - рожа человека, который сутки потратил на умиротворение с конференцией по
Григорию Сковороде, дабы потом расшатать амплитуду от хохота до рыдания. 

Я не скажу, кто у меня лично в пятерке лучших. Намекну. Один молодой гений
любви, один мужественный выразитель темы Победы в ВОВ в той ее жёсткой и
яростной, радикально-триумфально-юной форме, в какой память 9 мая должна Жить
Дальше, один классик в духе Кабанова (ибо вечен), одна совершенно Достоевская и
не раздражающе постмодерная героиня Линча с неимоверной аудиовизуалкой и то,
супротив чего Бильченко - бессильна и тает: русский интеллектуальный рэп из
глубинки.

Оргам: Мише Вэю и Грише Белову - поклон и благодарность.

PS. Все это время, пока судила, БЖ не курила. Это - вместо всех рецензий.
