% vim: keymap=russian-jcukenwin
%%beginhead 
 
%%file 01_11_2020.fb.igor_lesev.1.moj_tupoj_prezident
%%parent 01_11_2020
 
%%url https://www.facebook.com/permalink.php?story_fbid=3689544304409979&id=100000633379839
%%author igor lesev
%%tags 
%%title Мой тупой президент
 
%%endhead 

\subsection{Мой тупой президент}
\Purl{https://www.facebook.com/permalink.php?story_fbid=3689544304409979&id=100000633379839}

Вы ведь знаете, что такое разочарование?  Даже не так. Вы знаете, что такое
источник разочарования? Разве вас разочарует хамство незнакомца в маркете? А
может вас разочаровать игра киевского «Динамо»? Разочаровать могут только те,
кто обнадеживал. При этом, чем меньше ожиданий, тем меньше повода к
разочарованию.

Вот возьмем, скажем, Петра Порошенко. Он меня ни разу в жизни не разочаровал.
Напротив. Часто удивлял. Даже потрясал. Вот как можно оставаясь 100\%
проспиртованной гнидой, петлять даже в тех местах, где уже даже веревочка
закончилась?

В конце 2018 я для себя однозначно решил. Если эта скотина остается на вторую
пятилетку, я сваливаю из страны. И не потому, что я какая-то цаца, которой
непременно что-то угрожало. Хотя ощущение физиологического дискомфорта, когда
могут загасить у подъезда, не покидало всю порошенковскую пятилетку. Уверен,
также наши прадеды чувствовали себя в 37-м. Тут Петру Алексеевичу спасибо за
историческую реконструкцию. Натуральные 3-D, а некоторые, вроде Бузины,
Калашникова и Шеремета даже не сумели выйти из кинотеатра.

Но главная причина была даже не в этом. Порошенко с его сектой расистов украл у
меня мою страну. А это как молодость. Честно скажу, не дорожил Украиной до
13-го года. Воспринимал ее как место обитания. Где работа, девчушки, бухло,
развлечения и мимолетные радости и горести. Но вдруг оказалось, что Украина
образца 2013 года – это моя Любима Родина. Хрен его знает, что вообще такое
«любимая родина». Но вот так вышло, что у меня другой не было, а в феврале
14-го у меня ее забрали.

Весна 2014-го. У каждого она своя. Одни заходят в социальные лифты в Киеве. У
других Крым, Одессе, Донбасс. Весной 2014-го мы начали друг друга фильтровать
не по категориям «тот парень со своей квартирой на Дарнице» и не «та девчушка с
4 размером сисек», а по невероятноым – еще пару месяцев назад нам всем
невероятным – другим критериям. Критериям, где слово «вата» использовалось
только в аптеке, а «укроп» на рынке.

Я возненавидел эту Украину. Она была грязной, ржавой и чужой. А ведь я не из
Донбасса, и не из Крыма. Я формально не Alien, не чужой для героя Сигурни
Уивер. Я родился в Гайсине. Моя мама родилась в Гайсине. Мой дед и бабулька
родились в пределах 60-100 км от Гайсина. Где родились их родители, могу только
предполагать, что не сильно далеко. Уж не кавказец, где 7 колен назовут без
алкогольных влияний. Но срань господня, я – уроженец своей земли – почувствовал
в 14-м себя чужаком в своей же стране.

Я. Гайсинчанин. Чужак. А что же говорить про девчонок из Макеевки? А про
парнишек из Керчи?  Смотрите. Это не пост «обиженного Майданом». При Януковиче
я жил в коммуналке на окраине Киеве. Майдан обрушил цены на недвижимость, и я
сумел купить полноценную квартиру. У меня нет только черных красок. Я о другом.
В 2019-м я голосовал в первом туре за Зеленского, как самого рейтингового
кандидата. 31 марта выборы. Вечером экзитпол и первые результаты – 30\%.
Тридцать процентов. А потом было утро 1 марта. Я вышел из подъезда и… это было
мое самое чудесное утро в моей жизни. Я никогда не сидел в тюрьме. Я никогда не
выигрывал крупную сумму в казино. И у меня никогда не было секса с молодой
Моникой Беллуччи (да чего уж там, не только с молодой). Но это 1 марта ощущения
были именно такие. Ты вышел с зоны, поднял бабло, а Моника тебе шлет вслед
пошлые смс-ки. Свобода. Бабло. Удовлетворение.

Друзья. Я ведь не топ-блогер. Не полутоп. Даже не полупокер. Так, мелководный сейнер и чуть-чуть рефрижератор. Но кто со мной на этой страничке помнят, как я топил в 19-м за Зеленского. Понятия не имею, кого я сумел тогда сагитировать за Зе. Но я и моя мама голоса отдали. Плюс два юнита. И это было искренне. И это было бесплатно.

Но знаете, что меня тогда вдохновляло больше всего? Меня, 100\% циника и 99\%
выгодополучателя от любой работы, где надо передвинуть жопу с одной половинки
дивана на другую? Я был уверен, что Зеленский меня НИКОГДА не сумеет
разочаровать. Не потому что Зеленский из себя что-то представляет. А потому что
фон Зеленского – это Порошенко.

И вот… Нет, нет никакого «и вот». Зеленский тихонько напрягать начинал с самого
начала. Потом бесить. Потом смешить. А смеяться с комика – это как учиться
красноречию у Кличко. Зеленский очень быстро стал временным балластом. Это
когда надо ждать, чтобы попрощаться. Вот как в ресторане с говяной едой. Ты уже
вилкой поковырял и даже официанту высказал свое недовольство. Но все равно по
итогу надо расплатиться.

Но все же «и вот» таки настало. Пост о судьях КС. Вот он пишет – «…заради чого
боролися на всіх наших Майданах». На всех наших Майданах. Сука ты, Зеленский.
Грязная. Нет, лично я не делю людей на тех, кто был и кто не был на Майдане. Я
даже слово «Майдан» стал писать с большой буквы, как и «в Украине», а не «на»,
вот только чтобы не разжигать. Мне лично пох, а кого-то колбасит. Так пусть
никого не колбасит.

Но что такое Наш Выбор образца 2019? Чтобы каждый из нас ответил на этот
вопрос, сделайте друзья гугловский камбэк назад в будущее в 2014 год. Половина
Украины поддерживает Майдан, а вторая – нет. Майдан – это протест, который стал
фартовым для его сторонников. Но ведь не для всей страны. И вот наш Выбор 2019
– это был выбор, где нас не делили по ширине черепа – кто «за», а кто «против»
Майдана. Мы – 73\% - во втором туре голосовали именно за это.

И вот Зеленский в этом посте все это опроверг. Нет, не просто опроверг. Он
предал меня. Мою маму. Предал тех людей, которые за него голосовали. Грязный.
Ржавый. Порохобот.
