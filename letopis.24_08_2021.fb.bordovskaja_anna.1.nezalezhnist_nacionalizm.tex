% vim: keymap=russian-jcukenwin
%%beginhead 
 
%%file 24_08_2021.fb.bordovskaja_anna.1.nezalezhnist_nacionalizm
%%parent 24_08_2021
 
%%url https://www.facebook.com/anna.bordovski/posts/3668276843397360
 
%%author Бордовская, Анна
%%author_id bordovskaja_anna
%%author_url 
 
%%tags farion_irina,nacionalizm,nezalezhnist,ukraina
%%title довгий половинчастий і непевний шлях України в лабіринтах безвольности та інфантилізму
 
%%endhead 
 
\subsection{Довгий половинчастий і непевний шлях України в лабіринтах безвольности та інфантилізму}
\label{sec:24_08_2021.fb.bordovskaja_anna.1.nezalezhnist_nacionalizm}
 
\Purl{https://www.facebook.com/anna.bordovski/posts/3668276843397360}
\ifcmt
 author_begin
   author_id bordovskaja_anna
 author_end
\fi

Зеленський прочитав непогану промову і закінчив гаслом тих, про яких жодним
словом не згадав у написаному слові. Це логічно. Для нього найбільший страх -
це саме національна і націоналістична Україна з Бандерою, Шухевичем і
Коновальцем. Тому Галичина в нього асоціюється тільки з "горнятком кави", а не
з боротьбою Січових стрільців, УВО-ОУН-УПА. А Тернопільщина - це лише Соломія
Крушельницька, а не Ярослав Стецько чи Йосип Сліпий. Луганщина - це чомусь
Даль, а не Іван Світличний. Донеччина - це Биков і Бубка, а не родина
Алчевських, Олекса Тихий та Іван Дзюба. А Харківщина - це аж ніяк не Микола
Міхновський.  Словом - це маніфест СОВКА, для якого найбільша паніка - це
червоно-чорна націоналістична повстанська протимосковська Україна.  

\href{https://www.radiosvoboda.org/a/video-den-nezalezhnosti-promova-zelenskoho/31425597.html}{%
Промова президента Зеленського до Дня Незалежності України (відео), radiosvoboda.org, 24.08.2021%
}

\ifcmt
  pic https://external-cdt1-1.xx.fbcdn.net/safe_image.php?d=AQH0pVjcxBXEf4NN&w=500&h=261&url=https%3A%2F%2Fgdb.rferl.org%2F6411ea7c-47ad-400c-88d6-939d9fd9ca44_w1200_r1.jpg&cfs=1&ext=jpg&_nc_oe=6e93c&_nc_sid=06c271&ccb=3-5&_nc_hash=AQFUIuzu0TSTLVjw
  width 0.4
\fi

Він, його оточення і значною мірою його виборець - це боягузтво, страх,
безхребетність і НЕЗНАННЯ сили самих українців. Це довгий половинчастий і
непевний шлях України в лабіринтах безвольности та інфантилізму. Буду активно
надалі працювати саме над пропагандою націоналістичних  ідей, що дають реальну
силу Україні, а не попсове видовище з беззахисною синьо-жовтою квіткою, що в
декого навіть викликає безвольно-сентиментальні сльози там, де має бути вістря
національної ідеї та брязкіт зброї.

© Ірина Фаріон
