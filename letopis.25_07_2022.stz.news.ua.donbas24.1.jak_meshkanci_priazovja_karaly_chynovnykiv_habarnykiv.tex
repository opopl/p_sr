% vim: keymap=russian-jcukenwin
%%beginhead 
 
%%file 25_07_2022.stz.news.ua.donbas24.1.jak_meshkanci_priazovja_karaly_chynovnykiv_habarnykiv
%%parent 25_07_2022
 
%%url https://donbas24.news/news/yak-meskanci-priazovya-v-minulomu-karali-necesnix-cinovnikiv-xabarnikiv-yaki-zlovzivali-vladoyu
 
%%author_id demidko_olga.mariupol,news.ua.donbas24
%%date 
 
%%tags 
%%title Як мешканці Приазов'я в минулому карали нечесних чиновників-хабарників
 
%%endhead 
 
\subsection{Як мешканці Приазов'я в минулому карали нечесних чиновників-хабарників}
\label{sec:25_07_2022.stz.news.ua.donbas24.1.jak_meshkanci_priazovja_karaly_chynovnykiv_habarnykiv}
 
\Purl{https://donbas24.news/news/yak-meskanci-priazovya-v-minulomu-karali-necesnix-cinovnikiv-xabarnikiv-yaki-zlovzivali-vladoyu}
\ifcmt
 author_begin
   author_id demidko_olga.mariupol,news.ua.donbas24
 author_end
\fi

\ii{25_07_2022.stz.news.ua.donbas24.1.jak_meshkanci_priazovja_karaly_chynovnykiv_habarnykiv.pic.front}

\begin{center}
  \em\color{blue}\bfseries\Large
Наприкінці XIX століття в Маріуполі один чоловік вирішив поборотися за
відновлення справедливості та провчити чиновника, який зловживав владою
та займався систематичним хабарництвом
\end{center}

Завдяки збереженій книзі \emph{\textbf{\enquote{Історія одного дня} Петра Каменського}} — депутата
третього скликання Державної Думи Російської Імперії від Катеринославської
губернії — до нас дійшла одна неймовірна історія, яка ілюструє кмітливість,
сміливість та неабияку винахідливість мешканців Приазов'я. Сама ж книга
заснована на матеріалах реальної кримінальної справи, а викладені в ній події
дуже нагадують зміст комедійної п'єси \emph{Миколи Гоголя \enquote{Ревізор}}. Історія,
присвячена одному чоловікові, який вирішив провчити колишнього члена грецького
суду Маріуполя Логафетова за зловживання довіреною йому владою.

\ii{25_07_2022.stz.news.ua.donbas24.1.jak_meshkanci_priazovja_karaly_chynovnykiv_habarnykiv.pic.1}

\subsubsection{Про постраждалого Піддубню}

\textbf{20 березня 1863 року} купець \emph{\textbf{Піддубня}}, який торгував товарами, розповідав двом
своїм знайомим \emph{\textbf{Мазину}} і \emph{\textbf{Колосовському}} та незнайомцю, який сидів поруч —
\emph{\textbf{Григорію Власову Ілляшенку}} — про нещастя, які трапилися з ним в місті
Маріуполь. Григорій був чоловіком 33 років, блондин, ніяких особливих прикмет
не мав. В описуваний час проживав із дружиною у місті Бердянськ, де займався
приватними креслярськими роботами місцевого архітектора. До прибуття до
Бердянська Ілляшенко перебував на службі в севастопольському інженерному
відрядженні морської будівельної частини креслярем і був нагороджений, як це
засвідчується офіційними документами, бронзовою медаллю на Андріївській
стрічці. Піддубня розповів своїм слухачам, як його обік\hyp{}рав колишній член
грецького суду \emph{\textbf{Микола Логафетов}}. У 1861 році він так вдало розторгувався, що
заробив 5015 рублів. У тому ж шинку, де він відпочивав після вдалої торгівлі,
був і Логафетов зі своїм загоном. Логафетов в стані надмірного алкогольного
сп'яніння став буйствувати і зажадав, аби музиканти, які розважали Піддубню,
негайно почали грати перед ним. Виявилося, що у Логафетова за спиною подібних
злочинів було безліч і він зі своїм загоном продовжував уважно \enquote{стежити за
порядком} в Маріуполі... 

\ii{25_07_2022.stz.news.ua.donbas24.1.jak_meshkanci_priazovja_karaly_chynovnykiv_habarnykiv.pic.2}

Піддубня намагався захищати свої права, говорячи, що музикантам заплатив він і
що він не відпустить їх \enquote{поки вони не відіграють свого}. Сталося непорозуміння,
внаслідок якого Піддубня виявився викинутим із зав'язаними назад руками в якусь
напівтемну комору, названу арестантською камерою. Тут Піддубня заснув міцним
п'яним сном, від якого отямився годин через п'ять, коли якась невідома людина
стала виштовхувати його на волю. Жах охопив в'язня, коли він помітив
відсутність грошей. Піддубня одразу протверезів і зрозумів, що вкінець
розорений, бо там під жилетом знаходився виторг за весь розпроданий товар.
Чоловік усвідомив, що тепер не було ані товару, ані грошей. Залишалася
бідність. Він звернувся за допомогою до місцевої влади, але це виявилося
марним.

\begin{leftbar}
\emph{\enquote{Логафетов багатий, у нього велика рідня, знайомства, наш брат нічого не вдіє!
Там прокляте грецьке царство}}, — пояснював Піддубня своїм слухачам. 
\end{leftbar}

Писав Піддубня скаргу і губернатору, що викликало розслідування особливо
відрядженого чиновника, який виписав багато паперів, але ні до якого результату
не дійшов...

\subsubsection{Про благородну справу Ілляшенка}

Після цієї розмови всі, крім головного потерпілого, вирішили їхати в місто
Маріуполь та карати Логафетова. Так, \emph{\textbf{купець Мазин, міщанин Колосовський та
відставний кресляр Григорій Власов Ілляшенко}}, який і очолив цю важливу і
благородну справу, відправилися до Маріуполя. \textbf{6 квітня 1863 року} вони заїхали
до Маріуполя з боку Бердянська. Того ж дня ввечері, коли вже стемніло,
Ілляшенко розшукав будинок Логафетова і прийшов до нього. Він жив приспівуючи:
був неодружений, багатий, не особливо старий, йому було близько 50 років,
недугами не страждав, у місті мав багатих і сильних родичів, серед яких він був
своєю і дорогою для них людиною. Словом, жилося йому добре, безтурботно,
спокійно; його службові подвиги не турбували, бо всі розслідування про його
минулі діяння скінчилися дуже благополучно: ні про які судові переслідування не
йшлося. Ось чому Логафетов зверхньо поставився до Ілляшенка, коли той почав
йому пояснювати, що йому загрожують неприємності за скаргами Піддубні та коли
Ілляшенко запропонував йому своє сприяння для залагодження цієї справи. Зрештою
Логафетов грубо випроводив Ілляшенка і, закриваючи за ним двері, пропустив повз
вуха звернені до нього слова:

\begin{leftbar}
\emph{\enquote{Ти мене ще згадаєш!}}, — наголосив Ілляшенко.
\end{leftbar}

\ii{25_07_2022.stz.news.ua.donbas24.1.jak_meshkanci_priazovja_karaly_chynovnykiv_habarnykiv.pic.3}

Озброївшись зухвальством, самовпевненістю і підробленими документами, Ілляшенко
після візиту до Логафетова починає наводити жах на провідних громадян
Маріуполя. Виявивши неабиякі акторські здібності, він представився
уповноваженим імператора і розпорядився 

\begin{quote}
\em\enquote{за грабежі та вбивство і взагалі за
всі зловживання позбавити (Логафетова) всіх прав зі засланням в алтайські
заводи у вічні працівники, а маєток його продати з публічного торгу і
задовольнити всіх боржників...}. 
\end{quote}

Маріупольська влада, налякана \enquote{начальницькою} особою, завзято виконала
всі вказівки, заарештувала Логафетова і навіть поголила йому півголови.

\ii{25_07_2022.stz.news.ua.donbas24.1.jak_meshkanci_priazovja_karaly_chynovnykiv_habarnykiv.pic.4}

Всі в Маріуполі повірили, що Ілляшенко є високопосадовцем і всі накази
справжні. Навіть капітан команди Іван Павлович Лисенко не впізнав почерк свого
писаря і виконував всі накази лженачальника. Більше того, на суді над
Логафетовим Ілляшенко настільки сміливо себе поводив, що всі остаточно
переконалися — перед ними справжній Уповноважений Государя. Логафетов змирився
зі своєю участю і одразу ж був поголений. Тепер вже в ролі арештанта він чекав
на виїзд з міста. А маріупольці робили все, щоб догодити високому начальнику,
Уповноваженому Государя Імператора Всеросійського Ілляшенку, який на застіллі,
влаштованому на його честь, так багато випив, що виїхати з міста одразу не зміг
і залишився ще відпочивати. Мабуть, якщо виїхав би раніше, ця історія
закінчилася б для нього цілковитим тріумфом. Однак в Маріуполі знайшлися ті,
хто зрозумів, що він — справжній самозванець і всі його накази є вигадкою.
Тому, на жаль, благородна справа Григорія Ілляшенка було розкрита... \emph{І все ж йому
вдалося помститися за свого знайомого Піддубню та покарати чиновника-хабарника,
який зловживав владою.}

\subsubsection{Про автора книги}

\ii{25_07_2022.stz.news.ua.donbas24.1.jak_meshkanci_priazovja_karaly_chynovnykiv_habarnykiv.pic.5}

Автор книги — російський громадський діяч і політик, член III Державної думи
від Катеринославської губернії \emph{\textbf{Петро Валерійович Каменський.}} Він займав посаду
почесного мирового судді Бахмутського округу (з 1881) і Маріупольського повіту
(з 1891 року), був гласним Маріупольського повітового і Катеринославського
губернського земських зборів, а також Маріупольським повітовим головою
дворянства (1899−1913). Оскільки в повіті практично не було дворянства, свою
діяльність переважно спрямовував в інтересах сільського населення: українських
та російських селян, колишніх німецьких колоністів, селян-греків і
євреїв-землевласників. Був відомим лектором і публіцистом. Якщо переглянути
список творів Петра Валерійовича, виникне враження, що це доволі серйозний
лектор, теми якого торкалися релігії, благодійності, освіти, боротьби з
бідністю... Але водночас цьому благородному діячу вдавалося писати з гумором та
легкою іронією, адже саме так написана книга, присвячена подіям, що відбулися в
Маріуполі \emph{\textbf{6 квітня 1863 року.}} Петро Валерійович Каменський наголошував, що вся
ця історія була реальною. Всі відомості П. Каменський отримав від
маріупольських старожилів, які були не тільки безпосередніми свідками, але й
навіть частково учасниками описуваних подій.

Нагадаємо, раніше Донбас24 розповідав про \href{https://donbas24.news/news/vulicya-bogdana-stupki-v-pokrovsku-yakim-buv-zittjevii-slyax-ukrayinskogo-mitcya}{%
вулицю Богдана Ступки в Покровську: яким був життєвий шлях українського митця}.%
\footnote{Вулиця Богдана Ступки в Покровську: яким був життєвий шлях українського митця, Алевтина Швецова, donbas24.news, 25.07.2022, %
\par\url{https://donbas24.news/news/vulicya-bogdana-stupki-v-pokrovsku-yakim-buv-zittjevii-slyax-ukrayinskogo-mitcya}}

ФОТО: з відкритих джерел.

\ii{insert.author.demidko_olga}
%\ii{25_07_2022.stz.news.ua.donbas24.1.jak_meshkanci_priazovja_karaly_chynovnykiv_habarnykiv.txt}
