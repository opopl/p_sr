%%beginhead 
 
%%file 24_02_2023.fb.demidko_olga.mariupol.1.rik_vijny_rik_zhyttja
%%parent 24_02_2023
 
%%url https://www.facebook.com/100009080371413/posts/pfbid02wBNNqaPVZt2my8qDYYsay45EndyJVkg78Dj6RaZYXG89AzY1khVVsLK7vARYg1Fgl
 
%%author_id demidko_olga.mariupol
%%date 24_02_2023
 
%%tags 
%%title Рік повномасштабної війни. І рік життя  у стані постійної загрози та невизначеності
 
%%endhead 

\subsection{Рік повномасштабної війни. І рік життя  у стані постійної загрози та невизначеності}
\label{sec:24_02_2023.fb.demidko_olga.mariupol.1.rik_vijny_rik_zhyttja}

\Purl{https://www.facebook.com/100009080371413/posts/pfbid02wBNNqaPVZt2my8qDYYsay45EndyJVkg78Dj6RaZYXG89AzY1khVVsLK7vARYg1Fgl}
\ifcmt
 author_begin
   author_id demidko_olga.mariupol
 author_end
\fi

Рік повномасштабної війни. І рік життя  у стані постійної загрози та
невизначеності. Рік, коли почала цінувати прості речі, такі як: ранкова кава,
можливість ходити на роботу, зустрічатися з друзями, навіть такі блага, як
світло та водопостачання сприймаються нами зовсім по-іншому. Рік тому багато
моїх знайомих зняли маски і показали себе справжніх... 

Але цей рік зробив кожного з нас точно сильнішим. Я радію, що мої друзі почали
говорити українською, вивчати нашу історію, культуру і традиції. Ми зрозуміли,
як важливо турбуватися одне про одного і бути поруч у найскладніші моменти. Рік
тому я відчула надвідповідальність за свого сина та батьків. Намагалася робити
все можливе, щоб маленькому Данилку не було страшно. Назавжди в пам'яті
залишаться кімнати мого будинку - затишні, комфортні, зроблені з любов'ю....
Вони зігрівали мою сім'ю в перші дні війни. Сьогодні я вдячна, що мій синочок
та батьки  вижили, що зараз ми всі разом.❤🙏 Ніколи не забуду своє місто рік
тому. Пам'ятаю, що було страшно їхати на Маріупольське телебачення знімати
сюжет, але я хотіла бути корисною. Вперше в житті всю дорогу їхала в порожньому
автобусі. Поруч була тільки налякана кондукторка. З кожним днем ставало все
важче. Рятували лише надія і віра. Все життя пам'ятатиму обличчя вдячної жінки
з Талаківки, будинок якої був зруйнований в перші дні. А у неї троє маленьких
діточок... Я зібрала іграшки малого, його одяг, подушки, ковдри, посуд і
відвезла їй. Вона обійняла мене і ми заплакали. Найважчі спогади, пов'язані з
моєю бабусею. Її привезли до мене з Лівого берега. У будинок, в якому вона
прожила більшу половину життя, потрапив снаряд. Їй було найважче.

Ніколи не зможу пробачити, що у моєї бабусі  забрали спокій в очах, забрали
життя... Не забуду погрози, які мені приходили лише за те, що я завжди була
свідомою громадянкою своєї країни. Назавжди зберігатиму вдячність і щиру любов
до своїх сусідів - дяді Серьожі, тітки Ріти, дідуся Петра та тітки Віти, які
прихистили мене з малим, які нас врятували.

Можливо, саме ми, ті хто пережили цей страшний рік війни,  будемо розповідати
онукам, як це жити без тепла, світла, зв'язку та води... Як це, коли тебе
охоплює страх та відчай, зберегти ясність розуму і намагатися врятувати свою
сім'ю,  намагатися вижити і зберегти Людяність...
