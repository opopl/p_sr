% vim: keymap=russian-jcukenwin
%%beginhead 
 
%%file slova.raduga
%%parent slova
 
%%url 
 
%%author 
%%author_id 
%%author_url 
 
%%tags 
%%title 
 
%%endhead 
\chapter{Радуга}
\label{sec:slova.raduga}

%%%cit
%%%cit_head
%%%cit_pic
\ifcmt
  pic https://avatars.mds.yandex.net/get-zen_doc/4697586/pub_60df31aabaf4b439d35409f4_60df31dcae4f98420edb72a5/scale_1200
	width 0.4
\fi
%%%cit_text
Беда в том, что живу в чудной стране в которой не возможно даже фоном без
тормозов запустить российские трансляции этого дивного чемпионата,
раскрашенного в цвета напористого прайдовского небинарного меньшинства. Брр!
Поэтому приходится слушать местных комментаторов, а это в купе с цветовым
винегретом косящим под \emph{радугу}, тот ещё симбиоз. Особенно в то время когда они
суматошно придумывают свой инклюзив, рождая каждый день по дюжине новых
непроизносимых буквенно-звуковых сочетаний, выдавая их за истинный и
неповторимый украинский язык. Аж сипало меня когда они раза три-четыре за два
тайма произносили последний из перлов
%%%cit_comment
%%%cit_title
\citTitle{Яйца, но не роковые}, Дмитрий Жук (ЦИНИК), zen.yandex.ru, 02.07.2021
%%%endcit
