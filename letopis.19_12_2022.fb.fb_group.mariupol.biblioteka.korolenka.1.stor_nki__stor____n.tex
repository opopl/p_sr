%%beginhead 
 
%%file 19_12_2022.fb.fb_group.mariupol.biblioteka.korolenka.1.stor_nki__stor____n
%%parent 19_12_2022
 
%%url https://www.facebook.com/groups/1476321979131170/posts/5665013773595282
 
%%author_id fb_group.mariupol.biblioteka.korolenka,lisogor_viktoria.mariupol
%%date 19_12_2022
 
%%tags mariupol,mariupol.istoria,jalynka,novyj_god
%%title Сторінки історії - Новорічна ялинка
 
%%endhead 

\subsection{Сторінки історії - Новорічна ялинка}
\label{sec:19_12_2022.fb.fb_group.mariupol.biblioteka.korolenka.1.stor_nki__stor____n}
 
\Purl{https://www.facebook.com/groups/1476321979131170/posts/5665013773595282}
\ifcmt
 author_begin
   author_id fb_group.mariupol.biblioteka.korolenka,lisogor_viktoria.mariupol
 author_end
\fi

\textbf{Сторінки історії}

\begingroup
\em\bfseries
Про те, як маріупольці готувалися до новорічних свят, у розповідях краєзнавців нашого міста.

Новорічна ялинка
\endgroup

Яким було це свято у дореволюційному Маріуполі? Ось те, що було почуте багато
років тому від людей, які період вітчизняної історії до 3 серпня 1914 року,
тобто до початку Першої світової війни, називали «мирним часом». Ялинку
зазвичай ставили на Різдво посередині найбільшої кімнати в будинку, попередньо
звільнивши для неї місце від зайвих меблів. Її вінчала шестикінцева зірка,
обклеєна блискучою золотистою фольгою, що символізувала зірку, що зійшла над
Віфлеємом і сповістила світ про народження Христа. На різдвяне дерево на
ниточках підвішували мандарини, яблука, груші, печиво, льодяники та шоколадні
цукерки, інші солодощі, іграшки-подарунки. Діти водили навколо ялинки хороводи,
а потім читали вірші, співали пісеньки, і за це кожен із них міг вибрати та
зняти собі подарунок.

Минулими роками серед старожилів нашого міста передавалася з вуст у вуста
історія про те, як святкували Різдво в одному з міщанських будинків на Торговій
вулиці. Ялинка для дорослих прикрашалася «мерзотниками» - пляшечками з
горілкою, коньяком і марочними винами, мисливськими сосисками, гірляндами з
нарізаних до товщини цигаркового паперу пластику шинки, балика, голландського
сиру, малосольними ніжними огірками, дрібними винних та горілчаних наклейок.
Після гри у «фанти» всі йшли до святкового столу і вже потім приступали до
танців під піаніно, за яке сідала одна з дочок господарів цього будинку. А
вранці, тільки-но відійшовши від сну, прямували до ялинкових прикрас, тут
щосили йшли і вміст «мерзотників», і закуска до них.

На різдвяні свята влаштовували і розваги. На озері Домаха – його наприкінці
двадцятих років минулого століття засипали – влаштовувалась ковзанка, освітлена
електричними лампочками, у центрі її встановлювали ялинку, ковзани давали
напрокат. У кіосках пихкали самовари - можна було випити склянку-другу чаю.
Пані та кавалери до пізньої ночі ковзали на ковзанах по льоду під звуки
духового оркестру. А ті, хто боявся застудитись на морозному повітрі, міг
вирушити до цирку братів Яковенка, де споруджувався скейтинг-ринг. На арені
робили дощатий настил, і всі бажаючі могли за порівняно невелику плату кататися
на ковзанах, тільки роликових.

Середина другого десятиліття ХХ ст. Оголошення в газеті «Маріупольське життя»:
«Ялинка та танцювальний вечір. Першого дня Різдва маріупольське товариство
піклування про дітей влаштовує у міській управі ялинку для дітей та
танцювальний вечір для дорослих. Початок ялинки о 4 годині та до 8 години
вечора, після 8 години вечора – танці для дорослих».

Ялинки різних розмірів отримані. Бондарна вулиця, №3 (на подвір'ї, де бондарна
майстерня Бочова). Ціни помірні». Бондарна вулиця – це ділянка нинішньої
Італійської від залізничного переїзду до Земської.

На благодійні вечори запрошували дітей із бідних сімей, їм дарували подарунки.
А для збору коштів влаштовували лотереї та аукціони, на яких розігрувалися
вишивки, плетіння з бісеру, різні вироби, пожертвувані більш менш заможними
людьми.

«26 грудня 1916 року у приміщенні реального училища Батьківський комітет
маріупольської жіночої гімназії влаштовує ялинку. Початок для молодших о 5
годині, для старших – о пів на 9 вечора. Плата за вхід – 75 коп.». 75 копійок –
це багато чи мало було для наших пращурів? Для порівняння наведемо опубліковані
у тому самому номері «Маріупольського життя» ціни на товари Маріупольського
міського продовольчого комітету: цукор пісок – 22 коп. за фунт (фунт відповідає
410 г – С.Б.), крупа гречана – 13 коп. за фунт, рис – 35 коп. за фунт. Судіть
самі.

Утримувачі видовищних підприємств перед святами мали звичай вітати через газету
своїх потенційних глядачів та слухачів. Ось три приклади таких поздоровлень.
Номер «Маріупольського життя» за 1 січня 1917 року: «Художній театр «ХХ
століття» вітає шанованих відвідувачів із Новим роком! Найкращі побажання!»,
«О.С. Подольний (власник одного з маріупольських кінотеатрів – С.Б.) вітає з
Новим 1917 роком і надсилає кращі побажання», «Влаштовувач концертів М.Я.
Немирівський має честь привітати з Новим роком і надсилає найкращі побажання».

6 грудня 1918 року в маріупольському порту висадився десант англо-французьких
військ, а наприкінці грудня місто зайняли загони Добровольчої армії під
командуванням генерала Май-Маєвського. "Маріупольські вісті" за 1 січня 1919
року. Оголошення: «У середу, 1 січня н.с. у приміщенні офіцерських зборів 48
пішого полку (магазин братів Адабашевих) буде влаштований вечір із благодійною
метою. Після вистави – музично-вокальний вечір за участю Ф.П. Городецького та
А.А. Терпіловського та учнів музичного училища. На закінчення – танці»...

Із встановленням радянської влади на території колишньої Російської імперії
влаштування новорічної ялинки було оголошено буржуазним пережитком і фактично
заборонено. 28 грудня 1935 року в газеті «Правда» було опубліковано невелику
замітку другого секретаря ЦК КП(б) України Павла Петровича Постишева під
назвою: «Давайте організуємо до Нового року дітям гарну ялинку!» Він писав, що
у дореволюційний час буржуазія та чиновники завжди влаштовували на Новий рік
своїм дітям ялинку. Діти ж робітників із заздрістю через вікно дивилися на
лісову красуню, що сяяла вогнями. Далі П.П. Постишев пише: «Чому в нас школи,
дитячі будинки, ясла, дитячі клуби, Палаци піонерів позбавляють цього чудового
задоволення дітлахів трудящих Радянської країни? Якісь, не інакше як «ліві»
загибелі, ославили цю дитячу розвагу як буржуазну витівку. Слід цьому
неправильному осуду ялинки покласти край. Комсомольці, піонер працівники мають
під Новий рік влаштувати колективні ялинки для дітей». І вже 1 січня 1935 року
в усіх дитячих закладах країни, і в Маріуполі, зокрема, з'явилися ялинки,
щоправда, замість традиційних Віфлеємських зірок на їхніх верхівках були
укріплені червоні п'ятикутні зірки.

Напередодні 1944 року. Йде війна. Маріуполь у руїнах. Німці ще у Криму. Ялинка
для дітей фронтовиків у середній школі №2 (вул. Радянська, 6). Невелика, метра
півтора заввишки сосонка, поставлена на табуретку. Мами з дошкільнятами.
Хороводи під музику баяніста-інваліда. Роздача подарунків за талонами.
Подарунки в маленьких кульках, склеєних зі старих газет і аркушів, видертих із
якоїсь довоєнної комори. У кульці – яблуко, кілька цукерок-подушечок, булочка
із сірого борошна розміром трохи більше горіха. У когось не було талону.
Сльози.

І в цей час «ялинка» вдома. Її роль виконує домашня квітка - китайська троянда,
прикрашена дивом, що з довоєнного часу збереглися ялинковими іграшками. Кілька
скляних куль із заглибленнями на боках, пофарбованих у яскраві, таємниче
мерехтливі кольори, картонні зайчики, рибки, метелики, що сяють станіолевим
блиском. Їх дістали з круглої капелюшної коробки, де вони зберігалися між
клаптями вати. Решта – саморобні прикраси. Ланцюги, склеєні з різнокольорових
обкладинок старих шкільних зошитів, звірятка та фрукти з вати, особливим чином
оброблені картопляним борошном, розфарбовані акварельними фарбами, гірлянди
прапорців – четвертинок старих кольорових листівок.

«Приазовський робітник», 1 січня 1945 р.: «Багато веселощів принесе Новий рік
нашим хлопцям. Сьогодні учні молодших класів зберуться вранці біля ялинки у
Будинку піонерів. Учні драмгуртка поставлять п'єсу «Серце друга». Потім
розпочнуться масові ігри навколо ялинки. Учні шкіл Іллічівського району
зустрінуться з учасниками Вітчизняної війни та розкажуть про свої успіхи у
навчанні. Хлопці побують у кіно, театрі, візьмуть участь у змаганнях із
штикового бою».

Ось такі вийшли строкаті нотатки про святкову ялинку в Маріуполі у різні роки
його історії.

С. Д. Буров

//Вечірній Маріуполь.-2010.-22 грудня.
