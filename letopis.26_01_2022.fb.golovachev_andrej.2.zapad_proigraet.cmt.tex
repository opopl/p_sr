% vim: keymap=russian-jcukenwin
%%beginhead 
 
%%file 26_01_2022.fb.golovachev_andrej.2.zapad_proigraet.cmt
%%parent 26_01_2022.fb.golovachev_andrej.2.zapad_proigraet
 
%%url 
 
%%author_id 
%%date 
 
%%tags 
%%title 
 
%%endhead 
\zzSecCmt

\begin{itemize} % {
\iusr{Serhiy Dovbnya}

Да, я тоже об этом давно думал. \enquote{Героїчна козацька доба} - это и есть
культурный код. Это приговор. Пока мы будем гордиться, что нашими истоками
является сообщество разбойников и рекетиров (чем по сути и было запорожское
казачество; оно даже принципиально отличается этим от Донского), до тех пор на
будущем можно ставить жирный крест. Украинское общество мало того что
православное - оно крайне ксенофобное, чем сильно отличается от других
православных народов: оно выковыряло евреев и поляков - до 1940-х они
составляли более 1/3 населения запада Украины. Фактически, шанс построения другой
Украины был потерян уже тогда

\begin{itemize} % {
\iusr{Нина Постой}
\textbf{Serhiy Dovbnya} кому що- а курці просо...

\iusr{Serhiy Dovbnya}
Армія, мова, віра! Ура!!!
\end{itemize} % }

\iusr{Виктор Марченко}
Я называю это явление генетическим кодом (или генетической памятью).

\iusr{Слава Марков}

Многое не сходится. Междоусобицы между славянскими княжествами были
исключительно кровопролитными. Турция со дня основания была светским
государством ( в отличии от Османской империи), а православный Константинополь
сгинул в лета и даже через три поколения не вернулся


\iusr{Даша Маркова}

Жаль, вы не поняли. Нет давно Украины, за долги куплена. И ща будет продана
саудитам. Летом запустили оттудова лоукосты, семьями прилетали - выбирать земли
себе.

Населению, кто еще не сбежал, расскажут - \enquote{почем вода в пустыне!}

\iusr{Татьяна Грицай}

Европейская демократия в ближайшем рассмотрении оказалась настолько аморальной
и опять же коррумпированно-продажной, да и внедрять ее хотели не из
бескорыстных соображений, а в обмен на бусики за большие деньги.


\iusr{Alexandr Barsukov}

Все это гадания на кофейной гуще. Какой православный мир, если самая
религиозная часть Украины- западная. А там католики и греко-католики. А на
востоке традиций на выходные в церковь ходить нет уже. Не так уж много людей в
Украине религиозных. И то, что в Украине один человек править будет как в
России, мне кажется тоже маловероятно. У нас бунт в крови. 2 майдана за 10
лет. И Украина разная от востока до запада. Неоднородное общество.

\end{itemize} % }
