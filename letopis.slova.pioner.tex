% vim: keymap=russian-jcukenwin
%%beginhead 
 
%%file slova.pioner
%%parent slova
 
%%url 
 
%%author 
%%author_id 
%%author_url 
 
%%tags 
%%title 
 
%%endhead 
\chapter{Пионер}

%%%cit
%%%cit_head
%%%cit_pic
\ifcmt
tab_begin cols=2
  pic https://icdn.lenta.ru/images/2021/07/27/14/20210727145757477/preview_beeb8136c5c30094dd01c7a925f91a37.jpg

	pic https://icdn.lenta.ru/images/2021/07/27/17/20210727173843680/preview_a9d4efb04945f37c08bc726643550ae4.jpg
tab_end
\fi
%%%cit_text
Валентин, Москва: Мне казалось, что \emph{пионерия} — это некое общее дело, которым мы
будем заниматься вместе с друзьями. Не просто будем хорошими сами по себе, а
целенаправленно.  Вся \emph{пионерская} история воспринималась как кружок, в котором
нужно будет выполнять задания, нацеленные на увеличение добра вокруг (даже не в
мире). Было приятно ощутить себя полезным и правильным.  Что такое \emph{пионерское}
движение, я знал только по детским книгам (вроде Тимура с командой). Мне
хотелось туда, потому что это было здорово — вместе с друзьями заниматься
чем-то интересным. У нас был хороший класс, мы дружили не только группами, но и
все вместе с удовольствием общались.  В то время я, как наивный идеалист, видел
все в радужных цветах. Может быть, оттого, что меня воспитывали мама и бабушка,
а в классе была очень уютная и домашняя училка. Мы были какими-то одуванчиками.
Не было строгости, не было какой-то жуткой «советской ответственности». Поэтому
быть \emph{пионером} было здорово
%%%cit_comment
%%%cit_title
\citTitle{«Это слишком жестко для ребенка» Последние пионеры СССР — о крахе страны и ее идеологии: Общество: Россия: Lenta.ru}, 
Михаил Карпов, lenta.ru, 04.08.2021
%%%endcit
