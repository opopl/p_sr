% vim: keymap=russian-jcukenwin
%%beginhead 
 
%%file slova.mir.peace
%%parent slova
 
%%url 
 
%%author 
%%author_id 
%%author_url 
 
%%tags 
%%title 
 
%%endhead 
\chapter{Мир (Не-Война)}
\label{sec:slova.mir.peace}

%%%cit
%%%cit_head
%%%cit_pic
%%%cit_text
Отсюда был и главный посыл Блогера – я принесу Украине \emph{мир}. И \emph{«мир»} это не
только, когда не стреляют и не убивают. \emph{Мир} начинается с прекращения
расчеловечивания. И там где \emph{мир} – там нет места сегрегации граждан по
идеологии, религии или языку. \emph{Мир} – это уважение инаковости.  И вот июнь 21
года, а Данилов стэндапит о водичке для крымчан. Хотели? Ах-ха, бог вам дал ее
многовато. Но разве до Данилова не было Резникова с его «раковой опухолью»
Донбасса? Или не было Кравчука, у которого Донбасс все та же раковая опухоль?
Или может не при Зеленском прямо сейчас травят на теме языка какую-то
харьковскую девчушку? Или людей у нас перестали увольнять за отказ обслуживать
клиента не на том языке?
%%%cit_comment
%%%cit_title
\citTitle{Бог Данилова - это бог расчеловечивания}, 
Игорь Лесев, strana.ua, 20.06.2021
%%%endcit


%%%cit
%%%cit_head
%%%cit_pic
%%%cit_text
\enquote{22 июня 1941 года, ровно 80 лет назад, началась Великая Отечественная война,
принеся за собой миллионы жертв и сломанных судеб.  Этот день навсегда изменил
жизнь свидетелей тех событий — ужасы войны коснулись каждой семьи. Мы помним и
скорбим вместе со всеми, кто потерял родных и близких.  Память об этих страшных
страницах истории нашей страны — урок для всех нас. Это напоминание о том, что
\emph{мир} — самая большая ценность. И наш долг — сделать все ради возвращения \emph{мира} в
нашу страну, чтобы Донбасс и вся Украина вернулись к \emph{мирной} и спокойной жизни.
Чтобы остановить гибель людей.  Этого ждут украинцы. И мы все сделаем все,
чтобы наша страна стала \emph{мирной}, стабильной и процветающей}, - написал Бойко и
опубликовал фото с возложения цветов 
%%%cit_comment
%%%cit_title
\citTitle{22 июня в Киеве - что говорили украинские политики}, Екатерина Терехова, strana.ua, 22.06.2021
%%%endcit
