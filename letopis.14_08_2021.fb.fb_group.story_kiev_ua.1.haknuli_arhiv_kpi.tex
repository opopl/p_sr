% vim: keymap=russian-jcukenwin
%%beginhead 
 
%%file 14_08_2021.fb.fb_group.story_kiev_ua.1.haknuli_arhiv_kpi
%%parent 14_08_2021
 
%%url https://www.facebook.com/groups/story.kiev.ua/posts/1729208207275973
 
%%author_id fb_group.story_kiev_ua,mednikov_jan.kiev
%%date 
 
%%tags kiev,kpi,obrazovanie,studenty
%%title КАК МЫ «ХАКНУЛИ» АРХИВ КПИ
 
%%endhead 
 
\subsection{КАК МЫ «ХАКНУЛИ» АРХИВ КПИ}
\label{sec:14_08_2021.fb.fb_group.story_kiev_ua.1.haknuli_arhiv_kpi}
 
\Purl{https://www.facebook.com/groups/story.kiev.ua/posts/1729208207275973}
\ifcmt
 author_begin
   author_id fb_group.story_kiev_ua,mednikov_jan.kiev
 author_end
\fi

КАК МЫ «ХАКНУЛИ» АРХИВ КПИ.

Сначала немного терминологии.

«Коза» – чертеж или образец проекта, который можно целиком или частично
скопировать при выполнении домашнего задания, курсовой работы или диплома.

«Козлотрон» – устройство для копирования бумажных чертежей, состоящее из
огромного толстого стекла (ширина до 2 метров), лежащего на двух опорах с
настольной лампой внизу. На стекло клали оригинал чертежа, поверх него чистый
лист ватмана. Лампа проецировала чертеж на чистый лист, и оставалось его только
обвести и проставить нужные цифры и надписи.

\ifcmt
  ig https://scontent-frx5-1.xx.fbcdn.net/v/t39.30808-6/238561716_4272572799477301_7304240216408622867_n.jpg?_nc_cat=105&ccb=1-5&_nc_sid=825194&_nc_ohc=OYBE96Ld2_YAX81CzZq&_nc_ht=scontent-frx5-1.xx&oh=00_AT84iEENAMZHojaJLoHTtib0-G9IxNb9EfXOAzd9n_yNNw&oe=61C21C77
  @width 0.4
  %@wrap \parpic[r]
  @wrap \InsertBoxR{0}
\fi


Одним из самых противных курсовых проектов, который должен был сдать каждый
инженер-механик, был ПТМ – «Подъемно-транспортные механизмы и устройства».
Конструкции кранов или чертежи грузовых крюков не менялись многие десятилетия,
а необходимость их рассчитывать и рисовать, вместо того, чтобы выбрать по
каталогу, сильно раздражала. С другой стороны, однотипность чертежей и расчетов
позволяли козлить старые проекты, которые в общаге передавались из рук в руки
многими поколениями студентов. Но коз было мало, а вариантов курсовых заданий
много.

Пришло время, и я получил свое задание на проектирование мостового крана.
Побродил по общаге, нашел что-то похожее, взял чертеж крюка и понес его в
дискотеку, в клуб «Политехник». Я был членом Совета клуба, а в Совете работал
Костя Зворыкин, аспирант с кафедры ПТМ. Между прочим, Костя был внук профессора
Зворыкина, выдающегося ученого и ректора КПИ. Я пытался показать чертеж Косте,
но он даже смотреть отказался.

- Наша кафедра борется с «козами» и это принципиальная позиция!

- Костя, ну какая борьба? Я тебе этих коз сто штук принести могу.

- Договорились: принесешь 100 коз – получишь «отлично».

- Костя, я погорячился. Во всем студгородке и 20 коз по ПТМ не наберется. Давай
сойдемся на реальной цифре?

- Нет. Слово – не воробей. Несешь 100 коз – получаешь «отлично». Я на кафедре
договорюсь.

Влип конкретно. Краткий анализ показал, что в зоне досягаемости есть только
одно место, где можно найти 100 проектов кранов, лебедок и транспортеров. Это
архив института. Надо включать мозги и действовать.

Через пару дней в архив института пришли два аккуратно одетых студента с
комсомольскими значками. Они рассказали о том, что студенческий театр ставит
пьесу про инженеров, а денег на декорации нет. Но декорации можно сделать из
старых чертежей. Сотрудница архива поняла молодежь с полуслова.

- Вон там лежат дипломы. Берите, сколько унесете.

Студенты взяли по десятку проектов, поблагодарили и ушли. Дипломы мы выбросили
вдали от архива, чтобы случайно не попалиться, присели, перекурили, обмозговали
ситуацию.

На следующий день, после пересменки, те же два студента вновь посетили архив и
изложили откорректированную легенду про декорации.

- Так вам нужны чертежи кранов? Это в той комнате на полках вдоль стены справа.

Сотрудниц архива отблагодарили «Пражским» тортом с индийским чаем, с расчетом,
что они будут заняты минимум пол-часа. Нам предстояло вынести 120 проектов. 120
рулонов плотного ватмана и 120 толстых папок с пояснительными записками и
расчетами. Представляете себе этот объем и вес? Мы справились и доставили
документацию в надежное место. Потом быстро изучили и рассортировали добычу.
Два лучших и наиболее подходящих проекта достались участникам операции. Не
забыли профсоюзного и комсомольского вожаков. Староста был отличником и в наших
подарках не нуждался. Друзьям сообщили, что после досрочной сдачи курсовых, все
уцелевшие проекты переместятся в общагу на благо нынешнему и будущим поколениям
машиностроителей.

В оговоренный день и час в аудитории напротив кафедры ПТМ были сложены ровно
100 коз курсовых проектов. Принимать бесценный подарок явился лично зав
кафедрой с тремя преподавателями. Эта была та еще сцена! Они вчетвером рвали
чертежи, кромсали пояснительные записки, уничтожая даже твердые переплеты.
Когда вакханалия закончилась, и мы протянули зачетки, от нас потребовали
вынести всю эту макулатуру за учебный корпус и сжечь ее в мусорном баке.
Кафедра банковала, так что пришлось потрудиться еще часик. В итоге нам вместо
«отлично» поставили «хорошо» и стороны разошлись с уверенностью в том, что они
наебали партнеров.

10 лет спустя я еще видел эти козы в общаге №20, а потом пришла эпоха
компьютеров, сканеров и AutoCAD.

(На фото: Учебный корпус КПИ №19 (справа). За ним мы сжигали курсовые. 1982
год. Скан с пленки ORWO)

\ii{14_08_2021.fb.fb_group.story_kiev_ua.1.haknuli_arhiv_kpi.cmt}
