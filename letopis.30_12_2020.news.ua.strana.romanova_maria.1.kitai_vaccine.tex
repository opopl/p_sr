% vim: keymap=russian-jcukenwin
%%beginhead 
 
%%file 30_12_2020.news.ua.strana.romanova_maria.1.kitai_vaccine
%%parent 30_12_2020
 
%%url https://strana.ua/news/309533-ukraina-zakupit-kitajskuju-vaktsinu-coronavac-chto-o-nej-izvestno.html
 
%%author 
%%author_id romanova_maria,ksenz_ljudmila
%%author_url 
 
%%tags 
%%title Китайская по цене американской. 5 главных вопросов о вакцине CoronaVac, которую решила закупить Украина
 
%%endhead 
 
\subsection{Китайская по цене американской. 5 главных вопросов о вакцине CoronaVac, которую решила закупить Украина}
\label{sec:30_12_2020.news.ua.strana.romanova_maria.1.kitai_vaccine}
\Purl{https://strana.ua/news/309533-ukraina-zakupit-kitajskuju-vaktsinu-coronavac-chto-o-nej-izvestno.html}
\ifcmt
	author_begin
   author_id romanova_maria,ksenz_ljudmila
	author_end
\fi

\ifcmt
  pic https://strana.ua/img/article/3095/ukraina-zakupit-kitajskuju-33_main.jpeg
  width 0.6
	caption В Китае разрабатывали вакцину от коронавируса с января 2020 года. Фото: globaltimes.cn
\fi

Украина договорилась о покупке 1,9 млн доз китайской вакцины\Furl{https://strana.ua/news/309536-ukraina-poluchit-1-8-milliona-doz-vaktsiny-ot-covid-19-.html} от коронавируса.

Первые дозы CoronaVac производства компании Sinovac могут поставить уже в
феврале, после того как препарат утвердят в Китае. 

И это будет, судя по всему, первая вакцина от коронавируса, которую получит
Украина, так как с поставками западных вакцин есть трудности (Украина там не в
первых рядах), бесплатные вакцины через механизм COVAX неизвестно когда
поступят, а российскую вакцина наши власти не хотят брать по политическим
причинам.

Прививки от китайского производителя в первую очередь сделают украинцам из зоны
риска - тем, кто работает с больными коронавирусом, а также работникам
критически важных сфер. Вакцинировать обещают бесплатно.

При этом цена закупки китайского препарата оказалась весьма недешевой.

Также с ним могут возникнуть и политические проблемы - связанные с привычкой
американцев блокировать украино-китайские сделки. 

Ответили на пять главных вопросов о китайской вакцине, которая должна скоро
поступить в Украину.

\subsubsection{1. Что представляет собой китайская вакцина?}

Sinovac Biotech - крупнейшая китайская компания по производству вакцин. Ее
вакцина от коронавируса - CoronaVac является инактивированной вакциной, которая
выпускается в шприце и хранится при температуре от +2 до +8.

Для иммунизации нужно два укола с разницей в две недели. Вакцина построена на
базе образцов коронавируса, полученных в разных странах мира, но прежде всего -
в Китае. 

По данным компании-разработчика, вакцина CoronaVac выдерживает 42 дня при
температуре +25 и 28 дней сохраняет свои свойства при температуре +37.

Работа над созданием вакцины в Sinovac Biotech началась в январе 2020 года,
сразу после того, как в Ухане официально подтвердили распространение пневмонии
неизвестного происхождения. В июне препарат поступил на испытания, которые, к
слову, идут по сей день. 

\subsubsection{2. Испытана ли вакцина CoronaVac?}

Официально эта вакцина еще не прошла третьего этапа испытаний, его окончание
планируется в январе, когда препарат представят на переквалификацию ВОЗ. 

При этом страны, которые уже закупили у Китая первые партии CoronaVac -
Индонезия, Турция, Бразилия - проводят третью фазу на своей территории. Сигналы
оттуда идут противоречивые. 

В Индонезии проверили 1 600 пациентов, вакцина показала эффективность 97\%. В
Турции - 1 325 пациентов, эффективность 91\%. 

А вот в Бразилии, где сейчас идет третий этап испытаний, заявили, что китайская
вакцина показала около 50\% эффективности. Об этом заявил лично президент
страны Жаир Болсонару.

Поэтому пока в бразильский план вакцинации от коронавируса входят российская
вакцина "Спутник V" и британский препарат от AstraZeneca (последнего ожидается
100 млн доз). А китайский препарат находится под вопросом. 

В этой связи вспоминаются заявления украинских официальных лиц о том, что
российская вакцина "Спутник V" не проверена и поэтому ее закупать не будут
(хотя "Спутник" уже прошел третью стадию испытаний и им активно вакцинируют
население в России, Беларуси, Аргентины, а скоро и в других странах).

А теперь Киев решил купить китайскую вакцину CoronaVac, которая третью стадию
испытаний еще не закончила и не получила регистрацию даже в Китае (это
ожидается только в январе).  

Кстати, интересно, что покупать ее будут у посредника - украинской компании
"Лекхим", которая официально финансировала предвыборную кампанию партии
"Голос". Она перевела на счета политсилы Вакарчука три миллиона гривен. 

Главой набсовета "Лекхима" является Валерий Печаев. На фармрынке он не
нуждается в особом представлении. Печаев возглавляет организацию работодателей
медицинский и микробиологический промышленности, которая, как говорят эксперты,
является "рупором" отечественных операторов фармрынка. Так, эта организация в
все время выступала претив передачи государственных закупок лекарств
международным организациям.

К слову, "Лекхим" начал договариваться с китайцами еще летом. В июле на
официальном сайте компании появилась новость о том, что "Лекхим" работает над
подписанием контракта на поставки вакцины против вируса SARS-CoV-2. 

А буквально через неделю появилась новость, что "Лекхим-Харьков" приступил к
проектированию мощностей для производства профилактических вакцин против:
SARS-CoV-2, вируса гриппа (четырехвалентная вакцина), полиомиелита. Планируемый
объём производства — 30 миллионов единиц вакцин в год. Новую линию планируют
ввести в эксплуатацию в первом квартире 2022 года.

"Очевидно, изначально "Лекхим" работал по своему сценарию - искали
международного партнера сначала под дистрибуцию, а потом - под производство
вакцины от ковида. Но когда под конец года оказалось, что все планы властей по
закупке вакцин с треском провалились, и начали спешно прорабатывать "план Б",
наработки "Лекхима" пришлись как нельзя кстати», — рассказал наш источник на
фармрынке.

\subsubsection{3. Почем вакцина и на сколько людей ее хватит?}

По компании "Медзакупки Украины", речь идет о контракте на 1,9 млн доз по 504
гривны за штуку. То есть в целом закупка CoronaVac обойдется почти в миллиард.

Напомним, что для иммунитета нужно две дозы на человека. То есть поставки от
Sinovac хватит на 950 тысяч украинцев. Это 2,5\% населения страны. 

Напомним, что бюджет Украины на закупку вакцин предусматривает 2,6 миллиарда
гривен. Если все эти деньги пойдут на закупку продукции Sinovac по указанной
цене, то денег хватит на вакцинацию около двух с половиной миллионов человек
(пять с лишним миллионов доз). 

Еще четыре миллиона человек власти обещают привить вакцинами в рамках
гуманитарной помощи по линии Covax (хотя западные СМИ пишут, что нам выделят
вдвое меньше доз). То есть, в оптимистическом сценарии, за счет бюджета получат
вакцину более шести миллионов украинцев, в пессимистическом - четырех.

Интересно также сравнить цену на китайскую вакцину и другие предложения. 504
гривны - это примерно 18 долларов за дозу.

Российская вакцина "Спутник V", как заявлялось ранее, будет продаваться за
рубеж не дороже 10 долларов за ампулу. Но на днях Минздрав РФ утвердил
максимальную экспортную цену на уровне 25 долларов сразу за две дозы. То есть
максимум по 12,5 долларов за инъекцию, но может быть и ниже. 

Учитывая, что Россия предлагала производить "Спутник" на украинских мощностях,
то не исключено, что цена была бы еще ниже. При том, что вакцина уже прошла все
этапы испытаний еще в начале декабря. 

Цены на западные вакцины - вопрос сложный. Они зависят от рыночной конъюнктуры.
Недавно в Бельгии разразился скандал: местная правительственная чиновница
опубликовала цены, по которым ЕС контрактует вакцины. Но после переполоха в
правительстве, она удалила свой твит.

\ifcmt
  pic https://strana.ua/img/forall/u/0/92/%D0%BF%D0%BE%D0%BB%D0%B8%D1%82%D0%B8%D0%BA.jpg
  width 0.4
\fi

Согласно этому сообщению, самой доступной для европейцев стала британская
Астразенека, а дорогой - американская Moderna.

\begin{itemize}
\item AstraZeneca: 2,18 доллара 
\item Johnson \& Johnson: 8,50 долларов
\item Санофи: 9,27 доллара 
\item Pfizer и BioNTech: 14,71 доллара 
\item CureVac: 12,26 доллара 
\item Moderna: 18 долларов
\end{itemize}

Как видим, Украина покупает китайскую вакцину по цене, которую ЕС платит за
самую дорогую американскую. Правда, Европа заказывает огромные партии,
соответственно и цена для нее ниже. Украина же по меркам ЕС покупает небольшое
количество препарата. Да еще и в срочном порядке. Поэтому китайцы могут играть
ценой и требовать за свой продукт больше денег. 

Но все равно цена очень высокая.

\subsubsection{4. Какой срок поставки?}

По условиям договора, первую партию вакцины в количестве 700 тысяч доз поставят
в Украину в течение 30 дней после официальной регистрации в Китае -
или одним из компетентных органов США, Великобритании, Швейцарии, Японии,
Австралии, Канады, Израиля, Индии, Мексики, Бразилии. Или же по
централизованной процедуре Евросоюза.

То есть первые дозы могут поступить в Украину в феврале, после намеченной
регистрации в Китае (она ожидается в течение января). Остальное должны прислать
до конца первого квартала - то есть к апрелю. 

\subsubsection{5. А что скажут американцы?}

Главный политический вопрос относительно китайской закупки - это позиция США. 

Осенью американцы открытым текстом запретили Украине покупать российскую
вакцину.\Furl{https://strana.ua/news/295565-posolstvo-ssha-zapretilo-brat-rossijskuju-vaktsinu-ot-koronavirusa-reaktsija-seti.html}

\ifcmt
  pic https://strana.ua/img/forall/u/0/92/%D0%BF%D0%BE%D1%81%D0%BE%D0%BB%D1%8C%D1%81%D1%82%D0%B2%D0%BE(11).png
  width 0.4
\fi

Беспокойство Штатов было явно геополитического, а не медицинского характера.

Позволить Украине покупать вакцину "страны-агрессора" означало подвергнуть
сомнению официальную доктрину о разрыве Киева с Москвой. К тому же американцы,
судя по всему, рассчитывали сохранить за собой украинский рынок медицинских
закупок. 

Однако что-то пошло не так.

Запад не спешил помогать Украине с вакцинами и поставил ее в конец очереди за
жизненно необходимым препаратом. Что вызвало возмущение даже у Зеленского,
который в интервью The New York Times поинтересовался, как он теперь должен
объяснять умирающим людям, почему не должен покупать вакцину "Спутник V".

Впрочем, закупать российский препарат все же не решились и пошли другим путем -
обратились к китайцам. Еще одному врагу США. Но для Украины - явно меньшее из
зол, поскольку с Китаем Киев пока еще в более-менее нормальных отношениях.

Тем не менее и здесь могут проблемы в отношениях с Вашингтоном.

За примерами далеко ходить не надо. На днях Вашингтон предписал Киеву заменить
оборудование компании Huawei, которое установлено в министерствах, на
аналогичные товары американской компании Sisco. Да еще и за свой счет (США
согласились разве что сделать скидку). 

Всем известен и пример с заводом "Мотор Сич", который Америка запретила
продавать китайским инвесторам и заставила отменить сделку. Будет ли с
китайской вакциной та же история - покажет время. 

Не исключено, что в данном случае США закроют на "китайское происхождение"
вакцины глаза. 

Потому что это позволит украинским властям как-то выкрутиться с объяснением
обществу, почему не закупается российская вакцина, при том, что западную мы
получить не можем.

В данном случае появляется возможность сказать, что Киев нашел заменитель -
китайскую вакцину. Пусть она пока еще не прошла всех испытаний и даже не
зарегистрирована на родине.

