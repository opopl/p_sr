% vim: keymap=russian-jcukenwin
%%beginhead 
 
%%file 09_08_2021.fb.podobna_evgenia.1.goruna_maguchih_vojna
%%parent 09_08_2021
 
%%url https://www.facebook.com/evgeniya.podobna/posts/4268783336581745
 
%%author Подобна, Евгения
%%author_id podobna_evgenia
%%author_url 
 
%%tags __aug_2021.maguchih.foto.olimpiada.lasickene,donbass,maguchih_jaroslava,mobilizacia,vojna
%%title Знаєте, якою була наша найбільша помилка у цій війні?
 
%%endhead 
 
\subsection{Знаєте, якою була наша найбільша помилка у цій війні?}
\label{sec:09_08_2021.fb.podobna_evgenia.1.goruna_maguchih_vojna}
 
\Purl{https://www.facebook.com/evgeniya.podobna/posts/4268783336581745}
\ifcmt
 author_begin
   author_id podobna_evgenia
 author_end
\fi

Знаєте, якою була наша найбільша помилка у цій війні? Те, що не було  масової
мобілізації. Щоб війна прийшла в життя і душу кожного, а не була обридлою
картинкою в вечірніх новинах. Щоб це стало дійсно спільною бідою і спільним
горем. Повірте, ніщо так не вивітрює «ми внє палітікі», «мишебратья», «ета всьо
бізнес», як вибухова хвиля снаряду, що бахнув поряд. 

Наше велике щастя, що ми маємо таку круту армію. Яка дозволяє жити звичайним
життям і відчувати себе в повній і абсолютній безпеці вже за пару десятків
кілометрів від лінії фронту, не кажучи про Львів, де під спідницею в теплі і
добрі сидів всі ці роки собі Стас Горуна. І раптом озвався. Наша, здається,
велика біда, що ми маємо настільки круту армію, що вже за пару десятків
кілометрів від фронту ми відчуваємо себе настільки в безпеці, що забуваємо її
ціну. І що деякі наші спортсмени, забувають, що виходять з жовто-блакитним
прапором, а не з триколором чергової бананової народной республіки, лише тому,
що були ті, чиї труни накривали тим самим жовто-блакитним прапором. 


\ifcmt
   pic https://scontent-cdt1-1.xx.fbcdn.net/v/t39.30808-6/233652358_4268779603248785_4810472190975015189_n.jpg?_nc_cat=105&ccb=1-5&_nc_sid=8bfeb9&_nc_ohc=xBwsx6mVP8sAX9neZ7L&_nc_ht=scontent-cdt1-1.xx&oh=753385ed6a771002b47a38602d3893ba&oe=611D7BEF
	 width 0.4
\fi

Зараз почнеться, звісно, про імідж країни, який буцімто ледь не з колін підняли
героїчні Ярослава Магучіх і Стас Горуна. Дзуськи. Бо в очах світової спільноти
Ярослава ще раз показала, що «не всьо так адназначна». Що запам ятає обиватель
з умовних Нідерландів, Японії, Іспанії? Яка крута Україна, що здобула бронзу?
Ага.. Що щось там не так, якщо дві віськовочлужбовиці воюючих країн
обіймаються. По ходу, права РФ: «гражанданская» там у них, а не окупація. Бо
хто ж обійматиметься при здоровому глузді на камери з чинною
військовослужбовицею окупаційних військ. На її сторінці - навала порожніх
бото-акаунтів, які тішаться цим обіймам. Російська «преса» теж смакує всі ці
мімімі з усіх боків. 

\ifcmt
  tab_begin cols=2

     pic https://scontent-cdg2-1.xx.fbcdn.net/v/t39.30808-6/234327388_4268779699915442_2048180149179012408_n.jpg?_nc_cat=107&ccb=1-5&_nc_sid=8bfeb9&_nc_ohc=2-l4U3DZXVAAX_9W6Rv&_nc_ht=scontent-cdg2-1.xx&oh=2d1d47fc6170071f7432e53aa3826c1d&oe=611E7EF0

     pic https://scontent-cdt1-1.xx.fbcdn.net/v/t39.30808-6/227794425_4268779636582115_8262656780486623923_n.jpg?_nc_cat=109&ccb=1-5&_nc_sid=8bfeb9&_nc_ohc=VxS-8J5Ym7wAX9Bt0cR&_nc_ht=scontent-cdt1-1.xx&oh=fc34622d7ebf7fd49f34dbdba0fbd574&oe=611DE74A

  tab_end
\fi

А тепер до Стаса Горуни, який там емоційного поста наваяв, що спорт поза
політикою і все таке. І «мать» попутно згадав. Так от. 

З 9 по 11 жовтня 2015 року Стасік давав семінар в Новосибірську. Бо ж «спорт
поза політикою». 

10 жовтня 2015 року на фронті загинули двоє. Їх звали Віталій і Артем. Артем
був вашим однолітком, Стасе.

10-11 грудня 2016 Стас давав майстер-класи в Москві. 

10 грудня 2016 загинуло троє. Під Красногорівкою, Стасе. Вам, звісно, чхати,
але це був страшний бій, Стасе. У Володі Шоломинського була маленька донька. І
Віктора Клименка було четверо дітей. Андрій був усього на кілька років старше
за вас, Стасе. 

А 11 грудня 2016, коли ви, Стасе, давали майстер-клас в Москві, загинув чоловік
на ім я Віталій, який приїхав з цієї самої Москви захищати вас. Доки ви рубали
бабло в країні, яка вбивала ваших захисників. 

Я не хочу вас, Стасе, ображати - вам відповідати за це перед вашою совістю. Я
рада вашій перемозі. Я рада, що ваш син дочекається вас з бронзовою медаллю
живого і здорового. А не медаль «За міжність ІІІ ступеня» в коробочці замість
найдорожчого. Але особисто для мене одна така «За мужність» значно важливіша за
усі ваші разом узяті. Але вам, звісно, цього не зрозуміти. Для вас це лише
«палітітка». «І трошки бізнес», як ви написали. 

На фото пост, через який мене розірвало. І дві події, які відбулись в один
грудневий день 2016 р. 

І вибачте, друзі, за емоції. Чим далі, тим важче мені тримати себе в руках.
