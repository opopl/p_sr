% vim: keymap=russian-jcukenwin
%%beginhead 
 
%%file 13_06_2018.fb.lesev_igor.1.singapur_vstrecha
%%parent 13_06_2018
 
%%url https://www.facebook.com/permalink.php?story_fbid=1950331328331294&id=100000633379839
 
%%author_id lesev_igor
%%date 
 
%%tags geopolitika,singapur,usa,vstrecha
%%title По сингапурской встрече Кима и Дональда
 
%%endhead 
 
\subsection{По сингапурской встрече Кима и Дональда}
\label{sec:13_06_2018.fb.lesev_igor.1.singapur_vstrecha}
 
\Purl{https://www.facebook.com/permalink.php?story_fbid=1950331328331294&id=100000633379839}
\ifcmt
 author_begin
   author_id lesev_igor
 author_end
\fi

По сингапурской встрече Кима и Дональда.

Итогов как таковых нет, потому что все подписанное – это абстрактные
словоблудливые намерения. Дружить, любить, заботиться о ближних, не обижать
зверей и стремиться к счастью. Ни одна из сторон не взяла на себя ВООБЩЕ
никаких обязательств. Да и не могла в принципе взять. Потому что, эта встреча о
символах, а не о сути.

\ifcmt
  ig https://scontent-frt3-1.xx.fbcdn.net/v/t1.6435-9/35162279_1950331121664648_2530976162553790464_n.jpg?_nc_cat=102&ccb=1-5&_nc_sid=730e14&_nc_ohc=QMMTQslpKbcAX_pUVJH&_nc_ht=scontent-frt3-1.xx&oh=161ce328f2a4d776e6e073f88b689b10&oe=61BAD92E
  @width 0.4
  %@wrap \parpic[r]
  @wrap \InsertBoxR{0}
\fi

А какие мы получили символы? Первый и самый главный – сильные разговаривают со
стержневыми ребятами, а не с подстилками. Даже если стержневая пацанва живет в
другом правовом измерении.

Второй символ – мир держится не на добром слове, а на возможности вознаипнуть
силой. А это сигнал уже всем остальным – арабам, индусам, латиносам –
вооружайтесь. И тогда, вернее, только тогда с вами будут считаться.

Третий символ – Америка последовательна в своей непоследовательности.
Нарушается все что только можно, спонтанно, алогично и эмоционально. В Штатах
доминирует инфоповод над логичной линией поведения. Ведь КНДР – это такой себе
мировой зажопинск, что о нем не вспоминали ни при Клинтоне, ни при
Буше-младшем, ни при Обаме. Маленькие самоизолированные дикари, не несущие
РЕАЛЬНОЙ угрозы особо никому. Трамп залез в этот спящий муравейник, довел дело
сначала до одного абсурда – направил (якобы) туда авианосные соединения, чтобы
завершить весь этот цирк личной встречей с ранее абсолютно нерукопожатым Ыном.

Что теперь? Денуклеаризации (ох уж любят выепнуться модными словцами)
Корейского полуострова, естественно, не будет. Хотя бы потому, что для этого
нужно вывести одну из самых крупных группировок Штатов за рубежом – около 30
тыс. военных – с Южной Кореи. И там американцы пасутся не столько из-за КНДР,
сколько «ради» своих более стратегических «друзей» - России и Китая. Если
американцы гипотетически уходят из Кореи – это ликвидация всей сложившейся
геополитической конструкции в тихоокеанском регионе. Сами южнокорейцы этого
боятся, наверное, даже больше. Как и японцы, потому что переписывается тогда в
регионе, да и в мире, абсолютно все правила игры и поведения.

Объединение Кореи тоже не может произойти без полного демонтажа системы власти
в одной из частей – северной или южной. «Объединение» может быть только через
поглощение с полным принятием правил одной из частей, как это произошло с ГДР.
А Ын не Хонеккер.

В итоге, мы получаем о дивный новый мир, в котором Штаты все чаще играют роль
шутов и дестабилизатора старой доброй системы. Трамп все больше напоминает
нашего Горби, подтачивающего свою гегемонию там, где не додумались бы до такого
даже спецслужбы иностранных государств. Подосрал с договором о климате.
Поставил на счетчик партнеров по НАТО. Кинул всех с Ираном. Посрался из-за
пошлин со всеми соседями/партнерами/бесплатными любителями Америки.
Удивительный персонаж и настоящая находка для русских, потому что именно с
Трампом Москва получила неожиданную передышку перед большим цивилизационным
замесом.

Ну а толстяк Ким теперь звезда мирового масштаба. На ровном месте.

\ii{13_06_2018.fb.lesev_igor.1.singapur_vstrecha.cmt}
