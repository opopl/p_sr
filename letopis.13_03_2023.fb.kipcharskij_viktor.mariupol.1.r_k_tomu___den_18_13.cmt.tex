% vim: keymap=russian-jcukenwin
%%beginhead 
 
%%file 13_03_2023.fb.kipcharskij_viktor.mariupol.1.r_k_tomu___den_18_13.cmt
%%parent 13_03_2023.fb.kipcharskij_viktor.mariupol.1.r_k_tomu___den_18_13
 
%%url 
 
%%author_id 
%%date 
 
%%tags 
%%title 
 
%%endhead 

\qqSecCmt

\iusr{Елена Девина}

Дякуємо за докладні оповідання. Боляче їх читати, але... У кожного маріупольцям
своя історія війни...

\iusr{Zhanna Savelyeva}

Лівий вже був вщент. Сусіди мами наважилися йти до доньки та онуки у ваш район.
Сусіда дорогою поранило, у спині був осколок, але життеві органи цілі. Осколок
на кухні вже у доньки (дійшли!) вилучили самі та самі і зашили... як могли,
трохи хтось був з меддосвідом.

\begin{itemize} % {
\iusr{Віктор Кіпчарський}
\textbf{Zhanna Savelyeva} 

Я багато разів шукав пояснення тому, як ми опинилися у \enquote{оці тайфуну} - чому
смерть і руйнування були навколо нашого двору, але до нас не зайшли?

Логічного пояснення не знайшов - тільки містичні...

\iusr{Zhanna Savelyeva}
\textbf{Віктор Кіпчарський} це диво. Неможливо знайти відповідь. Дяка Господові.

\iusr{Віктор Кіпчарський}
\textbf{Zhanna Savelyeva} Славимо Його!

\iusr{Віктор Кіпчарський}
\textbf{Zhanna Savelyeva} 

\enquote{Випадковість - то непізнана закономірність}.

Я дуже серйозно займався математичним моделюванням і впевнений, що пояснити
(описати, вирахувати) можна навіть рух молекул у океані, аби кожної часточки
борошна у міксері.

Навколо нашого двору, який стояв мало не впритул до госпіталю, на відстані
плюс/мінус кілометру було чотири батареї. І ми могли спати у ліжках, готувати
їжу...

Як? Диво.

\iusr{Zhanna Savelyeva}
\textbf{Віктор Кіпчарський} 

даже дуже висока вирогідність це все ж таки тільки вирогідність. (Теорія
вирогідностей добряче пройшла повз моїх вух😆). Або є якесь пояснення, час щось
відкриє, мабуть.

\iusr{Віктор Кіпчарський}
\textbf{Zhanna Savelyeva} Це виглядає приблизно так: під час дощу, на відкритому полі є метрове коло, на яке не впало ані краплі.

\iusr{Віктор Кіпчарський}
\textbf{Zhanna Savelyeva} Тобто є вірогідність вважати це дивом?

\iusr{Віктор Кіпчарський}
\textbf{Zhanna Savelyeva} 

\enquote{Тому що немає нічого прихованого, яке б не виявилося, і немає таємного, яке б
не стало явним. Якщо хто має вуха, щоби слухати, нехай слухає!}

Євангеліє від Луки, 8: 17-18.

\iusr{Roy Evans}
\textbf{Віктор Кіпчарський} 

скоріше за все, вдале розташування дому. У нас будинок був як раз у верхній
точці траєкторії, тому у двір часто прилітало, так само як і через дорогу у
будинок, бо стояли на одній лінії.

\iusr{Віктор Кіпчарський}
\textbf{Roy Evans} 

На відстані до двох кілометрів від нас були: батарея ППО, дві артбатареї,
вогнева САУ, поліція в Дружбі та на Жилкопах, зовсім поруч - госпіталь.

Навіть те, що перелітало батарею на Шлачці до нас не долітало...

У 1980-му році над Ждановом промчав смерч. Ми з батьком саме поїхали на
риболовлю, а перед тим заїхали на заправку на Донецьку трасу. Побачивши, що
коїться, рвонули додому: проспект Металургів перед Заворуєва завалило деревами
і бетонним стовпами; Заворуєва перед Карпінського - так само; Менжинського в
балці - теж. У будинках на проспекті позривало дахи, на заводі Б повалило кран
і вбило людину.

Перелякані ми по Мазая доїхали до дому і ... на клумбах на квітках - роса!!! Її
навіть не струсило!!!

Я все розумію про \enquote{око тайфуну}, але чому воно вже вкотре над нашим двором?

\iusr{Roy Evans}
\textbf{Віктор Кіпчарський} 

Дружба, відділок поліції, шпиталь - це все або висока будівля, або дуже
відокремлений комплекс. Стосовно природних явищ - то теж ніякого везіння, десь
тепліші потоки, десь вітерець не так загорнувся, і все, ви в \enquote{оці}. Просто
сукупність факторів. Як і те, що ми взагалі вижили. Ми спускалися готувати їжу
біля під'їзду, і трохи теревенили стояли біля самого входу до нього, так от
стояли стояли, я кажу сусідам, пішли до дому, бо неможна довго стояти на
вулиці. Я пішов, вони залишились. Тільки піднявся на свій поверх, як у сусідній
під'їзд прилетіло, зачепивши палець на нозі одному сусіду, та у декількох
сантиметрах позаду від ноги другого сусіда дірка в стіні. В 7-10 метрах від
гаражу, де я ховав автівку, в дах приватного будинку прилетіло. Автівка та
гараж абсолютно цілі, тільки багато пилу від вібрації було. Їхав по Нахімова,
на перетині Нахімова-Бахчіванджи за хвилину до того прилетіло, водія автівки,
який проїджав там поранено. Виходили готувати їжу вже коли нас загнали у підвал
по Будівельників 98, у дах окремої доповерхової будівлі щось прилетило, ми всі
залишились живі. І такого дуже багато. Просто сукупність факторів, не більше.
Щоправда, нажаль, не у всіх так склалося(

\end{itemize} % }

\iusr{Александр Колобов}

Не памятью якого дня, але вже ні коли не забуду як це було... Пішли по воду на
криницю к мечеті, біля "дачі Савчука", от Крандштатской по сходам, вниз к морю.
Людей багато. Черга. Тільки набрали свої пластикові бутилі.. "Фрррр... і зверху
дві міни... За забором дачі вибух (ми снизу і нижче), другий вибух біля сходів к
морю. Убило чоловіка на сходах, і поранено жінку і хлопця. Хватаємо свою воду,
і вниз к морю, к "Барбарису". Біля моря зупинились... і знову, зверху "Фррр..."
метрів 40 від нас, в море падають дві міни... Сплеск, пар... і тішина...
Попадали в море і не взірвались... Тікаем швиденько, біжим в сторону
Азовстальской водной. Кругом побитті дерева, запах вибухівкі... Піднімаемося
металевими сходами вгору, к майданчику. І знову... Фрррр... Чотири міни
пролетіли на головами і впали в воду біля Азавстальской водной. І знову не
взірвались... Прилітає пакет Граду, десь біля профілакторія "Азовмаша" Залізли
під металеві сходи, лежим на землі, слухаем... Трошки стало тихше, бігом до
хати, на пр. Нахімова. Прибіг домой і в підвал...

Сходив по водичку...

\begin{itemize} % {
\iusr{Віктор Кіпчарський}
\textbf{Александр Колобов} Про що Ви в ці секунди думали?
Можете відповісти в Месенджері.
\end{itemize} % }

\iusr{Елена Девина}

Часто вспоминала и вспоминаю известное народное выражение: \enquote{Кому суждено утонуть, тот не повесится}

\iusr{Александр Колобов}

За себе не думав. Думав за рідних.

\begin{itemize} % {
\iusr{Віктор Кіпчарський}
\textbf{Александр Колобов} 

Я, мабуть, некорректно сформулював питання.

Мені здається, що я хвилювався за дітей та онуків, які ще не насолодилися
життям.

Після обстрілів, про минуле я не згадував, питанням \enquote{чому одразу не поїхали} не
переймався, з майна шкодував лише те, що залишилося у гаражі, а могло
придатися, наприклад, двуручна пилка, паяльна лампа тощо.

Навіть не думав, що з собою взяти (окрім, звичайно, документів - вони у
\enquote{тривожному рюкзаку} лежали ще з 2014-го року).

Наперед планував хіба що добування їжі: полювання на голубів, риболовлю.

Отаке написалося, коди вже поїхали:

Ми вижили - тому, що жили...

\iusr{Александр Колобов}
\textbf{Віктор Кіпчарський} 

\enquote{тривожний рюкзак} зібрали зразу. Виїхати хотіли при першому \enquote{зеленому
корідорі}. Тільки почали підходити к Ільічівському спорткомплексу, сповістили, що
коридору не буде. Так було декілька разів. Ловити рибу, голубів та горобців
навчився в дитинстві. Так що це не турбувало.. А виїхали несподівано і
випадково, 17 березня. Тільки одні документи, зразу с підвала, без речей. В 23. 00
були в Запоріжжі.

\iusr{Віктор Кіпчарський}
\textbf{Александр Колобов} їхали своєю машиною чи перевізником?

\iusr{Александр Колобов}
\textbf{Віктор Кіпчарський} Під'їхали родичі...бігом в машину і поїхали...

\iusr{Віктор Кіпчарський}
\textbf{Александр Колобов} А як їхали? Через Токмак?
Нас привели до Епіцентру, а ночували ми у дітсадочку Надія.

\iusr{Александр Колобов}

Так же іхали і ми...

\end{itemize} % }

\iusr{Александр Колобов}

До 24 числа у мене був дім, щасливе життя, спокійна старість... Все це відібрали...
