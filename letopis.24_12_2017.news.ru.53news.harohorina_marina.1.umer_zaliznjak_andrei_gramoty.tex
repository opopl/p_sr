% vim: keymap=russian-jcukenwin
%%beginhead 
 
%%file 24_12_2017.news.ru.53news.harohorina_marina.1.umer_zaliznjak_andrei_gramoty
%%parent 24_12_2017
 
%%url https://53news.ru/novosti/35819-umer-issledovatel-novgorodskikh-berestyanykh-gramot-andrej-zaliznyak.html
 
%%author Харохорина, Маргарита
%%author_id harohorina_marina
%%author_url 
 
%%tags beresta_gramoty,novgorod,smert
%%title Ушел из жизни исследователь новгородских берестяных грамот Андрей Зализняк
 
%%endhead 
 
\subsection{Ушел из жизни исследователь новгородских берестяных грамот Андрей Зализняк}
\label{sec:24_12_2017.news.ru.53news.harohorina_marina.1.umer_zaliznjak_andrei_gramoty}
\Purl{https://53news.ru/novosti/35819-umer-issledovatel-novgorodskikh-berestyanykh-gramot-andrej-zaliznyak.html}
\ifcmt
	author_begin
   author_id harohorina_marina
	author_end
\fi

\ifcmt
pic https://53news.ru/images/wsscontent/articles/2017/12/ushel-iz-zhizni-issledovatel-novgorodskikh-berestyanykh-gramot-andrej-zaliznyak.jpg
\fi

\textbf{Сегодня, 24 декабря, на 83-м году жизни скончался известный лингвист,
академик РАН Андрей Зализняк. Об этом сообщил сотрудник Института русского
языка РАН Дмитрий Сичинава.}

\index[deaths.rus]{Зализняк, Андрей Анатольевич!Лингвист, исследователь берестяных грамот, 24.12.2017}
\index[names.rus]{Зализняк, Андрей Анатольевич!Лингвист, исследователь берестяных грамот!Смерть, 24.12.2017}

«Умер А. А. Зализняк. Я иногда задумывался над тем, как это будет перенести, и
ничего не придумал. Не держат ноги, стою прислонившись к стене», – написал он
на своей странице в «Фейсбуке».\Furl{https://www.facebook.com/mitrius/posts/10155959649688064?pnref=story} О причине смерти не говорится.

Андрей Зализняк родился 29 апреля 1935 года в Москве. Более 50 лет преподавал
на филологическом факультете МГУ. В 1990-е годы читал лекции в
Экс-ан-Прованском, Парижском и Женевском университетах. Также был приглашенным
профессором в университетах Италии, Германии, Австрии, Швеции, Англии и
Испании.

С 1982 года Андрей Анатольевич Зализняк вел систематическую работу по изучению
языка берестяных грамот, как уже известных, так и вновь найденных на раскопках
в Новгородской области. Он является соавтором издания «Новгородские грамоты на
бересте». Обобщающим трудом академика в этой области стала книга
«Древненовгородский диалект», где представлен грамматический очерк
древненовгородского диалекта и даны с лингвистическим комментарием тексты
практически всех берестяных грамот. В 2000 году при раскопках в Великом
Новгороде была обнаружена древнейшая книга Руси — Новгородский кодекс. Андрей
Анатольевич Зализняк многие годы посвятил реконструкции «скрытых» текстов этой
находки.являлся исследователем новгородских берестяных грамот и «Слова о полку
Игореве».

За многолетний исследовательский труд Андрей Зализняк был удостоен нескольких
премий, включая Государственную премию России в 2007 году.  В мае 2011 года ему
присвоили звание «Почетный гражданин Великого Новгорода».

В конце октября ученый рассказал о находках археологического
сезона-2017.\Furl{https://53news.ru/novosti/33966-video-akademik-andrej-zaliznyak-rasskazyvaet-o-berestyanykh-gramotakh-najdennykh-v-2017-godu.html}
Год назад «53 новости» публиковали материал по его лекции в МГУ, где он поведал
о переписке новгородских ювелиров XII
века.\Furl{https://53news.ru/novosti/26046-na-lektsii-v-mgu-andrej-zaliznyak-rasskazal-o-perepiske-novgorodskikh-yuvelirov-xii-veka.html}

