% vim: keymap=russian-jcukenwin
%%beginhead 
 
%%file 13_01_2022.fb.fb_group.story_kiev_ua.2.vzlety_i_padenia.pic.4.cmt
%%parent 13_01_2022.fb.fb_group.story_kiev_ua.2.vzlety_i_padenia
 
%%url 
 
%%author_id 
%%date 
 
%%tags 
%%title 
 
%%endhead 

\iusr{Мария Бутковская}
3акая красивая мама ! @igg{fbicon.thumb.up.yellow}  @igg{fbicon.face.blowing.kiss} 

\iusr{Елена Слободяник}

Чёрно-белые фото лично мне нравятся больше. Они лучше передают не только
внешность, но и душевность, если можно так сказать. Прекрасная память в них.

\iusr{Георгий Майоренко}
Матроски на детях это - дореволюционная традиция! Знатное фото!

\iusr{Анна Дубницкая}
\textbf{Георгий Майоренко} 

У мене в сина була матроска в 1990 році, правда сірого кольору. Є й зараз,
зберегла на пам'ять.

\iusr{Людмила Билык}

Очень красивая мама...  @igg{fbicon.thumb.up.yellow}  @igg{fbicon.rose}
@igg{fbicon.hands.pray} 

\iusr{Владимир Новицкий}
\textbf{Людмила Билык} 

Да, мама была красивая, всегда носила шляпки, следила за собой!!

\iusr{Надежда Лабик}
И у меня была такая матроска.

\iusr{Лена Шклярська}
Какие милые и мама и дети... а как мне найти рассказ «Талисман» ?

\iusr{Irena Visochan}
Какая мамочка красавица! Светлая ей память! @igg{fbicon.hands.pray} @igg{fbicon.heart.suit}
