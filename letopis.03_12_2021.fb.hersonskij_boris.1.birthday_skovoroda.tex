% vim: keymap=russian-jcukenwin
%%beginhead 
 
%%file 03_12_2021.fb.hersonskij_boris.1.birthday_skovoroda
%%parent 03_12_2021
 
%%url https://www.facebook.com/borkhers/posts/4960021214032665
 
%%author_id hersonskij_boris
%%date 
 
%%tags birthday,filosofia,lomonosov_mihail,mudrost,skovoroda_grigorii,ukraina
%%title З Днем народження, пане Григоріє!
 
%%endhead 
 
\subsection{З Днем народження, пане Григоріє!}
\label{sec:03_12_2021.fb.hersonskij_boris.1.birthday_skovoroda}
 
\Purl{https://www.facebook.com/borkhers/posts/4960021214032665}
\ifcmt
 author_begin
   author_id hersonskij_boris
 author_end
\fi

\headCenter{З Днем народження, пане Григоріє!}

\begin{multicols}{2}
\obeycr\noindent
1.
\smallskip
Благий Господь. Він - фортеця у дні війни.
Знає  вірних своїх. У безпеці вони!
Ой, що нам безпека, коли літо та спека,
на даху гніздо, а в гнізді - лелека.
\smallskip
А війна десь гримить, як гроза далека.
Й пливе по небу велика хмара,
невідворотна, як Божа кара.
\smallskip
Біжи з міста, філософе, по степу блукай,
сиди на пагорбі, на сопілці грай.
На сопілці грай, там, де гай та річка,
У Софії-мудрісті є дурість-сестричка.
У дурості  в косах рожева стрічка.
\smallskip
Вечор була дівка, з ранку вона молодичка.
З дурістю балакати - то погана звичка.
Аврелій з Сенекою звернулись до Пана
бо віра Христова розумна й бездогана.
\smallskip
Щось таке безглузде у душі зростало:
Літо настало, а козля цапом стало,
цап бородатий, йому життя мало.
\smallskip
Йому життя мало, й те на сміх курам.
Живи з афінейським Христом - Епікуром.
Російська з латиною, та ще й мова грецька,
та й думка єретична, напів-мистецька.
\smallskip
Був майже всюди, але мріяв про втечу.
Всюди - безглуздість, всюди - ворожнеча.
Поранена Європа, ось і кровотеча.
\smallskip
Не лай мене, мати, не кляни мене, сину,
за те, що московщину приніс в Україну,
бо з'їв нашу мову Михайло Ломоносов.
Мало українців - багато малоросів.
\restorecr
\end{multicols}

\ii{03_12_2021.fb.hersonskij_boris.1.birthday_skovoroda.pic.1}

\begin{multicols}{2}
\obeycr\noindent
2.
\smallskip
Ось - сижу на пагорбі, на сопілці граю,
наначе в раї. й не потрібно іншого раю,
не потрібно іншого раю, не хочу іншого краю,
\smallskip
люблю стадо бидляти, ненавиджу вовчу зграю,
давногрецьку знаю, знаю також латину,
та ось - на сопілці граю, та ось - люблю Україну.
\smallskip
То ж скажіть цариці-матінці, що зустріли мене при дорозі,
що сидів філософ наодинці, на розі
земного й небесного, у чистому полі,
на межі долі й недолі, волі й неволі.
\smallskip
То ж скажіть цариці-матінці, щоб мене не чекала
в Петербурзі, й тут мене не шукала,
Я там, де дуб зеленіє, де пасеться вівця,
й сопілка моя дорожча мені царева вінця.
\smallskip
4.
\smallskip
В юные годы я посещал Сады
Божественных Песен Григория Сковороды,
кроны шумели архаикой, немыслимым языком,
который откуда-то-ниоткуда был мне едва знаком,
\smallskip
Так в шуме толпы и крыльев, растерявшись сперва,
ловишь мысли и различаешь слова.
В мире философов растут такие сады,
где не срывают цветы и не вкушают плоды.
\smallskip
Там Страстная Суббота срастается с Рождеством.
Там Давид и Мария гордятся дальним родством.
Там байраки и символы, пагорбы и горбы.
Там разум сдается Откровению без борьбы.
Там жива в забвении славянская старина,
что Григорий Сковорода, что Франциск Скорина.
\smallskip
За философом ходит мир, что твой бородатый цап.
Увидит и - цап, и не выпустит душу из лап.
Мир не поймал его душу, но лицо он поймал,
на купюре размножил - пятьсотгривневый номинал.
\smallskip
Философ к философу - туго набит кошелек.
расплатись - и увидишь, как он от тебя далек.
Увидишь - не горячись, затвердевай и остынь,
как в украинской форме расплавленная латынь.
\smallskip
2019
\restorecr
\end{multicols}

\ii{03_12_2021.fb.hersonskij_boris.1.birthday_skovoroda.cmt}
