% vim: keymap=russian-jcukenwin
%%beginhead 
 
%%file slova.myshlenie
%%parent slova
 
%%url 
 
%%author 
%%author_id 
%%author_url 
 
%%tags 
%%title 
 
%%endhead 
\chapter{Мышление}
\label{sec:slova.myshlenie}

Мышление, смысл, мыслить, осмыслить, мысль

Давайте скажем прямо: управленческий успех в мировосприятии
среднестатистического украинца совершенно не отличим от зрады.  Мы давно
поставили между ними постмодернистский знак равенства.  Когда Украине пророчат
будущее \enquote{серой зоны}, надо понимать, что прежде всего это -
\enquote{серая зона смысла}.  Успех Украины - экономический, геополитический,
социальный - может состояться. Лично для меня это предмет не веры (в данном
случае, вера - антипод \emph{мышления}), но надежды.  Я надеюсь, что в Украине рано
или поздно сформируется волевой центр - субьект, а не обьект истории.  Однако,
без смены \enquote{регистра \emph{мышления}} этого никогда (!) не произойдет,
\citTitle{Что имеют в виду, когда пророчат Украине будущее серой зоны?}, Владислав Михеев, strana.ua, 10.06.2021

Прежде всего, что это будет серая зона \emph{смысла}.  Успех? А как он туда
попал?  Знаете, что лично меня реально забавляет в современном дискурсе?  Две
вещи.  Во-первых, ежесекундное ожидание Апокалипсиса.  Да, не заслужили вы его
еще пока.  Не заслужили! Хотя... ) Во-вторых, репортажи о невиданных успехах:
от строительства лучших за всю историю Украины дорог до пластики лучших в
Украине губ, нереально счастливых семей и непревзойденных карьерных
достижений.  Агонизирующее благополучие... ,
\citTitle{Что имеют в виду, когда пророчат Украине будущее серой зоны?}, Владислав Михеев, strana.ua, 10.06.2021

%%%cit
%%%cit_head
%%%cit_pic
%%%cit_text
Сегодня исполняется 80 лет с тех пор, как Германия атаковала Советский Союз.
Это произошло 22 июня 1941 года. Так начался самый кровопролитный этап Второй
Мировой войны - Великая Отечественная война.  Украинский политический мейнстрим
представляет эту войну как борьбу двух тоталитарных режимов, одинаково
враждебных идее украинской государственности, проводится \emph{мысль}, что это
была \enquote{не наша} война.  На правом же фланге этого мейнстрима и вовсе царит
уверенность, что лучше было бы с немцами, чем с Советами.  По крайней мере
деятельное сотрудничество украинских националистов с нацистами в глазах
нынешних властей не считается чем-то слишком уж зазорным. В честь таких
персонажей называют улицы, вводят памятные даты, а марш в годовщину создания
дивизии СС \enquote{Галичина} этой весной впервые прошел по Киеву
%%%cit_comment
%%%cit_title
\citTitle{22 июня - 80 лет нападения на СССР. Что немцы готовили для украинцев}, 
Максим Минин, strana.ua, 22.06.2021
%%%endcit

%%%cit
%%%cit_head
%%%cit_pic
%%%cit_text
Давайте начнем с того, что начнем \emph{размышлять}: «А что же такое, что уже
не одно столетие мы пытаемся в той же схеме получать одну и ту же Руину. Может
что-то не так в \emph{мышлении}, а?». Так поменяйте структуру \emph{мышления}.
Может с этого надо начать? «Не. Гей. Як наші діди і батьки вмирали, так і ми
вмирати будемо». Ну и толку из этого, что вы умираете? Никакого толка в смерти
нет. Толк есть в жизни, который сопровождается \emph{сложным мышлением}. В этом
\emph{смысл}. \emph{Переосмыслить} все нужно. Очищение нужно. Прежде всего
моральное. Принцип общего блага должен быть. Солидарное будущее должно быть.
Договорное. Понимаете? А когда к вам приходят и говорят, что ты сволочь должен
говорить на вот этом вот языке, если ты не будешь говорить, на тебя донос, а
потом штраф, это тупик. Это тупик
%%%cit_comment
%%%cit_title
\citTitle{Сергей Дацюк: Украина сегодня - не просто попрошайка, она на мусорнике истории}, 
Сергей Дацюк; Людмила Немыря, hvylya.net, 28.06.2021
%%%endcit

%%%cit
%%%cit_head
%%%cit_pic
%%%cit_text
Цивилизационное \emph{мышление} видит мир целиком. Оно видит опыт таких же в
прошлом феодально-сословных сообществ, которые волевым усилием первых лиц
страны, волевым усилием госаппарата вырвались из третьего мира. Которые за одно
поколение не только избавились от бедности, но и ворвались в первый мир -
создали такие условия жизни и такую бизнес-среду, что туда устремились люди и
деньги со всей планеты. Народ Сингапура в 50-х, 70-х не был ни образованнее, ни
осознанее, ни продвинутее украинцев. Он был таким же, как и любой другой народ.
и где теперь Сингапур, и где Украина?  И мы знаем имя Ли Куан Ю. Наверно, один
из самых знаменитых Премьеров мира. А чем запомнится Яценюк, Гройсман,
Гончарук, Шмигаль? Чем запомнятся наши президенты? И самое главное - никто из
них не то, чтобы не смог. Никто из них даже не хотел. \emph{Колониальное
мышление} толкает страну или к России, или к Западу.  \emph{Колониальное
мышление} не генерирует субъектности
%%%cit_comment
%%%cit_title
\citTitle{Украина разделена на два типа мышления - колониальный и цивилизационный}, 
Павел Себастьянович, strana.ua, 14.07.2021
%%%endcit

%%%cit
%%%cit_head
%%%cit_pic
%%%cit_text
Нынешняя власть продолжает крутить старую шарманку, что Россия отсталая страна,
плохо управляемая страна, которая вот-вот развалится. Украинская политика
официально призывает не принимать интересы России всерьез, изгонять российский
бизнес с украинских рынков, отгородиться от России стеной и базами НАТО. Это,
мягко скажем, неумная, недальновидная политика, которая угрожает не столько
России, сколько существованию украинского государства. Сведение \emph{смысла}
украинской государственности к противостоянию с Россией автоматически
уничтожает этот \emph{смысл}, если Европа и Россия договорятся. И тогда государство
Украина становится ненужным, и неважно, кем будет контролироваться эта
территория. Значит, нам нужен \emph{другой смысл} существования украинской
государственности
%%%cit_comment
%%%cit_title
\citTitle{О будущем украинского и русского народов}, 
Виктор Медведчук, strana.ua, 15.07.2021
%%%endcit

%%%cit
%%%cit_head
%%%cit_pic
%%%cit_text
И в этом \emph{смысле} поражает согласное молчание зеленого клоуна, который все еще
бездарно пытается играть роль Гаранта. Ведь его дед воевал! Уроженец Кривого
Рога Семен Иванович Зеленский, 1924 года рождения (данные о дате смерти в сети
отсутствуют), был участником Великой Отечественной войны, командовал минометным
взводом, затем — стрелковой ротой 174-го гвардейского стрелкового полка 57-й
гвардейской стрелковой дивизии. Окончил войну в звании гвардии лейтенанта.
Кавалер двух орденов Красной Звезды. А в Великую отечественную ордена просто
так не раздавали...
%%%cit_comment
%%%cit_title
\citTitle{Краткий словарь грантоедов под редакцией СНБО / Лента соцсетей / Страна}, 
Александр Карпец, strana.news, 26.10.2021
%%%endcit

%%%cit
%%%cit_head
%%%cit_pic
%%%cit_text
Але годі. Це не основне. Це — прелюдія до найважливішого. Тоді був морок,
безнадія, психічна яма. Дні, місяці, роки. Не рахував. Це все було невимірно,
вічно, безвихідно. Я намагався віднайти в моєму новому стані бодай якийсь
\emph{смисл}, бодай шматочок доцільності. Знав, читав про героїв, про терпіння, про
незвичайну віру. Та все це годилося тоді, коли людина приймала самоцінність
життя, біологічного функціонування. Я ж ніколи не вважав життя самоцінністю.
Воно було для мене стежкою пізнання, пошуків, боротьби. А тепер — в
непорушності— яка боротьба? Які пошуки? Навіщо бути тягарем для світу, для
людей — близьких чи далеких, — котрі повинні віддавати свій повноцінний час
безпомічному каліці? До того ж я відчув, що хвороба моя прогресує, вже
віднімалися руки, я почав погано розмовляти. Мені найняли няню, щоб прибирати в
кімнаті й біля мене. Це було тяжко й соромно. Інколи хотілося виповзти з ліжка,
добратися до вікна, вивалитися на асфальт... Кімната була на шостому поверсі,
череп розлетівся б на шматки. Але навіть такого я не міг вчинити... не вистачало
сил...
%%%cit_comment
%%%cit_title
\citTitle{Вогнесміх}, Олесь Бердник
%%%endcit

%%%cit
%%%cit_head
%%%cit_pic
%%%cit_text
Вниманием к мелочам, контексту могут похвастаться единицы. Когда сбили "Боинг"
наш в Иране, читаю "самым маленьким участником катастрофы был трехлетний
ребенок". Объясни, - говорю, - как может трехлетний ребенок может быть
участником катастрофы? Ты как это представляешь? Выходит человек и спрашивает:
Товарищи, кто сегодня участвует в катастрофе? Лес рук? Участие требует акта
воли, то есть человек сознательно принимает решение. И очевидно, что никто не
хочет участвовать в катастрофах как в марафонах. За редкими исключениями
суицидников, которые имеют соответствующие мотивы. Но такая цепочка вопросов не
возникает. Клиповое \emph{мышление}
%%%cit_comment
%%%cit_title
\citTitle{Система образования Украины напрочь убивает мышление / Лента соцсетей / Страна}, 
Юрий Романенко, strana.news, 31.10.2021
%%%endcit

%%%cit
%%%cit_head
%%%cit_pic
%%%cit_text
Месяц назад при одной только \emph{мысли} об этом Харлану стало бы не по себе.
Его родное 95-е Столетие осталось далеко в прошлом. Это был век патриархальных
традиций, в котором атомная энергия находилась под запретом, а всем
строительным материалам предпочитали дерево. Век славился своими напитками,
которые в обмен на семена клевера вывозились почти во все другие Столетия. Хотя
Эндрю Харлан не был дома с тех пор, как он в пятнадцать лет стал Учеником и
прошел специальную подготовку, его никогда не оставляла тоска по родным
Временам. Между 95-м и 2456-м Столетиями пролегло почти двести сорок тысяч лет,
а это ощутимый промежуток даже для закаленного Вечного. При обычных
обстоятельствах все было бы именно так.  Однако сейчас Харлану было не до
абстрактных размышлений. Рулоны перфолент оттягивали его карманы, планы тяжким
грузом лежали на сердце, \emph{мысли} были скованы страхом и неуверенностью. Он
машинально остановил капсулу в нужном Столетии
%%%cit_comment
%%%cit_title
\citTitle{КОНЕЦ ВЕЧНОСТИ}, Айзек Азимов
%%%endcit

%%%cit
%%%cit_head
%%%cit_pic
%%%cit_text
Миллионы лет прошли, пока родилась эта \emph{мысль}. Но еще десятки тысяч лет
пройдут, пока она утвердится окончательно. Ибо и до сих пор бродит по свету
несчастное число цивилизованных дикарей, убежденных, что «плохо, если у меня
украдут, и хорошо, если украду я сам».  Но заповедь «Не укради», при всем ее
величии, есть тоже только «дифференциальное уравнение», правда, для огромного
круга явлений. На этом огромном круге жизнь бесчисленных тысячелетий есть
только «бесконечно малый» отрезок. Для этого отрезка заповеди Моисея и все, что
из них вытекает, непреложный закон. Но чтобы знать закон для всего круга, надо
«проинтегрировать заповеди»...
%%%cit_comment
%%%cit_title
\citTitle{Три Столицы}, В. В. Шульгин
%%%endcit

%%%cit
%%%cit_head
%%%cit_pic
%%%cit_text
Хотя, конечно, если начать искать \emph{смыслы} даже в таких идиотских подделках, то
вышел показательный вариант картины, как отражение нынешнего состояния Украины,
раздираемой гражданской войной, геноцидом, меняющей одного за другим срамных
правителей. Как результат: исковерканный бездарными руками образ великого
русского князя, одного из тех, кто заложил фундамент русской цивилизации,
русского мира, русской государственности»
%%%cit_comment
%%%cit_title
\citTitle{Небратья обокрали русского художника}, Бунтовский, zen.yandex.ru, 08.11.2021
%%%endcit
