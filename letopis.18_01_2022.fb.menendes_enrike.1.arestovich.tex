% vim: keymap=russian-jcukenwin
%%beginhead 
 
%%file 18_01_2022.fb.menendes_enrike.1.arestovich
%%parent 18_01_2022
 
%%url https://www.facebook.com/e.menendes/posts/6849561511752883
 
%%author_id menendes_enrike
%%date 
 
%%tags arestovich_aleksei,politika,tkg,ukraina,uvolnenie
%%title Увольнение Арестовича
 
%%endhead 
 
\subsection{Увольнение Арестовича}
\label{sec:18_01_2022.fb.menendes_enrike.1.arestovich}
 
\Purl{https://www.facebook.com/e.menendes/posts/6849561511752883}
\ifcmt
 author_begin
   author_id menendes_enrike
 author_end
\fi

Ушла эпоха... Арестович уволился с поста советника главы Офиса Президента и
спикера украинской делегации в ТКГ.

В таких случаях СМИ обычно делают подборку самых ярких, может быть, скандальных
высказываний, чем запомнился человек. Но если примерить это к Арестовичу, то у
него скандальным было каждое заявление. Каждый раз, когда человек что-то
говорил, хотелось покрутить пальцем у виска. Тут в пору попробовать сделать
анти-список, типа «адекватные заявления Арестовича». Но этот список будет пуст.

Конечно, проблема не в самом Арестовиче. Я много раз пересекался с ним на
эфирах и это обаятельный, не глупый человек, который просто выполняет
определённую роль. Почему он на неё соглашается оставлю за скобками, потому что
это отвлекает нас от главного – неадекватного Офиса Президента. 

Под словом «неадекватный» я имею в виду, что главный переговорный орган страны
просто несоразмерен масштабу проблем, которые перед нами стоят. Он просто на
просто не в состоянии хотя бы вести переговоры, не говоря уже о том, чтобы дать
позитивный результат. Я как-то писал, что вместо всей украинской делегации в
минской ТКГ можно поставить картонный трафарет спайдер-мэна и ничего не
изменится. Я был не прав. Трафарет смотрелся бы солиднее.

Наша делегация (за редким исключением, ведь там есть и хорошие люди) это
заевшая пластинка, которая постоянно ретранслирует глупости. Вредные глупости.
А иногда ещё и портит воздух в комнате. Согласитесь, что трафарет выглядит на
этом фоне выигрышнее. И это безусловная проблема ОП, ведь рыба гниёт… блин,
кого я обманываю – рыба сгнила.

В то время, как нам критически нужны новый переговорный формат с Россией,
перезапуск минского трэка и оживление Н4, мы судорожно готовимся – подумать
только, что мы всерьёз это обсуждаем, - к вторжению! Зачем нам тогда вообще ОП?
Если война это наша единственная альтернатива, то это задача Генштаба. А дороги
может и местная власть строить.

В последнем абзаце хочется воззвать к совести и разуму людей, которые принимают
решения, но что-то мне говорит о том, что это бесполезно. Хотя попробую –
АСТАНАВИТЕСЬ!

P.S. Друзья, писать тексты почти каждый день – это работа. И мне будет приятно,
если вы поддержите меня на Патреоне. Один доллар в месяц – это не большая сумма
(но можно и больше), а у меня пойдёт на благое дело. Вот ссылка, где можно
подписаться \url{patreon.com/Enrique_Menendez}

\ii{18_01_2022.fb.menendes_enrike.1.arestovich.cmt}
