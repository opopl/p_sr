% vim: keymap=russian-jcukenwin
%%beginhead 
 
%%file 29_03_2022.fb.solovjov_mikita.harkov.demsokyra.2.o_metro_ubezhischah.cmt
%%parent 29_03_2022.fb.solovjov_mikita.harkov.demsokyra.2.o_metro_ubezhischah
 
%%url 
 
%%author_id 
%%date 
 
%%tags 
%%title 
 
%%endhead 
\zzSecCmt

\begin{itemize} % {
\iusr{Anna Baklygina}
Полностью согласна. Проблемы с психикой у людей уже катастрофически а будет ещё хуже..

\begin{itemize} % {
\iusr{Ирина Булгакова}
\textbf{Анна Баклыгина} заметила, что у тех, кто сидит ночью дома под обстрелами, с психикой ещё хуже

\iusr{Anna Baklygina}
\textbf{Ирина Булгакова} , я нахожусь дома и вроде ещё могу адекватно реагировать. Метро замкнутое пространство и концентрация людей большая и это не есть хорошо. Временно да, постоянно нет.

\iusr{Ирина Булгакова}
\textbf{Анна Баклыгина} расскажите это моей знакомой, которая звонит мне сейчас с ППоля и дрожащим голосом вперемежку со слезами рассказывает про то, как прилетело возле её дома

\iusr{Anna Baklygina}
\textbf{Ирина Булгакова} , вы идиотка. Писать в соц.сети место и время прилёта уголовно наказуемое преступление сейчас. Посему, идите на хуй. Микита Соловйов, прошу прощения что у Вас под постом.

\iusr{Roman Frolov}
\textbf{Anna Baklygina} та вроде никто ничего не вьідал
тут же про реальную проблему, как люди реагируют на близкую опасность (замените мьісленно тот район на любой другой) и не надо посьілать.

\iusr{Anna Baklygina}
\textbf{Roman Frolov}, А Вы изначальный пост читали или Вам просто поговорить и вставить свой пятак? Мужчина с закрытым профилем..
\end{itemize} % }

\iusr{Yuliia Pimenova}

Ну это ж не цветочки дарить в метро, а целые задачи и процессы. Сильно
сомневаюсь, что возьмутся

\iusr{Oleg Chernyak}

Ты шо, а где же герой обороны Харькова Терехов пиар-концерты проводить будет?

\begin{itemize} % {
\iusr{Andrey Fedorinin}
\textbf{Oleg Chernyak} 

безотносительно Терехова и без сарказма - метро это убежище, точка. К
сожалению, в нашем с тобой городе нет достаточного количества бомбоубежищ (не
нужно рассказывать про подвалы и т.д. - это не бомбоубежища). Да, проблема с
отсутствием быстрого и безопасного транспорта существует, но на данный момент
метро решает проблему банального здорового сна для многих людей. Относительно
того, что не выходят на улицу - бред сивой кобылы, из того, что я вижу в своем
районе. Респираторные заболевания - я лично заболел после полуторачасовой
очереди на кассу в классе, а не потому, что спускаюсь спокойно поспать в
бомбоубежище. На счет эвакуации... Ну камон - реально переселить людей в
села... А по ним ничего не прилетает? Про психологические проблемы вообще
промолчу... выводить на 500-600 метров погулять... Да ну еб вашу машу... Может
где-то это и нужно делать, не спорю, но не везде - из того что вижу конкретно
тут - отсутствуют такие, кто месяц сидит на жопе в метро и разводит
антисанитарию. Вообще можно написать все, что угодно по этому поводу, но
хочется сказать одно - если в городе все так безопасно - хочу видеть работающий
наземный транспорт.

\end{itemize} % }

\iusr{Нина Юхименко}

Все равно не понимаю, почему россии можно стрелять по Украине, а Украине по
россии нельзя?

\begin{itemize} % {
\iusr{Viktoriia Levizka}
\textbf{Нина Юхименко} потому, что НЕЧЕМ! у нас практически нет наступательного вооружения. скажите спасибо зегандонам

\iusr{Kostyantyn Filonenko}
\textbf{Нина Юхименко} там де змогли - там в'єбалі. Наприклад, Мілерово

\iusr{Pavlo Vinnyk}
\textbf{Нина Юхименко} думаю проблема в том что нам нечем стрелять на такие расстояния

\iusr{Григорий Степанов}
\textbf{Viktoriia Levizka}
ЧЕМ СТРЕЛЯТЬ?!

\iusr{Sergiy Fakas}
\textbf{Нина Юхименко} 

Самое дальнее наше оружие єто советская ракета Точка-У. Дальность 120 км.
Именно ей ударили по Миллерово и потопили БДК \enquote{Саратов} в Бердянске.

\iusr{Serge Bikhunenko}
\textbf{Sergiy Fakas} 

я так понимаю, что проблема еще и с точностью - \enquote{точка} старая, без жпс и
прочих современностей наведения. А то так вполне можно было бы по скоплениям в
белгородской области кинуть.

И мало их осталось  @igg{fbicon.frown} 

\iusr{Sergiy Ryabykin}

Тому що більшість, в тому числі харків'яни, обрали Зелю. А він позакривав
проекти розвитку ракет типу Вільха (130км) та Нептун (25км).

Озерніться навкруги і подякуйте.

\iusr{Sergiy Fakas}
\textbf{Serge Bikhunenko} Та нормально там с точностью, Саратову хватило  @igg{fbicon.smile} . Но мало их скорее всего.

\iusr{Oleksii Fesenko}
\textbf{Нина Юхименко} 

потому что мы не они. Мы не убиваем мирных просто из ненависти из за того что
они не такие как мы. Тот факт что с территории рф можно ударить по Харькову с
ураганч означает что с Харькова можно ударить по рф, но зачем?  @igg{fbicon.smile} 

\iusr{Andriy Trushevskyi}
\textbf{Sergiy Ryabykin}, 

ніхто нічого не закривав. Вільха вже на озброєнні, навіть вільха-м, що на 180км
стріляє (хоча мабуть не так багато як хотілось би), а нептун мали прийняти на
озброєння в квітні.

Ось про Вільху з вікипедії: \enquote{РСЗВ «Вільха» використовувалась з перших днів
російського вторгнення в Україну 2022 року. Станом на 2 березня було зроблено
близько 50 влучних ракетних ударів.}

\url{uk.wikipedia.org/wiki/Вільха_(ракетний_комплекс)}

\iusr{Khrystyna Kuznietsova}
\textbf{Oleksii Fesenko} в смысле, зачем? Люди не должны платить за свои поступки?

\iusr{Oleksii Fesenko}
\textbf{Кристина Кузнецова} 

для этого вводятся санкции и репарации. Мы не в ХV веке чтобы в ответ на
убийства убивать, как бы сильно кому-то этого не хотелось.

У меня тоже такие мысли бывают но с ними нужно бороться

\iusr{Oleksii Fesenko}
\textbf{Кристина Кузнецова} 

а еще смотрите, психика ломается не только у жертв, но и у убийц, как рф
образца 90х. Нам бороться со своими демонами, им со своими. Зачем нам лечиться
сначала от \enquote{нас убивали} а потом от \enquote{мы убивали}?

\iusr{Марійка Мороз}
\textbf{Нина Юхименко} 

тому, що війна не об'явлена. Де-юре. Будь-який постріл по території рф розв'яже
хуйлу руки на тотальну примусову мобілізацію. Він зараз потрапив у свою ж
пастку: війни нема, \enquote{спецоперація}, тому він не може поставити у військо усіх
чоловіків країни. Поки він це робитиме завуаліровано, а значить - в рази
повільніше. Час грає на нас: в них закінчуватимуться ресурси, а економіку
обвалюють санкції. Але якщо він оголосить в рашці воєнний стан, це вийде нам не
на користь.

\iusr{Viktoriia Levizka}
\textbf{Григорий Степанов} вы МЕНЯ спрашиваете?

\end{itemize} % }

\iusr{Женя Дисс}
Повністю згодна. І пустити метро.

\iusr{Женя Дисс}
Давно думаю про це.

\iusr{Sergii Gizelo}

І є ще один нюанс:

Я от бажаю запустити своє виробництво, маленьке, але все ж таки.

А зібрати людей до праці без метро майже неможливо, пішки по двадцять
кілометрів на день не находиш, автівками - варіант економічно безглуздий

\iusr{Artem Korotenko}

Чимось схожа історія з залишками майдану, який простояв до літа 2014 року. Всі
просто затягують рішення проблеми, що назріла

\iusr{Svitlana Polyakova}

Есть две стороньі медали.

Я сидела до 20. 03 в Киеве (понимаю, что єто не Харьков), периодически спускаясь в
б/у в соседнем доме. А потом вьіехала в євакуацию, через Польщу.

І, хочу сказать, что моя психика больше страдает вдали от опасности
,т. к. євакуация, єто то еще испьітание. При том, что я об'ездила (поездки на 50\%
бьіли связаньі с работой) более 10 стран на разньіх континентах-Азия Америка
,Европа.

Євакуация + беженство -не туризм!

\begin{itemize} % {
\iusr{Микита Соловйов}

Так сидеть дома и периодически спускаться в подвал это совершенно другой
вариант. Сейчас в Харькове так делают многие. А вот сидеть сутками в метро, это
и по психике, и по санитарии и по всему остальному жесть полная.

\iusr{Оля Хабарова}
\textbf{Свитлана Полякова} 

а почему, объясните,пожалуйста.

Моя дочка в эвакуации неделю плакала, конечно, а сейчас, вроде бы все в
порядке.

Может просто времени ещё мало прошло?

\iusr{Ирина Булгакова}
\textbf{Микита Соловйов} 

да никто и не сидит, в 6.00 платформа стремительно пустеет, все разбегаются по
домам, именно чтобы помыться, поесть, заниматься ежедневными делами.
Большинство приходит только переночевать.

\iusr{Svitlana Polyakova}
\textbf{Оля Хабарова} Напишите, плз, где ваша дочь и сколько ей лет?
Єто тоже имеет большое значение.

\iusr{Оля Хабарова}
\textbf{Свитлана Полякова} в Германии, дочке 32, внуку 3,5.

Приютили очень хорошие люди. Пара примерно моего возраста, у них тоже двое
внуков, живут отдельно, но недалеко.

Добирались дочка с внуком очень тяжело, эвакуационным поездом, потом
переночевали в Варшаве и поехали дальше

\iusr{Svitlana Polyakova}
Да, Германия отличается неожиданной доброжелательностью .

\iusr{Svitlana Polyakova}
\textbf{Оля Хабарова} 

Но вьі же понимаете, кто остался в метро? К єтим людям необходим особьій
подход. И да, возможно и психолог.

Более решительньіе давно вьіехали.

\iusr{Оля Хабарова}
\textbf{Свитлана Полякова} это я понимаю, конечно.

\iusr{Оля Хабарова}
\textbf{Свитлана Полякова} 

немцы даже называют украинцев гостями, а не беженцами. Принимают, как членов
семьи. Это потрясающе, на самом деле

\iusr{Marina Novoselskaya}
\textbf{Svitlana Polyakova} 

я вас розумію, я думала, якщо поїду, буде трохи легше морально, а вийшло так,
що мені ще важче, хоча дякувати Богу, в нас неймовірна кількість прекрасних
людей, які підтримують та допомагають. А додому тягне нереально. Коли бачу, що
у людей розбиті домівки і їм нема куди вертатися, мені аж ножем по серцю (

\end{itemize} % }

\iusr{Olexandr Burlaka}

Метро \enquote{Тракторний завод} Якщо вийти з нього, то жодної зруйнованої будівлі
знайти важко. Навіть розбите скло припадає переважно натабачні кіоски, що стали
жертвою мародерів

\iusr{Andrey Bogdanovich}

С первого дня войны живу дома, на салтовке, студенческая. В подвале провел
только первый день, только день, ночевали уже дома. Если чесно, то я вообще не
понимаю как в метро можно столько высидеть?! У кого психика слабая давно
уехали, мои жена и дочь в их числе, в глубь страны, где потише. Мы с мамой
остались. Жизнь продолжается, а не рабочее метро приносит определенные
неудобства, и вспышки какой-нибудь заразы вполне реальны! Стопроцентно
согласен! Рискуем нажить еще проблем. Короче говоря я двумя руками \enquote{за}!

\begin{itemize} % {
\iusr{Elena Hizhnaya}
\textbf{Андрей Богданович} 

я тоже на Студенческой живу, но временно уехала. Скажите пожалуйста, там же уже
не так страшно? Очень хочу домой, но боюсь.

\iusr{Elena Hizhnaya}
\textbf{Андрей Богданович} 

в первый день я спустилась в метро, увидела толпу людей с животными, детьми на
полу и у меня начался приступ панической атаки. Я не смогла там находиться
из-за концентрации страха. В подвале тоже не ночевала, там холодно и сыро

\iusr{Микита Соловйов}
\textbf{Андрей Богданович} Вот спускаться или нет в подвал, я отказываюсь обсуждать. Тем более на Салтовке.

\iusr{Elena Hizhnaya}
\textbf{Микита Соловйов} я читала что в подвале нужно иметь два выхода, иначе это братская могила будет. У нас часть подвала под какой-то спортзал захвачена

\iusr{Ирина Кравченко}
\textbf{Elena Hizhnaya} Живу недалеко от Студенческой. Недавно в районы поблизости на Академика Павлова было несколько прилетов. Разнесло несколько домов в частном секторе. Бахает очень сильно постоянно. Хотя сегодня по городу ездила, в других районах было тихо.

\iusr{Elena Hizhnaya}
\textbf{Ирина Кравченко} я только собралась домой ехать, прочитала про случай на новой почте и осталась. Хотя каждый день думаю о городе. Не хочу нигде жить кроме Харькова

\iusr{Viktoriia Levizka}
\textbf{Elena Hizhnaya} там и не было ТАК страшно. ул Барабашова. пески

\iusr{Andrey Bogdanovich}
\textbf{Елена Хижная} 

страшно, потому-что прилетает, и ты не знаешь куда прилетит. Но прилетает везде, безопасных мест в городе нет, с точки зрения прилета. Просто нужно принять это и жить дальше, насколько это возможно в этой ситуации. Переходить дорогу иногда тоже страшно, масса случаев тому подтверждение. Но страх этот притуплен, т.к. мы не обращаем на него внимания в силу его повседневности, примерно так-же нужно поступить со страхом прилета, прилет так-же возможен как и вылетевший на красный автомобиль, нужно просто научиться воспринимать сложившуюся реальность, ее мы уже имеем и нам с ней жить.

\iusr{Elena Hizhnaya}
\textbf{Андрей Богданович} спасибо большое за ответ.

\iusr{Andrey Bogdanovich}
\textbf{Елена Хижная} не за что. Учимся жить в новой реальности! Израиль в ней давно живет. Привыкли.

\end{itemize} % }

\iusr{Viktoriia Levizka}
на Студняке то же люди нормально живут

\begin{itemize} % {
\iusr{Elena Hizhnaya}
\textbf{Viktoriia Levizka} как я рада это слышать!!! Я тоже оттуда

\iusr{Viktoriia Levizka}
\textbf{Elena Hizhnaya} 

иногда проблема с коммуникациями. отопление, газа не было 4 дня. ни разу не
ходили даже в подвал. во двор было 2 прилета еще в начале. все. а про НП - то
была дальняя. это очень редко и нигде не застраховано(( просто не нужно торчать
на открытых местах. даже в квартирах правило второй стены работает

\iusr{Elena Hizhnaya}
\textbf{Viktoriia Levizka} спасибо большое за ответ!

\iusr{Viktoriia Levizka}
\textbf{Elena Hizhnaya} 

у меня все на Салтовке. в районе Краснодарской и то было более стремно, чем в
районе Студняка. на Мавзолее, еще меньше проблем. только вода было с пониженным
тиском. ну и какое-то время ночевали в коридоре

\end{itemize} % }

\iusr{Олена Монова}

Все логічно і правильно, але гнилі помідори ловитимеш все одно

\iusr{Микита Соловйов}
Ну если бы я этого боялся, то не писал бы примерно 90\% текстов )

\iusr{Игорь Дубровський}

Групу потрібно створювати. Не мережеву. Місто, метро, нуо. Чия ініціатива буде
сприйнята? Не знаю. Але згоден на сто відсотків

\iusr{Yulia Vepritskaya}
Поддерживаю. Их нужно вытаскивать оттуда и эвакуировать желающих.

\iusr{Anton Bondarev}

Ууу.. От чую... Шо скоро набегуть, обвинять, проклянуть, изувером и аспидом
окоянным кликать стануть, у изуверы запишуть.Ну и вестимо, опосле сожгуть твоё
чучело... На рельсах промеж станций @igg{fbicon.face.wink.tongue}{repeat=3} 

\begin{itemize} % {
\iusr{Svitlana Polyakova}

Єто не смешно, и нет причин для стеба.

\iusr{Anton Bondarev}
\textbf{Svitlana Polyakova} 

да ни кто не стебеца. это скорее грустная ирония. Тема актуальная и для города
злободневная. Но я на днях на странице знакомой которая написала на эту же тему
столько проклятий, оскорблений и угроз прочёл от оппонентов что был в шоке...

\iusr{Микита Соловйов}
\textbf{Антон Бондарев} Удивляет, что до сих пор не.

\iusr{Svitlana Polyakova}
\textbf{Anton Bondarev} Просто не пришел еще тот....

\iusr{Anton Bondarev}
\textbf{Микита Соловйов} 

может просто потому шо у Вашего Благородия слишком уж интелегентная и утонченно
изысканная публика/почитатели?  @igg{fbicon.beaming.face.smiling.eyes}{repeat=3} Вот и хамить и дерзить не изволят @igg{fbicon.face.wink.tongue} 

\iusr{Микита Соловйов}
\textbf{Антон Бондарев} А еще очень помогает висящий у входа топор и большая табличка \enquote{Осторожно! Здесь могут послать нахуй!}
\end{itemize} % }

\iusr{Женя Дисс}

У нас одного разу зламалась автівка, десь на Сході. Зима, холодно, не супер,
десь -5.

За кілька хвилин біля нас зупинилась автівка, люди запропонували допомогу( і
допомогли). Пропонували пересісти до них, погрітись.

І я зрозуміла, як люди замерзають насмерть, коли загубились, наприклад.

Бо мені так не хотілось виходити, взагалі змінювати позу. Я навіть розуміла, що
так не можна, але виходити із зони дискомфорту не було ніяких сил.

І це відбулося зі мною за 20 хвилин. А люди так сидять місяць майже. Це дупа.
Але потрібно дуже акуратно їх виводити тепер

\iusr{Оксана Масалітіна}

Я узнала что оказывается в Киеве, где можно сказать вообще спокойно, и уж точно
намного безопаснее пойти домой, собрать вещи и сесть потом на поезд в
эвакуацию, так вот, что в Киеве неделями в метро живут люди. Я не представляю
что там у них с психикой и как это вообще потом можно будет починить ((( но с
этим точно нужно что-то делать.

\iusr{Olena Kryzhanivska}

Абсолютно аналогічна ситуація у Києві. І транспортні проблеми через
розірваність транспортних потоків величезного працюючого міста, та ще й з
непрацюючими мостами через Дніпро! І таки-да дикі проблеми з психікою у тих,
хто там живе другий місяць. І санітарні проблеми. Дякую за підняття цієї теми.

\begin{itemize} % {
\iusr{Микита Соловйов}
\textbf{Олена Крижанівська} 

Честно говоря, в Киеве для меня это вообще непонятно. Там же реальная угроза
даже по сравнению с Харьковом микроскопическая. Но лезть со своими советами к
киевлянам я не буду, сами без меня разберутся.

\iusr{Оксана Масалітіна}
\textbf{Olena Kryzhanivska} 

от я взагалі не розумію нащо у Києві жити тижнями у метро((( ну буває лячно,
буває дуже дуже лячно, особливо якщо нема кому надати психологічну допомогу чи
\enquote{сусіди по платформі} накрутять. Але можна ж поїхати з міста. Там вже матраци,
намети (!) поставили і оселилися ледь не назавжди.

\iusr{Roman Frolov}
\textbf{Olena Kryzhanivska} в Киеве вроде метро работает по одной стороне как челнок, но не на всех линиях

\iusr{Olena Kryzhanivska}
\textbf{Микита Соловйов} 

Саме тому і дякую. Що без порад, але тема - архіважлива. Я розумію, що Києву ще
дістанеться десь наприкінці квітня - орки будуть біснуватися до свого
прибацаного \enquote{пабєдобєсія}. Але все одно - уявити в Києві людину з нормальною
психікою, що сидить другий місяць у метро, дуже важко. Ще раз вдячна за допис
та тему, про яку всі бояться згадати.

\iusr{Оксана Масалітіна}
\textbf{Микита Соловйов} все правильно ты говоришь
Но они как зашли туда в конце февраля, так и живут там

\iusr{Оксана Масалітіна}
\textbf{Roman Frolov} поезда раз в час, не останавливается на станциях в центре города и нет сообщения левого берега с правым.

\iusr{Roman Frolov}
\textbf{Оксана Масалітіна} 

ну вот у нас хотя бы начать - по самой длинной линии, чтоб обязательно была
остановка на южд, это сразу уберет проблему как вывозить людей с района хтз и
дальше на вокзал

\iusr{Оксана Масалітіна}
\textbf{Roman Frolov} да, с пролета и ХТЗ на вокзал или работу добираться это вообще жесть

\iusr{Марійка Мороз}
\textbf{Оксана Масалітіна} намети у нас в метро теж стоять! Побачила вчора фотографію м.Київська - офігєла.

\iusr{Марла Сингер}
\textbf{Олена Крижанівська} 

алкоголт з 1 квітня (якщо не жарт, звісно) запустять, може й метро хоча б не
раз в півтори години... бо кажуть працювати бізннсу а як

\end{itemize} % }

\iusr{Svitlana Polyakova}

Конечно, их нужно вьіводить из метро. Главное сейчас - не навредить больше! Єто
огромная проблема-куда везти? В Закарпатье - негде жить. Нужньі временньіе лагеря,
В Польшу - переполнена.

И добавится к имеющимся психическим проблемам, еще и стресс при перемещении + язьіковьій барьер.

\begin{itemize} % {
\iusr{Оксана Масалітіна}
\textbf{Svitlana Polyakova} да, это уж точно не те люди, которые легко приспособятся к жизни в другой стране

\iusr{Микита Соловйов}
\textbf{Свитлана Полякова} 

Негде жить" по сравнению с чем? Наши ребята занимаются расселением на западе
Украины. И условия далеко не 5*. Но уж точно в разы лучше, чем в метро.

А Польша, Чехия, Германия и т.д. это возможность вообще возвращаться к
нормальной жизни. Там везде сейчас дают право на работу.

\iusr{Elena Babenko}
\textbf{Микита Соловйов} 

право то дають, але роботи в Польщі немає - тільки для чоловіків. Або саджати
дерева. Не все так райдужно в Польщі

\iusr{Svitlana Polyakova}
\textbf{Микита Соловйов} 

Ладно, подкину вам \enquote{веселой жизни в ШВЕЙЦАРИИ}.

Группа -\enquote{Українці в Швейцарії}

Окунитесь в среду .\url{https://www.facebook.com/groups/679158755543458/?ref=share}

\iusr{Tetyana Sfandex}
\textbf{Микита Соловйов} 

\obeycr
право на работу не равно получить работу
Но ещё примерно месяц - и можно будет жить в летних домиках и даже палатках до осени
На дачах всяких
Собственно мы жили около Днепра на дачах в начале матра когда выехали
Да +13 в домике вначале, но явно не хуже чем метро
Знакомые тоже в летних домиках на ЗУ сейчас
Уже нет -15 ночью
\restorecr

\iusr{Svitlana Polyakova}
\textbf{Микита Соловйов} Єто вьі пишите, не оказавшись в \enquote{шкуре} беженца.
Страньі разньіе и условия пребьівания тоже разньіе

\iusr{Viktoriia Levizka}
\textbf{Микита Соловйов} вот только работы нет. та, что до войны стоила 1200-1300, теперь за 500 с руками и ногами((

\iusr{Olga Geraschenko}
\textbf{Свитлана Полякова} расскажите про свой опыт. Очень интересно

\iusr{Svitlana Polyakova}
\textbf{Olga Geraschenko} 

Я вьіпадаю из \enquote{правила}.

У меня есть опьіт 6 лет проживания в Женеве, в среднем по полгода в году. Моя
дочь 7 лет работала в Женеве, а я приезжала периодически помогать ей с
ребенком.

Т. к у дочери много друзей мьі не имеем проблем с проживанием.

Дочь с внуком, так сложилось, приехали сюда раньше.

А я вьіезжала из Киева 20.03.

Автомобилем до Львова. Там поезд на Пшемьісль. А оттуда 24 часа автобусом(за
свои деньги) до Женевьі.

Живем у друзей, на программу S подали, но швейцарцьі все умеют считать, и я еще
не встречала человека, которьій получил интервью, по программе для
временноперемещенньіх лиц.

Некоторьіе ждут уже 4недели.

При єтом не знаю, за что они питаются. Но в лагерях для перемещенньіх лиц как -
то кормят.

\iusr{Svitlana Polyakova}
В разньіх кантонах по разному.

\iusr{Olga Geraschenko}
\textbf{Свитлана Полякова} спасибо. Очень познавательно

\iusr{Микита Соловйов}
\textbf{Свитлана Полякова} 

Так я никого не призываю уезжать, кажется. Я вот сижу в Харькове и мне, честное
слово, хватает чем заниматься. Но аккуратно позволю себе предположить, что
найти работу в Швейцарии сейчас не сложнее чем в Харькове.  @igg{fbicon.smile} 

\iusr{Oleksii Fesenko}
\textbf{Svitlana Polyakova} Румыния, Венгрия, Болгария...

\iusr{Svitlana Polyakova}
\textbf{Микита Соловйов} 

Да, владеющим франц-немецким язьіком ,реже англ ,имеющим соответствующее
образование или работу, которую можно делать руками -маникюр, парикмахер и
прочее.

\iusr{Svitlana Polyakova}
\textbf{Олексій Фесенко} Согласна ,при условии наличия работьі.

\iusr{Женя Дисс}
\textbf{Svitlana Polyakova} Польша знаходить місця все ще. Ми туди возимо.

\iusr{Маша Бахтігозіна}
\textbf{Микита Соловйов} 

какую работу? Нелегальную? Ок, большой риск вылететь из страны сразу после того
как Вы предложите кому-то работать без оформления. Легальную? Без языка
довольно проблематично и нужно ждать, пока получишь право на работу. Мы
приехали в Дрезден 16 марта, регистрацию мне назначили на 20 мая. До этого я и
рыпнуться в сторону работы не могу. Выучить польский - ок, могу себе
представить. Выучить немецкий или французский за эти два месяца - ну Вы сами
понимаете...

\iusr{Микита Соловйов}
\textbf{Маша Бахтігозіна} 

Естественно, я говорю о легальном трудостройстве. И естественно, пока не выучен
язык, ни о какой работе по специальности или около речь не идет. Но
невозможность получить работу по специальности и близко не равно невозможности
получить какую-то работу вообще. Низкоквалифицированной (и низкооплачиваемой,
конечно) работы же, не требующей особенного общения за пределами пиджин, по
моей информации в большинстве европейских стран хватает.

\end{itemize} % }

\iusr{Ирина Федотова}

\obeycr
Згодна на всі 200\%.
Я працюю із дітками в метро.
Це просто жахливо, в яких вони там в і умовах.
Треба негайно зачиняти метро у якості сховища, та відкривати у якості транспорту.
Більш того.
Майже всі мешканці метро вдень зараз чимчикують додому і повертаються тільки на ночівлю.
Чому?
Бо в метро завжди є їжа, є корм для тварин, є безкоштовна електроенергія.
Справді наляканих людей, які бояться звідти виходити - одиниці.
Більшості там просто зручно і безплатно.
\restorecr

\begin{itemize} % {
\iusr{Светлана Прокопенко}
\textbf{Ирина Федотова} 

ще можливо що деяким мешканцям насправдi краще спати в метро, бо вдома вони всю
нiч в станi напруги та панiки чекають на на бомби та ракети.

\iusr{Женя Дисс}
\textbf{Ирина Федотова} якщо так, то добре. Нехай людям буде зручно і безплатно. Головне- щоб не вивчена безпомічність.

\iusr{Маша Бахтігозіна}
\textbf{Светлана Прокопенко} 

я не могла спати вдома. Просто уявляла собі, що за це доведеться заплатити
власним життям або життям моїх дітей і не могла залишитися. Правда, усього 4
дні, але нам просто пощастило, що друзі запропонували виїхати до Мерефи. А так
-не знаю, бігти було нікуди і нічим.

\iusr{Ирина Федотова}
\textbf{Светлана Прокопенко} 

розумієте у чому справа... Тепер вся Україна вимушена буде жити в постійному
очікуванні бомбардування, як зараз Ізраіль, бо в нас ж божевільний сусід з
купою боєприпасів. Але ж неможливо постійно використовувати метро, як
сховище...

Метро потрібно місту, аби працювати. Багато хто з бізнесу готовий відкриватися,
але вони елементарно не можуть зібрати на роботу працівників бо ті не можуть
дістатися до роботи.

Треба шукати інші варіанти. Бомбосховища у школах та ін. А \#метроповиннопрацювати

\iusr{Ирина Федотова}
\textbf{Женя Дисс} 

ви не розумієте, що на \enquote{безкоштовно} потрібні чиїсь гроші?! Принаймні, гроші
громади або міста. Якщо в місті не може працювати бізнес з-за того, що
працівники не можуть дістатися до роботи, це \enquote{безплатно} може дуже швидко
закінчитися.

\end{itemize} % }

\iusr{Igor Feldman}
Без транспорта Харьков не оживет. А никакого транспорта сейчас нет.

\begin{itemize} % {
\iusr{Микита Соловйов}
\textbf{Игорь Фельдман} Это отдельная тема. И через день-два подробно распишу.

\iusr{Igor Feldman}
\textbf{Микита Соловйов} 

наземный транспорт сейчас проблема, и потому, что это опасно, и потому, что
судя по всему его стало сильно меньше из-за обстрелов, в том числе депо. Так
что только метро сейчас способно выполнять роль городского транспорта. А оно
стоит. Значит стоит и бизнес, который мог бы работать

\end{itemize} % }

\iusr{Наталія Пахніна}
Підтримую, спілкувалась з тими, хто сидить в метро, наявні ознаки психічних розладів в них є

\iusr{Anna Slavutskaya}
\textbf{Наталія Пахніна} Може, це не причина, а наслідок (в метро залишилися тількі супертривожні люди)

\iusr{Наталія Пахніна}
\textbf{Anna Slavutskaya} , можливо

\iusr{Анна Кошелева}

Харькову нужен наземный и подземный транспорт! Людям-подземникам помощь в
психиатрическом лечении и социальной адаптации. Хотя кто этим будет заниматься.
Моим пенсионерам горсовет всё никак не довезет обещанную гумпомощь и валерьяну
с цитрамоном..

\iusr{Ірина Форстер}
Абсолютно согласна! Не такой страшный черт как его рисуют.

\iusr{Виталий Буняев}

Кстати, о санитарии.

Пункты выдачи гуманитарки нужно оборудовать передвижными биотуалетами в
достаточном количестве или чем-то подобным. Потому что одновременные 500-1000
человек, стоящие по нескольку часов в очереди, и ежедневные несколько тысяч,
проходящие через пункт, превращают в адовый звиздец окрестные улицы, переулки и
дворы. Как минимум, в общественный сортир и свалку. И так изо дня в день, из
недели в неделю.

Образовалась целая прослойка, которая кочует от пункта к пункту, а потом с
благоприобретенным спускается в метро и там просто живет.... их легко опознать
со стороны.

С этим всем точно нужно что-то делать.

\iusr{Anna Korol}
Ви очевидець, ви там живете, все бачите, значить так і є.

\iusr{Светлана Иванова}

Ч полторы недели просидела в метро, но я выходила на несколько часов домой.) В
метро ходила только на ночь, не могла заснуть под гупанье

\iusr{Dmytro Kurilo}

А какая подготовка нужна для того, чтобы запустить метро? Я думаю в случае
штатной работы эволюционно решаются сразу обе проблемы: 1. Метро как транспорт
реально нужен городу. 

2. \enquote{Сидельцы} за пару дней, посмотрев на живых людей, сами потихонечку
начнут выдвигаться к своим домам или к эвакуаторам.

Что-то мне подсказывает, что немногих оставшихся будет легче социализировать.

В принципе, это тоже самое, про что и пост, но только чуть меняется порядок
действий.

Но, в принципе, поддерживаю автора обеими руками - зачем оттягивать
неизбежное?!

\begin{itemize} % {
\iusr{Ерофей Подольский}
\textbf{Dmytro Kurilo} надо сделать как в Киеве. Метро работает днём.

\iusr{Dmytro Kurilo}
\textbf{Ерофей Подольский} Для старта - неплохой ход

\iusr{Микита Соловйов}
\textbf{Дмитрий Курило} Нет, так не получится. Для того, чтобы возвобновить работу метро, придется СНАЧАЛА выселить сидельцев. И там будет еще не на один день работы после этого.

\iusr{Dmytro Kurilo}
\textbf{Микита Соловйов} не поясните? Разве метро нельзя запускать по частям? Линиями, станциями? Другие варианты?

\iusr{Ерофей Подольский}
\textbf{Микита Соловйов} тем не менее в Киеве так всё работает. Есть люди которые сидят постоянно, есть люди которые ночуют. Метро работает только по одной стороне, вагон едет раз в час...

\iusr{Olga Suvorova}
\textbf{Дмитрий Курило} люди в том числе и в вагонах живут, которые стоят на станциях.

\iusr{Dmytro Kurilo}
\textbf{Olga Suvorova} Ужас какой. Перебор...
\end{itemize} % }

\iusr{Viktoria Nesterenko}
Підтримую на всі 100\%! І дякую, що пишете про це  @igg{fbicon.smile} 

\iusr{Irina Nesvitaylo}

Поддерживаю, надо запускать метро, это самый безопасный и быстрый транспорт для
Харькова. Но там есть люди, которые лишились жилья совсем....с ними будет
проблема, куда им идти?

\begin{itemize} % {
\iusr{Tetyana Sfandex}
\textbf{Irina Nesvitaylo} к друзьям, знакомым. Спрашивать кто может пустить. Многие выехали и в принципе жильё по городу есть.

\iusr{Микита Соловйов}
\textbf{Ирина Несвитайло} Если людям негде жить в Харькове, то тогда уж лучше в эвакуацию, чем в метро.
\end{itemize} % }

\iusr{Natalie Neviasky}

Микutа, я пережила длительную бомбардировку в Израиле в 91м году (но у вас
ситуация намного хуже сейчас). Этих людей просто так уговорить в этой стадии
будет крайне трудно. Здесь нужно собрать оставшихся психологов и психиатров, и
всех туда забросить. Большинству из находящихся там сейчас уже необходима
психологическая и психиатрическая помощь.

\begin{itemize} % {
\iusr{Микита Соловйов}
\textbf{Natalie Neviasky} Я человек довольно резкий. И в данном случае говорю не столько об убеждении, сколько о принудительном выселении. Естественно, эту неделю дать на то, чтобы свыкнуться с мыслью.
Да, и я Микита, а не Микола.

\iusr{Natalie Neviasky}
\textbf{Микита Соловйов} izvinite, ya znayu- typo (typing through Translit- no Cyrillic here - at work)

\iusr{Natalie Neviasky}
\textbf{Микита Соловйов} im nuzhna psihologicheskaya pomoshch, a ne rezkost', oni navernoe tam vse s det'mi ili starikami
\end{itemize} % }

\iusr{Лена Полторацкая}

Встречаю в саду Шевченко жителей метро. Бледные, с потеряным
взглядом, оцепенелые. Им реально необходимо выходить наружу, участвовать в любых
процессах.

\iusr{Boris Sevastyanov}

На Гертруде был сегодня 2 часа. Ну да, слышно, и приходы и уходы. Но
адаптируется быстро. Главное научить людей различать. А так - тишь, благодать.

\iusr{Дмитрий Баевский}
\textbf{Boris Sevastyanov} на северной 2 люди живут. Не только в убежище. Но то уже - сами говорят, на Бога полагаются..

\iusr{Boris Sevastyanov}

И ещё говорили сегодня как раз про запуск метро. Хоть раз в час электричка.
Людям по делам надо иногда, что-то открывать заново, работать. На такси не
наездишься. Хотя вчера сегодня и в этом смысле не ощутил какой-то обдираловки.
Нормальные цены в онтакси. Но, это уже мой вопрос нелюбви к метро )

\iusr{Микита Соловйов}
\textbf{Борис Севастьянов} 

Там масса своих технических вопросов будет с запуском. Обсуждаем несколько
последних дней. Но пока этот режим убежища не отменить, запуск в принципе
нереален. В реверсном варианте как в Киеве шансов нет чисто технически на двух
из трех веток.

\iusr{Igor Prykhodko}

Есть одна инфекционная хрень, которую никто не отменял: ковид. Хоть он сейчас и
не такой страшный, как в начале пандемии, но всё равно доставляет много хлопот
врачам, оставшимся в городе. Плюс сезонные вирусные инфекции, которые могут
быть очень неприятными. Плюс обычая бытовая антисанитария...

Хотя бы с этой точки зрения с убежищами в метро надо что-то решать.


\end{itemize} % }
