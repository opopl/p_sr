% vim: keymap=russian-jcukenwin
%%beginhead 
 
%%file 28_11_2021.fb.kofman_aleksandr.doneck.dnr.1.religii_ustareli_vse
%%parent 28_11_2021
 
%%url https://www.facebook.com/permalink.php?story_fbid=4702605896468655&id=100001578740019
 
%%author_id kofman_aleksandr.doneck.dnr
%%date 
 
%%tags chelovek,obschestvo,religia,world
%%title Религии устарели. Все
 
%%endhead 
 
\subsection{Религии устарели. Все}
\label{sec:28_11_2021.fb.kofman_aleksandr.doneck.dnr.1.religii_ustareli_vse}
 
\Purl{https://www.facebook.com/permalink.php?story_fbid=4702605896468655&id=100001578740019}
\ifcmt
 author_begin
   author_id kofman_aleksandr.doneck.dnr
 author_end
\fi

Религии устарели. Все.

Тот, кто изобретет новую, объединяющую различные течения религию, будет владеть
миром.

\begin{itemize} % {
\iusr{Alexandr Kolesnik}
Она уже есть. Сатанизм называется.

\begin{itemize} % {
\iusr{Димьян Донецкий}
\textbf{Alexandr Kolesnik} Саня привет, скорее эта религия деньги

\iusr{Александр Борисовский}
\textbf{Alexandr Kolesnik} формально, сатанизм это ересь в рамках христианства, так что нет.
Та же история с исламом.
\end{itemize} % }

\iusr{Борис Борисович}
Ну тогда Ом-минь, правоверныя!)))

\iusr{Юлий Буркин}
\textbf{Борис Борисович}
Воистину, акбкар.

\iusr{Юлий Буркин}
Чтобы заставить мир во что-то верить, нужно самому верить в это фанатично. А верить в то, что сам рассудочно изобрел, невозможно.
Такая религия должна сложиться как бы сама собой, через большие жертвы. Иначе это будет такой же "всеобщий язык", как никому не нужное эсперанто.

\iusr{Владимир Федоров}

Полежав в ковидной реанимации, мне сложно рассуждать о том, устарели ли
религии. Как по мне, всё на месте. И Бог, и его оппонент, к сожалению, тоже.  @igg{fbicon.smile} 

\begin{itemize} % {
\iusr{Yevgeniy Tinianskiy}
\textbf{Владимир Федоров} нет никакого оппонента. И не было никогда

\iusr{Владимир Федоров}
\textbf{Yevgeniy Tinianskiy} не готов вести теологические споры, тем более с представителем религии, в них изрядно поднаторевшей.  @igg{fbicon.smile} 
\end{itemize} % }

\iusr{Владимир Федоров}
Да и сложно говорить об "устаревании религии", глядя на взрывной рост исламского фундаментализма.

\iusr{Олег Бондарев}
Ломать сформировавшиеся механизмы дико сложно... (

\iusr{Эдуард Гродзинский}
Такой не будет.
Финансовые потоки будут сильно возражать )

\iusr{Ирина Ремезова}
Уже и храм новой религии в Эмиратах построили

\iusr{Светлана Никулина}

\obeycr
Какая Богу разница
В какой идёшь ты храм,
Коль сердце твоё чёрствое
К Божественным делам.
Будь милосерден, честен,
Не лги и не блуди,
Души людской погибели
Ты в жизни не верши.
Религия единая -
Будь Человеком ты,
Тогда земля окрепнет,
Не будет и войны!
\restorecr

\iusr{Polina Anenkova}

Да, да, ждите, он придет. Мессия новой религии на три года. Все это давно описано

\begin{itemize} % {
\iusr{Инна Сасим}
\textbf{Polina Anenkova} тоже об этом подумала. Даже жутко немного .

\iusr{Polina Anenkova}
\textbf{Инна Сасим} 

ну так а что мы можем сделать? И потом, это только на три года. А потом Встреча
с Богом. Так что пусть все будет, как будет

\end{itemize} % }

\iusr{Аркадий Окунев}
Теисты уже пробовали. Не прижилось.

\iusr{Abraimov Leonid}

Религия всегда была нужна для подчинения и управления, и сейчас эту роль
выполняет потреблянство всяких форм, и воздвигнуто множество храмов религии
потребления, и молятся богам её с самых высоких трибун во всём мире.


\iusr{Роман Бобков}
Ну, я изобрёл. Высший синкретический дзен-пофигизм.
Где расписаться в получении мира?

\iusr{Николай Белов}
Сразу видно, посмотрел Зов из Ада.

\begin{itemize} % {
\iusr{Александр Кофман}
\textbf{Николай Белов} нет, а что это?

\iusr{Николай Белов}
\textbf{Александр Кофман} Сериал. Там как раз про новую религию.


\iusr{Александр Кофман}
\textbf{Николай Белов} хммм. Надо посмотреть.
\end{itemize} % }

\iusr{Алексей Маштаков}
нет Бога кроме.... да его просто нет.

\iusr{Zahar Kariagin}
Так устарели или нужна новая?

\begin{itemize} % {
\iusr{Александр Кофман}
\textbf{Zahar Kariagin} именно потому что устарели и нужна.

\iusr{Zahar Kariagin}
\textbf{Александр Кофман} это не всегда связанные вещи
\end{itemize} % }

\iusr{Оляша Фёдорова-Пандази}
Погнали изобретать

\begin{itemize} % {
\iusr{Елена Баззаева}
\textbf{Оляша Фёдорова-Пандази} вышивать первый этап научишь?  @igg{fbicon.face.blowing.kiss}  @igg{fbicon.face.wink.tongue}  @igg{fbicon.face.shushing}  @igg{fbicon.laugh.rolling.floor}{repeat=5} 
У человека без веры нет опоры в жизни , от сюда страх о будущем и соответственно рабское состояние .
Если есть опора веры - то хоть болото , хоть угли проходить человеку.
Вот и сейчас те кто в вере на Бога ( Аллаха и так далее ) надеятся и стоят крепче , чем те кто атеист ..
А потому "вышивка" опорой стать возможно сможет , но не надолго
Веру в сердце хотя бы в светлое будущее Донбасса надо давно сделать не идеей , а именно делом.
Ох... @igg{fbicon.face.pensive}  @igg{fbicon.heart.broken} 
\end{itemize} % }

\iusr{Михаил Боревский}

Уже сама идея религии устарела.
Нужна Вера. И новая форма.
Пока религию заменяет пропаганда и маркетинг

\iusr{Dennis Sched}
да, этот вопрос стоит очень остро последние несколько десятков лет. Нужен культ адаптированный к последним достижениям науки и техники

\iusr{Елена Баззаева}

У человека без веры нет опоры в жизни, от сюда страх о будущем и соответственно рабское состояние .
Если есть опора веры - то хоть болото, хоть угли проходить человеку.
Вот и сейчас те кто в вере на Бога ( Аллаха и так далее ) надеятся и стоят крепче, чем те кто атеист ..
А потому "вышивка" опорой стать возможно сможет, но не надолго
Веру в сердце хотя бы в светлое будущее Донбасса надо давно сделать не идеей, а именно делом.
Ох... @igg{fbicon.face.pensive}  @igg{fbicon.heart.broken} 

\begin{itemize} % {
\iusr{Abraimov Leonid}

Вера - это состояние духа, религия - процедура, ритуал, должный привести в
нужное состояние душу. И о рабстве. Во всякой религии есть идея покорности воле
Бога.

Но без Веры жить трудно, а всякой вере нужна точка опоры, вот её и принято
называть Бог. А религия, в данном случае, рычаг, а рычаги м.б. короткими,
длинными, прочными, гнилыми и т.д.

В общем, религия - это инструмент манипуляции сознанием, и самая господствующая
сейчас религия (как я выше говорил) - потреблянство. Так что предпочитаю
атеизм.

\end{itemize} % }

\iusr{Vlad Donskoy}

Всё уже придумано до нас ©
Новая религия - Дзен-суннизм
из цикла "Дюна" Фрэнка Хэрберта

\iusr{Надежда Аникина}
Так и придёт Антихрист

\iusr{Игорь Шенгальц}

Так изобрели уже и владеют миром, корона-религия ) У нас вон 2г+ правило ввели,
мало быть просто привитым, но и актуальный тест необходим


\iusr{Алексей Лотов}

Не будет никаких религий в свете
Новой парадигмы мировоззрения
с критерием истины. Книга написана в четвёртой итерации.
Квантовый мир один, целостен, логичен, а все мы его неотъемлемые части.
Самое простое определение бога - Бог есть самая сложная сущность Мира.
Вера есть яркая визуализация достижимой цели, иначе это фантазия.
Вечность есть интервал времени содержащий в себе любой интервал времени.
Квантовый феномен сознания необходим для адекватного отражения квантового Мира.
И так далее

\begin{itemize} % {
\iusr{Sergius Alex}
\textbf{Алексей Лотов} продолжайте... оочень интересно.

\iusr{Алексей Лотов}
\textbf{Sergius Alex} книгу
Новая парадигма мировоззрения
легко скачать бесплатно в интернете с ridero ru
\end{itemize} % }

\iusr{Катя Сказка}

Как в принципе можно "уверовать" в "изобретённую/придуманную" религию?🤦 Одно
это словосочетание показывает весь маразм данного действия)))

Истинную религию, если таковая была на самом деле, давно видоизменили, исказили
и переписали... И то, что мы имеем сейчас, ничего общего с настоящей религией
не имеет)))

И какая бы "новая" не была придумана - будет носить разряд "новой секты", и
ничем никому не поможет, ибо правды в ней нет!)

\begin{itemize} % {
\iusr{Александр Кофман}
\textbf{Катя Сказка} путаешь религию и веру.
\end{itemize} % }

\iusr{Gogi Lortkipanidze}
А зачем ему?  @igg{fbicon.smile} 

\iusr{Oleg Divov}

Лично я выкрестился в пофигизм, чего всем вам советую. Очень комфортное
вероисповедание. Видал всех в гробу.

\begin{itemize} % {
\iusr{Иван Наумов}
\textbf{Oleg Divov} пока что можем себе это позволить

\iusr{Oleg Divov}
\textbf{Иван Наумов} Если я правильно понял, что ты имеешь в виду - да ну, фигня это все, Вань. Есть одна вещь, которую забывают люди, "обращающиеся к Богу", когда их клюнут в темечко жареные петухи старости и слабости. Господь не заключает сделок. Попытки продать Ему душу - это даже не смешно. Искренняя бескорыстная любовь к Господу штука довольно редкая. Я, допустим, так умею. Именно поэтому мне от Него ничего не надо  @igg{fbicon.smile} 

\iusr{Иван Наумов}
\textbf{Oleg Divov} а я не знаю, как именно ты меня понял. Всё ок, пока остаётся свобода вероисповедания или не-исповедания. Если же, гипотетически, и не обязательно именно вера, а даже просто определённое мировоззрение укрепится в социуме до уровня возможности ограничения этой свободы, то мало никому не покажется.

\iusr{Dmitry Serpent}
\textbf{Oleg Divov}
Я многое повидал и прошел. Но как был похуистом, так и остался.  @igg{fbicon.wink} 

\iusr{Роман Бобков}

\ifcmt
  ig https://scontent-frt3-2.xx.fbcdn.net/v/t39.30808-6/260846827_1383565095394030_5492832024061176324_n.jpg?_nc_cat=101&ccb=1-5&_nc_sid=dbeb18&_nc_ohc=beFGU3swsKEAX9PImAj&_nc_ht=scontent-frt3-2.xx&oh=0e2bbe3a841d2d826b1171b8bb55b2ae&oe=61A8CD12
  @width 0.4
\fi

\end{itemize} % }

\iusr{Иван Наумов}
И не факт, что её приход нас обрадует...

\iusr{Zahar Kariagin}

Кстати! Она же исподволь уже сформировалась, эта самая новая религия -
охлоцентрический инфантилизм. Соцсети - церковь ея, а Малахов предтеча ея, и
Цукерберг с Дуровым - пророки ея.

\iusr{Наташа Хоменко}

НАМ ОЧЕНЬ НУЖНА ПОМОЩЬ!

Общественная организация инвалидов регулярно проводит благотворительные акции
для людей с инвалидностью, которым всегда было не легко, а сейчас очень тяжело,
кроме всех жизненных проблем прибавился локдаун.

Сейчас мы готовимся к благотворительной акции к 3декабря, Международному дню
людей с инвалидностью.

У нас на учёте 1558 человек, это люди с ограничеными возможностями, которым
постоянно нужна поддержка и внимание.

Кроме взрослых, у нас 118 детей инвалидов детства.

\obeycr
Ещё есть детки в которых родители с инвалидностью.
Будем благодарны за любую помощь!
Нам нужны: продукты, моющие средства, деньги, постельные пренадлежности, памперсы, пелёнки, средства личной гигиены, пренадлежности для рукоделия
Наш адрес:
Хоменко Наталия Георгиевна
ул. Полевая, 16
с. Горбачёвка ( отделение связи с. Вознесенское )
Броварской район
Киевская область
07652
Карта ПриватБанка
5168 7574 2455 8262
Хоменко Наталия Георгиевна
Карта МоноБанка
4441 1144 1075 4806
NATALIIA KHOMENKO
Контактный телефон:
099 233 11 22
3аранее благодарны за понимание!
Всем Божьего Благословения!
\restorecr


\iusr{Алексей Евтушенко}
Не согласен, но спорить не стану. Пока  @igg{fbicon.smile} 

\iusr{Вадим Денисов}

Экуменизм есть всего лишь необходимость для создания мондиализма — планетарного
сверхгосударства с единым правительством и суперрелигией, и мы знаем, кто там
хочет заправлять. Давно придумано, но уже лет пятнадцать как заморожено,
поднадоели разговоры мечтателям, не находят пока англосаксы дорожных карт. Так
что ничего нового ты не предложил.

\iusr{Александр Захарин}

"В лисьих норах - атеистов нет!" (с) ген. Д. МакАртур. Религии не устарели, это
люди зажрались и утратили смысл. Настоящей войны давно не было.

\iusr{Филипп Руссо}
А зачем владеть миром? Чтобы что?

\begin{itemize} % {
\iusr{Александр Пащенко}
\textbf{Филипп Руссо} Надо  @igg{fbicon.smile} 

\ifcmt
  ig https://scontent-frx5-1.xx.fbcdn.net/v/t39.30808-6/260880473_932068507417160_7026098606642107085_n.jpg?_nc_cat=110&ccb=1-5&_nc_sid=dbeb18&_nc_ohc=ZFA3ve8Hmd4AX94PXY6&_nc_ht=scontent-frx5-1.xx&oh=e0ee741a269f38181727f1d98c64472c&oe=61A8DCA5
  @width 0.4
\fi

\end{itemize} % }

\iusr{Вера Камша}
Ну и хвала Дионису!

\iusr{Alex Dudchak}
Православие не устаревает.

\iusr{Алексей Мартынов}
Ханука, Саша! Все будет!


\end{itemize} % }
