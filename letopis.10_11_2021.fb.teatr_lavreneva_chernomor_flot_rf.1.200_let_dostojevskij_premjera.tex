% vim: keymap=russian-jcukenwin
%%beginhead 
 
%%file 10_11_2021.fb.teatr_lavreneva_chernomor_flot_rf.1.200_let_dostojevskij_premjera
%%parent 10_11_2021
 
%%url https://www.facebook.com/teatrlavreneva/posts/1144680842730073
 
%%author_id teatr_lavreneva_chernomor_flot_rf
%%date 
 
%%tags dostojevskii_fedor,literatura,premjera,rossia,rusmir,teatr
%%title ПОСВЯЩАЕТСЯ 200-летию ФЁДОРА МИХАЙЛОВИЧА ДОСТОЕВСКОГО
 
%%endhead 
 
\subsection{ПОСВЯЩАЕТСЯ 200-летию ФЁДОРА МИХАЙЛОВИЧА ДОСТОЕВСКОГО}
\label{sec:10_11_2021.fb.teatr_lavreneva_chernomor_flot_rf.1.200_let_dostojevskij_premjera}
 
\Purl{https://www.facebook.com/teatrlavreneva/posts/1144680842730073}
\ifcmt
 author_begin
   author_id teatr_lavreneva_chernomor_flot_rf
 author_end
\fi

ПОСВЯЩАЕТСЯ 200-летию ФЁДОРА МИХАЙЛОВИЧА ДОСТОЕВСКОГО

13 НОЯБРЯ/ ОТЕЦ МОЙ КАРАМАЗОВ/ ПРЕМЬЕРА

Впервые Фёдора Достоевского осудили в 28 лет - «за недонесение о
распространении преступного о религии и правительстве письма литератора
Белинского». Он часто потом был судим - молвою, ближними и дальними,
коллегами-литераторами и читателям.

\ifcmt
  tab_begin cols=3

     pic https://scontent-frt3-1.xx.fbcdn.net/v/t39.30808-6/254588107_1144680502730107_8599396576121526743_n.jpg?_nc_cat=102&ccb=1-5&_nc_sid=730e14&_nc_ohc=LJCcDZpuqF0AX_XHu0S&_nc_ht=scontent-frt3-1.xx&oh=a16783153d47e71b40a0aadda1101a1c&oe=619A330D

     pic https://scontent-frx5-1.xx.fbcdn.net/v/t39.30808-6/254580639_1144680559396768_7218147997477486842_n.jpg?_nc_cat=111&ccb=1-5&_nc_sid=730e14&_nc_ohc=rBxV-O9vWYIAX-iKYz1&tn=lCYVFeHcTIAFcAzi&_nc_ht=scontent-frx5-1.xx&oh=b5eb49b4f5691b8302096bcedd781795&oe=619AFB62

		 pic https://scontent-frx5-1.xx.fbcdn.net/v/t39.30808-6/253740017_1144680549396769_8802676377183100491_n.jpg?_nc_cat=110&ccb=1-5&_nc_sid=730e14&_nc_ohc=K2arqcX3bK0AX-7nodi&_nc_ht=scontent-frx5-1.xx&oh=04f5c7058a37f42d54661a1f169c9e92&oe=619B40D3

  tab_end
\fi

После его земного ухода герои писателя остались на суде - пожалуй, в мировой
литературе нет другого такого случая, когда вымышленные персонажи и их автор
подвергались такому скрупулёзному, серьёзному и суровому расследованию.
Наверняка в школьных историях многих из вас было место "Суду над
Раскольниковым". А ведь это что! Некоторых героев Достоевского в разное время
привлекали к вполне реальной ответственности за соучастии в преступлении!

\ifcmt
  tab_begin cols=3

     pic https://scontent-frt3-1.xx.fbcdn.net/v/t39.30808-6/255003276_1144680662730091_750926895898996831_n.jpg?_nc_cat=102&ccb=1-5&_nc_sid=730e14&_nc_ohc=-oSpR0DxKg8AX_CFjP_&_nc_ht=scontent-frt3-1.xx&oh=af189aa825d3aa00c073d3289b478ce2&oe=619B40C7

     pic https://scontent-frx5-1.xx.fbcdn.net/v/t39.30808-6/255493073_1144680652730092_6395335697826680055_n.jpg?_nc_cat=110&ccb=1-5&_nc_sid=730e14&_nc_ohc=t_UjTPyEZn4AX9x29Xf&_nc_ht=scontent-frx5-1.xx&oh=de71c5a5bd18fd1e5c2f3329d39b9dad&oe=619BBB73

		 pic https://scontent-frt3-2.xx.fbcdn.net/v/t39.30808-6/255557783_1144680759396748_5291300990035130986_n.jpg?_nc_cat=101&ccb=1-5&_nc_sid=730e14&_nc_ohc=cJ1Sz4ZlcaEAX-uVaHL&_nc_ht=scontent-frt3-2.xx&oh=38f933097c8d385b56be185a77b56cc8&oe=619BD2E7

  tab_end
\fi

13 ноября мы тоже пригласим вас в зал суда, которым станет на несколько часов
зал зрительный. Нам нужно будет решить много важных вопросов, которые уместно
обсуждать вероятно только перед лицом Суда - того, о котором всю свою жизнь
помнил Фёдор Михайлович Достоевский и о котором так ёмко и исчерпывающе
выразился: "Единый суд – моя совесть, то есть сидящий во мне Бог."

13 НОЯБРЯ в 18.30 - премьера мениппеи "ОТЕЦ МОЙ КАРАМАЗОВ".

Режиссёр, художник - Николай НЕЧАЕВ.

Художник по свету - Роман КЛОЧКОВ.

В ролях: заслуженная артистка Республики Крым Светлана АГАФОШИНА, заслуженный
артист Республики Крым Андрей ДЗУБАН, заслуженный артист Республики Крым Илья
ДОМБРОВСКИЙ, заслуженный артист Республики Крым Виталий МАКСИМЕНКО, артисты
Андрей БРАЖНИК, Игорь ЛУЧИХИН, Анастасия МОРОЗОВА, Игорь СМИРНОВ.

фото Екатерины Нечаевой

 @igg{fbicon.pushpin}  Если вы почему-либо споткнулись о слово "мениппея" или смущены им -
продержитесь, пожалуйста, до вечера: всё расскажем! 13, 20 и 27 ноября - ещё,
конечно, и покажем.

\#театрфлота \#театрЛавренёва \#достоевский200 \#культураСевастополя
