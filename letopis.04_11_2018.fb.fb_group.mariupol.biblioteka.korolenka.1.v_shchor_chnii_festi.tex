%%beginhead 
 
%%file 04_11_2018.fb.fb_group.mariupol.biblioteka.korolenka.1.v_shchor_chnii_festi
%%parent 04_11_2018
 
%%url https://www.facebook.com/groups/1476321979131170/posts/1943073929122637
 
%%author_id fb_group.mariupol.biblioteka.korolenka,lunina_tetjana.mariupol
%%date 04_11_2018
 
%%tags mariupol,festival,poezia,shkola,dity
%%title V щорічний фестиваль "Батьківщина у моєму серці" серед шкіл III освітнього округу
 
%%endhead 

\subsection{V щорічний фестиваль \enquote{Батьківщина у моєму серці} серед шкіл III освітнього округу}
\label{sec:04_11_2018.fb.fb_group.mariupol.biblioteka.korolenka.1.v_shchor_chnii_festi}
 
\Purl{https://www.facebook.com/groups/1476321979131170/posts/1943073929122637}
\ifcmt
 author_begin
   author_id fb_group.mariupol.biblioteka.korolenka,lunina_tetjana.mariupol
 author_end
\fi

Останній місяць мальовничої осені  розпочався для співробітників Центральної
бібліотеки ім. В. Г. Короленка з чарівних віршів. Адже представника бібліотеки
було запрошено на V щорічний фестиваль «Батьківщина у моєму серці» серед шкіл
III освітнього округу, який пройшов на базі школи № 18, з почесною місією члена
комісії журі.  

У конкурсі читців брали участь учні з різних шкіл міста, вкотре доводячи, що
Маріуполь – це місто талановитої молоді. Усі вірші лунали українською мовою, та
торкалися різноманітних тем. Хлопці та дівчата обрали для свого виступу твори
видатних українських письменників, присвячених нашій рідній землі, сьогоденню і
звичайно  незмінній темі, яка в усі часи надихала митців на прекрасне –
коханню. Особливо приємно було почути не тільки відомих класиків української
літератури, але  й сучасних авторів. Так наприклад дует із ЗЗСО №67 Лотвінов О.
та Лавриненко В. дуже зворушливо прочитали вірш С. Жадана «Птах уночі
забивається до кімнати».

Членам комісії журі Коляде Н.Д (голова громадської організації \enquote{Ми поруч}),
Керімовій І. Т. (поет Маріупольського літературного об`єднання \enquote{Світанок}),
Луніной Т. С. (представник Центральної бібліотеки ім. Короленка), Тельбізову А.М.
(режисер народного театру \enquote{Театроманія})  було дуже складно обрати серед
учасників найкращого. Тільки після довгого обговорення нагороди знайшли свого
власника. 

Отже поспішаю поділитися з Вами результатами конкурсу:

I місце отримав учень ЗЗСО №25 Вадим Ялі 

II місце отримав учень ЗЗСО № 52 Віталій Ходукін 

III місце посіли учні ЗЗСО № 18 Тимофій Садирін і Софія Бабинець.

В завершення хочеться ще раз привітати усіх учасників конкурсу, та побажати
нових звершень та натхнення на наступні перемоги. А організаторам фестивалю
виказати вдячність за розкриття та розвиток талантів міста.
