% vim: keymap=russian-jcukenwin
%%beginhead 
 
%%file 11_11_2021.fb.fb_group.respublika_lnr.1.nasha_obschaja_bolj_ostanki
%%parent 11_11_2021
 
%%url https://www.facebook.com/groups/respublikalnr/posts/917484465554012/
 
%%author_id fb_group.respublika_lnr,zimina_olesja
%%date 
 
%%tags agressia,donbass,ukraina,vojna,zahoronenie
%%title НАША ОБЩАЯ БОЛЬ. Рабочая группа извлекла останки 292 жертв Украины
 
%%endhead 
 
\subsection{НАША ОБЩАЯ БОЛЬ. Рабочая группа извлекла останки 292 жертв Украины}
\label{sec:11_11_2021.fb.fb_group.respublika_lnr.1.nasha_obschaja_bolj_ostanki}
 
\Purl{https://www.facebook.com/groups/respublikalnr/posts/917484465554012/}
\ifcmt
 author_begin
   author_id fb_group.respublika_lnr,zimina_olesja
 author_end
\fi

ГАЗЕТА "РЕСПУБЛИКА" (№45, 2021г).

НАША ОБЩАЯ БОЛЬ. Рабочая группа извлекла останки 292 жертв Украины

Завершились работы по извлечению останков тел из спонтанного массового
захоронения на окраине Луганска. Полевой командой Межведомственной рабочей
группы по розыску захоронений жертв украинской агрессии, их идентификации и
увековечению памяти из двух траншей были подняты 165 тел.

\ifcmt
  ig https://scontent-frt3-1.xx.fbcdn.net/v/t39.30808-6/256402105_257257646454342_5432738145783172016_n.jpg?_nc_cat=102&ccb=1-5&_nc_sid=825194&_nc_ohc=rYUrElawdK0AX8_Zi6L&_nc_ht=scontent-frt3-1.xx&oh=5f458d6a23059e34d6ab100f16f13e97&oe=6192C020
  @width 0.4
  %@wrap \parpic[r]
  @wrap \InsertBoxR{0}
\fi

Работы на месте захоронения велись с 8 октября по 5 ноября. В полевом лагере
судмедэкспертами проводились все необходимые процедуры, а также отбор
биоматериала для проведения генетических исследований. 

– Первичный осмотр остатков одежды, характер положения тел позволяет судить о
том, что это мирные граждане, погибшие летом – осенью 2014 года в результате
украинской агрессии, – подчеркнула руководитель Межведомственной рабочей
группы, Первый замминистра иностранных дел Анна Сорока. – Вся информация,
полученная за три месяца, процессуально зафиксирована и станет частью
уголовного дела, находящегося в производстве Министерства государственной
безопасности по статье «Применение запрещенных средств и методов ведения
войны». Также на основании данной информации составятся материалы, которые
будут направлены в международный уголовный суд. Скорее всего, это будут
коллективные обращения от территориальных общин, где были вскрыты захоронения. 

В связи с погодными условиями полевые работы останавливаются, но в холодное
время года мероприятий предстоит не меньше.

– Впереди – серьезная скрупулезная работа: через оперативно-розыскные
мероприятия будут устанавливаться родственники и личности похороненных, –
отметила Анна Сорока. – Кроме того, будет проводиться анализ как проведенной
работы, так и информации, поступавшей в течение трех месяцев на нашу горячую
линию.

Всего за три месяца луганской группой были подняты останки 292 тел из массовых
и одиночных стихийных захоронений в Славяносербском, Краснодонском и
Антрацитовском районах, а также городе Первомайске. В Луганске было вскрыто
самое масштабное в Республике стихийное захоронение, которое было организовано
по причине массированных неизбирательных бомбежек города, в том числе и
городских кладбищ. На месте захоронения в 2016 году был установлен крест, а
позже – часовня. В будущем планируется возведение на этом месте масштабного
мемориального комплекса, который увековечит память всех жертв украинского
кровавого режима.

Сообщить о случаях одиночного или массового захоронения жертв украинской
агрессии или о пропавшем без вести родственнике (в случае пропажи в период с
апреля 2014-го по настоящее время в зоне активных боевых действий в Донбассе)
можно по телефону горячей линии межведомственной рабочей группы: 

+38 (072) 216-80-30, или в письме на электронную почту wg-img-lpr@ya.ru.

Оксана ЧИГРИНА

Фото автора

ГАЗЕТА "РЕСПУБЛИКА" (№45, 2021г).

\#газета \#республика \#эксгумация\_останков
\_жертв\_украинской\_агрессии
\#Межведомственная\_группа\_по
\_розыску\_захоронений
