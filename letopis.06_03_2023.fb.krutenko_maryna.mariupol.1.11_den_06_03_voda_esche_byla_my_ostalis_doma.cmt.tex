% vim: keymap=russian-jcukenwin
%%beginhead 
 
%%file 06_03_2023.fb.krutenko_maryna.mariupol.1.11_den_06_03_voda_esche_byla_my_ostalis_doma.cmt
%%parent 06_03_2023.fb.krutenko_maryna.mariupol.1.11_den_06_03_voda_esche_byla_my_ostalis_doma
 
%%url 
 
%%author_id 
%%date 
 
%%tags 
%%title 
 
%%endhead 

\qqSecCmt

\iusr{Irina Paul}

Я помню этот день, тогда десятки машин приехали к спорткомплексу "Ильичевцу", и
многие наши тоже засобирались, услышав про эвакуацию, но напрасно... позже мы в
мегафоны, что эвакуация отменяется и просят разойтись по домам...

\iusr{Maryna Krutenko}

Мы пытались ещё 10.03 выехать. Тогда очень много машин ждали на разрешение, чтоб выехать.

\begin{itemize} % {
\iusr{Irina Paul}
\textbf{Maryna Krutenko} да, я помню, многие пытались и вернулись
\end{itemize} % }

\iusr{Irina Matushina}

а мы в этот день в 5 утра выезжали на границу в Краковец.

\iusr{Таша Мелина}

Мы тоже пытались. Пока еще был бензин в баке и была машина... И каждый раз
такое разочаровние и бессилие!

\iusr{Юлия Дворниченко-Развозова}

😞

\iusr{Svetlana Didenko}

Помню этот день, так как живу возле "Ильичевца". Люди шли к спорткомплексу с
детьми, с вещами, животными. Это был просто муравейник. Потом шли обратно.

\iusr{Валентина Самарина}

Я помню как наши соседи усаживали соседку она слепая и без ноги несли с 3 го
этажа в машину усадили и поехали колонной мы молились что бы они доехали ..к
вечеру они все вернулись
