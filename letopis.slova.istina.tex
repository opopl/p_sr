% vim: keymap=russian-jcukenwin
%%beginhead 
 
%%file slova.istina
%%parent slova
 
%%url 
 
%%author_id 
%%date 
 
%%tags 
%%title 
 
%%endhead 
\chapter{Истина}
\label{sec:slova.istina}


%%%cit
%%%cit_head
%%%cit_pic
%%%cit_text
А нематематики этого не знают. Они не знают разницы между бесконечно малым и
конечно великим. Они закон, правильный для искры костра, прилагают к
огнедышащей горе. Правило, законное для прямой доски в полу их комнатушки,
прилагают к огромному кругу земли, orbis terrarum. Они мгновенное,
преходящее, собственное свое чувство принимают за «естественное право» народов,
человечества, вселенной...  Можно знать \emph{«истины»} разных величин, но не
дано знать \emph{«Истину»}. Маленькие \emph{истины} могут знать все люди и даже
животные. «Если у меня украдут, это плохо» — это \emph{истина}, которую знает
всякий дикарь. Но он не может сообразить, что и всякий другой дикарь так
думает. И поэтому он, не позволяя красть у себя, сам крадет вовсю. Дикарь не
может проинтегрировать своего маленького дифференциала, своей маленькой истины.
Эту работу за него делают великие люди, гении, пророки, учителя. Они делают
скачок интуиции и из маленькой \emph{истины} каждого дикаря «плохо, если у меня
украдут» выходит великая заповедь Моисея: VIII. — Не укради. (То есть не воруй
сам, если не хочешь, чтобы тебя обкрадывали)
%%%cit_comment
%%%cit_title
\citTitle{Три Столицы}, В. В. Шульгин
%%%endcit

