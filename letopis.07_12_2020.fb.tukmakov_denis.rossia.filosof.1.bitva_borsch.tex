% vim: keymap=russian-jcukenwin
%%beginhead 
 
%%file 07_12_2020.fb.tukmakov_denis.rossia.filosof.1.bitva_borsch
%%parent 07_12_2020
 
%%url https://www.facebook.com/tukmakov/posts/3729748000422391
 
%%author Тукмаков, Денис (Россия, Философ)
%%author_id tukmakov_denis.rossia.filosof
%%author_url 
 
%%tags bitva,borsch,filosofia,kuhnja,kulinaria,rossia,tradicii,ukraina
%%title В битве под Борщом украинец дерётся изо всех сил
 
%%endhead 
 
\subsection{В битве под Борщом украинец дерётся изо всех сил}
\label{sec:07_12_2020.fb.tukmakov_denis.rossia.filosof.1.bitva_borsch}
 
\Purl{https://www.facebook.com/tukmakov/posts/3729748000422391}
\ifcmt
 author_begin
   author_id tukmakov_denis.rossia.filosof
 author_end
\fi

В битве под Борщом украинец дерётся изо всех сил: блещет остроумием, сыплет
шутками, всю душу вкладывает, лишь бы побольнее уязвить оппонента.

Особенно ему удаётся миллион раз повторённая острота "Щи — это то, что осталось
в вымытой после борща кастрюле". Но и другие репризы, про мокшу и лапоть, тоже
идут в ход.

Русский же на битву не явился. Он смотрит на эти упражнения соседа скорее с
долей брезгливости: "Ишь взбеленился". Для него не только спор о борще
представляет в лучшем случае этнографический интерес, но и вся эта тема с
"мокшей, мерей, чудью" кажется полной ерундой. 

Русского, посреди его большой работы, в окружении громадных строек, плавучих
АЭС и вакцин, просто не может задеть это ковыряние в древних племенах,
свойственное скорее архаичным, затёртым между соседями народцам. Которым, за
неимением достойных поводов, нестерпимо важно знать, что именно они первыми
догадались сварить свёклу с мясом.

Но что, наверное, самое нестерпимое: для русского не только борщ — русский, но
и сам украинец — тоже, как ни крути, русский. Только, как бы сказать,
некормленный, что ли. Как и Киев, как и вся Украина — по недоразумению отпавшая
от Русского мира земля, где говорят на наших диалектах, ходят в те же церкви и
ставят на стол привычные с детства блюда.

Поэтому русский смотрит на эту возню даже с некоторым сочувствием: "Вот же
угораздило дурного". В какой-то момент украинец заметит это, сникнет,
насупится. Тогда русский придвинет ему похлёбку: "Ешь давай, не ерепенься". И
тот, поломавшись ещё немножко для вида, дохлебает всё — до последней ложки.

\ii{07_12_2020.fb.tukmakov_denis.rossia.filosof.1.bitva_borsch.cmt}
