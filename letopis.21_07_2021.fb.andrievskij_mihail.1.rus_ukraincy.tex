% vim: keymap=russian-jcukenwin
%%beginhead 
 
%%file 21_07_2021.fb.andrievskij_mihail.1.rus_ukraincy
%%parent 21_07_2021
 
%%url https://www.facebook.com/groups/367613903441695/posts/1885791541623916/
 
%%author 
%%author_id andrievskij_mihail
%%author_url 
 
%%tags identichnost',jazyk,mova,obschestvo,ukraina
%%title ПРО РОСІЙСЬКОМОВНИХ УКРАЇНЦІВ (допис Elizabeth Bielska)
 
%%endhead 
 
\subsection{ПРО РОСІЙСЬКОМОВНИХ УКРАЇНЦІВ (допис Elizabeth Bielska)}
\label{sec:21_07_2021.fb.andrievskij_mihail.1.rus_ukraincy}
 
\Purl{https://www.facebook.com/groups/367613903441695/posts/1885791541623916/}
\ifcmt
 author_begin
   author_id andrievskij_mihail
 author_end
\fi

Хто ви за національністю? Яке у вас громадянство? Яка ваша рідна мова? А якою
мовою ви говорите і чому?

Багато хто називає себе російськомовними українцями. Тобто усвідомлює себе
покручем, гібридом. 

Якби такі існували, то логічно було б зустрічати в Росії здебільшого
українськомовних росіян, в Німеччині – івритомовних німців, а у Франції часто
чути англійську з вуст корінних парижан?

\ifcmt
  pic https://scontent-lga3-1.xx.fbcdn.net/v/t1.6435-9/217951996_4178088545559946_2059304590004694254_n.jpg?_nc_cat=103&ccb=1-3&_nc_sid=730e14&_nc_ohc=VN-KoV7lu3AAX-GLa31&_nc_ht=scontent-lga3-1.xx&oh=3c6db4f9365cec56241e868354e68938&oe=611FF532
  width 0.4
\fi

Словосполучення «російськомовний українець» викликає у мене таку асоціацію:
українська мова в Україні - це запашне яблучко, а російська – шарудлива
цибулина, від якої очі неприємно сльозяться. Якщо вони лежатимуть поряд, то чи
стане яблуко цибулевим або цибуля яблуневою? – Точно ні. Але що напевне
відбудеться – яблуко набереться цибулевих випарів і смердітиме. Й стане
непридатним до споживання.

Отже, висновок. Російськомовних українців не існує. А є зросійщені чи
русифіковані. І якщо яблуку вже ніщо не допоможе, то ви можете досить швидко,
майже безболісно і назавжди виправити цю ситуацію. Для цього маєте себе
правильно ідентифікувати, залежно від того, ким ви є.

Наприклад:

1. Росія 300 років нищила українців, забороняла нашу мову, примусовою
русифікувала, стирала національну пам’ять. Я досі говорив російською лише тому,
що нас зробили рабами. І моя генетична пам’ять містить у собі страх перед
українською, бо за неї моїх бабусь і дідусів вбивали. Сьогодні я це усвідомив.
І більше не буду жертвою. Це моя земля. Моя Україна. Я українець і пишаюся цим.
Тому повертаю собі рідну мову, і більше нікому не дозволю її у мене забрати.

2. Я іноземець. У мене є власна мова. Але оскільки я живу в Україні, то
вивчатиму українську, а не російську. Бо поважаю цю країну, її культуру та
традиції. А українці й росіяни точно не один народ. Бо якби було так, Росія б
не окупувала українські території та не вбила б понад 15 000 українців на
Донбасі. Я українськомовний ... (вірменин, болгарин, кримський татарин, єврей,
кореєць).

Це мій робочий уривок із нашої спільної з Антіном Мухарським книжки "ЯК ПЕРЕЙТИ
НА УКРАЇНСЬКУ", яка побачить світ цієї осені.

Підтримати видання можна у два способи:

\begin{itemize}
\item 1. Передзамовити книжку за 300 грн зараз, щоб отримати з автографом серед перших: \url{http://www.ukrideabook.com.ua/}
\item 2. Підтримати видання сумою від 1000 грн, тоді ваше ім'я буде вказане на
спеціальній сторінці подяк у книзі, яку отримаєте, щойно вона вийде друком,
разом з автографом, дипломом-подякою та браслетом "Шляхетні люди говорять
українською".
\end{itemize}

Більше про це тут: \url{http://www.ukrideabook.com.ua/pidtrimka}

PS Поширення допису теж є великою допомогою!

\ii{21_07_2021.fb.andrievskij_mihail.1.rus_ukraincy.cmt}
