% vim: keymap=russian-jcukenwin
%%beginhead 
 
%%file 22_12_2021.fb.fb_group.story_kiev_ua.1.kiev_bani_rejting
%%parent 22_12_2021
 
%%url https://www.facebook.com/groups/story.kiev.ua/posts/1824360227760770
 
%%author_id fb_group.story_kiev_ua,bubnov_jurij
%%date 
 
%%tags banja,gorod,kiev,rejting
%%title Рейтинг киевских бань: "клубничка", с риском и в мраморе
 
%%endhead 
 
\subsection{Рейтинг киевских бань: \enquote{клубничка}, с риском и в мраморе}
\label{sec:22_12_2021.fb.fb_group.story_kiev_ua.1.kiev_bani_rejting}
 
\Purl{https://www.facebook.com/groups/story.kiev.ua/posts/1824360227760770}
\ifcmt
 author_begin
   author_id fb_group.story_kiev_ua,bubnov_jurij
 author_end
\fi

Рейтинг киевских бань: \enquote{клубничка}, с риском и в мраморе.

С наступлением холодов логично прогуляться по тем местам, где можно не только
прикоснуться к истории города, но заодно и погреться, а значит, самое время
заглянуть в городские бани. Вплоть до 60-х годов прошлого века, когда киевляне
начали обзаводиться квартирами с ванными комнатами, и потребность в
общественных банях отпала, здесь не только мылись, но также стриглись,
стирались, пили пиво, ели раков, совершали покупки, обсуждали новости. В XIX
веке в столичных парилках стирали, ели и делали покупки. В начале ХХ века, если
верить статистическим сведениям, каждый киевлянин посещал баню минимум пять раз
в год. Кстати, любопытная деталь: несмотря на гостеприимность киевлян, приезжих
в бани не пускали: здесь мылись только горожане.

\begin{multicols}{2} % {
\setlength{\parindent}{0pt}
\ii{22_12_2021.fb.fb_group.story_kiev_ua.1.kiev_bani_rejting.pic.1}
\ii{22_12_2021.fb.fb_group.story_kiev_ua.1.kiev_bani_rejting.pic.1.cmt}

\ii{22_12_2021.fb.fb_group.story_kiev_ua.1.kiev_bani_rejting.pic.2}
\ii{22_12_2021.fb.fb_group.story_kiev_ua.1.kiev_bani_rejting.pic.2.cmt}

\ii{22_12_2021.fb.fb_group.story_kiev_ua.1.kiev_bani_rejting.pic.3}
\ii{22_12_2021.fb.fb_group.story_kiev_ua.1.kiev_bani_rejting.pic.3.cmt}
\end{multicols} % }

ДОСТУПНАЯ РОСКОШЬ

Баня Михаила Бугаева (ул. Андреевская, 4), построенная в 1887 году, несмотря на
мраморный антураж, газовое освещение и такую барскую роскошь, как душевые, была
сравнительно недорогой. Вход в общий зал стоил от 6 (для студентов) до 25 коп.,
отдельные \enquote{нумера} — от 50 коп. до 2 руб. (Двухрублевые \enquote{нумера} были
оборудованы мраморными ванными.) Вода в баню подавалась по специально
построенному водопроводу из Днепра и тщательно фильтровалась. Ну и, конечно же,
здесь продавали пиво и раки.

КИНО И БАНИ

Ярославов Вал, 40

Бани Каплера на Ярославовом Валу, 40, носили гордое название
\enquote{Московские}.  Назвали их так не без влияния очерков Владимира
Гиляровского, в которых знаменитый журналист воспевал бани Москвы. Их построил
в 1900 году купец первой гильдии Яков Каплер, известный киевский домовладелец.
Любопытное совпадение: сын Якова Каплера, Алексей, стал именитым советским
кинорежиссером, а в наше время в банях, построенных его отцом, находится
\enquote{киношный} институт им.  Карпенко-Карого.

МЫТЬЕ С \enquote{КЛУБНИЧКОЙ}

Малая Житомирская, 3а

Центральные бани на Малой Житомирской, 3-А, — единственные, которые дожили до
наших дней. Они были построены в 1897 году помещиком Гаврилом Позняковым и были
известны также в качестве Купеческих бань. Это была первая столичная баня,
построенная почти в центре города — спасибо водопроводу. (До того подобные
заведения строились вблизи Днепра.) Пикантная особенность: здесь располагались
не только банные \enquote{нумера}, но также гостиничные, куда водили девушек легкого
поведения.

БАНЯ-ДОЛГОЖИТЕЛЬ

Красноармейская, 66

Троицкие бани на Большой Васильковской, 66, также располагались рядом с рынком.
(На Троицкой площади, рядом с нынешним НСК \enquote{Олимпийский}, с 60-х по
90-е годы XIX века размещался Конный базар, где проводились ярмарки лошадей.)
Но воришек, в отличие от бубновских бань, отсюда отваживали, потому горожане
ходили сюда с большим удовольствием. Хотя условия здесь были самые что ни на
есть спартанские — даже пива не продавали. Кстати, продержались Троицкие бани
вплоть до 70-х годов прошлого века.

ПОМЫТЬСЯ С РИСКОМ

Жилянская, 146

Баня Бубнова, расположенная в самом конце Жилянской улицы (сейчас на ее месте
стоит дом №146), считалась самой демократичной баней Киева. Вход в общее
отделение стоил 7 коп., отдельный \enquote{нумер} — всего 75 коп. Дешевизна объяснялась
просто — по соседству находился Евбаз, один из крупнейших киевских рынков, и к
Бубнову ходили мыться, преимущественно, рыночные торговцы и... воришки, которые
активно промышляли в бане своим ремеслом. Горожане старались сюда не ходить.

У БРИКА: ДЕШЕВО И СЕРДИТО

Баня госпожи Ивановой находилась на ул. Бульварно-Кудрявской, 4. До наших дней
почти не дошло никаких сведений ни о госпоже Ивановой, ни о ее бане. Кто такая
госпожа Иванова — тоже загадка: баню построил в 1874 (по другим сведениям —
1875) году купец второй гильдии по фамилии Брик. Плата за вход составляла от 10
до 40 коп., а за веники, простыни и полотенца надо было платить отдельно.

\enquote{КАРАВАЕВСКИЕ}: ПАР И МРАМОР

Пушкинская 42/4

Там, где сейчас располагается Бизнес Центр Киев-Донбасс, раньше находились бани
\enquote{Караваевские}. Знаменитые киевские бани на ул. Ново-Елизаветинской открыл в
1877 году купец 1 гильдии Фридрих Михельсон. Это было заведение, как сказали бы
теперь, премиум-класса: с мраморным бассейном для плавания и мраморными
ваннами. Вход в общий зал стоил 40 копеек (для гимназистов и студентов — 25
коп.), одиночная парная — полтора рубля, а семейный \enquote{нумер} — целых 3. Для
сравнения: курица стоила рубль, а бутылка пива — 12 копеек. \enquote{Караваевские} бани
были самым настоящим развлекательным центром — с буфетами, пивными, магазинами
и парикмахерскими.

А. Росовецкий

PS. Ставлю в известность читателей: Несмотря на то, что я и мои предки являемся
коренными жителями Евбаза, к упомянутой в этой публикации \enquote{бане Бубнова},
никакого отношения не имеем. А жаль....(особенно себя). Был бы знаменитым.))

Фотографии. 

1. 1980-е. Вход в \enquote{Караваевские} бани. 

2. 1956 год. Баня (бывшая \enquote{бане Бубнова}) на ул. Жилянская, 146. 

3. Центральные бани на Малой Житомирской, 3-А.

\ii{22_12_2021.fb.fb_group.story_kiev_ua.1.kiev_bani_rejting.cmt}
