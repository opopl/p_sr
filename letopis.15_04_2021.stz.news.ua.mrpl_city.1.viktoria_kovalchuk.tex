% vim: keymap=russian-jcukenwin
%%beginhead 
 
%%file 15_04_2021.stz.news.ua.mrpl_city.1.viktoria_kovalchuk
%%parent 15_04_2021
 
%%url https://mrpl.city/blogs/view/svitlij-pamyati-viktorii-volodimirivni-kovalchuk
 
%%author_id demidko_olga.mariupol,news.ua.mrpl_city
%%date 
 
%%tags 
%%title Світлій пам'яті Вікторії Володимирівні Ковальчук
 
%%endhead 
 
\subsection{Світлій пам'яті Вікторії Володимирівні Ковальчук}
\label{sec:15_04_2021.stz.news.ua.mrpl_city.1.viktoria_kovalchuk}
 
\Purl{https://mrpl.city/blogs/view/svitlij-pamyati-viktorii-volodimirivni-kovalchuk}
\ifcmt
 author_begin
   author_id demidko_olga.mariupol,news.ua.mrpl_city
 author_end
\fi

\ii{15_04_2021.stz.news.ua.mrpl_city.1.viktoria_kovalchuk.pic.1}

4 квітня Україна втратила ще один унікальний і неповторний талант. Від
нещадного ковіду померла українська художниця-гра\hyp{}фік, ілюстраторка, дизайнерка
та літераторка і неймовірно світла людина Вікторія Володимирівна Ковальчук. За
своє яскраве і цікаве життя художниця оформила близько 200 книжок різних
жанрів, серед них понад 30 книг для дітей. Вікторія Володимирівна – лауреатка
державної премії імені Лесі Українки та літературної премії імені Олени Пчілки.
Народилася в Ковелі Волинської області, жила і працювала у Львові. Розробила
оригінальне оформлення українського \emph{\enquote{Букваря}}, який отримав диплом першого
ступеня на всеукраїнському конкурсі \enquote{Мистецтво книги} у 2000 році і був
названий найкращою книгою року. Буквар витримав дев'ять перевидань загальним
накладом понад 200 тисяч примірників.

Чотири роки художниця працювала над книгою, присвяченою народному вбранню
\emph{\enquote{Український стрій}}. Збирала польовий матеріал по всій країні, розробила
оригінальну концепцію для науко\hyp{}во-популярних ілюстрацій. Книга вийшла в 2000
ексклюзивним тиражем. У 2001 була визнана найкращою книгою року. Окрім того,
вона є авторкою книжок \enquote{Про фею Дорофею} (Маленькі оповідки для маленьких
дітей), \enquote{Подорож з Нічним постояльцем} (Бучач очима Шмуеля Агнона), \enquote{Гусінь
Мотя, війна і любов – Правдива казочка}, \enquote{Казка про ворону, яка хотіла лишитися
чорною}. Водночас пані Вікторія була унікальним фотографом, дизайнеркою
комп'ютерної графіки, писала серйозні статті на роботи видатних художників та
письменників.

\ii{15_04_2021.stz.news.ua.mrpl_city.1.viktoria_kovalchuk.pic.2}

Впевнена, що в усіх містах, де проводилися виставки чи творчі зустрічі з
Вікторією Володимирівною вона залишала яскравий слід у серцях містян. Маріуполь
не став виключенням. 6 грудня 2016 року художниця завітала до нашого міста, що
стало однією з найяскравіших подій в житті маріупольців. Цей приїзд був
організований в рамках проєкту \enquote{Схід та Захід разом} Маріупольським Управлінням
культури та художнім відділенням школи мистецтв з метою пропаганди творчої
національної культурної спадщини високого ґатунку. Найбільша заслуга в
організації всіх заходів і довгому перебуванні художниці в Маріуполі належить
\emph{\textbf{Любові Василівні Макаренко}} – викладачці Маріупольської школи мистецтв. Саме
вона завдяки широкій та різноманітній програмі заходів відкрила для Маріуполя
таку яскраву і багатогранну постать. Розпочалося знайомство Вікторії Ковальчук
з містом з моєї екскурсії. Людини з такою потужною позитивною енергетикою я
раніше ніколи не зустрічала. Наша екскурсія вийшла дуже душевною і для мене
незабутньою. Завершили нашу прогулянку біля моря, яке того дня було
неймовірним. Особисто мене Вікторія Володимирівна навчила більше уваги звертати
на кожну деталь, адже, розповідаючи про відому будівлю, можна пропустити
непримітний будинок, який теж зберігає свої загадки і таємниці.

\ii{15_04_2021.stz.news.ua.mrpl_city.1.viktoria_kovalchuk.pic.3}

9 грудня 2016 року у Художньому музеї ім. А. І. Куїнджі відбулося відкриття
персональної виставки творів Вікторії Володимирівни на тему \enquote{Зустріч}. Ця
виставка стала справжнім відкриттям для багатьох містян. Стиль художниці багато
спеціалістів називають \enquote{міфічним реалізмом}. І дійсно за кожною ілюстрацією
вибудована ціла система цінностей. Насправді ілюстрації пані Вікторії можна
розглядати годинами і щоразу знаходити для себе нові символи. Малюнки настільки
детальні та  продумані, що сюжет розкриваються на різних рівнях розуміння.
Оскільки Вікторія Володимирівна затрималася в Маріуполі на тиждень, вона змогла
стати рідною для багатьох творчих маріупольців. Її майстер-клас \emph{\enquote{Ілюструємо
літературні твори}} надихнув багатьох учнів художнього відділення школи мистецтв
та недільної Грецької школи. До речі, Вікторія Володимирівна намагалася й
надалі підтримувати учнів та слідкувати за їхніми успіхами. До Дня Святого
Миколая чарівна пані Вікторія познайомила наймолодших маріупольців зі своїми
творами в межах дитячого клубу \enquote{Маріуполь читає дітям}. Наймолодші містяни
порівнювали її зі справжньою феєю. А ще була проведена низка круглих столів,
творчих зустрічей з маріупольськими митцями пензля, зі студентами-журналіста\hyp{}ми
МДУ, з представниками ЗМІ; літературних виставок у бібліотеках. При цьому у
Центральній міській бібліотеці ім. В. Короленка відбулася зустріч з
маріупольськими поетами, де пані Вікторія прочитала свої вірші. У бібліотеці
ім. М. Свєтлова провели зустріч з підлітками, а в бібліотеці ім. М. Горького –
з дітьми. На всіх заходах Вікторія Володимирівна залишалася привітною і уважною
до кожного. Цікаво, що за ініціативою Федерації грецьких товариств України
вперше була опублікована книга письменниці \enquote{Гусінь Мотя, війна і любов –
Правдива казочка}. Зі свого боку Вікторія Володимирівна подарувала Художньому
музею ім. А. І. Куїнджі 30 своїх робіт – ілюстрацій до збірки віршів для дітей
\enquote{Бузиновий цар} Ліни Костенко. Як вона сама зізнавалася, Маріуполь став для неї
дуже близьким і рідним. Тому художниця підтримувала з багатьма містянами
листування, надсилала оригінали і репродукції своїх робіт, допомагала і
матеріально тим маріупольцям, хто хворів чи був у скрутному становищі.
Найбільша дружба у неї склалася з Любов'ю Макаренко, яка дуже тепло згадує пані
Вікторію, її світлий будинок у Львові, ошатну майстерню, де кожна річ мала свою
історію. Коли Вікторія Ковальчук приїхала до Маріуполя в 2018 році, містяни її
зустрічали вже як дуже близьку і рідну людину.

\ii{15_04_2021.stz.news.ua.mrpl_city.1.viktoria_kovalchuk.pic.4}

Влітку 2021 року мала відбутися в Маріуполі ще одна виставка художниці... Дуже
сподіваюся, що до нас приїде син Вікторії Володимирівни з її роботами  і
книгами. Маріупольці пам'ятатимуть пані Вікторію чарівною та життєрадісною,
чуйною і глибокою людиною. Творча та спрагла до нових справ вона й інших
спонукала до пошуку й реалізації власного потенціалу. Світла пам'ять яскравій,
небайдужій і світлій художниці, чий талант і доброта  житимуть в серцях людей,
яким пощастило її пізнати.

\textbf{\emph{Світлини з Особистого архіву Любові Макаренко.}}
