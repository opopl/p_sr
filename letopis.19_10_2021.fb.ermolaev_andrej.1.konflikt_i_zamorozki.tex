% vim: keymap=russian-jcukenwin
%%beginhead 
 
%%file 19_10_2021.fb.ermolaev_andrej.1.konflikt_i_zamorozki
%%parent 19_10_2021
 
%%url https://www.facebook.com/yermolaievandrey/posts/943543522924669
 
%%author_id ermolaev_andrej
%%date 
 
%%tags donbass,konflikt,obschestvo,ukraina,vojna,zamorozka
%%title Конфликт и "заморозки"
 
%%endhead 
 
\subsection{Конфликт и \enquote{заморозки}}
\label{sec:19_10_2021.fb.ermolaev_andrej.1.konflikt_i_zamorozki}
 
\Purl{https://www.facebook.com/yermolaievandrey/posts/943543522924669}
\ifcmt
 author_begin
   author_id ermolaev_andrej
 author_end
\fi

Конфликт и "заморозки"

"Замороженный конфликт" - это не просто временный status quo, когда не стреляют
и не посылают ДРГ. 

"Замороженный конфликт" должен означать и одновременную "разморозку"
миротворческого процесса в самом широком спектре: "большая дипломатия",
консультационная работа, разворачивание "гражданской дипломатии";
законодательное обеспечение уже достигнутых договоренностей, расширение свобод
перемещения людей, товаров и услуг; экономическое де-блокирование и поэтапное
восстановление торговых и, в оговоренных пределах, производственных связей,
полноценная реализация социальных обязательств государства в отношении своих
граждан, проживающих в регионе. 

"Заморозка" конфликта должна обеспечить создание базовых условий полного выходе
из конфликта, обеспечения устойчивого мира, компромиссного
политико-административного обустройства региона и  создание условий для
формирования общего гражданско-правового, экономического, гуманитарного
пространства и пространства безопасности.

Без полноценной миротворческой политики, кратковременные "заморозки"
оборачиваются новыми претензиями, спонтанными конфликтами по "линии
соприкосновения", и в итоге - очередным срывом "режимов прекращения огня". 

Судьбу миллионов решают несколько десятков политиков, силовиков и наемных
"артистов разговорного жанра".  Остальные миллионы притихли, замерев у экранов
мыльных политических ток-шоу...

А размороженный миротворческий процесс - это десятки и сотни новых контактов,
сотни людей, вовлеченных в реальный диалог и в разные форматы взаимодействия.
Это контакты на уровне самоуправления с двух сторон, бизнеса, гражданских
структур. Это ДЕЯТЕЛЬНОЕ миро-творчество, на основе взаимо-действия. И - что
важно - проснувшееся гражданское общество, которое сейчас рассечено и отсечено
от диалога о будущем, которому заткнули рот и накрыли прессом  из спецслужб и
армады пропагандистов. 

Может, именно поэтому конфликт в хроническом "подогретом" состоянии?

\ii{19_10_2021.fb.ermolaev_andrej.1.konflikt_i_zamorozki.cmt}
