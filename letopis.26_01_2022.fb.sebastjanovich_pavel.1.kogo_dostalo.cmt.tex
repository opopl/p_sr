% vim: keymap=russian-jcukenwin
%%beginhead 
 
%%file 26_01_2022.fb.sebastjanovich_pavel.1.kogo_dostalo.cmt
%%parent 26_01_2022.fb.sebastjanovich_pavel.1.kogo_dostalo
 
%%url 
 
%%author_id 
%%date 
 
%%tags 
%%title 
 
%%endhead 
\zzSecCmt

\begin{itemize} % {
\iusr{Леонид Шехтер}
В добрый путь!

\iusr{Олександр Жимінюк}

На каждых выборах очень много прилипал это те которые хотят любой ценой к
корыту. Не кто уже не верит лозунгом. Перевернуть все это с головы на ноги
могут только радикально настроенные люди к нынешней власти. Другого варианта не
будет. У всех чиновников самая главная ценность, это деньги. Вся институция
повалена, и поэтому нужны профессионалы дела, а не языка. Все кто это на себе
уже почувствовал, это малый, и средний бизнес. Это те люди которые в нашей
стране платят только налоги, и штрафы и при этом не видят не кокой перспективы
в этой стране.

\iusr{Игорь Ковальчук}
Электоральный Майдан ?  @igg{fbicon.smile} 
Точно, Зе и его дримтимы это ж всё предыдущая власть - Порошенки  @igg{fbicon.smile} 
П/с

Янукович ушел через уличный Майдан и призыв ввести российские войска, Порошенко
в результате самых демократичных выборов (электоральный Майдан). Вы точно
уверены, что \enquote{новые Порошенки} уйдут как те кого вы им приписываете ?

\iusr{Yurii Romanov}
Это все понятно, но когда Петя должен подключится к процессу?

\iusr{Олена Полулях}
В будь якрму випадку акціі пІд ВР на руку РФ

\begin{itemize} % {
\iusr{Святослав Дубина}
\textbf{Олена Полулях}, а хто є вигодоотримувачем свавілля чиновників і некомпетентності влади?

\iusr{Сергей Чепалда}
\textbf{Олена Полулях} Напоминает времена пороха.- делай с народом что хочешь, а кто против, тот рука кремля

\iusr{Татьяна Сердечная}
\textbf{Олена Полулях}, то бишь склонить голову в поклоне и не трепыхаться?

\iusr{Владислав Колодницький}
\textbf{Олена Полулях} вам к психиатру
\end{itemize} % }

\iusr{Світлана Костюк}

Вчера после длительного оплачиваемого отпуска, изнеженные на экзотических
островах депутаты, не вышли к тем, кто работает и платит налоги круглый год без
отпусков. Это было показательное шоу зажравшейся власти.

\begin{itemize} % {
\iusr{Yurii Romanov}
\textbf{Світлана Костюк} 

а шо там есть власть, понабирали хренотень какую то безвольную, еще и
неприкосновенности лишили. Вышли бы они или не вышли, что они решают?

\iusr{Микола Чепурний}
\textbf{Yurii Romanov} 

Если они не решают, следовательно не там протестная акция проведена. И не
экономические требования нужно выдвигать, а политические..., типа: \enquote{Распустить
хренотень ...неприкосновенную и изнеженную!} А для этого НЕОБХОДИМО иметь
ПАРТИЮ, но не \enquote{попа Гапона}

\end{itemize} % }

\iusr{Сергей Чепалда}

Есть такая партия - это партия за радикальные либеральные эконом реформы 5.10.
Балашов много доказывал что этих двух налогов хватит. Не знаю

\begin{itemize} % {
\iusr{Юрій Макарський}
\textbf{Serhiy Chepalda} Ви чули, що він ніс коли вправаджувався владою \enquote{базар} грунтів? отож

\iusr{Alexey Ponomarov}
\textbf{Юрій Макарський} слышали. что то плохое предлагал?

\iusr{Юрій Макарський}
\textbf{Alexey Ponomarov} 

дивлячись, які цілі у влади при запровадженні обігу грунтів - чи це базар, чи це
засіб для створення мільйонів рабочих місць.

\iusr{Юрій Макарський}
\textbf{Alexey Ponomarov} 

а Балашов топив за чистий базар грунтів, який веде до закріплення феодалізму на
землі і аргентинізацію з знищенням сіл.

\iusr{Alexey Ponomarov}
\textbf{Юрій Макарський} 

это повысит стоимость земли, феодализм был с мараторием,когда за копейки можно
орендовать 10000000 гектар, попробуй купить за 30/га столько же

\iusr{Юрій Макарський}
\textbf{Alexey Ponomarov} 

до чого тут мараторій? він свою функцію худо бєдно виконав, а в нас ще є
можливість створити країну самозайнятих куркулів (не дивлячись на сегрегуючий
самозайнятих селян з землі закон 1914 чи прийнятий на користь саме орендаторів
\enquote{базар} грунтів), а не базар грунтів для холдінгів (а за ними ТНК) і
рант'є-міщан, які до \enquote{трудової} діяльності на землі ніяким боком, а лише виводять
ренту в міста (в кращому випадку, а в більшості зовсім в економіку закордоном).

\iusr{Юрій Макарський}
\textbf{Alexey Ponomarov} 

як підніме вартість моїх гектарів 1, в кращому випадку 2 конкуренти? От як би
було так як в цівілізованих світі, що за кожну сотку рій сусідів покупців
відносно малоземельних, можно було б за якесь підвищення гектару заїкатись  @igg{fbicon.frown} 

\iusr{Микола Чепурний}
\textbf{Юрій Макарський} Ось саме тому потрібно мати ПОЛІТИЧНУ національну партію, а не економічну приватну балашовську !

\iusr{Юрій Макарський}
Як поховала гебня Рух, то й не стало таких партій, а чи будуть?
\end{itemize} % }

\iusr{Валерий Метелкин}

Головне коли цього дракона вб'ють не стати новим драконом, був вчора там, дуже
злий(((

\iusr{Сергей Чепалда}

На долгосрочную. Политику Гетьманцева и его самого нужно делать известным и
токсичным. Тогда есть шанс что после очередных выборов что-то изменится

\iusr{Юра Сокол}
ЛЮТО!!!! ПЛЮСУЮ!!!!

\iusr{Татьяна Новикова}
\textbf{Павел Себастьянович} Любая власть делает с народом только то, что ей позволяют

\iusr{Теодор Дяків}

Ще один Реанімаційний пакет реформ ... - \enquote{модернизация}, тобто підмальовування,
підрихтовування московсько-радянської політичної системи та відповідної їй
економіки в Україні.

І, звичайно ж, жодного слова про першочергову потребу політичних реформ ...

\begin{itemize} % {
\iusr{Павел Себастьянович}
\textbf{Теодор Дяків} а что вы называете политическими реформами?

\iusr{Теодор Дяків}
\textbf{Pavel Sebastianovich} 

Політичні реформи - послідовне заміщення успадкованої від СРСР
московсько-радянської політичної системи українсько-європейською. Конкретніше -
зміна характеру, змісту, сутності ключових інститутів держави - місцевого
самоврядування, виборчої, судової і правоохоронної систем - їх призначення,
завдань, функцій, методів роботи.. Тобто, завданням політично притомної частини
суспільного активу є, вважаю, популяризація і організація послідовного переходу
від імітації державою Республіки до реальної Республіки.

В умовах теперішнього характеру політичних інститутів держави в Україні ніякого
економічного розвитку не може бути.

\iusr{Теодор Дяків}
\textbf{Микола Чепурний} Поляки Республіку побудували, і українці рано чи пізно побудують.

\iusr{Микола Чепурний}
\textbf{Теодор Дяків} 

В Польщі США не влаштували ТВД для війни з Кремлем і влили купу долярів. А нам
зробили війну і сварки з усіма сусідами заради збільшення впливу на РФ, та ще й
МВФ зробив наших правнуків боржниками. Різницю хіба не видно ?

\iusr{Теодор Дяків}
\textbf{Микола Чепурний} Різниця в політичній свідомості поляків і українців.

\iusr{Микола Чепурний}
\textbf{Теодор Дяків} 

Є така! Але на початку \enquote{старту} найважливішою різницею була наявність
патріотичної свідомості у Валенси і шкурних інтересів у Кучми

\iusr{Теодор Дяків}
\textbf{Микола Чепурний} 

Не варто зводити успіхи чи проблеми багатомільйонних країн до однієї людини -
польське суспільство ніколи не вибрало б \enquote{двічі несудимого} Януковича чи
теперішнього блазня ...

\iusr{Микола Чепурний}
\textbf{Теодор Дяків} 

\enquote{Не варто} нехтувати \enquote{впливом особистості на хід історії}! Як би польське
суспільство було розграбовано, аморалізовано і розділено так, як це зробив
куЧмо із \enquote{синами} юЩенком, Яником та поТрохом, то не відомо які блазні і як
прийшли б до влади у них. Дуже важко \enquote{зводити до спільного знаменника}
історично + територіально + ментально + олігархічно

відмінні країни=народи

\iusr{Теодор Дяків}
\textbf{Микола Чепурний} 

Історично, ментально і територіально українці і поляки є дуже близькими, а в
минулому були навіть рівноправними частинами однієї держави. А от політично ці
народи сьогодні дуже різні : - поляки почали відриватися від Москви при першій
же нагоді у 1989-90 роках, а Україна - тільки у кінці 2013 року.

\end{itemize} % }

\iusr{Веніамін Тимошенко}

Системна помилка, яка ніколи не призведе до успіху.

Всі активісти-реформатори говорять і говорять про якусь міфічну політсилу, в
якій будуть бездоганні люди, які одразу наведуть лад в країні.

Не наведуть, тому що в партії буде пару тисяч членів і хай ще навіть 200 в
парламенті, а на місцях сотні тисяч вороватої мразі, яка буде саботувати всі
реформи і продовжувати красти. А якщо у вас не вдасться зайти в парламент, бо
оскотинені знову проголосують серцем за своїх старих вороватих кумирів? Що,
чергові 5 років грабунку та глуму і мрії про міфічну політсилу?

Воювати з диржавою потрібно просто зараз, точніше ще вчора, точніше - ще
позавчора.

Працівникам потрібно зупиняти в законних страйках всю залізницю, аеропорти,
автотранспорт, всю торгівлю, все обслуговування, а бізнесу припиняти платити
податки.

І все, це смердюче шапіто зупиниться і почне слухати ВИМОГИ, а не камлання
лоховатих терпіл. Через конкретні щохвилинні збитки та зупинку інфраструктури,
а не молитви та стони безпомічних філософів.

Ах да, всі мають озброїтись, щоб навіть не було думок спробувати розігнати все
силою.

Тут і зараз, а не колись можливо.

\begin{itemize} % {
\iusr{Андрей Гарнат}
\textbf{Веніамін Тимошенко}

А як ти бачиш наслідки, які держава може законно організувати будь підприємству
після «зупиняти платити податки»?

\iusr{Веніамін Тимошенко}
\textbf{Андрей Гарнат} 

страйк - це не шабаш з прапорцями, а зупинка роботи. Якщо немає роботи - немає
доходів і відповідно податків. Зарплат теж немає, значить немає ПДФО і ЄСВ.

\iusr{Константин Орда}
\textbf{Веніамін Тимошенко}, ФОП на єдиному податку зобов'язаний його платити незалежно від отриманих доходів. Як страйкувати?

\iusr{Андрей Гарнат}
\textbf{Веніамін Тимошенко}

Ну вот ты некоторое время начни не получать зарплату, и на что ты будешь жить?

Или если ты руководитель предприятия, возьми останови работу предприятия и
перестань выплачивать зарплату сотрудникам, и посмотришь, что начнём
происходить, так как твои предпринимательские проблемы сотрудникам не
интересны, и так на любом предприятии, независимо от его размера.

\iusr{Веніамін Тимошенко}
\textbf{Андрей Гарнат} 

страйк починають працівники, а не керівник, щоб якось триматись треба страйкові
фонди. Для ефекту достатньо тижня простоя всієї країни, а скиглити і мріяти
можна ще 30+ років.

\iusr{Веніамін Тимошенко}
\textbf{Константин Орда} 

ви можете організовано призупинити діяльність. Ну і масова непокора в кінці
кінців, не платіть і все, але всі.

А зборища під Радою варто припиняти, це безглузде витрачання часу та ресурсів.

\iusr{Веніамін Тимошенко}
\textbf{Андрій Гарнат} 

а на ці твої слова я знову тобі нагадаю свою фразу: \enquote{підприємство - це місце
для синергії керівника та його працівників, це команда}, натомість більшість
підприємців вважають свій персонал власністю чи навіть сировиною.

Результат на табло.

\end{itemize} % }

\iusr{Сергей Кононенко}

Підтримую. А вот эти рожи, которые объединились ради \enquote{спасения} страны и не
вылазящие со Свободы слова - порошенко, яценюк, аваков и т.д - уже вызывают
своим умничанием и \enquote{патриотизмом} рвотный рефлекс.

\begin{itemize} % {
\iusr{людмила карпович}
\textbf{Сергей Кононенко} а Зеленский вызывает высшее наслаждение?

\iusr{Viktoria Sklyarova}
\textbf{людмила карпович} нет, рвотный тоже

\iusr{Сергей Кононенко}
\textbf{людмила Карпович} По-Вашему, если кто-то не за петю, он автоматически за зеленского?

\iusr{людмила карпович}
\textbf{Сергей Кононенко} а что были ещё варианты?
\end{itemize} % }

\iusr{Леся Ворона}
це добре, але не розумію російську, оскільки живу в Україні. Це ви путіна так розмазали?

\iusr{Дарія Матвіїв-Веклин}
\textbf{Lesya Vorona} так \enquote{какая разніца} - це бальзам для України. Твердолобі.

\iusr{Петро Гарматюк}

Невже і цей раз хтось клюне на \enquote{нову} партію, на \enquote{нових}
професіоналів ?!

Мені смішно це читати.

І смішно, і сумно.

Зовнішній воріг - Росія звозить війська і зброю до наших кордонів, весь світ
переймається проблемою як упередити наміри Путіна, як зупинити фашиста, щоб не
напав на Україну.

Внутрішні вороги - петі, юлі, віті, ренати та інші згрупувалися, щоб роздерти
нашу Україну з середини, розсварити нас, розділити, натравити одне на одного.

Президент зі своєю командою робить майже неможливі речі, знаходячи підтримку
Європи, та Америку, Україна отримує допомогу, про яку і мріяти раніше не могли.
До нас прибуває зброя, фінанси, наші дружні країни готові прислати своїх
добровольців. Росії дають знати чим для неї закінчиться не просто
широкомаштабне вторгнення, а спроба переступити наш кордон російському солдату.

Всі українці, весь світ живе в напрузі, живе в очікуванні що ж буде дальше.

ОКРІМ ОДНИХ !!! \enquote{УКРАЇНА 30 +}

У цих одна задача - скористатися надважкою ситуацією і дорватися до влади. Вони
готові і Росії допомогти, лиш аби до корита, лиш аби красти.

ЯК НА МЕНЕ, ТО ВСЯ ЦЯ ШУШЕРА НЕ ВАРТА ВІЇДЕНОГО ЯЙЦЯ.

Судячи з того, що навіть з російської на українську не переклали - дуже спішать
відрапортувати перед Московією.

ГНАТИ В ШИЮ ЦИХ НОВОСПЕЧЕНИХ \enquote{ПРОХВЕСІОНАЛІВ}.

\begin{itemize} % {
\iusr{Веніамін Тимошенко}
\textbf{Петро Гарматюк} 

ви той приклад, кого потрібно вже селити в резерваціях з позбавленням права
голосу, саме ви довели країну до такого ублюдочного стану.

\iusr{Boris Kurchenko}
Що там з САП?

\iusr{Viktoria Sklyarova}
\textbf{Петро Гарматюк} полікуватися б,вам...

\iusr{Святослав Дубина}
\textbf{Петро Гарматюк}, вам треба в контррозвідці працювати! Кремлівських агентів на раз викриваєте  @igg{fbicon.face.tears.of.joy} 

\iusr{Леонид Шехтер}
\textbf{Петро Гарматюк} , 

в армии не сразу, а недавно сделаны титанические сдвиги. Это плюс, и браво!
Тут, как и во внешней политике, горжусь своим президентом ! Но! Почему то же я
не могу сказать о политике закручивания гаек гетьманцевым, например?!!!

\end{itemize} % }

\iusr{Александр Глущенко}

НЕТ такой партии. НЕТ такого лидера. Общество заполитизировано, и искусственно
доведено до нищеты.

\begin{itemize} % {
\iusr{Теодор Дяків}
\textbf{Oleksandr Glushchenko} 

На жаль, в суспільстві, на екранах ТБ немає дійсно політичних дискусій - ніхто
ж реально не аналізує сучасну політичну систему в Україні, не пропонує чогось
(та хоч чого-небудь!) сутнісно іншого. 30 років спостерігаються теревені, які
десятою дорогою обходять успадковану Україною від СРСР московсько-радянську
політичну систему - причину всіх українських проблем.

\end{itemize} % }

\iusr{Тарас Плахтій}

українофобські проекти, що просувають двомовність, а фактично - витіснення
української мови з ужитку, не мають жодних вагомих перспектив в Україні)

\end{itemize} % }
