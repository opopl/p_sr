% vim: keymap=russian-jcukenwin
%%beginhead 
 
%%file 28_11_2021.fb.baumejster_andrej.kiev.filosof.1.hajdegger
%%parent 28_11_2021
 
%%url https://www.facebook.com/andriibaumeister/posts/4481851601936364
 
%%author_id baumejster_andrej.kiev.filosof
%%date 
 
%%tags baumejster_andrej.filosof.kiev,filosofia,hajdegger.filosof,ukraina
%%title Вечер первого воскресенья Адвента. Можно и о Хайдеггере поговорить...
 
%%endhead 
 
\subsection{Вечер первого воскресенья Адвента. Можно и о Хайдеггере поговорить...}
\label{sec:28_11_2021.fb.baumejster_andrej.kiev.filosof.1.hajdegger}
 
\Purl{https://www.facebook.com/andriibaumeister/posts/4481851601936364}
\ifcmt
 author_begin
   author_id baumejster_andrej.kiev.filosof
 author_end
\fi

Вечер первого воскресенья Адвента. Можно и о Хайдеггере поговорить... 

"Онтология (герменевтика фактичности)" - это 13-часовой курс летнего семестра
1923 года. Это последний лекционный курс, который Хайдеггер читал в статусе
приват-доцента Фрайбургского университета, перед переездом в Марбург. 

\ifcmt
  pic https://external-frt3-2.xx.fbcdn.net/safe_image.php?d=AQG3TWrPFHAsV6f8&w=500&h=261&url=https%3A%2F%2Fi.ytimg.com%2Fvi%2FFT-I8ADLuFQ%2Fmaxresdefault.jpg&cfs=1&ext=jpg&_nc_oe=6f1f6&_nc_sid=06c271&ccb=3-5&_nc_hash=AQGfGt3hTo0bTh-R
  @width 0.8
\fi

\href{https://www.youtube.com/watch?v=FT-I8ADLuFQ}{%
Мартин Хайдеггер. Онтология (герменевтика фактичности). На подступах к "Бытию и времени", %
Andrii Baumeister, 28.11.2021%
}

Как говорит Хайдеггер в самом начале лекционного курса, его "спутником в
поисках был молодой Лютер, образцом (Vorbild) - Аристотель. Толчком стал
Кьеркегор, а глаза открыл Гуссерль". 

Для меня этот лекционный курс - одна из самых дерзких и рискованных попыток
философии ХХ века: объединить мастеров строгого и методичного мышления
(Аристотель и Гуссерль) с теологией (Лютер) и радикальной религиозной
философией (Кьеркегор). Это как если бы кто-то попытался соединить Фреге и
Бердяева, или Рассела и Шестова. 

Не менее дерзкая попытка - создать синтез из онтологии, герменевтики,
феноменологии и трансцендентальной философии. Результатом этих попыток стал
трактат "Бытие и время". 

То, что Хайдеггер в этом лекционном курсе называет герменевтикой - это не
теория интерпретации текста. Это подход, который вобрал в себя метод деструкции
традиции в предыдущих курсах и который был нацелен на толкование способов бытия
"радикального сущего" - Dasein. 

В рамках новосозданного философского словаря Хайдеггер пытается подорвать
изнутри главные методологические подходы схоластики и философии Модерна, а
также подвергнуть радикальной деструкции  господствующие дискурсы начала ХХ
века (неокантианство, неопозитивизм, социологию Макса Вебера, антропологию
Макса Шелера). 

Этот необычный синтез гетерогенных методологий и способов мышления в конце
концов разрушится ко второй половине 30-х годов. Вначале исчезнет слово
"онтология", потом "Герменевтика" и "феноменология". Но этот краткий промежуток
времени (с 1920 по 1927 годы) останется одним из самых интенсивных и
напряженных опытов мышления за последние сто лет. И мне кажется, вряд ли
кому-то удастся повторить что-нибудь подобное в ближайшем будущем. 

Предлагаю краткий разбор этого ключевого текста под странным титулом.
