% vim: keymap=russian-jcukenwin
%%beginhead 
 
%%file slova.putin
%%parent slova
 
%%url 
 
%%author 
%%author_id 
%%author_url 
 
%%tags 
%%title 
 
%%endhead 

\emph{Путин} трясет \enquote{Ящик Пандоры}, но пока не открывает,
obozrevatel.com, 24.05.2021

Навіть попри те, що \emph{Путін} не почав нову фазу завоювань в Україні –
останній епізод кілька тижнів тому змусив людей у Вашингтоні і Києві
понервувати, radiosvoboda.org, 24.05.2021

В особі Лукашенка ми маємо справу з \emph{Путіним}, radiosvoboda.org, 25.05.2021

За його словами, нинішня ситуація є на руку російському президентові Володимиру
\emph{Путіну}, який має «набагато кращий шанс протиснути свій сценарій зміни
влади у Білорусі», а це несе набагато більші загрози для України,
radiosvoboda.org, 25.05.2021

На вашому YouTube каналі одне із рекордсменів за переглядами – відео \enquote{Чи
матюкались давні українці?}. Давайте розставимо крапки над і щодо лайок в
українській мові. Чи винен в цьому умовний колективний \enquote{путін}?
pravda.com.ua, 25.05.2021

Клутин - \emph{Путин}, вакцинация - эскалация. Байден забыл, как и что
называется, news24ua.com, 16.04.2021

\emph{Путін} застосовує «стратегію божевільного» – Дмитро Кулеба, radiosvoboda.org, 01.05.2021

Единственный пункт, который вызвал панику у придворного журналистского
профсоюза НСЖУ, — это право СБУ блокировать любые сайты в рамках
«информационных операций» против агентов \emph{Сами Знаете Кого}, \textbf{На
границе Ада. Появилась финальная версия законопроекта о реформе СБУ},
ukraina.ru, 26.05.2021

\emph{Путин} разорвал налоговое соглашение с Нидерландами, lenta.ru, 26.05.2021

Портрет \emph{Путіна} з вирваних зубів виставлений на аукціон, radiosvoboda.org,
27.05.2021

Росія – централізована автократія з ручним керуванням \emph{Путіна}, на нього
зав'язано все, radiosvoboda.org, 30.05.2021

Росія страждає від структурних проблем, які виходять за рамки особливостей
режиму \emph{Путіна}, і тому навіть з його смертю або відходом від влади
політичний курс в Росії може не змінитися, radiosvoboda.org, 30.05.2021

Як ще \emph{Путін} при цьому не застрелився – не знаю, це ж ціла система ППО!
Гаразд, годі сарказму. Лише дуже наївна людина може повірити, що в липні чи
серпні воювала якась «нова армія», не та, яка звільняла Красний Лиман. Як і в
те, що \emph{Путін} справді планував захопити Київ,
\textbf{П'ять цікавих фактів – від \emph{УПА} до АТО}, Павло Зуб'юк, zaxid.net, 01.06.2021


Истеричный лукашизм, или Подарок для \emph{Путина}, Во избежание спекуляций как
с той, так и с другой стороны автор сразу же отметит, что, имея белорусские
корни, а также придерживаясь старой доброй советской традиции, с искренней
симпатией относится к сябрам, причем прежде всего именно к Белоруссии и к
близкородственным нам, выросшим с нами из одного древнерусского и советского
корня братьям-белорусам. Поэтому вся эта антибелорусская, совершенно идиотская,
скотская, подонковская, гнусная, ублюдочная (далее уже только нецензурные
выражения!) истерия, которая развернута в Украине в последние дни в исполнении
разного рода \enquote{актывиздов} и Зе-власти, вызывает рвотный рефлекс и классовую
ненависть,
\textbf{Европа впала в иррациональный истеричный лукашизм}, Александр Карпец,
strana.ua, 03.06.2021

Причем наиболее вероятен именно второй вариант, поскольку \emph{Путин} опыт
Украины, надо полагать, учел, и Беларусь просто так не отдаст. А если так, то
наиболее лакомые куски белорусской экономики пойдут под дерибан околопутинским
олигархам, а остальное будет уничтожено за ненадобностью. Этим
\enquote{кремлевский вариант} кончины белорусской государственности отличается
от \enquote{условно демократического}, при котором будет просто уничтожено все,
за очень редким исключением, вроде добычи калия,
\textbf{Европа впала в иррациональный истеричный лукашизм}, Александр Карпец,
strana.ua, 03.06.2021

