% vim: keymap=russian-jcukenwin
%%beginhead 
 
%%file 10_11_2020.fb.roman_barashev.1.mir_ili_vojna
%%parent 10_11_2020
 
%%url https://www.facebook.com/roman.barashev/posts/2160865440712590
%%author 
%%tags 
%%title 
 
%%endhead 
\subsection{Мир или война? Свобода или страх?}
\label{sec:10_11_2020.fb.roman_barashev.1.mir_ili_vojna}

\Purl{https://www.facebook.com/roman.barashev/posts/2160865440712590}
\Pauthor{Барашев, Роман}
\index[names.rus]{Могильницкий, Максим Сергеевич!адвокат,Киев}

Мир или война?  Свобода или страх?

Всё упирается в эти вечные понятия. В этот непростой выбор.

Можно тщательно скрывать, что половина Украины -- это Украина, говорящая на
русском языке. Можно пытаться это менять, вытравливать, запугивать,
дискриминировать. 

А можно признать, принять, развивать.

\ifcmt
pic https://scontent.fiev22-1.fna.fbcdn.net/v/t1.0-9/124852135_2160865380712596_5780198781634063114_n.jpg?_nc_cat=107&ccb=2&_nc_sid=730e14&_nc_ohc=Q9xUrMMkSaQAX91P9zr&_nc_ht=scontent.fiev22-1.fna&oh=245057805e725eab09e77c45f07f54a4&oe=5FD20565
\fi

Презентовать, строить Украину на русском языке -- это ещё и проявлять
миролюбие, широту души и мировоззрения, строить мирное будущее для детей.

Строить Украину, где национальный язык и вера прикреплены к армии, борясь с
"чуждыми" языком и мышлением -- значит, подогревать воинственность,
враждебность в сознании народа.

Но мова, приравненная к армии -- не перо, приравненное к штыку. В случае с
пером человек сам решает, воевать ему или не воевать. 

Такие мысли пришли в голову после нашего разговора-интервью с автором ПЕТИЦИИ к
президенту ПРОТИВ ЯЗЫКОВОЙ ДИСКРИМИНАЦИИ в Украине, адвокатом Максимом
Сергеевичем Могильницким, желаю всем такого защитника, и успехов ему и всем нам
в борьбе за здравый смысл и взаимопонимание! 

Оказалось, наши дети в одну и ту же школу ходят, надо же!

Ссылка на петицию -- в 1-м комментарии.\Furl{https://petition.president.gov.ua/petition/108306?fbclid=IwAR0TaVvjtiFSp0YLZgEWgFHEXWgwR2itEVx3VTUBFZzS6vfcZgX8A3fSLa0}
