% vim: keymap=russian-jcukenwin
%%beginhead 
 
%%file 08_12_2022.fb.ryzhenko_sergij.ua.likar.dnepr.1.sljozy_doloni
%%parent 08_12_2022
 
%%url https://www.facebook.com/rsa010963/posts/pfbid02xhHfmz8QwP2YXKEEaJCXBKp3EserMJ97qmxghx4qzxxZmsPuKCBuEiQMGc522edbl
 
%%author_id ryzhenko_sergij.ua.likar.dnepr
%%date 
 
%%tags medicina
%%title У нього були сльози на брудних долонях...
 
%%endhead 
 
\subsection{У нього були сльози на брудних долонях...}
\label{sec:08_12_2022.fb.ryzhenko_sergij.ua.likar.dnepr.1.sljozy_doloni}
 
\Purl{https://www.facebook.com/rsa010963/posts/pfbid02xhHfmz8QwP2YXKEEaJCXBKp3EserMJ97qmxghx4qzxxZmsPuKCBuEiQMGc522edbl}
\ifcmt
 author_begin
   author_id ryzhenko_sergij.ua.likar.dnepr
 author_end
\fi

У нього були сльози на брудних долонях...

Як важко вижити і так легко помирати, - каже анестезіолог Тетяна Тешина.

Кома, поранення в серце, осколок в лівій гемісфері мозочка, розірваний уламками
живіт, який залетів до кишківника.  Михайлу тільки виповнилось 22, він з Києва.

Хірург \href{https://www.facebook.com/profile.php?id=100029563000279}{Ярослав Медведик} пішов на операцію відчаю.

Трахеотомія, втручання кардіохірурга, який ушив поранення лівого шлуночка, і це
все в одну ніч.

Перелито більше чотирьох літрів препаратів крові.

Михайло все витримав.

Тільки що на обході із задоволенням всі зітхнули – на ранок поранене серце
працює.

Ситуації, які сотнями вдень і вночі проходять в лікарні Мечникова, показують
всі жахи та напругу війни.

У Михайла ще осколок в голові, який видалить професор
\href{https://www.facebook.com/andrey.sirko.1}{Andrey Sirko} після
стабілізації.

Людські трагедії та їх світи, які ми втрачаємо через цю ненависну війну.

Сьогодні маленька перемога – серце Михайла тихо б’ється, зняли з апарату ШВЛ.

Але більше всього герой переживає, що рідні - мама, бабуся і сестра - не
перенесуть, коли побачать його пораненим.

He had tears on his dirty palms...

"It is hard to survive and so easy to die,"  says anesthesiologist Tetyana
Teshina.

He was in a coma, with a wound in his heart, a piece of shrapnel in the left
hemisphere of his cerebellum, a badly damaged stomach, and fragments of
shrapnel in the intestine.

Mykhailo has just turned 22. He is from Kyiv.

Surgeon Yaroslav Medvedyk started a surgical procedure with great despair.

A tracheotomy, a cardiosurgical procedure that involved sewing the wound in the
left ventricle, was performed overnight.

More than four liters of blood products were transfused.

Mykhailo has withstood everything.

Everyone caught their breath at the morning round — the wounded heart was
working.

Situations like this, where hundreds of people pass through the Mechnikov
Hospital at all hours of the day and night, displaying all the horrors and
pressures of war.

Mykhailo still has a fragment of shrapnel in his head; it will be removed by
Professor Andrey Sirko after the stabilization of his condition.

There are lots of human tragedies because of this hateful war.

We have a small victory today — Mykhailo's heart is beating quietly, and he has
been taken off the ventilator.

The hero worries most of all about his family — his mother, grandmother, and
sister. He is afraid that they would not bear seeing him wounded.

\ii{08_12_2022.fb.ryzhenko_sergij.ua.likar.dnepr.1.sljozy_doloni.orig}
\ii{08_12_2022.fb.ryzhenko_sergij.ua.likar.dnepr.1.sljozy_doloni.cmtx}
