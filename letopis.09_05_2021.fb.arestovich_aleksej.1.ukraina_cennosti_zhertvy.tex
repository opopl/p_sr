% vim: keymap=russian-jcukenwin
%%beginhead 
 
%%file 09_05_2021.fb.arestovich_aleksej.1.ukraina_cennosti_zhertvy
%%parent 09_05_2021
 
%%url https://www.facebook.com/alexey.arestovich/posts/4312468992150481
 
%%author 
%%author_id 
%%author_url 
 
%%tags 
%%title 
 
%%endhead 
\subsection{И если вырвать занозу из нашего сердца пока не удаётся, то хотя бы пошатать}
\Purl{https://www.facebook.com/alexey.arestovich/posts/4312468992150481}

\ifcmt
  pic https://scontent-frt3-1.xx.fbcdn.net/v/t1.6435-0/p180x540/181674817_4312468565483857_3293560069489204936_n.jpg?_nc_cat=109&ccb=1-3&_nc_sid=8bfeb9&_nc_ohc=aI0pBorZR8wAX-55bk2&_nc_ht=scontent-frt3-1.xx&tp=6&oh=9ae0a32f45ac728276ff298d226c3687&oe=60BE96D4
\fi

- Государства взаимно признают тесты на ковид,  так же, как и паспорта.

А вот вакцины признают не все.

Это значит, что в выборе они руководствуются разными наборами ценностей:

- при тестах формальными,
- при вакцинах реальными.

Ценностная шизофрения - наиболее характерный знак пандемии, хорошо становится
видна разница между декларируемые в мире ценностями (то, что мы, люди,
заявляем) и реальными (то, как мы реально поступаем). 

О реалиях распределения и резервирования вакцин говорить вообще не очень
принято, ибо ужасные страшилки про «…золотой миллиард, эксплуатирующий
остальное человечество», становятся нехорошо похожими на правду.)

Этика человеческая в свете ее презентации делится на три вида:

- идеальная (как человек хотел бы поступать),
- реальная (как поступает),
- декларируемая (рассказывает, как надо поступать).

Когда у человечества «все хорошо» (что было при известном рассмотрении являлось
весьма сомнительным тезисом и в доковидные времена, глядя на 9 миллионов
человек, ежегодно умирающих от голода в мире), мы лайкаем постеры с
сентиментальными декларациями.

Когда розовый крем слегка опадает, проступают стальные пирамиды, спрятанные внутри розовых тортов.

Слабых бьют.

Украина станет сильной и самостоятельной, или ее сомнут. 

На этом пути наиболее сложной нашей задачей является не строительство могучих
вооружённых сил или гибкой, производительной экономики, а выпутывание из самой
любимой украинской игрушки - комплекса вселенской жертвы, весьма ярко
проступившем вчера в ленте.

Жертвы, жертвы, жертвы.

«Никогда больше» - это вообще-то декларация отказа от борьбы, уход в
оборонительную позицию, которая всегда проигрышная.

«Каждый раз, когда захотим» - вот правильный девиз.

Декларация вторичной реактивности от действий окружающих, ставшая едва ли не
национальной идеей, вот - главная отравленная заноза в украинском сердце.

На идее «отъебитесь от нас» мы никуда не уедем, мы дождёмся. В очередной раз. 

Те, кто хочет «…мирно жить и просто строить» выбрали не ту страну для проживания.

В 2014 году мы заснули в Европе (которая спит до сих пор), а проснулись в Израиле.

Украина - страна на библейском разломе истории и географии, и просто спрятаться
за забором НАТО, чтобы снова «…мирно жить» (как об этом мечтает половина
страны) не удасться.

Все равно догонит разлом смыслообразования.

Только логика экспансии: смыслов, культуры, силы и экзистенции - единственная
дорога для Украины, по ходу которой мы не сотремся и не растаем, как дым.

Защищаться - проиграть.

Наступать - жить.

Украинское кино в украинском павильоне на Марсе - жизнь.

«Сады опрыскивать» - смерть.

И если вырвать занозу из нашего сердца пока не удаётся, то хотя бы пошатать.
