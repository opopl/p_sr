% vim: keymap=russian-jcukenwin
%%beginhead 
 
%%file 01_09_2018.stz.news.ua.mrpl_city.1.pervoje_sentjabrja_1955_goda
%%parent 01_09_2018
 
%%url https://mrpl.city/blogs/view/pervoe-sentyabrya-1955-goda
 
%%author_id burov_sergij.mariupol,news.ua.mrpl_city
%%date 
 
%%tags 
%%title Первое сентября 1955 года
 
%%endhead 
 
\subsection{Первое сентября 1955 года}
\label{sec:01_09_2018.stz.news.ua.mrpl_city.1.pervoje_sentjabrja_1955_goda}
 
\Purl{https://mrpl.city/blogs/view/pervoe-sentyabrya-1955-goda}
\ifcmt
 author_begin
   author_id burov_sergij.mariupol,news.ua.mrpl_city
 author_end
\fi

\ii{01_09_2018.stz.news.ua.mrpl_city.1.pervoje_sentjabrja_1955_goda.pic.1}

1 сентября 1955 года в аудиториях Ждановского металлургического института
приступил к занятиям новый набор студентов. Он был четвертым по счету после
восстановления сожженных немцами в начале сентября 1943 года бывшего штаба
238-го стрелкового полка, бывшего детского дома, а еще раньше – мариупольского
епархиального училища. Немцы здание сожгли, немцы же его и восстанавливали,
правда, теперь уже военнопленные. Работу свою они выполнили прекрасно. После
обшарпанных коридоров и классных комнат школы помещения института казались
залами дворца. Еще сохранился едва уловимый запах лаков и красок. Зеркально
блестели паркетные полы. Идеально были окрашены стены.

Для бывшего еще два месяца назад школьника-десятиклассника многое было новым.
\enquote{Новое} начиналось сразу за входной дверью. С левой ее стороны, если смотреть
на выход, была прикреплена доска, на которой еще несколько дней назад парни и
девушки, успешно сдавшие все пять экзаменов, с замиранием сердца искали свои
фамилии среди принятых в вуз, висело расписание занятий. В том числе и для
группы МЧ-55-4. Память не сохранила, кто нам, группе МЧ-55-4, представил
старосту – Леню Чугунова. Это был рыжеватый парень, вероятно, на пару лет
старше большинства новоиспеченных студентов. На нем был синий форменный китель
связиста. Старше его были двое, явно старше Чугунова. Они сразу запомнились.
Митя Данилов и Юра Евстифеев - демобилизованные сверхсрочники. Позже узнали,
что Митя был уже женат и жил с семьей в нашем городе. А Юра приехал из
Ужгорода. Как был одет Митя, не отложилось, а вот на Юре были темно-синие
галифе, заправленные в офицерские сапоги, и самый что ни на есть гражданский
пиджак. Нужно заметить, что приверженность к форменной одежде не было \enquote{форсом}.
Купить пальто, брюки, а тем более костюм, было непросто. Да и средств не в
каждой семье на это хватало.

\textbf{Читайте также:} 

\href{https://mrpl.city/blogs/view/o-vyzovah-i-vozmozhnostyah-dlya-mariupolya-v-novom-uchebnom-godu}{%
О вызовах и возможностях для Мариуполя в новом учебном году, Лилия Борисенко, mrpl.city, 01.09.2018}

В группе было только пятеро местных – Кирилл Брызгунов, Эрнст Голодов, Виталий
Митько, Эрнст Петров и автор этих строк. Украшением нашей мужской группы были
три девушки – Вера Карманова, Надя Савина и Марина Телелейко, которая вскоре
покинула институт. Года два поучившись, оставили нашу группу Вера и Надя по
причинам житейским. Ушел из группы и Петров. Кто откуда приехал? Из каких краев
Марина – так и не успели узнать. Верочка - из какого-то приволжского города,
Надя - из города Сасово, что на Рязанщине. Коля Гаврик - из шахтерского поселка
Сутаган Луганской области, из той же области и Игорь Лютый, Витя Катаев - из
Омутнинска, города на Северном Урале, Боря Коржов и Витя Несветов - из
Дружковки, Толя Кубышкин - из нынешнего города Новоазовска. Володя Бондаренко -
из села в Белгородской области, Виталий Приступов - из Енакиево. Гена Мааль -
из Евпатории, Гена Пухальский - из Белоруссии, Рудольф Матрахин добрался до
наших краев из Казахстана, Валера Руднев -  из Ельца, Эдик Стаханов – парень из
Татарии. А вот откуда Коля Нестеренко и Кашубин - вылетело из головы. Это уже
потом разобрались, кто есть кто и кто откуда. А познакомились лишь на
практических занятиях, на вузовском сленге – проработках. Ведь в лекционных
аудиториях, где присутствует несколько групп сразу, как разобраться, кто из
твоей, а кто из чужой?

В наборе 1955 года на дневное отделение института было принято 250 человек. Из
них сформировали две группы по специальности \enquote{Металлургия черных металлов}
(сталеплавильщики), две группы – \enquote{Обработка металлов давлением} (одна –
прокатчики, вторая – кузнецы), две группы – \enquote{Металловедение и термообработка},
две группы – \enquote{Сварочное производство} и две группы – \enquote{Механическое оборудование
металлургических цехов}. В этот год в нашем городе окончили средние школы (по
разным оценкам) от двух до двух с половиной тысяч юношей и девушек.
Значительная часть их принесли заявление в единственный тогда институт. К этому
нужно прибавить выпускников предыдущего года, \enquote{провалившихся} на приемных
экзаменах в столичных учебных заведениях. А еще были ребята и девчата,
окончившие десятилетки в других городах и поселках. К тому же обладатели
золотых и серебряных медалей, а также выпускники техникумов с \enquote{красными}
дипломами принимались в вузы без экзаменов. Можно себе представить, какой был
конкурс на одно место в Ждановском металлургическом институте. Запомнилось, как
в первый день сдачи экзаменов, а это был экзамен по русскому языку (сочинение),
на втором этаже невозможно было протолкнуться. Но на следующие дни ряды
жаждущих получить высшее образование становились все реже. А в последние два –
три дня в коридорах можно было увидеть лишь небольшие группки абитуриентов...

\textbf{Читайте также:} 

\href{https://mrpl.city/news/view/hrustalnye-lyustry-zerkalnyj-holl-s-divanami-tehlitsej-mariupolya-udivil-ministra-fotofakt}{%
Хрустальные люстры, зеркальный холл с диванами. Техлицей Мариуполя удивил министра, mrpl.city, 31.08.2018}

Не удалось вспомнить, кто нам читал самую первую лекцию в институте. Но можно
биться об заклад, что происходило это в 24-й аудитории, и это был один из трех
преподавателей, каждый из которых достоин, чтобы его вспомнили. Математику, а
точнее, аналитическую геометрию, читал доцент, кандидат физико-математических
наук Яков Брониславович Рутицкий. Он был молод, на нем был безукоризненно
отглаженный серый костюм. Запомнилось, как он мелком изображал необходимые
рисунки с четкими буквами и цифрами. Его объяснения сути декартовых координат,
решения задач по определению расстояния между двумя точками и задач более
сложных были просты и доходчивы. Доцент, кандидат химических наук Михаил
Аркадьевич Рабкин преподавал общую химию. Иногда по ходу лекции он показывал
химические опыты, ему помогал лаборант. Порой на лекции мог привести пример из
повседневной жизни. Например, почему нужно полоскать белье в холодной воде, а
не в теплой, а тем более, горячей. Желающим показывал лепесток из золота,
который на просвет оказывался розовым. Михаил Аркадьевич как-то необычно держал
указку, пальцами большим, указательным и мизинцем. Позже кто-то рассказал, что
во время Отечественной войны на фронте он был ранен в руку. Вопреки ожиданию,
что предмет \enquote{Основы марксизма-ленинизма} будет скучным, оказалось, что слушать
преподавателя Бориса Васильевича Николаева было интересно. Он был хорошим
рассказчиком, именно рассказчиком, а не педагогом или ученым. У него
идеологическая сторона предмета была как бы отодвинута, а внимание студентов
сосредоточилось на примерах Бориса Васильевича из истории. Были еще
практические занятия, на студенческом сленге – проработки. По математике их вел
Бурназов, занятия в химической лаборатории вели ассистенты Екатерина Сидоровна
Погребная и Сима Борисовна Торговицкая.

\ii{01_09_2018.stz.news.ua.mrpl_city.1.pervoje_sentjabrja_1955_goda.pic.2}

Сначала учеба в институте очень нравилась. Не нужно было выполнять домашние
задания. Спросить могли лишь на проработках и то не всех. Оказалось много
свободного времени. В конце семестра безделье аукнулось бессонными ночами перед
зачетами и экзаменами. Прошли первые десять дней учебы в институте. И вдруг
объявили, что студенты всего первого курса, за исключением больных, будут
направлены в ближайшие села для оказания помощи сельскохозяйственным
предприятиям. Нашу группу привезли в кузове грузовой машины в Володарск, а
точнее, на стан одной из бригад колхоза. С места в карьер нам дали работу –
погрузить арбузы на стоящий рядом грузовик. Работа была выполнена довольно
быстро. Наградой за \enquote{доблестный труд} было несколько огромных спелых арбузов.
Местные ребята ограничились двумя-тремя скибками. А вот приезжие из краев, где
сладкая ягода не произрастает, навалились с энтузиазмом. И некоторые
поплатились за это на некоторое время своим здоровьем.

Для обустройства нам выдали огромные мешки-наволочки для матрасов, а также
сухие оболочки кукурузных кочанов для набивки упомянутых ранее матрасов. Нас
распределили по хатам, где спать пришлось на матрасах, разложенных прямо на
глинобитный пол. На следующий день началась работа на кукурузных полях. Это
была уборка кукурузы. Нужно был отламывать кочаны, сбрасывать на небольшие
кучи. Уже в первые дни тыльная сторона правой руки была изрезана острыми, как
бритва, уже сухими листьями кукурузы. Было жарко. С непривычки было очень
тяжело. Руководителем нашей группы был Георгий Федорович Косогов, за два года
до того окончивший наш вуз и работавший на одной из кафедр института. Он
назначал каждому рядок кукурузы для уборки, а также перерыв для отдыха, во
время которого рассказывал разные истории о науке и ученых.

* * *

Было непонятно, почему не с 1 сентября послали нас в колхоз, прервали занятия.
Через время было объяснено, что будто бы существовало положение, что если
принятый на первый курс студент не приступит к учебе в течение десяти дней, его
отчисляют из вуза. Так ли это?
