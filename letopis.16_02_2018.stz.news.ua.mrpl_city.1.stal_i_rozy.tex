% vim: keymap=russian-jcukenwin
%%beginhead 
 
%%file 16_02_2018.stz.news.ua.mrpl_city.1.stal_i_rozy
%%parent 16_02_2018
 
%%url https://mrpl.city/blogs/view/stal-i-rozy
 
%%author_id burov_sergij.mariupol,news.ua.mrpl_city
%%date 
 
%%tags 
%%title Сталь и розы
 
%%endhead 
 
\subsection{Сталь и розы}
\label{sec:16_02_2018.stz.news.ua.mrpl_city.1.stal_i_rozy}
 
\Purl{https://mrpl.city/blogs/view/stal-i-rozy}
\ifcmt
 author_begin
   author_id burov_sergij.mariupol,news.ua.mrpl_city
 author_end
\fi

От Всеволода Александровича Кулика, ветерана комбината им. Ильича, довелось
услышать об одной из трех на весь бывший Советский Союз женщин – начальников
смен в цехах черной металлургии. Это инженер-металлург Александра Ивановна
Савченко – помощник мастера, мастер, начальник смены мартеновского цеха № 2
завода имени Ильича. Именно Савченко ввела Всеволода Кулика, только что
окончившего вуз юнца, в суровый и вместе с тем прекрасный мир сталеварения.

Александра Савченко родилась в 1914 году в г. Лисичанске. По семейным
обстоятельствам вместе с матерью переехала в Мариуполь. Ей было семнадцать лет.
В 1931 г. устроилась в контору завода. В 1932 году поступила на вечернее
отделение Мариупольского металлургического  института. В 1937 г. окончила вуз,
получив диплом с отличием. На заводе, где она уже работала шесть лет, ей
предложили должности в Центральной заводской лаборатории или в один из отделов
заводоуправления. Она отказалась. Попросилась в мартен. Ее приняли в
мартеновский цех на должность диспетчера. Конечно, это было не то, что она
хотела, но главное – она уже была в штате цеха.

\ii{16_02_2018.stz.news.ua.mrpl_city.1.stal_i_rozy.pic.1}

Через время инженера Савченко назначили помощником мастера в печной пролет.
Там-то и оценили ее знания и навыки руководить людьми. Она умела как-то без
нотаций и крика заставить своих подчиненных добросовестно выполнять свою
работу. Следующей ступенькой ее продвижения было назначение в 1940 году на
должность начальника смены цеха. В январе 1941-го всесоюзный журнал \enquote{Работница}
опубликовал большую статью, посвященную Александре Ивановне. Это было время,
когда многие женщины стали осваивать сугубо мужские профессии. Достаточно
вспомнить трактористку Пашу Ангелину, летчиц-рекордсменок Валентину
Гризодубову, Полину Осипенко, Марину Раскову, помощника механика на паровозе
Анну Васильеву и других. Казалось, жизнь складывалась как нельзя лучше. У нее
был сын, она любила свою работу. Конечно, ей было приятно, что о ней написали в
журнале.

\ii{16_02_2018.stz.news.ua.mrpl_city.1.stal_i_rozy.pic.2}

Но фашистская Германия развязала войну. Фронт стремительно приближался к
Мариуполю. Александра Ивановна принимала активное участие в подготовке
оборудования к эвакуации на Восток страны. Однако случилось так, что сама
осталась на оккупированной территории. В 1942 году оккупанты отправили ее в
Германию на принудительные работы. Девятилетнего сына пришлось оставить на
попечение близких людей. На чужбине за скудный паек она работала швеей на
фабрике за колючей проволокой, на которой шили военную форму.

В 1945 году американцы освободили узников концлагеря и Александра Ивановна
вернулась домой, в Мариуполь. Стала работать начальником смены в том же
мартеновском цехе, что и до войны. По своей ли воле или по воле начальства
Савченко переводят мастером на блок из трех печей. Внештатный корреспондент
\enquote{Ильичевца} Анатолий Томаш писал в одном из своих очерков: Ее мастерство
выразилось в оценке труда подопечных. Одним из первых заслуженным металлургом
Украины и кавалером ордена Ленина стал Виталий Ермолаевич Ясиновой. За трудовые
заслуги сталевар Володарчик тоже был награжден орденом Ленина. Сама же
Александра Ивановна Савченко была отмечена почетным званием \enquote{Мастер
отличного качества}. Вот так. Сталевары получили государственные награды, а
мастер, который вел плавки, отмечен наградой на уровне профкома завода.

\ii{16_02_2018.stz.news.ua.mrpl_city.1.stal_i_rozy.pic.3}

Всеволод Александрович Кулик так охарактеризовал Александру Ивановну: \enquote{Женщина
у руля производства стали. Это странно и в то же время, ну, настолько
интересно, что об этом можно рассказывать много и долго. Мягкая по натуре, она
твердо управляла бригадами сталеваров. В ее присутствии ни один, горячий даже
подручный сталевара или сталевар не позволяли себе сказать крепкое слово.
Спокойная, уравновешенная, она влияла своим поведением, своим отношением к
производству на всех. Ну, и вот мне повезло, мне повезло в первые годы, вернее,
в первые моменты моей трудовой деятельности работать с Александрой Ивановной}.

В 1949 году трагически погиб единственный сын Александры Ивановны Владимир.
Стоит ли говорить, какой это был удар для матери. От тоски спасала работа.
Вместе с ней в металлургическом институте в параллельной группе учился Иван
Кузема. Савченко была знакома с Иваном Демьяновичем. Он работал в центральной
заводской лаборатории. В том же году его тоже постигло  горе. После тяжелой
болезни умерла жена. На руках вдовца осталось двое детей - Лидия и Валерий,
четырнадцати и одиннадцати лет. Одинаковая судьба сблизила Александру Ивановну
и Ивана Демьяновича Кузему. В 1952 году они поженились. Александра Ивановна
сделала все, чтобы дети мужа приняли ее как мать. И дети признали ее матерью.

\ii{16_02_2018.stz.news.ua.mrpl_city.1.stal_i_rozy.pic.4}

В 1953 году Александра Ивановна родила дочку Елену. Елена Ивановна Литвинова
рассказывала: \enquote{Родители мне обеспечили счастливое детство. Не в материальном
плане каком-то. А в плане моральном. Мне уделялось много внимания. Очень много.
И со стороны папы, и особенно со стороны матери. Мама много сделала для того,
чтобы дать мне хорошее образование. Она учила меня музыке, занималась уроками.
В общем-то, добилась того, что я отлично окончила школы: и музыкальную, и
общеобразовательную}. Александра Ивановна довольно продолжительное  время
работала в мартеновском цехе, цехе с вредными условиями  труда. Это позволило
уйти на заслуженный отдых раньше, чем ее сверстницы, трудившиеся где-нибудь в
конторах. 

У Ивана Демьяновича с детства была повреждена стопа, он заметно хромал. С
возрастом ему все труднее стало ходить. И тогда на семейном совете решили
купить автомобиль. И автомобиль купили – \enquote{Запорожец}. Александра Ивановна со
свойственным ей прилежанием окончила автошколу, на экзаменах точно и правильно
ответила на все вопросы из Правил дорожного движения и получила водительские
права. Из воспоминаний В.А. Кулика: \enquote{Ее часто можно было видеть на улицах
Ильичевского района на автомобиле. Горбатый автомобиль \enquote{Запорожец}, очень
интересно было наблюдать и видеть по району, что из числа мужчин водителей
выделяется своим искусством Александра Ивановна Савченко – женщина}.

Александре Ивановне судьба уготовила череду тяжелейших испытаний как
физических, так и моральных. Казалось бы, она могла сникнуть, уйти в себя,
обозлиться на столь несправедливый мир, отрешиться от всех благ земных. Что
заставило ее быть оптимисткой? Пожалуй, ответ содержится в цитате из
воспоминаний Елены Ивановны Литвиновой, ее дочери:  \enquote{Жизнелюбие, интерес ко
всему, что ее окружает. Она много времени уделяла семье, прекрасно готовила,
она шила, вышивала, вязала. Она играла на фортепиано, много читала. Ее любимыми
писателями был Диккенс, Голсуорси и особенно Чехов. Главная любовь ее были
цветы. Она настолько серьезно увлекалась цветоводством, что была членом
Всесоюзного общества цветоводов-любителей, ездила в Латвию на съезд лилиеводов,
в Москву - на съезд цветоводов, на съезд в Сочи. Цветами был усажен наш двор
прекрасными, она занималась кактусами, лилиями, тюльпанами. Вела огромную
переписку с цветоводами всего Союза}.

Александра Ивановна Савченко ушла из жизни в сентябре 1987 года, на шесть лет
пережив мужа.
