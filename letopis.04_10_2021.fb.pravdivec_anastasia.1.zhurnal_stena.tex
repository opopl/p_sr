% vim: keymap=russian-jcukenwin
%%beginhead 
 
%%file 04_10_2021.fb.pravdivec_anastasia.1.zhurnal_stena
%%parent 04_10_2021
 
%%url https://www.facebook.com/anastasiyapravdyvets/posts/4417510164982046
 
%%author_id pravdivec_anastasia
%%date 
 
%%tags kultura,obschestvo,teatr,ukraina,zhurnal
%%title Ура!!! Свежий номер Всеукраинского молодежного журнала “Стена” №4 (2021) подписан в печать!!!
 
%%endhead 
 
\subsection{Ура!!! Свежий номер Всеукраинского молодежного журнала \enquote{Стена} №4 (2021) подписан в печать!!!}
\label{sec:04_10_2021.fb.pravdivec_anastasia.1.zhurnal_stena}
 
\Purl{https://www.facebook.com/anastasiyapravdyvets/posts/4417510164982046}
\ifcmt
 author_begin
   author_id pravdivec_anastasia
 author_end
\fi

Ура!!! Свежий номер Всеукраинского молодежного журнала “Стена” №4 (2021)
подписан в печать!!! Слоганом номера стали навсегда актуальные слова Антона
Павловича Чехова: «Берегите в себе Человека!», а на обложке изображены герои
легендарного спектакля о жизни Чехова «Насмешливое моё счастье», Вячеслав
Езепов (Чехов), Лариса Кадочникова (Книппер), Наталья Доля (Лика), реж. М.
Резникович, автор картины В. Плавун. 7 октября 2021 года Вячеславу Езепову
исполнилось бы 80 лет... 

\ifcmt
  ig https://scontent-frt3-1.xx.fbcdn.net/v/t1.6435-9/244448577_4417511321648597_5095782819957522417_n.jpg?_nc_cat=104&_nc_rgb565=1&ccb=1-5&_nc_sid=730e14&_nc_ohc=g4P7lP3-_f0AX-vGrNi&_nc_ht=scontent-frt3-1.xx&oh=fa963a6a2b7ab2d57e3fd473e1c174e1&oe=618170A7
  @width 0.4
  %@wrap \parpic[r]
  @wrap \InsertBoxR{0}
\fi

Так случайно получилось, что этот спектакль красной нитью проходит через весь
свежий выпуск журнала...  Поздравляем, читателей, команду, гостей номера и всех
причастных!!

Уже через неделю журнал появится в продаже по всей территории Украины, в
почтовых ящиках наших подписчиков и в библиотеках. Следите за новостями на
сайте \url{http://stina.kiev.ua/}

В свежем номере читайте: 

.::Эксклюзив номера::.

- Особая форма театральной одержимости. Интервью с Олегом Вергелисом.

"Театр для меня - спасение от плохого антиТеатра, которым утрамбована наша
жизнь, политическая и общественная. Когда после клинической эпопеи я оказался
дома на реабилитации, то театр онлайн спасал меня от уныния и бесцельности дней
нашей жизни.

Если проще, - театр, как состояние души и профессиональная одержимость."

- Первая Всеукраинская выставка Передвижников 21 века посвящённая 190-летию
Николая Николаевича ГЕ под эгидой журнала «Стена».

Настоящее творчество всегда полно оптимизма. Творец не может быть в унынии.
Строитель полон знания в избрании лучших материалов. Живое сердце понимает, как
нужно сейчас дать людям возможность СТРОЕНИЯ.

.::Творческий  форум::.

- Творческий Форум журнала «Стена».  Итоги Международного
литературно-поэтического конкурса «Пусть дорога вдаль бежит»:

Виктория Осташ

Люся Лагутина

Александр Апальков

Елена Богданова

Оксана Дмитренко

Лидия Ляшенко

Недзельницкий Андрей

.::Фабрика грез::.

- Новые аспекты привычного героя. 70 лет со дня выхода картины “Тарас Шевченко” 

За каждым символом стоит живой человек

- Всё, что было. История Бессарабского соловья.

Наряду с Вертинским, Утёсовым, Пётр Лещенко был легендой своего времени.  Его
песни любили все - от рядовых до маршалов, а эммиграция ходила на него, как на
воздух, которым можно было дышать, словно он прилетел с просторов Родины.
Родившись на сломе эпох, в начале бурного 20 века – в 1898 году, объехав пол
мира, в начале 50-х на нарах румынской тюрьмы кумир довоенной Европы умирал от
голода и пыток. Триумф и трагедия, тайное и явное

.::Коллаж::.

- Сказка ложь, да в ней намёк! Добрым молодцам урок

Перед тем как менять мир, научись менять себя. Чтобы изменить себя, узнай, кто
ты. Чтобы узнать, кто ты, узнай, кто твои предки.

- Гефсиманский сад Василия Григорьевича Перова

В 1866 году Василий Григорьевич Перов создал одну из своих самых знаменитых
картин, которая обросла в истории флёром таинственности, называется она –
"Тройка". Уже работая над картиной, Василий Григорьевич всё никак не мог найти
образ мальчика – центрального в тройке. Такого, который сразу бы привлёк
внимание. И вот, как всегда бывает, когда ищешь, гуляя по городу, художник
увидел бедно одетую женщину с ребёнком, именно таким, которого хотел
изобразить. Он подошёл, познакомился, и женщина по имени Марья рассказала, что
она и её сын Васенька...

- Это ты?

Среди глубокого мрака вдруг отворяется железная дверь тюрьмы, и сам старик
великий инквизитор со светильником в руке медленно входит в тюрьму. Он один,
дверь за ним тотчас же запирается. Он останавливается при входе и долго, минуту
или две, всматривается в лицо его. Наконец тихо подходит, ставит светильник на
стол и говорит ему: «Это ты? Ты?

Автор картины на обложке — Валентина Плавун, автор оборота обложки — Виктор
Зинченко
