% vim: keymap=russian-jcukenwin
%%beginhead 
 
%%file 20_12_2020.news.ua.strana.venk_victoria.1.zelenski_nyt_interview
%%parent 20_12_2020
 
%%url https://strana.ua/news/307719-intervju-zelenskoho-the-new-york-times-20-dekabrja-hlavnye-zajavlenija.html
 
%%author Венк, Виктория
%%author_id venk_victoria
%%author_url 
 
%%tags zelenskii_vladimir,sputnik_v
%%title "Невозможно объяснить, почему мы не берем "Спутник". Главные заявления Зеленского в интервью New York Times
 
%%endhead 
 
\subsection{\enquote{Невозможно объяснить, почему мы не берем }Спутник\enquote{. Главные заявления Зеленского в интервью New York Times}
\label{sec:20_12_2020.news.ua.strana.venk_victoria.1.zelenski_nyt_interview}
\Purl{https://strana.ua/news/307719-intervju-zelenskoho-the-new-york-times-20-dekabrja-hlavnye-zajavlenija.html}
\ifcmt
	author_begin
   author_id venk_victoria
	author_end
\fi

\ifcmt
  pic https://strana.ua/img/article/3077/intervju-zelenskoho-the-19_main.jpeg
  caption Владимир Зеленский. Фото The New York Times 
  width 0.5
  fig_env wrapfigure
\fi

Владимир Зеленский дал большое интервью газете The New York Times.

Самое интересное - признание Зеленского,\Furl{https://strana.ua/news/307713-vladimir-zelenskij-v-intervju-the-new-york-times-rasskazal-chto-dumaet-orossijskoj-vaktsine.html} что Запад не ставит Украину в
приоритет в очереди за вакцинами от коронавируса. Поэтому идея обращения за
российской прививкой "Спутник", судя из слов президента, на Банковой пока не
отброшена. 

Правда, пока эти слова похожи скорее на попытку шантажировать американцев, чем
на реальные намерения купить более доступный для Украины российский препарат.

Показательные заявления прозвучали по Донбассу. Зеленский дал понять, что
никаких уступок местному населению при реинтеграции территорий давать не
намерен.

В том числе усиливать права русского языка. 

Зато Зеленский пообещал не создавать парламентскую коалицию с Порошенко.
"Страна" проанализировала интервью президента, полный текст которого можно
прочитать на русском языке по этой ссылке.\Furl{https://www.nytimes.com/2020/12/19/world/europe/trump-zelensky-biden-ukraine.html?searchResultPosition=1}

\subsubsection{Коронавирус и вакцина}

Зеленский рассказал, что ежедневно просыпается в семь часов утра и начинает
день с чтения сводок о коронавирусе.

По его словам, месяц назад в Украине начиналась вторая волна болезни, которая
добавляла по тысяче заболевших в сутки. Но благодаря действиям властей, волну
получилось сбить, да и вообще - в Украине дела обстоят лучше, чем в Евросоюзе.

"Страна" уже неоднократно разбирала, почему это, мягко говоря, не так.

Вкратце - Минздрав искусственно занижает число тестов и, судя по всему,
подгоняет под них и суточный прирост по заболевшим (детальнее - в материале
"Страны" В Украине снова обвал по заболевшим, но при этом - резкий рост
госпитализаций. Как такое может быть?\Furl{https://strana.ua/news/306503-koronavirus-v-ukraine-pochemu-upalo-chislo-zabolevshikh14-dekabrja-.html}).

Зеленский в интервью еще больше повысил статус достижений своей команды в
борьбе с пандемией, заявив, что "медицинская система в Украине была развалена".

Учитывая, что с 2016 по 2019 год ее "разваливала" министр-американка Ульяна
Супрун, то становится неясным, почему ее реформы при новом президенте свернуты
не были. Их разве что приостановили с оглядкой на коронавирус. 

Далее в интервью журналист обнародовал сенсационный факт - оказывается, Украина
и компания Pfizer остановили переговоры по закупке вакцины от коронавируса.
Причина - запрет на экспорт вакцины из США, пока не будет вакцинировано
американское население. 

Украина об этой истории умалчивала и заявляла, что ведутся переговоры. Издание
поинтересовалось у Зеленского, будет ли ввиду этого Киев покупать вакцину из
России? 

Президент ответил, что если США и Британия поставят Украину в конец очереди за
вакциной, то это будет "геополитический удар" и "сильнейшая информационная
война со стороны России". 

"Конечно, объяснить украинскому обществу, почему если Америка и Европа не дают
тебе вакцину, ты не должен брать ее в России, - любому человеку, который
умирает, объяснить это невозможно", - заявил Зеленский. 

При этом он тут же, противореча себе, объяснил, что брать российскую вакцину
нельзя, поскольку она, по мнению Зеленского, не прошла всех испытаний.

"Мы не должны допускать того, чтобы Украина принимала российскую вакцину,
которая не прошла все испытания. У нас с вами нет реальных доказательств, что
эта вакцина имеет стопроцентный положительный эффект. Более того, я не могу
взять ответственность за вакцину, последствий действия которой мы не знаем", -
заявил он. 

Проверенными вакцинами президент назвал американские Moderna и Pfizer, а также
британскую AstraZeneca. Которую отправили на доработку после провала испытаний
и пока никого ею не прививают. В то время как в России уже началась массовая
вакцинация населения. Более того, AstraZeneca начала сотрудничество с
российским центром Гамалеи и начала прививать испытуемых параллельно вакциной
"Спутник". 

"Страна" уже рассказывала, что не существует объективных данных о том,
"проверена" ли любая вакцина от коронавируса. Кроме того, еще ни одна вакцина
(включая Moderna и Pfizer) не рекомендована ВОЗ. Единственный критерий - это
заявление самих производителей и получение государственной регистрации
препарата.

Но даже до этой регистрации ведущие страны мира уже контрактуют сотни миллионов
доз у производителей. В том числе у России. Формально таким образом все вакцины
пока "не проверены". Но Украину и Зеленского здесь почему-то смущает лишь
российская прививка. То есть в ход идут, судя по всему, только политические
соображения. 

Правда, судя по интервью, идея закупок российской вакцины на Банковой
окончательно не отброшена.

Во-первых, как элемент шантажа Запада: мол, не дадите вашу вакцину - возьмем
российскую. И будет вам "геополитический удар". 

Во-вторых, Зеленский обмолвился, что у западных производителей вакцин есть
заводы в России. И не исключил, что вакцина может прийти также и с них. 

Вот как дословно звучит ответ президента на вопрос журналиста, будет ли Украина
покупать российскую вакцину, если она будет безопасной: "Украина будет бороться
за то, чтобы наши партнеры дали нам вакцину, которая подтверждена - Moderna,
Pfizer, AstraZeneca. Это то, что подтверждено. Мы понимаем, что некоторые из
этих компаний имеют заводы по всей Европе и даже в Российской Федерации,
насколько мне известно. Поэтому здесь вопрос прежде всего в качестве вакцины".

Из указанных Зеленским компаний завод в России есть только у британской
AstraZeneca. Он расположен в Калужской области. Чем, видимо, и объясняется уже
упомянутое сотрудничество этой компании с Россией по вакцинам. 

То есть украинский президент намекает, что российскую вакцину Киев может
получить, если она будет произведена на мощностях AstraZeneca.

\subsubsection{Реформы и коррупция}

В интервью Зеленский делал особый акцент на то, что при нем активизировались
реформы, на которых настаивал Запад (запущен Антикоррупционный суд, принят
закон о рынке земли). 

Также на следующий год он анонсировал судебную реформу, которую также давно
требуют США и ЕС вместе с МВФ, так как она предусматривает передачу полномочий
по отбору судей неким "международным экспертам". То есть фактически переводит
под контроль Запада судебную систему.

Президент вновь обрушился на Конституционный суд и олигархов, которые мешают
этим реформам.

В то же время, как уже писала "Страна", Зеленский и его команда, несмотря на
всю риторику, на деле спускает на тормоза многие пожелания Запада. Связано это
с влиянием на президента крупного украинского бизнеса, который хочет уменьшить
(или как минимум не дать увеличить) степень внешнего управления Украиной со
стороны США и международных финансовых структур.

Типичный пример - история с отменой Конституционным судом уголовной
ответственности за недостоверное декларирование - это был важный рычаг влияния
Запада на украинских чиновников и политиков через подконтрольные
антикоррупционные органы.

На словах Зеленский возмущался решению КСУ, однако по итогу парламент
восстановил ответственность, но в сильно урезанном варианте. Это не понравилось
западным структурам, и президент закон не подписал, намекая, что может его
ветировать. Но до сих вето не наложено, что переносит решение вопроса на
следующий год. И по факту сейчас никакая ответственность за декларирование не
действует (подробнее о ситуации можно прочитать здесь\Furl{https://strana.ua/articles/analysis/305864-chto-budet-s-antikorruptsionnym-zakonodatelstvom-posle-reshenija-venetsianskoj-komissii-.html}).

На Западе уже по этому поводу появляются публикации об "опасной игре"
Зеленского.\Furl{https://strana.ua/news/307218-bloomberg-napisal-o-zelenskom-i-eho-provalakh-na-postu-prezidenta-ukrainy.html} А МВФ затягивает с решением по возобновлению кредитования.

Видимо, чтобы перебить этот тренд в глазах американского истеблишмента (и в
первую очередь команды Байдена), Зеленский, собственно, и дал интервью.

\subsubsection{Отношения с Порошенко и Коломойским}

Зеленскому задали щекотливый вопрос: как быть с тем, что фракция "Слуга народа"
перестала голосовать монобольшинством? Пойдет ли президент на сотрудничество с
партией Порошенко "Европейская солидарность"?

Президент в ответ начал рассуждать о достижениях своей команды - рынке земли,
"Большой стройке", запуске Антикоррупционного суда и так далее. Видимо, таким
образом показывая, что первая часть вопроса не в тему - раскола во власти нет.

Что же до монобольшинства, Зеленский попенял на "влияние финансовых групп". В
том числе на его фракцию влияют "группы Медведчука и Порошенко". Однако,
заверил президент, "законы голосуются, и результат - на табло". 

"С олигархом Порошенко я не готов сотрудничать", - ответил Зе и на вторую часть
вопроса. 

Интересной была формулировка вопроса по поводу олигарха Игоря Коломойского.
Журналист издания напомнил, что бизнесмен критиковал МВФ и грозил разворотом
Украины в сторону России. "Что вы делаете сейчас, чтобы показать, что не
разделяете его взглядов?", - строго спросил журналист. 

Такая постановка вопроса вызвала раздражение даже у Зеленского, которого, по
сути, обязали отмежеваться от слов Коломойского: "Я не уверен, что должен
что-то показывать, потому что показывают обычно те, кто хочет на публику
продемонстрировать недостоверность своих внутренних решений".

Но потом Зе привычно начал расписываться в лояльности Западу. 

"Это заявления одного из бизнесменов, одного из олигархов. Что касается наших
стратегических партнеров, то Украина выбрала себе стратегических партнеров. ЕС,
Альянс (НАТО - Ред.), США - это наши важные партнеры, которые поддерживают нас
и в санкционный политике против российской агрессии в Донбассе", - заявил
президент. 

Однако журналист NYT потребовал от Зеленского снова "показать, что он не будет
влиять на реформу в банковской сфере". Зе ответил, что на нее никто не влияет.

\subsubsection{О новом президенте США }

Не обошлось без вопроса о новом президенте Соединенных Штатов. У Зеленского
спросили, какими он видит отношения с новой администрацией Белого дома. 

Зе дал понять, что Штаты поддерживают Украину при любых президентах, и
напомнил, что именно Трамп первым ввел санкции против "Северного потока - 2".
Но потом отпустил комплимент и в адрес Байдена - причем заявив, что он для
Украины лучше, чем Трамп.

"Новый президент Джо Байден, мне кажется, знает Украину лучше, чем его
предшественник. Потому что еще до своего президентства он, так сказать, имел
глубокие отношения с Украиной и хорошо понимает россиян, хорошо понимает
разницу между Украиной и Россией и, мне кажется, хорошо понимает ментальность
украинцев", - отметил президент. 

\ifcmt
  pic https://strana.ua/img/forall/u/0/92/%D0%B7%D0%B5%D0%BB%D0%B5%D0%BD%D1%81%D0%BA%D0%B8%D0%B9(3).jpg
  caption Владимир Зеленский в своем кабинете. Фото The New York Times
  width 0.5
  fig_env wrapfigure
\fi

Еще один вопрос по США был об участии Украины в конфликте Байдена и Трампа - с
намеком на дело "Бурисмы", в которой работал сын избранного президента и
получал там непонятно за что огромные деньги. 

Трамп пытался сделать эту историю своим козырем против Байдена, но Киев сыграл
в этой истории на стороне демократов. Журналист NYT спросил, считает ли себя
Украина свободной от этого конфликта (читай: будут ли еще предприниматься
попытки очернить Байдена?). 

Зеленский ожидаемо дал уклончивый ответ, говоря, что страну "затягивали в
конфликт", но не вышло. Но дальше начал говорить, что не о том надо думать
Украине и Америке. И увел разговор в космос - напомнив, что надо развивать
совместные космические проекты, которые со времени полета Каденюка почему-то
заглохли. 

"Мы - космическая держава, Соединенные Штаты Америки тоже, мы можем вспомнить
те же девяностые годы, когда у нас был полет космического корабля с нашим
первым украинским космонавтом Каденюком, и тогда вместе с США, с партнерами
готовился кейс. Прошло более 20 лет, а больше каких-то таких открытий в науке,
в искусстве, в технологиях, в медицине, таких глобальных современных вещей не
было. Вот куда надо двигать таланты одной и другой страны", - предложил
Зеленский.

Под конец интервью он даже предложил, чтобы американцы построили в Украине
филиал Голливуда и Диснейленд. А также дали украинским туристам безвизовый
режим. 

\subsubsection{Россия, Донбасс, Нормандия}

Журналист спросил у Зеленского, хочет ли тот, чтобы США вошли в Нормандский
формат и Байден стал играть более весомую роль в переговорах с Россией по
Донбассу. 

Украинский президент рассыпался в комплиментах американцам и сказал, что без
них конфликт на Донбассе не решить и Крым не вернуть.

Кстати, на Крыме Зеленский сделал особый акцент, сказав, что намерен привлечь
американцев к работе с "крымской платформой", чтобы вновь вернуть в мировую
повестку тему аннексии полуострова, которую ранее заслонил Донбасс ("Страна"
уже подробно анализировала,\Furl{https://strana.ua/articles/analysis/304832-krymskaja-platforma-zachem-ee-pridumal-i-s-kem-ona-mozhet-possorit-kiev.html} что такое "крымская платформа" и зачем она нужна). 

Далее Зеленский пожалел, что с Россией перестал вести переговоры
спецпредставитель США Курт Волкер. И предложил вернуть переговоры Москвы и
Вашингтона на уровне таких представителей. 

"Мне кажется, что вот это понимание новым президентом США реальности в России,
в Украине, в Европе даст большую возможность решения всех конфликтных задач", -
предположил украинский президент.

"Страна" уже моделировала ситуацию, при которой Байден действительно может
достичь прогресса в переговорах\Furl{https://strana.ua/news/299500-bajden-pobezhdaet-na-vyborakh-v-ssha-k-chemu-hotovitsja-ukraine-i-rossii.html} с Россией по Донбассу - поскольку его руки, как
у Трампа, не будут связаны на российском направлении. Да и в целом эксперты
Демпартии давно призывали перестать накалять отношения с РФ\Furl{https://strana.ua/news/283256-novyj-kurs-ssha-v-otnoshenii-rossii-chto-predlozhili-v-amerike.html} и свести к нулю
вероятность военных сценариев. 

Другой вопрос - что Минские соглашения не работали уже при демократах. Хотя и
были подписаны в том числе с отмашки администрации Обамы. То есть каких-то
больших шансов на то, что компромисс будет достигнут, нет и сейчас. 

Но вернемся к интервью. На вопрос, готова ли Украина менять конституцию под
Минские соглашения, Зеленский ответил, что лишь в рамках общей децентрализации. 

"Менять Конституцию в том направлении, о котором мы иногда слышим в
медиапространстве от Российской Федерации, - здесь они знают мою позицию, я
говорил Путину прямо, что здесь я не согласен", - сказал Зеленский. 

По словам президента, "мы хотим, чтобы на местах у людей было больше
полномочий, больше выбора, чтобы они распоряжались деньгами... Мы это и так
делаем. Если захотим в этом направлении Донбассу больше преференций, вот такие
компромиссы могут быть", - отметил он. 

Но какими могут быть эти преференции - Зеленский четко не пояснил,
ограничившись общими слова про расширение прав, в том числе и по вопросам языка
(об этом подробнее ниже). Хотя особые права региона и есть главный вопрос
урегулирования на Донбассе. Более того, представители президента в
Трехсторонней контактной группе за более чем год президентства Зе так и не
сформулировали, какие полномочия готовы отдать послевоенным властям региона. 

То есть реальных предложений у Киева к Донецку и Луганску, а также к Москве
просто нет. Что ставит под сомнение искренность заявлений президента о желании
вернуть регион дипломатическим путем. 

\subsubsection{Русский язык и Карабах}

Журналист спросил, нет ли признаков потери Россией влияния на постсоветском
пространстве. И привел в пример Карабах, Молдову и Беларусь. 

Зеленский ответил, что, судя по Украине, Россия своего присутствия не
ослабляет. И заявил, что нельзя сравнивать украинский кейс с Карабахом: по
словам Зеленского, на Донбассе воюют между собой люди одной веры и языка. 

"Сравнивать военный конфликт между Арменией и Азербайджаном и между Украиной и
Россией невозможно, потому что мы понимаем, например, что такое разная
ментальность и разное вероисповедание, разные религии. Это то, что там. Сложный
конфликт основан на этом. Что касается нас - у нас люди, которые защищают
Украину на Донбассе, говорят как на русском, так и на украинском. Вопрос языка
у нас не стоит, у нас люди воюют за независимость своей страны. К тому же у нас
люди воюют с людьми такой же религии", - объяснил Зе. 

Правда, тут же президент обмолвился, что вопрос языка в украинском конфликте не
важен и то, что на Донбассе собирались притеснять русский, - "манипуляции
Российской Федерации".

Напомним, что всплеск пророссийских настроений после победы Майдана в 2014 году
вызвала отмена языкового закона, который давал русскому языку статус
регионального. Без этого решения постмайданной Рады "манипуляции РФ" вряд ли бы
сработали. Это было вообще похоже на провокацию, когда и без того разогретые
протесты на юго-востоке Украины подстегнули запретом на русский язык. 

Противоречит словам Зеленского и нынешняя политика украинизации, которая с этой
осени поставила вне закона все русские школы, а с января переводит на мову всю
сферу услуг.\Furl{https://strana.ua/news/306995-ukrainizatsija-sfery-usluh-chto-izmenitsja-s-16-janvarja-2021.html}

При этом, упоминая от Минских соглашениях, президент туманно дал понять, что не
против усиления роли русского языка на Донбассе. 

"Я понимаю, что мы хотим отдельные полномочия, право выбора, чтобы никто потом
не спекулировал на вопросах языка, чтобы жители Донбасса разговаривали, как они
хотят, - пожалуйста, на все эти вещи, я считаю, что мы должны быть максимально
либеральными", - сказал Зеленский. 

Правда, пока признаков такого либерализма незаметно. Школы подконтрольной части
Донбасса украинизируются ударными темпами и переводились на мову даже до
вступления в силу языковых норм. Не делается для региона исключений и по другим
направлениям.

То есть, по сути, это просто слова. И если понимать их буквально, то читается,
что в обмен на возвращение Украине остальной части Донбасса, Зеленский согласен
позволить людям "разговаривать" на родном языке. 

Однако на самом деле речь идет о другом - о праве вести бизнес, учиться или
потреблять культурный продукт на языке, который является родным не только для
Донбасса, но и для многих регионов Украины. В том числе и для самого Владимира
Зеленского. 

Но для этого нужно как минимум отменить принятый в апреле 2014 года закон о
тотальной украинизации. Однако в этом направлении никаких движений команда
Зеленского так и не сделала.

\subsubsection{Выводы из интервью}

Телеграм-канал "Политика Страны" делает краткое резюме из слов президента.

"По интервью Зеленского The New York Times. Краткий пересказ содержания.

\begin{itemize}

\item 1. Отчитался о победной поступи согласованных с Западом реформ (включая
				анонсированную им на следующий год судебную реформу), несмотря на
				сопротивление олигархов и судей КСУ. Смысл - хочет получить новые
				транши МВФ, хотя многие условия фонда не выполнены и, скорее всего,
				выполнены не будут (включая судебную реформу).

\item 2. Много ругал олигархов, давая понять, что приоритет для него -
				ориентировки "западных партнеров". Смысл - хочет получить новые транши
				МВФ, хотя крупные олигархи сейчас имеют определяющее влияние на его
				Офис.

\item 3. Сказал, что Коломойский не влияет на политику государства, хотя и без
				какой-либо конкретики. Смысл - хочет получить новые транши МВФ и вообще
				понравиться команде Байдена (где Коломойского не любят), хотя олигарх
				продолжает пользоваться многими преференциями (в частности, по
				"Центрэнерго") и пытается сейчас ещё больше укрепить своё влияние,
				проталкивая Витренко на пост первого вице-премьера.

\item 4. Много хвалит Байдена, который, по его словам, "глубоко в теме" по
				Украине. Смысл - хочет получить новые транши МВФ и вообще понравиться
				команде Байдена, за внимание которой конкурирует с другими группами в
				Украине (в первую очередь с Порошенко и "соросятами").

\item 5. Говорит много плохого про Россию, обещает не идти на выполнение
				политической части Минских соглашений (то есть не вносить особый статус
				в Конституцию). Смысл - показывает, что он не русский шпион (о чем
				постоянно рассказывают Вашингтону Порошенко и Ко).

\item 6. Торопит (\url{https://t.me/stranaua/10376}) Запад с поставками вакцины в
				Украину. Смысл - на Банковой отдают себе отчёт, что если с поставками
				западных вакцин в Украину будет проблема, но при этом наша страна по
				политическим причинам и под очевидным давлением США откажется брать
				российскую вакцину, то это будет очень сильным ударом по президенту и
				вообще по прозападного курсу страны.

\item 7. Сказал, что хочет построить в Украине Диснейленд. Смысл - привнести
				чуток няшности в интервью и чтоб "хоть что-то для людей".
\end{itemize}
