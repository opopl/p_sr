% vim: keymap=russian-jcukenwin
%%beginhead 
 
%%file 30_04_2020.fb.fb_group.story_kiev_ua.2.kievskie_mozaiki.pic.27
%%parent 30_04_2020.fb.fb_group.story_kiev_ua.2.kievskie_mozaiki
 
%%url 
 
%%author_id 
%%date 
 
%%tags 
%%title 
 
%%endhead 

\ifcmt
  ig https://scontent-frt3-1.xx.fbcdn.net/v/t1.6435-9/94308969_3174547935912125_1549512041620832256_n.jpg?_nc_cat=104&ccb=1-5&_nc_sid=b9115d&_nc_ohc=ktx-aud7aKMAX-X1tJA&tn=lCYVFeHcTIAFcAzi&_nc_ht=scontent-frt3-1.xx&oh=b978977ef31c3c5d678b6cca8ac280f1&oe=61B40735
  @width 0.4
\fi

\iusr{Ирина Петрова}
Фонтан Зірки та сузір'я

\iusr{Nadiya M Shana}
А де ж це він?

\iusr{Наталия Меженная}
\textbf{Nadiya M Shana} КПДЮ, київський палац дітей та юнацтва (Центральний палац піонерів)

\iusr{Nadiya M Shana}
\textbf{Natali Mezhenna} Дякую!

\iusr{Nina NinaNina}

вул. Івана Мазепи, 13, фонтан перед Центральним палацом дітей та юнацтва почали
реставрувати у 2017, закінчили у 2019, до цього фонтан не працював майже чотири
десятиріччя. Комплекс фонтанів, як і мозаічні панно, були створені Адою Рибачук
та Володимиром Мельниченко
