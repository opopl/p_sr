% vim: keymap=russian-jcukenwin
%%beginhead 
 
%%file 11_10_2018.fb.lesev_igor.1.ukr_voen_doktrina
%%parent 11_10_2018
 
%%url https://www.facebook.com/permalink.php?story_fbid=2123801400984285&id=100000633379839
 
%%author_id lesev_igor
%%date 
 
%%tags doktrina,donbass,ukraina,vojna
%%title В чем вообще смысл военной доктрины Украины?
 
%%endhead 
 
\subsection{В чем вообще смысл военной доктрины Украины?}
\label{sec:11_10_2018.fb.lesev_igor.1.ukr_voen_doktrina}
 
\Purl{https://www.facebook.com/permalink.php?story_fbid=2123801400984285&id=100000633379839}
\ifcmt
 author_begin
   author_id lesev_igor
 author_end
\fi

Разбанили, но уверен, это где-то на дня три. Успею написать 2-3 поста, затем
найдут какой-то коммент из 2016 года и досвидос еще на месяц. ФБ делает все,
чтобы пипл сваливал на другие площадки. Пока что весомых альтернатив нет, но
запрос уже огромный. Смысл сидеть на площадке, у которой нет никаких внятных
правил игры?

\ifcmt
  ig https://scontent-frt3-1.xx.fbcdn.net/v/t1.6435-9/43878446_2123801287650963_892094888114388992_n.jpg?_nc_cat=107&ccb=1-5&_nc_sid=730e14&_nc_ohc=N_KZVlOU2fcAX-dwsAM&_nc_ht=scontent-frt3-1.xx&oh=74b7e310650dba09380ad05f32c9d40b&oe=61B9D65E
  @width 0.4
  %@wrap \parpic[r]
  @wrap \InsertBoxR{0}
\fi

Ну а пока еще не забанили, рассуждение на серьезную тему. Сегодня пошли сводки
с Донбасса – убито четверо военнослужащих Украины. Это четыре похоронки. Четыре
убитых матери. А если пацаны успели сделать детей, то и они теперь никому,
кроме своих матерей, не нужны.

Но если отойти от лирики и перейти к государственному цинизму. А какой смысл в
этих смертях? Какой смысл в постоянной эскалации на своей же территории, на
которую та же Россия юридически не претендует? В чем вообще смысл военной
доктрины Украины?

В мире более 200 государств и у каждого свой резон в наличии армии. Но для
подавляющего большинства стран мира армия – это обуза. Это как полное КАСКО или
добровольные пенсионные отчисления. Вроде как понимаешь, что надо, но есть
сомнения, потому что дорого.

Для геополитических игроков армия – это эффективное средство продвижения на
внешних рынках. Это как для таксиста машина. Ты на армии зарабатываешь.
Например, текущий военный бюджет РФ составляет очень скромные 46 ярдов зелени.
Меньше, чем у Франции, Британии и даже саудитов. С американцами даже сравнивать
не стоит. 15-кратное отставание. Но у русских пакет военных заказов превышает
50 миллиардов долларов. А это значит, что работает сотни смежных предприятий.
Т.е., есть какая-то экономическая движуха. Плюс нахрапом берутся новые рынки.
Сирия. Сейчас будет, возможно, Ливия. Иными словами, траты на оборонку и
ведение сразу нескольких войн для РФ как-то обоснованы. Это же относится,
только в гораздо больших масштабах, и к Штатам.

В мире есть с пару десятков стран, для которых наличие сильной армии – это
гарантия сохранения своего суверенитета, а также сдерживающих фактор для своих
недружелюбных соседей. Египет, Израиль, Турция, Греция, Иран, Саудовская
Аравия, Пакистан, Индия, Вьетнам, Тайвань... У этих стран есть устоявшийся кейс
из постоянных срачей и войнушек различной интенсивности и поэтому сильная армия
им нужна как воздух. Они же и являются основными покупателями вооружения у
Штатов и России. Но у всех этих стран есть еще одна важная особенность – они
обладают высокой степенью суверенитета. Условный Байден или Медведев у них не
может заседать во главе правительства. И военные конфликты эти страны
развязывают по своему внутреннему усмотрению. Правда, завязывают, как правило,
уже под давлением более сильных внешних игроков.

Но даже эти страны, при чудовищных тратах на оборонку, не ведут беспорядочные
войны со своими недружелюбными соседями. Например, Кипр. Посрались турки с
греками, чуть постреляли, зафиксировали линию разлома и все – тишина. Тайвань и
Китай. Были разные истеричные всплески. Но не более того. Индия и Пакистан. Три
войны. Перманентно накидывают говно на вентилятор с интенсивностью раз в 5 лет.
Десантура, стрельбы, локальные наступление, гневные заявления и... все. 3-4 дня
войны раз в пять лет. Проверяют бдительность. Но там не ипошат друг друга
ежедневно и на протяжении многих лет.

Другой тип военных бюджетов. Европейский. Эти ребята реально не понимают,
нахрена им тратить деньги на оборонку. «Русская угроза» с 46 ярдами бюджета
впечатляет кого-то в Таллине, но не в Берлине. Но главная причина пассивности
европейцев – это отсутствие геополитической субъектности. Это как выделять
деньги на содержание бабы, зная, что ее трахает кто-то другой. Какой смысл
немцам тратиться на оборонку, если они не принимают решения даже в том, могут
или нет на ее территории находиться американские базы?

К слову, для Франции и Британии – официальных членов «ядерного клуба» - эта
проблема имеет тот же характер. Геополитическая субъектность Парижа и Лондона
закончилась в 1957 году во время Суэцкого кризиса, когда те без согласования со
старшим американским братом полезли в Египет. Тогда Вашингтон и Москва вместе
(!) показательно трахнули французов и британцев. И хотя Лондон и Париж
продолжают тратить на оборонку гигантские суммы, обоснованность этих трат для
их же внутренних элит не совсем понятна.

Есть еще один тип оборонных бюджетов – латиноамериканский. Ребята когда-то
основательно срались друг с другом, но в основном все крупное было в самом
начале ХХ века. Потом дерзили только всякие карлики, вроде Гондураса и
Сальвадора. Но в Латинской Америке армия – это встроенная каста в саму
политическую систему государств. Хочешь стать президентом – иди не в
парламент/политическую партию, а поступай в военное училище. С 90-х годов эта
традиция уходит в прошлое, но само допущение неконституционной смены власти у
латиноамериканцев в традиции. А потому траты на армию там – это больше фактор
внутренней политики, а не попытка воздействия на своих соседей.

Ну и последний тип оборонных бюджетов – африканский (тот, что южнее Сахары). И
вот здесь Украина находит своих собратьев по уму и оружию. В Африке живут
крокодилы, удавы и дебилы. Самые опасные последние. Потому что они постоянно
воюют. При этом война для них – такой же процесс, как поглощение шаурмы или
секс в подъезде. Вроде неэстетично, но почему бы и нет? Все африканские войны –
это череда бессмысленности и алогичности. Сомали напала на Эфиопию, чтобы
создать Великое Сомали, но в результате превратились в Большое Говно из десятка
бандитских государств. Сама Эфиопия трахалась со своей приморской провинцией
Эритреей и дотрахалась до потери выхода к морю. Эритрея получила независимость
и гражданку с кучей трупов и невозможностью жить по-человечески. Зато Эритрее
помогал соседний Судан, но допомогался до образования Южного Судана... И это
только краткая история Африканского Рога.

В Африке живут самые тупые люди на Земле. Они так и не создали ни одного
комфортного для проживания общества. Ни одной эффективно работающей экономики.
Была Южная Африка, да и ты скоро сплывет вместе с последним эмигрирующим буром.
И очевидно, что самые тупые и бессмысленные в мире люди ведут самые тупые и
бессмысленные войны в мире.

Война на Донбассе в этом плане очень африканская. Мы тратим на оборонку по
нашим меркам непростительно дикие деньги. Это деньги проедания. Деньги, которые
не становятся потом новыми деньгами. У нас стремительно сокращаются оборонные
заказы за рубежом. Мы не решаем геополитических и даже тактических задач этими
тратами. Мы сокращаем свои трудовые ресурсы, сокращая тем самым и ВВП. Наконец,
мы воюем на своей территории. И уничтожаем свою же инфраструктуру, на которую
же и претендуем.

Я бы еще понял, если бы киевские власти подписали какие угодно соглашения с
Россией, а потом тыдыщ – и обманули ее. И вместо амнистии сепаратистам –
зачистки в Донбассе и факельные марши по Донецку. В этом ведь есть какая-то
логическая последовательность.

Я бы также понял, если бы ЛДНР обнесли колючей проволокой и повесили на воротах
волосатую дилду. Мол, ипитесь сами. А мы вкладываем бабло в инфраструктурные
проекты. И в то, что в среднесрочной перспективе нам принесет бабло. А там где
бабло – там всегда любовь.

Но в чем смысл вот такой войны, которая работает исключительно на истощение
только Украины, непонятно. Хотя нет, есть одно предположение. Возможно,
некоторые гондоны просто тырят львиную часть оборонного бюджета – статьи ведь
все равно закрыты, тендеры непрозрачны и хер что ты проверишь. А до армии
доходят только крохи. А чтобы держать пипл в тонусе, нужно воевать «до
победного конца». Но воевать там, где нет угрозы настоящей войны. На Крымском
перешейке или на границе с Курской областью, заметьте, никто не стреляет.

А вот матерям погибших пацанов скажут, что их дети были героями. И дадут
посмертно ордена. Был сын, а теперь у мамы в коробочке орден. Адекватная такая
замена. А те, кто ордена заказывает – у тех дети учатся в Лондоне, а виллы на
Сардинии. Не у всех же дети герои.
