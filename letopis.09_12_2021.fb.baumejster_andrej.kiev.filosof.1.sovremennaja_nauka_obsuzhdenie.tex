% vim: keymap=russian-jcukenwin
%%beginhead 
 
%%file 09_12_2021.fb.baumejster_andrej.kiev.filosof.1.sovremennaja_nauka_obsuzhdenie
%%parent 09_12_2021
 
%%url https://www.facebook.com/andriibaumeister/posts/4517677645020426
 
%%author_id baumejster_andrej.kiev.filosof
%%date 
 
%%tags chelovek,filosofia,miroporjadok,nauka,teologia
%%title Что происходит с современной наукой?
 
%%endhead 
 
\subsection{Что происходит с современной наукой?}
\label{sec:09_12_2021.fb.baumejster_andrej.kiev.filosof.1.sovremennaja_nauka_obsuzhdenie}
 
\Purl{https://www.facebook.com/andriibaumeister/posts/4517677645020426}
\ifcmt
 author_begin
   author_id baumejster_andrej.kiev.filosof
 author_end
\fi

Что происходит с современной наукой? Почему современная наука вырождается в
агрессивный сциентизм и вслед за Богом выбрасывает \enquote{с корабля современности} и
человека? Такой сциентизм сам становится квази-религией. 

Может ли в ближайшее время возникнуть новый синтез естественной науки,
философии и теологии? 

\ifcmt
    pic https://external-mxp1-1.xx.fbcdn.net/safe_image.php?d=AQFckONRibkF-Wer&w=500&h=261&url=https%3A%2F%2Fi.ytimg.com%2Fvi%2Fh2cWKWp0AkU%2Fmaxresdefault.jpg&cfs=1&ext=jpg&_nc_oe=6f314&_nc_sid=06c271&ccb=3-5&_nc_hash=AQEHs7Er8fitm2Ds
    @width 0.4
\fi

Почему сегодня вместо свободы нам предлагают \enquote{безопасность} и \enquote{спокойствие}?
Чем опасна \enquote{мягкая} и ползучая тирания наших дней?  

Мы начали разговор с обсуждения главных идей статьи Алексея: 

\href{https://snob.ru/profile/27355/blog/170135}{%
Опасная мудрость Запада I, Алексей Буров, snob.ru, 13.09.2020%
}

\begin{multicols}{3}
\setlength{\parindent}{0pt}

Помещаемый ниже текст представляет собой первую треть статьи, разбитой для
удобства чтения ее публикатором Валерием Петровичем Лебедевым, редактором
веб-альманаха «Лебедь». 

\headTwo{ЧЕЛОВЕК КАК ВОПРОС}

Что составляет сущность человека, его самое главное? Есть немало глубоких
ответов на этот вопрос, объединенных импликацией его важности. А раз так, раз
важность вопроса первична в отношении любого конкретного ответа, то не
содержится ли ядро ответа уже в вопросе? Иными словами, нельзя ли определить
человека через вопрос о себе, как ищущего свое определение, истину о своем
бытии? Помимо определенной элегантности, такой самореферентный ответ был бы
поддержан не только авторитетом Дельфийского божества с его вечно-волнующим
познай самого себя, но и общим центром всех великих учений. Накопленная
человечеством мудрость в конечном счете антропоцентрична; в ее фокусе всегда
сияет один и тот же инвариант — вопрос о сущности человека, о том, кто я. Но
позвольте — могут здесь возразить — ведь люди редко так уж прицельно задаются
этим вопросом; люди в основном заняты конкретными задачами: работают, учатся,
заботятся о близких, и так далее, так что на такие философские вопросы обычно
не остается ни времени, ни возможности; и что же, они от того уже и людьми
считаться не должны? Всё верно, ответил бы я, но если человек уже до такой
степени заземлился, что стал полностью глух, безразличен к вопросу о своей
сущности, то не следует ли его считать заживо умершим или, как минимум, крепко
спящим? Не знаю кто как, но я бы дорого дал, чтобы Бог избавил меня от встреч с
подобными зомби, безнадежно спящими как-бы людьми. Какие бы ни были у человека
проблемы с работой, семьей, государством, каковы бы ни были его познания по
физике или истории, разделяй он ту, иную, или никакую религию — встреча с ним
может раскрыться как благо. Но если этот человек столь беспробуден — нет, пусть
мои пути никогда не пересекутся с подобными существами. Прав был Эрнест Ренан:
«в действительности индивидуум является в большей или меньшей мере человеком, в
большей или меньшей мере сыном Божьим». Остается актуальным комментарий Жюльена
Бенда, данный этому высказыванию почти век назад в «Предательстве
интеллектуалов»:

Современные эгалитаристы, не понимая, что равенство может быть лишь в области
абстрактного, что сущность конкретного — неравенство, показывают, кроме своего
недопустимого политического невежества, чрезвычайную грубость ума.

Хорошо, примем такую дефиницию человека, через вопрос о себе, как его центр и
сущность. Значение вопроса становится предельным. Предельно значимые вещи имеют
особое наименование — святыни. Просто ценные вещи можно менять одну на другую,
при ценовом эквиваленте. Святыня же цены не имеет, или таковая бесконечна.
Святыни не обмениваются ни на что, даже на другие святыни; а если все же
приходится выбирать между ними, то вопрос оказывается трагически неразрешимым.
Подчеркнем, что данная святыня решительно отличается от частных, особенных
святынь, ограниченных индивидуальными, семейными или национальными границами.
Святыня человека как такового, к которой принадлежит и вопрос о нем,
безусловна, универсальна, категорична. Она и задает кантов категорический
императив: «поступай так, чтобы ты всегда относился к человечеству и в своём
лице, и в лице всякого другого так же как к цели, и никогда не относился бы к
нему только как к средству».

Итак, вопрос о сущности человека столь же свят, сколь и сама сущность,
человечность сама по себе, ибо одно неотделимо от другого. Вопрос, однако же,
требует ответа или ответов; иначе он бессмыслен. Что же может быть здесь
ответом? А что не может? Безусловная святыня есть универсальная бесконечная
ценность, а стало быть, ее уничтожение есть, со всей категоричностью,
бесконечное злодеяние. А раз так, то категорически неприемлемы те ответы, что
уничтожают вопрос, снимая его, дискредитируя, или закрывая.

К такого рода ответам относятся все отсылки к произвольности, случайности или
бессмысленности ответов. Например — человек есть то, кем ему вольно себя
определить. Если сущность человека — не более, чем его произвольное мнение,
если все утверждения о сущности равноправны, то они и равно-ничтожны, будучи
приравнены к произвольным фантазиям и капризам. Есть и объективистские варианты
уничтожения человека — например, сциентистский: человек есть результат
эволюции, законов природы и случайности. О природе законов, почему они именно
таковы, почему они оказались познаваемы объявившимися необходимо-случайными
существами, при этом либо не спрашивают, либо полагают законы тоже делом
случая. Бессмысленные законы и случай оказываются ответом на вопрос о человеке,
бессмысленном детище незнамо откуда взявшихся безумных сил. Вопрос оказывается
снят через дискредитацию, что и обнаруживает, стало быть, ложность такого
ответа. Злодеяние, стоящее за этим ответом, не обязательно входит в намерение
отвечающего, но оно присуще ответу по существу: человек таким ответом уже
уничтожен, став столь же случайным и бессмысленным, как и любой ответ о нем.
Философское основание любому преступлению тем самым уже подведено.

Другой способ снятия вопроса — его табуирование. Вопрос может оказаться в числе
неприличных, или слишком опасных, или всегда неуместных, так или иначе
нехороших. Мол, да, проблема есть, но говорить о ней — признак дурновкусия,
глупости и невоспитанности. Или — да, вопрос есть, но есть и правильный ответ:
бороться со всем прогрессивным человечеством за все хорошее против всего
плохого, и лучше всего на этом тему закрыть, ради нашей с вами безопасности.
Люди, принимающие такие варианты, совершают философское самоубийство,
превращаются в зомби — тем вернее, чем надежнее их табу.

Вопрос может уничтожаться и отключением ядра личности, откуда он исходит —
через наркомании, иррациональные экстатические культы, любые формы
самозабвения, вплоть до самоубийства. Особым способом устранения вопроса
является подключение к тоталитарным группам, где индивидуальность теряется,
растворяясь в подчиненном лидеру едином «мы». Идеология таких групп может
носить самый разнообразный характер, от псевдо-мистического до
псевдо-рационалистического, от эзотерически-замкнутого до
утопически-революционного. 

Но каким же может быть ответ на вопрос о человеке — совместимый с его
достоинством, святостью его сущности? Сказанное означает, что ответ должен
даваться, но не может исчерпываться. Он не может состоять из конечного набора
элементов, ибо все конечное ничтожно по последнему счету. Ответ должен быть,
быть важным и содержательным, и вместе с тем не вполне быть — оставаться
открытым, незаконченным, требующим существенных дополнений. Вопрос, получивший
законченный ответ, перестает быть вопросом. Ответ должен быть сразу и
универсальным, и индивидуальным. Если он лишь универсален — он безразличен ко
мне, и мне останется лишь отплатить ему тем же. Если он лишь индивидуален — он
уничтожается временем. Ответ, уничтожаемый временем, обнуляет и самого
человека, независимо от того, идет ли речь о времени жизни человека или
вселенной. А коли так, то уничтожаемый временем ответ должен быть признан
ложным и потенциально гибельным.

Ответы, так или иначе реализующие обозначенные требования, даются определенными
религиозно-философскими учениями. Наиболее мощные комплексы некоторых из них
порождают цивилизации, развиваясь далее внутри них. По цивилизациям можно
судить об их комплексах учений. Цивилизация есть опыт применения такого
комплекса; в ней раскрываются его истины и заблуждения. Учения можно судить по
их цивилизациям как по плодам — по плодам их узнаете их. В определенном смысле,
цивилизация есть эксперимент, в отношении которого ее комплекс учений играет
роль теории. Истина теории раскрывается через эксперимент и наблюдение. 

\headTwo{ЦИВИЛИЗАЦИЯ КАК ОТВЕТ}

Подчеркнем еще раз: цивилизация, взятая вместе со своим генетическим комплексом
учений, своей мудростью, может пониматься как ответ на вопрос о человеке, о
смысле, или смыслах, жизни, ее высших задачах, ценностях, святынях. Отношение
индивидуума к святыне — служение и жертва, прежде всего. Создание и развитие
цивилизации — дело трудное и долгое, требующее многовекового конструктивного
сотрудничества, жертв сегодняшним ради будущего, в конечном счете — ради высших
святынь. Все это требует твердых нравственных качеств, которые и задаются
базовыми учениями. Если эти качества оказываются слабы, если базовые учения с
их святынями перестают восприниматься всерьез — цивилизация будет разрушаться и
гибнуть. Строительство цивилизации, подчеркнем еще раз — долговременная
коллективная жертва или инвестиция. Зачем горожанам веками строить
величественный собор, требующий немалых средств? Не лучше ли пропить и проесть
эти денежки, погулять, пока живы? Зачем судье судить справедливо, если за
несправедливый приговор хорошо заплатят, а за справедливый могут и убить?
Почему бы служащему не прикарманить из казны, если не заметят? Зачем ему вообще
служить старательно, если и так сойдет? Почему бы в случае финансовых
затруднений человеку и не ограбить кого-то? Почему бы полицейским не приехать
попозже на место преступления, не рискуя уже ничем, а то и войдя в число
бенефициаров преступного мира? Эти и подобные им деструктивные мотивации
многократно усиливаются бременем накапливающихся, смешанных с завистью, обид,
ресентимента — на близких и дальних, а в особенности на тех, кто преуспел
несправедливо, будь эта несправедливость реальной или иллюзорной. Зачем я буду
вкладывать силы в мир, если он весь во зле, если преуспевают и правят лишь злые
и бесчестные? Начни копить обиды — и дело пойдет… Важнейшая сторона мудрости
цивилизации — снятие ресентимента.

Ресентимент неразрывно связан со свободой, чьи возможности кажутся загубленными
злыми людьми или несправедливыми небесами. Поэтому один из традиционных путей
его снятия — принципиальное избавление от свободы, акцент на неколебимом
послушании гармонии природы и традиции, на безусловном следовании рутине или
мудрости руководства. Это лекарство может быть весьма эффективно, но у него
есть существенный дефект: подавление свободомыслия, а стало быть, и
общественного развития. Учения послушания либо замораживают достигнутый
общественный уровень, либо ведут к постепенной деградации; таковы доминанты
традиционных религий ближнего и дальнего Востока. Второй путь освобождения от
ресентимента реализован учениями отречения: все материальные обстоятельства
рассматриваются как иллюзорные или несущественные. Свобода здесь вполне может
сохраняться; ресентимент, однако, теряет свое жало вместе со значением
материального мира. Таковы по преимуществу учения Индии, включая буддизм;
таковы же платонизм и гностицизм. 

\headTwo{ОСНОВАНИЕ ЗАПАДА}

Генетический комплекс учений Запада, его особая мудрость, включает Библейскую
веру, Греческую философию и связанные с ними Греко-Римские политические идеи и
практики универсального государства и права. Кроме того, в становлении Запада
важную роль играл феодальный плюрализм, разделение центров церковной власти и
многочисленных центров власти светской. В силу исторических обстоятельств, это
разделение произошло на Западе Средиземноморья и не произошло на Востоке.
Уникальный творческий импульс, задаваемый противоречивым синтезом этих начал в
условиях политического плюрализма Запада, находил свое выражение в росте
университетской учености, развитии искусств и ремесел, географических и научных
открытиях, техническом прогрессе и расширении сферы свобод Западной
цивилизации.    

Религия Запада, Христианство, уже есть определенный синтез Библейской веры и
Греческой философии. Как и другие великие религии, она содержит начала
послушания и отречения. Запад, однако же, не стал бы тем, что он есть, без
третьего начала христианства, творческого. Это третье начало выражено в
сказаниях о сотворении мира как хорошего и прекрасного, о сотворении человека
по образу и подобию Создателя и о шаткости и слабости свободной человеческой
души, о трудных задачах человека, его восхождениях и падениях, о действиях
Бога, нацеленных на  возвышение человека. Это начало выражено в образе
воплощенного Бога-Сына, мудрого и кроткого учителя, в жертве Его Собой ради
спасения каждого из нас. В силу этого начала, человеку не должно уходить ни от
мира, на что склоняет второе, индо-гностическое начало, ни от свободы, на что
склоняет первое, ближне-дальне-восточное, но принимать мир как трудную задачу,
разделяя с Небесным Отцом ответственность за всё и всех. Все обиды здесь могут
быть сняты. Вопрос за что мне такое от Тебя? замещается другим — что я еще могу
сделать вместе с Тобой и ради Тебя? Среди всех элементов Св. Писания христиан
нет ничего значительней того короткого текста, что положено произносить хотя бы
трижды в день: после пробуждения, перед основным приемом пищи и перед сном —
молитвы Господней, «Отче наш». Этот самый главный текст христианства, прямое
обращение к Богу как Отцу, указывает лишь на одно конкретное дело
произносящего, «и остави нам долги наши, яко же и мы оставляем должникам
нашим», и указывает столь эффективно, что тем дело и исполняется. Если прощение
Богом моих проступков зависит от прощения мною моих злодеев и обидчиков, то что
и говорить о зависти к тем, кто мне вообще ничего дурного не делал. Обращаться
к Богу с такой молитвой и лелеять при этом в душе злые чувства — просто
невозможно.

Библейское учение о мире как произведении всеблагого Творца и о человеке как
созданном по образу и подобию Бога имело первостепенное значение для развития
естествознания. Оно задавало безграничную возможность познания Космоса и его
святость, как особой формы причастия Небесному Отцу. Греческий Рационализм
перешел в Средневековье своими обоими составляющими, Платоновой и
Аристотелевой. Аристотелевский дискурс вел к эмпирическому познанию, к
детальным рассмотрениям и классификациям природы, культуры и социума;
Платоновский же побуждал к поискам универсальных математических теорий, как
самих по себе, так и в плане естествознания. Оба направления развивались в
росшей общеевропейской сети университетов и научных обществ, изъясняясь вплоть
до эпохи Просвещения преимущественно на едином языке учености, Латыни. Научный
эксперимент ставил запрос на технические изобретения, побуждая и к
изобретательству для практических нужд.

Величайшая опасность для Христианства содержится, однако же, в нем самом.
Расширение пространства свободы и веры в разум с неизбежностью приводили к
критике Христианства — вначале его институтов, из чего родилась Реформация, а
затем и самих религиозных основ, что привело к Просвещению с его деизмом,
скептицизмом и атеизмом. Глубокие и таинственные истины Христианства скроены из
иной материи, чем представляющиеся ясными и отчетливыми истины разума.
Поверившим в последние как в основание всего и вся, ослепленным их ясностью,
первые будут казаться нелепыми и абсурдными предрассудками. Уничтожая
'предрассудки', они будут губить самое основание цивилизации, сердцевину
понимания человека, его высших смыслов и святынь, превращая тем самым человека
в безумного монстра, а социум — в дом умалишенных. В определенной степени
таковой была уже Французская Революция XVIII века; в еще большей степени
таковыми были большевицкая и все марксистские революции.

Подлинная мудрость соткана из тех начал, которые Нильс Бор предложил называть
дополнительными, представленными через инь и янь на его рыцарском гербе, как
строгость и милосердие, как дерзновение и покорность, как вечное и творческое,
как свобода и предопределенность, как логос и мифос, игра за белых и игра за
черных. Когда одно из дополнительных начал сверх меры усиливается за счет
другого, мудрость оборачивается глупостью, бессмысленностью или безумием.
Великие умы Нового Времени, основоположники физики, от Галилея до Ньютона, не
были фанатиками разума; книга божественных откровений читалась ими с тем же
святым вниманием, как и книга природы. Декартова философия начинается с
постановки вопроса о достоверности знания. При всей глубине и смелости его
размышления, оно касается истин веры лишь в одной, философски необходимой
точке: Бог не обманывает и не даст нам стать безнадежными жертвами обмана и
заблуждений, откуда бы таковые ни исходили. Никаких иных теологических вопросов
Декарт философски не рассматривал. Католическое учение он брал на веру, молясь
Пресвятой Деве как добрый католик, а не как блюститель ясности и отчетливости.
В грандиозном успехе физики Ньютона нетрудно, однако же, было усмотреть
сплошное торжество разума в космическом масштабе — особенно людям
узко-рационалистического склада ума. Вольтер прямо писал, что небесная механика
Ньютона доказывает существование Бога — но не того, что слышит молитвы, а
Мастера, собравшего вселенную как часы, что идут потом сами собой. Не слишком
удивительно, что эпоха Просвещения, воодушевленная успехами естествознания,
уверенно поставила ясный и отчетливый разум высшим судией всех без исключения
истин. Разум, как известно, не терпит противоречий, а мудрость, с ее
дополнительностями, вся из их сопряжений и состоит; так что провал мудрости на
экзамене разума был предопределен.

\headTwo{РЕДУКЦИЯ МЫШЛЕНИЯ}

Культ разума, провозглашенный Просвещением, потребовал объяснять мир и человека
в ясных и отчетливых терминах, через отыскание понятных научных механизмов
возникновения и существования. Закон Всемирного Тяготения прекрасно объяснил
орбиты планет — вот в таком духе и следует мыслить о чем бы то ни было. Что же
до темных вопросов — почему законы такие, а не другие, почему они оказываются
познаваемыми, в чем смысл и ценность познания,  кто такой субъект и откуда он
взялся, что такое добро и зло — эти и подобные им вопросы надо оставить поэтам
и попам, а серьезным мыслителям даже и не к лицу отвлекаться на них. Темные
вопросы разумом неразрешимы, а стало быть и значения не имеют, одна пустая
трата времени с ними, да еще и с опасными конфликтами впридачу. Культ разума,
таким образом, редуцировал мышление до рационально-методического, соотносимого
с универсальными наблюдениями. Обо всем, включая и самого человека, требовалось
думать именно в таком ключе: найти кубики, из которых состоит объект познания,
и найти закон взаимодействия этих кубиков — вот такие задачи только и могут
иметь смысл. Законы природы надо объяснять из еще более общих законов, пусть
пока и неизвестных, а откуда самые общие законы взялись — нет смысла
спрашивать. Может, через тысячу лет наука найдет такой смысл — ну тогда и
поговорим. Ценности можно описывать тоже через какие-то кубики — выживаемость,
биологическая предрасположенность, случайность — но утверждать истинность одних
ценностей против других нет никаких научных оснований. Наука ничего не может
утверждать об истинности или ложности должного, только объяснять представления
о должном через элементы, из которых они сложились под действием определенных
безличных причин. Химия скажет о молекулярном составе человека, анатомия — о
его основных органах, когнитивистика — об алгоритмах обработки данных
мозгом-компьютером, психология — о механизмах психики, социология покажет
работу этих алгоритмов и механизмов на уровне общества. Вот вам и ответ, кто
есть человек. Физика объясняет вселенную, а биология — жизнь. Дарвин объяснил
происхождение видов. Это не то, что вы искали? Ну, идите тогда к попам и
поэтам, они наговорят всего, чего хотите. А серьезные ученые дают вот такие
строгие ответы, даже если они кому-то и не нравятся. Да, много еще
неизвестного, но любознательные учёные работают, находят новые ответы, а из них
следуют новые вопросы.  

Таков порожденный культом разума сциентизм, редукция мышления до научного и
наукообразного. Будучи оборванным и некогерентным мышлением, сциентизм захватил
академические массы не в силу своих интеллектуальных или философских
достоинств, но строго наоборот — в силу своей узости, сосредоточения на
методическом познании и отбрасывания самых главных вопросов. Ницше писал о
духовном илотстве ученых: жесткая профессиональная конкуренция требует
овладения гигантским объемом специальных знаний и навыков, так что ни на какое
иное образование уже не хватает сил и времени. Но картина мира создается не
методическим, а свободным мышлением. Нет и не может быть учебников и инструкций
по отношению к самым важным вопросам, по построению картины мира. Если это
построение подлинно, оно начинает с постановки под вопрос всего дотоле
построенного, двигаясь далее свободно, а не по методе. Но в университетах учат
другому — наработанным методам решения конкретных задач, учат профессии. И
после университета еще много сил надо вкладывать, чтобы в профессии удержаться
и преуспеть. Так что, одно из двух: или методическое мышление и есть
единственно ценный и надежный путь к истине, пусть и ограниченной, а
философское свободное мышление — это забава или самоутешение или еще что, но
только не путь познания реальности. Или же, напротив, именно философское
мышление и выводит к самым важным вопросам о мире и человеке, и, может быть,
даже их решениям, а методическое мышление, в лучшем случае, предоставляет для
того лишь полезный материал или данные. В первом случае, эксперт оказывается
причастным самому главному, подлинному пути к истине, по которому он сделал
какие-то шаги. А во втором, эксперт, даже весьма успешный, но чуждый философии,
оказывается в положении человека низшего ранга, даже не узнавшего, что такое
настоящее мышление. Много ли найдется ученых, да еще и преуспевших, готовых
хотя бы гипотетически допустить второй вариант? И спрашивать нечего, вопрос
риторический. В этом все и дело, здесь и корни сциентизма, этого шовинизма
научного мышления. Характерно, что среди основоположников физики, вплоть до
квантовой механики, сциентистов не было, за единственным возможным исключением
молодого Дирака. Но и Дирак уже к среднему возрасту стал осознавать философские
основания бытия и познания, выступать как философ-платоник; закончил же он
жизнь регулярным церковным прихожанином. Такое решительное отличие между отцами
физики и ученой массой в целом неудивительно. Установление основания науки
требует не методического, а свободного мышления; методы появляются позже, как
постройки на уже положенном основании. Иными словами, отцами науки могли быть
лишь философы. Ученые же, предоставленные самим себе, в своей массе обречены на
сциентистскую редукцию мышления, последствия чего глубоко деструктивны. Будучи
апологией бессмысленности, сциентизм уничтожает философию, этику и религию. Его
разрушительная сила весьма велика; она проистекает из соединения восхищения
наукой, ограниченности объективированного мышления, экзистенциального страха и
гордого стремления удержать достоинство в бессмысленности. Освобожденное
либеральным мировоззрением от необходимости религиозно-философского
образования, ученое сообщество падает в сциентизм с неизбежностью камня,
потерявшего опору. Будучи весьма крупным, этот падающий камень сносит на своем
пути всё, все основания цивилизации — не только общечеловеческую этику, но и
саму веру в разум, лежащую в основании науки. Если спрашивать о смысле
вселенной, о смысле жизни бессмысленно, если любой ответ может иметь причину,
но не резон, то и в науке, и в самом разуме смысла никакого быть не может. А
раз так, то все нарративы равноправны; любая декларация неравноправия есть акт
произвола и подавления одного мышления другим. Таким образом сциентистская
редукция мышления переходит в постмодернистское самоубийство разума. 

\end{multicols}

Вторая часть: 

\href{https://snob.ru/profile/27355/blog/170283}{%
Опасная мудрость Запада, II \& III, Алексей Буров, snob.ru, 20.09.2020%
}

\begin{multicols}{3}
\end{multicols}

А затем перешли к анализу главных для нас вопросов об основаниях  науки и
реальности. И конечно, коснулись самых острых и актуальных вопросов
современности: упадок свобод, сползание к миропорядку, в котором главные
лозунги:

\obeycr
- Гарантии
- Безопасность
- Спокойствие.
\restorecr

Что скрывается за этими лозунгами? Об этом - в нашей беседе.  

Цитата, которую я упоминал в разговоре: 

\begin{zznagolos}
"В науке мы не в последнюю очередь руководствуемся понятиями прекрасного и
симметрии, поиском простоты и элегантности, стремлением объяснить множество
феноменов с помощью минимального числа возможных принципов. Возникший в
результате эволюционный эпос, возвышаясь над совокупностью содержательных
частей, потенциально придает значение и рациональность ранее бессмысленным
явлениям... 

Возможно, теперь настало время расширить пути познания, распространить
интеллектуальное усилие за пределы конвенциональной науки, то есть
включить в исследование широкий круг не-научных сообществ философов,
теологов... 

Человеческий род входит в эпоху синтеза, что случается лишь однажды в несколько
поколений, а, может быть, в несколько столетий". 
\end{zznagolos}

- Eric Chaisson. Cosmic
Evolution: The Rise of Complexity in Nature. - Cambridge Mass.: Harvard UP,
2001. - P.211-213 [цитата по: Роберт Белла. Религия в человеческой эволюции. -
М.: ББИ, 2019. - С.51-52]. Эрик Чейссон - профессор физики и астрономии
университета Тафта. 

О Льве Тихомирове: \url{https://ru.wikipedia.org/wiki/Тихомиров,_Лев_Александрович} 

Название книги Роджера Скрутона: \enquote{Дураки, мошенники и поджигатели.
Мыслители новых левых}. - М.: Издательский дом ВШЭ, 2021.

\href{https://www.youtube.com/watch?v=h2cWKWp0AkU}{%
Опасная мудрость Запада. Философские разговоры с физиком Алексеем Буровым. Беседа 3, %
Andrii Baumeister, youtube, 09.12.2021%
}

\ii{09_12_2021.fb.baumejster_andrej.kiev.filosof.1.sovremennaja_nauka_obsuzhdenie.scr.1}

\begin{multicols}{2}
\iusr{Myroslav Gashuk}

Очень нужная дискуссия.
Данные вопросы Меня уже давно волнуют. Услышал нужные формулировки.

\iusr{Богдан Вандюк}
Благодарю Вас  за интересную дискуссию! @igg{fbicon.thumb.up.yellow} @igg{fbicon.heart.red}

\iusr{I. P.}

Народные таймкоды:

\obeycr
0:00 Введение. 
1:44 Основные тезисы статьи А.В. Бурова
8:55 Какие есть противоядия от упрощения наукой мира?
13:35 Контраргумент. Наука разрушает иллюзию.
18:26 Смысл может быть ложным. Нигилизм. Эрзац-революциолизм. Затягивание в революционизм.
25:08 Контраргумент. Зато мы живем удобнее и лучше. Чьи плоды?
31:20 Марксизм - синтез сайентизма и ресентимента
32:40 Отче наш снимает ресентимент. Революционеры его разжигают
34:15 Контраргумент. Реальный мир предполагает неравенство. Как правильно учить о неравенстве?
38:40 А что вы предлагаете? Власть святых? 
40:40 Современный человек боится больше предыдущих поколений.
45:10 Свобода - это бремя. Новые революционеры пытаются снять это бремя. Актуальность легенды о Великом инквизиторе.
49:20 Протестантская этика выдыхается. Отцы-основатели становятся неугодными. 
54:30 Тоталитарное государство - культ нигилистов. Майкл Шелинбергер как Лев Тихомиров наших дней.
1:00:00 Революционеры приписывают себе роль Бога. Кулаковский - фасад Рая на земле и за ним Гулаг. 
1:05:05 Основатели науки не говорили о науки в утилитарном смысле. Интерес к Платону родил математическое естествознание. 
1:17:34 Бог является необходимой идеей науки. Интеллектуальная любовь к Богу. 
1:22:55 Среди основоположников науки не было атеистов. Сайентизм - это увлечение Снежной королевой.
1:27:03 Выводы от А.О. Баумейстера. 
1:30:19 Видение будущего от А.В.Бурова. Мы живем в эпоху космического знания. Но золотая жила, возможности быстрых и дешевых открытий закончились.
\restorecr

\iusr{Andrii Baumeister}
О, спасибо большое за помощь!!

\iusr{Natali}
Только начала слушать, уже хочется сказать вам спасибо за выпуск )

\iusr{Наталія Компанієць}
Спасибо! Столько вместилось в одном видео!

\iusr{Tatsiana Moon}
Как всегда, было очень интересно и познавательно! Благодарю!

\iusr{Green}
Очень рад опять увидеть Алексея

\iusr{Интеллектуальный клуб Эмпирос}

\enquote{Мудрость запада} скользкий термин все-таки. Четче будет говорить про
\enquote{негативные плоды атеизма и философского равнодушия и нигилизма}. И плоды эти
не то, что не очевидны, они просто пока еще исторически не наступили. Их могут
предвосхищать только люди, которые мыслят парадигмально в исторической
перспективе, кто прослеживает идеи и их плоды. Остальные не могут
предвосхищать, и по факту пока еще ничего страшного не происходит. Мы в
интересном периоде. Как вы, Андрей, когда-то художественно сказали \enquote{гвоздь
вынули, и полагаем, что картина будет висеть}. Картина пока левитирует в
воздухе в эти секунды истории, которые составляют наши десятилетия-столетия.

\iusr{Alex Stepanov}
Вы не годитесь в паству Андрея. Вы думаете ясно и логически, такие не подходят. Следующий. )

\iusr{Boris Hecht}

Спасибо. Андрей,
за Ваши неудобные вопросы, на которые ещё нет готовых ответов

\iusr{Наталия Пугач}

С отходом от религии новое естественно- научное мышление явно  не имело ничего
общего.Вернер Гейзенберг.

\iusr{Ирина Гонжа}

Спасибо! Очень интересно и актуально.

\iusr{Andrii Baumeister}
Благодарю, Ирина!

\iusr{Natalia Kantovich}
Спасибо за рекомендации авторов и книг!

\iusr{Михаил Иванов}

Предвкушаю...
Больше Бурова! Диалоги с ним во многом мне кажутся самыми \enquote{объёмными} и информативными.
Эдакий \enquote{Дайжест здравомыслия})
Когда какому-то знакомому нужно дать \enquote{red pill}, но много времени он тратить не хочет, по многим вопросам направляю именно на ваши диалоги с Буровым)

\iusr{D.Y. N.}
Великолепная беседа! Великолепные диалоги! Пир настоящей мудрости! Огромное спасибо!

\iusr{Andrii Baumeister}
Благрдарю за инетерс!

\iusr{Natalia Kantovich}

Спасибо за дискуссию. По сути все очень правильно. Единственное, с чем сложно
(невозможно) согласиться, это - упор на Христианство. Я бы говорила о пути к
Б-гу, как о цели (смысле) человеческой жизни, а не о конкретной религии (путей
к Б-гу много, они разные, и каждый человек волен выбрать тот путь, который для
него подходит. Это - одно из проявлений свободы выбора человека).

\iusr{Alexander Davidenko}

Андрей и Алексей, спасибо, было интересно в конце видеть, что вы и видите, и
пока не видите... ТРИ не случайные вещи:

1-я: (видите) что за последние 100 лет иссякла \enquote{импирическая жилка открытий
методом чувственного намацывания} 

2-я: (не видите и не понмаете) что за последние 100 лет
чувственно-намацывательную-науку и теологию некто разгромил в пух и прах, его
книги набирают оборот, их читают,...

3-е: (видите, но не пнимаете) в самом конце вы говорите о том, что некоторые
имеющие вес учёные агитируют за объединение и науки и теологии, вы только не
знаете, что эти некоторые учёные читают ЕГО книги, читают и воруют его идеи,
выдают их за свои... 

Рад вашей прозорливости. Молодцы. Если обстоятельства не изменятся, то мы ещё
увидимся и многое обсудим. А пока, многие читают не те книги, мягко говоря.

\iusr{Alexander Davidenko}

Алексей, посмотрел уже 51мин вы и Кротов. Не бритый варвар «моноложит», но не
беседует. Если бы Кротов не манипулировал и стремился сохранить
конструктивность, то не пришлось бы его назвать варваром. Впрочем, у Кротова
хорошая память и хорошая способность сравнивать (по памяти), нужно было ему
показать, как он неспособен оперировать качественными понятиями. Цитировать по
памяти разное это не тоже, что анализировать. 

Жаль про целостность вы ему не объяснили. И он понёс иррациональное, хаос, в
глазах обывателя вы ему тут слегка проиграли. Я вас понимаю. И у Бога всё в
порядке, а не в хаосе или иррациональности. Яков Кротов варварски не прав.

Следующий раз Кротова бейте его жадностью переговорить, жадностью доминировать,
и, главное, агрессией исходящей от клирика. Получается, яков сам попутал себя.
Он не умеет слушать, но умеет перебивать, навязывать и затыкать рот И ЭТО ОН
НАЗЫВАЕТ ДИАЛОГОМ С «ХАЛЕСТИАНИНОМ»!!!??? Надеюсь, ещё не досмотрел, вы ему в
конце поясните.

И эволюции нет. Если бы была, то появлялись бы новые формы жизни и появлялись
бы новые животные (как результат эволюции). Смысл слова эволюции: хаотически
бороться, сражаться и убивать других за право выжить, и изменяться без цели, на
абум, и дай случай не быть убитым Природой, приспособиться и прожить до
трансформации в новое. (Запомните. Это таков смысл дарвиниста.) 

Есть трансформация. Целевое движение по известным точкам для демонстрации
единства мироздания (материального мира) возникающего из нематериального мира и
снова схлопывающегося в нематериальный мир.

\iusr{Serguei Badaev}

Алексей Буров считает, что христианство должно пониматься как отражение идеи
развития и человеческой свободы. Но это если и возможно, то лишь на уровне
отвлеченной философской мысли. А на уровне социальной практики на первый план
выдвигается христианская догматика и регуляция повседневной жизни с жестким
кодексом морального и ритуального поведения. По сравнению с христианством наука
обладает встроенным механизмом развития, который помогает соотносить практику и
знание, отвергать неверное знание и корректировать ошибки. Из этого и
складывается развитие науки. Лично я не вижу какого-то внутреннего механизма,
присущего христианству, который бы обеспечивал такое же развитие религиозный
мысли и практики, которое существует в науке и технологии.

\iusr{Интеллектуальный клуб Эмпирос}

Тонкое замечание. Действительно, чтобы достучаться до мыслящих людей нужно еще
тщательней проговаривать этот момент (условно говоря) \enquote{двух разных
христианств}: \enquote{практики христианства} (несовершенной церкви земной) и философии
на базе христианской матрицы. Потому что для большинства в возврате к
христианству слышится и видится лишь призыв к невежественному праксису
народного христианства. Которое не то что даже не учит мыслить, но и не
выполняет даже минимальных этических практических призывов Христа.

\iusr{Эдуард Коваленко}
Спасибо.

\iusr{Пифагорейская Вселенная}

Спасибо за великолепные, остро отточенные вопросы, Андрей, и за беседу вообще!
Все беседы с тобой для меня чрезвычайно важны.  Дорогие оппоненты, учитесь
искусству \enquote{адвоката дьявола} у хозяина этого канала. Самоуверенность и сарказм,
как ни странно это может показаться, лишь снижают остроту возражения. Алексей
Буров.

\begin{itemize} % {
\iusr{Ограниченная Ответственность}

Самым веским аргументом в любом споре, кроме сарказма и самоуверенности считаю
кирпич:)) Особенно в наше время посткоммуникации

\iusr{Евгений Воловой}

К Алексею Бурову есть вопрос, а именно: как в голове физика вмещается
метафизика, не возникает ли когнитивный диссонанс,  в следствии нарушения 2
закона логики.  Не является ли метафизика изучением совпадений.  Заранее
благодарю.

\iusr{Пифагорейская Вселенная}

 @Евгений Воловой  Когнитивный диссонанс должен бы возникать в головах
 сциентистов, приравнивающих мышление к явлениям природы.

\iusr{Alex Stepanov}
Да, готовились, пару дней.
\end{itemize} % }

\iusr{Neli Engelhardt}

Большое спасибо за беседу. Сциентизм/сайентизм - это важное направление
мировоззрения сегодня. Агрессия имеет место быть. Приятно было услышать ссылку
на работу Льва Тихомирова. Жду продолжения.

\iusr{Andrii Baumeister}
Благодарю!

\iusr{Petr Knei}

Спасибо Андрей Олегович за очередную содержательную беседу с Алексеем
Владимировичем. Как всегда очень интересные и актуальные темы поднимаете.

\iusr{Kulzada Turlanova}

Спасибо большое за очень интересную и познавательную беседу с интересным гостем!

\iusr{Andrii Baumeister}
Благодарю

\iusr{Olga Atman}

Отличная беседа! Два интеллекта с пламенеющим духом! Очень хотелось бы
как-нибудь послушать вашу беседу на тему \enquote{Почему христианство закончилось на
Западе?} Как так вышло, что способствовало? Эта тема цепляет все прочие
вытекающие. Ведь от традиционалистов слышится только \enquote{запад загнивает}, но
почему - самокритики не хватает. Ведь все, что загнивает, не от внешних
каких-то врагов загнивает. И даже не всегда от внутренних врагов. Идет масса
процессов - те же научные открытия, это не враги, но исторически они попали в
оппозицию церковной политики. Это тоже факт. Алексей затронул эту тему с 54:54,
про \enquote{души, жаждущие смысла} среди тех же левых и атеистов (из-за парадигмы на
дворе), и ее важно развить. 

P.S. Мне очень понравилась беседа Алексея с упомянутым о. Яковом)) Алексей
умудрился вложить максимум смысла в краткие отведенные минуты успевая снимать с
себя \enquote{шкуры верблюда} одну за другой, и еще возвращаться к теме)) Это было
достойное испытание.

\iusr{Nestor Mahno}

Спасибо Огромное!!!! Американские горки для мышления! Бесы Достоевского
-зеркало сегодняшнего дня! Жаль, что в школьных программах нет книги Скотный
двор Джорджа Оруэлла! Отдельное спасибо за  упомянутых авторов и их книги!!!!!

\iusr{Олег Павлов}

Природа не имеет свободы и с ней « проще общаться» учёным и находить какие- то
законы, которые поэтому можно доказать. Человек как часть природы, имея
биологическое тело, ощущает на себе и удовольствие и боль этого заточения. «
Программа существования тела» греховна с религиозной точки зрения и человеку
имея ментальное оснащение требуется выход из этого существования путём
ограничения власти природы. Атеисты и верующие согласны с тем, что сейчас конец
2021 года, а он как известно от рождества Христова!

\iusr{Нинель Химич}
Очень полезный диалог. Благодарю  @igg{fbicon.heart.sparkling}.

\iusr{Boris Hecht}

То, что в науке сейчас существует много противоречивых взглядов на устройство
мира, легализует хаос мнений в обществе. Долгая ситуация неопределённости
разрешает отказатся от поиска нового единого ответа.

Мы не знаем собственных границ, пока не определяем их готовыми ответами. Но мы
можем отказаться от готовых ответов в пользу свободы задавать неудобные
вопросы, на которые ещё нет готовых ответов.

\iusr{Boris Hecht}
Философы строят мосты, там где специалисты роют ямы

\iusr{Y Polk}

Мечта \enquote{Гарантии, Безопасность, Спокойствие}  реализована в птичнике для
бройлеров.  Свобода от чего и для чего и кого???  Свобода вместо Бога??? В
ответ - тяжелое замыленное молчание. Математическая красота мира это не о
закрытой замкнутой системе, к которой подспудно скатываются в обсуждении
(\enquote{обоев и занавесок}).

\iusr{Dmitry Pligin}
Спасибо, очень вовремя, недавно и я формулировал похожие мысли.

\begin{itemize} % {
\iusr{Alex Stepanov}
А вы на кого работаете?
\end{itemize} % }

\iusr{Златокрылец Александр}

Теософия является синтезом \enquote{науки, религии и философии}, уже больше 100
лет в виде теософского общества

\iusr{Alex Stepanov}

Вот теперь всё прояснилось.

\iusr{Leonid2020 Павлов}

Галилей:  книга природа написана языкоматематики, а человек  это часть природы,
следочательно и он этим языком написан, как и все прочее...

\begin{itemize} % {
\iusr{Пифагорейская Вселенная}

Для Галилея и его великих последователей ложность этого \enquote{следовательно} была
само собой очевидной. Истины о природе природными процессами не открываются.

\end{itemize} % }

\iusr{Пётр Волков}

Тема обессмысливания мира наукой является острой и как это ни странно не
исследованной. Наука это прекрасное \enquote{как}, но \enquote{зачем, смысл} на
это нет ответа.

\iusr{roxana romano}

Спасибо огромная за такие интересные и важные беседы! И за ваш выбор и открытие
для нас современных мыслителей!

\begin{itemize} % {
\iusr{roxana romano}
Извиняюсь за опечатку (это умная техника откорректировала @igg{fbicon.grin} )
\end{itemize} % }

\iusr{СЛОЖНОЕ НАЧАЛО}

Неравенство -норма, но проявлять волю и требовать свой кусок, тоже норма, такая же как и неравенство.

П.С. - Соблюдайте очередь.

\iusr{Костя Гнедін}
Вечір пройде не даремно, дякую

\iusr{Mila Mila}
Благодарю  @igg{fbicon.hands.pray} 

\end{multicols}
