% vim: keymap=russian-jcukenwin
%%beginhead 
 
%%file 23_10_2021.fb.druzenko_gennadiy.1.zerkalo_ukraina_amerika
%%parent 23_10_2021
 
%%url https://www.facebook.com/gennadiy.druzenko/posts/10158571968723412
 
%%author_id druzenko_gennadiy
%%date 
 
%%tags amerika,evropa,ideologia,missia,nacidea,strana,ukraina,usa
%%title АМЕРИКАНСЬКЕ ДЗЕРКАЛО УКРАЇНИ
 
%%endhead 
 
\subsection{АМЕРИКАНСЬКЕ ДЗЕРКАЛО УКРАЇНИ}
\label{sec:23_10_2021.fb.druzenko_gennadiy.1.zerkalo_ukraina_amerika}
 
\Purl{https://www.facebook.com/gennadiy.druzenko/posts/10158571968723412}
\ifcmt
 author_begin
   author_id druzenko_gennadiy
 author_end
\fi

АМЕРИКАНСЬКЕ ДЗЕРКАЛО УКРАЇНИ

Я давно стверджую, що покликання України стати «європейською Америкою». Не раз
я писав про разючі паралелі між сучасною Україною та ранніми США. Але
віднедавна я помічаю все більше фундаментальних схожестей між сучасним
українським суспільством і сучасним американським. 

\ifcmt
  ig https://scontent-lhr8-2.xx.fbcdn.net/v/t39.30808-6/242179960_10158571968673412_141351620183045207_n.jpg?_nc_cat=102&ccb=1-5&_nc_sid=8bfeb9&_nc_ohc=4HgA2g4NM1oAX_g4s8W&_nc_ht=scontent-lhr8-2.xx&oh=d7ebbc2e2972466975df1559074ea302&oe=618411A4
  @width 0.4
  %@wrap \parpic[r]
  @wrap \InsertBoxR{0}
\fi

Насамперед це, за словами Івана Крастєва та Олівера Голмса, «перетворення
політики громадян на політику футбольних фанатів», сплеск сучасного
трайбализму, віра в те, що ідентичність важливіша за етику, що перемога лідера
своєї спільноти важливіша за державу і власне державність.

Сьогодні знову переконався, що мої спостереження невипадкові. Наш Формат видав
український переклад чудової книжки Бена Шапіро «Як зруйнувати Америку за три
прості кроки». Автор починає її з невеселих міркувань, що «новітній трайбалізм»
розриває американську політію. Але воднораз нагадує американцям ту велику
істину, яку я волію раз-у-раз намагаюсь нагадувати українцям: українська справа
— це важка боротьба за дедалі кращий проект, в якому беруть участь люди з дуже
різним минулим, а також мовним, культурним та релігійним бекграундом. Тому
Україна — це про спільне майбутнє, а не про спільне минуле. Це насамперед про
усвідомлений громадянський обов’язок, а не про всмоктані з молоком матері мову,
віру та історичний наратив». 

Далі — пряма мова:

«…є інший напрямок. Упродовж усієї історії Америки ця думка перемагала — хоч
давалося це завжди борнею і стражданнями, часто й ціною життя. Ця філософія
стверджує, що сили, які єднають Америку, значно переважають сили роз’єднання,
що наші взаємні обітниці скріплено кров’ю, що долі наші переплетені нерозривно.
Роз’єднання означатиме смерть для всіх нас. 

Нитка цієї думки проходить від батьків-засновників, через Абрагама Лінкольна й
до руху за громадянські права. Вона встановлює вищість розуму й універсальної
моралі над пристрастю й племінним духом; виникла вона з віри в цінність
демократії та особистих прав — принципів, які завжди були істинними, проте
ніколи не застосовувалися належно. Представники цієї течії вважають, що Америка
завжди була недосконалим союзом, але, так чи інакше, союзом залишається, і що
ми перебуваємо в постійному процесі зміцнення та збільшення цього союзу,
заснованого на певних ідеалах.»
