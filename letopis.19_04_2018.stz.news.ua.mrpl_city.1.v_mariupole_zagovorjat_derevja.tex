% vim: keymap=russian-jcukenwin
%%beginhead 
 
%%file 19_04_2018.stz.news.ua.mrpl_city.1.v_mariupole_zagovorjat_derevja
%%parent 19_04_2018
 
%%url https://mrpl.city/news/view/v-mariupole-zagovoryat-derevya-foto
 
%%author_id news.ua.mrpl_city
%%date 
 
%%tags 
%%title В Мариуполе "заговорят" деревья (ФОТО)
 
%%endhead 
 
\subsection{В Мариуполе \enquote{заговорят} деревья (ФОТО)}
\label{sec:19_04_2018.stz.news.ua.mrpl_city.1.v_mariupole_zagovorjat_derevja}
 
\Purl{https://mrpl.city/news/view/v-mariupole-zagovoryat-derevya-foto}
\ifcmt
 author_begin
   author_id news.ua.mrpl_city
 author_end
\fi

\ii{19_04_2018.stz.news.ua.mrpl_city.1.v_mariupole_zagovorjat_derevja.pic.1}

Деревья-патриархи в Мариуполе поведают свои истории жителям и гостям города,
передает MRPL.CITY.

Об этом рассказала мастер \enquote{Зеленстроя} Ирина Хая, которая руководит Городским
садом. Она отметила, что в планах сотрудников парка проведение
исследовательской работы, касающейся истории появления самых старых деревьев на
аллеях Городского сада.

\ii{19_04_2018.stz.news.ua.mrpl_city.1.v_mariupole_zagovorjat_derevja.pic.2}

%https://mrpl.city/news/view/v-mariupole-raspuskayutsya-tsvetochnye-kartiny-iz-bolee-20-tys-tyulpanov-foto
\textbf{Читайте также:} \href{https://archive.org/details/18_04_2018.mrpl_city.kartiny_tulpany}{%
В Мариуполе распускаются цветочные картины из более 20 тыс. тюльпанов (ФОТО), mrpl.city, 18.04.2018}

Ирина Хая подчеркнула, что совместно со специалистами краеведческого музея,
возможно, будет определено, кто и когда высаживал деревья-патриархи. Данная
информация будет размещена на табличках, установленных у зеленолистных
свидетелей истории.

\ii{19_04_2018.stz.news.ua.mrpl_city.1.v_mariupole_zagovorjat_derevja.pic.3}

По словам мастера \enquote{Зеленстроя}, на территории Городского сада находится 70
деревьев, которым свыше 150 лет. Самому старшему из них – 200 лет. Эта
шелковица находится на выходе с главной аллеи парка, под ним установлено
декоративное кресло для свиданий.

Напомним, что в этом году парку исполняется 155 лет и историческое изыскание
поможет лучше узнать жителям  свой город.

\ii{19_04_2018.stz.news.ua.mrpl_city.1.v_mariupole_zagovorjat_derevja.pic.4}
