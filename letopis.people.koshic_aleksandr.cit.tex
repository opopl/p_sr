% vim: keymap=russian-jcukenwin
%%beginhead 
 
%%file people.koshic_aleksandr.cit
%%parent people.koshic_aleksandr
 
%%url 
 
%%author_id 
%%date 
 
%%tags 
%%title 
 
%%endhead 

%%%cit
%%%cit_head
%%%cit_pic
\ifcmt
  tab_begin cols=3
     pic https://avatars.mds.yandex.net/i?id=4768b9ca1db16ac81cf7432ae4324dae-5311590-images-thumbs&n=13
     pic https://focus.ua/static/storage/thumbs/x1200/c/49/0b94b091-3d00d31d686627746212d8c89efb849c.jpg?v=0976
		 pic https://focus.ua/static/storage/thumbs/x1200/c/3d/bfa1d274-e1a8562ac570b67f90405a61b546d3dc.jpg?v=1123
  tab_end
\fi
%%%cit_text
Пам’ятна дошка Олександру Кошицю – продовження великого громадського проєкту
\enquote{Українська історія в камені}, ініційованого громадськими спілками \enquote{Народний
музей України}, \enquote{Музичний батальйон} тощо.  Торік у рамках ініціативи
встановили пам’ятник полководцю Петру Болбочану, цьогоріч – пам’ятник політику
та дипломату Василю Вишиваному (Вільяму фон Габсбургу) (спільно з ГО
\enquote{Всесвітній день вишиванки}), а також пам’ятник митрополиту Тимофію Щербацькому
та пам’ятну дошку Олександру Кониському.  Для широкої спільноти Олександр
Кошиць відомий насамперед як керівник Української республіканської капели УНР,
яка вперше тріумфально презентувала закордоном у світовому турне \enquote{Щедрика} в
обробці Миколи Леонтовича. Саме завдяки цим виступам \enquote{Щедрик} настільки
запам’ятався іноземцям, що став улюбленою різдвяною піснею мільйонів по всьому
світу.  Втім, творчий доробок Олександра Кошиця значно ширший. Так, окремої
уваги заслуговують кубанські експедиції етнографа.  Узимку 1903 року етнограф
отримав доленосного листа від Миколи Лисенка про те, що уряд Кубанського
Козачого війська звернувся до нього в справі записування пісень козаків. Кошиць
із великим ентузіазмом пристав до цієї чудової справи, домовившись викладати у
Тифлісі впродовж навчального року, а літо присвятити розвідці козацьких пісень
на Кубані. Не чекаючи бюрократичних рішень, пов’язаних із фінансуванням, він
власним коштом розпочав експедицію. \enquote{Треба сказати, що з боку обставин, в яких
буде провадитись музикальна робота, ніхто не міг дати мені яких-небудь
вказівок. Прийшлось їхати навмання, і я вирішив починати просто з першої,
ближчої до Катеринодара станиці, а далі йти за вказівками самих козаків}, –
згадував Кошиць
%%%cit_comment
%%%cit_title
\citTitle{Історична експедиція Олександра Кошиця: 1000 українських пісень Кубані}, 
Ігор Шейкін, slovoprosvity.org, 08.11.2021 
%%%endcit
