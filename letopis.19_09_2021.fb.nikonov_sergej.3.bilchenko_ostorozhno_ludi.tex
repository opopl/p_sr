% vim: keymap=russian-jcukenwin
%%beginhead 
 
%%file 19_09_2021.fb.nikonov_sergej.3.bilchenko_ostorozhno_ludi
%%parent 19_09_2021
 
%%url https://www.facebook.com/alexelsevier/posts/1585287208483255
 
%%author_id nikonov_sergej,bilchenko_evgenia
%%date 
 
%%tags bilchenko_evgenia
%%title БЖ. Осторожно, люди!
 
%%endhead 
 
\subsection{БЖ. Осторожно, люди!}
\label{sec:19_09_2021.fb.nikonov_sergej.3.bilchenko_ostorozhno_ludi}
 
\Purl{https://www.facebook.com/alexelsevier/posts/1585287208483255}
\ifcmt
 author_begin
   author_id nikonov_sergej,bilchenko_evgenia
 author_end
\fi

Ещё один пост Евгении Витальевны Бильченко. Размышление. Что касается меня, то
мыслители в данный конкретный момент источники текстов, как файлы для Microsoft
Word в момент их беспристрастного сохранения и дальнейшей нейтральности.

\headCenter{БЖ. Осторожно, люди!}

Юрий Михайлович Лотман - семиотик русский, которого нельзя цитировать на
Украине, - сказал: прошлое рождается в будущем. Осип Эмильевич Мандельштам -
петербургский и всея Руси поэт, которого на Украине объявили украинцем, -
сказал: Пушкин ещё не родился. Антиглобалист Зигмунт Бауман - британский учёный
с мировым именем, запрещённый на Украине за критику внешней политики США, -
сказал: люди фотоальбомов имеют память, а люди смартфонов - манкурты.

Все эти три прекрасных человека имели в виду одно: прошлое - это горизонт
будущего. Изменишь прошлому - нет будущего. Что и сделала Украина.

\ifcmt
  tab_begin cols=3

     pic https://scontent-frt3-2.xx.fbcdn.net/v/t1.6435-9/242294123_1585283578483618_7073086315447767328_n.jpg?_nc_cat=101&ccb=1-5&_nc_sid=730e14&_nc_ohc=x6bPqES0D40AX-lUz0P&_nc_ht=scontent-frt3-2.xx&oh=d79a9be21cba6314467aadf87b33cdc0&oe=616DCB59

     pic https://scontent-frt3-2.xx.fbcdn.net/v/t1.6435-9/242342860_1585283561816953_4314098726089088635_n.jpg?_nc_cat=101&_nc_rgb565=1&ccb=1-5&_nc_sid=730e14&_nc_ohc=aNCiMHLjU4YAX-yS9fJ&_nc_ht=scontent-frt3-2.xx&oh=292b1a33dfa63195a851a0dc1d9f62e5&oe=616E8FBF

		 pic https://scontent-frx5-2.xx.fbcdn.net/v/t1.6435-9/242357849_1585283568483619_8005075720709798711_n.jpg?_nc_cat=109&_nc_rgb565=1&ccb=1-5&_nc_sid=730e14&_nc_ohc=BGlkALztPvMAX8dd-a9&_nc_ht=scontent-frx5-2.xx&oh=d04a676c50398ad316ee11926f51b942&oe=616DBEFD

  tab_end
\fi

На этих фото: я - в шинели. Я больше не хожу в шинели, а ношу цветное бохо,
потому что шинель у меня - внутри, с маузером впридачу. Обычно камуфло носят
пацифисты, когда у них нехватка. У меня - избыток. На фото - студент Влад.
Плюнул на универ после закрытия моей школы и ушел в Маркса. На фото - студентка
Юна. Ушла из универа после увольнения меня в чистое русское православие. На
фото - магистр Коля. Игнорит универ, ушел в мистику Востока. На фото - аспирант
Леша. Ушел шумно, как Холден Колфилд, в медленно вызревающую русскую ярость.

Все эти три прекрасных человека имели в виду одно: прошлое - это горизонт
будущего. Изменишь прошлому - нет будущего. Что и сделала Украина.

На этих фото: я - в шинели. Я больше не хожу в шинели, а ношу цветное бохо,
потому что шинель у меня - внутри, с маузером впридачу. Обычно камуфло носят
пацифисты, когда у них нехватка. У меня - избыток. На фото - студент Влад.
Плюнул на универ после закрытия моей школы и ушел в Маркса. На фото - студентка
Юна. Ушла из универа после увольнения меня в чистое русское православие. На
фото - магистр Коля. Игнорит универ, ушел в мистику Востока. На фото - аспирант
Леша. Ушел шумно, как Холден Колфилд, в медленно вызревающую русскую ярость.

Никто из них не начал повторять речевки сугсов, наоборот, стало все ещё хуже.
Хотя мы все очень разные, но каждый в сердце несёт прошлое нашей курилки. А
прошлое рождается в будущем. Дорогие украинские националисты, сегодня им по
двадцать, завтра они придут за вами. Вы зря думаете, что их мало, что это -
единичные фрики. Имя вам легион. Имя им воинство. Двадцать лет работы с
молодежью вашего, да, врага, антифашиста - меня.

Мы - не люди смартфонов. Мы - люди фотоальбомов. Мы - просто люди.

Осторожно.
