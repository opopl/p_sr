% vim: keymap=russian-jcukenwin
%%beginhead 
 
%%file 12_02_2022.stz.news.lnr.lug_info.1.krasa_donbassa_2022
%%parent 12_02_2022
 
%%url https://lug-info.com/news/bolee-400-devushek-iz-lnr-i-dnr-podali-zayavki-na-konkurs-krasoty-krasa-donbassa-2022
 
%%author_id news.lnr.lug_info
%%date 
 
%%tags devushka,dnr,donbass,konkurs,konkurs_krasoty,krasota,lnr,zhizn
%%title Более 400 девушек из ЛНР и ДНР подали заявки на конкурс красоты "Краса Донбасса-2022"
 
%%endhead 
 
\subsection{Более 400 девушек из ЛНР и ДНР подали заявки на конкурс красоты \enquote{Краса Донбасса-2022}}
\label{sec:12_02_2022.stz.news.lnr.lug_info.1.krasa_donbassa_2022}
 
\Purl{https://lug-info.com/news/bolee-400-devushek-iz-lnr-i-dnr-podali-zayavki-na-konkurs-krasoty-krasa-donbassa-2022}
\ifcmt
 author_begin
   author_id news.lnr.lug_info
 author_end
\fi

Более 400 девушек из ЛНР и ДНР подали заявки на участие в кастинге конкурса
красоты \enquote{Краса Донбасса - 2022}. Об этом ЛИЦ сообщил председатель общественной
организации (ОО) \enquote{Молодая гвардия} Даниил Степанков.

\ii{12_02_2022.stz.news.lnr.lug_info.1.krasa_donbassa_2022.pic.1}

Луганский этап отбора претенденток на участие в конкурсе состоялся сегодня во
дворце культуры имени Ленина.

\ii{12_02_2022.stz.news.lnr.lug_info.1.krasa_donbassa_2022.pic.2}

\enquote{На сегодняшними день заявки на участие подали более 400 девушек. Сегодня те
девушки, которые прошли отбор на этапе электронных заявок, проходят очный
кастинг на территории Луганской Народной Республики. Завтра, 13 января, будет
проходить отбор на территории Донецкой Народной Республики. После этого, через
пару дней, будет объявлены результаты девушкам, и всего 20 из них примут
участие в финальном этапе, который состоится на территории ДНР и на котором
одна из них получит титул \enquote{Краса Донбасса – 2022}}, - сообщил Степанков.

\ii{12_02_2022.stz.news.lnr.lug_info.1.krasa_donbassa_2022.pic.3}

Представитель оргкомитета уточнил, что в финале встретятся 10 лучших девушек из
двух Республик в возрасте от 14 до 17 лет и 10 девушек в возрасте от 18 до 25
лет.

\ii{12_02_2022.stz.news.lnr.lug_info.1.krasa_donbassa_2022.pic.4}

В состав жюри вошли юноши и девушки из общественных организаций \enquote{Молодая
гвардия} (ЛНР) и \enquote{Молодая Республика} (ДНР), а также организатор ежегодного
конкурса красоты \enquote{Мисс Луганск} Юлия Байрамова.

\ii{12_02_2022.stz.news.lnr.lug_info.1.krasa_donbassa_2022.pic.5}

Степанков отметил, что этот проект представительницы из ДНР стал победителем
конкурса грантов в рамках форума молодежи Донбасса \enquote{Море - Лес}.

\ii{12_02_2022.stz.news.lnr.lug_info.1.krasa_donbassa_2022.pic.6}

\enquote{Таким образом, гранты реализуются с помощью нашего форума
\enquote{Море – Лес}. Мы желаем всем успеха, и пусть победит сильнейший, и
пусть весь мир смотрит, что на территории Донбасса самые красивые девушки}, -
добавил он.

\ii{12_02_2022.stz.news.lnr.lug_info.1.krasa_donbassa_2022.pic.7}

Напомним, организаторами форума стали проект \enquote{Молодая гвардия} ОД \enquote{Мир
Луганщине} и общественная организация \enquote{Молодая Республика} при поддержке главы
ЛНР Леонида Пасечника, главы ДНР Дениса Пушилина и российской партии \enquote{Единая
Россия}. Три тематических смены форума прошли на территории ДНР, а три – в ЛНР. 

\ii{12_02_2022.stz.news.lnr.lug_info.1.krasa_donbassa_2022.pic.8}
