% vim: keymap=russian-jcukenwin
%%beginhead 
 
%%file 06_04_2021.fb.zharkih_ekaterina.1.taksi_karantin_it
%%parent 06_04_2021
 
%%url https://www.facebook.com/permalink.php?story_fbid=126222656185071&id=112669904207013
 
%%author 
%%author_id 
%%author_url 
 
%%tags 
%%title 
 
%%endhead 

\subsection{Первый день локдауна: когда айтишники завидуют таксистам }
\label{sec:06_04_2021.fb.zharkih_ekaterina.1.taksi_karantin_it}
\Purl{https://www.facebook.com/permalink.php?story_fbid=126222656185071&id=112669904207013}

Вы не раз слышали про невидимую руку рынка, правда? С сегодняшнего дня многие
жители Киева ясно почувствовали, как она нагло шуршит в их карманах и
кошельках.   

После ограничения работы общественного транспорта с сегодняшнего дня в
столичных службах такси уже отметили повышенный спрос, что привело к резкому
скачку тарифов на проезд. 

Теперь поездка в 10 км стоит приблизительно 320-500 гривен – эдак в 2 – 2,5
раза дороже обычного. Такой себе тариф-новогодний, только без ёлки и
новогоднего настроения. Ну и, предположительно, на целых 2 недели, до пятницы
16 апреля.


\ifcmt
  pic https://scontent-yyz1-1.xx.fbcdn.net/v/t1.6435-9/170255837_126222619518408_367586407384378309_n.jpg?_nc_cat=105&ccb=1-3&_nc_sid=8bfeb9&_nc_ohc=AtetD0dwp6IAX9oP8q9&_nc_ht=scontent-yyz1-1.xx&oh=62f2787a14f5c7396043daf991a3f866&oe=60934D92
  width 0.4
\fi


А что вы хотели, рыночек порешал. Ведь кому-то всё равно нужно ездить на работу
5-6 дней в неделю, а пропуска выдаются прежде всего медперсоналу, спасателям,
работникам коммунальной, транспортной и социальной сфер, военным - иными
словами, персоналу критически важной инфраструктуры. И ещё продавцам. И, если
почитать, то и многим иным профессиям, вплоть до банкиров, священников и
сотрудников СМИ. Но. Во-первых, пропусков всего полмиллиона -  явно меньше, чем
работающих в столице людей. 

Во-вторых, бюрократия. В Киевсовете обращают внимание, что количество
спецпропусков ограничено. Заявки от физических лиц не принимаются, только от
организаций. С заполнением заявок-анкет, объяснениями, почему вам нужен
пропуск, которые рассмотрят и может быть соизволят дать пропуск. Которого ещё
нужно дождаться. 

А на работу нужно уже сейчас. Так что, плачу 2 счётчика, шеф ) 

Сегодня Кличко допустил введение комендантского часа в столице, хотя и отметил,
что решения по введению чрезвычайного положения принимает правительство
Украины, а не лично он.

Правительство у нас, как мы знаем, мудрое и дальновидное. Так что можно ожидать
не только ЧП, но и чего веселее. Одно лишь добавлю напоследок с надеждой: уж
лучше чрезвычайное положение и комендантский час от коронавируса, чем военное
положение и комендантский час от бомбёжек и стрельбы.
