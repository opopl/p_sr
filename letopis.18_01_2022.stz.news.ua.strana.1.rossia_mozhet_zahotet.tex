% vim: keymap=russian-jcukenwin
%%beginhead 
 
%%file 18_01_2022.stz.news.ua.strana.1.rossia_mozhet_zahotet
%%parent 18_01_2022
 
%%url https://strana.digital/news/372265-rossija-mozhet-zakhotet-atakovat-ukrainu-v-ljuboj-moment-belyj-dom.html
 
%%author_id 
%%date 
 
%%tags 
%%title "Крайне опасная ситуация". В Белом доме считают, что Россия может захотеть атаковать Украину в любой момент
 
%%endhead 
\subsection{\enquote{Крайне опасная ситуация}. В Белом доме считают, что Россия может захотеть атаковать Украину в любой момент}
\label{sec:18_01_2022.stz.news.ua.strana.1.rossia_mozhet_zahotet}

\Purl{https://strana.digital/news/372265-rossija-mozhet-zakhotet-atakovat-ukrainu-v-ljuboj-moment-belyj-dom.html#}
\ifcmt
 author_begin
   author_id  
 author_end
\fi

Пресс-секретарь Белого дома Джен Псаки заявила, что в Вашингтоне считают
сложившуюся ситуацию вокруг Украины чрезвычайно опасной и исходят из того, что
России предстоит решить, продолжать дипломатическое взаимодействие или нет. 

Об этом Джен Псаки заявила на брифинге для журналистов во вторник, 18 января.

\enquote{Мы считаем, что это крайне опасная ситуация. На текущем этапе Россия может
захотеть атаковать Украину в любой момент}, - утверждает пресс-секретарь Белого
дома.

По версии Псаки, госсекретарь США Энтони Блинкен на предстоящей в пятницу, 21
января, в Женеве встрече с министром иностранных дел РФ Сергеем Лавровым
\enquote{призовет Россию незамедлительно предпринять шаги по деэскалации}.

\enquote{Наша позиция с самого начала была очень ясной. Есть два пути: дипломатический,
и мы, конечно, надеемся, что они выберут его, и другой путь. Конечно, от
россиян зависит то, какой путь они выберут. Последствия будут жесткими, если
они не выберут путь дипломатии}, - предупредила Джен Псаки.

Напомним, что в Государственном департаменте США заявили, что Россия продолжает
нагнетать обстановку, несмотря на призывы к деэскалации, и в последние дни
создала чрезвычайно опасный уровень угрозы нового масштабного наступления на
Украину.

Ранее мы писали, что в Пентагоне не планируют отправлять дополнительный военный
контингент в Украину.

Также \enquote{Страна} сообщала, что американские сенаторы пообещали передать Украине
\enquote{Джавелины}, \enquote{Стингеры} и другое оружие в случае вторжения РФ.
