% vim: keymap=russian-jcukenwin
%%beginhead 
 
%%file 21_11_2021.fb.dolzhenkov_aleksandr.kiev.1.den_dostoinstva_svobody
%%parent 21_11_2021
 
%%url https://www.facebook.com/permalink.php?story_fbid=3039685246244737&id=100006098745165
 
%%author_id dolzhenkov_aleksandr.kiev
%%date 
 
%%tags den_gidnosti_svobody,dostoinstvo,maidan2,revolucia,strana,svoboda,ukraina
%%title С 2017 ежегодно 21 ноября в Украине отмечается День Достоинства и Свободы...
 
%%endhead 
 
\subsection{С 2017 ежегодно 21 ноября в Украине отмечается День Достоинства и Свободы...}
\label{sec:21_11_2021.fb.dolzhenkov_aleksandr.kiev.1.den_dostoinstva_svobody}
 
\Purl{https://www.facebook.com/permalink.php?story_fbid=3039685246244737&id=100006098745165}
\ifcmt
 author_begin
   author_id dolzhenkov_aleksandr.kiev
 author_end
\fi

С 2017 ежегодно 21 ноября в Украине отмечается День Достоинства и Свободы...
Этот день 8 лет назад стал отправной точкой массовых протестов вполне немирных
на Майдане, формальным поводом которых было решение Кабмина остановить
подписание Соглашения об ассоциации с ЕС.

\ifcmt
  ig https://scontent-frx5-1.xx.fbcdn.net/v/t39.30808-6/258876444_3039683969578198_7415107917646223678_n.jpg?_nc_cat=110&ccb=1-5&_nc_sid=8bfeb9&_nc_ohc=td5QAq2sIS0AX_6TepK&_nc_ht=scontent-frx5-1.xx&oh=51c8bbe31a3d552a2a4c4937d9951eaa&oe=61A394D2
  @width 0.4
  %@wrap \parpic[r]
  @wrap \InsertBoxR{0}
\fi

Многие люди, которые вышли тогда, хотели закрепить европейскую идентичность, но
не понимали, что их просто использовали мерзавцы, желавшие дорваться к власти
даже ценою жизней протестующих. Личность Януковича была токсична для многих
из-за его противоречивого прошлого, логика принимаемых тогда властью решений
затмилась антипатией к его персоне, нежеланием сближаться с Россией. 

Многие уже себя увидели в Европейском Союзе.

Что же произошло в реальности?

Насколько мы после этих событий повысили наше Достоинство, если ходим с
протянутой рукой, прося дополнительные кредиты от МВФ, Европейского банка
реконструкции и развития (ЕБРР)?

Насколько мы стали свободны, когда в стране укрепилось внешнее управление
Запада? Представители Майдана обвиняли Партию Регионов в том, что они получают
указания с Кремля (что было, конечно, неправдой). Но неужели сейчас действующая
власть не следует указаниям «коллективного Запада» (послы G7)?

Этот день стал отправной точкой новой истории Украины, которая уже не знает
благополучия, уверенности в завтрашнем дне, законности и порядка...

Экономика в упадке, тарифы выросли в разы, политики занимаются лишь
откладываниями, а не решают проблемы людей, отсутствует системное видение
развития страны, ее место в мире, кроме плацдарма для противостояния с Россией,
что может для нас трагически завершиться...

Надеюсь, это будем уроком для тех, кто поддерживал в своё время Майдан,
поскольку последствия таких потрясений приводят только к ухудшению ситуации в
стране.

\ii{21_11_2021.fb.dolzhenkov_aleksandr.kiev.1.den_dostoinstva_svobody.cmt}
