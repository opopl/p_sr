% vim: keymap=russian-jcukenwin
%%beginhead 
 
%%file 23_08_2021.fb.skubchenko_aleksandr.1.velichie_nezalezhnist
%%parent 23_08_2021
 
%%url https://www.facebook.com/skubchenco/posts/4057170641049029
 
%%author_id skubchenko_aleksandr
%%date 
 
%%tags nezalezhnist,ukraina,velichie
%%title Я желаю своей Украине вернуть былое величие
 
%%endhead 
 
\subsection{Я желаю своей Украине вернуть былое величие}
\label{sec:23_08_2021.fb.skubchenko_aleksandr.1.velichie_nezalezhnist}
 
\Purl{https://www.facebook.com/skubchenco/posts/4057170641049029}
\ifcmt
 author_begin
   author_id skubchenko_aleksandr
 author_end
\fi

Флаги на "самых высоких" флагштоках поднимают тогда, когда нет самых больших
достижений науки и промышленности, на которых можно эти флаги поднять. 

Украина 30 лет назад была одним из 9 государств в мире с полным циклом
производства самолётов — самолёты мы уже не строим. 

Мы строили самые большие корабли — сегодня мы их покупаем у запада. Наша
промышленность уже вообще ничего не строит, потому что мы обязаны всё покупать
у запада. 

В УССР установили Родину Мать — спустя 30 лет независимости Украина даже
демонтировать (декоммунизировать) её не в состоянии, так как нет необходимой
для этого техники. 

Я желаю своей Украине вернуть былое величие. И даже стать гораздо лучше. А
гордиться самыми высокими флагштоками — увольте, слишком мелко.
