% vim: keymap=russian-jcukenwin
%%beginhead 
 
%%file slova.knjaz
%%parent slova
 
%%url 
 
%%author 
%%author_id 
%%author_url 
 
%%tags 
%%title 
 
%%endhead 
\chapter{Князь}

%%%cit
%%%cit_head
%%%cit_pic
%%%cit_text
За кілька років після хрещення \emph{князь} Володимир Святий вирішив розмежувати на
Русі небесне і земне. У 995/996 році він видав Десятинній церкві грамоту на
отримання десятини, яка до 1011 року перетворилася на перший церковний статут.
Незважаючи на свою назву, документ стосувався устрою всієї державної влади,
оскільки розмежував юрисдикцію світських і церковних судів, а також віддав
священикам контроль мір і ваг. Всі інші питання залишилися у віданні \emph{князя}.
Пізніше свій церковний статут видав \emph{князь Ярослав}, а потім канонічне право
остаточно відокремилося від державного
%%%cit_comment
%%%cit_title
\citTitle{Українській Конституції – 2300 років}, 
Сергій Громенко, gazeta.ua, 26.06.2021
%%%endcit

%%%cit
%%%cit_head
%%%cit_pic

\ifcmt
  tab_begin cols=2
     width 0.4
     pic https://img.strana.ua/img/article/3410/zelenskij-vnes-v-33_main.jpeg

     pic https://strana.ua/img/forall/u/11/52/%D0%B3%D0%B5%D1%80%D0%B1.jpg
  tab_end
\fi

%%%cit_text
Главным элементом Большого герба, согласно проекту, является Знак \emph{Княжеского}
Государства Владимира Великого (малый Государственный Герб Украины) золотого
цвета, размещенный на синем пятистороннем щите с закругленными нижними боковыми
углами с золотой каймой.  Над щитом - изображение \emph{великокняжеского} венца
(короны) Ярослава Мудрого и пурпурно-золотой шатер в виде растительного
орнамента. Щит держат: с левой стороны - лев (герб Галицко-Волынского
\emph{княжества}), с правой - воин-казак с ружьем (герб Войска Запорожского) под щитом
- лента из двух горизонтальных полос синего и желтого цветов, под лентой - два
золотых колоска пшеницы, переплетенные гроздью калины пурпурного цвета со
стилизованным листьями пурпурно-золотого цвета
%%%cit_comment
%%%cit_title
\citTitle{Зеленский внес в Раду законопроект о большом гербе. Эскиз}, 
, strana.ua, 29.06.2021
%%%endcit
