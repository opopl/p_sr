% vim: keymap=russian-jcukenwin
%%beginhead 
 
%%file poetry.rus.dnr.vladislav_rusanov.bio
%%parent poetry.rus.dnr.vladislav_rusanov
 
%%url 
%%author 
%%tags 
%%title 
 
%%endhead 

\subsubsection{Биография}

\ifcmt
pic https://stihi.ru/photos/brendarcher.jpg
\fi

Родился 12 июня 1966 года.

По образованию инженер-геологоразведчик, кандидат технических наук по
специальности "Бурение скважин". Живу и работаю в Донецке.  Прозаик, поэт,
переводчик.

Автор 28 книг, выходивших в разное время в издательствах "Крылов", "Астрель"
(Санкт-Петербург), АСТ, ЭКСМО (Москва), "Шико" (Луганск, Севастополь).

Переводил с английского прозу Артура Конан Дойла, Грегори Киза, Джорджа
Мартина, Джо Аберкромби, Гилиан Флинн и других, стихотворения Эдгара Алана По,
Редьярда Киплинга и других.

Редактор-составитель сборника гражданской поэзии Донбасса "Час мужества"
(обладатель специальной премии в номинации "Поэзия" Московской международной
книжной ярмарки 2015 года).

Лауреат премии "Лунная радуга" в области литературы (2014 год).  Кавалер Ордена
им. Ф. М. Достоевского 1-й степени (2016 год).

Член Союза писателей Донецкой Народной Республики. Почётный член
Интернационального Союза писателей. Член Союза писателей России. Член Совета по
фантастической и приключенческой литературе при Союзе писателей России.

"Отважная фантазия, столкновение впитанной киношной романтики с дьявольским
окружающим реализмом - всё это стихи Русанова..." Вячеслав Сухнев, ведущий
редактор журнала "Стратегия России"
