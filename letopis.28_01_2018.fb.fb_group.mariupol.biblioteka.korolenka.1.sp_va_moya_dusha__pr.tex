%%beginhead 
 
%%file 28_01_2018.fb.fb_group.mariupol.biblioteka.korolenka.1.sp_va_moya_dusha__pr
%%parent 28_01_2018
 
%%url https://www.facebook.com/groups/1476321979131170/posts/1569544423142258
 
%%author_id fb_group.mariupol.biblioteka.korolenka,kibkalo_natalia.mariupol.biblioteka.korolenko
%%date 28_01_2018
 
%%tags sosjura_volodymyr,poezia,mariupol,literatura
%%title Співа моя душа, прозора і крилата…
 
%%endhead 

\subsection{Співа моя душа, прозора і крилата...}
\label{sec:28_01_2018.fb.fb_group.mariupol.biblioteka.korolenka.1.sp_va_moya_dusha__pr}
 
\Purl{https://www.facebook.com/groups/1476321979131170/posts/1569544423142258}
\ifcmt
 author_begin
   author_id fb_group.mariupol.biblioteka.korolenka,kibkalo_natalia.mariupol.biblioteka.korolenko
 author_end
\fi

Співа моя душа, прозора і крилата...

Несмотря на форс-мажорные обстоятельства, в частности отсутствие электричества,
26 января читальный зал Центральной библиотеки им. В.Г. Короленко был
переполнен. На литературно-музыкальную композицию \enquote{Співа моя душа, прозора і
крилата...}, посвященную 120-летию со дня рождения Владимира Николаевича
Сосюры, собрались ценители украинской лирики - и молодежь, и мариупольцы
«золотого» возраста. Инициатором мероприятия стал исследователь творчества В.Н.
Сосюры, композитор-аматор, лауреат премии им. В. Сосюры, почетный краевед
Донетчины Григорий Григорьевич Кабанцев.

Ознакомив с основными вехами жизни и творчества Владимира Сосюры, ведущая
вечера Людмила Галбай открыла музыкальную программу. Концерт был очень
насыщенным. Звучали патриотические, лирические, философские стихи «соловья
украинской поэзии». Ко многим произведениям В. Сосюры Григорий Кабанцев написал
музыку. Песни «Любіть Україну», «Білі акації будуть цвісти»,  «Коли потяг у
далі загуркоче», «Такий я ніжний, такий тривожний» Григорий Григорьевич
исполнил лично под аккомпанемент Лины Азаровой. 

В теме «Лирика любви» прозвучали два романса. Романс «Так ніхто не кохав»,
посвященный первой жене поэта Вере Берзиной, исполнил мужской ансамбль
городского Дворца культуры (худ. рук. Лина Азарова). Посвященный второй жене
Сосюры Марии Даниловой романс «Марія» также собрал многочисленные аплодисменты.
Он прозвучал в исполнении Александра Прокопенко. Чудесным танцем сопровождали
декламацию стихотворений  В. Сосюры учащиеся Школы искусств Олег Пантюшенко и
Лада Булатова (преп. Токарева И.И.). 

Вниманию любителей украинской классики библиотекари представили книжную
выставку-персоналию «Яке щастя, що я – українець!». Познакомившись с
посвященными Сосюре изданиями, поклонники его творчества еще раз убедились в
неповторимой внутренней мелодии, искренности, задушевности, яркой образности
произведений выдающегося украинца.

Завершилась литературно-музыкальная композиция символическими строками
Владимира Сосюры: 

«Знаю я: моє не згасне ім'я

На прийдешній сонячній путі,

Кров свою віддам пісням моїм я,

Що мене продовжать у житті!».
