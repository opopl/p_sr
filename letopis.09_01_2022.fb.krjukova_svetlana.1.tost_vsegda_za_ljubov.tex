% vim: keymap=russian-jcukenwin
%%beginhead 
 
%%file 09_01_2022.fb.krjukova_svetlana.1.tost_vsegda_za_ljubov
%%parent 09_01_2022
 
%%url https://www.facebook.com/kryukova/posts/10159880174478064
 
%%author_id krjukova_svetlana
%%date 
 
%%tags chelovek,kiev,ljubov,rasskaz
%%title Что удивительно, несмотря на печальный опыт, тост - всегда за любовь!
 
%%endhead 
 
\subsection{Что удивительно, несмотря на печальный опыт, тост - всегда за любовь!}
\label{sec:09_01_2022.fb.krjukova_svetlana.1.tost_vsegda_za_ljubov}
 
\Purl{https://www.facebook.com/kryukova/posts/10159880174478064}
\ifcmt
 author_begin
   author_id krjukova_svetlana
 author_end
\fi

Воскресный кабак на автовокзале без таблички и имени шумит гостями как пчелиный
улей. Бархатные занавески, деревянные бусы как дождик в ряд в качестве
перегородок, бордовые скатерти под стеклом и официант Света, она же и хозяйка,
хорошо усвоившая золотое правило ведения ресторанного бизнеса - вкусная
домашняя еда важнее модного интерьера.

Место для своих. Киевляне и понаехавшие с пропиской греются горячей едой и
алкоголем, вроде как дрова, заброшенные внутрь, - греют лучше батарей. 

Застолье подогревают беседами, а не наоборот. 

За столиком слева, двое мужчин за 60 анализируют события в Казахстане и тяжкие
гастрономические последствия праздников. Через ох и ах. И что всему есть
предел, но не сегодня. Суп с фрикадельками. Фасолевый «Лоббио» и блинчики с
мясом и со сметаной. И ещё по сто. 

За столиком справа большая еврейская семья с тремя пухлыми внуками обсуждают
безусловные таланты своих детей. У старшего шахматы, у среднего теннис, у
младшего - невероятная тяга к математике и странное хобби - коллекционировать
пуговицы. Все разные, но общее у них одно - лучшие мальчики в мире. 

За окном романтично валит снег, упрятывая следы пассажиров троллейбуса \#43 от
ближайшей остановки. 

За столиком у окна две женщины за 50, в укладках и помадах, допивают бутылку
коньяка и моют кости бывшим. Делают это романтично, через тост и признания.
Пил, курил, храпел, веники не носил, по бабам шлялся. На утро приходил, и
говорил, что задержался, потому что его напугали петарды. А когда окончательно
ушёл, прихватил все золотые бряцки, которые подарил за годы терпения. 

Что удивительно, несмотря на печальный опыт, тост - всегда за любовь!
