% vim: keymap=russian-jcukenwin
%%beginhead 
 
%%file 15_12_2020.news.ua.zn.1.kuleba_russia_covid
%%parent 15_12_2020
 
%%url https://zn.ua/UKRAINE/kuleba-pozhalovalsja-na-voz-ne-prenebrezhitelnoe-otnoshenie-rf-k-pandemii-covid-19-v-krymu-i-ordlo.html
 
%%author 
%%author_id 
%%author_url 
 
%%tags covid,russia,krym
%%title Кулеба пожаловался ВОЗ на пренебрежительное отношение РФ к пандемии COVID-19 в Крыму и ОРДЛО
 
%%endhead 
 
\subsection{Кулеба пожаловался ВОЗ на пренебрежительное отношение РФ к пандемии COVID-19 в Крыму и ОРДЛО}
\label{sec:15_12_2020.news.ua.zn.1.kuleba_russia_covid}
\Purl{https://zn.ua/UKRAINE/kuleba-pozhalovalsja-na-voz-ne-prenebrezhitelnoe-otnoshenie-rf-k-pandemii-covid-19-v-krymu-i-ordlo.html}

\textbf{Россия, будучи государством-оккупантом, нарушает Четвертую Женевскую конвенцию
о защите гражданского населения.} 

\ifcmt
  pic https://zn.ua/img/article/4396/39_main-v1608063099.jpg
  caption Кулеба обсудил с Клюге доступ международных медицинских организаций в оккупированные районы Донбасса и Крым
  width 0.5
  fig_env wrapfigure
\fi

Российская Федерация не принимает необходимых мер для борьбы с пандемией
COVID-19 и защиты здоровья и жизней гражданского населения в оккупированных
районах Донбасса и Крыму. На этот факт обратил внимание министр иностранных дел
Украины Дмитрий Кулеба в ходе встречи с директором Европейского бюро Всемирной
организации здравоохранения Хансом Клюге в Киеве\Furl{https://zn.ua/UKRAINE/v-ukrainu-pribyl-hlava-evropejskoho-bjuro-voz-obsuzhdat-podhotovku-k-vaktsinatsii} 15 декабря, сообщает
пресс-служба МИД.

«Россия как государство-оккупант продолжает уклоняться от выполнения своих
международных обязательств в рамках Четвертой Женевской конвенции о защите
гражданского населения. Жители оккупированных частей Донбасса и Крыма не
обеспечены надлежащей медицинской и социальной защитой, что недопустимо на фоне
продолжающейся пандемии», - констатировал глава МИД.

Собеседники обсудили вопрос доступа международных организаций на временно
оккупированную территорию Украины в контексте необходимости борьбы с пандемией.

Ранее представитель отдельных районов Донецкой области в ТКГ от Украины Сергей
Гармаш сообщил о том, что подконтрольные РФ незаконные вооруженные формирования
в Донбассе не могут обеспечить помощь населению ОРДЛО в рамках борьбы
коронавирусом. В ОРДЛО не хватает медикаментов, кадров. Все усугубляет тяжелая
экономическая ситуация. Гармаш также добавил, что смертность от коронавируса на
оккупированных территориях Донбасса значительно превышает среднюю мировую
статистику.\Furl{https://zn.ua/UKRAINE/harmash-rasskazal-i-vysokoj-smertnosti-ot-kovid-19-v-okkupirovannom-donbasse.html}

В октябре в Офисе омбудсмена также сообщали о том, что ситуация с коронавирусом
в ОРДЛО критическая.\Furl{https://zn.ua/UKRAINE/situatsija-s-koronavirusom-v-ordlo-kriticheskaja-ofis-ombudsmena.html} По словам бывшего представителя омбудсмена в Донецкой и
Луганской областях Павла Лисянского, больницы на оккупированной территории
переполнены, средств индивидуальной защиты в аптеках ОРДЛО не хватает,
ограничительные меры, такие как социальная дистанция и масочный режим, не
действуют.
