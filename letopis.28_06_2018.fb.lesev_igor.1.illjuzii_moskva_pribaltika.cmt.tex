% vim: keymap=russian-jcukenwin
%%beginhead 
 
%%file 28_06_2018.fb.lesev_igor.1.illjuzii_moskva_pribaltika.cmt
%%parent 28_06_2018.fb.lesev_igor.1.illjuzii_moskva_pribaltika
 
%%url 
 
%%author_id 
%%date 
 
%%tags 
%%title 
 
%%endhead 
\subsubsection{Коментарі}

\begin{itemize} % {
\iusr{Евгений Отовчиц}
В Прибалтике уровень жизни в несколько раз выше? Игорь, вы иногда подменяете здравый смысл и объективность эмоциями и манипуляцией.

\begin{itemize} % {
\iusr{Игорь Лесев}
вы не поленитесь посмотреть среднюю зп в Эстонии и в России, а потом продолжите рассуждения об эмоциях и манипуляциях

\iusr{Евгений Отовчиц}
\textbf{Игорь Лесев} а я не поленился. И в Латвии, и в Литве тоже. Какое еще "в разы"?..

\iusr{Игорь Лесев}

500 против 900 у.е. - по моему, достаточно оснований писать "в разы". И сюда
включается не только чистая зп, но и сама возможность зарабатывать. Вы видели
латышей-гастеров в Москве или Питере? А они гастарбайтерят, только в Британии
или Швеции. И остаются латышами, зарабатывая ДРУГИЕ от ру-реальности деньги


\iusr{Евгений Отовчиц}
\textbf{Игорь Лесев} в разы - это в 2-3 раза МИНИМУМ.
500 и 900 - это на порядок, существенно.
Но не в разы.

\iusr{Петр Шелест}
Вна порядок - это в 10 раз)))

\iusr{German Gorozhanski}

Ну, так шоб "без трепатни, без дураков", как шутил А.Райкин, то 900 ПОЧТИ в два
раза больше, чем 500. Не в разы и не на порядки. Но в цифрах нужно, видимо,
учитывать и процент от з/п на оплату ЖКХ, мед.обслуживание, образование и т.д.
Так же то, что прибалтов до 20-го года продолжает спонсировать ЕС. Население -
это, в основном, старшее поколение и их внуки. Среднее уехало на заработки в
ЕС, по этому города достаточно немноголюдны.

\end{itemize} % }

\iusr{Евгений Водовозов}
Как будто в аннексии Крыма была какая то логика...

\iusr{Матвей Кублицкий}
согласен. от нехерделать захватили....

\iusr{Михаил Подоляк}

отличный микро/анализ входов-выходов в сложной теме... проблема в том, что ты чересчур умён для нынешних кабинетов (военных в том числе) и слишком логически чётко рассуждаешь... а в этих кабинетах везде дремучая дрем) и потому они на мир смотрят через, прости, задницу)...

\iusr{Константин Бондаренко}

Очень хороший анализ!

\iusr{Игорь Лесев}
Спасибо, Константин

\iusr{Евгений Отовчиц}
\textbf{Михаил Подоляк}, Константин Бондаренко да, только кому он нужен, этот анализ о Прибалтике и России.

\iusr{Дмитрий Коломийченко}

Прибалтика удобный плацдарм для первой операции против России и неудобное место
для обороны от неё. Пока российская армия считалась Западом величиной
несерьёзно по сравнению с собственными возможностями, прибалтийская позиция
никаких проблем не вызывала. Когда же ситуация изменилась, любой адекватный
анализ показал, что удержать эти страны принципиально невозможно, исключая
глобальный ядерный конфликт. Теперь возникает объективная необходимость
ответить на потенциальную русскую угрозу. Но если начинать концентрировать в
Прибалтике значительные воинские контингенты, то это многократно усиливает
напряженность, является неприкрытой угрозой (помним, что нападать от туда
легче) и поводом к конфронтации. При всей нелюбви к России оно им не нужно.
Поэтому ограничиваются средним вариантом с размещением постоянных небольших
контингентов НАТО, демонстрирующих флаг. Плюс учения по развертыванию в
угрожаемый период. Плюс учения по парированию русского удара в Прибалтике.
Сейчас вот такое в Норвегии проходит. Дополнительно Прибалтийские страны бедные
как церковные мыши, им будет очень трудно без европейских дотаций, поэтому они
рассматривают контингенты НАТО, как источник постоянного дохода. И чем больше
будет контингент, тем лучше. В общем это отражение объективной ситуации
контроля остро атакующей позиции.

\begin{itemize} % {
\iusr{Евгений Отовчиц}
Никто на Россию нападать не станет. Поэтому эти все домыслы не стоят ни гроша.

\iusr{Матвей Кублицкий}
\textbf{Евгений Отовчиц} а как же Грузия - расстрелявшая российских миротворцев? это же своего рода и было нападение на РФ

\iusr{Евгений Отовчиц}
\textbf{Матвей Кублицкий} Грузия не в НАТО. И там был абсолютно невменяемый ручной товарищ. Он бы и здесь в атаку на Донбассе пошел бы. Главное - погромче.

\iusr{Матвей Кублицкий}
\textbf{Евгений Отовчиц} за ним стояли именно товарищи из сша и он им полностью подчинялся. это и была проверка РФ на умение защититься

\iusr{Евгений Отовчиц}
\textbf{Матвей Кублицкий} а Крым, видимо, была проверка на возможность напасть?
В случае с Грузией было понятно, что шансы неравны. Грузия - это даже не Чечня. Там никто воевать не хочет.

\iusr{Матвей Кублицкий}
\textbf{Евгений Отовчиц} в крыму тоже никто не воевал. народ с радостью свалил с украины

\iusr{Евгений Отовчиц}
\textbf{Матвей Кублицкий} мне сложно представить латышей с ружьями.
Да и не нужно это никому. Прибалтика - обрусевшие немцы. Венигрет. Поэтому, когда Германия с Россией окончательно найдут общие цели на будущее (а на фоне торговой войны с США это сделать намного проще) - тогда сразу и Прибалтика окажется "за".

\iusr{Матвей Кублицкий}
\textbf{Евгений Отовчиц} не соглашусь что прибалты - обрусевшие немцы. совсем нет. даже далеко нет. это сами по себе отдельные народы со своей историей. и они точно не будут воевать с россией - ибо нафиг не надо. но даже когда германия и россия договорятся - то и тогда прибалты будут исполнять амерские приказы - в ущерб себе же

\iusr{Евгений Отовчиц}
\textbf{Матвей Кублицкий} но это путь в никуда. Худший вариант для самих же. Украина тому яркий пример.

\iusr{Матвей Кублицкий}
\textbf{Евгений Отовчиц} не спорю. они сами выбрали свой путь. и на их месте и на месте украины - самый выгодный путь развития был в дружбе и с ЕС и с РФ. умный телок у двух маток сосет - украинская народная мудрость. Тех же прибалтов россия уже 15 лет как потихоньку отодвигает от своей кормушки

\iusr{Евгений Отовчиц}
\textbf{Матвей Кублицкий} это оправданная практика США - дробить страны и создавать небольшие нетерпимые антироссийские анклавы. Это сделано в Прибалтике, Югославии, теперь в Украине. Маленький но гордый кусочек обходится дешевле, чем целая Украина или Югославия.

\iusr{Матвей Кублицкий}
\textbf{Евгений Отовчиц} вам не кажется что это уже было: разделяй и властвуй

\iusr{Дмитрий Коломийченко}
Военные другими мотивами руководствуются.

\iusr{Матвей Кублицкий}
\textbf{Дмитрий Коломийченко} военные выполняют команды политиков

\iusr{Дмитрий Коломийченко}
\textbf{Матвей Кублицкий} Я не про команды, а про мотивы писал. Разницу узнаете, поговорим.

\iusr{Матвей Кублицкий}
\textbf{Дмитрий Коломийченко} мотивы у всех разные. и меняются постоянно. вот вы про каких военных и про какие мотивы?

\iusr{Дмитрий Коломийченко}
\textbf{Матвей Кублицкий} Про нормальных, которые умеют оценивать потенциал угроз.

\iusr{Матвей Кублицкий}
\textbf{Дмитрий Коломийченко} не ответ. амеры одни угрозы видят. российские другие. украинские третьи

\iusr{Дмитрий Коломийченко}
\textbf{Матвей Кублицкий} В штабах принято оценивать всё объективно, если вы хотите продолжать меня троллить, то увольте.

\iusr{Матвей Кублицкий}
\textbf{Дмитрий Коломийченко} извините. даже не пытался троллить. просто хочу понять точно что обсуждаем. ваша точка зрения слишком расплывчата

\iusr{Дмитрий Коломийченко}
\textbf{Матвей Кублицкий} 

Есть понятие позиции, из анализа её потенциала исходят штабисты. Так как
геополитическая борьба идёт многие годы, то большинство основных участников
обладает одинаковым подходом к оценке геополитических позиций в отношении
потенциала. Политики оперируют именно этой оценкой. Характерный пример Крым.


\iusr{Матвей Кублицкий}
\textbf{Дмитрий Коломийченко} позиции чего? потенциал позиции - это что вообще? и каким боком тут крым? может начнете изъяснятся просто и понятно?. а то я могу пяток версий выдвинуть оценки позиций и потенциалов...

\end{itemize} % }

\iusr{Виктория Павликова}

В Прибалтике некому нападать на Россию, хоть там и создаются некие отряды
самобооны, но скорее для получения спонсорской помощи и удержания в тонусе
русскоязычного населения. В принципе, все очень уныло. Прибалты тужатся в
русофобии теряя прибыль от российского трафика уверенно сворачивающего мимо
поледние годы, а теперь еще теряют и белорусский. Шаг за шагом. Если бы не
калининградский плацдарм))) - с прибалтийскими тиграми уже было бы покончено и
территории заросли естественный образом лесом, что весьма бы способствовало
экологическому процветанию данных территорий.

\begin{itemize} % {
\iusr{Евгений Отовчиц}
А когда воссияет русско-германский союз - тогда Прибалтике станет веселее всех.

\iusr{Евгений Отовчиц}
\textbf{Игорь Ротмистров} Меркель уйдет (итальянцы помогут), Северный Поток 2 закончат, ЕС с США окончательно перейдут в режим торговой войны - и воссияет.
Не волнуйтесь.

\iusr{Евгений Отовчиц}
\textbf{Игорь Ротмистров} наверное, намного больше чести вывозить лес и брать кредиты.
\end{itemize} % }

\iusr{Матвей Кублицкий}

США ничего не делают просто так. всё ради денег. Российская угроза -
обоснование существования НАТО. И отчислений от стран-участников - 2\% ВВП. О
чем европейцам напомнил Трамп. РФ не пугается - просто должно отвечать на
угрозу, чётко понимая возможность агрессии в адрес нашей страны. Примеры есть -
Грузия, напавшая на российских миротворцев.

\iusr{Михаил Романов}

В политическом плане все верно, а вот анализ военной составлющей говорит о том,
что автор далек от военного планирования, комплекса оборонных мероприятий и
прочей штабной дребедени о которых известно тем, кто служил в армии

\iusr{Матвей Кублицкий}

Кстати, Игорь! про высокий уровень жизни прибалтики: товарищ на днях вернулся с
покатушек по европе. перед возвращением зарулил в Ригу. Сказал что Ригу можно
обозначить лишь одним словом - уныние

\begin{itemize} % {
\iusr{Матвей Кублицкий}
\textbf{Игорь Лесев} умыл... полностью умыл. даже не буду оправдываться

\iusr{Матвей Кублицкий}
\textbf{Игорь Лесев} но с другой стороны: судьба прибалтики предрешена и всем известна.а твоё, Игорь, отношение к российским вооружениям слишком субъективно и отражает незнание темы
\end{itemize} % }

\end{itemize} % }
