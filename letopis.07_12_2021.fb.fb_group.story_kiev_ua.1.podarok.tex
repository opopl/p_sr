% vim: keymap=russian-jcukenwin
%%beginhead 
 
%%file 07_12_2021.fb.fb_group.story_kiev_ua.1.podarok
%%parent 07_12_2021
 
%%url https://www.facebook.com/groups/story.kiev.ua/posts/1813294208867372
 
%%author_id fb_group.story_kiev_ua,fedjko_vladimir.kiev
%%date 
 
%%tags cvety,kiev,podarki,semja
%%title Подарунок: Символ чи обов’язково матеріальна річ?
 
%%endhead 
 
\subsection{Подарунок: Символ чи обов’язково матеріальна річ?}
\label{sec:07_12_2021.fb.fb_group.story_kiev_ua.1.podarok}
 
\Purl{https://www.facebook.com/groups/story.kiev.ua/posts/1813294208867372}
\ifcmt
 author_begin
   author_id fb_group.story_kiev_ua,fedjko_vladimir.kiev
 author_end
\fi

Подарунок: Символ чи обов’язково матеріальна річ? 

19-го липня 2019-го ми відзначили 45-річчя шлюбу! 

Напередодні я засидівся за комп’ютером до середини ночі, тому солодко спав десь
до 10-ї. Прокинувся, а Надійки нема – кудись побігла по своїх справах. 

Поснідав, випив кави і пішов за квітами... Стандартний ринковий асортимент мене
абсолютно не надихнув... І тоді я пішов у Пуща-Водицький ліс. Години дві бродив і
конструював букет з лісових квітів. Повернувся додому і подарував ікебану
Надійці!

Букет їй дуже сподобався!!! Надійка попросила мене сфотографувати букет і
одразу розіслала фото доньці, онукам та подругам. 

Через декілька днів, по справах, мав розмову з однією знайомою... років 50-ти.
Вона із захватом розповідала мені та ще декільком співробітникам про 25-річний
ювілей весілля своєї подруги і перераховувала подарунки, які подружжя
подарували один одному, та які подарунки вони отримали від рідні і друзів. 

Я необережно промовився, що у нас теж днями був ювілей – 45 років подружнього
життя. Знайома одразу запитала: «А що ти подарував дружині?»

• «Букет лісових квітів!»

• «І все!!!???!!!»

По очах знайомої бачу, що мій рейтинг стрімко котиться до рівня плінтусуІ 

• «Так, все!»

• «А вона тобі!?

• «Дружина мене поцілувала!»

• «І все!?»

• «Ні, не все. Я її теж поцілував! Хотів встановити рекорд поцілунку – 45
хвилин! Але дружина на 39-й хвилині втратила від щастя свідомість і довелося
віднести її в ліжко...»

• «І все!?»

• «Так, тепер все!»

Безмежне розчарування мною...

***

Ювілейний липень закінчувався і 31-го вранці я сказав, що ікебана вже майже
засохла і, можливо, букет треба прибрати. «Хай стоїть», — сказала Надійка і
мене поцілувала.

Не думаю, що якби я їй подарував якусь прикрасу чи матеріальну річ, то реакція
була б такою ж. 

І тільки коли букет майже осипався і втратив свій вигляд ми його прибрали.

Подарунок: Символ чи обов’язково матеріальна річ? 

***
