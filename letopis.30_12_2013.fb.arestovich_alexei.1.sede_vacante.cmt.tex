% vim: keymap=russian-jcukenwin
%%beginhead 
 
%%file 30_12_2013.fb.arestovich_alexei.1.sede_vacante.cmt
%%parent 30_12_2013.fb.arestovich_alexei.1.sede_vacante
 
%%url 
 
%%author_id 
%%date 
 
%%tags 
%%title 
 
%%endhead 
\subsubsection{Коментарі}

\begin{itemize} % {
\iusr{Анатолий Неёлов}
Алексей, ты тупо прав... однако...

\iusr{Алексей Арестович}

Однако, общество еще не попало в серьезную беду. Так - насморк.
Когда попадет - тогда все изменится.

\iusr{Andriy Horbal}
Не завалити найбільшого мамонта, а змінити клімат.

\iusr{Roman Bas'kov}
и тут Остапа понесло...)))

\iusr{Stas Boiko}

Эта задача стоит на несколько уровней выше, чем сейчас способно решать наше
общество исходя, как минимум, из текущего умения сотрудничать и доверять,
которое только начинает зарождаться. А в долгосрочной перспективе да, надо
стремиться к такой модели.

\iusr{Jaroslava Igorivna}
Кого виштовхати? Добровільно!

\iusr{Наталья Ната}
Мне кажется , что настолько народ устал смотреть на цирк власти, что уже готов на все. Раньше ляжем раньше Встанем!!!

\iusr{Сусанна Тритяченко}

Власть- это отложенное, но предъявленное насилие, остальное притчи.

\iusr{Алексей Арестович}
Насилие бывает очень разных видов. @igg{fbicon.smile} 

\iusr{Сусанна Тритяченко}
Совершенно верно, моральное, физическое и на благо тоже. заставлять умываться, это тоже насилие)) даже уговорами))

Dmitri Gurewitsch
Я надеюсь все догадались, что народ наконец то должен себя вытолкнуть во власть! Время представителей прошло.

\iusr{Наталья Ната}
Да... :((( :))))) но хочеться думать о позитиве!

\iusr{Лариса Васильченко}

Ви мені неприємний, оце Ваше дозоване інформування, це так противно. Ви наче
зверху і все знаєте. І Вам насрати на цих людішок, а якщо дехто з них оцінить
Вашу мудрість - їх щастя

\iusr{Алексей Арестович}
Природа насилия тоже изменилась. @igg{fbicon.smile} 

\iusr{Лариса Васильченко}
медитація триває

\iusr{Алексей Арестович}
\textbf{Лариса Васильченко}, если бы мне было насрать, зачем бы я писал?:)

\iusr{Юрий Юхман}
вырвать больной зуб - это насилие?

\iusr{Сусанна Тритяченко}
Да, только инстинкт власти пока еще нет, вот и несрастец.))

\iusr{Julia Polibina}

Из опыта: если не появляется лидер, то на сцену выдвигается анти-лидер...
Психологи говорят, что это - закономерность... Может, следует принять меры -
чтобы анти-лидер не появился хотя бы? (я к тому, что пост Президента можно было
бы временно упразднить - это бы сняло накал страстей немного)

\iusr{Dmitri Gurewitsch}
@Лариса: чтоб воспринимать информацию надо до нее дорасти

\iusr{Алексей Арестович}
Нам спешить некуда. Вокруг - Вечность. @igg{fbicon.smile} 

\iusr{Гелена Мурлян}
Нанять немецких администраторов и хозяйственников и все! Пусть научат!

\iusr{Лариса Васильченко}
Дмітрію Гуревичу - якщо пишете російською, то хоч її вивчіть - правильно "дорАсти"

\iusr{Наталья Ната}
Можно вопрос: а как вам львовяне? После семинара? в общем?

\iusr{Татьяна Архипова}

Алексей, а "серьезная беда" -это как? Когда уже умер или парализованный
инвалид? И в том качестве как есть сейчас народ сможет ли этот народ увидеть
того, кто поможет ему? Или, как всегда, нужные дяди таки красиво "подскажут"
кого "нужно". А потом будут убеждать, что все по решению народа... Пока
качество человека не измениться ничего не изменится у нас. Даже лидера, что
сможет помочь, увидеть и вытолкнуть к власти не смогут...

\iusr{Юрий Юхман}
вечность сама по себе, а мы сами по себе.

\iusr{Алексей Арестович}

Так секрет нового лидера для народа, не в лидере, а в народе.
Серьезная беда - это тяжелое положение, коснувшееся лично всех при условии, что никто не знает, что делать.

\iusr{Лариса Васильченко}
Так як Вам, Алєксєй, спєшшить нєкуда і купа у Вас співбесідників, ВЕСЕЛИХ ВАМ СВЯТ!

\iusr{Dmitri Gurewitsch}
@Лариса: пардон... исправлю

\iusr{Андрей Любалин}
Amor wincit omnia

\iusr{Алексей Арестович}
\textbf{Лариса Васильченко}, и Вам!:)

\iusr{Алексей Арестович}
Кто будет субъектом люстрации?

\iusr{Molekula Valentyna}
Зрозуміла кого треба штовхати у владу - одного мого голосу замало. Допоможіть організувати процес!

\iusr{Наталья Ната}
Тот кто будет выдвигаться на пост
\iusr{Olesia Grabovska}

Очень глубокий пост. Единственное, что меня остановило - "лидер, которого
назначили против его воли". Это сразу как-то нивелирует для меня предыдущие
слова "...власть новых времен - это власть, которую не надо навязывать вопреки
сопротивлению тех, над кем она осуществляется". Я думаю, что принцип не
навязывания власти (как и всего другого) должен действовать в обе стороны - и
обществу, и лидеру. И кроме того, очень сомнительно, что можно грамотно
распорядиться навязанной властью... Может, я чего-то недопоняла в выводах?


\iusr{Роман Цапик}

А не готов народ к новым лидерам, не готов... Потому как что б быть новыми
лидерами надо говорить правду, мало того надо делать, очень много делать, а еще
больше надо делать народу, а еще больше надо думать, учиться, проявлять
инициативу.... А народ не готов, делать, не готов думать, не готов учиться он
даже правду слушать не готов... Его этому не учили. Да и лидеры тоже не готовы
(я не о горыныче), у лидеров пока есть что терять, и пока есть что терять
систему не завалить... Система любого лидера съест и не подавиться. Печалька...
Но ничего не поделать, пока.

\iusr{Татьяна Архипова}

ребенок- семья- человек- народ-лидер. Меняться нужно нам. И революция здесь
никак не поможет, только эволюция человека как вида. Вопрос в том, хватит ли
времени что бы пройти череду изменений и стать действительно человеками? Или
"серьезная беда" придет раньше?:)

\iusr{Наталья Ната}
Я уверена , что есть такие люди! Но они наверное скромные вот и надо их выдвигать! Может и насильно...

\iusr{Andrew Sivach}
Достойных людей нужно уговаривать править.

\iusr{Andrew Sivach}
Это факт

\iusr{Александр Грабин}

Да и сам по себе лидер без команды либо впадет в дисперсию, если совестливый,
либо плюнет на все и примет правила игры окружения... Выталкивать во власть
надо целую толпу единомышленников!


\iusr{Julia Polibina}

Есть лидеры, которые не воспримутся сейчас... Люди не готовы их принять. Один
пример того что говорит В. Горбулин в статье "ЗН"... Я читала комменты - и вижу
что здравые мысли воспринимаются в штыки именно из-за того что "харизма лидера"
для текущего момента не воспринимается

\iusr{Andrew Sivach}
а можно ссылочку на статью?

\iusr{Julia Polibina}

\href{http://gazeta.zn.ua/internal/kamo-gryadeshi-_.html}{%
Камо грядеши?, Владимир Горбулин, zn.ua, 27.12.2013%
}

\iusr{Ігор Скальський}

\href{http://www.snob.ru/profile/5232/blog/56510}{%
Борис Акунин: Теория поплавка, snob.ru, 13.01.2013%
}

це хвороба, хворіє зранений
організм, з кожним просуванням Ординської хвороби, в 13ст, в 17ст, 18ст, 20ст,
рецидивами, хвилями, все далі на Захід, ми відбиваємось, а воно знову накочує .
Пора йти назад 

\url{http://www.youtube.com/watch?v=zdrcgLqdbnE}

\iusr{Наталья Ната}

А Соскин !? смотрела на 112 кан сегодня. Тоже, так не скрывая правду говорил.
Мне нравятся его статьи. Умный человек


\iusr{Ирина Стежка}

Есть же такое понятие БРЕМЯ власти, тягар по-нашему)) Когда была жива лидер
сахаджа-йоги, то вибрационно выбирался самый центральный человек (гармоничный), в
подробности сейчас не буду вдаваться, и он был лидером - страны или города, но
лидерство ограничивалось координаторством, а если начинал давить, то у него сразу
конечно падал вибрационный фон и его снимали)) Думаю, что по-большему счету в
будушем лидеры будут такими координаторами, которые будут заинтересованны в
том, чтобы все ладилось, а значит было в руках умных и знающих людей, профи и
конечно порядочных, ведь без этого ничего не скоординируешь. При такой ситуации
эго не задействовано - лидер как бы и не делает ничего, есть команда, где каждый
в ответе за свой участок. Это высшая форма лидерства, когда люди думают, что они
сами все делают и над ними нет никого.

\iusr{Julia Polibina}

Да, Наталья! Я смотрела его - и не раз! Очень разумный подход... НО! Он сам
говорит: его НЕ ЗОВУТ, не просят поделиться мнением и рассказать... в нем не
заинтересованы. Именно об этом я и говорю. Не в лидерах вовсе дело. "Тройняшек"
устраивает ситуация - они вовсе ничего менять не хотят. И не собираются! Именно
поэтому есть опасность возникновения "анти-персонажа", который может негативно
повлиять на ситуацию...

\iusr{Sergey Marchuk}

Для того, чтобы эта абсолютно правильная идея овладела массами и в конце концов
реализовалась, - одного поста в одном фейсбуке недостаточно. Даже с учетом
перепостов. @igg{fbicon.smile} 

\iusr{Роман Цапик}

Достойный, лидер, править, эффективно не бывает.... Ирина Стежка опередила
задача не править а координировать и модерировать работу.

\iusr{Петро Обухов}
habemus papam

\iusr{Віталій Коваляк}

Поки що люди хочуть, щоб прийшов хтось і вирішив всі їхні проблеми. Але без
їхньої участі. Ми завжди знайдемо тисячу відмазок, лиш би не змінюватися самим.

\iusr{Valerii Borys}

Гадаю, духовні лідери є (Філарет, Гузар, Ліна Костенко, ініціатива 1 грудня
тощо) і було (Шевченко, Франко, Бандера) вже їх достатньо, і зробили свою
справу: повстали люди, стоїть Майдан і активісти піднімаються скрізь по країні

\iusr{Максим Банасевич}

Я сегодня много от кого слышу что "есть лидеры"... ребята - это не лидеры. Это
люди которые хотели бы ими быть... Реальный лидер - очень быстро соберет вокруг
себя инициативную группу. Или придет и подчинит уже существующую (я не про ту
которая сегодня сидит в доме профсоюзов). Не знаю как вы - пока что я таких
лидеров не видел. То что люди поддаются риторике оппозиции - тут все понятно.
За не имением других - слушают тех кто есть. Но - предчувствую что это скоро
переменится.

\iusr{Наталья Ната}
Юля! Согласна с вами.

\iusr{Максим Банасевич}
По поводу "уговаривать" - я бы не согласился. Хотя и такой вариант может быть. Всяко история знала.

\iusr{Максим Банасевич}
я бы не согласился что это тот случай =))) в смысле  @igg{fbicon.face.smiling.eyes.smiling} 

\iusr{Julia Polibina}

Лидер - это человек, который не только согласен "нести лавровый венок", но и
готов во имя ОБЩЕГО ДЕЛА всего себя, полностью и безраздельно ему посвятить. И
быть готовым к тюрьме, к пыткам и к смерти... И что?... Кто-то пойдет на это
если его "заставить"? Тут нужна не только воля и вера, не слепое подчинение
"толпе" - нужно осознанное ПОНИМАНИЕ ЦЕЛИ! Нет лидеров... нужно ситуацию
выравнивать... Алексей написал - стволов на руках много (и это, увы, правда)...

\iusr{Алексей Арестович}
Спасибо, Антон. @igg{fbicon.smile} 

\iusr{Игорь Зюбин}
\textbf{Алексей Арестович}, я выталкиваю Вас во власть! Кто за? Прошу голосовать!

\end{itemize} % }
