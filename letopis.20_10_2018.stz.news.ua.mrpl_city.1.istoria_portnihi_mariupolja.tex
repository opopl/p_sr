% vim: keymap=russian-jcukenwin
%%beginhead 
 
%%file 20_10_2018.stz.news.ua.mrpl_city.1.istoria_portnihi_mariupolja
%%parent 20_10_2018
 
%%url https://mrpl.city/blogs/view/istoriya-portnihi-mariupolya
 
%%author_id burov_sergij.mariupol,news.ua.mrpl_city
%%date 
 
%%tags 
%%title История: портнихи Мариуполя
 
%%endhead 
 
\subsection{История: портнихи Мариуполя}
\label{sec:20_10_2018.stz.news.ua.mrpl_city.1.istoria_portnihi_mariupolja}
 
\Purl{https://mrpl.city/blogs/view/istoriya-portnihi-mariupolya}
\ifcmt
 author_begin
   author_id burov_sergij.mariupol,news.ua.mrpl_city
 author_end
\fi

\ii{20_10_2018.stz.news.ua.mrpl_city.1.istoria_portnihi_mariupolja.pic.1}

Не говоря уже о годах войны, в первое десятилетие после нее бесполезно было
искать в мариупольских магазинах готовое дамское платье, блузку или тем более
пальто. Такие вещи можно было купить в Мариуполе разве что на \enquote{толчке}
- огороженной площадке на главном и старейшем в городе базаре, где торговали
всем тем, что можно надевать, обувать, на чем спать, в чем готовить пищу, чем
стирать, украшать быт и тому подобное, что обозначалось придуманным в советскую
эпоху словечком – \enquote{промтовары}. Где базар находился? А там, где в
начале проспекта Ленина стоит здание канувшего в Лету ДОСААФ, а также вся
площадь за ним.

Вот на этом-то \enquote{толчке} мариупольские модницы и могли прикупить себе
наряд. Это могло быть крепдешиновое платье – последнее достояние какой-нибудь
вдовы, сохранившееся с довоенных времен, или наряд из белого льняного полотна,
излюбленное облачение пожилых интеллигенток, переживших революцию семнадцатого
года, можно было встретить также \enquote{трофейные платья и костюмы},
привезенные из Германии демобилизованными офицерами, реже - солдатами. Здесь
можно было встретить юбку, перешитую из офицерской диагоналевой гимнастерки,
блузку из парашютного шелка, \enquote{шикарный} жакет, сооруженный из портьеры.
Для краткости скажем, что фасоны дамского ассортимента \enquote{толчка} по
временным меркам простирались от начала века двадцатого до его середины, не
пропуская ни единого этапа перемен в модах. Но и это еще не все - из-под полы
можно было купить отрез ткани, в том числе и входящего в моду штапельного
полотна.

\textbf{Читайте также:} 

\href{https://mrpl.city/news/view/mariupolets-dedushka-vova-raskryl-neskolko-faktov-iz-svoej-zhizni-video}{%
Мариуполец дедушка Вова раскрыл несколько фактов из своей жизни, Олена Онєгіна, mrpl.city, 20.10.2018}

Итак, обнова приобретена. Если что-нибудь готовое (а оно далеко не всегда
подходило по размеру), - приходилось что-то подрезать, подшивать, что-то
доставлять. Хорошо, когда счастливая обладательница нового наряда  или куска
ткани \enquote{дружила} с ножницами и иголкой. А если нет? Что тогда делать с куском
ткани? Вот на эту тему и пойдет разговор. Исполнение фасонов той поры требовали
большого мастерства -  сложный крой, всевозможные украшения: воланы, оборки,
рюши, аппликации, фестоны, плиссе. И только опытные \textbf{портнихи} – их называли в
народе \textbf{модистками} - могли воплотить мечту или каприз заказчицы в реальный
наряд.

\ii{20_10_2018.stz.news.ua.mrpl_city.1.istoria_portnihi_mariupolja.pic.2}

В то достославное время в нашем городе существовали швейные мастерские под
громким названием - \textbf{ателье мод.} Их было очень мало, но они были. Однако
пользовались ими местные модницы почему-то неохотно – то ли дорого было, то ли
не устраивало качество исполнения заказов, но факт остается фактом – туда мало
кто обращался. И вся нагрузка ложилась на модисток, практикующих, так сказать,
частным образом. Занятие это для них было сопряжено с неприятными
обстоятельствами. Дело в том, что всякая ремесленная деятельность облагалась
таким налогом, что мастеру или мастерице, заплатив его, не всегда оставалось
достаточно средств даже на скромное пропитание. Тем более для портних, у
которых не всякий месяц были заказчицы. Вот и шла между ними и фининспекторами
незримая борьба. Первые придумывали, как обойти уплату налога, а вторые
старались выследить нарушительницу закона, а если такое удастся, получить
взятку.

Одна из модисток, назовем ее, допустим, Варварой Степановной, принимала заказы
только от тех клиенток, у которых дома были швейные машины. Так она и кочевала
по Мариуполю из дома в дом, зарабатывая себе на кусок хлеба. Ее привечали не
только за то, что она хорошо шила, но и за осведомленность в делах и событиях
многих мариупольских семейств, сообщения ее всегда были только хорошего и
наилучшего содержания. Это гарантировало, что и из этого дома не будет \enquote{вынесен
сор из избы}.

Ходила в былую пору байка, как Ангелина Матвеевна – едва ли не самая популярная
портниха у мариупольских модниц – была \enquote{застукана} фининспектором. Она
сметывала детали будущего платья, когда в дверь кто-то вкрадчиво постучался.
Ангелина Матвеевна выглянула в окно и обмерла. У порога ее дома стоял враг –
фининспектор, а на столе у нее лежит явная \enquote{улика} - наполовину сметанное
произведение ее искусства.  \enquote{Произведение}, которое ни по росту, ни по объему
не могло бы шиться для себя. Что делать? Решение пришло само собой. Ангелина
Матвеевна открыла верхнюю крышку пианино и сунула внутрь инструмента свою
скомканную недошитую работу. Теперь можно было открывать дверь. Представитель
власти поздоровался, сел без приглашения за стол и завел разговор о том о сем.
И тут хозяйка квартиры увидела - на столе остались предметы ее ремесла:
наперсток, сантиметр, подушечка с воткнутыми в нее иголками и булавками, мелок.
Однако гость делал вид, что ничего не заметил. Он обвел взглядом комнату,
лениво подошел к пианино, открыл клавиатуру, ткнул несколько раз пальцем в
клавиши. Вместо звона струн послышались глухие удары молоточков о что-то
мягкое...

\textbf{Читайте также:} 

\href{https://mrpl.city/news/view/novyj-god-ne-za-gorami-mariupoltsy-gotovyat-podarki-v-doma-prestarelyh}{Новый год не за горами: Мариупольцы готовят подарки в дома престарелых, Яна Іванова, mrpl.city, 17.10.2018}

Не все детали будущего наряда модистки делали сами. Например, плиссе - мелкие,
заложенные одна на другую параллельные складки на материи – они заказывали
Эстонке, так называли мастерицу, у которой национальность превратилась в имя
собственное. Как ее звали на самом деле, никто не знал. Более того, из тех, кто
заказывал ей плиссе, никогда ее не видел. Известен был только небольшой домик
на Мало-Садовой улице, в котором она обитала. Было ли это строение ее
собственностью или она квартировала здесь у хозяев – оставалось загадкой.
Впрочем, никто из портних не пытался проникнуть в тайны Эстонки. А как же
удавалось делать заказы? Да очень даже просто. Подходили к домику Эстонки,
вызывали ее троекратным деликатным стуком в окно. Край белоснежной
накрахмаленной занавески чуть-чуть отгибался и тут же падал. Открывалась узкая
форточка в окне, высовывалась сухонькая кисть руки, тонкие цепкие пальцы
принимали заказ – пакетик с полосками ткани для плиссировки и деньгами за
работу. Модистки знали, что за выполненным заказом нужно будет прийти завтра, в
то же самое время. Так что ни заказчица, ни исполнительница заказа ртов не
открывали.

Работа у портних была нервная. Мало того, что они жили в постоянном страхе быть
\enquote{застуканными} фининспектором, нужно было не ошибиться в раскрое платья, чтобы
для него хватило материала. Да и заказчицы попадались часто капризными. Многие
из них, имея комплекцию Одарки из оперы \enquote{Запорожец за Дунаем}, хотели
выглядеть, как Любовь Орлова – советский эталон женской красоты. Атрибуты
портняжного ремесла – манекены, на которых производилась подгонка деталей
изделия, исчезли из употребления портних-одиночек  еще в первые годы
индустриализации, когда началось наступление на ремесленников. Исчезли потому,
что наличие в квартире манекена прямо указывало на то, чем занималась ее
обитательница. Поэтому приходилось первичную сметку, например притачивание
рукавов, производить на ком-то из близких – сестре, взрослой дочери, а если не
было таковых под рукой, то и на подростке-мальчишке.

Но были и такие портнихи, которые не хотели связываться с заказчицами, а может,
просто не обладали достаточным мастерством. Представительницей такой
разновидности профессии была Марта Ивановна. Она занималась шитьем флотских
брюк. Этот предмет мужской одежды пользовался спросом местных франтов. В таких
брюках, а еще в белой сорочке, в распахнутый ворот которой выглядывали полоски
матросского тельника, к тому же иногда еще в фуражке-мичманке на голове
\enquote{модники} щеголяли на проспекте Республики летними вечерами.

Эту радость лжеморяков Марта Ивановна изготавливала в будние дни, а в
воскресенье выносила их на \enquote{толчок}. Она была бездетной вдовой, на иждивении
которой была старшая сестра – выпускница Мариупольской Мариинской женской
гимназии, считавшая зазорным заниматься портновским ремеслом и домашними
делами, ее дочь – актриса местного театра, брошенная очередным мужем,
малолетняя внучка сестры, а еще сын сестры - семнадцатилетний даун Валька.
Именно Валька, как ни странно, был главным помощником Марты Ивановны. Пока его
тетка торговала, он шатался по \enquote{толчку}. Но это не было пустым
времяпрепровождением. Дело в том, что власти запрещали торговать ремесленникам
своими изделиями. За тем, чтобы этот запрет не нарушали, следила милиция. И в
задачу Вальки входило вовремя предупредить тетку о приближении милиционера,
дабы она успела спрятать брюки в кошелку, наполовину заполненную рваньем. Но
однажды Марту Ивановну все-таки страж порядка потащил в участок. Одной рукой он
вел задержанную, а в другой нес вещдок – новенькие флотские штаны. Валька,
увидев эту безрадостную картину, поплелся вперевалку в участок. Там он увидел
заплаканную тетушку и злополучные брюки на столе у начальника. И тут Валька
проворчал: \emph{\enquote{Дура, зачем купила мне эти штаны. Я ж тебе долбил, долбил: большие
они на меня}}. И к милиционеру: \emph{\enquote{Начальник, отпусти ее, это я послал тетку
продавать штаны}}. Тетка была отпущена...

\textbf{Читайте также:} 

\href{https://mrpl.city/news/view/v-mariupole-luchshij-uchitel-priazovya-posvyatila-pobedu-osobennym-detyam-foto}{В Мариуполе лучший учитель Приазовья посвятила победу особенным детям, Анастасія Папуш, mrpl.city, 19.10.2018}

Героини этого повествования давно ушли в мир иной. Они, конечно, не могли
предположить, что наступят такие времена в нашей стране, что портняжным
ремеслом можно будет заниматься открыто и даже почетно.
