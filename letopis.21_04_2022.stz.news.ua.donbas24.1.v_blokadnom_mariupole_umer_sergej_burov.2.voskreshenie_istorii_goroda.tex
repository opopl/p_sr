% vim: keymap=russian-jcukenwin
%%beginhead 
 
%%file 21_04_2022.stz.news.ua.donbas24.1.v_blokadnom_mariupole_umer_sergej_burov.2.voskreshenie_istorii_goroda
%%parent 21_04_2022.stz.news.ua.donbas24.1.v_blokadnom_mariupole_umer_sergej_burov
 
%%url 
 
%%author_id 
%%date 
 
%%tags 
%%title 
 
%%endhead 

\subsubsection{Воскрешение истории города}

Сергей Буров любил Мариуполь и его прошлое. Чтобы восстановить страницы
истории, он опрашивал людей, узнавал, как жили в старинных домах горожане, и
все описывал в своих очерках. Точность и достоверность фактов при их создании
имели для него принципиальное значение.

Он был членом краеведческого общества и тесно сотрудничал с городским музеем.

По словам заведующего отделом научно-просветительской работы Мариупольского
краеведческого музея Александра Горе, роль Сергея Давыдовича в жизни города
была велика — он внес значительный вклад в воскрешение утерянных страниц
истории или неправильно трактованных.

\ii{21_04_2022.stz.news.ua.donbas24.1.v_blokadnom_mariupole_umer_sergej_burov.pic.3}

\begin{leftbar}
	\begingroup
		\bfseries
\qbem{Мы виделись с ним перед войной. Сергей Буров постоянно поддерживал с
нами связь, участвовал в мероприятиях. Он был из наиболее известных,
авторитетных и уважаемых краеведов города, который своей работой
отчасти возвращал память о тех периодах истории, свидетелем которых был
сам, проводил очень огромную работу, опрашивая людей. Он активно
сотрудничал с коллективом музея и над программами, которые знал весь
город, никому не отказывал в консультациях. Его публикации и издания
касались дореволюционного периода и послереволюционной жизни. Понимаю,
как ему было тяжело в такой ситуации — город исчезал на глазах}, —
сказал Александр Горе.
	\endgroup
\end{leftbar}

\ii{21_04_2022.stz.news.ua.donbas24.1.v_blokadnom_mariupole_umer_sergej_burov.pic.4}

