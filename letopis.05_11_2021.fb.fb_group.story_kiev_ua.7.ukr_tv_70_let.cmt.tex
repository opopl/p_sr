% vim: keymap=russian-jcukenwin
%%beginhead 
 
%%file 05_11_2021.fb.fb_group.story_kiev_ua.7.ukr_tv_70_let.cmt
%%parent 05_11_2021.fb.fb_group.story_kiev_ua.7.ukr_tv_70_let
 
%%url 
 
%%author_id 
%%date 
 
%%tags 
%%title 
 
%%endhead 
\subsubsection{Коментарі}

\begin{itemize} % {
\iusr{Nina Lazarenko}
У нас був такий телевізор який показував одну програму із лінзою яка збільшувала єкран

\iusr{Ольга Кожедуб}
Благодарю, Максим за информацию! Красиво и душевно написано! СПАСИБО!

\iusr{Георгий Горбенко}

У мас дома тоже такой был, только линза квадратная. Позже в сарае был - там у
меня ,,штаб,, был - две программы ловил - без ничего. Хорошоя штука - примерно
до конца 1979 года - работал, потом мы переехали.

\begin{itemize} % {
\iusr{Maksim Pestun}
такой?

\ifcmt
  ig https://scontent-frx5-1.xx.fbcdn.net/v/t39.30808-6/253719338_3049324725337688_8164047446433517493_n.jpg?_nc_cat=100&ccb=1-5&_nc_sid=dbeb18&_nc_ohc=xmcVcj5O4DwAX9R9vtx&_nc_ht=scontent-frx5-1.xx&oh=b1f4249d39d2d1a7ec2f8c3db3310993&oe=618D7AF9
  @width 0.4
\fi


\iusr{Георгий Горбенко}
Точно такая. Однажды мы надолго оставили отодвинутую линзу летом, так у нас стул начал тлеть. Хорошо что пожара не было.

\iusr{Мария Дубровская}
\textbf{Георгий Горбенко} У моей тети тоже с квадратной линзой был
\end{itemize} % }

\iusr{Нина Логвина}
А мы из такой линзы сделали аквариум.

\iusr{Vitali Andrievski}
У нас был один из первых на Подоле ткой телевизор назывался КВН

\begin{itemize} % {
\iusr{Maksim Pestun}
КВН-49

\iusr{Vitali Andrievski}
Спасибо что напомнили. Я помню что были какие то цифры, но не помню какие. 49 это наверное год начала производства этого чуда тогдашней техники
\end{itemize} % }

\iusr{Нина Логвина}

Жили на Горького, в коммунальной квартире (5 семей). Телевизор КВН был только у
нас. Каждый вечер соседи приходили к нам "на телевизор", иногда со своими
стульями, было весело.

\iusr{Вера Гержод}

Был такой же и в нашей семье. Вот только линза была овальная. Долго нам служил,
а потом появился телевизор "Березка". Для нас это был уже "космос", т.к. экран
побольше и линза уже не нужна.

\iusr{Надежда Залескова}
А у нас Знамя , он сразу был с « большим» экраном))))

\iusr{Maksim Pestun}

а у нас на Ленина был такой Ленинград

\ifcmt
  ig https://scontent-mxp1-1.xx.fbcdn.net/v/t39.30808-6/254063010_3049331378670356_4363182306049872749_n.jpg?_nc_cat=105&ccb=1-5&_nc_sid=dbeb18&_nc_ohc=gnQzRvCAsQMAX-B-eEu&_nc_ht=scontent-mxp1-1.xx&oh=d3a4a732a7ba6e667883b76585f1549e&oe=618C273D
  @width 0.4
\fi

\begin{itemize} % {
\iusr{Вадим Кречмар}
\textbf{Maksim Pestun} с приемником и проигрывателей ?

\iusr{Вадим Кречмар}
Мой отец настаивал вышку и оборудование на Крещатике.

\iusr{Maksim Pestun}
вроде бы нет... не помню

\iusr{Людмила Каштан}
\textbf{Maksim Pestun} Да, были приёмник и проигрыватель. У нас потом долго на даче пластинки лежали

\iusr{Анна Солдатова}
\textbf{Maksim Pestun} Там шторка правая двигается, можно закрыть, спрятать экран телевизора
Сверху проигрыватель...
Радио тоже, кажется, есть

\iusr{Людмила Мозговая}
\textbf{Maksim Pestun} такой я помню только у друзей моих родителей...

\iusr{Наталия Педос}
\textbf{Maksim Pestun} Крутой!
Спасибо Вам за интересный рассказ, вернувший в далекие-далекие и безвозвратно ушедшие времена детства!

\iusr{Виктория Штамковская}
\textbf{Maksim Pestun} И у нас первый такой был , левая сторона радиоприёмник.
\end{itemize} % }

\iusr{Павел Бурунов}
Из линз аквариумы делали.

\iusr{Вадим Чемерис}
Тільки лінза заливалась не водою, а гліцеріном

\begin{itemize} % {
\iusr{Maksim Pestun}
это по идее! @igg{fbicon.smile} 

\iusr{Ирина Казачкова}
\textbf{Вадим Чемерис} в аптеке покупали дисцилированную воду и заливали

\iusr{Maksim Pestun}
а если не было "гербовой", то лили простую...
\end{itemize} % }

\iusr{Ірина Оснач}
А у нас линза была квадратная. А сам телевизор у меня на даче, жаль выбрасывать.

\begin{itemize} % {
\iusr{Елена Гугунава}
\textbf{Ірина Оснач} совсем скоро станет антиквариатом! Храните)
\end{itemize} % }

% -------------------------------------
\ii{fbauth.shurhoveckaja_alla.kiev.ukraina}
% -------------------------------------

У нас тоже, один из первых в 2 х эт. Доме на ул. Артема. 12 м. Комната. И каждый
со своим стульчиком нсмотрел эту диковинку по тем временам. Это был 1952-53 г.

\begin{itemize} % {
\iusr{Алла Шурховецкая}

\ifcmt
  ig https://scontent-frx5-2.xx.fbcdn.net/v/t39.1997-6/p480x480/105941685_953860581742966_1572841152382279834_n.png?_nc_cat=1&ccb=1-5&_nc_sid=0572db&_nc_ohc=Hf5CVU3vpT8AX_wQkdB&_nc_ht=scontent-frx5-2.xx&oh=daab3ee9bb0ca206f94e69c2686c79f5&oe=618C7BCB
  @width 0.2
\fi

% -------------------------------------
\ii{fbauth.shvecova_natalia.kiev.ukraina}
% -------------------------------------

\textbf{Алла Шурховецкая} 

Аллочка! У нас та же история - соседи приходили вечером, усаживались, правда
стулья были наши. У соседей был мальчик - на пару лет старше нас с сестрой. А
тогда были фильмы "детям до 14 нельзя". Он оставался смотреть, а нас мама
выпроваживала. Обидно до слёз. А теперь все это вспоминается с таким теплом.

\begin{itemize} % {
\iusr{Алла Шурховецкая}
\textbf{Наталья Швецова} 

Да. Приятные воспоминания.  Наш путь был сложным и одновременно счастливым. У
меня лично нет претензий к той системе и времени.ж

\iusr{Алла Шурховецкая}

\ifcmt
  ig https://scontent-frx5-2.xx.fbcdn.net/v/t39.1997-6/s168x128/47270791_937342239796388_4222599360510164992_n.png?_nc_cat=1&ccb=1-5&_nc_sid=ac3552&_nc_ohc=UxuRvQAuwp8AX8iWqBe&_nc_ht=scontent-frx5-2.xx&oh=f52e49e8b3394e7acfe9778e04e2efbb&oe=618BA136
  @width 0.1
\fi

\end{itemize} % }

\end{itemize} % }

\iusr{Melnicuk Vladimir}
Отето ГАДЖЕТ!!!

\iusr{Виктория Самарина}
Очень интересно!

\iusr{Vadym Shvydkiy}

Да... Помню... И про "пульты" помню и про ДЗЗ на Первомайского и про огромные
ПТСы (пел в Большом Детском Хоре Гостелерадио УССР у Татьяны Николаевны
Копыловой)... Было время...

\iusr{Анна Сидоренко}

Очень хорошо всё помню, первый телевизор не все могли купить, у моей подружки
по школе отец был машинистом поезда, там зарплата была хорошая и они первыми
купили, а мы, дети, ходили к ним на интересные передачи, они сделали даже
маленькие стульчики, а кому не хватало, то просто на полу располагались.

\iusr{Лидия Перкина}

У нас был такой телевизор с линзой, помню бабушка очень любила смотреть
фигурное катание, брат пробовал окрашивать жидкость или наклеивал пленку с
разными цветами, чтобы создать эффект цветного изображения

\begin{itemize} % {
\iusr{Regina Kostetska}
\textbf{Лидия Перкина} Особенно фигурное катание с Белоусовым и Протопоповой!

\iusr{Лидия Перкина}
\textbf{Regina Kostetska} Совершенно верно

\iusr{Лариса Олейникова}
\textbf{Regina Kostetska} Белоусова и Протопопов.
\end{itemize} % }

\iusr{Елена Ходорко}

...у нас такой был, КВН, с линзой, мамочка в ней периодически меняла воду....

\iusr{Kravets Irina}
Поздравляю, Максим!
И спасибо за интересные воспоминания! @igg{fbicon.thumb.up.yellow} 

\iusr{Валентина Белозуб}

Я тоже закончила КПИ электроакустический факультет! Только мы учились на
"собачке" как отдельное государство! Отличное было время! А затем 40 лет было
отдано служению Украинскому телевидению на Крещатике 26, а затем на
Мельникова! Поздравляю, коллеги с праздником Телевидения!

\iusr{Светлана Онойко}

И у нас был такой в моем детстве.

\iusr{Светлана Аникина}

Я маленькая совсем была и пыталась в этой линзе увидеть рыбок - думала, что это
такой аквариум

\iusr{Лариса Давиденко}

Да уж, на подъезд был один такой телевизор.
Каждый со своим стулом приходил.
Если не сделал уроки, плач в кулочок!!

\iusr{зоя Барабанщикова}

Эх, вспомнить бы первого диктора нашего телевидения Ольгу Даниленко! Мы были
влюблены в неё. Может, у кого-то есть фото?

\iusr{Gennady Henry Sergienko}

Спасибо большое за воспоминания!

Помню, в середине 70-х, после того как появилась третья программа, по квартирам
ходили народные умельцы, которые за небольшую мзду настраивали переключатель
телевизора так, чтобы все программы переключались подряд.

\begin{itemize} % {
\iusr{Maksim Pestun}
И я помню
\end{itemize} % }

\iusr{Олена Медведева- Прицкер}

А я вспомнила магазин, в котором продавали эти первые телевизоры. На
Крещатике, рядом с Центральным гастрономом. Там записывались в очереди: «за
большим» - Ленинград и «за маленьким» - КВН

\begin{itemize} % {
\iusr{Volodymyr Volodymyr}
\textbf{Олена Медведева- Прицкер}, у бабушки с дедушкой был "Ленинград" - телевизор с прямоугольной водной линзой, совмещённый с радиоприемником.

\iusr{Алексей Бречко}
\textbf{Олена Медведева- Прицкер} ты, что Вечная?

\iusr{Алексей Бречко}
дай бог тебе здоровья !
\end{itemize} % }

\iusr{Милана Пономаренко}

Раньше телевидение не раздражало.....

\iusr{Ada Dubinski}

Наш первый телевизор КВН наклеивали пленки чтобы он выглядел цветным и помню
закрашивали линзу чтобы казался цветным

\iusr{Ludmila Vorobiova}

Максим, думаю, что Вы должны помнить маму моей школьной подруги-Цилечку
Юровскую, звукорежиссера Украинского телевидения, которая отдала ему с любовью
всю свою жизнь.

\begin{itemize} % {
\iusr{Maksim Pestun}
По имени помню. Но я работал в отделе Кинохроники, и у нас был свой звукоцех.

\iusr{Неонила Сваток}
\textbf{Ludmila Vorobiova} 

Я помню милейшую Цилю Самойловну. Она перед каждым эфиром или записью говорила,
обращаясь сразу ко всем - режиссерам, инженерам и ведущим: "Деточка, давай
сделаем пробочку!" Как-то возле четвертой студии повесили плакат, там было
поздравление с юбилеем и написано было "Поздравляем Цецилию... "Я как раз
столкнулась с ней под этим плакатом и сказала: "Какое у вас красивое имя!" - "Да
нет, я просо Циля, не знаю, почему они так написали"

\iusr{Ludmila Vorobiova}

Спасибо Вам большое за память. Циленька была человеком огромной доброты, которой
она согревала всех окружающих, в моей памяти она осталась навсегда.

\begin{itemize} % {
\iusr{Неонила Сваток}
\textbf{Ludmila Vorobiova} 

В 90-х из Израиля мне прислала фотографии режиссер нашей редакции Вита
Белявская. Она там встретилась с Цилей Самойловной и ее мужем, который сказал:
"Я выучил на иврите 6 слов, и мне хватает, а Циля все ходит и еще учит"

\iusr{Ludmila Vorobiova}

Спасибо Вам за добрые слова. Эта семья в Киеве была уникальная. Фимочка и Циля
принимали всех. В их доме всегда было тепло всем, их друзьям и нам друзьям
девочек. Ты всегда знал, если тебе что-то надо, говори тете Циле -поднимет весь
Киев. В Ерусалиме мне говорили-таких домов здесь больше нет, а я отвечала-и в
Киеве таких не было. Светлая память этим замечательным людям.

\end{itemize} % }

\iusr{Светлана Веник}
\textbf{Ludmila Vorobiova} 

Помню Цилю Соломоновну. Всегда весёлая, задорная и как-то особенно по-доброму
относилась к нам, молодняку.

\end{itemize} % }

\iusr{Георгий Столяров}

Автору инженеру акустику сообщу, что у большинства других в линзах была не
мутноватая вода, а китайский глицерин.

\begin{itemize} % {
\iusr{Valery Sakhno}
\textbf{Георгий Столяров} 

єто сейчас глицерин бьі бьіл китайским, а тогда бьіло столько мьіловаренньіх
заводов, шо ой-ей. Ну, как-бьі, главньій продукт-мьіло, хотя на самом
деле-мьіло =побочка. Другое дело, шо лучший глицерин тогда шел на взрьівчатку,
и даже назьівался "динамитньім". Но на телевизорьі хватало, бо партия сказала,
строго: "надо"

\begin{itemize} % {
\iusr{Георгий Столяров}
\textbf{Valery Sakhno} У меня долго хранилась жестяная канистра с иероглифами.

\iusr{Valery Sakhno}
\textbf{Георгий Столяров} китайские товарищи тогда еще, похоже о производстве динамита не парились, поєтому и поставлялю союзу его. За технику отсюда
\end{itemize} % }

\iusr{Maksim Pestun}
я знаю, что был глицерин. Но здесь ключевое слово "был"...

\end{itemize} % }

\iusr{Marina Kovalenko}
Я помню такой телевизор @igg{fbicon.beaming.face.smiling.eyes} 

\iusr{Юрій Марков}

"КВН" - это не название "клуба весёлых и находчивых", а Кенигсон, Варшавский,
Николаевский, т е. аббревиатура этих фамилий, основных конструкторов этого на
то время "чуда техники". Да были телевизоры, в которых щелкающие переключатели
нередко выходили из строя (потому что переключать нужно было, как правильно
здесь уже было сказано через несколько перещелков и с УСИЛИЕМ) и приходилось
брать плоскогубцы... Передачи: Время, Кабачок 13 стульев (к названным добавим
колоритных пана Гималайского, пана спортсмена, пана профессора....), В мире
животных, Клуб кинопутешествий... ну последние это было попозже... До и после
полуночи с Владимиром Молчановым... много было интересных передач... Утренняя
почта с Юрием Николаевым...

\begin{itemize} % {
\iusr{Георгий Столяров}
\textbf{Юрій Марков} да. А КВН был на три канала. Рычажок сзади. Плоскогубцы не нужны!

\begin{itemize} % {
\iusr{Юрій Марков}
\textbf{Георгий Столяров} да я уже про другие: Весна, ещё был такой старенький Рекорд что-ли...

\iusr{Георгий Столяров}
\textbf{Юрій Марков} да, эти я знаю!
\end{itemize} % }

\end{itemize} % }

\iusr{Людмила Любарская}
И у на такой был !!!

\begin{itemize} % {
\iusr{Людмила Любарская}

Да !!! Представляешь ,к нам приходила вся коммуналка смотреть новости !!! Помню
прямой эфир когда убили Кенеди !!! Может и не прямой ))
\end{itemize} % }

\iusr{Кретов Андрей}
С Праздником! Ну, у нас был "Рекорд", более поздний. Служил долго!

\begin{itemize} % {
\iusr{Олег Сошевский}
\textbf{Кретов Андрей} , "Рекорд 12" долго прослужил у нас ,потом пришла на смену Ладога 205 с большим экраном
\end{itemize} % }

\iusr{Владимир Романюк}
Спасибо, все правильно, все так и было, но ностальгия по тем временам иногда наплывает и даже бурными волнами воспоминаний.

\iusr{Maria Rasin}

В вашем рассказе сквозит доброта ! Я тоже помню появление телевизора очень
хорошо ,а в линзу потом добавляли краску ,чтобы был цветной !!!

\begin{itemize} % {
\iusr{Maria Solodko}
\textbf{Maria Rasin} зелёнку

\iusr{Maria Rasin}
\textbf{Maria Solodko} нет ! Что-то голубое

\iusr{Maria Solodko}
\textbf{Maria Rasin} значит другой индикатор!
\end{itemize} % }

\iusr{Ludmila Vorobiova}

Я помню как у нас появился телевизор с линзой. Была одна программа, которая
появлялась где-то после обеда Приехала бабушка из деревни. И я с ней, старый и
малый ставили перед телевизором стулья, садились и долго смотрели таблицу
настройки, это было чудо.

\iusr{Эльвира Костарева}
У моєї подруги був такий. Ми дивились фільм "На хуторі поблизу Диканьки"

\iusr{Александр Муратов}

Помню такую диковинку. Первый раз увидел его и смотрел пердачу в полквой
библиотеке 15 стр. полка Гв. Таманской дивизии, когда служил там зимой 1948-49
г. Тогда и линзы еще незнали. А в 1950-х годах в Киеве на Крещанике были очерди
у магазина "Радио". Счастлвчикам удавалось купить.

\iusr{Зина Козинец}
Первые трансляции передач по КВНу шли несколько дней в неделю...

\iusr{Boris Lyudmirsky}
КВН: Купил - Включил - Не работает

\begin{itemize} % {
\iusr{Maksim Pestun}
нужно с ним лет на 60 назад вернуться...
\end{itemize} % }

\iusr{Света Света}
У меня есть тоже такой телевизор. Оставила на память. И он рабочий.

\iusr{Rimma Turovskaya}
Интересная жизнь у вас была, Максим! Столько всего повидали. @igg{fbicon.grin} 

\iusr{Олег Єгоров}
"Для всех жителей усср" вызвало гомерический смех.

\iusr{Bob Voronkov}
А якщо зрівняти з теле того часу на Заході?...  @igg{fbicon.face.upside.down}  як і решта для людей, а не для війни ...

\iusr{Pavel Nibyt}
У нас был такой же.

\iusr{Тамара Мельничук}
у нас був такий десь у 60 роках в лінзі була дистильована вода яка збільшувала екран

\iusr{Лариса Олейникова}

Спасибо за прекрасный пост-воспоминание! С праздником!

\iusr{Valentina Urban}

У нас был точно такой телевизор с линзой в конце 50-х, линза двигалась в
направлении вперед-назад для получения увеличения и четкости изображения. При
любой важной трансляции многие соседи нашего парадного 5-ти этажки в которой
мы жили приходили в нашу однокомнатную квартиру на Печерске чтобы стать
очевидцами увиденного и услышанного. В июне 1962 годы мы привезли из Москвы
телевизор "Заря", который имел приличного размера экран без линзы и
серебристо-серая коробка была металической, вместо деревянной коричневой.

\iusr{Веле Штылвелд}

Такой КВН был у бабушки Евы и дедушки Наума на ул. КрасноармейскаЯ, 6 в г. Киеве.

\iusr{Vladimir Zhuravsky}

Помню восторги ребятишек возле окон телеателье на углу Ярославской и Волошской,
там были выставлены линзами к улицепераые квн-49!!!

\iusr{Виктория Зайцева}

Да, сериалы в то время были знаковыми. Я просто обожала "Четыре танкиста и
собака", первый раз смотрела в трех-летнем возрасте, болея за собачку. А вот к
"Семнадцать мгновений весны" у меня особое отношение. Этот сериал спас нам
жизнь. Мы смотрели сериал, и как раз в подвале нашей хрущевки начался пожар -
горели веники, тряпки, ящики, которые там хранили дворники. Дыма было столько,
что выйти из квартиры уже было нельзя. Мы заметили, что полно дыма в коридоре,
когда во всем доме погас свет - провода начали гореть. Это нас и спасло. Никто
не спал, все смотрели кино. Зато потом было "кино и немцы" когда приехали
пожарные, а в машине не оказалось воды, машину ударило током, потом приехало 5
машин, а весь район сидел без света.

\iusr{Алла Донская}

Я помню у нас такой телевизор был с линзой! Это он раритетный, или я
антика)) @igg{fbicon.monkey.see.no.evil}  @igg{fbicon.laugh.rolling.floor} 

\begin{itemize} % {
\iusr{Maksim Pestun}
Мы все уже...

\iusr{Алла Донская}
\textbf{Maksim Pestun} ой, молчи... @igg{fbicon.face.grinning.smiling.eyes} 
\end{itemize} % }

\iusr{Eugen V Pryhod'Co}

ответ из газетного киоска: правды нет советская россия продана остался только труд. за две копейки

\iusr{Valentina Kharenko}

Мы - ровесники. Я танцевала во Дворце пионеров, и нас иногда приглашали на
телевидение. Несколько мальчиков и девочек танцевали африканский танец - эскизы
к костюмом и сами костюмы заказывали в мастерских оперного театра. Нас
занримировади, надели парики. Первый раз после выступления смывали грим прямо
на телевидении. Я тогда заболела - холодная вода, зима. После второго
выступления бабушка вела меня по Крещатику, мы ехали в троллейбусе до Большой
Житомирской. Зрелище было странное - 1957 год, чёрная девочка идёт по
заснеженному городу в сопровождении белой бабушки. Бабушка ещё завела меня в
гастроном, чтобы показать знакомой кассирше Зиночке. Воспоминания на всю жизнь.

\iusr{Tatiana Thoene}
Помню это чудо техники!

\iusr{Helen Ferrum}
Было  @igg{fbicon.smile} 
У нас был первый "КВН" на улице .
Весь двор ходил телик смотреть !

\iusr{Erena Sevohena}
Мы у соседки смотрели по такому тв фестиваль из Сопота

\iusr{Электрозвук Плюс}

Когда транслировали балет, дед пытался снизу заглянуть под телевизор, на
балерин. КВН49, 60-е годы, был на ходу

\begin{itemize} % {
\iusr{Алла Вайнерман}
\textbf{Электрозвук Плюс} Сначала был КВН 47
\end{itemize} % }

% -------------------------------------
\ii{fbauth.barabanschikova_zoja.kiev.ukraina}
% -------------------------------------

\ifcmt
  ig https://scontent-mxp1-1.xx.fbcdn.net/v/t1.6435-9/252713304_4553784351348365_5930511775946284276_n.jpg?_nc_cat=101&ccb=1-5&_nc_sid=dbeb18&_nc_ohc=x7hCu5q1vgAAX8RXCF2&_nc_ht=scontent-mxp1-1.xx&oh=85e51b7812d06b2354d4280a444ce908&oe=61ABA2E1
  @width 0.4
\fi

\iusr{Polina Feldman}
это история но это было

\iusr{Дмитрий Даен}

Был такой телевизор КВН и у нас дома на Дмитриевской 8. Больше всего меня
интересовала вода в линзе. Она забавно покачивалась, стоило только слегка
толкнуть стол, на котором телевизор стоял.

\iusr{Владимир Девятияров}
Я помню,но линза была квадратная

\iusr{Виктор Михайлюк}

У батиного дядьки был такой, только с квадратной линзой. Он где то добыл плёнку
со всеми цветами радуги и закрепил её возле линзы, получился такой себе цветной
телевизор!!! Это примерно 1965-66 год!

\begin{itemize} % {
\iusr{зоя Барабанщикова}
\textbf{Виктор Михайлюк} 

в это время такие телевизоры уже уходили в небытие - появились «Ленинград»,
Темп-2. А такой (кажется, КВ) - это 50- 56гг. В 51-52 бегала к подружке на
такой телевизор.

\begin{itemize} % {
\iusr{Виктор Михайлюк}
\textbf{зоя Барабанщикова} Я написал о том что видел в свои пять лет.

\iusr{зоя Барабанщикова}
\textbf{Виктор Михайлюк} да, конечно, у кого был куплен такой телевизор, продолжали пользоваться. Тогда ещё так быстро телевизоры не меняли.
\end{itemize} % }

\end{itemize} % }

\iusr{Светлана Митюненко}
Написано чудесно Вся жизнь воспоминания встают перед глазами Огромное спасибо Хотелось бы ещё читать такие рассказы

\iusr{Tamara Khmara}
И я в конце пятидесятых на старой квартире Киева на Пирогова у соседей смотрела КВН.

\iusr{Lyudmila Zhuchkova}

В 70-х уже был у нас телевизор без линзы, "Берёзка". Уже экран был нормальный.
Пульта не было, вставали и переключали. Была у нас соседка, психолог, кандидат
наук. Она принципиально не покупала ТВ. Но когда были интересные сериалы ходила
к нам смотреть. А когда появился первый цветной ТВ, она сразу купила и нас
приглашала смотреть и всё приговаривала, что цветной, это совсем другое
качество от чёрно белого... Да, есть хорошие воспоминания. Благодарю Вас за
рассказ!!!

\iusr{Людмила Старовойтенко}

Ах воспоминания спасибо за рассказ и прекрасные коментарии

\iusr{Світлана Куликова}

\ifcmt
  ig https://scontent-frx5-2.xx.fbcdn.net/v/t39.1997-6/s168x128/16781161_1341101952618574_7704631035023065088_n.png?_nc_cat=1&ccb=1-5&_nc_sid=ac3552&_nc_ohc=9dQ3UxEXGQ4AX-RiagO&_nc_ht=scontent-frx5-2.xx&oh=8b8d1a86f95fc3aec0c3deb46101e9fd&oe=618C836B
  @width 0.1
\fi

\iusr{Генадий Пинский}

У нас был такой телевизор КВН 49 с линзой. В эту линзу мы наливали глицерин.

\begin{itemize} % {
\iusr{Natalia Grebelnik}
\textbf{Генадий Пинский} І у нас був КВН з лінзою
\end{itemize} % }

\iusr{Наталия Педос}

Ой, как много знакомого! У нас в 1960-м году появился телевизор КВН с
прямоугольной линзой. Мама заливала в нее подсиненную водичку. Телевизор был
один на всю нашу длиннющую улицу и наличие этого чуда техники стало настоящим
бедствием для нашей семьи. Дело в том, что в тот год родился мой младший брат,
а один из самых активных зрителей, живший на соседней улице, имел привычку
приходить где-то за час до начала передачи. Мама сильно нервничала и говорила:
"Так рано приходить, що й дитину не можу нормально покупати!"

\iusr{Наталья Соколовская}
У нас, у бабушки с дедушкой был такой. Я, малышкой будучи, даже англ. язык
учила по нему. Передача для детей была. Это 1960-1967 годы.

\iusr{Люба Кириленко}
У нас появился КВН в 1957 году. Фсе соседи по вечерам приходили как в кинотеатр.

\iusr{Люба Кириленко}
Очрень познавательно! Спасибо! Чем старше становишся, тем больше нового хочеться узнать. Но боюмсь, что уже не успею, а очень жаль!

\iusr{Светлана Лапчик}
У бабушки и дедушки был такой.

\iusr{Светлана Лысань}

В Олівце музей организовали, небольшой, в нем есть и телеки и камеры и
оборудование Можно посетить его. Есть экскурсии

%1.png

\iusr{JUrij PAnin}

В год моего рождения (1954) сослуживцы моих родителей сделали такой подарок. Ни
разу не ломался до 1975г. Только один раз поменяли стабилизатор. А ещё в линзу
вставляли цветную плёнку. Красиво было...

\iusr{Микола Кокоша}
КВН.

\iusr{Liudmyla Liudmyla}

Насчет того, что в каждом доме? У нас был ТВ один на дом из 12 квартир. Вначале
трансляция была только по четвергам, это потом уже стали появляться различные
программы.. Благодаря телевидению, я ребенок познакомилась с классическим
операции, узнала Татьяну Шмыгу, известных художников. Было очень много
познавательных программ.

\begin{itemize} % {
\iusr{Liudmyla Liudmyla}
Операми, композиторами.

\iusr{зоя Барабанщикова}
\textbf{Liudmyla Liudmyla} Да, тогда телевидение, кроме идеологии и агиток, о которых сейчас многие говорят ( даже те, кто этого «не нюхал»), несло в массы знание, культуру, просвещение, а дикторы были образцом грамотности и образованности.
\end{itemize} % }

\iusr{Тала З.}

Да! Помню! Такой КВН был у моих бабушки и дедушки, только линза была другой
формы))

\iusr{Aleksander Kudielkin}

А у нас помню это чудо в комнате. Пол-дома приходило на футбол. Что-то там
мелькало на экране, а мяча ваапще не видно было. Помню яркость тяжело было
удержать.

\ifcmt
  ig https://scontent-mxp1-1.xx.fbcdn.net/v/t1.6435-9/254821206_10209693974943631_4982646463547166060_n.jpg?_nc_cat=105&ccb=1-5&_nc_sid=dbeb18&_nc_ohc=vsJ6b_s0szMAX9Tx1gS&_nc_ht=scontent-mxp1-1.xx&oh=32c396bea0fa6370939cc3fc7a77da93&oe=61AEEC3E
  @width 0.4
\fi

\iusr{Alexander Gorochovski}
Замечательный рассказ, спасибо Вам.

\iusr{Anatoliy Pilipenko}
Хорошие воспоминания!!!

\iusr{Valeri Ko}
Пацаны всего дома приходили на фильм "Чапаев" в 1954г. По диагонали без линзы аж 12 см

\begin{itemize} % {
\iusr{Maksim Pestun}
Папа жил на Чеховском
\end{itemize} % }

\iusr{Владимир Попов}
у родителй был телевизор с линзой... 1960г.... давно это было..

\iusr{Николай Судакевич}
Был такой аппарат и у моего родича. Смотрели матчи чемпионата мира и Европы по хоккею целой своей командой.

\iusr{Николай Беляев}
Похоже, что спирт из линзы слили...)))

\iusr{Тетяна Гавриш}

Дуже пам'ятаю такий телевізор. Всі сусіди, родичі збирались ввечері у т.Оксани,
двоюрідної сестри мого батька, для перегляду всього, що показували. Це було
цікаво для всіх, і для дорослих, і для дітей. Один випадок запам'ятався на все
життя. Як завжди, ввечері всі зібрались для перегляду. Транслювався фільм про
Пушкіна. В момент епізоду дуелі, коли Дантес вистрілив, я не сримала емоцій,
і по обличчю покотились сльози. Крім мене ніхто не плакав, я була ще зовсім
мала. Хтось засміявся над моїми емоціями ( не від великого розуму, як я тепер
розумію), і я стала виправдовуватись, що в око порошинка попала, що сльози від
цього потекли...

Так, до цього часу, якщо я плачу, то вже старенька т. Оксана з доброю посмішкою
питає, чи не порошинка попала в око?))

А потім, того ж 1959 року, батько поїхав в Гомель і з Білорусії привіз
телевізор Неман, у якого екран був значно більший за КВНівський і дивились ми
його без лінзи ))

\iusr{Ирина Архипович}

Чудесный рассказ!!  @igg{fbicon.thumb.up.yellow}  @igg{fbicon.hand.ok}
@igg{fbicon.face.happy.two.hands} 

\end{itemize} % }
