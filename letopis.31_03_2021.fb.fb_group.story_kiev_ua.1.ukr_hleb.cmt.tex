% vim: keymap=russian-jcukenwin
%%beginhead 
 
%%file 31_03_2021.fb.fb_group.story_kiev_ua.1.ukr_hleb.cmt
%%parent 31_03_2021.fb.fb_group.story_kiev_ua.1.ukr_hleb
 
%%url 
 
%%author_id 
%%date 
 
%%tags 
%%title 
 
%%endhead 
\subsubsection{Коментарі}

\begin{itemize} % {
\iusr{Полина Полякова}
Відразу відчула смак \enquote{Українського} хліба. Аж, слина потекла. Обожнюю його.

\iusr{Наталия Давиденко}

Хлебушек вкусный, жаль, что у нас в стране так мало маленьких частных пекарен
@igg{fbicon.exclamation.mark.double}

\iusr{Ігор Крижановський}
ПІДУ... КРУМОЧКУ ВІДЛОМЛЮ..... ДЯКУЮ)

\iusr{Sara Shpilman}
Прямо до слёз...

\ifcmt
  ig https://scontent-frx5-1.xx.fbcdn.net/v/t1.6435-9/167551833_283167820060903_2700705601996121034_n.jpg?_nc_cat=111&ccb=1-5&_nc_sid=dbeb18&_nc_ohc=kw_q6aor2t8AX_x1lMq&_nc_ht=scontent-frx5-1.xx&oh=00_AT9G5uqmQoS5F98vqsIgJpgpR5OmdqfeWY0JKva5Rp-Ivw&oe=61DEC75D
  @width 0.3
\fi

\iusr{Елена Сидоренко}

Умеете Вы написать так, что наяву представляешь себя ребёнком, грызущим корочку
свежего хлеба. @igg{fbicon.face.savoring.food} Завод этот работает до сих пор,
и на Подоле пахнет хлебом. @igg{fbicon.heart.beating} 

\iusr{Олег Самченко}

\ifcmt
  ig https://scontent-frx5-1.xx.fbcdn.net/v/t39.1997-6/p240x240/47294511_540271073156464_2498170897404264448_n.png?_nc_cat=107&ccb=1-5&_nc_sid=0572db&_nc_ohc=iHcKZzNY2mkAX_JWvcZ&_nc_ht=scontent-frx5-1.xx&oh=00_AT90iXMN0ohtEU2KyO7w-G6O5sdG82UqRsszON_CP3mXtw&oe=61BD0699
  @width 0.2
\fi

\iusr{Ольга Яворовская}
Дякую за смачний екскурс!
Смачн!шого не !ла !

\ifcmt
  ig https://scontent-frx5-1.xx.fbcdn.net/v/t39.1997-6/p240x240/47294511_540271073156464_2498170897404264448_n.png?_nc_cat=107&ccb=1-5&_nc_sid=0572db&_nc_ohc=iHcKZzNY2mkAX_JWvcZ&_nc_ht=scontent-frx5-1.xx&oh=00_AT90iXMN0ohtEU2KyO7w-G6O5sdG82UqRsszON_CP3mXtw&oe=61BD0699
  @width 0.2
\fi

\iusr{Lola Shpilskaya}
Я тоже всегда прошу привезти из Киева хлеб

\iusr{Любов Огородня}

Дуже смачно, Дякую Петюнчику! Але, чому на Кожум'яки за хлібчиком?
Гончарна-Воздвиженська. На Комуж'яках був продовольчий магазин @igg{fbicon.wink} . А щодо
хлібзаводу на Подолі, між двома вулицями - Щекавицькій і Турівській. Пам'ятаю,
у 1987 році, ВО \enquote{Київоблгаз}, який розташовувався на Гончарній, 3 (я там
працювала), переїхав на Турівську, якраз навпроти хлібзаводу, поруч з магазином
Галантерея. Аромат навкруги неповторний. Наше керівництво домовилося з
хлібзаводом про харчування працівників Облгазу в їдальні заводу. Ми на обід
ходили на територію заводу. Хліб там був на столах завжди свіжий, гарячий і
безкоштовно.

\iusr{Петр Кузьменко}

Бачиш, Любаня, ти пам'ятаєш ці неповторні миті нашого дитинства краще ніж я.
Старішаю... @igg{fbicon.laugh.rolling.floor} 

\iusr{Semyon Belenkiy}

А хлебозавод, который вы упоминаете, это хлебокомбинат \#2, специализировавшийся
на изготовлении украинского хлеба. Где то до 80-х, по-моему, это был хлебозавод
\#2, а затем произошло объединение с хлебозаводом \#7 ( ул. Кирилловская 65,
бывшая Фрунзе), который производил продукцию из пшеничной муки, бублики,
баранки, кондитерские изделия и в том числе Киевский, фирменную Ромашку и т. д.
и получился хлебокомбинат.

В молодости и сам работал на 7-ом хлебозаводе. Такая история, а за рассказ
большое спасибо!

\iusr{вита шаповал}

Дякую. Дуже смачно написано. Але зараз цей хліб не смачний, вони там хімічать)
Мої сільські родичі і їх сусіди завжди чекали приїзду городських) бо смачного
хліба привезуть. Тітка вийшла заміж за француза, приїхала в 1980 р. до Києва
погостювати. Коли відїждала до Франції взяла декілька буханців українського
хліба та пахучу соняшникову олію

\iusr{Раиса Карчевская}
Да пост прекрасный.

Написано так вкусно, что сразу почувствовала запах и вкус Украинского хлеба. Я
его обожаю и всегда, когда едешь за границу все очень ждут наш прекрасный
вкуснющий хлеб.

\iusr{Константин Панасенко}

Тоді це був дійсно ХЛІБ! А зараз - це порнографія, а не хліб! Допоки теплий, ще
нічого, а тільки став холодним їсти не можливо! Є з чим зрівнять. Хлібзавод
номер 6 випікав в 60х-70х такий хліб, що з магазина до дому доносив тільки
половинку і ту без країв!

\begin{itemize} % {
\iusr{Ірина Оснач}

Колись той хліб був мені смачніший за будь-яке печиво чи торт. Потім його
замінили на \enquote{Дарницький}, то вже був інший смак, на жаль. І паляницю тепер, на
жаль, не печуть, її хрумкий гребінець з"їдався одразу, за що мене постійно
сварили, бо по магазинах з раннього дитинства в сім"ї ходила я.

\iusr{Галина Компанченко}
\textbf{Константин Панасенко} Та яким би він не став- не можна так про хліб..

\iusr{Игорь Бала}
\textbf{Галина Компанченко} не можно выпекать такое \enquote{Г} как сейчас
\end{itemize} % }

\iusr{Ольга Григор'єва}
Сьогодні купила половинку  @igg{fbicon.wink}  Цікаво куштувати і нові смаки, але "Український" купляю завжди.

\iusr{George Vishnevsky}

Нажал, сьогоднішній Український відрізняється від того, що був колись. Немає
тієї хрусткої скоринки, смак схожий , але не той. І головне - інгрідієнти інші,
бо, хліб довго не сохне, а у поліетилені пліснявіє.  Теж саме з Батоном, тільки
той з другого дня починає не сохнути, а сипатися крихтками.  Взагалі, ХЛІБ в
Києві став гірше...

\iusr{Екатерина маковецкая}

самый вкусный всегда был с подольского хлебзавода, бабушка жила рядом на
Щекавицкой, аромат украинского хлеба там был всегда)))) аж слюнки текли!
вкуснее нигде не пекли, т.к. печь украинский сложно, он кислый, плохо подходит
и пропекается, для этого нужны специальные кирпичные тонельные печи, они
сохранились на Подольском хлебзаводе

\iusr{Marina Gromova}

Шматочок свіжого українського, змастити соняшниковою, не рафінованою олією і
посипати сіллю, смакота з дитинства  @igg{fbicon.wink} 

\iusr{Наталья Ершова}
\textbf{Marina Gromova} и сейчас так вкусно!!!

\iusr{Наталия Вигерина}
і паляниць вже не печуть @igg{fbicon.cry} 

\begin{itemize} % {
\iusr{George Vishnevsky}
\textbf{Наталия Вигерина}

\ifcmt
  ig https://scontent-frt3-1.xx.fbcdn.net/v/t1.6435-9/166882013_4075385425834399_7599132278821478542_n.jpg?_nc_cat=102&ccb=1-5&_nc_sid=dbeb18&_nc_ohc=b7cCot6BJB4AX-1lgon&_nc_ht=scontent-frt3-1.xx&oh=00_AT8Xh1ePAEinJJyNoasJuTZLPlkhZpu3BLYn5Xt6ftrKsg&oe=61DE5358
  @width 0.4
\fi

\iusr{Игорь Васильевич Переводов}
\textbf{George Vishnevsky} І де?

\end{itemize} % }

\iusr{Лариса Жабкіна}
Запах хлеба и хлебного магазина на Константиновской я помню до сих пор.

\iusr{Наталья Ершова}
Сейчас хлеб украинский совсем другой, тот украинский делали честно и с душой!!! А сейчас его и покупать нет желания!!!

\iusr{Вера Долженко}
\textbf{Наталья Ершова} Вот именно, я и покупать перестала. А был хлеб вкуснейший.

\iusr{Александр Полищук}

В моїй родині тільки одне слово вимовляється з неправильним наголосом - хліб
укрАїнський, а все інше украЇнське. Чомусь здається, що, якщо ти назвеш хліб
украЇнським, він не буде таким смачним.

\begin{itemize} % {
\iusr{Volodymyr Nekrasov}
\textbf{Александр Полищук} тут нема нічого неправильного. УкрАїнський може бути ще й похідним від «укрАяти», себто відрізати. А ще є містечко УкрАїнка, що на Київщині. Теж всі називають з наголосом на А. Зрештою і назва Україна є похідною від КрАй, крАяти.
\end{itemize} % }

\iusr{Ирина Шевцова}
А ещё арнаутка серая какая была ароматная!!!

\iusr{Olga Zyrina}
Я дуже любила свіжу четвертинку українського з пломбіром після школи (якщо ті
цінні 20 копійок на морозиво були).  @igg{fbicon.smile} 

\iusr{Виктория Довгая}

Дааа, везде, куда бы мы не ездили вне Киева, нас просили привезти буханки
Украинского хлеба. Мама ходила и закупала буханки. А я очень любила бублики с
маком и булочки за 3 копейки. Купишь бублик, съешь - вкуснотища..

\iusr{Маргарита Марченко}
Дуже гарний і неповторний був смак!



\end{itemize} % }
