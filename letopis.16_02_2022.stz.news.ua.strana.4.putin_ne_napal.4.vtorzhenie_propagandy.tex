% vim: keymap=russian-jcukenwin
%%beginhead 
 
%%file 16_02_2022.stz.news.ua.strana.4.putin_ne_napal.4.vtorzhenie_propagandy
%%parent 16_02_2022.stz.news.ua.strana.4.putin_ne_napal
 
%%url 
 
%%author_id 
%%date 
 
%%tags 
%%title 
 
%%endhead 

\subsubsection{\enquote{Вторжение топорной пропаганды}}

Еще одна тема, которую обсуждают в связи со \enquote{вторжением} - это День единения,
который сегодня стал главным официальным событием дня. Его транслировали по
всем телеканалам, власти призвали всех надеть национальную символику и хором
петь гимн.

На фоне всего этого пиарился Зеленский, а страницы министерств и ведомств были
заполнены этой темой. 

Хотя праздник этот не совсем понятен. Никаких исторических предпосылок он не
имеет, к тому же дублирует уже существующий День соборности. 

То есть это праздник-однодневка, чтобы Зеленский мог обозначиться на фоне
\enquote{вторжения} как объединитель украинцев. 

В сети сравнили происходящее с \enquote{Лебединым озером}, которое тоже шло по всем
каналам в дни ГКЧП. Назначенный властями День единства называют топорной
пропагандой в комсомольском стиле.

\enquote{По всем каналам, национальным и провластным, в том числе 1+1, Украина, Интер,
ICTV и другие, кроме санкционных, - сбывшаяся мечта Зеленского - транслируют
марафон \enquote{єднання}. Зеленский в камуфляже. Фейковая война, фейковая свобода
слова, фейковый президент} - написала заместитель главного редактора \enquote{Страны}
Светлана Крюкова.

\ifcmt
  ig https://strana.news/img/forall/u/0/34/%D0%A1%D0%BD%D0%B8%D0%BC%D0%BE%D0%BA(1537).JPG
  @wrap center
  @width 0.8
\fi

"Зато сегодня наблюдается вторжение топорной пропаганды Зеленского на всех
каналах. Крутят марафон в вышиванках и с лентами. Выглядит максимально жлобски.

Я все ещё жду выкладки сала в форме Тризуба на площадях", - написала
спецкорреспондент \enquote{Страны}, ведущая блога \enquote{Ясно.Понятно} Олеся Медведева.

\ifcmt
  ig https://strana.news/img/forall/u/0/34/%D0%A1%D0%BD%D0%B8%D0%BC%D0%BE%D0%BA(1536).JPG
  @wrap center
  @width 0.8
\fi

Нардеп Макс Бужанский ставит под сомнение смысл, заложенный властями в
праздник. 

"День Единства, это хорошо.

В стране, где одна половина граждан считает себя победителями на Майдане над
другой половиной.

Где одни могут работать на родном языке, другие нет.

Где для одних есть школы на родном языке, а для других их не существует.

Да, день Единства, это очень хорошо.

С остальными 364 днями не очень" - написал депутат от "Слуги народа" Макс
Бужанский.

\ifcmt
  ig https://strana.news/img/forall/u/0/34/%D0%A1%D0%BD%D0%B8%D0%BC%D0%BE%D0%BA(1526).JPG
  @wrap center
  @width 0.6
\fi

\enquote{День единства – это не про единство, не про диалог между разными регионами и
политическими группами, нет. Это про то, что Зеленский и его чиновники на всех
каналах страны рассказывают, какие они молодцы и как они в едином порыве
остановили Путина. Не перепутайте}, - написал журналист Вячеслав Чечило.

\ifcmt
  ig https://strana.news/img/forall/u/0/34/%D0%A1%D0%BD%D0%B8%D0%BC%D0%BE%D0%BA(1527).JPG
  @wrap center
  @width 0.5
\fi

\enquote{Включаю телевизор - по всем каналам \enquote{Лебединое озеро} ...} - написал
экс-депутат Александр Ржавский с намеком на \enquote{марафон единения}, который
сегодня показывали везде.

\ifcmt
  ig https://strana.news/img/forall/u/0/34/%D0%A1%D0%BD%D0%B8%D0%BC%D0%BE%D0%BA(1525).JPG
  @wrap center
  @width 0.6
\fi

"Опять все проспали... На часах 8.25. Тишина. Очередная "точная дата"
наступления провалилась, но никто из этих оракулов-предсказателей не будет
нести ответственности. Назначат новую. Позже. Когда снег растает. Хотя нет,
наоборот, когда грязь замёрзнет.

Зато, мы готовы были встретить врага во всеоружии. Во всех школах детей сегодня
сказали одеть в вышиванки и петь гимн. Вместо урока. Веселуха такая
патриотическая. Особенно это мощно на удаленке. По Zoom.

Новая игрушка для Зе - День единения!!! Хотя, закрывая без суда и следствия
канал НАШ, Президент сказал, что ни один по-настоящему наш украинский канал не
пострадал, вычеркнув несколько миллионов граждан из списка наших. Поделил
Украину на наших и не наших. А сегодня про единение рассказывает. И детей
использует. "Брежнев" наш 43-летний.

Я-то думал, что если война, то Верховный какие то иные Указы должен подписать.
Ан нет, стрички, флаги, гимн и…. еднатися будем.

Ну и правильно. Есть, вокруг чего. Например, наша курятина стала стоить на 20%
дороже, чем в Польше, Румынии, Венгрии. И муку из нашей пшеницы покупаем у
турков, шоб хлеб сделать. Да много чего есть, шо всех объединяет. 

Тупость, вороватость, профнепригодность и безнаказанность власти. Вот, против
чего я хотел бы всех объединить….

Так что, Еднаемось!!!! Кстати, а День Соборности шо, уже отменили. Доброе
мирное утро, страна!!! Быть Добру!!!" - написал телеведущий Дмитрий Спивак.

\ifcmt
  ig https://strana.news/img/forall/u/0/34/%D0%A1%D0%BD%D0%B8%D0%BC%D0%BE%D0%BA(1528).JPG
  @wrap center
  @width 0.5
\fi

"Доброе утро! Путин не вторгся. Зато Зеленский вторгся во все телеканалы.
Заставил их показывать одну и ту же картинку - телемарафон государственного
канала \enquote{Дом}.

Показывают даже частные и даже каналы Порошенко, которые обычно игнорируют
просьбы власти. Чиновники второй час рассказывают как они любят наш флаг и
сколько всего полезного сделали для нас.

Прям камбек в СССР - одна картинка на всех каналах и начальники делают вид, что
тебя любят.

Зеленский придумал странный праздник в честь не вторжения - День единения,
который по смыслу повторяет день Соборности. Но как его праздновать - не
придумал" - написал директор информагентства \enquote{Украинские новости} Денис
Иванеско.

\ifcmt
  ig https://strana.news/img/forall/u/0/34/%D0%A1%D0%BD%D0%B8%D0%BC%D0%BE%D0%BA(1529).JPG
  @wrap center
  @width 0.6
\fi

\enquote{Не понятна суть сегодняшнего государственного праздника, который два дня назад
ввел Зеленский. Какое оно имеет значение? С кем сегодня объединяется Украина?
Со всем демократическим миром? Тогда, наверное, и другие страны должны объявить
о таком дне}, - написала экс-депутат Виктория Пташник.

\ifcmt
  ig https://strana.news/img/forall/u/0/34/%D0%A1%D0%BD%D0%B8%D0%BC%D0%BE%D0%BA(1530).JPG
  @wrap center
  @width 0.6
\fi

"Телемарафон единства превратился в скучное бубнение чиновников разного уровня
о "своих достижениях"", - выразил мнение экс-министр Павел Розенко.

\ifcmt
  ig https://strana.news/img/forall/u/0/34/%D0%A1%D0%BD%D0%B8%D0%BC%D0%BE%D0%BA(1531).JPG
  @wrap center
  @width 0.6
\fi

\enquote{С днём очередной комсомольщины, ребят. Смотрю я на этих патриотов и снова
четко понимаю - что-то спи@дили} - написал телеведущий Макс Назаров.
