% vim: keymap=russian-jcukenwin
%%beginhead 
 
%%file 10_11_2020.news.ua.strana.4.karabah_dogovor
%%parent 10_11_2020
 
%%url https://strana.ua/news/300067-rossijskie-mirotvortsy-v-karabakhe-kak-zakonchilas-vojna-v-rehione.html
%%author 
%%tags 
%%title 
 
%%endhead 

\subsection{Российские миротворцы, удар по Пашиняну, уроки для Украины. Что означают новые соглашения по Карабаху}
\label{sec:10_11_2020.news.ua.strana.4.karabah_dogovor}

\Purl{https://strana.ua/news/300067-rossijskie-mirotvortsy-v-karabakhe-kak-zakonchilas-vojna-v-rehione.html}
\Pauthor{Венк, Виктория}
\index{Карабах!Война!Cоглашение (Москва, Баку, Ереван, 2020)}

\ifcmt
img_begin 
	url https://strana.ua/img/article/3000/67_main.jpeg
	caption Российские миротворцы летят в Карабах. Фото Минобороны РФ 
	width 0.7
img_end
\fi

Сегодня ночью Москва, Баку и Ереван подписали историческое соглашение по
Карабаху\footnote{\url{https://strana.ua/news/300038-o-chem-dohovorilis-po-karabakhu-aliev-pashinjan-i-putin.html}}.
Оно знаменует по большому счету конец войне.

Азербайджан получает территории, на которые претендовал, за Арменией остается
большая часть Карабаха, а под охрану его берут российские миротворцы. 

Что означают эти договоренности - разбиралась "Страна". 

\subsubsection{Два слагаемых мира в Карабахе}

Попытки установить перемирие и закончить новую войну в Карабахе (или хотя бы
приостановить ее) делались неоднократно.

Впервые Армения и Азербайджан договорились об остановке огня еще в октябре при
посредничестве Москвы. Но перемирие сорвалось почти сразу. В дальнейшем было
еще несколько подобных попыток. Одну из них промодерировал даже Госдепартамент
США. Но все было тщетно - бои продолжались.

Одновременно весьма прохладно к попыткам перестать стрелять относилась Турция -
главный патрон Азербайджана в этой войне. Уже это поддерживало веру Баку в то,
что лучше ковать железо, пока горячо - пока армяне отступают. 

Это была первая причина постоянных срывов перемирия. Азербайджану оно было не
нужно. А подписывал его Баку, чтобы показать свою конструктивную позицию и не
рассориться с Москвой. 

При этом было очевидно, что президент Алиев не остановится, пока не удержит
какой-то убедительной победы. Таковой, конечно, не могли быть захваты
второстепенных населенных пунктов наподобие Гадрута или Магадиса. Война
затевалась ради настоящего реванша. Как минимум - возвращения территорий,
отторгнутых Арменией за пределами Карабаха еще в 90-е годы. 

И нельзя сказать, чтобы Москва считала эти требования несправедливыми. Еще до
обострения Еревану предлагали отдать эти районы без войны, но демилитаризовать
их, чтобы обезопасить Карабах от нападения азербайджанской армии. Однако армяне
отказывались.

Ситуацию могло бы спасти контрнаступление Армении, но ее оборона вокруг
Карабаха пала. 

В общем, держать перемирие у Баку не было никакой необходимости. Однако все
изменилось на выходных после взятия важнейшей цитадели - города Шуши в десяти
километрах от столицы
Карабаха\footnote{\url{https://strana.ua/news/299764-kak-situatsija-v-karabakhe-pokhozha-na-donbass-i-smozhet-li-ukraina-zachistit-rehion.html}}.

Это ставило всю непризнанную республику под угрозу быстрой ликвидации: через
Шуши обеспечивалась связь с Арменией. 

По сути, это была стратегическая победа Азербайджана, после которой уже можно
было диктовать условия мира. 

Второй фактор, который провоцировал постоянные срывы перемирия - отсутствие в
регионе третьей силы (или сил), которые бы развели враждующие стороны по углам.
То есть миротворцев. И вот сегодня ночью эта переменная в уравнении появилась. 

\subsubsection{О чем договорились Путин, Алиев и Пашинян  }

"Страна" уже публиковала план из 10 пунктов\Furl{https://strana.ua/news/300038-o-chem-dohovorilis-po-karabakhu-aliev-pashinjan-i-putin.html}, который подписали два президента -
Азербайджана и России - и премьер Армении. Дадим их краткое описание: 

Огонь прекращается с 10 ноября, все стороны остаются на занимаемых позициях (то
есть Шуши армяне теряют).

Более того, армяне до 15 ноября возвращают Азербайджану Кельбаджарский район, а
до 20-го - Агдамский и часть Газахского района. До 1 декабря под контроль Баку
переходит Лачинский район. Все это территории, захваченные армянами в 90-х
годах.  

Впрочем, в Лачинском районе под армянами остается "дорога жизни" - Лачинский
коридор, соединяющий Карабах с Арменией. А, чтобы азербайджанцы его не
захватили, на всем протяжении коридора вводятся российские миротворцы.

Они также заходят на всю линию фронта между армянами и азербаджанцами.
Миротворцев будет 1960, они вооружены стрелковым оружием, бронетранспортерами,
спецтехникой.

Заходить на территорию Азербайджана россияне будут параллельно выходу армян из
удерживаемых ими селений. 

Миротворцы вводятся на пять лет с автоматическим продлением этого срока.
Вывести их можно, если хотя бы одна из сторон заявит о разрыве соглашений. Но
сделать это можно только за полгода до требуемой даты вывода.

То есть явный предохранитель для Карабаха: если Баку решит захватить
непризнанную республику и убрать перед этим россиян, то у армян будет полгода
на подготовку к войне либо эвакуации. 

Но вернемся к Лачинскому коридору. Он проходит мимо Шуши, который теперь под
контролем Азербайджана. Баку гарантирует безопасность движения по нему. Но
стороны договорились о том, что будут перестраивать и перенаправлять эту дорогу
- видимо, с тем чтобы отвести ее подальше от Шуши и вывести таким образом
из-под угрозы захвата Азербайджаном.

Обновленный коридор после перестройки также берут под контроль россияне. 

Армения в свою очередь обязуется разблокировать транспортное сообщение между
Нахичеванью и Азербайджаном. Нахичевань - это автономия Азербайджана, которая
отделена от него территорией Армении. Регулировать транспортное сообщение в
этом районе будут пограничники ФСБ России. 

Суть соглашений прокомментировал президент России в официальном видеообращении\Furl{https://youtu.be/Har9oJSkMxo}.

То есть на этот раз все гораздо серьезнее и скорее всего план будет исполнен. 

Миротворцы уже начали заходить на территорию. Они прибывают на военных грузовых
"Илах". Видео публикуют российские СМИ\Furl{https://youtu.be/1FHSUJO_d2A}. 


AddThis Sharing Buttons
Share to Twitter
Twitter
Share to Facebook
Facebook
Share to Telegram
Telegram
Share to Viber
Viber
Российские миротворцы летят в Карабах. Фото Минобороны РФ
Российские миротворцы летят в Карабах. Фото Минобороны РФ

Сегодня ночью Москва, Баку и Ереван подписали историческое соглашение по Карабаху. Оно знаменует по большому счету конец войне.

Азербайджан получает территории, на которые претендовал, за Арменией остается большая часть Карабаха, а под охрану его берут российские миротворцы. 

Что означают эти договоренности - разбиралась "Страна". 
Два слагаемых мира в Карабахе

Попытки установить перемирие и закончить новую войну в Карабахе (или хотя бы приостановить ее) делались неоднократно.

Впервые Армения и Азербайджан договорились об остановке огня еще в октябре при посредничестве Москвы. Но перемирие сорвалось почти сразу. В дальнейшем было еще несколько подобных попыток. Одну из них промодерировал даже Госдепартамент США. Но все было тщетно - бои продолжались.

Одновременно весьма прохладно к попыткам перестать стрелять относилась Турция - главный патрон Азербайджана в этой войне. Уже это поддерживало веру Баку в то, что лучше ковать железо, пока горячо - пока армяне отступают. 

Это была первая причина постоянных срывов перемирия. Азербайджану оно было не нужно. А подписывал его Баку, чтобы показать свою конструктивную позицию и не рассориться с Москвой. 

При этом было очевидно, что президент Алиев не остановится, пока не удержит какой-то убедительной победы. Таковой, конечно, не могли быть захваты второстепенных населенных пунктов наподобие Гадрута или Магадиса. Война затевалась ради настоящего реванша. Как минимум - возвращения территорий, отторгнутых Арменией за пределами Карабаха еще в 90-е годы. 

И нельзя сказать, чтобы Москва считала эти требования несправедливыми. Еще до обострения Еревану предлагали отдать эти районы без войны, но демилитаризовать их, чтобы обезопасить Карабах от нападения азербайджанской армии. Однако армяне отказывались.

Ситуацию могло бы спасти контрнаступление Армении, но ее оборона вокруг Карабаха пала. 

В общем, держать перемирие у Баку не было никакой необходимости. Однако все изменилось на выходных после взятия важнейшей цитадели - города Шуши в десяти километрах от столицы Карабаха. Это ставило всю непризнанную республику под угрозу быстрой ликвидации: через Шуши обеспечивалась связь с Арменией. 

По сути, это была стратегическая победа Азербайджана, после которой уже можно было диктовать условия мира. 

Второй фактор, который провоцировал постоянные срывы перемирия - отсутствие в регионе третьей силы (или сил), которые бы развели враждующие стороны по углам. То есть миротворцев. И вот сегодня ночью эта переменная в уравнении появилась. 
О чем договорились Путин, Алиев и Пашинян  

"Страна" уже публиковала план из 10 пунктов, который подписали два президента - Азербайджана и России - и премьер Армении. Дадим их краткое описание: 

Огонь прекращается с 10 ноября, все стороны остаются на занимаемых позициях (то есть Шуши армяне теряют).

Более того, армяне до 15 ноября возвращают Азербайджану Кельбаджарский район, а до 20-го - Агдамский и часть Газахского района. До 1 декабря под контроль Баку переходит Лачинский район. Все это территории, захваченные армянами в 90-х годах.  

Впрочем, в Лачинском районе под армянами остается "дорога жизни" - Лачинский коридор, соединяющий Карабах с Арменией. А, чтобы азербайджанцы его не захватили, на всем протяжении коридора вводятся российские миротворцы.

Они также заходят на всю линию фронта между армянами и азербаджанцами. Миротворцев будет 1960, они вооружены стрелковым оружием, бронетранспортерами, спецтехникой.

Заходить на территорию Азербайджана россияне будут параллельно выходу армян из удерживаемых ими селений. 

Миротворцы вводятся на пять лет с автоматическим продлением этого срока. Вывести их можно, если хотя бы одна из сторон заявит о разрыве соглашений. Но сделать это можно только за полгода до требуемой даты вывода.

То есть явный предохранитель для Карабаха: если Баку решит захватить непризнанную республику и убрать перед этим россиян, то у армян будет полгода на подготовку к войне либо эвакуации. 

Но вернемся к Лачинскому коридору. Он проходит мимо Шуши, который теперь под контролем Азербайджана. Баку гарантирует безопасность движения по нему. Но стороны договорились о том, что будут перестраивать и перенаправлять эту дорогу - видимо, с тем чтобы отвести ее подальше от Шуши и вывести таким образом из-под угрозы захвата Азербайджаном.

Обновленный коридор после перестройки также берут под контроль россияне. 

Армения в свою очередь обязуется разблокировать транспортное сообщение между Нахичеванью и Азербайджаном. Нахичевань - это автономия Азербайджана, которая отделена от него территорией Армении. Регулировать транспортное сообщение в этом районе будут пограничники ФСБ России. 

Суть соглашений прокомментировал президент России в официальном видеообращении. То есть на этот раз все гораздо серьезнее и скорее всего план будет исполнен. 

Миротворцы уже начали заходить на территорию. Они прибывают на военных грузовых "Илах". Видео публикуют российские СМИ. 

\subsubsection{Чья победа?}

Документы не просто фиксируют приобретения Баку в ходе войны, но и возвращают
ему территории, захваченные ранее. Армения же остается собственно только с
Карабахом - в границах его автономии времен Советского Союза. Да и то не всей -
Шуша и весь юг Карабаха остаются под контролем Азербайджана.

На карте синим цветом помечены территории, контролируемые Азербайджаном прямо
сейчас, а красным - те, которые перейдут под Баку до декабря.

Зеленые зоны под контролем российских миротворцев, которые фактически взяли под
защиту собственно Нагорный Карабах.

\ifcmt
img_begin 
	url https://strana.ua/img/forall/u/0/92/%D0%BA%D0%B0%D1%80%D1%82%D0%B0(17).jpg
	caption 10th November agreement on Karabakh
	width 0.7
img_end
\fi

\ifcmt
img_begin 
	url https://strana.ua/img/forall/u/0/92/Location_Nagorno-Karabakh[1].png
	caption Территория Нагорно-Карабахская автономной области времен СССР
	width 0.7
img_end
\fi

В общем, Баку победил, но по сути - он просто вернул отторгнутые у него 30 лет
назад территории "большого Азербайджана". К чему, кстати, всегда призывала
Армению та же Россия. 

При этом очевидно, что Алиев мог взять и весь Карабах - вместе со
Степанакертом. Но не стал и пошел на соглашения. Причем такие, которые на этот
раз нарушить будет практически нельзя без прямого военного конфликта с Россией. 

Почему Азербайджан остановился за несколько шагов до победы? 

\begin{itemize}
\item \textbf{Во-первых}, взятие всех территорий было бы чревато большой
				кровью. Которую Алиеву теперь не нужно проливать. Он выполнил
				заявленную программу-минимум - возвращение всех районов, захваченных
				Карабахом, часть из них получив вообще без войны. Это красивая победа. 

\item \textbf{Во-вторых}, "окончательное решение карабахского вопроса" с
				захватом Степанакерта несет явную угрозу этнических чисток. А это могло
				бы спровоцировать два нежелательных сценария: уже полноценную войну с
				Арменией (чей официальный союзник - Россия) и вероятные санкции со
				стороны ЕС. 

\item \textbf{В-третьих}, сдача собственно Карабаха стала бы потерей лица для
				России как главного армянского союзника. Одно дело - быть в стороне,
				когда Азербайджан забирает своё, другое - смотреть, как зачищают давнюю
				армянскую автономию.
\end{itemize}

Видимо, Москва очертила перед Алиевым и Эрдоганом некие красные линии, после
которых она может начать военную помощь армянам. Или вообще вступить в боевые
действия. Что и было истолковано в пользу компромисса. 

То есть РФ - еще один условный победитель в этом конфликте. По его итогам
позиции Москвы как миротворца в регионе укрепились. Война, судя по всему,
окончена, при этом Россия смогла пройти меж двух огней и не ввязаться в драку
на какой-то одной стороне. Сохранив прежние отношения с тремя столицами - Баку,
Ереваном и Анкарой. 

При этом учитывая, что в Карабах зашли миротворцы, вероятность возобновления
конфликта становится минимальной. По крайней мере, на ближайшие годы. 

А вот главный проигравший - конечно, Армения. Где уже шатается кресло под
премьером Николом Пашиняном. Что многими в России воспринимается как еще одна
победа: Пашиняна там считают прозападным и не слишком лояльным политиком. 

\subsubsection{Мятежная ночь в Ереване}

Вчера Пашинян сделал заявление, в котором признал поражение в Карабахе. Он
заявил, что вооруженные силы страны не смогли бы продолжать сопротивление,
мобилизационного резерва не хватало, и лучшее, что можно было сделать, сделали. 

"Я принял трудное, чрезвычайно трудное решение для себя и для всех нас. Я
подписал заявление с президентами России и Азербайджана о прекращении
карабахской войны в 01:00. Текст уже опубликованного заявления невыразимо
болезнен для меня лично и для нашего народа", - написал он у себя в Фейсбуке.

Премьер отметил, что это наилучшее возможное решение сложившейся ситуации, о
чем он обещает подробно рассказать в ближайшие дни.

"Это не победа, но и поражения не будет, пока вы не признаете себя проигравшим.
Мы никогда не признаем себя проигравшими, и это должно стать началом нашей эры
национального единения и возрождения", - отметил он.

По его словам, сейчас нужно проанализировать годы независимости, чтобы
планировать свое будущее и не повторять ошибок прошлого.

"Я преклоняю колени перед всеми нашими мучениками. Я преклоняюсь перед всеми
нашими солдатами, офицерами, генералами, добровольцами, которые защищали и
защищают Родину своей жизнью. Они самоотверженно спасли армян Арцаха. Мы
боролись до конца. И мы победим. Арцах держится", - написал Пашинян.

После этого к Площади Республики в центре Еревана начали съезжаться автоколонны
с возмущенными людьми. Автомобили громко сигналили. 

\begin{center}
	\begin{fminipage}{0.7\textwidth} 
		К Площади Республики в центре Еревана
		стекаются колонны автомобилей с громко включенными сигналами.
		Граждане Армении в считанные минуты отреагировали на заявление
		премьер-министра Никола Пашиняна о прекращении войны в Нагорном
		Карабахе и сдаче территорий.
		\url{https://t.me/SputnikArmenia/9969}
  \end{fminipage}
\end{center}

Причем это произошло в считанные минуты после обращения премьера. Вокруг стоял
крик и свист, некоторые использовали нецензурную лексику. Ситуация у здания
правительства накалялась с каждой минутой. После чего разъяренные граждане
начали выламывать двери Дома правительства\Furl{https://youtu.be/eeMIyy7ZNBs}. 

Охрана правительственного здания не оказала протестующим никакого
сопротивления, пытаясь вступить с ними в диалог. Это позволило манифестантам
прорваться в кабинет Пашиняна, все круша на своем пути. Толпа разгромила
помещение и забрала госсимволы\Furl{https://youtu.be/UyVb-3ShDZw}.

Затем они поднялись на второй этаж, где расположен зал заседаний правительства,
зал встречи и кабинет премьер-министра. Один из манифестантов занял кресло
премьера и неоднократно спрашивал представителя сил правопорядка, где этот …
(далее ненормативная лексика). Тот пожал плечами.

\ifcmt
pic https://strana.ua/img/forall/u/0/0/25252786.jpg
\fi

Ворвавшиеся в парламент Армении демонстранты требовали явки главы Генштаба ВС
Оника Гаспаряна, другие же требовали от депутатов срочно явиться на работу,
поскольку считают, что они могут отменить подписанное Николом Пашиняном
совместно с президентами России и Азербайджана заявление по Карабаху. 

Они ходили по коридорам парламента, заходили в кабинеты депутатов, изредка
повреждали мебель. Мужчины были настроены агрессивно, постоянно матерились. 

\ifcmt
pic https://strana.ua/img/forall/u/0/0/25253738.jpg
pic https://strana.ua/img/forall/u/0/0/25254714.jpg
\fi

Тем временем у парламента Армении произошла стычка между демонстрантами и
охранниками спикера парламента Арарата Мирзояна.

Сам мужчина находился в это время в автомобиле. Протестующие пытались силком
достать его из машины, окружив его и выкрикивая ругательства в его адрес.
Группа военных была вынуждена выйти на защиту.

Но после небольшой стычки между охранниками и демонстрантами, группа мужчин все
же добралась до Мирзояна и избила его. Сейчас спикер в больнице\Furl{https://youtu.be/_m0zXOdldRk}.

После этих событий Никол Пашинян сообщил, что его жизни ничего не угрожает. При
этом он обещал наказать зачинщиков погромов и призвал сторонников верить ему.

"Тех, кто не имеет отношения к погромщикам, прошу уйти домой. Все те граждане,
которые верят мне и верят нам, прошу подготовиться к борьбе", - написал он в
Фейсбуке.

Однако заявления властей уже не могли сдержать протестов. Демонстранты также
ворвались в апартаменты премьера и вынесли из резиденции символы государства и
личные вещи.

В сети появилось видео, на котором видно, как несколько человек заходят в
апартаменты Пашиняна и берут часы, духи и водительские права супруги премьера
Анны Акопян\Furl{https://youtu.be/eoeUOsXmXBc}.

"Из резиденции президента украли компьютер, часы, духи, водительские права и
другое. Это все конечно же "во имя родины", - написал Пашинян о краже на своей
странице в Фейсбуке.

Утром 10 ноября стало известно, что протестующие разгромили также кабинеты
депутатов Национального Собрания Армении. На кадрах видно, как они
переворачивают шкафы и выносят из помещения документы, удостоверения и другие
вещи\Furl{https://youtu.be/DmASgGu4my4}.

Протестующие покинули зал заседаний парламента Армении только под утро, но
часть участников акции продолжает ждать депутатов в коридорах. Один из них,
командир отряда добровольцев "Арабо" Манвела Егиазаряна, ранее обратился к
депутатам, призвав начать процесс выхода из соглашений по Карабаху, подписанных
ночью. Он отметил, что депутаты должны явиться в парламент до полудня, им
гарантируют безопасность. 

В общем, в стране назревает новый Майдан - только уже не за Пашиняна, который
таким образом и стал премьером, а против. 

Потеря Шуши и территорий вокруг Карабаха - действительно большой удар по
национальному самосознанию многих армян. И такой провал на этом направлении,
который произошел в премьерство Пашиняна, конечно, может стоить ему кресла. 

Удержится Никол или нет - покажут события ближайших дней. Также много будет
зависеть от позиций РФ, на которую ориентируется весомая часть населения
Армении.

С одной стороны, в Москве не слишком жалуют Пашиняна: он был лидером "цветной
революции". С другой - приход к власти радикалов, которые выйдут из только что
подписанных договоренностей, может вызвать новый виток войны с Азербайджаном.

А это уже чревато серьезными проблемами для России. Поэтому, скорее всего,
Москва будет настаивать на стабилизации ситуации в Ереване. И смене власти,
если она необходима, более легитимным путем. 

\subsubsection{Уроки Карабаха для Украины}

Вчера, после известий о падении Шуши, в Украине начали раздаваться голоса, что
можно военным путем вернуть и Донбасс\Furl{https://strana.ua/news/299764-kak-situatsija-v-karabakhe-pokhozha-na-donbass-i-smozhet-li-ukraina-zachistit-rehion.html}. Отдельные сторонники "партии войны"
предлагали применить опыт Азербайджана - тем более что Украина тоже пытается
стать союзником Турции и закупает беспилотники "Байрактар". 

Однако сегодня ситуация изменилась - в ходе наступления Азербайджана свои
позиции укрепила именно Россия, которая ввела в Карабах своих миротворцев.
Какие из этого выводы следуют для Украины, анализирует главный редактор
"Страны" Игорь Гужва. 

"Интересно наблюдать, как за одну ночь исчезли все эксперты-бонапарты, которые
призывали Украину повторить опыт Азербайджана и вернуть Донбасс военным путём.

Заключенное соглашение о прекращении огня между Арменией и Азербайджаном при
участии России менее всего напоминает то, чего хотелось бы получить "партии
войны" от наступления на востоке.

По договорённостям стороны останавливаются на уже занятых позициях, а на линию
соприкосновения в Карабахе (то есть на международно признанную территорию
Азербайджана) официально вводятся российские миротворцы. Также Армения передает
Азербайджану все занятые в 1994 году районы вокруг Карабаха.

Параллельно развиваются события в Ереване. Пришедшего к власти на волне Майдана
и пользовавшегося поддержкой Запада премьера Пашиняна люди обвиняют в
предательстве и капитуляции, громят здание правительства. Требуют от парламента
отменить заключенные соглашения.

И большой вопрос, с какой страной в данной ситуации стоит отождествлять Украину
– с Азербайджаном или с Арменией.

Очевидно, что ситуация на Донбассе не просто иная, а прямо противоположна той,
что была в Карабахе накануне начала войны.

Во-первых, для России война в Карабахе была изначально чужая. Армяне сами
отвоевали территории у Азербайджана в 1994 году и Москва тогда лишь выступила
посредником для заключения перемирия между двумя странами.

Во-вторых, Азербайджан --- это не враждебная России страна. Да и с Турцией при
всех нюансах отношения у Москвы, мягко говоря, не предвоенные.

В-третьих, Москва сама убеждала Армению поступиться районами вокруг Карабаха.
Но Ереван не согласился.

В-четвертых, лично премьер Пашинян не является другом и союзником Москвы и
ориентируется на страны Запада.

В-пятых, военно-техническое превосходство азербайджано-турецкого альянса над
Арменией было подавляющим. При этом у России было объективно мало возможностей
это превосходство нивелировать (даже если бы она этого захотела) в виду
отсутствия общей границы с Арменией и Карабахом. Но тем не менее Москва, пусть
и дипломатическим путем, но добилась остановки наступления азербайджанских
войск и не допустила полного разгрома Карабаха, хоть после падения Шуши это и
было делом времени.

О том, что далеко не всем и с азербайджано-турецкой стороны нравятся условия
перемирия, говорит и загадочный инцидент со сбитым азербайджанцами российским
вертолетом, который можно трактовать как желание сорвать в последний момент уже
подготовленные договоренности.

Но это отдельная тема.

Вернемся к нашей ситуации.

Если бы Украина начала наступление на Донбассе, то это привело бы к
результату, который сейчас имеет Армения --- потеря новых территорий за счет
превосходящей военной силы противника, принуждение к заключению нового
перемирия (условных новых Минских соглашений) на куда худших условиях
(вплоть до согласия на введение российских миротворцев, а может и еще
больших уступок), что, вероятно, имело бы фатальные последствия для
действующей украинской власти да и для страны в целом.

И об этом стоит задуматься тем, кто в последние дни любил сравнивать Шушу с
Горловкой".

\ifcmt
pic https://strana.ua/img/forall/u/0/92/%D0%B8%D0%B7%D0%BE%D0%B1%D1%80%D0%B0%D0%B6%D0%B5%D0%BD%D0%B8%D0%B5_2020-11-10_113229.png
\fi
