% vim: keymap=russian-jcukenwin
%%beginhead 
 
%%file slova.krym
%%parent slova
 
%%url 
 
%%author 
%%author_id 
%%author_url 
 
%%tags 
%%title 
 
%%endhead 
\chapter{Крым}

\enquote{... Ці крики та істерику підхопили різні кремляді з провладного пулу. І вони не
вщухають досі. Недоімперія верещить і пускає шмарклі, тому що їх у черговий раз
ткнули, що \emph{Крим} — це не їхня територія. Вся ця історія прекрасно показує, що в
Кремлі все ще усвідомлюють, що \emph{Крим український}, а вони його банально вкрали.
Вірніше, скримздили. А те, що вкрадене, рано чи пізно треба повертати. Поки що
вони не хочуть. Але прийде час... Він обов'язково прийде, і ця постсовкова
недоімперія розвалиться, як СРСР. І тоді ми точно повернемо \emph{Крим}. А може, й
раніше. Поки ж пропоную насолодитися танцями святого Вітта у виконанні
кремляді. І нехай їх корчить і надалі від цілісної України та від слів: Слава
Україні! Героям слава!}, — написав Береза,
\citTitle{Нова форма збірної України з футболу - реакцію Росії аналізує Борислав Береза}, , fakty.ua, 06.07.2021

«Форма футболистов с очертаниями \emph{Крыма} не подстегнет напряженность.
Спорт есть спорт», - разъяснил Песков и заявил, что не стоит разжигать
ненависть.  Пусть себе бегают по футбольному полю с изображением \emph{нашего
Крыма}, пусть ностальгируют, если им так грустно. Есть же пословица: «Чем бы
дитя ни тешилось, лишь бы не вешалось!» Фантасты вот тоже много навоображали в
своих книгах – и что? От этого их фантазии стали былью? Просто не стоит
заострять внимание на больном воображении – тогда клиент быстрее выздоровеет))
И еще поддерживаю нашего \emph{крымского} депутата Госдумы Наталью Поклонскую,
которая написала в Facebook: «Если футболисты на Украине решили гонять мяч в
майках с нашим полуостровом, то предлагаю совместный матч в \emph{Крыму} с
нашими командами, раз уж они так скучают по нам - \emph{крымчанам}. А вообще,
есть такая примета: на крымчан посягнешь - свое потеряешь»,
\citTitle{Форма не главное – главное содержание! И в спорте тоже...}, Мысли Бабы Яги, zen.yandex.ru, 07.06.2021

В 2014 Яценюк говорил - \enquote{Цэ наши территории!}. И ведь не врал,
оговорился всего лишь. Территории - это всё, что их волновало, а на мнение
населения плювать с крыши ненаглядных автогаджетов. Хотя крымчане ясно дали
понять 16 марта того года - где Родина, а где вынужденно находились
\enquote{благодаря} ЕБН... ,
\citComment{KortesMatveich},
\citTitle{Форма не главное – главное содержание! И в спорте тоже...}, Мысли Бабы Яги, zen.yandex.ru, 07.06.2021

Почему бы на нашу форму не \enquote{нарисовать} территорию России с \emph{Крымом}! Интересна реакция хохлов и уефа!,
\citComment{Дмитрий Дружинин},
\citTitle{Форма не главное – главное содержание! И в спорте тоже...}, Мысли Бабы Яги, zen.yandex.ru, 07.06.2021

%%%cit
%%%cit_pic
%%%cit_text
Еще и у Союза европейских футбольных ассоциации (УЕФА) стали требовать
запретить эту форму, хотя известна позиция европейцев на принадлежность нашего
полуострова. Украина может себе хоть весь земной глобус нарисовать и бегать
потом с ним. От этого ничего не изменится. Такая ситуация и с \emph{Крымом}. У них в
Киеве целые ведомства, которые представляют \emph{Крым} находятся, но от этого ничего
не меняется. Как \emph{Крым} был в составе нашей страны, так он там и остается и
останется
%%%cit_comment
%%%cit_title
\citTitle{Какая разница на каких футболках Крым, главное он в составе России}, 
Добрый Человек, zen.yandex.ru, 11.06.2021
%%%endcit


