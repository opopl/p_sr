% vim: keymap=russian-jcukenwin
%%beginhead 
 
%%file slova.zoloto
%%parent slova
 
%%url 
 
%%author_id 
%%date 
 
%%tags 
%%title 
 
%%endhead 
\chapter{Золото}

%%%cit
%%%cit_head
%%%cit_pic
\ifcmt
tab_begin cols=3
  pic https://strana.news/img/forall/u/0/34/b62c7a55642442d8d3406a1abbb8f2c2.jpg
	pic https://strana.news/img/forall/u/0/34/ddfb5d17538f82f9ef91838c5046102f.jpg
	pic https://strana.news/img/forall/u/0/34/2cbf05cc95b761af2a9790a69ec9f5b1.jpg
tab_end
\fi
%%%cit_text
6 октября Апелляционный суд Амстердама объявил решение о судьбе "скифского
\emph{золота}". Древнюю коллекцию постановили отправить в Украину.  Суд
посчитал, что хотя объекты происходят из Крыма, они являются частью наследия
Украины после того, как она стала независимой в 1991 году.  После вынесения
решения спорные экспонаты останутся в Нидерландах еще минимум на три месяца, в
течение которых возможно обжалование этого решения в суде высшей инстанции.
После решения голландского суда Владимир Зеленский заявил, что вслед за
\emph{"скифским золотом"} Украина вернет и Крым. Такая запись появилась в его
Twitter-аккаунте
%%%cit_comment
%%%cit_title
\citTitle{Впереди новые тяжбы. Как Украина выиграла суд за \enquote{скифское золото} и почему его пока не отдают}, 
Оксана Малахова, strana.news, 26.10.2021
%%%endcit
