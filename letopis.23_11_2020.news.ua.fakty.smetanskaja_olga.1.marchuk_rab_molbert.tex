% vim: keymap=russian-jcukenwin
%%beginhead 
 
%%file 23_11_2020.news.ua.fakty.smetanskaja_olga.1.marchuk_rab_molbert
%%parent 23_11_2020
 
%%url https://fakty.ua/361541-ivan-marchuk-ya-rab-prikovannyj-k-molbertu
 
%%author Сметанская, Ольга
%%author_id smetanskaja_olga
%%author_url 
 
%%tags 
%%title Иван Марчук: «Я раб, прикованный к мольберту»
 
%%endhead 
 
\subsection{Иван Марчук: «Я раб, прикованный к мольберту»}
\label{sec:23_11_2020.news.ua.fakty.smetanskaja_olga.1.marchuk_rab_molbert}
\Purl{https://fakty.ua/361541-ivan-marchuk-ya-rab-prikovannyj-k-molbertu}
\ifcmt
	author_begin
   author_id smetanskaja_olga
	author_end
\fi
\index[names.rus]{Марчук, Иван!Художник}

\ifcmt
	 pic https://fakty.ua/photos2/article/36/15/361541w540zc0.jpg?v=233022
	 caption Иван Марчук в Испании
\fi

11:34 — 23 ноября 2020
Ольга СМЕТАНСКАЯ, «ФАКТЫ»

\begin{leftbar}
	\bfseries
Несмотря на пандемию, жизнь продолжается. В Мадриде, в Музее авангардного
искусства, который находится в самом центре столицы Испании, в помещении
бывшего железнодорожного вокзала девятнадцатого века, открылась
персональная выставка украинского художника, вошедшего в список гениев
современности, Ивана Марчука. Его творчество сегодня широко известно
во всем мире.
\end{leftbar}

О своей поездке в Испанию и о выставке в эксклюзивном интервью
он рассказал «ФАКТАМ».

\subsubsection{«У меня брали интервью самые известные в мире искусствоведы. Один из них работал с Пабло Пикассо»}

— Иван Степанович, от всей души поздравляю вас с приятным событием!

— Спасибо. Я был в Испании девять дней. Моя ретроспективная выставка,
организованная при поддержке посольства, охватывает работы от 1960-х годов
по сегодняшний день. На ней представлено 184 мои картины. Распаковав
работы, испанцы были потрясены. Выставка получила огромный резонанс.
Столько отзывов! Буквально на третий день с просьбой организовать у себя
мою выставку обратились Валенсия и Андорра. Испанцы поймали «золотую
рыбку» и теперь хотят нарадоваться (улыбается).

— Сколько же времени продлится выставка?

— Два с половиной месяца. Это редкость. Пусть люди смотрят! Кроме того,
сейчас идет активная работа над каталогом. Все это для моей репутации
новая ступень.

\ifcmt
pic https://fakty.ua/user_uploads_new/images/articles/2020/11/23/361541/%D0%BD%D0%B0%D1%82%D1%8E%D1%80%D0%BC_n.jpg
caption Натюрморты художника всегда необычны (фото из альбома Ивана Марчука)
\fi


— Столько приятного на фоне пандемии!

— Да. Посетители выставки смогут увидеть, как я говорю, 12 разных
Марчуков. И они удивятся, что один художник может создавать такие разные
картины. В творчестве я ненасытный, голодный, одержимый! Постоянно думаю
о том, что еще показать из того, чего в мире еще не было.

Читайте также: Иван Марчук: «Уже работаю над новым циклом картин — душа,
как писал Шевченко, «оживає і сміється знову»

— Как проходило открытие выставки в новых реалиях?

— Все были в масках. Количество посетителей регулировалось. У меня брали
интервью самые известные в мире искусствоведы. Один из них работал с Пабло
Пикассо. Эти искусствоведы дадут свои резюме для каталога.

\subsubsection{«Мне очень понравились испанские вино и сыры»}

— Испания изменилась в связи с пандемией?

— А я до пандемии в этой стране не был, поэтому не знаю. Все люди
в масках. Мне удалось побывать в нескольких музеях. Конечно, в музее
Прадо. Особенно интересно было посетить Толедо. Вспоминается картина Эль
Греко, где он нарисовал это удивительное место. Толедо — чудо архитектуры.
К слову, в путешествиях она меня всегда интересует больше, чем живопись.
Архитектектура — лицо городов.

\ifcmt
pic https://fakty.ua/user_uploads_new/images/articles/2020/11/23/361541/%D0%9C%D0%B0%D1%80%D1%87%D1%83%D0%BA%20%D0%BC%D0%B0%D1%81%D0%BA%D0%B8_n.jpg
caption На открытии выставки все были в масках (фото из альбома Ивана Марчука)
\fi


— Как вас принимали в Мадриде?

— Прекрасно. Я жил в гостеприимной семье украинцев.

— Знаменитую испанскую паэлью попробовали?

— Мне очень понравились вино и сыры. Я же стремился для друзей, у которых
жил, приготовить что-то свое. Делал салаты, варил суп с чечевицей, так как
они там первое не готовят.

— Что-то купили в Испании?

— Знаете, я никогда ниоткуда не привожу сувениры. Даже конфеты не покупаю
за границей. Сам их не ем и дарить не хочу, так как сладкое вредно.

— Поездка вас вдохновила на новые произведения?

— Я не завишу от вдохновения. Вдохновить меня могут только женщины
(улыбается). Каждое утро иду в мастерскую и знаю, что должен что-то
делать. Сейчас вот, к примеру, рисую сумерки: солнце прощается с днем, оно
оставляет нас. И мне хочется передать это настроение…

— Счастливый вы человек, Иван Степанович!

— Смотря что считать счастьем. Если так, как его обычно понимают люди,
то по-человечески считаю себя самым несчастным человеком — я раб,
прикованный к мольберту!

— И все же что-то вас, наверное, в жизни радует?

— Радуюсь, когда вижу красивую женщину, которая мне улыбается, когда
со мной кто-то здоровается на улице, и я говорю: «Я вас не знаю»,
а в ответ слышу: «Зато мы все вас знаем!» По пути домой из мастерской
несколько человек обязательно со мной здороваются. Это приятно!

\ifcmt
pic https://fakty.ua/user_uploads_new/images/articles/2020/11/23/361541/%D0%BF%D0%B5%D0%B9%D0%B7%D0%B0%D0%B6%20%D1%81%D0%BD%D0%B5%D0%B35_n.jpg
caption Пейзажами Ивана Марчука можно любоваться бесконечно (фото из альбома Ивана Марчука)
\fi


\subsubsection{«Ценю удобство. Считаю, что лучшие брюки — джинсы»}

— Наверное, на открытии выставки в Мадриде у вас все хотели взять
автограф?

— Да, раздал их немало. На следующий день после открытия была встреча
с диаспорой. Присутствовало много молодежи. И все просили автографы
и сфотографироваться на память.

Читайте также: Иван Марчук: «Больно за Украину. Почему мы на райской
земле живем, как в пекле?»

— Смотреть на ваши картины можно бесконечно…

— Я весь мир уже ими очаровал (улыбается). И мне хочется, чтобы все больше
людей в разных странах увидели мои произведения. Искусство должно
принадлежать человечеству.

— Пандемия закрыла границы, но ваше творчество их не знает. Недавно в США
на одном из сайтов демонстрировалась коллекция одежды дизайнера Алены
Серебровой, созданная по мотивам вашего творчества.

— Очень красивые платья! Я бы хотел, чтобы мои картины печатались на целых
полотняных гектарах. Их нужно увеличивать! Маленькие картины я рисовал
в период, когда был запрещен, чтобы их можно было переслать.

— А вы любите красиво одеваться?

— Я ценю удобство. Считаю, что лучшие брюки — джинсы. Они у меня и для
парадного выхода, и для работы.

— Что бы вы пожелали читателям «ФАКТОВ»?

— Мир прекрасен и безграничен! Планета Земля — это чудо. Любите ее!
Интересно все: каждая травинка, деревце, цветок… Получайте удовольствие
от всего этого! Будьте добрыми, счастливыми и радостными. Любите
искусство!

Читайте также: В КГБ от меня требовали покаяться «в ошибках» и «признать
вину», — художник Иван Марчук

Фото в заголовке из альбома Ивана Марчука

1609

Читайте нас в Telegram-канале, Facebook и Twitter

Сделайте "ФАКТЫ"
избранным источником в Google News
fakty logo
Google News
К содержанию раздела Заметили ошибку? Выделите её и нажмите CTRL+Enter
* Печатная версия
* Письмо редактору
* Пожаловаться
Введите вашу жалобу
* _________________________________________
_________________________________________
_________________________________________
_________________________________________
_________________________________________
_________________________________________
_________________________________________
* Введите вашу жалобу
* * _____________________ Ошибка!
* [ Отмена ]
* [ OK ]
*   Поделиться
*   Поделиться
*   Твитнуть
Сейчас читают
Андрей Богдан

Андрей Богдан получил ответ Кремля на свой призыв к Путину

граница «ДНР» и России

Устроили гетто: жители "ДНР" шокированы решением российских пограничников

Киперман Соня

"А лифчик где?" Дочь Веры Брежневой устроила смелую фотосессию возле
лимонного дерева

Родион Гайсинский

Сын Кернеса после долгой паузы опубликовал фото из Берлина

Следующий материал
Читайте также
Новости партнеров
Коронавирус в Украине
(260-й день карантина)
Заболели:
677189
+15331
Вылечились:
317395
+9617
Умерли:
11717
+225
карта Украины

Статистика заболеваемости и
смертности по областям

Контакт-центр Минздрава по борьбе с COVID-19:
0 800 60 20 19 (круглосуточно)


Популярное

* 1Из-за боли уже собирался менять тазобедренный сустав, но мне помогла
клеточная терапия, о которой прочел в «ФАКТАХ»

22 ноября 2020 6:55
* 2Как справиться с одышкой и кашлем при COVID-19: советы врача

24 ноября 2020 7:46
* 3«Зачем вы сюда приехали? У нас нет мест»: истории тяжелых пациентов,
которых больницы отказывались принимать

20 ноября 2020 20:31
* 4Иван Марчук: «Я раб, прикованный к мольберту»

23 ноября 2020 11:34
* 5Игорь Луценко: «Если бы мы не выстояли на Майдане, никакой Украины не
было бы»

24 ноября 2020 10:27

* Рейтинг формируется за последние 7 дней

Паперовий
тижневик
* Головне за тиждень
* TV-програма
* Гороскоп
* Медичні поради
55
Передплачуй
зараз!
А тепер ще і у форматі PDF >>
Анекдот от «ФАКТОВ»

Исаак Соломонович был в прекрасной спортивной форме. Правда, она... не
застегивалась у него на животе.

Пасмурно, дымка
Киев
+3

Ветер: 4 м/с  Ю-3
Давление: 749 мм

Погода в других городах

Больше в разделе:
* 26 ноября 2020 13:40

«Господи, сколько еще выдержим?!» — откровения Дмитрия Комарова о проекте
«Мир наизнанку»

* 26 ноября 2020 11:33

Роман Виктюк: «Я буду посылать сигналы из своей вечности. Потому что
творческая энергия не исчезает никуда»

* 25 ноября 2020 19:06

Главред "ФАКТОВ" Швец: "Сейчас вижу себя со стороны маленького, 6-летнего,
который кричит: “Не опускайте маму в могилу!”

* 24 ноября 2020 20:04

«Нос словно онемел»: Анатолий Анатолич рассказал свою историю борьбы с
коронавирусом

* 24 ноября 2020 12:13

Звезда сериала «Ферзевый гамбит» Аня Тейлор-Джой: «Шахматный мир полон
страсти и невероятно сексуален»

Карта портала
* Новости

* Украина

* Политика

* Деньги

* Происшествия

* Мир

* Здоровье

* Прямая линия

* Спорт

* Футбол

* Культура

* Шоу-бизнес

* Фигурка дня

* Интервью со звездой

* Именинники дня

* Общество

* Стиль жизни

* Люди и звери

* Житейские истории

* История современности

* Наука и технологии

* Клуб потребителей

* Перед лицом Фактов

* Спросите у Фактов

* Блоги

* Погода

* ФотоФакты

* Профиль пользователя

* Гороскоп

* Регистрация

© 1997—2020 «Факты и комментарии®»

Все права на материалы сайта охраняются в соответствии с законодательством
Украины

Материалы под рубриками "Официально", "Новости компаний", "На заметку
потребителю", "Инициатива", "Реклама", "Пресс-релиз", "Новости отрасли" а
также помеченные значком публикуются на правах рекламы и носят
информационно-коммерческий характер

* Реклама
* Подписка
* Контакты
* Поздравления и соболезнования
* Правила пользования сайтом fakty.ua
BigMir counter
PHP version: 7.4.12
start: 1606426405.9537
finish: 1606426405.9954
render time: 0.041771173477173
