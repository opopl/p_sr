% vim: keymap=russian-jcukenwin
%%beginhead 
 
%%file 19_01_2022.yz.maj_dnr.1.russkij_plohoj
%%parent 19_01_2022
 
%%url https://zen.yandex.ru/media/id/5f8f226b1fe36c1d9e02a36b/russkii--znachit-plohoi-61e7d9bfffe51956d68bea7b
 
%%author_id yz.maj_dnr
%%date 
 
%%tags napadenie,rossia,russkie,russofobia,ugroza,ukraina
%%title Русский – значит плохой?
 
%%endhead 
\subsection{Русский – значит плохой?}
\label{sec:19_01_2022.yz.maj_dnr.1.russkij_plohoj}
 
\Purl{https://zen.yandex.ru/media/id/5f8f226b1fe36c1d9e02a36b/russkii--znachit-plohoi-61e7d9bfffe51956d68bea7b}
\ifcmt
 author_begin
   author_id yz.maj_dnr
 author_end
\fi

Когда моя дочь была маленькой, я пугала ее «страшным дядей милиционером». С
воодушевлением я рассказывала ей что он «украл» бутылочку, любимую соску,
похищает непослушных детей, следит за тем, чтобы Маша быстро ела кашу и не
оставляла ничего на тарелке, а по вечерам «проверяет», все ли дети легли спать
вовремя, словом, этакий плохой Бабайка.

\ii{19_01_2022.yz.maj_dnr.1.russkij_plohoj.pic.1}

Однажды в магазине дочь увидела охранника в форме и, вырвав свою ручку из моей
подбежала к нему со звонким криком и размахивая маленьким кулачком: «Страшный
дядя милиционер! Я тебя не боюсь, верни мне мою любимую соску, иначе я буду с
тобой драться!». Покупатели и продавцы просто легли от смеха, парень растерялся
и покраснел, а я, оттаскивая Машуню, только и нашлась разъяснить дочери, что
она ошиблась, и это не милиционер, а военный, который, наоборот, защищает
детей. Ну, и все в этом духе. С того дня страшный дядя «М» исчез из нашей
жизни. А на Машкины вопросы, куда это он делся, я пояснила, что дядя испугался
храброй Машеньки и исправился – теперь он защищает всех деток и взрослых от
разбойников.

С сыном я уже не ставила подобных экспериментов, решив, что подобными
манипуляциями храброго мальчика не воспитаешь...

А теперь о деле

Никто уже не сомневается в том, что слухи о грядущем нападении России усиленно
распространяют американцы. Собственно, США эту истерию и придумали, из их
методичек украинские спикеры и называют веселящие цифры «разного достоинства»
от 100 тысяч до 145 миллионов (да, да, скрин сказанного Кулебой у меня лежит в
архивах). Понятно, что этот слушок сам по себе является мощным антироссийским
снарядом, который создаёт невыгодную для России информационную обстановку во
всём мире.

Но я сейчас не об этом. Много лет спустя моя дочь мне заявила: «Мама,
отсутствие у меня внутреннего неприятия людей в форме – это не твоя
недоработка, а чудо истины, действующей вопреки твоим воспитательским изыскам».
Понимаете, о чем это? О создании недоверия, отрицания и даже ненависти к
определенному слою населения или, как в случае со слухами о нападении, к целой
стране. Это такой элемент холодной войны, закрепляющий в каждой голове
безусловный рефлекс «Русский – значит плохой». И ведь не только в украинских
гражданах, хотя они и есть главный сегмент «покупателей» этого мема, но и
насчёт всего мира ребята пыхтят продать эту дезу подороже. Еще скажу, что
американцам удалось преизрядно напугать украинских граждан. Такого уровня
нервного страха я не припомню, даже в 2014 такого не было.

Еще одна немаловажная деталь: эти слухи, безусловно, оказывают негативное
влияние на экономику России, и это, на минуточку, происходит до введения США
всяких санкций. Я уже не говорю о внешних, но даже внутренние инвесторы
оглядываются в поисках ответов на вопрос: «А сколько правды выдают американские
СМИ? А вдруг?» забывая простую геббельсовскую истину «Чем чудовищнее ложь, тем
быстрее в нее поверят».

Я говорю о том собственно, что нагнетание военных слухов антироссийской
командой – это целая операция, самодостаточная, отдельная, со своими целями,
задачами и ожидаемыми результатами. Одно «но» есть у этой операции – она
краткосрочная. А без должной постоянной поддержки тезиса «Русский – значит
плохой», она еще и недолговечная. Мало того, повторять дважды, а тем более
несколько раз такую операцию невозможно, иначе сработает сказка про волков и
пастушка, крикам «Волки, волки!» которого на третий раз уже никто не поверил. И
то сказать, человек, а тем более, человеческое общество не могут долго жить в
состоянии нервного перевозбуждения. В конце концов, американцам перестанут
верить даже их собственные граждане. А потом над журналистами, рассказывающими
про мифическую российскую угрозу, станут просто смеяться.

Кстати, при всей нервозности обстановки на Украине, очень многие стали писать
издевательские комментарии под сообщениями типа: «!!!Россия планирует взять
Киев и свергнуть действующую украинскую власть». Например, прочтите комментарии
под постом украинского телеграмм канала «Орен Юкрейн»:
\url{https://t.me/OpenUkraine/5099}

Пастушку уже не верят.

Это я к чему, собственно? А к тому, что:

1. Россия правильно сделала, что запаслась терпением и деятельно пережидает
очередную волну русофобских атак, раз за разом разъясняя, что не стоит верить
вракам про нападение. Убеждена, волна спадет до того, как исчерпаются резервы.

2. Стоит отметить, что американские потуги изрядно бьют по самой Украине.
Во-первых, все большее количество украинских граждан убеждается в лживости
американских и украинских СМИ. Во-вторых, обвал украинских еврооблигаций и
активный, до панического, вывод средств из украинских внутренних облигаций
буквально топит, из без того помирающую, украинскую экономику.

3. Наблюдаю за странной тенденцией: все больше украинцев в сетях пишут что-то
вроде «Ну когда уже русские придут и спасут нас? Мы тут заждались!». И это тоже
добрый звоночек. Новое смысловое качество оксюморона «Русский – значит плохой»
заключается в том, что все большее количество людей в мире начинают считать
«русского медведя» едва ли не спасителем Человечества. Ну, по-хорошему, это так
и есть.

Думаю, приходит конец эры «плохих русских» в кино и сериалах, в компьютерных
играх и текстах песен, этот стереотип перестает быть модным, причем,
чрезмерными усилиями самих же русофобов. Оно всегда так – пережаренные котлеты
несъедобны, даже если они изготовлены из мраморной говядины.
