% vim: keymap=russian-jcukenwin
%%beginhead 
 
%%file 31_03_2018.fb.fb_group.story_kiev_ua.1.babulichka_julia_6
%%parent 31_03_2018
 
%%url https://www.facebook.com/groups/story.kiev.ua/posts/781824018681068
 
%%author_id fb_group.story_kiev_ua,bochkovskaja_evgenia
%%date 
 
%%tags kiev,pamjat,semja
%%title МОЯ  (НАША) БАБУЛИЧКА ЮЛИЯ! (продолжение 6)
 
%%endhead 
 
\subsection{МОЯ  (НАША) БАБУЛИЧКА ЮЛИЯ! (продолжение 6)}
\label{sec:31_03_2018.fb.fb_group.story_kiev_ua.1.babulichka_julia_6}
 
\Purl{https://www.facebook.com/groups/story.kiev.ua/posts/781824018681068}
\ifcmt
 author_begin
   author_id fb_group.story_kiev_ua,bochkovskaja_evgenia
 author_end
\fi

МОЯ (НАША) БАБУЛЧКА ЮЛИЯ !

(продолжение 6)

После предыдущей главы прошло много времени... И, в силу сложившихся
обстоятельств, у меня была возможность ещё раз «перелистать книгу», той
когда-то такой счастливой, оставшейся в прошлом жизни... Со всеми её трудностями,
неустроенностями и без каких-либо намёков на сегодняшний комфорт. 

1953 год и 2018 год… 

Разрыв всего лишь 65 лет ! Но, как видим уже сегодня, - это целая эпоха во
времени… Очень невероятно удивительным вообще вспоминается время той прошлой
жизни. По крайней мере в нашей «деревеньке» - тогда по ул.Церковной, а потом
по  ул.Верхней. 

Дрова привозились. Их нужно было занести во двор. В каждом дворе были козлы, на
которых эти дрова (поленья, доски... - что кому удавалось раздобыть) нужно были
распилить... - пилить нужно было только вдвоём... Помню эту большую ленточную
пилу с двумя ручками, крупные зубья которой регулярно точились, потом порубить
(рубила бабуля) эти \enquote{пеньки}... и сложить в сарай, чтобы можно было потом
растопить печь (грубу), согреть дом и приготовить еду. У нас была «русская»
печь в которой можно было печь, готовить на регулируемых конфорках, и на
которой можно было и спать... Зад этой печи выходил в нашу с бабулей комнатку.
Вдоль печи было множество печурок-углублений, где всегда что-то подсушивалось
(и травки тоже...). И под печью было открытое окошечко в подпечье... А нас очень
любили мышки... И их норки, в основном, и выходили в это подпечье.

И только мне было «доверено», вследствие моей тощести и малолетства залазить в
это окошечко в подпечье и замазывать глиной со стеклом и замуровывать жестяными
пластинами из под консервных банок ходы в эти норки... Никогда не забуду ночной
«шкряб...» этих мышек. Кот Степан очень ответственно относился к своим
обязанностям..., но не всегда успевал... Поэтому мышеловки тоже имели место...
Короче, флора и фауна (цветочки и всякая живность) в нашем доме жили дружно...
Всем хватало места... И всем «уделялось» внимание... В общем углу четырёх комнаток
была сооружена груба с духовкой, которая тоже редко пустовала в холодное время
года... Бабуля в ней что-то то тушила, то «подпекала»... И моя кровать ногами была
как раз в эту грубу... Всегда было сухо, тепло и очень уютно... И кот Стёпка в
ногах... Дрова «имели» запах... Тепло тоже пахло «сушкой» из яблок и слив...
Незабываемо...

А в примыкающей кухне вовсю уже хозяйничали примус и ДВОЙНОЙ ! керогаз... Для
примусов и керогазов покупался керосин в специальных «лавках» неподалёку, но на
окраинах. Керосинки уже отходили... Потихоньку стали проводить свет !!! Это уже
был БОЛЬШОЙ ПРОГРЕСС ! Но утюги, как и прежде, грели на плитах или угольками,
которые специально закладывали во внутрь утюга. И надо было гладить очень
аккуратно, чтобы искры и сами угольки «не выпрыгивали» во время глажки. А
зимой, бельё постиранное в балии руками и на «таре» простым хозяйственным мылом
и вываренное в выварке (хозйственное мыло тёрлось на тёрку, добавлялись всякие
«тонкости» для белизны - ...силикатный клей и даже ...\enquote{пурген}..., а полоскалось
обязательно с крахмалом и синькой при последнем полоскании, вывешивалось на
мороз, замерзало «дыбом...», и потихоньку на верёвке во дворе вымораживалось, и
его, не до конца выморозив (высушив), несли в дом с таким невероятно чистым и
никогда НЕ ЗАБЫВАЕМЫМ запахом зимы... И «раскочегарив» утюги (обычно их было два
- на смену, - один гладил, другой грелся), спешили скорее всё прогладить, пока
бельё хранило влагу и легче гладилось. 

И от крахмала всё бельё было такое белоснежное, гладенькое и лоснящееся..., а от
мороза такое душистое... И как же пахла эта постель...! А белизна белья всегда
характеризовала хорошую хозяйку...

Но сколько же воды нужно было принести в вёдрах из колонок-слоников, которые
находились не рядом с домом... Но гладить уже можно было и вечером... Уже был свет.
Так что моим бабуле и мамуле доставалось на полную катушку... Мне только доверяли
полотенца... В доме мужчины не было... Папу отправили на новое место работы и мы
остались втроём. Но в то время в наших условиях «частных» домов все так жили.
Вспоминаю это всё С УДОВОЛЬСТВИЕМ... Но, одновременно, сравнивая... - думаю..., как
же тяжело тогда было женщинам... Удобств никаких не было... Всё было во дворе...
Но жили тогда все как-то радостно, доверчиво и очень дружно. За детьми
приглядывали все по соседски... Нас детей все любили. Всё было на глазах... Даже,
если мы ганяли по кладбищу в \enquote{стариков-разбойников}... Там, прямо у входа, на
самой территории кладбища тоже жили люди... Все друг друга давно знали. 

И ближние соседи всегда бегали одалживали друг у друга разное..., вдруг нужное
по хозяйству... По улице частенько ездила телега старьёвщика с лошадью. 

«Лужу, точу, паяю, ветошь собираю»...! К нему сразу выстраивалась очередь из
кастрюль..., вёдер, ножей, ножниц и пил... Детвора вылазила на заборы, чтобы лучше
увидеть весь этот процесс...

Особенно всех завораживал процесс точения... Специальная педаль крутила колесо,
которое в свою очередь крутило круглое точило... и от него во время заточки во
все стороны сыпались искры... 

Незабываемое для детворы зрелище... 

И точно также при первых звуках траурного марша на заборы взлетала вся детвора
и отслеживался весь ход похоронной процессии на кладбище, которое находилось за
углом..., а потом ...шло обсуждение... - молодой..., старенький... и другие
подробности... 

Шла активная деревенская городская жизнь...

Огороды сажались... - на зелёный борщ и салат хватало..., яблоки, сливы и ягоды
сибирались... Варенье во дворах в медных тазах на двух кирпичах варилось... «Пенки»
не успевали собираться... А потом зимой пили чай из медного самовара в прикуску с
вареньем или кусковым рафинадом... Или пеклись пироги с вареньем... А бабуля
ещё из варенья варила и кисели... Всегда вкусные... Любые...

Гигиена поддерживалась походами раз в неделю в баню на ул.Московскую. Ехать
туда надо было трамваем или №16, или №30. На первом этаже находился кинотеатр
«Авангард», а рядом отдельный вход в баню. Внизу был буфет с пивом. А моечные
залы - мужской и женский были на втором этаже. К ним вела лестница вверх. Под
выходные вдоль этой лестницы стояла двойная очередь в мужской и женский зал.
Мужчины стояли слева под стенкой, т.к. вход в их зал был сразу первый слева. А
женщины стояли справа, облокотившись на перила... Вход в женский зал был
следующий за мужским. И там же на площадке были семейные номера... Вожделенная
тайная мечта всех детей покупаться там, только со своей мамой... И вот в самом
начале зимы 1952 года я с мамой и бабулей тоже пошла в баню. Очень хорошо помню
большой просторный зал раздевалки с номерками, ящиками, сидениями, простынями и
весами. Тогда все взвешивались... И считалось ХОРОШИМ тоном..., если вес был больше
предыдущего... Считалось, что если у тебя «хорошие» формы..., то ты хорошо
питаешься и у тебя «благополучная» и устроенная жизнь... 

Вот такая была «узость» мышления на тот момент после всех голодовок... В моечный
зал вела дверь, из которой, когда она открывалась, валил белый душистый
туман-пар... Зал был большой, с высокими потолками. В стене зала был также вход и
в парную... Из дверей парной в моечный зал валил ещё больший пар... По залу
рядами стояли большие мраморные скамьи – где-то 150 см на 60 см. Красивые, «в
крапочку», чистые, гладкие. В раздевалке мы брали свободные шайки-тазики. С
собой всегда мочалки, которые «росли...» и мыло земляничное для тела и яичное –
для головы... Мама вымыла лавку, чтобы меня посадить... Села я..., мылилась-мылилась...
и свалилась с лавки... Перелом ключицы... И это было третье горе моего детства...
Гипс до пояса..., плечо и рука в гипсе... ГОРЕ...!!! Все на санках... А я рисую, пишу,
читаю... Благо, что правая рука была здоровая... Но к Новому 1953 году уже всё было
хорошо и мы снова пошли в баню, чтобы смыть всю «гипсовую» пыль...

Март 1953 года... Смерть Сталина... Я горя в доме не помню... Была просто ТИШИНА...
Над дверью в передней висела радио-тарелка, которая, практически, всегда
вещала... Вот она и «говорила»... Мама тогда уже работала на П/Я... Бабуля в
арсенальском детсаду... Моя тётя уже работала в кабмине. Её муж работал в своей
сельхоз-лес академии... У меня уже были две двоюродные сестрички... Частенько все
они приезжали к нам в гости. И тогда у нас дома всё было вверх тормашками... А
как могло быть по другому в дружной семье... Детишки скакали... - прямо с печки...
! на кровать...! (тогда на кроватях лежали перины и пуховые подушки, а ватные
одеяла бабуля шила сама), а взрослые кушали... и беседовали... У каждого был
свой интерес... ВСЕМ БЫЛО ХОРОШО !

Но пора было меня и в школу отправлять... Моя 88 школа ещё только
достраивалась..., опять таки, теми же пленными немцами... Весь этот микрорайон по
ул. Е. Бош и ул. Бастионной тогда был построен ими.

Поэтому 1-го сентября наши четыре первых класса пошли в 133 школу, которая
находилась совсем рядом, на ул. Бастионной. И сразу напротив 133 школы открылся
новый магазин «Книга», где для нас и были приобретены наши самые первые
учебники - Букварь, Чтение, Арифметика..., первые тетради в густую косую
линеечку, тетради в клеточку, альбомы, карандаши, ручки, ПЕРЬЯ..., чернильницы...,
резинки, деревянные пеналы... Всё это сокровище было такое новое, такое красивое...
и так пахло типографской краской... И, правда, очень хотелось учиться и
стараться... Это я говорю искренне. И дальше по всей моей жизни этот магазин
"Книга" оставался моим самым любимым книжным магазином, пока он существовал.
Очень хорошие у меня были и отношения с его директором. 

Очень хороший, эрудированный был человек. Этот магазин всегда обеспечивал всеми
учебниками все школьные программы. А сколько хороших книг там было приобретено.
Сейчас его уже нет... Книги сейчас не в почёте... ОНИ ТЯЖЁЛЫЕ... Их нужно
носить... 

Моя первая любимая учительница Магилат Елена Гавриловна жила недалеко от нас на
нашей улице. Её уже нет… Но у меня о ней самая добрая память. 

Бабуля пошила мне мою самую первую шерстяную коричневую форму с юбочкой «в
складку»..., белый с кружевом передник. Белый кружевной воротничок и кружевные
манжетики... Купили на «стройке» в галантерейном магазине атласные белые
широкие ленты для бантов в косички на праздник и коричневые атласные ленты на
каждый день. И коричневый портфель... отделанный под крокодиловую кожу ! 

Первые две четверти мы проучились в 133 школе. А к концу третьей четверти
дружно, с цветочными горшками в руках, перешли строем под барабан в очень
солнечный весенний день в построенную, совсем уже нашу, пахнущую краской,
совершенно новую, очень просторную и с красивым очень паркетом в залах, школу
№ 88! 

В школу меня никто не водил. Я «девка» взрослая..., сознательная..., дорогу знаю...
Дома должна была быть, когда положено... Библиотека и секции поощрялись. Но
обязанности по дому строго отслеживались... А как по другому...? За тебя никто
ничего не сделает... Все заняты...

Самостоятельность не проходила без «анализа»... Во втором классе перед Новым
годом писали диктант по русскому языку... Тема была зимняя... Дословно уже не
помню..., но суть такая... - диктуется предложение..., а мне в этом предложении
очень не нравится одно слово..., не уверена в СВОЕЙ правильности... Не уверена...,
- значит надо заменить на близкое по смыслу... ЗАМЕНИЛА !!! Получила «КОЛ» на
всю страницу !!! - от своей любимой Елены Гавриловны... и ВЫЗОВ! мамы в школу...!
Елена Гавриловна даже маме записку ...адресовала ! ...Вот так ребёнка отличника
можно и до суицида довести... Дома был разбор «полётов»... Не знаю, - говорю...
Что-то с диктантом..., а сама трясусь...

Мама же в школу завтра пойдёт... А ЧТО ПОТОМ БУДЕТ...???!!! Мама пошла... Я трясусь...
Зовут меня... - ...ты поняла, что ТАК ДЕЛАТЬ НЕЛЬЗЯ ?!... Конечно ...Я ПОНЯЛА...
Конечно ...НЕ БУДУ...! А сама трясусь..., понимаю, что теперь и дома «разбор» будет...
Прихожу домой..., тихонько стою под дверью... - зайти не решаюсь... И тут слышу за
дверью смех... - «не пропадёт..., выход найдёт...». Решилась войти... - уже не
смеются..., но глаза выдают... В общем пожурили..., а в итоге сказали : ты теперь ВСЁ
ПОНЯЛА...? Конечно поняла и запомнила...

На этом, можно сказать, детство и закончилось... Всему, что я умею, я обязана
бабуле. В школе на уроках труда нам девочкам дали азы шитья. Тогда женского
нормального белья, как такового, я не помню. Нас по «Бурде» научили шить себе
«бюстики» и трусики... Как бабуля делала выкройки, я тоже подсматривала... И в 15
лет я уже себе сама шила из ситчиков по той же «Бурде» платьица на нижних
накрахмаленных юбочках и костюмчики... На 14-летие бабуля подарила мне «золотую»
Кулинарную книгу... К этому мне тоже интерес привили... \enquote{Всё должна уметь сама...
Неумеху замуж не возьмут}...! Во все мои тяжёлые моменты и радости бабуля
всегда была рядом... и делила их со мной... Я, правда, не всегда была подарок...
Порой Я... занималась воспитанием бабули... Мне сейчас так смешно это... Мне
казалось, что она не всегда была права..., когда касалась личной жизни моей мамы...
И я бабуле устраивала голодовки... в знак протеста... 

А она придёт... и сразу к кастрюлькам... - ела я или нет... Моя бабулечка ! Потом
мы с ней выдали моего мамчика замуж... за хорошего человека... И остались с
бабулей уже вдвоём строить дальше мою уже взрослую жизнь. На все Новогодние
карнавалы у меня в школе всегда были из марли разные карнавальные костюмы...
Помню в четвёртом классе, как раз по Хрущёву, - костюм кукурузы... с короной
оклеенной кукурузой... Все призы всегда были мои... 

В 1971 году мы провели дома грандиозные перемены. Подвели в дом настоящий газ,
а не балонный, которыми пользовались раньше... Поставили газовый котёл,
поставили батареи и подключили водяное отопление в дом. А наши ВСЕ ИСТОРИЧЕСКИЕ
ПЕЧИ... убрали... Конечно было очень жалко... Но сразу освободилось много
места. И бабуля была очень довольна, что наконец-то завершилась бесконечная
эпопея с дровами. Но я до сих пор помню \enquote{дух} дров и печей... 

Хотя, конечно, с отоплением значительно стало комфортней и легче... 

Очень тяжело бабуля переживала Чернобыль. 

Ушла от нас за месяц до 90-летия в 1989 году. Прожила сложную, не лёгкую,
полную испытаний жизнь.

Все наши друзья очень помнят мою бабулю, говорят, что они с ней себя
чувствовали, как с подружкой... Очень честный, чистый, добрый, справедливый была
человек. Может порой и резковатый, но в глаза... 

Газеты всегда читала, журналы толстые, \enquote{Огонёк}. Всегда была в курсе всех
событий. Госпожу Тетчер очень уважала... Хрущёва не очень уважала... Сталина
презирала... Кстати, после смерти Сталина, осенью вернулся и наш двоюродный
дедушка. Без здоровья, но живой... 

Бабуля всеми политиками всегда интересовалась и всех мировых политиканов на тот
момент знала и всем давала свою оценку...

А сейчас ей было бы очень сложно и тяжело дать оценку всему происходящему... Но
ОНА бы дала..., однозначно резко, зло и осуждающе... Собственно, вся моя семья,
которая и происходит от самых обычных, мирных, честных, простых, трудящихся
людей, проявившаяся в 16 – 17 веке и дала повод для моего такого откровенного
«изложения»... Ведь, если разобраться...,
- КТО МЫ ВСЕ ПО СУТИ СВОЕЙ...?! ОТКУДА МЫ ВСЕ ?! 

И если Америку, в основном, осваивали авантюристы жаждущие быстрейшей
мгновенной наживы..., то в Украине всегда стремились жить мирные люди умеющие
работать, строить, созидать, выращивать хлеб, строить дома, семьи, растить
детей. НО НЕ РАЗРУШАТЬ ! Смотрите сколько умных людей в нашей стране, сколько
учёных у нас..., и сейчас тоже... Какая сейчас у нас умная и перспективная
молодёжь... Почему же их предки и их родители пустили свои корни здесь, на нашей
земле...?!, а их дети вынуждены сейчас уезжать, чтобы свои таланты открывать и
вкладывать не дома... в Украине..., а улучшать жизнь где-то, кому-то..., и при этом
быть не хозяином на своей земле..., а рабом..., но на чужой... 

Не такую хотели видеть Украину все мои предки, которые лежат на Зверинецком
кладбище. Они всегда впереди видели свет и солнце…

Они строили… Они верили... 

Очень благодарна всем, кто понял меня и услышал...

Жизнь продолжается... И строить её уже будут наши внуки...

ПУСТЬ ВСЁ У НИХ ПОЛУЧИТСЯ !

И СЛАВА УКРАИНЕ !
