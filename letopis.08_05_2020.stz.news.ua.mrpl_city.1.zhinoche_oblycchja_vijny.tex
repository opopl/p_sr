% vim: keymap=russian-jcukenwin
%%beginhead 
 
%%file 08_05_2020.stz.news.ua.mrpl_city.1.zhinoche_oblycchja_vijny
%%parent 08_05_2020
 
%%url https://mrpl.city/blogs/view/pro-zhinoche-oblichchya-vijni
 
%%author_id demidko_olga.mariupol,news.ua.mrpl_city
%%date 
 
%%tags 
%%title Про жіноче обличчя війни...
 
%%endhead 
 
\subsection{Про жіноче обличчя війни...}
\label{sec:08_05_2020.stz.news.ua.mrpl_city.1.zhinoche_oblycchja_vijny}
 
\Purl{https://mrpl.city/blogs/view/pro-zhinoche-oblichchya-vijni}
\ifcmt
 author_begin
   author_id demidko_olga.mariupol,news.ua.mrpl_city
 author_end
\fi

З весни 1945 року пройшло рівно 75 років, але Велика Перемога буде залишатися в
людських серцях, та сама Перемога, сповнена одночасно радістю і болем, щастям і
сумом. З кожним роком з нами стає все менше свідків тих подій. Проте кожна
сім'я дбайливо зберігає пам'ять про своїх Героїв – учасників війни,
солдатів-фронтовиків і трудівників тилу. Кажуть, що у війни не жіноче обличчя,
але вона торкнулася і жінок, змінивши їхні долі та життя назавжди. Пропоную
познайомитися з історіями маріупольчанок, яким довелося зрозуміти, що війна не
робить винятків. Кожна з героїнь переживала ті трагічні події по-своєму, але
всі вони є прикладом незламності духу та вражаючої сили...

Роль жінки під час війни не завжди розкривалася повною мірою. Ми можемо
дізнатися про жінку у військовій формі на передовій, про санітарку, що виносила
поранених під зливою куль. А от про жінку, яка просто чекала своїх синів або
рятувала чужих дітей, вижила або загинула під час Голокосту в радянські часи
навіть не згадувалося. Дозволялося писати лише про героїзм і жертовність... Я
хотіла б почати свою розповідь про маріупольчанку, якій дивом вдалося
врятуватися.

Як відомо, нацистські окупанти зайняли Маріуполь 8 жовтня 1941 року. Що
насправді довелося пережити маріупольцям найправдивіше розповіла в своєму
знаменитому щоденнику маріупольчанка \emph{\textbf{Сарра Глейх}}. Шкільний зошит з червоною
обкладинкою, в якому розповідається про те, як в жовтні 1941 року гітлерівці
знищили понад 30 тисяч єврейського населення Маріуполя, отримав світову
популярність. Це сталося, коли російський письменник, поет і публіцист \emph{Ілля
Еренбург} включив щоденник Сарри в свої спогади \emph{\enquote{Люди. Роки. Життя}}, перекладені
на десятки мов. 18-річна Сарра Глейх, (її навіть не поранили) зуміла вибратися
з-під тіл і пробратися до Маріуполя та розповісти про знищення єврейського
населення. Зараз складно уявити, що довелося пережити юній дівчині, яка серед
мертвих тіл розшукувала свою матір, батька, сестер, племінника... Біль, відчай і
великий сум передає кожне слово щоденника... 

\begin{quote}
\em\enquote{Дійшла черга і до нас, і вся
картина жаху і покірливої смерті постала перед нашими очима, коли ми
попрямували за сараї. Тут вже десь лежать трупи тата і мами. Відправивши їх
машиною, я скоротила їм життя на кілька годин. Нас гнали до траншей, які були
вириті для оборони міста. У цих траншеях знайшли собі смерть 9000 осіб.
єврейського населення, більше ні для чого вони не знадобилися. Нам веліли
роздягтися до сорочки, потім шукали гроші і документи і відбирали, гнали по
краю траншеї, але краю вже не було, на відстані в півкілометра траншеї були
наповнені трупами, вмираючими від ран і шукачам ще однієї кулі, якщо однієї
виявилося замало для смерті. Ми йшли по трупах. У кожній сивій жінці мені
здавалося, що я бачу маму. Один раз мені здалося, що старий з оголеним мозком –
це тато, але підійти ближче не вдалося. Ми почали прощатися, встигли
поцілуватися. Згадали Дору. Фаня не вірила, що це кінець – \enquote{невже я вже ніколи
не побачу сонця і світла}  – говорила вона, обличчя у неї синьо-сіре, а Владя
все питав: \enquote{Ми будемо купатися? Навіщо ми роздяглися? Йдемо додому, мама, тут
не добре}. Фаня взяла його на руки, йому було важко йти по слизькій глині. Фаня
обернулася і відповіла: \enquote{З ним я вмираю спокійно, знаю, що не полишаю сироту}.
Це були останні слова Фані. Більше я не могла витримати, схопилася за голову і
почала кричати якимось диким криком, мені здається, що Фаня ще встигла
обернутися і сказати: \enquote{Тихіше, Сарра, тихіше} і на цьому все обривається...}.
\end{quote}

Мабуть, Саррі судилося вижити не просто так. Вона повинна була відкрити правду
про ті трагічні події всьому світові і їй це вдалося. Хоча дівчина і розуміла,
що її порятунок за тих обставин можна вважати справжнім дивом. В Маріуполі
Сарру обігріли, дали пальто і відправили на схід, де вона врешті-решт опинилася
на радянській території.

\ii{insert.read_also.demidko.elizaveta_birjukova}
\ii{08_05_2020.stz.news.ua.mrpl_city.1.zhinoche_oblycchja_vijny.pic.1}

Під час звільнення Маріуполя мужність і відвагу проявила сан\hyp{}інструктор 695-го
стрілецького полку \emph{\textbf{Парасковія Ковальова}}. Нехтуючи смертельною небезпекою, вона
надавала допомогу пораненим і під вогнем винесла з поля бою понад сорок
солдатів і офіцерів. Сама присутність на позиціях цієї мужньої дівчини вселяла
в бійців впевненість у перемозі, і багато хто, незважаючи на поранення, не
покидали поле бою. Складно уявити зараз, через стільки років, які тяготи
довелося перенести тендітним дівочим плечам... Пригадуються рядки радянської
поетеси \emph{Юлії Друніної}:

\begin{quote}
\em\enquote{Я только раз видала рукопашный,

Раз наяву. И тисячу – во сне.

Кто говорит, что на войне не страшно,

Тот ничего не знает о войне}.
\end{quote}

За героїзм, проявлений у цих боях, старшина медичної служби Парасковія
Ковальова була нагороджена орденом Червоного Прапору.

\ii{08_05_2020.stz.news.ua.mrpl_city.1.zhinoche_oblycchja_vijny.pic.2}

Одна з найбільш яскравих актрис маріупольського театру \emph{\textbf{Людмила Радіонова}} теж
стала безпосередньою учасницею воєнних дій. Вона працювала в Маріупольському
театрі з 22 років. Цікаво, що акторський шлях почала одразу з головних ролей.
Але, коли почалася війна, талановита Люда не змогла залишатися осторонь. Після
закінчення спеціальних курсів стала санінструктором та ворошиловським стрілком.
Жінка була впевнена, що нарешті знайшла свою головну роль. Відступаючи із
зайнятого німцями Маріуполя, Людмила зустріла на Міусі передові частини 75-го
стрілецького полку 31-ї Сталінградської дивізії. Одразу прийшла до командира і
попросилася в полк. Чоловіка вразила наполегливість дівчини, яку він вирішив
перевірити на полі бою. Довго чекати не довелося. В боях біля села Миколаївка
бійці дивізії відбили кілька атак. Людмила виносила поранених до стіни будинку,
де розташовувалася санчастина. Але раптовий вибух збільшив кількість вбитих та
поранених бійців. Один німецький танк розстріляв артилерійський розрахунок,
інший крутився на неглибоких окопах, ховаючи живцем солдатів. А третій йшов до
будинку, біля стін якого лежали поранені.

\ii{08_05_2020.stz.news.ua.mrpl_city.1.zhinoche_oblycchja_vijny.pic.3}

Люда стиснула в руці наган і зробила крок назустріч танку. Два постріли влучили
прямо в оглядову щілину. Дівчина схопилася на броню. З вежі з'явилася голова в
чорній пілотці. Родіонова вистрілила ще. У відкритий люк вона пускала кулю за
кулею, поки не скінчилися патрони. З посадки виїхав другий танк зі свастикою,
вдарила кулеметна черга. Людмила рухнула, залишаючи на броні кривавий слід.

Підхопив дівчину лікар, що поспішав в операційну і вже не чув, як до Миколаївки
прийшла підмога.

Пізніше, в госпіталі, Людмила Антонівна Родіонова дізналася, що розстріляла
весь екіпаж ворожого танка. Ця затримка в русі нацистських машин допомогла
бійцям відбити атаку і врятувати поранених.

Потім був Сталінград, битва за Дніпро, визволення Одеси. Після чергового
поранення лікарі списали старшого лейтенанта медичної служби Людмилу Антонівну
в запас.

На її грудях виблискували два ордени Червоної Зірки, медаль \enquote{За відвагу}. А
орден Леніна, найперша її нагорода за подвиг на Міус-фронті, знайшла її через
32 роки після закінчення війни. Людмилі Родіоновій вручена медаль \emph{Флоренс
Найтінгейл} за самовіддані дії у Другій світовій війні, під час порятунку 179
поранених.

Назва книжки \emph{Світлани Алексієвич \enquote{У війни не жіноче обличчя...}} стала із часом
крилатим висловом. Але чи є у війни обличчя? Є біль, страждання і надія на
порятунок. Яке у Другої світової війни обличчя кожен вирішує для себе сам...
Точно відомо, що вона прийшла несподівано і миттєво, зірвала з кожного маски і
навчила цінувати життя.
