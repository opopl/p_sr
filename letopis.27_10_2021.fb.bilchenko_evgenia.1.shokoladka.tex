% vim: keymap=russian-jcukenwin
%%beginhead 
 
%%file 27_10_2021.fb.bilchenko_evgenia.1.shokoladka
%%parent 27_10_2021
 
%%url https://www.facebook.com/yevzhik/posts/4380897225278687
 
%%author_id bilchenko_evgenia
%%date 
 
%%tags bilchenko_evgenia,chelovechnost,poezia
%%title БЖ. Шоколадка
 
%%endhead 
 
\subsection{БЖ. Шоколадка}
\label{sec:27_10_2021.fb.bilchenko_evgenia.1.shokoladka}
 
\Purl{https://www.facebook.com/yevzhik/posts/4380897225278687}
\ifcmt
 author_begin
   author_id bilchenko_evgenia
 author_end
\fi

\ifcmt
  ig https://scontent-lhr8-1.xx.fbcdn.net/v/t1.6435-9/249480586_4380897161945360_6991365118546746035_n.jpg?_nc_cat=106&ccb=1-5&_nc_sid=8bfeb9&_nc_ohc=NtvUXh2qJI4AX_8BgLP&_nc_ht=scontent-lhr8-1.xx&oh=3fe56abd6f6b20061871b75679dc3d1d&oe=619ED335
  @width 0.4
  %@wrap \parpic[r]
  @wrap \InsertBoxR{0}
\fi

БЖ. Шоколадка
Много-премного пришло учёных: рассуждали о человечности.
О нашей увечности и о вечности, о времени быстротечности.
О бесконечности космоса. Всяк говорил своё,
Ибо то была конференция, а не абырвалг, ё моё.
Я тоже там умное говорила, снобствуя и цитируя.
У меня же на кухне, в самом холодном углу квартиры,
Кутаясь в мою кофту - драную, серую, шерстяную, - по горло,
Сидела Алёнка, Алёшка, Алёнушка - шоколадка моя из Горловки.
Её волосы, блеском каштановым окутывающие скулы,
Напоминали конфеты советские - "Кара-Кум".
А глаза её, карие, золочёные Тицианом,
От горя сами себя очищали, от калия будто цианистого.
Алёнушка ехала в Киев через Россию - тридцать восемь часов пути.
Её бабушка ехала столько же, чтоб для пенсии получить id.
На путь туда и обратно даже в плацкартах 
Уходило всё, что капало на проклятую эту карту.
В вагоне Алёнушка-шоколадка откровенно борзела,
Отвечая военным из Украины: "Зачем вы стреляете в нашу землю?"
У Алёнки было целых три деда, и о них я хочу отдельно
Рассказать дельно: не в пример конференции, ибо истина - неподдельна.
Первый дед её умер в сорок девять. Танкист. Живым дошёл до Берлина.
Потом он стал журавлём, вышибая клин песенный птичьим клином.
Второй дед её умер тоже в сорок девять: судьбе решать-то,
Кому и с кем совпадать: подорвал дыхалку на ртутной шахте.
А третий - самый красивый - из шахты угольной
Гарцевал Ванюшей, завидным парнем, по главной улице.
И дожил он до девяноста, сохранив при обстрелах крылья.
Похоронить Алёнке его не дали: блок-посты в тот день перекрыли.
А потом пришло много-премного учёных, и всяк испытал кАтарсис,
Рассуждая о человечности: к чему она там, блин, катится.
Я вышла из зума, будто из комы, мат загнав под щёки и веки,
И вернулась на кухню... Алё, Алён, шоколадка моя - навеки.
27 октября 2021 г.
