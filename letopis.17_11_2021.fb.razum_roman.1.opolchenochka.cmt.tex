% vim: keymap=russian-jcukenwin
%%beginhead 
 
%%file 17_11_2021.fb.razum_roman.1.opolchenochka.cmt
%%parent 17_11_2021.fb.razum_roman.1.opolchenochka
 
%%url 
 
%%author_id 
%%date 
 
%%tags 
%%title 
 
%%endhead 
\subsubsection{Коментарі}

\begin{itemize} % {
\iusr{Богдан Мл Соболев}
С музыкой как раз все в порядке, а вот с событиями... Мне знающие люди сказали что были два сценария, и практически фильм зарубили и вышло то, что вышло... Непосредственные участники событий очень разочарованы...

\iusr{Александр Фомин}
Фильм не понравился, но всё равно спасибо.

\iusr{Аделина Волкова}
А что с событиями?

\iusr{Аделина Волкова}
Все знают, что прообразом Атамана Егора являлся Павел Дрёмов. Именно его памятник появился в кадре.

\iusr{Евгений Фарахов}
Фильм- говно!!! Лучше бы эти деньги отдали на войну с раком и новой болезьнью- это фронт по страшнее 14 го!!!!

\iusr{Аделина Волкова}
Фильм замечательный. Так рассуждает только тот, кто войну видел только по телевизору.

\iusr{Аделина Волкова}
Что здесь не правда? Танк-памятник под Хрящеватым? Бои под Новосветловкой? Да, и не стоит забывать, что фильм художественный, а не документальный.
\end{itemize} % }
