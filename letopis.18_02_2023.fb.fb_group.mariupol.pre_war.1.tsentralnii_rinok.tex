%%beginhead 
 
%%file 18_02_2023.fb.fb_group.mariupol.pre_war.1.tsentralnii_rinok
%%parent 18_02_2023
 
%%url https://www.facebook.com/groups/1233789547361300/posts/1412513659488887
 
%%author_id fb_group.mariupol.pre_war,kipcharskij_viktor.mariupol
%%date 18_02_2023
 
%%tags mariupol,mariupol.pre_war,mariupol.rynok,rynok
%%title Центральний ринок
 
%%endhead 

\subsection{Центральний ринок}
\label{sec:18_02_2023.fb.fb_group.mariupol.pre_war.1.tsentralnii_rinok}
 
\Purl{https://www.facebook.com/groups/1233789547361300/posts/1412513659488887}
\ifcmt
 author_begin
   author_id fb_group.mariupol.pre_war,kipcharskij_viktor.mariupol
 author_end
\fi

Центральний ринок

\enquote{Щодня, крім понеділка та дощу...}

Центральний ринок... Колись він був Новим – після того, як закрили Старий – де
зараз височить будівля ДТСААФу... Його дах нам наводили на лекції з \enquote{Нічорта ні
поніметрії} (\enquote{Начертательной геометрии}) як приклад ламаних поверхонь.

У роки \enquote{розвитого оптимізму} (\enquote{розвитого соціалізму}) він мало чим міг нам
прислужитися – зарплатні двох інженерів вистачало хіба що на рибок подивитися,
тобто, не на \enquote{шопінг}, а на \enquote{зирінг} (позирити). Іллічівський, або колгоспний
був більш демократичним – там торгували люди, які привезли \enquote{дари городів}
електричкою і мали швиденько все розпродати, \enquote{скупитися} (щось купити до дому)
та встигнути повернутися наступною електричкою. Іноді вони починали торгувати
ще на пероні зупинки \enquote{Заводська} або \enquote{Робоча площадка}, а іноді – прямо у
електричці  

А от коли прийшли часи \enquote{економічних реформ}, ринок став годувальником.
Кмітливі люди привозили товари (переважно – овочі)  звідти, де вони були
дешевшими і продавали просто з кузовів машин. При цьому їм треба було встигнути
розпродати свій товар, доки він не почав псуватися: взимку від холоду, а влітку
– від спеки, тож ціни іноді були значно нижчими, ніж в магазинах. До того ж,
ринок був мені по дорозі на роботу і з роботи: доїхав до ринку, купив,
наприклад, кілька кілограмів картоплі (грошей на мішок картоплі могло й не
бути!) і поніс на роботу (чи додому). 

Одяг та взуття, особливо для дітей (які мали шкідливу звичку швидко рости!),
привозили \enquote{човники} здебільшого з оптових ринків у великих містах: Одеси,
Харкова. Такими товарами торгували \enquote{в рядах}. За цінами треба було слідкувати:
тільки-но поширювалися чутки про те, що на якомусь з великих заводів починають
видавати заробітну платню, ціни починали рости...

\enquote{З машин} торгували \enquote{внизу} ринку – уздовж вулиці Карла Лібкнехта (зараз -
Мітрополітської). Машини стояли передом до вулиці, а товар відпускали через
задній борт. Якось в черзі за картоплею перед мною стояв чолов’яга у
спецовці (біля ринку були ЖЗТО, трампарк та кондитерська фабрика, завод
пральних машин, тож люди у спецодягу не були чимось екзотичним). Цей, судячи
за запахом емульсії, був з ЖЗТО, а саме – з цеху, де стояли
токарно-револьверні верстати.  Один продавець під машиною зважує картоплю та
бере гроші, а інший в кузові набирає картоплю у металеве відро та ставить на
борт, тож черга рухається швидко. Підійшла черга до робочого, а він
замислився. Продавець нетерпляче:

- Скільки тобі? 

А той:

- От не знаю: проставляюсь бригаді, бо син народився, взяв оселедців (показує),
а скільки картоплі треба на 8 чоловіків?

- А горілки скільки?

- Два літри самогонки...

Продавець зазирнув у відро біля борту і кричить:

- Оцю висип та набери трохи більше піввідра з правого кутка, бо там та, що не
розвариться у кисіль, і не велику, а щоб варити у мундирах та найкращу вибери,
бо чоловік сина обмиває!

Вибрали. Зважили. Покупець:

- Скільки з мене? 

- Подарунок сину! Піди от там капусти квашеної візьми! 

- Машо!!! Підійде хлопець, дай йому тої капусти, що мене пригощала! Та знайди
банку, налий розсолу! Він банку потім принесе.

І до покупця:

- Ти ж післязавтра банку повернеш?

- Та завтра ж з ранку поверну!

- Завтра в тебе голова боліти буде, бо ти ще вдома добавиш...

Отакий був базарний сервіс 90-х років.

Трохи далі стояли машини з ковбасами ... але я намагався швидко проходити повз
них, аби від тих пахощів у голові не запаморочилося...

Якось, відчитавши першу чи другу пару, йду я на базар від трамвайного
кільця, роздивляюся на викладений з обох боків тротуару на
газетках-клейонках товар. З лівого боку \enquote{прилавки} виклали на край тротуару,
а продавці стоять на дорозі – майже в струмочку під бордюром. Тротуаром
(посередині між \enquote{торговельними рядами}) біжить собака, який несе в зубах
шматок біляша. Йому перегороджує шлях купка людей: хтось звузив прохід, бо
приміряє взуття. Собака приймає рішення перестрибнути з тротуару на дорогу
через викладений на газету \enquote{товар}. Бо на дорозі продавці стоять майже
впритул один до одного і лише в одному місці, де продавчиня виклала дві
газети, є вільне для стрибка місце. Пес обережно, не торкнувшись газети,
стрибає, але начинка під час стрибка випадає з біляша і падає на \enquote{прилавок}
серед товару. Пес на дорозі розвертається, кладе тісто і намагається забрати
м’ясну начинку. Продавчиня бачить, як пес лізе до її товару і починає
кричати і тупотіти ногами, проганяючи його. Відскочивши, пес дивиться на неї
поглядом, в якому чітко написано: \enquote{Що за справи!? Це моя здобич! Піди сама собі знайди!}. 
Друга спроба пса закінчується так само, бо продавчиня не
бачить, що серед її товару лежить його здобич. Тоді пес бере зубами тісто
від того біляша, кладе під ноги продавчині і поглядом каже: \enquote{Ну то на тобі
ще й це!} Повертається, піднімає ногу і пускає струмочок на товар на
\enquote{прилавку}...

Інша сцена починається так само: назустріч потоку людей гордо біжить собака,
який несе в зубах ... палку копченої ковбаси!!!

СОБАБА НЕСЕ ПАЛКУ КОПЧЕНОЇ КОВБАСИ!!!

Якби мені ... ні, не палку (звідки стільки грошей взяти, коли мало не щодня
робиш вибір: купити собі цигарок, чи дітям булочок \enquote{по три копійки}?) – якби
хоч трішечки, аби пригадати смак... Мабуть, не я один був такий, бо люди, що
йдуть тротуаром назустріч псу, з повагою поступаються йому дорогою!!! Ви б
бачили очі того пса: \enquote{Так, це моя здобич! А вам слабо?!}

Серед великих заводів одним з перших \enquote{захитався} ЖЗВМ (в той час вже
Азовмаш), тож ряди найманих продавців (реалізаторів) поповнили його
конструктори та технологи.  Потім Центральний Ринок на кілька років випав з
поля мого зору – чи то ми стали краще жити, чи, скоріше, через брак часу.
Але, скоріше за все, в той час ми \enquote{скуплялися} на ринку на восьмих воротах.

На початку 1990-х я почав пішки ходити на роботу – не намагаючись скоротити
витрати на транспорт (витрати на зношене взуття перекривали економію на
квитках), а щоб остаточно прийти до тями, бо в той час я щороку видавав
конспект лекцій (скромно називаючи їх книжками!). Робочий день у мене
починався десь о першій чи другій ночі і я працював до шостої в цілковитому
спокої (всі домашні сплять). Потім снідав, збирався і йшов на роботу,
прокачуючи киснем мізки (я просив диспетчерів, щоб мені ставили перші пари,
які зазвичай викладачі не любили через необхідність рано вставати та ще й
добиратися у переповненому у годину пік транспорті). Зазвичай в мене ще був
час до початку пари, тож я міг не одразу піти до універу, а зайти на ринок. 

В той час мене цікавили старі технічні довідники та підручники – в моїй
бібліотеці згодом їх стало кубометри півтора... Скільки разів я обіцяв собі
не купувати книжок, молодших за мене! І не стримувався. Час від часу я
спостерігав, як біля "мого" букініста стоїть хтось із викладачів і дістає з
валізи книжки... Це означало, що людина збирається кудись виїхати. Скоріше
за все – за кордон...

Повз мою увагу пройшло те, як і коли з центру ринка зникли металеві прилавки
і з'явилися цегляні, з широкими дахами-навісами, які захищали від сонця й
опадів не лише продавців та їх товар, а й покупців. Між ринком та вулицею
Миколаївською теж почали будувати закриті торгівельні місця, які на ніч
закривали ролетами. В продавців з'явилася можливість залишати товар на ніч
просто на прилавках, адже до цього вранці його привозили, а наприкінці
робочого дня – забирали. Власники приватних будинків навколо ринку здавали
під склади свої сараї, літні кухні, гаражі, встановлювали у городах морські
контейнери. Вранці чоловіки з тачками вивозили з тих складів на ринок
картаті \enquote{челночні} валізи та торби, підписані прізвищем продавця чи номером
торгового місця, а ввечері відвозили назад. Поява \enquote{ролетів} призвела до
занепаду такого ринку праці.

З часом ринок почав розростатися ще більше: вздовж стіни трампарку виросли
саморобні різноманітні і різнокольорові будочки-халабудочки. На іншому боці
вулиці Казанцева (не пам'ятаю, що на тому місці було раніше), з'явилася
\enquote{залізка}(\enquote{железка}), де торгували переважно інструментами, запчастинами до
них та до транспорту, будівельними матеріалами тощо. Там був такий собі
Михалич, який скуповував старі електроінструменти, відновлював їх, збираючи
з кількох один і продавав... В нього я купував дрілі, точила, планетарні
редуктори від шурупокрутів, крокові двигуни від різної оргтехніки і інші
\enquote{скарби}...

Згодом вулицю Казанцеві перекрили згори (від трамвайних рейок) і знизу (від
вулиці Мітрополітської) залізними заборами з воротами, які замикали на ніч.
Оце зараз виникло питання: як ввечері додому потрапляли мешканці кількох
приватних подвір'їв, які виходили на Казанцева?

До 2014-го року був огороджений весь ринок: коли у 2014-му році \enquote{народна
міліція} намагалася проникнути на ринок, аби \enquote{шукати правий сектор}, охорона
ринку не пустила озброєних бойовиків. Завдяки цьому робочі місця торговців
не були пограбовані.

Навколо трамвайного кільця виник \enquote{блошиний} ринок, де \enquote{з землі} торгували
переважно пенсіонери своїми домашніми скарбами і незабаром він дотягнувся до
пам'ятника Висоцькому і самого проспекту Миру. Його кілька разів намагалися
\enquote{розігнати}, але він знову і знову виникав на тому ж місці. 

Наприкінці 2010-х років на \enquote{залізці} теж почали ставити \enquote{ролети} – з їх
появою металеві прилавки вивезли на кільце, аби з них торгували пенсіонери.
Блошиний ринок на деякий час зник з Казанцева вище Соборної (Варганова), але
час від часу виникав знову, хоч вже не в такому масштабі.

На кільці торгові місці дуже швидко захопила \enquote{мафія}: ті, хто мешкав поруч,
приходили раніше і займали місця для своїх знайомих. Цим були незадоволені ті,
хто приходив пізніше і намагався стати на порожні місця. Їм пояснювали, що це
місце вже зайнято, нібито \enquote{людина от щойно відійшла за товаром} і так далі.
Торговці за багато років здружилися, спільно \enquote{перекушували}, пригощаючи один
одного принесеною з дому їжею і, чого скривати, випивали... Особливо взимку
\enquote{аби зігрітися}. Девіз та розпорядок роботи ринку, приведений на початку, я
почув від одного годинникаря: в понеділок ринок майже не працював. Майже – бо в
понеділок приходили ті, кому не було місця в інші дні. Вони ставали на кільці і
вздовж ролетів по Казанцеві аж до повороту трамвая і торгували старим одягом,
краватками, нитками для в’язання, спицями і таким іншим. Ці люди, зазвичай, не
зазивали покупців, а мовчки стояли біля свого \enquote{товару}, бо звикли, що він не
користується попитом... Та й хто ж ходить на ринок у понеділок...

На блошиному ринку поза кільцем в мене було кілька \enquote{своїх} продавці (чи то я в
них був \enquote{своїм} покупцем), серед яких я виділяв Влада та Валеру. З ними можна
було перекинутися словом-другим, почути якусь історію (Влад колись \enquote{плавав}, а
Валера був біженцем з 2014-го року...), замовити якусь потрібну мені річ... В
моїй машині вогнегасник, аптечка, та \enquote{краватка} (буксирувальна линва) та 
\enquote{прикурював} (дроти) лежать у чемодані від друкарської машинки \enquote{Оптіма} від
Влада, а від Валери в мене складаний стільчик та залишений вдома дипломат, у
якому розміщені паяльники та все, необхідне для паяння...

А на кільці стояв такий собі Гриша, який торгував слюсарним інструментом
(звичайно, б/у). В нього я якось побачив ковадло (наковальню), майстерно
зроблену з шматка залізничної рейки... Таке враження, що її робили як прикрасу
столу якогось начальника: дбайливо вирізана, відполірована до дзеркального
блиску... справжній виріб мистецтва... Кожного разу, зупинившись біля Гриші, я
любувався тим ковадлом... У січні 22-го року мені треба було здати на перевірку
лічильник води. Тим інструментом, що був у мене вдома, я не зміг відкрутити від
лічильника одну трубку, а ворота гаражу надійно \enquote{запечатав} снігом грейдер. У
водоканалі дівчата такий лічильник брати відмовилися, хоча саме їх слюсар так
затягнув ту трубку, встановлюючи лічильник. В гаражі водоканалу не виявилося
ані слюсарів, ані лещат (\enquote{оптимізація}...), тож я пішов до Гриші і провокаційно
запитав: \enquote{Закладаємося на пиво, що не відкрутиш?}. І програв! До речі, те пиво
я так і не віддав...

Якось я прочитав (здається, у Фазіля Іскандера), що на Сході люди ходять на
ринок поспілкуватися, дізнатися новини, зустріти знайомих, ну і щось купити.

Центральний ринок для мене був отаким собі місцем психотерапії та розвантаження
від проблем...

Звичайно, на ринку було \enquote{підвищене криміногенне оточенні}, але про це вже в
наступному спогаді... 

Фото з Інтернету

З різних сайтів та груп
