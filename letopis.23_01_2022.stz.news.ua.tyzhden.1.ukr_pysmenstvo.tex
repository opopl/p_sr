% vim: keymap=russian-jcukenwin
%%beginhead 
 
%%file 23_01_2022.stz.news.ua.tyzhden.1.ukr_pysmenstvo
%%parent 23_01_2022
 
%%url https://tyzhden.ua/Columns/50/254147
 
%%author_id ljutyj_taras
%%date 
 
%%tags istoria,kultura,ukraina
%%title Давнє українське письменство
 
%%endhead 
 
\subsection{Давнє українське письменство}
\label{sec:23_01_2022.stz.news.ua.tyzhden.1.ukr_pysmenstvo}
 
\Purl{https://tyzhden.ua/Columns/50/254147}
\ifcmt
 author_begin
   author_id ljutyj_taras
 author_end
\fi

Матеріал друкованого видання № 3 (739) від 19 січня

\begin{zznagolos}
Для нашого сучасника все, що діялося в Україні сто років тому, є давниною.
Водночас усякого майстра пера ми звемо класиком. А тоді ким вважати того, хто
писав іще раніше, скажімо, понад 300 років тому? Хіба нас не хвилює, які
завдання ставили собі письмовці, чим переймалися, чи мали якусь спільноту? І чи
виявляли свою приналежність до певної культурної традиції?	
\end{zznagolos}

Завжди непросто робити висновки про діла минулих літ. Та все одно, попри
великий відтинок часу, нам хочеться бачити хоча б якусь суголосність із давніми
подіями. Водночас не вдаватися до жодних інтерпретацій означає не помічати
тяглості й відмінності. Візьмімо, для прикладу, статтю Миколи Сумцова «Про
літературні звичаї південноруських письменників XVII століття» (1906). Що ми
бачимо? Зважаючи на історичні реалії, нас заледве здивують польські впливи на
тодішнє українське письменство. Якщо в усній творчості переважали думи, то в
літературному процесі вирували полемічні твори й точилися суперечки про
морально-релігійні та політичні засади життя. 

Так само, як і нас, авторів турбували питання про роль книжки. Хіба що заскочує
те, що все видане підлягало цензурі. Щодо самого письма, то дехто не цурався
простого мовлення, а хтось уважав за потрібне вживати церковнослов’янську,
книжну руську чи польську мови. Зокрема Захарія Копистенський міркує про
можливість об’єднання слов’янських народів на основі церковнослов’янської мови.
Водночас потреба перекласти Святе Письмо простою мовою визріла ще в XVІ
столітті.

Як і в останні наші роки, у тодішній Україні воєнна загроза нависала постійно.
Та й рівень національного та політичного усвідомлення не був стабільним.
Доводилося обирати між двома цивілізаційними орієнтирами: культурно вищим, але
чужим варіантом польського католицизму чи відсталим, хоч і ближчим за вірою й
сильнішим військово московським царством. Звідси деякі неоднозначності. Нині
гірко читати, як Іоаникій Ґалятовський, байдужіючи до рідного краю, переповідає
подвиги Речі Посполитої. Чи як Атанасій Кальнофойський шанує князя
Святополка-Четвертинського за приборкання запорожців. Обоє були кліриками й
обстоювали духовну владу, позаяк віра відігравала ключову роль у
самовизначенні. Ще князь Костянтин Острозький застерігав польських магнатів від
утиску православних, адже це неминуче позначилося б на громадянській свободі
всіх вірян. Отже, православне духовенство поволі здобувало авторитет у
формуванні трибу суспільного життя, що відбилося на письменстві. Але втішно,
коли Лазар Баранович в одному з листів до Інокентія Ґізеля пише про
впевненість, що малоруський народ навіть у дрібницях колись не потребуватиме
зовнішньої допомоги у розв’язанні своїх справ.

Серед застарілих нині речей доречно згадати проповіді, що виникли під впливом
шкільництва. Їх оздоблювали химерними схоластичними пасажами, щоб
продемонструвати вченість, переобтяжували казуїстикою, недоречним гумором,
улесливими надмірними тропами, іншомовними словами й іншими пустоцвітами. Тому
Теофан Прокопович зневажливо іменує казнодіїв базікалами. Але це не дивина.
Таке саме подибуємо й у поляків. Саме на цей штиб написано й чимало віршів. Та
їхні автори не завжди мали до цього схильності. Зате ті оди залюбки приймали за
щедру винагороду до своїх уродин спраглі до вихваляння московські царі,
польські можновладці та місцеві багатії.

Вочевидь, літературні запозичення були не копіюванням, а радше осмисленим
наслідуванням. Розуміння письменниками відставання від Заходу не спричиняло
розчарувань або засліпленої пихи. Навпаки, вони спромоглися об’єднатися, щоб
заснувати нові інституції та використати знання для піднесення своєї культури.
Їм поталанило організувати осередки критики й мати сміливість виносити власні
опуси на обговорення. Щоправда, дискутувати наважувалися лише компетентні
особи. Диспути бували непростими й іноді виходили за межі академічності. Інакше
до нас не дійшло б те, що Кирило Терлецький привіз із Рима непристойні
картинки, Стефан Зизаній не знав, ким був його батько, а Касіян Сакович не мав
одного вуха, бо в дитинстві його відкусила свиня. Відомо й те, що Ґізель і
Симеон Полоцький вдавалися до нестримних висловлювань. Останній у звинуваченні
опонента міг запропонувати обезголовити його й віддати рештки звірям. Одначе
деякі автори, як-от Баранович, закликали до спокою, дозволяючи собі лише злегка
шпиняти візаві.

Хай там як, а найбільш властивою письменницькою діяльністю була просвіта.
Сумцов наголошує, що головною рисою південноруських літераторів є демократизм.
Але чи не відчутне тут народництво самого дослідника? Безперечно, чуйне
ставлення до простолюду притаманне майже кожній освіченій людині. Баранович,
який засуджує здирництво щодо селян і суперечки за монастирські маєтності,
викликає повагу. Проте лояльності до посполитих із цього не виснувати.
Натомість на увагу заслуговують свідчення про духовну спільність і взаємну
підтримку письменників. А особливо — вміння відстоювати питомі заведення на
своїй землі, прагнення змінювати їх і увиразнювати самобутність, порівняно зі
звичаями сусідів. 
