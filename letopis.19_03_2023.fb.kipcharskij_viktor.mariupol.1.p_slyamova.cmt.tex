% vim: keymap=russian-jcukenwin
%%beginhead 
 
%%file 19_03_2023.fb.kipcharskij_viktor.mariupol.1.p_slyamova.cmt
%%parent 19_03_2023.fb.kipcharskij_viktor.mariupol.1.p_slyamova
 
%%url 
 
%%author_id 
%%date 
 
%%tags 
%%title 
 
%%endhead 

\qqSecCmt

\iusr{Елена Девина}

Дякуйте Богові, або вищим силам за те, що вашу родину, будинок, сусідів оминула
біда! Подяка ніколи не буває зайвою!

\begin{itemize} % {
\iusr{Віктор Кіпчарський}
\textbf{Елена Девина} 

11.03.22 у нашому дворі я сказав, що оскільки свічки нам потрібні для
освітлення і обігріву, то я вранці замість свічки буду запалювати сигарету і
молитися.

Кожного ранку, коли маю певність що зможу повернутися на п'ятий поверх, у будь
яку погоду я виходжу на вулицю, запалюю сигарету (ту ж саму марку - Captain
Black Cherry), дякую за наш порятунок і молюся:

\begin{itemize}
  \item - за звільнення з полону друзів і усіх полонених (звільнено двох; один ще досі у полоні);
  \item - за зцілення поранених (двоє повернулися на фронт, двоє звільнених з полону ще лікуються);
  \item - за збереження життя захисників (нажаль, один загинув);
  \item - за упокій вбитих та померлих (нажаль, цей список росте...).
\end{itemize}

В так вже один рік, один тиждень і один день.

Я вже не кажу про те, коли ми проходимо повз церкву, костьол чи бодай капличку.

\iusr{Roy Evans}
\textbf{Віктор Кіпчарський} 

є серіал, мабуть, більше для молоді, з Духовним, але там цікава назва книги
письменника \enquote{Бог ненавидить віх нас}. І от це, мабуть більш доречне, бо
насправді, неможливо таке, щоб той самий Бог дозволив стільком дітям загинути.
І ще, інший серіал, який має привернути вашу увагу, The Fringe його назва, то
там кажуть, що є таке прислів'я: будь краще, ніж твій батько. Там говорять, що
це грецьке прислів'я, але я не вірю, та й це не так важливо. Для кожного з нас,
батько найкращій, а бути краще ніж найкращій, або хочаб йти до того - це топ.
Тому, будемо кращі з кращих. Цвітом нації. Багато полегло, багато віддали
здоров'я та життів. Ми не можемо дозволити собі жити не гідно. Слава нації!

\iusr{Віктор Кіпчарський}
\textbf{Roy Evans} 

Як кажуть філософи: \enquote{Перш ніж сперечатися, треба визначити терміни}. Тож, що є
Бог? В кожній релігії Він свій. В дитинстві у мене була стара подерта книжка
індійських казок чи притч - без обкладинки, без початку і кінця. Мені
запам'яталася притча про сліпих, кожен з яких помацав слона: один за вухо, інші
за живіт, хобот, ногу.

Потім вони ділилися враженнями: слон - як лист лотоса, сказав перший. Ні, він
цистерна, сказав другий. Він змія, сказав третій. Колона - запевняв четвертий.
І вони побилися.

Друге моє переконання, Бог (доля, карма, природа і т.д.) не дозволяє і не
забороняє: хіба щось забороняють закони природи?

Є ще одна притча, більш сучасна: чому Бог дозволяє існувати злу? Є світло і є
темрява. Що є світло? Сукупність електромагнітних хвиль видимого діапазону. А
що є темрява? Відсутність світла. Є Добро. А зло - це відсутність Добра.

Згідно закону природи, вода тече з гори додолу: біля берегів течія повільніша,
посередині ріки - швидша; є спокійні заводі і є вири, перекати і водоспади.
Хтось, або щось може опинитися у заводі, або у вирі. І не завжди з власної
волі.

Зерня, що падає з рослини, може впасти на землю і прорости, а може впасти на
камінь, або у воду. Все це за законами природи. І нема сенсу питати: чому
природа дозволяє гинути зерну.

Людей і дітей вбивають інші люди. І зупиняти і карати їх мають теж люди, бо
безкарність породжує зло.

А релігія (одна з релігій) каже не лише \enquote{Не вбий!}, а й \enquote{Око за око}.

Якось так.

Я нічого не проповідую: лише намагаюсь зрозуміти: чому наш двір став
\enquote{оазою тиші}. І чому загинув мій Друг, який казав: \enquote{Треба, щоб
людям було добре}. І завжди робив людям добро.

\iusr{Елена Девина}
\textbf{Віктор Кіпчарський} з Вами цікаво! Продовжуйте свої спогади, роздуми... А може вже час писати книгу?

\iusr{Віктор Кіпчарський}
\textbf{Елена Девина} 

На все свій час: є час збирати події і є час писати спогади...

Якщо що - то це я вигадав.

Колись я робив доповідь про класифікацію способів поверхневою обробки деталей.
І от один слухав сказав, що класифікацією мають займатися поважні старі вчені,
якім же не здатні вигадувати нове.

Так от, роздумами мають займатися поважні люди. А мене час від часу ще тягне побешкетувати.

Пару років тому я пускав дрон Emotion - були такі емоції, коли він залетів у
Колибу (біля Катеру) і ч поліз його доставати через паркан. З моєю вагою мені
було простіше зламати той паркан, а але я переліз. Двічі.

А інший сів на воду - і я поплив його діставати...

Як казав син мого друга (чи друг мого сина?), коли його питали, чому він не
одружується: \enquote{Я ще неготовий ділитися своїми іграшками}.

\iusr{Елена Девина}
\textbf{Віктор Кіпчарський} 

Про книгу-просто виникла на думці така порада. Вважайте це, як відчуття читача
(що теж цінно, я вважаю). Коли приходе на думку вірш, мелодія, або пісня, автор не
роздумує, чи треба слухати когось. Воно з'являється, і все! Тому, вже те, що
записували, ділитесь - це і є, мабуть, Ваша \enquote{книга}.

\end{itemize} % }

\iusr{Алла Коломієць}

\ifcmt
  igc https://i.paste.pics/6efbc6b1c74330d689afebc198f528e2.png
	@width 0.1
\fi

\iusr{Светлана Водзянская-Живогляд}

Коли ми виїздили, я взяла молитвослов, то на багатьох блок постах ми з Сергійом
його читали перед проходженням, просто відкривали в довільному порядку і
починали читати звідки око чіплялося, поки чоловік мене тримав на Бердянському
посту, Сергій при світлі фар поруч читав молитви. Дякуйте своєму янголу
охоронцю! Він добре постарався виконуючи божу волю у Вашому порятунку
