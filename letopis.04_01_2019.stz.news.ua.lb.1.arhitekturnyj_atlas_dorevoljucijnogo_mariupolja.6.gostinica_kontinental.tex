% vim: keymap=russian-jcukenwin
%%beginhead 
 
%%file 04_01_2019.stz.news.ua.lb.1.arhitekturnyj_atlas_dorevoljucijnogo_mariupolja.6.gostinica_kontinental
%%parent 04_01_2019.stz.news.ua.lb.1.arhitekturnyj_atlas_dorevoljucijnogo_mariupolja
 
%%url 
 
%%author_id 
%%date 
 
%%tags 
%%title 
 
%%endhead 

\subsubsection{Гостиница \enquote{Континенталь}}

Один из самых ценных архитектурно-исторических памятников дореволюционного
Мариуполя – гостиница \enquote{Континенталь}. Трёхэтажное изящное здание кирпичной
постройки, неизменно привлекает взгляды всех, кто проходит по проспекту Мира. В
начале прошлого века оно принадлежало купеческой семье итальянского
происхождения по фамилии Томазо. Ни имя архитектора, ни точная дата постройки
здания неизвестна, что в прочем, характерно для большинства мариупольских
памяток. Предположительно основной трехэтажный корпус для отеля на 30 номеров,
возведен в период между 1895-1898 годами, а концертный зал-ресторан заработал
под новый 1911 год.

\ii{04_01_2019.stz.news.ua.lb.1.arhitekturnyj_atlas_dorevoljucijnogo_mariupolja.6.gostinica_kontinental.pic.1}

Гостиница \enquote{Континенталь} на протяжении всего своего существования была самым
роскошным местом дореволюционного Мариуполя, где городская элита могла
проводить свой досуг. В советский период здание использовали в сугубо
практических целях. В январе 1918 года в нём заседал временный исполнительный
комитет, и большевики брали гостиницу штурмом. В 1920 году в дом Томазо
заселился штаб Начальника морских сил Черного и Азовского морей, а в зале
ресторана заработала одна из первых советских столовых. Во времена НЭПа на
первом этаже находился магазин Центрального рабочего кооператива (ЦЕРАБКООП). В
тридцатых годах, когда началось строительство завода \enquote{Азовсталь}, здание
передали в распоряжение этого предприятия.

\ii{04_01_2019.stz.news.ua.lb.1.arhitekturnyj_atlas_dorevoljucijnogo_mariupolja.6.gostinica_kontinental.pic.2}

С 1933 года его занимал клуб металлургов, позже получивший название ДК
\enquote{Азовсталь}. Во время войны, видимо, навсегда исчезли решетчатые балконы, и
шатровые завершения крыши в центральной композиции здания, которые хорошо
заметны на старых открытках. В постсоветский период здание перешло из
ведомственной в коммунальную собственность и получило название ДК \enquote{Молодежный},
а совсем скоро должно стать центром современного искусства. Как ни странно, но
именно принадлежность к городскому бюджету залог благополучного существования
дома Томазо. Для частной собственности такой \enquote{дворец} был бы слишком
обременителен. Мариупольские миллионеры уже не те, что прежде.

\ii{04_01_2019.stz.news.ua.lb.1.arhitekturnyj_atlas_dorevoljucijnogo_mariupolja.6.gostinica_kontinental.pic.3}
\ii{04_01_2019.stz.news.ua.lb.1.arhitekturnyj_atlas_dorevoljucijnogo_mariupolja.6.gostinica_kontinental.pic.4}
\ii{04_01_2019.stz.news.ua.lb.1.arhitekturnyj_atlas_dorevoljucijnogo_mariupolja.6.gostinica_kontinental.pic.5}
