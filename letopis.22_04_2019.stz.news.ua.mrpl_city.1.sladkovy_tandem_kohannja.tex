% vim: keymap=russian-jcukenwin
%%beginhead 
 
%%file 22_04_2019.stz.news.ua.mrpl_city.1.sladkovy_tandem_kohannja
%%parent 22_04_2019
 
%%url https://mrpl.city/blogs/view/mariupoltsi-mariya-i-oleksandr-sladkovi-tandem-stvorenij-v-kohanni
 
%%author_id demidko_olga.mariupol,news.ua.mrpl_city
%%date 
 
%%tags 
%%title Маріупольці Марія і Олександр Сладкови: тандем, створений в коханні
 
%%endhead 
 
\subsection{Маріупольці Марія і Олександр Сладкови: тандем, створений в коханні}
\label{sec:22_04_2019.stz.news.ua.mrpl_city.1.sladkovy_tandem_kohannja}
 
\Purl{https://mrpl.city/blogs/view/mariupoltsi-mariya-i-oleksandr-sladkovi-tandem-stvorenij-v-kohanni}
\ifcmt
 author_begin
   author_id demidko_olga.mariupol,news.ua.mrpl_city
 author_end
\fi

\ii{22_04_2019.stz.news.ua.mrpl_city.1.sladkovy_tandem_kohannja.pic.1}

Продовжуючи знайомитися з видатними маріупольцями, пропоную відкрити для себе
одразу двох героїв, чия історія доводить, що союзи, створені на небесах, дійсно
існують. Вони надихають один одного, їх поєднали близькість інтересів, любов до
Батьківщини, щасливі обличчя маріупольських глядачів і справжнє кохання.
Сьогодні \textbf{Олександр і Марія Сладкови} є найбільш творчим подружжям Маріуполя, яке
завдяки енергійній діяльності надихають маріупольців на нові звершення.

\ii{insert.read_also.demidko.ukrainceva}

Марія народилася у місті Маріуполь (тоді м. Жданов), а Олександр – у місті
Дебальцево Донецької області. Вони познайомилися ще у студентські роки, під час
навчання у Жданівському металургійному інституті (зараз – ПДТУ). Марія
навчалась на зварювальному факультеті, а Олександр – на енергетичному. На той
момент Олександр займався у студклубі, грав на бас-гітарі у студентському гурті
і був керівником та ді-джеєм студентської дискотеки \enquote{Карусєль}. Потім додалися
студентські кафе \enquote{Меотіда} і \enquote{Орбіта}, де він брав участь як ді-джей. Марія
керувала агітбригадою, де писала сценарії для різноманітних програм. І ось, на
одному із вечорів вони зустрілися. Але після заходу Олександр більше її не
побачив. Знав тільки ім'я і факультет. Довго думав, як її знайти - і придумав.
Він купив цукерки дівчатам, які працювали у студентському медпункті, вони
виписали йому прізвища всіх Марій зварювального факультету, потім він взяв
телефонний довідник і став шукати ці прізвища і телефонувати. Отак Олександр її
знову знайшов і вони почали зустрічатися.

\ii{22_04_2019.stz.news.ua.mrpl_city.1.sladkovy_tandem_kohannja.pic.2}

Освіта Олександра – інженер-теплоенергетик – не завадила йому почати писати
музику. Одного разу його спитала знайома, чи зможе він зробити аранжування. На
той момент він награвав для себе на гітарі і не мав поняття, як це робити, але
сів за комп'ютер і став шукати в інтернеті, які є програми. Згадав теорію
музики, адже раніше він займався у музикальній студії по класу гітари. Так
поступово Олександр почав писати музику. Сьогодні Олександр весь вільний час
присвячує вивченню нових технологій у шоу-бізнесі. Це і написання музики,
вивчення пісень різноманітних музичних напрямів, пошук нових технічних прийомів
і технологій. Він часто говорив Марії, що дуже хотів би, щоб вона почала писати
вірші і пісенні тексти. Мабуть, чоловік відчував хист коханої дружини краще за
неї. Отже, у 2014 році Марія все ж таки почала писати вірші. Її поезію
відрізняє чистота, благородство, м'який ліризм та неабияка щирість. У 2016 році
вірші Марії увійшли до двох збірок, одна з них – це поетичний альманах, якій
був надрукований у Москві на підтримку України. На майбутнє у планах в поетеси
випуск збірки віршів і пісень.

У ПК \enquote{Молодіжний} Олександр почав працювати у 2014 році, до цього він працював
звукорежисером у караоке-барі і вже мав досвід звукорежисури. Він запропонував
тогочасному директору Дмитру Візенкову створити на базі ПК свою студію
звукозапису і клуб, який би розвивав сучасну музику на базі пісень Марії і
Олександра. Робота дуже подобається Олександру. Наразі, крім звукорежисури і
ді-джеінга, він вивчає основи ві-джеінга, відеомонтажу і світлорежисури, тому
що на останніх авторських проектах всі ці моменти він розробляв сам.

\ii{insert.read_also.dkm.berkova}

\ii{22_04_2019.stz.news.ua.mrpl_city.1.sladkovy_tandem_kohannja.pic.3}

Марія стала працювати у ПК \enquote{Молодіжний} на рік пізніше, з 2015 року. Вона пише
у різних напрямах – це вірші, гуморески, проза, пісенні тексти, а також
сценарії, як в прозі, так і у віршах.

Сьогодні у творчої пари вже є низка реалізованих проектів. Вони використовують
тільки свій авторський матеріал, а це: сценарії, вірші, пісні з аранжуванням і
потім, з готовою фонограмою, розробка відео і створення готового відеоконтенту,
створення світового рішення. Все це вони роблять вдвох. У 2018 році на день
міста Олександр і Марія показали програму \textbf{\enquote{Маріуполь-240. Перезавантаження
\#ТутВарто бути щасливими}}. Для цього проекту вони написали нові пісні, вірші і
прозу. Проект видався дуже цікавим, тому що подібного у місті ще не бачили. І
як завжди, для цього проекту вони написали фінальну пісню, тому що це стало вже
їхньою традицією.

Проекти подружжя вражають самобутністю та оригінальністю рішення. Можливо, саме
тому вони надихають маріупольську молодь. Зокрема, після проекту \enquote{Мрії Тараса},
який вони показали у ПК \enquote{Український дім} 13 березня цього року, маріупольські
учні захотіли написати твір про мрії Тараса Шевченка. Марія і Олександр
підтримали цю ідею і приготували подарунки найкращому твору.

Незабаром у пари стартує новий проект: це їхні філософські творчі роздуми під
назвою \textbf{\enquote{Да, я такой или откровенный разговор}}.  Далі готуються різноманітні
проекти з іншими колективами, а також підготовка і випуск поетичної збірки.

Найбільше Олександр і Марія люблять море, захід сонця і тишу. Їх завжди
підтримують сім'я та друзі. Зокрема, син завжди цікавиться новими творами
батьків. Вражає, що в цьому союзі все по-справжньому гармонійно: вони надихають
один одного, підтримують у всьому.

\textbf{Читайте також:} \emph{Гранд-опера и балет кранов: \enquote{Мариупольское телевидение} покажет изюминку \enquote{ГогольFest}}%
\footnote{Гранд-опера и балет кранов: \enquote{Мариупольское телевидение} покажет изюминку \enquote{ГогольFest}, Кіра Булгакова, mrpl.city, 22.04.2019, %
\par\url{https://mrpl.city/news/view/grand-opera-i-balet-kranov-mariupolskoe-televidenie-pokazhet-izyuminku-gogolfest}}

\textbf{Порада маріупольцям:} \enquote{Будьте завжди цікавими для оточуючих вас людей, випромінюйте позитив і тоді навколо вас буде все змінюватися на краще}.

\textbf{Курйозний випадок з життя:} 

\begin{quote}
\em\enquote{У нас курйозний випадок виникає після кожного
нашого проекту. А почалося це з проекту \enquote{Біле і чорне}, де поряд з творами
В. Висоцького лунали вірші і проза Марії, а його пісні лунали з моїм
аранжуванням. До нас підходили глядачі і задавали питання:  де ми знайшли ці
твори В.Висоцького. А коли ми казали, що це наш з Марією продукт, то вони не
вірили. Деякі, навіть після цього, шукали в інтернеті ці твори, не знаходили і
знову приходили до нас і задавали те ж питання. А коли у своїх проектах ми
повністю використовували тільки свій матеріал, то до питання \enquote{Хто автор
віршів?} додалося питання \enquote{Хто автор пісень?}. Такі курйози нас ще більше
надихають та спонукають на нові творчі досягнення}.
\end{quote}

\ii{22_04_2019.stz.news.ua.mrpl_city.1.sladkovy_tandem_kohannja.pic.4}
