% vim: keymap=russian-jcukenwin
%%beginhead 
 
%%file 30_10_2020.news.ua.strana.2.maidan_dtp
%%parent 30_10_2020
 
%%url https://strana.ua/news/298147-avarija-na-kreshchatike-sehodnja-vse-podrobnosti-chto-izvestno-o-vinovnike-dtp-na-majdane-jurii-nazarenko.html
%%author 
%%tags 
%%title 
 
%%endhead 

\subsection{Смертельный таран на Майдане. Как бывший чиновник устроил кровавое ДТП в центре Киева}

\Purl{https://strana.ua/news/298147-avarija-na-kreshchatike-sehodnja-vse-podrobnosti-chto-izvestno-o-vinovnike-dtp-na-majdane-jurii-nazarenko.html}
\Pauthor{Венк, Виктория}

\ifcmt
img_begin 
  url https://strana.ua/img/article/2981/47_main.jpeg
  caption ДТП на Майдане 30 октября и Юрий Назаренко, который был за рулем. Коллаж "Страны" 
  width 0.7
img_end
\fi

Сегодня, 30 октября, в Киеве случилось ужасное ДТП. В результате аварии погибли
два человека, трое в больнице.

Трагическое происшествие произошло в самом центре столицы - на пешеходной и
проезжей части Майдана Незалежности и Крещатика.

Внедорожник пролетел из улицы Институтской через толпу людей. На кадрах не
видно, что водитель пытался тормозить. Остановилась машина лишь о плотный поток
авто, которые тянулись в пробке. По ходу движения Range Rover наехал на людей.

По нашим данным, виновником ДТП стал 66-летней Юрий Назаренко, который до
выхода на пенсию был специалистом по ценным бумагам и фондовому рынку. И служил
чиновником во времена Ющенко и Януковича.

"Страна" собрала все, что известно об этой аварии и личности водителя. Мы
публикуем фото и видео с места событий.

Внимание! В этом материале публикуются кадры 18+. 

\subsubsection{Что известно об аварии}

Страшная авария произошла в Киеве во второй половине дня пятницы.

На Майдане гуляют киевляне и гости города. Туристов призывают сделать фото
ростовые куклы. Людей много - кто-то идет буквально через каждый пару метров. 

Внезапно со стороны улицы Институтской на солидной скорости едет внедорожник
Range Rover.

Водитель не пытается тормозить. Он пролетает через саму площадь, по которой
идут люди. Некоторые успевают разбежаться. Те, кто стоял на остановке и кому
авто двигалось со спины - сбиты. 

Известно, на месте скончались два человека. Еще троих увезла скорая.
Остановился внедорожник лишь о машины, которые медленно ехали в пробке на
Крещатике, путем взятия их на таран. Всего разбито пять автомобилей. 

Нацполиция опубликовано полное видео жуткого ДТП на Майдане. Из него видно, как
машина попала на площадь.
\url{https://t.me/stranaua/9006}

\ifcmt
img_begin 
  url https://strana.ua/img/forall/u/10/91/2020-10-30_18.47_.13_.jpg
  caption Последствия аварии на Майдане. Четыре фото: DTP KIEV CHAT
  width 0.7
img_end
\fi

\ifcmt
pic https://strana.ua/img/forall/u/10/91/2020-10-30_18.47_.27_.jpg
pic https://strana.ua/img/forall/u/10/91/2020-10-30_18.47_.32_.jpg
pic https://strana.ua/img/forall/u/10/91/2020-10-30_18.47_.38_.jpg
\fi

Одна из погибших - женщина, второй человек накрыт пледом. Машина ехала быстро,
в связи с этим травмы от столкновения оказались несовместимыми с жизнью. 

\ifcmt
pic https://strana.ua/img/forall/u/10/91/2020-10-30_18.47_.48_.jpg
\fi

В Киеве уже начинало смеркаться, но врачи скорой продолжали оказывать помощь и
госпитализировать пострадавших. Движение в центре столицы в месте аварии
ограничили. 

\ifcmt
pic https://strana.ua/img/forall/u/10/91/2020-10-30_18.48_.05_.jpg
pic https://strana.ua/img/forall/u/10/91/2020-10-30_18.48_.10_.jpg
\fi

Сообщение об аварии опубликовали на странице в Facebook сотрудники Нацполиции
Украины.

По сути силовики указывают: "Предварительно установлено, что водитель иномарки
не справился с управлением и въехал в остановку общественного транспорта, где
находились люди"

\ifcmt
pic https://strana.ua/img/forall/u/10/91/%D0%A1%D0%BD%D0%B8%D0%BC%D0%BE%D0%BA_%D1%8D%D0%BA%D1%80%D0%B0%D0%BD%D0%B0_2020-10-30_%D0%B2_18.22_.26_.png
\fi

"Страна" поговорила с очевидцами аварии. "Автомобиль выехал на пешеходную зону
и по ступенькам вниз... Люди разбегались. Скорость была за 40 км/ч точно", -
сказал нам мужчина.

"Сначала подумал салют, перестрелка какая-то... Я пришел, люди стоят, подумал
митинг какой-то, а это авария. Девушка лежала. Там рядом бабушка торговала и ее
машина сбила", - рассказал другой очевидец.

\subsubsection{Что было с водителем?}

Полиция сообщала, что проверяет водителя на предмет опьянения. А очевидцы
утверждали, что из машины выпала пустая бутылка вина. 

Однако измеритель не показал у водителя наличие алкоголя. Судя по видео, прибор
показывает "по нулям".

\begin{center}
  \begin{fminipage}{0.7\textwidth}
Рассказ очевидцев трагедии. Люди говорят, что из машины выпала пустая бутылка
вина. И также говорят, что измеритель не показал у водителя наличие
алкоголя.
\url{t.me/stranaua/9003}
  \end{fminipage}
\end{center}

СМИ же сообщают, что водитель Range Rover не справился с управлением из-за
того, что потерял сознание. Причиной ухудшения его здоровья называется недавняя
операция на сердце (стентирование). 

На видео выше заметно, что на шее у водителя пластырь. 

\begin{center}
  \begin{fminipage}{0.7\textwidth}
Появилось видео замера алкоголя у водителя, который сбил на Майдане насмерть
двух людей. Прибор показывает «по нулям».
\url{t.me/stranaua/9002}
  \end{fminipage}
\end{center}

\subsubsection{Юрий Назаренко - что известно о водителе Range Rover}

Водителю, который наехал на толпу людей, 66 лет. По данным "Страны", за рулем
автомобиля был Юрий Николаевич Назаренко 1954 года рождения.

При президентстве Ющенко и Януковича он несколько лет работал на руководящих
постах в Национальной комиссии по ценным бумагам и фондовому рынку.
  
\ifcmt
img_begin 
  url https://strana.ua/img/forall/u/0/36/image_2020-10-30_175241.png
  caption Юрий Назаренко
  width 0.7
img_end
\fi  

Он занимал занимал должность Председателя коллегии Агентства по развитию
инфраструктуры фондового рынка Украины (АРИФРУ), а в июле 2009 года был
назначен членом Национальной комиссии по регулированию рынков финансовых услуг
(Нацкомфинуслуг) бывшим главой комиссии Сергеем Петрашко.

В 2014-м году подпадал под увольнение в рамках закона "Об очищении власти" (о
люстрации), согласно которому люстрированные чиновники в течение 10 лет не
могут занимать высокие государственные посты. Однако успел подать заявление на
увольнение еще до вступления в силу люстрационного закона.

20 октября 2014 года президент Порошенко подписал приказ об увольнении
Назаренко в связи с выходом на пенсию. Таким образом, принудительного
увольнения в ходе люстрации бывшему члену регулятора удалось избежать.
