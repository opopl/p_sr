% vim: keymap=russian-jcukenwin
%%beginhead 
 
%%file 10_04_2022.fb.kirychenko_svitlana.harkiv.1.post_vdjachnosti
%%parent 10_04_2022
 
%%url https://www.facebook.com/SvitlanaKirychenko/posts/1738039639874567
 
%%author_id kirychenko_svitlana.harkiv
%%date 
 
%%tags 
%%title Пост вдячності
 
%%endhead 
 
\subsection{Пост вдячності}
\label{sec:10_04_2022.fb.kirychenko_svitlana.harkiv.1.post_vdjachnosti}
 
\Purl{https://www.facebook.com/SvitlanaKirychenko/posts/1738039639874567}
\ifcmt
 author_begin
   author_id kirychenko_svitlana.harkiv
 author_end
\fi

Пост вдячності.

@igg{fbicon.flag.ukraina} @igg{fbicon.hands.pray}{repeat=2} 

Через два тижні після початку війни, 14 березня, вчителі ліцею почали
онлайн-консультації для учнів Харківського ліцею № 89. З підвалу, з метро, з
квартири у Харкові (бо вже не було сил ходити в укриття), з хатинки на
Харківщині, з різних куточків України та з-за кордону.

Так, не мали ані напрацьованих матеріалів, ані підручників, ані якісного
Інтернету, але мали в серці  @igg{fbicon.heart.sparkling}  велике бажання спілкуватися із найдорожчими у
світі, мали велике бажання бути корисними, бажання допомогти, долучитися,
підтримати.

\ifcmt
  ig https://scontent-mxp1-1.xx.fbcdn.net/v/t39.30808-6/277808569_1738039386541259_8526676345726492592_n.jpg?_nc_cat=100&ccb=1-5&_nc_sid=8bfeb9&_nc_ohc=fS-U0W9gERAAX8NztJx&_nc_ht=scontent-mxp1-1.xx&oh=00_AT9dGeL5IhyIO_2WmD54zHb_595yYmy3kLGCGgq6jnnM5A&oe=6269C52A
  @wrap center
  @width 0.8
\fi

Точно знаю, що деякі вчителі поєднали викладання з іншою, дуже важливою задля
нашої перемоги, діяльністю  @igg{fbicon.hands.pray} 

У перший тиждень нас, учителів, було досить небагато, але ми змогли охопити
навчанням  @igg{fbicon.woman}  @igg{fbicon.school}  всі класи нашого улюбленого ліцею.

Потім нас стало більше, потім ще більше  @igg{fbicon.hands.raising} 

Чітко пам’ятаю свою підготовку до уроків 14 березня. Відчуття, як перед першим
уроком у своєму житті. Все з нуля, по пам’яті. Цілу ніч. Хвилювання неймовірне
 @igg{fbicon.person.gesturing.ok}  Розумію, що приблизно те ж саме відчували і колеги.

І ось він: перший урок, 14 березня. Попросила дітей включити камери, щоб
побачити, щоб упевнитись, що живі, що у безпеці  @igg{fbicon.face.smiling.halo}
Очікувано не стримали сліз  @igg{fbicon.heart.with.arrow} 

Перше запитання від дітей, скоріш все ж не запитання, а ствердження:

«Ми скоро переможемо?!»

Моя відповідь: «Безумовно!!!!!», в яку я вложила все серце, всю душу, всю
любов, усю віру  @igg{fbicon.hands.pray} 

А далі уроки, уроки, уроки.

Кожного дня.

Минулого тижня з різних шкіл Харкова до уроків, які проводять учителі ліцею,
приєднувалося вже більше тисячі дітей, які зараз знаходяться майже в усіх
куточках не тільки України, а й світу  @igg{fbicon.globe.europe.afrika}  І всі діти мріють про скоріше
повернення до своїх, таких рідних, шкіл!

Я безмежно вдячна вчителям ліцею, всім учителям України @igg{fbicon.flag.ukraina}, за те, що вони,
незважаючи на умови, якісно роблять свою справу. Роблять те, що вміють, те, що
люблять. І, на мою думку, це, безумовно, наближає нас до нашої перемоги!

Я безмежно вдячна дітям, які мріють про мир, які чітко розуміють, що саме вони
будуть відновлювати своє місто, свою Україну, що саме вони - майбутнє нашої
країни. І це розуміння, як ніщо інше, мотивує наших дітей, підтримує їх, дає їм
силу та віру! 

Дякую!!!! 

\#Харьков \#городгерой \#містогерой \#Україна \#нетвойне \#дети \#діти \#вчителі

\ii{10_04_2022.fb.kirychenko_svitlana.harkiv.1.post_vdjachnosti.cmt}
