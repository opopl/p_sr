% vim: keymap=russian-jcukenwin
%%beginhead 
 
%%file 10_11_2018.stz.news.ua.mrpl_city.1.kino_nashej_junosti
%%parent 10_11_2018
 
%%url https://mrpl.city/blogs/view/kino-nashej-yunosti
 
%%author_id burov_sergij.mariupol,news.ua.mrpl_city
%%date 
 
%%tags 
%%title Кино нашей юности
 
%%endhead 
 
\subsection{Кино нашей юности}
\label{sec:10_11_2018.stz.news.ua.mrpl_city.1.kino_nashej_junosti}
 
\Purl{https://mrpl.city/blogs/view/kino-nashej-yunosti}
\ifcmt
 author_begin
   author_id burov_sergij.mariupol,news.ua.mrpl_city
 author_end
\fi

\ii{10_11_2018.stz.news.ua.mrpl_city.1.kino_nashej_junosti.pic.1}

Когда случайно оказываешься перед закрытыми кинотеатрами им. Шевченко и
\enquote{Юбилейный} в Ильичевском районе или \enquote{Буревестник} в Приморском районе,
охватывают чувства противоречивые. С одной стороны, понимаешь, что этим
учреждениям в том виде, в котором они находились в последние десятилетия XX
века, невозможно тягаться с современным цветным, теперь еще и цифровым
телевидением, а тем более с Интернетом. А с другой стороны - щемящее сознание
утраты чего-то близкого и дорогого из молодости, в которой были семейные походы
в кино, где играли оркестрики перед сеансом, где на стенах фойе были развешаны
портреты любимых артистов, где в буфетах был нехитрый ассортимент сладостей и
напитков...

\ii{10_11_2018.stz.news.ua.mrpl_city.1.kino_nashej_junosti.pic.2}

Что тогда в первую очередь занимало умы и сердца взрослых и юных мариупольцев
из репертуара кинематографа пятидесятых-шестидесятых годов? Взрослые,
пережившие ужасы тяжелейшей войны, принесшей горе в каждый дом и в каждую
семью, предпочитали смотреть довоенные фильмы. Пусть их копии были донельзя
изношенными, пусть сеансы часто прерывались из-за порывов киноленты, и
приходилось сидеть в кромешной тьме, пока киномеханик в своей будке склеивал
ленту. Они все терпели, ведь фильмы, полные оптимизма, музыки, счастливо
завершавшиеся, были частью полной надежд жизни до страшного дня - 22 июня 1941
года.

\textbf{Читайте также:} 

\href{https://archive.org/details/22_06_2018.sergij_burov.mrpl_city.mariupol_22_iunja_1941_goda}{Мариуполь: 22 июня 1941 года, Сергей Буров, mrpl.city, 22.06.2018}

Они от души хохотали, когда смотрели комедии Григория Александрова \enquote{Веселые
ребята}, \enquote{Волга-Волга}, \enquote{Цирк}, \enquote{Светлый путь} с блистательной Любовью Орловой
- первой советской кинозвездой, с музыкой Исаака Дунаевского. Им близка была
мелодрама Ивана Пырьева \enquote{Свинарка и пастух}, в которой трогательную любовь
сыграли совсем еще молодые Марина Ладынина и Владимир Зельдин. Они с
удовольствием смотрели восстановленные копии еще немых советских фильмов
\enquote{Катька-бумаж\hyp{}ный ранет} с красавицей образца 1926 года Вероникой Бужинской в
главной роли, \enquote{Праздник святого Йоргена} режиссера Якова Протазанова с Игорем
Ильинским и Анатолием Кторовым, создавших образы проходимцев, потешавших
зрителей своими проделками.

Большой популярностью пользовались американские фильмы с участием голливудской
кинозвезды Дины Дурбин \enquote{Сестра его дворецкого} и \enquote{Сто мужчин и одна девушка},
не меньшей – английская музыкальная комедия \enquote{Джордж из Динки-джаза}. Но с ними
невозможно сравнить бешеный успех у зрителей \enquote{Серенады солнечной долины}. Там
было все: и знаменитая норвежская фигуристка Соня Хени, кстати, первая и
единственная трехкратная олимпийская чемпионка, и горы, укрытые сверкающим
блестками снегом, а главное - искрометная, зажигательная джазовая музыка в
исполнении оркестра под управлением Гленна Миллера. Неудивительно, что уже
после первых сеансов местные музыканты старались подобрать мелодии из
\enquote{Серенады} - кто на стареньком баяне, кто на трофейном аккордеоне, а кто и на
пианино. Через короткое время эти на свой лад аранжированные мелодии уже
звучали на танцплощадках и перед сеансами в кинотеатрах. Немало было выплакано
слез мариупольчанок во время демонстрации картин \enquote{Мост Ватерлоо} и \enquote{Леди
Гамильтон} с английской актрисой Вивьен Ли, исполнившей главные роли в них с
необыкновенным проникновением и достоверностью.

\textbf{Читайте также:} 

\href{https://mrpl.city/news/view/iz-mariupolya-v-rim-studentka-mariupolskogo-vuzaprizer-mezhdunarodnogo-kinematograficheskogo-konkursa-foto-plusvideo}{Из Мариуполя в Рим: студентка мариупольского вуза – призер Международного кинематографического конкурса, Олена Онєгіна, mrpl.city, 09.11.2018}

Пожалуй, через год или два после войны, а может, чуть позже на экранах
Мариуполя появились кинокартины, взятые у побежденной Германии в качестве
трофеев, многие из которых были подлинными шедеврами мирового кино. Среди них
был музыкальный фильм \enquote{Петер} с венгерской актрисой Франческой Гааль,
приключенческий - \enquote{Индийская гробница}. Среди особо рьяных мариупольских
поклонников кино выработалась своя оценочная терминология для кинолент. В их
устах фильмы были \enquote{переживательными}, \enquote{отдыхающими}, \enquote{про войну} и
\enquote{жизненными}.

Пристрастия же мальчишек были иными. Они могли бес­счетное количество раз
смотреть фильмы \enquote{Щорс} и \enquote{Александр Пархоменко}, \enquote{Два бойца} и \enquote{Иван Никулин -
русский матрос}, \enquote{Парень из нашего города} и \enquote{Чапаев}. После про­смотра
очередного фильма они \enquote{сражались} в погорелках, стараясь воспроизвести
увиденное на экране. Незавидные роли \enquote{беляков} и \enquote{фашистов} доставались самым
безот­ветным и беззащитным мальчуганам и девчонкам. С появлением на экранах
нескольких серий \enquote{Тарзана} именно главный герой этого фильма с незамысловатым
сюжетом стал примером для подражания сорванцов. Они лазали по деревьям,
бельевые веревки заменяли им лианы. Они улюлюкали, имитируя воинственные кличи
африканских дикарей...

Все это ушло в прошлое, и для молодежи мало что скажут перечисленные названия
фильмов. А люди, так сказать, в возрасте посетуют, наверное, почему здесь не
названы \enquote{Большой вальс}, \enquote{Сказание о земле Сибирской}, \enquote{Дитя Дуная}, \enquote{Петер},
\enquote{Молодая гвардия}, \enquote{Первоклассница}, \enquote{Весна}. Что ж, фильмов было много, все
не перечислишь...

\textbf{Читайте также:} \href{https://mrpl.city/blogs/view/kino-v-klenovyh-zaroslyah}{Кино в кленовых зарослях, Яна Иванова, mrpl.city, 13.06.2018}
