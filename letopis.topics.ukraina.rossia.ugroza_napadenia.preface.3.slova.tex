% vim: keymap=russian-jcukenwin
%%beginhead 
 
%%file topics.ukraina.rossia.ugroza_napadenia.preface.3.slova
%%parent topics.ukraina.rossia.ugroza_napadenia.preface
 
%%url 
 
%%author_id 
%%date 
 
%%tags 
%%title 
 
%%endhead 
\clearpage
\subsubsection{Слова, слова, слова...}

Касательно мнимого вторжения России в Украину, и словесном потоке, его
сопровождающем.  Знаете, это даже интересно за этим наблюдать. Какой-то
массовый психологический эксперимент происходит, и самое удивительное, что в
нем принимают участие самые могущественные страны и люди, - США,
Великобритания, президент США, премьер-министр Британии... Так и хочется
ущипнуть себя за щеку и спросить - божечки, неужели это таки происходит на
самом деле, они там что, совсем все уже ку-ку? Чего вы там все так подвисли на
Украину и Россию, у вас что, своих проблем не хватает?

Вроде с осени 2021 пишут, что вот-вот уже сейчас вторгнется, а сейчас уже
февраль 2022 года, - и поразительно! - в Киеве спокойно и тихо. Спокойно
работает метро, ездят машины, ходят пешеходы. Влюбленные парочки рассматривают
Днепр, а компания подростков что-то там увлеченно обсуждает.  Как же так? Уже
столько написано, уже как бы ясно - что все это - один гигантский
информационный пузырь, а до сих пор находятся люди, которые во всю эту чушь
неистово верят. На границе страшный враг, он обязательно хочет нас уничтожить,
злобный карлик в Кремле день и ночь не спит, думая, как бы захватить бедную
несчастную Украиночку. В чем же дело?  Почему постоянная ложь и нагнетание со
стороны Западных сми до сих пор проходит в Украине? Наверное, тут есть
несколько причин:

\begin{itemize} % {
\item в информационном обществе 21 века ценность слова как такового, да, именно
Слова, с большой буквы напишем, вот, - стала очень низкой, и вообще
отсутствующей. Люди в фейсбуке, телеграме и СМИ пишут что попало и как попало.
Какой-то сплошной базар, один кричит одно, другой кричит другое, и громче всего
кричат те, у кого совести и разума меньше всего.  А в обычной жизни как? Если
ты что-то пообещал, своему другу например, - то надо исполнить обещанное, - а
то можно и друга потерять, нажив вместо него врага. Это называется по-простому
отвечать за свой базар, или же, пацан сказал, - пацан сделал.
\end{itemize} % }
