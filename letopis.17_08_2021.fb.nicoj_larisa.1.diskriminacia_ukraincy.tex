% vim: keymap=russian-jcukenwin
%%beginhead 
 
%%file 17_08_2021.fb.nicoj_larisa.1.diskriminacia_ukraincy
%%parent 17_08_2021
 
%%url https://www.facebook.com/nitsoi.larysa/posts/972096986690007
 
%%author Ницой, Лариса
%%author_id nicoj_larisa
%%author_url 
 
%%tags belenjuk_zhan,diskriminacia,nicoj_larisa,ukraina,ukraincy
%%title Нас, українців дискримінують на кожному кроці
 
%%endhead 
 
\subsection{Нас, українців дискримінують на кожному кроці}
\label{sec:17_08_2021.fb.nicoj_larisa.1.diskriminacia_ukraincy}
 
\Purl{https://www.facebook.com/nitsoi.larysa/posts/972096986690007}
\ifcmt
 author_begin
   author_id nicoj_larisa
 author_end
\fi

Я заздрю Жану Беленюку. 

Все думаю про випадок з Беленюком, який, попри ганебність ситуації – у цій
ситуації щасливчик. Він від проявів цькування і дискримінації захищений законом
і суспільною думкою. 

Мене кожного дня цькують і дискримінують. Я отримую пачками листи, що мене
згвалтують, поріжуть на шматки і закопають мою сім’ю – і це НЕ ВВАЖАЄТЬСЯ
цькуванням і дискримінацією. Бо подібні випади проти українськомовних українців
в Україні - явище звичне.

\ifcmt
  tab_begin cols=2

     pic https://scontent-cdt1-1.xx.fbcdn.net/v/t1.6435-9/238987853_972096476690058_8881535896456051145_n.jpg?_nc_cat=110&_nc_rgb565=1&ccb=1-5&_nc_sid=730e14&_nc_ohc=IC4m0bA1afAAX_dkYWR&_nc_ht=scontent-cdt1-1.xx&oh=0c73e5dc69f39d83f67a16e008fb5ff1&oe=614600DC

     pic https://scontent-cdg2-1.xx.fbcdn.net/v/t1.6435-9/238148168_972096816690024_708749160138901371_n.jpg?_nc_cat=102&_nc_rgb565=1&ccb=1-5&_nc_sid=730e14&_nc_ohc=9UyElSfDJUgAX-K2JOo&_nc_ht=scontent-cdg2-1.xx&oh=4d9082610b421285b8c3153a1fbd46ee&oe=61443A51

  tab_end
\fi

Беленюка зацькували один раз (підозрюю, що насправді не раз у його житті) – але
ж яка піднялася суспільна хвиля (і я в тій хвилі теж) на його захист!

Мене ж цькують КОЖНОГО ДНЯ і кожного дня протягом свого життя я стикаюся з
проявами дискримінації від водія маршрутки, продавця в магазині – до всіх
національних ЗМІ. Відкрийте заголовки в Інтернеті, що ви там побачите?
«Одіозна» Ніцой. «Скандальна» Ніцой, тощо. І це вже увійшло в історію. Уже це
не вигризеш. А мені ж із цим іти до дітей і батьків, мені з цим іти в школи. Чи
зачиняються через це переді мною двері? Звісно.  Але чому я одіозна? Бо кажу,
що в Україні повинна панувати українська мова – мова мого народу і цієї землі.
Це одіозність в інших країнах? Ні, це норма. Але мене так називають, бо цькують
і дискримінують. Не хочуть давати мені право на мою мову. Чому я скандальна? Бо
кажу: «Українською, будь ласка». Це скандал? Ні. Це лагідна чемна фраза. Але
водночас – це скандал з боку моїх співрозмовників. Як посміла українка в
Україні вимагати українську мову для себе? Це великий скандал ПРОТИ такої
українки. І це велика ДИСКРИМІНАЦІЯ мене і таких, як я. 

Порівняйте з Беленюком. Навколо нього зчинився скандал, але його назвали
скандалістом? Ні. Порівняйте (не в тему про чемпіона, але в тему до ситуації) з
Мендель. Попри її висловлювання, які день за днем спричинюють скандал – її
називають скандалісткою чи одіозною? Ні. А мене називають. Чому? Бо мова про
УКРАЇНОМОВНІСТЬ. А що стосується утвердження україномовності – супроводжується
цькуванням і дискримінацією.

Нас, українців дискримінують на кожному кроці. Дискримінація відбувається по
всій Україні в загальнонаціональних масштабах.

Беленюк – щасливчик, він захищений. Захист таких, як Беленюк, (за расовою
ознакою, за кольором шкіри) – це не наша заслуга. Такий захист і такі закони
прийшли до нас з інших країн. Ми їх запозичили. Ми запозичили цю справедливу
норму від інших народів, які її встановили і вибороли, а ми перейняли. Такий
захист Беленюк отримав з Америки, з Європи і має в Україні.

А з якої країни мені чекати захисту? Мені, українці, яка живе в Україні? Яка
країна і яка нація повинна написати для мене закон і встановити захист мені і
таким як я? Хто визнає, що позбавлення українців в Україні своєї української
мови і права на своє рідномовне середовище – це ДИСКРИМІНАЦІЯ?  Хто допоможе
українці, яка ПОЗБАВЛЕНА на своїй землі у своїй країні свого рідномовного
українськомовного середовища, яка МРІЄ в далекій Україні про своє рідномовне
середовище (нормальне явище для всіх інших народів) – і за своє бажання щодня
піддається дискримінації, яка НЕ ВИЗНАНА ДИСКРИМІНАЦІЄЮ? 

Мені немає звідки чекати такої допомоги. Закон про дискримінацію українців на
мовному ґрунті можуть написати лише в Україні, але коли і хто його напише, якщо
самі українці не розуміють, що обмеження їх у праві на свою українську мову у
своїй країні – це кричущий приклад ДИСКРИМІНАЦІЇ на ЗАГАЛЬНОНАЦІОНАЛЬНОМУ
рівні!

Я заздрю Беленюку. На захисті його кольору шкіри і його ПРАВА почуватися вільно
і комфортно в Україні, на захисті від проявів дискримінації таких, як Беленюк
стоїть не лише мій маленький голос, а увесь цивілізований світ. Увесь світ
розуміє, що це дискримінація.

Українськомовні українці перебувають у гіршому становищі, ніж Жан.
Україномовних українців позбавили права в Україні перебувати в комфортному їм,
рідному українськомовному середовищі. Українськомовні українці позбавлені права
виховувати своїх дітей в українськомовному середовищі,  а намагання повернути
своє право наражаються на цькування і дискримінацію – і при цьому
дискримінацією це не названо НІДЕ, ні в Україні, ні (я вже мовчу) на
міжнародному рівні.


