% vim: keymap=russian-jcukenwin
%%beginhead 
 
%%file 27_01_2021.fb.fb_group.story_kiev_ua.1.osvobozhdenie_kieva.cmt
%%parent 27_01_2021.fb.fb_group.story_kiev_ua.1.osvobozhdenie_kieva
 
%%url 
 
%%author_id 
%%date 
 
%%tags 
%%title 
 
%%endhead 
\zzSecCmt

\begin{itemize} % {
\iusr{Елена Мирошниченко}
Плачу.

\begin{itemize} % {
\iusr{Мария Каменцова}
\textbf{Елена Мирошниченко} . 

Я сама плакала, когда писала... И мама, когда мне рассказывала. Я постаралась
воспроизвести ее рассказ с фотографической точностью. Все детали...


\iusr{Лариса Заболотная}
\textbf{Елена Мирошниченко} и я...

\end{itemize} % }

\iusr{Александр Лесик}
кошмар. моей бабке повезло она в каком то другом Киеве оккупацию пережила.

\begin{itemize} % {
\iusr{Мария Каменцова}
\textbf{Александр Лесик} . Да жить в оккупацию, мама рассказывала, можно было... Только когда немцев стали выбивать из города, они начали свирепствовать!

\begin{itemize} % {
\iusr{Александр Лесик}
\textbf{Мария Каменцова} 

а моя говорила что тихо ушли. за пару дней до освобождения. просто вдруг много
бродячих собак появилось. (немцы собак беспризорных отстреливали). не стало
немцев. пришли собаки. а потом коммунисты начали Киев обстреливать в котором
кроме жителей не было уже никого (но это понятно они ж предатели не
самоубились). а еще говорила - когда советские Киев сдавали то повзрывали все
эшелоны с зерном на Дарницком вокзале. чтоб не досталось никому. а фашисты когда
бежали-склады с едой (которую не успели вывезти) не взрывали. наверно взрывчатку
пожалели.

\iusr{Мария Каменцова}
\textbf{Александр Лесик} . 

Я так понимаю, что немцы тихо ушли, но до этого пытались уничтожить то и тех,
кого могли. Передаю просто рассказ моей мамы. Она девочкой была, рассказала то,
что сама видела. Может быть, другие видели и знали больше!

\iusr{Александр Лесик}
\textbf{Мария Каменцова} наверно.у мамы медали были ?

\iusr{Мария Каменцова}
\textbf{Александр Лесик} . Ей было 14 лет. Медали у нее были, но не боевые, юбилейные, и довольно много! Они сохранены.

\iusr{Александр Лесик}
\textbf{Мария Каменцова} 

ага у всех так. и дед и бабушка вся грудь в медалях. как начал вникать оказалось
все юбилейные. 20 лет победы 30. и т д. даже орден вов. и тот юбилейный. а за
реальные боевые достижения дед ..благодарность.. от товарища Сталина.. (письмецо
на тетрадном листе написанное) получил. ъ

\iusr{Tatiana Thoene}
\textbf{Александр Лесик} 

моя мама тоже вспоминала, что немцы ушли быстро и вроде «тихо». У них во дворе,
на Коминтерна, была немецкая полевая кухня. Когда наши стали приближаться к
Киеву, собрали манатки и пошли строем к вокзалу. Наверное, не везде
свирепствовали и кому-то повезло.

\end{itemize} % }

\end{itemize} % }

\iusr{Aleksandra Gorovaya}
Огромное спасибо! Слов нет - только слезы.

\begin{itemize} % {
\iusr{Мария Каменцова}
\textbf{Aleksandra Gorovaya} . Я, пока писала (вначале на компьютере), тоже целый день плакала...

\iusr{Мария Каменцова}
\textbf{Aleksandra Gorovaya} . Мама и папа, оба пережившие оккупацию в Киеве, рассказывали много, я здесь еще покажу! Некоторые рассказы прямо необыкновенные! Хотя они почти дети еще были, 14 лет...

\begin{itemize} % {
\iusr{Aleksandra Gorovaya}
\textbf{Мария Каменцова} Мои смогли уехать из Киева, хотя соседки по коммунальной квартиры, уговаривали бабушку остаться. (Как потом выяснилось , дворник выдал немцам всех евреев в доме). Школьная подруга моей матери, которая осталась со своей мамой в оккупированом Киеве, рассказывала, что немцы разрешили учиться в школе только детям, которые занимались балетом.
\end{itemize} % }

\end{itemize} % }

\iusr{Тома Храповицкая}
Спасибо... больше нет слов, Спасибо Вам и Вашей Маме за такие тяжёлые но нужные всем Нам воспоминания!!!!!!!!!

\iusr{Мария Каменцова}
\textbf{Тома Захарова} . Без памяти нет Родины.

\iusr{Ирина Иванова}

Машуня, до слез!!!! И какой же нежной, чистой как первая снежинка на фото твоя
мама. По фото не скажешь, что у этой девочки столько силы..... .Светлая,
добрая память живет во мне о тете Нине

\begin{itemize} % {
\iusr{Мария Каменцова}
\textbf{Ирина Иванова} . Вот она и побыла с нами как будто воочию... Вечная память! Рассказов как ее, так и папы множество. Я тут еще покажу...

\iusr{Оксана Тернавская}
\textbf{Ирина Иванова} мы все помним и её, и всех наших близких с ул.Герцена. Тем более, жизнь так связала наши судьбы.
\end{itemize} % }

\iusr{людмила петрухина}

Моя бабушка тоже пережила этот ужас, пряталась в Голосеевском лесу со своей
подругой, тоже рассказывала об освобождении Киева, о том, что кругом были одни
развалины, сколько же им пришлось пережить.

\iusr{Мария Каменцова}
\textbf{людмила петрухина} . Мы обязаны об этом помнить и рассказывать. Чтобы не повторилось...

\iusr{Giorgi Begeluri}
Kakaia krasivaiajenshchina!

\begin{itemize} % {
\iusr{Мария Каменцова}
\textbf{Giorgi Begeluri} . Потом она немного снималась в кино, ее пригласили... Стала еще красивее!

\iusr{Giorgi Begeluri}
\textbf{Мария Каменцова} ona vasha mama?

\iusr{Мария Каменцова}
\textbf{Giorgi Begeluri} . Да, конечно!  @igg{fbicon.smile} 
\end{itemize} % }

\iusr{Виктория Аксененко}

Мой дедушка был военный и успел вывезти бабушку с моей мамой и тетей. А как
только освободили Киев они вернулись в свою квартиру на ул. Чкалова. Мама тоже
много расказывала уже про освобожденный Киев. Страшное время было..... Мы должны
знать и помнить те года.....


\iusr{Татьяна Чернышева}
Мама такая красивая@igg{fbicon.heart.red}

\iusr{Мария Каменцова}
\textbf{Татьяна Чернышева} . Спасибо! Я еще буду рассказывать, покажу ее фото постарше, когда расцвела – она была прямо кинозвезда!

\iusr{Marina Chernaya}
Как ценны воспоминания очевидцев, когда так нещадно переписывают историю!

\begin{itemize} % {
\iusr{Мария Каменцова}
\textbf{Marina Chernaya} . Критерий всего – ИСТИНА.

\iusr{Marina Chernaya}
\textbf{Мария Каменцова} 

истину знает только Бог, порядочные историки, используя научные методы, вне
всякой политики и пропаганды, пытаются к ней приблизиться. В истории всех
народов и стран были светлые и темные страницы, нельзя этим манипулировать. Как
сказал А. Мень, правда одна, остальное - ложь.

\end{itemize} % }

\iusr{Елена Свец Баркан}

Моя бабушка с двумя детьми 36 и 41 года рождения (мой дядя и мама) была все
время оккупации в Киеве. Не могу даже себе представить, каково ей было

\begin{itemize} % {
\iusr{Александр Лесик}
\textbf{Елена Свец Баркан} может надо было спросить?

\begin{itemize} % {
\iusr{Елена Свец Баркан}
\textbf{Александр Лесик} 

кое что бабушка рассказывала, но она умерла, когда мне было только 16. Что
успела - рассказала. Как ходила по селам под Киевом менять вещи на продукты.
Когда проходила через посты если стояли немцы, то проверяли документы и
пропускали, а если были полицаи, то часто забирали вещи или продукты.

\iusr{Александр Лесик}
\textbf{Елена Свец Баркан} ага. а энкаведисты с 1918 до 39 просто стреляли и даже не закапывали

\iusr{Ольга Жук}
\textbf{Александр Лесик} вас, лично стреляли?
\end{itemize} % }

\iusr{Лариса Заболотная}
\textbf{Елена Свец Баркан} и мои... Бабушка 1905 года, мама 1929...

\iusr{Наталия Погорелова}

Моя бабушка 1908 года с двумя дочками 1928 и 1933 тоже остались в оккупации.
Жили на Сталинке в подвале деревянного домишки на 4 семьи. Мама очень много
рассказывала о том времени...

Перед освобождением Киева немецкий офицер, занимавший в их доме пустую
квартиру, предупредил, что город будет взорван и чтоб уходили поскорее. Мамина
семья с соседями, со своим каким-то скарбом, ушли в сторону Мышеловки. Ночевали
и дневали (в ноябре!) под открытым небом. Но ночами мама с её тетей (1920 г.р.,
8 лет разницы всего, так что были практически подругами) ходили в Киев
покормить лежачую бабушку тетиного жениха (он воевал, а тётя за ней
присматривала)...

Тетю эту, кстати, отправляли в начале оккупации в Германию, но она как-то
умудрилась сбежать - выпрыгнула из вагона и вернулась в Киев.

\iusr{Victor Rozhkov}

Мой отец тоже был в оккупации. Но такого как расстрел мирных не было. Наоборот
немцы уводили людей чтоб они не пострадали. А хуже всех в окупации были полицаи
из наших. Это слова отца. И он ещё жив.

\end{itemize} % }

\iusr{Тетяна Бондаренко}
Читала про саме цей момент в одній книзі про Київ в окупації!

\iusr{איליאנה ז׳וקוב}
 @igg{fbicon.face.downcast.sweat}  @igg{fbicon.flag.izrail}

\iusr{Lola Madino}

Щось я чула, та й документи свідчать - німці ВИСЕЛЯЛИ мирне населення. А не
вбивали... Так, за непідкорення наказу - розстріл. Але можна було скоритись
наказу та піти.

\end{itemize} % }
