% vim: keymap=russian-jcukenwin
%%beginhead 
 
%%file 28_02_2021.fb.stepanov_farid.1.solovjina_mova
%%parent 28_02_2021
 
%%url https://www.facebook.com/StepanovFarid/posts/527017341606749
 
%%author Степанов, Фарид
%%author_id stepanov_farid
%%author_url 
 
%%tags jazyk,mova
%%title СОЛОВ'ЇНА МОВА
 
%%endhead 
 
\subsection{СОЛОВ'ЇНА МОВА}
\label{sec:28_02_2021.fb.stepanov_farid.1.solovjina_mova}
 
\Purl{https://www.facebook.com/StepanovFarid/posts/527017341606749}
\ifcmt
 author_begin
   author_id stepanov_farid
 author_end
\fi

СОЛОВ'ЇНА МОВА.

Если вы думаете, что нельзя в одном предложении использовать сразу три разных
языка? То вы - ошибаетесь. Сам слышал. Случайно. 

Мимо меня, в направлении школы, пронеслась школьница, на вид, 14-15 лет. На
ходу разговаривая по телефону. То, что люди на улице  громко разговаривают по
телефону, давно привык. Это ещё пол-беды. Что хуже - они, увлечённые
разговором, ещё и не замечают уже никого и ничего вокруг себя. 

\ifcmt
  pic https://scontent-cdg2-1.xx.fbcdn.net/v/t1.6435-9/154927594_527017314940085_3281157971814610814_n.jpg?_nc_cat=107&ccb=1-5&_nc_sid=8bfeb9&_nc_ohc=fhvql3jiX4sAX9BeV6P&_nc_ht=scontent-cdg2-1.xx&oh=f8c8c467f45268985882eab21e33c8f7&oe=61444840
  width 0.4
\fi

То, что я услышал, тянуло на шедевр:

"Ну, ok, я отута вже жду тебе цілих півчаса... "

Совместить в одном предложении: украинский, русский и английский, усилив их,
для глубины впечатления, суржиком, дорогого стоит. 

"Я украинский бы выучил только за то..."

Почти цитата.

Люди, если вы так не любите нашу страну Украину, если вы так не уважаете
украинский язык, то хотя бы пожалейте уши окружающих. 

А если любите и уважаете, то выучите, что:

"Горжусь" на украинском - "пишаюся", "жду" - "чекаю", что "на протязі" - "на
сквозняке", а "на протяжении" - "протягом", что "час" - "година"! 

Учите украинский язык, читайте много книг, штудируйте словари, слушайте
настоящих носителей языка. Только не смотрите сериалы на украинском и не
слушайте ток-шоу. Иначе: "Ну, ok, я отута вже жду тебе цілих півчаса... "
покажется высоким образцом "солов'їної мови". 

Сразу вспомнил эпизод из старого советского фильма "Чапаев". 

Василий Иванович (Чапаев) учит английский. Заходит Петька (его ординарец).
Далее, не ручаюсь за дословность, но смысл следуюший:

- А что, Василий Иванович, английский сложно учить? 
- Нет, Петька, ничего сложного. Слова как у нас. Просто в некоторых нужно поменять окончание. 
- Как это? 
- Например, не нравится им наше слово "революция", так они в конце шипят на него - "революшн". 

Так вот, что бы разговаривать на украинском языке, недостаточно взять русский,
изменив ударение и окончание. 

Я свободно говорю, читаю и пишу на русском и украинском. На высоком уровне на
английском. 

Мне повезло: в школе и КГУ я учился на украинском. Потом преподавал на
украинском историю, право и философию. И всё равно, бывает, делаю ошибки. Не
считаю зазорным взять словарь и проверить. И благодарю, когда меня исправляют.

Потому, что люблю и уважаю Украину и украинский язык. Хотя, в повседневной
жизни, разговариваю на русском. Так исторически сложилось - думаю на этом
языке. 

PS:

У Лены была языковая ситуация. Она обратилась в Киеве на улице к женщине. А
женщина решила её повоспитывать:

"Ото воно понаїхали сюда. І тута на руськом розговарюють". Видимо, попрекая
Лену тем, что она обратилась к ней на русском. 

В следующий раз ответь: "Так и Вы на украинском не говорите" - посоветовал ей.
