% vim: keymap=russian-jcukenwin
%%beginhead 
 
%%file 20_11_2020.news.ua.gazeta.1.russki_jazyk_silikonovy
%%parent 20_11_2020
 
%%url https://gazeta.ua/blog/54216/russkij-yazik-mi-povinni-zberigati-dlya-silikonovih
 
%%author 
%%author_id 
%%author_url 
 
%%tags 
%%title "Русскій язик" ми повинні зберігати для "силіконових" 
 
%%endhead 

\subsection{\enquote{Русскій язик} ми повинні зберігати для \enquote{силіконових}}
\label{sec:20_11_2020.news.ua.gazeta.1.russki_jazyk_silikonovy}

\Purl{https://gazeta.ua/blog/54216/russkij-yazik-mi-povinni-zberigati-dlya-silikonovih}
\Pauthor{Несенюк, Микола}

\begin{leftbar}
	\bfseries
	Інакше Путін нападе
\end{leftbar}

Невимушено почув учора діалог двох киян. Таке трапляється, коли не хочеш, але
мусиш вислуховувати те, що говорять довкіл. Тим більше - самі співрозмовники не
дуже зважають на присутніх.

Отож, двоє відносно молодих, від 30 до 40 років, людей обговорюють сусідку по
дачній ділянці - а кого ще обговорювати \enquote{карєнним кієвлянам}?

І він, і вона, очевидно, не бідні – мають пристойні авто і працюють не
двірниками. Розмова, ясна річ, ведеться на \enquote{київ-рашен} - київському варіанті
російської із характерним \enquote{шоканням та гаканням}. Обговорювана парою сусідка,
судячи із діалогу, загалом оцінюється позитивно. Є лише один недолік, на який
вказує той, що старший:

- Штота наша Андрєєвна стала дружить с Бахусом.

- С сабакай?

У розмові зависає пауза. А що? Звідки 30-річній \enquote{русскаязичній} киянці знати,
хто такий Бахус? Їй для успішного життя цілком вистачає словникового запасу
російських серіалів 90-х років минулого століття, на яких вона виросла у
столиці незалежної України. Ви би у неї ще про щось із \enquote{Євгенія Онєгіна}
запитали. Аби дізнатися, що Лєнскій – це варіант собачої або кошачої клички.

Саме, так, дорогі мої співвітчизники! Саме для таких успішних киянок із
силіконовими губами ми повинні зберігати \enquote{русскій язик}. Щоб не створювати їм
незручностей. Інакше Путін нападе!

Микола Несенюк, для Gazeta.ua

