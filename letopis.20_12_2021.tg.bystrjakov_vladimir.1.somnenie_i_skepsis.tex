% vim: keymap=russian-jcukenwin
%%beginhead 
 
%%file 20_12_2021.tg.bystrjakov_vladimir.1.somnenie_i_skepsis
%%parent 20_12_2021
 
%%url https://t.me/bystryakovvladimir/29
 
%%author_id bystrjakov_vladimir
%%date 
 
%%tags 
%%title Сомнение и Скепсис.....
 
%%endhead 
\subsection{Сомнение и Скепсис.....}
\label{sec:20_12_2021.tg.bystrjakov_vladimir.1.somnenie_i_skepsis}

\Purl{https://t.me/bystryakovvladimir/29}
\ifcmt
 author_begin
   author_id bystrjakov_vladimir
 author_end
\fi

Сомнение и Скепсис..... 

Два понятия, в основе которых на первый взгляд лежит одна и та же мысль -
НЕВЕРИЕ.... Вроде бы два родственных понятия, но какая же, с моей точки зрения,
между ними  лежит колоссальная пропасть.... Поясню, как я это понимаю. Сомнение
- качество любого ТАЛАНТЛИВОГО человека, способному к СОЗИДАНИЮ. 

На любой стадии процесса человеком может овладевать сомнение. В правильности
метода, в использовании тех или иных красок, приемов и прочих деталей своего
Основного труда.... 

Наконец - в своем Главном Жизненном Выборе... 

и зачастую бороться с ним (с пришедшим сомнением) достаточно трудно. но -
необходимо ! Потому как, преодолев его, ты.... обязательно поднимешься на
ступень выше, найдешь вместе с верой в свое Предназначение и новые приемы, и
усовершенствуешь метод, и, что Главное - удесятеришь свои творческие силы. Я
подозреваю, что так поступали многие выдающиеся деятели  искусства (вспомним
того же Ван Гога, продавшего при жизни лишь....ОДНУ (!) свою картину, но при
этом не утратившего стимула к творчеству), Создатели, которых оценили только
после их ухода.... Но ведь что-то заставляло их преодолевая сомнения,
продолжать свой как многим казалось \enquote{напрасный} каторжный труд... Что-то ведь
мучило их души, но, одновременно... вело сильной рукой к мольберту, к роялю, к
письменному столу...

Вывод - СОМНЕНИЕ есть непременный спутник Творчества. 

И это прекрасно !

И совсем другое дело - СКЕПСИС. Он - ржавчина для любой свежей творческой
мысли.... он - раковая опухоль, что способна разъесть новый смелый проект на
корню, навести опустошительный разгром в уме человека, решившего взять новую,
до того неведомую нам высоту. Не секрет, что многие успешные проекты, кажущиеся
сегодня нам аксиомой, естественной частью нашего сознания, поначалу вызывали
оторопь у людей нормальных, людей талантливых. Тот же Эмиль Золя,
организовавший целую кампанию из УМНЫХ людей против ....Эйфелевой башни... 

Ответьте, сегодня кто-то может себе представить Париж без нее?! 

А ведь тогда такой гвалт подняли.. \enquote{...снести к ядреной фене, дабы не позорила
своим дурацким вид красоту города !}
и т.д. 

Послушали бы тогда умных скептиков, и на что теперь любовались бы... ?!

Почему я вдруг поднял тему скепсиса сегодня, когда у нас с вами и без
Творчества вкупе с другими \enquote{высокими} темами хватает насущных проблем ?! А
навеяло мне это большим количеством СКЕПТИЧЕСКИХ комментов на мой пост о съезде
Союза Левых Сил. Хочу поинтересоваться у вас, скептики вы наши \enquote{вумные}.... А
не надоело ли вам сидеть в полной заднице вместе со всей страной и ее
\enquote{керовництвом}?! Или вы считаете, что наша нынешняя \enquote{дыра} - это наше ВЕЧНОЕ и
ЕСТЕСТВЕННОЕ состояние ?! Или может, будем ждать...... Путина ли, Байдена ли, а
может  ....инопланетян ?! Иными словами, анализируя мотивацию сих \enquote{оппонентов},
можно сделать только один вывод - об их колоссальном ЭГОИЗМЕ !   (это в том
случае, ежли сии \enquote{скептики} не являются \enquote{патриотами на доверии и бюджете})

Мол, пусть наши проблемы решают \enquote{где-то ТАМ, подале....} а мы тут будем
вумничать и подвергать скепсису (читай \enquote{обсерать}) любое свежее движение, любую
свежую инициативу ...

Как тот \enquote{Премудрый пескарь} из классической литературы....

Проводить дурацкие псевдо \enquote{исторические} аналогии, дискредитировать людей
неравнодушных, стремящихся к возникновению по-настоящему демократических сил,
старающихся хоть что-то изменить в разъеденной раком Стране, в губительном для
всех нас курсе ее \enquote{кормчих} - это все - дело конечно легкое, простое..... 

Сиди себе \enquote{увдома} перед компом, и сливай все говно, распирающее
\enquote{патриотическую} душу,  в сеть.... 

И самоутверждайся перед женой, тещей и парой таких же как сам... А потом можно
будет позволить себе и.... \enquote{прозреть}. 

И влиться под шумок в колонны \enquote{бывших врагов}, а может даже и возглавить ...
Сейчас иногда наблюдаю за парой таких, особо шустрых.... и чего-то видятся они
мне в.... буденовках и с георгиевской лентой в косоворотках...
