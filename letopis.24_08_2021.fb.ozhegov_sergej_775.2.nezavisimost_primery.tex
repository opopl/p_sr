% vim: keymap=russian-jcukenwin
%%beginhead 
 
%%file 24_08_2021.fb.ozhegov_sergej_775.2.nezavisimost_primery
%%parent 24_08_2021
 
%%url https://www.facebook.com/sergey.ozhegov.775/posts/537258420828470
 
%%author Ожегов, Сергей
%%author_id ozhegov_sergej_775
%%author_url 
 
%%tags nezalezhnist,ukraina
%%title Наглядные примеры послемайданной "незавимости"
 
%%endhead 
 
\subsection{Наглядные примеры послемайданной \enquote{незавимости}}
\label{sec:24_08_2021.fb.ozhegov_sergej_775.2.nezavisimost_primery}
 
\Purl{https://www.facebook.com/sergey.ozhegov.775/posts/537258420828470}
\ifcmt
 author_begin
   author_id ozhegov_sergej_775
 author_end
\fi

Наглядные примеры послемайданной "незавимости":

\begin{itemize}
  \item - чиновник США в кресле президента Украины; 
  \item - преклонение офицеров украинской армии перед американскими и английскими послами с передачей им оружия, стоя при этом на коленях; 
  \item - проход военных США по Крещатику. Не в парадной форме и прогулочным шагом;
  \item - национальный гопак перед иностранными военными.
\end{itemize}

И это на фоне того, что богатейшая ресурсами страна, в 1991 году получившая
уникальную возможность стать самым богатым государством в Европе, за 30 лет
превратилась в самое бедное государство. При самых богатых в Европе элитах,
безсовестно грябящих население своей страны.

\ifcmt
  tab_begin cols=3

     pic https://scontent-cdg2-1.xx.fbcdn.net/v/t1.6435-9/238172988_537258234161822_1975849738939031246_n.jpg?_nc_cat=100&_nc_rgb565=1&ccb=1-5&_nc_sid=8bfeb9&_nc_ohc=-SLi7aPrYhoAX9O0H_N&_nc_ht=scontent-cdg2-1.xx&oh=6ad44a4b98dd0b32aef379e12ef89d93&oe=614B8AF4

     pic https://scontent-cdg2-1.xx.fbcdn.net/v/t1.6435-9/239115718_537258274161818_4922458769550552024_n.jpg?_nc_cat=102&ccb=1-5&_nc_sid=8bfeb9&_nc_ohc=m35hzSAEkBEAX8JYoAZ&_nc_ht=scontent-cdg2-1.xx&oh=3bd61c4104400197ded3f4c1c8d7c507&oe=614C90F1

		 pic https://scontent-cdt1-1.xx.fbcdn.net/v/t1.6435-9/239326458_537258320828480_9004844319098223147_n.jpg?_nc_cat=110&_nc_rgb565=1&ccb=1-5&_nc_sid=8bfeb9&_nc_ohc=09VzA-tr6UoAX8frsoN&_nc_ht=scontent-cdt1-1.xx&oh=36cd6cdf16ea846946bf05e5f5ae29c5&oe=614BE32A

  tab_end
\fi

Видеть, осознавать всё это и при этом называть независимостью, и главное верить
в это - это оруэловское двоемыслие.

Для тех, кто не читал антиутопию Оруэлла "1984", ставшую методичкой для наших
властей, кратко приведу суть понятия: 

Двоемыслие — способность искренне верить в две взаимоисключающие вещи, либо
менять своё мнение на противоположное при идеологической необходимости.

\ifcmt
  tab_begin cols=2

     pic https://scontent-cdt1-1.xx.fbcdn.net/v/t1.6435-9/239270295_537258360828476_7781126873350059949_n.jpg?_nc_cat=103&_nc_rgb565=1&ccb=1-5&_nc_sid=8bfeb9&_nc_ohc=vCr7Dc0DxSIAX9zwzwg&_nc_ht=scontent-cdt1-1.xx&oh=edc8905be48ef975f54de4fc0f695599&oe=614E0BDC

     pic https://scontent-cdt1-1.xx.fbcdn.net/v/t1.6435-9/240212795_537258404161805_3630937739993658468_n.jpg?_nc_cat=110&_nc_rgb565=1&ccb=1-5&_nc_sid=8bfeb9&_nc_ohc=ZrRhme1-cJAAX9LCkwD&_nc_ht=scontent-cdt1-1.xx&oh=9d5f99ab94781b1d88f1998750b38474&oe=614BF148

  tab_end
\fi


В романе правящая партия постоянно переписывает историю, и для того, чтобы люди
это принимали, им меняется мировозрение.

Лозунги государства в романе:

\begin{itemize}
  \item Война — это мир
  \item Свобода — это рабство
  \item Незнание — сила
\end{itemize}

Применимо к нынешней Украине двоемыслие в следующем:

\begin{itemize}
  \item 1.Война - это мир / Чтобы жить в мире непременно нужно воевать. Мирный вариант решения конфликта - недопустим, т.к. это война.
  \item 2.Свобода - это рабство / Быть свободными - это тотальное поклонение Западу и полная зависимость от него, несмотря на то, что тот открыто использует народ, грабит его и сокращает.
  \item 3.Незнание  - сила / Здесь самое важное: необходимо приучить народ не желать знать правду. Сила в слепой вере в пропаганду власти.
\end{itemize}

В романе власти регулярно применяют образ врага, периодически проводя на
телеэкранах т.н. "двухминутки ненависти", при которых винит во всём соседнее
государство и его представителя. Народ, глядя на экран, видя врага, впадает в
неистовство от ненависти.

В стране регулярно меняют язык, постоянно сокращая слова и убирая образность
слов. Тем самым отупляя народ.

...

Несмотря на всё это, у нас ещё есть надежда на то, что здравый смысл всё же
победит. И только тогда можно будет сказать, что Украина по настоящему обрела
независимость.

Подписывайтесь

\url{t.me/iskra1}
