% vim: keymap=russian-jcukenwin
%%beginhead 
 
%%file 26_01_2022.fb.fb_group.story_kiev_ua.1.kiev_mosty_1941.cmt
%%parent 26_01_2022.fb.fb_group.story_kiev_ua.1.kiev_mosty_1941
 
%%url 
 
%%author_id 
%%date 
 
%%tags 
%%title 
 
%%endhead 
\zzSecCmt

\begin{itemize} % {
\iusr{Павел Руденко}
Сейчас опять враг войной на нас пойдет, придется взрывать

\begin{itemize} % {
\iusr{Артур Игнатов}
\textbf{Павел Руденко} я на левом живу. Не нужно.

\iusr{Павел Руденко}
\textbf{Артур Игнатов} паром будет)

\iusr{Артур Игнатов}
\textbf{Павел Руденко} думаю, наши мосты скоро и без войны окажутся в воде.

\iusr{Надежда Владимир Федько}
\textbf{Артур Игнатов} Без паніки)
\end{itemize} % }

\iusr{Igor Rostov}
Мой текст из ЖЖ. Удивительно, через сколько лет его прочитали.

\iusr{Jon Yossarian}

И надолго это остановило Вермахт?

Бей своих, что б чужие боялись. Нет в истории человечества более ублюдочного
образования, чем совок. @igg{fbicon.face.symbols.mouth} 

\begin{itemize} % {
\iusr{Юрий Блохин}
\textbf{Jon Yossarian} вы вообще верите тому что пишите, или это не главное ? Ведь люди вашу глупость всерьёз принять могут.

\iusr{Jon Yossarian}
\textbf{Юрий Блохин} вы на вопрос ответьте, прежде чем слюной плеваться.
Насколько задержало уничтожение мостов, своих войск и гражданских на этих мостах продвижение Вермахта на левый берег?.
Заодно откройте тайну, как помогло борьбе с оккупантами взрыв Крещатика и прилегающих районов? Осветите отдельно судьбу жильцов этих домов, а так же заложников, расстрелянных в отместку за взрывы.
Если не можете, то заткните пасть, совок.

\iusr{Юрий Блохин}
\textbf{Jon Yossarian} 

У меня нет желания отвечать на глупые вопросы, ответы на которые есть даже у
людей изучавшие военную подготовку в гражданском ВУЗе. Я так понял, что вы из
той породы людей, которые орут на каждом шаге, что не хрен было сопротивляться
, Кац предлагает сдаться .... а там пили бы \enquote{Баварское} .... манкурт вы наш  @igg{fbicon.wink} 


\iusr{Jon Yossarian}
\textbf{Юрий Блохин} ну так и заткнитесь, болтун находка для шпиона

\iusr{Надежда Владимир Федько}
\textbf{Юрий Блохин} 

Повідайте нам, будь ласка, яка мотивація була у Київського угрупування ЧА
здатися в полон в кількості 665 000 (!) Чи про це на кафедрі військової
підготовки Вам не розповідали)

\iusr{Юрий Блохин}
Я так понял что я только что раскрыл вашу тайну  @igg{fbicon.face.grinning.sweat} 

\iusr{Дмитрий Артюк}
\textbf{Юрий Блохин} Когда у человека нету аргументов, он начинает затыкать рот и плеваться.  @igg{fbicon.smile} 
\end{itemize} % }

\iusr{Dmitrii Pisanenko}
Враг сейчас не ходит, а летает! Это еще хуже, захватит и все.

\iusr{Владимир Картавенко}

За три дня до того как советские войска покинули Киев, на Рыбальском создана в
ч 20884 и в 1967 установлен памятник военным морякам - монитор Железняков.

В 1999 умер отец Путина и нам предложили заменить монитор на подлодку сс-310.
Пробную на Коса Тузла выиграли, суды тоже. Под угрозой ядерного уничтожения
народа Украины подписали соглашения. ..


\iusr{Андрій Чмир}

Какие автоматчики? Фильм посмотрите \enquote{Город, которьій предали.} Три дня ! до 19
числа в Киеве НЕ бьіло немцев. И как взорвали мост с отходящими частями. Там
воспоминания детей-очевидцев.

\begin{itemize} % {
\iusr{Надежда Владимир Федько}
\textbf{Андрій Чмир} Чудовий фільм!

\iusr{Галина Зуева}
\textbf{Андрій Чмир} На мосту были и раненые и гражданские. Взорвали вместе с нашими людьми. Зато как \enquote{красиво} пишет в мемуарах

\iusr{Надежда Владимир Федько}
\textbf{Галина Зуева} Мемуари писалися \enquote{по канону}, написаному в Ідеологічному відділі ЦК КПСС.
\end{itemize} % }

\iusr{Владимир Мареев}
Страшное дело эта война. Мама 1934 г.р. всегда плакала когда войну вспоминала.
Пережила окупацию в Чернигове.

\iusr{Oleksandr Oleksandrovich Maslov}
Шел третий месяц ВОВ...

\iusr{Ковальская Татьяна}
Преступление 20 века!

\iusr{Людмила Пащинська}
Кто считает - что надо было оставить все немцам - на блюдечке с голубой каемочкой ?

\begin{itemize} % {
\iusr{Надежда Владимир Федько}
\textbf{Людмила Пащинська} Знищення інфраструктури - військовий злочин!

\iusr{Tovarish Inka}
\textbf{Надежда Владимир Федько} Це ви зараз про сьогодення. А що стосовно оборони та військових дій в минулому столітті?

\iusr{Надежда Владимир Федько}
\textbf{Tovarish Inka} Міжнародне законодавство про правила ведення війни і військові злочини були прийняті ще у 1907 році.

\iusr{Tovarish Inka}
\textbf{Надежда Владимир Федько}

\ifcmt
  ig https://scontent-frx5-1.xx.fbcdn.net/v/t39.30808-6/s851x315/272822672_1073689263214692_1574443218841073449_n.jpg?_nc_cat=100&ccb=1-5&_nc_sid=dbeb18&_nc_ohc=p9tZX8KVI58AX9UwZEw&_nc_ht=scontent-frx5-1.xx&oh=00_AT88oOZMo1CK9DrGk-C-TqBOBuvekKmXB_YQFqtHsTfxBQ&oe=61F7EA57
  @width 0.3
\fi

\end{itemize} % }

\end{itemize} % }
