% vim: keymap=russian-jcukenwin
%%beginhead 
 
%%file 15_01_2022.stz.edu.lnr.lgaki.1.masterstvo_i_azart
%%parent 15_01_2022
 
%%url https://lgaki.info/novosti/masterstvo-i-azart-luchshie-pedagogi-shkol-iskusstv-lnr-demonstriruyut-talant-na-konkurse-v-akademii-matusovskogo
 
%%author_id lgaki
%%date 
 
%%tags donbass,konkurs,kultura,lgaki,lnr,lugansk
%%title Мастерство и азарт: лучшие педагоги школ искусств ЛНР демонстрируют талант на конкурсе в Академии Матусовского
 
%%endhead 
\subsection{Мастерство и азарт: лучшие педагоги школ искусств ЛНР демонстрируют талант на конкурсе в Академии Матусовского}
\label{sec:15_01_2022.stz.edu.lnr.lgaki.1.masterstvo_i_azart}

\Purl{https://lgaki.info/novosti/masterstvo-i-azart-luchshie-pedagogi-shkol-iskusstv-lnr-demonstriruyut-talant-na-konkurse-v-akademii-matusovskogo}
\ifcmt
 author_begin
   author_id lgaki
 author_end
\fi

Сегодня в главном творческом вузе Донбасса – Академии Матусовского проходит
второй, очный, этап Республиканского смотра-конкурса педагогического мастерства
среди преподавателей школ искусств. По итогам первого — жюри конкурса во главе
с первым проректором вуза Ириной Цой из 250 участников отобрало 100 лучших в
номинациях «Музыкальное искусство», «Театральное искусство», «Хореографическое
искусство», «Изобразительное искусство», «Творческий проект».

— Цель конкурса, который мы традиционно проводим уже не первый год при
поддержке Министерства культуры, спорта и молодежи ЛНР, — создание условий для
развития творчества преподавателей школ искусств, содействие обмену лучшим
педагогическим опытом, — рассказала руководитель Учебно-методического центра
Академии Матусовского, который координирует взаимодействие со школами искусств
всей Республики, Мария Чепрасова.

Первой на главную сцену музыкального корпуса ЛГАКИ вышла сегодня известный
музыкант, лауреат международных конкурсов, преподаватель Луганской музыкальной
школы № 1, педагог с 20-летним стажем, к слову, Юлия Челышева.

— Хочу держать себя в тонусе, не забрасывать инструмент! — ответила она на
вопрос о том, зачем участвует в конкурсе. — На конкурсах всегда волнуюсь,
особенно перед своими учениками.

Юлия Евгеньевна исполнила «Цыганку» Фрица Крейслера и «Испанский танец» Мануэля
де Фальи — одни из любимых своих произведений. И задала традиционно высокую
планку!

Инесса Яковлевна Мичко из Алчевской детской музыкальной школы № 1 имени Ивана
Алчевского преподает уже почти 42 года! Среди ее выпускников и нынешних
учеников – немало лауреатов конкурсов разных уровней. И тем не менее сегодня с
особым чувством она вышла на сцену Республиканского смотра-конкурса – в составе
педагогического инструментального ансамбля «Аккорд».

— Игра в ансамбле народных инструментов – это особенная красота и вообще моя
любовь, — призналась с улыбкой.

Преподаватель детской школы искусств № 5 Стаханова по классу «хореографическое
искусство» Елена Чикина совсем недавно праздновала значительную победу – стала
лучшей в номинации «Открытие года в области культуры и искусств»
Республиканского молодежного конкурса «Достояние Республики». А в прошлом году
заняла 1-е место на таком же смотре-конкурсе педмастерства, который проходит в
Луганске сегодня.

— На победу рассчитываете?

— Конечно! — ответила не раздумывая.

Хотя позже и она, и многие другие из участников подтвердили: победа – это
приятно, конечно, но главное, что заставляет их, педагогов выходить на сцену в
качестве конкурсантов — стремление продемонстрировать мастерство и… азарт!

Гала-концерт победителей Республиканского смотра-конкурса педагогического
мастерства среди преподавателей школ искусств и церемония их награждения
намечена на 28 января. И пройдет на главной сцене Академии Матусовского – в
корпусе на Красной площади, 7. Обязательно приходите посмотреть!

Итоги 2 этапа республиканского смотра-конкурса педагогического мастерства

Фото – Марина Машевски.
