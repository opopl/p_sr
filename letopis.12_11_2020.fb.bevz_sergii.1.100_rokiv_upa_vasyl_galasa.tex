% vim: keymap=russian-jcukenwin
%%beginhead 
 
%%file 12_11_2020.fb.bevz_sergii.1.100_rokiv_upa_vasyl_galasa
%%parent 12_11_2020
 
%%url https://www.facebook.com/spollloh/posts/205490764360942
 
%%author Бевз, Сергій
%%author_id bevz_sergii
%%author_url 
 
%%tags 
%%title 100 років від дня народження полковника УПА Василя Галаси (псевдо «Орлан»)
 
%%endhead 
 
\subsection{100 років від дня народження полковника УПА Василя Галаси (псевдо «Орлан»)}
\label{sec:12_11_2020.fb.bevz_sergii.1.100_rokiv_upa_vasyl_galasa}
\Purl{https://www.facebook.com/spollloh/posts/205490764360942}
\ifcmt
	begin_author
   author_id bevz_sergii
	end_author
\fi
\index[names.rus]{Галаса, Василь!Полковник УПА, псевдо «Орлан»}

Сьогодні виповнюється рівно 100 років від дня народження полковника УПА Василя
Галаси (псевдо «Орлан»). Він належить до незаслужено призабутих героїв
визвольної боротьби. 

Народившись у звичайній селянській родині на Тернопільщині, юнак змалку
вирізнявся патріотизмом. У підлітковому віці Василь створює гуртки самоосвіти,
де навчає однолітків історії України. Уже в 17 років він поповнює лави ОУН,
жадаючи вибороти краще майбутнє для своєї Батьківщини. 

\ifcmt
pic https://scontent.fiev6-1.fna.fbcdn.net/v/t1.0-9/124849464_205490671027618_2743175634491648417_n.jpg?_nc_cat=103&ccb=2&_nc_sid=8bfeb9&_nc_ohc=p1ARnHk3iKIAX9qwcDi&_nc_ht=scontent.fiev6-1.fna&oh=692a9e56ab23b733d07d8751c767c412&oe=5FE32FCF
caption Полковник УПА, Василь Галаса
\fi

Ставши на революційний шлях, Василь Галаса швидко здобув авторитет завдяки
своєму завзяттю та мужності. Його не зламали польські та московські катівні.
Уже в 1942 він очолює Тернопільську округу ОУН, в 1943 --- Перемишльську, бере
активну участь в розбудові відділів УПА. Найяскравіше «Орлан» проявив свої
здібності на посаді заступника крайового провідника ОУН на Закерзонні. Він
розгорнув активну пропагандивну роботу, став співорганізатором низки блискучих
закордонних рейдів УПА до Чехословаччини. На інформаційному фронті війни Василь
Галаса досягнув неабияких успіхів. Завдяки його роботі тисячі іноземців
проникнулись симпатіями до української визвольної боротьби. Саме з  ініціативи
Василя Галаси було проведено славетний рейд сотень УПА на Захід 1947, який
закінчився проривом залізної завіси. 

Після завершення депортацій українців із Закерзоння Василя Галасу призначили
крайовим провідником ОУН Північно-західних українських земель. Він проявив
непересічні лідерські якості, керуючи підпіллям в нерівному протистоянні з
червоними окупантами. Став співорганізатором рейду УПА до Прибалтики. 1953 р.
через зраду Василь Галаса опинився в руках ворога. До 1960 перебував у неволі,
згодом знаходився під пильним наглядом окупантів.

Василю Галасі судилося дожити до відновлення незалежності України. В нових
реаліях він активно долучився до популяризації визвольної боротьби та руйнації
радянських міфів. Помер у жовтні 2002 на 82 році життя.

Пам’ятаємо, пишаємось!
