% vim: keymap=russian-jcukenwin
%%beginhead 
 
%%file 17_03_2021.fb.fb_group.story_kiev_ua.1.sobaki.cmt
%%parent 17_03_2021.fb.fb_group.story_kiev_ua.1.sobaki
 
%%url 
 
%%author_id 
%%date 
 
%%tags 
%%title 
 
%%endhead 
\subsubsection{Коментарі}

\begin{itemize} % {
\iusr{Любовь Линник}

\ifcmt
  ig https://scontent-frx5-2.xx.fbcdn.net/v/t39.1997-6/s168x128/12057035_446661978873394_934856931_n.png?_nc_cat=1&ccb=1-5&_nc_sid=ac3552&_nc_ohc=0hf2SHcaUE8AX8Qb5fz&_nc_ht=scontent-frx5-2.xx&oh=00_AT-HR-F3N6PTA24_OR_a_8bGMk4a9sErUozjB1Vin_gCuQ&oe=61BCDF02
  @width 0.1
\fi

\iusr{Irena Visochan}
Так трогательно! Спасибо! @igg{fbicon.hearts.two} 

\iusr{Наталья Писная}

Замечательный рассказ! Так сразу и я вспомнила о своих пуделях, тоже проживших
с нами десятилетие и так много приятных минут радости нам подаривших !

\iusr{Галина Кардаш}
Спасибо. Очень добрый рассказ собачника))

\iusr{Всеволод Шевчук}
Петрику ти пам'ятаєш як вони заплигуввли до мене на руки коли я приїжджав в гості?

\begin{itemize} % {
\iusr{Петр Кузьменко}
\textbf{Всеволод Шевчук} було!

\iusr{Всеволод Шевчук}
\textbf{Петр Кузьменко} чудові були час, а в подальшому будуть, ще кращі!
\end{itemize} % }

\iusr{Lesik Machynsky}

Уважаю серьезных собачников. Всегда с симпатией присматривался к безстрашным и
энергичным таксам. У самого были и маленькая дворняжка, в детстве, и лайка
Полкан, и бассет Патти-Кальмина фон Йоги-Амфитрион с дочерью, и овчарка
полукровка Малыш, гроза котов, и приблукавшийся гигантский бордосс Рекс...
Последнее время, один за другим мопсы. Абсолютно не собачье поведение,
психология и повадки.

\iusr{Ludmila Teslenko-ponomarenko}

Потрясающий рассказ! Спасибо, Петр! Я на собаку так и не решилась, но ангорская
белая кошка Соня прожила с нами 17 лет.

\iusr{Анна Невская}

\ifcmt
  ig https://scontent-frt3-1.xx.fbcdn.net/v/t39.1997-6/p480x480/28775943_155886991792975_937989508480106496_n.png?_nc_cat=102&ccb=1-5&_nc_sid=0572db&_nc_ohc=sICZT74X2XwAX8GD_n9&_nc_ht=scontent-frt3-1.xx&oh=00_AT-mtPSDwN_wklokoSDf7Y-Hmrc1Bmf4L1Vkjky5L_DzXg&oe=61BD3A9F
  @width 0.2
\fi

\iusr{Анна Невская}
Спасибо большое, Пётр за такой интересный рассказ о братьях наших меньших.

\iusr{Светлана Волошина}

моя самая первая такса была приобретена в УООРе. Это и неудивительно. В то
время таксу можно было только в охот.обществах приобрести. И Веру Андреевну
Масс помню, бывала у неё дома, в гостях, воднокомнатной квартире на первом
этаже на Печерске. И на притравчные станции в Бучу ездили, и на Турухановом
острове помню нору, тоже туда приезжали. Там моя такса взяла не по месту
лису..... Норный мастер расслабился, закурил аппетитно... осенний день,
солнышко, красота.... вот и потерял бдительность. \enquote{Содрали} с меня тогда за
лису аж 70 американских рублей...

\ifcmt
  ig https://scontent-frt3-1.xx.fbcdn.net/v/t1.6435-9/161674630_3646595815469130_4485721917423633934_n.jpg?_nc_cat=106&ccb=1-5&_nc_sid=dbeb18&_nc_ohc=QRwJ-S7bu_0AX-PXXq-&_nc_ht=scontent-frt3-1.xx&oh=00_AT_HoYVptvGialKSwKwzUhifyioD0cDw4XFKrC8s_stLAQ&oe=61DE8CED
  @width 0.5
\fi

\begin{itemize} % {
\iusr{Светлана Волошина}

вот ещё одно архивное фото.... норный мастер со своей воспитанницей..... Ныне
на этом месте - многоквартирные и частные дома, всё застроено.

\ifcmt
  ig https://scontent-frx5-1.xx.fbcdn.net/v/t1.6435-9/161621617_3646700368792008_2836057526281897760_n.jpg?_nc_cat=110&ccb=1-5&_nc_sid=dbeb18&_nc_ohc=WewJ5D3yMI8AX9PM7fJ&_nc_ht=scontent-frx5-1.xx&oh=00_AT__NwmIN0UAij_VyYV4S1jKajhccTkbP7g6alEV44HK4A&oe=61DCFF76
  @width 0.4
\fi

\end{itemize} % }

\iusr{Semyon Belenkiy}
С душой написано, спасибо

\ifcmt
  ig https://i2.paste.pics/d68b73f042bef16a625080d3dbe9cede.png
  @width 0.1
\fi

\iusr{Нелли Кузьменко}

\ifcmt
  ig https://scontent-frx5-2.xx.fbcdn.net/v/t39.1997-6/p240x240/851592_490565377723574_924545120_n.png?_nc_cat=1&ccb=1-5&_nc_sid=0572db&_nc_ohc=WC_AOwZHBXcAX_jVdba&tn=lCYVFeHcTIAFcAzi&_nc_ht=scontent-frx5-2.xx&oh=00_AT8JOn9wClGYOnm4_2IgV7ZiNwmH5mT-J8D1YcomSE_i5g&oe=61BD5E6F
  @width 0.2
\fi

\iusr{Раиса Карчевская}
Пётр!
Спасибо большое за потрясающий рассказ о Ваших собаках

\iusr{Дарья Кирносова}
Ужас. Не понимаю подобных развлечений.
Умиляют комменты про \enquote{трогательность}...
Притравочные станции и подобные натаскивания собак - живодерство в чистом виде.

\begin{itemize} % {
\iusr{Петр Кузьменко}
\textbf{Дарья Кирносова} 

живодёрство в чистом виде это стаи беспризорных животных. Это уничтожение собак
и кошек, как брошенных \enquote{сердобольными дюдьми} так и домашних. Более
ответственных собаководов, чем заводчики охотничьих собак трудно найти. А
деятельность притравочных станций была чётко регламентированна. Звери для
притравок и натаски разводились специально и жили лучше их диких собратьев.
Теперь по требованиям \enquote{защитников природы} станции закрыты. Собак охотники
обучают в лесу сами, теряя массу зверья подранками, калеча и губя собак.
Недостаток поголовья норных собак в Украине обусловил небывалую вспышку
бешенства плотоядных, болезни опаснейшей для человека и животных. Основными
разносчиками бешенства являются лисы и енотовидные собаки. Когда-то они были
объектами норной охоты... @igg{fbicon.face.sad.but.relieved} 

\begin{itemize} % {
\iusr{Дарья Кирносова}
\textbf{Петр Кузьменко}, ожидаемый ответ.
Особенно про бешенство. Стандартные отписки охотников.
В нормальных странах с бешенством борются превентивными мерами, а н как не отстрелом.
Про замечательные условия жизни притравочных животных - пишите сказки дальше, кто-то да поверит...
Отвечать больше не буду...
За все каждый когда-то ответит... По заслугам...

\iusr{Иванна Зацнова}
\textbf{Петр Кузьменко} В клетках покалеченые животные жили лучше чем на воле! Жестокий циник.

\iusr{Нелли Кузьменко}
\textbf{Иванна Зацнова} 

Вы бывали на притравочных станциях? Видели животных здоровых, привитых и
откормленных? Они обучены работать с собаками. Есть фото нормастера с
притравочной лисой выше в комментариях. Если Вы так хорошо в этом разбираетесь,
ответьте мне, пожалуйста, на простой вопрос какова продолжительность жизни
дикого зверя в нашей нынешней природе? На нашей притравочной станции лисовин
прожил пять лет и был отправлен на заслуженный отдых в зооуголок охотхозяйства
в вольер. Так в чьих комментариях и поступках больше цинизма?

\end{itemize} % }

\iusr{Петр Кузьменко}
\textbf{Дарья Кирносова} ожидаемая реакция диванного \enquote{защитника природы} @igg{fbicon.face.sad.but.relieved} 

\end{itemize} % }

\end{itemize} % }
