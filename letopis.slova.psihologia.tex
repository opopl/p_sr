% vim: keymap=russian-jcukenwin
%%beginhead 
 
%%file slova.psihologia
%%parent slova
 
%%url 
 
%%author 
%%author_id 
%%author_url 
 
%%tags 
%%title 
 
%%endhead 
\chapter{Психология}
\label{sec:slova.psihologia}

%%%cit
%%%cit_head
%%%cit_pic
%%%cit_text
Эта мысль пришла на ум после реакции на матч Украина-Швеция. Пора бы создать
Общественный Совет Национальной \emph{Психологической} Безопасности) Пригласить
в состав маститых \emph{психологов}, психотерапевтов, психиатров и
невропатологов. А в качестве экспертов - социологов и культурологов. И дать им
право вводить персональные санкции с целью временной изоляции от общественности
- в случае откровенной болтовни, тупого хайпа и саморекламы на чужих успехах,
параноидальных заявлений, идиотских оценок и призывов, шизофренических
инициатив.  Не взирая на лица!  А то - беда...  p.s. Тоже нервы не выдерживают
после услышанного и прочитанного о футбольном матче Украина-Швеция.  А ребятам
все равно спасибо за победу)!
%%%cit_comment
%%%cit_title
\citTitle{Пора бы создать Совет Национальной Психологической Безопасности / Лента соцсетей / Страна}, 
Андрей Ермолаев, strana.ua, 01.07.2021
%%%endcit

%%%cit
%%%cit_head
%%%cit_pic
%%%cit_text
Вклад Маркса в \emph{психологию} долгое время оставался неизученным, а во многом он
остаётся таковым по сей день. Принято считать, что Маркса интересовало общество
в целом, а не отдельный человек. Фромм назвал несколько причин такого
пренебрежения \emph{психологическими} взглядами Маркса.  Во-первых, Маркс не успел
изложить свои \emph{психологичекие идеи} в систематической форме. Они разбросаны по
всем его трудам — от «Экономико-философских рукописей» 1844 года до последних
страниц «Капитала», в третьем, посмертном, томе которого говорится об условиях
труда, «наиболее достойных человеческой природы и адекватных ей». Чтобы понять
\emph{психологические} взгляды Маркса, их нужно свести вместе.  Во-вторых, наиболее
распространены вульгарные представления, согласно которым Маркс якобы
интересовался исключительно экономическими явлениями, а исторический
материализм считает главной движущей силой человека стремление к экономической
выгоде. Это крайне искажает реальные представления Маркса о человеке и мешает
осмыслить его вклад в \emph{психологию}
%%%cit_comment
%%%cit_title
\citTitle{Карл Маркс – психоаналитик и религиозный экзистенциалист / Статьи}, 
Александр Карпец, fraza.com, 08.05.2018
%%%endcit

