% vim: keymap=russian-jcukenwin
%%beginhead 
 
%%file 21_11_2014.fb.fb_group.patrioty_ukrainy.1.godovschina_evromaidan
%%parent 21_11_2014
 
%%url https://www.facebook.com/groups/patriotofukraine/posts/1498399290427639
 
%%author_id fb_group.patrioty_ukrainy
%%date 
 
%%tags godovschina,maidan2,obschestvo,ukraina
%%title Гірка річниця - Пройшов рівно рік з початку Євромайдану
 
%%endhead 
 
\subsection{Гірка річниця - Пройшов рівно рік з початку Євромайдану}
\label{sec:21_11_2014.fb.fb_group.patrioty_ukrainy.1.godovschina_evromaidan}
 
\Purl{https://www.facebook.com/groups/patriotofukraine/posts/1498399290427639}
\ifcmt
 author_begin
   author_id fb_group.patrioty_ukrainy
 author_end
\fi

Гірка річниця

Пройшов рівно рік з початку Євромайдану. Всього лише рік, а здається так багато
часу пройшло з подій на Майдані Незалежності. Якою величезною кров'ю далася ця
Революція Гідності для українського народу і як раділи люди поваленню режиму
донецької олігархії в органах управління країною. І що ми маємо сьогодні на
річницю Євромайдану?

\ifcmt
  ig https://scontent-frt3-2.xx.fbcdn.net/v/t1.18169-9/10622885_313580605495538_417885339806403174_n.jpg?_nc_cat=103&ccb=1-5&_nc_sid=825194&_nc_ohc=13JzzyO6wlUAX9Usroc&_nc_ht=scontent-frt3-2.xx&oh=144b1cfbf7b642466f3aa80f9293125c&oe=61BC4183
  @width 0.4
  %@wrap \parpic[r]
  @wrap \InsertBoxR{0}
\fi

Давайте проведемо паралелі - що було і що стало.

Рік тому ми жили в єдиній і мирній країні, яка переживала перманентну
економічну та політичну кризу, люди скаржилися на несправедливість і
олігархічну владу, обговорювали переваги і недоліки вступу України в союз з
Європою чи з Росією.

Рік по тому ми живемо в поділеної і воюючої країні, яка котиться в прірву
економічних і соціальних проблем, вихід з якої не знайдеться в найближчі роки.
Країна повністю змінилася і не в кращий бік ...

Політики Майдану виявилися фокусниками-шарлатанами, які пускають перед людьми
мильні бульбашки - обіцянки, жодный з яких не судилося збутися. І Майдан,
завдяки цим людям виявився пустушкою, не більше ніж спогадом для людей про
битву проти тиранії влади, яка призвела до катастрофічних наслідків. Якщо
зараз, через 365 днів після початку змін в політичному курсі держави ми маємо
такі невтішні результати, то що буде далі ?! Вже зараз люди думають про третій
Майдан, щоб скинути нинішню, нову владу ...

Жодна революція не обходиться без жертв, але коли жертва - ціла країна - це
рідкість. По суті, Майдан справив позитивний вплив лише на українське
суспільство, яке згуртувалося разом і виступило проти сформованого політичного
режиму. Але що ми маємо зараз ... Війна на Сході, де вже сформувалося дві
окремих незалежних народних держави, Крим більш не належить Україні, а увійшов
до складу РФ. Країна переживає важку економічну кризу, значна частина населення
живе в бідності, а гривня щоденно продовжує своє повільне падіння вниз, по всій
країні відбувається зупинка виробництва і економічних процесів ... Економіка,
політика, культура ... все зараз знаходиться в глибокій кризі.

Згадайте, чого ви очікували від революції? Чи цього хотіли?

Не змінилося нічого - корупція у владі як була, так і є, все ті ж корупційні
схеми, все ті ж способи викачування державних коштів з бюджету.

Вибори не поміняли нічого, а в черговий раз довели, що політична система
держави давно застаріла, і саме з неї треба розпочати реформування суспільства.
Хоча в Раду і пройшли промайдановскі партії, але міжусобна боротьба за
лідерство в коаліції і місця та портфелі в майбутньому уряді довела, що в першу
чергу політики переслідують свої інтереси і інтереси бізнесу, який привів їх до
парламенту. Майдан для них був лише одним із способів досягнення своїх цілей, а
саме потрапляння в Раду.

Для мене Євромайдан виявився розчаруванням саме через дії влади, тому що
величезна кількість людей боролося і віддало свої життя за ідею, яку політики,
що стояли з людьми пліч о пліч на Майдані, просто розтоптали.

\#Майдан \#Євромайдан \#РеволюціяГідності

\ii{21_11_2014.fb.fb_group.patrioty_ukrainy.1.godovschina_evromaidan.cmt}
