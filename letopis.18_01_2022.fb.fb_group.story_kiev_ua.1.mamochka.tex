% vim: keymap=russian-jcukenwin
%%beginhead 
 
%%file 18_01_2022.fb.fb_group.story_kiev_ua.1.mamochka
%%parent 18_01_2022
 
%%url https://www.facebook.com/groups/story.kiev.ua/posts/1842856412577818
 
%%author_id fb_group.story_kiev_ua,novickij_vladimir
%%date 
 
%%tags kiev,kievljane,mama,pamjat,semja
%%title История жизни моей мамочки
 
%%endhead 
 
\subsection{История жизни моей мамочки}
\label{sec:18_01_2022.fb.fb_group.story_kiev_ua.1.mamochka}
 
\Purl{https://www.facebook.com/groups/story.kiev.ua/posts/1842856412577818}
\ifcmt
 author_begin
   author_id fb_group.story_kiev_ua,novickij_vladimir
 author_end
\fi

Сегодня я хочу Вам рассказать историю жизни моей мамочки.     

Родилась она в Киеве на улице Дмитриевской № 42 в доме, который до революции
принадлежал их семье. Дед мой был из купеческой семьи и владел фабрикой по
изготовлению настенных часов с боем, которую ему подарили на свадьбу родители.
В семье было пятеро детей. Старший брат Саша, который погиб при авиакатастрофе
ещё до войны. Он был уже лётчиком.  Мама 1910 года рождения, затем брат Коля,
которого расстреляли немцы в первые же дни захвата Киева немцами по доносу
соседки и ещё сестра Лиза и брат Миша, которые умерли сосем маленькими, от
болезней, которые тогда ещё не лечили. 

Мама моя в детстве занималась балетом и
хорошо играла на фортепьяно, которое стояло у нас в гостиной. Кроме того она
была очень красивая и за ней ухаживали многие молодые люди. Но выбрала она,
кстати  как и я,  себе половинку по соседству. Я напротив, прямо через
дорогу в № 21 мою жену Лилю, которую сейчас  уже знает, по предыдущим моим
рассказам   половина участников нашего клуба « Киевские Истории».  А мама
выбрала мне папу буквально  рядышком, даже через дорогу переходить не надо
было, в № 38, который вообще по понятиям того времени считался одним двором,
так как между этими домами не было забора. Дед мой был не в восторге от её
выбора. Папа на тот период был сиротой и как он говорил — голодранцем. Его
отец — умер в1912 году, когда папе было всего 7 лет. Его дедушка и бабушка,
которые были очень знатного происхождения и богатыми людьми и на содержании
которых они с бабушкой находились, - после революции уехали в Париж, чтобы,
как они считали  переждать эти беспорядки в спокойном Париже и вернуться,
когда всё успокоится. Уезжали они, как думали, не навсегда, поэтому и не
взяли с собой внука с невесткой, так как он учился в гимназии. Но время
распорядилось по другому. Разлука оказалась долгой. Вначале они помогали
материально на жизнь бабушке и папе. Бабушка, воспользовавшись тем бардаком,
который был в то время в  Украине и в частности в Киеве — смогла сменить
метрику папе, чтобы скрыть его дворянское происхождение. Теперь он был не их
дворян, а из крестьян Ровенской области, а так же сменила ему год рождения с
1905 на 1907, почему сменила  и год не знаю, видимо была похожая запись на
мальчика в реестрах по рождению Ровенской области, чтобы было всё более
достоверно и запутать власти ещё больше. . Власти в Киеве менялись чуть ли не
ежемесячно, а то и по несколько раз на месяц. Большевики, Рада,
Зелёные,Махновцы, немцы, затем снова большевики и так далее и тому подобное.
Скажу , что папа всю свою оставшуюся жизнь с ужасом заполняя бесконечные
анкеты, которые так любили при Советской власти, чтобы контролировать всё и
всех-  боялся что этот обман раскроется и его репрессируют.  Но Бог миловал.
Всё обошлось. Правда дедушка с бабушкой перестали посылать денежную поддержку
из Парижа - они боялись этим ему навредить. А потом вообще связь с ними
пропала, и мы так и не знаем их дальнейшую судьбу. Когда мы с Лилей были в
Париже, то прошлись по нескольким «Русским» кладбищам, но найти могилки с
фамилией Новицкие не смогли.  Бабушка моя, хоть была и образованная, закончила
Институт благородных девиц, но была очень не практичная  и не могла
приспособиться к жизни при новой власти, поэтому папе пришлось рано пойти
работать. Он устроился на стройку — учеником каменщика. Помню как он мне
рассказывал, как таскал на «козе» ( это такое приспособление, которое
надевалось на плечи и на него накладывали кирпичи, чтобы можно было их носить
на верхние этажи. Подъемных кранов же в то время не было. Тяжёлая я Вам скажу
работа, но благодаря этому он смог закончить вечерний институт  Рабфак, (
рабочий факультет) - так  тогда они назывались. Службу в армии он проходил
ежегодно по несколько месяцев  на  местных  курсах ( сборах)  На очередные
военные сборы его призвали  в мае 1939 года, когда мне было 5 месяцев, Эти
сборы растянулись на целых 6 лет. Так как Первого сентября 1939 началась Вторая
Мировая Война и увидел я его впервые уже после войны, когда учился в
Хореографическом Училище на углу улиц Воровского и Тургеневской.  Я понимал,
когда впервые его увидел, что это мой папа, по фото  узнал, но  ещё долго
стеснялся  к нему подойти. Так война лишила нас общения, такого  нам
необходимого.

Жили мы буквально в двух шагах от знаменитого «Евбаза», который был намного
больше не менее знаменитого Одесского «Привоза». На нём можно было купить всё
от продуктов, гвоздей и даже пулемёта. Я не шучу, моя мама, которой  во время
окупации Киева немцами приходилось ежедневно бывать на нём, что бы заработать
на жизнь, всё видела  своими глазами. Даже как то призналась, что помогала
партизанам  в одной подобной сделке. Была посредником продажи  венгерским
офицером ( они были, по её рассказам, не такие как немцы, во всяком случае те,
которых она знала)  партии оружия представителю партизан. Коммерцию война не
отменяла. « Война — войной, а коммерция — коммерцией». Правда она сказала, что
очень сильно после этого испугалась и больше в подобных опасных сделках не
участвовала. Ведь у неё на руках были мы -двое сыновей и родители, а также
папина мама, и всех  надо было кормить.    Как она это делала?

А очень просто, сейчас правда люди для этого кончают Университеты по изучению
«маркетинга». А что собственно маркетинг — это знание реальной стоимости
товара, с целью купить дешевле и продать дороже, чтобы получить прибыль, вот
этот « маркетинг» её заставила изучить сама  жизнь +природная сообразительность
и талант, Но для того , чтобы купить, кроме таланта, нужны как минимум деньги,
а их , как раз у неё и не было. Вначале, сколько могла всё продавала из дому,
но это очень быстро кончилось.  Жила рядом женщина у которой мама  брала деньги
под процент. Если утром, перед базаром она брала 1000 рублей ( условно), то
возвращаясь с базара должна была отдать 1200. И она это делала, кроме того
приносила ещё и сумку еды. Кроме чистой купли и последующей продажи, но уже
дороже, мама часто покупала вещи, которые можно было переделать ( перешить) .
Тогда вся наша коммуна вооружившись бритвочками  садилась и распаривала эту
вещь , а затем мама удалив  всё лишнее, перелицовывала её и получалась почти
«новая» вещь, но немного меньших размеров. Труд был огромный , но овчинка, как
говорится стоила выделки. Это давало нам возможность жить, а иногда я получал
под подушку конфетку или пирожок. Но для этого я должен был пропеть «колядин,
колядин я  у батька один, по колiнцi кожушок, дай мамо пирожок». Конечно,
поскольку она ежедневно была на базаре она знала всю его подноготную и знала
воришек всех мастей, которые орудовали на базаре. Был на Евбазе один громадного
вида человек  с  ещё более громадной головой — звали его  «Лампадка» Он
постоянно разбивал, причём именно головой, прилавки и рундуки тех торговок,
которые не платили ему дань на выпивку .Во время возникающей паники и шума —
вольготно  орудовали местные базарные воришки, возможно он работал с ними
заодно. Часто они пытались ей продать по дешевке сворованное, но мама всегда
отказывалась брать ворованные вещи на продажу, хотя был соблазн легко
заработать. Во первых ей было неприятно помогать воришкам, а во вторых она
понимала, что взяв только что украденную у кого то вещь  она  могла
встретиться  с человеком у которого эту вещь украли и что бы она ему сказала?

У мамы был очень большой риск попасть в облаву, которые немцы очень часто
проводили на базарах, где они вылавливали молодых людей для отправки в Германию
на работы. Маме было в ту пору 31 год и её вполне могли отловить и отправить.
Как она смогла избежать этого. Рядом с базаром на Дмитриевской №38 жила папина
мама, куда идя на базар заводила меня мама. Как только бабушка слышала , что
начиналась облава, а это определялось легко. Немцы на грузовиках и мотоциклах
окружали базар, это было хорошо слышно у неё дома, она брала меня за руку и на
костылях ( у ней была сломана нога) шла к Босяцкому магазину в условленное
место, к которому подходила и мама. Маленький Вовочка между немцами проходил в
объятия своей мамочки и она с ним выходила из облавы. Дело в том, что с детьми
они не брали. А вот мамину куму  тётю Лиду, у которой был грудной Женечка
забрали и отправили  в товарном вагоне в Германию. Сколько она им не говорила,
что у неё грудной ребёнок — ничего не помогало.  Ей повезло, что поменялся  на
ближайшей станции конвоир и она вынув грудь  начала ему цыркать  молоком и
говорила «Киндер, киндер» он  сжалился и отпустил её. Так через несколько дней
она вернулась в Киев.   Мы уже перестали и надеяться её увидеть и мама стала
кормить Женечку кашками и козьим молоком.

По вечерам вся  наша коммуна,  собиралась дома, после 5 вечера был
комендантский час и рассказывали о том, как прошёл день, а я сидел и тихо
слушал. Вот то, что я запомнил и стараюсь Вам рассказать.

Память моя конечно перепутала всё то, что я сам видел, что мне потом
рассказывали дед, мама и то, что я понял потом сам.

Теперь расскажу как мы жили после освобождения Киева от немцев. Ну во первых мы
возвратились из дома кумы Лиды на Татарке,   в свою квартиру по улице
Дмитриевской 42, где жили до нашего выселения немцами, после расстрела маминого
брата Коли.  Первые дни после освобождения Киева запомнились тем, что
заработала почта и люди стали бояться почтальонов. Как только почтальон заходил
во двор — наступала напряжённая тишина. Все смотрели к кому он  идёт. Все
боялись получить похоронку — известие о гибели близких людей. А таких похоронок
приходило много. Ведь почти три года люди не имели сведений о судьбе своих
близких. Живы они или нет К нашему счастью мы получили письмо от папы, где он
писал как он нас любит, что жив и здоров . Мы стали получать, как семья офицера
— паёк и другие льготы. Жить стало легче. Все люди тогда были дружны и помогали
друг другу. Если улицу на которой ты жил, считали домом, то двор  -  это вовсе
была почти квартира, её первая большая комната.

Как театр начинался с вешалки, двор начинался с ворот и закрывающейся на ночь
калитки. За воротами тянулся подъезд, стены которого  украшали исполненные
углём или мелом рисунки, не всегда пристойные, а также объявления о заседаниях
домового комитета. 

И лишь за подъездами начинались дворы — простые и проходные. Больше всего
конечно было простых дворов. Проходных было меньше, но они обладали прелестью,
что знающий их, мог  зайти на одной улице и выйти уже на другую. Их называли
«сквозняки». Когда говорили, что он или она сделали  кому то  сквозняк, всем
было ясно, что кто-то от кого- то смылся. Ко двору относилось всё, что не
принадлежало  непосредственно квартире. Так двором  были сараи , тянувшиеся по
периметру двора, где хранились дрова и уголь на зиму.  В погребах засаливали
бочки с капустой и держали  на зиму картошку.  Многие в них держали курочек,
кроликов и даже свиней. На чердаках, которые также были общими, сушили бельё в
дождливую погоду. Чердаки были запретным и вожделенным местом наших
мальчишеских игр, их «штабом» и наблюдательным пунктом. Кроме того мы держали
голубей,  почти в каждом дворе была своя голубятня. 

Центром двора  всегда был водопроводный кран — колонка. Возле неё полоскали
бельё, купали замурзанных детей, чистили рыбу,  чтобы дома не воняло. Зимой,
чтобы колонка не замёрзла её укутывали, чтобы не замёрзла вода, она должна была
постоянно течь. И если кто - то по невнимательности или забывчивости кран
закрывал и вода замерзала, то ему нужно было нанимать слесаря, который паяльной
лампой его разогревал.

Но возле крана собирались не только  воды ради — упаси Боже! Это был клуб,
судилище, форум, своеобразный вариант знаменитого Гайд-парка, где говорили обо
всём и обо всех, не становясь даже на символическую трибуну.

Во дворе был прекрасный обычай угощать соседей, которые были, как члены
большой семьи, своими коронными блюдами. Никого совсем не удивляло, когда одна
соседка могла попросить другую, приготовит на именины, то блюдо, которое
только она умела так вкусно приготовить. Или тётя Надя говорила тёте Фане
-  «Фаничка Ваш Бог не обидится, если Вы скушаете кусочек этой чудесной пасхи».
Если сейчас соседи часто не знают друг друга, то раньше во дворах они выручали,
помогали друг другу, вместе радовались и грустили.  Гордились успехами  своих
детей и кочевал по по свадьбам и именинам единственный на весь двор старенький
патефон и обязательная толстая длинная доска, которую укладывали на два стула
вдоль праздничного стола, который в тихие и тёплые вечера накрывали прямо во
дворе.. На этих вечерах гости сидели не семьями , а домами, верней — дворами,
потому как чаще всего говорили не «из нашего дома» а из «нашего двора». Дворы и
сами являли собой независимую колоритную картину, специфически озвученную с
утра и до вечера. Утро начиналось пронзительными криками молочниц:
«МОЛОКО-О-О-О!!!. Затем появлялись уличные мастера, нищие с гармошками и без,
старьевщики. Паять лудить вёдра, чайники. Липкая бумага. Купите и можете спать
без мух! Точить ножи ,ножницы, мясорубки и так далее. Сольные партии
гастролёров звучали на фоне многоголосого дворового хора: «Тётя Соня вы были
на Евбазе, почём сегодня мясо?» « Наденька  идите быстрей, в Босяцком выбросили
головы и ножки». Зиночка сбегай детка и посмотри — открыли ли уже керосин?
Дядя Гриша киньте мне пару спичек. Фира Борисовна- это не у Вас горит тряпка?
Мадам Новицкая, который уже час? Во дворах прозвища давали  с детства и они
сопровождали жильца всю его жизнь. И эти клички так прилипали к тем кому их
дали, что зачастую  некоторые родители   звали  домой детей тоже по кличкам.
Двор казался мне таким большим, мы в нём играли в футбол, цурки-палки, коцы и
другие игры,  а так же давали представления для соседей. Когда же я приехал в
Киев и зашёл на то место, где  был наш двор, я не мог понять, как мы
умудрялись всё это делать. Он мне показался маленьким. Из детских воспоминаний
запомнились также наши набеги на сады на примыкавшей к нам Тургеневской улице.
Чтобы нас не покусали собаки, мы длинными палками (шополками) с приделанными к
их концам мешочками умудрялись с сараев  обрывать груши и яблоки  и лакомиться
витаминами, которых так не хватало нашим детским организмам.

Но вернёмся к рассказу о маме. Она была у меня очень строгой, во всяком случае
в детстве я так о ней думал и иногда сердился на неё за её не сговорчивость. И
как же я теперь понимаю, что она  делала это только ради моего же блага.  Когда
я закончил школу, она заказала мне на выпускной вечер костюм и сказала, как
отрезала «Если я не пойду дальше учиться, буду шалопаем, то - это  будет
последнее, что она для меня сделала, никакой помощи мне в дальнейшем  не будет.
Поверьте, что так оно и было бы, я в этом не сомневался , так как её слово у
нас в семье было — законом. Несмотря на то, что папа после войны работая на
заводе Большевик, занимал должность Зам Директора по снабжению  и получал не
плохую по тем временам  зарплату и кроме того пенсию — как инвалид войны, семью
в основном содержала мама.  У неё было много клиентов, которым она шила. Кроме
того, она могла в воскресенье пойти на толкучку и за день заработать больше
чем папа  получал за месяц , как говорится — беда научила.

А как она относилась к моей жене Лиле, с которой у неё дружба завязалась ещё
задолго до нашей свадьбы. Характерно и то, что прожив в одной квартире, а
главное  на одной кухне 5 лет на Дмитриевской а затем 17 лет на Русановке, пока
мы не построили себе кооператив , я не помню случая, чтобы они не только
поругались, но и  просто поспорили, хотя, сами понимаете , конфликтных ситуаций
за такой долгий период было более чем  достаточно. Она  никогда не вмешивалась
в наши отношения и никогда не вставала на сторону кого то из нас в наших
спорах..  Всегда говорила « Разбирайтесь сами,  вы взрослые люди. Я вам тут не
советчица»  Это во многом предрекло то, что мы сохранили нашу любовь и в
прошлом году отметили 62 годовщину нашей свадьбы. Сколько я знаю случаев, когда
вмешательство  родителей  разрушало семьи.  А как она её обшивала. Она всегда
говорила Лиле — я хочу , чтобы ты у нас была красивей и лучше  всех одета. Как
только появлялся новый журнал « Бурда» с новыми модами они садились и выбирали,
что она пошьёт  ей новенькое.  Она ей шила не только платья , юбки, блузки  но
также  шубки и шляпки.  Такие прекрасные отношения, которые сложились в нашей
семье   между невесткой и свекровью  - говорят о том, что моя мама  была  очень
мудрым и добрым человеком, при всей её строгости и требовательности. Впрочем
она никогда не считала её невесткой а скорей считала дочерью. Дело в том что в
нашей семье не было девочек, если не считать маминой сестры Лизочки, которая
умерла младенцем. У папы не было сестёр, у них первенцем  был сын  Жора и они
ожидали, что   вторым ребёнком  будет  девочка, но родился я. У брата Жоры было
два сына и у нас первенцем был  тоже сын Миша. Поэтому , когда у нас  родилась
девочка, наша дочь Таничка ( первая девочка  в семье Новицких  за  70-80 лет)
тут уж мама отвела душу. Понимаете, ей хотелось передать весь свой женский опыт
и умения  девочке  и она, забегаю вперёд — это наконец осуществила. Наша дочь
Таня — по характеру, складу ума,  умению всё сама делать и  жизненному опыту —
точное повторение моей мамы. Но в то время всего этого ещё  не было и она
видела в Лиле — дочь а не невестку. А Лиля ей отвечала взаимностью — видела в
ней не свекруху а мать.  Такой расклад устраивал всех и особенно меня.

Кроме того она многому меня научила, за что я ей безмерно благодарен по сей
день.  Научила меня не отчаиваться и уметь собираться в трудную минуту и
находить выход из самых, казалось бы, безвыходных положений. Никогда не
отчаиваться, научила думать и не быть наивным. Научила, что прежде  чем что-то
сделать, надо хорошо всё обдумать, как это отразится на  твоей судьбе в
дальнейшем и только потом принимать решения.  Научила разбираться в людях. Она
всегда говорила мне и это подтвердила вся моя последующая  жизнь, что не всякий
кто красиво говорит — обязательно хороший человек. А тот, кто явно льстит тебе
—  если не враг, то уж   точно плохой человек и его надо опасаться. Научила
уважать старших и прислушиваться к их мнению. Научила быть благодарным, не быть
Фомой, не помнящим  добра и родства.  И ещё многому, многому другому. Кстати
она научила меня  хорошо готовить пищу,   мне кажется у меня она неплохо
получается, но это не мне судить. Это в последствии  — стало моим хобби.

В самые тяжёлые  минуты жизни, которых у меня к сожалению — было много я
приходил и советовался с мамой. Даже тогда, когда её не стало я  утром (
обязательно ранним, чтобы никого из посторонних не было и нам не помешал) ехал
к ней  на Байковое, шёл по центральной алеее,  в конце которой, прямо за
церковью она покоится и мысленно просил у неё совета. Уходил одухотворённый и
это всегда мне помогало.

В заключении скажу своей любимой мамочке; Ты счастье вложила своею рукой в мою
колыбель при рождении. Не расплатиться мамуля  ничем мне с тобой. Я просто
встаю пред тобой на колени.
