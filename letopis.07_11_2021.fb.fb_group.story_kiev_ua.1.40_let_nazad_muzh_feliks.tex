% vim: keymap=russian-jcukenwin
%%beginhead 
 
%%file 07_11_2021.fb.fb_group.story_kiev_ua.1.40_let_nazad_muzh_feliks
%%parent 07_11_2021
 
%%url https://www.facebook.com/groups/story.kiev.ua/posts/1792324784297648/
 
%%author_id fb_group.story_kiev_ua
%%date 
 
%%tags chelovek,kiev,medicina,semja,ukraina,vrach,zhizn
%%title Эта киевская история началась 40 лет назад
 
%%endhead 
 
\subsection{Эта киевская история началась 40 лет назад}
\label{sec:07_11_2021.fb.fb_group.story_kiev_ua.1.40_let_nazad_muzh_feliks}
 
\Purl{https://www.facebook.com/groups/story.kiev.ua/posts/1792324784297648/}
\ifcmt
 author_begin
   author_id fb_group.story_kiev_ua
 author_end
\fi

Эта киевская история началась 40 лет назад. В ноябре 1981 года в трудовой
книжке моего будущего мужа Феликса появилась запись, которая стала первой и
единственной записью о поступлении на работу. После окончания школы он начал
свой путь на скорой помощи. Познакомившись с Феликсом в далеком 1983 году, уже
тогда была наслышана о его призвании и успехах на медицинском поприще. Мир
тесен. Так получилось, что моей одногруппницей в КГУ, подругой, а в будущем и
свидетельницей, была дочь врача-кардиолога, которая работала на одной
подстанции скорой помощи с Феликсом и рассказывала дочери о молодом и грамотном
новом сотруднике. 

\ifcmt
  tab_begin cols=3

     pic https://scontent-mxp1-1.xx.fbcdn.net/v/t39.30808-6/254989969_1533321337031509_4496824319146989338_n.jpg?_nc_cat=105&ccb=1-5&_nc_sid=b9115d&_nc_ohc=Kz4-bzOU2oQAX-QnGjj&_nc_ht=scontent-mxp1-1.xx&oh=c1e53a402d6f484b765daf9cd09e0545&oe=618CAA7A

     pic https://scontent-mxp1-1.xx.fbcdn.net/v/t39.30808-6/254773073_1533321547031488_2691014161682941333_n.jpg?_nc_cat=106&ccb=1-5&_nc_sid=b9115d&_nc_ohc=NA7NwyTAu4gAX92Eoq0&_nc_ht=scontent-mxp1-1.xx&oh=30c2960f75e491b1e4ca6f06d6680c61&oe=618D1F8D

		 pic https://scontent-mxp1-1.xx.fbcdn.net/v/t39.30808-6/254548769_1533321690364807_995497756781489716_n.jpg?_nc_cat=101&ccb=1-5&_nc_sid=b9115d&_nc_ohc=HST3oI0CCXcAX9Ma40_&_nc_ht=scontent-mxp1-1.xx&oh=9ea949f06c3921b9e503d6bc7f13c4d5&oe=618DC7BF

  tab_end
\fi

Пройдя срочную воинскую службу, он вернулся в 1985 году на свою подстанцию № 6.
В 1986 был переведен на только что открывшуюся новую подстанцию № 12, где
продолжает трудиться до сих пор. Я не представляю Феликса, да и он сам не
представляет себя, на другом месте. 

Человек с абсолютной скоропомощной ментальностью не только по специальности, но
и по жизни. Ему важно принятие быстрых и правильных решений, когда человек,
который вот-вот стоял на краю жизни, и смерть уже дышала в затылок, вдруг
приходит в себя!

Сколько спасенных жизней было за эти годы, сколько слов благодарности от
больных и их родственников, даже появление новой жизни происходило с его
участием и на его руках! Некоторые свои истории Феликс рассказывал читателям
нашей группы.

Человек, который постоянно совершенствует свои знания, причем не только те,
которые непосредственно связаны со скорой помощью, а и других областей
медицины. Человек, который внимательно и неравнодушно относится к больным, с
пониманием к их родственникам и близким. Человек, которого и после дежурства
беспокоит состояние больного. 

Я живу с этим человеком почти 36 лет.

И сегодня он на очередном дежурстве на страже здоровья киевлян!
