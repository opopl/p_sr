% vim: keymap=russian-jcukenwin
%%beginhead 
 
%%file slova.vy
%%parent slova
 
%%url 
 
%%author 
%%author_id 
%%author_url 
 
%%tags 
%%title 
 
%%endhead 
\chapter{Вы}

%%%cit
%%%cit_head
%%%cit_pic
%%%cit_text
\emph{Вы}, те, кто сейчас ищет спрятанных в парках, музеях и стрип-клубах покемонов,
не можете в это поверить? И \emph{вы}, сидящие с планшетами в метро над 10 серией 7
сезона \enquote{Игр престолов}, тоже? Что и \emph{вы}, вышедшие в город без комендантского
часа с пивком в рюкзаке посидеть на скамейке? Для \emph{вас} название \enquote{Горловка} - это
оккупированная территория, многострадальная земля геноцида русского народа,
серая зона конфликта на Донбассе? А это просто город, один из тысяч советских
постиндустриальных городов, где живут любящие его люди. Со своими недостатками,
трагедиями, радостями и достоинствами. Также как и \emph{вы}, видящие покемонов на
экранах своих смартфонов, горячо спорящие на лавочке о судьбе Джона Сноу и
обсуждающие западноевропейскую поэзию под пластиковый шелест пивных бокалов.
Просто два года назад здесь под минометным огнем и снарядами РЗСО \enquote{Град} в
воскресный день погибли люди и с тех пор продолжают гибнуть от осколков
снарядов. Мечтая, как и \emph{вы}, свозить детей на море, купить кроссовер и сварить к
обеду суп
%%%cit_comment
%%%cit_title
\citTitle{Gorlovka.ua: Город, рожденный жить - Блоги}, 
Егор Воронов, gorlovka.ua, 29.07.2016
%%%endcit

