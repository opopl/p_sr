% vim: keymap=russian-jcukenwin
%%beginhead 
 
%%file 05_11_2021.fb.fb_group.story_kiev_ua.1.rasskaz_borsch.cmt
%%parent 05_11_2021.fb.fb_group.story_kiev_ua.1.rasskaz_borsch
 
%%url 
 
%%author_id 
%%date 
 
%%tags 
%%title 
 
%%endhead 
\subsubsection{Коментарі}

\begin{itemize} % {
\iusr{Лилия Момотюк}
Супер! Катя то вышла замуж за гостя после борща и груши?)

\begin{itemize} % {
\iusr{Олег Коваль}
Лилия, все может быть.  @igg{fbicon.smile} 

\iusr{Наталья Лаврухина}
\textbf{Лилия Момотюк}  @igg{fbicon.laugh.rolling.floor} хороший вопрос

\iusr{Лилия Момотюк}
\textbf{Наталья Лаврухина} вот и нихххто ответа дать не может - шожыж с Катей
\end{itemize} % }

\iusr{Ирина Иванченко}

А я - то думаю, чего с утра моя душенька просит, блинов ли, расстегаев ли
;тут,,прилетает" ваше - точно! Борщ! Ох, наварю, наваристый,( прощенья просим-
с за тавтологию - оправдана - с ) со свининкой свеженькой с Лукьяновки, да с
перчиком, сметанкой, с зеленью. Фотоотчёт , Олег, ,,за вдохновение" предоставлю
позже.

Доброе утро!

\begin{itemize} % {
\iusr{Олег Коваль}
Ирина, доброе! Жду фотоотчет.  @igg{fbicon.smile} 

\iusr{Александр Асатуров}
\textbf{Ирина Иванченко} и с фасолью!!!

\begin{itemize} % {
\iusr{Ирина Иванченко}
\textbf{Александр Асатуров} , так ,Саша, я тут не совсем поняла...Вы что, ко мне на обед заскочить собираетесь? Фасоль ,,с собою", если что ...

\iusr{Александр Асатуров}
\textbf{Ирина Иванченко} эх времена.... заскакивают в макдональдс..... а на борщ со свининой с Лукьяновки званое приглашение надобно, шоб костюм из химчистки забрать, шоб в паликмахтерскую.... борщ это ритуал!

\iusr{Ирина Иванченко}
\textbf{Александр Асатуров} , ну, ладно, хорошо хоть не китайский... ритуал.

\iusr{Александр Асатуров}
\textbf{Ирина Иванченко} не.... слава богу
\end{itemize} % }

\end{itemize} % }

\iusr{Зіна Бублей}
Класс! Прочитала на одном дыхании. Дякую!

\iusr{Игорь Кокарев}

Эврика! Моя семья как раз из Анатолии с отдыха скоро возвращается, а я все
голову ломаю, что приготовить к их приезду. Буду варить киевский борщ!

Спасибо за идею )

\iusr{Раиса Карчевская}
Олег!
Спасибо за классный рассказ

\iusr{Валентина Горковенко-Спицына}
Интересно, но очень уж правильная речь у деда Коли, литературная (правда, с парой-тройкой ошибок), прямо-таки научный труд!  @igg{fbicon.grin}  @igg{fbicon.wink}  @igg{fbicon.face.kissing} 

\iusr{Виктор Задворнов}

По моим подсчетам старик принял на грудь в течение небольшого времени грамм так
250 отборного алкоголя. Отборного, в смысле, памяти. Но, судя по интересной
истории, именно памяти он и не теряет. А говорят алкоголь - это яд. Брешут,
поди)))

\begin{itemize} % {
\iusr{Ирина Иванченко}
\textbf{Виктор Задворнов} , в вас зарыт талант великого Бухгалтера, Виктор! Как чётко все вы просчитали

\begin{itemize} % {
\iusr{Виктор Задворнов}
\textbf{Ирина Иванченко} ну, бухгалтером - никогда-с! Это, видимо, от мамы. Бухгалтер по профессии, экономист - в душе. Плюс отец - инженер-нормировщик. По-хорошему, у \textbf{Олег Коваль} получилось сделать фотографию рабочего дня репортера. Главное - дело сделал, деда напоил, а сам остался ни в одном глазу. Иначе, изрядно употребив накануне, репортаж о встрече с интересным собеседником не сделаешь. Проверено!

\iusr{Ирина Иванченко}
\textbf{Виктор Задворнов} , no comments...
\end{itemize} % }

\end{itemize} % }

\iusr{Анна Сидоренко}

Я , когда читала первый раз нахохоталась по поводу дедовых 50 грамм, думала
второй раз будет не интересно,ан нет опять насмеялась, но дед тот ещё
интересный рассказчик, ещё мне понравилось про борщ, а на десерт желтые груши
на вербе, классно. Спасибо.

\begin{itemize} % {
\iusr{Tatiana Klotchko}
\textbf{Анна Сидоренко} да, и томаты из Анатолии))))
\end{itemize} % }

\iusr{Yevhen Holub}

Цікаво, але для вивчення історії непридатно. Хоча, природньо, ніхто і не очікує
від опису байок діда Миколи історизму. Ще два рази по 50 і скіфи з оселедцями
стануть засновниками Мінойського царства. Хоча за Котляревським все майже так і
сталося.

До речі, томати не могли з‘явитись раніше за картоплю. Декому ще потрібно було
відкрити Америку.

Маєте хист до письменництва, на мою думку, просто не обтяжуйте його билінністю,
наслідуючи Нестора.

Бажаю творчіх успіхів.

\begin{itemize} % {
\iusr{Олег Коваль}
\textbf{Yevhen}, дякую.

\iusr{Yevhen Holub}
\textbf{Олег Коваль} . Прошу.
\end{itemize} % }

\iusr{Ирина Иванченко}

Уж и не знаю, когда гажеты смогут передавать запах, вкус... поверьте на слово -
это обааааалденно вкуууусно !

\ifcmt
  ig https://scontent-frt3-1.xx.fbcdn.net/v/t1.6435-9/253792408_584973469476903_5548134950518187536_n.jpg?_nc_cat=102&ccb=1-5&_nc_sid=dbeb18&_nc_ohc=k8GqorKQv8EAX_GSGeX&_nc_ht=scontent-frt3-1.xx&oh=0455f7bb0332aed8b0edb4894bdec8af&oe=61A9F28C
  @width 0.3
\fi

\begin{itemize} % {
\iusr{Олег Коваль}
Ирина, чувствую... пасиб.  @igg{fbicon.smile} 

\iusr{Александр Асатуров}
\textbf{Ирина Иванченко} запахи не чувствуем.... с фасолью?

\iusr{Ирина Иванченко}
\textbf{Александр Асатуров} ,было ж сказано: ,,с собой", а нет так нет...
\end{itemize} % }

\iusr{Марина Волков-я}
Такое... Особенно про томат...

\iusr{Татьяна Сирота}
О-о-о!!! Борщ - это наше ВСЁ @igg{fbicon.face.upside.down} 
А если ещё тоненько порезанное сальцо с чесноком и... 50 грамм. М-м-м!!! Аж слюньки потекли, пока читала и комментарий писала @igg{fbicon.face.upside.down}{repeat=3} 
\end{itemize} % }
