% vim: keymap=russian-jcukenwin
%%beginhead 
 
%%file 10_09_2020.fb.himich_roman.1.rjabchuk_statja_malorossia.cmt
%%parent 10_09_2020.fb.himich_roman.1.rjabchuk_statja_malorossia
 
%%url 
 
%%author_id 
%%date 
 
%%tags 
%%title 
 
%%endhead 
\subsubsection{Коментарі}
\label{sec:10_09_2020.fb.himich_roman.1.rjabchuk_statja_malorossia.cmt}

\begin{itemize} % {
\iusr{Roman Fomin}
Отличная рецензия, хотя опус Рябчука не читал, ибо у них там все одно и то же.

\begin{itemize} % {
\iusr{Роман Химич}
Похоже, он мастер одного удара.

\iusr{Roman Fomin}
Такое впечатление, что у национал-патриотов один мозг на всех - условного Рябчука от условного Бондаря хрен отличишь.
\end{itemize} % }

\iusr{Klepko Theodore}

Как же мне нравится читать этого автора! Умно, экспертно, иронично, грамотно и
креативно. За словесные обороты - отдельное спасибо


\iusr{Alexander Rumiantsev}

Хорошо, что общество в основном живет в отрыве от своих интеллектуалов. Бо так
реально до двох Україн дограємось.

Ещё конечно удивляет сравнение українських націонал демократів и
посткоммунистической власти в Чехии/Польше. Да, у Чорновола найдётся много
общего с Гавелом или Валенсой, но закономерности из этого я бы не рискнул
выводить

Причины успеха Чехии или Польши вообще не заканчиваются на личностях из
послекоммунистических правительств, имхо

\begin{itemize} % {
\iusr{Роман Химич}
Начать с того, что Черновол был окружен, в том числе, совершенно дремучими персонажами вроде Левка Лукьяненко.

\iusr{Alexander Rumiantsev}
\textbf{Роман Химич} знаем, доводилось читать поздние сочинения уважаемого диссидента.

\iusr{Роман Химич}
Да, какая независимость такой и Батько. И наоборот

\iusr{Roman Fomin}
Черновол признавал культурные и экономические отличия внутри страны и подумывал о федерализации.
А вот его кенты Костенко и Удовенко по сути отжали партию Рух, а потом раскололи и слили её)

\iusr{Роман Химич}
\textbf{Roman Fomin} як ото діди отжимали и кололи!
\end{itemize} % }

\iusr{Александр Гросман}
Хорошая рецензия, жаль, что из-за n-word могут легко заблокировать по жалобе.

\iusr{Roman Fomin}
\textbf{Сергей Лунин}

\iusr{Roman Eridan}

Ох, ці вічні «та якби не гітлер зі сталіним, було б вам худо», а далі «росіяни
навчили вас штани застьогувати і не видувати шмарклі на паркет, а ви,
стурбовані, не хочете цінувати блеат! » (гнівно потрясає рукою)

\begin{itemize} % {
\iusr{Роман Химич}

Достатньо приїхати в славетне (колись) місто Львів та завітати на площу Ринок,
аби переконатися, що навіть матню застібати дехто до кінця не навчився.

Отриманий нашару житловий фонд перебуває у жалюгідному стані. Видно, що
тубільцям подобається колоніальна архітектура, серед них вважається статусним
жити в сецессійних будиночках, аніж в панельках, але витрачати гроші на
підтримання їх у належному стані панство не готове.

Ну а той жах, який вони ліплять, коли через родоплемінні звязки отримують
можливість побудувати щось самі, не піддається ніякому описанню. Одразу видно,
хто тут дикун


\iusr{Roman Eridan}

слід сказати ще, що, Львів, і два інших голичанських обласних центра,
знаходяться в найкращому стані, порівняно з іншими. Не тому, що в Львуві добре
керують, а тому, що в інших керують ЩЕ гірше.

\iusr{Roman Eridan}

До речі журнал Фокус, він не галичанський часом? Кому він належить? UPD тому що
там пише, ці міста кращі в країні по бізнес-клімату. Про стан культурної
спадщини важко судити, де краще/гірше зберігається.

\iusr{Роман Химич}
\textbf{Roman Eridan} нічого не знаю про Фокус.
\end{itemize} % }

\iusr{Евгений Голуб}
Яркий текст! Отличный. «Бытовая телепатия» - чудесно!

\iusr{Roman Eridan}

Отут гарно

«««« Так само, як ірландський. Тому що він показує, що є нації, які можуть
існувати, втративши мову. Все одно вони залишаються ірландськими
націоналістами. Вони вибудували потужну державу, яка має один із найвищих ВВП
на душу населення.»»»»

В мене до малоросів куча претензій, але «мова агресора111111» там буде на 18
місці. А на першому: «чому ви, падли, не «ірландці»? «Де ваш експорт одних
біотехнологічних продуктів на 4 млрд\$ як в Ірландії ?. Ваш уголь і ваше зерно,
ваш захист національного виробника вже задрав. Ви ще більші совки ніж ми !»

І всякі ідентичності пофіг мені ) Аби гроші приносив цей «континуум». І проблем
не приносив, і грошей на утримання не вимагав. І можна жити не тужити.

\begin{itemize} % {
\iusr{Роман Химич}
\textbf{Roman Eridan} так це до вас, пане, це питання в першу чергу ))
Де ВВП?!! Де літаки і всяке таке?!!! )))

\iusr{Roman Eridan}
що з нас, рагулів полонинських, взяти? Совість є, га?

\iusr{Roman Eridan}

ми сільські інтелігенти, вчителі рідної рагульської мови, на більше не
претендуєм. В ракетах не розбираємся, рівняння не зводим, знаєм кілька віршів
напамять, і на Справжню Еліту дивимся з надією. Вже 30 років


\iusr{Роман Химич}
\textbf{Roman Eridan} та якби ви мовчки дивилися, ви ж лізете керувати!

\iusr{Roman Eridan}
в ракети не лізем! правда же?

\iusr{Roman Eridan}

тільки в мовно-кулІтурні питання, і ще трошечки в таможенні, але дуже трошечки.
І це все! Погодьтесь.

\iusr{Roman Eridan}
\textbf{Roman Khimich} 

але ми відхилились від теми.

Чи можна вважати, що умовно «гіперукраїноцентрична» інтелігенція, типу Рябчука,
в формування економічної політики не втручалась?

Адже економіка, як не крути, це основне. Вона все решту визначає.

\iusr{Роман Химич}
\textbf{Roman Eridan} 

втручалась самым активным образом, как только появилась возможность.

Во-первых, продвигая представления о том, что индустриальное наследие это зло,
некруто, пуповина, связывающая с Москвой и совком и т.д. и т.п. Иными словами,
создавая медийный интеллектуальный фон для катастрофической деиндустриализации.
Впрочем, на этот поприще отличилась совецкая интеллигенация в целом.

Во-вторых, с начала 90-ых нац.-демы озаботились украинизацией делопроизводства
на пром. предприятиях. А также "коренизацией" руководящих кадров. Что не пошло,
мягко говоря, на пользу.

\iusr{Roman Eridan}

Ого!

«продвигая представления»

Ок, можна «продвигать представления», однак, на голоси-то, в Раді, і на
виділення грошей на «підтримку національного виробника» це хіба впливало?

Коли рябчуки були при владі? І коли впливали на гроші?

Епоха Кравчука = епоха печатання грошей = підтримки червоних директорів. Кучми
також?

І при чому тут советская интеллигенция, вона не мала впливу на виділення грошей
в 90х також, правда?

Чи в позитивну, чи в негативну сторону жодна з цих «інтелігенцій» - не могла
вплинути

Українізація ділопроізводства - це взагалі притягнуто якось, в надії на тупість
співрозмовника )

Хіба червоні директори і олігархи, які мало не до 2004 (?) панували
безроздільно в економіці, - слухали якихось рябчуків, чи то «радянську
інтелігенцію»?

Хто прислухався, Кучма, Горбулін, Табачник, (?),

\iusr{Roman Eridan}
\textbf{Roman Khimich} давайте якось серйозніше )
Серйозніше подивіться на це, без штампів
Без жодних симпатій/антипатій до окремих осіб, ідеологій, регіонів, промислових політик, ..., ... .
Намагаюсь розібратись в причинах, «чому все так», і «хто в поворотний момент прийняв неправильне рішення»
\end{itemize} % }

\iusr{Юрій Чорноморець}
хороша рецензія

\iusr{Ievgen Yelisieiev}

"якби в кожному місті Західної України 10-30\%-50\% населення становили поляки -
із своєю мовою, потужною культурою та впливовою історичною Батьківщиною просто
за дверима. А в містах Правобережної України аналогічну долю населення складали
б євреї. І не ті зручні, тихенькі секулярні "жидобандерівці", які навіть кіпу
не ризикують надягати, аби не зачепити вічно набряклі "національні почуття"
аборигенів, а люди, скажімо так, набагато більш яскраві та розкуті"

Якщо це супроводжується свободою для української культури, тобто ніякої
насильної русифікації в певні періоди чи розстріляного відродження - то було б
цікаво, можливо всі б взаємно збагатилися.

"Зняти огидні колонізаторські брюки та краватки, прикрасити себе елегантними
татуюваннями та намистами, заспівати та затанцювати як ото сторіччями діди
співали-танцювали, поки не прийшли білі чорти і не загнали їх на плантації."

Це якийсь товстий натяк, що без умовних "білих" умовні чорні так і лишилися б
"дикунами".

\begin{itemize} % {
\iusr{Aleksander Wójcik}
Знаючи, що текст писав Роман, в мене виникло лише одне, коротке питання: WTF?

\iusr{Ievgen Yelisieiev}
Трохи занесло, як на мене.

\iusr{Роман Химич}
\textbf{Ievgen Yelisieiev} 

"Якщо це супроводжується свободою для української культури, тобто ніякої
насильної русифікації в певні періоди чи розстріляного відродження - то було б
цікаво, можливо всі б взаємно збагатилися".

До чого русифікація, коли мова йде про поляків, ті і самі непогано справлялися.
От уявимо, що Сталін вирішив не депортувати поляків із ЗУ, а навпаки залишити.
А німцям не вдалося через якісь обставини влаштувати Голокост.

Міста ЗУ залишилися б в розпорядженні поляків та євреїв, до яких додалася б
невелика кількість переселенців з внутрішніх областей СРСР.

На кого б тоді скаржилися українці? Що клятий Сталін не захотів прибрати
поляків та євреїв? Що насправді Львів український?

Прихильники концепції російськомовного креольства самі обрали метафору,
відповідно до якої знаходяться на найнижчому щабелі. Це їх вибір, не мій. Я
можу лише глузувати з нього, чим і займаюся.

До речі, як тільки мова заходить про колоніальну політикку де-інде в світі,
прихильники цієї концепції ототожнюються саме з колонізаторами та окупантами,
посилаючись на їхню цивілізаторську місію. Тож не розумію, що саме викликає
обурення чи подив

\iusr{Ievgen Yelisieiev}

Українська чи бєларуська зараз саме у такому стані не тому, що програли у
конкурентній гонці із російською.

Тож порівнювати російську чи польску, нав'язану різними методами з імперського
центру, із тими ж самими мовами/культурами, що просуваються локально
рівноправними етноменшинами, - це очевидна як на мене помилка.

\iusr{Роман Химич}
\textbf{Ievgen Yelisieiev} 

концепція чесної конкуренції прямо передбачає наявність а) якихось правил б)
заборону на агресивні дії учасників перегонів та в) арбітра, який слідкує за
порядком. У випадку мовної політики про це годі й говорити.

Українська програла, оскільки була мовою переважно селянського етносу, який не
мав ані елітарної (шляхетської), ані державної традиції. За радянських часів її
підтримували, але не в такому обсязі, аби вона створювала конкуренцію
загальнодержавній.


\iusr{Ievgen Yelisieiev}
Якийсь дивний дискурс.
А чому вона була мовою села, як так сталось?

\iusr{Роман Химич}
\textbf{Ievgen Yelisieiev} я не готовий переповідати розлоги тексти

\iusr{Ievgen Yelisieiev}
Само якось)

\iusr{Роман Химич}
\textbf{Ievgen Yelisieiev} это ваша версия ))

\iusr{Aleksander Wójcik}
\textbf{Roman Khimich} 

твердження "Українська програла, оскільки була мовою переважно селянського
етносу, який не мав ані елітарної (шляхетської), ані державної традиції."(с) не
відповідає історичній дійсності. Простіше кажучи, це не аргумент, а лише
видавання бажаного за дійсне. Не буду посилати тебе за підтвердженнями
істориків, оскільки переконаний, що ти й сам це чудово знаєш, просто пробуєш
втримати власну логічну конструкцію, хибну в своєму засновку.


\iusr{Ievgen Yelisieiev}
\textbf{Roman Khimich} ви просто якусь дічь несете, а знаючи вас, я припускаю, що робите це свідомо й навмисно. Тобто це не місце для дискусії.
Ну таке.

\iusr{Роман Химич}
\textbf{Aleksander Wójcik} і як же виглядає історична дійсність?

\iusr{Роман Химич}
\textbf{Ievgen Yelisieiev} 

мені здається, ми розмовляємо різними мовами, точніше, вкладаємо різний зміст в
одні й ті слова. Який сенс обговорювати чи могла українська не програти
конкуренцію російській або польській, якщо вона знаходилася в настільки
нерівних умовах?

Теоретично, можливо, могла. На практиці дива не сталося


\iusr{Aleksander Wójcik}
\textbf{Roman Khimich} 

помилкою є наділення мови якоюсь суб'єктністю. Українська мова нічого нікому не
програвала - зникла держава, котра в допереяславський час формувала українську
націю. Тому тут цілком правий Рябчук, котрий розглядає це питання в контексті
процесу формування держав та, як похідне від них - націй. Втім, він тут зовсім
не оригінальний, а лише відображає позицію європейських мислителів. Якщо ці
загальновідомі факти та обставини якось зачепили твої особисті почуття та
викликали такий шторм емоцій, то що вже тут вдієш... Зачекаємо трохи))

\iusr{Aleksander Wójcik}

\href{https://www.facebook.com/natalya.starchenko.1/posts/2985835954854742}{%
ЩЕ РАЗ ПРО ЛЮБЛІНСЬКУ УНІЮ ТА РЕМЕСЛО ІСТОРИКА, Natalya Starchenko, facebook, 11.09.2020%
}

- як маленький фрагмент великого полотна.

\end{itemize} % }

\iusr{Aleksander Wójcik}

Якби не рецензія, не мав би нагоди ознайомитись зі статтею п.Рябчука. Тепер
змушений констатувати, що рецензія якось дивно викручує зміст статті, тобто я
її коректною назвати не можу: Рябчук вибудовує цілком зрозумілі та логічні
конструкції, котрі носять абсолютно нейтральний характер. На відміну, до речі,
від рецензії, яка здивувала мене не найкращим чином.

\begin{itemize} % {
\iusr{Роман Химич}

Рябчук, як він сам визнає, вирішує суто ідеологічне завдання, намагаючсь
вписати Україну в загальносвітовий контекст. Для цього він притягнув за вуха
концепцію колоніальних страждань і решту мотлохоу.

Україна не була колонією ані в РІ, ані в СРСР, натомість вона була провінцією.
А це геть інший статус та доля. Частина українського народу, дійсно, була
обєктом колоніальної експлуатації, а саме селяни. Десь всередині 70-х, коли
режим відкрив для себе вуглеводородну скарбницю, ця експлуатація змінилася
іншими крайнощами.

В будь-якому випадку міське населення знаходилося в привелійованому стані, в
тому числі як один з бенефіціарів надексплуатації селян.

Тож фактична картина набагато складніша за примітивні схеми Рябчука


\iusr{Aleksander Wójcik}

Взагалі-то сама стаття стосується питання ідентичності, причому саме як
загального поняття, а вже потім як окремого українського контексту. На мою
думку, цілком зважена та аргументована позиція автора. Тому, якщо ти вирішив
"проїхатись" не по змісту статті, а по самій позиції та постаті автора, то
вибрав не той випадок, і, абсолютно напевно, не той тон.

\iusr{Роман Химич}
\textbf{Aleksander Wójcik} 

Рябчук цю стюардесу використовує вже 20 років поспіль. Саме в українському
контексті. Нічого зваженого я в його позиції не знаходжу. Деякі причини навів.

Логічні дефекти його міркувань такі кричущі, що, за великим рахунком,
дискредитуються і самого автора, і його погляди.


\iusr{Aleksander Wójcik}
\textbf{Roman Khimich}, 

якщо з рецензії забрати припущення, оціночні судження та образливі
висловлювання на адресу навіть не автора, а цілих спільнот, то залишиться
цікаве та, як на мене, вдале порівняння примітивного націоналізму з BLM-рухом -
але, на жаль, це порівняння до тексту статті відношення не має. Тому в сухому
залишку в мене нічого, крім питання "Що це було?" не залишилось. Дякую за твої
відповіді на мій коментар, але враження, що тут бракує логіки, зате є надмір
негативних емоцій, не пропало. Ще раз: я не знайомий ні п.Рябчуком, ні з його
працями, тому просто висловив свою думку виключно в контексті статті та
рецензії.

\iusr{Роман Химич}
\textbf{Aleksander Wójcik} а які спільноти я ображаю, окрім прихильників етнічного націоналізму?

\iusr{Aleksander Wójcik}
\textbf{Roman Khimich} 

"Що б не намагалися змайструвати в царині правосуддя тубільні ентузіасти -
САП/НАБУ/НАЗК/ДБР/ВАС/АРМА..." - не знав, що в нас ці органи утворені етнічними
націоналістами)) З іншого боку, сам факт навмисної образи будь-якої української
спільноти мав би оцінюватись як явище припустиме?


\iusr{Роман Химич}
\textbf{Aleksander Wójcik} 

хто ж ці чудові люди, які шостий рік будують літачки з хмизу? Чому серед них
так багато людей, які вважають за потрібне за будь-якої нагоди присягнутися у
своїй відданості ідеалам Національного Відродження?

Ну реально, це ж без винятнку публіка націонал-стурбована.

\iusr{Aleksander Wójcik}
\textbf{Roman Khimich} , 

не вірю, що ти в це сам віриш)) Вибач, але плутати бутафорію та декорації з
реальністю - геть не твій стиль. Тому: все по-Станіславському.

\end{itemize} % }

\iusr{Irina Gladuniak}
ще б проаналізувати вплив урбанізації та смерті села на гетто - було б ще цікавіше

\iusr{Nataly Bezmen}
Отличный текст и анализ. Эталонный!

\iusr{Iurii Volochai}

Я заметил одну не большую, но важную разницу между рос и укр националистами. В
РФ хотят, что бы ВАЗ стали хорошей машиной и все ездили на них, а в Украине
националисты хотят ездить на мерседесах... Собственно это и есть вся суть
украинского национализма. Для этого не нужно заниматься 'интеллектуальным'
онанизмом


\iusr{Блюминов Алексей}

рябчук это такой унылый шарлатан от гуманитаристики известный больше свонй
русофобией и антикоммунизмом а не достижениями в какой либо из наук.

\iusr{Роман Химич}
\textbf{Блюминов Алексей} именно что шарлатан, торгующий набором фобий

\iusr{Aleksander Wójcik}
Я так розумію, це позиція самого Друзенко:

\href{https://censor.net/ru/blogs/3218760/ukranske_getto_chi_patrotichniyi_front}{%
УКРАЇНСЬКЕ ГЕТТО чи ПАТРІОТИЧНИЙ ФРОНТ?, Геннадій Друзенко, censor.net, 11.09.2020%
}

\begin{multicols}{2}

Myk Riabchuk прочитав у Харкові чудову лекцію, гловні тези якої законспектував
та оприлюднив "Збруч": \url{https://cutt.ly/ifYJg7M}. В ній він розвиває "єресь" про
мову як простий засіб комунікації, що для сили-силенної людей не має
сакрального значення, і пропонує переосмислити термінологічний апарат, яким
звикла користуватись українська гуманітарна інтеліґенція.

В основі головної тези Миколи лежить цікаве спостереження, що від 2012 року
більш як 80\% українців вважають себе патріотами України. Підтримка незалежності
України підскочила 2014-го року і стабільно сягає більше 75\% (останні 2 роки –
більше 80\%). Додам від себе, що згідно зі свіжими соціологічними замірами,
більш як 70\% вважає українську мову важливим атрибутом незалежності України.
Так само абсолютна більшість українців (80\%) згодна з тим, що всі керівники
держави та державні службовці повинні у робочий час спілкуватися державною
мовою: \url{https://cutt.ly/XfYJx44}. Статус української як єдиної державної від
2014-го стабільно підтримує 75-80\% українців.

За статус російської як другої державної, приєднання України до інтеграційних
утворень на чолі з Росією, виступають в Україні 15-25\% українців. Тобто базовий
розподіл тих, хто за незалежну Україну та кому незалежність муляє душу й очі
приблизно 80\% на 20\%. За іронією долі ці цифри точно віддзеркалюють пропорцію
американських колоністів, що повстали за свободу та незалежність далекого 1775
року, та лоялістів, які підтримували панування Британської імперії у Новому
світі. На відміну від України, в результаті війни за незалежність та перемоги у
ній повсталих колоній американські лоялісти мусили емігрувати в Канаду: п’яту
колону у новоствореному союзі ніхто терпіти не збирався.

Наведені цифри переконливо свідчать: "бінго" – Україна відбулась!!!

А далі починається те, що нас розділяє. Статус російської як регіональної –
мінус 15\% від вісімдесяти. Прихильники позаблокового статусу України — мінус
26\% (за підтримки вступу до НАТО в районі 50\%). При тому, що на референдумі про
вступ до ЄС "за" проголосували б майже 60\% громадян України, лише 27,5\% з них
вважає, що нам потрібно домагатися членства в Євросоюзі, а 41,7\% – стверджує,
що Україні слід "визначати свій власний шлях розвитку і спиратися виключно на
свої ресурси" (і це за цьогорічним опитуванням СОЦІСа, афілійованого через
Гриніва з партією "ЄС"!: \url{https://cutt.ly/sfYJnlO}).

Ми об’єктивно розмовляємо в побуті різними мовами: від 97\% українською в
Івано-Франківську та 95\% — у Львові до 97\% російською в Маріуполі і 95\% – у
Сєвєродонецьку, хоча майже всі добре володіємо обома. І наша фактична
двомовність нікуди не подінеться, бо ми розуміємо один одного без перекладача і
знаємо про це, а це означає, що наш фактичний білінгвалізм цілком задовольняє
комунікативну функцію мови. Для виконання символічної функції достатньо статусу
української як єдиної державної: шанування державного прапору не передбачає, що
він має висіти на стінці у кожній хаті, та майоріти на кожному подвір’ї та
балконі.

Нас також розділяє ставлення до радянської (та імперської) спадщини: це був
злочинний колоніальний режим чи трагічна, але наша історія, яку творили наші
батьки і в якій є як те, чим варто пишатися, так і те, від чого слід жахнутись
та засудити? Нагадаю, що згідно з опитуваннями "Демініціатив", ставлення
українців до засудження СРСР як комуністичного тоталітарного режиму, що
здійснював політику державного терору, фактично розкололо націю навпіл: 34\%
ставляться позитивного до такого засудження і 31\% – негативно. Так само 30\%
українців схвалюють перейменування населених пунктів та вулиць, названих на
честь комуністичних діячів, тоді як 44\% ставляться до цього негативно. Решті –
байдуже: \url{https://cutt.ly/0fYJQar}.

І тут починається українська версія "прокляття ідентичності", яку людині, за
слушним спостереженням Френсіса Фукуями, властиво весь час уточнювати, а отже
асоціювати себе з меншими й меншими спільнотами. Замість створювати широкий
патріотичний фронт на підвалинах цінностей, які нас єднають: незалежність та її
символічні атрибути, включно з державною мовою, ми розбігаємось по
партизанських загонах, кожен з яких прагне більше прав та привілеїв для свого
хутору коштом сусіднього села. Попри те, що широкий суспільний консенсус
полягає у статусі української як державної (що це означає чудово розтлумачив
Конституційний суд України ще далекого 1999-го: \url{https://cutt.ly/wfYJYrp}), але
збереженні вільного обігу російської та інших мов нацменшин у приватній сфері,
частина українців з енергією вартою кращого застосування нав’язує країні
наратив про російську як "мову агресора". Хоча чеченці на Донбасі розмовляли
російською набагато гірше за тих, хто по інший бік фронту боронив українську
незалежність. Примусова українізація сфери обслуговування, яка не має нічого
спільного з функціюванням української як державної, як і непродумана
декомунізація, що часто перетворюється на топографічну бандеризацію України (я
навскидь нарахував 70 вулиць, 2 проспекти та 3 провулки Степана Бандери – і цей
список далеко не повний), радикально звужує український фронт, але натомість
виступає дріжджами, на яких активно зростає рейтинг ОПЖЗ.

І так 80\% патріотів України перетворюються на 15\% "справжніх українців",
згуртованих навколо "сивочолого гетьмана". За злою іронією долі,
політтехнологія Петра Порошенка "або я або Путін", яка передбачала реінкарнацію
в Україні п’ятої колони, повернення в Україну та політичне воскресіння
демонізованого Медведчука — все це з метою продемонструвати патріотичному
електорату реальність загрози (про)російського реваншу і в такий спосіб
консолідувати навколо п’ятого президента усі проукраїнські сили, повернулась до
нього кармічним бумерангом. Тепер не так Медведчук, Бойко та компанія
мобілізують електорат Петра Олексійовича, який уперся в стелю в 15\%, як
Порошенко мобілізує та консолідує електорат ОПЖЗ, який незначно, але стабільно
перевищує кількість прихильників "ЄС".

Оскільки п’ятий президент залишається стабільним лідером народної недовіри
(\url{https://cutt.ly/wfYJD7K}), а в другому турі виборці, як відомо, голосують не
так \enquote{за}, як \enquote{проти}, Петро Олексійович не має жодних шансів повернутись на
Банкову демократичним шляхом — хіба що в результаті державного перевороту. І
тому "ЄС" фактично створила та активно працює над консолідацією та консервацією
зелотського гетто в Україні. Яке не має жодних шансів на перемогу на вільних,
відкритих та демократичних виборах, але дає його добровільним мешканцям
відчуття винятковості та вищості щодо решти громадян України — в їхній
термінології \enquote{п’ятої колони}, \enquote{хахлів} та \enquote{малоросів}.

Втім сучасні електоральні війни виграють не 300 спартанців, а широкі політичні
фронти. Замикатись в культурному гетто у власній державі — абсурд та
найкоротший шлях до поразки та втрати України! А ми досі ладні бачити ворога в
кожному, хто говорить російською мовою або вживає вираз \enquote{на Україні},
хоча саме такий вираз знаходимо у \enquote{Львівському літописі},
\enquote{Конституції Пилипа Орлика}, Тараса Шевченка, Лесі Українки, Івана
Франка, Михайла Грушевського, Симона Петлюри, Дмитра Донцова, Юрія Шевельова і
(sic!) самого Степана Бандери: \url{https://cutt.ly/1fYJB3Z} та
\url{https://cutt.ly/7fYJVqN}.

Американці, які заснували США, були значно мудрішими за нас. Вони чудово
усвідомлювали, що питання рабства здатне розколоти та знищити новоутворену
конфедерацію вчорашніх колоній. І попри те, що Декларація Незалежності (1776)
стверджує: "Ми вважаємо за самоочевидні істини, що всіх людей створено рівними;
що Творець обдарував їх певними невідчужуваними правами, до яких належать
життя, свобода і прагнення щастя", ані в Статтях Конфедерації (1777), ані в
оригінальній редакції Конституції США (1787) ми не знайдемо подібних тверджень.
Чому? Бо розв’язок питання рабства, ставлення до якого кардинально відрізняло
південні та північні штати, відклали до кращих часів, аби вберегти
новостворений союз. Ба більше, розділ 9 ст. 1 Конституції США прямо позбавляв
Конгрес права забороняти завезення до країни нових рабів впродовж 20 років
після набрання Конституцією чинності. Такий (а)моральний компроміс був ціною
існування самої американської державності.

Як відомо, відкладене питання рабства вибухнуло в США через 80 з гаком років і
було розв’язане в результаті кровопролитної громадянської війни, що коштувала
американцям більше життів, аніж будь-яка війна із зовнішнім ворогом. Але на той
час США остаточно відбулись як незалежна держава. Від часу останньої
інтервенції британців, коли був узятий Вашингтон і спалений Білий дім, минуло
півстоліття. Суверенітет Сполучених Штатів ніхто не ставив під питання. І тоді
прийшов час відкинути рабовласницьку ідентичність як неамериканську. Хоча
остаточно питання расової дискримінації не вирішено в США до цього часу,
яскравим свідченням чому є цьогорічні протести під прапорами руху Black Lives
Matter.

Нам потрібно навчитись у американців стратегічному мисленню. А воно спонукає
творити спільний патріотичний фронт, який потенційно підтримують 80\% українців
(чи бодай 60\%, тобто ті, хто вважає Росію державою-агресором), а не замикатись
в зелотському гетто, яке плекає у його добровільних мешканців почуття
винятковості та героїзму, але прирікає їх на електоральне фіаско. Навіть у
двобію з п’ятою колоною Медведчука-Бойка-Льовочкіна-Рабіновича.

Час дорослішати та обирати стратегію перемоги. Бо за свою інфантильну
ексклюзивність та "справжню українськість" ми ризикуємо поплатитись самим
існуванням справжньої України, яка набагато розмаїтіша за уявну Україну зелотів
з українського гетто.

\end{multicols}

\end{itemize} % }
