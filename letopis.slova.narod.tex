% vim: keymap=russian-jcukenwin
%%beginhead 
 
%%file slova.narod
%%parent slova
 
%%url 
 
%%author 
%%author_id 
%%author_url 
 
%%tags 
%%title 
 
%%endhead 
Что я хочу предложить и почему. Сначала почему. Я считаю, что \emph{народ}
совершенен.  Любой \emph{народ}. Его менталитет всегда и везде сформирован
внешними условиями.  История, традиции действует на людей только благодаря
пропаганде власти. Наши дети уже не знают, кто такой Ленин, не говоря о
Брежневе и Хрущеве. После 1991 уже родилось два поколения новых людей,
\textbf{Не нужно никого лишать избирательного права}, Павел Себастьянович,
strana.ua, 27.05.2021

...Одна из проблем, которую нельзя решить высокоточными ракетами, - миллиарды
недорослей, недоучек, недоразвитков. Примитивные \emph{народы} умели
воспитывать своих мальчиков и девочек. Простая культура целиком влезала в одну
голову, и в каждой голове были необходимые элементы этики и религии, а не
только техническая информация. Культура была духовным и нравственным целым.
Естественным примером этой цельности оставались отец с матерью. Сейчас они
банкроты. Тинейджер, овладевший компьютером, считает себя намного умнее деда,
пишущего авторучкой.  Мир изменился, каждые пять лет он другой, и все старое
сбрасывается с корабля современности. Растут миллиарды людей, для которых
святыни, открывшиеся малограмотным пастухам, не стоят ломаного гроша. Полчища
Смердяковых, грядущие гунны, тучей скопились над миром. И они в любой день
готовы пойти за Бен Ладеном или Баркашовым. Записку Иконникова гунны не прочли
(а если б и прочли - что им Иконников? Что им князь Мышкин?). Судьбу Другого
они на себя не возьмут… Одно из бедствий современности - глобальная пошлость,
извергаемая в эфир.  Возникает иллюзия, что глобализм и пошлость - синонимы. И
глобализм уже поэтому вызывает яростное сопротивление. Не только этническое. Не
только конфессиональное...,
Григорий Соломонович Померанц (13 марта 1918, Вильно, Литва — 16 февраля 2013, Москва, Россия)

