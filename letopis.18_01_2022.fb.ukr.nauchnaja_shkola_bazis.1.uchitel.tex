% vim: keymap=russian-jcukenwin
%%beginhead 
 
%%file 18_01_2022.fb.ukr.nauchnaja_shkola_bazis.1.uchitel
%%parent 18_01_2022
 
%%url https://www.facebook.com/sciencebasis/posts/1256154411546326
 
%%author_id ukr.nauchnaja_shkola_bazis
%%date 
 
%%tags obrazovanie,shkola,uchitel,ukraina
%%title Як це круто, коли вчитель знаходиться на одній хвилі з учнями
 
%%endhead 
 
\subsection{Як це круто, коли вчитель знаходиться на одній хвилі з учнями}
\label{sec:18_01_2022.fb.ukr.nauchnaja_shkola_bazis.1.uchitel}
 
\Purl{https://www.facebook.com/sciencebasis/posts/1256154411546326}
\ifcmt
 author_begin
   author_id ukr.nauchnaja_shkola_bazis
 author_end
\fi

Як це круто, коли вчитель знаходиться на одній хвилі з учнями. Він може і
пожартувати, і складну тему подати цікаво, і гру придумати, щоб простіше було
засвоїти матеріал...

\ii{18_01_2022.fb.ukr.nauchnaja_shkola_bazis.1.uchitel.pic.1}

З ним урок пролітає майже непомітно, а після заняття діти не зазубрюють весь
вечір пройдений матеріал, а шукають додаткову інформацію про тему, яка їх
зацікавила.

В Базисі взаємовідносини між учнями та вчителями завжди були на першому місці.
Вони побудовані на довірі та взаємоповазі. Кожен учень знає, що вчитель – це
його друг, до якого можна підійти на перерві поспілкуватися, а після уроків
порозв’язувати разом складні задачки.

\ii{18_01_2022.fb.ukr.nauchnaja_shkola_bazis.1.uchitel.pic.2}

Саме таке відношення до дітей допомагає їм розкритися і проявити свої таланти,
мотивує їх навчатися та досягати результатів. Наше завдання – показати учням,
що вони здатні на більше, дати їм якісні знання для реалізації своїх
можливостей.

\begin{verbatim}
	#Basis
\end{verbatim}
