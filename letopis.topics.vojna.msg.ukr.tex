% vim: keymap=russian-jcukenwin
%%beginhead 
 
%%file topics.vojna.msg.ukr
%%parent topics.vojna.msg
 
%%url 
 
%%author_id 
%%date 
 
%%tags 
%%title 
 
%%endhead 

https://www.facebook.com/groups/story.kiev.ua/posts/1914763742053751/

Kiev Fortress Валентина Фищук Відкрийте будь-ласка, твори Григорія Саввича
Сковороди - хоча він й є наш великий український філософ, але писав свої твори
про Бога та Людину російською та латиною. Його теж треба викорінити? І щодо
латини та мови. Специфіка будь-якої мови полягає в тому, що її неможливо просто
так "викорінити", як би те комусь не хотілось би. Будь-яка мова, і російська в
тому числі, є живим організмом, який живе за власними законами, і який
неможливо просто так загнати в якісь межі законами або ж циркулярами. Ось та же
латинська мова. Римська імперія у свій час жорстоко насаджувала свій лад у
поневолених країнах, не менш жорстокіше ніж росімперія в Україні або Польщі або
британська імперія в індії. Римська імперія давно вже зникла, а латина живе.
Хоча й мертва мова в тому, що нею ніхто не розмовляє в звичайному житті, але
вона живе в словах, живе в творах, живе як мова богослужіння. Латина - офіційна
мова Ватикану. Також, "специфіка" , календар, інформація, реформація - всі ці
слова є спадком жорстокої, аморальної римської імперії, в якій Нерон вбив свою
матір та спалив Рим, а Калігула вїжджав в сенат на коні. Тим не менш,
незважаючи на все це, латинська мова пережила 15 сторіч, і вже ніколи не
зникне, хоча й існує лише як писемна мова. Тим більше російська мова, як мова
живого спілкування, як мова, що розповсюджена далеко за межами московії,
переживе і криваву зловісну брехливу московію, переживе і минулу свою імперську
провину перед Україною, і буде жити далі, і в Україні зокрема. А якщо вам не
подобається росмова під час війни - адресуйте, будь-ласка, зокрема свої
занепокоєння до конкретних українців, що розмовляють росмовою і широко відомі в

Відкрийте будь-ласка повість Захар Беркут Івана Франка... і перші рядки це
будут рядки з Пушкіна (цитата з поеми Руслан і Людмила). як це не дивно. Пушкін
не обмежується лише Полтавою або ж Клеветникам России, і зокрема наш Каменяр,
Іван Франко досить високо цінував творчість Пушкіна. Але звісно, гнів та
ненависть з боку українців до сучасної московії є абсолютно законна річ,
злочини сучасної московії просто перетнули всі можливі людські межі, але все ж
таки тут важливо підходити до таких речей, як знесення памятників та
перейменування вулиць, з більшим розумом, ніж це зараз відбувається. На наш
погляд, це є явне розпорошення розумових та емоційних сил, які можна було би
більш продуктивно витратити на допомогу нашим хлопцям, зокрема Маріуполю.

https://www.facebook.com/kyjeslav.borshchahivets/posts/5195164247219634

У вас тута трохи якась каша. То парк чудовий, то він слугував русифікації. Ну -
слугував, ну і шо. росімперія давно вже вмерла - а чудовий парк залишився. От
ви коли ідете в Київскому метро, яке відкрили в 1960 році, ви ж не
задумувались, що його будувала компартія, яка винна в багатьох злочинах. срср
вмер - а чудове гарне метро залишилось, хоч би якою тоталітарною державою совок
не був. Або ж гімн Києва, Як тебе не любити, Києве Мій. Теж був створений в
совку в 1962, і взагалі його першим виконавцем був росіянин, Юрій Гуляєв. І
тому подібне. А щодо Пушкіна, відкрийте Захар Беркут Івана Франка, і там ви
побачите цитату Пушкіна. Іван Франко взагалі досить високо оцінював творчість
Пушкіна. Так само в ширшому значенні і творчість поетів та письменників. Можете
знести хоч всі памятники, всі до одного, але творчість поетів та письменників,
зокрема Пушкіна, буде продовжувати жити в серцях багатьох людей, буде
продовжувати жити також на сторінках книг (через цитати, згадки і тому
подібне), і зокрема також творів українських письменників, таких як Іван
Франко, бо ж Пушкін не обмежується лише Полтавою або ж Клеветникам России, і
звісно справжня цінність Пушкіна не вимірюється кількістю памятників. Щодо
порівняння московії і російської імперії, знаєте, сучасна московія, тобто
путінська московія, навіть гірша і за росімперію і за срср, бо московія, бридке
хиже утворення на тілі Землі, зараз нищить те, що будували таким великим коштом
і такими великими зусиллями і росімперія, і срср, як-от найбільший у світі
літак Мрія АН-225. московія не має ніякого відчуття цінності і власне Пушкіна,
бо вона бомбить Місто, Вічне Місто Київ, де той самий памятник і знаходиться
(або ж Одесу, де навіть є музей Пушкіна).

розумієте, тут ви теж впадаєто в крайнощі, тому що не ви єдиний, хто любить
Україну. І кожен любить Україну трохи по-своєму. Жодного відношення не мають,
чи мають... Ви що у нас геній на кшталт Франка, що є такі самовпевнені і у
всьому на світі розбираєтесь? Все у світі так чи інакше має відношення до
України, тому что Україна є частиною цього великого світу. Крім того, у вас
немає ніякої особистої монополії вирішувати, що має відношення до України, а що
ні. Це справи громади або віча, де збираються люди, що живуть безпосередньо у
Києві, наприклад. Ось громада збирається і вирішує, бути тим чи іншим
памятниками чи ні в цьому районі чи ні. І судячи з досить жвавої дискусії,
якраз досить велика частина української громади вважає, що або це питання не на
часі, або ж памятник треба залишити, так, як він і був. Суть України полягає ж
не тільки у Пушкіні, чи Шевченкові, а у тому, чи здатні ми сукупно та мирно
вирішувати ти чі інші спірні питання. В цьому і є суть демократії. Ми самі
вважаємо, що це питання абсолютно не на часі, бо це тільки розпорошує наші
сили, зараз війна зі страшним аморальним ворогом, ви хіба забули. А після війни
- проголосуємо, обговоримо - якщо що, знесемо, нема питань.

визволяти. справа не зовсім у цьому. це лише один із словесних приводів до
вторгнення, і не самий найголовніший. Основною тезою роспропаганди, на нашу
думку, є міф про те, що в Україні "неонацисти", які служать Заходу, який хоче
знищити расєю-матушку. І вбивця - якщо він хоче вбити - він привід знайде.
Стоїть Пушкін, є ті, хто розмовляє російською - йдемо визволяти молодших
братів. Пушкіна звідусіль знесли, всі перейшли на українську - йдемо мститися
за Пушкіна, Україна вже повністю бандерівська стала, так чому б не знищити її
взагалі одним ядерним ударом. Розумієте. Вони там просто всі як скажені звірі
стали. От ви самі можете зайти на однокласники, і перевірити. Як барани без
мізків, що їм вкладуть, у те вони й вірять. І привід до насильства завжди
можна знайти. І тут справа не в тому, що ми персонально Пушкіна любимо чи ні.
Справа в принципі демократії та громадянського суспільства. Україна зможе
протистояти московії тоді, коли вона стане по справжньому згуртованою, а одним
із чинників цієї згуртованості є в тому, що треба навчитися спірні питання
вирішувати по-дружньому, злагоджено. Іще головне питання щодо московії є в
тому, як зламати, знищити саму ідеологію московії. Але це складне питання, яке
потребує напружених інтелектуальних зусиль.

ви не туди дивитесь. Розумієте, можна спалити книжки (що є поганий прецендент
сам по собі, і тут справа не в змісті цих книжок, а в самому сутті цього явища
спалювання книжок,  що ставить інформаційно для світу нас як варварів ),
викинути памятники пушкіну, але саму мову - в сенсі інструменту людського
спілкування  по собі неможливо просто так знищити. Сама історія заборон
української мови тут є найкращим свідченням. Її забороняли, принижували - а
українська мова все це пережила, і нарешті розквітає по справжньому в
незалежній Україні. І тут справа не в російській мові як такій, а в самому
понятті мови як перш за все інструменту спілкування. Так, можна казати, що
росмова це мова путіна, мова оккупанта, давайте все заборонимо, все спалимо, і
тому подібне, але від того нікому краще не буде. Такі штампи є просто
неправильним осмисленням самого факту росмови в Україні, і що з нею далі
робити. І тут також проблема в тому, що люди не усвідомлюють суть самого явища
мови. Будь-яка мова, українська, російська, англійська - це живий організм сам
по собі, який живе за своїми законами. Так що ви можете бути персонально проти
росмови в Україні, але факт є факт. росмова в Україні є і буде на час нашого
життя, і напевне буде ще надовго, а може і назавжди. Вам напевне це не
подобається, і ви цього не хочете, але це факт. Ось наприклад, ви ж знаєте, що
існує латина - мова Римської імперії. Хіба Рим був краще за московію. Може й
гірше у свої часи, аморальна, жорстока імперія... Але латина надовго пережила
Рим, і продовжує жити в медицині та запозиченнях, і навіть є офіційною мовою
Ватикану. Сковорода наприклад писав латиною (крім росмови), як і багато інших
філософів і вчених. А росмова - це жива мова, і нею говорять не тільки хуйло з
скабеєвою, а наприклад скажімо також той герой родом з Харкова або Одеси, що
вчора підбив черговий ворожий танк десь на Донбасі. Так що питання не в
існуванні росмови як такої, а питання в тому, які меседжі, які смисли генерують
ті люди в Україні, що розмовляють росмовою. Якщо досить велика частина
російськомовних людей в Україні є патріотами України, а таких дуже-дуже багато,
дуже гарних, розумних, хоробрих людей, щирих серцем і душею, якщо вони
підтримують та захищають Україну, так в чому питання? І більш того. Якщо ми
хочемо знищити московію, ми в тому числі повинні знищити її ідеологічно, тобто
перепрошити мізки зомбованим баранам запоребріком, щоби вони більш не
загрожували Україні. А цю контр-пропаганду неможливо згенерувати інакше як
росмовою, бо москалі в більшості своїй просто ніяких інших мов просто не
знають.

% Охтирка оперативна
https://www.facebook.com/groups/498652738278706/?multi_permalinks=552444902899489&comment_id=553493676127945&notif_id=1652342891110460&notif_t=feedback_reaction_generic&ref=notif

Kiev Fortress Julie Tolmacheva так, звичайно. Але ви неуважно читали, що ми
хотіли донести. Так, це мова ворога, звісно. Але це також і мова друзів. До
біса українців, справжніх патріотів України, і не тільки українців, а й друзів
України по всьому світу, розмовляють росмовою, збирають гроші, виходять на
демонстрації на підтримку України, знімають кліпи, пишуть в телеграмах,
фейсбуках, при цьому абсолютно будучи патріотами України. Наприклад, ви
телеграм канал Казанського не читаєте, напевно. Або хроніки Харкова Соловйова.
А чому. Нащо оце зациклення на московії. Що, тільки московія розмовляє
росмовою, чи що. Що, хуйло персонально винайшов росмову, чи написав хоч один
словник росмови, що кажуть, що це мова путіна. Он наш Григорій Саввич Сковорода
набагато гарніше писав на росмові, ніж уся потворна московія разом. Або ми їх,
або вони нас. Абсолютно вірно. Ми маємо справу з жахливим, брехливим,
аморальним ворогом. Але перемогу України над московією, якщо брати ідеологічний
вимір, неможливо досягти простим тотальним відторгненням усього російського.
По-перше, це неможливо в принципі, як би вам не хотілося цього досягти. А
по-друге. А що ж потрібно, власне кажучи. Тут потрібен розум. Крім емоцій і
законної люті, потрібно дуже багато розуму. Дуже-дуже багато розуму, науки.
Вивчення філософії, психології, математики, фізики, суспільних наук. Ми повинні
стати настільки розумними, наскільки можемо. Бо якщо ми не станемо розумними,
ми не зробимо нову якісну зброю, яка буде захищати нас від ворога. Якщо ми не
станемо розумними, ми не побудуємо кращий суспільний лад в Україні. Якщо ми не
станемо розумними, ми не створимо нові технології, та нові робочі місця в
Україні. Якщо ми не станемо по справжньому розумними як суспільство в цілому,
ми так і залишимося найбіднішою наскрізь корумпованою країною Європи, поруч із
злим жахливим сусідом, із постійною небезпекою нових нападів та нових ракетних
ударів. Так що тут напевне варто нагадати. Учітеся, читайте, і чужому
навчайтесь, і свого не цурайтесь.

Kiev Fortress Катя Сосна як ми вибрали не Кравчука, а Чорновола. Якби то, якби
те. Якби ми б всі розмовляли б українською, то путін би не прийшов визволяти.
Якби влада була б українською, то все було б добре. Знаєте, питання не у тому,
а у чому. Приказка є така. Якби бабуся була б дідусем, то у неї була б борода.
Розумієте. Оці осі нескінченні якби б то, якби б це - це ментальна пастка.
Нескінченна демагогія, яка ні до чого не веде. Кравчук вже помер, залиште його
у спокої. І крім того, у свій час він власне і став одним із засновників
незалежної України, підписавши угоди в Біловежській пущі. За це йому честь і
хвала. Також за те, що отримання незалежності пройшло безкровно. У свій час він
був на своєму місці, так, комуняка звичайно, але також і хитрий лис. В Україні,
якщо вам нагадати, не було кровопролиття, незалежність була отримана фактично
безкровно.

щодо спалювання або заборони російських книжок. Тут ми в цьому коментарі не
агітуємо особливо за чи проти самої роскультури або мови в Україні, але
просто... смішно читати про це у цифрову епоху, коли зміст всіх цих книжок
легко доступний в інтернеті. Кому треба - зайде в інет і прочитає. І чим більше
буде таких заборон, тим більше буде охочих скоштувати забороненого. Заборонене
завжди має присмак... солодкого, чи як сказати. Ну, забороните, ну, спалите, ну
і що. Крім того, заборони, викидання на макулатуру чи просте спалювання не
можуть знищити ні росмову, ні рослітературу, ні зокрема зміст цих книжок, не
можуть дати нам перемоги над московією, і не повернуть нам життя наших воїнів
та невинно загиблих. Але сам факт такого створює дуже негарний прецендент,
інформаційний прецендент, оскільки позицінує українців для навколишнього світу
як націю, що не поважає книгу як таку.  Так, звісно, розпіареність роскультури
не відповідає її контексту в Україні, особливо з урахуванням жахливих злочинів,
скоєних російською армією, і ненависть та лють до загарбників і окупантів є
абсолютна законна і зрозуміла річ, але нищити книжки по бібліотеках, особливо
під час війни, коли десь там гинуть наші воїни, і їм потрібна вся наша
допомога, і всі наші зусилля повинні бути спрямованими на перемогу, ну... це
якось  надто тупо. Просто тупо. А українці повинні бути, точніше повинні стати
розумною нацією, найрозумнішою нацією на світі, нацією не тільки хоробрих
воїнів або волонтерів, а по-справжньому нацією вчених, інженерів, філософів,
науковців та письменників. Інакше Україна просто не витримає конкуренції в
навколишньому світі поміж інших держав.  

Це звичайно дуже круто, що ви таки знайшли україномовного тренера для своєї
дитини. Це офігенно як круто. І обовязково треба було пнути словесно
російськомовних, це теж круто. Дитинка ваша буде щаслива і вдоволена, що вона
не буде чути жахливої страшної росмови, а як же інакше. Але у нас зараз війна,
якщо ви забули. Війна зі страшним, жахливим, аморальним ворогом. Дякуйте Богові
що взагалі живі, і що вашу дитинку десь не вбили або не покалічили. Дякуйте
Богові, що Київ взагалі стоїть, що місто відносно ціле, і що орки відійшли.
Видно, що Київ ожив, і деякі особливо розумні вирішили, що можна зайнятись
словесною маячнею і пустослов'ям. Краще б написали що коїться у Херсоні, або ж
у Маріуполі, або ж як живе Харків, і чим можна допомогти тим або іншим. А оця
маячня що ви написали - абсолютно не на часі, це є просто словесний бур'ян у
голові. Так викидуйте якнайшвидше його з голови, і почніть спрямовувати свої
інтеллектуальні та душевні сили на те, що країні по справжньому потрібно для
перемоги над московією. А те, що ви знайшли україномовного тренера для своєї
дитинки - вибачте, але це перемогу над московією аж ніяк не пришвидшує. Якась
херня просто. Ну, знайшла, ну і шо. Купіть собі конфєтку, ви ж такий молодець.

ниче. смеетесь россиянцы. рано смеяться начали скоро перестанете смеяться.
плакать будете. горькими слезами плакать будете. не далек тот час когда рассея
развалится ко всем чертям. вот будет потеха-то.

вы наверное плохо изучали историю. иначе бы знали, что гетьман Сагайдачный в
свое время дошел до Москвы. И украинцы всю свою историю тоже воевали, так что
Украина тоже не в первой воюет, и к тому же на своей родной земле против
наглых, лживых, омерзительных оккупантов. Крейсер Москва хорошо горит, да,
россиянцы. Люлей еще мало получили, видно не нахватались пока еще смертей. Пока
еще в патриотическом угаре поколение Зет. Плавает в говне, а думает, что то
божья благодать. Ниче, время поплакать, постоять на коленях и посыпать голову
пеплом после поражения рассеи у вас еще найдется. Прошлый неуемный рассейский
патриотический угар в 1914 году кончился распадом империи, и в этот раз будет
то же самое, распад рассеи. только второго ссср уже не будет, будет просто
полная жопа. Вот и все.

икак никак. да все так, таки будет. а почему - долго объяснять. сами увидите
скоро. вкратце, нападение на Украину - это просто мега-глупость тысячелетия,
это выстрел себе в голову, это рассея просто убила себя саму идеологически,
ментально, культурно. и наверное, вы при падении рассеии просто умрете, ну то
такое. А то что вы этого не видите, и начнете выть, когда будет уже поздно. Ну
так извините, ваши проблемы. Если вы свою голову употребляете только на то,
чтобы потреблять соловьева и пускать слюнки на путена и великаю рассею и жрать
котлеты за обедом, ну так кто ж вам доктор. Видно, смертей еще не нахватались.
Еще в угаре находитесь. Ну что ж, угар пройдет. А что придет взамен на рассею.
А просто жопа. Полная и всеобъемлющая жопа. Возможно, через пару лет, или
позже, но это случится обязательно. Потому что иначе никак. Потому что таковы
неумолимые законы истории, по которым был Римская империи - а потом империи не
стало. есть москва - а потом и москвы не станет. Все ж так просто. И потому что
за преступлением обязательно будет наказание. Про это ж еще федормихалыч (ТМ)
писал.

да месідж взагалі не про це. ну, пішла - відвела до тренера, ну і добре. Але
нащо з цього розводити цілу демагогію, оце незрозуміло. В контексті війни нас
от більш хвилює питання наприклад, коли звільнять Херсон або чи можливо
врятувати захисників Маріуполя, а не те, що хтось там бідкається, що не може
знайти тренера, або щоби дитинку відгородити від росмови. На жаль, проблема є в
тому, що у багатьох українців відсутня якість виставляти правильні пріорітети,
оце є біда. А в результаті маємо, те, що маємо.

зїздіть тоді краще в Маріуполь або в Херсон, якщо нема чим зайнятись, окрім
відповідати в такому дусі. От тоді і побачите, де у кого які проблеми, якщо
взагалі залишитесь живі.

дякую, що намагаєтесь розібратись в темі. Буду розказувати на прикладі міст, в
яких жила. Донецька обл. історично Україна і історично україномовна, як і решта
України. Конкретно про Донецьку область мені не потрібні супер історичні
розвідки, щоб це знати. Просто приїжджаю в Лиман або Слов‘янськ, або Курахове і
слухаю, як говорять люди. В самому ж Донецьку українська мова була викорінена і
переселенням туди росіян, наративами пропоганди в стилі «українська не
важлива», і «так поймуть», «дєдьі воєвалі», «бандеровци шота там, не ясно шо і
де, але погані». Ці наративи створювали ситуацію, коли за українську на Донбасі
мало хто боровся, бо складена суспільна думка була такою, що тільки скажеш
щось, то викличеш бурю емоцій у людей, що мислять як ви зараз. І деякі з них
матимуть навіть гарні наміри, а проте будуть обслуговувати чужі цілі. І от ми
зараз з вами переписуємось під час війни, яка почалась тому, що росіяни думали,
що можуть швидко перемогти нас. Ми переписуємось про те, чи мають право діти в
Києві ходити на україномовні гуртки. І вас не цікавить, що цих гуртків нема, і
права українських дітей, травмованих війною дітей, порушуються, бо нема цих
гуртків. Ніяких. Ви пишете щось в дусі рос пропаганди про не на часі і решта
все. Чому? Чому вам не здається, що українська дитина може ходити на гурток
українською в Києві. Не колись потім, а зараз? І як ці гуртки повинні
з‘явитись, якщо таким як ви завжди вони не доречні і не на часі?

Olga Maksymchuk Kiev Fortress тільки не розумію чому ви згадали Херсон і
Маріупіль. Це прекрасні міста, які я сподіваюсь, скоро будуть звільнені. Але до
чого вони в розмові про те, що в Києві неможливо знайти український гурток? Це
називається маніпуляція. Вона теж з шаблонів рос методичок. І ви можете цього
не усвідомлювати, проте наслідуєте патерн поведінки.

чудик. здесь никто не истерит. мы просто говорим как оно есть. В отличие от
многих других, мы читаем книги и умеем анализировать что было, что есть, и что
будет потом. Так вот. Жалко вас в чем-то, олухов рассейских. Вроде люди - а по
факту - выродились в диких злобных безмозглых животных. Скоро у вас будет
полная жопа. Знаете такое слово - жопа... замечательное кстати слово - жопа.
Емкое и краткое. Так вот, вас ожидает жопа, настоящая и всеобъемлющая. А вы все
раздуваете лоснящиеся и заплывшие от безумной гордыни щечки. То возьмем! Это
возьмем! Бойтесь нас, мы сильные и страшные! Херсон взяли, Мариуполь взяли, аа!
Буквой Зю все обрисовали, аа! Посмотрите же на нас, какие мы сильные и злобные
обезъяны! Да никакие не сильные и страшные, а просто напросто омерзительные и
глупые до безобразия... Многие уже гниют в украинской земле... а многих еще
ожидает точно такая же участь. Да. Как это ни удивительно, а гордая
тысячелетняя рассея скоро рассохнется и рухнет с громким писком. Вот и все.
Кстати. А че это вы тута во вражеской сети вообще забыли-то. Вам же вроде
роскомнадзор приказал не лазить по фейсбукам, запретили же вам слушать
вражеские голоса, так чо лазите-то, россиянцы-московитики.

ларисочка, нехрен шастать по фейсбукам. вам же вроде роскомнадзор запретил,
да. У вас же там... ээ... росграм... налей сто грам есть, правда. Ну так и
чеши туда, в свой родной росграм-гиппопотам, нефиг засорять фейсбук рассейским
пометом. Фейсбук - это сеть для приличных людей. Московитам сюда вход
воспрещен... а вы все лезете и лезете, никак не уйметесь. Ну а когда интернет
у вас упадет, вообще полная потеха начнется! То, что рутюб упал на 9 мая, или
около того, это такое. Только начало. А цветочки еще впереди. Еще очеень много
всего интересного будет.

потому как если первые лица московии безбожно врали и продолжают врать, то чего
уж ожидать от простых рядовых московитов, истово облизывающих буквочки Зю на
парадах. Ничего кроме лжи, в общем-то. Уж извините, если огорчили.

git restore --source=HEAD :/

Ганна Терпиловська

Kiev Fortress Ні, не так. Ніяк думки про Маріуполь чи Херсон не мають заважати продовжувати виховувати своїх дітей. Особливо людям, які знаходяться в місцях, де є для того умови. В контексті війни, як Ви пишете, нас хвилюють всі питання. Нікому думки про Маріуполь не заважає в Києві снідати і вечеряти, читати новини і жити, як це вдається. Незрозуміло, чому Ви вважаєте, що треба плюнути на виховання дитини, тому що от зараз війна.

А те, що людина поділилася радістю від того, що знайшла україномовного тренера,
розуміють всі, хто таких викладачів шукав. Я живу в Києві, і десь приблизно з
15-16 року постійно була в пошуці гуртків і секцій, де навчання проводиться
українською, і це було дуже важко. Не дарма авторка написала, що це не бойовий
гопак. Тому що українською можна було знайти малювання, фольклорний спів,
фольклорний театр і бойовий гопак. Я шукала звичайний театр дуже довго, і мені
всюди відповідали, що Вишотакехочете, ні, ні, і не буде українською, хочете
українською, йдіть вчіть колядки в етнотеатр, а класика в нас буде класична
російська. Знайшла там, де пообіцяли 50 на 50, насправді було 25 на 75, але, як
виявилося, багато людей теж захотіли українською і поступово стало таки 50 на
50, а потім за пару років почали відкривати групи українською. В Києві була
група пошуку гуртків українською мовою, і, повірте, це була велика проблема,
навіть в гуманітарній сфері, а в спорті ніхто навіть не розглядав можливість
займатися з дітьми українською. Можливо, тим, в кого діти розмовляють
здебільшого російською - все одно. А тим, в кого діти виховані вдома
українською, навіть якщо відкинути ідейну складову, важко займатися російською.
І неможливість реалізувати бажання дитини навчатися рідною мовою в своїй країні
- це дико. Тому радість автора я дуже добре розумію, як мама, яка вишукувала
дев'ять років для своєї дитини україномовну середу існування.

%https://www.facebook.com/groups/story.kiev.ua/posts/1923987197798072

так, звісно ідеологічний вимір важливий, і кількість назв, повязаних с
роскультурою напевне є надмірною. Але ви помиляєтесь у тому, що ракети падають
тому, що 30 років було не на часі.  Ракети падають тому, що в кремлі сидить
божевільний. Ракети падають тому що народ російський перетворився на стадо
диких зомбованих баранів. Ракети падають також тому, що Україна не вкладалась
достатньо в розробку нових озброєнь. Ракети падають, тому що... Взагалі, оце на
часі, не на часі. Не на часі, тому що одразу розгорається лютий срач, одразу
починаються суперечки. Піднімається потужна ментальна хвиля. Люди засмічують
свої мізки цією балаканиною, і просто тупо витрачають свої ментальні сили в
нікуди. А в цей час хтось сидить в окопі, поки ми тута розводимо балаканину і
йому абсолютно по барабану, чи є там памятник чи нема. Знаєте, це як в
анекдоті. Це до того, що є багато речей в Україні, які самі по собі на хід
війни не впливають, але якщо їх підняти зараз, то йде потужне відволікання
уваги від по справжньому важливих речей - як от. Як звільнити Херсон. Як
звільнити Маріуполь. Як визволити бійців Азову. Як добути більше зброї,
броників. Так от, чи є памятник пушкіну, чи його немає прямо зараз - для ходу
війни абсолютно по барабану. Хід війни не визначається тим, чи дивиться пушкін
на корпуса політеху чи вже ні. Поки це питання не підняли, ніхто не пушкіна
особливо не думав. Всі ці суперечки аж ніяк не можуть пришвидшити перемогу
України, але навпаки, можуть сприяти в кінці болісній поразці, бо мізки були
направлені не та те, що потрібно. Так от, анекдот. Звісно, аналогія дуже-дуже
крива, але все ж таки. Йде професор з величезною бородою, назустріч йому
студент. Студент питає - професоре. А коли ви спите, ви кладете бороду на
ковдру, чи під ковдрою. Професор - я ніколи не задумувався. Пройшов день -
студент йде і професор такий злий-злий - каже йому - ах ти ж паскуда! всю ніч
не спав!! то на ковдру кладу, то під ковдрою, все свербить, ніяк не міг
заснути! От. є також інша приказка. Не діліть шкуру невбитого медвідя. От
переможемо - перейменуємо все що завгодно, знесемо одні памятники, поставимо
інші. Народ збереться, обговорить і винесе якесь прийнятне рішення. А зараз це
все є якийсь просто лютий треш, просто голова болить вже від цього. Кажете -
привід для нової війни, якщо не знесемо. Це абсурд. Привід для вбивці завжди
найдеться, було б бажання напасти. Тим більше що привід можна буде ще легше
знайти. А як запобігти новій війні. Бути згуртованими як один, сильними,
розумними. Бути найрозумнішою, найзгуртованішою нацією в світі.

шановний чудік під імям Сашко Папуга. Цей акаунт вже пару місяців як ми
створили, і ми є абсолютно живі люди. Хтось веде акаунт роками і має тисячі
підписників, хтось створює його вчора і починає писати, маючи нуль друзів
спочатку. Треба ж з чогось починати. І в чому взагалі питання, не зрозуміло.
Адекватна людина, може і так, ну і що. А що, адекватні люди у нас вже прям
стали святі, що їх не можна критикувати. Окей. Далі. Ну настрочив ти гнівний
коментар ну і що. По суті того що ми висловили нічого не сказав, по змісту
нічого розумного не сказав (як наприклад шановна пані в попередньому коментарі
вище), просто інформаційного шуму додав. Так що піди купи собі конфєтку теж,
маладєц. Щодо буряну в своїй свідомості, бурян вже давно прополений. Так
барвисті квіточки ростуть і гарні розлогі дерева. Там дзюрчить холодна водичка,
і гріє тепле сонечко, у нашій свідомості. Співають пташки соловїним співом. І
там все у нас зі свідомістю добре, не хвилюйся.

але якщо вам цікаво, чим ми займаємось, окрім цих коментарів. Ми ведемо
літописи війни по різним тематикам, і насправді це досить напружена робота
кожного дня. Якщо вам цікаво, ми можемо кинути вам посилання в приваті. У нас є
збірка і про Харків, а ви як видно з Харкова. У вас до речі видно по
прискіпливості аналітичний склад розуму, не дивно як для людини з науковою
освітою.

Щодо пари місяців спілкуйтеся із самим Фейсбуком, адже це саме він характеризує
так, як показано на картинці.  "Адекватна людина, може і так, ну і що". -
Заздріть мовчки!  "А що, адекватні люди у нас вже прям стали святі, що їх не
можна критикувати". - Критикувати можна. Писати нісенітниці, називаючи це
критикою, теж можна - і виглядати при цьому дурнем ТЕЖ можна.  "Ну настрочив ти
гнівний коментар ну і що". - Гнівний? А, може, більш глузливий, ніж гнівний?
"По суті того що ми висловили нічого не сказав, по змісту нічого розумного не
сказав". - Ви були значно неуважні. 😄 А якщо читати уважно, то можна побачити
те розумне, що було сказано досить чітко: хтось вивалив свою недолугу маячню у
коментарі і зовсім не потурбувався про те, чи буде воно стосуватися теми
розмови. Те інше, що можна було ще сказати по суті (невеличкій суті, але вже
якій є), вже сказано іншими. Я хіба що міг перефразувати.  "Так що піди купи
собі конфєтку теж, маладєц". - Мрійте і далі собі про конфєтки. Якщо згодом
почнете мріяти про цукерки, то це буде суттєвим кроком уперед.  "Щодо буряну в
своїй свідомості, бурян вже давно прополений". - А сувора практика, тобто
наявні коментарі показують зовсім інше.

здравствуйте. Привет из Киева. Если бы как бы, то во рту росли б грибы.
Конечно, воюет именно с народом Украины. И народ Украины дает рассее люлей. А
рассея скоро просто развалится от вони и гнилья, старая мерзкая старушонка
рассея... 

аллочка. все у нас есть. и все у нас будет. и уголь будет. и газ. и зерно -
сколько бы вы не наворовали ли бы во временно оккупированной Херсонской
области. Кстати, буряки лучше еще будут расти - хорошо парнями из Иркутска и
Самары удобрили в этом году! 27 000 200-тых это уже не шуточки! Да, и Бензин
тоже есть, и будет в каком угодно количестве. А вообще дело не в бензине и не в
газе. А просто в том, что у нас - самая лучшая в мире земля, самая лучшая в
мире страна - Украина. Прекрасная, удивительная страна. С удивительными
прекрасными людьми, способными творить чудеса. С людьми, которых ничто не может
сломить. Вы, - злобные карлики из преисподней,  пришли на нашу землю и
оккупировали Херсон и Запорожскую область - ну что ж, вас оттуда выгонят либо
просто перережут. Ночью. Умело и спокойно, без пощады. Как гайдамаки в свое
время или запорожские козаки. Вы взорвали прекраснейший в мире самолет Мрия
АН-225 - дикие злобные обезъяны, рвущиеся к Киеву - но получившие по зубам, и
убравшиеся восвояси от Стольного Града Киева - Вечного Города Киева - мы
отстроим и Мрию, и даже не одну. Вы изуродовали Мариуполь - и обрекли на слезы
и смерть десятки тысяч людей - ну что ж. Мариуполь будет возвращен в Украину,
что бы там лавров не лепетал, а вам будет наказание. Жестокое наказание. Жопа
будет, короче. Все мы отстроим и будет все в тыщу раз круче, чем было до того.
А вы несчастные злобные московиты, не думающие ни о чем другом, кроме как о
газе, бензине, бабках и территориях, будете получать люлей везде. Люлей и в
военном плане, и в обычном житейском. Вам будут закрывать двери, отказывать в
ресторанах, в кафе, в магазинах везде - и в Египте, и в Турции. Мы уж - Украина
- постараемся - а нас миллионы таких - чтобы вы стали нерукопожатыми везде - от
Африки до Америки, от Европы до Азии. Вас сейчас ненавидят, злобных диких
обезьян - но знаете, пройдет время..., ненависть уляжется, и вы будете вызывать
лишь смех и презрение, как дикие животные, которых нужно держать в зоопарке. Ну
как скажем гиена. Злобное дикое животное, питающиеся падалью. Все еще будет,
болезненькие московитики. Да... а бензин. А бензин у вас скоро кончится, не
волнуйтесь. И газ скоро кончится. Все кончится. путен кончится, лавров
кончится, все ваши вожаки кончатся. Вы то сейчас знаете как насильник
действуете и думаете. Насилует, насилует, получает кайф, и думает, что ему за
это ничего не будет, потому что он такой большой, и никто ему на районе по
морде не даст, потому что он такой накачанный. Так вот... Район весь соберется
и будет бить вас сообща. Все будут бить вас, и Украина, и Турция, и Китай, и
США, и Европа. И наступит жопа у вас. Полная, всеобъемлющая жопа. Вот тогда-то
завоете по настоящему, запляшете. Завоете от безнадеги, от страха, от ужаса.
Будете с факелами бегать по тайге и петь Катюшу в трусах. Все будет. Мы уж вам
устроим, бесам из преисподней, веселую жизнь.

чудик. приветики. ваших уже 27 000 покрошили. протухшая рассея, бессильно
надувающая щечки от собственного величия и гордыни... продолжай так же ленечка.
И вообще, приезжай лучче к нам, постреляешь, а потом - упокоишься. Нам
удобрения нужны. буряки лучче расти будут.

ромочка. ты наверное коньяка перепил, успокойся. Да че уж стесняться, давай
мильйон нарисуй сразу. Так и скажи - укров уже мильйон покрошили. Че
стесняться. Ну а... русккие по доброй воле отошли от Харькова, как раньше от
Киева... Ну а потом вы будете рапортовать о том, что уничтожили уже два мильона
нацистов, но при этом отошли назад к Москве. Все ж так просто, ромочка.

а кто поправляет географию, наверное в школе не учился, иначе знал бы наверное,
что Украина - суверенное независимое государство, в чей состав входят АР Крым,
а также Донецкая область и Луганская тоже. Так вот, гопникам, которые принялись
кому-то поправлять географию, будут поправлены и мозги, и печенка, и сердце, и
почки. Потому что бить будут гопника все разом, и по наиболее чувствительным
точкам. Все это в процессе, крейсер Москва это только цветочки, и дальше будет
еще интереснее. Очень будет все интересно. Гопник же дурной и глупый, и не
понимает, что за наезд по беспределу обязательно будет ответка.

Не понимаешь, Егор. Не способный к изучению языков, ясно. Кроме магучего узика
ничего так и смог освоить... Так научим, не волнуйся. Придем и научим. Будеш по
утрам бегать на пробежку под звуки Государственного Гимна Украины. Будешь еще,
как миленький, хоть бы ты аж во Владивостоке бы прятался от мобилизации, воин
диванный ты этакий. Потому что знаешь, пословица такая есть, не буди лихо, пока
тихо. А вы Украину разбудили. Ну что ж - разбудили - будете получать по полной,
пока нам не надоест.

ivan demidov 2replied to ОДЕССА
одессочка. Украина уже тыщу лет как живет. и будет жить вечно. а рассея
сдохнет, это верняк. не веришь - ну так скоро сама проверишь. взвоешь еще как
взвоешь. жаль конечно что ты не испытала величие рассеи на корабле москва когда
в него попала ракета нептун... было бы проще здесь писать. а то лезешь, лезешь
со своими глупыми картинками, все никак не успокоишься. Ну что ж, придет время
и успокоишься. Будешь на коленях стоять, грехи свои вымаливать.

мирные жители. Вот я например - мирный житель. Никого не трогал, никого не
обижал. Жил себе-поживал в Киеве. И тут - на меня и мой Город - самый лучший
Город в мире - Киве - полетели ракеты, полез какой-то непонятный вонючий
кадыров. Полезла какая-то нечисть обрисованная буква Зю. Полезла нечисть,
полезла... Рвалась нечисть в Святой Город Киев, Город Лавры и Софии. Рвалась
нечисть, облизываясь на Киев и на Днепр, наш величавый Днепр. Но... получила
нечисть люлей. и убралась нечисть от Киева восвояси. По доброй воле убралась
))) и так будет и с нечистью в Херсоне, и в Мариуполе, и в Запорожской области,
и в Крыму, и в Донецке. Нечисть рашисткая будет разбита и изгнана в свою
протухшую берлогу. Русский злобный медведь будет окружен, и усыплен. Методично,
спокойно и беспощадно. А что же с тобой, Костя, будет. А плевать. Наверное,
исчезнешь где-то в складках времени... потеряется твой след и потомки о тебе не
вспомнят, н е вспомнят... что был такой Константин Жабин, пользователь
Одноклассников, тратящий свое время на бессмысленные комментарии. Вот и все.

в отставке, костенька. Снова куча бессмысленного словесного гавна про нациков.
Вы там наверное слово нацик уже по сто раз в день повторяете. Ну как попугаи...
Нацики, бандеры, укры, потом снова -
нацики-нацики-бандеры-хохлы-рассея-ура-рассея-вперед-рассея-непобедима-у-нас-есть-ядерное-оружие-и-нам-все-нипочем.
Ну так отставь свою глупость, свое тщеславие, свое высокомерие и гордыню, и
постарайся стать просто Человеком. Понимаешь. Просто Человеком, способным на
Сострадание, на Разум, на Анализ причинно-следственных связей. Человеком, у
которого есть Совесть и Сердце, а не только насквозь промытый мозг.

понимаешь. никогда не поздно стать Человеком, даже если всю жизнь прожил как
баран. Никогда не поздно осознать себя. Никогда не поздно начать думать.
Никогда не поздно проявить Сочувствие и Разум. Никогда не поздно, даже есть это
придет в твой последний день. Никогда не поздно.

мечтай дальше костенька. рассея по факту - это дохлый труп. идеологически,
ментально, религиозно. Труп и в политическом смысле, и в плане
научно-техническом. Это просто труп. А что насчет наших земель, насчет
незаконно захваченного Херсона. Сейчас там стоит оккупационный гарнизон. И что.
А я не удивлюсь... если например завтра их просто всех перережут. Так что
мечтай костенька дальше, витай в своих облаках. Только не плач, когда придет
жопа, настоящая жопа. Потому что за преступлением - а преступления России
против народа Украины совершенно ужасны и отвратительны во всех отношениях -
обязательно последует возмездие, обязательно будет наказание. Это ж еще
федормихалыч писал.

костенька. опять оскорбления. ну че ты такой злой. чем тебя там кормят, что ты
как дикая собака бросаешься на Людей.

ivan demidov 2 забавные вы рассеяны. животинки этакие злобные и паскудные.
лечить вас придется долго и тщательно, прежде чем вы вернетесь в нормальное
человеческое состояние.

зеленского обкуренного. может быть такое и было. Может быть раз где-то там
закурил косячок. Ну и что. Ну, покурил, ну и что. Это что смертный грех что-ли
разок покурить марихуану какую-нибудь, расслабиться. Не в этом же дело. Мы ж
его не за это уважаем. А в том - что он настоящий мужчина и отличный Президент.
В нужный момент он сделал все, чтобы Украина устояла, и он уже вошел в Историю.
И как раз здесь никакой наркоманией и не пахнет совершенно. В нужный момент он
- Зеленский Владимир Александрович - Президент Украинской Республики, законно
избранный шестой президент Украины - проявил и продолжает проявлять себя как
мужик с яйцами. Тут факты налицо. Так что нет костенька ты ошибся. Наркоша это
ты как раз Костя. Потому что это видно по твоему психо-эмоциональному
состоянию. По тому, что ты ничего кроме агрессии в окружающую среду не
транслируешь. Потому что ты в этих комментариях показываешь себя как злобного
зомбированного барана. И да, ты костенька ничем как раз не известен. Помрешь -
и никто о тебе особо и не вспомнит. Я вот например тоже не вспомню. Был такой
себе Костя в одноклассниках. Жил-жил себе, постил гадкие комменты про Украину и
украинцев. Потом умер. Похоронили. Вот и все.

о! тазик вещь хорошая. Но у нас подход другой немного. Для нас все что на
рассее говорят - в первом приближении - заведомая ложь и информационный шум.
Тут чтобы смотреть каналы рф - мы их не смотрим, но иногда правда в телеграме
видим отрывки - нужна позиция врача. Вот больные какие интересные, болезнь
называется рашизм. Целый народ заболел, целая огромная страна больна. Больна
злобой и глупостью, больна безумием и гордыней. Вот соловьев истерит, а вот
скабеева, животинка этакая. Интересно-с... а вот тута еще один зверек снова про
денацификацию снова говорит... (вроде это шахназаров стерилизацию предлагал).
Наверное, этот подход и есть тот пресловутый тазик, о котором вы говорите. Тут
вопрос есть еще, а что нам в Украине собственно со всем этим делать. Как
сделать так, чтобы рассея в конце концов успокоилась и перестала быть угрозой
Украине. Уничтожить физически всех поголовно рашистов - не получится заведомо.
А вот уничтожить идеологию, и перепрошить мозги, пробудить человечность,
совесть, разум, сердце, что в итоге приведет к покаянию перед измученной и
обескровленной Украиной - теоретически можно. Но тут нужна долгая кропотливая
работа. 

рассея в харьковской области. скоро рассеи там уже не будет. и кстати в
белгороде тоже. и в курске тоже. это все Украина будет, представляешь. Да и
масква тоже в состав Украины войдет. будет пограничная область Украины. На
Кремле будет украинский флаг развеваться. Все будет, любочка, не волнуйся. Ты
только потерпи еще немножко, и мы к тебе придем.

любочка. потопленный крейсер москва тебя на никакие мысли не наводит, а. пожар
в белгороде тоже нет. То, что рассеян во всем мире ненавидят и презирают, тоже
нет. То, что под Киевом получили люлей, и вообще уже почти как 30 000 ваших
парней, нет, не наводит на некоторые мысли. А очень жаль. Очень жаль. Похоже
надо еще пару десятков тысяч ваших войск положить из-за таких недалеких злобных
теток как ты, любочка.

мы кстати к любой инфе из рассеи, там... нацисты уж извини. равнодушны. это же
все ложь, это и ослику понятно. Ложь про нацистов, про великую непобедимую
армию, про биолаборатории, про наркомана зеленского, все несусветная ложь и
ерунда. Поразительно, да. Но вы так истово верите вашему путену, что просто
поразительно. Зло оно такое. Обаятельное, привлекательное, в галстуках и в
парадах. ветераны там, цветочки, буквы Зю везде. А по факту - ЗЛО. Огромное,
вонючее зло, залезшее в души миллионов людей на одной шестой суши...

ну вот пожалуй пока и хватит, любочка-солнышко. проблема таких как ты любочка,
что вы не хотите посмотреть на себя в зеркало. упорно не хотите. думаете, что
вы такие все белые и пушистые, прям святые все поголовно в рассее. прям одна
святость так и струится. а по факту - нация зомби. нация злобных диких обезъян,
неспособных ни думать, ни анализировать, ни сочувствовать. вот и все.

й там шалава по прозвищу добрая ведьма. мы ради интереса даже набрали слова из
твоей картинки. нихрена такого нет в гугле. как обычно солгала. Это и
неудивительно. Потому как ничего кроме лжи от рассеи ожидать нельзя. Лгали,
лгут и будут лгат. Пока не перестреляют как бешеных собак.

нацисты немецкие кстати лучше за вас были, рашистов. Те хоть не врали, говорили
как есть - мы евреев убьем, славян поработим. А вы, позор славян, вообще
неописуемое убожество. Мало того что убиваете и насилуете, воруете зерно, так
еще и врете. Безбожно врете по самое не могу. Ей-богу, это просто какой-то
феномен. Зомби-вруны в масштабах одной шестой суши. Уму непостижимо, как же
можно было в такое скотство впасть за такое короткое время

https://www.facebook.com/groups/story.kiev.ua/posts/1923987197798072/?comment_id=1924120101118115&reply_comment_id=1926254284238030&notif_id=1652686632863619&notif_t=group_comment_mention

ви і помиляєтесь, і одночасно праві. Так буває. Тому що у кожного українця, у
кожнього свідомого громадянина України - свідомого в сенсі - щирості своєї
любові до України - свій погляд на те, що є культурою України. Один любить
Булгакова, але вважає себе українцем, і ненавидить московію і рашизм. А інший -
плювати хтів на того Булгакова. Один памятає про 9 травня, а інший хоче
святкувати 8 травня. Один шанує пушкіна, хоч би хто про нього не казав як про
імперського поета, а інший, навпаки хоче знести всі памятники пушкіна. А
третьому байдуже, чи є той пушкін, чи нема. І тим не менш, вони всі на даний
час разом в одному окопі проти московії, проти тиранії і рабства. І так дале. А
згуртованість нації - це набагато більше, ніж просто згуртованість навколо
своєї культури і історії. Згуртованість - це практичний навик, перш за все. Це
здатність обєднуватись заради спільної мети.  Коли ти дійсно робиш щось для
цього, окрім гасел. Ми повинні обєднатись навколо мови! Ми повинні шанувати
культуру, українську! Ура! І тому подібне. Це набагато більше. Це досконале
знання і історії, і культури, і літератури, і більш того, розуміння що є
важливе, а що ні. і Тут треба було би цілу дисертацію написати. А щодо
культури. Тоді до вас питання. Іван Франко знав 14 мов, і написав чи то 50
томів, чи то 100 томів своїх праць. І якщо згуртованість розуміється в сенсі
згуртованості навколо питання історії, мови, культури, то треба підніматись до
рівня Франка і Лесі Українки. Так от, питання до вас. Якщо ви хочете
гуртуватись навколо культури, якщо ви хочете стати найрозумнішою почність
наприклад з Франка. Спробуйти піднятись до його рівня, хоча б подумки. Скільки
ви творів його прочитали, які висновки ви з них зробили. Що з того, що вони
написали, є більш важливим, а що ні. І тому подібне. Це є велике питання, що
таке є згуртованість нації, і в одному коменті, звичайно, не вмістити всю
глибину цього питання. А щодо того, що нам треба все це робити прямо зараз.
Помиляєтесь. Зараз нам треба виграти війну, бо інакше України просто не буде.
Просто тупо не буде. А ви цього очевидно ще не усвідомили в повній мірі. Так що
навчіться будь-ласка виставляти правильні пріорітети. Бо за кожною помилкою, за
кожним необдуманим словом або ж вчинком стоять життя конкретних людей.  І поки
ви тут хочете знести памятники, хтось сидить в окопі, хтось ранений страждає
від ран у госпіталі, а когось із наших воїнів може сьогодні вбють. Бо - війна
це війна. І те що ви тут можете писати коменти про культуру і так далі,
оплачено дуже дорогою ціною. Дуже дорогою ціною. Неймовірно дорогою ціною.

ну що за маячня. що це за пораженчество. бідкатись - від слова біда. Так чого
ви біду кличете. Треба не бідкатись, а брати руки в ноги. Розум, Сердце, Волю
покупи і йти допомогати своїй Вітчизні. Або брати кулемет, або волонтерити, або
заробляти гроші, а гроші віддавати на допомогу, або ж освоїти нову професію,
корисну для фронту, наприклад пошиття одягу для воїнів. Є купа всього чим можна
зайнятись, бо кожна хвилина на рахунку, бо кожна хвилина цінна, бо кожної
хвилини хтось зараз бється без сну і покою, а хтось вмирає на операційному
ліжку у воєнному госпіталі. А тренер - ну хрен з ним, тренером. Нікому цей
тренер не потрібен, окрім пані, що запостила пост. Нам він точно не потрібен.
Мова - ну так ми і пишемо, і говоримо мовою, і для нас це зараз є достатнім, в
чому питання. Отой згаданий тренер... Нам він точно не цікавий, нам цікава
Україна, нам цікава перемога, і нам цікаво, щоби Україна перемогла і розквітла.
А ви на жаль займаєтесь демагогією, і ваше бідкання особливо ні на шо не
впливає, вибачте. Ваш комент - це просто інформаційний викид в нікуди.

добре. ви хочете позбутись всього російського. Дуже добре. Може ми теж хочемо
так само, як ви, позбутись всього російського. Але на відміну від вас, у нас
більш прагматичний склад розуму, і ми навчені думати, перш ніж щось казати.
Можливо, це через те, що заробляємо гроші програмуванням, а у програмуванні
кожна помилкова літера означає, що програма просто не буде працювати. Також
відносно добре обізнані із природничими науками. А у природничих науках так
само. Якщо ти добре не подумав спочатку, і допустив помилку у розрахунках,
наприклад, при проектуванні літака, може статись велика біда. Люди сядуть у
літак, літак злетить. А потім станеться катастрофа, бо не врахували якийсь
важливий фактор. Якщо ти неправильно зпроектував мост, і пив пиво під час
розрахунків, так само, мост в певний час несподівано рухне, і загине багато
людей. А потім іди розбирайся, хто там був винуватий. Так от. Проблема у вас, і
більш широко, багатьох українців, що українці дуже розумні люди. Навіть занадто
розумні і талановиті люди. Неймовірно інтелектуально розвинуті і талановиті
люди. І через цю талановитість українці часто починають придумувати теорії,
придумувати собі химери, які на жаль, слабко вписуються у реальність, як би ця
химера не виглядала привабливо. От вірять люди у щось, дуже вірять. Дуже-дуже
вірять у шось. А їм  кажеш. А подумати не хочеш. Не хочу, йди собі, я і так все
знаю, нащо мені твоя думка. Так що змушені вас розчарувати, позбутись всього
російського, якщо брати за основу саме таке словесне формулювання, якщо брати
вимір цілої України - а ми мислимо не в термінах своїх особистих уподобань - а
мислимо щодо цілої України - це просто неможливо. фізично, ментально неможливо.
Неможливо культурно, неможливо мовленнєво, неможливо в багатьох речах. Це -
неправильна постановка цілі, мисленнєво неправильна. А неправильна постановка
цілі сама по собі, особливо під час війни, коли пріорітет взагалі повинен бути
на інших речах, призводить до насправді болісних втрат, до смертей, каліцтв
тощо. 

а, чому. тому що вони ненавидять Україну, їх тошнить від України. Тошнить від
української мови, від української культури. Вони всю Україну записали в
нацисти. Вони просто хворі. Вони просто стали зомбі, масово стали зомбі. Це -
клінічний казус в масштабах цілої країни. Це окрема велика тема для досліджень
психіатрів-вчених. Якщо ви не вірете - зайдіть заради цікавості в однокласники
- почитайте, скільки там лайна ллється щосекунди. Там не люди, а зомбі, і так
до них і треба відноситись. Те, що вони міняють таблички - а що, вам легше було
б, якщо б не міняли, чи що. От у анексованому Криму незаконна влада рф офіційно
встановила українську мову як третю державну. Вам від цього легше, чи ні.
Головне тут в тому, що Маріуполь - величезне, гарне місто - зруйновано. Стільки
горя, сліз, смертей. Головне в злі те зло є зло. Зло є зло, якою б мовою, якими
б гаслами воно б не прикривалося. А щодо табличок. А якщо путін раптом завтра
офіційно буде звертатись до українців українською, це щось змінить чи ні.
Маньяк і вбивця же залишається маньяком, якщо у нього в душі одна суцільна
темрява і зло, хоч би він всього Шевченка вивчив на спір. Гітлер от здається
захоплювався Моцартом.

крім того. є фактор інформаційного суспільства, у нього свої закони.
Інформаційний простір подібний великому океану, по якому весь час прокотуються
хвилі. Інформаційні хвилі. Коротко кажучи, є збуджувальні сигнали, на які
суспільство реагує, і виставляє певний пріорітет, як от наприклад та табличка,
які дають нелінійну експоненціально велику відповідь (почитайте у вікіпедії, що
таке є експонента). Є новина про табличку. Збудились, і дуже швидко. І є
постійні новини, що наші захисники там вмирають, і в оточенні. Так от.
Інформаційне збудження від факту таблички порівняно набагато більше, ніж
збудження від новин про захисників, що їм потрібна поміч, і шо вони тримаються
як супер-герої. Так от. На нашу думку, порівняно велике збудження від однією
таблички, порівняно з насправдні важливими речами, речами практичними, які
потрібно вирішувати прямо зараз, прямо зараз і негайно, інакше наші хлопці там
помруть, як от, стан наших захисників, є річчю досить таки ненормальною.

а щодо роскультури, росмови і її місця в Україні. треба проводити грунтовну,
цілеспрямовану ревізію. чи потрібно нам стільки памятників, чи ні. потрібне
широке грунтовне обговорення, потрібний широкий аналіз, глибокий аналіз, чому і
як. Це завдання для мирного часу. Це все звісно потрібні і важливі речі, і
культурна деколонізація повинна проводитись і щодо памятників, і щодо назв
вулиць тощо. Але вони мають інший порядок пріорітету, іншу структуру, інший
часовий вимір. Культурологічні питання, це питання глибокі, це питання, які
потребують розумових зусиль, великих розумових зусиль. І є ж іерархія
пріорітетів. Вам же знайомо таке слово, пріорітет, так. Так от, зараз війна. І
всі ці запеклі обговорення пушкінів під час війни - це марнослов'я, вибачте
вже. Поки пушкіна не підняли - нікому той пушкін не заважав, про нього просто
не думали. Думали про те, яка сьогодні лінія фронту. А ось підняли - комусь
засвербіло - і понеслась хвиля. Просто розтрата дорогоцінного часу і психічних
резервів суспільства. А вони не є безмежні, як і не є безмежними запаси зброї,
та кількість боєздатних захисників на лінії фронту..

Kiev Fortress Nina Ischenko може й так. але питання пушкінів, як ви самі вже
бачите, це не питання, чи працює метро, і як часто воно ходить, чи ні. Ось
питання про працююче метро, або чи вводити комендантську годину в Києві, або ж
чи можна вже проїхати в Бучу або Ірпінь, або там ще небезпечно, і не все ще
розмінували - це типово цивільні питання, і їх дійсно вирішує цивільна влада. І
добре, що так є. Це тилові цивільні питання. А питання пушкінів, достоєвських і
так далє - це глибинні світоглядні питання, які збурюють усе суспільство разом,
які здатні засмітити психіку і мізки на раз. Ми ж тута вже приводили анекдот
про професора з бородою. Чи був пушкін великим, чи він є імперець, що там про
мазепу, чи це повне лайно. чи то, чи це. ааа! ааа! голова вже тріщить, а все
пишуть і пишуть, все пишуть і пишуть. а по факту вже настрочили тисячу
коментів, і все в нікуди. Бо нездатні виставляти пріорітети, нездатні просто.
Все їм подавай все і зразу. І пушкіна знести прям зараз, і всі вулиці
перейменувати прям зараз. А що там на фронті. А на фронті все гаразд, ми
обовязково переможемо, перемога буде за нами, не хвилюйтесь... Бо можна було б
настрочити тисячу коментів, і купити чергову партію в тисячу броників, або
скинутись і купити черговий пікап. А тут у нас пікапа немає - а є срач про
пушкіна, з якого можна було би уже цілу книжку скласти, - і окупований Херсон,
який поки що ніхто ще не звільнив, і захисники в Маріуполі, які там кожний
божий день в пеклі воюють з нелюдями. Розумієте. Ось подивіться, є такий сайт
dou.ua. Це сайт програмістів України. Там є стаття про те, що за день (так, за
день!!! ) зібрали мільон доларів і купили якийсь там міні-літачок для фронту.
Оце круто. Оце классно. Оце неймовірно. Взялись, скинулись - і комусь внаслідок
цього врятували життя, і не одне. https://dou.ua/forums/topic/37811/

%https://www.facebook.com/eversti.rymin/posts/1187260715419516?notif_id=1652627483284696&notif_t=comment_mention&ref=notif
14:18:39 16-05-22

спасибо! пока что да, пока что они еще находятся в угаре. А что же в
дальнейшем. У нас такое понимание. С одной стороны поражения на фронте и сама
реальность, все более и более хреновее для них будут все больше и больше
вынуждать их как бы чувствовать эту самую реальность, шкурой своей
чувствовать. Конечно, это все болезненно, и они будут пытаться спрятаться -
что они и сейчас делают - от этой самой реальности - в свою зону комфорта
пропаганды - великого путена, великай рассеи, и все такое. Но мыслишки-то
будут проскакивать, хоть бы их и гнали прочь... что шо то не то...Что чтобы
там не вещали соловьев на пару с шахназаровым, все-таки шо то не совсем то.
Шо-то не совсем то. Такие же мыслишки то не сразу появляются и не у всех. Шо
шото не то творится, что надо, товарищи. Уже ж вроде мильон нацистов убили,
вон, всю авиацию украинскую уничтожили, но ничего так толком и не взяли, и не
добились. А шо ж такое, как же так. Вроде же и мирных жителей не убивали, не
насиловали, и вообще рассея вся белая и пушистая, и воюет с сатаной в лице
Запада, а чо ж тогда они нас - обычные граждане Украины - так ненавидят и
презирают. Шо ж такое. Как у классиков, знаете - меня терзают смутные
сомнения... А с другой стороны, есть такое. Мы сейчас живем в информационном
веке, и информация есть суть мощное оружие. Проблема в том, что та информация,
которая показывает, что наше дело правое, что они действительно воюют с
народом, что они действительно убийцы и зомби, пропитанные ненавистью и
злобой, пока что мало собрана. мало собрана в кулак, понимаете. Очень много
всякой информации, но она пока что разбросана по разным местам. 

Доброго дня. Щодо Пушкіна. Є така думка. По-перше, йде війна, страшна війна, з
аморальним ворогом, який вийшов за усі загальнолюдські моральні межі. Херсон
окупований, захисників Маріуполя дотепер не врятували, - купа інших питань, і
такі ідеологічні питання є більш питаннями мирного часу. Поки питання не
підняли, нікому той пушкін не заважав в сенсі воєнному. Але ящо вже підняли це
питання, то є така думка. Справа не в самому Пушкіні чи Достоєвському як
великому чи імперському поеті. Чи він там виспівував імперію чи ні, чи любив
Україну чи ні. Чи "винний" він у тому, що не завадив звірствам у Бучі та
Маріуполі. У нього є різні твори, і як поет, він був людиною з власною думкою і
власним поглядом, і не завжди він був у ладах із царським режимом. Так само як
і Булгаков, якого радянська критика чмирила весь час практично всього його
життя в срср. І до речі Іван Франко досить високо цінував Пушкіна, і навіть
поставив цитату з нього в початку повісті Захар Беркут, не беручи вже до уваги
його інші дослідницькі статті з літератури. Тут справа трохи в іншому. Це
питання деколонізації мислення і наяв скритого почуття страху, що якщо Пушкін
десь стоїть з 1964 року, то це є вже загроза існуванню України, що от Пушкін
стоїть, і він буде погано впливати на мізки підростаючого покоління. Що от
треба обовязково знищити, щоби навіть і духу його не було більше в Україні,
щоби тільки українські поети - а все інше, особливо російське - зась. Тут видно
совкове, навіть більшовистське мислення у тому, що от - а от він поганий, він
імперець, люди будуть ходити крізь нього, і він буде ідеологічно впливати на
них. Колись під час революцій (1917) так само зносили памятники царям, коли
розпалась росімперія. Во-перше, є історія України, і в певних періодах цієї
історії Україна була частиною тих чи інших імперій. І памятник Пушкіну - це не
якийсь новодєл, який поставили вчора під час війни як ідеологічну диверсію
проти України, а його поставили ще в 1964 році. Це частина історії України в
камені, частина радянського спадку, згадка про те, яким був Київ у ті роки, це
згадка про радянський період Києва та згадка що Україна колись була міцною
частиною ідеологічного простору срср. Так само як працююче Київське метро, -
або інші речі, наприклад Андріївська церква - взагалі спадок росімперії. Київ
сам по собі, як і інші міста України є сплавом різних епох та цивілізаційних
устроїв. Весь центр Києва до речі в сенсі архітектурному - це все срср та
росімперія в основному. Хрещатик, інше - це ж все було знищено, і відбудовано
заново. Так от, звісно неможливо тут все написати, бо це досить глибоке
питання, але. Дивіться, інші країни, особливо європейські, більш нормально
відносяться до своєї історії, яка була не менш кривавою, ніж наша.

Наприклад, в Англії стоїть памятник Оліверу Кромвелю - але він стратив короля -
його осудили як злочинця - але памятник стоїть, хоча в Англії монархія, і
Королівська Влада - один із стовпів англійської державності. Королева для них -
це щось сакральне. Ніхто в Франції не забуває про період в історії, коли були
королі і феодальна знать - хоча гімн у них Марсельеза. Так от, щодо росімперії
та срср, вони вже давно розпалися, і памятники - є лише згадка про них, згадка
про колоніальний період в історії України. Це згадка про минуле, що от колись
так було. Ніякої особливої загрози ідеологічної самі по собі ці памятники не
несуть (хоча навпевне їх все ж таки трохи забагато, як і назв вулиць), бо вже
сама історія винесла вирок тим режимам - срср та росімперії. І відносинам
України і росії так само винесено вирок,. Самою росією, війною, сама росія
застрелила і вбила будь-яке позитивне підношення до росії як до держави, та
росіян як нації. Але позбутись взагалі всього російського на світоглядному та
культурному рівні - це трохи інша річ, і це окрема тема, і це є тема чутлива
для багатьох. І так чи інакше, всі твори пушкіна і так можна прочитати в інеті,
і хоч цілий день сидіти і дрочити на "геніального" царя петра, яким він його
описав у Полтаві. А у нас зараз інформаційна епоха, і основний вплив йде через
інтернет та змі. І треба напевне виробляти якесь більш рівне, розсудливе
мислення щодо своєї історії. Так от. Крім того, ще зараз війна, і ми думаємо,
що взагалі треба витрачати свої інтелектуальні і психічні резерви на питання
практичні, як-то - чи волонтерити, чи йти в армію, чи воювати на кіберфронті
проти московії. Бо перемоги ще ніякої нема, а є важка війна з ворогом, і кожна
людина, що свідомо і з душею вкладається задля перемоги, є важлива.

дякуємо. Якщо вас хвилюють колоніальні мітки в сенсі культурному, то, як ми вже
сказали, колоніальних міток в абсолютному сенсі цього слова позбутись не можна
взагалі, бо неможливо видерти сторінки з історії, неможливо видерти той період,
коли Україна була частиною інших держав. Ми абсолютно не проти перейменування
вулиць або станцій метро. Але до цього треба підходити творчо, з розумом. І є
також проблема в мисленні. Саме по собі мислення є колоніальне, що ось є мітки,
що ось колись був срср і росімперія тут у нас, і треба всі ці мітки вичистити,
щоби і згадки про колоніальний період не було. Це вибачте трохи є абсурдна
постановка питання, це мисленнєва пастка, яка веде в нікуди. Крім того, якщо
вас хвилює Пушкін і його вплив на душі українців, то основний вплив Пушкіна -
як і будь-якого іншого поета та письменника, йде не через памятники, а через
слово, через його твори, які легко можна знайти в інтернеті. Ми вже згадали про
Івана Франка та його твір Захар Беркут. Відкрийте - перші рядки - цитата з
Руслана і Людмили. Колоніальна мітка - для вас можливо. Але цю мітку видерти
неможливо, бо ця мітка вже живе в літературі. Тому... добре, досить. У нас
немає задачі тут багато щось доводити, просто рано встали сьогодні і були такі
думки. Але напевне досить вже строчити.

росмова - це також і мова друзів України по всьому світу, які збирають гроші,
виходять на демонстрації проти путіна, окрім мови багатьох патріотів України,
та багатьох видатних людей, що зробили багато для України в минулому, як-то наш
лікар Микола Амосов. Окрім того, це мова творів Григорія Сковороди, нашого
видатного українського філософа-просвітителя, якому в цьому році 300 років. Не
вірите - ну так самі прочитайте. Чи ви тільки путіна з соловьовими і дивитися,
що для вас все, що говориться або пишеться росмовою умістилось у зловісне
московитське лайно? Може час вже позбуватись звички весь час оглядатись на
москву, позбуватись ментального рабства, плазування перед московією? І більше
того, Ви можете скільки завгодно гаркати на росмову, або будь-яку іншу мову,
але Україні врешті решт це мало допоможе. Бо мова сама по собі - будь-яка мова
- і та ж російська мова - яка має певний негативний тягар минулого - має оцей
імперський шлейф з минулого  - це окремий організм, живий організм, в
лінгвістичному сенсі, який живе за своїми законами, як інструмент живого
спілкування людей. Як приклад. Латинська мова була у свій час мовою Римської
імперії, зловісної, хижої імперії, яка нічим не була краще московії. Але
імперія зникла, а латина дотепер живе як мова творів, мова молитви, і навіть як
офіційна мова Ватикану.  А пройшло вже більш ніж тисячоліття з моменту розпаду
імперії! Тому ми можемо вас розчарувати - росмова в Україні не зникне, як би
цього комусь не хотілось би. А за що ми воюємо, за Україну, звісно. А Україна в
найглибшому сенсі - це Воля. Вільна Земля Вільних Людей. Звісно, українська
мова займає дуже важливе значення в сутності України, і її треба розвивати, і
розмовляти нею, співати нею. Але не через коліно, а через Сердце і Розум. 

простим перейменуванням на рівні офіційному ви на жаль політику московії щодо
нас не зруйнуєте, бо це тільки початок, хоча й добрий початок. Це суто
офіційний вимір, який по факту мало що ще змінить, на жаль - хоча початок
добрий, бо дійсно правильно рф називати московією, а росіян - московитами.  А
для остаточної та повної перемоги України над московією, щоби московія зникла
взагалі як явище, зруйнувати треба ідеологію московії на рівні мізків мільйонів
людей, на рівні звичайного пересічного громадянина в московії. Отоді дійсно
ідеологія московії буде зруйнована. Це дійсно повинна бути масштабна робота на
рівні мільйонів людей, і тут потрібні дійсно великі інтелектуальні зусилля,
щоби таки зламати навіки цю зловісну ідеологію.  Бо перед ракетами і бомбами на
наші міста на протязі багатьох років йшла масована та цілеспрямована обробка
мізків населення, яка перетворила росіян на стадо зомбованих баранів (якщо
подивитись на те, що вони пишуть у своїх соцмережах, то ніякого іншого враження
і не виникає - а це тонни і тонни інформаційного лайна ).

так. наміри добрі, але є велика різниця в тому, між намірами та
перейменуваннями на офіційному рівні, та фактичним зруйнуванням ідеології
московії на рівні мільйонів людей. І ми ставимо так питання. Насправді в
коментах тут  Яким чином, теоретично та практично, на масовому рівні мільйонів
людей, на рівні соцмереж треба вести контр-пропаганду, треба роз-зомблювати
зомбі (тобто перетворювати раша-зомбі назад в людей, через усвідомлення своєї
колективної провини, через щире покаяття перед Україною ), щоби ідеологія, яка
засіла в мізках мільйонів людей, була зруйнована. Це насправді важливе питання,
бо одне діло теоретизувати, от перейменуємо оте, або те, і одразу вже щось
почне валитись там у них. Насправді, ні. Ідеологія московії - досить інерційна
річ, досить отруйна річ, неймовірно отруйна штука ментально, і тут потрібна
насправді систематична важка робота на протязі років, і на рівні мільйонів
звичайних громадян. 

ви трохи помиляєтесь, і праві одночасно. Проблема в тому, що ті люди (які є
зомбі), яким марно щось доводити як здоровим адекватним людям, для яких не
працює аргументація та факти, є насправді просто психічно хворі люди. Люди
хворі ментально. Хворі злобою, ненавистю, жагою до імперської величі. У
звичайних людей таке відхилення, наприклад, називається манія величі. І таких
людей в московії мільйони. І якщо задатись ціллю знищити ідеологію московії,
треба прийняти позицію, що ціла країна є психічно хворою, і з цих позицій
розробляти відповідну методику. Ну, ми так думаємо. Потрібно трохи навчитись
прийняти позицію психіатра, який має справу із психічно хворою людиною, і
дивитись на проблему московїі з цієї точки зору. Бо якщо зовсім нічого не
робити, і залишити зомбоване населення московії само собі, то будуть і нові
путіни, і нові війни, і нове лихо. московія зараз - це зла мавпа, яка
замкнулась сама на собі. Ну, ми це кажемо з точки зору психології, так би
мовити. А щодо перейменувань - після вторгнення і трьох місяців цього всього
лиха люди в переважній більшості своїй, і на тому ж південному відносно більш
російськомовному Сході, вже дуже добре зрозуміли і так, адекватні люди, що є
таке московія насправді. Сама ж московія їм в цьому і допомогла, повністю
розкривши свою внутрішню брехливу та отруйну сутність.

Доброго дня! Слава Україні! До речі. Якщо вам цікаво таке -  ми пишемо літопис
війни, збираємо матеріали, публікації, потім структуруємо по різним темам про
війну - пости з фейсбуку, пісні, вірші, аналітика, волонтерство, по містам
також - Харків, Київ тощо. Потом оформляємо у вигляді окремих (часто досить
великих файлів на сотні сторінок) PDF-файлів, які викладуємо у хмарне сховище
(як-то гугл-диск). Якщо вам, цікаво, ось посилання на гугл-папку, де ми
розмістили файл про Харків (198 сторінок, 73 мегабайти). Там є навігація по
заголовкам публікацій та авторам. Скріншот файлу прикладаємо. Мотивація в тому,
щоби зберегти та структурувати правду про події в Україні. Бо хоча в фейсбуку і
телеграмі дуже багато публікується, але самі по собі телеграм та фейсбук в
принципі є ненадійними. Бо сьогодні пост є, а завтра фейсбук взяв і видалив
пост. От. Нижче ми посилаємо два скріншота з файлу, а потім саме посилання на
гугл-папку. Сподіваємось, цей літопис вам стане у нагоді. Попереду є ще багато
роботи.

“Змія і світлячок” (Притча).
Змія почала переслідувати світлячка, який літав низько у траві. Світлячок зупинився і промовив до неї:
– Чи можу я задати вам три питання?
Змія відповіла:
– Так.
Світлячок розпочав:
– Я належу до вашого харчового ланцюжка?
Змія відповіла:
– Ні.
Тоді світлячок продовжив:
– Я зробив вам щось погане?
Змія знову ж відповіла коротко:
– Ні.
Тоді світлячок задав їй останнє питання:
– Тоді скажіть, чому ви хочете мене з’їсти?
Змія без жодних роздумів відповіла:
– Тому що мені нестерпно бачити, як ти..... сяєш.
Мораль цієї історії:
Часто деяким людям нестерпно бачити, як ви сяєте, тому вони діють як змії, тишком-нишком і готові знищити вас просто за те, що ви є щасливішими/красивішими/успішнішими/… за них.
Є два типи заздрісних людей: перші хочуть все, як в тебе, а інші – щоб в тебе взагалі нічого не було… Бережіться і тих, і інших!

Знаєте, шо є в нашого народу, а нема в тих у*обків? Справжні жінки! Нє, не баби, які якогось хєра в горящу ізбу лізуть, остановівши коня на скаку…
Наші жінки не такі дурні, щоб за конями по полю ганятись. Ще два місяці тому вони планували робити кар’єру, няньчити діток, готувати якісь смаколики, стояти в планці, бо ж скоро пляжний сезон, і купити ту таку гарну весняну сукенку в горошок. Але тут якесь падло вирішило ці плани порушити…
І поки русскі матері вздихають по втраченим синам, думаючи, як то ше нарожать для русскої їпанутої армії, наші жінки розізлились. Лють української жінки настільки сильна, що, здається, навіть на небесах зараз тихенько всі присіли і повдягали шапки!
Українська жінка не боїться стояти за спиною в свого воїна. Вона його підтримує і оберігає. Вона ліпить тонни вареників на фронт, плете кілометри сітки, возить допомогу, шукає бронежилети, тягає мішки з піском на блокпости. За 24 години вона організує тобі 20 фур з продуктами, з 10 кілометрів маскувальних сіток, гарячі пиріжки і БТР!
Вона народжує в сирому смердючому підвалі прекрасних дітей, які з її молоком вбирають всю силу материнського духу…
Вона публічно говорить слова, які не мають знати її діти, але слова ті правдиві і сповнені люті…
Вона кидає коктейль, б’ється у всі двері, мерзне, не спить, посміхається, плаче, стає перед танком, затуляє собою, прикрашає коциками бомбосховища, популяризує квіти і стає Бабою Надьою століття… І при цьому не забуває набрати і запитати: “Ти точно поїв і в шапці?”.
І коли це все закінчиться, вона зніме з себе цю лють, покладе на очі патчі, шоб приховати легку втому, поправить зачіску, і таки одягне ту бажану сукенку в горошок, шоб красиво сходити в магазин за гарним тортиком по засіяній соняшниками алеї.
Але вона ніколи не забуде…
Бо нєхєр лізти в наші, курва, плани!
Ольга Магас





https://drive.google.com/drive/folders/1Ei0FyhFFLDctQ91F3rbWs2o3kF8rmz3Y?usp=sharing
