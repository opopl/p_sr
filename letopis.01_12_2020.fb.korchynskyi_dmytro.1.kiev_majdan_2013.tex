% vim: keymap=russian-jcukenwin
%%beginhead 
 
%%file 01_12_2020.fb.korchynskyi_dmytro.1.kiev_majdan_2013
%%parent 01_12_2020
 
%%url https://www.facebook.com/korchynskyi/posts/3529658813766292
 
%%author Корчинський, Дмитро
%%author_id korchynskyi_dmytro
%%author_url 
 
%%tags 
%%title Сім років тому, 1 грудня 2013 року на вулиці Києва вийшли сотні тисяч обурених громадян
 
%%endhead 
 
\subsection{Сім років тому, 1 грудня 2013 року на вулиці Києва вийшли сотні тисяч обурених громадян}
\label{sec:01_12_2020.fb.korchynskyi_dmytro.1.kiev_majdan_2013}
\Purl{https://www.facebook.com/korchynskyi/posts/3529658813766292}
\ifcmt
	author_begin
   author_id korchynskyi_dmytro
	author_end
\fi

Сім років тому, 1 грудня 2013 року на вулиці Києва вийшли сотні тисяч обурених
громадян. Стільки не виходило ні до ні після.  Влада не очікувала такого і не
попіклувалася зібрати достатню кількість правопохоронців для протидії.

Я, як історик-аматор, одразу зрозумів, що революцію можна здійснити за один
день - захопити центральні державні установи. Президент якщо його вигнати з
Адміністраці Президента вже ніким, навіть силовиками не буде сприйматися, як
президент.

\ifcmt
pic https://scontent.fiev6-1.fna.fbcdn.net/v/t1.0-9/128880449_3530098157055691_5746033568520646444_n.jpg?_nc_cat=105&ccb=2&_nc_sid=730e14&_nc_ohc=-Q8aeWFHclIAX8TPfng&_nc_ht=scontent.fiev6-1.fna&oh=9b532d503f6ecf91f52c169aabb24b38&oe=5FEE21D2
fig_env wrapfigure
width 0.4
\fi

До адміністрації, прихопивши грейдер для пробивання встановлених ментами
загород і попрямувала кмітлива молодь.  Так розпочався перший в історії
новітньої України справжній штурм влади.

Чисельність повстанців на Банковій з кожною хвилиною збільшувалася.

Спецпризначенці намагалися оборонятися газом та світлошумовими гранатами: але
зростала їхня розгубленість.

На жаль парламентська опопзиція перелякалася можливості швидкої перемоги,
перелякалася гніву януковича вразі невдачі повсталих й вирішила пустити
революцію за тим сценарієм за яким нині здійснюється білоруська \enquote{революція} -
за безрезультатним.

Ці демократичні лідери стали між повстанцями і беркутом, щоб захистити беркут.
Вони переконали більшу частину людей піти геть. Найбільш активних вони об'явили
провокаторами і найманцями януковича, чомусь навіть мене, хоча я не мав до того
жодного стосунку (я ніколи ні до чого не маю стосунку, окрім як до української
літератури). Деякі з вождів Майдану аби продемонструвати свою лояльність до
влади, навіть написали на мене заяви в міліцію.

За кілька годин, коли на Грушевського залишилося мало людей - переважно
випадкових громадян і журналістів -  беркут наважився піти в атаку, щоб
помститися. Але між ним і громадянами жоден демократичний опозиціонер не став,
щоб захистити громадян, як до того захищав беркут.

Найкращій частині повстанців тоді довелося сховатися аж до початку активної
фази революції.

Я опинився в дурному становищі: з одного боку мене шукають менти і сбшники,
вдираються в мою домівку з обшуком, оголошують терористом, подають мене у
національний та міжнародний розшук, з другого боку, введені в оману майданівці
вважають мене провокатором і найманцем януковича, ненавидять і цькують. Їхні
вожді прагнуть видати мене ментам, якщо я з'явлюся на Майдані.

Саме така поведінка легальної опозиція призвела до того, що революція
розтягнулася на місяці, замість того, щоб перемогти за години, що менти вбили
сотню людей, що багато разів Майдан був на волосину від поразки і лише дивом
вистояв.

Згодом, свої методи боротьби з януковичем, вони вирішили застосувати до
боротьби з путіним, тому в нас забрали Крим і ледь не забрали решту України.

Втім, в порівнянні з нинішньою владою вони здаються робесп'єрами й наполеонами.
Хоча треба віддати належне і зеленському: янукович таки гірший за нього, бодай
трошки...
