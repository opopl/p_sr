% vim: keymap=russian-jcukenwin
%%beginhead 
 
%%file 30_04_2022.fb.pogrebnoj_jurij.herson.1.koloborant
%%parent 30_04_2022
 
%%url https://www.facebook.com/yuripogrebnoy/posts/5291641687585809
 
%%author_id pogrebnoj_jurij.herson
%%date 
 
%%tags 
%%title Колоборант - человек, который сотрудничает с оккупантами во вред интересам своего государства (Википедия)
 
%%endhead 
 
\subsection{Колоборант - человек, который сотрудничает с оккупантами во вред интересам своего государства (Википедия)}
\label{sec:30_04_2022.fb.pogrebnoj_jurij.herson.1.koloborant}
 
\Purl{https://www.facebook.com/yuripogrebnoy/posts/5291641687585809}
\ifcmt
 author_begin
   author_id pogrebnoj_jurij.herson
 author_end
\fi

Колоборант - человек, который сотрудничает с оккупантами во вред интересам
своего государства (Википедия).

В очень сложное положение поставлены сейчас в оккупированном Херсоне врачи,
учителя, работники коммунальных предприятий, банковские служащие, городской
голова и главы районных советов. С одной стороны - надо обеспечивать
жизнедеятельность города, а с другой - можно потом получить обвинение в
колоборационализме.

Там, где кто-то принял добровольное решение и назначен на какие-то должности
оккупантами - всё понятно. А что делать, например, работникам водоканала,
Херсонгаза, Облэнерго и т.д.? Их же фактически подчинили оккупационным властям.
Сабботировать? Под дулами автоматов?

До сих пор никаких вразумительных рекомендаций этим людям не дали ни бывший
глава ОДА, ни кабмин, ни СБУ...

И ещё есть одна неприятная история. Она касается депутата горсовета
\href{https://www.facebook.com/karamalikov.ilya}{Илья Карамаликов}, который
после оккупации Херсона, когда не осталось в городе ни одного украинского
полицейского и ни одного представителя исполнительной власти, кроме мэра,
организовал и координировал работу городской народной дружины. Они, как
волонтёры, охраняли город от мародёров, развозили гуманитарку, следили за
общественным порядком и т.п. Сейчас Карамаликова СБУ задержала и предъявила
обвинение в государственной измене за контакты с военными оккупационного
гарнизона. А как можно было не контактировать, если народная дружина работала и
в период комендантского часа, в том числе, ночью?  Или надо было нихрена не
делать и дать мародёрам разграбить город?

Я, конечно, не знаком со всеми материалами этого дела и не могу сейчас
категорически за кого-то ручаться, но со стороны похоже, что Илья - один из
немногих депутатов -  взялся помогать родному городу и пострадал (((

Сейчас множество людей в городе не понимают, как себя вести, а чётких
разъяснений из Киева пока нет.

Я не очень дерзко изложил суть проблемы?

\ii{30_04_2022.fb.pogrebnoj_jurij.herson.1.koloborant.cmt}
