% vim: keymap=russian-jcukenwin
%%beginhead 
 
%%file 31_03_2021.fb.fb_group.story_kiev_ua.1.ukr_hleb.pic.6.cmt
%%parent 31_03_2021.fb.fb_group.story_kiev_ua.1.ukr_hleb
 
%%url 
 
%%author_id 
%%date 
 
%%tags 
%%title 
 
%%endhead 

\iusr{Шлёма Облап}
в селе или в 90-е?

\iusr{Петр Кузьменко}
\textbf{Шлёма Облап} это в Киеве в 80-х, я думаю...

\iusr{Марина Теклишина}
\textbf{Петр Кузьменко} вряд ли такая цена могла быть: украинский стоил в 80-е 28 коп.

\iusr{Петр Кузьменко}
\textbf{Марина Теклишина} возможно. В конце 80-х мы были очень далеко от родного Города.

\iusr{Шлёма Облап}
батон 22 коп, укр 14-16, арнаут 16, паляныця 18, пшеничный 28

\iusr{Yuri Arkadyev}
\textbf{Shlyoma Oblap} ещё булочки по 3 копейки... вкууусные  @igg{fbicon.wink}  и рогалики с вареньем

\iusr{Юлия Павлова}

В моем детстве, я с подругой складывались по 2 копейки и покупали чертветинку
хлеба, и получался импровизированный обед на свежем воздухе

\iusr{Любов Огородня}
на вітрині, поруч з українським, видно паляницю

\ifcmt
  ig https://scontent-frt3-1.xx.fbcdn.net/v/t1.6435-9/167992368_4033639663370399_1641448445583048311_n.jpg?_nc_cat=102&ccb=1-5&_nc_sid=dbeb18&_nc_ohc=rv8m-Mlsnk4AX-IYZ4T&_nc_ht=scontent-frt3-1.xx&oh=00_AT_Xp_NdOBWiDojX0BtCpmMTvWEhg8Vaj5OiRVI0AXH_6g&oe=61DDAF24
  @width 0.3
\fi

\iusr{Vasili Vlasenko}

Украинский, киевской выпечки, имел особый вкус. Наши черновицкие друзья всегда
увозили домой пару буханок Украинского. И сейчас, живя в Америке, часто
вспоминают его вкус.

\iusr{Yuri Kladiiov}

Колись (за совітів)випікали хліб українсьний вагою 1кг та 1,2кг і ніхто
навіть не міг і уявити вкрасти 50 грамів на одному кг, а продавати у тисячу
разів дорожче!

