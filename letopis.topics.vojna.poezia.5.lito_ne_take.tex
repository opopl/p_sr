% vim: keymap=russian-jcukenwin
%%beginhead 
 
%%file topics.vojna.poezia.5.lito_ne_take
%%parent topics.vojna.poezia
 
%%url 
 
%%author_id 
%%date 
 
%%tags 
%%title 
 
%%endhead 

 літо не таке...
 і люди, мов примари...
І небо голубе кудись веде отари...
І ми всі не такі....
Розкидані по світу...
Часи тепер важкі...
Гуляють душі з вітром...
Надщерблено життя...
Розірвано на шмаття...
Дорога в укриття...
Ну, що? Спасибі "браття"
За літо... і за лютий...
За те, що син без тата
За те, що день прикутий 
Чорнющим словом: "втрата"...
Тепер нам вже знайомі
Тісні зв'язки "родинні"
Коли стріляють в домі
Сказавши: "самі винні"
І сиплять бомби, гради...
Туди, де просто жили,
Де не чекали зради,
А рідним дорожили
Де зустрічали сонце,
Вели дітей до школи
Де Бог був охоронцем, 
Гуділи влітку бджоли
І все якось складалось
В малесенькі хвилини
Можливо, щось не вдалось...
Та ми у себе жили
У себе в рідній хаті 
В маленькій Україні
По своєму багатій,
По своєму єдиній.
Не має вам тут місця,
Не має і не буде!!!
Любов, що в нашім серці 
Не зникне вже нікуди
Ніколи та нізащо
Її не загасити
Не може люд пропащий 
Нас вчити, як любити!
                      ©️ Аліна Войтенко
