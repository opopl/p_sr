% vim: keymap=russian-jcukenwin
%%beginhead 
 
%%file 21_06_2021.fb.buzhanskii_max.1.vojna_22_06_1941
%%parent 21_06_2021
 
%%url https://www.facebook.com/permalink.php?story_fbid=1973295642834905&id=100004634650264
 
%%author 
%%author_id buzhanskii_max
%%author_url 
 
%%tags buzhanskii_maksim,germania,istoria,mirvojna2,nacizm,sssr,vojna,vov,vov.22.06.1941
%%title А завтра была война
 
%%endhead 
 
\subsection{А завтра была война}
\label{sec:21_06_2021.fb.buzhanskii_max.1.vojna_22_06_1941}
\Purl{https://www.facebook.com/permalink.php?story_fbid=1973295642834905&id=100004634650264}
\ifcmt
 author_begin
   author_id buzhanskii_max
 author_end
\fi

\obeycr
А завтра была война.
Это неправда, что все знали и чувствовали, это в мемуарах потом напишут военачальники, которые частично действительно знали, а у остальных, у остальных послезнание ужаса.
Да, говорили, не дремлет враг, сплотить бы ряды ещё на пятилетку и тд, но ведь сплотить, а не сеять панику.
А потом наступило завтра.
Кто то стоял в этой толпе на двести миллионов человек у репродуктора утром, хмуро стоял, видел десятки фото, хмурые все, а не испуганно.
Кто то уже ехал на фронт, ещё до того, как у репродуктора ехал, не сидел, стоял всю дорогу, не было сил сидеть.
Кто то прибежал на работу, метался пустыми коридорами института, гулкое эхо шагов, запертые двери.
В эти двери кто то уже никогда не зайдёт, вообще никуда не зайдёт, дверная ручка, до блеска отполированная тысячами прикосновений, такое простое и привычное движение, и больше никогда...
У кого то был брат.
Ещё сегодня был старший брат, даже попрощаться не успели, просто обнял в тесной прихожей, уже выходя в дверь, и всё, навсегда.
Что то важное вспомнится, миллионы людей вспомнят что то важное, то, что так никогда и не сказали тем, за кем только что гулко захлопнулась дверь.
Будут писать, крошечные бумажные треугольнички понесут эти слова, кого то найдут, кого то успеют найти, кого то уже навсегда нет.
Обычная мирная суббота огромной страны.
Парк с девушкой, фонтаны, мороженое, обычный летний день.
Вечером за учебники, но завтра всё равно воскресенье, можно просто отдохнуть, ни о чём не думать, посвятить время себе.
Кто то переживал, что дети поссорились, что младшие разбили окно соседу футбольным мячом, а старший-упрямец, и хочет поступать куда то не туда, куда хотел бы отец.
Обычный день, соленые брызги моря, Керчь и Одесса, коробки рыбачьих лодок на горизонте, ленивая жара, семечки.
Вы думаете, это смешная деталь, семечки?
Нет, семечки, это мирный денёк, беззаботно щёлкаешь, лениво наблюдая, как летит время...
Впереди у миллионов людей была жизнь.
А завтра....
А завтра была война.
\restorecr

\ii{21_06_2021.fb.buzhanskii_max.1.vojna_22_06_1941.cmt}
