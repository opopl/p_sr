% vim: keymap=russian-jcukenwin
%%beginhead 
 
%%file 12_11_2021.fb.fb_group.story_kiev_ua.1.leto_1979_lavra_pushka
%%parent 12_11_2021
 
%%url https://www.facebook.com/groups/story.kiev.ua/posts/1796225080574285
 
%%author_id fb_group.story_kiev_ua,bubnov_jurij
%%date 
 
%%tags 1979,kiev,lavra,pushka,rodina_mat.kiev
%%title Лето 1979-го - прогулка возле Лавры
 
%%endhead 
 
\subsection{Лето 1979-го - прогулка возле Лавры}
\label{sec:12_11_2021.fb.fb_group.story_kiev_ua.1.leto_1979_lavra_pushka}
 
\Purl{https://www.facebook.com/groups/story.kiev.ua/posts/1796225080574285}
\ifcmt
 author_begin
   author_id fb_group.story_kiev_ua,bubnov_jurij
 author_end
\fi


Лето 1979-го. Не помню, по какой причине я оказался в районе частного сектора,
который находится или находился (так быстро всё меняется) между Киево-Печерской
Лаврой и будущим монументом Родина-мать. 

\begin{multicols}{2} % {
\setlength{\parindent}{0pt}

Пройдя по узким улочкам, с обеих
сторон огороженными заборами, за которыми стояли совершенно разные, похожие на
дачные, дома и поднявшись по тропинке вверх настолько, насколько это было
возможным, передо мной открылась такая панорама, что не сделать несколько
снимков я просто не мог. Жаль, но в моем архиве сохранились только две
фотографии.

\ii{12_11_2021.fb.fb_group.story_kiev_ua.1.leto_1979_lavra_pushka.pic.1}
\ii{12_11_2021.fb.fb_group.story_kiev_ua.1.leto_1979_lavra_pushka.pic.2}
\ii{12_11_2021.fb.fb_group.story_kiev_ua.1.leto_1979_lavra_pushka.pic.2.cmt}

\end{multicols} % }

Расскажу интересный, на мой взгляд, эпизод. Поднимаюсь по узенькой, едва
протоптанной тропинке к месту съёмки. Подъём был достаточно крутой и местами
приходилось вставать на четвереньки, чтобы двигаться дальше. Так вот, в
очередной раз, находясь в указанной позе, я увидел прямо перед глазами что-то
блестящее, величиной с ладонь. Этим \enquote{что-то} оказался металлический предмет,
отполированный до блеска ногами и руками, таких же \enquote{путешественников}, как и я.
В нём угадывались округлые бока, и подумав, что это неразорвавшийся снаряд,
решил обойти предмет стороной. Но, любопытство взяло верх, и меня остановило
также то, что металл был красно-оранжевого цвета, явно похожий на медь.
Неподалёку нашел старый ржавый обруч от бочки и осторожно приступил к работе.
Грунт был очень плотный, тут нужен был лом. Провозившись минут сорок, и
раскопав предмет на глубину сантиметров пять и определив его начало и конец, я
увидел перед собой маленькую, не больше метра в длину, но массивную старинную
пушку, стрелявшую когда-то ядрами. По-моему, такую или похожую, я видел в
Историческом музее. Замаскировав свои \enquote{раскопки} землей и травой, уже
планировал, как приеду с товарищем, раскопаем эту пушку и .... Меня остановило
то, что судя по внешнему виду, вес этой пушки был такой, что для того, чтобы её
поднять, потребовалось бы человека четыре. А если она была ещё и на лафете? В
общем, на этом всё закончилось, а вскоре и забылось. В тех краях я больше не
бывал.

\ii{12_11_2021.fb.fb_group.story_kiev_ua.1.leto_1979_lavra_pushka.cmt}
