% vim: keymap=russian-jcukenwin
%%beginhead 
 
%%file 14_07_2021.fb.jushkov_slava.1.mova_pochemu
%%parent 14_07_2021
 
%%url https://www.facebook.com/SlavaYushkov/posts/2916788325261593
 
%%author_id jushkov_slava
%%date 
 
%%tags identichnost',mova,ukraina,ukrainizacia
%%title ЧОМУ Я ГОВОРЮ УКРАЇНСЬКОЮ?
 
%%endhead 
 
\subsection{ЧОМУ Я ГОВОРЮ УКРАЇНСЬКОЮ?}
\label{sec:14_07_2021.fb.jushkov_slava.1.mova_pochemu}
 
\Purl{https://www.facebook.com/SlavaYushkov/posts/2916788325261593}
\ifcmt
 author_begin
   author_id jushkov_slava
 author_end
\fi

ЧОМУ Я ГОВОРЮ УКРАЇНСЬКОЮ?

Я ще жодного разу не відповідав на це публічно і завжди плутано особисто.
Мабуть, настав час вбити головну інтригу цього акаунту. Все ж таки вже 10
місяців послуговуюсь мовою. Але помре інтрига лише в кінці посту після повного
ознайомлення з ним. Так, я підступне падло, але бачили очі на кого
підписувалися.

\ifcmt
  pic https://scontent-yyz1-1.xx.fbcdn.net/v/t1.6435-9/216976420_2916788131928279_930212383876704962_n.jpg?_nc_cat=105&_nc_rgb565=1&ccb=1-5&_nc_sid=730e14&_nc_ohc=rZt7GrpDLK0AX8i12HN&_nc_ht=scontent-yyz1-1.xx&oh=2328fedc170a714f7d990339ee6ebceb&oe=6179D42C
  @width 0.8
\fi

Для тих, хто тут чисто лайкнути фотку і най буде, є проста відповідь. Я
перейшов на українську, бо в результаті занурення в український культурний та
історичний контекст та пошуку відповідей на питання самоідентичності, аргументи
за перехід на мову перемогли інертне бажання мозку триматися за \enquote{великий и
могучий язык Кивы и Моргенштерна}. Так, це проста відповідь.

Якщо ще простіше, то подивіться серіал \enquote{Друзі} в українському дубляжі, а потім
на \enquote{языке, который нужно защищать танками и долбоебизмом}. Невже ще є якісь
питання? Тоді ласкаво прошу в мій світ відсутності лапідарності та легких слів.

Все почалося з подорожей за кордон. Мені було важливо, щоб місцеві розуміли, що
я з України. Але я розмовляв російською, що спонукало їх ідентифікувати мене
росіянином, а мені запускати протокол роз’яснення. Так тривало, поки один
португалець не сказав, мовляв яка різниця: Україна, Росія… Мене це зачепило, а
він по-португальськи спокійно відповів: \enquote{Але ж ви розмовляєте однією мовою}.
Саме тоді мій раціональний мозок усвідомив: пояснювати, що ти українець,
російською мовою при наявності української - це якась зайва ітерація. Інакше
кажучи, нахіба ловити рибу пісюном, якщо ти тверезий і в тебе є вудочка? 

Розуміємо ми це чи ні, але говорити українською - це один з базових способів
жити українцем. На цю думку мене наштовхнули подорожі в Естонію та Литву і
найбільше - у Вірменію. Як людина з вірмено-українською кров’ю, я бачу багато
спільного в історії цих країн. Українців знищували голодомором, вірмени
пережили геноцид з боку турків. Обидві країни мають втрачені території, обидві
країни на Росії вважаються \enquote{чимось своїм, що вони не здатні ні любити, ні
відпустити}. Але там неможливо зустріти людину, яка не знає, що Арарат - це
Вірменія. Ба більше, всі вірмени говорять вірменською. Так само як литовці і
естонці говорять своїми мовами. І, о диво, там є багато людей \enquote{нашого} віку,
які взагалі не знають \enquote{языка, на котором говорил Пушкин, влюблялся Есенин и
рыгал в бассейн в Египте Володя из Рязани}.

Ці спостереження все частіше змушували мене думати про українську мову.
Фактично, я був вже готовий до переходу, але мені бракувало якогось поштовху.
Саме тоді мій мозок люб’язно перейшов на лаконічний пафос: \enquote{Ти знаєш російську,
бо ти цього не обирав. Говорити українською означає, що тепер ти можеш
обирати}. Бачите, ця штука в моїй голові може формулювати коротко. Але рідко.

1 вересня 2020 я обрав говорити українською. Це був пробний місяць, який триває
досі. Це не означає, що я відмовився від російської. Вона вже є в моєму житті і
нікуди не дінеться, як і іронія до її імперського контексту. Це означає, що
мені подобається звучати українською. І я хочу оволодіти нею на тому рівні, щоб
це звучання не викликало нарікань навіть у моєї внутрішньої прискіпливої
людинки з її одвічним: \enquote{Агов, юначе, що ти собі думаєш? Ти можеш краще.} Це
буде складно і довго. Але я готовий. Я думаю, що російська мова вже відкрила в
моєму персонажі всі доступні характеристики. Але я відчуваю, що в мені є ще не
одна карточка під знаком питання, які я зможу відкрити тільки українською. Чим
все закінчиться я не знаю. Та це й немає значення. В цьому питанні я самурай:
сам шлях і є метою. А ще я ніндзя: непомітно продерся у вашу стрічку і розмахую
україномовними нунчаками.

А тепер я спробую добити інтригу. Чому ж все-таки я говорю українською? Мабуть
тому, що мені хочеться вірити, що настане той день, коли в Україні нікому не
потрібно буде відповідати на питання: \enquote{А чому ти вирішив говорити українською?}
Досі не зрозуміло? Тоді так: вивчання мов у зріломі віці зменшує ризик деменції
та хвороби Альцгеймера. Я говорю українською, щоб якомога пізніше виникала
безумовна тяга мазати стіни лайном.

П.С. Цей пост почне цикл публікацій, в які увійдуть мої мовні рефлексії та
досвід. Якщо вас цікавлять якісь конкретні теми чи питання - ласкаво прошу в
коментарі. Або в дірект. Моя лагідна українізація розглядає усі канали
комунікації. Окрім вайбера. Все ж таки є кордони, які не можна порушувати. До
речі, останнє речення неможливо зрозуміти російською.

\#словомпромову
