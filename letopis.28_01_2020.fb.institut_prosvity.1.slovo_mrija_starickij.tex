% vim: keymap=russian-jcukenwin
%%beginhead 
 
%%file 28_01_2020.fb.institut_prosvity.1.slovo_mrija_starickij
%%parent 28_01_2020
 
%%url https://www.facebook.com/ProsvitaInstitute/posts/3003652596313713
 
%%author Інститут Просвіти
%%author_id institut_prosvity
%%author_url 
 
%%tags filologia,franko_ivan,kotljarevskij_ivan,kultura,lesja_ukrainka,mova,slovnyk,slovo,starickij_mihail.ukr.dramaturg,ukraina
%%title А ви знаєте, що слово "мрія" придумав Михайло Старицький?
 
%%endhead 
 
\subsection{А ви знаєте, що слово \enquote{мрія} придумав Михайло Старицький?}
\label{sec:28_01_2020.fb.institut_prosvity.1.slovo_mrija_starickij}
 
\Purl{https://www.facebook.com/ProsvitaInstitute/posts/3003652596313713}
\ifcmt
 author_begin
   author_id institut_prosvity
 author_end
\fi

А ви знаєте, що слово "мрія" придумав Михайло Старицький, а авторство лексеми
"мистецтво" належить матері Леся Українки – Олені Пчілці?

За походженням українська мова неоднорідна: є в ній чимало запозичень зі
слов’янських та інших мов. Але найбільший шар словникового запасу української
мови становить саме українська лексика. Тому слів багаття, бандура, баритися,
батьківщина, вареники, вибалок, держава, деруни, заздалегідь, козачок, лелека,
малеча, напувати, самітність, щодня не знайдете в жодній іншій мові 😉 Бо є
вони питомо українськими.

І хоч імена творців цих слів загубилися у вирі часу, все ж авторство деяких
лексем збереглося.

Наприклад, саме Іван Котляревський в \enquote{Енеїді} вперше вжив слова
\enquote{несамовитий} та \enquote{угамуватися}.

Михайло Старицький створив такі слова, як "мрія", "майбутнє", "байдужість",
"завзяття", "темрява", "пестливий", "привабливий". До речі, слово "мрія"
походить від "мріти", що означає "ледь виднітися, бовваніти".

Леся Українка є авторкою таких на той час неологізмів, як "провесна" та
"промінь", а з легкої руки Олени Пчілки в українські мові прижилися слова
"мистецтво", "переможець", "променистий", "палкий" і "нестяма".

Перу Івана Франка належать такі лексеми, як "отвір", "привид", "чинник".

Тарас Шевченко створив слово "високочолий",  "вогняний", "мордуватися",
"почимчикувати", "фортеця".

Класик українського красного письменства Іван Нечуй-Левицький є автором лексем
"самосвідомість" та "світогляд". На його рахунку також "стосунок", “перепона”,
“квітник”, “вигукування”, “сміливість”.

Григорій Квітка-Основ’яненко збагатив українську словесність такими поняттями,
як “буденний” та “перемагати”.

Леонід Глібов увів такі мовні одиниці, як “бурмотати”, “пожовкнути”, “соромно”.

1904 року «батько українського тіловиховання» професор Боберський випустив
працю \enquote{Забави й ігри рухові}, де з’являються терміни копаний м’яч (футбол),
кошиківка (баскетбол), стусан (бокс), наколесництво (велоспорт), відбиванка
(волейбол), пориванка (гандбол) та багато інших.

\verb|#Не_бійтесь_заглядати_у_словник|

\ii{28_01_2020.fb.institut_prosvity.1.slovo_mrija_starickij.cmt}
