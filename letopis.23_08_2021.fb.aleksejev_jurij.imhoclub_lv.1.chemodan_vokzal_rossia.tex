% vim: keymap=russian-jcukenwin
%%beginhead 
 
%%file 23_08_2021.fb.aleksejev_jurij.imhoclub_lv.1.chemodan_vokzal_rossia
%%parent 23_08_2021
 
%%url https://www.facebook.com/yury.alekseev.94/posts/4321783174556067
 
%%author_id aleksejev_jurij.imhoclub_lv
%%date 
 
%%tags chemodan_vokzal_rossia,rodina,rossia,russkie,sssr
%%title ЧЕМОДАН-ВОКЗАЛ-РОССИЯ! К юбилею распада СССР
 
%%endhead 
 
\subsection{ЧЕМОДАН-ВОКЗАЛ-РОССИЯ! К юбилею распада СССР}
\label{sec:23_08_2021.fb.aleksejev_jurij.imhoclub_lv.1.chemodan_vokzal_rossia}
 
\Purl{https://www.facebook.com/yury.alekseev.94/posts/4321783174556067}
\ifcmt
 author_begin
   author_id aleksejev_jurij.imhoclub_lv
 author_end
\fi

ЧЕМОДАН-ВОКЗАЛ-РОССИЯ! К юбилею распада СССР.

Президент современной Украины Зеленский на днях изрёк: "Жители Донбасса,
которые считают себя русскими, должны переехать в Россию, поскольку рано или
поздно регион вернется в состав Украины, а без нее там не будет цивилизации..."

Боже, как это знакомо... 30 лет назад, когда распался СССР, мы, прибалтийские
русские, это услышали сразу: эй, русские, вон из нашей "цивилизации",
чемодан-вокзал-Россия! И с тех пор слышим регулярно. Латыши, литовцы, эстонцы
все эти 30 лет строят свою "цивилизацию", постепенно вымирая. 

\ifcmt
  pic https://scontent-frx5-1.xx.fbcdn.net/v/t39.30808-6/239321223_4321781891222862_6222908181603449268_n.jpg?_nc_cat=110&_nc_rgb565=1&ccb=1-5&_nc_sid=730e14&_nc_ohc=rSB4d7H5ovkAX9YASMi&_nc_ht=scontent-frx5-1.xx&oh=7e3403c81a8488a2583aec4a656052a5&oe=613E1C6E
  @width 0.7
\fi

Президент Зеленский тоже включился в строительство своей "цивилизации".
Украинский язык он знает плохо, мучительно пытается его изобразить,
жестикулируя-интонируя, подбирая слова, кряхтит, э-кает, мэ-кает, лицом
шевелит... Еврейский русскоязычный мальчик из Кривого Рога, стендапит из себя
"щирого украинца". А дома в семье, как и 95\% всех этих "щирых", говорит
по-русски. Вытерев пот со лба от "украинского" напряжения.

Но я вообще хотел порассуждать не об этом. Начав писать статьи для "Взгляд.РУ"
и иных неглупых российских СМИ, я постоянно читаю комментарии в стиле: что ты
там делаешь в Латвии, Алексеев, зачем мучаешься, переезжай к нам в Россию. Тот
же самый императив Зеленского "чемодан-вокзал-Россия", но уже с другой, "своей"
стороны.

Нет, вопрос - резонный. Но ёрнический: ты там в Евросоюзе процветаешь,
получаешь "немецкие зарплаты" и "шведские пенсии", приезжай к нам! Помучаемся,
дескать, вместе.

Так вот, братцы, российские русские, "мучащееся" в России уже тысячу лет,
отвечу как на духу. В 1991-м году, ровно 30 лет назад Великая Россия
добровольно отказалась от "окраин-украин" - 14 своих республик. За границей
Империи остались кинутыми примерно 25 миллионов русских. В Прибалтике, в
Средней Азии, на Кавказе... 

Все они должны были немедленно отбыть в съёжившуюся Россию? Стесняюсь спросить:
а Россия готова была одновременно принять 25 миллионов соплеменников? С малыми
детьми, с пенсионерами мамами-папами, дать элементарное жильё, работу,
школы-садики-больницы-пенсии? Всем 25 миллионам? Точно?

Мой знакомый, полковник СА, вышедший в отставку в Прибалтийском военном округе,
получил от "независимой" Латвии в 1992-м предписание покинуть страну в течение
90 дней, военных пенсионеров латыши высылали в первую очередь (оккупанты же!).
А у отставного полковника была квартира в Риге (получил на склоне лет) и трое
несовершеннолетних детей. Он пытался эту квартиру продать хотя бы за пол
цены... 

А не вышло, латыши её у него просто отобрали, И выгнали на все четыре стороны с
детьми и чемоданами. (Кстати, его квартира была построена в начале 1980-х на
средства Министерства Обороны СССР, специально для отставных военных.) И
полковник эту квартиру заслужил честно: воевал, Анголу и Афганистан прошел,
парадный китель в боевых орденах, три ранения. 

И что? Родина-мать о нём как-то позаботилась? Защитила? Или хотя бы просто
достойно приняла? Полковник написал мне потом письмо из Вологодской области:
работает в автосервисе, гайки крутит за еду, жена, кандидат биологических наук,
торгует на вещевом рынке. Живут в бараке с печным отоплением, пенсии полковника
хватает на 7 "сникерсов". Больше писем не было... 

И как вы думаете, если со своими заслуженными Солдатами Родина-мать обходилась
таким образом, то что могли ожидать мы, простые смертные количеством в 25
миллионов бывших граждан СССР, русских, в один памятный день в августе 1991-го
вдруг оказавшихся "за границей"? 

А я скажу: нас в России встречали и до сих пор встречают на всех уровнях, от
больших чиновников до соседей по бараку, сакраментальным "мы вас сюда не
звали". Точно так же встречали израненных физически и морально воинов-афганцев
в конце 1980-х: "мы вас на войну не посылали". Помните?

Кстати про Афган. Сейчас россияне смеются, что после 20 лет "успешного
строительства демократии", американцы сваливают из Кабула на вертолётах, бросая
всё "нажитое непосильным трудом", какая в штаны от усердия и страха. Не
смейтесь над ними. Тридцать лет назад из Прибалтики Российская Федерация
(наследница СССР) выводила свои войска точно так же, бегством, панически...

Бросая всё, что не смогла унести в двух руках. Военную технику, имущество,
склады, базы-полигоны, военные училища, порты, военные аэродромы, военные
заводы, военные городки... Да что там, городки  -  целые города бросала! Если не
лениво  -  кликните: СССР в Латвии построил два военных города  -  Ирбене
\url{https://mikeseryakov.livejournal.com/88955.html} и Скрунда-локатор
\href{https://press.lv/post/nachat-demontazh-voennyh-stroenij-v-gorodke-skrundskogo-lokatora}{%
Начат демонтаж военных строений в городке Скрундского локатора, press.lv, 15.01.2016%
}
.... Со школами,
поликлиниками, детскими садами, магазинами-клубами. Бросили, даже не попросивши
оплаты на переезд.

Эксперты НАТО, прибывшие на брошенные Россией военные объекты Латвии уже через
месяц после ухода Армии РФ, несказанно удивились: так нам же здесь даже строить
ничего не надо! Всё готово! Только таблички на казармах-базах-полигонах
переклеить! Всё досталось ДАРОМ!

Причём, армии РФ в Прибалтике не дышали в спину талибы со "Стингерами" и
реактивной артиллерией, как в Афганистане. Армию бывшего СССР, победительницу
Гитлера, в 1991-м испугали танцы и песни селян  -  латышей-литовцев-эстонцев...
И вот отступила Россия отсюда, бросив на многие десятки миллиардов военного
имущества, ну и нас, 2 млн прибалтийских русских, заодно.

Никаких обид на матушку-Россию. Россия  -  великая континентальная Империя, она
с момента своего зарождения "пульсирует", расширяется-сжимается. История не
вчера началась, и не завтра закончится, Фукуяма ошибся. И у большинства из 25
миллионов русских, кинутых "за границей" бывшим СССР/Россией в 1991-м, обид на
Родину нет, свидетельствую. 

Как сказал один умный президент, "Распад СССР  -  "крупнейшая геополитическая
катастрофа века". А когда случается катастрофа, тогда главный принцип выживания
 -  "спасайся кто может и как может". Спасались все, как могли. И сама Россия от
себя самой в том числе. Бежали в Россию и русские и со своей территории  -  из
Чечни, например, неся на руках своих детей и стариков, побросав всё... Всякое
бывало...

А вот теперь  -  главное. Мне не понятно, почему лозунг "Чемодан-вокзал-Россия",
объединяет в едином порыве и прибалтийских нацистов, и "нэзалэжных" зеленских,
и российских патриотов. Это  -  что ж, товарищи? Каждый раз, когда Великая
Империя "съёживается", русские должны сажать на телеги своих детей-стариков и
уныло брести в пределы Московской области? А если ещё и Московскую область
займут какие-нибудь фашисты, что было совсем недавно, а если и саму Москву
займут какие-нибудь поляки-французы, что тоже случалось? Куда дальше бежать
нам, русским, с окраин Империи? За Урал? Каждый раз?

Вот прикиньте: если бы русские в Крыму вняли призыву "чемодан-вокзал", который
им 25 лет подряд вбивали в головы и Великие Зеленские, и российские диванные
патриоты. Собрали бы они манатки в начале 1990-х и переехали бы в Московскую
область. И стал бы тогда в 2014 году Крым  -  наш?

Ведь закончится мировая гегемония США, уже закончилась. Ведь закончится
процветание Четвёртого Рейха  -  Евросоюза, уже заканчивается. Начнёт Россия
снова собирать свои окраины, уже начала... 

А мы  -  уже тут.

Первоисточник: \url{https://imhoclub.lv/ru/author_page/91}
