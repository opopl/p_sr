%%beginhead 
 
%%file 10_06_2021.fb.denga_alena.kyiv.1.spodobavsya_men__mar
%%parent 10_06_2021
 
%%url https://www.facebook.com/alena.denga/posts/pfbid0kwedgzftvMzPA68aFUbifScjPVv3r2UbfkBE4wud3iR96YH3DtCveEydpwi8HCCHl
 
%%author_id denga_alena.kyiv
%%date 10_06_2021
 
%%tags mariupol,foto,mariupol.pre_war,2021,vpechatlenie
%%title Сподобався мені Маріуполь, от в саме сердечко
 
%%endhead 

\subsection{Сподобався мені Маріуполь, от в саме сердечко}
\label{sec:10_06_2021.fb.denga_alena.kyiv.1.spodobavsya_men__mar}

\Purl{https://www.facebook.com/alena.denga/posts/pfbid0kwedgzftvMzPA68aFUbifScjPVv3r2UbfkBE4wud3iR96YH3DtCveEydpwi8HCCHl}
\ifcmt
 author_begin
   author_id denga_alena.kyiv
 author_end
\fi

Сподобався мені Маріуполь, от в саме сердечко ❤ Коли згадують це місто, то
одразу говорять про екологію: так, там з цим біда. Але зараз хочу про ті хороші
моменти, які підмітила в місті, бо дечому може заздрити й Київ:

- Красивий парк в центрі міста з багаторічними рослинами на клумбах, нормальною
плиткою, а не з заводу Омельченко, сучасними лавочками й охайними газонами, на
яких відпочивають люди. По периметру кілька таких самих охайних закладів
харчування. Без мафів, торгівлі з лотків. Поряд є чистенька громадська
вбиральня, в яку не страшно зайти. 

- Поряд дитячий майданчик наче в якомусь Берліні: класний, інтерактивний, в
нормальних природніх кольорах. В Києві подібного не бачила у відкритому
доступі, все якесь вирвиоко.

- Поряд відкритий новий фонтан, що б'є з-під землі, тому тут влітку часто
купаються й діти, й дорослі. Навколо ряд лавочок. Ввечері на площі життя вирує
навколо фонтану: діти та підлітки на сунітах, велосипедах та самокатах, дорослі
на лавочках, навколо гуляють парочки. Таких фонтанів по місту кілька.

- Пусті вулиці з рівними дорогами. Так, автівок закономірно тут мало, але й
багато поліції – місто прифронтове. Тому всі намагаються їздити чемно, хоча за
відсутності пробок, певно, й пропадає цей істеричний стиль водіння.

- Новенькі тролейбуси, електронні табло з розкладом руху й нові зупинки. 

- Дуже класна набережна, сучасна, красива. На вході штук 5 однаково оформлених
точок фуд-кортів, зони відпочинку. Коли добираєшся сюди, починаєш вже
сумніватися, що ти в Маріуполі.

- Громадські пляжі заслуговують високої оцінки. Не за якість води, але за
інфраструктуру: лежаки, затінки, лавочки, пляжні душові та переодягальні – все
це охайне, доступне й безкоштовне. А головне – відсутність курортного
базару-вокзалу довкола.

- Низька забудова міста, переважно 1-2 поверхові старі будинки. Але оку є за що
зачепитися: чимало чепурненьких будівель тут збудував архітектор німецького
походження Нільсон. Але й багато закинутих, напіврозвалених часом та байдужість
будівель. Іронічно, що саме в такому плачевному стані садиба самого
архітектора.

- Достатньо закладів \enquote{все як ми любимо}, де можна поїсти і випити хорошої кави.
Третя хвиля дійшла до Маріуполя й надійно закріпилася в місті.

- В місті розвиваються різні громадські простори: Платформа ТЮ, Центр сучасного
мистецтва \enquote{Готель Континенталь}. 

- Вже третій рік поспіль тут проходить Гогольфест, що є ковтком свіжого повітря
для місцевих. Це круто, можна в місто й інші фестивалі везти.

Версії чому так гарно в Маріуполі розходяться: за однією – сюди вкладають
державні й кошти міжнародних донорів, щоб показати \enquote{Маріуполь вистояв, в
Україні краще, ніж в ОРДЛО}; за другою – це \enquote{компенсація} містянам за жахливу
екологію. Хоч би що там було, в Києві якість повітря вже втричі гірша за норму,
але подібних речей ніхто не робить. Тож варто на вихідні рванути в Маріуполь
або нарешті вирватися на наступний Гогольфест.

%\ii{10_06_2021.fb.denga_alena.kyiv.1.spodobavsya_men__mar.cmt}
