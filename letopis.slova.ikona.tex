% vim: keymap=russian-jcukenwin
%%beginhead 
 
%%file slova.ikona
%%parent slova
 
%%url 
 
%%author_id 
%%date 
 
%%tags 
%%title 
 
%%endhead 
\chapter{Икона}

%%%cit
%%%cit_head
%%%cit_pic
\ifcmt
  tab_begin cols=3
		 pic https://cdn1.ozone.ru/multimedia/1015649029.jpg
     pic https://rusbuk.ru/uploads/books/916285/8315773ebd602017207a775a8b77a0166acb52e2Max.jpg
		 pic https://i74.fastpic.org/big/2016/0307/7e/9fbff70e98194a1b3cf47b7dc8d9f17e.jpg
  tab_end
\fi
%%%cit_text
\emph{«Ікони»} наукпопу змінюються залежно від нових викликів, однак спільним
лишається прагнення знайти нового світського месію, який навчить жити та
сформулює те, що ми давно усвідомлювали, однак не могли дібрати потрібних слів.
Так, у Радянському Союзі, де були чудові серійні видання нонфікшну («Еврика»,
«У світі науки і техніки», бібліотечка «Квант» тощо), до безтями зачитувалися
Дейлом Карнеґі з його лицемірними «Як здобувати друзів і впливати на людей»
(«How To Win Friends and Influence People», 1936). Ще пам’ятаю, як під час
перебудови модними були «Закони Паркінсона» та різні брошурки про дієти, а
особливо цінною вважали книженцію Пола Бреґґа про голодування
%%%cit_comment
%%%cit_title
\citTitle{Не сподівайтеся позбутися літератури}, Максим Нестелєєв, tyzhden.ua, 08.12.2021%
%%%cit_url
\href{https://tyzhden.ua/Columns/50/253787}{link}
%%%endcit
