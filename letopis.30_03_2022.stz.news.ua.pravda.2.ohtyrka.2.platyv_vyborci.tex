% vim: keymap=russian-jcukenwin
%%beginhead 
 
%%file 30_03_2022.stz.news.ua.pravda.2.ohtyrka.2.platyv_vyborci
%%parent 30_03_2022.stz.news.ua.pravda.2.ohtyrka
 
%%url 
 
%%author_id 
%%date 
 
%%tags 
%%title 
 
%%endhead 

\subsubsection{Я 20 років \enquote{платив} своїм виборцям}
\label{sec:30_03_2022.stz.news.ua.pravda.2.ohtyrka.2.platyv_vyborci}

Охтирка – це маленьке затишне місто на березі Ворскли, зі спокійними заможними
людьми. 

Колись ми були нафтовою столицею, містом нафтовиків – у нас в районі є багато
свердловин. І до сьогодні, до війни, у нас добувалась і нафта, і газ у великій
кількості. 

Сам я живу тут з 2000-го року, приїхав, як закінчив медичний університет у
Харкові. 

У 2020-му виграв вибори з бюджетом у 50 тисяч гривень. Бо я, як кажу, "платив"
своїм виборцям усі ці 20 років – я 20 років вночі працював хірургом, а вдень
ортопедом-травматологом.

\ii{30_03_2022.stz.news.ua.pravda.2.ohtyrka.2.platyv_vyborci.pic.1}

Коли мені казали, що в невеликих містечках Сумської області вибори вигравали з
бюджетом від 2 до 10 мільйонів, то я просто дивувався: а як ви, люди добрі,
хочете їх відробляти? Красти?

У мене в лікарні було півтори ставки, це 8-10 тисяч. Тож, думав, на зарплату
міського голови в 24 тисячі я якось проживу. Але потім лікарям докинули грошей
і деякі, зокрема сімейні, почали отримувати значно більше, ніж міський голова.
