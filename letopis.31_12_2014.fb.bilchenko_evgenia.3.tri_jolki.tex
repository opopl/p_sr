% vim: keymap=russian-jcukenwin
%%beginhead 
 
%%file 31_12_2014.fb.bilchenko_evgenia.3.tri_jolki
%%parent 31_12_2014
 
%%url https://www.facebook.com/yevzhik/posts/769256006442845
 
%%author Бильченко, Евгения
%%author_id bilchenko_evgenia
%%author_url 
 
%%tags bilchenko_evgenia,jalynka,novyj_god,pozdravlenie,prazdnik
%%title БЖ. ТРИ ЙОЛКИ НА ОДНУ БЖ
 
%%endhead 
 
\subsection{БЖ. ТРИ ЙОЛКИ НА ОДНУ БЖ}
\label{sec:31_12_2014.fb.bilchenko_evgenia.3.tri_jolki}
\Purl{https://www.facebook.com/yevzhik/posts/769256006442845}
\ifcmt
 author_begin
   author_id bilchenko_evgenia
 author_end
\fi

БЖ. ТРИ ЙОЛКИ НА ОДНУ БЖ, ИЛИ КАК ПОЭТ ФОРМИРОВАЛ НОВОГОДНИЕ АТРИБУТЫ СВОЕГО
МАРГИНАЛЬНОГО ЖИЛИЩА. ПОПЫТКА ПОЗДРАВЛЕНИЯ ЧИТАТЕЛЕЙ С НАСТУПАЮЩИМ НОВЫМ
РАЗВЕСЕЛЫМ ГОДОМ.

Ребята, я вас всех сердечно поздравляю с наступающим Новым Годом уже сегодня,
потому что завтра уже с утра начинается трудный волонтерский день, с хлопушками
не связанный, о чем будет рассказано отдельно и будет мало времени для звонков
и, тем более, не признаваемых моим поэтическим Логосом и гастритным желудком
\enquote{советских} салатов оливье вперемежку с телеизображениями, для которых
я не могу подобрать определения нематерного происхождения. 

\ifcmt
  tab_begin cols=3
	width 0.3

     pic https://scontent-lga3-1.xx.fbcdn.net/v/t1.18169-9/10906129_769254733109639_8574115461321044975_n.jpg?_nc_cat=108&ccb=1-3&_nc_sid=cdbe9c&_nc_ohc=vpss1nKjhbIAX-nffBA&tn=lowUrFCbCbt-jOWu&_nc_ht=scontent-lga3-1.xx&oh=f82617e4301c8fd8ce9d9a0127682db5&oe=60E7BD5C
	width 0.27

     pic https://scontent-lga3-1.xx.fbcdn.net/v/t1.18169-9/10881836_769254743109638_237069685545722845_n.jpg?_nc_cat=100&ccb=1-3&_nc_sid=cdbe9c&_nc_ohc=p7nTEwo66hgAX9cdlfK&_nc_ht=scontent-lga3-1.xx&oh=e67653d2cfcc90923d0beb0ada4a71c3&oe=60E8A2D6
	width 0.23

		 pic https://scontent-lga3-1.xx.fbcdn.net/v/t1.18169-9/10888977_769254746442971_7679443811878471255_n.jpg?_nc_cat=109&ccb=1-3&_nc_sid=cdbe9c&_nc_ohc=t4S_NZ2qTTgAX_Rd3lM&tn=lowUrFCbCbt-jOWu&_nc_ht=scontent-lga3-1.xx&oh=0cec6c60c5b69b1c7cbb52b585b0b431&oe=60E7D2AF
	width 0.35

  tab_end
\fi

Убедительно и сердечно прошу вас ввиду моего непроходимо тотального одиночества
писать мне новогодние поздравления в \enquote{личку} или в комментариях под постами -
таким образом я смогу освободить стенку для нужной социальной (пока
волонтерской) информации и пообщаться с каждым не \enquote{лайком}, который иногда
обозначает просто пустой знак, а нормальным человеческим (или звериным - в
зависимости от степени опьянения или приема таблеток) языком. 

\ifcmt
  tab_begin cols=3

     pic https://scontent-lga3-1.xx.fbcdn.net/v/t1.18169-9/10428419_769254786442967_4160164478714508413_n.jpg?_nc_cat=109&ccb=1-3&_nc_sid=cdbe9c&_nc_ohc=fYLzWzmkXFUAX-HfEw4&_nc_ht=scontent-lga3-1.xx&oh=93438e2cff537fe3e54e6687ab8d5fb3&oe=60E745CB

     pic https://scontent-lga3-1.xx.fbcdn.net/v/t1.18169-9/10881864_769254793109633_9152546180333321312_n.jpg?_nc_cat=107&ccb=1-3&_nc_sid=cdbe9c&_nc_ohc=JgG4SxVz2s4AX9vvLtZ&_nc_ht=scontent-lga3-1.xx&oh=d6227406c94deab04c64b2e0c6e2fa83&oe=60E83727

		 pic https://scontent-lga3-1.xx.fbcdn.net/v/t1.18169-9/10898271_769254803109632_6681109618886750229_n.jpg?_nc_cat=100&ccb=1-3&_nc_sid=cdbe9c&_nc_ohc=AUxqW_pxO3AAX_8N8Dk&_nc_ht=scontent-lga3-1.xx&oh=3ac0c7c12e65ad75a0bfbed53a8c42a3&oe=60E856F4

  tab_end
\fi

То есть ответить каждому. Если это не будет происходить сразу, то это не потому
что я \enquote{плохая}, а мне просто завтра (31 декабря) надо встретить львовян,
достать 5 наборов теплой формы, получить деньги из США и Панамы на АТО,
желательно все-таки одеть друга-правосека Леху и выудить из пучины алкоголия
друга-правосека Миху. 

\ifcmt
tab_begin cols=3
  pic https://scontent-lga3-1.xx.fbcdn.net/v/t1.18169-9/10906213_769254839776295_1548806656309597248_n.jpg?_nc_cat=109&ccb=1-3&_nc_sid=cdbe9c&_nc_ohc=__5NenC7i7IAX-_5h60&_nc_ht=scontent-lga3-1.xx&oh=28f2b170ab74ee70f61628aa55caaf97&oe=60E76D63
  width 0.4

	pic https://scontent-lga3-1.xx.fbcdn.net/v/t1.18169-9/10891654_769254846442961_3129909527296273071_n.jpg?_nc_cat=109&ccb=1-3&_nc_sid=cdbe9c&_nc_ohc=XlzFdK8khLMAX9ni3hN&_nc_ht=scontent-lga3-1.xx&oh=e9f588eca1e6bf00f32199f887662d17&oe=60E89985

	pic https://scontent-lga3-1.xx.fbcdn.net/v/t1.18169-9/1907362_769254853109627_8007968836821661380_n.jpg?_nc_cat=110&ccb=1-3&_nc_sid=cdbe9c&_nc_ohc=gLjTe3yyyoMAX8VBSwL&_nc_ht=scontent-lga3-1.xx&oh=b22f7e56a16bf99a8b31330005a8c942&oe=60E8744B

tab_end
\fi

Я уже молчу, что выуживать из сей благородной античной пучины придется еще и
родного брата, потом долго спорить с единоутробной сестрой о прелестях Родины,
а затем - выуживать из алкоголия и транквилизаторов саму себя, искать приличную
не слишком хипповую длинную юбку с вышиванкой и ехать на ночь читать стихи
священникам УПЦ МП... о национальных ценностях. Если эти парадоксы 31 декабря
состоятся - Рязанов будет отдыхать.


%\ifcmt
  %tab_begin cols=3

     %pic <++>

     %pic <++>

     %pic <++>

  %tab_end
%\fi


Посему, побратимы мои, близкие и далекие, ваш \enquote{укропчик} решил сегодня немного
повеселиться, отдав дань радости своему несчастью, и развлечь Вас тем, что он
впал в окончательную шизуху и украсил у себя в комнате сразу 3 елки, условно
назвав их: \enquote{Советская} (\enquote{детсадовская}), \enquote{европейская} (рациональная) и
\enquote{поэтическая} (патриотическая, или родная). Хочу познакомить Вас со всеми
тремя, а также с дорогими для меня предметами, которые мне в праздники порой с
успехом заменяют людей. 

ЙОЛКА ПЕРВАЯ, СОВЕТСКАЯ - большая, спереди расфуфыренная, зад, как и положено,
подобен плащу господина Портоса без перевязи, много игрушек, много дождика,
много исколотых пальцев, кривой ствол, падает на батарею, пока украсила,
выкурила три сигареты, \enquote{Ирония судьбы, или С Легким паром!}, \enquote{А
вот в наше время....}, \enquote{\enquote{Русские ракеты} - самые большие ракеты
в мире}, несколько пластмассовых раритетов, дальше предсказуемо, но - зато
много ностальгии о поре, когда ты был настолько наивен, что тебе жилось хорошо
и светло, хотя пионерский галстук я не носила - пугал он меня даже тогда.
Атрибут елки: конечно, Дед Мороз. 

ЙОЛКА ВТОРАЯ, ЕВРОПЕЙСКАЯ - экологически грамотно составлена не из спиленного
дерева, а из отдельной веточек. Все шарики - миниатюрные, импортные, одинаковой
формы. Увенчана золотой короной, символизирующей западный центризм и гегемонию
проклятых \enquote{америкосов}, ее атрибуты: телефон и телефонная книжка, напоминающая
о звонках диссертанта утром 1 января, и хилая шишечка, символизирующая
состояние психики БЖ после звонка и прочтения его диссертации.

ЙОЛКА ТРЕТЬЯ, ЛЮБИМАЯ, ПАТРИОТИЧЕСКАЯ символизирует мою поэзию и бандеровскую
сущность соответственно. У основания ёлки сидит суровая Снегурка,
экстремистская вожачка ОУН-УПА в сопровождении своих подельников - пьяного
плюшевого Мишутки и лютого Конька-Горбунка, готовых выползать из берлоги и
скакать по малороссийским степям с целью осуществления геноцида и произвола.

Рядом с ними - памятник исконной земле ОУН-УПА - волынский горшочек, сделанный
руками студентов Острожской Академии, куда сепаратисты прячутся от злобных
бандеровцев, как Буратино - от Карабаса. Прямо над ними - их гуцульские тотемы:
деревянные фигурки яремчанского брелока - Великий Отец и Великая Мать, в жертву
которым \enquote{бандеровцы} каждый Новый Год приносят русскоязычных жителей Горловки. 

Объединительным сакральным началом сего нуминозного тендема Анимы и Анимуса
украинских ацтеков является главный тотем БЖ - Белка, охраняющая ёлку. Над
головой этого агрессивного животного завязана белая лента героев Болотной и
символ Марша Мира в Москве. На ветках ёлки висят несколько дорогих сердцу
предметов: четки с крестиком, маленький поросенок, подаренный покойной любимой
бабушкой, сердечко на нитке от читателя и вещь, которую узнают только мои
друзья из Тамбова - Олька и Петька - наш бисерный браслет, купленный на
московском вокзале. Таких у людей только три сейчас (тиражи мы не считаем,
элита ибо вредная). 

Как видим, бандерлоговская ёлка наполнена исключительно украинскими атрибутами,
не имеющими к России никакого отношения. И, наконец, верхушка ёлки - истертая
майдановская ленточка с моего запястья, видавшая снайперов, которую охраняют
Поэт (беленькая девочка ,похожая на Ангела), кривоватый Санта (странного вида
мальчик, который меньше и слабее своей Герды: мой классический вариант
отношений с мужчинами) и Колокол (главный герой моих текстов). \enquote{Звездочкой} на
этой елочке является сине-желтая синичка, подаренная в Питере писателем Ольгой
Кушлиной. Интересен и фон елки: мое пианино, мое же детское фото, моя же
пепельница, фрагмент картины с рекой Прут кисти Ивана Попова и гуцульская
салфетка, на которой это всё стоит - поле наших бандеровских происков.

А еще я решила запечатлеть вещи, с которыми я после транквилизаторов
разговариваю: первый подарок Юры Крыжановского мне - полотно известного
украинского живописца-примитивиста Светланы Мурованной \enquote{Контакт} (очень
дорогое, есть в международных каталогах, висит у меня над столом), отдельно -
сфотографировала глаза рыжей девушки - героини этой картины, художественное
фото Михаила Булгакова (автор - Виктор Нагорный), который (Булгаков) дружит с
верховинскими писанками и хипповыми совушками с лубочной подольской картинки,
мой портрет кисти покойной талантливейшей художницы Наташи Антоновой (друга
поэта-хулигана Тараса Липольца) в стиле хиппи, Божья коровка на ладошке
(подарок читателя из Москвы), парик клоуна, в котором я ходила на Майдане в
прошлый Новый Год, веселя Беркут своим праворадикальным оптимизмом, задумчивое
лицо йогина с художественной фотографии (подарок из ныне отказавшегося от меня
Ильичевска за нежелание изменять Украине и прославлять Малороссию на языке
\enquote{розы-морозы}), изображение \enquote{Инь и Ян} со студенческой фотокартины, медаль мужа
жене за ее дурное поведение и ее националистическое надругательство над
подарком мужа в виде обрамления сей медальки, купленной (о ужас!) в Крыму,
фольклорной украинской конякой, коей стало мое гран-при конкурса \enquote{Подковы
Пегаса}, и лемковскими бусами. 

Из любимого: маленькое лицо хасида из Израиля в виде крохотного сувенира, белка
номер два (символ одиночества - сидит на компьютерном динамике и думает, что
мир, собственно, - говно, а все мужики - сво...), священное для меня боевое
знамя с баррикады на метро \enquote{Театральная} (сзади - пыль и пятна), окруженное
любимыми игрушками, и старинные иконы в красном углу. 

Разве человеку нужно больше друзей? 

Пусть вас хранит Бог, мои родные. Добра вам - и ничего не бояться: все равно,
как пела Умка, \enquote{Не волнуйся, мама, ничего не будет хорошо} ))) Ваш БЖ. 

\ii{31_12_2014.fb.bilchenko_evgenia.3.tri_jolki.cmt}
