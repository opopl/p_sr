% vim: keymap=russian-jcukenwin
%%beginhead 
 
%%file 28_09_2021.fb.fb_group.story_kiev_ua.1.gogol_i_kiev.cmt
%%parent 28_09_2021.fb.fb_group.story_kiev_ua.1.gogol_i_kiev
 
%%url 
 
%%author_id 
%%date 
 
%%tags 
%%title 
 
%%endhead 
\subsubsection{Коментарі}

\begin{itemize} % {
\iusr{Victor Gregory}
С географией у Коли было так себе, раз по пути из Полтавы в Нежин проскочил Киев.

\iusr{Milana Piskalenko}
Браво!

\iusr{Tatiana Berlin}
Во торкнуло.....

\iusr{Ольга Кирьянцева}

Просто замечательно! Тонкая ирония Вам очень удалась.
@igg{fbicon.hands.applause.yellow}  @igg{fbicon.face.happy.two.hands} 

\iusr{Vladimir M. Pushkarev}
Маленькая неточность: Паниковский там не жил, а работал.

\begin{itemize} % {
\iusr{Аркадій Друкер}
\textbf{Vladimir M. Pushkarev} а рядом жил. Потомушо у него нога была больная. Куда же ему с больной ногой? Поезжайте в Киев и спросите.

\iusr{Vladimir M. Pushkarev}
\textbf{Аркадій Друкер} Рядом жить он не мог - не по Сеньке шапка!

\iusr{Lara Ilich}
\textbf{Vladimir M. Pushkarev} и до сих пор работает...)

\iusr{Yan Nudelman}
\textbf{Vladimir M. Pushkarev} Конечно жил. Мы были соседями по коммунальной квартире.

\iusr{Аркадій Друкер}
\textbf{Yan Nudelman} Теперь я догадываюсь, кто у него трость оторвал. За что вы его так не любили?!

\iusr{Vladimir M. Pushkarev}
\textbf{Yan Nudelman} Тогда коммунальных квартир еще не было - большевики еще не угнездились.
\end{itemize} % }

\iusr{Victor Gregory}
Кстати, на углу Крещатика и Кадетской ещё перепичку не продавали, иначе он бы ни в какие Италии не уехал.

\iusr{Наталья Дрягун}
Милый рассказ  @igg{fbicon.smile} 

\iusr{Ирина Иванченко}

Вот до чего прекрасна страна/ народ, когда про её / его лучшие красоты /
чесноты самым талантливым мозгам которой/ ого гораздо продуктивней/ эффективней
описывать их ( красоты)из - за границы!

\begin{itemize} % {
\iusr{Vadim Basovskiy}
\textbf{Ирина Иванченко}

\enquote{Почему иностранец менее стремится жить у нас, чем мы в его земле? Потому, что
он и без того уже находится за границей}. Козьма Прутков.


\iusr{Michael Viz}
\textbf{Vadim Basovskiy} это про Россию. Не перепутали?

\iusr{Dubizhansky Ludmila}
\textbf{Ирина Иванченко} Дык большое видится на расстоянии! Это еще классик заметил

\iusr{Галина Петрова}
вот чего?...- ни чего!)

\end{itemize} % }

\iusr{Тамара Ар}

Как всегда, типа в духе \enquote{сатиры}, прошлись по Николаю Васильевичу Гоголю!
Автор, вам до него все равно далеко! Не утруждайтесь

\begin{itemize} % {
\iusr{Ірина Кравець}
\textbf{Тамара Ар} А Ваше мнение истина в последней инстанции?

\iusr{Аркадій Друкер}
\textbf{Toma Ar} но вот вы же утруждаетесь. Не пишите. И будет вам щасте.

\iusr{Tatiana Thoene}
\textbf{Toma Ar} и Вы не утруждайтесь читать. Пропустите и промолчите.

\iusr{Kate Levenko}
\textbf{Toma Ar} Підкажіть, а ви буваєте чимось задоволені? Ну там, зранку встали і все добре, сонечко світить, пташки співають? Що не коментар - негатив, вже почала вас впізнавати за аватаркою  @igg{fbicon.beaming.face.smiling.eyes}  А автор молодець, тонко підмітив такий от типаж  @igg{fbicon.wink} 

\iusr{Наталия Веремеенко}
\textbf{Тамара Ар}
Нате і мою баночку на
свячену воду!.. @igg{fbicon.face.eyebrow.raised} 

\iusr{Наталья Эгатова}

\ifcmt
  ig https://scontent-mxp1-1.xx.fbcdn.net/v/t39.1997-6/p480x480/105941685_953860581742966_1572841152382279834_n.png?_nc_cat=1&ccb=1-5&_nc_sid=0572db&_nc_ohc=_iE3XtYBdGYAX_yWoLc&_nc_ht=scontent-mxp1-1.xx&oh=c51abc2b22b58bf5986c988f967a31d1&oe=61B7FD4B
  @width 0.2
\fi

\iusr{Наталья Твердохлеб}
\textbf{Тамара Ар} Мне кажется, что автор и не думал соревноваться с Николаем Васильевичем. @igg{fbicon.smile}  Он просто хотел сделать наше утро добрым. Я, например, благодарна @igg{fbicon.smile} 

\iusr{Света Арсеньева}
\textbf{Тамара Ар} Как по мне Золотушкин покруче Гоголя.  @igg{fbicon.wink} 

\iusr{Alik Perlov}
\textbf{Тамара Ар} С утра негатив у вас))). Съешьте конфетку Шоколадную, попустит ))

\iusr{Yan Nudelman}
\textbf{Тамара Ар} Дык это написано не для вас.

\end{itemize} % }

\iusr{Татьяна Вакуленко}
Вот хорошо же!
Сделали мое пасмурное утро!

\iusr{Надежда Кальная}
Мне понравилось! Короткий, но всеобъемлющий экскурс в жизнь великого! @igg{fbicon.hands.applause.yellow} 

\iusr{Ірина Кравець}
Какая с утра недобрая! Какая-то желчная и вдругих коментах/как-то уж заметно/не
хотелось писать, но уж очень явно

\begin{itemize} % {
\iusr{Анатолий Золотушкин}
\textbf{Ірина Кравець} О них и написано
\end{itemize} % }

\iusr{Валерий Якимчук}
\enquote{У нас тут своих с носами некуда девать}.(( Увы, это исторически не
достоверно... @igg{fbicon.wink} 

\iusr{Энна Кот}
Треба цінувати своїх геніїв!

\begin{itemize} % {
\iusr{Аркадій Друкер}
\textbf{Enna Kot} але це не означає, що не можна жартувати, використовуючи їх імена. Гадаю, вони не образилися би, маючи гарне почуття гумору.
\end{itemize} % }

\iusr{Татьяна Тихонова}
Точно, ёмко, в меру саркастично, спасибо!

\iusr{Виктор Задворнов}

Ой, зря вы так о Нежине. Как давно вы там были? Да, это не столица. Здесь
уютнее, спокойнее... И главное для меня, в местном драмтеатре иногда и Гоголя
дают.

\begin{itemize} % {
\iusr{Анатолий Золотушкин}
\textbf{Виктор Задворнов} прошу прощения за Нежин. Просто к слову пришлось

\begin{itemize} % {
\iusr{Виктор Задворнов}
\textbf{Анатолий Золотушкин} 

Нежин - прекрасный городок. Во все времена и сезоны. Навестите, убедитесь.
Удручает лишь одно место. Желдорвокзал. Увы: бомжи \enquote{слетаются} сюда по вечерам.
Неспокойно даже в зале для отдыха (за деньгу). А так - масса локаций для
неутомимых туристов.

\iusr{Наталья Твердохлеб}
\textbf{Виктор Задворнов} По этой части наш киевский вокзал все равно победить невозможно.  @igg{fbicon.smile} Думаю, по сравнению с ним, нежинский — Букингемский дворец.  @igg{fbicon.smile} 

\iusr{Виктор Задворнов}
\textbf{Наталья Твердохлеб} 

если бы полиция не выгоняла бомжей с киевского вокзала, их бы вообще не было на
нежинском, где полиция отсутствует. СообЧающиеся сосуды - это о вокзалах.

\iusr{Наталья Твердохлеб}
\textbf{Виктор Задворнов} 

С «черного» входа в вокзал (со стороны улочки Пестеля), где все эти «обитатели»
открыто торгуют наркотиками, пьют и спят в вечной грязи и мусоре, никогда нет
никакой полиции. Стражи порядка «служат народу» с другой стороны, с
центральной. Так что, с нашими все хорошо. @igg{fbicon.smile} 

\iusr{Alik Perlov}
\textbf{Виктор Задворнов} А река в Нежине есть??

\iusr{Виктор Задворнов}
\textbf{Alik Perlov} а как же! Хотите вплавь добраться!?))

\iusr{Alik Perlov}
\textbf{Виктор Задворнов} Переселиться )) реку люблю, рыбалку @igg{fbicon.wink} 

\iusr{Kate Levenko}
\textbf{Наталья Твердохлеб} 

Не бували ви на Ніжинському вокзалі  @igg{fbicon.smile}  У нас, звісно, то все
в масштабі, але на Ніжинському настільки концентровано... Ууух. Останній раз
була там років 10 тому, і, чесно кажучи, не скучила
@igg{fbicon.beaming.face.smiling.eyes}  Але якщо їхати в Ніжин на поїзді, чи
\enquote{підвищеній} електричці, можна на вокзал не заходити - бігом по перрону
і в місто  @igg{fbicon.wink} 

\iusr{Наталья Твердохлеб} 
\textbf{Kate Levenko} 

Врахую @igg{fbicon.smile}
Дякую)

\iusr{Kate Levenko}
\textbf{Alik Perlov} 

З Ніжина всі їздять на Десну, багато містян мають дачі в селах в тому напрямі,
ціни на будиночки дуже низькі. Там рибалка супер, навіть я, не фанат, люблю
туди їздити рибалити. Бо ж ловиться. Лящі, щуки, а карасі, красноперка там - то
як сорна риба, для котів  @igg{fbicon.smile}  Кладьківка-Авдіївка-Салтикова Дівиця - пляжі кращі
за наше Чорноморське узбережжя, неймовірно красива природа, все дешево. Ну і
навколо Ніжина ліси, де реально повно грибів, ось цього року чоловікові батьки
знайшли білий на 600 грамм. Річка Остер в місті чисто для краси... В Графському
парку озеро, рибаки сидять, але ну таке...


\iusr{Alik Perlov}
\textbf{Kate Levenko} щиро дякую за росповидь ))) Треба помиркувати над цим)))

\iusr{Alik Perlov}
\textbf{Kate Levenko} Скильки километров вид Киева???

\iusr{Kate Levenko}
\textbf{Alik Perlov} 

Ніжин десь 150, села так само, від Ніжина до Десни десь 45-50 км. Схожі місця є
у нас вздовж Десни, Вища Дубечня і далі, біля Остра - але людей, звісно,
більше, ніж там. Ну і ціни там на нерухомість вражають, батьки чоловіка кілька
років тому купили будинок і 40 сотих за 2500\$, перед будинком тільки поле і
далі Десна. Влітку живуть більше там, зробили ремонт, террасу і насолоджуються
звідти заходом сонця  @igg{fbicon.smile} 


\iusr{Alik Perlov}
\textbf{Kate Levenko} Дякую за инфу)))

\iusr{Kate Levenko}
\textbf{Alik Perlov} 

Там є кілька баз відпочинку - можна просто поїхати подивитись на місця, \enquote{Дім
хаскі}, \enquote{Убідь} (цивілізація таки наступає). В домі хаскі прогулянки з хаскі по
луках, катання на конях, ми колись їздили на екскурсію - особливо дитина
вражена була.

\end{itemize} % }

\iusr{Ирина Иванченко}
\textbf{Виктор Задворнов} , недавно Оксана Дубинина сделала обзор экскурсии по нежинским местам, очень, очень мило, не отказалась бы её,,рейд" повторить.

\iusr{Tatiana Thoene}
\textbf{Виктор Задворнов} ведь написали о Нежине во времена Гоголя; а Вас ведь тогда там не было ))

\begin{itemize} % {
\iusr{Виктор Задворнов}
\textbf{Tatiana Thoene} вчитайтесь в \textbf{Анатолий Золотушкин}!!!

\iusr{Tatiana Thoene}
\textbf{Виктор Задворнов} вчитываюсь и с удовольствием.
\end{itemize} % }

\iusr{Анна Загорулько}
\textbf{Виктор Задворнов} Согласна! Нежин обижать не надо! Отличный город!

\iusr{Kate Levenko}
\textbf{Виктор Задворнов} 

Так, приємне містечко, мій чоловік звідти, дуже жаль, що не дуже популярне у
туристів і туристичної інфраструктури майже немає. Тут посилання не можна, але
за пошуком в ютьюбі можна знайти його відео: \enquote{Тисячолітнє місто Ніжин}. Іще
нещодавно траплялась справжня екскурсія по місту, можна в ютьюбі знайти за
запитом: \enquote{Город Нежин. Ритм жизни}.


\iusr{Lara Ilich}
\textbf{Виктор Задворнов} у меня сын знакомых в Нежине учится. Как ему там, спрашиваю - \enquote{сильно хлопци все бьются } - ответил.


\iusr{Виктор Задворнов}
\textbf{Michael Viz} ЛИцей

\iusr{Michael Viz}
\textbf{Виктор Задворнов} Царскосельский Лицей

\iusr{Michael Viz}
\textbf{Виктор Задворнов} т9 и не такое может выдать, дяк

\iusr{Виктор Задворнов}
\textbf{Michael Viz} Т9, коньячно

\iusr{Michael Viz}
Шупер!!

\iusr{Виктор Задворнов}
\textbf{Michael Viz} это вы куда?

\iusr{Аркадій Друкер}
\textbf{Виктор Задворнов} то же самое хотел отметить )))

\end{itemize} % }

\iusr{Наталия Бакалейко}
 @igg{fbicon.laugh.rolling.floor}  @igg{fbicon.hands.applause.yellow}{repeat=3} 

\iusr{Tatiana Thoene}
Золотушкину - браво! Легко и с юмором. А у кого чувства юмора нет или оно их покинуло - искреннее сочувствие.

\iusr{Анатолий Золотушкин}
\textbf{Tatiana Thoene} спасибо @igg{fbicon.face.grinning.big.eyes} 

\iusr{Радикс Рад}
Как можно уехать из Киева, погуляв по Владимирской горке!!! Как?!!

\begin{itemize} % {
\iusr{Анна Загорулько}
\textbf{Радикс Рад} Молча - возвращаться будет приятно)

\iusr{Радикс Рад}
\textbf{Анна Загорулько} возвращаться можно на Монмартр, а на Владимирской нужно периодически гулять, для восполнения сил и энергии)

\iusr{Анна Загорулько}
\textbf{Радикс Рад} Я вижу, Вы поклонник именно Владимирской горки! Я тоже люблю ее, но не только)

\iusr{Радикс Рад}
\textbf{Анна Загорулько} не только. Есть масса любимых мест. Вчерась гулял в Муромце - релакс  @igg{fbicon.smile} 

\iusr{Alik Perlov}
\textbf{Радикс Рад} И как было?? Как кухня? Или это не ресторан ? @igg{fbicon.wink} 

\iusr{Радикс Рад}
\textbf{Alik Perlov} я про парк. Есть на острове и рестораны, но пока ещё мною не проверялись)
\end{itemize} % }

\iusr{Нина Волченкова}

\ifcmt
  ig https://i2.paste.pics/6fb9dacbae68a3773bc4959d0c8f011a.png
  @width 0.3
\fi

\iusr{Elen Zhorniak}
Браво

\ifcmt
  ig https://i2.paste.pics/12e5d3fc740dd48c05d934a581eac88b.png
  @width 0.4
\fi

\iusr{Elen Zhorniak}
Родилась под Полтавой, Киев не проскочила, прямо и давно...
@igg{fbicon.face.tears.of.joy}{repeat=3} А вот Рим, можно подумать уже.... Ибо
@igg{fbicon.flame}{repeat=4} 

\iusr{Елена Чичик}

\ifcmt
  ig https://scontent-mxp1-1.xx.fbcdn.net/v/t39.1997-6/s168x128/17640308_1652591141433953_2515677274297073664_n.png?_nc_cat=1&ccb=1-5&_nc_sid=ac3552&_nc_ohc=yClL-RbVaBQAX_Kfil-&_nc_ht=scontent-mxp1-1.xx&oh=78de06a53fd416e63ac5f72a2358a9d3&oe=61B78E2E
  @width 0.1
\fi

\iusr{Наталия Кирьянова}

\ifcmt
  ig https://i2.paste.pics/a9825c78009aabbb1b5f1f8865c3a1d4.png
  @width 0.3
\fi

\iusr{Сергей Кравцов}
Золотушкин говорите?

\iusr{Анатолий Золотушкин}
\textbf{Сергей Кравцов} ?

\iusr{Alik Perlov}
\textbf{Анатолий Золотушкин} Павлины говоришь.... Хе))))

\iusr{Сергей Кравцов}
Вот и я о том же!

\iusr{Игорь Семенов}
Гоголь со своей птицей, предвосхитил разом всех модернистов, видение такое у художника

\iusr{Rayisa Dmytrenko}
Його звали Миколка.

\iusr{Сергей Кравцов}
Это какой такой теперь Гоголь? Писательница теперь Ницой! И не важно, что она еще и рыбой торгует.

\iusr{Аркадій Друкер}
\textbf{Сергей Кравцов} так и Гоголь не только поэмы писал.

\iusr{Savelyeva Olena}
Отличный текст!

\ifcmt
  ig https://scontent-mxp1-1.xx.fbcdn.net/v/t39.1997-6/s168x128/17634213_1652591098100624_731967241620291584_n.png?_nc_cat=1&ccb=1-5&_nc_sid=ac3552&_nc_ohc=wNTmUISdYpQAX8GkND6&_nc_ht=scontent-mxp1-1.xx&oh=53378bb964a0f43a6e01ea3af1efacb2&oe=61B800F7
  @width 0.1
\fi

\iusr{Анатолий Золотушкин}
\textbf{Savelyeva Olena} спасибо @igg{fbicon.heart.red}

\iusr{Арт Юрковская}

Ещё его упрекнули, почему он на русском языке написал, а не на державний. Ведь
украинский хлеб ел и вообще был украинцем. А то, что ездил работать в Россию -
"чемодан, вокзал, Россия". Но он чемодан, вокзал, Рим" решил по безвизу. И там
написал: "Ну, шо сынку, помогли тебе твои ляхи?" И тут почалось...

\begin{itemize} % {
\iusr{Юрий Прокопец}
\textbf{Арт Юрковская} он ещё и Яновским был , т.е. поляком . Да .

\iusr{Людмила Лысюк}
\textbf{Art Yurkovska} 

перед тем как делать подобные "умозаключения", неплохо бы получить историю и
разобраться, а не демонстрировать свое невежество. А так выходит по принципу:
оплюю кого-нибудь "великого" - так и сам поднимусь в своих глазах."

\begin{itemize} % {
\iusr{Арт Юрковская}
\textbf{Людмила Лысюк} Я ни на кого не плевала.Начинается?

\iusr{Арт Юрковская}
\textbf{Людмила Лысюк} 

А юмор и иронию отменили? В ФБ люди серьезные, сгрызут и не то что Гоголя. Знаю
по себе. За публикацию рецепта борща из мемуаров Вертинского, чуть не убили. И
отбили, но не почки, конечно, а охоту писать интересное.

\iusr{Аркадій Друкер}
\textbf{Art Yurkovska} Не этот ли рецепт? 5 л НЕГАЗИРОВАННОЙ воды...

\ifcmt
  ig https://scontent-mxp1-1.xx.fbcdn.net/v/t39.30808-6/243012674_10219418056188512_2057625143861819247_n.jpg?_nc_cat=101&ccb=1-5&_nc_sid=dbeb18&_nc_ohc=7kX6dXlPQysAX9t8juB&_nc_ht=scontent-mxp1-1.xx&oh=00bd17a6fd9489d575e6dce410e32cb0&oe=61B97D15
  @width 0.4
\fi

\iusr{Арт Юрковская}
\textbf{Аркадій Друкер} 

Нет. Там на свекольном квасе с балабушками из щуки и грибными ушками.)))

\end{itemize} % }


\iusr{Oleksa Vovk}
\textbf{Art Yurkovska} В те мрачные времена державным был расейский язык.

\begin{itemize} % {
\iusr{Владимир Шмуклер}
\textbf{Oleksa Vovk} для Гоголя в Риме - мрачные времена? За державный счет? Это Тараса Григоровича за словесную, да еще и на недержавном языке, невоздержанность к царским особам отправили на курорт Мангышлак. А так он тоже не бедствовал в Италии.
\end{itemize} % }

\end{itemize} % }

\iusr{Сергей Кравцов}

А вообще еврем очень не любят Гоголя. За его их описание, в том же \enquote{Тарасе
Бульбе}. И вполне понятно, почему из Хайфы такие о нем тексты.

\begin{itemize} % {
\iusr{Анатолий Золотушкин}
\textbf{Сергей Кравцов} ошибаешься, любезный. Люблю Гоголя и текст написан с любовью

\iusr{Сергей Кравцов}
С любовью... большой любовьюю, к ненавистникам Гоголя.

\iusr{Yan Nudelman}
\textbf{Сергей Кравцов} Этот текст не для всех. Вы в этом списке.

\iusr{Сергей Кравцов}
\textbf{Yan Nudelman} ок, помещаю и вас в список

\iusr{Тетяна Леонова}
\textbf{Сергей Кравцов} ,расслабьтесь. Вьі еще Резника покритикуйте,у вас получиться... Люди ульібнуться по-доброму не прочь с утра, но всегда начеку и бдят правдолюбители с дегтем...

\iusr{Tatiana Thoene}
\textbf{Сергей Кравцов} да бросьте! Он употреблял слово «жид», которое в те времена было обычным, а не ругательством. И персонажа Янкеля он описал с симпатией.

\iusr{Сергей Кравцов}
Гм, если я сейчас с такой же симпатией опишу Золотушкина, то меня тут же обвинят. Увы, у Гоголя было больше свободы.
\end{itemize} % }

\iusr{Жанна Павильч}
 @igg{fbicon.thumb.up.yellow} салатик

\iusr{Hennadii Tsurkov}
Неплохо и в духе Хармса)

\iusr{Сергей Кравцов}
\textbf{Hennadii Tsurkov} Ювачев Золотушкина того бы в Пакины бы определил. Ну или в Ракукины.

\iusr{Татьяна Сирота}

\ifcmt
  ig https://i2.paste.pics/b1a3ed54e082ad2607e7c1222a33437c.png
  @width 0.3
\fi

\iusr{Юрий Панчук}

А мне Нежин понравился, не знаю как насчет жить там, но учиться Гоголю было
наверно комфортно.

\begin{itemize} % {
\iusr{Kateryna Masliechkina}
\textbf{Юрий Панчук} Как раз не очень было комфортно. Он скучал очень в Нежине и не слишком хорошо отзывался о качестве обучения. Мечтал о Петербурге. Но Нежин действительно сейчас красивый старинный и уютный городок.
\end{itemize} % }

\iusr{Сергей Кравцов}

Кстати, Ирка Фарион о Гоголе такое же навалять могла, но на мове. Она баба
старой коммунистической закалки и Гоголя ненавидеть умеет правильным образом.


\iusr{Петр Закалюжний}
за Нежин обидно...

\begin{itemize} % {
\iusr{Анатолий Золотушкин}
\textbf{Петр Закалюжний} извините, хотел исправить, но не получилось
\end{itemize} % }

\iusr{Евгения Ерёменко}

Простите, уважаемые одногруппники, не кажется ли нам, что подобный тон
преподнесения материала нас не возвышает?

\begin{itemize} % {
\iusr{Виктор Задворнов}
\textbf{Евгения Ерёменко} есть послевксие.

\iusr{Сергей Кравцов}
Как раз задача стоит обратная

\iusr{Yan Nudelman}
\textbf{Евгения Ерёменко} Над кем вы хотите возвышаться? Закажите себе трехметровый памятник. Знаете, как возвышает!

\begin{itemize} % {
\iusr{Евгения Ерёменко}
\textbf{Yan Nudelman} очевидно, моё пожелание к Вам не относится. Бывает

\iusr{Ольга Патлашенко}
\textbf{Yan Nudelman} Не умно. Правильно пишет Евгения, что эти насмешки и сарказм в адрес Н. В. Гоголя не делают нам чести и не возвышают. Нам до классика ой как далеко. И не стоит так саркастично писать о нем. Он после себя оставил такое наследие. И всегда помнил, что его Родина - Украина.

\iusr{Yan Nudelman}
\textbf{Ольга Патлашенко} В душе был русский и писал по русски. Предал украинскую нацию. Сегодня ему бы запретили вьезд на Украину.

\iusr{Ольга Патлашенко}
\textbf{Yan Nudelman} примитивно мыслите. Никого он не предавал. В чем предательство? В том, что писал по русски? И страна была другая. Не было тогда такой страны, как Украина к сожалению. Была царская Россия.
И сами-то - пишете тоже по русски.
И не на Украину, а в Украину
\end{itemize} % }

\end{itemize} % }

\iusr{Тамара Ар}
Нежин

\ifcmt
  ig https://scontent-mxp1-1.xx.fbcdn.net/v/t39.30808-6/243132421_4416840175069871_2810654837286840525_n.jpg?_nc_cat=111&ccb=1-5&_nc_sid=dbeb18&_nc_ohc=Lro2IaKrfCUAX81zhtS&_nc_ht=scontent-mxp1-1.xx&oh=bf702f4fb34184fbd7e911975442dcfa&oe=61B93F04
  @width 0.4
\fi

\iusr{Тамара Ар}
Греческая церковь. Нежин

\ifcmt
  ig https://scontent-mxp1-1.xx.fbcdn.net/v/t39.30808-6/243535194_4416845631735992_6971727199104564188_n.jpg?_nc_cat=111&ccb=1-5&_nc_sid=dbeb18&_nc_ohc=mmSUwCCuljgAX__cxPT&_nc_ht=scontent-mxp1-1.xx&oh=029f74ed34502417412765313838ee19&oe=61B81AB6
  @width 0.4
\fi

\iusr{Тамара Ар}
Собор Иоана Златоуста. Нежин

\ifcmt
  ig https://scontent-mxp1-1.xx.fbcdn.net/v/t39.30808-6/243519284_4416846641735891_5955182790464725801_n.jpg?_nc_cat=103&ccb=1-5&_nc_sid=dbeb18&_nc_ohc=e9hvjeTJAC4AX9-rV-L&_nc_ht=scontent-mxp1-1.xx&oh=771b54a215bbfdddf4988b63c1e3ecd8&oe=61B91A11
  @width 0.4
\fi

\iusr{Kateryna Masliechkina}

За Нежин хочу заступиться - это не просто мой родной город, в котором я провела
полжизни, но и древний, уютный и красивый городок со множеством прекрасных
храмов, хотя Николай Васильевич действительно не любил Нежин во время своей
учебы и , как всякий молодой человек, мечтал о столице - о Петербурге. Где Вы
взяли, что он мечтал о Киеве не понятно) Изначально, он стремился в Петербург,
но из-за сырого климата и более, чем скромных доходов он подумывал о Москве.
Желание переехать в Киев возникло позже и в связи с надеждами получить место в
Университете. Недавно читала его письма.

\begin{itemize} % {
\iusr{Yan Nudelman}
\textbf{Kateryna Masliechkina} Конечно мечтал. Все творческие личности мечтают о Киеве.

\iusr{Yan Nudelman}
\textbf{Евгения Ерёменко} Вы всем раздаете оценки?

\iusr{Vladimir M. Pushkarev}
\textbf{Yan Nudelman} Что есть - те есть!

\iusr{Антон Пинчук}
\textbf{Kateryna Masliechkina} лицей один из лучших в империи. Автор поста ни слухом, ни духом.
\end{itemize} % }

\iusr{Тамара Ар}
Крашенки на Пасху. Нежин

\ifcmt
  ig https://scontent-mxp1-1.xx.fbcdn.net/v/t39.30808-6/243025599_4416848091735746_2852853763547119457_n.jpg?_nc_cat=109&ccb=1-5&_nc_sid=dbeb18&_nc_ohc=i-FeSl8_MWUAX9ILgpD&_nc_ht=scontent-mxp1-1.xx&oh=eb727fb6dbbda44dd513d4f55e660853&oe=61B901A6
  @width 0.4
\fi

\iusr{Тамара Ар}
В парке им. Гоголя. Нежин

\ifcmt
  ig https://scontent-mxp1-1.xx.fbcdn.net/v/t39.30808-6/243106978_4416848861735669_7884828173141089422_n.jpg?_nc_cat=109&ccb=1-5&_nc_sid=dbeb18&_nc_ohc=a3Qdz3DHWWgAX8OZyUF&_nc_ht=scontent-mxp1-1.xx&oh=ef7eab03b5c449447d9b0a1a549fa9b5&oe=61B89D7B
  @width 0.4
\fi

\iusr{Тамара Ар}
Здание старинного почтамта. Нежин

\ifcmt
  ig https://scontent-mxp1-1.xx.fbcdn.net/v/t39.30808-6/243008105_4416849808402241_7009877229176938887_n.jpg?_nc_cat=101&ccb=1-5&_nc_sid=dbeb18&_nc_ohc=HlLBV7bENCsAX_8b-8w&_nc_ht=scontent-mxp1-1.xx&oh=12bebe6dcf3cb15edb5c3b376e6b1eb3&oe=61B94A1E
  @width 0.4
\fi

\iusr{Тамара Ар}
Там же

\ifcmt
  ig https://i2.paste.pics/8cf84b85ad5baf01e29e845b60f22d31.png
  @width 0.3
\fi

\iusr{Арт Юрковская}

Супер.

\iusr{Тамара Ар}

\obeycr
Нежинское небо,
Воздух - как кристалл.
Нежин, я приехал,
Город, я устал.
Твой прозрачный воздух
Пью как эликсир.
Всех путей возможных
Ты - ориентир.
Будь благословен ты –
«Город огурцов»,
Торгашей, студентов,
Зодчих и певцов!
\restorecr

\iusr{Kateryna Masliechkina}
\textbf{Тамара Ар} Браво!

\iusr{Тамара Ар}
И еще раз, Николаевский собор. Нежин. Украинское барокко

\ifcmt
  ig https://scontent-mxp1-1.xx.fbcdn.net/v/t39.30808-6/243052938_4416862651734290_4996673432804811918_n.jpg?_nc_cat=101&ccb=1-5&_nc_sid=dbeb18&_nc_ohc=TnQeyql5Pl4AX_ZZbuJ&_nc_ht=scontent-mxp1-1.xx&oh=6495ad7184b7b2bdefa3c888971d4e41&oe=61B985D5
  @width 0.4
\fi

\iusr{Тамара Ар}
Читайте Гоголя, господа! @igg{fbicon.smile} 

\iusr{Сергей Кравцов}

О Гоголн и университете. Он в Питере жил весьма не богато и друзья
рекомендовали его в университет, читать лекции. Но не его это было дело. Лектор
он оказался весьма посредственный и вскоре дело это забросил вовсе.

\iusr{Антон Пинчук}

Весь пост с искажённой историей. Киевляне хотели видеть Гоголя возглавляющим
кафедру истории. Российская цензура и жандармерия не разрешила. Автор себя к
Гоголю приравнял ?  @igg{fbicon.smile}  смешно

\begin{itemize} % {
\iusr{Сергей Кравцов}
Увы, преподаватель и администратор Гоголь был слабый.

\begin{itemize} % {
\iusr{Антон Пинчук}
\textbf{Сергей Кравцов} 

а где он преподавал? Что вы знаете? Он должен был возглавить кафедру истории.
он писал в письмах \enquote{ах, какую я историю Украины напишу!} Испугались, не
допустили!

\iusr{Сергей Кравцов}
\textbf{Антон Пинчук} он преподовал краткое впемя в университете Санкт-Петербурга. Почитайте кнр биографию внимательно. О том, что чиновник он никакой, он и сам не скрывал.

\iusr{Антон Пинчук}
\textbf{Сергей Кравцов} он служил в Министерстве вначале, и о той работе он писал, что это не его стезя. На кафедре истории в С-П универе, он занялся историей Украины, за что и не позволили ехать в Киев. Про его неумение преподавать, нигде ни слова. Как раз все отмечают, что он был хорошим рассказчиком.

\iusr{Сергей Кравцов}
\textbf{Антон Пинчук} Рассказчиком да, но не преподавателем. Почитайте воспоминания о нем.

\iusr{Антон Пинчук}
\textbf{Сергей Кравцов} я как раз читал. И что вы имеете ввиду преподавателем. Он собирался возглавить кафедру истории. Но ему, в силу национальности, не дали.

\iusr{Сергей Кравцов}
Найдите и прочтите. А после спорьте и рассказывайте.

\iusr{Антон Пинчук}
\textbf{Сергей Кравцов} я биографию Гоголя знаю хорошо. Потому если вы знаете больше, давайте ссылку и первоисточник

\iusr{Сергей Кравцов}
\textbf{Антон Пинчук} Если интересуетесь, то найдете. Я не учитель и не помощник в чьм-то учении. О преподавании Гоголя в университете СПб информация довольно распространенная.

\iusr{Сергей Кравцов}
\textbf{Антон Пинчук} А слово "давайте" оставьте для тех, кто от вас зависим.

\iusr{Антон Пинчук}
\textbf{Сергей Кравцов} все понятно. Данных нет.

\iusr{Антон Пинчук}
\textbf{Сергей Кравцов} Если вы вступаете в полемику, будьте готовы, что люди могут знать больше, чем вы. Всегда интересовался Гоголем, перечитал всю его переписку и о том, что плохой преподаватель, прочитал у вас. Да и перепутали вы его работу в Министерстве, в качестве чиновника и на профессорской должности. На этом откланивпюсь.

\iusr{Сергей Кравцов}
\textbf{Антон Пинчук} Увы, в сети полно таких вот \enquote{спецов} из википедии. Не знающих и ждущих, что кто-то за них все им покажет.

\iusr{Антон Пинчук}
\textbf{Сергей Кравцов} я вам сказал, я изучал биографию и читал всю переписку. Прощайте.

\iusr{Сергей Кравцов}
\textbf{Антон Пинчук} Даю пкедаля Пинчуку и отправляю изучать материалы о Гоголе. Особенно о его питерском периоде.
\end{itemize} % }

\end{itemize} % }

\iusr{Таня Куликова}
 @igg{fbicon.face.grinning.big.eyes} как всегда замечательно

\begin{itemize} % {
\iusr{Анатолий Золотушкин}
\textbf{Таня Куликова} спасибо

\iusr{Dubizhansky Ludmila}
\textbf{Анатолий Золотушкин} Саме так треба писати, щоб з самiсeнького ранку усiх небайдужих розбурхати! Здоровיячка!

\iusr{Анатолий Золотушкин}
\textbf{Dubizhansky Ludmila} дякую
\end{itemize} % }

\iusr{Svitlana Agafonova}
слог легок, как и ГОГОЛЬ ВЕЛИК, но сий пасквиль ни о чем(((
Может не знаю сути?

\iusr{Анатолий Золотушкин}
\textbf{Svitlana Agafonova} если надо объяснять, то не надо

\iusr{Тетяна Леонова}
. @igg{fbicon.face.eyes.star} 
Антолий Золотушкин, мьі вас любим и я, хоча україномовна, абсолютно сприймаю
Миколу Васильовича таким, яким він був  @igg{fbicon.face.astonished} 

\iusr{Анатолий Золотушкин}
\textbf{Тетяна Леонова} дякую за любов @igg{fbicon.heart.red}

\iusr{Elena Efimenko}
Каждый человек прежде всего критиковать и оценивать должен себя. Произведения
Гоголя и Пушкина любли и восхищаюсь полетом их мысли

\iusr{Татьяна Рачинская}
Супер, а мне понравился такой рассказ быль

\iusr{Анатолий Золотушкин}
\textbf{Татьяна Рачинская} спасибо. Насчёт были сомневаюсь

\iusr{Татьяна Рачинская}

\ifcmt
  ig https://i2.paste.pics/7c51fff4305323525f647e20052535cf.png
  @width 0.3
\fi

\iusr{Александр Асатуров}
ода комментаторам))

\iusr{Zoya G Zoia}
Какая классная ирония. Удовольствие. @igg{fbicon.heart.red}

\iusr{Анатолий Золотушкин}
\textbf{Zoya G Zoia} спасибо @igg{fbicon.heart.red}

\iusr{Мария Константиновская}
Мне показалось, слишком панибратский текст.

\iusr{Viktoria Terpylo}
А почему бы не поехать в Рим?...ведь он там прожил четыре года на довольствии
государя-батюшки

\iusr{Gennadiy Ab}
\textbf{Сергей Кравцов} 

Николай Васильевич Гоголь ...Когда я читаю его, я чувствую себя...

При чем тут евреи? Это часть жизни и быта нашего народа.. Начиная от Исуса, от
имен(Ивана и Николая),от музыки и культуры.. Меня всегда раздражает, когда свои
портянки подкладываем кому то....


\iusr{Nadiia Basalai}
\obeycr
Я хоч і вчилася класичній філології, але маю свої погляди і почуття гумору.
Мені дуже сподобалася Ваша інтерпретація.
Усі класики такі ж само люди, як і ми.
Десь, можливо, так і відбувалося б в реаліях нашого часу.
А стосовно коментаторів –влучно @igg{fbicon.hand.ok} 
\textbf{Олег Коваль}
Мені подобається підхід адмінів однієї із груп до коментаторів.
Авторський тест – це світогляд автора, його право писати своє бачення і думки.
Має місце тільки необразлива пропозиція, а все інше – критика, а тим більше, хамство, образи і навіть погрози неприпустимо.
Усі тексти проходять модерацію, потім навіть текст не завжди можна відредагувати.
І якщо вже текст надруковано, які претензії можна висувати автору?
Давайте поважати один одного і підтримувати саме авторів, адже завдяки їм група живе.
\restorecr

\iusr{Сергій Авдєєв}
Гоголь би зацінив - це без сарказму

\iusr{Светлана Адаменко}

А вот за Нежин таки обидно ! Это я вам как простой народ киевский классику
\enquote{Киевских историй} говорю.  Вот вы поезжайте в Нежин и спросите !  @igg{fbicon.smile} 

\iusr{Анатолий Золотушкин}
\textbf{Светлана Адаменко} я хотел исправить, не получилось

\iusr{Ludmila Krywicka}
Браво автору!!!!!

\iusr{Анатолий Золотушкин}
\textbf{Ludmila Krywicka} спасибо

\iusr{Yulianna Portnoy}

Обмелел наш Днепр, слегка позеленел от постов и слов коментаторов, но верю - он
не обидчивый, отойдет, облака поплачут над ним, люди добрые неводы забросют,
слова да грязь всякую, маски да перчатки выловят, зимой снег остудит, по весне
всё оттает, и покатит наша река волны, воды чистые.

И все у нас будет хорошо

\begin{itemize} % {
\iusr{Тамара Ар}
\textbf{Yulianna Portnoy} 

позеленел наш Днепр от многочисленных стоков фосфатов, обмелел от засилья
водяного ореха на киевском море, как только почистят орех, закрывающий
прорастанием попадание кислорода в воду, как только стоки не будут сливаться в
великую реку, сразу и небо поголубеет, и Днепр чистым станет, как в гоголевские
времена, и все у нас будет хорошо


\iusr{Yulianna Portnoy}
\textbf{Тамара Ар} аминь

\iusr{Тамара Ар}
\textbf{Yulianna Portnoy} Посмотрим,,,,,

\end{itemize} % }

\iusr{Boris Shulman}

Когда Гоголь писал \enquote{редкая птица долетит до середины Днепра}, он подразумевал,
если она будет лететь - вдоль, а не поперёк. Длина реки более 2.000 км, так что
он был почти прав. Хотя орнитологи, наверное, и не согласятся.

\iusr{Галина Рябенькая}
Хорошо написано: с юмором, иронией! Читается легко и интересно!

\iusr{Анатолий Золотушкин}
\textbf{Галина Рябенькая} спасибо

\iusr{Галина Петрова}
Великолепно! Спасибо)) 
@igg{fbicon.beaming.face.smiling.eyes}{repeat=3} ) и да, Гоголю бы точно
понравилось! @igg{fbicon.hands.applause.yellow}{repeat=5} 

\iusr{Анатолий Золотушкин}
\textbf{Галина Петрова} спасибо

\iusr{Галина Петрова}
\textbf{Анатолий Золотушкин} я поделилась))

\iusr{Вероника Иванова}
 @igg{fbicon.laugh.rolling.floor} 

\iusr{Наталья Каминская}
\enquote{То ль дело Киев? Что за край!»

\iusr{Вита Вовченко}

С удовольствием прочитала! И, да, насчёт Нежина вы правы: ничего там хорошего
нет, увы...

\begin{itemize} % {
\iusr{Андрей Надиевец}
\textbf{Вита Вовченко} 

Огурцы, помидоры, кабачки, недавно скидка была в магазине \enquote{Фора} на
баклажаны в банке с красивой этикеткой, все с Нежина..

\begin{itemize} % {
\iusr{Вита Вовченко}
\textbf{Андрей Надиевец} В СССР можно было из Нежина купить только маринованные огурцы в трехлитровой банке, остальные - маленькие, все шли на экспорт. Или в Москву.)))

\iusr{Тамара Ар}
\textbf{Андрей Надиевец} М да,,,, на чёрное, говорят белое, на белое, чёрное,,,,, тактика

\iusr{Тамара Ар}
\textbf{Андрей Надиевец} Это о старинном городе, я

\iusr{Андрей Надиевец}
\textbf{Вита Вовченко} Как же вы правы, Виктория, что в переводе с американского языка- Победа... Маленькие, но по три, на Рублевку, большие, но по пять, в Жмеринку .

\iusr{Vladimir M. Pushkarev}
\textbf{Андрей Надиевец} 

\enquote{Нежинские} огурцы закрывали по всей Украине! Я в 7-м классе летом
работал на консервном заводе в г. Городенка Ив.-Франк. обл. - мы там клеили
этикетки на банки с нежинскими огурцами.


\iusr{Андрей Надиевец}
\textbf{Vladimir M. Pushkarev} Жаль, этикеток не осталось,красивые были..

\iusr{Vladimir M. Pushkarev}
\textbf{Андрей Надиевец} И огурцы были ничего...

\iusr{Вита Вовченко}
\textbf{Vladimir M. Pushkarev} 

Тогда получается, это просто сорт "Нежинский"? В самом Нежине был(есть?)
консервный завод. Оттуда продукцию развозили по другим городам и странам. В
самом Нежине видела только соки в трехлитровых банках, но это в конце 80-х.
Знакомый работал экспедитором, так угощал огурцами и помидорами: отличные на
вкус, нужно отметить.


\iusr{Kateryna Masliechkina}
\textbf{Андрей Надиевец} Сейчас новые этикетки. Огурцы по- прежнему вкусные и по-прежнему можно встретить в магазинах в
Москве и в городах Украины.

\iusr{Андрей Надиевец}
\textbf{Kateryna Masliechkina} А Москва, это где?? И в каких городах Украины они есть?

\iusr{Kateryna Masliechkina}
\textbf{Андрей Надиевец} Во всех городах Украины. Москву на карте можете найти.

\iusr{Андрей Надиевец}
\textbf{Kateryna Masliechkina} Спасибо вам большое за огурцы.

\iusr{Kateryna Masliechkina}
\textbf{Андрей Надиевец} Не за что. Не моя заслуга)

\iusr{Андрей Надиевец}
\textbf{Kateryna Masliechkina} Город древний, город длинный, имярек Екатерины, даже свод тюрьмы старинной там положен буквой Е. Стихи о вас, обошлось без огурцов

\iusr{Kateryna Masliechkina}
\textbf{Андрей Надиевец} Не о мне, конечно, но, да ладно.

\end{itemize} % }

\iusr{Мария Тимошенко}
\textbf{Вита Вовченко} 

Нежин был очень хороший городок там родина его матери он там учился братья
Безбородьки дали деньги на лицей прекрасная библиотека торговый путь церкви
рядом синогога католическая провославная кто виноват что счас все запустении
там церьков по каторой написано Вий

\begin{itemize} % {
\iusr{Вита Вовченко}
\textbf{Мария Тимошенко} Дело в том, что я училась в Нежинском пединституте. Да, красивое здание, да есть исторические ценности, но то, о чем я говорю, может понять лишь тот, кто там пожил))) Кстати, о лицеистах гоголевских времён до сих пор легенды ходят по Нежина, о том, как их боялись, дочерей прятали по хатам и т. п. А Нестор Кукольник выбросился из окна аудитории прямо во время лекции... Энергетика там, прямо скажем, странная.

\iusr{Андрей Надиевец}
\textbf{Вита Вовченко} Вы правы, энергетика там странная, но намного хуже в Трускавце, где я учился, с окон тоже выбрасывались и могут ещё, там публика ещё та...

\iusr{Мария Тимошенко}
\textbf{Вита Вовченко} да я тоже знала что лециисты вели себя не одекватно может вы и больше знаете я была там на экскурсии
\end{itemize} % }

\end{itemize} % }

\iusr{Андрей Надиевец}
Мне повезло, я знал Гоголя...

\iusr{Юрий Прокопец}
Очень может быть, что Николая Васильевича в детстве звали Мыколкою... Полтава всё-таки .

\begin{itemize} % {
\iusr{Арт Юрковская}
\textbf{Юрий Прокопец} А ничего,что он поляк?

\iusr{Юрий Прокопец}
\textbf{Арт Юрковская} по польски - Миколай

\iusr{Volodymyr Nekrasov}
\textbf{Art Yurkovska} віросповідання православне, «душа хохлацкая»

\iusr{Tatiana Klotchko}
Думаю, его звали Николенькой !
\end{itemize} % }

\iusr{Виктор Штонда}
вот так врут о Гоголе, я о посте @igg{fbicon.face.angry} 

\iusr{Vadim Vadim}
Раньше, Днепр был ширше, а птицы жирше))

\begin{itemize} % {
\iusr{Ирина Иванченко}
\textbf{Vadim Vadim} , наваристей ?

\iusr{Vadim Vadim}
\textbf{Ирина Иванченко} Жирной, потому и падала в Днепр)

\iusr{Ирина Иванченко}
\textbf{Vadim Vadim} , ну, да, на нежинских харчах не полетаешь много...

\iusr{Vadim Vadim}
\textbf{Ирина Иванченко} не той тепер Миргород))

\iusr{Ирина Карлашова}
\textbf{Vadim Vadim} Птицы жирше, а потому и не перелетемши!
\end{itemize} % }

\iusr{Татьяна Остапец}

\ifcmt
  ig https://scontent-mxp1-1.xx.fbcdn.net/v/t39.1997-6/p480x480/105941685_953860581742966_1572841152382279834_n.png?_nc_cat=1&ccb=1-5&_nc_sid=0572db&_nc_ohc=_iE3XtYBdGYAX_yWoLc&_nc_ht=scontent-mxp1-1.xx&oh=c51abc2b22b58bf5986c988f967a31d1&oe=61B7FD4B
  @width 0.2
\fi

\iusr{Volodymyr Nekrasov}
Гірким словом моїм посміюся.  @igg{fbicon.books} 

\iusr{Людмила Рябчук}
Не будьте слишком серьезными, улыбайтесь, немного юмора никому не повредит! @igg{fbicon.face.zany} 

\iusr{Leonid Dukhovny}

\obeycr
"Здесь Гоголь, интелект свой подтверждая,
За птицей редкой над Днепром следит
И профиль его странный наблюдая
"Он - наш, не наш" - решает эрудит"....
( песня "Русановка", из "Киевского цикла")
\restorecr

\begin{itemize} % {
\iusr{Анатолий Золотушкин}
\textbf{Leonid Dukhovny} пришлите ссылку на тексты, пожалуйста. В личку, а то в группе нельзя

\iusr{Leonid Dukhovny}
\textbf{Anatol Zolotushkin} Не понял, что нельзя
\end{itemize} % }

\iusr{Татьяна Пурченошвили Филимонова}
Смешно! @igg{fbicon.face.grinning.smiling.eyes}  @igg{fbicon.beaming.face.smiling.eyes}  @igg{fbicon.face.tears.of.joy} 

\iusr{Dmitrii Pisanenko}
Не красота, а смех -великая сила!

\iusr{Наталья Корниенко}
Вижу каждое утро этот памятник посредине Днепра, проезжая мост Патона скульптор Васильев, 2013, сварка и металл, Дню Киева открыт

\iusr{Vicki Seplarsky}

Анатолий, я стою в очереди на вашу книгу. Правда, в силу разницы во времени,
мне приходить отмечаться не очень удобно. Решила в этом году на Халловин
одеться панночкой. Не подскажите прикид полтавской ведьмы, а то все
американские ведьмы одеты очень скучно.  Кстати, у нас в Netflix идёт фильм
Gogol -балдёж от американского перевода!

\iusr{Анатолий Золотушкин}
\textbf{Vicki Seplarsky} я не специалист по ведьмам. Это к Гоголю

\iusr{Михаил Рабинович}

Птичка симпатичная но идея мне кажется малость сомнительная. Это вроде
надгробного памятника - типа долетела птичка до середины и тут и осталась
навеки.

\iusr{Сергей Власов}

Редкая птица долетит до середины Днепра не потому, что не может, а потому, что
оно ей нафик не нужно....

\iusr{Мария Тимошенко}

Когда в Киеве был руководил Давыдов на месте Бабьего Яр хотели построить
развлекательный парк и тогда от туда пошла ловина грязью на Куреневку 2 метра
высотой скорость была что догоняла автобусы на воротах травайного парка повисли
люди и погибли детей снимали с крыш верталетами много погибли надо всем помнить
что без наказания ничего не пройдет


\end{itemize} % }
