% vim: keymap=russian-jcukenwin
%%beginhead 
 
%%file 18_09_2021.fb.donchenko_elena.1.kurjer_mova.cmt
%%parent 18_09_2021.fb.donchenko_elena.1.kurjer_mova
 
%%url 
 
%%author_id 
%%date 
 
%%tags 
%%title 
 
%%endhead 
\subsubsection{Коментарі}

\begin{itemize} % {
\iusr{Світлана Весна}

\obeycr
В Києві купа співбесід проходила на російській мові ( зі сторони роботодавця). Я принципово не переходила на звичну їм мову. Анкети до речі теж, більшість надруковані російською. Зате їхні назви магазинів на українській.
- Светлана Викторовна? ( звучить вчергове в телефонній розмові)
- Мы рассмотрели ваше резюме! ( Та що ви говорите!? Не пройшло і пів року!)
- Вы нам подходите!
-А ви мені ні!
- Ідіть на \#уй , господа! Ідіть на х\#й!
\restorecr

\begin{itemize} % {
\iusr{Olena Donchenko}

А потім ці люди поширюють маніпулятивні сентенції на зразок "Лучше быть
русскоязычной няшечкой, чем украиноязычной сволочью", і ти мусиш постійно
обтікати в своїй же країні просто через те, що бажаєш реалізувати своє
справедливе і законне право чути рідну державну мову.


\iusr{Світлана Весна}
\textbf{Олена Донченко} Це якийсь треш!(

\iusr{Надія Віннік}
\textbf{Світлана Весна} жодного разу не чула, щоб українською проводили співбесіду, завжди мокшанською. Одного разу мені навіть сказали, що я їм не підходжу, бо керівник компанії спілкується російською, а я ні.

\iusr{Світлана Весна}
\textbf{Nadia Vinnik} 

Вони там дійсно прикукурені! Мене не взяли працювати,, Молоко від фермера,, ( в
них теж все російською і співбесіда, і анкета, і дебільні задачки з ребусами)
тому що їх,, напружила ,, , моя сторінка в Інстаграм. Сцуко! Там лише 6 фото на
той момент було. І у Вайбері на фото в мене половина обличчя закрита
бейсболкою.

- Светлана Викторовна, вы от нас что то скрываете? Что именно? Вам есть что
скрывать, если вы не хотите вылаживать это в сторис!

Приходите через две недели, если надумаете нам рассказать, что вы скрываете. Может тогда мы возьмëм вас на стажировку .

Дебіли! Я не встигала ахуївати від цього гівна у вуха і очі!


\iusr{Надія Віннік}
\textbf{Світлана Весна} мда... То, мабуть, випускники трускавецьких курсів, ну, десь з тієї категорії людей. ))

\iusr{Світлана Весна}
\textbf{Nadia Vinnik} Таке враження що я хотіла пробратися працювати в СБУ)

\iusr{Надія Віннік}
\textbf{Світлана Весна} це точно!

\end{itemize} % }

\iusr{Veronika Beleteeva}
І я сумую за попереднім хлопцем. Чемний був(

\iusr{Yelizaveta Cherednichenko}
Замовляла тур в Єгипет в Поїхали с нами - жодна з менеджерів не розмовляє українською і на мову відповідають російською

\begin{itemize} % {
\iusr{Andriy Borodavko}
\textbf{Yelizaveta Cherednichenko}
Звертався кілька разів до одного з відділень.
Перемкнув на українську вдох менеджерок та директрису/власницю.
\end{itemize} % }

\iusr{Наталія Владимирцева}
В куми бабуся впадає в істерику, коли та розмовляє українською зі своїми дітьми, а ви про кур'єрів... сумно

\iusr{Алена Заглада}

Боже, попередній кур'єр був прекрасний, він всього раз мені дзвонив російською.
Цей же- я чесно не знаю, як звать- він таке відчуття, що принципово
узькощелепний. Ось саме сьогодні був. Оболонь.\textbf{Ukraine Express} це політика
компанії- обслуговування іноземною мовою?

\begin{itemize} % {
\iusr{Olena Donchenko}
\textbf{Альона Заглада} Справедливості заради - це не є політикою компанії.

\iusr{Алена Заглада}
\textbf{Олена Донченко} не знаю. Компанія про це знає, як мінімум- ти ж писала про це вже.

\iusr{Olena Donchenko}

Я писала про це раз, так, і ситуація змінилася належним чином, не мала після
того зауважень взагалі. Пишу вдруге, знову сподіваюся, що проведуть своєму
працівнику політінформацію. За всі роки співпраці, а це вже дуже-дуже багато, я
вперше почула від їхнього співробітника, що він не розуміє української. Тож
будьмо справедливі.

\end{itemize} % }

\end{itemize} % }
