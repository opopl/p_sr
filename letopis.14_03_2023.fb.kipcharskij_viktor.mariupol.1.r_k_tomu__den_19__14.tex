%%beginhead 
 
%%file 14_03_2023.fb.kipcharskij_viktor.mariupol.1.r_k_tomu__den_19__14
%%parent 14_03_2023
 
%%url https://www.facebook.com/permalink.php?story_fbid=pfbid02orw1Q98DaGvWnmoA3G343H96jAfh9a9pzZL7tFBdtJEVkw2FXW4MgAGGAAmH55nRl&id=100006830107904
 
%%author_id kipcharskij_viktor.mariupol
%%date 14_03_2023
 
%%tags mariupol,mariupol.war,dnevnik,14.03.2022
%%title Рік тому: День 19. 14-03-22. Понеділок
 
%%endhead 

\subsection{Рік тому: День 19. 14-03-22. Понеділок}
\label{sec:14_03_2023.fb.kipcharskij_viktor.mariupol.1.r_k_tomu__den_19__14}

\Purl{https://www.facebook.com/permalink.php?story_fbid=pfbid02orw1Q98DaGvWnmoA3G343H96jAfh9a9pzZL7tFBdtJEVkw2FXW4MgAGGAAmH55nRl&id=100006830107904}
\ifcmt
 author_begin
   author_id kipcharskij_viktor.mariupol
 author_end
\fi

Рік тому:

День 19. 14-03-22. Понеділок.

0:50. Відносно тихо.

Якщо будемо виїжджати - віддам ключі від гаражу Ігореві - у гаражі є дрова, а у
підвалі може бути їжа закладена у 2014-го році.

\ii{14_03_2023.fb.kipcharskij_viktor.mariupol.1.r_k_tomu__den_19__14.pic.1}

Сьогодні очікуємо на 4-й раунд перемов.

6:45. Спочатку гупає рідко, а потім все частіше і гучніше і так до 6:00.

Поснідав вівсянкою. Пішов кип'ятити воду собі та Едіку. Хлопці вирішили зрубати
акацію у дитячому садку, бо біля балки прилітає все частіше: на Шлаковій горі
стоїть наша артбатарея, по якій луплять з-за Азовсталі і подекуди снаряди
перелітають і літять до нас. 

Акація дуже міцна та пружна - тупа сокира відскакує. Спробували включити від
генератора точило - на ньому спрацьовує захист від перевантаження. Запропонував
гострити сокиру на бордюрному камені.

Недолік моєї пилки - надто коротка: хід малий. Недолік іншої - тупа і
нерозведена. Потрібна велика пилка - дворучна. В гаражі є, запропонував піти,
забрати пилку та дрова але хлопці вирішили, що воно того не варто. Тож акацію
пиляли та рубали тим, що було. Буксирувальними линвами нахиляли її, аби не
впала на паркан садка і знову пиляли та рубали. Нарешті її завалили. Хлопці
почали пиляти її на колоди по півметра. Я наполіг, щоби пиляли по півтора
метри: так швидше і можна буде на ніч занести у під'їзди. Майже переконав:
попиляли на метровій цурпалки.

Одну колоду принесли до вогнища бо тополі вологі і маже не горять. Почали
пиляти колоду навпіл. Я згадав дитинство у Кандалакші і, заявивши, що свій
перший палець я мало не відрубав у п'ять років, наполіг, аби її кололи цілком.
. Сергій спробував рубати як звичайно: всадив сокиру і намагався підняти й
вдарити. Я показав як колоти, використовуючи сокиру як колун: Стас тримав
сокиру, а Сергій гамселив по ній іншою. Спочатку йшло не дуже добре і хлопці
вже збиралися знову пиляти, але я переконав продовжити, розповівши, що саме так
вікінги кололи колоди на дошки для своїх кораблів. Данило слухав про вікінгів
мало не з розкритим ротом. Сокира увійшла по обух і застрягла. Я вирубав з
гілки клина і почав вбивати його, аби витягнути сокиру. Хлопці кепкували:
"Палич дерево деревом рубає!". Витягнувши сокиру, я продовжив. Нарешті колода
тріснула й розвалилася навпіл. Наталчин Сергій: "Неуки! Не сперечайтеся з
освіченою людиною. Данило! Вчися, бо будеш такий як я!". Сергій (що роздавав
чай) взявся колоти (не рубати!) половинки, а потім так захопився, що
роздягнувся до футболки і переколов усі колоди.

\ii{14_03_2023.fb.kipcharskij_viktor.mariupol.1.r_k_tomu__den_19__14.pic.2}

Я взяв із багажника складаний стілець і сидячи - "ніяк не відвикну від сидячої
роботи" -  рубав гілки. З підвійного дому молоді чоловік та жінка прийшли
розбирати "наш павільйон". Ми не дали. Пришельці почали тиснути на жалість:
"Нам дітей годувати треба". "То шукайте дрова!'. Жінка: "Як я буду тягати
дрова?. "А що, у вашому будинку чоловіків нема? Ми усім домом шукаємо дрова і
всі разом готуємо, а не кожний сам по собі!"

Павільйон відстояли. (Вже після нашого від'їзду у нього влучили міна і
розвалила...).

Ми з сином пішли по воду до гаражів. Після того, як вчора були "прильоти" по
приватному сектору, відпускати його одного не хотів, хоча це було дурістю: аби
виїхати, має бути хоч один водій. За 42-м будинком на дорозі все ще лежить
снаряд.

Приходили свати. Розповідали "жахалки" про руйнування. Я в цей час спав -
раніше в цей час не спав, використовуючи світло, але на цей раз заснув посеред
дня. 

Грів їжу на обід. У повітрі літає горіле листя з камишів - горить гирло
Кальчику або Кальміусу, або обох разом.

Повернулася сусідка, яка переходила жити на Правий берег - там дуже часто
прилітає.

Діти з онукою пішли на Нептун за пайком "дітям до шести років". Принесли банку
зеленого о консервованого гороху та банку маринованих огірків. Я похвалив: "От
розумниця онучечка! Принесла дідусеві чим закусити!". Наталчин Сергій: "То я
збігаю!". Онука міцно притисла банку до себе: не віддам!

Данило ходить за мною: "Розкажіть ще щось"...

Невістка пекла на великій пательні лаваші з тіста на кефірі зі спеціями та
рослинною олією.

До вогнища підійшов жінка: пропонувала печінку тріски в обмін на крупи.

Оля з невісткою зварили суп, який заправили копченою ковбасою - оскільки вона
пролежала у теплі більше тижня, вирішили її проварити. Онучка здивувалася: "Суп
з ковбасою!". Ми з нею їли суп з лавашем і маринованими огірками. По обіді грів
воду мити посуд. До вогнища вийшов Сашко Щит, якого нещодавно несли до дому -
його зустріли оплесками.

Я виніс книжки Спіллейна та Чейза - Стас просив для жінки - Атлантиду для
Данила. Щит взяв Чейза.

Підійшли молоді хлопці з четвертого під'їзду: вони знайшли старий приймач, але
він під "радянські" батарейки. Я знайшов у "закромах"  контейнер на чотири
батарейки формату АА і подовжив на ньому дроти. Де взяти батарейки? З дитячих
іграшок. Хлопці принесли батарейки  ААА. Я показав, як їх встановити за
допомогою харчової фольги. Приймач ожив.

Подарував Толику із Запоріжжя "казацьку" люльку - в нього був день народження.

Кажуть, що біля "Квартального" магазину іноді буває зв'язок - мабуть у
госпіталі встановили засоби зв'язку. (Пізніше ми дізналися, що до Маріуполя
кілька разів проривалися вертольоти - вони й привезли Старлінки).

19:10. Гупало дуже гучно, аж дриготіли стіни і кімнатою гуляло повітря.
Стрекочуть автомати.

19:55. Кілька кулеметних черг по 5-6 пострілів. Це мені нагадало, як на
риболовлі на Чермалику по нас стріляли з того берега - там пройшла ротація і
новачки  пристрілювали кулемет. Теж черги по 5-6 пострілів. Так, згадуючи
друзів-рибалок і заснув. 

Фото сина:

1. Бібліотека-барикада на вікні: книги не лише знання, але й захист.

2. Приймач - "брехунець': єдине джерело новин на весь двір.

3. Дим над Азовсталлю.

%\ii{14_03_2023.fb.kipcharskij_viktor.mariupol.1.r_k_tomu__den_19__14.cmt}
