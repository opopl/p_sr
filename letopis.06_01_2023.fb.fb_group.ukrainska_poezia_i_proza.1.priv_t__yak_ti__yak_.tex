%%beginhead 
 
%%file 06_01_2023.fb.fb_group.ukrainska_poezia_i_proza.1.priv_t__yak_ti__yak_
%%parent 06_01_2023
 
%%url https://www.facebook.com/groups/255425469158429/posts/756475109053460
 
%%author_id fb_group.ukrainska_poezia_i_proza
%%date 06_01_2023
 
%%tags 
%%title Привіт! Як ти? Як в тебе справи? Ти не голодний?
 
%%endhead 

\subsection{Привіт! Як ти? Як в тебе справи? Ти не голодний?}
\label{sec:06_01_2023.fb.fb_group.ukrainska_poezia_i_proza.1.priv_t__yak_ti__yak_}

\Purl{https://www.facebook.com/groups/255425469158429/posts/756475109053460}
\ifcmt
 author_begin
   author_id fb_group.ukrainska_poezia_i_proza
 author_end
\fi

Привіт! Як ти? Як в тебе справи? Ти не голодний? Тобі тепло? Як твої побратими
з небесного війська? Сподіваюсь, ви на пухнастих хмаринках...

Пам'ятаєш, як ми зустрілися перший раз? Нам було по 16. Ти мене познайомив зі
своїм другом, бо в тебе була дівчина, але чомусь доля вирішила за нас. Наші
юнацькі зустрічі біля цегельного заводу, високо на горі, звідки ми бачили наше
місто. Пам'ятаєш? Я все пам'ятаю. 18 років...наші мрії, які вони були казкові,
ми мріяли про Андрюшку, ми вже з ним розмовляли, ніби він в нас був😄.

20 років. Ми одружилися. Ти працював на "шинному" заводі, 30 квітня ми з
твоїми співробітниками, після нічної зміни пішли на шашлики, поверталися
додому, і в 12.30, я відчула, що наш синочок хоче народитися🥰. А в 20.30 він
з'явився на світ. Ти з мамою піднявся на 3-й поверх, і вам його показали. Який
він був гарний, пам'ятаєш, найкращий серед всіх дітей. Бо він наш. Перший і
єдиний син❤. І ось ми повноцінна  сім'я. Життя йде вперед. Андрюшенька росте.
Ми і в горі, і в радості. Всяке було, але і ми були разом.  Нас розлучила
відстань. Я поїхала з дитиною за кордон, але він завжди повертався додому, бо
там був ти, його найкращий тато. Наша дитина розривалася на дві країни. Так,
це сталося, ми розлучилися. Нові події, нові сім'ї... Згодом, наша дитина йде
в армію, сам вирішує, що йому так краще, не дивлячись на мої прохання.
Закінчив службу, їде в АТО. Мої ридання знову не допомагають. Плач, відчай,
страх за життя сина. Потім кохання❤. Радіємо всі разом. Поїздки, зустрічі,
знайомства з нареченою. Щастя без меж. В січні, тепер вже дітИ, приїжджають до
нас. Скільки радості. Але вони військові, і мають повертатися на службу. Я не
розуміла, чому цей раз було найболючіше прощання. Материнське серце, воно
відчувало...24 лютого в 5.30 ранку дзвінок від сина:  "Мама, почалася
повномасштабна війна. Ми виїжджаємо" 😭. Перші дні були трохи на зв"язку. Я
знала, що ти на другий день війни вже був поруч з сином. Щоразу, я питала, як
тато? Де він? Ви разом? Він з тобою? Я хвилювалася за вас обох, але я нікому
цього не казала, а мені боліло. В один день смс від сина: "Буду без зв'язку,
їдемо на завдання, батько зі мною". Фуф, легше. Тато поруч. Рахую хвилини,
години, дні. Він вийшов на зв'язок через три дні, в 9.15 ранку,  коротенька
смс-переписка, радію, що живий🙏. В 19.00 дзвінок від нареченої, синочок
поранений. А як Діма? Хлопці? У відповідь-неправда. 

Ти загинув. Ти залишив нашого сина самого. Ти для нього був самий кращий тато.
Як він тепер без тебе? Сам?

Я поспішала. Дуже. Я хотіла попрощатися з тобою, я хотіла з тобою побачитися
в останнє і назавжди. Тобі було всього 44 роки, а я тримала в руках 44
червоні троянди. Час йде вперед,  але пам'ять, спогади ніколи не викарбувати,
не забути. Спи спокійно, оберігай нашого сина з небес. Він заслуговує на
щастя. Слава Героям України. Низький уклін

P.S. Кашура-Масальський Дмитро Васильович. Народився 21 серпня 1977 року,
загинув 24 липня 2022 року, під час виконання бойового завдання. Вічна
пам'ять🙏



