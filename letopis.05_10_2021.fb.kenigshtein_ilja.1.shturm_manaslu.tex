% vim: keymap=russian-jcukenwin
%%beginhead 
 
%%file 05_10_2021.fb.kenigshtein_ilja.1.shturm_manaslu
%%parent 05_10_2021
 
%%url https://www.facebook.com/kenigshtein/posts/4325399104179837
 
%%author_id kenigshtein_ilja
%%date 
 
%%tags alpinizm,gory,manaslu.gora.nepal,nepal
%%title Штурм вершины Манаслу, как это было
 
%%endhead 
 
\subsection{Штурм вершины Манаслу, как это было}
\label{sec:05_10_2021.fb.kenigshtein_ilja.1.shturm_manaslu}
 
\Purl{https://www.facebook.com/kenigshtein/posts/4325399104179837}
\ifcmt
 author_begin
   author_id kenigshtein_ilja
 author_end
\fi

Штурм вершины Манаслу, как это было. 

Накануне, у нас было 3 дня отдыха и подготовки к решающему выходу. Мы с группой
провели это время в базовом лагере. Мой высотный костюм (down suite) и второй
спальник, а также несколько баллонов с кислородом  накануне перенесли в camp 1
и camp 3 соответственно. Планировалось, что мы переходим в camp 1, там спим
ночь, утром переходим в camp 2, там также спим ночь и утром следующего дня
переходим в camp 3, где начинаем готовиться к штурму. 

\ii{05_10_2021.fb.kenigshtein_ilja.1.shturm_manaslu.pic.1}

Штурм начался 27 сентября в 18.00 из camp 3. Часть моих коллег по группе решили
штурмовать из camp 4, который расположен на высоте 7,400 м т.к. ребята
планировали совершить восхождение без кислорода. Но основная часть группы шла
на кислороде, поэтому организаторы Seven Summits решили, что мы стартуем прямо
из camp 3, который расположен на высоте 6,800 м. 

\ii{05_10_2021.fb.kenigshtein_ilja.1.shturm_manaslu.pic.2}

Таким образом, нам необходимо
было набрать почти 1,400 м за раз и это даже по профессиональным альпинистским
меркам много. Объяснялось это тем, что на высоте 7,400 м в camp 4 организм не
восстанавливается в принципе, и нахождение там смертельно опасно. Возвращаться
мы также должны были в camp 3 (а то и ниже), то есть таким образом суммарно
весь выход и возвращение длилось более 20 часов. 

\ii{05_10_2021.fb.kenigshtein_ilja.1.shturm_manaslu.pic.3}

Прямо из лагеря мы вышли уже на кислороде. Сразу расскажу что означает нехватка
кислорода для организма - как это ощущается на практике подъема. Вроде вы
дышите воздухом, и вроде всё ок. Но через 10 минут у вас начинается жёсткая
отдышка и все движения замедляются. Мозг начинает паниковать. Ощущение ни с чем
несравнимо, самый близкий пример - если закутаться в одеяло с головой. Совсем
быстро внутри становится нечем дышать и организм начинает вам об этом сообщать.
Появляется слабость, болит голова, становится темно в глазах. Вот примерно те
ощущения, которые вы испытываете на высоте. Только в отличие от тёплого одеяла,
вокруг вас - холодный и безжизненный воздух, которым невозможно надышаться.
Поэтому многие альпинисты на таких высотах используют кислород в баллонах. Он
подаётся через трубку в маску, которую вы крепите у себя на лице. Дышать
становится легче, но появляется другая проблема - ощущение того, что на вас
надет противогаз. Попробуйте походить с противогазом на лице. А теперь
попробуйте идти с противогазом 20 часов. 

\ii{05_10_2021.fb.kenigshtein_ilja.1.shturm_manaslu.pic.4}

По состоянию на 2008 год, согласно Википедии, Манаслу считалась одним из самых
опасных восьмитысячников мира - после Нанга Парбат, К2 и знаменитой своими
лавинами Аннапурны. Времена изменились, но уровень смертности альпинистов там
всё-равно достаточно высок, и составляет 17,8\% - за историю восхождений. Смерти
в основном связаны с гипотермией, горной болезнью и лавинами, которыми
накрывает высотные лагеря camp 2 и camp 3 - как это, например, произошло в
сентябре 2012 года. Тогда сошедшая со склонов лавина унесла жизни 12
альпинистов. 

\ii{05_10_2021.fb.kenigshtein_ilja.1.shturm_manaslu.pic.5}

Итак, мы вышли. Первые 5 часов штурма - просто ад. Сразу резкий набор высоты,
идём вдоль перил, периодически перещёлкиваясь. Впереди идут люди, сзади идут
люди - все двигаются медленно и равномерно, без остановок, освещая пространство
перед собой налобными фонариками. Вокруг непроглядная тьма, ночь, дикая
горизонтальная метель из снега, рассекающего кожу на лице и руках - если снять
очки, маску или перчатки. Ветер, холод, лёд, высота 7,000 - 7,200 и выше
метров. Дышать тяжело, приспускаешь маску - моментально задышка, короче - ад.
Идём сквозь верхнее облако в надежде поскорее из него выйти и оказаться над
ним. Это первое жесткое испытание, которое резко отличается от всего того, что
мы уже проходили между лагерями ниже. Где-то наверху - camp 4, но до него ещё
далеко, и никто не планирует там останавливаться надолго. Кажется, что это
длится вечность. 

\ii{05_10_2021.fb.kenigshtein_ilja.1.shturm_manaslu.pic.6}

Через какое-то время люди начинают выпадать из шеренги, просто садясь на склон,
упираясь кошками, и сидят не двигаясь, уставившись в одну точку. Никто не
ожидал, что будет настолько жестко. Отдых - это ресурс, энергия, которую пьёшь
глотками. Гора проверяет, тестирует. 

Склон становится круче, мы по прежнему ориентируемся только на перила и на
ведущего гида с фонариком. Появляется раздражение, злость - на всё подряд. Мой
шерпа Дорчи идёт впереди, иду за ним практически вплотную и изучаю задники его
альпинистских ботинок - это хоть немного успокаивает. Ветер не стихает, впереди
- склон за склоном, в который ты практически вгрызаешься зубами. Всё, что может
заледенеть - леденеет, даже брови. 

\ii{05_10_2021.fb.kenigshtein_ilja.1.shturm_manaslu.pic.7}

А потом, внезапно равнина, и первая остановка на чай и воду. Вышли. Смотрю на
часы - 2 часа ночи, высота 7,400 метров. Мертвая зона. Ничто не выживает на
такой высоте. Организм не восстанавливается, и задача любого альпиниста
подняться на вершину, спуститься с неё и выйти за пределы мёртвой зоны прежде,
чем последствия для организма станут необратимыми. 

Но с другой стороны ад закончился, дальше начинается широкий пологий склон,
размером с футбольное поле - только таких полей по дороге к самой высокой точке
- десятки. Ветер стихает, метель заканчивается. Оглядываюсь по сторонам и вижу
вокруг ночь, поля и дорогу вперёд, по которой также как и я бредут люди
группами, освещая свой путь фонариками словно светлячки. Дальше будет легче. 

\ifcmt
  ig https://scontent-frx5-1.xx.fbcdn.net/v/t39.30808-6/244282303_4325392247513856_7276580816302353338_n.jpg?_nc_cat=105&ccb=1-5&_nc_sid=8bfeb9&_nc_ohc=Z4Y-F3LqyJUAX8LuLIS&_nc_ht=scontent-frx5-1.xx&oh=67e927a7c411916590406dd680c4223f&oe=619D84BC
  @width 0.4
  %@wrap \parpic[r]
  @wrap \InsertBoxR{0}
\fi

Дальше стало тяжелее. Мы идём по бескрайним полям, постепенно набирая высоту,
поднимаясь всё выше и выше. Слева темной массой еле видны густые ночные облака.
Начинаю хотеть пить, но выясняется что наша вода во флягах замёрзла. Меняем
кислородный баллон на новый. 

И идём. Долго, медленно, утомительно. Хочется спать, мы в пути уже 9 часов. Жду
рассвет, знаю что с его приходом станет легче. А пока что терплю. Отчасти
помогает адреналин, знаю что впереди, уже совсем скоро - вершина. 

Вижу фонарик, который спускается сверху - кто-то уже был на вершине и идёт
вниз. «Илья, привет! Тебе ещё 2 часа до вершины!» - товарищ, который вышел из
camp 4 раньше, зашёл на вершину и спускается вниз. «Илюх, держи мой изотоник,
правда он вонючий, ну ты в курсе» - и протягивает мне флягу. Жадно пью как
будто это не вода, а сама жизнь. «Вот ещё держи телефон, жене наберёшь с
вершины, как спустишься - отдашь» - и хочется его в этот момент обнять. 

А потом рассвет. Рассвет на 8,000 м наконец даёт понимание где я сейчас
нахожусь. Финальный взлёт Манаслу, гора медленно пускает, появляется 100\%
уверенность что дойду. 

Иду и вою, сжимаю зубы и делаю следующий шаг. Всё неважно уже, главное сейчас -
это цель, она понятна и она впереди. Шаг за шагом, вдох за вдохом, последняя
битва с самим собой. Зрение становится туннельным, как у лошади на глаза
которой надеты шторки. Я больше не Илья, нет больше меня. Есть цель, идея,
миссия. Я есть воплощение пути. Уже не смотрю по сторонам, смотрю вниз - на
свои следующие 3 шага. Раз, два, три. Раз, два, три. 

И потом внезапно вижу перед собой задники чьих-то ботинок. Которые стоят.
Поднимаю голову, вижу перед собой 20-25 человек, стоящих в ряд. Очередь. К
вершине. Вижу вершину. Я дошёл. 

Мысли медленно выравниваются, словно вода с осадком в стеклянном сосуде -
понимаю, что дошёл, и это очередь на сам пик, а стою я на финальном взлёте,
наверху которого небольшая площадка, за которой - 20 метров и вершина, на
которой может поместиться от силы 3-4 человека. Перевожу дыхание, надеваю
наушники, произвольно выбираю трек на своих гарминах. Вижу, что правый наушник
сломан, и работает только если поднять голову. Включаю трек, поднимаю голову и
замираю. 

\ifcmt
  ig https://scontent-frt3-2.xx.fbcdn.net/v/t39.30808-6/244253376_4325392510847163_7975228719863173412_n.jpg?_nc_cat=103&ccb=1-5&_nc_sid=8bfeb9&_nc_ohc=v7vUpu5wmYMAX_vBPQO&_nc_ht=scontent-frt3-2.xx&oh=61541bc8bab31b21754ec068c0d17572&oe=619C76DA
  @width 0.4
  %@wrap \parpic[r]
  @wrap \InsertBoxR{0}
\fi

Это Daniel Hart - A Love Adventure из фильма Light Of My Life. Один трек. И нет
ничего более грандиозного, прекрасного и подходящего - просто нет. Я шёл сквозь
ночь, метель, холод, превозмогая усталость, боль и жажду, чтобы прийти сюда и
услышать этот трек. Это - моё посвящение Манаслу. Именно те две минуты, которые
мне так были нужны. 

Потом была вершина. Мы снимали себя, записывали видео. Я развернул флаг
Creative States. Я позвонил своей жене, услышал её сонный голос, её
«возвращайся домой» и мне стало очень тепло на душе и спокойно на сердце,
появились силы для спуска вниз. 

Вершина - это как та комната в «Сталкере». Внезапно пропадают ветер, усталость
и отдышка. Это мистика, но дышится без кислородной маски легко и свободно. И
говорится тоже. То был короткий перерыв, полный счастья и энергии. Мы сидели и
болтали, смеялись и подшучивали друг над другом. Знаете, такой спокойный,
уверенный, ироничный мужской разговор. На высоте 8,163 м. А потом встали,
подняли рюкзаки и пошли вниз. 

Примерно через час после начала спуска, мой шерпа мне сообщил: «Босс, кислород
закончился. Новый баллон в camp 4, в полутора часах ходьбы, так что придётся
потерпеть». Уже ярко светило солнце. Я увидел, как впереди меня двое шерпов
тащат тело умершего альпиниста. Позже я узнал, что это был канадец, который
умер на высоте 7,800 м в результате инсульта. В момент смерти ему было 37 лет. 

Моя дорога в camp 3 была долгой и изнурительной. Мы решили не спускаться в camp
2 и остановиться на отдых в camp 3, так как силы были на исходе. Помню, как
последние 500 метров к лагерю я шёл 1 час. Чтобы прийти, выпить чаю, залезть в
спальный мешок и отключиться на 16 часов. Я не уставал так никогда в жизни. Но
я это сделал - поднялся на вершину Манаслу. И спустился вниз - живым и
невредимым. 

А дальше уже было просто. Утром мы спустились в camp 2, потом в camp 1 и
перешли в базовый лагерь. Я только в базовом лагере понял, где я был. 

Я ещё напишу про своё восхождение. Понимаете, это всё про собственные лимиты,
свои возможности, свой потолок. Но прежде всего, это про те самые две минуты,
которые есть у вас, когда вы находитесь в совершенно немыслимых условиях, при
этом оставаясь наедине с самим собой. Со сломанными наушниками, но с открытым
вершине сердцем. Только так туда можно попасть и вернуться назад. Потому что
поднимаемся мы в горы, но познаём самих себя. 

Посвящаю это восхождение моей жене Lena Kenigshtein. Люблю тебя бесконечно.

\ii{05_10_2021.fb.kenigshtein_ilja.1.shturm_manaslu.cmt}
