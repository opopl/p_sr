% vim: keymap=russian-jcukenwin
%%beginhead 
 
%%file 25_10_2020.sites.ru.nauka_i_zhizn.antropova_anastasia.1.chudo_kotoroje_proizoshlo_1
%%parent 25_10_2020
 
%%url https://www.nkj.ru/open/39683/
 
%%author Антропова, Анастасия
%%author_id antropova_anastasia
%%author_url 
 
%%tags 
%%title Чудо, которое произошло. Часть первая
 
%%endhead 
 
\subsection{Чудо, которое произошло. Часть первая}
\label{sec:25_10_2020.sites.ru.nauka_i_zhizn.antropova_anastasia.1.chudo_kotoroje_proizoshlo_1}
\Purl{https://www.nkj.ru/open/39683/}
\ifcmt
  author_begin
   author_id antropova_anastasia
  author_end
\fi

\begin{leftbar}
        \bfseries Как применить научные знания в творческой профессии? Что
        делать, чтобы приблизиться к детской мечте? Зачем учёным «таймлапсы» и
        как грибы могут привести на телевидение? Об этом и о многом другом
        рассказывает Илья Долгов, видеооператор живой природы, выпускник
        биофака НГПУ и автор блога \verb|@bbc_na_kolenke| в Инстаграм.
\end{leftbar}

\ifcmt
pic https://www.nkj.ru/upload/medialibrary/963/963d8a42554671c7ebc2983257c02027.jpeg
\fi

\textbf{Илья, расскажите, пожалуйста, что вас мотивировало поступить на биофак и почему
вы не продолжили научную карьеру?}

Мне с детства нравилось наблюдать за окружающим миром. Ещё в школе, когда
учился в 8 классе, я попал в Институт систематики и экологии животных (ИСиЭЖ СО
РАН – прим. ред.), где занимался систематикой пилильщиков и даже был в
соавторстве в нескольких скучнейших для обывателя научных статьях. Поэтому на
биофак Новосибирского пединститута я поступил исключительно «по любви», а не от
безысходности.

Но биолог из меня не получился. Классическая фундаментальная наука оказалась
для меня скучноватым занятием. И к тому же, если честно, научная и околонаучная
среда очень и очень специфична. Как и любое закрытое сообщество, оно очень
тесное, затхлое, если уместно так говорить. Зачастую в ней находятся люди
определенного склада. Очень многие (не все, конечно) замкнутые и
некоммуникабельные. В общем, человеческий фактор взаимодействия с учёными
людьми вызывал у меня некоторые сложности.

\textbf{Почему вы решили заниматься съёмками природы?}

Мне хотелось приблизиться к детской мечте: путешествовать и снимать передачи о
природе. А каждая съёмка – это особое исследование. Я часто узнаю что-то новое
не из книг, не из интернета, не из чьих-то уст, а наблюдая за какими-то
процессами. Меня просто удивляет то, что я вижу… Я ещё не утратил способность
удивляться. В моей работе это часто случается и очень вдохновляет. 

\textbf{И как складывался ваш путь от детской мечты к её профессиональному воплощению?}

К мечте привела череда случайностей, в общем-то, как и всегда в моей жизни.
После института я сорвался и переехал вслед за своей девушкой в Москву. Но ехал
я именно за девушкой, а не покорять столицу. Заниматься достигаторством – не
моё. Правда, ещё в школе у меня был период, когда я кайфовал от того, что всё
успеваю – хорошо учиться, заниматься спортом, что есть успехи по всем
направлениям. Но потом у меня в голове резко что-то щёлкнуло. Когда я поступил
в институт, то вдруг понял, что можно спокойно жить и получать удовольствие, не
достигая каких-то сверхцелей. Успешность – не самоцель. 

И я поехал в Москву по принципу «почему бы и нет?». Примерно год прожил в
столице без особо интересных событий, пока мне не позвонил мой друг-востоковед
по образованию, который переехал в Китай. Он какое-то время поработал
барменом, тогда как раз нашёл новое место работы и предложил мне своё место в
баре. Я уволился, купил билет и уже через месяц ровно разливал водочку на
границе Китая и Гонконга.

Я задержался там на полгода, и это был максимально благополучный период жизни:
постоянная не пыльная работа, вкусная еда, квартира на 32 этаже небоскрёба с
видом на Гонконг и залив. В общем, красота, живи – не хочу! Но мне стало
скучно, это было совершенно не творческое занятие, неинтересное для меня. Для
меня стало открытием, что, как бы классно и интересно не было за границей, нас
там никто не ждёт, если, конечно, ты не какой-то матёрый физик-ядерщик или
молекулярный биолог. Есть такая иллюзия, что сейчас мы всё бросим, уедем
куда-нибудь подальше, поменяем свою лыжню, и у нас всё наладится. Нет, не
налаживается. Каким бы крутым ты ни был, сложно куда-то попасть с улицы. «Не о
таком театре мы мечтали». Поэтому я всё бросил и вернулся. 

\ifcmt
pic https://www.nkj.ru/upload/medialibrary/f00/f0055306678ba29c4741591fa4f5f783.jpg
\fi

После Китая я работал в океанариуме. Это был первый большой океанариум в
Москве на Дмитровке в РИО (торговый центр – прим. ред.). Сначала я работал
экскурсоводом, а потом перевёлся в отдел ихтиологии ухаживать за рыбками:
кормить и иногда их лечить (моя дипломная работа как раз была связана с
физиологией рыб). К нам всё время приезжали снимать телепрограммы. Съёмочных
групп было настолько много, что все уже от них шарахались. Меня выловили в
коридоре и отправили рассказывать что-то про наш океанариум. Я обрадовался,
что могу не работать какое-то время, и провёл экскурсию оператору, который
снимал передачу.

Параллельно мы болтали о жизни, и я его расспрашивал о том, как он стал снимать
передачи о природе. Эта кухня мне была совершенно непонятна. Впоследствии мы
стали друзьями, но ещё в нашу первую встречу он мне дал такой совет: «Если ты
увлекаешься фотографией, научись снимать таймлапсы» – то есть ускоренную съемку
очень медленных процессов. Тогда это только входило в моду, просто потому, что,
имея очень малые ресурсы (фотоаппарат, штатив, объектив), можно было получать
красивые кадры кинематографического качества. Лично для меня это был самый
простой способ попасть в операторскую профессию.

И я купил какую-то совсем дешёвенькую камеру на Авито, пульт и... начал
снимать. Стартанул я с классики: облака плывут над городом. Потом подумал, что
можно было бы сделать что-то поинтереснее, и выкопал маленькие грибочки на
газоне за домом, направил на них камеру и дома под столом снял, как они растут
(идею мне подкинули Би-би-сишники, в фильме которых «Планета Земля» это было
очень подробно и красиво снято). Я показал результат своему другу-оператору,
тот похвалил, сказал: «молодец, здорово». Но тут история не кончается.

Однажды у меня было свободное время и я захотел пересечься с тем оператором, но
он сказал, что едет «сливать» материал после съемки, и предложил поехать с ним.
Я поехал. Выяснилось, что направлялись мы на детский канал «Карусель». Мой друг
презентовал меня своим коллегам так: «Это мой товарищ, он снимает грибы».

«Что за грибы?» – спросили они меня. Я им показал свой ролик.

«Ты ещё можешь что-то так снять?»

«Конечно! – убедительно соврал я, – каждый день это делаю!»

А через пару дней мне позвонили и предложили снимать для детского канала
программы о том, как всякие грибочки растут. Тут уже пригодилось, что я учитель
биологии: я сам писал сценарии для этих программ. Бюджет у программы был не
очень большой – и я стал ещё и ведущим на своём проекте. Так что меня привела к
мечте череда совпадений. Обыкновенное чудо, которое произошло. 

\textbf{Были у вас какие-то особенные трудности в работе для телевидения?}

Мне как новоиспечённому сценаристу было тяжело работать с телевизионными
редакторами. Я пишу простым языком текст, но его потом переделывают.
Литературно он становится красивее, но смысл искажается. Приходится объяснять
редакторам, что, хоть их текст и интересно будет слушать, информация стала
недостоверной и нужно это исправить.

Я стремлюсь к тому, чтобы у передачи был образовательный элемент (пединститут,
против этого не попрёшь), а не только развлекательный, но телевидение с этим
как будто бы борется. Нужно всё максимально упрощать. У неразвлекательных
программ рейтинги ниже, меньшая аудитория захочет смотреть передачу, если в ней
будут умничать. Что-то образовательное – это больше интернет-формат для очень
ограниченной аудитории. 

\textbf{Ваше образование биолога-учителя помогло вам в подготовке познавательных
программ. А мог бы пригодиться ваш опыт работы для телевидения в научной
деятельности?}

У меня совсем недавно состоялся разговор с моим знакомым ботаником Михаилом
Серебряным, который работает в ботаническом саду в Москве. Он подкинул
интересную идею. Есть растение, которое в простонародье называют огоньки –
купальница. Есть похожие виды, у которых по-разному идёт процесс увядания
цветков и это может служить важным отличительным признаком. Если получится
снять, как увядают цветки разных видов, то можно было бы доказать, что
особенности увядания цветка – важный систематический признак. 

\ifcmt
pic https://www.nkj.ru/upload/medialibrary/01e/01e593c1387b76eb1acaa7342d7dc7b1.jpg
\fi

Во время работы я очень много взаимодействую с учёными, но для меня в новинку,
что мои навыки можно применить так. Когда показываю свои видео исследователям,
чтобы уточнить, что же конкретно я наснимал (когда моих знаний не хватает
определить это на глаз), они, кроме ответа на мой вопрос, говорят все в один
голос: «Мы всегда знали, как то или иное явление происходит, но никогда прежде
этого не видели!» Было бы интересно иллюстрировать науку – объединить
телевидение и науку, эти полярные миры.

Я могу разговаривать с учёными более-менее на одном языке. Учёные стараются
уходить от конкретных формулировок (чем больше ты знаешь, тем больше у тебя
сомнений), а телевизионщики рубят сплеча, стремятся к конкретике, стремятся всё
упростить. Поэтому традиционно одни других сильно недолюбливают. Было бы
неплохо стать связующим звеном, мостиком между научным миром и обывательским. 

\textbf{Для каких телевизионных передач вы готовили видеосюжеты?}

На канале «Карусель» у меня сразу же была авторская программа
«Видимое-невидимое» про малозаметные чудеса природы: ускоренная макросъёмка,
всё, что не видно глазом. И вроде бы всё шло хорошо: рейтинги были неплохие
(дети после мультфильмов не переключали канал). Но руководство канала пришло к
выводу, что 13 минут для ребёнка – это многовато. Тем более что аудитория – это
совсем мелкие ребята. Программу сократили до 5 минут и убрали ведущего, что
даже упростило задачу. Съёмочная группа была теперь не нужна, материал для
такой программы я мог добывать в одиночку. Сначала придумывал тему, находил
место и снимал. Появилась возможность снимать за границей. Когда мне было
скучно, я выбирался в какую-нибудь прекрасную тропическую страну и снимал
зарисовки, а после писал к ним сценарий.

Программа просуществовала два года, мы выпустили примерно 60 серий. И потом
из-за специфики детского телевидения нам сказали: «горшочек, не вари». Этих
программ, с учётом выпусков первого проекта, хватает на пару лет, затем малыши
вырастают и перестают смотреть канал, приходят новые телезрители, и можно опять
показывать эти передачи с самого начала.

Ещё удалось поснимать для популярной телепередачи «Диалоги о животных» (правда,
длилось это недолго). Кто бы мог подумать: в детстве я взахлёб смотрел
программы Ивана Затевахина, а потом вдруг получилось там поработать! Если бы
мне мелкому кто-то сказал, что так сложится, я бы не поверил! 

\ifcmt
pic https://www.nkj.ru/upload/medialibrary/f28/f285f21e743d553bcab49ea619912170.jpg
\fi

Сейчас я работаю на телеканале «Живая планета». В проекте «Царство грибов» был
оператором (снимал исключительно таймлапсы о том, как растут грибы). И теперь
мы запускаем новый проект с моим участием, немного похожий на
«Видимое-невидимое», только для аудитории постарше. Так что есть где
разгуляться, можно будет поумничать в кадре.

\textbf{Есть какие-то особенности в подготовке программ о живой природе?}

На производство программы закладывается очень мало времени. У съёмочной группы
есть несколько дней на работу, а это чертовски мало. Пытаешься снять что-то
интересное, но приходится рассчитывать на повезет – не повезёт, успеешь – не
успеешь. Так работают на нашем телевидении, а Би-би-сишники, например, делают
всё иначе.

Однажды я почти приблизился к юношеской мечте – поснимать для Би-би-си. Пару
лет назад мне позвонил мой приятель – Саша Семёнов (морской биолог и прекрасный
подводный фотограф) – и позвал на Белое море снимать морских ангелов. Он должен
был погружаться в воду, а я снимать их (морских ангелов) в аквариуме во всех
подробностях.

Выяснилось, что это нужно было для большого фильма Би-би-си о северных морях.
Они вышли на Сашу в интернете, посмотрели его работы и предложили снять видео.
Би-би-си выискивает людей по всему миру с опытом необходимых съёмок. В итоге
они набрали несколько съёмочных групп, разбросанных по всему миру. Мы,
например, снимали на Белом море. Очень мне понравилось то, что можно было
сконцентрироваться на процессе и получить необходимый результат.

У нас же всё делают по старинке. Собираются режиссёр, сценарист и оператор,
берут ведущего и выезжают делать программу. Самый главный ресурс – это время,
которого всегда в обрез. В том же Би-би-си понимают, что с первого раза может
не получится снять то, что нужно. Взять, к примеру, сюжет о том, как
размножаются морские окуни. Караулить их придётся очень долго, но это настолько
быстрый процесс, что можно просто не успеть снять. Поэтому в случае неудачи
съемочная группа едет на следующий год снова снимать окуней: вопрос времени не
стоит так остро, как у нас.

Из-за коронакарантина мы недосняли тот сюжет о морских ангелах для Би-би-си:
биостанцию закрыли, и мы не смогли закончить съёмки. Но я сильно не
расстроился, думаю, всё ещё впереди. 

\ifcmt
pic https://www.nkj.ru/upload/medialibrary/35f/35ffb61b9a74820c4525adb5faae9681.jpg
\fi

Конец первой части.\href{https://www.nkj.ru/open/39690/}{Вторая часть.}

Автор: \textbf{Анастасия Антропова}

Источник: \textbf{«Наука и жизнь»} (nkj.ru) 
