% vim: keymap=russian-jcukenwin
%%beginhead 
 
%%file 31_01_2022.fb.uljanov_anatolij.1.internet_podsrekateli
%%parent 31_01_2022
 
%%url https://www.facebook.com/dadakinder/posts/5208846459134475
 
%%author_id uljanov_anatolij
%%date 
 
%%tags internet,obschestvo,ukraina,vojna
%%title Интернет-подстрекателям нужна большая кровавая война в Украине
 
%%endhead 
 
\subsection{Интернет-подстрекателям нужна большая кровавая война в Украине}
\label{sec:31_01_2022.fb.uljanov_anatolij.1.internet_podsrekateli}
 
\Purl{https://www.facebook.com/dadakinder/posts/5208846459134475}
\ifcmt
 author_begin
   author_id uljanov_anatolij
 author_end
\fi

Есть нечто комичное и жуткое в том, как зашипели на Зеленского клавиатурные
патриоты нашего «гражданского общества». 

\ii{31_01_2022.fb.uljanov_anatolij.1.internet_podsrekateli.pic.1}

Маленькая неоколониальная мразь недовольна тем, что президент Украины возразил
американскому солнцу. Они не понимают, что в этом возражении заключено редкое
мгновение субъектности их государства. Их страна сказала барину попуститься, и
не кошмарить украинскую экономику. В этот момент страна была. 

А им обидно. Обидно, что таким голосом Украина деэскалирует угрозу страданий.
Даёт обществу надежду на мир. И отнимает у интернет-подстрекателей надежду на
их большую кровавую войну. 

Войну, где эти кредитные хомяки будут рвать голыми руками, разумеется, не
агрессора, как нам обещает The Times, а свою крайнюю плоть. Всё это – глядя из
фейсбуков и гастропабов на то, как в бессмысленной мясорубке сгорают их
малоимущие соотечественники. И спуская в тарелки свои рефлексии о трагедии
этого зрелища.

Моча уже хоть чуточку начала остывать, или ещё булькает? Понимают ли господа
разволновавшиеся психопаты, как легко манипулировать человеком в угаре? 

Читая «аналитику», которую выдают наши лидеры грантовых мнений, я понимаю, что
её производят люди, живущие в мире драконов. Рассуждающие о «чувствах Путина»,
дающих оценку его «обсессиям». Описывающие свои чувства по этому поводу… Это –
волшебное сознание церковной свечницы. Поэтика Пачамамы.

Мысля так, можно уверовать в портал, побеседовать с лешим. Я тоже люблю все
прелести этого газа. Но, к сожалению, прогулки по миру в таком состоянии не
позволяют читать ландшафт и быть субъектом реальности; тем более
трансформировать её так, чтобы мы, а не нами.

\ii{31_01_2022.fb.uljanov_anatolij.1.internet_podsrekateli.cmt}
