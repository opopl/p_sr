% vim: keymap=russian-jcukenwin
%%beginhead 
 
%%file introduction
%%parent _main_
 
%%endhead 
 
\section{Introduction}

This package is essentially a replacement---partial or total---for the 
\LaTeX{} macros related with sections---namely titles, headers and 
contents.  The goal is to provide new features unavailable in current 
\LaTeX{}; if you just want a more friendly interface than that of 
standard \LaTeX{} but without changing the way \LaTeX{} works you may 
consider using \textsf{fancyhdr}, by Piet van Oostrum, \textsf{sectsty},
by Rowland McDonnell, and \textsf{tocloft}, by Peter Wilson, which you
can make pretty things with.\footnote{Since the sectioning commands 
are rewritten, their behaviour could be somewhat different 
in some cases.}

Some of the new features provided are:
\begin{itemize}
\item Different classes and ``shapes'' of titles, with tools for very 
fancy formats.  You can define different formats for left and right 
pages, or numbered and unnumbered titles, measure the width of the 
title, add a new section level, use graphics, and many more.  The 
Appendix shows a good deal of examples, so jump forward right now!

\item Headers and footers defined with no |\...mark| intermediates,
and perhaps containing top, first \emph{and} bot marks at the same time.
Top marks correctly synchronized with titles, without 
incompatibilities with the float mechanism. Decorative elements easily
added, including picture environments.

\item Pretty free form contents, with the possibility of grouping 
entries of different levels in a paragraph or changing the format
of entries in the middle of a document.
\end{itemize}
\textsf{Titlesec} works with the standard classes and with many
others, including the AMS ones, and it runs smoothly with
\textsf{hyperref}.\footnote{However, be aware the AMS classes
reimplement the original internal commands.  These changes will be
lost here.  The compatibility with \textsf{hyperref} has been tested
with \textsf{dvips}, \textsf{dvipdfm} and \textsf{pdftex} but it is an
unsupported feature.  Please, check your version of
\textsf{hyperref} is compatible with \textsf{titlesec}.}
Unfortunately, it is not compatible with \textsf{memoir}, which
provides its own tools with a limited subset of the features available
in \textsf{titlesec}.

As usual, load the package in the standard way with 
|\usepackage|.  Then, redefine the sectioning commands with the 
simple, predefined settings (see section ``Quick Reference'') or with 
the provided commands if you want more elaborate formats (see section 
``Advanced Interface.'')  In the latter case, you only need to 
redefine the commands you'll use.  Both methods are available at the
same time, but because |\part| is usually implemented in a
non-standard way, it remains untouched by the simple settings and
should be changed with the help of the ``Advanced Interface.''

