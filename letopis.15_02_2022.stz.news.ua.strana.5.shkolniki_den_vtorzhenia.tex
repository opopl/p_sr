% vim: keymap=russian-jcukenwin
%%beginhead 
 
%%file 15_02_2022.stz.news.ua.strana.5.shkolniki_den_vtorzhenia
%%parent 15_02_2022
 
%%url https://strana.news/news/377015-shkolniki-otmetjat-den-vtorzhenija-v-ukrainu-natsionalnym-himnom-i-v-vyshivankakh-.html
 
%%author_id terehova_ekaterina
%%date 
 
%%tags deti,patriotizm,rossia,shkola,ugroza,ukraina,vtorzhenie
%%title Школьники Украины отметят день "вторжения России" в вышиванках и пением гимна
 
%%endhead 
 
\subsection{Школьники Украины отметят день \enquote{вторжения России} в вышиванках и пением гимна}
\label{sec:15_02_2022.stz.news.ua.strana.5.shkolniki_den_vtorzhenia}
 
\Purl{https://strana.news/news/377015-shkolniki-otmetjat-den-vtorzhenija-v-ukrainu-natsionalnym-himnom-i-v-vyshivankakh-.html}
\ifcmt
 author_begin
   author_id terehova_ekaterina
 author_end
\fi

Украинским родителям приходят смс из школ - настоятельно просят завтра нарядить
детей в вышиванки. Таким образом будут отмечать учрежденный Зеленским День
единения.

Во время занятий дети будут слушать гимн и смотреть патриотическое кино, исходя
из рекомендаций Минобразования.

\ii{15_02_2022.stz.news.ua.strana.5.shkolniki_den_vtorzhenia.pic.1}

Киевляне сообщили \enquote{Стране} о получении от администрации школы сообщения такого
содержания:

\enquote{Уважаемые родители. Завтра праздник День единства. В 10.00 ВСЯ школа будет
исполнять гимн. Есть ли возможность прийти в школу в вышиванках?}.

Или вот ещё пришло такое смс: \enquote{Добрий день. Шановні колеги, вийшов Указ
Перезидента України. 16 лютого - День єднання. Завтра приходимо у вишиванках. О
10.00 лунатиме гімн України}.

Напомним, Зеленский объявил \enquote{день вторжения} 16 февраля Днем единства (по
случаю распускаемых западными СМИ слухах о том, что в этот день вторгнется
Россия). По словам Зеленского, по всей стране будут вывешены сине-желтые
национальные флаги, а люди наденут ленты под цвета флага и \enquote{покажут всему миру
свое единство}.

А в пятницу, 18 февраля, во всех школах страны по инициативе Минкульта
заведения пройдет урок медиаграмотности.

\enquote{Медиаграмотность – это превентивное оружие против дезинформации и жизненный
необходимый навык, который нужно развивать с детства. Урок призван привлечь
внимание молодого поколения к развитию критического мышления. Вместе с
учителями школьники проговорят способы распознавания фейков, виды манипуляций в
сети и рассмотрят процессы создания новостей}, — говорится в анонсе.

\enquote{Прихожу домой, а там уже рыба, шуба, оливье}. Нардепы рассказали, как будут
праздновать Новый год
