% vim: keymap=russian-jcukenwin
%%beginhead 
 
%%file slova.druzhba
%%parent slova
 
%%url 
 
%%author 
%%author_id 
%%author_url 
 
%%tags 
%%title 
 
%%endhead 
\chapter{Дружба}
\label{sec:slova.druzhba}

%%%cit
%%%cit_head
%%%cit_pic
%%%cit_text
У проросійському болоті велика заруба. Чубляться Портнов/Лукаш і Шарій. Ще
вчора вони були найліпшими \emph{друзями}, а сьогодні вже один одного тваринками
називають. Що ж сталося? Там всього намішано. І президентські амбіції
Портнова, і \enquote{туалетна історія} Шарія, і бажання їхніх учорашних спільних \emph{друзів}
вислужитись перед Зеленським, десь і Медведчук є на радарах
%%%cit_comment
%%%cit_title
\citTitle{Шарія кинули всі. Втече до Москви?}, Олексій Братущак, blogs.pravda.com.ua, 20.06.2021
%%%endcit

%%%cit
%%%cit_head
%%%cit_pic
\ifcmt
  tab_begin cols=2
		width 0.4
     pic https://strana.ua/img/forall/u/0/36/8870e29b30de20a65403b30c30b4b37560dc7956d45f7795881761.jpg
		width 0.46

     pic https://strana.ua/img/forall/u/0/36/2021-07-01_09h35_48.png
  tab_end
\fi
%%%cit_text
Подробности нападения. 29 июня на матч Украина-Швеция, который завершился
выходом украинской сборной в четвертьфинал Евро-2020, пришли не только
болельщики команд-соперниц, но и российский фанат Денис Смоленский вместе со
своими \emph{друзьями} - украинцем и поляком. Трое \emph{товарищей} выбрали
яркие образы, чтобы показать приверженность к разным странам, но в то же время
единство.  Россиянин выбрал одежду с российской символикой, которая отличала
его на \enquote{сине-желтом} секторе - он был в красной футболке сборной
России, с флагом РФ и в шапке-ушанке.  И, как видно по снимкам, в начале матча
украинские болельщики спокойно реагировали на болельщика, укутанного в триколор
%%%cit_comment
%%%cit_title
\citTitle{На Евро 2020 избили российского фаната Дениса Смоленского, что известно об инциденте}, 
Анна Копытько, strana.ua, 01.07.2021
%%%endcit
