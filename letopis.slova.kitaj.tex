% vim: keymap=russian-jcukenwin
%%beginhead 
 
%%file slova.kitaj
%%parent slova
 
%%url 
 
%%author 
%%author_id 
%%author_url 
 
%%tags 
%%title 
 
%%endhead 
\chapter{Китай}

%%%cit
%%%cit_pic
%%%cit_text
Современные украинские политики, объявившие в 1991 году независимость УССР от
СССР, тоже не придумали ничего нового, как назвать новую страну Украиной, то
есть Окраиной. При этом они называют страну великой, совершенно забывая о том,
что, прежде всего, державу, даже не великую, характеризует в первую очередь
название. Вот посмотрите на названия ведущих стран мира: GB (Великая
Британия), США (Соединенные Штаты Америки), РФ (Российская Федерация), КНР
(\emph{Китайская Народная Республика}), даже Гондурас и Зимбабве официально называются
Республиками. И только Украина как была Окраиной, так ей и осталась даже в
названии, невзирая даже на то, что некогда ее территория была основой довольно
мощного государства – Киевской Руси
%%%cit_title
\citTitle{Почему современную Украину назвали «окраиной», а не более престижно - Киевской Русью?}, 
Исторический Понедельник, zen.yandex.ru, 22.02.2021 
%%%endcit

