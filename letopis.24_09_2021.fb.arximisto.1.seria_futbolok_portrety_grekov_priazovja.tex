%%beginhead 
 
%%file 24_09_2021.fb.arximisto.1.seria_futbolok_portrety_grekov_priazovja
%%parent 24_09_2021
 
%%url https://www.facebook.com/arximisto/posts/pfbid0JxSdMEzS41UcbrojVAmXnjrFU5k9taztaWsKnwjRFbLo2sYtrdWudWUbE77tTgVbl
 
%%author_id arximisto
%%date 24_09_2021
 
%%tags 
%%title Авторская серия футболок с портретами выдающихся греков Приазовья
 
%%endhead 

\subsection{Авторская серия футболок с портретами выдающихся греков Приазовья}
\label{sec:24_09_2021.fb.arximisto.1.seria_futbolok_portrety_grekov_priazovja}

\Purl{https://www.facebook.com/arximisto/posts/pfbid0JxSdMEzS41UcbrojVAmXnjrFU5k9taztaWsKnwjRFbLo2sYtrdWudWUbE77tTgVbl}
\ifcmt
 author_begin
   author_id arximisto
 author_end
\fi

\enquote{Архи-Город} инициировал авторскую серию футболок с портретами выдающихся греков Приазовья.

\#новости\_архи\_города

К фестивалю греческой культуры Мега Йорты ГО \enquote{Архи-Город} вместе с
мариупольской художницей Дарьей Галько запустил авторскую серию футболок с
портретами выдающихся греков Приазовья. Первой стала футболка с портретом
Архипа Куинджи, знаменитого мариупольского художника, \enquote{мастера света}. Средства
от продажи будут направлены на создание веб-портала Мариупольского Светлого
Некрополя. Партнер проекта – культурно-ту\hyp{}ристический центр \enquote{Вежа}.

Наш проект – это дань уважения к вкладу приазовских греков в историю и культуру
Мариуполя, Приазовья и Украины. Многие из них оказали значительное влияние на
развитие всей мировой культуры, как сообщил Андрей Марусов, директор ГО
\enquote{Архи-Город}.

Проект также нацелен на сохранение и продвижение историко-культурного наследия
Мариуполя и Приазовья. Часть средств от продажи футболок будут направлены на
создание веб-портала Мариупольского Светлого Некрополя, который хранит память,
в том числе, о греках-первопоселенцах XVIII века.

Для меня участие в проекте — это возможность создать новые образы выдающихся
греков Приазовья: яркие, современные, запоминающиеся. Это позволит привлечь
внимание к уже известным личностям, таким как Архип Иванович Куинджи. И
познакомит всех желающих с другими, практически неизвестными, но не менее
значимыми людьми, роль которых долгие годы сознательно замалчивалась, как
отметила художница Daria Galko.  

Я уже завершила портреты живописца Архипа Куинджи, Феоктиста Хартахая, который
инициировал открытие первых гимназий в Мариуполе, и Георгия Костоправа —
основателя литературы приазовских греков на румейском языке, который был
репрессирован в 1938 году...

Мы надеемся на продолжение проекта и приглашаем всех художников к
переосмыслению образов приазовских греков и их культурного наследия, как заявил
А. Марусов.

В настоящее время выпущена лимитированная серия футболок с портретом Архипа
Куинджи. Футболки с портретами Феоктиста Хартахая и Георгия Костоправа
изготавливаются по отдельному заказу.

Квк и где приобрести футболки? 

Цветные футболки с образом Архипа Куинджи (надписи на украинском и греческом)
будут продаваться с субботы, 25-го сентября, на Мега Йорты (парк им. Н. Гурова)
и в культурно-туристи\hyp{}ческом центре \enquote{Вежа}.

Их также можно заказать с помощью мессенджера \enquote{Архи-Горо\hyp{}да}, по вайберу
(096 463 69 88), указанному телефону или по e-mail arximisto@gmail.com.

Стоимость – 400 грн. (универсальные размеры S, M, L, XL). Доставка наложенным
платежом по всей Украине через \enquote{Нову пошту} (пожалуйста, указывайте
размер, номер мобильного, адрес или отделение \enquote{Нової пошти}). 

Футболки с образами Георгия Костоправа и Феоктиста Хартахая, а также
черно-белая футболка с образом Архипа Куинджи продаются по отдельному заказу.
Стоимость – 500 грн.

Мы признательны каждому покупателю и покупательнице за поддержку наших усилий
по продвижению историко-культурного наследия Мариуполя и Приазовья!

%\ii{24_09_2021.fb.arximisto.1.seria_futbolok_portrety_grekov_priazovja.cmt}
