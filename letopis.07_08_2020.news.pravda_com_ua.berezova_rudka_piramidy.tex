% vim: keymap=russian-jcukenwin
%%beginhead 
 
%%file 07_08_2020.news.pravda_com_ua.berezova_rudka_piramidy
%%parent 07_08_2020
 
%%endhead 
\subsection{Піраміда, таємне кохання Шевченка та червона Мата Харі і віцепрем'єр Британії. Чому потрібно побувати в Березовій Рудці на Полтавщині}
\label{sec:07_08_2020.news.pravda_com_ua.berezova_rudka_piramidy}
\url{https://www.pravda.com.ua/articles/2020/08/7/7262066/}
  
\vspace{0.5cm}
 {\ifDEBUG\small\LaTeX~section: \verb|07_08_2020.news.pravda_com_ua.berezova_rudka_piramidy| project: \verb|letopis| rootid: \verb|p_saintrussia| \fi}
\vspace{0.5cm}

Від Києва до Каїра понад 3 600 кілометрів. До села Березова Рудка всього 150.

У Рудки з Каїром є дещо спільне --- в цьому невеличкому селі в 30 кілометрах від
Пирятина і в 213 від Полтави, знаходиться піраміда. Одна з двох пірамід
України. Друга --- теж на Полтавщині, в Комендантівці.

Побудована в 1899-м дев'ятиметрова усипальниця --- найдивніший і найпопулярніший
туристичний об'єкт Березової Рудки.

Вибравши правильний ракурс, тут можна зробити фото для соцмереж з жартівливою
геолокацією "Єгипет". В епоху пандемії COVID-19 і закритих кордонів --- не
найгірший варіант для нових емоцій.

Утім піраміда --- далеко не єдина пам'ятка села. Тут знаходиться колишній маєток
знаменитого козацького роду Закревських. Він зберігає чимало таємниць, трагедій
минулого і любовних інтриг.

Саме в Березовій Рудці Євген Гребінка написав легендарні "Очі чорні". А в 1921
році в цьому селі народився Дмитро Луценко, автор неофіційного гімну столиці
"Як тебе не любити, Києве мій!".

Чим закінчилося кохання Тараса Шевченка до заміжньої жінки, як покарали Гната
Закревського, котрий назвав Росію "варварською", який зв'язок між Березовою
Рудкою та колишнім віцепрем'єром Британії Ніком Клеггом і
письменником-соцреалістом Максимом Горьким --- репортаж УП з Полтавщини.

\subsubsection{зеркало і бичок}

Чорні джинси, темно-синя футболка і сіро-зелені кросівки --- репортера УП
відображає велике дзеркало, в яке 170-175 років тому вдивлявся сам Тарас
Шевченко.

Витонченості потужному предмету побуту 19 століття надає різьблена горіхова
оправа.

Експозицію історико-краєзнавчого музею розмістили у флігелі, в якому гостював
Шевченко Всі фото: Українська правда

Всю поверхню венеціанського скла, від лівого кута навскоси до низу, розрізає
тріщина. Історія про те, як вона з'явилася, сповнена іронії з нальотом смутку.
І дуже схожа на притчу, яка закликає українців не повторювати помилки минулого.

--- Після революції 1917 року буквально за одну ніч розграбували маєток
Закревських, --- розповідає Валентина Гончар. --- Один із селян потягнув додому це
дзеркало. Але воно було завеликим, щоб пролізти в двері хатини.

Він вирішив поставити його в сінник. А там бичок, побачивши своє відображення,
дзеркало стукнув. Так і вийшла тріщина.

Валентина Гончар --- директорка і по суті єдина штатна співробітниця, на якій
тримається історико-краєзнавчий музей в Березовій Рудці. 

Його відкрили в 1974-му на території колишнього палацу легендарного сімейства
Закревських, який з 1929-го займає аграрний технікум.

У музеї Валентина Гончар працює останні двадцять років

Валентина родом з Харківської області. В Рудці опинилася в 1966-му, багато
років працювала бібліотекаркою, а потім очолила музей.

Історією про дзеркало і бичка вона ділиться, присівши на ковану лавку, що
стоїть поруч. Таких було багато в тутешньому розкішному саду понад століття
тому.

--- З 70-х минулого століття люди почали здавати в музей все, що в них було
цінного з маєтку, --- говорить директорка. --- По селу, на подвір’ях залишилося ще
багато (предметів з палацу Закревських --- УП). Але зараз люди не дуже вже і
віддають.

Валентина Гончар показує залишки масивних меблів з дуба, які вдалося зберегти.
І підкреслює, що на них теж міг сидіти Тарас Шевченко.

--- Подивіться, яке там різьблення --- вгорі лев'ячі голови, спинка різна і ніжки
які, --- по-дитячому захоплюється вона. --- Коли палац розграбували, Закревських
тут вже не було. Вони поїхали від селянських бунтів та руйнування.

Вони залишили в маєтку керуючого. А землі віддали в оренду бельгійцям. До
розграбування тут був великий господарчий двір: стайня з породистими рисаками,
багато рогатої худоби та п’ять тисяч гусей.
  
\subsubsection{Серцевий напад в Каїрі}

Біля типового сільського магазину в Березовій Рудці розташована її головна
визначна пам'ятка ---  піраміда Закревських висотою дев'ять метрів. Але у
місцевих вона зацікавлення не викликає.

До дивної для тутешніх країв будівлі кілька поколінь жителів Березової Рудки
давно звикли. Настільки, що тут був і склад молокозаводу після Другої світової,
й імпровізований туалет, і звалище.

На місці покинутої будівлі сільського клубу раніше була дерев'яна церква

Сьогодні об'єкт теж у занепаді. Хоча крізь зачинені грати на вході видно, що
його намагаються хоч якось реставрувати.

Громадський туалет сільського типу тепер знаходиться в декількох метрах. За ним
--- кладовище.

Піраміду в 1899-му році побудував Гнат Закревський, який бував у Єгипті та
марив всім, що було з ним пов'язано. До того ж був масоном, для яких трикутник
грав символічну роль в зображенні Всевидячого ока.

Гнат Закревський, який побудував піраміду, був одним з найвідоміших
представників відомого козацького роду

Подібних пірамід, які були родовою усипальницею й каплицею, в Європі всього
три. Причому дві з них в Україні, в Полтавській області: в селі Комендантівка і
в Березовій Рудці. Третя --- в Римі.

Піраміда виглядала екзотично поруч з православною церквою, яка була тут раніше

У 1801 році в Рудці збудували Свято-Троїцьку церкву, біля якої ховали
Закревських.

--- Гнат Платонович вирішив закрити родинний склеп пірамідою-каплицею. 

В неї було три поверхи, в тому числі підземний. У самій піраміді --- хрест,
невеличкий вівтар, --- розповідає Валентина Гончар.

Усередині піраміду розписали малюнками на біблійну тематику та фресками в
єгипетському стилі. Світло в приміщення проникало через чотири вікна з
красивими ажурними гратами.

Піраміду Закревських розграбували після революції 1917 року

Цегляні стіни залили бетоном, розділивши візуально на блоки так, що можна було
подумати, ніби вони облицьовані мармуровими плитами.

У будівництві використовували цеглу з місцевого заводу, яким володіли
Закревські.

Охороняла піраміду статуя богині Ісіди, створена, за деякими даними, в другому
тисячолітті до нашої ери. Її разом з іншим антикваріатом Гнат придбав під час
однієї з подорожей до Єгипту.

Помер Гнат Закревський у 1906-му році, в Єгипті. Піраміда в Березовій Рудці
стала йому в нагоді. Як стародавньому фараону.  

--- Його підводило серце, --- говорить Гончар. --- Поїхав в Каїр, щоб підлікуватися.
До того ж узяв двох доньок, які були якраз на виданні. Єгипет був у той час
дуже модним. Там було багато відомих та заможних людей. 

Дрон дозволяє зняти піраміду без зайвих радянських будівель, які з'явилися біля
неї в 20 столітті

---  Він вирішив, що, можливо, знайдеться підходяща партія для доньок, ---
продовжує Гончар. ---  Але там у нього стався серцевий напад. Його тіло,
забальзамоване в меду, доньки перевезуть в Березову Рудку, і воно знайде
спочинок у ним же збудованій піраміді.

Проте ненадовго. Через кілька років, на хвилі революції, усипальницю
пограбують. Останки Закревських хаотично перепоховають на сусідньому кладовищі.

\subsubsection{Замість шляхти --- агрономи}

Село Березова Рудка в 1717 році заснував гетьман Іван Скоропадський. Він
подарував мальовничі землі своїй дружині Насті.

\emph{Біля палацу Закревських залишилася частина саду, в якому відпочивала полтавська
аристократія}

У 1752-му Рудку викупив Осип Закревський, який в 1756-му став генеральним
бунчужним Війська Запорізького. Дружиною Осипа була сестра гетьмана Кирила
Розумовського Ганна, це допомогло зробити Закревському кар'єру.

Облаштував родинний маєток у Березовій Рудці син Осипа Григорій.

\emph{Після революції 1917 року палац Закревських "націоналізували" і перетворили в технікум}

Центральним у садибі був двоповерховий палац у класичному стилі. На його місці після пожежі в 1838-му побудували новий, який сьогодні показують туристам.

\emph{Гостьовий флігель, де зупинявся Шевченко, зроблений в стилі скромного класицизму}

Стомлена спекою, Валентина Гончар веде гостей музею стежками минулого. Періодично зупиняється.

\emph{Сьогодні сходи в палаці Закревських пофарбовані, як багато під'їздів радянських багатоповерхівок}

Перевівши дух, жінка похилого віку із захопленням розповідає про веранди, які
не збереглися. Про фріз і ліпнину, які прикрашали фасад садиби.

--- Через цей маєток пройшло п’ять поколінь Закревських, з 1752 по 1917 роки, ---
зауважує она. --- Два флігелі для гостей, в одному з яких зупинявся Тарас
Григорович Шевченко, збудовані у 1800 році.

В палаці проживали самі господарі.

Піднімаючись сходами, якими ходив Шевченко, Валентина Гончар розповідає про
пишні бали Закревських. Грав оркестр, шаруділи сукні прекрасних дам, стукали по
паркету низькі підбори їхніх залицяльників.

\emph{У бальній залі збирався вищий світ Полтавщини. Тут танцювали і влаштовували театральні вистави}

Там, де була танцювальна зала площею 230 квадратних метрів, сьогодні приміщення, де навчають агрономів.

--- Зверніть увагу на сходинки, які збереглися з того часу, та на орнаменти.
Пелюстки розкритих квітів, недоплетений вінок. 

Цей вхід до зали освітлювали два великих арочних вікна. А там був вихід на
оглядову терасу, з якої прекрасно проглядалося все подвір’я з фонтаном у
центрі.

Висота стелі в кожній кімнаті палаца п’ять метрів. А в бальній залі, де були
п’ять великих вікон та колони, --- вісім, --- ділиться Гончар з усіма, хто заїжджає
до неї в гості.

\subsubsection{Ніколи ти не здавалася мені такою гарно-молодою}

Тарас Шевченко гостював у Березовій Рудці кілька разів у різні роки.

Крім природи та тиші, його вабила пристрасть до Ганни Закревської, молодої
дружини Платона Олексійовича Закревського.

\emph{Шевченко часто згадував про Ганну Закревської у засланні. Присвятив їй кілька віршів}

Кажуть, у Шевченка і Ганни навіть була дитина, але ця інтригуюча версія не
підтверджена документально.

--- Оці два портрети (Платона і Ганни --- УП) написав Шевченко, --- показує
репродукції директорка музею. --- Ганна написана з великою любов’ю, це найкращий
жіночий портрет Шевченка. На жодному не було такого життя очей, такого
світлого, ніжного, промовистого погляду. 

\emph{Портрети подружжя Закревських Тарас Шевченко написав під час відпочинку в їхньому маєтку в 1843 році}

Критики відзначають різницю в тому, як зобразив Шевченко Ганну та її чоловіка.
У холодному і трохи гордовитому погляді Платона відчувається антипатія автора.

Закревська не була щасливою в заміжжі, тому і звернула увагу на молодого поета
і художника.

--- Вона заміж вийшла зовсім юною, 17-річною, не по своїй волі, --- пояснює
Валентина Гончар. ---  Батько Ганни служив разом з її майбутнім чоловіком,
Платоном Олексійовичем. Одного разу він поїхав провідати свого товариша й
побачив молоду, дуже красиву доньку. 

А на другий раз приїхав вже просити руки. Уявіть собі, батько дав згоду, знаючи
статки Платона, знаючи, що багатющий. Думав, раз багатий, то буде щасливою. Але
ж щастя не сталося. В них була вікова різниця --- 21 рік. 

Вперше Тарас Шевченко і Ганна Закревська зустрілися на балу у поміщиці
Волховської в селі Мойсіївка Черкаської області. 

Їй було 21, йому --- 29. У неї вже було двоє дітей, серед яких той самий Гнат,
який побудує піраміду в кінці 19 століття.

\emph{Палац в Березовій Рудці наочно показує, як жила українська шляхта в 19 столітті}

За словами Гончар, обставини зустрічі були романтичними. Тарасу дали право
відкривати вечір, а Ганна була королевою балу.

Платон Закревський здогадувався про роман молодої дружини з поетом, але вирішив
не виносити сміття з хати.

Мине кілька років, Шевченко опиниться у засланні, де напише вірш, адресований
Г.З. --- Ганні Закревській, своїй таємничій коханій з Березової Рудки:

Немає гірше, як в неволі

\begin{multicols}{2}
	\obeycr
	\restorecr
Про волю згадувать. А я

Про тебе, воленько моя,

Оце нагадую. Ніколи

Ти не здавалася мені

Такою гарно-молодою

і прехорошою такою,

Так, як тепер на чужині,

Та ще й в неволі. Доле! Доле!

Моя ти співаная воле!
\end{multicols}

\subsubsection{Незаконнонароджена, баронеса і віце-прем'єр Британії}

Історія роду Закревських --- наче клубок дорогої нитки. Розплутуючи його,
допитливий дослідник може виткати яскраве полотно.

Гнат Закревський, який побудував піраміду в центральній Україні, був відомим
юристом. Працював суддею в Петербурзі  та Варшаві, прокурором у Харкові та
Казані. Дослужився до сенатора.

--- Обер-прокурор першого департаменту правлячого сенату --- так називалася його
посада, --- уточнює Валентина Гончар. --- Його кар'єра дуже стрімко підіймалася, і
так само стрімко впала. Він написав сміливу, дуже критичну статтю в англійську
"Times" по справі Дрейфуса. 

Всі прогресивні люди виступили на захист Дрейфуса. Гнат Платонович написав: "Як
це так, що країна (Франція --- УП), яка називає себе самою передовою й
прогресивною в Європі, допустила таку погрішність у своєму судочинстві?". І
додав: "Можливо, через те, що вона подружилася з варварською Росією?".

Після такого випаду Закревський потрапив у немилість царя.

У Гната були чотири доньки і син. 

\emph{До революції 1917 року палац Закревських і сад навколо були комфортними для життя}

Як каже Валентина Гончар, про існування четвертої доньки, Дарини, дізналися в
2004-му, коли в Рудку з Люксембургу приїхала її онучка --- Колетт Артіч.
Випускниця Сорбонни, перекладачка з п'яти мов.

Дарину народила проста селянка Тетяна Іваненко, яка працювала на Закревських.

--- Це була родинна таємниця. І навіть діти між собою гралися, і ніхто ні разу не
сказав, що вона їхня сестра. Вона була просто як вихованка. Її пізніше вивезли
в Швейцарію, влаштували в пансіонат на навчання, --- розказує директорка музею.

Доля ще однієї дочки Гната, Марії Закревської-Бенкендорф-Будберг, гідна
екранізації. Про неї складають легенди, і приписують роботу відразу на три
розвідки --- СРСР, Британії та Німеччини.

\emph{Максим Горький присвятив Марії Закревській свій останній роман "Життя Клима Самгіна"}

Марію Закревську прозвали "червоною Мата Харі". Вона двічі була заміжня за титулованими особами.

Закревська працювала секретаркою Максима Горького і стала його коханою,
провівши до смертного одра. А потім жила в любові з Гербертом Уеллсом, до самої
смерті фантаста.

Баронеса Марія Закревська-Бенкендорф-Будберг померла в 1974-му в італійській
Тоскані. Її поховали в Лондоні.

Привід згадати червону Мата Харі з'явився в 2010-му. Тоді заступником
прем'єр-міністра Великобританії став ліберал Нік Клегг, якого британські ЗМІ
називають двоюрідним правнуком баронеси.

\subsubsection{``Лише час боїться пірамід''}

Сьогодні нащадки Закревських розкидані по Європі. По можливості вони
допомагають зберегти родовий маєток у Березовій Рудці. В українському бюджеті
на такі об'єкти грошей традиційно немає.

Чотири останніх зими будівлю музею не опалюють, і його директорці Валентині
Гончар доводиться не солодко. З останніх сил вона оберігає уламки минулого.

\emph{За допомогою самоварів у Березовій Рудці заварювали чай з місцевих трав}

Залишки посуду, бронзові кавник і молочник, частина канделябра --- німі свідки життя козацької родини, яку перемололи дві імперії. Але не змогли остаточно стерти пам'ять про неї.

--- Є таке арабське прислів’я, --- говорить Гончар. --- Все на світі боїться часу, лише час боїться пірамід.

Незважаючи на руйнування і наругу, піраміда в Березовій Рудці вистояла, щоб підтвердити цю східну мудрість.

У 20-х роках минулого століття в старовинній Свято-Троїцької церкві влаштували сільський клуб. Валентина Гончар згадує, як після пожежі на його місці побудували новий.

У розграбованій піраміді поставили дизельний генератор, який давав електрику. У 70-х її хотіли знести. Але з якоїсь причини не вийшло.

--- Просто, мабуть, бог зберіг, --- припускає Гончар.

Колишній клуб, який побудували на місці церкви, тепер зяє очницями вибитих вікон. Відгукується хрускотом битого скла та порожніми пляшками, якими вистелені його коридори і кімнати.

\emph{У музеї Березової Рудки чекають закінчення пандемії і туристів}

Сьогодні, проводжаючи нечисельних гостей на порозі будинку, в якому жив Тарас
Шевченко, Валентина шкодує тільки про одне.

--- Музей та історія, мабуть, --- це те, з чого мені треба було починати. На жаль,
так сталося, що це прийшло до мене вже в кінці мого життя. Але це те, що дає
мені наснагу. 

Мені подобається спілкуватися з людьми та розповідати про минуле. А воно у нас
дуже цікаве, --- каже вона на прощання, і, поки в неї ще є час, запрошує всіх до
Березової Рудки.

Євген Руденко, Ельдар Сарахман, УП
