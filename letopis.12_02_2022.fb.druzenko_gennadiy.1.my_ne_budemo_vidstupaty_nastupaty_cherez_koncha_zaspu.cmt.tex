% vim: keymap=russian-jcukenwin
%%beginhead 
 
%%file 12_02_2022.fb.druzenko_gennadiy.1.my_ne_budemo_vidstupaty_nastupaty_cherez_koncha_zaspu.cmt
%%parent 12_02_2022.fb.druzenko_gennadiy.1.my_ne_budemo_vidstupaty_nastupaty_cherez_koncha_zaspu
 
%%url 
 
%%author_id 
%%date 
 
%%tags 
%%title 
 
%%endhead 
\zzSecCmt

\begin{itemize} % {
\iusr{Андрей Юрьевич}

Однозначно одновременно с вторжением одичалых надо раз и навсегда покончить с
пророссийскими упырями в Козине, Пуще-водице Горенке и на берегах Киевского
водохранилища. Чтоб в спину не стреляли. И тут то есть возможность министрам соц
политики Реве, Соколовой, Лазебной, руководству Пенсионного фонда припомнить
все мытарства военных пенсионеров ... От себя добавлю, что это уже не их
волосы, а мои скальпы.

\iusr{Станислав Краснов}
\textbf{Андрей Юрьевич} я з тобою на всі 100!)) і не тільки я

\iusr{Ola Hnatiuk}
Оце супер стратегія оборони - закликати до зведення прорахунків.

\begin{itemize} % {
\iusr{Андрей Юрьевич}
\textbf{Ola Hnatiuk} 

для особо тупых поясняю: ЧТО Б НАМ В СПИНУ НЕ СТРЕЛЯЛИ ! Хотя, кому я поясняю,
безликому существу сидящему в Варшаве  @igg{fbicon.smile} 

\iusr{Gennadiy Druzenko}
\textbf{Ola Hnatiuk} 

за всієї поваги, а що Ви знаєте про російсько-украінську війну не з книжок і не
з преси?

Скільки Ваших близьких загибли через (у кращому разі) злочинну (без)діяльність
української влади?

І де Ви побачили «стратегію» в цьому дописі?

\iusr{Ola Hnatiuk}
\textbf{Андрей Юрьевич} дуже Ви вже ввічливі, Пане Андрію. Вітання з Києва.

\iusr{Ola Hnatiuk}
\textbf{Gennadiy Druzenko} ясно, жінки, діти і риби голосу не мають.

\iusr{Андрей Юрьевич}
\textbf{Ola Hnatiuk} Вы определитесь, откуда, потом приветствуйте.

\iusr{Gennadiy Druzenko}
\textbf{Ola Hnatiuk} 

це не відповідь. Ми ж люди з університетського середовища. І мали б апелювати
до раціо навіть в ситуації, коли емоціо природно вирує.

Я описав, які відчуття та передчуття переповнюють ту частину ветеранської
спільноти, з якою я спілкуюсь і до якої належу. Я навів історичний приклад,
коли формально легітимна влада делегімітмзувала себе колаборацією з ворогом. Я
апелював до критичної недовіри населення до влади, яку фіксують усі соціологи.
Де Ви побачили заклик зводити порахунки - для мене загадка. До речі, @Андрей
Юрьевич приймав бій під Слов’янськом, після якого почалось АТО. «Спіймав» три
кулі чудом вижив. Тепер роками судиться з Пенсійним фондом за свою військову
пенсію. І навіть вигравши всі суди, не отримує її. Як і ветерани ПДМШ
заслуженого статусу УБД. На що заслуговує така влада?

\iusr{Ola Hnatiuk}
\textbf{Gennadiy Druzenko} 

Пане Геннадію, дозвольте, що у відповідь перефразую Ваш посил до мене: «при
всій повазі до Ваших заслуг», це згубна стратегія випинати власні заслуги і
картати всіх інших. Бо не йдеться про Вашу, мою чи ще багатьох інших критичну
оцінку дій влади. Я не люблю пафосу, тому не напишу, про що йдеться. І
впевнена, що для Вас це слово так само святе.

\iusr{Ola Hnatiuk}
\textbf{Gennadiy Druzenko} 

Пане Геннадію, дозвольте, що у відповідь перефразую Ваш посил до мене: «при
всій повазі до Ваших заслуг», це згубна стратегія випинати власні заслуги і
картати всіх інших. Бо не йдеться про Вашу, мою чи ще багатьох інших критичну
оцінку дій влади. Я не люблю пафосу, тому не напишу, про що йдеться. І
впевнена, що для Вас це слово так само святе.


\end{itemize} % }

\iusr{Ольга Михайлова}

Ця логіка мені нагадує логіку петлюрівців зразка листопада 1918: мовляв,
знесемо спочатку владу багатіїв, уособлену режимом Скоропадського, а потім з
народним урядом зможемо на довірі здійснити спротив російській агресії. Але
десь за пару місяців, отримавши чоботи і зброю, переважна більшість вояків
розбрелися по домах. Отаманщина вона така.

\iusr{Микола Рудаков}

Взагалі, якась дивна політика МИРУ. Усі вимагають від України виконання якихось
мінських угод (написаних під градом \enquote{Градів}), забувши про те, що московія напала
на Україну, порушивши всі міжнародні договори...

\iusr{Сергей Дацюк}

І в чому стратегія? Перетворити визвольну антиросійську війну на громадянську
війну в Україні? Так в цьому план Путіна і полягає.

\iusr{Станислав Краснов}

Заголовок дуже сподобався))

Чесно кажучі в своїх колах ще не чув, всі готуються до відбиття нападу на
кордоні/підходах до Києва. Обов‘язково озвучу концепцію )?

\iusr{Богдан Панкевич}

8 років тому після розстрілів на Майдані у Львові була Ніч Гніву. Потім майже
три тижні у місті не було поліції, а ми самі патрулювали і забезпечували
порядок. На автомобілях, велосипедах і пішки. Не було мародерів, а злочинність
зменшилася в рази. Тобто у разі нападу Росії ми знаємо як забезпечити порядок в
тилу. Я особисто охороняв тоді Генконсульство Росії від гарячих голів
провокаторів. Знаючи, що там сидять вороги. Фейкова тероборона чи ні, але
добре, що почали створювати. Всі, хто не поїде зустрічати ворога на схід будуть
самотужки контролювати свої міста і села.

\iusr{Sergii Kostezh}

На карте одесситы нападают на непризнанное приднестровье?)))) зачем?)))) кто
это рисовал?))))

\iusr{Olexandr Yaretskiy}
\textbf{Sergii Kostezh} 

это не одесситы, а морской десант. И не нападают, а пробивают коридор, рассекая
территорию Украины.

\iusr{Rostyslav Vorobiov}
Конча Заспа не пострадает при любом раскладе, даже в случае ядерной войны.))

\iusr{Аліна Подолянка}

Мінусова довіра громадян до політичного проводу? Ну довіра громадян до армії
висока. А армія існує не сама по собі. Нею керує міністр Оборони і решта,
призначені саме політичним проводом.

\iusr{Андрей Гарнат}
Знадто занирювати у чужу гру - от то є насправді небезпечно.

\end{itemize} % }
