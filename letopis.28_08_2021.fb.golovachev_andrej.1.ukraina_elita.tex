% vim: keymap=russian-jcukenwin
%%beginhead 
 
%%file 28_08_2021.fb.golovachev_andrej.1.ukraina_elita
%%parent 28_08_2021
 
%%url https://www.facebook.com/permalink.php?story_fbid=2687077508268677&id=100008993618796
 
%%author Головачев, Андрей
%%author_id golovachev_andrej
%%author_url 
 
%%tags elita,future,gosudarstvo,obschestvo,ukraina
%%title Откуда   возмется новая элита?
 
%%endhead 
 
\subsection{Откуда   возмется новая элита?}
\label{sec:28_08_2021.fb.golovachev_andrej.1.ukraina_elita}
 
\Purl{https://www.facebook.com/permalink.php?story_fbid=2687077508268677&id=100008993618796}
\ifcmt
 author_begin
   author_id golovachev_andrej
 author_end
\fi

Откуда   возмется новая элита?

Итак, большинство  моих подписчиков  более или менее согласилось  с тем ,  что
невозможно ничего  изменить в стране , пока к власти не придет принципиально
новая элита,  реально  обладающая  демократическим  мировоззрением.    Эта
элита должна предложить обществу и сама продемонстрировать своим примером
принципиально   новый тип отношений между  государством и гражданами, т е.
другой тип нравственности.

Нынешняя элита уже ни на что не способна и в этом смысле любые выборы это
просто потеря времени, потому  что все претенденты   являются плоть от плоти из
нынешней элиты , все  обладают криминальным сознанием и  по этому причине
наивно ожидать от них каких либо улучшений. Наоборот,  будет еще хуже!

Но откуда возьмется   эта новая элита?  Кто ее допустит к власти ? 

Тут надо сразу сказать, что вопрос прихода новой элиты к власти  - это событие
строго случайное. Повезет -придет, не повезет - не придет.   Если сложатся
благоприятные условия, то новая элита может прийти к власти завтра, а может -
через 50 лет, а может  и никогда.   В истории нет ничего заранее
определенного, не существует никаких  объективных законов развития, нет  какой
то исторической траектории, вдоль которой якобы двигаются все страны- кто
впереди , а кто сзади , но все, якобы  в одном направлении, вслед за Западом.
Это все химеры - история абсолютна случайна  и глобальное потепление может
вызвать такие социальные потрясения и войны, которые полностью изменят все
политические режимы в мире, как когда то их не раз меняли надвигающиеся ледники
.

Но все таки, каков  мог бы быть   механизм прихода к власти новых элит?   Как
это в принципе может произойти?

В любом обществе есть так называемая базовая культура и есть маргинальные
культуры -флуктуации.   Наша базовая культура сейчас  традиционно -воровская.
Носителями этой так называемой сельской системы ценностей является сейчас где
то 80-85\% населения.  Именно по этой причине у власти и находится
клептократическая элита- общество не протестует, потому что само разделяет их
систему ценностей. Люди не осуждают свою элиту, а скорее завидуют ей: вот
умеют, мол , люди жить!   И по этому причине   общество продолжает голосовать
за нынешнюю элиту. Ну, в самом деле почему люди голосуют , например, за таких
патентованных воров, как Левочкин, Бойко, Рабинович, Порошенко, Тимошенко и
т.д.?

Но внутри нас есть такие себе " первые христиане" , маргиналы-чудаки,
которые являются носителями  другой ,перспективной западной  культуры,
которые презирают воров и коррупционеров и не допускают по моральным
соображениям такое поведение для себя. Их мало- 5-7\%, но они есть и их
количество медленно растет за счет культурного  воздействия на Украину  со
стороны Запада

Если в результате какого то социального конфликта,  волнений , майдана,
революции и т.д.   в общественный разлом сможет прорваться такая перспективная
маргинальная культура и станет доминирующей, базовой , то тогда Украина получит
шансы на построение цивилизованного и преуспевающего общества.   Если нет, то
нет.  

Итак,  все дело в уникальных благоприятных условиях, которые должны сложиться в
удачный пазл в результате социальных катаклизмов. Например, таким образом
к власти  совершенно  непрогнозируемо пришел Саакашвили и  привел  с собой
группу людей, которые предложили обществу совершенно другой тип отношений.  Это
было чисто случайное событие. Михо мог и не прийти к власти и  Грузия до сих
пор  могла бы быть такой какой она была  как при Шеварднадзе, т.е. воровским
государством с нищим населением.

Мирным эволюционным  путем, с помощью демократических выборов и т.д
принципиально новая элита прийти к власти не в состоянии.  Совершенно
бесполезно также пытаться искусственно  стимулировать революцию или майдан
чтобы облегчить приход новой элиты.  Вместо здорового ребеночка получиться
выкидыш.  Это все может произойти  только абсолютно спонтанно, непрогнозируемо,
как , например, пришел к власти в Армении сейчас Пашинян.  В Румынии начались
массовые протесты против коррупционной власти - румыны отчаялись что либо
поменять к лучшему с помощью выборов.   Посмотрим к чему это приведет.

Для того чтобы что-то  поменялось  в Украине в лучшую сторону, должна смениться
базовая культура общества , а культуры меняются только в результате социальной
революции, которые приводят к власти  новую элиту, которая быстро внедряет
новую нравственную систему ценностей , время для которой уже пришло.

Будем надеется, что Украине повезет и новая элита все таки придет к власти в
ближайшие годы,  потому что любое  криминальное  государство, окруженное  по
периметру более сильными в экономическом  отношении соседями, долго не живет.
Разорвут.

\ii{28_08_2021.fb.golovachev_andrej.1.ukraina_elita.cmt}
