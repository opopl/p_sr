% vim: keymap=russian-jcukenwin
%%beginhead 
 
%%file 12_06_2018.fb.lesev_igor.1.nacia_idiotov.cmt
%%parent 12_06_2018.fb.lesev_igor.1.nacia_idiotov
 
%%url 
 
%%author_id 
%%date 
 
%%tags 
%%title 
 
%%endhead 
\subsubsection{Коментарі}

\begin{itemize} % {
\iusr{Станислав Бочкур}

Ну, когда в сегодняшний день смотреть стыдно, а в завтрашний - страшно, только
и остаётся, что мриять о грядущих перемогах лет через 10-15, по принципу "а
вдруг?"

\iusr{Сергей Куприянов}

Подозреваю, что если на Кубани какого-нибудь Петренко назвать украинцем, то
можно отгрести и в морду. (рассуждения по поводу картинки).

\begin{itemize} % {
\iusr{Станислав Бочкур}
Откуда взяться украинофобии на Кубани?

\iusr{Сергей Куприянов}
Дело не украинофобии, а в идентификации себя, как русского, особенно на фоне событий на Украине.

\iusr{Станислав Бочкур}
\textbf{Сергей Куприянов} О, нет. Многие на Кубани и не только идентифицируют себя как украинцы. Но не как жёлтоголубисты, конечно.

\iusr{Сергей Куприянов}
\textbf{Станислав Бочкур} На этнографическом уровне, не более.

\iusr{Станислав Бочкур}
\textbf{Сергей Куприянов} Ага. А другого и не надо.

\iusr{Таня Кушнир}

Вы были на Кубани?  @igg{fbicon.smile}  уверена, что все строго бы наоборот случилось. Этот
кубанский Петренко бы вам радостно рассказал, сколько у него в роду украинцев,
кто где воевал и кем он гордится. Они простые и наивные. И совершенно
естественно быть русским и с дедом-бабкой украинцами.

\end{itemize} % }

\iusr{Иван Данилов}

Так вся политика властей Украины после 91 года основана на этом. Простому
народу впаривается национализм, а элита под шумок дерибанит советское
наследство

\iusr{Надежда Топоровская}
Вернет, когда попадет в состав нового союза.

\iusr{Михайло Бойченко}
Между прочим русские тоже мечтали вернуть крьім полвека... Дождались

\begin{itemize} % {
\iusr{Матвей Кублицкий}
только до этого русские им двести лет владели и кровушки за него пролили не меряно

\iusr{Михайло Бойченко}
Кто им только не владел и не мечтал! Думаю, Турция и сейчас не против его вернуть @igg{fbicon.wink} 

\iusr{Станислав Бочкур}
\textbf{Михайло Бойченко} Кто бы ещё спрашивал ту Турцию...

\iusr{Матвей Кублицкий}
\textbf{Михайло Бойченко} она им практически не владела. Крымское ханство просто присягнуло османам, сохранив в какой-то мере свободу

\iusr{Михайло Бойченко}
Обьічно забирают без спроса @igg{fbicon.wink} 

\iusr{Станислав Бочкур}
\textbf{Михайло Бойченко} Таким мы традиционно говорим "молон лабе"...

\iusr{Михайло Бойченко}
Ну вот, не хотел ханов вспоминать... А они думаете не хотят? @igg{fbicon.wink} 

\iusr{Станислав Бочкур}
\textbf{Михайло Бойченко} Увы, такую валюту, как "хотелка", можно обменять разве что на другую "хотелку", посмелее. Нам тоже много чего хочется...

\iusr{Михайло Бойченко}
Поживем - увидим, чья хотелка более живучая... @igg{fbicon.wink} 

\iusr{Матвей Кублицкий}
\textbf{Михайло Бойченко} вот как раз через 15 лет и надо смотреть. Украина перед серьезной проблемой в виде кредитов оказалась. проблема то с годами пройдет, но лет 10 не до развития будет. Тут как раз Рабинович прав - надо сделать так чтоб крымчане сами назад захотели

\iusr{Марина Прохорова}

Русские никогда не считали Крым украинским. Нелепый подарок безумного Хрущёва в
рамках единого Союза никто не воспринимал всерьёз. И даже замена СССР на СНГ
поначалу виделась чистой формальностью. Но потом взыграла нэзалэжнисть, теперь
ещё долго придётся расхлёбывать последствия ((

\iusr{Матвей Кублицкий}
\textbf{Игорь Лесев} нет. не соглашусь. Игорь. тут совпало. если б россия не забрала крым - там бы щас амеровская база была. они эти планы не скрывали

\iusr{Станислав Бочкур}
\textbf{Игорь Лесев} Там была поначалу призрачная надежда, что ваши резко включат заднюю и сделают вид, что переворота никакого не будет - так, встали с ног на голову, да и обратно с головы на ноги. Но - не включили, там уж и нам пришлось заходить по-настоящему.

\iusr{Матвей Кублицкий}
\textbf{Игорь Лесев} согласен со второй частю - КАК это происходило. но база НАТО там - реалные планы. просто щас представь себе там базу НАТО - этоже полный контроль всего черного моря. мечта всех натовцев во все времена. вспомни войну в крыме в середине 19 века...

\iusr{Игорь Лесев}
ну, плюс/минус тоже склоняюсь к такой версии

\iusr{Матвей Кублицкий}
\textbf{Станислав Бочкур} будте любезны по четче писать. из вашего текста куча непонятного

\iusr{Станислав Бочкур}
\textbf{Matvey Kublitsky} Чего именно непонятно?

\iusr{Игорь Лесев}
\textbf{Матвей Кублицкий} не преувеличивай значение Черного моря. Кроме того, Россия последовательно готовилась выходить из Севаса, а значит, и из Крыма. Новороссийск объявили главной базой ЧФ. На Кубани стали строить аналог крымской Нитки. Короче, ушли бы потихоньку, добазарились бы о нейтральном статусе ЮА и никакой трагедии. Но все все ускорили в 14-м

\iusr{Матвей Кублицкий}
\textbf{Игорь Лесев} ну не знай. четкости этих планов не было никогда. наши и давили на цены газа чтоб оставаться в крыму. новороссийск имеет кучу проблем для того чтоб стать базой. порт то ограниченный. а там и гражданский флот и военный. в общем находимся в зоне предположений и домыслов

\iusr{Игорь Лесев}
\textbf{Матвей Кублицкий} это были не планы, а проходила последовательная работа. Строились пирсы, углублялось дно, в Крыму сокращался личный состав, ЧФ не обновлялся, русские РЕАЛЬНО готовились к уходу из Крыма

\iusr{Матвей Кублицкий}
\textbf{Игорь Лесев} ты был в новороссийском порту? я был. он географически ограничен. половину моря выкапывать? опять же дело же не где будет российский флот. проблема - чтоб в крыму не было флота НАТО. Чисто стратегия. какой смысл от порта Новороссийска - если его можно раздолбать малой кровью из крыма

\iusr{Игорь Лесев}
\textbf{Матвей Кублицкий} так и от Севаса какой смысл, есть проливы контролирует НАТО?))

\iusr{Матвей Кублицкий}
опять же. разборки эти международные сколько продлятся то? ну пять лет, десять... потом кому разница будет чей крым? лишь бы отдыхалось комфортно

\iusr{Матвей Кублицкий}
проливы турки контролируют. по особому договору. там много что прописано

\iusr{Матвей Кублицкий}
кстати, менно из-за турок на украину никогда не придет сжиженный газ - вот и нато.

\iusr{Игорь Лесев}
сжиженный газ не придет не из-за турок, а из-за отсутствия бабла на него) это совсем разные вещи

\iusr{Станислав Бочкур}
\textbf{Игорь Лесев} видимо, бабла хватает лишь на купить ребят, которые расскажут, что газа нет из-за турок  @igg{fbicon.wink} 

\iusr{Матвей Кублицкий}
\textbf{Игорь Лесев} я про турецкий запрет на проезд газгольдеров по проливам. но и твоя мысль в тему. денег нет. с другой стороны - нафтогаз показал в прошлом году дикую прибыль в полтора лярда. хотя тут я могу и соврать....

\end{itemize} % }

\iusr{Елена Несветайлова}
Последнее предложение - молодец.)

\begin{itemize} % {
\iusr{Елена Несветайлова}
Так и знала, что так напишешь. То, что выше - обсуждаемо. А вывод = аксиома.

\iusr{Елена Несветайлова}
Импровизируй.) Смелее!)
\end{itemize} % }

\iusr{Евгений Отовчиц}
Вернем. Сперва Донбасс, потом Крым, потом Волынь и Закарпатье, Буковину вернем, Львовщину возвращать будем смело...

\iusr{Евгений Водовозов}

При сильном желании называть украинцев нацией идиотов любой повод хорош, как я
вижу. Не вернётся Крым? Нереально?

На наших глазах произошли более нереальные вещи - распад совка, например. Кто
бы мог подумать, что такое возможно?

Ну а примеров, подобных возврату аннексированному Крыму, можно привести массу.

Судетские немцы спаковали чемоданчики и мирно потопали в Германию.
Аннексированные Латвия, Литва и Эстония сейчас свободные страны. Если копнуть
глубже, то можно вспомнить и Донецко-Криворожскую, почившую в бозе, исполнив
свою функцию. Из того же времени можно вспомнить и Финляндию, ценой Карелии
обретших свободу.

Можно и немного натянутые аналогии - падение Берлинской стены. Чем ГДР не
аннексированная территория?

А тут гляди ещё Кореи объединятся - ваще из области фантастики.

Вроде норм набросал примеров? Только вот идиотов искать не хочется)

Таким образом дураком может оказаться не тот, кто думкою богатие)

\begin{itemize} % {
\iusr{Игорь Лесев}

Все приведенные примеры ни к черту. Судетские немцы – это из главы «Горе
побежденным». Чтобы такое произошло, надо раздолбать силовым методом страну. В
данном случае, Россию. Если вы рассматриваете такой вариант серьезно, не буду
вам мешать думкой богатеть. Прибалтам оставили субъектность в виде нацреспублик
и не их заслуга, что они получили независимость, а рядовых москвичей, которые
остановили ГКЧП. Учите историю. Украина по этому же принципу получила в 1991-м
независимость. К чему вы приплели Донецко-Криворожскую республику, вообще не
пойму. «Финляндия ценой Карелии обретшая свободу» - это из какого украинского
сказочного учебника вы такое выудили? Финляндия получила независимость
благодаря помощи кайзеровской Германии, добровольному уходу разложившейся
царской армии и бездарным действиям красной армии, которой тогда было особо и
не до финнов. Ну а при чем тут Карелия к Финляндии, которая еще с петровских
времен в составе России, понять вас сложно. Пример с ГДР еще более нелепый. А
про объединение Кореи… ну, когда-то может и объединятся, но не через 8 и даже
не через 15 лет. А когда-то там было три Кореи. Всякое бывает. Но даже эти
корявые примеры показывают, что возврат чего-либо к тому как было – случаи
исключительные и практически невозможные.

\iusr{Евгений Водовозов}

Какая разница какой ценой что произошло и чья это заслуга - главное
свершившийся факт. И Крым может вернуться не обязательно силовыми методами, во
всяком случае не со стороны Украины. Совок распался на раз-два, без видимых
силовых методов, а это событие поглобальнее, чем возврат небольшой территории.

Тут дело в другом. Я хочу, чтоб вернулся Крым, и лежу в этом направлении). Вы,
как я понял - нет). Поэтому и аргументация у нас разная.

\iusr{Игорь Лесев}
\textbf{Евгений Водовозов} 

знаете, я хочу стать долларовым миллионером. Думаю, вы тоже. Но один из нас
будет утверждать, что лежать в этом направлении - это путь к успеху. А другой
будет чуть реальнее смотреть на вещи, которые даже и очень желанны. Союз
распался хотя бы потому, что в его конституции это ДОПУСКАЛОСЬ. И вообще, во
всем есть предпосылки и степень вероятности. Куда больше оснований допускать,
что республики Северного Кавказа в случае катаклизмов в РФ обретут
независимость, чем Крым, который более русский чем Белгородская или Воронежская
область, станет снова украинским. Даже если в России к власти придет
Навальный/Явлинский, начнется ельцинская эпоха-2, Запад продавит отказ от ЯО и
РФ начнет распадаться на части, нет НИКАКИХ оснований предполагать, что Крым в
таком случае вернут непременно Украине. Запад бесится не из-за потери Украины,
а из-за геополитического усиления России. Выходите из своих теплых иллюзий и
смотрите на мир чуть реальнее.

\iusr{Андрей Фролов}

если даже Россия потеряет Крым - то его скорее всего подберрет Турция.. она
имено с этой целью и работала с крымтатарами - постепенно выдавить русских с
полуострова, после чего сделать татарское государство под турецким патронажем..
Украине то в любом случае Крыма не видать

\iusr{Евгений Водовозов}
\textbf{Игорь Лесев} Ну, время покажет. Я просто хотел сказать, что в этом мире может произойти что угодно, и не стоит быть столь категоричным в суждениях)

\iusr{Андрей Фролов}
\textbf{Евгений Водовозов} 

в этом мире может произойти что угодно, но извиняйте - слабых грабяти бьют, а
не прносят подарки на блюдечке с голубойц каемочкой ..вечное правило "терпилу
не спрашивают, терпила всем по жизни должен"

\end{itemize} % }

\iusr{Дмитрий Коломийченко}

Два момента создающие эту ситуацию - инфантилизм и двоемыслие. Второе к
сожалению весьма распространено в нашем обществе. Например, когда какие-нибудь
деятели активно выступают против олигархии, но при этом столовываются у
олигарха; демократы, призывающие к репрессиям и т.д.

\begin{itemize} % {
\iusr{Игорь Лесев}

ну, это как раз новояз... типа ЛДПР Жириновского (либералы/демократы). Главное,
первым перетянуть название с подсознательно-позитивным смыслом, и гнать через
него любую пургу

\end{itemize} % }

\iusr{Василий Январев}

Тем не менее в Китае ведь многие верят в возвращение Тайваня, а в Сербии ждут
возвращения Косово и т.д. по списку.

\begin{itemize} % {
\iusr{Игорь Лесев}

Естественно. Ждут/верят/надеются. Только это все в разделе хотелок. Где-то были
предпосылки реальные, как договор с Гонконгом, подписанный на 99 лет.
Дождались, но там была зацепка. Где-то нет вообще никаких предпосылок, кроме
ожидания. Это не означает, что тот же Тайвань когда-то не станет под юрисдикцию
континентального Китая. Возможно, станет. А может и нет. "Может" - ключевое
слово во всех этих планах.

\iusr{Василий Январев}
\textbf{Игорь Лесев} Ну так чем тогда отличаются украинцы от сербов?

\iusr{Игорь Лесев}

как раз от сербов ничем, потому что сербский и украинский вопрос вообще
тупиковый... от китайцев, вами же упомянутых, сильно отличаются. У КНР, по
крайней мере, имеются действенные рычаги для реализации своих хотелок -
национальный суверенитет, растущая военная мощь и крутая экономика. Это не
значит, что они ТОЧНО вернут Тайвань, но это ПРЕДПОЛАГАЕТ такую возможность

\iusr{Василий Январев}
\textbf{Игорь Лесев} 

Да, с Китаем конечно попроще. Но из сабжа можно вытащить и другие примеры
иррациональных хотелок. Греческие хотелки по по поводу Северного Кипра. Войну с
турками греки не потянут. Ну если не хотят повторения 22 года. Индия/Пакистан -
там патовая ситуация. Кроме повторения очередной почти ничейной по результатам
войны (причем, возможно, и ядерной) результата не будет. Так что не одни
украинцы верят в иррациональное. Вообще - вера в иррациональное свойственна
человечеству.

\iusr{Игорь Лесев}
\textbf{Василий Январев} 

Кипр - классическая тупиковая ситуация. Страна в ЕС, но это не стало конфеткой
для Севера. Большая Греция и Турция в НАТО - но это не предотвратило войны. И
все. Без перспектив. Так и живут, пока лет через сколько-то может еще что-то
поменяется. А может и нет. Кстати, на Кипре также пасутся и британские военные
базы-анклавы, которые тоже оттуда не собираются сваливать. И ничего. Все
привыкли.

\end{itemize} % }

\iusr{Дмитрий Коломийченко}

Территориальные изменения отражают объективную суть исторических процессов, а
не чью-то злую или добрую волю. Конечно бывает так, что кому-то удаётся
обмануть историю, но это не надолго. Поэтому объединение Германии произошло
после окончания ХВ, так как само разъединение логикой ХВ и было определено.
Относительно Крыма и Донбасса, Россия может их вернуть Украине на условиях
нынешних киевских властей только в случае тотального геополитического поражения
в конфликте с США. Не с Украиной, а с США. Вот только я не уверен, что США
играют именно в тотальное поражение Москвы. Относительно разделения Украины.
Никакое здравое правительство не будет играть в сепарацию соседнего
государства. Особенно достаточно большого. Новый баланс выстраивается
непредсказуемо, требует ресурсов. Характерный пример Югославия. Сколько лет
прошло, а стабильности нет (Македония, которая не Македония, Косово, БиГ). И
это при наличии такого геополитического магнита как ЕС.

\begin{itemize} % {
\iusr{Игорь Лесев}

в том-то и дело, если гипотетически Россия потерпит сокрушительное поражение,
нет никаких оснований предполагать, что кто-то будет отдельно расписываться за
Украину... получают сильные и наглые, а не скулящие и просящие

\iusr{Дмитрий Коломийченко}
\textbf{Игорь Лесев} 

При всём желании, я не вижу никаких плюсов для Украины от распада России. Если
конечно не считать за плюс злорадство. Проблемы с газом в трубе, углем, потеря
рынка и риск большой войны на границе.

\end{itemize} % }

\end{itemize} % }
