%%beginhead 
 
%%file 07_01_2023.fb.kotikova_oksana.mariupol.1.lyudi__ukra_nts____n
%%parent 07_01_2023
 
%%url https://www.facebook.com/oksana.kotikova/posts/pfbid0d852SXd4RsvGCPMgix7a9iSytSViRKC6mqXx5jmi2hgihDrcQzKWENW3TkQMZ7VTl
 
%%author_id kotikova_oksana.mariupol
%%date 07_01_2023
 
%%tags chelovek
%%title Люди. Українці.  Неймовірні і незламні. Сильні і вільні. Дякую вам за це
 
%%endhead 

\subsection{Люди. Українці.  Неймовірні і незламні. Сильні і вільні. Дякую вам за це}
\label{sec:07_01_2023.fb.kotikova_oksana.mariupol.1.lyudi__ukra_nts____n}

\Purl{https://www.facebook.com/oksana.kotikova/posts/pfbid0d852SXd4RsvGCPMgix7a9iSytSViRKC6mqXx5jmi2hgihDrcQzKWENW3TkQMZ7VTl}
\ifcmt
 author_begin
   author_id kotikova_oksana.mariupol
 author_end
\fi

Люди.

Українці. 

Неймовірні і незламні. Сильні і вільні. Дякую вам за це.


Дякую всім, на чиїх плечах наразі тримається наша незалежність: 

хто боронить її від ворога зовнішнього із зброєю в руках; 

хто виборює її право на власне існування, на її унікальність, обособленність
від ворога на світових політичних аренах; 

волонтерам, про яких в майбутньому складатимуть легенди, як про чарівників та
чарівниць; 

безтрашним медикам, що не маючи ані спокою, ані відпочинку щоденно рятують
життя людей; 

дякую всім, завдяки кому попри непогоду та небезпеку ми маємо світло в наших
домівках та зв’язок з рідними; 

дякую тим, хто не боючись переслідувань продовжує говорити правду, продовжує
боротьбу із ворогом серед нас, страшнішим від зовнішнього, бо він намагається
виглядати, як ми, але не є нами; 

дякую тим українцям, що усвідомили ціну «галочки» в бюлетні; 

дякую всім українцям, що допомагають один одному. 

Дякую тим, завдяки кому і ми з Сашком врятовані. 

Ви щиро раділи, коли ми вийшли з пекла і з’явилися в мережі. Ви підтримали нас
морально і матеріально. Ви піклувалися про наші потреби. Ви мовчали і говорили
вчасно і тими словами, що залікували нам рани. Без вас ми б так і залишалися в
мороку пекла. Бо ми винесли його в своїх душах, серцях, спогадах, уявленях,
баченнях.

Нещодавна я давала інтерв’ю (як завжди прохлопала вухами назву кому, головне -
для західного світу) і мене спитали - де ви знаходите сили? Я навіть не
замислювалася. Звісно ж в людях! В друзях старих і тих, що з’явилися. Важко
вчинити самогубство, коли Господь підтримує тебе руками стількох людей.

Тож так, завдяки вам всім ми врятовані. Ми пройшли ту стежку мороку, тому що ви
кликали нас і підсвічували нам той шлях. І я розумію, що не всім так пощастило.
І від того боляче за них.

Я дуже хвилююсь зараз, коли пишу цей допис. Я пишу його вже майже тиждень. Тому
що я боюсь, що я когось не згадаю, комусь не подякую. Тому, якщо вразі дійсно
таке трапиться, прошу - не тримайте на мене образи. Просто пам’ятайте, що якусь
частину часу по виході ми не були у свідомості, все для нас було, наче в
тумані. Я майже нічого не пам’ятаю про перші три - чотири місяці. Сподіваюсь,
що Сашко мені допоможе.

Якби просторово-часові рамки дозволили, я б згадала вас всіх водночас, тому що
кожен з вас має своє місце в моєму серці. Але в цьому світи є і час, і є
простір. І чудом є, що кожен з вас з’являвся саме у той потрібний час, коли
відчай душив зсередини, хватав нас за руки і витягував на світло. Я ніколи не
зможу віддячити вам тим само, але я щиро бажаю, щоби ви такого ніколи не
потребували! Щоби ваше життя було настільки нудно стабильним, що за фантазіями
як себе розважити та як собою зайнятися ви йшли до царини мистецтва!

Пані Оксано - уклінно дякуємо за наданий нам прихисток. Ваша оселя, наповнена
особливим затишком. Тут відчувається, що тут любили щиро один одного і Україну.
Стіни вашого дома вилікували та наповнили нас силами, повернули здатність
мріяти та бажання діяти. 

Наталочко, без перебільшень, завдяки тобі ми не стали одноокими піратами 🙂
Хоча Сашко вперто стверджує, що я б була шармАн з пов’язочкою і одного ока мені
цілком би вистачало виразно глядіти. ))) А потім ти ще й подарувала нам світло
)))) І втілення моєї дурної мрії - ванна при свічах по всіх кутках. І читати
або в’язати я тепер можу коли захочу. Дякую. Дай Боже наступні свята вашій
родині зустрічати у вашій вже новій оселі в родинному колі і якщо і плакати,то
від щастя, що все це нарешті скінчилося.

Лануся, сонячний промінчику, дякуємо тобі за організовану тобою спецоперацію
«врятувати  рядових котиків», яку ти розгорнула серед дівчат на весь фейсбук 🙂
Я не знаю, здається я ніколи тобі цього не казала, але коли ми нарешті зарядили
телефони і спіймали зв’язок, ми побачили, що по суті влада робить потужні пуки
ротом, а рятуванням людей займаються волонтери. Це був відчай. Він був
настільки сильним і підсилювався ще й тим, як дорого вибратися звідти
(перевізники), у нас - не було грошей, і ми не бачили, як ми виживатимемо там,
в любій серцю Україні. Був момент, коли ми абсолютно серйозно приймали рішення,
що нам далі робити: чи не повернутися і дістатися до мого будинку і там
переховуватись, чи все ж таки виїжджати з пекла. І ви з Людмилкою тоді написали
дописи і люди накидали нам стільки грошей, що ми зрозуміли - так, нам буде
важко. Втратили все і найголовніше - здоров’я. Але в нас є друзі! З Божою
ласкою і підтримкою друзів вишкребемо! Як бачиш, не в грошах справа (вони й не
знадобилися на перевізника, але допомогли нам в подальшому) - а у підтримці,
яку продемонстрували через гроші мої друзі і люди, яким я була, можливо, навіть
не знайома. Ось що було рятивною палицею.

Людмилко, твоє «я завжди поруч» - неоціненне! Все написане вище відноситься і
до тебе. Я спостерігаю, як ти допомагаєш зараз на два фронти - масікам і
маріупольцям і плачу від вдячності. Я бачу, яка ти втомлена. І бачу, яка ти
сильна.

Януся, квіточко, щиро бажаю тобі міцного здоро’я! Ми згадуємо, як ти нас
частувала, обдаровувала подарунками з ніжною теплотою і подякою. Подарованій
тобою вишиванці завжди сиплються компліменти.

Всі ви сильні. Надихаюсь вами. Стараюсь. Росіянє вбили в нас надії своїми
ракетами добра, а ви повернули надії. Хочу, щоби всі ви знали: ви врятували
чотири людських душі (окрім нас з Сашком ще брат і свекров)!!!

Кожній, хто чекає бажаю дочекатіся. Чекаємо разом з вами. Мрію бачити ваші
родинні фото із щасливими мармизами все життя, що ми матимемо після нашої
перемоги!


Марта, непосидюча дівчисько, дякую за твою енергію - вона нас зарядила. Ти
вдягла мені Сашка, а твої кросівки допомогли мені швидко бігати Львовом 🙂 Ну і
на зв’язку я завдяки твоєму телефону (як працевлаштуюсь і зможу купити собі
свій - поверну, най комусь ще допоможе у скрутну хвилину).

Пані Марія - ви чудо! Знаходячись так далеко, на острові через море, Ви
обдаровуєте нас подарунками, наче ми ваші рідні! Я дивуюсь цьому, вражаюсь і
дякую. Це наче пульсація української крові, що тече в жилах різних людей, що
живуть на різних континентах, в різних країнах по всій планеті від одного на
всіх серця, що б’ється в один ритм, ритм боротьби за наше майбутнє. 

І тільки не кидайте в мене капцями, люди, але завдяки пані Марії я той, нахабно
шикувала на Різдвяні свята - ми їли бутіки з чорною ікрою, аж соромно. Але це
було тааааак смааачно! Дякую.

Пані Олеся, Оксана Михайлівна і колектив Тернопільської ЗОШ№ 14 - у вас
величезні серця! Ви відгукнулися, організувалися, почали допомагати
постраждалим українцям одразу, коли пекло почалося. Ви вставали о шостій ранку,
щоби у ваших підопічних були свіжі сніданки та лягали пізно в вечорі. Ви не
бачили ваших родин через це. Ви поєднювали вчителювання з волонтерством. Ваша
саможерованість вражаюча. Ми дякуємо до неба за вашу підтримку, за той час, що
ми блукали в тумані, що ми наново вчилися жити, наче малі діти вчаться ходити,
а дорослі їх підтримують, так і ви нас підтримали, поки ми наново навчилися
ходити.

Дякуємо волонтеркам Карині та Яні, двом молоденьким маріупольчанкам, в яких
вистачить мужньості на добрий десяток, що вони організували евакуацію людей і
тим дуже вчасно врятували наші життя. Такого не можливо забути! Бережи вас Бог
і допомагай в усьому!

Дякую волонтерському центру Вінниці, що знаходиться на третьому поверсі (чи
знаходився) Аквапарку. Ви були перевалочним центром для тих, хто мав транзитом
їхати далі. Але ви опікувалися нами, поки ми чекали на наших брата і маму, яких
росіянє затримали на окупованих територіях. Мені здавалося, що ми пробули у вас
цілий місяць, Сашко стрерджує, що лише три дні. Повірити не можу! Дякую, що так
ненав’язливо дали нам виплакати першу хвилю болю, що проривався назовні. Бо
нарешті було можна. Ваші пиріжки, тим не менш, я добре запам’ятала 🙂 Дуже
смачні! 🙂

Дякую всім, хто нас вдягли і нагодували смаколиками, обдарували парфюмами та
прикрасами, ниточками та книжками! 

Пані Вікторія та львівські господині, Галинко, Надійко, Іринко, Наталочко,
Світланко, отець Володимир і пані Юлія, Вікторія, Сергію, Лідіє, Олюню,
Оксаночко, Світланко, Наталочко, Наталочко, Ксеня, Іринко, Світлано,....
Сподіваюсь нікого не забула... Бачите, як вас багато і ви нас підтримали, коли ми
були, наче малі діти безпорадними та розгубленими і наново вчилися жити,
радіти, довіряти (ооо, як же скалічили ті покидьки вміння щиро довіряти! Щоб їм
самим до віку страждати від страху і недовіри!). Дякую вам всім за це! Обіймаю
подумки кожного і кожну з вас!

Коли я з кимось спілкуюсь і озвучую, що ми вийшли пішки і з цінностей в нас
була Мавка та документи, а перевдягнутися лише на раз в чисте, люди кидають на
мене здивований погляд, оглядаючи з голови до п’ят і доводиться пояснювати, що
мої друзі нас вдягли 🙂 Здається дехто мені все ж таки не вірить. 🙂

Мінігачкування, я абсолютно впевнена в цьому, навчило мою голову знову
працювати і концентрувати увагу на роботі 🙂 Можна сказати, що ви врятували
мене від дуже ранньої деменції. Дякую вам.

Щодо книжок: одна голова - то є добре, а коли їх багато і вони визнані
людством, як розумні - то шикарно! Книги повернули мені здатність бачити світ в
кольорах.

Дякую Антоніні за те, що допомогла мені піклуватися про мій фасад 🙂

Дякую пані Любові, хоч вона і образилася на мене, бо вона виклала на аукціон в
допомогу нашій родині зроблену нею прикрасу. Мені шкода, що люди наразі такі
виснажені та надають словам оточуючих власного значення. Але я з тим нічого не
подію. І я пам’ятаю про всі добрі діла, які були зроблені для мене.

Дякую тим журналістам, які комунікують зі мною і допомагають донести до країн
Європи щиру правду, щире бачення істиної російської огидної душонки!

Дякую волонтерам, які продоажують допомагати тим, що залишилися там, які
допомогли нам забрати Грейс.

І безумовно я дякую Анастасії та її чоловіку Андрію, які подарували мені цей
нойтбукі я завдяки ньому працюю, навчаюсь, спілкуюсь і залишаюсь на зв’язку! А
головне - дякую, що вітире в мене.

Українці - дивовижні люди. Так, ми кожен собі господар. І знаєте, це ж не є
недоліком! Це ж і є нашою перевагою! Нашею зброєю! Що там ляпнув абдристович
про монархію?! Дурненький, про яку монархію мже йти мова серед вільних людей,
господарів свого життя, які згуртувалися попри всі перепони зі сторони влади і
боронять свою країну, допомагають тим, кому симпатизують і тим, хто не
довподоби, але ж ніхто не є досконалим, тільки Бог!

І все це, наші дії, сварки, примирення - наче одна пісня на всіх, пісня про
нашу Україну. А може це вона так і співає сама?

Люблю людей, навіть коли здається, що ненавиджу 🙂

(фото ще з осені, бо якось ми забуваємо морди зафіксувати для історії)
