% vim: keymap=russian-jcukenwin
%%beginhead 
 
%%file 22_11_2021.fb.miheev_vladislav.1.zavorozhennost_maidan
%%parent 22_11_2021
 
%%url https://www.facebook.com/vladislav.mikheev.5/posts/4665129746887097
 
%%author_id miheev_vladislav
%%date 
 
%%tags chelovek,maidan2,obschestvo,psihika,psihologia,ukraina
%%title Завороженность Майданом
 
%%endhead 
 
\subsection{Завороженность Майданом}
\label{sec:22_11_2021.fb.miheev_vladislav.1.zavorozhennost_maidan}
 
\Purl{https://www.facebook.com/vladislav.mikheev.5/posts/4665129746887097}
\ifcmt
 author_begin
   author_id miheev_vladislav
 author_end
\fi

Завороженность Майданом.

Она присутствует как у сторонников, так и у противников его.

Но, мне кажется, это событие следует вынести за скобки и жить дальше. Также как
люди выносят за скобки землетрясения и цунами, отстраивая заново после
катастроф свои города.

Такие сбои не только в природе, но и в программе  человеческих сообществ, увы,
случаются.

Социальный катаклизм - всего лишь разновидность некой общей закономерности.

Разница в том, что песчинка, захваченная торнадо, не испытывает по его поводу
никаких эмоций.

А участник бунта испытывает.

Коллективная эмоция - очень мощный наркотик, на который  легко подсесть.
Алкоголик или наркоман в завязке, у которого уже отваливается печень, нередко
ностальгирует по  былому кайфу.

Даже под угрозой окончательно развалить государственный организм, конечно бы
снова вышел, и конечно бы снова ширнулся...

Эмоциональный пик Майдана именно этим многим дорог и памятен. Чувство свободы,
романтика протеста,  рядом - тысячи горящих глаз соратников, которые плечом к
плечу выступили на стороне добра, спасая страну от тирании и зла... Это ли не
кайф?! Быстрый, неподдельный, яркий!

В обыденной жизни его не встретишь. Там все долго и скучно - надо  учиться,
работать, растить детей,  строить карьеру.

Чтобы государственную систему менять, тоже нужны годы рутинного менеджмента,
просвещение, образование, трансформация институций.

Меняться самому -  опять не вариант: это требует рефлексии, больших
энергозатрат и времени, а кроме того больно бьёт по тщеславию.

Еще гностики открыли для маленького человека этот простой и увлекательный путь:
если мир вокруг плохой, а ты хороший, ты вправе делать с этим несовершенным
миром, что угодно во имя добра.

Бунт - самый простой способ маленького человека стать большим. Увы, не надолго.

И тут, конечно, Достоевский с Булгаковым нам в помощь...

Торнадо пройдёт. Катастрофы имеют свойство заканчиваться. Новые барыги и тираны
придумают маленькому человеку  новое позорное несовершенство и новую
несправедливость. И только маленький человек останется  прежним: ни умней, ни
нравственней, ни свободней, ни богаче он не станет. Разве что у него останется
память о том, как он был великой песчинкой великого торнадо!

Стоит ли осуждать его за это? Ажно судьи мы, вечно  недовольные чем-то
фейсбучные черви, этому маленькому человеку?

Согласитесь, реальное проектное мышление и реальная картина мира это реальная
скукотища! То ли дело гностический миф об избранничестве, магия толпы и
постреволюционные каргокульты.

Необычные явления, в первую очередь катастрофические, с древности вызывали у
людей сильнейшие эмоции и вследствии этого становились объектом мифологизации и
ежегодно повторяющихся ритуалов.

Вообщем, люди как люди, только Шустер с Гордоном их сильно испортили.

\ii{22_11_2021.fb.miheev_vladislav.1.zavorozhennost_maidan.cmt}
