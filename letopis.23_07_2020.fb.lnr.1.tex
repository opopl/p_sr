% vim: keymap=russian-jcukenwin
%%beginhead 
 
%%file 23_07_2020.fb.lnr.1
%%parent 23_07_2020
 
%%endhead 
\subsection{Страна победившей шизофрении --- Год бессрочного перемирия, захват заложников в Луцке и "террористы" в ЛДНР}
\url{https://www.facebook.com/groups/LNRGUMO/permalink/2869468399831382/}
  
\vspace{0.5cm}
{\small\LaTeX section: \verb|23_07_2020.fb.lnr.1| project: \verb|letopis| rootid: \verb|p_saintrussia|}
\vspace{0.5cm}

\index{Авторы!Денис Григорюк}

В Донбассе уже перестали следить за тем, какое по счету перемирие было
объявлено во время встречи Трехсторонней контактной группы. Думаю, что не
солгу, если скажу, что многие забыли о том, что 17 июля 2019 года, когда
президент Украины Владимир Зеленский только начинал свою политическую карьеру и
старался хотя бы создавать видимость выполнения данных обещаний, был подписан
документ, который провозглашал «бессрочное перемирие».

Спустя год специалисты Представительства ДНР в СЦКК подвели итоги очередного
отрезка времени, который в официальных документах принято считать режимом
прекращения огня.

Выше я не зря поставил «бессрочное перемирие» в кавычки, так как, по данным
СЦКК, уже на следующий день после торжественного подписания украинская армия 7
раз нарушила режим прекращения огня. В графике, который опубликовали
специалисты Совместного центра, отмечено, что больше всего ВСУ стреляли осенью
2019 и весной 2020 года, то есть перед отведением вооружений в трех точках на
линии соприкосновения и в период разгара пандемии коронавируса.

Кроме того, в СЦКК назвали точное количество снарядов, которое насчитали
сотрудники ведомства — 34189 боеприпасов было выпущено только по территории
ДНР. В это число входят и запрещенные Минскими соглашениями артиллерийские
снаряды.

Но это всего лишь сухие цифры, которые, безусловно, впечатляют, но нужно
понимать, что каждый из них несет опасность для жизни человека. Всего за период
с 21 июля 2019 по 20 июля 2020 года погибли 5 человек и еще 60 человек получили
ранения различной степени тяжести.

Если говорить о жертвах, то большая часть из списка погибших была убита весной
2020 года. Это говорит о том, что в текущем году Зеленский и его команда
окончательно отказались от мирного плана и приняли решение следовать примеру
его предшественника Петра Порошенко. В отчете СЦКК фигурируют случай гибели
молодой девушки под Горловкой в результате атаки украинских ударных
беспилотников, ранения детей в Александровке и другие вопиющие случаи
террористических атак ВСУ на мирное население ДНР.

В отчете говорится, что весной 2020 года ОБСЕ стали фиксировать переброску
военной техники украинской армии к линии фронта. Должно быть, это и стало
причиной, почему ВСУ в июне текущего года стали бить по средствам и
оборудованию, которое ОБСЕ использует для фиксации нарушений. Только за прошлый
месяц украинские военные уничтожили три комплекса видеонаблюдения
мониторинговой миссии. Глава ОБСЕ Яшар Чевик вынужден был признать, что в
период с апреля по июнь количество жертв среди мирного населения выросло в 5
раз. Возможно, этот факт не устроил украинское командование, раз нападения на
беспилотники ОБСЕ увеличились. В это же время сами ВСУ увеличили количество
атак с применением дронов.

На фоне опубликованной СЦКК информации из Луцка приходит информация о
террористе, который захватил автобус с пассажирами. Как стало известно, фанатик
выдвигает абсолютно абсурдные требования, но от этого он лишь опаснее для
населения, так как не преследует каких-то материальных целей, а представляет
угрозу населению своей деструктивной идеологией. Он спланировал свои действия,
в здравом уме (если можно так сказать о человеке, который решился на захват
заложников) и, в целом, действовал осознанно. Если можно верить информации,
которую публикуют Telegram-каналы, то на этот поступок его вдохновил инцидент с
захватом моста в Киеве, когда террориста украинские силовики не просто
отпустили, но сам глава Министерства внутренних дел Украины Арсен Аваков
заступился за боевика, так как он воевал в Донбассе на стороне ВСУ.

И в это же время, вопреки вышеупомянутой информации с обстрелами, гибелью
мирных жителей Донбасса, захватом моста бывшим участником боевых действий на
стороне Украины, официальный Киев называет террористами представителей ЛДНР.

В украинских городах регулярно находятся фанатики, сумасшедшие и военные с
посттравматом, которые устраивают захваты мостов, автобусов, бросают гранаты в
толпу людей под Верховной радой. Делают это всё люди с проукраинской позицией.
Если они и имеют отношение к событиям в Донбассе, так они воевали за
"суверенитет Украины", то есть в составе подразделений ВСУ или нацбатах.

Но, согласно заявлениям официальных представителей украинского государства и
СМИ, "террористами" являются жители ЛДНР, потому что они не согласились с
событиями 2014 года в центре Киева. Хотя ни один житель Донбасса за время
конфликта не совершил ни одного теракта на украинской территории.

Понимаю, что мысль банальная, но иногда нужно проговаривать вслух очевидные
вещи, чтобы люди перестали верить украинской лжи.

Денис Григорюк
  
