% vim: keymap=russian-jcukenwin
%%beginhead 
 
%%file 07_04_2018.stz.news.ua.mrpl_city.1.cerkov_cv_marii_magdaliny
%%parent 07_04_2018
 
%%url https://mrpl.city/blogs/view/tserkov-sv-marii-magdaliny
 
%%author_id burov_sergij.mariupol,news.ua.mrpl_city
%%date 
 
%%tags 
%%title Церковь св. Марии Магдалины
 
%%endhead 
 
\subsection{Церковь св. Марии Магдалины}
\label{sec:07_04_2018.stz.news.ua.mrpl_city.1.cerkov_cv_marii_magdaliny}
 
\Purl{https://mrpl.city/blogs/view/tserkov-sv-marii-magdaliny}
\ifcmt
 author_begin
   author_id burov_sergij.mariupol,news.ua.mrpl_city
 author_end
\fi

Большинство людей, кто видел своими глазами величественный мариупольский храм
во имя святой равноапостольной Марии Магдалины, ушло в мир иной. Судите сами:
церковь эта была варварски разрушена воинствующими безбожниками в середине
тридцатых годов прошлого века, скорее всего, в 1936 году. А те жители города,
которым в ту пору было по семь-десять лет, сейчас девяностолетние старики и
старухи.

Прежде чем продолжить рассказ об этом храме, обратимся к его предыстории. На
том месте, где в наши дни Греческая улица пересекается с проспектом Мира,
азовский губернатор Василий Алексеевич Чертков \enquote{по собственному плану}, как это
сказано в главе \enquote{Православные храмы в Мариуполе}, в хорошо известном краеведам
сборнике \enquote{Мариуполь и его окрестности}, в 1778 году заложил фундамент главной
церкви строящегося города Павловска. Тогда, конечно, ни здания банка, ни других
привычных для нас строений не было и в помине. Более того, и самих улиц тогда
не существовало.

\ii{07_04_2018.stz.news.ua.mrpl_city.1.cerkov_cv_marii_magdaliny.pic.1}

Будущая церковь, как говорится в упомянутом выше источнике, была названа во имя
\textbf{Марии Магдалины в честь великой княгини Марии Федоровны – жены наследника
престола Российской империи Павла Петровича, а затем императора Павла I. Святая
равноапостольная Мария Магдалина была  небесной покровительницей великой
княгини.} Подтверждением этого может служить такой факт.
София-Доротея-Августа-Луиза, принцесса Вюртемберг-Монбельярская 14 сентября
1776 года приняла крещение по православному обряду и была наречена Марией
Федоровной. 26 сентября того же года состоялось ее венчание с Павлом
Петровичем. А на следующий день, как пишет биограф, молодоженам от имени
Священного Синода были преподнесены две иконы с образами их святых покровителей
- святого Апостола Павла и святой равноапостольной Марии Магдалины. 

\ii{07_04_2018.stz.news.ua.mrpl_city.1.cerkov_cv_marii_magdaliny.pic.2.chertkov_va}

Церковь Марии Магдалины, \textbf{заложенная В. А. Чертковым}, строилась долго. И только в
1791 году она была освящена. На плане города 1811 года она обозначена соборной.
Через сто лет после этого события  была запечатана из-за ветхости. Позже она
была и вовсе разобрана, а на ее месте возвели часовню в память чудесного
спасения цесаревича Николая Александровича, будущего императора Николая II во
время его путешествия по Японии. Кстати, часовня, которую в народе называли
капличкой, стояла на своем месте еще в начале тридцатых годов прошлого
столетия, а потом ее убрали, так как она мешала прокладке двухпутной трамвайной
линии по проспекту Республики, так прежде назывался современный проспект Мира.
История первой церкви Марии Магдалины завершилась.

\ii{07_04_2018.stz.news.ua.mrpl_city.1.cerkov_cv_marii_magdaliny.pic.3}

Но задолго до сноса церкви, задуманной В. А. Чертковым, в 1862 году был заложен
храм с тем же названием. Большой, с престолами во имя святой Марии Магдалины,
святого Иоанна Крестителя и Покрова Пресвятой Богородицы. Строительство его шло
целых тридцать пять лет. Еще в шестидесятых годах ХХ века пришлось услышать от
стариков то ли быль, то ли легенду, что кладку храма вели только летом,
прихватывая май и сентябрь, в холодное же время года работы приостанавливались.
Когда к ним возвращались вновь, то несколько верхних слоев кладки удаляли, а на
их место укладывали на свежем растворе новые кирпичи. Будто бы поэтому его
долго и строили. Только ли из-за этого? Проверить достоверность этого изустного
сказания не удалось. Но точно известно, что храм освятили в 1897 году и в нем
начали править службы. Вокруг величественного храма были высажены деревья,
образовавшийся сквер был назван Александровским - в честь царя Александра II,
вошедшего в историю как освободитель крестьян от крепостной зависимости и,
кроме того, проведшего ряд других либеральных реформ.

После Харлампиевского собора, который возвышался в самом начале Екатерининской
улицы, новая церковь Марии Магдалины была самой большой в Мариуполе. Купола
Марии Магдалины и колокольня Харлампия  - так несколько фамильярно называли
храмы наши пращуры – были видны с моря за несколько миль от берега. Местные
рыбаки использовали их как ориентиры для своих нехитрых навигационных расчетов...

\textbf{Читайте также:} \href{https://mrpl.city/blogs/view/vtracheni-hrami-mariupolya}{%
Втрачені храми Маріуполя, Ольга Демідко, mrpl.city, 16.03.2018}

Бытует мнение, что здание драматического театра построено на месте церкви Марии
Магдалины, чуть ли даже не на ее фундаменте. Это утверждение не соответствует
истине. Для того чтобы в этом убедиться, достаточно внимательно рассмотреть
старые карты Мариуполя. На них крестиком обозначено пересечение продольной и
поперечной осей церкви, т.е. центр строения. Этот центр совпадает с точкой
пересечения осей Больничной и Таганрогской улиц. В советское время эти две
улицы были объединены в одно целое названием - именем Артема (С. Сергеева),
советского государственного и партийного деятеля. Церковь еще не была заложена,
а на карте 1826 года ее месторасположение уже указано на том самом месте, о
котором говорилось выше. То же самое можно сказать, посмотрев на карты 1891 и
1930 годов. Если же обратиться к современным картам нашего города, то видно –
здание театра смещено относительно бывшей церкви в западном направлении.

\ii{07_04_2018.stz.news.ua.mrpl_city.1.cerkov_cv_marii_magdaliny.pic.4.cerkov_sv_magdaliny_zimoj}

Но не только с помощью карт можно убедиться в этом. Сохранилось несколько
дореволюционных фотографий с изображением церкви Марии Магдалины. На одной из
них, снятой с южной стороны улицы Больничной, теперешней Куинджи, хорошо видна
церковь. А если в наши дни с этой же точки фотографировать, то театр не будет
виден. Есть и косвенное доказательство, что церковь стояла на оси улицы Артема.
До разрушения храмов в Мариуполе на этой же оси стояло еще две церкви: Успения
Пресвятой Богородицы на Марьинске (район междугородного автовокзала) и
Рождества Пресвятой Богородицы на Карасевке (район школы № 11), поэтому третью
церковь привязали, как говорят архитекторы, на той же оси. И еще, когда церковь
Марии Магдалины в сквере была взорвана, то вряд ли извлекли ее фундамент, дело
это трудоемкое и дорогое. Вместе с тем известно, что строители избегают
сооружать новые здания, частично опирающиеся на старые фундаменты.
Любознательные читатели могут своими глазами увидеть точку пересечения осей
церкви Марии Магдалины. В нескольких метрах от лестницы, ведущей к служебному
входу в театр, лежит стальной лист, который прикрывает устройство для крепления
новогодней елки, которую еще несколько лет назад устанавливали здесь. Это и
есть упомянутая выше точка.

\textbf{Читайте также:} \href{https://mrpl.city/news/view/mariupol-k-240-letiyu-napolnyat-installyatsiyami-v-ramkah-proekta-gorod-marii-video}{%
В Мариуполе нашли остатки храма Марии Магдалины, Яна Іванова, mrpl.city, 16.02.2018}

В послевоенные годы на этом месте был сооружен круглый павильон. Там можно было
купить ситро \enquote{Дюшес}, мороженое, два-три вида пирожных, папиросы
\enquote{Беломорканал}, \enquote{Шахтерские} или \enquote{Казбек} и прекрасное пиво \enquote{Жигулевское}
местного пивоваренного завода, если повезет. Это строение нарисовал акварелью с
натуры архитектор-художник Николай Иосифович Никаро-Карпенко. Сейчас этот
рисунок хранится в фондах Мариупольского краеведческого музея. В свое время его
сын – известный краевед Олег Николаевич, ныне покойный, на основе анализа
старинных фотографий и с помощью инженерных расчетов воссоздал перспективное
изображение церкви Марии Магдалины.

Кисти нашего современника, \textbf{одаренного мариупольского художника Валерия
Александровича Остапенко} принадлежит еще одна историческая реконструкция этой
церкви и Александровского сквера, его окружавшего, какими они могли быть летним
воскресным днем 1913 года. Автор в комментарии к картине написал: \textbf{\em\enquote{Пусть этот
солнечный уголок того ушедшего в прошлое Мариуполя перенесет зрителя в тот,
такой еще безмятежный, поющий теплым чистым летним воздухом южного городка,
счастливый мир сиюминутного счастья}}.

\ii{07_04_2018.stz.news.ua.mrpl_city.1.cerkov_cv_marii_magdaliny.pic.5.aleksandrovskij_skver_valerij_ostapenko}
