% vim: keymap=russian-jcukenwin
%%beginhead 
 
%%file 21_09_2021.fb.nikonov_sergej.5.bilchenko_recenzia_teatr_inercia_ili_stagnacia
%%parent 21_09_2021
 
%%url https://www.facebook.com/alexelsevier/posts/1586063651738944
 
%%author_id nikonov_sergej,bilchenko_evgenia
%%date 
 
%%tags bilchenko_evgenia,chehov_anton.rossia.pisatel,jazyk,kiev,kritika,kultura,recenzia,teatr,teatr.kiev.khat,ukraina
%%title БЖ. Украинский театр. Инерция или стагнация?
 
%%endhead 
 
\subsection{БЖ. Украинский театр. Инерция или стагнация?}
\label{sec:21_09_2021.fb.nikonov_sergej.5.bilchenko_recenzia_teatr_inercia_ili_stagnacia}
 
\Purl{https://www.facebook.com/alexelsevier/posts/1586063651738944}
\ifcmt
 author_begin
   author_id nikonov_sergej,bilchenko_evgenia
 author_end
\fi

И снова пост. На этот раз рецензия на спектакль. Причем негативная. Я не был в
театре давно. И в КХАТ не был. Поэтому просто репощу. И никак не выдаю своего
мнения. Даже лайки не ставлю. 

БЖ. Украинский театр. Инерция или стагнация?

Да, мне не нравятся университеты, из которых прогоняют талантливых профессоров.

Да, мне не нравятся театры, из которых прогоняют талантливых актеров.

Да, мне не нравятся ректоры и примадонны с непомерным самолюбованием и
замашками диктаторов, замаскированными под самореализацию.

Да, мне не нравится, когда стихи все время пишут про розы и морозы или про
виски и кофе.

Да, мне не нравится, когда в текстах нет сюжета, композиции и психологии, а
эффект достигается заламыванием рук.

\ifcmt
  ig https://scontent-yyz1-1.xx.fbcdn.net/v/t1.6435-9/242325506_1586062335072409_7854728254371580629_n.jpg?_nc_cat=106&ccb=1-5&_nc_sid=730e14&_nc_ohc=MX7KY1YV_p8AX-5_R0i&_nc_ht=scontent-yyz1-1.xx&oh=e33dc02877a3b3a9057832e1e1c11d6a&oe=6170B61C
  @width 0.4
  %@wrap \parpic[r]
  @wrap \InsertBoxR{0}
\fi

Да, мне не нравится, когда плохую игру маскируют красивыми платьями, голыми
грудями, аппетитными попками и всем, что вызывает эрекцию у зрителя-мужчины. Я
не мужчина. У меня не встало. Не может быть ярмарки. Ни в поэзии, ни в театре.
Это искусство, а не шоу. Исключение: "Кабаре" с Лайзой Миннелли, но КХАТ до
такого антифа не допрыгнет, испугается.

Да, мне не нравится, когда поэт пишет по стиху в десять лет, а репертуар годами
не обновляется.

Да, мне не нравится, когда извиняются за русский и ставят при этом Чехова.
Крест или трусы.

Да, мне не нравится карнавал нарядов вместо "Верю" Станиславского и весь этот
пестрый провинциализм.

Да, мне не нравится, когда малороссийские героини Гоголя играют изысканных
дворянок.

Да, мне не нравится, когда все должны петь дифирамбы одному начальнику или
начальнице.

Да, мне больше не нравится театр КХАТ.

И, так как муж у меня - не подкаблучник, а мужчина, жадный до юных нимф в
шелках, наши мнения не совпали. Ему и лирика моя в авангарде не вся нравится,
вешаться из-за того, что любимый не понимает неологизмов, я не собираюсь.

Хочу во МХАТ. В КХАТ без своих не предавших меня друзей не хочу. По причине
украинизации тоже, но больше - из-за прогрессии китча и инерции репертуара.

Нет, я не злая, я честная. Я долго ждала, когда пафос сменится на реализм, а
Виктор Кошель сыграет свою истинную трагедию, а не клоунаду под дурачка.

Ну, извините. У меня не было ещё негативных рецензий, пора начинать.

\#рецензииотБЖ
