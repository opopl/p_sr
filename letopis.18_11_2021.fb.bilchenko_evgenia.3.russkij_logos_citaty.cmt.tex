% vim: keymap=russian-jcukenwin
%%beginhead 
 
%%file 18_11_2021.fb.bilchenko_evgenia.3.russkij_logos_citaty.cmt
%%parent 18_11_2021.fb.bilchenko_evgenia.3.russkij_logos_citaty
 
%%url 
 
%%author_id 
%%date 
 
%%tags 
%%title 
 
%%endhead 
\subsubsection{Коментарі}

\begin{itemize} % {
\iusr{Евгения Бильченко}
\textbf{Алексей Бажан} 

Любовью. Любовь помогла нам победить нацизм. Любовь заставила моих 55 студентов
подставиться под СБУ. Любовь привела вчера ко мне двух семидесятилетних
ленинградок, поделившихся последним куском со мной, чтобы я справилась. Любовь
ко мне завела моего мужа за колючую проволоку на таможне. Любовь к Донбассу
заставила меня потерять профессию. Любовь изливается из глаз Мамы Казанской и
из глаз моего друга атеиста-коммуниста, спешащего на помощь, когда на меня
напали националисты на Крещатике. Любовь собирает нас на Ваське, и каждый несёт
что может: кто улыбку, кто кекс. Любовь сделала так, что из батальона Аз.
(запр. в РФ) ушел, начитавшись меня, русский солдат, проведя сутки в подвалах с
лужами крови, написав перед этим: "Спасибо деду за победу!" Любовь побудила
второго солдата из ополчения таскать мои сумки. Любовь привела ко мне домой
учителя из Ропши с нимесилом. Любовь сделала седым моего брата из 2 мая Одессы
и влечет его за мной по всем городам, и мы не знаем, в какой следующей стране
обменяемся значками с изображением Высоцкого. Любовь, которая заставляет общее
страдание переживать как личное и личное как общее. Любовь, которая заставляет
меня отвечать даже вам: интеллектуалу-постмодернисту, с интересными знаниями,
нулевым милосердием и кучей апломба человека-цифры. Любовь заставляет меня
отвечать вам, который одобряет пародию на Бессмертный полк в свином рыле только
потому, что там бывали мои друзья с Рубинштейна, 13 и Елизаров. Вот эта любовь
из меня выплёскивается, она меня сжигает, за нее мне звезду на спине рисуют.
Вот эта любовь позволяет мне быть тем, чем считают себя другие цивилизации.
Ведь Европе можно иметь латинский Логос, Индии - можно, вообще Востоку, США -
можно, лишь нам нельзя, потому что мы - евреи 21 века, нам ничего нельзя,
только заткнуться и забавлять вас Шанталь Муфф и Деррида. Только я - не из
таких. Не заметили ещё? Стыд, что ты счастлив, потому что другие несчастны, -
это модель Христа. Как вы кощунственно толкует дядю Федора. Если вы ставите
знак равенства между абсолютным добром Благодати и абсолютным злом гитлеризма,
между сжигающими в печах и спасающим из печей - да, теперь я знаю имя ада. Оно
у меня описано в книге по критике постмодерной иронии. За сим откланяюсь.
Выражаясь вашим языком, вы нарушили вашу любимую мультикультурную
толерантность, стигматизировав мою идентичность. Я теперь с вами буду на языке
ЕС говорить. Просто не дождусь вашего развоплощения. С кафедры или трибуны мне
либералов парировать - это, как игра: я все ходы знаю наперед. Вот vatanam, тем
сложнее: они по сути правы, но они не метят дракону в глаз. А бить надо
сократовский лёгким уколом в глаз, а не плясать на лапах Левиафана. Спасибо.

\begin{itemize} % {
\iusr{Тим Печеніг}
\textbf{Евгения Бильченко} 

лавры евреев 21 века, это вы уже загнули, Евгения Витальевна, ваше право
симпатизировать РФ и ее богатому наследию, описанные нападения радикалов на вас
действительно ужасны и вызывают беспокойство, но в общем и целом, на Украине
абсолютно свободно дышится русскоязычным гражданам то

\ifcmt
  ig https://scontent-frx5-1.xx.fbcdn.net/v/t39.30808-6/257132037_10220296618634647_2607046844446732815_n.jpg?_nc_cat=100&ccb=1-5&_nc_sid=dbeb18&_nc_ohc=hiQb5y9pnzQAX-5A240&_nc_ht=scontent-frx5-1.xx&oh=86591a554dfeba89ea3414a1d8daf5de&oe=619D18E9
  @width 0.4
\fi

\iusr{Алексей Бажан}
\textbf{Евгения Бильченко} Что было сказано о гердеровских "мыслях Бога о человечестве", надеюсь Вам известно, потому ограничусь одним вопросом - откуда лично Вам известно о Божественном замысле о России и русском народе? По такому ответственному вопросу, извините, готов выслушать исключительно подлинные высказывания прославленных Церковью святых (т.е. "пророчества" Лаврентия Черниговского и т.п. не прокатят).

\iusr{Тим Печеніг}
\textbf{Антонина Бажукова}! 

Спасибо что прочли, я то живу в Израиле, но очень часто на Украине бываю, и при
том в разных регионах, немало знакомых, людей творческих именно в области
русской словесности, давече переселилось туда из субъектов РФ и чувствуют себя
, опять это совершенно мое абсолютно субъективное суждение и может быть я
ошибаюсь и ввожу в заблуждение, короче дышат они там намного лучше

\iusr{Тим Печеніг}

\ifcmt
  ig https://scontent-frx5-1.xx.fbcdn.net/v/t39.30808-6/257460641_10220298895971579_1148063823600328339_n.jpg?_nc_cat=105&ccb=1-5&_nc_sid=dbeb18&_nc_ohc=glr58hMzX3MAX9VYaT2&_nc_ht=scontent-frx5-1.xx&oh=6d311ff31968ee674aa25a6233c5b2b3&oe=619C2F27
  @width 0.4
\fi

\end{itemize} % }

\iusr{Тим Печеніг}
Бертогольц интересна

\iusr{Евгения Бильченко}
\textbf{Тим Печеніг} "Интересна", это мягко говоря.

\iusr{Алексей Бажан}
\textbf{Евгения Бильченко} Борис Корнилов, когда не халтурил, писал лучше.

\iusr{Ольга Батурина}
Это правда!

\iusr{Алексей Бажан}

Полагаю, не стоит вменять Достоевскому слова его персонажа, как и делать
Берггольц (ее первый муж поэт куда более яркий, кстати) сторонницей софиологии.
Кстати, чем Логос народа отличается от давно вышедшего в тираж Volksgeist'а?

\end{itemize} % }
