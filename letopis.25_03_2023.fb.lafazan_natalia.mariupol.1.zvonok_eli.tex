%%beginhead 
 
%%file 25_03_2023.fb.lafazan_natalia.mariupol.1.zvonok_eli
%%parent 25_03_2023
 
%%url https://www.facebook.com/permalink.php?story_fbid=pfbid0tPhF5A7druaegQrPaUquXQdmSdrRhNzFmfNSGEh72DtUzSnrvEhRNuyiHQgMb87vl&id=100030592628843
 
%%author_id lafazan_natalia.mariupol
%%date 25_03_2023
 
%%tags mariupol,mariupol.war,harkov
%%title Звонок Эли для меня всегда желанный
 
%%endhead 

\subsection{Звонок Эли для меня всегда желанный}
\label{sec:25_03_2023.fb.lafazan_natalia.mariupol.1.zvonok_eli}

\Purl{https://www.facebook.com/permalink.php?story_fbid=pfbid0tPhF5A7druaegQrPaUquXQdmSdrRhNzFmfNSGEh72DtUzSnrvEhRNuyiHQgMb87vl&id=100030592628843}
\ifcmt
 author_begin
   author_id lafazan_natalia.mariupol
 author_end
\fi

Звонок Эли для меня всегда желанный. Но тот звонок стал глотком свежего
воздуха... И флешбеком в мирную жизнь. Эля без долгих раздумий спросила: "
Помнишь, ты приглашала к себе в гости?" Я ответила: "Да!". В голове закрутились
воспоминания: лето 2021, большой надувной бассейн, я с Анастасія Янковська на
веранде, наши долгие разговоры. Тогда, под мирным небом Мариуполя, где война
длилась с 2014, мы чувствовали себя комфортно, уверенно и защищено. Настя
ходила с сыном на море, мы работали и в этой идеальной картинке не хватало
только Эли. Она всё собиралась приехать, но работа и ее неимоверная
ответственность перед ее клиентами, не позволяли ей этого сделать. Потом в
голове закрутились воспоминания встреч в Харькове. В то время мы могли поехать
туда на выходные. Встретиться с нашей старшей доченькой и увидеть своих друзей.
Потом начало войны и кошмар, который был в Харькове. У нас в Мариуполе в то
время ещё не было ада. Новости с Харькова охватывали сознание ужасом. Эля
спросила, точно ли ее приезд мне удобен. Я отвечала, что конечно да, а она
начинала сомневаться. Потом долгожданная встреча. И все время Эля спрашивала,
ничем ли она меня не обидела. Я искренне доказывала, что все отлично и я
счастлива, что мы встретились, что я познакомилась с ее подругой Алёной. Потом
их планы изменились и они, найдя своих харьковчан в соседнем городе, уехали на
день раньше. Меня спас дождь. Они уезжали, но слезы на моих щеках были не
видны. Они вырвали меня из привычного "ещё один день прошел". Но они уезжали, и
когда будет наша следующая встреча с ними я не знаю. Также, как и с моими
соседями, моими родителями, моими кумовьям, моими друзьями, моими коллегами,
моими сватами, моим двоюродным братом и его семьёй, моими племянниками... Эля
стала глотком свежего воздуха в бесконечном ожидании победы Украины и триггером
воспоминаний. Она умеет искренне любить, жить для всего мира и, при этом,
оставаться самой собой. Прошло уже три месяца с той встречи, но то, что она
подарила мне самые теплые воспоминания с начала войны - неоспоримо. Все буде
Україна і ми ще обов'язково зустрінемося на подвір'ї мого зруйнованого будинку
та у Харкові! Эличка Шилина - обов'язково! З Настею та Павлом!

%\ii{25_03_2023.fb.lafazan_natalia.mariupol.1.zvonok_eli.cmt}
