% vim: keymap=russian-jcukenwin
%%beginhead 
 
%%file slova.mir.world
%%parent slova
 
%%url 
 
%%author 
%%author_id 
%%author_url 
 
%%tags 
%%title 
 
%%endhead 
\chapter{Мир (Планета, Вселенная)}
\label{sec:slova.mir.world}

%%%cit
%%%cit_pic
%%%cit_text
\emph{Современный мир} совсем другой по сравнению с феодально-аграрным строем, когда
\enquote{размер имел значение}. В своей книге \enquote{Антихрупкость. Как
извлечь выгоду из хаоса} знаменитый финансист Нассим Талеб доказывает, что
самыми устойчивыми и эффективными экономическими объектами являются маленькие.
Во время кризисов лопаются сотни крупных банков, но ни один ларек в Украине не
обанкротился.  МАФы, с которым отчаянно борется недалекий Кличко, именно так
живучи потому что они сверхэффективны.  Мы живем в \emph{мире}, когда начинают
пересматриваться устаревшие ценности. Малые страны осознали это первыми.
Большие страны-бронтозавры все еще воюют за свою территориальную целостность и
проигрывают конкуренцию \enquote{малышам}
%%%cit_comment
%%%cit_title
\citTitle{Мы живем в мире, когда размер перестал иметь значение}, 
Андрей Головачев, strana.ua, 13.06.2021
%%%endcit

%%%cit
%%%cit_head
%%%cit_pic
%%%cit_text
Найти свое место в \emph{мире}.  Просто существование украинского национального
государства не имеет смысла для сложных современных \emph{мировых} политических и
экономических процессов. Важна роль, которую играет Украина сегодня и будет
играть в будущем. И никто нам эту роль в \emph{мировом} распределении труда, мировой
политике и влиянии просто так не уступит. Но мы эту роль можем заслужить,
заработать. Именно заработать, создать, а не выпросить, выклянчить, взять
шантажом, как это делает современное украинское руководство. С 2005 года
Украина стремительно превращается из субъекта международной политики в ее
объект и закономерно завершает этот путь в качестве страны, находящейся под
полным внешним управлением
%%%cit_comment
%%%cit_title
\citTitle{О будущем украинского и русского народов}, 
Виктор Медведчук, strana.ua, 15.07.2021
%%%endcit

%%%cit
%%%cit_head
%%%cit_pic
%%%cit_text
В конечном счете, враг и враждебность есть способ изжить-избыть упрощенное
мышление за счет уничтожения носителей упрощенного мышления — как с одной
стороны, так и с другой стороны. Лишь наследники проявленной вражды принимают
более сложно устроенную реальность, участники вражды не способны усложнять
реальность, иначе бы они не стали врагами и не создали бы себе врагов.  \emph{Мир} без
врага — это \emph{мир}, в котором есть место сложному мышлению, которое осуществляет
непрерывное преобразование, непрерывно порождает новый порядок, умеет
поддерживать равновесие и договор с другими порядками.  Скажи мне, кто твой
враг, и я скажу тебе, что ты к мышлению непричастен
%%%cit_comment
%%%cit_title
\citTitle{Враг}, Сергей Дацюк, analytics.hvylya.net, 18.11.2021
%%%endcit
