% vim: keymap=russian-jcukenwin
%%beginhead 
 
%%file 24_08_2021.fb.butusov_jurij.1.vpechatlenija_parad_kiev
%%parent 24_08_2021
 
%%url https://www.facebook.com/butusov.yuriy/posts/6220315288008715
 
%%author Бутусов, Юрий
%%author_id butusov_jurij
%%author_url 
 
%%tags kiev,nezalezhnist,parad,ukraina,vpechatlenie
%%title 8 впечатлений от парада на День независимости и Марша защитников Украины
 
%%endhead 
 
\subsection{8 впечатлений от парада на День независимости и Марша защитников Украины}
\label{sec:24_08_2021.fb.butusov_jurij.1.vpechatlenija_parad_kiev}
 
\Purl{https://www.facebook.com/butusov.yuriy/posts/6220315288008715}
\ifcmt
 author_begin
   author_id butusov_jurij
 author_end
\fi

8 впечатлений от парада на День независимости и Марша защитников Украины.

1. Мы оторвались от России навсегда. У нас большие проблемы с государством и
организацией, но уровень общественного развития на порядок выше. Украинцы любят
свою страны не из-под палки ОМОНа, это искренний порыв, это национальное
самосознание. Те чувства, которые вызвало участие в этом празднике невозможно
сравнить ни с чем. Спасибо всем вам, дорогие сограждане - это настоящая
Независимость и Свобода.

\ifcmt
  pic https://scontent-cdt1-1.xx.fbcdn.net/v/t1.6435-9/240553535_6220322781341299_3642403221803958915_n.jpg?_nc_cat=105&ccb=1-5&_nc_sid=730e14&_nc_ohc=VJ3NQtAuT7cAX9RZ6Fx&_nc_ht=scontent-cdt1-1.xx&oh=5652b965316fdff61185e0932bdb86ee&oe=614CEFF9
  width 0.4
\fi

2. Офис президента организовал торжественную часть на хорошем уровне. Можно
высказать много замечаний, но в целом смотрелось достойно. Меня более всего
впечатлило награждение званием Герой Украины полковника Евгения Сидоренко,
одного из настоящих героев боев под Иловайском, представление на которого было
подписано Генштабом и министром обороны еще в сентябре 2014-го, но 7 лет
пролежало без движения на Банковой. Справедливость наконец восторжествовала. 

Да, хотелось бы более резких слов и смыслов Зеленского против России, хотелось
оглашения каких-то обязательств по военной реформе,  об оценках "перемирия" и
что делать дальше, но любой прогресс стоит поддерживать.

3. Беспрецедентное участие авиации НАТО в параде - это во многом прорывная для
Украины инициатива. Впервые в небе над Киевом пролетели 4 польских истребителя
F-16 и 2 британских истребителя Тайфун прошли в одном строю с 4 украинскими
МИГ-29. F-16 и Тайфуны имеют огромное преимущество над всеми типами российских
истребителей, включая новейшие Су-35 и Су-57, за счет  современных радаров с
активной фазированной антенной решеткой и более эффективных и дальнобойных
ракет воздушного боя AIM-120C-7 и С-8. Истребители НАТО над Киевом - это
демонстрация прежде всего для России. Если Путин посмеет начать агрессию с
воздуха, в расчете на  слабую украинскую авиацию и ПВО, то истребители НАТО
способны быстро устроить бесполетную для россиян зону, также как НАТО закрыло
небо для российских самолетов над северной Сирией. На  фото - пилот польского
F-16, принимающий участие в параде над Киевом. 

4. Беспрецедентное количество украинцев, которые приняли участие в мероприятиях
в День независимости в центре Киева. Буквально яблоку негде было упасть в
центре - целое море людей, от края до края Хрещатика, куда только достает
взгляд, вышиванки, национальные цвета, флаги. Причем добрая половина пришедших
так и не смогла удобно посмотреть парад, видеотрансляция была гораздо удобней,
потому что на тротуарах, на оградах, балконах и деревьях не было свободного
места.   Но люди не ломились в первый ряд, а пришли на парад, чтобы просто
ощутить единение и национальный подъем, поведение украинцев было просто
восхитительным. Полиция не мешала и ее было много только вокруг официальной
фан-зоны, люди сами себе обеспечили порядок. 

5. Парад получился самый большой в истории Украины. 32 парадных "коробки" -
ВСУ, СБУ, Нацгвардия, Нацполиция, Госпогранслужба, Госслужба чрезвычайных
ситуаций. Это было очень сложное мероприятие - много боевой техники пришло с
заводов прямо на парад, много экипажей были сборными из разных подразделений.
Проблем было много, но все получилось. О самих проблемах напишу позднее. Люди
овациями встречали украинских воинов, технику на земле и в воздухе.

6. Марш защитников Украины также превосходил воображние. Несмотря на жару,
после военного парада десятки тысяч людей с Хрещатика не ушли. В парке Шевченко
сформировалась гигантская колонна в 64 группы, с представителями 42
общественных организаций, для участия в марше. И на всем пространство на
отрезке от бульвара Шевченко до Майдана с обеих сторон выстроились киевляне,
чтобы поприветствовать участников Марша. Колонна шла более двух часов
непрерывным потоком. На марше полиция была представлена совсем малозаметным
числом "полиции диалога" - за порядком на марше следили сами люди, и огромные
массы людей передвигались просто идеально и без малейших проблем. Это не
Россия, и никогда не будет Россией, украинцы вышли на куда более высокий
уровень общественного развития. Про Марш защитников напишу еще позднее.

7. Количество делегаций дружественных стран  на 30-летний юбидей - также было
самым большим. В 2017-м году участие в параде принял министр обороны США Джеймс
Мэттис, сегодня США представляла министр энергетики. На марше защитников
Украины очень большую поддержку киевлян вызвала колонна патриотов Беларуси,
которые получили много аплодисментов своими энергичными речевками против Путина
и Лукашенко.

8. Речевка "Путин - ...уйло!" благодаря истерике российской пропаганды стала
просто неофициальным паролем и парада и марша. На Марше защитников Путина
постоянно вспоминали все колонны, Путин был самым часто упоминаемым в словесных
речевках политиком.
