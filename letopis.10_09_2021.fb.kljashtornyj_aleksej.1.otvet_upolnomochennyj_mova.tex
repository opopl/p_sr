% vim: keymap=russian-jcukenwin
%%beginhead 
 
%%file 10_09_2021.fb.kljashtornyj_aleksej.1.otvet_upolnomochennyj_mova
%%parent 10_09_2021
 
%%url https://www.facebook.com/Klasztorny/posts/378692000516687
 
%%author_id kljashtornyj_aleksej
%%date 
 
%%tags mova,ukraina,ukrainizacia
%%title Отримав сьогодні ВРАЖАЮЧУ відповідь Секретаріяту Уповноваженого
 
%%endhead 
 
\subsection{Отримав сьогодні ВРАЖАЮЧУ відповідь Секретаріяту Уповноваженого}
\label{sec:10_09_2021.fb.kljashtornyj_aleksej.1.otvet_upolnomochennyj_mova}
 
\Purl{https://www.facebook.com/Klasztorny/posts/378692000516687}
\ifcmt
 author_begin
   author_id kljashtornyj_aleksej
 author_end
\fi

Отримав сьогодні ВРАЖАЮЧУ відповідь Секретаріяту Уповноваженого із захисту
державної мови про причини нез’яви їхніх представників на акції громадського
контролю за дотриманням мовного законодавства на торгових рядах коло станції
метра «Академмістечко» у Святошинському районі Києва. Дякую за оперативну
реакцію! Пропоную ВСІМ прочитати її: ЦЕ ВАЖЛИВО!..

\ifcmt
  ig https://scontent-frx5-1.xx.fbcdn.net/v/t39.30808-6/241735509_378690320516855_9164393327661169098_n.jpg?_nc_cat=105&_nc_rgb565=1&ccb=1-5&_nc_sid=730e14&_nc_ohc=gpdeobWYIrMAX-F5kK6&_nc_ht=scontent-frx5-1.xx&oh=2a6f7d4f1c924af901a9f2f163e15d3d&oe=6141502B
  @width 0.4
  @wrap \InsertBoxR{0}
\fi

Виявляється, Уповноважений та його Секретаріят дотримуються позиції, що вони,
згідно Закону... не мають показувати носа з офісу! Буквально так!!!

Цитую: «нормативно-правовими актами не передбачено здійснення державного
контролю за застосуванням державної мови шляхом проведення виїзних перевірок
(за місцезнаходженням суб’єкта контролю)...» 

Із чого, судячи з контексту відповіди, панове Тарас Кремінь та керівник його
апарату Юрій Зубко роблять висновок, що й НЕ ПОВИННІ контролювати мови
обслуговування споживачів безпосередньо у місцях обслуговування споживачів, а
лише реаґувати на докази (наприклад – відео чи авдіозаписи), надані їм
скаржниками. НУ ДУЖЕ «ЗРУЧНА» ПОЗИЦІЯ, скажу я вам, пані та панове!

\ifcmt
tab_begin cols=2
  ig https://scontent-frx5-1.xx.fbcdn.net/v/t39.30808-6/241764807_378690417183512_5583296113580012927_n.jpg?_nc_cat=105&ccb=1-5&_nc_sid=730e14&_nc_ohc=2S2ANpz-Kh4AX9YtZuU&tn=lCYVFeHcTIAFcAzi&_nc_ht=scontent-frx5-1.xx&oh=f30ae1ab65121aeec6a49b48105f03e8&oe=6140AC40

  ig https://scontent-frt3-2.xx.fbcdn.net/v/t39.30808-6/241760122_378690480516839_4880300264259113370_n.jpg?_nc_cat=103&ccb=1-5&_nc_sid=730e14&_nc_ohc=_c7Y7DUnfwkAX9-T-Au&_nc_ht=scontent-frt3-2.xx&oh=b66b13d10266e6acb1dfaddbea5d8525&oe=614126CC
tab_end
\fi

Але ця «логіка» трохи хибує!  @igg{fbicon.wink}  Адже Закон не предбача не лише «виїзних», але й
«камеральних» та й будь-яких инших «перевірок», замінюючи їх терміном
«здійснювати контроль». І не приписує при цьому, з якого саме фізично місця він
має здійснюватись: із офісу Уповноваженого, хідника коло МАФу, ресторану
«Велюр», круїзного лайнера чи космічного корабля!

Закон просто не деталізує цього, бо в законі не можна детально розписати
абсолютно все!

Відмова проводити контроль за дотриманням законодавства про мову безпосередньо
в місцях обслуговування споживачів під приводом того, що «закон не передбача
виїзних перевірок» – приблизно те ж саме, що не складати актів про порушення
сидячи за столом, бо «закон не передбачає» використання стола в процесі
складення акту. Нема в Законі і чітких вказівок про підписування акту ручкою. А
раз нема, то використання ручки для підписування акту, ймовірно, є
протиправним?..  @igg{fbicon.wink}  

На відміну від поліції та служби газу, Уповноважений дійсно «не зобов’язаний»
приходити на мій виклик, як і будь-хто із його представників. Але як громадяни
України вони мають ПОВНЕ ПРАВО ТА МОЖЛИВІСТЬ зробити це!

Скарга будь-якої людини навіть не є, за Законом, конче потрібною Уповноваженому
для здійсненя контролюючих дій. Закон надає йому можливість робити це «за
власною ініціативою», див. ст. 56 ЗПМ.

Тобто: прийшов собі представник Уповноваженого (не він навіть сам!) на наше
запрошення на базар коло станції метра «Академмістечко» (має ж право там
ходити?..), ПОБАЧИВ вивіску не державною мовою, ЗАХОТІВ «здійснити контроль»,
не відходячи від каси ПРИЙНЯВ РІШЕННЯ це зробити, НА МІСЦІ Ж «здійснив» його,
за столиком поруч склав три документи («акт», «протокол» та «вимогу») в усній
формі «оголосив попередження», особисто передав (не заборонено ж?..) «вимогу»
припинити порушення Закону правопорушнику і... готово!

Друзі, це все виглядає як страшенно «багатоетапний процес», проте цілком може
бути виконано за пів години, чи навіть менше!

Ми – учасники акції – готові були бути СВІДКАМИ ФАКТУ, що порушення МАЛО МІСЦЕ,
«попередження» – оголошене, а «акт» - вручений!

Гаразд, на практиці це не так, УПОВНОВАЖЕНИЙ НЕ РОБИТЬ ЦЬОГО, та РОБИТЬ ВИГЛЯД,
що «не має права». Хоча це – неправда. 

Вони просто хочуть сидіти в офісі або телестудії та чекати, що небайдужі
громадяни ВСЕ ЗРОБЛЯТЬ ЗА НИХ. На жаль...

PS Готовий до дискусії будь-з-ким, хто спробує аргументовано довести, що я не
маю рації!..

