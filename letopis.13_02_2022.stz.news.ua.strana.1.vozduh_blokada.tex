% vim: keymap=russian-jcukenwin
%%beginhead 
 
%%file 13_02_2022.stz.news.ua.strana.1.vozduh_blokada
%%parent 13_02_2022
 
%%url https://strana.news/news/376492-vtorzhenie-v-ukrainu-12-fevralja-hlavnye-vyvody.html
 
%%author_id minin_maksim
%%date 
 
%%tags aviacia,blokada,rossia,ugroza,ukraina,vtorzhenie
%%title Запад готовит воздушную блокаду Украины. Почему из страны вывозят иностранцев и что происходит с "вторжением"
 
%%endhead 
 
\subsection{Запад готовит воздушную блокаду Украины. Почему из страны вывозят иностранцев и что происходит с \enquote{вторжением}}
\label{sec:13_02_2022.stz.news.ua.strana.1.vozduh_blokada}
 
\Purl{https://strana.news/news/376492-vtorzhenie-v-ukrainu-12-fevralja-hlavnye-vyvody.html}
\ifcmt
 author_begin
   author_id minin_maksim
 author_end
\fi

После того, как США в пятницу резко усилили нажим с \enquote{вторжением}, события стали
развиваться стремительно (детально мы их отслеживали в нашем онлайне).

Во-первых, многократно выросло количество стран, которые призвали своих граждан
уехать из Украины. Это главный фактор, создающий панический фон. 

\ii{13_02_2022.stz.news.ua.strana.1.vozduh_blokada.pic.1}

Но связан он скорее всего не напрямую с тем, что все внезапно поверили в
нападение Путина. А с тем, что внешнее авиасообщение с Украиной скоро может
быть прервано. 

Во-вторых, резко усилилась телефонная дипломатия между Москвой и Западом,
кульминацией чего стал новый разговор Путина с Байденом. Во время которого
президент США сделал некие новые предложения по безопасности.

Это значит, что трек военной разрядки в Европе, начатый Москвой и Вашингтоном,
продолжает работать под информационную канонаду о \enquote{вторжении}. 

Подводим главные итоги субботы по теме \enquote{нападения Путина}.

\ii{13_02_2022.stz.news.ua.strana.1.vozduh_blokada.1.ishod_inostrancev}
\ii{13_02_2022.stz.news.ua.strana.1.vozduh_blokada.2.telefonnaja_diplomatia}
\ii{13_02_2022.stz.news.ua.strana.1.vozduh_blokada.3.ukraina_vtorzhenie}




