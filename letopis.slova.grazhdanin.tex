% vim: keymap=russian-jcukenwin
%%beginhead 
 
%%file slova.grazhdanin
%%parent slova
 
%%url 
 
%%author 
%%author_id 
%%author_url 
 
%%tags 
%%title 
 
%%endhead 
\chapter{Гражданин}
\label{sec:slova.grazhdanin}

%%%cit
%%%cit_head
%%%cit_pic
%%%cit_text
\enquote{Германский народ! Национал-социалисты! Обремененный тяжкими заботами,
принужденный молчать месяцами, я дождался часа, когда, наконец, могу говорить
открыто}.  \enquote{\emph{Граждане} и \emph{гражданки} Советского Союза!
Советское правительство и его глава товарищ Сталин поручили мне сделать
следующее заявление: сегодня, в 4 часа утра, без предъявления каких-либо
претензий к Советскому Союзу, без объявления войны, германские войска напали на
нашу страну}. Так начинались два выступления, прозвучавших по радио 22 июня
1941 года. Одно - из Берлина, другое - из Москвы. Несмотря на крайнюю важность
обеих речей, с ними выступили не первые лица государств - Адольф Гитлер и Иосиф
Сталин, - а их ближайшие помощники
%%%cit_comment
%%%cit_title
\citTitle{Как начиналась война Германии против СССР 22 июня 1941 года}, Дмитрий Коротков, strana.ua, 21.06.2021
%%%endcit

