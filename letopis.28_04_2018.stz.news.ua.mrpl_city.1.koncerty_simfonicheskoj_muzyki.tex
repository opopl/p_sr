% vim: keymap=russian-jcukenwin
%%beginhead 
 
%%file 28_04_2018.stz.news.ua.mrpl_city.1.koncerty_simfonicheskoj_muzyki
%%parent 28_04_2018
 
%%url https://mrpl.city/blogs/view/kontserty-simfonicheskoj-muzyki
 
%%author_id burov_sergij.mariupol,news.ua.mrpl_city
%%date 
 
%%tags 
%%title Концерты симфонической музыки
 
%%endhead 
 
\subsection{Концерты симфонической музыки}
\label{sec:28_04_2018.stz.news.ua.mrpl_city.1.koncerty_simfonicheskoj_muzyki}
 
\Purl{https://mrpl.city/blogs/view/kontserty-simfonicheskoj-muzyki}
\ifcmt
 author_begin
   author_id burov_sergij.mariupol,news.ua.mrpl_city
 author_end
\fi

\ii{28_04_2018.stz.news.ua.mrpl_city.1.koncerty_simfonicheskoj_muzyki.pic.1.estrada_v_gorodskom_sadu}

Несмотря на то что Мариуполь был не очень большим и откровенно провинциальным,
многие его обитатели знали толк в музыке, в том числе и классической. Среди
местной интеллигенции – врачей, преподавателей местных гимназий и городских
училищ, чиновников средней руки - их жены и старшие дети на досуге
музицировали: играли на фортепиано, скрипках, виолончелях, других инструментах.
Играли музыку все больше серьезную, для чего выписывали ноты из Петербурга,
Москвы, Харькова и даже из-за границы. Лет этак пятьдесят назад у потомков этих
музыкантов-любителей можно было встретить ноты, отпечатанные на плотной бумаге,
с обложками с гравированными вензелями, витиеватыми надписями на французском
или немецком языках. Нотные страницы, как правило, были испещрены пометами –
стало быть, нотами пользовались. В частной музыкальной школе Каневского – она
помещалась в доме, стоящем чуть выше нынешнего кинотеатра \enquote{Победа} -
преподавание велось по консерваторской программе. Более или менее состоятельные
жители города старались дать своим детям, особенно девочкам, музыкальное
образование. Для них приглашали преподавателей или отдавали в частные школы,
которых было в Мариуполе несколько. Пианино или даже рояли в домах были не
такой уж и редкостью. Газета \enquote{Мариупольский справочный листок} летом 1900 года
анонсировала концерты местного симфонического орке­стра под управлением М. Я.
Свердлова. Оркестр маэстро Свердлова состоял из тридцати музыкантов, в том
числе пяти солистов: скрипач, виолончелист, арфист, гобоист и кларнетист. К
слову, оркестранты были выпускниками Санкт-Петербургской и Московской
консерваторий. 

Листая подшивки дореволюционной газеты \enquote{Мариупольская жизнь}, не раз и не два
наткнешься на объявления о гастролях то концертирующих скрипачей, то пианистов,
то камерных оркестров. Местные врачи, педагоги, инженеры, юристы, получившие
высшее образование в университетах и институтах Санкт-Петербурга, Москвы,
Киева, Харькова, Одессы, были, как правило, достаточно музыкально образованны.
Ведь посещение оперных спектаклей и концертов симфонической музыки было в
прошлом едва ли не главным досугом студенческой молодежи. После Октябрьской
революции огромная масса молодежи из рабочих и крестьян заполнила аудитории
вузов. Их-то и взяли под свою опеку ведущие деятели музыкальной культуры
столичных городов. Читали для студентов лекции, сопровождая их небольшими
концертами. Теперь уже новые студенты заполнили галерки столичных филармоний и
оперных театров. Некоторые из них после, завершения учебы в вузах, оказались в
Мариуполе. Так что слушателей и ценителей серьезной музыки в нашем городе и
после революции было более чем достаточно.

В двадцатые-тридцатые годы ХХ века, уже в советское время, гостями
мариупольской публики были выдающиеся скрипачи-виртуозы \textbf{Мирон Полякин, Михаил
Эрденко, Давид Ойстрах, Борис Гольдштейн}, пианисты \textbf{Лев Оборин, Эмиль Гилельс} и
менее именитые, но, тем не менее, талантливые исполнители. Эти концерты
проходили в переполненных слушателями залах и всегда с большим успехом. Очень
показательный факт из прошлого музыкальной жизни нашего города. Когда в марте
1936 года была завершена реконструкция зимнего театра, на его открытие был
приглашен симфонический оркестр Московской государственной филармонии.
Дирижировали этим прославленным музыкальным коллективом \textbf{Александр Иванович
Орлов}, чья творческая карьера началась еще в дореволюционные годы, а
закончилась за пультом Большого симфонического оркестра Всесоюзного радио, и
\textbf{Борис Эммануилович Хайкин}, который впоследствии стал музыкальным руководителем
Большого театра СССР. На каждом из двух концертов – они состоялись 14 и 15
марта – зрительный зал зимнего театра был полон слушателями...

\ii{28_04_2018.stz.news.ua.mrpl_city.1.koncerty_simfonicheskoj_muzyki.pic.2.estrada_dovoennaja}

Когда-то в Городском саду стояло три могучих дерева. Сейчас осталось одно. Все
три, как определили в свое время ботаники – тутовые деревья. Они как бы
ограничивали музыкальную площадку с деревянными лавками-сиденьями и эстрадой в
виде раковины. Поэтому горожане и называли ее раковиной. Раковина была
сколочена из дерева. Говорили, что под полом эстрады был слой битого стекла.
Будто такое устройство обеспечивало лучший звук, когда на эстраде играли
музыканты. А на ней действительно летними вечерами демонстрировали свое
искусство или местные духовые оркестры (их было несколько в старом Мариуполе)
или заезжие симфонические оркестры. И те и другие пользовались огромным успехом
у горожан.

\textbf{Читайте также:} 

%\href{https://mrpl.city/news/view/v-mariupole-zagovoryat-derevya-foto}{В Мариуполе \enquote{заговорят} деревья, mrpl.city, 19.04.2018}
\href{https://archive.org/details/19_04_2018.mrpl_city.v_mariupole_zagovorjat_derevja}{В Мариуполе \enquote{заговорят} деревья, mrpl.city, 19.04.2018}

Судя по фотографиям на почтовых открытках разных лет, на месте, осененном, как
мы уже знаем, тутовыми деревьями, в разное время стояли две разные раковины.
Одна из них была построена еще до Октябрьской революции. Ее фронтон был
украшен изображением лиры. Это украшение оставалось на своем месте и в
советское время. В годы первых пятилеток к нему добавили изображение фигуры
рабочего перед входом, а также лозунги, содержания, соответствующего, как
тогда говорили, текущему моменту. На открытке начала тридцатых годов ХХ века
можно разглядеть такие призывы: \enquote{Мобилизуем пролетарскую смекалку на успешное
выполнение 5 в 4 года}, \enquote{Ко второму дню индустриализации дадим свои
рационализаторские предложения}, \enquote{Вище соцбудівництво}. На открытках начала
сороковых годов ни лозунгов, ни рабочего из фанеры более не видно. Вместо них
появилось легкое обрамление входа с короткой надписью: \enquote{Искусство в массы}.

Исчезла раковина во время хозяйничанья гитлеровцев в Мариуполе. То ли она
сгорела, то ли жители близлежащих домов растащили ее для отопления своих жилищ
в зиму 1941 – 1942 года, когда свирепствовали невиданные в наших краях морозы.
Потом в городском саду была сооружена новая эстрада (скорее всего сразу после
войны), но не на месте сгоревшей, а с противоположной стороны главной площади
Городского сада. Она представляла собой три кирпичные стены с легкой крышей.
Места для зрителей были окружены довольно массивным забором, сложенным также из
кирпича. Естественно, что это сооружение не отвечало самым минимальным
акустическим требованиям.

Вероятно, между 1946 и 1949 годами была построена новая раковина на том же
самом месте, где стояла ее дореволюционная предшественница. Во всяком случае,
когда в 1949 году впервые после войны в наш город на летние гастроли приехал
симфонический оркестр Сталинской (теперь Донецкой) областной филармонии, свои
концерты он давал уже на новой эстраде. Такие встречи повторялись каждое лето
на протяжении многих лет. Чтобы занять сидячие места перед летней эстрадой,
многие приходили заранее, ну а тем, кому таковые не доставались, стоя слушали
концерт на протяжении полутора-двух часов стоя. Такая публика располагалась в
пространстве от небольшой оградки, окружающей ряды лавок перед эстрадой и почти
до фонтана.

Старожилы города с теплой грустью вспоминают эти концерты. Музыку слушали,
затаив дыхание. Бурными аплодисментами сопровождали исполнение каждого номера.
Были завсегдатаи, которые за время гастролей не пропускали ни одного
концертного вечера. Были и знатоки музыки, которые посещали лишь те концерты,
на которых исполнялись произведения их любимых композиторов. Но и те и другие
были безмерно рады случаю послушать классическую музыку вживую. Правда, среди
слушателей встречались и такие, кто приходил на концерты не из любви к
искусству, а ради моды. Посещение концертов симфонической музыки среди
мариупольской интеллигенции той эпохи было признаком хорошего тона.

Вспоминается еще одна картинка из того времени. Кое-кто из местных
преподавателей музыки, руководителей музыкальных коллективов, студентов и
старшеклассников приходил днем на репетиции оркестра, которые назначались после
посещения оркестрантами пляжа. Дирижер, полистав на пульте партитуру, не
торжественно, как на концертах, а как-то буднично поднимал свою палочку,
призывая свою паству к порядку. Начиналась репетиция. Время от времени он
стучал палочкой по пульту, все замолкали. Тогда-то и делалось замечание
сфальшивившему музыканту. Дирижер называл цифру, с которой следует продолжить
исполнение, и музыка звучала вновь.

Гастрольный сезон начинался, как правило, в конце июня или начале июля.
Продолжался он около месяца. За это время гости успевали дать двадцать пять –
двадцать семь концертов. Для примера приведем программу выступления музыкантов
из областного центра на открытии летнего сезона 1951 года, состоявшегося 23
июня. Итак, вниманию местных меломанов были предложены \textbf{увертюра к опере Лысенко
\enquote{Тарас Бульба}}, \textbf{симфония Овсянико-Куликовского}, \textbf{увертюра к опере Глинки \enquote{Руслан
и Людмила}}, \textbf{\enquote{Половецкие пляски} из оперы Бородина \enquote{Князь Игорь}}, другие
произведения. В тот июньский вечер управлял оркестром его главный дирижер
\textbf{Соломон Фельдман}.

\ii{28_04_2018.stz.news.ua.mrpl_city.1.koncerty_simfonicheskoj_muzyki.pic.3}

Сезон 1955 года открылся 25 июня. В этот вечер были исполнены \textbf{Вторая симфония
Ревуцкого}, уже упоминавшаяся \textbf{увертюра к опере \enquote{Тарас Бульба}}, \textbf{арии из
опер Римского-Корсакова}, \textbf{торжественная увертюра Петра Ильича Чайковского
\enquote{1812 год}}.  Во время этих гастролей оркестром дирижировали \textbf{С.
Фельдман, В. Краснощек, Л.  Авербах}. 17 июля 1955 года оркестр аккомпанировал
нашему выдающемуся земляку – народному артисту Советского Союза,
лауреатуГосударственной премии, солисту Киевского государственного
академического театра оперы и балета \textbf{Михаилу Гришко}.  В тот необыкновенный
вечер Михаил Степанович пел арии из опер \textbf{\enquote{Князь Игорь}, \enquote{Богдан
Хмельницкий}, \enquote{Алеко}, украинские и русские народные песни, романсы}.

Летом 1956 года на эстраду-раковину, на возвышение перед оркестрантами поднялся
незнакомый мариупольцам дирижер. Он был молод, строен и красив. Это был
выпускник Киевской консерватории \textbf{Вадим Гнедаш}. Тогда ему только что исполнилось
двадцать пять лет. Он приезжал в наш город на летние гастроли еще пять лет
подряд, пока работал в Донецкой филармонии. С 1961 года Вадим Борисович Гнедаш
стал художественным руководителем и главным дирижером симфонического оркестра
Украинского республиканского телевидения и радио. И еще один музыкант, знакомый
мариупольским поклонникам классической музыки, у которого начало творческой
деятельности связано с Донецкой областной филармонией – \textbf{Иван Гамкало}. Когда
Ивану Дмитриевичу перевалило за шестьдесят, на его визитной карточке значилось:
генеральный директор - художественный руководитель Национального заслуженного
академического симфонического оркестра Украины, дирижер Национальной оперы
Украины, народный артист Украины, профессор.

Кроме уже перечисленных выше маэстро, мариупольцы имели возможность в разные
годы слушать оркестр, о котором здесь идет речь, под управлением дирижеров, как
постоянно работавших в Донецке – \textbf{Бориса Векслера}, \textbf{Ильи Островского}, \textbf{Валентина
Куржева, Владимира Агафонова, Евгения Заблодского}, так и гастролеров – \textbf{Марка
Павермана} из Свердловска, \textbf{Владимира Кожухаря} из Киева, болгарина \textbf{Илии Темкова} и
многих других.

Сейчас трудно перечислить все произведения, которые прозвучали в исполнении
симфонического оркестра Донецкой филармонии с эстрады Городского сада за годы и
годы его гастролей. Перечислим лишь авторов, чьи имена задержались в памяти.
Зарубежные композиторы Гендель, Гайдн, Моцарт, Бетховен, Берлиоз, Лист,
Дворжак, Брамс, Дебюсси, Равель и отечественные - Глинка, Римский-Корсаков,
Бородин, Чайковский, Танеев, Рахманинов, Лысенко, Ревуцкий, Штогаренко,
Шостакович, Кабалевский, Пейко, Эшпай... 

Нужно отметить, что программы концертов составлялись очень продуманно. В будни,
когда концерты посещали искушенные слушатели, а их у нас в городе было немало,
играли сложные и серьезные номера, ту же \textbf{Пятую симфонию Чайковского} или,
скажем, \textbf{произведения Дебюсси, Берлиоза}. В воскресные же дни, так сказать, для
всех, исполнялась музыка, которую снобы называли \enquote{садовой}. Это \textbf{вальсы Штрауса,
венгерские танцы Брамса, увертюры к известным операм...}

Со временем гастроли донецких музыкантов в нашем Городском саду устраивались
все реже и реже. А потом и вовсе прекратились. Эстрада-раковина по недосмотру
кого-то сгорела. Вместо нее на новом месте построили возвышение, закрытое от
ветра листами железа, окрашенного в разные цвета. И когда года три или четыре
назад к нам в Городской сад вновь приехал симфонический оркестр Донецкой
областной филармонии, впечатление от концертов было подпорчено из-за
неприспособленности новой эстрады...

В историю Мариуполя вписано два эпохальных события. 5 августа 1971 года во
Дворце культуры завода \enquote{Азовсталь} (теперь это ДК \enquote{Молодежный}) этот день дал
концерт выдающийся музыкант ХХ века виолончелист \textbf{Мстислав Ростропович}. 14 марта
1967 года во Дворце культуры \enquote{Искра} состоялся концерт \textbf{Арама Хачатуряна}.
Гениальный композитор дирижировал симфоническим оркестром Донецкой областной
филармонии.

Редкими гостями в нашем городе симфонические оркестры. И иной раз летним
вечером неведомо откуда донесется в открытое окно почти забытый пряный запах
ночных фиалок. И перед внутренним взором вдруг возникает эстрада, освещенная
желтоватым светом ламп, белизна сорочек и блузок оркестрантов, подчеркивающая
бронзовый загар их лиц и рук, дирижер в черном фраке, пробирающийся сквозь лес
пюпитров к своему пульту. По пути он на ходу жмет руку \enquote{первой
скрипки}, поднимается на возвышение, легкий поклон публике, поклон оркестру,
руки его занимают горизонтальное положение: \enquote{Внимание}! Взмах палочки -
и полились далеко окрест звуки чарующей музыки...
