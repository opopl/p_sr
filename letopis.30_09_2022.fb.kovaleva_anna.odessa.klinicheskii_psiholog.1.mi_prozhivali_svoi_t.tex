%%beginhead 
 
%%file 30_09_2022.fb.kovaleva_anna.odessa.klinicheskii_psiholog.1.mi_prozhivali_svoi_t
%%parent 30_09_2022
 
%%url https://www.facebook.com/permalink.php?story_fbid=pfbid02vMSkvdM6kjRSZuJWDBZPc71CCPKUnzmGbtqkshNqpHjLJ6ZTHpdRLyNnziYuUGrjl&id=100007996034347
 
%%author_id kovaleva_anna.odessa.klinicheskii_psiholog
%%date 30_09_2022
 
%%tags 
%%title Мы проживали свои трудные моменты и всегда справлялись, так почему же мы не верим в своих детей?
 
%%endhead 

\subsection{Мы проживали свои трудные моменты и всегда справлялись, так почему же мы не верим в своих детей?}
\label{sec:30_09_2022.fb.kovaleva_anna.odessa.klinicheskii_psiholog.1.mi_prozhivali_svoi_t}

\Purl{https://www.facebook.com/permalink.php?story_fbid=pfbid02vMSkvdM6kjRSZuJWDBZPc71CCPKUnzmGbtqkshNqpHjLJ6ZTHpdRLyNnziYuUGrjl&id=100007996034347}
\ifcmt
 author_begin
   author_id kovaleva_anna.odessa.klinicheskii_psiholog
 author_end
\fi

Наблюдается тенденция делания из наших детей сейчас бедных жертв, которые живут
во время войны, которым выпало тяжелое бремя и поэтому их надо всячески
оберегать и попускать.

Да, баловать, обнимать, поддерживать и все такое надо, безусловно.

Оберегать от всего не получится, столько соломки не найдется.

От самой жизни не уберечься, ее придется жить, иначе потом только за соломинку
и останется держаться...

Мы проживали свои трудные моменты и всегда справлялись, так почему же мы не
верим в своих детей?

Не надо делать из них моральных калек и заранее тяжело психически
травмированных личностей.

...Не за себя боюсь, за детей, они такие сейчас бедные, как им дальше жить...

...Да ладно тебе с этой школой, тут хоть бы выжить...

... Да пусть не моет ту посуду, ему и так выпало жить в это тяжелое время...

Нет, так через год на выходе мы действительно получим ленивых «травматиков»,
созданных нашими же собственными руками.

Детская и подростковая психика еще гибкая и податливая. И очень тонко
воспринимает вибрации взрослых.

Причем, в зависимости от трансляции этих вибраций.

...А, я бедный и несчастный, тогда я могу спекулировать на этом и ничего не
делать.

.... Так война же, завтра может быть ядерный взрыв, на хрена те уроки...

Давайте помнить о том, что детям нужно передавать ответственность за их жизни и
будущее.

И переводить нашу тревогу за них в позитивное русло.

Дайте им другие маркеры, чтобы они могли раскрасить мир в свои краски, которые
им помогут жить счастливо.

Думать и обсуждать с ними темы того - как мы будем жить после победы?

Когда будет мир, то все эти знания тебе понадобятся. Ты сможешь найти призвание
и быть успешным.

Через пару лет ты уже сможешь жить один и у тебя будет навык мыть после себя
посуду.

Будет много рабочих мест и вакансий, давай подумаем, чем бы ты хотел
заниматься?

Будущее никто не отменял. 

И нашим детям его строить.

И от нас напрямую сейчас зависит какая нация выйдет в свет.

По прежнему жертв обстоятельств, которые себя жалеют или нация победителей,
которые выжили в темные времена и продолжали жить, учиться и работать.

Для того, чтобы дать им эти навыки, нужно сначала переформатировать свое
мышление, это труднее всего.

Но я очень верю в наших родителей) потому что если не для себя, то для детей уж
точно они совершат невозможное)

Верим в Победу

Держим строй

Даем детям уверенность в их силах, они справятся.

Да, им выпало жить в нелегкое время, поэтому они уже победители и сильные духом
люди, просто им нужно об этом сказать ♥️

И конечно похвалить себя за стойкость и осознанное родительство)

🇺🇦💙🇺🇦💛
