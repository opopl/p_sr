% vim: keymap=russian-jcukenwin
%%beginhead 
 
%%file 13_10_2021.fb.fb_group.story_kiev_ua.3.pokrov
%%parent 13_10_2021
 
%%url https://www.facebook.com/groups/story.kiev.ua/posts/1774089429454517/
 
%%author_id fb_group.story_kiev_ua,kuzmenko_petr
%%date 
 
%%tags gorod,kiev,pozdravlenie,prazdnik,prazdnik.pokrova
%%title Покров Пресвятой Богородицы
 
%%endhead 
 
\subsection{Покров Пресвятой Богородицы}
\label{sec:13_10_2021.fb.fb_group.story_kiev_ua.3.pokrov}
 
\Purl{https://www.facebook.com/groups/story.kiev.ua/posts/1774089429454517/}
\ifcmt
 author_begin
   author_id fb_group.story_kiev_ua,kuzmenko_petr
 author_end
\fi

От души поздравляю всех православных друзей - одногруппников с великим
Праздником, наступающим через несколько часов Покровом Пресвятой Богродицы!
Здоровья и счастья вам, добрые люди! Оно необходимо всегда, особенно в наше
непростое время. Но, здесь, не об этом. Я намеренно поставил картинку не с
Покровской церковью на родном Подоле, находящейся чуть ниже, которую тоже
люблю. Я, в очередной раз предлагаю полюбоваться на купола самого близкого мне
и многим киевлянам (не только христианам) величественного Андреевского храма. 

В этот Светлый Праздник хочу напомнить всем нам о хрупкости и зыбкости всего
того, что делает наш Город неповторимым, лучшим, самым родным местом во
Вселенной. Сейчас постепенно исчезают с лица земли многие старые и даже древние
здания нашего Подола и других знаковых районов Киева. 

Что можем сделать мы? Я много раз задавался этим вопросом. Пока, в связи со 100
000 рубежом участников нашей любимой, самой душевной и доброй киевской
фейсбучной группы КИ, на мой вопрос кто стал стотысячным юбиляром не получил
ответ администратора Олег Коваль, что пиарщики наших властьпредержащих
сработали чётко и в унисон, подав, практически одновременно, заявки на
юбилейного участника группы от имени Владимира Зеленского и Виталия Кличко.
Надеюсь, их приняли в группу. И, возможно они прочитают этот пост. 

Но, сейчас и это не главное. Ведь стотысячная аудитория реальных людей, пусть и
просто пользователей Фейсбука, не тоько нынешних но и киевлян из разных городов
и стран, это реальная сила. И даже, если большинство из нас выскажет своё
мнение в нашей и других киевских группах, да просто на своих страницах, с этим
придётся считаться. Власти, обществу и пусть в небольшой степени, но людям от
которых сейчас зависит будущее Киева. Я очень хочу, чтобы эти люди отметили для
себя, что это будущее зависит от них в какой-то мере только сейчас, на мизерном
промежутке более чем полуторатысячной истории нашего вечного Города. Он всё
преодолеет. Но, хотелось бы, чтобы в нём остались для потомков, не только
охраняемые ЮНЕСКО, значимые исторические памятники, но и места, пусть не всегда
яркие и презентабельные, но дорогие сердцу каждого истинного Киевляниа. 

С Праздником, друзья! Пусть Пресвятая Богородица укроет от горестей и невзгод,
своим величественным защитным покровом, наш любимый неповторимый Город и всех
любящих его людей!

