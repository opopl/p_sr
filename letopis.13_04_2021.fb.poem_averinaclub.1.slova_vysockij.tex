% vim: keymap=russian-jcukenwin
%%beginhead 
 
%%file 13_04_2021.fb.poem_averinaclub.1.slova_vysockij
%%parent 13_04_2021
 
%%url 
 
%%author 
%%author_id 
%%author_url 
 
%%tags 
%%title 
 
%%endhead 

\subsection{Красивых любят чаще и прилежней}
\label{sec:13_04_2021.fb.poem_averinaclub.1.slova_vysockij}


\ifcmt
  pic https://scontent-bos3-1.xx.fbcdn.net/v/t1.6435-9/172970527_1877242209096428_3241654953143103958_n.jpg?_nc_cat=104&ccb=1-3&_nc_sid=8bfeb9&_nc_ohc=clAcmIhw9VkAX8HCXeO&_nc_ht=scontent-bos3-1.xx&oh=b2c69c6debfc5012be0da98b27f5a90c&oe=609C32B0
\fi

Красивых любят чаще и прилежней, 
Весёлых любят меньше, но быстрей, 
И молчаливых любят, только реже, 
Зато уж если любят, то сильней. 
Не кричи нежных слов, не кричи, 
До поры подержи их в неволе. 
Пусть кричат пароходы в ночи, 
Ну а ты — промолчи, помолчи, 
Поспешишь — и ищи ветра в поле. 
Она читает грустные романы. 
Ну, пусть сравнит, и ты доверься ей. 
Ведь появились чёрные тюльпаны, 
Чтобы казались белые белей. 
Не кричи нежных слов, не кричи, 
До поры подержи их в неволе. 
Пусть поэты кричат и грачи, 
Ну а ты — помолчи, промолчи, 
Поспешишь — и ищи ветра в поле. 
Слова бегут, им тесно — ну и что же! 
Ты никогда не бойся опоздать. 
Их много — слов, но всё же, если можешь, 
Скажи, когда не можешь не сказать. 
Но не кричи этих слов, не кричи, 
До поры подержи их в неволе. 
Пусть кричат пароходы в ночи. 
Замолчи, промолчи, помолчи, 
Поспешишь — и ищи ветра в поле.

Владимир Высоцкий 
Красивых любят чаще и прилежней
