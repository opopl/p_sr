% vim: keymap=russian-jcukenwin
%%beginhead 
 
%%file 07_12_2021.fb.fb_group.story_kiev_ua.2.vasyl_stus_film
%%parent 07_12_2021
 
%%url https://www.facebook.com/groups/story.kiev.ua/posts/1813940082136118
 
%%author_id fb_group.story_kiev_ua,fedjko_vladimir.kiev
%%date 
 
%%tags film,stus_vasyl
%%title «Ваш Василь»… Чесний фільм про Василя Стуса
 
%%endhead 
 
\subsection{«Ваш Василь»... Чесний фільм про Василя Стуса}
\label{sec:07_12_2021.fb.fb_group.story_kiev_ua.2.vasyl_stus_film}
 
\Purl{https://www.facebook.com/groups/story.kiev.ua/posts/1813940082136118}
\ifcmt
 author_begin
   author_id fb_group.story_kiev_ua,fedjko_vladimir.kiev
 author_end
\fi

«Ваш Василь»... Чесний фільм про Василя Стуса.

23 серпня 2019 року, у День Державного прапора України і напередодні Дня
Незалежності, я записався у документальному фільмі про Василя Стуса – «Ваш
Василь». 

\ii{07_12_2021.fb.fb_group.story_kiev_ua.2.vasyl_stus_film.pic.1}
\ii{07_12_2021.fb.fb_group.story_kiev_ua.2.vasyl_stus_film.pic.1.cmt}

Моя місія як військового історика та історика діяльності спецслужб полягала в
тому, щоб розказати глядачам про організацію репресій проти українського
визвольного та дисидентського руху керівництвом КПРС/КПУ і виконання цієї
злочинної партійної політики органами КДБ СРСР/УРСР; розкрити механізм
оперативно-розшукової діяльності (ОРД) спецслужб Радянського Союзу,
використовуваний проти українських патріотів і дисидентів, зокрема проти Василя
Стуса.

\begin{multicols}{2} % {
\setlength{\parindent}{0pt}

\ii{07_12_2021.fb.fb_group.story_kiev_ua.2.vasyl_stus_film.pic.2}
\ii{07_12_2021.fb.fb_group.story_kiev_ua.2.vasyl_stus_film.pic.2.cmt}

\ii{07_12_2021.fb.fb_group.story_kiev_ua.2.vasyl_stus_film.pic.3}
\ii{07_12_2021.fb.fb_group.story_kiev_ua.2.vasyl_stus_film.pic.3.cmt}
\end{multicols} % }

Атмосфера запису, створена режисером Світланою Рудюк та її колегами, була дуже
дружньою і творчою! Загалом провів у студії більше семи годин – з 13:30 до
21:15. В чистому вигляді я відповідав на запитання режисера і розповідав годин
п’ять. Було дві перерви на каву і одна на спільний обід.

\ii{07_12_2021.fb.fb_group.story_kiev_ua.2.vasyl_stus_film.pic.4}
\ii{07_12_2021.fb.fb_group.story_kiev_ua.2.vasyl_stus_film.pic.4.cmt}

Безумовно, що у фільм могла увійти тільки мізерна частина з моїх розповідей,
але я вважав необхідним дати максимально повну інформацію про організацію і
керівництво репресіями з боку Ідеологічного відділу Комуністичної партії;
звітування керівництвом КДБ про виконання вказівок партійних органів; про
оперативно-розшукові засоби, які використовувалися при цьому (оперативні
обліки, наявний агентурний апарат та довірені особи, вербування агентури з
близького оточення фігурантів, перлюстрація кореспонденції, прослуховування
телефонів і приміщень, зовнішнє спостереження з фото та  кіно-документацією,
агентурна розробка засуджених патріотів в місцях їхнього ув’язнення тощо). 

\ii{07_12_2021.fb.fb_group.story_kiev_ua.2.vasyl_stus_film.pic.5}
\ii{07_12_2021.fb.fb_group.story_kiev_ua.2.vasyl_stus_film.pic.4.cmt}

В процесі запису я чітко наголосив, що організатором репресій в Радянському
Союзі з 1917-го і до розпаду СРСР у 1991-му була Комуністична партія, яка і
несе, в першу чергу, відповідальність за масові репресії, позасудові розстріли
та ГУЛАГ! А  орденом мечоносцем –  виконавцем злочинної політики партії, – був
«озброєний загін партії»: ВЧК – НКВД – МГБ – КГБ!

\ii{07_12_2021.fb.fb_group.story_kiev_ua.2.vasyl_stus_film.pic.6}
\ii{07_12_2021.fb.fb_group.story_kiev_ua.2.vasyl_stus_film.pic.4.cmt}

У створенні фільму взяли участь: Василь Овсієнко, Левко Лук’яненко, Михайло
Горинь, Марія Стус, Дмитро Стус, Борис Довгань, Маргарита Довгань, Володимир
Федько, Іван Драч, Віктор Медведчук.

\ii{07_12_2021.fb.fb_group.story_kiev_ua.2.vasyl_stus_film.pic.7}
\ii{07_12_2021.fb.fb_group.story_kiev_ua.2.vasyl_stus_film.pic.4.cmt}

***

Через рік, 24 серпня 2020 року, під аркою Дружби народів, в рамках фестивалю
«Молодість», відбулася прем'єра фільму «Ваш Василь», присвяченого українському
поету і політв'язню Василю Стусу, вбитому у радянському політичному концтаборі
в Кучино. 

\ii{07_12_2021.fb.fb_group.story_kiev_ua.2.vasyl_stus_film.pic.8}
\ii{07_12_2021.fb.fb_group.story_kiev_ua.2.vasyl_stus_film.pic.4.cmt}

Фільм чудовий і, головне – чесний! В ньому ми бачимо реального Стуса та цинічну
жорстокість радянської системи. ЗАМОВНИК РЕПРЕСІЙ проти української
інтелігенції – КОМУНІСТИЧНА ПАРТІЯ РАДЯНСЬКОГО СОЮЗУ!

Після прем'єри команда фільму відповіла на запитання глядачів. Перед присутніми
також виступили Марія Стус, Василь Овсієнко, Маргарита Довгань.

А я поспілкувався з командою фільму і висловив їм подяку за гарно зроблений
фільм! 

***

Команда фільму:

Авторка ідеї та сценаристка: Світлана Рудюк.

Режисери: Світлана Рудюк, Олександр Авшаров.

Оператори: Христина Лизогуб, Костянтин Пономарьов.

Художниця: Ірина Шумейко.

Композитор: Степан Бурбан.

Монтаж: Анна Перепелиця, Максим Рудюк.

Продюсер: Артем Денисов.

Компанія виробник: UM-Group, Kyiv, Ukraine.

***
\ii{07_12_2021.fb.fb_group.story_kiev_ua.2.vasyl_stus_film.cmt}
