% vim: keymap=russian-jcukenwin
%%beginhead 
 
%%file 16_04_2021.fb.barashev_roman.1.djachenko_lidia_smert
%%parent 16_04_2021
 
%%url https://www.facebook.com/roman.barashev/posts/2299575496841583
 
%%author 
%%author_id 
%%author_url 
 
%%tags 
%%title 
 
%%endhead 
\subsection{Лидия Николаевна Дьяченко}
\Purl{https://www.facebook.com/roman.barashev/posts/2299575496841583}

Мир оставил замечательный человек, выдающийся дипломат, русская женщина с украинской фамилией. Лидия Николаевна Дьяченко.

\ifcmt
  pic https://scontent-mad1-1.xx.fbcdn.net/v/t1.6435-0/s640x640/174721881_2299510960181370_2516308438676267318_n.jpg?_nc_cat=104&ccb=1-3&_nc_sid=730e14&_nc_ohc=3xNaVy82UxIAX9BDdbj&_nc_ht=scontent-mad1-1.xx&tp=7&oh=70ac401ddc3ad365bf7ded45122558d7&oe=60A3882D
\fi

Человек из Острова надежды, как она называла центр русской культуры в Киеве,
который возглавляла до совсем недавнего времени и который недавно было решено
\enquote{потопить} с высочайшего повеления.

Мало осталось таких, как она, людей с улыбающейся миру душой, так искренне,
прямо и открыто. В ее глазах горел вечный жизнерадостный девиз "Любовь,
комсомол и весна!" 

В ее сердце, в ее центре, в Киеве на Подоле звучали романсы и песни советской
эстрады, Шопен, Рахманинов и Свиридов и песни Великой Отечественной, и песни
бардов Киева, Москвы, Питера, Харькова, Одессы, Владивостока... Как-то и она
мне один мотив дворовой песни напела -- помнишь, такую? Нет, не вспомнил.

Когда мои архаровцы читали стихотворения в день рождения клуба \enquote{Занимательная
журналистика для школьников}, Лидия Николаевна всей душой откликнулась на
стихотворение Евгения Евтушенко \enquote{Не надо бояться!}

Она хотела, чтобы мы не боялись. Любить, дружить, жить.

Одно только исключительное радушие-очарование -- её ответ миру на его маразмы, смерти и запугивания. 

Мы вспоминали Олеся Бузину, когда разговаривали в последний раз, совсем недавно. И вот... Ушли в один день...     

Нам будет Вас не хватать, Лидия Николаевна!

Благодарим Вас!

Царствие Небесное!

И мы ещё споём!

Не надо бояться густого тумана,
Не надо бояться пустого кармана.
Не надо бояться ни горных потоков,
Ни топей болотных, ни грязных подонков!

Не надо бояться тяжёлой задачи,
А надо бояться дешёвой удачи.
Не надо бояться быть честным и битым,
А надо бояться быть лживым и сытым!

Умейте всем страхам в лицо рассмеяться, -
Лишь собственной трусости надо бояться!
