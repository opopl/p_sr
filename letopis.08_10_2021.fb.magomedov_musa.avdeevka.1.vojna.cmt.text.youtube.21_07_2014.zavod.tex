% vim: keymap=russian-jcukenwin
%%beginhead 
 
%%file 08_10_2021.fb.magomedov_musa.avdeevka.1.vojna.cmt.text.youtube.21_07_2014.zavod
%%parent 08_10_2021.fb.magomedov_musa.avdeevka.1.vojna.cmt
 
%%url 
 
%%author_id 
%%date 
 
%%tags 
%%title 
 
%%endhead 

\begin{multicols}{2}

Около 21:00 в результате перестрелки днр с нацгвардией, снаряд попал на
территорию завода, повредив коммуникации, остановлены вспомогательные цеха,
продукты коксования выбрасывались в воздух

ситуация на 22.07.14: 

Благодаря слаженным и профессиональным действиям сменного персонала завода и
сотрудников авдеевского подразделения МЧС удалось локализовать пожары и
избежать человеческих жертв. В настоящий момент пожары потушены, ведутся
восстановительные работы на поврежденных участках. Ожидается, что на ликвидацию
последствий потребуется не менее двух суток. Главная задача - полностью
восстановить электроснабжение завода.

В течение этого времени выпуск кокса на АКХЗ будет сокращен наполовину. Тем не
менее, в целом такая ситуация не повлияет существенным образом на работу ряда
металлургических предприятий Группы, использующих продукцию АКХЗ -
Енакиевского метзавода и Мариупольского металлургического комбината имени
Ильича. Руководство Металлургического дивизиона Группы Метинвест оперативно
разработало схему замещения поставок кокса на эти заводы продукцией других
коксохимических предприятий холдинга - Днепродзержинского, Енакиевского
коксовых заводов и коксовых батарей комбината «Азовсталь».

Сегодня утром авдеевские коксохимики вышли на работу в соответствии с
утверждёнными графиками. Вспомогательные службы предприятия осуществляют свою
деятельность в обычном режиме.

Генеральный директор Группы Метинвест Юрий Рыженков:

- Мы приложим все усилия, чтобы максимально быстро вернуть производство в
нормальный ритм. Авдеевский коксохим - это химически опасное производство.
Случившееся на заводе потенциально может нести риск уничтожения целого города
Авдеевки. И в этом смысле совершенно неважно, с какой стороны велся огонь. В
такой ситуации последствия военных действий могут оказаться одинаково
трагичными для всех участников конфликта. Уничтожение АКХЗ, который производит
40\% кокса в стране - это не просто риск для Группы Метинвест. Это настоящая
катастрофа для всей металлургии и, как следствие, для экономики Украины. Кроме
того, остановка завода повлечет за собой безработицу для десятков тысяч людей в
Авдеевке, Донецке, Мариуполе, Кривом Роге. Мы еще раз призываем немедленно
остановить подобные военные действия, отказаться от разрушения предприятий,
дорог, линий электропередач, трубопроводов. Это может поставить города и весь
Донбасс на грань выживания.

\end{multicols}

