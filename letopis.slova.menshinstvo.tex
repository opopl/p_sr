% vim: keymap=russian-jcukenwin
%%beginhead 
 
%%file slova.menshinstvo
%%parent slova
 
%%url 
 
%%author 
%%author_id 
%%author_url 
 
%%tags 
%%title 
 
%%endhead 
\chapter{Меньшинство}

%%%cit
%%%cit_head
%%%cit_pic
%%%cit_text
Поэтому, собственно говоря, успешное будущее Украины зависит от этой узкой
прослойки, которая может еще развернуть Украину в какой-то конструктив. Это,
скажем так, не проблема в том, что эти люди в \emph{меньшинстве}, потому что в
принципе всегда меньшинство определяло курс в любой стране. Просто идеи
\emph{меньшинства} со временем становятся идеями большинства, для этого
требуется время.  Но вот эта совокупность багов и совокупность наследия,
которое нам досталось от Российской империи, от СССР, усугубились тем, что хомо
советикус превратился в такой хомо жлобикус в украинских реалиях. Когда самое
худшее что было в совке, потому что в совке ж были и нормальные моменты, чего
там греха таить...
%%%cit_comment
%%%cit_title
\citTitle{Украина в плену консервативного мышления: почему 30 лет шли не туда и что делать}, 
Сергей Иванов; Юрий Романенко, hvylya.net, 15.07.2021
%%%endcit

