% vim: keymap=russian-jcukenwin
%%beginhead 
 
%%file 22_03_2021.stz.news.ua.mrpl_city.1.teatralna_artil_dramkom
%%parent 22_03_2021
 
%%url https://mrpl.city/blogs/view/teatralna-artil-dramkom
 
%%author_id demidko_olga.mariupol,news.ua.mrpl_city
%%date 
 
%%tags 
%%title Театральна артіль "ДрамКом"
 
%%endhead 
 
\subsection{Театральна артіль \enquote{ДрамКом}}
\label{sec:22_03_2021.stz.news.ua.mrpl_city.1.teatralna_artil_dramkom}
 
\Purl{https://mrpl.city/blogs/view/teatralna-artil-dramkom}
\ifcmt
 author_begin
   author_id demidko_olga.mariupol,news.ua.mrpl_city
 author_end
\fi

\ii{22_03_2021.stz.news.ua.mrpl_city.1.teatralna_artil_dramkom.pic.1}

З нагоди Всесвітнього дня театру я вирішила присвятити свої нариси найбільш
яскравим театральним колективам міста. Почну з Театральної артілі \enquote{ДрамКом},
колектив якої до 27 березня готує справжній сюрприз для своїх глядачів.
\enquote{Драмком} став справжнім відкриттям для маріупольців і вже на першій прем'єрі
викликав величезний інтерес у глядачів.

Театр був створений \emph{10 жовтня 2019 року}. Засновницею Театральної артілі є
\emph{\textbf{Наталя Олександрівна Гончарова}} – репетитор з техніки мови в Донецькому
академічному обласному драматичному театрі (м. Маріуполь) та викладач у
Маріупольському коледжі мистецтв. Наталя Гончарова також відома глядачам
завдяки режисерським роботам в Народному театрі \enquote{Театроманія} (\enquote{Готель двох
світів}, \enquote{Королівські ігри}, \enquote{Викрадення Дженні}, \enquote{Мій великий кенгуру}).

Театральна артіль \enquote{ДрамКом} сповідує традиційні театральні цінності і є
потенційно репертуарним колективом. Артіль прагне, щоб її вистави були різними
за жанром та естетичної спрямованості, при цьому підбирає унікальний ключ до
кожного драматурга. Дати маріупольському глядачеві якісний театральний продукт
і атмосферу, до якої захочеться повертатися знову і знову – головна мета, яку
переслідує \enquote{ДрамКом}.

Театр складається з 16 акторів (\emph{Ганна Авдєєва, Тетяна Федорова} (обидві
закінчують навчання на акторському відділенні коледжу мистецтва), \emph{Юлія Корсун,
Ольга Новікова, Ганна Кучеренко, Євген Сосновський, Андрій Бурябаш, Микита
Армен, Роман Ерінський, Дмитро Синяков} (закінчив акторське відділення коледжу),
\emph{Ігор Волоський, Дмитро Синяков, Ігор Грецький}). Нещодавно до колективу
приєдналися \emph{Тетяна Цикун, Ксенія Гузь, Олена Чорноморець} (солістка ансамблю
української народної музики\par\noindent\enquote{Шумка}). Всі актори театру харизматичні та
обдаровані. Взяти хоча б талановитого маріупольського фотографа Євгена
Сосновського, образи якого стали улюбленими для багатьох маріупольців. Нині у
репертуарі театру наступні вистави:  \enquote{Контрабас розбушувався}, \enquote{Мій великий
кенгуру}, \enquote{Та, що осяяла темряву}. У 2000 році театр брав участь у першому
міжнародному театральному інтернет-конкурсі \enquote{Театр.NET}, де був відзначений
дипломами за постановку і режисуру в номінації \enquote{Аматорський театр}.

\ii{22_03_2021.stz.news.ua.mrpl_city.1.teatralna_artil_dramkom.pic.2}

Цікаво, що \emph{\textbf{Олександр Володарський}} – відомий радянський і український
письменник, драматург і сценарист, сам запропонував Наталі Гончаровій поставити
виставу за його твором. Маріупольська режисерка з радістю погодилася. Так в
репертуарі  \enquote{ДрамКому} з'явилася перша вистава – \enquote{Контрабас розбушувався}, на
яку, до речі, Олександр Володарський планує приїхати.

Сьогодні театр працює у приміщенні Маріупольської камерної філармонії.
Родзинкою театру є співпраця з \emph{\textbf{камерним оркестром \enquote{Ренесанс}}}, який майже у
повному складі з'являється на сцені у виставі \enquote{Контрабас розбушувався}. А
головний диригент оркестру \emph{\textbf{Василь Крячок}} навіть грає у цій виставі справжню
роль – роль Диригента.

До Міжнародного дня театру, який відзначається 27 березня, колектив готує
прем'єру \textbf{\emph{\enquote{Та, що осяяла темряву}}}, яка нікого не залишить байдужим. У виставі
задіяні майже всі актори Театральної артілі \enquote{ДрамКом}. П'єса американського
драматурга Вільяма Гібсона \enquote{Та, що створила диво} сподобалася режисерці ще,
коли вона навчалася в університеті. Це реальна історія вчительки Енн Саліван та
її учениці Хелен Келлер, яку вдалося завдяки великим зусиллям, терпінню і,
головне, любові –  перевиховати! Це історія боротьби і перемог, любові і
ненависті.

Цікаво, що всі костюми для акторів у відповідності з вимогами тієї епохи, яка
зображується у п'єсах, баченням постановника і драматурга, а також особистістю
актора створила актриса і дизайнерка одягу Театральної артілі \enquote{ДрамКом} \textbf{\emph{Юлія
Корсун}}.

\ii{22_03_2021.stz.news.ua.mrpl_city.1.teatralna_artil_dramkom.pic.3}

Спектакль \enquote{Та, що осяяла темряву} окремо планують ставити для людей з
порушенням слуху. У виставі багато мовчазних сцен, без музики, без тексту,
актриси часто розмовляють руками, адже вони вивчили мову жестів для цього
спектаклю. Саме тому, на думку художньої керівниці та режисерки театру Наталі
Гончарової, після декількох вистав вони хочуть долучити до роботи
сурдоперекладача і запросити тих людей, які не завжди можуть чути театр.
Вистава стане справжнім подарунком для всіх театралів і дуже актуальна для
батьків, вихователів, вчителів. Вдруге спектакль відбудеться 30 квітня.

Попереду у театральної артілі \enquote{ДрамКом} ще багато планів і задумів. Головне, що
колектив готовий надалі приємно вражати свого глядача.
