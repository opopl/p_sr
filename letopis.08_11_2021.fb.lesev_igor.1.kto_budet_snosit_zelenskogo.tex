% vim: keymap=russian-jcukenwin
%%beginhead 
 
%%file 08_11_2021.fb.lesev_igor.1.kto_budet_snosit_zelenskogo
%%parent 08_11_2021
 
%%url https://www.facebook.com/permalink.php?story_fbid=4758936350804097&id=100000633379839
 
%%author_id lesev_igor
%%date 
 
%%tags politika,strana,ukraina,zelenskii_vladimir
%%title Кто будет сносить Зеленского
 
%%endhead 
 
\subsection{Кто будет сносить Зеленского}
\label{sec:08_11_2021.fb.lesev_igor.1.kto_budet_snosit_zelenskogo}
 
\Purl{https://www.facebook.com/permalink.php?story_fbid=4758936350804097&id=100000633379839}
\ifcmt
 author_begin
   author_id lesev_igor
 author_end
\fi

Кто будет сносить Зеленского

На одной из первых своих пресс-конференций Зеленский как-то обронил журналисту
фразу «вы хотите, чтобы я остался один»? Тогда это была еще одна ложечка
пафоса, а вот сегодня это уже почти реальность. Блогер таки остается один,
вернее, в окружении с теми дураками, с которыми он так много всего нашел.

\ifcmt
  ig https://scontent-mxp1-1.xx.fbcdn.net/v/t39.30808-6/254792110_4758936214137444_8613889892019184327_n.jpg?_nc_cat=108&ccb=1-5&_nc_sid=730e14&_nc_ohc=3QwNpVgRptsAX9tPXmP&_nc_ht=scontent-mxp1-1.xx&oh=a553120ffe7b72d5df01e2765f701988&oe=618DC220
  @width 0.4
  %@wrap \parpic[r]
  @wrap \InsertBoxR{0}
\fi

Итак, против Вовчика сложились самые разношерстные коалиции. Первая, назовем ее
широко Ахметовской. Это уже четко видно по движухе на его каналах. Первой
выстрелила Скороход, которая фактически обвинила власть (Зеленского) в убийстве
нардепа Полякова. Интервью получилось ярким, но все же, главной темой там был
даже не покойный Поляков, а отставной Разумков. О нем много не говорили, но
обозначили базовое – вокруг него будут лепить АнтиЗеленского. И на той же
неделе со старта нарисовали Диме 8\%. Учитывая, что против Разумкова практически
нет никакого токсичного говнеца – офшоры, «снял квартиру у тещи» и пр. – это
вполне перспективный задел. По крайне мере, пока нет на горизонте никого более
узнаваемого и перспективного, кушайте вот это.

Дальше бомбанул Аваков. Думаю, многие смотрели его словесный камбэк после
обидной для него отставки. Сугубо в электоральном плане Авакова рассматривать,
естественно, бессмысленно. Вряд ли у него есть какой-то персональный рейтинг. А
если и есть, то приблизительно как у Яценюка, недаром оба были в одной конторе.
А с другой стороны, какой рейтинг есть у Шмыгаля? А у Арахамии вместе с
Корниенко? Или у Ермака? Аваков – это фигура в украинской политтусовке, которая
оказывает влияние непубличными методами. Он ведь никогда не был «просто главой
МВД». Просто глава МВД у нас Монастырский. И Аваков в отставке, это тоже не
«просто еще один отставной политик». И вот теперь мы точно знаем, что Аваков
очень плохо попрощался с Зеленским.

Кроме того, Аваков успел обозначить всю клиентелу Ахеметова. Юля и ее
«Батькивщина», упомянутый уже Разумков, Гройсман, Смешко и подзабытый своим
колоритом Ляшко. Да, специфическая компашка, но когда кот в мешке оказался
Зеленским, и этот веселый балаган становится вполне конкурентными оппонентами.

Следующая антизеленая коалиция – Петровская. И ведь это самая боевая и
сплоченная секта. Если ее не вытравили за первые два с половиной года, то
ничего уже не сделают и теперь. Более того, Порошенко на данным момент второй
по популярности политик в стране. А ведь какие-то два с половиной года назад
казалось, что персонаж будет постоянным клиентом прокуратуры, а то и вовсе
после очередного туристического вояжа забудет вернуться домой. Но Зеленский
сотворил невероятное.

Знаете, будь я Станиславским, то не поверил бы, что Петро сумеет до следующих
президентских сохранить свои позиции и выйти с Вовой во второй тур. Не верю
даже не столько в Петра – это как раз упорная шельма – сколько в зеленого
вундеркинда. И все же, будь две жизни, очень бы хотелось увидеть, как во второй
тур выходят Зеленский и Порошенко и там побеждает Петро. Не то, чтобы сильно
хочется реставрации петровского расизма, но посмотреть за репортажами, как
зеленая шобла штурмует Борисполь и Жуляны, ну согласитесь же, по-голливудски
дорогая картинка.

Третья коалиция, назовем ее условно Соровской. У пацанвы нет ярко выраженного
электорального представительства. Иногда просачиваются прожекты вроде «Голоса»,
но это так, на злобу дня и для особо утомленных от секты Петра. К тому же,
персонажи там очень неоднозначные и их настроение зависит от источника
кормления. Лещенко, Милованов или Безуглая пристроены хорошо и попа Зеленского
их полностью устраивает. А Рябошапка, Гончарук или Данилюк из очень обиженных,
а потому и попа Блогера для них давно уже стала жопой.

Впрочем, сила Соровской коалиции не в электоральном перетягивании голосов, а в
информационных атаках. Их прелесть в разгар президентской кампании ярко ощутил
тот же Петро. Можете представить, чтобы нечто подобное по тому же
«Укроборонпрому» опубликовали какие-то «ватные» журналисты? При Петре в СИЗО
сидели по статье «госизмена» и за более невинные публикации. И что такое
соровские информационные атаки, уже прекрасно ощущает на себе зеленая команда.
Грозев, похоже, уже озолотился на Ермаке с историей о «вагнеровцах». А офшорная
тема хорошо так срезонировала на рейтинге Зеленского.

И ведь в Офисе их по-настоящему боятся. Во-первых, потому что их нельзя
закрыть, они внешние. А во-вторых, непонятно с кем и на каких условиях
договариваться. Всех не купишь, да и набсоветов на всех не напастись. А потому
и прилетает по самой непредсказуемой траектории. К тому же, каждая такая
информационная атака – это сигнал всей местной элитке, что дурачка можно рвать.
И рвут ведь.

Ну и четвертая коалиция, назовем ее собирательно Московской. Все эти ОПЗЖ,
шарийцы и просто неприкаянные «ватники», имя которым Легион. Формально и не
формально это самая бесправная пацанва постмайдановской Украины. Их стреляли,
жгли, садили, а сегодня ими «просто» пугают из каждой подворотни. Если
проводить аналогию с кастами в Индии, то это такие местные шудры.

Но реальность такова, что без Московской коалиции в Украине решить хоть
какую-то системную проблему невозможно. Вот по определению, даже не буду
расписывать, почему. Никакого Большого договора, ни вопроса войны и мира, вот
ничего без «ватников» тут не будет. А что будет? Вот то, что сейчас есть, вот
такое и будет.

Но мы ведь сейчас о Зеленском. А без «ваты» у него тоже все выходит грустно.
Московская коалиция не может провести своего кандидата на высший пост, это
очевидно. Но ее слово становится решающим, когда из всех вариантов выбирается
самый умеренный. Блогер в этом плане давно уже потерял берега и топчется на
чуждом огороде Петра. А этот огород удобрял не Зеленский, и не ему там собирать
урожай. А назад уже тоже хер вернешься. Он одними только своими
«поздравлениями», даже не плюет, а прямой наводкой срет в души «ватникам». Нет,
возможно, ему это даже приятно. Но персонаж ведь хочет зайти на второй срок, мы
же об этом, а не о каких-то высоких материях.

И вот со всей этой гигантской коалицией нашему пластилиновому Наполеону
предстоит схлестнуться. Понятно, что где-то будут пытаться договориться (это же
Украина), а где-то точечно бить врага по отдельности. Но ситуация таково, что
Блогер и его инициативные дураки объединили против себя практически всех.

Зеленский пытается играть в юного Путина, но по итогу каждый раз у него все
равно выходит Зеленский. Много причин, почему так. Начиная с того, что мало и в
России, и у нас смогут назвать с ходу главу администрации президента РФ, потому
что он обслуга, а не управляющий Путиным. И заканчивая кучмовским «Украина – не
Россия». В общем, Путин не лепится.

Но также не лепится и все остальное. Два с половиной года прошло – а мы так и
не знаем, что строит Зеленский и Ермак. Собственно, они и сами не знают.
Оставили идеологическую модель Петра, перенесли Раду и Кабмин в Офис, и
окучивают индивидуальное кормление для причастных. Занимательная такая
получается Украина, в которой нет места всем остальным. Поэтому, и появилось
коллективное желание указать заслуженное место зеленым вундеркиндам.

\url{https://t.me/Lesev_Igor}
