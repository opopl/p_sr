% vim: keymap=russian-jcukenwin
%%beginhead 
 
%%file 16_11_2021.fb.fb_group.story_kiev_ua.2.an70.cmt
%%parent 16_11_2021.fb.fb_group.story_kiev_ua.2.an70
 
%%url 
 
%%author_id 
%%date 
 
%%tags 
%%title 
 
%%endhead 
\zzSecCmt

\begin{itemize} % {
\iusr{Volodymyr Pultsyn}

Все це дуже добре, але... чому Ви не написали про те, що завадило цьому
проекту? Не тільки політичне рішення ФРГ. Були і конструктивні недоліки і
особливо цікава історія з двигунами для цього літака. Хоча сам по собі літак є
дуже цікавий, перспективний і добрий.

Та все ж таки, в його, поки що, не дуже вдалої історії, винні корупція, совкові
метастази та вимушена співпраця з московитськими 'партнерами' на той час.

Дуже бажаю цьому проекту нового життя та відродження!

\begin{itemize} % {
\iusr{Eugen Zlyvko}
\textbf{Volodymyr Pultsyn} 

«Он мог бы прямо на митингах деньги зарабатывать, – мутно мечтал пес, –
первоклассный деляга" С таким набором штампов и устойчивых оборотов точно мог
бы на митингах выступать  @igg{fbicon.beaming.face.smiling.eyes}{repeat=3}  \enquote{московитськие партнёры} это нечто
 @igg{fbicon.beaming.face.smiling.eyes}{repeat=3} 

\begin{itemize} % {
\iusr{Volodymyr Pultsyn}
\textbf{Eugen Zlyvko}  @igg{fbicon.face.tears.of.joy}{repeat=3}  подивись у дзеркало- побачиш штампи!))))

\iusr{Юрий Стебельский}
\textbf{Eugen Zlyvko} 

прикольно, когда русские люди на Руси добровольно начинают, как Владимир
Пыльцин, балакать на мове а не на своём родном, русском языке.

Уверен, что на тридцатый год незалежности Косово там появятся албаномовные
сербы, которые будут считать себя этническими косоварами - жертвами
многовековой сербизации.

\iusr{Volodymyr Pultsyn}
\textbf{Yuriy Stebelskiy} 

дуже смішно, коли всеякі 'социальні експєрди' такі як Ви, починають розкидати
навколо себе лайно, вважаючи свою думку якоюсь цінністю. Доречі, штампи про всі
ці наративи, щодо 'русских людей' і т.п. є певною ознакою невиглавства, тем
більш, коли ви звертаєтеся до незнайомої людини, нічого не розуміючи ні про
неї, ні про будь-які обставин її життя та життєвого шляху.

Це дуже склизька доріжка, бо іноді трапляється так, що можливо придется і
вибачатися, а іноді і з сльозами... @igg{fbicon.face.rolling.eyes} 

 · Ответить · Поделиться · 3 д.
\iusr{Надежда Владимир Федько}
+2
\textbf{Юрий Стебельский} 

Цікаво, звідки походять ці \enquote{русские люди на Руси}? Московія походить від
угро-фінів і ніколи не мала нічого спільного з Київською Руссю.

\iusr{Eugen Zlyvko}
\textbf{Volodymyr Pultsyn} 

ты прям по Геббельсу вещаешь))) значит у меня и лицо неподходящее, форма
черепа?) Володимир, тебя бы на 70 лет назад  
@igg{fbicon.thumb.up.yellow}{repeat=3}  @igg{fbicon.wink} 

\iusr{Юрий Стебельский}

Дорогая русская барышня Надежда Владимир, на свете существует только один
исторический источник по древней Руси. Это Ипатьевская летопись. Единственный
сохранившийся исторический документ.

И в нём написано, что первые русские князья пришли в хазарский на тот момент
Киев из Новгорода - из Новгородской Руси.

Это грустный факт, но некоторым его надо мужественно принять:

\enquote{ - Чей се град буде?  - Кия. А платим мы дань хозарам.}

(Нестор летописец).

\iusr{Volodymyr Pultsyn}
\textbf{Yuriy Stebelskiy} заткнить собі ції 'аргументи' в дупу і живіть спокійно  @igg{fbicon.face.tears.of.joy}{repeat=3}  @igg{fbicon.hands.applause.yellow}{repeat=3} 

\iusr{Костянтин Усенко}
\textbf{Yuriy Stebelskiy} Дехто з читачів навіть розуміє різницю між тими, чий град й кому \enquote{дань платят}.

\iusr{Юрий Стебельский}
\textbf{Костянтин Усенко} 

но лишь немногие читали Ипатьевскую летопись - \enquote{Повесть временных лет.}

\enquote{Се повесть временных лет. Откуда пошла русская земля, кто в Киеве нача первее
княжити и откуда русская земля стала есть.}

Зная только мову, Повесть прочитать не получится. Она написана на русском
языке, который для этого необходимо знать.

\iusr{Костянтин Усенко}
\textbf{Yuriy Stebelskiy} 

Глибше помилки у аматорів російської версії історії Київської Русі не буває:
\enquote{написана на русском}

Великорусское наречие, як його означив укладальник першого словника цієї мови,
почало складатись по завоюванні частин України та Литви Московським царством
під проводом Петра І - останнього московського царя.

\iusr{Volodymyr Pultsyn}
\textbf{Yuriy Stebelskiy} що москалики, дуже зачепило?)))))
А нужбо їще нам щось про нас розкажіть!!! @igg{fbicon.face.tears.of.joy}{repeat=3} 

\iusr{Юрий Стебельский}

Дорогой селянин Костянтин Усенко, ты хочешь сказать, что Ипатьевская летопись
написана на мове?  А скинь ка сюда в подтверждение любой фрагмент текста из
\enquote{Повести временных лет.}


\iusr{Юрий Стебельский}
\textbf{Volodymyr Pultsyn} 

(Владимир Пыльцин) что про вас - вырусь, рассказывать ещё? Вы сами о себе
трещите на каждом углу из каждой сковородки.

О янычарах ещё поразузнавай. О манкуртах справки наведи.

Забавно наблюдать, как некоторые русские люди на Руси добровольно становятся
нерусью.

\iusr{Костянтин Усенко}
\textbf{Yuriy Stebelskiy} 

Дешёвый бомжара, разобрался бы, чем церковнославянский отличался от
разговорного.

Да и чем княжеский род Рюриковичей отличался от автохтонного населения.

Що Рюрик, що хазари, однаково зайди були на київських землях.

А історію завжди пише переможець. Недарма глаголицю викреслено було з писемної
практики.


\iusr{Юрий Стебельский}
\textbf{Костянтин Усенко} 

дорогой хазарянин печенежский, русскость свою на Руси позабывший - в
тринадцатом веке монгол-татары вырезали всех автохтонных на Киевщине. Кто смог
и успел уйти в Русь Залесскую - тот спасся и уцелел.

А после на обезлюдевшую Малую Русь потянулись переселенцы, разные там угры,
валахи и другие чернобривые, родства не знающие, русскими себя не считающие.

Живите себе на Руси. Щёки только не надувайте сильно.


\iusr{Андрій Дідик}
\textbf{Eugen Zlyvko} бл*ть, опять этот великорусский бред. В эту ахинею верят только тупые наши соседи.
Из угро-финских краев приши русские князья... анекдот.

\iusr{Eugen Zlyvko}
\textbf{Андрій Дідик} эммммм, это точно мне комментарий адресован?))
\iusr{Volodymyr Pultsyn}
\textbf{Eugen Zlyvko} тобі, тобі, проковтни вже, чудило!!! @igg{fbicon.face.tears.of.joy}{repeat=3} 

\iusr{Eugen Zlyvko}
\textbf{Volodymyr Pultsyn} Володька, иди ханки лакни, бо ты под утро трезветь начинаешь  @igg{fbicon.beaming.face.smiling.eyes}{repeat=3}  плесень, блять  @igg{fbicon.face.tears.of.joy} 

\iusr{Volodymyr Pultsyn}
\textbf{Eugen Zlyvko} 'володька' тебе під тамбовом чекає у овразі @igg{fbicon.face.tears.of.joy}{repeat=3} 

\iusr{Костянтин Усенко}
\textbf{Yuriy Stebelskiy} Озлобленість в дискусії є однією з ознак невіри у власну аргументацію. А з ухилом у мовні викривлення демонструє, наскільки слабо відрізняється суспільство країни побєдівшего соціалізьма від суспільства країни націонал-соціалізьма.

\iusr{Олег Курилов}
\textbf{Надежда Владимир Федько} какой же бред Вы несёте....

\end{itemize} % }

\iusr{Василь Гордієнко}

Європейський тендер. Багато що можно сказати з цього приводу - від ідеї,
складнощив політичних, військових, технічних до успіхів та причин обрання
європейцями іншого проекту свого європейського літака. Вивчаючи АН-70-й ,
європейці дивувалися технічним рішенням талановитих українських інженерів, які
потім таки перейшли в іхній власний проект літак. Але й наші конструктори
збагатилися досвідом західного літакобудування та дуже швидко втілили його в
конкретному літаку. Головний тормоз - цього започаткового співробітництва -
непередбачуванність політики РФ! Європейці зрозуміли небезпеку проекту АН-70-го
для себе через його занадто глибоку зав'язку на РФ - комплектуючі, виробничі
можливості, фінансування в основному через МО РФ... Був й негативний вплив
інших російських КБ. Саме в цей час почала завершуватися єльцінська
демократизація. Тому європейці виявилися більш передбачливими, немаючи
впевненності у майбутньому воєнно-політичному курсі РФ та у співробітництві з
недоімперією, яка \enquote{почала вставати з колін}.

\begin{itemize} % {
\iusr{Volodymyr Pultsyn}
\textbf{Василь Гордієнко} Дякую за дуже правильну оцінку та аналіз цього питання!!!

\iusr{Юрий Стебельский}
\textbf{Василь Гордієнко} 

зачем брехать, товарищ? Половина интеллектуальных наработок Ан-70 принадлежит
России, 90 процентов комплектующих российские. Авиаконструкторы Антонов,
Балабуев, Кива - русские.

Сам авиазавод Антонова был российским и после Великой Отечественной войны был
подарен Украине и перевезен в Киев вместе с Антоновым.

Невозможность строить Ан-70 обусловлена нарушением Украиной Будапештского
меморандума - нарушение внеблоковости и стремление в НАТО.

Маемо тэ, що маемо.

\iusr{Василь Гордієнко}
\textbf{Yuriy Stebelskiy} Ми все и наши заводи советские. А кто-то оочень советский.

\iusr{Юрий Стебельский}
\textbf{Василь Гордієнко} 

на тридцатом году незалежности все заводы до сих пор советские?

Стыд то какой, товарищ.

А что из этих заводов хоть осталось? Веркон, Арсенал, ХТЗ, Южмаш, Буревестник?
АТЭК, КРЗ? Николаевская судоверфь? Киевский мотозавод? ЛАЗ?

Велосипеды \enquote{Украина} Украина хотя бы производит?

\enquote{А що, товарищ, помогли вам новые заводы построить ваши ляхи?}

\iusr{Василь Гордієнко}
\textbf{Yuriy Stebelskiy} 

Юра, ты ж всё про \enquote{подарки} российские Украине рассказываешь. А я тебе про то,
что мы вышли из эсэсэсэра и все было советское, колхозное. А потом все делили
\enquote{по-братски}. Например, из закордонной недвижимости \enquote{братским} народам не
досталось аж ничего! Украине все здания для посольств пришлось покупать. Как и
другим \enquote{братьям} впрочем. А чехи и словаки все чехословацкое поделили
цивилизовано. И обошлось без щедрых \enquote{подарков}!

\iusr{Юрий Стебельский}

Не свисти, наци(онали)ст Василь Гордієнко. Укронационалисты радостно готовы
были выскочить из СССР на любых условиях. Лишь бы поскорее и чтобы не
затягивать с платой по долгам СССР. Они быстро получили свою незалежность и
успели на ней обогатиться. А простонародье теперь разводит руками.

\iusr{Андрій Дідик}
\textbf{Yuriy Stebelskiy} 

не свисти, разграбили ве те, кто был у кормушки, комуняки и комсомольцы.

Остальеое само подохло, так как ничего конкурентноспособного сделать не могло

\end{itemize} % }

\iusr{Валентина Герасимова}
\textbf{Volodymyr Pultsyn} 

Не надо гнать херни, суть в том, что именно рашка сделала все чтобы равалить
дело, начиная с вывоза техдокументации и юрдич. перепонов по собственности
АНТОНОВ... все к чему прикладывается рука плешивого, там и разруха даже в
собственном доме.

\begin{itemize} % {
\iusr{Юрий Стебельский}

Дорогая русская женщина \textbf{Валентина Герасимова}, а разве не украшка слила
айрбасу всю техдокументацию и ноу-хау, в результате которой айрбас построил
копию Ан-70 - Airbus A400M?

Всё, к чему протянут руку хуторяне - протягивает ноги.

\iusr{Volodymyr Pultsyn}
\textbf{Yuriy Stebelskiy} 

'русскую женщину' ти довбень зустріш десь на Валдаї
@igg{fbicon.face.tears.of.joy}{repeat=3} 

\iusr{Владимир Каледин}
\textbf{Валентина Герасимова} 

уверены!? У них-то, как раз, все предприятия работают! В том числе и авиапрома!
Интересно, что же там \enquote{не работает}, к чему \enquote{плешивый}, как вы
выразились, \enquote{прикоснулся} !!!? Это у нас ничего уже за последние восемь лет ни
хера не
работает, благодаря \enquote{патриоту} Порошенко и его кодлы \enquote{глашатаев российской
агрессии} и благодаря \enquote{гнусавому} сейчас! А \enquote{Антонову} и его собственности
ничего не угрожало до 2004 года! Пока не пришли к власти \enquote{национал-
патриотческие} запроданцы западных хозяев!

\iusr{Валентина Герасимова}

Если откинув психоз и нетерпимость, то факты ; еще Кучма начал сливать АО
Антонов ему так выкручивали руки.. а чтоб не вильнул, документацию более
70\%вывезли в рф, юридически создали СП с Антонов и жопа... Показательно что АНы
сегодня делает и обслуживает только Киев в рфии отказались и это факт, авионика
и сер. оборудование получают из Франц., потму Антонов имет заказы, а рашка
болт, звиняюсь-заклепку..

\iusr{Корытин Александр}
Не до деланная

\iusr{Валентина Герасимова}
\textbf{Корытин Александр} 

Доделаный ты наш, смотри отчет Росстат.. упадок полнейший, это у Соловьева все
хорошо-догнали и перегнали..

\end{itemize} % }

\iusr{Oleksandr Kalinin}
\textbf{Volodymyr Pultsyn} 

Наскільки відомо, унікальні характеристики Ан-70 мають місце в т.ч. завдяки
Д-27. Цей двигун часто наводиться в іноземних підручниках як ілюстрація до
гвинто-вентиляторних двигунів. Цікаво, Д-27 це вже труп на користь варіанту з
ТВРД (Ан-188), чи ще є якась надія, що Д-27 в якійсь інкарнації доведуть колись
до розуму?

\end{itemize} % }

\iusr{Кирилл Антонов}
Классный самолёт. Жаль, что не производится серийно.

\iusr{Юрий Петров}
анекдот был, почему у самолёта такой острый нос ?

\begin{itemize} % {
\iusr{Вика Нурибекова}
\textbf{Юрий Петров} у них у всех такой, когда делают \enquote{первые шаги}

\iusr{Дмитрий Вакуленко}
\textbf{Юрий Петров} это ДУАС, датчик угла атаки и скольжения.
\end{itemize} % }

\iusr{Дмитрий Вакуленко}

В 2011-2012году я и моя напарница Серинэ Гурджян были ведущими технологами по
модернизации навигационного и электронного оборудования АН-70. Помню, когда
начинался проект никто не захотел (из технологов) брать его, а мне было
интересно. Бонусом нам дали кабинет в 10м цеху, дальний огромный ангар, мы
компанией молодёжи переехали туда. Это была своя романтика. Самолет очень
интересный, особенно в деталях. Чем больше я им занимался тем больше я находил
нестандартных и уникальных решений, которые были применены только в этом
самолёте. Назначения для некоторых систем я так и не понял, их знали только
старейшины, которые раньше принимали участие в проектировании этого самолёта.

\iusr{Сусанна Ковальская}

Історія літака - це прекрасно! А яке НАДСУМНЕ сьогодення!!! СОРОМ НАМ, не
зберегли, не примножити, не покращили.....

\iusr{Вика Нурибекова}
\textbf{Сусанна Ковальская} 

це і є політичні рішення. Зараз знов розглядає Уряд закордонні інвестпроекти,
проте в майбутньому як пощастить з геополітикою невідомо.

\end{itemize} % }

