% vim: keymap=russian-jcukenwin
%%beginhead 
 
%%file 26_03_2019.stz.news.ua.mrpl_city.1.volodymyr_kozhevnikov_kerivnyk_prof_teatru
%%parent 26_03_2019
 
%%url https://mrpl.city/blogs/view/volodimir-kozhevnikov-kerivnik-profesijnogo-teatru-mariupolya
 
%%author_id demidko_olga.mariupol,news.ua.mrpl_city
%%date 
 
%%tags 
%%title Володимир Кожевніков: керівник професійного театру Маріуполя
 
%%endhead 
 
\subsection{Володимир Кожевніков: керівник професійного театру Маріуполя}
\label{sec:26_03_2019.stz.news.ua.mrpl_city.1.volodymyr_kozhevnikov_kerivnyk_prof_teatru}
 
\Purl{https://mrpl.city/blogs/view/volodimir-kozhevnikov-kerivnik-profesijnogo-teatru-mariupolya}
\ifcmt
 author_begin
   author_id demidko_olga.mariupol,news.ua.mrpl_city
 author_end
\fi

З нагоди чудового свята – Всесвітнього дня театру, який щорічно відзначається
27 березня, пропоную познайомитися ближче з виваженим і відповідальним
керівником професійного театру Марі\hyp{}уполя – \textbf{Володимиром Володимировичем
Кожевніковим}, який з 2016 року обіймає посаду генерального директора Донецького
академічного обласного драматичного театру (м. Маріуполь).

\ii{26_03_2019.stz.news.ua.mrpl_city.1.volodymyr_kozhevnikov_kerivnyk_prof_teatru.pic.1}

Володимир народився в селі Урада Куєдинського району Перм\hyp{}ського краю, куди його
батьки-медики були спрямовані за розподілом на роботу. Мама, Валентина
Федорівна, працювала фельдшером-акушеркою, батько,  Володимир Олександрович,
завідував лікарнею. Цікавитися музикою маленький Володимир почав ще в
дитинстві, вже в 4 роки він отримував призи на регіональних конкурсах. А ще у
хлопчика були улюблені іграшки – маленькі піаніно і скрипка, і, почувши по
радіо мелодію, він намагався відразу її підібрати. Оскільки у віддаленому селі
отримати музичну освіту можливостей не було, сім'я переїхала на батьківщину
мами – до Маріуполя. Мама влаштувалася на станцію швидкої медичної допомоги,
батько почав працювати в ковальському цеху \enquote{Азовмаша}, а юний Володимир вступив
до музичної школи № 5 по класу акордеона. У Маріуполі на сім'ю чекало
поповнення – народилася молодша донечка Наталя. Після закінчення музичної школи
і восьмирічки хлопець вступив до музичного училища.

\ii{insert.read_also.demidko.krjachok}

Вищу освіту Володимир Кожевніков отримав на диригентсько-оркестровому
відділенні Орловської філії Московського державного інституту культури. У роки
навчання Володимир був учасником студентського ансамблю народних інструментів,
який щороку виїжджав з концертами в сільські райони. За щасливим збігом
обставин його майбутня \textbf{дружина Наталя} вчилася на тому ж відділенні. Молоді люди
одружилися ще в період студентства, в 1985 році. Завдяки справжньому коханню та
спільним інтересам (обидва музиканти) їм вдалося створити міцну і щасливу
родину. Наталя в усьому підтримує чоловіка і пишається ним. Вільний час
Володимир полюбляє проводити з родиною та друзями на природі.

\ii{26_03_2019.stz.news.ua.mrpl_city.1.volodymyr_kozhevnikov_kerivnyk_prof_teatru.pic.2}

Після випуску наш герой проходив службу в місті Броди Львівської області
радіорозвідником у військах спецпризначення. На службі не вистачало музики, але
цей період став важливим етапом на шляху становлення Володимира як особистості.

\ii{26_03_2019.stz.news.ua.mrpl_city.1.volodymyr_kozhevnikov_kerivnyk_prof_teatru.pic.3}

Працювати в Маріупольському театрі Володимир Володимирович почав з 1991 року.
На посаді завідувача музичною частиною він пропрацював плідних 25 років.
Неабиякий талант і невтомна праця допомогли йому створити музичне оформлення
понад ста вистав, урізноманітнити театральний репертуар, написавши авторську
музику для багатьох постановок. Зокрема, завдяки Володимиру Кожевнікову
маріупольський глядач побачив вперше на сцені обласного театру мюзикли \enquote{Острів
скарбів} та \enquote{Червоні вітрила}. Роботи композитора неодноразово відзначалися
нагородами на театральних конкурсах і фестивалях. А у 2012 році він став
переможцем конкурсу \textbf{\enquote{Маріуполець року}} в номінації \enquote{За розвиток культури та
внесок в духовне виховання містян}.

\ii{insert.read_also.demidko.teatr_okupacia_1}

Авторська музика Володимира Кожевнікова відрізняється оригінальністю музичних
тем, вишуканістю стилю і форми, проникливою мелодійністю. Робота композитора
перетворила музику в дієву опору режисерського задуму, створюючи ритмічну та
емоційну основу акторської гри. Бездоганний музичний смак і високий
професіоналізм притаманні аранжуванням, зробленим В. Кожевніковим.
Маріупольські глядачі захоплюються здатністю композитора відчувати події у
виставі, вміти точно передати потрібну атмосферу.

\ii{26_03_2019.stz.news.ua.mrpl_city.1.volodymyr_kozhevnikov_kerivnyk_prof_teatru.pic.4}

Творчий підхід, виняткова відповідальність, надійність і зацікавленість завжди
відрізняли Володимира Володимировича в роботі з режисерами та артистами театру.
А ще багатий досвід роботи і відповідна освіта привели його до заслуженого
обрання у 2016 році художнім керівником і генеральним директором Донецького
академічного обласного драматичного театру (м. Маріуполь).

\ii{insert.read_also.demidko.teatr_okupacia_2}

Завдяки зусиллям генерального директора у театрі виникла не\hyp{}обхідна робоча
атмосфера. Володимир Володимирович створює потрібні умови для багатогранного
розвитку акторів. Сьогодні репертуар театру відрізняється високим рівнем,
розмаїттям, багатою палітрою художніх засобів, які використовуються під час
постановки вистав. Водночас драматичні актори відчули прагнення до вокального
самовираження. Більшу частину постановок театру зараз можна вважати повноцінно
музичними. Колектив Донецького академічного обласного драматичного театру (м.
Маріуполь) на чолі з Володимиром Володимировичем готовий дивувати
маріупольського глядача, \enquote{розкриватися} та яскраво виявляти свої здібності у
спільній справі, тому ми можемо очікувати від них нових мистецьких звершень та
творчих успіхів.

\ii{26_03_2019.stz.news.ua.mrpl_city.1.volodymyr_kozhevnikov_kerivnyk_prof_teatru.pic.5}

Особисто для мене маріупольський театр посідає особливе і неповторне місце в
історії українського мистецтва. І я відкрито пишаюся тим, що театральна
культура Маріуполя відрізняється унікальністю і неабиякою самобутністю, а
завдяки мудрому керівництву Володимира Володимировича професійний театр міста
лише примножує свої здобутки.

\ii{insert.read_also.demidko.saburov}

\textbf{Улюблені книги Володимира Володимировича:} \enquote{Мистецтво війни} Сунь-Дзі, \enquote{Соляріс}
Станіслава Лема.

\ii{26_03_2019.stz.news.ua.mrpl_city.1.volodymyr_kozhevnikov_kerivnyk_prof_teatru.pic.6}

\textbf{Курйозний випадок з життя:} Після закінчення музичної школи Володимир
Володимирович засумнівався у виборі професії. Тоді батько взяв його з собою на
роботу і влаштував екскурсію ковальським цехом. Юнак був вражений, в яких
умовах працюють робітники, йому здавалося, що від гуркоту земля йде з-під ніг.
Саме тоді він зрозумів, що завод – це точно не його професія.

\textbf{Улюблені фільми:} \enquote{Список Шиндлера} Стівена Спілберга (неймовірний саундтрек).

\textbf{Порада маріупольцям:} \enquote{Любіть себе, любіть свою родину, любіть свою країну. Не
забувайте театр, бо він теж потребує любові! Тоді все буде добре!}

\ii{insert.read_also.demidko.mariupolchanky}
