% vim: keymap=russian-jcukenwin
%%beginhead 
 
%%file 30_01_2022.fb.fb_group.story_kiev_ua.1.ulica_bogdana_hmelnickogo_10.cmt
%%parent 30_01_2022.fb.fb_group.story_kiev_ua.1.ulica_bogdana_hmelnickogo_10
 
%%url 
 
%%author_id 
%%date 
 
%%tags 
%%title 
 
%%endhead 
\zzSecCmt

\begin{itemize} % {
\iusr{Dmitriy Gernak}
Мой родной дом. Спасибо за пост.

\iusr{Анатолий Борозенец}
\textbf{Dmitriy Gernak} Будь ласка.

\iusr{Ксения Погорлецкая}
В Советское время балконы были. В здании находился Институт педагогики. В конце 70-начале 80-х сотрудники выходили на балконы.

\begin{itemize} % {
\iusr{Валерий Лысенко}
\textbf{Ксения Погорлецкая} Мой отец с сотрудниками

\ifcmt
  ig https://scontent-frx5-1.xx.fbcdn.net/v/t39.30808-6/272807083_5042892565755287_903200249462058807_n.jpg?_nc_cat=105&ccb=1-5&_nc_sid=dbeb18&_nc_ohc=7CNGF6lgk6gAX-1LmXv&_nc_ht=scontent-frx5-1.xx&oh=00_AT8YcrBSXxo4Mm9kdR4wxwAFAzkxgptU-kjfdrHtNgeLGA&oe=61FAA54A
  @width 0.3
\fi

\iusr{Ксения Погорлецкая}
\textbf{Валерий Лысенко} Он работал в институте педагогики? Фамилию не вспоминаю. Работала там в 77-79 г.г. в секторе физики.

\iusr{Анатолий Борозенец}
\textbf{Ксения Погорлецкая} Так балкони були, принаймні, в 1950-1970-х. Ось цей знімок зроблено з балкону будинку №10

\ifcmt
  ig https://scontent-frt3-2.xx.fbcdn.net/v/t39.30808-6/273112584_3099629656959414_746420271947957747_n.jpg?_nc_cat=103&ccb=1-5&_nc_sid=dbeb18&_nc_ohc=fUipxcuIbiIAX9a_VIE&_nc_ht=scontent-frt3-2.xx&oh=00_AT-V-tBw1iQh3F3w3Fg2FeKl3tSZWHxQbkdCP3PxkBkGCg&oe=61FBE72C
  @width 0.2
\fi

\begin{itemize} % {
\iusr{Ксения Погорлецкая}
\textbf{Анатолий Борозенец} Институт работал - кажется - до середины 90 в этом здании. Балконы, наверное, были демонтированы после. В 95 жила в этом доме и балконы еще были - вспомнила!

\iusr{Ксения Погорлецкая}
Фото возвращают в те годы, в наш цветущий Киев.

\iusr{Анатолий Борозенец}
\textbf{Ксения Погорлецкая} Мешканці будинку цілком довіряю в цьому питанні ) Я орієнтувався на фотографії, що мав, а їх не так багато. Дякую за уточнення!
\end{itemize} % }

\end{itemize} % }

\iusr{Ольга Арканова-Чёрная}

Не правда - балконы были и мы выходили на них. И учебники там писали правильные
и правдивые. Потом здание УНДИПа продали и лаборатории перевели на окраину
города. Конечно! Кому нужна педагогика!

\begin{itemize} % {
\iusr{Alexey Novozhylov}
\textbf{Olga Arkanova-Chornaya} Про Фінляндію я так розумію теж пісалі учєбнікі правільниє і правдівиє??  @igg{fbicon.face.tears.of.joy}{repeat=3} 

\iusr{Татьяна Соловьева}
\textbf{Ольга Арканова-Чёрная} Моя мама работала в УНДИПе с 1974 по 2006 год. Ленина, 10 - столько раз приходила к ней на работу)

\iusr{Ксения Погорлецкая}
\textbf{Татьяна Соловьева} Возможно, помню Вашу маму.

\iusr{Ольга Арканова-Чёрная}
\textbf{Татьяна Соловьева} у меня отец возглавлял лабораторию методики физики

\iusr{Tamara Sorochan}
\textbf{Ольга Арканова-Чёрная} да, перевели, но не так уж и на окраину. Січових стрільців, 52А.
\end{itemize} % }

\iusr{Ирина Дрюкова}
Спасибо за информацию это мой родной дом я здесь родилась

\iusr{Анатолий Борозенец}
\textbf{Ирина Дрюкова} Будь ласка.

\iusr{Наталия Жминко Сычевска}

На счет Аллы Тарасовой - на Печерске, в районе Генеральной прокуратуры, была
школа номер 7. Там еще есть табличка, что ее закончила Тарасова. Или ?

\begin{itemize} % {
\iusr{Maryna Chemerys}
\textbf{Natalia Zhmin'ko Sychevskaya} 

Алла Тарасова вчилась спочатку у Печерській приватній гімназії на Різницькій
2/34, а потім у приватній гімназії Титаренко на Фундуклеєвській

\iusr{Наталия Жминко Сычевска}
\textbf{Maryna Chemerys} спасибо.

\iusr{Ксения Погорлецкая}
\textbf{Maryna Chemerys} В этом здании на Резницкой впоследствии была вечерняя школа? Здание давно снесено.

\iusr{Татьяна Гурьева}
\textbf{Наталия Жминко Сычевска} 

Печерська приватна гимназия - майбутня школа \#7. Там був музей, прекрасне
примищення. Але зараз бизнес-центр. Все знищено

\iusr{Ольга Патлашенко}
\textbf{Наталия Жминко Сычевска} 

Школу, вернее гимназия, где училась Тарасова, давно, лет 10 назад, благополучно
снесли

На ее месте давно высочеет огромный бизнес - центр. Зачем же нам исторические
памятники, памятники архитектуры? Печерск! Земля то дорогая!

\end{itemize} % }

\iusr{Валерий Лысенко}

По цьому будинку є купа інформації, але ж тут, напевне, не можна ставити
посилання?

\begin{itemize} % {
\iusr{Elen Tsarovska}
\textbf{Валерий Лысенко} 

i додай - документальных свiдчень, а не недолугих фантазiй \enquote{на тему трикутника}
(як у автора) @igg{fbicon.face.smiling.eyes.smiling} 

\begin{itemize} % {
\iusr{Анатолий Борозенец}
\textbf{Elen Tsarovska} 

Пані має долугі фантазії з цього приводу? Поділіться ними, будь ласка. Чи може
взагалі не вмієте фантазувати?  @igg{fbicon.smile} 

\iusr{Elen Tsarovska}
\textbf{Анатолий Борозенец} 

не маю бажання фантазувати i \enquote{высасывать великие истины из... пальца}.
Волiю спиратися на документ та факти.

\iusr{Анатолий Борозенец}
\textbf{Elen Tsarovska} 

А хіба я в дописі стверджував, що то є істина та ще й велика? Це Ви точно із
пальця висмоктали ) Чорним по білому ж написав \enquote{трохи пофантазую}. До того ж
допис у Фейсбуці це не дисертація і навіть не реферат, де мають бути
посиланнями на першоджерела, правда ж?

\iusr{Анатолий Борозенец}
\textbf{Elen Tsarovska} 

І не мають бажання фантазувати, зазвичай, ті, хто цього не вміє, але чомусь
береться оцінювати фантазії інших )

\iusr{Elen Tsarovska}
\textbf{Анатолий Борозенец} 

берусь ще й борюсь. Бо цi \enquote{фантазii} вже геть зовсiм \enquote{задрали}
Знаэте вислiв: за деревами лicу не видно... Запевняю, що пересiчна публiка
читаэ виключно fb, а не дисертацii... От в головах чужi фантазii помноденi на
власнi i производят до жахливоi кашi. У Вас пристойний цiкавий фактаж. Навiщо
Вам дешева \enquote{робота на публiку}? А стосовно моix фантазiй- чудова тема:
спадкоэмцем \enquote{треугольника} в радян. часи став \enquote{красный
треугольник} Уявляете - трикутник, ще й червоний!!! @igg{fbicon.face.zany} Бажаю успiху. I дарнемно
не ображайтесь

\iusr{Анатолий Борозенец}
\textbf{Elen Tsarovska} 

Давно боретеся? А хто ще в групі, крім мене, Вас так задрав? Я так розумію, що
результати боротьби маєте слабенькі, якщо так Вас задрали. Не думали поміняти
методи боротьби? Почніть писати щось своє. І так, щоб без фантазій. Тільки сухі
факти і посилання на першоджерела. Обіцяю почитати. Теж бажаю успіху в Вашій
нелегкій боротьбі. Я на Вас зовсім не ображаюсь, не вигадуйте. Чи це з Вашого
боку була спроба пофантазувати, Олено? )


\iusr{Elen Tsarovska}
\textbf{Анатолий Борозенец} У мене iншi методи боротьби,

\iusr{Анатолий Борозенец}
\textbf{Elen Tsarovska} Інші? Які саме?
\end{itemize} % }

\iusr{Анатолий Борозенец}
\textbf{Валерий Лысенко} Спробуйте. Або доповніть, будь ласка, у форматі коментарів.

\iusr{Elen Tsarovska}
Впевнена, що Вам уже не стане в нагодi

\iusr{Анатолий Борозенец}
\textbf{Elen Tsarovska} А я впевнений, що у Вас фантазії не вистачить навіть, щоб придумати ті \enquote{методи} )

\end{itemize} % }

\iusr{Татьяна Гурьева}
Было дело, жили здесь мои родственники. Шикарный дом

\begin{itemize} % {
\iusr{Александр Длугач}
\textbf{Татьяна Гурьева} был шикарным...

\iusr{Татьяна Гурьева}
\textbf{Александр Длугач} да, шикарный Киев исчезает на глазах
\end{itemize} % }

\iusr{Любовь Подлесная}
Здесь
жила моя одноклассница. Здесь находился паспортный стол и я получала свой первый паспорт. И ещё много др. Родной район.

\iusr{Ксения Погорлецкая}
\textbf{Любовь Подлесная} 

Паспортный стол все еще есть в прежнем помещении и даже внутри мало что
изменилось с тех лет когда мы получали свои первые паспорта (примерно в одно
годы). Там все родное.


\iusr{Татьяна Данилова}
Дякую за докладну розповідь, дуже цікаво і корисно ! @igg{fbicon.thumb.up.yellow}{repeat=3} 

\iusr{Анатолий Борозенец}
\textbf{Татьяна Данилова} Будь ласка! Дякую за відгук!

\end{itemize} % }
