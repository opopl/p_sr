% vim: keymap=russian-jcukenwin
%%beginhead 
 
%%file 03_12_2021.fb.drjomov_aleksandr.kiev.1.tatarooligarh
%%parent 03_12_2021
 
%%url https://www.facebook.com/aleksandr.driomov/posts/6456653454405733
 
%%author_id drjomov_aleksandr.kiev
%%date 
 
%%tags batyj.han,jaroslav_mudryj,jumor,kiev,kievljane,oligarh,verhovna_rada,zakon
%%title .. и слеза катилась по лицу, по лицу татароолигарха..
 
%%endhead 
 
\subsection{.. и слеза катилась по лицу, по лицу татароолигарха..}
\label{sec:03_12_2021.fb.drjomov_aleksandr.kiev.1.tatarooligarh}
 
\Purl{https://www.facebook.com/aleksandr.driomov/posts/6456653454405733}
\ifcmt
 author_begin
   author_id drjomov_aleksandr.kiev
 author_end
\fi

.. и слеза катилась по лицу,

по лицу татароолигарха..

021121

Однажды, Ярослав Мудрый, прогуливаясь по крутым холмам Киева, молвил окружившим
его киевлянам:

\ii{03_12_2021.fb.drjomov_aleksandr.kiev.1.tatarooligarh.pic.1}

- Аль не построить церковь имени себя, иль права не имею я?

- Ой лепо, кнезе, лепо паче чаяния! - вскричали киевляне, - строй!

- Аз, есмь, великий, князь! - грозно ответствовал Ярослав, кротко добавив, -
посему строить будете вы, а за это я вам буду платить серебром, ибо его
непереводно у меня.

- Ну так бы и речил сразу! - гаркнули кияне, - и выстроили церковь, и назвали
её Георгиевской, т.к. Ярослав в крещении был ещё и Георгием.

В общем праздник, князь доволен, киевляне пьют меда, ощипывают лебедей и
вкушают запечённые, медвежьи лапы.

Аж до пока, некто Батый из клана Чингисхана, не развалял пол Киева с
окрестностями.

Шайтан его задери.

Через ~500 лет, церковь отстроили заново - спасибо весёлой императрице
Елизавете Петровне Разумовской.

А ещё через пару столетий, пришли пламенные революционеры и сподобилась
развалить её, да так, что Батый  в аду, прямо со сковородки аплодировал и пел
Интернационал.

...

От древних славян повелось, что Георгий - он же Гога, он же Егорий, он же и
Юрий.

Как у каждого истинного святого, у Юрия был свой день в святках. 

Его так и звали - Юрьев. 9 декабря.

За неделю до него, крестьянам милостевейше дозволялось перейти в услужение
другому барину, наверное более доброму и красивому.

Так же милостевейше сие право упразднили, привязав крестьян навечно к
кормильцу.

- Вот тебе бабушка и Юрьев день! - говорили они, и заложив последнюю шапку
отмечали его в кабаке, к тому времени появившимся на Руси.

Очень вовремя.

...

И вот, аккурат к древнему празднику, Верховная Рада приняла новый законопроект,
где крестьянам и олигархам наказали платить больше и не утаивать от них подати
казённые ибо тюрьма, сума и тлен.

5600.

- Это удар по олигархам! - радостно делились впечатлениями, законотворци. -
Теперь они будут платить ещё больше налогов, станут беднее и послушнее! -
ЦэБэззапэречнаПэрэмога! - Это удар по хитрым селянам, сдающим паи на шару! -
Это удар по нечестным фермерам!  - Это даст столько денег в казну на Вэлыкэ
Будивныцтво, что можно будет покрыть всю страну асфальтом и сверху выстроить
президентский университет майбутнього!

- Это заставит Киевстар уменьшить количество дней в пакете! Ну и т.д…

Испугались все олигархи и от страха повысили цены на всё подряд, чтобы
крестьяне оплатили неудобства барину, собственно как сотни лет и происходило.

- Вот тебе бабушка и Юрьев день! - гаркнули крестьяне и побежали по кабакам,
отмечать, потому что введение нового налогового закона повысит акциз аква виты
на 5\%, а значит на 25\% для простых смертных. Будьмо!

 @igg{fbicon.face.happy.two.hands} 

\begin{itemize} % {
\iusr{Георгий Солоневич}
Глубоко

\iusr{Наталья Токарева}
Чорт же возьми... То же, что и у нас, что ли? Олигархам хоть бы хрен. А на нас положили...  @igg{fbicon.thinking.face} 
\end{itemize} % }
