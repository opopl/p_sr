% vim: keymap=russian-jcukenwin
%%beginhead 
 
%%file 30_03_2022.stz.news.ua.pravda.2.ohtyrka.4.vorog
%%parent 30_03_2022.stz.news.ua.pravda.2.ohtyrka
 
%%url 
 
%%author_id 
%%date 
 
%%tags 
%%title 
 
%%endhead 

\subsubsection{Хлопці, що вам треба? – Ворог! Ми хочемо більше його накришити!}

Охтирка повністю знищена ворогом, але, попри це, у нас залишаються жителі –
десь 20 з 48 тисяч (інші виїхали – УП), та військові. Удень місто живе.

Над нами постійно літають їхні БПЛА, \enquote{Орлани}, а потім вони нас
обстрілюють. 

Зараз вони підводять нас до того, щоб на наступний рік у нас був голод,
гуманітарна, екологічна катастрофи. Вони знищують інфраструктуру по всіх
напрямках: зерносклади, техніку по селах, ангари. Це ж все, щоб не було
посівної!

А за рік вони скажуть: а ми тут ні до чого, ми воювали тільки з військовими; а
те, що вони вмруть без їжі – не наша проблема. 

Вони були впевнені, що все буде за тиждень-два, а вже четвертий тиждень іде, і
ми перемагаємо й знищуємо їх на всіх фронтах. Охтирка – це Україна, була, є і
буде.

\ii{30_03_2022.stz.news.ua.pravda.2.ohtyrka.4.vorog.pic.1}

Чому росіянам не вдалось її взяти? На те є багато причин.

Їхня найбільша надія була на перших три дні. А коли ми за той час не зламались,
то вони почали боятися Охтирку, як чорт ладану. 

Ми вистояли і почали потихеньку робити оборонні лінії. 

Були такі моменти, коли ми питали військових на передовій: \enquote{Що вам
потрібно, хлопці? Їжа, телефони, спальники?} А вони: \enquote{Нам потрібен
ворог! Ми хочемо його більше накришити!} Тобто вони сміються, що їм треба
тільки москаля, знищити його за те, що він прийшов на нашу територію. 

А той москаль почав нас обходити \enquote{окольными путями}. Ми його стримали,
біля Харкова його стримали, і він пішов через Суми, через Тростянець. 

Через Суми пройшло декілька тисяч військової техніки, вони там на БТР їздили –
не розумію, чого говорять, що Суми не пропустили ворога.

Якби ми їх пропустили, то, думаю, що війна б закінчилась капітуляцією країни. 

Завдяки Охтирці вони не пройшли на Полтаву, завдяки Охтирці й Харкову – не
пройшли на Дніпро.

Але, пройшовши через Суми, Тростянець, Лебедин, вони просто загубились у полях.
Там вони не знали – куди, що і як, і їх там потихеньку почали палити.
