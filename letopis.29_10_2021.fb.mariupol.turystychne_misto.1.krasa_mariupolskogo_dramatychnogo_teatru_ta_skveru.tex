%%beginhead 
 
%%file 29_10_2021.fb.mariupol.turystychne_misto.1.krasa_mariupolskogo_dramatychnogo_teatru_ta_skveru
%%parent 29_10_2021
 
%%url https://www.facebook.com/mistoMarii/posts/pfbid02ck9JaZks1fAA8yAZsVjtn6GFTgUmqbqVNmCgAMZNfXE4sv561YSW9qtdf8NoJSq1l
 
%%author_id mariupol.turystychne_misto,oleynik_yaroslav
%%date 29_10_2021
 
%%tags 
%%title Краса маріупольського Драматичного театру та скверу навколо нього
 
%%endhead 

\subsection{Краса маріупольського Драматичного театру та скверу навколо нього}
\label{sec:29_10_2021.fb.mariupol.turystychne_misto.1.krasa_mariupolskogo_dramatychnogo_teatru_ta_skveru}
 
\Purl{https://www.facebook.com/mistoMarii/posts/pfbid02ck9JaZks1fAA8yAZsVjtn6GFTgUmqbqVNmCgAMZNfXE4sv561YSW9qtdf8NoJSq1l}
\ifcmt
 author_begin
   author_id mariupol.turystychne_misto,oleynik_yaroslav
 author_end
\fi

Краса маріупольського Драматичного театру та скверу навколо нього 🥰

Yaroslav Oleynik:

На центральній площі Маріуполя - Театральній розташувався Донецький академічний
обласний драматичний театр, один з найстаріших театрів Лівобережної України.

Початком театру в Маріуполі вважають 1847 рік, коли до міста вперше приїхала
театральна трупа під керівництвом антрепренера В. Виноградова. 

А 1878 року в місті була заснована  перша професіональна театральна трупа.
Василь Шаповалов, орендував придатне для потреб театру приміщення.

В 1947 році театр закрили.

А 1959 року було розпочато будівництво нової будівлі. До цього на цьому місті
була -  трьохпрестольна церква Святої Марії Магдалини (1862 - 1930).

Нещодавно, коли я був в місті, я застав вже традиційне закриття мотосезону 2021
року. Більше сотні мотоциклів, мопедів і квадрациклів з'їхалися до драмтеатру,
звідки колоною рушили через весь Маріуполь.

А ми помандрували вулицями міста фотографуючи цікаві об'єкти.

Ось, що в нас вийшло!
