% vim: keymap=russian-jcukenwin
%%beginhead 
 
%%file 24_08_2021.fb.chehonadskij_andrej.1.strana_nezavisimost_formulirovka
%%parent 24_08_2021
 
%%url https://www.facebook.com/chehonadskiy/posts/977913269774342
 
%%author Чехонадский, Андрей
%%author_id chehonadskij_andrej
%%author_url 
 
%%tags 1984_kniga,dvojemyslie,nezalezhnist,orwell,strana,ukraina
%%title Очень точная формулировка - День Независимости. День страны, от которой  ничего не зависит
 
%%endhead 
 
\subsection{Очень точная формулировка - День Независимости. День страны, от которой  ничего не зависит}
\label{sec:24_08_2021.fb.chehonadskij_andrej.1.strana_nezavisimost_formulirovka}
 
\Purl{https://www.facebook.com/chehonadskiy/posts/977913269774342}
\ifcmt
 author_begin
   author_id chehonadskij_andrej
 author_end
\fi

Очень точная формулировка - День Независимости.

День страны,от которой  ничего не зависит.

Что,кто-то этого не знает?

Знают все.

В чем же дело?

А дело в том, что Украина - это Оруэлл с его "1984".

Мышлением и поведенческими особенностями украинского населения рулит
двоемыслие.

"Двоемыслие означает способность одновременно держаться двух противоположных
убеждений. 

Партийный интеллигент знает, в какую сторону менять свои воспоминания;
следовательно, сознает, что мошенничает с действительностью; однако при помощи
двоемыслия он уверяет себя, что действительность осталась неприкосновенна. Этот
процесс должен быть сознательным, иначе его не осуществишь аккуратно, но должен
быть и бессознательным, иначе возникнет ощущение лжи, а значит, и вины. 

Двоемыслие — душа ангсоца, поскольку партия пользуется намеренным обманом,
твердо держа курс к своей цели, а это требует полной честности. 

Говорить заведомую ложь и одновременно в нее верить, забыть любой факт, ставший
неудобным, и извлечь его из забвения, едва он опять понадобился, отрицать
существование объективной действительности и учитывать действительность,
которую отрицаешь, — все это абсолютно необходимо. 

Даже пользуясь словом «двоемыслие», необходимо прибегать к двоемыслию. Ибо,
пользуясь этим словом, ты признаешь, что мошенничаешь с действительностью; еще
один акт двоемыслия — и ты стер это в памяти; и так до бесконечности, причем
ложь все время на шаг впереди истины. В конечном счете именно благодаря
двоемыслию партии удалось (и кто знает, еще тысячи лет может удаваться)
остановить ход истории.

Все прошлые олигархии лишались власти либо из-за окостенения, либо из-за
дряблости. 

Либо они становились тупыми и самонадеянными, переставали приспосабливаться к
новым обстоятельствам и рушились, либо становились либеральными и трусливыми,
шли на уступки, когда надо было применить силу, — и опять-таки рушились. Иначе
говоря, губила их сознательность или, наоборот, атрофия сознания. 

Успехи партии зиждятся на том, что она создала систему мышления, где оба
состояния существуют одновременно. 

И ни на какой другой интеллектуальной основе ее владычество нерушимым быть не
могло. Тому, кто правит и намерен править дальше, необходимо умение искажать
чувство реальности. Ибо секрет владычества в том, чтобы вера в свою
непогрешимость сочеталась с умением учитья на прошлых ошибках.

Излишне говорить, что тоньше всех владеют двоемыслием те, кто изобрел
двоемыслие и понимает его как грандиозную систему умственного надувательства. 

В нашем обществе те, кто лучше всех осведомлен о происходящем, меньше всех
способны увидеть мир таким, каков он есть. В общем, чем больше понимания, тем
сильнее иллюзии: чем умнее, тем безумнее. Наглядный пример — военная истерия,
нарастающая по мере того, как мы поднимаемся по социальной лестнице. Наиболее
разумное отношение к войне — у покоренных народов на спорных территориях. Для
этих народов война — просто нескончаемое бедствие, снова и снова
прокатывающееся по их телам, подобно цунами. Какая сторона побеждает, им
безразлично. Они знают, что при новых властителях будут делать прежнюю работу и
обращаться с ними будут так же, как прежде. Находящиеся в чуть лучшем положении
рабочие, которых мы называем «пролами», замечают войну лишь время от времени.
Когда надо, их можно возбудить до исступленного гнева или страха, но,
предоставленные самим себе, они забывают о ведущейся войне надолго. Подлинный
военный энтузиазм мы наблюдаем в рядах партии, особенно внутренней партии. В
завоевание мира больше всех верят те, кто знает, что оно невозможно. Это
причудливое сцепление противоположностей — знания с невежеством, циничности с
фанатизмом — одна из отличительных особенностей нашего общества. Официальное
учение изобилует противоречиями даже там, где в них нет реальной нужды. Так,
партия отвергает и чернит все принципы, на которых первоначально стоял
социализм, — и занимается этим во имя социализма. Она проповедует презрение к
рабочему классу, невиданное в минувшие века, — и одевает своих членов в форму,
некогда привычную для людей физического труда и принятую именно по этой
причине. Она систематически подрывает сплоченность семьи — и зовет своего вождя
именем, прямо апеллирующим к чувству семейной близости. Даже в названиях
четырех министерств, которые нами управляют, — беззастенчивое опрокидывание
фактов. Министерство мира занимается войной, министерство правды — ложью,
министерство любви — пытками, министерство изобилия морит голодом. Такие
противоречия не случайны и происходят не просто от лицемерия: это двоемыслие в
действии. Ибо лишь примирение противоречий позволяет удерживать власть
неограниченно долго. По-иному извечный цикл прервать нельзя. Если человеческое
равенство надо навсегда сделать невозможным, если высшие, как мы их называем,
хотят сохранить свое место навеки, тогда господствующим душевным состоянием
должно быть управляемое безумие"

Джордж Оруэлл,"1984".
