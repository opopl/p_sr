% vim: keymap=russian-jcukenwin
%%beginhead 
 
%%file 26_01_2022.fb.golovachev_andrej.2.zapad_proigraet
%%parent 26_01_2022
 
%%url https://www.facebook.com/permalink.php?story_fbid=2789837907992636&id=100008993618796
 
%%author_id golovachev_andrej
%%date 
 
%%tags civilizacia,ukraina,zapad
%%title Борьба за Украину: Почему Запад к конце концов проиграет?
 
%%endhead 
 
\subsection{Борьба за Украину: Почему Запад к конце концов проиграет?}
\label{sec:26_01_2022.fb.golovachev_andrej.2.zapad_proigraet}
 
\Purl{https://www.facebook.com/permalink.php?story_fbid=2789837907992636&id=100008993618796}
\ifcmt
 author_begin
   author_id golovachev_andrej
 author_end
\fi

Борьба за Украину: Почему Запад к конце концов проиграет?

Сэмюель Хантингтон, автор нашумевшего некогда бестселлера \enquote{Столкновение
цивилизаций}, помимо прочего утверждал что ни одна страна не может вырваться
за пределы своей цивилизации и перейти в другую. При этом Хантингтон выделял
девять цивилизаций:

\obeycr
Западная
Православная
Исламская
Индуистская
Синская
Японская
Африканская
Буддистская
Латиноамериканская
\restorecr

Хантингтон допускал, что некая внутренняя или внешняя политическая сила может
с помощью радикальных реформ, невероятных усилий, а также прямого насилия и
террора временно вырвать ту или иную страну из ее цивилизации и навязать ей
принципиально другой тип культуры.

Например, Ататюрк силовым путем пытался вырвать Турцию из исламской цивилизации
и направить ее по западному пути развития. Большевики через красный террор
пытались навязать православной России и Украине принципиально новую
атеистическую цивилизацию. Европейский союз надеется что ему удалось полностью
сменить православный код Греции и вовлечь ее в западную цивилизацию. 

Примеров таких немало.   Однако, Хантингтон утверждал что как бы не старались
реформаторы, максимум через три поколения страна все равно возвратиться к своим
корням, к своей базовой  культуре, в основе которой лежит, конечно,
соответствующая религия.   

Похоже, что Хантингтон был прав.  Как ни старались большевики, через три
поколения Россия все равно вернулась на свой традиционный имперско-
православный путь, в основе которого лежат принципы: самодержавие, православие,
народность. И никакую демократию, верховенство права и равенство всех перед
законом в России не построить.  

Как ни пыталась турецкая армия и США удержать Турцию на светском пути развития,
сколько не совершали государственных переворотов, а через три поколения уже
всем стало ясно, что Турция возвращается в свою традиционную исламскую
цивилизацию. В ближайшие годы в Турции будет изменена Конституция, она выйдет
из всех европейских структур и из блока НАТО. Это вызовет потрясения в Европе и
окончательно добьет ЕС и НАТО.  

Европейский союз, который пытался соединить в едином политическом союзе
западную, православную и исламскую цивилизации, на базе химерических
общечеловеческих ценностей, уже не жилец. А ведь не  прошло и одного
поколения! А еще через два поколения эксперимент по созданию единых
общечеловеков, будут вспоминать как недоразумение. 

Греция останется внутри православной цивилизации и гей парады будут отменены во
многих странах бывшего Евросоюза, где возрастет роль церкви и базовой религии.  

Восемь  лет   назад, после победы евромайдана,  Запад всерьез поверил что он в
состоянии сменить православный код Украины, ее традиционные
крестьянско-консервативные ценности  и привить ей западный либерализм,
индивидуализм, парламентаризм, права человека  и т.п  Это было наивно. Впрочем
Запад уже осознал,  что он взялся за неподъемную задачу и он уже почти не
предпринимает попыток обратить Украину в свою верю. Разница в масштабах
проводимых гей парадов и Крестных ходов в Украине наглядно продемонстрировала
Западу все безнадежность его усилий.  Запад уже смирился с этим и дал понять
Украине, что у нее нет никаких шансов войти в ЕС. Правда, неугомонный старик
Сорос все еще никак не успокоится и безнадежно финансирует  своих бесплодных и
жадных до его грантов соросят. 

Трех поколений  в Украине ждать не придется. Через 15-20  лет Украина  обвинит
Запад в предательстве, окончательно закрепит свой православный статус и
откажется от европейского вектора.   Страна возвратится к 100\% президентской
форме правления, как в  России. При этом отношения с католической Польшей
будут на порядок хуже, чем отношения с православной Россией. 

Примечательно, что тот же Хантингтон утверждал что войны между странами,
принадлежащими одной цивилизации редко бывают длительными и ожесточенными.
Напротив, войны между странами, принадлежащими  к разным цивилизациям, часто
перерастают просто в бойни, в которые втягиваются многие другие страны. 

От судьбы, точнее от своей цивилизации, не уйти!

\ii{26_01_2022.fb.golovachev_andrej.2.zapad_proigraet.cmt}
