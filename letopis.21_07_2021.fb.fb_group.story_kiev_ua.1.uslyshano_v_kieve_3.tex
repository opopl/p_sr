% vim: keymap=russian-jcukenwin
%%beginhead 
 
%%file 21_07_2021.fb.fb_group.story_kiev_ua.1.uslyshano_v_kieve_3
%%parent 21_07_2021
 
%%url https://www.facebook.com/groups/story.kiev.ua/posts/1711708039025990/
 
%%author Киевские Истории
%%author_id fb_group.story_kiev_ua
%%author_url 
 
%%tags gorod,kiev,poslovica,slovo,ukraina,uslyshano_v_kieve
%%title Услышано в Киеве - ЧАСТЬ ТРЕТЬЯ
 
%%endhead 
 
\subsection{Услышано в Киеве - ЧАСТЬ ТРЕТЬЯ}
\label{sec:21_07_2021.fb.fb_group.story_kiev_ua.1.uslyshano_v_kieve_3}
 
\Purl{https://www.facebook.com/groups/story.kiev.ua/posts/1711708039025990/}
\ifcmt
 author_begin
   author_id fb_group.story_kiev_ua
 author_end
\fi

ЧАСТЬ ТРЕТЬЯ. 

И, вероятно, последняя.  Комменты для первых частей поста уже закрыты.

В ответ на слабое, но конструктивное обвинение в самозахвате, назовем наше ностальгирование «Услышано в Киеве». 

Итоги второй части поста. 500 комментов и великолепные киевские перлы!

Среди особо изысканных: 

\begin{itemize}
  \item - Він мов гівно з просом – він гливкий як в’ялений інжир. 
  \item - А Сара хочет негра – автоматический ответ моего папы на любое «Я хочу…».
  \item - Позовите городового - пусть он отведёт меня в сумасшедший дом – в конфликтных ситуациях местного значения.
\end{itemize}

Все киевские ФРАЗЫ вставил в словарь. Прозвища, клички, эпитеты и пр.  – темы отдельных постов, за которые никто не хочет браться. 

Украинских фраз по-прежнему маловато, или они не совсем киевские. Польских фраз нет совсем☹ 

Вспоминается старый анекдот. - У Вас есть экономическое образование? – Да. У меня обе бабушки – еврейки.

Откровением явилось, количество, казалось уже забытых, в опустевшем Городе, идишизмов, но разве можно забыть еврейскую бабушку?

ЧАСТЬ ВТОРАЯ.

Подведем итоги первой части поста и начнем вторую.

1300 коментов, говорят о том, что тема насущная! Не ожидал такого. 

Админы! Хельг наш Ковальский! Подхватите тему, пжл! Нет сил все окучить! 

90\% — это, конечно, прозвища, на что киевляне всегда были мастаками. Предлагаю
желающим открыть самостоятельно темы киевских прозвищ, взяв себе какой-нибудь
из разделов: магазины (объедков и т.д.), улицы (Гапкинстрит и Фраерштрассе и
т.д.) районы (Троя и т.д.), архитектурные объекты (Мать-анархия и т.д.),
личности (гвардеец тыла и т.д.), предметы (бычок-окурок и тд.), фрукты, овощи
(синенькие и т.д.), профессии (балагула, табакрошки и т.д) и пр.

А, что касается выражений…

Вставил все киевское, по моему мнению, что подкинули знатоки. Немного с
натяжкой. Комментариев море, а выражений – не очень. Украинских – совсем мало.
Будем обнулять и пойдем по второму кругу. 

Киевляне – напрягите все свое и понесите людям!

ЧАСТЬ ПЕРВАЯ.

Предлагаю составить словарь чисто киевских выражений. Не иронические прозвища,
на что киевляне мастера, таких вроде «Ручки от склона» для Арки или «Отдай
жвачку» для Самсона, а именно устоявшихся «ходячих», сейчас или раньше,
выражений. 

Тексты брал из Макарова, Галайбы и от прочих знатоков.

Итак, мой и ваш, уже совсем не короткий список:

\begin{itemize}
\item  Аз ох-н-вей (идиш/иврит) -  когда (хочется сказать) «ох!» и «вей!».
Первое междометие употребляется не от хорошей жизни, второе же слово означает
«горе». 
\item  А он крутой шнарант – от фамилии хозяина дома №63 на углу Межигорской и
Еленовской. Он был знаменит тем, что сдавал временную жилплощадь торгашам. Там
заключались сделки насчет купли-продажи оптовых партий товаров и
останавливались разношерстные аферисты. Именно поэтому впоследствии слово
«шнарант» стало олицетворять авантюриста, прохиндея, нечистого на руку дельца.
Когда дом Шнаранта сносили, в стенах нашли тайники с деньгами, расписками, а
рабочие обнаружили в подвале шкатулку с золотыми монетами.

\item  А Сара хочет негра – в ответ на «Я хочу…».
\item  А там вся аристократия липовая — ироническое наи¬менование владельцев усадеб на Липках. В стари¬ну это выражение не имело того саркастическо¬го смысла, который вкладывается в него теперь.
\item  Ах ты старая кошолка (от кошеля) – Евбазовское приветствие.
\item  Біля Лева печінка дешева - рядом с Самсоном был базарчик.
\item  Вей из мир (идиш) – горе мне.
\item  Він мов гівно з просом – він гливкий як в’ялений інжир. 
\item  В инвалидку за бананами – пойти в магазин для инвалидов.
\item  В Киеве и кирпич не краснеет – поговорка богомольцев. 
\item  В Киеве одни стены святы, а люди поганы - поговорка богомольцев.
\item  Вона так кричить, аж ВИРЯЧИЛА (выпучила) очи - без комментариев.
\item  В свинячий голос - когда что-то делается не вовремя или с опозданием.
\item  Встретимся на шариках - на улице Мечникова у дома с шарами на парапете. 
\item  Встретимся у пеликанов - у памятника пеликанам  на Лютеранской.
\item  Встретимся у Черного моря -  в парке Шевченко у фонтанчика.
\item  Вус трапилось (идиш/укр.) – что случилось?
\item  Вы посмотрите на эту дарницкую дворянку – возглас возмущения евбазовских торговок.
\item  Гейн какен вайтер (идиш – иди какать подальше) – ласковый посыл.
\item  Гиб а кик, шо робыться (идиш/укр.) – Взгляни, что творится.
\item  Граф Алексей Бобринский - «А кто это будет делать - Бобринский?», "Теж мені, графиня Бобринська!", и пр. - на месте Щорса, стоял памятник графу Бобринскому, который много в жизни успел сделать. Вот оттуда и пошло.
\item  Грузи на Бессарабку – динамовско-стадионное.
\item  Дай мне кецик - кусочек чего-либо съестного. Евбазовское.
\item  Двойная половинка - первые кофейни на Крещатике и на Заньковецкой, в которых готовили аппаратный кофе, в 80-е назывались кафетериями и были оснащены автоматами, настроенными на стандартную порцию кофе на стакан воды. Напиток получался так себе. Для кофеманов работница кафетерия вручную отмеряла две порции кофейного порошка и половинку порции воды. Получалось полстакана (разливали в граненые стаканы) кофе, носившего название «двойная половинка». 
\item  Дрек мит фефер (идиш– гавно с перцем) – крученый пройдоха. 
\item  Если хочешь поработать ляжь поспи и все пройдет - без комментариев.
\item  Жизнь дала трещину еду на Троещину - без комментариев.
\item  Забегалась, как Ицикова сучка - без комментариев.
\item  Їхати на ковбасі  - У старих трамваях був ззаду маленький майданчик, на який ставали чоловіки, а тримались за драбину, яка вела на дах. Отак їхали від зупинки до зупинки.
\item  Иди на Евбаз курочек ловить – ласковый посыл.
\item  Из под пятницы - суббота - если из-под рубашки видно было майку или из-под пиджака - подол рубашки - это считалось очень неряшливо.
\item  Их вейс? (идиш) – откуда я знаю?
\item  Как ты мене дорог -говорилось если треплют  нервы.
\item  На городі бузина, а в Києві дідько - що ти дурню верзеш.
\item  Насипь мені миску борщу - без комментариев.
\item  Не дрек мне копф (идиш/рус.) - не гамни мне голову. 
\item  Ну ты и босяк – так звались голоногие изгои, артельная подольская «босая команда» временные грузчики на Притыке (пристани) еще при царе. Они спали там босиком и на подошвах у них мелом были написаны цены, за которые они нанимались.
\item  Он из бичевых (аристократов) – из польских помещиков, которые ездили с бичами (кнутами). 
\item  Опять за рыбу гроши – ну сколько можно?
\item  Підожды - без комментариев.
\item  Плевать с лаврской колокольни - без комментариев.
\item  Позовите городового - пусть он отведёт меня в сумасшедший дом – в конфликтных ситуациях местного значения.
\item  Поедем на эмочке? – на такси.
\item  Розумному лихо, дурному радість - без комментариев.
\item  С Короленка видна Ленка – с Короленка, 15 (с милиции) видна Сибирь…
\item  Сиди жды, покi, жаба цицки дасть – жди, мечтай - может дождешся.
\item  Так не фонтан, так не шпрынцает -  из рук вон плохо.
\item  Танцуем от печки - из лексикона знаменитого киевлянина Соломона Шкляра, героя известной песни, который, вообще-то, был парикмахером и работал на Бибиковском бульваре. Однако решил, что сможет преуспеть и совершенно неожиданно открыл в доме Пфалера на большой Васильковской, 10 школу танцев. Нанятые учителя танцев преподавали, а Шкляр, как сказали бы сейчас, занимался конферансом. Он объяснял фигуры танца, беспрестанно шутил, много комментировал, давал рекомендации, короче, создавал настроение. 
\item  Танька (или любое другое имя) на базаре семечками торгует - в ответ на грубо-уменьшительное имя.
\item  Та тож кожемяцкие аристократы - без комментариев.
\item  То цирульник из-за канавы – без комментариев.
\item  Тю… – без комментариев.
\item  Ума нет- на Бесарабке не купишь - без комментариев.
\item  Усюсяный-ухвысяный (совр. уси-пуси) - Это означало, что младенец "любленный-голубленный".
\item  Хочу на копки-баранки – посади меня на плечи.
\item  Цирк на дроті — о нелепой ситуации.
\item  Через духовку на Куреневку – не в том направлении.
\item  Чорти шо и с боку бантик - на какую - то нелепость.
\item  Шо за фойле штык (идиш) – что за дурная шутка?
\item  Щас, возьму разбег с Бессарабки – выражение торговок Сенного рынка на выпады на счет цены товара.
\item  Що з возу впало, то пропало – тысячелетняя киевская поговорка с Андреевского спуска, где была таможня.
\item  Що ти швендяєш? – чего гуляешь?
\item  Это форменное надувательство – киевские мясники «надували» мясо для улучшения внешнего вида;
\item  Это что за шарашкина контора – контора строительного подрядчика Шарашкина, неофициальным девизом которого было: «Главное подрядиться», т.е. взять аванс. Называлась в противовес конторе Льва Гинзбурга.
\item  Яхна ты такая – грубая, не образованная еврейка. Так ул. Ярославскую называли Яхнославской.
\end{itemize}

Одесситы и житомиряне, если захотят опротестовать – пусть доказывают😊

Добавляйте, пжл, а если я что пропустил, то предлагайте понастойчивее.
