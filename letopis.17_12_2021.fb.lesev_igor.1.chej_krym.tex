% vim: keymap=russian-jcukenwin
%%beginhead 
 
%%file 17_12_2021.fb.lesev_igor.1.chej_krym
%%parent 17_12_2021
 
%%url https://www.facebook.com/permalink.php?story_fbid=4889762677721463&id=100000633379839
 
%%author_id lesev_igor
%%date 
 
%%tags krym,maidan2,rossia,ukraina
%%title Чей Крым?
 
%%endhead 
 
\subsection{Чей Крым?}
\label{sec:17_12_2021.fb.lesev_igor.1.chej_krym}
 
\Purl{https://www.facebook.com/permalink.php?story_fbid=4889762677721463&id=100000633379839}
\ifcmt
 author_begin
   author_id lesev_igor
 author_end
\fi

Чей Крым?

Вчера у меня был эфир с московским ютуб-каналом. Вроде как тема была Донбасс и
Украина в целом. В ру-политтусовке обожают украинскую тему. Обычно она
сопровождается менторским поучением и вопросами а-ля почему вы там такие тупые,
какие у вас перспективы, и не боитесь ли еще больше деградировать и
развалиться?

\ii{17_12_2021.fb.lesev_igor.1.chej_krym.pic.1}

Но я давно уже заметил одну особенность, которая чертовски одинакова у русских
ватников и украинских славаукраинцев. И те, и те не любят против шерсти. А еще
они синхронно живут проблемами друг друга. Они натуральные микробиологи по
изучению друг друга. Только настоящий патриот Украины знает точную дату
рождения Путина. И только настоящий патриот России дословно процитирует
заявление внештатного помощника замминистра Имярека, о том что русские – это
свиньи.

Патриоты с ру-пропиской болезненно реагируют, когда им напоминаешь об
односторонней передаче амурских островов Китаю. Они не любят обсуждать, почему
в Чечне, да и в целом, в северокавказских республиках вообще не живут русские.
Раньше жили, а вот теперь вообще их там нет.

Еще их очень сильно волнует притеснение русского языка в Украине. Сразу скажу,
меня это волнует гораздо больше. И спасибо, что переживаете. Ценю. Но чем же
вам не угодили русские и русскоязычные в Казахстане, которых вот прямо сейчас
задрачивают на языковой теме не меньше, чем в Украине?

В общем, эфир вышел в таком ключе с гораздо большим географическим расширением
и обильными аналогиями из самой России. И я бы даже о нем не написал, но меня
занесло зайти по ссылке на комментарии к нему. Даю выборочно несколько наиболее
ярких пассажей:

«Пытался держаться до последнего, н в конце поперло украинство.»

«Да, типичный украинский нацист. Всё ждёт развала России».

«Мужик ничего не знает о том, какие мнения в России, но утверждает постоянно и
высказывает эти как бы существующие мнения ....  Чуть - чуть ушел от нациков,
но не очень далеко»

«всё - таки русофоб, хоть и старается вывернуть как-нибудь более приемлемо,
иногда передергивая....»

«Теряюсь как его назвать!! Вот не знаю!!! Эмоции без контрольны, в душе-точно
русофоб. Вера.Н он Вас не слушал и слышать не может,потому как один из всех
невмерлых! Как Вы всё выдерживаете?? Спасибо!!»

Ну, думаю этого достаточно. Если меня когда-то таки припрет СБУ за
«антиукраинскую деятельность», думаю, комментарии к этому эфиру будет моим
лучшим адвокатом.

Смотрите, начну из общетеоретического издалека. Государство. Любое. Без
разницы. У него есть три основных признака – территория, население и
суверенитет. Это общий курс политологии.

Исходя из этого, какой критерий успешности государства? Сохранение территории,
генерация лояльного этой территории населения, а также самоуправление этой
территорией без вмешательства извне. Если что-то в этой цепочке нарушается, мы
получаем на выходе Гаити, Бангладеш или Украину.

Это вводная теория. А теперь история. История, особенно мировая – это фатальная
сука. Вот как секс без резинки, а потом бац – и «вдруг» СПИД.

В нашей общей истории СПИД случился в 1991 году. Мне тогда было 10 лет, но у
истории – пусть она и фатальная сука – есть одно неубиенное достоинство. Ее
можно познавать и много лет спустя после прошедших событий. Поэтому, коротко
напомню, что в том году произошло. Весна – референдум о сохранении СССР, где
большинство наших мамок-папок сказали «да». Лето – ГКЧП. Осень – парад
суверенитетов. Зима – Беловежское соглашение и вынос тела СССР в морг.

Для совсем хроников напомню, что ГКЧП снесли не в Киеве, не в Вильнюсе, и не в
Тбилиси. Там как раз сидели ровно и готовы были принять любой расклад. ГКЧП
снесли сотни тысяч москвичей и ленинградцев, которые отстояли на улицах Б.
Ельцина. Они и сыграли решающую роль в славных похоронах Союза ССР.

А тем, кто особо сожалеет о развале Союза, приведу пример из того же 1991 года.
Югославия. Она как раз тогда же и начала сыпаться. Но сербы хотя бы сделали
попытку сохранить свою страну. А что сделали русские? Молча встретили Новый
1992 год в новой обрубленной стране.

И всё, карта бита. Нет больше ни русского Кустаная, ни русской Одессы, ни
русской Нарвы. И да, ни первые, ни последние. Где та греческая Смирна сегодня?
А польский Львов? А кто те безымянные китайские дураки, которые за две встречи
на безвозмездной основе отдали России гигантские территории Дальнего Востока –
Приамурье и Приморье – и случилось это почти позавчера, всего-то во второй
половине XIX века?

А потом СПИД пришел уже в Украину. Мы знаем когда. 2014 год. Напомню коротко
хронологию. Майдан – расстрелы на Институтской – госпереворот – «крымская
весна» - Донбасс. Насыщенный такой год, компактная версия 1991-го, только для
одной уже Украины.

И тут я вернусь к упомянутому в начале эфиру. Ведущая зачитала вопрос какого-то
зрителя. Он звучал приблизительно так – «почему украинские националисты так
сильно боятся нападения России, какой смысл ей вообще нападать на Украину»?

Думаю, от меня ожидали ответа в стиле, «потому что украинские националисты
тупые, а Россия ни на кого не нападет, она только защищается»… ну, короче, мы
ведь с вами иногда смотрим Соловьева, знаем, как надо правильно отвечать. Но я
ответил, что боятся, потому что Россия УЖЕ напала, и при этом не на Донбассе, а
в Крыму.

Для полной ясности приведу глоссарий Игоря Лесева, не только же ветеринару в
СНБО выделываться.

Первое. В Киеве в феврале 14-го произошел банальный государственный переворот,
который тут же был легализован странами Запада. Это кардинально нарушило как
сложившийся баланс политических сил внутри Украины, так и геополитическое
равновесие в регионе.

Второе. «Крымская весна» - это агрессия России против Украины, приведшая к
присоединению части территории государства А к государству Б. И это прямая
ревизия Беловежского соглашения 1991 года, итогом которого было появление как
Украины, так и России в новых ее границах.

Третье. Жители Крыма в своей массе поддержали вхождение в состав РФ. Напомню, в
1991-м больше половины жителей полуострова поддержало независимость Украины.
Узаконивает ли этот факт как-то пункт 2 для России? И да, и нет. Узаконивает ли
появление Косово на карте мира, только потому что там живет большинство
албанцев, и они не хотели жить в Сербии? А узаконивает ли пребывание Голанских
высот в составе Израиля, а Западной Сахары в составе Марокко только на том
основании, что они давно уже в составе этих стран? США во всех этих случаях
ответили «да», хотя это нарушает базовый принцип международного права о
нерушимости границ, с которым мировое сообщество апеллирует к России по
крымскому вопросу.

Четвертое. Новая украинская власть, осуществившая государственный переворот и
сместившая законно выбранного и признанного мировым сообществом президента В.
Януковича, по каким-то причинам так и не удосужилась явиться на войну с РФ,
отстаивая юридически законные права Украины на Крым. Это называется – особо
тяжкое преступление против основ государственности.

Пятое. Новая власть В. Зеленского, отказываясь проводить расследование и
наказывать виновных, которые причастны к потери Крымского полуострова, по сути:

а) разделяет ответственность за потерю части территории Украины и 

б) тем самым упрощает и ускоряет легализацию Крыма в составе России.

Шестое. Украина не сможет вернуть Крым обратно. Точка.

Седьмое. Ключевое. Крым делает невозможным нормализацию российско-украинских
отношений в среднесрочной перспективе. Именно мирно отжатый Крым, а не
заваленный трупами Донбасс. Потому что Донбасс еще не всё для Украины, а Крым –
уже всё.

И не обязательно в этих пунктах искать что-то анти/за украинское или
российское. В Украине произошел государственный переворот, который привел к
фатальным последствиям для страны и большей части ее населения. И в этом
очевидном утверждении нет ничего антиукраинского. А Россия совершила агрессию,
присоединив часть территории сопредельного государства – и в этом утверждении
тоже нет ничего антироссийского.

А вот пламенные патриоты России и Украины очень здорово похожи друг на друга
своими стеклянными глазами.

\url{https://t.me/Lesev_Igor}

\ii{17_12_2021.fb.lesev_igor.1.chej_krym.cmt}
