% vim: keymap=russian-jcukenwin
%%beginhead 
 
%%file 18_10_2021.fb.bilchenko_evgenia.6.durochka.cmt
%%parent 18_10_2021.fb.bilchenko_evgenia.6.durochka
 
%%url 
 
%%author_id 
%%date 
 
%%tags 
%%title 
 
%%endhead 
\subsubsection{Коментарі}

\begin{itemize} % {
\iusr{Ольга Матвиенко}
Ух, классно!! Прямо манифест. Спасибо, Женечка!

\iusr{Natalia Rezanova}
Очень!!!

\iusr{Николай Енакий}
Хорошее фото!

\begin{itemize} % {
\iusr{Евгения Бильченко}
\textbf{Николай Енакий} ну, хоть шото хорошее)
\end{itemize} % }

\iusr{Алексей Бажан}

Про Флоренского правильно говорят, из его разысканий в поддержку кровавого
навета, между прочим, произросла в Мастере и Маргарите сюжетная линия,
связанная с головой Берлиоза. Что либералов корежит от Летова-мл., не думаю;
вот, кстати, классный кавер на концерте в поддержку Серебренникова.

\href{https://www.youtube.com/watch?v=IV6LmZQdfaw}{%
Всё идёт по плану, Александр Горчилин, (гражданская оборона) Гоголь центр-Нам 6 лет, %
Julia WoaWeb, youtube, 17.02.2019%
}

\begin{itemize} % {
\iusr{Евгения Бильченко}
\textbf{Алексей Бажан} 

корежит: Летова не присвоить, он гений. Отца не обижайте Флоренского и Ильина
тоже. Там любовь есть. Она и в Марксе есть. Мне хватает везде красоты на уровне
Логоса.

\iusr{Алексей Бажан}
\textbf{Евгения Бильченко} 

То-то Медуза организовала трибьют (плохой) Летову. Вот Ромычу даже заукраинство
не помогает - не будет у него ангажемента в либеральной среде. А ведь какие
роскошные версии у Анатолия! 

\href{https://www.youtube.com/watch?v=cfJXP8IIJA0}{%
Анатолий Благовест - В России, youtube, 18.04.2021%
}

\iusr{Алексей Бажан}
\textbf{Евгения Бильченко} 

Про то, какая у Флоренского "любовь", давайте лучше не будем. Хотя тема в
определенных кругах модная.

\iusr{Зоя Белоголова}

Люблю у Егора Летова песню - Слово Товарищ.

\end{itemize} % }

\iusr{Алексей Бажан}

Вот песен Дугина в плейлисте рукопожатные люди точно не поймут, а от кое-какого
рэпа им и поплохеет. 

\href{https://www.youtube.com/watch?v=Naa3biMAwKU}{%
Babangida - Коба, youtube, 04.07.2009%
}

\begin{itemize} % {
\iusr{Евгения Бильченко}
\textbf{Алексей Бажан} у Дугина интересные композиции.

\iusr{Алексей Бажан}
\textbf{Евгения Бильченко} 

У него еще вроде сборник стихов выходил под псевдонимом (А. Штернберг -
"Флаконы Лжи"), в сети есть промо-версия, где в предисловии обсуждается
авторство. 

\href{https://www.youtube.com/watch?v=T18v96BG2rI}{%
Александр Гельевич Дугин – Злые духи (La Nuit de Noёl) (Александр Вертинский кавер), youtube, %
28.12.2020%
}

\end{itemize} % }

\iusr{Борис Никифоров}
✍ Просто песня ей Богу... И так точно... Спасибо

\iusr{Юна Щудло}

Женя, прямо в сердце, спасибо!
"слышу, как плачут в земле кости..."


\end{itemize} % }
