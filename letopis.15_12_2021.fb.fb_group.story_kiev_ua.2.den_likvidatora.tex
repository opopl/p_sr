% vim: keymap=russian-jcukenwin
%%beginhead 
 
%%file 15_12_2021.fb.fb_group.story_kiev_ua.2.den_likvidatora
%%parent 15_12_2021
 
%%url https://www.facebook.com/groups/story.kiev.ua/posts/1819665341563592
 
%%author_id fb_group.story_kiev_ua,ugrjumova_viktoria.kiev.pisatel
%%date 
 
%%tags 1986,avaria_na_chaes,chaes,chernobyl
%%title ДЕНЬ ЛИКВИДАТОРА
 
%%endhead 
 
\subsection{ДЕНЬ ЛИКВИДАТОРА}
\label{sec:15_12_2021.fb.fb_group.story_kiev_ua.2.den_likvidatora}
 
\Purl{https://www.facebook.com/groups/story.kiev.ua/posts/1819665341563592}
\ifcmt
 author_begin
   author_id fb_group.story_kiev_ua,ugrjumova_viktoria.kiev.pisatel
 author_end
\fi

ДЕНЬ ЛИКВИДАТОРА

Место, в котором всегда будет 26 апреля 1986 года (из книги Чернобыль. 120
символов, на англ. языке, Скай Хорс)

\ii{15_12_2021.fb.fb_group.story_kiev_ua.2.den_likvidatora.pic.1}

Медсанчасть №126 находится к востоку от центральной площади города Припять. Это
длинное пятиэтажное здание с большим парадным входом, где размещается крупный
медицинский комплекс, включающий роддом, станцию скорой помощи, детское
отделение, стационар, морг и т. д. Сюда в ту ночь доставляли первых
пострадавших ‒ сотрудников ЧАЭС и пожарных. Их одежда — робы, ботинки, каски —
до сих пор свалена в подвале и зашкаливает радиометры: радиоактивный фон от
этих вещей составляет не менее одного рентгена в час.

За прошедшие после аварии 30 с лишним лет Чернобыльская Зона изменилась до
неузнаваемости, стала особенным, отдельным затерянным миром со своей историей и
судьбой. И хотя время здесь течёт иначе, чем за границей Зоны Отчуждения, здесь
тоже наступил 21 век. Только в подвале медсанчасти, кажется, по–прежнему длится
все тот же нескончаемый субботний день, 26 апреля 1986 года.

\begingroup
\em
Как организация МСЧ №126 продолжает существовать и поныне. Сейчас она
расположена в городе Чернобыль на улице Кирова. В ней проходят медосмотр и
медицинское обслуживание сотрудники Зоны Отчуждения и самосёлы.
\endgroup

\ii{15_12_2021.fb.fb_group.story_kiev_ua.2.den_likvidatora.cmt}
