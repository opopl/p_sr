% vim: keymap=russian-jcukenwin
%%beginhead 
 
%%file 25_05_2018.fb.lesev_igor.1.mh17
%%parent 25_05_2018
 
%%url https://www.facebook.com/permalink.php?story_fbid=1929650457066048&id=100000633379839
 
%%author_id lesev_igor
%%date 
 
%%tags mh17,rassledovanie,ukraina
%%title По МН 17
 
%%endhead 
 
\subsection{По МН 17}
\label{sec:25_05_2018.fb.lesev_igor.1.mh17}
 
\Purl{https://www.facebook.com/permalink.php?story_fbid=1929650457066048&id=100000633379839}
\ifcmt
 author_begin
   author_id lesev_igor
 author_end
\fi

По МН 17.

По большому счету, уже абсолютно нет никакой разницы, кто на самом деле сбил
«боинг». Обе стороны – Вашингтон и Москва – обозначили виновных вне зависимости
от того, как все было на самом деле.

Расследование Нидерландов и Австралии могло быть каким угодно, с привлечением
самых лучших, квалифицированных и авторитетных специалистов. С аргументами,
доказательной базой и железобетонными доводами. Единственно, что в этом
расследовании ТОЧНО не могло произойти, это назначение кого-то другого, кроме
России, виновной стороной.

\ifcmt
  ig https://scontent-frt3-2.xx.fbcdn.net/v/t1.6435-9/33475203_1929650353732725_8764514099070500864_n.jpg?_nc_cat=101&ccb=1-5&_nc_sid=730e14&_nc_ohc=QCPMC7R0pucAX8Eh3qE&_nc_ht=scontent-frt3-2.xx&oh=cf5fc4929aadcdfc246825b815adbbf3&oe=61B88CB4
  @width 0.4
  %@wrap \parpic[r]
  @wrap \InsertBoxR{0}
\fi

Точно также и позиция России будет строиться ВСЕГДА на одном и том же тезисе –
«боинг» сбила Украина. Потому что... и дальше по пунктам.

И хотя в Москве, Вашингтоне да и в Киеве посвященные ТОЧНО знают, кто сбил
самолет, для мировой общественности эта тема приобретает сугубо религиозный
окрас. «Я не верю, что это могла сделать Украина». «А я верю, что это сделали
именно украинские военные».

Вопрос теперь в другом – как официальные стороны будут дальше разгонять тему
«боинга». Вообще, сбитый гражданский самолет – это, конечно, из ряда вон
происшествие, но точно не уникальное. Советский Союз сбил известный корейский
самолет. Точно также когда-то Штаты сбили пассажирский иранский самолет в
Персидском заливе. Украина сбивала над Черным морем самолет (российский,
кстати). Даже болгары в далекие 60-е как-то подбили израильский гражданский
борт. А палестинцы в 1970-м году как-то за раз и вовсе угнали аж 5 гражданских
самолетов и все их ухайдохали.

Обычно, если это не касается каких-то особо неприступно-наглых стран, вроде
Штатов или Союза, которые просто ложатся в таких случаях на мороз,
провинившаяся сторона платит компенсацию родственникам жертв инцидента, ну и
перевозчику возмещает стоимость самолета. В случае с малазийским «боингом»,
речь может идти о 500-600 млн. дол. (в среднем, по ляму за погибшего +
стоимость самолета). Это цена вопроса. Куда, естественно, не входят
репутационные издержки. А именно они самые болезненные.

Но проблема конкретно с этим «боингом», который и не позволяет его гладко
разрешить – это использование инцидента, как средства жесткого политического
давления. На Москву пытаются повесить не «случайно сбитый самолет», а
«сознательный террористический акт». А это уже ливийский вариант со всеми
вытекающими. Русские это особо хорошо чувствуют, а потому и упираются
максимально.

Украина во всей этой истории, вне зависимости от ее роли, полностью вне удара.
Даже несмотря на то, что Киев, отвечающий за воздушной пространство над своей
территорией, не запретил полеты над Донбассом. «Боингом» бьют Москву, а не
Киев.

И теперь самое интересное. Стороны обозначили свои позиции. Американцы, пока
что устами голландцев и австралийцев, обвиняют уже в «окончательном варианте»
русских. «Не верить выводам комиссии у нас нет никаких оснований», - уже
отметили в Госдепе. У русских позиция тоже простая – «в работе комиссии мы не
участвовали, а значит и выводы ее не признаем». Все. Но теперь мяч на стороне
Вашингтона. Все друг друга послали нахер и нужно делать конкретику. Какую? Вот
это самое оно.

Первое – это обвинение руководства России в террористической атаке. Второе –
введение новых (каких? Уже непонятно что еще вводить) санкций против РФ. Третье
– ужесточение позиции по болезненным для русских направлениям. А тут вариантов
не так много. «Северный поток-2», Сирия и Украина. Ну, возможно, еще разрыв
дипотношений с Австралией и Голландией (хотя последних будет тяжело на это
протолкнуть). Но в любом случае, «боинг» уже давно стал давней историей.
Толкать им Кремль можно было летом 14-го, ну плюс-минус еще весь следующий год.
К тому же, определенная сатисфакция уже была получена со сбитым русским
самолетом над Синаем. Так что втюхать эту историю как-то серьезно уже не
получится. Все равно, что обвинить Кремль в расстреле польских офицеров под
Катынью. Хороша предъява к обеду.

\ii{25_05_2018.fb.lesev_igor.1.mh17.cmt}
