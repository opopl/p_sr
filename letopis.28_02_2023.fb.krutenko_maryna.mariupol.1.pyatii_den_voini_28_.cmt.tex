% vim: keymap=russian-jcukenwin
%%beginhead 
 
%%file 28_02_2023.fb.krutenko_maryna.mariupol.1.pyatii_den_voini_28_.cmt
%%parent 28_02_2023.fb.krutenko_maryna.mariupol.1.pyatii_den_voini_28_
 
%%url 
 
%%author_id 
%%date 
 
%%tags 
%%title 
 
%%endhead 

\qqSecCmt

\iusr{Irina Matushina}

Каждый день мне всё тяжелее это читать....

\begin{itemize} % {
\iusr{Maryna Krutenko}
\textbf{Ирина Матюшина} это только начало! Запасись валерьянкой (можно было бы попкорном, если бы это была комедия)
\end{itemize} % }

\iusr{Ольга Паламарчук}

28 к нам в дом пришли украинские военные. Они сказали, уходите ближе к центру,
здесь будет прорыв. Мы ушли на Кирова в бывший садик и дом свой больше не
увидели, на месте нашей улицы руины.

\iusr{Анастасия Верещагина}

😔💔

\iusr{Андрей Лысенко}

Так, так все і було!! Я пам'ятаю як мене питали навіщо я закрашиваю стовпи!

\begin{itemize} % {
\iusr{Maryna Krutenko}
\textbf{Андрей Лысенко} 

нас не те що питали, нас навіть не побили! Казали, що дерева помітили
Зеленстрой, що дерева хворі і їх будуть спілювати. Коли я показала мітки на
всіх деревах, вони почали задумуватись 😮🙅

\end{itemize} % }

\iusr{Андрей Мартынов}

Не так страшен враг как брат..

\iusr{Helen Terliga}

Страшно этл все вспоминать, страшно даже сейчас больше, думаю о том, почему мы
так долго там оставались... и, действительно, мы думали, что все очень быстро
закончиться

\begin{itemize} % {
\iusr{Maryna Krutenko}
\textbf{Helen Terliga} 

а у нас был выбор? До 15 марта, выбора вообще не было. Мы выехали в первый день
так называемого \enquote{зеленого} коридора.

\iusr{Helen Terliga}
\textbf{Maryna Krutenko} в первые дни были эвакуационные поезда и автобусы от ж/д вокзала

\iusr{Maryna Krutenko}
\textbf{Helen Terliga} в первые два дня мы ещё ума не набрались)))))

\iusr{Helen Terliga}
\textbf{Maryna Krutenko} мы не собирались уезжать, думали будет, как в 2014г. и к нам на Черемушки не достанет, ошибались...

\iusr{Maryna Krutenko}
\textbf{Helen Terliga} мы тоже так думали!

\end{itemize} % }

\iusr{Svetlana Didenko}

В общежитие возле рынка \enquote{Застава} привезли много пострадавших от обстрела а
Сартане. Я пошла в АТБ, там полки с конфетами и канцелярия. Купила конфеты,
альбомы, ручки. Все, чтобы как-то развлечь детей. Отнесла. Казалось, война
пришла только к ним, и скоро все закончится. Приморский район. Вечный огонь.
Гимназия 2. Хлеба уже не было...

\iusr{Анна Кузьменко}

Я помню этот день, на нас тогда полицию вызвали, уж очень мы с тобой подозрительными казались.

\iusr{Maryna Krutenko}

Мы с тобой крепкие орешки! Пока наши мужчины в госпитале помогали, мы на своём фронте работали. 😉
