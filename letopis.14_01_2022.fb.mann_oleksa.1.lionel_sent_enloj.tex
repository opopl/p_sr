% vim: keymap=russian-jcukenwin
%%beginhead 
 
%%file 14_01_2022.fb.mann_oleksa.1.lionel_sent_enloj
%%parent 14_01_2022
 
%%url https://www.facebook.com/olexa.mann/posts/5495169327165948
 
%%author_id mann_oleksa
%%date 
 
%%tags gaiti,hudozhnik,isskustvo,kartina
%%title Ліонел Сент-Елой
 
%%endhead 
 
\subsection{Ліонел Сент-Елой}
\label{sec:14_01_2022.fb.mann_oleksa.1.lionel_sent_enloj}
 
\Purl{https://www.facebook.com/olexa.mann/posts/5495169327165948}
\ifcmt
 author_begin
   author_id mann_oleksa
 author_end
\fi

Ніколи не стикався з вудуістичним сучасним мистецтвом. Хоча завжди цікавився,
що відбувається на Гаїті. Звісно, що я знав про «Острів магії» Вільяма Сібрука.
Цікавився по мірі сил вуду, культом Смерті і що там за космогонія, в рамках
інтересу до антропології. І життям дикого упиря «Папаши» Дювальє, який
представлявся бароном Суботою і засновувався на вудуізмі в своїй диктатурі,
також цікавився. І різна інформація про абсолютно реальну зомбацію крестьян на
Гаїті, з промиванням мізків, закопуванням живими в землю, примусовою зміною
особистості і таким іншим теж поступала. 

\ii{14_01_2022.fb.mann_oleksa.1.lionel_sent_enloj.pic.1}

А ось з мистецтвом якось не складалось. 

А тут ось потрапив на такого цікавого гаітянського художника. Звати його Святий
Елой, що вже забов’язало мене з ним познайомитись поближче.

\ii{14_01_2022.fb.mann_oleksa.1.lionel_sent_enloj.pic.2}

«Ліонел Сент-Елой народився в 1950 році в Порт-о-Пренс. 

Святой Елой син маляра і швачки. Це художник церемоній вуду. Він живе в
багатоповерховому будинку, схожому на замок, із звивистими сходами на пагорбі в
столиці — будівлі, настільки ж фантастичної, як і бачення, яке він створює на
полотні. 

Святий Елой також скульптор, виготовляє химерних ангелів з металу, дзеркал,
друшляків, ланцюгів, віконних рам та інших знайдених матеріалів.

\ii{14_01_2022.fb.mann_oleksa.1.lionel_sent_enloj.pic.3}

Він є учасником мистецького руху, відомого як «Пото-Мітан», де він малює, грає
на музичних інструментах і танцює. 

\ii{14_01_2022.fb.mann_oleksa.1.lionel_sent_enloj.pic.4}

Він є одним з найбільш ексцентричних і цікавих гаїтянських художників. Зазвичай
його роботи це фантасмогоричні і релігійні уявлення на тему вуду або
християнства. Він також відомий тим, що виготовляє прапори Вуду.»

На першій композиції зображена Народна армія папаши Дювальє в дії. Робота
зроблена Святим Елоєм у 1986 році.

\ii{14_01_2022.fb.mann_oleksa.1.lionel_sent_enloj.pic.5}
