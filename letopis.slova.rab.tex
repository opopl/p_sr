% vim: keymap=russian-jcukenwin
%%beginhead 
 
%%file slova.rab
%%parent slova
 
%%url 
 
%%author 
%%author_id 
%%author_url 
 
%%tags 
%%title 
 
%%endhead 
\chapter{Раб}

%%%cit
%%%cit_head
%%%cit_pic
%%%cit_text
Интересно, как они будут объяснять отрывок из летописи (1015 год) о князе
Владимире: Держат в памяти его русские люди, вспоминая святое крещение, и
прославляют Бога молитвами, песнями и псалмами, воспевая их Господу, новые
люди, просвещенные Святым Духом, ожидая надежды нашей, великого Бога и Спаса
нашего Иисуса Христа; он придет воздать каждому по трудам его неизреченную
радость, которую предстоит получить всем христианам.  Следуя всё той же
«украинской» логике получается, что князь крестил только русских людей, то есть
\emph{рабов}, а русам-украинцам крещение было без надобности
%%%cit_comment
%%%cit_title
\citTitle{В чём нас обвиняют на сей раз}, Досужник, zen.yandex.ru, 02.08.2021
%%%endcit
