% vim: keymap=russian-jcukenwin
%%beginhead 
 
%%file 08_09_2021.fb.krjukova_svetlana.1.svetofor_podderzhka_voditel
%%parent 08_09_2021
 
%%url https://www.facebook.com/kryukova/posts/10159650675653064
 
%%author_id krjukova_svetlana
%%date 
 
%%tags cenzura,kiev,krjukova_svetlana,podderzhka,strana_ua,ukraina,voditel
%%title Помощь приходит в том месте, где ее совершено не ждёшь. И ты понимаешь. Что все было не зря
 
%%endhead 
 
\subsection{Помощь приходит в том месте, где ее совершено не ждёшь. И ты понимаешь. Что все было не зря}
\label{sec:08_09_2021.fb.krjukova_svetlana.1.svetofor_podderzhka_voditel}
 
\Purl{https://www.facebook.com/kryukova/posts/10159650675653064}
\ifcmt
 author_begin
   author_id krjukova_svetlana
 author_end
\fi

\obeycr
Красный как огненная Луна светофор фиксирует остановку. В этот вечер автомобили на Бульваре Леси притормаживают со свистом как участники киношного ралли. 
Водитель справа внимательно разглядывает водителя слева, меня, через закрытое и, вот черт, не тонированное стекло. Сигналит тремя короткими неглубокими звуками, пытаясь привлечь внимание. Я открываю окно, рассчитывая на плохие вести. По всей видимости лопнуло колесо, или  по обыкновению еду с открытым багажником, из которого выпадают ленты и апельсины. 
- Приветствую! 
- Привет. 
- Меня зовут Сергей. 
- А. Привет, Сергей. У меня колесо пробило?
- Нет. 
- Нет?
-Я просто хотел сказать «Спасибо!»
- Ээээ. Спасибо...? За что? 
- За то, что боретесь. И не сдаётесь. У вас много союзников. Вы просто об этом пока ещё не знаете. 
И вот желтый, зелёный и ты мчишь как на крыльях. 
Всегда так. Ждёшь импульса от своих, от него, от неё, от тех самых. Но помощь приходит в том месте, где ее совершено не ждёшь. 
И ты понимаешь. 
Что все было не зря.
\restorecr

\ii{08_09_2021.fb.krjukova_svetlana.1.svetofor_podderzhka_voditel.cmt}
