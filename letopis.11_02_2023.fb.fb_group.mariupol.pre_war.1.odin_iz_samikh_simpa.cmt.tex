% vim: keymap=russian-jcukenwin
%%beginhead 
 
%%file 11_02_2023.fb.fb_group.mariupol.pre_war.1.odin_iz_samikh_simpa.cmt
%%parent 11_02_2023.fb.fb_group.mariupol.pre_war.1.odin_iz_samikh_simpa
 
%%url 
 
%%author_id 
%%date 
 
%%tags 
%%title 
 
%%endhead 

\qqSecCmt

\iusr{Тимофій Неживенко}

Знакомое место? В районе Площади на Левом?

\begin{itemize} % {
\iusr{Eleonora Ehlkina}
\textbf{Тимофій Неживенко} дворы пр. Мира - Архитектора Нильсена, напротив Драмтеатра

\iusr{Alex Rud}
\textbf{Тимофій Неживенко} 

Нет это не на Левом, это в самом центре, двор дома Энгельса 39, проще - за
Детским миром (ПУМБОМ). Но Вы правы, дворы в районе площади на Левом очень
похожи.

\iusr{Катаріна Шрамко}
\textbf{Тимофій Неживенко} 

на левый похожи только дома выкрашенные, а так кирпичных таких в районе площади
не было. Но я тоже не угадала, подумала, что это дворы за домом со шпилем, а
это чуть выше, правильно?

\end{itemize} % }

\iusr{Людмила Миснянкина}

Это мой любимый дом и двор возле Драмтеатра.Здесь прошла моя юность, зрелые
годы и вот старость. Мой родной дом - ты мужественно выстоял под обстрелами,
спасая нас от смерти, укрывал и согревал нас твой подвал, где мы укрывались от
холода и бомбежки. Спасибо тебе большое. Не дай Бог, чтобы это ещё
повторилось.

\iusr{Elena Ganzina}

Сколько красивых дворов было в нашем любимом Мариуполе.....

\begin{itemize} % {
\iusr{Alex Rud}
\textbf{Елена Ганзина} Да, Мариуполь весь родной, для меня все улицы и все дворы родные. Есть еще фото, со временем поделюсь.

\iusr{Elena Ganzina}
\textbf{Alex Rud} спасибо, очень жду я и все мариупольчане!
\end{itemize} % }

\iusr{Vadim Melnik}

У меня в этом дворе родители жили до 2020-го...

\iusr{Лена Куцарская}

А у нас уже нет ни квартиры, ни дома, ни двора. Только вид на море остался.

\iusr{Олена Сугак}

За ПУМБ (детским миром). Клумба виднеется - был когда-то фонтан.
