% vim: keymap=russian-jcukenwin
%%beginhead 
 
%%file 06_04_2021.fb.promovugroup.2.mova
%%parent 06_04_2021
 
%%url https://www.facebook.com/groups/promovugroup/permalink/948490065724855/
 
%%author 
%%author_id 
%%author_url 
 
%%tags 
%%title 
 
%%endhead 

\subsection{Аваковсько-менделівське «українська російська» зачепило}
\label{sec:06_04_2021.fb.promovugroup.2.mova}
\Purl{https://www.facebook.com/groups/promovugroup/permalink/948490065724855/}

Аваковсько-менделівське «українська російська» зачепило.
Вчора видав блог про це - і полетіло мені: а як же швейцарська німецька? А як американська англійська? Можна ж?
Ага. І в Латинській америці більше десятка різних іспанських.
Хоча в тому ж Парагваї – ще і гуарані є державною. А кечуа і аймара – другою державною мовою Болівії і Перу.
Проте, як створювалися ці нації в двох реченнях не розкажеш.
І так, дійсно, і раніше, і зараз є багато етносів двомовних. Ну, ті ж кечуа чи гуарані.
Схожа з нашою ситуація – здавалося б - є ситуація в Ірландії.
Де ірландська активно витіснялася протягом століть, коли ірландський народ був гнобленим у Британській імперії.
Де кращі представники народу, як от Оскар Вайлд в гонитві за визнанням, писали (по аналогії з нашим Гоголем) – англійською.
Але ми розуміємо, що мова – не єдиний (хоч і головний) маркер для самоідентифікації.
От у ірландців – це була релігія. Можна було говорити англійською, але бути добрим католиком – і тебе сприймали як свого.
У нас напаки. Більшість українців були православними.
Тому Шевченкова мова і виділялася, як народна і своя. Маркер.
Інший відомий маркер самоідентефікації – протиставлення. Ми – не такі, як вони. Бо вони схизматики/гої/роми/євреї/ізвращенци/орки/кавказці/тощо – а ми - нормальні.
Але я знаю, що цей маркер якраз і використовують ті, хто розпалює мову ненависті. І це гарний компост для послідовників Адольфа Алоїзовича.
Тому хотів би зазначити, що нормальний етнолог не скаже – що котрийся маркер самоідентичності є визначальним. Але при цьому скажуть, що якщо їх спробувати рейтингувати, то мова мабуть буде здаватися більшим маркером.
Тому, щоб визначитися, хто ж насправді правий – Аваков і Мендель? Чи ті хто заперечують існування якогось варіанту російської для українців?
Спробуємо таку тезу. Із зовсім іншої площини.
1) Щоб сталося, якби міністр внутрішніх справ Ірландії заявив, що він англомовний націоналіст, ірландською не говорить, але він найкращий фахівець і тому фіолетово, якою мовою забезпечувати захист простих людей? Бо відомо ж, що такі цінності як ковбаса і безпека – цінніші за гідність чи мову спілкування? 
І хтось, безумовно, кивне – все правильно чувак сказав. А що не так?
Але ж ми розуміємо, що там не тільки міністр, там увесь уряд після таких слів піде у відставку. І це при тому, що Ірландія і Британія вже сто років як припинили війну і живуть мирно.
А чи не тому Ірландія у Європі, а ми ні? Може тут справа у відношенні міністрів до народу? Чи нас у оцінці їх дій? Чи готова країна до вступу в Євросоюз, якщо міністр, який має забезпечувати захист – займається протилежним?
2) Інша ситуація. Кому вигідно, щоб міністр свідомо хизувався своїм просуванням в маси мови – офіційної у державі-агресорі.
От де зараз Авакову і Мендель аплодують? Невже не у Кремлі?
І для зовсім диких. В усьому світі бережуть зникаючі мови. Мови народів, які не мають власної державності
У тих же Нідерландах уряд робить надзусилля для збереження фризької мови. 
А у нас треба підтримувати кримсько-татарську і гагаузьку.
Ну, приватно можна бути і російськомовним націоналістом.
Але російськомовний міністр в Україні має піти у відставку.
А не нагло нести пургу на догоду Москві.
Ну, і Мендель це теж стосується.
