% vim: keymap=russian-jcukenwin
%%beginhead 
 
%%file 30_01_2022.fb.fb_group.story_kiev_ua.1.ulica_bogdana_hmelnickogo_10.cmt
%%parent 30_01_2022.fb.fb_group.story_kiev_ua.1.ulica_bogdana_hmelnickogo_10
 
%%url 
 
%%author_id 
%%date 
 
%%tags 
%%title 
 
%%endhead 
\zzSecCmt

\begin{itemize} % {
\iusr{Dmitriy Gernak}
Мой родной дом. Спасибо за пост.

\iusr{Анатолий Борозенец}
\textbf{Dmitriy Gernak} Будь ласка.

\iusr{Ксения Погорлецкая}
В Советское время балконы были. В здании находился Институт педагогики. В конце 70-начале 80-х сотрудники выходили на балконы.

\begin{itemize} % {
\iusr{Валерий Лысенко}
\textbf{Ксения Погорлецкая} Мой отец с сотрудниками

\ifcmt
  ig https://scontent-frx5-1.xx.fbcdn.net/v/t39.30808-6/272807083_5042892565755287_903200249462058807_n.jpg?_nc_cat=105&ccb=1-5&_nc_sid=dbeb18&_nc_ohc=7CNGF6lgk6gAX-1LmXv&_nc_ht=scontent-frx5-1.xx&oh=00_AT8YcrBSXxo4Mm9kdR4wxwAFAzkxgptU-kjfdrHtNgeLGA&oe=61FAA54A
  @width 0.3
\fi

\iusr{Ксения Погорлецкая}
\textbf{Валерий Лысенко} Он работал в институте педагогики? Фамилию не вспоминаю. Работала там в 77-79 г.г. в секторе физики.

\iusr{Анатолий Борозенец}
\textbf{Ксения Погорлецкая} Так балкони були, принаймні, в 1950-1970-х. Ось цей знімок зроблено з балкону будинку №10

\ifcmt
  ig https://scontent-frt3-2.xx.fbcdn.net/v/t39.30808-6/273112584_3099629656959414_746420271947957747_n.jpg?_nc_cat=103&ccb=1-5&_nc_sid=dbeb18&_nc_ohc=fUipxcuIbiIAX9a_VIE&_nc_ht=scontent-frt3-2.xx&oh=00_AT-V-tBw1iQh3F3w3Fg2FeKl3tSZWHxQbkdCP3PxkBkGCg&oe=61FBE72C
  @width 0.2
\fi

\begin{itemize} % {
\iusr{Ксения Погорлецкая}
\textbf{Анатолий Борозенец} Институт работал - кажется - до середины 90 в этом здании. Балконы, наверное, были демонтированы после. В 95 жила в этом доме и балконы еще были - вспомнила!

\iusr{Ксения Погорлецкая}
Фото возвращают в те годы, в наш цветущий Киев.

\iusr{Анатолий Борозенец}
\textbf{Ксения Погорлецкая} Мешканці будинку цілком довіряю в цьому питанні ) Я орієнтувався на фотографії, що мав, а їх не так багато. Дякую за уточнення!
\end{itemize} % }

\end{itemize} % }

\iusr{Ольга Арканова-Чёрная}

Не правда - балконы были и мы выходили на них. И учебники там писали правильные
и правдивые. Потом здание УНДИПа продали и лаборатории перевели на окраину
города. Конечно! Кому нужна педагогика!

\begin{itemize} % {
\iusr{Alexey Novozhylov}
\textbf{Olga Arkanova-Chornaya} Про Фінляндію я так розумію теж пісалі учєбнікі правільниє і правдівиє??  @igg{fbicon.face.tears.of.joy}{repeat=3} 

\iusr{Татьяна Соловьева}
\textbf{Ольга Арканова-Чёрная} Моя мама работала в УНДИПе с 1974 по 2006 год. Ленина, 10 - столько раз приходила к ней на работу)

\iusr{Ксения Погорлецкая}
\textbf{Татьяна Соловьева} Возможно, помню Вашу маму.

\iusr{Ольга Арканова-Чёрная}
\textbf{Татьяна Соловьева} у меня отец возглавлял лабораторию методики физики

\iusr{Tamara Sorochan}
\textbf{Ольга Арканова-Чёрная} да, перевели, но не так уж и на окраину. Січових стрільців, 52А.
\end{itemize} % }

\iusr{Ирина Дрюкова}
Спасибо за информацию это мой родной дом я здесь родилась

\iusr{Анатолий Борозенец}
\textbf{Ирина Дрюкова} Будь ласка.

\iusr{Наталия Жминко Сычевска}

На счет Аллы Тарасовой - на Печерске, в районе Генеральной прокуратуры, была
школа номер 7. Там еще есть табличка, что ее закончила Тарасова. Или ?

\end{itemize} % }
