% vim: keymap=russian-jcukenwin
%%beginhead 
 
%%file 19_07_2020.fb.lnr.5
%%parent 19_07_2020
 
%%endhead 

\subsection{Сегодня Михаилу Толстых исполнилось бы 40 лет.}
\label{sec:19_07_2020.fb.lnr.5}
\url{https://www.facebook.com/groups/LNRGUMO/permalink/2860052257439663/}
  
\vspace{0.5cm}
{\small\LaTeX~section: \verb|19_07_2020.fb.lnr.5| project: \verb|letopis| rootid: \verb|p_saintrussia|}
\vspace{0.5cm}

\index{Люди!Михаил Толстых (Гиви)}

Его военная карьера развивалась стремительно. Уже в 2014 году он возглавлял
отряд ополченцев, которые прошли Славянск и Миусинск и вместе с командиром
укрепили свой военный авторитет победой в битвах за Иловайск.

Все, кто знал Гиви, отмечают его решительность, работоспособность и
хладнокровие перед лицом опасности. Он был военачальником от бога. Вехи его
жизненного пути отражены в документальных свидетельствах, которые мама Михаила
Толстых, Нина Михайловна, передала в столичный Музей Великой Отечественной
войны, где они и хранятся.

Со дня убийства легендарного комбата прошло три с половиной года. Боль от
потери не утихает, как и горячее желание добиться наказания для всех причастных
к его смерти.

Мы навсегда запомним Гиви как человека слова и дела, верного товарища, на
которого всегда можно было положиться. 
