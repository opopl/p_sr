% vim: keymap=russian-jcukenwin
%%beginhead 
 
%%file 26_11_2021.fb.baumejster_andrej.kiev.filosof.1.schastje_zhurnalist_politolog
%%parent 26_11_2021
 
%%url https://www.facebook.com/andriibaumeister/posts/4475665615888296
 
%%author_id baumejster_andrej.kiev.filosof
%%date 
 
%%tags obschestvo,politika,politologia,schastje,strana,ukraina,zhurnalist
%%title Какое счастье быть украинским политологом. Какое счастье быть украинским журналистом
 
%%endhead 
 
\subsection{Какое счастье быть украинским политологом. Какое счастье быть украинским журналистом}
\label{sec:26_11_2021.fb.baumejster_andrej.kiev.filosof.1.schastje_zhurnalist_politolog}
 
\Purl{https://www.facebook.com/andriibaumeister/posts/4475665615888296}
\ifcmt
 author_begin
   author_id baumejster_andrej.kiev.filosof
 author_end
\fi

Какое счастье быть украинским политологом. Какое счастье быть украинским
журналистом. Всегда есть темы для нескончаемой болтовни. 

Какая радость для интеллектуалов и обитателей социальных сетей в нашем
отечестве. На сетевых кухнях иронизировать, зубоскалить, придумывать мемы,
упиваться собственным превосходством. И теснее сжимать ряды. 

Это наш ответ на проект Мета. Глубокий уход в крутой виртуал. Всё дальше от
реальности, все дальше от тем и вопросов, которые требуют обсуждений и
немедленных решений.

И то, что идёт параллельно с мемотворчеством и зубоскальством: торговля
страхами и тревогами. Раздача обвинений в предательстве, в работе на врагов,
наконец, во всех смертных грехах.

Блаженное веселье и страх, зубоскальство и ненависть, мания величия и
болезненная зависть идущие вместе. 

Все это фон последних двух месяцев и сегодняшнего дня.

И совсем вдогонку. Так, как некоторые журналисты сегодня позволяли себе
говорить с президентом Зеленским (Бутусов и Ко), - это за гранью. Президент
может быть неправ. Он может ошибаться. Он может у некоторых вызывать различные,
даже недобрые чувства.

Но свободный человек должен вести себя благородно. А раб срывается в хамство.
Особенно, если знает, что ему это хамство сойдёт с рук. И даже добавит ему
очков. Когда же раб чует опасность, он льстит, пытается услужить и заискивающе
смотрит в глаза.

До сих пор актуальна мысль Аристотеля: свободный человек умеет управлять и
умеет подчиняться. Только сегодня вместо подчинения я бы говорил об умении
понимать природу власти.

Надеюсь, я никого не задел и не обидел. Это всего лишь беглые впечатления...

\ii{26_11_2021.fb.baumejster_andrej.kiev.filosof.1.schastje_zhurnalist_politolog.cmt}
