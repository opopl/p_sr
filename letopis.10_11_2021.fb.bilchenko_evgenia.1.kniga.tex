% vim: keymap=russian-jcukenwin
%%beginhead 
 
%%file 10_11_2021.fb.bilchenko_evgenia.1.kniga
%%parent 10_11_2021
 
%%url https://www.facebook.com/yevzhik/posts/4426359244065818
 
%%author_id bilchenko_evgenia
%%date 
 
%%tags bilchenko_evgenia,kniga
%%title БЖ. Моя книга - уже во Вселенной
 
%%endhead 
 
\subsection{БЖ. Моя книга - уже во Вселенной}
\label{sec:10_11_2021.fb.bilchenko_evgenia.1.kniga}
 
\Purl{https://www.facebook.com/yevzhik/posts/4426359244065818}
\ifcmt
 author_begin
   author_id bilchenko_evgenia
 author_end
\fi

БЖ. Моя книга - уже во Вселенной

Проект "Сентиментальное насилие либерализма", который начался в Киеве благодаря
Олег Никоф и издательству "Дом Федорова", был мной переработан и стал ПОЛНЫМ
достоянием прав "Алетейи". В результате к Стефания Данилова как к автору
предисловия прибавились Александр Секацкий и Григорий Тульчинский. Самый крутой
питерский традиционалист и самый известный культуролог постсоветского мира.
Потом книга прошла без меня презентацию в ВШЭ. И это было, судя по фото, круто,
на знаменитых октябрьских платформах. Сейчас все это повезде лежит в околицах
Невского, и, когда я шла, у меня был сюр: доктор наук, продаваемых в ведущих
магазинах самой большой страны, ставший бомжом на Украине, работает корректором
и безвестным литературным исполнителем.  Бродский, кури и молчи. Лимонов, вот
сейчас, вообще сиди на небесах и не отсвечивай: такая же у меня жизнь, БЖ не
жалуется. БЖ опять завернула на курс бытия-к-смерии, так проще. 

\ifcmt
  ig https://scontent-lga3-1.xx.fbcdn.net/v/t1.6435-9/254679950_4426359167399159_1566943219663089587_n.jpg?_nc_cat=101&ccb=1-5&_nc_sid=8bfeb9&_nc_ohc=RRW-YrErU_oAX8X3wk7&_nc_ht=scontent-lga3-1.xx&oh=b116d4cbe19db5f8e9ba004e40fa7a63&oe=61B2D668
  @width 0.4
  %@wrap \parpic[r]
  @wrap \InsertBoxR{0}
\fi

Ноябрь. Вчера. Пишу на ходу, в пути, на коленке, под сильным впечатлением от
ночных бесед с женой Виктора Малахова Татьяной Чайкой из Наарии, предисловие к
английской версии этого дела, подписываю договор с Братиславой: теперь у меня
нет авторских прав ни на русскую, ни на английскую версии. Зато есть
возможность познакомиться с будущими рецензентами: это - Наоми Кляйн и Славой
Жижек. Братислава все ещё ищет питерских культурологов, ау!

Приведу вам текст нового предисловия как анонс. Мне разрешили коллеги, с
которыми в Словакии вчера подписан договор.

Евгения Бильченко:

"Сентиментальное насилие либерализма: от шока к китчу" - уникальная в своем
роде книга. Это первое, системно изложенное, написанное доступным для каждого
читателя языком, глубокое научное исследование механизмов контроля глобального
мира цифрового капитализма, в котором живём все мы. Показательно, что эта книга
не содержит никакой эзотерики и никакой показной конспирологии, потому что
автор считает, что истерическая подозрительность, как любая экзальтация, не
только не разрушает объект критики, но и возвышает его в собственных глазах. А
нам не нужны мифы - нам нужна правда и рациональные знания в пространстве, где
истина больше не скрывается за ложью, но играет роль самой себя превращаясь в
хайп. Так что, объект критики у автора - очень серьезный. Серьезный, потому что
неуловимый. Ибо мы живём в мире несвободы, которая называет себя свободой. Мы
существуем в мире политики, претендующей на аполитичность. Мы барахтаемся в
бесконечной войне за мир, именующей себя пацифизмом. Когда фашизм называет себя
антифашизмом, а агрессивность разыгрывает перформанс толерантности,
сопротивляться - гораздо сложнее, чем в классическом тоталитарном режиме ХХ
века, с его вождизмом, однопарттйностью и прямыми директивными запретами. Здесь
- все совершенно иначе. У нас нет слов для описания насилия, приходящего к нам
в улыбчивый маске большой лояльности, за которой скрывается смерть любого
Иного.  Речь идёт о манипуляциях, осуществляемых транснациональными элитами в
странах-колониях глобализма, среди которых находится и Украина. 

Отметим, что именно за научную критику американизма под предлогом русофобии
автор - умеренный антиглобалист - была уволена с работы, будучи самым молодым
за всю историю Украины доктором наук - она получила докторскую степень в 32
года и должность профессора в 33. Главная идея книги - заимствованная из
фильмов Д. Линча стратегия "ленты Мёбиуса". Лента имеет всего одну сторону, но
кажется, что две. За каждой блондинкой скрывается брюнетка. За каждой помадой -
кровь. За любовью - ненависть. За милосердием - жестокость. Но это - особое,
сентиментальное либеральное насилие. Оно - символично. Оно не выходит на улицу
напрямую в виде неонациста со свастикой на предплечье: оно спонсирует его через
креативные классы, облагораживая до менеджера. Оно на запрещает правду факта,
но смешивает факты и ложь в едином показном потоке, предлагая нам правду более
страшную, чем правда факта: правду наших фатальных желаний, обыкновенный фашизм
внутри нас. Отныне ничего не надо скрывать. Достаточно сделать из нескрываемого
зла медиа-спектакль в виде поклонников Вермахта, которые разгуливают в Киеве в
шапках Санта-Клауса, раздавая детям шоколадки, или в виде постмодерных проектов
рекламной памяти, уравнивающей агрессора и жертву в виде культа вечной
виктимности, когда агрессор становится жертвой. 

Сначала к нам обращаются на языке якобы гуманности и диалога. Затем нам
предлагают "сломать стереотипы" и "релятивировать" устойчивые ценности.
Кажется, что плохого в релятивации? "Ведь Гитлер носил детские платьица и тоже
любил маму... А итальянские антифашисты, ликвидировавшие колонну СС СД, - не
такие уж и герои, ведь именно из-за них бедных граждан Рима нацисты казнили в
Ардеатинских пещерах". Так начинается ад: ангелы становятся демонами, а демоны
- ангелами.  Уравнение гитлеризма и коммунизма как "одинаково агрессивных"
проектов постепенно, следуя окну Овертона, приводит к оправданию гитлеризма как
жертвы. То есть - к потаканию агрессии и к поощрению абсолютного социального
зла.  Пассивная толерантность постмодерна оборачивается репрессивной: сквозь
слащавую завесу, драпированную либеральной терпимостью, проступает хищный оскал
неонацизма. Мультикультурализм не просто поощряет все виды извращений, анархия
- это ещё полбеды. Он отдает предпочтение наиболее худшим из них. Так идея
уважения прав Другого превращается в синдром общих своих, среди которых есть
"хорошие", одомашненные, другие, которые нас больше не пугают: шахид в
Макдональдсе или русский с книгой Алексея Навального в руках - это "хорошие
другие"...

Но русский, живущий на Донбассе и читающий Захара Прилепина, - избыток и
подлежит уничтожению. То же касается "неправильно" чтущих Холокост евреев, не
согласных с толеризацией убийц своих предков, "неправильных" украинцев,
считающих себя цивилизационно русскими, и так далее. Сколько имен нам
придумали, чтобы не допустить нашего сближения,чтобы держать нас на безопасном
расстоянии искусственной политики идентичностей, периодически стравливая нас, а
потом вновь побуждая к некому, разыгранному по нотам либеральной гегемонии,
диалогу! 

Искажение антифашистской идеи диалога - это внутреннее убийство Бога как
Большого Другого в каждом из нас. Поэтому в условиях, когда диалог стал
разменной монетой символического обмена баннерами, автор ищет иные пути
развития общества, которые он видит в возрождении традиции как поэтической силы
Логоса, способной привести за собой консервативную революцию. Когда либерализм
стал властью, традиционализм стал контркультурой. Автор - убежденный
контркультурщик. Но в мире сентиментального насилия контркультура - это медаль
прадеда с георгиевской лентой, а не цветные волосы хипстера. 

Всякая традиция, если она - поэтична, открыта миру благодаря своей
чувствительности. Поэтому автор не противоставляет традицию и универсум,
оставаясь русским поэтом и при этом человеком мира, интернационалистом и
антифашистом. Автор видит универсальную этику как левое крыло, а традиционные
ценности как правое крыло не оседланного глобализмом мира: мира людей космоса и
мира людей цивилизации. Но не правых и левых либералов, ибо обе эти группы
Только диалог между условно "правыми" и условно  "левыми" антиглобалистами,
консерваторами и марксистами, универсалистами и традиционалистами снимет сами
условности деления нас, при помощи которого нами управляют. 

Люди не делятся в храме, в окопе и на курилке. Пусть эта книга станет для вас
храмом, окопом, курилкой и тем, что вы сами себе представляете, когда говорите:
"Я не преодолеваю смерть, я просто знаю место, где для смерти нет места, - это
любовь".

\ii{10_11_2021.fb.bilchenko_evgenia.1.kniga.cmt}
