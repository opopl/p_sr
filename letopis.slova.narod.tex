% vim: keymap=russian-jcukenwin
%%beginhead 
 
%%file slova.narod
%%parent slova
 
%%url 
 
%%author 
%%author_id 
%%author_url 
 
%%tags 
%%title 
 
%%endhead 
\chapter{Народ}
\label{sec:slova.narod}

Что я хочу предложить и почему. Сначала почему. Я считаю, что \emph{Народ}
совершенен. Любой \emph{Народ}. Его менталитет всегда и везде сформирован
внешними условиями. История, традиции действует на людей только благодаря
пропаганде власти. Наши дети уже не знают, кто такой Ленин, не говоря о
Брежневе и Хрущеве. После 1991 уже родилось два поколения новых людей,
\textbf{Не нужно никого лишать избирательного права}, Павел Себастьянович,
strana.ua, 27.05.2021

...Одна из проблем, которую нельзя решить высокоточными ракетами, - миллиарды
недорослей, недоучек, недоразвитков. Примитивные \emph{народы} умели
воспитывать своих мальчиков и девочек. Простая культура целиком влезала в одну
голову, и в каждой голове были необходимые элементы этики и религии, а не
только техническая информация. Культура была духовным и нравственным целым.
Естественным примером этой цельности оставались отец с матерью. Сейчас они
банкроты. Тинейджер, овладевший компьютером, считает себя намного умнее деда,
пишущего авторучкой.  Мир изменился, каждые пять лет он другой, и все старое
сбрасывается с корабля современности. Растут миллиарды людей, для которых
святыни, открывшиеся малограмотным пастухам, не стоят ломаного гроша. Полчища
Смердяковых, грядущие гунны, тучей скопились над миром. И они в любой день
готовы пойти за Бен Ладеном или Баркашовым. Записку Иконникова гунны не прочли
(а если б и прочли - что им Иконников? Что им князь Мышкин?). Судьбу Другого
они на себя не возьмут… Одно из бедствий современности - глобальная пошлость,
извергаемая в эфир.  Возникает иллюзия, что глобализм и пошлость - синонимы. И
глобализм уже поэтому вызывает яростное сопротивление. Не только этническое. Не
только конфессиональное...,
Григорий Соломонович Померанц (13 марта 1918, Вильно, Литва — 16 февраля 2013, Москва, Россия)

Как можно проводить пир во время чумы. Люба ты высказала мысли 90\% \emph{народа},
\citComment{Валентина Костюченко}, 
\citTitle{Cpoчнo! PE3KOE oбpaщeниe Tитapeнкo к 3eлeнcкoмy ШOKИPOBAЛO Kиeв!},
youtube.com, 07.06.2021

В общем, президентский законопроект о \emph{коренных народах} является
правильным, но к нему есть существенное замечание – в нем четко не очерчено то,
что именно украинцы являются коренным народом. Именно украинцы укоренились
здесь и выросли на этой земле. Разве они – не древний \emph{народ}? Если так
пойдет дальше, мы, украинцы, рискуем остаться без своей территории, без своей
земли.  Закон является правильным, но слишком либеральным – нужно было четко
определиться, кем являемся мы сами, украинцы, потому что пока что в
национальном вопросе у нас полная вакханалия. Мы думали, что вокруг нас
дружественные и братские \emph{народы}, но как только \enquote{старший брат}
посягнул на нашу территорию, то сразу зашевелились и Донбасс, и Крым, и
Закарпатье,
\citTitle{Как законопроект о коренных народах Украины выбил почву из-под ног Москвы - Главред}, 
Григорий Перепелица, opinions.glavred.info, 10.06.2021

%%%cit
%%%cit_pic
%%%cit_text
\emph{Народные} песни. Ночь эта лунная, зорькою ясная, Ночь эта лунная,
зорькою ясная!  Видно, что ночь для чудес.  Выйди любимая, за день уставшая,
Хоть на минуточку в лес.  Сядем, обнимемся тут под калиною, И над панами я пан!
Глянь же ты рыбонька – волной серебристою Стелется полем туман.  Лес
зачарованный, нимбом овеянный, Толь задремал, толи спит. Лишь на высокой и
стройной осине Ветер листвой шелестит.  Небо бескрайнее осыпано зорями – Какая
же божья краса!  Искрами ясными там под деревьями, Брызжет, играет роса.  Ты не
пугайся, что ножками босыми, Станешь в холодную росу, Я же тебя верная, в дом
по тропиночке, сам на руках донесу.  Ты не пугайся замерзнуть лебёдушкой, Тепло
– и нет в небе хмарь.  Я же прижму тебя близко к сердечку, Сердце пылает как
жар. Ты не пугайся, что могут подслушать, О чём ты шептала в лесу.  Ночь
уложила всех, снами окутала – За нами следить не досуг.  Спят твои недруги,
бытом замучены, Не потревожит их смех.  Разве обделенным нам долею нашею
Минутка свидания грех?
%%%cit_comment
%%%cit_title
\citTitle{Украинские Песни Русскими Словами}, 
БРАТИНА, zen.yandex.ru, 15.12.2020
%%%endcit

%%%cit
%%%cit_pic
%%%cit_text
Ох, труден в осмыслении триединый и (ныне трехглавый) \emph{русский народ}. Одни себя
в пограничники записали, окраины от москалей стерегут, другие никак не решат:
из каких именно краев себе на холку «русь» варяжскую призвали и что это вообще
значит
%%%cit_comment
%%%cit_title
\citTitle{Белоруссия: две загадки одного названия...}, 
Исторические Напёрстки, zen.yandex.ru, 28.03.2021
%%%endcit

%%%cit
%%%cit_head
%%%cit_pic
%%%cit_text
Донёс до нас в своих романах Достоевский, Что ложь - как зло, а преступленье -
это грех! Литературная культура - повод веский, Тем самым к лучшему он изменил
нас всех. А если честно это глупое сравненье, Объединяет здесь нас всех -
славянский род, Хотя и разное имеем поведенье, Но москали, да и хохлы - один
\emph{народ}!
%%%cit_comment
%%%cit_title
\citTitle{Как отличить украинца от русского?}, 
Уткин, zen.yandex.ru, 14.06.2021
%%%endcit

%%%cit
%%%cit_head
%%%cit_pic
%%%cit_text
И Рада, и Кабмин, и Офис президента враз опустеют. Истосковавшись по вниманию
СМИ, Лизавета Богуцкая внесла законопроект о введении уголовной ответственности
за надругательство над украинским языком. Тепла цветочку не достает. Понять
можно.  Если наш «ЦК» не планирует какой-нибудь стройки века с привлечением
осужденных, то лишь полоумные отдаст свой голос за этот глупенький
законопроект.  У нас ведь почти вся госслужба и две трети политиков говорят на
чудовищном суржике. Это ж сколько народа за надругательство по этапу пустить
придется? И Рада, и Кабмин, и даже Офис президента враз опустеют
%%%cit_comment
%%%cit_title
\citTitle{Депутат Богуцкая подала закон о надругательстве над украинским языком}, 
Максим Могильницкий, strana.ua, 19.06.2021
%%%endcit


%%%cit
%%%cit_head
%%%cit_pic
%%%cit_text
Именно поэтому, кстати, нас пытаются убедить в том, что сразу после майдана
власть являлась учредительной и честно исполняла волю \emph{народа}.  Правовую оценку
действиям парламента в 2014-м, а также прокуроров, которые расследуют
сфабрикованные дела и ставят решение КСУ под сомнение, дает мой коллега
Владимир Богатырь в своей новой статье. Почитайте. Она интересная.  Меня же
сегодня занимает вопрос отсутствия у Турчинова полномочий для подписания
законов, принятых в период с 23-го февраля по 2-е марта 2014-го. Тогда ведь
Конституция образца 96-го действовала и обязанности президента должен был
исполнять премьер-министр, а не какой-то Турчинов
%%%cit_comment
%%%cit_title
\citTitle{Ранее, чтобы выглядеть прилично, власть меняла Конституцию / Лента соцсетей / Страна}, 
Максим Могильницкий, strana.ua, 28.06.2021
%%%endcit


%%%cit
%%%cit_head
%%%cit_pic
%%%cit_text
Мне все происходящее напоминает период падения Великого Инки с развращенной
\enquote{аристократией}, оторванной от простого \emph{народа}. Впереди - вполне возможно,
пришествие \enquote{конкистадоров}....  Моя цитата из статьи: \enquote{Украина оказалась в
своего рода демографической воронке.  У нас низкая рождаемость и высокая
смертность, в итоге мы ежегодно теряем фактически население одного областного
центра. Плюс - высокая миграция за границу.  В итоге по темпам сокращения
населения мы бьем все возможные антирекорды}, — говорит экономист Алексей Кущ
%%%cit_comment
%%%cit_title
\citTitle{Происходящее в Украине напоминает период падения Великого Инки / Лента соцсетей / Страна}, 
Алексей Кущ, strana.ua, 29.06.2021
%%%endcit

%%%cit
%%%cit_head
%%%cit_pic
\ifcmt
  pic https://img.strana.ua/img/article/3413/zelenskij-prokommentiroval-slova-64_main.jpeg
	width 0.4
\fi
%%%cit_text
\enquote{Давайте, наконец, расставим точки над \enquote{i}. Мы точно не один
\emph{народ}. Да, у нас есть много общего. У нас общая часть истории, память,
соседство, родственники, общие победа над фашизмом и общие трагедии. Да, все
это крайне важно, и мы об этом помним. И, возможно, что-то общее у нас будет в
будущем, если еще не поздно остановить тотальное разделение между нашими
странами. Но мы, повторю еще раз, \emph{не один народ}. Были бы \emph{одним народом}, то в
Москве, скорее всего, ходили бы гривни, а над Государственной думой развивался
желто-голубой флаг. Так что, мы точно не один \emph{народ}. У каждого из нас свой
путь. Но цель у нас действительно должна быть одна - закончить войну на
Донбассе, вернуть Украине ее территории. И еще кое-что - невозможно
одновременно говорить об \enquote{одном народе} и открыто захватывать наши
территории и продолжать бойню на Донбассе.  Это же очевидно}, - заявил
Зеленский
%%%cit_comment
%%%cit_title
\citTitle{Зеленский прокомментировал слова Путина об одном народе}, Игорь Рец, strana.ua, 01.07.2021
%%%endcit


%%%cit
%%%cit_head
%%%cit_pic

\ifcmt
  tab_begin cols=2
		width 0.4
		 caption Владимир Зеленский и \enquote{Квартал} в середине апреля 2014 года на Донбассе 

     pic https://img.strana.ua/img/article/3416/zelenskij-o-russkikh-65_main.jpeg

     pic https://strana.ua/img/forall/u/10/91/%D0%B1%D0%B5%D0%B7%D0%BB%D0%B5%D1%80.png
  tab_end
\fi

%%%cit_text
Третий день подряд в Сети продолжают появляться \enquote{таблетки для памяти} -
заявления Владимира Зеленского семилетней давности.  Связано это с недавними
словами Зе о том, что украинцы и русские - это не один \emph{народ}. Президент
произнес их в ответ на заявления российского коллеги Владимира Путина о том,
что россияне и украинцы - это даже не \emph{братские народы}, а единое целое.  Однако
Зеленскому тут же напомнили, что еще будучи шоуменом он придерживался абсолютно
иной точки зрения на этот вопрос.  Уже после начала боевых действий на Донбассе
нынешний президент Украины, будучи еще актером, высказывался в защиту русского
языка; говорил, что русские и украинцы - \emph{один народ}; утверждал, Украина
заслуживает более серьезных людей в политике, чем он
%%%cit_comment
%%%cit_title
\citTitle{Зеленский о русских и русском языке. Что будущий президент говорил в 2014 году}, 
Екатерина Терехова, strana.ua, 03.07.2021
%%%endcit

%%%cit
%%%cit_head
%%%cit_pic
%%%cit_text
Как там напутствовал Карнеги: не держите все яйца в одной корзине. И вот они
рады тому факту, что их откатили в отдельное лукошко. Орут что те резанные о
своей эксклюзивности и несхожести с русскими. Как смеют там в Москве называть
их \emph{одним народом}. Они иные — бритые, а значит бриты им роднее и ближе. При этом
бреют их все и во всём, отвлекая их внимание на красивые эротические
(патриотические) картинки. Теребят их напряжённые патриотические нервы ложными
поводами. А покумекать возможности просто не существует. Природа так устроила.
Можно сказать, наказала или функцией такой наградила. Им сказали что Петро
Сидоренко не брат Петру Сидорову, при этом Зеленский, Шендерович, Коломойский и
прочие Нетаньяху \emph{одним народом} быть могут
%%%cit_comment
%%%cit_title
\citTitle{Яйца, но не роковые}, Дмитрий Жук (ЦИНИК), zen.yandex.ru, 02.07.2021
%%%endcit

%%%cit
%%%cit_head
%%%cit_pic
%%%cit_text
И Беларусь, как никто нам нужна в примирении нас как внутри Украины, так и с
Россией. Ибо только глупые или лживые политики говорят нам, что проблемы
Украины в режиме Путина или Лукашенко, а сменив их - мы решим все наши невзгоды
и вернём все наши территории. Это не так, Путин и Лукашенко лишь внешнее
проявление воли своих граждан.  И наши проблемы не закончатся никогда, потому
как начиная с 2004 года украинские политики повернули нашу страну против
русских, а последнее время и против белорусов не спросив этого у собственного
украинского \emph{Народа}. Сменить своих соседей мы не сможем, менять нужно
прежде всего политику в отношении собственных граждан не согласных с
националистами и американистами, и политику в отношении России и Беларуси.
Ведь это наши братские \emph{народы}.  Украине пора проснуться от этого пьяного
угара. Пора идти на диалог, делать прагматичные поступки и научиться держать
слово. Прежде всего перед своими людьми - тебя избравшими"
%%%cit_comment
%%%cit_title
\citTitle{Крым и Донбасс - это только начало". Что говорят в Украине о статье Путина}, 
Екатерина Терехова, strana.ua, 13.07.2021
%%%endcit

%%%cit
%%%cit_head
%%%cit_pic
%%%cit_text
В Україні від президента до істориків, публіцистів та просто користувачів
соцмереж обговорюють статтю президента Росії Володимира Путіна, опубліковану на
його сайті у понеділок, де він вчергове обґрунтовує свою тезу про те, що
росіяни і українці – це один \emph{народ}, а отже українців як таких і нема зовсім. І
поки історики – професійні та аматори – доводять безпідставність тез Володимира
Путіна, які переважно повторюють ще радянські шкільні підручники з історії,
спеціалісти з протидії дезінформації радять приглянутися до послання Путіна як
до спецоперації, що виконується в певний час і має свою мету та аудиторію
%%%cit_comment
%%%cit_title
\citTitle{Спецоперація під назвою «стаття Путіна»}, 
Марія Щур; Сашко Шевченко, www.radiosvoboda.org, 13.07.2021
%%%endcit

%%%cit
%%%cit_head
%%%cit_pic
%%%cit_text
Есть другой путь.  Большинство украинских политиков пытаются убедить
\emph{народ} Украины, что у украинской государственности нет иного выхода, чем
быть «анти-Россией»: только борьба с Россией способствует становлению
украинской государственности, только борьба с русским языком, русской культурой
и российской экономикой способна сделать украинское государство процветающим.
Сразу скажу: это ложь. И ложь опасная, способная привести к уничтожению
украинской государственности. Государство, которое объявило смыслом своего
существования вечную борьбу с соседями и превосходство своей расы и культуры
над ними, в Европе это уже было, просуществовало недолго, а конец и вовсе был
позорным, не стоит повторять эти ошибки.  Потому ответом на вызовы, поднятые в
статье В. Путина должны быть следующие политические процессы в Украине
%%%cit_comment
%%%cit_title
\citTitle{О будущем украинского и русского народов}, 
Виктор Медведчук, strana.ua, 15.07.2021
%%%endcit

%%%cit
%%%cit_head
%%%cit_pic
%%%cit_text
Государство с нынешней антироссийской, мононациональной идеологией не может
служить интересам своего \emph{народа} по определению. Власти говорят, что задача
\emph{украинского народа} – защищать «цивилизованную Европу» от «российских варваров».
Но ни один \emph{народ} долго не может существовать исключительно как защитник другого
\emph{народа} от третьего. История показывает, что в этом случае поглощение неизбежно.
Либо тем, от кого защищают, либо тем, кого защищают. И нет никакой альтернативы
этому процессу: либо мы защищаем Европу от России, даже ценой собственного
существования, либо мы защищаем интересы \emph{своего народа} и занимаемся развитием
своей страны, а не приносим себя в жертву на алтарь европейских и заокеанских
интересов
%%%cit_comment
%%%cit_title
\citTitle{О будущем украинского и русского народов}, 
Виктор Медведчук, strana.ua, 15.07.2021
%%%endcit

%%%cit
%%%cit_head
%%%cit_pic
\ifcmt
  pic https://img.strana.ua/img/article/3460/ukraintsy-i-russkie-7_main.jpeg
  width 0.4
	caption Украинцы и россияне - один народ, считает 41\% опрошенных 
\fi
%%%cit_text
Социологическая компания "Рейтинг" опубликовала опрос, который уже стал
сенсацией. Главным образом потому, что украинцев спросили согласны ли они с
тезисом Владимира Путина о том, что русские и украинцы - \emph{один народ}.  Более 40\%
украинцев ответили на этот вопрос "да". И надо понимать, что это еще не
спрашивали у жителей неподконтрольных территорий.  Были и другие столь же
показательные моменты.  "Страна" проанализировала этот опрос.  \emph{Один народ} или
нет?  С утверждением Владимира Путина о том, что украинцы и русские - один и
тот же \emph{народ}, внезапно согласился 41\% опрошенных.  Для Украины, где последние
семь лет шла мощная антироссийская пропаганда, а два самых дружественных к
Москве региона вообще откололись - это, конечно, сенсация.  Надо думать, что с
Крымом и Донбассом эта цифра была бы еще выше и превысила бы 50\% ответов "да"
%%%cit_comment
%%%cit_title
\citTitle{41\% согласных с Путиным. Что показал опрос о единстве украинцев и русских}, 
Максим Минин, strana.ua, 28.07.2021
%%%endcit

%%%cit
%%%cit_head
%%%cit_pic
%%%cit_text
Тому не вони, а ми зобов’язані запитувати Московію скрізь і завжди, що
відбувалося з українцями та іншими \emph{народами} впродовж ХХ–ХХІ ст. Куди поділися
мільйони українців, чому поспіхом, одномоментно українці Ставропілля і Кубані
ставали \enquote{русскіми}, чому в Росії не функціонує жодної україномовної школи, а
українські громадські організації репресовані і закриті?  Не перелічити
злочинів, скоєних проти українських громадян на тимчасово окупованих
територіях. Але чи варто кидати бісер перед тими, хто вкрав вівцю, краде й
вовну? Лише дія, спротив, боротьба до перемоги може уберегти \emph{народ} від
упокорення. Проте мусить бути якісно іншою політика стосовно вічно
\enquote{братерського} ворога
%%%cit_comment
%%%cit_title
\citTitle{Правда в Рідному Слові}, Георгій Філіпчук, slovoprosvity.org, 12.07.2021
%%%endcit

%%%cit
%%%cit_head
%%%cit_pic
\ifcmt
  pic https://odnarodyna.org/sites/default/files/sites/default/files/2021/k04082102.jpg
  width 0.4
\fi
%%%cit_text
4 августа (23 июля по ст. ст.) 1854 года в деревне Заньки Нежинского уезда
Черниговской губернии (ныне Черниговской области) в семье помещика родилась
Марья Константиновна Заньковецкая (Адасовская) (1854-1934) – актриса. Первый
выход на сцену состоялся в 1876 году в Нежине. С юных лет выступала в
любительских концертах. Дебют на профессиональной сцене состоялся в
Елисаветграде в опере «Наталка Полтавка». Служила в крупнейших малороссийских
труппах под руководством М.Л. Кропивницкого, М.П. Старицкого, Н.К. Садовского,
П.К. Саксаганского, И.К. Карпенко-Карого. После Октябрьской революции
возглавляла \emph{Народный театр} в Нежине, участвовала вместе с Саксаганским в
создании \emph{Народного театра} в Киеве, её имя носит Украинский драматический театр
во Львове. Стала первой обладательницей звания «Народный артист Украинской ССР»
%%%cit_comment
%%%cit_title
\citTitle{2 августа — день в нашей истории — Одна Родина}, , odnarodyna.org, 08.04.2021
%%%endcit

%%%cit
%%%cit_head
%%%cit_pic
%%%cit_text
«Ми є \emph{народ}, у якого вкрали назву» – так за пару століть означив ситуацію
академік Михайло Грушевський.  Доктор історії Ярослав Дашкевич прокоментував
подію розлогіше: «Московщина по суті загарбала назву Русь, яка своїм питомим
змістом – етнічним, географічним, устроєвим – цілком відповідає сучасному
термінові Україна… Це дало Московщині, хоча й підроблений, але все ж таки,
блиск культурної, цивілізованої держави з давньою історичною традицією, з
візантійсько-київською церковною метрикою»
%%%cit_comment
%%%cit_title
\citTitle{Як Московія стала Росією? До 300-ліття «викрадення» назви українського народу}, 
Ірина Костенко; Ірина Халупа, www.radiosvoboda.org, 21.10.2021
%%%endcit
