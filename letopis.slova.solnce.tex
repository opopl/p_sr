% vim: keymap=russian-jcukenwin
%%beginhead 
 
%%file slova.solnce
%%parent slova
 
%%url 
 
%%author 
%%author_id 
%%author_url 
 
%%tags 
%%title 
 
%%endhead 
\chapter{Солнце}
\label{sec:slova.solnce}

%%%cit
%%%cit_pic
%%%cit_text
Села птаха белокрылая на тополь. Слова Нины Матвиенко. Села птаха белокрылая на
тополь, Село \emph{Солнце}, красным заревом крича. Полюбила, полюбила я до
боли, Молодого, молодого скрипача. Я влюбилась, околдованная струною,
Заблудилась та мелодия в лесу
%%%cit_comment
%%%cit_title
\citTitle{Украинские Песни Русскими Словами}, 
БРАТИНА, zen.yandex.ru, 15.12.2020
%%%endcit

%%%cit
%%%cit_head
%%%cit_pic
\ifcmt
  pic https://strana.ua/img/forall/u/11/52/1625582748-930.jpg
	width 0.4
\fi
%%%cit_text
Когда вспышки на \emph{Солнце} такого класса направлены непосредственно на Землю,
самые сильные из них могут угрожать жизням астронавтов и работе спутников в
космическом пространстве, а также создавать помехи электрических сетей на
Земле.  \emph{Солнечные вспышки} классифицируются латинскими буквами А, В, С, М или Х
в зависимости от величины пика интенсивности рентгеновского излучения вспышки
(А - самая маленькая, Х - самая большая). За буквой указывается также индекс,
который уточняет значение вспышки и может быть от 1,0 до 9,9.  Ученые
утверждают, что пятно AR2838, вызвавшее вспышку Х1,59, является новой активной
областью на \emph{Солнце}
%%%cit_comment
%%%cit_title
\citTitle{На Солнце произошла самая мощная вспышка за последние годы}, 
Анна Копытько, strana.ua, 06.07.2021
%%%endcit

%%%cit
%%%cit_head
%%%cit_pic

\ifcmt
  tab_begin cols=3
     pic https://regnum.ru/uploads/pictures/news/2021/10/01/regnum_picture_16330825827663862_normal.JPG
     pic https://regnum.ru/uploads/pictures/news/2021/10/01/regnum_picture_16330826013161964_normal.JPG
		 pic https://regnum.ru/uploads/pictures/news/2021/10/01/regnum_picture_16330826384000816_normal.JPG
  tab_end
\fi
%%%cit_text
В основном в октябре будет облачно с пояснениями, во вторую неделю возможны
осадки, но есть вероятность, что установится \emph{солнечный} период. Дневная
температура во второй половине октября опустится до 5 градусов, ночью до 2
градусов тепла
%%%cit_comment
%%%cit_title
\citTitle{В Москве ждут бабьего лета-2021 — фоторепортаж}, 
Наталья Стрельцова, regnum.ru, 01.10.2021
%%%endcit


