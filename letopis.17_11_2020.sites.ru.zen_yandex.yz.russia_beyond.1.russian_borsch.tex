% vim: keymap=russian-jcukenwin
%%beginhead 
 
%%file 17_11_2020.sites.ru.zen_yandex.yz.russia_beyond.1.russian_borsch
%%parent 17_11_2020
 
%%url https://zen.yandex.ru/media/rusbeyond/chem-russkii-borsc-otlichaetsia-ot-ukrainskogo-5fb24f2a1064d30b6c622d22
 
%%author Russia Beyond
%%author_id yz.russia_beyond
%%author_url 
 
%%tags borsch,ukraina,russia
%%title Чем русский борщ отличается от украинского?
 
%%endhead 
 
\subsubsection{Чем русский борщ отличается от украинского?}
\label{sec:17_11_2020.sites.ru.zen_yandex.yz.russia_beyond.1.russian_borsch}
\Purl{https://zen.yandex.ru/media/rusbeyond/chem-russkii-borsc-otlichaetsia-ot-ukrainskogo-5fb24f2a1064d30b6c622d22}
\ifcmt
	author_begin
   author_id yz.russia_beyond
	author_end
\fi

\index[rus]{Блюда!Борщ!Чем русский борщ отличается от украинского, 17.11.2020}

Это блюдо есть у многих славянских народов, и каждый считает свой рецепт
каноническим.

АННА СОРОКИНА

\ifcmt
pic https://avatars.mds.yandex.net/get-zen_doc/3994559/pub_5fb24f2a1064d30b6c622d22_5fb24f87f2466e1810a8e5ba/scale_1200
\fi

В русской и украинской кухнях есть очень много общих блюд, что в общем-то,
неудивительно, учитывая и близкое соседство и родство, и то время, когда Россия
и Украина были частями одной страны (а это, на минуточку, три с половиной
ВЕКА). В советские годы вообще дружба народов была одним из столпов
государственной идеологии и касалась, в том числе, и еды: блюда региональных
кухонь образовали одну советскую, в которую вошел и узбекский плов, и
кавказский шашлык, и сибирские пельмени, и старинный славянское первое блюдо
борщ.

На самом деле кушанье под названием «борщ» варили в семьях восточных славян с
незапамятных времен (он упоминается в русских летописях с 16 века): блюда с
похожим названием есть не только у русских и украинцев, но и белорусов,
поляков, румын и молдаван. И у каждого он свой. Даже в «Книге о вкусной и
здоровой пище», главном кулинарном сборнике СССР (впервые издана в 1935 году),
рецепты борщей занимают несколько страниц, и они все очень разные.

\subsubsection{Древний борщ был не таким}

\ifcmt
  pic https://avatars.mds.yandex.net/get-zen_doc/1710127/pub_5fb24f2a1064d30b6c622d22_5fb24fad8d19932be111572d/scale_1200
  caption (с) Третьяковская галерея. Ф.Г. Солнцев. Крестьянское семейство перед обедом, 1824 год
\fi

Сегодня под борщом многие понимают суп красного цвета из свеклы и капусты, но
изначально так называли похлебку из съедобной разновидностью борщевика,
растения, похожего на гигантский укроп (его современная разновидность, борщевик
Сосновского, летом становится одним из самых ядовитых сорняков, а вот в
древности на этих территориях рос вполне себе съедобный борщевик). Листья
квасили, а потом добавляли в суп.

Позднее появился вариант борща на свекольном квасе, который разбавляли водой,
добавляли туда капусту и морковь и томили в печи. Уже потом стали варить и на
мясном бульоне, причем крестьяне могли себе позволить это только по праздникам,
а в будни добавляли в суп сало с чесноком для сытости. «Один из первых
достоверных рецептов свекольного борща встречается только у Василия Левшина в
«Словаре поваренном» 1795 года, -
пишет\Furl{https://eda.ru/media/vopros/chem-borshch-otlichaetsya-ot-shchey}
кулинарный историк Павел Сюткин. - Там есть перечисление ингредиентов для борща
— «говядина кусками, много ветчины, курицу варить с водой…».  

\subsubsection{Борщ у каждого свой}

«Борщ — лишь термин, в который каждый из народов вкладывал свое понимание», -
считает\Furl{https://zen.yandex.ru/media/eda.ru/chem-otlichaetsia-russkii-borsc-ot-ukrainskogo-5deb955ff7e01b00ad7414af?integration=vgtrk_widget&place=teasers} кулинар Ольга Сюткина, добавляя, что в рецепты этого супа шли простые и
доступные ингредиенты, которые были в каждом регионе свои. На юге России и
Украине были доступны свежие и разнообразные овощи, а вот северянам приходилось
заготавливать на зиму квашеные, считает она.

Оттого у славянских народов есть огромное количество приготовления этого супа,
который варьируется не только по региональному признаку, но и по погодному -
летний свекольный вегетарианский суп называют «холодником» или «свекольником»,
и это тоже одна из разновидностей борща.

\ifcmt
pic https://avatars.mds.yandex.net/get-zen_doc/1860621/pub_5fb24f2a1064d30b6c622d22_5fb24fd38d19932be1119aa5/scale_1200
caption (с) Legion Media. Польский белый борщ
\fi

В Польше едят так называемый «белый борщ» на основе закваски из ржаной муки,
без свеклы, с добавлением вареных яиц и колбасы. Есть также варианты с грибами
или рыбой. В Румынии борщом называют кислый овощной суп с квасом. На юге России
готовят совершенно удивительный «таганрогский» (он же «монастырский») борщ без
свеклы, из томатов (которые вообще для русской кухни раньше были экзотикой, но
в Таганроге, портовом городе, были совершенно обычным делом), на бульоне из
бычьих хвостов или курицы (и никогда на свином). Готовят борщи также с фасолью
или картошкой.

\ifcmt
pic https://avatars.mds.yandex.net/get-zen_doc/1877958/pub_5fb24f2a1064d30b6c622d22_5fb2502c70f5da1bda49e0e5/scale_1200
caption (с) Legion Media. Борщ с пампушкой

pic https://avatars.mds.yandex.net/get-zen_doc/1704908/pub_5fb24f2a1064d30b6c622d22_5fb25053f2466e1810aa68bf/scale_1200
caption (с) Legion Media. Борщ с черным хлебом
\fi

В современной России, Украине и Беларуси наиболее популярен горячий красный
борщ, который подается со сметаной и хлебом, однако канонических рецептов нет
не только в одной стране, но даже в одном регионе.

\subsubsection{Так в чем же разница между русским и украинским борщом?}

В разговорной речи слово «борщ» у русских (и вообще жителей бывшего СССР) даже
означает «микс из всего, что попадется под руку». И это вполне справедливо, так
как у этого супа и правда нет единого рецепта.

Сам процесс приготовления вызывает ожесточенные кулинарные дебаты: под любым
рецептом в русскоязычном интернете будут комментарии вроде «не умеете готовить
- не беритесь», «у вас халтура, а не борщ», «у меня свой рецепт, и все просят
добавки, а у вас просто щи со свеклой».

Во-первых, бульон. Считается, что русский борщ готовят на говяжьем костном
бульоне, а вот украинский на свиных ребрышках, но это очень условно. Можно
варить борщ и на курином бульоне или на воде. Свекольник, например, делают
вообще на кефире.

\ifcmt
pic https://avatars.mds.yandex.net/get-zen_doc/1926164/pub_5fb24f2a1064d30b6c622d22_5fb250821064d30b6c64c380/scale_1200
cpx (c) Legion Media. Борщ с фасолью

pic https://avatars.mds.yandex.net/get-zen_doc/3952637/pub_5fb24f2a1064d30b6c622d22_5fb250a28d19932be1131d8a/scale_1200
cpx (c) Legion Media
\fi

Во-вторых, капуста. Некоторые кулинары пишут, что в русском варианте
преобладает квашеная капуста, которая кладется в начале, чтобы стать мягкой, а
в украинском - свежая, которая кладется в конце варки и остается хрустящей, а
другие - что капуста вообще не обязательна, зато без фасоли борщ не настоящий.
Белорусский борщ скорее всего будет с картошкой.

В-третьих, сорт свеклы. Говорят, что в борщ нельзя добавлять ту же свеклу, что
и в винегрет, якобы она не даст нужный цвет.

И главное - дополнительные ингредиенты! Современные кулинары всех стран нередко
кладут в зажарку с морковью, луком и свеклой томатную пасту, которая делает
борщ более сочным и ярким. Но в то же время адепты традиционных борщей говорят,
что раньше томатов не было, и этого делать ни в коем случае нельзя. Кто-то
добавляет уксус, лимон, сахар (да, прямо в суп!), варит бульон с луком в
шелухе, а кто-то предпочитает кроме мяса еще и шкварки сала.

Интересно, что проще всего отличить происхождение борща по способу подачи: чаще
всего русский борщ подают с черным (ржаным) хлебом, а украинский - с чесночными
булочками-пампушками. Белорусский борщ иногда украшают также зеленью и
половинкой яйца. Но это не точно.
