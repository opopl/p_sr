% vim: keymap=russian-jcukenwin
%%beginhead 
 
%%file 04_06_2020.fb.lesev_igor.1.unitarizm_ukraina.cmt
%%parent 04_06_2020.fb.lesev_igor.1.unitarizm_ukraina
 
%%url 
 
%%author_id 
%%date 
 
%%tags 
%%title 
 
%%endhead 
\subsubsection{Коментарі}
\label{sec:04_06_2020.fb.lesev_igor.1.unitarizm_ukraina.cmt}

\begin{itemize} % {
\iusr{Василий Стоякин}
И это всех устраивает

\begin{itemize} % {
\iusr{Игорь Лесев}
всех, кто попал в кормовую цепочку

\iusr{Василий Стоякин}
\textbf{Игорь Лесев} Всех кто не попал тоже

\iusr{Игорь Лесев}
\textbf{Василий Стоякин} ну теперь хоть буду знать, что и меня тоже устраивает))

\iusr{Дмитрий Коломийченко}
\textbf{Василий Стоякин} А как они могут проявить своё недовольство?

\iusr{Игорь Потысьев}
\textbf{Василий}, нет, их не устраивает.

\iusr{Игорь Потысьев}
\textbf{Дмитрий}, чемодан, самолет, другая страна
\end{itemize} % }

\iusr{Александр Каревин}
Мажоритарка - не выход. Округа тупо покупаются.

\begin{itemize} % {
\iusr{Игорь Лесев}

я вовсе не топлю за мажоритарку. Просто констатирую - это сегодня единственный,
пусть и далеко не совершенный механизм представительства регионов в центре.
Другого ничего больше нет, а скоро и этого не станет

\iusr{Александр Каревин}
\textbf{Игорь Лесев} 

Не будет никакого представительства регионов. Округа покупаются не для
регионов, а для обеспечения собственных интересов покупателя или тех, кто за
ним стоит.

\iusr{Игорь Лесев}
\textbf{Александр Каревин} 

ну это все тоже понятно. Только даже при таких скупках есть неформальные
правила поведения, те же депутаты выступают "от имени округа", чего-то там
решают на местах и типа топят за местных. К тому же, большинство мажоритарщиков
хотя бы живут на две хаты - киевскую и местную.

\iusr{Александр Каревин}
\textbf{Игорь Лесев} 

Та пофиг, где они живут. У таких, как правило, всегда есть условная "хатынка в
Швейцарии". Где банк, в котором их деньги - это более значимо. А выступать от
имени округа и реально защищать интересы округа - это две большие разницы.

\iusr{Владимир Егоров}
\textbf{Александр Каревин}, а двухпалатный сенат?
\end{itemize} % }

\iusr{Сергей Сергеев}
Хорошее наблюдение.

\iusr{Елена Понизовская}
Блин, стройно изложено(

\iusr{Марина Прохорова}

Ну, прописаны они в Киеве. А поскрести - каждый второй - "кыянка Зи Стрыя".
Поэтому Киев стабильно голосует за ССволоту и переименовывает улицы в честь
мутных галычанских митрополитов ((

\begin{itemize} % {
\iusr{Владимир Мерцин}
Можно сюда добавить аналитику и про учебные заведения и места стажировки свежих депутатов. Тогда точно все встанет на свои места.
\end{itemize} % }

\iusr{Влад Винница}
Тот случай, когда автор умеет анализировать ситуацию и синтезировать выводы! Спасибо! Лайк и репост)))

\iusr{Татьяна Черногорец}
Фото из моего городка...  @igg{fbicon.thinking.face} 

\iusr{Игорь Лесев}
и точно такие же строения в моем городке

\iusr{Олег Хавич}
Я так люблю Украину, что хотел бы, чтобы их было четыре-пять  @igg{fbicon.wink} 

\iusr{Дмитрий Коломийченко}
Дотянулся проклятый Ульянов. Сделал УССР унитарной национальной республикой и получили. Даже в Евросюз не вступишь. Вот ведь гад.

\iusr{Yartsev Anatoliy}
Поэтому Крым и убежал от Киева в 2014 как Финляндия от Москвы сто лет назад  @igg{fbicon.face.smirking} 

\iusr{Юрий Лукашин}

Что тут еще добавить. Оттого и в умных странах давно изобрели понятие -
федерализм (Германия, США, Канада, РФ, Швейцария, Бельгия, Австрия, ОАЭ и
т.д.). Он не везде официально признан, но как принцип во многом реализован по
факту и даже порой юридически (хотя и не называясь федерализмом) в еще большем
числе стран. Порой причудливо пересекаясь и с конституционной монархией, и с
советскими принципами управления государством и обществом - Китай,
Великобритания, Норвегия, Дания, Италия, Испания и др. Строго унитарные страны
- это чаще всего Африка, значительная часть стран Ближнего Востока (типа
Саудовской Аравии, Ирана, Турции) и Азии. То, что элитка территориально во всех
странах стремится оседать поближе к столице - тут особо ничего удивительного.
Просто именно федерализм позволяет этот нездоровый процесс существенно
сглаживать, а кое-где и напрочь перебороть. Если бы Украина с 1991 г.
развивалась по принципу федерализма, ее история сложилась бы совершенно иначе.
Индустриальные (т.е. "ватные") регионы уже лет 20 назад окончательно бы
оторвались от аграрных (а может даже и от Киева) в уровне благосостояния, в
привлекательности для жизни, в плане инвестиций и т.д. Как мы с вами понимаем,
политически это была бы уже совершенно другая страна. Но так как такой
политический сценарий полностью противоречил планам режиссера, федеративные
Штаты (которые у нас тут много "демократии" сеяли все эти годы) ни единого разу
за все это время не высказывались в пользу внедрения в Украине хотя бы отчасти
того же принципа устройства, что они имеют и всячески лелеют у самих себя. Это
был бы страна совершенно с другим политическим балансом, с другой
ментальностью, с другой экономической, а значит и другой внешней политикой.
Сегодня же это совершенно марионеточный, без внятных перспектив на будущее,
депрессивный Рагулестан.

\begin{itemize} % {
\iusr{Сергей Киселев}
\textbf{Юрий Лукашин} да в тех же Италии и Франции по факту федерализм с элементами союзного государства и конфедерации
\end{itemize} % }

\iusr{Алекс Логвинов}
Добавить нечего. Только немного Харьков выделяется , но под Кернеса копают постоянно

\iusr{Анатолий Козак}

Давайте скажем честно, что на сегодня мы имеем парламент "карманных депутатов".
Они принадлежат "большому баблу" и их задача обеспечить боссам грабеж страны.
Сейчас время, условно, группы Коломойского. Дограбить страну их первая и
последняя задача. Когда малец не смущаясь и не сомневаясь кидает на "распил"
родную (для нас) и безразличную (для него) землю, о чем можно говорить?

Государство начинается с идеологии. Стержневой идеи. Проще говоря: государство
для народа или государство для кучки "рамсы попутавших" жлобов. Если мы хотим
государство для народа, тогда депутат должен жить постоянно у себя на округе,
лечиться в поликлинике по месту жительства, а дети должны ходить в обычную
школу. Лишь тогда депутатская тварь будт подумывать об интересах своих
избирателей.

\iusr{Александр Реушев}

Россия должна быть федерацией! Все намного хуже, чем на Украине. Только вот
когда беда пришла и дед закрылся в бункере, то неожиданно вспомнили, что у
регионов есть какая то субъектность

\iusr{Светлана Воробьева}

Так ведь нет никакого запроса общества ни на что. Вот вы пишите, что в 100\%
русскоязычном Запорожье депутаты не могут открыть русские школы. А они
кому-нибудь нужны? Фактически школы украинизировали лет 20 назад, если не
больше. И вот был при позднем Януковиче закон Колесниченко-Кивалова, когда
родители могли собраться и постановить, чтобы их детей учили на русском языке.
Кто-нибудь им воспользовался? Всем пофиг, чему и на каком языке учат их детей.
Вот "западенцам" не пофиг, они и продвигают свою программу. Вот кто-то тут
вверху написал, что и финансовые потоки перенаправили в свои регионы.

\end{itemize} % }
