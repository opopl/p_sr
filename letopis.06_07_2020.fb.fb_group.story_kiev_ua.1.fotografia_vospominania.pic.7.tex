% vim: keymap=russian-jcukenwin
%%beginhead 
 
%%file 06_07_2020.fb.fb_group.story_kiev_ua.1.fotografia_vospominania.pic.7
%%parent 06_07_2020.fb.fb_group.story_kiev_ua.1.fotografia_vospominania
 
%%url 
 
%%author_id 
%%date 
 
%%tags 
%%title 
 
%%endhead 

\ifcmt
  ig https://scontent-frx5-1.xx.fbcdn.net/v/t1.6435-9/106925946_3380588998641350_299727335656753079_n.jpg?_nc_cat=100&ccb=1-5&_nc_sid=b9115d&_nc_ohc=ysUrhOrf91cAX_3hvC9&_nc_ht=scontent-frx5-1.xx&oh=fa85139266ee92072c3afe4e5630d77b&oe=61B58D46
  @width 0.4
\fi

\iusr{Ирина Петрова}
Іван Якович

\iusr{Татьяна Оксаненко}
\textbf{Ирина Петрова} 

Памятник ивану Франко, он находится между зданием Киевэнерго АК (как раз видно
здание), а по правую руку - театр им. Франко. В компании я работала, так,, что
место знакомо

\iusr{Ирина Петрова}
\textbf{Татьяна Оксаненко} 

да, абсолютно точно)))) тогда там, позади дома, был двор, большой, неухоженный,
грязноватый. Но! Мы знали одну дверь, проскочив в которую, поднявшись на второй
этаж, можно было попасть в туалет! Он был жутковатый, со страшными ржавыми
унитазами, но, позволял нам не отрываться надолго от игр. Сейчас это кажется
нереальным - ребятня могла запросто зайти в Киевэнерго)))) как-то было попроще,
что-ли... и ещё там валялись отличные куски цветных проводов, из которых
умельцы плели кольца!

\iusr{Татьяна Оксаненко}
\textbf{Ирина Петрова} Все правда!!!

\iusr{Татьяна Оксаненко}
\textbf{Ирина Петрова} 

Мы сейчас, уже в таком возрасте, когда трогает любое воспоминание
детства, юности, прямо до мурашек. А Октябрьский дворец? Сколько было кружков, как
были заняты дети по интересам, какие Елки?! А подарки в Новый Год! НУ, ЗДОРОВО ЖЕ
БЫЛО!

\iusr{Ирина Петрова}
\textbf{Татьяна Оксаненко} 

да! Я туда в библиотеку ходила, там был клуб юных любителей книг. Там таааак
пахло, этот запах библиотек до сих пор один из любимейших. Сейчас так пахнет в
Парламентской. Утренники в Октябрьском! Когда-то даже довелось выступать на
сцене в каком-то школьном спектакле, помню мне папа туфельки красил белой
лаковой краской, а мама пришила розовые банты... и мы ещё потом с этим же
спектаклем выступали в кинотеатре Украина перед сеансом, переодевались за
занавесом, перед экраном и кто-то упал, запутавшись в занавесе, выполз из-под
него на сцену, грохнул смех, успех нам был уже гарантирован. И вот странно - я
очень много помню о детстве, а что был за спектакль - "чистый лист.".. ой,
когда тянет на воспоминания, говорят, это уже возраст даёт первые звоночки @igg{fbicon.face.grinning.squinting} 

\iusr{Alik Perlov}
\textbf{Ирина Петрова} Ты бы вспомнила ещё, кто с тобой выступал в этом спектакле в \enquote{Украине}

\iusr{Татьяна Оксаненко}

Хорошо, что хоть помним и можем воспроизвести

\ifcmt
  ig https://scontent-frx5-2.xx.fbcdn.net/v/t39.1997-6/p600x600/106398290_953858951743129_8907968730258864485_n.png?_nc_cat=1&ccb=1-5&_nc_sid=0572db&_nc_ohc=yeeQMTz8yLUAX8C-Q3-&_nc_ht=scontent-frx5-2.xx&oh=50983b2166b68425cf72bececb97a405&oe=619621DC
  @width 0.15
\fi

\iusr{Ирина Петрова}
\textbf{Alik Perlov} ты! По-моему, ты был чародеем?

\iusr{Alik Perlov}
\textbf{Ирина Петрова}  @igg{fbicon.face.blowing.kiss}{repeat=2} Я. Только не помню, что за пьесу мы давали!!

\iusr{Ирина Петрова}
\textbf{Alik Perlov} вот! В темно-синем атласном балахоне и остроконечной шапке!!! Сама не могу вспомнить пьесу... Это ещё с Тамарой Федоровной)

\iusr{Alik Perlov}
\textbf{Ирина Петрова} вроде кто-то фоткал, нет???

\iusr{Ирина Петрова}
\textbf{Alik Perlov} 

не помню... у меня, увы, таких фотографий нет. Ни у кого не удалось в свое
время стащить  @igg{fbicon.face.eyes.star} 
