% vim: keymap=russian-jcukenwin
%%beginhead 
 
%%file 09_11_2021.fb.fb_group.story_kiev_ua.1.rasskaz_sssr_kiev
%%parent 09_11_2021
 
%%url https://www.facebook.com/groups/story.kiev.ua/posts/1793704854159641/
 
%%author_id fb_group.story_kiev_ua,kuzmenko_petr
%%date 
 
%%tags gorod,kiev,sssr
%%title Закон о декоммунизации не нарушен
 
%%endhead 
 
\subsection{Закон о декоммунизации не нарушен}
\label{sec:09_11_2021.fb.fb_group.story_kiev_ua.1.rasskaz_sssr_kiev}
 
\Purl{https://www.facebook.com/groups/story.kiev.ua/posts/1793704854159641/}
\ifcmt
 author_begin
   author_id fb_group.story_kiev_ua,kuzmenko_petr
 author_end
\fi

Закон о декоммунизации не нарушен.

Люди, не любящие больших ностальгических опусов, могут смело прокрутить этот
мой пост в ленте новостей не читая. Отмечу лишь, что сам удивился своей
графомании и объему выданного за вечер ночь и утро материала. Подтолкнула меня
к написанию этого рассказа, сама того не ведая, душа и модератор нашей любимой
группы, одноклассница и друг Светлана Манилова. Но, Светик, к огромному
сожалению, очень рано покинула нашу школу и не была участницей описываемых
событий. Однако, в группе всё - же найдутся свидетели правдивости моего
повествования. 

Итак, дорогой читатель, запасёмся терпением и окунёмся в
начало 80-х прошлого века. Подол. 100 школа и не только...  Я уже писал в
нашей лучшей киевской группе несколько своих подольских воспоминаний о нашей
школе. Теперь это респектабельный лицей "Подол". Но моё повествование
несколько дополнит (в определённом ключе🤣) мои предыдущие посты. Я расскажу
о после олимпиадных временах, а именно начале 80-х годов прошлого века,
выпавших на наши 9-й и 10-й классы. 

\ifcmt
  tab_begin cols=3

     pic https://scontent-mia3-1.xx.fbcdn.net/v/t1.6435-9/255288435_4699798966739017_7806854757161218460_n.jpg?_nc_cat=104&ccb=1-5&_nc_sid=b9115d&_nc_ohc=rqk6-XvmWPIAX-e2-kk&_nc_ht=scontent-mia3-1.xx&oh=83c382338a214be910e8667c9dbb2afa&oe=61AE9E99

     pic https://scontent-mia3-1.xx.fbcdn.net/v/t1.6435-9/255614280_4699799160072331_1931588226702417165_n.jpg?_nc_cat=104&ccb=1-5&_nc_sid=b9115d&_nc_ohc=5YLbhyiM214AX-YoXhn&_nc_ht=scontent-mia3-1.xx&oh=0e79a18aab8fcb75b487bf28b935cdd7&oe=61B154D3

		 pic https://scontent-mia3-1.xx.fbcdn.net/v/t1.6435-9/254955829_4699799323405648_432950129972712640_n.jpg?_nc_cat=111&ccb=1-5&_nc_sid=b9115d&_nc_ohc=0pt0qjFfdwMAX9UPMlS&tn=lCYVFeHcTIAFcAzi&_nc_ht=scontent-mia3-1.xx&oh=7e68a53cc6392b0f5ec36be1f80bc3f9&oe=61B13DC1

  tab_end
\fi

Во все времена школьники старших классов
задумывались и задумываются над своим будущим, и ваш покорный слуга не
исключение. До своей безмерной и затмившей всё влюблённости в одноклассницу и
будущую единственную подругу жизни Неличку, я более всего склонялся к
поступлению в Духовную семинарию. Не скажу, что я вёл смиренный и
богобоязненный образ жизни и систематически посещал Храм. Просто я вырос на
Подоле, мама была верующей и часто брала меня с собой на богослужения в
уютный и близкий многим подолянам Флоровский монастырь. Иногда мы посещали
величественный Владимирский собор. 

Во Флоровском был добрый и эрудированный
священник, знавший меня практически с рождения. С батюшкой было интересно
общаться не только на религиозные темы. Он здорово рассказывал об истории
православия, закон Божий, в его устах, из строгих церковных канонов
превращался в живую и захватывающую науку. Но, простите друзья я, как обычно,
отвлёкся от нити своего рассказа. 

\ifcmt
  tab_begin cols=3

     pic https://scontent-mia3-2.xx.fbcdn.net/v/t1.6435-9/255927937_4699799423405638_3887566504449742229_n.jpg?_nc_cat=107&ccb=1-5&_nc_sid=b9115d&_nc_ohc=p__ZAQlFsAsAX-HAvYJ&_nc_ht=scontent-mia3-2.xx&oh=5f1bac61c960643611831504d8b001a1&oe=61AFF860

     pic https://scontent-mia3-2.xx.fbcdn.net/v/t1.6435-9/255046643_4699799533405627_2275527958091780081_n.jpg?_nc_cat=102&ccb=1-5&_nc_sid=b9115d&_nc_ohc=IAF_bDmLricAX-qLn4Q&tn=lCYVFeHcTIAFcAzi&_nc_ht=scontent-mia3-2.xx&oh=b8a315250eee79360bca0df331ba30c1&oe=61B0B823

		 pic https://scontent-mia3-1.xx.fbcdn.net/v/t1.6435-9/255710706_4699799770072270_1859367450072262174_n.jpg?_nc_cat=104&ccb=1-5&_nc_sid=b9115d&_nc_ohc=HPd9_GDt43AAX_ow26c&_nc_oc=AQnNT-sIQlpq9u4sGyPhkLkkg4SbePuHeg73nODIxlfJxMAGACksCek0tn8vajhy2WQ&_nc_ht=scontent-mia3-1.xx&oh=16ef77934b3340fb9806b35dbf2322a7&oe=61AF3D7B

  tab_end
\fi

Итак, далее. Ответившая на мою страсть,
после бурных ухаживаний, избранница, из партийно - функционерской семьи,
категорически не хотела быть впоследствии попадьёй. Через много - много лет
она несколько пожалела о своём тогдашнем решении, когда мы познакомились с
семьёй священника подольской Ильинской церкви, которая жила этажом ниже в
нашем доме на Почайнинской. Это произошло, когда она познакомилась с
Татьяной, женой отца Виталия и матерью семейства. Её трудно было назвать
"матушкой", в одежде от известных кутюрье и с макияжем от лучших визажистов.

Снова ушёл в сторону от рассказа. Извините. Итак, благодаря решению Нелли мне
пришлось пересмотреть свои планы. Мой выбор был изменён на первый взгляд
кардинально. С Духовной семинарии на военно-морское политическое училище. Но,
это только на первый взгляд. У этих, казалось бы разных профессий одна
специализация. Работа с людьми, имеющая в основе определённую идеологию.
Христианскую или коммунистическую. Это сейчас, умудрённый жизненным опытом
(хотелось написать седовласый, но это не так) пожилой джентльмен, я понимаю
огромную разницу между православием и марксизмом - ленинизмом, а тогда
девятиклассник, считающий почти тождественными Нагорную проповедь Христа и
Моральный кодекс строителя коммунизма, читающий то Библию, то теоретические
работы Ленина, что я мог иметь в юношеской голове? 

Но, решение было принято и нужно было воплощать его в жизнь. Невзирая на все
идеологические и романтические завихрения в мозгах, я с детства был человеком
довольно практичным. Посмотрев на статистику абитуриентов - школьников (была
огромная разница между поступающими со службы, коим отдавалось предпочтение, и
ребятами  после школы) в КВВМПУ предыдущих лет я понял, что барьер был
несколько выше, чем в Одесской Духовной семинарии (заведение данной
специализации появилось в Киеве лишь в 1989-1990гг.) Это объяснялось тем, что
Духовных семинарий было всё-таки несколько в СССР, а военно-морское училище,
выпускающее специалистов с гуманитарной и технической специальностью, одно.
Уяснив для себя цель, я стал обдумывать средства её достижения. Первым
подспорьем для меня стало то, что мы с друзьями - одноклассниками уже пару лет
занимались в военно - патриотическом объединении "Океан". И нашими наставниками
были курсанты и преподаватели вожделенного Морполита. 

Я понимал, что одно преимущество перед обычными вчерашними десятиклассниками
мне обеспечено. Я знаю морскую терминологию, фрагментарно историю Флота, умею
вязать морские узлы, изучил "морзянку" и флажный семафор. Также на экзаменах и
перед Приёмной комиссией я предстану в морской форме, с якорьками и надписью
ВПО "Океан" на погонах. 

Вторым средством достижения поставленной цели стало комсомольское поручение
быть полит информатором класса и дважды в неделю, после уроков, знакомить
одноклассников с главными событиями политической и экономической жизни страны и
зарубежья. К этому поручению я отнёсся в большой ответственностью. 

И тут мы подходим к изюминке моего рассказа. Думаю, вступление было хоть
довольно продолжительным, но не нудным.

Итак, как ответственный полит информатор, я обзавёлся письмом директора школы
(а им тогда была дочь дважды Героя Советского Союза героя - партизана
отечественной войны А.Ф.Фёдорова Нина Алексеевна Фёдорова) с подписями и
печатями секретарей партийной и комсомольской организаций школы в магазин
политической и пропагандистской литературы находившийся напротив Бесарабского
рынка. Там я пополнял запас нужной литературы. И на СЛЕДУЮЩИЙ день после
окончания 26-го съезда КПСС, проходившего в Москве с 23 февраля по 3 марта
1981 года я уже имел в своём распоряжении изданные материалы съезда, причём
обложка издания была синего цвета. Как я потом узнал это было первое
предсъездовское издание для партийной номенклатуры. (На фотографии
представлена книга в красном переплёте для обычных читателей). 

Именно 26
съезд КПСС и досконально изученные мной отчётный доклад Генерального
секретаря ЦК КПСС Л.И.Брежнева, а так же принятые на этом партийной форуме
"Основные направления экономического и социального развития СССР на 1981—1985
годы и на период до 1990 года." сыграли, хоть во многом саркастическую, но
решающую роль в осуществлении моей мечты. Начнём с того, что в праздничный
День СА и ВМФ 23 февраля 1981 года, когда девочки по устоявшейся советской
традиции уже поздравили мальчиков с утра и одарили подарками, все ученики
нашего класса в весёлом и приподнятом настроении расселись в кабинете истории
смотреть торжественное открытие партийного съезда по цветному телевизору.

Занятия по этому значимому для страны поводу были отменены. Мероприятие
возглавляла Циля Семёновна Карпиловская преподаватель истории и секретарь
парторганизации школы. Мы все шептались, передавали записочки, хихикали и
обращали мало внимания на телевизор. ЦСК, как иногда мы называли историчку
между собой, тоже была занята своими учительскими бумагами и обращала на нас
мало внимания. И тут зазвучали фанфары и с соответствующей телекартинкой и
прогремели первые аккорды партийного гимна Интернационал. 

Я встал, внимательно глядя на экран. Циля Семёновна сразу заметила это и
спросила у меня причину моего телодвижения. Я ответил, что не могу сидеть,
когда звучит партийный гимн и на трибунах стоят члены ЦК партии, а в зале
кремлёвского дворца делегаты съезда. Наш преподаватель настоятельно попросила
встать весь класс. 

Я до сих пор помню "ласковый" взгляд любимой, сидевшей за одной партой со мной,
и "доброжелательные" лица одноклассников обращённые ко мне. Позже, на уроках
новейшей истории и обществоведения, я всегда использовал в своих ответах
материалы прошедшего съезда. Однажды Циля Семёновна в сердцах сказала мне, что
я ни за что не поступлю в КВВМПУ потому, что завал выпускные экзамены. Я
промолчал на этот выпад, но запомнил его. 

Пришла экзаменационная пора в 10-м классе. На все экзамены, начиная с самого
первого в пятом классе, я всегда ходил первым. Лучше отстреляться сразу и быть
свободным. Тем более первому отвечающему многие преподаватели делали скидку за
смелость. Эта привычка осталась у меня на всю жизнь. Извините за отступление. 

Первой мы сдавали биологию. Знал я её относительно. Но тут, о ЧУДО, мне попался
билет в первом вопросе которого было что-то о будущем развитии биологии в СССР
в ближайшую пятилетку. Алла Анатольевна, наш учитель биологии которую мы все
любили, несколько раз пыталась прервать мой обоснованный и аргументированный
ответ на первый вопрос, но я аргументируя тем, что затронут важный аспект
развития не только биологии, но и всех естественных наук, продолжал цитировать
выдержки из программы развития страны, принятой 26 съездом партии.

Подойдя вплотную ко мне, Алла Анатольевна сказала, что за такой блестящий
ответ на первый вопрос она сразу ставит мне "отлично" потому, что уверена и в
превосходном знании мной второго вопроса. Математику вытянуть на съезде не
удалось, но поскольку я целый год дополнительно занимался ей с нашим учителем
Львом Ихиловичем, зная, что этот предмет будет одним из основных и на
вступительных экзаменах, мне пришлось по рекомендациям и телефонным звонкам
Льва Ихиловича, ездить на такси вечером им в первую половину ночи перед
экзаменом по его коллегам - преподавателям особо информированным об утренних
экзаменационных заданиях. 

В итоге, к утру, у меня были шпоры обоих вариантов.
Для меня и Нелли. В общем математика "отлично". Сочинение я выбрал на
общественную окололитературную тему, где можно было "налить воды" и к месту
использовать выдержки из отчётного доклада и других материалов партийного
съезда, что естественно обеспечило мне высший бал по русскому языку и
литературе. На истории, которую я люблю и смею надеяться до сих пор неплохо
знаю, в послесъездовский год вопросы конечно же были в порядке: сначала
обществоведение, затем история. Практически все вопросы были в той или иной
мере связаны с "историческими" решениями 26 съезда КПСС. Я, как всегда
первым, отстрелялся на "отлично" и потом ждал под дверью и быстренько писал
шпору по выпавшим Нелли вопросам и передавал её с бегавшим "мочить тряпку для
доски" дежурным. Замечу, что первым вопросом у неё был "решения 26 съезда
КПСС о развитии советского общества", или что-то такое. 

Партийный съезд, да и
саму партию, любимая терпеть не могла всегда, но достойную четвёрку получила.
При том, что Циля Семёновна относилась к Нелли точно так же как та к партии,
и предрекала ей судьбу Наташи Ростовой. Правда, мы с Несей, до сих пор не
понимаем, что плохого она находила в судьбе толстовской героини. На химии,
Небо в лице её преподавателя Наума Ильича, снова послало мне подарок. Мне
выпал первый билет, где первым вопросом стояло: "Решения 26 съезда КПСС о
развитии химии и химической промышленности в СССР." 

Выпал он тут не просто так. Накануне по просьбе того же Наума Ильича мы с
другом - одноклассником Игорем Вересом потратили весь световой день, даже
добавив солидный отрезок тёмного позднего вечера, выполняя поручение нашего
химика где-то в частном секторе за Сталинкой на горе у "черта на куличках".
Может, там сейчас и отличный район Города, застроенный ульями - многоэтажками,
не знаю. Но тогда это было довольно унылое, если не сказать дикое, место. 

Помню мы очень долго взбирались на какую-то возвышенность пешком долго искали
адресата, передавали и забирали какие-то бумаги и свертки, долго их ждали.
Отмечу, что ни до и никогда после того дня, мне не пришлось бывать в том
микрорайоне Киева.

Нужные и отлично вызубренные билеты, на экзамене ждали нас в определённом ряду
на нужных местах на столе. Я отвечал первым, Игорь вторым. Итог - химия
"отлично". 

Физик Павел Витович, бывший ассистентом на экзамене по химии, был очень удивлён
и сказал, что у него на экзамене "шары" не будет. А физика была для нашего
физико-математического класса ведущим предметом. Однако он не учёл, что
ассистентом на экзамене по физике был Наум Ильич. Павел Витович даже перемешал
все билеты и снова разложил их передо мной. 

Однако, обладая зрением с очень сильным минусом (он носил очки с толстенными
линзами) Павел Витович не смог заметить малюсенького накола, говоря языком
картёжников крапа на нужном мне билете. Отвечая, я вновь разглагольствовал о
задачах, поставленных партийным съездом перед физикой и физической наукой. 

Второй вопрос и физическая задача были вызубрены мной накануне на подольском
пляже Труханова острова, где мы с Нелли, ежедневно учили билеты перед
экзаменами, совмещая это занятие с солнечными ваннами и купанием в Днепре.
Часто к нам присоединялись одноклассники и одноклассницы. Таким образом, моя
цель в виде аттестата 4,8 и нужных для поступления характеристик от
администрации, комсомольской и партийной организаций школы была наконец
достигнута, в чём огромная заслуга 26 съезда КПСС и его решений. 

Писать в своём поступлении в училище, в летнем Лютежском лагере, здесь не буду,
поскольку и так здорово утомил читателя своими школьно - выпускными
ностальгическими воспоминаниями.

Я сердечно благодарю тех, кому хватило терпения дочитать этот опус до конца.  Я
сам не знал, что он получится таким продолжительным и нудным. Огромное спасибо,
дорогой читатель!

\ii{09_11_2021.fb.fb_group.story_kiev_ua.1.rasskaz_sssr_kiev.cmt}
