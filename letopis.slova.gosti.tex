% vim: keymap=russian-jcukenwin
%%beginhead 
 
%%file slova.gosti
%%parent slova
 
%%url 
 
%%author 
%%author_id 
%%author_url 
 
%%tags 
%%title 
 
%%endhead 
\chapter{Гости}

%%%cit
%%%cit_head
%%%cit_pic
%%%cit_text
Коли ми з Наталією Березою розписалися і стали офіційно чоловіком і дружиною,
то теж перший рік чудили й експериментували. Наприклад, пам'ятаю, як Наташа
психонула і законсервувала 337 банок. Ой, чого ми тоді тільки не закрутили в
банки. Огірки і лечо, кабачки і половинки помідорів. Коли прийшла зима, то
почали пробувати. Вся консервація поділилася на три види - смачно, дуже смачно
і неймовірно. Але вже через деякий час ми зрозуміли, що з'їсти все це ми не в
змозі, навіть з урахуванням того, що до нас регулярно приходили гості. Ми
знайшли рішення. Частину нашої консервації ми віддали моїм батькам, частину -
батькам Наташі. А решту почали дарувати. Приходить до нас хтось в \emph{гості} і каже,
що гострі помідори у нас просто чудові. І ми йому банку цих помідорів із собою
давали. Або чиясь дитина почала проявляти своє захоплення персиками з компоту,
а ми його батькам банку цього самого компоту з цими самими персиками даємо з
собою. І всі начебто щасливі. І друзі з \emph{гостей} зі смаколиками йдуть, і ми
роздаємо наші поклади. Так що чудили ми. Але шкоди від цього нікому не було
%%%cit_comment
%%%cit_title
\citTitle{Шостий президент так начудив, що наздоганятимуть і його, і його свиту}, 
Бopиcлaв Бepeзa, gazeta.ua, 24.06.2021
%%%endcit
