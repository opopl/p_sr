% vim: keymap=russian-jcukenwin
%%beginhead 
 
%%file 08_09_2021.fb.nikonov_sergej.1.bilchenko_desovetizacia_dehumanizacia
%%parent 08_09_2021
 
%%url https://www.facebook.com/alexelsevier/posts/1576843029327673
 
%%author_id nikonov_sergej,bilchenko_evgenia
%%date 
 
%%tags degumanizacia,desovetizacia,obrazovanie,ukraina
%%title БЖ. Десоветизация или дегуманизация? Вся правда про вузы в Украине
 
%%endhead 
 
\subsection{БЖ. Десоветизация или дегуманизация? Вся правда про вузы в Украине}
\label{sec:08_09_2021.fb.nikonov_sergej.1.bilchenko_desovetizacia_dehumanizacia}
 
\Purl{https://www.facebook.com/alexelsevier/posts/1576843029327673}
\ifcmt
 author_begin
   author_id nikonov_sergej,bilchenko_evgenia
 author_end
\fi

Я снова без коментариев. копия без слов и иных выражений отношения.
Евгения Бильченко.

БЖ. Десоветизация или дегуманизация? Вся правда про вузы в Украине.

Украина настолько озабочена десоветизацией, что сносит памятники дружбы с
Москвой и Мемориалы воинам-освободителям. Отринув все лучшее, что дала Украине
советская власть, мы оставляем все худшее, унаследованное от прошлых лет. Мы
любим говорить про тоталитаризм и про борьбу с ним. Между тем мы построили
вполне завершенную тоталитарную систему, без тени всякой демократии, но с
молчаливой поддержки наших западных партнеров. И эта система только по цвету
флага отличается от времен СССР, стиль мышления остался тем же. Поменялись
только символы. Точнее она – хуже советской, потому что запреты остались,
гонения остались, травля инакомыслящих осталась, а качества былого нет –
разрушили.

\ifcmt
  ig https://scontent-frt3-1.xx.fbcdn.net/v/t1.6435-9/241629320_1576837079328268_4273163428084011397_n.jpg?_nc_cat=104&ccb=1-5&_nc_sid=730e14&_nc_ohc=XBR65BJCO44AX9UdSKN&_nc_ht=scontent-frt3-1.xx&oh=9212895f87e39be0d1bdb3e545d18a9c&oe=6160567D
  @width 0.5
  @wrap \parpic[r]
\fi

Давайте обратимся к фактам и только к фактам. Их в моей жизни накопилось
достаточно.

Контроль органов безопасности над системой образования уже давно никто не
скрывает. В 2018 году у меня был первый допрос СБУ в Академии руководящих
кадров культуры и искусств (город Киев). Люди в штатском приехали прямо в
кабинет ректора, закрыли меня там, а ректор при этом называл их "своими
друзьями". Пока меня буквально зажали в кольцо в отдельной комнате несколько
человек, выпытывая подробности моих литературных связей с Россией и в частности
со сторонником СССР Захаром Прилепиным, господин ректор, бывший главный
редактор советского издательства "Молодь", изображал "участие", периодически
забегая к нам в кабинет с церковной просфорой.

Цирк закончился на том, что Финский комитет мира воззвал к здравому смыслу
через Миссию ООН Украины, и от меня отстали. Но из ученого совета Академии я
была изгнана, как мне объяснила глава ученого совета, "за стихи о мире с
Россией". Сама глава, Валерия Дмитриевна Шульгина, видимо, забыла, что в 1972
году свою диссертацию, посвященную украинскому искусству, она защищала в
Москве, в институте художественного воспитания. Почему не в Киеве?

Еще через год меня убирают из следующего ученого совета, на этот раз - по
религиоведению в Национальном педагогическом. Причину мне так и не объяснили,
но на последнем совете, где я полемизировала с очередным брачующимся с наукой,
меня возмутило, что вместо академической проблемы в диссертации обсуждается,
как плох "русский мир" и как надо с ним "бороться". 

Глава совета – Юрий Павлович Черноморец – успешно поменял несколько конфессий,
отколовшись от опасной УПЦ и примкнув к компании ПЦУ. Он любит хвастать своими
связями в высоких политических кругах. Меня бы совершенно не интересовали его
связи, если бы Юрий Павлович не стал гонителем православной церкви, что ученому
не приличествует, ученый вообще не должен давать оценки конфессиям, а тем более
сортировать их на "достойные" и "недостойные".

У нас же "мультикультурализм" и "полирелигиозность", нет?

Недавно мы беседовали с журналистом Василием Муравицким, которого продержали в
тюрьме за то, что в его речи якобы встречались "сепаратистские призывы". Я уже
знаю, кто производит контент-анализ наших лекций. Это – так называемые
карандашные эксперты органов безопасности Украины.

Во время допроса в Академии мне прямым текстом предлагалась такая роль, но я
отказалась, оповестив международные правозащитные органы. Когда в университете
надо мной устроили судилище, "опасные слова" в моих публичных лекциях вычитывал
назвавшийся "экспертом" Юрий Павлович Черноморец. Чем вам не Советский Союз?
Пусть на меня не обижаются сторонники СССР, я не являюсь его противником,
наоборот, я его очень и очень ценю, но я говорю сейчас о них – "самых свободных
людях в мире", – которые уничтожают советскую память, но не брезгуют при этом
любыми методами контроля, бывшими в СССР.

Ну, и наконец, интересно само судилище в университете. Это была провинциальная
версия суда над Бродским. За меня пытались вступиться студенты. Они принесли
пятьдесят подписей! 

Пятьдесят – против десятка кляуз молодых националистов, которые меня даже не
знали в лицо. Но моим воспитанникам в жесткой форме приказал закрыть рот декан.
Им также запрещали записывать действо, студенты делали это под партами. Потом в
украинских СМИ написали ложь о неприятии меня студентами.

Это – "демократия"?

Декан с удовольствием присоединился к травле. Дробот Иван Иванович защитил в
свое время кандидатскую диссертацию по вопросам "партийной идеологии и
политической пропаганды в совестких молодежных организациях" периода борьбы с
фашизмом. Именно Дробот, сын советского фронтовика, выказал себя сейчас ярым
сторонником неонациста Сергея Стерненко. Еще бы! Сергей Стерненко, осужденный
за убийство человека, лично пригрозил ректору нашего университета, Виктору
Андрущенко, что, если он меня не уволит, его коррупционные схемы будут
обнародованы. Потом Сергей показал публике неслабое имущество нашего ректора.
Кстати, если верить Википедии, в 1980-е годы Виктор Петрович плотно занимался
вопросами марксизма-ленинизма и советской пропаганды. Даже монографии по этому
делу выпускал. Почему-то не сидел он в тюрьме рядом с Василием Стусом и другими
националистами. Зато студенты жаловались, что Виктор Петрович активно сдавал в
аренду принадлежащие студенческой общественности участки вуза.

Вот так все просто в "молодом демократическом государстве": новая неонацистская
поросль на прикорме СБУ против старой советской бюрократии на том же прикорме.

Но, если в Советском Союзе у нас была великолепная фундаментальная наука, то
теперь на волне приватизации уничтожаются все теоретические специальности. Этой
стране мыслящие люди не нужны. Например, полностью уничтожена моя магистратура
по философии марксизма и европейского постмодерна. Ладно, почему бывшие
марксисты борются с Марксом, мне понятно, приспосабливаются. Но Европа-то чем
не угодила?

Что ж, продолжаем готовить вместо студентов послушников религиозной секты
Стерненко.

Евгения Бильченко, \verb|#112_Ua_TV|, \verb|#обозрение|

Ссылка на блог: \url{https://112ua.tv/profiles/evgeniya-bilchenko-3395.html}
