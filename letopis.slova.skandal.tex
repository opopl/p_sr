% vim: keymap=russian-jcukenwin
%%beginhead 
 
%%file slova.skandal
%%parent slova
 
%%url 
 
%%author 
%%author_id 
%%author_url 
 
%%tags 
%%title 
 
%%endhead 
\chapter{Скандал}
\label{sec:slova.skandal}

%%%cit
%%%cit_head
%%%cit_pic
%%%cit_text
\enquote{Так называемые украинцы}. Письмо Бормана Розенбергу.  В связи с последним в
1942 году возник даже \emph{скандал}. На оккупированной украинской территории кто-то
из немецких чиновников начал блокировать поставки противозачаточных средств -
под предлогом, что \enquote{недочеловекам} не нужны медикаменты.  Об этом стало
известно Гитлеру, и он сильно рассердился. Итогом стало письмо его соратника
Мартина Бормана в адрес Розенберга. В нем говорится, что славянам нужно давать
сколько угодно противозачаточных средств. А также изложены взгляды фюрера в
целом на будущее Украины.  Зацитируем этот небольшой, но очень показательный
документ из восьми пунктов. Украинцы в нем определяются эпитетом \enquote{так
называемые}
%%%cit_comment
%%%cit_title
  \citTitle{22 июня - 80 лет нападения на СССР. Что немцы готовили для украинцев}, Максим Минин, strana.ua, 22.06.2021
%%%endcit

%%%cit
%%%cit_head
%%%cit_pic
%%%cit_text
По словам девушки, огласка ситуации принесла пользу предприятию, где она
работает.  \enquote{Очень забавно, но я вчера увидела пост на странице нашей конторы,
что нам наконец-то выдали форму, никогда не выдавали, а теперь после \emph{скандала}
решили выдать. Так благодаря \enquote{Стране} ребята наконец-то получат форму, так как,
когда я пришла, покупала ее за свои деньги. Кроме того, пообещали выдать новую
машину скорой помощи, так как у нас машины в таком состоянии, что водители их
чинят посреди дороги, двери отпадают на ходу}, — сообщила Маргарита
%%%cit_comment
%%%cit_title
\citTitle{Пожаловавшуюся на украинский харьковскую студентку хотят уволить и угрожают полицией}, 
Наталья Полулях, strana.ua, 18.06.2021
%%%endcit

%%%cit
%%%cit_head
%%%cit_pic
\ifcmt
  pic https://img.strana.ua/img/article/3485/skandal-s-zhanom-60_main.jpeg
  width 0.4
	caption Жан Беленюк сфотографировался с российской журналисткой Ольгой Скабеевой во время Олимпиады. Фото: t.me/skabeeva
\fi
%%%cit_text
В Украине разгорелся очередной \emph{спортивно-политический скандал.} Не успели
утихнуть страсти после того, как легкоатлетка Ярослава Магучих на Олимпиаде
обнялась с российской спортсменкой, как в похожую историю попал борец и
обладатель золотой медали на Олимпиаде в Токио Жан Беленюк.  Он заявил, что к
нему сегодня подошли в Киеве неизвестные и начали провоцировать на конфликт,
выкрикивая расистские оскорбления
%%%cit_comment
%%%cit_title
\citTitle{\enquote{Украинец ли я для своего государства?} Как после фото со Скабеевой на Беленюка напали в Киеве}, 
Оксана Малахова, strana.ua, 13.08.2021
%%%endcit

