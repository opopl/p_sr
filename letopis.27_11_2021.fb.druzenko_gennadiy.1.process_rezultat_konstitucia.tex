% vim: keymap=russian-jcukenwin
%%beginhead 
 
%%file 27_11_2021.fb.druzenko_gennadiy.1.process_rezultat_konstitucia
%%parent 27_11_2021
 
%%url https://www.facebook.com/gennadiy.druzenko/posts/10158636799113412
 
%%author_id druzenko_gennadiy
%%date 
 
%%tags gosudarstvo,konstitucia,obschestvo,ukraina
%%title КОЛИ ПРОЦЕС ВАЖЛИВІШИЙ ЗА РЕЗУЛЬТАТ
 
%%endhead 
 
\subsection{КОЛИ ПРОЦЕС ВАЖЛИВІШИЙ ЗА РЕЗУЛЬТАТ}
\label{sec:27_11_2021.fb.druzenko_gennadiy.1.process_rezultat_konstitucia}
 
\Purl{https://www.facebook.com/gennadiy.druzenko/posts/10158636799113412}
\ifcmt
 author_begin
   author_id druzenko_gennadiy
 author_end
\fi

КОЛИ ПРОЦЕС ВАЖЛИВІШИЙ ЗА РЕЗУЛЬТАТ

Мене часто питають, що не так з українською конституцією. У відповідь я можу
довго розповідати про недосконалості її формулювань та колізії між її нормами. 

Але це дрібниці проти її головного ґанджу – українські громадяни не готові її
захищати. Бо ми не знаємо власної конституції і не розуміємо, навіщо вона нам.
Тому що народ України ніколи не був субґєктом конституційного процесу. А
національні (псевдо)еліти, пактом між якими став свого часу Основний закон, не
подужали свою історичну місію і першими почали порушувати засадничі правила
гри, про які вони наче домовились тільки вчора...

\zzrule

\ifcmt
  pic https://external-frt3-2.xx.fbcdn.net/safe_image.php?d=AQG25nCJ8U4nGAwX&w=500&h=261&url=https%3A%2F%2Fi.ytimg.com%2Fvi%2FQ8JYHMvuwuA%2Fmaxresdefault.jpg&cfs=1&ext=jpg&_nc_oe=6f28e&_nc_sid=06c271&ccb=3-5&_nc_hash=AQHZTw-YTGM2I9pR
  @width 0.8
\fi

\href{https://www.youtube.com/watch?v=Q8JYHMvuwuA}{%
\enquote{Чи вдасться Зеленському побудувати автократію? НЕ ВІРЮ!} Кириченко, youtube, %
27.11.2021%
}

\begin{multicols}{2}
Конституція має задавати рамки політичного процесу. Всі наші президенти хотіли
її змінити. Майдан був запитом на новий суспільний договір. Народ вийшов за
основну конституційну цінність – гідність.

Зеленський тягне до президентської  республіки, старі та нові еліти – до
парламентської. Зараз дуже низька політична культура влади. Як подолати
відчуження між народом та владою? Чи вдасться Зеленському побудувати
автократію? Що йому може завадити?

У новому випуску програми \enquote{Конституційна кухня} експерт з конституційного права
Юлія Кириченко. Спостерігайте за цікавою розмовою !

\url{https://www.patreon.com/druzenko}

\begin{itemize}
  \item 00:00 Інтерв’ю з Юлією Кириченко 
  \item 01:47 Про конституційну реформу 
  \item 05:00 Майдан був запитом на новий суспільний договір. Народ вийшов за основну конституційну цінність – гідність
  \item 10:29 Чому народ не реагує на свавілля?
  \item 13:09 Відчуження між народом та владою
  \item 17:19 У українців з давнини збереглося відчуття, що влада – це не \enquote{цар-самодержець}
  \item 20:00 Зеленський тягне до президентської  республіки, старі та нові еліти – до парламентської
  \item 29:00 Кто за кермом України зараз. Про низьку політичну культуру влади
  \item 40:20 Родовий гріх нашої конституції
\end{itemize}

Юлія Кириченко – керівниця проєктів з питань конституційного права Центру
політико-правових реформ, членкиня ради «Реанімаційного пакету реформ».

\enquote{Конституційна кухня} – авторський проект правника та публічного інтелектуала
Геннадія Друзенка, в якому він запрошує гостей обговорити вузлові конструкції
конституційної архітектури України. По суті це розмови про правильну модель
української державності. Де проходить легітимна межа втручання держави у життя
людини та суспільства? Яка конституційна логіка підштовхує президентів
узурпувати владу? Навіщо країнам конституційний суд? Як створити належний
баланс між центральною владою та місцевим самоврядуванням? На ці та інші
питання конституційної архітектури держави ви почуєте відповіді на
«Конституційній кухні». Чи радше навіть не готові відповіді, а приклади
правильної та глибокої дискусії, без якої ми ніколи не змоделюємо успішний
український проект.

\end{multicols}

\zzrule

Нам байдужа конституція, бо ми, українці, досі не домовились навіщо нам
держава. Ми інтуїтивно відчуваємо і навіть напевно знаємо, що державність – як
право самим визначати свою долю – нам вкрай потрібна. Ми готові вбивати і
вмирати за неї. А от чим має бути наповнена ця державність ми не тільки не
домовились – ми навіть не почали справжню дискусію про наш "суспільний
договір". Про те що перетворює мільйони етнічних українців, вірмен, кримських
татар, росіян, румунів, євреїв, білорусів, поляків тощо, що говорять різними
мовами, ходять (чи не ходять) до різних церков, слухають та співають різні
пісні, голосують (чи не голосують) за різних політиків, але спільно проживають
на українській землі на політичну націю, а країну – на політію. 

Я вірю, що справжній конституційний процес – щира суспільна дискусія, яка саме
Україна нам потрібна – може стати тими дріжжами, які перетворять мешканців на
громадян, та заквасять тісто мешканців України у політичну націю. Бо РЕСПУБЛІКА
як спільна справа апріорі неможлива без вільних, активних та відповідальних
громадян. 

Засаднича проблема України – це те, що замість якісних цеглин громадян, ми
маємо сипучий пісок споживачів української державності. І що б ми не будували з
цього піску: президентську, парламентську, премґєрську, федеративну чи унітарну
державу – ця будівля знов і знов на очах перетворюватиметься на виборну
монархію чи невиборний олігархат. Бо ми хочемо бачити у державі матір – замість
сприймати їх як спільну справу. А в президентові воліємо бачити батька –
замість тимчасового очільника нашого спільного проекту...

Дякую, \href{https://www.facebook.com/juliakyr}{Юлія Кириченко} за цю чудову розмову та суголосні думки...


