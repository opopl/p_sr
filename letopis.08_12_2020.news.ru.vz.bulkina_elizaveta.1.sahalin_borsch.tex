% vim: keymap=russian-jcukenwin
%%beginhead 
 
%%file 08_12_2020.news.ru.vz.bulkina_elizaveta.1.sahalin_borsch
%%parent 08_12_2020
 
%%url https://vz.ru/news/2020/12/8/1074662.html
 
%%author Булкина, Елизавета
%%author_id bulkina_elizaveta
%%author_url 
 
%%tags borsch,sahalin
%%title Сахалинский газовщик потушил пожар борщом
 
%%endhead 
 
\subsection{Сахалинский газовщик потушил пожар борщом}
\label{sec:08_12_2020.news.ru.vz.bulkina_elizaveta.1.sahalin_borsch}
\Purl{https://vz.ru/news/2020/12/8/1074662.html}
\ifcmt
	author_begin
   author_id bulkina_elizaveta
	author_end
\fi

\ifcmt
pic https://img.vz.ru/upimg/m10/m1074662.jpg
\fi

\index[rus]{Блюда!Борщ}

\textbf{На Сахалине сотрудник газовой службы залил вспыхнувший в ходе проверки
газового оборудования пожар борщом.}

Работник «Сахалиноблгаза» пришел в одну из квартир Охи проверить исправность
оборудования. «Утром позвонили из газоцеха, сказали – проверка приборов. Монтер
надел бахилы, прошел на кухню, разложил приборы. Я стою рядом. Он намазал
нижний кран какой-то замазкой, которая проверяет, нет ли где утечки газа.
«Видите, пузырится?» – спрашивает у меня. Я не увидела, но он взял инструмент и
стал откручивать кран, через который газ идет к плитке. Я говорю, что нужно
проверить и другой кран. Он проворачивает гайку, и раздается сильное шипение. Я
очень испугалась, вижу, что и мастер испуган, у него затряслись руки, он
пытается провернуть гайку назад. И вдруг взлетает пламя. Факел просто ужасный,
до потолка», – передает слова хозяйки квартиры «Сахалинский нефтяник».

Муж женщины бросился на первый этаж и перекрыл центральный газовый вентиль, сын
вызвал пожарных. Женщина начала звонить в газоцех, а на шум прибежали соседи.
«Зато монтер был на удивление спокоен. Взял кастрюлю только что сваренного
борща и выплеснул ее на пламя», – добавила хозяйка. 

В 2019 году Минстрой России принял новый свод правил, предполагающий
обязательное оснащение жилых новостроек газовой сигнализацией и легко
вышибаемыми при взрыве стеклопакетами. Специалист по проблемам коммунального
хозяйства Константин Крохин рассказывал, что в России взрывы бытового газа в
жилых домах чаще всего происходят в декабре – январе.
