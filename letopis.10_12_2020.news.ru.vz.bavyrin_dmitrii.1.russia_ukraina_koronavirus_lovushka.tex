% vim: keymap=russian-jcukenwin
%%beginhead 
 
%%file 10_12_2020.news.ru.vz.bavyrin_dmitrii.1.russia_ukraina_koronavirus_lovushka
%%parent 10_12_2020
 
%%url https://vz.ru/world/2020/12/10/1074902.html
 
%%author Бавырин, Дмитрий
%%author_id bavyrin_dmitrii
%%author_url 
 
%%tags covid,vaccine,russia,ukraina
%%title Россия загнала Зеленского в коронавирусную ловушку
 
%%endhead 
 
\subsection{Россия загнала Зеленского в коронавирусную ловушку}
\label{sec:10_12_2020.news.ru.vz.bavyrin_dmitrii.1.russia_ukraina_koronavirus_lovushka}
\Purl{https://vz.ru/world/2020/12/10/1074902.html}
\ifcmt
	author_begin
   author_id bavyrin_dmitrii
	author_end
\fi

\ifcmt
  pic https://img.vz.ru/upimg/m10/m1074902.jpg
  width 0.5
  fig_env wrapfigure
\fi

\textbf{Вакцина «Спутник V» оказалась в центре политической интриги, направленной
против Владимира Зеленского. Он не может принять щедрое предложение российской
стороны, как следствие, станет соучастником массового убийства украинских
граждан, предупреждают в «пророссийской» партии «Оппозиционная платформа».
Именно эта сила набирает сейчас очки, подталкивая президента Украины к краху.}

На протяжении года после президентских выборов на Украине государственные СМИ
России и ассоциированные с властью политологи, критикуя общее состояние дел в
этой стране, избегали личных нападок на президента Владимира Зеленского. Если
это не случайность, а то, что можно назвать информационной политикой, ее логика
прозрачна.

Пришел новый человек, получил солидную поддержку населения, говорит по-русски и
далек от русофобии, безразличен к украинским национальным жупелам вроде Бандеры
и томоса, обещает добиться устойчивого мира в Донбассе через компромисс – чем
плохо? Пусть попробует, не нужно его поносить заранее только за то, что
отношения России и Украины сейчас не просто плохие, а отвратительные. 

К настоящему времени Владимир Александрович Зеленский в отражении российских
медиа стал значительно менее симпатичным персонажем – не нейтральным, а резко
отрицательным. И дело тут уже не в информационной политике, а в объективных
обстоятельствах: в отношении русского языка и России вообще он продолжает
политику своего предшественника, а в Донбассе ситуация подошла к новому кризису
из-за отказа Киева\Furl{https://vz.ru/politics/2020/11/24/1071966.html} выполнять Минские соглашения и наглых эскапад его
представителей.\Furl{https://vz.ru/world/2020/12/6/1074226.html}

Проще говоря, в случае Зеленского уже не на что надеяться и некого жалеть. Если
и существует шанс на упадок русофобии и улучшение российско-украинских
отношений, он связан с теми людьми, которые придут после него. В этот транзит,
судя по всему, теперь решили вложиться. По крайней мере, глава украинского
государства был загнан в простую, изящную, но опасную для него ловушку,
соорудить которую без российского участия было нельзя. 

Накануне на сайте партии «Оппозиционная платформа – За жизнь» (то есть той
единственной крупной силы, которую можно отнести к пророссийским) появился
видеосюжет, снятый в Москве, конкретно – в НИЦЭМ им. Н.Ф. Гамалеи, то есть там,
где была создана антикоронавирусная вакцина «Спутник V».\Furl{https://vz.ru/society/2020/12/8/1074643.html} В нем один из лидеров
этой партии Виктор Медведчук заявил о принципиальной договоренности\Furl{https://vz.ru/news/2020/12/8/1074650.html} с
российской стороной о том, что Украина сможет производить вакцину на своей
территории, причем не только для собственных нужд, но и для экспорта в третьи
страны.

Глава Центра Гамалеи Александр Гинцбург\Furl{https://vz.ru/news/2020/12/8/1074673.html} подтвердил готовность «перенести
технологию и обучить специалистов». А пресс-секретарь Кремля Дмитрий Песков
впоследствии уточнил, что решение этого вопроса зависит сейчас только от доброй
воли украинских властей.

Эта новость быстро разнеслась по СМИ, вызвав самую разнообразную реакцию. Но
что касается доброй воли украинских властей, тут даже двух вариантов быть не
может – команда Зеленского точно откажется от щедрого предложения. 

\begin{leftbar}
	\begingroup
		\em
			И тем самым, как заметил депутат от ОПЗЖ Олег Волошин, «станет
				соучастником в убийстве собственных граждан», что сами граждане
				наверняка поймут (за исключением совсем уж упоротых). 
	\endgroup
\end{leftbar}

Собственно, она уже отказалась. В начале сентября это сделал глава МИД страны
Дмитрий Кулеба, заявивший, что Украина будет ждать американского или
европейского препарата и готова производить на своих мощностях только западную
вакцину. В середине октября эту позицию подтвердило министерство
здравоохранения. Высказывались в аналогичном ключе и другие высокопоставленные
чиновники – как до того, как препарат вообще появился, так и после запуска
программы вакцинирования в России.

Иного результата быть не могло. Дело не только в базовых пропагандистских
установках «нельзя ничего брать у агрессора» и «все российское – плохое». Отказ
от наших вакцин – это своего рода национальная стратегия украинской власти, при
реализации которой там не считаются с человеческими жизнями.

До Евромайдана Украина активно закупала российские вакцины, но после прихода
националистов к власти резко от них отказалась. Результатом стали вспышки и
полноценные эпидемии опаснейших заболеваний:\Furl{https://vz.ru/world/2018/3/22/913485.html} кори, ботулизма, коклюша, дифтерии
и даже полиомиелита. Оказалось, что заменить российскую продукцию нечем –
альтернатива была либо хуже (препараты из развивающихся стран), либо
значительно, в десятки раз, дороже (если поставки шли из ЕС, США и Канады). 

Корень этой проблемы даже не в русофобии новой власти, а в безудержной
коррупции – на переделе рынка\Furl{https://vz.ru/world/2015/9/7/765412.html} наживались «правильные люди». Изменить ситуацию
не смогли даже увещевания со стороны ВОЗ – там, глянув на цифры прироста жертв
особо опасных инфекций на Украине, забили тревогу на международном уровне.

Та же ситуация повторяется со «Спутник V». Причем в харьковской компании
«Биолик», упомянутой в качестве возможного места для будущего производства
российской вакцины, заявили о «гнусной лжи и отвратительной провокации, корни
которой произрастают в Российской Федерации».

Отказ Киева на фоне эпидемии COVID-19 на Украине и перегруженности местной
инфраструктуры здравоохранения (и без того дряхлой) теперь выглядит так, что
Зеленский своей несговорчивостью напрямую вредит украинцам. А у не оправдавшего
ничьих надежд Зеленского и вне темы борьбы с пандемией серьезные проблемы,
связанные, что самое интересное, со все тем же Медведчуком. 

Во-первых, страна близка к политическому коллапсу. Конституционный суд страны
упразднил большую часть закона «О предотвращении коррупции».\Furl{https://vz.ru/world/2020/10/30/1068013.html} Следствием этого
должен стать отказ международных финансовых институтов от кредитования Украины
и отмена знаменитого евросоюзовского безвиза. Офис Зеленского в панике, но
повлиять на ситуацию законным путем не может, а незаконный уже отвергло вроде
как пропрезидентское большинство в Верховной раде.

Большая часть членов КС были назначены президентом Петром Порошенко, партия
которого действует сейчас по принципу «чем хуже, тем лучше». Но жалобу в суд
направила партия Медведчука, мотивировав ее тем, что антикоррупционные
ведомства стали механизмом внешнего управления Украиной со стороны Запада.

Во-вторых, команда президента потерпела сокрушительное поражение\Furl{https://vz.ru/world/2020/10/26/1067303.html} на местных
выборах, в том числе от членов ОПЗЖ. Наиболее символичным стало то, что член
пророссийской партии выиграл у пропрезидентского кандидата даже в родном городе
Зеленского – Кривом Роге.\Furl{https://vz.ru/news/2020/12/7/1074487.html}

В-третьих, рейтинг действующего главы государства продолжает падать\Furl{https://vz.ru/politics/2020/6/30/1046630.html} и рискует
упасть до критических значений. А параллельно в лидеры электоральных симпатий
вышла все та же ОПЗЖ.\Furl{https://vz.ru/world/2020/11/14/1070097.html}

История со «Спутником V» – очередной удар по президенту со стороны оппозиции.
Отражать его будет нечем, а на горизонте уже маячит окончательная катастрофа –
дефолт.\Furl{https://vz.ru/world/2020/11/29/1072689.html}

Но глава украинского государства, кажется, до сих пор не понял, что тактика
бездействия, вызванного безволием, в кризисных ситуациях приводит к краху даже
быстрее, чем борьба за болезненные, но необходимые перемены. «Политик-надежда»
оказался «политиком-пустышкой». И кто бы ни пришел на его место, и для России,
и для Украины лучше, чтобы это произошло как можно скорее.

\begin{itemize}

\iusr{Мысль Пришла} Москва 2 дня назад  
В Цэевропе ширится и набирает обороты бандеровское движение "Нет хотим оккупации москальской вакциной!".

\iusr{Среднерусская Пчела} 4 дня назад  
Голобродько-Зеленский, - как кривое зеркало кривой украинской революции.

\iusr{Евгений Железников} Екатеринбург4 дня назад  
Мне лично как-то пофиг, что будет с Украиной и украинцами. Они сами выбрали свою судьбу, пущай теперь ее и принимают.

\iusr{Защитник веры} Сургут 4 дня назад  

Пока ещё ныне действующий презик Хохляндии поступает совершенно правильно.
Послностью поддерживаю его политику по невакцинированию своих подопечных
селюков дуже поганой москальской вакциной, а другой пока и близко не
предвидится. Количество селюков с этой территории должно неуклонно, неумолимо
сокращаться. Это является прямой, осознанной политикой Запада в отношении
хохлов. И здесь у нас с ними нет разногласий.

\iusr{Олицетворение добра} 4 дня назад  
Да ладно, статья для кацаполохов. В Украине даже близко не рассматривается
вопрос производства ослиной мочи под названием "Спутник V".

\iusr{narg Безымянный} 4 дня назад
Спаси мир от дебилов - пользуйся презервативом!!! 

\iusr{Яков Слащёв} 3 дня назад

На Украине вообще не планируют ничего производить. Даже ослиной мочи
собственного производства. Ослы все передохли. А западная ослиная моча им будет
впариваться по цене божьей росы. Хотя при этом будет оставаться ослиной мочой.
А на неё у них денег нет...

\iusr{Evgeny A} Москва 4 дня назад  

Нет у Зеленского никакой ловушки. Ему и его кастрюлеголовому стаду, которых в
сумме всего-то ничего, помогут с вакциной западные хозяева. А остальных чего
жалеть. Бабы еще нарожают.
\end{itemize}
