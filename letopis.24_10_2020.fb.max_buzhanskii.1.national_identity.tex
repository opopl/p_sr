% vim: keymap=russian-jcukenwin
%%beginhead 
 
%%file 24_10_2020.fb.max_buzhanskii.1.national_identity
%%parent 24_10_2020
 
%%url https://www.facebook.com/permalink.php?story_fbid=1779163805581424&id=100004634650264
%%author max buzhanskii
%%tags national identity
%%title Национальная идентичность
 
%%endhead 

\subsection{Национальная идентичность}

\Purl{https://www.facebook.com/permalink.php?story_fbid=1779163805581424&id=100004634650264}
\Pauthor{Бужанский, Максим}

Институт Национальной Памяти тонет в онлайн конференциях, отчаянно пытаясь понять, что же такое национальная идентичность и кого считать героями.
Тут, в принципе, фокус очень простой, тех, кто герой, того и считать, но тогда немного другая картина складывается.
Как бы и Институт Нацпамяти не совсем нужен, чтоб не сказать, не нужен вообще.
Берётся какой то персонаж, и рассказывается какая то душераздирающая история про то, как он отчаянно боролся с советами.
Тут всегда возникает вопрос, так он слегка коллаборант, вон на фото, фуражка немецкая так лихо заломлена, что Гиммлер от зависти места себе не находит.
Та нет, говорят в УИНПе, не обращайте внимания, то у него просто такая форма непримиримой борьбы была, а так, он вообще не советский гражданин, и в Украине то и не бывал толком.
Ну нет, так нет, не наш, так и не наш, так с чего вдруг надо проникнуться таким щемящим чувством к польскому то коллаборанту в таком случае?
Ннну, отвечает УИНП, ответ на этот вопрос мы как раз ищем, надо как то так по хитрому идентичность выписать, чтобы само всё сложилось.
И вон тот, в фуражке под Гиммлера как то оказался приличным человеком, а не тем, что есть.
Ну ок.
Иван Бойко, крестьянин.
Как водится у издевавшихся своими социальными лифтами совков, один из шести детей, село Жорнице под Винницей.
Не Католический лицей и не Иезуитский колледж, нет, сельская школа семилетка и вкалывал   с 11 лет.
Героическая карьера Петлюры в Украине закончилась, когда Бойко как раз исполнилось 10, поэтому с советами он не боролся до этого легендарного момента, и как то не боролся после.
Можно, конечно, порассуждать об идентичности и тут, но так вышло, что он и был этими самыми советами.
В 30м году, в 20 лет, добровольцем в армию, Ульяновское Танковое, школа, курсы, из крестьянского пацана делали офицера, глубоко наплев на вековые традиции царской армии, о которых давно успели благополучно забыть.
Теория о том, что только лорд в седьмом поколении может быть офицером сменилась практикой.
Берёшь сельского пацана, и хорошо учишь его, а Родину он и сам любит не хуже остальных.
Бойко любил.
И на Халхин-голе любил, уже командиром танковой роты.
С июля 41 на фронте.
Когда говорят, что вся кадровая армия погибла летом-осенью 41го года или попала в плен, слегка врут, нарочно или по незнанию.
Не вся.
Так вышло, что кадровая то как раз уцелела, а вот мобилизованные в начале Великой Отечественной, в очень большом количестве, нет...
Подробно боевой путь Ивана пересказывать не буду, просто отмечу одну деталь.
Обе свои Золотые Звезды Героя Советского Союза, Бойко получил за три  месяца.
10.01.44 и 26.04.44.
Что нужно было сделать в Красной Армии, чтобы получить ДВЕ!!! Звезды Героя за сто дней?
Два ордена Ленина к ним, плюс три ордена Красного Знамени, орден Богдана Хмельницкого , орден Суворовова, Отечественной Войны и Красную Звезду, за весь период войны.
Вопрос.
Нуждается ли украинец до мозга костей Иван Никифорович Бойко в национальной идентификации?
Ответ.
Нет, не нуждается.
Вопрос.
Нуждается ли Бойко в героизации?
Вежливый ответ.
Нет, спасибо, у него всё в порядке с героизацией.
Вопрос.
Чем он не подходит Институту Национальной Памяти?
Ответ.
Ничем.
Может он как то меньше любил свою страну, чем Шухевич?
Не похоже, судя по боевым заслугам.
Может он чего то не понимал, политически слепым был?
Вряд ли, его успехам в Лондоне и Вашингтоне аплодировали, похоже случайно угадал с политическим вектором.
Так куда его девать, на его фоне весь раздел учебника истории меркнет и совсем уж смешным выглядит, а таких как он, по селам, на командирские курсы сотни тысяч понабиралось.
Как куда?
Вычеркнуть из учебника.
Вопрос.
А кто тогда освободил Черновцы?
Ответ.
Надо искать ответ на вопрос о национальной самоидентификации...
Не лицо ж, свирепая морда, ну разве б такого пустили играть на рояле в абвере, а?
Да нет, не пустили бы, посмотрите на него, какой тут абверовский рояль?

T.me/MaxBuzhanskiy

Комментарии: 216
