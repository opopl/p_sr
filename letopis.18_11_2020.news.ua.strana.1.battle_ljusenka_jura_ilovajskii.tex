% vim: keymap=russian-jcukenwin
%%beginhead 
 
%%file 18_11_2020.news.ua.strana.1.battle_ljusenka_jura_ilovajskii
%%parent 18_11_2020
 
%%url https://strana.ua/news/301611-arestovich-i-butusov-o-chem-battl-dvukh-adeptov-partii-vojny.html
%%author 
%%tags 
%%title 
 
%%endhead 

\subsection{Битва \enquote{Люсеньки} с \enquote{Юрой Иловайским}. Как и почему \enquote{фельдмаршал} Бутусов и Арестович устроили публичный баттл}
\label{sec:18_11_2020.news.ua.strana.1.battle_ljusenka_jura_ilovajskii}
\Purl{https://strana.ua/news/301611-arestovich-i-butusov-o-chem-battl-dvukh-adeptov-partii-vojny.html}
\Pauthor{Венк, Виктория}
\par 18:10, сегодня

\index[names.rus]{Бутусов, Юрий!18 ноября 2020, Страна.УА}
\index[names.rus]{Арестович, Алексей!18 нояюря 2020, Страна.УА}

\ii{18_11_2020.news.ua.strana.1.battle_ljusenka_jura_ilovajskii.intro}
\ii{18_11_2020.news.ua.strana.1.battle_ljusenka_jura_ilovajskii.predmet_spora}
\ii{18_11_2020.news.ua.strana.1.battle_ljusenka_jura_ilovajskii.butusov_i_ilovajsk}
\ii{18_11_2020.news.ua.strana.1.battle_ljusenka_jura_ilovajskii.smysl_battla}

\subsubsection{В чем смысл баттла?}

Конфликт Арестовича и Бутусова - эхо войны, которая идет между Офисом
президента и командой Турчинова-Порошенко. У бывшего гаранта таким образом
консолидируют свой электорат, раскачивая тему "зрады" Зеленского.

И, надо сказать, тактика дала свой эффект: на местных выборах
"Евросолидарность" перетянула на себя часть националистически настроенных
избирателей "Слуги народа". Активно в этой истории поучаствовал и Бутусов,
раскручивая, к примеру, громкую историю о госизмене Ермака по делу
"вагнеровцев". 

Но вернемся к Арестовичу. Его взяли на работу в ТКГ вскоре после
увольнения умеренного экс-премьера Витольда Фокина.\Furl{https://strana.ua/news/282441-vitold-fokin-chto-izvestno-o-pervom-premere-ukrainy-kotoroho-mohut-otpravit-na-donbass.html} Того в свою очередь
пригласили накануне местных выборов, чтобы создать иллюзию прогресса в
минском процессе. И остановить отток юго-восточных избирателей к
"Оппозиционной платформе". 

Однако после того, как Фокин в интервью "Стране" заявил,\Furl{https://strana.ua/news/282441-vitold-fokin-chto-izvestno-o-pervom-premere-ukrainy-kotoroho-mohut-otpravit-na-donbass.html} что нужно
исполнять Минские соглашения во всей полноте - его убрали. А через
какое-то время поставили Арестовича, который до этого считался один из
главных ЛОМов условной "партии войны". Что, видимо, по задумке Зеленского
и Ермака, должно было накал "зрады" снизить. 

Однако, как показала ситуация с Бутусовым, "партии войны" абсолютно без
разницы, какие декорации на Банковой возводят вокруг темы Донбасса. Атака
идет в принципе на весь процесс отношений с РФ в рамках минской площадки.

Цель при этом двоякая.

Во-первых - загнать президента за флажки, "красные линии", прочерченные
Порошенко и Ко, ликвидировав даже теоретическую возможность каких-либо
компромиссов с Россией и "республиками". 

Во-вторых - дискредитировать Зеленского среди промайданной части
избирателей, представив его едва ли не агентом Кремля. Хотя по факту его
политика по "Минску" мало чем отличается от времен Порошенко. У
последнего, впрочем, можно найти еще больше "зрады" - он не только
подписал Минские соглашения, но и продавил в Раде принятие закона об
особом статусе Донбасса. 

Поэтому - если Офис президента наймет для своего пиара по Донбассу даже не
Арестовича, а, к примеру, самого Парубия, отношение "партии войны" к
переговорам и к самому Зеленскому это никак не изменит.
