% vim: keymap=russian-jcukenwin
%%beginhead 
 
%%file 16_12_2021.stz.news.lnr.lug_info.1.shkolniki_novyj_god_ukrashenia
%%parent 16_12_2021
 
%%url https://lug-info.com/news/shkol-niki-lnr-predstavili-bolee-400-novogodnih-ukrashenij-na-vystavke-v-luganske
 
%%author_id 
%%date 
 
%%tags deti,donbass,lnr,novyj_god,ukrashenie,tvorchestvo,shkola,lugansk
%%title Школьники ЛНР представили более 400 новогодних украшений на выставке в Луганске
 
%%endhead 
\subsection{Школьники ЛНР представили более 400 новогодних украшений на выставке в Луганске}
\label{sec:16_12_2021.stz.news.lnr.lug_info.1.shkolniki_novyj_god_ukrashenia}

\Purl{https://lug-info.com/news/shkol-niki-lnr-predstavili-bolee-400-novogodnih-ukrashenij-na-vystavke-v-luganske}

Школьники ЛНР представили более 400 поделок на выставке-конкурсе новогодних
украшений \enquote{Зимняя сказка} в Республиканском центре научно-технического
творчества (РЦНТТ), выступившем организатором мероприятия. Об этом ЛИЦ сообщил
временно исполняющий обязанности директора РЦНТТ Виктор Бончук.

\ii{16_12_2021.stz.news.lnr.lug_info.1.shkolniki_novyj_god_ukrashenia.pic.1}

\enquote{Сегодня открывается республиканская выставка-конкурс новогодних
украшений \enquote{Зимняя сказка}. В данной выставке представлено более 400
экспонатов }, - сказал он.

Бончук уточнил, что на выставке представлены работы учащихся школ,
школ-интернатов и организаций дополнительного образования городов и районов
Республики. Возраст участников - от 7 до 18 лет.

\enquote{Данная выставка проходит ежегодно, это выставка уже седьмая. Традиционно в
рамках акции \enquote{Единение детских сердец} представленные на выставке новогодние
маски, костюмы, открытки и другие украшения будут направлены в детские дома,
школы-интернаты ЛНР для оформления новогодних и рождественских праздников}, -
пояснил руководитель РЦНТТ.

\ii{16_12_2021.stz.news.lnr.lug_info.1.shkolniki_novyj_god_ukrashenia.pic.2}

Он рассказал, что конкурсные работы оценит компетентное жюри, в состав которого
вошли представители Министерства образования и науки ЛНР, образовательных
организаций дополнительного образования и учреждений культуры Республики.

Воспитанница народного художественного коллектива \enquote{Бисерное искусство}
Луганского учебно-воспитательного объединения (ЛУВО) \enquote{Восток} Елизавета
Селизнева рассказала, что представила на выставке-конкурсе работу \enquote{Ожерелье для
Снегурочки}.

\ii{16_12_2021.stz.news.lnr.lug_info.1.shkolniki_novyj_god_ukrashenia.pic.3}

\enquote{Занимаюсь бисером уже семь лет - с четырехлетнего возраста. Ожерелье для
выставки делала очень долго - где-то около двух месяцев. Я вдохновилась узорами
на окнах. Мне они очень сильно нравятся, как и сама зима. Вот я и решила
повторить эти узоры в своей работе}, - пояснила школьница.

\ii{16_12_2021.stz.news.lnr.lug_info.1.shkolniki_novyj_god_ukrashenia.pic.4}

Руководитель народного художественного коллектива \enquote{Бисерное искусство} ЛУВО
Анна Пономаренко отметила, что ее ученики ежегодно участвуют в
выставке-конкурсе \enquote{Зимняя сказка}.

\enquote{В этом году у нас на выставке представлены три работы. В целом выставка очень
интересная, здесь представлено много идей для воплощения, и для своего
творчества есть, что здесь подсмотреть}, - добавила она.

Выставка продлится до 25 декабря. Ее можно посетить с понедельника по пятницу с
9:00 до 15:00 по адресу: Луганск, ул. Ватутина, 87

\ii{16_12_2021.stz.news.lnr.lug_info.1.shkolniki_novyj_god_ukrashenia.pic.5}

