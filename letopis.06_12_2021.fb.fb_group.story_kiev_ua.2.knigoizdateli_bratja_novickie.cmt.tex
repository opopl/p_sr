% vim: keymap=russian-jcukenwin
%%beginhead 
 
%%file 06_12_2021.fb.fb_group.story_kiev_ua.2.knigoizdateli_bratja_novickie.cmt
%%parent 06_12_2021.fb.fb_group.story_kiev_ua.2.knigoizdateli_bratja_novickie
 
%%url 
 
%%author_id 
%%date 
 
%%tags 
%%title 
 
%%endhead 
\zzSecCmt

\begin{itemize} % {
\iusr{Лариса Мысник}
Спасибі. Дуже пізнавально

\iusr{Евгения Ерёменко}
\textbf{Лариса Мысник}  @igg{fbicon.face.happy.two.hands} 

\iusr{Ирина Архипович}
Очень серьёзно потрудилась, Женечка !!  @igg{fbicon.thumb.up.yellow}  @igg{fbicon.face.happy.two.hands} 

\iusr{Евгения Ерёменко}
\textbf{Ирина Архипович} так о нашей же alma mater!

\iusr{Андрей Сидоровский}

Ни фига себе! Очень серьёзная работа по зарождению полиграфии Киева, просто
потрясающе!. Мне кажется, вполне заслуженно может стать монографией, браво!

\begin{itemize} % {
\iusr{Евгения Ерёменко}
\textbf{Андрей Сидоровский} 

Андрей, благодарю за добрые слова, но не стоит преувеличивать всё же
@igg{fbicon.face.grinning.smiling.eyes} 

\iusr{Андрей Сидоровский}
\textbf{Евгения Ерёменко} 

хммм...а почему бы и нет? Вы бы видели, что печатается в \enquote{Web of science} или в
\enquote{Scopus}, (по идиотским требованиям на звание кандидата наук, там должно что-то
быть), Ваша работа просто жемчужина в потоке плагиата и компиляции.


\iusr{Евгения Ерёменко}
\textbf{Андрей Сидоровский}  @igg{fbicon.face.smiling.eyes.smiling} 
\end{itemize} % }

\iusr{Ирине Вильчинская}

блестяще! спасибо! умыкну, с вашего позволения на святошинские странички. Как
раз активисты борются за сохранение тех остатков святошинских дач, над которыми
невисли ковши экскаваторов застройщиков, и пытаются отстоять экскурсионный
маршрут по святошинским дачам. Этот материал просто находка!

\begin{itemize} % {
\iusr{Евгения Ерёменко}
\textbf{Ирине Вильчинская} спасибо, Ирина! В святошинскую группу тоже отправила, надеюсь Кирилл выставит в ближайшее время. Собственно мысль об экскурсионном маршруте мне и не даёт спать  @igg{fbicon.face.happy.two.hands} 

\iusr{Ирине Вильчинская}
\textbf{Евгения Ерёменко} да, эта идея классная! сделать маршрут-реконструкцию по Святошину. Опираясь на оставшиеся строения и дополнив фотоматериалами. это было бы просто роскошно!

\iusr{Евгения Ерёменко}
\textbf{Ирине Вильчинская} и фотоархив уже вполне достаточный тоже поднакопился  @igg{fbicon.face.grinning.smiling.eyes} .
\end{itemize} % }

\iusr{Марина Антощук}
Интересно! Какие замечательные люди жили здесь и творили.

\iusr{Марина Антощук}

Да, это действительно может стать основой для маршрута. Я даже и не знала, что
на месте гимназии восточных языков была парковая зона.

Ее можно восстановить! Ведь идея каскада парков и туристического маршрута по
святошинским дачам уже у всех на слуху.

\begin{itemize} % {
\iusr{Ирине Вильчинская}
\textbf{Марина Антощук} 

Обычно хорошая, правильная идея \enquote{обрастает} дельными инициативами и интересными
материалами, идеями. Это было бы просто шикарно! Евгения Еременко публикует
просто ценнейшие материалы, делясь своими находками.

\begin{itemize} % {
\iusr{Марина Антощук}
\textbf{Ирине Вильчинская} Да, нам пора уже объединять усилия. Вырисовывается замечательная картина.

\iusr{Ирине Вильчинская}
\textbf{Марина Антощук} 

да, и группа краеведов-исследователей \enquote{нарисовалась} - Кирилл Степанец, Евгения
Еременко...Недавно видела группу, которую вела по Святошину
женщина-экскурсовод. Район интересный, исторический и, безусловно,
привлекателен для туристов, которые уже знают киевские привычные маршруты, а
этот - внове! И с новыми гранями.


\iusr{Марина Антощук}
\textbf{Ирине Вильчинская} Да, думаю, что Екатерина Кольт и Vira Stepanova присоединятся.

\iusr{Ирине Вильчинская}
\textbf{Марина Антощук} обязательно! Катя тоже столько усилий прилагала для сохранения дач, Святошина...

\iusr{Нина Светличная}
\textbf{Ирине Вильчинская} 

Вообще то имеет смысл подумать об издании сборника. Понимаю, что сейчас полетят
в меня тапки, но я буду отбиваться) Прекрасно знаю, что такое издать сейчас
книгу. Но нужно с чего то начинать, собрать воедино материалы, отобрать
иллюстрации. Девиз моей бывшей \enquote{шефини} - капля камень точит.

\iusr{Евгения Ерёменко}
\textbf{Ирине Вильчинская} 

да, Ирина, я сознательно ограничила себя рамками дореволюционного дачного
Святошина. Во-первых, небольшой временной охват, во-вторых, это зеркало
киевской жизни в лучшей ее части, ведь здесь и правда элита жила. Оттуда и
Академгородок мой родной (сколько выдающизся учёных было!), и Авиагородок
(конструкторы жили, тот же Григорович - не путать с Григоровичами-Барскими!).

Ещё одна потрясающая тема - по науковедению, если кому по силам. Столько
всемирно известных учёных рядом жило и трудилось, столько в мировую науку
внесено... Один ин-т материаловедения чего стоит. Кстати, в нём завотделом был
С. Л. Берия - сын того самого человека  @igg{fbicon.face.smiling.eyes.smiling} .

В ин-те металлофизики стараниями д.т.н. В. М. Надутова был создан институтский
музей. Не уверена, что после его смерти работает


\iusr{Ирине Вильчинская}
\textbf{Евгения Ерёменко} 

вот это именно те новые аспекты, которые я имела ввиду. У нас ведь не только
древнейшая история (безусловно, тоже интереснейшая и уникальная), но и новейшая
не менее увлекательная и, практически, не раскрытая по-новому! А старые,
советские штампы (которые тоже не всегда были плохи!) все же требуют нового и
осмысления, и подачи.


\iusr{Евгения Ерёменко}
\textbf{Ирине Вильчинская} именно так

\end{itemize} % }

\end{itemize} % }

\iusr{Елена Мельникова}
Интереснейшая статья!

\iusr{Евгения Ерёменко}
\textbf{Елена Мельникова} 

благодарю, Елена! В наступающем году желаю всего доброго
@igg{fbicon.face.smiling.eyes.smiling} 

\end{itemize} % }
