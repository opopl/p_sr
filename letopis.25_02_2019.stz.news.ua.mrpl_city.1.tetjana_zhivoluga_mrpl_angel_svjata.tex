% vim: keymap=russian-jcukenwin
%%beginhead 
 
%%file 25_02_2019.stz.news.ua.mrpl_city.1.tetjana_zhivoluga_mrpl_angel_svjata
%%parent 25_02_2019
 
%%url https://mrpl.city/blogs/view/tetyana-zhivoluga-mariupolskij-angel-svyata
 
%%author_id demidko_olga.mariupol,news.ua.mrpl_city
%%date 
 
%%tags 
%%title Тетяна Живолуга: маріупольський ангел свята
 
%%endhead 
 
\subsection{Тетяна Живолуга:\ \ маріупольський ангел свята}
\label{sec:25_02_2019.stz.news.ua.mrpl_city.1.tetjana_zhivoluga_mrpl_angel_svjata}
 
\Purl{https://mrpl.city/blogs/view/tetyana-zhivoluga-mariupolskij-angel-svyata}
\ifcmt
 author_begin
   author_id demidko_olga.mariupol,news.ua.mrpl_city
 author_end
\fi

Дуже енергійна, творча, сильна, наполеглива і водночас тендітна… У неї за
спиною невидимі крила, в серці – безмежна любов і неймовірна сила, а в житті –
потужна діяльність, завдяки якій у Маріуполі народжуються незабутні свята.

\ii{25_02_2019.stz.news.ua.mrpl_city.1.tetjana_zhivoluga_mrpl_angel_svjata.pic.1}

Наша героїня – \textbf{Тетяна Володимирівна Живолуга} – Маріуполь полюбила з дитинства.
Вона згадує, як з батьками гуляла вулицями міста і насолоджувалася чарівністю
неба, загадковістю маріупольських вуличок. Обожнює й Азовське море. Більшу
частину життя мати Тетяни пропрацювала завідувачкою дитячим садом. А в дитячий
садок прийшла працювати нянею, бо її Таточка весь час там плакала. І, хоч
працювала не в групі, де була її донечка, але так її сердечко менше
розривалося. Потім вона стала вихователем. Тетяна Володимирівна, наголошує, що
її Мама - Педагог з великої літери і для неї головний Вчитель. До речі, у
Тетяни три улюблених жіночих образи: мама, Мати Тереза і Маргарет Тетчер.

\textbf{Читайте також:} \emph{Праздники в стиле китч-лакшери в Мариуполе: Сказки от Татьяны Живолуги (ИНТЕРВЬЮ)}%
\footnote{Праздники в стиле китч-лакшери в Мариуполе: Сказки от Татьяны Живолуги (ИНТЕРВЬЮ), Яна Іванова, mrpl.city, 02.12.2018, \par
\url{https://mrpl.city/news/view/prazdniki-v-stile-kitch-laksheri-v-mariupole-skazki-ot-tatyany-zhivolugi-intervyu}
}

\ii{25_02_2019.stz.news.ua.mrpl_city.1.tetjana_zhivoluga_mrpl_angel_svjata.pic.2}

Вдома завжди була атмосфера казковості й дива. Цю атмосферу підтримував і
батько. У школі юна Тетяна теж брала участь в організації культурних заходів.
Вона з дитинства була оточена творчістю і позитивом, який і сьогодні продовжує
дарувати всім навколо.

Тетяна має дві освіти – викладач педагогіки і дошкільної психології та
практичний психолог. Має педагогічний стаж, адже 7 років пропрацювала
вихователем у дитячому садочку. Працювати у сфері культури почала з 2010 року,
коли стала завідувачкою художнім сектором в Міському палаці культури
Лівобережного району. Саме на цій посаді Тетяна отримала дуже корисний і
важливий для неї досвід. Культурним організатором в ПК \enquote{Молодіжний}
пропрацювала два роки (з 2016 до 2018 року), але за дуже короткий час змогла
провести безліч яскравих і унікальних загальноміських \enquote{івентів}, що
відрізняються нестандартністю і креативним підходом.

\ii{25_02_2019.stz.news.ua.mrpl_city.1.tetjana_zhivoluga_mrpl_angel_svjata.pic.3}

У 2018 р. Тетяна Володимирівна знову почала працювати в Міському палаці
культури Лівобережного району. Зараз всі сили спрямовані на підготовку до
святкування 8 березня, яке, судячи з розповіді, буде відрізнятися великою
кількістю квітів, а особливо, бузку. Сама Тетяна переживає, чи вдасться їй
дістати стільки квітів, але якщо вирішила, то значить повинно вийти! Загалом
Тетяна Живолуга – справжній генератор ідей, завдяки нестандартному і
креативному мисленню з нею легко і цікаво працювати. Для неї не існує
стереотипів. Її Діди Морози та Снігуроньки в різних точках міста проводять
флешмоби, раптово влаштовують веселощі в маршрутках, змагаються в батлах. Вона
здібний організатор і талановитий педагог, їй вдається об'єднати навколо себе
молодь, яку навчає працювати у команді. Вражає, що наша героїня постійно
займається самовдосконаленням, кожного дня продовжує навчатися, адже вважає, що
робити все треба якісно і гарно. І вона робить! Кожного дня Тетяна пам'ятає і
нагадує іншим, що необхідно за собою прибирати, не шкодити іншим, любити себе,
але жити безпечно для тих, хто поруч. Нести позитивні емоції та створювати
комфортні умови для існування, а якщо плакати, то тільки від щастя.

\textbf{Читайте також:} \emph{Как Деды Морозы и Снегурочки ходили в гости к мариупольцам и веселили их на площадях}%
\footnote{Как Деды Морозы и Снегурочки ходили в гости к мариупольцам и веселили их на площадях, Яна Іванова, mrpl.city, 17.01.2018, \par%
\url{https://mrpl.city/news/view/kak-dedy-morozy-i-snegurochki-hodili-v-gosti-k-mariupoltsam-i-veselili-ih-na-ploshhadyah-foto}
}

\ii{25_02_2019.stz.news.ua.mrpl_city.1.tetjana_zhivoluga_mrpl_angel_svjata.pic.4}

Одним з головних проектів Тетяни залишається \enquote{ЛюдиСвята}. Створений у 2016
році, він і сьогодні є важливим здобутком міста. Колектив проекту складається з
талановитих, творчих і здібних молодих людей, які вміють створити потрібний
настрій. Проте за усміхненими і переодягненими в яскраві костюми героями стоїть
колосальна робота. Сьогодні учасники колективу Тетяни вміють не тільки створити
образ, але й самі шиють собі костюми, наносять грим. Як наголошує наша героїня:
\emph{\enquote{Прийшов час універсальних людей. Ти повинен вміти багато}}. У Тетяни
Володимирівни напрочуд легко виходить передавати знання молоді, що з нею
працює, хоча вона й відрізняється вимогливістю, а інколи й жорстким підходом.
Найважливіше, що Тетяна готова до сучасних викликів часу і знає, як їм
відповісти. Звідки у неї стільки сил і ідей, залишається загадкою. Її оптимізму
та енергійності можна позаздрити, але мало хто замислюється, скільки невтомному
організатору свят доводиться працювати і з колективом, і над собою. Вона
справжній перфекціоніст, для неї головне - не просто зробити, а зробити якісно
і правильно, по-справжньому гарно.

\ii{25_02_2019.stz.news.ua.mrpl_city.1.tetjana_zhivoluga_mrpl_angel_svjata.pic.5}

Родина для Тетяни – справжня опора і надія. Розмовляючи з нею, розумієш,
скільки вона має загальних інтересів з синами. Вони разом читають, дивляться
фільми, разом вчаться долати невдачі і здобувати перемоги. Загалом Тетяну
надихає багато речей. Це і її сім'я, і Маріуполь, і гарна атмосфера. Вона може
насолоджуватися приємним світлом ліхтарів на вулицях Маріуполя чи міськими
подвір'ями, старовинними маріупольськими будівлями, дивлячись на які, уявляє,
як жили раніше. А може сісти в автобус і поїхати в інше місто, де буде гуляти і
насолоджуватися кожною миттю. Тетяна Живолуга любить, щоб було красиво: і у
вчинках, і в душі, і візуально!

\textbf{Читайте також:} \emph{Вікторія Лісогор: берегиня книжкового фонду Маріуполя, Ольга Демідко, mrpl.city, 19.02.2019}
\footnote{\url{https://mrpl.city/blogs/view/viktoriya-lisogor-bereginya-knizhkovogo-fondu-mariupolya}} %
\footnote{Internet Archive: \url{https://archive.org/details/19_02_2019.olga_demidko.mrpl_city.viktoria_lisogor_beregynja_fondu_mariupolja}}

Цікаво, що Тетяна Володимирівна обожнює колекціонувати різні маленькі версії
всього. Це можуть бути різноманітні речі, які відрізняються за ціною, але мають
унікальний вигляд.

Тетяна вірить, що найважливіші перемоги і відкриття у неї та її дітей ще
попереду. І вона готова йти їм назустріч, але при цьому кожного дня
продовжувати робити все якісно і гарно.

\ii{25_02_2019.stz.news.ua.mrpl_city.1.tetjana_zhivoluga_mrpl_angel_svjata.pic.6}

\textbf{Улюблені книги Тетяни Володимирівни:} трилогія Юрія Германа \enquote{Дело,
которому ты служишь. Дорогой мой человек. Я отвечаю за все}; поезія Василя
Симоненка, Ліни Костенко, Максима Рильського та Ігоря Сєверяніна.

\textbf{Улюблені фільми:} \enquote{Король Лев}.

\textbf{Порада маріупольцям:} \emph{\enquote{Любити Маріуполь просто і безумовно не за
що-небудь, адже рідне місто, як і Батьківщину, як маму не обирають. Ми повинні
знайти, як зробити гарніше і щасливіше, потрібно зберегти чистоту і в душі, і
навколо своїх під'їздів}}. Тетяна Володимирівна сподівається, що молоді
маріупольці не будуть їхати з міста, а залишатимуться і намагатимуться
створювати щось нове, розвиватимуть рідний Маріуполь.

\textbf{Читайте також:} \emph{Маріуполець Євген Сосновський: таке красиве життя, Ольга Демідко, mrpl.city, 19.01.2019}%
\footnote{\url{https://mrpl.city/blogs/view/mariupolets-evgen-sosnovskij-take-krasive-zhittya}} %
\footnote{Internet Archive: \url{https://archive.org/details/19_01_2019.olga_demidko.mrpl_city.sosnovsky_take_krasyve_zhyttja}}

