% vim: keymap=russian-jcukenwin
%%beginhead 
 
%%file slova.doroga
%%parent slova
 
%%url 
 
%%author_id 
%%date 
 
%%tags 
%%title 
 
%%endhead 
\chapter{Дорога}
\label{sec:slova.doroga}

%%%cit
%%%cit_head
%%%cit_pic
%%%cit_text
Інша \emph{дорога} виглядає начебто утопічною, хоч насправді і є успішною. Її проклали
українські Майдани: самоорганізація на основі чітко усвідомлених і засвоєних
цінностей. Це – українська демократична традиція, базована на громадах. Це –
\enquote{єдність у багатоманітті}, коли ми погоджуємося бути різними за нашими
вподобаннями, але залишаємося єдиними й солідарними за нашими цінностями.  Що
це за цінності? Вони протилежні до антицінностей, що їх пропагує Кремль, тобто
спираються на людську гідність, вірність правді, взаємній пошані, готовності до
розважливості та справедливості. Саме за ними мають упізнати, що ми – не Росія.
Згадаймо, ці етичні принципи принесли нам успіх на трьох українських Майданах,
оскільки саме вони уневажнили тодішні кремлівські сценарії. Чому б нам не
спертися на ці принципи, щоб уневажнити ще й нинішній сценарій?
%%%cit_comment
%%%cit_title
\citTitle{Надходить велика доба}, 
Ініціативна Група Першого Грудня, www.pravda.com.ua, 01.12.2021
%%%endcit
