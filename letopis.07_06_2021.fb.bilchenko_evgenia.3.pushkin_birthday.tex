% vim: keymap=russian-jcukenwin
%%beginhead 
 
%%file 07_06_2021.fb.bilchenko_evgenia.3.pushkin_birthday
%%parent 07_06_2021
 
%%url https://www.facebook.com/yevzhik/posts/3959167390785008
 
%%author Бильченко, Евгения
%%author_id bilchenko_evgenia
%%author_url 
 
%%tags 
%%title БЖ. С днём рождения, Пушкин!
 
%%endhead 
 
\subsection{БЖ. С днём рождения, Пушкин!}
\label{sec:07_06_2021.fb.bilchenko_evgenia.3.pushkin_birthday}
\Purl{https://www.facebook.com/yevzhik/posts/3959167390785008}
\ifcmt
 author_begin
   author_id bilchenko_evgenia
 author_end
\fi

БЖ. С днём рождения, Пушкин!
Родня! Поздравляю Вас с днём рождения нашего золотого колокола - Александра Сергеевича Пушкина! Всегда мечтала, как Марина Ивановна, сбежать с ним за руку по коктебельскому холму и поговорить о главном глубокой ночью на советской кухне (Пушкин ценил дружбу с некрасивыми девочками мужского склада, оставляя страсти для хорошеньких). Всегда считала, как Осип Мандельштам прав, говоря, что Пушкин "ещё не родился", - в том смысле, что русский Логос, как и любая Традиция, - это живое внутреннее начало свободы, а не внешний мертвый конец принуждения. Никогда не была "Пушкиным в меру пушкиньянца", ибо не "с этой безмерностью - в мире мер" (МЦ). Понимаю его всего: от романтики декабризма до царского православия, от классики до новостроек языка, от тщательной редактуры до полного игнора ударений, от Анны Керн до Наташи Гончаровой и их деток с выпадающими молочными зубками, от щедрости абсолютного поступка в себе до подсчитывания копеек на балы и мазурки. Всего - его. О, родной мой, тебя бы сейчас назвали хайпометом и креатором, ибо "уже прокатилась дурная слава", что "похабник" ты и "скандалист" (Есенин).
Моя бабушка перечитала все его письма. И мне в наследство досталось петербургское издание его заметок, посмертное, уже после дуэли.
Ну, и поскольку наш родной и любимый парень, юноша, мужчина, гений, красавец, упоительный и живой человечище, ценил шутку, расскажу вам шутку. Давеча на одной конференции местного разлива некая украиноцентрическая дама вместо доклада в рамках темы мероприятия по экранной культуре, кино и театру зачем-то пришла лечить свой идеологический ковид в кабинет к гинекологу, и, несмотря на полный неадекват по профилю, сформулировала (правда, без явления себя лично: мои оппоненты всегда так делают) тему явно в мой адрес: "Что прячут современные культурологи под неокоммунизмом?" (это мой парафраз, программа есть на сайте).
Намек, очевидно, должен был повергнуть БЖ в дрожь разоблаченного агента Мадридского двора, но тайны как-то не вышло. Мы ничего под марксизмом не прячем. Иногда используем как пинцет: в ваших дырах поковыряться. Мы смело идём по жизни с Пушкиным, с русской культурой, с родной духовностью, со славянской соборностью, со священным православным Логосом и с пронзительным Голосом великого и могучего языка того, кто сказал нам:
— Что ж нового? «Ей-богу, ничего».
— Эй, не хитри: ты верно что-то знаешь.
Не стыдно ли, от друга своего,
Как от врага, ты вечно все скрываешь.
Иль ты сердит: помилуй, брат, за что?
Не будь упрям: скажи ты мне хоть слово…
«Ох! отвяжись, я знаю только то,
Что ты дурак, да это уж не ново».
\textbf{#бытьрусскиммодно}

\ifcmt
  pic https://scontent-cdg2-1.xx.fbcdn.net/v/t1.6435-9/196601934_3959167320785015_3583164931523776008_n.jpg?_nc_cat=111&ccb=1-3&_nc_sid=8bfeb9&_nc_ohc=jpmsuZqKMhMAX-bDbnB&_nc_ht=scontent-cdg2-1.xx&oh=729045b1277de0464aae1629aec767f6&oe=60E3F03D
\fi

