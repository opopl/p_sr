% vim: keymap=russian-jcukenwin
%%beginhead 
 
%%file slova.serdce
%%parent slova
 
%%url 
 
%%author 
%%author_id 
%%author_url 
 
%%tags 
%%title 
 
%%endhead 
\chapter{Сердце}
\label{sec:slova.serdce}

%%%cit
%%%cit_head
%%%cit_pic
%%%cit_text
Камнем преткновения среди украинских поклонников игры стало название на русском
языке - \enquote{\emph{Сердце} Чернобыля}. Его разработчики перевели на
английский путем транслитерации как \enquote{\emph{Heart} of Chernobyl}.
Националисты настаивают, что по законам украинского языка правильно было бы
писать Chornobyl. Они недоумевают: почему украинские разработчики позволили
себе выпустить трейлер на русском языке и допустили подобный перевод, который
может создать впечатление того, что игра разработана в России, а не Украине.  В
Twitter даже запустили специальный хэштег \verb|#ChornobylNotChernobyl|.
\enquote{При этом весь трейлер на русском. Теперь на Западе игру считают
российской, а русня бегает по сети и кричит про \enquote{отечественный}
геймдев}, - пишет пользователь Даниил Максименко.  \enquote{Потому что
Чернобыль - это Украина, и название должно быть украинское}, - считает еще один
пользователь соцсети
%%%cit_comment
%%%cit_title
\citTitle{Сталкер 2 Сердце Чернобыля - почему хейтят видеоигру украинских разработчиков}, 
Екатерина Терехова, strana.ua, 16.06.2021
%%%endcit

%%%cit
%%%cit_head
%%%cit_pic
%%%cit_text
Глотнув конскую дозу узбагоительного, шпрехен-фюрер далее в отчете указал, что
в \emph{сердце страны} Киеве, молчу уже про юго-восток, более половины учителей
позволяют себе на уроках говорить на русском языке и, о ужас, также
предпочитают русский вне школы и школьных занятий
%%%cit_comment
%%%cit_title
\citTitle{Speak на мове please}, 
Terra Incognita, zen.yandex.ru, 30.04.2021
%%%endcit

%%%cit
%%%cit_head
%%%cit_pic
\ifcmt
  pic https://img.strana.ua/img/article/3428/v-kieve-na-10_main.jpeg
	width 0.4
	caption Светофор-сердечко на Крещатике продержался недолго. Фото: Instagram
\fi
%%%cit_text
На днях на Крещатике в Киеве заметили светофор, на котором во время красного
сигнала появлялось \emph{сердце}. Неизвестный установил на нем трафарет. Сейчас
его уже убрали, а автора такого перформанса разыскивает полиция.  Об этом со
ссылкой на Центр организации движения пишет \enquote{Вечерний Киев}.  Необычный
светофор появился на перекрестке бульвара Тараса Шевченко и Крещатика. Трафарет
демонтировали 7 июля - сразу же, как только его заметили, поскольку такая форма
сигнала не соответствует стандарта
%%%cit_comment
%%%cit_title
\citTitle{В центре Киева красный сигнал светофора заменили на сердечки. Полиция ищет хулигана}, 
Игорь Кулик, kiev.strana.ua, 09.07.2021
%%%endcit


%%%cit
%%%cit_head
%%%cit_pic
\ifcmt
  pic https://strana.ua/img/forall/u/0/34/225473508_993836914713733_8330773426907809635_n-1024x681.jpg
  width 0.4
	caption Сергей Волошин в третьем ряду в центре. Фото: Facebook
\fi
%%%cit_text
Скандал с флагом и моряками.  Снимки, из-за которых разгорелся скандал, мэр
города Хорол Сергей Волошин опубликовал 25 июля. Вместе с другими бывшими
моряками (сам Волошин служил в советском морском флоте с 1988 по 1991 года) по
старой традиции он праздновал день ВМФ.  В советское время дата выпадала на
последнее воскресенье июля. Сейчас в этот день День ВМФ отмечают в России, а
Украина в рамках "декоммунизации праздников" установила другой день - первое
воскресенье июля.  Мэр написал, что у тех, "в чьем сердце навсегда живет море"
существует давняя традиция собираться в День военно-морского флота в селе
Вишняки на Полтавщине.  По его словам, члены районной организации моряков
"Шторм" собрались вместе, чтобы поприветствовать новых членов и помянуть
товарищей, которые уже покинули этот мир.  На торжественном построении они
подняли флаги ВМС Украины и военно-морской флаг СССР
%%%cit_comment
%%%cit_title
\citTitle{Носить Гитлера безопасней". Как мэру Хорола могут дать десять лет за флаг советского флота}, 
Оксана Малахова, strana.ua, 28.07.2021
%%%endcit

%%%cit
%%%cit_head
%%%cit_pic
%%%cit_text
В добру хвилю потім залунали ліси й полонини хрипливим ревом жубрових рогів.
Немов величезна хвиля, покотився голос по лісах і зворах, розбиваючися,
глухнучи, то знов подвоюючись. Пробуркалися ліси. Заскиглила каня над
верховіттям смереки; зляканий беркут, широко розмахуючи крилами, піднявся на
воздухи; захрустів звір поміж ломами, шукаючи безпечної криївки. Нараз рик
рогів утих, і ловці пустилися в дорогу горі плаєм. Усіх \emph{серця} билися живіше
ожиданиям незвісних небезпек, бою і побіди. Обережно пробирались вони рядами;
передом ряд боярський, за ним парубоцький ряд; Максим ішов попереду, пильно
надслухуючи та слідячи звірину. Цар ломів, медвідь, ще не показувався
%%%cit_comment
%%%cit_title
\citTitle{Захар Беркут}, Іван Франко
%%%endcit
