% vim: keymap=russian-jcukenwin
%%beginhead 
 
%%file books.malanjuk_evgen.malorosijstvo_1959
%%parent books.malanjuk_evgen
 
%%url 
 
%%author 
%%author_id 
%%author_url 
 
%%tags 
%%title 
 
%%endhead 


Малоросійство
Євген Маланюк

Євген МАЛАНЮК (1959)

МАЛОРОСІЙСТВО

І.

Поняття малоросійство, в тім сенсі, в якім тут ужито, не мало б обмежуватися українським лише світом.

Кожна многонаціональна держава, не виключаючи найбільш навіть
національно-упорядкованих імперій, в процесі свого історичного існування
витворювала своєрідний пересічний тип імперської людини. Згадаймо ще донедавна
пам'ятний нам тип, наприклад, австрійця, який, без особливих перешкод, міг бути
одночасно чехом чи хорватом, поляком чи русином-українцем. Політична мудрість
старого Риму, який очолював світову імперію, але ніколи не "романізував" своїх
колоній, як би ще дотлівала в устрою віденської імперії Габсбурґів. Правда,
протягом XIX ст. в бувшій Австро-Угорщині проривалися тенденції до
адміністраційного, якщо вже не німчення, то, принаймні, вживаючи зловісної
назви недавнього минулого, гляйшальтування (вирівнювання — Л. К.). Але такі
намагання були випадкові й ніколи не носили характеру чогось упланованого. Тип
австрійця, цебто — поміж— чи понаднаціональний тип громадянина й мешканця
наддунайської монархії — формувався автоматично, хоч і досить зовнішньо,
розуміється. Більш ускладнені; але все ж подібні процеси в'язалися з
виникненням типу бритійця або колоніального (а раніш і метропольного) француза.
Вже досить відмінний характер носили аналогічні явища на терені германськім,
особливо — германо-прусськім, отже вже східноєвропейськім. Але засадничо
інакший образ бачимо на властивім сході Європи.

На теренах, що офіційно називаються нині СРСР, а зазвичай, просто "Росією",
процеси не так творення, як роблення загальноімперського типу — мали характер
радикально відмінний вже хоча б тому, що державний устрій московської, згодом
петербурзької, а від сорока літ знову московської імперії — не знав, не знає і
знати не хоче жодної особовості: ані особистої, ані суспільної, ані
національної, ані навіть регіональної чи класової. Тому процес формування в
Росії типу імперської людини — "росіянина" — не був ані процесом історичним,
ані процесом взагалі (1). (Примітки див. наприкінці публікації) Це був
брутальний, масово-механічний в и —р і б, виконуваний терористично-поліційною
машиною тотально-зцентралізованої держави. І, власне тому, що був то
механістичний виріб, цебто акція видирання з живого організму підбитих і
обезвладнених націй — окремих їх шматків, — імперський тип "росіянина" не
постав. В умовах характеристичної мішанини понять "нація — етнос" і "держава —
імперія" державна машина в практиці механічно нагинала старі національні
організми під етнічний рівень московської маси з метою, розуміється, творити
єдинонеділимий нарід — русскій, російський чи совєтський, вірніш, "народ" — в
специфічно-російськім розумінні цього слова.

Результатом такої наполегливої й планової чинності державної машини було
з'явлення типу не росіянина, а лише: малороса, малополяка, малогрузина і навіть
малосибіряка.

До речі, щоб це не звучало парадоксом, пошлюся на недавнє минуле суспільства
польського, в якім тип польського малороса був безперечним соціологічним
фактом. Духовим виразом і як би сублімацією цього факту був, наприклад, Роман
Дмовський як політик та ідеолог, і ціла дмовщизна т. зв. демократії народової,
що саме своїм малоросійством отруїла в зародку новопосталу Польську державу і в
львиній долі спричинилася до її упадку (2).

Що ж таке малорос?

Це — тип національно-дефективний, скалічений психічно, духово, а — в наслідках,
часом — і расово. На нашій Батьківщині, головнім історичнім родовищі цього
людського типу, він набрав особливо патологічного і зовсім не такого простого
характеру, як, на перший погляд, здавалося б. Завдяки такому, а не іншому,
перебігові історичного процесу на нашій землі, тип малороса ставав (принаймні,
по містечках і містах) масовим, а що найгірше, традиційним. І треба припускати,
що способи, так би мовити, малоросійського виробництва вже на Москві були
розроблювані протягом не одного століття, а система тієї продукції не віднині
має під собою солідну, сказати б, наукову базу. Сучасні совєтські специ, хоч як
озброєні терором і модерними здобутками науки Павлова, безумовно дуже уважно
перелистують і вчитуються в історичні документи багатющих архівів Посольського
(й Питошного) Пріказу ХУІ-ХУІІ ст., Малоросійської Колегії XVIII ст., III
Отдєлєнія та Охранки XIX ст.

У нас фатально закорінилося майже переконання, що малорос — то, мовляв,
неосвічений, примітивний, недорозвинений українець без національної свідомості,
словом, як то кажуть, темна маса. Вистачить, мовляв, його при помочі "Просвіти"
просвітити, переконати й усвідомити — і справа полагоджена. Але кожен, хто
давав собі труду зупинятися на цій проблемі, знає, наскільки вищеподана схема
далека від дійсності.

Малоросійство, хоч явище часте і кількісне, — найменш дотикало основну нашу
національну масу — селянство (що нас не мусить особливо тішити, бо не маса
творить історію). У нас малоросійство було завжди хворобою не лише
пів-інтелігентською, але — й передовсім — інтелігентською, отже, поражало
верству, що мала виконувати роль мозкового центру нації.

І в цім — суть проблеми.

І тут одразу ж треба виключити той тип простолюду, який любив повторяти "моя
хата скраю", або при польських конскрипціях називав себе поліщуком чи тутейшим,
як при совєтських переписах записує свою національність "русскій": то є лише
мімікрія і самооборона, за якими тягнуться віки гіркого досвіду. Скажемо
коротко і забігаючи наперед: проблема українського малоросійства є однією з
найважніших, якщо не центральних проблем, безпосередньо зв'язаних з нашою
основною проблемою — проблемою державності. Що більше: це є та проблема, що
першою встане перед державними мужами вже Державної України. І ще довго, в часі
тривання й стабілізації державності, та проблема стоятиме першоплановим
завданням, а для самої державності — грізним мементо (нагадуванням).

Малоросійство-бо — наша історична хвороба (В. Липинський називав її хворобою
бездержавності), хвороба многовікова, отже, хронічна. Ні часові застрики (уколи
— ред.), ні навіть хірургія — тут не поможуть. її треба буде довго-довгі
десятиліття і з ж и в а т и.

II.

Намагаючись викладати цю тему мовою стислою, з нахилом навіть до академізму,
все ж відчуваєш труднощі при формулюванні думок, щоб вони були легко-схопні,
дохідливі. Випливає це з самої скомплікованої (складної — ред.) істоти теми.
Тому звернемось до ілюстрацій, взятих чи то з історії, чи то з літератури.

Не сягатимемо аж до праісторії. Зазначимо лише, що малоросіянству, певно,
віддавна сприяли природні багатства і щедре підсоння нашої єдиної на світі
Батьківщини. Плями майбутнього малоросійства, помітні вже в нашім Середньовіччі
("люди татарські"), яскравіють в добах литовській та козацькій. За
гетьманування Виговського і в наступній Руїні — малоросійство українське стає
вже чинником політичним і виходить на арену історії. Переяслав р. 1654 той
чинник як би залегалізував, і він, бувши спочатку чисто чуттєвою й
психологічною вадою, об'єктивно дає пізніш очевидний параліч
національно-державній волі, а дедалі — агентуру й п'яту колону Москви.
Брюховецький — з одного боку, Тетеря — з другого: ось два обличчя малоросійства
за Руїни. Але ще Мартин Пушкар, полковник полтавський, встає зловісним символом
малоросійства похмельниччини, символом знівечення перемоги під Конотопом,
символом, що веде хід нашої історії аж до полтавської катастрофи, бо
Кочубеївщина — то був плід довгих десятиліть. Якщо колись з'явиться історія
Українського Малоросійства, то автор її саме Мартина Пушкаря, правдоподібно,
проголосить батьком малоросійства. Бо, повторюю, всупереч популярній у нас
думці, малоросійство — то не москвофільство і не ще яке-небудь фільство. То —
неміч, хвороба, каліцтво внутрішньонаціональне. Це — національне пораженство.
Це, кажучи московською урядовою мовою XVII століття, — шатость чєркасская, а
кажучи мовою такого експерта, як цариця Катерина Друга, це — самоотвер-женность
малороссійская. Отже, є то логічне степенування: хитливість, зрадливість, зрада
і агентурнісп?. Аж до часів наступних і нам найближчих. Москвофільство чи інше
фільство (їх було кілька в нашій історії) — то є можливий напрям нашої
національної політики, і в цім сенсі був "переяславський" момент москвофільства
в політиці Богдана Великого, як москвофільськими були цілі десятиліття
національної політики великого Івана Мазепи. Як туркофільство Петра Дорошенка.
Як змушене полонофільство Виговського чи — на наших очах — Петлюри.

Всі ці приклади — то політика чи тактика. Але малоросійство — це не політика і
навіть не тактика, лише завжди апріорна і тотальна капітуляція. Капітуляція ще
перед боєм. Москвофільство (як всяке інше фільство), перепокладає власну
контрольовану волю, отже, теоретично кажучи, може міститись в границях
національної чи, тим більше, державної політики, але малоросійство, як яскравий
прояв паралічу політичної волі і думки, завжди є поза межами якої-будь
раціональної політики — взагалі. Діячі Центральної Ради малоросами не були.
Вони були здебільше тими, що їх окреслюється виразом "свідомі українці" (р.
1918 досить двозначно казали "переконаний українець"). Але майбутній історик не
зможе політику Центральної Ради пояснити інакше, як присутністю в ній
політичного малоросійства, яким були люди того покоління отруєні: брак
найелементарнішого національного інстинкту і параліч політичної волі — були
наявні. З коротеньких спогадів генерала Врангеля, який в імперській гвардії був
підлеглим Скоропадського, видно, що гетьман роздумував навіть над історіософією
України (3). Але ці роздуми відбувалися літом р. 1918 — отже, були і спізнені,
й абстрактні: історію треба було тоді вже творити, владу, якою б обмеженою вона
тоді не була, — здійснювати щохвилини, бо час був рахований. Гетьман Павло
Скоропадський під багатьма оглядами стояв неспівмірно вище від більшості
лідерів Центральної Ради, але й він був і сином доби, і ровесником покоління.
Якщо не комплекс малоросійства, то напевно "комплекс Гоголя" — тяжів і над ним.
Павло Скоропадський залишив нам класичний приклад українського політичного
гамлетизму, який, до речі, тісно зв'язаний з традиційним малоросіанством. І в
цім була трагедія і Батьківщини, і її нещасливого сина, який на чужині
залишився їй вірним аж до смерті.

В нормальній, незмалоросійщеній психіці кожного сина свого народу існують
своєрідні "умовні рефлекси" національного інстинкту: чорне — біле, добре — зле,
вірне — невірне, чисте — нечисте, Боже — диявольське. В малоросійстві ці
рефлекси пригасають і слабнуть, часом аж до повного їх занику. В такім стані
сама-но праця інтелекту не помагає, бо буде завжди с п і з н е н а. Навіть
розроблена, зафіксована і по архівах (зазвичай нечитаною) переховувана
національно-політична мисль не може винагородити того інстинктовного розуму, що
його називаємо часом природним, часом мужицьким, і який тісно зв'язаний з волею
і характером. Цього "природженого" розуму не може замінити жодна школа, жоден
титул чи диплом. Історичне малоросійство або той розум редукувало до розмірів
"мужицької арихметики", або його наркотизувало різними міфами (спільна віра,
спільний цар, спільний соціалізм чи брак географічних кордонів), або — просто
вижирало, як мікроб вижирає живу тканину організму.

Що ж таке малоросійство? Це також затьмарення, ослаблення і — з часом — заник
історичної пам'яті. Тому і колишній Петербург, і теперішня Москва,
розпоряджаючи з централізованим шкільництвом, таку велику вагу надавали й
надають науці Історії, яка, в сполученні з відповідно підібраною літературою (а
т. зв. російська література є першорядний чинник деморалізації, про що
попереджував, але безуспішно, ще Володимир Антонович) — забиває історичну
пам'ять української дитини з першим днем вступу її до школи. Придивімось, що
робиться в сучасній совєтській історіографії! Яку до дрібниць продуману
програму здеформованої й зденатурованої Історії там розроблено! Бо то —
найважливіший відділ лабораторії малоросіанства. Малоросійсіво, як показує
досвід, одночасно плекається Також систематичним впорскуванням комплексу
меншовартості ("ніколи не мали держави", "темне селянство", "глупий хохол" і т.
п.), насмішкуватого відношення до національних вартостей і святощів. Це —
систематичне висміювання, анекдотизування й глузування зі звичаїв, обичаїв,
обрядів, національної етики, мови, літератури, з ознак національного стилю,
реалізації якого ставляться систематичні, планові й терором підперті перешкоди.
А коли пісню чи танець висміяти не вдається, тоді їх вульгаризується й
примітивізується ("пісні народів СССР") так, щоб гопак непомітно переходив в
камарінскій, а бандура — через різні "капели" — в балалайку чи гармонь. Коли ж
в області науки чи мистецтва постає твір українського національного духу
вартості бездискусійної і самопереконливої, тоді приходить просто реквізиція чи
"соціалізація" і твір проголошується "нашим" ("русекім" чи — тепер —
"совєтським").

Для сучасників Шевченка національність нашого великого математика Михайла
Остроградського була річчю очевидною. Сьогодні це "русекій учоний" — вже для
цілого світу. Ми знаємо, що один з фундаторів науки про міцність матеріалів та
будівельної механіки — Степан Тимошенко — є син нашого народу і найстарший член
Наукового товариства ім. Шевченка, але для цілого світу він сьогодні навіть не
американець, а просто "русекій", і його, навіть перекладені, підручники в СССР
давно вже "націоналізовані" для генія "совєтського народу".

Ось свіжий приклад: видатний наш мистець Олександр Довженко, який в умовинах
власної держави виріс би на світового генія, пильнував аж до своєї смерті, щоб
в кінці кожного його, Москвою соціалізованого, твору стояло: переклад з
української. Але не вспів він склепити очі, як його теста-ментарну "Зачаровану
Десну" видано було в Москві велетенським накладом по-російському, а одночасно —
англійський переклад з зазначенням, що він "совєтський", отже — для
"англомовних" народів — "русскій" письменник.

Уявім собі, що диявол помагатиме так само далі — і тоді, за 20-30 літ, Довженко
буде так само "русскім", як Тимошенко і Богомолець, як Бортнянський,
Боровиковський, Гоголь, Мечников, Куїнджі, Самокиш чи (до речі — з львів'ян)
Айвазовський...

Київську Всеукраїнську Академію наук перетворено на провінційну філію
московської з публікаціями "на общепонятном". Славна Київська Академія мистецтв
обернулася на провінційний "художній" Інститут, а її фундатор — геніальний
графік Юрій Нарбут — просто викреслений з історії: його пам'ять зліквідовано
навіть у "всесоюзній" скалі. "Нєт, нє било і бить не может..." Те саме з
музикою, оперою. Один з авангардових театрів XX ст. — театр Курбаса-Куліша —
знищено до кореня і на його місце привернуто побутовий, фактично малоросійський
театр, який, одначе, порівняти не можна-з класичним побутовим нашим театром
Кропивниць-кого — Карпенка-Карого: театр "на украінском язикє" в УССР опинився
на рівні — увічненого ще Винниченком — Гаркуна-Задунайського.

Дуже прикметний недавній випадок з поеткою Ліною Костенко. По виданні тільки
двох книжечок поезій — вона опинилася з кляпом в роті. Не з огляду на тематику
(про кохання й соловейка така В. Ткаченко воропає собі без перешкод), хоч Л.
Костенко мала необережність писати про... море, забувши, що тема моря для
українців була заборонена ще в кінці 20-х рр. ("ніззя"). Ні, справа була не в
тематиці, а в занадто певнім тоні, занадто суверенній інтонації і занадто
яскравій літературній культурі, яка — ретроспективно — виявляла рівень 20-х
років, неокласиків, Плужника ба й... Яновського, словом — зраджувала непер е р
в н и й процес... А, на біду, — поетка справжня та ще й з власним стилем. Це й
припечатало її долю. Вона фактично вже задушена, не вспівши навіть заквітнути.

Випадок з Ліною Костенко, може, найяскравіше показує справді сатанинську чуйність совєтського апарату малоросизації.

Побіжно відзначені явища дають мірило сучасної совєтської малоросизації нашої
культури і показують обсяг продукції всеохоплюючого малоросійства на нашій
Батьківщині, де, до речі, вже офіційно не фігурує нарід український, лише від
двох десятків літ просто "народ України", отже, не нація, а населення,
мешканці, ріоріе, або, як тепер кажуть, жителі цієї "республіканської" колонії
"совєтського государства".

IV.

Тільки на тлі імлисто-хиткого, зрадливо-безформного, невиразно-обопільного і часом просто юдиного малоросійства можна відчути й зрозуміти, чому саме ім'я гетьмана Мазепи прошиває ворога, як жагуча стріла, чому це ім'я змушує тремтіти його, як ту євангельську осику, на якій Юда повісився, чому сама згадка про Мазепу кидає ворога в холодний піт.

Історія наша згадує про річки крові в руйнованім, ґвалтованім і ганьбленім р. 1709 Батурині. Історія подає нам, як московським канчуком було зігнано наших єпископів до глухівського собору, щоб там побілілими зі смертного жаху українськими устами проголосити "Івашці Мазепі" — фундаторові храмів! — блюзнірчу й страшну, істинно диявольську "анатему". Це був прилюдний ґвалт наїзника-варвара, перш за все, над найвищими святощами нації — нашою Церквою, нашою Вірою Благочестивою... І від часів коронованого ката

Петра, якого ліпша частина власного народу, з сином-престолонаслідником на чолі, справедливо мала за антихриста, — протягом двох з половиною століть іде різними способами і з різних сторін оббріхування, оплюгавлювання, оганьблювання, викорчовування й викорінювання найменшого сліду Мазепи й мазепинської доби. Ми досі не можемо знайти навіть вірнішого портрету того, хто, напевно, був за життя портретований десятки разів. 0, допоки буде Росія Росією і Москва Москвою, навіть історіографічної "реабілітації" Мазепи ніколи не наступить, хоч би сталося ще кільканадцять октябрських чи февральських революцій! Ворог, якого національний інстинкт — при всім варварстві й дикунстві — був, є і буде найбільш живучий і безпомилковий, ворог, який на жодне малоросіянство, протягом століть своєї історії, ніколи не хворів, — на пункті Мазепи і мазепинства є тотально-непримиримий. І має цілковиту рацію. Мазепинство бо й є яскравою протилежністю, яскравим запереченням, нещадним демаскуванням і найрадикальнішим ліком саме на малоросійство. Бо що ж є мазепинство, як не чинна свідомість Нації та інстинктовно зв'язана з тією свідомістю політична й мілітарна воля Нацією бути? Навіть за ціну Батурина чи Полтави.

В капітальній повісті Б. Антоненка-Давидовича "Смерть" (в ній чомусь тепер почали шукати порнографії, а не проблеми малоросійства, якій вона в цілості присвячена) є яскрава сцена. Окупаційний комісар — політично свідомий українець, отже, змушений грати роль самоотверженого малороса, при ревізуванні сільської школи натрапляє на портрет Мазепи, маскований сусідством портрета Драгоманова. Буря почувань, викликана образом гетьмана, мусить бути з місця згашена, понадто в присутності учителя школи, типового "просвітянина" і "щирого українця", але політичного сліпця, отже — в істоті своїй — натурального малороса.

Сцена, розуміється, кінчиться голосним наказом: "В радянській школі не місце одвертій контрреволюції!"

А скільки ж то попотів тероризований III Отдєлєнієм віртуоз віршу Олександр Пушкін, щоб протягом рахованих тижнів виконати замовлену царем "Полтаву" так, щоб в ній героя байронівської поеми, за всіма канонами соціалістичного реалізму, представити "злодєєм", про якого

"Немногим, может быть, известно, Что он не ведает святыни, Что он не знает благостыни. Что он не любит ничего, Что кровь готов он лить, как воду, Что презирает он свободу, Что нет отчизны для него."

А пригадується, як ці блюзнірчо-страшні й сатанинсько-брехливі рядки за спеціальними інструкціями міністерства освіти мусили учні середніх шкіл царської Роси (Росії — ред.) вивчати обов'язково напам'ять і ще півдитячими устами рецитувати вголос. Як, додамо, ці самі рядки мусить вивчати й рецитувати наша молодь нині в школах т. зв. Радянської України. А написані ж ці рядки були не яким-будь Сурковим чи Еренбурґом, а таки визначним класиком і, можна сказати, Моцартом російської, ним же створеної й розвиненої, літературної мови. Один опис полтавської баталії в поемі Пушкіна вартий десятикратної Сталінської премії!

Так працювала машина малоросійства за царів, зеленцем нищучи й отруюючи українську душу. І працює, може, більш примітивно, але й більш брутально та одверто, тепер — з перспективою "освоєнія цєліни" в Азії і з тінню нагана на стіні.

Як єдиним радикальним ліком на хворобу малоросійства є державність, так упадок державності, смерк державницької ідеї і всяка "руїна", від Руїни XVII ст. почавши, були і є тим ґрунтом, на якім малоросійство виростало, квітло і давало плоди.

Сучасну Руїну, як ґрунт під малоросійство, використовує ворог планово, безоглядно і приспішено, бо "время ґарячее". І було непростимою, злочинною наївністю недооцінювати цей факт, або, що гірше, збувати його псевдопатріотичною фразою чи ледачою вірою в автоматизм т. зв. історичного прогресу.

Тут не місце давати якусь дешеву рецептуру. Тим більше, що рецептура та спроваджується зазвичай до того традиційного, спростаченого "просвітянства", з яким боролися нечисленні представники політичної мислі, люди суверенного національного розуму чи живого національного інстинкту — ті, що спромоглися не зважати на диригентську паличку ззовні, а дерзали, як писав колись Хвильовий, у тій області самостійно. Напружене творення Духової Суверенності — ось рецепт, що був, є і буде найбільш трудний, але й найбільш істотний і всеобіймаючий. Цей рецепт, до речі, виключає — якнайгостріше — імітацію, декламацію, патріотичну позу, барокове "здаватися, а не бути", як і всіляке "погрожування пальцем в чоботі".

Коли ми зупинимося на області національного інстинкту, то, враховуючи всі здобутки науки (з Павловим включно), мусимо признати, що плекання того інстинкту є питанням двох, вельми коштовних чинників: часу і обставин. Це — родина, рамки національного (не етнографічного!) стилю, магія національного обряду, атмосфера національної етики і національної естетики. А в першу чергу — національне по-

ступовання й ділання, бо в області чуття — віра без діл мертва є, як каже Святе Письмо.

Коли ж ми зупинимося над областю національного інтелекту, то автоматично приходимо до поняття знання, ц. т. досліду, студій, висновків і сформулювань. Наше-бо знання не має носити характеру абстрактного, а мусить бути направлене, остаточно, на одержання знаття — отого, власне, приповідкового знаття, якого так часто бракувало землякам, коли-то, чухаючи потилицю, вони post faktum нарікали: "Якби ж то було знаття!.."

Оце-то знаття і є те місце психіки, де національне чуття — в ідеалі — гармонійно сполучується з національним розумом в їх синтетичний вислід: національну волю. Говоримо: в ідеалі. В дійсності, як вчить нас недавній історичний досвід, ці дві основні психологічні категорії, наслідком малоросійського паралічу (і малоросійського розкладу) давали в області чуття — отаманщину (і махнівщину), а в області розуму — мертвий, отже завжди спізнений формалізм, ялову "принципіальність" (а не творчу засадничість) і ті чи інші "дискусії" в скалі від "високого рівня" аж до Гоголевої повісті про те, як посварився Іван Іванович з Іваном Ничипорови-чем...

Саме усвідомлення собі комплексу малоросійства — було б вже значним кроком вперед, так само, як поставлення діагнозу є початком лікування.

В своїй величезній мистецькій та інтелектуальній творчості саме Микола Гоголь, тепер канонізований "русскій пі-сатєль" і як би прапор політичного малоросійства, дав не-перевершений досі матеріал для студій над розкладом національної психіки і переходом її в стан малоросійського гниття. Життєвий шлях Гоголя і його психіка — українця перелому епох — зробили те, що він, в зударенні з Петербургом, спалахнув був майже революційним націоналізмом (листи до М. Максимовича) і в р. 1836 фактично емігрував, — наслідком надломлення характеру та данайської "помочі" і недвозначної "опіки" уряду — почав "пропагандово" плутати Русь і Росію. А в т. І "Мертвих душ" несподівано посадив нашу, історичну Русь на московську "тройку" з москалем-"ямщіком".

Так Гоголь фатально (і, певно, несподівано для самого себе) стався фундатором міфу Руси-Россії, а — політично — дав під малоросійство своєрідну, хоч і дуже двозначну "ідеологію". Та тут не місце на розвинення цієї, важливої для проблеми малоросійства, але бічної теми.

Гоголь, син своєї доби і свого душевно вже напівмертвого суспільства, подав його жахливу (в "Мертвих душах") панораму — сміючись, хоч "сміючись крізь сльози". У нього ще півспівчутлива, півпрезирлива поблажливість до пізніх нащадків героїчної Хмельниччини, хоч, одночасно, він, ціле життя працюючи над "Тарасом Бульбою", старається поставити перед мертвими душами сучасників постаті й чини козацької Іліади, в якій брав участь його предок — Остап Гоголь, полковник брацлавський, славний хмельничанин і один з найнезламніших мужів доби Руїни.

Але вже тільки трохи молодший сучасник Гоголя, автор вимовного посланія "М. Гоголю" ("ти смієшся, а я плачу") — Шевченко тієї поблажливості не має. Обурення, погорда, презирство, пекучий сарказм: "славних прадідів великих правнуки погані", "раби з кокардою на лобі, лакеї в золотій оздобі", "не заріже батько сина... викохає та й продасть в різницю — москалеві" — це вже портрети закінчених малоросів XIX ст.

Шевченко перший вжив це слово, за його часів ще зовсім необразливе (ще у 70-80 pp. його писали з великої букви: "Малоросіяне"), саме як слово ганьби й погорди. "А на Україну не поїду, цур їй, там сама Малоросія", — писав він в однім листі. Так Шевченко поставив діагноз і сформулював національне каліцтво, якому пізніш Іван Франко дав максимально згущений і, фактично, властивий вираз в "На ріках вавилонських":

І хоч душу манить часом волі приваб,

Але кров моя — раб! Але мозок мій — раб.

Ця формула — належить підкреслити: всеукраїнська і "соборна" — дана була Франком на самім початку нашого століття. Але це не перешкодило віками культивованому рутено-малоросійству відіграти свою фатальну ролю у вирішальному трьохріччі 1917, 1918 і 1919 років, а головне, у болюче-змарнованих місяцях весни року 1917.

* * *

Оскільки Франко ніколи не був, сказати б, загальнодоступним і легким до вульгаризації, оскільки Шевченко в короткім часі "потрапивши під стріхи" (побожне бажання Міцкевича), був завжди предметом особливих заходів і турбот Росії, не менш від Мазепи. Від "лівих", як критик В. Бєлінський, і аж до генералів тайної поліції — і "громадськість", і уряд робили все, щоб його спочатку висміяти, а коли це не вдалось, то зденарутувати і — при помочі цензури — зредукувати до "крестьянского поета", співця "селянської недолі" і т. п. В цих заходах свою помічну і не малу роль відограло рідне малоросійство також. Малоросійство-бо було в тій акції — малоросизації Шевченка — заінтересоване безпосередньо.

. Як вже згадувалося, комплекс малоросійства є складний і заплутаний. Він — в своїй змінливості — має багато облич і сторін. Він часто є замаскований, особливо в останніх десятиліттях, коли, буваючи знаряддям в чужих руках, він зазвичай маскується гопако-шароварництвом, відповідно спростаченою мовою ("поддєлуєця под мужицький разговор", як каже один з персонажів Винниченка), ховається за лжепатріотичною віршографією й етнографічною патріоткою взагалі. Довгі десятиліття малоросійство, при допомозі чужої поліції, "пристосовувало" Шевченка. І не без успіху. Лідер воюючого київського малоросійства, славнозвісний Васілій Віталієвіч Шульґін — не без гордості стверджував ще перед р. 1917, що існує, мовляв, два Шевченки: "наш", як він казав — "богданівський", і "їх" — мазепинський. Термінологія, як бачимо, досить довільна, але тим більш характеристична.

Так. За царів малороси препарували ("закобзарювали") Шевченка досить наполегливо, але й досить "кустарно", в порівнянні з тією індустріалізованою препарацію, що її доконано протягом сорока літ в УССР. Знечулювалося у читача саму шевченківську емоцію, самий інстинкт шевченківства для того, щоб — у відповідний момент — вирізати Шевченка з національної психіки майже хірургічним способом. А якщо й ні, то, принаймні, зробити з Шевченка провансальського Ф. Містраля або шкотського Р. Бернса (4). Наскільки не можна легковажити цієї акції малоросизації Шевченка, свідчить один епізод з порівняно недавнього минулого.

Покійний Максим Славинський, приятель Лесі Українки, довголітній співредактор західницького петербурзького місячника "Вестник Европи" і наш дипломат в часах державності, десь в середині 20-х рр. щиро признався був у розмові, що страшне пророцтво Шевченка

"Та не однаково мені,

Як Україну злії люде

Присплять, лукаві, і в огні її,

окраденую, збудять," — було для нього довший час незрозуміле. Що то значить "присплять"? І чому "збудять в огні"? І чому "окраденую"? Як це можна "окрасти" цілу країну, цілий народ? Славинський згадував, як він (та й не він один) ту "неясну" шевченківську строфу клав на карб "слабої обробленості вірша", "малої освіти", мовляв, "самоука" і тому подібних інтелігентських забобонів здрагоманізованого й звинниченкізованого покоління. І аж, як казав він, ось тепер, по всім, що було протягом 1917-20 рр., він зрозумів, яке прозріння і яка осторога містилася в тій "неясній" і "необробленій" строфі.

Цей епізод згадався, щоб додати ще одне визначення малоросійства: воно є еквівалентом нашої окраденості. В сорок першу річницю проголошення державності.

Примітки:

1) Це історично-спізнене слово тепер все частіше пускають в обіг керівники т. зв. російської еміграції, як би попереджаючи, поки що масковані совєтською термінологією,
інтенції керівників совєтської імперії. Історично — це слово мало свій сенс в заранні імперії, в часах Петра І й Катерини II, і в тім сенсі — імперсько-державнім, а не національнім — вживалося протягом XVIII ст. Допіру в столітті XIX урядові чинники безуспішно намагаються надати йому сенс етнічно-національний, згодом остаточно перейшовши до явного
фальсифікату: "русскій". (Тут і далі примітки Є. Маланюка).

2) Замордування першого президента відновленої Польщі — Габрієля Нарутовича (інженера-гідравліка і славного
швейцарського професора) — допіру тепер частина польського суспільства починає собі належно усвідомлювати.
Убійник — Еліґіуш Нєвядомський, людина релігійна, маляр,
абсольвент, щоправда, Петербурзької Академії мистецтв, —
доконав цього з мотивів патріотичних, як партійний ендек
(націонал-демократ). Пілсудський, з властивою йому інтуїцією, досить точно
визначив був тоді ж, що то була "куля зі Сходу".

3) Подаю, на жаль, з пам'яті такий діалог:

Врангель: Чи Україна — це поважно?

Гетьман: Цілком. Я от лише думаю, з ким їй бути: зі Сходом
чи Заходом.

Цей епізод, поданий людиною честі і шведським бароном
з походження, до речі, спростовує знане "свідоцтво" ген. Денікіна, що гетьман, мовляв, "своєчасно склав би Україну до стіп царя" і тому подібні "запобігавчі" інсинуації.

4) Виразним підтвердженням цього є остання "радянська" публікація під назвою "Тарас Григорович (так!) Шевченко — літературний портрет" двох авторів: О. Білецького
та О. Дейча. Висновок з цієї ніби академічно написаної книжечки такий: Шевченко був, щоправда, досить високого (але, розуміється, нижчого від Пушкіна та Лєрмонтова) рівня, але виразно "общерусский", що писав двома мовами і безмежно любив спільне "отечество", Пушкіна та "велику російську літературу й культуру" (так!), як "рідну". Треба співчувати О. Білецькому, що дав своє авторитетне
ім'я цьому майже жандармському елаборатові (трактатові — ред.). Але не можна недооцінювати можливого впливу книжечки на попередньо скомсомолізовавого читача.

