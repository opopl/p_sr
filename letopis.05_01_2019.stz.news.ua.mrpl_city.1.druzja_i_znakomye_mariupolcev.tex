% vim: keymap=russian-jcukenwin
%%beginhead 
 
%%file 05_01_2019.stz.news.ua.mrpl_city.1.druzja_i_znakomye_mariupolcev
%%parent 05_01_2019
 
%%url https://mrpl.city/blogs/view/druzya-i-znakomye-mariupoltsev
 
%%author_id burov_sergij.mariupol,news.ua.mrpl_city
%%date 
 
%%tags 
%%title Друзья и знакомые мариупольцев
 
%%endhead 
 
\subsection{Друзья и знакомые мариупольцев}
\label{sec:05_01_2019.stz.news.ua.mrpl_city.1.druzja_i_znakomye_mariupolcev}
 
\Purl{https://mrpl.city/blogs/view/druzya-i-znakomye-mariupoltsev}
\ifcmt
 author_begin
   author_id burov_sergij.mariupol,news.ua.mrpl_city
 author_end
\fi


Содержание этого очерка не касается наших земляков, которые в разное время
покинули малую родину, обосновались в столицах, достигли выдающихся успехов в
искусстве, науках или общественной деятельности и, будучи сами людьми
неординарными, естественно, жили и творили в окружении знаменитостей. Речь
пойдет о тех жителях Мариуполя, которым посчастливилось учиться, работать бок о
бок или, находясь в родных пенатах, переписываться, а то и поддерживать
дружественные отношения с людьми, чьи имена занесены в энциклопедии в знак
признания их выдающихся заслуг.

\vspace{0.5cm}
\begin{minipage}{0.9\textwidth}
\textbf{Читайте также:}

\href{https://mrpl.city/news/view/v-mariupole-razrabotali-pyatikilometrovyj-marshrut-po-istoricheskim-mestam-foto}{%
В Мариуполе разработали пятикилометровый маршрут по историческим местам, Роман Катріч, mrpl.city, 04.01.2019}
\end{minipage}
\vspace{0.5cm}

В этом ряду, безусловно, одно из первых мест занимает Георгий Георгиевич
Псалти, выпускник Мариупольской Александровской мужской гимназии, получивший
образование во Франции, чьими трудами много сделано для озеленения нашего
города. Ему довелось в конце 1896 года познакомиться и подружиться с
малоизвестным в то время журналистом \textbf{Александром Серафимовичем Поповым}
(1863-1949), прибывшим в наш город для работы в корреспондентском пункте газеты
\enquote{Приазовский край}. А. С. Попов в январе 1898 года покинул Мариуполь, стал
писателем, вошедшим в литературу под псевдонимом Александр Серафимович, а
приятельские отношения и обмен письмами между ним и Г. Г. Псалти сохранялись на
протяжении нескольких десятилетий.

\ii{05_01_2019.stz.news.ua.mrpl_city.1.druzja_i_znakomye_mariupolcev.pic.4.serafimovich}

Жил в довоенные годы в нашем городе священник отец Михаил Арнаутов. Кроме
службы в церкви, он был любителем-садоводом. Его внучка, Маргарита Михайловна
Астахова, вспоминала: \emph{\enquote{Дедушка вел переписку с \textbf{Иваном Владимировичем Мичуриным}
(1855-1935), и по его совету сделал прививку на абрикос. Получился новый сорт,
названный им \enquote{Ивановкой}. Может быть, в честь Мичурина – знаменитого в то время
селек\hyp{}ционера-самоучки}}.

\ii{05_01_2019.stz.news.ua.mrpl_city.1.druzja_i_znakomye_mariupolcev.pic.7.michurin}

Коль скоро зашла речь об отце Михаиле, стоит, наверное, вспомнить и его сына
Виктора Арнаутова – художника-монументалиста, профессора Станфордского
университета, жизнь которого в разные ее периоды была тесно связана с
Мариуполем. Открыв книгу Виктора Михайловича \enquote{Жизнь заново}, узнаем о
том, как он в середине двадцатых годов ХХ века работал со знаменитым
мексиканским художником \textbf{Диего Риверой} (1886 — 24 ноября 1957) – мастером
мирового уровня.

\ii{05_01_2019.stz.news.ua.mrpl_city.1.druzja_i_znakomye_mariupolcev.pic.6.rivera}

\textbf{Читайте также:} 

\href{https://mrpl.city/news/view/mariupoltsyam-na-dovgi-novorichni-svyata-najkrashhi-ukrainski-knigi-2018-go-roku}{%
Маріупольцям на довгі новорічні свята: найкращі українські книги 2018 року, mrpl.city, 29.12.2018}

\ii{05_01_2019.stz.news.ua.mrpl_city.1.druzja_i_znakomye_mariupolcev.pic.9.makarenko}

В начале тридцатых годов ХХ века на площадке строящегося завода \enquote{Азовсталь}
местный археолог-любитель Григорий Федорович Кравец обратил внимание на
необычные красные пятна на поверхности грунта. Он сообщил об этом научным
сотрудникам Мариупольского краеведческого музея. Те обратились за помощью в
Киев. Оттуда приехал знаменитый археолог, профессор \textbf{Николай Емельянович
Макаренко}. Под его руководством начались раскопки, в результате которых был
открыт неолитический могильник, как оказалось впоследствии, мирового значения.
Общее дело сдружило Г. Ф. Кравца и киевского профессора. Пытливый любитель и
маститый ученый обменивались письмами. В фондах музея хранятся оттиски научных
работ Николая Емельяновича с теплыми дарственными надписями, адресованные
Кравцу.

\ii{05_01_2019.stz.news.ua.mrpl_city.1.druzja_i_znakomye_mariupolcev.pic.8.strazhesko}

Многие жители Мариуполя обязаны здоровьем, а то и жизнью, замечательному врачу
Ивану Илларионовичу Саенко. Он окончил Киевский медицинский институт, где его
учителем был профессор \textbf{Н. Д. Стражеско}. Николай Дмитриевич позже стал Героем
Социалистического Труда, действительным членом нескольких академий. Его имя
известно медикам всего мира. Именно по настоянию Николая Дмитриевича молодой
врач Саенко был оставлен в аспирантуре. Но обстоятельства сложились так, что
Ивану Илларионовичу пришлось прервать свою научную работу и покинуть не только
институт, но и Киев. Прощаясь с учеником, профессор Стражеско (1876—1952)
подарил ему свой стетоскоп – деревянную трубочку для прослушивания больных. С
этим предметом Иван Илларионович не расставался до самой смерти.

\ii{05_01_2019.stz.news.ua.mrpl_city.1.druzja_i_znakomye_mariupolcev.pic.1.krymskii}

Многие годы жизни востоковеда, переводчика и экономиста Евгения Филипповича
Ребрика связаны с Мариуполем. В наш город его привезли в раннем детстве. Здесь
он учился в Александровской мужской гимназии, после окончания которой уехал в
Москву, где поступил в знаменитый Лазаревский институт восточных языков. Там
среди его наставников был профессор \textbf{Агатангел Ефимович Крымский} (1871 – 1942),
известный востоковед, славист, а кроме того, фольклорист, этнограф, полиглот,
оригинальный поэт и самобытный прозаик. Именно по рекомендации А. Е. Крымского,
Евгения Ребрика после окончания в 1911 году направили в Персию - в Тегеран -
для работы в учебно-ссудный банк. Только в 1927 году Е. Ф. Ребрик с семьей
вернулся в Мариуполь. В 1930 году его приглашают в Харьковский техникум
востоковедения, где он не только преподает, но и много занимается переводами на
украинский язык произведений основоположника таджикской литературы Айни. Эти
переводы до сих пор остаются самыми лучшими. Увы, техникум не предоставил
жилья, и ему пришлось возвратиться в Мариуполь. Он работал днем банковским
служащим, а все свободное время посвящал переводам с восточных языков,
поддерживая переписку со многими авторами и учеными-лингвистами.

\textbf{Читайте также:} 

\href{https://mrpl.city/blogs/view/preobrazovaniya-goroda-marii-ili-chem-zapomnilsya-mariupoltsam-2018-god-1}{%
\enquote{Преобразования города Марии}, или Чем запомнился мариупольцам 2018 год, Константин Карцев, mrpl.city, 30.12.2018}

\ii{05_01_2019.stz.news.ua.mrpl_city.1.druzja_i_znakomye_mariupolcev.pic.3.beketov}

Архитектор-художник Николай Иосифович Никаро-Карпенко приехал на постоянное
жительство в наш город в 1947-м, где жил до своей кончины в 1963 году. Но еще
до войны бывал у нас. По его проектам было построены многоэтажные дома в
Ильичевском районе. Став жителем нашего города, он участвовал в восстановлении,
реконструкции и новом строительстве жилых домов и общественных зданий. В свое
время Николай Иосифович несколько лет работал в Харькове под руководством
выдающегося украинского архитектора, профессора Харьковского
инженерно-строительного института \textbf{Алексея Николаевича Бекетова} (1862-1941),
который называл среди лучших своих учеников и последователей и Никаро-Карпенко. 

\ii{05_01_2019.stz.news.ua.mrpl_city.1.druzja_i_znakomye_mariupolcev.pic.5.filatov}

Большой известностью и уважением у мариупольцев пользовался врач-офтальмолог
Василий Моисеевич Пашковский. Полгорода носили очки, подобранные им. Доктор
Пашковский делал тончайшие операции на глазах, тем самым возвращая людям
способность видеть. Еще в пятидесятых годах ХХ века ездил в Одессу на курсы
усовершенствования врачей в знаменитый институт, основанный и руководимый тогда
академиком, Героем Социалистического Труда и лауреатом Государственной премии
\textbf{Владимиром Петровичем Филатовым} (1875-1956). Пашковскому посчастливилось не
только слушать великого хирурга, но и ассистировать ему на операциях. Академику
приглянулся трудолюбивый, одаренный и пытливый мариупольский врач, и, когда
позже Василий Моисеевич обращался за советом к В. П. Филатову, тот никогда ему
не отказывал в помощи.

\ii{05_01_2019.stz.news.ua.mrpl_city.1.druzja_i_znakomye_mariupolcev.pic.2.rajkin}

В 1959 году на базе Кировоградского и Енакиевского театров был создан Донецкий
областной русский драматический театр с постоянным базированием в нашем городе.
Пока в центральном сквере достраивалось для него здание, актеры играли
спектакли в летнем театре в Городском саду. Местные театралы сразу обратили
внимание на талантливого актера Алексея Чернова – высокого, статного, с
благородными чертами лица. Алексей Михайлович окончил Ленинградский театральный
институт, где среди его сокурсников и друзей был и ставший впоследствии
знаменитым \textbf{Аркадий Райкин} (1911-1987).  По воле сложившихся обстоятельств
жизненные пути этих людей разошлись еще в довоенные годы. А встретились вновь в
шестидесятых годах, когда Аркадий Райкин был в зените славы. Райкин принял
своего однокашника тепло и сердечно. После этого они стали обмениваться
письмами и поздравительными открытками. Некоторые из них с автографами А. И.
Райкина хранились в личном архиве Тамары Александровны Щекатуровой – вдовы А. М.
Чернова, в прошлом актрисы нашего театра.

\textbf{Читайте также:} 

\href{https://archive.org/details/04_08_2018.sergij_burov.mrpl_city.v_chest_kogo_nazvan_nash_gorod}{%
В честь кого назван наш город?, Сергей Буров, mrpl.city, 04.08.2018}

Конечно, это лишь малая толика наших земляков, которым довелось общаться со
знаменитостями, но и этого, пожалуй, достаточно, чтобы признать: Мариуполь не
был такой уж глухой провинцией, где жители его были сосредоточены на своих
личных житейских проблемах, оторванные от цивилизованного мира. Многие из них
дружили, встречались, учились, переписывались с людьми известными и даже
знаменитыми, знакомясь с новинками литературы и искусства, так сказать, из
первых рук. И общение с неординарными личностями обогащает, а порой определяет
дальнейшую судьбу человека.
