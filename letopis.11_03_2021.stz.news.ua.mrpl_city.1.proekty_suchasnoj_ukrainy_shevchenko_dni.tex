% vim: keymap=russian-jcukenwin
%%beginhead 
 
%%file 11_03_2021.stz.news.ua.mrpl_city.1.proekty_suchasnoj_ukrainy_shevchenko_dni
%%parent 11_03_2021
 
%%url https://mrpl.city/blogs/view/proekti-suchasnoi-ukraini-do-shevchenkivskih-dniv
 
%%author_id demidko_olga.mariupol,news.ua.mrpl_city
%%date 
 
%%tags 
%%title Проєкти "Сучасної України" до шевченківських днів
 
%%endhead 
 
\subsection{Проєкти \enquote{Сучасної України} до шевченків\hyp{}ських днів}
\label{sec:11_03_2021.stz.news.ua.mrpl_city.1.proekty_suchasnoj_ukrainy_shevchenko_dni}
 
\Purl{https://mrpl.city/blogs/view/proekti-suchasnoi-ukraini-do-shevchenkivskih-dniv}
\ifcmt
 author_begin
   author_id demidko_olga.mariupol,news.ua.mrpl_city
 author_end
\fi

9 березня 1814 року народився визначний поет, класик української літератури,
мислитель, художник \emph{\textbf{Тарас Григорович Шевченко}}. Україна відзначила 207-річчя від
дня народження письменника. А 10 березня \emph{\textbf{160 років тому}} серце Великого Кобзаря
перестало битися. Без його імені сьогодні не можна уявити нашої літератури,
нашої культури, нашої країни. Його творчість невмируща. 1384 – стільки
пам'ятників Тарасові Шевченку встановлено в світі. З них в Україні налічується
1256 монументів, і ще 128 в 35 країнах світу – Бразилії, США, Китаї тощо.
Цікаво, що за життя Кобзаря більше цінували як художника, ніж як поета.
Сьогодні Тараса Григоровича шанують як автора кількох поетичних збірок, поем,
драми \enquote{Назар Стодоля}, російськомовних повістей; ідейного натхненника
Кирило-Мефодіївського товариства. Водночас він був дійсно талановитим
художником (випускник Петербурзької академії мистецтв, академік гравюри),
автором чималої кількості олійних полотен, зокрема, портретів та автопортретів,
акварелей, рисунків. Не всім відомо, що Тарас Шевченко був офортистом (одним із
перших у тогочасній Російській імперії), автором багатьох офортів на класичні
та власні сюжети.

\ii{11_03_2021.stz.news.ua.mrpl_city.1.proekty_suchasnoj_ukrainy_shevchenko_dni.pic.1}

Він основоположник нової української літератури, громадський діяч,
просвітитель – своїм коштом склав та видав \enquote{Буквар} для недільних шкіл,
планував видати також граматику, арифметику, географію; брав участь у
підготовці першого номера журналу \enquote{Основа}; популяризатор класичного мистецтва
серед простого люду.

Спадок, який створив Тарас Шевченко понад 2 століття тому, став джерелом
натхнення для багатьох сучасних митців. Завдяки креативному підходу та свіжому
погляду, безліч текстів, образів, картин Кобзаря отримали друге життя у XXI
столітті.

У Маріуполі до шевченківських днів вже не один рік створюють самобутні та
яскраві проєкти найбільш творче подружжя міста \emph{\textbf{Марія і Олександр Сладкови}}.
Сьогодні вони працюють у МПК \enquote{Український дім}: Марія – керівницею \emph{\textbf{творчої
студії \enquote{Сучасна Україна}}}, а Олександр – звукорежисером. Їх надихає поетичне
слово Тараса Шевченка, його потужна діяльність. Загалом любов до творчості
Т. Шевченка – одне з джерел натхнень для Олександра і Марії. Проєкти Сладкових
не про Шевченка, а про велику енергетику його поетичного слова, яка впливає на
творчість сучасних авторів. Вони надихають маріупольську молодь. Зокрема, після
проєкту, який вони показали у МПК \enquote{Український дім} 13 березня 2019 року,
маріупольські учні захотіли написати твір про мрії Тараса Шевченка. Марія і
Олександр підтримали цю ідею і приготували подарунки найкращому твору. Всі
проєкти Олександра і Марії до шевченківських днів мають спільну назву \emph{\textbf{\enquote{МРІЇ
ТАРАСА}}}. Кожен – різний за контентом, але спільний за своєю ідеєю: любов до
рідної землі, до рідної мови, до України.

\ii{11_03_2021.stz.news.ua.mrpl_city.1.proekty_suchasnoj_ukrainy_shevchenko_dni.pic.2}

Цього року вони підготували декілька проєктів. Будуть представлені нові вірші,
відбудеться прем'єра пісні. Презентація обох проєктів проходитиме онлайн у
березні. Зокрема, один проєкт вже був представлений 9 березня у програмі \enquote{Ранок
Маріуполя}, відбулася презентація відео, в якому учасники читали \enquote{Заповіт}
Т. Шевченка на різних мовах (взяли участь студенти та викладачі МДУ і актор
Донецького академічного обласного драматичного театру (м. Маріуполь) Ігор
Курашко). Презентація другого проєкту відбудеться теж  онлайн 26 березня.

Організатори радіють, що кожного року до їхніх заходів долучається все більше
нових учасників, що свідчить про актуальність і потрібність проєктів \textbf{\enquote{Мрії
Тараса}}.

Цьогоріч через запровадження карантинних заходів Сладкови вперше представили
свої проєкти \textbf{\enquote{Мрії Тараса}} в онлайн режимі.

\ii{11_03_2021.stz.news.ua.mrpl_city.1.proekty_suchasnoj_ukrainy_shevchenko_dni.pic.3}
