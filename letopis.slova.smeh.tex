% vim: keymap=russian-jcukenwin
%%beginhead 
 
%%file slova.smeh
%%parent slova
 
%%url 
 
%%author 
%%author_id 
%%author_url 
 
%%tags 
%%title 
 
%%endhead 
\chapter{Смех}

%%%cit
%%%cit_head
%%%cit_pic
%%%cit_text
Крымский полуостров, над которым \emph{посмеялся} на днях секретарь СНБО
Алексей Данилов, продолжает бороться с последствиями наводнения.  Дожди пока
что перестали заливать Южный берег Крыма и пик наводнения вроде как пройден. Но
спасательные работы еще в самом разгаре.  Местное население, коммунальщики,
военные и все неравнодушные бросились откачивать воду из затопленных домов,
убирать мусор с улиц и перегораживать горные реки, которые вышли из берегов и
продолжают извергать воду на города и поселки ЮБК.  Черное море на побережье в
одночасье замусорилось, купаться во многих местах нельзя. Но запреты некоторые
туристы игнорируют даже в Ялте - не пропадать же отпуску.  Зато теперь
полуостров гарантированно обеспечен водой до конца года, а то и гораздо дольше.
И это уже не шутка.  \enquote{Страна} разобралась, что сейчас происходит на
южном берегу и будут ли еще дожди
%%%cit_comment
%%%cit_title
\citTitle{Ялта - наводнение в Крыму. Что происходит 21 июня и как ликвидируют катастрофу}, 
Максим Минин, strana.ua, 21.06.2021
%%%endcit


