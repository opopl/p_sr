% vim: keymap=russian-jcukenwin
%%beginhead 
 
%%file 06_05_2023.stz.news.ua.suspilne.1.ludi_knizhki_bagattja.3.mangush
%%parent 06_05_2023.stz.news.ua.suspilne.1.ludi_knizhki_bagattja
 
%%url 
 
%%author_id 
%%date 
 
%%tags 
%%title 
 
%%endhead 

\subsubsection{Мангуш. Хліб, ковбаса \enquote{Макєєвська} та сльози}

Сіли в машину. Нас було п'ятеро: я з донькою, моя подруга з донькою й ще один
хлопець. Нам водій одразу сказав: ляжте на підлогу й не висовуйтесь зовсім. Ну,
як же не висовуватись. Ми дивились у вікна — це був жах: дуже багато було
загиблих людей, вони просто лежали на вулиці, їх ніхто не збирав, бо дівати
було нікуди. Коли ми виїхали за Маріуполь, він тоді сказав: все, видихайте, вже
більш менш безпека. Це було Портовське, дорога на Мангуш.

Ми виїхали з бібліотеки о пів на дванадцяту, і до Мангуша ми потрапили о 8-й
годині вечора, бо там була колона. що просто все тихенько, тихенько, все було в
полях це все ми стояли довго й довго. Й там в полях пробився перший зв'язок. Я
тоді змогла зв'язатись з чоловіком, з мамою, яка залишилась там під Маріуполем.
Дізнались, що живі.

\ii{06_05_2023.stz.news.ua.suspilne.1.ludi_knizhki_bagattja.pic.6}

У Мангуші ми зустріли друга чоловіка, він нас поселив. І ми помились перший раз
за ці всі дні. Були трохи здалеку вибухи, але ми знали, що сюди вже нічого не
прилетить, бо тут — вони вже окупували Мангуш. Це перша ніч, яку ми більш менш
спокійно спали, хоча все чули, знали, що бомблять вже Азовсталь.

Як раз в цей день, 16 березня, коли ми виїхали — прилетіло у драмтеатр. Дві
доби ми провели в Мангуші. І от вранці ми із подругою пішли щось купити поїсти,
і ми пройшли по Мангушу, побачили всіх цих "денерівців" з цими білими
пов’язками, ці черги за хлібом. Нічого не схотілось, прийшли з нею мовчки,
лягли й 2-й годині і просто мовчали, бо це був, якийсь такий, якась така
ситуація, що ми не вірили, що це все з нами проходить. Знайшли там у хазяїв,
щось там перекусили, так пройшов день.

Телефонуємо чоловікам: що нам далі. Кажуть, ми шукаємо, як вас забрати. Бо тоді
ще не розуміли, як їхати, це тільки-тільки початок тиках виїздів був, ще не
було фільтрації.

\ii{06_05_2023.stz.news.ua.suspilne.1.ludi_knizhki_bagattja.pic.7}

Наступного ранку ми таки перший раз купили в Мангуші хліб: одна булка в руки.
Купили ковбасу — \enquote{Макєєвська}, як зараз пам'ятаю. Всі ж були голодні, й ми
тільки зайшли до квартири, дівчата наші побачили цей хліб й почали плакати, хоч
вже і дорослі, і все розуміють, але почали плакати, бо за 20 днів не бачити
хліба — це в голові не вкладається.

Саме тоді прийшов товариш мого чоловіка, який нас привіз до Мангуша, і сказав
бігом збиратися, він відвезе нас до Бердянську. До Бердянська дорога була
важча, довша. Ми виїхали о 10-й ранку, а потрапили туди, коли вже почалась
комендантська година. Щоправда, нас пропустили. Там вже і блокпости були, і з
дороги не можна було відійти кудись на якусь полосу, бо там були міни.

\begin{qqquote}
Паспорти, перевірки, \enquote{чому ви їдете}, — вже тримались з останніх сил. А подруга
моя двічі переселенка, з Донецька — і в неї така лють була, така ненависть. Я
кажу: ну потерпи, ну потерпи, зараз же, інакше зараз всі будемо \enquote{на підвалі}.	
\end{qqquote}

