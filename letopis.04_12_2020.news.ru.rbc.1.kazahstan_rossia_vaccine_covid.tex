% vim: keymap=russian-jcukenwin
%%beginhead 
 
%%file 04_12_2020.news.ru.rbc.1.kazahstan_rossia_vaccine_covid
%%parent 04_12_2020
 
%%url https://www.rbc.ru/rbcfreenews/5fc9eaec9a794752c271ad16
 
%%author 
%%author_id 
%%author_url 
 
%%tags 
%%title Казахстан назвал дату начала производства российской вакцины от COVID-19
 
%%endhead 
 
\subsection{Казахстан назвал дату начала производства российской вакцины от COVID-19}
\label{sec:04_12_2020.news.ru.rbc.1.kazahstan_rossia_vaccine_covid}
\Purl{https://www.rbc.ru/rbcfreenews/5fc9eaec9a794752c271ad16}

\index[rus]{Коронавирус!Вакцина!Спутник V, Казахстан, Россия, 04.12.2020}

\ifcmt
pic https://s0.rbk.ru/v6_top_pics/resized/1180xH/media/img/9/55/756070691076559.jpg
cpx Фото: пресс-служба АФК «Система» / РИА Новости
\fi

В Казахстане с 22 декабря приступят к производству российской вакцины против
коронавируса «Спутник V». Об этом премьер-министр Аскар Мамин доложил
президенту Касым-Жомарту Токаеву, сообщается на сайте главы государства.

«Отдельным блоком главе государства сообщили, что в соответствии с его
договоренностью с президентом России Владимиром Путиным с 22 декабря текущего
года в Казахстане начнется производство российской вакцины «Спутник V», —
говорится в сообщении.

Массовая вакцинация населения планируется с начала 2021 года, она будет
добровольной. В первую очередь прививку смогут сделать граждане из группы
риска: врачи, педагоги, сотрудники силовых структур.

Кроме того, глава правительства сообщил, что в республике 29 ноября завершились
первая и вторая фазы клинических испытаний казахстанской вакцины против
COVID-19. Мамин отметил, что препарат дал положительный результат, побочных
эффектов не зафиксировано. В рамках третьей фазы исследований с 25 декабря
вакцину получат 3 тыс. добровольцев, уточнил он. Испытания казахстанского
препарата планируется завершить в конце марта 2021 года.

В середине ноября президент Касым-Жомарт Токаев заявил, что в стране будет
построен завод для производства российской вакцины.

Первой страной, получившей российскую вакцину против COVID-19 для испытаний,
стала Белоруссия. В конце ноября посол России в Минске Дмитрий Мезенцев
сообщил, что в республике начнут применять первую партию препарата в декабре.

В августе Россия объявила о регистрации первой вакцины против коронавирусной
инфекции, разработанной Национальным исследовательским центром эпидемиологии и
микробиологии имени Н.Ф. Гамалеи. Вакцина получила торговое название «Спутник
V». В октябре был зарегистрирован второй российский препарат, «ЭпиВакКорона» —
разработка Государственного научного центра вирусологии и биотехнологии
«Вектор». Кроме того, к регистрации готовится третья вакцина, созданная
Федеральным научным центром исследований и разработки иммунобиологических
препаратов им. М.П. Чумакова.

В начале декабря в Российском фонде прямых инвестиций (РФПИ) сообщили, что
получили заявки на приобретение 1,2 млрд доз вакцины «Спутник V» из более чем
50 стран.



