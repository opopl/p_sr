% vim: keymap=russian-jcukenwin
%%beginhead 
 
%%file 12_11_2022.fb.kokotjuha_andrij.1.donbass_zvilnemo_nam_radi
%%parent 12_11_2022
 
%%url https://www.facebook.com/andriy.kokotuha/posts/pfbid05PwoBnr6h85rLxRfh5ap83Qq448FYmWmTJ3YZaydFuvUnmaCmKNu3D44ZfDFVS4hl
 
%%author_id kokotjuha_andrij
%%date 
 
%%tags donbass,ukraina
%%title Ми звільнимо Донбас. Але чи будуть нам там раді?
 
%%endhead 
 
\subsection{Ми звільнимо Донбас. Але чи будуть нам там раді?}
\label{sec:12_11_2022.fb.kokotjuha_andrij.1.donbass_zvilnemo_nam_radi}
 
\Purl{https://www.facebook.com/andriy.kokotuha/posts/pfbid05PwoBnr6h85rLxRfh5ap83Qq448FYmWmTJ3YZaydFuvUnmaCmKNu3D44ZfDFVS4hl}
\ifcmt
 author_begin
   author_id kokotjuha_andrij
 author_end
\fi

Цей допис напевне викличе в частини читачів трихвилинку ненависті до його
автора. Але пишу, про що справді думаю. А думаю про наступне. Херсон звільнили
- і мешканці зустрічали ЗСУ так, що в старих циніків типу мене пробилися сльози
з очей вперше після звільнення Київщини, Чернігівщині й Сумщини. Я ніколи не
був у Херсоні. Мені казали - регіон складний у проукраїнському плані, загалом
депресивний, занепадає тощо. Проте з перших днів окупації люди виходили на
мітинги протесту. Не 100\% херсонців - але 100\% ніде не виходить і нічого не
робить. Проте люди радісно вітали ЗСУ. Чи буде це в Донецьку та Луганську? Моя
відповідь: ні. Я згоден, колеги, що там 2014 року були мітинги. Згоден, що там
убивали за українську мову. Згоден, що в Донецьку досі діє концтабір Ізоляція і
що там тримають людей і за проукраїнські погляди. Натомість у Львові і загалом
західних регіонах України знаходять прибічників русского мира. До чого
порівняння? До того, що у Львові таких відносно небагато. Як відносно небагато
в Донецьку та Луганську проукраїнської публіки. Там ждуни. 2014-того чекали на
Росію. Бо думали: годуватиме. Тепер, можливо, чекають на Україну - але з метою
виставити рахунок. \enquote{Вы нам должны за войну. Вы не мирились с Россией - мы из-за
этого страдали. Зашли назад, власть? Хотите тут мову? А хер вам! Платите
компенсацию за свою войну!} Приблизно таким буде формат діалогу з неодмінно
звільненими Донецьком та Луганськом. Звільняти їх треба, бо це українська
територія. Проте ходити там доведеться, постійно озираючись. Приклади
українського Донбасу знаю без нагадувань. Ці достойні люди або вже не живуть,
померши своєю смертю, або вбиті СССР і Росією, або виїхали звідти і дев'ятий
рік вперто виправдовують свій регіон. І епілог: якби Донецьк і Луганськ вдалося
звільнити ще 2014-того, українських воїнів там теж би не вітали. Камінь за
пазухою, дуля в кишені. До Херсону - запрошуйте...

\ii{12_11_2022.fb.kokotjuha_andrij.1.donbass_zvilnemo_nam_radi.orig}
\ii{12_11_2022.fb.kokotjuha_andrij.1.donbass_zvilnemo_nam_radi.cmtx}
