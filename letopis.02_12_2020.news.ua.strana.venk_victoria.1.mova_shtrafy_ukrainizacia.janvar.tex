% vim: keymap=russian-jcukenwin
%%beginhead 
 
%%file 02_12_2020.news.ua.strana.venk_victoria.1.mova_shtrafy_ukrainizacia.janvar
%%parent 02_12_2020.news.ua.strana.venk_victoria.1.mova_shtrafy_ukrainizacia
 
%%url 
 
%%author 
%%author_id 
%%author_url 
 
%%tags 
%%title 
 
%%endhead 

\subsubsection{Что будет со сферой услуг в январе}
\label{sec:02_12_2020.news.ua.strana.venk_victoria.1.mova_shtrafy_ukrainizacia.janvar}

Тарас Креминь у себя в Facebook написал, что с 16 января 2021 года сфера
обслуживания переходит на украинский язык - в соответствии со статьей 30 закона
\enquote{Об обеспечении функционирования украинского языка как государственного}. 

\enquote{Все поставщики услуг, независимо от формы собственности, обязаны
обслуживать потребителей и предоставлять информацию о товарах и услугах на
государственном языке. В супермаркете и в интернет-магазине, в кафе, банка, на
АЗС, в аптеке или библиотеке, где бы ты ни был - обслуживание должно быть
украинским. Только по просьбе клиента его персональное обслуживание может
осуществляться на другом языке}, - сообщил уполномоченный. 

Он призвал жаловаться на нарушителей и оставил e-mail.

\ifcmt
pic https://strana.ua/img/forall/u/0/34/%D0%A1%D0%BD%D0%B8%D0%BC%D0%BE%D0%BA(427).JPG
caption Скриншот из Facebook
\fi

В комментариях под постом Креминя пользователи интересуются, что предусмотрено
для нарушителей этой нормы. И какие будут штрафы. 

\ifcmt
pic https://strana.ua/img/forall/u/0/34/%D0%A1%D0%BD%D0%B8%D0%BC%D0%BE%D0%BA(428).JPG
caption Скриншот из Facebook
\fi

При этом пока что беспокоиться предпринимателям рано. Штрафы заработают лишь
через полтора года - с 16 июля 2022 года. До этого времени, по сути, никаких
санкций для тех, кто обслуживал покупателей на не государственном языке, быть
не может, хотя механизм наказания нарушителей уже прописан. 

В нем говорится, что при выявлении нарушения - читай: общения с клиентом на
русском языке - уполномоченный или его представитель составляет акт и объявляет
фирме предупреждение. И требование устранить нарушения в течение 30 дней. 

В случае повторного нарушения в течение года составляется протокол. И
выписывается постановление о штрафе в размере от 300 до 400 необлагаемых
налогом минимумов (5100-6800 грн). Его нужно будет оплатить в течение 15 дней.

Повторное админнарушение в течение года - влечет штраф в размере от 8500 до 11
900 грн. При этом постановление о штрафе можно попробовать обжаловать в суде. 

