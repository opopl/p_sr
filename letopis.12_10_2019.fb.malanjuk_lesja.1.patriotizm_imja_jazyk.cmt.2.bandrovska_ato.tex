% vim: keymap=russian-jcukenwin
%%beginhead 
 
%%file 12_10_2019.fb.malanjuk_lesja.1.patriotizm_imja_jazyk.cmt.2.bandrovska_ato
%%parent 12_10_2019.fb.malanjuk_lesja.1.patriotizm_imja_jazyk.cmt
 
%%url 
 
%%author_id 
%%date 
 
%%tags 
%%title 
 
%%endhead 

\paragraph{Vira Bandrovska - Патриотизм - Зона АТО}

\begin{itemize} % {
\iusr{Vira Bandrovska}

Я українка, спілкуюся українською мовою, але не згодна ,' бо в зоні АТО, багато
добровольців, які пішли у 2014році, розмовляли російською. І тому так категорично
писати, що патріоти ,лише ті, які розмовляють українською, невірно. Бо скільки
таких, що розмовляють українською, повтікали за кордон, щоб не ідти в АТО.

\begin{itemize} % {
\iusr{Леся Маланюк}
А за що вони воюють у тій «АТО»?

\iusr{Ігор Кушнірчук}
\textbf{Vira Bandrovska}, 

ті, що послуговуються московитською і є московитами. Вони живуть в віртуальній
Московії і все московитське для них своє. І хуйло має рацію, коли повторює
формулу нікмені Катерини 2: "Ґдє руцкій язик - там расєя". Це московитомовні -
брат'я московитам, бо від московитів їх відрізняє тільки паспорт.

\iusr{Оксана Захарчин}
\textbf{Леся Маланюк} я знаю-за русский язык! Щоб потім отут нам усім "какати" і "штокати".

\iusr{Vira Bandrovska}
\textbf{Фрейр Бинтмакерь} Не ображайте хлопців, які воюють і загинули за Україну, розмовляючи російською.

\iusr{Vira Bandrovska}
\textbf{Фрейр Бинтмакерь} І щоб щось тут доказувати ,потрібно не створювати пост одноденний,під чужим іменем. А краще ідіть на війну,,а коли повернетесь,, будете когось судити

\iusr{Vira Bandrovska}
\textbf{Леся Маланюк} У тому числі за таких ,як ви ,що їх зневажають.

\iusr{Vira Bandrovska}
\textbf{Леся Маланюк} 

Я люблю українську мову , але ніколи не можна бути такою впертою, зневажати
людей, які віддають життя за таких ,як ви , бо їхній один недолік, що вони
проживали при таких обставинах, і спілкуються російською мовою, це не по людськи
і не по христиЯнськи, і скорше ,вибачте, дурість

\iusr{Леся Маланюк}
\textbf{Віра Бандровска}, вони можуть змінюватися, а ми маємо їм у цьому допомогти.

\iusr{Оксана Захарчин}
\textbf{Леся Маланюк} обов'язково .Допоможемо.

\iusr{Костянтин Собіченко}
\textbf{Vira Bandrovska}, 

я спілкувався колись російською мовою (обставини, як ви кажете), за що мені
тепер соромно. Якщо хтось скаже, що я був ідіотом - погоджусь, бо так воно і є.
І нічого - нормально почуваюсь, визнавши свій ідіотизм. Дивно було би, аби я
його не визнав.

\iusr{Ольги Мусієнко-Боровик}
\textbf{Vira Bandrovska} якби не ця гібридна війна, то можна так говорити, але вперто дуркують саме ті, хто продовжує щось говорити про захист якоїсь окупантської " язИки" та ще й завжди апелюють до російськомовних воїнів....

\iusr{Vira Bandrovska}
\textbf{Ольги Мусієнко-Боровик} Я не захищаю ,російську мову. Мова іде про інше Але ви зациклена І лишнє вести з вами мову.

\iusr{Леся Маланюк}
\textbf{Віра Бандровска}, а що ви захищаєте? Право рускаязичних защітніков на москвинську? Навіщо?

\iusr{Ольги Мусієнко-Боровик}
\textbf{Vira Bandrovska} 

чому ж так жорстоко зі мною ? Якби розциклилися ті толеранти зі збайдужілими
разом, було б дуже добре для УКРАЇНИ. Як цього не розуміти? Чом так вперто
чіплятися за все чуже. Про що " христиЯнське" мова, коли підступно вбивають
українців?

\iusr{Ігор Кушнірчук}
\textbf{Vira Bandrovska}, 

ті хлопці, що загинули московитомовними загинули марно, бо вони навіть ціною
свого життя не можуть врятувати Україну від руцкаґа міра, що їх вбив, бо вони
самі є провідниками того руцкаґа міра й п'ятою колоною хуйла. А якщо вони цього
не розуміли, як не розумієте цього й ви, то це зовсім не відміняє об'єктивної
реальності. Я вам відповів, бо написали ви українською, а те, що ви займаєтеся
демагогією та пересмикуванням за московськими методичками вам окреме "дякую"
від хуйла.

\iusr{Руслана Курах}
\textbf{Vira Bandrovska} 

це їхня біда, що вони розмовляли й розмовляють російською, і не робить їм
честі, не робімо з того прикладу іншим, якщо хочемо мати Україну, а не
жовто-блакитну росію, тож ваш стокгольмський синдром і приступ невігласької
доброти тут не є доречним.

\iusr{Vira Bandrovska}
\textbf{Руслана Курах} 

Знаєте, щоб вести такі дискусії ,про хлопців ,які воюють в АТО, підніміть свій
зад, і їдьте на фронт, і самі побачите. А говорити з дивана дуже легко.
Говорити, писати, і при цьому дивитися українські серіали, російською мовою. Ось
де біда наша, а не хлопці ,які виросли у Миколаївській, Одеській областях ,і не
зовсім по своїй волі розмовляють російською. В Україні повинні серіали бути
українською, передачі, і т. д. І тоді і вони зміняться. А сидіти на дивані, і
поливати брудом наших захисників ,тільки за те що вони розмовляють не
українською, не є правильно.

\iusr{Vira Bandrovska}
\textbf{Руслана Курах} Вам шана.

\iusr{Vira Bandrovska}
допомагайте .

\iusr{Vira Bandrovska}
\textbf{Ольги Мусієнко-Боровик} Ніж сидіти на дивані і писати філософські статті поцікавтеся трохи як стоять речі на фронті.

\iusr{Леся Маланюк}
\textbf{Віра Бандровска}, а ви де зараз сидите?

\iusr{Руслана Курах}
\textbf{Vira Bandrovska} не треба мені приписувати тепер всяке, просто припиніть толерувати, бо ваша толерантність дорого нам обходиться

\iusr{Руслана Курах}

мда.. хто б тут казав про російські серіали, коли у самої рускій мір на
сторінці, відео "рєбйонок вижил" і всякі нечистоти.. от так насрано в людських
головах, і головне - подивіться, скільки їх є...(пролайкало людей оте її
бєзразніци), я в шоці, цим нехлюйним людям вже нічого не допоможе.. уже б тоді
не лицемірили сумними пичками й свічками на чергову загибель..

\iusr{Vira Bandrovska}
\textbf{Леся Маланюк} Не коло телевізора.

\iusr{Vira Bandrovska}
\textbf{Руслана Курах} Такі як ви ,"дуже патріоти" якщо щось зміниться ,першими підуть здавати людей.

% -------------------------------------
\ii{fbauth.bandrovska_vira.ukraina}
% -------------------------------------

\textbf{Руслана Курах} 

Мої два сини, не утікали у Росію на заробітки, а пішли добровольцями на фронт.
Разом з ними пліч опліч воювали і ті ,хто розмовляв українською, і ті хто
розмовляв російською. Вони там не робили різниці. Вони захищали і захищають
Україну. На похорон до мого сина приїхали побратими, які розмовляли російською.
Вони для мене такі ж герої, як і ці що розмовляють українською. І скільки
повтікало українців у Росію в2014 2015роках, щоб не викликали у армію. Не
закидайте мені ,що я захищаю руський мір. Я нікого не захищаю ,а просто веду до
того ,що мають пошану усі воїни ,які боронять Україну . Можете писати ,усе що
хочете, вичитувати з мої сторінки ,що хочете, я маю свою думку. А таких
"патріотів і захисників мови" ніколи не б хотіла у друзі, людина розумна,
ніколи не є категорична. . Уміє відрізнити біле від чорного, знайти середину.

\iusr{Костянтин Собіченко}
\textbf{Vira Bandrovska}, 

оце й погубить вас, а разом і нас усіх, - ви не бачите різниці в тому, хто якою
мовою говорить. А це фундаментальні речі, світоглядні. Всі російськомовні
можуть скільки завгодно помирати за Україну, але навіть смертями своїми
героїчними України не збудують. Бо не будується вона посередництвом ЧУЖОЇ мови.

Можливо, колись зрозумієте.

\iusr{Vira Bandrovska}

КостянтинНе перекручуєте .Я різницю у мові бачу. Але ,ще є вчинки, які не
залежать від мови ,шановний. Чому я написала про українські серіали ,російською
мовою, бо мене це болить. Тому, усі повинні бороться, з такими явищами. А не з
хлопцями, які воюють ,і не поважати їх. А як усі фільми, телепередачі, шоу будуть
тільки українською, тоді будете, говорити про хлопців на фронті. Собіченко

\iusr{Костянтин Собіченко}
\textbf{Vira Bandrovska}, а до того не можна? )

Поки оті російськомовні будуть на фронті - то і серіали для них мають бути. А
чому ні, вони ж важливі люди, чи не так? Чому з цього боку не підійти?) Чому ви
не поважаєте, в такому разі, священне право російськомовних на священні
російськомовні серіали? Вони ж нібито, з ваших же слів, заслужили на багато
чого?

\iusr{Костянтин Собіченко}
\textbf{Vira Bandrovska}, та я пам'ятаю про своє прізвище - не нагадуйте.)

\iusr{Vira Bandrovska}
\textbf{Костянтин Собіченко} При чому прізвище?

\iusr{Vira Bandrovska}
\textbf{Костянтин Собіченко} Я не відповім вам нічого. Дарма тратити час,.Пишіть ,що хочете.

\iusr{Леся Маланюк}
\textbf{Віра Бандровска}, 

ви закликаєте всіх боротися з такими явищами, а самі чому поширюєте дописи
бидлоязиком? І скільки разів ви сказали рускаязичьним байцам, що держава, яку
вони захищають, називається Україна, і мова наша, відповідно, українська?

\iusr{Костянтин Собіченко}
\textbf{Vira Bandrovska}, прізвище ні до чого, то так, жарт.)
Як знаєте, не відповідайте.
А між іншим, у вас ще є чимало українського. І це добре. Якщо думаєте про серіали українською мовою - то це вже багато про що говорить.
За синів - окрема, і велика дуже, подяка! Тут без питань.

\iusr{Руслана Курах}
\textbf{Vira Bandrovska} 

Тепер кожен почне захищати Україну рускім ізиком, бо "там є російськомовні
захисники в окопах", то ми хіба залишатимемось Україною? Не зациклюйтесь лише
на захисті територій, дірявлячи своїми ворожими закликами тил. Той факт, що не
всі захищали Україну усвідомлено, багато це робили за гроші, не повинен
слугувати як прикриття для пропаганди російського. Героїзуючи росмовних
захисників,ми завтра матимемо втарой гасударствєнний, бо хтось дуже добрий і
дуже ватний, власні емоції ставить понад національні інтереси.

\iusr{Vira Bandrovska}
\textbf{Руслана Курах} підростіть трохи ,щоб мене учити.

\iusr{Ольги Мусієнко-Боровик}
\textbf{Vira Bandrovska}\textbf{Bandrovska} будь ласка, вчіть мову- це гарний реальний крок до України і її народу...досконало вчіть, а не абияк.

\iusr{Ольги Мусієнко-Боровик}
\textbf{Vira Bandrovska} стоять??? Вас задовольняє те, що абсурдні толеранти з
неповагою до України і її народу чіпляються як тільки можуть за " язик"?
Доречі, на диванах саме правдиві українці точно не сидять- або гарують на
роботі поза визначеним часом, або впорядковують негаразди, що створює "
насєлєніє ".

\iusr{Руслана Курах}
пані Віра нарешті зависла, втрапивши у внутрішній когнітивний дисонанс

\end{itemize} % }

\iusr{Igor Bezuglof}
Це тіпа ... як назвеш так і попливьот ?!  @igg{fbicon.face.tears.of.joy} 

\begin{itemize} % {
\iusr{Леся Маланюк}
Саме так. Або потоне.))

\iusr{Ігор Кушнірчук}
\textbf{Леся Маланюк}, лайно не тоне.
\end{itemize} % }

\iusr{Ігор Кушнірчук}

Прочитав допис й подумав, зараз захисників руцкаґа міра налетить й будуть
намагатися довести, що вони "тожеукраінци". Так і є.

\begin{itemize} % {
\iusr{Костянтин Собіченко}

Більше того, не лише тожеукрАінци - навіть ті, що справжніми є... нібито
українською пишуть... от лише ім'я своє з москвинської ніяк... ну ніяк, ну, не
можуть люди... його, те ім'я, українською написати.) От все вони розуміють, все
сприймають, але ось цей пункт... проблемний дуже. Вперлись - і клавіші не
натискаються на перейменування. Так, як не натискались вони на український лад
тоді, коли вони вперше оприлюднювали ймення своє для фб-спільноти.)

\end{itemize} % }

\iusr{Тетяна Гога}
Повністю згодна з вами. Бо Маші, Даші, лєни це внутрішнє визнання меншовартості

\iusr{Jaroslav Datzik}
Навіть дуже патріотичні галичани кличуть свою доню КАТЯ.

\begin{itemize} % {
\iusr{Юрій Іщук}
\textbf{Jaroslav Datzik} стоп!!! А що, Катя- це раіське ім'я????доведіть, або зітріть цей пост...

\iusr{Ігор Кушнірчук}
Те, що ці люди походять з Галичини, не робить їх ідеальними українцями.

\iusr{Jaroslav Datzik}
\textbf{Юрій Іщук} Ваня, Петя, Вася, Маша, Катя ,Валя, Таня, Міша, Поля,Толя, Паша , Даша, Ґлаша... Пане Іщук, вам пороблено.

\iusr{Jaroslav Datzik}
Коли Вашу доню звати Катерина, то це КАТРУСЯ!
\end{itemize} % }

\iusr{Jaroslav Datzik}
А небога моєї дружини назвала свою доню САША. Це Львів.

\iusr{Valentina Sidorenko}

Інколи навіть не виникає бажання читати дописи російською мовою. Якщо людина
вважає себе патріотом і українцем, тоді в першу чергу має вивчити мову.

\begin{itemize} % {
\iusr{Галина Литвин}
\textbf{Valentina Sidorenko} А я і не читаю. пропускаю мовчки.
\end{itemize} % }

\iusr{Светлана Антонюк}
ЦЯ Леся не сповна розуму...

\begin{itemize} % {
\iusr{Леся Маланюк}
Чому ви так вирішили?

\iusr{Ігор Кушнірчук}
\textbf{Леся Маланюк}, бо не співпадає з її московитським уявленням яким мав би бути українець.
\iusr{Леся Маланюк}
\textbf{Фрейр Бинтмакерь} (\textbf{Ігор Кушнірчук}) А сповна розуму для них — це ті, що відмовились від свого й повзають на колінах перед окупантом. Оце правильні українці!))

\iusr{Костянтин Собіченко}
\textbf{Светлана Антонюк}, Свєта Антонюк - оце поєднання! Ніби й українка... а ніби й Свєта... Чимало розуму.)

\iusr{Светлана Антонюк}
\textbf{Костянтин Собіченко} не жалуюсь

\iusr{Светлана Антонюк}
\textbf{Леся Маланюк} бесполезная полемика, если НЭ ПОНИМЭ

\iusr{Костянтин Собіченко}
\textbf{Светлана Антонюк}, то чого ти сюди приперлося, нє жалуйся там десь в однокласніках!) Там поймут, поддєржат, обласкают і согрєют. Своі люді. )

\iusr{Светлана Антонюк}
Сам то понял ШО сказал? @igg{fbicon.face.tears.of.joy}{repeat=3}  костя

\iusr{Костянтин Собіченко}
\textbf{Светлана Антонюк}, 

свєта, всьо нармальна. Я понял. А ти?) До речі, ти вже поняла за цей час, що
"несповна" одним словом пишеться? Ну, воно ніби й таке, нєнужная вєщь нєнужной
нікаму мови... А раптом?) Підтягуй рівень, і пам'ятай про цитату Гете!)) Всі
москвини й малороси її так шанують. Ось лише в інший бік розвернути не здатні.)

\iusr{Леся Маланюк}

\ifcmt
  ig https://scontent-frx5-2.xx.fbcdn.net/v/t39.1997-6/s720x720/851559_177569735758623_1106841098_n.png?_nc_cat=1&ccb=1-5&_nc_sid=0572db&_nc_ohc=0CWlo0lq-FYAX9_Gcy9&_nc_ht=scontent-frx5-2.xx&oh=9fd2142c8e719d1f231bc0d428362dd9&oe=616B82E9
  @width 0.2
\fi

\iusr{Наталія Вальчак}
\textbf{Светлана Антонюк}

\ifcmt
  ig https://scontent-frx5-2.xx.fbcdn.net/v/t39.1997-6/s168x128/851575_126362090881921_1049355036_n.png?_nc_cat=1&ccb=1-5&_nc_sid=ac3552&_nc_ohc=1_mw1w8NYA0AX9tFti4&_nc_oc=AQl5GyBngeb67WBWLmmtP27Kx4dAREPti1fVDshRU6kRU24wIo87dX2-f7cvbpptpiY&tn=lCYVFeHcTIAFcAzi&_nc_ht=scontent-frx5-2.xx&oh=e4a122c18bde68237ad8d68640ca8f93&oe=616B472F
  @width 0.1
\fi

\iusr{Светлана Антонюк}
\textbf{Наталія Вальчак} когда нечего сказать @igg{fbicon.face.tears.of.joy}{repeat=3} 


\end{itemize} % }

\iusr{Kyrylo Bohdanenko}

У фейсбуці є можливість додати транслітерацію свого імені різними мовами. Якщо
ви бачите моє ім'я латинкою, значить мова вашого фейсбука не українська  @igg{fbicon.wink} 

\begin{itemize} % {
\iusr{Ігор Кушнірчук}
\textbf{Кирило Богданенко}, українці бачать моє справжнє ім'я. @igg{fbicon.wink} 
\end{itemize} % }

% -------------------------------------
\ii{fbauth.klejmenov_aleksandr.kiev.ukraina.programming.utrecht}
% -------------------------------------

Це - впихування людей в якісь умовні шафи і мірила. Примітивне мислення із
середньовіччя. Я не можу уявити, щоб в Америці чи в якійсь європейській країні
людині казали, що як вона не напише своє ім‘я «як треба», то буде зрадником
країни. Тьфу, яке лайно смердюче.

\begin{itemize} % {
\iusr{Оксана Завадська}
\textbf{Alexander Kleimenov} Америку згадувати не варто, бо це збірна світу, але не можу уявити німця, француза, чи британця, що не є етнічним азіатом, але називає дитину азійських іменем.варто подивитись як називають дітей наші найближчі сусіди.

\iusr{Alexander Kleimenov}
\textbf{Oksana Zavadska} кожна людина має право називатися і називати своїх дітей, як їй заманеться. І ніхто не має права нікому вказувати, як це робити. Бо це - повний трешовий комуністичний совок, який сидить у головах, тільки його віночками і синьо-жовтими стрічками пообплітали.

\iusr{Ігор Кушнірчук}
\textbf{Alexander Kleimenov}, 

а литовці так не вважають, так само як ісландці, бо вони хочуть зберегти свою
ідентичність, а московити не мають жодної ідентичності, крім видуманої
імперської. Саме тому для них національна ідентичність не має жодної цінності.
Ба більше вони агресивно зневажають всі національності. Саме ксенофобія,
рабство та дологічне мислення характерні для московитів. (О.Борґардт "Дві
культури").

\iusr{Alexander Kleimenov}
\textbf{Ігор Кушнірчук} пост нагорі є взірцем ксенофобії.

\iusr{Ігор Кушнірчук}
\textbf{Alexander Kleimenov}, ні це намагання пояснити "тожеукраінцам", що мова має значення.

\iusr{Alexander Kleimenov}
\textbf{Ігор Кушнірчук} 

мова має значення, але до права людини самій вирішувати, як називатися і
називати своїх дітей, це не має жодного стосунку. А має, радше, до РПЦ, у
церквах якої відмовляються робити молебни за людей з «неправильними» іменами.
Тому московщина стирчить з цього посту ногами і вухами. Але ж у нас будь-яку
московщину радо ковтають, якщо на неї почепити жовто-блакитні стрічки.

\iusr{Костянтин Собіченко}
\textbf{Alexander Kleimenov}, більш смердючого лайна, ніж малорос не буває.

Нещасний, європейські народи відбулись як нації, вони можуть собі дозволити
багато чого! Ти ж, пишучи своє ім'я москвинським язиком, дозволяєш москвинам
панувати тут. Подумай, якщо зможеш.

\iusr{Alexander Kleimenov}
\textbf{Костянтин Собіченко} набагато гірше для України те, що за її патріота видає себе звичайне невиховане ксенофобське хамло. Оце тикання незнайомій людині—типовий стиль більшовиків і комуняк. Вичавлюйте краще це з себе, замість розповідати іншим, як їм свої імена писати.

\iusr{Ігор Кушнірчук}
\textbf{Alexander Kleimenov}, той, хто толерує ворога вже переможений. В Україні війна з московитами, а ви граєте ворогові на руку, бо не даєте можливості відрізнити з першого слова ворога і "тожеукраінца", який ніби патріот України.

\iusr{Alexander Kleimenov}
\textbf{Ігор Кушнірчук} ворогові на руку граєте ви, на жаль, навіть не усвідомлюючи цього.

\iusr{Костянтин Собіченко}
\textbf{Alexander Kleimenov}, немає нічого гіршого для України, як оці численні Алєксандри.

\iusr{Ігор Кушнірчук}
\textbf{Alexander Kleimenov}, 

знову ви, московити та колаборанти-малароси хочете нав'язати, що мова не на
часі. Я не ділю українців, я не об'єдную московитів з українцями в "адін нарот"
й називаю московитів московитами, хоч би в нього був український паспорт.

\iusr{Alexander Kleimenov}
\textbf{Ігор Кушнірчук} 

я десь сказав, що мова не на часі? Я десь сказав хоч щось з того, що ви
нафантазували? Ви брехун, ось ви кто. Ви розповсюджуєте наклепи і цим працюєте
на ворога. Називайте себе брехлом, і це буде чесно. Вичавлюйте з себе потроху
раба, це корисно.


\iusr{Костянтин Собіченко}
\textbf{Alexander Kleimenov}, 

а я гадав, що на руку ворогові грає той, хто підписується його, ворожою, мовою.
А виходить - ні: я Алєксандр - но я за Украіну!) Власне, так, ви всі за
"Украину".

\iusr{Alexander Kleimenov}
\textbf{Костянтин Собіченко} наплели казнащо.

\iusr{Костянтин Собіченко}
\textbf{Alexander Kleimenov}, та правильно наплів.)

\iusr{Ігор Кушнірчук}
\textbf{Alexander Kleimenov}, 

не кажете, що мова не на часі, але всіх переконуєте, що українці себе можуть
називати московитською й це нормально. А це одне й те ж, що "мова не на часі".

Брехня у ваших словах в кожному дописі, ви себе написали московитською (вас же
ніхто не заставив, а своєю власною рукою записали, чи не так?), з запіненим
ротом доводите, в українській групі, що для українця послуговуватися
московитською добре й зручно, й в цьому нічого поганого немає, а коли вам
вказують на алогічність, ви переходите на особистості й звинувачуєте инших
"поділі українців". І коли знову вашу брехню викрито, наведенням ваших цитат,
ви знову ж таки переходите до особистих образ. Та це не має жодного значення,
бо ви московит і власною рукою це вписали в фейсбук.

\iusr{Alexander Kleimenov}
\textbf{Ігор Кушнірчук} 

ви поширююте наклепи, вигадуєте те, що я ніколи не казав. Якщо у вашій порожній
голові «московитська мова» пишеться латинкою, то вам варто повернутися в школу
і заново пройти курс загальної освіти, бо пишете дурню несусвітню.

\iusr{Ігор Кушнірчук}
\textbf{Alexander Kleimenov}, 

\obeycr
вам наведені ваші ж цитати. Тобто вашу брехню доведено й це є доконаним фактом.
Я не писав, що треба писати московитською латинкою, бо московитська мене не обходить. Це ваша чергова маніпуляція.
А щодо моєї голови, та що мені робити, то це не ваша проблема, брехуне московитський.
Продовжувати спілкування з московитом не бачу сенсу тому, прощавайте!
\restorecr

\end{itemize} % }

\iusr{Stepan Telychko}
Худобі не відоме поняття патріот... Істоти.

\begin{itemize} % {
\iusr{Ирина Мовсумова}
\textbf{Stepan Telychko} у кожного свій світогляд.

\iusr{Stepan Telychko}
\textbf{Irina Mov} Світогляд мають свідомі люди, а не худоба...
\end{itemize} % }

\iusr{Александр Плясов}

Ім'я дають батьки, а не ми вибираємо. І виростаємо ми з ним, а тому й
ідентифікуємо себе через нього. Алєксандр чи Олександр ім'я теж не українське
так чи сяк, а грецього походження.

А які чисто українські? і що то таке?

Запишіть сюди й усіх Олегів, Костянтинів та багато ще яких імен, як би вони не
писалися...бо вони походять з інших мов.

До того ж, часто мені чомусь в приват пишуть "а чого ж це ви Алєксандр?" (крім
того, що вважаю це не їхньою справою) - рекомендую змінити налаштування
телефону НА УКРАЇНСЬКУ. Бо ВАША російська версія ФБ імена ПЕРЕКЛАДАЄ... і в
бан, бо ідіоти.

Якже добре отим новомодним - Герман, чи Аліна, чи що там ще - бо у них немає
двійників в нашій мові.

\begin{itemize} % {
\iusr{Леся Маланюк}
Дивно, що ви не зрозуміли, про що допис.

\iusr{Александра Стрельникова-Криницкая}
Ти сама розумієш, що надряпала?!!..

\iusr{Леся Маланюк}
І обов'язково якась чучундра невихована вилізе.))

\iusr{Ольга Когнітенко}
О, російськомірська пропоганонка Aleksandra Krinitskay не забарилася свій писок висунути @igg{fbicon.face.grinning.smiling.eyes} 

\iusr{Костянтин Собіченко}
\textbf{Aleksandra Krinitskaya}, а ти взагалі усвідомило для чого живеш на цьому світі, хоч приблизно? Хоч раз за життя замислювалась?

\iusr{Александр Плясов}
\textbf{Леся Маланюк} мені здається - зрозумів. Про імена. А ми їх собі не вибираємо.
Я знаю одну - її зовуть Аеліта. Її так назвали батьки. Ну то чи є у неї вибір, як себе представляти? Вона виросла з цим іменем і асоціює себе з ним, бо вибору в неї не було і немає.. І так: над нею насміхалися всі - від дитсадка до університету.

\iusr{Леся Маланюк}
\textbf{Олександр Плясов} (\textbf{Alexander Plassov}) Та ні, не про імена. Про те, що якщо ти українець, то чому підписуєшся моквинською?
Коли вибір є, але вибирають чомусь не українське.

\end{itemize} % }

\iusr{Оксана Прокуратор}

А мене нервує, коли до польських прізвищ в жіночому варіанті додають закінчення
ая. Богуцкая, Валевская... Вбила б..

\begin{itemize} % {
\iusr{Александра Стрельникова-Криницкая}
\textbf{Оксана Прокуратор}, так давай! Чого сидиш?!.

\iusr{Оксана Прокуратор}
\textbf{Aleksandra Krinitskaya} та ви вже й так скарані))
\end{itemize} % }

\iusr{Александра Стрельникова-Криницкая}

Правильно, треба всім імена з грецької, латини, івриту тощо поперекладати на
суто українські: не Алєксандра, а Захисниця Козаків, не Єлена, а Смолоскип, не
Федір, а Здоровань тощо!

\begin{itemize} % {
\iusr{Леся Маланюк}
Крініцкая — це по-якому?))

\iusr{Оксана Завадська}
\textbf{Aleksandra Krinitskaya} Хоч поцікався що означає Федір, те ж що Теодор, Богом даний, по нашому Богдан.

\iusr{Ігор Кушнірчук}
\textbf{Леся Маланюк}, то по-манкуртськи.

\iusr{Костянтин Собіченко}
\textbf{Aleksandra Krinitskaya}, до яких тільки маніпуляцій не дійдеш, аби захистити своє московитське нутро.) Побачили загрозу в дописі? Це добре.)

\iusr{Marko Rudyk}
\textbf{Aleksandra Krinitskaya} ну Вам пояснили шо пост не про імена як такі, чо ви дальше прете прете свою версію ....

\iusr{Marko Rudyk}
\textbf{Костянтин Собіченко} це як націоналізм і шовінізм... воно вузькуеонаправлене москалятко...

\iusr{Инна Дзюба}
\textbf{Костянтин Собіченко}

\ifcmt
  ig https://scontent-frt3-2.xx.fbcdn.net/v/t1.6435-9/72897280_2338643619595731_6509235537969152000_n.jpg?_nc_cat=101&ccb=1-5&_nc_sid=dbeb18&_nc_ohc=Jjhe5MWdGPMAX-gJN33&_nc_ht=scontent-frt3-2.xx&oh=2f388ef9c682d3055957ef32783a1cff&oe=618B68DF
  @width 0.4
\fi

\iusr{Леся Маланюк}
Воно-Инно — так, немудре, згодна.))

\iusr{Костянтин Собіченко}
В мене щось навіть картінка від Инни відкриватися не хоче - патамуша какая разніца: аткриваєцца ілі нєт.)

\iusr{Леся Маланюк}
\textbf{Костянтин Собіченко} Повідомила Вам, що Воно, тобто Инно, дуже дурне. Ну, щоб Ви вже про це точне знали й не сумнівалися.))

\iusr{Костянтин Собіченко}
\textbf{Леся},
але ж инна инною... та українською мовою відкопала плакатик десь... Воно все не так і погано - нехай поширюють такі речі, я потерплю.) Навіть радий.)

\iusr{Костянтин Собіченко}
Инна Дзюба, Инно, бачиш, яка мова на картинці? Напиши й своє ім'я такою ж! Гарно ж сказано, правда? Не бійся, можна й українською багато чого робити.)

\iusr{Леся Маланюк}
\textbf{Костянтин Собіченко} Про плакатики... А як Вам цей? Теж иннин, до речі, з її сторінки. Були москвлі — був хліб на столі?!

\ifcmt
  ig https://scontent-frt3-1.xx.fbcdn.net/v/t1.6435-9/72787905_1094372440773527_7819148899087548416_n.jpg?_nc_cat=107&ccb=1-5&_nc_sid=dbeb18&_nc_ohc=hzD_srSZBKkAX_2Jv6g&_nc_ht=scontent-frt3-1.xx&oh=cac0f20813a01c3d8045fb71e0470efa&oe=618A1D52
  @width 0.4
\fi

\iusr{Костянтин Собіченко}
\textbf{Леся Маланюк}, це їхнє, вони тішаться щиро таким юмаром, для ідіотів-малоросів в кремлі (а проте, майстерно) склепаним. Так на душу хохла лягає... як масло на хліб. Чи сало. )
Ця істота знає, що робить, між іншим. А скільки дурників поширюють - бо ж інтєресно зроблено!
Я вже бачив сьогодні, в одного патріота. Питаю - навіщо взагалі?

\end{itemize} % }

\iusr{Жанна Боднарук}

А ще до вищезгаданої проблеми, з якою цілковито погоджуюсь, додам ще одне, що
мене бісить!)) Це закінчення імен на -аша, -юша. Типу Ксюша, Андрюша...!

\begin{itemize} % {
\iusr{Леся Маланюк}
Згодна, пані Жанно! Москвинське це, не наше, не українське.

\iusr{Жанна Боднарук}
\textbf{Леся Маланюк} Я завжди виправляла всіх, хто мого сина Андрійка Андрюшою називав))

\iusr{Андрій Кривобабко}
\textbf{Жанна Боднарук} А де це українці Андрія називають Андрюшою? Мені дуже цікаво.

\iusr{Жанна Боднарук}
\textbf{Андрій Кривобабко} Як де? В Україні)) Навіть мої знайомі з Буковини)

\iusr{Андрій Кривобабко}
\textbf{Жанна Боднарук} Я маю зрілий вік. І прожив з цим ім'ям. Від жодного УКРАЇНЦЯ такого не чув. І Буковина незразок українства.

\iusr{Жанна Боднарук}
\textbf{Андрій Кривобабко} Вам пощастило)
\end{itemize} % }

\iusr{Tak Smith}
Смішно.

\begin{itemize} % {
\iusr{Леся Маланюк}
Коли не вистачає розуму зрозуміти, то залишається лише сміятися. З себе.

\iusr{Костянтин Собіченко}
\textbf{Tak Smith}, очєнь.)
\end{itemize} % }

\iusr{Костянтин Собіченко}

Вчора мені було прикро, а сьогодні зрадів навіть.) Таки правильний допис,
актуальний. Тут вам і малороси з москвинськими іменами, й імперці-шовіністи,
почітатєлі русскіх традіций, дурненкі хохлята також присутні...
маніпулятори-оповідачі про свободу самоствердження та досвід передових
європейських країн - також на місці. Всі разом. Свято просто якесь! )

Дякую, Леся Маланюк!)

А, чекаю на одвічно-сакральне - про Швєйцарію с єйо четирмя(!) гасударствєннимі
язикамі хочу почути. На класику, знаєте, потягнуло.)

\begin{itemize} % {
\iusr{Yaroslav Okhrymchuk}
\textbf{Костянтин Собіченко} Костянтине, к чому Ви радієте? Ваше ім’я наче не українське етнічно! То ж яке Ви маєте право щось тут писати, чи не так Лесю?

\iusr{Костянтин Собіченко}
\textbf{Yaroslav Okhrymchuk}, я радію "к тому", що своє ім'я, не українське за походженням, я навчився писати українською мовою, що для інших чомусь є непідйомним тягарем.

\iusr{Руслана Курах}
а я чекала на оте "та у вас ім'я не є українське" ахаха))

\iusr{Костянтин Собіченко}
\textbf{Yaroslav Okhrymchuk}, а чому українець класичний, з ім'ям справжнім, древнім, слов'янським, вирішив, що йому латинська абетка миліша?
\end{itemize} % }

\iusr{Ирина Мовсумова}
Я на ФБ 10 років. Відредагувати ім'я не вдається. Хотіла англомовний варіант,
бо знайомі з усього світу.

\begin{itemize} % {
\iusr{Костянтин Собіченко}
\url{https://www.facebook.com/UkrayinskiyFeysbuk/videos/1668537060095326/}
Ось так можете. Англомовний варіант - Iryna. Але навіщо вам під той світ підлаштовуватися - будьте собою, Іриною.

\iusr{Костянтин Собіченко}
\textbf{Irina Mov}, ну що там у вас? Чому не редагуєте?
\end{itemize} % }

\iusr{Іван Іванович}

Цілком підтримую тому що наше суспільство русифіковане і потрібно проводити
дерусифікацію суспільства !!! Тому що з русифікованого суспільства неможливо
сформувати українську націю !!! Тому у нас такі проблеми , як тільки проведем
дерусифікацію суспільства все кардинально зміниться в кращу сторону багато
проблем зникне !!!


\iusr{Marko Rudyk}
Шкода що не можу поширити

\begin{itemize} % {
\iusr{Леся Маланюк}
Дивно, а чому? У мене налаштування кофіденційності «публічно», тому, по ідеї, мало б поширюватись. Бачу, інші це роблять без проблем.
\end{itemize} % }

\iusr{Ярослав Яніцький}
+

\iusr{Александр Колесник}
А якщо я Олександр?.....

\begin{itemize} % {
\iusr{Леся Маланюк}
\textbf{Якщо ви Олександр}, то я б зараз бачила не «Александр Колесниченко», а «Олександр Колесніченко».

\iusr{Александр Колесник}
\textbf{Леся Маланюк} Зрозумів!

\iusr{Костянтин Собіченко}
\textbf{Александр Колесниченко}, о, і я хочу побачити!)

\iusr{Леся Маланюк}
\textbf{Костянтин Собіченко} Рік минув. Ви ще маєте надію побачити?)

\iusr{Костянтин Собіченко}
\textbf{Леся Маланюк} , він зрозумів, але єму так удобнєй.)
\end{itemize} % }

\iusr{Инна Дзюба}

До чого тут імена до патріртизму???? Як вже задовбали своїм патріотизмом!!!!

\end{itemize} % }
