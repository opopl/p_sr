% vim: keymap=russian-jcukenwin
%%beginhead 
 
%%file 21_09_2021.fb.nikonov_sergej.2.bilchenko_pticy
%%parent 21_09_2021
 
%%url https://www.facebook.com/alexelsevier/posts/1586032675075375
 
%%author_id nikonov_sergej,bilchenko_evgenia
%%date 
 
%%tags bilchenko_evgenia,poezia
%%title БЖ. Птицы
 
%%endhead 
 
\subsection{БЖ. Птицы}
\label{sec:21_09_2021.fb.nikonov_sergej.2.bilchenko_pticy}
 
\Purl{https://www.facebook.com/alexelsevier/posts/1586032675075375}
\ifcmt
 author_begin
   author_id nikonov_sergej,bilchenko_evgenia
 author_end
\fi

Ещё одно стихотворение. Не смею комментировать величие Христа или Будды и не стану реагировать  на комментарии других.

\ifcmt
  ig https://scontent-yyz1-1.xx.fbcdn.net/v/t1.6435-9/242610308_1586032495075393_5133251085677024345_n.jpg?_nc_cat=105&_nc_rgb565=1&ccb=1-5&_nc_sid=730e14&_nc_ohc=sRWfKf1_1wUAX-R3stm&_nc_ht=scontent-yyz1-1.xx&oh=2ed2d5c0749dc92c56d6236f7ba2e77e&oe=616F9E8B
  @width 0.4
  %@wrap \parpic[r]
  @wrap \InsertBoxR{0}
\fi

БЖ. Птицы
Покуда добрые люди мира побеждали мирское зло,
Держа нос по ветру, ухо востро, шашки же наголо,
Он вышел нА гору, но не с наглостью, а с какой-то нагостью,
И всем было стыдно, хоть уши прячь, что Он говорил инаково.
Слово дробилось зёрнышком в птичьем клюве:
"Ибо если вы согрешения будете прощать людям, -
Падала Весть, и росток её почвой, казалось, схватится:
"То согрешения ваши простит и вам Отец".
И все Его так любили, что даже потом убили.
И все Его понимали и даже с креста снимали.
И хоронили, и причитали, рыдали навзрыд о Нём все...
И никто потом не поверил, что Он вознёсся.
А в Индии одна девочка делала пранаяму.
Она не слыхала о Сыне Божьем, но Иссу ибн Марьяма
Знавал ее дядя, купец араб: он пугал её адской карой,
Но она не верила в рай и ад: зачем, если рулит карма?
И всё-таки было что-то нежное в той угрозе.
И она пошла по русской дороге, по малорусской "дорозі",
По иудейской тропке в пустыне села Текоа...
Она не могла себе объяснить, что с ней стряслось такое.
И она видела эту гору: Китеж, оливу, Буян-остров.
И среди львов, куропаток, тигров, с Чехова книгой, Он стоял.
"Как ты простишь меня, если я сама себя не простила?" -
Спросила буддисточка из Агры и потеряла силу.
А потом спала она... Долго... Глухо... Как - от гидазепама.
И это была - не мокша. Звезда поднялась над храмом.
И не нирвана тоже. А просто - необратимость.
Золушка обратилась
С просьбой о нищем к фее...
Звезда упала.
Зло запятнало себя.
Сплюнула ил Офелия.
И все простили - и всем простилось.
Стих
Пятнадцатый
От Матфея.
20 сентября 2021 г
Иллюстрация к нему:
