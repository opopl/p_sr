% vim: keymap=russian-jcukenwin
%%beginhead 
 
%%file 23_11_2019.stz.news.ua.mrpl_city.1.golodomor_to_chto_zapomnilos
%%parent 23_11_2019
 
%%url https://mrpl.city/blogs/view/golodomor-to-chto-zapomnilos
 
%%author_id burov_sergij.mariupol,news.ua.mrpl_city
%%date 
 
%%tags 
%%title Голодомор. То, что запомнилось...
 
%%endhead 
 
\subsection{Голодомор. То, что запомнилось...}
\label{sec:23_11_2019.stz.news.ua.mrpl_city.1.golodomor_to_chto_zapomnilos}
 
\Purl{https://mrpl.city/blogs/view/golodomor-to-chto-zapomnilos}
\ifcmt
 author_begin
   author_id burov_sergij.mariupol,news.ua.mrpl_city
 author_end
\fi

\ii{23_11_2019.stz.news.ua.mrpl_city.1.golodomor_to_chto_zapomnilos.pic.1.babushka_tatjana_kirillovna}

У бабушки Татьяны Кирилловны была своя, своеобразная периодизация отечественной
истории. \enquote{У мирноє врем'я} - имелось в виду до июля 1914 года, когда, как
известно, началась Первая мировая война. Потом – \enquote{до першого голоду}, т.е. до
1921 года, который охватил тогда почти всю страну. Затем \enquote{до другого голоду} -
период Голодомора. И, наконец, \enquote{до війни з німцями}, т.е. до начала Второй
мировой войны. Бабушка, делясь воспоминаниями о страшных голодовках, всегда
говорила, что пестрое по национальному составу население Приазовья спасло море.
А точнее, тюлька. К счастью, тогда еще не догадались организовать промышленный
лов этой рыбешки, которая является главным кормом рыбы более крупной.

\textbf{Читайте также:} 

\href{https://mrpl.city/blogs/view/kak-otaplivali-doma-v-mariupole-v-byloe-vremya}{%
Как отапливали дома в Мариуполе в былое время, Сергей Буров, mrpl.city, 15.12.2018}

Бабушкино крестьянское происхождение и прожитые годы научили бережно относиться
к хлебу. Всякий недоеденный кусочек отправлялся в духовку для превращения его в
сухарь. Сухари складывали в холщовую \enquote{торбынку}, которая висела в углу рядом с
печкой. Сухари же превращались со временем в квас. Отвлекаясь от темы, можно
сказать, сколько ни довелось пробовать разных квасов, купленных в магазинах или
«на разлив», ни один из них не приносил такого удовольствия, как бабушкин
хлебный напиток. Случай из студенческих лет. В середине 50-х годов прошлого
века было принято отправлять учащихся вузов на помощь ближайшим колхозам и
совхозам. Перед одной из таких поездок бабушка сунула в самодельный
рюкзачок-\enquote{ворочечок}. Уже в колхозе обнаружилось, что это сухари. Несколько
человек из нашей группы оказались на приличном расстоянии от центральной
усадьбы. Для ночевки выделили кошару, из-за отсутствия овец наполовину
заполненную сеном. Столоваться ходили на полевой стан - своеобразный штаб
бригады на время разгара полевых работ, с временной кухней, учетчиком и т.п. Но
пошел дождь, грунтовая дорога к стану размякла, а большинство будущих инженеров
были обуты в туфли. Как \enquote{форсировать} водно-грязевые препятствия? Решили
переждать непогоду. Ждать пришлось долго. Вот тогда-то и пригодились бабушкины
сухарики. Они по-братски были распределены между пятью постояльцами колхозной
кошары...

Дополнением к познанию истории родной страны были беседы с отцом. Когда ему
было четырнадцать лет, умерла его мама. Старшему брату было шестнадцать,
младшему – двенадцать лет. До начала индустриализации было еще далеко.
Безработица стала бичом Мариуполя. Но братьям повезло. Первым стал зарабатывать
себе на хлеб старший из них. Он устроился рабочим в артель, которая
изготавливала стальные бочки. Не каждый мог выдержать постоянный грохот
молотков, которыми клепали емкости для разных промышленных нужд. Он выдержал.
Младшего взял лаборантом к себе известный мариупольский фотограф Стояновский.
Отца, как только ему исполнилось шестнадцать лет, пристроили по большому
знакомству на местную швейную фабрику, где он стал работать
грузчиком-упаковщиком. В его обязанности входило связывать в кипы готовую
продукцию, - а это были фартуки из брезента, из этого же материала рукавицы,
стеганые фуфайки-телогрейки, другую спецодежду. Потом все это грузили на
подводы, которые отвозили добро на завод Ильича и другие предприятия. Платили
за это мало. В столовке едва хватало на порцию макарон, на которых кое-где
поблескивали кусочки поджаренного лука. Он ел это блюдо с хлебом. Его более
удачливые сотрудники высмеивали. Но так было сытнее.

\textbf{Читайте также:} 

\href{https://mrpl.city/blogs/view/svetilniki-kogda-v-mariupole-ne-bylo-e-lektroe-nergii}{%
Светильники: когда в Мариуполе не было электроэнергии, Сергей Буров, mrpl.city, 08.12.2018}

Отец самой большой удачей в жизни считал, что поступил на рабфак. Его приняли,
несмотря на то что у него за плечами было лишь образование в шесть классов.
Рабфак находился в здании в то время уже бывшей Мариупольской мужской
Александровской гимназии. Сейчас там помещается Мариупольский колледж
Приазовского государственного технического университета. Преподаватели на
рабфаке были гимназические – умелые и вместе с тем очень требовательные.
Учиться было очень тяжело. Многие бросили учебу. После окончания рабфака в 1930
году папу направили в Харьковский инженерно-строительный институт, созданный на
базе строительного факультета Харьковского технологического института. Выбора
не было, рабфаковцев распределяли по вузам специальные комиссии. Но отец был
несказанно рад уже тому, что он стал студентом.

Годы его учебы совпали с Голодомором 1932 – 1933 годов. Он видел трупы людей на
Харьковском вокзале. Голодал. В студенческой столовой кормили редким супом,
состоящим из воды и гнилой капусты. Крошечная хлебная норма не могла утолить
постоянное желание кушать. Во время всех каникул работал то на строительстве
цехов завода \enquote{Азовсталь}, то на сооружении трамвайного моста через Кальчик, то
на возведении Кальчинского водохранилища. В родном городе отъедался. Отец на
собственном опыте изучил, что такое профессии землекопа, подручного каменщика,
десятника строителей. Интересно было слушать, как он и его друзья из группы
отметили окончание института. Купили водку, а денег, оставленных на закуску,
хватило только, чтобы приобрести кулек вишен. С этими покупками новоиспеченные
инженеры-строители отправились в дальнюю аллею парка, там и пировали.

\textbf{Читайте также:} \href{https://mrpl.city/blogs/view/staraya-gazeta-chto-volnovalo-mariupoltsev-bolee-100-let-nazad}{%
Старая газета: Что волновало мариупольцев более 100 лет назад?, Сергей Буров, mrpl.city, 17.11.2018}
