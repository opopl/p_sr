%%beginhead 
 
%%file 04_12_2018.fb.sosnovskij_evgenij.mariupol.1.gastroli_teatra_franka
%%parent 04_12_2018
 
%%url https://www.facebook.com/evgeny.sosnovsky/posts/pfbid0rbHrrM9VhJTmYVU84sqTGJiUAxHietXZHg8TMvGWPxs14pLNY3ex5CssUncaR8p7l
 
%%author_id sosnovskij_evgenij.mariupol
%%date 04_12_2018
 
%%tags mariupol,mariupol.pre_war,kultura,teatr.kiev.franka,gastroli,shvejk
%%title Дводенні гастролі Національний академічний драматичний театр імені Івана Франка
 
%%endhead 

\subsection{Дводенні гастролі Національний академічний драматичний театр імені Івана Франка}
\label{sec:04_12_2018.fb.sosnovskij_evgenij.mariupol.1.gastroli_teatra_franka}

\Purl{https://www.facebook.com/evgeny.sosnovsky/posts/pfbid0rbHrrM9VhJTmYVU84sqTGJiUAxHietXZHg8TMvGWPxs14pLNY3ex5CssUncaR8p7l}
\ifcmt
 author_begin
   author_id sosnovskij_evgenij.mariupol
 author_end
\fi

Не очікував побачити ці вистави у Маріуполі, але це дійсно так і вже цього
тижня у приміщенні драмтеатру пройдуть поки що дводенні гастролі Національний
академічний драматичний театр імені Івана Франка, 

6 грудня, 19.00  -  \enquote{Швейк} (за мотивами роману Я. Гашека \enquote{Пригоди бравого
солдата Швейка})

\ii{04_12_2018.fb.sosnovskij_evgenij.mariupol.1.gastroli_teatra_franka.pic.1}

Вистава \enquote{Швейк} - це більше, ніж вільна інтерпретація  роману Ярослава Гашека.
Багато в чому вона стала можливою завдяки двом неординарним акторським
особистостям: Богдану Бенюку та Анатолію Хостікоєву. Саме їх багаторічний
творчий тандем спровокував ідею створення вистави про істинну чоловічу дружбу,
про доброту та людяність, які не підвладні ані часу, ані будь-яким
випробуванням.

\ii{04_12_2018.fb.sosnovskij_evgenij.mariupol.1.gastroli_teatra_franka.pic.2}

Цю виставу неможна переповісти, вона кожного разу звучить по-іншому. І не
тільки за рахунок імпровізації, а й за рахунок чутливості та розуміння зали,
від якої актори набувають енергію творчості.

Не випадково, протягом років, \enquote{Швейк} лишається одним з найпопулярніших
спектаклів театру імені Франка.

7 грудня, 19.00 -  \enquote{Шельменко-денщик} (за п'єсою Г. Квітки-Основ'яненка)

Ця класична комедія-водевіль з успіхом грається на сцені театру Франка вже
протягом двох десятиліть. І цілком закономірно, адже тут присутні усі складові
успіху: соковитість, яскравість характерів, хитромудра інтрига, що її плете
бувалий в багатьох бувальцях Шельменко, прагнучи з'єднати закохану пару. А
поруч наївно-комічні батьки - патріархальні поміщики Шпаки і, ясна річ, жених,
що безнадійно залицяється. Усе це, помножене на акторський кураж та
багатобарвність театральних засобів, породжує веселе, дотепне не без моралі
дійство.

Тож не пропустіть цю значну подію у театральному житті Маріуполя.
