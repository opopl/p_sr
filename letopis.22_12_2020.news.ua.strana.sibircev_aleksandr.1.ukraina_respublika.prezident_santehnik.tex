% vim: keymap=russian-jcukenwin
%%beginhead 
 
%%file 22_12_2020.news.ua.strana.sibircev_aleksandr.1.ukraina_respublika.prezident_santehnik
%%parent 22_12_2020.news.ua.strana.sibircev_aleksandr.1.ukraina_respublika
 
%%url 
 
%%author 
%%author_id 
%%author_url 
 
%%tags 
%%title 
 
%%endhead 

\subsubsection{Президент-сантехник \enquote{республики Украина}}
\label{sec:22_12_2020.news.ua.strana.sibircev_aleksandr.1.ukraina_respublika.prezident_santehnik}

Сейчас же участники нового движения "народного самоуправления" пошли
дальше, чем предшественники.

На днях заявили о создании "государства Украина (республики)", избрали
"первого президента Украины", "Верховную Раду" и ее "главу", а также
создали "народное СБУ".

\ifcmt
pic https://strana.ua/img/forall/u/11/2/%D0%BF%D1%80%D0%B5%D0%B7%D0%B8%D0%B4%D0%B5%D0%BD%D1%82_%D1%83%D0%BA%D1%80%D0%B0%D0%B8%D0%BD%D0%B0.jpg
\fi

Все это было сделано на так называемом "Всеукраинском конгрессе местного
самоуправления", который прошел в Киеве.

Как рассказал "Стране" Анатолий Балахнин, который представился нам как
"Первый президент Украины (республики)", на конгрессе присутствовало 200
делегатов от 19 областей Украины.

"Были приняты решения о создании Государства Украина (республики). Помимо
провозглашения, были поданы документы в ООН и другие международные
организации для признания легитимности Государства Украина (республика)",
- сказал Балахнин.

По мнению Анатолия Балахнина, государственные органы Украины и само
государство до сих пор фактически действуют на основании Конституции 1978
года. А конституция Украины от 1996 года, дескать, была принята не на
всеобщем народном референдуме, а народными депутатами Верховной Рады 2-го
созыва. Которые избирались еще по старому законодательству и не могут
юридически считаться представителями всего народа. То есть, по логике
активистов движения "народного самоуправления", существующее государство
нелегитимно и не имеет полномочий управлять страной.

\ifcmt
  pic https://strana.ua/img/forall/u/11/2/%D1%81%D0%B0%D0%BD%D1%82%D0%B5%D1%85%D0%BD%D0%B8%D0%BA.jpg
  caption Анатолий Балахнин на фоне флага "Украины (республики)"
  width 0.5
  fig_env wrapfigure
\fi

Активисты движения приняли на конгрессе свой флаг Украины, а также
назначили сроки "внеочередных выборов" на посты "президента", в "Верховную
Раду", а также в органы местного самоуправления.

А также приняли порядок "выборов". Которые должны произойти уже через
несколько месяцев. Должности "первого президента государства Украина" и
"главы Верховной Рады", которые заняли Анатолий Балахнин и Павел Билецкий
(кстати, один из основателей движения в 2015-2016 года - Ред.), сами
активисты называют "временными". Мол, на внеочередных "выборах" будут
избраны постоянный глава государства, а также депутаты всех уровней.

Анатолий Балахнин по профессии действительно сантехник, но не видит в этом
никаких проблем. Мол, Лех Валенса также был электриком, что не помешало
ему стать главой обновленной Польши. 

Попутно с проведением конгресса активисты движения «народного
самоуправления» побывали на Майдане ФОПов в Киеве и провели с бунтующими
предпринимателями беседу, агитируя поддержать свое движение. "Стране"
активисты движения рассказали о жестком прессинге со стороны СБУ и
полиции.

"Обыскивали и прессовали почти всех наших активистов, врывались в дома.
Прессинг жесточайший. Представители региональных властей пытались не дать
добраться в Киев многим нашим делегатам", - пояснил "Стране" Павел
Билецкий.

Стоит отметить, что только агитацией сторонники нового движения не
ограничиваются. По словам Анатолия Балахнина, в регионах уже созданы
"органы местного самоуправления" всех уровней. "Есть даже прецедент, когда
один из таких советов местного самоуправления уже выдал человеку акт на
право пользования землей"", - сказал Балахнин.

Вероятно, новое движение "параллельного государства" также, как и в 2016
году, подвергнется жесткому давлению со стороны полиции и СБУ. Однако
сторонники движения готовы к уголовным делам и собираются идти вплоть до
Европейского суда по права человека.
