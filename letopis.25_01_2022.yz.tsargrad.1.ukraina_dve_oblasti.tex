% vim: keymap=russian-jcukenwin
%%beginhead 
 
%%file 25_01_2022.yz.tsargrad.1.ukraina_dve_oblasti
%%parent 25_01_2022
 
%%url https://zen.yandex.ru/media/tsargrad.tv/ukraine-ostaviat-vsego-lish-dve-oblasti-vse-ostalnoe-russkie-zemli-61efbd5ba4e98573a61f573d
 
%%author_id tsargrad
%%date 
 
%%tags istoria,ukraina
%%title Украине оставят всего лишь две области. Всё остальное - русские земли
 
%%endhead 
 
\subsection{Украине оставят всего лишь две области. Всё остальное - русские земли}
\label{sec:25_01_2022.yz.tsargrad.1.ukraina_dve_oblasti}
 
\Purl{https://zen.yandex.ru/media/tsargrad.tv/ukraine-ostaviat-vsego-lish-dve-oblasti-vse-ostalnoe-russkie-zemli-61efbd5ba4e98573a61f573d}
\ifcmt
 author_begin
   author_id tsargrad
 author_end
\fi

Ситуацию вокруг Украины нагнетают с каждым днём: \enquote{Россия нападёт на Украину. Не
сегодня, так завтра!} – любимая тема рассуждений для западных СМИ и политиков.
При этом другой рукой Запад подталкивает Зеленского к нападению на Донбасс.
Полное впечатление, что скоро там всё равно полыхнёт – как Воронья слободка в
\enquote{Золотом телёнке}, которая просто не могла не сгореть, потому что очень многие
этого хотели. Вот только переживёт ли Незалежная такую политическую авантюру
своего руководства – большой вопрос. Особенно если вспомнить, из каких земель
сложился \enquote{пазл} нынешней Украины и на какие её территории в случае чего будут
претендовать её соседи.

\ifcmt
  ig https://avatars.mds.yandex.net/get-zen_doc/4480952/pub_61efbd5ba4e98573a61f573d_61efbd8eefaff71e04617b72/scale_1200
  @caption ФОТО: VIRTIS/SHUTTERSTOCK.COM
	@wrap center
	@width 0.8
\fi

Чтобы немного разрядить столь напряжённую за последнее время обстановку вокруг
Украины, анекдот в тему, из недавнего прошлого, мало отличающегося от
настоящего:

\begin{zznagolos}
Путин звонит Порошенко и спрашивает: – Ще не вмерла Украина? Порошенко в ответ
смеётся: – Ще не вмерла. Путин: – Хорошо, я тогда попозже позвоню...	
\end{zznagolos}

Смех – смехом, но если не о предсмертных судорогах, то о смертельно опасной
болезни Украины говорить можно давно. Унаследованную от СССР мощную экономику
(60-е место в мире) Незалежная, прямо скажем, раздербанила и профукала. Госдолг
вырос в десятки раз. Население с советских 51 миллиона человек только
официально упало до 42, а фактически (по оценкам многих экспертов) – до 36, да
и то в лучшем случае. Потери в разы бо́льшие, чем в годы Великой Отечественной и
пресловутого \enquote{голодомора} вместе взятых.

Выход у украинских временщиков один – маленькая война. И не то чтобы
\enquote{победоносная}, но позволяющая \enquote{обнулить долги} и запустить на
свою территорию для \enquote{защиты от российской агрессии} войска НАТО. Так,
первый министр госбезопасности ДНР, исполнительный директор Союза добровольцев
Донбасса Андрей Пинчук в комментарии \enquote{Первому русскому} подчеркнул:

\begin{zzquote}
Сегодня имеет место высочайший уровень вероятности войны. Режим президента
Зеленского шатается, и для него война – единственная возможность удержать
власть. Но Зеленский – лишь кукла в руках Запада, которому нужна война. Чтобы
\enquote{сдержать} Россию, начавшую перезагрузку ОДКБ и дипломатическое
наступление. И преодолеть фактор раскола в Европе по отношению к России. Для
прекращения \enquote{излишней самостоятельности} ряда стран под предлогом
\enquote{атлантического единства} необходимо дружить против общего врага.	
\end{zzquote}

Вот только закончиться эта авантюра может для Украины... полным распадом этой
страны, а точнее, недогосударства.

\subsubsection{Границы Украины: откуда что взялось?}

Можно сколько угодно повторять завиральные теории одного из творцов украинского
псевдоисторического мифа Михаила Грушевского о никогда не существовавшей в
реальности \enquote{козацкой государственности}. Но факты – упрямая вещь (причём в
данном случае не только в отношении Незалежной). Практически все крупные города
Украины построены во времена нашей общей русской государственности (будь то
Киевская Русь, Российская Империя или СССР).

Так, Харьков основан царём Алексеем Михайловичем в 1630-х годах и заселялся
эмигрантами в Россию из \enquote{польского} правобережья Днепра. С городом Сумы
(основан в 1655 году) – та же история. Императрицей Екатериной II во второй
половине XVIII века основаны Днепропетровск (Екатеринослав), Луганск, Херсон,
Николаев, Одесса, Севастополь, Симферополь, Мариуполь, Кривой Рог, Запорожье.
Добавьте сюда основанный в 1754 году Елизаветой Петровной с недавних пор
именуемый Кропивницким Кировоград (Елисаветград) и Александром II Донецк
(1869), и что тогда от Украины останется?

Ах, да – Чернигов, старинный русский город, известный с X века. Но и он в
составе России был с 1503 года (исключая период 1611–1653 годов, когда его
отобрали и разорили поляки). Киев, кстати, юридически – тоже не украинский: его
поляки передали России в 1667 году по условиям Андрусовского перемирия, сначала
на время, а потом – продали за немалые по тем временам деньги (146 тысяч
рублей). Так что и здесь, пардон, это наша \enquote{нерушимая частная собственность}.

\ifcmt
  ig https://avatars.mds.yandex.net/get-zen_doc/1360848/pub_61efbd5ba4e98573a61f573d_61efbd5f3615c85db556486e/scale_1200
	@wrap center
	@width 0.8
\fi

Если же посмотреть, кем были подарены нынешней Украине (в девичестве –
Украинской ССР) её нынешние территории, получится ещё \enquote{чудесатее}. От
Царей и Императоров Всероссийских ей достались Киевская, Черкасская,
Полтавская, Винницкая, Хмельницкая, Житомирская, Черниговская и Сумская
области. Ленин (памятники которому неблагодарные необандеровцы почему-то
ломают) добавил к этому Одесскую, Николаевскую, Херсонскую, Донецкую,
Луганскую, Харьковскую и большую часть Запорожской области. Сталин прирезал к
УССР весь её запад: Львовскую, Закарпатскую, Ивано-Франковскую, Тернопольскую,
Волынскую и Ровненскую области. И, наконец, \enquote{щедрый} Хрущёв добавил
\enquote{вишенку на торте} – Крым.

Вот и выходит, что изначально то, что можно было бы назвать Украиной (точнее,
окраиной Речи Посполитой, отколовшейся при Хмельницком и присоединившейся к
России в 1654 году), это всего две нынешние области – Кировоградская (ныне
Кропивницкая) и Днепропетровская, ну и кусочек Запорожской. Всё же прочее –
подарки ненавистных царей да генеральных секретарей, которые, отрекаясь от
"колониального прошлого", вообще-то полагалось бы вернуть.

\subsubsection{\enquote{Мы делили апельсин: много нас, а он один}}

А теперь немного \enquote{повангуем}. Предположим, что, уверовав в помощь Запада,
очередной безумный \enquote{гетьман} решит всё же напасть на ДНР-ЛНР по-серьёзному. При
этом Россия вынуждена будет вступить в конфликт, защищая получивших наши
паспорта донбасских соотечественников, и конфликт этот с высокой вероятностью
закончится потерей Украиной как минимум всей Новороссии, а то и всего
Левобережья. И это запустит распад всей страны.

Румыния мечтает о возвращении Буковины, присоединённой к Украине Сталиным ещё
перед Великой Отечественной. Венгрия претендует на когда-то входившее в
Австро-Венгерскую империю Закарпатье с Ужгородом. Но больше всех на Западе
аппетиты у Польши: она хочет вернуть Галичину – Львовскую, Волынск,
Ивано-Франковскую, Тернопольскую и Ровненскую области. Поляки не забыли, что
основанным в XIII веке Даниилом Галицким Львовом они владели большую частью его
истории – 450 лет, и фактически весь его построили. Что Ивано-Франковск – это
польский Станиславов, а Луцк уже через 200 с небольшим лет после своего
основания перешёл во власть Гедиминовичей. Что Ровно – это вотчина литовских
князей Острожских, а Тернополь – польского магната Яна Тарновского…

Причём при всей демонстрируемой \enquote{любви} к Незалежной, эти господа всё понимают,
ждут неизбежного и к нему готовятся. Пока украинские власти заходятся в
праведном гневе по поводу выдачи российских паспортов в Донбассе, поляки
активно раздают на Украине \enquote{карту поляка}, которая сразу даёт право на
получение вида на жительство и работу в Польше, а через год – и паспорта. Её
обладателей ещё 4 года назад было более 100 тысяч, а сколько сегодня?

Тем же, несмотря на украинские протесты, занимается Венгрия в Закарпатье. Что
интересно, венгерские паспорта уже получили вдвое больше украинцев (по
некоторым данным – 300 тысяч), чем записано этническими венграми. Не отстают и
румыны – 100 тысяч румынских паспортов получили только в Черновицкой области.

Легко представить, что произойдёт, если при военном поражении в Донбассе или
внутриполитической смуте Украина начнёт разваливаться. Опыт отделившейся
Новороссии тут же захотят повторить у себя обладатели венгерских, румынских и
польских паспортов. А когда Украина попытается им помешать, "соотечественникам"
тут же окажут братскую военную помощь соответствующие страны.

Некоторые остряки даже шутят, что, не сумев \enquote{уйти на Запад} целиком, Украина
сделает это по частям. Вот только не выйдет ли ей это боком? К примеру, те же
доселе бредящие идеями Ржечи Посполитой \enquote{от можа до можа} (от моря до моря)
поляки подсознательно воспринимают украинцев как своих бывших \enquote{хлопов},
сбежавших вместе с землями (их они именуют \enquote{восточные кресы}) в Россию. Кстати,
и \enquote{Волынскую резню} в Польше никто не забыл, так что дайте только время и
возможность – всё всем припомнят. По словам Андрея Пинчука:

\begin{zznagolos}
В геополитической перспективе распад Украины неизбежен: слишком из разных
кусков она \enquote{сшита}. Трагедии в этом нет: объединилась же Германия, распалась
Югославия – и ничего, произошла очередная перезагрузка реальности. Для нас же
это возможность зафиксировать нужные нам реалии и исправить геополитические
ошибки прошлого.	
\end{zznagolos}

Что в таких условиях делать России, которой совершенно не улыбается появление
по западным границам этакого огромного \enquote{Гуляй-поля}, где бродят полоумные
вооружённые \enquote{нацики}? И при этом имеются четыре атомных электростанции, плюс –
чернобыльский саркофаг, невесть в каком состоянии пребывающий. При таком
раскладе даже раздел Украины с \enquote{возвращением домой} целого ряда некогда
\enquote{подаренных} территорий – единственно разумный вариант.

\ifcmt
  ig https://avatars.mds.yandex.net/get-zen_doc/5236269/pub_61efbd5ba4e98573a61f573d_61efbd5f5d05e66deb7cd265/scale_1200
	@wrap center
	@width 0.8
\fi

А значит, Россия придёт и возьмёт своё, когда-то \enquote{подаренное}. Истерика Запада
и санкции при этом, разумеется, будут. Так они и так будут – в любом случае.

\subsubsection{Что с того?}

Как справедливо заметил мудрый израильский экс-шпион Яков Кедми:

\begin{zznagolos}
Главная угроза Украине – это нарушение самой Украиной Минских соглашений.	
\end{zznagolos}

Штука, однако, в том, что Украина давно – не самостоятельный игрок, а лишь
пешка в чужой игре. Прикажут выпрыгнуть из окна – прыгнет, не задумываясь, к
чему её упорно подталкивают. Ибо чем вечно кормить, проще разделить и скушать
по частям. Так что нельзя исключить, что Украина, как Голлум из
\enquote{Властелина колец}, скоро сама сиганёт в эту пропасть, с радостным
взором глядя на Запад и вопя \enquote{моя прелесть!} Или, быть может, другой
вариант – путь домой, в состав России?
