% vim: keymap=russian-jcukenwin
%%beginhead 
 
%%file 24_08_2020.fb.fb_group.story_kiev_ua.1.boris_paton_kievljanin
%%parent 24_08_2020
 
%%url https://www.facebook.com/groups/story.kiev.ua/posts/1438573019672828
 
%%author_id fb_group.story_kiev_ua,udovik_sergej.kiev
%%date 
 
%%tags kiev,kievljane,paton_boris
%%title Борис Патон – знаменитый киевлянин
 
%%endhead 
 
\subsection{Борис Патон – знаменитый киевлянин}
\label{sec:24_08_2020.fb.fb_group.story_kiev_ua.1.boris_paton_kievljanin}
 
\Purl{https://www.facebook.com/groups/story.kiev.ua/posts/1438573019672828}
\ifcmt
 author_begin
   author_id fb_group.story_kiev_ua,udovik_sergej.kiev
 author_end
\fi

Борис Патон – знаменитый киевлянин.

19 августа 2020 года на 102 году жизни от нас ушел Борис Патон – легендарный
ученый из легендарной династии. 22 августа прошло прощание с ним и захоронение
на Байковом кладбище.

\ii{24_08_2020.fb.fb_group.story_kiev_ua.1.boris_paton_kievljanin.pic.1}

Проститься с ним пришли сотни людей, включая Президента Украины В. Зеленского с
супругой. Его достижения, порядочность и человечность, как и его легендарного
отца , нет смысла пересказывать. Борис и Евгений Патоны были символом
аристократии духа и крови, символом инноваций и устремления в будущее. Он -
автор и соавтор более 720 изобретений (500 иностранных патентов), 20 научных
монографий.  И как это ярко выделяется на фоне нашей рустикальной элиты,
которая погрязла в мелочных вопросах дележа наследия страны и устремления в
архаическое прошлое.

\ii{24_08_2020.fb.fb_group.story_kiev_ua.1.boris_paton_kievljanin.pic.2}

К сожалению, в годы независимости такие личности как Борис Патон были
отодвинуты на задний план. Героями стали олигархи и коррупционеры, которые
работая на госслужбе и не создав даже «собачьей будки», обзавелись миллионными
состояниями и виллами в престижных местах Европы. И теперь ведомая Борисом
Патоном  НАНУ станет маркером – будут ли ее огромное имущество дерибанить
коррупционеры, или она станет символом инновационного возрождения Украины.
Например, в США в науку вкладывают деньги не только государство и корпорации,
но и частные лица. Например, Harvard Management Company (фонд Гарварда)
оценивается в 41 миллиард долларов и активно управляет самым известным
университетом мира.

\ii{24_08_2020.fb.fb_group.story_kiev_ua.1.boris_paton_kievljanin.pic.3}

Почему развитие Украины как передовой инновационной страны не строится на таких
личностях, как Борис Патон? Ведь это прекрасный символ инновационной креативной
Украины. А мы, наоборот, видим, как пигмеи подобные вятровичам насаждают в
стране культ Жертвы колониальных режимов и трактовкой, что в период 1917-1991 в
Украине была только разруха и правление «злочинного режиму», к которому как раз
и относился Борис Патон - член Президиума Верховного Совета УССР (1963-1980),
член ЦК КПУ (1960-1991) и ЦК КПСС (1966-1991), награжден 4 орденами Ленина.

\ii{24_08_2020.fb.fb_group.story_kiev_ua.1.boris_paton_kievljanin.pic.4}

Поэтому писать о нем хвалебные статьи – опасно, поскольку по закону «Про
засудження комуністичного та націонал-соціалістичного (нацистського)
тоталітарних режимів» можно получить срок до 5 лет. Правда там есть нечеткие
исключения для «цитат осіб і назв», связанных «з розвитком української науки та
культури». Тогда мы должны признать, что в то время было развитие украинской
науки и культуры. И именно эту преемственность мы должны положить в основу
развития технологической, креативной, социальной и успешной Украины.

И здесь монументальная фигура Бориса Патона как нельзя лучше символизирует эту
аристократическую и креативную Украину. Будем надеяться, что Президент В.
Зеленский выполнит свое обещание и учредит престижную премию имени Бориса
Патона
