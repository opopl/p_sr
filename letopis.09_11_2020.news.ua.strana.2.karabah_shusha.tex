% vim: keymap=russian-jcukenwin
%%beginhead 
 
%%file 09_11_2020.news.ua.strana.2.karabah_shusha
%%parent 09_11_2020
 
%%url https://strana.ua/news/299764-kak-situatsija-v-karabakhe-pokhozha-na-donbass-i-smozhet-li-ukraina-zachistit-rehion.html
%%author 
%%tags 
%%title 
 
%%endhead 

\subsection{"Горловка - украинская Шуша". Как в Украине призывают к военной зачистке Донбасса по примеру Карабаха}
\label{sec:09_11_2020.news.ua.strana.2.karabah_shusha}

\Purl{https://strana.ua/news/299764-kak-situatsija-v-karabakhe-pokhozha-na-donbass-i-smozhet-li-ukraina-zachistit-rehion.html}
\Pauthor{Харченко, Александра}

\ifcmt
img_begin 
	url https://strana.ua/img/article/2997/64_main.jpeg
	caption Азербайджан заявил о взятии "сердца" Нагорного Карабаха, города Шуша, армяне отрицают 
	width 0.7
img_end
\fi

В карабахской войне, которая длится уже больше месяца, настал переломный
переломный момент.

Азербайджан заявил о взятии ключевого города Шуши (азербайджанцы называют его
Шуша), около столицы Нагорного Карабаха - Степанакерта. Армянские военные еще
вчера отрицали потерю и утверждали, что бои продолжаются. Но уже сегодня власти
НКР подтвердили, что город взят противником. 

Контроль над Шуши - это контроль над дорогой, соединяющей Карабах с Арменией.
Кроме того, город находится на возвышенности над Степанакертом и с него удобно
вести обстрел.

Говорят, "кто владеет Шуши, тот владеет Карабахом". И взятие города может
означать стратегический перелом в пользу Азербайджана, при котором падение
всего остального Карабаха будет вопросом времени.

Но интересен и другой аспект этой истории - украинский. Если Баку сумеет
вернуть себе Карабах, это будет первый на постсоветском пространстве прецедент
военного решения вопроса непризнанных территорий. Некоторые украинские
комментаторы, вдохновленные примером Баку, уже предлагают подобным образом
готовиться к военной зачистке Донбасса. 

Разбирались, что происходит в Нагорном Карабахе и как это скажется на
проблематике Донбасса и других непризнанных территорий в бывшем СССР. 

\subsubsection{"Карабах - наш". Что говорили в Баку и Ереване}

В воскресенье, 8 ноября, в Азербайджане заявили о взятии города Шуши в Нагорном
Карабахе. Об этом в обращении к нации сообщил президент Ильхам Алиев.

"С большой гордостью заявляю, что город Шуша освобожден от оккупации! Шуша
наша! Карабах наш!.. Шуша, находившаяся под оккупацией 28,5 лет, была
освобождена", - заявил Алиев.

Он напомнил, что у конфликта Азербайджана и Армении есть военное решение.

"Мы одержали эту победу на поле боя, а не за столом переговоров. Я много раз
говорил, что, несмотря на все заявления, у этого конфликта есть военные
решения... и сегодня мы доказали это" - сказал президент Азербайджана.

По его словам, "почти 30 лет бессмысленных переговоров" не привели ни к какому
результату, и древние исторические земли Азербайджана все это время "были
оккупированы".

Он пригрозил, что если Армения не освободит территории, Баку продолжит бои в
регионе.

"Мы пойдем до конца. Никто нас не остановит", - заявил Алиев.

\ifcmt
img_begin 
	url https://strana.ua/img/forall/u/0/34/124720486_3704749322915090_370549226352439991_n.jpg
	caption Ильхам Алиев. Фото: facebook.com/wwwmodgovaz
	width 0.7
img_end
\fi

С победой Баку поспешил поздравить президент Турции Реджеп Эрдоган.

"Освобождение Шуши --- символ скорейшего освобождения оставшихся оккупированных
земель Азербайджана. От своего имени, членов своей семьи и всего народа Турции
поздравляю собратьев с этой значимой победой", - цитирует Эрдогана агентство
Anadolu.

По его словам, Минская группа ОБСЕ 30 лет игнорировала интересы Баку, в то
время как азербайджанский народ сохранял терпение и наконец своими силами начал
достигать победы.

"Минская группа 30 лет не могла решить эту проблему, больше уже нельзя было
терпеть, и наконец Азербайджан добился победы. Это знак того, что общая победа
близка. Радость Азербайджана --- это и наша радость", - заявил турецкий президент
Реджеп Тайип Эрдоган.

"Шуша освобождена от армянской оккупации. Поздравляю с победой, мой дорогой
брат", - написал и вице-президент Турции Фуат Октай, обращаясь к Алиеву.

Однако в Ереване взятие города Шуши сначала опровергали. 

"Бои в Шуши продолжаются, ждите и верьте нашей армии", --- сообщил официальный
представитель Минобороны Армении Арцрун Ованнисян в Фейсбуке.

\ifcmt
pic https://strana.ua/img/forall/u/0/34/%D0%A1%D0%BD%D0%B8%D0%BC%D0%BE%D0%BA(350).JPG
\fi

Пресс-секретарь министерства обороны Армении Шушан Степанян сегодня с утра
заявила, что карабахские силы успешно решают боевые задачи в окрестностях Шуши.

"Бои в окрестностях Шуши продолжаются. Подразделения Армии обороны успешно
выполняют поставленные перед ними задачи, лишая противника инициативности", -
написала Степанян в Фейсбуке.

\ifcmt
pic https://strana.ua/img/forall/u/0/34/%D0%A1%D0%BD%D0%B8%D0%BC%D0%BE%D0%BA(351).JPG
\fi

Однако уже после обеда в понедельник 9 ноября глава Нагорного Карабаха Ваграм
Погосян признал, что город захвачен азербайджанцами. "К сожалению, цепочка
неудач пока нас преследует, и город Шуша полностью вне нашего контроля", -
написал он в Facebook.

Также он написал, что бои идут на подступах к Степанакерту. "Существование
столицы в опасности", - добавил Погосян.

\ifcmt
pic https://strana.ua/img/forall/u/0/92/%D0%BF%D0%BE%D0%B3%D0%BE%D1%81%D1%8F%D0%BD(1).png
\fi

\subsubsection{Значение Шуши}

Значение города сложно переоценить. Это древняя крепость-цитадель в горах,
стоящая на возвышении.

Сейчас она в десяти километрах от Степанакерта и прикрывает Лачинский коридор -
главную дорогу между Карабахом и Арменией. Если ее перерезать, то армянский
анклав будет, по сути, отрезан от внешнего мира и его падение станет вопросом
времени. 

В 1992 году Шуши именно с этой целью захватили армяне - чтобы обеспечить
коммуникации НКР и обезопасить столицу Карабаха от обстрелов, которые из Шуши
вел Азербайджан.

Можно уверенно утверждать, что захват Шуши предопределил победу Армении в
нагорно-карабахском конфликте 90-х годов. 

Для Баку это по той же причине была огромная потеря. А для Армении - событие,
которое празднуют ежегодно: в прошлом году в Шуши премьер Никол Пашинян устроил
в честь годовщины радостные танцы. Что вызвало возмущение президента Ильхама
Алиева.

Да и в целом для Азербайджана возвращение Шуши было наиболее заветной целью и
символом реванша за поражение почти 30-летней давности.

Интересно, что 9 ноября в Азербайджане отмечают день государственного флага. Не
исключено, что президент Алиев решил приурочить заявление о взятии Шуши к этому
празднику, который является очень важным для страны (это всеобщий выходной).

К слову, день взятия Шуши в Армении тоже носит символический характер: его
празднуют 9 мая, одновременно с Днем победы над нацизмом. В эту же дату армяне
празднуют день создания армии обороны Арцаха (так в Ереване называют Карабах). 

\subsubsection{Последствия взятия Шуши для региона}

Итак, взятие Шуши может стать крупной победой Азербайджана и вызовет коренной
перелом в конфликте. За которым просматривается два сценария - либо зачистка
всего Карабаха от армян, либо остановка на этой линии и начало мирных
переговоров. 

Пока все развивается по первому варианту, учитывая военное преимущество
Азербайджана и турецкую поддержку. Однако компромисс тоже возможен. И у него
сейчас два варианта. 

Первый - если произойдут успешные контрудары армян, а главное - они сумеют
отбить Шуши и прилегающую трассу на Лачин и Степанакерт. 

Второй - это консенсус мировых держав, опосредованно влияющих на конфликт.
Главным образом России и Турции. Вечером 7 ноября Путин созвонился с Эрдоганом.
Учитывая, что это было в разгар боев за Шуши, то понятно, что стало центральной
темой разговора. 

Официально о беседе известно немного. Но турецкие СМИ сообщили, что Эрдоган
предложил Путину создать двустороннюю группу по урегулированию конфликта вместо
нынешней, Минской площадки под эгидой ОБСЕ. И якобы Путин согласился. 

Такой опыт у Анкары с Москвой есть - речь о соглашении по сирийскому Идлибу.
Правда, там все весьма небезоблачно: в начале марта дело едва не дошло до
военных столкновений России и Турции на этой территории. 

Впрочем, здесь другая ситуация: в Азербайджане войска обеих стран не
присутствуют (по крайней мере, официально). 

На чем может быть основан компромисс - вопрос переговоров. Но очевидно, что
обязательной частью уравнения будет возвращение Азербайджану территорий,
которые Армения захватила за пределами Карабаха, чтобы обеспечить безопасность
Арцаха. Не исключено, что с последующей демилитаризацией этих буферных зон
(чтобы снять угрозу обстрелов и срыва перемирия). 

По мнению политолога Руслана Бортника, главная цель Азербайджана - хотя бы
частично занять потерянные территории, чтобы компенсировать ранее понесенные
издержки политического, экономического и военного характера.  

"На фоне экономического кризиса, который захлестывает регион, обострения могут
быть политически компенсаторным механизмом для катализации общественного
недовольства", - объясняет Бортник.

Среди политических последствий - падение Шуши может стоить кресла премьеру
Николу Пашиняну. В Армении Пашиняна, который пришел на волне Майдана 2018 года,
оппозиция обвиняет в некомпетентности по защите Карабаха и в ссоре с главным
союзником - Россией. Противники премьера намерены потребовать его отставки.

Эксперт-международник и нардеп Олег Волошин тоже связывает потери в Карабахе с
позицией Еревана. 

"Армянам давно предлагали разумный размен с этими районами. Они не хотели.
Потом армяне играли в Майданы, искали партнёрства с ЕС и США, и теперь Москва
должна таскать для них каштаны из огня? Алиев все делает грамотно - не вылазит
с российского ТВ и при этом всячески доказывает, что союз с Турцией вовсе не
тождественен курсу на вступление в НАТО", - пишет он в Фейсбуке. 

При этом Волошин считает, что если возникнет угроза уничтожения армянского
Карабаха, против Турции в Европе могут ввести санкции. Что ударит по военному
сотрудничеству Анкары с Киевом, который рассчитывает на военные контракты с
Эрдоганом. 

"Самый большой зуб на Эрдогана у французов. Чем слабее будут позиции армян в
Карабахе, тем больше армянская диаспора будет давить на Макрона действовать
жёстче в отношении Турции. А у Парижа и своих претензий к Анкаре масса. У
Зеленского об этом даже не задумывались, когда согласились быть у нового
султана на подтанцовке. Скоро ЕС введёт санкции против Турции. Это ослабит и
без того находящуюся в кризисе экономику и сделает Эрдогана более сговорчивым
на других направлениях", - написал Волошин.

Вероятные санкции против турок могут сделать последних сговорчивее и по вопросу
Карабаха. Что также поспособствует сценарию компромисса. 

\subsubsection{Карабах и Украина}

Взятие Шуши вызвал восторженные комментарии многих в Украине. Начали
раздаваться голоса, что Азербайджан показал, как Украина должна вернуть себе
территории Донбасса - военным путем и при поддержке союзников.

"Нам очень наглядно показали, как следует побеждать на Донбассе вооруженные
формирования, оснащенные по принципам 1970 годов.

"Барьяктары", сотрудничество с Турцией и Азербайджаном, собственный опыт и
возможности - это все у нас уже есть или может быстро появиться", - считает
журналист Денис Попович.

\ifcmt
pic https://strana.ua/img/forall/u/0/92/%D0%B8%D0%B7%D0%BE%D0%B1%D1%80%D0%B0%D0%B6%D0%B5%D0%BD%D0%B8%D0%B5_2020-11-09_122022.png
\fi

О том же говорит и националистка Олена Билозерская в своем видеоблоге. 

"Смотрю новости о войне в Нагорном Карабахе и думаю… угадали, о Донбассе думаю.
Что-то наши соотечественники много беспокоятся об абсолютно надуманной проблеме
– "ой, а стоит ли вообще возвращать тот Донбасс, а там же так много "ваты",
которая ненавидит все украинское, что мы будем с ними делать, а они же сначала
проведут выборы, потом проведут своих людей в Верховную Раду. Ой, там же за эти
годы столько выросло детей, которые уже ненавидят Украину.

Друзья мои, это вообще не проблема. Посмотрите, как правильно сейчас делает
Алиев --- он говорит, что обещает армянам полную безопасность и сохранение их
прав. Армяне их внимательно слушают и убегают. Как вы думаете, через трое суток
после того, как украинская армия возьмет Горловку --- это наша украинская Шуша,
как вы думаете, сколько "ваты" еще останется в Донецке, а сколько сбежит? 50 на
50, 30 на 70, 20 на 80?

Это все решается просто и элементарно, нужно только иметь других союзников –
тех, кто обеспечивает "Байрактарами", а не хвалят нас за миролюбие".

\Purl{https://youtu.be/93IFh2z6WFA}

В Украине такие рассуждения от "партии войны" появились не вчера. Но после
появления новостей об успехах Азербайджана под Шуши они зазвучали по-иному.

И если Азербайджан действительно сможет одержать военную победу в Карабахе,
идеи военного решения вопроса Донбасса перейдут в политический мейнстрим. Ведь
Баку таким образом покажет первый на постсоветском пространстве прецедент, как
можно возвращать неподконтрольные территории военной силой (аналогичная попытка
у Саакашвили с Южной Осетией, напомним, закончилась в 2008 году провалом). 

Но применима ли эта аналогия к Донбассу?

В лице Армении, которая контролирует Карабах, Азербайджан имеет сопоставимого
военного противника. И даже менее технически оснащенного, что показала ситуация
с ударными беспилотниками, которые в первые дни войны нанесли решающие потери
армянским войскам. То есть Баку начинал войну с Ереваном, зная, что его армия
объективно слабее. 

Также Алиев понимал, что Россия открыто заступаться за Карабах не будет.
Во-первых, потому что российских интересов там нет. Во-вторых - в силу
дружеских отношений с Баку и партнерских - со стоящей за азербайджанцами
Турцией. В-третьих, Москва сама убеждала Армению поступиться территориями,
которые были ранее захвачены у Азербайджана. Но Ереван не согласился.
В-четвертых, лично премьер Пашинян не является верным союзником Москвы и
ориентируется скорее на страны Запада. В-пятых, у России нет общей границы с
Арменией и Карабахом, а потому даже число технически и логистически военную
помощь им Москве оказывать было трудно.

Исходя из этих стратегически важных вводных, Азербайджан начал войну и добился
существенных успехов (хотя и об окончательной победе говорить еще рано). 

Теперь сравним с непризнанными "республиками" Донбасса. За ними стоит
непосредственно Россия - многократно превосходящая Украину военная держава.
Которая уже показывала, что введет войска в регион, если будет риск военного
поражения "ДНР" и "ЛНР" (так было в августе 2014 года в Иловайске). 

В этом смысле получение "Байрактаров" или других вооружений вызовет накачку
современным оружием и с той стороны (причем, скорее всего, оружие будет
поставляться сразу с войсками). И вместо зачистки Донбасса по сценарию "партии
войны", Украина получит большую войну на своей территории, которая быстро
выплеснется за пределы "ДНР" и "ЛНР", приведя к потере новых территорий и,
возможно, к фатальным последствиям для украинской государственности. 

Все это понимают и на Западе, который именно поэтому продавил Минские
соглашения и выступает против обострения войны на Донбассе. Ведь это чревато
для США и Европы либо потерей Украины, либо их непосредственным вовлечением в
войну с Россией (и того и другого Западу бы очень не хотелось). 

Тем не менее "горячие головы" есть не только в Украине. В других странах, где
стоит проблема сепаратизма, могут воспринять возможный успех Баку как
руководство к действию.

То же Приднестровье общей границы с Россией или ее союзниками не имеет. И в
случае победы на выборах президента Молдовы прозападной Майи Санду, есть риск
того, что конфликт попытаются разморозить. 
