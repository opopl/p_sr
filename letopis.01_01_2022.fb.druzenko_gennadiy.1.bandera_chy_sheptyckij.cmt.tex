% vim: keymap=russian-jcukenwin
%%beginhead 
 
%%file 01_01_2022.fb.druzenko_gennadiy.1.bandera_chy_sheptyckij.cmt
%%parent 01_01_2022.fb.druzenko_gennadiy.1.bandera_chy_sheptyckij
 
%%url 
 
%%author_id 
%%date 
 
%%tags 
%%title 
 
%%endhead 
\zzSecCmt

\begin{itemize} % {
\iusr{Nikita Kimovish}
Дякую. Я підозрював, що саме так і було, проте не знав, що так трагічно.

\iusr{Slavko Polyatynchuk}
Генадію, я не знайшов у Шептицького жодної згадки про причетність Бандери до загибелі Бабія

\begin{itemize} % {
\iusr{Yaroslav Znych}
\textbf{Slavko Polyatynchuk} 

\enquote{Чи не найнапруженішим конфлікт між Шептицьким та молодими націоналістами був у
1934 році, коли вони вбили Івана Бабія, директора української академічної
гімназії. Шептицький різко їх засудив: @igg{fbicon.exclamation.mark} «[Я]кщо хочеться зрадливо забити тих,
які противляться вашій роботі, доведеться вам вбити усіх вчителів та
професорів, які працюють для української молоді, всіх батьків і матерів
українських дітей». Він нагадав їм, що «злочин є завжди злочин» і «що святій
справі не можна служити закривавленими руками» (цит. за Лисенко 2000: 38-39). В
українському суспільстві цю заяву митрополита сприйняли як надзвичайно сильне
засудження діяльності націоналістів (Паньківський 1965: 140).  @igg{fbicon.face.worried} У відповідь
ОУН, за мовчазної згоди іншого єпископа спробували усунути Шептицького з
престолу митрополита 4.}

\end{itemize} % }

\iusr{Богдан Панкевич}

Шептицький завжди засуджував злочини, особливо - вбивства. Вперше різко засудив
у 1908 році М.Січинського за вбивство намісника Потоцького. Що терором не можна
служити українській справі. Але часто голос Шептицького волав у пустелі
примітивізму і хибного розуміння патріотизму. Січинський на еміграції потім був
великим другом злочинного радянського режиму, показав своє нутро. Але це не
зупинило придурків, які декілька років тому повісили йому меморіальну таблицю
на будинку Львівської облради. На жаль Церква не популяризує величезну частину
спадщини Шептицького бо не може (чи не хоче) дієво слідувати його заповітам у
соціальній сфері, розбудові громадянського суспільства, економічному розвитку,
кооперації і т.д.

\begin{itemize} % {
\iusr{Gennadiy Druzenko}
\textbf{Богдан Панкевич} на жаль...

\iusr{Богдан Панкевич}
\textbf{Gennadiy Druzenko} 

А що тепер робити, коли термін "бандерівець" означає прихильника незалежної,
антимосковської України? Тому я, все знаючи і розуміючи - бандерівець.

\iusr{Gennadiy Druzenko}
\textbf{Богдан Панкевич} 

як казав великий українець Микола Зеров, ad fontes - повернутись до джерел.
Врешті-решт образ Бандери як головного борця з російщенням нав’язав нам Кремль.
І це спонукає до рефлексій.

\iusr{Nikita Kimovish}
\textbf{Богдан Панкевич} 

Да, но не забывайте, что большая часть украинцев относится к культу Бандеры
достаточно прохладно, однозначно негативно - приблизительно процентов сорок.
Что делать? Кстати, вы, надеюсь, в открытую войну верите? Если да, сообщите,
когда начинается в этом году или мне поездку на Крит планировать?

\iusr{Yaroslav Znych}
\textbf{Богдан Панкевич} можете бути націоналістом-патріотом @igg{fbicon.face.monocle} 

\iusr{Богдан Панкевич}
\textbf{Nikita Kimovish} 

Дуже багато українців знає власну історію у російсько-радянській брехливій
версії. І не лише українці - он президент Ізраїлю наговорив дурниць у нашій
Верховній Раді, їхні політики також у полоні старої радянської пропаганди. В
найкращих університетах світу донедавна вивчали історію Росії на основі
фальсифікату ще часів Катерини Другої. Потім так навчені студенти ставали
експертами і урядовцями. Тому ще не скоро світ і самі українці зрозуміють
правду і повірять у неї.

\iusr{Богдан Панкевич}
\textbf{Yaroslav Znych} 

У нас націоналізм означає те саме, що в інших країнах - патріотизм. Бо інші
країни не переживали такого неймовірного геноциду титульної нації.

\iusr{Nikita Kimovish}
\textbf{Богдан Панкевич} 

Пан Богдан. Вы же в университете учились. И в советское время жили, коммунистом
были, кстати? 1. Украина была столбом, причем подпорным Империи. \enquote{Родина
застоя} 70 процентов сержантско-старшинского состава РККА а после СА составляли
этнические украинцы. 40 процентов состава ВМФ - украинцы. Какой геноцид? Я уже
не говорю о культуре, о промышленности на Украине. Ваша позиция мне непонятна,
коль вы, судя по всему, образованный человек и свидетель былого расцвета
Украины. Ну, а если хотите жить в Лемберге.... Тут я вам не помощник.

\iusr{Богдан Панкевич}
\textbf{Nikita Kimovish} 

росіяни знищили у ХХ столітті близько 20 мільйонів українців. Що не завадило
іншим українцям стати опорою імперії. Так само як колись їхні предки були
економічною опорою Речі Посполитої. Зараз маємо свою державу і у ній має бути
своя правда. Не російська і не польська. Оскільки ми досить близькі культурно
до поляків та росіян то для утвердження власної ідентичності мусимо поставити
між собою і цими народами максимально високу стіну. Будь яка близькість до
росіян загрожує незалежності України. Росія розпочала війну тому, що не бачила
інших можливостей втримати Україну під своїм впливом.

\iusr{Yaroslav Znych}
\textbf{Nikita Kimovish} 

1) \enquote{...70 процентов РККА/СА, ...40 процентов состава ВМФ - украинцы.} Дозвольте
уточнити: були не українці, а укрАинцы, а частіше \enquote{хахлы}... А скільки \% було
серед старших офіцерів?! @igg{fbicon.no.entry} ️ "...Анализ национального состава старших офицеров и
генералитета РККА демонстрирует, что чем выше было воинское звание, тем выше
был удельный вес русских.

Так, среди майоров русских числилось 66,83\%, среди подполковников — 68,86\%,
среди полковников — 73,47\%, а среди генералов в конце войны — 77,45\%.!?

2) культуру представляли Хор Верьовки і Ансамбль Вірського, а \enquote{настоящая}
культура - опера, балет були в Москві і Лєнінграді. 

3) промисловість з усім екологічним негативом була по всій УРСР і вишенька на
тортику - ЧАЕС поруч з Києвом...

\iusr{Богдан Панкевич}
\textbf{Nikita Kimovish} 

побачив, що Ви професор українського університету. І тут коментуєте російською
мовою. Поширюючи російську пропаганду. Це огидно. Припиняю дискусію.

\iusr{Yaroslav Znych}
\textbf{Богдан Панкевич} 

помиляєтеся, цей пан - професор колишнього \enquote{советского университета}...
@igg{fbicon.laugh.rolling.floor} 

\iusr{Nikita Kimovish}
\textbf{Богдан Панкевич} 

.....богданчик, ты уже старик, а уме как не было так и нет. Все равно ты ничего
не решаешь, \enquote{патриот} немощный. \textbf{Yaroslav Znych} славочка, я с ботами не
разговариваю. Принцип. Ты, кстати, мальчик или девочка, цветок запоздалый.

\end{itemize} % }

\iusr{Vitaliy Kolomiets}

Це не зрада Шептицького.
Зрада це протиставляти Шептицького Бандері.

\begin{itemize} % {
\iusr{Gennadiy Druzenko}
\textbf{Vitaliy Kolomiets} 

насправді або інклюзивна Україна Шептицького, сповнена любові та мудрості, або
ексклюзивна Україна Бандери, яка проростає з ресинтементу та віри у прості
розв‘язки складних проблем.

Обирати кожен має сам. Бо дякувати Богові, ми живемо в державі, де страчувати за погляди — моветон.

Врешті-решт, це прямо випливає зі слів Шептицького, які я цитую.

\iusr{Vitaliy Kolomiets}
Це протиставлення типу: або старі або молоді.
(І далі типу Tertum non datur).
Це хибне твердження.
Як хибне твердження з позиції інтересів нації протиставляти Мельника і Бандеру.

\iusr{Gennadiy Druzenko}
\textbf{Vitaliy Kolomiets} 

йдеться ж не про вік, а про вміння мислити стратегічно. Це засадниче питання
про устрій українського дому: опонент, який також любить Україну, але бачить її
розвиток по-іншому, - це ворог, що підлягає знищенню, чи побратим? Бандеризм
відповідає: «ворог» і тим самим веде визвольну боротьбу до поразки.

\iusr{Vitaliy Kolomiets}
\textbf{Gennadiy Druzenko} 

вміння мислити, як і досвід мислення змінюється з віком. Ти робиш узагальнення
о окремого звернення Шептицького чи його товариша Мельника до хлопця, якому
станом на 1934 рік лише 25 роки.

Думаю, що 40 річний Бандера не менш завзято зміг би подискотувати з собою
25річним. Як і кожен з нас.

\iusr{Gennadiy Druzenko}
\textbf{Vitaliy Kolomiets} 

вірогідно. Але відомо, що Бандера до смерті залишився на позиціях
ексклюзивності та авторитаризму і був переконаний, що велика мета виправдовує
будь-які засоби. Я йому не суддя, але він точно не приклад для наслідування для
сучасної України. Непересічний історичний діяч - так. Герой вартий наслідування
- ні. І тому мене непокоїть саме глорифікація Бандери в сучасній Україні.

\iusr{Vitaliy Kolomiets}
\textbf{Gennadiy Druzenko} 1954

\ifcmt
  ig https://scontent-mxp2-1.xx.fbcdn.net/v/t39.30808-6/270778683_5306105742752141_4024062333731299057_n.jpg?_nc_cat=101&ccb=1-5&_nc_sid=dbeb18&_nc_ohc=rvs8k8UUwtsAX-gsc7U&_nc_ht=scontent-mxp2-1.xx&oh=00_AT_dbQ_v_3N4hzS99tFSr34xPfXTTRNE7I8ZTXrK_nXtWQ&oe=61D8CF56
  @width 0.2
\fi

\iusr{Микита Василенко}
\textbf{Vitaliy Kolomiets} Да. Прочел вашу полемику. Не убедили. Кстати, будьте корректнее по отношению к Друзенко. Ваше мнение не есть истиной, во всяком случае для участников дискуссии.

\iusr{Yaroslav Znych}
\textbf{Vitaliy Kolomiets} як мінімум треба знати про позиції обох фігурантів і криваві наслідки протистояння. Через майже 82 роки вже час зробити висновки, визнати помилки і не займатися взаємо поборюванням.

\end{itemize} % }

\iusr{Роман Химич}

Говорячи про спадщину Шептицького, як не згадати його позицію щодо ритуалів ОУН
(обох крил одразу), висловлену в листопаді 1941 року.

Акти Митрополичого Ординаріяту

Док. № 17

Ч. 78

Чит. 11.ІХ.1941

Тризуб без хреста

Не знати з якої ініціативи і для якої причини укр[аїнські] патріоти носять
часом малу відзнаку тризуба без хреста. Коли св. Володимир і по ньому усі
укр[аїнські] князі уживали яко державного знамени тризуба з хрестом, треба
признати, тризуб без хреста за символ повороту до поганства і за сумний признак
переваги безбожницьких течій серед нашої нещасної суспільності.

Може бути, що приклонники того поганського символа відкликуються на те, що
такого тризуба уживала укр[аїнська] Центральна рада з 1917 і 1918 р. Таке
оправдання не мало би ніякої рації. Центр[альна] рада допустилася много ошибок
і, зокрема, під впливом засідаючих в ній безбожників спричинила національну
руїну і помогла большевикам запанувати над Україною. Хотіти повторювати її
ошибки було би нерозумною політикою.

Подібним проявом безбожництва буває заступлювання словом \enquote{Слава
Україні} відвічну похвалу віддавану Христу: \enquote{Слава І[сусу] Христу!}.
Очевидно, ніхто з українців не може мати нічого проти поклику \enquote{Слава
Україні}, але тим словом заступити акт релігійного прославлення Христа є
виразною тенденцією усунути Христа і поставити батьківщину на його місце,
значить є ознакою виразної безбожницької тенденції, що обманює наївних
українських патріотів.

Україна не може взагалі існувати як самостійна держава, а тим менше може бути
славною державою без волі Царя царствующих і Господа господствующих Предвічного
Бога І. Христа, нашого Спасителя, Батька і Пана. Взиваю усіх християн єпархії
поборювати ті два прояви безбожництва в практиках укр. патріотизму.

Від М.О.

У Львові 6.ІХ.1941

† А.

Джерело: \url{http://www.ji.lviv.ua/n29texts/akty.htm}

\iusr{Yaroslav Znych}

Бандера - суперечлива фігура, яка \enquote{містить} як позитив, так і негатив. Його
ім'ям можуть привести на барикади і в окопи тисячи патріотів і, в той же час,
розділити на два протилежні/ворожі табори десятки, а можливо, і сотні тисяч
пересічних громадян... Позиції Митрополита Шептицького і Бандери (радикальних
націоналістів) дуже часто, м'яко кажучи, не співпадали. Більше тут

\href{https://commons.com.ua/uk/andrej-sheptitskij-ta-banderivskij-ruh}{%
Християнство та радикальний націоналізм: митрополит Андрей Шептицький та бандерівський рух, %
Химка Іван-Павло, commons.com.ua, 02.01.2015%
}

\begin{itemize} % {
\iusr{Nikita Kimovish}
Подписчики: 24 человека \textbf{Yaroslav Znych}. Слава, кто вы по гендлерному признаку, если не секрет?
\end{itemize} % }

\iusr{Slavko Polyatynchuk}

Генадію, ... людина яка ніколи не воювала проти СССР ???... так СССР воював
проти Бандери до його фізичного знищення ...  ти б ще Сташинському лайк
поставив .. якось так тексти твої складаються не по нотах

\begin{itemize} % {
\iusr{Gennadiy Druzenko}
\textbf{Slavko Polyatynchuk}, друже, важко воювати проти СРСР, сидячи в Мюнхені. Стосунки Бандери з запіллям в Україні були складні і суперечливі. А вбивство провідника навряд чи сильний аргумент. За такою логікою Литвиненко та Срипаль воювали з Росією.
\end{itemize} % }

\iusr{Pavlo Zhovnirenko}

1. В Українському Пантеоні мають бути і є не «або - або», а «і - і»: і
Шептицький, і Бандера.

2. Бути ЗАРАЗ з нинішніх позицій суддями тих, хто жили У МИНУЛОМУ - принаймні
некоректно.

3. Джузеппе Ґарібальді був терористом для австрійців та французів, з якими він
боровся за незалежність Італії. Менахем Бегін був терористом для Великої
Британії, проти якої він воював за створення Ізраїлю. Але для італійців та
євреїв вони - безперечні герої.

4. Кожна порядна людина, яка знає дійсну біографію Степана Бандери, не може не
схилити в пошані голову перед цією особистістю, яка горіла ідеєю свободи Людини
й Нації і заради цього пройшла через тюрми й табори окупантів, створила потужну
організацію та армію, які без жодної підтримки ззовні боролися і врешті-решт
домоглися перемоги на першому етапі боротьби українців за свою державу.

Він та його побратими і зараз, на другому етапі, говорять всім нам: попри все -
дій, \enquote{бо плач нікому ще не дав свободи!}.

Вічна пам'ять!

\begin{itemize} % {
\iusr{Роман Химич}
\textbf{Pavlo Zhovnirenko} 

про яку свободу Людини може йти мова в контексті державного ідеалу Бандери, а
саме однопартійної ідеологізованної диктатури із пожиттєвим диктатором на чолі?
Ви не в курсі тієї параної, яка культивувалася в ОУН(р), безкінечного пошуку
ворожих елементів? Терору, спрямованого проти усіх одразу, включно із етнічними
українцями?


\iusr{Pavlo Zhovnirenko}
\textbf{Roman Khimich} 

1. Гасло було? Було.

2. Параноя була, але повторюю: з позицій сьогодні судити ні ви, ні я нікого з
тодішніх патріотів не маємо права - хоча б тому, шо вони не можуть ні
опонувати, ні захищатися. Вони - з плюсами, які незрівняно перевершують їхні
мінуси - однозначно у списку Героїв Нації. Вам це не подобається? Путіну теж.
Будете в дуеті?

\iusr{Gennadiy Druzenko}
\textbf{Pavlo Zhovnirenko} так ніхто і не судить. Мені глибоко огидний сучасний культ Бандери, а не він сам як історичний персонаж. Бо цей культ - примітивізує та нищить Україну зсередини.

\iusr{Pavlo Zhovnirenko}
\textbf{Gennadiy Druzenko} А що ж це у дописі, як не суд?! А - «вміння критично поглянути на своє минуле». ну да, ну да....
Я не бачу в Україні культу Бандери, у мене - просто повага до нього, можливо, трохи гіпертрофована - але виключно через а)війну з Росією і б) триваючі і вічні намагання Москви знищити будь-які точки кристалізації нашої нації - від символів до героїв.

\iusr{Yaroslav Znych}
\textbf{Pavlo Zhovnirenko} "2. Бути ЗАРАЗ з нинішніх позицій суддями тих, хто жили У МИНУЛОМУ - принаймні некоректно." 1)тут ніхто нікого НЕ судить 2)прояснити позиції Шептицького і Бандери на підставі документів, наслідків вчинків, зокрема, вбивства політичних/ідеологічних опонентів/соратників та дати їм оцінку - наш обов'язок, щоби сьогодні, в нових реаліях, не розділяти, а єднати Україну.

\iusr{Pavlo Zhovnirenko}
\textbf{Yaroslav Znych} сьогодні у нас найголовніша реалія - війна з Москвою. Переможемо - тоді й подискутуємо. А до того - як раз і робити розкол, в який Москва все глибше вставлятиме свого чобота.

\emph{Yaroslav Znych}
\textbf{Pavlo Zhovnirenko} 

згоден, \enquote{сьогодні у нас найголовніша реалія - війна з Москвою.} Для перемоги
потрібна єдність країни @igg{fbicon.exclamation.mark} Героїзуючи і міфологізуючи Бандеру, ми продовжуємо
розкол у суспільстві або, щонайменше, створюємо напругу  @igg{fbicon.face.worried} .

\iusr{Роман Химич}
\textbf{Pavlo Zhovnirenko} 

1. В СРСР, якщо ви, раптом, застали ті часи, гасла теж були нівроку. І про
свободу Людини також, а як же. І що це доводить?

2. Якщо керуватися вашою логікою, не можна судити ані Бандеру, ані Сталіна, ані
Щербицького. Я виходжу з того, що християнське вчення дає надійні моральні
критерії та дороговкази, які дозволяють оцінювати будь-які вчинки та політичні
практики минулого сторіччя. Або треба голосно заявити, що Христос вам не цейво.
Тоді все буде і логічно, і послідовно.

3. Що за плюси в Бандери із Арсеничем? Чого спромоглися?

4. Використання Reductio ad Hitlerum (ad Putinum в нашому випадку) саме по собі
є яскравою та вичерпною характеристикою людини. Як і побутова телепатія,
наприклад.

\iusr{Pavlo Zhovnirenko}

2. Прошу мою цитату, яка дала вам підстави стверджувати, що «за моєю логікою не
можна судити Сталіна чи Щербицького».

3. Воювали за Україною проти однієї з світових потуг без шансів на перемогу -
воювали, не здалися. Як приклад для нас, нинішніх, що воюють з нею зараз із
шансами на перемогу значно більшими.


\iusr{Yaroslav Znych}
\textbf{Pavlo Zhovnirenko} 

до речі, виходячи з контексту вашого допису, я теж хотів написати про Сталіна і
тп, але вирішив не загострювати дискусію @igg{fbicon.face.nerd} 


\iusr{Yaroslav Znych}
\textbf{Pavlo Zhovnirenko} путін, як ніхто інший зацікавлений у міфологізації Бандери. Здивовані? Тож ще невідомо, хто кому підіграє (по факту, а не за благими намірами) і співає у дуеті... @igg{fbicon.wink} 

\iusr{Pavlo Zhovnirenko}
\textbf{Yaroslav Znych} а де у мене про міфологізацію Бандери? Не чіпляйте мені свої придумки, щоб героїчно з ними боротися.)

Повторюю вище сказане:

Я не бачу в Україні культу Бандери, у мене - просто повага до нього, можливо,
трохи гіпертрофована - але виключно через а) війну з Росією і б) триваючі і
вічні намагання Москви знищити будь-які точки кристалізації нашої нації - від
символів до героїв.


\iusr{Роман Химич}
\textbf{Pavlo Zhovnirenko} 

1. \enquote{з позицій сьогодні судити ні ви, ні я нікого (...) не маємо права -
хоча б тому, шо вони не можуть ні опонувати, ні захищатися}.

2. Воювали за абсолютно виразний проект облаштування України - однопартійну
ідеологізовану диктатуру із пожиттєвим вождем на чолі. Без профспілок, євреїв
(щонайменше), вільної преси і будь-якої опозиції взагалі. Для чого українцям
було таке щастя - питання для мене риторичне.

3. Готовність воювати без шансів на перемогу як і готовність вбивати та вмирати
самі по собі для мене не є позитивною характеристикою. Будь-який ІДІЛ або
терористичний рух займався та займається тим самим. Навіжених на українських
теренах завжди вистачало.

4. Теза про \enquote{не здалися} виглядає не дуже переконливо, зважаючи на долю
і Василя Кука, і Степана Андрійовича.


\iusr{Pavlo Zhovnirenko}
\textbf{Roman Khimich} це, вибачте, ница підтасовка - замість «нікого з патріотів» ставити «нікого ...». Далі не бачу сенсу марнувати свій час. Бувайте, маніпуляро.

\iusr{Роман Химич}
\textbf{Pavlo Zhovnirenko} я вже зрозумів, що для вас \enquote{патріот} це індульгенція.
\enquote{Центр Стратегічних Досліджень}, OMG

\end{itemize} % }

\iusr{Pavlo Mryhlockyi}

Ну. У випадку війни за виживання Ґанді - не кращий герой, що надихатиме
боротися зі звірями, в цьому випадку росіянами. Треба просто визначитись з
пріоритетами: бути праведниками, котрих замочили, щоб звільнити ресурси для
власного споживання, чи вижити, і розвиватись, відбираючи ресурси в інших, якщо
свої дорожче виробити? Так збудовано цю систему життя? Кращий той, хто вижив.
Будь-якими способами.

\begin{itemize} % {
\iusr{Gennadiy Druzenko}
\textbf{Pavlo Mryhlockyi} 

Ґанді виграв, а Бандера програв. Але навіть не в цьому справа. Вийдіть на
вулицю і озирніться навколо - Україна доконаний факт. І саме тому
ексклюзивно-авторитарний світогляд наразі контрпродуктивний. Зараз ми маємо
вірити: всі, хто не проти нас, - з нами. Бандера вірив навпаки.

\iusr{Pavlo Mryhlockyi}
\textbf{Gennadiy Druzenko} . 

Хм. Ґанді мав справу з британцями. Ми - з святим народом, що охороняє всесвітню
духовность. Тобто таким, що ні разу не зупиниться перед геноцидом неслухняних
священній волі  @igg{fbicon.smile}  Бандера програв але його організація
трималась 20 років проти одного з переможців світової війни. Україна - не є
доконаний факт на сьогодні.  Адже є сусідні держави, що ставлять її під сумнів
і одна з них готова збройно знищити її.

Те, що ми когось визнаємо національним героєм, не означає, що ми приймаємо його
ідеологію. Те, що прем'єр-міністр Японії вшановує пам'ять Хдейоші Тойотомі, не
означає, що Японія планує повернутися до сьогунату. І тим більше не означає, що
планується грузити кораблі вухами, відрізаними у корейців...

\iusr{Роман Химич}
\textbf{Pavlo Mryhlockyi} у вас дивні уявлення про британців ))

\iusr{Pavlo Mryhlockyi}
\textbf{Roman Khimich} . Цілком реалістичні. По жорстокості та злобі вони - діти перед москалями....

\iusr{Роман Химич}
\textbf{Pavlo Mryhlockyi} популярна и зручна точка зору

\ifcmt
  ig https://focus.ua/static/storage/thumbs/1088x/d/3c/fbc51191-58b1a99f7a9f48d8ab5bf009424073cd.png?fbclid=IwAR1RIeCiOCwEXeqGhuReG6GHPL_4gUxRg-j2thO2lEU4YnM8l4F_X6boIEs
  @width 0.4
\fi

\iusr{Роман Химич}
Досліджував це питання, результати невтішні

\href{https://focus.ua/opinion/opinions/459279-irlandskii_golodomor_pochemu_ukrainskie_uchebniki_istorii_sochuvstvuiut_britanskomu_imperializmu}{%
Ирландский Голодомор: почему украинские учебники истории сочувствуют британскому империализму, %
Роман Химич, focus.ua, 21.07.2020%
}

\iusr{Pavlo Mryhlockyi}
\textbf{Roman Khimich} . І що, перевершили вони творіння Сталіна і Кагановвича в Україні 1932 - 1933 років?

\iusr{Роман Химич}
\textbf{Pavlo Mryhlockyi} невже наведеного мною графіка недостатньо?

\end{itemize} % }

\iusr{Аліна Подолянка}

Моїй підсвідомості не подобається зовнішність Бандери. Типаж схожий зовнішньо з
Путіним. Ну от є таке) А взагалі то дуже серйозну тему Ви підняли. І тут справа
не у Бандері. Замість того, щоб запропонувати поцінювачам Росії багату,
процвітаючу, справедливу, без корупції Україну патріоти їх принижують... і який
результат? Ми живемо в інфобульбашці патріотичній. Молодь здебільшого
аполітична. І оцей розумний пласт... айтишники, яких в Україні, слава богу,
багато.. вони космополіти... мріють жити в Пітері, бо там атмосферно... культурна
столиця, андеграунд... для тих хто не в собі. А що можемо запропонувати ми їм?
Смолоскипна марші? Тут не про гроші мова... Тут про спосіб життя і майбутнє.

\iusr{Zinoviy Svereda}

Шептицький і Бандера - це дві різні постаті. Це як казати: хто сильніший,
баскетболісти чи гандболісти. Це різні напрямки.

\end{itemize} % }
