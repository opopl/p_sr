% vim: keymap=russian-jcukenwin
%%beginhead 
 
%%file people.palamarenko_anatolij_chtec_ukr
%%parent people
 
%%url 
 
%%author 
%%author_id 
%%author_url 
 
%%tags 
%%title 
 
%%endhead 
\section{Паламаренко Анатолий}
\label{sec:people.palamarenko_anatolij_chtec_ukr}

\begin{longtable}{|p{0.45\textwidth}|p{0.45\textwidth}|}

\citTitle{Герой України, народний артист Анатолій Паламаренко: 
«Буде мова — буде нація, буде нація — буде держава»}, 
Ліна Тесленко, www.umoloda.kiev.ua, 02.07.2021

Герой України, народний артист Анатолій Паламаренко: «Буде мова — буде нація,
буде нація — буде держава»

Анатолій Паламаренко — давній шанувальник і передплатник «України молодої».

Бо думками і почуттями з нашою газетою він «на одній хвилі».

А ще «УМ» і Анатолій Паламаренко борються за Україну спільною зброєю — словом. «Боротимусь за Україну поки ноги носять.

І навіть на спині боротимусь!» — каже народний артист. Тож у рамках проєкту
«Україна молода: 30-й рік Незалежності» говоримо з паном Анатолієм одразу про
два ювілеї...

& 

\citTitle{На Украине интеллигенция подняла голову. Запахло перегаром и не свежими мыслями...}, 
Вечер у камина, zen.yandex.ru, 05.07.2021

\enquote{Мы получили независимость, но мы не получили Украину}, - сетует
Анатолий Несторович Паламаренко. Не абы кто, а прямо целый народный артист
Украинской Советской Социалистической Республики.

Товарищу нынче 83 года, но он бодр, весел, полон сил и гноя. \enquote{Потому что
мыслями и чувствами он с нашей газетой, он с нами на одной волне}, - говорят в
редколлегии издания \enquote{Украина молодая}, разбрызгивая освежитель воздуха послу
ухода мэтра.

\enquote{Видите ли, как получилось?} - раздумывает эстет из городка Макаров на Киевщине:
\enquote{Мы не получили Украину. Украинскую Украину, где была бы в своей хате своя
правда и сила, и воля. То есть где был бы свой язык, своя история, свои герои.
И сегодня мы бы не спорили, кому ставить памятники и чьи имена должны носить
наши улицы}.

\\

«Щоб бути українцем, треба трохи потрудитися»

— Анатолію Несторовичу, незабаром  будемо святкувати 30-річчя відновлення нашої
Незалежності. Як вважаєте, Україна сповна використала свій шанс, який випав їй
у серпні 1991 року?

— У мене таке враження, що не з того ми тоді почали. Була така ейфорія — більше
90 відсотків населення підтримали незалежність України. І от на цій хвилі треба
було сотворити ідеологічну машину під нашу незалежність. Бо бачте, як вийшло?

Ми отримали незалежність, але ми не отримали Україну. Українську Україну, де
була б «в своїй хаті своя правда і сила, і воля», як писав наш пророк Тарас
Шевченко. Тобто де була б своя мова, своя історія, свої герої. І ми сьогодні не
сперечалися б, кому ставити пам’ятники та чиї імена мають носити наші вулиці.

Спочатку треба було сотворити націю, з населення, з народу України сотворити
український народ. Немає народу Росії, Франції або Німеччини. Є російський
народ, французький, німецький... Щоб бути українцем, треба трохи потрудитися.
Бо якщо ви живете в державі, назва якої «Україна», значить, треба знати і
поважати її цінності, історію, традиції, мову.

& 

И то правда, как была Украина - своя хата, правда, сила и воля с краю, ничего
не знаю, так и осталась. Народный артист ничего нового не открыл. Зато
украинцы, может быть, открыли для себя народного артиста. А то живет человек,
звания получает, а народ не в курсе.

Вы только послушайте глубину его мысли: \enquote{Сначала надо было сотворить нацию, из
населения, из народа Украины сотворить украинский народ. Нет народа России,
Франции или Германии. Есть русский народ, французский, немецкий}. Глубоко
копнул, основательно. Теперь, поди, разберись, он сам-то, народный артист
Украины или украинский народный артист?

\enquote{И тут я обращаюсь к Шевченко}, продолжает народный артист —
\enquote{его слова, это наше священное писание: \enquote{А мы смотрели и
молчали, и молча почесывали чубы, немые, подлые рабы}. И вот рабство до сих пор
сидит в нас}.

\\

— Мабуть, ми й справді не з того почали, якщо на 30-му році незалежності все ще
мусимо боронити свою мову, доводити, що «какая разніца» рано чи пізно вилазить
боком...

— Мова — це мати єднання, батько громадянства і сторож держави. Дуже великий
мудрець сказав ці слова і я написав би їх на всіх білбордах як нагадування
кожному, що мова — це та першооснова, яка згуртовує народ у націю. Буде мова —
буде нація, буде нація — буде держава.

Ми багато за цей час помилок наробили. І тут я звертаюся до Шевченка — його
слова, то наше святе письмо: «а ми дивились та мовчали та мовчки чухали чуби,
німії, підлії раби». І оце рабство до сих пір сидить у нас. Невже 15 тисяч
людей поклали голови, щоб сьогодні в Україні говорили мовою ворога? Це як треба
себе не поважати? Я не можу спокійно пропустити повз, коли говорять російською.
Постійно в суперечки встряю! І на базарі, і в ліфті — де б не був. Кажу: «А
мова Шевченка набагато красивіша, ніж путінська. Чули про такого? На
стогривневій купюрі він, візьміть почитайте!».

& 

Опять глубоко копнул, народный подлый раб, почесывая чуб.

\enquote{У меня по субботам в консерватории занятия со студентами. А там арендуют зал
разные детские коллективы. И так, бывает, душа болит: и дети, и их наставники —
все не по-нашему говорят. Не выдерживаю, останавливаюсь: \enquote{Вы откуда, неужели
московиты? Нет? Так почему же языком врага говорите?},} плачет пенсионер. Так
срочно бегите, народный артист, к директору, и требуйте переименования
консерватории! Как у вас только штиблеты поворачиваются ходить в учебное
учреждение имени Петра Чайковского?

Ты, братец, любишь Русь,
Как хлеб, как ломоть сала, —
Я лаю сотни раз,
Чтобы она не спала, 

- вспоминает мозг артиста стихи Ивана Франко. 

Как народный артист Украины может вспоминать стихи поэта, носящего
предательское имя Иван? Вообще непонятно. Зато его мысль, как слеза - мутная и
соленая: \enquote{Так и я должен лаять, чтобы Украина не дремала. Потому что
государство молчит}.

\\

У мене по суботах у консерваторії заняття зі студентами. А там орендують зал
різні дитячі колективи. І так, буває, душа болить: і діти, і їхні наставники —
всі не по-нашому говорять. Не витримую, зупиняюся: «Ви звідки, невже московити?
Ні? То чому ж мовою ворога говорите?» Мовчать і швиденько тікають.

— Але ж буває, що не мовчать? Мабуть, і неприємні речі у відповідь кажуть?

— Всяке буває! І огризаються, і образливе щось сказати можуть. Якось у рідному
Макарові (а його вже теж окацаплюють!) один мені кричав: «Ви націоналіст!»
Кажу: «Я цим горджусь! Бо для мене слово «націоналіст» святе!» Дуже чітко Іван
Франко зобразив різницю між націоналістом і патріотом у «Сідоглавому»:

Ти, брате, любиш Русь,
Я ж не люблю, сарака!
Ти, брате, патріот,
А я собі собака.
Ти, брате, любиш Русь,
Як хліб і кусень сала, —
Я ж гавкаю раз в раз,
Щоби вона не спала.
Бо твій патріотизм –
Празнична одежина,
А мій — то труд важкий,
Гарячка невдержима.

Отак і я мушу гавкати, щоб Україна не дрімала. Бо держава мовчить. А наші
владні структури такі в’ялі, такі далекі від справжньої України. Важливі
визначні дати і свята ми відзначаємо на народному рівні, за покликом душі.

Ось недавно було 150 років від дня народження Лесі Українки і Василя Стефаника.
Та які ж це постаті в Україні! Як їх треба знати і на якому рівні треба
святкувати! 160 років, як перевезли на батьківщину тіло Шевченка. Хтось поїхав
у Канів із владної верхівки? Ніхто! І ми чекаємо, що ця влада виховає нас
українцями?

& 

\enquote{Вот недавно было сто пятьдесят лет со дня рождения Леси Украинки и Василия
Стефаника. Да какие же это фигуры на Украине! Как их надо знать и на каком
уровне надо праздновать! Сто шестьдесят лет, как перевезли на родину тело
Шевченко!}, - восторгается артист, привлекая к себе внимание, чтобы и его тело
не бросили на социальном погосте, а возложили поближе к Шевченко, а еще лучше к
Леси. И не хочет вспоминать артист, что именно в издательствах СССР
популяризировали украинских, киргизских, таджикских, эстонских и иных поэтов,
писателей и прочих композиторов. Не будь Совета Национальностей Верховного
совета СССР, не было бы и народного чтеца Украинской СССР Паламаренко, не было
бы и Розы Рымбаевой, Палада Бюльбюль оглы, Софии Ротару, Вахтанга Кикабидзе,
Николая Гнатюка и прочих нынешних русофобов. 

\\

«Я б ОПЗЖ підйомні заплатив, аби вони вибралися десь у Ростов»

— Чому, на вашу думку, країнам Балтії вдалося, так би мовити, «вибитися в
люди», а нам ні? Це тому, що ми, як казав Віктор Ющенко, «надто довго жили в
неволі»?

— Та ще й у якій неволі... Болгарія 300 років була під турком. А тепер вільна —
йде своїм шляхом, розвивається. А нам доля послала оцю орду, яка століттями
знущається над нами. Вкрали в нас усе — і нашу історію, і назву нашу — все
забрали.

& 

И вот к этому моменту товарищу приходят исторические знания из космоса:
\enquote{Болгария триста лет была под турками. А теперь свободна — движется своим
путем, развивается. А нам судьба послала эту орду, которая веками издевается
над нами. Украли у нас все — и нашу историю, и название наше — все забрали}.

Интересный подход, как у болгар. Были болгары 300 лет под турками, а потом раз,
и уже свободные, развиваются. А на Шипке, русские просто так из пушек палили,
фейерверки запускали, братушек развлекали. В ходе \enquote{карнавала} русские части
потеряли около пяти с половиной тысяч солдат и офицеров. Турецкая же армия
Вессель-паши, просто перестала существовать.

\\


Чому Польщі було легко стати незалежною європейською сучасною країною? Тому що
її громадяни духовно пов’язані історією, традицією, вірою і, головне, мовою.
Там із пелюшок ти є поляк! А у нас? Дивлюся на малих дітей, а вони вже
окацаплені, в них сидить оцей «рускій мір» із молоком матері.

Та що там говорити, ми навіть у межах власної країни маємо такі приклади. Землі
Західної України, які менше перебували під впливом московитів, а належали до
Польщі та Австро-Угорщини, дуже відрізняються від решти України, заколисаної
«колискою трьох братніх народів». Нас, українців, — 86 відсотків.

& \\

Але надто багато орда виховала перевертнів, про яких писав знову ж таки
Шевченко: «а тим часом перевертні нехай підростають та поможуть москалеві
господарювати...» І таких перевертнів ще й досі маємо як у низах, так і верхах.
Та ж таки ОПЗЖ — це воріженьки наші, колаборанти. Я б їм навіть підйомні
заплатив, аби тільки вони вибралися десь у Москву чи бодай Ростов — отам ваш
«рускій мір», там його і будуйте. А нам не заважайте будувати Україну.

& 

\enquote{Но слишком много орда воспитала оборотней, о которых писал опять же Шевченко:
\enquote{а тем временем оборотни пусть подрастают и помогут москалю хозяйничать...} И
таких оборотней мы до сих пор имеем как в низах, так и в верхах}, - страдает
урожденный в СССР Анатолий Паламаренко и делает фееричный вывод: \enquote{Построй
Украину в себе, а не себя в Украине}. И теперь понятно, почему анкета артиста
начинается со слов: \enquote{С малых лет Анатолий Паламаренко увлекался художественным
словом}. \enquote{Даже страшно представить, что мы могли бы жить, как Белоруссия или
Россия}, - заговаривает себя пенсионер.

\\


Думаю, якби на перших виборах президента ми зробили правильний вибір і обрали
націоналіста Чорновола (хай царствує!), то все, може, було б геть інакше. Але
хіба ми зараз думаємо, обираючи президента?

Наш народ, на жаль, піддався оцій дуже хитрій і талановитій наживці, яку
принесли нам чужаки. «Квартал», кіно з Голобородьком... І оце наївне дитя —
«мальчік Мотл» (герой оповідання Шолом-Алейхема. — Авт.), тобто наш народ,
клюнув! Ця рабська вдача знову зіграла з українцями злий жарт: хотіли кращого
життя, а отримали урок. Бо, розбудовуючи країну, в першу чергу побудуй Україну
в собі.

Вивчи її мову, історію, передай це своїм дітям, і ти вже зробиш для України
велику справу. Хочеш нормально жити — побудуй у собі українця. Якби ми всі були
монолітною нацією, чи пішов би Путін на нас? А так він знайшов кого захищати —
російськомовних, бо для нього Росія там, де є російська мова. А наш нинішній
президент... Розуміти Україну він не може. Йому це природою не дано! Він
Україні дуже далекий і чужий.


— Але в Зеленського ще є кілька років, аби схаменутися і ви­правитися...


— Мені так бачиться, що він свою каденцію не добуде. Так чогось здається. Вже
народ починає зріти, вже набирається оця вулканічна сила протистояння...

Але якщо подивитися з іншого боку...

Ми звикли орієнтуватися на хороші приклади і порівнюємо свою ситуацію з країнами Балтії. Україна, Грузія, а тепер ще й Молдова принаймні розвернулися в бік Європи і пов’язують своє майбутнє з нею. Водночас у багатьох інших республіках лише вивіски змінили, а суть залишилася та сама. Ми принаймні право вибирати президента собі вибороли. Хай обирати правильно нам ще треба вчитися, але ж маємо вже шостого. А в деяких інших республіках Союзу мали двох-трьох, а то й й узагалі одного, як у Білорусі.
«Ми не Білорусь, бо все-таки козацького роду»

— Навіть лячно уявити, що ми могли б жити, як Білорусь чи Росія.

& \\

— Ми відрізняємось, ми зов­сім інші. Бо, як би там не було, ми таки козацького
роду. І показали це нашими майданами всьому світу. І десь у генах сидить у нас
ота волелюбність і прагнення до свободи. Навіть у найпо­хмуріші радянські часи
знаходилися сміливці, які не корилися системі. Козацтво — то було унікальне
явище в середньовічній Європі. Так, козаками були не всі, це було ядро
найсміливіших, найволелюбніших. Так і зараз ми маємо ядро тих, хто не дає
приспати Україну.

& 

А дальше он, скорее всего, начал прибухивать, ибо долго: \enquote{Где-то в генах сидит
у нас это свободолюбие и стремление к свободе}, - шепчет артист, а мы
вспоминаем, что он только что цитировал Шевченко: \enquote{А мы смотрели и молчали, и
молча почесывали чубы, немые, подлые рабы}. Но его это уже не останавливает.
Его мысли, его скакуны: \enquote{Даже в самые мрачные советские времена находились
смельчаки, которые не повиновались системе. Казачество — это было уникальное
явление в Средневековой Европе. Да, казаками были не все, это было ядро самых
смелых, самых любвеобильных. Так и сейчас мы имеем ядро тех, кто не дает
усыпить Украину}. Это просто феерия слов - любвеобильные средневековые казаки в
мрачные времена СССР не повиновались системе...

\\

— І достойних президентів, які запам’яталися не золотими батонами, а справами на благо України, ми приводили до влади саме з майданів...

— Віктор Андрійович для мене свята людина. Він великий інтелігент і до нього
треба було дотягнутися, дорости, а ми були надто далеко тоді від його розуміння
України. Це він першим копнув у глибини нашої самоідентичності, дістав з
архівів нашу минувшину та героїв. І показав, які вони насправді, ті «браття», і
чому маємо ставитися до них обережно. Своєю любов’ю до України і баченням її
розвитку Ющенко став дуже страшним для ворогів. Він єдиний, кого хотіли знищити
фізично. Це він започаткував і Томос, і «безвіз», і НАТО — готував ґрунт і
засівав саме він. А продовжив Петро Порошенко.

& \\

Коли Петро Олексійович був президентом, я був спокійний за Україну. Йому
вистачило снаги написати на білбордах давню мрію українців «Геть від Москви!»
Мова, віра, армія — він чітко розумів, на яких трьох китах тримається держава.
Як оратор — він златоуст! З ним ніхто не порівняється в цьому мистецтві!
Зовнішня політика в нього була відмінна, він зумів згуртувати навколо себе
діячів світу.

& 

"Когда Петр Алексеевич был президентом, я был спокоен за Украину. Ему хватило
сил написать на билбордах давнюю мечту украинцев \enquote{Прочь от Москвы!}.
Прямо вижу, как из последних сил, Петр Алексеевич размахивает кистью и
расписывает плакаты. Только хотел он писать \enquote{Прочь от Киева}, а подумал
о Москве. И этот народный артист, уже допивая свою сливовицу, думал о Киеве, а
написал о Москве. 

\\


Але ж скільки довкола ворогів-воріженьків! Нерозумні люди наші, я не називаю їх українцями, а хохлами, піддалися, мов ті туземці, на блискучі брязкальця. Комусь не подобалося, що він багатий. То заглянімо в історію. А що, Хмельницький був бідний? А Мазепа? А інші наші гетьмани і полковники? Вони були досить заможними людьми, і до цього треба звикнути.

Якщо накрався — це інша справа, а якщо людина збудувала успішний бізнес, отже,
має здібності й хист. Мазепа і Хмельницький будували церкви і школи. І це наша
традиція. А скільки Порошенко віддав на армію? Хтось із багачів його наслідує?
Щось не бачу! Мало бути просто багатим. Треба вміти бути багатим.

Та, попри все, я вірю в Україну, вірю в українську молодь, яка часто є набагато
мудріша і сміливіша за наше покоління, бо росте в зовсім інший час.


— З нагоди ювілею незалежності плануються грандіозні святкування. Вас до участі в концертах запрошували?

— Поки що мовчать, мабуть, і не запросять. Я вже до цього звик. Але, хвалити
Бога, я маю свого глядача, маю до кого звернутися словами класиків. Бо ж
Паламаренко — це не тільки гуморески. Основа моєї творчості — класична
література. Це Шевченко, Гоголь, Нечуй-Левицький, Чингіз Айтматов, Довженко. От
учора (розмова відбулася 18 червня. — Авт.) в моєму Макарові вітав медиків.

Думали, що вийду на 15 хвилин, але люди із такою насолодою слухали, що я
простояв на сцені понад годину. В мене аж сорочка мокра була. От і сьогодні на
концерті в Будинку офіцерів читатиму Довженка... Люди скучили за літературним
словом. Такий зараз час, що українці нічого не читають. Раніше в метро книжки,
газети читали, а зараз — усе знай «довбуть» у ті екрани телефонів. І лягають
спати з ними, і прокидаються...


— Так сталося, що українська Незалежність тісно пов’язана з історією нашої
газети. «Україна молода» лише трішки старша за сучасну Україну Незалежну — 4
липня їй теж виповниться 30 років. Знаю, що ви давній шанувальник нашої
газети...

— О, так! Здавна передплачую «Україну молоду», дуже люблю її і шаную! Сам читаю
й іншим раджу! Оце якось у Макарові хотів купити номер (бо передплата мені в
Київ приходить) — в один кіоск, у другий, на пошту — нема. «Давайте запишемо,
будемо просити цю газету», — кажуть мені. Та що ж це таке? Найкраща газета, і
купити ніде не можна!

Щиро вітаю ваш колектив за правдиве слово, яке несете в люди вже багато років.
Що гуртуєте націю, повертаєте її забуту історію. Нехай ще не один ювілей буде
попереду. А якщо будете святкувати — кличте. Приїду в редакцію, почитаю вам: і
посмієтесь трошки, і поміркуєте над словами наших класиків...

& 

Ну, и в конце интервью, народный чтец Украины пытается поклониться
"журналистам" Україна молода, не вставая с кресла, чтобы не расплескать бокал.
"Спасибо!", - восклицает он, "Что спаи... сплачиваете нацию, возвращаете ее
забытую историю. Пусть еще не один юбилей будет впереди. А если будете
праздновать — зовите!"

Вот такой пожилой весельчак с кашне на шее живет на Украине. Он очень любит
молодых и коньяк. Он любит читать вслух то, что написали другие и не любит
читать то, что наговорил сам. Он любит аплодисменты, но их нет в продаже. Он
любит себя, но давно не смотрел в зеркало и не менял носки. Его публикуют в
Україна молода, ибо больше публиковать им нечего. Потому как хороший коньяк и
хорошие мысли давно кончились...

Читайте и подписывайтесь на наш канал в Telegram.  Ваш, "Вечер у камина".

\\
  
\end{longtable}
