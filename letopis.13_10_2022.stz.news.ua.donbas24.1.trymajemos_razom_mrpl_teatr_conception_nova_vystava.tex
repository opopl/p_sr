% vim: keymap=russian-jcukenwin
%%beginhead 
 
%%file 13_10_2022.stz.news.ua.donbas24.1.trymajemos_razom_mrpl_teatr_conception_nova_vystava
%%parent 13_10_2022
 
%%url https://donbas24.news/news/trimajemos-razom-mariupolskii-teatr-conception-predstavit-novu-vistavu-foto
 
%%author_id demidko_olga.mariupol,news.ua.donbas24
%%date 
 
%%tags 
%%title "Тримаємось разом" — маріупольський театр "Conception" представить нову виставу
 
%%endhead 
 
\subsection{\enquote{Тримаємось разом} — маріупольський театр \enquote{Conception} представить нову виставу}
\label{sec:13_10_2022.stz.news.ua.donbas24.1.trymajemos_razom_mrpl_teatr_conception_nova_vystava}
 
\Purl{https://donbas24.news/news/trimajemos-razom-mariupolskii-teatr-conception-predstavit-novu-vistavu-foto}
\ifcmt
 author_begin
   author_id demidko_olga.mariupol,news.ua.donbas24
 author_end
\fi

%\ii{13_10_2022.stz.news.ua.donbas24.1.trymajemos_razom_mrpl_teatr_conception_nova_vystava.txt}

\ii{13_10_2022.stz.news.ua.donbas24.1.trymajemos_razom_mrpl_teatr_conception_nova_vystava.pic.front}

\begin{center}
  \em\color{blue}\bfseries\Large
  Актори театру авторської п'єси \enquote{Conception} підготували нову музично-поетичну виставу
\end{center}

Маріупольський \href{https://donbas24.news/news/mariupolci-vrazili-svojeyu-premjeroyu-kiyan}{театр авторської п'єси \enquote{Conception}}%
\footnote{Маріупольці вразили своєю прем'єрою киян, Ольга Демідко, donbas24.news, 29.07.2022, \par%
\url{https://donbas24.news/news/mariupolci-vrazili-svojeyu-premjeroyu-kiyan}%
} готовий представити нову
музично-поетичну виставу \enquote{Тримаємось разом}. Актори зачитають вірші Ліни
Костенко, Сергія Жадана, Петра Маги, а також твори власного авторського
написання.

\begin{leftbar}
\emph{\enquote{Поєднані однією історією, одним болем, ми тримаємось усі разом,
хто постраждав від рук війни...}}, — наголосили актори театру. 
\end{leftbar}

\ii{insert.read_also.demidko.donbas24.teatr_brama_detail}

\subsubsection{Як виникла ідея створити нову виставу?}

\ii{13_10_2022.stz.news.ua.donbas24.1.trymajemos_razom_mrpl_teatr_conception_nova_vystava.pic.1}

Режисер театру авторської п'єси \enquote{Conception} Олексій Гнатюк розповів, що
спочатку актори театру виступили на літературному вечорі з віршами Ліни
Костенко та Сергія Жадана, який було підготовлено до Дня Незалежності України.
Творчий вечір маріупольців підтримало київське видавництво \enquote{Саміт Книга},
запросивши відомих літераторів Києва. Так колектив познайомився з українським
актором, поетом-піснярем та телеведучим, який наразі служить в ЗСУ \href{https://www.facebook.com/people/Петро-Мага/100011134235936}{Петром
Магою}.%
\footnote{\url{https://www.facebook.com/people/Петро-Мага/100011134235936}} %
Автор подарував акторам свої брошури з віршами, які надихнули весь
колектив. Саме вірші Петра Маги лягли в основу вистави, адже вони дуже
актуальні та злободенні. У цьому спектаклі задіяні всі актори театру. Також
завдяки грі на фортепіано Людмили Гричаненко вистава матиме живий музичний
супровід.

\begin{leftbar}
\emph{\enquote{Актори читатимуть вірші наших сучасників, але будуть присутні і елементи
театралізації. Нашому колективу тема війни і втрати власної домівки
дуже близька, тому актори не зможуть просто читати вірші. Звісно, вони
будуть і грати}}, — підкреслив режисер театру Олексій Гнатюк.
\end{leftbar}

\ii{insert.read_also.demidko.donbas24.teatromania_rozv_pidtrym_kulturu_mrpl}

\subsubsection{Які ще плани має колектив?}

\ii{13_10_2022.stz.news.ua.donbas24.1.trymajemos_razom_mrpl_teatr_conception_nova_vystava.pic.2}

Маріупольський колектив театру авторської п'єси \enquote{Conception} нещодавно
повернувся з турне українськими містами. Театр представив свою виставу \textbf{\emph{\enquote{Обличчя
кольору війна}}} за документальними подіями. Режисер і актори майстерно
висвітлили тему війни, виживання і людяності. У кожному місті вистава проходила
при переповнених залах та бурхливих оваціях захоплених та вдячних глядачів.

\begin{leftbar}
\emph{\enquote{Найбільше нас вразила реакція глядачів у Вінниці. Глядачі аплодували 15 хвилин
стоячи. Це нас дуже надихнуло на нові звершення}}, — розповів Олексій Гнатюк. 
\end{leftbar}

\textbf{Читайте також:} \emph{Драматичний театр Маріуполя виступив у Польщі — актори взяли участь у міжнародному фестивалі}%
\footnote{Драматичний театр Маріуполя виступив у Польщі — актори взяли участь у міжнародному фестивалі, %
Еліна Прокопчук, donbas24.news, 12.09.2022, \par%
\url{https://donbas24.news/news/dramaticnii-teatr-mariupolya-vistupiv-u-polshhi-aktori-vzyali-ucast-u-miznarodnomu-festivali}%
}

\ii{13_10_2022.stz.news.ua.donbas24.1.trymajemos_razom_mrpl_teatr_conception_nova_vystava.pic.3}

Наразі театр поповнився новими акторами. До нього приєдналися маріупольські
актриси, яким вдалося виїхати з окупованого Маріуполя, актор з \enquote{Молодого
театру} та актриса з Івано-Франківська. Репетиції проходять у приміщенні
Київнаукфільму. Колектив театру вже відновлює виставу \enquote{Другий шанс} і готує
відеопривітання до Дня захисників України.

\begin{leftbar}
\emph{\enquote{Головним нашим завданням зараз є інтегруватися в київську культуру і знайти
спонсорів, адже ще не вистачає реквізиту та потрібного обладнання}}, — поділився
Олексій Гнатюк.
\end{leftbar}

\ii{insert.read_also.demidko.donbas24.u_kyevi_pokazhut_vystavu_pro_mariupol}
\ii{13_10_2022.stz.news.ua.donbas24.1.trymajemos_razom_mrpl_teatr_conception_nova_vystava.pic.4}

Ще одним проєктом, який незабаром побачить глядач, стане телевізійне ток-шоу
\textbf{\enquote{Мистецький максимум}}, телеведучим якого стане актор театру \href{https://www.facebook.com/MrEvangelion}{%
Дмитро Гриценко}.%
\footnote{\url{https://www.facebook.com/MrEvangelion}} %
Проєкт буде присвячено співпраці київської та маріупольської культур. Один раз
на тиждень Дмитро проводитиме інтерв'ю з діячами культури Києва, які
розповідатимуть про свою готовність допомагати розвитку культури Маріуполя.

А музично-поетичну виставу \enquote{Тримаємось разом} можна буде побачити \textbf{\emph{17 жовтня о
17:00}} в Київському академічному театрі українського фольклору \enquote{Берегиня} (вул.
Івана Миколайчука, 3а), яка відбудеться в рамках співпраці з видавництвом
\enquote{Саміт-Книга}, Міжнародним благодійним фондом \enquote{Твоя Доброта}, за підтримки DDI
Croup. Вхід вільний.

\ii{13_10_2022.stz.news.ua.donbas24.1.trymajemos_razom_mrpl_teatr_conception_nova_vystava.pic.5}

Раніше Донбас24 розповідав, \href{https://archive.org/details/31_07_2022.olga_demidko.donbas24.jaki_teatr_proekty_mrpl_zakordon_mytci}{які театральні проєкти та культурні заходи,
присвячені Маріуполю, реалізуються закордонними митцями}.%
\footnote{Які театральні проєкти та культурні заходи, присвячені Маріуполю, реалізуються закордонними митцями, %
Ольга Демідко, donbas24.news, 31.07.2022, \par%
\url{https://archive.org/details/31_07_2022.olga_demidko.donbas24.jaki_teatr_proekty_mrpl_zakordon_mytci}%
}

Ще більше новин та найактуальніша інформація про Донецьку та Луганську області
в нашому телеграм-каналі Донбас24.

ФОТО: з відкритих джерел.

\ii{insert.author.demidko_olga}
