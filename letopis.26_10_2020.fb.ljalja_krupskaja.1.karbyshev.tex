% vim: keymap=russian-jcukenwin
%%beginhead 
 
%%file 26_10_2020.fb.ljalja_krupskaja.1.karbyshev
%%url https://www.facebook.com/N.Krupskay/posts/3466249326816452
%%parent 26_10_2020
 
%%endhead 

\subsection{Я совестью и Родиной не торгую... - генерал Карбышев}

\url{https://www.facebook.com/N.Krupskay/posts/3466249326816452}
\index[names.rus]{Карбышев, Дмитрий}

Я совестью и Родиной не торгую...

26 октября 1880 года родился мой земляк, герой Советского Союза, генерал
Карбышев, замученный в лагере смерти Маутхаузен.....  Между прочим, это военный
инженер Карбышев принимал участие в модернизации Брестской крепости - одним из
славных символов героизма советских солдат. 

Карбышев - участник еще одного славного дела, Брусиловского прорыва. Построив
десятки укрепрайонов от Дальнего Востока до Черного моря, Дмитрий Карбышев
разработал рекомендации по прорыву линии Маннергейма - еще одного символа
воинственного 20-го века.  Карбышеву принадлежит наиболее полное исследование и
разработка вопросов применения разрушений и заграждений. Значителен его вклад в
научную разработку вопросов форсирования рек и других водных преград. Он
опубликовал более 100 научных трудов по военно-инженерному искусству и военной
истории. Его статьи и пособия по вопросам теории инженерного обеспечения боя и
операции, тактике инженерных войск были основными материалами по подготовке
командиров Красной Армии в предвоенные годы. 

А еще православный советский генерал консультировал Учёный совет по
реставрационным работам в Троице-Сергиевой Лавре - крепость, всё-таки… 

Немцы взяли его контуженным, в бессознательном состояниии - и отправили в
страшный последний путь по лагерям: Замосць, Хаммельбург, Флоссенбюрг,
Майданек, Аушвиц, Заксенхаузен и Маутхаузен. Путь был долог - плененного в
августе 41-го, немцы убили его 18 февраля 45-го: пытали, поливали ледяной водой
на морозе, убили, наконец, и сожгли в лагерной печи. Почему-то ночью, с 17-го
на 18-е - не спалось фашистским упырям… 

…Генерал Карбышев вовсе не позабытый герой — сотни названных в его честь улиц,
десятки памятников. Я училась в школе имени его...Есть даже памятная доска на
Стене плача в Маутхаузене. Если не сорвали еще — в соответствии с новыми
веяниями в истории.
