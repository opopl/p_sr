% vim: keymap=russian-jcukenwin
%%beginhead 
 
%%file 23_08_2021.fb.filindash_evgenij.1.nezalezhnist_30_let
%%parent 23_08_2021
 
%%url https://www.facebook.com/e.filindash/posts/4094673783992354
 
%%author_id filindash_evgenij
%%date 
 
%%tags 30_let,nezalezhnist,strana,ukraina
%%title Как по-разному можно потратить 30 лет
 
%%endhead 
 
\subsection{Как по-разному можно потратить 30 лет}
\label{sec:23_08_2021.fb.filindash_evgenij.1.nezalezhnist_30_let}
 
\Purl{https://www.facebook.com/e.filindash/posts/4094673783992354}
\ifcmt
 author_begin
   author_id filindash_evgenij
 author_end
\fi

Как по-разному можно потратить 30 лет.

Страна, вырвавшаяся из-под оккупации врагом, за 30 лет прошла долгий путь. Все
последствия оккупации были ликвидированы, и страна бурно развивалась одними из
самых высоких в мире темпами. 

Было построено с нуля или полностью модернизировано больше 10 тысяч только
крупных промышленных предприятий. Включая несколько ракетокосмических,
авиационных и автомобильных заводов, предприятий по выпуску компьютерной и
разнообразной (телевизоры, холодильники и т.д.) бытовой техники. Не говоря о
металлургических, химических и многих других предприятиях. 

\ifcmt
  pic https://scontent-frt3-1.xx.fbcdn.net/v/t1.6435-9/240390273_4094673700659029_2934110716825941518_n.jpg?_nc_cat=107&_nc_rgb565=1&ccb=1-5&_nc_sid=730e14&_nc_ohc=IhlECg_HvVUAX9Z6iuO&_nc_ht=scontent-frt3-1.xx&oh=9373861dd7721c2280a42cb4aa788986&oe=615D2B38
  width 0.8
\fi

Страна является одним из мировых технологических лидеров во многих сферах,
развивается наука, в разы увеличилось количество ВУЗов и молодых учёных, за
последние 30 лет в экономике выросла доля наукоёмкой, высокотехнологической
продукции. 

Производство электроэнергии выросло более чем в 60 раз: с 3,2 млрд. кВт·ч до
195 млрд. Были построены десятки электростанций: тепловых, гидро, строились
новые атомные электростанции, а цены на электричество снижены. 

Открыто и введено в промышленную эксплуатацию крупнейшее в Европе месторождение
природного газа, строятся масштабные новые газопроводы и газохранилища, газ
дешевеет для населения и промышленности.

Страна производит 8-10\% мирового производства стали, чугуна, тракторов, сахара,
добычи угля, природного газа, при том что доля населения страны составляет
только 1,4\% от мирового.

Население страны, кстати, тоже увеличилось за 30 лет более чем на 17 млн.
человек, примерно в полтора раза, и продолжает расти. Рождаемость значительно
превышает смертность, выросла средняя продолжительность жизни. 

И действительно, почему бы населению не расти, если реальные темпы роста
доходов населения составляют 20\%, уровень жизни для подавляющего большинства
населения постоянно улучшается, по всей стране массово строятся доступные для
всех граждан жилые кварталы со школами и больницами, а также новые мосты,
дороги, развязки, в мегаполисах – линии метрополитена, и т.д. К стране были
присоединены и новые территории.

Политическая элита страны управляет или, как минимум, влияет на половину
планеты, предложенные ею идеи имеют множество сторонников по всему миру, их
берут на вооружение десятки других стран, а с враждебно настроенными
государствами разговаривает с позиции силы. 

А теперь угадайте, о какой стране идёт речь? 

В мировой истории и современности крайне мало стран, подходящих под это описание.

То, что здесь описана не сегодняшняя Украина, чьи правители готовятся завтра с
помпой отпраздновать 30-летие провозглашения независимости, понимают, думаю, и
самые упоротые вышиватники. 

Евромайданная Украина если что и построила масштабного, то разве что гигантские
флагштоки для флагов, некоторые из которых уже успешно падают на припаркованные
рядом элитные «Лексусы» кристально честных и очень патриотичных чиновников,
ответственных за эти самые флагштоки.

Речь и не о некоей гипотетической идеальной Украине, в духе альтернативной
истории: «а что было бы, если бы…».

И всё же несколькими строками выше по тексту я описал именно Украину. Только не
сегодняшнюю деградирующую и вымирающую евромайданную Украину, находящуюся под
почти неприкрытой оккупацией Запада, где нет и речи ни о какой реальной
независимости. 

А Украину 1975 года – Украинскую Советскую Социалистическую Республику в
составе Советского Союза. И тот путь, который Советская Украина проделала за
предшествующие 30 лет – с 1945-го года, после кровопролитнейшей войны, когда
вся Украина лежала в руинах. 

И рост населения за те годы с 31,3 млн. до 48,8 млн., и строительство (или
восстановление после разрушений в годы войны) тысяч крупных, современнейших в
те годы предприятий и конструкторских бюро: от Днепропетровского «Южмаша» и КБ
«Южное», КБ Антонова, киевского и харьковского авиазаводов, и до днепровского
каскада ГЭС, и всё остальное, о чём было сказано выше, – всё это произошло в
УССР за 30 лет: с 1945 по 1975 гг. 

Как и присоединение к Украине Закарпатья и Крыма, а также окончательное
закрытие вопроса о международном признании присоединения Западной Украины.

И управляли в те годы одной из двух тогдашних сверхдержав – Советским Союзом, а
также ориентированной на СССР половиной Европы и мира, во многом украинцы и
выходцы с Украины: лидер СССР, генеральный секретарь ЦК КПСС Леонид Брежнев,
председатель Президиума Верховного Совета СССР (по конституции — высшая
государственная должность) Николай Подгорный, министр обороны Андрей Гречко,
министр внутренних дел Николай Щёлоков, Генеральный прокурор Роман Руденко,
председатель КГБ (в 1960-х гг.)  Владимир Семичастный, заместитель главы
правительства СССР (позже – глава правительства) Николай Тихонов, члены
Политбюро ЦК КПСС Владимир Щербицкий, Андрей Кириленко и многие другие.

Принятые ими решения были тогда обязательны для Берлина и Варшавы, с их мнением
вынуждены были считаться в Вашингтоне и Лондоне.

Более чем наглядная иллюстрация того, как очень по-разному потратили 30 лет в
проклятом тоталитарном СССР и в очень прогрессивной, прозападной и национально
свидомой Украине
