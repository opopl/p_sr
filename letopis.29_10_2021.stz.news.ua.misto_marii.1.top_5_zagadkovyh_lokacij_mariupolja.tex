% vim: keymap=russian-jcukenwin
%%beginhead 
 
%%file 29_10_2021.stz.news.ua.misto_marii.1.top_5_zagadkovyh_lokacij_mariupolja
%%parent 29_10_2021
 
%%url https://mistomariupol.com.ua/uk/top-5-mistychnyh-ta-zagadkovyh-lokaczij-mariupolya
 
%%author_id news.ua.misto_marii
%%date 
 
%%tags 
%%title ТОП-5 загадкових локацій Маріуполя
 
%%endhead 
 
\subsection{ТОП-5 загадкових локацій Маріуполя}
\label{sec:29_10_2021.stz.news.ua.misto_marii.1.top_5_zagadkovyh_lokacij_mariupolja}
 
\Purl{https://mistomariupol.com.ua/uk/top-5-mistychnyh-ta-zagadkovyh-lokaczij-mariupolya}
\ifcmt
 author_begin
   author_id news.ua.misto_marii
 author_end
\fi

\ifcmt
  ig https://i2.paste.pics/PIJ0F.png
  @wrap center
  @width 0.9
\fi

\begin{quote}
\em
Ну що ж наближується свято Гелловіну, яка ж містична атмосфера без легенд та
розповідей про жахи та страшні місця? Ми зібрали невеличку добірку із
загадковими та містичними локаціями міста, які й сьогодні містять у собі багато
чуток та загадкових моментів
\end{quote}

\subsubsection{Маєток купця Регіра}

\ii{29_10_2021.stz.news.ua.misto_marii.1.top_5_zagadkovyh_lokacij_mariupolja.pic.1}

Перше, що спадає на думку, коли замислюєшся про загадкові місця Маріуполя, це
маєток купця Регіра. Петро Регір був купцем першої гільдії та власником
кампанії \enquote{Петро Регір та син}, яка входила у топ-5 судновласницьких
кампанії Російської імперії. Свій розкішний будинок на вулиці Італійській
заможний маріуполець побудував на початку XX століття. Він одружився з молодою
гречанкою, його власність тільки зростала – ніби-то будинок повинен був стати
місцем радості та посмішок. Але після 1917 року все почало змінюватися, у місто
входила нова-радянська влада, і до представників першого класу та буржуазії
стосунок різко змінився.

Родина Петра Регіра виїхала, а сам він залишався в місті. У 1919 році в його
величному будинку сталася трагедія. Більшовики вбили власника будинку, але
ходять також чутки, що він вмер від голоду, адже з його маєтку його просто
вигнали, і чоловіку не було куди подітися. Так то чи ні – і досі не відомо,
адже тіло так і не знайшли, так само як і не має могили Петра Регіра на
маріупольському Некрополю. Маріупольські сталкери розповідать про підвал та
підземний хід біля будинку Регіра, а також видають версії, що Петро Регір
просто врятувався через цей тунель та не загинув у власному маєтку.

Будинок же так і не прийняв нового хазяїна, кажуть, навіть що гріфони на фасаді
відвернулися від будинку. І зараз будівля стоїть зовсім покинута, але своєю
красою продовжує надихати фотографів, художників та прихильників маріупольської
історії.

\ii{29_10_2021.stz.news.ua.misto_marii.1.top_5_zagadkovyh_lokacij_mariupolja.pic.2}

\subsubsection{Загадки старовинного Некрополю}

\ii{29_10_2021.stz.news.ua.misto_marii.1.top_5_zagadkovyh_lokacij_mariupolja.pic.3}

Старовинне кладовище завжди має певну містичну та загадкову атмосферу.
Маріупольський Некрополь, офіційна дата заснування якого, не пізніше 1832 року,
також приховує у собі багато таємниць та цікавинок. Прогулюючись \enquote{Містом
мертвих} можна побачити поховання людей, які розвивали наше місто, а саме:
династії Хараджаєвих, Гофів, Найдьонових... Дізнатися де знайшла свій останній
притулок чарівна парижанка Емма Сторе; побачити місце захоронення доньки
першого просвітника, історика та засновника перших гімназій нашого міста –
Феоктиста Хартахая; почути цікаву історію про священника, який вінчав Архипа
Куїнджі та Віру Шаповалову. Склепи, усипальниці, старовинні плити різних стилів
– Некрополь має ще багато таємниць та історій маріупольських династій!

Існує також легенда про молоду доньку купця Найдьонова, яку поховали в склепі у
весільному вбранні. Коли в склеп прийшли зловмисники, аби вкласти ювелірні
вироби з тіла дівчини – вона розсипалася у них на очах, і тільки її сукня
залишилася лежати у склепі.

То ж якщо вам в житті не вистачає таємниць та містики – ви можете долучитися до
волонтерів \enquote{Некрополю}, кожних вихідних вони збираються на кладовищі, аби
розшукувати заховання видатних маріупольців та відновлювати дореволюційні
пам'ятники та склепи \enquote{батьків нашого міста}.

\subsubsection{Таємничий підвал та сходи Готелю Континенталь}

\ii{29_10_2021.stz.news.ua.misto_marii.1.top_5_zagadkovyh_lokacij_mariupolja.pic.4}

Будівля готелю \enquote{Континенталь} є першої триповерховою будівлею нашого міста. У
XIX столітті це був найвідоміший готель нашого міста. Тут зупинялися заможні
гості Маріуполя, а самі маріупольці приходили розважитися у не зовсім законне
казино. І зараз у підвалі будівлі можна побачити шахту першого грузопід'ємного
ліфту, підіймач для ресторану готелю. А якщо придивитися уважніше на світлини
сходів – ви побачити їхнього автора, адже вони були зроблені на дореволюційному
заводі Сойфера, також можна побачити зірку Давида – як посилання на єврейське
походження власника. І також карткові масті – як згадка про казино.

\ii{29_10_2021.stz.news.ua.misto_marii.1.top_5_zagadkovyh_lokacij_mariupolja.pic.5}
\ii{29_10_2021.stz.news.ua.misto_marii.1.top_5_zagadkovyh_lokacij_mariupolja.pic.6}
\ii{29_10_2021.stz.news.ua.misto_marii.1.top_5_zagadkovyh_lokacij_mariupolja.pic.7}

\subsubsection{Будинок архітектора Нільсена}

\ii{29_10_2021.stz.news.ua.misto_marii.1.top_5_zagadkovyh_lokacij_mariupolja.pic.8}

Містичною атмосферою та легендою й досі оповитий другий будинок архітектора
Нільсена. Кажуть, що цю будівлю зодчий присвятив своїй доньці, яка померла в
досить молодому віці від епідемії тифу. Саме тому серед багатьох маріупольців
ходять чутки, що обличчя доньки й зображено у вигляді маскарону на фасаді
будинку. І також навколо усієї будівлі можна побачити 14 однакових віночків у
вигляді ліпнини, пов'язують цю цифру із віком доньки архітектора – 14 років.

Так то чи ні, й досі не відомо. Але коли йде дощ, по обличчю жіночого маскарона
так стікає вода, що здається це сльози. Ніби то ліпнина плаче за юною донькою
Нільсена. А місцеві називають цей будинок – \enquote{Дім з плачущою німфою}.

\subsubsection{НКВС та ГЕСТАПО у будинку Юр'єва}

\ii{29_10_2021.stz.news.ua.misto_marii.1.top_5_zagadkovyh_lokacij_mariupolja.pic.9}

Одним з відомих вишуканіших будинків центру Маріуполя – є маєток успішного
повіреного Іллі Юр'єва. Сьогодні цей дім – це справжня прикраса сучасного
проспекту Миру, адже у 2020 році будинок був повністю реконструйований.

Небагато хто знає його історію та сумний розділ існування цієї споруди.
Орієнтовно будівля з'явилася на головній вулиці міста наприкінці XIX століття.
Після революції 1917-го року, маєток було націоналізовано. І з 1930-х років
його зайняв комітет НКВС. У цій будівлі вироки не тільки підписували, але й
здійснювали. Всі розстріли та катування проходили у підвалі маєтку. Після того,
як у Маріуполь під час Другої світової війни прийшли німецько-фашистські
загарбники, вони демонстративно винеслі тіла жертв більшовіцького режиму на
проспект Миру. Але потім продовжили цю кроваву справу. У будівлі почав
розташовуватися ГЕСТАПО. Зараз можна уявити скільки жаху та смерті бачив підвал
цієї будівлі.
