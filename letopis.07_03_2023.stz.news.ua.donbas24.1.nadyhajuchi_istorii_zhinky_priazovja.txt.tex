% vim: keymap=russian-jcukenwin
%%beginhead 
 
%%file 07_03_2023.stz.news.ua.donbas24.1.nadyhajuchi_istorii_zhinky_priazovja.txt
%%parent 07_03_2023.stz.news.ua.donbas24.1.nadyhajuchi_istorii_zhinky_priazovja
 
%%url 
 
%%author_id 
%%date 
 
%%tags 
%%title 
 
%%endhead 

Надихаючі історії про непересічних жінок Приазов'я

В історію Маріуполя увійшли сильні жіночі особистості

Аннали історії Маріуполя та Приазов'я загалом зберігають унікальні історії про
жінок, чия непересічність, самовіддана діяльність і винятковий талант
надихають. Вони тендітні та сміливі водночас, вродливі, ніжні, проте нескорені
та незламні духом. Кажуть, що сила жінки в її слабкості, проте життя доводить,
що сила жінки — в її енергії, здібностях та вродженій внутрішній силі.

Читайте також: Парамедик з Маріуполя Тайра номінована на премію Держдепа США

Віра Шаповалова

Досить поширеним є твердження, що за «кожним великим чоловіком стоїть велика
жінка». В історії є безліч прикладів, які підтверджують цю тезу. У Маріуполі
також збереглася своя історія. Зокрема, видатного українського
живописця-пейзажиста Архипа Івановича Куїнджі надихнула на неймовірну роботу
над собою та боротьбу за спільне майбутнє дочка багатого грецького купця,
талановита піаністка Віра Шаповалова. Вона своєю самовідданістю і неповторною
жіночністю повністю підкорила ще не відомого світові художника. Батько Віри,
незадоволений тим, що дочка проводить багато часу зі злиденним художником,
запитав її, чи не зібралася вона за нього заміж? На що дівчина одразу
відповіла: «Якщо не за Архипа, то тільки в монастир». Шаповалов поставив
Куїнджі умову: принесеш сто рублів золотом — Віра твоя. Для Архипа це була
непідйомна сума. Він вирішив їхати в Петербург на заробітки. Та завдяки
закоханості у Віру Архип Куїнджі підкорив Санкт-Петербург і, повернувшись до
неї, отримав дозвіл від її батька на одруження. Однак чекати закоханим довелося
довго — цілих 12 років. І все ж таки сильна Віра дочекалась свого єдиного і
неповторного Архипа Івановича. Після вінчання Куїнджі написав портрет дружини.
Її обличчя на картині світилося любов'ю. А щасливий Архип Іванович з того часу
творив один шедевр за іншим.

Людмила Радіонова

Одна з найбільш яскравих актрис Маріупольського театру, який працював до Другої
світової війни, була Людмила Радіонова. Працювала вона в Маріупольському театрі
з 22 років. Цікаво, що свій акторський шлях вона почала блискуче — відразу з
головних ролей. Лариса в «Безприданниці» О. Островського відразу ж зробила їй
ім'я. Незабаром жінка-комісар в «Оптимістичній трагедії» В. Вишневського, ролі
в «Любові Яровій», «Марійці». Всього за один рік актриса встигла зіграти
стільки яскравих ролей, скільки іншим, дай Бог, за все життя зіграти. М.
Соколов, народний артист Української РСР, згадуючи про Людмилу Радіонову,
підкреслив: «Так, талантом її природа обдарувала щедро. Але, мабуть, всього
важливіше її талант людини — мужньої, красивої, яскравої; її талант жити без
оглядки, не шкодуючи себе». Втім талановита актриса вирішила, що під час війни
вона не зможе лишатися осторонь, тому одразу ж стала безпосередньою учасницею
воєнних дій. Її бойовий шлях відзначений орденом Леніна, двома орденами
Червоної Зірки, сімома бойовими медалями, медаллю Міжнародного Червоного
Хреста.

Читайте також: Жінкам-підприємицям надають гранти в 100 тис. грн та подорож у
Париж: як отримати

З початком війни Л. Радіонова стала санінструктором 75-го стрілецького полку. У
с. Миколаївці під Таганрогом вона винесла з вогню і врятувала від неминучої
загибелі сорок п'ять поранених бійців і командирів. У бою під Таганрогом
відважна Людмила розстріляла з револьвера весь екіпаж нацистського танку, за що
була нагороджена орденом Леніна. З 4 по 6 березня 1944 р. надала допомогу 213
бійцям і командирам, пораненим в бою. Була нагороджена вищою відзнакою
Міжнародного комітету Червоного Хреста — медаллю «Флоренс Найтінгейл». Ця
нагорода носить ім'я однієї з засновниць Міжнародного Червоного Хреста і
вручається медсестрам і санітаркам, які «відзначилися виключною самовідданістю
при догляді за пораненими та хворими під час війни або громадського лиха». У 27
років вона стала інвалідом. Театр довелося залишити... 

Єлизавета Бірюкова 

Під час Другої світової війни маріупольський інженер рибної промисловості
Єлизавета Бірюкова врятувала не менше 150 осіб. Вона не встигла евакуюватися,
хворіла на скарлатину. Молода жінка з вищою технічною освітою під час війни
залишилася в місті й наважилася піти працювати на німців, хоч і прекрасно
усвідомлювала, чим це може для неї закінчитися. Працювала на рибоконсервному
комбінаті сумлінно і налагодила настільки потужне виробництво консервів, що
німці не могли натішитися її роботою. Та, судячи з матеріалів справи, далі вона
абсолютно системно починає рятувати людей, допомагає військовополоненим і
містянам. Наприклад, Єлизавета Бірюкова наказувала залишати на відрізаних
головах риб м'ясо, яке потім варила і відносила в табір військовополонених.
Після звільнення Маріуполя жінку звинуватили в колабораціонізмі й доносах. Їй
присудили 15 років позбавлення волі. Засуджена Єлизавета Бірюкова відбувала
покарання у Воркуті — там, незважаючи на важкі каторжні роботи, на всю
несправедливість долі, вона продовжувала зберігати людське обличчя.
Маріупольці, яких врятувала Єлизавета Бірюкова, писали листи Берії, Сталіну,
головам НКВС, благаючи про помилування. Люди заявляли: «Єлизавета Бірюкова
перетворила рибний комбінат на табір спасіння, забирала чоловіків з таборів для
військовополонених, запевняючи офіцерів вермахту, що це найкращі спеціалісти,
без яких комбінату не обійтись. Годувала у столовій членів сімей комуністів та
євреїв. Немічним та хворим возила харчі додому. Загалом від долі остарбайтерів
врятувала 150 людей».

Читайте також: Переселенка з Донеччини вивезла швейну фабрику та відновила
бізнес у Полтаві (ФОТО)

Завдяки листам батька Єлизавети, небайдужих маріупольців та її скарзі головному
військовому прокурору Збройних сил СРСР, яку вона написала, коли отримала
дозвіл на листування, справу Є. Бірюкової все ж таки переглянули. Повторний суд
відбувся у лютому 1948 року в м. Сталіно. Однак Єлизавета Пилипівна була
визнана винною в пособництві, формою якого назвали ефективну роботу на керівній
посаді рибоконсервного комбінату. Мірою покарання було названо лише чотири роки
й чотири місяці каторжних робіт, які на момент суду були відбуті. Втомлена, але
незламна жінка не погодилася з вироком і все життя намагалася отримати
реабілітацію. Все ж Єлизавета Пилипівна була реабілітована в 1992 році, але за
фатальним збігом обставин і незрозумілим сценарієм життя цього ж року вона
померла.

Світлана Отченашенко

Однією з найпопулярніших у Маріуполі актрис завжди залишалася і залишається
народна артистка України, лауреат премії ім. Заньковецької Світлана
Отченашенко. Створені нею образи ставали предметом глядацького відгуку й
захоплення. На відміну від Полтави, в Маріуполі молоду Світлану Отченашенко
відразу ж ввели в поточний репертуар театру. Вистава за п'єсою Ю. Едліса
«Крапля в морі» (1966 р.) стала першою роботою актриси. Олівцем від руки в
програмку вписано: «Віра — Отченашенко». Після ролей Валі Анощенко («Російські
люди» К. Симонова), Ніси («Дурочка» Лопе де Вега) і Тані Свєтлової («Гліб
Космачов» М. Шатрова) в колективі закріпився статус С. Отченашенко як однієї з
провідних актрис театру. І як підтвердження тому — звання заслуженої артистки
України, отримане в 1972 р., у 27 років! З кожною новою роллю зростала і
майстерність актриси. Одна з ключових ролей для Світлани Іванівни — це роль
Марії Каллас («Майстер-клас» «Страсті по Марії»). Загалом, про зіграних
персонажів народної артистки України Світлани Отченашенко можна сказати, що всі
вони сильні, неоднозначні та незабутні. Попри те, що вона була актрисою
далекого провінційного театру, їй вдалося стати відомою в столичних професійних
колах. Світлані Іванівні неодноразово присвячував статті відомий журнал
«Театр». Актриса, в чиєму репертуарі Аркадіна, Раневська, Аббі Патнем, Марія
Каллас і театральний критик — в одній особі.

Читайте також: 20-річна дівчина із Закарпаття чекає на первістка від 60-річного
переселенця з Маріуполя

У актриси був досвід і в кіно. Вона зіграла бабу Зою у фільмі «Зачароване
кохання» (2008 рік). Завдяки енергійній діяльності актриса змогла зберегти
історичну театральну спадщину Маріуполя, надихнувши телеканал «Сигма» на
створення унікального циклу, присвяченого історії та людям Маріупольського
театру. Мало кому відомо, що актриса брала активну участь в створенні
бібліотеки в маріупольському театрі. Світлана Іванівна дуже любила
маріупольців, підкреслювала, що вони мають виняткову театральну культуру, дуже
вдячні глядачі. Актриса завжди щиро переживала за своє місто і дуже боляче
переживала за те, що місто почали руйнувати… Великим ударом для неї було
знищення військами рф будівлі Донецького академічного обласного драматичного
театру (м.Маріуполь), в якому вона пропрацювала понад 50 років. Вона вирішила
залишатися у рідному місті попри воєнні дії і небезпеку для життя. Померла
Світлана Іванівна Отченашенко 7 квітня внаслідок набряку легенів у 2 міській
лікарні Маріуполя.

Валерія Карпиленко

Валерія Суботіна (Карпиленко) на позивний Nava — маріупольська поетеса,
кандидатка наук із соціальних комунікацій, у минулому випускниця й викладачка
Маріупольського державного університету, яка разом з «Азовом» захищала рідний
Маріуполь. Валерія завжди відрізнялася неабиякою силою волі, оптимізмом та
винятковим талантом. Для багатьох маріупольців вона є справжнім взірцем для
наслідування. 24 лютого Валерія Карпиленко відчувала — на порозі повномасштабна
війна. Жінка пішла до військкомату і її мобілізували до полку «Азов». Разом зі
своїм коханим — Андрієм Суботіним (позивний Борода) вони опинилися
заблокованими на «Азовсталі». Проте там вони майже не бачились і на спілкування
часу не було. Історія кохання Валерії Карпиленко та прикордонника Андрія стала
відомою не тільки в Україні, але й за кордоном. Пара познайомилась, коли
Валерія працювала в пресслужбі Донецького прикордонного загону. В облозі на
«Азовсталі» вирішили одружитись. Датою обрали 5 травня, бо це день народження
полку «Азов». Проте 7 травня Андрій загинув. Валерія вже 9 місяців перебуває в
полоні…

Читайте також: Від математикині — до першої леді: вісім українок увійшли в
сотню найвпливовіших жінок року

Маріуполь пов'язав долі Віри Куїнджі, Людмили Радіонової, Єлизавети Бірюкової,
Світлани Отченашенко та Валерії Карпиленко, які, безумовно, заслуговують
сьогодні особливої уваги та пам'яті. Ці сильні жіночі особистості є важливою
частиною не тільки історії Приазов'я, але й загалом історії нашої незламної
країни.

Нагадаємо, раніше Донбас24 розповідав, що жінки з Донбасу та чотирьох областей
зможуть отримати гуманітарну допомогу.

Ще більше новин та найактуальніша інформація про Донецьку та Луганську області
в нашому телеграм-каналі Донбас24.

ФОТО: з особистого архіва Демідко Ольги та з відкритих джерел
