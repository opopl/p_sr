% vim: keymap=russian-jcukenwin
%%beginhead 
 
%%file _main_
%%parent 
 
%%endhead 
 
%%file f_main

\newif\ifDEBUG
\DEBUGfalse

\newif\ifINDEX
\INDEXtrue

\newif\ifcmt
\cmtfalse

\def\PROJ{letopis}
\def\ii#1{\InputIfFileExists{\PROJ.#1.tex}{}{}}
 
\ii{preamble}
\ii{defs}
 
\begin{document}

%\ii{titlepage}
 
\ii{listfigs}
\ii{listtabs}
\ii{tabcont}
 
\ii{body}
\ii{index}
%\ii{bib}

\end{document}

%ombaranovskyi (at) gmail dot com
%https://www.facebook.com/yaremenko.sergiy/posts/pfbid0rm7p8Qsi41hgemJrkwn7rgMGXSLtLW6ZXAegKGFaFwvhfa5Zx7GLigabcC4rVjJbl
%https://www.facebook.com/desn.rda/posts/pfbid028j7NvqGmhQF9fULm5xmT6GLCvtkuD3vC5V5okompQK8EvLsEx9fqVSskAjwGWvqtl

Доброго вечора. Слава Україні! 

Сьогодні 37-річниця Чорнобильської аварії. Для мене як для киянина... короче,
ціла величезна тема, і я про це напишу окремо. А зараз я хотів би трохи
подумати над таким, оскільки я програміст (про все це я теж напишу більш
докладно). Справа в тому, что ЧАЕС свого часу вважалась найсучаснішим
реактором. По-моєму, академік Александров, чи хтось інший, казав, що
ймовірність аварії на ЧАЕС така, що ... не памьятаю точно його вираз... ну
скажімо така, що ви йдете Хрещатиком і Вам на голову падає астероїд. Короче,
було таке собі шапкозакідательство, що мирний атом ляляля, абсолютно надійно,
товариші фізики все знають, все чьотко. Як воно вийшло насправді (також
враховуючи пізнішу атомну катастрофу на Фукусімі), всі все це знають. А я хотів
тут дещо сказати в розрізі моєї професії як програміста. Ну ок. Тут такий трохи
потік думок... але.. ЧАЕС був продукт свого часу, і всі вважали її абсолютно
надійною, як і інші атомні станції. Але станція вибухнула і спричинила
неймовірно кількість нещасть по всій планеті. От воно як виходить. Вчені щось
створили начебто абсолютно надійне, а воно взяло і вибухнуло. Так от. Я хотів
би сказати пару слів про соцмережі та Інтернет взагалі. Справа в тому, что
зараз у нас іде війна із московією, і однією із суттєвих особливостей цієй
війни, навіть я би так сказав, такою особливістю, яка докорінно відрізняє її
від минулих воєн, як-от друга світова війна, або фінська кампанія, або В'єтнам,
або якась інша війна, є те, що вона також відбувається у кіберпросторі, і також
у ментальному та метафізичному просторах. Кіберпростір - ну тобто є наші
канали, сторінки, і т.д. і є орківські, там, бухі медвєдєви, машки-алкашки і
т.д. Воно може і смішно звучить, аде з їх сторони все це лайно потім влізає в
голови простих росіян, і в результаті отримується така собі армія зомбі, яка
підтримала і продовжує підтримувати вже всі злочини росіян в реальному
просторі. Тобто, смерть, зло, дурість, безумство, руйнування у віртуальному
світі потім призводить до руйнувать і смертей реальних.

І в принципі всі це і так знають, але можливо не всі задумуються над тим, що
Інтернет в глобальному сенсі, і фейбук або телеграм зокрема, є таким собі
досить особливим світом. Окрема Віртуальна Реальність, у якої свої закони,
якісь закономірності і т.д. І насправді, серйозні вчені, навіть фізики та
математики, зараз активно займаються дослідженням Інтернету та соцмереж як
окремого явища,... І т.д. Тут у мене зараз немає багато часу писати. Але.
короче. Внаслідок цієї війни є дуже багато дуже цінної інформації, як-от 

(1) щоденники блокадного Маріуполя, у вигляді фейсбук-сторінок, які самі по
собі є доказом злочинів росіян

(2) всі ті фейсбук-сторінки Маріуполя. до 24.02.2022, є частиною Цифрової
Спадщини Маріуполя, є Конденсованим Згустком Духовної Енергії цього дуже
цікавого, енергійного, та красивого міста, у якому, я на жаль, ніколи не був.
І надійне збереження всієї цієї цифрової спадщини в максимально близькому до
оригіналу форматі є дуже важливою передумовою подальшого духовно-культурного
відродження Міста Марії. Це дійсно так.

(3) інші фейсбук сторінки, альбоми і т.д. і т.д. - тисячі, мільйони записів
різних авторів і т.д. в Україні за різні роки - все це разом є сукупним
Цифровим Образом України, і має величезну цінність саму по собі. 

Ну а тепер. Все це до чого. Справа в тому (обмежимось фб) що все це може бути
зруйновано за мить, внаслідок якоїсь наприклад технічної помилки з боку
сервера. Або. Якийсь топ-модератор десь або програміст там в Каліфорнії напише
програмку і вирішить у свій вільний час трохи... поексперементувати... Ну от
прямо як робітники на ЧАЕС, які проводили (начебто контрольований) експеримент,
який в результаті призвів до глобальної ядерної катастрофи.

А щодо Інтернету та Фейсбуку... короче, почну з простого прикладу. На моєму
ноуті стоїть операційна система Убунту. Там є можлива практично така цікава команда,
називається так

sudo rm -rf /

Пояснення:

sudo - права адміністратора
rm - remove - видалити
/  - кореневий каталог, тобто в ньому міститься все інше
-rf - два ключі - ( -r recurse рекурсивно по директоріям -f force насильно ) 

Короче. Достатньо мені по дурі або по пьяні сісти за ноут, зайти як адмін і
набрати цю команду і натиснути Enter, як все, в принципі все, що важливе для
мене міститься у мене,  на моєму ноуті зникне. Ну просто зникне і все. І усьо.
Але тут в принципі трагедії вселенської немає ніякої. Ну, дурний, витер собі
все на ноуті. Жалко фоточок, там, якихось файлів і усьо. Сам винуватий, пішов
собі, можливо важкими зусиллями відновив частину файлів і т.д. Але. Справа в
тому, что на глобальному рівні - тобто на рівні наприклад фб, а це реально
мільйони - мільярди користувачів по всьому світу - глобальний збій, чиїсь дурні
експерименти на рівні адміністрування всього фейсбуку або чиїсь спеціально
створений вірус може запросто зробити недоступними на тривалий час, або знищити
назавжди, абсолютно безцінні цифрові дані, як от ті ж самі щоденники, які,
звісно, буде неможливо потім відтворити. Також, дуже багато всього управляється
через інтернет, там, АЕС, транспорт і т.д.

Так, тему я задав, а розпишу якось пізніше. Висновок такий.  Чорнобиль - ядерна
катастрофа у фізичному просторі вже стався 37 років назад, хоча в масовій
свідомості атомна енергії було синонімом абсолютно безпечної енергії. Зараз у
нас Цифрова Ера. Ми всі користуємось смартфонами, ноутами, оплачуємо,
спілкуємось, репортажі з фронту, ролики, що роблять самі військові і т.д.
Інтернет, фб, телеграм стали невідьємною частиною нашого життя до такого рівня,
всі вже настільки звикли до цього, що в принципі до-цифрова епоха уявляться
трохи вже як така собі епоха печерних людей. Як це так, а що, дійсно колись
люди не мали фб або навіть смартфона? аааааааааааааа, що ж це таке, щоби
домовитись з дівчиною про побачення, треба було іти до таксофона на вулиці і
дзвонити у гуртожиток і просити, щоби та конкретна дівчина підійшла до
слухавки?!  А якщо раптом телефон в общазі змамався, що ж тоді робити?  ото то
дикі темні часи були!... А колись і того не було, і телефону і телеграфу,
тільки поштові станції між містами, так то напевне взагалі часи динозаврів
були! Також, лист міг йти до адресата цілий місяць, замість того, щоби просто
зробити скайп-дзвінок, - це взагалі неможливо собі уявити, в яких жахливі умови
люди були поставлені! (а за часів Д'артаньяна та трьох мушкетерів, якщо читали,
- Двадцять років опісля, - Д'артаньян взагалі спочатку не міг знайти своїх
старих друзів, тому що він загубив листа від Араміса на полі бою, і все -
зв'язок був утрачений, де той Араміс - в якомусь монастирі... їх тисячі по всій
Франції! - а якшо він у жіночому монастирі стирчить, то що робити тоді?)  Але
Слава Богу, зараз вже все по іншому! - якось хтось десь мені так сказав,
турботливо з посмішкою поглажуючи пальчиками, як якесь маленьке кошеня, свій
найновіший Айфон... Так от. Цифровий Чорнобиль - тобто, глобальний збій Мережі,
фб, телеграму, АБСОЛЮТНО МОЖЛИВИЙ. (в 2020 вже був такий відносно малий збій в
роботі фб та телеграму  - то чому не може статися чогось набагато більш
серйознішого - ну звичайно може!!) І це буде мати абсолютно катастрофічні
наслідки для всього людства. В меншому масштабі, в масштабі України і контексті
нашої перемоги над московією і шаого прагнення покарати росію за всі її
злочини, це призведе до втрати назавжди величезної кількості неймовірно
важливих даних, як от тих самих Маріупольських блокадних щоденників, або ж
памьять про наших загиблих Воїнів, як от Дениса Галушко, який загинув в січні
під Бахмутом, вічна йому Пам'ять!  ( маю на увазі його фейсбук-сторінку - якщо
ця фейсбук сторінка буде втрачена внаслідок якогось технічного збою, ну... це
буде жахливо. Смерть людини це завжди трагедія, але смерть пам'яті - можливо,
ще страшніша річ... - АЛЕ. На щастя, майже 40 постів із сторінки Дениса я
зберіг, і цифровий Чорнобиль конкретно щодо пам'яти цього Героя вже не так
страшний).

А як це все може виглядати в реалі (глобальна цифрова катастрофа, глобальний
Цифровий Апокаліпсис), можете подивитись фільм Міцний Горішок-4 

to be continued

%21:57:12 27-04-23
Андрій Жованик. Світ як воля та уява фільм
