% vim: keymap=russian-jcukenwin
%%beginhead 
 
%%file 21_10_2021.fb.fb_group.story_kiev_ua.1.lavra_1898_hromolitografia
%%parent 21_10_2021
 
%%url https://www.facebook.com/groups/story.kiev.ua/posts/1779812152215578
 
%%author_id fb_group.story_kiev_ua,dubinina_oksana
%%date 
 
%%tags kiev,lavra
%%title Общий видъ Киево-Печерской Лавры. 1 мая 1898 год
 
%%endhead 
 
\subsection{Общий видъ Киево-Печерской Лавры. 1 мая 1898 год}
\label{sec:21_10_2021.fb.fb_group.story_kiev_ua.1.lavra_1898_hromolitografia}
 
\Purl{https://www.facebook.com/groups/story.kiev.ua/posts/1779812152215578}
\ifcmt
 author_begin
   author_id fb_group.story_kiev_ua,dubinina_oksana
 author_end
\fi

Общий видъ Киево-Печерской Лавры. 

1мая 1898 год

Хромолитография Е. И. Фесенко, Ришельевская 47, въ Одессъ

От С.-Петербуржского Духовного Цензорного Комитета печатать дозволяем.

Старший цензор Архимандритъ Климентъ 

Меня заинтересовала хромолитография нашей древней и всегда величественной
Печерской Лавры. Рисунку более 120 лет, а черты Лавры почти неизменны. Разве
что удивляет наличие восточных путей подъезда к храму, земляные тропы по
территории  и еще свечной заводик)

\ii{21_10_2021.fb.fb_group.story_kiev_ua.1.lavra_1898_hromolitografia.pic.1}

На каждом объекте можно рассмотреть номера и вот подписи к ним:

1. Соборъ; 2. Колокольня; 3. Святыя ворота; 4. Владычные покои и Церковь
Христовая; 5. Церковь подъ покоями; 6. Трапезная церковь; 7. Братская кухня; 8.
Живописная школа; 9. Типография; 10. Корпусъ типографской братии; 11.
Словолитня; 12. Палата провизіи; 13. Церковь при входъ въ ближнія пещеры; 14.
Тёплая церковь ближнихъ пещер; 15. Корпус братіи ближн. пещер; 16. Кръпостная
башня; 17. Колодезъ; 18. Кръпостная башня; 19. Церковъ на гостинницъ; 21.
Колокольня ближн. пещеръ; 22. Корридоръ дальнія пещеры; 23. Корпус для 4-хъ
братій; 24. Тоже; 25. 1-й Гостиннецкій копусъ; 26. 2-ц Гостинн. корп,; 27. 3-й
Гостинн. корп.; 28. Трапеза странниковъ; 29. Церковь на гостинницъ; 30. Угол
свъчнаго завода; 31. 8-й Корпусъ на гостинницъ; 32. Башня съ Юго-западнаго
угла.

* это не опечатки)

\ii{21_10_2021.fb.fb_group.story_kiev_ua.1.lavra_1898_hromolitografia.cmt}
