% vim: keymap=russian-jcukenwin
%%beginhead 
 
%%file slova.konkurs
%%parent slova
 
%%url 
 
%%author_id 
%%date 
 
%%tags 
%%title 
 
%%endhead 
\chapter{Конкурс}

%%%cit
%%%cit_head
%%%cit_pic
\ifcmt
  tab_begin cols=3
     pic https://storage.lug-info.com/cache/9/d/416c8418-6edf-4881-a962-1e4fd49df6ab.jpg/w1000h616%7Cwm
     pic https://storage.lug-info.com/cache/6/8/c4c339cf-b91b-48e4-ae3d-a92bea81c152.jpg/w1000h616%7Cwm
		 pic https://storage.lug-info.com/cache/a/0/d046628c-21e5-4f43-ac3c-0895f14a73a0.jpg/w1000h616%7Cwm
  tab_end
\fi
%%%cit_text
Учащиеся общеобразовательных учреждений Республики представили почти 1 тыс.
работ на \emph{конкурс} рисунка \enquote{Мой город в радуге профессий}, который организовал
Республиканский центр занятости (РЦЗ). Об этом сообщили в учреждении.
Творческое состязание было направлено на профориентацию учащихся и повышение
престижа рабочих профессий.  \enquote{\emph{Конкурс} проводился в два этапа. Для участия в
первом этапе \emph{конкурса}, который проводился территориальными отделениями (РЦЗ),
233 общеобразовательными организациями представлено почти 1000 детских работ,
выполненных в разных жанрах. В работах ребят отображены наиболее востребованные
сегодня профессии: врач, парикмахер, повар, сварщик, строитель, учитель,
шахтер, швея и другие}, - говорится в сообщении.  Во второй этап прошли 126
работ, авторами которых стали учащиеся 94 общеобразовательных организаций.
Оценивание работ проводилось в трех возрастных категориях: с 1-го по 4-й класс,
с 5-го по 8-й класс и с 9-го по 11-й класс
%%%cit_comment
%%%cit_title
\citTitle{Школьники представили почти 1 тыс. работ на конкурс рисунка \enquote{Мой город в радуге профессий}}, 
, lug-info.com, 17.11.2021
%%%endcit

%%%cit
%%%cit_head
%%%cit_pic
\ifcmt
  tab_begin cols=2
     pic https://lgaki.info/wp-content/uploads/2021/11/izobrazitelnyj-diktant-1536x709.jpg
  tab_end
\fi
%%%cit_text
29 и 30 ноября 2021 года студенты 1-го и 2-го курсов специальности «Живопись»
отделения изобразительного искусства колледжа Академии Матусовского впервые
приняли участие во Всероссийском изобразительном диктанте-2021, который
проводится в рамках Международного благотворительного конкурса «Каждый народ —
художник». Организатор конкурса — Международный союз педагогов-художников.
Общая тема \emph{конкурса} в этом году: «Моя страна – моя история». Нашим
участникам организаторы предложили самим выбрать темы, связанные с историей
города и края, историческими личностями и местными достопримечательностями.
Студенты работали в течение трех часов, каждый по-своему воплощая неповторимый
образ Донбасса, его героев, историю и культуру родной земли.  — Этот
\emph{конкурс} помог каждому участнику оценить свои способности, закрепить
знания и умения, полученные на занятиях, а также почувствовать себя частью
большого дружного художественного сообщества, — рассказали наши участники
диктанта
%%%cit_comment
%%%cit_title
\citTitle{Впервые участвовали во Всероссийском изобразительном диктанте}, 
, lgaki.info, 30.11.2021
%%%cit_url
\href{https://lgaki.info/novosti/vpervye-uchastvovali-vo-vserossijskom-izobrazitelnom-diktante}{link}
%%%endcit
