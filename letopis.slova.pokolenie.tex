% vim: keymap=russian-jcukenwin
%%beginhead 
 
%%file slova.pokolenie
%%parent slova
 
%%url 
 
%%author 
%%author_id 
%%author_url 
 
%%tags 
%%title 
 
%%endhead 
\chapter{Поколение}
\label{sec:slova.pokolenie}

%%%cit
%%%cit_head
%%%cit_pic
%%%cit_text
Коли з’явилась українська мова, як говорили і писали на Русі, чому у російських
билинах багато українських слів? Про мовний штаб Путіна, хто вигадав
російськомовних українців, як СРСР створив \emph{покоління} мовних калік.
Тепер ми в Instagram. Підписуйтеся, Вам сподобається.  У свіжому випуску «Без
брому» інтерв’ю із мовознавцем, очільником Інституту української мови Павлом
Гриценком
%%%cit_comment
%%%cit_title
\citTitle{«Ми маємо кілька поколінь мовно скалічених людей»}, 
zaxid.net, 06.07.2021
%%%endcit

%%%cit
%%%cit_head
%%%cit_pic
%%%cit_text
Да чего далеко ходить. Каких-то 30 лет назад распался Советский Союз. И хотя
ответ на вопрос «как дальше» был дан буквально в ходе этого распада – по
республикам, но для миллионов переживающих этот переход (всего напоминать не
буду, многие еще помнят), - это было время катастроф.  Подавляющее большинство
тех, кто представлял наше \emph{поколение} – «перестроечное», этот переход
восприняли болезненно, но с надеждами. Мы готовились перестраивать старый
рушащийся «мир» на новый, и были уверены, что «темное время» ненадолго.  И та,
пост-советская Украина была новым, но близким и понятным «миром», который нужно
обустроить как можно быстрее.  Но... «Конец» делал паузы, но не прекратился.
Страна, люди, территории, духовная жизнь, экономические устои и нормы
социальной жизни изменились сегодня до неузнаваемости. Новые кумиры – вчерашние
артисты, спортсмены, прохвосты и криминал. Бизнес пронизал всю жизнь насквозь,
и как мне с грустью заметил один из замов генпрокурора (уже из бывших), теперь
«госслужба – это место для первоначального накопления капитала»
%%%cit_comment
%%%cit_title
\citTitle{Пришел конец старому теплому украинскому миру. Начинается что-то другое / Лента соцсетей / Страна}, Андрей Ермолаев, strana.news, 29.11.2021
%%%endcit
