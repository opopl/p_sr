% vim: keymap=russian-jcukenwin
%%beginhead 
 
%%file 28_02_2022.fb.fb_group.story_kiev_ua.2.vypechka.cmt
%%parent 28_02_2022.fb.fb_group.story_kiev_ua.2.vypechka
 
%%url 
 
%%author_id 
%%date 
 
%%tags 
%%title 
 
%%endhead 
\zzSecCmt

\begin{itemize} % {
\iusr{דני כהן}
Никогда не доносил до дома эту корочку целой... ммммммммм....

\iusr{Наталия Озерова}
Улюблений хліб- паляниця!

\iusr{Александр Венге}
\textbf{Natalia Ozerova} а ещё пароль!

\iusr{Наталия Даниленко}
Давно його не коштувала, не печут!

\iusr{Дмитрий Гаранжа}

Самый простой заменитель хлеба.

Что нужно:

Мука, соль вода, любой жир (хотя можно и без жира)

Грамм 300 муки залить примерно 100 граммами кипятка, добавить щепотку соли и
начать перемешивать массу вилкой. (эстетам можно добавить специи типа перца,
куркумы и т.п.)

Когда масса остынет продолжить перемешивание руками. Разделить массу на шарики,
размером с куриное яйцо или мячик для пинг-понга. Раскатать с помощью скалки
или бутылки в блин. Нет скалки, можно руками. Бросить на сковородку с любым
маслом или жиром и обжарить с двух сторон. (Блин желательно проткнуть вилкой,
чтоб не пузырился) Вишенка на торте - выдавить пару зубчиков чеснока с помощью
чеснокодава в чашку, залить кипятком и сбрызнуть этой жидкостью готовые хлебные
лепешки

Те, кто попробует эти хлебные лепешки задумаются, а стоит ли вообще покупать
хлеб.

Если нет масла, жира и т.п. можно запечь лепешки в духовке или микроволновке,
но будет суховато и не так вкусно

\begin{itemize} % {
\iusr{Оксана Дубинина}
\textbf{Дмитрий Гаранжа} хлебные лепёшки - ням-ням)) Спасибо!
Вообще в мужские руках еда получается особо вкусной!

\iusr{Дмитрий Гаранжа}
\textbf{Оксана Дубинина} )

\iusr{Ассоль Грей}
\textbf{Dmitriy Garanzha} ...зато полезно)

\iusr{Дмитрий Гаранжа}
\textbf{Ассоль Грей} И вкусно и полезно. Я другого и не посоветую - только проверенное и самое лучшее

\iusr{Даша Андреева}

Дуже вредна їда жарена @igg{fbicon.face.sad.but.relieved} 

\iusr{Дмитрий Гаранжа}
\textbf{Даша Андреева} ну так готуйте все на пару якщо так вважаєте
\end{itemize} % }

\iusr{Ирина Клименко}

А как мой од нлкий старенький папа, 86 лет, живя на троещине, может испечь
хлеб? Мы к нему не можем с правого переправить я. Что делать!? SOS!

\begin{itemize} % {
\iusr{Оксана Дубинина}
\textbf{Ирина Клименко} попробуйте написать объявление, я думаю отзовутся немало соседей. У нас в группе много с Троещине

\begin{itemize} % {
\iusr{Ирина Клименко}
\textbf{Оксана Дубинина} все рядом соседи уехали.

\iusr{Оксана Дубинина}
\textbf{Ирина Клименко} а группе напишите небольшой пост об этом. А кто сможет с Вами в личке пообщаются

\iusr{Ирина Клименко}
\textbf{Оксана Дубинина} скажите пожалуйста, какой именно группе, он живёт на Маяковского, в конце...

\iusr{Оксана Дубинина}
\textbf{Ирина Клименко} напишите у нас, в Киевских историях. Может фото папы, можно без фото. Вот увидите, отзовутся люди. А потом надо поискать местные группы на Троещине.
Мне сегодня батюшка-священник прислал контакты, кто помогает нуждающимся. Сейчас сюда пришлю

\iusr{Оксана Дубинина}
\textbf{Ирина Клименко}
\end{itemize} % }


\end{itemize} % }

\end{itemize} % }
