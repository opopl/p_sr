% vim: keymap=russian-jcukenwin
%%beginhead 
 
%%file slova.kitaj
%%parent slova
 
%%url 
 
%%author 
%%author_id 
%%author_url 
 
%%tags 
%%title 
 
%%endhead 
\chapter{Китай}
\label{sec:slova.kitaj}

%%%cit
%%%cit_pic
%%%cit_text
Современные украинские политики, объявившие в 1991 году независимость УССР от
СССР, тоже не придумали ничего нового, как назвать новую страну Украиной, то
есть Окраиной. При этом они называют страну великой, совершенно забывая о том,
что, прежде всего, державу, даже не великую, характеризует в первую очередь
название. Вот посмотрите на названия ведущих стран мира: GB (Великая
Британия), США (Соединенные Штаты Америки), РФ (Российская Федерация), КНР
(\emph{Китайская Народная Республика}), даже Гондурас и Зимбабве официально называются
Республиками. И только Украина как была Окраиной, так ей и осталась даже в
названии, невзирая даже на то, что некогда ее территория была основой довольно
мощного государства – Киевской Руси
%%%cit_title
\citTitle{Почему современную Украину назвали «окраиной», а не более престижно - Киевской Русью?}, 
Исторический Понедельник, zen.yandex.ru, 22.02.2021 
%%%endcit

%%%cit
%%%cit_head
%%%cit_pic
%%%cit_text
При этом торговый партнер номер один для Украины – \emph{Китай}, с которым успели
испортить отношения декоммунизацией и изгнанием \emph{китайских инвесторов} из
\enquote{Мотор-Сичи}. А также \enquote{забыв} в 2019-м поздравить с Днем
образования \emph{КНР}.  Наша промышленная продукция не нужна в ЕС, но при этом нас
кормят сказками о \enquote{промышленном безвизе}, который лишь ускорит
деиндустриализацию украинской экономики.  Ведь европейцы не покупают наши котлы
или насосы не из-за того, что на них нет принятых в ЕС сертификатов, а потому
что они им просто не нужны, что с сертификатами, что без.  В то же время наша
промышленная продукция нужна на постсоветском пространстве
%%%cit_comment
%%%cit_title
\citTitle{Отрицательное торговое сальдо Украины с РФ - патриотичная дыра / Лента соцсетей / Страна}, 
Алексей Кущ, strana.ua, 24.06.2021
%%%endcit

%%%cit
%%%cit_head
%%%cit_pic
%%%cit_text
К 100-летию компартии \emph{Китая}.  Во второй половине 80-х, во времена перестройки,
будучи вольным журналистом, имел возможность общаться с закрытыми в специальную
группу академиками, которые для Горбачева разрабатывали варианты будущего пути
для СССР. Один из них был \enquote{\emph{китайский}}. Предполагавший полное сохранение
лидерских позиций КПСС в стране.  \enquote{Горбатый} отверг этот вариант, настояв на
полном демонтаже командно-административной машины во главе с партией. Что
получили - всем уже очевидно.  Как \emph{китайцы} получили доступ к концепции
\enquote{китайского пути} разработанного учеными СССР доподлинно не знаю. Хотя не
секрет их умение \enquote{тырить} чужую интеллектуальную собственность. И распоряжаться
ею настолько эффективно, что на родине ее авторов, увы не способны. Обидно
одно: шанс у СССР был. Сегодня мы могли бы быть на месте \emph{КНР} в табели о
геополитических рангах.  Не случилось. А \emph{китайцев} можно поздравить. Более
подробно текст ниже
%%%cit_comment
%%%cit_title
\citTitle{На месте Китая могли и должны были оказаться мы / Лента соцсетей / Страна}, 
Валерий Песецкий, strana.ua, 02.07.2021
%%%endcit

%%%cit
%%%cit_head
%%%cit_pic
%%%cit_text
Но в Украине довольно быстро попытались сгладить конфликт и заявили, что с
\emph{Поднебесной} конфликта нет.  Сохранить отношения с \emph{Китаем} на
Банковой хотят по двум причинам: во-первых, это крупный рынок сбыта и торговый
партнер, а также источник инвестиций.  Во-вторых, у Зеленского пытаются создать
противовес давлению США, которые за небольшие кредитные вливания требуют слить
под них судебную систему и государственные компании. И, кстати, в лагере
проамериканских активистов начали власть за \enquote{\emph{китайский} поворот}
критиковать.  За последние дни \emph{прокитайская} риторика только усилилась.
Разбирались, что это значит
%%%cit_comment
%%%cit_title
\citTitle{\enquote{Китай - партнер номер один}. Как Зеленский и \enquote{слуги} хвалят коммунистов КНР и что за этим стоит}, 
Максим Минин, strana.ua, 14.07.2021
%%%endcit

