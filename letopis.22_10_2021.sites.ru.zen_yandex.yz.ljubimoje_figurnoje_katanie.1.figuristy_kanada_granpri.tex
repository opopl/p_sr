% vim: keymap=russian-jcukenwin
%%beginhead 
 
%%file 22_10_2021.sites.ru.zen_yandex.yz.ljubimoje_figurnoje_katanie.1.figuristy_kanada_granpri
%%parent 22_10_2021
 
%%url https://zen.yandex.ru/media/lovely_fk/obeskurajivaiuscaia-novost-nashi-luchshie-figuristy-mogut-ostatsia-bez-etapa-gran-6172e5d46bb5a46713572e63
 
%%author_id yz.ljubimoje_figurnoje_katanie
%%date 
 
%%tags devushka,figurist,figurnoje_katanie,rossia,sorevnovanie,sport
%%title Наши лучшие фигуристы могут остаться без этапа гран-при в Канаде
 
%%endhead 
 
\subsection{Наши лучшие фигуристы могут остаться без этапа гран-при в Канаде}
\label{sec:22_10_2021.sites.ru.zen_yandex.yz.ljubimoje_figurnoje_katanie.1.figuristy_kanada_granpri}
 
\Purl{https://zen.yandex.ru/media/lovely_fk/obeskurajivaiuscaia-novost-nashi-luchshie-figuristy-mogut-ostatsia-bez-etapa-gran-6172e5d46bb5a46713572e63}
\ifcmt
 author_begin
   author_id yz.ljubimoje_figurnoje_katanie
 author_end
\fi

Обескураживающая новость: наши лучшие фигуристы могут остаться без этапа
гран-при в Канаде.

Под вопросом участия: Камила Валиева, Елизавета Туктамышева и Алена Косторная,
а также Евгений Семененко и Макар Игнатов, включая их тренеров Алексея Мишина и
Евгения Рукавицына.

\ifcmt
  pic https://avatars.mds.yandex.net/get-zen_brief/5282241/pub_6172e5d46bb5a46713572e63_6172ea36ca7a3f4a9de7945c/scale_1080
  @width 0.7
  %@wrap \parpic[r]
\fi

Как сообщает федерация фигурного катания России (ФФКР), фигуристы и тренеры до
сих пор не получили паспорта с канадскими визами и не получат их раньше
понедельника, хотя визы были одобрены еще в середине месяца. Сборная России
должна вылететь в Канаду в это воскресенье, 24 октября.

В данный момент ФФКР пытается решить проблему с организаторами турнира и ISU.
Будем надеяться, что удастся изменить даты вылета и маршрут перелетов, иначе
Skate Canada рискует потерять сильнейших участников. Кто будет смотреть женский
турнир без Валиевой, Туктамышевой и Косторной?
