% vim: keymap=russian-jcukenwin
%%beginhead 
 
%%file 05_04_2021.fb.bilchenko_evgenia.2.greh
%%parent 05_04_2021
 
%%url https://www.facebook.com/yevzhik/posts/3778242315544184
 
%%author 
%%author_id 
%%author_url 
 
%%tags 
%%title 
 
%%endhead 
\subsection{Грех}
\label{sec:05_04_2021.fb.bilchenko_evgenia.2.greh}
\Purl{https://www.facebook.com/yevzhik/posts/3778242315544184}

\ifcmt
  pic https://scontent-bos3-1.xx.fbcdn.net/v/t1.6435-9/169912713_3778242012210881_5470967239592431380_n.jpg?_nc_cat=105&ccb=1-3&_nc_sid=8bfeb9&_nc_ohc=s7TbL_pEUiUAX9nHnFq&_nc_ht=scontent-bos3-1.xx&oh=df97922b3a242c4d91a6cd96686b9ec8&oe=6094B621

	pic https://scontent-bos3-1.xx.fbcdn.net/v/t1.6435-9/169065870_3778242095544206_9333681266157558_n.jpg?_nc_cat=104&ccb=1-3&_nc_sid=8bfeb9&_nc_ohc=OZkns4gQQKAAX_ga3zl&_nc_ht=scontent-bos3-1.xx&oh=596f3247c8cfe73cb2318d28968effde&oe=60936787

	pic https://scontent-bos3-1.xx.fbcdn.net/v/t1.6435-9/168439630_3778242212210861_3973934922422169097_n.jpg?_nc_cat=106&ccb=1-3&_nc_sid=8bfeb9&_nc_ohc=0RMGhfFTjAMAX-waqZ4&_nc_ht=scontent-bos3-1.xx&oh=ea6731a4208c1cf5311c055be945c340&oe=6093F506
\fi

Из выступления профессора Владимира Степановича Возняка на вчерашней
конференции по Андрею Тарковскому.

…Пауль Тиллих пишет: «Мужество быть это мужество утверждать нашу собственную
разумную природу вопреки всему тому в нас, что противостоит нашему единению с
разумной природой самого бытия» [4]. Итак: утверждать разумную природу, нашу
собственную разумную природу. Она – тождественна разумной природе самого бытия,
а посему мужество быть решительно противостоит нашему разъединению с разумной
природой бытия. Впрочем, об этом же говорил и Гераклит Эфесский. И далее у
Пауля Тиллиха: «Мужество быть – это мужество утверждать превосходство нашей
собственной разумной природы вопреки всему случайному в нас» [4]. – Пожалуй,
лучше не скажешь.

Павел Флоренский напрямую связывает само-утверждение с грехом: «Грех – в
нежелании выйти из состояния само-тождества, из тождества Я=Я, или точнее – Я!
Утверждение себя, как себя, без своего отношения к другому, – т.е. к Богу и ко
всей твари, само-упор вне выхождения из себя и есть коренной грех, или грех
всех грехов» [5, с. 160]. Утверждать себя как только себя – значит желать
только себя: «Желая только себя, в своем «з д е с ь» и «т е п е р ь», злое
само-утверждение негостеприимно запирается ото всего, что не есть оно; но,
стремясь к само-божеству, оно даже самому себе не остается подобным и
рассыпается и разлагается и дробится во внутренней борьбе. Зло по существу
своему – “царство разделившее на ся”» [5, с. 156]. Зло свершает
«раздробительное действие». Распад, дробление, расщепление, поругание цельности
и целостности, «трещина во всеединстве» (С.Л. Франк), противопоставление
истинно всеобщему как противопоставление истине и истинному. – Вот к чему ведёт
жажда человеческого самоутверждения. Утверждения себя за счёт других и против
других. «Само-утверждение личности, – полагает П.А. Флоренский, –
противо-поставление ее Богу – источник дробления, распадения личности,
обеднения ее внутренней жизни; и лишь любовь, до известной степени, снова
приводит личность в единство. Но если личность, уже отчасти распавшаяся, опять
не унимается и хочет сама стать богом, - «как боги», - то неминуемо постигает
ее новое и новое дробление, новый и новый распад» [5, с. 157]. И далее: «И
разве не видим мы, как на наших глазах, - то под громким предлогом
«дифференциации» и «специализации», то по обнаженному вожделению бесчиния и
безначалия – разве не видим мы, как на наших глазах рассыпается и общество, и
личность, до самых тайников своих, желая жить без бога и устраиваясь помимо
Бога, само-определяться против Бога» [5, с. 157].

Согласно мт. Антонию Сурожскому, грех есть не просто нарушение некоторых
законов, заповедей, а в первую очередь и по существу своему – утрата человеком
связи со своей собственной глубиной. Можно это интерпретировать сугубо
психологически, но не нужно: вряд ли стоит “глубину”, тем более “собственную”,
брать столь поверхностно. А в пределах настоящей глубины различие между своим и
не-своим исчезает.

Можно вслушаться в умные размышления В.В. Бибихина:  «В том, что Гегель
называет интимной собственностью духа, собственность в конечном счете уходит в
такую себя, о которой бессмысленно спрашивать, чья она. Она своя.   В самом
деле, что в личности кроме ограниченности, дурных привычек, скрытости, личин,
из которых часто состоит вся ее индивидуальность, принадлежит ей, а не
человечеству как роду. Утаиваемые слабости, так тревожащие личность, по сути
присущи всем и все их одинаково скрывают. Наоборот, всего реже случается и
по-настоящему уникально то, что составляет суть каждого и чего обычно не
наблюдаешь в полноте, родное и родовое. Не вмещаясь ни в ком отдельно, оно
желанно каждому, кто хочет быть собой, и достижимо только в меру превращения
человека в человека. Стань наконец человеком, говорю я себе то, что говорят
миллиарды, и одновременно совершенно конкретное и неповторимое, не потому что я
особенный человек и взращиваю в себе какую-то небывалую человечность, а как раз
наоборот, потому что самое общее (Гераклит), в котором я спасен и укрыт, и есть
настоящее я (курсив мой. – В.В.)».

Именно в ситуациях само-утверждательства человек теряет связь со своей
собственной глубиной (так можно соединить мысли митр. Антония Сурожского и отца
Павла Флоренского о природе греха), хотя человеку и мнится, что он-то как раз и
утверждает исконно своё, глубинное, собственное, а на деле – только и
исключительно свою поверхность.  Вряд ли стоит «глубину», тем более
«собственную», брать столь поверхностно,  интерпретируя ее сугубо
психологически. Глубина – это наполненность чем-то. Чем же? – Вспомним Пушкина:

«Да ты чем полон, шут нарядный?

 А, понимаю: сам собой;
 Ты полон дряни, милый мой!»

Чем же душе наполняться, быть наполненной и исполненной? – Конечно же, не собой
– тобой.  «Моя душа полна тобой» (Алексей Фатьянов). Исполненность «Тобой» –
лучшее лекарство от само-утверждения и реальный способ ухода от поверхности в
собственную глубину, в глубину собственного, в, собственно, глубину. 

Иными словами, человечность находится решительно по ту сторону самоутверждения, утверждения себя, своего. Воистину, как говорит Белла Ахмадулина:  

«Свести себя на нет,
чтоб вызвать за стеною
не тень мою, а свет,  
не заслоненный мною». 

А если и светить – то благородно и благодарно отражённым светом (светом бытия,
в том числе – вернее, в первую очередь – человечества во мне) и ни в коем
случае не претендовать быть источником света, «светочем разума»…  Вот почему в
максиме: «Пусть будет воля Твоя, не моя» – содержится высшая человечность. И
это – чистая отнология, ибо соответствует истинному способу построения и
осуществления всеединого бытия.

Правда, только ли в творчестве происходит «возвращение абсолюта» и наша встреча
с ним? Интересно послушать размышления В.В. Бибихина. Считается, что приобщение
Абсолюту (бытию), Соборному Разуму дается не всегда, а лишь в творчестве.
Полемизируя по этому вопросу с П.А. Флоренским, В.В. Бибихин  спокойно
отвечает: нет, не так. «Не нужно творчества для того, чтобы Логос давал о себе
знать: он дает о себе знать всегда, везде, на улице, не обязательно в
творчестве, а в искренней человеческой растерянности, и в речи маленьких
детей». Вот так-то. Не только в творчестве. Нет никаких оснований творчество
возводить в абсолют, ведь «никакое творчество не обеспечивает ничего. Среди
творчества – и острее, чем без и вне творчества – страшнее, грознее – встает та
же необеспеченность нам Бытия, та же невозможность нам осуществиться иначе, как
– не в вещах – а в Бытии, как его присутствие, а не присутствие ничто». Что же
обеспечивает? В.В. Бибихин спрашивает: философия? Нет. Религия, богословие?
Нет. Наука? Нет. Ведь «<…> вовсе не творчеством только поэтическим, но и
простой совестливой искренностью всякого слова и даже молчания блюдется
равновесие (обоих начал – Разума соборного и индивидуальной, личной мысли –
В.В.), если оно блюдется <…>». Значит, дело не в «обеспечении», а в
человечности собственно человеческих состояний. 

PS. Точнее о самости как об отказе от эгоизма не скажешь: имеет значение, не кто впервые слово употребил, а кто суть воплотил. Стефания Данилова
 , к нашей ночной кандидатской беседе для коллег из СПбГУ.
