% vim: keymap=russian-jcukenwin
%%beginhead 
 
%%file slova.slovo
%%parent slova
 
%%url 
 
%%author 
%%author_id 
%%author_url 
 
%%tags 
%%title 
 
%%endhead 
\chapter{Слово}
\label{sec:slova.slovo}

Рада после \emph{слов} \enquote{свободовки} о \enquote{героях-бойцах СС} провалила
проект о запрете героизации нацизма,
Эллина Либцис, strana.ua, 03.06.2021

И вначале было \emph{Слово} и \emph{Слово} было у Бога и \emph{Слово} было Бог,
от Иоанна, Евангелие (Благая Весть)

Во-первых, Зеленский совершенно не умеет давать интервью, что странно для
человека, много лет говорящего со сцены, и исходя из этого, должного владеть
\emph{словом}. У него совершенно путанные мысли, отсутствие логических связок и
переходов при формировании ответов, фразы не выверенные, а
импровазиционно-эмоциональные, противоречивые по содержанию и сути. Такой себе
Голобородько-стайл, хотя за два года уже можно было бы освоить совершенно
другой уровень общения со СМИ – президент-стайл,
\textbf{Это странно, но Зеленский совершенно не умеет давать интервью}, Светлана Крюкова, 
strana.ua, 03.06.2021

Друзі! Тижневик \enquote{\emph{Слово Просвіти}} — один із небагатьох нині
існуючих островів незалежного українського ­ \emph{слова} в інформаційному морі
— за певного \enquote{сприяння} нашої влади опинився на межі припинення свого
існування. І тільки ми самі — ті, хто любить, шанує і цінує українське
правдиве і об'єктивне \emph{слово}, можемо допомогти тижневику. Будь-яка
фінансова підтримка видання — це вагомий внесок в українську справу. Тож
подаємо Вам наші банківські реквізити: АТ \enquote{Альфа-Банк}, р/р UA 30
300346 0000026002016994001; ­ отримувач: ГО \enquote{Всеукраїнське товариство
\enquote{Просвіта} ім. Тараса Шевченка}; код ЄДРПОУ/ІНН: 00031756.  Призначення
платежу ­заповнюється платником так, щоб надати повну інформацію про платіж.
На наше прохання про допомогу для газети \enquote{\emph{Слово Просвіти}}
відгукнулися: Кропивницька Мар'яна Андріївна – 800 грн.  Стебловська Леонілла
Петрівна – 100 грн.  Лизанчук Василь Васильович – 500 грн.  Стасик Юлія
Іванівна – 100 грн.  Мельниченко Володимир Юхимович – 1000 грн,
\textbf{Звернення до наших читачів, передплатників, просвітян}, slovoprosvity.org, 17.05.2021

%%%cit
%%%cit_pic
%%%cit_text
У Зеленского были все возможности выполнить обещание - и снизить тарифы.  Но,
видимо, ему скучно держать данное \emph{Слово} ...  Впрочем, Зеленскому давно
привычно жить по принципу одного героя известного советского кинофильма: я -
хозяин своего \emph{Слова}. Я его дал, я его обратно взял. Тем более, что еще
столько можно хорошего наобещать избирателю, что к чему исполнять? Скучно ведь
станет жить. Буднично. Серо
%%%cit_title
\citTitle{У Зеленского были все возможности выполнить обещание - и снизить тарифы}, Валерий Песецкий, strana.ua, 11.06.2021
%%%endcit

%%%cit
%%%cit_pic
%%%cit_text
Есть здесь и откровенная дичь, которую сами составители «Стратегии» в силу
своих особых способностей наверняка даже не заметили. В одном месте они
перечисляют страны, территориальную целостность и суверенитет которых
поддерживает Украина – Азербайджан, Молдавия, Грузия. Сербии в этом списке как
бы «случайно» нет. Ну нет и нет. Как говорит Блогер, \emph{Cлова} не нужны.
Главное – это действия.  И вот вам действия. В другом месте они же пишут –
Украина продолжает брать активное участие в миссиях под руководством НАТО в
Афганистане и... Косово. В Косово, блд. Но вы же дебилы сами на официальном
уровне Косово продолжаете считать территорией Сербии. И одновременно без
согласия Белграда принимаете участие в миссиях НАТО на ее территории в Косово.
И тут же на каждой страничке отражаете раз за разом ру-агрессию и декларируете
приверженность международному праву
%%%cit_title
\citTitle{Украинская власть разродилась стратегией внешнеполитической деятельности}, Игорь Лесев, strana.ua, 11.06.2021 
%%%endcit

%%%cit
%%%cit_pic
%%%cit_text
Никакой свободы \emph{Слова} больше не будет, цензура становится нормой для
информационной политики Свободного Мира, а единое пространство глобального
интернета уже вскоре разделят на огороженные друг от друга сегменты. Новая
Берлинская стена – только распространившаяся практически повсеместно.  А если в
Евросоюзе заблокируют Telegram, его немедленно запретят в свободной
демократической Украине – вдобавок к уже запрещенным VKontakte, Одноклассникам,
Яндексу, Касперскому, итп. Потому что это станет отличным поводом расправиться
с неугодными для власти тг-каналами. Свободный Мир запрещает – и мы тоже.  Не
говорите, что вас не предупреждали
%%%cit_comment
%%%cit_title
\citTitle{Власти Германии грозят заблокировать Telegram}, Андрей Манчук, strana.ua, 13.06.2021
%%%endcit

%%%cit
%%%cit_head
%%%cit_pic
%%%cit_text
\enquote{Як знаєте, мої друзі, я не є приятелем гомінких \emph{Слів}. Я не
знаю, чи і наскільки нам пощастить розвинути діло, за яке ми ось тут
прийнялися. Це залежатиме від праці, зусиль, послідовности і жертовности нас
усіх. І це залежатиме від того, наскільки ми нашу справу зуміємо зробити
зрозумілою всьому українському громадянству ...  Спротиви, які зустрінемо на
нашому шляху, будуть велетенські. Бо ж віднова Соборної Української Держави,
сама собою однозначна з ліквідацією московської імперії, як і польського
історичного імперіялізму, спричинить таку докорінну перебудову цілого Сходу
Европи і великої частини Азії, що це з конечности вплине не менш глибоко й на
політичний вигляд всієї решти світу}. (4, с.  223-224)
%%%cit_comment
%%%cit_title
\citTitle{Евген Коновалець: від життя до смерти у безсмертя (тези за телепрограмою \enquote{Ген українців})}, 
Ірина Фаріон, blogs.pravda.com.ua, 14.06.2021
%%%endcit

%%%cit
%%%cit_head
%%%cit_pic
%%%cit_text
Безусловно, надо все рычаги, а их два основных (структура госбюджета и
структура налогов), максимально задействовать, чтобы повернуть страну с
монопольно-олигархической системы на инновационный путь экономического
развития.  Но кто всё это будет делать? То есть кардинально менять
экономические отношения и повышать уровень профессиональных знаний? Хватит ли
компетенции и политической воли у представителей верхних эшелонов власти?  Или
традиционно дальше \emph{слов} дело не пойдет?
%%%cit_comment
%%%cit_title
\citTitle{Украину нужно повернуть с олигархического на инновационный путь развития}, 
Александр Гончаров, strana.ua, 17.06.2021
%%%endcit

%%%cit
%%%cit_head
%%%cit_pic
\ifcmt
  pic https://avatars.mds.yandex.net/get-zen_doc/3725151/pub_60d0b464b2ba075e653d51f9_60d0b72e5fc3481f3fca0c3b/scale_1200
\fi
%%%cit_text
Вначале было - \emph{Слово}. И \emph{Слово} было у Бога. И \emph{Слово} было Богом.  Пришли черти и
\emph{Слово} стало убого. Программа беса - смутить, свою муть из своего омута выдать
за чистую воду. В тихом омуте... смутьяны водятся.  Писания - это тоже образ.
Это мироздание, построенное на правдивой базе. Само мироздание строй как
хочешь, ведь оно - поверхностное. Чтобы оно не рухнуло, за основу бери правду.
Такое здание простоит долго и люди часть правды и ВСЮ ложь будут воспринимать
как ВСЮ правду. Хитро, не правда ли? Да, Искуситель хитёр и коварен
%%%cit_comment
%%%cit_title
\citTitle{Чтобы мир извратить, нужно просто исказить исКОНную Суть Слова}, 
Ирина, zen.yandex.ru, 22.06.2021
%%%endcit

%%%cit
%%%cit_head
%%%cit_pic
%%%cit_text
\emph{Слово} - сгоВОР - ить всё поясняет да и проясняет, кому не ЯСНО.  Это ещё
цветочки. Ягодки будут впереди, когда мы им лайки подсунем и лайкать всех
заставим, АХ..вот и ГАД ЖЕ ТЫ! Хи - хи - хи. Вот такой он - ОН лай н. Н - наш,
наше. Искусителя сила искусная - в умении да в учении. Они - то НАШЕ \emph{Слово}
учат, в отличие от нас
%%%cit_comment
%%%cit_title
\citTitle{Чтобы мир извратить, нужно просто исказить исКОНную Суть Слова}, 
Ирина, zen.yandex.ru, 22.06.2021
%%%endcit

%%%cit
%%%cit_head
%%%cit_pic
%%%cit_text
Хіба це не історія, подумали німці, вирішивши залишити \emph{слова} жити, аби й надалі
не забувати, хто вони насправді.  Злочини нацистської Німеччини були
унікальними. Залишивши \emph{слова} солдата країни-переможця на будівлі свого
парламенту, Німеччина показує, що вона засвоїла важливі уроки зі свого
минулого, помістивши все в акуратні архіви історичної пам’яті. Така вона й є.
Німці тримають історію перед очима завжди і скрізь, вивчають у школі, говорять
з незнайомцями, досі відчуваючи провину перед євреями, звільняючись таким чином
від гіршого. Вони були такими. Їх такими знав увесь світ. Сьогодні вони
змінилися. Дивися, світе, і вчися
%%%cit_comment
%%%cit_title
\citTitle{Українці не розуміють одне одного не через мову, а через небажання слухати, чути і сприймати}, 
Юлія Мендель, www.pravda.com.ua, 07.07.2021
%%%endcit

%%%cit
%%%cit_head
%%%cit_pic
%%%cit_text
Весьма примечательны \emph{слова}, которые прозвучали в Киево-Печерской Лавре во время
круглого стола «Украина в православно-католическом диалоге», где также
присутствовал и кардинал Кох. Тогда представитель УПЦ МП священник Николай
Данилевич заявил: «Сейчас назрела необходимость решать напрямую проблемы,
которые были или остаются у греко-католиков и православных на Украине. То есть
между нами самими, а не в Москве, которая далеко и, возможно, не всегда в курсе
того, что на самом деле происходит»
%%%cit_comment
%%%cit_title
\citTitle{Украина в крови. Новый крестовый поход Католической церкви. Часть 1 / Статьи}, 
Александр Вознесенский, fraza.com, 08.07.2021
%%%endcit

%%%cit
%%%cit_head
%%%cit_pic
%%%cit_text
А это и есть проявление политико-государственного рукоблудия, щедро сдобренного
патриотической \emph{словесной} шелухой и бесплодной географической настойчивостью.
Пыль в глаза. Видимость вместо реальности. \emph{Слова} вместо дел. Игра вместо
действительности. Сомнительное самоутешение вместо получения настоящего
удовольствия и настоящих поводов реально утешиться. И наконец-то начать
работать себе во благо, а не ради каких-то химер и бесплодных мифов.
Но получается пока по-другому: как и заметил как-то хитромудрый и речистый мэр
Киева Виталий Кличко, «лучше рука в небе, чем синица в журавле»…
%%%cit_comment
%%%cit_title
\citTitle{«Синица в журавле». Вокабулярная война Украины за виртуальные границы}, 
Владимир Скачко, ukraina.ru, 26.07.2021
%%%endcit

%%%cit
%%%cit_head
%%%cit_pic
%%%cit_text
Ще 140 років тому (1881) царський апологет О. Дондуков-Корсаков так описував
найбільшу загрозу для імперії: \enquote{Заміна малоросійською мовою мови російської у
викладанні, хоча б навіть у початкових школах, заміна російських підручників
написаними малоросійською мовою, логічно призводячи у майбутньому до
запровадження малоросійської мови у вищих навчальних закладах і до заміни нею
мови державної у законодавстві, суді й адміністрації, загрожує незліченними
ускладненнями та небезпечними змінами у державному ладі єдиної Росії} (кн.
Українська ідентичність і мовна політика в Російській імперії, 2013).  Тому
допоки \enquote{толстови} базікатимуть у ПАРЄ, українці як націєдержавотворчий Народ
мусять обстоювати на власній землі свою Правду, Волю і Долю. Духовний світ
Української Мови! Бо, як говорив великий серб (не з таких, що підтримують нині
ординську Росію) Стефан І Неман, засновник династії володарів Сербії (ХІІ ст.):
\enquote{Народ, який губить свої \emph{слова}, перестане бути народом}. Наша Правда міститься
в животворному Рідному \emph{Слові}!
%%%cit_comment
%%%cit_title
\citTitle{Правда в Рідному Слові}, Георгій Філіпчук, slovoprosvity.org, 12.07.2021
%%%endcit

%%%cit
%%%cit_head
%%%cit_pic
%%%cit_text
"Нож в спину - это всегда больнее, чем в грудь".  Многих разочаровал поступок
спортсменки. Ведь ранее они ее поддерживали и защищали от травли.  "Смотрите
какая адекватная девушка, наша гордость, а разжигатели идут у дупу" - написал
утром ведущий Макс Назаров, комментируя видео с ответом Магучих по прилету в
Киев. "Беру свои утренние \emph{слова} назад", - написал он же вечером после
последнего заявления спортсменки
%%%cit_comment
%%%cit_title
\citTitle{\enquote{Многие вас поддерживали и теперь разочарованы}. Что пишут в сети о покаянном заявлении спортсменки Магучих}, 
Оксана Малахова, strana.ua, 12.08.2021
%%%endcit

%%%cit
%%%cit_head
%%%cit_pic
\ifcmt
  pic https://img.strana.news/img/article/3592/dtp-na-haharina-32_main.jpeg
  @width 0.4
	@caption Авария произошла накануне и унесла жизнь двоих человек. Фото: соцсети 
\fi
%%%cit_text
По ее \emph{словам}, она поговорила с братом до того, как его госпитализировали
в больницу. Он рассказал, что накануне проводил время с другом и своей
девушкой.  К ним подъехала ее подруга Лиза вместе с Николаем на Infiniti.
Последний начал хвастаться авто и сказал, что права и машину ему подарили
родители. После этого компания согласилась прокатиться на Infiniti.  По
\emph{словам} сестры, Михаил снимал на телефон все разговоры и происходящее в
авто, однако после ДТП его iPhone украли с места происшествия
%%%cit_comment
%%%cit_title
\citTitle{"Пошли все на х@й, меня отмажут". Очевидцы рассказали, как себя вел мажор на "Инфинити" после смертельного ДТП}, 
Карина Вольтер, strana.news, 27.10.2021
%%%endcit

%%%cit
%%%cit_head
%%%cit_pic
%%%cit_text
\emph{Словарный} запас грантоедских зелебобиков и пляски на памяти о Великой войне
Контора под названием Зе-СНБО предала гласности краткий \emph{словарный} запас
грантоедов и зелебобиков. Над этим можно было бы поиздеваться, как Ильф и
Петров поиздевались над \emph{словарным} запасом Эллочки-людоедки и ее товарки Фимы
СОбак (ударение на первый слог!) в «Двенадцати стульях». Но в этой «азбуке
идиота» от СНБО и некоего «центра противодействия дезинформации» значится, что
определение «Великая Отечественная война» — оказывается- теперь объявляется...
«фейком». В переводе с нынешнего идиотического новояза, \emph{слово} «фейк» можно
перевести как «чушь», «бред сумасшедшего» или «бред собачий», «брехня» и так
далее
%%%cit_comment
%%%cit_title
\citTitle{Краткий словарь грантоедов под редакцией СНБО / Лента соцсетей / Страна}, 
Александр Карпец, strana.news, 26.10.2021
%%%endcit

%%%cit
%%%cit_head
%%%cit_pic

\ifcmt
  tab_begin cols=2
     pic http://archivsf.narod.ru/1969/vsp/65.jpg
     pic https://fb.ru/media/i/6/3/7/3/4/6/i/637346.jpg
  tab_end
\fi
%%%cit_text
Минув час, і стіну (тепер я вже не пригадую навіть яку) обмазали чи обшкрябали,
і напис зник. саме так ось уже протягом двох століть поводяться з чудовими
храмами середньовіччя. їх нівечать всіляко — зсередини і зовні.  священик їх
перефарбовує, архітектор обшкрябує, а згодом приходить юрба, яка їх руйнує.
Отож нічого вже не лишилося від таємничого \emph{слова}, викарбуваного на стіні
похмурої вежі собору паризької богоматері, нічого не лишилось і від невідомої
долі, про яку так сумовито розповідало це \emph{слово}, — нічого, крім
нетривкого спогаду. людина, яка начертала це \emph{слово} на стіні, багато
століть тому зникла, так само зникло із стіни храму \emph{слово}, та й сам
храм, може, незабаром зникне з лиця землі. саме це \emph{слово} й спричинилося
до написання цієї книги
%%%cit_comment
%%%cit_title
\citTitle{Собор Паризької Богоматері}, Віктор Гюго
%%%endcit

%%%cit
%%%cit_head
%%%cit_pic
%%%cit_text
Евангелие от Иоанна открывается строкой: «Вначале было \emph{Слово}». Это
канонический перевод. В оригинале, правда, используется термин «Логос», который
можно перевести 34 различными способами. Среди возможных вариантов перевода:
«разум» и «учение». Согласитесь, что строка «И \emph{Слово} было Бог» будет
звучать совсем иначе в вариантах: «И Разум было Бог» или «И Учение было Бог».
Тем не менее, я думаю, что не случайно практически никто не берётся оспаривать
верность канонического перевода данного текста с древнегреческого. Обратившись
как к нашей истории, так и к повседневной практике, мы обнаружим, что
утверждение «Вначале было \emph{Слово}, и \emph{Слово} было у Бога, и
\emph{Слово} было Бог», — в полной мере описывает реальную действительность со
свойственными древним краткостью и афористичностью
%%%cit_comment
%%%cit_title
\citTitle{О Слове, Боге и разделённом человечестве и Крошке Еноте}, 
Ростислав Ищенко, ukraina.ru, 30.10.2021
%%%endcit

%%%cit
%%%cit_head
%%%cit_pic
%%%cit_text
Поколение 20+, 25+ выглядит неспособными к осознанию тех понятий с которыми
сталкивается. Чем дальше, тем больше это приобретает характер эпидемии и даже
катастрофы. Я по роду деятельности заметил, что 20+ читая тексты не понимают
значения \emph{слов}. Например, делают тебе расшифровку беседы и в ней легко меняют
значение на прямо противоположное. Штук 5-10 таких изменений в каждой беседе
может быть. Например, в фразе ""Я далек от мысли, что кто-то построит объект,
который не будет безопасным. Построят. Заставят принять, по бумаге он будет
безопасным". В тексте вместо "не будет безопасным" читаю "будет безопасным". То
есть легким движением руки смысл становиться абсурдным, но сознание это не
фиксирует. От \emph{слова} вообще. Ну нет разницы, не отбивается она в матрице. Потому
что на ЗНО заставляют заучивать, как баранов, псалмы непонятно чего. Потому это
не проблема какого-то конкретного человека, а именно поколения. У 30+ это
встречается, но редко. У 20+ это катастрофа
%%%cit_comment
%%%cit_title
\citTitle{Система образования Украины напрочь убивает мышление / Лента соцсетей / Страна}, 
Юрий Романенко, strana.news, 31.10.2021
%%%endcit

%%%cit
%%%cit_head
%%%cit_pic
\ifcmt
  pic https://avatars.mds.yandex.net/get-zen_doc/3141623/pub_612a62b1538b705fd6669995_612a64bb7c289b323891279a/scale_1200
  @width 0.4
\fi
%%%cit_text
Как выглядит калькирование? Вы берёте иностранное \emph{слово} и не просто
заимствуете его, а переводите каждую его составную часть. Ну вот например
французское influence. Делим его: in заменим на «в», а fluent – на лить.
Получилось «влияние». Вуаля, заимствованное \emph{слово}, которое встало как
родное. Таким способом Карамзин создал множество \emph{слов} и выражений:
«расположение», «быть не в своей тарелке» и так далее.  Есть ещё другой тип
калек, семантический. Это когда мы заимствуем не отдельные части \emph{слова},
а только кусочек его смысла. Например, у русского \emph{слова} «трогать» было
буквальное значение 'прикасаться', но отсутствовало переносное – 'касаться
струн души, вызывать чувство умиления или жалости'.  Карамзин просто взял этот
дополнительный смысл с французского, применил его к исконно русскому
\emph{слову} – и заодно на этой волне образовал \emph{слово} «трогательный».
Так русский язык обогатился множеством новых элементов, которые в отличие от
\emph{слов}, образованных ревнителями чистоты русского языка с их
«мокроступами», встроились в систему легко и непринуждённо – в духе светских
салонов
%%%cit_comment
%%%cit_title
\citTitle{Как Карамзин навсегда изменил русский язык?}, Катехизис и Катарсис,
zen.yandex.ru, 29.08.2021  
%%%endcit

%%%cit
%%%cit_head
%%%cit_pic
%%%cit_text
Перш за все щодо вашого повернення до мого будинку після цілком заслуженого
вигнання, — він випнув наперед бороду й задерикувато глянув на мене, немов
запрошуючи заперечити йому. — А повернулися ви сюди після вашого, як я вже
сказав, цілком заслуженого вигнання через відповідь, яку ви дали цьому
причепі-полісменові. Мені здалося, в ній був натяк на певну порядність, якої я
не звик бачити у представників вашого фаху. Визнавши себе за винного, ви
виявили деяку розумову незалежність і широчінь поглядів, що й притягло до себе
мою прихильну увагу. Різновид породи людської, до якої ви маєте нещастя
належати, завжди був поза моїм розумовим обрієм. А ваші \emph{слова}
несподівано піднесли вас вище, до рівня моєї уваги. Ви зацікавили мене. Ось із
цих причин я й запросив вас до себе, поклавши познайомитися з вами ближче.
Попіл я попрошу струшувати в японську попільничку, яка стоїть на бамбуковому
столику ліворуч од вас.  Цю коротеньку промову він виголосив, відрубуючи кожне
\emph{слово}, мов лектор перед аудиторією. Обернувшись обличчям просто до мене,
він сидів на своєму обертовому кріслі, відхиливши голову назад,
настовбурчившись, як велика жаба, і примруживши трохи зневажливо повіки. Потім
раптом повернувся до мене боком, через що мені було видно лише його червоне
вухо та скуйовджене волосся. Він став длубатись у стосах паперу на столі,
видима річ, шукаючи чогось. Нарешті я побачив у нього в руках щось схоже на
пошарпаний старий альбом для малюнків
%%%cit_comment
%%%cit_title
\citTitle{Утрачений світ}, Артур Конан Дойл
%%%endcit
