% vim: keymap=russian-jcukenwin
%%beginhead 
 
%%file 02_03_2023.fb.krutenko_maryna.mariupol.1.sedmoi_den_voini__02.cmt
%%parent 02_03_2023.fb.krutenko_maryna.mariupol.1.sedmoi_den_voini__02
 
%%url 
 
%%author_id 
%%date 
 
%%tags 
%%title 
 
%%endhead 

\qqSecCmt

\iusr{Inna Drozdenko}

Початок апокаліпсису.

\begin{itemize} % {
\iusr{Maryna Krutenko}
\textbf{Инна Дрозденко} в мене досить пмнаннічо.... як в Мариуполі хоча комусь вдалось вижити?! І ще одне, як той народ що таке пережив, наразі співає алилуйя рузьким солдафонам????

\iusr{Inna Drozdenko}
\textbf{Maryna Krutenko}, в мене нема логічних пояснень цим речам.
\end{itemize} % }

\iusr{Irina Matushina}

Я помню, что в этот день Алина - дочь Ольги выложила историю в инстаграм и ФБ и
моя Катя первая кто за кадром услышала твой голос. Это как раз, когда ты
спрашивала прохожую про воду в «Терраспорт», мы скачали это видео слушали твой
голос и все плакали.

\iusr{Оксана Ррр}

Читаю и проживаю все вновь...

\iusr{Lyudmila Kretova}

\ifcmt
  igc https://i.paste.pics/07f972d7470ca2b8e985f52db37120e8.png
	@width 0.1
\fi

\iusr{Lyudmila Kretova}

Спасибо, Марина, что ты находишь силы, чтобы описать наше состояние. Мы 55 дней
провели в аду, но я как и ты, не понимаю, как можно сейчас радоваться приходу
тварей нашим жителям? Мерзко, до тошноты

\begin{itemize} % {
\iusr{Maryna Krutenko}
\textbf{Людмила Кретова} как вы выжили эти 55 дней? Ночью с 14 на 15 марта, обстреляли каждый квадратный метр. Я всю ночь слышала, как ближе и ближе ложились снаряды и .... Вуаля, снаряд упал ровно под наше окно....

\iusr{Lyudmila Kretova}

Меня и мужа смогли вывезти только 18 апреля, поэтому 55 дней

\iusr{Maryna Krutenko}
\textbf{Людмила Кретова} ужассссс.

\end{itemize} % }

\iusr{Irina Matushina}

Марина, не забываю писать кто уничтожал мариупольцев..... Одноклассница моих
младших дочерей рассказывала, что когда они выезжали из Мариуполя на блок посте
один русских солдат придрался, что она очень грустная. Он ей сказал: «
Запомни, вас обстреливали Украина, а мы вас пришли защитить». Девочка очень хотела
плюнуть ему в лицо, но испугалась потому что он был с автоматом. Но она точно это
запомнить на всю жизнь, как её детство было разрушено орками в 14 лет. И кстати
у неё день рождения 24 февраля, теперь она решила его отмечать в другой день.

\iusr{Leonid Omelchenko}

Ужас!

\iusr{Настюша Мироненко}

Читаю і знову переживаю все це....... Біль жах.........

\iusr{Настюша Мироненко}

Моя знайома мені стверджує, що росія захищала, хоча вона все бачила, жила біля
мене, но що в її голові блін......

Так як звільнять Маріуполь прибіжить як нічого і не було.....

\begin{itemize} % {
\iusr{Maryna Krutenko}
\textbf{Настюша Мироненко} 

у меня не хватает нормативной лексики..... от чего ее спасли? Он нормальной
жизни? Почему эта не совсем здоровая на голову дама, которая к стати чудом
выжила. Не собрала свои пожитки и не уехала в рашу если она так плохо жила и ее
обижали????? Умом Россию не понять, это точно!!!!

\iusr{Настюша Мироненко}
\textbf{Maryna Krutenko} она уехала, но ждёт возможности вернуться, а в Украине чутли не за воздух платила, все не так, Азов плохой и тд..... И только русняки приехали на автобусе и вывезли их в рашку

\iusr{Настюша Мироненко}

Я когда услышала ее разговор, хотела водки выпить, впервые в жизни, она сказала, что там хорошо....

\iusr{Настюша Мироненко}

У меня есть своя четкая позиция, Я Украина!

До дупы таких, хто перейшов на бік темряви.

\iusr{Настюша Мироненко}
\textbf{Maryna Krutenko} ну ви Крутелики, завжди допомогаете, молодці.
\end{itemize} % }

\iusr{Ольга Шишигина}

Маринка, ты - молодец, что решила описать каждый день нашей оккупации в
Мариуполе. Мне больно возвращаться в эти воспоминания. Но я хочу добавить, что
Бог нас всех хранил и выводил. У нас у всех много свидетельств о том, как Он
Всемогущ. Его покров был над всеми нами. Он все сделал как обещал!!! Ему только
Слава!!!!

\iusr{Андрей Дудин}

Откройте, для того чтобы поделится.

\begin{itemize} % {
\iusr{Maryna Krutenko}
\textbf{Андрей Дудин} сейчас разберусь как это сделать. Не ты один попросил....

\iusr{Maryna Krutenko}
\textbf{Андрей Дудин} ничего не получается...., у меня закрытый профиль, может мне нужно его открыть, чтоб ты смог поделится?

\iusr{Ольга Шишигина}

Я тоже хотела поделиться, но потом просто дала ссылку на твой аккаунт тем, кому
интересно. Он открыт, кстати.

\iusr{Андрей Дудин}
\textbf{Maryna Krutenko} я честно сказать не знаю, как это делается.

\iusr{Анастасия Верещагина}
\textbf{Maryna Krutenko} я знаю как это сделать

\iusr{Анастасия Верещагина}
\textbf{Maryna Krutenko} в настройках страницы

\ifcmt
  igc https://scontent-fra3-1.xx.fbcdn.net/v/t39.30808-6/334284086_743926940597031_1213159193853943324_n.jpg?stp=cp6_dst-jpg&_nc_cat=105&ccb=1-7&_nc_sid=dbeb18&_nc_ohc=GZ3-o5gYnzcAX-tq_vM&_nc_ht=scontent-fra3-1.xx&oh=00_AfDfiCEXYZve2uC51U5z2lBSx84a_9THRb5RNhoMRHOqSQ&oe=640951CA
	@width 0.4
\fi

\iusr{Анастасия Верещагина}
\textbf{Maryna Krutenko} называется Настройки профиля и меток

\iusr{Анастасия Верещагина}
\textbf{Maryna Krutenko} если и так не разрешит, хотя должно! Тогда нужно будет открыть страницу

\iusr{Maryna Krutenko}

У меня включено, наверное нужно открыть аккаунт

\iusr{Анастасия Верещагина}
\textbf{Maryna Krutenko} думаю да, попробуй

\iusr{Maryna Krutenko}

Открыла профиль, теперь можно поделится?

\iusr{Андрей Дудин}
\textbf{Maryna Krutenko} нет. Пока нет, кнопки поделится.

\iusr{Анастасия Верещагина}
\textbf{Андрей Дудин} уже есть)

\iusr{Анастасия Верещагина}
\textbf{Maryna Krutenko} да

\end{itemize} % }

\iusr{Татьяна Демченко}

То что пережила ты, тоже самое пережила и я. Но мне совсем не хочется этого
вспоминать. Очень тяжёлыйе воспоминания. Всё как в фильме ужасов.

\begin{itemize} % {
\iusr{Maryna Krutenko}
\textbf{Татьяна Демченко} мозги хотят забыть тот ужас, но я хочу записать и оставить на память, а потом можно и забыть.

\iusr{Татьяна Демченко}

Жизнь разделились на до и после.

\iusr{Maryna Krutenko}
\textbf{Татьяна Демченко} это точно
\end{itemize} % }

\iusr{Ruslan Paul}
\textbf{Maryna Krutenko} 

Марина, как ты помнишь по дням подробно события? Ты дневник писала?? У меня
тоже множество воспоминаний, но точно сказать в какой день что было, я не
скажу.

\begin{itemize} % {
\iusr{Maryna Krutenko}
\textbf{Руслан Пауль} я как в фильме «я всегда знаю где и с кем я была»))))) помню и все))) Дневник не вела.
\end{itemize} % }

\iusr{Валентина Самарина}

Я не вела дневник но помню четко до подробностей некоторые моменты

24.02 первый день как собралась на работуи услышала взрывы потом на работе
сидели с коллегами и не понимали ничего все были в шоке..

1.03 задражал дом и прилетел снаряд во двор напротив горели машины и школа с
садиком без окон

13.03 в окно увидели асвабадителей которые шли колонной и направлялись к нашему
дому

16.03 готовили еду на площадке т.к на улицу не пускали крик соседа бежим вниз
потом свист темнота очнулась тряс сосед и что то кричал ни чего не слышала и не
видела потом отползли все тряслось очнулась лицо в крови руки то же но уже чуть
слышала и видела сильно болела рука

Позже оказалось контузия и порваны мышцы плеча слух вернулся на 60\% рука в
норму так и не пришла.. часто все это снится правду говорят мы не живые мы
выжившие... страшно и больно

\iusr{Irina Matushina}

Жаль, что нет функции поделиться...

\begin{itemize} % {
\iusr{Maryna Krutenko}
\textbf{Ирина Матюшина} еду к тебе. Завтра разберёмся как это сделать. Подключим молодёжь 😉
\end{itemize} % }

\iusr{Natalia Azovskaya}

Спасибо, Марин, что делишься. Я до какого-то момента фиксировала события, потом
уже не до того было. Фото несколько сделала один раз. Мой напарник, когда мы
пытались добежать с Приморского до 17 мкр (не добежали, на Ильюше жестяк был) ещё
и фото делал. Интересные кадры, когда летят мины 😫 Пробегали, помню, в твоём
районе. Тогда ещё даже девятиэтажки на площади Свободы стояли не обгоревшие. Но
перекрёсток Ленина и Строителей стремно было пересекать, бежала, как заяц. И на
протяжении всего пути, как минимум, миномёты бахали весь путь. То и дело в
подвалы домов падали.

\iusr{Валентина Самарина}

Мне высали интервью моей соседки для росийского тв из Марика. показала
Украинцам в Польше одна сказала виключи меня тошнит
