% vim: keymap=russian-jcukenwin
%%beginhead 
 
%%file 31_12_2021.tg.lesev_igor.1.moja_doroga
%%parent 31_12_2021
 
%%url https://t.me/Lesev_Igor/257
 
%%author_id lesev_igor
%%date 
 
%%tags 
%%title По крупинкам, по сахаринкам, но я пришел к убеждению, что моя дорога из желтого кирпича не напрасна
 
%%endhead 
\subsection{По крупинкам, по сахаринкам, но я пришел к убеждению, что моя дорога из желтого кирпича не напрасна}
\label{sec:31_12_2021.tg.lesev_igor.1.moja_doroga}

\Purl{https://t.me/Lesev_Igor/257}
\ifcmt
 author_begin
   author_id lesev_igor
 author_end
\fi

Ну что? Надо-таки что-то черкнуть. Размазать сопли по столу в
празднично-икорном настроении. К тому же, как единственно ведущий на своем же
тг-канале, кто как не я?

Начну так. Этот год у меня вышел чертовски суровым. Переломным. Я даже не
спотыкался, а падал. Я падал в своих сомнениях. Лежал лицом в грязи. Образно,
конечно. Уж лучше натурально в грязи, чем чувствовать себя в грязи душевно. Да
вы и сами это знаете.

Мне казалось, что я не там и занимаюсь не тем. Что я не реализован. Не слышим.
Незначим. Неудачлив во всех своих начинаниях. Знаете, друзья. Я просто
собирался свалить из страны. Не в Париж на 4 дня, а вот так с концами. Даже в
Магадан, как тот дружбан Высоцкого.

И все же я прямо сейчас лежу на кухонном диванчике на Термках и именно отсюда
строчу эти сопли. Ко мне пришло... нет, не видение, а уверенность. Я теперь точно
знаю, что обязан совершить на ЭТОЙ земле – земле, где я родился – миссию.
Достучаться до других сердец. И я вижу, что у меня это уже получается. По
крупинкам, по сахаринкам, но я пришел к убеждению, что моя дорога из желтого
кирпича не напрасна. Вот это был главный итог моего года.

Да, в этом месяце в Варшаве на пересадке я встретил интересного чудика из
Киева. Я его спросил, где тут можно пожрать, а по возвращению купил ему пива.
Потом он купил мне пива. И мы бухали всю ночь, пока ждали борт на Киев. Пацан
майдановец, за-порошенковец, абсолютно русскоязычный, но поставил себе цель
«перейти на родной украинский язык». Медведев (который он нам не Димон) написал
о таких «вывернутый наизнанку человек».

Но знаете, мне этот парень чертовски понравился. И я ему. Нет, даже не так. Он
в меня просто влюбился. (Если что, у него трое дочек в Киеве). Он вызвался нас
с женой подвести от Жулян, приглашал на лыжи, и в целом, я ему зашел в душу.

Но вот какое дело, друзья. Мы живем с вами в говяной стране и в еще более
говяном сообществе. Помните ту финальную фразу из «Касабланки»? «Думаю это
начало большой дружбы». Будь я с этим парнем в Варшаве, Москве или Париже – мы
бы, несомненно, дружили. Дружба – это та же любовь, только без секса. Это
эмоциональное состояние.

А Украина – это страна ненависти. Именно поэтому я хотел отсюда свалить в пока
еще этом году. Но нет, не свалил. И мне кажется… по крайней мере, я хочу в это
верить, что я смогу поменять что-то кардинально на этой земле. Не говорю
специально «на Украине». Говорю – на СВОЕЙ земле.

Вот ради этого я здесь. И стучусь в двери. Звеню в колокольчики. Бью в
барабаны. Глашатаю. И наяриваю на клавиши. С наступающим Новым годом, друзья!

\ii{31_12_2021.tg.lesev_igor.1.moja_doroga.cmt}
