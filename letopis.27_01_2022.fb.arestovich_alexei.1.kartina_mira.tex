% vim: keymap=russian-jcukenwin
%%beginhead 
 
%%file 27_01_2022.fb.arestovich_alexei.1.kartina_mira
%%parent 27_01_2022
 
%%url https://www.facebook.com/alexey.arestovich/posts/5153371624726876
 
%%author_id arestovich_alexei
%%date 
 
%%tags chelovek,psihologia
%%title Картина мира и базовые убеждения чрезвычайно важны
 
%%endhead 
 
\subsection{Картина мира и базовые убеждения чрезвычайно важны}
\label{sec:27_01_2022.fb.arestovich_alexei.1.kartina_mira}
 
\Purl{https://www.facebook.com/alexey.arestovich/posts/5153371624726876}
\ifcmt
 author_begin
   author_id arestovich_alexei
 author_end
\fi

- Картина мира и базовые убеждения чрезвычайно важны.

Они определяют то, что мы считаем реально существующим в мире: фигуры, объекты,
силы, законы, связи, логику и эмоции.

Возможности, которые мы упускаем - это лакуны в картине мира.

\ii{27_01_2022.fb.arestovich_alexei.1.kartina_mira.pic.1}

То, на что мы пытаемся опереться - базисные силы нашей картины мира, а
существуют ли они при этом на самом деле - это бааальшой вопрос.

Картина мира расположена слоями.

С каждого предыдущего слоя последующий видится есть не прямым обманом, то, как
минимум, сильно искажённым пазлом, в котором не хватает многих элементов, а те,
что есть - расположенными неправильными.

Через два слоя погружения, верхние выглядят совершенным заблуждением.

Параллельно с проникновением в глубину своей картины мира у человека резко
возрастает его способность к пониманию и управлению собственной психикой: «не
названо - не операбельно».

Выход же на пре-онтологию своей картины мира развязывает такие силы внутри
человека, о которых большинство из нас даже не подозревает.

Не следует думать, будто человек так уж свободен, как это подаётся в
современной либеральной культуре.)

Мышление - это всегда мышление об основаниях. 

Только выход на основания своего мышления покажет Вам - насколько человек
определён (почти полностью) картиной мира, сформированной его образованием,
воспитанием, мышлением.

Этот выход и предлагает нам Павел Щелин в своём модуле «Археология
консерватизма», который начнётся завтра, в 19.00.

Павел покажет нам глубинные установки господствующей у современной западной
цивилизации картины мира, вместе со всеми ее достоинствами и недостатками, и
самое главное - возможную альтернативу, которая лежит в духовных глобинах той
же самой цивилизации, но была ею незаслуженно забыта.

Я иду:

\url{https://apeiron.school/conservatism}
