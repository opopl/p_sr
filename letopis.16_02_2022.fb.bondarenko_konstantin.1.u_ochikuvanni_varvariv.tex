% vim: keymap=russian-jcukenwin
%%beginhead 
 
%%file 16_02_2022.fb.bondarenko_konstantin.1.u_ochikuvanni_varvariv
%%parent 16_02_2022
 
%%url https://www.facebook.com/bonkost/posts/5407989982569006
 
%%author_id bondarenko_konstantin
%%date 
 
%%tags napadenie,poezia,rossia,ugroza,ukraina,varvary
%%title У очікуванні варварів
 
%%endhead 
 
\subsection{У очікуванні варварів}
\label{sec:16_02_2022.fb.bondarenko_konstantin.1.u_ochikuvanni_varvariv}
 
\Purl{https://www.facebook.com/bonkost/posts/5407989982569006}
\ifcmt
 author_begin
   author_id bondarenko_konstantin
 author_end
\fi

Багатьом відомий вірш класика грецької літератури Константіноса Кавафіса «У
очікуванні варварів». Написаний у 1904 році, цей вірш чудово передає настрої
правлячої верхівки України 16 лютого 2022 року, після того, як Путін так і не
напав).

Український переклад вірша здійснив Григорій Кочур:

- Чого чекаємо, чого на площі з'юрмились? - Бо прийдуть варвари до нас
сьогодні.

\obeycr
- Чого ж тоді сенат принишк, не порядкує,
Сенатори сидять, не видають законів? -
Таж прийдуть варвари до нас сьогодні,
Та нащо нам сенаторські закони?
Ось прийдуть варвари, вони й закони видадуть.
- Чого ж то імператор встав удосвіта,
Чого при вході в місто перед брамою
В короні, пишно вбраний, сів на троні він? -
Таж прийдуть варвари до нас сьогодні,
То імператор там зустріти хоче їх ватажка. 
Вже навіть і пергамен є,
Де писано усі почесні титули,
Що ними того ватажка вшанує він.
- Чого ж це претори та ще й два консули
В гаптованих червоних тогах вийшли?
Чому браслети начепили з аметистами,
Перснями зі смарагдами виблискують?
Чому в руках у них коштовні посохи,
Різьблені, сріблом-золотом цяцьковані? -
Таж прийдуть варвари до нас сьогодні,
А ця пишнота варварів засліплює.
- А чом шановних риторів нема ніде,
Чому промов їх звичних ми не чуємо? -
Таж прийдуть варвари до нас сьогодні,
А варвари, либонь, промов не полюбляють.
- Чого ж це порожніють площі й вулиці.
Так, ніби щось незрозуміле скоїлось -
Збентеження якесь чи то розгубленість, -
В задумі по домівках всі розходяться? -
Бо смерклося, а варварів не видко,
А тут іще з кордону поприходили
Та й кажуть, що нема ніяких варварів.
- То як же нам - що діяти без варварів?
Це ж хоч який там, але був би вихід.
\restorecr
