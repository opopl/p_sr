% vim: keymap=russian-jcukenwin
%%beginhead 
 
%%file slova.kontekst
%%parent slova
 
%%url 
 
%%author 
%%author_id 
%%author_url 
 
%%tags 
%%title 
 
%%endhead 
\chapter{Контекст}

%%%cit
%%%cit_head
%%%cit_pic
%%%cit_text
Я также не посещаю театральные постановки на украинском языке. Из разного рода
культурных событий, предпочитаю концерты инструментальной музыки и
художественные выставки. На них ничто не напоминает о параллельном культурном
\emph{контексте}. Избегаю я его не сознательно, а инстинктивно. Думаю, большинство
моих читателей делают так же.  Пожалуй, с параллельным \enquote{миром} я сталкиваюсь
лишь во время визитов на телеканал \enquote{Прямой}. И это выглядит очень забавно. У
них там действительно какая-то своя, не связанная с нашей, реальность. Другие
проблемы, другой язык, другая культура и ценности
%%%cit_comment
%%%cit_title
\citTitle{Разные группы украинцев живут в разных информационных и культурных реальностях / Лента соцсетей / Страна}, 
Даниил Богатырев, strana.ua, 05.07.2021
%%%endcit
