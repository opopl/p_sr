% vim: keymap=russian-jcukenwin
%%beginhead 
 
%%file 17_02_2022.stz.news.ua.fraza.1.kak_zapadnyje_partnery_opuskajut_ukrainu.5.nas_naglo_ispolzujut
%%parent 17_02_2022.stz.news.ua.fraza.1.kak_zapadnyje_partnery_opuskajut_ukrainu
 
%%url 
 
%%author_id 
%%date 
 
%%tags 
%%title 
 
%%endhead 

\subsubsection{Нас нагло используют!}
\label{sec:17_02_2022.stz.news.ua.fraza.1.kak_zapadnyje_partnery_opuskajut_ukrainu.5.nas_naglo_ispolzujut}

Нападать Путин точно не будет, но даже пара-тройка ракетно-бомбовых ударов по
Украине тоже будет на руку Западу, поскольку обесценят ее активы сильнее, чем
многолетний кризис. И это позволит западным структурам взять интересующие их в
Украине объекты с минимальными затратами, не говоря о решении своих внутренних
и глобальных проблем вроде вытестения Москвы с европейского энергорынка
санкциями и так далее...

Кстати, показательный факт. В январе 2022 года впервые в истории экспорт
сжиженного природного газа из Соединенных Штатов в Европу превысил поставки по
трубопроводу из России. Вот так легко Запад за счёт Украины зарабатывает
миллиарды, а наша экономика терпит убытки. Кроме обещаний мы ничего не получили
от США, а выделяемые кредиты вынуждены отдавать с большими процентами. Выходит,
для Запада Украина просто игрушка, на которой зарабатывают американские
компании, а украинские граждане вынужден платить за коммуналку в 5 раз дороже
по причине того, что «рыночек», видите ли, цены на газ поднял. Типа сам по себе
— вот так взял и поднял...

Таким образом, пока украинская экономика терпит убытки, а рост инфляции и цен
на энергоресурсы могут стать основной для социального взрыва на Западе,
транснациональные корпорации активно богатеют.

Еще раз повторим, эта война или истерия вокруг возможной войны нужна, прежде
всего и исключительно англо-американцам по очень многим причинам. А Украину
Запад и Москва непосредственно или, так сказать, дистанционно, «на удаленке»
используют в своих империалистических разборках. Как предупреждал 100 лет назад
товарищ Ленин в своих бессмертных о государственно-монополистическом
капитализме в стадии империализма и империалистических войн за рынки сбыта и
сферы влияния. Кстати, читайте бессмертную классику, ибо там уже все
написано...
