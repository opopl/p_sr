% vim: keymap=russian-jcukenwin
%%beginhead 
 
%%file 13_10_2021.fb.bobrov_gleb.1.savenkova_mirotvorec
%%parent 13_10_2021
 
%%url https://www.facebook.com/glebbobrov0665926209/posts/4431929736920446
 
%%author_id bobrov_gleb
%%date 
 
%%tags donbass,mirotvorec_sajt,savenkova_faina,ukraina,vojna
%%title Очередной "испанский" стыд от небратьев
 
%%endhead 
 
\subsection{Очередной \enquote{испанский} стыд от небратьев}
\label{sec:13_10_2021.fb.bobrov_gleb.1.savenkova_mirotvorec}
 
\Purl{https://www.facebook.com/glebbobrov0665926209/posts/4431929736920446}
\ifcmt
 author_begin
   author_id bobrov_gleb
 author_end
\fi

Очередной "испанский" стыд от небратьев. На главной украинской информационной
помойке "Миротворец" размещены данные 12-летнего луганского драматурга и
прозаика Фаина Савенкова. Киев в очередной раз наглядно продемонстрировал с кем
на самом деле сражаются украинские силовики в Донбассе - с женщинами и детьми.
Ссылка на материал в первом комменте...

====================

Скандально известный украинский сайт "Миротворец" публикацией личных данных
12-летней писательницы из Луганска Фаины Савенковой проявил запредельную
низость. Об этом ЛИЦ заявил глава комиссии по вопросам развития культуры,
образования и науки Общественной палаты ЛНР, председатель правления Союза
писателей (СП) ЛНР Глеб Бобров.

\ifcmt
  ig https://scontent-frt3-1.xx.fbcdn.net/v/t39.30808-6/245196051_4431927806920639_3077370488699669578_n.jpg?_nc_cat=102&ccb=1-5&_nc_sid=730e14&_nc_ohc=UiaqJPZMBWAAX_uTAg-&_nc_ht=scontent-frt3-1.xx&oh=6c77bde729ad49555d8b9bb69bd498db&oe=616C098D
  @width 0.4
  %@wrap \parpic[r]
  @wrap \InsertBoxR{0}
\fi

"Миротворец" разместил в своей базе личные данные, фотографии Савенковой и
скриншоты ее публикаций. Девочку обвиняют в том, что она "участвует в
антиукраинских пропагандистских мероприятиях", а также "называет себя
прозаиком, драматургом и ФАНТАСТОМ!" "Миротворец" просит правоохранительные
органы Украины "рассматривать данную публикацию на сайте как заявление о
совершении этим гражданином осознанных деяний против национальной безопасности
Украины, мира, безопасности человечества и международного правопорядка, а также
иных правонарушений".

"Очевидно, что размещение личных данных 12-летнего ребенка на сайте
"Миротворец" – однозначный акт устрашения и запугивания. Дескать, мы вас всех
перепишем и, рано или поздно, накажем. Такое послание и есть главный
экзистенциальный посыл этого позорного украинского проекта. Запугивать ребенка
за его взгляды и убеждения – запредельная низость. Хотя ниже падать Украине в
глазах Донбасса, казалось бы, и так уже некуда", - сказал Бобров.

Он отметил, что "травля девочки вызвана не "фейковой информацией и собственными
домыслами", как заявлено на сайте, а тем, что ребенка слышат во всем мире".

"Эссе Фаины о нынешней войне переведены на пяток-другой европейских языков и
размещены в десятках печатных и электронных СМИ ближнего и дальнего зарубежья.
Ее видеообращение усилиями (первого заместителя постоянного представителя
России при ООН) Дмитрия Полянского увидели сотни дипломатов ООН. Постоянное
участие Савенковой в различных международных конкурсах и фестивалях, где она,
кстати, успешно собирает номинации и награды, делает голос девочки из Донбасса
слышимым. Она самим своим присутствием прорывает информационную блокаду и
наглядно демонстрирует всему миру, с кем на самом воюют украинские силовики и
спецслужбы. И здесь попадание Фаины в базу "Миротворца" - еще одна жирная точка
украинского позора. Воевать с детьми - это просто какой-то "испанский стыд"", -
резюмировал писатель.

Луганская писательница и драматург Фаина Савенкова – самый юный член СП ЛНР.
Несмотря на возраст, у нее уже есть серьезные литературные достижения. В 2019
году ее пьеса "Ежик надежды" вошла в шорт-лист Международного конкурса
современной русской драматургии "Автора – на сцену!", проводимого Гильдией
драматургов России, а также получила специальный приз жюри на Всероссийском
конкурсе драматургии для детской, подростковой и молодежной аудитории ASYL. В
2020 году вошла в короткий список международного конкурса "Евразия-2020" театра
Николая Коляды. Ее произведения опубликованы в двух литературных сборниках,
включая "Всходы", в четырех всероссийских "толстых" литературных журналах и уже
переведены на чешский, болгарский, сербский, французский и итальянские языки.
