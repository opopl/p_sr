%%beginhead 
 
%%file 14_01_2022.fb.mariupol.turystychne_misto.1.pershyj_informacijnyj_pilon_vezha
%%parent 14_01_2022
 
%%url https://www.facebook.com/mistoMarii/posts/pfbid02pBbi2tXdQxuYdw5UPkN4pD4yrCihg2hzpkqkLi3fgoMh1WC8hrH7GbnM2eEZQibXl
 
%%author_id mariupol.turystychne_misto
%%date 14_01_2022
 
%%tags 
%%title Перший інформаційний пілон біля водонапірної Вежі
 
%%endhead 

\subsection{Перший інформаційний пілон біля водонапірної Вежі}
\label{sec:14_01_2022.fb.mariupol.turystychne_misto.1.pershyj_informacijnyj_pilon_vezha}

\Purl{https://www.facebook.com/mistoMarii/posts/pfbid02pBbi2tXdQxuYdw5UPkN4pD4yrCihg2hzpkqkLi3fgoMh1WC8hrH7GbnM2eEZQibXl}
\ifcmt
 author_begin
   author_id mariupol.turystychne_misto
 author_end
\fi

Перший інформаційний пілон з'явився біля головного туристичного символу міста -
водонапірної Вежі. 

Основна концепція навігації – сталь та море. Пілон, як і вся навігація, за
стилістикою відповідає бренду міста.  Матеріал - маріупольський метал  -
підкреслює індустріальну унікальність нашого міста. 

• Комфорт  перебування туристів у Маріуполі 

• Простота – аби кожна людина могла швидко зорієнтуватися на новій місцевості. 

• Доступність – пілон містить аудіосупровід, а на табличках інформацію нанесено
шрифтом Брайля.

Команди m.EHUB\footnote{\url{https://www.facebook.com/mrpl.EHUB}} та MoreInfo
\footnote{\url{https://www.facebook.com/moreinfo.ua}} впродовж року розробляли
навігаційну систему Маріуполя.

\ii{14_01_2022.fb.mariupol.turystychne_misto.1.pershyj_informacijnyj_pilon_vezha.pic.scr.1}

Незабаром у місті Марії з'являться стели, пілони та таблички українською та
англійською мовами, із основною інформацією про туристичні об'єкти та мапою для
зручнішої навігації. 

Компанія MoreInfo  існує уже 5 років і за цей час встигла зробити внесок у
розвиток зручного доступного туризму в багатьох містах України, серед яких
Запоріжжя, Одеса, Херсон, Миколаїв та інші.

Завдяки таким проектам туристична інфраструктура Маріуполя стає ще більш
зручною, простою та орієнтованою абсолютно на всіх жителів і гостей міста.
