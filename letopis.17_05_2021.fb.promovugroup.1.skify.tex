% vim: keymap=russian-jcukenwin
%%beginhead 
 
%%file 17_05_2021.fb.promovugroup.1.skify
%%parent 17_05_2021
 
%%url 
 
%%author 
%%author_id 
%%author_url 
 
%%tags 
%%title 
 
%%endhead 
\subsection{Уявіть собі, як багато бракує в курсі саме Шкільної Історії}
\Purl{https://www.facebook.com/groups/promovugroup/permalink/972540506653144/}

2345 або 2350 років тому македонський воєвода Алєксандра Великого на ім'я
Зопіріон очолив похід 30-тисячного війська на Царство Скіфів. 

Він дійшов до Ольвії і взяв місто в осаду, але не зміг його захватити. На
зворотньому шляху скіфи наздогнали його і вбили разом з усім військом. Отим
самим, македонським, непереможним, яке прокотилось Елладою, Персією, Індією. 

Розгром відбувся в "гетській пустощі", ймовірно в Бесарабії. 

У наших підручниках ця славетна перемога, безумовно, маюча прямий відносок до
України, не представлена ніяк. Перемог у нас мало, поразок багато. 

Поляки ще в середні віки зметікуавали записати це собі, як початок "згадок про
лехітів", хоча і самі визнавали, що йдеться про скіфів, однак там один з воєвод
був, нібито, Лєх.

Взагалі історія багатьох держав починається з епохи Алєксандра, і далеко не
кожна з них може похвастатись саме успішною обороною македонців, а не їхнім
тріумфом. 

Для України така проблема не стоїть - перед тим була славетна перемога скіфів над Дарієм І, а можливо і до того - над Кіром І. 

Та й ще раніше були переможні (і не дуже) походи скіфів на Азію, описані
античними і вавилонськими авторами. 

За легендою, самі скіфи подавали свої грабіжницькі походи як "помсту
єгиптянам", що останні мовляв колись дуже давно на короткий час завоювали
Скіфію (протягом життя лише одного фараона). 

Уявіть собі, як багато бракує в курсі саме Шкільної Історії. 

При чому, як назло, саме тих моментів, де Україна виграє, де її населення - переможець, про якого складають історії. 

Це можна опрокинути і далі на всю історію 

Потім смішно читати що давня історія України - "то якісь безіменні черепки"

Одних тільки надписів з Херсонесу, Ольвії, інших колоній дорійців, ахейців та
іонійців на теренах України нараховується тисячами. 

З них встанвлено не те що чи не повний перелік посадових осіб, а навіть роки
життя горшечників! дати похорон відомих громадян, з вказанням їх подвигів. 

А ще є Боспорське царство, яке воювало з самим Цезарем, Митридатом, потім зі
скіфами і сарматами - і достояло не те що до самих готів, а аж до самих гуннів. 

Ні, нашим дітям це не потрібно. 

Це потрібно російським дітям, у них якраз проблеми з удревнінням історії. 

А наші діти нехай краще плачуть над важкою долею, гнітом, щож-за-народ-у-нас-такий. 

"Не можна читати без брому" щоби, інакше то вже не історія, а якийсь комікс.
