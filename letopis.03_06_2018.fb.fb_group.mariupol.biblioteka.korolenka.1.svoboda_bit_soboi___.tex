%%beginhead 
 
%%file 03_06_2018.fb.fb_group.mariupol.biblioteka.korolenka.1.svoboda_bit_soboi___
%%parent 03_06_2018
 
%%url https://www.facebook.com/groups/1476321979131170/posts/1697979576965408
 
%%author_id fb_group.mariupol.biblioteka.korolenka,kibkalo_natalia.mariupol.biblioteka.korolenko
%%date 03_06_2018
 
%%tags mariupol,vystavka,hudozhnik
%%title Свобода быть собой - выставка - Евгения Яковлева
 
%%endhead 

\subsection{Свобода быть собой - выставка - Евгения Яковлева}
\label{sec:03_06_2018.fb.fb_group.mariupol.biblioteka.korolenka.1.svoboda_bit_soboi___}
 
\Purl{https://www.facebook.com/groups/1476321979131170/posts/1697979576965408}
\ifcmt
 author_begin
   author_id fb_group.mariupol.biblioteka.korolenka,kibkalo_natalia.mariupol.biblioteka.korolenko
 author_end
\fi

Свобода быть собой

Мобильная галерея «Мир увлечений» Центральной библиотеки им. В.Г. Короленко
активно популяризирует творчество мариупольских аматоров. 2 июня в галерее
открылась новая выставка, получившая название «Свобода быть собой». Автор
представленных творческих  работ - молодая мариупольская  художница Евгения
Яковлева - психолог по образованию, философ по состоянию души, поэтесса.
Девушка  представила в экспозиции галереи 50 полотен, большинство из которых -
сюрреалистические картины. Поэтому эпиграфом к вернисажу послужило высказывание
знаменитого Сальвадора Дали: «Сюрреализм — полная свобода человеческого
существа и право его грезить». 

Фантазийные истории полотен Е. Яковлевой притягивают и вовлекают зрителя в свой
сюжет, приглашают поразмышлять. Несколько работ молодой художницы иллюстрируют
произведения американского писателя Г.Ф. Лавкрафта. На выставке представлена
также серия реалистических пейзажей краеведческого плана:  интересные уголки
Старого Мариуполя, умиротворяющие морские пейзажи, торговый порт. 

Заглянувшая на открытие выставки Галина Гарькина (Рудзят), член Национального
союза художников Украины, очень тепло отозвалась о работах Яковлевой,
акцентировала ее внимание на необходимых усилиях для продолжения творческого
роста.

Среди гостей вернисажа были представители Международного литературного клуба
«Моряна», членом которого является Евгения Яковлева. Евгения читала свои стихи,
а впечатленные картинами молодой художницы литераторы высказывали свое мнение.
Сергей Катасонов, Анатолий Рычагов, Евгения Кашанская искренне пожелали девушке
дальнейших творческих успехов. Библиотекари подарили Евгении на память о
выставке персональный буклет.

Празднику в галерее добавили энергию музыкальных инструментов представители
Школы искусств под руководством преподавателя Владимира Солонского. Музыканты
исполнили на фортепиано и кларнете несколько интересных композиций К. Шатровой,
Дж. Мессанжера, В. Солонского. 

И организаторы выставки, и гости вернисажа остались довольны встречей в
Мобильной галерее «Мир увлечений».
