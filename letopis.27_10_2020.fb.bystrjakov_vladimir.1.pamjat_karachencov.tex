% vim: keymap=russian-jcukenwin
%%beginhead 
 
%%file 27_10_2020.fb.bystrjakov_vladimir.1.pamjat_karachencov
%%parent 27_10_2020
 
%%url https://www.facebook.com/permalink.php?story_fbid=1640420629444394&id=100004294186082
 
%%author_id bystrjakov_vladimir
%%date 
 
%%tags karachencov_nikolaj.akter,pamjat,rossia,smert
%%title День Памяти, переходящий в День рождения...
 
%%endhead 
 
\subsection{День Памяти, переходящий в День рождения...}
\label{sec:27_10_2020.fb.bystrjakov_vladimir.1.pamjat_karachencov}
 
\Purl{https://www.facebook.com/permalink.php?story_fbid=1640420629444394&id=100004294186082}
\ifcmt
 author_begin
   author_id bystrjakov_vladimir
 author_end
\fi

День Памяти, переходящий в День рождения...

Да, это о Нём, моём Друге и одним из самых талантливых моих Исполнителей.... о
Николае Караченцове. 

27 октября он явился на этот свет, а 26 октября навсегда ушёл от нас. Весёлый,
шумный, до ужаса обязательный 
(на себе испытал), работоголик каких мало, анекдотчик, центр любой компании,
ПЕВЕЦ, так и не раскрытый до конца (мешали театр и кинематограф)... Короче,
перечислять все винтики, из которых состоял Коля, всё равно, что перечислять
все награды "нашего дорогого Леонид-Ильича". 

\ifcmt
  pic https://scontent-frt3-1.xx.fbcdn.net/v/t1.6435-9/122727209_1640420432777747_3984679793606283611_n.jpg?_nc_cat=104&ccb=1-5&_nc_sid=730e14&_nc_ohc=LIJ0FKRAXgIAX-b-ff8&_nc_ht=scontent-frt3-1.xx&oh=d56364c2b0ed7f66812accc76f1aff76&oe=61D3FDBC
  @width 0.6
\fi

Это фото - последнее наше с ним. С сыном Жориком и с женой Олей мы приехали к
нему в больницу....

Помню, я был приятно удивлён тем, что он выглядел румяным и даже каким  -то
округлившимся. 

Но сын Андрюша сказал мне.... "Это не то, что ты думаешь, Володя ! Это ....", и
он замолчал.

А потом через несколько дней мы улетели в Турцию, и уже там я увидел на мобиле,
что звонит Андрей, и понял...

Знаете, я очень редко плачу...последний раз года три назад... у нас в
Крестовоздвиженской, на исповеди, когда слёзы полились сами, помимо меня,
полились градом.... а потом сразу вдруг стало легко-легко, и я понял, что
....отпустило. 

А вот тогда в Турции... вдруг тоже.... градом...но только потом... не
отпускало. Долго-долго...И ещё...пошли воспоминания... Их было много... 30 лет
всё ж-таки... от первой записи для мультика "Алиса в Зазеркалье", 

и потом.... через "Дорогу к Пушкину", через Америку, Австралию, Израиль,
Казахстан, Украину, Россию....Концерты, посиделки, новые знакомства, ночные
многочасовые перезвоны (..."Слушай сюда, Вова ! Есть свежий анекдот... - Коля !
Да сейчас три ночи !!! - Ух ты, а мы во Владивостоке уже позавтракали... Ну
ничего, выспишься, слушай ! (и рассказывается тот самый, что я ему рассказал
месяц тому...) 

У них в театре шла жестокая борьба за звание короля анекдотов, (и мастеров по
этому делу хватало), но уж если кто-либо приносил наисвежайший "свежак", потом
мог ходить спокойно месяц в королях...

...Когда меня вдруг "торкнуло" на писательство, то именно Коля дал мне столько
благодарного материала для моих "баек". Это всё были случаи из его
разнообразнейшей жизни, или из наших совместных казусов... 

Знаете, стыдно признаться, но для меня он умирал дважды. Первый раз, когда я
увидел его впервые после аварии, 

и вдруг с ужасом понял, что... НИКОГДА больше не услышу его непередаваемый
хриплый баритон. Этого голоса просто не существовало теперь во всей Вселенной. 

И означало, что НИКОГДА он не споёт того, что напишу для него. Я тогда
сдержался, а моя доця Ира, уже после, в лифте разрыдалась. Сильно, безудержно,
беззащитно... Как маленькие дети, потерявшиеся посреди большого города...

...На часах уже начало второго ночи, и день его 
ухода сменился днём его рождения. 

(говорят, только отмеченные Богом уходят накануне своего дня рождения. Так ушёл
мой отец Быстряков Юрий Иванович, так ушёл Коля...) 

И не хочется заканчивать этот мой пост на миноре. 

И хочется верить, что Природе не свойственно разбрасываться человеческим
материалом, уничтожая безвозвратно всё наработанное человеком по жизни...
Интеллект, мастерство, опыт, любовь в разных её проявлениях... всё то, из чего
формируется Личность. 

И мне ближе всё ж-таки теория реинкарнации, теория Вечной Неумирающей Души. 

А раз так, то осмелюсь сказать : С днём рождения, Колюня! Мы знаем, что ты
СЛЫШИШЬ нас !....

\ii{27_10_2020.fb.bystrjakov_vladimir.1.pamjat_karachencov.cmt}
