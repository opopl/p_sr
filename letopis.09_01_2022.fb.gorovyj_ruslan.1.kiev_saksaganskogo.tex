% vim: keymap=russian-jcukenwin
%%beginhead 
 
%%file 09_01_2022.fb.gorovyj_ruslan.1.kiev_saksaganskogo
%%parent 09_01_2022
 
%%url https://www.facebook.com/gorovyi.ruslan/posts/7013129488697976
 
%%author_id gorovyj_ruslan
%%date 
 
%%tags ekonomika,jazyk,kiev,magazin,mova,ukraina,ukrainizacia
%%title Київ, Саксаганського
 
%%endhead 
 
\subsection{Київ, Саксаганського}
\label{sec:09_01_2022.fb.gorovyj_ruslan.1.kiev_saksaganskogo}
 
\Purl{https://www.facebook.com/gorovyi.ruslan/posts/7013129488697976}
\ifcmt
 author_begin
   author_id gorovyj_ruslan
 author_end
\fi

Київ, Саксаганського.

- Ой, здравствуйтє, - посміхається мені продавчиня магазину, яка вийшла
перекурить за відсутності покупців, 

- І вам доброго дня.

- Давно вас нє відна, нє заходітє.

- Та я до ваших конкурентів ходжу,  - кажу, - вони украксаїнською говорять.

- Так а ми тоже можем!

- Та я інтуїтивно розумію, що можете, та жодного разу не говорили. Тож отак.
Гарного дня вам. 

І пішов.

Я вперто ігнорую заклади де зі мною говорять російською. Не шкандалю, не
вимагаю, однак якщо вони не переходять на українську взагалі під час відвідин,
то після «спасіба, пріходітє ісчо», зазвичай чують, «та Боже збав, далі без
мене». 

Мені не важко пройти пару метрів, а навіть і десять і сто, щоб витратити гроші
там, де мені приємно і дати заробити тим, хто принаймні в мовному питанні дає
відпочинок моїм вухам.

Отака в мене щоденна лагідна українізація з економічним присмаком. Всім гарного
дня. Любіть Україну і не забувайте хто хуйло.

\ii{09_01_2022.fb.gorovyj_ruslan.1.kiev_saksaganskogo.cmt}
