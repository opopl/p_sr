%%beginhead 
 
%%file 31_12_2022.fb.galushko_denis.1.zima_bude_vkrai_vazh
%%parent 31_12_2022
 
%%url https://www.facebook.com/HalushkoDenis/posts/pfbid02HHCco1R4LBjzib5zyZi5MosCizD3u1aXrvKBqScwhRLVeHPbv37KwSVvzmwB5kCnl
 
%%author_id galushko_denis
%%date 31_12_2022
 
%%tags 
%%title Зима буде вкрай важкою, але, але мені хочеться літа...
 
%%endhead 

\subsection{Зима буде вкрай важкою, але, але мені хочеться літа...}
\label{sec:31_12_2022.fb.galushko_denis.1.zima_bude_vkrai_vazh}

\Purl{https://www.facebook.com/HalushkoDenis/posts/pfbid02HHCco1R4LBjzib5zyZi5MosCizD3u1aXrvKBqScwhRLVeHPbv37KwSVvzmwB5kCnl}
\ifcmt
 author_begin
   author_id galushko_denis
 author_end
\fi

\#запискидилетантазсу

Колись один військовий авторитет, сказав, що взимку йому легше воювати, а тим
паче в обороні. Суто зі стратегічної точки зору, я його розумію.  Але, але, я
хочу літа. 

Минулої зими я ставив, собі на меті дожити до весни. 

Я робив це паралельно вивчаючи на позиції розрахунку ппо, принцип роботи
\enquote{стінгера} який лежав поруч, у ютюбі, командир в принципі так і сказав: \enquote{та що
там стріляти, в ютюбі подивитеся...}  

І на початку квітня всі п...ри получили п..зди на околицях Києва і області. 

І перспектива пожити весною вже не була такою захмарною. 

Навесні була нова мета, дожити до дня народження, а воно в кінці літа.

Йшов час, йшли від нас побратими, на майдані лунав гімн, ми з похмурими
обличчями проводжали наших. 

Було пару моментів,  під Марїнкою та Пісками, що я думав, це говорила втома,
робота танку, палаючі навколо будинки від чергових приходів... 

Але ми вже 4-й місяць в Бахмуті, ми не піхота,  не вигрібаємо як вони і тут
не буде жодних пафосних слів чи епітетів, їх очі, ви просто будете після
цього їх впізнавати в натовпі, без зайвих питань, це буде особливий погляд з
безодні. 

Вже минуло декілька ротацій різних бригад, сказати що тут тяжко це нічого не
сказати деяких розбирають ще до виходу на позиції, це місто і його околиці
це резиденція пекла на землі, на сьогодні.

Нічого не міняється в моїх планах, варто дожити до весни.

Так дожити, щоб вони колосально вигрібли п...ди, як і минулої весни, а літньою
компанією ми  добили їх залишки на території нашої країни.

Все поступово, спочатку весна, без жодних ілюзій, що буде легко, але варто
бути спокійним і робити свою роботу.

Зима буде вкрай важкою, але, але мені хочеться літа...

На фото, літо 22 і будинок (позиція), Водяне, звичайно якого вже нема, його
розібрали ще при нас. 

Побажання на наступний рік, живіть так, аби вам усміхалося літо, з прийдешнім,
все буде рок-н-рол!
