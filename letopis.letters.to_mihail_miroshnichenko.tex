% vim: keymap=russian-jcukenwin
%%beginhead 
 
%%file letters.to_mihail_miroshnichenko
%%parent letters
 
%%url 
 
%%author_id 
%%date 
 
%%tags 
%%title 
 
%%endhead 

Добрый вечер, Михаил! Меня зовут Иван, я киевлянин, программист и живу и работаю
в Киеве. Много сейчас читаю про Мариуполь, у меня есть также несколько книжек
про Ваш замечательный Город. К сожалению, до войны не успел побывать в
Мариуполе, так что меня... с Мариуполем связывает только воспоминание о том,
как я когда то сел на поезд Киев-Мариуполь, и потом сошел на промежуточной
станции... Давно это уже было... Но... я принимаю близко к сердцу, все, что
произошло с Мариуполем и с мариупольцами. Страшная, невообразимая трагедия...
которая забрала ваших близких, родных, детей, родителей, раскидала вас,
мариупольцев, по всему миру... Тем не менее, я уверен, лучшие времена для
Мариуполя еще впереди, и отчаиваться точно не надо... Да, как я сказал выше, я
программист. В свободное время от работы... я также занимаюсь систематизацией
разных публикаций о войне, у меня есть отдельный проект Летописи Войны, который
я веду - пока что сам - довольно уже давно, собственно говоря, я сохраняю от
забвения посты на фейсбуке, потом раскидываю их по авторам, темам. Технология,
которую я использую, называется LaTeX (не знаю, имеете ли вы отношение к физике
или математике, но это та технология, которая повсеместно используется учеными
в научном мире для записи и публикации своих статей и исследований ). Так вот.
Я сохраняю в печатном виде, то, что я читаю, считаю важным для сохранения...
Про Мариуполь я в последнее время тоже сохраняю, и довольно много. И про
довоенный мирный Мариуполь, и для ужасы войны. У меня есть телеграм канал
https://t.me/kyiv_fortress_1, там уже довольно много выложено, так же как и на
этом фейсбук - аккаунте, также я постепенно выкладываю собранные  материалы - с
уважением к авторам - я всегда указываю, кто и когда что создал - на
https://archive.org - это интернет сайт, посвященный сохранению всего в
Интернете, ссылка на меня там https://archive.org/details/@kyiv_chronicler.
Зачем все это делается... знаете... Мариуполь физически то убили... но
Мариуполь не умер. Мариуполь остался в Духовном Пространстве, и это очень очень
важно... Но если не записывать... если этим не заниматься... есть риск, что
Мариуполь умрет духовно... и это будет уже навсегда. А духовная смерть - это...
еще страшнее, чем смерть физическая... Поэтому я этим и занимаюсь, хотя я и
киевлянин, вообще то говоря. Хочется, знаете, тоже приехать в наш Украинский
Возрожденный Мариуполь, как гость, в свое время ) Надеюсь, все о чем я пишу,
Вам будет интересно. 

Я вижу, Вы в последнее время выкладываете в фейсбуке свой дневник о блокадном Мариуполе,
буду рад, если Вы разрешите его тоже выложить в сохраненном виде. 
К сожалению, если не записывать систематически, а просто выкладывать в фейсбуке,
то есть риск, что (1) мало людей смогут узнать Вашу историю (2) фейсбук сам по себе очень плох
в плане сохранения памяти, из-за своей непредсказуемой политики (3) ну и конечно, нельзя исключить
такого, что фейсбук просто рухнет сам по себе, и все, что в нем есть, просто исчезнет навсегда. 

Отдельно хочу отметить ужасную ситуацию в Киеве и вообще по Украине - с книжками про Мариуполь - 
их очень мало, либо вообще нет. Об этом я более детально пишу в своем посте здесь 
https://www.facebook.com/ivan.ivan.kyiv/posts/pfbid02WyjjCrDz4v3PJuHKkKMm2Fxg1ninb3ZibGgekSsQuF2WYZ7npVwk7h1qEcLV9u4sl

Я послал Вам запрос на дружбу в фейсбуке, буду рад
подружиться с Вами здесь. 

С уважением, Иван.

%Михаил
%Михаил Мирошниченко
%Доброго здравия Иван, я не против если вы будите сохранять мой дневник, только без искажений. Я и сам интересовался историей нашего славного города и очень рад что вы проявляете к нему интерес. Вы делаете хорошее дело, пусть Бог поможет вам в этом. Будьте здоровы.

Добрый день, Михаил! Большое спасибо! Вы тоже делаете большое дело, я вижу,
сколько Вы пишете, и очень много постов у Вас, которые как раз поднимают дух!
Да, конечно, без искажений! Стараюсь насколько это возможно, сохранить как
есть, проверяю внимательно все... Недавно вот записал пост Наталии Дедовой,
список погибших в Мариуполе, 130 страниц получилось, вот это было тяжело -
потому что реально много материала... и нужно проверять каждую букву... 4 дня
где то потратил на запись (этот пост можно найти по ссылкам, которые я написал
выше - там у меня есть записи как трагических вещей, так и постов о довоенном
Мариуполе). А один из Ваших постов Вашего дневника в записи выглядит так (см.
изображения ниже). В таком виде ему уже не грозит забвение, и я надеюсь,
намного больше людей смогут прочитать его в свое время (там, если у Вас есть
замечания, как Вас записать на украинском языке как автора, могу подправить).
Вообще, глобально моя идея состоит в том, чтобы в одном месте собирать и
сохранять важные нужные посты о Мариуполе (ну и вообще говоря, не только про
Мариуполь, а вообще про всю Украину), как мирном довоенном, так и о трагических
событиях. 

%13:22:12 09-03-23
Щодо сьогоднішнього дня, Ви спитали, як у нас всіх справи. Але... Сьогодні до
речі день народження нашого Кобзаря, - Тараса Григоровича Шевченка, - йому вже
виповнилось 209 років, про це я у Вас нічого, на жаль, не побачив... І Ви
знаєте, це звичайно добре, що Ви пишете багато про війну, найостанніші
найтрагічніші новини публікуєте. Але я хочу зауважити таке. Якщо весь час
писати про війну, весь час писати про те, що там знищене... тут... там...
плавати в усіх оцих тяжких новинах. То... це все врешті решт просто висмокче
душу. Якщо писати тільки про війну, забуваючи про те, щоби також знаходити
Радість, хвилинки Щастя в повсякденному житті... то... перемоги не буде,
оскільки не буде духовних, душевних сил, щоби врешті решт до цієї - такої,
жаданої перемоги, дійти, дожити... Так, звісно росіяни - то є люті звірі, що
давно вже втратили людську подобу, і мені це стало зрозуміло вже давним-давно,
ще коли вони вчинили Бучу та Маріуполь.. Але... читання всіх новин підряд,
особливо тяжких новин... занурення день і ніч у війну, особливо якщо ти живеш у
відносно спокійному місці... війну виграти точно не допоможе. Я так думаю.
Непотрібно читати всі новини підряд, непотрібно бігти, щоби бути в курсі всіх
останніх новин. Треба знаходити час, так, треба знаходити час, щоби плекати мир
в душі, щоби читати книжки, думати про Вічне, незважаючи на будь-які зовнішні
обставини. Тому я... наприклад. У мене сьогодні був такий день. Я вранці поїхав
до центру, побачив, як люди співають біля пам'ятнику Шевченку. Пісня була із
слів Реве та Стогне Дніпр Широкий, сердитий вітер завива... Я якраз наскільки я
пам'ятаю - почув строки - Та ясен раз-у-раз скрипiв... Дуже добре співали...
Невелика кількість людей, але дуже душевно... Ну добре... Я подивився... теж
трохи підспівав... і пішов собі далі... йду собі Києвом... йду... думаю...
врешті решт дійшов туди... сюди... бачив те... бачив се... а зараз я вдома. Ну
от. Прохання таке - сьогодні день народження Шевченка. Напишіть щось, так, від
душі, скільки хочете, стільки і пишіть... Розумієте... Про Мир, Культуру,
Поезію, Вічність, Красу, Силу... потрібно писати теж! Багато-багато писати,
так, і щоби цього було більше, ніж війни та розпачу! І... якщо ми навіть в такі
знаменні дати забуваємо про таких визначних людей, натомість обираючи у своїх
(інформаційних) уподобаннях війну, нещастя, руйнування... замість Вічності,
Культури, Сильного Слова, то як ми взагалі збираємось перемагати? Так що...
Борітеся — поборете! Вам бог помагає! За вас правда, за вас слава І воля
святая! Учітесь - читайте... і чужому навчайтесь, і свого не цурайтесь...  фото
з сьогоднішнього мого дня (1) - люди біля пам'ятнику Шевченку (2) - мозаїка на
станції метро Золоті Ворота.

Добрый день, Надежда. Я вижу Вы удалили меня из списка друзей на фб. Для меня
это вообще не проблема, и я даже ничуть не обиделся. Я понимаю, что вы -
мариупольцы... как вы сами сказали в книжке, вас никто не понимает... Ну
ладно... Да... Не забудьте забанить меня в телеграме, а то я случайно что то не
то напишу часом ))) а из Вашего канала я удалился сам, чтобы не нервировать
Вас. Так что хорошего дня! Не забудьте посмотреть фильм Город 44, Вы мне
обещали! И жду от Вас новых книжек про Мариуполь!
