% vim: keymap=russian-jcukenwin
%%beginhead 
 
%%file 21_11_2021.fb.bilchenko_evgenia.2.chelovek_ne_sverh_ne_polu.cmt
%%parent 21_11_2021.fb.bilchenko_evgenia.2.chelovek_ne_sverh_ne_polu
 
%%url 
 
%%author_id 
%%date 
 
%%tags 
%%title 
 
%%endhead 
\subsubsection{Коментарі}

\begin{itemize} % {
\iusr{Алексей Бажан}

Я, вообще-то, не только все песни слышал, в порядке их появления, но и прозу
Елизарова читал. Так вот, Библиотекарь роман, если угодно, антисоветский: все
светлое советское прошлое в прагматике текста абсолютно фиктивно. Что до
Сталинского костюма, в России Рабфак у всех более-менее на слуху и очевидную
аллюзию люди считывают сходу. Наверное, не все, находятся же товарищи, серьезно
обсуждающие "единомышленника" 

\url{https://www.youtube.com/watch?v=Naa3biMAwKU}

\iusr{Александр Мурашов}
\textbf{Алексей Бажан} я об этом как раз и написал в комменте вверху.

\iusr{Ольга Батурина}
\textbf{Евгения Бильченко}, блестяще!
Как же я люблю Вас читать!

\iusr{Александр Мурашов}

Неосталинизм это как раз чистый постмодерн, где образы одного времени
политтехнологами профессионально переносятся на другую реальность. Но так в
политтехнологиях, но не так в искусстве. Помним, как в мае, Красном мае 1968,
европейские интеллектуалы в захваченной ими(или освобожденной) Сорбонне весело
юзали образы тоталитаризма как альтернативы ,либо троллинга общества
потребления- анархист, антифашист или анархо синдикалист, сидящий под портретом
Сталина ,Троцкого и (или) Мао (прикол в том, что для парижск. студентов даже де
Голль был тоталитарным политиком !) Когда в конце 1980-х угрюмая
партноменклатура угрюмо строила общество "рынка, который все порешает", поэты и
интеллектуалы создали НБП как чистый перформанс, как андеграундный кич, когда
Лимонов, Летов, пришедший из мамлеевского кружка Дугин сварганили карнавальный
фейкрверк из "всего самого красно коричневого",где никакой идеологии по сути не
было, а было чистое искусство и пощечина постмодерну. Вот и Елизаров
"тоталитарными" образами каранвально троллит обывателя, стонущих от
кредитов-ипотек мерчандайзеров и прочих менеджеров среднего звена. Конечно он
трикстер. Это Мамонов в роли Ивана Грозного прямо из саксофониста Такси блюз

\iusr{Krzysztof Węgrzwienievcki}
Книжки он пишет жуткие, до мурашек.

\end{itemize} % }
