% vim: keymap=russian-jcukenwin
%%beginhead 
 
%%file 24_10_2020.fb.vasiliy_volga.1.kiev_bander
%%parent 24_10_2020
%%url https://www.facebook.com/Vasiliy.volga/posts/2732433890407359
%%tags kiev,bandera
 
%%endhead 

\subsection{Дорогой мой бандеровский Киев!}
\label{sec:24_10_2020.fb.vasiliy_volga.1.kiev_bander}

\url{https://www.facebook.com/Vasiliy.volga/posts/2732433890407359}
\index[authors.rus]{Василий Александрович Волга}

Доброе утро, мой город, который уже давно стал не моим. Доброе утро, жители
Киева, среди которых киевлян уже почти не осталось, что собственно говоря и
показали выборы. Доброе утро рагулье, которое вытеснило культуру и здравый
смысл из древнего русского города, и которое так прочно засело на семи киевских
холмах. 

Что ж, киевляне, мы проиграли - надо признать. У нас с вами не оказалось той
животной силы, которую, по всей видимости, дают карпатские горы, где новые
хозяева нашего города еще совсем недавно пасли овец и растили свиней. 

Как мощно выглядит в этом смысле интервью кандидата в мэры нашего города от
партии Зеленых, недавней львовянки - панянки Верещук, которое она давала
Гордону. 

Гордон у нее спрашивает: «Вы считаете себя киевлянкой?»

Она отвечает: «Конечно».

Гордон: «А когда Вы приехали в Киев и как Вы здесь оказались?».  Она: «В 2014.
Сначала муж, а затем и мы (семья), паровозиком». 

Смеется. 

Гордон: «А во Львов вернетесь когда-нибудь?»

Она: «Никогда. Я киевлянка. Киев нуждается в таких, как я. Я европейка».  Что
тут можно сказать?! 

У меня прямо сейчас пальцы чешутся и руки неестественно напряжены, словно они
помимо моей воли хотят кого-то за горло подержать, но не смогу же я подержать
за горло сотни тысяч тех, кто наводнил своим жлобством некогда великий древний
город и миллионы тех, кто переродился под их влиянием? У меня руки две, а их
уже миллионы.  Доброе утро, бандеровский город Киев. Доброе утро, чужой город.
