%%beginhead 
 
%%file 09_01_2020.fb.mariupol.centr.kuindzhi.1.rizdvo_u_ridnomu_misti
%%parent 09_01_2020
 
%%url https://www.facebook.com/100023161160311/posts/pfbid02ogkRFwv324fvfCZQFa5eVeHvNNqQgpxsLJpRX8v9qsV2yKaXeofZgKLvuQtaCyxTl
 
%%author_id mariupol.centr.kuindzhi
%%date 09_01_2020
 
%%tags 
%%title "Різдво у рідному місті" - різдвяна театралізована програма
 
%%endhead 

\subsection{\enquote{Різдво у рідному місті} - різдвяна театралізована програма}
\label{sec:09_01_2020.fb.mariupol.centr.kuindzhi.1.rizdvo_u_ridnomu_misti}

\Purl{https://www.facebook.com/100023161160311/posts/pfbid02ogkRFwv324fvfCZQFa5eVeHvNNqQgpxsLJpRX8v9qsV2yKaXeofZgKLvuQtaCyxTl}
\ifcmt
 author_begin
   author_id mariupol.centr.kuindzhi
 author_end
\fi

Сьогодні, 9 січня 2020 року у Центрі сучасного мистецтва і культури ім.Куїнджі
волонтерський гурт \enquote{АТМА} презентував різдвяну театралізовану програму \enquote{Різдво
у рідному місті}.

На початку заходу усіма учасниками хвилиною мовчання було вшановано пам'ять
загиблих у авіакатастрофі.

В стародавніх українських традиціях закарбовані духовні знання, сакральні
символи, значення яких зараз поступово відновлюється. У  програмі були
представлені вертеп, колядки та щедрівки, різдвяні історії, легенди про древні
українські традиції та їх значення. Це важливо для духовного, культурного
відродження українського народу, для розуміння національної сутності як на
індивідуальному, так і на колективному рівні. Саме за культурою, освітою та
духовністю світле майбутнє нашої країни!

Щиро дякуємо учасникам волонтерського гурту \enquote{АТМА} за чудову програму і
настрій, а також усім, хто завітав сьгодні до нас у гості.

Ласкаво просимо знову!
