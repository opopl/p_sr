% vim: keymap=russian-jcukenwin
%%beginhead 
 
%%file 19_08_2021.fb.golovachev_andrej.1.krym_platforma
%%parent 19_08_2021
 
%%url https://www.facebook.com/permalink.php?story_fbid=2680916195551475&id=100008993618796
 
%%author Головачев, Андрей
%%author_id golovachev_andrej
%%author_url 
 
%%tags 2014,geopolitika,krymskaja_platforma,rossia,ukraina
%%title Что я думаю о Крымской платформе
 
%%endhead 
 
\subsection{Что я думаю о Крымской платформе}
\label{sec:19_08_2021.fb.golovachev_andrej.1.krym_platforma}
 
\Purl{https://www.facebook.com/permalink.php?story_fbid=2680916195551475&id=100008993618796}
\ifcmt
 author_begin
   author_id golovachev_andrej
 author_end
\fi

Что я думаю о Крымской платформе.

Любое обсуждение проблемы Крыма , на мой взгляд, совершенно бесплодно, рутинно
и обманчиво,   если   украинское общество  до сих  пор не решилось   дать
честный и объективный  ответ на неприятный вопрос:

ПОЧЕМУ мы отдали Крым , отдали без единого выстрела.?  

Ведь,  в  сущности, Россия пришла и поставила нас перед фактом, что она забрала
Крым.  

Пока нет честного ответа на этот вопрос всякие "крымские платформы " это не
более чем  бессмысленное  собрание недругов России.  Подчеркиваю, не друзей
Украины, а недругов России.  Это , как бы перекличка тех кто против России. 

Степень  недружественности к России может быть разной . Может быть сильной.
Тогда такая страна посылает на Крымскую платформу высшее должностное лицо -
Президентов, как  сделают балтийские страны и Польша. 

Может быть умеренной , тогда приедут главы МИД . Или символической , тогда
приедут министры транспорта , послы или депутаты.

Крымская платформа, по этой причине, будет абсолютно бесплодным мероприятием,
так оно  также не сможет и не захочет  поставить  на обсуждение  прОклятый
вопрос:

Почему Крым был отдан без единого выстрела?  

Пока нет честного ответа на это вопрос , совершенно невозможно  даже
подступиться к решению проблемы Крыма, но Крымская платформа совершенно не
собирается решать эту проблему.   Еще раз: Это просто ПЕРЕКЛИЧКА за на счет.

 тем не менее , хотелось бы напомнить , что в свое время   на  переговорах  с
 Зеленским   бывший госсекретарь США Майк Помпео заявил буквально следующее:

 " Украина отдала Крым и  поэтому потеряла  его навсегда". 

Предельно жесткое заявление!!   Но в   Украине   его  предпочли не заметить и
Зеленский продолжает   взывать к мировой общественности  подсобить вернуть
Крым.

Что интересно, заявление Помпео не вызвало резонанса  не только в Украине , но
и  в ЕС и в самих США. Сложилось впечатление , что мир смирился с тем , что
Украина ,действительно,  сама отдала Крым.

Тогда в чем смысл подобных Крымских платформ, если никто не опровергнул слова
Помпео?   Если есть негласный  консенсус , что Украина сама прос....ла Крым.
Ответ уже дан выше.

Так почему  же все таки Украина  фактически отдала Крым?  

Вот официальные данные, обнародованные  в пятую годовщину аннексии Крыма с трибуны ВР:

Не изменили своей стране, находясь в автономии во время ее аннексии: 

\begin{itemize}
  \item -из 13 468 солдат и офицеров Вооруженных Сил Украины - лишь 3990 военнослужащих (29,6\%); 
  \item -Министерство внутренних дел Украины: из 10 936 - только 88 (0,8 \%);
  \item -Государственной пограничной службы Украины: из 1 870-ти только 519 (29,7\%)
  \item -Служба безопасности Украины: из 2 240 только 242 (10,8) процента; 
  \item -Управление государственной охраны - из 527 - только 20 
\end{itemize}

То есть, почти все те, кто должны были защищать Крым, оказались изменниками Родины.

Может быть  поэтому Помпео и сказал  " Украина отдала Крым"

Но почему имело место массовая измена Родине?  

Почему наши военнослужащие массово отказались защищать страну и позволили
России  абсолютно беспрепятственно аннексировать  Крым ?

Факт настолько беспрецедентный в  истории , что до сих пор мы боимся не то ,
что дать честное пояснение происшедшему  , но даже обсуждать его не беремся
из-за стыда.

Причина , однако , на мой взгляд,  на поверхности.  Это  катастрофический
провал гуманитарной политики в Крыму, которым Россия   воспользовалась,  почти
не напрягаясь.

В этой связи я вспоминаю великого Ли Кван Ю. Когда он впервые стал премьер
-министром Сингапура  в 1959г , то его страна состояла  из :

\begin{itemize}
  \item китайцы - 75.4\%
  \item малайцы -13.6\%
  \item индийцы-  8.6\%
\end{itemize}

Далее  я передаю слово самому Ли Кван Ю:

"Мы полагали, что в интересах будущего Сингапура нам следовало воссоединиться с
Малайей, поэтому в сентябре 1963 года мы вошли в состав единого государства –
Малайзии. Но не прошло и года, как в июле 1964 года Сингапур стал ареной
расовых столкновений между малайцами и китайцами. Мы попали в ловушку и
оказались вовлеченными в тяжелую борьбу с малайскими экстремистами из правящей
Объединенной малайской национальной организации (ОМНО – United Malay National
Organisation). Они стремились создать «Малайзию для малайцев», в которой
малайцы играли бы доминирующую роль. Малайские националисты использовали
межобщинные столкновения, чтобы запугать нас. Чтобы противостоять им, мы
сплотили малайцев и немалайцев по всей Малайзии в Малайзийское объединение
солидарности (Malaysian Solidarity Convention), целью которого было создание
«Малайзии для малайзийцев». Тем не менее к августу 1965 года у нас уже не
оставалось иного выбора, кроме как выйти из состава Малайзии. 

Столкнувшись с угрозой межрасовых столкновений и запугиванием, жители Сингапура
исполнились решимости пережить все трудности, связанные с созданием
независимого государства. Болезненный опыт межрасовых столкновений сделал меня
и моих коллег убежденными сторонниками построения многонационального общества,
в котором всем гражданам независимо от расы, языка или религии гарантировались
бы равные права.  ЭТО БЫЛО КРЕДО, ОПРЕДЕЛИВШЕЕ  НАШУ ПОЛИТИКУ"

Вот и весь рецепт  от Ли Кван Ю : 

"Построение многонационального общества, в котором всем гражданам независимо от
расы, языка или религии гарантировались бы равные права." 

И никаких мовных или иных инспекторов , никаких квот и наказаний и прочей
чепухи.   Именно на этой либеральной  основе Ли Кван Ю смог добиться
высочайшего благосостояния для  своей многоязычной и многоконфессиональной
страны и причем  сделал это не в самом благоприятном окружении соседей.

Ли Кван Ю не проводил   никаких Сингапуских платформ, этих бесплодных ,
безжизненных, бюрократических и дорогостоящих мероприятий. Он один раз обжегся
и все сразу понял.  Величие Ли Кван Ю как раз и заключается в том, что он ,
никогда не наступал  на одни  и те же грабли дважды. 

Если бы Украина неуклонно  придерживалась  этого мудрого   совета Ли Кван Ю ,
не занималась  бы социальной инженерией,  не пыталась  бы лепить единого
унифицированного украинца, то она не потеряла бы Крым.   И не было бы
гражданского   конфликта  на Донбассе и российской интервенции.    Да и самого
евромайдана, наверное, тоже не было бы. Просто не было бы причин и оснований
для всего этого  кошмара.

Но для этого надо найти и  привести к власти своего украинского Ли Кван Ю.
Увы, вместо Ли Кван Ю  у нас Зеленский со его  мовными омбудсменами и
крымскими платформами. 

Что, ж каждому свое.
