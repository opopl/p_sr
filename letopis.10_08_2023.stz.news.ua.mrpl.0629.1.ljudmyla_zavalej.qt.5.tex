% vim: keymap=russian-jcukenwin
%%beginhead 
 
%%file 10_08_2023.stz.news.ua.mrpl.0629.1.ljudmyla_zavalej.qt.5
%%parent 10_08_2023.stz.news.ua.mrpl.0629.1.ljudmyla_zavalej
 
%%url 
 
%%author_id 
%%date 
 
%%tags 
%%title 
 
%%endhead 

\begin{quote}
\em\enquote{Ми ще на початку повномасштабного вторгнення поділили місто на зони
відповідальності. Тому що через обстріли важко було пересуватись. Я відповідала
за Приморський район, Черьомушки. Моя дорога в центр міста раз на декілька днів
виглядала так: завершився черговий обстріл, значить, у мене є хвилин 30, щоб
сісти в машину і летіти до гуманітарного штабу. І так само у зворотному
напрямку.

Одного разу я вимушена була шукати тимчасове укриття, тому що за ці 30 хвилин
перерви не встигла дістатись місця. Тож заїхала у санаторій-профілакторій
\enquote{Чайка}. Я була шокована тим, що побачила там. Людей з інвалідністю, старих із
будинку ветеранів ніхто не евакуював, і медичний персонал, медсестрички на
каталках, по одному перевозили самотужки цих людей в укриття. У них там не було
їжі, води. Це був просто жах...}
\end{quote}
