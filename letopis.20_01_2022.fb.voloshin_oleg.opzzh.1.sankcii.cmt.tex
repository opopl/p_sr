% vim: keymap=russian-jcukenwin
%%beginhead 
 
%%file 20_01_2022.fb.voloshin_oleg.opzzh.1.sankcii.cmt
%%parent 20_01_2022.fb.voloshin_oleg.opzzh.1.sankcii
 
%%url 
 
%%author_id 
%%date 
 
%%tags 
%%title 
 
%%endhead 
\zzSecCmt

\begin{itemize} % {
\iusr{Андрій Єрмолаєв}

Не переживай, Олег, это они с перепугу. Еще впереди \enquote{Украина-гейт} - и с
Буризмой, и со средствами МВФ, и со \enquote{словацким хабом}. Уотергейт и Ирангейт
покажется \enquote{цветочками}. Потому и лупят, прикрывая списками уже \enquote{списанных и
сбежавших}. Зато какой кадровый потенциал выращивают - Яценюк, Гончарук,
Вакарчук, Шабунин... Просто поп-арт-галерея)))

\begin{itemize} % {
\iusr{Oleg Voloshin}
\textbf{Андрій Єрмолаєв} 

спасибо, Андрей Васильевич. Часть американского истеблишмента так глубоко
увязла в коррупционных схемах в Украине, что это очень исказило их оптику и
связало руки. Потому они вынуждены потакать всяким уродам (и прикрывать их)
лишь из-за того, что те слишком много об их делах знают.


\iusr{Андрій Єрмолаєв}

Учитывая, какая впереди выборная кампания в США, Байдену стоит 10 раз подумать,
какие там у него \enquote{команды А}, \enquote{Б} и прочие советники, и в какую \enquote{трубу} они
загоняют деда. Один Афган чего стоил...

\end{itemize} % }

\iusr{Игорь Волошин}

Перед тем как пасть под ударами варваров великие империи часто оказывались
мертвы изнутри. Их организм был выеден злоупотреблениями, популизмом и
громадным расслоением общества. В итоге они сначала теряли престиж, потом лицо
и даже минимальное уважение.

Очень хочется верить в то что американское общество имеет здоровые корни и не
повторит печальную судьбу павших гегемонов.


\iusr{Николай Моисеенко}
Брейндед госуправления

\iusr{Юрий Лукашин}

Нужно Штатам, как граду демократии на холме, подсказать: че уж там
церемониться, пусть вводят сразу санкции против миллионов граждан Украины,
которые посмели голосовать за ОПЗЖ. А может и саму партию сразу запретить.
Можно даже на половину страны ввести санкции, которая все годы здесь голосовала
против их прозападных партий и политиков, чем как бы мешала продвижению
интересов США в регионе. А вообще, если задуматься, то к таким \enquote{демократическим
санкциям} против политиков других стран, по-моему, даже при Сталине не
додумались бы. Вообще, многие бы самые махровые диктаторы постеснялись бы так
беспардонно лезть во внутренние дела других стран. На самом деле, для любой
страны мира такие действия были бы позорищем, а не внешней политикой. Видимо,
утрачены там окончательно даже базовые навыки грамотной политики, вообще
каких-либо правил дипломатии. Как будто школьники необразованные и
закомплексованные у руля страны засели. Всех оппонентов накажем и покараем.
Демократы, че))))

\iusr{Семён Александров}

Великий гегемонишко в порыве бессильной злобы, решил наср@ть под дверь


\end{itemize} % }
