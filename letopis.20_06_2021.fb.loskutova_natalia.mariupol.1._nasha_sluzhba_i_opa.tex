%%beginhead 
 
%%file 20_06_2021.fb.loskutova_natalia.mariupol.1._nasha_sluzhba_i_opa
%%parent 20_06_2021
 
%%url https://www.facebook.com/permalink.php?story_fbid=pfbid02dXmDFhTtDcDf6JcCsWa5zZMbf4p49e15ipVekh5scT1C3icUetAUPbcrnm1ad4jQl&id=1427894275
 
%%author_id loskutova_natalia.mariupol
%%date 20_06_2021
 
%%tags mariupol,perevod.language,francia
%%title "Наша служба и опасна и трудна" або один день з небуденного життя перекладача
 
%%endhead 

\subsection{\enquote{Наша служба и опасна и трудна} або один день з небуденного життя перекладача}
\label{sec:20_06_2021.fb.loskutova_natalia.mariupol.1._nasha_sluzhba_i_opa}

\Purl{https://www.facebook.com/permalink.php?story_fbid=pfbid02dXmDFhTtDcDf6JcCsWa5zZMbf4p49e15ipVekh5scT1C3icUetAUPbcrnm1ad4jQl&id=1427894275}
\ifcmt
 author_begin
   author_id loskutova_natalia.mariupol
 author_end
\fi

\enquote{Наша служба и опасна и трудна} або один день з небуденного життя перекладача.

Ніколи в житті не могла навіть подумати, що я колись опинюся у Широкіно, у
цьому чудовому селищі з ласкавим Азовським морем і дивовижними краєвидами. У
цьому місці, яке стало болючою кривавою раною на тлі України.... 

У минулу середу мені зателефонувала представниця Посольства Франції в України
та звернулася з проханням супроводжувати французьку делегацію під час візиту в
Маріуполь як перекладачка. Звісно, я погодилася: хто ж відмовиться від такої
пропозиції та від можливості поспілкуватися мелодійною французькою мовою й
вкотре перевірити свої професійні навички!!! Хоча серце калатало і колінки
дрижали аж до самої зустрічі з французами))) 

Отож, до Маріуполя завітав депутат Парижу, колишній радник Емануеля Макрона під
час його передвиборчої кампанії, член Комісії з економічних питань Пьер Персон
та підприємець Едвард Мейєр. Метою візиту високих гостей було ознайомлення з
містом, з'ясування ситуації на кордоні з тимчасово окупованими українськими
територіями, а також зустріч з мером міста та обговорення ініціатив розвитку
Маріуполя. 

Спочатку ми попрямували до КПП Гнутове, де представники силових структур
докладно описали ситуацію на прикордонній території та роз'яснили особливості
функціонування КПП. Наразі КПП діє лише в одну сторону, а краще сказати зовсім
не діє, оскільки українська сторона згодна пропускати на територію України та
випускати своїх громадян, але інша сторона нікого не пропускає і не випускає...
Отака свобода вибору... Було моторошно бачити на під'їзді до КПП таблички, які
попереджали про неможливість зупинки на узбіччі, оскільки це несе загрозу для
життя: все заміновано... Уявіть, тиша, зелене розмаїття, степовий запах трав,
річечка і міни... Смерть... 

А потім ми попрямували до Широкіно... На БТРі, бо звичайний та звичний нам
транспорт там не їздить. Хоча якихось сім років тому (а я востаннє була у цьому
курортному містечку саме на початку літа 2014) траса аж гула від приміських
автобусів та приватних автівок: 20 км. від міста - і ти на чистенькому пляжі
дихаєш не смогом і не смородом заводів, а солоним морським повітрям,
насолоджуєшся спекотним азовським сонцем, слухаєш шелест прибою та крики чайок.
Прибій, чайки, море та сонце нікуди не ділися: чудова днина, прозоре повітря,
білі полохливі птахи та абсолютна, гнітюча, вбиваюча тиша...Де сміх дітей?
Зойкіт тих, хто вперше пірнає у воду? Крики торговок: Пахлава медовая!!!????
Безжальна і смертоносна тиша. Руїни від колись квітучого поселення. Вибиті
шибки, розтрощені двері, уламки стін, дома без дахів ніби роззявили рота до
неба та благають: \enquote{Зупиніться!!! Досить!!!} Боляче, перехоплює подих, серце
стікає кривавими сльозами. А я ж лише випадкова туристка... А що ж відчувають
ті, хто залишив там все? Хто не має змоги повернутися на рідне подвір'я? Хто
ось уже сім довгих років не відвідує могили близьких?... Отож, ми проїхали повз
зруйноване вщент селище та дісталися до штабу. Військові на макеті наочно
продемонстрували, де ми, де супротивник, звідки і як ведуться обстріли, яку
зброю при цьому застосовують. Я слухала, перекладала, а самій зовсім не
вірилося, що ось тут війна, озброєні до зубів солдати, міни та руїни, а в
якихось 20 км. від цього страшного місця - спокійний та квітучий Маріуполь...
Потім гостям продемонстрували фото місцевих будинків 2013 року та поруч - свіжі
світлини 2021 року. Жахливе враження: колись і зараз, мир і війна, життя і
смерть... І наостанок - ще одна екскурсія по Широкіно, наразі вже піша. Нас
попередили, щоб ми йшли лише по тротуару, ні в якому разі не заходили на
узбіччя, ні на що не наступали, а у разі обстрілу - рятувалися у БТР, який їхав
за нами. Таке собі, скажу вам, напуття під час перекладу. Тут думки: як же
правильно перекласти всі ті калібри, типи вибухівки, дрони, сховища... А тут ще
й зважати, куди ногу поставити... Нас, звичайно, супроводжували військові і я
впевнена, що у разі небезпеки, вони б намагалися нас захистити, але я зовсім
була не готова до обстрілу, вибухів і т.д. Дякувати Богу, все минулося. Ми
пройшли зруйнованою вулицею, вийшли на покинутий усіма, порослий травою та вже
котрий рік недоглянутий пляж. Звісно, гості були шоковані... І під час
коротенького інтерв'ю військовому журналісту пан Персон висловив своє
захоплення мужністю українських військових, пообіцяв персонально доповісти про
все побачене Президенту Франції та запевнив про всіляку підтримку України з
боку Французької Республіки. 

Далі ми попрямували до виїзду з Широкіно, проїхали по всім вирвам від
обстрілів, нас сильно покидало по БТР і нарешті - мирне шосе і ми на безпечній
території. Ви навіть не уявляєте, з якою насолодою я скидала важелезний
12-килограмовий бронежилет та каску. І як приємно було опинитися у зручній, а
найголовніше, мирній та звичній автівці. Ми попрямували до Маріуполя і я зовсім
іншими очима подивилася на його затишні вулиці, не спаплюжені війною. Хоча
нашому місту все ще болять і події 2014 року, і страшний обстріл у січні
2015... Але ми це подолали, пережили, встали з колін і стали "вітриною"
Донбасу, містом сильних людей та безмежних можливостей. 

Далі все було більш-менш прозаїчно і звично для мене: візит до ресторану,
зустріч з мером міста Маріуполя Вадимом Бойченко, під час якої мер докладно
проінформував про набутки міста, розповів про спільні проєкти з Французькою
Республікою (реконструкція маріупольських пляжів та будівництво водоочисного
заводу), та поділився ідеями реалізації проєктів, які спрямовані на розвиток
міста, а саме: будівництво аеропорту, будівництво нової об’їзної бетонної
дороги, покращення залізничного сполучення між столицею та Маріуполем.
Французькі гості відзначили динамізм розвитку Маріуполя та пообіцяли всіляко
допомагати реалізації цих проєктів. Візит вийшов дуже насиченим та емоційним,
гості були і збентежені від побаченого у Широкіно, і задоволені від разючого
контрасту мирного та квітучого Маріуполя. 

А я познайомилася з цікавими людьми, набула нового досвіду та вивчила багато
нової лексики. А ще я дотепер приходжу у себе від пережитих почуттів, від
болючих вражень, від \enquote{руського миру} та це більше ціную своє мирне життя у
найкращому місті на узбережжі найтеплішого моря)))
