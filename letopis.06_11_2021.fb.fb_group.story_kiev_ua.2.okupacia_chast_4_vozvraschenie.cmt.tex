% vim: keymap=russian-jcukenwin
%%beginhead 
 
%%file 06_11_2021.fb.fb_group.story_kiev_ua.2.okupacia_chast_4_vozvraschenie.cmt
%%parent 06_11_2021.fb.fb_group.story_kiev_ua.2.okupacia_chast_4_vozvraschenie
 
%%url 
 
%%author_id 
%%date 
 
%%tags 
%%title 
 
%%endhead 
\zzSecCmt

\begin{itemize} % {
\iusr{Тамара Ар}
Сколько наших солдат погибало, зажав в руке фото любимой женщины,,,,,,,

\iusr{Тома Храповицкая}

Такие тяжёлые воспоминания... Столько пережить... Ваши воспоминания бесценны,
не беда что Вы многое не запомнили из разговоров родных, Главное что Вы описали
и поделились тем, что Ваша Память запомнила и сохранила... Спасибо Вам Yuriy...


\iusr{Регина Кучеренко}
Боже мой, какое счастье, что мои родители пережили войну и остались живы.

\iusr{Юрий Петров}
как у немцев оказался американский мотоцикл индиан?

\iusr{Юрий Карпенко}
Трофей. Они поставлялись в РККА

\iusr{Юрий Петров}
тем более коляска от м 72

\iusr{Елена Полякова}
Очень интересные воспоминания, прочитала с большим интересом. Спасибо

\iusr{Наталия Озерова}
Сколько горя принесла война людям! Как смогли пережить такое? Наш бедный Город! Спи спокойно!

\iusr{Tatiana Thoene}

Ваши воспоминания - бесценны, и читать интересно и профессионально,
художественно написаны. Я себе ярко представила семью моей мамы, и как они
пережили оккупацию.

\iusr{Sergey Rabchuk}
Каких это вы красно!!!армейцев упоминаете?
Вам же сказали сверху - антигитлеровская коалиция!
А Красная Армия - харам !

\iusr{Микола М. Демянов}

Да, жахливо все це навіть читати. Ще в 60-ті ми бачили на вулицях багато
покалічених війною чоловіків з культяпками, колодками, з костурами, на
інвалідських візках. Страшна річ війна.

\iusr{Таня Гур}

\ifcmt
  ig https://scontent-frx5-2.xx.fbcdn.net/v/t39.1997-6/s168x128/93118771_222645645734606_1705715084438798336_n.png?_nc_cat=1&ccb=1-5&_nc_sid=ac3552&_nc_ohc=2JYZKuDGW-oAX_CtvbU&tn=lCYVFeHcTIAFcAzi&_nc_ht=scontent-frx5-2.xx&oh=00_AT_TLMr5fyBeKG9OSgUdJso2TUPhNDfeul8reHrEchodQQ&oe=61CCAEBF
  @width 0.1
\fi

\iusr{Людмила Билык}

Благодарю за воспоминания... Ах, война, что ты подлая сделала?...
@igg{fbicon.face.pensive}  @igg{fbicon.face.sad.but.relieved} 

\iusr{Елена Ткаченко}
Очень интересно. Спасибо.

\iusr{Dmitrii Pisanenko}

Сочувствую, точно такое пережил и я, когда мы добирались из села Требухово в
Киев после освобождения. Я даже нес рюкзачок с листьями табака и периодически
отец нес и меня и этот табак. Когда мы пришли к родственникам на Подоле на
улицу Ярославскую, я 2 дня лежал на кровати и держал ноги на подушке. Такие
дела...

\iusr{Alisa Brekhova}
Прочитала с интересом. Большое спасибо

\iusr{Ковальская Татьяна}

Наша семья тоже вернулась в свою квартиру, из которой нас выделили и почти 3
года мы жили на Сталинке, на кагатах, наш дом по Владимирской 39, угловой,
чудом не сгорел, квартира была разграблена, но мы были счастливы, соседей в
живых не было, ушли в Бабий яр, грустно, царство небесное!

\iusr{Исаев Георгий}
Очень интересно, спасибо  @igg{fbicon.wink} 

\iusr{Tamara Pankratova}
З великою увагою і інтересом прочитала Ваші спогади. Дякую.

\end{itemize} % }
