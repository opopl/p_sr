% vim: keymap=russian-jcukenwin
%%beginhead 
 
%%file 16_02_2022.stz.news.ua.strana.3.putin_ne_napal_infvojna.4.zagnat_ukrainu_pod_kolpak
%%parent 16_02_2022.stz.news.ua.strana.3.putin_ne_napal_infvojna
 
%%url 
 
%%author_id 
%%date 
 
%%tags 
%%title 
 
%%endhead 

\subsubsection{4. Загнать Украину под колпак}
\label{sec:16_02_2022.stz.news.ua.strana.3.putin_ne_napal_infvojna.4.zagnat_ukrainu_pod_kolpak}

Наконец есть еще одна версия почему в США и Британии раскручивают истерию
вокруг \enquote{войны с Россией}.

И она касается конкретно Украины.

Как известно, после Майдана американцы очень плотно влияют на внутренние
процессы в нашей стране.

Однако до сих пор этот контроль не является тотальным. Попытки выстроить
полностью управляемую систему влияния через антикоррупционные органы и суды
наталкивается на скрытое сопротивление украинской власти и элиты (как во
времена Порошенко, так и во времена Зеленского). Украинские власти хотят и
далее жить по принципу \enquote{Техас должны грабить техасцы}. Классический пример -
затягивание назначения директора САП.

Тем более, что благоприятная конъюнктура экспорта позволяла Украине снизить
зависимость от внешних кредитов. 

Эта ситуация не очень нравилась \enquote{внешним управителям}. Поэтому истерия вокруг
\enquote{вторжения Путина} имеет еще одну вполне прикладную цель - сделать украинские
власти еще более послушными к требованиям Вашингтона. 

Паника уже привела к тому, что обрушился украинский рынок евроооблигаций и наша
страна уже не может занимать деньги на внешнем рынке кроме как у правительств
других государств или же под их гарантии. 

Кроме того, ограничения, которые ввели страховые компании на авиабилеты и
морские перевозки в Украину наносят сильный удар по экономике. 

А учитывая, что Запад обороты по нагнетанию истерии сбавлять не намерен,
ситуация будет только ухудшаться.

Поэтому можно констатировать, что чем бы не закончились глобальные торги Запада
и России главной проигравшей в них уже сейчас является украинская экономика.

В такой ситуации украинским властям было бы правильно приходить максимум
усилий, чтоб сбить напряжение. В том числе - заявить о внеблоковом статусе (то
есть об отказе от вступления НАТО, тем более, что альянс уже многократно
заявил, что не намерен воевать за Украину и не собирается нашу страну принимать
в свои ряды в ближайшее время). А также пойти на выполнение Минских соглашений
для мирной реинтеграции Донбасса.

Однако пока трудно представить, чтоб украинские власти пошли бы по этому пути.
Пока они идут совсем в ином направлении. 


