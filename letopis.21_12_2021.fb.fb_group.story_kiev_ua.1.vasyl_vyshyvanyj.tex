% vim: keymap=russian-jcukenwin
%%beginhead 
 
%%file 21_12_2021.fb.fb_group.story_kiev_ua.1.vasyl_vyshyvanyj
%%parent 21_12_2021
 
%%url https://www.facebook.com/groups/story.kiev.ua/posts/1823485377848255
 
%%author_id fb_group.story_kiev_ua,poljakova_galina.kiev
%%date 
 
%%tags istoria,kiev,vyshyvanyj_vasyl
%%title Вильгельм Габсбург (Василь Вышиваный)
 
%%endhead 
 
\subsection{Вильгельм Габсбург (Василь Вышиваный)}
\label{sec:21_12_2021.fb.fb_group.story_kiev_ua.1.vasyl_vyshyvanyj}
 
\Purl{https://www.facebook.com/groups/story.kiev.ua/posts/1823485377848255}
\ifcmt
 author_begin
   author_id fb_group.story_kiev_ua,poljakova_galina.kiev
 author_end
\fi

Спасибо всем, прочитавшим мою миниатюру. Спасибо за комментарии. Комментарии
очень разны, они полярны. Вот еще одна миниатюра. Этот человек в Киеве не жил.
Он в Киеве умер. Мир его праху. Особо его не ругайте, как Игнатьева. Возможно,
что и у него были свои недостатки. Но я пишу только об очень симпатичных людях. 

\ii{21_12_2021.fb.fb_group.story_kiev_ua.1.vasyl_vyshyvanyj.pic.1}

Эрцгерцог Вильгельм Франц Йозеф Карл фон Габсбург\hyp Лотарингский. Нужно ли
перечислять его славных предков и знатных родственников? Кстати, такое
происхождение далеко не всегда обеспечивало беззаботное существование. Мама
юного эрцгерцога была итальянкой, поэтому дети сначала заговорили
по-итальянски. Но папа-то был австрийцем, поэтому детишек учили и немецкому, а
для порядка еще английскому и французскому языкам. Выйдя в отставку, отец
обосновался в одном из имений в Польше. Понятно, что мальчик выучил и польский
язык. Зачем такие подробности? Ребенок не просто говорил на этих языках, он
изучал культуру этих народов. 

В 10-летнем возрасте Вильгельм поступил в Венское реальное училище, а затем
окончил Терезианскую военную академия. Воспитанный в спартанском духе, легко
овладевал военной наукой. В качестве факультативного курса освоил украинский
язык. Вот это новое знание изменило его жизнь. 

В феврале 1915 года Вильгельм Габсбург получил звания лейтенанта австрийской
армии и был направлен в 13-й уланский полк, квартировавший в Золочеве подо
Львовом. Австрийский эрцгерцог внезапно влюбился в Украину. Еще в детстве он
много путешествовал с братьями и отцом. Они объездили всю Европу, побывали в
Африке, Азии, Америке. Но именно Украина стала его судьбой. 

В полку его прозвали Василем Вышиваным, вероятно за то, что он носил вышиванку,
подаренную ему украинскими солдатами. Прозвище ему понравилось, и он стал
подписывать так свои стихи. Кстати, стихи он писал по-украински. 

По закону Австро-Венгрии каждый член императорской семьи по достижении 21 года
автоматически становился членом сената, поэтому в 1916 году Вильгельм Габсбург
вошел в состав парламента. Понятно, почему с тех пор правительство стало
обращать больше внимания украинскому вопросу. Молодой эрцгерцог окунулся в
дипломатическую работу, наладил отношения с УНР и настолько мощно поддерживал
Украину в эти очень бурные годы, что правительство вынуждено было его отозвать
в Вену. 

В соответствии с условиями Брестского мирного договора отряды Австро-Венгерских
войск были передислоцированы в Украину. Молодой Габсбург- Вышиваный оказался в
своей стихии и 1 апреля 1918 года принял командование корпусом УСС – Украинских
Сичовых Стрельцов. А в декабре того же года получил звание полковника армии
УНР. Поступали многочисленные предложения принять титул гетьмана и даже короля,
которые он решительно отклонял. При этом яростно защищал интересы Украины и
разругался с Петлюрой из-за Варшавского договора 1920 года, по которому к
Польше отходили западноукраинские земли, включая Галичину. Возмущенный
Вильгельм уехал в Вену и оттуда вел борьбу за целостность Украины, чем
чрезвычайно разгневал отца, которого прочили на польский трон.

Теперь в Украину ему путь был закрыт. Решительно отказавшись от предложенных
булавы и короны, активно интересовался событиями в Украине и поддерживал
украинские эмигрантские круги. Занялся собственным бизнесом, жил
преимущественно во Франции, Испании, в Вене. Женился. Родились двое сыновей. Во
время Второй мировой войны, как и все Габсбурги, отказался от сотрудничества с
нацистами и находился под надзором Гестапо. И не зря. Поскольку сотрудничал с
британской разведкой.  

Однако была еще одна спецслужба, которая не спускала с него глаз – советская.
26 августа 1947 Вильгельма Габсбурга арестовал СМЕРШ. Ему инкриминировали
шпионаж в пользу западных держав, кстати, союзников СССР по антигитлеровской
коалиции, и связь с ОУН.

В конце ноября 1947 Вильгельма Габсбурга перевезли в Лукьяновскую тюрьму в
Киеве. В течение полугода его допрашивали ежедневно, вернее еженощно. Эрцгерцог
отрицал свое участие в каких-либо заговорах, отказался оговаривать соратников
по эмиграции. Но все же все вместе сулило 25 лет лагерей. Но условия содержания
в Лукьяновке и методы допросов привели к тому, что в ночь на 18 августа 1948
года Вильгельм Габсбург скончался в тюремной больнице. Место захоронения
эрцгерцога неизвестно: где-то возле ограды Лукьяновского кладбища.   

Его называли украинским королем, аристократом, романтиком, мечтателем,
дипломатом, военным, шпионом, бизнесменом. Кем же он был? Наверное, все
приведенные определения вполне соответствуют этой исключительной личности. А
если коротко? Наверное, скажем, что Вильгельм Габсбург был офицером и поэтом.

\ii{21_12_2021.fb.fb_group.story_kiev_ua.1.vasyl_vyshyvanyj.cmt}
