% vim: keymap=russian-jcukenwin
%%beginhead 
 
%%file 19_01_2023.fb.korovanenkova_natalia.mariupol.1.19_marta_2022___mors.cmt
%%parent 19_01_2023.fb.korovanenkova_natalia.mariupol.1.19_marta_2022___mors
 
%%url 
 
%%author_id 
%%date 
 
%%tags 
%%title 
 
%%endhead 

\qqSecCmt

\iusr{Евгения Павлюк}

Наташа, этот ужас никогда из памяти не уйдет:(

\begin{itemize} % {
\iusr{Natali Korovanenkova}
\textbf{Евгения Павлюк} 

я каждый день наблюдала в окно, пока оно у меня было, как на обочину дороги,
проспекта, привозят на санках, приносят тела и оставляют..

\iusr{Евгения Павлюк}
\textbf{Натали Корованенкова} вообще:(((

\iusr{Nadezhda Gorchakova}
\textbf{Евгения Павлюк} 

те, кто пережил не забудут этого никогда, остальные забудут и очень быстро (((
Максимально распростронять и всем рассказывать, только так

\iusr{Larisa Sinchuk}
\textbf{Nadezhda Gorchakova}, не забудут.

\end{itemize} % }

\iusr{Natali Korovanenkova}

Когда стреляли еще \enquote{не очень} сильно, то есть интенсивные обстрелы были в других
районах, у нас в основном самолет и арта, и мы еще могли выходить готовить
возле подьезда на кострах, при крике \enquote{самолет} успевали вбегать в поьезд и
прятаться за стены.. я сейчас понимаю, какие мы были не досвідченні((.. нам
просто вовезло ..нас би ничто не спасло за теми стенами..

Когда стало более интенсивно, но еще можно было выходить к костру, выходили
парами, соседка и мой сын, я и супруг соседки оставались в укрытии, если вдруг
с ними что то случится, мы смогли !смогли? им помочь, Потом менялись, выходили
мы, а они оставались..

\begin{itemize} % {
\iusr{Katerina Rubtsova}
\textbf{Натали Корованенкова} Наталья мурашки по коже ..когда читаю это
Ужас ,что вы пережили

\iusr{Natali Korovanenkova}
\textbf{Katerina Rubtsova} 

незнаю.. это только \enquote{вершина айсберга}(, мариупольцы многие молчат или як це
було между собой рассказывают, потому что действительно понять может только
тот, кто пережил, не дай боже комусь це знати((

\end{itemize} % }

\iusr{Natali Korovanenkova}

Накрыло сейчас мариупольцев с новой силой боль Днепра, потому что мы все это
пережили и переживали каждый день и кажый час..

Я разговаривала эти дни с мариупольцами, соседями, друзьями, кто остался в
Мариуполе .. с новой силой переживают все то что происходило, боль, открытая
рана... перед глазами руины города... помощи нет .т и ш и н а.

\begin{itemize} % {
\iusr{Katerina Rubtsova}
\textbf{Натали Корованенкова} по нашим каналам, покалывают, как они счастливы 🤦♀️

\iusr{Olga Pidgaetska}
\textbf{Natali Korovanenkova} 

я впм скажу... накрило всіх.... хто має душу. Я з міста яке ттт тільки раз зазнало
такого удару... і скажу, що Маріуполь болів і болить, що Дніпр.... що
Бровари.... Мені іноді страшно дивитись на ваші фотоспогади. Тому шо страшно
уявити то через шо проходили ви і мешканці....

\iusr{Natali Korovanenkova}
\textbf{Katerina Rubtsova} 

возможно, но тогда где логика, если Мариуполь звал, ждал, зачем убивать тех кто
тебя ждет? Если в Мариуполе в каждом доме бендеры, то тогда, оставшиеся в
живых, не могут быть рады.. сейчас в Мариуполе много понаевших, гораздо
больше, чем местных осталось

\end{itemize} % }

\iusr{Iren Iren}

100 процентов, новая боль!

\iusr{Viktoriia Stavriienko}

Морской, 42. Это рядом со мной. На их девятиэтажке сидел снайпер. Сначала
\enquote{Азова}, потом русский. Это был угловой дом, удобное место для обстрела. Вот
туда и летело постоянно со всех сторон..

\begin{itemize} % {
\iusr{Natali Korovanenkova}
\textbf{Viktoriia Stavriienko} на нашей то же сидел снайпер и корректировщики, и нам летело((
\end{itemize} % }

\iusr{Natali Korovanenkova}

Наш район, 5й и до поворота, занят был самый последний, когда зетки перешли
мост, по Металлургов и вошли через наш двор, расположившись во дворе школы 28,
город был полность аккупирован,со стороны Искры, поворот, со стороны Тополиной, и
со стороны М. Мазая( шлакова, гаражи), где шел танковый бой дня три, уже были
zетки, так что мы оставались \enquote{на закуску}(((

Это было 27-28марта, перестали бомбить самолеты, утихла арта, кое где слышались
одиночные выстрелы ..мост разминировали и можно было пойти на речку набрать
воды

На мосту блок пост, днр, докуметы проверяли, когда ты шел за водой и обратно..

\iusr{Natalya Airin}
\textbf{Натали Корованенкова} 

\enquote{моряки} и Черёмушки ещё позже, мы 31 марта оттуда уехали, потому как горело уже
всё вокруг и с половины второго ночи сидели в подвале и ждали рассвета, чтобы
собраться и выехать

\begin{itemize} % {
\iusr{Natali Korovanenkova}
\textbf{Наталья Айрин} 

да, там горело и догоралоно, 30 я уже была в ценре, на тысячку можно было только
со стороны обжоры, а к дому матери на \enquote{зеркальный} меня не пустили, там шел бой,
и были снайперы, но по городу уже везде были зетки, а 31 я выеха через Кирова, на
Тополиную и на Метро..

\iusr{Елена Корнеева}
\textbf{Натали Корованенкова} 

а мы с Черёмушек вышли 22 марта, в нашем квадрате несколько дней не было
обстрелов, нас только с самолёта бомбили, когда утром вышли, встретили наших на
машине, у них ещё пушка была, это были остатки нашей армии там ( на въезде в
Мелекино уже было днр

\iusr{Natali Korovanenkova}
\textbf{Елена Корнеева} в том то и дело, город защищать было нечем((

\iusr{Елена Корнеева}
\textbf{Натали Корованенкова} нечем, автоматы были, пару танков на въезде в город, сражайтесь лопатами

\iusr{Елена Корнеева}
\textbf{Натали Корованенкова} 

ну и главное, что наши люди свято верят, что это именно наши все разрушили, чем
интересно? Самолётами, кораблями, градамии, смерчами ? У нас и не было
ничего, вот и получилось как получилось, убили и мирных и военных

\end{itemize} % }

\iusr{Тая Аксенова}

😢

\iusr{Natali Korovanenkova}

12 марта самолет разбомбил 66 школу, мы были рядом, получали
гуманитарку.. красный крест выдавал ее, пока бомбили школу мы прятались в каком
то разбитом магазине за такими же разбитыми прилавками....

в то время шел бой возле института, где то в час дня мы вернулись домой с
гуманитаркой, нас довезли военные на машине или вернее на том, что от нее
осталось и она каким то чудом двигалась, а в 16. 00, там где мы получали гум, был
днр и снайпера зачищали местность, дома дворы, дет садик..

\begin{itemize} % {
\iusr{Nadezhda Gorchakova}
\textbf{Natali Korovanenkova} Який жах, читаю і знову рана болить, неначе інша реальність(((
\end{itemize} % }
