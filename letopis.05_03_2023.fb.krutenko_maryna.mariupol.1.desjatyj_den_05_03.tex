%%beginhead 
 
%%file 05_03_2023.fb.krutenko_maryna.mariupol.1.desjatyj_den_05_03
%%parent 05_03_2023
 
%%url https://www.facebook.com/marinakrytenko/posts/pfbid02AzvpHAA3kqPo87x1Xdsb43GdYUG39BG57FHdNP5NWTBSHSH7Eja6R5LecKM9pULZl
 
%%author_id krutenko_maryna.mariupol
%%date 05_03_2023
 
%%tags 05.03.2022,dnevnik,mariupol,mariupol.war
%%title ДЕСЯТЫЙ ДЕНЬ ВОЙНЫ 05.03.22
 
%%endhead 

\subsection{ДЕСЯТЫЙ ДЕНЬ ВОЙНЫ 05.03.22}
\label{sec:05_03_2023.fb.krutenko_maryna.mariupol.1.desjatyj_den_05_03}

\Purl{https://www.facebook.com/marinakrytenko/posts/pfbid02AzvpHAA3kqPo87x1Xdsb43GdYUG39BG57FHdNP5NWTBSHSH7Eja6R5LecKM9pULZl}
\ifcmt
 author_begin
   author_id krutenko_maryna.mariupol
 author_end
\fi

ДЕСЯТЫЙ ДЕНЬ ВОЙНЫ 05.03.22

Мы остались дома, потому что, ЗА НОЧЬ ВЫПАЛ СНЕГ, утром началась оттепель и мы
просто подставляли ведра под сливную трубу крыши. 

Люди вышли на улицу с ведрами и лопатами и набирали воду из луж. 

Так мы радовались снегу! Снег- это вода. Так же мы узнали, что деньги - это
бумага, за которую уже ничего нельзя купить. Телефон - это пластмасс, с
которого сквозь обстрелы нужно найти место где ловит связь и позвонить родным и
сказать, что мы ещё живы. Телефоны включались перед тем как собирались
позвонить и после, сразу же выключались, экономили батарею. 

На момент когда выключили свет, у нас были заряжены все устройства телефоны,
компьютеры и повербанк на 100\% 

Мы занялись домашними делами. Вытянули из морозильника мясо и все что там
хранилось и вынесли в коробке на улицу. Хорошо, что ночами был мороз.

По радио мы узнали о переговорах Украины и России о \enquote{зелёном коридоре}
(гуманитарном коридоре). FM волны вещали только радио ДНР, Украинское радио
могли слушать только на волнах AM, и то радио глушили. Прежде мы находили
новости Украины, нам приходилось выслушивается в язык государства которого мы
слушали. Китай, Италия.... и все они говорили об Мариуполе.

Женя сказал, чтоб я собрала вещи и документы. Я ответила, что если хотя бы
Никита не поедет, я не поеду и пусть уговаривает Никиту. Никита вроде
согласился, вещи у него с первого дня были сложены в сумку. 

Женя и Сергей поехали на бывшее предприятие,  им дали распоряжения забрать
аккумуляторные батарейки для фонариков, мясо которое хранилось в морозильнике и
упакованные комплекты термобелье. 

На машине мы \enquote{написали} строительным скотчем \enquote{ДЕТИ} с одной и другой стороны. На
зеркала навязали белые тряпочки. 

Легли спать в ожидании чуда. Выезжать при первой возможности должны были: я за
рулем, Никита, Ольга и Алина. 

Продолжение следует....
