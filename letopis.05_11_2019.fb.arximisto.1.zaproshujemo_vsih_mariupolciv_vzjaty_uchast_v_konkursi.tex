%%beginhead 
 
%%file 05_11_2019.fb.arximisto.1.zaproshujemo_vsih_mariupolciv_vzjaty_uchast_v_konkursi
%%parent 05_11_2019
 
%%url https://www.facebook.com/arximisto/posts/pfbid02u6cvJtJWqMN3hdnAngMovu7AXdMqiW4gyHQZEhYtwAV6QJoowtMtXPjRsuB2dYYhl
 
%%author_id arximisto
%%date 05_11_2019
 
%%tags 
%%title Запрошуємо всіх маріупольців взяти участь в конкурсі!
 
%%endhead 

\subsection{Запрошуємо всіх маріупольців взяти участь в конкурсі!}
\label{sec:05_11_2019.fb.arximisto.1.zaproshujemo_vsih_mariupolciv_vzjaty_uchast_v_konkursi}

\Purl{https://www.facebook.com/arximisto/posts/pfbid02u6cvJtJWqMN3hdnAngMovu7AXdMqiW4gyHQZEhYtwAV6QJoowtMtXPjRsuB2dYYhl}
\ifcmt
 author_begin
   author_id arximisto
 author_end
\fi

Запрошуємо всіх маріупольців взяти участь в конкурсі! 🙂

====================

УВАГА❗️

міський конкурс фотографій серед учнів закладів освіти
\enquote{Повернення Маріуполю його імені – очима маріупольців} ⚓️💕

\ii{05_11_2019.fb.arximisto.1.zaproshujemo_vsih_mariupolciv_vzjaty_uchast_v_konkursi.pic.1}

До участі запрошуються:

✅ Учні закладів освіти Маріуполя незалежно від їхнього типу, форми власності та підпорядкування.

✅Вчителі історії закладів освіти Маріуполя незалежно від їхнього типу, форми власності та підпорядкування

Предметом конкурсу є фотографії з сімейних архівів маріупольців щодо двох тем:

1) Різноманітна громадянська діяльність маріупольців у 1985 – 1991 рр. (рухи
\enquote{За Маріуполь}, \enquote{За чистий Маріуполь}, Народний Рух, \enquote{За перебудову} тощо;
ініціативи з культурного відродження приазовських греків, українців, німців,
росіян, євреїв, інших етнічних груп міста; мистецьке об’єднання \enquote{Маріуполь-87};
організації \enquote{афганців} та ліквідаторів аварії на Чорнобильській АЕС; релігійне
відродження тощо)

2) Повсякденне життя маріупольців у другій половині 1985 – 1991 рр. (будь-які
життєві ситуації – народження, дитинство, робота, навчання, кохання, побут,
хобі, робота на дачі, подорожі, відпочинок, участь в мітингах, демонстраціях,
служба в армії, Афганська війна, ліквідація аварії на Чорнобільській АЕС,
різноманітна творчість, відвідування концертів та вистав, спорт тощо)


Подання \textbackslash~відсилка копій фотографій на конкурс

- Учні відшукують в сімейних фотоальбомах фотографії, які відповідають темам конкурсу, роблять їхні електронні копії (смартфоном, сканером, фотоапаратом, будь-яким іншим пристроєм) та відсилають зроблені електронні копії на 📧 mariupol1989@yahoo.com.

У листі з копією \textbackslash~ копіями фотографій повинна міститися наступна інформація:

o П.І.Б. учня\par
o Назва закладу середньої освіти та клас\par
o Контактний телефон учня (або батьків, опікунів)\par
o Контактний адрес електронної пошти учня (або батьків, опікунів)\par
o П.І.Б. вчителя історії (якщо учень займався \textbackslash~ -лася пошуком фотографій у шкільній групі під керівництвом вчителя історії)\par
o Опис фотографії: П.І.Б. автора (якщо відомий), власник \textbackslash~ -ця фотографії (якщо відрізняється від родини учня), зміст фотографії (хто та що саме зображено на фото), приблизна дата зйомки фотографії.\par

В темі листа бажано вказувати номер \textbackslash~ назву закладу загальної середньої освіти

Вітаються короткі відео коментарі учнів щодо того, чому вони обрали ту чи іншу
фотографію, хто та що на ній зображено тощо. Їх слід надсилати разом з листом з
копією фотографії, якої вони стосуються (\enquote{історія однієї фотографії}).

-  Один учень може надати на конкурс не більше 9 фотографій

-  Копії фотографій, які були подані без інформації про учня, не приймаються на конкурс

-  Вчителі історії організують пошукові групи з учнів своїх шкіл, розповідають
про значення громадської діяльності з повернення Маріуполю його історичного
імені, суспільне піднесення в місті та Україні у другій половині 1980-х – 1991
років, одночасні демократичні події у республіках колишнього СРСР,
соціалістичного блоку, падіння Берлінської Стіни у листопаді 1989 року тощо.

За потреби вони допомагають учням надсилати копії фотографій на конкурс.

Бажано, щоб вчителі історії повідомили оргкомітет про свою участь в конкурсі,
надіславши листа на 📧mariupol1989@yahoo.com.

- Приймаються лише копії фотографій, які зроблені з існуючих оригіналів (паперових, негативів).

- Приймаються копії як кольорових, так і чорно-білих фотографій

- Копії студійних фотографій або фотографій, зроблених професійними фотографами
(\enquote{на паспорт}, сімейні, дітей, весільні, фотопортрети тощо), не приймаються до
участі в конкурсі

- Копії фотографій пейзажів, будівель тощо приймаються, якщо загалом фотографії відображають певну життєву ситуацію маріупольців

- Якість копії має бути достатньою, щоб можна було побачити предмет \textbackslash~ предмети
фотографії Всі копії фотографій, прийнятих до участі в конкурсі, будуть
оприлюднюватися на спеціальній сторінці конкурсу та проекту, присвяченого
30-тій річниці повернення Маріуполю його імені, в мережі Facebook
\url{https://www.facebook.com/groups/mariupol1989/} (у разі наявності відео коментаря
учнів – разом з таким коментарем).

Школярі-переможці конкурсу отримають подарункові сертифікати на фотосесію від
маріупольських рекламно-інформаційного агентства Борис Дембицкий, фотостудії
Лилия Ровенская та Клубу любителей фотографии (керівник Клубу - Дмитрий Ганжеев
(Dmitry Ganzheev)).

Загальнонаціональна Книгарня \enquote{Є} та Видавництво \enquote{Темпора} нададуть призи для
вчителів історії \textbackslash~ вихователів закладів дошкільної та позашкільної освіти. 

Партнером конкурсу є Центр громадянського суспільства \enquote{Друкарня} (Друкарня / Drukarnia).

За згоди власників, \enquote{Архі-Місто} зробить якісні цифрові копії їхніх фотографій,
а Центральний державний кінофотофоноархів України ім. Г. Пшеничного візьме їх на
зберігання, з можливістю показу на загальнонаціональних виставках
(\url{https://tsdkffa.archives.gov.ua/}).

Початок конкурсу – 4 листопада 2019 р.

Кінцевий термін подання копій фотографій на конкурс – 27 листопада 2019 р.

Відкриття виставки в Центрі сучасного мистецтва ім. А. Куїнджі та нагородження переможців – 30 листопада 2019 р., субота.

Закриття виставки – 8 грудня 2019 р.

Детальний опис конкурсу міститься в
\href{https://archive.org/details/mariupol1989_competition_statute_final}{Положенні
про конкурс} \url{https://u.to/Qmu3Fg}. 

Контактна особа конкурсу:

Андрій Марусов, +38096 463 69 88

Андрей Марусов (Andrei Marusov)
