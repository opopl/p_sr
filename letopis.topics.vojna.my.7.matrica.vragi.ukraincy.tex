% vim: keymap=russian-jcukenwin
%%beginhead 
 
%%file topics.vojna.my.7.matrica.vragi.ukraincy
%%parent topics.vojna.my.7.matrica.vragi
 
%%url 
 
%%author_id 
%%date 
 
%%tags 
%%title 
 
%%endhead 

\paragraph{13:10:19 15-08-22, Любовь Адианова (Эпова)}

У вас даже в паспорте не стоит ,что ты украинец...
Ха. Ха ...
Нет ТАКОЙ нации...
Я же говорю...
Без Родины без флага
:-D :-D :-D :-D

\paragraph{14:32:45 15-08-22 Ирина Иванова}

Нет такой задницы, через которую евроукры что-то бы ни сделали ― это
утверждение, как ни одно другое, идеально подходит для современной Украины. Что
они ни делают ― не идут дела. В чём причина? Воздух неправильный?
Судьба-злодейка? Почему граждане великой и незалэжной ни от кого державы ни
сами не живут хорошо, ни другим не дают? Одна из теорий гласит, что всему виной
менталитет: та самая загадочная украинская ментальность, от которой украинцы
страдают уже много лет, и чем дальше, тем их страдания сильнее. И вот о десяти
наиболее ярких особенностях украинского менталитета сейчас и пойдёт
речь.Особенность № 1. ХатаскрайничествоЕсли о русских говорят, что в драке
не помогут, а в войне победят, то об украинцах можно сказать, что они ни в
драке не помогут друг другу, ни в войне. Просто потому, что
среднестатистический украинец, так уж исторически повелось, старается скрывать
свой внутренний мир, убеждения и мысли. Хотя бы просто потому, что это может
быть невыгодно или банально опасно. Быть принципиальным человеком в селе,
которое сегодня захватили поляки, завтра немцы, послезавтра белые, а затем
красные ― очень, знаете ли, вредно для здоровья. Поэтому и выработалась эта
ментальная установка ― ни во что не лезть дальше своего огорода. С одной
стороны, этих людей можно понять: вот подходит к такому человеку на улице
журналист и интересуется его политическими взглядами. На это часто следует
ответ: «Вы знаете, я вне политики». Так часто говорят публичные персоны,
которые не хотят портить свой бизнес политическими высказываниями. Но с другой
стороны, когда в одном месте собираются десятки миллионов таких вот
«хатаскрайников», то подобное становится настоящим проклятьем для общества:
общества равнодушных приспособленцев, аморфных людей без принципов. Территория,
населённая такой вот серой массой, обречена, что, собственно, мы и имеем
возможность наблюдать.

\paragraph{14:33:05 15-08-22 Ирина Иванова}

Особенность № 8. Обострённое достоинство. Украинский обыватель иногда может
начать корчить из себя эдакого вольнолюбивого индивидуалиста, потомка каких-то
там рыцарей или, как правило, казаков, которые все как на подбор были
благородными и смелыми. Настоящие бессребреники, честнее которых нет и не было
больше на земле. Умные люди давно это поняли, и теперь, когда им нужно в чём-то
убедить украинскую «спильноту», они всенепременно нажимают на это самое
«лыцарство», действуя на воспалённое чувство собственного достоинства рядового
евроукра, которое как воспалилось в 1991 году, так и не думает успокаиваться.
Как там говорил классик: «Не дай боже с холопа пана». Для холопа, которого
унижали поколениями польские феодалы, понятие чести и достоинства ― это
какие-то эфемерные и недосягаемые вещи. Но опять же, мы вспоминаем ещё одного
классика, утверждавшего, что украинцы не могут творить мелкого, поэтому всегда
хватаются за великое. В итоге и получается, что этот самый вчерашний украинский
холоп хватается за самое великое, чего у него никогда не было, ― за
достоинство, и при разговоре о достоинстве он снимет себя последние портки.
Именно поэтому умные дяди из спецслужб назвали последний украинский майдан
Революцией достоинства, и именно поэтому бараны, похожие на людей, у которых
берут интервью украинские телеканалы, на вопрос о позитивных плодах майдана
часто отвечают, что они обрели достоинство. И есть мнение, что где-то тут уже
заканчивается психология и начинается психиатрия.

\paragraph{14:33:31 15-08-22 Ирина Иванова}

Особенность № 9. Молодец против овец, а против молодца ― и сам овца. Парадокс,
но годы упоительного самовнушения насчёт собственного достоинства нисколько не
смогли добавить мужества украинскому обывателю, его общественным лидерам и
большим политикам. Современное украинство ― это трусость и страх понести
наказание. Поэтому не стоит удивляться такой бытовой шизофрении, когда сейчас
ты рыцарь, преисполненный достоинства, а через минуту ты уже стоишь на коленях
в грязи у дороги, по которой едет каток, укладывающий асфальт. Последнее ―
реальный случай из Львовской области: коленопреклоненная массовая благодарность
асфальтоукладчику (пускай и ироническая). Всё-таки столетия крепостничества
имеют большую силу в сознании обывателя, чем какое-то там достоинство. Отсюда и
хроническая трусость. Первый вал карателей, прибывших на Донбасс, в основном
состоял из жителей западных областей. Но, когда их сотнями начали укладывать в
землю, резко наступило просветление, и теперь в т. н. АТО (ООС) в основном
участвуют простые украинизированные русские, воюющие против таких же русских по
ту сторону фронта. Отважные галичане, мнящие себя нравственной элитой страны,
прячутся от призыва в армию по подвалам, в лесах и даже за границей. Потому что
страхов много, а жизнь одна. Именно поэтому, как уже было сказано ранее,
гитлеровцы и использовали бандеровцев только в карательных операциях против
безоружных. В настоящем бою они были бесполезны. 
