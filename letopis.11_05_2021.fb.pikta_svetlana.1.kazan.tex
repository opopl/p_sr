% vim: keymap=russian-jcukenwin
%%beginhead 
 
%%file 11_05_2021.fb.pikta_svetlana.1.kazan
%%parent 11_05_2021
 
%%url https://www.facebook.com/svetilkin.lanna/posts/4343610512338030
 
%%author 
%%author_id 
%%author_url 
 
%%tags 
%%title 
 
%%endhead 
\subsection{Жуткие кадры, как дети выпрыгивают с третьего этажа школы в Казани}
\Purl{https://www.facebook.com/svetilkin.lanna/posts/4343610512338030}

Жуткие кадры, как дети выпрыгивают с третьего этажа школы в Казани. Это как же
они напуганы!! Страх детей - то что невозможно себе простить, когда ты
взрослый. 

Господи, укажи нам путь, как нам воспитывать наших детей.

Сейчас век оскудения любви. Цинизма. На любовь не хватает энергии ни у кого.
Равнодушие друг ко другу. Погруженность в свои проблемы. 

Много закрытых детей, которые растут наедине со своим смартфоном. Институт
семьи разорён, но и в чудом сохранившихся семьях царит одиночество, чувство
оставленности. 

Стоит, наверно, хотя бы формально исполнять дела любви. Даже если не чувствуешь
ничего.

Есть подтверждённые случаи, когда в результате это помогало наполнить форму
содержанием. Мы так устроены: не можем терпеть расхождения между формой и
содержанием. Либо сойдём с ума, либо начнём чувствовать то, что делаем внешне. 

Надо рисковать.

Надо любить свою жену, даже если она уже "никуда не денется". Дарить ей цветы, хвалить. Это даст ей много энергии любить детей!

Писать о любви родителям, даже если есть тяжёлый многолетний конфликт. 

Устраивать детям праздники, даже если тебе никто никогда праздники не устраивал.

Нужно как-то преодолевать этот ад.

В Радоницу нет ни одного мертвого, они рядом с нами, они радуются воскресшему
Христу.

Надо понуждать себя к воскресению, надо становиться живыми, а это может только
Любовь. Она строга, участлива, горяча. Только не равнодушие!!  Тогда не будут
вот так погибать наши дети.

