% vim: keymap=russian-jcukenwin
%%beginhead 
 
%%file 18_11_2020.news.ua.strana.romanova_maria.1.draki_kislorod_kovid.vlasti
%%parent 18_11_2020.news.ua.strana.romanova_maria.1.draki_kislorod_kovid
 
%%url 
 
%%author 
%%author_id 
%%author_url 
 
%%tags 
%%title 
 
%%endhead 

\subsubsection{Что собираются делать власти?}
\label{sec:18_11_2020.news.ua.strana.romanova_maria.1.draki_kislorod_kovid.vlasti}

17 ноября проблематика - в частности, с кислородом - была поднята на уровне правительства и Минздрава. Премьер-министр Денис Шмыгаль отчитался о выделении средств на кислород для больных ковидом.

\enquote{Средства на кислород были выделены ориентировочно месяц назад, первые более
570 миллионов гривен. Сейчас было выделено еще более 870 тысяч гривен. Всего
практически почти более 1,5 миллиардов выделено на подведение кислорода}, -
рассказал глава правительства.

Далее глава правительства рассказал о планах власти. Больницам ставится задача
до конца года подключить к подаче кислорода не менее 80\% коек. В абсолютных
числах это около 50 тысяч койко-мест. 

Более подробно тему развила первый замминистра здравоохранения Ирина Садовьяк.
Из ее слов следует, что нехватка кислорода - вопрос не только денег, но в
первую очередь времени и человеческого ресурса. 

\enquote{Даже несмотря на выделенные средства на проведение магистралей, на закупку
кислородных концентраторов, есть ряд проблем, которые возникают на местах. Не
хватает работников, проводящих эти магистрали, не хватает компаний, которые
могут сделать техническую документацию. Я вас уверяю, что в каждом регионе, в
котором я бываю, нет равнодушных ни среди власти, ни среди врачей к тому, чтобы
этого не сделать. К сожалению, нужно время и большинство работ для того, чтобы
довести показатель до 80\%. Сегодня мы понимаем, что это будет не раньше
декабря}, - сказала Ирина Садовьяк. 

Отсюда можно сделать несколько выводов. 

\begin{itemize}
\item 1. Оформлять техническую документацию и прокладывать магистрали для
				подачи кислорода в больницы можно (и нужно) было летом. Однако в теплый
				период власть не стала усиленного готовиться ко второй волне ковида,
				которую сама же прогнозировала. Вместо этого деньги усиленно тратились
				на \enquote{Большое строительство}. 

\item 2. Выделение средств на увеличение койко-мест, оснащение их кислородом
				руководство страны также запустило лишь в октября  - после резкого
				всплеска заболеваемости в конце сентября. 
\end{itemize}

Почему летом власть усиленно не готовилась к осеннему пику пандемии? Этот
вопрос тысячи украинцев, которые заболевают ковидом сегодня, могут задать
власти. 

Но ответ - что в этот период власти тратили деньги из коронавирусного фонда на
дороги\Furl{https://strana.ua/articles/interview/301594-ihor-umanskij-objasnil-kak-kartel-iz-shesti-kompanij-razvoroval-denhi-iz-koronavirusnoho-fonda-na-bolshoj-strojke.html} - граждан вряд ли устроит. Поэтому уже сейчас понятно: руководство
страны до последнего будет скрывать коллапс медицинской системы и отрицать
нехватку кислорода на местах. 
