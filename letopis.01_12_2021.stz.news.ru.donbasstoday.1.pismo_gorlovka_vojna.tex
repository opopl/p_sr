% vim: keymap=russian-jcukenwin
%%beginhead 
 
%%file 01_12_2021.stz.news.ru.donbasstoday.1.pismo_gorlovka_vojna
%%parent 01_12_2021
 
%%url https://donbasstoday.ru/strashno-kogda-utrom-vyglyadyvaesh-iz-okna-a-vo-dvore-net-igrayushhih-detej-zhitelnicza-doneczka
 
%%author_id 
%%date 
 
%%tags 
%%title Страшно, когда утром выглядываешь из окна, а во дворе нет играющих детей – жительница Горловки
 
%%endhead 
\subsection{Страшно, когда утром выглядываешь из окна, а во дворе нет играющих детей – жительница Горловки}
\label{sec:01_12_2021.stz.news.ru.donbasstoday.1.pismo_gorlovka_vojna}

\Purl{https://donbasstoday.ru/strashno-kogda-utrom-vyglyadyvaesh-iz-okna-a-vo-dvore-net-igrayushhih-detej-zhitelnicza-doneczka}

Почти всю свою сознательную жизнь я осознавала себя полноправной гражданкой
Украины, наделённой всеми правами и свободами суверенного государства, жила с
непоколебимой верой в светлые идеалы равенства и братства всех без исключения.
Культуру своей страны я, несомненно, считала самой яркой и самобытной, язык –
самым мелодичным, искусство – неповторимым. Все эти мысли прививали мне с
самого детства и в школе, и дома. Я с почтением относилась к символам своего
государства и его законам. Я наивно полагала, что люди, живущие в моей стране,
наделены множеством положительных качеств. Я верила, что обидеть слабого,
поднять руку на ребёнка или старика было бы недостойно для моих
соотечественников.

Я знала, что если потребуется, люди моей страны не только встанут на защиту
своей семьи, но и смогут бороться и отдать жизнь ради будущего своей Родины.
Без сожаления, смело и гордо, как это было в годы Великой Отечественной войны,
когда все от мала до велика встали на защиту своего Отечества. Так сделал и мой
прадед, который ещё совсем молодым познал все ужасы войны: голод, холод,
нестерпимую боль, смерть друзей и однополчан. Он, как и весь народ в то ужасное
время, понимал, что отступать нельзя, что нужно идти только вперёд, защищая
родную землю, свободу своего государства, свою семью и будущее детей. И я
благодарна всем дедам и прадедам, которые выстояли в нелегкой борьбе, не дали
распространиться фашизму на нашей земле. Благодаря им, несколько поколений
людей могли жить и трудиться под мирным небом, не слыша свиста пуль и грохота
снарядов.

Я, как и множество детей, чувствовала себя счастливой и беззаботной. Но пришло
настоящее, и розовые очки упали с моих глаз. Вернее, не упали, а их смели
сегодняшние события. Действия Украины по отношению к моему родному краю
ошеломляют меня своей жестокостью. Стрельба, взрывы, разрушенные дома, гибель
мирных жителей, закрытые предприятия, дети и старики, вынужденные сидеть в
подвалах во время обстрелов, – это наше настоящее. Страшное настоящее. Когда
снаряды падают близко, дрожат окна – да что там окна, весь дом дрожит... Когда,
выглядывая утром из окна, видишь и слышишь не играющих во дворе детей, а
пустоту и тишину, а потом опять– близкие и отдалённые взрывы. Когда видишь
множество разрушенных зданий, вид которых угнетает...

В душе поселяется панический ужас, когда в новостях сообщают о «прилётах» на
знакомых улицах, по которым ты когда-то гулял или где живут твои родные. И это
гнетущее чувство, и рука, которая сама тянется к телефону, чтобы позвонить и
узнать, всё ли у них в порядке. Становится очень горько на душе, когда видишь
плачущих стариков, которые в одну секунду лишились всего: дома, земли,
имущества, или, что намного хуже, – собственных детей. Ведь для любого родителя
нет ничего хуже, чем пережить своего ребенка. А теперь представьте, сколько ещё
совсем молодых людей погибло на нашей территории, даже не успев толком пожить.
Сколько детей, невинных малышей лишились жизни. И за что? В чём они
провинились? А сколько людей навсегда покинули наш край в поисках даже не
лучшей жизни, а простого спасения! Ведь, согласитесь, сложно жить и верить во
что-то хорошее, когда на окнах вместо стёкол – полиэтиленовая пленка и доски, а
у соседнего дома не хватает стены. Думать обо всем этом очень страшно.

Но я всё равно верю, что настанет мир в моем родном краю. Пусть не сегодня и не
завтра, а через месяц или через год, но война обязательно закончится. И можно
будет мирно жить и учиться, ничего не боясь. Я верю, что уже мои дети не будут
знать, что такое настоящая война, ночи в сырых подвалах и свист снарядов над
головами. Что мирное детство, отобранное у нас, будет у следующего поколения,
которому не придется взрослеть под грохот проезжающих мимо танков. Что в
будущем наш регион снова станет богатым и процветающим...

Стилистика и пунктуация автора сохранена

\textSelect{\em Анна К. на момент начала боевых действий 17 лет, учащаяся Горловской
общеобразовательной школы I -III ст. № 41}

В проекте «Как я встретил начало войны» каждый житель Донбасса может
рассказать, как именно изменила война его жизнь, что произошло в его судьбе с
началом боевых действий в Донбассе. Необходимо, чтобы весь мир узнал о тех
тревожных днях 2014 года, когда началась гражданская война.

Вы можете отправить свою историю нам на почту: \url{pismo@donbasstoday.ru}

Все письма можно прочесть —  \href{https://donbasstoday.ru/category/segodnya/pisma}{здесь}
