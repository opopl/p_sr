% vim: keymap=russian-jcukenwin
%%beginhead 
 
%%file 12_01_2021.stz.news.ua.zaxid.1.ukraina_ne_antirossia
%%parent 12_01_2021
 
%%url https://zaxid.net/ukrayina__ne_antirosiya_n1533737
 
%%author_id zubʼjuk_pavlo
%%date 
 
%%tags 
%%title Україна – не Антиросія
 
%%endhead 
\subsection{Україна – не Антиросія}
\label{sec:12_01_2021.stz.news.ua.zaxid.1.ukraina_ne_antirossia}

\Purl{https://zaxid.net/ukrayina__ne_antirosiya_n1533737}
\ifcmt
 author_begin
   author_id zubʼjuk_pavlo
 author_end
\fi

\begin{zznagolos}
Україні було б вигідно мати російське громадянське суспільство своїм союзником
\end{zznagolos}

\ii{12_01_2021.stz.news.ua.zaxid.1.ukraina_ne_antirossia.pic.1}

Одна моя знайома феміністка зробила спільний проєкт з російськими колегами,
після чого довелося довго вибачатись та обіцяти, що більше це не повториться,
бо ж не можна співпрацювати з ворогом, який забрав Крим. Дивно, виявляється
Крим забрали феміністки. Гаразд, я, в принципі, і так хотів написати про
багатовимірність ставлення українського суспільства до Росії, а тут якраз
незайве нагадування про те, що тема важлива й актуальна. Отже, як кажуть
видатні люди, «Паєхалі!..».

Радянські люди періоду застою зростали з розумінням того, що таке справжня
Велика Вітчизняна війна. Має бути фронт і все для фронту. Здвиг усього народу,
самопожертва. Має бути безкомпромісне «Убей немца!». Ну і логічний результат –
велика Перемога. Цікаво, що такі люди могли бути цілком навіть антирадянськи
налаштовані – візія «правильної» війни була вмонтована десь на споді їх
свідомості. І тут – така «неправильна» війна! На фронті вмирають наші воїни, а
в тилу кожен підліток може завиграшки скачати ворожий серіал про ворожих
солдатів і співчувати його героям! Логіка «Великої Вітчизняної» потребує карати
зрадників. Ну і ще сильніше ненавидіти ворога, тих самих «німців», у війну з
якими радянські діти гралися аж до 90-х. От тільки замість німців – росіяни.

Взагалі, ні для кого не є секретом, що найлегше ненавидіти, зневажати або
просто байдуже ставитися до смерті того, хто на тебе несхожий. Саме тому
нацистська пропаганда з усіх сил намагалася карикатуризувати євреїв – мовляв,
погляньте, хіба ж ці горбоносі темношкірі персонажі в недолугому одязі можуть
бути частиною одного з вами суспільства? Ну от тепер «щирі патріоти» з-поміж
колишніх радянських людей вирішили у такий самий спосіб карикатуризувати
росіян. В хід пішов весь арсенал того, що трапилося під руку. Росіянам вирішили
нагадати, що вони нащадки угро-фінів, а зовсім не слов'яни, через що ніби
зневажливо стали називати їх «мокшанами». Я досі не можу зрозуміти, як
угро-фінськість може асоціюватися з чимось поганим: усі без винятку незалежні
держави угро-фінів суттєво заможніші та впорядкованіші за Україну. Але тут годі
чекати на логіку. Далі витягли латинську назву міста Москва (Москва латиною
називається Московія, та само як Варшава – Варсовія) і вирішили, що буде дуже
принизливо назвати Росію (грецька назва Русі) саме латинською назвою її
столиці. Поза тим, у цьому середовищі стало можна смакувати всі трагедії, які
стаються в Росії: чи то алкоголік зарізав колегу, чи то клієнти ТРЦ загинули
під час пожежі – завжди з'являються в коментарях «наші партизани» з репліками
типу «як добре, що поменшало мокшів».

А потім – щире здивування: «Як?! Якийсь юний київський відеоблоґер сказав, що
Росія – це не погано?!! Що не так з цим молодим поколінням?!! Ми ж йому котрий
рік пояснюємо, що Україна – останній форпост Європи перед дикою Ордою, що проти
нас воюють дикі болотяні люди, схильні до зловживання алкоголем і безпричинної
аґресії!!! І от як, як ці жовторотики примудряються вважати мокшанських дикунів
гідними поваги?!!».

Розгадка проста. Зараз не 1921 рік. У 1921-му можна було б розповісти мільйонам
селян, які далі сусіднього містечка виїздили кілька разів за життя, що Росія
населена собакоголовими ведмедями. Сучасний підліток має Інтернет. Він точно
знає, що Москва – це не хаотичне нагромадження дерев'яних ізб. Він слухає
російську музику, дивиться російські фільми та відеоролики. Що більше
«патріотичні доброзичливці» малюватимуть відірваний від реальності Мордор, то
більше українців не сприйматиме це серйозно. І добре, якщо «це» – фантазії про
Мордор, гірше – коли несерйозне ставлення поширюється на протистояння
російській експансії загалом.

Путін колись сказав, що ворожий Захід намагається зробити з України Антиросію.
І – дивна річ – знайшлися тисячі українців, особливо «новонавернених», часто з
російськими прізвищами та незнанням української мови, які сприйняли його слова
за чисту монету. «Наш ворог – не путінський режим, наш ворог – російський
народ, російська мова, російська культура! Ми будемо цькувати російських
опозиціонерів ще сильніше, ніж російську владу! Все російське має зникнути!»
Грубо кажучи, вітчизняні дурники повторюють усі путінські марення, тільки те,
що Путін подає як погане, вони трактують як хороше. І Путін, і «українські
патріоти» вірять, що от-от колективний Захід у чорному фраку, циліндрі і з
моноклем прийде до України як до такого собі Мальчіша-Плахіша, дасть бочку
варення і кошик печива за антиросійськість. Насправді ні.

Фактом є те, що Росія – велика і досить сильна сусідня держава. Особисто я
хотів би, щоб вона була меншою і слабшою, але реальність не відповідає моїм
хотілкам. У Росії є путінський режим і є досить цікаве громадянське
суспільство, яке в умовах авторитарного тиску розвивається й міцнішає. Так,
росіяни виділяють на опозиційні організації більше власних грошей, ніж виділяли
українці на початку 2014 року на підтримку Майдану. І Україні було б вигідно
мати російське громадянське суспільство своїм союзником, а не вправлятися перед
ним у демонстративній і тим більш недолугій ненависті до всього російського.
Хіба що ми не хочемо перемогти, а лише погратися у крутих «бендер-фашистів, які
їдять російськомовних снігурів». Така собі малоросіянська гра, мало пов'язана з
реальністю. Тільки що в реальності в цей час виграватиме Путін. Але кого з
грайликів це цікавить? Вони сконструюють собі паралельну реальність зі «сивим
гетьманом» і «зеленими зрадниками», а потім чудово в ній почуватимуться.

Так, я переконаний, що українському суспільству пора дорослішати. І перший крок
– це чітко і впевнено самим собі сказати: Україна – це не Росія і не Антиросія.
І не намагається бути ні першим, ні другим.


