% vim: keymap=russian-jcukenwin
%%beginhead 
 
%%file 07_07_2019.stz.news.ua.mrpl_city.1.larysa_petrova_malenkymy_krokamy_do_velykoj_mety
%%parent 07_07_2019
 
%%url https://mrpl.city/blogs/view/larisa-petrova-malenkimi-krokami-do-velikoi-meti
 
%%author_id demidko_olga.mariupol,news.ua.mrpl_city
%%date 
 
%%tags 
%%title Лариса Петрова: "Маленькими кроками до великої мети!"
 
%%endhead 
 
\subsection{Лариса Петрова: \enquote{Маленькими кроками до великої мети!}}
\label{sec:07_07_2019.stz.news.ua.mrpl_city.1.larysa_petrova_malenkymy_krokamy_do_velykoj_mety}
 
\Purl{https://mrpl.city/blogs/view/larisa-petrova-malenkimi-krokami-do-velikoi-meti}
\ifcmt
 author_begin
   author_id demidko_olga.mariupol,news.ua.mrpl_city
 author_end
\fi

\ii{07_07_2019.stz.news.ua.mrpl_city.1.larysa_petrova_malenkymy_krokamy_do_velykoj_mety.pic.1}

Як свідчать соціологічні дослідження, більшість українців переоцінюють рівень
своєї фінансової грамотності. Цьому предмету не вчать в школі, і відсутність
елементарних знань про те, як управляти своїми грошима, вже в дорослому віці
призводить до плачевних результатів. Однак у Маріуполі є справжні фахівці, які
готові навчати і допомагати людям не просто зберігати свій капітал, а
примножувати його і передавати наступним поколінням. Пропоную познайомитися з
наступною героїнею моєї серії нарисів, присвячених яскравим маріупольцям –
маріупольським фахівцем з фінансової грамотності \textbf{Ларисою Петровою}. Вона
відрізняється високим рівнем емпатії і дуже чутлива до проблем інших. Її досвід
і унікальні знання сьогодні можуть стати в нагоді кожному пересічному
маріупольцю. Як створити власний бюджет, що таке \enquote{фінансова подушка безпеки} -
про це і багато іншого може з легкістю і у формі гри розповісти громадська
діячка Маріуполя Лариса Петрова.

Лариса народилася в дружній родині у селі Володимирівка Дніпропетровської
області Межівського району. Мама працювала старшою медсестрою, батько – водієм
на швидкій допомозі. З мамою були дуже близькі стосунки. Вона згадує, що її
мама, \emph{Людмила Олександрівна}, завжди всім допомагала, була дуже чуйною і щирою
людиною. У Лариси є молодший брат, з яким вона підтримує теплі відносини.
Загалом діти пішли по стопах батьків. Син, як і батько, став водієм, а донька,
як мати, почала безкорисливо допомагати іншим. Лариса з дитинства мріяла бути
вчителем. Але обставини склалися інакше. Вступила до Дніпропетровського
промислового економічного технікуму. У місто Дніпро закохалася відразу.
Отримавши спеціальність \enquote{Бухгалтер} за перерозподілом у 18 років вона почала
працювати в Управлінні капітального будівництва \enquote{МК \enquote{Азовсталь}}. Так вона
потрапила до Маріуполя. Проте робота не дуже була до вподоби юній, ще й
амбітній дівчині. Вона розуміла, що кар'єру тут не побудувати. У 1993 році
Лариса вийшла заміж. Чоловік для неї став опорою і підтримкою у всьому. Вона за
ним була як за міцною і надійною фортецею.

\ii{insert.read_also.demidko.minjajlo}
\ii{07_07_2019.stz.news.ua.mrpl_city.1.larysa_petrova_malenkymy_krokamy_do_velykoj_mety.pic.2}

У 1994 році Ларису запросили до першого комерційного банку \enquote{Inko}
стажером-економістом. Саме у цьому банку почалася кар'єра нашої героїні.
Пізніше вона змінила багато банків, зокрема працювала в банках МФ КАБ
\enquote{Слов'янський}, АКБ \enquote{Укрсиббанк}, МФ Банк \enquote{БІГ Енергія}. Але де б не працювала
Лариса, вона завжди знаходила індивідуальний підхід до людей, дослухалася до
кожного клієнта. Лариса зрозуміла, що люди не знають багатьох простих речей: як
обрати кредит, визначаючись з валютою, на який термін оформляти депозит. Вона
завжди приділяла багато часу кожному клієнту. Для неї було найголовнішим
пояснити так, щоб людина дійсно зрозуміла. У 1996 році Лариса народила донечку
– \emph{Карину}. Завдяки підтримці коханого чоловіка вона отримала вищу освіту в ПДТУ,
вступивши на спеціальність \enquote{Бухгалтерський облік і аудит}. На жаль, чоловік
рано пішов з життя. У 2003 році він помер від інсульту. Ця подія розділила
життя героїні на \enquote{до} і \enquote{після}. Їй було важно наодинці зіткнутися з багатьма
проблемами, дякуючи друзям, наставникам, колегам вона справлялася. Але сила
духу жінки і розуміння, що заради донечки треба жити далі, допомогли їй.
Сьогодні донька Лариси працює помічником бухгалтера та мешкає в Харкові.
Дівчина займається волонтерською діяльністю, чуйна і щира, як і її мама.

\ii{07_07_2019.stz.news.ua.mrpl_city.1.larysa_petrova_malenkymy_krokamy_do_velykoj_mety.pic.3}

У 2008 році сталася економічна криза, яка вплинула на стан банківської системи
в країні. Ряд банків потрапив під ліквідацію, клієнти втрачали гроші, бо Фонд
гарантування вкладів не забезпечував повернення депозитів у повному об'ємі, так
прийшла думка змінити напрям діяльності, отримати нові знання, які будуть
корисними для людей. Після ліквідації банку, отримавши статус безробітної,
Лариса вирішила спробувати свої сили в страховій компанії. Побачила вакансію
начальника відділу продажів та подала резюме. Пройшла навчання, здала тести,
так розпочалася робота в 2011році в ПрАТ СК \enquote{Фортіс страхування життя}, в
подальшому ПрАТ СК \enquote{Aegon Life} Україна в якості директорки ЦСП Маріуполя. З
2012 року Лариса – приватний підприємець і сьогодні продовжує надавати послуги
в сфері страхування.

\ii{07_07_2019.stz.news.ua.mrpl_city.1.larysa_petrova_malenkymy_krokamy_do_velykoj_mety.pic.4}

Однак події 2014 року змінили життя нашої героїні. У 2015 році вона потрапила
на захід, який проводив Відкритий Університет Майдану. Її вразили і надихнули
такі яскраві громадські діячі як: Валерій Пекар, Анна Валєнса, Наталя
Склярська, Олександр Солонтай. Саме завдяки участі в цьому заході
маріупольчанка почала займатися громадською діяльністю. У травні 2016 року
Лариса разом з командою маріупольських волонтерів стала одним з ініціаторів
проведення в Маріуполі акції \textbf{\enquote{Народжені у вишиванці}} (авторка
акції – Леся Воронюк). Тепер ця акція проводиться на постійній основі і
викликає лише захват. Команда маріупольських волонтерів до всесвітнього дня
вишиванки дарує вишиванки новонародженим. Ця акція завжди дарує гарний настрій
молодим матусям та всьому медичному персоналу, тому не може залишити когось
байдужим. Того ж року Лариса стала координатором локації Міський Сад та
куратором \textbf{\enquote{Майданчика NGO}} в проекті фестиваль
\textbf{\enquote{З країни в Україну}}. На локації фестивалю об'єднали 32
громадські Організації Маріуполя та Приазов'я для інтерактивного проведення
різноманітних громадських, культурних, навчальних, просвітницьких заходів. На
локації Міського саду Лариса координувала роботу 6 майданчиків. Пізніше вона
стала співорганізаторкою проекту \textbf{\enquote{Стадіон-57}}, реалізованого
за сприяння USAID UCBI \enquote{Українська ініціатива зміцнення та розбудови
міського простору, міської спільноти}. Лариса разом з командою маріупольських
активістів почала розбудову шкільного спортивного майданчика ЗОШ №57, куди
потрапили \enquote{гради} під час обстрілу мікрорайону \enquote{Східний}.

\ii{07_07_2019.stz.news.ua.mrpl_city.1.larysa_petrova_malenkymy_krokamy_do_velykoj_mety.pic.5}

У 2017 році вона стала одним з координаторів проекту \enquote{Маріуполь туристичний},
спрямованого на розкриття туристичного потенціалу Маріуполя.

\ii{07_07_2019.stz.news.ua.mrpl_city.1.larysa_petrova_malenkymy_krokamy_do_velykoj_mety.pic.6}

Проте головний проект Лариси – це фінансова грамотність. У 2017 році вона стала
автором проекту \textbf{\enquote{Розвиток волонтерства на парафіях}} \enquote{Фінансова грамотність для
підлітків}, який було реалізовано за підтримки БО \enquote{Львівська освітня фундація}.
Вона вважає, що кожен українець повинен бути фінансово грамотним, адже кожна
людина повинна мати свій капітал та збільшувати його.

Наразі Польща і Україна займають останні місця з фінансової грамотності, тому
тренінги і заняття необхідні. Проводячи консультації та надаючи рекомендації,
не вистачало інструменту, який би дозволяв кожній людині відчути себе
підприємцем, навчитися формувати фінансову подушку безпеки, проаналізувати
наскільки використовуємо ресурси. У 2017 році, поїхавши до доньки в Харків,
Лариса познайомилася з \enquote{Діловим клубом \enquote{Партнер}}, представники якого є
розробниками соціально-економічного симулятора по управлінню ресурсами
\enquote{Життєвий Капітал}. Пізніше вона пройшла навчання та отримала сертифікат
тренера, після чого відкрила регіональне представництво ГО \textbf{\enquote{Діловий клуб
\enquote{Партнер}}} м. Маріуполь.

\ii{insert.read_also.burov.arihbaeva}
\ii{07_07_2019.stz.news.ua.mrpl_city.1.larysa_petrova_malenkymy_krokamy_do_velykoj_mety.pic.7}

З 2018 року Лариса стала тренером курсу \enquote{Фінансова грамотність} в проекті
\enquote{Підготовка вихованців інтернату та дітей з родин СЖО до самостійного життя} БФ
\enquote{Карітас} за підтримки платформи \enquote{Київський діалог}. До Лариси приходять
підлітки, батьки, підприємці. Вона наголошує, що модель \enquote{я сам} більше не
працює! Треба об'єднуватися. Лариса у формі гри навчає основам фінансової
грамотності, якої так не вистачає маріупольським сім'ям. З тренінговою грою
\enquote{Життєвий капітал} вона з командою була учасником Global Money Week (Всесвітній
тиждень грошей) – щорічна глобальна ініціатива у Національному банку України.
Нещодавно випустили гру \enquote{Лісові комерсанти}, яка розрахована на дітей від 6 до
12 років. Ця гра дозволяє малечі потроху ознайомлюватися з цікавим і
несподіваним світом фінансів. Ларису надихає, коли маріупольці завдяки її
допомозі починають розуміти навіщо потрібно заощаджувати, формувати фінансову
подушку безпеки, розкривати в собі новий потенціал та втілювати мрії в життя.

\ii{07_07_2019.stz.news.ua.mrpl_city.1.larysa_petrova_malenkymy_krokamy_do_velykoj_mety.pic.8}

У 2018 році вдруге долучилася до команди Фундації Соціальних Інновацій
Фестивалю \textbf{\enquote{З країни в Україну}} як менеджер майданчику
\enquote{Інтелектуальні ігри}.  Цього року докладено багато зусиль для
проведення півфіналів в 13 містах України. Наразі проходить підготовка до
фіналу, який проходитиме в жовтні у Києві, де зберуться з 24 областей всі
переможці півфіналу.

Громадська діячка вважає, що маленькими кроками слід йти до великої мети –
працюючи з молоддю в напряму освіти та культури. Кожна родина може ставати
заможнішою, фінансово забезпеченішою і передавати цю забезпеченість з покоління
в покоління. Наразі, на жаль, складається інакше. Як правило, передають борги і
кредити. У Лариси багато однодумців. Близько 70 клубів відкрито по всій
Україні. Започатковано онлайн-навчання. Дуже часто Лариса проводить тижні
фінансової грамотності. Це вже стало вагомою частиною її життя.

На відпочинок часу залишається небагато, але наша героїня дуже любить Азовське
море, де могла б сидіти годинами. Маріуполь для Лариси важливий і вже давно
став рідним. Сподівається, що місто продовжить розвиток у туристичному плані.

\textbf{Хобі:} подорожі.

\textbf{Улюблена книга:} \enquote{Найбагатша людина у Вавилоні} Джорджа Клейсона, \enquote{Алхімік} Пауло Коельо.

\textbf{Улюблений фільм:} \enquote{Аватар}, \enquote{Вартові Галактики}.

\textbf{Порада маріупольцям:} \emph{\enquote{відкривайте себе, не бійтесь запитувати, позбувайтеся страхів, звертайтеся за порадами до експертів і приходьте на наші заходи}.}

\ii{insert.read_also.demidko.krjachok}
