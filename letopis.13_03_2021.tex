% vim: keymap=russian-jcukenwin
%%beginhead 
 
%%file 13_03_2021
%%parent mar_2021
 
%%url https://www.facebook.com/yevzhik/posts/3711762768858806
 
%%author 
%%author_id 
%%author_url 
 
%%tags 
%%title 
 
%%endhead 
\subsection{БЖ. Донецкого благородства пост}
\Purl{https://www.facebook.com/yevzhik/posts/3711762768858806}

Сегодня получила знаковое письмо из Донецка, автор которого захотела поставить
свое имя. Много получаю, но не каждое публикую. Контент письма и мужество
автора сразили. Сначала пару слов скажу от себя.

\ifcmt
  pic https://scontent-frt3-2.xx.fbcdn.net/v/t1.6435-9/160569081_3711762728858810_8743563347753132289_n.jpg?_nc_cat=103&ccb=1-3&_nc_sid=8bfeb9&_nc_ohc=JX7UUm5VgGIAX_yKIbm&_nc_ht=scontent-frt3-2.xx&oh=ea666b5a2249d3022cc3f8572e83f610&oe=60A24242
\fi

Честно, у меня бы не хватило благородства на такое письмо. Меня не
обстреливали, моего ребенка не насиловали в подвале, я не прятала дедушку в
ванную от снарядов. Меня просто в Украине убили морально. Подумаешь, беда: меня
лишили сцены, кафедры и слова. Как следствие - напрочь здоровья. 

Для меня - лучше бы они меня убили. Но они знали, что для меня - лучше и
выбрали самое адское наказание. Меня выставили психом на посмешище черни и
вытерли об меня ноги. 

Честно? Я не верю в будущее этой страны. Этой страны нет, она умерла в 2014
году. Кто не захотел в ад вместе с ней, как я, оказался в ещё худшем аду, в
десятом круге. 

Я совершенно понимаю людей, которые меня не прощают: они пытаются сохранить
остатки своего отчаявшегося гетто и не пускать в него "бывших" майданеров из
раскаявшихся убийц страны. Так теплее, так менее страшно. Да.

Я знаю, что по сравнению с Донбассом я пережила мало. Я просто объяснила, как
человек, бывавший на войне и привыкший к военным (разным, противоположным,
всяким), что пуля для меня - не смерть. То, в чем я сейчас, - это СМЕРТЬ. 

Я начинаю ненавидеть улицы, по которым я еле хожу, речь, которую я слышу, я не
хочу просыпаться, телом и духом не хочу и не могу. Да, я не могу ни забыть, ни
простить того, что сделала со мной толпа и особенно университет. Я просто не
могу видеть никого из них. Не могу писать им. Презираю их? Да. 

На фоне моей греховности находятся люди, которые ещё могут верить в Украину
после всего, что с ними сделали. Вот - одна из этих людей. Украинцам должно
быть стыдно читать это письмо:

\ifcmt
  pic https://scontent-frt3-2.xx.fbcdn.net/v/t1.6435-9/160569081_3711762728858810_8743563347753132289_n.jpg?_nc_cat=103&ccb=1-3&_nc_sid=8bfeb9&_nc_ohc=JX7UUm5VgGIAX_yKIbm&_nc_ht=scontent-frt3-2.xx&oh=ea666b5a2249d3022cc3f8572e83f610&oe=60A24242
\fi

\enquote{Жень, привет! Я 20 лет от рождения (1967) жила в Донецке. Потом ещё 20 в
Киеве... Потом начался глобальный кризис, потом вот эта война. Мне бы хотелось,
чтобы разумные люди в Украине услышали нас.

И вот щас - супер-курсивом...

Здесь, в Донбассе никто не был нелоялен к Украине, никто - даже на фоне войны -
ее не осуждал. Было и есть просто изумление... За что, почему? Здесь никто не
только не враждебен к Украине - и это после всего! - но здесь все ещё ее любят,
разве что не могут переступить через убитых... Жень, мне давно думается, что в
Украине должны об этом знать. Люди, живущие здесь, даже после всего
случившегося, не испытывают ненависти к Украине. Они просто недоумевают - за
что с ними так поступили}.

Елена Герасимова
