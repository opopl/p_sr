% vim: keymap=russian-jcukenwin
%%beginhead 
 
%%file 03_06_2017.stz.news.ua.mrpl_city.1.legendy_ot_djadi_kosti
%%parent 03_06_2017
 
%%url https://mrpl.city/blogs/view/legendy-ot-dyadi-kosti
 
%%author_id burov_sergij.mariupol,news.ua.mrpl_city
%%date 
 
%%tags 
%%title Легенды от дяди Кости
 
%%endhead 
 
\subsection{Легенды от дяди Кости}
\label{sec:03_06_2017.stz.news.ua.mrpl_city.1.legendy_ot_djadi_kosti}
 
\Purl{https://mrpl.city/blogs/view/legendy-ot-dyadi-kosti}
\ifcmt
 author_begin
   author_id burov_sergij.mariupol,news.ua.mrpl_city
 author_end
\fi

\ii{03_06_2017.stz.news.ua.mrpl_city.1.legendy_ot_djadi_kosti.pic.1}

Летом, в конце августа, вечерами, когда день заметно становился короче, и
гонять тряпичный мяч в сумерках было неинтересно, а до времени, когда матери
звали своих чад, чтобы те вымыли ноги и укладывались спать, было далеко,
ребятня усаживалась на вытесанные из известнякового камня ступени. Они
прилепились к высокому цоколю дома и вели к его двери. Все сидели тихо, и
потому что устали от беготни, и потому что ждали дядю Костю, который не любил
суеты, и, а тем более - шума. Наконец в проеме двери появлялся Константин
Иванович. Он был смугл. У него были черные блестящие прямые волосы без малейших
признаков седины. Из-за поврежденных шейных позвонков его голова навсегда была
наклонена к правому плечу и слегка повернута. Это придавало его взгляду некое
саркастически-презрительное выражение.  Константин Иванович был жестянщиком и
трудился на Мариупольском рыбоконсервном комбинате.

Итак, появлялся дядя Костя. Он присаживался на верхнюю ступеньку. Неспешно
доставал из кармана металлическую коробочку из-под довоенного зубного порошка.
доставал оттуда щепоть самосада, насыпал, его на листочек газетной бумажки и
желтыми прокуренными пальцами делал самокрутку, прикуривал и смачно
затягивался. И до того момента, пока он щелчком длинных сухих пальцев не
сбрасывал окурок на землю, нельзя было произнести хотя бы одно слово, даже
шепотом. Только после завершения такого ритуала можно было к нему обратиться.
Откуда же у дяди Кости такая власть над детьми? А основой ее была извечная
детская потребность слушать сказки и занимательные истории. Правда, сказки он
не рассказывал, а вот истории с занимательными сюжетами – это, пожалуйста.
Повествование его начиналось как бы, само собой. Велось оно неспешно, в
полголоса, с продолжительными паузами... 

В его историях подвиги героя из одного романа или повести приписывались
действующему лицу другого произведения, исторические события смещались из одной
эпохи в другую. Но все это не мешало ребятам с напряженным вниманием слушать
рассказчика, который мог так закрутить сюжет, сложенный из фрагментов
приключений Гулливера, легенд об Илье Муромце, казаке Голоте, похождений
бравого солдата Швейка и дедуктивного метода Шерлока Холмса, что дух
захватывало. Но с самым большим вниманием слушала ребятня мариупольские легенды
в исполнении дядя Кости. Они всегда были \enquote{привязаны} к конкретным улицам, а
иногда и к домам. Правда, многие дома были сожжены немцами, и от них остались
одни остовы. Но от представления, где происходили события, повествования дяди
Кости становились еще ярче.

Вот одно предание в свободном изложении.  Однажды, во время Гражданской войны,
по дороге из Мариуполя в одно из соседних сел к строго охраняемому мосту
двигалась длинная похоронная процессия. Как водится, впереди высокий парень нес
деревянный крест. За ним шагали мужики с венками. Следующими были дроги с
огромным гробом, дьякон, машущий кадилом, батюшка, то и дело вертящий головой
по сторонам. Одни женщины плелись следом за батюшкой, другие - сидели в
тачанках. Тачанок было много. Видно хоронить везли тело богатого хуторянина.
Караульные у моста сняли фуражки и принялись усердно креститься, когда с ними
поравнялись дроги. Старший из охраны махнул рукой, и похоронный поезд двинулся
через мост, вглубь села. Через небольшой промежуток времени со стороны, в
которую недавно удалилась печальная процессия, раздались пулеметные очереди,
винтовочные выстрелы, и взрывы гранат. Через несколько минут к мосту прискакал
всадник на неоседланном коне. \enquote{Бросайте все, и бегите, Батька Махно в селе}.
Что же случилось? Оказывается, как только участники предстоящих похорон
достигли центра селения, Нестор Иванович сбросил с себя облачение священника, а
его хлопцы сорвали с себя женское одеяние. Из гроба достали пулемет, а из
тачанок винтовки. Начался бой. Внезапность и военная хитрость сделали свое дело
– враг был разгромлен. Дядя Костя деликатно умалчивал, кто был враг – красные
или белые. Теперь известно – почему. В то время еще жив был великий вождь и
учитель всех народов. И говорить лишнее было опасно.

Кто такая Клавдия Шульженко мариупольским мальцам не нужно было объяснять.
Почти каждый день из репродукторов звучал ее голос. И однажды, неожиданно для
себя, они узнали от дяди Кости, что любимица публики - их землячка. Он
клятвенно утверждал, что Клава родилась на Слободке. Что отец ее Иван был
паровозным машинистом, поэтому-то она училась в Железнодорожной школе. Это
подкупало, потому что серое здание на Первой слободке они хорошо знали. Более
того дядя Костя делился воспоминаниями о том, как будучи подростком Клава пела
в клубе железнодорожников под аккомпанемент баяна, на котором играл ее отец.
Кстати, эта байка о мариупольских корнях великой певицы бытует в нашем
приморском городе до сих пор.

7 июня 1938 года в Москве был арестован директор завода имени Ильича Николай
Радин. На следующий день, 8 июня, в его квартире в Мариуполе был проведен
обыск, и в этот же день приказом №173 Народного Комиссара оборонной
промышленности был назначен новый директор завода – Александр Фомич Гармашев,
ему тогда было всего тридцать лет. Новый директор был не только страстным
любителем футбола, но и игроком заводской футбольной команды. Конечно, до
назначения на пост руководителя завода. Вот, что довелось услышать от дяди
Кости о Гармашеве.  У Гармаша, - так болельщики называли будущего директора
завода, - был \enquote{пушечный} удар, когда он бил по мячу. Особенно - правой ногой.
Чтобы Гармаш ненароком не убил вратаря, ему запретили бить правой ногой.

Предания дяди Кости воспринимались без тени сомнения, безоговорочно верили
каждому его слову. И лишь его рассказы о старом мариупольском базаре вызывали
подозрение, что это выдумка. Действительно, как можно было поверить, что перед
лавками могли стоять бочки, наполненные орехами, изюмом, сушеными фруктами. И
никто их не крал. Но разрешалось взять горстку изюма или орехов и попробовать.
Гомерический мех вызывал рассказ дяди Кости о магазине, где будто бы были все
размеры ботинок и туфель. Но уж слова о том, что к вошедшему покупателю выбегал
приказчик, показывал всю обувь, а затем став на одно колено, собственноручно
примерял мещанке или крестьянке ботинок, или сапог. Тут ему не верили не одному
слову. Уж очень откровенной была ложь. Только недавно была отменена карточная
система. Далеко не все погорелки были восстановлены. И для того, чтобы купить
парусиновые туфли, нужно было выстоять огромную очередь.

Такими были мариупольские легенды, рассказанные жестянщиком Константином
Ивановичем на излете лета в одном из дворов старинной Торговой улицы. 
