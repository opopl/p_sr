% vim: keymap=russian-jcukenwin
%%beginhead 
 
%%file 13_02_2022.stz.news.ua.strana.3.tenj_kabula
%%parent 13_02_2022
 
%%url https://strana.news/news/376444-chto-zapadnye-smi-pishut-ob-anonsirovannoj-ssha-vojne-rossii-s-ukrainoj-.html
 
%%author_id harchenko_aleksandra
%%date 
 
%%tags rossia,ugroza,ukraina,vtorzhenie
%%title "Тень Кабула". Что пишут о "вторжении" России в Украину западные СМИ
 
%%endhead 
 
\subsection{\enquote{Тень Кабула}. Что пишут о \enquote{вторжении} России в Украину западные СМИ}
\label{sec:13_02_2022.stz.news.ua.strana.3.tenj_kabula}
 
\Purl{https://strana.news/news/376444-chto-zapadnye-smi-pishut-ob-anonsirovannoj-ssha-vojne-rossii-s-ukrainoj-.html}
\ifcmt
 author_begin
   author_id harchenko_aleksandra
 author_end
\fi

\raggedcolumns
\begin{multicols}{3} % {
\setlength{\parindent}{0pt}

Информационная лавина на тему якобы близящегося \enquote{вторжения Путина} заполонила
передовицы западных СМИ.

Британские, американские, немецкие, французские и издания других стран мир с
обложек транслируют озвученные Вашингтоном призывы покинуть Украину.  

При этом некоторые прочат повторения афганской истории с позорной сдачей Кабула
\enquote{Талибану} и провальной эвакуацией, когда люди цеплялись за самолеты.

\ii{13_02_2022.stz.news.ua.strana.3.tenj_kabula.pic.1}

\enquote{Помочь американцам в случае бегства}

\enquote{Украинский кризис: тень Кабула нависла над планами эвакуации США}, - с таким заголовком вышла британская Times. 

Издание пишет, что президент Байден одобрил планы размещения американских войск
в Польше, чтобы \enquote{помочь эвакуировать тысячи американцев из Украины в
случае вторжения России, поскольку Белый дом стремится избежать повторения
хаотического вывода войск из Афганистана}. 

Польские и американские войска начнут строить палаточные лагеря и
контрольно-пропускные пункты, \enquote{чтобы помочь американцам в случае
бегства}.

Впрочем, в Украине пока нет паники, чем искренне удивляются западные СМИ.

\enquote{В условиях, когда иностранные посольства отзывают сотрудников, а ряд
стран теперь призывают своих граждан покинуть Украину, Киев все еще не
чувствует себя городом, находящимся в кризисе}, - отмечает дипломатический
корреспондент BBC Паули Адамс.

\ii{13_02_2022.stz.news.ua.strana.3.tenj_kabula.pic.2}

\enquote{Правительство здесь говорит людям сохранять спокойствие и единство и, как
говорится в сегодняшнем заявлении, воздерживаться от действий, которые
подрывают стабильность и сеют панику. Президент Зеленский заявил, что страна
должна быть готова к любым неожиданностям. По всей Украине иностранные граждане
сейчас строят поспешные планы. Стюарт Маккензи, который живет в Киеве уже 28
лет и ведет успешный бизнес, надеется отправить жену и двух сыновей на самолет.
Но он готов, если надо, запихнуть семью в машину и проехать 300 верст до
Польши. Он любит Украину и не может поверить, что до этого дошло}, - продолжает
корреспондент BBC.

В британском посольстве молчаливые сотрудники загружают сумки в машину и
уезжают. Казалось, никто не хочет говорить.

\enquote{Недалеко к северу, за границей в Белоруссии, сейчас полным ходом идут военные
учения России. На фотографиях Министерства обороны России, опубликованных
сегодня утром, видны выстрелы из нескольких реактивных установок. Москва
по-прежнему заявляет, что не планирует вторжения. Но русские могут многое
сделать, даже не ступая на территорию Украины}, - считает автор BBC.

\ii{13_02_2022.stz.news.ua.strana.3.tenj_kabula.pic.3}

Читатели Washington Post в комментариях под статьей о телефонном звонке Байдена
в Кремль, во время которого он снова грозил Путину карами за никак не
наступающее вторжение на Украину, рассуждают о возвращении во времена холодной
войны.

\enquote{Железный занавес снова возникнет. Это разбивает сердце}, - пишет один из читателей.

\ii{13_02_2022.stz.news.ua.strana.3.tenj_kabula.pic.4}

Нагнетание и паника раскручиваются со скоростью света.

Американский политик Берни Сандерс написал статью в Guardian, в которой он
описывает ужасы войны и призывает решать украинский кризис дипломатическими
мерами и признавая за Россией сферу ее интересов.

Сандерс описывает апокалиптический сценарий, если США начнут войну с Россией, и
призывает договариваться.

\enquote{Я тревожусь, когда слышу в Вашингтоне знакомый шквал требований
\enquote{продемонстрировать силу}, потому что в Европе может начаться самая страшная
война за последние 75 лет, - пишет политик в своей колонке. - Войны чреваты
непредвиденными последствиями. Они редко проходят так, как нам рассказывают
эксперты. Вы спросите тех руководителей, которые розовыми красками рисовали
сценарии войн во Вьетнаме, Афганистане и Ираке, а в итоге оказывалось, что они
абсолютно и ужасно неправы. Спросите матерей тех солдат, которые погибли на
этих войнах или получили ранения. Спросите миллионы мирных жителей, ставших
\enquote{сопутствующими потерями}. Вот почему мы должны сделать все возможное для
поиска дипломатического решения и предотвращения войны на Украине, которая
может стать ужасно разрушительной. Никто точно не знает, каковы будут людские
потери на этой войне. Но есть оценки, согласно которым на Украине могут
погибнуть более 50 000 гражданских лиц, а миллионы беженцев хлынут в соседние
страны, спасаясь от самого страшного после Второй мировой войны конфликта в
Европе. Кроме того, погибнут многие тысячи военнослужащих российской и
украинской армий. Не исключено также, что эта \enquote{региональная} война
распространится на другие страны Европы. И тогда произойдет нечто намного более
ужасное. Но и это не все. Санкции против России и ответные российские меры,
которыми грозит Москва, приведут к мощнейшим экономическим потрясениям. Это
отразится на энергетике, банках, продовольствии и повседневных потребностях
простых людей во всем мире. Вполне вероятно, что от санкций пострадают не
только русские, но и другие народы. Между прочим, все надежды на международное
сотрудничество в борьбе с экзистенциальной угрозой глобального климатического
кризиса и будущих пандемий рухнут в одночасье}.

Также он винит Штаты в борьбе за сферы влияния, которая уже чуть было не
привела к ядерной войне во время Карибского кризиса 60 лет назад.

«Это лицемерие, когда Соединенные Штаты утверждают, что мы не признаем никакие
\enquote{сферы влияния}. Последние 200 лет наша страна руководствуется доктриной Монро,
исходя из того, что является господствующей державой в Западном полушарии, а
посему имеет право вмешиваться в дела любой страны, которая может угрожать
нашим мнимым интересам, - пишет Сандерс. - Действуя в соответствии с этой
доктриной, мы подорвали и свергли не меньше десятка правительств. В 1962 году
мы оказались на грани ядерной войны с Советским Союзом, когда тот разместил
ракеты на Кубе в 150 километрах от американского побережья, а администрация
Кеннеди посчитала это недопустимой угрозой нашей национальной безопасности».

На счёт требований России не принимать Украину в НАТО Сандерс фактически
принимает сторону Москвы, приведя аналогию: «Неужели кто-то верит, что США
промолчали бы, реши Мексика вступить в военный альянс с врагом Америки?» Из его
статьи ясно, что он против приема Украины в НАТО.

«Страны должны свободно принимать решения в области внешней политики, но это
должны быть разумные решения, в которых учитываются издержки и выгоды. То
обстоятельство, что США и Украина углубляют свои отношения в сфере
безопасности, чревато очень серьезными издержками, причем для обеих стран», -
обьясняет свою точку зрения политик.

Telegraph в свою очередь пугает, что \enquote{Путин вот-вот совершит роковую
ошибку}.

\enquote{В отличие от простого и легкого \enquote{захвата Крыма} в 2014 году,
сейчас Путин рискует оказаться в совершенно другой ситуации}, уверен
обозреватель издания Кон Кафлин. Он предупреждает российского президента об
ошибке \enquote{катастрофических масштабов} в случае наступления на Украину.  

\enquote{Аннексия Крыма и дестабилизация на востоке Украины породили
неопределенность, и никто не знает, как далеко готов зайти Путин}, пишут авторы
статьи National Interest. Они предупреждают: масштабное \enquote{вторжение} на
Украину может выйти России боком, хотя не исключают и \enquote{перемоги}
Москвы. 

\enquote{В результате масштабной войны на Украине Россия окажется в еще большей
политической изоляции и понесет серьезные потери, в то время как ее армия и без
того испытывает серьезное напряжение, сдерживая угрозы на Кавказе, в
Центральной Азии, Сирии и на китайской границе. Можем ли мы говорить о том, что
к нам снова вернулась холодная война? Некоторые знающие комментаторы именно так
и считают}, - пишет National Interest.

\enquote{Некоторые аналитики и российские официальные лица также полагают, что, если
Кремль решит начать крупную военную операцию против Украины, он с легкостью
одержит победу, ибо украинская армия сможет оказать лишь слабое и
непродолжительное сопротивление имеющим огромное превосходство сухопутным
войскам, военно-воздушным силам и военно-морскому флоту России. Но, если Россия
решится на масштабное вторжение вместо проведения ограниченных операций в
Донбассе, она наверняка поймет, что ее надежды на быструю и решительную победу
малой кровью не имеют под собой оснований. Некоторые комментаторы не учитывают
все факторы, которые могут сорвать российские планы. НАТО настойчиво делает
вид, что двери для вступления Украины должны оставаться открытыми, а Россия не
менее упорно требует от альянса публично и заблаговременно отказаться от любых
попыток предложить Украине членство в альянсе. Парадокс заключается в том, что
обе стороны отказываются признать военную и стратегическую реальность. НАТО не
готова и не склонна защищать Украину военными средствами, а также приглашать ее
к вступлению в альянс. Однако стягивание российских войск к украинским
границам, как это ни парадоксально, способствует укреплению политического
единства НАТО и повышению ее боевой готовности, а также усиливает ее желание
рассмотреть вопрос о принятии Украины в альянс}, - считает издание.

\ii{13_02_2022.stz.news.ua.strana.3.tenj_kabula.pic.5}

Боле того, в СМИ открыто пугают ядерной войной.

\enquote{Важнее всего то, что военное столкновение между НАТО и Россией будет
происходить на фоне ядерной эскалации. Стороны по опыту холодной войны должны
знать о том, что угроза ядерного конфликта неизменно нависает над любым военным
противостоянием между Соединенными Штатами и Россией. Даже если НАТО и Москва
будут воздерживаться от открытых угроз применить ядерное оружие, прямая военная
конфронтация всегда создает возможности для неверного понимания намерений
противника, эскалации из-за просчетов и постепенного разрастания конфликта,
который в итоге выходит из-под контроля политического руководства}, - пишет
National Interest. 

\enquote{Путин будет хозяином резиденции президента Франции} 

А вот телеведущий Fox News Такер Карлсон (известный тем, что задается вопросом,
почему США должны защищать Украину) прочит американцам поражение в
гипотетической схватке с Россией.

\enquote{Война с Россией - это не война с Каддафи при помощи дронов. Это может
быстро стать войной, которую невозможно контролировать. С нашей стороны войну
будут вести те же генералы, которые не смогли победить Талибан, бойцов в
сандалиях, не использующих туалетную бумагу}, - иронизирует он.

\ii{13_02_2022.stz.news.ua.strana.3.tenj_kabula.pic.6}

Россия направила в Черное море большие десантные корабли, пишет Polityka. Все
это происходит в рамках учений, однако автор статьи считает, что Москва
готовится использовать Черное море как альтернативное направления для
\enquote{удара} по Украине. Провести атаку с моря и с воздуха россиянам будет
легче, чем с суши, утверждает он. Хотя Россия неоднократно заявляла, что не
планирует ни на кого нападать.

Не обходится и без сарказма. 

Французская газета с карикатурами Charlie Hebdo шутя прочит, что Путин
\enquote{через две недели будет хозяином резиденции президента Франции}. 

Повсеместно размышляют по поводу намерений Путина. Guardian вышла со статьей
\enquote{На грани войны: чего именно Путин хочет от Украины?}.

Масштабное наращивание Россией военной мощи, по мнению Guardian, может быть
\enquote{блефом или политической уловкой, рассчитанной на российскую аудиторию}.

Издание выдвигает несколько версий, что стоит за эскалацией.

\enquote{Говорят, что Путин хочет восстановить российскую сферу влияния в Восточной
Европе, главным образом охватывающую бывшие советские республики, такие как
ныне независимые Эстония, Латвия, Литва, Беларусь, Грузия и Украина. Он часто
оплакивал их \enquote{потери} после распада Советского Союза. Путин также может
надеяться продемонстрировать Западу (и русским), что страна по-прежнему
является сверхдержавой, хотя по большинству показателей (за исключением запасов
ядерного оружия и географического положения) она представляет собой
несостоятельную державу среднего размера}, - пишет Guardian.

По версии Guardian, Путин якобы опасается, что \enquote{стратегически важная Украина,
контролирующая юго-западный фланг России, ассимилируется с Западом}.

На \enquote{вторжение}, по мнению издания, Путина могла сподвигнуть слабость
Запада.

\enquote{В прошлом году НАТО потерпела унижение в Афганистане, и Джо Байден,
который выступал за прекращение войн, а не за новые, переориентировал
американскую внешнюю политику и военные ресурсы на Китай, а не на Европу. Также
предполагается, что Путину нужна большая победа, чтобы укрепить свою внутреннюю
поддержку, оправдать свою антизападную политику, оправдать необузданную
коррупцию режима и клептоманию, а также оправдать лишения, которые переживают
россияне в результате западных санкций, введенных после его первого нападения
на Украину в 2014 году. Именно тогда он аннексировал Крым и фактически взял под
свой контроль восточную часть Донбасса}, - пишет издание.

Так чего же, по мнению западных СМИ, добивается Путин?

\enquote{Чтобы положить конец противостоянию (возможно), Путин хочет, чтобы НАТО
пообещала никогда не принимать Украину (или Грузию и Молдову) в члены. Он
хочет, чтобы альянс отошел от \enquote{прифронтовых} стран, таких как Польша, Румыния и
Болгария, бывших членов несуществующего Варшавского договора. Он хочет, чтобы
Киев принял автономный статус Донбасса и отказался от претензий на Крым (в
рамках так называемых Минских соглашений). Он хочет ограничить или остановить
развертывание в Восточной и Южной Европе новых американских ракет средней
дальности. Еще более амбициозным является то, что он хочет изменить
\enquote{архитектуру безопасности} Европы, восстановить влияние России и расширить ее
геополитическое влияние. Большинству из них США говорят \enquote{нет}. Отсюда и
нынешний кризис}, - продолжает Guardian.

Звучат и иные прогнозы - о том, что никакого вторжения не будет (к слову,
Россия также отрицает такие намерения). 

\enquote{Я не верю в вероятность крупномасштабного нападения на Киев}, - так
сказал Vocal Europe, оценивая реакцию Запада на украинский кризис, Бен Ходжес -
экс-командующий европейской армии США, заведующий кафедрой стратегических
исследований Першинга в Центре анализа европейской политики.


\end{multicols} % }
