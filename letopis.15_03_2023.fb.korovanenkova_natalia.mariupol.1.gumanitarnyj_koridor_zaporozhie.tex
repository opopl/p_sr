%%beginhead 
 
%%file 15_03_2023.fb.korovanenkova_natalia.mariupol.1.gumanitarnyj_koridor_zaporozhie
%%parent 15_03_2023
 
%%url https://www.facebook.com/natali.korovanenkova/posts/pfbid02BrQ45ZDMGr3Ss2R9YKiNJsPde5kAi3MVWru2KMGVgcmafCaYDKrbwAhcMjQ7qj7il
 
%%author_id korovanenkova_natalia.mariupol
%%date 15_03_2023
 
%%tags mariupol.war,mariupol,dnevnik,album.korovanenkova.jak_ce_bulo,15.03.2022,zaporozhie,koridor.gumanitarnyj
%%title 15.03.2022 - відкрито гуманітарний коридор для приватного транспорту у напрямку Запоріжжя
 
%%endhead 

\subsection{15.03.2022 - відкрито гуманітарний коридор для приватного транспорту у напрямку Запоріжжя}
\label{sec:15_03_2023.fb.korovanenkova_natalia.mariupol.1.gumanitarnyj_koridor_zaporozhie}

\Purl{https://www.facebook.com/natali.korovanenkova/posts/pfbid02BrQ45ZDMGr3Ss2R9YKiNJsPde5kAi3MVWru2KMGVgcmafCaYDKrbwAhcMjQ7qj7il}
\ifcmt
 author_begin
   author_id korovanenkova_natalia.mariupol
 author_end
\fi

\#маріуполь\_2022 \#як\_це\_було

БЕРЕЗЕНЬ💔Так сталося, що ми всі знову опинилися в пекельному березні, ми
знову в Маріуполі і день за днем ми проживаємо знову і знову

Хто пише, хто мовчить та плаче... але наш березень зайшов на другий виток

2 березня не стало світла, місто занурилося в непроглядну темряву, телефони
розрядилися, зв'язок пробивався дуже рідко.

Я написала на зошитному листку календар, відзначала кожен день тижня, та що
сталося за день.

Номери телефонів на листку і на стіні  біля дверей, кому зателефонувати якщо
мене не стане

Стіна розлетілася при прямому влученні. А аркуш паперу залишився.

15 березня було відкрито гуманітарний коридор для приватного транспорту у
напрямку Запоріжжя, багато людей змогли вибратися з блокадного міста, але
тільки ті хто мали транспорт, бензин та зв'язок і знали про коридор.

Ті, кому пощастило.

%\ii{15_03_2023.fb.korovanenkova_natalia.mariupol.1.gumanitarnyj_koridor_zaporozhie.cmt}
