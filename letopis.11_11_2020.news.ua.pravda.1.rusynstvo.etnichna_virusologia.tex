% vim: keymap=russian-jcukenwin
%%beginhead 
 
%%file 11_11_2020.news.ua.pravda.1.rusynstvo.etnichna_virusologia
%%parent 11_11_2020.news.ua.pravda.1.rusynstvo
 
%%url 
%%author 
%%tags 
%%title 
 
%%endhead 

\subsubsection{Етнічна вірусологія}

Потяг робить петлю навколо села Опорець, щоб згодом увійти в Бескидський
тунель, що розмежовує Львівщину та Закарпаття.

\textemdash Яка мета будь-якого вірусу? Створювати свої копії, \textemdash продовжує свої
вірусологічно-геополітичні розвідки мій співрозмовник. \textemdash Вірус починає
передаватися у спадок, як звичайна мутація ДНК. Тому, називаючи себе
підкарпатським русином, чітко усвідомлюю, що насправді я етнічний мутант.

Я з цікавістю вивчаю "етнічного мутанта". Йому років 25, худий, з дещо нудним
обличчям.

Правильна мова одразу видає в ньому філолога. Здається, закарпатські "фуй!"
("погано") та "фоссо!" ("дуже добре") він додає виключно, щоб потішити чужинця.

Хлопець, що сидить навпроти мене, називає себе сучасним "радикальним ідейним
русином". На цьому місці я насилу заспокоюю себе, щоб не запанікувати \textemdash ось
він, живий сепаратист, заколот, зрада. Намагаюсь стриматися, щоб терміново не
телефонувати на "гарячу лінію" СБУ. 

Правда, його "ідейність" зводиться до того, що він дуже не любить
праворадикальних українських націоналістів з організації "Карпатська Січ". Та
ратоборствує у соцмережах за визнання підкарпатських русинів окремим етносом, а
так звану "русинську бесіду" \textemdash окремою мовою.

А ще він мріє перекласти русинською "Ілюзії" Річарда Баха.

\ifcmt
img_begin 
    url https://img.pravda.com/images/doc/7/5/750d57a-dscf1068.jpg
    caption На полицях книгарні "Файні книгы", заснованої 30 років тому, можна знайти як сучасні русинськомовні видання, так і перевидані книги, зокрема, "Грамматику руського языка" Івана Гарайди або "Книгу варенія для сельских карпаторусских женщин" Анни Микити. всі Фото: Євген Кудрявцев.
    width 0.7
img_end
\fi

Згодом, коли потяг вже починає гальмувати перед Мукачевом, філолог-вірусолог
застібає куртку і раптом каже: "А, може, все навпаки, і саме русинство є
вірусом?".

Я витріщився на нього. Він ловить погляд і пояснює.

\textemdash Якщо я прохолов або перенервував, в мене на губі може виникнути герпес. Це не
означає, що я недавно десь підчепив герпес. Ні \textemdash весь цей час вірус жив у моєму
організмі і заявив про себе, коли імунна система дала тимчасовий збій.

Я до чого це кажу? Імунна система колективного організму під назвою Україна
збоїть вже багато років. А тому не дивно, що вірус русинства раз у раз проявляє
себе на Закарпатті.

Мій попутник сходить, а я, дещо спантеличений, їду далі, до Ужгорода.

