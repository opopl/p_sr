% vim: keymap=russian-jcukenwin
%%beginhead 
 
%%file 22_01_2020.stz.news.ua.mrpl_city.1.10_faktiv_den_sobornosti
%%parent 22_01_2020
 
%%url https://mrpl.city/blogs/view/10-tsikavih-faktiv-pro-den-sobornosti
 
%%author_id demidko_olga.mariupol,news.ua.mrpl_city
%%date 
 
%%tags 
%%title 10 цікавих фактів про День Соборності
 
%%endhead 
 
\subsection{10 цікавих фактів про День Соборності}
\label{sec:22_01_2020.stz.news.ua.mrpl_city.1.10_faktiv_den_sobornosti}
 
\Purl{https://mrpl.city/blogs/view/10-tsikavih-faktiv-pro-den-sobornosti}
\ifcmt
 author_begin
   author_id demidko_olga.mariupol,news.ua.mrpl_city
 author_end
\fi

\begin{center}
\textbf{10 цікавих фактів про День Соборності}
\end{center}

\begin{quote}
\em Сьогодні, 22-го січня, Україна святкує День Соборності. У далекому 1919-му році
відбулося проголошення \textbf{Акта злуки Української Народної Республіки та
Західноукраїнської Народної Республіки}. Це свято нагадує всім українцям, що
наша сила – в єдності. Від часу проголошення Акту накопичилося чимало
унікальних фактів. Пропоную ознайомитися з найцікавішими з них. 
\end{quote}

1. Перемовини про об'єднання Наддніпрянської України з Над\hyp{}дністрянською
ініціювали лідери саме Західноукраїнської Народної Республіки.

2. 1-го грудня 1918-го року УНР і ЗУНР підписали \enquote{передвступний} договір про
злуку у Фастові, після якого й прийняли ухвалу про наступне об'єднання двох
частин України в одну державу.

3. Проголошення Акта злуки УНР та ЗУНР в єдину незалежну державу відбулося 22
січня 1919 року на Софійському майдані в Києві. У зачитаному на зборах
\enquote{Універсалі соборності}, зокрема, відзначалося: \enquote{Однині воєдино зливаються
століттями одірвані одна від одної частини єдиної України – Західноукраїнська
Народна Республіка (Галичина, Буковина, Угорська Русь) і Наддніпрянська Велика
Україна. Здійснились віковічні мрії, якими жили і за які умирали кращі сини
України. Однині є єдина незалежна Українська Народна Республіка}.

\ii{22_01_2020.stz.news.ua.mrpl_city.1.10_faktiv_den_sobornosti.pic.1}

4. Виходячи з попереднього пункту, 22-го січня важливо відзначати і як День
Соборності, і як \emph{День Першої Незалежності України}.

\textbf{5.} В роки радянського тоталітарного режиму проголошення не\hyp{}залежності УНР і День
Соборності не відзначалися. Із утвердженням влади російських більшовиків \textbf{ці
\enquote{контрреволюційні свята} стерли із суспільної свідомості}.

6. \textbf{Перше офіційне відзначення свята Соборності на державному рівні відбулося 22
січня 1939 року} в столиці Карпатської України м. Хусті. Це була наймасовіша
демонстрація українців, адже в акції взяло участь понад 30 тисяч осіб.

\ii{22_01_2020.stz.news.ua.mrpl_city.1.10_faktiv_den_sobornosti.pic.2}

7. Головна традиція на День Соборності (зміцнилася наприкінці ХХ століття) –
поява \enquote{живого ланцюга}, що складається із людей з українською символікою. Люди,
тримаючись за руки, розтягуються, як живий ланцюжок, вулицями рідного міста, і
навіть за його межами, між містами, по мостах через річки, поєднуючи береги і
символізуючи єдність різних регіонів України. У 2011 року цю традицію
підтримали у більш ніж двадцяти містах України.

8. 21 січня 1990 року, в переддень свята Злуки, відбулася акція \textbf{\emph{\enquote{Українська
хвиля}}} одна із наймасштабніших акцій подібного роду в історії. \enquote{Живий ланцюг}
став одним із важливих кроків до відновлення Української держави. Люди взялися
за руки, щоб продемонструвати Соборність України на шляху до незалежності. Цю
акцію організував Народний Рух України, його очолював Іван Драч, а головою
виконавчого органу – Секретаріату – був колишній політв'язень Михайло Горинь.
Підготовка \enquote{Живого ланцюга} почалася ще у вересні 1989 року, організатори від
різних областей збиралися на наради щомісяця чи й раз на два тижні. Ланцюг
починався в Івано-Франківську від Центрального народного дому (колишньої
резиденції парламенту ЗУНР і місця ухвалення Акту Злуки), йшов через Стрий
(звідси йшло його відгалуження на Закарпаття), Львів, Тернопіль, Рівне, Житомир
до Києва. Початок ланцюга в Івано-Франківську зумовлений тим, що саме
Івано-Франківськ (тодішній Станіславів) у 1919 році був столицею ЗУНР.
\enquote{Українська хвиля} простягалася на близько 700 кілометрів. У \enquote{ланцюзі} взяли
участь, за офіційними даними радянського режиму, близько 450 тисяч осіб. За
неофіційними оцінками – від 1 до 5 мільйонів.

\ii{22_01_2020.stz.news.ua.mrpl_city.1.10_faktiv_den_sobornosti.pic.3}

9. Цього дня по всьому світу, де існують організовані об'єднання закордонних
українців, проходять молебні за Українську державу! А останнім часом до
української акції \enquote{живого ланцюга} приєднуються люди в інших країнах.
Зокрема, у 2018 році до акції долучилося Посольство України в Канаді, яке
провело урочисте прийняття до Дня Національної Єдності. У заході взяли участь
міністр оборони Канади  Х. Сейджан, делегація Секретаріату Кабінету Міністрів
України на чолі з Міністром О. Саєнком, члени української громади, представники
канадських урядових структур.

\ii{22_01_2020.stz.news.ua.mrpl_city.1.10_faktiv_den_sobornosti.pic.4}

10. У 2019 році на честь 100-річчя свята в Києві проводилася акція об'єднання
двох берегів Дніпра живим ланцюгом, в якій взяли участь кілька сотень людей. На
100-ту річницю злуки мешканці Івано-Франківська стали в живий ланцюг завдовжки
майже 9 кілометрів, який простягнувся через усе місто. У Дніпрі курсанти
влаштували святковий марш, утворивши велике серце. А у Вінниці люди
вишикувалися так, що утворили собою слово \enquote{соборність}. У Харкові Україну
символічно возз'єднав двома рушниками, які зв'язали прапором. А на річці Збруч,
між Тернопільською та Хмельницькою областями, де раніше пролягав кордон між
двома українськими республіками, встановили рекорд – підняли понад три тисячі
прапорів.

Обєднавча акція 1919 року залишила глибинний слід в історичній пам'яті
українців. Саме тому цього дня важливо пам'ятати про ціну, яку доводиться
платити за єдність та аналізувати й враховувати помилки минулого.

За матеріалами: \href{https://archive.org/details/2016.uinp.22_sichnja_den_sobornosti_ukrainy}{Український інститут національної пам'яті}%
\footnote{22 січня – День Соборності України. Інформаційні матеріали, УІНП, 2016, \par%
\url{http://memory.gov.ua/news/22-sichnya-den-sobornosti-ukraini-informatsiini-materiali}, \par%
Internet Archive: \url{https://archive.org/details/2016.uinp.22_sichnja_den_sobornosti_ukrainy}
}

Посольство України в Канаді
\footnote{\url{https://canada.mfa.gov.ua/ua/press-center/news/62837-svyatkuvannya-dnya-sobornosti-ukrajini-v-ottavi}}

Радіо Свобода
\footnote{\url{https://www.radiosvoboda.org/a/archive-photos-zhyvyi-lantsiug-lviv-kyiv/30386492.html}}
