% vim: keymap=russian-jcukenwin
%%beginhead 
 
%%file 24_10_2022.stz.news.ua.donbas24.1.kartyny_oleksandr_lukjanov.txt
%%parent 24_10_2022.stz.news.ua.donbas24.1.kartyny_oleksandr_lukjanov
 
%%url 
 
%%author_id 
%%date 
 
%%tags 
%%title 
 
%%endhead 

Ольга Демідко (Маріуполь)
Маріуполь,Україна,Мариуполь,Украина,Mariupol,Ukraine,Художник,Artist,Мистецтво,Исскуство,Art,War,Війна,Война,date.24_10_2022
24_10_2022.olga_demidko.donbas24.kartyny_oleksandr_lukjanov


Поранене та ув'язнене місто у роботах маріупольського художника

Маріупольський художник Олександр Лукьянов створює картини, присвячені
Маріуполю та маріупольцям, постраждалим від війни

Після повномасштабного вторгнення рф в Україну художники з усього світу
намагаються у своїх роботах зобразити всю трагедію війни. Серед них і
маріупольські художники. Нещодавно відома маріупольська громадська діячка та
журналістка Алевтина Швецова на своїй сторінці у Facebook опублікувала роботи
ще одного маріупольця Олександра Лукьянова, які вразили всіх маріупольців своєю
майстерністю та глибиною.

Читайте також: Як створювалася «Маріупольська мотанка»: історія талановитої ілюстраторки (ФОТО)

Про Олександра Лукьянова

Олександр народився 31 березня 1963 року у сім'ї художників. Батько працював на
заводі художником-оформлювачем, мама теж малювала. Картини писали і дідусь і
бабуся, тож запах лляної олії та фарби Сашко ввібрав із молоком матері. Влітку
на канікулах у селі міг довго спостерігати, як малює дід, потім біг на річку
ловити в банку піскарів, а дорогою назад заглядав у траву і спостерігав, чим
там займаються комахи. У 1980 році закінчив художню школу. Служив у групі
Радянських військ у Німеччині, де з'явилася можливість проявити себе
художником. Перші виставки картин проходили у Будинках офіцерів.

Читайте також: Про втрачені мозаїки Маріуполя зняли фільм «Жевріння. Дерево
життя»: унікальні кадри (ВІДЕО)

Після служби в армії він закінчив Приазовський державний технічний університет,
потім працював інженером. У 90-х роках почав працювати у кооперативах. Тоді
дізнавшись про творчість Сальвадора Далі, зрозумів, що у своїй творчості треба
щось змінювати. Намагався працювати у різних стилях. Проте життя диктувало свої
закони. Щоб утримувати сім'ю, організував власне виробництво меблів, розробляв
дизайн інтер'єрів, опанував техніку аерографії, але завжди думав, що не
вистачає часу на улюблену справу. До повномасштабного вторгнення присвятив себе
створенню нових робіт та вихованню дочок. Найголовнішим хобі була риболовля.
Наразі картини знаходяться у приватних колекціях країн СНД та за кордоном.
Чому художник почав зображувати війну?

Олександр Лукьянов, як і всі маріупольці, на власні очі побачив всі жахи війни
і повною мірою відчув її трагедію на собі.

«Будинок трясло від снарядів. Попадало у магазини, сусідні будинки, люди під
розривами грабували магазини з їжею, аби вижити. Сніг, справжня зима у березні
мінус 10. Трупи були скрізь. Мертвих клали у вирви під будинками. Закопувати
було неможливо. Пораненим не пощастило, допомоги чекати не було звідки.
Обстріли не припинялися ні вдень, ні вночі», — розповів Олександр про своє
життя в Маріуполі у березні.

Читайте також: «Розбери русню на сувеніри»: переселенець з Маріуполя виготовляє
унікальні брелоки

Художнику і його сім'ї довелося пережити і життя під обстрілами, і втрату
власної домівки, і страх померти від голоду та холоду. Олександр бачив на
власні очі, як помирає його власне місто, як знищували його будівлі та вбивали
громадян.

«У сусідньому будинку на воротах крейдою написано: "У будинку мертва 8-ми річна
Даша, дідусь та бабуся". В іншому загинули мати та син. І так весь район через
будинок. Чи згоріли, чи зруйновані. На вулиці з'являлися трупи місцевих
мешканців та військових. З міста під обстрілами потяглася нескінченна низка
машин та людей. Старі та діти пішки, як мурахи тікали від війни. Гуманітарний
конвой у місто не пускали».

Читайте також: У Києві відкрилася виставка маріупольських художників (ФОТО)

Коли художнику та його сім'ї все ж вдалося вибратися з Маріуполя, він зрозумів,
що повертатися вже немає куди.  «Я став вимушеним свідком цього пекла, тому моя
схильність до малювання дала мені можливість показати людям тисячну частку
пережитого мною і тисяч подібних до мене», — наголосив Олександр.

Художник підкреслив, що якщо йому коли-небудь доведеться повернутися до міста,
він одразу ж відправиться на море.

«Насамперед я помчуся на море. Зайду в цю зелену воду і закричу від щастя. Тут
пройшло моє дитинство, все моє життя... А зараз, без нього, хочеться вити...».

Читайте також: Як у Німеччині створюються футболки, присвячені Маріуполю —
деталі

Реакція маріупольців

Роботи Олександра Лукьянова вразили багатьох маріупольців, яким вдалося виїхати
з Маріуполя. Вони наголошують, що більш правдивих картин вони ще не бачили.

«Плачу... Настільки влучні картини», — наголосила Ірина Мальовнича.

«Всі картини можна друкувати. Дуже сильні емоції, до самого серця», — зазначила
Лариса Алтухова.

Читайте також: Злочини росіян проти культурної спадщини Донбасу

Водночас маріупольчанка Юрія Крючкова зауважила, що художнику, як нікому
раніше, вдалося повністю розкрити феномен «Азову».

«Для кожного українця зараз „АЗОВ“ — це найсучасніші Геракли, Прометеї, герої
всіх часів, але наші. Це подвиг, який бачили своїми очима мільйони нас. Ми не
просто мусимо „не забувати“, ми своєю волею і силою повинні вирвати їх з прирви
пекла полону. Друзі, прошу поширення робіт Олесандра Лук'янова. Пішіть про
нього, показуйте друзям, знайомим, дуже хочеться персональну виставку для
нашого талановитого земляка», — зазначила Юлія Крючкова.

Наразі з усіма роботами художника, присвяченими Маріуполю та маріупольцям під
час війни, можна ознайомитися на його сторінці в Facebook.

Раніше Донбас24 розповідав, що окупанти збираються знищити маріупольський
символ надії на мир — мурал «Мілана».

Ще більше новин та найактуальніша інформація про Донецьку та Луганську області
в нашому телеграм-каналі Донбас24.

ФОТО: з відкритих джерел.
