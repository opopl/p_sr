% vim: keymap=russian-jcukenwin
%%beginhead 
 
%%file slova.vojna
%%parent slova
 
%%url 
 
%%author 
%%author_id 
%%author_url 
 
%%tags 
%%title 
 
%%endhead 
\chapter{Война}
\label{sec:slova.vojna}

С \emph{войной} в родной стране никак нельзя примириться. О книге, которая
приобрела новый вкус после семи лет конфликта. Ще у 2018 році на книжковому
форумі купила книгу Серые пчёлы Андрея Куркова!  Гортала, тоді не пішла. Зараз,
коли минає вже сім років цієї жахливої \emph{війни}, а світла в кінці тонелю
влада так і не засвічує, просто ковтнула цю оповідь.  Ця \emph{війна} так
болить ....  Просто замисліться і відчуйте. Не змиряймося з такою ситуацією..
Ніяк. Не можна!,
\textbf{С войной в родной стране никак нельзя примириться},
Марина Ставнийчук, strana.ua, 07.06.2021

%%%cit
%%%cit_pic
%%%cit_text
\emph{Воїн}, який захищає Україну, звертається до дівчини з окопів: \enquote{Я
запрошую тебе, твого чоловіка, офіційно до себе на позицію, де ми
\enquote{прохолоджуємося в тіньочку під деревцем, ножками махаємо} } і додає:
\enquote{Поки ти працюєш, відпочиваєш, гуляєш, спиш вдома поруч зі своїм
чоловіком}.  Ще один \emph{воїн} каже: \enquote{Я кинув роботу, батько трьох
дітей, прийшов захищати свою країну}. Жінки-\emph{військовослужбовці} з фронту
також звернулися до Яни: \enquote{Мороз, погода, непогода, мінус 25. Коли по
тобі гатять дві з половиною години, мінометний обстріл. Коли ти чуєш ці виходи,
чуєш свист, рахуєш ті секунди, доки летить ця міна}. Журналісти хотіли
передати запрошення від \emph{бійців} Яні особисто, але за місцем реєстрації
б'юті-майстрині не знайшли. Де Яна нарощує вії журналісти також не довідались,
офіційних даних про її салон і чи сплачує вона податки немає
%%%cit_comment
%%%cit_title
\citTitle{Українські воїни звернулися до горе-блогерки}, 
gazeta.ua, 14.06.2021
%%%endcit

%%%cit
%%%cit_head
%%%cit_pic
%%%cit_text
Однажды, в какой-то \emph{военный} зимний день, мне приснился сон. Очень странный сон,
должен вам признаться. Словно иду я через степь и разговариваю с железными
травами, некогда бывшими фугасами, танковой броней, минометами и пулями. И
каждая \enquote{травинка} говорит мне примерно одно и тоже - что прожила немалую жизнь
в сырости складов, мечтая о войне. Не то, чтобы мечтала, скорее четко понимала
смысл своей \enquote{жизни}. А после того, как отслужила свое, превратилась в степную
траву. И так ей жить нравится больше, чем существовать орудием для убийств.
Просто жить. В степи, без границ, национальных интересов и реющих флагов
%%%cit_comment
%%%cit_title
\citTitle{Gorlovka.ua: Город, рожденный жить - Блоги}, 
Егор Воронов, gorlovka.ua, 29.07.2016
%%%endcit

%%%cit
%%%cit_head
%%%cit_pic
\ifcmt
  pic https://gorlovka.ua/Images/Upload/NewsArticle/5qzXPLR/_k2haChT5v8dE.jpg
  caption Война присылает Рите новые \enquote{подарки} почти ежедневно
\fi
%%%cit_text
\emph{Война} присылает Рите новые \enquote{подарки} почти ежедневно. \emph{Войска ДНР} базируются
чуть не доезжая села Зайцево. \emph{Украинские силовики} разбили позицию на той
стороне поля, за деревьями — до них метров триста. Раньше их было видно хуже,
из-за сарая во дворе — но недавно дяди постарались, и теперь у сарая нет
задней стенки. Там Рита нашла ещё одну хвостовую часть мины, которая и
разворотила кирпичную кладку. Теперь под камнями лежит покорёженный Ритин
велосипед и коляска. По коляске Рита не скучает, а вот велосипед совсем другое
дело — на нём можно было кататься хотя бы днём, пока дяди опять не начали
взрывать свои взрослые игрушки. Трёхгодовалая Рита помнит, что соседний
перекрёсток им с мамой лучше быстро-быстро пробегать, потому что дяди за
кустами любят иногда устраивать \enquote{бум}, просто забавы ради. Рита понимает, что
это такие правила во взрослых играх. Девочка не знает, когда они наиграются,
но каждый день ранним утром, когда обстрелы стихают, она выбегает на улицу или
во двор искать новые подарки, которые прислала \emph{война}
%%%cit_comment
%%%cit_title
\citTitle{Жизнь в Зайцево глазами российских журналистов из Life News и 3-летней девочки Риты}, 
Gorlovka.ua, gorlovka.ua, 06.06.2016
%%%endcit

%%%cit
%%%cit_head
%%%cit_pic
%%%cit_text
Вот тогда, и только тогда, у нас по-настоящему закончится \emph{война}.
Народ на Меркель дивится, а чего? Немцев простили давно. Некоторые даже не
сильно и обижались, полагая, что при ином исходе их бы поили баварским (и таки
поили бы, только в бараках по нормам еженедельной выдачи). А вот своих простить
не могут. Все никак не довоюют задним числом, не выместят старые обиды,
уверенно побеждая мертвых.  И \emph{война} закончится не тогда, когда \enquote{будет похоронен
последний солдат}. Слишком это маловероятно. \emph{Война} закончится, когда многие
ныне живущие перестанут видеть врага в своем ближнем. Разглядят в нем родню -
как немцы без особых трудностей разглядели ее в бывших военнослужащих Вермахта.
Но поскольку люди эти очевидно не дозрели до взрослых и мудрых решений, \emph{война}
занимает все их воображение. А каждый получает именно то, чего хочет больше
всего
%%%cit_comment
%%%cit_title
\citTitle{Живущие должны перестать видеть врага в своем ближнем}, 
Дмитрий Заборин, strana.ua, 19.06.2021
%%%endcit

%%%cit
%%%cit_head
%%%cit_pic
%%%cit_text
А когда \emph{война} уже началась, идеи Розенберга были отброшены даже на
теоретическом уровне. Будучи министром по восточным территориям (1940-1945), он
проводил ту политику, которую определял фюрер.  Практически же ее воплощали
вообще другие люди - в частности, гауляйтер Украины Эрих Кох, который будучи
убежденным расистом и палачом, не допускал и мысли, что на территории Украины
может быть хотя бы номинальная власть местных народов.  Кох был в большом
фаворе у Гитлера, и практическую политику по той же Украине определял чаще
именно он, а не Розенберг
%%%cit_comment
%%%cit_title
\citTitle{22 июня - 80 лет нападения на СССР. Что немцы готовили для украинцев}, 
Максим Минин, strana.ua, 22.06.2021
%%%endcit

%%%cit
%%%cit_pic
%%%cit_text
Отметим, что большая часть этих планов не только не была реализована - но и не
появилась в форме четких директив, напрямую утвержденных Гитлером.  Этот факт,
к слову, часто используется ревизионистами, которые говорят, что \enquote{План Ост} -
это фантазии, которыми занимались второстепенные немецкие чиновники. И к
реальной политике нацистского руководства отношения все это не имеет.  В
частности, такая точка зрения популярна среди тех в Украине, кто защищает
коллаборационистов, служивших немцам.  Однако тому, что \enquote{План Ост} не вышел за
рамки черновика, есть вполне очевидные причины - провал \enquote{\emph{молниеносной войны}} и
плана \enquote{Барбаросса}. Сначала \emph{войну} надо было выиграть, а уж потом всерьез
разрабатывать планы по зачистке территорий от местного населения
%%%cit_comment
%%%cit_title
 \citTitle{22 июня - 80 лет нападения на СССР. Что немцы готовили для украинцев}, Максим Минин, strana.ua, 22.06.2021
%%%endcit

%%%cit
%%%cit_head
%%%cit_pic
%%%cit_text
Данный тип конфликта далеко не новый, как нам пытаются представить.  В 18 веке
было немало подобных \emph{войн}, когда многие короли находились в родственных
отношениях, торговали, не прерывали дипломатических отношений, но \emph{военным} путем
выясняли территориальные споры.  Современная \emph{гибридная война} это когда лидеры
воюющих сторон готовы встречаться друг с другом, улыбаться и давать совместные
пресс конференции
%%%cit_comment
%%%cit_title
\citTitle{Гибридные войны велись и много столетий тому назад}, 
Андрей Головачев, strana.ua, 23.06.2021
%%%endcit

%%%cit
%%%cit_head
%%%cit_pic
%%%cit_text
Собственно, \emph{война} между ведомством Айваруса Абромавичуса и
\enquote{Антоновым} началась уже давно. \enquote{Укроборонпром} организовал на
\enquote{Антонове} масштабный аудит, и, как говорят на предприятии, начал
\enquote{копать} под руководителя завода Александра Донца. В итоге последний
был уволен, в связи с чем и возмутился трудовой коллектив. \enquote{Собрание
связано с обострением ситуации относительно неправомерных действий управляющей
структуры - ГК \enquote{Укроборонпром}, направленных на остановку работы
предприятия}, - говорится в резолюции по итогам собрания.  Антоновцы
утверждают, что \enquote{Укроборонпром} предпринимает попытки едва ли не
\enquote{убить} предприятие. В концерне же дают понять, что коллектив просто
будоражит директор, к которому \enquote{есть вопросы}.  \enquote{Страна}
разбиралась, что стоит за конфликтом на \enquote{Антонове}
%%%cit_comment
%%%cit_title
\citTitle{Почему \enquote{Укроборонпром} воюет с \enquote{Антоновым} и чем это закончится}, 
Людмила Ксенз, strana.ua, 10.06.2020
%%%endcit

%%%cit
%%%cit_head
%%%cit_pic
%%%cit_text
Mи зaзвичaй нaвiть нe здaємo coбi cпpaви в тoмy, нacкiльки Дpyгa cвiтoвa \emph{вiйнa}
вплинyлa нa фopмyвaння нaшoї cвiдoмocтi i вiдфopмaтyвaлa нaшe cycпiльcтвo в
цiлoмy. Пyтiн i Лyкaшeнкo, Кpaвчyк, Кyчмa, Ющeнкo, Янyкoвич, Пopoшeнкo - цe вce
дiти \emph{вoєннoгo} пoкoлiння. Boни виxoвaнi тими, xтo бeзпocepeдньoгo пepeжив \emph{вiйнy}
i вoювaв нa нiй. Я зapaз нe кaжy виняткoвo пpo бaтькiв: вчитeлi y шкoлi,
cyciди, нaвiть випaдкoвi пepexoжi в їxньoмy дитинcтвi - цe бyли \emph{люди вiйни}.
Пoкoлiння, для якoгo пyблiчнi poзcтpiли, бoмбapдyвaння, cтpax i гoлoд -
пpиpoднa чacтинa їxньoгo життєвoгo дocвiдy
%%%cit_comment
%%%cit_title
\citTitle{Beликий вiтчизняний пocттpaвмaтичний cиндpoм - ZAXID.NET}, 
Пaвлo Зyбʼюк, zaxid.net, 24.06.2021
%%%endcit

%%%cit
%%%cit_head
%%%cit_pic
\ifcmt
  pic https://storage.lug-info.com/cache/2/a/fc774cb0-b14b-475a-975e-3639deaca2db.jpg/w1000h616
	width 0.4
	caption Военнослужащий Народной милиции ЛНР погиб в результате обстрела со стороны киевских силовиков
\fi
%%%cit_text
\emph{Военнослужащий} Народной милиции ЛНР погиб в результате обстрела со стороны
киевских силовиков. Об этом сообщил официальный представитель оборонного
ведомства Республики Иван Филипоненко.  Ранее представительство Республики в
Совместном центре по контролю и координации режима прекращения огня сообщало,
что киевские силовики пять раз за сутки нарушили \enquote{режим тишины}.  \enquote{С
прискорбием сообщаем, что, защищая Республику от украинской агрессии, ранения
несовместимые с жизнью получил \emph{военнослужащий} Народной милиции.
Выражаем искренние соболезнования родным и близким погибшего защитника
Республики}, - сказал он
%%%cit_comment
%%%cit_title
\citTitle{Луганский Информационный Центр – Защитник ЛНР погиб при обстреле со стороны ВСУ – Народная милиция}, 
, lug-info.com, 02.07.2021
%%%endcit

%%%cit
%%%cit_head
%%%cit_pic
%%%cit_text
И еще один важный штрих к портрету Богдана. В 14-м он увидел \emph{войну}. Он собирал
трупы украинских военных и по итогу не выдержал. Он уволился из армии и
проходил курс реабилитации. А спустя несколько лет все равно вернулся. Еще раз,
я понятия не имею, идейным он был или нет. Поддерживал ли Майдан, считал ли,
что там \emph{война} с ру-армией или только с местными сепарами, не знал и, наверное,
уже и не узнаю. Но я знал, что его семья была в стесненных обстоятельствах, а
украинская армия сегодня – это стабильный и по местным меркам сравнительно
высокий источник дохода. Война на Донбассе – это в первую очередь \emph{война}
бедняков. «Продал ролс-ройс – пошел на фронт» - это для дураков
%%%cit_comment
%%%cit_title
\citTitle{Наши ребята гибнут там, где у власти нет ни стратегической, ни тактической цели}, 
Игорь Лесев, strana.ua, 08.07.2021
%%%endcit

%%%cit
%%%cit_head
%%%cit_pic
%%%cit_text
А от життя та доля його дружини і сестри мого дідуся Галі були менш тривалими і
більш трагічними. \emph{Війну} вона провела також на фронті, медсестрою. Від того часу
у нас із дідусем залишився її портрет у \emph{військовій} формі та пілотці, зроблений
у \emph{військовому} шпиталі пораненим бійцем кольоровими олівцями. Дуже гарної якості
портрет, що може свідчити про те, що вона була гарною медсестрою і її любили
бійці. Після \emph{війни} Галя возз’єдналася зі своїм чоловіком, Григорієм Чорнишем,
та прожила недовго
%%%cit_comment
%%%cit_title
\citTitle{«Заборонені» спогади з фронтів війни Німеччини і СРСР 1941–1945 років}, 
Ігор Роздобудько, www.radiosvoboda.org, 08.07.2021
%%%endcit

%%%cit
%%%cit_head
%%%cit_pic
%%%cit_text
Потому что у нас исчез сильный центр. Сильный центр был вокруг сильного
президента, условно говоря. И поэтому после того, как Кучма ушел, у нас
разыгрывалась карта юго-востока, запада, центра, на них постоянно педалировали
те или иные политики. В конечном итоге это закончилось \emph{войной}.
\emph{Война}, с одной стороны, на время позволила объединить людей по всем
понятным причинам, почему это произошло. А потом опять-таки, когда Порошенко
начал армовировский курс, с циничным, с прагматичным расчетом, абсолютно вот,
Грынив и вся его команда. Они рассчитывали на то, что они 40\% мобилизуют таким
образом, остальных там как-то докрутят, докупят и т. д., сманипулируют на
угрозе, и все будет хорошо. Но фактор Зеленского это поломал, потому что
опять-таки, общество ответило неприятием этого, во-первых, неприятием той
политики экономической, политической, которую проводил Порошенко
%%%cit_comment
%%%cit_title
\citTitle{Украина в плену консервативного мышления: почему 30 лет шли не туда и что делать}, 
Сергей Иванов; Юрий Романенко, hvylya.net, 15.07.2021
%%%endcit

%%%cit
%%%cit_head
%%%cit_pic
\ifcmt
  pic https://i.obozrevatel.com/news/2021/7/15/filestoragetemp-234.jpg?size=1944x924
  width 0.4
\fi
%%%cit_text
Пять дней проведут в трауре по погибшему земляку – 43-летнему \emph{воину} ООС
Владимиру Яськиву – жители поселка городского типа Большие Бирки на
Тернопольщине. Вскоре \emph{воин} должен был приехать домой, чтобы отпраздновать свой
день рождения и снова вернуться на восток. Вместо этого приехал в гробу –
навсегда
%%%cit_comment
%%%cit_title
\citTitle{\enquote{Мечтал и хотел жить}: что известно о погибшем на Донбассе Владимире Яскиве}, 
Лилия Рагуцкая, incident.obozrevatel.com, 16.07.2021
%%%endcit

%%%cit
%%%cit_head
%%%cit_pic
%%%cit_text
Забавно, но это чуть ли не первый случай, когда украинские исполнители повели
себя так, как полагается рокерам - в их исконной, западной или советской
традиции. После Евромайдана наши певцы проявляли себя исключительно в качестве
угодливых конформистов, которые всегда подмахивали властям, обнимались с
шароварными чиновниками, зиговали на сцене с бонами, заливали аудиторию
патриотическими соплями - помогая заливать страну кровью.  Никакой социальной
повестки, никакой рефлексии по поводу катастрофического кризиса, никакой
критики политических репрессий, милитаризма, нацизма. Украинские рокеры и
рэперы критиковали власть только справа - требуя еще больше славанации и \emph{войны}.
Потому что за это дают госфинансирование, гранты, премии, публикации и эфиры
%%%cit_comment
%%%cit_title
\citTitle{Green Grey повели себя так, как, по идее, и полагается рокерам}, Андрей Манчук, strana.ua, 31.07.2021
%%%endcit

%%%cit
%%%cit_head
%%%cit_pic
%%%cit_text
Неоголошена \emph{війна}, яку Путін розпочав у 2014 році, забрала життя понад 14 тис.
людей і призвела до переміщення мільйонів українців. Це спричинило
безпрецедентний розкол між двома країнами, який проявився в різних сферах - від
торгових зв'язків і ЗМІ до культури і релігії. Росія зникає з усіх сфер.
Замість того, щоб визнати саморуйнівний характер \emph{війни}, Путін і Медведєв твердо
заперечують її, вважаючи за краще замість цього вказувати пальцем на втручання
іноземців та українських екстремістів. Оскільки століття російського панування
над Україною, схоже, добігають кінця, вони, зі зрозумілих причин, відчайдушно
намагаються звинуватити кого-небудь, крім самих себе, в зростаючій катастрофі
%%%cit_comment
%%%cit_title
\citTitle{Стаття Медведєва – тривожний сигнал для світу. Росія не має наміру жити мирно}, 
Пітер Дікінсон, gazeta.ua, 13.10.2021
%%%endcit

%%%cit
%%%cit_head
%%%cit_pic
\ifcmt
tab_begin cols=2
  pic https://strana.news/img/forall/u/10/91/SDaSUk6DiFc.jpg
  @width 0.4

	pic https://strana.news/img/forall/u/10/91/%D0%A1%D0%BD%D0%B8%D0%BC%D0%BE%D0%BA_%D1%8D%D0%BA%D1%80%D0%B0%D0%BD%D0%B0_2021-10-21_%D0%B2_12.12_.13_.png
  @width 0.4
tab_end
\fi
%%%cit_text
Впрочем, очередной обстрел Донецка произошел уже после отъезда Остина в Румынию
- то есть это не была демонстративная акция для привлечения внимания шефа
Пентагона. Скорее это выглядит как мессидж России - что Киев получил "добро" на
продолжение обстрелов (независимо от того, давалось это "добро" или нет). 
И это соображение интересно в контексте другого недавнего визита -
замгоссекретаря Виктории Нуланд в Москву. О его итогах просачивались крайне
скупые дозы информации. И можно было предположить, что стороны как минимум
продолжат наводить мосты по Донбассу.  
Но не исключено, что такие контакты беспокоят "\emph{партию войны}", которой выгодно
продолжение конфликта. Поэтому и совершаются действия, которые могут любые
переговоры сделать бессмысленными. Например, с расчетом, что после обстрела
Донецка на американцев в Кремле "обидятся" и контакты свернут
%%%cit_comment
%%%cit_title
\citTitle{Соседям снесло крышу". Почему опять стали обстреливать Донецк и как это связано с Нуланд в Москве}, 
Максим Минин; Екатерина Терехова, strana.news, 21.10.2021
%%%endcit

%%%cit
%%%cit_head
%%%cit_pic
%%%cit_text
Самой же гнусной является попытка сочинителей этого «глоссария» в очередной раз
вымарать из истории и медиапространства \emph{Великую Отечественную войну}. Эту
\emph{страшную войну}, которая перепахала массовую психику на поколения вперед,
которая именно под таким именем вошла не просто в историю, а в коллективное
бессознательное миллионов, превратившись, по сути, в «Карло-Юнговский» архетип,
то есть аффективно-нагруженное содержание коллективной психики, и так далее...
Так вот, эту \emph{войну} поместили в раздел... фейков. Великая \emph{война}, коснувшаяся
буквально всех в этой стране — это фейк, Карл!
%%%cit_comment
%%%cit_title
\citTitle{Краткий словарь грантоедов под редакцией СНБО / Лента соцсетей / Страна}, 
Александр Карпец, strana.news, 26.10.2021
%%%endcit

%%%cit
%%%cit_head
%%%cit_pic
%%%cit_text
Антигитлеровская коалиция, о которой пишет глава правительства, 28 октября 1944
года освободившая западные области и выгнавшая последних нацистов с территории
Украины, выглядела примерно так: 1-й Украинский фронт во главе с Маршалом
Советского Союза И. С. Коневым и 4-й Украинский фронт во главе с генералом
армии И. Е. Петровым.  А вот страх премьер-министра употребить слова Советская
Армия и \emph{Великая Отечественная война} - напоминает реакцию отдельных категорий
граждан, к которым сейчас принято проявлять толерантность, на распятие и святую
воду
%%%cit_comment
%%%cit_title
\citTitle{Власть панически боится употребить слова Советская Армия / Лента соцсетей / Страна}, 
Елена Дьяченко, strana.news, 28.10.2021
%%%endcit

%%%cit
%%%cit_head
%%%cit_pic
%%%cit_text
Поскольку подобный демарш Мельник устраивает уже не впервые, но все равно сидит
в своем кресле, уже можно однозначно сделать вывод: все это говорится при
полном одобрении Банковой, которой, видимо, уже плевать на отношения с
Берлином.  Но хуже то, что Банковой плевать и на то, о чем говорит Берлин (а
вчера то же заявил и Париж), – что эскалация на Донбассе ведет к новой
полномасштабной \emph{войне}. Из всего этого можно сделать вывод, что немцев
беспокоит, будут украинцы массово погибать или нет, а собственно украинскую
власть такая перспектива не тревожит
%%%cit_comment
%%%cit_title
\citTitle{Киевский полулокдаун, "анонимные" освободители Украины, украино-немецкий скандал. Итоги "Страны"}, 
, strana.news, 29.10.2021
%%%endcit

%%%cit
%%%cit_head
%%%cit_pic
%%%cit_text
Улыбка Тревиза не изменилась. Он сказал: – Разве вредно говорить, что город
спасен? И что мы сделали это без \emph{войны}?  – Здесь некому было сражаться,
– сказал Кампер. У него были желтые как масло волосы, небесно-голубые глаза, и
он всегда противился искушению изменить эти немодные цвета.  – Ты никогда не
слышал о \emph{гражданской войне}, Кампер? – сказал Тревиз. Он был высок,
черные волосы лежали приятными волнами, и он, как всегда, засунул большие
пальцы за мягкий кожаный пояс на костюме для улицы.  – О \emph{гражданской
войне} на пространстве столицы.  – Этого вопроса достаточно, чтобы вызвать
Кризис Селдона. Это погубило политическую карьеру Хэмиша и включило тебя и меня
в Совет на последних выборах и результат замедлился… – Тревиз медленно поводил
рукой, вроде уравновешивая стрелки
%%%cit_comment
%%%cit_title
\citTitle{Кризис основания}, Айзек Азимов
%%%endcit

%%%cit
%%%cit_head
%%%cit_pic
%%%cit_text
Дело в том, что мы все живем в своем собственном, неповторимом мире, из
которого изредка высовываемся, чтобы понять, нет ли какой-то опасности. Мы
живем настолько обособленно, что ни совместное пребывание на одном кусочке
земли, ни чтение одних и тех же книг, ни одинаковая пища, ни работа, ни учеба,
ни \emph{война} не делают нас одинаковыми. Каждый остается самим собой. Более того, и
\emph{войну}, и любовь, и работу каждый переживает по-своему, лишь приклеивая
общеупотребительные слова к своему уникальному опыту.  Я думаю об этом часто.
И, быть может, думают об этом проводники, провожающие взглядом людей,
приехавших, казалось бы, из одного и того же места, но на самом деле побывавших
в совершенно разных местах
%%%cit_comment
%%%cit_title
\citTitle{Возвращение в Рай}, Андрей Ткачев
%%%endcit

%%%cit
%%%cit_head
%%%cit_pic
%%%cit_text
Переселившись на інший континент, вони ніби позбувалися, так би мовити,
первородного гріха. Виживаючи в чужому світі, вони не почувалися зобов’язаними
танцювати ритуальні історичні танці і взагалі виконувати купу зайвих рухів.
Вони реалізовувалися, не затрачаючи сил та енергії на другорядні речі. Правда,
тут варто зауважити, що сказане не стосується організованих форм життя в
діаспорі. Що перенесення в діаспору старих партійно-політичних поділів було
також вкрай шкідливим. Хоча й зумовлене низкою об’єктивних причин. По-перше,
дією задавнених гострих конфліктів між бандерівцями і мельниківцями. По-друге,
необхідністю захищатися від свого недавнього минулого, особливо часів Другої
світової \emph{війни}. Заради справедливості треба згадати й те, що довший час йшлося
не так про возвеличення своєї боротьби, як про протидію нападкам з боку
радянських спецслужб і тих, хто полював за нацистськими злочинцями
%%%cit_comment
%%%cit_title
\citTitle{Мертві герої проти живих людей}, 
Василь Расевич, zaxid.net, 29.10.2021
%%%endcit

%%%cit
%%%cit_head
%%%cit_pic
%%%cit_text
Реакция за рубежом оказалась прямо противоположной. «Это испытание ввергло меня
в состояние такого шока, какого я не испытывал никогда ранее», — признался
премьер-министр Японии Хаято Икэда.  Взрыв стал кульминацией \emph{холодной
войны} и до сих пор значится в книге рекордов Гиннесса как не имеющий аналогов
в истории. Там говорится, что взрывная волна обошла земной шар три раза, сделав
первый оборот за 36 часов 27 минут
%%%cit_comment
%%%cit_title
\citTitle{«Дом будто ножом срезало» 60 лет назад СССР взорвал «Царь-бомбу» — самую мощную в истории. Что помнят о взрыве очевидцы?: Общество: Россия: Lenta.ru}, Дмитрий Окунев, lenta.ru, 30.10.2021
%%%endcit

%%%cit
%%%cit_head
%%%cit_pic
%%%cit_text
Зупинити наступ Збройних сил Росії також не допоможуть дуже сильно
розрекламовані "Байрактари", які дуже вразливі для російської ППО і РЛС, які
здатні їх виявляти. Нажаль, українці дуже сильно недооцінюють Російську
Федерацію, бо навіть деякі військові експерти зі США вважають, що Сухопутні
війська збройних сил Росії є найсильнішими в світі. Тож вже варто, нарешті,
визнати, що Україна ніколи не зможе змагатися з Росією симетрично, у нашого
північного сусіда завжди буде \emph{військова} перевага над нами
%%%cit_comment
%%%cit_title
\citTitle{В яку "Гру кальмара" грають з Україною}, 
Стефан Закревський, analytics.hvylya.net, 30.10.2021
%%%endcit

%%%cit
%%%cit_head
%%%cit_pic
\ifcmt
  tab_begin cols=2

     pic https://avatars.mds.yandex.net/get-zen_doc/5234524/pub_60e42e2ffc247b0058ba99a4_60e485768d827a792be46c41/scale_1200
		 @caption Взятие Киева Юрием Долгоруким в 1155 году

     pic https://avatars.mds.yandex.net/get-zen_doc/5174875/pub_60e42e2ffc247b0058ba99a4_60e439f8e80c5522be58fc02/scale_1200
		 @caption Взятие Киева войсками Андрея Боголюбского. Миниатюра из летописи

  tab_end
\fi
%%%cit_text
Что вы знаете о Российско - Украинской \emph{войне}? Ничего на ум не приходит?
А между прочим их было аж несколько и речь идет вовсе не о событиях на Донбассе
или в Крыму как кто то возможно подумал. Как только Украина в 1991 году
получила суверенитет, некоторые украинские историки занялись написанием новой
украинской истории. Новое государство - новая история, что тут такого? Только
вот в придумывании собственной истории украинские историки зачем то
сконцентрировались на том, что Россия на них постоянно нападала и унижала.
Причем проделывать это по их версии она умудрялась даже тогда, когда собственно
никакой России и не существовало. Как собственно и Украины. Была Русь,
состоявшая из нескольких княжеств, со столицей сначала в Ладоге (сейчас Старая
Ладога), потом в Новгороде, потом в Киеве, потом во Владимире
%%%cit_comment
%%%cit_title
\citTitle{Как Россия с Украиной воевала}, Илья Duke, zen.yandex.ru, 06.07.2021
%%%endcit

%%%cit
%%%cit_head
%%%cit_pic
%%%cit_text
Є давнє твердження: «Горе переможеним, бо їхню історію писатимуть переможці,
їхні вороги». Свого часу президент Віктор Ющенко стверджував, що ми пережили 28
\emph{російсько-українських воєн}. І навіть добре вивірена Вікіпедія, в якій
представлено хронологію російсько-українських \emph{воєн} після 1658 року, налічує їх
приблизно двадцять. І після кожної із них, особливо після Полтави, наслідком
якої для нас стала Велика Руїна; після ліквідації УНР, як і після Другої
світової війни, у світі поширювались тонни фальсифікацій нашої історії. Отже,
сьогодні історію слід переписувати і переосмислювати тому, що писали її наші
вороги, які хочуть і надалі перемагати, — тож лише усвідомлення нами нашої
справжньої як давньої, так і новітньої історії не дасть їм цього зробити.
Відомий український аналітик Володимир ГОРБУЛІН у книзі «Як перемогти Росію у
війні майбутнього» відносить ідейну переконаність українців, яка повинна
базуватися на глибокому знанні власної історії, до видів зброї, яка, крім
іншого, може гарантувати перемогу в \emph{гібридній війні}, що її веде нині Росія
проти України
%%%cit_comment
%%%cit_title
\citTitle{Чому історію потрібно переписувати і переосмислювати?}, Микола Семена, day.kyiv.ua, 04.11.2021
%%%endcit

%%%cit
%%%cit_head
%%%cit_pic
\ifcmt
  tab_begin cols=2
     pic https://avatars.mds.yandex.net/get-zen_doc/1863556/pub_618004c5d7c9b83c8df51d8c_618007957bd5a20be36c966c/scale_1200
     pic https://avatars.mds.yandex.net/get-zen_doc/1570751/pub_618004c5d7c9b83c8df51d8c_61800972d51b313bac054e4c/scale_1200
  tab_end
\fi
%%%cit_text
Дальше речь идёт о годах Гражданской (по-нашему) \emph{войны}, но с этим периодом
украинским коллегам, чувствую, ещё долго разбираться. Хочу обратить внимание
только на один факт: 25 апреля 1920 года началось наступление объединенных
украинско-польских войск на Правобережную Украину, во время которого уже 7 мая
они овладели Киевом.  Красная армия отступала почти без боя. ... Вскоре
соотношение сил в этой \emph{войне} сложилось в пользу большевиков, которые начали
наступление силами Юго-Западного и Западного фронтов. 12 июня польские войска
оставили Киев.  Я надеюсь, вы обратили внимание: овладели Киевом
украинско-польские войска, а оставили его только польские. А Красная армия
победила только в силу сложившихся обстоятельств. Каких? Кто сложил? А разве
это интересно?
%%%cit_comment
%%%cit_title
\citTitle{История запараллеленного мира (Украины)}, 
Досужник, zen.yandex.ru, 01.11.2021
%%%endcit

%%%cit
%%%cit_head
%%%cit_pic
%%%cit_text
До какой моральной пропасти мы дошли, если врачи ( ) вполне серьезно в
телевизоре поднимают вопрос о неоказании медицинской помощи невакцинированным
людям или оказание ее не на равных условиях с вакцинированными. Что это?
Безграмотность или желание кому-то понравиться?  Даже раненных и больных
подбирают во время \emph{войн} и оказывают им мед. помощь, согласно Женевской
конвенции 1949 г., следуя здравому смыслу, человечности и гуманизму. Я уже не
говорю о ст. 139 УКУ, которая предусматривает ответственность медицинского
работника за такие действия. Вы подумали о том, как это читать и слушать тем
пациентам, которые сейчас тяжело болеют. От врачей!!!  Не выдерживает никакой
критики манипулятивная т.н. «аргументация» адептов этого бреда, ибо болеют и
умирают, к сожалению, во всем мире все - и вакцинированные и невакцинированные
%%%cit_comment
%%%cit_title
\citTitle{Врачи уже говорят о неоказании медицинской помощи невакцинированным / Лента соцсетей / Страна}, 
Ольга Голубовская, strana.news, 01.11.2021
%%%endcit

%%%cit
%%%cit_head
%%%cit_pic
%%%cit_text
Ещё один признак «колонии»... Кровожадные колонизаторы передали Украине: Крым,
Галичину, Волынь, часть Бессарабии и Буковины, русино-мадьярское Закарпатье.
Ещё раньше большевики присоединили земли Донецко-Криворожской Республики,
всегда принадлежавшие великороссам. Вчистую продув абсолютно все \emph{войны} за
«независимость» в 1918-1920-ом... Украина (уже как советская республика)
приросла землями, о которых «батько» Грушевский, Петлюра и Центральная Рада
даже не мечтали.  Украинские Генсеки дали Украине что только можно и нельзя:
лучшую индустрию, передовую науку, сельское хозяйство, культуру, передовой
спорт, инновационное здравоохранение
%%%cit_comment
%%%cit_title
\citTitle{Как Украина была колонией СССР... Не наоборот ли?}, 
Исторические напёрстки, zen.yandex.ru, 02.11.2021
%%%endcit

%%%cit
%%%cit_head
%%%cit_pic
%%%cit_text
Когда в 1943 году Геббельс призвал немецкую нацию к \emph{«тотальной войне»},
подразумевалось, что начались \emph{война} не армий, а тотальная \emph{война} наций. И в этой
\emph{войне} хороши все способы — полное подавление инакомыслия, мобилизация всех
ресурсов, задействование против иной нации диверсантов и террористов и т. п.
«\emph{Тотальная война}» идёт не только на фронте, но и в душах людей.  На
Украине объявлена тотальная \emph{война} разуму и совести. Что казалось ранее
недопустимым и презренным — стало нормой. Причем разгул стукачества — лишь одна
из примет нового времени. Некоторые, ссылаясь на благо «нации», личный
«патриотизм» и грядущие «улучшения», скатились до абсолютной мерзости, а другие
во всем готовы их оправдывать.  Вот в Харькове молодой «активист» подкрался
сзади к пожилому человеку (вся «вина» которого состояла лишь в том, что на нем
была надета старая спортивная кофта с надписью «СССР»), окликнул его и с криком
«Слава Украине» залил глаза перцовым газом. Причем видео нападения выставил сам
негодяй, явно рассчитывая на одобрение окружающих — и он его получил.  В Киеве
мовная «активистка» Ольга Кириленко в паре со шпрехенфюрером Тарасом Креминем
сновала по киевским магазинам и специально заводила разговор с продавцами на
русском языке, а потом фиксировала нарушение «закона про мову»
%%%cit_comment
%%%cit_title
\citTitle{Геть от цивилизации! Одна нация, одна мова, один гетман}, 
Константин Кеворкян, ukraina.ru, 11.11.2021
%%%endcit

%%%cit
%%%cit_head
%%%cit_pic
%%%cit_text
Враг и \emph{война}.  Враг является персонификацией угрозы и опасности, воплощающихся
в кризисах.  Формой онтологического и транзитологического кризиса является
\emph{война}.  Враг очень часто бывает весьма необходим, ибо вражда сплачивает
враждующих.  \emph{Война} тем плоха для враждебности, что иногда приводит к ее
уничтожению.  Переход от враждебности к \emph{войне} является необычайно важным
предприятием.  Превращение враждебности в \emph{войну} означает, что хотя бы одна из
враждебных сторон принимает онтологический вызов враждебности и отвечает на
него путем его уничтожения.  Чтобы \emph{война} случилась, нужно от враждебности
перейти к прямой агрессии. В этом смысле, агрессирование суть особый этап
подготовки к \emph{войне}. В этой подготовке создается особая солидарность в отношении
\emph{войны}, подавление всяческого пацифизма, происходит поиск и преследование
внутренних врагов.  Для \emph{войны} агрессия должна быть поднята до предела, когда
уже нет возможности ее выдерживать без прямого непосредственного столкновения с
врагом. Столкновение может быть каким угодно, не обязательно физическим, но оно
должно быть непосредственным
%%%cit_comment
%%%cit_title
\citTitle{Враг}, Сергей Дацюк, analytics.hvylya.net, 18.11.2021
%%%endcit
