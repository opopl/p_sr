% vim: keymap=russian-jcukenwin
%%beginhead 
 
%%file 07_11_2019.fb.sotnik_katerina.1.bulgakov_parad_kiev.cmt
%%parent 07_11_2019.fb.sotnik_katerina.1.bulgakov_parad_kiev
 
%%url 
 
%%author_id 
%%date 
 
%%tags 
%%title 
 
%%endhead 
\zzSecCmt

\begin{itemize} % {
\iusr{Катерина Сотник}
На першому фото – О. Козир-Зірка, він же – булгаківський \enquote{полковник Козырь-Лешко}.

\iusr{Глеб Бараев}

\url{https://uk.wikipedia.org/wiki/Олесь_Козир-Зірка}

\begin{itemize} % {
\iusr{Катерина Сотник}
\textbf{Глеб Бараев}
Благодарю Вас.
Однако уровень \enquote{Википедии} мною давно преодолён.
И в сфере непосредственной теоретико-прикладной деятельности, где наличествует и PhD, и Dr.habilitatus (см. инфо в профиле).
В исторических материалах, – являющихся моим эпизодическим хобби: также стараюсь пользоваться оригинальными, а не научно-популярными и герменевтико-мультиинтерпретативными источниками.
Однако – спасибо за внимание и заботу  @igg{fbicon.wink} .

\iusr{Глеб Бараев}
\textbf{Катерина Сотник} Дело не в уровне, а в указании на дату рождения. Вы ведь писали, что неизвестен его возраст. Интересно также указание на место рождения, оно совпадает с местом рождения другого атамана - Андрея Рыбалко-Зирка. Уж не родственники ли?

\iusr{Катерина Сотник}
\textbf{Глеб Бараев}
Спасибо.
Вам – просто следовало уточнить, к чему относится Ваша ссылка. Но, тем не менее, в украинских архивах Мюнхенского украинского университета, Институте украинских исследований Гарвардского университета, украинских спец.-иссл.центрах Канады, имеющих бОльший нежели в Украине объём материалов, даты рождения и смерти данного исторического субъекта – до сих пор спорны.

\iusr{Катерина Сотник}
\textbf{Глеб Бараев}
Спасибо за интересную информацию об А.Рыбалко-Зирка.
Попробую сделать запрос в бумажные архивы.
А пока что – не ли у Вас какой-нибудь эл.ссылки о нём ?

\iusr{Глеб Бараев}
\textbf{Катерина Сотник} О Рыбалко-Зирка специальных работ нет. Он упоминается лишь в связи с предательством 1921 года, но это Вы наверняка знаете.

\iusr{Катерина Сотник}
\textbf{Глеб Бараев}
Нет, вроде бы не знаю об этом событии – я ведь не историк.

\iusr{Глеб Бараев}
\textbf{Катерина Сотник} Тогда имеет смысл начать с этой статьи 

\href{https://zn.ua/SOCIETY/opozdali.html}{%
ОПОЗДАЛИ, Как и почему сорвалась попытка поднять в Украине всенародное восстание против большевиков, %
zn.ua, 14.03.1997%
}

\iusr{Глеб Бараев}

И дополнительно - что остается за пределами исторических публикаций: кто
возглалял противника - Екатеринославскую ГубЧК. Это был Трепалов, вошедший в
историю как \enquote{первый начальник московского уголовного розыска}, о нем в
последние годы даже сериалы снимают, но только о деятельности в МУРе.

\end{itemize} % }

\iusr{Катерина Сотник}

\begin{multicols}{3} % {
\obeycr
« – За що ж боролись ми з ляхами?
– За що ж ми різались з ордами?
– За що скородили списами
московські ребра !??
\smallskip
Засівали,
І рудою поливали...
І шаблями скородили.
Що ж на ниві уродилось??!
\smallskip
Уродила рута... рута...
Волі нашої отрута.
А я, юродивий, на твоїх руїнах
Марно сльози трачу; заснула Вкраїна,
\smallskip
Бур’яном укрилась, цвіллю зацвіла,
В калюжі, в болоті серце прогноїла
І в дупло холодне гадюк напустила,
А дітям надію в степу оддала.
\smallskip
А надію...
Вітер по полю розвіяв,
Хвиля морем рознесла.
Нехай же вітер все розносить
На неокраєнім крилі,
Нехай же серце плаче, просить
Святої правди на землі.
\smallskip
Спи ж, повитий жидовою,
Поки сонце встане,
Поки тії недолітки
Підростуть, гетьмани.
Помолившись, і я б заснув...
\smallskip
Так думи прокляті
Рвуться душу запалити,
Серце розірвати.
Не рвіть, думи, не паліте,
Може, верну знову
Мою правду безталанну,
Моє тихе слово.
\smallskip
Може, викую я з його
До старого плуга
Новий леміш і чересло.
І в тяжкі упруги...
Може, зорю́ переліг той,
А на перелозі...
\smallskip
Я посію мої сльози,
Мої щирі сльози.
Може, зійдуть, і виростуть
Ножі обоюдні,
Розпанахають погане,
Гниле серце, трудне...
\smallskip
І вицідять сукровату,
І наллють живої
Козацької тії крові,
Чистої, святої!!!
\smallskip
Може... може... а меж тими
Меж ножами рута
І барвінок розів’ється,
І слово забуте,
Моє слово тихо-сумне,
\smallskip
Богобоязливе,
Згадається — і дівоче
Серце боязливе
Стрепенеться, як рибонька,
І мене згадає...
Слово моє, сльози мої,
\smallskip
Раю ти мій, раю!
Нехай гинуть
У ворога діти,
Спи, гетьмане, поки встане
Правда на сім світі».
\restorecr
\end{multicols} % }

P.S. Вагомий мемуарист Т. Шевченка,

поляк Афанасьєв-Чужбинський відзначає повсякденну дружбу Шевченка з євреями,
котрих він ніколи не ототожнював з ідеологічними «жидами», з-поміж багатьох
випадків, описує один з епізодів, коли горіли бідні євреї, а всі мешканці
стояли збоку і дивилися:

– «Шевченко у розпачі кинувся їх рятувати й витягати з вогню, зі злобством й
розпачем та плачучи кричав цим обивателям: \enquote{Чого ж ви стоїте? Це ж люди !
Рятуйте їх}.

Т.Шевченко «Полякам»:

\obeycr
– «Отак-то, ляше, друже, брате!
Неситії ксьондзи, магнати
Нас порізнили, розвели,
А ми б і досі так жили.
Подай же руку козакові
І серце чистеє подай!»...
\restorecr

Цей вірш Т.Шевченко присвятив й вперше усно читав своїм найближчим друзям –
З.Сераковському (провідному герою Польського повстання 1863-67 рр.),
А.Коасовському, ксьондзу М.Зельонці (котрий зі сміхом сприймав покаянні
екскурси \enquote{елекцій} Тараса проти ксьондзів, котрий, кожного разу перепрошував
свого друга...).

P.P.S. Найкращим художньо-документальнтм біографістом Т. Шевченка вже багато
років вважається лише поляк Єжі Єнджієвіч, автор книги «Українські ночі, або
Родовід генія».

Нет описания фото.

\ifcmt
  ig https://scontent-frt3-1.xx.fbcdn.net/v/t1.6435-9/74888328_539434066882019_4199391947107860480_n.jpg?_nc_cat=102&ccb=1-5&_nc_sid=dbeb18&_nc_ohc=1zs8L0b7-MEAX-59tDI&_nc_ht=scontent-frt3-1.xx&oh=00_AT_SSiKoQgKfEzNJnCtsOJISjodInrrS4tZAoSmgS28lpg&oe=61E22C00
  @width 0.4
\fi

\begin{itemize} % {
\iusr{Катерина Сотник}

\href{https://diasporiana.org.ua/literaturoznavstvo/461-yendzheyevich-ye-ukrayinski-nochi-abo-rodovid-geniya}{%
Єнджеєвич Є. Українські ночі або Родовід генія, diasporiana.org.ua%
}

\iusr{Катерина Сотник}

«В той час, як у Прилуках перепрягали нам коней — це було вночі, — на сусідній
вулиці сталася пожежа. Горіла вбога халупа. Народ збігався, але гасили й
допомагали здебільшого євреї, бо в халупі жив їхній одновірець.

Ми також прибігли на пожежу, і Тарас Григорович кинувся рятувати майно
погорільців. Він нарівні з іншими виносив різний мотлох і, коли все
закінчилось, виголосив промову до християнського населення, що діяло якось
неохоче...

Шевченко палким словом докоряв присутнім за байдужість, доводячи, що людина у
скруті й біді, хоч би якої була нації, хоч би яку сповідувала віру, стає нам
найближчим братом».

( О. С. Афанасьєв-Чужбинський. СПОМИНИ ПРО Т. Г. ШЕВЧЕНКА)

\href{http://litopys.org.ua/shevchenko/spog24.htm}{%
О. С. Афанасьєв-Чужбинський, СПОМИНИ ПРО Т. Г. ШЕВЧЕНКА, litopys.org.ua%
}

\iusr{Катерина Сотник}

\url{https://viewer.rusneb.ru/ru/rsl01003820454?page=5}


\ifcmt
  tab_begin cols=3,no_fig,center

     pic https://i2.paste.pics/5068fe141568790aa6f13aed0bf470ec.png
		 pic https://i2.paste.pics/d952ef77b2b8183b85bf6bb7812975e0.png
		 pic https://i2.paste.pics/4d8465f48b7581bb603094feb7c556be.png

  tab_end
\fi


\end{itemize} % }

\iusr{Катерина Сотник}

Це – про Першу Світову війну, коли українці (та їхні субетноси) опинилися
розділеними між одними з найбільших європейських континентальних "гравців"
геополітики – Австро-Угорською імперією та Російською Імперією...

В аналогічному стані перебували й поляки, алеивони в силу 1) більшої
згуртованості, 2) більшої ненависності/націонал-шовінізму (може українці й без
нього обійдуться ?), 3) більшого геополітичного вміння, 4) більшого
міжнародного прагматизму – змогли вберегти у період 1917-20 років можливість
незалежної держави.

– «Війна — не тільки фронт, а повсюдне людське життя у всіх проявах: фронт і
тил, героїзм і маркірантство, рев гармат і пісня, любов і ненависть, — тож
несправедливо протиставляти фронтовиків тиловикам, загиблих — новонародженим,
наїдених — голодним, скалічених — уцілілим; війна — це круговерть, хаос, у
якому кожної миті можна перейти з одного становища в інше: ніхто не може
гарантувати для себе цілковитої безпеки.

То чи треба звинувачувати того, який скористався нагодою побути хвилину в
затишку? І чи можна осуджувати тих, які, опинившись у пеклі, мріють вирватися з
нього?

Усе на світі стало нестійким, і врятувати людину від морального краху може
тільки одна категорія — фаталізм...

Бо війна — війною!

“— Я прочитаю вам великодні молитви за Україну, — продовжував четар. — Слухайте
спочатку, чого просять усусуси (Українські січові стрільці – \enquote{УСС}, котрі потім
стали основою Західно-Української Народної Республіки та Української Галицької
армії):

1) – "Боже, дай нам Україну: Коломию, Снятии,
Станіславів, Городенку, Надвірну, Делятин !".

Ніхто не засміявся: надто мало просять у Бога стрільці. "Невже це й уся наша
політична програма?" — подумав Михайло.

\obeycr
2) — А поляки хочуть набагато більше:
– "Panie Boże! Polski naród! My się przecież znamy,
Dla nas całej Ukrainy nie prosim — żądamy!"★.
— О, ляхи мають набагато кращий апетит! — вигукнув хтось.
— Слухайте, чого хочуть євреї:
3) – "Боже, волю дай Вкраїні, — просим й ми, євреї,
На Вкраїні краще буде нам, ніж в Галілеї !".
Сміх: жид завжди хитрий!
— І на кінець — гаряча молитва нашого голови й політиків:
4) – "Боже, вольну дай Вкраїну від Сяну до Дону,
– Крім родини, не дам у ній панувать нікому !".
Вибухнув голосний регіт...»
(Р.Іваничук. Бо війна – війною: роман [про визвольні змагання першої третини ХХ ст.]).
P.S. ★ переклад з польської:
«Пане Боже! Народ польський!
Ми один одного знаєм,
Ми всієї України
Не просим – жадаєм !»
\restorecr

\iusr{Катерина Сотник}

Ни А.И.Деникин, ни остальное командование ВСЮР не издали НИ ОДНОГО приказа против погромов...

«"...Всіх же, що підбурюватимуть вас на погроми, рішуче наказую викидати геть з
нашого війська й віддавати під суд як зрадників Вітчизни. Суд же нехай судить
їх по їхніх вчинках, не жаліючи для злочинців найсуворіших кар закону. Уряд
УНР, розуміючи всю шкоду від погромів для держави, видав до всього населення
краю відозву, в якій закликає виступити проти всіх заходів ворогів, що
підіймають погроми єврейського населення.

Наказую всьому вояцтву пильно прислухуватись до цієї відозви й найширше
розповсюдити її серед населення та серед товариства. Наказ цей прочитати в усіх
дивізіях, бригадах, полках, куренях та сотнях як Наддніпрянської й
Наддністрянської армій, так і серед повстанських частин".

/Головний отаман Петлюра/».

\ifcmt
  ig https://scontent-frt3-1.xx.fbcdn.net/v/t1.6435-9/75627449_548312145994211_3868727573244542976_n.jpg?_nc_cat=102&ccb=1-5&_nc_sid=dbeb18&_nc_ohc=K3JswlptdHIAX9gh2G6&_nc_ht=scontent-frt3-1.xx&oh=00_AT_Z5oz6Dz5nJqTW0c04AwYtV3TqNm7m82pgX0LshXRO8g&oe=61E2FF21
  @width 0.4
\fi

\iusr{Катерина Сотник}

Я считаю, что бОльшая вина деникинских офицеров ВСЮР по сравнению с
армейцами-погромщиками Директории УНР в том, что в отличии от петлюровцев
(большинство которых составляли необразованные селяне) – как раз деникинцы в
силу гораздо бОльшего уровня культуры и образованности, должны были и проявить
их в этих обстоятельствах.

А не вести себя, сбросив оболочку \enquote{интеллигента}, – как тёмный
люмпен...

Как выразился киевлянин-монархист журналист и писатель В.В.Шульгин: «Белые
проиграли из-за того, что стали \enquote{серыми}».

\href{https://lechaim.ru/ARHIV/164/dostup.htm}{%
ЕВРЕЙСКАЯ ДЕЛЕГАЦИЯ У ГЕНЕРАЛА ДЕНИКИНА, lechaim.ru%
}

\iusr{Катерина Сотник}

Так вот, мы, украинцы, должны признать – евреи действительно страдали в годы
становления УНР. Но все дело в том, что нужно глубоко разобраться в причинах
этих злодеяний и, не оправдывая виновников, указать в каждом конкретном случае
на них. И тогда в этом ряду мы увидим не только стихийную украинскую атаманию,
громившую евреев под воздействием брошенных большевиками лозунгов об
экспроприации буржуазии, но и организованную травлю со стороны деникинцев, не
имевшую по размаху и зверству аналогов, а также погромы польских войск и
большевистских, в частности казаков Буденного.

Всестороннее изучение документов однозначно засвидетельствует: правительство
УНР, возглавляемое Петлюрой (как и армия, которой он руководил), не имеет
никакого отношения к организации еврейских погромов. Наоборот, именно
Директория с самого начала пресекала погромщиков. А если мы попытаемся
разобраться с причинами погромов, то обязательно убедимся: они были на руку
противникам украинской государственности и возникали каждый раз, когда Украина
заявляла о самостоятельном пути развития.

\href{https://www.interesniy.kiev.ua/simon-petlyura-kak-protivnik-evreyski}{%
Симон Петлюра как противник Еврейских погромов, interesniy.kiev.ua%
}

\iusr{Катерина Сотник}

Ну, а пока, на Украине-Руси, как всегда, скорее всего, благополучно обойдутся и
без \enquote{шведов}, варягов, московитов, америкосов, китаёзов и прочих
\enquote{доброжелателей-абрикосов}:

\begin{multicols}{2} % {
\obeycr
«И стояла Русь без бога
После княжьей пьянки,
И крутилось все, как прежде,
На манер шарманки.
\smallskip
Помирал, как прежде, старый,
Малый — нарождался,
Работяга делал дело,
Пьющий — напивался.
\smallskip
После яблока и груши
Слива поспевала,
После всякого ненастья —
Вёдро наставало.
\smallskip
Среди дня сияло солнце,
Месяц — среди ночи.
Летом князь потел от зноя,
Как и всякий прочий.
\smallskip
Вырастало в поле жито,
И бурьян — в овраге,
Языком паны трудились,
А горбом — бедняги.
\smallskip
В пользу тех, кто даст побольше,
Разрешали тяжбу;
Голод пищей утоляли,
А водою — жажду.
\smallskip
Мокрота была в озерах,
А в каменьях — твердость.
Голодранцы осуждали
Богача за гордость.
\smallskip
Дворянин с простолюдином
Не общался сроду.
Шинкари, как прежде, в пиво
Подбавляли воду.
\smallskip
Молодые торопились,
Старцы ковыляли.
В бочку меда ложку дегтя
Всюду подбавляли.
\smallskip
Мироеды залезали
К беднякам в карманы.
Мудрецы встречались редко,
Чаще же — болваны ...».
\restorecr
\end{multicols} % }

(К. Гавличек-Боровский. Крещение святого Владимира).

\iusr{Катерина Сотник}

10 марта 1940 г. в 48 лет скончался Михаил Афанасьевич Булгаков.

Памяти М. А. Булгакова

\begin{multicols}{2} % {
\obeycr
Вот это я тебе, взамен могильных роз,
Взамен кадильного куренья;
Ты так сурово жил и до конца донёс
Великолепное презренье.
\smallskip
Ты пил вино – ты как никто шутил
И в душных стенах задыхался,
И гостью страшную ты сам к себе впустил
И с ней наедине остался.
\smallskip
И нет тебя, и всё вокруг молчит
О скорбной и высокой жизни,
Лишь голос мой, как флейта, прозвучит
И на твоей безмолвной тризне.
\smallskip
И кто подумать смел, что полоумной мне,
Мне, плакальщице дней погибших,
Мне, тлеющей на медленном огне,
Вcex потерявшей, всё забывшей,
\smallskip
Придётся вспоминать того, кто, полный сил,
И светлых замыслов, и воли,
Всё кажется, вчера со мною говорил,
Скрывая дрожь предсмертной боли.
\restorecr
\end{multicols} % }

(Анна Ахматова, 1940 г.).

\ifcmt
  ig https://scontent-frt3-2.xx.fbcdn.net/v/t1.6435-9/89775048_637927360366022_5229844933170429952_n.jpg?_nc_cat=101&ccb=1-5&_nc_sid=dbeb18&_nc_ohc=OGLejeaHpT0AX9Fxd9C&_nc_ht=scontent-frt3-2.xx&oh=00_AT9K1XJTMxGW3L2yLzSYRnP5u7D_5ZSe-Ql4wSn2g05wDg&oe=61E5362C
  @width 0.4
\fi

\iusr{Катерина Сотник}

Как известно, Булгаков любил Киев, но не Украину и не украинский язык.

Прямо так в «Белой гвардии» и писал — как минимум, говорил устами своих героев
— русских офицеров.

Бесспорно, что роман Булгакова имперский.

И, соответственно, поскольку место его действия — «самостийная Украина» и
собственно русский город Киев, это роман антиукраинский. Чему немало
доказательств в его тексте.

–Все знают, что "Белая гвардия" Булгакова (точнее его пьеса, которую весьма
любил отец народов Сталин "Дни Турбиных" по изначальному роману "Белая
гвардия", а еще больше - российская экранизация 2012-го года) - вещь весьма
анти-украинская, и уж точно анти-петлюровская, анти-незалежная.

Но те, кто неоднократно читал оригинал, смотрел прекрасную экранизацию В.
Басова конца 1970-х "Дни Турбиных", читал сопутствующие рассказы Булгакова,
внесенные в экранизацию 2012-го, и вот что что я вам скажу, господа, насчет
"анти-украинскости" Булгакова.

– Он, Булгаков, сам служил медиком в армии УНР, то есть у Петлюры. А также он,
Булгаков, присутствовал и при входе войск УНР в Киев в конце 1918-го, и при
входе в Киев "долгожданной" Красной Армии за год до того, в январе-феврале
1918-го. Да и годовалое присутствие сил Украинской Народной Республики в Киеве
в 1917-18гг сам Булгаков воочию наблюдал, хотя и описал ожидание и вход войск
УНР в Киев в конце 18-го как нечто новое, страшное, невиданное, адское,
чудовищное.

–История, кстати, его изложения не подтверждает - ничем особо чудовищным оба
киевских периода УНР не запомнились. Бесталанностью, бестолковостью и
неоправданными надеждами они да запомнились. Всяко хуже душегубов-большевиков
они не были.

–В начале 18-го, при входе в Киев, (этого у Булгакова ясен пень нету) красный
командарм Муравьёв, невзирая на мирность и беззащитность Киева, подверг город
чудовищному артобстрелу и уничтожению - 15 тысяч снарядов по центру, потом
ГАЗОВАЯ атака на город боевыми отравляющими веществами, а потом и резня с
грабежом и тотальным многодневным пьяным беспределом краноармейцев в столице
Украины. Пять тысяч человек вырезали, за НЕДЕЛЮ.

– Сам Муравьёв писал: "Я занял город, бил по дворцам и церквям... бил, никому
недавая пощады! 28 января Дума (Киева) просила перемирия. В ответ я приказал
душить их газами. Сотни генералов, а может и тысячи, были безжалостно убиты...

Так мы мстили. Мы могли остановить гнев мести, однако мы не делали этого."

– Они мстили... Это писал человек, всего за ДВА месяца до того променявший
золотые погоны полковника Русской Армии на комиссарский "маузер" и мандат от
товарищей Ленина и Троцкого. Мстил он, ага.

Так вот, господа, будучи всё-таки советским писателем, М. Булгаков написал в
"Днях Турбиных" то, что и должен был написать советский писатель второй
половины 20-х - он в своей работе просто ПОМЕНЯЛ стороны местами - "зверьми" он
изобразил войска Петлюры, вместо Красной Армии, хотя и знал правду, наблюдал
варварство "красных" видел всё это сам. Писал он то, что надо было писать в
имеющейся ситуации. Не на Мон-Мартрах парижских он находился, а в советской
Москве, в Москве сталинской, а там уж изволь...

Вот еще чуть-чуть, что пишут об взявшем Киев красном командарме Муравьёве
советские энциклопедии:

"Будучи сторонником лозунга « Россия единая, великая и неделимая » Муравьёв был
ярым противником « украинизации», а «украинцев» считал «предателями-
мазепинцами » и «австрийскими шпионами». Войска Муравьёва проводили массовые
репрессии против украинской интеллигенции, офицеров, буржуазии, говорить на
улицах на украинском языке стало смертельно опасно"...

\iusr{Катерина Сотник}

Михаил Булгаков и украинский язык:

\href{https://www.facebook.com/groups/kiev.klab.group/posts/465339534329679}{%
Михаил Булгаков и украинский язык, Катерина Сотник, facebook, 05.11.2019%
}

\iusr{Катерина Сотник}

«...28 ноября стало известно, что петлюровские войска заняли Одессу. 30 ноября
немцы по требованию украинского правительства увели свои войска с пунктов,
прикрывавших Киев и 1-го декабря \enquote{сЄчевики} под начальством Коновальца12
вступили в бой с гетманскими войсками. Немцы держали себя нейтрально, охраняя
однако безопасность Скоропадского.

[...]

На стороне гетмана была только сердюкская дивизия хлеборобов да русские
добровольческие офицерские отряды, объединенные генералом графом Келлером.

[...]

Бой гетмана с Петлюрой был неравный. Сердюки держали себя двусмысленно и стали
переходить на сторону Петлюры. Русские добровольцы были и малочисленны и
неорганизованны, да у них и не было морального подъема. У Петлюры все же была
дисциплинированная сила в лице корпуса \enquote{сiчiвиков} под начальством Коновальца.
Опираясь на эту силу, Петлюра мог использовать своих отаманов...

[...]

Искали главным образом, конечно, русских офицеров. Они их или убивали на месте
или тащили в Педагогический музей \enquote{в следственную комиссию}, которая должна
была сыграть роль Мариинского парка времён муравьёвщины. Здесь нужно отдать
должное немцам, не допустившим массового уничтожения русского офицерства.

Однако им не удалось спасти генерала графа Келлера. Его украинцы арестовали. Он
был заключен в Михайловский монастырь. Во время перевода его украинцами в
другое место конвойные застрелили его на площади у памятника Богдану
Хмельницкому.

[...]

Официальными представителями Добровольческой армии в Киеве были генерал
Ломновский и кн[язь] Долгоруков.  ...

У Ломновского с Долгоруким происходили нелады, не было единства действий. Как
мне говорили, Долгоруков склонялся к политике признания самостийной Украины,
Ломновский расчленения России не признавал.

В общем же ни тот, ни другой никакого содействия организации русского
офицерства в Киеве не оказали».

(Мемуары генерала П. С. Махрова [1876-1964] \enquote{Развал русского фронта в 1917 году и
немецкая оккупация Украины в 1918 г.} – 1953 г.)

\ifcmt
  ig https://scontent-frt3-2.xx.fbcdn.net/v/t1.6435-9/151731472_888402085318547_2393565268790623674_n.jpg?_nc_cat=103&ccb=1-5&_nc_sid=dbeb18&_nc_ohc=By-7FmSsOhoAX-2NUaw&_nc_ht=scontent-frt3-2.xx&oh=00_AT9yKxvQclNBhy_BDf_z0FEuIEGUZSrGGylWk7VNmknLrA&oe=61E4F462
  @width 0.4
\fi

\iusr{Катерина Сотник}

Предыстория событий:

– В ноябре 1918 г. Германия и её союзники потерпели поражение в Первой мировой
войне. Это означало скорое изменение положения оккупированных германскими
войсками территорий, в том числе Украины, где существовал зависимый от Берлина
режим гетмана П.П. Скоропадского.

Скоропадский начал искать пути выхода из ожидаемого кризиса. После неудачи в
переговорах с украинскими социалистическими организациями, 17 октября 1918 г.
вышло правительственное постановление об организации в целях поддержания
законности и порядка добровольческих дружин (известных по роману Булгакова).
Можно сказать, что гетман постепенно стал менять политический курс в сторону
сближения с русскими/российскими кругами антибольшевистской направленности. К
началу "петлюровского движения" формирование добровольческих дружин ещё не
завершилось.

Кроме того, не всегда офицеры желали участвовать в борьбе с петлюровцами –
некоторые стремились только к борьбе с красными.

В ночь на 14 ноября 1918 г. украинские национальные социалисты сформировали
Директорию и объявили о воссоздании Украинской народной республики и днем – о
восстании против гетмана П.П. Скоропадского.

В тот же день гетман обнародовал федеративную грамоту, в которой говорилось о
будущей Украине как автономии в составе федеративной России, освобождённой от
большевиков.

В результате развернувшейся вооруженной борьбы войска Директории 14 декабря
заняли Киев, Скоропадский бежал в Германию.


\iusr{Катерина Сотник}

Май 1918 года стал началом грандиозной крестьянской войны, которая очень скоро
охватила всю территорию Украины. Главные причины этой войны: возобновление
помещичьего землевладения и террор карательных и реквизиционных отрядов
интервентов, которые грабили, казнили, пороли розгами непокорных селян. Против
насилия австро-немецких войск и гетманской «варты» (охраны) выступила
организованная и влиятельная в украинских селах сила вольное казачество,
отвернувшееся от своего недавнего гетмана.

В ходе локальных восстаний украинских крестьян только за шесть первых месяцев
пребывания иноземных армий в Украине было убито около 22 тысяч австро-немецких
солдат и офицеров (по данным немецкого Генерального штаба) и более 30 тысяч
гетманских вартовых. Фельдмаршал фон Эйхгорн указывал, что более 2 миллионов
крестьян в Украине выступило против австро-немецкого террора. Можно сказать,
что только в повстанческих вооруженных отрядах в мае — сентябре 1918 года
успело побывать до 100 тысяч человек.

Лидер белогвардейцев генерал Деникин, далеко не симпатизирующий крестьянской
борьбе, в своих воспоминаниях указывал, что украинское село поднялось против
немцев как грабителей и защитников помещиков. Восстания крестьян практически
сорвали сбор и вывоз из Украины продовольствия. До ноября 1918 года из Украины
в Германию и Австро-Венгрию было вывезено только 113 тысяч тонн муки (около
9300 вагонов хлеба), около 30 тысяч вагонов продуктов и сырья... 19,5 тыс.
вагонов было отправлено в Австро-Венгрию, 16,5 тысячи вагонов — в Германию, 271
вагон — в Турцию и 130 вагонов — в Болгарию. Интервенты, рассчитывающие на
большее, так и не смогли преодолеть продовольственный кризис в Германии и
Австрии за счет Украины.

Первое крупное восстание произошло в конце мая 1918 года в районе
Елизаветграда, когда против грабителей выступило около тысячи вооруженных
повстанцев-крестьян.

\iusr{Катерина Сотник}

В начале июня поднялась Екатеринославщина и Уманщина, где в повстанческих
отрядах сражалось уже до пяти тысяч человек. Наибольшую угрозу для режима
представляло крестьянское восстание, начавшееся 2 июня 1918 года на
Звенигородчине и Таращанщине (юг Киевской губернии). Поначалу отряды повстанцев
формировались из участников вольного казачества бывшим организатором этого
казачества, а в мае 1918 года — полномочным представителем гетманской державы
Юрием Тютюнником. Пользуясь своим служебным положением, Тютюнник смог тайно
передать повстанцам 10 тысяч винтовок, более 200 пулеметов, 2 пушки, большое
количество патронов и военного снаряжения. Это позволило создать целую
«повстанческую армию» (около 15 тысяч человек), которая вскоре захватила
уездный городок Таращу.

16 июня 1918 года, во время разрастания Звенигородско-Таращанского восстания, в
Киеве собрался Земский всеукраинский съезд. Оправдывая восстания и ища их
причину, съезд указал на политику «безоглядной реакции и реставрации старого
строя», которую проводит гетман Скоропадский. В меморандуме Земского съезда
звучал протест против расправы австро-немецких карателей над крестьянами,
против «нечеловеческого насилия», которому постоянно подвергались жители сел.
Земские деятели требовали немедленного созыва Украинского Учредительного
собрания, которое должно решить аграрный вопрос и вопрос о пребывании
интервентов в Украине. Петлюра тогда заявил гетману: «Мы требуем, чтобы не
нарушались элементарные права человека, как это было в царское время».

В августе — сентябре 1918 года германским и гетманским войскам с трудом удалось
подавить Звенигородско-Таращанское восстание. Но восстание вспыхнуло в новых
регионах.

На Полтавщине и Черниговщине — под руководством большевиков и левых элементов
из украинских партий эсдеков и эсеров. На Екатеринославщине и в Северной Таврии
— под началом анархистов и левых эсеров. В забытой Богом и людьми крохотной
степной Гуляй-Польской волости Александровского уезда Екатеринославщины знамя
восстания подхватил анархистский атаман — батька Нестор Махно. Вокруг
харизматической фигуры Махно — защитника обездоленных уже собиралась
многотысячная крестьянская армия всех недовольных режимом.

Подрывную работу против гетмана вела глубоко законспирированная организация
офицеров, ранее служивших в войсках Центральной Рады, а потом перешедших в
армию гетмана, — «Украинский офицерский союз — Батькивщина (Отечество)». В этот
союз вошли будущие руководители восстания полковники Васыль Тютюнник и
Александр Осецкий, возглавил союз генерал Александр Греков. «Батькивщина»
поддерживала отношения с Национальным Союзом и с некоторыми атаманами
крестьянских повстанцев.

\iusr{Катерина Сотник}

В конце мая 1918 года был создан ещё один центр оппозиции режиму — межпартийный
Украинский Национальный Союз. Поначалу он ограничился умеренной критикой режима
и кабинета министров Лизогуба, как «неукраинского в своем составе и по своей
политической ориентации». Главой Национального Союза сначала был умеренный
федералист Андрей Никовский. Но с ослаблением германского влияния, после неудач
на Западном фронте в июле — августе 1918 года ослабли и позиции гетмана.
Национальный Союз с осени 1918 года становился все более радикальным. К нему
присоединились украинские эсдеки и эсеры «центра», «Селянська спилка»,
петлюровский Земский союз. В середине сентябре 1918 года главой Национального
Союза стал эсдек Владимир Винниченко. Высказывая идею «широкого единого
национального демократического фронта», Винниченко уже с сентября 1918 года
стал искать контакты с повстанческими атаманами, надеясь превратить
Национальный Союз в повстанческий центр.

Винниченко и Никита Шаповал тайно от других лидеров Национального Союза пошли
на переговоры с советскими представителями, которые находились в Киеве как
участники переговорного мирного процесса между Гетманской державой и Советской
Россией. Эти представители России (Раковский и Мануильский) надеялись, со своей
стороны, подтолкнуть все оппозиционные силы к восстанию против гетмана и
укрепить большевистское влияние в Украине. Советские представители обещали
заговорщикам из Национального Союза помощь деньгами и оружием. «Дипломаты»
предлагали «организовать военные стычки на российско-украинской границе для
того, чтобы оттянуть гетманские войска от Киева в момент восстания». Они
обещали Винниченко, что в случае победы украинских социалистов Советская Россия
признает новое правительство Украинской республики и не будет вмешиваться во
внутренние дела Украины.

Атаман Григорьев организует на Елизаветградщине отряды из восставших крестьян
для борьбы с австро-немецкими карательными отрядами и «вартой». Первый
повстанческий отряд, около 200 крестьян, Григорьев собрал в селах Верблюжки и
Цыбулево. Первой операцией григорьевцев было нападение на гетманскую полицию
(варту). Вскоре григорьевцы разгромили большой карательный отряд, захватив
четыре пулемета и пушку. Следующая победа принесла большие трофеи. Повстанцы
напали на австрийский военный эшелон на станции Куцивка, захватив большое
количество патронов, гранат, винтовок и пулеметов. Этим оружием Григорьев
вооружил полторы тысячи повстанцев. В начале октября 1918 года Григорьеву
удалось объединить Под своим руководством до 120 мелких повстанческих отрядов
Херсонщины. Однако в конце октября 1918 года григорьевцы потерпели ряд
поражений от войск карателей и были вынуждены отойти на север, к границе
Киевской и Херсонской губерний.

К середине ноября 1918 года Григорьев восстановил свое влияние в степях
Украины. Григорьевцы выбили немцев и гетманцев из Села Верблюжки и Александрии,
после чего Григорьев провозгласил себя «атаманом повстанческих войск
Херсонщины, Запорожья и Таврии», хотя он контролировал тогда один уезд
Херсонщины, а на Запорожье и в Таврии никогда не появлялся (Махно, истинного
предводителя повстанцев Запорожья, возмущало такое самозванство Григорьева). В
начале декабря 1918 года отряды Григорьева, действовавшие против войск гетмана,
вторглись в земли Причерноморья со стороны Вознесенска и Жмеринки. Против пяти
тысяч наступавших повстанцев гетманцы смогли выставить только 545 человек. Этот
заслон был опрокинут, и Григорьев 13 декабря 1918 года разгромил сводные отряды
гетманцев, белых и немецких солдат у станции Водопой, на окраинах Николаева. По
договоренности с немецким командованием войска Григорьева вошли в Николаев.

\iusr{Катерина Сотник}

Понимая, что власть уже не может опираться только на германские штыки, гетман
стал лавировать, искать путь к сохранению власти и к налаживанию союза со
странами-победительницами. Надеясь на формирование новой опоры режиму, гетман
огласил будущую аграрную реформу и выборы в Национальный парламент. Гетман
официально пригласил членов Национального Союза на переговоры по формированию
нового правительства «национального доверия». Винниченко согласился на участие
Национального Союза в формировании нового кабинета министров. 24 октября 1918
года был окончательно сформирован новый министерский кабинет, в котором
Национальный Союз получил только четыре далеко не ключевых портфеля. Однако
после создания коалиционного кабинета министров Винниченко неожиданно огласил,
что Национальный Союз не берет на себя ответственности за политику кабинета и
остается к режиму гетмана в оппозиции. Многим тогда стало ясно, что Винниченко
взял курс не на эволюцию режима, а на революцию.

Винниченко понимал, что восстание необходимо провести немедленно, пока Антанта
не успела определенно высказаться в пользу сохранения режима, пока большевики
не перехватили инициативу восстания, пока у гетмана не было крупных военных сил
(только с 15 ноября 1918 года гетман планировал провести первый призыв в
украинскую армию, для формирования новых частей уже были готовы штабы и
офицерские кадры). Заговорщики рассчитывали и на неизменные восстания
мобилизуемых — спутник большинства насильственных мобилизаций.

***

Национальный Союз требовал от гетмана созвать уже 17 ноября 1918 года
Национальный конгресс с функциями парламента. Но гетман, пообещав созвать
Конгресс, вскоре полностью отказался от идеи его проведения. Он боялся, что
Конгресс выскажется за ликвидацию гетманата и за восстановление республики.

Долгое время невыясненными оставались отношения гетмана с Антантой. Французское
командование осенью 1918 года отрицательно относилось к самому факту
самостоятельности Украины. Французские эмиссары подталкивали гетмана отказаться
от самостоятельности Украины, только в этом случае Антанта готова была
оказывать помощь. Но в начале ноября 1918 года гетман получил из Ясс заявление
французского командования о том, что он должен договориться о сотрудничестве с
широкими украинскими кругами, а возможно и передать власть тем, кто не запятнал
себя сотрудничеством с немецким командованием. Главным противником немцев в
Украине был Петлюра; и уже 11 ноября 1918 года он был освобожден из тюрьмы и
принят министром юстиции и самим гетманом.

10 ноября гетман обратился с воззванием к гражданам, разоблачая планы
возможного восстания и обещая созвать демократический парламент. 11 ноября
казалось, что гетман решился на уступки «украинским кругам», но уже на
следующий день стало ясно, что он готовит совершенно иной поворот. 12 ноября
стало днем окончательного выбора дальнейшего курса. Гетман решил провозгласить
федерацию с Великороссией, запретить подготовку Национального конгресса,
распустить коалиционный кабинет и отправить в отставку семь украинских
министров, настаивающих на созыве Конгресса.

Утром 13 ноября стало известно, что гетман решил ужесточить режим и пойти на
воссоединение с белогвардейской Россией, что коалиционный кабинет будет заменен
кабинетом сторонников ликвидации украинской независимости. В окружении гетмана
проводились консультации по созданию пророссийского кабинета премьера Сергея
Гербеля. В тот же день на тайном заседании Национального Союза решалась судьба
восстания. Восстание уже приобретало характер национальной обороны, и свежая
информация о смене ориентиров Скоропадского подтолкнула ЦК УСДРП и ЦК УПСР дать
согласие на руководство восстанием. Петлюра заявил о своем участии в акции
восстания, был намечен революционный триумвират, который должен был возглавить
новое революционное правительство: Владимир Винниченко, Симон Петлюра, Никита
Шаповал.

\iusr{Катерина Сотник}

14 ноября гетман П.Скоропадский подписал подготовленную ещё несколько дней
назад «Грамоту» — манифест, в котором заявлял, что будет отстаивать «давнее
могущество и силу Всероссийской державы». Гетман призывал к строительству
Всероссийской федерации как первого шага для воссоздания великой России.
«Грамота» ставила крест на всех стараниях по созданию украинской
государственности. Этот документ окончательно оттолкнул от гетмана большую
часть украинских федералистов, украинских военных и сознательную украинскую
интеллигенцию.

Антанта еще не успела определённо поддержать гетмана. Только 22 ноября 1918
года она пообещала помочь гетману войсками и кредитами, де-факто признав
гетманское правление, заявив, что пришлет свои войска в Одессу к 1 декабря 1918
года.

Четырнадцать заговорщиков собрались вечером 14 ноября в кабинете Министерства
железных дорог. Среди заговорщиков присутствовали: директор одного из
департаментов железных дорог А. Макаренко, генерал А. Осецкий, полковники В.
Тютюнник и Н. Аркас, полковник Е. Коновалец с несколькими лидерами сечевых
стрельцов, представители украинских партий эсеров, федералистов,
«самостийныкив» и социал-демократов, и лидеры восстания Винниченко и Шаповал.
Настроение собравшихся было приподнято-нервозное, ведь по всему городу их уже
искали, чтобы арестовать... На собрании было провозглашено начало всеобщего
восстания против гетмана, — сформирована альтернативная власть в Украине.
Петлюры на этом собрании не было. Сечевые стрельцы потребовали введения Петлюры
в состав новой революционной власти — Директории и утверждения его командующим
революционными войсками. Он вполне мог претендовать даже на положение главы
Директории.

Директория имела функции коллективного президента, диктаторскую власть и
формировалась на основе компромисса различных политических сил. Присутствующие
на собрании избрали «директоров» единогласно. Было решено, что Директория
останется у власти только до ликвидации режима Скоропадского, а после победы ее
заменит представительская власть. Кроме Винниченко (эсдека), избранного главой
Директории, и Петлюры (эсдека) — наиболее популярного лидера, в Директорию
вошли малоизвестные в партийных кругах деятели, которые оказались тогда под
рукой: университетский профессор Федор Швец — как представитель эсеров и
«Селянской спилки», адвокат Опанас Андриевский — от
националистов-»самостийныкив», беспартийный служащий Андрей Макаренко — от
железнодорожников. Было заявлено, что три «директора» — Винниченко, Петлюра,
Швец — были главными «директорами», а двое других «вспомогательными».

Коновалец 15 ноября встретился с гетманом Скоропадский и потребовал от него
отказаться от манифеста, расформировать русские отряды, перевести стрельцов в
Киев, собрать Национальный конгресс. Но гетман наотрез отказался от этих
предложений.

Через несколько дней после «Грамоты» гетмана по всей Украине начало
распространяться воззвание Директории, объявившее гетмана «предателем»,
«узурпатором», а новое правительство Гербеля — «реакционным». Их власть
объявлялась недействительной, провозглашалось возрождение власти Украинской
Народной Республики (УНР), а народ призывался к восстанию против гетманского
режима. Одновременно с воззванием Директории появляется отдельное воззвание от
имени Петлюры, в котором он, как верховный главнокомандующий — Головный
(Главный) атаман, призвал всех «солдат и казаков» выступить против
Скоропадского, запрещал, под страхом военного суда, помогать гетману спрятаться
от возмездия. В воззвании были такие слова: «обязанность каждого гражданина,
который живет в Украине, арестовать генерала Скоропадского и передать его в
руки республиканских властей».


***

\end{itemize} % }
