% vim: keymap=russian-jcukenwin
%%beginhead 
 
%%file 11_07_2018.fb.lesev_igor.1.po_bljaheram.cmt
%%parent 11_07_2018.fb.lesev_igor.1.po_bljaheram
 
%%url 
 
%%author_id 
%%date 
 
%%tags 
%%title 
 
%%endhead 
\subsubsection{Коментарі}
\label{sec:11_07_2018.fb.lesev_igor.1.po_bljaheram.cmt}

\begin{itemize} % {
\iusr{Марина Прохорова}
Автомайдан по праву требует своей доли пирога

\begin{itemize} % {
\iusr{Игорь Лесев}
пусть еще потребуют заправляться по полцены... хотя эти как раз могут

\iusr{Марина Прохорова}
А АТОшники требуют бесплатного проезда.

\iusr{Евгений Отовчиц}

Я не понимаю упреков. Да, они сейчас все заезжают НЕРАСТАМОЖЕННЫМИ. Но
абсолютно все публичные объединения евробляхеров выступают ЗА растаможку, ни
разу не слышал "а давайте оставим, как есть, или сделаем просто шару".

Единственное, что возмущает людей - почему за автомобиль, стоимостью 2000-3000
евро нужно платить таможенные пошлины, втрое превышающие его стоимость?!!

Купить товар за 2000 евро и заплатить 1000 евро таможенных сборов - согласны
все.

Еще раз - не оставить, как есть, не сделать растаможку бесплатной, а ввести
справедливые тарифы.

На всякий случай - я автомобиля не имею, покупать не собираюсь, считаю их
вообще вчерашним днем))

\iusr{Игорь Лесев}
\textbf{Евгений Отовчиц} 

вы чудесны в своих рассуждениях. Не нравится норматив растаможки - откажитесь
от бляхи и купите проездной на метро. Не нравятся цены на машины в Украине -
переезжайте в ЕС. Не нравятся действующие законы - избирайте власть, которая
удовлетворит ваши требования. Но законы накуя нарушать?

\iusr{Евгений Отовчиц}
\textbf{Игорь Лесев} 

в данном случае они не нарушают, а используют "дырку", позволяющую не
растамаживать. Вы же сами пишете, и абсолютно правы.

Если автомобиль нарушает Закон (например, разрешенный срок непрерывного
пребывания в Украине) - он подлежит конфискации. Ни много, ни мало.

Вполне справедливо, я считаю.

\iusr{Игорь Лесев}
\textbf{Евгений Отовчиц} 

не выдумывайте. Машины ИЗНАЧАЛЬНО ввозят по фиктивной схеме. Никто из
"бляхеров" не работает в литовских или польских фирмах. Это ложь и обман. Так
что нет здесь никакой "дырки". Это чистой воды коррупция и мерзость.

\iusr{Олег Резник}

Насчёт того, что автомобиль-вчерашний день- полностью согласен. Тем более их
количество перешло разумные пределы. И во всех отношениях. А ещё плюс менталитет
нашего товарыства( если что, машина у меня есть, ищу случая как избавиться).

\iusr{Александр Вееруга}
\textbf{Евгений Отовчиц}, 

бляхеры нарушают закон. То, что они массово существуют как явление - это воля
власти, сознательное попустительство.

Зачем - другой вопрос.

\iusr{Евгений Отовчиц}
\textbf{Александр Вееруга} это утверждение не имеет смысла.

\iusr{Евгений Отовчиц}
\textbf{Александр Вееруга} они НЕ получают автомобиль в свою собственность. А управлять чужим автомобилем никаким законом не запрещено.

\iusr{Александр Вееруга}

Можно

а) подгрузить ментам к базе угонов и проч. данные с таможни о просроченных
бляхах;

б) расширить список требований к пресечению границы, т.е., водитель обязан
подтвердить свое сотрудничество с работодателем договором, банковскими
справками о зарплате, страховым договором, документами по уплате социальных
взносов и прочее;

в) данные водителя привязать к авто и залить в ту же базу полиции, и учтите -
любые доверенности и передача управления третьему лицу - нарушение закона,
Стамбульская конвенция это не дырка в законе, это реальный механизм для
упрощения ведения бизнеса и только.

Это навскидку. Можно усложнять квест, не выходя за рамки конвенции.

И вишенка на тортик - похерить конвенцию. Кто-то скажет, что это гевол и
алярм)) Ничего подобного. В истории ЕС масса примеров, когда его члены херили
ратифицированные акты. Повоняют и перестанут.


\iusr{Евгений Отовчиц}
\textbf{Александр Вееруга} 

а дешевле и проще всего - можно упростить и удешевить процедуру растаможки. О
чем, собственно, и речь. Купил за 2000, заплатил еще 1000 в казну, пошел за
регистрацией на укр. номерах.

Вы считаете, что все автомобили на европейских номерах нарушили законные сроки пребывания в Украине?

\iusr{Александр Вееруга}
купил за 2000, заплатил еще 1000 в казну))
И завтра соберётся майдан за право растоможки за 200 евро.
Потому что грабительскую тысячу народ не в силах платить))
ОК, отменим растаможку. Исчезнет доходная статья бюджета. Чем компенсируете?

\iusr{Евгений Отовчиц}
\textbf{Александр Вееруга} перестаньте. Нет 3000 евро - не покупай за 2000.
Слушайте, 50\% от стоимости товара - не достаточная растаможка? Обязательно нужно 400\% отдавать государству?
Вас немного заносит.
Окей. Давайте оставим, как есть. С растаможкой "0".

\iusr{Александр Вееруга}
Акциз на табачные изделия, впятеро от цены нового изделия, вот где грабёж. Айда на майдан?)))
Такова структура сборов, ничего не поделаешь.
Хотите снизить акциз - найдите чем компенсировать или завтра закроют школу, в которой учатся ваши дети.

\iusr{Евгений Отовчиц}
\textbf{Александр Вееруга} да у вас в голове перемешалась причинно-следственная связь)))
Автомобиль - акцизный товар?)))
Окей. Я понял. Давайте, значит, на ВЕСЬ импорт 400\% пошлины. На телефоны, ноутбуки, телевизоры, китайские инструменты, итальянские макароны, немецкие шоколадки и английский чай.
И лекарства.
Вы же справедливости хотите?)))

\iusr{Александр Вееруга}

Насчёт "меня заносит". Лично я вообще считаю, что накануне выборов бляхеры
пропихнут растаможку своих корыт по двести евро. Потому что триста тысяч семей
это миллион голосов. За их голоса будут бороться и дадут всё что угодно.

Не сегодня. Но скоро.

Это не меня заносит, это страну колбасит.

\iusr{Евгений Отовчиц}
\textbf{Александр Вееруга} растаможка даже за 200 евро - это больше, чем сегодня. Сегодня они платят кому угодно, но не в бюджет.

\iusr{Александр Вееруга}
Евгений, так сложилось, автомобиль облагается заоблачными сборами. В Сингапуре всё ещё хуже и ничего, процветают. У них тоже так сложилось.
Когда и если в бюджете будут лишние деньги, будем вместе требовать снизить налоговый пресс.
Я вообще за отмену НДС. Та ещё кормушка для пузатых пиджаков.

\iusr{Евгений Отовчиц}
\textbf{Александр Вееруга} с отменой НДС совершенно согласен.
Но компенсаторы в данном случае вас не волнуют, а только с бляхами?))
Безусловно, необходимо ослаблять налоговое давление на заработанные деньги. И повышать (затем) пошлины на импорт, тарифы и т.д.
Я думаю, многие согласятся платить коммуналку 5000 грн с зарплатой 25000 грн, чем 1000 с 6000, хотя в первом случае это 20\% от дохода, а во втором - 15\%.
Человек, при зарплате 10000 грн на руки уплачивает еще столько же в виде налогов, точнее, работодатель на него уплачивает. Потом из этой половинки зарплаты 20\% уходит на коммуналку, и еще 18\% - НДС при покупке товаров и услуг. Итого реально государство из фактических 20000 грн забирает около 14000. И только 6000 остаются в обращении между субъектами экономики.
Так ничего никогда не вырастет, не разбогатеет и не сможет покупать новые автомобили))

\end{itemize} % }

\iusr{Елена Несветайлова}
"Чем тебя породил, тем тебя и убъю.") Одним хочется бабла со всех каналов, другим хочется их не давать. А вообще - дурь-ситуация.

\iusr{Дмитрий Коломийченко}
Ситуация с бляхерами прекрасно иллюстрирует состояние государства и его перспективы в случае продолжения инерционного курса.

\iusr{Светлана Соколова}
Всякого рода привилегии — это могила для свободы и справедливости

\iusr{Александр Власенко}
Так анархия - это же прекрасно. Это хаос, в котором лишь и способен родиться какой-то порядок.

\iusr{Матвей Кублицкий}

Изначально тема с пошлинами на авто была логичной. копировали российское
правительство, которое благодаря высоким пошлинам на ввоз авто поддержало
отечественных производителей и стимулировало множество импортных автокомпаний к
разворачиванию производств в России. Только на Украине это не сработало. И
сейчас эти пошлины выглядят нелепыми и глупыми. Посмотрим что дальше будет

\begin{itemize} % {
\iusr{Игорь Лесев}
у нас только успевай следить... каждый раз что-то новое можно упустить

\iusr{Матвей Кублицкий}
\textbf{Игорь Лесев}  @igg{fbicon.laugh.rolling.floor}{repeat=3} у вас всё новое - такое шумное - хрен упустишь... я с белорусами сравниваю - там всё тихо и под ковром
\end{itemize} % }

\end{itemize} % }
