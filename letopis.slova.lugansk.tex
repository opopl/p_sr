% vim: keymap=russian-jcukenwin
%%beginhead 
 
%%file slova.lugansk
%%parent slova
 
%%url 
 
%%author 
%%author_id 
%%author_url 
 
%%tags 
%%title 
 
%%endhead 
\chapter{Луганск}

– Ну и вопросы у Вас... Тут ведь как – когда тебе рассказывают, что ты дурак,
ты сам начинаешь в это верить. Пропаганда работает на все 100 процентов. Тут
ведь как считают? Во всем виноват Майдан. В \emph{Луганск} перся «Правый сектор», в
Станице прямо эшелоны останавливали. Но Вы же понимаете, все с телевизора...
Украина говорит, что Россия виновата. Россия – наоборот. Гонят друг друга и
результата нет. У меня большинство друзей уехали... Я этого не понимаю. Я живу в
своем доме – это главное. Да, теперь в кошельке русские деньги. Но как
говорится ведь, деньги не пахнут. Лишь бы они были,
\textbf{«Никто нам не вернет украденные годы»: как в \emph{Луганске} вспоминают захват города},
Донбас.Реалії, Записки з окупації, radiosvoboda.org, 02.06.2021

А я вирішив іти на війну, коли подивився сюжет, який мене вразив. Він був про
те, як у \emph{Луганській} області місцеві жителі пікетують якусь військову частину. І
показують срочника в радянській касці і в якомусь засраному броніку, з
автоматом. А потім його якийсь алкоголік витягує з воріт і здирає з нього каску
зі словами: \enquote{Снимай это говно}. Я подумав тоді, що, бля, ми б на Майдані за
каску, бронік, автомат вбили б, бо нічого з цього у нас не було. А тут якийсь
лиган, якого можна просто лящами прогнати, у пацана все забирає. Тоді я
зрозумів, що, мабуть, Збройним силам треба трохи допомогти, 
\textbf{Доброволець Олександр Воробєй: \enquote{Війна не змінює людину, вона
просто розкриває її сутність. Якщо людина була гандоном, то після фронту вона
стане махровим п\#дарасом}},
Віка Ясинська, censor.net.ua, 02.06.2021


Багатьом \emph{луганчанам} у чужих стінах найманих квартир сниться дуже схожий сон. Ми
блукаємо вночі \emph{Луганськом} серед знайомих місць, вулиць і провулків, парків,
вокзалів, будинків і кав’ярень. Усе це перемішується в фантасмагоричний
лабіринт. Комусь сниться рідна оселя, яку він не бачив сім років, а у когось
марення зациклюється на одному сценарії - доходиш до власної квартири і не
можеш до неї потрапити. Начебто ось двері, ось замок, але ключ або не
встромляється, або заклинює... І головна думка, що переслідує серед навіяних
образів, лише одна - як безперешкодно вибратись звідси? Саме так – назад,
подалі від того, що належить тобі. Потрапивши в своє місто, дійшовши до свого
дому, діставшись дверей квартири, втекти геть. Втекти в суцільну хмару ночі з
поодинокими сумними ліхтарями, які в пам'яті асоціативно освіжать забуті
свідомістю місця. Так, забуті, але уві сні вони на диво чітко проступають, як
колись у дитинстві чорно-білі контури на фотопапері, яку в червоній напівтьмі
занурюєш у хімічний розчин «проявлювача»,
\textbf{Позахмарний Київ}, Валентин Торба, day.kiev.ua, 30.05.2021

Це сон довжиною вже в сім років. Це сон, який належить багатьом, адже містичним
чином сниться всім, хто полишив рідне місто, яке ми недолюбили, бо або не
встигли, або не змогли. До нього можна було звикнути, поселити в нього свої
юнацькі враження і переживання, з ним можна було співіснувати й мати справу,
але любити його, як люблять ті, хто присвячує своїй малій батьківщині пафосні
вірші, нереально. Місто як даність, як доля, як приреченість і як підсвідомий і
водночас реальний переслідувач, що тримає в заручниках твою пам'ять, твої
пов’язані з ним переживання і полишені речі. Нав’язане місто, куди тебе наче
підкинули, як кошеня в бетонний підвал, і тепер мусиш приймати його і себе
таким, як є, бо там народився і від того мав пізнати свою Україну ще й з такого
боку, а не лише мальовничо-казкову, квітучо-вишиванкову, поетико-мрійливу. Мій
\emph{Луганськ}, назва якого химерно, але мило асоціюється з простором і свободою і
який багато разів навмисно перейменовували в протезне, штучне і гуркітне, наче
постріл з дробовика, - \emph{Ворошиловград},
\textbf{Позахмарний Київ}, Валентин Торба, day.kiev.ua, 30.05.2021

Натомість, коли десь на початку 2000-х мені вперше довелось їхати до зовсім
незнайомого Києва у відрядження, я незбагненним чином відчував, наче...
повертаюсь додому. Це також містика. Так само, як і інший сон зі школи, де
вчителька історії, яка ще в сьомому класі «середньої школи номер ...надцять м.
\emph{Луганська}» доводила мені, що «Мазепа – ворог і зрадник», залишила мене на
незнайомому київському вокзалі з купою перонів без зворотного квитка, а сама
прудко перекинулась через височенну бетонну стіну, не зачепившись навіть за
«вінок» колючого дроту,
\textbf{Позахмарний Київ}, Валентин Торба, day.kiev.ua, 30.05.2021

