% vim: keymap=russian-jcukenwin
%%beginhead 
 
%%file 06_05_2021.fb.respublikalnr.1.faina_savenkova_kiev_pamjat
%%parent 06_05_2021
 
%%url https://www.facebook.com/groups/respublikalnr/permalink/807979043171222/
 
%%author 
%%author_id 
%%author_url 
 
%%tags 
%%title 
 
%%endhead 
\subsection{Детский взгляд. Фаина Савенкова:  Забытые судьбы}
\label{sec:06_05_2021.fb.respublikalnr.1.faina_savenkova_kiev_pamjat}
\Purl{https://www.facebook.com/groups/respublikalnr/permalink/807979043171222/}

\ifcmt
  pic https://scontent-bos3-1.xx.fbcdn.net/v/t1.6435-9/182484385_130325029147605_7564097943587029491_n.jpg?_nc_cat=109&ccb=1-3&_nc_sid=825194&_nc_ohc=bJNBqAxG3zUAX9pSB5Q&_nc_ht=scontent-bos3-1.xx&oh=6745ddb9ec5136306e54f04880778497&oe=60BAE263
\fi

Листая старые семейные альбомы, я вглядываюсь в лица на пожелтевших от времени
фотографиях. Обычные, ничем не примечательные люди. Многих из тех, чью жизнь
запечатлела бесстрастная пленка, я даже не знаю. Мне неизвестно по какой
причине они оказались в тот застывший момент прошлого рядом с моими
родственниками, какие дороги свели их вместе. Разные судьбы, почти забытые и
затерявшиеся где-то на страницах истории, которую они создавали все вместе,
подчас жертвуя ей свои жизни.

Моя прабабушка родилась и провела детство намного западнее Луганска – в
Винницкой области УССР. Она и две её сестры повзрослели слишком рано. Дети,
оставшиеся без родителей и выжившие в нечеловеческих условиях, – жестокая
обыденность военных лет. Да и послевоенных тоже… Ведь там, где жила семья
родного мне человека, после освобождения территорий от немецко-фашистских
захватчиков ещё долго орудовали банды УПА. 

Моей прабабушки Лидии Петровны нет с нами уже 2 года. А в 2014-м, во время
просмотра новостного сюжета о бандеровском факельном шествии в Киеве, она
сказала только: «Что ж их всех в сорок пятом-то не добили?»  Почему? Все
просто: в её памяти оживали тревожные события детских лет. Она часто
вспоминала, как те, кого в современной Украине называют «героями», держали
людей в страхе и напряжении. Она, ребёнок войны, рассказывала, как однажды
бандиты, грозившие вырезать пол деревни, из-за тумана не дошли до их
населённого пункта. Кто знает, возможно, из-за этого обстоятельства моя
прабабушка осталась жива. И могу жить я.

Сейчас мне сложно понять, как так получилось, что усилия многих свидетелей
Великой Отечественной по восстановлению мира на родной земле, рассыпались в
прах и обесценились всего за несколько лет. Разве возможно оставаться
безразличными, держа в руках старые фотоснимки своих прадедушек и прабабушек?
Неужели не возникает желание понять, почему же они стали такими, какими мы их
помним: сильными, любящими, созидающими и радующимися простым вещам? Люди,
которые сегодня отрицают, переписывают историю по своему усмотрению, не предают
ли они не только своих предков, но и себя?..

На самом деле у меня нет ответов на эти вопросы. Думаю, нет их и у многих
других. Но что у меня есть, так это вера в то, что ошибки настоящего будут
исправлены, пока еще не поздно. И в Киеве снова будут чтить память тех, кто
освобождал этот город, а не тех, кто убивал его жителей. Мы должны сохранять
связь поколений, помнить подвиги настоящих героев-победителей Великой
Отечественной войны! Помнить ради справедливости, ради счастья наших детей. Оно
видится мне в стремлении к миру и созиданию, а не в стремлении все разрушить.
Пожалуй, это самое главное, что нужно вспомнить в преддверии праздника Победы.
И больше никогда об этом не забывать!

Фаина САВЕНКОВА, 12 лет, г. Луганск, ЛНР
