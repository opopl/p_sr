% vim: keymap=russian-jcukenwin
%%beginhead 
 
%%file 21_09_2021.fb.bereza_borislav.1.kafe_kiev_beseda_zelenskii
%%parent 21_09_2021
 
%%url https://www.facebook.com/borislav.bereza/posts/6863827923643255
 
%%author_id bereza_borislav
%%date 
 
%%tags future,imperia.rimskaja,kiev,politika,prezident,ukraina,zelenskii_vladimir
%%title Кафе в центре Киева - беседа - Масштабность Зеленского
 
%%endhead 
 
\subsection{Кафе в центре Киева - беседа - Масштабность Зеленского}
\label{sec:21_09_2021.fb.bereza_borislav.1.kafe_kiev_beseda_zelenskii}
 
\Purl{https://www.facebook.com/borislav.bereza/posts/6863827923643255}
\ifcmt
 author_begin
   author_id bereza_borislav
 author_end
\fi

Кафе в центре Киева. За соседним столом сидят двое. Одному около 60, а второму
не более 30. Тот, что постарше, в тёмном вельветовом пиджаке, светлой рубашке и
темно-бордовом шейном платке. Его же собеседник в свитере с надписью Армани.
Молодой доказывает, что люди сейчас не в состоянии понять масштабность
Зеленского. Рассказывает о том, что тот много хочет достигнуть, но мешают козни
предшественников, отсутствие полномочий и слабые кадры на местах.

- Так кадры он же сам расставляет, - парирует тот, что постарше. 

- А кого выбрать? Либо верных дураков, либо враждебных профи. Ты бы кого
выбрал? Ведь тех, кто верен, - отвечает ему молодой. 

- Знаешь, Владушка, твой Зеленский сильно мне напоминает императора Западной
Римской империи Ромула Августа, - вдруг говорит старший собеседник и делает
глоток вина.

- Чем? Я же не силен в истории, папа. Ему тоже пришлось нелегко?

- Ему? Он был последним императором. На нем история Древнего Рима закончилась.
Его имитация правления длилась менее года и привела к тому, что его свергли и
на этом Римская Империя прекратила свое существование. 

- Зеленский уже более двух лет при власти, папа. Какая-то неуместная аналогия. 

- Видимо запас прочности Украины чуть больше. Но посмотрим, что будет дальше. А
я возвращаюсь в Лондон. Когда у вас тут все на@бнется прилетай. 

- Пап, не на@бнется.

- Зря ты, Владушка, историю не любишь. Зря. Но не буду мешать тебе получать
опыт через набивание шишек. Ты, кстати, не ешь. Все уже остыло, - закончил
старший собеседник свою тираду.

Мне очень хочется, чтоб старший собеседник оказался не прав, а в их споре
победил молодой. Но пока все наоборот. И дело не только в том, что я читал
Гиббона и знаю историю падения Римской империи, но и в том, что при Зеленском
Украина действительно превращается в какое-то квазигосударство с неработающими
институциями. И пока все движется к тому, что Украина может и не пережить
правление величайшего политического лидера современности.

\ii{21_09_2021.fb.bereza_borislav.1.kafe_kiev_beseda_zelenskii.cmt}
