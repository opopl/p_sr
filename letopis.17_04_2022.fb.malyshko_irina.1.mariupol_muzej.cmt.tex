% vim: keymap=russian-jcukenwin
%%beginhead 
 
%%file 17_04_2022.fb.malyshko_irina.1.mariupol_muzej.cmt
%%parent 17_04_2022.fb.malyshko_irina.1.mariupol_muzej
 
%%url 
 
%%author_id 
%%date 
 
%%tags 
%%title 
 
%%endhead 
\zzSecCmt

\begin{itemize} % {
\iusr{Оксана Ушакова}

В Мариуполе уничтожен и музей одного из моих любимых художников - Куинджи. Ты
не упоминула в числе тех, кто должен оплакивать судьбу этого города-героя
русских. Но я хочу сказать, что каждый кадр оттуда рвёт душу. И особенно
страшно, что это творит моя страна. И я не знаю как жить с этим...

Каждый день этой проклятой войны я плачу. Каждый день — уничтожение людей,
городов, культуры, истории. Искажается всё  @igg{fbicon.man.facepalming}   перечеркивается. Страшно,
больно, кошмарно. Я ненавижу происходящее. И мою страну тоже уничтожает наша
 @igg{fbicon.face.symbols.mouth}{repeat=3}  власть. История не прощает такого

\begin{itemize} % {
\iusr{Irina Malyshko}
\textbf{Оксана Ушакова} 

Краеведческий музей включал три музея - сам главный, с уникальной библиотекой и
аохивом, у них была публицистика по Мариупрьскому уезду и реиональные газеты
20х. Плюс отделно - етногоафический музей, и музей Куинджи. Я обобщила. Я не
знаю как с этим жить даже, нам. А как вам тем более не знаю. Как со всем этим
вообще можно дальше жить

\iusr{Vera Skvirskaja}
\textbf{Irina Malyshko} 

Главное выжить украинцам, а все музеи отстроят и восстановят. И если что,
сделают цифровые экспозиции по новым технологиям, и не будет ничего потеряно.
Главное, что бы остались живы наши люди, которые смогут это смотреть. Берегите
себя.

\iusr{Irina Malyshko}
\textbf{Vera Skvirskaja} 

выжить, это вне всего, это первое и главное. У меня в Мариуполе до сих пор
близкие мне люди так и не вышли на связь. И я все равно жду и верю, что они
спасутся. Но есть и культурные ценности, которые не восполнимы. Утрачены целые
пласты истории. И это не этно-одежды или фото, там были уникальные экземпляры

\end{itemize} % }

\iusr{Vira Baldyniuk}
Ох, як недарма ти займалася історією цього краю. І яка пустка від знищеного.

\iusr{Irina Malyshko}
\textbf{Vira Baldyniuk} так, але зараз розумію як мало зробила, щось завжди відклаємо на завтра, яке вже ніколи не натупить, нажаль

\end{itemize} % }
