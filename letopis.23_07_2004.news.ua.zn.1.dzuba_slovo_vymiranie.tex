% vim: keymap=russian-jcukenwin
%%beginhead 
 
%%file 23_07_2004.news.ua.zn.1.dzuba_slovo_vymiranie
%%parent 23_07_2004
 
%%url https://zn.ua/ART/vymiranie_slova.html
%%author 
%%tags 
%%title 
 
%%endhead 

\subsection{Вымирание слова}

\Purl{https://zn.ua/ART/vymiranie_slova.html}
\Pauthor{Дзюба, Иван}

\textbf{Казалось бы неуместно говорить о вымирании Слова в эпоху его гиперпродукции,
небывалой экспансии и агрессивности, когда все мировое пространство
загромождено человеческими голосами и письменами...}

Казалось бы неуместно говорить о вымирании Слова в эпоху его гиперпродукции,
небывалой экспансии и агрессивности, когда все мировое пространство
загромождено человеческими голосами и письменами. Но имеется в виду не сумма
продукции органов звукообразования и графической аппаратуры, не те мертвые
слова, которые господствуют в «структуре владения» (Э. Фромм), а Слово с
большой буквы, которое связывает человека с глубинами бытия — в измерениях
космическом, историческом, этнокультурном, личностном. Слово не просто как член
предложения, а как носитель мудрости, софийности, в значении, близком к
древнегреческому «Логос». Конечно, это слово еще не совсем исчезло, но его все
жестче вытесняют в узкие специальные сферы — философии (она как раз в ХХ в.
дала особенно глубокие анализы феномена языка!), культурологии, этнологии,
элитарной поэзии и тому подобного, — и парадоксальным образом именно в этом
якобы очень благоприятном контексте, но под действием других факторов (Жиль
Липовецки: «пустыня побеждает») Слово перестает быть реальным духотворящим
началом в человеческих сообществах. Именно в этом смысле и приходится говорить
о вымирании Слова, и этот процесс, на мой взгляд, имеет несколько измерений.

В первую очередь бросается в глаза (или в уши?) утрата Словом сакральности. Под
сакральностью я здесь понимаю не только его непосредственную принадлежность к
сфере религиозных таинств (хотя в этом свойственном значении отход от
сакральности, ее формализация особенно очевидны — учитывая угасание религиозных
чувств в христианском мире и угрожающую политизацию их в мире мусульманском), —
а понимаю вообще власть Слова над человеком как силы внеиндивидуальной и
сверхрациональной, мироупорядочивающей, данной от богов, от космических
энергий, от великих поэтов.

Мирча Элиаде считал, что сакральным миром был мир кочевников и оседлых
земледельцев, а мы живем в десакрализированном мире. Однако и в
десакрализированном мире, по-видимому, могут сохраняться какие-то реликты былой
сакральности. По крайней мере в Слове, особенно в Слове поэзии. Божественная
природа Слова поэта, как и Слова жреца, не вызывала сомнений в патриархальном
обществе. «Кто я такой, чтобы носить это жгучее пламя на своей непокрытой
голове?» — говорит о своей миссии жрец Езеулу в романе Чинуа Ачебе «Стрела
Бога». О великом скальде Оулавюре Каурасоне в романе Ласнесса говорится, что он
«держит за хвост молнию». Великие поэты и пророки владели теми «глаголами
вечной жизни», о которых говорится в Евангелии. «Не хотите ли и вы уйти?» —
спрашивает Иисус Христос у своих апостолов. «Господи! К кому же нам идти? У
тебя есть глаголы вечной жизни».

Эти глаголы — радикальное отрицание профанного языка, и шестикрылый серафим
должен вырвать грешный язык у героя Пушкинского «Пророка» и «десницею кровавой»
вложить ему «жало мудрыя змеи», прежде чем прозвучит ему Божий приказ: «…И
обходя моря и земли, Глаголом жги сердца людей». Шевченковский Перебендя
разговаривает с Богом, скрываясь от людей, поскольку «на Божеє слово вони б
насміялись» (самоощущение, вроде бы противоположное пророческому). Но это
очуждение проходит под давлением непреодолимой потребности, в которой Шевченко
молит Богоматерь:

Пошли мені святеє слово…
Скорбящих радосте! Пошли,
Пошли мені святеє слово,
Святої правди голос новий!
 23 июля, 2004, 00:00
Распечатать
Выпуск № 29, 23 июля-30 июля 2004г.

    Иван Дзюба

Автор

    Иван Дзюба

Статьи авторов

    Може, саме тепер формується політична українська нація, про яку весь час говорилося В інтерв'ю DT.UA Іван Дзюба поділився думками про українську політику та її діячів, російську пропаганду, причини невдачі Народного Руху та формування національної самобутності українців.
    Может, именно теперь формируется политическая украинская нация, о которой все время говорилось В интервью ZN.UA Иван Дзюба поделился мыслями об украинской политике и ее деятелях, российской пропаганде, причинах неудачи Народного Руха и формировании национальной самобытности украинцев.
    Пятьдесят лет назад... При всей своей импульсивности Хрущев был неизменно безошибочен в определении направления идеологических ударов, и общегосударственная облава на "формалистов", "абстракционистов", "дегтемазов" не была его глупой прихотью

Все статьи автора
Все авторы
Поделитесь с друзьями
Комментировать

Казалось бы неуместно говорить о вымирании Слова в эпоху его гиперпродукции, небывалой экспансии и агрессивности, когда все мировое пространство загромождено человеческими голосами и письменами...

Казалось бы неуместно говорить о вымирании Слова в эпоху его гиперпродукции, небывалой экспансии и агрессивности, когда все мировое пространство загромождено человеческими голосами и письменами. Но имеется в виду не сумма продукции органов звукообразования и графической аппаратуры, не те мертвые слова, которые господствуют в «структуре владения»
(Э. Фромм), а Слово с большой буквы, которое связывает человека с глубинами бытия — в измерениях космическом, историческом, этнокультурном, личностном. Слово не просто как член предложения, а как носитель мудрости, софийности, в значении, близком к древнегреческому «Логос». Конечно, это слово еще не совсем исчезло, но его все жестче вытесняют в узкие специальные сферы — философии (она как раз в ХХ в. дала особенно глубокие анализы феномена языка!), культурологии, этнологии, элитарной поэзии и тому подобного, — и парадоксальным образом именно в этом якобы очень благоприятном контексте, но под действием других факторов (Жиль Липовецки: «пустыня побеждает») Слово перестает быть реальным духотворящим началом в человеческих сообществах. Именно в этом смысле и приходится говорить о вымирании Слова, и этот процесс, на мой взгляд, имеет несколько измерений.

В первую очередь бросается в глаза (или в уши?) утрата Словом сакральности. Под сакральностью я здесь понимаю не только его непосредственную принадлежность к сфере религиозных таинств (хотя в этом свойственном значении отход от сакральности, ее формализация особенно очевидны — учитывая угасание религиозных чувств в христианском мире и угрожающую политизацию их в мире мусульманском), — а понимаю вообще власть Слова над человеком как силы внеиндивидуальной и сверхрациональной, мироупорядочивающей, данной от богов, от космических энергий, от великих поэтов.

Мирча Элиаде считал, что сакральным миром был мир кочевников и оседлых земледельцев, а мы живем в десакрализированном мире. Однако и в десакрализированном мире, по-видимому, могут сохраняться какие-то реликты былой сакральности. По крайней мере в Слове, особенно в Слове поэзии. Божественная природа Слова поэта, как и Слова жреца, не вызывала сомнений в патриархальном обществе. «Кто я такой, чтобы носить это жгучее пламя на своей непокрытой голове?» — говорит о своей миссии жрец Езеулу в романе Чинуа Ачебе «Стрела Бога». О великом скальде Оулавюре Каурасоне в романе Ласнесса говорится, что он «держит за хвост молнию». Великие поэты и пророки владели теми «глаголами вечной жизни», о которых говорится в Евангелии. «Не хотите ли и вы уйти?» — спрашивает Иисус Христос у своих апостолов. «Господи! К кому же нам идти? У тебя есть глаголы вечной жизни».

Эти глаголы — радикальное отрицание профанного языка, и шестикрылый серафим
должен вырвать грешный язык у героя Пушкинского «Пророка» и «десницею кровавой»
вложить ему «жало мудрыя змеи», прежде чем прозвучит ему Божий приказ: «…И
обходя моря и земли, Глаголом жги сердца людей». Шевченковский Перебендя
разговаривает с Богом, скрываясь от людей, поскольку «на Божеє слово вони б
насміялись» (самоощущение, вроде бы противоположное пророческому). Но это
очуждение проходит под давлением непреодолимой потребности, в которой Шевченко
молит Богоматерь:

Пошли мені святеє слово…
Скорбящих радосте! Пошли,
Пошли мені святеє слово,
Святої правди голос новий!
І слово розумом святим
І оживи, і просвіти!
Молю, ридаючи, пошли,
Подай душі убогій силу,
Щоб огненно заговорила,
Щоб слово пламенем взялось,
Щоб людям серце розтопило.
І на Украйні понеслось,
І на Україні святилось
Те слово, Божеє кадило,
Кадило істини…

Неудивительно, что сегодня поэт так говорить не может. Наверное, Артюр Рембо
был среди последних, кто думал, что «с помощью метафоры можно изменить мир».
Вопрос в другом: какую власть имеют «Божьи слова» великих поэтов прошлого над
современной публикой? Ведь если верить Леви-Строссу, «слова — это знаки»,
перестав быть ценностью (правда, у него это утверждение направлено и против
поэтов). А Гастон Башляр утверждает: «Слово утратило свое бытие, оно есть
мгновение особой семантической системы». Это и есть потеря сакральности Слова —
в пользу операционности.

Но и христианство, возможно, в чем-то ограничило полномочия и возможности
Слова. С одной стороны, Слово — первокреатив Мира, тот первовзрыв, творческий
акт, с которого все началось, в котором (Слове) все запрограммировано (Мир как
текст?). (Михаил Максимович: «…Через слово мы достигаем Того, Кто дал нам Слово
вместе с жизнью и светом, с Чьего всемогущего, первородного Слова возник
мир…»). Но с другой стороны — в христианстве проигнорировано то глубокое
понимание, или, скорее, переживание отношений человека и природы, которое было
в языческих религиях. Слово стало великим разговором между Богом и человеком,
но перестало быть разговором между человеком и природой — деревом, птицей,
зверем, то есть Миром вообще. Затем этот разговор возобновился в рамках наук,
но это был уже разговор рацио, а не души, и, собственно, уже не диалог, а,
скорее, анализирующий монолог человека.

Далее можно говорить об уменьшении миропознавательных возможностей и
мирообъясняющей способности Слова. До эры развитых технологий именно в Слове и
с помощью Слова осуществлялись познавательные операции и именно Слово несло
груз знаний о мире. С развитием научных исследований Слово все больше уступало
математической формуле, геометрической функции, статистическому графику.
Сегодня в создании фундаментальной картины мира слову принадлежит только
вспомогательная роль. Возможно, это временное состояние познавательного
прогресса человечества; возможно, вообще все не так, так только кажется. Мартин
Хайдеггер писал: «Наука не мыслит. Она не мыслит, поскольку ее образ действия и
ее средства никогда не дадут ей мыслить — мыслить так, как мыслят мыслители.
Только это дает ей возможность войти в нынешнюю предметную сферу и поселиться в
ней». То есть наука принадлежит к иному роду мышления — тому, которое он
называл вычисляющим. Собственно, вот его мнение:

«Вычисляющее мышление — это не осмысливающее мышление, оно не способно подумать
о смысле, который господствует во всем, что есть. Следовательно, есть два вида
мышления, причем существование каждого из них необходимо для определенных
целей: вычисляющее мышление и осмысливающее размышление». Очевидно,
«осмысливающее размышление» — это миссия философии. Хотя «на виду» ныне не
философия, а таки наука…

Если — в этом контексте — о мирообъясняющей миссии Слова еще можно спорить, то,
на мой взгляд, более очевидным является обеднение его жизненно-предметного
наполнения. Оно обусловлено обеднением естественной жизни современного
человечества и, соответственно, языка. Городской быт — а он фактически уже
доминирует во всех так называемых развитых обществах — означает
катастрофическое сужение и выхолащивание непосредственной причастности к жизни
животного и растительного мира, потерю ощущения земли как основы своего бытия,
— отсюда и отмирание целых пластов языка. Человек уже не способен адекватно
поименовать мир в его хоть и исчезающем, но все еще существующем богатстве. А
без наименования нет владения. За исключением очень узкого круга очень узких
специалистов, — что в этом случае не имеет значения, поскольку говорим об
общенациональной речи, а не профессиональной терминологии, — никто уже ничего
не знает о том, что растет на земле, бегает или ползает по земле, роется в
земле, летает в воздухе. От тысяч конкретных имен скоро останутся только общие
понятия: это птица, это дерево, это зверь, это или цветок, или сорняк, —
лишенные чувственной и эмоциональной наполненности, конкретных образных
представлений и ассоциаций. (Я уж не говорю о забвении символических значений
многих и многих имен природного мира, которые остались в языках опустевшей
звуковой формой.)

И эти потери ничем не компенсировать, в отличие, скажем от потери определений
тысяч трудовых операций, ушедших в прошлое, — на смену им пришли тысячи других.
Хотя и здесь равноценной компенсации нет и не будет: работа за
станком-автоматом или на панели дистанционного управления какой-то
чудо-техникой не требует того очеловечивания, той степени идентичности субъекта
и орудия труда, как, скажем, вспахивание земли плугом или косьба, —
следовательно, и не сможет вызвать равноценные переживания и спровоцировать
поэтическую образность, равноценную той, которая много веков наполняла
человеческий язык. Соответственно, теряется восприятие тех предметных и
чувственных смыслов, которые несла эта образность, пронизывая смысловое поле
речи глубинными ассоциациями. «Ночами небо пахнет, как посев», — писал когда-то
Назым Хикмет. Что скажет это сравнение человеку, который никогда не стоял на
распаханной земле, держа в горсти зерно, а в небо смотрит, только чтобы увидеть
пролетающий реактивный самолет или фейерверк по случаю какой-нибудь
общественнозначимой глупости? Но ведь именно такой стандарт человека воцаряется
в современном цивилизованном мире! Когда парагвайский поэт Эльвио Ромео
сравнивает высохшую землю со слезой — сухость с влажностью! — то такое
парадоксальное сравнение могло родиться только из личного переживания засухи
как самого горького соленого бедствия и из личной причастности к этой трагедии
хлеборобского люда. Современному читателю это нужно объяснять, а объясняемое
теряет силу непосредственного впечатления.

Обедненность представления природного мира в душевной жизни человека упрощает,
опресняет психологический механизм его реакций и усиливает поведенческую
агрессивность. Это, в свою очередь, стимулирует вульгарность высказывания,
порождает стилевую асимметрию речи. Хотя работают на это, разумеется, и другие
факторы.

Антрополог Эдвард Холл говорил, что «человек изобрел инструменты, которые
являются продолжением почти всех частей его тела». Под этим углом зрения он
оценивал технические достижения человечества: транспортные сети заменили ноги и
спины; одежда и дома — производные от теплорегулирующих механизмов
человеческого тела; эволюция оружия начинается с зубов и кулаков и
заканчивается атомной бомбой. Если так, то, очевидно, наиболее интенсивно
человечество «продолжало в пространство» свои голосовые связи: азбука,
письменность, книгопечатание, телефон, радио, мегафон, компьютер, Интернет… что
дальше? Однако это фантастическое расширение пространства речи умножало
возможности не только Слова, но и, так сказать, Антислова. Соответственно с
тем, как всякое человеческое изобретение является продолжением (пользуясь
упомянутым определением Эдварда Холла) не только добрых, но и злых возможностей
человека.

В системе речи испортился тормозной механизм, который — как и во всякой системе
— отвечает за безопасность движения. В прошлом человечество, зная и о
живительной, и о разрушительной, убийственной силе слова, вырабатывало
своеобразный самоконтроль: слово было предметом нравственной оценки, которая
различала слова хорошие и плохие — по их возможному влиянию на человеческую
душу. Постепенно эти нравственные границы размывались, а сегодня имеется
настоящий апофеоз так называемой ненормативной лексики.

…Иду возле станции метро «Лукьяновская» и вижу такую картину. Стоит посреди
толпы какой-то немолодой уже пацан, держит у открытого рта мобильный телефон и
шлет в мировое пространство бесконечный поток матерных слов. Подумалось: не это
ли символ причудливой судьбы всех гениальных изобретений человечества? И чем
отличается от этого задолбанного укрпацана блестящая английская кинозвезда,
которая, по свидетельству английской же прессы, несколько минут поливала свою
телеаудиторию отборными евроатлантическими матами? А что уж говорить о
болгарско-российском «зайке», «звездном матерщиннике» (по определению
«Известий») Филиппе Киркорове?! Давно прошли времена, когда лексика пьяного
русского мужика удивляла мир. Сегодня уже пьяный мужик может позавидовать
лексике культурной элиты определенного сорта.

Казалось бы, это явление — периферийное, где, как и в бизнес-рекламе,
банальность сюжета должна компенсироваться игрой на грани фола и скачком за эту
постоянно передвигаемую границу. Но на самом деле это симптом более широкой и
глубокой болезни человеческого духа. Это — вершина айсберга. Подводная же его
часть — омертвение тонких структур души, сформированных многими тысячелетиями
развития рода человеческого. Все это отражается в речи, особенно в тех ее
участках, которые «отвечают» за интимную сферу жизни, за деликатные стороны
межчеловеческих отношений. Любимого и любимую заменили сексуальный партнер и
партнерша. Слова, передававшие гамму чувств влюбленных, отходят в историю
литературы, и еще недавно их замещало оперативное «переспать», а уже на смену
ему пришло боевое «трахать». Не любовники, а спецназ. Такие вот теперь «глаголы
вечной жизни».

Могут сказать, что всегда, во все эпохи, была стилистика социального дна. Но
теперь это дно мощно всплыло наверх и начало править как законодатель мод, как
элитарность. То есть речь идет уже не о стилистических сдвигах в культуре, что
является закономерностью ее развития, а об изменениях качества человека, об
отмене того, что отличает его от животного, — чувства стыда. А это означает
разрушение всей структуры духа.

Одним из свидетельств этого является засилье в современной литературе — во
всяком случае в украинской литературе определенного рода — различных
«трахкачів», «спермачів», «піхвачів». Хотя они относят себя к постмодерну,
читать их тошно. Поскольку для них не существует та огромная дистанция — не
временная, а психологическая — между первым влюбленным взглядом и полной
интимной взаимопринадлежностью, дистанция, на которой раскрывалась личность
человека и с которой, следовательно, начались вечные произведения искусства.
Лихорадочные дерганья в обход человеческого не могут принести новое Слово.

Эти искажения неотвратимо выражались в искажениях речи, и еще Аристофану они
давали материал для комедий. В античных же временах берет начало и философская
традиция исследования языковых западней, представленная многими великими
именами. (Хотя философия и сама создавала такие западни.) Но жизнь создавала и
создает их с растущей неутомимостью. Наряду с языком как называнием мира она
продуцирует антиязык как подделывание мира. Этот антиязык, с легкой руки
Оруэлла, который зафиксировал его расцвет и триумф в тоталитаризме ХХ в..,
назван новоязом. Но сам по себе новояз сопровождает существование всех обществ,
где правящий слой создает свою идеологию из потребности обосновать и
увековечить государственные священнодействия. Конечно, тоталитаризм
абсолютизирует новояз и впервые делает его обязательным для всего общества,
загоняя адекватный язык в подполье. Но ведь и сам тоталитаризм опирается на
богатое историческое наследие. Потенциал его вызревал в разных обществах. На
мой взгляд, в исследование феномена новояза (если брать его общеисторически, а
не конкретноисторически) не меньший вклад, чем Оруэлл, сделал полузабытый ныне
русский гений — Салтыков-Щедрин. Он, собственно, показал два новояза: новояз
царской бюрократии и новояз псевдолиберальной интеллигенции, раскрыв их
манипулятивную природу. Знание об этих двух новоязах дало бы ключи к
непарадному подъезду русской истории, но сегодня это знание крайне
нежелательно.

Языку всегда приходится платить не за свои грехи. Вспоминается классическое — с
этой точки зрения — заявление одной московской учительницы, опубликованное на
заре перестройки в журнале «Огонек» (1987, № 29): «Нам надоел русский язык,
русскими словами слишком долго врали». Русскому языку приходилось отвечать и за
русских колонизаторов. Украинскому же языку приходилось платить и за униженное
положение своего народа, и за подавление его культуры, и за хохлов-вертухаев в
сибирских концлагерях с немалым количеством узников-украинцев; и за тех
«дядьків отечества чужого», о которых писал Шевченко, которые укрепляли власть
империи над своим и другими народами. Ныне украинский язык расплачивается за
непривлекательный образ и неясные перспективы украинского государства.

(Вспоминается более чем полуторавековой давности отчаянный крик украинского
поэта-романтика Амвросия Могилы: «Кто взвесит, сколько светлых и прекрасных
идей в будущих исследователях человеческого слова должны померкнуть и исчезнуть
по причине забвения одного из языков, тлеющих ныне в могиле презрительного
невнимания?»)

В отношении к языку проявляется вся глубина нашей некультурности и невежества.
«Мы ленивы и нелюбопытны», — писал Пушкин. Кажется, нигде это так не
подтверждается, как в отношении со словом. Особенно же в отношении украинской
публики с украинским словом. На каждом шагу слышишь — в том числе и от
читателя, который считает себя рафинированным интеллигентом, —негодующее:
«Откуда он (писатель) это слово взял?», «Я такого слова не слышал», «Зачем
такое выдумывать?» — вплоть до сакраментального: «У нас так не говорят!» А
более принципиальный товарищ еще и начнет доискиваться идеологической диверсии:
это полонизм, а это галицизм, а там германизм, а там русизм, и вообще это не
язык Шевченко. Ему невдомек, что у Шевченко не было доброй половины той
лексики, которой обогатили украинский язык более поздние писатели. Тем более
невдомек ему, что украинский язык — это не просто знаменитый киевско-полтавский
диалект, освященный самим Сталиным, а язык всех украинцев — и слобожан, и
полещуков, и поляков, и галичан, и буковинцев, и закарпатцев. И почему-то
никому не приходит в голову, что, когда наталкиваешься на слово, которого до
сих пор не знал, то не возмущайся, а радуйся, что отныне будешь знать, попробуй
понять его значение хотя бы из контекста — это в основном нетрудно, или спроси
у знающего человека, или загляни в словарь — ты ведь интеллигент, а бывает, что
и академик-гуманитарий, хотя и галичанофоб. Словарный резервуар развитых языков
содержит двести, триста, пятьсот тысяч лексем. А простой человек в быту
употребляет триста-пятьсот единиц. И считает это достаточным основанием для
санационных намерений в отношении языка. И создает атмосферу стихийной, а
иногда и официозной языковой «зачистки».

Особенно удручает возобладание в массовом сознании современных обществ, в
первую очередь советского и якобы постсоветского украинского, — а собственно
ментально вечно советского* — твердого в своей примитивности представления о
том, что язык всего лишь прагматическое средство общения людей, а не эквивалент
человеческого духа, не носитель бытийности, глубокого опыта поколений и
этической муки личности, не само национальное бытие в его прошлом, настоящем и
будущем. Для Украины, для украинского языка такое репрессивное убеждение
является своего рода смертным приговором, который в голове русско-украинского
обывателя выражен знаменитой максимой: «А какая разница?» Оставляю в стороне
очевидную фальшь этого якобы безразличия. На самом деле для этого равнодушного
«две большие разницы», потому что он требует уступки от другого, а сам никогда
своим не поступится, скорее устроит гражданскую войну, хотя бы районного
масштаба. Но за другим права на собственную «разницу» ни за что не признает.

Поль Рикер в работе «Победа языка над насилием. Герменевтический подход к
философии права» писал: «В больших исторических конфликтах защита и признание
родного языка (langue) всегда играли важную роль. К тому же то, что мы называем
нашей идентичностью, является в значительной степени языковой (de langage)
идентичностью. Принадлежность к определенному лингвистическому пространству —
это первая форма, в которой мы являемся субъектами права (…) Право говорить на
своем языке показывает нам, что урок права начинается уже с первой формы
субъекта».

Наше общество беспредельно далеко от понимания этих вещей. Все у нас поставлено
с ног на голову. Вот один из тысяч такого рода примеров, который может быть
своеобразной притчей. Украинская школа в столице Украины. Июнь 2004 года. Три
педагога беседуют с детишками из украинского же детского садика на предмет
принятия их в школу. Разговор записан дословно. Педагог: «У тебе є друзі в
садочку?» Ребенок: «Є». Педагог: «А ти з ними сваришся?» Ребенок: «Сварюсь».
Педагог: «Наведи приклад». Ребенок: «Вчора посварилась з Машею». Педагог:
«Чому?» Ребенок: «Я хотіла гратися по-українському, а вона не захотіла».
Педагог (удивленно): «Ну то й що? А як у мене чоловік росіянин, то я повинна
вигнати його з дому?» Ребенок (виновато): «Та я хотіла гратися
по-українському». Вмешивается вторая женщина-педагог: «Вот так растут маленькие
шовинисты». Думаю, понятно, кто здесь мудрее: ребенок или педагоги украинской
школы в столице Украины. Но беда не в этом, а в том, что такими педагогами
являются и большинство наших политиков, и большинство интеллектуалов, и едва ли
не все общество вообще. Однако оставим эту абсолютно безнадежную тему.

Хочу вернуться к вещам, в которых легче поладить. Выше говорилось о том, что в
познании мира, по крайней мере в создании его фундаментальной физической
картины, Слово как «осмысливающее размышление» все больше уступает место
средствам «вычисляющего мышления» наук (Мартин Хайдеггер). Но складывается
впечатление, что Слово утрачивает свои позиции и в искусстве, в культуре
вообще. Скажем, очевидно не на Слово делает ставку современный театр. Еще более
очевидна деградация слова в современной поп-культуре, в песенной и эстрадной
продукции. Свою лепту в примитивизацию Слова вносит телевидение. Интернет,
становясь массовым и приобретая функции интеллектуальной игрушки, создает
собственный стиль непритязательного языкового ширпотреба, который вряд ли
облагородит Слово. Все это кажется неизбежным, так как Слово предпочитает
рождаться в сосредоточенности, а не в суете.

А впрочем, те или другие оценки могут быть субъективными и недальновидными.
Возможно, создается новый тип цивилизации, цивилизации поствербальной, в
которой Слову будет принадлежать сугубо вспомогательная роль обслуживания
супертехнологичной деятельности напористого человечества. А может, философия и
поэзия сохранят его для новой жизни.

В свое время Николай Рерих убеждал: «Кризис мира отнюдь не материальный, а
именно духовный. Он может быть излечен только духовным обновлением. Холодный
язык мозга ввел в заблуждение считающих, и снова крайне необходимо обратиться к
тому вечному языку сердца, которым создавались эпохи расцвета».

Утопический призыв великого гуманиста потерялся в водовороте новых, еще более
страшных кризисов.

Пессимистических прогнозов хватает. А. Кожибский даже считает, что народам, в
первую очередь американскому, угрожает шизофрения, которая будет развиваться в
тех участках мозга, которые управляют речью, — из-за асинхронии в эволюции
реальности и социальных отношений, с одной стороны, и языка — с другой.

Можно возразить, что апокалипсические прогнозы о будущем человечества еще
никогда не подтверждались. Однако точно так же никогда не подтверждались и
оптимистические сценарии.

Но кое-что давно подсказал нам Клод Леви-Стросс: «Мы еще недостаточно отдаем
себе отчет в том, что язык и культура являются двумя параллельными
разновидностями деятельности, которые принадлежат к более глубокому слою. Я
думаю, что этот гость был среди нас, хотя никто не догадался пригласить его на
наши дебаты: это человеческий дух».

Итак, будем чаще приглашать этого гостя — человеческий дух — и на публичные
дебаты, и на интимные встречи тет-а-тет.

Доклад, представленный 29 июня 2004 г. на ХІІІ научной конференции «Язык и
культура» имени проф. С.Бураго.

* Мы живем в странном обществе, которое переживает две одновременные
реставрации, якобы противоположно направленные, но симбиотические по отношению
друг к другу. Первая — это реставрация дикого капитализма (фактически уже
состоялась), а вторая — реставрация советской идеологии, но уже откровенно
подшитой российской имперскостью (вспомним хотя бы агрессивность «единороссов»
— православцев, трогательную дружбу коммунистов и правящей элиты с московской
церковью, триумфальное восстановление всех советских ритуалов, вплоть до
оглушительной трескотни комсомольских юбилеев).
