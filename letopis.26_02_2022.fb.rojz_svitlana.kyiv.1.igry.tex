% vim: keymap=russian-jcukenwin
%%beginhead 
 
%%file 26_02_2022.fb.rojz_svitlana.kyiv.1.igry
%%parent 26_02_2022
 
%%url https://www.facebook.com/svetlanaroyz/posts/pfbid0BJZ9tCgEQoUHjBH7a9XRfeqfEXvVAGcZSfWoZXbYC6Bif6DGhbgKC9ge2WDP6HaHl
 
%%author_id rojz_svitlana.kyiv
%%date 
 
%%tags 
%%title Идеи для игр, в которые можно играть в убежище
 
%%endhead 
 
\subsection{Идеи для игр, в которые можно играть в убежище}
\label{sec:26_02_2022.fb.rojz_svitlana.kyiv.1.igry}
 
\Purl{https://www.facebook.com/svetlanaroyz/posts/pfbid0BJZ9tCgEQoUHjBH7a9XRfeqfEXvVAGcZSfWoZXbYC6Bif6DGhbgKC9ge2WDP6HaHl}
\ifcmt
 author_begin
   author_id rojz_svitlana.kyiv
 author_end
\fi

Идеи для игр, в которые можно играть в убежище

Нам важно помнить – что аудиальный канал восприятия во время гула сирен в
особой нагрузке – нужно его «занять». А во время стресса игры детей могут стать
"регрессивными" - старшие могут опять играть в игры малышей

\begin{itemize}
  \item 1. Петь хором
  \item 2. Орать кричалки против утипутина и российской армии
  \item 3. Слушать в наушниках аудиосказки и музыку
  \item 4. Взять с собой гитару, устроить для всех концерт
\end{itemize}


 Тело

\begin{itemize}
  \item 1. Пластилин (сейчас лепить может быть эффективнее, чем рисовать) Сегодня дочка лепила утипутина, танк, войска
  \item 2. По возможности взять с собой подушки, чтоб сделать домик
  \item 3. Нитки, чтоб вокруг себя «прочертить» круг безопасности
  \item 4. Оригами
  \item 5. Кинетический песок
  \item 6. Делать мотанки из платочков и раздавать детям в убежище
  \item 7. Учиться жонглировать
  \item 8. Волшебные прикосновения - представить, что тело - музыкальный инструмент. И каждая часть тела издает свой звук. Прикасаться ладошкой, а ребенок поет, мычит, кукарекает и проч.
  \item 9. Я - животное. Показать жестом, позой, звуками - животное. Второй партнёр по игре отгадывает. (Образ животного всегда "про жизненность")
  \item 10. Выдувать мыльные пузыри
  \item 11. «Докричаться до Луны» — представить, что нужно крикнуть послание космонавтам или инопланетянам. Можно кричать из разных углов комнаты друг другу, можно направлять звук «далеко-далеко».
  \item 12. «Выдохнуть облако» — представить, что вдохнули облако и выдохнуть его со звуком, можно с грозой и молнией (топая ногами).
  \item 13. «Погода» — можно идти, можно стучать подушечками пальцев (лучше ногтями): вначале солнышко и легкий ветер (идем-стучим медленно, мягко и тихо), затем ветер крепчает, начинается дождь (идем быстрее и топаем). Дождь усиливается — топаем сильнее, ливень — прыгаем, гроза — ураган — прыгаем, дуем и кричим, ветер успокаивается, дождь стихает — постепенно идем тише и медленнее. Заканчиваем «хорошей погодой»..
  \item 14. «Клац-клац» — клацать зубами, будто от холода, затем показывать зубы, как обезьянки, скалиться и рычать, как тигры. После этого превратиться в людей.
  \item 15. «Лети!» — дуть друг на друга, как на перо. Позволить себе подуть на родителя, чтобы «сдуть» его с места.
  \item 16. «Толкалки» — «оттолкнуть воздушной волной» родителя (партнера по игре), сделав жест руками. Родитель «отлетает» от ребенка.
  \item 17. «Подушечный бой».
  \item 18. «Борцы сумо» — засунуть подушку под футболку и толкаться, подушка будет смягчать удар.
  \item 19. Рвать бумагу, а потом собрать из обрывков коллаж.
  \item 20. «Шкрябалки и дряпалки» — ногтями «прошкрябать» рисунок в тесте, в фольге, в крупе. В закрывающийся пакет залить краску, ребенок подушечками пальцев и ногтями процарапывает узоры.
  \item 21. Боулинг — сбивать любые предметы (позволить нарушить порядок).
  \item 22. «Злой бобер» — придумать злое животное и вести себя так, как, нам кажется, может себя вести это животное. Потом превратиться в доброе животное. Затем в человека.
  \item 23. «Выпустить когти» — показать, как тигренок шипит и показывает когти, когда хочет защититься. Потом превратиться в человека.
  \item 24. «В цель» — смять в руках бумагу и, как мячиком, попасть в цель.
  \item 25. Кричать в «мешочек для крика». Или в стаканчик
  \item 26. «Бодалки да-нет» — бодаться руками или головой, говоря по очереди «да» или «нет».
\end{itemize}



 Настолки

\begin{itemize}
  \item 1. Шахматы – шашки
  \item 2. Причем, шашки – сбивать щелбанами (нам нужны «агрессивные действия»
  \item 3. Любые настолки
  \item 4. Карты
  \item 5. Чем проще (тупее) игра, тем лучше
  \item 6. Расставлять доминошки и устраивать "цепную реакцию"
\end{itemize}


 Рисовать

\begin{itemize}
  \item 1. Мандалы, домики, радуги
  \item 2. Мечту
  \item 3. Самое безопасное, защищенное место
  \item 4. То, что хочется делать, сразу после войны
  \item 5. Раскраски
  \item 6. Ангелов
\end{itemize}



Игры со словами

Меня дети в убежище научили игре – «вопрос» - один игрок задает вопрос – второй
должен ответить на него вопросом – игра идет до тех пор, пока кто-то скажет
предложение без вопроса.

(например – ты ел сегодня? – а почему ты интересуешься? Разве это секретный вопрос? И т.д.)

\begin{itemize}
  \item 1. Четыре слова - (детям от 8 лет можно предложить и 5-6 слов) - выбираем 4 совершенно не связанных между собой слова. Используя их, составляем по очереди маленький рассказ.
  \item 2. «Множество слов» Выбираем большое слово. Из его букв каждый на своём листе составляем и записываем много маленьких. (Дети могут вначале быть в команде со взрослыми, а затем, они быстро становятся самостоятельными игроками).
  \item 3. «Рисование словом» - выбираем «сложное» для ребёнка слово, которое сложно писать или произносить. Придумываем, что именно будем рисовать. Например, нарисовать фигурку кота, используя только загаданное слово (пишем его много раз в разных плоскостях)
  \item 4. «Наоборот» - говорить слова наоборот, не записывая их. (например, Чайник - кинйач, стена - анетс...)
  \item 6. Написать вразброс на листе буквы, соединять буквы линиями в слова
  \item 7. «Покажи слово» - один игрок загадывает слово и показывает его пантомимой
  \item 8. «Я загадал» - рассказать о загаданном слове, не называя его (например. «Я загадал - белый, холодный, скрипит под ногами, тает...)
  \item 9. В моем детстве эта игра называлась «индукция - дедукция». Игрок загадывает предмет. Остальные должны его угадать, задавая вопросы, на которые он может ответить только Да или Нет (с малышкой мы начинали с загадывания животных и частей тела)
  \item 10. Последняя буква. Каждый игрок говорит слово, которое начинается на последнюю букву предыдущего слова.
  \item 11. Рифма. Называем 4 (или два) случайных слова, на которые ищем рифму или с которыми придумываем четверостишие.
  \item 12. «Буква» - называем слова только на заданную букву. (Вариант для взрослых: слова - предметы на загаданную букву, которые можно положить в трёхлитровую банку)
  \item 13. Ищем сходное. Называем два случайных слова - чем они могут быть объединены?
  \item 14. Волшебный язык. Для детей, которые учатся разбивать слова на слоги. Разговариваем, перед каждым слогом добавляя дополнительный слог (жи, ма, хрю....)
  \item 15. «Чепуха». Говорим начало предложения, к которому нужно придумать совершенно «чепуховое», нереальное продолжение
  \item 16. «Сказка» - сочиняем сказку - каждый говорит по предложению
  \item 17. Ассоциации - один игрок называет слово, следующий - свою ассоциационная это слово (детям можем объяснить - первое, что приходит в голову, когда слышишь это слово).
  \item 18. «Это слово умеет...» - описываем слово его «умениями»
  \item 19. «Картина» - называем слово. По очереди рассказываем, какую «объемную» картину мы видим, когда произносим это слово (например, для одного игрока слово «чай» означает: семья сидит за столом и разговаривает, для другого - Чайная плантация и собиратели чая, для третьего - стол накрыт скатертью, играет Музыка, для четвёртого - Чайная церемония, для пятого - человек сам стоит у окна, наслаждаясь одиночеством и чаем.
  \item 20. Антонимы - называем слова и ищем к ним антонимы
  \item 21. Несуществующее слово - придумываем несуществующие слова, и что бы они могли означать на языке инопланетян
  \item 22. Тарабарский язык - говорим друг с другом на «тарабарском языке», пытаясь по интонации угадать - о чем каждый говорит
\end{itemize}

Ну и мультики

И игры в телефоне, где задействовано пространственное восприятие - типа тетриса

держитесь, дорогие!!! 

\ii{26_02_2022.fb.rojz_svitlana.kyiv.1.igry.orig}
\ii{26_02_2022.fb.rojz_svitlana.kyiv.1.igry.cmtx}
