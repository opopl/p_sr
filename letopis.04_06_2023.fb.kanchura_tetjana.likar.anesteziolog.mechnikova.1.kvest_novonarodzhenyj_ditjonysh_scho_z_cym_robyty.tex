%%beginhead 
 
%%file 04_06_2023.fb.kanchura_tetjana.likar.anesteziolog.mechnikova.1.kvest_novonarodzhenyj_ditjonysh_scho_z_cym_robyty
%%parent 04_06_2023
 
%%url https://www.facebook.com/tanjushka.djan/posts/pfbid0S5rceQmDfydiZy1K7V5sooTXfELAL7KKCQMKq9pwTc7VGPw85pseXpnwhEV1Vn1tl
 
%%author_id kanchura_tetjana.likar.anesteziolog.mechnikova
%%date 04_06_2023
 
%%tags 
%%title Перший рівень квесту "Новонароджений дітьониш і шо з цим всим робить" пройдено
 
%%endhead 

\subsection{Перший рівень квесту \enquote{Новонароджений дітьониш і шо з цим всим робить} пройдено}
\label{sec:04_06_2023.fb.kanchura_tetjana.likar.anesteziolog.mechnikova.1.kvest_novonarodzhenyj_ditjonysh_scho_z_cym_robyty}

\Purl{https://www.facebook.com/tanjushka.djan/posts/pfbid0S5rceQmDfydiZy1K7V5sooTXfELAL7KKCQMKq9pwTc7VGPw85pseXpnwhEV1Vn1tl}
\ifcmt
 author_begin
   author_id kanchura_tetjana.likar.anesteziolog.mechnikova
 author_end
\fi

1/12 \par
Перший рівень квесту \enquote{Новонароджений дітьониш і шо з цим всим робить} пройдено.\par
Короткі висновки:\par
- Найбільше я люблю час проведений з дитиною.\par
- Найбільше я люблю час проведений без дитини.\par
- Перші два пункта не суперечать один одному, навіть якщо вам так здалось.\par
- Шива точно був годуючою, працюючою жінкою. Бо інакше нашо мати стільки рук? \par
- Не залишати роботу - одне з найкращих рішень.\par
- Моя улюблена мантра зараз \enquote{Совєт свой сєбє посовєтуй}.\par
- Їсти, спати і ходити в туалет треба тоді коли є час. Тут всьо, як на чергуванні в реанімації. Нічого нового.\par
- Не псіхуй. Всьо минає. Ще б пригадувати цей пункт о четвертій ранку)\par
- Винахіднику робота-пилососа мають дати нобелевку. Краще дві.\par
- Винахіднику бордюрів і всім причетним до їхнього встановлення вже розігріли казан у пеклі. Два казана.\par
- Дві фрази, які я зараз чую найчастіше: \enquote{Якось неочікувано ти народила…} і \enquote{А твоя мама тоже з фб дізналась?}\par
- Попрасовані з чотирьох сторін пелюшки, це не про дитину, а про сублімовану тривогу.\par
- У мене найкращий чоловік на світі. Хоча це я знала ще до декретів.\par
P. S. Я знаю, що місяць був вчора, але я \#годуючаженщінавдєкрєті мені можна викладати дописи не вчасно.\par
