% vim: keymap=russian-jcukenwin
%%beginhead 
 
%%file 13_11_2021.fb.kyshtymov_aleksandr.1.russkie_i_ukraincy.cmt
%%parent 13_11_2021.fb.kyshtymov_aleksandr.1.russkie_i_ukraincy
 
%%url 
 
%%author_id 
%%date 
 
%%tags 
%%title 
 
%%endhead 
\subsubsection{Коментарі}

\begin{itemize} % {
\iusr{Павел Резник}

Когда в дело влезает политика, всегда потом влезает язык, очень советую в Ютубе
беседа Вассермана и полиглота Дмитрия Петрова, кто хочет думать, посмотрите эту
передачу


\iusr{Tatjana Voronina}

Не понимаю одного: если украинцы не хотят быть родственниками, то почему
русские так настойчиво хотят этого? И так радуются, когда находится возможность
опереться на чей-то авторитет...

Ну хотят люди жить отдельно, зачем лезть к ним в родственники? Живут же
отдельно немцы и австрийцы, швейцарцы - французы - итальянцы - хотя там уже
тоже веками все перемешано. И никто не говорит, что мы более великие, потому мы
главные и отдайте нам то, что принадлежало нам когда-то... Последним таким был
Гитлер кажется...

\begin{itemize} % {
\iusr{Александр Кыштымов}
\textbf{Tatjana Voronina} 

немцы, австрийцы и швейцарцы живут как дружеские страны, без противопоставления
друг другу. Что же касается проекта Украины, то он предусмотрен как антипод
России и поэтому, пока Украина будет идти по этому пути конфликты неизбежны


\iusr{Maxim Gerasimenko}
\textbf{Tatjana Voronina} 

К сожалению, в ситуации Украины с Россией сошлись два основных фактора -
синдром "младшего брата" и нужда местной элиты во внешнем враге, чтобы отвлечь
народ от катастрофической внутренней коррупции. (Есть, правда, и другие
факторы)


\iusr{Tatjana Voronina}

Ну вот объясните мне, даже если это проект ( хотя почему на допустить, что
просто страна, которая хочет самостоятельно и независимо жить в оговорённых
границах ), даже если как антипод России, - то зачем конфликтовать пт этому
поводу? Пусть живет, как хочет и как знает. Почему ей обязательно надо быть
похожей на Россию? Зачем это надо России? Почему это ее так волнует? Ведь
действительно в самой России есть очень много мест, куда можно было бы
приложить усилия с пользой для своих граждан, разве нет?

\iusr{Tatjana Voronina}
\textbf{Maxim Gerasimenko}

Я бы ещё, наверное, добавила «синдром старшего брата». Посмотрите на
комментарии. Хотя я и русская, но мне всегда стыдно и неприятно видеть, с каким
пренебрежением люди, считающие себя наследниками великой русской культуры,
называют граждан соседнего государства самыми унизительными кличками.

И эти, конечно, на добавляет желания соседям признать себя родственниками.

Я живу в Словакии и вижу, как словаки и чехи относятся друг к другу после
распада их общей страны. И тихо им завидую...


\iusr{Татьяна Пашаева}
\textbf{Tatjana Voronina} Наблюдаю за русскими и украинцами. Должна сказать, что украинцы не отстают, а даже опережают "москалей" в оскорблениях. Так что все хороши.

\iusr{Tatjana Voronina}
\textbf{Татьяна Пашаева}
А вам никогда не казалось, что они так защищаются?..

% -------------------------------------
\ii{fbauth.koreljakova_tatjana.rossia.hudozhnik}
% -------------------------------------

\textbf{Tatjana Voronina} 

у вас какая-то память короткая. Вам не кажется, что все 'клички' - это ответ на
десятилетия русофобии. Не вспомните клички, которые потоками лились сначала от
украинцев? За что боролись, на то и напоролись. И может начинать надо именно с
этого - свое поведение скорректировать. А то двойные стандарты - мы вам будем
дули крутить, а вы нам кланяйтесь в пояс.

\iusr{Alena Morgan}
\textbf{Татьяна Корелякова} а вас-то кто и где обозвать успел? Что вам украинцы вообще плохого сделали? Или по телеку "слова передали"?!

\iusr{Татьяна Корелякова}
\textbf{Alena Morgan} 

Так я думаю, что и вы лично ни от одного русского никаких оскорблений не
слышали!  @igg{fbicon.grin}  а только ваш телек вам передаеь и передает уже 30 лет. А вы хоть
одну передачу антиукраинскую найдите на российском ТВ раньше 2014? Вот то то
же.

\iusr{Татьяна Корелякова}
\textbf{Alena Morgan} 4 года когда?

\iusr{Татьяна Корелякова}
\textbf{Alena Morgan} ну так это своим соотечественникам и их художествам в ножки
поклонитесь. Я 30 лет в Московском регионе, уроженка Украины - ни разу, нигде
никогда не сталкивалась с унижением украинцев. И государство в котором жили не
надо поносить задним числом - это некрасиво.

\iusr{Alena Morgan}
\textbf{Татьяна Корелякова} 

звучит так, как будто это мы тонны грязи каждый день на российском ТВ выливаем.
Смешно. Вам ничего не говорят, потому что не знают, что вы \enquote{уроженка Украины}.
Видно, что вы никак себя даже и не ассоциируете. Представьте себе, когда вам в
лицо говорят что вашей страны не существует))

\iusr{Татьяна Корелякова}
\textbf{Alena Morgan} 

а вы как ассоциируете интересно? В какой форме? @igg{fbicon.grin} видимо, именно за это
неправильное проявление идентичности и получаете негатив.

\iusr{Alena Morgan}
\textbf{Татьяна Корелякова} я украинка и люблю свою страну. Вот вам закат с пожеланиями доброй ночи.

\ifcmt
  ig https://scontent-frt3-1.xx.fbcdn.net/v/t39.30808-6/255260481_10223336280509760_7726368349278552201_n.jpg?_nc_cat=107&ccb=1-5&_nc_sid=dbeb18&_nc_ohc=otQUH4h2UQMAX-70qTm&_nc_ht=scontent-frt3-1.xx&oh=eeef6dce38852fc052a46c8737a7c32b&oe=61A6DC94
  @width 0.4
\fi

\iusr{Татьяна Корелякова}
\textbf{Alena Morgan} по поводу ТВ - встречный вопрос - а кто льет по украинскому ТВ тонны грязи уже 30 лет?

\iusr{Alena Morgan}
\textbf{Татьяна Корелякова} до аннексии Крыма мы все, к сожалению, считали Россию братской страной. Это было огромной ошибкой.

\iusr{Татьяна Корелякова}
\textbf{Alena Morgan} и тем не менее по ТВ шли антироссийские передачи и массированно. Так что на себя надо смотреть.

\iusr{Alena Morgan}
\textbf{Татьяна Корелякова} приведите примеры таких передач.

\iusr{Татьяна Корелякова}
\textbf{Alena Morgan} вы сами можете погуглить.

\iusr{Alena Morgan}
\textbf{Татьяна Корелякова} я не знаю, про что вы вообще. Поэтому и спрашиваю. Что именно я должна гуглить? Ваши фантазии?))

\iusr{Татьяна Корелякова}
\textbf{Alena Morgan} 

ну фантазии не фантазии, а знаменитые скачки не в павильонах Мосфильма были
сняты. И в повестке каждого Майдана была тема противостояния с Россией. Вот
тоже удивительно почему? Вроде внутренние вопросы решали, а в итоге вся политика
строилась именно на противостоянии России. Я этому удивлялась, потому что до
2014 ниеакой украинской повестки не было в России нигде. И вы слепая? Вы не
видите что ли какой вред вашей любимой стране от того, что разорваны связи с
Россией? Кредиты, упадок производства и отток населения - и все это собственными
руками, бездумно и беспощадно. Зато кричитечто любите Украину - странная какая-то
любовь, от которой все разрушается.

\iusr{Анна Снегина}
\textbf{Tatjana Voronina} 

там сжигают и расстреливают людей как это делают чубатые, там пропагандируют
откровенный фашизм? Или уроки ВОВ ничему не учат?

\iusr{Юнна Мориц}
\textbf{Tatjana Voronina} 

А Вам не стыдно и приятно видеть и слышать, как скачут в\textbackslash на Украине с воплями
\enquote{Москаляку на гиляку!}, как там в школьном буфете стоит компот в кувшинах с
надписью \enquote{Кровь славянских младенцев}, как там напоказ выставляют консервные
банки с надписями о том, что в этих банках - мясо, тушёнка из людей Донбасса???
И не притворяйтесь, не врите, что Вы слышите об этом впервые!.. Всё Вы
прекрасно знаете.

\iusr{Максим Степанюк}
\textbf{Yunna Morits} 

на щет воплей я не поспорю, но остальное бред сивой кобылы. Меньше телек
смотрите, он на вас, россиян, имеет какое-то странное влияние.

\iusr{Максим Степанюк}
\textbf{Tatyana Korelyakova} 

странно, что не владея истинными фактами, Вы, россияне, так верите в свои
слова. Ни один майдан не был направлен на агрессию с Россией. Даже в 2013 не
было призывов убивать россиян. Боролись за вступление в Евросоюз, для уравнения
рыночных цен и привлечения инвесторов. А Россия испугалась, когда лояльного к
ней призидента выперли, за убийства по его приказу его не выдают. Почему
Россия, которая судя по Вашему, хотела мира, не выдала преступника? И напала
подло, как нацисты в 41. Это указывает на то, что Россия - агрессор, и ни чем
не отличается от Германии, периода Гитлера. Спросите у любого ветерана, сильно
ли они любили нацистов? Тогда не удивляйтесь всем кричалкам и остальному
происходящему после 14. Россияне нормально живут в Украине, никто не трогает
русскоговорящих, это выдумки ваших Пиз....Ду...нов. Российские ведущие уже
тонут во враньё, потому и соответственно такое отношение. Так что перестаньте
писать чушь

\iusr{Дммтрий Кулаков}
\textbf{Максим Степанюк}

Если нормально русские живут в Украине, то что же вы со своей демократической
властью не сядете за стол переговоров с Донбассом и не решите проблемы его
автономии, использования там русского языка наравне с государственным,
отчисления налогов Киеву и пр.?

% -------------------------------------
\ii{fbauth.kulakov_dmitrij.almaty.kazahstan.permj}
% -------------------------------------

А решаете вопросы авиабомбежками и обстрелами из Градов жилых кварталов Донецка
и Мариуполя?

% -------------------------------------
\ii{fbauth.stepanjuk_maksim.kiev.ukraina}
% -------------------------------------

\textbf{Дммтрий Кулаков} 

я задам Вам вопрос сначала, это будет и ответ на Ваш! Почему в 41 после
нападения нацистов на СССР и захвата половины государства, Сталин не сел за
стол переговоров с Гитлером и не договорился \enquote{полюбовно}? Почему выгнав с
територии СССР нацистов, их гнали до самого Берлина? Почему всех тех, кто
служил нацистам, судили как предателей, а не дали им возможности решить, в
какой стране жить? И не дали им территорию со своей автономией? У нас Крым был
автономным. А Донецкие не хотят автономии, а хочет передачи земли в
собственность РФ. А это уже нарушение территориальной целостности, а за
территорию будет воевать любой народ.

% -------------------------------------
\ii{fbauth.smirnov_igor.riga.latvia}
% -------------------------------------

\textbf{Максим Степанюк}, 

интересно, вы сами верите в то, что пишете ?.. Впрочем, этот вопрос
риторический.

Но давайте по пунктам. \enquote{Москоляку на гиляку} мы услышали ещё до
антиконституционного переворота 2014 года. Наверное, это не было призывом
убивать русских... Так активисты Майдана призывали вступить в ЕС, привлекали
инвесторов, развивали экономику...

Теперь о Януковиче. Скажите, а зачем за полгода до выборов устраивать переворот
? Ведь можно просто переизбрать другого президента. Зачем убивать сотни людей?
Ужели демократический путь предписывает массовые убийства для достижения цели
?!!

Далее, Россия не испугалась. Ни нового президента (посаженного Майданом Ющенко
уже видели), ни чего либо другого. Януковичу показали в своё время ответные
шаги России по защите своего рынка в случае ассоциации Украины с ЕС, он всё
понял. Поняли ив ЕС, что Украина не станет транзитом для товаров из ЕС в России
без пошлины. Потому Янукович и не решился потерять рынок России. А его
преемникам - новоявленной колониальной администрации Украины - на экономику
было наплевать. Их интересовал только собственный кошелёк. Результат сегодня
налицо.

Далее, Янукович, конечно, дерьмо. Слабак. Воришка. Но преступником в человека в
демократическом государстве может назвать только суд. 

И только по решению суда ( легитимного суда, а не сборища назначенных из-за
границы бомжей). Кстати, вы не задумывались, почему среди \enquote{небесной сотни} нет
ни одного из лидеров Майдана. Только \enquote{беркут} и случайные люди? Ведь если
расстреливать оппонентов, так отстрелять Порошенко, Кличко, Яценюка было бы
куда эффективнее для погашения выступлений! А стреляли в рядовых. Что только
разжигало конфликт... Да и стреляли из зоны, контролируемой Майданом. Впрочем,
это не впервые. Такое уже было. В Риге, в Вильнюсе, в Каире, в Алжире и т.д.

Теперь о нападении. Если вы о Крыме, то спросите крымчан, кто напал, когда была
оккупация, как сейчас. Они вам ответят совсем не то, что вы слышите по своему
ТВ. И решать, чей Крым, не надо. Крым принадлежит в первую очередь крымчанам.
Если же вы о Донбассе, тогда смотрите пункт первый: Донбасс принадлежит тем,
кто там живёт. И найдите в интернете кадры, как Луганск бомбили самолёты ещё до
начала военных действий, как жители Донбасса пытались остановить танки руками.

И перестаньте писать чушь!

\iusr{Максим Степанюк}
\textbf{Igor Smirnov} 

хорошо, что Вы верите в свою чушь! Свернуть с дороги поможет лишь гора, но
твердолобый и она не помеха. Умный в гору не пойдет, умный гору обойдет. Удачи
при подъёме Вам. Соглашусь с Вами лишь по поводу тех кто живёт, но
территориальную целостность страны никто не давал права нарушать. Никому,
никогда, ни под какими бы то ни было предлогами.

\iusr{Игорь Смирнов}
\textbf{Максим Степанюк}, 

отвечу вашими же словами: хорошо, что вы верите в свою чушь.

А далее снова по пунктам.

Во-первых, никогда не говорите НИКОГДА.

Во-вторых, незнание не может быть аргументом в дискуссии. Я уже не говорю о
нарушении территориальной целостности СССР, в результате чего, в частности, на
карте мира появилась Украина. Но когда разделили сначала Югославию, а потом
выделили из Сербии Косово, это не было нарушением территориальной целостности,
которую признал весь так называемый цивилизованный мир?

Как там насчёт \enquote{Никому, никогда, ни под какими бы то ни было предлогами} ?!!

И кто после этого твердолобый ?!.

\iusr{Максим Степанюк}
\textbf{Igor Smirnov} 

кстати Вы преподнесли лично проверенную информацию? Ездили, спрашивали? Или
есть источники?

\iusr{Ирина Николаевна Онищук}
\textbf{Tatjana Voronina} 

потому что не страна, а колония, потому что отдельно жить хотят меньшинство
населения. Большинство как раз хотят наоборот. Но хозяева не позволяют.

\iusr{Максим Степанюк}
\textbf{Igor Smirnov} 

мне не интересно про Югославию, Косово. Потому, что можно сюда приписать и
Грозное, Кавказ и т.д. мне интересно, почему до 14 года все было ОК, а потом
Россия проявила агрессию? Пойдя по пути нацистов, и отмазки кстати те же

\iusr{Максим Степанюк}
\textbf{Igor Smirnov} кстати судя по Вашим словам: Путин напал на Украину, потому, что не захотел терять рынок сбыта?

\iusr{Игорь Смирнов}
\textbf{Максим Степанюк}, 

во-первых, имейте мужество признаться в том, что погорячились насчёт \enquote{нигде и
никогда}. В мире такое применялось \enquote{всегда и везде}! И именно \enquote{цивилизованным
миром}.

Во-вторых, не надо отбрасывать прецеденты с криками \enquote{мне это неинтересно}, если
вы действительно хотите понять, что произошло. В-третьих, до 2014 года тоже
было всё нехорошо. В те моменты, когда мне приходилось посещать Украину с
начала 90х всё было плохо. И становилось всё хуже и хуже. Кстати, Янукович
вовсе не был лоялен России. Он был лоялен своему кошельку. И готовил себе
личную гвардию из нациков с Западенщины. Подготовил. Эта гвардия его и урыла.
Кстати говоря, вот его мне ни капельки не жаль.

\iusr{Игорь Смирнов}
\textbf{Максим Степанюк}, 

где по моим словам \enquote{Путин напал на Украину} ??? Это по вашим словам так. А по
мои словам Украина давно и надёжно оккупирована САСШ. И во главе Украины давно
нет никакого правительства или президента. Есть колониальная администрация, во
главе которой стоит Петрушка. И уже не Первый, а Второй. Который не приходит в
сознание от кокса. Первый не приходил в сознание от бухла.

\iusr{Анна Снегина}
Воронина явный провокатор и упырь

\iusr{Максим Степанюк}
\textbf{Igor Smirnov} 

1) прециндентов много (кстати у Вас в этом слове ошибка), и их можно толочь
как говно в ступе. Я не одобряю ни одной стычки, за всю свою жизнь подрался
пару раз, всегда решал вопрос мирно.

2) укажите, пожалуйста, что Вы имеете ввиду: \enquote{нигде и никогда}.

3) я 30 лет живу в Украине и не разу не столкнулся (до 14 года) в ущемлении
россиян, у меня брат и племянник в Смоленске, ничего не было плохого. Зато
много раз слышал от приезжих россиян оскорбления в адрес украинцев.

4) на счёт Яныка согласен на  @igg{fbicon.100.percent} 

\iusr{Tatjana Voronina}

Не хочу больше продолжать дискуссию, потому что уже вижу бредовые аргументы не
из личного опыта, а из телевизора.

Просто скажу, что пришлось по делам съездить во Львов (!!) именно в феврале
2014 года, когда там оплакивали погибших на Майдане. Боялась, очень. Но надо
было. И из-за любопытства даже пошла в центр. На украинском не говорю. Но в
кафе меня накормили, дорогу найти помогли и банок с русскими людьми не видела.
И тогда ещё раз убедилась, что критерий истины - это личная практика, а не
телевизор. После этого уже много раз ездила и во Львов, и в Киев.

Но да, я не кричала на каждом углу, что украинцы - это фашисты, и не указывала,
как им жить и поступать в своей стране. У меня есть свои заботы. Тем более, со
стороны невозможно разобраться, кто прав, а кто виноват.

\iusr{Володимир Кучерявий}
\textbf{Tatjana Voronina}, 

последним таким реваншистом является Владимир Путин, который плохо знает
мировую историю, а потому придумывает исторические мифы и небелицы.

\iusr{Alena Morgan}
\textbf{Юнна Мориц} это же надо писать подобный бред и ещё в него верить.

\iusr{Alena Morgan}
\textbf{Татьяна Корелякова} 

конечно, это ваши выдумки, заложенные вашей пропагандой. Я лично не была на
Майдане, но по его окончании, через день после бегства Януковича, прошлась по
баррикадам и там не было ни одного антироссийского плаката. Но я вас не виню,
ибо вы сами обмануты. Мне жаль.

\iusr{Игорь Смирнов}
\textbf{Максим Степанюк},

1 - это у вас в этом слове ошибка, правильно пишется прецЕдент. Можете
справиться в интернете. Я крайне редко делать ошибки в русском языке.

2 - никогда и нигде.

Ни один спорный вопрос между странами никогда не решался путём переговоров.
Всегда всё заканчивалось \enquote{последним доводом королей}. Даже в послевоенный
период \enquote{столп демократии} устроил более 60 вооруженных конфликтов по всему
миру. В качестве примера приведу самые одиозные: Тонкинский инцидент,
послуживший началом Вьетнамской войны, провокация с подругой с белым порошком,
послужившая поводом для агрессии против Ирака. Гренада, Панама, Сомали, Корея,
Югославия, Азии, Тунис, Ливия, Лаос, Камбоджа, Колумбия, Корея... Сколько
провокаций было против Кубы, устранение считать.

3 - каждый видит то, что хочет. Когда мне довелось проехать через Волынь и
Западенщину в 1990м году, я познакомился с "гостеприимством" галичан. Мы зашли
перекусить в придорожный ресторан, где нам на мове доходчиво объяснили, что
продуктов нет, готовить не из чего. 

Я перевёл своим спутникам на латышский, потому что они не поняли ничего.
Услыхав другой язык, рестораторы поинтересовались, откуда мы. Узнав, что не из
России, накормили без вопросов.  Напоминаю - 1990 год. До Майдана 2014го ещё 24
года. 

Потом в Полтаве меня
проверяли на \enquote{поляницю}. Могу ли я это произнести. В противном случае
фотокарточка была бы повреждена. Это уже 1994й. До Майдана ещё 20 лет...

Заметьте, что за всё время дискуссии я не применил ни одного кривого слова
против украинцев. Люди вообще ни в чём не виноваты. Они в своей массе всегда
ведомые. Очень подвержены пропаганде. Я даже тех, кто доставлял мне
неприятности в этом не обвиняю. Это политики сделали из них такое...

\iusr{Татьяна Корелякова}
\textbf{Alena Morgan} 

какая пропаганда, простите? Как легко на нее все списывать. Я лично следила за
тем как развивались события, чтобы понимать что и как происходит и кто первый
начал. А то, как мы видим, это потом быстро забывают и начинаетя 'а че? а где? не
был, не состоял.' Но вот стишок 'никогда мы не станем братьями', который сразу
стал гулять -это про кого он был, а кричалки про рабов и ватников? Они кому
были адресованы? Януковичу? Для меня лично это был полный шок, потому что я в
российском инфополе ни одного слова плохого про Украину и украинцев не
слышала, не понимала за что столько ненависти и злобы.

\iusr{Alena Morgan}
\textbf{Татьяна Корелякова} 

ваше мнение подготовили к тому, чтобы отнять Крым и Донбасс. В это вложили
миллионы рублей, этим занимались спецслужбы. Как кто-то может вас теперь
разубедить?)) Оставайтесь при своём мнении уже, так даже проще в среде вашего
обитания.

К слову, если до аннексии и войны это было неправдой, то сейчас настроения
вполне однозначные. Так что, действительно, наш спор не имеет смысла.

\iusr{Дммтрий Кулаков}

Конечно, вы правы.

Никто на самостийность и незалежность Украины не претендует.

Пусть живут, как хотят.

Только вот посетил давеча один укр сайт, так там все уверены (причем, 80\%
женщины), что Россия вот-вот нападет и захватит Украину. Все, прям, патроны
своим мужчинам подносить собираются.

Хотя, любому здравомыслящему понятно, что Украина России не нужна в плане
захвата, а если бы нужна была, то через неделю после начала российские танки
лязгали бы гусеницами по Крещатику

\iusr{Татьяна Корелякова}
\textbf{Alena Morgan} 

то есть эти песенки про небратьев распевали спецслужбы? Они же отправляли
'поезда дружбы' в Крым и танки на Донбасс в нарушение Конституции использовать
собственную армию против народа? Эти приказы давали спецслужбы? Тогда это повод
не мне, а вам задуматься, чего стоили все эти майданы и к чему они привели. А
мнение свое я складывала сама, сопоставляя и фиксируя факты. И они, к
сожалению, не в пользу Украины. Потому что все, что было сделано было не в
правовом поле - это вред и народу, и государству.

\iusr{Alena Morgan}
\textbf{Татьяна Корелякова} 

\enquote{неправильное проявление идентичности} — это просто быть украинкой и женщиной.
Как только звучало слово \enquote{украинец} или \enquote{Киев} — далее у некоторых случался
поток сознания)) Там хоть стой, хоть падай. Не звонить же каждый раз в
психушку))


\iusr{Alena Morgan}
\textbf{Татьяна Корелякова} как вы фиксируете факты, я уже поняла)) Из телека)) Развлекайтесь))

\iusr{Татьяна Корелякова}
\textbf{Alena Morgan} 

так тут же дело как раз в деталях, о которых вы вряд ли расскажете. А обмороков
от слова 'Москва' я тоже насмотрелась! Я теперь специально людей проверяю им на
адекватность  @igg{fbicon.grin}. Если после произнесения не пошло дебильных придуманных
исторических выкладок и майданных лозунгов, значит можно общаться.

\iusr{Татьяна Корелякова}
\textbf{Alena Morgan} 

ой, ну из Словакии конечно ви дней  @igg{fbicon.grin}{repeat=4} а вы откуда информацию берете можно
узнать? Вам наверное все президенты каждую неделю отчеты шлют?  @igg{fbicon.grin}  ну про
'телек' это уже такой тухлый аргумент, что я даже не знаю.... но надо отдать ему
должное - напугал он вас хорошо и поделом, будете в следующий раз понимать, что
ответка прилетит.

\iusr{Игорь Смирнов}
\textbf{Alena Morgan}, 

не знаю, как насчёт миллионов рублей, но то, что только по официальным данным в
Майдан госдеп вложил 5 млрд долларов, сомнений не вызывает. В смысле вызывает,
что только пять...

Что касается настроений, то для из оценки хорошо бы побывать в тех местах, за
которые расписываетесь. Насколько же я понимаю, вы там не бывали уже довольно
давно. Если бывали когда-то вообще. Да и настроения на Украине разные. Градиент
настроений по долготе очень разнится. В одну телегу впрячь не можно коня и
трепетную лань! То, что хорошо для Галичины, может оказаться неприемлемым для
Харькова. Так что спор действительно не имеет смысла.

\iusr{Alena Morgan}
\textbf{Татьяна Корелякова} 

никаких деталей не было)) Просто какие-то стереотипы прошлого, плюс пропаганда.
Например, если я украинка — то почему-то полячка, должна работать на панели,
должна отказаться от гражданства и быть этому рада, что в Киеве нечего есть и
прочее... И это если человек сдерживает себя и не говорит о политике))
Утомительно...

\iusr{Maxim Gerasimenko}

Ответ всем: \enquote{Тереть} на эту тему можно бесконечно. По-моему, единственный
способ переломить ситуацию, это набраться мудрости и обнулить взаимные обиды
лет, эдак, на двадцать пять.

... Но это если есть желание ситуацию изменить.

\iusr{Maxim Gerasimenko}
\textbf{Tatyana Korelyakova} 

Я тоже сталкивался здесь исключительно с интересом и уважением. Негатива не
встречал (не считая персонажей из talk-show)

\iusr{Татьяна Корелякова}
\textbf{Maxim Gerasimenko} я полностью с вами согласна. Украинцы очень устали от этой свистопляски,видно уже,что на пользу оно не пошло.

\iusr{Alena Morgan}
\textbf{Игорь Смирнов} 

вы, наверное, расписались о получении)) Во что там было 5 млрд вкладывать? В
палатки?)) Очень смешно, простите. 5 млрд долларов — очень приличная сумма,
вероятно, вы не совсем представляете, насколько))

Нет, я не отрицаю, что там были технологии применены, это уже всё триста раз
обсудили, но сам факт вашей уверенности в этих цифрах забавляет)


\iusr{Игорь Смирнов}
\textbf{Alena Morgan}, 

вопросы не ко мне. Вопросы к представителям госдепартамента. Конкретно - к
Виктории Нуланд. Именно она озвучила эту цифру.

И да - незнание не может быть аргументом в дискуссии. Прежде чем сморозить что-то, изучите матчасть...

\iusr{Alena Morgan}
\textbf{Игорь Смирнов} как у вас с английским вообще?))

\iusr{Владимир Домны}
\textbf{Tatjana Voronina} Господи, но мозг для чего-то существует.

\iusr{Наталия Щербо}
\textbf{Tatjana Voronina} потому что кому-то надо занять украинскую территорию и промывают мозги нашим братьям с Украины, а русским не промывают в этом плане, и мы жалеем своих родичей, которые попались на политическую удочку

% -------------------------------------
\ii{fbauth.schukina_aleksandr.nizhnevartovsk.rossia.lvov}
% -------------------------------------

\textbf{Tatjana Voronina} 

ну с чего вы взяли, что России этого хочется? Позиция Украины была следующая:
мы хотим получать заказы на от России на украинские заводы, российский газ по
льготной цене для украинской промышленности и бытовых нужд населения,
направлять на работу в Россию вахтовым методом своих рабочих, которым должны
платить зарплату, которая в экономическом обороте России не участвует. А за
собой Украина оставляли право преследовать русскоговорящих, кричать на площадях
\enquote{москаляку на гіляку}, \enquote{москалі геть з України}, обвинять во всех бедах
русских и делать прочие гадости России. Какое-то очень неравное \enquote{жить по
своему}. Пусть живут, никто не против, но свои проблемы решают сами.


\iusr{Максим Степанюк}
\textbf{Igor Smirnov} 

нас вообще политики такими сделали. Я не навижу националистов, не смотря на
национальность. Не смотря что я патриот, это не значит, что я люблю свое
правительство. Украинцы и россияне дружественные народы, я не имею ничего
против дружбы, но не те нами правят.

\iusr{Tatjana Voronina}
\textbf{Татьяна Корелякова} 

Ни разу не слышала, чтобы украинцы про или русских \enquote{кланяться им в пояс}. Слышу
только то, что они просят отстать от них, оставить их в покое и дать
возможность жить в своей стране так, как они считают нужным.

А разговоры о моей памяти, о дулях и о том, кто первый начал, - пардон, но это
детский уровень.

\iusr{Татьяна Корелякова}
\textbf{Tatjana Voronina} 

на каком уровне вы рассуждаете, на таком и отвечаю. Просто удивительно, что после
семи лет настоящего ада в обмене любезностями кто-то еще может хлопать
ресницами и утверждать, что украинцы 'совсем ничего такого'... это тоже
инфантильность та еще. и в этой теме вопрос кто первый начал очень даже
существенный и как бы ключевой! И, возможно, если тот, кто начал, прекратит и
станет договороспособным и будет вытягивать то, что уже наворотили к каким-то
вменяемым формам, то и прекратится то, что так не нравится. Этого же не было, это
совершенно закономерная ответная реакция! И в покое они сами быть не хотят -
свидетельство тому вся внешняя и внутренняя политика Украины пропитанная
русофобией и постоянным переписыванием истории, доходящим до откровенных
маразмов.

\iusr{Tatjana Voronina}
\textbf{Ирина Николаевна Онищук} 

А какое дело до всего этого вам, например? И откуда вы знаете, где большинство
и меньшинство? И кто каких \enquote{хозяев} хочет - вот почему это должно волновать
граждан другой страны?

Мне кажется, чем меньше ты лезешь в жизнь другого человека, тем спокойнее и
лучше живёшь. Разве нет?


\iusr{Tatjana Voronina}
\textbf{Владимир Домны}
Мозг существует для того, чтобы не отвечать \enquote{не читал, но осуждаю}...
А ещё для того, чтобы не считать себя умнее оппонента  @igg{fbicon.face.smiling.eyes.smiling} 

\iusr{Tatjana Voronina}
\textbf{Наталия Щербо}

Я уже писала, что по работе часто бываю в К Еве и во Львове. И убедилась что
все ужасы, которые рисует 1 канал, ооочень сильно преувпличены. Мне тоже многое
не нравится в Украине. Но, живя в другой стране и будучи русским человеком, не
считаю себя вправе лезть в чужой монастырь со своим уставом. И не понимаю,
почему те, кто живёт в России, считают себя в праве.

\iusr{Ирина Николаевна Онищук}
\textbf{Tatjana Voronina} 

потому что я лучше знаю ситуацию на Украине и настроения людей. Потому что вы
из Братиславы почему-то пытаетесь лезть.

\iusr{Tatjana Voronina}
\textbf{Александра Щукина} 

Ну вот, мне кажется, на этих разногласия и стоило бы разойтись и забыть друг о
друге на пару десятилетий. Чтобы обнулились отношения. Чтобы все улеглось.
Чтобы пришли к власти более мудрые люди.

И только потом начать понемногу думать, братья ли мы или побили горшки
навсегда... А сейчас - это все по гарячему, это углубляет пропасть и заводит в
тупик

\iusr{Tatjana Voronina}
\textbf{Ирина Николаевна Онищук} 

Вы грубиянка, оказывается.  @igg{fbicon.face.smiling.eyes.smiling}  Ну да, люди
везде разные бывают...

\iusr{Ирина Николаевна Онищук}
\textbf{Tatjana Voronina} 

я - грубиянка? Ответила вам вашими же словами  @igg{fbicon.laugh.rolling.floor} 
Вы ничего не знаете об истории Украины, о менталитете регионов, но браво кинулись на защиту.

\iusr{Игорь Смирнов}
\textbf{Максим Степанюк}, 

в первую очередь, не те Вами правят. В России тоже куча проблем, но да Украины им далеко...

И я хотел бы сделать Вам именно независимости. Чего, кстати говоря, и делал
своему однокурснику, проживающему в Киеве ещё в 2013м. Но не случилось ...

\iusr{Marina Griesbaum}
\textbf{Tatjana Voronina} 

Не хотят быть родственниками? А они не родственники, они части одного, нашего
общего организма. Да не может голова жить отдельно от туловища, как и наоборот.
Если у вас что-то болит, надо пойти к врачу и вылечить. Если терапия не
помогает, есть и радикальные методы. Раковые опухоли тоже лечатся. А такой
раковой опухоли из-за бугра денег подкидывают и уговаривают оттуда не лечиться,
что асоциально, аморально, противоречит здравому смыслу. Да, им оттуда хочется,
чтобы организм умер, что же, умереть по их желанию?


\iusr{Максим Степанюк}
\textbf{Igor Smirnov} Вы верите, что после всего будет так как раньше?

\iusr{Игорь Смирнов}
\textbf{Максим Степанюк}, 

ох не знаю. С одной стороны, 30 лет пресса пропаганды не подходит даром. С
другой стороны, если охлосу переключить программу телевидения, все может
поменяться за пару месяцев. Я лично знаю несколько жителей Украины, которые
упёрто верят в американские байки почти 350 дней в году. Оставшиеся 15 они
проводят у родственников в России. И -Аллилуйя - их взгляды меняются
диаметрально ...

Так что ответа однозначного у меня нет. Да и что такое \enquote{как раньше} ?.. Я уже
говорил, что имел некоторые неприятности и до 2014 года ...


\iusr{Максим Степанюк}
\textbf{Igor Smirnov} Вы ведь не напостой имели неприятности?

\iusr{Игорь Смирнов}
\textbf{Максим Степанюк} , сорри, не понял ...

\iusr{Максим Степанюк}
\textbf{Igor Smirnov} я имею в виду: не постоянно, не с каждым, и не везде?

\iusr{Tatjana Voronina}
\textbf{Ирина Николаевна Онищук}
Вечер перестаёт быть томным ... @igg{fbicon.face.tears.of.joy}  А это не интересно  @igg{fbicon.face.smiling.eyes.smiling} 

\iusr{Tatjana Voronina}
\textbf{Marina Griesbaum}
Ой- ой... Вот он, корень всех проблем...

\iusr{Виктор Иванов}
\textbf{Tatjana Voronina} Вы не забывайте, что не все жители хотят жить отдельно. Почему Вы наше мнение не учитываете?

\iusr{Наталия Щербо}
\textbf{Tatjana Voronina} телевизор не смотрю. Не лезу, но вижу сколько людей бежит из Украины, ища политического убежища в России и Европе.

\iusr{Игорь Смирнов}
\textbf{Максим Степанюк}, 

люди вообще не одинаковые. Поэтому, конечно, не постоянно. Но не говорите, что
этого не было. Было. И сегодня я встречаю (уже на просторах интернета) примерно
то же соотношение мнений, что было и в 1990м. Конечно, если исключить
проплаченных...

\iusr{Максим Степанюк}
\textbf{Igor Smirnov} идиотов хватает. Мир? @igg{fbicon.fist.right.facing}  @igg{fbicon.fist.left.facing} 

\iusr{Tatjana Voronina}
\textbf{Виктор Иванов} 

Да я вообще ничего не решаю. Просто обсуждаю тему, стараюсь без эмоций, н кого
не обижая и не обвиняя, а просто задаю вопросы @igg{fbicon.face.smiling.eyes.smiling} 

Согласна, что единодушия часто даже в маленькой семье не бывает, не говоря уже
об огромной стране. Печально это, конечно...

\iusr{Игорь Смирнов}
\textbf{Максим Степанюк}, 

собственно, мы и не ссорились. Вы не находите ? Дискуссия не всегда
предполагает ругань. Более того, ругаясь, мы исключаем возможность консенсуса.
А это не конструктивно.

В спорах рождается истина. В ссорах рождается вражда...

И я Вам благодарен. За то, что несмотря на различие во взглядах, мы нашли
возможность поспорить, не теряя человеческое достоинство.

\iusr{Олег Гераськин}
\textbf{Tatjana Voronina} 

Вы неправы. Никто, во всяком случае в здравом уме, не хочет мешать украинцам
\enquote{жить отдельно}. А вся эта волна \enquote{интереса} к русско-украинскому вопросу
проистекает от агрессивного и упорного желания многочисленных украинских
националистов противопоставить русских и украинцев, вплоть до полного отрицания
собственной истории и имеющих цель - воспитание ненависти ко всему русскому,
даже, если это часть истории Украины. А ещё то, что, якобы, заботясь об
украинской культуре, они полностью отрицают культуру и язык народа, веками
живущего на этой территории. Причём делают это нагло и цинично, и не только в
быту, но и переворачивая с ног на голову и историческую науку и школьные
учебники. Ну, а про средства массовой информации и говорить не приходится...


\iusr{Максим Степанюк}
\textbf{Igor Smirnov} благодарен Вам за интересный разговор. Будете в Украине, заезжайте в Бородянку. Будем рады гостю.

\iusr{Игорь Смирнов}
\textbf{Максим Степанюк} , не поеду. Уж извините...

\iusr{Максим Степанюк}
\textbf{Igor Smirnov} всего доброго Вам. Побольше добрых людей на пути

\iusr{Tatjana Voronina}
\textbf{Олег Гераськин}

Именно я не права? А где находится мерило правды?

Какое дело Вам, например, до того, как соседнее государство расценивает свою
историю, свой язык, свои законы и тд? Почему Вас должно заботить то, любят Вас
там или не любят? Вам же не понравится, если Германия заявит: переводите всю
жизнь Калининградской области на немецкий язык, это были наши земли? Или не
менее набитый оскомину пример с Японией...

У меня один единственный вопрос: почему русские люди всегда говорят: это наша
страна, как хотим, так и живём, не лезьте к нам, и в то же время считают до
себя возможным рассказывать другим народам, как им жить? Почему?

\iusr{Toma Toma}
\textbf{Alena Morgan} 

Ой, я постоянно матюки от украинцев ловлю в ФБ. ))). Хотя сама наполовину
украинка.

У них там что-то совсем страшное происходит. Море агрессии, мата, хамства,
грубости. И самое печальное, что от женщин чуть ли не больше, чем от мужчин
(((.

Жалко мою родную неньку и нормальных украинцев.

\iusr{Ирина Кочетова}
\textbf{Tatjana Voronina}, 

Меркель сообщила Путину что на территории восточной Германии найдено более 700
русских поселений. С какого перепуга чужой народ строил столько своих городов
на чужой территории? Может потому, что эта земля была русской? Иначе какой
правитель разрешил бы строить чужие города на своей земле? Из этого следует,
что восточная Германия была нечем иным как русской землей. Русские строили свои
города на своей земле. Вот и получается, что русских много веков сгоняют со
своей земли. Кто такие русины? Никто иной как остатки русских которых согнали
со своей русской земли! Украинцы никто иные как русские забывшие, что они
русские. Доколе мы будем отдавать свои земли европейским интеграторам? На мой
взгляд пора встать в полный рост и ни пяди не отдавать своей земли!

\iusr{Ирина Кочетова}
\textbf{Tatjana Voronina}, 

кто Вам сказал что украинцы не хотят быть родственниками русским? Это Вы кому
говорите? Тем у кого 70\% русско-украинские семьи? Сами то поняли что сказали?

\iusr{Ирина Кочетова}
\textbf{Tatjana Voronina},

если Вы русская вам не нужно объяснять то, что происходит между Украиной и
Россией! Если Вы русская Вам не должно быть стыдно за Российское государство,
за свое Отечество! Но если не понятно, объясню! На референдуме в 91 году у
народа спросили хотим ли мы жить отдельно от России. И народ 90\% ответил, НЕТ.

С какого перепуга у нас образовались союзные государства?! Марионетки любящие
Европу обозначили государство Украину как проект Антиросссия. У кого живущих на
Украине они спросили, а хочет народ что-бы государство Украина развивалось как
Антироссия? Возмем 2014 год. Произошел государственный переворот! Вдумайтесь!

Гаранты Германия и Франция заверили действующего Президента Януковича в
законной передаче власти и в туже ночь произошел государственный переворот. А
где демократическая передача власти гарантированная Германией и Францией?

Отвечаю. Тех кого американцы выбрали в правители Украины при голосовании
пролетели бы мимо кресла президента на раз, два! Для этого и нужен переворот,
что бы навязать марионеток америки в руководство страны. 

Вопрос: И где здесь воля народа? Страной управляют последователи фашизма. Где
здесь воля народа?

\iusr{Ирина Кочетова}
\textbf{Alena Morgan}, 

отвечу за Татьяну. Как только не называли нас живущих на Донбассе
младоукраинцы. И роли не играло какой национальности был их аппонент! И Фейсбук
их не банил! А за ответ кастрюлеголовым Донбасс и пол России сидело в бане
целый год.

\iusr{Alena Morgan}
\textbf{Ирина Кочетова} 

с тех пор прошло 7(!) лет и уже 2,5 года как 73\% голосовавших выбрали
Зеленского. И не нужно говорить, что он \enquote{фашист}. Кроме того, дважды
переизбирали Верховную Раду, городские и местные советы. Почитайте хоть что-то
про децентрализацию в Украине. Не позорьтесь.

\iusr{Alena Morgan}
\textbf{Ирина Кочетова} \enquote{оппонент}. С запятыми у вас беда. Не позорьтесь.

\iusr{Юрий Григорович}
\textbf{Tatjana Voronina} Это ответы тем, кто искусственно разжигает рознь, находя \enquote{различия} одних от других.

\iusr{Ирина Кочетова}
\textbf{Максим Степанюк}, 

Максимка сколько тебе лет? Наверное не много! Но не в этом дело! «Москоляку на
гиляку» кричали еще в 91 году на западной Украине не добитые нашими дедами
бандеровцы, которые для вас теперь национальные герои! Из-за младости Вы так и
не поняли, что ВСЕ майданы были направленны на агрессию с Россией. 

Найдите проект «Петля Анаконды»! Окружить Росссию не дружественными
государствами. А дальше просто логика. Если Украину отрывают от России, что это
значит? Если Россия это «+» , а Европа это «-«. Оторванная от России Украина
автоматически из + перехотит в « –« по отношению к России.

Кого Янукович убил? Бандерлогов напавших на страну? Ну так это святая
обязанность любого Президента любой страны! Напомните (себе) что гарантировала
Гермния и Франция Януковичу? По закону передачу власти через перевыборы! С
какого перепуга в туже ночь был переворот? Потому что не за Турчинова и тем
более за Порошенко НИКТО на Украине не проголосовал бы! Американцами были
предприняты усилия закончившиеся государственным переворотом. И до переворота
американцами были озвучены кандидатуры на управленческие посты. Это что был
выбор украинского народа? Вы знаете сколько \% от украинского бюджета составляла
торговля Украины с Европой? При переподписании ассоциации сколько осталось \%?
Потеря колосальная! Если Украина торговала с Европой в течении целого года, то
теперь квоты заканчиваются в феврале. 

И потом кто Украину звал в Европу? Если звали то почему восьмой год держат
Украину на европейском крыльце? И открою тайну, Россия на Украину не нападала и
при сильном желании Европы, Америки и Украины на войну так и не явилась за 8
лет! 

И как жительница Луганска достоверно говорю, что наш мирный город бомбила
украинская авиация 2 июня 2014 года! Равна как Семеновку, Горловку, Донецк,
Славянск и остальные восточные города расположенные на Украине. 

Украинские военные убивают гражданских на
востоке Украины. Это гражданская война на территории Украины! И Россия здесь не
причем! Сколько украинские СМИ сперли у нас видеоролики где граждане говорят на
камеру что их обстреляли украинские военные. Только на самом интересном месте
ролик начинает озвучивать диктор и смысл меняется на противоположный! За 8 лет
можно было бы пообщаться с жителями Донбасса и роликов мы прислали бы вам кучу,
где видно и понятно все.

На счет только в 2014 году к русским появилось такое отношение. Есть такой
художник курт осберг посмотри и скажи дату его рисунков, когда он «брал»
штурмом Донецкий аэропорт? И видео «москоляку на гиляку» гуляет с 2010 года. С
2010 года! Что русские вам сделали в 2010 году? И российские ведущие говорят
правду потому что сами многократно посещали города Донбасса в отличии от
украинских СМИ которые только и могут что создавать фэйки! И на счет
рускоговорящих не надо писать чушь! Избивают и убивают тех кто говорит на
русском, не везде это правда. А то так можно договориться что не вы Одессу
сожгли!

\href{https://www.youtube.com/watch?v=WKzikHhtfI4}{%
14.10.13 Марш УПА в Киеве: кто не прыгает - тот москаль!, Newsroom, youtube, 15.10.2013%
}

\ifcmt
  ig https://scontent-frx5-1.xx.fbcdn.net/v/t39.30808-6/255603801_10209251277674846_1530296541226723724_n.jpg?_nc_cat=105&ccb=1-5&_nc_sid=dbeb18&_nc_ohc=T1z__AQeaBMAX-KVkVv&_nc_ht=scontent-frx5-1.xx&oh=f05e88b0538fd70b5c3894d14b13bba3&oe=61A72D7E
  @width 0.5
\fi

\iusr{Ирина Кочетова}
\textbf{Tatjana Voronina},

Татьяна не путайте народ Украины с руководством страны, которые продали свои
души за доллары. Америка вложила в Майдан 3 миллиарда долларов, они
выгодаполучатели. И как водиться у марионетак америки должны быть «псы войны».
А народ Украины прижал уши и ждет куда вывезет «зигзаг удачи». Кого достало или
задело, берет оружие и защищает свои интересы. У кого-то интерес награбить
побольше, а у когото защитить свою семью и честь своих дедов.

\iusr{Ирина Кочетова}
\textbf{Alena Morgan}, 

Алена интересно знать, если бы Вас до в усмерть избил бы муж Вы развелись бы с
ним? В Корсунь-Шевченский убивали крымчан которые возвращались с Майдана.
Крымчане бандерлогам пришедшим к в васти в Украине не простили смерти крымчан и
решили развестись. В чем они были не правы? Убивают это не значит любят! С
Донбассом поступили так же. Приезжайте в Донецк на \enquote{Аллею ангелов} там более
200 имен детей не доживших до совершенноления. И их убили украинские военные

\iusr{Alena Morgan}
\textbf{Ирина Кочетова} мадам, меньше нужно пить))

\iusr{Ирина Кочетова}
\textbf{Maxim Gerasimenko}, 

\enquote{трем} уже восьмой год на минских площадках, а воз и ныне там! Украинцы срывают
заседания, не разрешают делать прямые трансляции минских заседаний что бы не
было видно как они их срывают.

\iusr{Ирина Кочетова}
\textbf{Tatjana Voronina}, 

я живу на Украине и НИКОГДА не просила Россию отстать от Украины и дать
возможность идти \enquote{самостоятельным} путем. Это все равно что дочь пошлет мать
идти на ... Всегда дети шли путем рода. Мы один народ и родовой путь у нас
один! А идущие своим путем попадали в тюрьму за свои деяния. С руководством
Украины будет так же!

\iusr{Ирина Кочетова}
\textbf{Tatjana Voronina}, 

мы украинцы будем ОЧЕНЬ благодарны если вы европейы отстанете от нас хотя бы на
пару десятилетий! А мы постараемся воссоедениться с матерью Россией крепко на
крепко. Мы теперь знаем кого выбирать в руководители страны. А вам европейцам
будем показывать направление движения когда будете вмешиваться в наши отношения
с Россией!

\iusr{Максим Степанюк}
\textbf{Ирина Кочетова} 

я говорю на российском, и? Говорю с теми, кто был в АТО! И? Ездил везде! И?
Меня почему не ущемляют? А насчёт \enquote{не россияне напали}, так от \enquote{не россиян},
у меня 5 друзей погибло. Не чешите пожалуйста. Погибли на границе, когда в 14 с
територии Ростовской области прилители \enquote{подарки}. В Крыму от \enquote{зелёных}
человечков, калеками остались 2 моих друга, они не захотели изменять присяге.

И посмотрите, как в Крыму, у женщины служившей в армии Украины, россияне
отбирают квартиру, которую выдал ещё Ющенко.

\iusr{Tatjana Voronina}
\textbf{Ирина Кочетова} 

знаете, как отвечают в Украине на такие заявления? - Чемодан - -вокзал Россия
@igg{fbicon.face.smiling.eyes.smiling} 

Так что это я вам буду благодарна, если вы избавит меня от общения с вами

Вот не хотелось грубить, но приходится...
@igg{fbicon.face.smiling.eyes.smiling} 

\iusr{Tatjana Voronina}
\textbf{Ирина Кочетова} 

вы ещё глубже покопайте курганы, ещё чьи-то следы обнаружите. Так можно до
бесконечности спорить о национальной принадлежности каждого раскопанного
черепка

\iusr{Tatjana Voronina}

Все, дорогие мои оппоненты, разрешите откланяться, вас ждёт ваш 1 канал, а мне
надоело говорить с глухими и слепыми @igg{fbicon.heart.eyes}{repeat=2} 

\iusr{Игорь Смирнов}
\textbf{Tatjana Voronina}, на это можно ответить - чемодан-вокзал-Европа. Многие так и поступают.

\iusr{Игорь Смирнов}
\textbf{Alena Morgan}, регулярно использую в бизнесе английский. А что ?

\iusr{Larisa Kosa}
\textbf{Tatjana Voronina}, кстати до распада они друг друга не очень-то жаловали. А теперь вдруг полюбили?

\iusr{Valentina Vachatová}
\textbf{Tatjana Voronina} 

а я живу в Чехии, полжизни, и знаю, как чехи рады, что отделились. Они да,
любят словаков, но счастливы, что не вместе. Уровень жизни в Чехии много выше,
чем у соседей и очень много их или на заработках или приходят жить. Но живут
мирно.

\end{itemize} % }

\iusr{Алексей Смоловой}

Белые и красные в гражданской войне часто были родными братьями... Сейчас то же
самое

\iusr{Вова Лобов}

Может их никогда и не было, а вот теперь есть

% -------------------------------------
\ii{fbauth.zaharova_marina.moskva.rossia}
% -------------------------------------

Только украинцы с этим решительно не согласны.

\begin{itemize} % {
\iusr{Константин Константинович}
\textbf{Марина Захарова} 

не украинцы, а рагули и идиоты. Не забывайте, Львов украинским сделала
Советская власть, Львов не открыл ворота Хмельницкому! Какой вывод из этого
напрашивается?

\iusr{Lobsang Rampa}
\textbf{Константин Константинович} молодец. Верно сказано

\iusr{Игорь Смирнов}
\textbf{Марина Захарова}, с этим не согласны только выходцы из Галичины. Всякие разные Ницой, Фарион и Чорновил.

\iusr{Дима Дербышев}
\textbf{Константин Константинович} идиоты, только идиоты

\iusr{Olga Spiwachyk}
\textbf{Марина Захарова} "упоротые" - эти да, не согласны.

\iusr{Аким Рычагов}

Особенно интересен нацсостав населения Галичины до мировых войн. Во Львове так
доля тех, кто мог бы себя назвать украинцем была исчезающе мала. Евреев было на
порядок больше, поляков тоже


\iusr{Анна Батори}
\textbf{Игорь Смирнов}, а всякие разные Смирновы или истории не знают, или осознанно несут бред.

\iusr{Анна Батори}
\textbf{Константин Константинович}, напрашивается вывод, что вы мало того, что не являетесь достойным уважения человеком, так еще и вопиюще невежественны.

\iusr{Константин Константинович}
\textbf{Анна Батори} в чем именно заключается мое невежество?

\iusr{Игорь Смирнов}
\textbf{Анна Батори}, 

если бы вы в своё время, когда все ваши сверстники оделись а одинаковое и
куда-то ушли, не положили собирать бычки по тротуарам, а прошли с ними, то
возможно, в следующие 10-15 лет вы узнали бы много нового и интересного. И
тогда смогли бы судить о тех вещах, названия которых вы выучили, но смысла из
не понимаете. Посему не стоит поминать историю. Равно как и пытаться обвинить в
невежестве людей, знающих настолько больше вас, что вы даже не представляете,
что столько существует знаний в этом мире.


\iusr{Игорь Смирнов}
\textbf{Константин Константинович}, оно не ответит. Ибо невежественно. А в методичке не написано...

\iusr{Анна Черненко-Огир}
\textbf{Марина Захарова} Вы ошибаетесь, по единицам, нельзя судить о всех. Спасибо.

\end{itemize} % }

\iusr{Анна Черненко-Огир}

Профессор очень правильно отметил какой мы народ - единый, родственный !!!

\iusr{Анна Снегина}

Автор, вы уверены во всех своих друзьях? Комменты смущают, а их жалобы тем паче

\iusr{Вера Пищугина}

Зачем сориться и ругаться? Политика вмешалась, а люди поддаются влиянию. Кто-то
наживается, а народ взбудоражен. Давайте жить дружно! Не поддавайтесь
провокаторам. Будьте здоровы!


\iusr{Юрий Смирнов}

Стоит ли говорить о национальных различиях, схожестях и предпочтениях в
наступающую эпоху всеобщей глобализации, когда исчезают не только расовые,
национальные и культурные различия, но и институты семьи, собственности и
государства. Поэтому любая попытка приостановить процесс наступления этой новой
реальности выглядит вполне позитивно, хотя может и являться элементом грязной
политики.

\iusr{Валентин Рогов}

Подойдём к этому вопросу с другого конца-что заставляет людей разных этносов не
перемешиваться друг с другом и веками, сосуществуя на одной часто территории,
не смешивать я? причин много и все они в итоге 1 создают некоторое отторжение на
определённом этапе сближения. Союз, с мпатии могут быть, но естьнезримая
граница, которую сам себе человек не позволяет пересечь. У этих трех народов,
если различия и есть, то настолько не существенные, что позволяют жить вместе.
В частности, такая вещь, как браки происходили совершенно свободно. ну а найти
повод поругаться можно и кровным братьям. Что сейчас и происходит.
Подогреваемое, кстати, теми извне, кому это выгодно. И, конечно, своими,
родными дураками, которых всегда немало в любом йплемени.

\iusr{Сизов Дима}

Он прав - потомки скифов! При этом Западная Украина не входила в этот этнос,
проживавший от Дуная до Байкала!

\begin{itemize} % {
\iusr{Захар Зв}
\textbf{Сизов Дима} 

нет Дмитрий, не так, все русины, живущие как в Червоной Руси (Галычыне), в
Закарпатской Руси, в Пряшивськой Руси, Русько - Лемковськой Республике и в
других странах, все наши.

\end{itemize} % }

\iusr{Язиката Хвеська}

Ага, значит убивать на моей земле - значит любить по-братски  @igg{fbicon.face.grinning.squinting} 

\begin{itemize} % {
\iusr{Дммтрий Кулаков}
\textbf{Язиката Хвеська}
Кто вас убивает?

\iusr{Дммтрий Кулаков}
Окстись

\iusr{Дммтрий Кулаков}
Не надо путать народ с политикой


\iusr{Александр Кыштымов}
\textbf{Язиката Хвеська} живет в польском городе, а вопит моя земля ))

\iusr{Язиката Хвеська}
\textbf{Александр Кыштымов}

\ifcmt
  ig https://scontent-frx5-1.xx.fbcdn.net/v/t39.30808-6/255584840_651056099590514_1359863788228837176_n.jpg?_nc_cat=105&ccb=1-5&_nc_sid=dbeb18&_nc_ohc=gh418SAtCDcAX_jiDNk&_nc_ht=scontent-frx5-1.xx&oh=b8103ba51f7395c69613d855574842a7&oe=61A6AF3A
  @width 0.4
\fi

\iusr{Язиката Хвеська}
\textbf{Дммтрий Кулаков}

\ifcmt
  ig https://scontent-frt3-2.xx.fbcdn.net/v/t39.30808-6/255501765_651056632923794_4701592309012544488_n.jpg?_nc_cat=101&ccb=1-5&_nc_sid=dbeb18&_nc_ohc=YV8R0uUl5f8AX_Vy3LQ&_nc_oc=AQmcapgqGTY9JK4jItwg3bw2afspixIo1ay2ihrVGet4r6ZAW2D_Nt0vGeJwBliR9dg&_nc_ht=scontent-frt3-2.xx&oh=8eb0752510e1ca7165d591619b674a33&oe=61A841F4
  @width 0.4

	ig https://scontent-frx5-1.xx.fbcdn.net/v/t39.30808-6/255617900_651057136257077_2886422460701473196_n.jpg?_nc_cat=111&ccb=1-5&_nc_sid=dbeb18&_nc_ohc=ftI4nFsm-VkAX8in0un&_nc_ht=scontent-frx5-1.xx&oh=73fea303e9a10aada563d4717c0cc719&oe=61A8232D
  @width 0.4
\fi

\iusr{Дммтрий Кулаков}
\textbf{Язиката Хвеська}
Ну и где здесь Украина?

\iusr{Язиката Хвеська}
\textbf{Дммтрий Кулаков} ну дурак  @igg{fbicon.face.grinning.squinting}{repeat=3} 

\iusr{Язиката Хвеська}
\textbf{Дммтрий Кулаков} дММитрий  @igg{fbicon.face.grinning.squinting} 

\iusr{Дммтрий Кулаков}
\textbf{Язиката Хвеська}
А ничего, что вы в Донбассе договориться со своими же гражданами не можете?

\iusr{Дммтрий Кулаков}
\textbf{Язиката Хвеська}
А на своей земле (в Донбассе) убивать мирных граждан - это нормально

\iusr{Язиката Хвеська}
\textbf{Дммтрий Кулаков} дММитрий иди домой, тут водки нет @igg{fbicon.face.grinning.squinting} 

\iusr{Язиката Хвеська}
\textbf{Дммтрий Кулаков} 

«народы Донбасса» так и не определились с названиием страны. Вроде, как
«официально» они же «республики» «лнр» и «днр», а их «защитники» все время
гибнут за какую-то призрачную «наварасию». Ну, вот как так? То и дело в
соцсетях и сайтах лугандона – «они погибли за наварасию». Где это? Что это?
Молчат!  @igg{fbicon.face.grinning.squinting} 

\ifcmt
  ig https://scontent-frt3-2.xx.fbcdn.net/v/t39.30808-6/255506351_651061236256667_4931294528323339430_n.jpg?_nc_cat=103&ccb=1-5&_nc_sid=dbeb18&_nc_ohc=Mx7nmcRPw7kAX_nz-oN&_nc_ht=scontent-frt3-2.xx&oh=cd2ace795afe8017671363f3dad858eb&oe=61A70BF3
  @width 0.4
\fi

\iusr{Язиката Хвеська}
\textbf{Дммтрий Кулаков} Русские будут дохнуть хоть за караван-сарай, хоть за Марс с Юпитером, абы деньги платили.
Русскому скажи, прими мусульманство, сделай обрезание, денег дам, он сделает. Разбей голову, прыгни с крыши, можно даже денег не давать, чисто на слабо или за водку. Прыгнут. Убьются.

\iusr{Язиката Хвеська}
\textbf{Дммтрий Кулаков} 

жители некогда мирных Луганской и Донецкой области не могут мирно
сосуществовать в одной стране, пусть даже фейковой, я вообще молчу. Донецкая
спесь не дает донецким подчиниться луганским, и наоборот! Вот здесь, даже
чуть-чуть горжусь. Даже Путину не по силе объединить этих донбасянских
«братьев».  @igg{fbicon.face.grinning.squinting} 

\iusr{Язиката Хвеська}
\textbf{Дммтрий Кулаков} 

Где на карте мира Россия? Не на карте России, нет. Ну, мало ли, что вы там себе
нарисуете, а вот на политической карте мира, где Россия? Тупик? России-то нет,
как и «наварасии». Это фейк. Есть официальная страна, с официальным названием,
гимном и флагом, и называется она – Российская Федерация. То есть содружество
федеративных республик. А Россия, это, извините, самопридумки. Есть конституция
России? Нет!

\iusr{Андрей Тихомиров}
\textbf{Язиката Хвеська} только убиваете вы! А Россия пока к сожалению, смотрит со стороны, но уничтожить Донбасс не даст, это точно!

\iusr{Александр Кыштымов}
\textbf{Язиката Хвеська} Донбасс -русская земля!

\iusr{Максим Степанюк}
\textbf{Дммтрий Кулаков} Вы это сами видели? Лично?

\iusr{Максим Степанюк}
\textbf{Александр Кыштымов} русская, но не российская.

\iusr{Ирина Кочетова}
\textbf{Язиката Хвеська}, а Львов бомбили россияне? А вот Луганск бомбили украинские военные! Так кто кого и от чего освобождает?

\iusr{Leonid Golberg}
\textbf{Александр Кыштымов} Неужели? Донбас - УКРАЇНА!

\end{itemize} % }

\iusr{Анна Снегина}

А чубатые себе приписали и Сикорского, и Королева и ТД. привычка

\begin{itemize} % {
\iusr{Alena Morgan}
\textbf{Анна Снегина} в смысле — приписали? Королёв, например, мой дальний родственник. Я из Киева, он из Житомира, это рядом. Может, это вы кого-то там "приписали"?))

\iusr{Роман Сулицкий}
\textbf{Alena Morgan} Когда делали окраинство, то постарались произношение вписать в грамматику. Так, например, Петренков, стал Петренко, Иваненков - Иваненко, а вот мой предок выходец с левобережья Днепра с переселением в Великороссию был и остался Клименковым. Королёв - даже не Короленко, а посему даже в псевдоокранцы записан быть не может. Это крестьян -онко - енькали, а купцы и дворяне носили русские прозвища. Часть козакыв носила татарские фамилии с окончаниями на -чук - щук.

\iusr{Alena Morgan}
\textbf{Роман Сулицкий} Зеленский — это у нас кто?))

\iusr{Роман Сулицкий}
\textbf{Alena Morgan} А Морган?  @igg{fbicon.laugh.rolling.floor} 

\iusr{Alena Morgan}
\textbf{Роман Сулицкий} там связи с викингами возможные))

\iusr{Роман Сулицкий}
\textbf{Alena Morgan} Нет такого слова, какое еврей не сделал бы фамилией. Еврейская пословица.

\iusr{Alena Morgan}
\textbf{Роман Сулицкий} у Королёва, простите, мама украинка, Москаленко, казачка, по ней у нас и родство. И родился он в Житомире. Папа — беларус или русский. Хотя, конечно, фамилия Москаленко тоже подозрительная))

\iusr{Анна Снегина}
\textbf{Alena Morgan} когда родился Королев понятия украинец не было. Малороссиянин

\iusr{Роман Сулицкий}
\textbf{Alena Morgan} 

Был у меня товарищ с фамилией Прохоренко. Отец по паспорту белорус, мать по
паспорту русская. Пошёл он получать паспорт, а в нём в графе национальность
стоит \enquote{белорус}. Кинулся возмущаться - как так?! Ему и ответствовали, мол ты,
раздолбай, в анкете не указал какую национальность выбираешь из двух
национальностей родителей, поэтому записываем таких по умолчанию по отцу. А
тебе какая разница? - Да ни какой, просто брат русский, а я - белорус, который
ни слова не знает по-белорусски. Пошёл в армию, оказался в элитной
радиоразведывательной части в Латвии. В учётной карточке указал, что русский.
По завершении учебки начальник части перед строем поздравляет всех прошедших
подготовку и порицает непрошедших. Смотрит в личное дело нашего
героя: \enquote{Прохоренко, ты же говорил, что - русский, а по паспорту - белорус. Аааа
хохол хитрый!} Вот в этом вся суть русского триединства - всем было пофиг, на
самом деле, кем человек записан, всё равно нет никакой разницы или возмущения
тем, что тебя считают русским, белорусом или украинцем - один народ. Есть,
конечно, какие-нибудь карпатско-закарпатские метисированные с венгро-румынами
племена, вот из них и надо было делать окраинский Диснейлэнд всем другим в
назидание, а считать казачку Москаленко иной окраинской народностью, це, пардон
муа, мовитон.

\iusr{Анна Снегина}
Конечно

\iusr{Alena Morgan}
\textbf{Анна Снегина}

\ifcmt
  ig https://scontent-frt3-1.xx.fbcdn.net/v/t39.30808-6/254961864_10223334777112176_6331750010354020324_n.jpg?_nc_cat=104&ccb=1-5&_nc_sid=dbeb18&_nc_ohc=D_5BEIypHjgAX8pIfWC&_nc_ht=scontent-frt3-1.xx&oh=be4a4e7242e64da9cb496900a012c0dc&oe=61A67B74
  @width 0.4
\fi

\iusr{Анна Снегина}
Не удивительно, все знают о насильственной украинизации с 30 х годов и коренизации

\iusr{Ден Владимирович}
\textbf{Alena Morgan} Гагарин ваш папа, Пушкин ваш прадед))) успокойтесь шумеры, борщ ваш)))

\iusr{Alena Morgan}
\textbf{Ден Владимирович} может, не будете хамить? Я назвала фамилию матери Королева, которая и моя родственница также. Привыкайте к аргументам. Никто вас за язык не тянет. Всем ментально убогим я всё равно не смогу ответить.

\iusr{Ден Владимирович}
\textbf{Alena Morgan} стуканула уже?)))

\iusr{Василий Писаренко}
\textbf{Анна Снегина} была в 92 году, кажется, издана книга \enquote{100 великих украинцев}, там первый Рюрик, в почётном пантеоне Чехов и Голда Мейр.

\iusr{Alena Morgan}
\textbf{Василий Писаренко} Голда родилась и жила в Киеве, она — украинская еврейка. А Чехов сам себя малоросом называл, бабушка точно его была малороской-украинкой, фамилия Шимко.

\iusr{Игорь Терзи}
\textbf{Alena Morgan} ох ты ж.... кто родители С. Королёва и деды с бабками? Откуда приехали?

\iusr{Alena Morgan}
\textbf{Игорь Терзи} по папе он или беларус или русский, а по маме — украинские казаки и греческие колонисты.

\iusr{Alena Morgan}
\textbf{Ден Владимирович} по вашим комментариям и видно, кто морально убогий.

\iusr{Анна Снегина}
\textbf{Василий Писаренко} 

есть у них в прессе и об украинских корнях даже Роберта оссейна. Бедные
азербайджанцы Хуссейны не знали, что проездом из Баку в Париж остановятся в
Киеве и сразу станут щирими

\iusr{Михаил Александрович Хаевский}
\textbf{Alena Morgan} по Украине проходила черта оседлости для ЕВРЕЕВ.
Вы их тоже запишите в украинцы ... больше народу будет и на Израиль можно права предъявить  @igg{fbicon.beaming.face.smiling.eyes} 

\iusr{Игорь Смирнов}
\textbf{Василий Писаренко} , 

помню-помню. Мне тут в 2015м киевлянин рассказал, что Чехов писал в биографии,
что родился в украинском городе Таганроге. Я не поверил. Прокопал интересно и
нашёл кучу ссылок на эту цитату. Но !!! Все из украинского сегмента. Тогда я
взял собрание сочинений Чехова, изучил и ... не нашёл этой фразы. Так вот и
создаются окна Овертона.


\iusr{Василий Писаренко}
\textbf{Игорь Смирнов} я читал, что у него мать украинка была.

\iusr{Игорь Смирнов}
\textbf{Василий Писаренко} , 

даже если бы это было так, неужели это оправдывает массовую ложь ???

Но если почитаете биографию матери А.П. Чехова, то все сомнения развеятся :

Чехова Евгения Яковлевна (в девичестве Морозова (1830 — 1919г), - мать А. П.
Чехова. Морозовы происходили из деревни Фофаново Владимирской губернии.
Семейное предание утверждает, будто Морозовы были из строобрядцев и приходились
дальней родней известным мануфактурщикам Савве и Викуле Морозовым.

Отец, Яков Герасимович Морозов, как рассказывали в семье, успешно торговал
сукнами в Моршанске Тамбовской губернии. В 1847 году он умер от холеры в
Новочеркасске, куда уехал с сыном Иваном по торговым делам. Его жена Александра
Ивановна (мать Евгении Яковлевны) вместе с дочерьми Феодосьей и Евгенией,
отправилась искать могилу мужа и оставшийся после него товар. Обратно они не
вернулись. В Ростове-на-Дону пересеклись пути двух семейств, Чеховых и
Морозовых. Они поселились в Таганроге, породнились, после женитьбы Павла
Егоровича на Евгении Яковлевне, в 1854 году.


\iusr{Василий Писаренко}
\textbf{Игорь Смирнов} спасибо, все врут под себя, получается.

\iusr{Игорь Смирнов}
\textbf{Василий Писаренко} , несомненно.
Но что касается некоторых молодых государств, им придумывают историю под копирку. И в основном за пределами этих стран. Собственно, это даже не страны, это колонии...

\end{itemize} % }

\iusr{Вера Попович}
О да о чём спор все от одной или обезьянки

\iusr{Михаил Соколов}

Нас с покон века разделяли, что бы мы шли друг против друга, что делают и
сейчас, пора положить этому конец, пока мы друг друга не уничтожили....

\begin{itemize} % {
\iusr{Максим Степанюк}
\textbf{Михаил Иванович Соколов} Путину это скажите

\iusr{Михаил Соколов}
\textbf{Максим Степанюк} всё вам кто то виноват, с себя надо начинать, вот и вся проблемма

\iusr{Максим Степанюк}
\textbf{Михаил Иванович Соколов} вот с этого и начинайте. Типа: сами виноваты.

Только это Россия увеличивает список недружественных стран, вместо того, чтобы
пойти на перемирие и не лезть как нацисты в чужое государство

\iusr{Михаил Соколов}
\textbf{Максим Степанюк} тут речь идёт о едином народе, не о чем говорить.

\iusr{Михаил Соколов}
А Россию обвинять в фашизме это перебор.

\iusr{Максим Степанюк}
\textbf{Михаил Иванович Соколов} 

та вы шо? 1941- нацисты нападают на СССР, с которыми был заключён договор о
ненападении и союзный договор. Отговорка: "защита прав нацменьшинства немцев,
проживающих на этой территории". И Гитлер тоже кричал, что арийцы - великая
нация.

Не вся Россия такая, не все россияне, но большинство

\iusr{Михаил Соколов}
\textbf{Максим Степанюк} 

стоп а вы совсем забыди, что СССР самые последние подписали
договор с германией, вижу с вами бесполезно спорить. прошу
прощения.

% -------------------------------------
\ii{fbauth.sokolov_mihail.riga.latvia}
% -------------------------------------

Суть идёт совсем о другом, о воссоединение Руси и хватит мерется у кого круче
яйца, от нас только этого и ждут чтбы мы друг друга уничтожели вот как то так.

\iusr{Максим Степанюк}
\textbf{Михаил Иванович Соколов} вот и надо начинать с правительства. Я не говорю что все россияне - нацисты. Среди нас тоже такие есть.

\iusr{Максим Степанюк}
\textbf{Михаил Иванович Соколов} я не говорю о подписании договоров. А говорю о перенятом опыте.

\end{itemize} % }

\iusr{Алекс Ван}
Все по плану, сначала разделить, потом столкнуть лбами, чтоб поубивали друг друга и ослабли. Разделяй и властвуй.

\iusr{Евгений Княгинин}
Вместо понятия \enquote{народ} Сикорскому следовало использовать термин \enquote{популяция}.

\iusr{Константин Лахненко}
РОДСТВЕНИКИ НЕ УБИВАЮТ СВОИХ.

\begin{itemize} % {
\iusr{Ольга Ивлева}
\textbf{Константин Лахненко} а как же Какие и Авель?
\end{itemize} % }

\end{itemize} % }
