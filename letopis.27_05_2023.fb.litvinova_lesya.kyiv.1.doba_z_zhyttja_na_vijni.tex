%%beginhead 
 
%%file 27_05_2023.fb.litvinova_lesya.kyiv.1.doba_z_zhyttja_na_vijni
%%parent 27_05_2023
 
%%url https://www.facebook.com/LitvinovaLesya/posts/pfbid02Ne6KBsE8q9WUL8hrEfr3w1CVUE1nDehbwJg94D5oA5ddL9DWSZF8U4MzY5RDE7H3l
 
%%author_id litvinova_lesya.kyiv
%%date 27_05_2023
 
%%tags vojna,vojna.2022
%%title Доба з  життя на війні
 
%%endhead 

\subsection{Доба з  життя на війні}
\label{sec:27_05_2023.fb.litvinova_lesya.kyiv.1.doba_z_zhyttja_na_vijni}

\Purl{https://www.facebook.com/LitvinovaLesya/posts/pfbid02Ne6KBsE8q9WUL8hrEfr3w1CVUE1nDehbwJg94D5oA5ddL9DWSZF8U4MzY5RDE7H3l}
\ifcmt
 author_begin
   author_id litvinova_lesya.kyiv
 author_end
\fi

Вибачте, це буде довгочит. Доба з  життя на війні.

......

Точка  старту  —   на околиці прифронтового  містечка.  Висуватися маємо  по
черзі трьома маленькими групами по  7-8 людей. Нові позиції —  в повній  дупі.
Але ніби друга лінія.  

Штаб запевняє,  що там раніше не було ніяких позицій. Просто порожня лісосмуга.
Нам треба окопатися й закріпитися. Сапери їдуть в кожній з груп, аби
забезпечити їм безпечний прохід. 

\ii{27_05_2023.fb.litvinova_lesya.kyiv.1.doba_z_zhyttja_na_vijni.pic.1}

Стартує перша  група. За пів години  друга. Я чекаю зі своєю третьою.  Поки
чекаю, біля нас починають зупинятися  автівки. З них просто на асфальт
вивантажують хлопців з важкими пораненнями та скляними очима. Хлопців, удвічі
молодших за мене і вдвічі ширших у плечах. Хтось лягає на траву, хтось сидить
на асфальті.  Мовчать. Дивляться одночасно за горизонт і кудись углиб себе. І
це страшно. 

— Хлопці, я медик. Допомога потрібна?

Не чують. Підходжу по черзі.  В одного тече з-під турнікенту — намотане
кітайське гімно. Накладаю поряд нормальний. Кровотеча зупиняється. Інший
відштовхує мене руками: "Не чіпай мене!". Відходжу до іншого. Зрізаю штанину,
з-під якої тече. Уламкове. Але можна обійтися бандажем.  

Потрошку починаємо спілкуватися. Це і є ті, в кого за спиною ми мали будувати
другу лінію. Останні. Тепер перед нами нікого немає. 

— Скільки вас? — ледь чутно питає  хлопець  з мертвими очима.

— Три відділення.

— Ви не втримаєте. Ніяк. До вас навіть в окопи заходити  не будуть. Роз'їбуть з
арти. Не їдьте. Нас не змогли евакуювати. Ми повзли майже добу, поки нас
знайшли. 

Чіпляється за мою руку. 

— Чуєш,  говори з командиром. Не їдьте...

Розвідка , яка заводить наші групи, і евак для хлопців приїжджають майже
одночасно. Короткий дзвінок командуванню, підтвердження наказу, незважаючи на
зміну обставин.  Хлопці зі скляними  очима вантажаться в евак. Ми — по своїх
аватівках. Розвідка  турботливо  намагається замінити мене на когось іншого.
Але добровольців  немає. Та я й не планую лишатися. Там — мої хлопці.  Мій
чоловік, мої побратими. Якщо дістануть поранення — я маю бути поряд. А там —
уже  як пощастить. 

Летимо  по розтрощеній  дорозі, намагаючись  лавіювати між ямами від снарядів
та підбитими  автівками. На ходу вилітаємо з автівки й одразу падаємо за
горбочок: розриви ліворуч, праворуч, неможливо зорієнтуватися, що саме. Бо
прильоти накладаються один на одний суцільною канонадою. Автівка, не
зупиняючись ані на секунду, робить розворот і  летить в зворотному напрямку.  

\ii{27_05_2023.fb.litvinova_lesya.kyiv.1.doba_z_zhyttja_na_vijni.pic.2}

...Вітаємо в пеклі. 

До початку посадки — метрів 700 полем. І ще метрів 800 самої посадки треба
перевірити. Бігом, бігом. Дистанція.  Прильот — упали. Підскочили. Бігом,
швидше. Прильот. Ще швидше. Дерева. Всі  живі? Добре. Видихаємо...

Хлопці з роти лежать, намагаючись злитися з землею.  Криє з усіх боків.  Я
обережно починаю заповзати в посадку. 

— Поки  я не скажу рухатися — до мене не підходити. Заходити чітко по моїй
стежці. Там біля вас двохсотий. Не чіпати. Це не наш. І взагалі нічого не
чіпати. Ніякі ніштяки не збирати. Зрозуміли? 

Посадка геть не схожа на "просто лісосмугу на другій лінії". Суміш завалених
снарядами дерев і гілок, чиїхось речей, якогось залізяччя, землі й ще
незрозуміло чого. Ну, як є.  Лежачи. По гілочці. Зі щупом. 

— Отут. Заходьте по моїх слідах. Від дерева до дерева — чиста смуга. І бігом
копайтеся. Щілину і нори. Не встигнете  — загинете. 

— А в мене лопати немає...

— Ти, блядь, чим думав? Ножа бери. Ножем і руками. Виходу немає. Ти або
вкопався, або труп. 

Трупом бути не хочеться, тому коли я вдруге перебігаю їхню точку, бачу, як
хлопець із шаленою швидкістю риє собі окоп ножем. 

— Шторм, Шторм, на зв'язок. Ви там як? Я своє зробила. Ви як? 

Виявляється,  що перша група не тільки вже почала копати, а ще й знайшла в
посадці сюрпрайз — окоп з дружнім підрозділом.  Дивом не поклали один одного.
Бо їм так само впевнено сказали, що посадка порожня. 

На радості, що прийшло хоч якесь підкріплення, саперів люб'язно запросили до
себе в окоп, чекати, поки можна буде вибратися на базу. 

Хай ручки не болять у того, хто  копав. Кілька ходів і навіть кілька ніш, у які
можна сховатись, якщо вже зовсім поливають. 

"Зовсім поливати" почали майже одразу.  Над окопом застигла ворожа пташка.
Потім ще одна. Потім позицію почали розбирати на частини.

— База, база, дайте підтримку артою. Нас криють.

— Робимо, що можемо.Тримайтеся. 

Ми тримаємось і рахуємо. На кожен постріл з нашого боку в зворотний бік летить
25-30 снарядів.  

— Де в них, цікаво, снарядний голод? — спокійно цікавиться сивий чоловік поряд
зі мною. Він, як і я, з Києва. Ми швидко знаходимо спільних знайомих, і нам
комфортно поряд. Десь за дві години він загине. А поки всміхається й
намагається пригостити мене снікерсом. 

У рації чутно, як панікує хтось на іншому боці посадки. Вони бачать, як
відходить ще один дружній підрозділ із флангу. 

— Нас тут взагалі скільки? 

— Ну, нас шестеро. А вас скільки зайшло? 

-— Двадцять вісім. З них троє  —   сапери.  У нас навіть бк із собою немає. Ми
на розвідку йшли. Так, по парі магазинів. Решта — саперні ніштяки, але тут вони
не згодяться. 

Пташка над головою висить низько,  але немає кута, під яким реально дістати.
Висунути ж голову  —   неможливо. В край бруствера летить стрілкотня, земля
гуде від розривів. 

— Ото вони щедрі... Чуєте? Це танчик. А вночі градами закидали. І птурів не
шкодують по піхоті-то.  Це ж які гроші зараз на нас витрачають... І не можуть
не бачити, що нас тут три десятки людей з калашами. Цікаво: як ми звідси
виходити будемо? 

— Та ніяк, по ходу. Займаємо кругову оборону. Що там у кого лишилось? 

Один за другим просто на бруствер кладуться постріли із СПГ.  Вилізати нема
куди — наш край посадки за лічені години викошений ущент.

...Спалах перед очима. Одночасно удар молотом по голові. На кілька секунд
відключаюсь. Приходячи до тями, бачу, що поряд лежать четверо людей. Один
стогне, інші мовчать. 

— Хімік, що в тебе?

Із сусіднього крила окопа чую його голос, але нічого не розумію, крім того, що
живий. У вухах гуде. Картинка перед очима мерехтить. 

Найближче  —     Влад. Засипаний пилюкою, в розколотому шоломі, по вуху тече
кров. Намагаюся зсунути його з місця в нішу. Майже нічого не розуміє, але
інтуїтивно відштовхується ногами, допомагаючи. Живий. Добре.

Ще двоє явно дихають і рухаються. Навіть матюкаються. З ними потім. Мел лежить
на животі і стогне. 

— Мел, ти мене чуєш? 

Не можу сама до нього дотягнутися. Навколо навалено рештки рюкзаків, понівечені
калаші, уламки гілок. 

— Допомагайте. Ну! Хлопці! Не стоїмо. Переверніть його. Я не впораюся. Не
стійте, бля, рухайтеся. 

Перевертаємо на спину. Під ним калюжа крові. Броник мокрий наскрізь, кров на
губах. І очі, які не можуть на мені сфокусуватися. 

- Тягніть  його в нішу. Треба броню знімати. Та посуньтеся ви! У вас максимум —
легка контузія. Дайте йому шанс. 

- Мел, дивися на мене. Говори, чорт тебе роздирай!!!

Знімаю броню, намагаємося розрізати одяг. Неушкоджена грудна клітина, але крові
— відро. Я очима бачу, як він умирає. І розумію, що скоро доведеться сказати
вголос "Мел — 200".  

— Не кидай мене,  - ледь чутно шепоче він мені. 

Обіцяю добити просто зараз, якщо не припинить молоти дурню. 

Нарешті дивом намацую дірку над ключицею, звідки б'є фонтан. Туди зручно
поміщається палець. Тисну. Кровотеча зупиняється. 

— Не сьогодні,  Мел. Не сьогодні...

Решта розгублено дивиться на мої спроби.  Влад усе ще не може прийти до тями. 

— Хімік, викликай евак. У мене важкий. Чуєш? 

— У мене теж. Надаю допомогу. Евак  викликав. 

Намагаюся знайти аптечку Мела — десь відірвана. Мій наплічник чи то розірваний,
чи то заритий. Не можна так робити, але розпаковую свою аптечку — там на трьох
вистачить. 

— Дістаньте мені хтось гермостатики і бандаж. Дайте в руку. 

Ніхто не розуміє, чого я хочу. Знову матюкаюся. Намацую сама, розриваю зубами.
Щільно тампоную отвір. Притискаю. 

— Хімік, ти зі своїм сам упораєшся? 

— Ні. 

Чорт.... Беру Владьку за руку, кладу їх Мелу на рану. 

— Тисни. Щосили. Ти мене розумієш? 

Киває. Тисне. Молодець. 

— Я повернуся. Тільки не відпускай.

Знов здригається земля навколо. Летить земля і щепки з дерев. 

У сусідньому рукаві окопа Хімік намагається накласти турнікет сивому чоловіку,
який пригощав мене солодощами. Той сидить,  прихилившись спиною до стіни окопа.  

— Кладемо. Я так не бачу. Ну! Давайте! Він важкий. 

Хлопці рухаються, як зомбі. Але на крик і погрози реагують. 

Чорт. Там не нога. Там розірваний пах. Краї рани навіть не бачу. Він ще живий.
Дивиться на мене й ніби навіть усміхається. Леплю якусь дурню про "розповім
потім дружині, що я тебе голяка бачила", поки зрізаю штани і труси, одночасно
намагаючись затампонувати  пахову артерію.. Вже бачу, що це марна справа,  бо
розірвані кишки. Але він усе ще живий і дивиться на мене. Навіть ковтає
знеболення . 

— Хімік, тримай тампонаду, мені Мела перев'язати треба. Як евак?

— Шукають.

Влад тримає гемостатик на рані так міцно, що я диву даюся, враховуючи, що він
сам майже без свідомості. 

— Бандаж. Я сама не накладу. Припідніміть його трохи. Давайте. 

- Мелік, тримайся. Не спи, гадюка. Нам ще до еваку дістатися. Будь з нами. 

Місце незручне. Два бандажі. Три бандажі. Ніби ок. 

— Бігом усі випили пілпаки! Чого я нагадую?! І знайдіть аптечку Мела, запхайте
в нього пігулки. Він ковтає. Води дайте запити.  Я — до Хіміка. 

Сивий чоловік лежить, дивлячись у небо. Хімік тримає тампонаду.  Він бачить те
саме, що й я. Перевіряю дихання і пульс. Невідомо кому кажу вголос: "Час смерті
— 16:01" 

В окопі лишаються четверо контужених,  один важкий, один загиблий, двоє
переляканих, але цілих, і Хімік.  Перелякані несуть щось про "ми тут усі
загинемо". Хімік спокійним голосом повідомляє, що якщо вони допоможуть винести
пораненого, можуть валити разом з еваком. Це не його підрозділ. Їхнього
командира вбито тільки що. 

Нарешті можу подивитися Влада. Голова ціла, вухо розірване, критичних кровотеч
немає, кінцівки цілі. Решта — потім. Неважливо. Рухатися можеш? Усе розумієш?
Добре. В нас попереду купа планів. Пам'ятаєш? Хер ми сьогодні загинемо. 

Оживає радейка. Евак виїхав. Чекати не буде. Дорога птуриться активно.
Встигнете заскочити на льоту — забираємо важкого й контузії.

— Хлопці, швидко. В нас 20 хвилин бігом. Якщо пощастить. Перекладаємо на ноші.
(Господи, дякую, що вони в мене є!) Кидайте все, що зайве. Нам — бігти. 

Десь на краю свідомості з'являється думка, що за кинуту зброю ми  огрібемо по
поверненню. З'являється і зникає. Бо відкопувати її залишки, втрачаючи час, —
це дурня. Головне — дотягти Мела. Іншого шансу у нього може не бути. 

На виході з окопу ставлю Хіміку безнадійне запитання, на яке розумію  відповідь
і так :

— Може, ти з нами? Що ти зробиш один?

— Наказу залишати позиції не було. Я залишаюся. 

Обіймаю його від усього серця.  Мені соромно його лишати самого, але ми вже не
бійці, а обуза. 

— Чуєш, Хімік, я маю за честь служити з тобою поряд. Я хочу, аби ти це знав.
Хай чим це закінчиться . 

(Хіміка поранять у ту саму ніч. Зможуть евакуювати з розтрощеними ногами. Але
позицію він утримає до заходу іншої групи.)

Ми йдемо по полю поміж прильотів. Уже не намагаючись падати й чекати. Бо: або
ми встигнемо, або Мел помре. А я йому обіцяла, що це буде не сьогодні.
Намагаюся підштовхувати тих, хто попереду, і контролювати Влада, який замикає.
Він увесь час намагається знепритомити, але вперто рухається.  Я вірю, що він
зможе. Що ми всі зможемо.  

Щільність обстрілу така, що здається, ніби ми біжимо крізь залізно-вогняну
зливу й не зрозуміло, чому все ще живі. 

До еваку ми добігаємо якраз тоді, коли він розвертається. Майже на ходу пхаємо
ноші на сидіння, поряд сідає  Влад. Дякувати Небу, він не бачить того, що бачу
я: на моєму місці в кузові двоє "переляканих". Мені місця не вистачить. 

Б'ю рукою по автіці, даючи знак, щоб рухалася. Кричу Владькові, що обов'язково
повернуся, сподіваючись, що він мене чує. 

— Ану бігом до нас! 

Це розвідка. У них там схованка на кілька людей. Навіть не одна. 

— Давай, давай, у бліндажик. Не маяч. Красуня ти наша. Виберемося. 

У бліндажику вже сидить хлопець з дружнього підрозділу з пораненням ока. Він не
може зв'язатися зі своїми, тож будемо якось витягати ми. 

— Ану покажи око! О! Уламочки. Та не лізь ти туди пальцями!  Щиток є? Нема?
Господи, хто вам все це комлектував.  Тримай мій. Зараз зафіксуємо. У тебе хоч
пластир і бинт є? Давай сюди. 

У бліндажик, у якому і так ледь вміщаємося ми двоє, вламується величезний
молодий хлопець.  Підкріплення на позицію. Чекати темряви, аби зайти. 

Я бачу, як йому страшно. Насправді. Але він усе одно сповнений рішучості.
Питає, як краще дістатися,  перепаковує рюкзак — уже зрозуміло, що спальник,
каремат і купа консервів — це зайвий вантаж. 

Знов пташка.  Просто над бліндажиком. Починаються прицільні прильоти. Накати
над нами здригаються, але тримаються. Разом з накатами здригаються хлопці. Той,
що з пораненим оком, відверто панікує. 

— Не ний. Чуєш? Ми майже вибралися. Не зможуть забрати — вночі вийдемо пішки.
Там кілометрів 7-10. Подолаємо. Не ний. 

Той, що молодший, мовчить і лише дивиться. 

— Тобі страшно?

— Звичайно, страшно. Я ж не дурень. Але я піду. Там хлопці. 

— А тобі рочків скільки?

— Двадцять чотири. 

Знов бахкає просто по бліндажику. Він інстинктивно ховає мені голову в коліна,
а я інстинктивно гладжу його по касці. 

— Дивись на мене. Ти — молодий, міцний. Я — вдвічі старша. В мене спина майже
не працює. В мене зайва вага й слабкі ноги. Я впоралась. І ти впораєшся. Я в
тебе вірю. 

Усміхається. 

 — Ей, красуне, лови! 

У бліндажик влітає шоколадка. Це розвідка з сусідньої дірки намагається мене
підбадьорити. 

— Ви цигарку мені кинули б, — намагаюсь обнагліти я, але черговий прильот
перекриває  мій голос. 

— Живі?

— А то! Сонечко сяде — спробуємо прорватися. Ти добіжиш до автівки? 

***

Я добігла, до речі.  І автівка летіла, як шалена, по дорозі, де майже не
лишилося вільного місця поміж палаючих розбитих автівок. Це потім я дізнаюся,
скільки екіпажів не змогли прорватися. Потім.А зараз ми летимо дорогою життя. 

Стабік. 

— Це не моя кров, це побратима. У мене дрібні уламки та контузія. Дайте
зв'язок. У мене чоловік і побратим  десь у лікарні. Мені домашніх треба
набрати. Та нікуди я не поїду. Я норм. Мені моїх треба відшукати. 

Але хто буде тут вже слухати контужену, правда? Я для них   вже не боєць, а
пацієнт. Лікарі — вони такі... 

Я це все до чого. Це не щось надзвичайне. Це звичайний робочий день на
передньому краї. І якщо ви не тут, ви маєте змогу підтримати тих, для кого це —
буденність.

\par\noindent\rule{\textwidth}{0.4pt}

Отримувач: БО "БФ "СВОЇ"\par
Код ЄДРПОУ: 39293651\par
IBAN: UA883052990000026004015002735\par
в АТ КБ "ПРИВАТБАНК"\par
Картка: 4246001030029122\par
@СВОЇ\par

Лише за тиждень з цього рахунку було сплачено майже чверть мільйона. На
кісткові імпланти для поранених. 

\par\noindent\rule{\textwidth}{0.4pt}

Карта 4149609014791422 (Ірина Гук/ Приватбанк – тактична медицина).

\url{https://leleka.care/donate}

Якби не вони - мені не було б чим надавати допомогу. І не тільки мені.

Зробіть репост, якщо не важко.
