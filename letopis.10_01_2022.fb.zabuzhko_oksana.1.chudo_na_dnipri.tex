% vim: keymap=russian-jcukenwin
%%beginhead 
 
%%file 10_01_2022.fb.zabuzhko_oksana.1.chudo_na_dnipri
%%parent 10_01_2022
 
%%url https://www.facebook.com/oksana.zabuzhko/posts/467521711405978
 
%%author_id zabuzhko_oksana
%%date 
 
%%tags maidan2,ukraina
%%title ЗАХИСНИКАМ МАЙДАНУ, АБО ПРО "ЧУДО НА ДНІПРІ"
 
%%endhead 
 
\subsection{ЗАХИСНИКАМ МАЙДАНУ, АБО ПРО \enquote{ЧУДО НА ДНІПРІ}}
\label{sec:10_01_2022.fb.zabuzhko_oksana.1.chudo_na_dnipri}
 
\Purl{https://www.facebook.com/oksana.zabuzhko/posts/467521711405978}
\ifcmt
 author_begin
   author_id zabuzhko_oksana
 author_end
\fi

ЗАХИСНИКАМ МАЙДАНУ, АБО ПРО \enquote{ЧУДО НА ДНІПРІ}. І ще таке. Нині, спостерігаючи за
трагедією Казахстану (де вже почалися репресії й катування), багато хто вперше
(вперше! навіть Білорусь недосить іще наочною була!!) починає усвідомлювати,
від чого нас урятував Майдан 2013-14 і складати нарешті докупи цілий той пазл,
ради якого я 5 літ тому видавала \enquote{І знов я влізаю в танк...}
( \href{https://komorabooks.com/product/i-znov-ya-vlizayu-v-tank-vybrani-teksty-2012-2016-statti-ese-interv-yu-spogady}{
«І ЗНОВ Я ВЛІЗАЮ В ТАНК...». ВИБРАНІ ТЕКСТИ 2012–2016: СТАТТІ, ЕСЕ, ІНТЕРВ’Ю, СПОГАДИ, komorabooks.com%
}). 

Згадують \enquote{Псков} і
\enquote{Пензу} в Маріїнському, провокаторів з чужим акцентом (і з рос. паспортами,
кого вдавалося відловити!) на самому Майдані, і автобуси спортивних, коротко
стрижених \enquote{паломників} перед Лаврою в кінці січня, і т.д. і т.п. - все те, про
що багато разів писалося, але тільки тепер, коли на прикладі чужої країни стало
видно, \enquote{як це працює} (ну так, лос-анжелеську лекцію Безменова 1983 р., де він
це все діаграмами на дошці малював, у навчальні програми ніхто не вводив!),
українці справді масово охнули - яким же чудом тоді Бозя нас на руках над
прірвою проніс!..

Яке ж це ЩАСТЯ, що нас тоді було так БАГАТО, що їхня \enquote{технологія} об нас
обламалась: забравши, правда, десятки життів, але порятувавши сотні тисяч... Бо
- так, це правда: казахів на протестах, з суто \enquote{гібридно-воєнного} погляду,
було МАЛО - замало, щоб унеможливити \enquote{підміну Майдану Антимайданом} і наступну
\enquote{зачистку} \enquote{запрошеними на підмогу} 70-ма бортами чужоземних військ (це дієво
проти 10-20 тис., але нічого не порадить проти 1-2 млн.!!!) 

Ще раз, і ще раз, не забуваймо: ми перемогли тоді - МАСОЮ, числом. (В кінцевому
підсумку - О. Субтельним, який встиг цілому поколінню школярів 1990-х вкласти в
голови: УНР програла, бо \enquote{Київ сидів удома}, висновок: коли є загроза країні,
НЕ СИДИ ВДОМА!). Це був той безцінний історичний момент, коли країну, дослівно,
врятував особисто КОЖЕН із тих, хто став на захист Майдану, хай не конче
свідомо: кінцева СУМА докладених зусиль врятувала. 

(Ось це й є те \enquote{чудо на Дніпрі}, яке майбутні покоління розглядатимуть крізь
лупу, намагаючись вирахувати загадкове \enquote{число Дельта}, котре - і тільки воно! -
дозволяє час від часу побачити в історії \enquote{народ} - як колективного індивіда...)

Ще раз, "борисфенці" (як звав нас Нострадамус): в 2014-му ми "дали Дельту". І
тільки ми одні її поки що дали - не тільки порівняно з білорусами чи казахами
(що вже всім очевидно), а на всьому просторі "Европы от Лиссабона до
Владивостока" (с). Увімкненим інстинктом колективної небезпеки, так. Бережімо
цей інстинкт: нічого ціннішого в нас наразі немає. Нічого безціннішого й
потрібнішого людству на цьому етапі історії.  

...Я часто згадую того хлопчика з відірваною рукою й вибитим оком, якого
однокурсники привели в лікарню 19.02.14, в розпалі терору, коли там оперували
нашого товариша, підстреленого вночі \enquote{кулею на кабана}. І от що подумалось. У
багатьох із нас є такий \enquote{знайомий знайомих}, кого тоді на Майдані - побили,
покалічили, поранили, хтось оклигав, комусь лишилася пам'ятка на всеньке життя,
війна на Донбасі якось те все заслонила, - і згадуєм ми про них усіх нараз
тільки на річницю розстрілів. А варт було б запровадити - якийсь окремий День
Подяки: не на те, щоб поминати, а щоб сказати \enquote{Дякую!} кожному з тих, перед
чиєю впертістю тоді мусили звалити з Києва (а потім - і з Харкова, і з
Запоріжжя, і з Одеси...) частини військ \enquote{специальных операций ВВС РФ}. Це не
була виграна війна, авжеж. Але це точно була виграна битва - поки що одна така
в цій ох же якій падлючій новій світовій війні...

Так, нам випав у ній передній край. І виборче \enquote{Чорне дзеркало}, розігране в
реалі, і \enquote{шоу-уряди}, виграні \enquote{по телевізору}, - все це теж уперше в \enquote{цифрову
добу} падає саме на наші плечі: ми - фронтир, \enquote{Боже грище} (див. розмову Дарини
з Вадимом у \enquote{Музеї покинутих секретів})... І нема на те ради: така доля, така
місія. А з долею не жартують, тут, як казали наші предки - або пан, або пропав. 

Тож хай наші вороги пропадають. Вище носа, \enquote{борисфенці}. ))

І не розслабляймося. 

\ii{10_01_2022.fb.zabuzhko_oksana.1.chudo_na_dnipri.pic.1}

(Це вулична скульптура з Філадельфії, хто їй домалював прапори - не знаю, взяла
зі стрічки.)
