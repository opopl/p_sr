% vim: keymap=russian-jcukenwin
%%beginhead 
 
%%file 20_11_2017.stz.news.ua.mrpl_city.1.tajemnyci_cvyntar
%%parent 20_11_2017
 
%%url https://mrpl.city/blogs/view/taemnitsi-starovinnogo-mariupolskogo-tsvintarya
 
%%author_id demidko_olga.mariupol,news.ua.mrpl_city
%%date 
 
%%tags 
%%title Таємниці старовинного маріупольського цвинтаря
 
%%endhead 
 
\subsection{Таємниці старовинного маріупольського цвинтаря}
\label{sec:20_11_2017.stz.news.ua.mrpl_city.1.tajemnyci_cvyntar}
 
\Purl{https://mrpl.city/blogs/view/taemnitsi-starovinnogo-mariupolskogo-tsvintarya}
\ifcmt
 author_begin
   author_id demidko_olga.mariupol,news.ua.mrpl_city
 author_end
\fi

Не всім маріупольцям відомо, що в Центральному районі міста (біля Центрального
ринку) розташований старовинний маріупольський цвинтар, який береже в собі
безліч таємниць та важливих відомостей про історичний розвиток Маріуполя. На
жаль, він знаходиться в занепаді, хоча там поховано чимало відомих і видатних
жителів міста.

Старе кладовище Маріуполя заснували в 1832 році для християн (євреїв ховали
окремо). Відомо, що спочатку старі цвинтарні поховання були на колишньому
Магдалінінському спуску (вул. Грецька) і на Олександрівській площі (міський
сквер). Згодом Маріуполь розростався і забудовувався новими будинками та
вулицями. Тому сьогодні найстаріше кладовище опинилося в центрі міста. Останнім
часом все частіше тут можна побачити маріупольців на екскурсіях, які з великим
інтересом відкривають для себе таємниці старовинного маріупольського цвинтаря.

\ii{20_11_2017.stz.news.ua.mrpl_city.1.tajemnyci_cvyntar.pic.1}

Неподалік від центрального входу на цвинтар у 1848 році на кошти відомого
маріупольця, колезького реєстратора Кирила Мат\hyp{}війовича Калері встановили церкву
в ім'я Всіх святих. Знесли її більшовики в 30-х роках минулого століття. Ця
церква була невелика, збудована з каменю, але з дерев'яним куполом. Цікавим є
той факт, що в ній у 1917 р., коли почалися єврейські погроми, настоятель
Михайло Арнаутов допомагав ховатися євреям. Зараз на місці цвинтарної церкви
встановлено пам'ятний хрест на честь жертв голодомору і репресій. 

\ii{20_11_2017.stz.news.ua.mrpl_city.1.tajemnyci_cvyntar.pic.2}

Неподалік розташований монументальний склеп, в якому за однією з версій, було
поховано сімейство купця Пілічова. Сьогодні відвідувачі можуть вільно заходити
всередину склепу, тут давно немає ні даху, ні вікон. На місці цього склепу
неодноразово проводили обряди представники різних субкультур, а місцеві
кінематографісти знімали фільми. За однією з легенд, колись зловмисники
замкнули в склепі доньку охоронця кладовища і вона провела там добу. Старожили
розповідали дві версії: коли дівчину знайшли, вона була вже мертвою, за іншою
версією, вона збожеволіла.

\ii{20_11_2017.stz.news.ua.mrpl_city.1.tajemnyci_cvyntar.pic.3}

Залишився на території кладовища і унікальний склеп сім'ї купця Найдьонова.
Частково він реконструйований, але оригінальними до наших днів збереглися лише
фрагменти. Тут розташовані пам'ятники купця Івана Найдьонова і його дружини
Параски. За легендою, внизу була похована незаміжня дочка Найдьонова. А коли
спробували пограбувати її могилу через кілька днів після поховання, тіло
дівчини нібито розсипалося. Залишилася тільки весільна сукня, в якій її
поховали. Про сім'ю Параски та Івана Найдьонова достовірно відомо небагато.
Вони приїхали до Маріуполя з іншого міста. Є версія, що Іван Іванович Найдьонов
прийшов до Маріуполя босоніж з Курської губернії, де проживала його рідня.
Починав хлопчиком на побігеньках, потім працьовитого хлопця помітили і стали
довіряти все більш відповідальні доручення. Він терпляче збирав гроші, які
перетворювалися спочатку на рублі, а потім і в тисячі рублів. Зібраний капітал
дозволив йому завести власну справу. До речі, саме І. Найдьонов подарував місту
дерев'яну будівлю, відому нам як Гоголівське училище на вул. Земській.

\ii{20_11_2017.stz.news.ua.mrpl_city.1.tajemnyci_cvyntar.pic.4}

Велика частина поховань на цьому кладовищі належить священнослужителям. Серед
них є могила отця Льва, який був похований в 1970 році, але нещодавно на могилі
встановили новий пам'ятник. На відміну від багатьох інших поховань, це має
дуже охайний вигляд. Місцеві жителі вірили, що отець Лев мав здібності до
зцілення, з цієї причини могила стала місцем паломництва.

\ii{20_11_2017.stz.news.ua.mrpl_city.1.tajemnyci_cvyntar.pic.5}

На цвинтарі також можна знайти могилу Олександра Давидовича Хараджаєва –
міського голови Маріуполя (1860 – 1864), купця першої гільдії і мецената.
Сьогодні пам'ятник почесного громадянина Маріуполя на жаль розбитий і лежить на
землі, він потребує термінової реставрації. Отже, статус міського голови на
кладовищі не дає жодних привілеїв.

\ii{20_11_2017.stz.news.ua.mrpl_city.1.tajemnyci_cvyntar.pic.6}

Поруч з могилою О. Хараджаєва знаходиться сімейний склеп Гофів. Це були досить
заможні люди, одна з найбільш багатих родин Маріуполя. Зокрема, Гавриїл
Ісидорович Гоф був власником особняка, де зараз розташована редакція газети
\enquote{Приазовский рабочий} (пр. Миру, 19).

На цьому ж кладовищі є поховання брата засновника першого стаціонарного театру
в Маріуполі Василя Леонтійовича Шаповалова  – Іллі Шаповалова, який був також
актором і режисером-аматором, брав участь у деяких постановках свого видатного
брата-антрепренера. Поховання практично зруйновано – розкидане каміння,
пам'ятник частково втрачений.

Поряд з пам'ятниками XIX століття бачимо могили радянських часів. Зокрема на
території кладовища є поховання героїв Радянського Союзу і братські могили.

\ii{20_11_2017.stz.news.ua.mrpl_city.1.tajemnyci_cvyntar.pic.7}

Втім, більшість поховань зруйновані, повиті рослинами, тому точні відомості про
всіх похованих не збереглися. Члени громадської організації \enquote{Фонд збереження
культурної спадщини Маріуполя} працюють над систематизацією відомостей про
похованих. Роботи по фотофіксації могил і систематизації цих світлин ведуться з
2003 року. Важливо створити електронний архів, щоб маріупольці могли знаходити
місця поховань своїх родичів через інтернет. Проте роботи ще дуже багато, щоб
скласти повну картину, не вистачає деяких відомостей, які містяться в так
званих цвинтарних книгах в архівах Донецьку.

\ii{20_11_2017.stz.news.ua.mrpl_city.1.tajemnyci_cvyntar.pic.8}

Старовинний маріупольський цвинтар площею близько 18 гектарів було закрито в
70-і роки минулого століття. Деякі краєзнавці стверджують, що небіжчиків тут
ховали в кілька шарів. Сьогодні його унікальні історичні могили потребують
негайного дослідження та термінової реставрації, адже він по праву вважається
меморіалом, і з нього можна було б зробити музей просто неба. На жаль місце, що
береже в собі тисячі таємниць і є невід'ємною частиною історії Маріуполя
перетворюється на руїни.

