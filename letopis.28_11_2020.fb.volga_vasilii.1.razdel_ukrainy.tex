% vim: keymap=russian-jcukenwin
%%beginhead 
 
%%file 28_11_2020.fb.volga_vasilii.1.razdel_ukrainy
%%parent 28_11_2020
 
%%url https://www.facebook.com/Vasiliy.volga/posts/2764134797237268
 
%%author Волга, Василий Александрович
%%author_id volga_vasilii
%%author_url 
 
%%tags 
%%title РАЗДЕЛ УКРАИНЫ
 
%%endhead 
 
\subsection{Раздел Украины}
\label{sec:28_11_2020.fb.volga_vasilii.1.razdel_ukrainy}
\Purl{https://www.facebook.com/Vasiliy.volga/posts/2764134797237268}
\ifcmt
	author_begin
   author_id volga_vasilii
	author_end
\fi

Возможно ли? Говорят об этом давно, хотя, конечно, в исторической перспективе
даже и тридцать лет - это всего лишь мгновение, но вот разговоры в очередной
раз усиливаются. Как это было и пару лет назад, этим разговорам дал новый
толчок Владимир Вольфович Жириновский.  Интересен Жириновский не только своим
политическим артистизмом, но часто и точностью формулировок и простотой
изложения. Не всегда, конечно, Владимир Вольфович попадал в десятку со своими
прогнозами, но оглядываясь назад, на всю его политическую карьеру, нельзя не
отметить, что в большинстве случаев он был прав. 

\ifcmt
pic https://scontent.fiev6-1.fna.fbcdn.net/v/t1.0-9/128221022_2764134707237277_5815604567203191449_n.jpg?_nc_cat=111&ccb=2&_nc_sid=730e14&_nc_ohc=K9BsRcJnmp8AX89DrRD&_nc_ht=scontent.fiev6-1.fna&oh=dd6f3712591cdfef49a266d0870d2590&oe=5FEB6FFC
\fi

И вот сегодня Жириновский дал прогноз на 2021-2022 годы. Прогноз о том, что уже
через полгода украинский Армагедон начнется, а через два года он будет закончен
тем, что Германия, как и сто лет назад, заберет себе Западную часть Украины, а
Центральная и Восточная Украина в полном своем составе войдут в обновленную
Российскую Империю. По мнению Владимира Вольфовича начаться все это должно с
военной операции Киева против Донбасса и Крыма, с целью вернуть себе эти
территории. 

В Киеве, конечно же, официальная пропаганда над этим прогнозом хихикает. Из её
нервического смеха становится понятно, что в атаку против Донбасса и Крыма мы
всё-таки пойдем, но о том, чтобы нашу армию мог хоть кто-нибудь разбить, мы не
верим, и даже смеемся над этими глупостями.  Что ж. Время покажет. Но логика
политического процесса с неизбежной необходимостью говорит о том, что
приближается некоторая реперная точка. Как минимум по Донбассу. Особенно
сильно, словно в турбулентном режиме, радикальный украинский политикум
заговорил сегодня о необходимости усиления давления на Россию, и о подготовке к
летней военной операции против ЛДНР. Причиной тому, конечно же, стала короткая
и успешная война Азербайджана против Армении за Нагорный Карабах. 

Ведь ситуация совершенно подобная. И Карабах и ЛДНР никем в мире не признаны.
Так как Армения не признавала Карабах, так и Россия, будучи даже защитницей
Донбасса от этнической агрессии Киева, не признала независимости восставших
республик. Мало того, Россия и её президент регулярно, на протяжении шести
последних лет, публично заявляли о том, что ЛДНР --- это законная территория
Украины. Т.е. ситуация еще более патовая, чем для Карабаха, ибо никто из
армянских президентов никогда не признавал Карабах территорией Азербайджана. 

Таким образом, налицо есть все основания для подготовки мирового общественного
мнения (проводя параллели с Карабахом и с поведением России в этом конфликте),
для начала военной агрессии против Донбасса.

В России это тоже понимают и Донбасс это понимает тоже, но Донбасс еще и устал
выше меры от продолжающейся уже шесть лет войны, блокады, нищеты и
беспросветности. Трудно сказать какой процент, но уж точно не менее трети
оставшегося населения республик, морально уже готовы к возвращению на Украину.
Они боятся, но готовы, рассчитывая на то, что хоть как-то можно прекратить их
непрекращающуюся пытку --- хоть отсечением головы, но прекратить. 

А что же на Украине? Так здесь тоже готовятся. Самым серьезным образом здесь
обсуждается возможность глобального «педагогического» акта по перевоспитанию
всего населения мятежных республик. Разрабатываются различные варианты, но во
всех почти без исключения вариантах присутствует идея лагерей для тех, кто не
сможет сдать первичные тесты на знание истории Украины и на любовь к ней. Те же
жители ЛДНР, кто будет уличен во владении российским паспортом, будут
подвержены уголовному преследованию, так как по украинскому законодательству
такой человек стал гражданином «страны агрессора» во время «войны с агрессором»
и уже тем самым способствовал «агрессии». 

Итак, есть два варианта: один с разделом Украины, другой без, и есть ощущение
реперной точки политического процесса, которая неумолимо становится все ближе.
Один вариант Жириновского, когда Киев не выдержит и начнет военную операцию;
второй вариант - когда не выдержат жители ЛДНР и не выдержит Москва и отдаст
таки мятежные республики Киеву, согласившись на какой-нибудь новый Минский
вариант.

Старая еврейская поговорка говорит о том, что если хочешь рассмешить Бога,
расскажи ему о своих планах. 

Жителей республик, конечно, жаль. Любой вариант для них плох. Они и в самом
деле заложники в пыточной камере. Шесть лет их пытают дыбой и выход у них,
похоже, только через колесование.
