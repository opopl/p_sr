% vim: keymap=russian-jcukenwin
%%beginhead 
 
%%file 26_11_2020.fb.volga_vasilii.1.ioann_zlatoust
%%parent 26_11_2020
 
%%url https://www.facebook.com/Vasiliy.volga/posts/2762476927403055
 
%%author Волга, Василий Александрович
%%author_id volga_vasilii
%%author_url 
 
%%tags 
%%title ИОАНН ЗЛАТОУСТ
 
%%endhead 
 
\subsection{ИОАНН ЗЛАТОУСТ}
\label{sec:26_11_2020.fb.volga_vasilii.1.ioann_zlatoust}
\Purl{https://www.facebook.com/Vasiliy.volga/posts/2762476927403055}
\ifcmt
	author_begin
   author_id volga_vasilii
	author_end
\fi

Удивительная судьба, удивительной силы ум, проповедник и учёный, один из трёх
столпов православного учения о Троице, о Христе, о Его любви к человеку.
Судьба Иоанна Златоуста непроста. Люди, которым он служил, и которые часто пели
ему «Осанна», а так же высшие иерархи церковной Администрации пятого века
расправились с ним, отправив в изгнание, оболгав его и оставив умереть в
одиночестве, нищите, голоде и болезни. 

\ifcmt
pic https://scontent.fiev6-1.fna.fbcdn.net/v/t1.0-9/127714810_2762476877403060_6160826317204973583_n.jpg?_nc_cat=103&cb=846ca55b-ee17756f&ccb=2&_nc_sid=8bfeb9&_nc_ohc=jF2LIfT9pqEAX82id5u&_nc_ht=scontent.fiev6-1.fna&oh=cc435363586898a9f9799cb6a36735db&oe=5FEB6093
\fi

Но этот великий человеколюбец никогда ни единого раза до самой своей смерти не возроптал и за все благодарил Бога.

Сегодня Православная Церковь вспоминает своего учителя, объяснившего ей Бога. 

С Праздником Вас, православные!

А всех Иванов с именинами!

Настоятель нашего сельского храма тоже о. Иоанн.

Отче, поздравляем Вас!)
