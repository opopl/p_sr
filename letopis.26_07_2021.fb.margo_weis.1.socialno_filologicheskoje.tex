% vim: keymap=russian-jcukenwin
%%beginhead 
 
%%file 26_07_2021.fb.margo_weis.1.socialno_filologicheskoje
%%parent 26_07_2021
 
%%url https://www.facebook.com/margo.weis.1/posts/336032788195245
 
%%author Margo Weis
%%author_id margo_weis
%%author_url 
 
%%tags kultura,mova,nauka,ukrainizacia
%%title Социально-филологическое
 
%%endhead 
 
\subsection{Социально-филологическое}
\label{sec:26_07_2021.fb.margo_weis.1.socialno_filologicheskoje}
 
\Purl{https://www.facebook.com/margo.weis.1/posts/336032788195245}
\ifcmt
 author_begin
   author_id margo_weis
 author_end
\fi

Социально-филологическое.

Поскольку в сети все больше постов и срачей на тему навязывания мовы, хотелось
бы публично пояснить свою позицию по этому вопросу. 

Сам по себе украинский - вполне себе язык как язык. Вот только этого языка нет
в большой науке, большой культуре, большом бизнесе и большой политике. И вместо
того, чтобы делать что-то для того, чтобы украинский язык там был, а контент на
украинском языке был интересен и востребован, ситуацию пустили совершенно по
противоположному пути. Наука убита, экономика развалена, образование опущено
ниже плинтуса, крупный бизнес на известном месте крутил щирых предпринимателей,
а что происходит в политике - и так все видят. Вот украинский язык и
превратился в мову, которую иначе, как насильно, в рот и в уши не вложишь.

Могу только искренне посочувствовать адекватным украиноязычным и билингвам.
Напоминаю, что у меня на украинском было выпущено три книги, а разбуди меня и
попроси рассказать стихотворение - я прочту стихов пять на украинском, три на
испанском, а потом вспомню что-то на русском из Лермонтова, Пушкина и
Маяковского. Но для меня вопрос решен: я не потребляю контент на мове и общаюсь
на украинском исключительно с теми, кто не владеет русским (таких много среди
диаспорян, и это совершенно нормально).

\ii{26_07_2021.fb.margo_weis.1.socialno_filologicheskoje.cmt}
