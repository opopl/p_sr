% vim: keymap=russian-jcukenwin
%%beginhead 
 
%%file 24_08_2021.fb.chernjak_evgenij.1.tochno_budet_horosho_ukraina
%%parent 24_08_2021
 
%%url https://www.facebook.com/evgenyjchernyak/posts/4212634652139566
 
%%author_id chernjak_evgenij
%%date 
 
%%tags nezalezhnist,ukraina
%%title Точно будет хорошо, Украина. Сегодня тебе 30
 
%%endhead 
 
\subsection{Точно будет хорошо, Украина. Сегодня тебе 30}
\label{sec:24_08_2021.fb.chernjak_evgenij.1.tochno_budet_horosho_ukraina}
 
\Purl{https://www.facebook.com/evgenyjchernyak/posts/4212634652139566}
\ifcmt
 author_begin
   author_id chernjak_evgenij
 author_end
\fi

\obeycr
Обнять, прижать к себе крепко, сказать извини и попросить немного подождать, всё изменится и будет хорошо. 
Точно будет хорошо, Украина. 
Сегодня тебе 30. 
Пока что большие флаги, вместо больших дел.
И чем меньше независимости, тем громче крики о ней. 
И вышиванка, как бронежилет для чиновника что ворует. 
И по каждому вопросу ровно пополам, всегда. 
И вопросы, бесконечные вопросы , кто на каком языке любит Родину. 
И на каком любить правильно. 
И каждый раз, каждый день, мы правильные , а вы нет, не так выглядите, не так говорите, потому, что на русском и молчите тоже не так. 
Любить Украину каждый день, а не только в День Независимости. 
Флаги - когда есть достижения, спортивные, научные, вершина Эвереста, Олимпийска медаль, математическая олимпиада мировая. 
Любить страну не словами, а делами. 
Смотреть кто сколько налогов заплатил и потом слушать, как Родину любит. 
Подьезд свой убрать, двор чистым сделать, помочь тем, кто рядом и нуждается в помощи, людям своим платить достойную зарплату и создать условия что бы не уезжали. 
С соседями не ругаться, землю свою беречь, а не рапсом засевать, обочины убрать, дороги построить. 
Дать этой команде что-то сделать, а если нет спросить жёстко, но не вообще, а каждого, персонально. 
Остановить, взять за руку и спросить глядя в глаза, как выполнил то, что обещал. 
Коррупционеру руки не подавать, а не пытаться договориться. 
Быть внимательным к тем, кто кричит, понимая что борется он не с воровством, а с тем, что ворует не он. 
Прошлых спросить по всей строгости, за всё, что натворили. 
Не просто спросить, а справедливо. Чтобы видно было. И слышно. 
Помнить настоящих, Макса Левчина, Яна Кума, Уэйна Гретцки и Сикорского. 
И помнить что великими они стали за пределами Украины, и сделать выводы , почему так. 
И от профессиональных патриотов спасти Украину, залюбят же до смерти, им же больше заниматься нечем, задач других нет, только любить страну с пользой для себя. 
Помнить, что любить свою Страну- это не значит ненавидеть другую. 
Это значит когда Украина в сердце, а не на калькуляторе для подсчета голосов на выборах, потом денег из бюджета. 
Как человек живущий за пределами, точно знаю, в самые тяжёлые моменты вспоминаешь свою Родину, вспоминаешь школу, где в лужах отражались звёзды, свой город Запорожье, где родился , Киев, Одессу и реку Днепр и Азовское Море, родное и тёплое, всегда, даже когда рядом огромный Океан, скучаешь за Азовским Морем. 
Родное, потому что. 
Сумбурно немного и не очень празднично. 
Праздник есть, а поводов праздновать пока немного. 
Но всё изменится. 
Всё будет хорошо. 
Я точно знаю. 
С Днём Рождения, Украина!
\restorecr
