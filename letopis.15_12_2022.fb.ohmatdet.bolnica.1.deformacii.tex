% vim: keymap=russian-jcukenwin
%%beginhead 
 
%%file 15_12_2022.fb.ohmatdet.bolnica.1.deformacii
%%parent 15_12_2022
 
%%url https://www.facebook.com/ndslohmatdyt/posts/pfbid0ETowfWmA46zGazJnrBiujuQvMpnaZ6WW2i3EmMMMYw8qfnSbEpqrjcDDJWf4MEEdl
 
%%author_id ohmatdet.bolnica
%%date 
 
%%tags 
%%title Ми виправляємо не деформації, ми виправляємо людські долі
 
%%endhead 
 
\subsection{Ми виправляємо не деформації, ми виправляємо людські долі}
\label{sec:15_12_2022.fb.ohmatdet.bolnica.1.deformacii}
 
\Purl{https://www.facebook.com/ndslohmatdyt/posts/pfbid0ETowfWmA46zGazJnrBiujuQvMpnaZ6WW2i3EmMMMYw8qfnSbEpqrjcDDJWf4MEEdl}
\ifcmt
 author_begin
   author_id ohmatdet.bolnica
 author_end
\fi


«Ми виправляємо не деформації, ми виправляємо людські долі»: спеціалісти
Охматдиту рятують дітей зі складними сколіозами⚡️

🔻Щоденно в умовах відділення ортопедії та травматології НДСЛ “Охматдит”
проводять складні хірургічні втручання щодо корекції складних деформацій та вад
розвитку опорно-рухового апарату на високому європейському рівні.
Використовується найсучасніше медичне обладнання, методики та апаратура.

🔻Про один з останніх кейсів хочемо розказати детальніше. S-подібна сколіотична
деформація хребта була виявлена лікарями Охматдиту в пацієнтки 11 років,
дівчинку звати Вікторія.

Про подробиці та перебіг лікування розповів Левицький Анатолій Феодосійович —
професор, лікар ортопед-травматолог НДСЛ «Охматдит», завідувач кафедри дитячої
хірургії НМУ імені О.О. Богомольця:

«Для віку 11 років викривлення грудного відділу 105 градусів та поперекового
відділу 98 градусів – це занадто багато. Враховуючи те, що захворювання
прогресуюче, страшно уявити, до якого рівня могла розвинутися деформація, якби
ми вчасно не допомогли. 

У цієї дитини ідіопатичний сколіоз. Складність полягає в тому, що конкретна
причина виникнення всіх таких випадків викривлень хребта невідома. На ранніх
етапах деформацій, можна носити корсети, проте дана ситуація виявилась більш
складною. Разом з тим дитина пережила ще й втручання на серці, бо мала вади
серця.

% 1,2,3
\ii{15_12_2022.fb.ohmatdet.bolnica.1.deformacii.pic.1}

Ми виконали багатоетапне хірургічне втручання. Перший етап складався з
накладання Hallo-гравітаційного витяжіння. Це тривало 2,5 місяці. За цей час ми
зменшили деформацію, аби мінімізувати крововтрату та неврологічні ризики.
Другий етап – це задня коригуюча інструментація хребта полісегментарною
конструкцією NovaSpine». 

% 4,5,6
\ii{15_12_2022.fb.ohmatdet.bolnica.1.deformacii.pic.2}

🔻Остання операція тривала 6 годин.

Тепер на дівчинку чекає процес відновлення рухової активності. Це приблизно 2-3
місяці. Проте, наші спеціалісти активізовують пацієнтів вже на наступний день
після операції. У перший день забирають дітей з інтенсивної терапії та
поступово ставлять на ноги. 

% 7,8,9
\ii{15_12_2022.fb.ohmatdet.bolnica.1.deformacii.pic.3}

🔻Можна спостерігати як змінюється зріст таких дітей. Наприклад Вікторія, наша
пацієнтка, після операції виросла на 5-6 см. Здебільшого діти починають ходити
самостійно після 4 дня реабілітації – тоді, коли прибирають дренажі, зонди,
знеболювальні. Саме після цього пацієнти стають більш активними.

% 10,11,12
\ii{15_12_2022.fb.ohmatdet.bolnica.1.deformacii.pic.4}

% 13,14,15,16
\ii{15_12_2022.fb.ohmatdet.bolnica.1.deformacii.pic.5}

% 17,18,19
\ii{15_12_2022.fb.ohmatdet.bolnica.1.deformacii.pic.6}

% 20,21,22
\ii{15_12_2022.fb.ohmatdet.bolnica.1.deformacii.pic.7}

Команда спеціалістів:

🔹Хірурги\par
- професор Левицький Анатолій Феодосійович\par
- Доляницький Микола Михайлович\par
🔹Асистенти\par
- Качковський Дмитро Любомирович\par
🔹Анестезіолог:\par
- Копил Лариса Георгіївна\par
🔹Нейрофізіолог:\par
- Курисько Кіра Сергіївна\par
🔹Операційна медична сестра:\par
- Бабчук Любов Романівна\par
🔹Анестезистка:\par
- Леус Соня Сергіївна \par

Ми дякуємо нашим медикам за безперервну роботу та допомогу дітям, навіть в умовах війни!🇺🇦

📸більше фото за посиланням: 

\url{https://photos.app.goo.gl/RgUdSReh52GhECzH8}
