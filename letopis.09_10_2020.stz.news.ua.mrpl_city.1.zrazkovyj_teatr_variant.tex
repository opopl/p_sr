% vim: keymap=russian-jcukenwin
%%beginhead 
 
%%file 09_10_2020.stz.news.ua.mrpl_city.1.zrazkovyj_teatr_variant
%%parent 09_10_2020
 
%%url https://mrpl.city/blogs/view/zrazkovij-teatr-variant
 
%%author_id demidko_olga.mariupol,news.ua.mrpl_city
%%date 
 
%%tags 
%%title Зразковий театр "Варіант"
 
%%endhead 
 
\subsection{Зразковий театр \enquote{Варіант}}
\label{sec:09_10_2020.stz.news.ua.mrpl_city.1.zrazkovyj_teatr_variant}
 
\Purl{https://mrpl.city/blogs/view/zrazkovij-teatr-variant}
\ifcmt
 author_begin
   author_id demidko_olga.mariupol,news.ua.mrpl_city
 author_end
\fi

\ii{09_10_2020.stz.news.ua.mrpl_city.1.zrazkovyj_teatr_variant.pic.1}

10 жовтня в Маріуполі розпочнеться Міський фестиваль театральних колективів
\enquote{Маріуполь театральний – 2020}. Я вирішила ближче познайомитися з колективами,
які стали учасниками цього унікального заходу. Почну з театру, який
відкриватиме фестиваль – це Зразковий театр \enquote{Варіант}, що вже завтра
представить маріупольцям свою виставу - казку \enquote{Жадібний гном}.

Зразковий театр \enquote{Варіант} виріс із зразкового театру \enquote{Ребята}, тому датою його
заснування можна вважати 1978 рік. Колектив дитячого театру \enquote{Ребята} об'єднав
навколо себе дітлахів, захоплених театральним мистецтвом. З роками це
захоплення настільки міцно увійшло в життя учасників колективу, що багато хто з
них, закінчивши школу, вирішив продовжити навчання у середніх та вищих
навчальних закладах культури. Дорослішаючи, вони також не поспішали розлучатися
з улюбленим колективом. Тому в \emph{1993 році} і був створений молодіжний театр
\enquote{Варіант}. Засновницею театрального колективу стала \emph{\textbf{Валентина Михайлівна
Кірєєва}}.

Усі ці роки, театральним колективом \enquote{Ребята}, в склад якого тоді і входив
колектив \enquote{Варіанта}, накопичувався великий і різноманітний репертуар. Були
поставлені вистави за п'єсами: М. Краль \enquote{Зачарована шкатулка}, Яна Уличанського
\enquote{Тік-так}, І. Медведєва \enquote{Чарівне зернятко}, \enquote{Посмішка Регіни}, Э. Шварца
\enquote{Снігова королева}, \enquote{Новий костюм короля}, \enquote{Попелюшка}, І. Токмакова \enquote{Морозко},
\enquote{Зачароване копитце}, Н. Шейко-Медведєва \enquote{Курочка Ряба}, Г. Андерсена
\enquote{Оле-Лукойє}, О. Проймер \enquote{Маленька баба-Яга}, А. Курляндского, А. Хайта \enquote{Ну, вовк,
постривай}, В. Зіміна \enquote{Жили-були дві лисички}, \enquote{Історія кота Філофея}, О.Афоніна
\enquote{Мимренок} тощо.

\ii{09_10_2020.stz.news.ua.mrpl_city.1.zrazkovyj_teatr_variant.pic.2}

Прем'єрою вистави І. Елагіна \enquote{Портрет мадмуазель Таржи} у 1993 році заявив про
себе, як про самостійний творчий колектив, молодіжний театр \enquote{Варіант}.

Задавши потрібний темп, театр \enquote{Варіант} впевнено і успішно рухався вперед,
розширюючи свій репертуар, завойовуючи нагороди. Так наприклад, за постановку
вистави \enquote{Портрет мадмуазель Таржи} колектив удостоєний звання Лауреата
обласного огляду-конкурсу юнацьких театральних колективів (м. Донецьк). У 1999
році в місті Євпаторія на міжнародному фестивалі аматорських театрів \enquote{Рампа
дружби – 99} колектив театру показує спектакль М. Павліковської \enquote{Шофер
Арчибальд}. Журі фестивалю назвало жіночий склад \enquote{Варіанта} найсильнішим, а
Євгенію Нестеренко (яка зіграла головну роль) нагородили дипломом першого
ступеня за найкращу жіночу роль.

З огляду на музичність постановок, колектив театру \enquote{Варіант} виступає як із
самостійними виставами, так і окремими номерами, бере участь в концертах, а
також готує самостійні концертні програми.

\ii{09_10_2020.stz.news.ua.mrpl_city.1.zrazkovyj_teatr_variant.pic.3}

15 березня 2001 року за досягнення у розвитку самодіяльної художньої творчості,
високий рівень виконавської майстерності та репертуару, активну роботу щодо
культурного обслуговування та організації змістовного дозвілля населення
молодіжному театру \enquote{Варіант} постановою Президії Ради Федерації професійних
спілок України за № ІІ-33-5г було присвоєне почесне звання \emph{\enquote{Народний
самодіяльний колектив профспілок України}}.

З січня 2017 року змінилася форма власності базового Палацу культури. Палац
став підрозділом департаменту культурно-громад\hyp{}ського розвитку Маріупольської
міської ради. І творчі колективи Палацу, зокрема і театр \enquote{Варіант}, знову
захищали свої звання. Рішенням колегії управління культури і туризму Донецької
облдержадміністрації від 19 грудня 2017 року театру \enquote{Варіант} було присвоєно
звання \emph{\enquote{Зразковий аматорський колектив}}.

А з серпня 2018 року театр очолює \emph{\textbf{Олександр Геннадійович Романюк}}, який об'єднав
театр зі \emph{Студією естрадних мініатюр \enquote{МарАКУйя}}. Раніше основна діяльність
Олександра Геннадійович була присвячена гумору та сміху. На думку художнього
керівника театру, колектив унікальний тим, що крім однієї-двох вистав в рік,
звітного спектаклю та новорічних казок діти ще чотири рази додатково виходять
на сцену, де змагаються у власній дотепності та винахідливості. Це дійсно
відрізняє колектив від інших театрів, де актори вчать тільки тексти та
намагаються перевтілитися в свого персонажа. Актори \enquote{Варіанта} ще й додають
свою творчість. Театр бере активну учать в змаганнях за сезонні кубки, тому
важливу роль грають тренери які готують дітей. Однією з найулюбленіших вистав
для багатьох юних театралів сьогодні стала \emph{\enquote{Експонат руками не чіпати}} –
сучасна казка, де всі персонажі оживають і отримують друге життя.

\ii{09_10_2020.stz.news.ua.mrpl_city.1.zrazkovyj_teatr_variant.pic.4}

Вихованці театру навчаються в Київському інституті культури, Харківській
академії культури на акторських та режисерських факультетах. Олександр
Геннадійович готував і Андрія Біднякова, який у 19-річному віці теж змагався за
кубок. Сьогодні Бідняков продовжує підтримувати колектив. Зокрема, український
актор і телеведучий профінансував нещодавню участь театру в
культурно-мистецькому фестивалі для обдарованих дітей та молоді \enquote{Жар птиця}, де
юні актори посіли 1 місце. Наразі колектив театру \enquote{Варіант} є змішаним у
віковій категорії, в ньому займається 46 осіб віком від 6 до 15 років. Юні
актори вже не уявляють свого життя без театру і вже знають безліч скоромовок та
методик, які допомагають їм вдосконалювати акторську майстерність.

Читайте також блог автора про \href{https://archive.org/details/25_08_2020.olga_demidko.mrpl_city.teatr_studia_feniks}{театр-студію \enquote{Фенікс}}.%
\footnote{Театр-студія \enquote{Фенікс}, Ольга Демідко, mrpl.city, 25.08.2020, \par%
\url{https://mrpl.city/blogs/view/teatr-studiya-feniks}, \par%
Internet Archive: \url{https://archive.org/details/25_08_2020.olga_demidko.mrpl_city.teatr_studia_feniks}
}
