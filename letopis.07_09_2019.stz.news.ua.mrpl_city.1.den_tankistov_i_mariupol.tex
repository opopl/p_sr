% vim: keymap=russian-jcukenwin
%%beginhead 
 
%%file 07_09_2019.stz.news.ua.mrpl_city.1.den_tankistov_i_mariupol
%%parent 07_09_2019
 
%%url https://mrpl.city/blogs/view/den-tankistov-i-mariupol
 
%%author_id burov_sergij.mariupol,news.ua.mrpl_city
%%date 
 
%%tags 
%%title День танкистов и Мариуполь
 
%%endhead 
 
\subsection{День танкистов и Мариуполь}
\label{sec:07_09_2019.stz.news.ua.mrpl_city.1.den_tankistov_i_mariupol}
 
\Purl{https://mrpl.city/blogs/view/den-tankistov-i-mariupol}
\ifcmt
 author_begin
   author_id burov_sergij.mariupol,news.ua.mrpl_city
 author_end
\fi

\ii{07_09_2019.stz.news.ua.mrpl_city.1.den_tankistov_i_mariupol.pic.1}

В этом году День танкистов отмечается 8 сентября. На первый взгляд, Мариуполь к
этому празднику не имеет отношения. Да, есть памятный знак
\textbf{\enquote{Танк-34}}.  Наверное, в нашем городе живут или жили люди,
которые служили в танковых частях. Так это в каждом городе или селе найдутся
бывшие танкисты. На самом деле наш город имеет непосредственное отношение если
не к танковым частям, то к танкостроению безусловно. Опыт производства броневых
сталей для военных кораблей на заводе \enquote{Никополь-Мариупольского
общества} был приобретен еще до Первой мировой войны. Не случайно в 30-е годы
прошлого столетия именно на заводе им. Ильича приступили к работам по
производству стальных листов для танков.

К началу 40-х была разработана технология выплавки, прокатки и термической
обработки металла для танкостроения. Испытания его на полигоне показали, что
ильичевская броня превосходит по бронестойкости известные тогда отечественные и
зарубежные образцы. Новые марки проката пошли в массовое производство. Началась
война. Когда угроза захвата Мариуполя стала очевидной, необходимое оборудование
и специалисты были эвакуированы на Урал. Там продолжалось производство металла
для танков. Вскоре после освобождения нашего города от немецко-фашистских
оккупантов необходимое оборудование было возвращено в цеха завода им. Ильича и
смонтировано вновь. На заводе им. Ильича из своего же листового металла
сваривали корпуса танков. Кроме того, впервые в истории танкостроения была
разработана конструкция литой башни, а также уникальная технология ее
изготовления.

\vspace{0.5cm}
\begin{minipage}{0.9\textwidth}
\textbf{Читайте также:}

\href{https://archive.org/details/19_01_2019.sergij_burov.mrpl_city.istoria_mariupolskij_zhestjanschik}{%
История: мариупольский жестянщик, Сергей Буров, mrpl.city, 19.01.2019}
\end{minipage}
\vspace{0.5cm}

Несколько интересных фактов. Уже в первые дни войны немецкие инженеры отобрали
пробы металла от корпусов подбитых советских танков Т-34. На заводах Круппа
выплавили металл с химическим составом, идентичным отобранным образцам. На
испытаниях на полигоне оказалось, что их бронестойкость низкая. Через
много-много лет довелось услышать от специалистов, что свойства ильичевской
брони обеспечивались не только ее химическим составом, но и оригинальной
технологией термической обработки. В 1944 году были возобновлены занятия в
Мариупольском металлургическом институте. При нем была открыта военная кафедра,
на которой учились молодые люди, в том числе и девушки. Кафедра готовила
офицеров – командиров взводов средних танков, т. е. Т-34. В свое время
принадлежность завода им. Ильича к наркомату танковой промышленности была
строго засекречена. 12 марта 1950 года директор завода им. Ильича Николай
Георгиевич Гавриленко был избран депутатом Верховного Совета СССР, в газетах же
напечатали - директор Ждановского... машиностроительного завода.

Всех мариупольцев, внесших вклад в развитие отечественного танкостроения, не
перечислить. Но есть люди, которых не вспомнить нельзя. \textbf{Константин Федорович
Челпан}, под руководством которого и при его личном участии был сконструирован и
изготовлен алюминиевый танковый дизель В-2 для танка Т-34. Это техническое
решение позволило существенно улучшить тактико-технические свойства боевой
машины. К сожалению, талантливый инженер был расстрелян в 1937 году по ложному
обвинению. В 1919 году он окончил Мариупольское реальное училище. \textbf{Николай
Владимирович Шмидт}, будучи молодым специалистом, в 1936 году приехал в
Мариуполь. С той поры он посвятил сорок шесть лет своей жизни сталям для
военной техники. Эти годы отданы созданию и совершенствованию брони для танка
Т-34 и других боевых машин. Инженер-литейщик \textbf{Иван Иванович Брагин}. При его
непосредственном участии  впервые была освоена серийная технология отливки
цельнолитых башен, которые до этого делались сварными. Естественно, сварная
башня была менее прочной, чем литая.

Так что День танкистов для мариупольцев праздник нерядовой!

\textbf{Читайте также:} 

\href{https://mrpl.city/news/view/stop-miny-so-dna-reki-pod-mariupolem-podnyali-snaryady-foto-plusvideo-1}{%
\enquote{Стоп - мины!}: со дна реки под Мариуполем подняли снаряды, Олена Онєгіна, mrpl.city, 06.09.2019}
