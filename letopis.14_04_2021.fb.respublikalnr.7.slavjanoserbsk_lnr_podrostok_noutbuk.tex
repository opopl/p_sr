% vim: keymap=russian-jcukenwin
%%beginhead 
 
%%file 14_04_2021.fb.respublikalnr.7.slavjanoserbsk_lnr_podrostok_noutbuk
%%parent 14_04_2021
 
%%url https://www.facebook.com/groups/respublikalnr/permalink/796324531003340/
 
%%author 
%%author_id 
%%author_url 
 
%%tags 
%%title 
 
%%endhead 

\subsection{НОВОСТИ ТЕРРИТОРИЙ ЛНР – СЛАВЯНОСЕРБСК. Председатель Народного Совета ЛНР подарил ноутбук пострадавшему от взрыва подростку}
\Purl{https://www.facebook.com/groups/respublikalnr/permalink/796324531003340/}

Председатель Народного Совета ЛНР, руководитель республиканского исполкома
общественного движения (ОД) \enquote{Мир Луганщине} навестил пострадавшего от взрыва в
Славяносербском районе подростка. 


\ifcmt
  pic https://scontent-bos3-1.xx.fbcdn.net/v/t1.6435-9/173661841_122250523288389_5820444396203735790_n.jpg?_nc_cat=105&ccb=1-3&_nc_sid=825194&_nc_ohc=oB12dM7teQcAX9dtmFy&_nc_ht=scontent-bos3-1.xx&oh=e837fbb20d0085e530702d0d30279a92&oe=609C762F
\fi


В ходе визита Денис Мирошниченко пообщался с Эдуардом и его семьей, подарил
ноутбук и передал привет от главы ЛНР Леонида Пасечника. Эдуард поблагодарил за
подарок и внимание, и рассказал о случившемся.

\enquote{Друг пригласил на прогулку. Мы пошли вдоль реки, затем по тропинке пошли и
увидели там какое-то мертвое тело. Друг подбежал и увидел на земле тело кабана.
Я начал опасаться и отговаривать друга идти дальше, но он решил идти. Затем
последовали взрывы, я успел отбежать, а он – нет. Царство ему небесное! Очень
жаль, что так вышло…}, – сказал подросток.

Эдуарда выписали из больницы через пять дней после происшествия, он постепенно
восстанавливается и продолжает обучение в техникуме. Посттравматический синдром
будет напоминать о себе ещё долго. На сегодняшний день мальчик испытывает
беспокойство, даже переступая с бордюра на тротуар, внимательно всматриваясь
под ноги. 

\enquote{У нас в семье три ребенка, мы им объясняли, что нельзя никуда ходить.
Рассказывали, что недавно был обстрел. Но вот так случилось, что не послушался
младший сын. В тот день мы очень сильно испугались}, – рассказала мама
подростка Елена и призвала всех жителей Донбасса не посещать небезопасные
места, а при обнаружении взрывоопасных предметов не приближаться и не
прикасаться к ним.

Напомним, 6 апреля 17-летний подросток погиб, еще один, 16-летний, получил
ранения при подрыве на неустановленном взрывном устройстве на окраине поселка
городского типа Славяносербск. По данному факту главное следственное управление
Следственного комитета России возбудило уголовное дело.
