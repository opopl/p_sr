%%beginhead 
 
%%file 13_03_2023.fb.krutenko_maryna.mariupol.1.18_den_13_03
%%parent 13_03_2023
 
%%url https://www.facebook.com/marinakrytenko/posts/pfbid0bwRAziX9SsaCg9MosMNa9LqWGYKh8xeLCr7DumoPxaMSR77kCAB2ZeMU4v1WRfMxl
 
%%author_id krutenko_maryna.mariupol
%%date 13_03_2023
 
%%tags mariupol.war,mariupol,dnevnik,13.03.2022
%%title 13.03.22 ВОСЕМНАДЦАТЫЙ ДЕНЬ ВОЙНЫ
 
%%endhead 

\subsection{13.03.22 ВОСЕМНАДЦАТЫЙ ДЕНЬ ВОЙНЫ}
\label{sec:13_03_2023.fb.krutenko_maryna.mariupol.1.18_den_13_03}

\Purl{https://www.facebook.com/marinakrytenko/posts/pfbid0bwRAziX9SsaCg9MosMNa9LqWGYKh8xeLCr7DumoPxaMSR77kCAB2ZeMU4v1WRfMxl}
\ifcmt
 author_begin
   author_id krutenko_maryna.mariupol
 author_end
\fi

13.03.22 ВОСЕМНАДЦАТЫЙ ДЕНЬ ВОЙНЫ

Всю ночь  слышали взрывы, утром узнали, что взорвали мост рядом с заводом
\enquote{Азовсталь} (сама не видела слухи ходили). Тем временем практически весь левый
берег был по оккупацией.  Мост соединял левый берег и город. Теперь все
желающие попасть в город должны были объезжать.....

Мужчины побежали домой за оставшимися продуктами и вещами.

С утра мы с Олей собирались идти искать воду. Нам сказали, что идти нужно либо
на источник, либо в сторону городского водоканала. Шёл бой, нам страшно было
куда-то идти. Ближе к обеду мы вышли на улицу, а нам во двор привезли цистерну
воды. Воду давали по 12 литров в одни руки. Мы позвали детей. Народ все так же
зверел и наглел, чуть ли не по головам шёл. Никита начал ругаться с взрослым
мужчиной, который лез без очереди, тот хотел полезть в драку..... мы еле
успокоили их. Первые набрали воду дети, мы приказали им, нас не ждать и бежать
домой. Сложно даже представить, чтоб в эту толпу прилетел снаряд..... Мы
периодически слышали, что снаряд прилетел в очередь под магазин, в то место где
люди готовили еду, в источник воды, где люди набирали воду. 

Рядом с нашим домом, один мужчина вынес генератор на улицу, заправил бензином и
разрешил всем желающим заряжать свои телефоны БЕСПЛАТНО (деньги уже ничего не
значили).

Люди выносили переноски, чтоб только подсоединить телефоны, найти связь и
позвонить родным. Люди стояли с молитвословами, с Евангелием и Библией, даже
ярые атеисты боялись умереть и предстать перед Творцом! 

В эти дни, для меня люди поделились на две категории, я уже об этом говорила.
Были люди которые взяли на себя ответственность, помогать людям: под обстрелами
привозили вод;у, выносили генераторы, тратили свой бензин; отдавали свои вещи и
продукты, делились с друг-другом. И были те, кто воровал то, что плохо лежит,
кто лез по трупам, чтоб как-то обогатится.....

На обед Женя с Сергеем подогревали пищу на костре, не далеко от них упал снаряд
от миномета. Взрывной волной поломало парня 22 лет, (третий раз когда смерть
дышала в спину).... Парень нёс воду из источника....

Вечером к Жени подошла женщина и попросила отвезти ее и ребёнка инвалида в
бомбоубежище. Я боялась, что нам не хватит бензина на эвакуацию, Женя ответил,
что если нас убьют или же снаряд попадёт в машину, мы вообще никуда не уедем.
Решил, что нужно помочь. У нас оставалось пол бака топлива. 

Вечером мы молились и благодарили Бога что днём выжили, а утром молились и
благодарили, что ночью выжили...

Немного о Ридике!

Ридика🐕 постоянно трясло, толи от холода, толи от страха. Мы нашли детский
комбинезон, ножницами отрезали ткань на том месте, что должно быть \enquote{на улице}.
Его поведение очень изменилось. Он вообще-то не очень дружелюбный пёс, на чужих
гавкал, с собаками агрегировал. Во время войны, он понял, что не стоит в жизни
напрасно тратить нервы. На людей и собак он больше не кидался, не время
выяснять отношения, на дворе война!

Продолжение следует....

%\ii{13_03_2023.fb.krutenko_maryna.mariupol.1.18_den_13_03.cmt}
