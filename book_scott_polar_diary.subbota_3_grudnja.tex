% vim: keymap=russian-jcukenwin
%%beginhead 
 
%%file subbota_3_grudnja
%%parent body
 
%%url https://www.facebook.com/ScottPolarDiary/posts/102843441672472
 
%%author 
%%author_id 
%%author_url 
 
%%tags 
%%title 
 
%%endhead 
\section{Субота, 3 грудня}
\Purl{https://www.facebook.com/ScottPolarDiary/posts/102843441672472}

Вчора до вечора вітер поступово вщух; на борт стало потрапляти менше води, а
отже і менше її протікало додолу, і невдовзі ми дісталися до машинного
відділення. Робота тривала безупинно двогодинними змінами. На десяту вечора
отвір у перегородці машинного відділення було завершено, і лейтенант Еванс,
звиваючись над вугіллям, проліз до шахти помпи і спустився нею вниз. Скоро він
прочистив всисну трубу \enquote{від вугільних кульок (суміші вугілля та
мастила), які її засмітили}, та на радість усім вперше з помпи вийшов гарний
струм води. З цієї миті стало ясно, що цю перешкоду ми здолаємо, і, хоча помпа
ще кілька разів забивалася, все ж вода з машинного відділення поступово
відходила.  Приємно було сьогодні вранці побачити, що вода вже не розтікається
там взад-вперед. По обіді розпалили топки: ручна помпа працювала в повну міць і
виссала воду з трюмів майже досуха, тож можна було набрати вдосталь вугілля і
попелу.

Тепер усе знову в порядку, ми впевнено йдемо на південь на парі й під
вітрилами, не більш як на два румби відхилившись від курсу. Кемпбелл і Боверс
зайняті переписом припасів на верхній палубі. Сьогодні вдень витягнули крізь
люк бака двох мертвих поні. Цікава видалась процедура, бо місця в проході було
замало. Зазирнули у льодовню, там усе в найліпшому стані.

Хоча ми ще й не в безпеці, адже ще одна буря може мати катастрофічні наслідки,
та все ж чудово усвідомлювати, як різко змінилося наше становище за останні
двадцять чотири години. Деякі члени команди зізналися, що вчора, як і я, мали
великі сумніви щодо нашої затії, але зараз ми всі знову сповнені надій.

Наскільки можна полічити збитки, окрім пошкоджень у фальшбортах, ми втратили
також двох поні, одного собаку, \enquote{10 тонн вугілля}, 65 галонів бензину і скриню
зі спиртом для біологічних дослідів – серйозна втрата, але значно менша, ніж я
очікував. \enquote{Зважуючи всі обставини, ми відбулися легко, та все ж страшенно не
поталанило вскочити в бурю у такий час}. Третій поні, який під час шторму був
на перев’язі, знову звівся на ноги. Вигляд у нього непевний, але, якщо не
налетить ще одна буря, має витримати. Осман – наш найкращий їздовий собака –
зранку мав жахливий вигляд, але день відлежався у сіні й теплі, і тепер йому
вже значно краще. \enquote{Іще декілька собак були в жалюгідному стані і потребували
піклування, аби вернути життєві сили}. Море й вітер знову починають
хвилюватися, а з південного боку маємо сильні брижі, але стовпчик барометра
високо, поки він не впаде, нової бурі не має бути.

