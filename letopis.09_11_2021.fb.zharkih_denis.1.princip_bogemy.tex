% vim: keymap=russian-jcukenwin
%%beginhead 
 
%%file 09_11_2021.fb.zharkih_denis.1.princip_bogemy
%%parent 09_11_2021
 
%%url https://www.facebook.com/permalink.php?story_fbid=3118899818323414&id=100006102787780
 
%%author_id zharkih_denis
%%date 
 
%%tags obschestvo,ukraina
%%title Принцип богемы
 
%%endhead 
 
\subsection{Принцип богемы}
\label{sec:09_11_2021.fb.zharkih_denis.1.princip_bogemy}
 
\Purl{https://www.facebook.com/permalink.php?story_fbid=3118899818323414&id=100006102787780}
\ifcmt
 author_begin
   author_id zharkih_denis
 author_end
\fi

Принцип богемы

Недавно ходил на лекцию Александра Роднянского у нас в Киеве. Тут в ФБ была
реклама, было безумно интересно узнать, что делает он тут, и что хочет сказать.
Его мне доводилось слышать ранее, интересно, что поменялось, ведь я знал его,
как продюсера 1+1, а тут он уже большая величина в мировом кино, короче,
интрига. 

Оказалось, что не изменилось ничего, все, что он говорил, я слышал от него и
ранее. Со мной были мои коллеги, они предположили, что Александр Ефимович
что-то скрывал. Не думаю, мы же приходили не пытать его, не допрашивать, а
просто послушать. Слушали мы его где-то полтора часа (больше нам не позволили
другие дела) - я лично информации получил ноль. Точнее, получил информацию, что
он говорит то же, что и десять-пятнадцать лет назад. Так что что-то я получил. 

А говорил он то, что обычно говорит высокопоставленная богема. Сетовал на
недостаток идей, талантливых людей, а после переходил на то, что этим, немногим
талантливым людям можно все, весь мир лежит у их ног, а они перебирают
предложения. То есть, есть некая талантливая аристократия, которой можно все, а
прочим плебеям нужно принимать все причуды этих господ. Себя, естественно,
люди, говорящие это причисляют к аристократии, а прочих к плебеям. Все я это
слышал ранее от многих, и именно эти рассуждения и эта среда оттолкнули меня от
мира кино, и, кстати, заставили на много лет покинуть журналистику. 

Я понимаю, что Роднянский успешный, в нынешнем понимании, человек, но все же
для меня это полный бред, которым завешивают определенные обстоятельства того,
или иного успеха. В бизнесе эту роль играют понты. Кино, как и ТВ является
коллективным процессом, а потому один гений не создаст ничего и никогда. Другое
дело, что все можно приписать одному человеку, и даже не гению. 

Роднянский постоянно путал мир идей, и мир коммерческого успеха. И тут не было
подвоха, он это делал искренно. Мир идей далеко не соприкасается с миром
успеха, а успех не всегда идет от идей. При этом Александр Ефимович понимает,
что успешные люди появляются на стыке первого и второго, только вот природа
этих миров для него точно загадка, он не столько хочет понять, сколько угадать.
И тут, естественно, лотерея, когда угадаешь, а когда нет. 

Вот со средой, которая вечно хочет угадать, мне говорить не о чем. Не люблю
казино и азартные игры, не интересно. Возьмем самого Роднянского. Дело не в том
талантлив он или не талантлив, а в том, что он чувствовал политическую
конъюнктуру, а именно соединял мир идей с коммерческим успехом. Судите сами
первый украинский общенациональный коммерческий канал 1+1. Канал занимался
вестернизацией Украины, по сути вел Украину в мировое культурное и коммерческое
пространство.  Хорошая идея и правильное коммерческое направление. Канал, в
итоге, захватил информационное пространство Украины. Не будем дальше
рассказывать во что он это пространство превратил, это было уже без
Роднянского. Сам Роднянский поехал в Россию формировать вкусы среднего класса. 

Все это привело к фильмам Звягинцева, которые вызывают у меня отвращение (так
же, как и современный 1+1). С одной стороны, Звягинцев титулованный и успешный
режиссер, то есть, по мнению богемы тот, кому все можно, гений. А по мне успех
Звягинцева стоит не на том, что он показывает истинную современную Россию, а на
том, что он показывает Россию такой, какой определенные силы хотят ее видеть на
Западе. Вот и весь рецепт гениальности. 

Для многих современных гениев потакание низким запросам толпы, политической
конъюнктуре, привело к огромным гонорарам и успеху на Западе. Но когда толпа,
не без помощи этих гениев, превращается в стаю животных, то гении поджимают
губки, говорят "кошмар", и едут тратить свои гонорары в страну, где они еще не
успели нагадить. 

И никакого покаяния, переосмысления, самокопания от этих гениев ждать не
приходиться. Сейчас я не говорю о Роднянском, я говорю о многих, Роднянский
просто помог мне об этом задуматься. Кошмар, который сегодня есть в Украине
создавали долго и упорно. И не стоит говорить, что народ плохой, он
сопротивлялся, просто определенные люди, в том числе и талантливые, его
продали. И это заставило многих отчаяться и потерять всякие ориентиры. Проблема
богемы именно в том, что она ориентиры угадывает. А угадывает потому, что не
хочет брать труд искать. И тогда народ ведут другие, а потом, глядишь, и нет
такого народа. Вот советского народа уже нет. Мы только помним о нем, как о
древних греках и византийцах. Но богема не понимает, что мы теряем, она же не
понимает, а угадывает.

\ii{09_11_2021.fb.zharkih_denis.1.princip_bogemy.cmt}
