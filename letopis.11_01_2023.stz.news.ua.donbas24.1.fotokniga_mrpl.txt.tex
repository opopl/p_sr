% vim: keymap=russian-jcukenwin
%%beginhead 
 
%%file 11_01_2023.stz.news.ua.donbas24.1.fotokniga_mrpl.txt
%%parent 11_01_2023.stz.news.ua.donbas24.1.fotokniga_mrpl
 
%%url 
 
%%author_id 
%%date 
 
%%tags 
%%title 
 
%%endhead 

Ольга Демідко (Маріуполь)
11_01_2023.olga_demidko.donbas24.fotokniga_mrpl

Маріуполь,Україна,Мариуполь,Украина,Mariupol,Ukraine,Photography,ЗСУ,ВСУ,Фотографія,Фотография,Book,Книга,date.11_01_2023

Військовослужбовець ЗСУ видав фотокнигу, присвячену Маріуполю (ФОТО)

Воїн ЗСУ Мирослав Кобилянський видав унікальну фотографічну книгу «Мій
Маріуполь та світ довкола»

Наприкінці 2022 — на початку 2023 року побачила світ фотокнига захисника
України, фотографа, мандрівника Мирослава Кобилянського «Мій Маріуполь та світ
довкола». Після повномасштабного вторгнення рф Маріуполь зазнав нищівного
руйнування, тому світлини, на яких зафіксована природа, архітектура Маріуполя і
навколишня місцевість, сьогодні мають ще більшу цінність та важливість.

«У пам'яті немає терміну придатності... В цій книзі я зібрав власні світлини
Маріуполя, створені в період його розквіту. Фотокнига є своєрідною візитівкою
міста. В ній глядач побачить головну гавань Приазов'я, привабливу вітрину
Донбасу — неймовірний Маріуполь до повномасштабного вторгнення. Маю надію, що
світлини, зібрані в цій книзі, закарбуються у пам'яті кожного, хто хоча б раз
перегорне глянцеві сторінки...», — зазначив автор книги Мирослав Кобилянський.

Читайте також: «Смерть ходила за нами» — Євген Малолєтка про зроблені кадри війни у Маріуполі (ФОТО)

Мирослав народився в Коломиї, з 2009 року служить у Збройних силах України. З
2015 до 2022 року він проходив службу в Маріуполі. Світлини фотохудожника з
висоти пташиного польоту не тільки Маріуполя, але й інших міст України та
Європи завжди відрізнялися особливою майстерністю та виразністю. Саме тому,
коли маріупольці дізналися про вихід фотокниги Мирослава, одразу ж намагалися
її якомога швидше придбати.  «Кожна його робота — це освідчення в коханні
рідному місту. Я із захопленням, болем та сумом знайомилася з фотоспогадами
автора. Панорамні фото — з таких ракурсів своє місто я побачила вперше.
Місто-фортеця, місто у моря, місто мрійників, місто якого нема... Але ще буде, і
всі ми, маріупольці, мріємо і знаємо, що Маріуполь це — місто-перемога!

Ця книга, пам'ять і надія, і вона вже у фонді незламної Донецької обласної
бібліотеки для дітей!», — наголосила Директорка Донецької обласної бібліотеки
для дітей Юлія Василенко.

Читайте також: Поранене та ув'язнене місто у роботах маріупольського художника

Книга включає п'ять розділів: «Приморський район», «Центральний район»,
«Лівобережний район», «Кальміуський район» та «Світ довкола». Кожен з розділів
містить унікальні та короткі авторські описи як зображених локацій чотирьох
адміністративних районів міста, так і території в межах 100 км навколо
Маріуполя. Директор Львівського мистецького видавництва «Світло й Тінь», яке і
випустило фотокнигу, Володимир Пилип'юк зауважив, що це один з найособливіших
та захоплюючих проєктів видавництва.

«Без перебільшення можу сказати: за всі 5 років мого керування видавництвом ця
книга — виняткова в культурному, історичному та художньому планах праця. Її
ідея, вміст та виконання, впевнений, неодмінно відгукнеться багатьом із вас.
Сторінки видання формують цілісний фотоспогад про Маріуполь, яким він був до
війни — красивим, омитим солоними азовськими хвилями, мрійливим, дещо строгим
та, на превеликий жаль, тимчасово ефемерним — а проте таким, яким назавжди
залишиться в пам'яті тих, хто тут проживав і гостював», — поділився думками
Володими Пилип'юк.

Читайте також: Слово 2022 року: який вираз став головним в рік війни

Важливо, що книга є чотиримовною: містить переклади з української на
англійську, німецьку та іспанську. Придбати книгу можна через львівське
видавництво «Світло і тінь» чи звернувшись особисто до Мирослава.

«Ця книга про тебе й для тебе, місто зі сталі — Маріуполь», — підсумував автор
фотографічної книги «Мій Маріуполь та світ довкола» Мирослав Кобилянський.


Раніше Донбас24 розповідав, що танцювальний колектив з України Light Balance
Kids викликав фурор на американському шоу талантів.

Ще більше новин та найактуальніша інформація про Донецьку та Луганську області
в нашому телеграм-каналі Донбас24.

ФОТО: з особистого архіву Мирослава Кобилянського
