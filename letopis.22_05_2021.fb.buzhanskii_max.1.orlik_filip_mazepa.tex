% vim: keymap=russian-jcukenwin
%%beginhead 
 
%%file 22_05_2021.fb.buzhanskii_max.1.orlik_filip_mazepa
%%parent 22_05_2021
 
%%url https://www.facebook.com/permalink.php?story_fbid=1950693948428408&id=100004634650264
 
%%author 
%%author_id buzhanskii_max
%%author_url 
 
%%tags mazepa_ivan,orlik_filip,ukraina,istoria,konstitucia,konstitucia.orlika
%%title Злополучный Филипп Орлик, жалкая тень Мазепы
 
%%endhead 
 
\subsection{Злополучный Филипп Орлик, жалкая тень Мазепы}
\label{sec:22_05_2021.fb.buzhanskii_max.1.orlik_filip_mazepa}
\Purl{https://www.facebook.com/permalink.php?story_fbid=1950693948428408&id=100004634650264}
\ifcmt
 author_begin
   author_id buzhanskii_max
 author_end
\fi

\obeycr
	Бестолочь.
Знаете, персонаж то совсем уж жалкий и никчёмный даже на фоне того сомнительного круга личностей, который нам активно продвигают в последние годы, и я б и не вспомнил его в жизни, если б не неожиданный праздник, в виду "конституции" за авторством персонажа, любезно одалживаемой нам нашими шведскими друзьями на 30 летие Независимости.
Не хочу его очернить, не пытаюсь создать негативный образ, просто решил написать пару слов для тех, кто готов поверить в "конституцию".

Орлик.
Злополучный Филипп Орлик, жалкая тень Мазепы, не дотянувшаяся даже до предсмертного уровня раздавленного крушением всего и вся гетмана.
Умел писать и был послушным.
Если коротко, то все достоинства изложены выше.
Если серьёзно, то в огромном административном водовороте мазепинских дел был незаменим, как всегда трезвый, абсолютно лояльный, лишённый всяческих амбиций и очень хорошо образованный человек.
Не обладающий никаким весом  среди Старшины в принципе, и потому не имеющий врагов.
Нечто вроде раннего Выговского при Хмельницком, только совсем лишённое стержня, и зубы готовое сцепить лишь над куском говядины.
История Мазепы и его драма, когда Гетман трагически ошибся в расчётах, это другая история.
В которой Орлику места нет, потому что его мнения вообще никто не спрашивал.
"Возьму одного Орлика и уеду!"- орал Мазепа шантажировавшей его и требующей перехода к Карлу Старшине, и никому в голову не приходило узнать, что по этому поводу думает Орлик.
В итоге, на измену то решились все, а вот рядом только Орлик и оказался, и ещё пару полковников вроде Данилы Апостола, получившего от Мазепы приказ ехать к Петру и пытаться обменять поимку Карла на прощение, да так и не вернувшегося.
В итоге, к Полтаве Мазепа прибыл с горсточкой своих людей, упомянутым Орликом, и.... запорожцами.
Которых Мазепа люто ненавидел, как злейших врагов хоть какого то порядка в Гетманщине, и здорово репрессировал в свое время, а Орлик- боялся до посинения, потому что саблю в принципе видел только в мазепином сундуке, а руками не трогал.
Но так легла карта, совпали интересы, других союзников не было.
Кстати, когда речь заходит о союзниках, меня всегда интересует вопрос:
Почему взятый штурмом как ресурсная база изменника Мазепы Батурин это драма, а вырезанный шведами Веприк- так вообще никогда не вспоминается?
Там в Веприке какие то меньше украинцы были, чем в Батурине?
Короче, Гетман умер сразу после Полтавы, в Бендерах, городе, давшем нам взамен другого Гетьмана, Сивочолого.
Умер, а Карлу Двенадцатому нужен был новый, чтобы под него оформлять союз с турками.
И запорожцам нужен был, чтобы под него получать от нового сюзерена финансирование, иначе к чему такое сюзеренство, а?
Вариантов не было вообще никаких, Старшина осталась с Петром, из старой, назовём это так, команды, под рукой был только Орлик, так как племянник Мазепы Войнаровский от высокой должности отказался наотрез, скромно предложив отдать ему всю мазепину казну или хотя бы ту часть, которую Карл ещё не успел запихать в свои сундуки.
Короче, гетмановали Орлика.
Но при этом, высокие договаривающиеся стороны предусмотрительно подозревали друг друга в немедленном предстоящем кидке.
Во избежание чего, и пришлось подписать внутреннее соглашение о разделе властных полномочий, в название которого на латыни, чтоб солидней было, и вписали нечто, звучащее как конституция.
И бывшее, на самом деле, обычным коалиционным соглашением.
Понимая, что кидок неизбежен даже на латыни,  в качестве гаранта вписали Карла Двенадцатого, что и привело к получению Швецией экземпляра опуса, который приедет к нам 24 августа.
Вписали, кстати, в качестве сюзерена и протектора, поэтому байки о борьбе Мазепы и тд за независимость выглядят немного дивно.
В смысле, ещё более дивно, чем всё остальное.
На этом всё, если не считать того, что уже в следующем году, Орлик вместе с запорожцами привел сюда орду крымских татар, которая грабила, жгла, убивала, и увела с собой огромное количество порабощенных украинцев, которыми Орлик банально расплатился за этот визит.
Это сейчас модная для героизации линия поведения борцов за независимость, согласитесь, есть ведь примеры и посвежее.
Всё, больше в жизни несчастного Филиппа ничего выдающегося не было, половину её он потратил, вымаливая у Петра прощение, вторую- вымаливая у Султана пенсию.
Я ему не судья, и не таких изламывала судьба, просто никак не пойму двух вещей.
Что положительного в самом Орлике, вот хоть на миллиметр, а?
И что такого значимого для страны в его договорняке с запорожцами, сугубо внутреннем документе, чтобы придавать этому такую значимость?
Особенно на фоне целого ряда документов, от Зборовского договора и Переяславских статей, до статей Коломацких.
Сказал бы, что тень Мазепы, но нет.
Тенёчек.
\verb|#Максории|
\verb|#Историидляленивых|
\restorecr

\ifcmt
  pic https://scontent-lga3-2.xx.fbcdn.net/v/t1.6435-9/189739963_1950694081761728_2552951254298048025_n.jpg?_nc_cat=111&ccb=1-3&_nc_sid=8bfeb9&_nc_ohc=UAoWE5b7LyUAX_x5mQo&_nc_ht=scontent-lga3-2.xx&oh=08c3952741a4005f2bedf795a0e85ae9&oe=60CCEDD5
\fi

