% vim: keymap=russian-jcukenwin
%%beginhead 
 
%%file 12_08_2019.stz.news.ua.mrpl_city.1.gennadij_djuzhok_zhyttja_velyka_meta
%%parent 12_08_2019
 
%%url https://mrpl.city/blogs/view/gennadij-dyuzhok-vse-zhittya-osyayane-velikoyu-metoyu
 
%%author_id demidko_olga.mariupol,news.ua.mrpl_city
%%date 
 
%%tags 
%%title Геннадій Дюжок: "Все життя осяяне великою метою!"
 
%%endhead 
 
\subsection{Геннадій Дюжок: \enquote{Все життя осяяне великою метою!}}
\label{sec:12_08_2019.stz.news.ua.mrpl_city.1.gennadij_djuzhok_zhyttja_velyka_meta}
 
\Purl{https://mrpl.city/blogs/view/gennadij-dyuzhok-vse-zhittya-osyayane-velikoyu-metoyu}
\ifcmt
 author_begin
   author_id demidko_olga.mariupol,news.ua.mrpl_city
 author_end
\fi

Є в Маріуполі \emph{люди високої мети}, у яких можна багато чому повчитися. Про них
необхідно писати і говорити, адже їхні досягнення та життєвий шлях надихають.
\textbf{Геннадій Іванович Дюжок} – дійсно людина високої мети. Всі його досягнення
складно перерахувати, одного нарису буде замало. Зараз він бізнес-тренер
міжнародного класу, парапсихолог, народний лікар, керівник ГО \enquote{КСК
\enquote{КІНОНАГАСІ}}, а в минулому – відомий донецький економіст, до якого возили дітей
на курси з усієї області. Однак війна змусила чоловіка повернутися до рідного
міста.

\ii{12_08_2019.stz.news.ua.mrpl_city.1.gennadij_djuzhok_zhyttja_velyka_meta.pic.1}

Народився Геннадій у Маріуполі. Згадує, що з дитинства мама дуже піклувалася,
хоча їй і було неймовірно важко. Тягнула кредит по квартирі, працювала на
декількох роботах. Батько їх рано залишив, потім з'явився вітчим, який і став
справжнім татом для Гени. У дитячому садочку хлопчика любили всі вихователі, у
школі – вчителі. Завдяки своїй поведінці і турботі мамі вдалося виховати
правильні риси у сина, адже він виріс благородною і відповідальною людиною.
Мамі допомагав просто так, без прохань. З дитинства Геннадій любив море. Майже
ціле літо проводив то у піонерських таборах, то у пансіонатах. Великий вплив на
його виховання мала друга мати – хрещена \textbf{Дробанова Олександра Федорівна} та її
чоловік (був старпомом в АМП) \textbf{Микола Полікарпович}, який брав хлопця та свого
сина (друзі дитинства) на суда та рейси. Також разом з його сином Гена ходив до
гуртка \enquote{Юний моряк}. Так море стало другою стихією у житті для нашого героя.
Геннадій згадує, що допомагали багато людей, бо мама була дуже товариською
людиною і також допомагала багатьом. Людяність та благородство – це те, що він
засвоїв з дитинства. Читав дуже багато, здебільшого фантастику, детективи,
мемуари учасників Другої світової війни. Щотижня хрещена мама влаштовувала
культурний похід у кіно (рідше театр), потім - кафе, пікнік чи застілля.
Водночас класний керівник в школі, \textbf{Лідія Сергіївна Попова}, весь час нагадувала,
що слова \emph{\enquote{Якщо не я, то хто?}} мають стати девізом в житті кожної людини.
Завдяки вчительці Гена виріс небайдужим до життя людей та міста. Згадує її з
великою повагою та вдячністю. Загальний рівень доходу в сім'ї хлопця був не
дуже високий, але завдяки хрещеній та її чоловіку з дитинства у нього були
гарні речі як з одягу, так і предметів побуту. З їжею було взагалі на дуже
високому рівні, бо обидві мами працювали у тресті столових та ресторанів. Отже,
дитинство було досить світлим періодом в житті Гени, хоча здоров'я не було
міцним (на жаль, була послаблена дихальна система).

\textbf{Читайте також:} \emph{Новая кровь: как в Мариуполе стать актером}%
\footnote{Новая кровь: как в Мариуполе стать актером, Богдан Коваленко, mrpl.city, 12.08.2019, \par%
\url{https://mrpl.city/news/view/novaya-krov-kak-v-mariupole-stat-akterom}
}

\ii{12_08_2019.stz.news.ua.mrpl_city.1.gennadij_djuzhok_zhyttja_velyka_meta.pic.2}

Освіта у Геннадія Івановича дуже різноманітна та цікава – вистачить на
декількох людей. По-перше, середня школа з медаллю, ходив грати в шахи –
багаторазовий чемпіон таборів, двічі грав з гросмейстером – нічия та раз виграв
(що є дійсно досягненням), займався боксом, закінчив музичну школу на відмінно
за класом скрипка. У музичній школі його навчала чудовий керівник, майже третя
мама – \textbf{Кануннікова Лідія Миколаївна}. Пізніше виступав в оркестрі \textbf{Семена
Рубінсона}, в театрі пісні \enquote{Радість} від Палацу культури металургів, де мав
насичене творче та гастрольне життя. Також займався в гуртку філателії,
морській школі при ОСОУ (ДОСААФ). Закінчив на відмінно Донецький державний
університет, економічний факультет. У студентські роки почав поглиблено
займатися спортом та бойовими мистецтвами, цілительством, езотеричними
напрямами, кікбоксінгом, карате, дзюдо, йогою, джиу-джитсу, китайськими та
в’єтнамськими напрямами. Працював викладачем у вишах, тренером з єдиноборств,
цілителем. Пройшов спеціальні курси для цього. Крім того, Геннадій Іванович
пройшов навчання на тренера-консультанта з бізнесу по міжнародним програмам,
працював у ведучій консалтинговій компанії в Донецьку. Наразі так і працює:
коуч, бізнес-тренер та консультант, тренер з джиу-джитсу та самооборони, цигун
та йоги, цілитель (має звання \enquote{Майстер \enquote{Золоті руки}}). Аналогів, за якими
працює Геннадій Іванович, сьогодні у Маріуполі та загалом в Донецькій області
немає. Зараз навчається у майстрів тих напрямів, які найбільш зацікавили.
Геннадій має посвяти та передачу дхарми від буддійських майстрів: Вон Мьон
синім, Дзюнсей Терасава синім, Ву Бонг синім, Лю Гуанлай, Фен Шао І (Фен Даші),
Намкай Норбу рінпоче, Кармапа, Далай лами.

Свою діяльність чоловік завжди починав з себе. Адже бажання покращити своє
здоров'я виникло ще у підлітковому віці. Саме тоді він захопився йогою,
Шаоліньцюань, цигун, бо фармакологічна медицина вичерпала свої можливості у той
період (йому ставили діагноз \enquote{бронхіальна астма}). Був активістом як у школі,
зокрема проводив політичні бої. Згадує свою вчительку з історії та права –
\textbf{Балабан Людмилу Володимирівну} – унікального педагога та неймовірно творчу
особистість. У виші боровся за права студентів, слідкував за порядком у
гуртожитку (керівник оперзагону), був старостою. Працюючи викладачем у вузі,
організовував різні громадські заходи для студентів. Як консультант вже більш
фахово брав участь в якості експерта різних проектів у Донецьку. Також у
Донецьку був куратором програми роботи зі школярами та студентами щодо
здорового способу життя та якісного навчання. Війна перевернула життя, тож
Геннадій повернувся до Маріуполя. Продовжив практично те, що робив у Донецьку,
але сфера діяльності значно розширилася. Сьогодні займається патріотичним
вихованням, оздоровленням, розвитком особистості.

\textbf{Читайте також:} \emph{В июле мариупольцы больше потратились на образование и медицину, но сэкономили на одежде и коммуналке}%
\footnote{В июле мариупольцы больше потратились на образование и медицину, но сэкономили на одежде и коммуналке, mrpl.city, 09.08.2019, %
\par\url{https://mrpl.city/news/view/v-iyule-mariupoltsy-bolshe-potratilis-na-obrazovanie-i-meditsinu-no-se-konomili-na-odezhde-i-kommunalke}}

\ii{12_08_2019.stz.news.ua.mrpl_city.1.gennadij_djuzhok_zhyttja_velyka_meta.pic.3}

Завдяки громадській організації Геннадія Івановича \enquote{Культурно-спортивний клуб
\enquote{Кінонагасі}} у Маріуполі відбувається патріотичне виховання – відвідування
шкіл, участь у військово-патріотич\hyp{}них заходах та іграх; працює гурток (секція)
по оздоровленню та самообороні (на базі джиу-джитсу) при Міському центрі
позашкільної освіти (керівник Балабан Людмила Володимірівна); духовне
відродження нації – спеціальні семінари та тренінги, вебінари. Водночас ГО бере
учать та допомагає іншим, наприклад: фестиваль \enquote{Азія-фест}, фестиваль \enquote{З Країни
в країну}, програми на телебаченні, міські програми, участь в деяких засіданнях
міськвиконкому, проектах щодо виховання дітей та підлітків. Діти, що займаються
в секціях, також беруть участь у міських змаганнях та проектах. Зокрема,
вдалося створити Кабінет по оздоровленню, він має тренажер \enquote{Правило}, на якому
можна розтягувати та корегувати поставу не тільки дітям та їхнім батькам, але й
усім бажаючим. Приходять іноді спортсмени інших секцій на реабілітацію,
військові, пенсіонери та звичайні люди, які потребують фахової допомоги.
Загалом клуб \enquote{Кінонагасі} має безліч секцій і напрямів діяльності: \emph{джиу-джитсу,
йога, цигун, біокорекція опорно-рухового апарату, реабілітація методиками:
остеопатії, миттєве зняття болю, ударно-хвильовий вплив, вісцеральна терапія,
рефлексотерапія, кінезіологія, аюрведи, психосоматіка та медитації}.

Члени клубу співпрацюють з іншими цілителями, зокрема з фітнес-клініки \enquote{Аурум}
(м. Київ). У Геннадія Івановича вже є свої учні-реабілітологи, які так само
ведуть прийом. Він підкреслює, що його діяльність самодостатня та не потребує
зовнішніх оцінок. Ця діяльність приносить радість та задоволення. Коли люди
радіють, що стали здоровішими, позбулися хвороби, болю, захистили себе та
близьких – він радіє, щонайменше, вдвічі. Щодо тренерської діяльності, то є
тренери, що готують саме чемпіонів, а є – що готують людей до життя. Нашому
герою імпонує другий варіант. Займе його учень на змаганнях перше, друге, третє
місця – \emph{\enquote{не це важливе, важливіше, що дитина вийшла, поборола страхи, стала
сильнішою}}. Однодумцями маріупольця є всі, хто займається у клубі, чи
займається аналогічними справами, або навіть тільки цікавиться. Порахувати їх
неможливо Співпрацюють з тренерами з інших спортивних секцій: боротьби, боксу,
кікбоксу, тайцзи, танців та інших.

У Маруполі народний лікар найбільше любить море, свій спортзал та \enquote{Порт-сіті}.
На думку Геннадія, місто дуже відстало від світових трендів, а намагання
поспішно наздогнати приводить не до гармонійного результату.

\textbf{Читайте також:} \emph{Художники из Германии и Словакии примут участие в благотворительном аукционе в Мариуполе}%
\footnote{Художники из Германии и Словакии примут участие в благотворительном аукционе в Мариуполе, mrpl.city, 09.08.2019, \par%
\url{https://mrpl.city/news/view/hudozhniki-iz-germanii-i-slovakii-primut-uchastie-v-blagotvoritelnom-auktsione-v-mariupole}
}

\ii{12_08_2019.stz.news.ua.mrpl_city.1.gennadij_djuzhok_zhyttja_velyka_meta.pic.4}

\textbf{Хобі:} \enquote{хобі – то кіно, я кіноман. Колекціоную переживання, а потім їх відпускаю
на свободу. В мене робота – то теж хобі}.

\textbf{Курйозний випадок:} 

\begin{quote}
\em\enquote{Якщо жити творчо, то життя настільки цікаве, що на курйози місця не вистачає.
Розкажу просто життєву історію. Мені не вдавалося взяти участь у
навчанні-семінарі у китайського майстра Лю Гуанлай, але я вибрав час,
відлучився з роботи і поїхав зустріти його та прихопив деякі смаколики, бо
людина з дороги, а ми українці – народ щирий та радушний. Майстер радісно мене
зустрів, привітав, засмутився, що я не зможу взяти участь у програмі навчання.
Я вручив йому пакет з їжею. В нього очі аж загорілися, бо він явно зголоднів, а
ніхто з інших учнів нічого не привіз та кафе не запропонував. Я вклонився,
сказав, що наступного разу візьму участь та пішов. Зненацька мене наздоганяє
хвиля енергії і моє тіло починає танцювати внутрішньо форми тайцзи. З того часу
відчуття внутрішньої наповненості кожної форми є завжди. Ось така вона буває
незрима, але відчутна передача від серця до серця Духу вчення}.
\end{quote}

\textbf{Порада маріупольцям:} 

\begin{quote}
\em\enquote{Починайте з себе, зверніть увагу на опорно-руховий апарат
як дітей, так і свій, на культурне виховання. На атракціони для дітей гроші
витрачають батьки, а прийти на біокорекцію опорно-рухового апарату чогось не
вистачає чи то усвідомлення необхідності, чи то культури. Працювати зі
свідомістю потрібно кожній людині, незалежно від соціального статусу, тоді
афектів стане менше, людина буде більш ефективною у своїй діяльності та
справжній (не для \enquote{галочки}) допомозі місту, іншим людям. Любов – то є дія.
Отже є концепція \enquote{маленьких кроків} - її треба втілювати у житті скрізь}.
\end{quote}
