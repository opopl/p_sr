
\isubsec{4_6_The_inner_workings_of_texht}{The inner workings of \texht}

An insight into how the system operates can help with installation or
with improving and extending its use. It can also explain the system's
capabilities and limitations.  Although most of these issues are
typically important for only a few, more advanced users, many readers
might like to skim this section quickly just to get a general impression
of the topics covered. 

\isubsubsec{4_6_1_The_translation_process}{The translation process}

The command line 

\begin{verbatim}
ht latex filename 
\end{verbatim}
 
%%page page_206                                                  <<<---3

requests a translation of the source \emph{filename} into HTML. The script \verb|ht| calls the 
different utilities involved in the translation process (\reffig{4-16}); it consists of five 
steps, of which the first three involve running \ \LaTeX: 

\begin{lstlisting}
latex filename
latex filename
latex filename
tex4ht filename
t4ht filename
\end{lstlisting}

\isubsubsec{4_6_2_Running_LaTeX}{Running \ \LaTeX}

The three compilations of the \ \LaTeX\  source by \ \TeX\  are needed to ensure both 
proper references within hypertext links and proper arrangements of cells in tables 
 
%%page page_207                                                  <<<---3
 
containing the \verb|\multicolumn| command. More compilations might be
needed for sources in which the \verb|\multicolumn| command merges a
large number of cells. 

The log file of the compilation will include information similar to that
shown in \reffig{4-17} (without the line numbers). The third line in the
example requests processing by the \verb|tex4ht| program of the output
file \verb|try.dvi| to produce a \verb|try.html| file and possibly other
files as well. The fifth line tells us about the style file, named
\verb|try.css|, supplied for the HTML output. The eighth line mentions a
bitmap file, named \verb|try0x.gif|, that needs to be produced by
\verb|t4ht| from the first figure of the \verb|try.idv| file (Section
4.6.4 on the facing page). 

The first and second lines show the two times that the package file
\verb|tex4ht.sty| is read. The first time gets the \verb|\Preamble|
command and about half a dozen hooks with default configurations. The
second time is activated by the \verb|\Preamble| command, and it reads
the portions of the style file selected by the options. The early set of
hooks allows the customization of the headers before they are written
during the second loading of the file. They include the hooks named
\verb|HTML|, \verb|HEAD|, \verb|BODY|, \verb|TITLE|, \verb|TITLE+|, and
\verb|HtmlPar| (use of these is discussed in Section B.2.1.4). 

During a compilation, \texht stores the entries for the tables of
contents in a \verb|toc| file; these are to be used in the next run of
the source document. The entries from the previous run are moved into an
\verb|otc| file so that they are available during the current run. The
file \verb|try.otc| shown in the seventh line of the example log file
shows such a file in use. 

\isubsubsec{4_6_3_Running_the_tex4ht_program}{Running the tex4ht program }

Running a source file with \LaTeX\  outputs a standard DVI file,
containing special instructions for the \verb|tex4ht| package. The
postprocessor program 
\ifx\HCode\undefined%
\footnote{%
The tex4ht utility is programmed in C with system calls to a very few
simple standard functions. Its hypertext fonts are system-independent
files of plain text and, like fonts of \TeX\  and its DVI output, are
portable across all systems. This means that the tex4ht utility is easy
to transport between different platforms, and the output independent of
the platform on which the program is run. 
}%
\else%
\fi%

\verb|tex4ht| uses these instructions to determine how the DVI code should be
processed. They tell \verb|tex4ht| where the output files should start
and end, what names should be given to the files, 
the HTML decorations to be assigned to the symbols of the different fonts, where 
the code for the bitmap pictures resides, and so on. 
%%page page_208                                                  <<<---3

To perform the translation into HTML, the \verb|tex4ht| utility needs to
know where the font metrics of  \TeX reside (\verb|tfm| files), where
the private hypertext fonts of \verb|tex4ht| are stored (see \verb|htf|
fonts, \refsec{4_6_7_The_font_control_files} on page 190), and other information that relates to
the environment in which the utility is working. Some of this
information might be included in the executable code of \verb|tex4ht|
during compilation.  The rest of it is specified in a control
(\verb|env|) file (see \refsec{4_6_8_The_control_file} on page 193). 

An invocation of \verb|tex4ht|, without any parameters, produces a usage message 
like the following: 

\begin{verbatim}
  tex4ht in-file[.dvi] 
				[-ttfm-font-dir] 
				[-ihtf-font-dir] 
				[-eenv-dir] 
				[-dout-dir] 
				[-gbitmap-file-ext] [-blg-divide-script] [-slg-gif-script] 
\end{verbatim}

Usually the user will simply give the name of the \verb|dvi| file. For instance, the 
command \verb|tex4ht try.dvi| produces a main file named \verb|try.html|. 

The command-line option \verb|-g|\emph{bitmap-file-ext} determines the
file extension of the bitmap files of picture symbols. The option
\verb|-d|\emph{out-dir} specifies a directory for the output files (for
instance, \verb|tex4ht try -d/tmp/| on a UNIX platform). The three
options -e\emph{env-dir}, -i\emph{htf-font-dir}, and
-t\emph{tfm-font-dir} specify directories to be searched for the control
file (see \refsec{4_6_8_The_control_file} on page 193), the virtual hypertext fonts, and
the TFM files. 

\reffig{4-18} shows an example of messages produced during a run of
tex4ht.  The second line shows the TEX metric cmr1O being found, and the
third line shows the corresponding virtual hypertext font being loaded.
The fifth line reports the loading of the file \verb|cmr.htf| instead of
\verb|cmbx1O.htf|. The symbols of the bold font have the same HTML
representations as those of the normal font. The two fonts, however,
will end up with different presentations, using font information
provided in the CSS file. 

The tenth line requests the execution of a script (see \refsec{4_6_6_A_taste_of_the_lg_file} on page 189). 

\isubsubsec{4_6_4_A_look_at_t4ht}{A look at t4ht}

When \verb|tex4ht| is done with processing the DVI file of \TeX, it
leaves behind a scaled-down DVI file with the extension name \verb|idv|.
This consists of all the DVI code fragments that need to be translated
into pictures, with each page holding exactly one picture. The first
half of the file is for pictures that are requested in the source IIHEX
file; the second half is for pictures requested in the fonts. 

%%page page_209                                                  <<<---3

The information on how the pictures should be named is recorded in a
\verb|lg| file; part of that information is also shown in the log file
of \TeX. For instance, in the log of \reffig{4-17}, the eighth line has
the statement \verb|try.idv[1] (try0x.gif)|. It says that the picture
from the first page of \verb|try.idv| will be stored in a file named
\verb|tryOx.gif|. 

The lg file may also contain style information for the document's css
file, user requests for calls to system functions, and other types of
entries. The lg file may, therefore, be regarded as a script to record
the actions that must be taken after the \verb|tex4ht| utility completes
its job. 

The \verb|t4ht| script has to execute the contents of \verb|lg| file.
This may be done by using a system-dependent program for interpreting
the script, or the script itself may be executed, if it is expressed in
terms of a scripting language recognized by the current platform. 

\isubsubsec{4_6_5_From_DVI_to_GIF}{From DVI to GIF}

\texht does not provide tools for converting DVI code into bitmap form. It relies 
on external tools being available for the task. 

The second and third lines of \reffig{4-19} show a two-step conversion of
a DVI picture into a bitmap GIF file. The first step uses the
\textbf{dvips}  driver to convert the picture into an intermediate file
in PostScript. The second step calls the \textbf{convert} program (part
of ImageMagick) to complete the task. 

The fourth line in the example says that the file \verb|cmsy10-2a.gif| already exists; 
therefore, it is not created again. 
%%page page_210                                                  <<<---3
The dimensions of the pictures depend on the size of the source documents 
and, in the case of picture symbols, on the sizes of the fonts in use. However, the 
dimensions and quality of the pictures may also depend on the settings chosen for 
the external utilities in use and, of course, on the display used to view the HTML 
s. 

\isubsubsec{4_6_6_A_taste_of_the_lg_file}{A taste of the lg file}

In essence, the \verb|lg| file is just a wish list written by the
\verb|tex4ht| program. Some of the entries originate in the program
itself. Other entries are requests made in the source document with
\verb|tex4ht| playing the intermediate role of passing the requests into
the file. 

Requests from the source document can be made with a command of the form 

\begin{verbatim}
\Needs{request} 
\end{verbatim}

They will end up in the \verb|lg| file embedded within an envelope. The envelope itself 
is configurable with the \verb|\Configure{Needs}{content}| command, where the latter 
command should use the token \verb|#1| to refer to the parameter of \verb|\Needs|. 

Consider the following \ \LaTeX\ code: 

\begin{lstlisting}

  \Needs{chmod 644 *.html} 
  \Configure{Needs}{#1} 
  \Needs{Say hello} 

\end{lstlisting}

With the configuration 

\begin{lstlisting}
  \Configure{Needs}{1. \the\inputlineno\space--- needs --- #1 ---} 
\end{lstlisting}

the source code contributes the following two lines to the \verb|lg| file: 

\begin{verbatim}

1. 12 --- needs --- "chmod 644 *.html" --Say hello 

\end{verbatim}

The default \verb|t4ht| program distributed with \texht will interpret the format 
of the first line 

\begin{verbatim}
1. integer --- needs --- "content" --\end{verbatim}

as a system call to the UNIX command \verb|chmod 644 * .html|. On the other hand, it 
ignores the second line because the utility was not programmed to recognize the 
pattern of that line. 


The \verb|lg| file starts with all the contributions made in the source
document, typically from \verb|\Needs| commands in style files. It then
lists the contributions originating in the \verb|tex4ht| utility. The
two types of contributions are separated by a distinguishing line, put
there by \verb|tex4ht|, in the \verb|lg| file. 
 
%%page page_211                                                  <<<---3
The default patterns for the contributions of the \verb|tex4ht| utility are determined 
at the time the utility is compiled for the platform in hand. These patterns can be 
overridden in the runtime control file. 

The lg files generally consist of requests to create bitmap pictures,
explicit contributions to the CSS file, and font information that
implicitly asks for contributions to the CSS file. 

%%%TODO

\isubsubsec{4_6_7_The_font_control_files}{The font control files}

Text in normal \ \LaTeX\  output often has many symbols coming from different fonts. 
These symbols are put in the source file by a character or macro pointing to a 
symbol in an actual font. Thus, when using the Computer Modern Roman 10pt 
font, character a in the input means ``set character number 98 from font cmr10''
and \verb|\Gamma| means ``set character number 1.'' Because \texht is not in charge of 
rendering the symbols, it supplies content for each symbol, instead of getting the 
glyph from the font file (as a DVI driver would usually do). This content is specified 
in \texht's virtual hypertext font files and is used by the Web browser to put a real 
character on the screen. 

\ipar{4_6_7_1_Using_the_files}{ Using the files}

For a given \ \LaTeX\  font, tex4ht assumes a virtual hypertext font,
the main filename of which is a subset of the \ \LaTeX\  font name. If
more than one such file is available, the one with the longest name is
assumed. Accordingly, \verb|tex4ht| searches in turn for the
\verb|cmr10.htf|, \verb|cmr1.htf|, \verb|cmr.htf|, \verb|cm.htf|, and
\verb|c.htf| files when it needs a virtual hypertext font for a \
\LaTeX\  font named \verb|cmr10|. 

Each virtual hypertext font file starts and ends with identical
identification lines that specify the font name, the character code of
the first symbol, and the character code of the last symbol (Figure
4.20). 

For each character, the file has a line consisting of three fields: a
text string, a class number, and a (possibly empty) comment. The first
and second fields must be delimited with a single character; any
delimiters can be used, but within any one line they must all be the
same. 
%%page page_212                                                  <<<---3

The class is a number between 0 and 255, where an empty class field is treated 
as 0. An entry with an even-numbered class contributes the content of the first field 
to the symbol. An entry with an odd-numbered class requests that a bitmap picture 
for the symbol be used, with the first field contributing an alternative content for 
character-based browsers. 

From input like ``\verb|a $\Gamma$|'', \texht will request a bitmap rendering
for \verb|\Gamma| in a file named \verb|cmr10-0.gif| and produce the output
\verb|a <IMG SRC="cmr10-0.gif" ALT="G">| in the HTML file. The \verb|cmr10| in
the filename indicates the \ \LaTeX\  font name; the 0 indicates the character
number in the font. 

The first field in the entry may refer directly to characters in a font by
placing the corresponding character code between backslashes. On the other
hand, a backslash character \verb|\| must be represented by a pair of backslash
characters \verb|\\|. 

The symbols \verb|<|, \verb|>|, and \verb|&| should be represented, for
instance, by the strings \verb|&1t;|, \verb|&gt;|, and \verb|&amp;|,
respectively. 

\ipar{4_6_7_2_Configuring_the_fonts}{Configuring the fonts}

The content retrieved for the symbols from the virtual hypertext font tables is
written into the HTML files in a format that is governed by the following
command.  It provides a seven-component template for the symbols of the
specified class. The delimiter must be a character that does not appear in the
components. 

\begin{lstlisting}

\Configure[htf]{class}{delimiter} 
  {parameter-1} {parameter-2} {parameter-3} 
  {parameter-4} {parameter-5} {parameter-6} 
  {parameter-7} 

\end{lstlisting}

For example, the \ \LaTeX\  code \verb|\textsc{a}| produces \verb|<SMALL>A</SMALL>| in HTML 
3.2 mode and \verb|<SPAN CLASS="small-caps">A</SPAN>| in HTML 4.0 Transitional
mode. The \verb|htf| font provides the content A of class 4 for the character
'a' in either mode. In the first mode, the markup is due to the default
configuration for symbols of class 4 set by the command 

\begin{lstlisting}
  \Configure{htf}{4}{+}{<SMALL>}{}{}{}{}{}{</SMALL>} 
\end{lstlisting}

The only difference in the second mode is in the default configuration 

\begin{lstlisting}
\Configure{htf}{4}{+}{<SPAN CLASS="}{}{}{}{}{small-caps">}{</SPAN>} 
\end{lstlisting}

For a symbol whose class is an even number, the first parameter is printed
literally. The second parameter should comply with the C language conventions,
and, if it is not empty, it is used to output the font name. The third and
fourth parameters are used in a similar manner for writing the font size and
its magnification, respectively. The remaining parameters are written
literally, where either the fifth 
 
%%page page_213                                                  <<<---3
 
or the sixth parameter must be empty. The string contributed from the htf file is 
introduced just before the last parameter. 

Symbols of odd classes use the parameters in a similar manner to output the 
font name, the alternate string from the htf font, a second copy of the font name, 
the font size, the font magnification when it differs from 100\%, and the character 
code. The configuration for class 0 is also used to provide extra markup to symbols 
of the other classes. 

The \verb|\NoFonts| and \verb|\EndNoFonts| commands suspend and resume, respectively, 
the contributions of the \verb|\Configure{htf}| command. 

\ipar{4_6_7_3_Adding_style}{Adding style}

Contributions of \verb|htf| fonts to the CSS file can be configured with the commands 

\begin{lstlisting}
	\Configure{htf-css}{class}{content} 
	\Configure{htf-css}{fontname}{attributes} 
\end{lstlisting}

The command 

\begin{lstlisting}
\Configure{htf-css}{4}{.small-caps {font-variant: small-caps; }} 
\end{lstlisting}

contributes \verb|. small-caps{font-variant: small-caps; }| for symbols of class 4. 

On the other hand, the command 

\begin{lstlisting}
\Configure{htf-css}{cmmi}{font-style: italic;} 
\end{lstlisting}

results in contributions like \verb|.cmmi-7{ font-size:70%; font-style: ita1ic;}| 
and \verb|.cmmi-10{ font-style: italic;}|. 

\ipar{4_6_7_4_Font_clues}{Font clues}

Existing virtual hypertext fonts may be redesigned by users to obtain alternative 
output; new ones may be produced to accommodate missing fonts. If no matching 
\verb|htf| can be found, tex4ht will issue warning messages such as 

\begin{verbatim}
---warning --- Couldn't find font 'fontname.htf'
(char codes: first--last) 
\end{verbatim}

until the new fonts are provided. 

With the package option \verb|ShowFont|, source code like 

\begin{lstlisting}
\font\x=fontname\ShowFont\x 
\end{lstlisting}

produces a picture showing the normal result for the different symbols in the given 
font. 
 
%%page page_214                                                  <<<---3
 
\isubsubsec{4_6_8_The_control_file}{The control file}

The task of the runtime control file is to allow the \verb|tex4ht| utility to
adjust itself to the platform on which it runs and to the needs of its users
without having to be recompiled. The file is called either \verb|tex4ht.env| or
\verb|.tex4ht|, and it might have more than one copy in a given installation.
For instance (in order of priority), one copy may reside in a directory
indicated with the -e option of the command line (see \refsec{4_6_3_Running_the_tex4ht_program} on page
186). A second file may reside in the working directory, and a third one may be
in a directory whose location is hard coded within the program. 

The file itself is made up of entries identified by the first character in each line. 
The following are some of the possible options: 

\begin{itemize}
  \item \textbf{t} Identifies a directory to be searched for the font metric (\verb|tfm|) files of TEX. 
  \item \textbf{i} Identifies a directory to be searched for the virtual hypertext fonts (\verb|htf|) of 
    \texht. 
  \item \textbf{a} Different fonts of \ \LaTeX\  may consist of identical sets of symbols that vary just 
    in size or style. Such fonts would translate to identical virtual hypertext fonts, 
    so the ``a'' character introduces font aliases. 
  \item \textbf{g} Identifies the extension name given to bitmap files. Currently such bitmap files 
    are used only for picture symbols in virtual hypertext fonts. 
\end{itemize}

Consider the following control file (the line numbers are not part of the file): 

\begin{verbatim}
1 t/n/candy/tex/texmf/fonts/tfm/! 
2 i/n/soda/tex4ht.dir/ 
3 1/n/soda/tex4ht.dir/ht-fonts/! 
4 acmbx cmr 
5 acmsl cmr 
6 g.jpg
\end{verbatim}

The first line points to a directory to search for font metric files that are not 
available in the current directory; the exclamation mark ! indicates that the search 
should extend to subdirectories of all depths. 

The second and third lines specify directories to locate \verb|htf|
fonts, where recursive searching into subdirectories is allowed within
the directory listed in the third line. 

The fourth line states that requests for \verb|cmbx| fonts should use
the \verb|htf| file \verb|cmr.htf|. The fifth line is ignored because it
starts with a blank character; this character is not within the options
available for entries of the control file. 

The sixth line requests an extension name of \verb|jpg|, instead of the
default extension \verb|gif|, for the bitmap files of the picture
symbols. 
 
%%page page_215                                                  <<<---3
 
