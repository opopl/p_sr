% vim: keymap=russian-jcukenwin
%%beginhead 
 
%%file 20_11_2020.fb.semesjuk_ivan.1.kvartira_dva_remonty
%%parent 20_11_2020
 
%%url https://www.facebook.com/ivan.semesyuk/posts/3831721826878356
 
%%author Семесюк,Іван
%%author_id semesjuk_ivan
%%author_url 
 
%%tags 
%%title Чи можна в одній квартирі зробити два принципово різні ремонти водночас?
 
%%endhead 
 
\subsection{Чи можна в одній квартирі зробити два принципово різні ремонти водночас?}
\label{sec:20_11_2020.fb.semesjuk_ivan.1.kvartira_dva_remonty}
\Purl{https://www.facebook.com/ivan.semesyuk/posts/3831721826878356}
\ifcmt
	author_begin
   author_id semesjuk_ivan
	author_end
\fi

Чи можна в одній квартирі зробити два принципово різні ремонти водночас? З
різним плануванням, різним призначенням одних й тих самих приміщень, різним
освітленням і матеріалами водночас?

Ні. Або заходить одна бригада будівельників з кресленнями і дизайн-проєктом,
або інша. Всі це розуміють, і саме так діють тоді, коли справа стосується їхніх
особистих інтересів. 

Чому тоді, питаю я вас, в питанні розбудови культурного поля в державі знову на
голубом глазу обговорюється кієвлянін Булгаков, і що ж саме сказав бик і
русскій продюсер з укрпаспортом по франшизі Юрій Бардаш, та й взагалі сама
можливість діалогу між різними будівельними бригадами на порозі однієї
квартири?

Діалог закінчився в 2014 році, Булгаков-Бардаш - це інша, при чому очевидно
ворожа щодо нас цивілізація. Лише по державному скудоумію Бардаш має
український паспорт, це й вся його українськість. Українськість усіх цих пєвіц
Полякових, і навіть нині обраного Президента, настільки неочевидна, що її нема.

Коли ж ви нарешті вдуплите, що маєте справу тупо з русскімі які плюють вам в
пику? Коли припините взагалі називати їх українськими письменниками і
українськими продюсерами? Це якесь божевілля.

Усі ці діалоги для них є лише інструментами війни, ви для них просто
випадковість, тимчасове помутніння яке скоро минеться. Ви для них існуєте лише
тому, що з вами загралися, не розчавили вчасно. Ви для них тупо помилка, якої
не мало би бути і яку треба виправити, щоби ви не заважали їм нормально жить і
работать в їхньому русском культурном полі. Вони щиро вважають, що це саме вони
у себе вдома, а українці тут – це просто якісь сумашедші дурачкі, пліснява, що
завелася від сирості. І вони ваш просушать, якщо ви й далі будете бавитися в
діалоги.

Вони у будь-яку мить готові врубити комісарський режим і убити вас фізично, бо
ви какойто странний, і зачємто по-украінскі размовляєтє как дурачок с сєла.
Так, простой музичний продюсер готовий це зробити, щойно матиме таку
можливість. Врешті, часто прямими нащадками червоних комісарів і русскіх
колоністів вони і є. Само по собі це не є злочином, я це знаю по собі, але
злочинними є тихі сімейні традиції, де саме такий погляд на вас с вашей
долбаной Украіной й культивується у них через покоління.

Очнітєсь же, васі.

\subsubsection{Комментарии}
317 коментарів 96 поширень

\paragraph{Іван Семесюк}

Бачу люди плутають поняття \enquote{водночас} і \enquote{паралельно}. Здавалося б, дорослі
кмітливі громадяни, а не розрізняють. Феноменально. Ти не можеш поставити дві
шафи на одне місце, можеш лише поряд. Оце останнє у нас і відбувається, але
вічно тривати так не може, хата не рєзінова

\paragraph{Іван Семесюк}

Цікаві взагалі техніки використовують, ідеально маніпулятивні. Коли кажеш їм
"йдіть но нахуй, заїбали", то вони лякаюсь нас культурною самоізоляцією без
їхньої дорогоцінної присутності, наче блять у світі ніхуя крім коцапні просто й
нема. Або коли їх починають заслужено притісняти, то закликають не уподоблятися
їм. Не треба спротиву, навіщо це насіліє, ви що, хочете бути такими ж хуйовими
як ми? Будьте вищими за нас, нє смикайтеся.

\paragraph{Borys Krimer}
Якщо б в квартирі жило не 1-4 людини, а 40 млн., боюсь, з будівельниками й
дизайн-проектом була б така ж історія))

\paragraph{Олександр Янковський}
Дякую, дядьку. Сьогодні я дізнався про існування бардаша.

\paragraph{Андрій Назаренко}
У Франківську прямо в дану секунду барабанник барабанить на центральному
майдані барабанну партію з Бі-2, пісня «Сєрєбро», гучно увімкнувши її. Це тіпа
захід України, дофіга патріотичний націоналістичний. 

\paragraph{Віктор Брильов}

Питання, чи можемо ми чогось навчитись в тієї ворожої цивілізації?
Наполегливості? Скупчення навколо лідера в скрутні часи? Відверта та деколи
нахабна реклама власної країни? Має ж бути там щось корисне що ми зможемо
згідно принципу \enquote{чужого навчайтесь; свого не цурайтесь} використати

\paragraph{Yevheniia Hurtova}
Мене теж ця тема вкурвлює. Особливо люблю коли вони «с придиханием»
розповідають «а дедушка в мене був полковник КГБ». 

\paragraph{Roman Dzyuba}
Нарешті чіткий публічний дискурс по відношенню до цих комісарських унучків.

\paragraph{Marianna Malina}
Вороги вони. І це треба вже всім українцям розуміти і не бавитися в
толерантність. Бо або ми їх, або вони нас.

\paragraph{Maksym Vichikov}

З тієї ж серії про Віктюка:
\begin{itemize}
  \item 2003 — Заслужений діяч мистецтв Росії.
  \item 2003 — Орден «За інтелектуальну відвагу».
  \item 2006 — Народний артист України.
  \item 2009 — Народний артист Росії.
\end{itemize}
Обмазувалося ним тут пів фейсбука

\paragraph{Сергій Мещик}

Їхав в маршрутці Київ-Умань, задні сидіння, поруч дівчина з мамою та дочкою,
україномовні. Дочка спочатку робила трелі губами, віршики читала, а потім як
заоре свинособачою \enquote{Я каралєва ночі сєводня точна!}. І тут, Валєр, я офігів, бо
Полякову теж від кацапки відрізняє тільки паспорт, а преференції у неї вищі, бо
типу своя, а деструкція та ж сама, вражає дітей, які ще не здатні зрозуміти де
добро, де зло. І мамаша з бабцьою на похуях. Сум.

