% vim: keymap=russian-jcukenwin
%%beginhead 
 
%%file 25_10_2021.fb.fb_group.story_kiev_ua.1.vitachev
%%parent 25_10_2021
 
%%url https://www.facebook.com/groups/story.kiev.ua/posts/1782842491912544
 
%%author_id fb_group.story_kiev_ua,dubinina_oksana
%%date 
 
%%tags foto,krasota,priroda,ukraina,vitachev.kiev_obl.ukraina
%%title Любуемся пейзажами Киевской области - Витачев
 
%%endhead 
 
\subsection{Любуемся пейзажами Киевской области - Витачев}
\label{sec:25_10_2021.fb.fb_group.story_kiev_ua.1.vitachev}
 
\Purl{https://www.facebook.com/groups/story.kiev.ua/posts/1782842491912544}
\ifcmt
 author_begin
   author_id fb_group.story_kiev_ua,dubinina_oksana
 author_end
\fi

Любуемся пейзажами Киевской области***

Село Витачо́в, еще многие называют его «Витачев» (українською мовою – Витачів).
Обуховский район.

(Кстати «Витачев» ближе всего к имени древнего предка села - города Витичев)

\raggedcolumns
\begin{multicols}{2} % {
\setlength{\parindent}{0pt}

\ii{25_10_2021.fb.fb_group.story_kiev_ua.1.vitachev.pic.1}
\ii{25_10_2021.fb.fb_group.story_kiev_ua.1.vitachev.pic.1.cmt}

\ii{25_10_2021.fb.fb_group.story_kiev_ua.1.vitachev.pic.2}
\ii{25_10_2021.fb.fb_group.story_kiev_ua.1.vitachev.pic.2.cmt}

\ii{25_10_2021.fb.fb_group.story_kiev_ua.1.vitachev.pic.3}
\ii{25_10_2021.fb.fb_group.story_kiev_ua.1.vitachev.pic.3.cmt}

\ii{25_10_2021.fb.fb_group.story_kiev_ua.1.vitachev.pic.4}
\ii{25_10_2021.fb.fb_group.story_kiev_ua.1.vitachev.pic.4.cmt}

\ii{25_10_2021.fb.fb_group.story_kiev_ua.1.vitachev.pic.5}

\end{multicols} % }

\raggedcolumns
\begin{multicols}{3} % {
\setlength{\parindent}{0pt}

\ii{25_10_2021.fb.fb_group.story_kiev_ua.1.vitachev.pic.6}
\ii{25_10_2021.fb.fb_group.story_kiev_ua.1.vitachev.pic.6.cmt}

\ii{25_10_2021.fb.fb_group.story_kiev_ua.1.vitachev.pic.7}
\ii{25_10_2021.fb.fb_group.story_kiev_ua.1.vitachev.pic.7.cmt}

% Памятный знак о православном храме
\ii{25_10_2021.fb.fb_group.story_kiev_ua.1.vitachev.pic.8}

\ii{25_10_2021.fb.fb_group.story_kiev_ua.1.vitachev.pic.9}

% Памятный знак о том, что здесь был древний город-порт Витичев.
\ii{25_10_2021.fb.fb_group.story_kiev_ua.1.vitachev.pic.10}

%\ii{25_10_2021.fb.fb_group.story_kiev_ua.1.vitachev.pic.11}

\ii{25_10_2021.fb.fb_group.story_kiev_ua.1.vitachev.pic.12}
\ii{25_10_2021.fb.fb_group.story_kiev_ua.1.vitachev.pic.12.cmt}

\end{multicols} % }

Местные жители древнего Витачова кличут пейзажные площадки над Днепром милым
словом «красовид». И действительно, здесь очень красивые виды!

Панорамные склоны в Витачове справедливо считаются одними из самых лучших на
днепровских берегах. Именно поэтому сюда тянутся многие, а по выходным
наблюдается паломничество любителей природных пейзажей. 

\ii{25_10_2021.fb.fb_group.story_kiev_ua.1.vitachev.pic.13}

Мне напомнили еще такое слово «левада», редко сейчас употребляемое, но очень
подходящее к этой местности у берега Днепра. Произносишь его и мелодично
представляются просторы у воды. 

Здесь мощная и особенная  энергетика и это  сложно переоценить. 

Село Витачов имеет тысячелетнюю интереснейшую историю и даже свои герб и флаг.

Постараюсь в нескольких предложениях изложить самые важные сведения.

Во времена Киевской Руси здесь был город-крепость Витичев, основанный в начале
5 века. 

\raggedcolumns
\begin{multicols}{3} % {
\setlength{\parindent}{0pt}

\ii{25_10_2021.fb.fb_group.story_kiev_ua.1.vitachev.pic.14}
\ii{25_10_2021.fb.fb_group.story_kiev_ua.1.vitachev.pic.14.cmt}

\ii{25_10_2021.fb.fb_group.story_kiev_ua.1.vitachev.pic.16}
\ii{25_10_2021.fb.fb_group.story_kiev_ua.1.vitachev.pic.16.cmt}

%\ii{25_10_2021.fb.fb_group.story_kiev_ua.1.vitachev.pic.17}
\ii{25_10_2021.fb.fb_group.story_kiev_ua.1.vitachev.pic.19}
\ii{25_10_2021.fb.fb_group.story_kiev_ua.1.vitachev.pic.21}

\end{multicols} % }

Интересно, что здесь была сигнальная башня, с неё огнём оповещали  Киев о
приближении врагов. 

Первые упоминания о Витичеве были в 949 году в трудах византийского императора
Константина Багрянородного. 

В «Повести временных лет» упоминается, что  в Уветичах  (Витичеве) в 1100 году
состоялся съезд князей во времена междоусобных войн на Руси.

\ii{25_10_2021.fb.fb_group.story_kiev_ua.1.vitachev.pic.15}

Есть интересная картина  конца 19 века художника Сергея Иванова  «Русские
князья заключают мир в Уветичах».  Когда нужно было решить какой-то особо
острый вопрос, князья с ближними боярами встречались и договаривались, а
дружины княжеские ожидали недалече.

\ii{25_10_2021.fb.fb_group.story_kiev_ua.1.vitachev.pic.20}

Мне кажется на этой картине можно разглядеть узнаваемые пейзажи Витачова. 

Сейчас в селе Витачов всего лишь около 7 сотен жителей, но для самых
любознательных существует множество сведений былой славной истории града
Витичев.

В окрестностях села можно найти остатки древних валов и  другие неожиданные
открытия для себя.

* уж простите за большое количество фото, мне кажется каждое особое, хочется
рассматривать))

\ii{25_10_2021.fb.fb_group.story_kiev_ua.1.vitachev.cmt}
