% vim: keymap=russian-jcukenwin
%%beginhead 
 
%%file 30_04_2020.fb.fb_group.story_kiev_ua.1.mozaika_kiev.cmt
%%parent 30_04_2020.fb.fb_group.story_kiev_ua.1.mozaika_kiev
 
%%url 
 
%%author_id 
%%date 
 
%%tags 
%%title 
 
%%endhead 
\subsubsection{Коментарі}

\begin{itemize} % {
\iusr{Kliuzko Victoria}
Первую видно со двора 17 или 19 дома по ул. Репина, сейчас Терещенковской

\begin{itemize} % {
\iusr{Ирина Петрова}
Так, звісно!

\iusr{Kliuzko Victoria}
\textbf{Ирина Петрова} у меня там бабушка жила... иногда заезжаю. Постою во дворе и дальше поеду...

\iusr{Ирина Петрова}
\textbf{Kliuzko Victoria} так, будинки дитинства - то магніти...
\end{itemize} % }

\iusr{Ирина Жуковская}

Ирина, спасибо за чудесную публикацию! Только, если позволите, маленькое
уточнение: соавтора Ады Рыбачук зовут Владимир Мельниченко.

\begin{itemize} % {
\iusr{Ирина Петрова}

Да, конечно, спасибо, сейчас исправлю. Пропустила, когда редактор телефона
написал то, что сам считает нужным. он живет своей загадочной жизнью и
подкладывает свинью не раз. Спасибо еще раз ! Извините меня. Это очень важно!

\iusr{Ирина Жуковская}
\textbf{Ирина Петрова} Да, Т9 - это ужас бесконечный! Иногда такое "исправляет", что во сне не приснится!))))))

\iusr{Ирина Петрова}
\textbf{Ирина Жуковская} 

его можно выключить, конечно, но, иногда он помогает быстро набрать текст. Надо
просто в ответственных текстах самой проверять, но, я когда увлекаюсь темой,
пропускаю.(((

\end{itemize} % }

\iusr{Таня Сидорова}

Я бачила стіну пам'яті, вона мене вразила, хоча здається до кінця ця робота не
була виконана. Досі не розумію - НАВІЩО її залили. Ні чого крамольного там не
було.

\begin{itemize} % {
\iusr{Ирина Петрова}

зараз вже важко сказати, що так не глянулось тому або тим чиновникам...мабуть,
їх вже й на світы немає.. але, може ще живі. Дуже хотілось би, щоб автор встиг
завершити реставрацію своєї роботи!

\iusr{Нина Светличная}
\textbf{Ирина Петрова} 

Я також бачила ту СТІНУ! Це КОСМОС! Така енергетика, така сила! Жодне
фото, жодне відео не передає і долі того магнетизму.. Я чудово розумію ЧОМУ
влада злякалась її.

\iusr{Ирина Петрова}
\textbf{Нина Светличная} 

чесно кажучи, я тієї стіни сама наживо не бачила. Так сталося, що тоді ми ще не
бували на кладовищі, декілька разів приходили до старих родичів, але вони біля
центральної алеї. Вперше почула про це, коли вже той акт вандалізму був
звершений... тільки фото бачила.

\iusr{Нина Светличная}
\textbf{Ирина Петрова} 

Ми потрапили випадково. Жили на Батиєвій горі, пішли бродити районом. На
кладовищі ще був будівельний майданчик крематорію. Ніким не охоронявся. І ми б
пішли вже геть, але випадково вирішили спуститись з гори і ......АБАЛДЕЛИ!
Навіть не зрозуміли до чого тут ЦЕ! Чому не десь в місті щоб можна було
дивитись? Шкода, що не було фотоаппарата. Розказали друзям - ніхто нічого не
чув про ту стіну. дехто над нами посміявся - мовляв - ну, ви і знайшли де
гуляти. Мабуть, не випадково ноги туди понесли в той день. Відновити Стіну
хотіли ще на початку "нульових". Але все вперлось в одне - немає грошей.

\iusr{Ирина Петрова}
\textbf{Нина Светличная} на жаль... одвічне фінансове питання... мецената якось не знайшлося...

\iusr{Людмила СУХАНЮК}
\textbf{Ирина Петрова}

\ifcmt
  ig https://scontent-frt3-1.xx.fbcdn.net/v/t1.6435-9/95283805_3418792701481962_7807316839712686080_n.jpg?_nc_cat=104&ccb=1-5&_nc_sid=dbeb18&_nc_ohc=GoIU4EcwkU0AX-RaURZ&_nc_ht=scontent-frt3-1.xx&oh=f802353457827004336491c211313a73&oe=61B489DE
  @width 0.3

	ig https://scontent-frx5-1.xx.fbcdn.net/v/t1.6435-9/95308651_3418793148148584_5046400184104255488_n.jpg?_nc_cat=105&ccb=1-5&_nc_sid=dbeb18&_nc_ohc=_Wg4CNBQosgAX9jmkVY&_nc_ht=scontent-frx5-1.xx&oh=e61305b3edab5248b6bde50ef2a2832d&oe=61B73076
  @width 0.3

	ig https://scontent-frt3-1.xx.fbcdn.net/v/t1.6435-9/95218150_3418793594815206_4570175571828408320_n.jpg?_nc_cat=106&ccb=1-5&_nc_sid=dbeb18&_nc_ohc=-ZQQhzWGRawAX-GzF8M&_nc_ht=scontent-frt3-1.xx&oh=08293aac1775afb8289a97f597eff53d&oe=61B43693
  @width 0.3
\fi

\iusr{Людмила СУХАНЮК}
\textbf{Ирина Петрова}

\ifcmt
	ig https://scontent-frx5-1.xx.fbcdn.net/v/t1.6435-9/95865192_3418794134815152_264421525594046464_n.jpg?_nc_cat=105&ccb=1-5&_nc_sid=dbeb18&_nc_ohc=U_ct-_eLA7IAX9x-rPX&_nc_ht=scontent-frx5-1.xx&oh=2122e8cf12539e1b03059ad599a4d706&oe=61B53454
  @width 0.3

	ig https://scontent-frt3-2.xx.fbcdn.net/v/t1.6435-9/95127137_3418794824815083_6836862150183485440_n.jpg?_nc_cat=101&ccb=1-5&_nc_sid=dbeb18&_nc_ohc=ZiKyUHeqcAQAX_SWeiL&_nc_ht=scontent-frt3-2.xx&oh=d2e060251ac1b34584cc43c1709af78e&oe=61B4E08F
  @width 0.3
\fi

\iusr{Людмила СУХАНЮК}
\textbf{Ирина Петрова} 

Цей шедевр мистецтва міг би прославляти нашу краіну на міжнародному рівні! Але
безбожники поглумилися над витвором мистецтва! Тепер іх душі неспокійні!

\ifcmt
  ig https://scontent-frt3-1.xx.fbcdn.net/v/t1.6435-9/95561154_3418803028147596_3525831454644764672_n.jpg?_nc_cat=102&ccb=1-5&_nc_sid=dbeb18&_nc_ohc=WKjU7VkemegAX9Oook3&_nc_ht=scontent-frt3-1.xx&oh=07877499c4582718bda71facdd8b68e5&oe=61B632CE
  @width 0.4
\fi

\iusr{Таня Сидорова}
\textbf{Людмила СУХАНЮК} 

спасибо за снимки, так хорошо я не помню, мне кажется что все изображения
отображали сотворение мира, жизнь на земле. Я только помню, что оторваться от
рассматривания изображений долго не могла. А вот от куда я узнала о стене, тоже
не помню, но пришла туда специально посмотреть.

\iusr{Maryna Chemerys}
\textbf{Tanya Sidorova} І я бачила - в студентські роки, дуже вразила, це було щось не бачене до того!

\iusr{Ирина Петрова}
\textbf{Людмила СУХАНЮК} 

дякую за фото! Епічне полотно... вдивляюсь, не можу зрозуміти, де і що здалося
крамольним...((( Якась маячня - що! тут може бути? Дух захопило... дякую!

\end{itemize} % }

\iusr{Тетяна Кутишенко}

Если зайти во двор дома №17 на Терещенковской то можно будет увидеть, полностью
или частично «Украинскую Мадонну», «Сердце Данко», «Портрет дочери», «Данко»,
«Киевские каштаны», «Журавля». Кроме того есть еще несколько мозаик которые
можно увидеть только из дворика музея. Это «Мамай» и «Голубь».

И все это работы Василия Фёдоровича Овчинникова, директора музея Восточного и
Западного искусства. На этой должности он проработал 42 года, до самой смерти в
1978 году.

\begin{itemize} % {
\iusr{Ирина Петрова}

И еще там есть цветы, насколько помню, похоже на розу и гвоздику (красные).
Светлая память Василию Федоровичу за все его труды, благодаря которым мы имеем
сейчас прекрасный музей. (Данко меня всегда пугал немного, такой похож на
цыгана, а их мы в детстве боялись...) мы долго не знали, что это Данко. Потом
думали, что это зарисовка к композиции дружбы народов, т.е. житель Африки)))
тогда как-то никто не рассказывал.

\end{itemize} % }

\iusr{Ivan Bratus}

Хочеться вірити, що колись поновлять... Про ці події є документальний фільм
"Стіна": пошук в ютуб за словами "Стена Режиссер и оператор: Израиль
Гольдштейн".

\begin{itemize} % {
\iusr{Ирина Петрова}
\textbf{Ivan Bratus} дякую, я не знала про фільм. Неодмінно пошукаю та посмотрю!
\end{itemize} % }

\iusr{Елена Никитина}

Пару лет назад я создавала петицию в КГГА о необходимости воссоздания Стены
Памяти. За нее проголосовало чуть больше 600 киевлян...

\begin{itemize} % {
\iusr{татьяна гордиенко}
\textbf{Елена Никитина} давайте еще раз петицию напише пожалуйста

\iusr{Елена Никитина}
\textbf{татьяна гордиенко} После карантина. Я сначала должна побывать там, чтобы должным образом зарядиться и написать новый проникновенный текст, а то прежнего уже не найду.

\iusr{татьяна гордиенко}
\textbf{Елена Никитина} ок ждем
\end{itemize} % }

\iusr{Ирина Петрова}

Це покажчик наявності, небайдужості та активності киян((( корінних, які ще
пам'ятають, що було там тоді... на жаль, в силу віку, вже багато з них не
володіють інтернетом, а молоді на це... що зробиш...

\iusr{Nina NinaNina}

Перше фото зроблено у дворі музею східного та західного мистецтва (нині це
національний музей мистецтв імені Богдана та Варвари Ханенко,
Терещенківська,15-17, побачила вперше у семидесятих - випадково визирнула у
вікно на другому поверсі, бо у двір тоді потрапити не було можливості, потім
прочитала книгу про монументаліста, директора музею, який у шестидесятих роках,
сам купуючи матеріали для мозаїки, здійснював свою мрію - прикрасити київські
будівлі чудовими малюнками, і саме Українська мадонна стала для мене символом
мрії Василя Федоровича Овчинникова. Наскільки знаю, його родина преїхала в
Америку, а дружина померла.

\begin{itemize} % {
\iusr{Ирина Петрова}
\textbf{Nina NinaNina} так, так воно й було. Дякую.
\end{itemize} % }

\iusr{Nina NinaNina}

мозаїчні панно з оленями постали у переході біля майдану восени в далекому 1968
році, спочатку там було закрито частину переходу, і було страшенно цікаво - що
ж буде. Птахи та олені з чудернацькими рогами мене просто зачарували тоді, весь
час хотілось помацати. Так само, як жар-птиця біля фотомагазину неподалік
палацу Україна (її зараз весь час затуляють припарковані машини) і мозаїка
улюбленої Ольги Рапай на бульварі Шевченка - вперше бачила її роботи на
художній виставці, і лише потім на вулицях міста. Звісно, хто автор панно (
Александр Ворона), дізналась набагато пізніше, як і те, що мозаїку готували у
керамічній майстерні Редьки, та про те, що ескізи були аврально швидко
зроблені, бо до свят залишалось лише 4 місяці..Цікаво знати, як склалась у
нього доля...

\begin{itemize} % {
\iusr{Ирина Петрова}
Дякую, пані Ніно, за розлогі коментарі!
\end{itemize} % }

\iusr{Наталия Трач}

я бачила Стіну на Байковому вона ще була сірою, і день був сірим та холодним,
листопадовим враження просто приголомшливе

\iusr{Виктор Бухтияров}

Найкращі мозаїки Рибачук та Мельниченка гарно зберіглися в Київському Палаці
дітей та юнацтва.

\end{itemize} % }
