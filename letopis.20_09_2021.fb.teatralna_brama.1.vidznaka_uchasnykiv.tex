%%beginhead 
 
%%file 20_09_2021.fb.teatralna_brama.1.vidznaka_uchasnykiv
%%parent 20_09_2021
 
%%url https://www.facebook.com/TeatralnaBrama/posts/pfbid02mqRijDZgvU6MVm6HgNearJCVD9M54qqvG8WGS9Q92Zwf7zD3MzLSJfFj7JmPQebzl
 
%%author_id teatralna_brama
%%date 20_09_2021
 
%%tags 
%%title Відзнака учасників
 
%%endhead 

\subsection{Відзнака учасників}
\label{sec:20_09_2021.fb.teatralna_brama.1.vidznaka_uchasnykiv}

\Purl{https://www.facebook.com/TeatralnaBrama/posts/pfbid02mqRijDZgvU6MVm6HgNearJCVD9M54qqvG8WGS9Q92Zwf7zD3MzLSJfFj7JmPQebzl}
\ifcmt
 author_begin
   author_id teatralna_brama
 author_end
\fi

Експертне журі

ІІ обласного відкритого фестивалю театрального мистецтва

ТЕАТРАЛЬНА БРАМА – 2021

відзначило учасників конкурсної програми фестивалю в наступних номінаціях:

🎭  ЗА ВИСВІТЛЕННЯ АКТУАЛЬНИХ СОЦІАЛЬНИХ ПРОБЛЕМ У ВИСТАВІ \enquote{ІГРИ НА ЗАДНЬОМУ
ПОДВІР'Ї} Е. Мазія

– Національний академічний театр російської драми імені Лесі Українки;

🎭  ЗА МАЙСТЕРНЕ ВТІЛЕННЯ СОЦІАЛЬНО-ПСИХОЛОГІЧ\hyp{}НОЇ ДРАМИ ВОЛОДИМИРА ВИННИЧЕНКА
\enquote{ЗАКОН}

– Кіровоградський академічний обласний український музично-драматичний театр
імені М. Л. Кропивницького;

🎭 ЗА ЦІЛІСНІСТЬ ВІЗУАЛЬНОГО ОБРАЗУ ВИСТАВИ \enquote{ЕЗОП} Г. ФІГЕЙРЕДО

– Дніпропетровський академічний обласний український молодіжний театр;

🎭 ЗА ВИСОКОПРОФЕСІЙНИЙ АКТОРСЬКИЙ АНСАМБЛЬ 

– Миколаївський національний академічний український театр драми та музичної
комедії, вистава \enquote{ІМ'Я} М. Делапорта, А. де ла Пательєра;

– Херсонський обласний академічний музично-драматичний театр імені Миколи
Куліша, вистава \enquote{БАБА ПРІСЯ АБО НА ПОЧАТКУ І НАПРИКІНЦІ ЧАСІВ} П. Ар'є;

🎭 ЗА НОВАТОРСТВО І ПОШУК ОРИГІНАЛЬНОЇ ФОРМИ

– Катерину Богданову, режисерку-постановницю вистави \enquote{КОЛИ ПОВЕРТАЄТЬСЯ ДОЩ}
Н. Нежданої в Миколаївському академічному художному російському драматичному
театрі;

🎭 ЗА ВИСОКУ КУЛЬТУРУ СЦЕНІЧНОЇ МОВИ

– Наталію Головіну за роль Тьоті Шури у виставі Харківського державного
академічного українського драматичного театру імені Т. Г. Шевченка \enquote{ХЛІБНЕ
ПЕРЕМИР'Я} С. Жадана;

🎭 ЗА ТВОРЧИЙ ПОТЕНЦІАЛ В РОБОТІ НАД ОБРАЗОМ

– Олександру Барсток за роль Дворі Махнес у виставі Національного академічного
театру російської драми імені Лесі Українки \enquote{ІГРИ НА ЗАДНЬОМУ ПОДВІР'Ї} Е.
Мазія;

– Сергія Пакулаєва за роль Толіка у виставі Харківського державного
академічного українського драматичного театру імені Т. Г. Шевченка \enquote{ХЛІБНЕ
ПЕРЕМИР'Я} С. Жадана;

– Анну Бабенко за роль Аліни у виставі Миколаївського академічного художнього
російського драматичного театру \enquote{КОЛИ ПОВЕРТАЄТЬСЯ ДОЩ} Н. Нежданої;

🎭 ЗА ВИСОКОХУДОЖНЄ ВИКОНАННЯ РОЛІ 

– Дар'ю Недавню за роль Жанни д'Арк в рок-опері Донецького академічного
обласного драматичного театру (м. Маріуполь) \enquote{БІЛА ВОРОНА} Г. Татарченка, Ю.
Рибчинського; 

🎭 ЗА ВИСОКУ МАЙСТЕРНІСТЬ У ВИКОНАННІ МОНОВИСТАВИ

– заслуженого артиста України Сергія Михайловського, вистава  Херсонського
обласного академічного музично-драматичного театру імені Миколи Куліша \enquote{ЮДА} за
Л. Українкою;

– Олену Дудич, вистава Львівського академічного обласного музично-драматичного
театру імені Юрія Дрогобича \enquote{МАРГАРИТА Й АБУЛЬФАЗ} за С. Алексієвич;

🎭 ЗА ВИСОКОХУДОЖНЄ СТВОРЕННЯ ОБРАЗУ

– народну артистку України Олену Галл-Савальську за роль Баби Прісі у виставі
Херсонського обласного академічного музично-драматичного театру імені Миколи
Куліша \enquote{БАБА ПРІСЯ АБО НА ПОЧАТКУ І НАПРИКІНЦІ ЧАСІВ} П. Ар'є;

– Віру Шевцову за роль Марусі Чурай у виставі Донецького академічного обласного
драматичного театру (м. Маріуполь) \enquote{МАРУСЯ} Л. Костенко;

🎭 ГРАН-ПРІ 

– вистава Донецького академічного обласного драматичного театру (м. Маріуполь)
\enquote{МАРУСЯ} за Л. Костенко. 

ВІТАЄМО!!🥳💐🎉🤩

фото Лев Сандалов
