% vim: keymap=russian-jcukenwin
%%beginhead 
 
%%file slova.zoloto
%%parent slova
 
%%url 
 
%%author_id 
%%date 
 
%%tags 
%%title 
 
%%endhead 
\chapter{Золото}

%%%cit
%%%cit_head
%%%cit_pic
\ifcmt
tab_begin cols=3
  pic https://strana.news/img/forall/u/0/34/b62c7a55642442d8d3406a1abbb8f2c2.jpg
	pic https://strana.news/img/forall/u/0/34/ddfb5d17538f82f9ef91838c5046102f.jpg
	pic https://strana.news/img/forall/u/0/34/2cbf05cc95b761af2a9790a69ec9f5b1.jpg
tab_end
\fi
%%%cit_text
6 октября Апелляционный суд Амстердама объявил решение о судьбе "скифского
\emph{золота}". Древнюю коллекцию постановили отправить в Украину.  Суд
посчитал, что хотя объекты происходят из Крыма, они являются частью наследия
Украины после того, как она стала независимой в 1991 году.  После вынесения
решения спорные экспонаты останутся в Нидерландах еще минимум на три месяца, в
течение которых возможно обжалование этого решения в суде высшей инстанции.
После решения голландского суда Владимир Зеленский заявил, что вслед за
\emph{"скифским золотом"} Украина вернет и Крым. Такая запись появилась в его
Twitter-аккаунте
%%%cit_comment
%%%cit_title
\citTitle{Впереди новые тяжбы. Как Украина выиграла суд за \enquote{скифское золото} и почему его пока не отдают}, 
Оксана Малахова, strana.news, 26.10.2021
%%%endcit

%%%cit
%%%cit_head
%%%cit_pic
%%%cit_text
Анна плакала і не хотіла їхати, брати-імператори, мабуть, умовляли її, чи як
воно там ведеться, Добриня розкладав перед священними очима дарунки князеві:
мед, віск, хутра, зброю і льон, що дорожчий за \emph{золото}. А що їм ті
дарунки?  Імператорська сестра знай розшукувалася: "Ліпше мені вмерти тут, ніж
іти в той полон". Була вже перезріла, мала літ двадцять і п’ять, могла б і
вмирати собі в своєму Царгороді, нащо така князеві Володимиру?
Брати-імператори Василій і Костянтин переконували сестру святістю подвигу, коли
вона всю безмежну землю Руську приведе до бога і водночас вибавить свою імперію
від лютої раті, від усіх бід, яких завдавала і грозиться завдати ще їм Русь
%%%cit_comment
%%%cit_title
\citTitle{Тисячолітній Миколай}, Павло Загребельний 
%%%endcit

%%%cit
%%%cit_head
%%%cit_pic
%%%cit_text
І зовсім не тому, що П'єр Гренгуар боявся чи зневажав пана кардинала. Йому не
були властиві ні така малодушність, ні така зарозумілість. Справжній еклектик,
як висловлюються сьогодні, Гренгуар належав до тих стійких і благородних умів,
урівноважених і спокійних, які завжди вміли у всьому додержуватися \emph{золотої}
середини, stare in dimido rerum, i, сповнені здорового глузду та ліберальної
філософії, водночас віддавали належне й кардиналам. Дорогоцінне, невмируще
плем'я філософів! Здається, що мудрість, немов нова Аріадна, дала їм клубок
ниток, і вони розмотуючи його від створення світу, проходять крізь лабіринт
усіх справ людських. Вони трапляються в усіх епохах, завжди однакові, тобто
завжди відповідають епосі. Якщо поминути П'єра Гренгуара, який, коли б нам
пощастило зобразити його так, як він на це заслуговує, міг би бути їхнім
представником у п'ятнадцятому столітті, то безперечним є те, що саме їхній дух
запалював старого дю Бреля, коли він у шістнадцятому столітті писав ці наївно
величні, гідні всіх віків слова: "Я парижанин родом і па-ризіанин словом, бо
грецькою мовою "parrhisia" означає "свобода слова", якої я додержувався навіть
по відношенню до монсеньйорів кардиналів, до дядька й брата монсеньйора принца
Конті, але все ж таки з пошаною до їхнього високого стану і не ображаючи нікого
з їхнього почту, а це вже немала заслуга"
%%%cit_comment
%%%cit_title
\citTitle{Собор Паризької Богоматері}, Віктор Гюго
%%%endcit
