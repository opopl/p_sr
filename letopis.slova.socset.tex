% vim: keymap=russian-jcukenwin
%%beginhead 
 
%%file slova.socset
%%parent slova
 
%%url 
 
%%author 
%%author_id 
%%author_url 
 
%%tags 
%%title 
 
%%endhead 
\chapter{Соцсеть (социальная сеть)}
\label{sec:slova.socset}

%%%cit
%%%cit_head
%%%cit_pic
\ifcmt
tab_begin cols=2
  width 0.2
  pic https://strana.ua/img/forall/u/0/36/208574553_3043720015904327_4166498339048538930_n.jpg

  pic https://strana.ua/img/forall/u/0/36/2021-06-30_10h07_14.png
  width 0.3
tab_end
\fi
%%%cit_text
Стоит ли говорить, что после этого \emph{соцсети} Баранского перешерстили вдоль и
поперек и собрали все его \enquote{неправильные} размышления.  В своих сторис
Баранский довольно давно выражал вполне определенную позицию в отношении
националистов.  \enquote{Зачем вам Моргенштерн, если есть сельская
\enquote{Плине кача}?} - интересовался он в Instagram
%%%cit_comment
%%%cit_title
\citTitle{Фоззи уволил топ-менеджера из-за Басты и наезда Стерненко}, 
Екатерина Терехова, strana.ua, 30.06.2021
%%%endcit

%%%cit
%%%cit_head
%%%cit_pic
%%%cit_text
И у \emph{соцсетей} нет никакого стимула соблюдать эти стандарты, потому что
они зарабатывают на сомнительном контенте. Получается, что остается только
стороннее регулирование?  Есть альтернативные сервисы, которые пытаются отойти
от пассивной модели потребления информации. По такой модели работает
телевидение: зритель просто смотрит то, что показывают. Поначалу интернет
работал по другому принципу: пользователь серфил по сети, сам выбирал, на какие
ссылки кликать и какие блоги читать. Но мы снова вернулись к пассивной модели:
просто бесконечно скроллим новостную ленту.  Даже Facebook когда-то отличался
активным потреблением: пользователи сами ходили по группам, на страницы к
друзьям. А сейчас все сосредоточено на бесконечной ленте новостей
%%%cit_comment
%%%cit_title
\citTitle{«У людей должен быть выбор» Как соцсети взламывают мозг человека и кто на самом деле их контролирует?: Coцсети: Интернет и СМИ: Lenta.ru}, 
Федор Тимофеев, lenta.ru, 28.10.2021
%%%endcit
