% vim: keymap=russian-jcukenwin
%%beginhead 
 
%%file 2007
%%parent articles
%%tags russia,moscow,alexander palii
%%url https://www.pravda.com.ua/articles/2007/02/6/3205873/
%%author oleksandr palii
 
%%endhead 

\subsection{Суперечка Русі з Московією закінчиться перемогою України}
\label{sec:2007}
\url{https://www.pravda.com.ua/articles/2007/02/6/3205873/}

\index[authors.rus]{Палій!Олександр}
\index[rus]{Русь}

Олександр Палій, для УП --- Вівторок, 6 лютого 2007, 13:23

Всі народи у свій час проходили бум уваги до історії. І це природно, адже лише
розуміння свого коріння дає народу і державі міцний ґрунт під ногами.

Навіть екс-президент Кучма по-своєму долучився до цієї дискусії: він заявив, що
парламент повинен прийняти закон, "який забороняв би історикам в лапках, які
пишуть підручники, займатися словоблуддям".

Між тим, сьогодні перед українськими істориками стоїть питання не про її
переписування, а навпаки --- про очищення української історії від непідтверджених
фактів нашарувань брехні.

Цю брехню було цілеспрямовано нав'язано за умов тотального ідеологічного
терору, коли імперські історики могли говорити все, що схочуть, а за правдиве
слово про українську історію і навіть за просте її вивчення сотні українських
вчених поплатилися кар'єрою і самим життям.

Одна з ключових імперських містифікацій --- теза про те, що нинішня Росія нібито
має якесь відношення до історії Київської Русі, крім того, що окремі її
території колись були підконтрольні Києву.

За часів СРСР вперто замовчувався елементарний факт, що за часів Київської
Русі, Русь --- це виключно територія Центральної України, тобто сучасних
Київської, Чернігівської, Житомирської, Сумської, а також частин Вінницької,
Черкаської та Полтавської областей.

З кінця XII століття Руссю починає зватися ще й Західна Україна. Доказів цьому
така кількість, що радянська історіографія просто не знала, що з ними робити, і
видавала в друк "перекази" літописів, а не їхні оригінали.

Що ж стосується літописів, то в них читаємо наступне.

Під 1146 роком: "І Святослав, заплакавши, послав до Юрія в Суздаль, сказав:
"Брата мені Всеволода Бог узяв, а Ігоря Ізяслав схопив. Піди-но в Руську землю,
до Києва".

Коли вигнаний із Київщини Ростислав 1149 року приїжджає до свого батька Юрія
Довгорукого в Суздаль, він каже: "Чув я, що хоче тебе вся земля Руська і чорні
клобуки".

Згодом літописець додає скаргу Юрія Довгорукого: "Синовець мій Ізяслав, на мене
прийшовши, волость мою розорив і попалив, а іще й сина мого вигнав із Руської
землі і волость йому не дав, і мене соромом покрив".

\ifcmt
pic https://img.pravda.com/images/doc/u/s/us_Picture_file_path_4292.jpg
\fi

Після того, як Ізяслав Мстиславич вчергове вигнав Юрія Довгорукого з Києва,
літопис під 1151 роком повідомляє, що син Юрія Андрій Боголюбський "тим часом
випросив у отця піти наперед до Суздаля, кажучи: "Осе нам уже, отче, тут, в
Руській землі ні раті, ні чого іншого. Тож затепла підем".

Про черговий невдалий похід Юрія Довгорукого на Київ літопис під 1154 роком
повідомляє: "У тім же році рушив Юрій з ростовцями, і з суздальцями, і з усіма
дітьми в Русь. І стався мор серед коней в усьому війську його, якого ж не було
ніколи".

Після смерті Київського князя Ізяслава Мстиславича "тої ж зими (1154 року)
рушив був Юрій у Русь, почувши про смерть Ізяславову".

1174 року Суздальський князь Андрій Боголюбський, відповідаючи на звернення
князів Ростиславичів, сказав: "Пождіте трохи, я послав до братів своїх в Русь.
Як мені вість буде од них, тоді й дам відповідь".

Коли суздальського князя Андрія Боголюбського вбили змовники у заснованому ним
містечку Боголюбові під Суздалем, його придворний Кузьмин, згідно з літописом,
сказав: "Уже тебе, господине, пахолки твої не признають. Не так, як колись,
коли купець приходив із Цесарограда, і з інших країв, і з Руської землі..."

Після вбивства Андрія Боголюбського володимирські бояри (з
Володимира-на-Клязьмі) говорили: "Князь наш убитий, а дітей у нього немає,
синок його в Новгороді, а брати його в Русі".

Під 1175 роком літопис повідомляє: "Того ж року коли оба Ростиславичі сиділи на
княжінні у землі Ростовській, роздавали вони були посадництва руським отрокам.
А ті велику тяготу людям сим учинили продажами і вірами".

Таким чином, "руські отроки", приведені Ростиславичами з Київщини, чужі на
Ростовщині.

Під 1180 роком повідомляється про похід чернігівського князя Святослава
Всеволодовича на Суздаль і Рязань: "І тоді спішно приготувалися до бою у
війську Святославовім, і Всеволод Святославич уборзі примчав до руських полків
зі своїм полком. І тоді рязанські князі втекли, а інших вони побили".

Коли Святослав повертався із Суздалі, літопис повідомляє: "А коли вийшов він із
Суздальської землі, то одпустив брата свого Всеволода, і Олега, сина свого, і
Ярополка в Русь, а сам із сином Володимиром пішов до Новгорода Великого".

В 1187 року князь Рюрик Київський послав у Суздаль до князя Всеволода сватати
його восьмилітню дочку Верхуславу за свого сина Ростислава. Всеволод згодився,
дав велике придане і відпустив її "в Русь".

Князь Рюрик справив пишне весілля, якого "несть бувало в Русі", а потім тих, що
привезли Верхуславу з Суздалю, "Якова свата і с бояре одпустив ко Всеволоду в
Суздаль". Яков "приїхав із Русі (в Суздаль), проводивши Верхуславу, і бисть
радість".

1223 року на допомогу українським князям у їхній боротьбі проти монголо-татар
послали з ростовським полком Василя Костянтиновича, але він буцімто, не встиг
"до них в Русь".

Під 1406 роком у вітчизняному літописі повідомляється, що "Свидригайло... почав
много зла з Москвою творити Литовській землі і Русі".

Під 1415 роком повідомляється: "Вітовт, Великий князь Литовський, бачивши, що
митрополити, приходивши з Москви в Київ, забирають з святої Софії все, що
красно ... і в Московську землю відносять", щоб "не умалялося багатство в землі
Руській", наказав обрати свого митрополита.

\ifcmt
pic https://img.pravda.com/images/doc/u/s/us_Picture_file_path_4293.jpg
\fi

Таких згадок безліч у літописах, і кожен читач при бажанні може легко знайти ще
пару сотень.

Між тим, ніколи в літописі не згадано ані "Суздальської Русі", ані "Заліської
Русі", ані "Московської Русі" --- усе це пізні вигадки імперських ідеологів. Русь
завжди була лише одна --- Київська.

Руссю не була не лише Північно-Східна околиця, а й інші підлеглі Києву землі.

У 1147 році, коли чернігівський князь Святослав Ольгович обступив місто
Неринськ (у Рязанському князівстві), літопис пише: "У той же час прибігли до
Святослава із Русі отроки". Таким чином, і Рязань не є Руссю.

Під 1148 роком літопис повідомляє про обмін дарами між Великим Київським князем
Ізяславом Мстиславичем і його братом Ростиславом Мстиславичем, князем
Смоленським: "Ізяслав дав дари Ростиславу, котрі од Руської землі і од усіх
цесарських земель (тобто земель, підвладних Київському князю Ізяславу, якого
часто називали цесарем, на зразок Візантійських імператорів), а Ростислав дав
дари Ізяславу, котрі од верхніх земель і од варягів".

Під 1154 роком Київський князь Юрій Довгорукий посилає по свого племінника
Ростислава у Смоленськ, кажучи: "Сину? Мені з ким Руську землю удержати? З
тобою. Поїдь-но сюди".

1173 року, коли Ростиславичам було запропоновано залишити Київ і навколишні
міста Білгород і Вишгород та піти до Смоленська, літопис повідомляє, що "І
заремствували вельми Ростиславичі, що він позбавляє їх Руської землі".

Тож, Смоленськ --- це "верхні землі" по Дніпру, які не є Руссю.

Усі ці дані абсолютно чітко й однозначно свідчать про те, що Суздаль і вся
Північно-Східна околиця Київської держави, ані Смоленськ, ані Новгород, ані під
час перебування під владою Києва, ані після сепарації, не називалася Руссю.

Русь --- це територія сучасної Центральної, а з кінця ХІІ століття і Західної
України, і ніяк інакше.

Цей факт для вчених --- загальновідомий "секрет Полішинеля".

Так, російський вчений Робінсон писав, що "серед усіх випадків вживання поняття
"Руська земля" в "Слові о полку Ігоревім" немає жодного, який говорив би за те,
що в ньому виражене уявлення автора про всі східнослов'янські князівства".

Інший російський історик Ключевський небезпідставно вважав суздальського князя
Андрія Боголюбського першим власне російським князем: "З Андрієм Боголюбським
великорос вперше вийшов на історичну арену".

Менш добросовісні російські ідеологи довго намагалися створити міф, що нібито
столиця Русі "переїхала" з Києва.

Теорія про масове переселення була сформована російським вченим Погодіним,
однак навіть за радянських часів була визнана ненауковою.

Річ у тім, що, хоча літописи фіксують переселення сотень людей, в жодному з них
немає згадки про масове переселення з Русі до Залісся.

Таке переселення досі є бездоказовим і ґрунтується виключно на здогадках, а
також на тому факті, що в деяких російських містах (Переяславлі-Заліському,
Рязані, Володимирі-на-Клязьмі тощо) назви річок повторюють назви Київщини і
Переяславщини --- Почайна (в російській мові вона стала "Пучай-рекою", Ручай,
Либідь тощо.

Однак хто здійснив це перейменування --- переселенці чи ностальгуючі за
Батьківщиною князі --- достеменно не відомо.

Насправді, нікуди столиця Русі не переїжджала, як не переїжджала й сама Русь.
Водночас, у предків росіян внаслідок сепарації від Києва утворилася своя
держава --- Суздаль, яка потім дістала назву Залісся, а ще пізніше --- Московія.

Ця держава стала для предків росіян більш актуальною, ніж колишня метрополія.
Із бездержавного статусу колишні фіно-угорські колонії Києва перейшли в
державний, ще кілька сотень років після того і не помишляючи називатися Руссю.

Цікаво, що й Новгород, з якого нині в Росії намагаються зробити "першу столицю
Росії", також не вважався Руссю.

Зокрема, у літописах читаємо, коли 1148 року Великий Київський князь Ізяслав
Мстиславич прийшов у Новгород, щоб разом із новгородцями, очолюваними його
сином Ярославом, йти на Юрія Довгорукого, київський князь каже новгородцям:
"Осе, браття, син мій, і ви прислали до мене, що вас ображає стрий мій Юрій. На
нього я прийшов сюди, залишивши Руську землю".

Коли того ж року київське і новгородське військо пішло війною на Суздальщину,
літопис чітко розділює новгородців і Русь: "І звідтіля послали вони новгородців
і Русь пустошити до Ярославля", "І в той час прийшли новгородці і Русь,
попустошивши, од Ярославля, і здобичі багато вони принесли".

Отже, новгородці не є русинами --- вони новгородці, і ніяк інакше.

У літописі за 1141 рік читаємо: "А коли Святослав (Ольгович) утікав із
Новгорода йшов у Русь до брата, то послав Всеволод назустріч йому". Тобто,
Новгород не є Руссю.

Цей факт є досить важливим. Єдину в усьому масиві літописів звістку про якесь
інше, не київське походження назви Русь записано в новгородському літописі, де
сказано, що "І од тих варягів, приходьків, назвалася Русь, і од них зветься
Руська земля; і новгородські люди до цього дня від роду варязького".

Водночас, у більш ранньому Початковому Київському зводі кінця ХІ століття, з
якого переписувалися всі ранні повідомлення Новгородських літописів, такого
уривку немає.

У самому Новгородському літописі стверджується, що Середня Наддніпрянщина
почала зватися Руссю ще до приходу варягів у Новгород, у 854 році.

Ці очевидні неузгодженості дали можливість російському академіку Шахматову
довести, що уривок у єдиному списку Новгородського літопису про походження Русі
від варягів як і про походження новгородців не від словен і чуді (як воно було
насправді), а від тих таки варягів, є дуже пізньою вставкою, зробленою
орієнтовно в першій половині ХV століття для зміцнення престижу Новгорода.

Новгородцям було конче потрібно прикласти до себе шляхетне варязьке походження
(хоча насправді в Новгороді жили словени й чудь) в ході суперечки за свій
статус.

Треба сказати, що лише з кінця XII століття, з часу діяльності князя Романа
Мстиславича Західну Україну починають називати Руссю.

Галицько-волинського князя Романа Мстиславича називають "самодержцем усієї
Русі", у той час як, скажімо, щодо Андрія Боголюбського літопис повідомляє, що
він "хотів бути самовладцем усієї Суздальської землі".

Відтоді всі галицько-волинські князі, продовжуючи цю традицію, називалися
князями і господарями "Руської землі" або "всієї Руської землі", а на їхніх
печатках був відображений титул "короля Русі" (Rex Russiae). І в XV, і в XVI, і
навіть у XVIII столітті у літописах географи чітко розрізняють Московію і Русь
(див., наприклад, французьку карту 1754 р.).

У самій Московії терміни "Росия", "Россия" на позначення країни вперше і дуже
обмежено почали вживати лише в XVI столітті, з того часу, як в Москві
з'являється ідея "Третього Риму", а московські царі починають претендувати на
землі України.

Московія перейменувалася за царськими командами 1713 і 1721 років. Етнонім
"русский" закріпився і того пізніше --- лише наприкінці XVIII століття, коли
цариця Катерина II "высочайшим повелением" остаточно наказала московському
народові називатися "руськими" і заборонила йому вживати назву "московитяне".

Цікаво, що Московщина взяла для своєї нової назви грецьку транскрипцію слова
"Русь", хоча, навряд чи є ще народ, який би взяв назву своєї країни з іноземної
мови.

До речі, Україна, як і Московія, теж міняла назву, але тільки один раз, і взяла
цю назву з власної мови.

Саме той факт, що Росія, отримавши свою назву в кращому разі в XVIIІ столітті,
претендувала на історичний спадок Русі, створений на сімсот років раніше, дав
підстави Карлу Марксу стверджувати у своїй праці "Викриття дипломатичної
історії XVIII століття", що "Московська історія пришита до історії Русі білими
нитками".

Додамо, ця праця Карла Маркса --- єдина, що ніколи не друкувалася в СРСР без
купюр.

Українці ніколи не визнавали крадіжки імені "Русь". Вже в середині 18 століття
в Україні з'являється "Історія русів", яка стверджує однозначно український
характер Русі.

Тарас Шевченко у своїх творах жодного разу не застосовує слів "Русь" і навіть
"Росія" і завжди пише про "Московщину".

В сучасній українській літературній мові утвердився етнонім "росіяни", у той
час як прикметник "руський" залишено для всього давньоукраїнського.

Наостанок треба сказати, що в Русі назвою народу були етноніми "русин" (в
знаменитій "Руській правді", княжих грамотах, літописах) і, зрідка, "рус". У
літописах рус чи русин --- це завжди мешканець Київщини.

Етнонім "русичі" зустрічається лише в "Слові о полку Ігоревім", і, на думку
сучасних істориків, в побуті не вживався, а був ознакою високого мовного стилю.

Саме цей етнонім "русин" масово зберігався на Західній Україні до ХХ століття,
а подекуди на Закарпатті --- й до сьогодні.

Виходячи з усього вищенаведеного, історія Росії має таке ж відношення до
історії Русі, як, скажімо, історія Анголи і Мозамбіку до історії Португалії.

Чи історія Індії --- до історії Великої Британії. Якби, наприклад, сьогодні Росія
захотіла перейменуватися на Китай, це зовсім би не означало, що разом із такою
нехитрою маніпуляцією вона б отримала у спадок кількатисячолітню китайську
історію і культуру.

Сьогодні є підстави чекати, що незабаром, у випадку успішності України,
ідеологічна суперечка за історичний спадок Київської Русі, яку протягом
останніх півтисячоліття вели Київ і Москва, закінчиться перемогою України.

Причина цього --- очевидність аргументів про приналежність Україні спадку Русі.
Ці аргументи в наш інформаційний вік не можна ні знищити, ні приховати.

\ifcmt
pic https://img.pravda.com/images/doc/u/s/us_Picture_file_path_4291.jpg
\fi

Автор: Олександр Палій, історик, для УП
