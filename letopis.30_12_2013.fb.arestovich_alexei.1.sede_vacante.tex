% vim: keymap=russian-jcukenwin
%%beginhead 
 
%%file 30_12_2013.fb.arestovich_alexei.1.sede_vacante
%%parent 30_12_2013
 
%%url https://www.facebook.com/alexey.arestovich/posts/639957802734970
 
%%author_id arestovich_alexei
%%date 
 
%%tags maidan2,narod,obschestvo,revolucia,ukraina,vlast
%%title SEDE VACANTE
 
%%endhead 
 
\subsection{SEDE VACANTE}
\label{sec:30_12_2013.fb.arestovich_alexei.1.sede_vacante}
 
\Purl{https://www.facebook.com/alexey.arestovich/posts/639957802734970}
\ifcmt
 author_begin
   author_id arestovich_alexei
 author_end
\fi

SEDE VACANTE.

Когда я раньше писал о том, что народу не нужны лидеры и народ сам все может, я
лукавил.

Это была цена за то, чтобы отбить у народа охоту инвестировать в лидеров
прошлого. 

Конечно, без лидеров нельзя. 

Говорят, между авторами американской конституции в момент ее написания возник
следующий диалог:

- Какие свободные выборы?!. Если мы дадим им право выбирать, они же выберут
СВОИХ!..

 - Успокойтесь. Мы им дали оружие, и они не поставили своих. Пишите. 

В этом диалоге - весь секрет власти. В чем же он?..

Многие наивные товарищи думают, что секрет власти в том, что она сильна и
вооружена, а общество - слабо и безоружно. 

Специалисты говорят, что в одном только Киеве - 400 тысяч одних только нарезных
стволов на руках. Официально зарегистрированных. А гладкоствольных? А
короткоствольных? А нелегальных?.. 

Часть из них, конечно, на руках у представителей власти, возможно, большая
часть, но все равно речь идет о десятках тысяч стволов в распоряжении граждан,
которые, по их словам, власть ненавидят. И что?..

Любой следователь и оперативник со стажем расскажет вам, что все задержанные
преступники обещают страшно отомстить - потом. И никто не мстит. В чем же
секрет?..

Секрет власти в том, что общество не может управлять само собой. И оно это
знает. 

Быть человеком власти, это все равно, что быть стоматологом. Или гинекологом.
Или патологоанатомом. Престижные, денежные специальности. Все ли могут ими
заниматься?.. 

Власть - это совершенно особый поток, страшный для случайных людей. Это
короткую сентенцию подтверждают результаты анализа смывов в туалетах Верховной
Рады, которые (как говорят) были проведены примерно год назад группой
энтузиастов - почти чистый кокаин. С небольшой примесью крэка. 

Случайным людям приходится допинговаться. Не выдерживают нагрузки. 

Власть - раскаленная сковородка. Уже больше месяца она валяется на улице и
никто не хочет ее подобрать. Ходят слухи, что некоторые командиры воинских
частей в четвертом квартале уходящего года неоднократно предлагали власть
лидерам оппозиции. Те отказались. 

Командиры воинских частей ищут лидера. Лидеры отказываются. Общество ищет
лидера. Лидер не появляется. 

Никто не хочет становиться под этот поток. Шутки же давно кончились. На новых
лидеров свалится огромное количество проблем, решать которые не пожелаешь и
врагу. Страна - банкрот. Причем, не только финансовый - моральный. 

Огромное количество случайных людей во власти, довели ее до ручки. 

Так что же это за штука такая "власть" - о которой столько говорят, и которую
никто не хочет брать, когда дело дошло до дела?.. 

Товарищ Сталин определял власть как то, чем можно злоупотребить.

Википедия определяет власть, как "...возможность и способность навязать свою
волю, воздействовать на деятельность и поведение других людей, даже вопреки их
сопротивлению."

Значит ли это, что общество ищет людей, которые способны будут навязать ему
свою волю, даже вопреки его, общества, сопротивлению и, кроме того, смогут этой
волей злоупотребить?. 

Хотите знать, почему не появляются новые лидеры?..

Потому, что историческая задача, которая перед ними стоит, состоит в том, чтобы
изменить саму природу власти.

Понимаете?.. Изменить природу власти. А не в том, чтобы модернизировать
экономику, куда-то там вступить или провести прозрачные выборы. Это - само
собой. 

Изменить природу власти необходимо так, чтобы ею нельзя было злоупотребить. Но
это еще - половина задачи. 

Суть задачи в том, что власть новых времен - это власть, которую не надо
навязывать вопреки сопротивлению тех, над кем она осуществляется.

Проблема наших "лидеров" и тех, кто претендует сейчас на эту роль, совсем не в
том, что они не знают, как реформировать экономику и снизить цену на газ. 

Их проблема в том, что они хотят власти. Такое положение - гарантия того, что
они будут навязывать свою волю другим и злоупотреблять тем, что они получат. 

Вторая их проблема в том, что они - бессовестные люди. Вы скажите - откуда ты
знаешь?!.. Очень просто: совестливый человек не может хотеть власти над другими
людьми. 

Главный орган власти у человека - совесть. Положение нашей страны таково, что
человек с совестью, не может править без слез. Сотни тысяч поломанных судеб.
Миллионы нищих. Разбитые дороги. Убитое здравоохранение. Коррупция.
Депрессивные регионы. Умирающая деревня. Десятки миллионов разочарованных до
состояния полного неверия. Два миллиона абортов в год. Обманутые надежды. 

Новым лидерам, взявшимся за  страну, придется ходить по растоптанным мечтам
своего народа, которыми выстелены наши разбитые, грязные улицы. 

Человек с совестью не может хотеть власти. При одной мысли о власти, у него
хребет должен ломаться от груза ответственности, которая на него ляжет. 

Однако, у нас сотни тысяч людей во власти и еще примерно столько же - желающих
хотя бы маленький кусочек таковой.  

Чего вдруг? При таком количестве проблем?.. Что они собираются с ней делать?..

В Древнем Риме, на заре его становления, был обычай наделять абсолютными
властными полномочиями гражданина, который по мнению сограждан, мог справиться
с критической ситуацией в которую попала республика. 

Один торговец коровами, два или три раза назначался народом диктатором, два или
три раза спасал страну и всякий раз добровольно слагал полномочия. Единственный
диктатор из всех римских диктаторов, который не пытался закрепить свое
положение, используя абсолютные полномочия, которые ему вручались народом. 

Времена изменились, но пример - поучителен. 

Секрет избрания власти новых времен в том, что в нее должны выталкивать. 

Просить занять. 

Тех людей, которые от нее бегают, не хотят ею заниматься. А не тех, которые
вываливают десятки миллионов долларов на самопродвижение и агитацию за самого
себя. 

А секрет самой власти новых времен в том, что она должна более вдохновлять,
помогать, вести, а не принуждать. Служить. 

Все изменилось, заставлять больше неэффективно. 

Лидерами нового времени могут стать лидеры мнений.

 Уже сейчас у таких лидеров мнений есть своя аудитория. 

Когда прижмет покруче, сообщество кинется к ним уже не поболтать - за помощью. 

Секрет контроля новой власти, секрет отсутствия злоупотреблений - в
самоорганизации нового типа. 

Когда группа аквтивистов добровольно и свободно просит о руководящей помощи
человека, на мнение, позицию, знания и умения которого, они считают нужным
ориентироваться. 

Все говорят, что лидер начинается со взятия ответственности. 

Это не так. Лидер начинается с занятия позиции. 

Общество устроено, как вода. Занявший позицию подобен свае, вбитой на бурном
переходе. Вода вынуждена его учитывать, переходящие могут на него опереться. 

Эта позиция не может быть иной, нежели ценностной. Речь, в конце концов, идет о
ценностях и о совести.  

Ищите людей, которые выдержат соблазны управления себе подобными.

Которых хочется слушать. За кем не страшно идти.

В таком случае, грех возможного злоупотребления ляжет на головы самого
сообщества: кого же вы выталкивали?.. 

Для чего просили помощи?..

Лидером может стать тот, кто не хочет этого делать, но делает, потому, что люди
попросили. Речь идет о выполнении своего гражданского долга перед согражданами.
Перед обществом, в котором они живут.

В случае неподчинения такие лидеры всегда могут спросить: мы не хотели этого,
зачем же вы нас назначали?.. Чтобы теперь не слушать?..

Лидеры нового времени смогут ходить без охраны и руководить без аппарата. Они
им просто будут не нужны. Зачем аппарат принуждения тем, кого попросили
помочь?.. 

Это - их главная примета. 

И, как всегда, притча.

Группа иностранных офицеров, в одной восточной стране, гостила у одного
муллы-исмаилита, в заброшенном Богом и песками городке. Мулла был лидером
общины, знал семь языков и закончил два университета. 

Они лежали на коврах в саду возле арыка и говорили о природе власти.

Мулла дождался паузы и вызвал охранника. 

- Достань гранату - сказал он ему.

Тот достал.

- Выдергивай кольцо.

Тот выдернул.

- Разожми ладонь.

Тот разжал (взрыв через три секунды).

Мулла чуть подождал, а потом спокойно сказал: 

- Выбрось.

Тот выбросил гранату в арык.

Когда пыль осела и грохот отзвенел, мулла сказал:

- Иди. 

Охранник вышел (молодой парень, лет двадцать пять). На протяжении всей сцены
выражение его лица не изменилось. 

Мулла повернулся к бледным, как мел, иностранным офицерам и сказал:

- Видите ли... Я у них тут религиозный лидер. Жизнь, смерть, свадьбы и
похороны, земля и вода - в моих руках. А я хотел бы быть духовным лидером.

- А это как?!..

- А это значит сделать так, чтобы никто на свете, не смог больше заставить
этого юношу выполнять такие приказы. 

Секрет новых лидеров для народа, не в лидерах, а в народе. Нужно быть очень
зрелым обществом, чтобы получить таких лидеров. 

Или - чудо. 

Sede vacante.
\ii{30_12_2013.fb.arestovich_alexei.1.sede_vacante.cmt}
