% vim: keymap=russian-jcukenwin
%%beginhead 
 
%%file 22_02_2021.stz.news.ua.mrpl_city.1.nilsen_150_rokiv.1.dytynstvo_ta_junist
%%parent 22_02_2021.stz.news.ua.mrpl_city.1.nilsen_150_rokiv
 
%%url 
 
%%author_id 
%%date 
 
%%tags 
%%title 
 
%%endhead 

\subsubsection{Дитинство та юність}

Віктор Нільсен народився у Санкт-Петербурзі в 1871 році в сім'ї лютеранського
пастора. Цікаво, що родина мала данське і німецьке походження. За розповіддю
сина архітектора його батько з дитинства захоплювався архітектурою, на що
вплинули розкішні міські ансамблі Санкт-Петербурга. Віктор Олександрович у 1896
році успішно закінчує курс навчання в Інституті громадянських інженерів, де
отримує диплом архітектора, який служив свідченням його ґрунтовної
підготовленості в галузі архітектурної та будівельної справи. Якщо ознайомитися
з основними тогочасними обов'язками архітекторів (спираюся на \emph{Енциклопедичний
словник Брокгауза і Ефрона}) можна зрозуміти, що люди, які обирали цю професію,
були широко обізнаними. Зокрема, архітектори мали знати основи будівельної
механіки, мінералогії, геології, фізики, хімії і навіть юриспруденції, адже
вони повинні були розуміти, що дозволяється, а що забороняється в будівельній
справі законодавством країни. З огляду на це можна зробити висновок, що молодий
Віктор Олександрович був досить розумним і кмітливим чоловіком, якщо зміг
опанувати таку непросту професію.
