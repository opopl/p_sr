% vim: keymap=russian-jcukenwin
%%beginhead 
 
%%file 01_11_2020.fb.alex_belyy.1.vybory
%%parent 01_11_2020
 
%%url https://www.facebook.com/permalink.php?story_fbid=177803093949537&id=100051595411712
%%author 
%%tags 
%%title 
 
%%endhead 

\subsubsection{Что дальше?}
\Purl{https://www.facebook.com/permalink.php?story_fbid=177803093949537&id=100051595411712}
\Pauthor{Белый, Алекс}

Выборы прошли! Что же у нас в остатке? Усталость кандидатов и их штабов, да и
всех нас от предвыборной агитации. Первый вывод можно сделать сразу и он
очевиден - граждане Украины, разочарованные Зеленским, проигнорировали выборы.
Явка была низкая по всей стране. 

Пандемия тоже не добавила желания идти на выборы, это  проблема, которую власть
тоже не решила и решать не хочет. Тупые вбросы о якобы появившейся украинской
вакцине вызвали у людей смех. Пять вопросов от Зеленского не добавили ни явку,
ни голосов партии «СН».

Бюллетени были напечатаны так, чтобы люди с плохим зрением не смогли ничего
увидеть. Понятно, для чего это было сделано. Сторонники оппозиционной партии
«За життя» - люди в возрасте и им было сложно разобраться в таких бюллетенях,
да и просто хоть что-то прочитать в них.  

Сама система выборов очень сложная и поэтому до сих пор во многих местах идёт
подсчёт или пересчёт голосов. Все это говорит о некомпетентности власти и
попытке искусственно повлиять на результаты выборов.

Полный провал на выборах «СН» был понятен задолго до выборов.  В разных
регионах их обошли и «За життя», и ПЕС, и местные партии, завязанные на
действующих мэров. Опять политическая карта Украины поделилась на восток и
запад. Ничего нового, Зеленский не объеденил страну. Есть разочарование и
президентом, и его командой.  

Ни одна партия не смогла стать доминирующей по всей стране. Раскол по языку и
идеологии никуда не ушёл. Власть на местах теперь сможет стать более
независимой от Киева. Конечно, зависимость от разного рода финансовых подачек
останется и заставит их договариваться с Киевом.
Маловато для склейки страны. 

Силовой блок под властью президента и пока ему вроде подчиняется. Но ничего у
Зеленскского больше нет. Информация о том, что готовится снятие большинства
губернаторов, отставка Кабмина, замена генпрокурора и министра МВД может на
время отвлечь людей от проблем и снять градус недовольства. Сможет ли новый
премьер предложить программу выхода из политического и экономического кризиса -
мы пока не знаем. Время покажет.  Выборы прошли, а усталость от них осталась.

