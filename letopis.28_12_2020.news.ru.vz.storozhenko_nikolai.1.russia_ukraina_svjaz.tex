% vim: keymap=russian-jcukenwin
%%beginhead 
 
%%file 28_12_2020.news.ru.vz.storozhenko_nikolai.1.russia_ukraina_svjaz
%%parent 28_12_2020
 
%%url https://vz.ru/world/2020/12/28/1077980.html
 
%%author 
%%author_id storozhenko_nikolai
%%author_url 
 
%%tags 
%%title Кулеба доказал неразрывную связь Украины с Россией
 
%%endhead 
 
\subsection{Кулеба доказал неразрывную связь Украины с Россией}
\label{sec:28_12_2020.news.ru.vz.storozhenko_nikolai.1.russia_ukraina_svjaz}
\Purl{https://vz.ru/world/2020/12/28/1077980.html}
\ifcmt
	author_begin
   author_id storozhenko_nikolai
	author_end
\fi

\ifcmt
  pic https://img.vz.ru/upimg/m10/m1077980.jpg
\fi

\begin{leftbar}
  \begingroup
    \em\Large\bfseries\color{blue}
«Разрыв Украины с «русским миром» окончателен», заявляет глава МИД страны
Дмитрий Кулеба. Тем самым он повторяет давнюю мантру, придуманную в Киеве еще
во времена Петра Порошенко. Скажем министру спасибо – с его подачи легко
увидеть, что на самом деле, несмотря на потуги киевских властей, Украина
по-прежнему часть большой русской цивилизации. И примеров этому множество.
  \endgroup
\end{leftbar}

До Нового года еще три дня, а в Киеве отмечание в самом разгаре. В этот раз
«наотмечался» глава МИД Украины Дмитрий Кулеба, заявивший изданию
Internationale Politik об окончательном разрыве с «русским миром». «Украина
продолжит свой путь как европейская страна, как часть Запада, так как ее разрыв
с «русским миром» окончателен. Все остальное – это вопрос времени и
дипломатии».  

\subsubsection{Связь из глубины веков}

Кулеба – вроде как министр при президенте Зеленском, но закалка явно еще
порошенковская. Мы хорошо помним, что номер «Остаточне «прощавай» был у
экс-президента Украины одной из «коронок» в его публичных выступлениях. И хотя
«остаточне» – это окончательное, но прощался Петр Алексеевич с Россией
регулярно, раз в месяц примерно. Прощался и не уходил. 

Впрочем, давайте серьезно, тем более что украинской власти эта проблема явно не
дает покоя. Итак, разорвала ли Украина связь с русским миром и насколько это
«остаточно»?

Начнем с основ. Киев и земли, ставшие впоследствии Украиной, были одним из
центров русской государственности более тысячи лет назад. В качестве примера
можно вспомнить Древний Рим, у которого (кроме собственно Рима) такими центрами
были Константинополь, Медиолан, Никомедия и несколько других. Знаменитую
формулу «Великий князь Киевский и всея Руси» мы впервые видим у сына Ярослава
Мудрого Всеволода Ярославича. Потом великие князья станут Московскими, но
приставка «...и всея Руси», конечно, останется. 

Однако смена центра государственности с Киева на Москву не означала обрыва этой
преемственности. Может, для пана Кулебы Москва – заграница, да еще и
недружественная. Работа у человека такая, что поделать. А для любого
московского князя, царя, императора и т. д. Киев целыми столетиями заграницей
не был и быть не мог. И пану Кулебе придется с этим как-то жить. 

Уже даже в этом, самом глубинном смысле Украина никогда не сможет порвать с
русским миром – потому что сам русский мир с Украиной порывать не желает. Ведь
это все равно что отказаться от многосотлетней истории и начать отсчет своей
государственности с XX века. Если самой Украине это нужно – Бог в помощь.  

\subsubsection{Запретный, но незаменимый язык }

Однако можно и не продираться вглубь веков для констатации очевидного. Вот
только в сентябре мы писали об анекдотическом случае: украинская сеть книжных
магазинов «Є», которая еще год назад призывала читателей бойкотировать другие
книжные магазины за продажу в них книг украинских «дочек» российских
издательств, признала, что сама ими торгует. И не просто торгует, а не может не
торговать, потому что на них приходится половина продаж. И если от них
отказаться, то магазины придется закрыть.

Несколькими месяцами ранее был и другой случай:\Furl{https://vz.ru/world/2020/7/10/1049302.html} приостановка контрабанды
российских книг на Украину оставила украинских врачей без самой актуальной
сегодня литературы – по иммунологии и вирусологии. Рынок на Украине слишком
мал, чтобы переводить и издавать такую специфическую литературу. Тут либо
каждый экземпляр будет «золотым», или сразу нужно переводить и издавать такое
за счет государства, чего, конечно же, никто делать не будет. Это ж не стена
Яценюка, тут не украдешь. 

\begin{leftbar}
  \begingroup
    \em\large\bfseries\color{blue}
    Специальная и научная литература – еще один пример такой связи Украины с русским миром.
  \endgroup
\end{leftbar}

Разорвать ее, конечно, теоретически можно. Но придется либо дорого за это
заплатить, либо смириться и жить дальше. Без науки и современной медицины. А
ведь болеют иногда даже министры иностранных дел Украины. 

И не только книги, к слову. Статистика поисковых запросов каждый раз ввергает
украинского патриота в уныние. Украинизировано уже все: школы, садики, вузы,
фильмов на русском в кинотеатрах нет в принципе, русскоязычному контенту на
радио и ТВ оставили жалкие проценты. Зато в украинском интернете царит русский
язык. Даже в западных регионах Украины поисковые запросы на нем – примерно две
трети пользователей. В целом же на русском – 70–80\% запросов.\Furl{https://tsargrad.tv/news/zrada-ot-google-bolshinstvo-ukraincev-predpochitajut-guglit-na-russkom-jazyke_236076}

Еще хуже дела с Youtube. В первой десятке музыкальных клипов, которые смотрели
украинцы в этом году, ни одной\Furl{https://itc.ua/news/chto-smotreli-v-ukraine-na-youtube-v-2020-godu/} украинской композиции. 

\subsubsection{Битвы за меню}

Можно и более уместный пример. Каждый год с декабря по январь медиапространство
Украины атакует вал однотипных статей (по ссылке – яркий пример). В них вам
расскажут, что Деда Мороза придумали в НКВД, чтобы скрывать с его помощью
репрессии. В деталях пояснят, чем святой Николай отличается от Деда Мороза и
почему первый – традиционный для Украины персонаж, а второй – морок темного
прошлого. Будет и пятиминутка ненависти к традиционному новогоднему меню, где
главной мишенью будет, конечно же, салат оливье. «...нельзя объединяться с
теми, кто ностальгирует по СССР, шпротам, оливье и колбасе по 2.20», – со
звериной серьезностью вещал экс-глава Украинского института национальной памяти
Владимир Вятрович в январе. 

Знаменитый салат он упомянул не зря: год назад на Украине разгорелась целая
оливье-битва. Ее началом стал предновогодний пост в Facebook кулинарного
эксперта Евгения Клопотенко о том, почему он не ест этот салат и никогда не
готовит его на Новый год: «Я не готовлю оливье на Новый год, потому что не хочу
видеть Советский Союз в своей тарелке. Это как выйти из тюрьмы, но продолжить
питаться тюремными блюдами. Я так не хочу». 

Пост тогда, что называется, удачно попал на вентилятор, а пользователи
мгновенно разделились на сторонников и противников салата. Первые ехидно
предлагали учредить должность салатных инспекторов (в дополнение к языковым),
которые в новогодние праздники будут ходить по домам украинцев и проверять,
какие блюда те готовят. Вторые скрежетали зубами и предлагали отказаться и от
других привычных блюд, вроде селедки под шубой, салата «Мимоза» и даже любых
полусладких игристых вин.

Кстати, недавняя битва за борщ – вполне себе продолжение этого сюжета. Просто
год назад «украинские патриоты» решили отказаться от салата (известного во всем
мире под вторым названием – «русский салат»). А в этом в очередной раз
попытались распилить общее культурно-гастрономическое наследие. Хотя, ей-богу,
им самим же было бы проще отказаться от него, как и от «Советского Союза в
тарелке». Ведь еще тысячу лет будут бегать и доказывать на всех углах, что борщ
– украинский, а все будут видеть, как украинцы и русские делят общий борщ.  

\subsubsection{Русское Рождество}

Если же серьезно, то новогодний период для украинцев и русских действительно
повод и время вспомнить об их неразрывной связи: два народа связаны Рождеством.
Нет, не в том глобальном смысле, после которого не стало ни эллина, ни иудея.
Известно, что РПЦ и УПЦ (наряду с Иерусалимской, Грузинской, Сербской и
Польской православными церквями) пользуются еще юлианским календарем. А
поскольку он отстает от григорианского, которым пользуются остальные
христианские церкви, то и «русское» Рождество отмечается как бы позже.

\begin{leftbar}
  \begingroup
    \em\Large\bfseries\color{blue}
    Среди церквей, пользующихся юлианским календарем, не хватает еще как минимум одной. 
  \endgroup
\end{leftbar}

Долгое время украинские греко-католики отмечали Рождество в один день с
православными. Традиция эта укрепилась, видимо, еще во время возникновения
Русской униатской церкви (опять-таки – русской!), то есть в конце XVI века. С
тех пор униатство было опорой католического мира в границах мира православного.
И старалось не выделяться. Поэтому греко-католики отмечали Рождество в русской
традиции (но только в пределах Украины). И даже глава недавно созданной
«порошенковской» церкви на вопрос о переходе на празднование 25 декабря
ответил, что сам-то он не против, но не готовы верующие. И чтобы избежать
раскола (!!), он подождет, пока к этому будут готовы прихожане «ПЦУ». 

Уже одно это способно взорвать мозг кому угодно из «патриотов», ведь
Константинопольская церковь, под патронатом которой существует «ПЦУ», празднует
Рождество по григорианскому календарю. И сам глава «ПЦУ», видимо, тоже. Но
чтобы не остаться без паствы, он вынужден беречь ту самую связь с русским миром
и служить Рождественскую службу в то же время, когда ее служат в УПЦ и РПЦ. Вот
вам и прощание с русским миром.

...Что же до Кулебы, то такие персонажи даже полезны. Иной раз задумаешься, а
что же общего у украинцев и русских? А они – хоп, и подскажут.  

\ii{28_12_2020.news.ru.vz.storozhenko_nikolai.1.russia_ukraina_svjaz.comments}
