% vim: keymap=russian-jcukenwin
%%beginhead 
 
%%file 23_03_2021.fb.dykyj_evgen.1.revgidnosti_ganjba
%%parent 23_03_2021
 
%%url https://www.facebook.com/evgen.dykyj/posts/10159207651738808
 
%%author Дикий, Євген
%%author_id dykyj_evgen
%%author_url 
 
%%tags 
%%title Сьогодні річниця нашої ганьби. Рівно сім років тому ми зрадили Революцію Гідност
 
%%endhead 
 
\subsection{Сьогодні річниця нашої ганьби. Рівно сім років тому ми зрадили Революцію Гідност}
\label{sec:23_03_2021.fb.dykyj_evgen.1.revgidnosti_ganjba}
\Purl{https://www.facebook.com/evgen.dykyj/posts/10159207651738808}
\ifcmt
 author_begin
   author_id dykyj_evgen
 author_end
\fi

Сьогодні річниця нашої ганьби. Рівно сім років тому ми зрадили Революцію Гідності.

В цей день менти без суду та слідства розстріляли Сашка Музичка - \enquote{Білого}.
Легендарного командира УНА-УНСО, героя Грузії, Чечні та Майдану. Незручного.
Некерованого. Нечемного. Незаконослухнняного. Часто-густо неприємного. Але
нашого.

Ми проковтнули це зухвале вбивство, і зробили вигляд що повірили у шиту білими
нитками офіційну версію МВС, освячену іменами новопризначеного \enquote{революційного}
міністра - так, того самого, що досі в цьому міністерському кріслі, та його
замів, рекрутованих з лав Майдану - яких ментовська машина вже за півроку
викинула зсередини як чужорідний елемент одноразового використання.

Ми ж насправді все знали та розуміли. Ми ж насправді ні на хвилину не вірили.
Але підіграли вбивцям та змовчали.

Я добре памятаю всі наші тодішні міркування. У нас була вагома аргументація, ми
вчувались мудрими та відповідальними. Ми зрадили не за тридцять срібняків, не
за страх чи ще щось подібне. Ми зрадили із найкращих міркувань, заради країни.
Можливо, це трохи помякшує нашу провину. Але зрада все одно лишається зрадою, а
безкарне беззаконнне вбивство - підступним вбивством. І ми, зрадники, лишаємось
ними попри всі помякшуючі обставини.

І коли через сім років після Революції Гідності ми змушені проводити довгі
години під судами над Ріфом та Стерненком, споглядати як судді, котрі виносили
нам вироки за Майдан, судять бійців добробатів за \enquote{утиски} окупантів, а молоді,
які вчасі Майдану вчились у школі, змушені вчитись писати на стінах \enquote{АСАВ} - це
чесна та справедлива плата за нашу тодішню зраду. Ми самі \enquote{злили} Революцію
Гідності, самі дали мусорській кривосудній системі карт-бланш на вбивства та
будь-яку сваволю, і це сталось 23 березня 2014 року. 

Тож тепер коли за черговим із нас прийдуть із абсурдними звинуваченнями, коли
виявитсья що черговий доброволець \enquote{плекав ідеї зверхності арійської раси} та
через це ще до свого народження застрелив Кеннеді, ну або на що там ще стачить
ментовських фантазій - не варто запитувати, за що нам таке. От саме за це. За
Сашка \enquote{Білого}, за його непокаране вбивство та за нашу \enquote{мудру} та
\enquote{державницьку} зраду.

В християнській культурі гріх передбачає шанс на спокуту. Я вірю, що в нас ще є
шанс якщо не виправти вчинене зло - Сашка вже не воскресити, - то принаймні
спокутувати його. Спокутувати, зламавши створену Дзержинським та Берією і досі
живу каральну машину ментовсько-суддівського кривосуддя. І створивши замість
неї справжні українські правоохоронні органи та справжні чесні суди.

І тоді ми прийдемо на могилу до Білого та проситимемо пробачити нас,
нерозумних. Але поки зарано - ми ще не заслужили права на цю прощу.
