% vim: keymap=russian-jcukenwin
%%beginhead 
 
%%file 09_03_2022.fb.fb_group.story_kiev_ua.1.hroniki_obretenia_i_poteri
%%parent 09_03_2022
 
%%url https://www.facebook.com/groups/story.kiev.ua/posts/1877595472437245
 
%%author_id fb_group.story_kiev_ua,stepanov_farid
%%date 
 
%%tags 
%%title ХРОНИКИ НЕОБЪЯВЛЕННОЙ ВОЙНЫ, Или: день десятый. Обретения и потери
 
%%endhead 
 
\subsection{ХРОНИКИ НЕОБЪЯВЛЕННОЙ ВОЙНЫ, Или: день десятый. Обретения и потери}
\label{sec:09_03_2022.fb.fb_group.story_kiev_ua.1.hroniki_obretenia_i_poteri}
 
\Purl{https://www.facebook.com/groups/story.kiev.ua/posts/1877595472437245}
\ifcmt
 author_begin
   author_id fb_group.story_kiev_ua,stepanov_farid
 author_end
\fi

ХРОНИКИ НЕОБЪЯВЛЕННОЙ ВОЙНЫ

Или: день десятый. Обретения и потери.

12:30

Уже второй день идёт \enquote{священная война} в сообществе бывших и нынешних
сотрудников Coca-Cola. Одна из крупнейших мировых корпораций приняла решение
продолжать работать в рф. И сообщество сотрудников Coca-Cola из Украины
разделилось на два лагеря. Одни, ссылаясь на решение Штаб-квартиры, разводят
руками: \enquote{Что же мы можем сделать? Мы не влияем на решения там. Мы же в
Украине помогаем армии. Что вы ещё от нас хотите?}. Другие отвечают им:
\enquote{Берите пример с Metro. Которые не приняли подобную позицию своей
Штаб-квартиры. И всеми доступными им способами объясняют, что не правильно
прикрываться в эти дни \enquote{мы вне политики}. Что выбор: с кем ты, все
равно придется сделать. И платить налоги в стране-агрессоре это - быть
сопричастным к убийству украинцев}

Я свой выбор сделал: последний глоток Coca-Cola и бутылка в мусорнике.
Coca-Cola - иди нахуй. В \enquote{родной гавани} тебя уже заждались.

16:00

\enquote{Возьми, пожалуйста, трубку, твой дядя звонит. Скажи, что я занята} - жена
показывает младшему на телефон, не отрываясь от приготовления борща.

Ещё 10 дней назад мои дети не знали, что у них есть дядя. Двоюродный или
троюродный, не очень разбираюсь в этих тонкостях родственных связей. Бабушка
жены родом из Сибири. У нее была родная младшая сестра. А дядя моих детей - ее
внук. 

\ii{09_03_2022.fb.fb_group.story_kiev_ua.1.hroniki_obretenia_i_poteri.pic.1}

\enquote{Я так понял, что Вы болеете за \enquote{Ливер} (Ливерпуль)? А такого
футболиста как Диего Армандо Марадона знаете?} - младший нашел благодатного
слушателя. И забыты уже вой сирен, глухие звуки взрывов, темнота комендантского
часа.

Ещё 10 дней назад моя супруга не знала, что у нее есть родственник, который
станет ей роднее, многих куда более близких людей. \enquote{Как вы?} - наш день
начинается с его звонка. \enquote{Приезжайте, я вас всех приму у себя}.

Ещё 10 дней назад я сомневался, что смогу с кем-то из-за поребрика нормально
обсуждать войну на уничтожение, которую ведут против моей страны. Чувствую, что
для него эти разговоры, как открывшееся окно в мир. Окно, которое для него
тщательно закрывали на засов, подпирая шваброй для надёжности. Десятую ночь, с
коротким перерывом на сон, мы разговариваем о Степане Бандере и событиях в
Одессе в мае 2014 года, о Майдане и Владимире Зеленском, о Донецке и Правом
секторе. Я давно не видел человека, который с такой заинтересованностью хотел
бы разобраться. 

Считаю, что если смогу хотя бы одному человеку из-за поребрика донести правду,
сделаю большое дело.

А у моего младшего уже большие планы: \enquote{Когда закончится война, мы вместе с ним
сыграем в футбол} 

18:00

Вангую, что эти апостолы \enquote{русского мира}, которые пришли нас \enquote{освобождать},
никогда не читали ни Достоевского, ни Чехова, ни Пушкина. А Лев Николаевич
Толстой и Алексей Николаевич Толстой - для них братья. Или я очень ошибался в
своем понимании русской культуры и ценностей. Которые и есть: сжигать,
уничтожать, насиловать, оставлять после себя пепелища и руины.

Вопль отчаяния и крик о помощи от нашего друга киевлянки. Ее родители остались
в небольшом поселке между Ирпенем и Бучей. Около часа назад к ним в дом пришли
\enquote{освободители}. В доме были ее пожилые мать, отец и его товарищ. Товарища
застрелили, отца ранили в живот. Он сейчас с женой в подвале дома. Скорая
приехать не может, уже были случаи, когда машины скорой обстреливали. Его они
не выпускают. Моя жена сейчас пытается найти врачей, которые окажут помощь по
телефону.

\enquote{Во всем есть черта, за которую перейти опасно; ибо, раз переступив, воротиться
назад невозможно}

Ф. М. Достоевский

PS:

На фото: Ирпень. Освобождаемые прячутся под развалинами моста от ковровых
бомбардировок освободителей.
