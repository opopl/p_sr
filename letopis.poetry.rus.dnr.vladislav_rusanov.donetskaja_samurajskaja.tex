% vim: keymap=russian-jcukenwin
%%beginhead 
 
%%file poetry.rus.dnr.vladislav_rusanov.donetskaja_samurajskaja
%%parent poetry.rus.dnr.vladislav_rusanov
 
%%url https://stihi.ru/2017/04/01/1895
%%author 
%%tags 
%%title 
 
%%endhead 

\subsubsection{Донецкая самурайская}
\label{sec:poetry.rus.dnr.vladislav_rusanov.donetskaja_samurajskaja}
\Purl{https://stihi.ru/2017/04/01/1895}

Владислав Русанов

Террикон, как Фудзияма,
на закате величав.
Я ползу, моллюск упрямый,
притяжение поправ.

Огонёк внизу на хате
так приветливо манит,
под ногой хрустит приятель —
допотопный аммонит.

Если на кон --- всё до нитки,
если бить, так наповал.
Здесь когда-то сон зенитки
самурай оберегал.

Не за деньги, не за славу,
не за почести держись.
Коль привык работать в лаве,
то в окопе тоже жизнь.

Со дворов, не из дворянства,
оказались на войне,
но не рады их упрямству
на небратской стороне.

«Перемирье мир не красит,
а Победа красит мир», —
отозвался о Донбассе
батальонный командир.

Шли в танкисты и в пехоту
из забоя --- от и до, —
не нарушив ни на йоту
славный кодекс бусидо.

Красоту родного края,
терриконы и поля,
отстояли самураи —
братья стали и угля.

2017 
