% vim: keymap=russian-jcukenwin
%%beginhead 
 
%%file 08_12_2021.tg.tkachev_jurij.1.vtorzhenie_putin_usa
%%parent 08_12_2021
 
%%url https://t.me/dadzibao_talk/307600
 
%%author_id tkachev_jurij
%%date 
 
%%tags 
%%title Даже в случае полномасштабного российского вторжения в Украину, США войска не пошлёт
 
%%endhead 
\subsection{Даже в случае полномасштабного российского вторжения в Украину, США войска не пошлёт}
\label{sec:08_12_2021.tg.tkachev_jurij.1.vtorzhenie_putin_usa}

\Purl{https://t.me/dadzibao_talk/307600}

\ifcmt
 author_begin
   author_id tkachev_jurij
 author_end
\fi

Между тем, из всего информационного мусора, который вылился на нас в последние
недели в контексте грядущего (но это неточно) российского вторжения в Украину,
можно выловить по крайней мере один информационный бриллиант, а именно: 

\textSelect{Даже в случае полномасштабного российского вторжения в Украину, США войска не
пошлёт.}

То есть, если бы Путин на самом деле хотел напасть, завоевать и так далее, он
мог бы это сделать хоть завтра. Худшее, чем он бы столкнулся в этом случае -
"экономические санкции". Что есть понятие весьма растяжимое и расплывчатое.

Причём высказывания украинских официальных лиц не оставляют сомнения в том, что
итог такого нападения и возможность своими силами отразить его в Украине
оценивают здраво: как нечто из области фантастики. Недаром нам уже начинают
рассказывать про \enquote{40 миллионов потенциальных партизан} и прочий бред И
даже творец сильнейшей армии в Европе Порошенко о победе не говорит, а говорит
лишь о том, что оккупация Украины России дорого обойдётся. 

В общем, мораль проста: если Украина до сих пор существует, то только потому,
что Путина это устраивает.

Да-да, политические украинцы. Вы существуете исключительно милостью Путина.
Живите теперь с этим.

\ii{08_12_2021.tg.tkachev_jurij.1.vtorzhenie_putin_usa.cmt}
