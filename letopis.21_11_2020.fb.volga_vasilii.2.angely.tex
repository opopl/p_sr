% vim: keymap=russian-jcukenwin
%%beginhead 
 
%%file 21_11_2020.fb.volga_vasilii.2.angely
%%parent 21_11_2020
 
%%url https://www.facebook.com/Vasiliy.volga/posts/2757973104520104
 
%%author Волга, Василий Александрович
%%author_id volga_vasilii
%%author_url 
 
%%tags 
%%title Ангелы
 
%%endhead 
 
\subsection{Ангелы}
\label{sec:21_11_2020.fb.volga_vasilii.2.angely}
\Purl{https://www.facebook.com/Vasiliy.volga/posts/2757973104520104}
\ifcmt
	author_begin
   author_id volga_vasilii
	author_end
\fi

Что это такое? И до какой степени серьезно можно относиться к тем людям,
которые о них рассуждают, или вот даже празднуют, например, (как это некоторые
из них делают сегодня) нечто вроде Дня Рождения всех ангелов вообще и их
предводителя самого главного ангела по имени Михаил, что в переводе с
древнееврейского языка означает «тот, кто как Бог»?

\ifcmt
pic https://scontent.fiev6-1.fna.fbcdn.net/v/t1.0-9/126939181_2757975437853204_4518512384899099663_n.jpg?_nc_cat=104&ccb=2&_nc_sid=730e14&_nc_ohc=xofY1_KdGs8AX_D09y2&_nc_ht=scontent.fiev6-1.fna&oh=b6de05568ef64fd207c36a9e04d73490&oe=5FEA309D
\fi

О чем это вообще?

Многих смущает то, как ангелов изображает церковное искусство. Мол, что это вы
себе придумали - крылья, шесть глаз, огненные мечи и прочая театральная
атрибутика? Ведь стыдно даже об этом говорить! Ведь взрослые люди,
образованные, век двадцать первый на дворе, а они про херувимов и серафимов
рассуждают. Смешно же, ей Богу.

Три года назад в Киеве взорвали автомобиль. В самом центре Киева, возле
Бессарабского рынка. За рулем была женщина, а на заднем сиденье в детском
креслице был её ребенок. Ребеночку было два года. Взрывное устройство было
заложено профессионально и было приведено в действие дистанционно. Женщина не
погибла, но лишилась ног, глаз, руки и получила множественные осколочные
ранения. Ребенок не пострадал совершенно.

Именно то, что так не понятно современному и слишком уж считающему себя
образованным человеку нашего века, и спало ребенка. Ангел укрыл ребеночка от
взрывной волны, и огня, и осколков.

Современному человеку вообще свойственно слишком много думать о себе. Хотя,
казалось бы. Даже данные современной науки, которые говорят о том, что человек
– это не так уж давно появившаяся на поверхности Земли форма жизни (что-то
вроде вредоносной бактерии), не могут уменьшить самомнение человека о себе
самом. Современный человек отказывает всему, что он не понимает, чего он не
может потрогать, понюхать, съесть. И чем, скажите мне, такой современный
человек отличается от своего давнего предка, собиравшего и употреблявшего в
пищу коренья? Ничем. Набором слов. И не более.

Правда о бытии, т.е. Истина, глубока многогранна и бесконечна. Она уходит
своими корнями в само Творение, в момент рождения Жизни, в момент появления
Бытия, и мы, люди, один только фрагмент этого огромного и бесконечного, как Сам
его Творец, мира. И как же велика Его благость, что он мир этот весь подарил
нам. И как же мы по-свински  с ним обходимся. Но это тема для другой
публикации.

Ну, а крылья, глаза, огненные мечи – это ведь все попытка человека рассказать
другому человеку о том, что описать человеческим языком невозможно. Да и
описывать это начали люди, жившие несколько тысяч лет назад, которые не знали
самолетов, компьютеров, космических ракет, но тем не менее начали сталкиваться
с этими силами. Как еще можно было сказать о том, что силы, выполняющие от
имени Высшего Существа работу во Вселенной, могут перемещаться так быстро, что
время и пространство для них не существует? Конечно, надо было наделить их
шестью крыльями.

Одним словом, дорогие братья и сестры, бесполезное ведь это дело - рассказывать
слепорожденному о том, что значит зеленый или красный цвет. Так и атеист не
поймет, о чем мы сегодня радуемся.

С Праздником вас, православные.

Ангела Хранителя вам и вашим семьям.
