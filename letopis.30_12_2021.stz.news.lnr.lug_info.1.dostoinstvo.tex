% vim: keymap=russian-jcukenwin
%%beginhead 
 
%%file 30_12_2021.stz.news.lnr.lug_info.1.dostoinstvo
%%parent 30_12_2021
 
%%url https://lug-info.com/news/novogodnij-blic-sociolog-andrej-yakovenko-glavnoe-sohranyat-chelovecheskoe-dostoinstvo
 
%%author_id 
%%date 
 
%%tags donbass,dostoinstvo,lnr
%%title Новогодний блиц. Социолог Андрей Яковенко: "Главное - сохранять человеческое достоинство"
 
%%endhead 
\subsection{Новогодний блиц. Социолог Андрей Яковенко: \enquote{Главное - сохранять человеческое достоинство}}
\label{sec:30_12_2021.stz.news.lnr.lug_info.1.dostoinstvo}

\Purl{https://lug-info.com/news/novogodnij-blic-sociolog-andrej-yakovenko-glavnoe-sohranyat-chelovecheskoe-dostoinstvo}

Заведующий кафедрой социологии и социальных технологий Луганского
государственного университета имени Владимира Даля, доктор социологических наук
Андрей Яковенко в традиционном новогоднем блиц-интервью ЛИЦ делится своим
видением итогов уходящего года и прогнозом на год грядущий.

\ii{30_12_2021.stz.news.lnr.lug_info.1.dostoinstvo.pic.1}

- Как вы оцениваете положение и ситуацию в Республиках Донбасса к началу 2022
года?

- Ситуацию в нескольких словах можно определить как стабильную, но сложную.
Давление внешних факторов и особенно информационных атак, с навязываемыми
апокалипсическими раскладами, оказывают неблагоприятное воздействие на
социальное самочувствие людей.

- Что вы считаете наиболее значимыми достижениями Луганской и Донецкой Народных
Республик и каковы были самые серьезные проблемы в уходящем году?

- Полагаю, что наиболее значимыми достижениями социально-экономического
характера выступили: открывающаяся возможность легальной широкомасштабной
торговли промышленной продукцией местного производства с Российской Федерацией
и снятие таможенных границ между Республиками. А в общественном плане,
независимо от календаря, главным нашим достижением остается способность
сохранять человеческое достоинство в самых сложных условиях.

Ключевой же проблемой уходящего года считаю даже не бесконечную вялотекущую
войну и весь тяжелый социально-психологический фон неопределенности, в котором
мы живем уже много лет, а непростую эпидемическую ситуацию, вызванную
коронавирусом. Утраты, потерянное здоровье, ограничения в работе целых
социальных институтов – драматическая канва 2021 года. Скорее бы она
прервалась...

- Ваш прогноз развития ситуации вокруг Донбасса в 2022 году?

- Прогнозы по-прежнему делать весьма сложно. Просто потому, что судьба
Донбасса, как хорошо известно, в заложниках у логики глобального
противостояния. А оно пока не снижается. Поэтому, как обычно, на \enquote{столе} три
сценария: вялотекущий, при котором сохраняется ситуация \enquote{ни мира, ни большой
войны}, и два радикальных – военно-политический и сугубо политический.
Последние сценарии, о чем периодически говорят аналитики, могут кардинально
переформатировать ландшафт и не только, а, возможно, и не столько в Донбассе...

Впрочем, по нашей культурной традиции принято всегда излучать оптимизм и верить
в лучшее. Продолжим же ей следовать. Иначе будет совсем тоскливо. Поэтому всех
дорогих земляков и персонально коллектив Луганского Информационного Центра
позвольте поздравить: с Новым 2022-м годом! Пусть для всех он сложится лучше
уходящего!
