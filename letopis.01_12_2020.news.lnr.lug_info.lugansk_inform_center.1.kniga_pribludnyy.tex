% vim: keymap=russian-jcukenwin
%%beginhead 
 
%%file 01_12_2020.news.lnr.lug_info.lugansk_inform_center.1.kniga_pribludnyy
%%parent 01_12_2020
 
%%url http://lug-info.com/comments/one/sekretar-sp-lnr-andrei-chernov-v-luganske-podobnogo-roda-knigi-ne-izdavalis-foto-807
 
%%author ЛуганскИнформЦентр
%%author_id lugansk_inform_center
%%author_url 
 
%%tags lnr,knigi,poetry
%%title Секретарь СП ЛНР Андрей Чернов: "В Луганске подобного рода книги не издавались" (ФОТО)
 
%%endhead 
 
\subsection{Секретарь СП ЛНР Андрей Чернов: \enquote{В Луганске подобного рода книги не издавались} (ФОТО)}
\label{sec:01_12_2020.news.lnr.lug_info.lugansk_inform_center.1.kniga_pribludnyy}
\Purl{http://lug-info.com/comments/one/sekretar-sp-lnr-andrei-chernov-v-luganske-podobnogo-roda-knigi-ne-izdavalis-foto-807}
\ifcmt
	author_begin
   author_id lugansk_inform_center
	author_end
\fi

\index[writers.rus]{Приблудный, Иван!Поэт, (1905-1937)}

\ifcmt
tab_begin cols=4
	caption Книга воспоминаний и писем вдовы поэта Ивана Приблудного

pic http://img.lug-info.com/cache/7/3/(1)_IMG_20201130_105829.jpg/1000wm.jpg
pic http://img.lug-info.com/cache/7/5/Natalya_Milonova._Do_aresta.jpg/1000.jpg
pic http://img.lug-info.com/cache/e/1/Pamyatnik_I._Pribludnomu_v_Novoajdare.jpg/1000.jpg
pic http://img.lug-info.com/cache/d/d/(43)_1_(2).jpg/1000.jpg
pic http://img.lug-info.com/cache/f/c/Natalya_Milonova._1945_g._Magadan._Cherez_god_posle_osvobozhdeniya_iz_lagerya._(Foto_iz_arxiva_A.P.Malceva).jpg/1000.jpg
pic http://img.lug-info.com/cache/8/a/Ivan_Pribludnyj._1930-e_gg..jpg/1000.jpg
tab_end
\fi

\begin{leftbar}
	\bfseries
О значимости и актуальности представленной\Furl{http://lug-info.com/news/one/soyuz-pisatelei-prezentoval-izdannye-v-lnr-vospominaniya-vdovy-poeta-ivana-pribludnogo-62342} накануне книги воспоминаний и писем
вдовы поэта Ивана Приблудного Натальи Милоновой ЛуганскИнформЦентру
рассказывает один из ее составителей и редакторов, секретарь правления
Союза писателей ЛРН Андрей Чернов.
\end{leftbar}

\subsubsection{ОПУБЛИКОВАНЫ ВПЕРВЫЕ}

У книги говорящий заголовок, полностью отражающий ее содержание – в ней
представлены воспоминания и письма вдовы Ивана Приблудного (1905-1937) Натальи
Петровны Милоновой. Без какого-либо преувеличения можно смело сказать, что это
уникальное издание – в Луганске подобного рода книги, как мне известно, не
издавались.

\subsubsection{СРЕДИ ЗВЕЗД}

Наталья Петровна, в девичестве Зиновьева, познакомилась с Иваном, будучи
студенткой Высшего литературно-художественного института в Москве. Приблудный
учился на год старше. Сокурсником Натальи Петровны был Михаил Светлов, многие
талантливые писатели. Вот об этой среде, о напряженной культурной и творческой
жизни 1920-х годов рассказывает эта книга. Она вспоминает и о студентах, и о
преподавателях вуза, который, к слову, возглавлял знаменитый Валерий Брюсов.
Литературные вечера тогда проходили с участием всех звезд литературы:
Маяковский, Есенин, Пастернак, Асеев. И об этом подробно пишет Наталья
Петровна.

Конечно, особая тема – Есенин. Наталье Петровне довелось с Сергеем Есениным
видеться несколько раз, но с великим русским поэтом был близок Иван Приблудный,
некоторое время входил в \enquote{свиту} поэта. Важен и такой момент: отец Милоновой
Петр Зиновьев – известный психиатр, классик советской психиатрии. В 20-е годы
он был ассистентом профессора Ганнушкина. Именно к отцу Натальи Петровны
обратились в 1925 году близкие Есенина, чтобы положить поэта в клинику
Ганнушкина.

\subsubsection{СКОРОТЕЧНОЕ СУПРУЖЕСТВО}

Семейная жизнь Милоновой и Приблудного была коротка. После рождения ребенка,
единственного сына Приблудного, они развелись. После астраханской ссылки
Приблудного, тем не менее, Наталья Петровна прописала бывшего мужа по своему
адресу. Когда Приблудного арестовали в 1937 году, именно этот факт послужил
поводом к последующему аресту самой Натальи Петровны. В отличие от Приблудного,
который был расстрелян в 1937-м, Наталья Петровна была освобождена в Магадане в
1944 году. В это время она вступает во второй брак – с ученым, также отбывавшим
заключение, Юрием Милоновым.

Многое, о чем пишет Милонова, будет интересно и литературоведам, и любителям
литературы, желающим проникнуться духом того времени.

\subsubsection{МАЛЬЦЕВСКИЙ ВКЛАД}

Тираж книги – 100 экземпляров. Она поступит в библиотеки, музеи, вузы ЛНР и
России.

Сама рукопись воспоминаний Милоновой поступила в архив Анатолия Петровича
Мальцева в 1990 году. Направляя машинопись своих воспоминаний, Наталья Петровна
поставила обязательное условие: не публиковать их при ее жизни. После смерти
Милоновой в 2000 году Мальцев задумался о публикации текста воспоминаний. В
2007 году часть воспоминаний Милоновой начала публиковаться в луганском
альманахе \enquote{Крылья}. В альманахе воспоминания публиковались не полностью – из-за
большого объема они были сокращены.

В январе 2020 года, еще при жизни Мальцева, я с ним обсуждал планы издать эту
рукопись. Мы рассчитывали это сделать к юбилею Приблудного – 1 декабря 2020
года исполняется 115 лет со дня рождения поэта. К сожалению, 4 февраля Анатолий
Петрович Мальцев скончался. Я посчитал своим святым долгом завершить это важное
дело.

Мальцев хранил и берег эту рукопись, сделал многое для популяризации наследия
Приблудного. К сожалению, многое осуществить он не успел, пришлось завершать
мне. И мне очень не хватало его мудрого, чуткого слова.

Структуру книги мы с ним обсуждали не раз, занимались подбором иллюстраций.
Анатолий Петрович поручил мне подготовить комментарии по персоналиям,
упоминаемым в письмах и воспоминаниях.

\subsubsection{НЕОБХОДИМОСТЬ ПРИМЕЧАНИЙ}

Публикацию самих воспоминаний я сопроводил примечаниями, поскольку многие лица,
которых упоминает Милонова, сейчас малоизвестны. Поэтому была необходимость
разъяснить, о ком идет речь. К тому же, поскольку воспоминания она писала в уже
преклонном возрасте, то там закрались ошибки – неверно написаны фамилии, имена,
инициалы. Там, где это существенно, я постарался исправить эти неточности.

\subsubsection{ЗНАЧЕНИЕ ПРИБЛУДНОГО}

Приблудный принадлежит поколению молодой литературы периода Гражданской войны.
Он был ребенком, когда стал \enquote{сыном полка} при кавалерийской части Красной
Армии. Конечно, никакого систематического образования не получил. Но его талант
заметили и направили учиться в Москву. Сам Брюсов отметил, что Иван – \enquote{поэт
милостью Божьей}. Эта оценка дорогого стоит, не каждый получал столь высокую
оценку от родоначальника русского символизма.

Приблудный широко публиковался в 20-30 годы. Но после расстрела в 1937 году его
имя оказалось \enquote{запретным}. Даже после реабилитации – почти полное забытье. Это,
несмотря на такого покровителя, как Леонид Осипович Утесов, который даже пел
песни на стихи Приблудного.

Но подлинное возвращение наследия Приблудного началось только с конца 1970-х
годов. В Луганске огромную роль в этом деле сыграл Олег Бишарев. В 80-е и 90-е
годы он способствовал переизданию его стихов, написал несколько книг о
творчестве Приблудного.

\subsubsection{РОЛЬ МИЛОНОВОЙ}

Наталья Петровна в короткой жизни Ивана Приблудного сыграла очень большое
значение. Это его единственная жена, мать единственного ребенка. С ней он не
порывал общение даже после развода. Во время астраханской ссылки она была среди
немногих, кто проявлял заботу о поэте. Ведь в 30-е годы оказаться под
репрессией – это обречь себя на одиночество. Друзья в миг от тебя
отворачивались. Наталья Петровна этого не сделала по отношению к Ивану, хотя
они уже не были мужем и женой. Кстати, именно во время стояния в очередях под
тюрьмой она впервые узнала настоящее имя бывшего мужа, отца своего ребенка –
Яков Овчаренко. До этого была уверена, что "Иван Приблудный" – настоящее имя.

После реабилитации в 1956 году Наталья Петровна вместе со своим вторым мужем,
доктором экономических наук Милоновым занимались восстановлением памяти о
Приблудном. Принимали участие в литературных вечерах, записях телевидения,
ходатайствовали о переиздании книг и стихов.

Но самое важное, что она сделала – это написала эти воспоминания. Книга –
исключительный памятник Приблудному, рисующий его облик живо и правдоподобно,
без прикрас и искажений.

Иван Приблудный (настоящее имя Яков Петрович Овчаренко, 1905-1937) – русский
советский поэт. Принадлежал к кругу \enquote{новокрестьянских} поэтов, входил в
окружение Сергея Есенина. Репрессирован и расстрелян в 1937 году,
реабилитирован посмертно. 

\textbf{ЛуганскИнформЦентр — 01 декабря — Луганск}

