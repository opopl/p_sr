% vim: keymap=russian-jcukenwin
%%beginhead 
 
%%file moje.kremlevskie_narrativy.jazyk.pchela_i_shershen
%%parent moje.kremlevskie_narrativy.jazyk
 
%%url 
 
%%author_id 
%%date 
 
%%tags 
%%title 
 
%%endhead 

\paragraph{Басня 27 - Пчела и Шершень}

— Скажи мне, Пчела, отчего ты столь глупа? Знаешь, что трудов твоих плоды не
столько для тебя самой, сколько для людей полезны, но тебе часто и вредят,
принося вместо награждения смерть, однако не перестаете дурачиться в собирании
меда. Много у вас голов, но безмозглые. Видно, что вы без толку влюбилися в
мед.

— Ты высокий дурак, господин Советник,— отвечала Пчела.— Мед любит есть и
Медведь, а Шершень тоже лукаво достает. И мы бы могли воровски добывать, как
иногда наша братия и делает, если бы мы только есть любили.

Но нам несравненно большая забава собирать мед, нежели кушать. К сему мы
рождены и не перестанем, пока умрем. А без сего жить и в изобилии меда есть для
нас лютейшая смерть.

Сила. Шершень есть образ людей, живущих хищением чужого и рожденных на то одно,
чтоб есть, пить и проч. А пчела есть герб мудрого человека, в сродном деле
трудящегося. Многие шершни без толку говорят: для чего сей, например, студент
научился, а ничего не имеет? На что-де учиться, если не иметь изобилия?.. Не
рассуждая слов Сираха: «Веселье сердца — жизнь человеку»,— и не разумея, что
сродное дело есть для него сладчайшее пиршество. Взгляните на правление
блаженной натуры и научитесь. Спросите вашу борзую собаку, когда она веселее? —
Тогда,— отвечает вам,— когда гоню зайца. Когда вкуснее заяц? — Тогда,— отвечает
охотник,— когда гоняю.

Взгляните на сидящего пред вами кота. Когда он куражнее? Тогда, когда целую
ночь бродит или сидит возле норки, хотя, уловив, и не ест мышь. Запри в
изобилии пчелу, не умрет ли с тоски в то время, когда можно летать по
цветоносным лугам? Что горестнее, как плавать в изобилии и смертно мучиться без
сродного дела? Нет мучительное, как болеть мыслями, а болят мысли, лишаясь
сродного дела. И нет радостнее, как жить по натуре. Сладок здесь труд телесный,
терпение тела и самая смерть его тогда, когда душа, владычица его, сродным
услаждается делом. Или так жить, или должно умереть. Старик Катон чем мудр и
счастлив? Не изобилием, не чином — тем, что последует натуре, как видно в
Цицеро-новой книжечке «О старости». Сия одна есть премилогордная мать и
премудрая путеводительница. Сия преблагая домостроительница несытому дарует
много, а мало даст довольному малым.

Но раскусить же должно, что значит жить по натуре. Но закон скотских членов и
похотей наших, но значит блаженное оное естество, называемое у богословов
трисолнечное, всякой твари свою для нее часть и сродность вечно предписывающее.
О сем-то естестве сказал древний Епи-кур следующее: «Благодарение блаженной
натуре за то, что нужное сделала нетрудным, а трудное ненужным».

А поскольку в Боге нет мужского пола, ни женского, но все в нем и он во всем,
для того сказывает Павел: «Который есть всяческое во всем...» 

