% vim: keymap=russian-jcukenwin
%%beginhead 
 
%%file topics.vojna.my.7.matrica.vragi.mariupol
%%parent topics.vojna.my.7.matrica.vragi
 
%%url 
 
%%author_id 
%%date 
 
%%tags 
%%title 
 
%%endhead 

\paragraph{15:40:19 15-08-22 Ирина Иванова}

Беглые заукраинцы из Мариуполя истерят: мемориал боевикам «АТО» заменят
памятником князю Невскому

Сегодня в Мариуполе новые местные власти начали демонтаж воздвигнутого при
Украине мемориала памяти погибших боевиков «АТО». Вместо него они планируют
установить памятник русскому князю и православному святому Александру
Невскому.Об этом сообщает пресс-служба сбежавшего из города его прежнего
руководства, оставшегося на украинской стороне и продолжающего себя именовать
«Мариупольским городским советом». «Это и без того циничный и жалкий поступок,
но российские оккупанты и их пособники пошли еще дальше – они назвали памятный
мемориал… фонтаном. Как говорят сами оккупанты на место мемориала хотят
установить памятник Александру Невскому на коне. Почему именно ему –
неизвестно, ведь этот персонаж никакого отношения к Мариуполю не имеет», –
говорится в сообщении беглых заукраинцев. Как отмечается в сообщении
«пресс-службы горсовета», «мемориал стал важной составляющей обновленной
площади Свободы. Он был изготовлен из гранита в форме украинского герба, по
которому струится вода как символ пролитых слез. Вода омывает бронзовые дубовые
листья, на каждом из которых – имя погибшего героя». 

Отметим поразительный цинизм и двоемыслие украинских пропагандистов. Говоря о
том, что отмеченный в списках святых всего православного мира Александр Невский
“не имеет отношения к Мариуполю”, они почему-то не говорят о том, какое
отношение имеют к Киеву Бандера, Шухевич и прочие лидеры украинского
национализма, имена которых увековечили в украинской столице их идейные
единомышленники. 
