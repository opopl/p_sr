% vim: keymap=russian-jcukenwin
%%beginhead 
 
%%file 09_09_2021.fb.vasiljev_maksim.1.sud_nestle_jazyk_rabota
%%parent 09_09_2021
 
%%url https://www.facebook.com/permalink.php?story_fbid=4522050747846443&id=100001246887005
 
%%author_id vasiljev_maksim
%%date 
 
%%tags ekonomika,jazyk,kompania.nestle,mova,prinuzhdenie,rabota,sud,ukraina,ukrainizacia
%%title Cуд - Nestle - примушування використовувати в роботі російську мову
 
%%endhead 
 
\subsection{Cуд - Nestle - примушування використовувати в роботі російську мову}
\label{sec:09_09_2021.fb.vasiljev_maksim.1.sud_nestle_jazyk_rabota}
 
\Purl{https://www.facebook.com/permalink.php?story_fbid=4522050747846443&id=100001246887005}
\ifcmt
 author_begin
   author_id vasiljev_maksim
 author_end
\fi

СУД МОЖЕ СТЯГНУТИ З КОМПАНІЇ НЕСТЛЕ ПІВ МІЛЬЙОНА ГРИВЕНЬ ЗА ПРИМУШУВАННЯ
ВИКОРИСТОВУВАТИ В РОБОТІ РОСІЙСЬКУ МОВУ!

Подільський районний суд міста Києва відкрив провадження у справі за позовом
Artem Kravchenko – колишнього працівника компанії Nestlé про стягнення
моральної шкоди, через тривале порушення його прав, шляхом фактичного
примушування використовувати при виконанні трудових обов’язків російську мову
(замість державної).

\ii{09_09_2021.fb.vasiljev_maksim.1.sud_nestle_jazyk_rabota.pic}

Подібні порушення більшість людей звикли сприймати за норму, або просто не
хочуть ризикувати роботою, становищем і так далі.

Але історія з Артемом зовсім інша, та точно послугує прикладом для багатьох.

Велику кількість доказів тривалого порушення (протягом більше трьох років)
компанією Нестле законодавства України ми вже надали суду. Зокрема, –  це
відеозаписи, копії електронного корпоративного листування, робочих матеріалів,
медичну документацію, яка підтверджує постійні стреси та інші розлади здоров’я
Артема, через порушення його законних прав та вчинення тиску керівництвом
компанії, в зв’язку з регулярними вимогами Артема надавати робочу інформацію
державною мовою.

Нами також подане клопотання про виклик низки свідків, які можуть підтвердити
факти порушень, та буде вчинено ряд інших процесуальних дій.

Більше того, у даній справі ми готові йти до кінця, навіть до Європейського
суду з прав людини! 

Якщо українці прагнуть до створення ПРАВОВОЇ ДЕРЖАВИ, та бути частиною
демократичного Світу, то ми самі повинні робити для цього КОНКРЕТНІ ДІЇ. Ніхто
не прийде і помахом чарівної палички за нас це не зробить! 

Саме такі РІШУЧІ та САМОВІДДАНІ ДІЇ вчинив Артем Кравченко, який маючи
можливість спокійно змінити роботу, чи просто закрити очі на ігнорування
роботодавцем законодавства (зокрема, й Закону про мову), зайняв та відстоював
ТВЕРДУ ПРОУКРАЇНСЬКУ ПОЗИЦІЮ. Працюючи в компанії Нестле, він систематично,
протягом більше трьох років, добивався від компанії дотримання його прав
використовувати при виконанні трудових обов’язків державну мову – українську (а
не російську, яка нав’язувалась компанією на всіх рівнях). 

Артем відстоював навіть не так свої особисті права, як права ВСІХ УКРАЇНЦІВ, та
можливість для кожного з нас жити в країні, де їх можна захистити, а не лише
знати, що ці права десь задекларовані та записані на папері. 

Я, зі свого боку, допоміг Артему перевести цю ситуацію в правове русло,
розпочати судовий процес, та зроблю все, від мене залежне, для захисту прав
Артема в суді.

Боремося спільно з Володимир Наконечний та Руслан Творецький.

Хочу звернути увагу, що Артем родом з Кривого Рогу (Дніпропетровська область),
тому це ще раз доводить, що українці, від Заходу до Сходу, НЕ ДОПУСТЯТЬ
знищення української мови – базової підвалини нашої державності і дадуть гідну
відсіч будь-яким спробам перетворити Україну на малоросію! 

Ще Леся Українка казала: «Нація повинна боронити мову більше, ніж свою
територію. Втратити рідну мову і перейняти чужу – це найгірший знак
підданства». 

Тому всіх, кому не байдужа доля нашої країни, всіх хто прагне ВИРВАТИ ЇЇ З ЛАП
«рускава міра» і «какой разніци» та інтегруватися в ЄС, закликаємо поширити цю
публікацію та долучитися до даного проекту всіма можливими способами, зокрема,
ПІДТРИМАТИ нас під Подільським районним судом міста Києва в день судового
засідання – 30-го вересня 2021 року о 10 год. та висловити протест проти
компанії Нестле та всіх інших, хто витирає ноги об Конституцію України та Закон
про мову, та ігнорує права громадян використовувати державну мову у всіх сферах
суспільного життя.

(всіх бажаючих долучитися прошу контактувати з цього приводу з Володимир Наконечний). 

Все буде Україна!
