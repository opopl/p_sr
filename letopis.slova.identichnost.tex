% vim: keymap=russian-jcukenwin
%%beginhead 
 
%%file slova.identichnost
%%parent slova
 
%%url 
 
%%author 
%%author_id 
%%author_url 
 
%%tags 
%%title 
 
%%endhead 
\chapter{Идентичность}
\label{sec:slova.identichnost}

%%%cit
%%%cit_head
%%%cit_pic
%%%cit_text
А ведь это еще не все. Оказывается, в понятие \enquote{патриотического воспитания}
входят еще и формирование поддержки вступления в НАТО, и уважительное отношение
к \enquote{героям борьбы украинского народа за обретение независимости}.  Но самый
главный пункт в этом перечне – последний: Кабмин предлагает привлекать на
работу в органы государственной власти и местного самоуправления только граждан
со \enquote{сформированной национальной \emph{идентичностью}}. Проще говоря, если у тебя
украинский не родной, если ты не поддерживаешь вступление в НАТО и не
относишься уважительно к Бандере и Шухевичу, дорога в органы власти тебе
заказана
%%%cit_comment
%%%cit_title
\citTitle{Миллионы на украинскую идентичность, зрада на Евро-2020, планы Путина по Украине}, 
, strana.ua, 01.07.2021
%%%endcit

%%%cit
%%%cit_head
%%%cit_pic
%%%cit_text
От урока истории Юлия Мендель перешла к урокам языка.  Начала издалека – о том,
что язык в Украине стал символом борьбы против Москвы и способом получения
украинской \emph{идентичности}. Негодяи, сталкивающие украинцев, – это
пророссийские политики и националисты: \enquote{Використовуючи нашу травму
історичних заборон говорити рідною українською і російську пропаганду, спочатку
нас роз’єднували, я навіть не скажу проросійські політики, а ті, хто не міг
відірватися від вимені Кремля і не мислив Україну незалежною.  Як годиться, з
другого боку підпряглися націоналісти. На цьому гріх було не заробити політичні
бонуси}.  Мендель при этом уверена, что языки защищать вообще не нужно. Потому
что украинцы, мол, вообще друг друга не понимают на почве взаимной неприязни
%%%cit_comment
%%%cit_title
\citTitle{\enquote{Я їб...в Гітлера в зад}. Что пишет Юлия Мендель в своей книге о двух годах у Зеленского}, 
Екатерина Терехова, strana.ua, 07.07.2021
%%%endcit
