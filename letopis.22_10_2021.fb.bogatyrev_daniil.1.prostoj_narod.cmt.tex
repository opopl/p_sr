% vim: keymap=russian-jcukenwin
%%beginhead 
 
%%file 22_10_2021.fb.bogatyrev_daniil.1.prostoj_narod.cmt
%%parent 22_10_2021.fb.bogatyrev_daniil.1.prostoj_narod
 
%%url 
 
%%author_id 
%%date 
 
%%tags 
%%title 
 
%%endhead 
\subsubsection{Коментарі}

\begin{itemize} % {
\iusr{Юрий Лукшиц}
Мне часто говорят: "Я политикой не интересуюсь". Очень жаль, потому что политика интересы простых граждан всегда затрагивает.

\begin{itemize} % {
\iusr{Даниил Богатырёв}
\textbf{Юрий Лукшиц} Я вообще стараюсь избегать слова "простой" в таком контексте. Если гражданин, то уже не простой. А если "простой", то уже не гражданин, а какое-то недоразумение.
\end{itemize} % }

\iusr{Елена Пожалова}
Никому простой народ не нужен!

\iusr{Ольга Котелкина-Агарычева}

Мы считаем, что "простой народ" - это просто народ, без вот этих новых "голубых
кровей" просто гражданин, рабоияга. А они считают, что простой это значит
глупый, послушный раб.

\begin{itemize} % {
\iusr{Даниил Богатырёв}
\textbf{Ольга Котелкина-Агарычева} 

В политическом отношении "простой народ" - это политически-безграмотная масса.
Нужно всеми силами выводить людей из этого жалкого состояния и формировать у
них гражданское самосознание. Грамотные граждане - сила. "Простой народ" -
ресурс для политики, не больше.

% -------------------------------------
\ii{fbauth.karpec_aleksandr.ukraina}
% -------------------------------------

\textbf{Даниил Богатырёв}

\index{Свобода! Цитата! Достоевский, Великий Инквизитор}

Ну-ну...

«…Ничего никогда не было для человека и человеческого общества невыносимее
свободы!.. Нет заботы бесперывнее и мучительнее для человека, как, оставшись
свободным, сыскать поскорее того, пред кем преклониться… Нет у человека заботы
мучительнее, как найти того, кому бы передать тот дар свободы, с которым это
несчастное существо рождается. Но овладеет свободой людей лишь тот, кто
успокоит их совесть… Ибо тайна бытия человеческого не в том, чтобы только жить,
а в том, для чего жить… Свобода и хлеб земной вдоволь для всякого вместе
немыслимы, ибо никогда, никогда не сумеют они разделиться между собой! Убедятся
тоже, что не могут быть никогда и свободными, потому что малосильны, порочны,
ничтожны и бунтовщики… Спокойствие и даже смерть человеку дороже свободного
выбора в познании добра и зла… Нет ничего обольстительнее для человека, как
свобода его совести, но нет ничего и мучительнее… Чего ищет человек на земле,
то есть: пред кем преклониться, кому вручить совесть и каким образом
соединиться наконец всем в бесспорный общий и согласный муравейник, ибо
потребность всемирного соединения есть третье и последнее мучение для людей… Мы
заставим их работать, но в свободные от труда часы мы устроим им жизнь, как
детскую игру, с детскими песнями, хором и невинными плясками. О, мы разрешим им
и грех, они слабы и бессильны… Мы дадим им тихое и смиренное счастье, счастье
слабосильных существ, какими они и созданы… [Это] избавит их от великой заботы
и страшных теперешних мук решения личного и свободного…[от] прокляти[я]
познания добра и зла»

"Великий Инквизитор"

Подробности читайте на Фразе:

\href{https://fraza.com/analytics/254181-otkrovenija-proroka-fedora-k-195-letiju-dostoevskogo-}{%
Откровения Пророка Фёдора. К 195-летию Достоевского,  Александр Карпец, fraza.com, 11.11.2016%
}

\end{itemize} % }

\iusr{Ирина Макаренко}

А как найти синоним слова "простой", т.к. "сложный" звучит многозначно!
Подскажете?

\begin{itemize} % {
\iusr{Даниил Богатырёв}
\textbf{Ирина Макаренко} А зачем его искать?

\iusr{Ирина Макаренко}
\textbf{Даниил Богатырёв} ведь как-то надо себя идентифицировать, чтобы не писать "простой". Согласитесь, писать мы "сложный народ" звучит неоднозначно!
Может, "думающий"? @igg{fbicon.wink} 

\iusr{Даниил Богатырёв}
\textbf{Ирина Макаренко} На мой взгляд, слова "граждане" достаточно.

\iusr{Галина Тучина}
\textbf{Ирина Макаренко} Так в том то и дело,что далеко не думающий народ. И таких,к большому сожалению не мало !

\iusr{Ирина Макаренко}
\textbf{Галина Тучина} не мало, но не большинство!
Большинство, на мой взгляд, ленивы, равнодушны, трусливы, словом, "болото"!
\end{itemize} % }

\iusr{Ольга Иванова}

Простой народ однажды уже повелся на демократию и заботу у нем... Может, хватит за простака хлопотать, а?

\iusr{Олег Крило}

Даниил, ну хоть бы смягчил "закономерно", а не "заслуженно")

\iusr{Александр Вербицкий}

Жители Громад-это уже сообщества. В некоторых случаях, они разбираются в
житейских и философских вопросах лучше, чем те, кто ими управляет. Степень их
активности зависит только от сложившейся ситуации внутри сообществ. Простого
народа, как такового, нет. Это штамп, придуманный давно и эксплуатируемый
нещадно) Как нет и среднестатистического жителя (тн душа населения). Есть
-житель определенной территории с её конкретными особенностями. В противном
случае, можно было бы говорить о среднестатистическом враче, учителе, военном,
депутате ВРУ.

\iusr{Александр Карпец}

Хорошо сказано... И лет этак 20-25 примерно такие мысли приходилось
высказывать, подобные идеи исповедовать...

Но многолетний производственный и просто жизненный опыт опустил с неба на землю
в отношении реальной "сучности" и потенций маленького человечка из толпы, равно
как и толпы в целом...

С возрастом, несмотря на левые, откровенно марксистские убеждения, в подобных
вопросах все больше клонит к старику Фридриху нашему Ницше... ))

\begin{itemize} % {
\iusr{Даниил Богатырёв}
\textbf{Александр Карпец} Так я вполне в ницшеанском духе и пишу. В моих убеждениях "левизной" и не пахнет)

\iusr{Александр Карпец}
\textbf{Даниил Богатырёв} 

Я о себе, а не о вас. Это о левизне. То, что вы в ней ничего не смыслите - это
понятно)) Извините уж...

Но у вас и не ницшеанство. Ницше никогда не возложил бы надежды на рост
маленького человечка, ибо Ницше - это даже не Достоевский ))

А вообще, Ничше - это далеко не только и не столько, об аристократах духа,
всадниках на коне и толпе. Там есть много чего такого, что вообще мало кто
понимает и на что не обращают внимание...

\index{Ницше, Фридрих}

\href{https://fraza.com/analytics/208276-o_chem_govoril_zaratustra_k_170_letiju_fridriha_nitsshe}{%
О чем говорил Заратустра. К 170-летию Фридриха Ницше, Александр Карпец, fraza.com, 15.10.2014%
}

\iusr{Даниил Богатырёв}
\textbf{Александр Карпец} 

Благо, я и в "левизне" смыслю и в Ницше) Но понимать те или иные философские
взгляды - не означает разделять их. Мои убеждения - традиционалистские.

"Рост маленького человечка" - это крайне вольная и превратная интерпретация
написанного мной. Я писал не об этом, а о необходимости низложить призванный
оправдать собственную серость термин "простой народ", перейдя наконец к
развитию личностей. А то, что ими станут не все, и к тому же - станут в разной
мере - это само собой.

\iusr{Александр Карпец}
\textbf{Даниил Богатырёв} 

Софистика, уважаемый, софистика!

"необходимость низложить призванный оправдать собственную серость термин
"простой народ", перейдя наконец к развитию личностей" - это значит, возлагать
надежды на рост того самого человечка. Как говорится то же самое, только в
другой проекции ))

Извините, но сразу видно, что в толпе, на заводе. в колхозе, на лесоповале и
так далее вы не работали. Это не для обидеть вас, а констатация. У вас хорошее
книжное образование, но нет реального праксиса.

что же касается "понимания вами левизны", то позвольте сильно сомневаться. Как
вам, например,постановка вопроса о религиозном экзистенциализме Карла Генриха
Маркса? )) А душа марксизма именно здесь, а не в экономическом детерминизме и
классовых интересах... Это вообще мало кто понимает...

\index{Маркс, Карл! Философ}

\href{https://fraza.com/analytics/268772-karl-marks-psihoanalitik-i-religioznyj-ekzistentsialist}{%
Карл Маркс – психоаналитик и религиозный экзистенциалист, Александр Карпец, fraza.com, 08.05.2018%
}

\iusr{Даниил Богатырёв}
\textbf{Александр Карпец} 

На вопрос "как мне?" отвечу: "Странно", как и любому, кто знаком с
академическими трактовками марксизма и опытом их практического воплощения.
Марксизм атеистичен. И его воплощение в виде советского и иного коммунизма было
атеистично. Собственно, это и не удивительно, когда речь идёт об одной из
идеологий модерна. Как по мне, попытки "натянуть сову на глобус", объединив
марксизм и религиозность, вызваны желанием совместить стремление к понятой
количественно социальной справедливости с приматом сознания по отношению к
бытию (то есть, собственно, с религиозным мировоззрением). Хотя, я не исключаю,
что при желании в трудах Маркса можно отыскать и это. Статью прочту, когда
будет время. Однако, утверждать, что "именно это и только это и есть марксизм,
а все, кто так не считают, ничего в нём не смыслят" - крайне странно. Если бы
это было так, то Ленина и всех последующих коммунистов 20-го века пришлось бы
"исключить из марксистов" задним числом.

Но повторюсь: точка зрения интересная. Хотя и бесперспективная/бесполезная как
по мне. Возможно, современные сторонники коммунистической идеи её оценят, но
благо, я к ним не отношусь.

Что же до "заводов, колхозов и лесоповалов" - да, там я не работал. И не вижу в
этом ничего дурного. Если у человека есть некие склонности, способности и
интересы, то нужно развиваться именно в их ключе, а не тратить время на
бесполезный монотонный физический труд. Я вижу в этом, скорее плюс, чем минус.

\iusr{Александр Карпец}
\textbf{Даниил Богатырёв} 

Чтобы продвигать идеи в массы, нужно уметь не столько излагать их в шоу-румах
на ТВ, но и уметь с ними выходить в толпу. Настоятельно рекомендую
присмотреться к товарищу Ульянову-Ленину, который умел держать внимание
огромной толпы и направлять ее в нужном направлении. Извините. но это перевесит
всю нашу с вами ученость многократно.

Что касается "атеистичности марксизма", то здесь как бы расхожая банальщина, но
она примитивна и ограничена. Вопрос о светской нетеистической религиозности
Маркса поставлен давно и не мной. Одних Эриха Фромма и нашего
земляка-киевлянина Николая Бердяева достаточно, чтобы, как минимум,
прислушаться и отнестись серьезно )) А ДАйсэцу Тайтаро Судзуки при зачитывании
Экономико-философских рукописей и вовсе говорил, что "это дзен" ))) Маркс ведь
вышел далеко не только из Фейербаха и даже Гегеля. Там били еще, как минимум,
Шеллинг и Фихте.

Рекомендовал бы вам все-таки прочесть опус по приведенной выше ссылке Впрочем,
дело ваше...

К марксизму Ленина тоже есть много вопросов. Да-да. Во времена Ленина львиная
доля работ Маркса вообще была неизвестна не только в России, но и в Европе. Там
кое-что на сей счет есть по приведенной ниже ссылке...

\end{itemize} % }

\end{itemize} % }
