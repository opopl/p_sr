% vim: keymap=russian-jcukenwin
%%beginhead 
 
%%file slova.ukraina
%%parent slova
 
%%url 
 
%%author 
%%author_id 
%%author_url 
 
%%tags 
%%title 
 
%%endhead 

Так и живем в \emph{Украине}: удочек нет у власти для простых \emph{украинцев},
и рыбы нет для страждущих, \textbf{Еще никому из президентов не удалось объединить
Украину}, Александр Гончаров, strana.ua, 24.05.2021

Євробачення 2021: \emph{Україна} посіла п'яте місце, beztabu.net, 23.05.2021

\emph{Украину} нужно лишить права оценивать выступления участников от
страны-агрессора: это аморально, Игорь Кондратюк, obozrevatel.com, 23.05.2021

Уберите российский сапог и \emph{украинская} культура побьет все мировые
рекорды, Елена Кудренко, obozrevatel.com, 23.05.2021

\emph{Украинские} туристы подрались в самолете: мама младенца озвучила свою
версию конфликта, Марина Петик, obozrevatel.com, 24.05.2021

Не просто \emph{Україна}, а \emph{Україна}-Русь. Як Михайло Грушевський захищає
державні кордони?  radiosvoboda.org, 23.05.2021

Навіть попри те, що Путін не почав нову фазу завоювань в \emph{Україні} –
останній епізод кілька тижнів тому змусив людей у Вашингтоні і Києві
понервувати,
radiosvoboda.org, 24.05.2021

«Дайте нам грошей!»: большая \emph{украинская} мечта о великой халяве, from-ua.com, 24.05.2021

«Евровидение» в Европрессе: Италия — победитель, Германия и Британия  «пасут
задних», а \emph{Украину} назвали «кошмарной танцевальной угрозой», sharij.net,
24.05.2021

Зато выступление GoA от \emph{Украины} с песней «Шум» назвали одним из лучших
выступлений вечера,
sharij.net, 24.05.2021

Паралимпийская сборная \emph{Украины} заняла первое место на чемпионате Европы,
завоевав 37 золотых медалей, sharij.net, 23.05.2021

Кладбище империй: Чему научила США и \emph{Украину} война в Афганистане,
sharij.net, 24.05.2021

Для \emph{Украины}, где миллионы людей с разных сторон, Холодная война долго не сможет быть холодной,
\textbf{Дело Протасевича - еще один повод раздуть Холодную войну}, Денис Жарких, strana.ua, 24.05.2021

Вот я читаю посты некоторых российских патриотов, и не нахожу особой разницы
между ними и патриотами \emph{украинскими},
\textbf{Дело Протасевича - еще один повод раздуть Холодную войну}, Денис Жарких, strana.ua, 24.05.2021

Тобто. вже рік тому можна було бачити, що Північний потік 2 для \emph{України}
- програна справа, \textbf{Наш МИД живет в каком-то выдуманном мире}, Владимир
Воля, strana.ua, 24.05.2021

У келіях Печерського монастиря запалав світильник \emph{української} культури. Саме
тут бере свій першопочаток давня \emph{українська} література, художнє мистецтво,
медицина. Нестор-літописець – перший історик \emph{України}-Руси, автор Повісті, що є
головним джерелом вивчення \emph{української} історії, Агапій – перший відомий лікар,
Аліпій – перший живописець...
radiosvoboda.org, 06.05.2017

Шановні туристи, ми не заперечуємо вагомості внеску цього київського духовного
центру до скарбниці всіх східнослов'янських культур, але наголошуємо, що Лавра
– це передусім феномен \emph{української} культури».
radiosvoboda.org, 06.05.2017

Відвідувачі можуть запитати, чому у тексті Повісті Нестора Літописця (мова
твору церковнослов'янська) древня обитель іменується лише Печерською. Один із
найкращих знавців давньоруських літописів професор Василь Яременко стверджує –
це лише доказ того, що розмовною мовою автора була \emph{українська}. Саме тому
«\emph{українська} лексика ллється суцільним потоком у Повісті: жито, колодязь,
подружжя, туга, печера…» – аргументує професор.
radiosvoboda.org, 06.05.2017

\emph{Украина} была одной из главных составных частей Советского Союза — чрезвычайно
развитой в экономическом отношении, многолюдной и комфортной республикой.
\emph{Украинцы} по сути являлись второй имперской нацией, а выходцы из республики со
своими многочисленными группами поддержки порою оказывались на самых вершинах
власти, \textbf{Надоела Украина? Ежедневное «иди и смотри»}, Константин Кеворкян, ukraina.ru, 24.05.2021

Скоро в \emph{Украину} ворвется лето: синоптики заявили о \enquote{шикарной} погоде и сделали детальный прогноз, obozrevatel.com,
25.05.2021

На реках и водоемах с начала года утонуло более 300 \emph{украинцев}: причины жуткой
статистики, obozrevatel.com, 25.05.2021

\emph{Україна} і ЄС готують рішення про припинення польотів до і над Білоруссю.
Якими можуть бути інші санкції? radiosvoboda.org, 25.05.2021

«Це незграбно і підло. Винуватці зриву спецоперації \emph{українських}
спецслужб мають бути встановлені і покарані. А спроби їх «відмазати» марні», –
написав у твітері колишній міністр, radiosvoboda.org, 25.05.2021

\emph{Украина} может стать главным карателем Белоруссии, vz.ru, 25.05.2021

Сама \emph{Украина} попадет в патовую ситуацию, ведь для нее полеты над Россией
уже и так закрыты, а теперь закроется небо еще и над Белоруссией. Облетать
\emph{украинцам} придется много и долго, vz.ru, 25.05.2021

Важной является позиция \emph{Украины}, которая очень часто действует в русле западных политиков,
vz.ru, 25.05.2021

Уж если борщ не \emph{украинский}, Царит не наша правота – Расколот мир! В нём нет
единства, Повсюду мрак и темнота, odnarodyna.org, 25.05.2021

Бойцы, хранитель святыни, Над нами реет вечный стяг!  Неспевший славу \emph{Украине}
Державе нашей – лютый враг, odnarodyna.org, 25.05.2021

Эй, Макаревич, дорогуша, Ты лоб, пожалуйста, не морщь!  Ты \emph{украинцам} плюнул в
душу, А заодно попал нам в борщ!
odnarodyna.org, 25.05.2021

Но независимой державой \emph{Украйне} быть уже пора: И знамя вольности
кровавой, Я подымаю на Петра, Александр Пушкин, Полтава

Під впливом «Історії Русів» писали свої твори на \emph{українську} тематику
Кіндрат Рилєєв, Микола Гоголь, Тарас Шевченко. Використовував «Історію Русів» і
Пушкін, пишучи «Полтаву». Саме з цього \emph{українського} твору з'явилися і в
творі Пушкіна мотиви героїчної \emph{української} історії, мотиви боротьби
\emph{України} за незалежність, radiosvoboda.org, 25.05.2021

Храня суровость обычайну, Спокойно ведал он \emph{Украйну}, Молве, казалось, не
внимал, И равнодушно пировал, Александр Пушкин, Полтава

На вашому YouTube каналі одне із рекордсменів за переглядами – відео
\enquote{Чи матюкались давні \emph{українці}?}. Давайте розставимо крапки над і
щодо лайок в \emph{українській} мові. Чи винен в цьому умовний колективний
путін?  pravda.com.ua, 25.05.2021

Остап \emph{Українець}: Твоя індивідуальна \emph{українська} мова не має бути такою, як
вимагає неіснуючий \enquote{золотий стандарт}, pravda.com.ua, 25.05.2021

Часом ви дозволяєте собі говорити страшні речі. Про те, що \emph{українська}
мова не наймилозвучніша в світі, що вона не походить безпосередньо від
фінікійського письма, що лайки прийшли в неї не виключно як запозичення з
російської, а \emph{українською} не лише з мальвами розмовляють. Чому ви досі
не в списку ворогів нації? pravda.com.ua, 25.05.2021

Якщо розбирати, яке слово де-факто прийшло з \emph{української}, а яке
запозичене, вийдуть дуже прикольні речі. Вийде, що слово \enquote{щур} в
\emph{українській} мові скоріш за все з російської, а російське слово
\enquote{крыса} скоріш за все з \emph{української}, просто свого часу ми ними
помінялися, pravda.com.ua, 25.05.2021

Верховный суд \emph{Украины} принял решение о том, что в Ровенской области настоятели
двух захваченных храмов УПЦ останутся проживать вместе со своими семьями в
церковных домах, fraza.com, 25.05.2021

Сладкие парочки: Россия и \emph{Украина}, mikle1.livejournal.com, 21.11.2009

Даже я понял, что он лично виноват в том, что Россия до сих пор не развалилась
на демократические Чечни и Запопинские республики Окрайны, Китайны и прочая.
Еще он виноват в \emph{украинском} голодоморе 1932 года, монголо-татарском
нашествии и проигрыше Мазепы в 1709 году. Змею, укусившую вещего Олега тоже
тренировал он, mikle1.livejournal.com, 21.11.2009

Такие вот культы личности с обратными знаками. 2 страны, 2 личности, 2 системы
ценностей и 2 результата. Все смелые, прогрессивные, правильные, модные,
продвинутые, революционные и прочая говорят, что лучше \emph{украинский}
вариант, mikle1.livejournal.com, 21.11.2009

Не возвращайся в \emph{Украину}!  Угроза жизни дышит в спину, В затылок дышит
подлецо – Фашизма подлое лицо! Юнна Мориц, www.owl.ru

Василь Шкляр: «\emph{Українською} заговорять навіть в Африці, якщо без неї неможливо
буде повноцінно жити», globlvillage.com, 15.07.2020

Тим часом, \emph{Україна} вже не вперше у своїй історії опинилася сам на сам з
найкровожерливішим агресором. Соромно бачити, як плазує перед Росією Франція,
як стримано поводиться Німеччина, наче вони досі не оговталися від давніх
поразок у війнах з цим монстром, Василь Шкляр, globlvillage.com, 15.07.2020

Зеленский уже и не скрывает, что он является марионеткой в руках иностранцев и
все делает в их интересах, а не интересах \emph{Украины} и \emph{украинцев}.
Вот честно и открыто рассказывают как будут продавать землю иностранцам о чем
ранее молчал, Ростислав Кравец, strana.ua, 25.05.2021

\enquote{Руслан и Людмила} в вольном пересказе первого премьер-министр
\emph{Украины} Витольда Фокина. Цікаво. Чудово, www.skadtalk.cc, 21.06.2020

\enquote{Господи милосердний, \emph{Україна} – це не Раша}, Ірина Фаріон,
obozrevatel.com, 16.06.2019

Недогромадяни. Як колаборантки викидали \emph{українські} паспорти і нарвалися
на неприємності, \url{https://www.youtube.com/watch?v=yMFZEZLmnec} 21.05.2021,

Викинула \emph{український} паспорт – отримала проблеми. Чому колаборантів
варто позбавити права голосу? - Сергей Стерненко,
\url{https://www.youtube.com/watch?v=DR-IeF3SKA0}, 20.05.2021

Цей пам'ятний день було запроваджено Організацією \emph{українських} націоналістів на
честь наших Героїв, щоб ми, \emph{українці}, про них не забували. Щоб від покоління до
покоління, до наших дітей і внуків передавалася \emph{українська} традиція, віра,
патріотизм, виховання, національна пам'ять та національні Герої, – наголосив
Олег Тягнибок, zz.te.ua, 25.05.2021

\enquote{Ты на \emph{Украине} сейчас находишься! Ты понял? Ты понял меня? Ты на
\emph{Украине} находишься!}, Менти настільки \enquote{переатестувались}, що
підзабули уроки Майдану, Євген Дикий, gazeta.ua, 23.05.2021

Але ні, ми тут, в \emph{Україні}, для чогось робили Революцію гідності - зокрема для
того, щоб подібні нащадки НКВД не мали жодного шансу безкарно вбивати нас.
Здається, за 7 років вони настільки \enquote{переатестувались}, що підзабули уроки
Майдану, Євген Дикий, gazeta.ua, 23.05.2021

Заявляю, что я нахожусь в \emph{Украине} и не собираюсь прятаться от
правосудия. Я и дальше буду участвовать в законных следственных действиях и
добиваться справедливости как для себя, так и для всех \emph{украинских}
избирателей, доверивших мне высокое звание народного депутата \emph{Украины}»,
— подчеркнул оппозиционный политик, fraza.com, 25.05.2021

Єдине «тішить» у цій ситуації, що й \emph{українці} дали «достойну відповідь»
слов'янам. \emph{Українські} глядачі лише проголосували за виступ від однієї
слов'янської країни – Росії. Хоча Росію (що показово!) представляла співачка,
яка не є слов'янкою. Але ж \emph{українці} без «братньої Росії» жити не можуть – хоча
й воюють із нею. Тому й дали співачці Маніжі від Росії 4 бали. Добре, що не 12, 
Про слов'янську солідарність? Петро Кралюк, day.kiev.ua, 24.05.2021

Участник кампании по вакцинации публичных людей от коронавируса в \emph{Украине} не
может найти вторую дозу препарата, strana.ua, 25.05.2021

В половине регионов \emph{Украины} приостановлена вакцинация от Covid-19,
strana.ua, 24.05.2021

Ковидные качели. В \emph{Украине} снова пошел рост числа новых зараженных. За
сутки плюс 3 395 человек, strana.ua, 26.05.2021

... и в целом, в \emph{Украине} властям можно всё, \textbf{Украина легализовала
сомалийские действия государства в воздухе}, Елена Лукаш, strana.ua, 26.05.2021

- Александр Гарриевич, не \enquote{в \emph{Украине}}, а \enquote{на
\emph{Украине}}. Это важно. \enquote{В \emph{Украине}} говорят
\emph{украинские} сепаратисты, а сторонники триединого русского мира говорят
\enquote{на \emph{Украине}}, Мак Сим, zen.yandex.ru, 08.04.2021

Ржу и плачу. У меня, в Беларуси, 15 суток за это. У немцев, наверное, не
меньше. На \emph{Украине} какая то собственная Европа?
Игорь Правлоцкий, комментарий к видео, \textbf{Посадки
самолётов от Порошенко до Лукашенко. Что нужно понимать по скандалу с
задержанием Протасевича}, Олеся Медведева, strana.ua, 24.05.2021

Предположительно, президент Беларуси под территорией, \enquote{где угробили 300
человек} имеет в виду \emph{Украину} - 17 июля 2014 года, в небе над Донбассом был
сбит самолет, который следовал из Амстердама в Куала-Лумпур, strana.ua, 26.05.2021

\enquote{Кто платил подонку, который убивал людей в братской Украине?} -
отметил Лукашенко, \textbf{\enquote{Убивал людей в братской Украине}. Лукашенко
впервые прокомментировал задержание Протасевича}, strana.ua, 26.05.2021

Зеленский готовит закон о поражении \emph{Украины} в войне России, Владимир
Воля, strana.ua, 26.05.2021

В \emph{УССР} не было школы и садика, где дети не учили бы \emph{украинский}
фольклор, Андрей Манчук, strana.ua, 26.05.2021

Богдана була носієм дуже важливих цінностей. Не тільки журналістських, але
цінностей ідентифікації людини до \emph{українського}, до \emph{українських}
тем, до \emph{української} позиції, до \emph{українського} вільнодумства, з
точки зору несення своїх персональних цінностей гідности, \textbf{У Києві
презентували книгу спогадів про журналістку Радіо Свобода Богдану Костюк},
radiosvoboda.org, 26.05.2021

\emph{Украинец} установил рекорд, внедрив с свое тело восемь чипов. С их помощью он
заводит авто и открывает квартиру, strana.ua, 26.05.2021

Іван Франко – титан \emph{українського} слова і \emph{української} думки,
Михайло Цимбалюк, censor.net.ua, 26.05.2021

Ми мусимо навчитися чути себе \emph{українцями} — не галицькими, не
буковинськими \emph{українцями}, а \emph{українцями} без офіціальних кордонів.
І се почуття не повинно у нас бути голою фразою, а мусить вести за собою
практичні консеквенції. Ми повинні — всі без виїмка — поперед усього пізнати ту
свою \emph{Україну}, всю в її етнографічних межах, у її теперішнім культурнім
стані, познайомитися з її природними засобами та громадськими болячками і
засвоїти собі те знання твердо, до тої міри, щоб ми боліли кождим її частковим,
локальним болем і радувалися кождим хоч і як дрібним та частковим її успіхом, а
головно, щоб ми розуміли всі прояви її життя, щоб почували себе справді,
практично частиною його, \textbf{Одвертий лист до галицької молодежі}, Іван
Франко, 1905

Не забувайте, що ми досі в Галичині жили з національного погляду крайнє
ненормальним життям. Велика більшість нашої нації лежала безсильна,
закнебльована, а ми, маленька частина, мали свободу рухів і слова. І нам іноді
здавалося, що ми – вся \emph{українська} нація, що ми її чільні ряди, її
репрезентанти перед світом. Тепер, коли на російській \emph{Україні} не
сьогодні то завтра повстануть десятки таких центрів, якими тепер являються
Львів та Чернівці, ся наша передова роля скінчилася, \textbf{Одвертий лист до
галицької молодежі}, Іван Франко, 1905

Якщо це Русь, то Русь, а не \emph{Україна}. Якщо це Московія, то Московія, а не
Росія, \textbf{Життя нації. Використання топонімів Русь, Московія, Україна,
Росія}, zrada.org, 12.03.2021

\emph{Украина} в топ-5 стран мира по количеству сомнительных бот-сетей в Фейсбуке, 
strana.ua, 26.05.2021

Число смертей от коронавируса в \emph{Украине} перевалило за 50 000 человек с начала
эпидемии, strana.ua, 27.05.2021

Соцопросы регулярно показывают, что у нас процентов 70 считают - \emph{Украина}
движется в неправильном направлении. Причем, в какой-то момент этот процент
социологи фиксируют при всех президентах \emph{Украины}, при всех Кабминах и Верховных
Радах. То есть, \emph{украинцы} из раза в раз выбирают во власть людей, которыми потом
недовольны, \textbf{Протасевич нам важнее, чем состояние собственной экономики}, Владислав Михеев, 
strana.ua, 27.05.2021

Возмущаться бардаком на соседнем участке, подглядывая за соседом в дыру забора,
для \emph{украинцев} почему-то намного органичнее, чем навести порядок на своем
огороде, \textbf{Протасевич нам важнее, чем состояние собственной экономики}, Владислав Михеев, 
strana.ua, 27.05.2021

\emph{Україна} – не США, проте ми країна, яка воює і відстоює свою
незалежність. Це те, що притягує людей в армію, \textbf{Чому українські
військові йдуть з армії},  pravda.com.ua, 27.05.2021

Падение капитальных инвестиций в \emph{Украине} - просто беспрецедентное,
Александр Гончаров, strana.ua, 27.05.2021

14 травня в \emph{Україні} вперше відзначили День пам'яті \emph{українців}, які
рятували євреїв під час Другої світової війни, \textbf{«Через страх перед
німецькою владою»}, Андрій Усач, zaxid.net, 26.05.2021

\emph{Украина} не смогла нажиться на авиационной блокаде Беларуси, Михаил
Чаплыга, strana.ua, 27.05.2021

\enquote{Мне в голову приходит госпожа Меркель и ее походы за продуктами. Мне
хочется, чтобы \emph{Украина} была как Германия}, - добавила она, \textbf{\enquote{Не
может пройти 30 метров}. На вокзале Киева ради удобства Порошенко перегнали
поезд на другую колею}, strana.ua, 27.05.2021

Первые 224 \emph{украинца} получили уколы еще незарегистрированной в стране
вакцины Johnson \& Johnson, strana.ua, 27.05.2021

\emph{Пенсионер} получает в Украине в среднем 3500 грн в месяц, а распорядители
пенсий по всей \emph{Украине} по 1800 баксов.  Я уверен, что ни в Польше, ни в
Литве государственный клерк не получает таких зарплат. А сколько же тогда
получает директор департамента? А сколько трудяги центрального аппарата?, -
удивляется киевский предприниматель Павел Себастьянович, который на своей
странице в Фейсбуке поделился скрином декларации чиновницы, \textbf{Было 12
тысяч, стало 30. Как при Зеленском пошли в рост зарплаты чиновников, опережая
частный сектор}, strana.ua, 26.05.2021

Госдума России призвала мир осудить тотальную \emph{украинизацию}, strana.ua,
20.01.2021

Терористи вважають Лукашенка негідним. А 30\% \emph{українців} - найкращим
президентом, Сергій Фурса, gazeta.ua, 25.05.2021

На Донбассе снайпер террористов убил \emph{украинского} воина, obozrevatel.com,
27.05.2021

Папа Римский обеспокоен эскалацией насилия на востоке \emph{Украины},
strana.ua, 18.04.2021,

В этом законе сегодня вы закладываете очень серьезную бомбу против командиров
\emph{украинских} Вооруженных Сил. С большим трудом мы вытащили сержанта Маркива,
которого схватили, задержали за рубежом. По этому закону, который вы сейчас
проголосуете, могут задержать его командира роты, его командира батальона и
командующего Нацгвардии, \textbf{До 15 лет за участие в боевых действиях на Донбассе. Кого будут судить по закону о военных преступлениях}, strana.ua, 27.05.2021

Запад говорит \emph{Украине}: \enquote{Ребята, может, вы, наконец, приберетесь
дома}, hvylya.net, 27.05.2021

Сегодня Петр Порошенко напомнил, что он чемпион \emph{Украины} по неявке на
допросы, \textbf{Навстречу солнцу и прочь от допроса. Как Порошенко уехал от
СБУ в клубничный тур на Донбасс}, strana.ua, 27.05.2021

\emph{Украинский} политический пиар – бессмысленный, беспощадный и обязательно
неадекватный, - пишет журналист Вячеслав Чечило, \textbf{Навстречу солнцу и
прочь от допроса. Как Порошенко уехал от СБУ в клубничный тур на Донбасс},
strana.ua, 27.05.2021

Оно ведь как? Все «патриоты» \emph{Украины} жаждут призвать к ответу режим
Лукашенко, но чужими руками и за чужой счет. На что у союзного интуриста
бюджетов не предусмотрено, то с радостью оплатит \emph{украинский}
налогоплательщик. Хочет он того или нет.  Хорошо устроились, дорогие борцы!
\textbf{\emph{Украинский} патриот жаждет бороться с режимом Лукашенко - но чужими
руками}, Максим Могильницкий, strana.ua, 27.05.2021

\emph{Украина} - как феодальная Англия времен шерифа Ноттингема, только без Робин
Гуда, Алексей Кущ, strana.ua, 27.05.2021

Ще у 2017 році Верховна Рада звернулась до Конгресу США з проханням надати
\emph{Україні} статус основного союзника поза НАТО, але, як відомо, наша країна
цей статус так і не отримала. По – перше, США не схильні підписувати угоду, яка
би вимагала від них прийти на допомогу \emph{Україні}. По – друге, американська
еліта не хоче надавши цей статус \emph{Україні}, підштовхнути Росію до Китаю,
\textbf{Битва за \emph{Україну}. Хто винен, та що робити?}, Стефан Закревський,
hvylya.net, 27.05.2021

Битва за \emph{Україну}. Хто винен, та що робити? hvylya.net, 27.05.2021

Співробітники компанії з країни-агресора працюватимуть в \emph{Україні} «за
обміном», \textbf{«А як \emph{українською} «дружба?»: агенція, що розробила бренд
Ukraine NOW, привезла до Києва російських рекламників}, kyiv.media, 27.05.2021

В США расследуют вмешательство \emph{Украины} в президентские выборы-2020 на
стороне Трампа, strana.ua, 28.05.2021

\emph{Украинские} дипломаты сейчас пытаются помочь с возвращением \emph{украинцам}, которые
лечатся в Беларуси и \enquote{застряли} там из-за прекращения авиасообщения,
\textbf{Кулеба объяснил, почему Киев не отзывает посла из Минска после ареста
Протасевича}, strana.ua, 28.05.2021

Сытник заявил, что директора НАБУ должны выбирать иностранцы. «Для сохранения
институциональной независимости Национального антикоррупционного бюро
\emph{Украины}», \textbf{Послушать Сытника, так \emph{украинскому} народу
нельзя доверять государство}, Вячеслав Чечило, strana.ua, 28.05.2021

Олег Скрипка назвал \enquote{не \emph{украинцами}} тех \emph{украинских}
артистов, которые поют песни на русском, strana.ua, 28.05.2021

А есть те, что ездят в Россию. Пусть зарабатывают, но для меня они не являются
\emph{украинцами}, они просто резиденты \emph{Украины}, они не разговаривают на \emph{украинском},
они не находятся в мировоззрении \emph{украинском}, они не живут в \emph{украинском} мире.
Они живут в паттернах, в постсовках. Я иногда, может, неправильно говорил на
этот счет или не так интерпретировали мои слова, что таких людей нужно
изолировать, - сказал Скрипка, strana.ua, 28.05.2021

\enquote{Это резиденты \emph{Украины}, это не \emph{украинские} артисты. Это
люди, которые живут на этой территории, но они не \emph{украинцы}. Это чисто
моя личная точка зрения. Для меня этот вопрос не является вопросом. Люди тут
живут, но они являются частью русского мира}, - сказал Олег Скрипка, strana.ua, 28.05.2021
