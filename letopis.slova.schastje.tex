% vim: keymap=russian-jcukenwin
%%beginhead 
 
%%file slova.schastje
%%parent slova
 
%%url 
 
%%author 
%%author_id 
%%author_url 
 
%%tags 
%%title 
 
%%endhead 
\chapter{Счастье}

%%%cit
%%%cit_head
%%%cit_pic
%%%cit_text
Нужно просто жить свободно и \emph{счастливо}.  Покидая Ливингстон - город основанный
беглыми рабами, я смотрел на остовы погибших кораблей. На птиц, что напоминают
души людей, освободившихся от рабства.  Вот почему жители этого городка на краю
Земли, выглядят более свободными и \emph{счастливыми}, чем жители «цивилизованных»
столиц? Почему в их взгляде нет злобы и высокомерия? Может потому что они не
переписывают никаких Деклараций о Свободе, а просто живут свободно и \emph{счастливо}?
%%%cit_comment
%%%cit_title
\citTitle{Для того, чтобы быть счастливыми, не надо переписывать Декларации о Свободе / Лента соцсетей / Страна}, 
Константин Стогний, strana.ua, 06.07.2021
%%%endcit

