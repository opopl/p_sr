% vim: keymap=russian-jcukenwin
%%beginhead 
 
%%file 17_12_2021.fb.fb_group.story_kiev_ua.1.kiev_oktjabrskij_dvorec.cmt
%%parent 17_12_2021.fb.fb_group.story_kiev_ua.1.kiev_oktjabrskij_dvorec
 
%%url 
 
%%author_id 
%%date 
 
%%tags 
%%title 
 
%%endhead 
\zzSecCmt


\begin{itemize} % {
\iusr{Margarita Kaminsky}

Thank you. I did not know the history of this palace.

\iusr{Victoria Novikov}

Спасибо, очень интересно. Побывала в нем очень много раз на концертах. Когда
Ленком привёз спектакль « Юнона и Авось» ( чуть ли не 40 лет назад!), то я
смотрела его, по-моему, именно в Октябрьском.

\iusr{Gennady Henry Sergienko}

Дякую за публікацію. Завжди приємно прочитати про перше місце роботи, хоча в
історії палацу є і сумні сторінки...

\iusr{Tatyana Mazerati}
spasibo za interesniy ocherk

\iusr{Chyzhyshyn Oksana}
Дякую, пізнавально.

\iusr{Юрий Стебельский}

Вот и стали прозывать петербуржца Дмитрия Бегичева Дмытром.  Скоро уже и Михаил
Афанасьевич Булгаков станет на селянский манер Мыхайлом Опанасовычэм, а там
глядишь и Джо Байден будет Йосыпом.

\begin{itemize} % {
\iusr{Roman Sytnyk}
\textbf{Yuriy Stebelskiy} А не соромно носити \enquote{селянське} ім'я Юрій (Юрко) замість Георгій (Джордж)? В мене два прадіди-селяни були Юріями.

\iusr{Вера Милашич}
\textbf{Юрий Стебельский} Торговый бренд Михаил Воронин так им и остался, как запатентован, так и есть, только биг борды написаны укр. буквами. Как хотите так и произносите вам никто не мешает. Не создавайте бурю в стакане воды!

\iusr{Darya Chertkova}
\textbf{Yuriy Stebelskiy} так Юрко!
\end{itemize} % }

\iusr{Татьяна Вакуленко}

Дворец моего детства и юности! Ансамбль бального танца \enquote{Ритм} под руководством
Валерия Николаевича Корзинина, Зайцевская \enquote{Горлица} и \enquote{Ветерок}, которым руководил
Дмитрий Семенович Подвысоцкий.

Огромная благодарность замечательным педагогам, которые ввели в мир танца тысячи детей!

А на елках мы выступали с самого детства)

\iusr{Нина Кубанова}
\textbf{Татьяна Вакуленко} СПАРТАК.-спортивная гимнастика. Ну и ...елки

\iusr{Галина Шабалтас}

А я - пела в хоре «Щедрик» при Октябоьском дворце! Мы обожали наш дворец, с его
мраморными лестницами и колоннами, старинными креслами, множеством укромных
местечек, которые мы исследовали в перерывах между репетициями. Однажды нас
послали в костюмерную, которая находилась где-то с тыльной стороны дворца ,мы
спускалиськ куда-то, потом долго поднимались, очутившись в мрачном коридоре, с
рядом дверей, расположенных на небольших расстояниях друг от друга,... «Как в
тюрьме»,- сказал кто-то, стало вдруг страшно, аж мороз по коже... Энергетика
здесь была совсем другая... мы быстро взяли костюмы и убежали оттуда. Спустя много
лет, я узнала о замученных в этом здании людях, годы прошли, а стены все
помнят!

\begin{itemize} % {
\iusr{Татьяна Ховрич}
\textbf{Галина Шабалтас}, 

а я танцевала, сначала, в детском ансамбле \enquote{Ветерок}, затем - в ансамбле
народного танца \enquote{Горлица}. Золотые были времена!

\iusr{Галина Шабалтас}

\textbf{Татьяна Ховрич} 

Лет в пять я думала, что мое призвание - балет! Мама отвела меня в правое крыло
Октябрьского дворца, где находилась студия балета. Ее руководитель, делая всем
замечания во время первой репетиции, сказала, что у меня «дурацкие коленки»! Я
разобиделась, мир лишился «великой» балерины! А уж через много лет я стала петь
в хоре... и не жалею, с ним связано столько чудесных воспоминаний! Так что Левое
крыло дворца - победило!

\end{itemize} % }

\iusr{Ирина Художница}
Чудова історія

\iusr{Larysa Karbysheva}

А мне как то неприятно подчёркивание того, что бомбила советская авиация.... Под
немцами оставаться надо было что ли?

\begin{itemize} % {
\iusr{Anton Shumsky}
\textbf{Larysa Karbysheva} за совєтів у підвалі Жовтневого палацу була влаштована катівня. Тому не дивно, що для когось совєти - такий самий ворог, як нацисти.

\iusr{Larysa Karbysheva}
\textbf{Anton Shumsky} А при советах инопланетяне были? Нет, все те же украинцы, наши родичи, бабушки и дедушки..

\iusr{Надежда Владимир Федько}
\textbf{Larysa Karbysheva} Так. Розстрілювали євреї, українці, росіяни - наші сородичі. Це так.

\iusr{Fabrizio Conti}
за копейки стирают из памяти нашу историю подменяя понятия и ход событий

\iusr{Надежда Владимир Федько}
\textbf{Larysa Karbysheva} Поясніть мені, будь ласка, різницю між \enquote{більшовицькою} і \enquote{нацистською} окупацією. Заздалегідь Вам вдячний)
\end{itemize} % }

\iusr{Ірина Бондарєва}

Моя прабабуся Серафима вчилася тут, в Інституті благородних дівчат. А я в
юності співала у Народній капелі при Жовтневому палаці, приходила кілька разів
на тиждень на репетиції і дуже була щаслива від того. У нас були два прекрасних
педагога з консерваторії. Після репетицій ми йшли з дівчатами Хрещатиком і
співали... настрій і почуття були надзвичайні... @igg{fbicon.heart.beating} 


\iusr{Ольга Литвиненко}
Дякую, цікаво...

\iusr{Наталия Руденко}

Я занималась в студии художественного слова 10 лет, пока училась в школе,
преподавателем была Татьяна Афанасьевна. Прекрасные воспоминания, участие в
новогодних спектаклях, концертах, катание на широких лестничных перилах дворца!


\iusr{Ines Kyiv}
Дякую ! Оригінальні плани бачу вперше. Дуже цікаво !

\iusr{Светлана Трегубова}

Новогодние каникулы билет с подарком в \enquote{Октябрьский} на Ёлку, это было Здорово!
Всегда чудесные впечатления. Потом был Хор \enquote{Щедрик}! Теперь
концерты, спектакли, творческие вечера и Всегда очень трепетно, торжественно
потому что это \enquote{Октябрьский}!


\iusr{Александр Иванов}

Красно дякую за матерiал. Одне питаннячко, що змусило Вас змiнити стать в
такому поважному вiцi? З.I. Пiдтримую трансгендерiв повнiстю!

\begin{itemize} % {
\iusr{Надежда Владимир Федько}
\textbf{Александр Иванов} Якби я не був шляхетною людиною, то відповів би вам круто російським матом!
Це акаунт моєї дружини, яка відійшла за завісу в липні цього року! Я продовжую вести нашу спільну сторінку! Тому на акаунті два імені - Надія і Володимир Федько.
А трансгендеристам місце у психіатричній лікарні!
Уфф... Ледве утримався в рамках пристойності, щоб Адміністратори не забанили мене.

\iusr{Александр Иванов}
\textbf{Надежда Владимир Федько} Примите мои соболезнования(
\end{itemize} % }

\iusr{Людмила Скомаровська}
Дякуємо за чудовий коментар!

\iusr{Ada Dubinski}

Да в детстве была пару раз в Октябрьском дворце приглашение на ёлку с подарком
какое было веселье радость подарок были содержал Мандаринки конфеты

\iusr{Ольга Dzhun}

В Октябрьском были чудесные елки для детей: анимация перед спектаклем, сам
спектакль, подарки, новогднее настроение, яркие и теплые воспоминания


\iusr{Наталья Драгальчук}
Дякую! Дуже цікаво та змістовно.

\iusr{Alexey Novozhylov}

Дякую за цікаву розповідь, колись працював в компанії, що мала офіс в цьому
палаці з довгим балконом, що виходив на майданчик з кінотеатром, ще до
встановлення меморіалу закатованим чекістами.


\iusr{Людмила Даниленко}
Дякую, я теж бувала у палаці на новорічні свята, дуже зворушливи спогади дитинства...

\iusr{Надежда Владимир Федько}

Друзі, можливо у когось збереглися фотографії з новорічної ялинки у Жовтневому
палаці...  Буде дуже цікаво подивитися.

\end{itemize} % }
