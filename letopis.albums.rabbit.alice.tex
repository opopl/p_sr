% vim: keymap=russian-jcukenwin
%%beginhead 
 
%%file albums.rabbit.alice
%%parent albums
 
%%url 
 
%%author_id 
%%date 
 
%%tags 
%%title 
 
%%endhead 

%http://ukrbooks.com/ua/alisa_v_kraini_chudes/

\begin{center}
\Large\bfseries\emph{В КРОЛЯЧІЙ НОРІ}
\end{center}

Алісі набридло сидіти без діла на березі. Разів зо два вона зазирнула в книжку,
що її читала сестра, але там не було ні рисунків, ні малюнків, а без них книжка
не книжка, — гадала Аліса.

\ifcmt
  ig https://scontent-fra3-1.xx.fbcdn.net/v/t1.6435-9/35477439_1747963305258960_5188267570980978688_n.jpg?_nc_cat=105&ccb=1-7&_nc_sid=5cd70e&_nc_ohc=1yNCqoMNB6cAX_0QrV4&_nc_ht=scontent-fra3-1.xx&oh=00_AfAee0215dsau18FlzdRaMV8WgbuAyt1tVEd27EE9jLJyQ&oe=6464FDB8
  @width 0.4
  @minipage 0.4
  @wrap \parpic[r]
\fi

Літня спека зовсім розварила й розморила її. Устати б оце, нарвати стокроток та
віночка сплести? Так ліньки, не хочеться ворушитись... Коли це де не взявся
білий кролик з рожевими очима, пробіг поперед нею.

Ну, кролик то й кролик, пробіг то й пробіг... І те Алісі не дивно, що кролик
говорив: \enquote{Ой лишенько, ой лишенько! Я спізнюся!} (Згадуючи про це пізніше, вона
дивувалась, але тепер їй здавалось, що це так і треба). Та коли Кролик при
цьому дістав з жилетної кишені годинника, поглянув на нього і стрімголов побіг
далі, Аліса зірвалася на рівні ноги, — де ж таки, вона зроду не бачила Кролика
в жилеті та ще й при годиннику. Вона аж затремтіла від цікавості і побігла
слідом за Кроликом через поле і ледве встигла помітити, що той шмигнув у велику
кролячу нору під живоплотом. Аліса відразу кинулася слідом, ні на мить не
замислюючись, як потім вибереться звідти.

Спочатку кроляча нора йшла прямо, мов тунель, потім раптово завертала вниз, так
раптово, що Аліса і незчулася, як полетіла в якийсь глибокий колодязь.

\ifcmt
  ig https://scontent-fra5-1.xx.fbcdn.net/v/t1.6435-9/35464434_1747963308592293_8406939049268346880_n.jpg?_nc_cat=111&ccb=1-7&_nc_sid=5cd70e&_nc_ohc=Q2dLuelKMAQAX8SFGZQ&_nc_ht=scontent-fra5-1.xx&oh=00_AfCTimkzNHT1KCHz2-dYAJYE4y333GbSGlnU6Qan5BjtcQ&oe=64651270
  @width 0.4
  @minipage 0.4
  @wrap \parpic[l]
\fi

Чи то колодязь був надто глибокий, чи вона падала надто повільно, але в неї
було досить часу, щоб роздивлятися навкруги і міркувати, що буде далі. Спочатку
вона спробувала глянути вниз, щоб дізнатися, куди вона падає, але там була
непроглядна тьма. Тоді вона придивилася до стін колодязя і побачила на них
багато поличок для посуду і книжок. Подекуди на кілочках висіли картини і
географічні карти. Пролітаючи повз одну з поличок, вона взяла з неї баночку з
написом: \enquote{Апельсинове варення}, але та, на превеликий жаль, була порожня. Аліса
не хотіла кидати баночку, щоб, буває, не вбити кого-небудь унизу, а
примудрилася поставити її в шафу, повз яку саме пролітала.

\enquote{Ну, — думала Аліса, — після такого падіння мені вже не страшно буде скотитися
зі сходів. Якою хороброю будуть вважати мене всі вдома! Та що там, я не скажу
ні словечка, якщо навіть упаду з даху будинку}. (І справді, мабуть, не сказала
б!)

\ifcmt
  ig https://lh3.googleusercontent.com/p/AF1QipO3g6pO2vcDcJvv8e1oHuFZVeMloDAxHyYiJxS8=s0
  @width 0.4
  @minipage 0.4
  @wrap \parpic[r]
	@no_caption 1
\fi

Вниз, вниз, вниз. Невже цьому падінню кінця-краю нема? — Цікаво, скільки миль я
пролетіла? — голосно заговорила Аліса. — Я, мабуть, наближаюся до центра землі.
Треба подумати: до нього щось близько чотирьох тисяч миль. (Аліса, бачите, дещо
вчила про це на уроках у школі, і хоч зараз була не дуже вдала нагода
похвалитися своїми знаннями, бо нікому було слухати, але не завадить повторити
для практики). Так, приблизно така відстань. Цікаво знати, на якій довготі та
широті я знаходжуся? (Аліса не мала навіть найменшого уявлення про довготу та
широту, але ці слова звучали так гарно і велично!)

— Цікаво, чи не пролечу я землю наскрізь? От смішно буде опинитися серед людей,
які ходять догори ногами! Антипуди, чи як там їх? (На цей раз вона була навіть
задоволена, що її ніхто не слухає, бо слово було трохи не те). Але, розумієте,
мені доведеться запитати назву країни. Пробачте, пані, це Нова Зеландія чи
Австралія? (При цьому вона намагалася зробити реверанс, — уявіть собі, що ви
робите реверанс, летячи в повітрі! Спробуйте!) Але вони подумають, що я зовсім
дурна, якщо я буду запитувати. Ні, питати не годиться... Може, я побачу десь
вивіску.

\ifcmt
  ig https://lh3.googleusercontent.com/p/AF1QipMyKzW5aoz7ZZk_CKlcSpyCREpLYc5uNyxuwXB0=s0
  @width 0.3
  @minipage 0.3
  @wrap \parpic[r]
	@no_caption 1
\fi

Вниз, вниз, вниз. Робити все одно було нічого, і Аліса незабаром знову почала:

— Ох же й скучатиме Діна без мене сьогодні ввечері. (Діна — це кішка). Якби ж
там хоч не забули налити їй молочка у блюдечко! Ой, Діночко! Як мені хочеться,
щоб ти була тут зі мною. Боюся, що в повітрі немає мишей, але ти могла б
зловити кажана, він же дуже схожий на мишу. А цікаво, чи їдять коти кажанів? —
Тут Алісу почав змагати сон, але вона продовжувала белькотати: — Чи їдять коти
кажанів? Чи їдять коти кажанів? — А іноді: — Чи їдять кажани котів? — Хіба не
все одно, як поставити питання, коли не знаєш відповіді? Аліса відчувала, що
зовсім засинає. Їй починало снитися, неначе вона йде з Діною і цілком серйозно
говорить: \enquote{Ну-бо, Діно, скажи мені правду, ти їла коли-небудь кажанів?}. Аж тут
раптом: бабах! І вона опинилася на купі хмизу і сухого листя. Приїхали!

Аліса нітрохи не забилася, тому хутко схопилась на ноги. Глянула вгору, але там
було зовсім темно. Попереду був довгий прохід, в якому ще виднівся Кролик, що
дуже поспішав. Не можна було втрачати ні хвилини: Аліса, мов вітер, помчала
слідом і встигла навіть почути, як він говорив, звертаючи за ріг: \enquote{Ой, бідні
мої вушка й вуса, як пізно!} Вона майже наздогнала його на розі, та коли
завернула, Кролика ніде не було. Вона опинилася в довгому низькому залі,
освітленому рядом ламп, що звисали із стелі.

З усіх боків залу були двері, але всі замкнені. Коли Аліса оглянула всі стіни,
пробуючи кожні двері, вона сумно попленталася до середини залу, роздумуючи, як
же їй вибратися звідти. Дивиться, аж там стоїть скляний столик на трьох ніжках.
На ньому не було нічого, крім маленького золотого ключика, і Аліса зразу ж
подумала, що він, мабуть, від якихось дверей в залі. Та ба! Чи замки надто
великі, чи ключик надто малий, але жодні двері не відімкнулися. Проте,
оглядаючи стіни вдруге, вона побачила невелику занавіску, якої не помітила
раніше, а за нею маленькі дверцята, дюймів 15 заввишки. Вона спробувала
вставити золотий ключик в замок, і, на превелику радість, він підійшов!

Аліса відчинила дверцята. Вони вели до вузенького проходу, трошки більшого від
нори пацюка. Вона стала навколішки і побачила по той бік нори дивовижно гарний
сад. Як їй хотілося вибратися з цього похмурого залу і поблукати серед он тих
пишних квітників і прохолодних фонтанів! Але в дверцята не пролазила навіть її
голова.

— А якби голова й пролізла, — міркувала бідолашна, Аліса, — що з того? Адже
плечі не пройдуть. От якби я могла складатися, як та підзорна труба, тоді б
пролізла... Як би мені оце скластися?

Останнім часом, бачите, трапилося стільки всяких див, що Алісі гадалося, ніби
на світі нема нічого неможливого.

Зрозумівши, що біля дверцят нічого не вистоїш, вона знову попрямувала до
столика, — чи не знайде там ще якого ключика або хоча інструкцію, як людині
скластися трубою. Цього разу вона побачила на столику пляшечку (\enquote{А її ж не було
тут раніше}, зауважила Аліса), на шийці якої був прив’язаний ярличок з словами
\enquote{випий мене}, чітко надрукованими великими літерами.

Добре було говорити \enquote{випий мене}, але розумна маленька Аліса і не думала
поспішати.

— Ні, я спочатку переконаюся, — вирішила вона, — що на ній немає напису
\enquote{отрута}.

Справа в тому, що вона не раз читала оповідання про людей, які згоріли, або
попали в пащу звірові, або зазнали ще якого лиха, і все це тільки тому, що не
бажали пам'ятати науки старших: не бери в руки розпеченої кочерги бо обпечешся;
не ріж пальця ножем, бо кров ітиме. І Аліса ніколи не забувала, що коли випити
зайве з пляшечки з написом \enquote{отрута}, то буде лихо рано чи пізно.

Проте на цій пляшечці не було напису \enquote{отрута}, тож Аліса зважилась покуштувати
крапельку. Те, що було в пляшечці, сподобалось їй: воно нагадувало смаком і
пиріг з вишнями, і заварний крем, і ананас, і смажену індичку, і тягучки, і
грінки на маслі, тому вона швиденько випила все.

* * *

— Ой, що це зі мною діється! — скрикнула Аліса. — Maбуть, я складаюсь, як
підзорна труба.

Аліса не помилилась. Вона справді поменшала і була всього дюймів десять
заввишки. От тепер вона пройде крізь маленькі дверцята в той прегарний сад!
Проте вона спочатку почекала, щоб переконатися, чи не буде зменшуватися далі. —
це її трохи турбувало.

— Чого доброго, я розтану зовсім, як свічка, — подумала Аліса. — Цікаво, яка я
тоді буду?

І вона намагалася уявити полум'я свічки після того, як його погасять, бо їй
досі не доводилося, наскільки вона пригадує, цього бачити.

Згодом, переконавшись, що не зменшується далі, вона вирішила негайно йти в сад,
але... Бідна Аліса! Коли вона підійшла до дверцят, виявилося, що вона забула
золотий ключик, а коли повернулася за ним до столу, виявилося, то не може
дістати його. Ключик добре виднівся крізь скло, і вона з усієї сили намагалася
видряпатися по ніжці стола, але та була надто слизька. Втомившись від марних
зусиль, сердешна мала сіла і заплакала.

— Перестань, сльозами лиху не зарадиш! — гримнула на себе Аліса досить різко. —
Раджу тобі, припинити це негайно!

Взагалі, вона давала сама собі дуже добрі поради (хоч дуже рідко дотримувалася
їх), а іноді так жорстоко картала себе, що аж сльози виступали в неї на очах і
навіть, пам'ятається, одного разу намагалася нам'яти собі вуха за те, що
обдурила сама себе, граючи за двох в крокет. Ця дивна дівчинка дуже любила
вдавати з себе двох осіб.

Згодом, переконавшись, що не зменшується далі, вона вирішила негайно йти в сад,
але... Бідна Аліса! Коли вона підійшла до дверцят, виявилося, що вона забула
золотий ключик, а коли повернулася за ним до столу, виявилося, то не може
дістати його. Ключик добре виднівся крізь скло, і вона з усієї сили намагалася
видряпатися по ніжці стола, але та була надто слизька. Втомившись від марних
зусиль, сердешна мала сіла і заплакала.

— Перестань, сльозами лиху не зарадиш! — гримнула на себе Аліса досить різко. —
Раджу тобі, припинити це негайно!

Взагалі, вона давала сама собі дуже добрі поради (хоч дуже рідко дотримувалася
їх), а іноді так жорстоко картала себе, що аж сльози виступали в неї на очах і
навіть, пам’ятається, одного разу намагалася нам’яти собі вуха за те, що
обдурила сама себе, граючи за двох в крокет. Ця дивна дівчинка дуже любила
вдавати з себе двох осіб.

\emph{КАЛЮЖА СЛІЗ}

— Все більш дивніше і більш дивніше, — скрикнула Аліса (вона з великого дива
забула всі правила граматики). — Тепер я розсовуюся, мов найбільша в світі
підзорна труба! До побачення, ніжки (бо коли вона глянула на свої ноги, їх
майже не було видно, так вони віддалилися). Ніженьки мої милі! Хто ж тепер вас
узуватиме? Я вже не зможу, не дістану! Взувайтеся самі, як знаєте. Але треба їх
жаліти, — міркувала Аліса, — інакше вони відмовляться іти туди, куди мені
заманеться. Треба подумати... Ага, я буду дарувати їм нову пару черевиків
кожного Нового року.

І вона міркувала собі далі, як це зробити.

— Доведеться надсилати їх з посильним, — думала вона. — Як це буде смішно,
надсилати подарунки власним ногам! І яка буде чудернацька адреса:

Пані правій нозі Аліси, Килимок, Поряд з камінною решіткою.  З привітом від
Аліси.

Ой лишенько, які дурниці я мелю!

