% vim: keymap=russian-jcukenwin
%%beginhead 
 
%%file 20_11_2021.fb.fb_group.story_kiev_ua.2.malenkaja_podolskaja_devochka
%%parent 20_11_2021
 
%%url https://www.facebook.com/groups/story.kiev.ua/posts/1802085179988275
 
%%author_id fb_group.story_kiev_ua
%%date 
 
%%tags gorod,kiev,zhizn
%%title Маленькая киевская история маленькой подольской девочки
 
%%endhead 
 
\subsection{Маленькая киевская история маленькой подольской девочки}
\label{sec:20_11_2021.fb.fb_group.story_kiev_ua.2.malenkaja_podolskaja_devochka}
 
\Purl{https://www.facebook.com/groups/story.kiev.ua/posts/1802085179988275}
\ifcmt
 author_begin
   author_id fb_group.story_kiev_ua
 author_end
\fi

Хочу рассказать маленькую киевскую историю маленькой подольской девочки. 

100-й садик на Подоле, как и всё, что связано с этим районом, оставил приятные
и яркие воспоминания в моей памяти. Здания садика давно нет, сейчас там новое
здание бизнес-центра. Иногда, приходя на предполагаемое место расположения
детского садика, пытаюсь по домам на противоположной стороне улицы
Борисоглебской определить точнее, где он находился. Не так давно мне
вспомнилась история с двумя фотографиями, сделанными на одном из утренников.
Девочкой я была очень застенчивой, мальчики мне нравились с тех пор, как помню
себя, но виду я не подавала и очень их стеснялась. К тому же, я не очень любила
фотографироваться. Воспитательница решила между мной и моей подругой Мариной
поставить мальчика (его имя в моей памяти не сохранилось). Идея мне эта не
очень понравилась, но что поделаешь. Когда он прикоснулся к моей руке, я вообще
растерялась... Но фото было сделано, сохранив навсегда то, что я чувствовала в
тот момент. Как оказалось, аналогичные чувства испытывала и Марина.
Предполагая, что фото получится не очень, я, не долго думая, попросила
фотографа сделать еще одно фото, но уже вдвоем с подружкой. Фотограф охотно
согласился, и я, радостная и спокойная, с облегчением и удовольствием
улыбнулась ему. 

\begin{multicols}{2}
\ii{20_11_2021.fb.fb_group.story_kiev_ua.2.malenkaja_podolskaja_devochka.pic.1}
\ii{20_11_2021.fb.fb_group.story_kiev_ua.2.malenkaja_podolskaja_devochka.pic.1.cmt}

\ii{20_11_2021.fb.fb_group.story_kiev_ua.2.malenkaja_podolskaja_devochka.pic.2}
\ii{20_11_2021.fb.fb_group.story_kiev_ua.2.malenkaja_podolskaja_devochka.pic.2.cmt}

\end{multicols}

Сейчас, когда прошло столько лет, я с улыбкой смотрю на эти две фотографии и
думаю о том, как хорошо, что можно, как сказал админ нашей группы в
приветственном посте, \enquote{вновь пережить нахлынувшие чувства, которые испытывали
тогда, – в те далекие тёплые дни.}

На фото я справа. @igg{fbicon.smile} 

\ii{20_11_2021.fb.fb_group.story_kiev_ua.2.malenkaja_podolskaja_devochka.cmt}
