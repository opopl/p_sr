% vim: keymap=russian-jcukenwin
%%beginhead 
 
%%file 26_01_2021.stz.news.ua.mrpl_city.1.anastasia_zhuravljova_vse_v_tvoih_rukah
%%parent 26_01_2021
 
%%url https://mrpl.city/blogs/view/anastasiya-zhuravlova-vse-v-tvoih-rukah
 
%%author_id demidko_olga.mariupol,news.ua.mrpl_city
%%date 
 
%%tags 
%%title Анастасія Журавльова: "Все в твоїх руках...".
 
%%endhead 
 
\subsection{Анастасія Журавльова: \enquote{Все в твоїх руках...}.}
\label{sec:26_01_2021.stz.news.ua.mrpl_city.1.anastasia_zhuravljova_vse_v_tvoih_rukah}
 
\Purl{https://mrpl.city/blogs/view/anastasiya-zhuravlova-vse-v-tvoih-rukah}
\ifcmt
 author_begin
   author_id demidko_olga.mariupol,news.ua.mrpl_city
 author_end
\fi

\ii{26_01_2021.stz.news.ua.mrpl_city.1.anastasia_zhuravljova_vse_v_tvoih_rukah.pic.1}

Цього разу героїнею мого нарису стане дуже яскрава дівчинка, чиї досягнення і
талант вражають не тільки маріупольців, але й увесь світ. Вона переможниця
багатьох творчих конкурсів національного, європейського та міжнародного рівнів.
Нещодавно \emph{Анастасія увійшла до книги рекордів України, а за підсумками
Міжнародного конкурсу \enquote{Світова суперзірка 2020} отримала гран-прі.}

Дівчинка малює з першого року життя. Спочатку разом з матусею намагалася
малювати всюди: на стінах, на асфальті, на одязі, на виробах з гіпсу. Перша
виставка в житті Настусі відбулася, коли їй було 5 років. Вчителька з малювання
вивісила її роботу на скло магазину канцтоварів. Щаслива родина зробила купу
світлин біля тієї вітрини. Але найголовніша і найповажніша перемога була
попереду. У 2017 році у Парижі вона отримала гран-прі конкурсу \enquote{Діти проти
війни}. На той момент вже 2 роки поспіль у номінації \enquote{Живопис} не було
переможців, тому перемога Анастасії була приємною несподіванкою. А для членів
журі маріупольчанка, яка за допомогою фарб зуміла показати глибину власних
переживань та унікальність свого таланту, стала справжнім відкриттям. Її робота
\emph{\enquote{Є речі, які не можна знищити}} пов'язана з подією, що закарбувалася в пам'яті
художниці назавжди. На початку війни родина нашої героїні на літо вирішила
поїхати до приміського будиночка, Там Настусі подарували дуже гарненького
кошеня, якого дівчинка часто брала з собою. Вона б ніколи не подумала, що це
чарівне пухнасте створіння принесе в її життя страшне випробування. Велика
зграя собак, полюючи за кошеням, напала на дівчину, яка, ховаючи маленького
друга, отримала багато пошкоджень і важких травм. Перебуваючи в операційній,
Настуся благала Божу Матір про допомогу. Всі рідні і друзі теж щохвилини
молилися. Тільки коли прийшов лікар і повідомив гарну новину, дівчинка
зрозуміла, що все минулося. Вона залишилася безмежно вдячною всім, хто був
поруч. Коли Настуся вийшла з лікарні, одразу пішла до храму. Того дня рідні їй
подарували ікону Пресвятої Богородиці.

\ii{26_01_2021.stz.news.ua.mrpl_city.1.anastasia_zhuravljova_vse_v_tvoih_rukah.pic.2}

На роботі, що була представлена у Франції, написана Богородиця з немовлям
(дівчина змальовувала її з тієї ікони, яку їй подарували у храмі), а позаду
Богородиці – війна. Страшна і кривава! Її робота – це заклик до людей всього
світу, до їхніх душ: прокинутися і зазирнути всередину себе... Зупинити війни та
вбивства, любити одне одного...

\href{https://mrpl.city/news/view/mariupolchanka-stala-mirovoj-superzvezdoj-v-las-vegase}{%
Читайте также как Анастасия стала мировой суперзвездой в Лас-Вегасе}%
\footnote{Мариупольчанка стала мировой суперзвездой в Лас-Вегасе, Ганна Хіжнікова, mrpl.city, 19.01.2021, \par%
\url{https://mrpl.city/news/view/mariupolchanka-stala-mirovoj-superzvezdoj-v-las-vegase}
}

Окрім паризької перемоги, Настя також брала участь і в багатьох інших
конкурсах, де здобула призові місця. Одна з найбільш значущих нагород – \emph{\enquote{Диво
дитина} – \enquote{Найбільш творча дитина року}}. Це найпрестижніша нагорода в Україні
для дітей до 13 років. За ескізом Анастасії створена більш ніж двохметрова
металева скульптура – \enquote{Закохані будиночки}, яка прикрашає улюблене місто
дівчини – Маріуполь.

Маріупольчанка проводила багато майстер-класів. Одного разу їй навіть довелося
це робити двома мовами у Хорватії для іноземних дітей. Це було дуже важко. У
художниці вже було декілька персональних виставок.

\ii{26_01_2021.stz.news.ua.mrpl_city.1.anastasia_zhuravljova_vse_v_tvoih_rukah.pic.3}

Її роботи були представлені в музеях багатьох країн: Греція (Афіни), Фінляндія
(це єдиний музей, в якому виставляються лише роботи дітей до 14 років зі всього
Світу), Японія, Китай (Пекін). В Україні роботи можна було побачити в
етнографічному музеї (найбільший музей України ), в мерії Львова, у Верховній
Раді України. Анастасія є переможницею багатьох всеукраїнських та міжнародних
конкурсів: Франція, Китай, Фінляндія, Польща, Японія, Індія, Румунія,
Туреччина, Америка, Чехія тощо. 23 рази отримувала Гран-прі! У 2019 році
отримала звання і орден \enquote{Гордість України}.

Дівчина є однією з ілюстраторок книг \enquote{Про живе}, \enquote{Лист дорослому},  \enquote{Планета
кішок}. У Львові вийшло видання з літературними творами Анастасії \enquote{Жінки
України}.

\ii{26_01_2021.stz.news.ua.mrpl_city.1.anastasia_zhuravljova_vse_v_tvoih_rukah.pic.4}

Анастасія – стипендіат міської ради, як обдарована дитина 2018-2019, 2019-2020
і 2020-2021 років. Сьогодні дівчина дуже вдячна всім вчителям своєї улюбленої
Спеціалізованої школи № 66 за підтримку і надані знання. Другим домом для
дівчинки стала Маріупольська школа мистецтв. З особливою вдячністю та теплом
відноситься до Макаренко Любов Василівни. Як наголошує мама талановитої Насті –
\emph{\enquote{всі наші перемоги – це командна робота. Це і натхнення та підтримка і
посмішки}}. У майбутньому Анастасія планує освоїти графічний дизайн. Є у
чарівної Анастасії ще одне хобі. Нещодавно почала ходити в школу моделей \enquote{Av
present models}. Перші результати говорять самі за себе. Анастасія природжена
модель.

\ii{26_01_2021.stz.news.ua.mrpl_city.1.anastasia_zhuravljova_vse_v_tvoih_rukah.pic.5}

Дівчина виховується в дуже дружній родині, хоча її братики не малюють зовсім.
Найменший Михайлик (6,5 років) любить математику і спорт. Миколі (13 років)
дуже подобається працювати з комп'ютером. Ходить в \enquote{Шаг}. Мама Насті – \emph{\textbf{Марина
Журавльова}} – вчителька 66 школи. Працює там 3 роки. Викладає французьку мову.
Свою професію дуже любить і від роботи отримує справжнє задоволення. Марині
подобається колектив, в якому вона працює, адже всі одне одного підтримують.
Жінка вдячна і директорці школи, Ровицькій Наталі Юріївні, і всім вчителям, які
щиро радіють за перемоги та здобутки Анастасії. 

\ii{26_01_2021.stz.news.ua.mrpl_city.1.anastasia_zhuravljova_vse_v_tvoih_rukah.pic.6}

На роботу Марина вийшла, коли діти трохи підросли. Зараз і Настя, і Микола
допомагають з Михайлом. Завжди намагається допомогти і улюблена бабуся. Чоловік
Дмитро теж займається дітьми, водить їх на секції. Подружжя дуже любить
понеділки, адже саме цього дня, коли діти в школі, чоловік і дружина гуляють,
милуються морем, п'ють каву вдвох.  Один тиждень Марина пропонує кудись
поїхати, інший – чоловік. В сім'ї намагаються кожному приділити увагу. Марина
наголошує, що дітей потрібно дуже любити і постійно надихати. Всі хочуть уваги.
Необхідно влаштовувати спільні заходи і більше розмовляти одне з одним. За
правилом мами на кухню телефон брати забороняється, адже він тільки заважає
насолоджуватися сімейним затишком.  Анастасію дуже надихає сім'я, яку вона
безмежно любить. Водночас Настуся обожнює рідне місто, про яке готова
розповісти всьому світу. Для неї велика шана представляти Маріуполь і
повертатися додому з перемогами. Свого девізу: \emph{\enquote{Світ чудовий...
все в твоїх руках}} Анастасія дотримується щодня.  Сподіваюся, що ця тендітна і
дуже талановита дівчинка стане прикладом для багатьох містян і надихне всіх нас
на нові звершення.
