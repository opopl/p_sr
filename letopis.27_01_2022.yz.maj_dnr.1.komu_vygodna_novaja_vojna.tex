% vim: keymap=russian-jcukenwin
%%beginhead 
 
%%file 27_01_2022.yz.maj_dnr.1.komu_vygodna_novaja_vojna
%%parent 27_01_2022
 
%%url https://zen.yandex.ru/media/id/5f8f226b1fe36c1d9e02a36b/komu-vygodna-novaia-voina-61f2d997ad2a25202d32a848
 
%%author_id yz.maj_dnr
%%date 
 
%%tags rossia,ugroza,ukraina,usa,vojna,zapad
%%title Кому выгодна новая война?
 
%%endhead 
 
\subsection{Кому выгодна новая война?}
\label{sec:27_01_2022.yz.maj_dnr.1.komu_vygodna_novaja_vojna}
 
\Purl{https://zen.yandex.ru/media/id/5f8f226b1fe36c1d9e02a36b/komu-vygodna-novaia-voina-61f2d997ad2a25202d32a848}
\ifcmt
 author_begin
   author_id yz.maj_dnr
 author_end
\fi

Не, ну, правда, кому оно надо-то? Кто гешефт получать будет? Всегда говорила: в
любой заварухе ищи заказчика по горке денег около его норки. Например, Вторая
мировая подняла экономику США, да и Британия тоже подлохматилась, а больше
всех, и в финансовом отношении тоже, пострадала Россия. А вот эта вот, новая,
около которой сейчас все крутится? Кому она «потрибна»? Ну да, про «потрибна» -
это не оговорка по Фрейду, это, так сказать, намек.

\ii{27_01_2022.yz.maj_dnr.1.komu_vygodna_novaja_vojna.pic.1}

Но давайте порассуждаем:

Нужна ли новая война Путину? А ему-то зачем? С чего бы? Рейтинги у него
порядке, его власти, в целом, ничто не угрожает, особенно после того, как
некоторые оппозиционные «силы» показали народу свое истинное лицо, и то
сказать, вы видите хоть одного конкурента? Я лично нет, и это не только потому,
что я живу в Донбассе, а без Слова Путина нам бы было ой-ей-ей, как паршиво, а
возможно, нам бы пришлось уйти в партизаны (я не шучу). Так вот, дело не только
в моей необъективности, а и в реальных причинах – их нет. Так, как развивается
Россия, не развивается ни одна страна в мире, что бы там не говорили. И многие
из тех, кто сейчас пыхтит на оппозиционной ниве, живет припеваючи не только за
западные гранты, но и благодаря Путину – точно знаю, сколько из нынешних
оппозиционеров в 90-х на сигареты сшибали, а первую тыщу долларов узрели с
наведением кой-какого порядка после 2000 года. Но то такое. Короче говоря,
Путину война не нужна. А значит, не нужна и России.

Европе? Боже упаси! Она, конечно, не прочь поживиться, если дадут и будет чем,
но во-первых, не факт, что дадут, а вот экономике, так долго гнущейся под
коронавирусом, точно придут кранты. Да много всего, что можно поставить в ряд
причин категорического нежелания войны со стороны Европы, но главная –
экономика, а память о 1945-м еще не остыла. Учтите еще момент – куда денутся
все коммерческие и прочие плюшки, которые сейчас Европа имеет от России?

Нужна ли война американцам? Безусловно да! Во-первых, следующие президентские
выборы демократы подуют всухую – это уже ясно. Во-вторых, внутренних проблем в
США вагон и маленькая тележка, а народное недовольство можно погасить только
перенаправив его на внешнего врага. В третьих, Америке совсем не нравится, что
она потеряла статус всемирного эээээ…. Хозяина (не будем называть вещи своими
именамиJ). Плюс кризис. Экономический кризис, я имею в виду. А единственный
способ, которым США отлично умеют пользоваться для вывода своей страны из
кризиса – это война на чужой территории. Я конечно, сейчас глобально утрирую но
в целом ответ на вопрос вы поняли.

Украине? А то! Снова по пунктам:

1. Зеленский сидит в безвыходном шпагате. С одной стороны хозяин колонии
требует, как сам, приказами из Вашингтона и американского посольства так и с
помощью подконтрольных украинских политиков и олигархов.

2. Без маленькой победоносной войнушки мечтать о следующей каденции, за который
можно будет зачистить «хвосты» первого срока, не получится.

3. Те, кто накачивает Украину летальным оружием, обязательно спросят: а зачем
просил? И никто не вспомнит что сам предлагал и даже навязывал.

4. «Минск» - миссия невыполнима. Зеленский это прекрасно понимает. Выйти из
международного НПА и не потерять лицо можно только одним способом – заставить
противника воевать. И не так важно, кто начнет, главное, кто громче кричать
будет «Хулиганы зрения лишают!».

5. Кризис, опять же. Зеленский уверен, что в случае войны, ему будет помогать
вся Европа (И США с Британией в первых рядах). Деньгами, вооружением,
технологиями, но в первую очередь деньгами, конечно – и мысль об этом греет,
как никогда раньше. К тому же на войну можно списать все промахи и просчеты
управления, если это слово применимо к нынешнему официальному Киеву. Т.е.,
оправдание перед собственным народом будет - просто зашибись. Еще и жалеть
будут, даже замерзая рядом с пустыми холодильниками.

Короче, ответ вы снова поняли.

Но не могу не сказать: простые, обычные граждане войны не хотят. Они же не
сумасшедшие, в самом деле! Не хотят от слова категорически. За неделю я
посмотрела огромное количество видео на украинских ресурсах, где люди,
во-первых, не верят утверждениям своей власти о скором нападении России,
во-вторых, считают русских и украинцев одним народом (к слову о статье
Владимира Путина), а в третьих, готовы выйти на майдан с протестами против
войны И это важно, я считаю.

Но вот на вопрос о вероятности я точно ответа дать не смогу. Могу только
пофантазировать на предмет вариантов.

1. Блеф и нагнетание продолжатся еще какое-то время под накачку Украины
вооружением НАТО. В конце пузырь лопнет и Байден объявит, что он предотвратил
большую войну. Потом все стихнет до начала предвыборной кампании в США, когда
Украина снова понадобится в качестве локтя, проталкивающего демократам путь на
американский трон. Почему я не исключаю этот вариант? Потому, что именно
сейчас, в начале 2022-го война не поможет демократам в конце 2024 года.

2. Украина организует провокацию в Донбассе или на границе с Крымом, получит по
зубам, хотя часть своих задач, поставленных США, выполнит. Но тогда на
политической карте мира многое изменится. Только не говорите мне о санкциях,
особенно в отношении Путина – какие такие активы за границей у него обнаружены?
А остальные ударят по затейникам больше, чем по России. Я исхожу из того что, в
отличие от США и Европы, у РФ есть запасные варианты, пути и ходы.

3. Украина организует провокацию в Донбассе, выйдет из Минска-2 в надежде на
Минск-3 (или как угодно называйте новый документ), но его не будет, потому что
о договороспособности Украины по миру легенды ходят. И…. смотри п.2.

Это не все мои версии, но остальные – только варианты. Почему я жду провокаций
только в Донбассе или в Крыму? Отвечу вопросом на вопрос: почему Украина,
которая говорит о скоплении российских войск у своих границ, нигде больше не
скапливает вооружения и армию – только около линии соприкосновения с
республиками? Не потому ли, что она прекрасно знает – Россия не собирается
нападать!

Давайте еще раз:

По-хорошему, война не выгодна никому. Никогда еще военным способом невозможно
было достичь счастья для народа. Ни для какого! Временных денег и власти – да.
Но благополучия, спокойствия, процветания – никогда. Россия это знает испокон
веков и никогда не нападала сама. И сейчас тоже, устами Владимира Путина,
Сергея Лаврова, Марии Захаровой и других российских политиков она говорит «Нет
войне!».

Донбасс тоже не молчит. Сколько раз уже было, когда, озвучивая планы Украины,
мы заставляли украинскую власть отказаться от них. Мы тоже не хотим войны, но
готовы отстаивать свою родную землю – другой у нас нет Сегодня наш Глава Денис
Пушилин в интервью «Украине.Ру» сказал замечательно точные слова: «Нет
мотивации выше, чем у тех, кто защищает свою землю, свои семьи и свои дома».

\href{https://denis-pushilin.ru/press/denis-pushilin-net-motivatsii-vyshe-chem-u-teh-kto-zashhishhaet-svoyu-zemlyu-svoi-semi-i-svoi-doma/}{%
Денис Пушилин: Нет мотивации выше, чем у тех, кто защищает свою землю, свои семьи и свои дома, denis-pushilin.ru, 27.01.2022%
}

И знаете, если каждый скажет «Войны не будет, она нам не нужна», то войны
обязательно не будет. Никогда.
