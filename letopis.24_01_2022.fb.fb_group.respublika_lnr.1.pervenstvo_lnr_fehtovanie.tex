% vim: keymap=russian-jcukenwin
%%beginhead 
 
%%file 24_01_2022.fb.fb_group.respublika_lnr.1.pervenstvo_lnr_fehtovanie
%%parent 24_01_2022
 
%%url https://www.facebook.com/groups/respublikalnr/posts/960354277933697
 
%%author_id fb_group.respublika_lnr,zimina_olesja
%%date 
 
%%tags fehtovanie,lnr,lugansk,molodezh,sorevnovanie,sport
%%title Первенство ЛНР по фехтованию среди юношей и девушек до 18 лет
 
%%endhead 
 
\subsection{Первенство ЛНР по фехтованию среди юношей и девушек до 18 лет}
\label{sec:24_01_2022.fb.fb_group.respublika_lnr.1.pervenstvo_lnr_fehtovanie}
 
\Purl{https://www.facebook.com/groups/respublikalnr/posts/960354277933697}
\ifcmt
 author_begin
   author_id fb_group.respublika_lnr,zimina_olesja
 author_end
\fi

СПОРТ. Более 70 спортсменов приняли участие в первенстве ЛНР по фехтованию
среди юношей и девушек до 18 лет 

Более 70 спортсменов стали участниками первенства ЛНР по фехтованию среди
юношей и девушек до 18 лет, которое 22-23 января прошло в Луганске. Об этом
сообщили в Федерации фехтования Луганщины, которая совместно с Министерством
культуры, спорта и молодежи (МКСМ) ЛНР выступила организатором соревнований.

\ii{24_01_2022.fb.fb_group.respublika_lnr.1.pervenstvo_lnr_fehtovanie.pic.1}

«На базах специализированных фехтовальных залов Комплексной детско-юношеской
спортивной школы (КДЮСШ) и КДЮСШ №2 прошли лично-командные соревнования на
рапирах и шпагах среди юношей и девушек 2005 года рождения и младше. В
соревнованиях приняли участие 75 спортсменов, представляющих отделения
фехтования четырех спортивных школ Луганска», – говорится в сообщении.

\ii{24_01_2022.fb.fb_group.respublika_lnr.1.pervenstvo_lnr_fehtovanie.pic.2}

По итогам личных соревнований победителями первенства среди рапиристов стали
спортсмены КДЮСШ Арсений Котуха и Дарина Помазанова, а воспитанники
Специализированной детско-юношеской спортивной школы олимпийского резерва
(СДЮСШОР) по пулевой стрельбе, фехтованию и современному пятиборью Артем Чалый
и Елизавета Капустина были признаны чемпионами в состязаниях шпажистов.

\ii{24_01_2022.fb.fb_group.respublika_lnr.1.pervenstvo_lnr_fehtovanie.pic.3}

В командных соревнованиях победителями первенства в своих спортивных
дисциплинах стали команды СДЮСШОР среди шпажистов, а также КДЮСШ среди
рапиристов.

Победители и призеры соревнований получили грамоты, медали и кубки МКСМ ЛНР.
