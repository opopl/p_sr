% vim: keymap=russian-jcukenwin
%%beginhead 
 
%%file slova.muzej
%%parent slova
 
%%url 
 
%%author 
%%author_id 
%%author_url 
 
%%tags 
%%title 
 
%%endhead 
\chapter{Музей}
\label{sec:slova.muzej}

%%%cit
%%%cit_pic
%%%cit_text
В Мелитополе откроют \emph{Музей} черешни.  Таким образом в город хотят
завлекать туристов.  Наверное, дело в том, что люди постоянно между Львовом,
Одессой и Мелитополем, выбирают не Мелитополь. Спросите, почему? - Там мало
\emph{музеев} и не было ни одного \emph{музея} черешни. Скорее всего так видит
ситуацию мэр города.  Уважаемые мэры городов. Хотите, чтобы в ваш город ехали
современные туристы, тогда стройте современные гостиничные комплексы, устричные
и страусиные фермы, эко-отели, парки активности и прочие современные объекты,
привлекающие людей и в которых вы сможете продать большое количество других
дополнительных услуг и тем самым развивать местный бизнес.  Что должен человек
купить в \emph{музее} черешни? Килограмм черешни? Магнитик или футболку с
изображением черешни?
%%%cit_title
\citTitle{В Мелитополе решили создать музей черешни}, Василий Апасов, strana.ua, 12.06.2021 
%%%endcit

%%%cit
%%%cit_head
%%%cit_pic
\ifcmt
  pic https://gdb.rferl.org/EFCF60F7-F119-4967-97EC-B7C92A38E225_w650_r0_s.jpg
	width 0.3
	caption «Берегиня березанської громади», засновниця місцевого краєзнавчого музею Галина Лаврентіївна Рих (1922–2021)
\fi
%%%cit_text
А воєнний портрет Галі, зроблений пораненим бійцем, ми з дідусем у середині
1980-х років віддали до краєзнавчого \emph{музею} міста Березані, батьківщини дідуся
та його сестри. Засновниця та директорка \emph{березанського музею} Галина Рих із
дитинства знала їх обох. Пам’ятаю, що вона тоді дуже зраділа нашому подарунку,
провела для нас екскурсію \emph{музеєм}, сказала що обов’язково під експонатом буде
підпис, що він подарований «школярем Ігорем Роздобудьком із Москви». Подальшої
долі цієї дорогої для нас реліквії я не знаю. Хотілося б вірити, що й досі вона
прикрашає один із залів березанського \emph{музею}
%%%cit_comment
%%%cit_title
\citTitle{«Заборонені» спогади з фронтів війни Німеччини і СРСР 1941–1945 років}, 
Ігор Роздобудько, www.radiosvoboda.org, 08.07.2021
%%%endcit

%%%cit
%%%cit_head
%%%cit_pic

\ifcmt
  tab_begin cols=3
     pic https://regnum.ru/uploads/pictures/news/2021/10/22/regnum_picture_16348942673307411_normal.JPG
     pic https://regnum.ru/uploads/pictures/news/2021/10/22/regnum_picture_16348943023363754_normal.JPG
		 pic https://regnum.ru/uploads/pictures/news/2021/10/22/regnum_picture_16348942833245180_normal.JPG
  tab_end
\fi
%%%cit_text
В выставочном проекте совместно с \emph{Историческим музеем} участвовали 15 участников
из России и Европы. Выставка открыта с 22 октября 2021 года по 14 февраля 2022
года в главном здании \emph{Исторического музея} по адресу Красная площадь, дом 1
%%%cit_comment
%%%cit_title
\citTitle{«Крузенштерн. Вокруг света» — фоторепортаж}, 
Наталья Стрельцова, regnum.ru, 22.10.2021
%%%endcit


