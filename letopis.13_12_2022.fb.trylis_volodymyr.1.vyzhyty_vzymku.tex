% vim: keymap=russian-jcukenwin
%%beginhead 
 
%%file 13_12_2022.fb.trylis_volodymyr.1.vyzhyty_vzymku
%%parent 13_12_2022
 
%%url https://www.facebook.com/VolodymyrTrylis/posts/pfbid02gPQz5rhAFgNp9iCfMCmjAuUkD4LRXg1LTBpuGtaEK8hMSLjaAWt6ZCbbHF6ghic4l
 
%%author_id trylis_volodymyr
%%date 
 
%%tags turizm.sportivnyj
%%title Як вижити взимку в польових умовах
 
%%endhead 
 
\subsection{Як вижити взимку в польових умовах}
\label{sec:13_12_2022.fb.trylis_volodymyr.1.vyzhyty_vzymku}
 
\Purl{https://www.facebook.com/VolodymyrTrylis/posts/pfbid02gPQz5rhAFgNp9iCfMCmjAuUkD4LRXg1LTBpuGtaEK8hMSLjaAWt6ZCbbHF6ghic4l}
\ifcmt
 author_begin
   author_id trylis_volodymyr
 author_end
\fi

Як вижити взимку в польових умовах

Володимир Триліс, МС України зі спортивного туризму

Ці рекомендації можуть бути корисними для воїнів, що боронять нашу
незалежність, для тих, кого воєнне лихоліття позбавило теплої домівки, для
всіх, кому доводиться жити і діяти в зимових польових умовах.

При їх написанні я спирався на свій багаторічний досвід зимових туристських
походів, який, звичайно, має певну специфіку, але я намагався це врахувати.
Тисячі туристів і альпіністів протягом багатьох років накопичили величезний
досвід існування і активної роботи в будь-яких, в т.ч. найтяжчих, погодних
умовах. В нинішні часи цей досвід може стати корисним для багатьох українців.

1. Холод

Сам по собі холод не такий вже й страшний. Північні кочові народи змалечку
звикають жити і діяти в умовах холоду, і особливого дискомфорту від того не
відчувають. Зовсім інша справа – міські жителі, які практично не стикаються з
холодом у своєму буденному житті. У них перебування на холоді викликає стрес,
який давить на психіку і досить швидко (за 2-3 доби у новачків) призводить до
апатії і непрацездатності/небоєздатності. Цей стрес називається холодовою
втомою, він може діяти навіть тоді, коли людині жарко (наприклад, вона йде під
рюкзаком), але кругом холод і нема можливості відпочити в теплі. По суті,
холодова втома – це наш страх перед холодом, що витікає з невпевненості у своїх
силах. Кожний наступний досвід довгого перебування на холоді покращує
підготовку і посилює впевненість у контролі над ситуацією і у власній здатності
з нею впоратись, щоразу відсуваючи критичну межу холодової втоми на 1-2 дні
(але тут є великі індивідуальні розбіжності). Як і всякий страх, холодова втома
підсилюється страхом оточуючих людей, поганим настроєм, відсутністю чіткого
плану дій. А зменшується, відповідно, хорошим настроєм, дружньою підтримкою і,
звичайно, впевненістю у своєму спорядженні. 

Тепер про спорядження.

2. Спальний мішок

Найголовніше при багатоденному перебуванні на холоді – це нормальний сон в
теплі. Особливо, віра в те, що, що б з вами не відбувалось вдень, вночі вам
буде тепло і затишно і ви зможете нормально відпочити. Отже, критичною є якість
спальника. Зараз, на фоні воєнних потреб, з’явилося безліч статей, що детально
розбирають характеристики і технології спальних мішків. Досі спальники були,
переважно, атрибутом туристів і альпіністів, а тепер їх потребують і бійці на
фронті, і жителі зруйнованих будівель, що залишились без тепла. І тут я хочу
сказати важливу річ: в таких умовах (коли спальник не потрібно постійно носити
в рюкзаку) нема потреби купувати суперсучасний дорогий спальник. Два дешевих
спальники, вкладені один в другий, гріють зазвичай краще, ніж один дорогий. А
три ще краще! Така багатошаровість має ще одну перевагу: при цьому мокне і
брудниться переважно зовнішній спальник, а внутрішній залишається чистим і
сухим. При цьому тонкий зовнішній спальник набагато легше висушити, ніж дорогий
товстий високотехнологічний спальний мішок. І його простіше замінити, якщо він
вийде з ладу, ніж міняти весь дорогий спальник. 

Тут слід згадати, що спальники (як і одяг) мокнуть не лише ззовні, а й
зсередини, від тої вологи, яку виділяє наше тіло. Цієї вологи чимало (пару
літрів за день, півлітра за ніч) і бажано, щоб вона кудись випаровувалась, бо
інакше все промокне. Тому, як на спальнику, так і на одязі верхній шар має бути
з дихаючої (промокаючої) тканини. Але на холоді ми стикаємось з таким явищем,
як «точка роси». Це те місце, де водяна пара, яку виділяє наше тіло,
перетворюється у рідку воду, яка називається конденсатом. Зазвичай, це
відбувається на внутрішній стороні зовнішнього шару тканини, яка нас оточує. У
спальнику конденсат накопичується у зовнішньому шарі. Якщо спальник один, то
поступово він весь намокає від конденсату. А якщо спальників 2-3, то намокає
лише зовнішній, а внутрішні залишаються сухими. А тонкий зовнішній, як ми вже
зауважили, висушити нескладно. Непогано працює комбінація «флісовий
вкладиш+середньої якості спальник+зовнішній захисний чохол з цупкої тканини».
Для тих, кому доводиться часом спати не роззуваючись, можна додати ще
внутрішній вкладиш для ніг, з одного шару тканини, глибиною приблизно по
коліна, от його краще зробити з тонкої водонепроникної тканини (поліетиленовий
мішок також згодиться).

Насправді, в критичних умовах замість спальників можна використовувати будь-що:
ковдри, пледи, покривала, килими, навіть старе ганчір’я, створюючи з цього
всього один чи декілька зовнішніх шарів. Три шерстяних ковдри чудово заміняють
пуховий спальник, загорнувшись в них можна комфортно спати до -20. До речі,
знизу звичайна шерстяна ковдра гріє краще, ніж пуховий спальник, бо під вагою
тіла пух стискається до майже повної відсутності теплоізоляції. Тож ще одна
стара ковдра може чудово замінити теплоізоляційний килимок (каремат).

(каремат).

Варто згадати також про захист від дощу, снігу та іншої вологи згори. Дощ в
цьому сенсі гірший за сніг, від нього захиститись важче. З вищеназваних
міркувань про конденсат випливає, що, маючи водонепроникний захист (такнину чи
плівку) краще не загортатись в неї, а зробити якийсь тент чи навіс, щоб
внутрішня волога могла вільно випаровуватись. А ще краще спати під якимсь
дахом, щоб хоч зверху не капало.

Загалом, спальник слід берегти від вологи значно ретельніше, ніж одяг. Бо
лежачи в спальнику ви майже не виділяєте тепла, тож спальник лише відсиріває
все більше. А в одязі ви, навпаки, більше рухаєтесь, тож одяг, відповідно,
швидше сохне, ніж мокне (і сушити його, зазвичай, легше). Звідси рекомендація
не сушити свій одяг і взуття в спальнику – краще хай вони сохнуть на вас, щоб
волога з них не потрапила в спальник. Спати бажано в сухому одязі, найкраще в
тонкому трикотажі або термобілизні. Товстий одяг утруднить теплообмін між
частинами вашого тіла і прискорить замерзання. Зайвий одяг (якщо він сухий)
краще покласти всередину спальника, утеплюючи ті місця, з яких «тягне» холодом.
Якщо вам дуже неприємно вранці надягати холодний мокрий одяг чи взуття, їх
можна на ніч покласти в спальник, у водонепроникному пакеті чи мішку (бережіть
спальник від вологи!). Тоді вранці вони будуть теплими, а спальник – сухим.

Дуже гарним засобом зігрівання є закладання в спальник пластикових пляшок з
гарячою водою – грілок. Чим більша фляга, тим довше вона тримає тепло.
2-літрова пляшка навіть на 30-градусному морозі до ранку залишається теплою. У
пляшку краще заливати не окріп (від нього пляшка деформується), а воду з
температурою 80-90°. Щоб не обпектися, на пляшку можна натягти теплі шкарпетки,
або обгорнути її футболкою. Таких пляшок може бути декілька. До речі,
шкарпетки, рукавиці і т.п., надіті на таку пляшку, за ніч непогано висихають
(але краще це робити не в спальнику). Навіть в черевики можна запхнути на ніч
пляшки з гарячою водою, і вони за ніч підсохнуть.

3. Одяг

На відміну від спальників, одяг значно більш багатофункціональний і
різноманітний, тому обмежуся тут лише кількома загальними принципами. Очевидно,
що сухий одяг гріє краще за вологий, тож його слід ретельно берегти від
потрапляння води або снігу, і при кожній можливості підсушувати. Одяг досить
швидко мокріє від поту, особливо на холоді (точка роси опиняється всередині
одягу), тому важливо, щоб одяг був «дихаючим». Водонепроникний одяг – це дуже
ризиковано, і, хіба що, сповільнює (але не припиняє) намокання під дощем. 

В одязі, так само як і в спальниках, важливо дотримуватись принципу
багатошаровості. Внутрішній шар – комфортний для тіла, середній шар –
утеплюючий (може складатись з кількох шарів одягу), зовнішній шар – захисний
(від вітру, снігу, бруду). Від дощу краще мати захист окремо (з вищеописаних
міркувань). Відповідно, зовнішній шар одягу краще щоб був без утеплювача (і
навіть без підкладки), бо він найпершим буде мокнути, бруднитись, рватись, а
сушити, чистити і ремонтувати значно простіше одношарову тканину, ніж складний
високотехнологічний комплекс сучасної теплої куртки (і коштує така зовнішня
штормовка-вітровка значно дешевше). Взимку важливо, щоб на зовнішній шар одягу
не налипав сніг, тож він має бути синтетичним і гладеньким. Внутрішні
кофти/светри можуть бути різними, в залежності від потреб і можливостей. При
наявності гарного зовнішнього захисного одягу, вимоги до них мінімальні.
Найзручніший (і найдешевший) на сьогоднішній день утеплювач – це фліс. Якщо
флісових кофт буде декілька, це дасть більше можливостей для комбінування і
адаптації під конкретні умови. Поєднання кількох функцій в одному елементі
одягу (наприклад, софтшел) незручне. Воно трохи економить вагу, але значно
підвищує вартість і зменшує діапазон можливостей. Якщо намокла, порвалась чи
забруднилась одна флісова кофта з кількох, то ви можете користуватись рештою. А
якщо те ж саме відбулося з вашою єдиною софтшельною кофтою, то ви залишились
без теплого одягу.

Найпершими, зазвичай, мерзнуть кінцівки – ноги і руки. Отже, за ними треба
доглядати якомога більш ретельно. Найкращий спосіб зберегти  ноги теплими і
сухими – це захистити взуття ззовні, за допомогою бахіл, про які я детально
писав тут 
\href{https://www.facebook.com/tourclub.universytet/posts/pfbid02DSHgG8cX9N4HfYa9bMEnS4UYahPz2UcFst1ompqQ5RdZ1kfgPRWSn8U3veR8vKLnl}{%
Зима близко! Изучаем высокие технологии, закупаем правильную ткань и галоши, учимся шить сами!, facebook, %
Турклуб "Університет", 24.11.2021%
}

Що стосується рук, то найкращий захист для них – це рукавиці (а не перчатки!).
Тут також важлива багатошаровість, бо рукавиці мокнуть, брудняться і рвуться
найпершими, тож важливо, щоб це траплялось не з усією рукавицею, а лише з її
зовнішнім шаром. Отже, зовнішній шар, так само, як і в одязі, і в спальнику,
має бути з міцної, дихаючої, сніговідштовхуючої тканини, бажано без всяких
підкладок і утеплювачів. Такі зовнішні рукавиці називаються верхонками, вони
приймають на себе основний удар зовнішнього середовища, практично не мокнуть
(бо там нема чому мокнути) і не рвуться (бо з міцної капронової тканини), їх
достатньо мати одну пару. А от внутрішніх теплих рукавиць (тут, як і в одязі,
найкращий матеріал – фліс) бажано мати декілька пар, щоб, поки одна сохне,
користуватись іншою. В сильний холод можна надягати кілька пар флісових
рукавиць одночасно, фліс дуже еластичний і легко дозволяє це робити. Необхідно
уважно слідкувати за сухістю рукавиць – у вологих рукавицях (а особливо – в
перчатках) на морозному вітрі пальці можна відморозити за кілька хвилин.

4. Укриття

Після холоду і вологи найбільшим ворогом незахищеної людини є вітер. Будь-який
рух повітря швидко охолоджує наше тіло (щоправда, забираючи і випарувану
вологу). Про це треба пам’ятати, і старанно ховатись від вітру. Будь-який
захист, що зменшує швидкість вітру, це вже добре. Тож будуючи/шукаючи укриття
слід пам’ятати не лише про захист від дощу/снігу, а й про захист від вітру. Так
само зовнішній шар спальника чи одягу покликаний перш за все зупинити вітер. І
зимові палатки виконують ту ж саму функцію (їх, зазвичай, роблять з
водопроникної тканини, щоб зменшити кількість конденсату). Наприклад, якщо
спальник просто обгорнути одним шаром водопроникної тканини, то це небагато
додасть до його теплоізоляції на відкритому повітрі, бо вітер легко видуватиме
з нього тепло. А якщо з цієї ж тканини зробити палатку, то спальник всередині
буде практично захищений від вітру, і теплоізоляція буде значно кращою.

Дуже гарним укриттям в польових умовах є снігові печери і нори. Сніг має гарні
теплоізоляційні властивості, спати на ньому тепліше, ніж на мерзлій землі чи на
бетоні. Якщо є можливість вирити в снігу нору чи печеру, в ній досить швидко
стає тепло і затишно, температура може навіть піднятись до плюсової, незалежно
від того, який мороз зовні. Якщо сніг пухкий, то можна нагорнути його велику
купу і вже в ній рити печеру. Туристський досвід свідчить, що таку печеру
нескладно зробити на десяток осіб, а на 1-2 особи нору викопати зовсім легко.
Спати в такій печері треба головою до виходу, щоб легко було вилізти в разі
чого. Туристи накопичили великий досвід риття снігових печер, його можна знайти
в інтернеті, в методичних статтях і в навчальних фільмах.

5. Активність

По суті, в холодних умовах головним джерелом тепла є ваше тіло. В нормі воно
виділяє достатньо тепла, якщо ним розумно розпорядитися. Основне тепло
виділяють м’язи при фізичній діяльності. Тому рух – це найкращий спосіб
зігрітися, а також підсушити на собі вологий одяг. Перегрів викликає спітніння,
що на холоді дуже шкідливо. Тож при інтенсивних нагрузках намагайтесь вчасно
роздягнутись до такого рівня, щоб не пітніти, але і не мерзнути. Пам’ятайте про
те, що зігрітись, коли вже замерз, набагато важче, ніж не дати собі замерзнути,
тож слідкуйте за своїм станом.

Отже: при найменшій нагоді намагайтесь рухатись. Заставляйте себе виконувати
якусь роботу. Холод викликає апатію, а апатія прискорює замерзання, не
потрапляйте у це замкнене коло. Якщо рухатись нема можливості, максимально
бережіть своє тепло, попереджаючи замерзання, а не долаючи його. Пам’ятайте, що
ніякі спальники і кожухи не гріють, вони лише зберігають ваше тепло, не лягайте
спати замерзлими, бо до ранку не зігрієтесь, а лише сильніше замерзнете. Не
намагайтесь зігрітись нерухомо, кутаючись в теплий одяг, так можна лише
сповільнити замерзання, а не зігрітись.

Не розраховуйте зігрітись біля багаття, це майже неможливо. Багаття створює
ілюзію тепла з одного боку, тоді як з трьох інших боків ваше тіло втрачає
тепло. Найкращий спосіб зігрітись біля багаття – це рубати для нього дрова.

6. Підсумок

Кілька узагальнених рекомендацій наостанок.

\begin{itemize}
  \item - Не бійтесь холоду, страх деморалізує і убиває. З холодом можна навчитись комфортно співіснувати. Ви це можете.
  \item - Вологі речі гріють погано, бережіть свій одяг і спальник (особливо!) від вологи, підсушуйте за кожної можливості.
  \item - Одяг і спальник не повинні бути водонепроникними, інакше вони швидко відсиріють зсередини (особливо – на холоді).
  \item - Дотримуйтесь принципу багатошаровості – зазвичай, це тепліше, надійніше і дешевше за використання одного теплого шару.
  \item - Докладіть максимум зусиль для того, щоб спати в теплі – це запорука психологічного комфорту і витривалості.
  \item - Бережіть кінцівки, приділіть їм особливу увагу.
  \item - Рух – це життя. Поки ви рухаєтесь, ви не замерзнете. За неможливості активно рухатись, бережіть своє тепло.
  \item - Голодна людина замерзає швидше. Чим більше ви перебуваєте на холоді, тим більше жирів має бути в раціоні.
\end{itemize}

Разом – переможемо! Слава Україні!

%\ii{13_12_2022.fb.trylis_volodymyr.1.vyzhyty_vzymku.orig}
%\ii{13_12_2022.fb.trylis_volodymyr.1.vyzhyty_vzymku.cmtx}
