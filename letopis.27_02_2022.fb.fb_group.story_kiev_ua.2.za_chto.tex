% vim: keymap=russian-jcukenwin
%%beginhead 
 
%%file 27_02_2022.fb.fb_group.story_kiev_ua.2.za_chto
%%parent 27_02_2022
 
%%url https://www.facebook.com/groups/story.kiev.ua/posts/1873870672809725
 
%%author_id fb_group.story_kiev_ua,bergelson_aleksandr.kiev
%%date 
 
%%tags 
%%title ВОТ ХОЧУ ПОНИМАТЬ – ЗА ЧТО?..
 
%%endhead 
 
\subsection{ВОТ ХОЧУ ПОНИМАТЬ – ЗА ЧТО?..}
\label{sec:27_02_2022.fb.fb_group.story_kiev_ua.2.za_chto}
 
\Purl{https://www.facebook.com/groups/story.kiev.ua/posts/1873870672809725}
\ifcmt
 author_begin
   author_id fb_group.story_kiev_ua,bergelson_aleksandr.kiev
 author_end
\fi

ВОТ ХОЧУ ПОНИМАТЬ – ЗА ЧТО?..

Сегодня спустил маму в убежище... Маме восемьдесят пять... Сердце, давление,
аритмия, проблемы с сосудами и с суставами, и с ногами... Острый живой ум и
неспособность усидеть на месте... Такое поколение... Такие люди...

Долгий путь с пятого на первый... А потом – в подвал... Под постоянное: - Я сама!..

\ii{27_02_2022.fb.fb_group.story_kiev_ua.2.za_chto.pic.1}

Лестница в подвал – деревянная... Из реек сбитая... Почти вертикально... А потом –
приступок в половину маминого роста... И сам подвал – место, где домовая рамка
сантехническая... Лабиринт-паутина труб в самых неожиданных местах и на
неожиданных высотах... Текущие запорные вентили... Горы битой штукатурки... Обрывки
стекловаты... Ржавые потеки под аккомпанемент падающих капель и тусклый свет пары
лампочек на свисающих со старой швабры проводах...

А вдоль стен... На всем, что можно было притащить из квартир, - ковриках,
табуретках, кусках пенопласта, целофана и картона... Старых махоньких (чтоб
поменьше места занимали) складных стульчиках и раскиданных поддонах... Серые от
страха и усталости загнанные люди... Старики, женщины и детишки... И запах подвала
смешивается с запахом безысходности и сомнабулического безразличия...

Комок в горле... Невозможно на это спокойно смотреть... Страна, загнанная и живущая
в подвалах... Кем и за что?.. Эти старики заслужили такую старость?.. Эти дети
заслужили такое детство?.. Женщины заслужили этот страх за своих детей?.. Моя
мама всей своей жизнью заслужила это, простите, говнище?.. Или миллионы других
мам?..

Кем же нужно быть, чтобы вот так – за несколько дней – сломать чужой мир?..
Уничтожить миллионы маленьких, всю жизнь выстраиваемых и бережно хранимых,
миров каждого из нас... Взмахом руки или росчерком зачеркнуть тысячи жизней... Чужих
и своих... Хотя – кому там они свои?.. Смахнуть огненной и свинцовой ладонью с
шахматной доски чужой, но НАШЕЙ земли тысячи пешек... Отработанных и ненужных...
Вопрос уже риторический...

За все в жизни придется отвечать... Там, где уже ни один вопрос не будет
риторическим... Там, где каждый вопрос станет пунктом обвинения... В земном суде
или в небесном... Свидетелей более чем достаточно... И здесь, и там... И зачтется
каждый из тех, которые там... И свои, и чужие... Каждому есть, что сказать и
предъявить... И мертвым, и живым...

Ах, как хочется, чтоб суд все же был земной... До которого дожить... И – за всех
НАШИХ мам и детей... И за ребят, которые остались лежать на СВОЕЙ земле... И за все
НАШИ разрушенные и растоптанные миры... А там, глядишь, до своих черед дойдет...

Котел точно будет отдельным!..

\ii{27_02_2022.fb.fb_group.story_kiev_ua.2.za_chto.cmt}
