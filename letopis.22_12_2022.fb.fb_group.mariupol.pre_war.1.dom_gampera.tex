%%beginhead 
 
%%file 22_12_2022.fb.fb_group.mariupol.pre_war.1.dom_gampera
%%parent 22_12_2022
 
%%url https://www.facebook.com/groups/1233789547361300/posts/1372855770121343
 
%%author_id fb_group.mariupol.pre_war,elena_mariupolskaja
%%date 22_12_2022
 
%%tags mariupol,mariupol.pre_war,mariupol.dom.gampera
%%title Дом Гампера
 
%%endhead 

\subsection{Дом Гампера}
\label{sec:22_12_2022.fb.fb_group.mariupol.pre_war.1.dom_gampera}
 
\Purl{https://www.facebook.com/groups/1233789547361300/posts/1372855770121343}
\ifcmt
 author_begin
   author_id fb_group.mariupol.pre_war,elena_mariupolskaja
 author_end
\fi

\#історіяДовоєногоМаріуполя

Немного истории

Это место всегда превлекало туристов, историков, любителей красивых фотографий
и даже входит в туристическую карту Мариуполя. Почему-же каменный спуск
называется Гамперским? Раньше его еще называли Докторским. А все потому что
рядом в необычном доме с островерхней крышей до 1911 года жил и работал очень
известный мариупольский доктор Сергей Федорович Гампер. История самого дома
уходит в начало 19-го века! Сначала это была двухэтажная усадьба. Потом над ней
возведен 3-й этаж и расширена часть флигеля. И только в 1897 году к дому
Гампера построили всем известную пристройку из красного кирпича. Эта пристройка
полностью изменила вид здания, т.к выполнена в стиле неоготика (слова откуда-то
такие знали в 1897 году!) и очень часто давало основание некоторым малосведущим
людям считать дом Гампера культовым зда­нием. Но это не мешало Сергею
Федоровичу вылечить в этом доме десятки тысяч людей!

Сейчас рядом с домом соседствуют ряд малопривлекательных сараев и
полуразрушенная дворницкая, а раньше к \enquote{готическому} строению примыкал
обширный двор с садом, огороженный внушительным забором. Внутрь двора мож­но
было проникнуть только через ворота мимо дворницкой, где днем и ночью дежурил
привратник. Первое, что бросалось в глаза посетителям, был бассейн, в котором
плава­ли прекрасные лебеди. Дорожка, посыпанная свежим морским песком, вела к
парадному входу... К сожалению этого ничего не осталось. Только одна ч\textbackslash б
фотография и та уже без забора и бассейна:

%\ii{22_12_2022.fb.fb_group.mariupol.pre_war.1.dom_gampera.cmt}
