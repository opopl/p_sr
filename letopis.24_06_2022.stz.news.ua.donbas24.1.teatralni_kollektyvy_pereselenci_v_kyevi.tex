% vim: keymap=russian-jcukenwin
%%beginhead 
 
%%file 24_06_2022.stz.news.ua.donbas24.1.teatralni_kollektyvy_pereselenci_v_kyevi
%%parent 24_06_2022
 
%%url https://donbas24.news/news/teatralni-kolektivi-pereselenci-z-mariupolya-xto-vidnoviv-robotu-v-kijevi-foto
 
%%author_id demidko_olga.mariupol,news.ua.donbas24
%%date 
 
%%tags 
%%title Театральні колективи-переселенці з Маріуполя — хто відновив роботу в Києві (ФОТО)
 
%%endhead 
 
\subsection{Театральні колективи-переселенці з Маріуполя — хто відновив роботу в Києві (ФОТО)}
\label{sec:24_06_2022.stz.news.ua.donbas24.1.teatralni_kollektyvy_pereselenci_v_kyevi}
 
\Purl{https://donbas24.news/news/teatralni-kolektivi-pereselenci-z-mariupolya-xto-vidnoviv-robotu-v-kijevi-foto}
\ifcmt
 author_begin
   author_id demidko_olga.mariupol,news.ua.donbas24
 author_end
\fi

\ifcmt
  ig https://i2.paste.pics/70e181eab7ca4e62459b95db61e73b3f.png
  @wrap center
  @width 0.9
\fi

\begin{center}
  \em\color{blue}\bfseries\Large
Маріупольські актори та режисери відновлюють творче життя після евакуації. Які
плани та де облаштувалися театрали
\end{center}

До повномасштабного вторгнення росії в Україну в Маріуполі нараховувалося 16
театральних колективів, які мали різну спрямованість, були як професійні, так і
аматорські. Більшість з цих колективів наразі не розпочали роботу через
перебування їхніх співробітників у різних містах України та Європи. А Донецький
академічний обласний драматичний театр (м. Маріуполь), який начебто відновив
свою роботу в м. Ужгород, через конфлікт з новопризначеною в. о.
директора-художнього керівника пані Людмилою Колосович так і не зміг зібрати
всіх маріупольських режисерів та акторів в одному місті.

Наразі єдиний театральний колектив, який зумів все ж відновити свою роботу
майже в повному складі в Україні — це Маріупольський театр авторської п'єси
Conception. Тепер його вистави можна буде побачити в Києві. 

\subsubsection{\enquote{Маріуполь: Мистецтво vs Війна}}

22 червня, у рамках відкриття триденного марафону \enquote{Маріуполь: Мистецтво vs
Війна}, театр показав епізод із майбутньої документальної вистави \enquote{Обличчя
кольору війна}. Захід проходив у Києві в приміщенні Центральної дитячої
бібліотеки ім. Т. Г. Шевченка. До коллективу приєдналися і актори з інших
театрів Маріуполя. Зокрема, це Євген Сосновський — актор Театральної артілі
\enquote{ДрамКом} та 12-ти річна Софія Алагірова — актриса Театру ляльок і учениця
Першої театральної школи-студії Маріуполя.

Незалежний театр авторської п'єси Conception, в якому можна знайти сучасну
авангардну режисуру і драматургію, був заснований 1 лютого 2019 року за
ініціативою творчого тандему поетеси Марії Грецької та режисера Олексія
Гнатюка. 

\begin{leftbar}
	\begingroup
		\bfseries
\qbem{Спочатку це був проєкт навчального плану — театральна студія, куди
приходили всі охочі займатися акторською майстерністю}, — розповів
Донбас24 Олексій Гнатюк. 
	\endgroup
\end{leftbar}

Пізніше режисер зрозумів, що студія готова стати справжнім театром, і почав
писати п'єсу, назва якої — \enquote{Рентген усміхнених сердець} — йому наснилася.
Колектив розпочав працювати над виставою, і вже під час роботи народилася ідея
створити театр саме авторської п'єси. У 2020 році театр переїхав до
Маріупольської камерної філармонії, де йому виділили декілька кімнат, і завдяки
зусиллям коллективу це приміщення стало досить комфортним і затишним. На стінах
можна було побачити світлини з вистав, погляд притягували і меблі, стилізовані
під старовину.

У складі трупи близько 20 осіб, серед них є досвідчені актори та ті, які
займаються в студії при театрі. В об'єднання театру входять актори з
багаторічним стажем, організатори культурно-дозвіллєвої діяльності, телеведучі,
бізнесмени, поети та студенти. Важливо, що Незалежний театр авторської п'єси
Conception не має вікових обмежень, головний критерій, щоб потрапити до трупи,
— це талант і бажання працювати. Актори наголошують, що театральне мистецтво
займає в їхньому житті найголовніше місце. При театрі продовжуватиме
функціонувати студія, де можна навчитися нелегкому ремеслу актора і закріпити
набуті навички на показах. Завдання студії — виховати актора, який досконало
володіє своїм тілом, як образним інструментом, голосом та уявою. Для цього в
процесі навчання акторській справі застосовуються різні методики і акторські
тренінги. У репертуарі театру є такі вистави: трагікомедія \enquote{Рентген усміхнених
сердець} та оперна театральна вистава за мотивом однойменної опери В. А.
Моцарта \enquote{Дон Жуан}, психологічна драма \enquote{Страхи в стилі НЮ} та фантасмагорія
\enquote{Другий шанс}.

\ii{24_06_2022.stz.news.ua.donbas24.1.teatralni_kollektyvy_pereselenci_v_kyevi.pic.1_2}

\begin{leftbar}
	\begingroup
		\bfseries
\qbem{У нас була неймовірна атмосфера, театр мав великий попит. За місяць
глядачі розкуповували всі квитки}, — пригадує засновник та режисер
театру Олексій Гнатюк. — \qbem{Мали 8 унікальних вистав, що проходили в
камерному форматі. Глядачі були учасниками спектаклів, бо в нас не було
сцени, що розділяла б акторів і глядачів. Кожну виставу намагалися
зробити як особливу подію, свято — з приємними зустрічами та живою
музикою}. 
	\endgroup
\end{leftbar}

\subsubsection{Евакуація та відродження}

До 27 березня чимало акторів знаходилися в театрі, підвал якого став укриттям.
Руйнування змусило їх полишити ріднемісто. Пішки вирушили на Мелекіне, звідти
дісталися до Бердянська, Запоріжжя і, нарешті, Києва.

Наразі у межах триденного марафону \enquote{Маріуполь: Мистецтво vs Війна} вперше
публічно показують фрагмент документальної вистави \enquote{Обличчя кольору війна}.
Олексій Гнатюк розповів, що вони збиралися протягом місяця: 

\begin{leftbar}
	\begingroup
		\bfseries
	\qbem{Перший, хто з акторів переїхав до Києва, — Олександр Пихтін, ми разом почали
пошук приміщення для театру. Звернулися до депутатки Київської міської ради
Ганни Старостенко, яка посприяла тому, що театр отримав офіс, де тепер можна
проводити збори, онлайн-конференції. Надалі приєднувалися інші актори.
Репетиції проводимо в університеті Карпенка-Карого, зараз працюємо над
виставою \enquote{Обличчя кольору війна}, презентаційний показ епізоду вже відбувся. До
речі, Дмитро Гриценко приїхав за декілька днів до прем'єри і вразив своїм
емоційним виступом. Наразі ми продовжуємо шукати власну сцену камерного
формату, щоб перенести ту атмосферу, яка в нас була в Маріуполі. На базі театру
є дитяча група, яка є безкоштовною для дітей та підлітків з Маріуполя. З
виставою \enquote{Обличчя кольору війна} плануємо гастролі
}. 		
	\endgroup
\end{leftbar}

Актор Дмитро Гриценко додав, що отримав приємні емоції від того, що знову був у
атмосфері театру.

\enquote{Це дозволяє відчувати себе живим, але все ж залишається і величезне почуття
порожнечі в середині, адже не всі з нашого театру зараз поряд і можуть
розділити емоції, отримані на сцені}.

Завдяки виставі \enquote{Обличчя кольору війна}, кожен з акторів зміг розповісти свою
історію. Як на початку війни намагалися жити і займатися звичними справами.
Однак \enquote{прильотів} ставало все більше, на п'ятий день вимкнули світло. Були
змушені під обстрілами готувати їжу. Як ховалися у підвалах і втрачали
знайомих. Глядачі не могли стримати сліз, пригадуючи власні переживання...

\begin{leftbar}
	\begingroup
		\bfseries
\qbem{Зараз театр готовий об'єднувати. Ми віримо, що мистецтво може зцілити
навіть понівечену душу. Знаємо, що обов'язково повернемося в рідний
український Маріуполь}, — завершує зворушливий показ один із акторів.
	\endgroup
\end{leftbar}

\ii{24_06_2022.stz.news.ua.donbas24.1.teatralni_kollektyvy_pereselenci_v_kyevi.pic.3_4}

\enquote{Ми хочемо показати, що ми не зломлені. Готові й надалі розвиватися,
відновлювати культуру нашого міста, — наголошує Олексій Гнатюк. — До нас
приєдналися й інші актори та культурні діячі Маріуполя, яким не вистачає сцени.
Плануємо велику виставу з використанням диджітал-технологій. Художники в
реальному часі будуть малювати події, що відбувалися в Маріуполі. Наразі в
трупі 14 людей, і до нас постійно приєднуються нові учасники}.

Отже, режисер Олексій Гнатюк разом із маріупольськими акторами готовий і надалі
завойовувати любов глядачів, адже актори постійно вдосконалюють свою
майстерність і завдяки креативному баченню режиссера будуть приємно вражати
своїми новими виставами та продовжувати розвивати театральну культуру
Маріуполя.

Нагадаємо, раніше Донбас24 писав про те, що \href{https://donbas24.news/news/mariupolski-mitci-u-stolici-pokazut-silu-ta-krasu-mista-mariyi-de-podivitisya}{маріупольські митці в столиці
покажуть силу та красу міста Марії}.

\textbf{Читайте також:} \href{https://donbas24.news/news/nezlamna-azovstal-top-12-mist-zi-stritartom-na-cest-zaxisnikiv-mariupolya}{ТОП-12 міст зі стріт-артом на честь захисників Маріуполя}.

Фото з репетиції: Євген Сосновський

\ii{insert.author.demidko_olga}

