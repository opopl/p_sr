% vim: keymap=russian-jcukenwin
%%beginhead 
 
%%file 24_11_2021.fb.bryhar_sergej.1.detsad_mova
%%parent 24_11_2021
 
%%url https://www.facebook.com/serhiibryhar/posts/1841453322721278
 
%%author_id bryhar_sergej
%%date 
 
%%tags deti,detsad,jazyk,mova,ukraina,ukrainizacia
%%title Всі державні дитячі садочки мають бути виключно україномовними
 
%%endhead 
 
\subsection{Всі державні дитячі садочки мають бути виключно україномовними}
\label{sec:24_11_2021.fb.bryhar_sergej.1.detsad_mova}
 
\Purl{https://www.facebook.com/serhiibryhar/posts/1841453322721278}
\ifcmt
 author_begin
   author_id bryhar_sergej
 author_end
\fi

\obeycr
Розповім вам про один показовий, як на мене, епізод.
Місце: східна околиця Одеси.
Дійові особи: 28-річна місцева мешканка і я.
- Всі державні дитячі садочки мають бути виключно україномовними, - говорю я. І додаю: - Нехай нацменшини самоорганізовуються за власні кошти. Це і є цивілізована практика.
- Я нє "нацменшина". Я украінка. Но говорю по-русскі. Моі дєті - тоже.
- От вона, нелогічність, яку потрібно змінювати.
- Зачєм?
- Для світлого майбутнього України.
- Ти новості смотріш? Я вот іногда смотрю, - питанням на питання відповідає співрозмовниця.
- Дивлюся. А частіше читаю.
\restorecr

\ii{24_11_2021.fb.bryhar_sergej.1.detsad_mova.cmtfront}

\obeycr
- Росія явно что-то задумиваєт на граніце.
- Так. Їхня мета - приєднання України. І вони не зупиняться.
- Ну так зачєм же рісковать?
- Тобто?
- Ну єслі оні нас захватят, а ми тут по-укрАінскі говорім, ілі наши дєті, нас же по головкє нє погладят, правільно?
- Стривай. А чому це ворог має гладити нас по голівці?
- У мєня двоє дєтєй, которих нужно корміть, обєспєчівать і обєрєгать.
- В мене теж двоє дітей. Їх теж потрібно годувати, вдягати, оберігати. І я вважаю, що найкращий шлях до безпеки - розбудова сильної держави з домінуванням національної мови та відповідного культурного фону.
- Ето бравада. А в жизні нужна прагматічность.
- Поясни.
- Бєзопаснєє тогда, когда ти подстраіваєшся под дєйствітєльность.
- На Донбасі теж багато тих, хто підлаштовувалися під якусь дебільну неукраїнську дійсність. І що, їх це вберегло від повної дупи?
- Ти мєня нє слишишь.
- Власне, як і ти мене...
Я не соціолог, щоб говорити про відсотки, але відчувається, що їх таких багато. Аж занадто. Вони, звичайно, не москвофіли, не колаборанти, на сепари. Їм просто все одно. Вони приймають те, що переважає, тобто російське. Вони готові прийняти інший прапор над міською радою і підлаштуватися під нову владу. Особливо якщо вона пообіцяє підвищить зарплати і пенсії, пообіцяє вирішити їхні проблеми, в ідеалі - без їхньої участі. А переймаються вони лише приватним, матеріальним. Інше - "лішнєє", "бравада", "нє прагматічно".
Їм часом ще й подобається бути "майже росіянами". Іноді вони вважають, що так безпечніше. Іноді не задумуються навіть про це...
\restorecr

\ii{24_11_2021.fb.bryhar_sergej.1.detsad_mova.cmt}
