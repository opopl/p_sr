% vim: keymap=russian-jcukenwin
%%beginhead 
 
%%file slova.most
%%parent slova
 
%%url 
 
%%author 
%%author_id 
%%author_url 
 
%%tags 
%%title 
 
%%endhead 
\chapter{Мост}
\label{sec:slova.most}

%%%cit
%%%cit_pic
%%%cit_text
Жизнь князя Владимира, крестившего Русь в 988 году, была бурной, неспокойной и
неоднозначной. И в мир оной он тоже отошел отнюдь не с улыбкой смирения на
устах. До последнего момента в нем кипела энергия. Тело уже не слушалась, но
мозг продолжал свою кипучую деятельность.  Летопись сохранила для нас
предсмертную фразу, произнесенную князем. Он не молился, не каялся, он отдавал
команду: \enquote{Требите путь и \emph{мостите мост}.  Куда собирался
отправляться Владимир? Для чего ему нужно было расчищать дорогу и строить
\emph{мосты}? Все просто. Он ни много ни мало собирался в очередной раз отправиться в
военный поход
%%%cit_comment
%%%cit_title
\citTitle{Мостите мост. Что означали последние слова князя Владимира?}, 
Русичи, zen.yandex.ru, 09.06.2021
%%%endcit

