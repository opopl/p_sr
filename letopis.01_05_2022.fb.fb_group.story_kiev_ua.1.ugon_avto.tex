% vim: keymap=russian-jcukenwin
%%beginhead 
 
%%file 01_05_2022.fb.fb_group.story_kiev_ua.1.ugon_avto
%%parent 01_05_2022
 
%%url https://www.facebook.com/groups/story.kiev.ua/posts/1914722575391201
 
%%author_id fb_group.story_kiev_ua,polovinko_svetlana
%%date 
 
%%tags 
%%title Вибачте я угнав ваше авто. Спасав родину
 
%%endhead 
 
\subsection{Вибачте я угнав ваше авто. Спасав родину}
\label{sec:01_05_2022.fb.fb_group.story_kiev_ua.1.ugon_avto}
 
\Purl{https://www.facebook.com/groups/story.kiev.ua/posts/1914722575391201}
\ifcmt
 author_begin
   author_id fb_group.story_kiev_ua,polovinko_svetlana
 author_end
\fi

%\ii{01_05_2022.fb.fb_group.story_kiev_ua.1.ugon_avto.eng}

\raggedcolumns
\begin{multicols}{2} % {
\setlength{\parindent}{0pt}

\ii{01_05_2022.fb.fb_group.story_kiev_ua.1.ugon_avto.pic.1}

Пишуть в неті, що це розповідь з Київщини, коли окупанти сподівалися зайти в
Київ.

Одна людина вийшла з бомбосховища і побачила біля магазину машину з ключами в
запаленні. Дві години спостерігав за нею - чекав господаря. Не дочекався. Взяв
родину сів у цю машину і виїхав у Вінницю до родичів. В бардачку знайшов номер
телефону. Дозвонився до власника: \enquote{Вибачте я угнав ваше авто. Спасав
родину}.

Господар відповів:

\enquote{Слава Богу, не хвилюйтеся, в мене чотири авто. На своєму джипі я вивіз
свою сім'ю. Інші авто заправив і лишив у різних місцях з ключами в запаленні;
номер телефону залишив у бардачку. І вже із всіх машин мені перезвонили.
Настане мир - зустрінемося. Хай Господь береже Вас.}... @igg{fbicon.heart.red}
@igg{fbicon.flag.ukraina} @igg{fbicon.thumb.up.yellow}

\end{multicols} % }

\ii{01_05_2022.fb.fb_group.story_kiev_ua.1.ugon_avto.cmt}
\ii{01_05_2022.fb.fb_group.story_kiev_ua.1.ugon_avto.cmtx}
