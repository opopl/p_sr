% vim: keymap=russian-jcukenwin
%%beginhead 
 
%%file slova.afrika
%%parent slova
 
%%url 
 
%%author 
%%author_id 
%%author_url 
 
%%tags 
%%title 
 
%%endhead 
\chapter{Африка}
\label{sec:slova.afrika}

%%%cit
%%%cit_pic
%%%cit_text
Жемчужину \emph{белой Африки} с правами и свободами для избранных.  В нашей
белой \emph{Африке}.  С интересом наблюдаю за славагеройской эпопеей на
футболках сборной Украины. В УЕФА сказали «да», в УЕФА сказали «нет», в России
возмущены, власти Украины восхищены... Не удивлюсь, если «героям слава» нарисуют
в виде пиктограммы, а может даже тайно зашьют на внутреннюю часть труселей. Тут
ведь уже движ пошел и его никакой разум не остановит.  Вспоминаю 14-й год, вся
страна, вернее, та ее часть, которой не нужен скипидар на жопу, закрашивала
движимое и недвижимое имущество в желто-синие цвета. Заборы, столбы, лавочки,
деревья, вот все. У меня у подъезда перекрасили даже урну. Феерическое зрелище.
Помойка в цвете национального флага. Правда, через несколько дней на ней же
дописали черным мелком «Росiя». Но это еще только больше подчеркнуло ебанариум
головного мозга желто-голубых покрасчиков. История с футболками – это из той
же клиники. Только процесс уже институализирован и продвигается
централизованно. Любое нейтральное явление, которое еще не разделяет общество,
вроде того же спорта, непременно надо засрать политическими лозунгами. И
обязательно бандеровскими. С милыми сердцу рунами и зигами
%%%cit_comment
%%%cit_title
\citTitle{Бандеровцы при власти создали в Украине уютненькую Уганду}, 
Игорь Лесев, strana.ua, 13.06.2021
%%%endcit

