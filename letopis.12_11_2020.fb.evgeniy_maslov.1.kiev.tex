% vim: keymap=russian-jcukenwin
%%beginhead 
 
%%file 12_11_2020.fb.evgeniy_maslov.1.kiev
%%parent 12_11_2020
 
%%url https://www.facebook.com/maslovevgeniy14/posts/1345513072457064
%%author 
%%tags 
%%title 
 
%%endhead 

\subsection{История моего Города... - Киев --- эпоха 60-х}
\label{sec:12_11_2020.fb.evgeniy_maslov.1.kiev}
\Purl{https://www.facebook.com/maslovevgeniy14/posts/1345513072457064}
\Pauthor{Маслов, Евгений}

\index[cities.rus]{Киев!История!60-тые}

История моего Города... - Киев --- эпоха 60-х
(материал взят из сайта ККК)
"Киев --- эпоха 60-х.

В 60-е годы, если кто впервые подъезжал поездом к Киеву с северного
направления, то был неожиданно удивлен и восхищен тем, что... оказывается, за
этими цветущими, историческими склонами на правом берегу Днепра скрывается
огромный, красивейший город!

\ifcmt
pic https://scontent.fiev22-2.fna.fbcdn.net/v/t1.0-9/125105808_1345511582457213_4147192108227799826_n.jpg?_nc_cat=111&ccb=2&_nc_sid=8bfeb9&_nc_ohc=fNTD_JxH79YAX8k9EuM&_nc_ht=scontent.fiev22-2.fna&oh=89a52ddcb7c8298e1ea5611d4473677e&oe=5FD4FB2B

pic https://scontent.fiev22-1.fna.fbcdn.net/v/t1.0-9/125152761_1345511629123875_1619093231775589298_n.jpg?_nc_cat=100&ccb=2&_nc_sid=8bfeb9&_nc_ohc=3bD8spqxJ1UAX98PLzk&_nc_ht=scontent.fiev22-1.fna&oh=df6f8508e1c447ac4abb285904b747d8&oe=5FD2BAA8

pic https://scontent.fiev22-1.fna.fbcdn.net/v/t1.0-9/124901310_1345511692457202_7065160434196303011_n.jpg?_nc_cat=109&ccb=2&_nc_sid=8bfeb9&_nc_ohc=J4YXyMMe4coAX9Q7_zz&_nc_ht=scontent.fiev22-1.fna&oh=aa744944ccdece185e498a36e2895e9e&oe=5FD1C861

pic https://scontent.fiev22-1.fna.fbcdn.net/v/t1.0-9/124578343_1345511759123862_622562091379951357_n.jpg?_nc_cat=105&ccb=2&_nc_sid=8bfeb9&_nc_ohc=aNYAMBN1hRkAX8DrDos&_nc_ht=scontent.fiev22-1.fna&oh=33c7171afe448a6c717fdcad76038b23&oe=5FD3F00A

pic https://scontent.fiev22-1.fna.fbcdn.net/v/t1.0-9/125064615_1345511795790525_4532824245940155473_n.jpg?_nc_cat=106&ccb=2&_nc_sid=8bfeb9&_nc_ohc=JIunAkKme7MAX9E2sV1&_nc_ht=scontent.fiev22-1.fna&oh=7c14ba9e750aaa9d607711046fb50553&oe=5FD22CD3

pic https://scontent.fiev22-1.fna.fbcdn.net/v/t1.0-9/125070157_1345511839123854_6600119526084032902_n.jpg?_nc_cat=109&ccb=2&_nc_sid=8bfeb9&_nc_ohc=4LgruzOlM10AX93WjlC&_nc_ht=scontent.fiev22-1.fna&oh=d633c627f160adb3d10d7feb8281c457&oe=5FD27B4C

pic https://scontent.fiev22-1.fna.fbcdn.net/v/t1.0-9/124657040_1345511882457183_1148153215983347932_n.jpg?_nc_cat=100&ccb=2&_nc_sid=8bfeb9&_nc_ohc=IbfU33LWeJ8AX8W5rxL&_nc_ht=scontent.fiev22-1.fna&oh=933fe2f4522e8f706048d2e97eb31609&oe=5FD28C81

pic https://scontent.fiev22-2.fna.fbcdn.net/v/t1.0-9/124968606_1345511919123846_3212518600024787888_n.jpg?_nc_cat=110&ccb=2&_nc_sid=8bfeb9&_nc_ohc=86TmuEnHHewAX9HiZxe&_nc_ht=scontent.fiev22-2.fna&oh=b70cee39c63e6a850691fd9a8b04dd4a&oe=5FD23570

pic https://scontent.fiev22-1.fna.fbcdn.net/v/t1.0-9/124977033_1345511995790505_4503016914671798723_n.jpg?_nc_cat=107&ccb=2&_nc_sid=8bfeb9&_nc_ohc=OdG0lRIQRVoAX-BZ5-A&_nc_ht=scontent.fiev22-1.fna&oh=a30f2e882192329f82ce91d64f3cadf7&oe=5FD4439A

pic https://scontent.fiev22-1.fna.fbcdn.net/v/t1.0-9/125229875_1345512075790497_6641733384824801913_n.jpg?_nc_cat=108&ccb=2&_nc_sid=8bfeb9&_nc_ohc=wqAyJzCSwxkAX8_Sbdm&_nc_ht=scontent.fiev22-1.fna&oh=b9001dd26bc2dbbc3cedeae051a9a2db&oe=5FD52906

pic https://scontent.fiev22-2.fna.fbcdn.net/v/t1.0-9/125175249_1345512135790491_6731009726725776791_n.jpg?_nc_cat=101&ccb=2&_nc_sid=8bfeb9&_nc_ohc=nC87PmikD6cAX-YX-r7&_nc_ht=scontent.fiev22-2.fna&oh=da6503ad317ae5dc266233d97186ecab&oe=5FD1AACC

pic https://scontent.fiev22-1.fna.fbcdn.net/v/t1.0-9/125114536_1345512235790481_1549718983910124094_n.jpg?_nc_cat=103&ccb=2&_nc_sid=8bfeb9&_nc_ohc=1tZE3BE-RzsAX9SOsdu&_nc_ht=scontent.fiev22-1.fna&oh=85d7a70873e4f9ef88bd0cfcd65c31da&oe=5FD48E07

pic https://scontent.fiev22-1.fna.fbcdn.net/v/t1.0-9/125147507_1345512282457143_1064507215451887220_n.jpg?_nc_cat=109&cb=846ca55b-311e05c7&ccb=2&_nc_sid=8bfeb9&_nc_ohc=TEDQeLFK9dgAX9MyBCJ&_nc_ht=scontent.fiev22-1.fna&oh=af4d4b0430c7b090ed81e9e4a2ab1730&oe=5FD4C195

pic https://scontent.fiev22-1.fna.fbcdn.net/v/t1.0-9/125164695_1345512329123805_8738578873613125722_n.jpg?_nc_cat=109&ccb=2&_nc_sid=8bfeb9&_nc_ohc=ACbMYuTjaMQAX8dHBoa&_nc_ht=scontent.fiev22-1.fna&oh=77e702ccbb2a1f037460b99773c6e489&oe=5FD4919B

pic https://scontent.fiev22-1.fna.fbcdn.net/v/t1.0-9/124940676_1345512405790464_619377425182520707_n.jpg?_nc_cat=109&ccb=2&_nc_sid=8bfeb9&_nc_ohc=xWg4nfr71VQAX-J08Tq&_nc_ht=scontent.fiev22-1.fna&oh=935a8b2a30365bd5ab9db5cbdf3f8b82&oe=5FD25D53

pic https://scontent.fiev22-1.fna.fbcdn.net/v/t1.0-9/125181269_1345512505790454_8609625740655379523_n.jpg?_nc_cat=103&ccb=2&_nc_sid=8bfeb9&_nc_ohc=I0aHf4AO_2EAX-KGgz3&_nc_ht=scontent.fiev22-1.fna&oh=66c6823db4e0d8f90cffe77bddb97e25&oe=5FD155FC

pic https://scontent.fiev22-1.fna.fbcdn.net/v/t1.0-9/125005035_1345512572457114_1125598101116374044_n.jpg?_nc_cat=107&ccb=2&_nc_sid=8bfeb9&_nc_ohc=3QMmQfEeXFoAX9qHxnf&_nc_ht=scontent.fiev22-1.fna&oh=3a4b339dda1a229aaafa945d8d467abe&oe=5FD2662D

pic https://scontent.fiev22-2.fna.fbcdn.net/v/t1.0-9/125187239_1345512612457110_440758516394773583_n.jpg?_nc_cat=102&ccb=2&_nc_sid=8bfeb9&_nc_ohc=A5ZoB-1sSXUAX8N3ia_&_nc_ht=scontent.fiev22-2.fna&oh=fde6882022d9d16ccfc762d61b6ecbea&oe=5FD3CA12

pic https://scontent.fiev22-2.fna.fbcdn.net/v/t1.0-9/125172751_1345512682457103_8797877896664458866_n.jpg?_nc_cat=104&ccb=2&_nc_sid=8bfeb9&_nc_ohc=fRhQyD8MyCkAX_0icZo&_nc_ht=scontent.fiev22-2.fna&oh=f1e2e2b3f7439e222bd711d069603738&oe=5FD496EB

pic https://scontent.fiev22-2.fna.fbcdn.net/v/t1.0-9/124768858_1345512759123762_721928292889120771_n.jpg?_nc_cat=110&ccb=2&_nc_sid=8bfeb9&_nc_ohc=8AAJOY5kYkoAX_DUDOU&_nc_ht=scontent.fiev22-2.fna&oh=cd26e43d802378dbe93024fc32d8c956&oe=5FD4B1C7

pic https://scontent.fiev22-2.fna.fbcdn.net/v/t1.0-9/125144776_1345512802457091_8608635195534996064_n.jpg?_nc_cat=110&ccb=2&_nc_sid=8bfeb9&_nc_ohc=HRl3FQk542YAX-LWJbU&_nc_ht=scontent.fiev22-2.fna&oh=42028f114be19155199981222daad40b&oe=5FD203A0

pic https://scontent.fiev22-1.fna.fbcdn.net/v/t1.0-9/124550707_1345512852457086_4555466519013835142_n.jpg?_nc_cat=105&ccb=2&_nc_sid=8bfeb9&_nc_ohc=eB1RS-hVPJEAX_2qx9Q&_nc_ht=scontent.fiev22-1.fna&oh=72da82d74f880f52581934ef2b5a11da&oe=5FD400F4

pic https://scontent.fiev22-2.fna.fbcdn.net/v/t1.0-9/125204601_1345512905790414_7972527130774095028_n.jpg?_nc_cat=101&ccb=2&_nc_sid=8bfeb9&_nc_ohc=0KJkCSKPy3kAX-se0H4&_nc_ht=scontent.fiev22-2.fna&oh=5e29f50e4ee42b9b1fd65243ec2dc94a&oe=5FD322D3

pic https://scontent.fiev22-1.fna.fbcdn.net/v/t1.0-9/124953014_1345512959123742_5761139803790944212_n.jpg?_nc_cat=105&ccb=2&_nc_sid=8bfeb9&_nc_ohc=xF-mVPaiissAX-7fUay&_nc_ht=scontent.fiev22-1.fna&oh=5521447004cbfc8a932f5256621add6b&oe=5FD1F036

pic https://scontent.fiev22-1.fna.fbcdn.net/v/t1.0-9/124642779_1345512999123738_4228731096240823772_n.jpg?_nc_cat=107&ccb=2&_nc_sid=8bfeb9&_nc_ohc=ZOykZmt8N0gAX9J4v6s&_nc_ht=scontent.fiev22-1.fna&oh=be6a2529e0613631431905c76f274dd1&oe=5FD260DF

pic https://scontent.fiev22-1.fna.fbcdn.net/v/t1.0-9/125152740_1345513045790400_5988129321790391036_n.jpg?_nc_cat=105&ccb=2&_nc_sid=8bfeb9&_nc_ohc=BFAx5K1FagAAX_4mlOy&_nc_ht=scontent.fiev22-1.fna&oh=f98731969b6fe904e9386b6ec1d73051&oe=5FD3BDBA
\fi

За последние лет шестьдесят Киев с полумиллионного населения вырос до четырех
миллионов, то есть почти в десять раз! Если в 60-е годы город вырос троекратно
за счет наплыва сельского населения, ранее отпущенного «вольной» Хрущева и
расселенного вскоре в новые дома, прозванные «хрущевками», то уже в конце 90-х
город разбух от стихийной, полулегальной рабочей силы со всех регионов (как
город «хлебный» Ташкент в 40-е). Если первая волна характеризовалась
промышленным бумом советской экономики, то вторая волна была связана с
патологическим кризисом чуть ли не всего сущего, когда ежедневный приток
рабочей силы с пригородов в 140 тысяч человек сливался с десятками тысяч новых
«остарбайтеров» отовсюду, чтобы обслуживать и обустраивать мегаполис с его
дачно-развлекательным комплексом.

С начала 60-х годов наши люди начинали жить значительно лучше, чем за все время до того. Однако стоит отметить, что в эти же годы в Европе, например в Швеции, Франции, особенно в Италии, было еще очень много нищеты, чего у нас уже тогда не было. Это потом, в 70-е годы Запад вырвался вперед благодаря «чуткому» руководству в лице деградировавшего генсека Брежнева со своим культом «военщины» и разбазаривания средств, так тяжело заработанных народом в 50-е годы. 

Тогда, в 60-е, в магазинах были высококачественные продукты. Тогдашний ГОСТ и
всеохватывающий контроль всяческих органов обеспечивали настоящие хлеб,
сливочное масло, колбасы, пиво, соки, конфеты и мороженое для детей, а не
теперешний фальсификат. В столовых был бесплатный хлеб, горчица, соль, перец; в
школах, детсадах было почти бесплатное питание. В летнее время детям, в целом,
был обеспечен отдых в пионерлагерях в лесной зоне города. Цены на продукты,
можно сказать, были смешные, хотя зарплата у большинства людей была и невысокая
(от 60 до 120 рублей), однако питание было легкодоступным и качественным. 

В начале 60-х, как эксперимент, в троллейбусах применялось самообилечивание,
когда пассажир сам бросал монеты в кассу, отрывал билет на полном доверии. И
зарплату на многих крупных предприятиях работники получали сами, беря ее с
пачки денег на общем столе с ведомостью, --- также на полном доверии. Появились
первые магазины-самообслуживания, где автоматы за жетоны подавали бутерброды и
кондитерские изделия, и такие теперь уже забытые «жулики», «марципаны»,
«сдоба»... 

На остановках троллейбуса люди стояли строго в очередь на заднюю площадку, и
никто не смел забегать наперед. Дети любили ездить на «колбасе», цепляясь сзади
трамвая или троллейбуса и спрыгивая на ходу. 

Киев был очень интернациональным и демократичным городом (хотя Одесса была
более). Украинцы, русские, евреи, поляки, белорусы, молдаване и т.д жили
дружно, но не без нюансов. Например, грузины и прибалты шли за иностранцев, за
их непохожесть и специфику. Грузины --- за свою «цитрусовую» обеспеченность,
латыши --- за их «малую» Европу с городом Рига, тогдашней законодательницей мод в
СССР наряду с Москвой и Ленинградом. У каждой национальности были и свои
прозвища: у грузин --- «кацо», у россиян --- «москаль/кацап», у белорусов –
«бульбаши», у молдаван --- «мамалыжник», у евреев --- «маланец», «парижанин». Тех,
кто из села, называли «черти». Очень распространено было слово «жлоб» (да и по
сей день). Если в каком-то давнем его начале оно носило оттенок происхождения,
то позже оно стало нарицательным за соответствующие качества. 

Некоторый шовинизм и пренебрежение коренных киевлян все же компенсировалось их
исконно славянской добротой, дружелюбностью. Подтверждение тому --- совместные
застолья, дружбы, браки, работа, увлечения (особенно рыбалка). Потому что, в
основном, все жили одинаково, и делить было нечего. 

Все эти нюансы были характерны всегда и для всего мира. В Европе, например,
французы насмешливы к немцам и бельгийцам, бельгийцы к фламандцам и
фландрийцам, испанцы к каталонцам, в Канаде Квебек к англичанам, даже у
американских штатов свои клички, а в России --- «челдоны», «чухонцы», «чурки»,
«пермяк соленые уши» и известный «лимита», московский шовинизм к остальной
части своей страны. Такова природа Homо S. 

В те же годы в Киеве пошла другая волна --- лавина иностранцев, но уже
чужеземцев: афроамериканцы, а позднее арабы. Недальновидная политика Политбюро
открыла страну чуждым и специфичным народам. Эта лавина, наряду с учебой в
лучших вузах страны, принесла чуждый менталитет в традиционные славянские
отношения между мужчиной и женщиной. Они искушали наших девушек «шмотками»,
купленными по ходу в Европе на барахолках, а ввоз пластинок и записей модной
рок-музыки развращал и сбивал с толку множество самобытных
вокально-инструментальных ансамблей (ВИА), у многих из которых были свои
талантливые направления, произведения и успех. 

Также стоит отметить и бурную «гендлевую» деятельность поляков из братского
соцлагеря, которые, также ввозя нам «шмотки», вывозили взамен высокопробные
золотые изделия и бриллианты (с нашими смешными ценами: золото --- 1 г за 5 руб.,
бриллианты --- 1 карат около 300 руб., ожерелье из натурального крупного жемчуга
– 400 руб.). Не говоря уже о серебряных и мельхиоровых изделиях, которые позже
массово вывозили арабы. Примечательно, но даже наши бабушки тех лет с пенсии
покупали внучкам на дни рождения золотые колечки и сережки! А сейчас?! 

Что уж говорить о представителях власти, номенклатуры, заслуженных деятелях и
их покупках и антикварных увлечениях. Вспомнить хотя бы примеры народной
артистки Зои Федоровой, снабжавшей балерона Рудольфа Нуриева в Париже
исключительно высокораритетными вещами через «окно» в Шереметьево, по связям в
КГБ; позже министра МВД Щелокова; Галины Брежневой; морские караваны леса с
министерства обороны; замминистра рыбной промышленности с черной икрой за
кордон в больших селедочных банках и т.д. и т.п. Хотя следует отметить и
«успехи» многих сегодняшних представителей власти, в сравнении с которыми те
некоторые были просто дети! 

Как некоторые руководящие коммунисты и сексотство в КГБ (НКВД) испортили
советский народ, так и иностранцы за «шмотки» испортили часть молодежи.

В те годы язык общения в Киеве был русский, хотя с характерным «шо», «гы», «о».
Разговаривать на украинском очень стеснялись. Только у интеллигенции (тогда она
еще была) да потомков мещан произношение было близко к ленинградскому. НО!!!
Украинский язык без ограничений был во всех официальных, публичных сферах.
Книжно-журнальной, газетной продукции на украинском языке было в несколько раз
больше, чем сейчас. Парадокс!

Огромный успех был у театра им. Франко с такими блестящими актерами, как Ужвий,
Юра, Бучма, Яковченко, Милютенко, Хвыля, Заднипровский. Яркими с национальным
колоритом бывали фильмы киностудии им. Довженко. Что только значит фильм «За
двумя зайцами» с непревзойденными Борисовым, Криницыной, Копержинской,
Яковченко! Танцевальный ансамбль им. Вирского и Хор им. Веревки соперничали с
российскими Моисеева и Пятницкого. В цирке проходили аншлаги: со знаменитым
иллюзионистом Кио, дрессировщицей Назаровой с тиграми, клоунами Карандаш и
феноменом Енгибаровым, и отчаянным Маяцким с трюком на мотоцикле в
раскрывающемся шаре под куполом! 

Дети в восторге ватагами бегали в цирк, в кино, многие «зайцем». В кинотеатры
вечером (за 30 коп.) трудно было попасть на знаменитые американские,
французские и итальянские фильмы, такие как «Великолепная семерка», «Фантомас»,
«Три мушкетера», «Развод по-итальянски», «Рокко и его братья». Да и «Тихий
Дон», «Война и Мир», «Хождение по мукам» и «Судьба человека» с не меньшим
успехом шли на экранах. Самыми посещаемыми были следующие кинотеатры:
«Довженко», «Спутник», «Киев», «Комсомолец Украины», «Коммунар», «Жовтень»,
«Зірка». 

Люди интересовались литературными новинками, поэзией, спектаклями. А концерт
Высоцкого в 1966 году в КПИ?! Что сказать, была, в целом, большая культура и
культурный народ в городе Киеве. Были очень популярны певцы югославской эстрады
– Джордже Марьянович и Радмила Караклаич, чешской --- Карел Готт и Гелена
Вондрачкова, певцы советские --- Эдита Пьеха, блистательный Муслим Магомаев, Майя
Кристалинская, великолепный цыганский певец Николай Сличенко, киевские --- Юрий
Гуляев, Дмитрий Гнатюк, Борис Гмыря, джазовые ансамбли Лундстрема и Эдди
Рознера…

Когда началось массовое увлечение фигурным катанием, то с огромными очередями
на выступления прошли «Венский балет на льду», «Праздник на льду» (США). А
знаменитая выставка «Средства связи США» в 1963 году с демонстрацией
невероятных для нас достижений в телерадиосвязи и, конечно, с массой в штатском
от КГБ?! 

В те годы были открыты: новое здание Цирка, Дворца спорта, первой линии
фантастического тогда метро (от ст. «Вокзальная» до ст. «Днипро»). И все это
происходило как праздник, в котором участвовали тысячи и тысячи людей --- от мала
до велика!  А официальные праздники, такие как 7 Ноября, 1 Мая и 9 Мая? Когда
перед этим красились скамейки в парках, белились деревья и бордюры тротуаров, в
воздухе витал особенный, свежий запах краски, что предвещало и уже радовало
тысячи и тысячи детей предстоящими вкусностями, сладостями, мороженым, особенно
сладкой водой (ситро, лимонад, крюшон). Продавались красные флажки с шариками,
а мальчишки в праздничной толпе стреляли с рогатки по шарикам, в восторге от их
лопанья. В парках играли духовые оркестры, из громкоговорителей на площадях
звучала торжественная музыка. Вечером на площадях люди танцевали под духовые
оркестры, а дети бегали между танцующими, играли, веселились вместе со
взрослыми. В эти праздники, уже с утра, все собирались у телевизора смотреть
грандиозный военный парад, в семьях накрывали праздничные столы, одевались во
все чистое, и даже в обновки. Тогда еще были такие старые традиции, как
«справить костюм» к праздникам. Дети тех лет запоминали это на всю жизнь.

В 1964 году киевляне, в основном молодежь, пристрастились к новому белому
крепкому вину, которое нарекли «Биомицин» (по 1 р. 22 коп.). Это действительно
было довольно качественное и вкусное недорогое вино, которое стало на несколько
лет сопровождать все совместные всевозможные выпивки, свидания и любовь. В те
годы киевляне, в основном молодежь, пили вина, несмотря на несметное количество
коньяка, веером выставленного в витринах, как и рыбные консервы, но не
получавшего одобрения из-за того, что, как определили опытные киевляне, «он пах
клопами». А полюбили все и сразу одно --- «Біле міцне», так называемый
«Биомицин»! Рабочая молодежь тех лет (тогда все работали) очень изрядно
потребляла и крепкие вина, которые в огромном и дешевом ассортименте были в
магазинах. Вечерами во дворах, парках и подъездах слышались «блатные» песни под
гитару, молодежное веселье и, конечно, не редкие драки. Из-за злоупотребления
такими часто суррогатными дешевыми винами (из-за них очень многие спивались),
типа «Портвейн» и «Вермут», на улицах и в транспорте часто происходили
негигиеничные последствия. 

Быстро вошел в новую закуску при застольях салат «оливье», который успешно
заменил старинный знаменитый его величество винегрет! Тогда же еще на некоторых
улицах города продавали на разлив белое столовое вино из цистерн на колесах по
30 коп. стакан. Из синих, дымящихся деревянных лотков на колесах продавались
горячие пирожки --- жареные, коричневые, с хрустящей корочкой и настоящим ливером
по 4 коп., да еще с газированной водой с сиропом по 4 коп.! Это яство с
удовольствием ели горожане и в обеденный перерыв, выбегая с работы пообедать
3-4 пирожками с газированной водой, за все --- 20 коп.! И это действительно было
очень своеобразно вкусно и, наверное, самый дешевый обед в мире. 

Были очень популярны и молочные кафе, где за копейки можно было перекусить
горячим молоком, простым кофе с (настоящим!) бубликом или городской булочкой.
Не говоря уже о богатом ассортименте таких вкусных хлебобулочных изделий, как
«арнаут», «паляница», «бородинский», «сайка», «жулик», «сдоба». Также и
кулинарные кафе, где за те же копейки был и горячий бульон в кружках, простые
горячие котлеты, шницель, не говоря уже о самых дешевых в мире столовых, пусть
и не высокого качества, но кормивших простой народ десятилетиями!  В то время
киевляне покупали продукты небольшими количествами, обычно после работы:
200-400 г вареной колбасы типа «докторская», отдельная, ветчинно-рубленая или
любительская (от 2 р. 10 коп. до 2 р. 80 коп. за 1 кг), масло сливочное (и
вологодское) настоящее 100-200 г (по 3 р. 50 коп. за 1 кг)! Никто не покупал
впрок, в запас, оптом. Это позже, в семидесятые, постепенно многого стало
недоставать, исчезать и появился злополучный дефицит. 

У киевлян в домашнем меню (кроме знаменитых киевских котлет и киевского торта)
преобладали всевозможные борщи, капустняки, рассольники, картошка в разных
видах, домашняя лапша, птица, жареная речная рыба, таранка (особенно, если в
семье были рыболовы, а таких семей было большинство). Морская рыба была
настолько дешевая, что пенсионерки кормили ею своих любимых кошек. На праздники
пеклись огромные пироги с вареньем, с узорами, различные пирожки, печенье
домашнее, мясное жаркое и, конечно, тоже его величество --- «холодец»! Еще все
варили варенье --- старейшая киевская традиция, начиная со знаменитого на весь
мир еще с 18 века «сухого» варенья. Вообще киевляне всегда питались очень
качественно и разнообразно, в отличие от обычной кухни России и Беларуси.
Большую роль в кухне киевлян сыграла еврейская кулинария с ее знаменитыми
форшмаком, фаршированной рыбой, паштетами, рулетами, блюдами из птицы. 

В то время в Киеве насчитывалось более 200 тысяч евреев на 1 миллион жителей
города. Этот народ сыграл большую роль в культуре города, в языке, манере и
даже акценте, подобно одесскому. Был даже период, когда считалось престижным
(выгодным) породниться с евреями, так как многие из них, если не большинство,
работали в сфере торговли --- «царице» благ того времени. Да и во власти, а
именно в органах до 50-х годов, они преобладали. 

Хоть у них и была своя культура со своей спецификой и клановостью, но в целом
это был, бесспорно, очень работоспособный, талантливый, жизнерадостный и
веселый народ, несмотря на свою печальную историю. Искренне жаль, что сейчас их
осталось буквально несколько человек: один аккордеонист, два политика да трое
олигархов. Ведь было время, что их обвиняли во всех бедах, что они даже, мол,
«выпили всю воду», а что оказалось на самом деле, то уже словами Евангелия: «Не
ищи соринку в чужом глазу, не видя бревна в своём».  На набережной Днепра,
напротив городского пляжа, летом располагалась целая колония
рыболовов-«босяков», «блатных» со знаменитого Подола. Это была тоже часть целой
эпохи 60-х в истории города. Начиная с конца мая и до августа на набережной они
с десятков самодельных деревянных вымостков круглосуточно ловили всевозможную
рыбу: язей, лящей, судаков, сазанов, сомов --- от 1 кг и даже до 20 кг штука! В
конце дня младшие из них выходили с уловами на дорогу вдоль набережной и
продавали старым киевлянам эту прекрасную, свежайшую рыбу по 1 рублю за 1 кг.
Эти деньги, и довольно в сумме немалые, старшие «переводили» в ящики, сумки с
крепким вином. Что было после? Что-то вроде «Вестсайдской истории» с блатными
песнями под гитару, где были любовь, дружба и почти шекспировские страсти, где
вместе с ними были и верные и всякие подружки, девушки --- все дети, в основном,
бедных, простых рабочих родителей с Подола. Было там и много судимых,
криминальных, но это было место, где осуществлялись их мечты, образ жизни.
Наверное, это были первые в мире хиппи! 

Надо отметить, что послевоенная безотцовщина в очень многих семьях рождала как
примеры талантливых и успешных в жизни, так и примеры дна.

Тогда еще были в ходу старые выражения типа «будьте добры», «ради бога
простите», «любезный», «неприлично», «неудобно», «Вы», «товарищ», «гражданин».
Это потом пошли «эй, мужчина», «ты», «козел», «фрайер», «машка» и прочие перлы
уголовного жаргона. Тогда же и началось падение городской культуры. «Матерщина»
становилась языком общения. Стремительно увеличивалось самогоноварение и
вследствие этого спаивание рабочей молодежи, началось повальное пьянство,
становившееся везде и на работе все более «уважаемым», чего никогда не было
раньше, где пьяниц знали одного-двух на всю улицу или двор. 

Еще памятны грандиозные тогда драки между районами. Например, была такая драка
за первенство между Чоколовкой и Шулявкой в 1959 году. В ней участвовали
десятки отчаянных парней с кастетами, ножами типа «финка», солдатскими ремнями
с пряжкой. Последствия таких драк часто были с нарядами милиции, с «посадкой»
некоторых особо агрессивных участников. Вообще, еще до конца 50-х дворы
враждовали с дворами, улицы с улицами, соперничали их «короли» со всевозможными
кличками типа «Серый», «Косой», «Лёва»... Для того времени были характерны
отчаянные парни с прическами полубокс и брюками «клеш», как правило, рослые,
симпатичные, мускулистые. Как называли их тогда старые киевляне --- «отпетые».
Это типично отражено в знаменитом американском фильме 50-х «Вестсайдская
история», который у нас пустили лишь в 80-е годы. 

В те времена провести девушку с чужого двора или улицы к ее дому было довольно
рискованным поступком. Молодые люди шли вместе, не касаясь рук, с гордо
поднятой головой, с еле скрываемым волнением, как бы бросая всем вызов. В те
времена девушки, грубо говоря, «не давали» без обязательств со стороны юношей и
его родителей, вплоть до ЗАГСа. Отношения между парнями и девушками оставались
еще очень строгими. Поцелуй был тогда событием, а близкие отношения или термин
«они живут» были как ЧП районного масштаба! Хотя в школах уже давно перешли на
совместное обучение девочек и мальчиков, но почти солдатская муштра школьников
со стороны «шкрабов» (школьных работников) держала всех в жесткой узде, строго
регламентированных отношениях. 

Как такового полового воспитания не проводилось, все было пущено на
традиционный самотек, а для родителей было как табу. Проявлялось это в школе,
во дворах, при играх, в записках избраннице, ношении ее портфеля домой, во
взглядах украдкой, в повышенном отношении и внимании --- и ничего больше! Что
бывало больше, было крайне редко и целым событием. Из-за этого иногда доходило
даже до смешного, потому что некоторые девушки считали, что от поцелуев могут
быть дети! О так называемой сексуальности никто понятия не имел, все было
положено на интуицию и опыт каких-то старших. Хотя и в Европе, особенно в
Италии, тогда тоже не далеко ушли (особенно в католических странах), и даже
похуже (исключая феномен Швеции). От такого положения вещей было и немало драм,
да и криминальных действий со стороны особо темпераментных юношей. Семья была
еще традиционно крепка, многодетна, со строгой ответственностью перед
обществом, школой. Школа была главным институтом контроля всей семьи, впрочем,
как и работа родителей. Даже разводы тогда еще публиковали в газете «Вечерний
Киев». Теперешняя же половая свобода дала другую крайность и последствия,
возможно, гораздо худшие по масштабам. 

Денежная реформа 1961 года заменила большие красивые ассигнации 1947 года на
небольшие бумажные, цветные, более удобные, а также монеты, но при этом
соотношение 1:10 ударило не только по торгашам и начинавшим тогда будущим
«цеховикам» и подпольным миллионерам, но и немного по простым людям. Так,
например, зелень на базарах в монетах стала в десять раз дороже. 

До конца 50-х самыми потребляемыми и дешевыми продуктами питания были:
хлебобулочные изделия (уже появился белый пшеничный хлеб), картофель в любых
видах, овощи, фрукты, всевозможная рыба и соленья, подсолнечное масло. Оттуда
же «cоветский бульон» --- кипяченая вода с покрошенным черным хлебом, луком и
уксусом, а также «советский шоколад» --- халва, и, конечно, уже вечные жареные
семечки. Только в 60-е народ начинал питаться более разнообразно и качественно. 

Еще до конца 50-х в некоторых многодетных семьях без отцов присутствовала
ужасающая нищета, где одинокие матери бились за выживание, борясь за изнуряющую
стирку белья соседям или продавая искусственные цветы для кладбища. Особенно
это было заметно в многодетных еврейских семьях, где один отец работал то ли
керосинщиком, то ли старьевщиком, то ли «гицелем» (ловцом кошек и собак на
мыло), едва зарабатывая всем на еду. Их дети не вылазили из кожных болезней, с
головами, покрытыми лишаями, все в «зеленке», в йоде. Если кто во двор выходил
с хлебом, намазанным маслом с посыпанным сверху сахаром или даже со смальцем,
то от «дай откусить» не было отбоя от таких вечно голодных детей. Никто из них
тогда не знал ни шоколада, ни колбас, да и простое мороженое было по
праздникам. Деликатесами для них были конфеты «горошек» или «гематоген» в
аптеках по 7 копеек. Только на официальные праздники такие дети могли получить
подарок с простыми конфетами и печеньем, если кто-то из родителей работал на
фабрике или заводе. Конечно, номенклатурные дети получали совсем другие по
качеству подарки. Тогда же появилась и школьная форма, красивого серого цвета с
красивой фуражкой со школьной кокардой, и опять же носили ее их же дети. 

При всей определенной справедливости советской власти, все же эта осточертевшая
всем и до сих пор существующая номенклатура всегда жила очень благополучно. Как
и то, что фронтовики с рядом медалей на груди, без ног, на досках с
подшипниками и с чушками в руках для отталкивания, еще долго после войны, уже
спившиеся, просили милостыню на базарах и возле церквей: «Подай сестричка,
Христа ради!», а их новоиспеченные командиры становились генералами, потому что
команда «Вперед!» часто не считалась с никакими самыми неоправданными жертвами,
за какую-нибудь никому не нужную высоту.

Еще до конца 50-х до Великого переселения всех из подвалов и бараков в новые
дома --- «хрущевки» --- с их газовой революцией, люди массово готовили пищу на
керосиновых примусах. Уголь и дрова еще были во множестве старых домов
Соломенки, Шулявки, Демеевки. Был тяжелый быт, с водой в колонке на булыжной
улице, с ведром на ночь в темном коридорчике, с массовым воскресным походом в
баню с семьями, где уже витал в буфете аппетитный запах свежего пива для отцов.
А эти почти безуспешные «войны» с клопами? Но наши люди жили все в одной
комнате, любили (часто через ширму), рождали детей и смеялись, несмотря ни на
что! 

Тогда же еще очень был высок уровень преступности, в основном грабеж, так
называемый «гоп-стоп» --- этот бич для людей, когда все гуляли допоздна (из-за
тесных коммуналок) и попадали в темных местах на раздевание или снятие наручных
часов. Потом пошло срывание «пыжиковых» шапок с головы некоторых прохожих (как
серьги с мочкой уха сразу после войны). В 1966 году вышел знаменитый Указ по
усилению борьбы с хулиганством. Сколько он пересажал, то и не счесть! Под эту
гребенку попадали и многие подростки за совершенно незначительные проступки
типа «пошел на...!», но сразу на пару лет в места не столь отдаленные, откуда
выходили многие уже совсем другими. Вообще, некоторые очень руководящие
коммунисты вкупе с «доблестными» органами кгб (НКВД) наделали нашему народу
неисчислимое количество бед!  До конца 50-х во многих киевских семьях с
достатком были домработницы. Это были девчата, тетки из забитых сел, работающие
за кров и еду, частенько спавшие на сундуках в коридоре, кладовках, которых с
богом отпускали их измученные родители-селяне, жившие в убогих хатах под
соломенной стрихой, с неровным глиняным полом полуголодного послевоенного села. 

Каждое воскресенье на Житнем и Сенном рынках был большой привоз
сельхозпродуктов. Цены были такие низкие, что некоторые селяне часто
возвращались домой буквально с пустыми руками. То ли сами обсчитывались по
своей малограмотности, то ли их дурили городские проходимцы. Но уже с начала
60-х (с массовым переселением селян в город) картина начала меняться. Тогда же
начались смешанные браки с городскими, часто из-за прописки, жилплощади
(прописка в Киеве до 80-х была очень строгая). Из-за этого часто происходило
немало семейных драм, вплоть до судов. 


Летом в 60-е город становился полупустым. Масса горожан уезжала в отпуск, кто
куда мог: одни в палатки и самодельные хижины из фанеры на берега Днепра и
Десны (на модную тогда йодную воду), другие в Крым (в пансионаты, санатории)
или для удешевления по две семьи в одну комнату по 1 руб. с человека, где часто
разругивались. Кстати, билет на самолет из Киева в Симферополь стоил 15 руб.
Дети разъезжались по пионерским лагерям в Ирпень, Бучу, Пущу-Водицу. Лето в те
годы было очень жарким, асфальт на улицах буквально плавился, и дети в
сандалиях это очень горячо ощущали. Тогда еще допоздна стоял гул от
многочисленных детских голосов и зовущих их из открытых окон рассерженных
родителей. 

Те, кто никуда не ехал, играли во дворах в «жмурки», в «соловьи-разбойники»,
«классики», футбол резиновым мячом, многие бегали на рыбалку, захватывающую и
плодотворную (рыбы в Днепре тогда было в изобилии). Часто бегали купаться на
разные водоемы: Совки, Пуща-Водица, озеро Выдубецкое, Матвеевский залив,
городской пляж, тогда же еще была переправа (до постройки Паркового моста в
1960 году, с набережной на Труханов остров) на судах типа баржи по 5 копеек с
человека. И при таких купаниях часто случались и беды с детьми на воде, с их
присущей отчаянностью и беспечностью. 

В середине июня перрон киевского вокзала всегда был заставлен десятками корзин
со свежей клубникой на Москву, благодаря этому уже зимой московские дети
ложками ели украдкой от родителей варенье, пока те не пришли с работы. Вообще
клубника, как и киевский торт, киевская котлета и горилка с перцем на, долгие
годы были знаменитыми на весь мир символами города. Как и слава Крещатика с его
неповторимой послевоенной архитектурой, о которой старые киевляне тогда шутя
говорили: «Для дома это слишком сладко, а для торта --- слишком высоко»! 

До сооружения в 1963 г. грандиозной Киевской ГЭС на Днепре выше Киева возле
древнего городка Вышгород каждый год в середине марта можно было с восхищением
наблюдать величественный, тысячелетний и уже теперь давно забытый ледоход с его
грохотом и треском огромных льдин над черной, глубокой, студеной водой великой
реки. А в конце мая, когда Днепр в весеннее половодье разливался на огромных
просторах левого берега, часто затапливались Осокорки, Русановские сады,
Центральный пляж с гардеробом, а тучи мошек вечерами заполняли близлежащие
улицы и площади, где прогуливающиеся горожане ветками свежей сирени или
черемухи не переставая отмахивались от них, мешающих кому выпить, а кому
целоваться.

Сооружение Киевской ГЭС, конечно, очень экономически необходимой, все же
кардинально нарушило тысячелетнюю экосистему Днепра с его тогда неистощимыми
запасами рыбы, которая сотни лет кормила на его берегах миллионы людей как в
городе, так и на всем протяжении реки. В те времена тысячи мальчишек и взрослых
наслаждались этим древним, страстным, благородным увлечением, да еще и просто
пропитанием своих семей!

В те годы многие горожане, в основном, на окраинах --- Подол, Шулявка, Сталинка,
Чоколовка --- страстно увлекались голубями, сооружением для них разнообразных
домиков на сваях --- голубятен. Каких только пород голубей тогда не было:
«турманы», «нежинские», «крюковские», «павлин», «трубач»... Это была большая,
можно сказать тоже эпоха страстной для всех возрастов охоты и увлечения!
Конечно, надо отметить еще одно повальное увлечение для всех возрастов и
«сословий» --- киевское мороженое, которое было в таком ассортименте и качестве,
что теперешним и не попробовать. Например, в кафе был пломбир, который
накладывали разноцветными шариками в мороженицах с ложечкой: шоколадный,
крем-брюле, сливочный, фруктовый, с шампанским, с газированной водой с сиропом,
фруктовой подливой, ягодной, шоколадной… 

На улицах с лотков, из больших алюминиевых, покрытых инеем цилиндров, мороженое
накладывалось в вафельные стаканчики. Также с лотков продавались в пачках:
сливочное с «лебедем» (13 коп.), молочное (9 коп.), фруктовое (7 коп.),
шоколадное «эскимо» малое (11 коп.), большое (22 коп.), «ленинградское»
шоколадное (22 коп.), в вафельном стаканчике сливочное с кремовой розочкой (28
коп.) и торт-мороженое с кремом. 

В 1965 году произошло огромное событие для народа --- субботу сделали выходным
днем, и теперь у людей стало два выходных дня в неделю! Это был еще один глоток
жизни для измотанного десятилетиями народа. Начался значительный подъем уровня
жизни, который вдохнул еще в 1956 году Никита Хрущев.

Люди стали лучше одеваться, «стиляги» начали отходить в прошлое. «Стиляги» –
газетный термин 50-х гг., когда молодые люди, в основном из обеспеченных семей,
одевались для того времени вызывающе: большой клетчатый пиджак, галстук с
пальмами, брюки «дудочкой», туфли на толстенной подошве, прическа «кок» в стиле
Прэсли, густо смазанная «бриолином», а также новая западная музыка рок-н-ролл,
а позже твист. Девушки были в ярких крепдешиновых платьях с широким поясом и
большой пряжкой, туфли «лодочкой», огромные бусы и серьги, высокая прическа
яркой блондинки в стиле Монро.

Если в 50-е образцами моды были дома моделей Риги, Москвы и Ленинграда, и
портные всей страны одевали часть людей в это модное, то в 60-х начала
появляться мода на все американское, и конкретно --- джинсы. Еще появилась
повальная мода на плащ «болонья» --- синтетический, коричневого цвета, а также на
белые нейлоновые рубашки, с узким черным галстуком, и черные туфли с
обрубленным или очень узким длинным носом. Влияние группы «Битлз» было даже на
модель пиджака (без воротника) и прически, густо зачесанные до бровей.

Хрущевская оттепель повлияла и на советскую моду, которая начала стремительно
уступать Западу. Подходил бум мини-юбок и брюк «Бэлз». Так появился термин:
«Тлетворное влияние Запада!» А кто виноват? Зачем годами все фабрики страны
шили один и тот же фасон костюмов --- широченный пиджак и брюки «мешком»?! Их
сразу после покупки почти все перешивали у модных тогда портных ателье
индивидуального пошива, да еще в очереди, где деньги мимо кассы, где и у
заведующих ателье, как и у всех тогда завмагов, была своя, немалая доля.

С того же периода началась бурная деятельность «фарцовщиков» --- нелегальных
торговцев западными «шмотками» (потом их преследовали нещадно и за валютные
дела). Позже подобные вещи «лепили под фирму» и поляки, и венгры из братского
соцлагеря (а в 70-е --- уже и в Грузии). «Фарцовщики» скупали вещи у иностранцев
и перепродавали населению, эффективно пользуясь нашим вечно тогда дефицитом.
Очень популярна была «толкучка» в Беличах под Киевом.

В конце 50-х советская парфюмерия была представлена в основном в нескольких
видах: «Красная Москва», «Шипр», «Сирень», «Ландыш» и «Тройной одеколон»,
который позже некоторые успешно пили. О дезодорантах тогда никто понятия не
имел (вплоть до 80-х). 

Горожане, парни и девушки, почти все были худые, но очень энергичные и
жизнерадостные, очень многие занимались спортом. О депрессии тоже никто понятия
не имел, не то, что сейчас, как процветающие Финляндия и Дания занимают в этом
первенство (еще один парадокс человеческого бытия). Большая часть тогдашней
молодежи одержимо училась, увлекалась спортом, активно участвовала во
всевозможных соревнованиях, в художественной самодеятельности. Другая часть --- в
выпивках и компаниях, и некоторая --- в криминале.

Вообще роль криминалитета, «блатного мира», была огромной с давних пор, еще с
царского времени. Эта своего рода субкультура особого образа жизни со своим
языком и манерой поведения. Корни были всегда одни: неблагополучные бедные
семьи, в которых дети почти никогда не могли выбиться в люди и выбирали
воровскую или грабительскую стезю. Эта среда настолько была влиятельна, со
своим ореолом романтики, что сбивала с пути и очень многих благополучных
молодых людей. На ее счету миллионы покалеченных судеб. Ее остатки в теперешнем
жаргоне («слэнге») и в уже современной, в общем, более-менее благополучной
молодежи, не подозревающей, какие черные значения это носит. 

В те далекие 50-е бывали еще солидные мужчины в светлых (и парусиновых)
костюмах, в шляпах, иногда с тростью с резной ручкой, куривших хорошие сигареты
типа «Дукат» с красивых мундштуков, в серебряных и даже золотых портсигарах.
Они были степенны, учтивы, с четкой речью уцелевших потомков дворянских и
военных сословий, возможно, и из мещан. Вообще, суетливость, многословие,
шумность, не говоря о сквернословии, были тогда у них признаками низкого
происхождения, даже слово такое было --- «низкий», «низость», и просто --- хам,
хамло.

В 60-е на Крещатике, в Пассаже был большой магазин кавказских «Минеральных
вод»; «Боржоми» и «Нарзан» наливали прямо с крана больших емкостей, постоянно
пополняемых натуральной свежей минеральной водой с Кавказа. Еще была реклама на
Крещатике, всего на нескольких высотных домах, типа: «Ешьте треску», «Пейте
сухое вино», «Храните деньги в сберегательной кассе» и «Летайте самолетами
Аэрофлота».

На центральном Бессарабском рынке всегда торговала самая привилегированная
часть советского народа --- грузины. Они торговали в основном цитрусовыми и
хурмой, чередуя друг друга походами с киевскими женщинами, девушками в
рестораны и гостиницы: «Кавказ», «Столичный», «Динамо», «Украина». Оттуда и
родилось выражение: «Кто девушку ужинает, тот ее и танцует!».

В 60-е годы грандиозные успехи страны в космосе, в спорте, в военном деле,
автомобилестроении вдохновляли молодежь на выбор профессии в этих почетных и
приоритетных направлениях. В киевских военных училищах (типа КВИРТУ, КВИАУ), в
известных вузах были огромные конкурсы. Курсы шоферов и телемастеров были
переполнены. Девушки мечтали быть врачами, учителями, геологами и даже
крановщицами! 

А какая слава была у киевского «Динамо»! Лобановский, Серебрянников, Паркуян,
Турянчик, Биба, Бышевец, Хмельницкий… А сейчас? Пришел культ денег --- и тут же
пропал дух команды! Может банально, но факт: все можно купить, кроме главных
ценностей, но некоторые разумеют их на свой лад. Как и то, что в одних странах
людей объединяют государственные деятели, а у нас, в Украине, объединяют людей
клоуны, юмористы типа Сердючки, «Кроликов», «Золотого Гуся»; в России –
Петросян, Гальцев и «Русские бабки»! Смешно все? И только ли это... Но время
идет. 

Прощай же тот Киев, которого уже больше нет!"
