% vim: keymap=russian-jcukenwin
%%beginhead 
 
%%file 18_07_2020.fb.lnr.12
%%parent 18_07_2020
 
%%endhead 

\subsection{Против Киева начали расследование о незакрытии неба в зоне крушения MH17}
\label{sec:18_07_2020.fb.lnr.12}
\url{https://www.facebook.com/groups/LNRGUMO/permalink/2855601207884768/}
  
\vspace{0.5cm}
{\small\LaTeX~section: \verb|18_07_2020.fb.lnr.12| project: \verb|letopis| rootid: \verb|p_saintrussia|}
\vspace{0.5cm}

\index{MH17}

Фонд безопасности полетов выяснит, почему Украина не закрыла небо над зоной
крушения малайзийского Boeing, сообщили РИА Новости в МИД Нидерландов.

В июне голландский адвокат Баудевейн ван Эйк, представляющий интересы
обвиняемого россиянина Олега Пулатова, сообщил, что Гаага согласилась изучить
обстоятельства, по которым Киев не закрыл воздушное пространство для полетов
гражданской авиации в зоне конфликта на востоке Украины. По его словам, такое
расследование может занять до шести месяцев. Пресс-секретарь прокуратуры
Нидерландов Брехтье ван де Мосдейк пояснила РИА Новости, что оно ведется
отдельно от уголовного процесса по авиакатастрофе.

Фонд безопасности полетов (Flight Safety Foundation) --- независимая
некоммерческая международная организация, занимающаяся исследованиями,
образованием, пропагандой и коммуникациями в области безопасности полетов. Она
базируется в США.

Российская сторона неоднократно поднимала вопрос об ответственности Киева за
незакрытие неба. Так, например, ранее в интервью РИА Новости директор Первого
европейского департамента МИД Алексей Парамонов отмечал, что Совместная
следственная группа (ССГ) не провела настоящего расследования роли Украины в
этой трагедии.
  
