% vim: keymap=russian-jcukenwin
%%beginhead 
 
%%file 06_12_2021. fb. fb_group. story_kiev_ua. 1. dorogi_kotoryje_my_vybirajem. cmt
%%parent 06_12_2021. fb. fb_group. story_kiev_ua. 1. dorogi_kotoryje_my_vybirajem
 
%%url 
 
%%author_id 
%%date 
 
%%tags 
%%title 
 
%%endhead 
\subsubsection{Коментарі}

\begin{itemize} % {
\iusr{Serge Ermakov}

В конце 80-х-начале 90-х, когда я был студентом химфака КГУ, диспозиция была
такая: проживание - Виноградарь, учились мы в основном на Владимирской, но
некоторые занятия были в районе ВДНХ. При этом сестра, у которой я часто
проводил выходные, жила на Дарнице, а институт, в котором я делал диплом,
располагался на Академгородке. Что предполагало регулярные и длинные поездки по
всему городу. Поездки 11 и 12 троллейбусом хорошо знакомы, ну а 38 автобус
вообще соединял общежитие химфака с универом, так что его знали все.

Любимым маршрутом был трамвай, которым я ездил с Ленинградской площади до
Почтовой, кажется, 21 номер. Через мост Патона, по набережной Днепра, если я
ехал от сестры в воскресенье вечером, народу было немного, можно было сесть, и,
глядя в окно, отпустить на волю поток мыслей и ассоциаций.

Самое жуткое - это было брать штурмом автобус на Святошино, чтобы ехать в
Академгородок, особенно простояв в его ожидании минут сорокв морозный день.
Впрочем, 102 автобус, соединявший Виноградарь с метро Оболонь (тогда - проспект
Корнейчука) был ненамного лучше.

\begin{itemize} % {
\iusr{Катя Бекренева}
\textbf{Serge Ermakov} Такое было и в 70-ые, я была ещё школьницей, тогда поликлиника и взрослая и детская были в Академгородке, иногда пользовалась маршрутным такси, чтобы не давиться в автобусе, тогда это было 15 копеек), а домой пешком, 25 минут быстрым шагом и дома, возле метро Святошино

\iusr{Наташа Гресько}
\textbf{Serge Ermakov} 97-й и 40-й автобусы. и "рафики" по 15 коп. но мы со "Святохи" в Академ - домой - чаще пешком ходили. помню, химфак был рядом с нами (с красным) - а сейчас на выставку переехали, как и биологи?

\begin{itemize} % {
\iusr{Serge Ermakov}

\textbf{Наташа Гресько} Я тоже часто ходил с Академа до Святошино пешком, ибо ждать автобус не было сил, а далеких пеших прогулок я никогда не боялся.
На химфаке не был с 2012, но насколько знаю, расположение не поменялось.

\iusr{Наташа Гресько}
\textbf{Serge Ermakov} а, на выставку - на военку, наверное? мы тоже туда раз в неделю катались (мне как раз по окружной было удобно - но долго! @igg{fbicon.face.smiling.eyes.smiling} ).

\iusr{Serge Ermakov}

\textbf{Наташа Гресько} Военка само собой, физика и физкультура.

\iusr{Татьяна Гордиенко}
\textbf{Наташа Гресько} , химфак на месте
\end{itemize} % }

\iusr{Татьяна Гордиенко}
\textbf{Serge Ermakov} , 38-й с дядей Васей - наше всё))

\iusr{Наташа Гресько}
\textbf{Татьяна Гордиенко} дядю Васю не помню), а 38-й автобус забыть нельзя))

\iusr{Татьяна Гордиенко}
\textbf{Наташа Гресько} , почил маршрут

\iusr{Клим Форманчук}
\textbf{Serge Ermakov} , 102 и сейчас на том же маршруте.

\end{itemize} % }

\iusr{Андрей Масловский}

Сейчас на общественном транспорте с ВДНХ на Лукьяновку только один вариант ) На
12 троллейбусе до вокзала, а потом на 15 или 18 трамвае до Лукьяновки )

\begin{itemize} % {
\iusr{Светлана Манилова}
Андрей, можно на метро. @igg{fbicon.smile} 

\iusr{Андрей Масловский}
\textbf{Светлана Манилова} можно ) но хотелось на тралике )

\iusr{Светлана Манилова}
Андрей, полностью поддерживаю!

\iusr{Елена Махова}
Есть 19 троллейбус на Лукьяновку, идет по Воздухофлотскому проспекту с пл Космонавтов, с Южного вокзала можно пройти до остановки.
\end{itemize} % }

\iusr{Alex Kunakh}
- Тролейбус відправляється у рейс за маршрутом - Виставка досягнень народного господарства - площа Ленінського комсомолу. Тривалість поїздки 39 хвилин. Обережно, двері зачиняються, наступна зупинка Магазин-салон Прилади....
Якось так)

\begin{itemize} % {
\iusr{Светлана Манилова}
\textbf{Alex}, у Вас потрясающая память! Спасибо!

\iusr{Alex Kunakh}
\textbf{Светлана Манилова} ))

\iusr{Светлана Манилова}
\textbf{Alex}, тоже учились в универе? @igg{fbicon.smile} 

\iusr{Alex Kunakh}
\textbf{Светлана Манилова} так, на біологічному.
Але я і виріс поруч, тому 11-й був один із варіантів проїзду до центру. З глибокого дитинства. Тому і запам'яталося)

\iusr{Otto Vitosch}
\textbf{Alex Kunakh} Магазин книги "Ювілейний"

\iusr{Tatyana Mazerati}
ona hotela bi jit na 5y avenue, no nastupna zupinka Darniza...

\iusr{Светлана Манилова}
\textbf{Tatyana}, если Вы обо мне, то не хотела бы... @igg{fbicon.smile} 
\end{itemize} % }

\iusr{Serge Ermakov}

В 2019 году я случайно обнаружил, что 575 маршрутка идет от парка Шевченко на
Виноградарь. Поскольку рабочий день у меня начинается и заканчивается рано,
время от времени теплым летним вечером я мог не спеша спуститься с Лукьяновки
по Дмитриевской, подняться по бульвару Шевченко до парка, а потом неспешно
ехать домой, рассеянно глядя в окно...

Увы, уже второй год, как поездки в транспорте, что долгие, что короткие,
потеряли всякую привлекательность...


\iusr{Татьяна Гурьева}

Как знакомо, мы все так ездили порой через весь город, это было нормально.
Сейчас все изменилось до неузнаваемости, как и количество людей, машин.


\iusr{Анна Загорулько}

Знакомые маршруты! Филологи учились в красном и желтом корпусе на Владимирской,
но физкультура и военка - на ВДНХ. А мне очень нравилось ходить на плавание в
бассейн.

\iusr{Светлана Маринич}

Светлана, с удовольствием прочитала Ваш пост. Вспомнила свою учебу. Правда у
нас в группе было всего три девочки ( из 25). Жизнь тоже разбросала по миру. А
дороги, которые мы выбираем- не знаю, почему тогда выдбрала такую профессию, но
ни разу не пожалела. Спасибо за Ваши воспоминания.

\begin{itemize} % {
\iusr{Светлана Манилова}
Светлана, спасибо Вам за отзыв! Какую дорогу выбрали, если не секрет? @igg{fbicon.smile} 

\iusr{Ирина Архипович}
\textbf{Светлана Манилова} 

Очень приятные воспоминания!  @igg{fbicon.hand.ok} Тоже часто бывала в том
районе!!  @igg{fbicon.heart.eyes}  @igg{fbicon.hearts.two} 

\iusr{Светлана Маринич}
\textbf{Светлана Манилова} Проектирование и строительство электростанций

\ifcmt
  ig https://i2.paste.pics/6aaae5600b03cbeb687bb41538e17999.png
  @width 0.4
\fi

\iusr{Светлана Манилова}

Светлана, мой папа инженер- электрик по специальности. Окончил КПИ,
электрические станции и системы, если не ошибаюсь. Всю жизнь занимался
проектной работой. @igg{fbicon.smile} 


\iusr{Светлана Маринич}
\textbf{Светлана Манилова} Как приятно узнать о коллеге. КПИ, теплофак. В каком проектном институте работал Ваш папа?

\iusr{Светлана Манилова}
Светлана, "Тяжпромэлектропроект", ВНИИПКнефтехим.

\iusr{Tatyana Mazerati}
\textbf{Светлана Маринич} Moi 4 druga zakonchili teploenergeticheskiy.

\iusr{Светлана Маринич}
\textbf{Светлана Манилова} Как приятно услышать о коллеге. Поскольку мир тесен, могут оказаться и общие знакомые. у

\end{itemize} % }

\iusr{Клим Форманчук}

К сожалению "... проделать путь на городском транспорте" у Вас может и
получится, вот только название остановок повергнет в шок. Вам будет казаться,
что едете вы не по Киеву, а по незнакомому, больше галичанскому райцентру, т. к.
название улиц так изменено, что многие горожане прожившие всю жизнь в своём
Городе совершенно не ориентируются как сейчас добраться на ту или иную ранее
знакомую улицу. Мне горько от того, что сделали и делают с моим любимым Киевом
...

\begin{itemize} % {
\iusr{Люала Пилипейко}
\textbf{Клим Форманчук} 

согласна, не знаю названия ни одной улицы, столько причем имена тех чьм именем
они названы тожене знаю. Живешь, как в чужом городе.

\end{itemize} % }

\iusr{Tatyana Mazerati}

SVETLANA, UDIVITELNO, CHTO NA MEH-MATE BILI V OSNOVNOM DEVOCHKI. OBICHNO
NAOBOROT!

\begin{itemize} % {
\iusr{Светлана Манилова}
\textbf{Tatyana}, почему? @igg{fbicon.smile} Это не технический ВУЗ.

\begin{itemize} % {
\iusr{Tatyana Mazerati}
\textbf{Светлана Манилова} 

Matematika vsegda bila slojnoy dlya devushek-tak iz moego opita. Ya ochen
lyubila matematiku. Bil v Kieve takoy znamenitiy uchitel-Yakov Ayzenshtat-mnogo
let nazad on prepodavalv 145 shkole. Ya zanimalas s nim dopolnitelno. Ot
uchenikov ne bilo otboya, prinimal 3 srazu.

\iusr{Светлана Манилова}
\textbf{Tatyana}, Яков Иосифович- мой репетитор. Я ему очень благодарна. @igg{fbicon.smile} 

\iusr{Светлана Манилова}
\textbf{Tatyana}, в мое время он преподавал в другой школе, по-моему, 171-й.

\iusr{Tatyana Mazerati}
\textbf{Светлана Манилова} 

eto je nado takoe! On togda jil na Piterskoy, CHOKOLOVKA, ETO BILI NEZAPAMYATNIE
VREMENA. Togda on, ego jena Balya i dochka ne pomnyu kak zvali jili v odnoy
komnate

\iusr{Tatyana Mazerati}

tut je i ucheniki. Potom oni perehali na Pechersk v perviy kievskiy
kooperativ, predsedatelem kotorogo bil otez moey podrugi Nikolay Matveevich
Polyakov, no eto uje sovsem drugaya istoriya...

\iusr{Светлана Манилова}
\textbf{Tatyana}, я ездила к нему на Кловский спуск.

\iusr{Tatyana Mazerati}
\textbf{Светлана Манилова} Tochno 171 shkola, ya oshiblas-ne mudreno-cherez 50 let i 2 emigrazii!

\iusr{Светлана Манилова}
\textbf{Tatyana}, его дочь тоже училась на мехмате.

\iusr{Tatyana Mazerati}
\textbf{Светлана Манилова} v noviy dom-ponyatno-a chto s nim bilo dalshe,on viehal ili nostalsya v Kieve?

\iusr{Светлана Манилова}
\textbf{Tatyana}, к сожалению, не знаю. Пыталась здесь, в группе, выяснить, но безрезультатно...

\iusr{Tatyana Mazerati}
\textbf{Светлана Манилова} 

po vidimomu dlya nee sdelali isklyuchenie s 5y grafoy i blizko ne podpuskali k
kievskomu universitetu-eshe na vecherniy so skripom kak to. Ya daje ne pitalas


\iusr{Светлана Манилова}
\textbf{Tatyana}, я сразу шла на вечерний.  @igg{fbicon.smile} 

\iusr{Tatyana Mazerati}
\textbf{Светлана Манилова} Jal, nezauryadniy bil chelovek, unikalniy uchitel i s chuvstvom yumora

\iusr{Светлана Манилова}
\textbf{Tatyana}, самые лучшие воспоминания. У меня до сих пор сохранились тетради (алгебра, геометрия, стереометрия).
 · Ответить · Поделиться · 19 ч.
Tatyana Mazerati
Светлана Манилова nu u menya konechno tetradey net posle vseh stranstviy,a vospominaniya ostalis.Mne povezlo-i v shkole-71ya i dopolnitelno bili potryasayushie uchitelya-moya mama pozabotilas ob etom -samoe glavnoe -ucheba.Medal poluchila i s ney v Kazan matushku na injener matematik.Pravda vsego na god-potom mama perevela v narodnogo hozyaystva na fakultet OMOEI-SAMIY MODNIY.Zvonili po moemu povodu iz ZK partii.Kogda ya sama prishla perevoditsya-skazali-takie nam ne nujni.Posle zvonka-kak raz takie nam nujni! Zapomnila na vsyu jizn!
 · Ответить · Поделиться · 19 ч.
Tatyana Mazerati
Светлана Манилова Tak vi toje iz nashego "franzuzskogo" kluba?
 · Ответить · Поделиться · 17 ч.
Tatyana Mazerati
a kto eshe na vechernem i hodit k Yashke-tak mi ego nazivali doma..
\end{itemize} % }

\end{itemize} % }


\end{itemize} % }

