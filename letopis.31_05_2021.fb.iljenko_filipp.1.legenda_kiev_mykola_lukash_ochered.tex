% vim: keymap=russian-jcukenwin
%%beginhead 
 
%%file 31_05_2021.fb.iljenko_filipp.1.legenda_kiev_mykola_lukash_ochered
%%parent 31_05_2021
 
%%url https://www.facebook.com/philip.illienko/posts/4247905641926424
 
%%author Ильенко, Филипп
%%author_id iljenko_filipp
%%author_url 
 
%%tags 
%%title Микола Лукаш - Київ - Черга - Дівчата - СРСР - Легенда
 
%%endhead 
 
\subsection{Микола Лукаш - Київ - Черга - Дівчата - СРСР - Легенда}
\label{sec:31_05_2021.fb.iljenko_filipp.1.legenda_kiev_mykola_lukash_ochered}
\Purl{https://www.facebook.com/philip.illienko/posts/4247905641926424}
\ifcmt
 author_begin
   author_id iljenko_filipp
 author_end
\fi

Більшості моїх фб-друзів не треба пояснювати, хто такий Микола Лукаш -
український перекладач, мовознавець і поліглот. Хто не знає – Вікіпедія й Гугл
вам в поміч. Але він відомий не лише своїми видатними перекладами. Завдяки
своїй вдачі дивакуватого інтелектуала він ще за життя став героєм безлічі
київських міських легенд. Я їх чув від свого батька і сьогодні мені пригадався
один такий апокриф.

\ifcmt
tab_begin cols=2

  pic https://scontent-bos3-1.xx.fbcdn.net/v/t1.6435-9/194951395_4247896401927348_2354929118782505918_n.jpg?_nc_cat=101&ccb=1-3&_nc_sid=730e14&_nc_ohc=wR2xtKL4wsYAX-3e4g8&_nc_ht=scontent-bos3-1.xx&oh=5ee3075ee9741ff4313f1800914260b0&oe=60DE06BB

  pic https://scontent-bos3-1.xx.fbcdn.net/v/t1.6435-9/195608820_4265756883447947_4803776771263615838_n.jpg?_nc_cat=111&ccb=1-3&_nc_sid=dbeb18&_nc_ohc=Ce7BqK45czUAX-kfxrd&_nc_ht=scontent-bos3-1.xx&oh=141b09257822ed61dad07cc63b4bcfc4&oe=60DE013E
  width 0.45

tab_end
\fi

Якось Микола Лукаш стояв у черзі за сметаною у магазині \enquote{Молоко} на перетині
вулиць Леніна та Франка. Це були 1980-ті роки, тому черги були довгими. Це я
добре пригадую, сам часто в тому магазині бував. І ось коли Микола Лукаш, який
чекав вже тривалий час, наблизився нарешті до прилавку і хотів вже звернутися
до продавщиці, прямо перед ним у чергу грубо вклинилися дві молоді товаришки,
одягнені та намальовані згідно останнього писку моди пізньорадянського періоду. 

Отямившись від такого втручання, Микола Лукаш спробував чемно пожурити цих
громадянок: \enquote{Як же так, такі красиві дівчата, і без черги?..} На це
одна з \enquote{дівчат} розвернулася до нього з класичною реплікою \enquote{А
можна на чєлавєческом язике?}

У тиші, що запанувала навколо, Микола Лукаш таким же люб'язним голосом
відповів: \enquote{Звичайно можна. ВЫ, ЧТО, БЛ@ДИ, СОВСЕМ ОХ@ЕЛИ?}

Так от, пригадався мені цей апокриф після ознайомлення зі змістом пояснювальної
записки нещодавно опублікованого законопроєкту під номером 5554, в якій
стверджується, що обов'язкове використання української мови при демонстрації на
українському телебаченні російськомовних серіалів \enquote{призведе до руйнування
кінематографічної та телевізійної галузі від чого, найбільших втрат зазнає
український глядач.}

В мене на сьогодні все. Добраніч.

\subsubsection{Комментарии}

\begin{itemize}
\iusr{Nataliya Zubar}

І магазин пам'ятаю, і переказ пам'ятаю. Не можу тільки згадати, хто розповів...
але приблизно в ті роки і розповіли. Відносно недалеко подруга родини
мешкала....

\iusr{Taras Tomenko}

А хто автор чергового циркуляру?

\iusr{Tatiana Nekriach}

Одна з моїх найулюбленіших оповідок про Миколу Олексійовича. У варіанті, який я
чула було \enquote{Почему без очереди лезете, бл*ди?} Що, власне, не становить
принципової різниці. Дуже шкодую, що не можу наводити цей приклад, пояснюючи
магістрантам, що таке контекстуальний переклад!:)

\iusr{Андрій Мохник}

Я теж давно чув цю історію, але там було: \enquote{Звичайно. КУДА ВИ БЛ@ДІ ПРЬОТЄСЬ?} Після чого їх як вітром здуло)

\iusr{Олег Медведчук}

Якже ж влучно підкреслено про чужу мову

Від себе додам, для тих, хто дуже дбає про «глядача»:
Не чипайте українську мову, інакше винесуть з кабінетів «кХу..ям»

\iusr{Petro Krasilnikov}

не пригадаю, чи є ця пригода у книзі Жолдака про Лукаша

\iusr{Любомира Яджин}

Прекрасно! Колись читала це. Але з насолодою ще раз перечитала...

\iusr{Viachek Kryshtofovych Jr}

мені цю прекрасну історію розповіли буквально тиждень тому. дивина та й годі...

\iusr{Діяна Тяско}

Класно. Я не знала, хто такий Микола Лукаш, але цю легенду чула в Черкасах. Так
шо це вже надбання не лише Києва, а й всієї України

\iusr{Iryna Rypianchyn}

Оце історія)

\iusr{Микола Зубков}

Це вже давно - класика, як і \enquote{гівно південне}\smile

\iusr{Larisa Oliinyk}

\enquote{чєлавєчєский язык}, він такий, зрозумілий, \enquote{чєлавекам}. Микола Лукаш все пояснив.

\iusr{Vitali Malicky}

Цей переклад я присвячував пам'яті Миколі Лукаша.
\url{https://toloka.to/t43983}

\iusr{Diveyeff-Tserkovny Alexei}

могу рассказать свою историю. когда у меня, ученика 10 класса физмат лицея в
1992г разболелся зуб, меня направили в студенческую поликлинику кгу на
ломоносова. я описал проблему, разместился в кресле и открыл рот. надо мной
склонилась врач-дантист с включённой бор-машинкой и спросила: \enquote{чому ти
розмовляєшь цією клятою мововою?!} зуб после этой \enquote{ликарши} болел
месяца 2. и желание ходить к стоматологам пропало на несколько лет...

\iusr{Andriy Borodavko}
\textbf{Diveyeff-Tserkovny Alexei} Чудова розповідь!  Вона Вам, нацику, другий апендицит
тоді не вставила?))

\iusr{Ольга Дерень}
\textbf{Diveyeff-Tserkovny Alexei} а мені 16 річній дитині, лікар в КТЕІ в тому самому
92 році сказав: я би тєбя фашистку сваімі руками би стрєлял.  Тому дантист ваша
хороша була лікар на відміну від мого. Знаєте чому?

\iusr{Віталій Корніяка}
\enquote{буде там плач і скрегіт зубів} (Мт. 8:12)

\iusr{Маргарита Сидоренко Дарда}
\textbf{Diveyeff-Tserkovny Alexei}, погано полікувала раз не навчила Вас, що в
Україні потрібно українською говорити.  В росії жорсткіше, там, або примусять
говорити російською, або вб'ють, а ця ще й зуба лікувала.

\iusr{Natalia Churikova}
\textbf{Diveyeff-Tserkovny Alexei} і Ви ще з усіма зубами додому пішли? Та вона ж ангел!

\iusr{Наталія Іваницька}
Скабєєва вже готує вечірній випуск пропаганди по цій замальовці. Лікарці респект

\iusr{Diveyeff-Tserkovny Alexei}

читаю и лишний раз убеждаюсь в какой единой соборной стране я живу...

\iusr{Ольга Дерень}

\textbf{Diveyeff-Tserkovny Alexei} Дійсно, звідки візьметься єдина і соборна
країна, якщо більшість какаяразніца і плюс ненависники всього українського.

\iusr{Diveyeff-Tserkovny Alexei}

простите, но судя по комментариям ненависть - это как раз ваше кредо...

\iusr{Ольга Дерень}

\textbf{Алексей Дивеев-Церковный} ну так, наше кредо:)))Ми в себе дома терпимо
ваше хамство, і маємо вас за це любити. Махохісти прямо якісь:)

\iusr{Diveyeff-Tserkovny Alexei}

учите историю, мадам. где бережаны, а где киев. и кто у кого дома. у нас дома -
сковорода, гоголь, вертинский, булгаков, ахматова. а у вас точно не все дома)))

\iusr{Ольга Дерень}
\textbf{Diveyeff-Tserkovny Alexei} який вихований кавалер:)))))
Бережани це Богдан Лепкий, Маркіян Шашкевич, Едуард Кофлер, Андрій
Чайківський, Іван Франко, Едвард Ридз-Смі́гли, Василь Іванчук та Мирося
Гонгадзе, Олег Шупляк і багато інших.  Тому історію України сам повчи, а не
тільки історію совка.

\iusr{Diveyeff-Tserkovny Alexei}
с нижайшим поклоном бережанам, и уважением к его культуре и традициям. но мы
про киев. не нужно в киеве нью-бережаны разводить...

\iusr{Ольга Дерень}

\textbf{Алексей Дивеев-Церковный} а я про Бережани слова не сказала.Київ мій т
дім, я тут живу більшу частину життя, тут народилися та виросли мої діти.Тому
ніхто нічого тут не розводить, крім вас,хизуючись своєю некультурою
спілкування.

\iusr{Diveyeff-Tserkovny Alexei}

ну видимо людей из бережан вывезти можно, а бережаны из людей сложнее...

\iusr{Diveyeff-Tserkovny Alexei}

\textbf{Маргарита Сидоренко Дарда} потрібно - это для вас, привыкшим по
потрибныкам жить. и свою хуторянскую пропаганду меньше разводите. а включите не
youtube кубанский казачий хор, и послушайте солов'їну мову і спів до которой
вам как до киева рачки...  и кстати приведу еще пример. месяц назад на самом
ненавистном вам канале:

\url{https://youtu.be/LJj7uoKzd_g}

p.s. после эфира всех детей расстреляли, а зрителей вывезли в гулаг...

\iusr{Ольга Дерень}

\textbf{Diveyeff-Tserkovny Alexei} Бережани не найгірший варіант у порівняні з
вами,який хоч і вважає себе столичним піжоном, та явно родич цирульника
голохвастова:)

\iusr{Alex Ruban}

\textbf{Diveyeff-Tserkovny Alexei} дядько, чому ти дурень такий?

\iusr{Ольга Дерень}

\textbf{Вікторія Полозенко} ось такі в нас київські жаби:))))))

\iusr{Nataly Lasoryb}
\textbf{Алексей Дивеев-Церковный} дышите, дышите

\iusr{Людмила Полока}

\textbf{Ольга Дерень} Підпудрене, підкрашене, хамовите... одним словом - чужинець

\iusr{Volodymyr Khovkhun}

\textbf{Diveyeff-Tserkovny Alexei} Добродію, то ви отой Олексій Гладченко, що
народився в Конотопі? Блукав світами разом із своїми батьками, а потім, осівши
в Києві, почав соромитись свого походження. Досить поширене явище, на жаль.
Утім, не гідні ви ні Києва, ні славного Конотопа. Перекотиполе...

\iusr{Маргарита Сидоренко Дарда}

Воно там трохи колись пострибало на українському телебаченні, а потім його
забули.  То треба ж про себе нагадати хоч якось, от воно й вищить.  Звичайний
представник «гонимого» народу без країни, роду та племені, що за гроші і в
церкві пердне.

\iusr{Віталій Корніяка}

Скупі дані з Вікіпедії повідомляють, що сам родом з Конотопа й батькове
прізвище Гладченко чомусь поміняв у 20 років на теперішнє. І чого б ото
цуратися свого славного походження, при цьому дорікаючи іншим \enquote{хуторянством},
Бережани протиставляти Києву (до речі, він як столиця належить усім українцям),
поводитися як жук у мурашнику, дражнитися й гудіти, показувати язьік, наражаючи
себе ж на захисну контратаку, коли сам таким самим мурашкою і є, тільки, бач,
замаскованим. Непереконливі приклади з Кубанню та українською піснею на рос.тб
(хор іще співає, а край кріпко поросійщений; українську пісню іноді можуть іще
в телевізор і навіть на Євробачення від Росії пустить, але тільки якщо
українське там або плаче, або нещасне, або дуркує, або виконано якось химерно,
як отими сумними дітьми-іноземцями про насправді щасливу дівочу ніч кохання, в
той час, як чи не єдиний на всю Росію умовний осередок української культури --
бібліотеку в Москві закрили, ще й бібліотекарку заарештували) та інші
маніпулятивні твердження ставлять під сумнів і розповідь про
дантистку-садистку. Жаль, що син свого батька так заблукав, але сподіваймося на
повернення. )

\iusr{Віталій Корніяка}

до речі, ось іще цитата з твору згаданого вище \enquote{корєннова (чи
коріннаґо?) кієвляніна}: «… а вот что делается кругом, в той настоящей Украине,
которая по величине больше Франции, в которой десятки миллионов людей, — этого
не знал никто. Не знали, ничего не знали, не только о местах отдаленных, но
даже — смешно сказать — о деревнях, расположенных в пятидесяти верстах от
самого Города. Не знали, но ненавидели всею душой».

\iusr{Наталія Іваницька}
Нехай зелений мотлох швидше розкриває свої плани. Позбудемося цього шапіто

\iusr{Liudmyla Pronenko}
\textbf{Наталія Іваницька} Вам ще замало?.... можна не встигнути позбутися.

\iusr{Наталія Іваницька}
\textbf{Liudmyla Pronenko} мені було досить ще до виборів. Але багатьом щось туман очі виїдав

\iusr{Liudmyla Pronenko}
\textbf{Наталія Іваницька} так

\end{itemize}
