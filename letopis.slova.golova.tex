% vim: keymap=russian-jcukenwin
%%beginhead 
 
%%file slova.golova
%%parent slova
 
%%url 
 
%%author_id 
%%date 
 
%%tags 
%%title 
 
%%endhead 
\chapter{Голова}
\label{sec:slova.golova}

%%%cit
%%%cit_head
%%%cit_pic
%%%cit_text
Лестница заскрипела, и Корнуэлл потянулся к рукоятке меча. Вихрь сена пронесся
мимо. Краем глаза он увидел, как Енот с выпущенными когтями прыгнул на
\emph{голову}, показавшуюся в отверстии сеновала. Енот опустился на эту голову,
и послышался приглушенный крик. Корнуэлл прыгнул к лестнице и начал быстро
спускаться. На полпути он уловил блеск вил, но увернулся. У основания лестницы
человек, поднимавшийся по ней, пытался освободиться от Енота, который вцепился
в голову и лицо своей жертвы. Свободной левой рукой Корнуэлл вырвал вилы у
этого человека.  Три кричащие фигуры устремились на него, и одна из них
выхватила меч. Корнуэлл отвел назад левую руку с вилами, а потом с силой бросил
их вперед. Держа перед собой меч, он побежал навстречу врагам. Щит все еще
висел у него за спиной, ему некогда было взять его в руки
%%%cit_comment
%%%cit_title
\citTitle{Зачарованное паломничество}, Клиффорд Саймак
%%%endcit

%%%cit
%%%cit_head
%%%cit_pic
%%%cit_text
Клеонові II не було діла до прецедентів. Скільки б їх не трапилося в минулому,
йому не ставало від цього ні крапельки легше. Не втішала його і думка про те,
що його прадід командував зграєю розбійників на якійсь третьосортній планеті, а
він, Клеон II, живе в розкішному палаці Амменетіка Великого і є спадкоємцем
давньої імператорської династії. Не радувало Клеона і те, що його батько
вилікував Імперію від міжусобиць і встановив в ній мир, подібний до того, яким
вона насолоджувалася при Станнеллю VI. Не приносив задоволення спокій у
віддалених провінціях, нічим не порушуваний ось уже двадцять п'ять років.
Імператор Галактики, Повелитель всього сущого застогнав і відкинув
\emph{голову} на подушки. Трохи заспокоївшись, він сів у ліжку і похмуро
втупився у дальню стіну спальні. Занадто велика кімната, в ній незатишно
одному. Інші кімнати теж занадто великі.  Втім, краще сидіти одному в спальні,
ніж дивитися на чепурних придворних, терпіти їх надто щедрі співчуття, слухати
порожню балаканину. Краще самотність, ніж суспільство неживих масок, під якими
ретельно прораховуються варіанти його смерті і успадкування престолу
%%%cit_comment
%%%cit_title
\citTitle{Фундація та Імперія}, Айзек Азімов
%%%endcit

%%%cit
%%%cit_head
%%%cit_pic
%%%cit_text
Я пішов до свого "Клубу Диких", але перед тим, як увійти туди, сперся на
поруччя Адельфі-террас і, замислившися, втупив погляд у брунатну, масну від
нафти річку. На вільному повітрі моя \emph{голова} завжди краще працює. Я витяг з
кишені список подвигів професора Челенджера і ще раз перечитав його при світлі
електричного ліхтаря. І раптом на мене зійшла ідея. Я чудово розумів, що мені,
як репортерові, не "пощастить знайти спільну мову з вередливим професором. До
того ж, наскільки я довідався з коротенької біографії, він був фанатик науки.
Може, в цьому слід пошукати його вразливого місця? Варто спробувати.  Я ввійшов
до клубу. Тільки-но вибило одинадцяту годину. У великій залі вже було майже
повно людей, хоч нові відвідувачі надходили й надходили. В кріслі проти каміна
я побачив високого, худорлявого чоловіка. Коли я підсунув крісло до нього, він
обернувся в мій бік. То був саме той, кого я найбільше хотів бачити — Тарп
Генрі з редакції журналу "Природа", він завжди радо допомагав усім своїм
знайомим. Не гаючи часу, я завів з ним розмову
%%%cit_comment
%%%cit_title
\citTitle{Утрачений світ}, Артур Конан Дойл
%%%endcit

%%%cit
%%%cit_head
%%%cit_pic
%%%cit_text
Дьявол извивался, пытаясь вырваться, однако я вцепился в него мертвой хваткой,
сомкнув руки на его брюхе. Шкура у него была вонючая. Мое лицо, поскольку я
поневоле прижался к ней, взмокло от его жирного пота. Он вырывался, ругался
последними словами, молотил по мне кулаками, но уголком глаза я все время видел
устремленное к нам острие копья. Стук копыт отдавался все громче, надвигался
все ближе, и вот копье с хлюпающим звуком вонзилось в цель. Дьявол упал. Я
выпустил его и тоже упал на тротуар, а нога так и застряла между прутьями.
Кое-как вывернув шею, я увидел, что копье ударило дьявола в плечо и пришпилило
его к ограде. Он извивался и подвывал, взмахивая руками. Изо рта у него текла
пена.  Дон Кихот попытался открыть забрало, но петли заело. Тогда рыцарь дернул
шлем так, что сорвал его с \emph{головы} целиком. Старческие пальцы не удержали
тяжесть, шлем вывалился из них и заклацал по тротуару.
— Мошенник! — вскричал Дон Кихот.  — Предлагаю тебе сдаться и дать
торжественный обет, что отныне и навсегда ты воздержишься от вмешательства в
дела людей... — Будь ты проклят на веки вечные, — отвечал дьявол.  — Уж кому
бы я сдался, так только не бестолковому добродею, который дня не может прожить,
не затеяв новый крестовый поход. Изо всего вашего племени ты несомненно самый
несносный. Ты способен учуять доброе дело за миллион световых лет и рвешься к
нему как одержимый. А я не хочу иметь с тобой ничего общего. Пойми ты наконец —
ничего общего!.. 
%%%cit_comment
%%%cit_title
\citTitle{Порождения Разума}, Клиффорд Саймак
%%%endcit
