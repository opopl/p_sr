% vim: keymap=russian-jcukenwin
%%beginhead 
 
%%file 14_04_2021.fb.butusov_jurij.1.vojna_7_let_donbass
%%parent 14_04_2021
 
%%url https://www.facebook.com/butusov.yuriy/posts/5564559943584256
 
%%author 
%%author_id 
%%author_url 
 
%%tags 
%%title 
 
%%endhead 

\subsection{Семь лет назад началась война на Донбассе. Напомним как это было}
\label{sec:14_04_2021.fb.butusov_jurij.1.vojna_7_let_donbass}
\Purl{https://www.facebook.com/butusov.yuriy/posts/5564559943584256}

\ifcmt
  pic https://scontent-bos3-1.xx.fbcdn.net/v/t1.6435-9/173292018_5564548150252102_5847781955081501238_n.jpg?_nc_cat=103&ccb=1-3&_nc_sid=730e14&_nc_ohc=eYCZl3RLrz8AX-APmmd&_nc_ht=scontent-bos3-1.xx&oh=e1d096812cfd9ae62b555695f4527b13&oe=609B876A
\fi

Семь лет назад началась война на Донбассе. Напомним как это было.
13 апреля утром на окраине Славянска произошла  встреча и совещание военнослужащих СБУ и ВСУ. Группа ВСУ: и.о. командующего Высокомобильных десантных войск полковника А.Швеца,  подразделение  80-й аэромобильной бригады под командованием комбрига полковника В. Копачинского в составе 3-й роты старшего лейтенанта В.Сухаревского. Группа СБУ: генералы В.Цыганок и В.Ягун, командир спецподразделения "Альфа" Г.Кузнецов, с которым было еще 5 сотрудников. Они встретились для обсуждения обстановки в захваченном Славянске.
К сожалению, встреча проводилась прямо на дороге и 8 БТРов и машины командования бросались в глаза всем проезжающим, поэтому командир спецотряда ФСБ РФ "Крым" полковник Игорь Гиркин в Славянске быстро получил информацию о месте нахождения украинских военных.
Гиркин в своих интервью рассказал, что  дал приказ устроить вооруженное нападение на украинских военных, чтобы начать войну и пролить кровь, "Спусковой крючок войны нажал именно я": 
https://censor.net/.../godovschina_okkupatsii_slavyanska...
Гиркин добавил: "Я не мог ждать, когда дончане самостоятельно "созреют" и "раскачаются", чтобы воевать" https://censor.net/.../polkovnik_fsb_girkin_rasskazal_kak...
Выполнить задачу полковник ФСБ поручил ударной группе наемников под командованием Сергея Журикова, позывной "Ромашка". Журиков до войны проживал в Киеве, уроженец Сумской области, который был, судя по фактам его деятельности, агентом российских спецслужб. Он был известен в Киеве в кругу профессиональных парашютистов, имел специальную военную подготовку, официально числился... фотографом и звонарем Киево-Печерской Лавры. Журиков с группой боевиков выдвинулся к месту встречи украинских военных, а непосредственно для нападения на месте была направлена машина с 4 местными боевиками, которые примкнули к Гиркину после захвата Славянска. Их задача была проста - открыть огонь и спровоцировать перестрелку. У Гиркина были веские основания предполагать, что нападение пройдет успешно. 
Боевики захватили машину местной охранной фирмы, на дороге было оживленное движение, ни одного выстрела на Донбассе не прозвучало, и, судя по открытому расположению, нападения украинские военные совершенно не ожидали.
Да, это было правдой. По приказу полковника Копачинского пулеметы БТРов были... разряжены. Копачинский неоднократно подчеркивал Сухаревскому свой приказ: "Огонь не открывать!" Как свидетельствует оператор-наводчик БТРа Олег Гапяк, полковник Копачинский разъяснял солдатам: "Якщо хтось буде стріляти, то кожну гільзу змусять знайти, прокуратура приїде і віддадуть під суд як на Майдані, тож не стріляйте". 
И еще одно трагическое совпадение - Копачинский передал приказ роте Сухаревского начать движение, и в результате было снято боевое охранение, рота вместе с Копачинским разместилась в БТРах. Сотрудники СБУ остались общаться у машин. 
Именно в этот момент к ним подъехала машина с наемниками спецотряда "Крым", раскрылась дверь и боевики внезапно в упор открыли огонь по сотрудникам СБУ. 
Одновременно из лесопосадки обстрел с дистанции нескольких сот метров начали боевики группы Журикова.
Обстановка была сложной - украинские военные были не готовы принять бой, боевого охранения не было. По ним стреляли гражданские люди, при этом на трассе было оживленное движение гражданского транспорта, рядом стояли зеваки.  
Командиры десантников Швец и Копачинский решения открыть огонь не приняли. Всю ответственность за ситуацию взял на себя старший лейтенант Вадим Сухаревский. 
Он оценил обстановку и крикнул наводчику-оператору своего БТРа Олегу Гапяку историческую фразу:
"Хули ты смотришь?! Ебашь!"
И этот приказ повторил всем экипажам. Наводчики начали заряжать пулеметы. Первым приготовился к бою оператор-наводчик БТР № 137 Николай Лавренчук Коля Лавренчук
: "Я бачив, що по нас ведуть вогонь з "зеленки" неподалік. Тож я дав попереджувальний постріл, а потім випустив кілька черг прицільних. Противник одразу припинив стріляти, бій швидко закінчився". (Джерело: https://censor.net/ua/r3122234)
Это был первый приказ украинского офицерав открыть огонь по врагу на украинской земле, и первые выстрелы.
Все это заняло буквально несколько минут.
В результате перестрелки погиб на месте капитан "Альфы" Геннадий Биличенко, были тяжело ранен полковник Кузнецов и еще один офицер "Альфы" Андрей Дубовик. 
Приказа преследовать отходящего противника или провести зачистку  десантники не получили, по приказу Швеца и Копачинского рота срочно покинула место боя. Поэтому противник отошел беспрепятственно. Затем боевики вернулись к месту боя и забрали оставленную на месте машину СБУ.
Сухаревский еще раз нарушил приказ Копачинского, и отправил два БТРа с ранеными "альфовцами" в больницу, они выжили. После боя он сказал, что открыл огонь сам, чтобы не подставлять под возможные разбирательства своих солдат, а отвечать самому.
Полковник Гиркин, конечно, свои потери в бою не афиширует, и назвал их в интервью "незначительными". 
По данным штаба пророссийских боевиков в Донецке они потеряли в этом первом боевом столкновении одного боевика убитыми и двух ранеными. Судя по данным противника, ответным огнем убит боевик Аванесян Рубен Робертович, 1985 года рождения, житель Донецка. 
Гиркин выполнил свою задачу, и теперь пришла очередь российских СМИ. Они опубликовали заявления, что "боевики "Правого сектора" напали на мирных граждан под Славянском". Таким образом, очевидно, что действия Гиркина были согласованы с руководством ФСБ, и российские СМИ немедленно начали дезинформационную операцию по легендированию нападения. 
Так началась война на Донбассе.
