% vim: keymap=russian-jcukenwin
%%beginhead 
 
%%file 20_01_2022.fb.fb_group.respublika_lnr.1.god_molodoj_gvardii_lnr
%%parent 20_01_2022
 
%%url https://www.facebook.com/groups/respublikalnr/posts/958165474819244
 
%%author_id fb_group.respublika_lnr,zimina_olesja
%%date 
 
%%tags donbass,lnr,mir_luganschine,molodaja_gvardia.lnr,zhizn
%%title ГОД МОЛОДОЙ ГВАРДИИ В ЛНР
 
%%endhead 
 
\subsection{ГОД МОЛОДОЙ ГВАРДИИ В ЛНР}
\label{sec:20_01_2022.fb.fb_group.respublika_lnr.1.god_molodoj_gvardii_lnr}
 
\Purl{https://www.facebook.com/groups/respublikalnr/posts/958165474819244}
\ifcmt
 author_begin
   author_id fb_group.respublika_lnr,zimina_olesja
 author_end
\fi

ГОД МОЛОДОЙ ГВАРДИИ В ЛНР. В луганской школе № 23 имени А. Н. Зозули провели
линейку, посвящённую памяти молодогвардейцев

Депутат Народного Совета ЛНР от Общественного движения «Мир Луганщине» Юрий
Юров и председатель Общественной организации «Молодая Гвардия» Даниил Степанков
20 января посетили Луганское общеобразовательное учреждение – среднюю
общеобразовательную школу № 23 имени А. Н. Зозули. Там состоялась линейка,
посвящённая памяти участников антифашистской подпольной молодёжной организации
«Молодая гвардия». Об этом сообщили в пресс-службе движения.

\ii{20_01_2022.fb.fb_group.respublika_lnr.1.god_molodoj_gvardii_lnr.pic.1}

Директор школы Ольга Сергеева рассказала, что ученики с интересом изучают
историю, чтят и помнят подвиги всех тех, кто стоял на защите Родины во время
Великой Отечественной войны.

\ii{20_01_2022.fb.fb_group.respublika_lnr.1.god_molodoj_gvardii_lnr.pic.2}

Она сообщила, что на территории посёлка Красный Яр находится дом отдыха
«Зелёная роща», а в военные годы там, на берегу Северского Донца, располагалась
школа особого назначения Украинского штаба партизанского движения. Или, как её
тогда называли, спецшкола. Там готовили специалистов для организации
партизанских отрядов, диверсионно-разведывательной работы и радистов. Эту школу
оканчивали и молодогвардейцы, в том числе Любовь Шевцова, Василий и Сергей
Левашовы, Виктор Третьякевич, Владимир Загоруйко.

\ii{20_01_2022.fb.fb_group.respublika_lnr.1.god_molodoj_gvardii_lnr.pic.3}

– Ещё в 2005 году мы обратили внимание на дом отдыха и увидели на фасаде здания
мемориальную доску. Конечно, после 2014 года мы несколько лет не ходили к
«Зелёной роще», ведь она находится у самой линии разграничения, посещать эту
территорию небезопасно. Но, когда пошли, увидели, что доски нет. Поэтому мы
хотим восстановить эту историческую память. Пока боевые действия не
прекратятся, мемориальная доска будет находиться в музейной комнате нашей
школы, – сказала директор учебного заведения.

Обращаясь к школьникам, Юрий Юров подчеркнул, что антифашистская подпольная
молодёжная организация «Молодая гвардия» – это гордость всех жителей бывшего
Советского Союза. Он отметил, что молодогвардейцы – это наша гордость, слава,
память, история, но и наша боль. Даже уйдя в вечность, эти молодые ребята
победили, они прогнали фашистов, а память о «Молодой гвардии» осталась навечно.
Парламентарий обратил внимание учеников, что в 2014 году сотни совсем юных
ребят тоже стали на защиту Донбасса.

– Пока мы здесь, пока мы верны своей земле и своей памяти, наша земля никому не
достанется и в любом случае победа будет за нами. Желаю всем, чтобы наконец-то
наступил мир, а Северский Донец был не линией фронта, а замечательным местом
для отдыха, чтобы на наших улицах были слышны не звуки выстрелов и взрывов, а
детский смех, – прокомментировал депутат.

Юрий Юров отметил, что информация о школе особого назначения, которую собрали
учащиеся учебного заведения, очень ценна и поблагодарил ребят за то, что они
интересуются историей, хранят память о молодогвардейцах.

Участникам мероприятия презентовали видеоролик о школе особого назначения,
которая находилась на территории дома отдыха «Зелёная роща». Его подготовили
волонтёры школы № 23 имени А. Н. Зозули, а именно: педагоги – активисты ОД «Мир
Луганщине» и члены детской организации «Юная гвардия».

Даниил Степанков поблагодарил присутствующих за то, что они знают свою историю
и прославляют имена настоящих героев.

– Чем старше становишься, тем больше понимаешь, насколько важно сохранять
историю. Мы поддерживаем вашу инициативу и постараемся помочь в восстановлении
мемориальной доски, – сказал председатель Общественной организации «Молодая
Гвардия».

По материалам пресс-службы ОД \enquote{Мир Луганщине}
