% vim: keymap=russian-jcukenwin
%%beginhead 
 
%%file slova.masshtab
%%parent slova
 
%%url 
 
%%author 
%%author_id 
%%author_url 
 
%%tags 
%%title 
 
%%endhead 
\chapter{Масштаб}

%%%cit
%%%cit_head
%%%cit_pic
\ifcmt
  pic https://day.kyiv.ua/sites/default/files/main/articles/22062021/do_tekstu_rizhkova.jpg
	caption Ось так подавався образ Гітлера-визволителя в колабораціоністській пресі на початку війни, коли вони ще не здогадувалися, яке майбутнє чекає на Україну... 
	width 0.4
\fi
%%%cit_text
Друга світова війна була найбільшим збройним конфліктом двадцятого століття і в
цілому в історії людства. Ще битви проходили практично на всіх континентах, а
жертвами стали близько 50 млн осіб. Однак для народів і країн-учасниць цієї
війни вона мала аж ніяк не однаковий характер і \emph{масштаб}. Багато держав
підключилися до сторін, що воювали, лише формально, а їхні матеріальна шкода і
людські жертви були мінімальними. Найголовніші втрати понесли Китай і
Радянський Союз
%%%cit_comment
%%%cit_title
\citTitle{Все ще дискусія? | Газета «День»}, Вадим Рижков, day.kyiv.ua, 22.06.2021
%%%endcit
