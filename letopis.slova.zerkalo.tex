% vim: keymap=russian-jcukenwin
%%beginhead 
 
%%file slova.zerkalo
%%parent slova
 
%%url 
 
%%author 
%%author_id 
%%author_url 
 
%%tags 
%%title 
 
%%endhead 
\chapter{Зеркало}
\label{sec:slova.zerkalo}

%%%cit
%%%cit_head
%%%cit_pic
%%%cit_text
А Данилов – это \emph{зеркало} Зеленского. Как и Потураев. Как Третьякова, как
Ткаченко, как Резников и как остальное стадо «слуг народа» системно и
последовательно расчеловечивающие наше общество.  Ну и чтобы для финального
контраста. «Настане день, коли можна буде пройтись недільним Донецьком,
Луганськом і Ялтою та побачити, як набережною гуляють родини з дітками,
фотографуються молодята, у парку дідусь вчить онучку кататись на велосипеді, а
поруч вуличний скрипаль неймовірно виконує мелодію Миколи Скорика. І все це
відбувається під синьо-жовтими прапорами». Это из поносика Блогера к 9 мая,
если кто не идентифицировал. Ну а пока кримці Бог вам послал \emph{водички}. Ха-ха, вы
же ее путиноиды хотели? Ну и, конечно же, мы любим вас, возвращайтесь
%%%cit_comment
%%%cit_title
\citTitle{Бог Данилова - это бог расчеловечивания}, 
Игорь Лесев, strana.ua, 20.06.2021
%%%endcit

%%%cit
%%%cit_head
%%%cit_pic
%%%cit_text
Хлопець хутенько вбрався, натягнув на себе шкіряну курточку, надів шоферського
кашкета. Заглянувши в \emph{дзеркало}, скептично хмикнув. Кашкет явно не пасував до
його видовженого засмаглого лиця, великих чорних очей, у яких ще хлюпалася
радість приємного сновидіння, буйних брів, що зрослися на переніссі. «Мої брови
— ліси захмарні!» — пробурмотів він рядок з вірша маловідомого поета, з яким
колись дуже заприязнився. Ну — пора! Він схопив зі столика сумку з
приготовленими книгами (у вільну хвилину перегляну на ходу!), крикнув на бігу:
«Полум’яний привіт, бабо Гапо! Воюйте з комунальними домовими і не здавайтеся!»
— і вискочив у клекіт липневого дня. Ранок уже набрав повної сили, світанкова
прохолода випарувалася під потужними променями літнього сонця. У дворику
гралися діти, билися за недавно розсипане пшоно голуби, голосно каркали ворони
у верхів’ях вікових осокорів
%%%cit_comment
%%%cit_title
\citTitle{Вогнесміх}, Олесь Бердник
%%%endcit
