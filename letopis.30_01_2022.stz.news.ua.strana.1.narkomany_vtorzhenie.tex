% vim: keymap=russian-jcukenwin
%%beginhead 
 
%%file 30_01_2022.stz.news.ua.strana.1.narkomany_vtorzhenie
%%parent 30_01_2022
 
%%url https://strana.news/news/374154-zelenskij-hovorit-chto-putin-ne-napadet-reaktsija-zapada.html
 
%%author_id minin_maksim
%%date 
 
%%tags narkomania,rossia,ugroza,ukraina,vtorzhenie,zapad,zelenskii_vladimir
%%title "Они там все наркоманы?" Как Запад реагирует на заявления Киева о том, что "вторжения России" не будет
 
%%endhead 
 
\subsection{\enquote{Они там все наркоманы?} Как Запад реагирует на заявления Киева о том, что \enquote{вторжения России} не будет}
\label{sec:30_01_2022.stz.news.ua.strana.1.narkomany_vtorzhenie}
 
\Purl{https://strana.news/news/374154-zelenskij-hovorit-chto-putin-ne-napadet-reaktsija-zapada.html}
\ifcmt
 author_begin
   author_id minin_maksim
 author_end
\fi

В западных СМИ все чаще поднимается тема о расхождениях между Киевом и
Вашингтоном по вопросу \enquote{российского вторжения}. 

\ii{30_01_2022.stz.news.ua.strana.1.narkomany_vtorzhenie.pic.1}

Пока в США на официальном уровне раздувают эту тему, Киев начинает все громче
заявлять, что проблема преувеличена. И признаков скорого нападения на Украину
нет.

Причем опровержения подготовки России ко вторжению идут от украинской власти
все в более резких тонах.

Эту позицию начали замечать и крупные мировые медиа, которые раньше писали о
\enquote{вторжении}, как о решенном деле. А особенно явственно тема несогласия с
Байденом прозвучала на встрече Зеленского с западными журналистами. 

И это противоречие - когда \enquote{жертва вторжения} говорит, что угрозы нет - может
кардинально поменять всю эту историю. Причем не только глобально, но и лично
для Зеленского.

\enquote{Страна} разбиралась, какие последствия будут у раскола между США и Украиной по
вопросу \enquote{нападения России}.

\subsubsection{Как на Западе реагируют на опровержения Зеленского}

Впервые о том, что расхождения существуют, заявил телеканал CNN в своем сюжете
по следам телефонного разговора Зеленского и Байдена 27 января.

Журналисты заявили, что президент США якобы пугал украинского коллегу скорым
нападением на Киев. А Зеленский не соглашался и говорил, что все не так
однозначно. 

Позже и в Белом доме, и в Офисе президента этот инсайд опровергли, а CNN даже
удалила свой сюжет. Но уже 28 января выпустила новый - с говорящим заголовком
\enquote{Напряженность между Вашингтоном и Киевом растет по мере затягивания
украинско-российского кризиса}. 

Агентство сообщило, что Зеленский остался недоволен реакцией США на угрозу
вторжения России. При этом Байден считает, что Киев недооценивает эту самую
угрозу.

\enquote{Почему вы затеваете ссору с США? Почему вы допускаете утечки и искажение
информации, когда США - один из ваших единственных настоящих друзей? И зачем
просить больше оружия, если вы говорите, что угроза не выросла?} – приводит CNN
слова американского чиновника.

На следующий день - после пресс-конференции Зеленского западным СМИ - с
материалом о \enquote{зраде} со стороны Киева вышло издание Politico. Заголовок -
\enquote{Почему \enquote{неизбежное} бесит Зеленского}.

Журналисты пишут: Киев опасается, что США намеренно преувеличивают российскую
угрозу. И делает это для заключения договоренностей с Москвой, например, по
Донбассу (об этом, кстати, еще на прошлой неделе сообщала \enquote{Страна}).

В статье говорится, что изменение позиции президента Украины по российской
угрозе связано с \enquote{растущим раздражением} в отношениях с США.

\enquote{Действия Зеленского также вызвали вопросы о том, почему США отправляют Украине
оружие и деньги, если Киев не слишком обеспокоен масштабным вторжением?} -
задается вопросом издание в день пресс-конференции, которую дал Зеленский
западным СМИ, чтобы сбить волну военных слухов, распространяемых Западом.

Один из последних аналогичных выпадов сделало 30 января издание Washington
Post. Оно пишет, что западных чиновников встревожил публичный скептицизм
Зеленского и его команды относительно \enquote{вторжения}.

\enquote{Есть раскол среди его административных советников и того, что он может слышать
от своих военных и разведывательных служб}, - сказал правительственный чиновник
США, специализирующийся на делах России. \enquote{Я думаю, что Зеленский и его
политическая команда работают, исходя из собственного набора приоритетов, и они
не обязательно совпадают с приоритетами разведки и военных}, - заявил он
изданию.

То есть Washington Post намекает, что украинский президент саботирует данные
разведки о \enquote{вторжении} ради каких-то своекорыстных целей. 

\subsubsection{\enquote{Зеленский некомпетентен}}

Как видим, американские медиа наконец-то заметили заявления Украины о том, что
вторжения не будет. Но вопросы в СМИ звучат не к Белому дому и практически не
ставят его трактовок под сомнение. А наоборот - рисуют Зеленского возмутителем
спокойствия. И даже грозят остановить поставки оружия, если тот продолжит
дезавуировать \enquote{линию партии}. 

\raggedcolumns
\begin{multicols}{2} % {
\setlength{\parindent}{0pt}

\ii{30_01_2022.stz.news.ua.strana.1.narkomany_vtorzhenie.pic.1.guzhva}


И это достаточно тревожный знак для украинского президента. 

"Ранее западные СМИ практически игнорировали тот факт, что об угрозе
\enquote{нападения Путина} говорят в основном в США и в Британии. А Киев эту
угрозу опровергает и, более того, считает ее преднамеренным раздуванием паники
и \enquote{подставой} со стороны западных партнёров, так как из-за этого
Украина несёт экономические потери.

Но после пресс-конференции Зеленского, который хоть и путано, но донёс
иностранным журналистам мысль о несогласии с Байденом относительно реальности
угрозы вторжения (собственно, это и была главная цель мероприятия), делать вид,
что ничего не происходит, для западных СМИ было уже невозможно.

И стали появляться публикации о недовольстве Зеленского нагнетаемой на Западе
паникой, а также о подозрениях Киева, что эта паника раздувается дабы принудить
украинские власти к уступкам по Минским соглашениям. То есть все то, о чем
\enquote{Страна} писала последние недели.

Теперь интересно, какой выход может быть из этой ситуации.

С точки зрения логики, после того как главная \enquote{жертва} потенциального
\enquote{вторжения России} заявила, что в угрозу этого вторжения не верит,
накал нагнетания в западных СМИ должен снизиться. Ведь глупо кричать
\enquote{Путин нападет на Украину}, если сама Украина говорит: \enquote{Путин
не нападет}.

\end{multicols} % }

Судя по всему, на такой эффект и рассчитывает Зеленский.

Но эффект может быть и другой: на Западе объявят, что Зе просто ослеп и не
видит угрозы. И вообще некомпетентен. А потому по-прежнему нужно бить во все
колокола. Причём ещё громче бить.

И эту тему очень охотно многие подхватят и внутри Украины", - пишет главный
редактор \enquote{Страны} Игорь Гужва.

Пока все выглядит так, что Штаты решили пойти по второму пути.

Накал заявлений об \enquote{угрозе российского вторжения} не ослабевает.

Американцы продолжают нагнетать как по официальным каналам, так и в мейнстримных медиа.

В Пентагоне вчера, 29 января, провели пресс-конференцию, где в жутких красках
расписали подробности \enquote{вторжения}. 

\enquote{С учетом того, какие силы собраны - сухопутные маневренные войска,
артиллерия, баллистические ракеты, военно-воздушные силы, - если все это вместе
будет применено против Украины, произойдет нечто очень серьезное, что приведет
к значительному числу жертв. Можете себе представить, как это будет выглядеть в
густонаселенных районах, около дорог и так далее. Это может быть просто
кошмарно. Может быть ужасающе}, — сказал министр обороны США Ллойд Остин.

А агентство Reuters вчера увидело \enquote{главный признак} того, что Россия
готовится напасть - к войскам РФ на границе начали якобы подвозить запасы
крови. К слову, эту информацию оперативно опровергли в Минобороны Украины.

Кроме того, в уже цитированной статье от Washington Post заявляется, что
украинская разведка эвакуировала секретные материалы и оборудование из своей
штаб-квартиры в Киеве на запад Украины. Об этом газете якобы сообщили
украинские разведчики на условиях анонимности.

Понятно, что официально такой эвакуации никто не подтверждал.

Но все эти заявления показывают, что Америка снижать масштабы военной истерии
не собирается даже после попыток Зеленского ее остановить. Что, как и
говорилось, уже оборачивается западными нарративами о некомпетентности или, что
хуже - намеренном \enquote{вредительстве} со стороны украинского президента. 

Причем не только из США доносятся уже такие упреки.

Депутат Европарламента Виола фон Крамон, которая у немецких \enquote{зеленых} отвечает
за внешнюю политику, резко отреагировала на слова главы фракции \enquote{слуг} Давида
Арахамии, который прямым текстом обвинил Запад в нагнетании паники вокруг
российского вторжения.

\enquote{Они все там употребляют наркотики в Украине или что там происходит?} - написал
Крамон в Твиттер.

Тут интересно, что имела в виду Крамон под словом \enquote{все}. Кто такие \enquote{все}? Не
исключено, что она намекает на слухи о наркозависимости самого Зеленского,
которые курсируют все более активно.

Но в целом понятно, что заявления Зеленского на Западе начинают использовать
для раскрутки критики против него.

\subsubsection{А что на самом деле происходит с \enquote{вторжением}?}

Эту историю на Западе раскручивают с октября.

Информационная кампания сначала пошла через крупные СМИ, близкие к Демпартии
США. Затем к ней присоединились Белый дом и Пентагон. 

Заявления о \enquote{вторжении} стали громче в преддверии видео-встречи Путина и
Байдена в декабре. Таким образом на Западе начали говорить об этой встрече
только в контексте якобы подготовки нападения России на Украину. Практически
умалчивая об остальных ее аспектах (например, обсуждении \enquote{Минска-2} и
сокращении направленных против РФ вооружений).

По той же модели происходили все дальнейшие контакты по линии Россия-США,
которые шли в продолжение переговоров президентов. Чем больше встреч - тем
больше разговоров о \enquote{вторжении}.

Уже тогда высказывались предположения, что идет создание отвлекающего фона от
неких истинных целей контактов между Москвой и Вашингтоном. На это указывали
два факта:

1. Россия не в одночасье создала на своих западных границах группировку в
100-130 тысяч военных. Такую цифру называл, к примеру, Зеленский еще весной
2021 года, когда Запад ни о каком вторжении не говорил. Позже это количество
подтвердила и американская разведка. 

При этом и украинская власть, и военные эксперты заявляли, что для полноценного
нападения на Украину такого количества военных явно недостаточно. Сопоставимые
цифры по личному составу Украина в разное время сосредотачивала, к примеру, на
Донбассе. То есть - против двух неполных регионов.

Таким образом, говорить о том, что стотысячная группировка россиян захватит всю
Левобережную Украину, а заодно Приднестровье, Одессу и Киев, как это подают в
США, - явная манипуляция.

Это не говоря уже о том, что никто внятно не объясняет, зачем России
понадобилось именно сейчас, на восьмой год конфликта, вторгаться в Украину -
ставя под угрозу запуск \enquote{Северного потока - 2} и получая болезненные санкции. 

К слову, сама Россия на протяжении всех месяцев этой кампании регулярно
отрицала подготовки ко \enquote{вторжению}. Причем в последние дни она начала это
делать со ссылкой на украинских официальных лиц.

2. К этой громкой кампании не присоединилась \enquote{старая Европа}. Германия,
Франция, Италия, Австрия, Евросоюз отказались даже эвакуировать посольства из
Киева.

Также они отмежевались от предложений американцев по самым деструктивным
санкциям против РФ - наподобие отключения SWIFT или нефтегазового эмбарго. А
Германия открыто отказалась присоединяться к поставкам оружия Украине. 

То есть англо-американцы в своих ничем не подкрепленных заявлениях о
"вторжении" остались одни даже внутри западного сообщества (в этом им
подпевают, да и то все менее уверенно, лишь некоторые страны Восточной Европы).

Это значит, что на европейском континенте мало кто вообще верит в то, что
Россия атакует Украину (в том числе не верят и украинские власти).

Поэтому все чаще идут разговоры о том, что американцы решают за счет раскрутки
этой кампании какие-то свои задачи.

Первая задача достаточно понятна - отвести от Байдена упреки в "зраде" от
внутриамериканских оппонентов. Они могут дискредитировать и поломать любые
договоренности с Россией, включающие уступки по тем или иным вопросам (пусть
даже и взаимные).

Поэтому сейчас официальная рамка переговоров с Москвой для самих США - это не
\enquote{большая сделка}, а \enquote{спасение Украины от вторжения}.

Что же в реальности на столе у сторон - вопрос пока не слишком публичный. Хотя
многие темы проговариваются достаточно четко: деэскалация на Донбассе,
отведение стратегических вооружений, сокращение числа учений в Европе, а также,
вероятно, негласные гарантии не принимать Украину в НАТО.

Вторая задача связана, скорее всего, с Украиной. Заявления о скором \enquote{вторжении}
оказывают мощное давление на украинскую экономику и пускают под откос гривну.
Что не может не конвертироваться в политическое давление США на Офис
президента.

Создавая рукотворный финансовый коллапс в стране, Вашингтон делает Киев более
зависимым от внешних займов. Причем не обычных займов через размещение
евробондов на открытом рынке без всяких особых условий (сейчас эта возможность
для Украины закрыта из-за поднятой паники вокруг \enquote{вторжения}). А от займов
международных финансов организаций или же под гарантии США, которые идут под
жесткие политические условия (именно таким образом страны Запада побудили
Порошенко подписать в 2015 году вторые Минские соглашения, так как Украине
тогда срочно нужен был кредит МВФ из-за падения гривны).

Зеленский уже рассказывал, что обсуждал с Байденом на этой неделе финансовую
помощь Украине. И можно вспомнить, как тот же Байден, будучи вице-президентом,
признавался, как обменял при Порошенко увольнение генпрокурора Виктора Шокина
на миллиард долларов от МВФ. 

То есть Байдену не впервой шантажировать Киев деньгами в обмен на исполнение
требований Штатов. 

С одной стороны, эти требования неизменно касаются лоббируемых американцами
назначений в правоохранительной системе Украины (например, утвердить лояльного
США главу антикоррупционной прокуратуры).

С другой - все больше признаков, что Штаты усилили давление на Киев по реальной
деэскалации на Донбассе и выполнению Минских соглашений. Что, видимо, делается
по договоренности с россиянами. На такое давление уже дважды намекал Зеленский.

Но пока большой вопрос, смогут ли американцы заставить его реализовать
"Минск-2", включая его политическую часть. 

У Зеленского считают это политически самоубийственным шагом. И вряд ли на него
пойдут без твердой уверенности, что Запад спасет Зе от \enquote{майдана}
антикапитулянтов.

Пока таких гарантий Банковой, судя по всему, никто не дает. Да и простым
обещаниям никто уже не верит - особенно после ситуации с Януковичем, который
после подписания соглашения с оппозицией при посредничестве Запада вынужден был
бежать из страны. Хотя должен был оставаться президентом до конца года. 

Как вариант, Зеленскому нужные какие-то прямые заявления и явные действия
западных кураторов против политических оппонентов Банковой. От прямых внушений
посольств тому же Порошенко - до разрешения на уголовные дела, санкции, аресты. 

Но в США не торопятся зачищать под Зе весь промайдановский лагерь. Наоборот,
там все чаще критикуют Зеленского за \enquote{преследование Порошенко}.

А, учитывая, что разговоры о \enquote{вторжении} в США не стихают, попытки додавить
Зеленского будут продолжаться. Причем, судя по негативной риторике на Западе в
отношении Зе, возможно, методы станут более жесткие, чем просто нагнетание
паники вокруг вторжения.
