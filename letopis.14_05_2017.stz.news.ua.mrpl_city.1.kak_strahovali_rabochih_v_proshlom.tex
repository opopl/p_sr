% vim: keymap=russian-jcukenwin
%%beginhead 
 
%%file 14_05_2017.stz.news.ua.mrpl_city.1.kak_strahovali_rabochih_v_proshlom
%%parent 14_05_2017
 
%%url https://mrpl.city/blogs/view/kak-strahovali-rabochih-v-proshlom
 
%%author_id burov_sergij.mariupol,news.ua.mrpl_city
%%date 
 
%%tags 
%%title Как страховали рабочих в прошлом
 
%%endhead 
 
\subsection{Как страховали рабочих в прошлом}
\label{sec:14_05_2017.stz.news.ua.mrpl_city.1.kak_strahovali_rabochih_v_proshlom}
 
\Purl{https://mrpl.city/blogs/view/kak-strahovali-rabochih-v-proshlom}
\ifcmt
 author_begin
   author_id burov_sergij.mariupol,news.ua.mrpl_city
 author_end
\fi

Этот довоенной моды ридикюль из коричневой кожи хранится в семье, как память о
горячо любимой маме, преждевременно ушедшей из жизни. Что в нем можно найти?
Прежде всего, тончайший, едва уловимый запах французской губной помады и
красный тюбик этого косметического продукта с рельефом \enquote{Paris}. А еще несколько
старых квитанций, свидетельство об окончании трудовой школы и документы
дедушки. Среди них - произведение мариупольской типографии \enquote{Печатное искусство}
размером чуть больше почтовой открытки, переплет которого обтянут серым
коленкором. На нем отпечатано: \enquote{Больничная касса при заводе Никополь –
Мариупольского Горного и Металлургического Общества. Членская книжка №...}. От
руки написано \enquote{5884}.

На первой странице указано, что участником кассы является Литвиненко Петр
Савельевич, родившийся 29 ноября 1881 года. Зафиксировано: \enquote{Книжка эта выдана в
доказательство участия в больничной кассе}. Есть дата выдачи – 23 декабря 1916
года. Страницы 4 и 5 отданы списку членов семьи, находящихся на иждивении
участника кассы. Это - жена Татьяна Кирилловна 32-х лет, дети – Александра 6-и
лет, Лидия – 3-х лет, Валентина – одного месяца от роду, а также мать –
Анастасия 60-ти лет. Последующие девятнадцать страниц членской книжки посвящены
Положениям государственного страхования рабочих на случай болезни и от
несчастных случаев, согласно закону от 23 июня 1912 года. Закон этот был принят
третьей государственной Думой Российской империи на волне подъема
революционного движения в стране. Он был результатом долгой и упорной борьбы
рабочих России за свои права.

Если можно так сказать, больничные кассы были инструментом, обеспечивающим
обязательное социальное страхование. Конечно, все содержание членской книжки
больничной кассы в газетной статье передать невозможно, но некоторые извлечения
из неё могут быть интересны современному читателю.

"1) Действию сего положения подчиняются те фабрично-заводские, горные,
горнозаводские, железнодорожные, судоходные по внутренним водам (по рекам,
каналам, внутренним морям и озерам) и трамвайные предприятия, в коих постоянно
заняты не менее 20 рабочих и применяются паровые котлы или маши­ны,
приводимые в  действие силами природы (воды, газа, электричества и т.п.) или
животных, а равно те из вышеуказанных предприятий,  в коих хотя и не
применяются паровые котлы или означенные выше машины, но число постоянно
занятых рабочих не менее 30.

4) Действию настоящего Положения подчиняются все, без различия пола и
возраста, лица, кои по найму заняты работами предприятия или службою в нем. Не
подчиняются действию сего Положения те из указанных выше лиц, кои наняты для
исполнения случайных работ, продолжающихся не более одной недели.

5) Во всех предусмотренных сим Положением случаях служащие в предприятиях
приравниваются к рабочим.

7) Лицам, подчиненным действию настоящего Положения, предоставляется врачебная
помощь и денежные пособия на основаниях, определенных сим Положением.

8) Врачебная помощь предоставляется за счет владельцев предприятий.

9) Денежные пособия выдаются больничными кассами за их счет.

43) Больничная касса может принять на себя расходы по предоставлению врачебной
помощи членам семейств участников кассы, состоящим на их иждивении, а также
выдавать сим лицам пособия по случаю болезни и на погребение и пособия, по
случаю родов женам участников. Размеры помощи и пособий определяются в
указанных уставом кассы пределах, при чем на покрытие означенных в настоящей
статье расходов больничная касса может назначать не более одной трети общей
суммы поступающих в кассу в течение года взносов и приплат.

63) Общему собранию больничной кассы предоставляется постановлять, что
участники кассы, нарушающие устав или постановления общего собрания
относительно порядка подачи заявлений о болезни и исполнения заболевшими
распоряжений врачебного персонала, пользующего больных, могут быть лишаемы
денежного пособия частью или полностью, либо подвергаемы денежному взысканию в
пользу больничной кассы. Размер этого взыскания определяется в каждом отдельном
случае постановлением правления кассы и не может превышать трех рублей за
каждое нарушение; означенное постановление Правления сообщается владельцу
предприятия, который удерживает при ближайшем платеже заработка наложенное
взыскание.

65) Размер взносов участников больничных касс устанавливается общим собранием
кассы в пределах от одного до двух процентов с суммы заработка. В больничных
кассах с числом участников меньше четырёхсот размер взноса может быть увеличен
до трех процентов. Для лиц, заработок которых превышает пять рублей в день или
содержание - одну тысячу пятьсот рублей в год, взнос исчисляется из пяти рублей
в день.

70) Взносы не удерживаются с участника кассы за все время, в течение которого
он лишен трудоспособности. Равным образом и владелец предприятия освобождается
в таких случаях от внесения соответственной приплаты".

Нужно отметить, что при заводе Никополь – Мариупольского Горного и
Металлургического Общества задолго до принятия закона от 23 июня 1912 года (в
1897 году) была устроена амбулатория, развернутая впоследствии по тем временам
вполне современную больницу, где оказывали медицинскую помощь рабочим
предприятия и их иждивенцам. С принятием закона, о котором здесь идёт речь, это
лечебное учреждение вошло в сферу деятельности больничной кассы при заводе.

Пока не удалось найти документов, которые бы свидетельствовали об эффективности
работы больничной кассы при заводе Никополь, кроме одного - это удостоверение
больничной кассы с таким текстом: \enquote{Сим удостоверяется, что рабочий
Никополь-Мариупольского завода Литвиненко Петр отправляется для климатического
лечения в село Кременное Купянского уезда Харьковской губернии} ...

Больничные кассы, в том числе и при заводе \enquote{Никополь} в Мариуполе, были
упразднены в 1921 г. в связи с введением государственного социального
страхования в Советской России.
