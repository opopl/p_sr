% vim: keymap=russian-jcukenwin
%%beginhead 
 
%%file 23_02_2022.fb.granovskij_vladimir.kiev.1.rasstavanie
%%parent 23_02_2022
 
%%url https://www.facebook.com/vladimir.granovski/posts/4985551388171964
 
%%author_id granovskij_vladimir.kiev
%%date 
 
%%tags kiev
%%title Расставанье
 
%%endhead 
 
\subsection{Расставанье}
\label{sec:23_02_2022.fb.granovskij_vladimir.kiev.1.rasstavanie}
 
\Purl{https://www.facebook.com/vladimir.granovski/posts/4985551388171964}
\ifcmt
 author_begin
   author_id granovskij_vladimir.kiev
 author_end
\fi

Расставанье.

Вы помните минуты раставанья? 

\ii{23_02_2022.fb.granovskij_vladimir.kiev.1.rasstavanie.pic.1}

Вы помните последние слова прощанья?  Не мучайте себя, последний поцелуй не
наполняет душу радостью желанья, последний вгляд не дарит счастья ожиданья. Мы
любим помнить первые свиданья, в глазах горит огнем закат в предверьи ночи
долгожданной, и дрожь по телу от первого касанья, и утром блажь от просыпанья.
А вы решитесь на шаги, что вопреки и вспомните, когда комок стоит в груди
дыханье прерывая, а сердце бьется внутрь и просит слов: «Нет подожди, постой,
прости, не уходи». Нет, уходи, забудь, что был и мир у ног, что Бог внутри и
жизнь, как праздник переполняла каждый шаг и ты шептал: «За что, за что такое
наказание, за что подарен этот детский всхлип в ночи?» Вы помните тот день,
когда ушли, оставив на столе ключи, когда уже в тиши безмолвной, вы думали о
том, что не смогли, что дни теперь не дни, что ночь теперь бессонна и каждый
день теперь одни воспоминанья. Я знаю вы не думали тогда, что даже позвонив
случайно, вы не вернетесь уж туда, где шли дожди, где слезы от души, а смех,
как грозы разрывает дни. Идите же туда, куда вы шли, но память иногда взорвет
тот миг, когда и вы, и вас любила нагота случайного признанья, где Киев был,
где будем мы, иносказанье. 

Киев, март, 2022 год
