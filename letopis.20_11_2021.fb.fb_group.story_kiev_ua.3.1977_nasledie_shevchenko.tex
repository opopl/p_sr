% vim: keymap=russian-jcukenwin
%%beginhead 
 
%%file 20_11_2021.fb.fb_group.story_kiev_ua.3.1977_nasledie_shevchenko
%%parent 20_11_2021
 
%%url https://www.facebook.com/groups/story.kiev.ua/posts/1801830113347115
 
%%author_id fb_group.story_kiev_ua,fedjko_vladimir.kiev
%%date 
 
%%tags kiev,shevchenko_taras
%%title 1977: Таємне жіноче товариство по вивченню забороненої літературної спадщини Шевченко
 
%%endhead 
 
\subsection{1977: Таємне жіноче товариство по вивченню забороненої літературної спадщини Шевченко}
\label{sec:20_11_2021.fb.fb_group.story_kiev_ua.3.1977_nasledie_shevchenko}
 
\Purl{https://www.facebook.com/groups/story.kiev.ua/posts/1801830113347115}
\ifcmt
 author_begin
   author_id fb_group.story_kiev_ua,fedjko_vladimir.kiev
 author_end
\fi

1977: Таємне жіноче товариство по вивченню забороненої літературної спадщини Шевченко

***

Кожна київська кав’ярня мала свою тусовку! 

З Будинку Художника, на Львівській площі, я часто ходив пити каву у
«Гастроном», що був навпроти (по діагоналі). Публіка там збиралася різна... Якось
я звернув увагу на одну молоду людину, яка своїм зовнішнім виглядом нагадувала
студента-народовольця XIX століття... Років 25-ти, худорлявий, високого зросту, з
інтелігентним обличчям... Одягнений у довгу чорну шинель з блискучими ґудзиками,
завжди з якоюсь книгою під пахвою. Як потім я дізнався, звали юнака Олександр
А-кий, а працював він кочегаром у газовій котельні неподалік від Львівської
площі. 

\ii{20_11_2021.fb.fb_group.story_kiev_ua.3.1977_nasledie_shevchenko.pic.1}

Досить часто О.А. приходив пити каву не один, а з жінкою. Жінки були різні, але
майже всі вони на вигляд були старші за нього. 

В друзях у мене була одна подружня пара, Павло – художник, Аня – інженер.
Іноді, в обідню перерву у Ані, ми втрьох зустрічалися на каві. Подружнє життя у
них не склалося і вони розлучилися. Але це не вплинуло на мої з ними стосунки.
Досить часто я пив каву з Анею, розмовляючи про балетний театр, літературу і
мистецтво, фотографію. 

Якось прийшовши на каву, я зустрів Аню в компанії О.А. Вона представила мені
Олександра і таким чином ми познайомилися, але далі ввічливих привітань при
зустрічі у нас стосунки не розвинулися. 

Одного разу Аня запросила мене додому на колективне читання журнальної
публікації про Григорія Распутіна – «Последний временщик последнего царя»
(рос.). При цьому попередила, що про ці посиденьки нікому не можна розповідати,
і що посиденьки будуть до ранку!

Годині о восьмій вечора на квартирі у Ані зібралася компанія осіб з десяти
молодих жінок. Я там був єдиним представником чоловічої частини людства і
нікого окрім Ані не знав. Всі жінки були у віці від 25 до 35 років (візуально)
і без обручок на руці, або з обручкою на лівій руці. Що свідчило про «холостий»
стан. 

Чесно кажучи, я відчував себе не дуже комфортно! Один у великій компанії жінок…
посиденьки до ранку… А коли на столі з’явилося декілька пляшок сухого вина,
«Празький» і кава, то я грішним ділом запідозрив, що мене запросили на оргію і
почав обдумувати плани своєї подальшої поведінки. Вже три роки, як я був
одружений; у мене чудова дружина і грішити з незнайомими жінками було мені не
комільфо.

Порізали тортик, відкоркували вино, розлили по бокалам… Після другого бокалу
Аня оголосила, що «приступаємо до читання…» і принесла декілька розшитих
журналів з публікацією. Процедура читання була такою… Аня прочитала перші пару
аркушів і передала їх по годинниковій стрілці подрузі, а сама почала читати
наступні аркуші. Поступово всі отримали аркуші і втягнулися у прочитання
повісті. Ну, і само собою, потихеньку смакували тортиком, вином, кавою…
«Поляна» була гарна… По ходу читання обмінювалися думками (коментарями) щодо
прочитаного.

Колективне читання починалося о 8-й годині вечора у п’ятницю, а завершилося
десь о 8-й годині ранку в суботу. 

(Забув сказати, що для домашніх мною була озвучена версія про «термінове
завдання» з «виїздом за межі Києва». Мобільних телефонів і пейджерів тоді ще не
було!)

***

Пройшло пару місяців... Зустрічаю Аню на каві і вона починає мені жалітися... Що
О.А. виявився негідником, що «він спав не тільки зі мною, а ще й з багатьма
жінками...» І що її неодноразово викликають в обласне управління КДБ (на вул.
Рози Люксембург)... На моє здивоване запитання, чому це КДБ раптом стало
цікавитися «хто з ким спить», Аня розповіла кумедну історію.

А-кий знайомився з жінками, заворожуючи їх і своїм виглядом і розмовами про
літературу, запрошував до себе в кочегарку, де читав їм українську поезію,
переважно Тараса Шевченко. Чудово володіючи українською мовою, він читав їм
вірші з обгорілих книжок, розповідаючи, що це «нелегальні видання «Кобзаря», з
університетської бібліотеки, спаленої комуністами-кадебістами»... 

[В 60-х роках дійсно була загадкова історія з пожежею в університетській
бібліотеці, яка тривала декілька діб. За офіційною версією підпал здійснив
психічно хворий працівник за допомогою фосфору.]

Також Олександр розповідав різні «романтичні історії» з життя Тараса
Григоровича, які зводилися до «сексуальних пригод Шевченка із світськими дамами
Петербурга, яких він малював в оголеному вигляді, за що його і було відправлено
у заслання із забороною малювати!»

Самотні жінки (а він був непоганим психологом і вибирав собі подруг з числа
«старих дів» і розведених) були від нього в захваті, приносили на зустрічі їжу
і спиртне. А він, таємничо від інших, призначав обраним індивідуальні зустрічі,
на яких схиляв їх до сексу.

З часом хтось із жінок дізнався, що вона не єдина обранка О.А. в сексуальних
розвагах! 

Образа! Ревнощі! І жага помсти!

Жага помститися привела жінку в Обласне управління КДБ по місту Києву і
Київській області. 

Боротьба з «українським буржуазним націоналізмом» завжди займала чільне місце в
діяльності партійного апарату КПРС/КПУ та органів державної безпеки. 

В УРСР головним ідеологом «боротьби» став Валентин Маланчук. Перебуваючи на
посаді секретаря Львівського обкому КПУ з ідеології (грудень 1964 – лютий 1967)
він набув слави теоретика й борця з українським націоналізмом, самочинно
призупинивши на території Львівської області рішення про обов'язкове введення
української мови у вищих навчальних закладах. У розріз із політикою тодішнього
керівництва ЦК КПУ Маланчук надрукував у Москві кілька статей, де боротьбу з
націоналізмом розглядав як основне завдання компартії.

Перший секретар ЦК КПУ Петро Шелест різко відреагував на виступ Маланчука, але
просто зняти Маланчука з роботи не зміг, бо той здобув підтримку кремлівських
ідеологів, в першу чергу Михайла Суслова, секретаря з ідеології ЦК КПРС
(1947-1982). Тому Шелест у лютому 1967 році перевів Маланчука на посаду
заступника міністра вищої та середньої спеціальної освіти Української РСР, що в
тодішніх умовах фактично означало зупинення партійної кар'єри. Офіційним
ґрунтом до усунення Маланчука послугувала його груба поведінка зі львівською
науковою та творчою інтелігенцією. 

Одразу після усунення Петра Шелеста з керівництва Компартією України і заміну
його більш лояльним до Москви Володимиром Щербицьким Маланчук 10 жовтня 1972
року на Пленумі ЦК КПУ був підвищений до посади секретаря ЦК КПУ з ідеології.
Вже 11 жовтня на засіданні Політбюро ЦК КПУ Валентину Маланчуку та начальнику
КДБ УРСР Віталію Федорчуку було  доручено розробити і внести на розгляд
Політбюро ЦК КПУ практичні заходи з питань посилення боротьби проти
націоналістичної діяльності та пропаганди на теренах УРСР. 

Почалася «епоха маланчукізму» в Україні! 

Тільки за перші півроку перебування Маланчука на посаді він зупинив у
видавництвах та друкарнях 600 версток готових до другу українських книг!

Компартія, КДБ, Прокуратура і суди СРСР чітко взаємодіяли – партія вказувала на
«ворогів» - інакомислячих; КДБ виконувало замовлення партії щодо оперативної
розробки «ворогів»; Прокуратура санкціонувала порушення кримінальних справ і
арешти; суди виносили вироки!

У 1972-му був засуджений Василь Стус і багато інших патріотів України! 

У 1974-му був висланий за межі Радянського Союзу Олександр Солженіцин! Був
вигнаний за кордон і киянин Віктор Некрасов. 

Західні радіоголоси щоденно розповідали про Солженіцина, Віктора Некрасова,
Гелія Снєгірьова, Леоніда Плюща, генерала Григоренко, академіка Сахарова і
Олену Боннер… Активно працювала «глушилка» – комплекс в Биківні, антенні поля
якого займали біля 160 гектарів землі..

І раптом у Києві існує ціла таємна жіноча націоналістична організація, про яку
КДБ нічого не відомо! У П’ятому управлінні відчули реальні перспективи
отримання зірочок і нагород!

Як і ведеться за інструкцією, ініціативну жінку завербували, і вона розкрила
явку, лідера та усіх членів «організованої націоналістичної групи»! Цілком
можливо, що вона провела і оперативні записи зібрань на спецтехніку. 

Закипіла оперативна розробка… Всіх членів «таємного гуртка» встановили і
почалася реалізація «агентурної справи»!

Коли в черговий раз жіноче товариство зібралося на читку «заборонених творів
Шевченка», у двір будинку, де розташована кочегарка, влетіли три «Волги». Люди
в цивільному затримали усіх присутніх, відправили їх спецтранспортом в
Управління, і провели ретельний обшук приміщення кочегарки. При цьому було
вилучено декілька обгорілих книжок з віршами Т.Г. Шевченко.

О.А. був затриманий в КПЗ на 72 години!

За цей час були проведені допити усіх членів «таємної організації» і проведені
експертизи книжок. Нічого націоналістичного в свідченнях не було зафіксовано! А
експертиза книжок, показала, що це звичайний «Кобзар» 1960 року видання;
парочка шкільних «Кобзарів», які можна було купити в будь-якому книжковому
магазині. А також «Кобзар» виданий у 1921 році і куплений у магазині «Букініст»
на вулиці Леніна. З книжок були обідрані обкладинки і їх було обсмалено на
газовій плиті!

О.А. випустили із КПЗ, декілька разів викликали на бесіди, як і інших жінок.
Але справа про «Таємне жіноче товариство по вивченню забороненої літ-спадщини
Шевченко» луснула!

***

Якось я побачив О.А. в кав'ярні і користуючись тим, що ми знайомі, поцікавився,
чим закінчилася справа. Він досить детально розповів мені про перебіг подій;
про те, що його офіційно попередили про відповідальність за «націоналістичну
пропаганду і агітацію» та спробували залучити до агентурної співпраці, але він
відмовився. 

***

З рукопису спогадів «Погляд у минуле», написаних для особового фонду Федька В.Ф
у Державному архіві Київської області.

У рукопису для архіву вказано повні імена діючих осіб.

\ii{20_11_2021.fb.fb_group.story_kiev_ua.3.1977_nasledie_shevchenko.cmt}
