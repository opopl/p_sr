% vim: keymap=russian-jcukenwin
%%beginhead 
 
%%file slova.gosudarstvo
%%parent slova
 
%%url 
 
%%author 
%%author_id 
%%author_url 
 
%%tags 
%%title 
 
%%endhead 
\chapter{Государство}
\label{sec:slova.gosudarstvo}

%%%cit
%%%cit_pic
%%%cit_text
И вот, в начале XX века, когда в результате Первой мировой войны, и
Австрийская, и Российская империи развалились, у западных русских, потерявших
свои корни и ставших «украинцами», появилась идея создания собственного
\emph{государства}. Ну, идеи создания собственных независимых \emph{государств} имеются не
только у национальных меньшинств, но даже и у однородных в этническом плане
социальных групп. Достаточно вспомнить Дальневосточную республику, образованную
в 1920 году и населенную русскими. Самый яркий пример – образование ряда
испаноязычных латиноамериканских \emph{государств}, образовавшихся на руинах Испанский
Америки. Можно вспомнить и прецедент образования США – ведь отделились от
Англии не индейцы и не негры, а те же самые англичане
%%%cit_title
\citTitle{Почему современную Украину назвали «окраиной», а не более престижно - Киевской Русью?}, 
Исторический Понедельник, zen.yandex.ru, 22.02.2021 
%%%endcit

%%%cit
%%%cit_head
%%%cit_pic
%%%cit_text
Українська нація на сучасному етапі своєї історії у здавалося б \enquote{незалежній}
\emph{державі} знаходиться у підневільному стані. Хоч ми вже чверть століття є начебто
\enquote{незалежними}, проте ще й досі не можемо цією незалежністю вдосталь
скористатися.  У наш час \emph{держава Україна} - неоколонія. Тобто, будучи \emph{державним
утворенням}, вона не є \emph{державою} для своєї титульної нації, а навпаки - всіма
можливими методами її поневолює та знищує, відстоюючи інтереси закордонних, як
відверто антиукраїнських сил, так і втілювачів ідеології ліберального
тоталітаризму у світовій політиці. А корінні українці тим часом і надалі
залишаються безправними на своїй землі. Жодні проблеми українців не вирішуються
на їх користь. Український народ загнали у ярмо, де його поступово і впевнено
знищують!
%%%cit_comment
%%%cit_title
\citTitle{Сучасний стан української нації та держави Україна}, 
, pravyysektor.info, 25.05.2017
%%%endcit

%%%cit
%%%cit_head
%%%cit_pic
%%%cit_text
Головна проблема українських бід полягає у \emph{бездержавності} української нації. У
тому, що у нас немає \emph{держави}, яка б забезпечувала українцям повноцінне
існування та розвиток, була центром об'єднання українців в Україні та за
рубежем. А те \emph{державне утворення}, що у нас наявне, українським по своїй суті не
є, оскільки в ньому забезпечуються інтереси будь-кого: чужоземців, педерастів,
сепаратистів, можновладців, олігархів, тобто всіх, окрім простих українців.
Представники титульної нації, яких у \emph{державі} 77\% (щонайменше), залишаються
безправними та беззахисними, неспроможними керувати своєю долею на своїй землі.
У цій \emph{державі} будь-яка українська націоналістична активність присікається, а
все, що є антиукраїнським, вітається чи мовчки допускається. Так було за кожної
влади в Україні, не був винятком навіть В. Ющенко, який вважався
\enquote{проукраїнським президентом} та на якого так покладали надії українці. Наразі
ми чітко бачимо, як лещата режиму внутрішньої окупації продовжують боротьбу на
знищення, допустивши війну з москалем, яку вигравати, очевидно, не збираються.
Створюється хіба вид лицемірної патріотичності та стурбованості, а паралельно
гине кращий генофонд нації у \enquote{перемир'ях} та у в'язницях
%%%cit_comment
%%%cit_title
\citTitle{Сучасний стан української нації та держави Україна}, 
, pravyysektor.info, 25.05.2017
%%%endcit
