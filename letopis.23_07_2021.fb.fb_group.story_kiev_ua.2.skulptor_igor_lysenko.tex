% vim: keymap=russian-jcukenwin
%%beginhead 
 
%%file 23_07_2021.fb.fb_group.story_kiev_ua.2.skulptor_igor_lysenko
%%parent 23_07_2021
 
%%url https://www.facebook.com/groups/story.kiev.ua/posts/1713011168895677
 
%%author_id fb_group.story_kiev_ua,majorenko_georgij.kiev
%%date 
 
%%tags kiev,kievljane
%%title Скульптор Игорь Лысенко
 
%%endhead 
 
\subsection{Скульптор Игорь Лысенко}
\label{sec:23_07_2021.fb.fb_group.story_kiev_ua.2.skulptor_igor_lysenko}
 
\Purl{https://www.facebook.com/groups/story.kiev.ua/posts/1713011168895677}
\ifcmt
 author_begin
   author_id fb_group.story_kiev_ua,majorenko_georgij.kiev
 author_end
\fi

Хочу сообщить печальную весть. Ушел из жизни представитель одной из самых
уважаемых киевских семей - замечательный скульптор Игорь Лысенко - сын
известнейшего киевоведа и экскурсовода Элеоноры Натановны Рахлиной и внук
выдающегося дерижера Натана Рахлина.

\raggedcolumns
\begin{multicols}{3} % {
\setlength{\parindent}{0pt}

% Светлая память Игорю
\ii{23_07_2021.fb.fb_group.story_kiev_ua.2.skulptor_igor_lysenko.pic.1}
\ii{23_07_2021.fb.fb_group.story_kiev_ua.2.skulptor_igor_lysenko.pic.1.cmt}

% Элеонора Натановна Рахлина...
\ii{23_07_2021.fb.fb_group.story_kiev_ua.2.skulptor_igor_lysenko.pic.2}
\ii{23_07_2021.fb.fb_group.story_kiev_ua.2.skulptor_igor_lysenko.pic.2.cmt}

% Грампластинка
\ii{23_07_2021.fb.fb_group.story_kiev_ua.2.skulptor_igor_lysenko.pic.3}

% Великий дед!
\ii{23_07_2021.fb.fb_group.story_kiev_ua.2.skulptor_igor_lysenko.pic.4}
\ii{23_07_2021.fb.fb_group.story_kiev_ua.2.skulptor_igor_lysenko.pic.4.cmt}

% figure 3

\end{multicols} % }

Мне довелось быть одноклассником Игоря, бывать у него в гостях, на Днях
рождения. О нем, о его маме остались самые теплые и добрые воспоминания. И от
этого утрата еще больнее.

Игорь Лысенко был представителем той легендарной киевской интеллигенции,
проживавшей в старинных кварталах Города. Традиции этих людей передавались из
поколения в поколение. Как и язык, песни, любовь к книгам, интерес к родной
истории. Такие киевляне понимают друг друга с полуслова. И, конечно же, Игорь
очень преживал, что того, знакомого с детства, старого доброго интеллигентного
Киева ставновится все меньше и мньше и ему на смену катит лавиной что-то
злобное, хамовитое, недоброе и чужое. Я думаю, многие поймут, о чем идет речь.

И вот, Игорь нас покинул.

Пусть же душа его покоится с миром, а мы сохраним добрую память о нем, как о
добром, талантливом и мудром человеке.

И КИЕВЛЯНИНЕ с большой буквы..
