% vim: keymap=russian-jcukenwin
%%beginhead 
 
%%file topics.vojna.my.7.matrica.vragi.teksty
%%parent topics.vojna.my.7.matrica.vragi
 
%%url 
 
%%author_id 
%%date 
 
%%tags 
%%title 
 
%%endhead 

\paragraph{12:07:27 21-08-22 Сила V правде Zа Победу 💚💕}
%img Screenshot from 2022-08-21 12-06-23.png

Украинцы с начала лета массово выставляют на продажу земельные участки,
сообщают украниские Телеграм-каналы.

Как отмечают в специализированных агентствах, с каждым месяцем число
предложений только растет. Особенно активно увеличивается количество объявлений
о продаже участков в Сумской, Черниговской, Житомирской, Киевской областях, а
также вблизи линии зоны боевых действий в Запорожской, Харьковской,
Николаевской областях.

Такая тенденция объясняется отсутствием средств у населения, а также кризисом
аграрной отрасли и слухами о том, что осеннюю посевную многие хозяйства
попросту не будут проводить.

При этом указывается, что найти покупателя на землю сейчас не так просто, а
потенциальные инвесторы, если и готовы приобретать паи, то только в отдельных
регионах (Полтавская, Кировоградская, Винницкая, Львовская области) и со
скидками.

Это уже обвалило ценники на рынке земли, и сейчас минимальную планку
ограничивает только законодательное требование, по которому продажная стоимость
земли не должна быть ниже нормативно-денежной оценки. Но появляются схемы по
обходу этого требования, и в итоге участки скупают по минимальным расценкам.
Источник
