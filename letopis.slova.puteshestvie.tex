% vim: keymap=russian-jcukenwin
%%beginhead 
 
%%file slova.puteshestvie
%%parent slova
 
%%url 
 
%%author 
%%author_id 
%%author_url 
 
%%tags 
%%title 
 
%%endhead 
\chapter{Путешествие}

%%%cit
%%%cit_head
%%%cit_pic
\ifcmt
	width 0.4
  pic https://cdn1.img.ukraina.ru/images/103178/19/1031781941.jpg
	caption Игорь Тамм. Нобелевский лауреат из Елизаветграда
\fi
%%%cit_text
8 июля 1895 года во Владивостоке в семье инженера немецкого происхождения
родился выдающийся советский физик, лауреат Нобелевской и двух Сталинских
премий, Герой социалистического труда Игорь Евгеньевич Тамм. В трёхлетнем
возрасте родители перевезли его в Елисаветград, где и сформировался этот
незаурядный учёный.  «…глядел в лицо цивилизации».  До того, как Транссибирская
магистраль была пущена в строй кратчайший путь из европейской части России на
Дальний восток и обратно пролегал через Одессу. Так и инженер Тамм, чтобы
оказаться в уездном городе Херсонской губернии Елизаветграде, совершил
\emph{путешествие} из одного конца России в другой морем через Японию, Индию,
Суэцкий канал и Константинополь в Одессу. А там было уже близко
%%%cit_comment
%%%cit_title
\citTitle{Игорь Тамм. Нобелевский лауреат из Елизаветграда}, Дмитрий Губин, ukraina.ru, 08.07.2021
%%%endcit

