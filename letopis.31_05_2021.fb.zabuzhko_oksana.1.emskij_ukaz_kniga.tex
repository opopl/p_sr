% vim: keymap=russian-jcukenwin
%%beginhead 
 
%%file 31_05_2021.fb.zabuzhko_oksana.1.emskij_ukaz_kniga
%%parent 31_05_2021
 
%%url https://www.facebook.com/oksana.zabuzhko/posts/10159375840713953
 
%%author Забужко, Оксана
%%author_id zabuzhko_oksana
%%author_url 
 
%%tags 
%%title Емський Указ - Річниця - Книжка
 
%%endhead 
 
\subsection{Емський Указ - Річниця - Книжка}
\label{sec:31_05_2021.fb.zabuzhko_oksana.1.emskij_ukaz_kniga}
\Purl{https://www.facebook.com/oksana.zabuzhko/posts/10159375840713953}
\ifcmt
 author_begin
   author_id zabuzhko_oksana
 author_end
\fi

Прочитала чудовий читацький відгук на \enquote{Довгу дорогу з Бад-Емса} - і розчулилась
мало не до сліз (дякую!). Тут ще й те додалося, що якраз учора була річниця:
145 років, відколи в Бад-Емсі Олександр ІІ підписав нам свій (як гадав)
смертний вирок, загнавши Україну в глухий кут безальтернативного (як здавалось)
вибору: або модернізація і \enquote{адіннарод} - або гетьте, малороси, на дно історії,
в \enquote{біологічний резерв}! 

За той горезвісний \enquote{Емський указ} ми заплатили поразкою у 1918-20-му, по тому
указу нас прийшли убивати з \enquote{нової-старої} Росії в 2014-му (бо, як пояснюють по
тамтешньому ТБ, \enquote{украинцы - это те же русские, которые об этом слегка
подзабыли, так надо им напомнить} (с)) - 145 років триває ця ОФІЦІЙНО ОГОЛОШЕНА
ВІЙНА, але от як ми зуміли тоді, в XIX-му столітті, попри (здавалось)
цілковитий брак ресурсу, створити тій убивчій махині організований опір
практично по всіх пунктах - це було й лишається одною з найзахопливіших
таємниць світової історії двох останніх століть, і в \enquote{Довгій дорозі...} я
називаю це - \enquote{українським чудом}, соррі за довгу автоцитату:

**********

"Розбещені (саме так!) апокаліптичними мартирологами 20-го століття, ми
втратили чутливість до «безкровних» репресій – до всіх інших форм винищення
еліт (елітоциду), крім кулі в потилицю (ну гаразд, іще скидання під поїзд на
повнім ходу, з наступним оголошенням, що це було «самогубство»: тут почерк
російських спецслужб від охранки до ФСБ не змінився), – і вже погано собі
уявляємо, як БАГАТО насправді потрібно культурі умов і догляду (куди більше,
ніж вишневому саду!) для повноцінного, нездеформованого розвитку, – і як легко
цей «сад» коли не вирубати сокирою, то занапастити на корню, щоб не дав ні
цвіту, ні плоду. Через цю викривлену оптику, замість навчати школярів на
прикладі української літератури, як їхні предки в 19-му столітті зуміли зробити
неможливе можливим (назвімо це, для зручности, «українським чудом» – успішний
культурний резистанс в умовах елітоциду, чи багато народів у новітній історії
можуть таким похвалитися?!), – ми нарікаємо, що наша класика для наших дітей
«надто депресивна» (вже ж, не «Вечерний квартал», але, як подумати: якби вона
справді здавалась сучасникам «депресивна», то кого б же вона пірвала за собою
«в українці» в час, коли такий вибір грозив Сибіром? може ж, таки не в класиці
справа, а в невмінні її прочитати?..), – а потім збиваємося з ніг у пошуках
тренінґів, «як мотивувати молодь». Я не жартую: недавно мене цілком поважно
просили взяти участь у такому тренінґу, і ці нотатки –  не тільки моя данина
світлій пам’яті Оксани Радиш, яка передала мені живий «інсайдерський» смак
нашого 19-го століття й остаточно зробила те століття для мене «своїм», а й
спроба нагадати, що мотивують молодь (якщо вона дійсно потребує мотивації, що
само собою підозріло: як писав Ю. Шевельов, «чого варта молодь, яка не хоче
змінити світ?») – не тренінґами, не онлайн-флешмобами й не перекладами книжок
Марка, прости-Боже, Менсона – а За́ХВАТОМ. Щирим за́хватом людьми, які «були тут
до тебе» – і створили й залишили тобі те, з чого ти вжиткуєш.

\ifcmt
tab_begin cols=2

  pic https://scontent-bos3-1.xx.fbcdn.net/v/t1.6435-0/s600x600/188047843_4168835813139874_2707422694101800270_n.jpg?_nc_cat=109&ccb=1-3&_nc_sid=8bfeb9&_nc_ohc=DKw_SeHCPE8AX-6at49&_nc_ht=scontent-bos3-1.xx&tp=7&oh=5591d06eb2b03339c63bf41f32215d9c&oe=60DAADC2

	pic https://scontent-bos3-1.xx.fbcdn.net/v/t1.6435-9/187399685_4168901393133316_3624347893210377287_n.jpg?_nc_cat=109&ccb=1-3&_nc_sid=8bfeb9&_nc_ohc=Ep6I6JaqAGEAX9EpVIV&_nc_ht=scontent-bos3-1.xx&oh=613fd155da1e3c3a01cf6ee2e43d7036&oe=60DB04BE

tab_end
\fi

Це може бути твоя галузь. Професія. Культура. Країна. Армія. Мова. І так далі.

І в кого як у кого, а в українців за плечима стільки приводів для за́хвату, що
кожному випускникові середньої школи мало б на весь вік вистачити на підвищений
рівень серотоніну без жодних стимуляторів. 

Якщо ж у молоді намічаються проблеми з серотоніном – це вже певний знак, що
пора переписувати шкільні програми". 

(\enquote{Як рубали вишневий сад, або Довга дорога з Бад-Емса}, книжка тут:
https://komorabooks.com/product/yak-rubaly-vyshnevyj-sad-abo-dovga-doroga-z-bad-emsu/
)

\url{https://www.facebook.com/permalink.php?story_fbid=4168935166463272&id=100000404906084}
