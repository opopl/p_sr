% vim: keymap=russian-jcukenwin
%%beginhead 
 
%%file 25_07_2021.fb.miroslava_iltjo.1.mova_zakon_jazyk_rodnoj.cmt
%%parent 25_07_2021.fb.miroslava_iltjo.1.mova_zakon_jazyk_rodnoj
 
%%url 
 
%%author 
%%author_id 
%%author_url 
 
%%tags 
%%title 
 
%%endhead 
\subsubsection{Коментарі}

\begin{itemize}
%%%fbauth
%%%fbauth_name
\iusr{Инна Гуриненко}
%%%fbauth_url
%%%fbauth_place
%%%fbauth_id
%%%fbauth_front
%%%fbauth_desc
%%%fbauth_www
%%%fbauth_pic
%%%fbauth_pic portrait
%%%fbauth_pic background
%%%fbauth_pic other
%%%fbauth_tags
%%%fbauth_pubs
%%%endfbauth
 

Мене дивують українці , які приїхавши до міста почитають говорити російською.
Тоді мені як людині з професійною освітою філолога-славіста аж вуха пухнуть ,
несила чути оті всі «авторські» інваріанти фонем і решту . Чому не навчитися
правильно говорити , а тоді вже практикувати , якщо такий потяг до мови

\begin{itemize}
%%%fbauth
%%%fbauth_name
\iusr{Мирослава Ільтьо}
%%%fbauth_url
%%%fbauth_place
%%%fbauth_id
%%%fbauth_front
%%%fbauth_desc
%%%fbauth_www
%%%fbauth_pic
%%%fbauth_pic portrait
%%%fbauth_pic background
%%%fbauth_pic other
%%%fbauth_tags
%%%fbauth_pubs
%%%endfbauth
 
\textbf{Inna Gurinenko} ну от мене російськомовні з роду ніколи не дратують, а тих, хто переходить, то постріляла би.

%%%fbauth
%%%fbauth_name
\iusr{Мирослава Ільтьо}
%%%fbauth_url
%%%fbauth_place
%%%fbauth_id
%%%fbauth_front
%%%fbauth_desc
%%%fbauth_www
%%%fbauth_pic
%%%fbauth_pic portrait
%%%fbauth_pic background
%%%fbauth_pic other
%%%fbauth_tags
%%%fbauth_pubs
%%%endfbauth
 
\textbf{Inna Gurinenko} у мене сестра, бляха, російською щебече за нагоди. Чи ж
не на одну пашу ми корів гонили, бляха.

%%%fbauth
%%%fbauth_name
\iusr{Инна Гуриненко}
%%%fbauth_url
%%%fbauth_place
%%%fbauth_id
%%%fbauth_front
%%%fbauth_desc
%%%fbauth_www
%%%fbauth_pic
%%%fbauth_pic portrait
%%%fbauth_pic background
%%%fbauth_pic other
%%%fbauth_tags
%%%fbauth_pubs
%%%endfbauth
 
\textbf{Мирослава Ільтьо} так само , справжня літературна російська - ніяк не
дратує , але ж людей , які нею правильно говорить одиниці

%%%fbauth
%%%fbauth_name
\iusr{Natalie Bortsova}
%%%fbauth_url
%%%fbauth_place
%%%fbauth_id
%%%fbauth_front
%%%fbauth_desc
%%%fbauth_www
%%%fbauth_pic
%%%fbauth_pic portrait
%%%fbauth_pic background
%%%fbauth_pic other
%%%fbauth_tags
%%%fbauth_pubs
%%%endfbauth
 
\textbf{Инна Гуриненко} це просто була схема вижити. Колись. Принаймні, якщо
вдавалося видертися з якогось голодного колгоспу. І потім "загубитися" у тому ж
Києві. Скоріш за все, зараз працює на рівні звички та генетичної пам'яті - роби
так, якщо хочеш нормально жити. Такий блок в голові. На жаль. З часом пройде..

%%%fbauth
%%%fbauth_name
\iusr{Инна Гуриненко}
%%%fbauth_url
%%%fbauth_place
%%%fbauth_id
%%%fbauth_front
%%%fbauth_desc
%%%fbauth_www
%%%fbauth_pic
%%%fbauth_pic portrait
%%%fbauth_pic background
%%%fbauth_pic other
%%%fbauth_tags
%%%fbauth_pubs
%%%endfbauth
 
\textbf{Natalie Bortsova} можливо, і дійсно навчання і робота мали
послуговуватися офіційною мовою , але - 30 років як це не актуально і як щодо
людей в тих же селах, які можуть писати підписи до постів чи повідомлення не
своєю мовою ?

%%%fbauth
%%%fbauth_name
\iusr{Natalie Bortsova}
%%%fbauth_url
%%%fbauth_place
%%%fbauth_id
%%%fbauth_front
%%%fbauth_desc
%%%fbauth_www
%%%fbauth_pic
%%%fbauth_pic portrait
%%%fbauth_pic background
%%%fbauth_pic other
%%%fbauth_tags
%%%fbauth_pubs
%%%endfbauth
 
\textbf{Инна Гуриненко} боюся, шо то історія не на 30 років, а дай боже на 130.
Але..але! Все одно, рух відчувається і дуже серьезний. Я добре пам'ятаю
абсолютно російськомовний Київ. Шалений спротив квотам на українську у ЗМІ у
дев'яностих) Зараз, це просто неможливо порівнювати! Це шалені зміни і з
історичної точки зору, дуже швидкі. Нічого..тут головне відходити від російкого
культурного простору. Чим більше буде ця прірва, чим більше буде нашого та
західного, тим легше піде)

\end{itemize}

%%%fbauth
%%%fbauth_name
\iusr{Сергій Смальчук}
%%%fbauth_url
%%%fbauth_place
%%%fbauth_id
%%%fbauth_front
%%%fbauth_desc
%%%fbauth_www
%%%fbauth_pic
%%%fbauth_pic portrait
%%%fbauth_pic background
%%%fbauth_pic other
%%%fbauth_tags
%%%fbauth_pubs
%%%endfbauth
 
Меншовартісні манкурти...

\begin{itemize}
%%%fbauth
%%%fbauth_name
\iusr{Мирослава Ільтьо}
%%%fbauth_url
%%%fbauth_place
%%%fbauth_id
%%%fbauth_front
%%%fbauth_desc
%%%fbauth_www
%%%fbauth_pic
%%%fbauth_pic portrait
%%%fbauth_pic background
%%%fbauth_pic other
%%%fbauth_tags
%%%fbauth_pubs
%%%endfbauth
 
\textbf{Сергій Смальчук} я багато кому розповідала історію про те, як моя
двоюрідна сестра в Києві говорила з подругою російською, а та подруга прекрасно
мене розуміла українською. І ось ця російська в її виконанні утвердила в мені
глибоке переконання, що я за жодних обставин так не принижуватиму всіх тих, хто
зігнив за мову в сталінських таборах. Чогось мені завжди згадувалися ці всі
Панаси, Марії і Аркадії, про яких я малою читала.
\end{itemize}

%%%fbauth
%%%fbauth_name
\iusr{Iryna Vorobets}
%%%fbauth_url
%%%fbauth_place
%%%fbauth_id
%%%fbauth_front
%%%fbauth_desc
%%%fbauth_www
%%%fbauth_pic
%%%fbauth_pic portrait
%%%fbauth_pic background
%%%fbauth_pic other
%%%fbauth_tags
%%%fbauth_pubs
%%%endfbauth
 
Миросю, дякую за цей пост! 💙💛

%%%fbauth
%%%fbauth_name
\iusr{Salko Ann}
%%%fbauth_url
%%%fbauth_place
%%%fbauth_id
%%%fbauth_front
%%%fbauth_desc
%%%fbauth_www
%%%fbauth_pic
%%%fbauth_pic portrait
%%%fbauth_pic background
%%%fbauth_pic other
%%%fbauth_tags
%%%fbauth_pubs
%%%endfbauth
 

Дякую за цей пост! Багато людей в Києві пробують повертатися до української, і
хто їм найчастіше "ламає крила"? Правильно, совєцька школа, де "главноє, чтоб
понимали, какая разница". Декілька людей з мого оточення так перепитувати у
мене - "а можна недосконалою українською говорити?"


%%%fbauth
%%%fbauth_name
\iusr{Andriy Kovalyov}
%%%fbauth_url
%%%fbauth_place
%%%fbauth_id
%%%fbauth_front
%%%fbauth_desc
%%%fbauth_www
%%%fbauth_pic
%%%fbauth_pic portrait
%%%fbauth_pic background
%%%fbauth_pic other
%%%fbauth_tags
%%%fbauth_pubs
%%%endfbauth
 

Цікаво, як співрозмовники реагують на такі лекції? Коли я починав щось подібне
розповідати, то завжди відчувалась певна роздратованість у співрозмовника, наче
йому щось нав’язують. Це цікаве питання для психологів, тому що частково це як
Стокгольмський синдром - ти наче і розумієш що колись над твоїми предками
відбулось насилля, але ж ти вже за це не відповідаєш був народжений з «родним
язиком».

\begin{itemize}
%%%fbauth
%%%fbauth_name
\iusr{Мирослава Ільтьо}
%%%fbauth_url
%%%fbauth_place
%%%fbauth_id
%%%fbauth_front
%%%fbauth_desc
%%%fbauth_www
%%%fbauth_pic
%%%fbauth_pic portrait
%%%fbauth_pic background
%%%fbauth_pic other
%%%fbauth_tags
%%%fbauth_pubs
%%%endfbauth
 
\textbf{Andriy Kovalyov} ну я ж категорична в будь-яких насильницьких методах,
а все роблю м'яко і з гумором здебільшого. ( Тут ще треба чути мій тон і бачити
вираз обличчя 😆)

Ну я бішу трохи людей, але що поробиш)

%%%fbauth
%%%fbauth_name
\iusr{Andriy Kovalyov}
%%%fbauth_url
%%%fbauth_place
%%%fbauth_id
%%%fbauth_front
%%%fbauth_desc
%%%fbauth_www
%%%fbauth_pic
%%%fbauth_pic portrait
%%%fbauth_pic background
%%%fbauth_pic other
%%%fbauth_tags
%%%fbauth_pubs
%%%endfbauth
 

Так від манери та тону «лекції» дуже багато залежить. Я коли гуляю з сином (3,5
роки) якщо хтось звертається до нього російською, просто кажу «Говоріть з ним
українською» - завжди спрацьовувало (в Києві, в інших містах не пробували ще \Smiley[1.0][yellow]
)

\end{itemize}

%%%fbauth
%%%fbauth_name
\iusr{Andriy Magera}
%%%fbauth_url
%%%fbauth_place
%%%fbauth_id
%%%fbauth_front
%%%fbauth_desc
%%%fbauth_www
%%%fbauth_pic
%%%fbauth_pic portrait
%%%fbauth_pic background
%%%fbauth_pic other
%%%fbauth_tags
%%%fbauth_pubs
%%%endfbauth
 

Де це так Вам відповідали?

\begin{itemize}
%%%fbauth
%%%fbauth_name
\iusr{Мирослава Ільтьо}
%%%fbauth_url
%%%fbauth_place
%%%fbauth_id
%%%fbauth_front
%%%fbauth_desc
%%%fbauth_www
%%%fbauth_pic
%%%fbauth_pic portrait
%%%fbauth_pic background
%%%fbauth_pic other
%%%fbauth_tags
%%%fbauth_pubs
%%%endfbauth
 
\textbf{Андрій Магера} у Чорноморську. А одна тітка доволі старша сказала, що ми не схожі на рускіх 😆

%%%fbauth
%%%fbauth_name
\iusr{Andriy Magera}
%%%fbauth_url
%%%fbauth_place
%%%fbauth_id
%%%fbauth_front
%%%fbauth_desc
%%%fbauth_www
%%%fbauth_pic
%%%fbauth_pic portrait
%%%fbauth_pic background
%%%fbauth_pic other
%%%fbauth_tags
%%%fbauth_pubs
%%%endfbauth
 

\textbf{Мирослава Ільтьо} на ка\#апствующих точно не схожі.
\end{itemize}

%%%fbauth
%%%fbauth_name
\iusr{Natalia Tsikra}
%%%fbauth_url
%%%fbauth_place
%%%fbauth_id
%%%fbauth_front
%%%fbauth_desc
%%%fbauth_www
%%%fbauth_pic
%%%fbauth_pic portrait
%%%fbauth_pic background
%%%fbauth_pic other
%%%fbauth_tags
%%%fbauth_pubs
%%%endfbauth
 

Дякую за допис, Миросю! А мене найбільше дратує, коли мої австрійські учні,
дипломати, які лише для чотирирічної каденції в Україні вивчають українську
мову, приїздять до Києва і на їхнє питання чи прохання українською їм
відповідають російською. І коли вони просять розмовляти з ними українською, бо
російської вони не знають, то їм не вірять. 🙁


%%%fbauth
%%%fbauth_name
\iusr{Любов Бурак}
%%%fbauth_url
%%%fbauth_place
%%%fbauth_id
%%%fbauth_front
%%%fbauth_desc
%%%fbauth_www
%%%fbauth_pic
%%%fbauth_pic portrait
%%%fbauth_pic background
%%%fbauth_pic other
%%%fbauth_tags
%%%fbauth_pubs
%%%endfbauth
 


На Волині було зросійщення, але воно відбувалося в основному через церкву.
Через те, що на Волині часто змінювались окупанти, наш край був найменш
освіченим, проте: найбільш стійким до впливу русифікації та полонізації. У всіх
витках визвольних змагань Волинь трималася найдовше. Доки були ліси, а окупанти
не мали високотехнологічної зброї. На Волині найважче приживалася пропаганда і
найдовше протрималася інституція родини. За обрусіння в наших краях били
боляче. Але... Російська попса і моспархат тихою сапою деградували людей. Та й
таке.


%%%fbauth
%%%fbauth_name
\iusr{Oksana Tymenko}
%%%fbauth_url
%%%fbauth_place
%%%fbauth_id
%%%fbauth_front
%%%fbauth_desc
%%%fbauth_www
%%%fbauth_pic
%%%fbauth_pic portrait
%%%fbauth_pic background
%%%fbauth_pic other
%%%fbauth_tags
%%%fbauth_pubs
%%%endfbauth
 
Дякую за пост!

%%%fbauth
%%%fbauth_name
\iusr{Олена Ревенко}
%%%fbauth_url
%%%fbauth_place
%%%fbauth_id
%%%fbauth_front
%%%fbauth_desc
%%%fbauth_www
%%%fbauth_pic
%%%fbauth_pic portrait
%%%fbauth_pic background
%%%fbauth_pic other
%%%fbauth_tags
%%%fbauth_pubs
%%%endfbauth
 

Цілком може бути рідною. В Україні чимало етнічних руських - зокрема через
сталінські переселення.

Я лише на 1/4 українка - та й та чверть з дуже східного регіону, на кордоні з
Росією.

У мене в родині ніхто не говорив українською.

Тож я не бачу нічого дивного в тому, щоб називати російську/руську мову своєю
рідною.

\begin{itemize}
%%%fbauth
%%%fbauth_name
\iusr{Мирослава Ільтьо}
%%%fbauth_url
%%%fbauth_place
%%%fbauth_id
%%%fbauth_front
%%%fbauth_desc
%%%fbauth_www
%%%fbauth_pic
%%%fbauth_pic portrait
%%%fbauth_pic background
%%%fbauth_pic other
%%%fbauth_tags
%%%fbauth_pubs
%%%endfbauth
 
\textbf{Олена Ревенко} ваша думка слушна, але і я не запевняю в тексті, що не
можна. Можна, але поряд з тим те, що я описала, незаперечні факти і в більшості
випадків йдеться про українців, а не етнічних росіян.

%%%fbauth
%%%fbauth_name
\iusr{Bohdana Stelmakh}
%%%fbauth_url
%%%fbauth_place
%%%fbauth_id
%%%fbauth_front
%%%fbauth_desc
%%%fbauth_www
%%%fbauth_pic
%%%fbauth_pic portrait
%%%fbauth_pic background
%%%fbauth_pic other
%%%fbauth_tags
%%%fbauth_pubs
%%%endfbauth
 
Тільки не руських, а русских, бо Русь - то не Московія. Але співзвучність
внаслідок крадіжки робить свою злу справу

%%%fbauth
%%%fbauth_name
\iusr{Мыкола Семена}
%%%fbauth_url
%%%fbauth_place
%%%fbauth_id
%%%fbauth_front
%%%fbauth_desc
%%%fbauth_www
%%%fbauth_pic
%%%fbauth_pic portrait
%%%fbauth_pic background
%%%fbauth_pic other
%%%fbauth_tags
%%%fbauth_pubs
%%%endfbauth
 
\textbf{Олена Ревенко} руська мова це українська мова (Іван Франко: як люблю я
тую руську мову", бо це від слова Русь), а російська - то російська

\end{itemize}

%%%fbauth
%%%fbauth_name
\iusr{Olena Bastun}
%%%fbauth_url
%%%fbauth_place
%%%fbauth_id
%%%fbauth_front
%%%fbauth_desc
%%%fbauth_www
%%%fbauth_pic
%%%fbauth_pic portrait
%%%fbauth_pic background
%%%fbauth_pic other
%%%fbauth_tags
%%%fbauth_pubs
%%%endfbauth
 

Виключно справедливості заради: в СРСР не було державної мови (!!!). Єдина
радянська республіка, яка у 80-х роках заявила про власну державну мову -
Грузія. Грузинську об’явила державною. В СРСР було поняття «язык
межнационального общения». А державної не було... такий ось парадокс.

І ще одне. Можу помилятися, але Петро другий, здається, помер геть малим. Не
можу зараз перевірити.

За пост подяка!

\begin{itemize}
%%%fbauth
%%%fbauth_name
\iusr{Мыкола Семена}
%%%fbauth_url
%%%fbauth_place
%%%fbauth_id
%%%fbauth_front
%%%fbauth_desc
%%%fbauth_www
%%%fbauth_pic
%%%fbauth_pic portrait
%%%fbauth_pic background
%%%fbauth_pic other
%%%fbauth_tags
%%%fbauth_pubs
%%%endfbauth
 
\textbf{Olena Bastun} в республіках були державні мови, і мови міжнацірнального спілкування

%%%fbauth
%%%fbauth_name
\iusr{Olena Bastun}
%%%fbauth_url
%%%fbauth_place
%%%fbauth_id
%%%fbauth_front
%%%fbauth_desc
%%%fbauth_www
%%%fbauth_pic
%%%fbauth_pic portrait
%%%fbauth_pic background
%%%fbauth_pic other
%%%fbauth_tags
%%%fbauth_pubs
%%%endfbauth
 

Не сперечатимусь. Бо пруфи наразі не зможу знайти. Але дуже добре пам‘ятаю,
який галас здійнявся після заяви Грузії. Тоді і відкрилося, шо СРСР юридично не
має державної мови. А як могли мати державні мови республіки, які юридично не
були державами?


%%%fbauth
%%%fbauth_name
\iusr{Мыкола Семена}
%%%fbauth_url
%%%fbauth_place
%%%fbauth_id
%%%fbauth_front
%%%fbauth_desc
%%%fbauth_www
%%%fbauth_pic
%%%fbauth_pic portrait
%%%fbauth_pic background
%%%fbauth_pic other
%%%fbauth_tags
%%%fbauth_pubs
%%%endfbauth
 

Республіки саме юридично були державами, бо це проголошували їх конституції,
але вони не були державами фактично, бо не мали атрибутів повної державності.
Однак державні мови вони мали поряд з мовою міжнаціонального спілкування...
\end{itemize}

% -------------------------------------
\ii{fbauth.grisha_kirill.kiev.ukraina.krivoj_rog}
% -------------------------------------

Ну таке: за часів СРСР таке Велике переселення відбулося, що в більшості сімей
намішано крові з усяких всюд. Тому найчастіш не доводиться говорити про етнічну
приналежність (шо ось ми, українці, а говоримо русською).

І тому отако просто ярлика навісить не не вийде...

Яким би не було бажання: відкалібрувати людину манкуртом, рагулєм, запроданцем
чи несвідомим.

Я оце собі лише вчора про себе самого думав, що це якесь викривлення свідомості
- визнавати свою українську кров, і забути (або, ще краще, витравити) кров
руську...

А ось іще каламбур: я полюбив українську тоді як померла моя перша вчителька
укр.мови (вона була дочкою політв'язнів і народилась десь в Сібіру)

Тому, отримавши нас в 1992 році, почала нелагідну українізацію, від чого ми
підох...ївали).

І коли Любов Іванівна померла (Царство їй Небесне), наступні вчителі донесли
Слово як Любов, як Красу, як Мелодію.

Тому, звісно, українська - рідна й ніжна. Але і російська - так само мова
пращурів і заслуговує шани не меншої.

\begin{itemize}
%%%fbauth
%%%fbauth_name
\iusr{Мыкола Семена}
%%%fbauth_url
%%%fbauth_place
%%%fbauth_id
%%%fbauth_front
%%%fbauth_desc
%%%fbauth_www
%%%fbauth_pic
%%%fbauth_pic portrait
%%%fbauth_pic background
%%%fbauth_pic other
%%%fbauth_tags
%%%fbauth_pubs
%%%endfbauth
 
\textbf{Кирилл Гриша} визнання національності по крові - це фашизм.
Нацаональність визначається по самоусвідомленню. Хто сам усвідомлює себе
українцем, той і є українець, не дивлячись, скільки якої крові в ньому
намішано, бо кров до свідомості не має ніякого відношення. Людське Я може бути
усвіломлене українським, російським, китайським, білоруським, якутським,
єврейським, чеченським тощо. Кров же вашого пра-пра-пра дідуся, якого ви ніколи
не бачили, не впливає на ваше усвідомлення себе тим чи іншим...

%%%fbauth
%%%fbauth_name
\iusr{Кирилл Гриша}
%%%fbauth_url
%%%fbauth_place
%%%fbauth_id
%%%fbauth_front
%%%fbauth_desc
%%%fbauth_www
%%%fbauth_pic
%%%fbauth_pic portrait
%%%fbauth_pic background
%%%fbauth_pic other
%%%fbauth_tags
%%%fbauth_pubs
%%%endfbauth
 
\textbf{Мыкола Семена} 
а як бачив? І навіть спілкувався з ним (а він ще за Колчака воював))))?

Ви про культурний код?

Так бува таке, шо людина мультикультурна (я, наприклад, не чураюся кращого що є
в мені, по за всякими отими)))

%%%fbauth
%%%fbauth_name
\iusr{Мыкола Семена}
%%%fbauth_url
%%%fbauth_place
%%%fbauth_id
%%%fbauth_front
%%%fbauth_desc
%%%fbauth_www
%%%fbauth_pic
%%%fbauth_pic portrait
%%%fbauth_pic background
%%%fbauth_pic other
%%%fbauth_tags
%%%fbauth_pubs
%%%endfbauth
 
\textbf{Кирилл Гриша} 

Ну це і значить, що в вас САМЕ ТАКЕ самоусвідомлення. Якщо вас це влаштовує, то
так і бути. Але якщо розмова йде про мову, то ви маєте визначитись на якійсь
одній, бо ви ж не можете говорити кількома мовами одночасно, ви можете говорити
чи тією, чи іншою, в одному випадку однією, в іншому іншою, але кожного разу
якоюсь однією. Це і є ваш код. То з ним і живіть, але у вас нема права за це
предїявити кому небудь претензії чи вимоги. Це ваше рішення, за нього ви
відповідаєте, і до інших не повинно бути претензій.


%%%fbauth
%%%fbauth_name
\iusr{Кирилл Гриша}
%%%fbauth_url
%%%fbauth_place
%%%fbauth_id
%%%fbauth_front
%%%fbauth_desc
%%%fbauth_www
%%%fbauth_pic
%%%fbauth_pic portrait
%%%fbauth_pic background
%%%fbauth_pic other
%%%fbauth_tags
%%%fbauth_pubs
%%%endfbauth
 
\textbf{Мыкола Семена} та які ж претензії...
\end{itemize}

%%%fbauth
%%%fbauth_name
\iusr{Тихолаз Ігор}
%%%fbauth_url
%%%fbauth_place
%%%fbauth_id
%%%fbauth_front
%%%fbauth_desc
%%%fbauth_www
%%%fbauth_pic
%%%fbauth_pic portrait
%%%fbauth_pic background
%%%fbauth_pic other
%%%fbauth_tags
%%%fbauth_pubs
%%%endfbauth
 

Зросійщені волиняки то сумно, еге ж, і пекельно... Я ще й завжди як приклад
згадую тих братів кононовичів, комуняк, і дивуюсь як вони вижили...


%%%fbauth
%%%fbauth_name
\iusr{Мирослава Ільтьо}
%%%fbauth_url
%%%fbauth_place
%%%fbauth_id
%%%fbauth_front
%%%fbauth_desc
%%%fbauth_www
%%%fbauth_pic
%%%fbauth_pic portrait
%%%fbauth_pic background
%%%fbauth_pic other
%%%fbauth_tags
%%%fbauth_pubs
%%%endfbauth
 
\textbf{Натка Кабаровська} здається, ти пропустила це)


\end{itemize}

