% vim: keymap=russian-jcukenwin
%%beginhead 
 
%%file topics.vojna.msg.rus.psycho.intro
%%parent topics.vojna.msg.rus.psycho
 
%%url 
 
%%author_id 
%%date 
 
%%tags 
%%title 
 
%%endhead 

Заметки об информационной войне и пропаганде.

Тут наверное пост не очень обычный, потому что мы здесь не будем приводить
особо фактов из театра военных действий, или же новостей с других фронтов. Речь
пойдет об методологии и подходах информационной войны в Интернете, в частности
в этой социальной сети. Пока что все что мы пишем выглядит наверное очень сыро,
но за то время, что уже было проведено в этой социальной сети в общении с
русней, накоплен определенный фактологический материал, и есть определенные
выводы по поводу рашистов и их психологии, а также как с ними стоит себя
вести.

Итак. Мы - украинцы и все кто против войны и за Украину и ее победу над
нацистской фашистской россии -  здесь много много всего пишем, очень много
пишем. Выставляем последние новости и факты. Тем не менее, вы может быть
заметили, что рашистские тролли упрямо лезут и лезут. Упрямо
все равно лезут. Лезут в чатики под постами, начинают хамить, оскорблять и
лично, и украинцев как нацию, и Украину как государство и страну. Постоянно
лезут, насмехаются, унижают, постят картинки с отвратительными смыслами, и
внешне отвратительными образами, часто с явно порнографическими элементами. Так
что получается так, что кроме того, что на Украину и так постоянно сыпятся
ракеты и бомбы, и что российское руководство явно нацелено на уничтожение
украинцев как нации, а Украины как государства и страны, русский народ, точнее,
та его часть, которая поддерживает политику путина, также обоими руками за
уничтожение Украины и укранцев. Они открыто радуются разрушениям, открыто
радуются, что Украина потеряла часть территорий, открыто радуются этой войне. В
то же время они приносят сюда элементы и сообщения российской пропаганды, как
например, Буча - это фейк, или же что украинская армия сама обстреливает свои
города, в то время как российская армия как будто стреляет только по военным
объектам. В общем, российские тролли все время залазят в чатики под нашими
постами, и все время гадят, и это очень даже заметно. В чем же причина, почему
так получается, что они чувствуют здесь себя вольготно и комфортно, почему так
получается. Многие люди предлагают их блокировать, но по факту (1) это не
всегда получается, как мы понимаем (2) можно заблокировать профиль, но человека
заблокировать нельзя, то есть в принципе человек может всегда сделать новый
профиль и может продолжать распространять гадости. 

Так вот. Обилие троллей в чатиках говорит кое о чем, а именно о том, что в
психологическом плане мы не научились еще как следует давать отпор на месте,
сразу же, в психологическом плане. Ну то есть мы слишком мягко ведем себя с
ними, если взять глобально. 
