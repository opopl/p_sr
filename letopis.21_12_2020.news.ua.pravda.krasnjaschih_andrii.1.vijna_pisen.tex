% vim: keymap=russian-jcukenwin
%%beginhead 
 
%%file 21_12_2020.news.ua.pravda.krasnjaschih_andrii.1.vijna_pisen
%%parent 21_12_2020
 
%%url https://www.pravda.com.ua/articles/2020/12/21/7277633/
 
%%author 
%%author_id krasnjaschih_andrii
%%author_url 
 
%%tags vojna,pesni
%%title Політолог Арве Хансен: Українсько-російська війна пісень — це війна між гумором і цинізмом
 
%%endhead 
 
\subsection{Політолог Арве Хансен: Українсько-російська війна пісень — це війна між гумором і цинізмом}
\label{sec:21_12_2020.news.ua.pravda.krasnjaschih_andrii.1.vijna_pisen}
\Purl{https://www.pravda.com.ua/articles/2020/12/21/7277633/}
\ifcmt
	author_begin
   author_id krasnjaschih_andrii
	author_end
\fi

\ifcmt
pic https://img.pravda.com/images/doc/c/b/cb96fac-arve-hansen.jpg
\fi

\index[rus]{Майдан!Война песен, 21.12.2020}

Сім років тому на київському Майдані можна було зустріти дивного чоловіка, який
виглядав саме так, як зображують вікінгів — світловолосий, блакитноокий,
спокійний. Але не суворий, а навпаки — усмішливий і відкритий.

Це був норвезький вчений з одним із найпоширеніших скандинавських прізвищ —
Хансен.

Шляхом стародавніх предків "із варягів у греки" Арве Хансен в 2006 році вирушив
у тривалу подорож Східною Європою. Але не торгувати, а вивчати і навчатися.

Спочатку навчався у Мінську. Потім працював в генконсульстві Норвегії в
Мурманську. І нарешті у 2013-му опинився у Києві, залишившись тут надовго.

Після Революції Гідності магістерська робота Арве "Відмінності української
опозиції від опозиції Білорусі і Росії" переросла у дисертацію. Вона про "площі
невдоволення" — майдани — у Києві, Мінську і Москві та про те, як ці площі
впливають на протестні рухи в столицях трьох країн.

Під час українського періоду свого життя Хансен вів популярний відеоблог, де
навчав всіх бажаючих норвезькій мові.

"Якби я навчав норвежців українській мові, я б запропонував спочатку вивчити
базові слова, такі як "дякую", "вибачте" і "вітаю", — говорить Арве. —
"Майдан", "гідність", "свобода" — ці слова потрібно знати, щоб зрозуміти
національний характер українців. 

"Пристрасть", "кохання" і "одруження" — тому що ці слова звучать дуже
по-українськи і часто йдуть одне за одним (Хансен одружений з українкою — УП). 

І останнє слово: "різноманіття" — тому що це єдине слово, яке, на мою думку,
найбільш точно характеризує Україну".

У 2019 році Арве став співавтором книги "Війна пісень: популярна музика і
новітні російсько-українські відносини". 

В 2021 році вийде з друку його книга "Міські протести: теоретичний підхід до
просторів невдоволення". Зараз він з колегами готує новий проєкт — про "війну
пісень" в Білорусі. 

В інтерв’ю "Українській правді" Арве Хансен згадав, як з кількох електронних
листів з темою "WoS" почалася книга-дослідження "А War of Songs", розповів як
український гумор в піснях став зброєю у війні з російським пафосом, пояснив,
як сам простір Майдану Незалежності допомагає протестувальникам, і порівняв
українські і білоруські протести. 

\video{https://youtu.be/eh--Y7ku_0Q}

\subsubsection{\enquote{Мене вразило як весь вагон метро почав скандували: ла-ла-ла!}}

\textbf{— Як вам прийшла ідея подивитися на українсько-російську війну з незвичайного
боку — на те, як вона віддзеркалюється в піснях?} 

— Випадково. Коли почалися військові події\Furl{http://www.istpravda.com.ua/articles/2019/03/22/153870/} у Криму і на Донбасі, ми з трьома
колегами в Норвегії та Німеччині — Андрієм Рогачевським, Інґваром Стейнгольтом
і Давід-Емілем Вікстрьомом — відправляли один одному посилання на пісні в
YouTube. Тема цих електронних листів була "WoS", це абревіатура від англійської
"War of Songs".

І незабаром помітили, що у нас дуже багато назбиралося цих "військових пісень":
"Воины света",\Furl{https://www.youtube.com/watch?v=D8HqRH5cHPo} "Никогда мы не будем братьями", "Никогда мы не будем братьями,
ответ украинке", "Путин — х*йло"…\Furl{https://www.pravda.com.ua/news/2019/04/26/7213707/}

\textbf{— До речі, на обкладинці книги під нотним станом уривок фрази "х*йло!" і
"ла-ла-ла". Коли ти її почув вперше?}

— Я вважаю, що "Путін — х*йло" — знакова пісня у цьому конфлікті. Пам’ятаю, як
в 2014 році я сидів у київському метро. Хтось зайшов у вагон і почав співати цю
пісню. А всі пасажири дуже голосно скандували: "ла-ла-ла!" Це мене вразило
тоді.

На обкладинці книги, до речі, ще одна пісня, яка стикається з українською
"Путін — х*йло". Це "A Million Voices", яку росіянка Поліна Гагарина співала
під час Євробачення в 2015 році. 

І разом вони демонструють, як музика двох країн у цій пісенній війні
відрізняється одна від одної. З одного боку, є українці, які користуються
гумором у боротьбі проти своїх "воріженьків". З другого — Росія, яка робить
вигляд, що війни немає, співає про мир, єдність і порозуміння. 

В якомусь сенсі можна сказати, що українсько-російська пісенна війна — це війна
між гумором і цинізмом.

\ifcmt
pic https://img.pravda.com/images/doc/0/9/092b0fd-1.jpg
caption Автори книги "Війна пісень: Популярна музика і новітні російсько-українські відносини" ("А War of Songs: Popular Music and Recent Russia-Ukraine Relations"). Зліва направо — Інґвар Стейнгольт, Давід-Еміль Вікстрьом, Арве Хансен, Андрій Рогачевський
\fi

\textbf{— Здається, що ця пісенька харківських футбольних фанатів, те ж саме, що "Лист запорожців турецькому султанові", тільки зовсім стисло. Можливо, на той момент в цих кількох словах була формула української національної ідеї або мрії?} 

— Цікаве порівняння, насправді виглядає як кавер. Після стількох воєн за
незалежність в історії України українці стали дуже цінувати свою свободу, як ці
козаки на полотні Іллі Рєпіна. Але я не думаю, що мрія українців в тому, щоб
постійно боротися. 

\textbf{— Що тебе вразило найбільше під час роботи над книгою, які пісні?} 

— Нас усіх здивувало, що українці, не дивлячись на криваву революцію,
вторгнення та війну, продовжували використовувати гумор у своїх піснях.

Мені здається, що це ваш спосіб не лише переживати складні ситуації, але й за
допомогою гумору об’єднувати людей. 

Згадуються такі пісні Майдану, як "Єврогойдалка" гурту "Фолькнери", "Вітя, чао"
Ольги Хуторянець і "Гімн тітушок" гурту "Телері". Або пісні Ореста Лютого
"Катя-ватница" і "Росіян в Донбасє нєт" під час кривавих військових події на
сході України.

\textbf{— Як відбирали пісенний матеріал, за якими критеріями?} 

— Ми починаємо з історії української музики протестів — від часу протистояння
радянській владі до 2014 року. Але найдокладніше перша частина книги
розповідає про музику під час Революції Гідності. 

Ми хотіли подивитися, як характер музики змінювався разом із перебігом подій
самого Майдану. З 21 листопада 2013 року було багато мотивуючих рок-пісень, як
під час Помаранчевої революції. 

З 30 листопада переважають гумористичні пісні. Але з моменту зіткнень на
Грушевського 19 січня 2014 року музика стає серйозною. А з лютого — сумною і
навіть траурною. Ми також написали про п’ятий період — після Революції, коли
люди намагалися зрозуміти, що насправді сталося під час Майдану і хто яку роль
відігравав.

\ifcmt
pic https://img.pravda.com/images/doc/4/0/40776b8-3.jpg
caption Арве Хансен на вулиці Грушевського в Києві, 19 лютого 2014 року 
\fi

Інші частини книги розповідають про заочний діалог між українськими і
російськими музикантами за допомогою музики. 

Наприклад, "Никогда мы не будем братьями" і всі відповіді на неї з боку
російських музикантів. Про державний гімн Росії і пародії на нього, наприклад,
"Гимн антирашистов" чи "Патриоты". 

Про Євробачення як геополітичне поле бою — причому в російсько-українських
відносинах не лише останнього часу, а вже багато років. Цей розділ нашої книги
має назву "ʽЛаша тумбайʼ чи ʽРаша, гудбайʼ" і демонструє, як через музику
регіональний конфлікт переноситься на міжнародну сцену. 

Так, як ми бачимо конфлікт — це не тільки військові дії, а й певний музичний
діалог між сторонами. Найчастіше злий і з образами, але він хоча б є. Артемій
Троїцький у передмові до нашої книги написав, що "війна пісень краще, ніж війна
куль і снарядів". 

\textbf{— Чи спілкувалися ви під час написання з авторами і виконавцями цих
пісень?}

— Марія Бурмака\Furl{https://www.pravda.com.ua/articles/2006/12/27/3192052/} погодилася розповідати про розвиток музики протестів 1980-х
років. Ще Марко Галаневич\Furl{https://life.pravda.com.ua/culture/2020/03/24/240324/} з гурта "ДахаБраха" і Євген Славянов з гуртів "OY
Sound System" і "StroOM" розповідали про музику під час Революції гідності. 

Ми також контактували з відомими артистами-політиками, такими як Вакарчук і
Руслана, але вони були надто зайняті, щоб брати участь у нашому науковому
проєкті. Можливо, погодяться на інтерв’ю для другого видання.

\subsubsection{"Піснею можна поранити ворога дуже сильно"}

{\bfseries 
— Які ще пісні, українські та російські, найбільше характеризують цю війну? 
}

— В цій війні дуже багато пісень, їх значення і роль залежать від того, для
кого вони призначені. 

Більшість пісень написані, так би мовити, "для внутрішнього користування", щоб
мотивувати співвітчизників до боротьби. Наприклад, українська пародія Ореста
Лютого на гімн РФ "Здохни, Імперія", чи "Наш президент Владимир Путин"
російського гурту "Черные береты". 

Є ще провокаційні пісні, що написані на адресу сусідів-антагоністів. В цій
музичній війні беруть участь не лише українці й білоруси. Тут можна згадати
"Podaj Rękę Ukrainie"\Furl{https://www.youtube.com/watch?v=QtClmp3zQBc} польського гурту "Taraka" — про те, що потрібно
підтримувати українців під час Революції. 

Або "Гарри" білоруського гурту "Brutto" " про молодого солдата, який пішов на
війну і, швидше за все, не повернеться.. Або, наприклад, "Никогда мы не будем
братьями" литовця Віргініуса Пупшиса на вірш Анастасії Дмитрук. Вона викликала
багато пісень-відповідей з російської сторони. Приміром, "Никогда мы не будем
братьями, ответ украинке" Леоніда і Гліба Корнілових. 

\video{https://youtu.be/jj1MTTArzPI}

Також є пісні, написані для світової спільноти. І тут, мені здається,
"Євробачення" гарний приклад. Цей конкурс, який мав би бути аполітичним, став
міжнародним полем бою. Полем, на якому російські музиканти хочуть показувати
свою країну як миролюбну і відкриту. 

Їм відповідають інші пострадянські країни, як Україна і Грузія. Як це було з
"1944" Джамали або з "We Don’t Wanna Put In" грузинського гурту "Stephane \&
3G", що на слух сприймається як "Ми не хочемо Путіна". РФ домоглася, щоб ця
пісня була знята з конкурсу, який тоді, в 2009-му, проходив в Москві. Відносно
Джамали, Росія вимагала від оргкомітету "Євробачення" того ж, але не вийшло.

\textbf{— Наскільки унікальна війна пісень як явище?} 

— Ми нечасто чуємо про це, але музика війни майже завжди з’являється там, де є
війна. 

Наприклад, коли нещодавно відбулися бойові дії в Нагірному Карабасі між
Вірменією та Азербайджаном, відомі азербайджанські музиканти зробили музикальне
відео\Furl{https://youtu.be/bSh5tm2Hmn0} на підтримку своїх солдатів. 

\textbf{— Хто сьогодні перемагає в пісенній війні: Україна чи Росія?}

— Я не думаю, що можливо перемогти у війні пісень, адже її проводять не
снарядами і кулеметами.

Але можна поранити ворога дуже сильно. Можна згадати пісню "Никогда мы не будем
братьями", як приклад. Судячи з безлічі відповідей на неї з російської сторони,
ми можемо стверджувати, що ця українсько-литовська пісня потрапила у болючу
точку. Певним чином це можна вважати маленькою перемогою для України. 

\textbf{— Чи слідкуєш і далі за новинками української пісні? Хто тебе останнім часом
зацікавив, взагалі — кого ти любиш і слухаєш?} 

— Мені дуже цікаво слідкувати за українською музикою тому, що країна, на мою
думку, зараз знаходиться в процесі національного пробудження. І з’являється
багато нової музики, і вона дуже оригінальна.

Особисто мені подобається alyona alyona.\Furl{https://life.pravda.com.ua/columns/2019/02/9/235507/} У неї дуже свіжий стиль, і я не знаю,
як це українською, але англійською я би сказав, що у неї справжнісінький
"flow". 

Також цікаво слідкувати за експериментальною музикою Onuka \Furl{https://sos.pravda.com.ua/articles/2020/06/19/7150801/} і 
"ДахаБраха".\Furl{https://life.pravda.com.ua/culture/2016/02/8/207736/} Люблю
слухати "Один в каное",\Furl{https://life.pravda.com.ua/culture/2019/03/10/235963/} "Бумбокс",\Furl{https://www.pravda.com.ua/news/2018/07/20/7186852/} і багато інших українських гуртів. 

\video{https://youtu.be/fg0_i_pwrK8}

\subsubsection{\enquote{Фізичний простір Майдану допомагає учасникам протесту}}

\textbf{— Окрім "пісень війни", ще одна тема твого наукового інтересу — територія
міських протестів. Які висновки ти зробив, коли жив і вивчав місця протесту в
Києві і Мінську?} 

— Майдан Незалежності — дуже зручне місце для протестів з трьох причин. 

По-перше, площа має дуже сильну символіку. На Майдані було багато успішних
протестів, три з яких стали революціями. До того ж "Майдан Незалежності" — це
два слова, які символізують незалежність, у тому числі від Москви. Назвали його
саме "майданом", а не "площею". "Майдан" — слово, якого немає в російській
мові.

По-друге, Майдан розташовано там, де постійно є люди: в долині між трьома
важливими центрами: політичним — на Печерських пагорбах, релігійним — на
Замковій горі, й історичним — на Старокиївській горі. І протести, які проходять
на Майдані, завжди помітні. 

По-третє, фізичний простір Майдану допомагає учасникам протесту. Наприклад, на
площу ведуть десять різних вулиць, кожна з яких розгалужується на інші вулички,
тому дуже складно контролювати ззовні те, що відбувається на Майдані. Складно
перекрити кожну вуличку, яка веде на площу.

У Мінську все інакше.

Все у політичному центрі — архітектура, пам’ятники, топоніміка — підтверджує і
підтримує ідеологію Лукашенко, яка у свою чергу ґрунтується на ідеології
Радянського Союзу, Великої вітчизняної війни і на політичному успіху самого
Лукашенко.

До того ж, владі легко контролювати центральний простір через його архітектурні
особливості. 

Візьмемо, приміром, площу Незалежності в Мінську, де можуть розміститися
десятки тисяч людей. Але немає єдиного простору: всі стоять окремими групами, і
їх легко ізолювати, потім заарештувати. А "аварійних" виходів, яких багато на
Майдані, практично немає на площі в Мінську. І це лише деякі негативні
властивості тієї площі. 

\ifcmt
pic https://img.pravda.com/images/doc/3/0/303840f-4.jpg
caption Арве Хансен: "У своїй дисертації я намагаюся зрозуміти, які площі використовуються для масових протестів, чому саме ці, та як простір впливає на учасників протесту" Фото: Хокун Бенямінсен
\fi

\textbf{— Виходить, нам, так би мовити, територіально більш пощастило, і топоси Києва
зручніші для протестних акцій, ніж топоси Мінська?} 

— Я так вважаю. Як підтвердження своїх думок, мені цікаво було побачити, що
безліч акцій протесту зараз відбувається не в центрі Мінська, біля політичних
інституцій, там, де було б логічним протестувати. Вони збираються на проспектах
Машерова і Переможців — досить далеко від центру. 

Думаю, що у людей погані асоціації з колишніми акціями, які відбувалися у
площах Жовтневій та Незалежності і зазнали поразки, і через те, що вони
відчувають себе вразливішими в центрі міста. 

\textbf{— Ти учасник нашого Майдану: порівняй, будь ласка, його з нинішнім білоруським
— чим вони ідеологічно і докорінно відрізняються?} 

— Я не брав активну участь на Майдані, але був там майже кожен день. Я
спостерігав, говорив з людьми, досліджував. Намагався зрозуміти, хто протестує,
чому і як.

Майдани багато в чому схожі, але і багато в чому відрізняються. Наприклад,
білоруси активно утримуються від західної символіки. Певно, через те, що
бояться російського втручання, як було після революції в Україні. Але також і
тому, що хочуть продовжувати тісні економічні та політичні відносини з Росією. 

\ifcmt
pic https://img.pravda.com/images/doc/7/5/754d19e-2.jpg
cpx Арве Хансен на Майдані, січень 2014 року Фото: Харальд Хофф
\fi

\textbf{— Які найяскравіші моменти київського Майдану ти досі згадуєш?}

— Я щойно сказав, що не був учасником Майдану, але це не зовсім правда.

20 лютого 2014-го я зустрічався з респондентом на Контрактовій площі. Ми почали
говорити, але йому зателефонували, і нам довелося зупинити інтерв’ю.

Виявилося, активісти боялися, що твори мистецтва, які тимчасово зберігалися в
Українському домі на Європейській площі, можуть бути пошкодженими. Будинок
профспілок згорів за день до того.

Я погодився допомогти перевозити величезні картини, старожитні меблі та інші
старовинні речі з Українського дому до Музею історії Києва.

І все це сталося дуже раптово. Ніяких координаційних рад або штабів
національного спротиву не потрібно було. Активісти самі все організували. Це
було неймовірно. 

\subsubsection{\enquote{Пісні підкреслюють паралель між Майданом і білоруськими подіями}}

\textbf{— Ви фактично заснували нову галузь політичної науки — музикознавчу
політологію. Плануєте у цьому напрямку розвивати своє дослідження?}

— Щойно закінчили "A War of Songs" у кінці 2018 року, ми почали розглядати
можливості писати про популярну музику сусідньої Білорусі.

Як музика розвивається під час конфлікту, ми вже досліджували в Україні, а в
роботі над Білоруссю хотіли піти ще глибше.

\textbf{— Що цікавого виявили щодо білоруських пісень протесту?} 

— Вже цієї весни нас цікавили відповіді на запитання: що відбувається зараз з
популярною музикою Білорусі? І чи можна побачити через музику, що країна стоїть
на порозі змін? 

Зараз бачимо, що у Білорусі є широкий національний рух, який не має собі рівних
в історії країни. Вже кілька місяців до так званих "виборів" скрізь грала
білоруська версія пісні Віктора Цоя "Перемен" з періоду Перебудови.

Зараз у білоруській протестній музиці можна знайти багато цікавих пісень.
Наприклад, працівники філармонії, священники і студенти співають у публічних
просторах народні і релігійні пісні, такі як "Купалінка" і "Магутны Божа". А
хтось співає пісні білоруського року з 2000-х, такі як "Тры чарапахі" гурту
"N.R.M." 

Але найцікавіше для нас — нові пісні, які створені під час протестів. Вони дуже
різні, і серед них можна відзначити декілька: "Баю-Бай" гурту "Дай Дарогу!",
"Жыве" чи "Мы не ‘народец’" гурту "TOR Band", чи "Тепло" Макса Коржа. 

Є ще і багато іноземних музикантів, які додають свої голоси до хору
невдоволення: "Выходи гулять" російського реп-гурту "Каста", чи "Героям"
українського рок-гурту "Беz Обмежень". Ця пісня частково співається
українською, частково білоруською, що підкреслює паралель між Майданом і
білоруськими подіями. 

Андрій Краснящих, для УП
