% vim: keymap=russian-jcukenwin
%%beginhead 
 
%%file 30_04_2022.fb.pogrebnoj_jurij.herson.1.koloborant.cmt
%%parent 30_04_2022.fb.pogrebnoj_jurij.herson.1.koloborant
 
%%url 
 
%%author_id 
%%date 
 
%%tags 
%%title 
 
%%endhead 
\zzSecCmt

\begin{itemize} % {
\iusr{Сергей Петренчук}

суть такова, что в сложившейся ситуации ничего делать нельзя за все накажут.

\begin{itemize} % {
\iusr{Юрий Погребной}
\textbf{Сергей Петренчук} , так что, отключить в городе воду, свет, газ и всё остальное?

\iusr{Сергей Петренчук}
\textbf{Юрий Погребной} думаю что рано или поздно это произойдёт с водой и отоплением.

\iusr{Сергей Петренчук}
\textbf{ЮрийПогребной} временно окупированный Крым остался без воды, хотя Крым это Украина
\end{itemize} % }

\iusr{Марина Кочубей}
Лучше изучить дело А потом комментировать Так будет объективно Вас же не обвинили

\begin{itemize} % {
\iusr{Юрий Погребной}
\textbf{Марина Кочубей} , Вы как-то странно комментируете мои посты. Можете воздержаться?

\iusr{Марина Кочубей}
Вы настаиваете?

\iusr{Юрий Погребной}
\textbf{Марина Кочубей}, очень прошу  @igg{fbicon.hands.pray}  @igg{fbicon.face.grinning.smiling.eyes} 

\iusr{Марина Кочубей}

У меня встречное предложение Не несите панику и не делайте утверждений на все
100 процентов .Договорились

\iusr{Юрий Погребной}
\textbf{Марина Кочубей} , всё, баню, Вы мне надоели  @igg{fbicon.face.grinning.smiling.eyes} 

\iusr{Nadia Mokryakova}
\textbf{Марина Кочубей} не раскрывай рот.....

\iusr{Владислав Григорьев}
\textbf{Юрий Погребной} Ю.Б., если бы ЭТА... была бы не женщиной, Вы могли бы показать, каким курсом направлен к - ь ,,Москва,,?
\end{itemize} % }

\iusr{Sergiy Romanyuk}

По Кармаликову вопрос сложный. Пусть СБУ разбирается. А вот то что его
Муниципальна Варта рекетом или скажем так КРЫШЕВАНИЕМ попыталась заняться это
факт.

\begin{itemize} % {
\iusr{Юрий Погребной}
\textbf{Sergiy Romanyuk} , у меня нет такой информации, но всё в этой жизни возможно ((

\iusr{Алена Чехун}
\textbf{Sergiy Romanyuk} так ось чому коли ми написали пост, як обдирає муніципальна варта то цей пан Ілля почав їх захищати і спілкуватись з нами нецензурно. Зрозуміло.
\end{itemize} % }

\iusr{Irina Turner}
Очень сложная ситуация.. Надеюсь, что разберутся...

\iusr{Dima Krava}
Вообще не дерзко,даже мягко)

\iusr{Sergiy Romanyuk}

Я понимаю что контролировать такую организацию не имея никаких инструментов для
этого УТОПИЯ. Возможно процесс вышел из под контроля. То что Кармаликов выехал
в Украину огромный ПЛЮС. Надеюсь что он просто идеалист .

\iusr{Тетяна Михайловська}
Думаю до простого робочого класу питань не буде, а от до керівників закладів мабуть що так.

\iusr{Zinaida Steblievskaya}
\textbf{Тетяна Михайловська} абсолютно правильно. Керівник може звілтнитися, на його місце швидко знайдуть підміну

\iusr{Наталья Зайкина}

Считаю, что главное - сохранить город, он один из немногих городов, сохранивших
нормальную жизнедеятельность в условиях полной оккупации, после Победы нужны
будут огромные ресурсы на восстановление инфраструктуры, пострадавшей нашей
области. А Херсон будет помогать в восстановлении. Надо максимально сохранять
все, что работает, функционирует. Разбираться, кто колоборант, а кто нет... Критерии
разработайте... А так вы предлагаете бросить на произвол оставшихся
херсонцев: без условий, с мусором, без медицинской помощи, еды, финансов. Или
заставить людей опять же покинуть город. А если люди любят Херсон, не хотят
уезжать. Это все, что власть может нам предложить???

\begin{itemize} % {
\iusr{Юрий Погребной}
\textbf{Наталья Зайкина}, я же не власть, я ничего не предлагаю. Вы у власти спрашивайте, а не у меня.

\iusr{Наталья Зайкина}
\textbf{Юрий Погребной} 

так я не вам вопросы ставлю, уважаемый Пан Юрий, я с вами полностью
согласна. Вопросы к тем, кто бегом назначает виноватых, ставит клеймо
колоборации. Вместо того, чтобы помочь выжить в этих условиях и дать возможность
сохранить честь и достоинство, И сохранить город для всех херсонцев, живущих
сейчас и вынужденных покинуть Херсон. Они там тоскуют, каждую минутуснкундочку
считают, когда смогут вернуться И вернуться не в руины, а в тот Херсон, который
они полюбили навсегда.

\iusr{Наталья Зайкина}

А вы сразу на свой счёт. Нет, вопросов ооочень много, на которые херсонцы хотят
по

\iusr{Наталья Зайкина}
Хотят получить ответы
\end{itemize} % }

\iusr{Di Ma}

Якщо без персоналій та конкретики, то дійсно ситуація з правилами поведінки в
окупації складна та давно потребує офіційного роз'яснення. Комунікація влади з
населенням окупованих територій відсутня з першого дня або обмежується
загальною інформацією, яка не дає жодних відповідей. Це стосується й пересічних
громадян, й представників органів самоуправління, й працівників різних
підприємств.  Повний провал з боку влади.

\iusr{Ivan Akimov}
Вы там держитесь, мы с вами.... Ответ один пока слышно. Нового не придумали.

\iusr{Луиза Кирсанова}
\textbf{Иван Акимов} нет, ещё есть \enquote{глубоке занепокоєння}((((!!!

\iusr{Татьяна Всяких}

Я так понимаю, что все, кто примет какое - либо решение в городе - будут иметь
право на наказание! А как можно жить и кормить семью в оккупации без принятия
решений? Первое решение было принято - не покидать город и за это решение нас
уже многие осуждают...

\begin{itemize} % {
\iusr{Елена Иванова}
\textbf{Татьяна Всяких} 

осуждают те кто всегда это делал это их жизненный принцип.

Я считаю, что государство в первую очередь должно обеспечить безопасность
своему населению, вы взрослый человек и понимаете что сейчас происходит.

Швеция, Ирландия, Нидерланды вы себе представляете, чтобы государство и мировое
сообщество в целом допустило уничтожать эти земли..?? У нас же пожалуйста всю
технику мира к нам на наш чернозем, задумайтесь глубже тогда решение слишком
очевидно.

Я просто сейчас размышляю в комментариях и высказываю лично свое мнение,
нивкоем случае никого не заставляю так думать

\iusr{Татьяна Всяких}
\textbf{Елена Иванова} 

согласна, что для многих, сейчас, осуждение стало смыслом жизни... Остальные
размышления, их много- не для публики., поглощенной поиском мальчиков для
битья...

\end{itemize} % }

\iusr{Ирина Еременко}
Человек помогал городу и людям и его же за это наказали !! Разви это справедливо !!

\begin{itemize} % {
\iusr{Lia Voronova}
 @igg{fbicon.hands.pray} спасибо Ирина за ваши слова поддержки  @igg{fbicon.hands.pray} 

\iusr{Ирина Еременко}
\textbf{Lia Voronova} 

Это действительно так большенство Херсонцев думаю со мной согласны !Всегда
ждали его эфира человек обьяснял что происходит в городе !! В городе не полиции
ни кого не было и нет и если даже он и разговаривал с российскими военными и
что ??? А что надо было делать в спину коктели молотова им кидать?? Что бы по
нашему городу в ответ стреляли !! Я просто в шоке !! Люди которые ни чего не
делали хорошие остались для власти а человек старался поддерживал жизнь и
порядок города и ЕГО ЖЕ ОБВИНЯЮТ !! Люди где справедливость !???

\end{itemize} % }

\iusr{Нина Сирик}

СБУ не мешало бы сначала из своих глаз бревна повытаскивать. Если следовать их
логике, то в ситуации полного отсутствия правоохранительных органов было бы
очень \enquote{патриотично} погрузиться в хаос.

. А люди, отвечающие за газоснабжение, электричество, подачу воды тоже
коллаборанты? Всем этим наши пришельцы пользуются

\iusr{Lia Voronova}
 @igg{fbicon.hands.pray} спасибо Нина очень верные слова ! @igg{fbicon.hands.pray} 

\iusr{Луиза Кирсанова}

Интересно, а как в Мариуполе? Они тоже в глубокой окупации уже давно! Где-то
было разъяснение, что ,если в окупации для сохранения своей жизни и жизни
семей, приходится работать( а где взять другую работу в окупации?), то это не
считается колоборанством, а работа под пренуждением. Как то так. Не могу
дословно передать посыл. Только хочу передать смысл. Ведь врачи не могу бросить
больных, а учители детей( слава богу, что до 30.04 работали). Дальше не буду
перечислять кто и что!

\iusr{Рита Козаченко}

Есть разъяснения зам министра юстиции Валерии Коломиец- колоборант это человек
принимающий участие, агитирующий за создание квазиреспублик, работающий на
созданную квазивласть. Те же, кто оказался в оккупации и занимается своей
работой не относятся к колаборантам. Сложнее с учителями. Требования оккупантов
преподавать на русском и по их программам не
законно.

\url{https://www.facebook.com/groups/345666560792223/permalink/393746269317585/}

\begin{itemize} % {
\iusr{Татьяна Сироткина}
\textbf{Рита Козаченко} 

с учителями проще - они закончили учебный год и официально в отпуске. Надеюсь,
что до 24 июня эта ситуация решится. А вот врачам и остальным работникам
комунальной сферы действительно не позавидуешь

\iusr{Рита Козаченко}
\textbf{Татьяна Сироткина} 

\url{https://fb.watch/cIxl9CE1o8/} вот только смотрела интервью с Аристовичем тоже
упоминается Херсон и в частности предприниматели, как им быть

\iusr{Татьяна Сироткина}
\textbf{Рита Козаченко} предприниматели - это понятно - они могут не работать, а городские службы? Оставить жителей города без воды, газа, сета, медицинских услуг?

\iusr{Рита Козаченко}
\textbf{Татьяна Сироткина} ну Херсон еще не квазиреспублика, пока все работают, ждём победу
\end{itemize} % }

\iusr{Лариса Смирнова}

А где ответственность тех, которые не эвакуировали население и как их после
этого назвать! Население эвакуировали и не надо предоставлять коммунальные
услуги !!

\iusr{Наталья Войченко}

Юрий, также по охране города и области от чрезвычайных ситуаций (пожары, стихийные
бедствия и т. д) несут службу сотрудники ДСНС... и живут их семьи, дети. Быть до
последнего - это наше решение!!!

\begin{itemize} % {
\iusr{Сергей Владислава Федоровы}
\textbf{Наталья Войченко} сказать спасибо, наверное мало. Благодарность и привет от наших детей, это их герои!

\iusr{Наталья Войченко}
\textbf{Сергей Владислава Федоровы}, огромное спасибо Вам за поддержу, этого нам не хватает...

\iusr{Сергей Владислава Федоровы}
\textbf{Наталья Войченко} напишите лс пожалуйста.
\end{itemize} % }

\iusr{Валентин Тимошенко}

А скільки Вам проплатили за цю статтю? Можете мене не банити. Я сам видпишуся.
Я був кращої думки про Вас.

\begin{itemize} % {
\iusr{Татьяна Сироткина}
\textbf{Валентин Тимошенко} а що саме Вам не сподобалося - роздуми як бути чи інфа про Карамалікова?

\iusr{Валентин Тимошенко}
\textbf{Татьяна Сироткина} Інфа про карамалікова.

\iusr{Татьяна Сироткина}
\textbf{Валентин Тимошенко} 

так давайте фактажно - я мешкаю на Острові і саме завдяки дружині в нас всі
крамниці цілі і є хоч якийсь товар. Це дійсно робота Кармалікова. І за це
дякую. Стосовно іншого - є правоохоронні органи - це їх парафія.

\iusr{Marina Dibrova}
\textbf{Татьяна Сироткина}, Таня, я не уверена, но говорят про рекет, не знаю как Карамликов, может он и с хороших побуждений, но потом группу людей сложно сдерживать.

\iusr{Татьяна Сироткина}
\textbf{Marina Dibrova} 

ничего не скажу - у меня нет фактов про рекет (возможно он и имел место быть,
потому что контингент был разношерстный), но про то что они действительно
патрулировали и не позволяли разграбить магазины - это 100\%

\iusr{Marina Dibrova}
\textbf{Татьяна Сироткина}, Танюшка, надеюсь, что разберутся скоро, и действительно достойные люди не пострадают  @igg{fbicon.hands.pray} 
\end{itemize} % }

\iusr{Juliya Bon}

Вы ещё забыли упомянуть ДСНС. Ребята, которые спасают жизни, тушат пожары не
знают  @igg{fbicon.shrug}  как им себя вести, чтобы не обвинили потом в коллаборации. А
если все уедут, то что будет с городом?

\iusr{Юрий Погребной}
\textbf{Juliya Bon}, да, спасибо! МЧС - это важно.

\iusr{Denis Kravetz}
Всі, хто знає Іллю особисто, знають, що він патріот Херсона і України @igg{fbicon.flag.ukraina}

\iusr{Lia Voronova}
Спасибо Денис @igg{fbicon.hands.pray} за ваши слова поддержки Илье. @igg{fbicon.hands.pray} 

\iusr{Анатолий Старюк}
А глава ОДА, уже бывший? Его все таки уволили за бездеятельность?

\iusr{Ievgen Kuleshov}

Дружина ни кое образом не сотрудничала с оккупантами. Дружина действовала в
рамках КП «Муниципальна Варта» по законам Украины. Илья карамаликов в целом был
тем человеком который толкнул этот воз, а вот какими доказательствами будет
обвинять СБУ  @igg{fbicon.thinking.face} ?

\iusr{Lia Voronova}
 @igg{fbicon.hands.pray} спасибо \textbf{Ievgen} за вашу позицию  @igg{fbicon.hands.pray}  всё очень правильно Вы говорите .

\iusr{Елена Гирич}
Мені здається, що час розставить все на свої місця.
Тримайтесь !
Разом ми сила.

\iusr{Jana Kosyk}
У меня дугой вопрос, куда с города пропали все СБУ?  @igg{fbicon.face.eyebrow.raised}  @igg{fbicon.face.monocle} 

\iusr{Lia Voronova}
Отличный вопрос .. и все мы знаем на него ответ ... Уе-али ! А теперь сидят и обвиняют других !

\iusr{Татьяна Гончаренко}
Вишукують відьом. А те, що самі здали нас, маючи армію...про таке не говорять

\iusr{Lia Voronova}
К сожалению именно так. Зато обвинять других это у них хорошо получается.(

\iusr{Sergey Bezvorotny}
Как по мне, мягковато

\iusr{Tatyana Tihomirova}
А что, в Херсоне продолжает работать и существовать СБУ? Или... как? Где именно СБУ задержвло Илью?

\begin{itemize} % {
\iusr{Sergey Bezvorotny}
\textbf{Татьяна Тихомирова} а он в Николаев уехал и не приехал

\iusr{Юрий Погребной}
\textbf{Татьяна Тихомирова}, его задержали на нашем блок-посту в Николаевской обл. Он семью вывозил.

\iusr{Tatyana Tihomirova}

Благодарю за разъяснения, друзья! Увы, всё слишком сложно! Но... судя по
информации, исходящей из даннлго Поста, мы видим, что НИКАКИХ УКАЗАНИЙ И
РАСПОРЯЖЕНИЙ ПО ЗАПРОСАМ, посланным их Херсона в Киев, НЕ ПОСЛЕДОВАЛО! Ведь
было открытое письмо: ЧТО ДЕЛАТЬ? КАК НАМ БЫТЬ? ... или я просто не поняла,
либо не был озвучен людям полученный ( возможно) ответ??? Закон о ЖЕСТОКОМ
НАКАЗАНИИ принят, а вот..... ЧТО ДЕЛАТЬ ЛЮДЯМ????- ОТВЕТА НЕТ...

\end{itemize} % }

\iusr{Тетяна Тихенька}

Спасибо что озвучили! Верим в справедливость! Илья Карамаликов должен быть на
свободе  @igg{fbicon.hands.pray}

\iusr{Lyudmila Seleskaya}
це велика проблкма! такі ж ситуації були і під час війни 1941-45 років

\iusr{Володимир Желуденко}

А є термін, яким можна зазвати державних діячів, що знали про можливий напад і
не дали ніяких інструкцій органам місцевого самоврядування, та комунальним
установам. Таким чином у руках окупанта вся конфіденційна інформація, не
говорячи про секретну. Не забезпечили в цьому плані ні законодавчих, ні
нормативних, ні відомчих інструкцій. Це при тому, що війна триває 8 років.

\iusr{Frozina Ksenchik}
\enquote{хто заказав Катю Гандзюк}...отак і тут буде...

\iusr{Гайдук Наталья}

МЧС почти все уехали, ребята давали присягу, врачи, большинство, при первой
возможности уедут. Вот такие настроения в городе.

\begin{itemize} % {
\iusr{Антон Ефанов}
\textbf{Гайдук Наталья} а кто же тогда пожары тушит и выезжает на вызовы если МЧС все уехали? В медицинской сфере в основном уехал младший медперсонал, главврачи все на месте, уехавших врачей стараются заменить врачи из частных практик.

\iusr{Гайдук Наталья}
\textbf{Антон Ефанов} 

да, остались некоторые, из мох знакомых уехали трое с семьями. Врачи сейчас
находятся не а лучшем положении! Лекарств нет, использует то, что передали с
Николаева, делит на дозы для пациентов. Когда узнал, что дома есть Диклофенак
,очень обрадовался. Вы думаете, что так может долго продолжаться ?

\iusr{Антон Ефанов}
\textbf{Гайдук Наталья} сейчас все находятся не в лучшем положении. Продолжатся может так долго, но нам необходимо продержаться буквально на один день дольше.
\end{itemize} % }

\iusr{Данило Бородашкін}

Здаэться в телемарафоні говорили про вчителів окупованих районів Запорізької
області, яких змусили вчити дітей за російською програмою. Цих вчителів назвали
заручниками, не колаборантами.

\iusr{Тетяна Тихенька}
\textbf{Данило Бородашкін} в Запорізькій області всі вчителі написали заяви на звільнення та завершили навчальний рік раніше.

\iusr{Гайдук Наталья}

Мы даже не представляем сколько колоборантов живёт среди нас, Зашла вчера в
Телеграм, сбросили ребята ссылку на страницу Далиев Виталий, ведущий юбилей
Херсон. Восемь лет, Херсон бомбили наши, дети Донбасса росли в подвалах,
россияне вежливые люди и т. д. В конце пожелал мира моему дому и удалил свои
голосовые. И таких ооочень много. Это невыносимо ! Хочется обглодать им лицо
или уехать с города к своим, пусть даже под бомбёжками, но среди своих !

\iusr{Людмила Вретык}

Дуже сподіваюся що Розберуться ! Бо виходить всі хто виїхав і повернеться білі
і пухнасті, а той що працював допомагав місту крайній, так не можна

\begin{itemize} % {
\iusr{Lia Voronova}
 @igg{fbicon.hands.pray}  спасибо Людмила
Правильно Вы говорите.. и как раз в случае с Карамаликовым .. кто остался работал и помогал теперь наказан ((((

\iusr{Ольга Шестакова}
\textbf{Lia Voronova} а такие как Лагута, который бросил город и выехал сразу, кто он? Герой или голова ОДА, который предал область и его жителей?
\end{itemize} % }

\iusr{Ирина Карченкова}

Насколько я помню, президент постоянно говорит о том, чтобы всячески
саботировать, и не сотрудничать с оккупантами.

\iusr{Ирина Карченкова}

В Оккупированном Мелитополе практически все учителя написали заявления на
увольнение. Со временем все Украинские учреждения будут переведены на
контролируемую Украиной территорию.

\iusr{Ирина Мухина}
\textbf{Ирина Карченкова} 

Раз херсонские учителя и воспитатели детских садов в отпуске, то вопрос с
переходом на русский язык на какое-то время закрыт. До осени будет полная
ясность

\iusr{Ирина Мухина}

Спасибо, Юрий Борисович, за поднятую тему. Если не ставить дерзких вопросов, то
никогда не найдёшь ответов, а значит, не примешь решения.

Как видно из обсуждений поста, то по многим рабочим ситуациям есть ответы.
Спасибо всем за внесённую ясность.

Касаемо Ильи Кармаликова, то не знаю пока, что и думать. Время покажет.

\iusr{Lia Voronova}

Спасибо Вам Юрий Борисович.

За Вашу статью и вашу позицию  @igg{fbicon.hands.pray}{repeat=3}  очень важно говорить об этом и разъяснять
людям. Так как сейчас в этом ужасе хаосе люди верят во все сплетни. очень легко
очернить человека в таких ситуациях. Спасибо всем кто думает и анализирует
 @igg{fbicon.hands.pray}  @igg{fbicon.flag.ukraina}

\iusr{Ihor Harahulia}

Вопрос поднятый Вами Юрий Борисович, очень важен для всех людей оставшихся в
Херсоне и делающих важнейшее дело - спасение себя и свои Семьи.

Как это делать?!? Конечно можно прислушиваться в своему сердцу, но как Вы
правильно отметили, есть тонкая грань перейдя которую для «компетентных
органов» каждый становится предателем и изгоем, о чем красноречиво говорит
пример с Илья Карамаликов которому выдвинула обвинения СБУ Херсонской области,
руководство которой признано в госизмене Президент Украины Зеленский Владимир
Александрович...

Странной выглядит молчание центральной власти относительно разъяснений о том,
как действовать простым работникам бюджетной сферы, ведь судят не по призыву
сердца, а по Закону написанному людьми.

Всем нам удачи и мудрости.

\iusr{Наталья Келеш}

Так обидно, когда близкие знакомые, друзья, сознательно сотрудничают с русским
миром и ещё и доказывают мне, как он хорош(( И обижаются, что общаться с ними не
хочу... Хочется там и правда лицо обгладать

\iusr{Oksana Mazur}

Очень правильно и своевременно вы озвучили суть этой ПРОБЛЕМЫ. @igg{fbicon.hands.shake}  Сейчас нужно
доносить это до верховных властей... пусть дают четкие рекомендации как себя
вести людям  @igg{fbicon.shrug} 

\iusr{Татьяна Татьяна}
Стаття - саме те, що відчуваємо... А за 2 і 3 речення я Вас традиційно обіймаю.

\iusr{Татьяна Буренко}
\textbf{Юрий Погребной} 

За справедливость Нужно бороться  @igg{fbicon.biceps.flexed} @igg{fbicon.flag.ukraina} своих не бросаем, хоть и далеко от Вас,
у Ильи очень одиозный адвокат и у нас все получится!  @igg{fbicon.biceps.flexed}  Разом до перемоги!

\iusr{Елена Дмитренко}

А если вы уезжаете, как с работой? Что пишете, за свой счёт? Или как? Меня
интересует такие предприятия, как горводоканал, тец, Херсонгаз, облэнеого.

\iusr{Юрий Погребной}
\textbf{Елена Дмитренко} , это вопрос к руководителям этих предприятий.

\iusr{Андрій Корнєєв}

Нет,Все Правильно, многие остались деньги заканчиваются,работы нет, Помощи нет,
в Общем печально. Много Вопросов пусть бы Лагута с Юрид отделом и давал
пояснения как Себя вести

\iusr{Ольга Васильевна Украинец}
\textbf{Андрій Корнєєв} какой Лагута? Он же ещё в, первые дни все в ФБ написал.

\iusr{Вера Тула-Онищенко}
Юрий, всегда с интересом читаю Ваши публикации...

\iusr{Татьяна Щербина}

Ну, ночью они не работали. Это точно, так как обращалась лично когда офис
разносила мародёрша.

Получила вразумительный ответ что ночью никто не поедет потому что нельзя.

Исходя из этого, договоренностей на ночное патрулирование города не было.

\iusr{Наталія Танклевська}

Юрий, Вы очень конструктивно изложили суть проблемы. И разъяснений так и нет(

\iusr{Iya Volikova}

Все верно! Инструкции нет ...все действуют интуитивно и по велению совести.
Кармаликов спасал продукты, развозил их по больницам организовал охрану аптек и
больниц с помощью безоружных инициативных жителей города, а теперь обвинения....

\iusr{Robert Neils}

It's a personal choice.

Myself would help but only if they agree not to harm civilians. Would be no
part of a regime that terrorizes innocent people.


\end{itemize} % }
