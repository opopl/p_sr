% vim: keymap=russian-jcukenwin
%%beginhead 
 
%%file 11_07_2021.fb.bilchenko_evgenia.1.travlja
%%parent 11_07_2021
 
%%url https://www.facebook.com/yevzhik/posts/4053513654683714
 
%%author Бильченко, Евгения
%%author_id bilchenko_evgenia
%%author_url 
 
%%tags bilchenko_evgenia,istoria,travlja,ukraina,volynskaja_reznja
%%title БЖ. Что со мной происходило в эти дни: травля учёных из-за волынского геноцида
 
%%endhead 
 
\subsection{БЖ. Что со мной происходило в эти дни: травля учёных из-за волынского геноцида}
\label{sec:11_07_2021.fb.bilchenko_evgenia.1.travlja}
\Purl{https://www.facebook.com/yevzhik/posts/4053513654683714}
\ifcmt
 author_begin
   author_id bilchenko_evgenia
 author_end
\fi

БЖ. Что со мной происходило в эти дни: травля учёных из-за волынского геноцида.

Родня. В моей жизни происходит нечто странное и совершенно невероятное. Я
неделю размышляла, стоит ли мне писать об этом в социальных сетях, учитывая
деликатность дела, но публичность сейчас, пока я ещё жива, - это единственный
способ, если не обезопасить себя, то хотя бы донести до моего украинского и
русского читателя правду. Я на опыте проверила, что огласка - это лучший способ
работы в информационном обществе постправды, где ничего нельзя скрыть, даже
личные письма.

5-6 июля 2021 года я и мой аспирант, журналист, философ Павел Волков должны
были принять  участие в антифашистской конференции Гуманитарно-экономического
университета в городе Замостье (Польша) по поводу Волынской резни. Я -
единственный учёный из Украины, кто не побоялся назвать резню резней и
составить доклад, раскрывающий, как либеральные механизмы насильного
всепрощения, принятые Институтом национальной памяти Украины, уравнивают
гитлеровских коллаборантов и их жертв. Польские коллеги восприняли мою
инициативу с энтузиазмом. Мой доклад был принят и я была приглашена.

Подчеркну: конференция имела сугубо академический характер и призвана была
исторически восстановить истину-событие. При участии в нем Украины это стало бы
первым международным шагом к солидаризации против героизации военных
преступников из ОУН-УПА, это смогло бы сильно напугать сотрудников нашего
института нацпамяти и иметь далеко идущие геополитические уже последствия.

До последнего на моей почте велась официальная переписка с паном Матеушем и
госпожой Магдаленой. Естественно, мы собрали и потеряли личные средства на ПЦР,
билеты, гостиницу. Почему потеряли?

А теперь внимание. 2 или 3 июля я начинаю получать очень жёсткие угрозы жизни и
здоровью от анонимов, которые представляются... россиянами. Усекли момент? 

Активация угроз началась после того, как мои антифашистские работы по memory
studies стали доступны миру через премиум-пакет академии Гугль, присвоенный мне
за количество процитировавших меня в месяц учёных. 

Дальше. 3 июля я получаю сообщение от представляющегося... русским хакера,
который сообщает мне о том, что мой личный маньяк (якобы тоже из не простивших
россиян) заказал ему взлом моей почты и всех личных ресурсов, но не заплатил.
Аноним мне предлагает заплатить, чтобы открыть его данные. Понимая, что это
мошенничество, я игнорирую письмо, пытаясь двигаться дальше.

Ещё через полчаса мой доклад... Снимают с конференции сами поляки. То есть,
доклад в защиту позиции Польши снят Польшей. Логика?Из личного разговора с
доктором Матеушем мне стало ясно, что решение - политическое и спущено сверху
по доносу. В этот же момент Гугль, который, в отличие от конспирологических
сплетен, - робот и беспристрастен, как всякая беспристрастная машинка,
отключает все мои ящики и высылает сообщение о взломе всех моих личных
аккаунтов, предлагая сменить пароли...

Меня охватывают паника и отчаяние. Поскольку выдержать оказываемое на меня на
Украине давление без психотропов невозможно, я принимаю последнюю разрешенную
врачом таблетку гидазепама и проявляю слабость. Я пишу мошеннику, чтобы
выяснить, кто он и откуда. Тот просит перевести ему деньги на рублёвый (!) счёт
через "Ракету". Делая вид, что я не умею ею пользоваться, я пытаюсь выманить
номер его карты: бесполезно. Я таки перевожу ему деньги уже в невменяемом
состоянии, естественно, ящик пропадает, и я теряю всё. Ошибка, согласна.

Дальше дней я не помню. В порыве неимоверной боли усилием воли я рву рецепт
гидазепама, не зная того, что с бензодиазепиновой зависимости соскакивают по
схеме 5-4-3-2-1-0, а не путем личного геройства. Дальше начинаются ад и ломка,
продолжающиеся до сих пор, но это - моя личная беда, как и потеря денег.

Теперь к делу. Включите бесстрастный логический анализ. Какой из сторон
выгодно, чтобы в Польше не было ни одного серьезного украинца, который
критикует украинскую политику института нацпамяти и солидарен с позицией
польских учёных? Мне очень трудно было наступить ногой на чувства и включить
чистого аналитика. Но логика победила.

Первое, что приходит в голову, - переполошились наши Вятровичи и Дробовичи,
которые через наши же спецслужбы взломали мне ящики с научной перепиской в
прокси-масках россиян. Это самая понятная версия. Именно их польские учёные
обвиняют в увековечивании памяти нацистов, а я должна была показать, как это
работает. Тогда в доносе на меня можно было написать бред об агентуре Кремля:
самую тупую отмазку, чтобы убрать противника.

Второе - мне это очень нелегко признать - ветер дует со стороны российских
патриотов, которым выгодно, чтобы в Украине не было ни одного
ученого-антинациста, способного поехать в Замостье. Там так и написали.

Дескать, тут тотальная тьма. Дальше никаких конструктивных предложений. Или их
не считают нужным демонстрировать. Тогда в доносе будут содержаться факты моего
майдановского прошлого, о котором, впрочем, поляки отлично знали, потому что
после 18 января историю моей жизни мир знает по тысяче медиа.

\ifcmt
  tab_begin cols=3

     pic https://scontent-lga3-2.xx.fbcdn.net/v/t1.6435-9/216864067_4053513421350404_4680132196810260869_n.jpg?_nc_cat=101&ccb=1-3&_nc_sid=8bfeb9&_nc_ohc=dNdCU25MhTMAX9sQ9kr&_nc_ht=scontent-lga3-2.xx&oh=631c3400b3051f8b4fb3c4235986a096&oe=60F2C145

     pic https://scontent-lga3-2.xx.fbcdn.net/v/t1.6435-9/217007251_4053513494683730_309060431257606481_n.jpg?_nc_cat=100&ccb=1-3&_nc_sid=8bfeb9&_nc_ohc=sCbuq3mJTXYAX-effDT&_nc_ht=scontent-lga3-2.xx&oh=26a301a9950eca0f862a4002e2971085&oe=60F29497

		 pic https://scontent-lga3-2.xx.fbcdn.net/v/t1.6435-9/216165730_4053513554683724_1545125090016982604_n.jpg?_nc_cat=111&ccb=1-3&_nc_sid=8bfeb9&_nc_ohc=r3C3dIEDGVwAX9Ks0hu&tn=lowUrFCbCbt-jOWu&_nc_ht=scontent-lga3-2.xx&oh=35504bdbb13bd93501c222d696e360ed&oe=60F2CB7A

  tab_end
\fi

Третье - дешёвка. Какой-то конкретный маньяк, тупо приватное лицо, заигрался
настолько, что ему уже плевать на интересы Украины, России, Польши, мира и
гуманитарные цели по борьбе с фашизмом. В случае с индивидуальными проявлениями
шизофрении логика - бессильна.

Доказательством, что мы на этой конференции от Украины должны были быть,
является уже опубликованная моя работа-доклад, первая программа конференции и
статья Паши Волкова в "Антифашисте": 

\url{https://antifashist.online/item/libo-assimilyaciya-libo-unichtozhenie-lozungi-ukrainskih-nacionalistov-ne-menyayutsya-so-vremen-volynskoj-rezni-itogi-nauchnoj-konferencii-v-polshe.html}

Я ощущаю себя волком, которого обложили со всех сторон, чтобы показать тупик,
но я не играю в double bind, я знаю, как это делается. Естественно, я заболела.
Я осознаю, что сейчас читаются все мои академические разработки в сети. Потому
у меня вопрос к русскому и украинскому читателю-антифашисту:

\begin{itemize}
\item - Кому выгодна моя моральная и затем физическая смерть? 
\item - Кому больше надо, чтобы здесь был сплошной сумрак и ни одного дееспособного
участника антифашистского сопротивления? Украине? России? 
\end{itemize}

Или тем глобальным силам, которые смотрят на нас, вечно прячутся за прокси, и
их не выщелкнуть? При этом пан Матеуш в интервью Паше Волкову выражает надежду,
что в Украине всё-таки найдутся смелые учёные, способные противостоять
госпропаганде. Я нашлась. Меня сами поляки и сняли. Это игра такая или что?

Повторюсь, я усилием воли заставляю себя мыслить рационально, без скидок на
личные привязанности. Иначе я бы тут же обвинила только команду
дробовичей/вятровичей и успокоилась бы на удобном для себя варианте. Мне
интересны ваши версии без эмоций. Просто уберите политические предпочтения и
подумайте об этом случае так, как если бы я жила в 17 веке в Южной Африке.
