% vim: keymap=russian-jcukenwin
%%beginhead 
 
%%file 03_01_2018.stz.news.ua.mrpl_city.1.jelochnyj_shar
%%parent 03_01_2018
 
%%url https://mrpl.city/blogs/view/elochnyj-sharbyl-uslyshannaya-ot-odnoklassnika
 
%%author_id burov_sergij.mariupol,news.ua.mrpl_city
%%date 
 
%%tags 
%%title Елочный шар (Быль, услышанная от одноклассника)
 
%%endhead 
 
\subsection{Елочный шар (Быль, услышанная от одноклассника)}
\label{sec:03_01_2018.stz.news.ua.mrpl_city.1.jelochnyj_shar}
 
\Purl{https://mrpl.city/blogs/view/elochnyj-sharbyl-uslyshannaya-ot-odnoklassnika}
\ifcmt
 author_begin
   author_id burov_sergij.mariupol,news.ua.mrpl_city
 author_end
\fi

В памяти Марка запечатлелись лишь два эпизода, в которых участвовал его папа.
Первый, когда, проснувшись в предновогоднюю ночь от резкого звука, он увидел
елку с несколькими игрушками, а под ней папа с мамой, собирающие осколки
елочного шара. Папа поднял голову и шепотом сказал: \enquote{Ничего, я куплю тебе
новый} - и подмигнул. Второй был связан с проводами отца на фронт. На вокзале
много людей, папа стоит в строю, человек в военной форме громко выкрикивает
какие-то слова, строй начинает движение, папа машет рукой, мама плачет,
прижимая одной рукой Марка к себе, а другой вытирает платком слезы.

Вскоре мама пошла работать на \enquote{Азовсталь} медсестрой. Он знал, где \enquote{Азовсталь}.
Если перейти Малофонтанную и подняться на пригорок, то видны были трубы. Мама
как-то показала Марку трубы и сказала: \enquote{Это – \enquote{Азовсталь}}. Потом воспоминания
смешались в кучу. Вагон, духота, перебранка взрослых. Взрывы бомб. Мама
прикрывает его своим телом. Беспрерывный стук колес. Наконец короткое слово
\enquote{Урал}. Переполненный людьми бревенчатый дом. Много детей. Шумно. Перебранка
взрослых. Хотелось есть. Лишь один эпизод запомнился четко, до мельчайших
подробностей. Соседка принесла маме конверт. Мама взяла его дрожащими руками.
Разорвала конверт, достала оттуда бумажку, что-то прочла и разрыдалась. Сквозь
слезы повторяла: \enquote{Марк, мы остались одни, Марк, мы остались одни, Марк, мы
стались одни...} Много позже она рассказала, что его папа геройски погиб в бою на
фронте...

В конце марта 44-го года Марк с мамой вернулись в Мариуполь. Сожженный вокзал
зиял пустыми проемами окон. Тут и там были погорелки. Но домик на Торговой
улице, где они жили до войны, остался цел. Немцы-поджигатели не успели дойти до
него. Марк отдыхал от шума. Старушка-соседка, дальняя родственница мамы,
научила его читать. Теперь у него появилось занятие. Он перечитал потрепанный,
весь в кляксах букварь. Потом его добровольная учительница подарила книжку с
названием \enquote{Родная речь}. Под темно-малиновым коленкоровым замусоленным
переплетом скрывались и рассказы, и стихи, были там также рисунки. Содержания
\enquote{Родной речи} хватило Марку почти на год. Не все, правда, понимал он из
прочитанного, но сам процесс чтения его увлекал необыкновенно.

1 сентября победного 45-го года мама отвела его в школу. Школа была недалеко от
их дома. В каких-то полутора кварталах. Это было двухэтажное здание из серого
кирпича, одно из немногих избежавших сожжения гитлеровцами. Марк оказался в
колонне наголо постриженных мальчишек, попарно построенных перед школой. Он
стал учеником 1 класса \enquote{А} Мужской неполной средней школы. Ему нравилась
учительница. Она его хвалила за хорошее чтение и за то, что смирно сидит на
уроках. В этом же классе учился и ее сын Толик. Вот на Толика она покрикивала.
А зачем надо было на него кричать? Ведь он хорошо учился. Не баловался, не
вертелся, всегда выучивал стихотворения и правильно решал все примеры на доске…

Когда ребята из бывшего 1 класса \enquote{А} перешли в третий класс, незадолго до
Нового года учительница сказала: \enquote{Ребята, принесите из дома игрушки для
украшения елки}. У Марка елочных игрушек не было, они пропали во время войны.
Где достать игрушки? Оставались считанные дни до праздника елки в школе. И тут
случилось чудо. Он шел после уроков из школы мимо тетки, торговавшей семечками.
А рядом с нею на скамеечке в открытой шляпной коробке на вате сверкало
несколько елочных шаров. Марку больше всех понравился самый большой из них. На
его боку была вмятина, в которой таинственно мерцал зеленый свет. Он спросил
цену шара и помчался домой. Едва открыв дверь, стал умолять маму купить шар. Но
мама сказала: \enquote{Сыночка, это для нас дорого, тебе варежки купить нужно}. Но Марк
все-таки настаивал на своем. И мама сдалась. Они купили шар. На следующий день
он принес шар в школу и отдал учительнице.

За два дня до новогоднего утренника из самого большого класса старшеклассники
вытащили в коридор парты. В центре поставили елку. Старшая пионервожатая и
молодые учительницы принялись украшать елку, которая на самом  деле была
сосной, игрушками, флажками и клочками ваты. На утренник возле елки собрались
школяры первых - четвертых классов. Среди них был и Марк. Старшая пионервожатая
водила хороводы вокруг елки, кто-то читал стишок. Потом мальчик из четвертого
класса бойко играл на аккордеоне, а его сменил баянист - папа одного из
учеников. Но Марк ничего не видел и никого не слушал. Он всматривался в
елку-сосну, пытаясь увидеть свой шар. Марк подумал: \enquote{Наверное, он с другой
стороны висит. Жаль, что не пошел я в хоровод, тогда бы непременно увидел шар}.
Марк дождался, когда все выйдут из класса, и принялся рассматривать лесную
красавицу со всех сторон. Он обошел ее несколько раз, но его шара среди игрушек
не было. \enquote{Наверное, разбили}, - подумал Марк.  

Он поплелся домой. У ворот его ждал Толик, сын учительницы: \enquote{Марик, куда ты
пропал? Папа и мама устраивают мне елку. И сказали, чтобы я пригласил своих
друзей-отличников. Ты ведь – отличник? Приходи на Новый год к нам к
двенадцати}. Когда мама пришла с работы, Марк рассказал о приглашении. Она
засуетилась, достала его праздничную белую рубашку, стала ее рассматривать со
всех сторон, нашла небольшую прореху, тут же принялась ее зашивать… И вот
настало 1 января. Мама осмотрела сына со всех сторон, что-то подправила, что-то
одернула, поцеловала его и напутствовала: \enquote{Веди себя хорошо и дай Толику
новогодний подарок}. На самодельной салфетке из белой материи лежал красный
пистолет, отлитый из карамели. Войдя в дом учительницы, Марк все сказал и
сделал, как велела мама, Толику передал подарок, всех поздравил. В комнате уже
были его одноклассники Гриша и Валера, тоже отличники. Гостей усадили за стол,
к чаю подали пирог с вишнями. Были на столе и вазочка с горкой леденцов и
конфет-подушечек, присутствовало абрикосовое повидло в большой тарелке.

Настала пора развлечений. Толик забренчал на пианино какую-то пьесу, несколько
раз ошибался, но все равно ему хлопали, Гриша с выражением, без единой запинки
продекламировал басню Крылова, Валера показал фокусы с картами. Настал черед
Марка. Он выучил стихотворение Пушкина. \enquote{Под голубыми небесами, великолепными...}
- тут поднял глаза и увидел на самом верху елки свой шар. Да, это был его шар.
Он разрыдался... Все его стали успокаивать, утешать: \enquote{Ничего, что забыл стих, со
всяким бывает...} Марк быстро оделся и побежал домой...

Когда через много лет одноклассники собрались у своей школы, а потом
отправились в ближайшее кафе. Там, перебивая друг друга, повзрослевшие ученики
1-го класса \enquote{А} вспоминали забавные случаи из школьной жизни, вспоминали и
первую учительницу. Лишь угрюмый Марк молчал.
