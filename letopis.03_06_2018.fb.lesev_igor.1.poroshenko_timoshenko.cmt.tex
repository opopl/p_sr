% vim: keymap=russian-jcukenwin
%%beginhead 
 
%%file 03_06_2018.fb.lesev_igor.1.poroshenko_timoshenko.cmt
%%parent 03_06_2018.fb.lesev_igor.1.poroshenko_timoshenko
 
%%url 
 
%%author_id 
%%date 
 
%%tags 
%%title 
 
%%endhead 
\subsubsection{Коментарі}

\begin{itemize} % {
\iusr{Stella Aksenova}
Срвершенно верно. От перемены мест слагаемых... какие бы не были величины... Итог один.

\iusr{Марина Прохорова}

Это всё так, но трудно спрогнозировать степень идиотизма оставшегося
электората. Пока вменяемые уезжают, куда глаза глядят, остающиеся свидомые,
активисты, бандиты на содержании и дураки по собственной инициативе, способны
на многое. Вариант финала Тимошенко против Ляшко - это ещё не самое страшное. А
Ляшко против Билецкого? - Проще утопиться))


\iusr{Виктория Форт}
\textbf{Игорь Лесев}, браво Вам! Написано блестяще, с юмором и с правдой жизни. Достойно.
Ходят слухи, что Вона не жлобится на работников пера, может, и брехня, но её пиарщики работают театрально и с огоньком. Вот даже сырники и новая мордочка подтянутая чего стоят!
Надеюсь, Вам уже сделали интересные предложения?  @igg{fbicon.wink} 

\begin{itemize} % {
\iusr{Игорь Лесев}
мой уровень писанины не соответствует высоким стандартам Юлии Владимировны

\iusr{Виктория Форт}
\textbf{Игорь Лесев}, нечего скромничать! Другое дело, что в штабах требуется полное послушание и выправка, а здесь народ избаловался своим альтернативным мнением. Однако, если жирные предложения и в лозунгах будет "за мир!", тогда, может, совесть не пострадает.

\iusr{Игорь Лесев}
совесть в любом случае не страдает, я с ней не знаком

\iusr{Виктория Форт}
\textbf{Игорь Лесев}, чтобы на кого-то работать, можно и не пить чай с сырниками.
\end{itemize} % }

\iusr{Дмитрий Коломийченко}

1. Тимошенко достаточно ума чтобы не повторяться. Поэтому сейчас дозированная
активность, больше экономики и меньше политики. 

2. Проблема Тимошенко в отсутствии адекватной вызовам команды. Прежняя была
тоже не очень, но сейчас и её нет. Вернувший назад Ваваков это не команда. 

3. Страна в системном кризисе.  Победа на выборах не даёт ресурса для его
преодоления. Скорее наоборот. Кризис начнёт съедать политический капитал с
большой скоростью. А его, судя по антирейтингу, не так и много. Любой кто
выиграет выборы быстро превратится в Порошенко-2. Поэтому более перспективная
стратегия - отказ от участия в выборах с прицелом на перехват власти после
начала активной фазы кризиса. Сюжет - возвращение короля (королевы). Вот под
эту ситуацию и нужно собирать команду.  Перебежчики и так будут. Только они
полезны когда есть на кого опереться кроме них. 

4. ПАП строит свою стратегию в логике переизбрания Ельцина. Вывод в второй тур
представителя реванша. Бойко идеальный кандидат. Только в этой схеме не хватает
харизматичного кандидата, который соберёт ожидания нового лица, простых
эффектных решений. Странно, что администрация не подготовила никого из военных.
Савченко неудачно вербовали? Так надо было альтернативных заготовить. Еще для
успеха этой стратегии не хватает консолидации элитных группировок (по схеме -
при всём богатстве выбора другой альтернативы нет). Слишком эгоистично ПАП вёл
себя. Креативные ходы по типу Бабченко все эти проблемы не заменят.

\iusr{Дмитрий Новаковский}
А сколько у Юли батальонов?

\iusr{Олег Резник}

Да, Юля сильнее его в пропагандистских делах, с её то опытом! Тертая баба. Мне
кажется, что она и без политтехнологов хорошо чувствует массу. И если у неё
получится, то счёт будет 2:0 в её пользу. А 1:0 было, когда она с премьерством об
ставила его при Ющенко. Про " весь в соплях" помните же?

\end{itemize} % }
