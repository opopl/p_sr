% vim: keymap=russian-jcukenwin
%%beginhead 
 
%%file 25_12_2021.yz.djadjka_chernomor.1.nuzhna_li_nam_takaja_ukraina
%%parent 25_12_2021
 
%%url https://zen.yandex.ru/media/chernomorr/i-vse-je-nujna-li-nam-takaia-ukraina-61c531b07d887c5335e4445f
 
%%author_id yz.djadjka_chernomor
%%date 
 
%%tags rossia,ukraina
%%title И всё же, нужна ли нам такая Украина
 
%%endhead 
 
\subsection{И всё же, нужна ли нам такая Украина}
\label{sec:25_12_2021.yz.djadjka_chernomor.1.nuzhna_li_nam_takaja_ukraina}
 
\Purl{https://zen.yandex.ru/media/chernomorr/i-vse-je-nujna-li-nam-takaia-ukraina-61c531b07d887c5335e4445f}
\ifcmt
 author_begin
   author_id yz.djadjka_chernomor
 author_end
\fi

Знаю, что многие досадливо поморщатся даже от названия, вынесенного в
заголовок. Дескать, ну надоело же читать об одном и том же. Тем, кому та тема
действительно не интересна, советую перелистнуть строку, правда сама проблема
от этого никуда не исчезнет. Мне же эта тема близка, по причине своего
проживания в Крыму, и наличие в ближайших соседях государства, которое
становится всё более недружественным, меня не просто волнует, но и беспокоит.

\ii{25_12_2021.yz.djadjka_chernomor.1.nuzhna_li_nam_takaja_ukraina.pic.1}

\enquote{Аффтором} опубликовано несколько статей на тему Украины и получены сотни
комментариев от тех читателей, кому она не безразлична. Хотя нет! Как раз
наоборот – большинство комментариев, говорят об обратном. Основной посыл такой:
будущее этой страны нас больше не волнует, они нам больше не братья, они
предатели и пусть катятся в Европу, которую от них тоже уже тошнит. И самое
главное – нам не нужна эта обнищавшая страна, которую придётся содержать и
восстанавливать за наш счёт.

Типа: украинцы сами себе выбирали таких неудачных президентов, сами прыгали на
майдане с кастрюлями на голове, сами жгли покрышки и теперь пусть сами отвечают
за свои слова и поступки. А нам теперь всё по барабану – как хотят так пусть и
выкручиваются.

Рискую привлечь на свою голову недовольство, а в отдельных случаях даже
неприкрытую неприязнь некоторых своих читателей, но считаю, что обязан
поделится своими сомнениями по поводу правильности их позиции.

Во-первых, виноваты ли они что выбирали таких президентов, которые, собственно,
и приходили к победе на выборах за счет обещаний наладить отношения с Россией,
или как в случае с их последним президентом, на желании сменить предыдущего
президента Украины, с явно антироссийскими настроениями и начавшим войну с
собственным народом пожелавшим остаться в русском гуманитарном мире?

Не мы ли, в своё время тоже неудачно выбирали президентом сильно бухающего
человека, едва не продавшего собственную страну и его народ, в услугу и угоду
западному миру. Хотя не факт, что именно мы, а не хитрые политтехнологи
устроили ему победу на выборах. Виноваты ли в этом случае мы сами? Слава Богу,
ИМЕННО БОГУ, а не нам, что у нас вскоре появился президент сумевший поднять
страну с колен. Украину всевышний таким президентом, к сожалению, не одарил.

Интересно знать, а кого бы Вы выбрали президентом Украины, будь вы гражданами
этой несчастной страны? Видели ли Вы в их кандидатах хоть одного достойного
человек, который бы действительно, а не на словах, гарантировал хорошие
отношения между нашими странами. Не мы ли своим бездействием допустили
расчистку украинского политического поля для превращения этого государства в
американскую экспериментальную политическую площадку для взращивания
украинского национализма. И теперь пожинаем эти горькие плоды. Так что виноваты
не только сами украинцы.

\ii{25_12_2021.yz.djadjka_chernomor.1.nuzhna_li_nam_takaja_ukraina.pic.2}

Во-вторых, изначально в майданах участвовала лишь очень малочисленная, но очень
организованная и хорошо проплаченная часть населения страны. Думаете сами по
себе появились на майдане палатки, автомобильные покрышки и полевые кухни. Или
пешком добирались на майдан из Западной Украины вооружённые боевики. Вся
всемирная история революций и контрреволюций подтверждает, что власть легко
переходит в руки меньшинства, если оно хорошо организовано, вооружено и имеет
достаточное финансовое обеспечение.

Примеров тому множество, от нашей октябрьской и не нашей кубинской революций и
до многочисленных переворотов в южной и центральной Америках или во многих
арабских и африканских странах. Большинство населения ничего не понимает в
происходящем и слепо следует ходу событий. А дальше под действием пропаганды
перешедшей под власть победителей средств массовой информации наступает
изменение сознания аморфного большинства под существующую власть. Как правило
это сопровождается большими бедами для большинства населения и это как раз их
беда в а вовсе не вина.

Еще меня удивляет непоколебимая уверенность, значительной части россиян, в том,
что в случае вовлечения Украину в сферу нашего политического и экономического
влияния, нам придётся взять на себя непосильные обязательства по её содержанию.
Ой, так ли это? Действительно, многие отрасли экономики Украины пришли в
упадок, но в подавляющем большинстве это устаревшие предприятия с изношенным
оборудованием, не выдерживающие конкуренции с современными
высокотехнологическими предприятиями развитых западных стран.

\ifcmt
  ig https://avatars.mds.yandex.net/get-zen_doc/5226294/pub_61c531b07d887c5335e4445f_61c5ff1babd3697a5be34e8f/scale_1200
	@caption Картинка взята из открытых источников
	@wrap center
	@width 0.7
\fi

Да у нас самих закрыты тысячи заводов и фабрик, не соответствующих современным
требованиям. И вместо них появляются новые, современные предприятия, с
конкурентной продукцией. Украина, пока ещё, относительно самодостаточное
экономическое пространство с грамотным и довольно трудолюбивым населением, во
всяком случае её потенциал пожалуй даже выше, чем у многих российских регионов,
и она способна, при соответствующих условиях, быстро показать рост своей
экономики. И вовсе не обязательно на невыгодных для нас условиях, а наоборот на
выгодных для нас кредитах или инвестициях. Не забывайте хотя бы то, что Украина
располагает самыми крупными в Европе чернозёмными площадями.

Многие финансовые организации и даже отдельные страны готовы вложить в Украину
значительные средства, в случае её политической стабильности и считают это
очень выгодным для себя вложением средств. Почему это не может быть выгодным
для нас?

Я, конечно не считаю себя ни экономическим гуру, ни политическим пророком, и
моё мнение – это мнение простого обывателя, но я, всё же не так прямолинеен,
как многие мои читатели и не склонен делить окружающий мир и происходящие
события исключительно на белое и чёрное, отбросив пятьдесят оттенков серого, не
говоря уж о существующем цветовой палитре, подразумевающем тысячи оттенков в
многообразии спектра картины мира.

Я также понимаю, что в комментарии пишут, как раз неравнодушные к происходящим
событиям люди, поэтому, вполне возможно, что их мнение не очень-то совпадает с
мнением весьма инертного большинства. Именно поэтому я и задаюсь вынесенным в
заголовок вопросом и предлагаю обычным гражданам России принять участие в
обсуждении этого важного для нас всех вопроса.

Оставляйте, пожалуйста свои комментарии, ставьте лайки, ПОДПИСЫВАЙТЕСЬ на канал
Дядька Черномор - ни один комментарий не останется без ответа автора.

