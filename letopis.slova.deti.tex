% vim: keymap=russian-jcukenwin
%%beginhead 
 
%%file slova.deti
%%parent slova
 
%%url 
 
%%author 
%%author_id 
%%author_url 
 
%%tags 
%%title 
 
%%endhead 
\chapter{Дети}
\label{sec:slova.deti}

\enquote{Сегодня у нас проходит мероприятие под названием \enquote{Самый
\emph{детский} день!}. Мы его так назвали, потому что в этот день вспоминают о
всех \emph{детках}, а у нас \emph{детки} особенные, которые требуют особого
внимания, особого к себе отношения, особой любви. Спасибо огромное за помощь и
поддержку нашим постоянным друзьям, которые помогают нашим \emph{деткам}}, -
поблагодарила Ильина, \textbf{Праздник для особых \emph{детей} \enquote{Самый
\emph{детский} день!} прошел в Луганске}, lug-info.com, 31.05.2021

\ifcmt
  pic http://img.lug-info.com/cache/1/5/1622459163_921.jpg/1000wm.jpg
\fi

Весь в крови и орет на пионеров. Тилль Линдеманн показал новое видео про СССР -
\enquote{Я ненавижу детей}, strana.ua, 01.06.2021

Як уберегти \emph{дітей} від небезпеки? Чи потребують змін правила організації
шкільних екскурсій, походів?  2 червня екскурсійна група у складі 44 учнів
львівської школи, трьох вчителів і двох батьків, приїхала на водоспад Кам'янка,
що у Карпатах, неподалік міста Сколе. \emph{12-річний школяр}, роблячи селфі,
виліз на каміння, послизнувся, впав униз водоспаду, зачепившись за дерево.
\emph{Дитина} перебувала під водою. Рятувальники, щоб витягнути \emph{школяра},
розпиляли дерево. На жаль, хлопець помер до приїзду медиків,
\textbf{Загибель 12-річного школяра змусила посилити заходи безпеки біля водоспаду у Карпатах},
radiosvoboda.org, 03.06.2021

«Чи дерево зіграло фатальну роль, чи спричинило смерть \emph{дитини}? Якби \emph{дитина} не
зайшла занадто близько до водоспаду, не переступила за обгородження, цього б не
сталося. Якби \emph{дитина} впала з цієї висоти просто у воду, то вдарилась би у
камінь і теж міг бути смертельний випадок. На цьому водоспаді рятувальники
вперше надавали допомогу, це перша така рятувальна операція, на жаль,
трагічна», – каже Радіо Свобода Віталій Туровцев, керівник пресслужби Головного
управління Державної служби з надзвичайних ситуацій у Львівській області,
\textbf{Загибель 12-річного школяра змусила посилити заходи безпеки біля водоспаду у Карпатах},
radiosvoboda.org, 03.06.2021

«Нічого кращого, як бути з \emph{дітьми}, говорити з ними, ділитися своїм життєвим
досвідом немає. \emph{Діти} добре засвоюють приклад батьків, як ті дали собі раду при
зустрічі з небезпекою. Шкільні екскурсії корисні для \emph{дітей}, бо це спілкування,
живий досвід взаємодопомоги, здобування досвіду походів, це здружує, виробляє
риси взаємовиручки, що основне у нашому у житті. Звісно, важко передбачити
поведінку підлітка, це вік турбулентності, то велика, то мала \emph{дитина}, а що
ввімкнеться, яка позиція, у який час, у який момент – точно не знаємо. Певні
вікові особливості закодовані природою, але те, що треба спілкуватися з \emph{дітьми},
мандрувати з ними – це дуже добре, це моменти задоволення і щастя. На жаль,
трагічні випадки будуть і від цього нікуди не подінешся, але оберігати і
застерігати, безумовно, потрібно», – наголосила психолог Галина Католик,
\textbf{Загибель 12-річного школяра змусила посилити заходи безпеки біля водоспаду у Карпатах},
radiosvoboda.org, 03.06.2021

Скільки людей щодня відвідують це відоме туристичне місце у Карпатах керівник
парку не сказав. Вхід на територію платний, але другий день після загибелі
\emph{школяра}, каса не працювала. Смерть \emph{дитини} не відлякала туристів. Сюди сьогодні
постійно приїжджали, як дорослі, так \emph{дитячі} екскурсійні групи,
\textbf{Загибель 12-річного школяра змусила посилити заходи безпеки біля водоспаду у Карпатах},
radiosvoboda.org, 03.06.2021

\enquote{На последнем совещании Леонид Иванович (Пасечник) сказал, что вопрос
\emph{детства} – самый главный. Это и подтверждает своими действиями. Сейчас мы
занимаемся распределением инвалидных колясок и передачей наиболее нуждающимся},
- уточнила уполномоченный по правам \emph{ребенка}. Она также выразила
признательность за доставку груза и его хранение Народной милиции ЛНР и
общественной организации инвалидов \enquote{Путь добра}, 
\textbf{Детский омбудсмен передала воспитаннице лутугинского интерната инвалидную коляску},
lug-info.com, 03.06.2021

\enquote{По поручению главы Луганской Народной Республики Леонида Пасечника 3 июня
уполномоченный по правам \emph{ребенка} ЛНР Юлия Назаренко передала \emph{девочке} с
нарушением опорно-двигательного аппарата, воспитаннице Государственного
образовательного учреждения Луганской Народной Республики \enquote{Лутугинская
специальная (коррекционная) школа-интернат} Валерии Веремеенко инвалидную
коляску для передвижения в помещении}, - говорится в сообщении,
\textbf{Детский омбудсмен передала воспитаннице лутугинского интерната инвалидную коляску},
lug-info.com, 03.06.2021

Государственное унитарное предприятие \enquote{Почта ЛНР} в рамках
благотворительной акции \enquote{Лучики добра} собрало 19 тыс. руб. для \emph{детских}
домов и интернатов. Об этом сообщили на предприятии.  \enquote{За период с
февраля по май 2021 года, благодаря совместным усилиям почтовиков и всех
неравнодушных жителей ЛНР, удалось собрать свыше 19 тыс. руб., из которых на
сумму свыше 9 тыс. руб. была оформлена подписка на \emph{детские} республиканские
периодические печатные издания, а на сумму около 10 тыс. руб. были осуществлены
благотворительные платежи на расчетные счета республиканских \emph{детских} социальных
учреждений}, - говорится в сообщении,
\textbf{\enquote{Почта ЛНР} в рамках акции собрала для детских домов и интернатов 19 тыс. руб.},
lug-info.com, 03.06.2021

\begingroup
Колектив Національного культурного центру України у м. Москві вітає з
Міжнародним днем захисту \emph{дітей}!, 
\textbf{Наші діти!}, ukrcentr.ru, 01.06.2021

Внаслідок російської агресії на сході України загинули 240 \emph{дітей}, -
Україна в ОБСЄ, day.kiev.ua, 03.06.2021
\endgroup

Гибель людей на Донбассе говорит о том, что наше общество больное. И, что
человек, что общество, приступают к лечению, когда чувствуют себя больными.
Именно чувствуют, но для этого надо эти самые чувства иметь. К сожалению, мы
продолжаем развивать бесчувственное общество потребителей, паразитов, не
имеющих чувств, а только желудок.  Вакциной от бесчувствия является творчество,
искусство, красота. И в этом плане творчество маленькой \emph{девочки} Фаины
Савенковой лучик света в этом огромном хаосе преступных желаний политиков и
безразличия масс. В творчестве Фаины нет пафоса, нет конъюнктуры, нет
романтизации войны - это творчество чистого \emph{Ребенка} повзрослевшего во время
войны более большинства взрослых,
\textbf{Гибель людей на Донбассе говорит о том, что наше общество больное},
Денис Жарких, strana.ua, 04.06.2021

Учреждения образования Луганска открыли 11 летних онлайн-лагерей для своих
воспитанников. Об этом сообщила пресс-служба столичной администрации со ссылкой
на заместителя начальника управления образования Анжелику Прядкину.  \enquote{У нас
открылись пришкольные онлайн-лагеря, как это было в прошлом году. Но если
минувшим летом их было 7, то теперь работают 11 лагерей, расположенных в разных
частях города, более чем для 300 детей}, - сказала она,
\citTitle{Луганский Информационный Центр — Учреждения образования Луганска
запустили 11 летних онлайн-лагерей – мэрия}, , lug-info.com, 07.06.2021

Активисты общественного движения (ОД) \enquote{Мир Луганщине} благоустроили
\emph{детскую площадку} в Луганске. Об этом сообщила пресс-служба движения.
\enquote{В Луганске 9 июня активисты общественного движения \enquote{Мир
Луганщине} в рамках акции \enquote{Начни с \emph{детства}} провели субботник на
\emph{детской} площадке, которая находится в квартале Жукова}, - говорится в
сообщении.  Пресс-служба отметила, что акция проходит уже второй год подряд в
период летних каникул под патронатом Луганского территориального отделения ОД
\enquote{Мир Луганщине}. В ней участвуют активисты проектов \enquote{Молодая
гвардия} и \enquote{Дружина} ОД \enquote{Мир Луганщине}: они убирают мусор с
\emph{детских} площадок, зачищают и красят турники, качели, лавочки, горки.
\enquote{Главная задача – сделать для \emph{детей} города Луганска пусть
небольшие, но красочные и уютные \emph{детские} площадки. Мы надеемся, что наша
работа им понравится}, – сказал член Молодежного совета при администрации
Луганска Максим Ефименко,
\citTitle{Луганский Информационный Центр — Активисты ОД \enquote{Мир Луганщине}
благоустроили детскую площадку в Луганске}, , lug-info.com, 09.06.2021

Алексей Попович
Вот  ....... реву как девчонка ... Плачу по нашему чистому \emph{детству}...
комментарий, Валерия Ланская \enquote{Прекрасное далёко}, youtube

Андрей Погорелый
Реквием по Советскому Союзу. Гимн счастливого Детства!!!
комментарий, Валерия Ланская \enquote{Прекрасное далёко}, youtube

%%%cit
%%%cit_pic
%%%cit_text
Знаете, я обожаю своё поколение.  Мы застали пластинки и бобины, перекручивали
на карандаше кассеты и сидели под стационарным телефоном в ожидании звонка; ещё
помним дискеты и диафильмы, но наравне с сегодняшними тинейджерами свободно
осваиваем все новинки техники.  Мы ещё любим Высоцкого и Магомаева, но и не
впадаем в эпилепсию от современных ритмов.  Более того, у нас и здесь нашлись
«любимчики».  Мы точно знаем, что в этом мире больше никогда не будет таких как
Майкл Джексон, Queen и Led Zeppelin, и никому до сих пор так и не удалось
заменить нам утраченных Белоусова, Талькова и Цоя.  Это мы в \emph{детстве} часами
выглаживали колючую советскую школьную форму и красный галстук, а сегодня также
трепетно заботимся о замшевой обуви
%%%cit_comment
%%%cit_title
\citTitle{Советскому поколению повезло родиться до того, как у детей отобрали право выбора}, 
Елена Лукаш, strana.ua, 13.06.2021
%%%endcit

%%%cit
%%%cit_head
%%%cit_pic
%%%cit_text
Если вы думаете, что украинский шпрехен-фюрер заточен только на выискивание
русской речи, то ничего подобного. Омбудсмен также наехал на венгерский гимн в
Закарпатье. Ну низяяяяяя, чего не понятно, только тотальная мова.  С гордостью
мовный шпрехен-фюрер в отчете отразил, что за 2020 год закрыто 300 русских
школ, но тут же тяжело вздохнул, указав, что в частных \emph{детских} садах как
говорили на русском так и говорят, и ничьи хотелки на это повлиять не могут, а
в одном частном \emph{детском саду} на Днепре вообще прямо отказались от мовы
заявив, что русский язык - это язык элиты
%%%cit_comment
%%%cit_title
\citTitle{Speak на мове please}, 
Terra Incognita, zen.yandex.ru, 30.04.2021
%%%endcit

%%%cit
%%%cit_head
%%%cit_pic
\ifcmt
  tab_begin cols=3

		 width 0.3
		 pic https://sharij.net/wp-content/uploads/2021/06/1-48.jpg?x35786

     pic https://sharij.net/wp-content/uploads/2021/06/2-22.jpg?x35786

     pic https://sharij.net/wp-content/uploads/2021/06/3-16.jpg?x35786

  tab_end
\fi
%%%cit_text
Материалы переданы в архив в рамках проекта «Без срока давности». В частности,
в них задокументированы случаи с вырезанием органов, испытаниями действий
препаратов и выкачиванием крови для раненых немецких военных.  Также
рассекречены сведения о нечеловеческих условиях содержания узников и убийствах
на территории концлагеря «Красный». Кроме того, раскрыты новые обстоятельства
преступлений против мирных жителей в Крыму, Запорожской и Николаевской областях
УССР. В числе жертв были и \emph{дети}
%%%cit_comment
%%%cit_title
\citTitle{В ФСБ рассекретили документы об экспериментах нацистских врачей в Крыму: вырезали органы и выкачивали кровь}, 
Светлана Комаренко, sharij.net, 22.06.2021
%%%endcit

