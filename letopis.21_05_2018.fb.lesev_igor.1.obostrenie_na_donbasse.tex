% vim: keymap=russian-jcukenwin
%%beginhead 
 
%%file 21_05_2018.fb.lesev_igor.1.obostrenie_na_donbasse
%%parent 21_05_2018
 
%%url https://www.facebook.com/permalink.php?story_fbid=1925315287499565&id=100000633379839
 
%%author_id lesev_igor
%%date 
 
%%tags donbass,obostrenie,ukraina,vojna
%%title Обострение на Донбассе
 
%%endhead 
 
\subsection{Обострение на Донбассе}
\label{sec:21_05_2018.fb.lesev_igor.1.obostrenie_na_donbasse}
 
\Purl{https://www.facebook.com/permalink.php?story_fbid=1925315287499565&id=100000633379839}
\ifcmt
 author_begin
   author_id lesev_igor
 author_end
\fi

Донбасское «обострение» во время проведения известного спортивного турнира,
естественно, будет, но не более того. У Киева так и нет гарантий, что русские
не впишутся за ЛДНР, а без таких гарантий само наступление теряет даже
тактический смысл. Есть, конечно, еще фактор американцев, но и те уже
окончательно запутались в последовательности давления на Москву. Новые санкции
вводятся без привязки вообще к чему-либо. Более того, особо уже и вводить то
нечего. Американский Минфин уже получает финансирование на то, чтобы
«планировать разработку новых санкций». Это значит, что если завтра кто-то
захватит условный Мариуполь, в Вашингтоне еще нужно придумать придумку нового
наказания Москвы.

\ifcmt
  ig https://scontent-frt3-1.xx.fbcdn.net/v/t1.6435-9/33074584_1925315117499582_547695711349637120_n.jpg?_nc_cat=104&ccb=1-5&_nc_sid=730e14&_nc_ohc=nRnvaOLEWa4AX-nRHxh&_nc_ht=scontent-frt3-1.xx&oh=0597edc7192a5af30a12aba7acf0b040&oe=61B7FDFB
  @width 0.4
  %@wrap \parpic[r]
  @wrap \InsertBoxR{0}
\fi

Соответственно, в геополитическом плане обострение на Донбассе теряет
первоначальный смысл. Да, можно унизить репутационно, например, захватом
крупного города, той же Горловки. Но эта цена вопроса, за которую должны будут
платить жизнями обладатели в основном украинских паспортов. Да и есть ли у
Киева возможность проведения подобной крупной войсковой операции – тот еще
вопрос. Наконец, риски в случае неудачи такого наступления конкретно для
товарища Первого несопоставимы с возможным успешным наступлением. К тому же,
нет никакой необходимости конвертировать всплеск национал-патриотических
настроений именно сейчас, за год до выборов. Власть у нас настолько чудесная,
что любой успех легко девальвирует падением мордой в грязь. А за год таких
падений еще будет ого-го сколько. Так что обострение на Донбассе – это
«последний довод королей» образца зимы-весны следующего года.

Ну а в целом, следует признать, что Донбасс (его наиболее населенный осколок)
Украина таки теряет. При этом, теряет безвозвратно. И это уже даже не вопрос
будущей власти в Киеве, которая де порешает на раз-два эту проблему. Во многом
против Украины играет именно внешняя политическая повестка, особенно выход
Штатов из «ядерной сделки» по Ирану. Здесь как раз Вашингтон очень здорово
подосрал именно Киеву, продемонстрировав условность своих договоренностей.
Теперь не только для России, но и для Европы стало очевидным, что «санкции в
обмен на Донбасс» не сработают. Мотивационная часть для русских по вопросу
Донбасса отсутствует. Это же видно и на примере миротворцев, тема которых
похоронена окончательно. А ведь миротворческая миссия – это была возможность
красивого ухода Москвы из ЛДНР под какие-то там гарантии и добрые слова. Но
теперь и этого нет.

В итоге, следующая киевская власть будет вести переговоры не об интеграции ЛДНР
обратно в состав Украины, а о вхождении самопровозглашенных республик на правах
некой субъектности. Такой себе союз меча и орала. А такой вариант не нужен
будет в Киеве, даже если президентом Украины станет какой-то там Бойко. А
значит мы приблизились вплотную к самому паршивому варианту решения проблемы –
абхаизации конфликта. Пока еще ЛДНР – это серая зона. Такая себе Абхазия и ЮО
до 2008 года, где была чересполосица, значительную часть территории
контролировали грузинские войска и в международном статусе республики были
никем. А у России в качестве «последнего аргумента» остается вариант признания
республик через процедуру «принуждения к миру», более того, еще и с выходом «на
законные границы» сепаратистских образований.

Как со всем этим жить России в условии санкций и западной изоляции – это ее
проблемы. А вот как жить нам с такой восхитительной властью, которая указывает
на стену и говорит, что это дверь – вопрос уже не праздный. Особенно для тех,
кто собирается жить в Украине и после выборов 2019 года.

\ii{21_05_2018.fb.lesev_igor.1.obostrenie_na_donbasse.cmt}
