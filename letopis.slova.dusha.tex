% vim: keymap=russian-jcukenwin
%%beginhead 
 
%%file slova.dusha
%%parent slova
 
%%url 
 
%%author 
%%author_id 
%%author_url 
 
%%tags 
%%title 
 
%%endhead 
\chapter{Душа}
\label{sec:slova.dusha}

Фаина ищет и находит простые слова, образы, сюжеты. У нее, в отличие от
современных потребителей, болит \emph{душа}. Не слышать ее может лишь тот, у кого \emph{души}
нет. Для кого патриотизм это не жертвенность во имя страны, а способ
обогащения, для кого война только прибыльный бизнес, для кого своя рубашка
ближе к телу, а хата с краю.  Послушайте ее обращение в ООН. Никакой
театральности, надрыва, пафоса. Она, скорее, робеет, чем кого-то обвиняет,
требует. Ей как-то неудобно, ведь она понимает, что дети не помнят мира, а
большие дяди в политике не понимают. Точнее, все эти дяди понимают, но просто
дяди негодяи, и заняты другим. А ей, Фаине, за этих дядей неловко и стыдно.
Хотя стреляют не по им, а по ней. Вот это и есть крест настоящего человека,
испытывать стыд за чужую подлость. Маленький человек, но большая \emph{Душа},
\textbf{Гибель людей на Донбассе говорит о том, что наше общество больное},
Денис Жарких, strana.ua, 04.06.2021

%%%cit
%%%cit_head
%%%cit_pic
%%%cit_text
Любовь \emph{душевная} – сложнее. Она может избегать выражений, присущих полу, но не
чуждается телесности. Так ребенок, любя мамочку, обхватывает ее за шею, не
отпускает, хочет вжаться в материнское тело до неразличимости. Но кто из нас
скажет, что любовь ребенка к маме – ненастоящая? Любящий любимого действительно
хочет даже съесть, и поэтому мать кормит дитя собою, равно как и Господь кормит
нас Своим Телом и Кровью. И старики могут любить подлинно и нежно, уже не имея
особых сил для телесных чувственных проявлений.  Грусть сопутствует \emph{душевной}
любви. Грусть со всем синонимическим рядом: с тоской, печалью, меланхолией,
томлением, жаждой неведомого, желанием распахнуть окно и смотреть на звезды.
Юношеское томление ищет выхода, старческое отличается созерцательностью. Но
часто это – лишь балансирование на жердочке. \emph{Душевное} в человеке непостоянно и
нетвердо. \emph{Душевность} либо соскальзывает вниз, в тот самый разгул плоти,
причастившейся злому духу, либо же стремится насытиться вверху, в духе,
объединяющем и тело, и \emph{душу}
%%%cit_comment
%%%cit_title
\citTitle{Возвращение в Рай}, Андрей Ткачев
%%%endcit
