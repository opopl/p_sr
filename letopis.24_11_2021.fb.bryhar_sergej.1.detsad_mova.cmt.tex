% vim: keymap=russian-jcukenwin
%%beginhead 
 
%%file 24_11_2021.fb.bryhar_sergej.1.detsad_mova.cmt
%%parent 24_11_2021.fb.bryhar_sergej.1.detsad_mova
 
%%url 
 
%%author_id 
%%date 
 
%%tags 
%%title 
 
%%endhead 
\subsubsection{Коментарі}

\begin{itemize} % {
\iusr{Руслан Шеремета}
Малороси, хохли... Тільки позбавляти права голосу.

\begin{itemize} % {
\iusr{Serhii Bryhar}
\textbf{Руслан Шеремета} Звичайно. Але немає кому. Для встановлення адекватної влади, на жаль, банально бракує здорового електродного поля..

\iusr{Євген Шпичка}
\textbf{Serhii Bryhar} думав, як описати відчуття, що виникли, коли прочитав допис. Перше на думку спало слово "гидко"
\end{itemize} % }

\iusr{Євген Шпичка}
Був час, коли такі ж їли своїх дітей від голоду

\begin{itemize} % {
\iusr{Мамина Доця}
\textbf{Євген Шпичка} Дуже не коректне порівняння , бо в ті часи московит не питав , а йшов та завойовував


\iusr{Serhii Bryhar}
\textbf{Євген Шпичка} ...буває й так, що дійшовши до повної сраки, вони так і не починають розуміти, що до чого. Власне, найбільший освітній здобуток цих людей - вміння читати-писати..


\iusr{Serhii Bryhar}
Московит ніколи не питає. А образи голодомору - як маркер, який варто використовувати, щоби пояснювати, що стається там, куди вони "приходять". Отак може і дійти. А може й не дійти..

\iusr{Дарія Матвіїв-Веклин}
\textbf{Serhii Bryhar} аж страшно від тупості тої дівки.
Та моя позиція в питанні домінування в Україні української - незмінна: тверда політична воля на вищому державному рівні в напрямку українізації суспільства.

\iusr{Євген Шпичка}
\textbf{Оксана Левит} якби боротьбу за незалежність активно підтримувало більше людей, то московити нікого б не завоювали. Але люди повелися на солодкі обіцянки і на свої сподівання, що якось воно буде. Так що не бачу некоректності у своєму порівнянні.

\iusr{Мамина Доця}
\textbf{Євген Шпичка} От власне тому що було дуже багато селянських повстань проти більшовиків , був створений терор і голодомор у відповідь саме селянам , саме в селах ...

\iusr{Євген Шпичка}
\textbf{Оксана Левит} аааа, ну тоді все ясно. Ну якби не повстання, то всі потім би просто мирно їхали б у табори, як "нармальниє савецкіє люді". В тому і справа, що значна частина повстань почали виникати тоді, коли вже була просрана можливість здобути незалежність, як реакція на відбір хліба і колективізацію. Тобто ж знову таки люди шлунками відреагували. Ну ладно тоді, тільки кожен четвертий писемний був, темних людей обдурити легко і пообіцяти "свєтлає будущєє". Але ж зараз? Між шансом розбудовувати нормальне супільство і шансом бути останньою з зарізнаих овець, ця "самиця" людини обирає останнє.

\iusr{Мамина Доця}
\textbf{Євген Шпичка} Не пересмикуйте , будь ласка , бо можливості не було , і це врешті саме ми маємо признати . Імперії так просто не здаються , і почалось це з часів Катерини , яка й зробила населення України неписемним , та кріпаками ... Я також мрію щоб українці стали зрілою нацією , але жодної ілюзії що це станеться швидко точно нема ...Довгий це процес ...

\iusr{Євген Шпичка}
\textbf{Оксана Левит} я не пересмикую. Просто погоджуюся з аксіомою, яка говорить про те, що вирішенню проблеми має передувати її визнання. А якщо вважати, що український нарід завжди був невинний у своїх бідах, то і зараз нічого не зміниться.

\iusr{Мамина Доця}
\textbf{Євген Шпичка} Так от і потрібно врешті визнати що десь помилились , десь промовчали , десь пробачили , десь прогнулись ...Життя штука складна ...


\iusr{Serhii Bryhar}

Знаєте, найкраще сказав Євген Сверстюк: "Кожна нація існує під небом для того,
щоб витворити інститут незалежної держави". Отже, якщо нації це з якоїсь
причини не вдається, вона починає марніти, деградувати, стає легкою здобиччю
шакалів. З цієї точки зору можна пробувати обережно так вести до того, що оце
от "ягражданінміра", "какаяразніца" - протиприродно і якось навіть гріховно.
Саме такі люди і підштовхують спільноту в бік деградації та зникнення.

\iusr{Мамина Доця}
\textbf{Serhii Bryhar} Знаєте Богдан Хмельницький перший гетьман України , і він точно мріяв її об'єднати , але при цьому саме він привів МП на наші терени ...То хто він патріот , какаяразніца , ворог ? От так досі й розібратись не можемо ...

\iusr{Serhii Bryhar}
\textbf{Оксана Левит} 

Я думав над постаттю Хмельницького, певне, років з 14, коли вирішив, що піду
навчатися на істфак. Багато думав. По-різному. І от як на мене, до нього просто
не вийде застосовувати сучасні поняття. Це категорично інші обставини. Тоді,
наприклад, ідентичність, сформована довкола конфесії, була ключовою... Тоді,
держави асоціювалися не з мовами чи культурними особливостями, а з монаршими
родами. Власне, московитів Хмельницький вибрав через головну для себе
ідентичнісну ознаку - релігійну. Хоча, це був десь вибір без вибору.
"Бусурмани" - недруги, поляки - взагалі без коментарів.. тут наче все і
зрозуміло... Зараз у нас є досить багато ознак соборної нації, але московська
отрута заважає рухатися далі, нормально розвиватися. Ми просто застрягли в
лайні неосвіченості, стереотипності, несміливості, безумовно, і може
насамперед, у путах бідності. Так, у нас є власна держава, але вона майже не
розвиваються. А отже, перебуває у великій небезпеці. Маємо багато деградаційних
тенденцій.. І це теж прямий наслідок того, що дуже багатьом мешканцям цієї
держави, як на мене, виразній більшості, перепрошую на слові, похеру, який
прапор висітиме на міськраді - їх цікавлять лише власні прагматичні справи.. я
знаю, чому все так. Але, на жаль, не знаю, як із цього кола вийти  @igg{fbicon.frown} .

\iusr{Мамина Доця}
\textbf{Serhii Bryhar} 

Хмельницький вчився в ізуітів , то яку він релігію сповідував ?) Залишив би він
Україну католицькою , все могло бути зовсім по іншому , але ...От так дуже
часто Україна опинялась перед вибором без вибору на жаль , бо лайно
неосвідченості це наслідок отого МП , адже католики точно живуть краще ...

\end{itemize} % }

\iusr{Radiy Radutny}
Скажіть їй, що у разі окупації її першу згвалтують.

\begin{itemize} % {
\iusr{Оксана Ковалишин}
\textbf{Radiy Radutny} А з чого Ви вирішили, що її це лякає?


\iusr{Serhii Bryhar}
\textbf{Radiy Radutny} Знищити віру в святе "подстраіваніє", насправді складно..

\iusr{Svitlana Rudych}
\textbf{Radiy Radutny} , грубо!

\iusr{Radiy Radutny}
\textbf{Svitlana Rudych} Повірте, процес буде ще грубіший за мої слова.

\iusr{Lukinov Glib}
\textbf{Radiy Radutny} головне, не Порошенко

\iusr{Тетяна Лук'янова}
\textbf{Radiy Radutny} , не повірить. І буде вона істотою другого сорту.

\iusr{Orest Kochan}
\textbf{Radiy Radutny} лутше бить ізнасілованной чем украінізірованой
\end{itemize} % }

\iusr{Віктор Іліка}
Якщо Україна не потрібна українцям, то ії не буде.

\iusr{Олександр Кондратюк}
Так. Дурнів — багато. На жаль.

\iusr{Ирина Невзорова}

Аж читать противно! И кстати, за разжигание вражды на межнациональной почве
есть уголовная ответственность. А от вас отписываюсь!

\begin{itemize} % {
\iusr{Serhii Bryhar}
\textbf{Ирина Воронцова} Щиро дякую  @igg{fbicon.smile} 

\iusr{Мамина Доця}
\textbf{Ирина Воронцова} Вона ж українка, то де тут міжнаціональний грунт ?)

\iusr{Вікторія Валевська}
І який дивний збіг - українофоб/антивакс

\iusr{Євген Шпичка}
так а де \enquote{разжиганіє}??

\iusr{Вуйко Місь}
\textbf{Ирина Воронцова} Яке горе ,прямо печалька. @igg{fbicon.face.tears.of.joy} 

\iusr{Оксана Ковалишин}
\textbf{Оксана Левит} Ви хочете, щоб у \enquote{женщіни} мізки закипіли?

\iusr{Мамина Доця}
\textbf{Оксана Ковалишин} Це не можливо, бо якби вони були, то не було б про почву

\iusr{Svitlana Chub-Krywuzka}
(грає марш \enquote{Прощаніє славянкі})

\iusr{Вуйко Місь}
\textbf{Оксана Ковалишин} там їх зроду віку не було.

\iusr{Тетяна Лук'янова}
Схоже, та сама потєряна намалювалась  @igg{fbicon.face.tears.of.joy} 

\iusr{Андрій Козачук}

Сергій, хоч і шоколадний заєць, але тут він ретранслював на широкий загал одну
з ватних версій власної інфантильності - я такого надскладного для їхнього
розуму пояснення ще від вати не чув( мо, сам придумав?)....вата - це не
розпалювання, якщо шо

\end{itemize} % }

\iusr{Оксана Ковалишин}
Люди-шлунки

\begin{itemize} % {
\iusr{Serhii Bryhar}
\textbf{Оксана Ковалишин} "Ходячі шлунки". От мені теж згадався образ Григорія Чубая..

\iusr{Оксана Ковалишин}
\textbf{Serhii Bryhar} Мені знаєте що ще смішно? Оті найпростіші постійно видають оте "ти мєня нє панімаєш". Господи, та що там розуміти?  @igg{fbicon.cat.tears.of.joy} 


\iusr{Serhii Bryhar}
...після кожного "нє слишишь", "нє понімаєш", мені згадується: "услиш Донбас".
\end{itemize} % }

\iusr{Наталя Дойкова}

Це наслідок поголовної неосвіченості. Люди не знають історії, не знають і знати
не вважають за потрібне, чим в минулому закінчувалась подібна відстороненість
для відсторонених також

\begin{itemize} % {
\iusr{Lukinov Glib}
\textbf{Наталя Дойкова} не завжди. В Європі освіта непогана, але вирощує здебільшого інфантилів, які хіба "занепокоєність" можуть показати у складній ситуації

\iusr{Наталя Дойкова}
\textbf{Lukinov Glib} в Європі своя ситуація, свої передумови і свої проблеми. В нас - все інше.
\end{itemize} % }

\iusr{Мара Доценко}
Таких море, жаль, що так

\iusr{Людмила Марченко}

Те саме було у Криму, мені приятелька телефонувала, та раділа, що скоро вони
будуть расієй, хотілось би знати, як їй тепер живеться

\begin{itemize} % {
\iusr{Serhii Bryhar}
\textbf{Ludmila Marchenko} Те саме, напевно було і там. Але тут я не бачу радості від потенційно можливого (хоча, варто наголосити, що наразі я не бачу такої аж прямої, безумовної небезпеки, але якась, зрозуміло, є)... Тут - байдужість. Глибока. На рівні: мой дом - Одєсса, а там пусть разбіраются бєз мєня... І навіть впевненість, що підлаштуватися можна під усе... включно з московитами (якщо стане потреба).
\end{itemize} % }

\iusr{Svitlana Rudych}

Дуже важливе питання підняли - тема своєрідної безхребетності малоросів. Але є
ще одна причина, і жінка її озвучила. Це безпека.

Рано чи пізно "вони" прийдуть і в Запоріжжя. Тому родичі просять мене
переїжджати звідси саме через небезпеку наруги і фізичної розправи над
україномовними. Це ж тварини без правил і навіть без голови!

Тому, в певній мірі та жінка має рацію. Хоч, звісно, це не привід забути свою
мову і в мирний час.

\begin{itemize} % {
\iusr{Мамина Доця}
\textbf{Світлана Рудич} Хто "вони", і чому мають рано, чи пізно прийти ?)
\end{itemize} % }

\iusr{Lina Kacmar}

Завжди казала і кажу, що в нас є 30\% заклятих ворогів, 30\% таких ось какаразніц
какбичєвонєвишло і 40 патріотів. Нас більше! Бо оте все болото буде сидіти тихо
по норах і чекати чия візьме?! І, як тільки буде ясно і зрозуміло, хто переміг,
то відразу приєднається до переможців і защебече украінською відразу наступного
дня! Ми не можемо розраховувати на їхню підтримку, але й ворогів наших вони не
підтримають, бо МИ переможемо!

\begin{itemize} % {
\iusr{Serhii Bryhar}
\textbf{Lina Kacmar} Знову ж таки, я не соціолог, щоб говорити про відсотки, але мені здається, що принаймні тут, де мешкаю я, оцих от "внєполітікі", "какаяразніца", "подстроімся", "договорімсяпосєрєдінє", більшість. Їх більше ніж нас, і більше ніж "їх". З одного боку, добре, що більшість становлять не сєпари, але й мої знайомі з Донецька, Луганська, Макіївка, теж кажуть, що більшість там складалася не з тих, хто лише і мріяли увійти у склад Росії, а з тих, кому "нєтразніци"..

\iusr{Lina Kacmar}
\textbf{Serhii Bryhar} по всіх показниках, по цілій країні, а не по регіонах є саме так, як я сказала. Давно відслідковую настрої насєлєнія. Їх більшість, якщо какаразніц додати до ворогів. Але вони не вороги, а тільки ті, які примкнуть до переможців. Чому вони такі? Ота жєнщіна вам ясно сказала - вона скоріше вірить в перемогу раші, хоч і сумнівається. Припускає, що може перемогти Україна, але вона не боіться українців, які будуть розпинати за узкій язик, а московитів боіться! Тому вона буде тільки рада перемозі України.

\iusr{Yura Yakubovsky}
\textbf{Serhii Bryhar} Насправді Ситуація зараз така, що дійсно "нет разніци". Ці люди мислять Матеріальними Цінностями, то ж не треба взивати до якогось Патріотизму. На запитання "Де ви хочете, що ви і ваші діти жили: Україна чи Росія?" будь яка Адекватна Людина наразі відповість: "Америка". Ви правильно кажете, для Перемоги над Ворогом і поверненням своїх територій потрібна Сильна Незалежна Держава, у якої Пріоритет - розвиток ВЛАСНОЇ Національної Культури. От тільки... Україна чомусь УПЕРТО такою ставати НЕ ХОЧЕ!!! І стане хотіти ще не скоро. Тому що у нас Демократична Республіка. Владу в нас обирає Більшість. Більшість складається з "КакіхРазніц". То і за кого проголосує ТАКА Більшість? За таких самих "КакіхРазніц". Поки в нас Демократична Республіка - виходу немає. Можливо колись у нас таки буде шляхом всезагальних виборів повністю Український Президент. Але це буде ще не скоро.

\iusr{Lina Kacmar}
\textbf{Yura Yakubovsky} а може бути і скоро. Не шляхом виборів, звичайно. Бо шляхом виборів не буде ніколи!
\end{itemize} % }

\iusr{Вікторія Стах}
Першими в пеклі опиняться "літеплі". Себто, - ніЯкі.

\iusr{Serhii Bryhar}
\textbf{Вікторія Стах} Так!

\iusr{Вуйко Місь}
Печально, але це боягузи і пристосуванці. Чернь.

\iusr{Олександра Решотко}

Рашоімперії потрібно все в Україні, включно з людськими ресурсами, які ця
імперія перетравлює зі швидкістю ненажерливих термітів. Тому, не варто тішити
себе ілюзіями, що у випадку капітуляції України, рашисти дозволять українцям
плекати щось своє фолькльорно- етнографічне в межах умовної Львівської чи
Франківської області. Так не буде, для росії і російських імперошовіністів є
неприйнятним сам факт існування українців і української України. Отже, вони
будуть діяти в уже неодноразово відпрацьованій ними самими парадигмі: "нє била,
нєт і бить нє может". Відтак, навіть, такі нібито українці, яким плювати на
Україну і все українське, росії потрібні тільки, у якості дешевої рабсили не
бескрайніх прасторах Сибіру і Далекого Сходу. Зрештою в українців, які це
розуміють нема іншої альтернативи- треба бути здатними і готовими давати відсіч
рашоорді на кожному кроці.

\begin{itemize} % {
\iusr{Serhii Bryhar}
\textbf{Олександра Решотко} 

Тут я погоджуся. Ті, хто мешкають у Львові чи Тернополі (я, до речі, теж хотів
би мешкати у Львові чи Тернополі - задовбало мене регулярне отакеот... - не
бачу сенсу це приховувати), і вважають, що у випадку, якщо відпадуть Харків і
Одеса, жити стане простіше, серйозно помиляються. Не стане. Бо наступними
будуть уже вони (ну може через кидок - ще ж є центр). Московитам важливо душити
Україну до кінця! Отже, так, південь і схід - теж Україна! Їх необхідно
берегти. І ситуація наразі дійсно загрозлива. Враховуючи потужні впливи
заплрєбріка і місцеву специфіку, я думаю, адекватна влада серйозно б задумалася
над впровадженням військово-цивільної адміністрації, ґрунтовною підготовкою до
потенційної агресії, масовою роз'яснювальною роботою серед "какаяразніц" (до
когось могло би і дійти)... Але ж нє Зє це забезпечуватиме(.

\end{itemize} % }

\iusr{Ntina Ntoubrova}

Оксана Забужко дуже добре це описала: ці люди просто роблять вибір: бути
частиною місцевої адміністрації колонізаторів - чи бути колонізованим, але
непокірним населенням. Їм хочеться бути саме адміністрацією. Заради цього вони
вивчать російську мову, водитимуть своїх дітей на фестівалі з кокошниками,
віддадуть своїх школярів в військові табори колонізаторів, щоб з них змалечку
ліпили сталевих агресивних імперських солдат та диверсантів-яничарів (а яничари
зазвичай набагато жорстокіші та байдужіші, ніж навіть представники імперської
національності чи верстви). І вони найбільш жорстко переслідуватимуть лідерів
колонізованого але непокірного населення: 1) щоб отримати вхід на верхню
сходинку та 2) щоб не було тих, які від них чимось відокремлюються (бо це
викликає ненависть, банально: повія люто ненавидить цнотливу жінку). Все це
відбулося і продовжує відбуватися в Криму та ОРДЛО. І ці люди готові зробити те
саме в Одесі. Чому? Бо в них немає СПІВЧУТТЯ, МИЛОСЕРДЯ та ГУМАНІЗМУ. Вони
звичайні мавполюди БЕЗ МОРАЛІ, які живуть звіриними поняттями ((


\iusr{Svitlana Chub-Krywuzka}

\enquote{Єслі оні нас захватят, а ми тут по-укрАінскі гаварім}
Ах, то он воно що.

\begin{itemize} % {
\iusr{Orest Kochan}
\textbf{Svitlana Chub-Krywuzka} дуже безпечно бути проти українців. То нічим не загрожує, а може приносити прибуток.


\iusr{Serhii Bryhar}
\textbf{Orest Kochan} Україна - це країна, в якій найзручніше бути "русскім".

\iusr{Orest Kochan}
\textbf{Serhii Bryhar} бо в тебе купа прав, тебе захищають як росіянці так і захід, та й місцеві теж зазирають до рота. І жодних обов'язків.
\end{itemize} % }

\iusr{Svitlana Chub-Krywuzka}
Між іншим, геть від раші - це економічно вигідно, про це прагматична дамочка не думала?

\iusr{Orest Kochan}
\textbf{Svitlana Chub-Krywuzka} напевне ні. В неї говорить страх, а раціо включається для того, щоб той страх логічно пояснити.

\iusr{Людмила Демянчук}

Це все наслідок знищення українства, духу свободи і волі напротязі століть.

Відчайдухи гинули, а сіра маса в отій хаті скраю, преспокійно переживала
скрутні часи, а коли ті часи в Україні були нескрутні ?

От і маєм штучний відбір, попсований генетичний фонд ( хай як жахливо це не
звучить! )

Слава Богу, що волелюбний дух наших славних предків не знищили докінця, Україна
як фенікс щораз відроджується, але може пора спинитись і таки втримати
суверенну державу ?!!

\iusr{Лариса Гуріна}

Ризикну припустити, що це наслідок суспільних травм. Як не крути, але найбільша
вірогідність вижити - у нащадків пристосуванців, які з покоління в покоління
передають свій досвід. Мімікруй, не висовуйся, голосно не висловлюй свою думку,
а краще - взагалі мовчи, підлаштовуйся та прогинайся - відголоски цієї нехитрої
«життєвої мудрості» можна бачити у багатьох проявах життя суспільства.

\iusr{Оксана Ковалишин}
\textbf{Лариса Гуріна} Блискуче сказано.

\iusr{Anatoliy Lustyk}
По Донецьку було 60\% таких. Інші - 20\% проукрїнські, 20\% - проросійські.


\iusr{Serhii Bryhar}
\textbf{Anatoliy Lustyk} Ну от, власне: думки донеччан, які вчасно виїхали, щодо цього питання, +- збігається.

\iusr{Jurko Zełenyj}
\textbf{Anatoliy Lustyk} sze poviru w 2\% pro-ukriwski na Dambassie, ale aż nijak ne 20\%!

\iusr{Наталия Швачий}
Знайомі слова...

\iusr{Юлия Кривенко}
На жаль.. чую це часто...

\iusr{Назарій Біленький}

Нехай мене пробачать люди, але таких треба стріляти задля життя нормальних
людей і їхніх дітей. Просто стріляти, з часом звикнуть, вони ж вміють
адаптуватися під реальність, а так як їх кожного дня буде менше, бо будуть
відстріляні, то і Україна вдихне свободу на повні груди!

\begin{itemize} % {
\iusr{Orest Kochan}
\textbf{Назарій Біленький} виглядає, що тільки тоді до них дойде, що треба перейти на українську, бо бєзопаснєє разговарівать па украінскі.

\iusr{Serhii Bryhar}
\textbf{Назарій Біленький} Це ж заклик до насилля..

\iusr{Віктор Фещенко}
А хто має стріляти, перепрошую??

\iusr{Назарій Біленький}
\textbf{Serhii Bryhar} це заклик зберегти Україну і наших дітей

\iusr{Назарій Біленький}
\textbf{Віктор Фещенко} а хоч би хто, от якби мода така пішла, як на кльош в 80-ті

\iusr{Назарій Біленький}
\textbf{Orest Kochan} а як на мене краще стріляти

\iusr{Мамина Доця}
\textbf{Назарій Біленький} Оце бажання стріляти всіх то часом не з совка, паночку ?) Думаєте мало бажаючих вас завалити ?) Так отож ...

\iusr{Назарій Біленький}
\textbf{Оксана Левит} ви праві, саме з совка, через розстріл совковим людям тільки і доходить, більш ніяк

\iusr{Мамина Доця}
\textbf{Назарій Біленький} Вам же ж абсолютно не дійшла проста істина про те що людей вчити потрібно , бо стріляли мільйонами , а дурні як були так і є

\iusr{Назарій Біленький}
\textbf{Оксана Левит} чому ви такі тупі що доказуєте свою маячню? Ми століттями будуємо щось хороше, а потім приходить якесь гівно і за секунду все рушить, ми тут в програшній позиції бо нам є що втрачати, а тупим москалям немає чого втрачати крім дирки з гівном у дворі!

\end{itemize} % }

\iusr{Тетяна Лук'янова}

Був би у цієї істоти мозок - був би струс. А ще допетрала би, що вона та її
діти є і будуть для росіян "своїми " тільки на словах. Та й то поки їм вигідно.
А так - хохли і всьо  @igg{fbicon.face.tears.of.joy} 


\iusr{Роман Клочко}

Згадується книжка Ірини Реви "По той бік себе" про соціально-психологічні
наслідки Голодомору і політичних репресій. Оце от пристосуванство - один з них.

\begin{itemize} % {
\iusr{Serhii Bryhar}
\textbf{Роман Клочко} Фактично - ключовий. Є й інші, і кожен із них, на жаль, безпосередньо заважає як гармонійно у розвитку особистості, так і нормальному розвитку національної спільноти.

\iusr{Роман Клочко}
\textbf{Serhii Bryhar} так. І долати його, як і інші, ще довго доведеться.
\end{itemize} % }

\iusr{Сергій Танасов}

Желудочнікі упускають момент самого процесу, можливої, окупації. Тобто бомби на
голови, тисячі смертей, репресії... Вони одразу уявляють "великі зарплати".


\iusr{Юлия Горда}
Божечки, як гірко!
Як боляче...
Стільки смертей за те, щоб Україна була українською, а нас оточують такі...

\iusr{Yaroslava Zheyko}
Вони уявляють, що бути росіянами безпечно. Не знають, що в росіє нікому не безпечно, крім путлєра і його прідворних

\iusr{Юрій Олександренко}

такі вони - нормальні люди. їх більшість. а меншість - свідома. меншість має
будувати ту реальність, під яку нормальна більшість має підлаштовуватись. і це
треба усвідомити і не чекати свідомості від нормальної більшості. ненормально
очікувати від більшості свідомості. :)))

\begin{itemize} % {
\iusr{Serhii Bryhar}
\textbf{Юрій Олександренко} І це теж вірно. Проте, варто зауважити, що без роботи в напрямку покращення людського капіталу, отого ключового ресурсу, розвиток стане такою ж утопією, як в Росії та в Білорусі. Здається, навіть от просто освічених людей в нас тут більше, ніж там, на диких землях, однак це ще нічого не гарантує. Більше ніж там, але значно менше, ніж у сусідів на Заході. Нас достатньо, щоби скинути лайно з трону, але замало, щоби потім облаштувати дім за нормальними правилами..

\iusr{Юрій Олександренко}
\textbf{Serhii Bryhar} головне правильне розуміння, а потім - правильні підходи. щодо кількості, то наполеон мав 70 тис солдатів - а москва згоріла  @igg{fbicon.smile} 
щодо виховання, то це... авторитарний процес. і демократія та лібералізм тут не допоможуть. одна заборона (реальна) - робить на кілька порядків більше, ніж найкраща лекція.

\iusr{Roman Sorokhtei}
Панове, ви доповнюєте один одного на 1000\%!
\end{itemize} % }

\iusr{Dmytro Dzyuba}

...дедалі частіше здається, що таких покручів, пристосуванців і ватніків у нас
90\%...

\begin{itemize} % {
\iusr{Serhii Bryhar}
\textbf{Dmytro Dzyuba} Я думаю, все ж таки, менше (патріоти, умовні патріоти, також вороги і майже вороги, ну і ті, кому все по цимбалах - їх, так, звісно, дуже навіть дофіга), але навіть якби нам вдалося визначити той відсоток, було б сумно.
\end{itemize} % }


\iusr{Semeniuk Oleksandr}
Люди-шлунки так і мислять, але далеко не всі мають сміливість то оприлюднювати.

\iusr{Oleksandr Master}

Тут ще важить що розмова була чоловіка з жінкою. По природі жінки більш
обережні, менше схильні ризикувати, скоріше підлаштуються. В складних умовах
завдання жінки не лише народити, а ще виростити любою ціною. Це Тарас Бульба
міг вбити рідного сина за зраду. А жінка швидше сама зрадить щоб син жив. В
Холодному Ярі є епізод як жінка через зраду хотіла спасти коханого. Можливо я
тут занадто виразно показав відмінність, бо в житті це менш помітно, але ідея
надіюсь зрозуміла.

\begin{itemize} % {
\iusr{Serhii Bryhar}
\textbf{Олександр Мастер} 

Але в нас відбуваються ті процеси, які, власне, змінюють акценти. От навіть
якщо згадувати місцевий майдан, то відразу можу виокремити провідну роль жінок.
В них достобіса лідерських якостей, активності, сміливості. Звісно, тут можна
сказати, що жінки діляться на активних та звичайних, але ж і чоловіки - також.
Звісно, жінку в образі Тараса бульби уявити не вийде. Але чи можливо уявити в
такому образі сучасного чоловіка? Я не знаю... Просто зараз мені здається,
гендерні особливості настільки розмилися, а десь навіть деформувалися, що вже
чоловіки ідуть за жінками на барикади. Але також варто зауважити, що питання на
зразок "боротися за нормальну державу, чи підлаштуватися під гниль?", або
"виїжджати з окупованої ворогом території, чи ні", це, все ж таки, більше
питання, що стосуються інтелектуальних здібностей, практичного розуму. Навпаки,
обережна мати мала би собі розуміти, що толерувати вороже, підлаштовуватися під
нього - це дуже небезпечно... Принаймні, це моя суб'єктивна думка.

\iusr{Oleksandr Master}
\textbf{Serhii Bryhar} все є, і розумні матері, і сміливі жінки, і пасивні чоловіки. Але в середньому чоловіки відрізняються від жінок (чи навпаки). Статеві відмінності стираються, але ж не зникають повністю.

\iusr{Serhii Bryhar}
\textbf{Олександр Мастер} Добре. Це зрозуміло. Але я, все ж таки, наполягаю на тому, що питання толерувати вороже, чи ні, це площина рівня освіченості, а не сміливості, активності, пасивності. Коли я чую, що "лучше уж бить (ну фактично) русскімі", то відразу ж думаю, що переді мною дурна людина. І найбільшу небезпеку вона несе для себе та близьких. Але, в якійсь мірі, звісно, - і для інших.

\iusr{Oleksandr Master}
\textbf{Serhii Bryhar} освоєння верхніх щаблів піраміди Маслоу допомагає із забезпеченням нижніх (базових) рівнів, включаючи безпеку. Але зрозуміти це може лише той підніметься наверх. Це як ходити - спочатку страшно бо можна впасти, а потім розумієш яка це перевага (хоч деколи все таки падаєш).

\iusr{Неонила Дрыгуш}
\textbf{Serhii Bryhar} Повністю підтримую Вашу думку!

\iusr{Оксана Ковалишин}
\textbf{Serhii Bryhar} 

От що до освіченості, тут я з Вами не погоджуюся. У освічених це просто
замасковано якоюсь софістичною дурнею. Здедільшого для власного заспокоєння.
Інакше, як людина за плечима якої університет, а то і не один, може називати
чужу мову рідною? Настільки дурне, щоб плутати мову роду з мовою вибраною, щоб
пристосуватися? Я навіть не знаю. Думаю, все ж, що це робиться для того, зоб
брехати собі. Вони ж хочуть бути елітою. Чи інтелігецією. А ці речі і
пристосуванство не сумісні.

\end{itemize} % }

\iusr{Jurko Zełenyj}
vyżyvajut najbilsz prystosowni!
E — Evolücïa...
 @igg{fbicon.face.smiling.sunglasses}  @igg{fbicon.hot.beverage} 

\begin{itemize} % {
\iusr{Serhii Bryhar}
\textbf{Jurko Zełenyj} От наприклад я... Насправді ж дуже некомпромісна людина. Завжди борюся, відстоюю, трансформую... Поки що виживаю). Може, звісно, не так, щоб аж за найвищими стандартами, але й не так, щоб аж погано). Отже, бути нонконформістом не смертельно навіть в українських реаліях.

\iusr{Jurko Zełenyj}
\textbf{Serhii Bryhar} ale jikszo wzėty w tryvaliszim vidrizku czėsu, to my - vymerajuczyj vyd...  @igg{fbicon.smile} 
\end{itemize} % }

\iusr{Tetyana Rozumenko}
так це ж нащадки колонізованого народу, який підлаштовується і не йде проти. це зрозуміло. постгеноцидна нація.

\iusr{Maria Kozyrenko}
порадьте їй одразу китайську вчити, ну щоб без оцих нікому не потрібних проміжних етапів

\iusr{Юрій Дубовик}
\textbf{Maria Kozyrenko} швидше ідиш з івритом... Хто при владі в Україні і на раші? А Китай то так для залякування і "лапша" на вуха...

\iusr{Елена Швець}

Не пам'ятаю де прочитала це на днях, але звучало приблизно так \enquote{якщо помічаєш
що ти згодний з більшістю, значить пора задуматись що щось не так}. І воно таки
правильно. Особливо як згадати вираз Любомира Гузара про стадо.

\iusr{Мамина Доця}
На жаль це буде ще довго, хоч раптом 30ть років тому ми лиш мріяти почали про те щоб здохла компартія ...

\iusr{Mashinskaya Anna}

А найцікавіше, що ця істота озвучила наратив, який нещодавно так непомітно
намагався вкласти в голови більш інтелектуальної публіки нецій деніс, здається,
казарін. Там він також ніс якусь ахінєю про прагматизм, піраміду маслоу, і все
інше, чим вербували українців ще років так сто тому.

Істоті просто варто нагадати, як саме зараз живуть в приєднаних до росіюшкі
територіях, і як живуть в неприєднаних і невизнаних ордло) І що найменш цікаво
там зараз жити саме дітям шкільного віку, і наскільки їм складно потім, коли
треба вступати в виші - як вони нікому не потрібні в росії, і як їм важко
приходиться в Україні.

Вона такого бажає своїм дітям?)

\iusr{Olga Tumenko}
Я таких називаю двужопими пристосуванцями. Таких багато є в Криму.

\iusr{Андрій Сокол}
Питайте в таких випадках, чи \enquote{підстроївся} під німців її дід в 1941-му.

\iusr{Василь Фелоненко}

Сумно, але це готовий продукт багатолітнього національного поневолення які
зводять свої цінності життя- хлібом насущним і видовищами

\iusr{Сергій Лащенко}

Звісно, ця молодиця повна дурепа і ватниця. Якщо вже так боїться за свою шкуру
і за дітей, то повинна б розуміти, що вивчивши українську, російську теж не
забудеш. Порозумілася б з окупантами навіть у випадку їхньої перемоги. Але в
цілому ймовірність перемоги москвинів різко зменшиться, коли наші діти
знатимуть і любитимуть українську.

\iusr{Андрей Роик}
Ось тому треба зробити так, щоб бути не українцем було сцикотно

\begin{itemize} % {
\iusr{Наталя Хоруженко}
\textbf{Андрей Роик}, ні силою добра не зробиш.


\iusr{Serhii Bryhar}
\textbf{Roik Andriy} Гарно сказано! Але я погано розумію, як це зробити. Десь наших достатньо. Десь - так собі. Десь і зовсім мало..

\iusr{Юлия Кривенко}
\textbf{Serhii Bryhar} а там де треба (школа, садочки),то з цим біда!

\iusr{Андрей Роик}
\textbf{Serhii Bryhar} Поступово. Через заохочення та заборони.
З одного боку щоб той хто гаваріт па русцкі сприймався нарешті як невиховане бидло та колгосп. З іншого щоб були ризики потрапити до резервації якщо що.
\end{itemize} % }

\iusr{Maryna Byshenko}
А потім дітки захочуть вчитися в Києві і тут перевчаться.

\iusr{Atmeshvar Anatoly}
І ще додам - їх 73\% ))

\iusr{Валентина Величко}

Я вже скоро сказюся від такого оточення. Діти у дворі сміються з української
мови, а на питання, в якій країні вони живуть, відповідають "росія"  @igg{fbicon.frown} 

\begin{itemize} % {
\iusr{Сергій Лащенко}
\textbf{Валентина Величко} Шматочок \enquote{ватної} Полтавщини? Кременчук доведеться дуже скоро \enquote{зачищати} від ватного бидла. Не фізично, але тиском і вихованням нового покоління

\iusr{Валентина Величко}
\textbf{Сергій Лащенко} Я зараз у дітей в Одесі

\iusr{Сергій Лащенко}
\textbf{Valentina Velichko} Одеса? О, так - ватність там зашкалює! Але й там небезнадійно.

\iusr{Nina Dmytrenko}
\textbf{Valentina Velichko} яке жахіття!
Це Ви живете в оточенні \enquote{пєрєсєленців}

\iusr{Валентина Величко}
\textbf{Ніна Дмитренко} А 87\% прАгАласувало за зє - це ж про щось говорить  @igg{fbicon.frown} 
\end{itemize} % }

\iusr{Павло Арей}

Стандартна російська ментальність - визнання сили та агресії, покірність перед
силою та агресією. Я б такій апелював би, що коли росіян переможуть бандерівці,
то вони її по голівці не погладять точно так само)))


\iusr{Алексей Гаврилов}
Я за те щоб державна мова була одна.
Але.
Що заважає існуванню, використанню і функціонуванню другої мови паралельно?
Язик повинен бути як у гадюки чи щелепи іншого крою?
Що принципіально заважає?

\iusr{Співак Дарія}
Це пристосуванці. Таких багато. І наше суспільство їх продукує кожний день.

\iusr{Сергій Тимощенко}

Людиною рухає страх, вона розуміє, що з московитами говорити не московською
небезпечно, натомість в Україні безпечно говорити як завгодно. Тут ні держава,
ні суспільство ніяк не реагує на це. Московити відверто цькують і глумляться
над тими, хто говорить інакше. Українці ж дозволяють цькувати себе. У своїй
країні.

\iusr{Анатолій Кухарчук}
Шлункоголові

\iusr{Руслан Шеремета}
Навіть в середовищі вчителів таких знгаю.

\iusr{Yura Yakubovsky}
Власне та жінка чітко розставила свої пріоритети. У неї на першому місці її діти. Все решту для неї не має значення.

\end{itemize} % }
