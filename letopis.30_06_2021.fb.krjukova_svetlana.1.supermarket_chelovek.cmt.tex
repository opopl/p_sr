% vim: keymap=russian-jcukenwin
%%beginhead 
 
%%file 30_06_2021.fb.krjukova_svetlana.1.supermarket_chelovek.cmt
%%parent 30_06_2021.fb.krjukova_svetlana.1.supermarket_chelovek
 
%%url 
 
%%author 
%%author_id 
%%author_url 
 
%%tags 
%%title 
 
%%endhead 
\subsubsection{Комментарии}
\label{sec:30_06_2021.fb.krjukova_svetlana.1.supermarket_chelovek.cmt}

\begin{itemize}
\iusr{Анатолій Фесенко}
А что Ви на ночь употребляете, коль зашли в продуктовий супермаркет?😉

\iusr{Svetlana Kryukova}

Анатолий Фесенко тю, Оливье с майонезом, что же ещё

\iusr{Анатолій Фесенко}
Svetlana Kryukova жестко!

\iusr{Петр Вдович}
Света, хватит писать о политике) Все равно там все и так понятно...
Начинайте писать о...жизни, о любви, о незащищённым сексе, в конце то концов) И особенно актуально, о пробках. Там уж точно...абсолютное равенство, и Жигулей и Мозератти)🙂

\iusr{Svetlana Kryukova}
Петр Вдович Гужва Игорь не разрешает)

\iusr{Olena Malysheva}
Люблю Вас читать. Очень интересно и содержательно. У вас талант.

\iusr{Петр Вдович}
Olena Malysheva один недостаток все же есть...замужем))

\iusr{Дмитрий Акимов}
"...Я стою еще некоторое время у окна и смотрю в ночной мрак, а потом начинаю стыдливо и совсем беззвучно шептать слова и фразы, которые мне очень хотелось бы когда-нибудь сказать кому-то, да вот некому, разве только Изабелле – хотя она даже не знает, кто я. Но кто из нас действительно знает, что такое другой человек?" ©Э.М.Ремарк

\iusr{Svetlana Kryukova}
\textbf{Дмитрий Акимов} красиво, черт побери, собирает слова в предложение

\iusr{Дмитрий Акимов}
\textbf{Svetlana Kryukova} кто бы говорил 😉 Очень красиво и у Вас

\iusr{Anton Egorov}
Я думаю, кто это возле аквариума с карпами в Новусе, на Печерске, посты в телефоне пишет😂😎🤘👌🔝
\iusr{Dankovicsné Ljudmilla}
Спасибо, Светлана. Добром повеяло от Вашего поста.. теплом и.. домом...
\iusr{Vitaliy Pantin}
Солидарен. Ждем наблюдений об охранниках и таксистах. По моим наблюдениям эта прям пятая колонна из "сбитых летчиков". И при умелом развертывании, учитывая там военных и других грамотных спецов то может и бахнуть
\iusr{Vitaliy Lukyanenko}
Это новый для вас жанр повествования. Можно только приветствовать! Но и потрудиться придётся много. Пока скучно и банально. Надеюсь, дальше пойдет лучше.
\iusr{Orlova Sabina}
Согласна с вашей бабушкой
\iusr{Michael Kot}
"Я в супермаркетах люблю потеряться, чтобы писать там посты, где-то у аквариума с карасями, или часами глазеть на людей." 🤦
\iusr{Sergiy Solomchenko}
Шикардос
\iusr{Роман Юсюк}

Тема для картины «22:15» ?)

\iusr{Dmitri Goncharuk}

Вот за столиком дама на даме панама под ней томный взгляд
Но панама упряма и клюет на панаму уже двадцать восьмой кандидат
И ее состоянье назовем ожиданьем

\iusr{Игорь Зубань}

Такие талантливые журналисты работают на издания, проталкивающие в Украине росиймкие интересы.
Жаль.
Впрочем, за 2 года знакомства лично от вас ни одного пророссийского текста не слышал.
Чего не скажешь о СЕГОДНЯ, в целом.
Ок, пишите. Читаю. Интересно.

\iusr{Станислав Ахмедов}

Отличный пост!
Если позволите, добавлю.
Моё любимое время наблюдения за особыми покупателями - это днём.

"Особые"- это пожилые бабушки и дедушки. Они приходят после получения пенсии,
предварительно оплатив коммуналку. Их редко интересуют деликатесы. Чаще они в
бакалейных и кондитерских рядах. Они иногда спорят, складывая цифры на
ценниках, иногда просто грустно проходят мимо дорогих товаров. Мне иногда
удаётся уговорить бабушку и купить такой себе продуктовый набор расширенного
ассортимента. Только собирать его в тележку , надо без их присутствия, иначе
добрую половину не дадут купить. Просто попросить подождать за кассой, в
условленном месте.

Чаще всего , я покупаю мягкие колбасы, мясо, курятину, хороший Голландский сыр,
качественное сливочное масло и растительного тоже пару бутылок. Беру гречку,
рис, чай и к чаю вафли, печенье. Ну и так ещё по- мелочи. Молочку и овощи не
покупаю - не донесут.

Пару пакетов очень привычных, но недоступных для них продуктов, делают их
счастливыми. Всегда стараюсь предложить подвести, но редко это удаётся сделать
- они ужасно стесняются и не хотят дополнительно обременять, иногда боятся,
чувствуя какой- то подвох.

Одним словом- это приятный день для меня и маленькое счастье , для них.

Честно, я этот позор за обездоленных стариков, проживших тяжелейшую жизнь и
реально построивших нам нашу страну, чувствую кожей, сердцем, душой. А вот умом
понять не могу! Таким способом, исключительно адресным, себя успокаиваю.

Если хотите, попробуйте, вам это точно понравится.

\iusr{Юлия Артёменко}

Утренние супермаркеты тоже хороши) Ожидание возможности в 10:00 купить
алкоголь, за которым уже с ночи мучаясь, наматывая круги вокруг магазина,
сбивались в стаи самые активные жаворонки и самые неугомонные противники
ночного сна и утреннего бодрящего кофе, которых кофе уже не бодрит 😆

\iusr{Dmytro Nogal}

Это смотря в какой гастроном зайти.

\iusr{Salaudin Akagadgiev}
Поставьте лайки кто внимательно прочитал этот пост от начала до конца.😑

\iusr{Evgeny Tsyganov}
и после 21-00 когда 50\% скидки на кулинарию ))))

\iusr{Katarzyna Alex}
Для этого есть адресная доставка из тех же супермаркетов))Плюс к тому же, куча супермаркетов в киеве круглосуточная

\iusr{Ирина Малина}

Супер! Хорошо пишете! Заматерели! И напоминает раннего Жванецкого, ещё
романтичного реалиста!)Если бы не знала, что Светлана киевлянка, то можно смело
решить, что одесситка!;)👏💪🌟Пишите- это ваше👌🏻

\iusr{Igor Gayvan}
Если не нравится ходить в супермаркеты, то, может быть, не стоит этого и делать? Сейчас все продукты можно заказать онлайн

\iusr{Вадим Карпяк}
Ну, такое.. Троещенский мопасан


\end{itemize}
