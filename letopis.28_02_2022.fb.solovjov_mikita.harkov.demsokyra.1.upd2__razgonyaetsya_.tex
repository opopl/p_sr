%%beginhead 
 
%%file 28_02_2022.fb.solovjov_mikita.harkov.demsokyra.1.upd2__razgonyaetsya_
%%parent 28_02_2022
 
%%url https://www.facebook.com/Mikita.Solovyov/posts/pfbid0RX7CbHEVWMDyGoTqnFEtxKw7hrADcaEDJHuYLQtyNBVh5jqYK5eJCzdwTNWDeS4tl
 
%%author_id solovjov_mikita.harkov.demsokyra
%%date 28_02_2022
 
%%tags 
%%title UPD2: Разгоняется волна паники
 
%%endhead 

\subsection{UPD2: Разгоняется волна паники}
\label{sec:28_02_2022.fb.solovjov_mikita.harkov.demsokyra.1.upd2__razgonyaetsya_}

\Purl{https://www.facebook.com/Mikita.Solovyov/posts/pfbid0RX7CbHEVWMDyGoTqnFEtxKw7hrADcaEDJHuYLQtyNBVh5jqYK5eJCzdwTNWDeS4tl}
\ifcmt
 author_begin
   author_id solovjov_mikita.harkov.demsokyra
 author_end
\fi

UPD2: Разгоняется волна паники. О сотнях и чуть ли не тысячах погибших, об
уничтоженных целых микрорайонах и т.д. Цель та же, что и самих обстрелов.
Вызвать волну паники и сломать ей. Это ложь! Возможно, у меня неполная
информация приведена. Даже скорее всего еще не все данные собраны. Но ни о
каких сотнях погибших речь не идет и близко. Ни один многоквартирный дом не
разрушен. 

UPD: Пока писал, продолжались обстрелы. Уже и по Харькову прилетает много
тяжелой арты. Общее количество жертв больше 7 погибших и 44 раненых.

Сегодняшние новости по Харькову. 

В районе часа дня был обстрел. В этот раз обстреливали Градами (возможно, еще
каким-то РСЗО, я не могу точно сказать) жилые кварталы. По Изюму валили
неуправляемыми авиабомбами. В Харькове обстреляли район 23-августа-Деревянко,
Сев. Салтовку. Клочки. Возможно, еще где-то, подробной информации у меня нет. 

Минимум одна жертва, больше десяти раненых. В основном это люди, которые в
момент обстрела были на открытых пространствах в непосредственной близости от
места прилета. По городу некоторое количество неразорвавшихся ракет и их боевых
частей. 

Диверсионных групп и прочих потеряшек сегодня в городе не было. Но это не
потому что их к нам не отправляли. По неофициальной информации, их сегодня на
нашем направлении в Украину зашло даже больше чем вчера. Но ЗСУ решило, что
сафари в городе это перебор. И вместо розничной перешло к оптовой переработке
биомусора. Как минимум две больших колонны были уработаны прямо на входе,
задолго до подходов к Харькову. 

За последние сутки восстановлен контроль над несколькими населенными пунктами
вокруг Харькова. "Пояс безопасности" вокруг Харькова расширяется. Но к
сожалению, еще не получается (и до этого еще далеко) сделать невозможными
обстрелы. Рашистские фашисты еще контролируют довольно много мест, из которых
возможны артиллерийские и рсзо обстрелы города. Поэтому в отношении обстрелом
расслабляться еще рано. 

Если в первые дни обстрелы велись в первую очередь по объектам инфраструктуры,
в первую очередь военной, то теперь противник перешел к неизберательному огню
жилых массивов. С сугубо военной точки зрения, эти обстрелы никакого смысла не
имеют. Я вижу только два объяснения, скорее всего имеет место их комбинация:

1) Основной целью является паника харьковчан и наши требования любой ценой остановить войну. 

2) Рашисты убедились в полной невозможности военной победы и теперь им уже все
равно кому именно наносить урон. Их интересует только максимальный суммарный
урон нам всем. 

Ситуация с снабжением продуктами и медициной сложная, но не ужасная.
Супермаркеты и часть сетевых магазинов работает, хотя с ограничениями и далеко
не все. Аптеки части сетей аналогично. Хлеб завозится, с очередями и т.д., но
его купить можно. Электричество, вода и связь в городе есть. Перебои изредка
бывают в небольшой доле районов, скорее даже микрорайонов. 

Вчера наконец до Харькова доехал груз Джавелинов, НЛАВов, Стингеров и других
плюшек. Если до того наши ребята успешно отбивались практически без них, то
сейчас прошло насыщение высокотехнологичным вооружением первой линии. С ним
утилизация орков пойдет успешнее.

Выше была известная мне информация, максимально без моих выводов и
предположений. Теперь мои выводы и личные впечатления. Здесь я не буду
стараться четко разделить чистые наблюдения и свои по ним выводы.Я не аналитик,
тем более не военный эксперт, эти оценки не являются профессиональным анализом. 

1) Харьков УЖЕ выстоял. И это уже стало абсолютно понятно даже командованию
рашистской швали. Переброс заходивших к нам сегодня ночью/утром частей были
последней значимой попыткой переломить ситуацию. Частично по инерции, частично
еще в надежде изменить расклад. 

2) Логистика, взаимодействие с соседями, взаимодействие между ЗСУ, ТрО и
другими подразделениями выстраиваются. Немного сумбурные (естественно) действия
в обороне первых часов вторжения уже сменились планомерной работой. Эффект
неожиданности уже полностью закончился. 

3) Боевой дух, мотивация и в целом эмоциональное состояние захватчиков на
уровне плинтуса и ниже. Перехваты их переговоров по обе стороны границы
однозначно показывают, что они в шоке, чувствуют, что их гонят на убой. Ищут
любые способы не входить в Украину. Многие сдаются в плен не в безнадежной
ситуации, а умышленно. И наконец до рядовых бойцов дошло, что освободителями их
здесь не считает никто.

4) Наши защитники наоборот, имеют желание и мотивацию убивать эту шваль и
очистить Украину от них. Они знают за что воюют и каждый день видят тех, кого
защищают. Кроме того, в отличие от противников, они уже распробовали вкус
победы.

5) Настроения харьковчан очень разные. Но первый шок начал постепенно
проходить. Сегодняшний обстрел и близко не привел к панике. Страх есть, но
ненависти намного больше. расчет на создание паники не просто не оправдывается.
Обстрелы мирного населения приводят к обратному результату. Вместо паники и
протестов против наших властей, идет консолидация общества. И пока наша власть
активно и последовательно дает отпор, ее поддержка только возрастает.

6) Общество саморганизуется. В волонерскую деятельность и взаимопомощь
включается все больше людей. В том числе тех, кто раньше подобным не занимался.
Только ко мне напрямую обращается какое-то запредельное количество людей,
ищущих как и кому помочь, где они могут быть полезными. Например, только за
вчерашний день трое людей через меня разобрались как и кому передать свои
автомобили нашим защитникам. Большинство запросов на помощь закрываются очень
быстро, а потом еще продолжают добегать люди, готовые сделать тоже самое, после
закрытия задачи. Это касается как помощи нашим защитникам, так и друг другу. 

Наша победа в этой войне уже очевидна. Вопрос сейчас только в том, какую цену
еще придется за нее заплатить.

Слава Украине! 

Низкий поклон нашим защитникам!
