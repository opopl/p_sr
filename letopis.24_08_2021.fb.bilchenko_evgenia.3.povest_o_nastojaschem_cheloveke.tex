% vim: keymap=russian-jcukenwin
%%beginhead 
 
%%file 24_08_2021.fb.bilchenko_evgenia.3.povest_o_nastojaschem_cheloveke
%%parent 24_08_2021
 
%%url https://www.facebook.com/yevzhik/posts/4181914895176922
 
%%author Бильченко, Евгения
%%author_id bilchenko_evgenia
%%author_url 
 
%%tags bilchenko_evgenia
%%title БЖ. Повесть о Настоящем Человеке
 
%%endhead 
 
\subsection{БЖ. Повесть о Настоящем Человеке}
\label{sec:24_08_2021.fb.bilchenko_evgenia.3.povest_o_nastojaschem_cheloveke}
 
\Purl{https://www.facebook.com/yevzhik/posts/4181914895176922}
\ifcmt
 author_begin
   author_id bilchenko_evgenia
 author_end
\fi

БЖ. Повесть о Настоящем Человеке.

Поздравляю вас с днём человека. Маленького человека. Большого человека. Очень
мужественного человека. Он не шел на компромисс и не делал из себя дурака. Он
был настоящим. Сегодня - его день. Евпл.

\ifcmt
  pic https://scontent-cdg2-1.xx.fbcdn.net/v/t1.6435-9/240472179_4181914578510287_341638295454133615_n.jpg?_nc_cat=102&_nc_rgb565=1&ccb=1-5&_nc_sid=8bfeb9&_nc_ohc=KYk1wnJ2dRMAX_qlYjN&_nc_ht=scontent-cdg2-1.xx&oh=8cd09a09767be36a1adb747a7f5375b4&oe=614C1D4E
  width 0.4
	fig_env wrapfigure
\fi

В девятое консульство Диоклетиана и восьмое консульство Максимиана, в третий
день майских календ, диакон Евпл пришел в суд города Катании и громким голосом
исповедал: «Я христианин и хочу умереть за имя Христово!» Магистрат Кальвисиан
приказал немедля привести к нему смельчака, дерзающего произносить такие речи.
Когда Евпла ввели к нему, в руках у него было Евангелие. Иллюстриссим Максим
заявил, что он держит книги, запрещенные императорами. «Где ты взял эти книги?
У себя дома?» – спросил Кальвисиан. Евпл ответил: «У меня нет дома, и мой
Господь это знает». Кальвисиан попросил зачитать отрывки из этих книг, и Евпл
прочел: «Блаженны изгнанные за правду, ибо их есть Царство Небесное» (Мф. 5:
10), и еще: «И кто не несет креста своего и идет за Мною, не может быть Моим
учеником» (Лк. 14: 27). «Что это значит?» – спросил судья. – «Это Закон Господа
Бога моего, каким он был мне дан». – «Кем?» – «Иисусом Христом, Сыном Бога
Живого».

Сочтя эти признания достаточными, Кальвисиан предал Евпла палачу для допроса.
Спустя некоторое время, в канун августовских ид, святого Евпла снова привели в
суд. Кальвисиан спросил его: «Хранишь ли ты еще писания, которые были запрещены
императором и которые ты должен был отдать властям?» Евпл отвечал: «Я –
христианин, мне нельзя их отдавать. Лучше умереть, чем отдать их. Они содержат
жизнь вечную. Чтобы не лишиться ее, я готов расстаться с земной жизнью». – «Где
они?» – еще раз спросил магистрат. «У меня», – ответил святой, показывая
жестами, что книг у него нет, но он знает их наизусть.

Тогда магистрат приказал подвесить его и пытать до тех пор, пока он не
согласится принести жертву богам. Во время мучений Евпл благодарил Христа за
то, что удостоился пострадать за Него, говоря: «Давно я желал этих мучений.
Делайте что хотите, мучайте меня сильнее, я – христианин!» Наконец правитель
приказал прекратить пытки и предложил святому принеси жертву богам. Евпл
ответил: «Я почитаю Отца, Сына и Святого Духа – Святую Троицу, и нет бога,
кроме Нее. Да разрушатся идолы, не создавшие ни неба, ни земли. Я –
христианин!»

Когда магистрат в последний раз потребовал от него принести жертву, если он
хочет сохранить жизнь, святой ответил: «Да, я принесу жертву, но я предложу
самого себя в жертву Христу. Мне больше нечего Ему пожертвовать. А твои усилия
напрасны. Я – христианин!» Правитель велел усилить пытки. Когда силы оставили
мученика и он почти лишился голоса, его губы продолжали шептать слова
благодарности. Тогда Кальвисиан произнес приговор, который секретарь записал на
таблички: «Поскольку христианин Евпл презирает императорские эдикты, хулит
богов и отказывается исправиться, я приказываю отрубить ему голову мечом».

На шею диакону повесили Евангелие, которое у него нашли при аресте, а шедший
перед ним глашатай кричал: «Евпл – христианин, враг богов и императоров!»
Святой, исполненный радости, ускорял шаг, спеша удостоиться венца, и
непрестанно повторял: «Слава Господу Христу!» Придя на место казни, он встал на
колени и долго молился, благодаря Бога. Потом сам преклонил голову и был усечен
мечом. Позже христиане взяли его тело и, умастив, похоронили (из Жития).

У меня всё.
