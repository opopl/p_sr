% vim: keymap=russian-jcukenwin
%%beginhead 
 
%%file slova.sosed
%%parent slova
 
%%url 
 
%%author 
%%author_id 
%%author_url 
 
%%tags 
%%title 
 
%%endhead 
\chapter{Сосед}
\label{sec:slova.sosed}

%%%cit
%%%cit_pic
%%%cit_text
Писательница привела пример, как две недели назад вспомнили о Владимире
Высоцком. \enquote{Кто такой \emph{соседний} Высоцкий - знают все украинцы. Кто
такой Пантелеймон Кулиш - половина украинцев не знает}, - написала Лариса.
\enquote{Вы думаете, чтобы уничтожить Украину, нужны танки? Совсем нет.
Достаточно не думать об украинском и хвалить \emph{соседнее}, увлекаться
\emph{соседним}, развивать \emph{соседнее}, изучать \emph{соседнее}, помнить о
\emph{соседнее}, популяризировать \emph{соседнее}.  Этого достаточно, чтобы
Украины не стало}, - высказалась Ницой
%%%cit_title
\citTitle{Ницой рассказала, что \enquote{убьет} Украину}, , strana.ua, 11.08.2019
%%%endcit

%%%cit
%%%cit_head
%%%cit_pic
%%%cit_text
Моя матушка 85 лет - типовой представитель аудитории ТВ, слух
у нее уже не тот, что прежде, поэтому даже в соседней комнате, не
прислушиваясь, я слышу, что вещают с экрана. А уж когда ее приходит навестить
80-летняя \emph{соседка} и начинается обсуждение увиденного... Российские
бабульки вполне политически активны. Сиделки же у другой \emph{соседской}
старушки - украинки (5 человек за шесть лет, тетки лет 60) смотрят на планшете
украинское ТВ и обсуждают, в том числе и со мной. Поэтому я в курсе, что
происходит и там и там, не будучи любителем ТВ))
%%%cit_comment
Yuliya Muatarova, \url{https://www.facebook.com/yuliya.muatarova}
%%%cit_title
\textbf{Бог Данилова затопил Крым}, Игорь Лесев, facebook, 20.06.2021
%%%endcit
