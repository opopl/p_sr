% vim: keymap=russian-jcukenwin
%%beginhead 
 
%%file poetry.rus.sokor.vsesvit_i_ljudyna
%%parent poetry.rus.sokor
 
%%url http://maysterni.com/publication.php?id=132719
%%author 
%%tags 
%%title 
 
%%endhead 

\subsubsection{Всесвіт і Людина}
\label{sec:poetry.rus.sokor.vsesvit_i_ljudyna}
\Purl{http://maysterni.com/publication.php?id=132719}

Чи початок, чи є кінець? В людини це первопричина.
Відлік початку почнемо, коли в світ прийшла людина.
Вона прийшла - світ вже жив й не було кінця й початку.
Природні ми кільця, у ланцюзі всесвітнього порядку.

Від початку плине час в сум'ятливій людській долі,
І турбують мрії нас, і тануть, як сніжинки на долоні.
Життя існує на Землі, й все живе у своєму русі,
Нас радує буття у привичнім часі.

Ти знати хочеш простоту сотворіння світу,
Зрозуміти пустоту оком нашим непомітну?
Хіба можливо осягти всесвіт в піднебесній,
Якщо явилися ми на мить в земний мир чудесний.

Завжди кидаємо зір у простір піднебесся,
Думки турбують нас: - Як цей світ створився?
Ми знаємо, там є тьма й мороз в просторі,
Там зірки, міри роять та народжуються нові.

Всесвіт сповнений буття у Святому дусі.
Міри рушаясь й знов творясь в непереривнім русі.
Творче так задумано світ, по волі Сина та Отця,
На світ являємся ми з материнського яйця. 
