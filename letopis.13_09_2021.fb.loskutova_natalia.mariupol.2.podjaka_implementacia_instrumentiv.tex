%%beginhead 
 
%%file 13_09_2021.fb.loskutova_natalia.mariupol.2.podjaka_implementacia_instrumentiv
%%parent 13_09_2021
 
%%url https://www.facebook.com/1427894275/posts/pfbid02xzXACsLEKri66gaY89sAdbF98wJH97RaP7MEvsCeuTQq3qzcfPhos61GmLFbNiAYl
 
%%author_id loskutova_natalia.mariupol
%%date 13_09_2021
 
%%tags mariupol,mariupol.pre_war,obrazovanie
%%title ПОДЯКА ЗА ІМПЛЕМЕНТАЦІЮ ІНСТРУМЕНТІВ З ОСВІТИ ПРАВ ЛЮДИНИ
 
%%endhead 

\subsection{Подяка за імплементацію інструментів з освіти прав людини}
\label{sec:13_09_2021.fb.loskutova_natalia.mariupol.2.podjaka_implementacia_instrumentiv}

\Purl{https://www.facebook.com/1427894275/posts/pfbid02xzXACsLEKri66gaY89sAdbF98wJH97RaP7MEvsCeuTQq3qzcfPhos61GmLFbNiAYl}
\ifcmt
 author_begin
   author_id loskutova_natalia.mariupol
 author_end
\fi

ПОДЯКА ЗА ІМПЛЕМЕНТАЦІЮ ІНСТРУМЕНТІВ З ОСВІТИ ПРАВ ЛЮДИНИ

Доцентка кафедри німецької та французької філології Наталія Лоскутова отримала
подяку від Голови Ради ГО \enquote{ЕдКемп Україна}, члена Консультативної ради з питань
сприяння розвитку системи загальної середньої освіти при Президентові України
Олександра Елькіна за організацію і проведення імплементації інструментів з
освіти прав людини за проєктом \enquote{SIMSCHOOL – СИМУЛЯЦІЙНІ ІГРИ ДЛЯ ВИВЧЕННЯ ПРАВ
ЛЮДИНИ}. Даний проєкт реалізується командою CRISP (Німеччина) та EdCamp Ukraine
у партнерстві з Європейським Союзом, метою якого є розробка нового підходу до
вивчення прав людини. Нагадаємо, що у квітні 2021 на факультеті іноземних мов
МДУ відбувся тренінг з імплементації симуляційних ігор для студентів перших та
других курсів. Запропоновані ігри та вправи сприяли розвитку ціннісного
ставлення до прав людини та почуття відповідальності, справедливості й
солідарності.

\ii{13_09_2021.fb.loskutova_natalia.mariupol.2.podjaka_implementacia_instrumentiv.pic.1}
