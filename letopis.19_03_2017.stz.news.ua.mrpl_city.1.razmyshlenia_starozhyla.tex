% vim: keymap=russian-jcukenwin
%%beginhead 
 
%%file 19_03_2017.stz.news.ua.mrpl_city.1.razmyshlenia_starozhyla
%%parent 19_03_2017
 
%%url https://mrpl.city/blogs/view/razmyshleniya-starozhila
 
%%author_id burov_sergij.mariupol,news.ua.mrpl_city
%%date 
 
%%tags 
%%title Размышления старожила
 
%%endhead 
 
\subsection{Размышления старожила}
\label{sec:19_03_2017.stz.news.ua.mrpl_city.1.razmyshlenia_starozhyla}
 
\Purl{https://mrpl.city/blogs/view/razmyshleniya-starozhila}
\ifcmt
 author_begin
   author_id burov_sergij.mariupol,news.ua.mrpl_city
 author_end
\fi


\ii{19_03_2017.stz.news.ua.mrpl_city.1.razmyshlenia_starozhyla.pic.1.nilsen}

Как это здорово, что во главу приоритетных задач развития Мариуполя поставлено
развитие культуры. Материалы на эту тему вызвали череду как своих воспоминаний
о родном городе, так и тех, что довелось услышать от родителей и бабушки, что
было прочитано в книгах. Все это привело к мысли, что для выполнения
поставленных городскими властями и общественностью амбициозных задач есть очень
солидная основа. Стало бы это реальностью.

\ii{19_03_2017.stz.news.ua.mrpl_city.1.razmyshlenia_starozhyla.pic.2.bazar}

Именно у нас был основан один из первых на территории Украины профессиональных
театров. К нам с удовольствием ехали столичные труппы, так как знали, что их
гастроли пройдут с аншлагом. На рубеже XIX и XX века Мариуполь располагал своим
симфоническим оркестром, который почти  полностью состоял из выпускников
Московской и Санкт-Петербургской консерваторий. Концерты этого музыкального
коллектива пользовались огромным  успехом у жителей  города. В 1876 году были
учреждены мужская и женские гимназии, - раньше, чем такие события произошли в
соседних с Мариуполем городах.  Не всякий город может похвастаться таким
историческим трудом, каким располагаем мы с 1892 года. Имеется в виду
\enquote{Мариуполь и его окрестности. Отчет об учебных экскурсиях Мариупольской
Александровской гимназии}, изданный на средства Д. А. Хараджаева, крупного
предпринимателя и мецената.

\ii{19_03_2017.stz.news.ua.mrpl_city.1.razmyshlenia_starozhyla.pic.3.ekaterininskaja_ulica}

Городские власти бдительно следили за внешним видом улиц. Городской архитектор
В.А. Нильсен собственноручно вычертил развертки фасадов домов на всех улицах
Мариуполя. Сооружение новых строений в обязательном порядке согласовывалось с
городским архитектором. Если фасады предполагаемых новостроек не гармонировали
с фасадами уже существующих домов, разрешение на стройку не давалось.
Оштукатуренные здания белились известкой перед Пасхой. Эта традиция
продолжалась и в советское время. Правда, обновление покраски теперь
проводилось к Первомаю. Также согласовывался вид вывесок на торговых и иных
заведениях. Интересная деталь. Городская дума запретила устройство нависающих
над тротуарами вывесок. Причина - \enquote{дабы обломившаяся вывеска не повредила
прохожих}. В начале ХХ века было решено установить тумбы для расклейки афиш и
объявлений. Тумбы установили. Старые люди рассказывали, что  если кто-то
пытался прицепить свою бумажку вне тумб, а это увидел городовой, то \enquote{можно было
и по мордасам получить}.

\ii{19_03_2017.stz.news.ua.mrpl_city.1.razmyshlenia_starozhyla.pic.4.torgovaja_ulica}

Как проводилось озеленение улиц. Стараниями Григория Георгиевича Псалти был
устроен питомник. Его территория находилась недалеко от нынешней площади
Освобождения. В 30-60-х годах этот питомник называли \enquote{Зеленстроем}. Там
выращивались саженцы деревьев и кустарников с учетом климатических условий
нашего края. Главенствующими из них были акации и клены. Владельцы домостроений
должны были купить саженцы и посадить их перед своими  домами по указанным для
них правилам.

\ii{19_03_2017.stz.news.ua.mrpl_city.1.razmyshlenia_starozhyla.pic.5.zhenskaja_gimnazia}

Нужно сказать, что Мариуполю раньше везло на главных архитекторов. Стоит
привести один лишь пример. В 1943 году гитлеровцы сожгли более 60 процентов
жилых домов, почти все школы и общественные здания. Часть зданий была
восстановлена к 1946 году. Но проблема с обеспечением трудящихся жильем
оставалась чрезвычайно острой. Руководители крупных заводов настаивали на сносе
разрушенных зданий в старой части Мариуполя и предлагали построить новый город
на Левом берегу реки Кальмиус, обосновывая это экономической целесообразностью.
Против этого решительно выступил главный архитектор города Александр
Митрофанович Веселов. Он убедил городское начальство, что такие предложения
ошибочны. Но для восстановления нужны были деньги из государственного бюджета.
И тогда председатель горисполкома Назар Львович Кудрявцев выехал в Москву
\enquote{выбивать} ассигнования. Его хлопоты дали результат. В 1950 году вышло
постановление Совета Министров СССР о выделении первых 120 миллионов рублей на
восстановление Мариуполя. К освоению этих денег приступили немедля,
финансировали возрождение исторической части города  и \enquote{Азовсталь}, коксохим,
завод имени Ильича и трест \enquote{Азовстальстрой}.

\ii{19_03_2017.stz.news.ua.mrpl_city.1.razmyshlenia_starozhyla.pic.6.bazar}

Да, чтобы восстановить былую красоту нашего города, нужно много потрудиться.
Сейчас такое впечатление, что каждый владелец того ли иного строения делает то,
что ему вздумается. Вот несколько примеров. На проспекте Мира, рядом с
библиотекой им. В.Г. Короленко, кто-то \enquote{слишком образованный} аляповато
размалевал часть монументального рельефа, созданного известными на всю Украину
художниками В.К. Константиновым и Л.Н. Кузьменковым. Дома №43 и 45 по проспекту
Мира, это у Театрального сквера.  Когда-то они были  излюбленным элементом
сюжетов как для профессиональных фотографов,  так и для любителей. Сейчас у
пятиэтажек, мягко говоря, вид непрезентабельный. Да, если бы их вид был
прекрасным,  рассмотреть их невозможно.  Их фасады закрыты  огромными
рекламными щитами. Один из щитов нависает над дорогой. Кстати, захотелось
узнать - есть ли еще города, у которых историческая часть города \enquote{украшена}
подобными щитами. В Интернете были просмотрены  Одесса и Львов, Тбилиси и
Ковель, Бердянск и Бердичев. Нет там таких щитов. Их нет даже в Урюпинске.
Нужно сказать, что обращение с историческими зданиями у нас очень вольное. Так
называемый дом адвоката Юрьева  (пр. Мира, 40). Какой-то доморощенный дизайнер
\enquote{украсил} вход в здание – памятник архитектуры начала ХХ века -
ядовито-желтого цвета треугольником. Конечно, это лишь малая толика варварского
отношения к культуре нашего города. Но сейчас появилась надежда, что лед
тронулся, что культура не останется на задворках сознания горожан и власти.
