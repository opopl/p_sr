% vim: keymap=russian-jcukenwin
%%beginhead 
 
%%file books.alternatyvna_evolucija.slovo_1_mirazhi_buttya
%%parent books.alternatyvna_evolucija
 
%%endhead 

\subsection{Слово перше МІРАЖІ БУТТЯ}

Найбільша біда сучасності — брехливість або відносна правдивість інформаційних
каналів: періодики, книг, радіо й телевізії, шкіл, кіно, традицій, історичних
та релігійних переказів. Лише окремі мислителі здатні усвідомити фальш
суспільного буття, і мало хто має мужність заявити про це.

Свідомість людства засмічена лавиною жупелів та штампів. Більшість людей
мислить газетними заголовками, фразами з популярних пісень, дотепними
анекдотами. Тому необхідно оголити проблему, як би не було боляче. Без болю не
відкинути тисячолітню кору сплячки й невігластва.

Отже, проблема катастрофи. Під загрозою не окремий острів чи континент, не
система чи ідеологія, навіть не цивілізація. Під загрозою повного зникнення зі
скрижалей Буття — саме Життя, а з ним — і Всесвіт, що усвідомив себе як чудо
свідомої Духосфери.

Не стихії, не тварини і не рослини поставили Світ на край загибелі. Це —
людська вина й проблема. Тож, якщо Людина, будучи лідером Життя, замахнулася на
Корінь цього Життя — на Цілість, якщо їй вдалося потрясти основи Буття, то
необхідно отримати точні відповіді на запитання:

Що таке Людина?

Що таке Світ, в якому вона перебуває і який нерозривний з нею?

Життя — це випадок чи сама сутність Буття?

Свідомість — хвороба чи основа, творець структури чи її функція?

Що рухає Світ — Випадок чи певна Воля?

Чи такий Світ, яким його вивчає наука, чи ми бачимо лише сни свідомості і
вважаємо їх дійсністю?

Що таке історія — реальність світу чи тимчасова координата загальнолюдської
свідомості?

Хто формував мови, науку, релігію, технологію?

Прогрес — реальність чи соціологічна функція?

Чи існує Світовий Лабіринт? Хто його Творець — Випадок чи розум?

Чи є так звана Еволюція і що це таке? Хто досконаліший: найпростіші чи
високорозвинуті істоти? Де критерій «досконалості»?

Чи існує таємний Світовий Уряд, і хто його очолює — Брати Людства чи Вороги
Людства?

Це не пусті запитання, і є досить підстав, аби їх розглянути. Ось лише кілька
передумов:

\begin{itemize}
	

\item • Людина, нібито мисляча істота, що має незаперечну перевагу над тваринами,
упродовж всієї відомої історії займається жорстоким самознищенням.

\item • Людина, яка розуміє Єдність Життя, старанно руйнує його, знищуючи вітку, на
якій росте.

\item • Людина, маючи мужність у потрібну мить померти ради якихось проблем чи
				цінностей, полохливо виконує накази нікчемних правителів під гнітом
								незрозумілого страху.

\item • Людина, котра явно бачить катастрофічність ситуації, поглиблює її й
				переслідує тих, хто намагається розбудити сплячих.

\end{itemize}

Ці та інші нез’ясовні факти свідчать про своєрідний гіпноз, в якому перебуває
Людина, або ж про древню Світову Хворобу, кульмінацією котрої буде грядуща
катастрофа Життя.

Так, гіпноз безсумнівний. Навіяні ним МІРАЖІ БУТТЯ і спричинили різнопланове
руйнування Свідомості. Але для того, щоб збагнути сутність цих міражів,
необхідно підійти до розуміння безумовної основи Буття й мислення, визначити те
безсумнівне, що всіх нас об’єднує.

Безумовні основи Буття — Свідомість і Космос, на полі яких відбувається
містерія Життя, та естафета Життя й Розуму, що передається з вічності у
вічність. Чому ця тріада є безумовною основою?

СВІДОМІСТЬ тому, що це єдине явище, про наявність якого ніхто не сперечається:
усі проблеми — це проблеми свідомості. Будь-який процес пізнання — це питання
свідомості і відповідь відомості. Аби про щось, говорити і за щось братися —
необхідне усвідомлення. Так визначається лідерство свідомості.

КОСМОС тому, що він — загальний для всіх мислячих і не мислячих істот. Він є
загальним фактом для свідомості всіх. Його походження — це інше питання, але
саме це питання й буде найголовнішим у визначенні катастрофічності ситуації.

ЕСТАФЕТА ЖИТТЯ Й РОЗУМУ тому, що є живий ланцюг поколінь, котрий свідомо і
несвідомо тче загальну тканину Буття. Саме естафета Життя й Розуму творить
релігію, соціологію, науку, космогонію і шукає осмислені моделі Світобудови.
Але чи могла ця естафета передати крізь ланцюг поколінь правдиве відображення
Цільного Буття, щоб уникнути міражів?

Розглянемо, що ж породило Свідомість, а отже, й розуміння Космосу, і оцінку
Історичної Естафети упродовж відомого періоду часу? Естафета Життя й Розуму
розкривається в трьох фазах суб’єктивного Часу: в минулому, сучасному й
майбутньому. Жорстокий взаємозв’язок явищ, невблаганна причинність ілюструє
жахливі феномени Буття.

\subsubsection{У МИНУЛОМУ}

Світи хаосу й титанічної боротьби — в космосі й на землі. Яре кипіння
Першожиття в океанах і початок взаємопожирання: десь там уже почалася лихоманка
буття і світова Хвороба. Ускладнюючись, породжуючи химерні форми Життя, Матерія
не вилікувалася, а втягнулася у ще жахливіші цикли потрясінь і конфліктів.
Довжелезні періоди царювання рептилій, їх страшна безкінечна боротьба,
припинена згодом космічним катаклізмом; поява ссавців, а з ними й приматів —
ембріонів Людини, — не зупинила лютої хвороби буття, а перевела її в план
духовного протистояння, у Сферу Розуму. І розгорнулась тисячолітня битва Світла
й Мороку, Гармонії й Хаосу, радості й відчаю. Так страшна естафета Життя й
Розуму передалася в сучасне.

\subsubsection{У СУЧАСНОМУ}

Розтерзаний, подрібнений світ. Кровоточива Планета, що пливе в невідомість.
Зримий Космос байдужий до долі Людини й Життя на Землі. Отже, вся суть проблеми
переноситься у Внутрішній Космос Духу.

Що ми отримали від минулого?

Сотні держав та племен, відгороджених стінами ворогування.

Загальне недовір’я, породжене розбоєм, грабунками, насиллям, шовінізмом. Ріст
злочинності, зростання жадоби до зовнішніх задоволень та споживацтва. Згасання
народної культури. Розповсюдження псевдокультури з допомогою потужної індустрії
світової інформаційної мережі. Сексуальне й ідеологічне розтління юних
поколінь. Падіння кращих традицій.

Безперервні війни. Нагромадження ядерної зброї, що не дає нікому жодного шансу
на перемогу, не кажучи вже про безцільність будь-яких «перемог».

Безперспективність дипломатичних зусиль. Брехливість політиків. Взаємні підозри
і загальний страх. Відсутність загальної для планети буттєвої платформи,
достойної Людини, — і це найголовніше. Жахлива сліпота більшості лідерів. Втома
й цинізм цілих народів. Руйнування тканини Життя, зникнення тисяч видів рослин
і тварин, отруєння океанів, морів, річок, землі, атмосфери. Повне спотворення
еволюційної ритміки Природи, котра не має сили відновлювати те, що ми руйнуємо…

\subsubsection{У МАЙБУТНЬОМУ}

Мільярди жадібних душ, позбавлених духовності й співстраждання, що прагнуть
лише розкоші та насолод: океан злочинів та збочень. Ніхто й ніщо не наповнить
цю бездонну прірву бажань!

Виснаження землі. Падіння врожайності. Задуха, бо флора гине.

Голод. Смерть мільйонів. Епідемії. Вимирання тварин, бо вони не зможуть жити в
агонізуючій Біосфері, серед лахміття екологічних зв’язків.

Деградація психіки. Натовпи дегенератів, девальвація знання, бо процес його
засвоєння перетворився в «заучування». Вузькі «спеціалісти», котрі більше
нагадують біороботів. Неконтрольовані спалахи насилля, тероризму, несподівані
війни, атомна катастрофа.

Це — кінець нашого, людського світу.

Якщо Природа й справиться з жахливим зараженням й отруєнням Сфери Життя, якщо
вона й відтворить Біосферу, то це вже буде не наш історичний світ…

Про що ж ми говоримо? Невже про грізні пророцтва, невже про апокаліптичні «чаші
помсти»? Чи не достатньо чули про це люди впродовж віків?

Жахи ніколи не стимулювали людей до пошуків кращого рішення. Навпаки — вони
вбивали ініціативу й змушували кидатися в обійми містики, фанатизму й покірної
приреченості. Тому ми згадали про грядущу катастрофу не для залякування, а щоб
нагадати про можливість Преображення, про можливість порятунку того, що може
бути врятоване.

Але що рятувати?

Залишки Біосфери? Для чого? Щоб розтягнути агонію?

Чи рятувати Людину як вона є — з усіма її збоченнями й жахіттями?

Суть проблеми не в тому, щоб самозберегтися (адже тілесно так чи інакше
помирають усі); суть у тім, щоб не тільки зупинити катастрофу, але й відтворити
Єдність, а отже — Радість Життя, котра суджена як головний прояв Самопізнання.

Аналізуючи минуле, сучасне й грядуще, ми бачимо, що всі жахи Буття — це єдиний
процес розвитку Космічної Хвороби, що уразила Життя.

Певно, знайдуться вчені апологети катастрофічного світосприймання, котрі
стануть стверджувати «законність» та «необхідність» такої ситуації. Знайдуться
і «святі отці», що вкажуть на «гріхопадіння» Людини і стануть стверджувати
невідворотність світового покарання.

Ми рішуче повстанемо проти приреченості. Ми стверджуємо Святість та Радість
людського Духу. Ми стверджуємо його Праведність та Космічне Синівство. Ми
стверджуємо, що Він здатен і мусить повернути Радість Життя. Отруйний вірус
падіння і хвороби необхідно вигнати — і відновити Цілість. Але лічені роки
зосталися до грізного порогу. Сплячі — не прокинуться! Але тим, хто не спить,
ми пропонуємо разом розглянути принципи Суми Любові, що кличе нас до
Материнського Світу Радості.

Але для того, щоб зрозуміти й прийняти суть цієї Альтернативи, необхідно
розглянути МІРАЖІ ПСЕВДОБУТТЯ, ці грандіозні космічні й історичні фікції, котрі
завели людство у лабіринт, порушивши гармонію Цілого.

МІРАЖ ЕНЕРГЕТИЗМУ або ЧАСУ — головний міраж, що став основою багатьох інших.

Зруйнована в результаті світової хвороби цілісність породила ритміку обмежених
проявів — Час, а разом з ним — Енергію як динаміку цих проявів.

Ефект розпаду Цілого, хвиля сили, відчужена від Внутрішнього Космосу і
відірвана від Джерела, оформлена у зовнішні стихії — ось що таке Енергія, що
лягла в основу тканини розпорошеного світу.

Не зрозумівши Руху та його проявів (Енергії, Сили, Могутності, Напруги), Людина
запрагнула оволодіти Зовнішнім Космосом, використовуючи часово-ефемерний аспект
Руху — Енергію. Погляньте неупередженим оком на творіння так званої технічної
революції, і ви збагнете, що ми ввійшли в протистояння з Цілим, що ми стали
його паразитами, ворогами, руйнаторами.

Ми творимо енерготарани для подолання простору і часу; ми використовуємо
могутність Матерії для руйнування того, що нездатні збагнути; ми схожі на
нерозумних дітей, котрі в пороховому погребі влаштовують феєрверк, коли
бомбардуємо потоками часток атомні мішені синхрофазотронів, ґвалтуючи
мільйоноградусну плазму в магнітних тюрмах. Ми грубо вторгаємося у світ атома,
клітини, психіки, ігноруючи єдність Сущого, і не розуміємо, що безвідповідальні
вторгнення в інші сфери життя можуть викликати ланцюгову реакцію розпаду.

Необхідно збагнути, що Енергія як прояв зовнішнього руху — це лише грубий
аспект світової Динаміки Духу, Космічної Свідомості. Ми будуємо цивілізацію на
цьому примітивному, часовому підмурку і тим прирікаємо себе на неминучу
загибель, не підготувавши свою свідомість до оволодіння Духовною Динамікою.

Уся наука, вся технологія збудована на ілюзорній ідеї, буцімто Всесвіт — це
джерело невичерпної енергії. Висуваються прожекти про грядуще використання
енергії в масштабах випромінювання зірок, галактик, метагалактик. Що за чудний
міраж? І для чого це потрібно, якщо навіть сучасна людина мріє про Тишу, цю
священну Матір Цілого?!

Збільшення технічної могутності вводить Людину в конфлікт з усією Світобудовою.
Ми знаємо, що від нерозумного використання і руйнування екологічного кільця
Планети деградує і рухається до загибелі вся Біосфера. Так само і при сліпому
оволодінні тими чи іншими енергорівнями мікрокосмосу чи Макрокосмосу ми
порушуємо стихійну рівновагу Природи, масштаби, походження і призначення котрої
неможливо пізнати з допомогою нашого деформованого бачення.

Енергетична криза, що уразила цивілізацію, — не випадкова. Надії на те, що
органічне паливо буде замінене термоядерним синтезом, — наївні. Енергія
творящого синтезу — це привілей Світу Єдності, а не світу руйнації, в якому ми
живемо і котрий живимо своїм розділено-дискретним мисленням.

Вимагають перегляду самі уявлення про СВІТЛО як про певну хвилю випромінювання
(чи про потік квантів, фотонів). Скоріше за все, СВІТЛО — це посередник між
внутрішнім та зовнішнім Космосом, котрий зв’язує Світового Суб’єкта зі Світовим
Об’єктом.

Оскільки Світло — найвище (для нашої свідомості) прояв Світової Динаміки, воно
тче світ форм, проникає в них і одухотворює речовинний світ. Те, що вимірює та
вивчає фізика, — лише динаміка речовини в полі гравітації, сплячої Свідомості.
Світло невловиме, ми помічаємо лише його ілюзорний слід, тільки його
енергетичну шкаралупу, що виникає при контакті зі світом форм, «заморожених»
Часом.

Використовуючи енергію, ми виснажуємо себе, свій власний дух. Це —
самопожирання в буквальному розумінні. Кільце взаємозв’язку дуже тісне, воно
губиться в безодні Часу й Простору, і важко побачити, звідки бере початок те
Джерело, з якого ми смокчемо енергетичну кров. Але грядуще відкриє нам ту
істину, що її знали древні мудреці, котрі подарували нам символ змії, що кусає
свій хвіст, чудовий символ самопоїдання.

Ясно лише одне: що більша енергетична потужність, котру ми використовуємо, то
більш напружене протистояння Космічного Цілого. Ми узурпуємо ті основи, що на
них тримається сам Фундамент Світобудови.

Сума Любові — Альтернативна Еволюція — кличе людство до пошуку нових шляхів
Буття: неенергетичних чи наденергетичних. Залишивши в спокої стихії,
співпрацюючи з ними в процесі творення гармонійних і прекрасних форм Життя, ми
отримаємо доступ до таких таємниць і глибин Буття, про які не можуть навіть
мріяти сучасні вчені. Ми агресивно відвойовуємо крихти зруйнованих, покалічених
таємниць, а необхідно увійти в Дім Таємниці друзями. Тоді й Таємниця зникне і
зітче для нас нову чудесну форму Життя.

Отже, підсумуємо.

Увесь енергетичний Всесвіт, зорі, галактики, мікросвіти, макросвіти — все це
лише наше колективне Творення, розіп’яте на безодні Часу і Простору в спробі
утвердити свою самообмеженість у певній системі координат. Дух Людини забув про
свою Прабатьківщину-Цілість, про своє право на Повне Буття і намагається
спорудити жебрацьке буття в сферах тління й смерті.

Чи наближаємося ми до таємниці буття, нагромаджуючи енергоможливості?

Ні. Ми тільки викликаємо більшу тугу і спрагу нових джерел енергії.

Всі наші зазирання в «інтимну таємницю кожного», у надра атома, зірок, у
глибини генів, у нейрони мозку показують нам лише невичерпні лабіринти
структур, безперервне бурління речовини в ритміці просторово-часових процесів.
Ніякої таємниці в цій фантасмагорії антижиття ми не відкриємо. Наші подорожі до
«інших світів» — це лише польоти у власну пустоту, непотрібне нагромадження
інформаційного сміття, що закриває від нас власну сутність. Новий Світ рішуче
позбудеться деспотії енергетизму як породження Хроносу-Часу.

Відомий астрофізик Козирєв стверджував, що час породжує енергію. Не будемо
аналізувати, що мав на увазі вчений, розробляючи цю концепцію. Але, зі свого
боку, повністю згодимося з ним. Саме час творить енергію. Саме Хронос рубає
Світове Ціле на часово-просторові прояви, творячи з цих осколків-квантів у
міріадах психік мозаїку оманливого буття. Динаміка цього кінематографу Всесвіту
і створює енергетичний світ.

МІРАЖ ПРОГРЕСУ або ІСТОРІЇ — не менш жахливий та підступний. Людство повірило в
грубу казку про те, що Світ прогресує, вдосконалюється, йде до гармонії. На
ґрунті цього міфу творилися соціологічні моделі, що кликали героїв та
подвижників до Преображення світу, до його перебудови. Але чи не дивний намір?
Де критерій покращення світу? Де зразок такого покращення? І що покращувати:
лише людину, чи тваринний світ, чи рослини і весь комплекс біологічних
структур? Саме питання «покращення» — фікція, міраж! Адже Всесвіт — не
спеціалізована машина, функції котрої можна вдосконалити, а таємнича Єдність
Внутрішнього й Зовнішнього, що було завжди і всюди. Отже, необхідне не
«вдосконалення» чогось, а розкриття глибинної сутності Ядра Життя, котре
виправить все, що деформоване.

Ми були свідками страшних злочинів в ім’я «прогресу»: колоніальні війни і
безжальне знищення туземного населення та грабунок їхніх багатств, загарбання
Америки і майже повне знищення оригінальної культури індіанців, жорстокі
європейські війни, столітні багаття для спалення інакомислячих під знаком Того,
хто Любов проголосив основою Нового Світу, криваві війни, революційні
потрясіння XX століття в ім’я людськості і свободи, нинішні нескінченні
локальні війни, перевороти, гризня політиків, переділ світу, змова за спиною
народів і багато чого, про що прекрасно знають всі мислячі люди.

То про який прогрес можна говорити? Як може жахливе минуле породити «сяюче
майбутнє», що його обіцяють пророки земного раю? Хто спекулює міфом «прогресу»,
той просто духовний підлотник і ошуканець…

О, ми знаємо і пам’ятаємо про подвиги тих чи інших людей, рухів, груп. Ми
знаємо про великі душі, що прийняли голгофу ради друзів своїх. Ми захоплюємося
відважними дослідниками, мандрівниками, духовними подвижниками, героями, що
зневажали смерть, великими трудівниками, що подарували світові красу творінь і
плоди дивовижних відкриттів.

Але хіба прийняв світ від них естафету любові, розуму і подвигу Краси? Світова
структура, розіпнувши їх, тільки зводить замучених у ранг «святих»,
використовуючи факт подвигу для того, щоби заштопати дірку, пробиту героями в
стіні тюремного світу; використовуючи ідеї тих, що впали, ветхий світ
виготовляє підробку, одягає маску змін і під цією маскою продовжує чинити
підлість ще більш витончено, аніж раніше.

Отже, можна говорити не про прогрес, а про періодичні спроби романтиків,
героїв, подвижників руйнувати твердиню оманливого, тіньового світу, зіткану
падкою свідомістю, — і про їх невдачу. Але… невдачі накопичуються, і саме вони
несуть в собі нові можливості.

Щоб прагнути до Нового Світу, до Нового Шляху, необхідно чесно й відкрито
поглянути на історичний балаган, потрібно зрозуміти, що він створений і
управляється вмілими й підлими руками. Потрібно прослідкувати, як створювалися
історичні міфи, брехливі традиції, фальшива історіографія, котра ліпила певний
образ народів і навіть сам тип Людини.

Не може бути й мови про те, щоб цей історичний світ ввести в нову епоху,
створити з цих фрагментів, з кривавого лахміття щось ціле, гармонійне.

Необхідно прийняти в серце біль і підлість сучасності, щоб пройти крізь вогонь
очищення й спрямувати себе до Радісного Преображення.

Та чи багато тих, хто зможе збагнути таємницю Преображення, що так просто
відкривається всім зрячим у трансформації повзучої гусені в крилату
психею-метелика?

МІРАЖ РЕЛІГІЇ більш витончений. Він виростає з віри в духовну Ієрархію
Всесвіту, у наявність за всією фантасмагорією Буття Батьків — Творців Сущого.

Усі релігійні моделі використовують інтуїтивне відчуття людською душею своєї
Єдиносутності з Коренем Буття, свого Космічного Синівства. Та замість того, щоб
природно розвивати це відчуття в почуття Всеєдинства й Всеродства, жерці
релігій створили могутній апарат церковної адміністрації, що зацікавлений уже
не в духовному розкритті людини, а в її самоприниженні.

Син Бога і Спадкоємець Космічних Скарбів стає нікчемним грішником, «блудним
сином», що упродовж тисячоліть повзає в мороці земного лабіринту, очікуючи
якогось «Страшного Суду», копійованого з жорстоких трибуналів середньовіччя.

Передчуття радісної Зустрічі з Батьком та Матір’ю Буття трансформується в страх
перед Небесним Судією, а отже, деградує внутрішній Божий Світ Душі,
припиняється ріст, розвиток, розкриття Духовного Космосу.

Людина починає шукати виходу із зачарованого кола: зовнішній світ — смерть,
жах, самотність; у сфері духу — караючий суддя! Поміж цими жорнами душа
перемелюється і від болю кидається від полюсу до полюсу: або стає рабом земних
диктаторів, або рабом лжебогів, що одне й те ж.

У релігій теж є свій «історизм» або ідея своєрідного прогресу. Очевидно, ця
ідея — відгомін древнього знання космоісторії. Спочатку — блаженний стан
Людини, потім — падіння і всі жахи буття поза Богом, потім — спокутуюча Жертва
Христа і грядуще воскресіння праведних в іншому світі.

Так, все логічно. Але логічно з нашої, земної точки зору — з точки зору
жорстоких людей, позбавлених співстраждання. Ми творимо богів «за нашою
подобою», навіть не сміючи допустити, що Найвища Сутність — Єдність і радість
Буття. Чи можуть від неї йти руйнівні чи нищівні імпульси?

Згадаємо Великого Вчителя Нового Заповіту. Ось де Син Людський, осяяний Світлом
Батька Сущого! Дух, що прагне до Преображення, Дух Великої Жертви, Дух, що
підносить кожну людину, бо Він нерушимо утвердив, що «Царство Бога внутрі нас».
А де Царство — там і Цар. Отже, Бог у нас. Ми його Діти. Ми — фундатори
Всесвіту, ми продовжувачі Його Творчої Дії. Ось яке покликання Людини!

Як далеко ми відійшли від Нового Заповіту! Як багато часу й можливостей
втратили. Міраж обманних релігій має бути розвіяний, щоби Людина могла
вернутися до судженого Всебуття…

Атеїзм — інший полюс тієї ж помилки. Поклонятися зовнішньому богу чи
розвінчувати його — це одне й те ж. І апологети, й супротивники утверджують
Володаря Світу Цього: теїзм та атеїзм — його права й ліва руки. У цьому ключі
треба розглядати всі окультні доктрини, всі містичні таємні товариства, всі
«одкровення», всі феномени-обіцянки небувалої могутності та нових можливостей.
Всі можливості знаходяться в Людині, і Божественний Птах всередині тільки й
чекає, щоб розірвати шкаралупу Земного Яйця і злетіти в Небо Свободи.

Саме до цього кличе Сума Любові.

МІРАЖ НАУКИ ще більш витончений. Відкинувши ілюзії «потойбічного світу, раю,
небесного царства», людина попрямувала до зовнішньої ілюзії, створивши
аналітичну науку. Вона увірувала в те, що, руйнуючи Ціле, розпинаючи, ріжучи і
класифікуючи частини, можна збагнути Ціле, оволодіти ним.

%6

Що за безглузда помилка! І яка злобна впертість!

Чим заслужила фізична людина — двонога істота з ряду приматів — щоб увесь
Всесвіт упав до її ніг? Що несе вона Природі навзамін того, що вона постійно
бере? А брати без віддачі — це космічний злочин.

Наука не замислюється над цим. Апологети «точної науки» крушать все навколо —
атомні структури, живі клітини, лізуть у психіку і гени, в минуле й майбутнє,
прагнуть до далеких світів, щоб і там ствердити свої честолюбство й зверхність.

Що вони намагаються знайти в мікросвіті? Ті ж битви й конфлікти, що й на Землі?

Що вони шукають в далеких світах? Про що хочуть говорити з іншими розумами,
якщо не можуть домовитися з такими ж, як і самі вони?

Справді, про що говорити із Зоряними Братами? Що розказати їм про нашу Землю?
Чим похвалитися? Безкінечними війнами? Отруєнням Планети? Тим, що озброїли
хижих політиків комічною зброєю загального руйнування?

Великий злочин Науки перед Розумом і Совістю. Бо найстрашніші диктатори не
змогли б нічого зробити, якби апологети науки не вклали в їхні руки таємну силу
землі й неба.

Воістину, Наука — хижак, що прикинувся велелюбним створінням, звір, який обіцяв
народам рай земний, а натомість укинув Планету в тяжкі кола техногенного Пекла.

Світ, як Жива Істота, зникає. Він закутий у бетон, у залізо й пластмасу. Він
обплутаний сіткою дроту й обманної інформації. Він знемагає під мільйонами
коліс і гусениць, що мучать його плоть. Він отруєний пестицидами та іншими
«цидами». Він задихається від випарів отруйного газу і нескінченного потоку
всілякої мерзоти, що пливе з мегаполісів та міст.

Навіть багато учених жахнулися від того, що самі накоїли. Дехто закликає до
здорового глузду, дехто заперечує небезпеку. І ті, й інші — ошуканці й
негідники!

Необхідні не запобіжні заходи (хіба не все одно, як убивати і мучити — швидше
чи повільніше?), а рішуча Альтернатива Буття. Необхідно збагнути небезпеку
зовнішньої аналітичної «науки» як ілюзорного інструменту псевдопізнання, бо
неможливо пізнати Ціле, дроблячи його на частки, агресивно вторгатися в
сокровенні глибини Життя, викликаючи тим самим його спротив. Ми не пізнаємо
сутність Буття, а воюємо з ним.

Наука Нового Світу вивчатиме не ілюзорне плетиво тіньових часових структур, а
невичерпні глибини Внутрішнього Космосу, котрий є зодчий і Космосу Зовнішнього.

Пізнавши себе, тобто Внутрішнє, Людина пізнає і Зовнішнє, стане Розумним Духом
Усесвіту. Не «закон» диктуватиме Людині сенс дій та волінь, а Людина творитиме
динамічні закони для Зовнішнього Космосу — Закони Радості, Краси і Єдності.

Хай не гніваються друзі пізнання, а добре подумають, що за інверсія відбулася з
проблемою гнозису-пізнання, кому стали слугувати нащадки древніх мудреців, що
свято берегли таємниці Матері Світу від жадібних прислужників Мороку? Збагнувши
свою зраду, вони рішуче спрямують свій шлях до Альтернативи!..

Небезпечний і МІРАЖ ЗОВНІШНЬОЇ ТВОРЧОСТІ. Він повів людей по шляху Каїна,
котрий, за біблейським міфом, був першим будівником, творцем наук, ініціатором
мистецтв, держав і т. д. Цей міраж обманув Людину надією на творення в
зовнішньому світі того, що було втрачено у внутрішньому. І що ж ми бачимо?

Безкінечне будівництво міст, палаців, пірамід, лабіринтів, храмів, картин,
скульптур, доріг, імперій, держав, ідеологій, наук, релігій, світобачень,
розваг і багато чого іншого. І вся ця «творчість» — всепожираючому Часу на
з’їжу. Тільки пилюга над колишніми імперіями, тільки каркання зловісних птахів
над шляхами кривавих завойовників, зотлілі кістки грізних диктаторів, лише
огризки древніх «знань» і «вірувань», тільки втома народів від нескінченного
шляху в нікуди, від непотрібної, тяжкої «творчості», від безперервного
будівництва «осяйного майбутнього».

Чому ж триває безглузда «творчість», чому пливуть потоком брехливі книги,
фальшиві картини, нікчемні теорії, сірі мудрування? Чому ростуть нові міста,
поглинаючи живу природу, готуючи все нові й нові комфортні в’язниці квартир,
офісів і ресторанів, з яких людям ніколи не вирватися?

Хтось скаже, що без такої творчості людина деградує, що в цьому — сенс Буття і
покликання Людини.

Це — напівправда. Людина, воістину, суджений Творець Всесвіту, але творець
духовний. Вона могла б стати Розумом і Духом Буття, направляючи Стихії й
Елементи Природи до гармонійного співзвуччя й Краси. Але не ліпити з крихт
розтерзаної Єдності, не творити тіньові образи занепалого Світу, — а прагнути
до Само-творчості, до Саморозкриття. Як це роблять зорі, квіти, окремі
мислителі, що вказали Шлях до Світів Духу.

Сума Любові не заперечує зовнішню творчість, ще протягом багатьох літ вона буде
необхідною як обрамлення головної будівлі — Світу Самотворчості, при якому
Людина почне ліпити сама себе й нові гармонійні форми Зладованого Космосу.

А зовнішня творчість (і тому є безліч прикладів) у більшості випадків веде в
ілюзорні світи безкінечної марноти…

МІРАЖ ТІЛЕСНОСТІ, або ФОРМИ — грізний міраж. Він полонив дух Людини і
нерозривної з нею фауни ілюзією окремості. Він, роздробивши Ціле, протиставив
форму формі, він розчленив Єдину Сутність Буття на океан дискретностей, частин,
що хаотично танцюють у безмежності під пресом «законів» тяжіння чи
відштовхування.

Тіло необхідне як ступінь самоусвідомлення серед Вічності, серед безодні Буття.
Це — сходи розкриття невичерпного потенціалу Духу, це — нові горизонти Світів і
Можливостей. Тіло — інструмент Творящого Духу для життя й діяльності в певній
сфері Буття. Але це в тому випадку, якщо тіло чи форма динамічні й слухняні до
волі Кореня Життя. А якщо вони самозберігаються і закривають зір Духу, тоді
вони перетворюються на космічну в’язницю, на міраж. Тоді тілесний інтелект
створює інструмент самозбереження у вигляді наук, релігій, соціології,
традицій, здорового глузду і т. д. Ілюстрація: кокон з метеликом, що
розвивається всередині. Якщо форма-кокон переконає метелика, що її треба
«зберегти», то метелик не з’явиться на світ, а задихнеться в самоствореній
тюрмі.

МІРАЖ ФОРМИ породжує ще один — МІРАЖ МНОЖИННОСТІ. Ця фікція дискретності,
роздробленості, роздільності Світу переслідує нас постійно, щохвилини буття.
Зорі, атоми, частинки, люди, звірі, хмари в небі, зотлілі кістки дідів, камені
колишніх палаців, новонароджені, мурашки, птахи, іскри сонця на хвилях води —
все тікає, розсипається, все це тримається разом тільки завдяки творчій потузі
Світової Свідомості, але чи надовго вистачить цієї потуги? Невже до
безкінечності слуги Мороку будуть терзати частини тіла Осіріса, кидати частини
розтерзаного Орфея в потік ілюзорного буття?

Ціле втратило Себе серед безодні і намагається з’єднатися із своїми ж
частинами, шукає їх, запалюючи маяки Любові, Краси й Розуму серед страшної
пустелі Небуття. Але цього не стається. Бо у Світі, поряд з істинними частинами
«Володаря Світла» діють самозванці, актори Мороку, що намагаються запобігти
Об’єднанню Світів. Головна їхня зброя — «Закон Дзеркала». Частини «Володаря
Світла» — Першосутності Духовного Світу — відображені в дзеркалах Мороку і
породжують легіони ілюзорних двійників. Маючи можливість діяти у світі форм,
паразитуючи на таємничій сутності Життя, псевдодвійники Реального постійно
тчуть примарне Буття Зовнішнього Світу, використовуючи духовний матеріал Синів
Світла. Так вони затримують Духів Творення в тюрмі Часу й Простору, змушуючи
крутити упродовж міріад років Колесо Обманного Буття.

Результат цього — страшна самотність серед пустелі зоряного світу. Самотність —
ось бич Буття. Біль самотності штовхає весь світ до пошуків втраченої Єдності
(чи Раю, за містичною термінологією), але ці пошуки ведуться на шляхах
суєтності й марноти. Саме біль самотності породжує агресію, ненависть і любовну
пристрасть — підробку Єдності.

Так, любов теж агресія, хоч і ніжна. Вона не задовольняється сама собою, але
намагається полонити іншого, щоб передати тому свій біль, щоб забути про
необхідність пошуку.

Самотність породила тілесність, тілесність — форму, форма — енергетизм,
енергетизм — силу, сила — перевагу, перевага — ієрархію Світобудови.

Так був розтерзаний Єдиний Світ, і Міражі Псевдобуття перемогли Дух Цілості.

Але де ж джерело цього світового гіпнозу? Хто навіяв страшний космічний сон?

Чимало традицій древності стверджують дуалізм Світобудови. Вони говорять про
наявність у світі певної сили, що зачаїлася біля самих витоків Еволюції і
руйнівно впливає на процес Життя.

Так, відповімо ми. Необхідно мужньо й відкрито глянути у вічі правді. Древній
Господар Світу — космоісторична сутність, із життям якої зв’язана проблема
Падіння. Говорячи сучасною термінологією, мова йде про Інтегральний Інтелект
Планети, або про Супермозок Біосфери Життя.

