% vim: keymap=russian-jcukenwin
%%beginhead 
 
%%file slova.skazka
%%parent slova
 
%%url 
 
%%author 
%%author_id 
%%author_url 
 
%%tags 
%%title 
 
%%endhead 
\chapter{Сказка}
\label{sec:slova.skazka}

%%%cit
%%%cit_head
%%%cit_pic
%%%cit_text
Який тільки народ не адаптував сюжет стародавньої \emph{казки} про відомих
українцям братів Правду і Кривду. Проте найкраще, здається, це вдалося зробити
в англосаксонському світі, де цю моралізаторську \emph{казку} назвали
«Засліплення правди фальшивістю». І тут йдеться не тільки про маніпуляції та
перебріхування.  Мова йде також про тих, хто облудою і фальшем забирає в інших
здатність бачити, чути і розуміти. Хоча у \emph{відомій казці} є безліч
незрозумілих для сучасної людини поведінкових практик, а сюжет містить багато
спрощень, основна мораль зрозуміла – неправдою щасливим не станеш. Але цього
разу нам важливо зосередитися на тому, що одному з братів за допомогою цинічних
маніпуляцій вдалося виманити у брата його коржа. Ще цікавішим є те, чому брат
Правда погодився на те, щоб його скалічили до напівживого стану
%%%cit_comment
%%%cit_title
\citTitle{Правда і Кривда}, Василь Расевич, zaxid.net, 04.06.2021
%%%endcit


%%%cit
%%%cit_head
%%%cit_pic
%%%cit_text
Проте ранiше чи пiзнiше настає мить, коли мисляча iстота Землi має проломити
планетарну шкаралупу, щоб вийти у зоряну сферу i народитися "громадянином
Всесвiту" (за висловом Цiолковського). Повитухою у таких вiдповiдальних пологах
i покликана бути лiтература мрiї та передбачення фантастика. Пiдготувати
свiдомiсть юних поколiнь до зiткнення з таємничим океаном буття, промоделювати
можливi варiанти подiй, навчити читачiв мислити неупереджено й парадоксально,
вiльно й розкуто. Ще донедавна точилися суперечки про значення й призначення
фантастики, i її, немов нерiдну доньку в \emph{казках}, намагалися запхати десь у
запiчок сучасної суспiльної динамiки, дозволяючи "оспiвувати" досягнення науки,
технологiї або розважати дiтвору закрученими сюжетами з космiчним антуражем.
Останнi ж десятилiття показали, що "географiя фантастики" визначає навiть
потужнiсть наукового потенцiалу країн, i нехтування цим жанром (або, скорiше,
цiєю пансофiчною лiтературою) призводить до деградацiї творчого мислення, до
цiлком конкретних втрат у пiдготовцi спецiалiстiв з прогностичними здiбностями,
а вiдтак - у освоєннi терра iнкогнiта (невiдомих континентiв прийдешнього).
Саме тому ми зобов'язанi не лише звернути найпильнiшу увагу на видання
фантастичної лiтератури, а й створювати прогностичнi об'єднання, клуби, групи,
що спроможнi будуть консолiдувати мрiйникiв, готувати новий рiвень суспiльного
мислення. Це те, що вкрай нам необхiдно в епоху багатомiрної перебудови всiх
сфер життя
%%%cit_comment
%%%cit_title
\citTitle{Вiдслонити завiсу часу!}, Олесь Бердник
%%%endcit
