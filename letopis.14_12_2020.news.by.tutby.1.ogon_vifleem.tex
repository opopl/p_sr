% vim: keymap=russian-jcukenwin
%%beginhead 
 
%%file 14_12_2020.news.by.tutby.1.ogon_vifleem
%%parent 14_12_2020
 
%%url https://news.tut.by/society/711290.html
 
%%author 
%%author_id 
%%author_url 
 
%%tags 
%%title На этой неделе в Гродно привезут Вифлеемский огонь. Отсюда он начнет «путешествовать» по Беларуси
 
%%endhead 
 
\subsection{На этой неделе в Гродно привезут Вифлеемский огонь. Отсюда он начнет «путешествовать» по Беларуси}
\label{sec:14_12_2020.news.by.tutby.1.ogon_vifleem}
\Purl{https://news.tut.by/society/711290.html}

\index[cities.rus]{Гродно!Беларусь!Вифлеемский огонь, 14.12.2020}
\index[rus]{Вифлеемский огонь, Гродно, Беларусь, 14.12.2020}

\begin{leftbar}
	\begingroup
		\em\large Вифлеемский огонь мира привезут в Гродно 17 декабря. Торжественное
				зажжение огня состоится в 17.30 в Фарном костеле, затем начнется
				торжественная служба, после которой огонь отправится в путь по нашей
				стране. В этот же день скауты завезут его во все католические храмы
				Беларуси, сообщает «Вечерний Гродно».
	\endgroup
\end{leftbar}

\ifcmt
pic https://tutby.gcdn.co/720x720s/n/regiony/06/3/grodno_vstrecha_vifliemskogo_ognia-6917.jpg
caption Вифлеемвский огонь, Гродно, Беларусь, 14.12.2020, Фото: Катерина Гордеева, TUT.BY
\fi

По традиции перед Рождеством Вифлеемский огонь доставляют в гродненский Фарный
костел. И уже из Гродно он начинает свое путешествие по Беларуси. Привозят его
в нашу страну польские харцеры, которые передают символическую лампадку на
границе белорусским скаутам.

Пламя символизирует рождение Иисуса Христа. Огонь зажигается в базилике
Рождества христова в Вифлееме, а потом доставляется в Вену (Австрия) и
передается в страны Европы и Северной Америки.

Именно в Австрии в 1986 году зародилась такая традиция. Первоначально акция
носила благотворительный характер и была направлена на внимание к
детям-инвалидам, больным и нуждающимся. Благодаря скаутам она стала
универсальной и распространилась по всей Европе.

В 1996 году пламя Вифлеемского огня было передано в 22 страны Европы, среди
которых была и Беларусь. Тогда вечером 24 декабря Вифлеемский огонь доставили
из Гродно в два минских прихода: церковь Пресвятой Троицы (св. Роха) на Золотой
Горке и церковь Св. Симона и Св. Елены. С 2008 года к акции присоединились
скауты из католических общин Беларуси.



