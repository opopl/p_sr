% vim: keymap=russian-jcukenwin
%%beginhead 
 
%%file 24_12_2021.fb.fb_group.story_kiev_ua.3.jolka
%%parent 24_12_2021
 
%%url https://www.facebook.com/groups/story.kiev.ua/posts/1825593977637395
 
%%author_id fb_group.story_kiev_ua,galaktion_ljudmila
%%date 
 
%%tags germania,jolka,kiev,kievljane
%%title ЁЛКА
 
%%endhead 
 
\subsection{ЁЛКА}
\label{sec:24_12_2021.fb.fb_group.story_kiev_ua.3.jolka}
 
\Purl{https://www.facebook.com/groups/story.kiev.ua/posts/1825593977637395}
\ifcmt
 author_begin
   author_id fb_group.story_kiev_ua,galaktion_ljudmila
 author_end
\fi

ЁЛКА

Ура! Наступают Новогодние праздники - лучшие в году.  Самые демократичные и
объединяющие. И совершенно не важно есть ли снег, нет ли снега. Главное - Ёлка:
пахнущая лесом или чуть химией, нарядная или другая – любая. Она -
олицетворение новогодних празднеств, повсюду на Земле. 

Рождественско-Новогоднюю ёлку человечеству подарила Германия. Её дебют
состоялся в 1527 г. в баварском городке Штокштадт (справка имеется...). Но далее
она века «кружила-бродила» лесными тропами, лишь изредка заглядывая в городские
гостиные на огонек. Популярность к ней пришла только в 19 в. Ёлку «продвинула»
художественная литература, прежде всего сказка Э. Гофмана «Щелкунчик и Мышиный
король» (1819, пер.–1835). А «раскрутила» – реклама и промышленное производство
ёлочных украшений (вт. пол.19 в.). Свою лепту внёс и вносит балет «Щелкунчик»
(с 1892, по одноименной сказке) с волшебной музыкой Петра Ильича. В странствиях
во времени и пространствах ёлка растеряла свой изначальный религиозный смысл
(см. ниже), превратившись в светский праздничный атрибут.

\ii{24_12_2021.fb.fb_group.story_kiev_ua.3.jolka.pic.1}

Киевлянам Рождественскую ёлку открыла немецкая колония, которая появилась в к.
18 в. на Киево-Подоле. Вследствие подольского пожара (1811) немцы перебрались
на Печерск, в Липки, где с 1812 г. находится лютеранская церковь св. Екатерины.
Одна из первых в городе ёлок засветилась в доме многодетного Георгия фон Брадке
-  «отце» Киевского университета. Будучи попечителем киевского учебного округа
(1832-1838), Брадке организовал перевод Кременецкого лицея в Киев и его
преобразование в университет. Наиболее сложной задачей оказался выбор
преподавателей, особенно на кафедру истории. На неё было несколько
претендентов, включая Н. Гоголя, имевшего влиятельных покровителей. Но Брадке
предпочел профессионала – магистра всеобщей истории Владимира Цыха. Он ему
импонировал своей искренностью и благородством. Раннее, работая в Харьковском
университете, Владимир обнаружил, что преподаватели меняют свои убеждения
сообразно убеждениям начальников. «Это так подействовало на Цыха, что он стал
решительно презирать человечество и пришел к убеждению, что все люди льстецы и
притворщики» (Г. Брадке. Автобиография). 

Владимир не подвел. Он проявил глубокое знание предмета, преподавательское
мастерство, талант организатора, став первым ректором (и. о.) университета св.
Владимира. 

В 1835 г.  Брадке пригласил неженатого Цыха на Рождество (Сочельник) в семейном
кругу.

В комнате было темновато, за стенкой слышался детский плач.  Откуда-то шло
легкое теплое дуновение. Надев пенсне, он обернулся и ОНА предстала перед ним в
своей почти ирреальной красоте. От восхищения он по-детски приоткрыл рот.
Нравиться – спросил Брадке, мягко беря Владимира под локоть. – Рождественское
дерево - продолжил он – символизирует Древо познания добра и зла в Эдемском
саду. Эти сверкающие золотом и серебром яблоки (шары) – напоминание о его
плодах, свечи - символ ангельской чистоты, а Вифлеемская...   Но в этот момент
камердинер распахнул дверь и всех пригласил в обеденный зал. На праздничном
столе уже ждал запеченный карп с картофелем, селедка (рыба – др. символ имени
Христа), квашенная капуста и другие деликатесы. 

В Рождественскую ночь Владимир медленно шел по скрипучему снегу. В ушах еще
звучали обрывки фраз, звон бокалов, смех. Когда-нибудь и я... Загадать? Он
посмотрел на небо. Звезд не было. Внезапно на нем появился огненный силуэт
черта. На мгновение всё вокруг озарились призрачным розоватым светом, И перед
изумленным Цыхом возник незнакомец с длинным носом. Он дружелюбно подмигнул и
Володя потерял сознание. Очнулся он у себя на постели. За окном уже серил день
Что же с ним было ... Как ни старался ничего вспомнить не мог. Ах да, Ёлка... и он
улыбнулся. В следующем 1836 году Цых был выбран ректором (без и. о.)
университета св. Владимира, вторым в его истории... Сбылось?!

А еще через год Цыха не стало. Ему было только 32...

\ii{24_12_2021.fb.fb_group.story_kiev_ua.3.jolka.cmt}
