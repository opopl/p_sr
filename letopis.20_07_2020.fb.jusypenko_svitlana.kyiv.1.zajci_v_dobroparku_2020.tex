% vim: keymap=russian-jcukenwin
%%beginhead 
 
%%file 20_07_2020.fb.jusypenko_svitlana.kyiv.1.zajci_v_dobroparku_2020
%%parent 20_07_2020
 
%%url https://www.facebook.com/media/set/?set=a.3660726390660727&type=3
 
%%author_id jusypenko_svitlana.kyiv
%%date 
 
%%tags dobropark,velykden.krolik
%%title Як себе почувають зайці в Добропарку 2020
 
%%endhead 
 
\subsection{Як себе почувають зайці в Добропарку 2020}
\label{sec:20_07_2020.fb.jusypenko_svitlana.kyiv.1.zajci_v_dobroparku_2020}
 
\Purl{https://www.facebook.com/media/set/?set=a.3660726390660727&type=3}
\ifcmt
 author_begin
   author_id jusypenko_svitlana.kyiv
 author_end
\fi


Зайці мають декілька локацій, вигляд чудовий, але ті, які мали зовнішні
прикраси, лишилися без прикрас. Діти допомагають зайцям змінювати позицію. Не
встигла озирнутися, як Самчик з кола перемістився під власний кущик петунії.
Здається, вони оживають, коли ти на них не дивишся. Люблю ці милі створіння.
Наче побачила друзів! upd вдруге шукали лавандового - не знайшли.
