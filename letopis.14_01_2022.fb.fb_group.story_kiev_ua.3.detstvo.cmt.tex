% vim: keymap=russian-jcukenwin
%%beginhead 
 
%%file 14_01_2022.fb.fb_group.story_kiev_ua.3.detstvo.cmt
%%parent 14_01_2022.fb.fb_group.story_kiev_ua.3.detstvo
 
%%url 
 
%%author_id 
%%date 
 
%%tags 
%%title 
 
%%endhead 
\zzSecCmt

\begin{itemize} % {
\iusr{Rima Ternovski}
Такой душевный пост, тепло и хорошо на душе. Благодарю Вас  @igg{fbicon.hands.pray} 

\iusr{Ірина Бондарєва}
Спасибо Вам! Мне очень приятно... @igg{fbicon.smile} 

\iusr{Анна Шустерман}
Ирина, замечательно! Очень перекликается с моей, когда-то, любовью поболеть.
Спасибо за пост! Пишите ещё, у Вас хорошо получается.@igg{fbicon.heart.red}

\iusr{Ірина Бондарєва}
Благодарю! Радостно и приятно! @igg{fbicon.face.smiling.eyes.smiling}  @igg{fbicon.heart.beating} 

\iusr{Igor Popell}

Я в далеком детстве любил листать фото-альбом у бабки с дедом на Подоле. У меня
даже название для него было copyrighted \enquote{дашле}. Потому что слово \enquote{дальше} не
выговаривывываловось! Поэтому было просто \enquote{давай дашле!} )))

\iusr{Ірина Бондарєва}
Здорово!  @igg{fbicon.smile} 

\iusr{Ludmila Teslenko-ponomarenko}
так, бабусіни альбоми - то любов назавжди! Чудова розповідь, пречудова! Хворіти я теж любила

\iusr{Ірина Бондарєва}
Щиро Вам дякую! Дуже приємно! @igg{fbicon.smile} 

\iusr{Юрій Федосєєв}
Яка тепла історія

\iusr{Ірина Бондарєва}
Щиро дякую! @igg{fbicon.smile} 

\iusr{Оксана Ходорич}
Дуже ніжно і смачно про дитинство

\iusr{Ірина Бондарєва}
Дякую! @igg{fbicon.smile} 

\iusr{Bella Tarapakina}
Очень хорошо написано, с удовольствием читала. Спасибо ⚘

\iusr{Ірина Бондарєва}
Вам спасибо большое! @igg{fbicon.smile} 

\iusr{Alla Yevstratova}
Очень душевно. Но что это за дикий текст. Никто не заметил?

\iusr{Ляля Пышная}
\textbf{Alla Yevstratova} , прочитайте в оригинале. У меня тоже бывает такой шок. А
потом понимаю, что читаю автоматический перевод.

\iusr{Alla Yevstratova}
Я имею ввиду неправильный, исковерканный русский язык и дикую стилистику.

\begin{itemize} % {
\iusr{Светлана Манилова}
\textbf{Alla}, пост в группе на украинском. Видимо, Вы читали на русском.

\iusr{Павел Гурнік}
\textbf{Alla Yevstratova} Не знаєте української мови ?
А що так ?

\iusr{Наталья Владимировна Ярмоленко}
\textbf{Alla Yevstratova} Аби шо!!
\end{itemize} % }

\iusr{Алина Шиндировская}
И мне мама рассказывала, как она любила с книжечками болеть

\iusr{Мария Горбатенко}
Із великим задоволенням читаю такі щирі і добрі пости @igg{fbicon.hands.pray} 

\iusr{Ірина Бондарєва}
Щиро Вам вдячна! @igg{fbicon.smile} 

\iusr{Ivan Krakovsky}
Душевно и очень интересно!

\iusr{Ірина Бондарєва}
Благодарю! @igg{fbicon.smile} 

\iusr{Klara Mezhebovsky}
Я тоже часто болела в детстве-очень похоже, все хорошо описано

\iusr{Ірина Бондарєва}
Спасибо! @igg{fbicon.smile} 

\iusr{Ольга Писанко}
Дуже теплий нарис! Теж згадала, як під час дитячих хвороб залюбки переглядала
сімейні фото, а бабуся розповідала де і хто...

\iusr{Ірина Бондарєва}
Дуже Вам дякую! Так, всі ми, чи майже всі, дітьми так любили роздивлятися фото
рідних... затамувавши подих... @igg{fbicon.smile}  @igg{fbicon.heart.beating} 


\iusr{Ольга Скуцкая}

Я теж народилася в лікарні Калиніна на вулиці Зоологічній. Пам'ятаю жартувала,
що народилася в зоопарку)))

\begin{itemize} % {
\iusr{Ірина Бондарєва}

Чудово! Ми, бувало, ходили до лікаря, вже коли я підросла, через Пушкінський
парк... любила гуляти в ньому... ну і зоопарк, зрозуміло, теж... @igg{fbicon.smile} 


\iusr{Ольга Скуцкая}
\textbf{Ірина Бондарєва},

а ми з бабусею в зоопарк завжди йшли через кіностудію Довженка, там у садочку
паслись коні і ми їх годували печівом)


\iusr{Ірина Бондарєва}
От про коней я не знала... на жаль...
\end{itemize} % }

\iusr{Наталья Рубанская}

І я народилась в Калініна, живу все життя на Борщагівській, а педіатр схожа у
мене була, Горбатенко Ріта Антонівна ( приймала на Комінтерну). Варню не варю, а
гриби мариную

\begin{itemize} % {
\iusr{Ірина Бондарєва}
Як чудово! А в якому номері живете?

\iusr{Наталья Рубанская}
\textbf{Ірина Бондарєва} в 12

\iusr{Ірина Бондарєва}
Це ближче до проспекту Перемоги, раніше Брест-Литовський... я у 21 номері жила. Навпроти нас був 16

\iusr{Наталья Рубанская}
\textbf{Ірина Бондарєва} це над трамваєм, 16 поверховий

\iusr{Ірина Бондарєва}

Ближче до \enquote{мостіка} - так дорослі казали про будинки, що були розташовані на
початку вулиці... \enquote{Він, вона живуть біля мостіка} Там ще фабрика була,
здається \enquote{Промткач} називалася

\end{itemize} % }

\iusr{Ирина Слободенюк}

Так уютно, тепло и нежно. Плед, варенье и даже гарчее молоко, которое пью до
сих пор, когда болею. Вот и сейчас с чашкой молока (горячего с маслом, мёдом и
содой) прочла Ваш душевный рассказ. Спасибо!

\begin{itemize} % {
\iusr{Ірина Бондарєва}

Вам большое спасибо! @igg{fbicon.smile} 
Как Вы его пьете?! Мне страшно вспоминать! Выздоравливайте!

\iusr{Ирина Слободенюк}
\textbf{Ірина Бондарєва} спасибо!

\end{itemize} % }

\iusr{Tetiana Desiatka}
Щиро дякую за ваш душевний допис. Згадала і своє дитинство. @igg{fbicon.hands.pray} 

\iusr{Ірина Бондарєва}
Дякую Вам! @igg{fbicon.smile} 

\iusr{Defektolog Logoped}

Душевно. Я тоже часто болела. Бабушка натирала козьим жиром. И меня также поили
молоком с содой и мёдом. Иногда вместо коровьего молока заставляли пить козье.
Это было ужасно (козье молоко) для меня. Фото храню в семейном архиве. Подобный
врач был у моих детей. Вчера ей исполнилось 85лет. Дети мои взрослые. А я
каждый день ей желаю здоровья. И таких,как я,очень много. Общаемся до сих пор.
Спасибо за пост.

\begin{itemize} % {
\iusr{Ірина Бондарєва}

Спасибо Вам большое! Да, козье молоко это ещё страшней! Козьим жиром меня тоже
натирали и ещё скипидарной мазью  @igg{fbicon.eyes} 

\iusr{Defektolog Logoped}
\textbf{Ірина Бондарєва} 

И скипидарной мазью также меня натирали. И смесь алоэ вера с медом я пила по
столовой ложке 3раза в день.. У нас всегда рос цветок. Почему-то бабушка
называла его агава.

\iusr{Ірина Бондарєва}

У нас тоже всегда рос этот цветок, но бабушка называла его столетник. Его соком
мне закапывали нос при насморке

\iusr{Нина Светличная}
\textbf{Ірина Бондарєва} 

Так, алое тоді душе широко використовували. Це вже пізніше я дізналась, що з
ним треба бути дуже обережним - різко підскакує тиск і, взагалі, багато
\enquote{побочок}(

\end{itemize} % }

\iusr{Марина Яловенко}
Теплые воспоминания!

\begin{itemize} % {
\iusr{Ірина Бондарєва}
Спасибо!

\iusr{Марина Яловенко}
\textbf{Ірина Бондарєва} ,особенно душевно о педиатре-редко сейчас таких врачей встретишь!!! потому, что раньше шли по-призванию

\iusr{Ірина Бондарєва}
Да, правда! Это действительно было служение, а не работа...
\end{itemize} % }

\iusr{Ассоль Грей}
Як Чудово і комфортно написано! Аж тепло на душі!

\iusr{Ірина Бондарєва}
Сердечно Вам дякую! @igg{fbicon.smile} 

\iusr{Ассоль Грей}
Гарненька дівчинка на фото!)

\iusr{Ірина Бондарєва}
Дякую! @igg{fbicon.smile} 

\iusr{Татьяна Иванова}
Спасибо!

\iusr{Ірина Бондарєва}
И Вам спасибо!

\iusr{Людмила Билык}

Яка гарна дівчинка... Які чудові і розумні очі... Мене так само мама
лікувала...  @igg{fbicon.grin} ... І я, так само, коли хворіла, присвячувала
час своїм улюбленим справам - читанню і перегляданню сімейних альбомів...
@igg{fbicon.wink}  Не було телевізора, інтернета, телефона врешті - решт...
Зате було багато часу для розвитку своєї уяви і фантазування... Дякую за тепло
спогадів, яке подарували...  @igg{fbicon.heart.sparkling}  А в лікарні ім.
Калініна я народила доньку...  @igg{fbicon.smile} 

\begin{itemize} % {
\iusr{Ірина Бондарєва}

Щиро Вам дякую за добрі слова! Так, правда, для розвитку уяви були всі умови і
все тому сприяло! Телевізор був, як з'явився, дуже якийсь обмежений час на
день.

\end{itemize} % }

\iusr{Валентина Самойленко}

Які приємні і душевні спогади, все, що з дитинства, чомусь саме особливе, чи є час
споглядати, а потім життя пробігло одна метушня. Я теж і зараз люблю переглядати
світлини, то було життя, у кожного своє, а все швидко минає тільки залишаються
спогади.

\end{itemize} % }
