% vim: keymap=russian-jcukenwin
%%beginhead 
 
%%file 31_12_2019
%%parent dec_2019
 
%%url https://www.facebook.com/larysa.nitsoi/posts/2907050412660405
 
%%author 
%%author_id 
%%author_url 
 
%%tags 
%%title 
 
%%endhead 
\subsection{31 декабря 2019, вторник}
\label{sec:31_12_2019}
\Purl{https://www.facebook.com/larysa.nitsoi/posts/2907050412660405}

Як школярі олів'є для фронту рятували.
Розповім вам теплу історію, новорічно-різдвяну. 
Є в Києві спільнота, яка з гумором називає себе «Орден «Олів’є»». Ця сільнота вже шостий рік поспіль напередодні нового року збирається разом і ріже олів’є на фронт нашим бійцям. Долучайються й інші люди, скидаються для тієї спільноти продуктами чи грошима – в результаті спільної праці, їде на фронт у новорічну ніч кілька тон олів’є.

\ifcmt
  pic https://scontent-iad3-1.xx.fbcdn.net/v/t1.6435-0/p180x540/80851134_2907050342660412_3891748761159860224_n.jpg?_nc_cat=102&ccb=1-3&_nc_sid=730e14&_nc_ohc=MxP7-rmBHdQAX-RU0O1&_nc_ht=scontent-iad3-1.xx&tp=6&oh=be5d6c0e1aebd2415ae3fea9d0df1cb5&oe=60CB1690
\fi


Цього року теж зібралися нарізати ні мало, ні багато - 3 тони, а як вийде - 6 тон салату. Та от халепа! Всі продукти є: картоплю, моркву ресторани поварили, 8 тисяч 400 яєць зварених привезли, ковбасу принесли, майонез відрами поскладали – огірків немає! Сталися непередбачувані обставини – не доїхали солоні огірки до Києва, 300 банок.
Кидає спільнота клич киянам у ФБ, хто має огірки, принесіть, будь ласка.
День проходить – немає огірків. Другий день проходить – шукають огірки. А час ікс з нарізування олів’є і Новий рік невблаганно наближаються. 
Ну, думаю, треба щось робити. Кволо несуть кияни огірки. Так уже сталося, свято – всі пороз’їжджалися, та й, якщо розібратися, де в киян на асфальті візьмуться солоні огірки. А час усе цокає. І подумалося мені, що треба звернутися в села. Село завжди рятувало Україну. А волонтери "ордену" мені й відповідають, якби був час, то й не проблема, можна звертатися в села, а тоді гратися в доставку бутликів з кожного села. Але ж немає часу. 
«Ну, виходить, треба звернутися в найближчі села», - подумала я і набираю свою любиму Бородянку. Це хоч і не село, та більше 30 разів виїжджала я в цей район по сільських школах. Є в мене в Бородянці головний "енерджайзер" тих письменницьких зустрічей по школах і бібліотеках, освітянка, Людмила Петрівна Кириленко. 
Телефоную. Так і так, треба солоні огірки для олів’є бійцям на фронт у новорічну ніч.
- Ой, боже, - каже  Людмила Петрівна, - якби ж не вихідний день за вікном, 25-те, Різдво, то без проблем, зібрали б,  але ж вихідний зв'язує руки.
- Людмилочко Петрівно, - кажу, - попросіть своїх дівчат з бібліотек, учителів, чиї телефони у вас є з тих сіл, що вздовж траси, нехай би спустилися в погрібець і в кого є, винесли по бутлику огірочків, а я проїду машиною і все зберу, завезу на Київ. Буде 5 бутлів – добре. Буде 10 бутликів – ще краще. 
- А скільки треба? – питає Людмила Петрівна. 
- Та, - кажу, - багато треба. Ви на те не дивіться, скільки треба. Скільки буде, за те й буду вдячна.
Все ж зізналася їй, що потрібно банок триста. Знову зітхнула Людмила Петрівна, що вихідний все ускладнює, та й стала обдзвонювати знайомих, просити огірки. Мовляв, Ніцой завтра по ті огірки приїде. Стільки разів була в районі, не відмовляла нікому, нічого не просила в обмін. Прийшов час віддячити. А тут ще й така свята справа, бійцям на фронт. 
І почалася ланцюгова реакція, власне, на яку я й розраховувала. Стали вчителі й бібліотекарі дзвонити одне одному, передавати, що на завтра дуже потрібні солоні огірки для олів’є бійцям на фронт. Почули про те учні та їхні батьки. Кому вчителі зателефонували, кому в групу скинули. 
І потяглися вранці до сільських шкіл вервечками школярики, які несли в руках банки з огірочками на новорічний салат для бійців. 
Нове Залісся, Загальці, Бородянка, Микуличі, Клавдієве, Бабинці, Немішаєве…
Ще не знала я, скільки буде банок, та відчула, що мого багажника стає замало. Треба десь брати бусика. Телефоную давньому знайомому, невідмовному в усіх добрих справах депутатові Київської обласної ради, свободівцю Ігореві Сабієві. Так і так, виручайте, треба бус. 
А той і каже:
- Зараз я наберу Бородянщину, є в мене там Близнюкова Людмила, в неї бус, заправлю.
Телефонує їй, питає, чим зайнята, а вона й відповідає: «Огірки ходжу по селу збираю». Сміється Сабій, каже, що він теж у цій справі телефонує, просить ще й буса взяти і допомогти ним до Києва ті огірки доправити.
Одне слово, коли зранку стали їздити з села до села по школах і збирати ті банки, які назносили діти, вчителі, бібліотекарі і всі, хто дізнався, то нарахували не 10 бутликів огірків, не 30, а близько трьохсот, як і було потрібно. 
Волонтери з "ордену олів'є" лише в долоні сплеснули від розчулення, коли ми приїхали до них і стали той бус розвантажувати.  Їм раніше банки з заводу доставляли, всі однакові, упаковані. А тут дістають з буса різнокаліберні баночки й бутлики, під металевими кришками й капроновими, різного посолу й рецептури, любовно обмотані газетами чи пакетами, щоб не побилися, упаковані в торбинки й кульочки… 
Уявили собі стежки в різних селах, якими одночасно поспішають до шкіл дорослі й особливо діти з банками в руках, щоб встигнути. І як перед тим, діставши банку з погреба, мами чи бабусі ті банки обмотують і наказують у слід дитині "ти ж не розбий!" - хотілося присісти й ті банки гладити. 
Згадала реакцію вчителів. Учителі дякували, що їх долучили до такої можливості хоч у такий мізерний спосіб висловити вдячність нашим воїнам. У цей час я подумала про патріотичне виховання... 
Коли спільнота олів’є з нами прощалася, нас  розчулено запитали: 
- Що написати бійцям на пластикових відрах, у які розфасують нарізаний салат?
Я попросила: 
- Напишіть «До ПЕРЕМОГИ»!
---------------------------------------------------------------------------------
Коли ви будете чути від когось, що українці не вміють єднатися, українці збайдужіли – обривайте такі слова й затикайте рота. То спеціально нас обмовляють, нав’язують нам думку, що ми якісь не такі. Навмисне нас налаштовують самих проти себе. Це неправда, що українці єднаються лише перед стратою. Українці завжди єднаються навколо гарної ідеї.
З НОВИМ РОКОМ!
