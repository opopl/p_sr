% vim: keymap=russian-jcukenwin
%%beginhead 
 
%%file 12_10_2020.news.pravda_com_ua.1_fedoriv_ukraine
%%parent 12_10_2020
 
%%endhead 

\subsection{Андрій Федорів: Україна --- це свобода, вольниця, бардак, хаос}

\url{https://www.pravda.com.ua/articles/2020/10/12/7269382/}

Кажуть, він різкий і категоричний. Якщо вважає, що проект --- лайно,
співпрацювати не буде.

За це його не люблять.

"Якщо ти думаєш, що світ --- це місце, де живуть виключно твої конкуренти, ти не
розвиваєшся сам. І не розвиваєш ринок", --- повторює він.

Те, що він має талант змінювати середовище навколо себе, визнають навіть ті,
хто його не любить.

Перелік того, що створив Андрій Федорів може бути задовгим. Він --- засновник
креативної інвестиційної компанії Fedoriv Group, маркетингової агенції Fedoriv
Agеncy, Фундації Дарини Жолдак, бізнес-простору Kooperativ, каналу Fedoriv Vlog
на YouTube, автор навчального курсу для власників бізнесу BrandFather.

У 2019 році спільно з Міністерством цифрової трансформації України його агенція
створила бренд цифрової держави "Дія", метою якого є оцифровка та автоматизація
всіх державних послуг.

Федорів --- один з найкращих в Україні фахівців з брендінгу. А ще він візіонер,
здатний відповісти не лише на питання "як?", але й на питання "куди?" і
"навіщо?".

Саме тому "Українська правда" запропонувала Андрію Федоріву розповісти, який
ребрендінг потрібен Україні і в чому наша конкурентна перевага.

В інтерв’ю УП Федорів розповів про Україну як стартап.

Про сильні ідеї, яким не будуть плескати в долоні, і які потрібно виборювати.

Про те, чому наші факапи є нашими можливостями.

Про те, чому шкодять порівняння на кшталт "Київ --- це новий Берлін".

Про прозорі стіни сучасного світу і про те, чому ребрендінг України не має
сенсу, допоки "Беркут" б’є людей.

Про своє бачення перемоги і фотографії майбутнього.

Обережно: спойлер. 

Під час розмови жодного разу не прозвучало слово "зрада". 

\subsubsection{"Україна – це голодний підліток з рюкзачком"}

─ Пропоную рольову гру. Уявіть, що я ваша потенційна клієнтка. Прийшла до
вашого офісу і кажу: "Андрію, у нас велетенська корпорація. Майже 42 мільйони
співробітників. Величезні ресурси. Давні традиції. Але щось пішло не так.

Ми, як спільнота, не знаємо, куди йдемо і до чого прагнемо. Ми потребуємо
чарівного пенделю ззовні, щоб набути другого дихання, наростити м’язи і почати
завойовувати нові ринки. Нам потрібен ребрендінг. Залишаємо лише назву –
Україна".

Візьметеся за такий проєкт?

─ Чудово. Мені подобається така корпоративна рольова гра. Те, що ви описали, є
класичним ребрендінгом. Це коли компанія, що пройшла довгий шлях і має певні
активи, ставить собі саме ті питання, що й будь-який стартап.

Класичний ребрендінг ─ завжди ре-брейнінг. Переосмислення. В цьому випадку ─
переосмислення України.

─ Тобто, ви погодилися би взятися за ребрендінг?

─ Не має значення. Ви прийшли до мене з пропозицією, яка потребує обговорення.

Будь-який ребрендінг ─ в першу чергу історія про зміни.

Хто буде лідером змін? Хто ця людина, яка візьме на себе емоційну
відповідальність за проєкт?

Хто є команда, той спецназ, який буде втілювати ідеї?

Хто ми? В якому сегменті працюємо? Хто наші конкуренти? Проти кого ми граємо?
Хто забирає наші гроші?

Куди ми йдемо? Яка наша велика амбітна мета?

Як про неї розповісти, щоб люди захотіли до нас приєднатися?

Це перше, про що б я запитав.

А ще обов’язково попередив би: коли йдеться про зміни, готуйтеся до спротиву.
Будь-яка ідея, а особливо сильна, ризикована, радикальна ідея – завжди
беззахисна. Вона як ембріон. Їй не будуть всі плескати в долоні. Її потрібно
виборювати.

- Які, на вашу думку, мають бути перші три кроки з ребрендінгу країни? 

─ З усією повагою до української історії та традицій, перше, що нам потрібно зробити, це визнати, що ми – стартап. 

Визнати, що ми підліток, як країна, як держава, як економіка. До моменту, поки
ми намагаємось виглядати або діяти як хтось інший, в нас є проблеми.

Другий крок – не намагатися реанімувати неефективні пострадянські гнилі
системи. Їх потрібно відключати. Не можна це міністерство зробити ефективнішим?
Окей, вимкнули і забули, що тут взагалі колись було міністерство.

Третій крок – сфокусуватися на точках сили. Не намагатися бути
надуніверсальними. Мовляв, і в цьому, і в цьому ми розумні й гарні. Ні.

─ Ви часто повторюєте, що маркетинг – це відповідь на просте питання: чим ви
відрізняєтеся від конкурентів? Чим ми, як спільнота, відрізняємося від інших? В
чому ця відмінність перспективна для позиціонування в світі?

─ Ми круто стрибаємо у висоту, бо в нас є Сергій Бубка.

У нас фантастичні архітектори, айтішники, у нас є музика. Питання як все це
інтегрувати в світовий контекст. Не намагаючись бути тими, ким ми не є.

В Україні нереальний сервіс, який можна отримати завдяки підприємцям.

Коли ми приходимо в Києві в будь-який ресторан і говоримо: будь ласка, можна
оцю страву без цього або без цього, чи можна нам просто зробити спагеті, в
дев’яноста відсотків зі ста скажуть: звичайно!

Коли ми з дітьми приходили до ресторанів у Нью-Йорку, нам казали – неможливо.

Особистий випадок, що яскраво ілюструє відмінності.

Моя дружина лікувалася в шпиталі в Берліні. Я був поруч. Різко змінилася
погода, я раптом став почуватися жахливо. Дуже боліла голова, стало якось зле.
Ми покликали медсестру, попросили виміряти мені тиск. В неї в руках був
тонометр. Вона відповіла: "Ні. Бо ви не є пацієнт. Пацієнт – ваша дружина. А ви
звертайтесь до лікаря".

Приватна клініка, за яку ми сплатили. Гроші не працюють. Працюють правила.В
Україні цього немає. 

В нашому ДНК немає культу раз і назавжди написаних правил світоустрою.

Ми стартап, і ми стартуємо знизу, а не зверху. Як кажуть інвестори, у нас
низька база.

Так, з моменту рестарту України в 1991 році ми багато часу змарнували. Або
нічого не робили. Або руйнували те, що було. Або забирала час необхідність
прийти до тями. Я, до речі, не впевнений, що ми до неї повною мірою прийшли.

Але є й хороші новини. Те, що ми не зробили колись – це наші можливості
сьогодні. 

Знаєте, є такий вислів: "Чим гірше, тим краще". Чому? Уявіть: приходить прораб
на замовлення в будинок. А там все ідеально. Все, що йому потрібно зробити, це
підключити кілька розеток. Чи це добре? Для будинку і для його мешканців
напевне так. Для прораба – ні. Бо він залишається без роботи.

─ Тобто наша головна конкурентна перевага в тому, що ми все ще на низькому
старті?

─ Неможливо на стартап накласти стандарти зрілої корпорації. Коли компанія
маленька і там працює 10-15 людей, у них не може бути чітко визначених
інструкцій і розписаних ролей. В будь яку мить можна сказати, окей, залишаємось
сьогодні на ніч. Ти менеджер, але ти сьогодні вночі будеш малювати макети. Бо
ми стартап з якостями стартапу.

Зрілі країни – це пенсійного віку пани і пані. В них вже все є, в них є, що
втрачати, в них побудована інфраструктура.

А Україна – це хлопець або дівчина з рюкзачком. В такої людини, окрім рюкзачка,
є лише одне – мотивація, пов’язана з почуттям голоду. Це сильна мотивація. 

Чому стільки людей їдуть працювати за кордон? Тому що вони голодні. Немає
можливостей та альтернативи, є тільки ця мотивація.

Зрозумійте, не існує шорт-листу, в якому Україна стоїть поруч з Англією,
Німеччиною, США. Україна з інфраструктурної, технологічної, бізнесової точки
зору є недорозвиненою.

Мені здається, що ми часто використовуємо хибні порівняння, які стимулюють
грати проти сильних гравців на їхньому полі. Ми говоримо: "Ми майже як
Німеччина. Київ – це новий Берлін".

Камон! По-перше, ми аж ніяк не Німеччина. Але й не треба казати "Ми –
недоНімеччина".

Нам потрібно шукати щось своє, свій осередок, своє повідомлення. Те, що ми
можемо запропонувати світові.

Ми граємо в іншій лізі. Це важко визнати, але це треба визнати.

Визнання цього вже є перемога.

