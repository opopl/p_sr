% vim: keymap=russian-jcukenwin
%%beginhead 
 
%%file 16_02_2022.tg.bogatyrev_daniil.1.nakal_propagandy
%%parent 16_02_2022
 
%%url https://t.me/polit_bogatyr/310
 
%%author_id bogatyrev_daniil
%%date 
 
%%tags patriotizm,propaganda,rossia,ugroza,ukraina
%%title Накал шароварно-вышиваночной пропаганды
 
%%endhead 
 
\subsection{Накал шароварно-вышиваночной пропаганды}
\label{sec:16_02_2022.tg.bogatyrev_daniil.1.nakal_propagandy}
 
\Purl{https://t.me/polit_bogatyr/310}
\ifcmt
 author_begin
   author_id bogatyrev_daniil
 author_end
\fi

Что по-настоящему раздражает, так это то, что на волне раздуваемой западными
СМИ военной истерии, украинские власти решили удвоить, а то и удесятерить накал
шароварно-вышиваночной пропаганды. Соцсети полны сообщениями от родителей, чьим
детям едва ли не в приказном порядке предписывают являться в школу в вышиванках
и разучивать гимн, чтобы хором петь его на втором уроке, в духе знаменитого
отрывка из фильма \enquote{Собачье сердце}, где большевики поют
\enquote{Суровые годы уходят}.

Форменный дурдом. Особенно - для детей из интеллигентных русских семей, которым
вся эта сельская эстетика глубоко чужда.

Собственно, культ села и примитивной культуры в противовес высокой культуре
города, в Украине пропагандировался с момента обретения независимости. Ещё в
период моего обучения в школе (1999 - 2009 гг.), во многих кабинетах были
расставлены глинянные горшки и развешаны \enquote{рушники}. Где-то пару раз
даже видел инсталляции из соломы. В среде городских интеллигентных людей,
которыми были подавляющее большинство учителей и учеников, всё это смотрелось
дико. Примерно как на экскурсии в этнографический музей. Вероятно, у
современных школьников это ощущение от происходящего решили довести до
терминальной стадии.
