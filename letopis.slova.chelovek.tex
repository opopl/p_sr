% vim: keymap=russian-jcukenwin
%%beginhead 
 
%%file slova.chelovek
%%parent slova
 
%%url 
 
%%author 
%%author_id 
%%author_url 
 
%%tags 
%%title 
 
%%endhead 
\chapter{Человек}
\label{sec:slova.chelovek}

%%%cit
%%%cit_pic
%%%cit_text
Самое дорогое - это \emph{Люди}
%%%cit_comment
%%%cit_title
\citTitle{Желая быть ангелом, превращаемся в демона}, 
Протоиерей Андрей Ткачёв, youtube.com, 04.06.2021
%%%endcit

У диплоїдних організмів, таких, як \emph{Людина}, присутні копії майже всієї
ДНК.  Зокрема, є пари хромосом по одній від кожного з батьків, однак для
дослідників цінність представляє сума тільки однієї копії пар хромосом.
Фактично, розмір геному \emph{Людини} являє собою приблизно 6 мільярдів пар
основ, однак вчені використовують тільки половину в якості
\enquote{репрезентативної} копії.  Вчені зазначили, що 8\% геному пропустили у
2003 році не через відсутність в них важливої інформації, а через технологічні
обмеження. Дослідники наголосили, що ці обмеження вдалося подолати завдяки
секвенуванню, яке розробили вчені з Pacific Biosciences і Oxford Nanopore,
\textbf{Вчені повністю розшифрували геном людини}, day.kiev.ua, 03.06.2021

В жизни любого \emph{человека} есть, как хорошие, так и отрицательные моменты.
Существует заблуждение, что психотерапия направлена на то, чтобы сделать ее
протекание счастливым и беззаботным. Но на самом деле, специалисты в этой сфере
работают над тем, чтобы последствия испытанного стресса, страха, неудач не
мешали вам в дальнейшем. Например, если вы не можете устроиться на хорошую
работу – психотерапевт не поможет. Но, если вы боитесь ходить на собеседования,
испытываете трудности в коммуникации с новыми людьми, рушите связи из-за
конфликтного характера – вполне,
\textbf{Для чего нужна психотерапия личности?}, beztabu.net, 02.06.2021

Фаина ищет и находит простые слова, образы, сюжеты. У нее, в отличие от
современных потребителей, болит душа. Не слышать ее может лишь тот, у кого души
нет. Для кого патриотизм это не жертвенность во имя страны, а способ
обогащения, для кого война только прибыльный бизнес, для кого своя рубашка
ближе к телу, а хата с краю.  Послушайте ее обращение в ООН. Никакой
театральности, надрыва, пафоса. Она, скорее, робеет, чем кого-то обвиняет,
требует. Ей как-то неудобно, ведь она понимает, что дети не помнят мира, а
большие дяди в политике не понимают. Точнее, все эти дяди понимают, но просто
дяди негодяи, и заняты другим. А ей, Фаине, за этих дядей неловко и стыдно.
Хотя стреляют не по им, а по ней. Вот это и есть крест настоящего
\emph{Человека}, испытывать стыд за чужую подлость. Маленький \emph{Человек},
но большая душа,
\textbf{Гибель людей на Донбассе говорит о том, что наше общество больное},
Денис Жарких, strana.ua, 04.06.2021

Ведь важно не то, что плетут политики, а важно, чтобы не стреляли, чтобы \emph{Люди}
не погибали, чтобы мы опять стали \emph{Людьми}, а не чинами и капиталами. Общество
больно, его покидает \emph{Человечность}. Очевидно, прохудилась общественная совесть.
Как вы думаете?,
\textbf{Гибель людей на Донбассе говорит о том, что наше общество больное},
Денис Жарких, strana.ua, 04.06.2021

Как жаль, как ужасно жаль, что не существует больше такого \emph{Человека}... Он
остался в прошлом. Он продолжает жить лишь в книгах, фильмах, рассказах наших
дедов...  \emph{Человек} утратил свое величие, и произошло это не в одно мгновение.
Ныне мы живем при капитализме, и нет больше тех великих идеалов - да вообще
никаких идеалов нет, а есть лишь личный интерес, не выходящий за рамки
материального: тепла, сытости и комфорта. Мы превратились в потребителей. Идея
низвергнута. У нас - тех, кто был когда-то велик как единый народ - ныне
раздор, вражда и отчужденность. Мы думаем по-другому, поступаем по-другому. Мы
извратили многие понятия. Мы упрекаем и обвиняем друг друга, завидуем и злимся,
предъявляем взаимные счеты. Мы больше не \enquote{Величество}...  Мы больше не братья.
Мы живем в разных государствах, между нами пролегли границы. Мы отдаляемся друг
от друга с огромной скоростью... ,

Каждый \emph{человек} — это целый огромный мир в себе, это сложная загадка с долгим
прошлым и бесконечным будущим, 

То що ж сталося у жовтні 2018 року? Можна з впевненістю стверджувати, що
«Портрет Едмунда Беламі» не має «фіксованого» автора, але має власників. Поміж
численними співавторами, які створили це зображення, має бути введена на рівних
нежива субстанція, якою є алгоритм. Тобто 2018 року у світ «високого мистецтва»
увійшла нова гібридна авторська сутність – \emph{людино-машина}. Це сталося
цілком очікувано. Технології, які значною мірою змінили навколишній світ, тепер
сфокусовані на \emph{Людині} і на всьому живому на планеті. Ми стоїмо перед
питаннями: Чи дійсно \emph{Людина} звільнилася від Бога і сама може впливати на
все живе? Чи може \emph{Людина} потрохи зникне на корить \emph{людино-машини}?
Чи не стане \emph{людське} життя лише виконанням програми, яку написали десь, у
науковій лабораторії? Яким буде поняття/розуміння \emph{Людини} в майбутньому?,
\citTitle{Мистецтво в добу «привидів у машинах»}, Богдан Шумилович, zaxid.net, 02.06.2020

Облишимо культурно зумовлені факти. Маркес, Борхес і Льйоса змусили мене
задуматися над іншим. Власне, над тим моментом зрілості, коли сам обираєш, як
ти називаєшся. Навіть прийнявши заготовлені родом варіанти, \emph{людина} має свободу
вибору. Хтось живе як Толік, хтось — як Анатолій Васильович, хтось взагалі стає
Анатоль. Кілька літер, додаткових звуків і відразу зовсім інший портрет. Навіть
те, як людина вимовляє власне ім'я, повільно й поважно, чи квапливо і голосно,
скаже чимало про власника і його ментально-душевний стан.  Напевне, в
дорослості, ми всі розуміємо — немає красивих і некрасивих імен. Є особистості,
що своїм досвідом і характером перетворили власне ім’я на символ. Символ сили
чи слабкості, перемоги чи підлості,
\citTitle{Як ти називаєшся?}, Анна Данильчук, day.kyiv.ua, 10.06.2021

%%%cit
%%%cit_head
%%%cit_pic
%%%cit_text
С минимальным наборов навыков уже можно идти во власть.  Очень обидно, что мы в
современном мире так быстро дошли до точки, когда \emph{человеку}, чтобы выжить,
вообще ничего не обязательно.  Ну, разве что держать ложку и уметь застегивать
ширинку. С этим набором умений уже можно придти к власти.  Арифметика - да
нахер ее арифметику, литература - сложно, гуманитарные знания - о чем это,
права \emph{человека} - заумно.  Прямохождение - ок навык, социальная лестница тебя
ждет, \emph{человек 21-го столетия}. Моргай, маши руками, топай, скоро даже кашлять
будет можно, но не всем. Победа будет за нами, одомашненные зверьки
%%%cit_comment
%%%cit_title
\citTitle{Мы дошли до точки, когда человеку, чтобы выжить, вообще ничего не обязательно / Лента соцсетей / Страна}, 
Евгений Ихельзон, strana.ua, 24.06.2021
%%%endcit

%%%cit
%%%cit_head
%%%cit_pic
%%%cit_text
Ну а лидером списка стал \emph{человек}, не имеющий ни украинского гражданства, ни
украинских корней. Это Кристина Квин – даже не посол, а временный поверенный в
делах США на Украине. Ее лидерство – констатация, что страна находится под
внешним управлением. Только в таких условиях дипломат может быть влиятельнее
генпрокурора
%%%cit_comment
%%%cit_title
\citTitle{Список самых влиятельных женщин стал позором Украины}, 
Станислав Борзяков, vz.ru, 26.06.2021
%%%endcit

%%%cit
%%%cit_head
%%%cit_pic
%%%cit_text
Ну, а если и примет, то только лишь для нашего общего блага. Потому что мы не
только правовое, но и социальное государство, где \emph{человек} есть высшая
ценность. ( статья 3). И ради этого \emph{человека} можно немного нарушить
Конституцию или даже узурпировать власть Всем, конечно, нельзя, но Президенту
можно. Он ведь всенародно избранный! А народ наш мудр! С днем Конституции!
%%%cit_comment
%%%cit_title
\citTitle{Конституция Украины содержит целый ряд замечательных положений / Лента соцсетей / Страна}, 
Андрей Головачев, strana.ua, 28.06.2021
%%%endcit

%%%cit
%%%cit_head
%%%cit_pic
%%%cit_text
Также он рассказал, что сегодня на предприятии числится 1580 \emph{человек}, но
реально работает около сотни специалистов.  \enquote{Сейчас мы выводим около
400 \emph{человек} работающих на заводе. Это \emph{люди} разного возраста. Это
обслуживающий персонал, охрана, работники жилищно–коммунальной сферы и около
ста производственников, работающих над задачей}, - сообщил Александр Кривоконь.
\enquote{А другая часть - более тысячи \emph{человек} сегодня находятся на
простое.  Завод у нас находится на простое. Мы работаем официально 4 часа в
неделю}, — разъяснил он
%%%cit_comment
%%%cit_title
\citTitle{ХГАПП больше не может производить самолеты}, Антон Щукин, strana.ua, 27.06.2021
%%%endcit

%%%cit
%%%cit_head
%%%cit_pic
%%%cit_text
Говорим на украинском, но находимся все там же.  За темой русского наследия,
русской культуры, которые якобы транслируются через русский язык независимо от
контента, ускользает вопрос прав \emph{человека}.  Закон про функционирование
государственного языка противоречит Всеобщей декларации прав \emph{человека} и нашей
Конституции. И главное - он ничего не меняет в общественно-политических, в
экономических отношениях. Т.е. нельзя сказать - вот до этого закона было так, а
после принятия закона стало вот так. Ничего не стало. Ничего не изменилось
%%%cit_comment
%%%cit_title
\citTitle{В плане развития страны увеличение числа украиноговорящих ничего не дает / Лента соцсетей / Страна}, 
Павел Себастьянович, strana.ua, 27.06.2021
%%%endcit

%%%cit
%%%cit_head
%%%cit_pic
%%%cit_text
Какое отношение Зеленский к дорогами имеет? А к земле? Никакого абсолютно! Но
он там, потому что требует запад и потому что там деньги. А языки и тарифы
\emph{людям} — причём тут Зеленский, да?  Не надо на людей перекладывать.
\emph{Люди} вас избрали в надежде, что исполните свои обещания. Но вы
исполняете желания олигархов, МВФ и внешнего управления, а на \emph{людей} вам
плевать. Но \emph{люди} у вас не такие, конечно же. А вы хорошие там все
%%%cit_comment
%%%cit_title
\citTitle{Самое простое для власти - переложить ответственность на рядовых людей / Лента соцсетей / Страна}, 
Александр Скубченко, strana.ua, 02.07.2021
%%%endcit

%%%cit
%%%cit_head
%%%cit_pic
%%%cit_text
Французский суд осудил вчера Брижит Бардо - потому что она назвала оппонента
\enquote{\emph{недочеловеком}}. А Олег Семинский, депутат украинского парламента от
партии президента Зеленского, открытым текстом пишет такое:
\enquote{Русские на законодательном уровне - не коренной народ, поэтому они не будут
иметь возможность в полной мере обладать всеми правами \emph{человека} и
основополагающими свободами, определенными нормами международного права, а
также предусмотренными в Конституции и законах Украины. В отличие от украинцев,
крымских татар, караимов и крымчаков}.
В принципе, он может сразу призывать к геноциду - никакого серьезного наказания
за это все равно не последует
%%%cit_comment
%%%cit_title
\citTitle{Семинский может открыто призывать к геноциду - все равно не накажут / Лента соцсетей / Страна}, Андрей Манчук, strana.ua, 03.07.2021
%%%endcit

%%%cit
%%%cit_head
%%%cit_pic
%%%cit_text
Построение \emph{человечеством} ужасающей антиутопии, от рабства в которой людей
спасет только второе пришествие Господа – один из центральных моментов
христианского учения. Знаки и приближение, неминуемое наступление апокалипсиса
христианское общество наблюдало и предсказывало на протяжении всех двух
тысячелетий со времени распятия Христа. И тем не менее каждый кризис, падение
христианской цивилизации оказывались лишь началом очередного пути ее развития.
В настоящий момент мирового кризиса христианства примеров которому трудно и
вспомнить, у России еще есть возможность реализовать себя в избранной и
провозглашенной миссионерской роли Третьего Рима, и столицы Империи Духа.  А
удастся ли ей это, даже не в руках и воле \emph{Человеческой}, но в Божиих
%%%cit_comment
%%%cit_title
\citTitle{Революция Духа – единственный путь спасения России}, 
Юрий Барбашов, voskhodinfo.su, 30.06.2021
%%%endcit

%%%cit
%%%cit_head
%%%cit_pic
%%%cit_text
Основной привлекательной чертой западного пути развития является предлагаемое
упрощение \emph{человека}, избавление его от тягот, труда по саморазвитию,
самосовершенствованию, перипетий социальной борьбы и даже трудностей реализации
в гендерной роли. Для любого живого существа привлекателен путь получения
минимального количества ресурсов и благ с минимизацией усилий. В конце концов,
кратчайшим путем достижения неограниченного удовлетворения и счастья отдельного
\emph{человека} является вживление электродов в мозг и раздражение непосредственно
центров удовольствия или центров выработки гормонов. С большой вероятностью,
что именно к такому способу «вознаграждения» и придет в скором времени западное
общество, вместо грубой замены тех же процессов доступным сексуальным
удовлетворением, алкоголем и наркотиками
%%%cit_comment
%%%cit_title
\citTitle{Революция Духа – единственный путь спасения России}, 
Юрий Барбашов, voskhodinfo.su, 30.06.2021
%%%endcit

%%%cit
%%%cit_head
%%%cit_pic

\ifcmt
  tab_begin cols=3
	width 0.3

		 pic https://avatars.mds.yandex.net/get-zen_doc/5326835/pub_60e3d43021f45644dfa3bfd7_60e40c3b2f1be51987934e66/scale_1200
	width 0.25

     pic https://avatars.mds.yandex.net/get-zen_doc/5238996/pub_60e3d43021f45644dfa3bfd7_60e40c7e4660f21f596e14fa/scale_1200
	width 0.33

     pic https://avatars.mds.yandex.net/get-zen_doc/5284326/pub_60e3d43021f45644dfa3bfd7_60e40cc572ae724b3ddc5778/scale_1200
  tab_end
\fi

%%%cit_text
Интересно было бы пообщаться со всеми этими \emph{людьми} хотя бы через месяц, но
наверняка они будут заняты обсуждением новых версий скорого краха России...
Так что мосты и телеграфы брать надо, а ТВ и Интернет
%%%cit_comment
%%%cit_title
\citTitle{То, что от вас скрывают прокремлевские СМИ. Обзор украинской прессы-2}, 
Мак Сим, zen.yandex.ru, 06.07.2021
%%%endcit

%%%cit
%%%cit_head
%%%cit_pic
\ifcmt
	width 0.4
  pic https://gdb.rferl.org/AF5469CE-AEDC-4722-9B20-D03E49643EA2_w1023_r0_s.jpg
	caption Ті фронтовики, що вижили в Харківському котлі, вважали винуватцем цієї катастрофи маршала Тимошенка
\fi
%%%cit_text
Сам маршал Тимошенко наказав забрати всі бойові знамена військових частин, що
були оточені, до свого особистого літака, і на тому літаку вилетів з фронту,
кинувши тисячі своїх бійців напризволяще. Для нього було головніше, що
«врятовані» знамена, адже поки знамена не здані ворогу, існують і ці частини,
хай навіть увесь людський склад тих частин зник у німецькому оточенні. \emph{Людей}
можна знайти нових, «баби ще понароджують», і зі старими знаменами ці нові \emph{люди}
підуть у новий бій, а про старих \emph{людей} просто забудуть. Не було їх, і все!
%%%cit_comment
%%%cit_title
\citTitle{«Заборонені» спогади з фронтів війни Німеччини і СРСР 1941–1945 років}, 
Ігор Роздобудько, www.radiosvoboda.org, 08.07.2021
%%%endcit

%%%cit
%%%cit_head
Пам'ятаємо, що той, хто рятує бодай одну людину, рятує цілий світ
%%%cit_pic
%%%cit_text
Що нам робити у цій ситуації? В першу чергу стати \enquote{праведниками
Білорусі}, подібно \emph{людям}, які рятували євреїв від нацистів. Виклики на
роботу від українських компаній, запрошення на конференцію, семінар чи абощо,
виклик від навіть дуже далеких родичів на весілля, уродини чи похорон, допомога
із першим житлом та першою роботою після перетину кордону – це все може
врятувати життя комусь із білорусів, наразі замкнених режимом у ізоляції.
Пам'ятаємо, що той, хто рятує бодай одну \emph{людину}, рятує цілий світ.
Наразі за лісами Полісся порятунку потребують тисячі світів, і кожний з нас
може допомогти бодай одному із них
%%%cit_comment
%%%cit_title
\citTitle{Лукашенко перекрив білорусам останню \enquote{дорогу життя}}, 
Євген Дикий, gazeta.ua, 05.07.2021
%%%endcit

%%%cit
%%%cit_head
%%%cit_pic
%%%cit_text
А Слуги народа, наконец, начали сознавать, что управлять не умеют.  СОЦИАЛЬНЫЙ
ВЗРЫВ БУДЕТ, ЕГО УЖЕ НЕ ИЗБЕЖАТЬ.  Безработица растет, её уровень по итогу
первого квартала 2021 года достиг 10,5\%, что является самым высоким показателем
с начала 2017 года, а реальные доходы подавляющего большинства граждан
неудержимо падают. \emph{Людям} разве это нужно? И новый раунд протестов ФОПов будет
пострашнее, т. к. обстановка в обществе слишком напряженная и даже тревожная:
чего еще отчебучат «Слуги народа», которые начали сознавать, что управлять они
не умеют
%%%cit_comment
%%%cit_title
\citTitle{В Украине нарастают предпосылки для социального взрыва}, 
Александр Гончаров, strana.ua, 09.07.2021
%%%endcit

%%%cit
%%%cit_head
%%%cit_pic
%%%cit_text
В 2007 году в Варшаве Институт Национальной Памяти Польши (государственная
организация) на польском языке и за бюджетные деньги издал уникальный сборник
«Кресовая книга праведников 1939-1945. Об украинцах, которые спасали поляков,
уничтожаемых ОУН и УПА» (под ред. Ромуальда Недзелько). Польские ученые за
десять лет исследований установили имена более 896 украинцев, которые защищали
поляков от уничтожения – \enquote{украинских праведников мира}. При этом
изучена архивная информация и свидетельства, установлены личности 384 украинцев
– погибших «украинских праведников мира». Известны имена и поляков защищавших
украинцев от карательных акций польских отрядов ...  Вот этим \emph{людям}
должны ставится памятники на Волыни, Галичине или Восточной Польше - эти
\emph{люди} в пекле Второй мировой войны остались \emph{Людьми} - смогли
преступить вековые противоречия, обиды, зависть, политические стереотипы во имя
\emph{Человечности}. Этими украинцами и поляками могут искренне гордится оба
народа и сейчас.  Именно \enquote{украинские и польские праведники мира} должны
стать основой украинско-польского исторического примирения.  Обо все остальном
пусть рассуждают историки в тихих пыльных кабинетах
%%%cit_comment
%%%cit_title
\citTitle{Национальный исторический миф нужно строить на героях и праведниках}, Руслан Бортник, strana.ua, 11.07.2021
%%%endcit

%%%cit
%%%cit_head
%%%cit_pic
%%%cit_text
Тому, на думку дослідниці, на такі послання все ж варто відповідати: «Багатьом
\emph{людям} досі потрібно пояснювати, чому не «один народ», чому різні мови, скільки
років Україні. Для різних аудиторій потрібно пояснювати по-різному. Комусь, те,
як виглядає спецоперація і деконструювати її, а комусь потрібно пояснювати ази
– про історію і мову»
%%%cit_comment
%%%cit_title
\citTitle{Спецоперація під назвою «стаття Путіна»}, 
Марія Щур; Сашко Шевченко, www.radiosvoboda.org, 13.07.2021
%%%endcit

%%%cit
%%%cit_head
%%%cit_pic
%%%cit_text
\emph{Люди} просто выживали. И единственное, что Вторая Украинская республика
могла сформировать, это прослойку \emph{людей} – со временем, спустя
десятилетия, вот мы с тобой к этой прослойке относимся – которая начала как-то
ориентироваться. У них сформировалась своя система координат за эти
десятилетия.  У тех, кто поездили по миру, заработали какой-то капитал,
получили какое-то образование и приспособились к тем правилам игры, которые
есть в современном мире, понимают их, и поэтому досягают какого-то успеха,
благополучия, самодостаточности и т. д. Эти \emph{люди} в абсолютном
меньшинстве. А те, кто не вписались, либо не сбежали отсюда, они в абсолютном
большинстве
%%%cit_comment
%%%cit_title
\citTitle{Украина в плену консервативного мышления: почему 30 лет шли не туда и что делать}, 
Сергей Иванов; Юрий Романенко, hvylya.net, 15.07.2021
%%%endcit

%%%cit
%%%cit_head
%%%cit_pic
%%%cit_text
Условно говоря, \emph{человек}, который думает, что его не могут достать, он
может быть агрессию внутреннюю, которая у него есть, изливать таким способом. В
реальности они чаще всего абсолютно иные. Те, кто допустим на меня в виртуале
гонят, когда я их встречаю в реале – милые, добрые, покладистые \emph{люди}.
Абсолютно.  Которые готовы подойти за ручку пожать и в глаза позаглядывать.
Понимаешь, это другая составляющая. Поэтому да, с одной стороны, нельзя
переоценивать вот эту вот виртуальную агрессию. Я, например, к ней отношусь
абсолютно спокойно.  \emph{Люди}, которые никогда не оказывались в эпицентре
таких волн ненависти, \emph{обычный человек}, допустим, какая-нибудь учительница, или
какой-нибудь металлург, который волею судьбы вдруг выносится на первый план в
алгоритмах Facebook, и начинают ему тысячи \emph{людей} писать, ах ты сука, мы тебя
ненавидим и т.д. Ему, конечно, это очень тяжело принять, это психологически
очень тяжело. Но нужно понимать, что чаще всего это ничем не заканчивается
%%%cit_comment
%%%cit_title
\citTitle{Украина в плену консервативного мышления: почему 30 лет шли не туда и что делать}, 
Сергей Иванов; Юрий Романенко, hvylya.net, 15.07.2021
%%%endcit

%%%cit
%%%cit_head
%%%cit_pic
%%%cit_text
Крім того, я зберіг дорогий для мене досвід із часів роботи в театрі імені Леся
Курбаса. Йдеться про можливість працювати з текстами у тривалій перспективі.
Завжди у новому часі відкриваються нові змісти, особливо якщо тексти хороші.
«Убогий жайворонок» – одна з останніх діалогічних притч Сковороди і певною
мірою його підсумкова робота. Цей текст особливий для мене. Велика цінність є в
якості, у простоті пояснення важливих складних речей. А ще якась любов до
\emph{людини}, адже він ставить на чолі усього щастя. Хочете бути щасливими? Дивіться:
треба так робити і так
%%%cit_comment
%%%cit_title
\citTitle{«Нам би хотілось закрутити всю Україну на цій осі»}, 
Катерина Сліпченко, zaxid.net, 29.07.2021
%%%endcit

%%%cit
%%%cit_head
%%%cit_pic
%%%cit_text
Відчутно, що змінилися тільки риторика, зовнішні форми, але залишилося
варварство за своєю сутністю, нагадуючи всім нам про деспотію царів, псарів і
більшовиків. Бо \enquote{русскіє} лише мавпують цінності європейської культури, але дух
\emph{загальнолюдських} чеснот був для них завжди чужим і неприйнятним.  Здається, що
найважнішим чинником для пропаганди всесвітнього блуду нинішньою РФ є брехня,
що сповна окутала не тільки офіційну державну політику, але й усі важливі
життєві сфери. Заперечення брехні вважається там державною зрадою. Адже
російська агресія, окупація Криму і частини українського Донбасу, катування й
ув’язнення українських громадян, нечувані для XXI століття порушення прав
\emph{людини} і міжнародних зобов’язань презентується путінським режимом як захист
\emph{людини}, співвітчизників, державних інтересів Росії
%%%cit_comment
%%%cit_title
\citTitle{Правда в Рідному Слові}, Георгій Філіпчук, slovoprosvity.org, 12.07.2021
%%%endcit

%%%cit
%%%cit_head
%%%cit_pic
%%%cit_text
Тож пусткою стоять не тільки хати і старі совєтські заводи, а вже й будівлі
профтехучилищ, гуртожитки, школи, корпуси лікарень, різноманітних управлінь і
установ, порожніють універмаги в райцентрах і громіздкі потвори мерій та
райдержадміністрацій, залізничні вокзали й церкви. Та й узагалі – \emph{людей} на
вулицях стає менше. І від цього стає страшно, бо панує відчуття, ніби щовечора
в якомусь вікні неподалік не загориться світло, а щоранку у новому будинку на
околиці не скрипне хвіртка. \emph{Люди} помирають, їдуть, зникають – і пустка, достоту
як сипучий пісок, поволі поширюється, підкрадаючись усе ближче до тебе. Аж поки
не настане час, коли тут не залишиться нікого
%%%cit_comment
%%%cit_title
\citTitle{Українці помирають, їдуть, зникають. Настане час, коли тут нікого не залишиться}, 
Андрій Любка, gazeta.ua, 31.07.2021
%%%endcit

%%%cit
%%%cit_head
%%%cit_pic
%%%cit_text
И повторюсь для \emph{русских людей} – показывайте себя, рассказывайте о себе,
гордитесь, любите, радуйтесь, ибо нет на Земле более талантливого, более умного
и прекрасного человека, который называется – \emph{Русский Человек}.  \emph{Русский человек}
– это просыпающийся Творец и потому он всегда сам себе на уме, и он победитель,
это НаРод не сотворенный, а рожденный, потому и Бог наш – Род наш.  Всех благ
вам, \emph{добрые люди}, уверяю вас, уйдет вся мерзость, а мы останемся, не забывайте,
мы Творцы всего, что было, есть и будет, и Творение – это наша Жизнь, жизнь
Творца в вечности и бесконечности
%%%cit_comment
%%%cit_title
\citTitle{Русских нет нигде, потому что они везде}, Вестник, zen.yandex.ru, 03.08.2021
%%%endcit

%%%cit
%%%cit_head
%%%cit_pic
%%%cit_text
Можно утверждать, что в процессе своего развития \emph{человек}, общество и
государство проходят три основных этапа. Сначала, после появления на свет, они
погружаются в созданные другими миры: в их законы и принципы, мудрость веков,
давящую на сознание своей обязательностью. Жизнь протекает в повиновении и под
тяжестью слова «должен». Поскольку без опыта не существует развития. Тот, кто
остановится на этой стадии, навсегда останется рабом, так никогда и не
познàющим радости свободы. Такая жизнь без смеха и веселья сотворена страданием
и бессилием. Однако, в качестве основного вывода из этого этапа развития,
возникает осознание факта, что нельзя всю жизнь трястись от страха и жить под
тяжестью чужих опыта или воли (выбора) – необходимо восставать и рисковать
%%%cit_comment
%%%cit_title
\citTitle{Замысел украинского государства: социальность, самодостаточность, независимость. Пятая часть}, 
Акулов-Муратов В. В., analytics.hvylya.net, 18.10.2021
%%%endcit

%%%cit
%%%cit_head
%%%cit_pic
%%%cit_text
И вот я задаюсь вопросом, кто виноват? Зеленский заставлял забивать болт на
других \emph{людей}? Или когда ковидные сертификаты врачи продают? Это ж
убийство в чистом виде и за это надо сажать. И самое главное, что огромное
количество \emph{людей} в Украине это пассивные убийцы. Потому что им все
равно, что из-за их действий кто-то даст дуба. Между телкой из Запорожья и
олигархом, который лоббирует энергетическую неэффективность нет никакой
разницы. В основе такого поведения примат личной выгоды за счёт жизней и
ресурсов других.  При этом сами по отдельности все \emph{милые люди}, но по
факту имеем стадо взбесившихся бабуинов. И отношение развитого мира к нам
именно как к бешеным бабуинам
%%%cit_comment
%%%cit_title
\citTitle{Огромное количество людей в Украине - это пассивные убийцы / Лента соцсетей / Страна}, 
Юрий Романенко, strana.news, 20.10.2021
%%%endcit

%%%cit
%%%cit_head
%%%cit_pic
\ifcmt
  pic https://img.strana.news/img/article/3584/kak-ezdil-transport-5_main.jpeg
  @width 0.4
\fi
%%%cit_text
"\emph{Людей} не пускали в автобусы без нужных документов плюс штрафовали за
отсутствие масок- по 170 гривен. В Черновицкой области сняли с автобуса
пассажира- иностранца, который въехал из Молдовы и направлялся в Тернополь. Что
ему делать дальше, и как попасть к месту назначения - непонятно. Бабушку,
которая была на госпитализации в областной больнице, а сегодня как раз
выписалась и пыталась попасть домой, не пустили в автобус. Старушка
расплакалась - денег на тест у нее не было. Пассажиры звонили даже не
правительственную линию, но так и не выяснили - как быть в таких случаях. Зато
ответ знают проверяющие во Львовской области. Они в ходе рейдов прямо говорили
пассажирам - не хотите, чтобы вас проверяли, ездите на бла-бла-каре", —
рассказывает Житинский
%%%cit_comment
%%%cit_title
\citTitle{Как ездил транспорт в Украине по новым правилам 21 октября}, Людмила Ксенз, strana.news, 21.10.202
%%%endcit

%%%cit
%%%cit_head
%%%cit_pic
%%%cit_text
Современная академическая и экспериментальная психология по большей части есть
наука, предмет которой – отчуждённый \emph{человек}, изучаемый отчуждёнными
исследователями при помощи отчуждённых и отчуждающих людей методов. Маркс
осознавал существование отчуждения и не считал \emph{отчуждённого человека}
естественным человеком, человеком как таковым. Идею об отчуждении Маркс
позаимствовал у мыслителей Просвещения, развил, углубил и сделал одной из
базовых в своём мировоззрении, о чём далее.  Через всё творчество Маркса
красной нитью проходит концепция «природы \emph{человека}». В
«Экономико-философских рукописях» он говорит о «сущности \emph{человека}». В
«Немецкой идеологии» он говорит, что эта «сущность» не является абстракцией, а
объективно существует. В «Капитале» Маркс, имея в виду всё ту же «сущность»,
говорит о «\emph{природе человека} вообще», в отличие от \emph{человеческой
природы}, видоизменяемой в каждый исторический период
%%%cit_comment
%%%cit_title
\citTitle{Карл Маркс – психоаналитик и религиозный экзистенциалист / Статьи}, 
Александр Карпец, fraza.com, 08.05.2018
%%%endcit

%%%cit
%%%cit_head
%%%cit_pic
%%%cit_text
Официальному Киеву явно не нравится, когда его обвиняют в военных
преступлениях. Чтобы подобных обвинений не было, у Украины есть два пути.
Первый, по которому предпочитает идти Киев, это язык запугивания и угроз. Это и
обстрелы мирных жителей, и обещание жёсткой зачистки республик ЛДНР после их
захвата, и сайт "миротворец". Русские должны испугаться и не открывать рта, а в
идеале убежать в Россию.  Второй намного проще и логичнее, но для его
вопрошания надо быть просто \emph{людьми}. Необходимо просто прекратить войну,
договориться по хорошему с Донбассом, тогда и повода для подобных обращений к
мировым лидерам не будет
%%%cit_comment
%%%cit_title
\citTitle{Или нормальное детство, или сайт "миротворец".}, Мак Сим, zen.yandex.ru, 23.10.2021
%%%endcit

%%%cit
%%%cit_head
%%%cit_pic
%%%cit_text
Лоуренс Беккет со своими \emph{людьми} засиделся за выпивкой. Ужинали рано, но они все
еще сидели за столами, на которых валялись кости и куски хлеба. Горожане,
завсегдатаи таверны, уже разошлись, и хозяин, отослав слуг, сам остался у
прилавка. Он хотел спать, и часто зевал, но не торопил гостей: не так уж часто
в «Кабаньей голове» появлялись посетители с таким количеством денег. Студенты
заглядывали редко и приносили больше беспокойства, чем прибыли, а горожане
прекрасно умели растягивать один стакан на целый вечер. «Кабанья голова» стояла
не на главной дороге, а на боковой улице, и купцы не часто находили сюда дорогу
%%%cit_comment
%%%cit_title
\citTitle{Зачарованное паломничество}, Клиффорд Саймак
%%%endcit

%%%cit
%%%cit_head
%%%cit_pic
%%%cit_text
Чого я прийшов до цих тимчасових \emph{людей}, чого сиджу серед них? Ми \emph{люди} державні,
самі собі не належимо, а ваш брат був \emph{державною людиною} найвищого рангу... Ранги...
Життя не має рангів, А \emph{людина} — вище держави. Скільки держав було на цій землі,
а брат — один. Він для мене тисячолітній, як і я. і ніхто не вбивав його, то
він сам себе вбив. Самоусунувся. Самоліквідувався. Або ж просто змінив сферу
перебування. Тут йому набридло. Або ми всі йому набридли, або ж він набрид
усім. Він міг умерти інакше: в постелі удосвіта від інфаркту, або ж у клініці,
коли хтось ненавмисне відключив штучне серце. Але Марко вибрав смерть гучну, з
вибухом, вогнем, нищенням, мало не з аннігіляцією. Удар КАМАЗ’а — як це
банально. А може, це болід, кульова блискавка, неймовірний заряд статичної
електрики, лазер з космічної платформи? Звичайні способи розслідування тут
нічого не дадуть. Вони надто примітивні. Я здійсню власне розслідування. Воно
не записане ні в яких кодексах і статутах, не обмежене куцими параграфами і
невблаганними статтями. Тисячолітнє розслідування
%%%cit_comment
%%%cit_title
\citTitle{Тисячолітній Миколай}, Павло Загребельний 
%%%endcit

%%%cit
%%%cit_head
%%%cit_pic
%%%cit_text
И ширваншах посла тотчас послал к шурину своему, князю кайтаков Халил-беку:
«Судно мое разбилось под Тарками, и твои \emph{люди}, придя, людей с него захватили, а
товар их разграбили; и ты, меня ради, \emph{людей} ко мне пришли и товар их собери,
потому что те \emph{люди} посланы ко мне. А что тебе от меня нужно будет, и ты ко мне
присылай, и я тебе, брату своему, ни в чем перечить не стану. А те \emph{люди} ко мне
шли, и ты, меня ради, отпусти их ко мне без препятствий». И Халил-бек всех
\emph{людей} отпустил в Дербент тотчас без препятствий, а из Дербента отослали их к
ширваншаху в ставку его - койтул. Поехали мы к ширваншаху в ставку его и били
ему челом, чтоб нас пожаловал, чем дойти до Руси. И не дал он нам ничего:
дескать, много нас. И разошлись мы, заплакав, кто куда: у кого-что осталось на
Руси, тот пошел на Русь, а кто был должен, тот пошел куда глаза глядят. А иные
остались в Шемахе, иные же пошли в Баку работать. А я пошел в Дербент, а из
Дербента в Баку, где огонь горит неугасимый; а из Баку пошел за море - в
Чапакур
%%%cit_comment
%%%cit_title
\citTitle{Хождение за три моря}, Афанасий Никитин (1433-1472)
%%%endcit

%%%cit
%%%cit_head
%%%cit_pic
%%%cit_text
Из не забавного( А нынешние \emph{люди-человеки} очень легко превращаются в зомби.
Социальные инженеры, с помощью машины СМИ, научились добротно раскалывать
группы населения для дальнейшей диффузии... Практически на какой угодно теме:
Путин. Язык. Вакцинация. "Харьковский мажор"...  Очень легко информационной
накачкой формируются костяки особо твердолобых и твердорогих в своих
"морально-ценностных" превосходствах. Правда, при этом, совершенно глухих и
слепых
%%%cit_comment
%%%cit_title
\citTitle{Нынешние люди очень легко превращаются в зомби / Лента соцсетей / Страна}, 
Александр Рябоконь, strana.news, 29.10.2021
%%%endcit

%%%cit
%%%cit_head
%%%cit_pic
%%%cit_text
Почти 30 лет лгали о доступной медицине, хотя все было за деньги. И у кого их
не было, просто умирали. Потому и смертность выше, чем в развитых странах.
Точно также врали о Будапештском меморандуме. Врали о реформах. Врали о
перемогах и прочая, прочая, прочая. Все украинское общество насквозь пронизано
ложью, которую начинают культивировать с пеленок.  Потому ковид просто лакмус,
который подсвечивает все стороны этого лживого, аморального общества-урода в
котором можно спокойно жертвовать другим \emph{человеком}, вообще не задумываясь о
последствиях. Харьковский мажор, который гонял по городу на скорости 200 км,
стрелял в людей и в финале убил \emph{человека} - венец этого общества. Он абсолютно
ничем не отличается от \emph{человека}, покупающего ковид-сертификат по своим мотивам.
Мотив везде один и базовый - мне удобно, остальные лохи
%%%cit_comment
%%%cit_title
\citTitle{Ковид-паспорта становятся лакмусом украинского позора / Лента соцсетей / Страна}, 
Юрий Романенко, strana.news, 29.10.2021
%%%endcit

%%%cit
%%%cit_head
%%%cit_pic
%%%cit_text
Третий путник был совсем еще молодым \emph{человеком}, почти мальчиком. Он ездил
сдавать документы в консерваторию и был плохо способен к поддержанию разговора.
Из всего существующего на свете занимала его только музыка. В Киеве он был
впервые и из всех впечатлений дня главными были испуг и усталость. \emph{Молодой
человек} был напуган многолюдством, суетой и расстояниями. Столица показалась
ему муравейником, в котором все спешат и все друг другу безразличны. Уже к
концу дня он смертельно устал от метро, от шума, от контраста между лицами,
улыбающимися с рекламных плакатов, и угрюмо сосредоточенными лицами на улицах.
Юноша привык слушать больше музыку, чем слова, и к концу этого дня звуки Киева
измучили его слух.  Как тот набоковский шахматист, которому мир представлялся
разбитым на клетки, а сама жизнь – похожей на хитрую партию с неизвестным
соперником, этот \emph{молодой человек} представлял мир зашифрованным нотными знаками.
Он еще не успел испытать ни любви, ни ненависти, он еще даже не начал бриться,
и не интересовало его покамест ничего, кроме специальных предметов,
преподаваемых в только что оконченном музучилище. Ему было отчасти неловко,
отчасти скучно. Но просто молчать и смотреть в окно он позволить себе не мог
%%%cit_comment
%%%cit_title
\citTitle{Возвращение в Рай}, Андрей Ткачев
%%%endcit
