% vim: keymap=russian-jcukenwin
%%beginhead 
 
%%file slova.fashizm
%%parent slova
 
%%url 
 
%%author 
%%author_id 
%%author_url 
 
%%tags 
%%title 
 
%%endhead 
\chapter{Фашизм}

%%%cit
%%%cit_head
%%%cit_pic
%%%cit_text
И там, и здесь мы видим торжество теории и практики простых решений. «Рада
міністрів була народжена Конституцією, але \emph{Вища фашистська рада} була
вищою за значенням, тому що була народжена Революцією». Беніто Муссоліні.
Тепер я знаю, звідки з'явилася ідея приймати основні політичні рішення через
Раду національної безпеки і оборони.  А ще згадалося одне з визначень
\emph{фашизму}: теорія і практика простих рішень. В розумінні: досягти
суспільного процвітання дуже просто - потрібно лише винищити євреїв або
комуністів (привіт Потураєву), згуртуватися довкола
дуче-фюрера-каудильо-поглавніка-Вождя, змусити всі типи власності працювати за
єдиним планом, перестати сумніватися в правильності політичного курсу,
витворити нову якість Людини... Структура, яку очолив в Україні Саакашвілі,
називається Офісом простих рішень. Симптоматично...
%%%cit_comment
%%%cit_title
\citTitle{Украинская власть технически повторяет практику фашизма}, 
Константин Бондаренко, strana.ua, 19.06.2021
%%%endcit

