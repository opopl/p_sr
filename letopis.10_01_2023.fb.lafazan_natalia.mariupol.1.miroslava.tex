%%beginhead 
 
%%file 10_01_2023.fb.lafazan_natalia.mariupol.1.miroslava
%%parent 10_01_2023
 
%%url https://www.facebook.com/permalink.php?story_fbid=pfbid0HFwfKaWsQnMUxUECAwgVSXkD42TnU82Fv8snLGNKtKSX8YQ3ie5G1EAynoLUWuCkl&id=100030592628843
 
%%author_id lafazan_natalia.mariupol
%%date 10_01_2023
 
%%tags mariupol,mariupol.war,dnevnik
%%title Мы с Ирой шли по улице практически ещё незнакомого нам города и встретили Мирославу
 
%%endhead 

\subsection{Мы с Ирой шли по улице практически ещё незнакомого нам города и встретили Мирославу}
\label{sec:10_01_2023.fb.lafazan_natalia.mariupol.1.miroslava}

\Purl{https://www.facebook.com/permalink.php?story_fbid=pfbid0HFwfKaWsQnMUxUECAwgVSXkD42TnU82Fv8snLGNKtKSX8YQ3ie5G1EAynoLUWuCkl&id=100030592628843}
\ifcmt
 author_begin
   author_id lafazan_natalia.mariupol
 author_end
\fi

Мы с Ирой шли по улице практически ещё незнакомого нам города и встретили
Мирославу. С ней мы познакомились несколькими днями ранее. Мира (так я называю
ее про себя) приветливо поздоровалась. Мы перекинулись парой фраз и вдруг она
спросила: "А хотите я вам сделаю подарок?". Пока Ира обдумывала вопрос я быстро
ответила: "Да!". Меня одолевало любопытство, что может нам подарить Мира, такой
же беженец, только из Бородянки. Она протянула талончик и сказала: "Идите в
пекарню через три дома, покажите эту бумагу и возьмите все, что захотите". Я
толком ничего не поняла, но пошла получать новый жизненный опыт. Зайдя в зал, я
почувствовала дурманящий аромат свежей выпечки. Запахи были разные, но все они
были потрясающими. Мы подошли к стойке и протянули талончик. Продавец что-то
нас спросила. Мы лишь догадались, что она интересуется, что нам дать. Я тыкнула
в красивую буханку хлеба. Она достала ее с витрины, заботливо упаковала и
протянула нам. Талончик нам она не вернула и по глазам я поняла, что диалог ещё
не завершен. Она подошла к соседней витрине и провела рукой по стеклу с ее
стороны прилавка. Потом начала показывать рукой на разные виды пирожных. Я
поняла, что нам предлагают выбрать. Тыкнула на первое попавшееся. Все действия
повторились. Она снова стала ждать моего указательного пальца по мою сторону
прилавка в сторону пирожных. Так мы выбрали три. Наверное, она бы продолжала
складывать в пакеты выпечку дальше, но меня уже к этому времени наполняло такое
чувство благодарности к этому незнакомому мне человеку, что я начала скрещивать
руки, ладонями к ней, показывая, что этого достаточно. Нам отдали талончик и мы
вышли на улицу. В глазах Иры стояли слезы. Я спросила ее: "Ну чего ты,
дереха?". Слезы она уже ели сдерживала, а самые проворные покатились по ее
щекам. Она взяла себя в руки и ответила: "Я никогда не просила ничего в своей
жизни. Тем более хлеб". Я промолчала. Мы в Мангуше, под Мариуполем, просили у
людей хлеб, так как последняя пачка печенья была съедена детьми в дороге.
Выезжали все даже не выпив чаю. Люди несли сало, суп, кисляк... Кто что мог. И
кормили многокилометровую мариупольскую пробку из разбитых машин и обездоленных
людей. С Иришей мы успокоились по дороге в лагерь. К слову, в лагере мы не
голодали - еды было достаточно и коллектив делал все для своих временных
постояльцев. В нашей комнате нас было всего 10. Считаю, что это шикарные
условия. Пять из нас взрослых, пять детей. Мы достали подаренные нам вкусняшки
и сказали детям разделить на пятерых. Они спросили, где мы это взяли. Опустив
подробности, мы ответили - помощь волонтеров. Это лишь одна история двух
мариупольчанок и девочки из Бородянки. Таких историй тысячи. Все благодаря
одному старику, выжевшему из ума. Сейчас мы все держимся вместе. В помощи
друг-другу не отказываем ни в каких обстоятельствах. Но мне кажется, что доброе
зерно в те наши закрытые и отчаявшиеся сердца, посеяла Мира. Я так ее и буду
продолжать называть, так как этого слова прям не хватает. Только про себя. Она
представилась Мирослава из Бородянки.

%\ii{10_01_2023.fb.lafazan_natalia.mariupol.1.miroslava.cmt}
