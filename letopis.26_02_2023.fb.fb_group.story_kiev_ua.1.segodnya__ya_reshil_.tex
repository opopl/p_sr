%%beginhead 
 
%%file 26_02_2023.fb.fb_group.story_kiev_ua.1.segodnya__ya_reshil_
%%parent 26_02_2023
 
%%url https://www.facebook.com/groups/story.kiev.ua/posts/2147766272086829
 
%%author_id fb_group.story_kiev_ua,novickij_vladimir
%%date 26_02_2023
 
%%tags kiev,semja,istoria,kniga
%%title Сегодня, я решил собрать все написанные мною рассказы  в единую книгу
 
%%endhead 

\subsection{Сегодня, я решил собрать все написанные мною рассказы  в единую книгу}
\label{sec:26_02_2023.fb.fb_group.story_kiev_ua.1.segodnya__ya_reshil_}
 
\Purl{https://www.facebook.com/groups/story.kiev.ua/posts/2147766272086829}
\ifcmt
 author_begin
   author_id fb_group.story_kiev_ua,novickij_vladimir
 author_end
\fi

Сегодня, я решил собрать все написанные мною рассказы  в единую книгу, в
которой отразить историю нашей семьи.

Делаю я это для того, чтобы последующие поколения нашей семьи смогли узнать
откуда они родом и как жили их предки на  исторической родине  Украине  в
городе Киеве на протяжение с середины 19 по первую половину 21 века.

В отличии от Вас, читающие эти строки, я  имел счастье с ними лично общаться и
таким образом узнать от них подробности жизни в Украине  и  Киеве в то
непростое и тяжёлое  время, которое, к глубокому моему сожалению, продолжается
и сейчас.

Думаю,  что  Вам всем будет  интересно об этом  узнать. Ваше счастье дорогие
мои потомки, что  родились  вы в мирное время и даже сейчас, когда  в Украине
продолжается война, так уж вышло, вы  родились и живёте на другом континенте и
не ощущаете того горя, которое выпало на долю  ваших сверстников в Украине.

Разница между мной и вами состоит ещё и в том, что Вы, хотя бы из моих
воспоминаний, сможете  узнать о моей судьбе и  судьбе ваших  далёких предков. Я
же, уже не смогу узнать о жизни многих из Вас — моих потомков, по одной простой
причине, что меня просто уже не будет к этому времени в живых, а жаль.  Как бы
мне хотелось увидеть всех Вас, погулять у каждого из Вас на свадьбе.
Посмотреть  своих пра-пра правнуков, но в одном я спокоен и уверен, что  все
они  будут прекрасными,  порядочными  и уважаемыми людьми, так как  все
унаследуют наши гены, а среди наших, то есть  Ваших предков людей с плохими
наклонностями  и  плохой наследственностью я не припоминаю.

Начать повествования о наших предках хочу с Ильи Федоровича Новицкого —
полковника  Запорожского войска. Знаю о нём только то, что гетьман Мазепа,
кстати первый из Украинских гетьманов сделавший  попытку  освободиться от
российского влияния и ценивший  полковника Новицкого как опытного
военоначальника, назначил его в 1696 году  главным начальником всех городов и
сёл на границе с татарскими владениями, чтобы защищать Украину от их частых
набегов. За мужество и воинские подвиги он был также пожалован разными
почестями с занесением рода Новицких в шестую часть родословной книги с
поколенной  росписью рода Новицких «Честь и совесть превыше всего» - девиз
рода.                       

Это  как бы возможная родня с давних времён, но есть ещё одна версия о которой
я услышал от своей бабушки на поминальные дни на Байковом кладбище, где с 1912
года лежит её муж, мой дед. Рассказывать тогда об этом было не безопасно, но
тут на его могиле она решила рассказать мне, ещё подростку, о нём поподробней,
чтобы я мог им гордиться. Она сказала, что дед, учился в кадетском — пажеском
корпусе и был штабс- капитаном, закончил Киевский Университет и что его отец,
мой дед - был генералом. Что он рос в военной семье и в их семье было ещё много
генералов. Вспомнив про этот давний  бабушкин рассказ, я уже сейчас, когда это
стало возможным, вышел на Википедию и попытался найти генералов царской армии с
фамилией Новицкие. Их, как и говорила бабушка, оказалось много. Одно фото меня
очень заинтересовало, так как оно было сделано в Киевском  фотоателье.
Оказалось — это наш земляк Киевлянин и звали его Новицкий Николай Дементьевич.
Но, что самое главное, мне показалось, что он, верней я очень похож на него.
Когда я позвал свою жену Лилю посмотреть на  это фото и спросил её «на кого
похож этот человек» - она сразу сказала, что на меня, когда я был ещё  молодым
и полным.

В генетику я верю, так  в нашей семье мой внук очень похож на моего деда, а
внучка на мою бабушку. Учитывая всё это,  можно предположить — что это наш
родственник.

