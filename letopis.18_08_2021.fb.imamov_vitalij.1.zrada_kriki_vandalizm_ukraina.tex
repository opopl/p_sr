% vim: keymap=russian-jcukenwin
%%beginhead 
 
%%file 18_08_2021.fb.imamov_vitalij.1.zrada_kriki_vandalizm_ukraina
%%parent 18_08_2021
 
%%url https://www.facebook.com/permalink.php?story_fbid=2838169636458013&id=100007950128049
 
%%author_id imamov_vitalij
%%date 
 
%%tags instaljacia,kiev,patriotizm,ukraina,vandalizm,zrada
%%title Менше криків про зраду, більше справ, і більше слів по ділу. Ось, що нам потрібно
 
%%endhead 
 
\subsection{Менше криків про зраду, більше справ, і більше слів по ділу. Ось, що нам потрібно}
\label{sec:18_08_2021.fb.imamov_vitalij.1.zrada_kriki_vandalizm_ukraina}
 
\Purl{https://www.facebook.com/permalink.php?story_fbid=2838169636458013&id=100007950128049}
\ifcmt
 author_begin
   author_id imamov_vitalij
 author_end
\fi

Україна - це країна історій, наративів.

Тут одночасно співживуть та протистоять один одному безліч поглядів на Україну
та її майбутнє (наративів).

Коли щось відбувається, якась яскрава чи значима подія, то представники кожного
такого наративу намагаються дати оцінку такій події в ракурсі їх бачення
України та поточної ситуації. 

\ifcmt
  ig https://scontent-frt3-1.xx.fbcdn.net/v/t39.30808-6/239608453_2838170573124586_5678452532877965109_n.jpg?_nc_cat=102&_nc_rgb565=1&ccb=1-5&_nc_sid=730e14&_nc_ohc=upETpjzsGawAX_wqcir&_nc_ht=scontent-frt3-1.xx&oh=eb03577dfb1f71252355b4f32bd31fe2&oe=613D9383
  @width 0.4
  @wrap \parpic[r]
\fi

Тобто намагаються вписати цю подію в ту історію, яку вони розказують, в свій дискурс.

І тут важливо бути "першим". Це важливо, бо йде боротьба за домінування в
інформаційному просторі (хто кого коментує, хто кому заперечує) та боротьба за
людей, за їх симпатії до відповідного бачення майбутнього України.

Інсталяція з радянським та пост-радянським побутом українців - це була лише
інсталяція, станом на той момент, коли виникла "буря в стакані" (емоційна
реакція фейсбуку).

Цей кейс чудово виявляє, що в людей в голові.

Люди не знали задуму щодо інсталяції, але "безапеляційно", з праведним гнівом,
висловлювали свою інтерпретацію, в ракурсі "їх" наративу.

І це, якби природно, і це поки варіант норми для нас нинішніх. 

Але цей варіант норми - це карнавал зради, який підживлюють з усіх боків
емоційними маніпуляціями.

Але кожен карнавал має закінчуватися і переходити в раціональну площину та
справу (діло).

Найбільше кричать про "зраду" і качають емоції ті, хто найбільше закликали
"думайте". І тут таки варто замислитися.

Менше криків про зраду, більше справ,  і більше слів по ділу. Ось, що нам
потрібно.

Благо, багатьох карнавал зради вже притомив, і вони потроху виходять з нього.

\subsubsection{Коментарі}
\begin{itemize}
%%%fbauth
%%%fbauth_name
\iusr{Роберт Леслі}
%%%fbauth_url
%%%fbauth_place
%%%fbauth_id
%%%fbauth_front
%%%fbauth_desc
%%%fbauth_www
%%%fbauth_pic
%%%fbauth_pic portrait
%%%fbauth_pic background
%%%fbauth_pic other
%%%fbauth_tags
%%%fbauth_pubs
%%%endfbauth
 
ВИ зробили це разом, ЗЕбіли йобані. ПНХ, баран
\end{itemize}

