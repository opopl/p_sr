% vim: keymap=russian-jcukenwin
%%beginhead 
 
%%file 05_03_2022.fb.petrova_aleksandra.1.dnevnik_ogloblin
%%parent 05_03_2022
 
%%url https://www.facebook.com/alexandra.petrova.900/posts/10229857027355725
 
%%author_id petrova_aleksandra,ogloblin_vladimir.harkov
%%date 
 
%%tags 
%%title Дневник Владимира Оглоблина
 
%%endhead 
 
\subsection{Дневник Владимира Оглоблина}
\label{sec:05_03_2022.fb.petrova_aleksandra.1.dnevnik_ogloblin}
 
\Purl{https://www.facebook.com/alexandra.petrova.900/posts/10229857027355725}
\ifcmt
 author_begin
   author_id petrova_aleksandra,ogloblin_vladimir.harkov
 author_end
\fi

В эти дни родилось (и еще появятся) очень много свидетельств об \enquote{окаянных
днях}, и продолжение \enquote{дневника} Владимира вместе с его фотографиями мне кажется
очень важным. 

Мы, находящиеся вдали от событий, переживаем их отраженно, и хотя мы не
прячемся от обстрелов, не ходим по разрушенным городам, только мы сами знаем,
как тяжело дается нам это отражение. Мир изменился навсегда, кто-то находится в
центре страшных событий, кто-то по краям, на обочинах и как будто очень далеко,
но он изменился для всех. 

Записывать хотя бы для себя о происходящем, куда-нибудь на манжеты.

\ii{05_03_2022.fb.petrova_aleksandra.1.dnevnik_ogloblin.pic.1}

Сообщения от Владимира Оглоблина:

Среда 12:16

\enquote{Живы. Но обстрелы и бомбежки не прекращаются. Харьков разрушают дальнейшем
артиллерией и бомбовыми ударами с самолетов. Такого не было даже от фашистов.}

\enquote{Это центр Харькова вчера}

"Теперь уже когда весь мир победит. Без Италии и других Украине не совладать с
чудищем огромным.

Были же в годы 2-й Мировой антифашисты. Теперь очередь антирашизма. 

Когда ещё 2008 году от пьяного росс. генерала в Подмосковье услышал о войне с
Украиной, то решил, что это бред пьяного генерала. Но позднее все стало
подтверждаться. Антирашизм - это не против России и российского народа. Как и
антифашизм. Это против системы, против диктатуры и имперских амбиций."

14:08

\enquote{Жизнь иногда иногда устраивает сюрпризы. Это не мои друзья... У меня сотни людей
остаются в России, с которыми сохранились человеческие отношения.}

14:17

\enquote{Харьков превращается в Сталинград. Конечно, я помню, не вся Россия идёт на
Украину. Всего 200 000 войска и + в 3/4 раза больше их родных. До миллиона.
Это 1/150 часть всего населения. Я не беру во внимание работников военных
ведомств, предприятий, СМИ и др. Надеюсь на разум людей…}

\ii{05_03_2022.fb.petrova_aleksandra.1.dnevnik_ogloblin.pic.2}

19:54

\enquote{Саша, слов нет. Начала авиация бомбить город Харьков, Киев. Не знаю что писать.}

\enquote{В этом большом здании и мой офис, моя Фотошкола. Не знаю что сейчас там. Окон
нет, что с водой в радиаторах не понятно. Завтра, если сумею, проеду на машине
в центр и заберу свой главный компьютер и мой супермонитор для профессиональной
обработки фото. Там много моих вещей и часть коллекции уникальных старых
фотоаппаратов. Но целое все ?}

\enquote{Там моя жизнь 15 лет}

\enquote{Завтра будет видно. Уже не знаешь, где безопасно. Стреляют издалека хаотично по городу. Как угадать?}

\enquote{Пролетели самолеты бомбить.}

\enquote{Свет как в погребе. Чтобы видеть клавиши. Сначала Корона, а теперь война.  Я
попробую заснуть ... Сегодня совсем не думается. Слышны взрывы бомб или ракет.
Уже сообщили о бомбовых ударах по большому заводу им. Шевченко. Дикость,
варварство, безумие.}

Четверг 09:37

[Харьков]Держится. Я ездил в офис в свой Дворец Труда. Там апокалипсис. Видели
Дом Павлова в Волгограде, после обстрелов фашистами? Дворец хуже после обстрела
ракетами русскими. Это нам по-братски..,

Сделал сьемку. Если доберусь домой, пришлю. Стрельба вокруг.

Спасибо за помощь. Помогайте - это важно.

Четверг 16:38

\enquote{Это сегодняшние. Не было времени долго снимать. Все время вокруг взрывалось и
пришлось пробежать дом-квартал и быстро уезжать. Проведал друга, который
сейчас с мамой. Ей 94 года. Не может оставить. Пообнимались, поговорили. Он ещё
рано утром купил себе и нам некоторые продукты. Есть возможность по 2-3 часа
стоять до открытия магазина.}

\enquote{Сумская — это вообще самая первая улица в городе ещё с конца 17 века. Из
деревянной крепости (1654/56гг) дорога вела на Суммы. Потом стала улицей.
Наверное самая старая. Вторая была Московская дорога, которая стала улицей
Московской, а теперь Московский проспект. Люди подали петицию о переименовании
проспекта. После войны так и случится. Нельзя сохранять названия вражеского
государства. Час назад три крылатые ракеты разбили 4 высотки. Два дома просто
сложились и похоронили много людей. Туда мимо меня помчалось 6-8 скорых,
пожарные. Как-то так.}

\enquote{Теперь эта улица уничтожается. Как уничтожается город, убиваются люди.}

19:11

\enquote{Сейчас ад адский. Бомбит авиация и обстрелы города ГРАДАМИ. Горят дома, депо с
новыми автобусами, огромный рынок. Ракетами попали в 4 дома высотных. Очень
много погибших. Не знаю что и писать… Я сегодня во Дворце Труда. Все, что
осталось от Дворца…}

22:13

Готовимся к ночи. Налетают самолеты и где-то сильно бомбят. Если переживем
ночь, то завтра можем уехать на Запад Украины. Пока к другу в Черкасскую
область, а потом видно будет... Может быть Ивано-Франковск. А там видно будет...

22:32

Уже как получится и куда смогу сначала на машине. А там видно будет. Бензина
нет на заправках – это проблема. У меня на 300 км есть, но это даже до Киева не
хватит.

Посмотрим.

22:35

\enquote{Я попробую до Ивано-Франковска доехать. Если повезёт с бензином. А там приятели.}

\enquote{Наверное посылал уже некоторые фото Дальнего Востока. Мне там было комфортно.
Ещё не убита природа людьми. 9 раз был на Колыме и 6 раз в Якутии. Это мое по
духу... Но туда дороги нет.}

\enquote{Да, я умею мечтать. Не представляю как разрешится эта трагедия. Войны
тоже заканчиваются. Правда, были и столетние войны.}

\enquote{Никогда не думал, что с россиянами (не русскими) будет так сложно}.

Пятница 12:12

Три часа едем в пробке до Полтавы. Тысячи машин

12:58

95 км от города

18:51

7 часов меня вымотали. Ночью спал 3 часа. 120 км за 7 часов – это рекорд из
Харькова в Полтаву

Всё так меняется в жизни... И как происходящее далеко от фотографии.

\ii{05_03_2022.fb.petrova_aleksandra.1.dnevnik_ogloblin.cmt}
