% vim: keymap=russian-jcukenwin
%%beginhead 
 
%%file 18_11_2020.fb.ivan_prymachenko.1.kovid_victims_ua
%%parent 18_11_2020
 
%%url https://www.facebook.com/primachenkoonline/posts/3321812057873006
%%author 
%%tags 
%%title 
 
%%endhead 

\subsection{Від коронавірусу помер батько моєї дівчини Микола - Іван Примаченко}
\label{sec:18_11_2020.fb.ivan_prymachenko.1.kovid_victims_ua}
\Purl{https://www.facebook.com/primachenkoonline/posts/3321812057873006}
\Pauthor{Примаченко, Іван}

\index[rus]{Коронавирус!Жертвы}

\ifcmt
pic https://scontent.fiev6-1.fna.fbcdn.net/v/t1.0-9/126074904_3321807904540088_2158998521028385431_o.jpg?_nc_cat=109&ccb=2&_nc_sid=730e14&_nc_ohc=WUL-or4vSJgAX90Mkme&_nc_ht=scontent.fiev6-1.fna&oh=8a8e67172498015f416bb87def87801f&oe=5FDC70A2
fig_env wrapfigure
\fi

Сьогодні Україна перетнула позначку в 10 000 загиблих від Covid 19. Цього разу
для мене це не просто чергове оновлення трагічної статистики - від коронавірусу
помер батько моєї дівчини Микола. 

Спочатку в нього піднялася невисока температура, потім почалися проблеми з
диханням, а через три дні після госпіталізації в чернівецьку лікарню він
раптово загинув, хоча лікарі говорили про стабільний стан. 

Я знав його не так добре, як хотілося б. Бачився з ним останній раз місяць
тому. Він був інженером, дуже розумним, дуже практичним. Завжди гостинним і
щедрим, вмів розповідати смішні історії. Витягнув родину з бідності в 90-ті.
Любив подорожувати і об'їздив півсвіту - від Франції до Китаю. На своє 70-річчя
хотів зібрати всю родину в одній мандрівці - але не дожив. 

Коли кількість загиблих в США від Covid-19 перетнула позначку в 100 000 New
York Times вийшла з іменами кількох сотень жертв вірусу. І графічною
візуалізацією всіх загиблих. Довго-довго гортаєш сторінку і бачиш: «Маріон, 85
– прабабуся, що багато сміялася», «Ромі, 91 --- врятував 56 єврейських родин від
Гестапо», «Джордж, 72 --- міг виростити все що завгодно». А між цими віконцями в
світи загиблих тисячі і тисячі тих, про кого звісток немає. 

Коли сьогодні я зайшов на головні сторінки п’яти українських видань, які
стабільно читаю, то не знайшов там нічого, крім нової порції сухої статистики
про загиблих. +12 496 захворіло --- майже антирекорд. 256 померло --- антирекорд. 

Сьогодні у одного з популярних блогерів я прочитав, що «ми мусимо думати про
тих, хто найбільше страждає від карантину - митці, ресторани, туризм та
авіалінії, оркестри, спорт, кінотеатри». І ми справді маємо думати і
підтримувати їх. 

Але не менше ми маємо думати про тих, хто постраждає від відсутності
карантинних заходів. Про тих, хто самотньо помре в лікарні без родичів. Про
одиноких старих, які побояться звернутися до лікаря і так і загинуть в своїй
квартирі. Про ще зовсім не старих, яким не вистачить місць в реанімації, якщо
не збити темпи поширення вірусу.

Більшість цих людей не має облікових записів в соціальних мережах, на них не
підписані тисячі людей. У них чи їх родичів не візьмуть інтерв’ю ЗМІ. Вони не
напишуть драматичну колонку. Але їх життя це те, про що ми маємо пам’ятати
кожного дня, коли вирішуємо для себе чи одягнути маску, чи знову вірити казкам,
що «Covid не страшніший за грип».

Я дуже хотів би побачити на головних сторінках українських медіа історії тих,
кого з нами більше немає в наслідок епідемії. Це було б гідно. Це було б
важливо. Хто були ці люди? Яке звичне чи зовсім незвичне життя вони прожили? Їх
імена. Десять тисяч людей --- яких більше немає.
