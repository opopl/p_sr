% vim: keymap=russian-jcukenwin
%%beginhead 
 
%%file 08_04_2021.fb.promovugroup.1.mova_vsu_dnepropetrovsk
%%parent 08_04_2021
 
%%url https://www.facebook.com/groups/promovugroup/permalink/949571432283385/
 
%%author 
%%author_id 
%%author_url 
 
%%tags 
%%title 
 
%%endhead 

\subsection{Неповна службову відповідність}
\label{sec:08_04_2021.fb.promovugroup.1.mova_vsu_dnepropetrovsk}
\Purl{https://www.facebook.com/groups/promovugroup/permalink/949571432283385/}

За зверненням Уповноваженого із захисту державної мови Тараса Кременя
Міністерство оборони України провело службове розслідування стосовно ситуації,
яка склалася 18 лютого 2021 року в комунальному закладі освіти «Середня
загальноосвітня школа № 11» Дніпровської міської ради за участі
військовослужбовців Збройних Сил України.

“За даним інцидентом було призначено службове розслідування, яке підтвердило
факти, висвітлені в ЗМІ. За результатами проведення службового розслідування
офіцеру Харківського національного університету Повітряних Сил імені Івана
Кожедуба, який взяв безпосередню участь у суперечці через проведення заходу
російською мовою, оголошено “про неповну службову відповідність”, по іншим
винним посадовим особам прийняті кадрові рішення”, - йдеться у листі
Міноборони, який надійшов на адресу Уповноваженого.

У міністерстві також повідомили, що з метою додаткового роз'яснення вимог
Закону України “Про забезпечення функціонування української мови як державної”
щодо обов'язкового застосування державної мови у службовій діяльності в
Збройних Силах України проведено цільове інформування усіх категорій
військовослужбовців та низку просвітницьких заходів. 

Як відомо, 18 лютого 2021 року у школі Дніпра стався мовний скандал за участю
військових, які прийшли до закладу освіти, щоб заохотити учнів вступати до
військових вишів. Військовослужбовці отримали зауваження від школяра за
спілкування російською мовою, а у відповідь вони почали погрожувати хлопцеві
«бесідою».

Нагадаємо, відповідно до статті 21 Закону України “Про забезпечення
функціонування української мови як державної”, мовою освітнього процесу в
закладах освіти є державна мова.

Відповідно до статті 15 Закону, мовою нормативних актів, документації,
діловодства, команд, навчання, виховних заходів, іншого статутного спілкування
та службової діяльності у Збройних Силах України та інших військових
формуваннях, створених відповідно до закону, є державна мова.

Пресслужба Уповноваженого

\ifcmt
  pic https://scontent-bos3-1.xx.fbcdn.net/v/t1.6435-9/169231251_342029454259756_7043746642627259739_n.jpg?_nc_cat=108&ccb=1-3&_nc_sid=730e14&_nc_ohc=A1kZCuayRtsAX9jFtYF&_nc_ht=scontent-bos3-1.xx&oh=a9a914873d9e5383468c2cf698bc2b63&oe=60939576
\fi

