% vim: keymap=russian-jcukenwin
%%beginhead 
 
%%file 24_10_2021.fb.vselennaja_mudrosti.1.volna_dobra
%%parent 24_10_2021
 
%%url https://www.facebook.com/Per.aspera.ad.astra.A.N/posts/1503764636677110
 
%%author_id vselennaja_mudrosti
%%date 
 
%%tags chelovek,dobro,volna
%%title ВОЛНА ДОБРА...
 
%%endhead 
 
\subsection{ВОЛНА ДОБРА...}
\label{sec:24_10_2021.fb.vselennaja_mudrosti.1.volna_dobra}
 
\Purl{https://www.facebook.com/Per.aspera.ad.astra.A.N/posts/1503764636677110}
\ifcmt
 author_begin
   author_id vselennaja_mudrosti
 author_end
\fi

ВОЛНА ДОБРА... 

Тащу ведро винограда, сзади в рюкзаке ведро айвы по спине телебонькает. И тут
слышу:

- ВАх, красавица! Как я счастлив тебя видеть!

Смотрю - навстречу идет мужчина, очень кавказской национальности. Смотрит не в
телефон. На меня смотрит. На ведро мое виноградное смотрит. И улыбается.

"Знаем мы вас, детей горячих гор. У вас все красавицы" - думаю я - а у самой
рот расползается до ушей. 

\ifcmt
  ig https://scontent-frx5-1.xx.fbcdn.net/v/t39.30808-6/248064545_1503764586677115_7847304815148771875_n.jpg?_nc_cat=110&ccb=1-5&_nc_sid=8bfeb9&_nc_ohc=5iTSKjfNOxoAX_9DJqO&_nc_oc=AQlFdkXocHl7Shcw9jmmVtfQoRSWvw0GF3pUNmfZjfqaMkEaat_1B1tuTBINNKnhd8s&_nc_ht=scontent-frx5-1.xx&oh=9487946740d2b3bee440985b13d32a0b&oe=61A4BFC1
  @width 0.4
  %@wrap \parpic[r]
  @wrap \InsertBoxR{0}
\fi

Захожу в магазин, смотрю в зеркальную витрину - ого, откуда то и румянец
взялся! Раскраснелась, глаза горят, на лице ямочки - а всего лишь мужчина мимо
шедший красавицей назвал.

Тут же и спина выпрямилась, и волосы со лба откинулись. Смотрю на продавщицу -
а у той тоска зеленая на лице. В каждом глазу по пулемету притаилось. Так
видать, достали покупатели и человечество в целом - что всех бы на фарш и
покрошила. Тем более, что мясорубка вон, за плечом стоит.

- Как я вас рада видеть! - восклицаю - Какой у вас красивый шарфик на шее -
просто необыкновенный.

Я что? Да, приврала трохи про шарфик! Но у меня собственного хорошего
настроения много неожиданно образовалось - через край переплескивается. Надо ж
поделиться!

Смотрю - пулеметы отъехали и бойницы закрылись в глазах. А вместо них свет
появился. Ласковый такой, цвета неба.

- Спасибо! - улыбается мне, - Что вам предложить? У нас сегодня грудинка свежая
- диво как хороша. 

Ну красавица, чисто красавица! И шарфик на шее и впрямь на шарфик похож стал, а
не на унылую тряпочку.

Возвращаюсь домой, прижимая к груди грудинку. А тут навстречу соседка - злая,
как сто тысяч чертей.

- Чего это вы своего кота тут завели и не кормите? Он у моего Тузика еду
ворует, скотина черная!

Тузик - это псина огромная, для которого мой котик все равно что бутерброд на
один зуб. Да у Тузика в миске костомахи больше чем мой котик вместе с хвостом.
Но соседке не докажешь. Она сейчас за своего Тузика готова порвать и меня, и
котика вридачу.

В другой раз, я бы обрадовалась поводу рассказать соседке, что я думаю по
поводу ее псинушки и ее самой. Но сегодня мне никак не ругается. У меня сегодня
такое настроение - что я готова любого голодного Тузика накормить.

- Как я рада вас видеть! - восклицаю - так люблю вашего щеночка милого - сил
никаких нет. Вот, хотите - грудинку ему эту отдам. Смотрите, какая она славная! 

Смотрю - а злобный соседкин оскал превратился в довольно милую улыбку.
Подтянулись брыли и превратились в обыкновенные щеки, волосы кудрями по плечам
раскинулись

- Да что вы, буду я еще этому гаду прожорливому грудинку вашу давать. Давайте я
лучше вашего котика угощу! Кыс-кыс-кыс! Иди сюда моя прелесть!

У моей вороватой прелести аж глаза на лоб полезли! Такой ласки от той, что
кроме как :"Убью, скотину!" говорила - она во век не слыхала...

А всего лишь один проходящий человек, сказал другому проходящему мимо человеку
: "Как я счастлив тебя видеть!"

Да, может, даже не ему сказал, а в телефон сказал.

Но все равно волна пошла!

Говорите друг другу больше хороших слов. Просто так.

И всё будет - зашибись!

Ирина Подгурская
