% vim: keymap=russian-jcukenwin
%%beginhead 
 
%%file slova.bot.internet
%%parent slova
 
%%url 
 
%%author 
%%author_id 
%%author_url 
 
%%tags 
%%title 
 
%%endhead 
\chapter{Бот (интернет)}
\label{sec:slova.bot.internet}

%%%cit
%%%cit_head
%%%cit_pic
%%%cit_text
Це зносини, так би мовити, між пасіонарними групами. А що відбувається в
головах? От дивись, що змушує пристойних, нібито нормальних людей, поводитись в
інтернеті принаймні, як абсолютні, вибач, єбанати.  Я називаю це, до речі,
феномен Гортензії. Це коли от жіночка з гортензією на аватарці, з онуками,
рецептами сирників, раптом починає писати, що вона хоче тебе вбити, з’їсти, а
те що не доїла ще й трахнути хоче. В чому проблема?  Ну смотри, во-первых, не
будем забывать, что у нас работает огромное количество \emph{ботоферм},
достаточно большое.  Ні, це реальні люди.  Смотри. То есть, есть
\emph{ботофермы}, которые накачивают это все, и которые создают эту виртуальную
агрессию и ощущение вот этой ненависти. Второе – реальные люди, которые ведут
себя так в инете, они себя так ведут, потому что они не сталкиваются с ответкой
%%%cit_comment
%%%cit_title
\citTitle{Украина в плену консервативного мышления: почему 30 лет шли не туда и что делать}, 
Сергей Иванов; Юрий Романенко, hvylya.net, 15.07.2021
%%%endcit

