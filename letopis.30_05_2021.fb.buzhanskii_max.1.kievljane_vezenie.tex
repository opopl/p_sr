% vim: keymap=russian-jcukenwin
%%beginhead 
 
%%file 30_05_2021.fb.buzhanskii_max.1.kievljane_vezenie
%%parent 30_05_2021
 
%%url https://www.facebook.com/permalink.php?story_fbid=1956794871151649&id=100004634650264
 
%%author 
%%author_id buzhanskii_max
%%author_url 
 
%%tags kiev,ukraina,den.kieva
%%title Киевляне, в вашем городе иногда везёт
 
%%endhead 
 
\section{Киевляне, в вашем городе иногда везёт}
\label{sec:30_05_2021.fb.buzhanskii_max.1.kievljane_vezenie}
\Purl{https://www.facebook.com/permalink.php?story_fbid=1956794871151649&id=100004634650264}
\ifcmt
 author_begin
   author_id buzhanskii_max
 author_end
\fi

Киевляне, в вашем городе иногда везёт.

Везёт случайно забрести куда то туда, где вы храните Киев для себя, не для
туристов.

Не для легионов чиновников и не для миллионов приезжих, которые пытаются тут же
переделать ваш город под привычное родное село.

Не со зла, просто так видят мир.

Забрести туда, где ваши дворы, и дворы ваших родителей, лавочки, на которых вы
сидели детьми и деревья, в тени которых читали книги.

Вы ведь читали тогда, мы знаем, и сейчас читаете, когда выключается режим
столицы и вы просто попадаете домой.

Достаточно зайти туда, в этот ваш Киев, которым вы никогда не делитесь ни с
кем, потому что это значит оторвать от себя часть, и сразу становится понятно,
о чём вы, когда говорите - Наш Киев.

Ваш!)

Только ваш, на самом деле, сколько бы нас, приезжих, не промелькнуло на его улицах.

С праздником вас, с Днём Вашего Города!)

Будьте всегда счастливы с ним, и всегда у себя дома!)

\begin{itemize}

\iusr{Владимир Волков}

В Киеве прожил более 15 лет. Знаю практически каждую улочку. Но, увы, детей
пришлось увезти пару лет назад.  Город с праздником!

\iusr{Ольга Епифанович}

Личное впечатление: Ярославов Вал состоит ровным счётом из того, из чего должен
состоять хороший город: из неширокой проезжей части, из невысокой достойной
застройки, из нормальных заведений питания и маленьких магазинов. Сверх
положенной нормы этой улице в самом историческом сердце столицы достались,
конечно, и весомые статусные дивиденды в виде таких шедевров архитектуры, как
«замок барона», караимская кенасса, доходный дом Родзянко и бывшая
экономическая школа, ныне корпус института Карпенко-Карого. Как и многое лучшее
в жизни, эту улицу формирует ряд совпавших в одном месте простых и правильных
вещей.

\iusr{Макс Бужанский}

Ольга Епифанович, искренне не имею понятия, о чём речь)))

\iusr{Артем Гриненко}

Макс Бужанский і не зрозумієте, бо не киянин, просто випадково потрапили до
Києва

\iusr{Сергей Удовик}

Макс Бужанский будете в Киеве, могу экскурсию по этой улице устроить, там ещё и
шикарный интерьер в стиле модерн театра Сузирья

\iusr{Ольга Епифанович}

Макс Бужанский Я искренне вам завидуют, сколько открытий вы можете сделать для себя в Киеве!!

\iusr{Aleksander Golov}

И, к сожалению, разгромленный дом Сикорского...

\iusr{Иван Шолль}

И главное - булочки в Ярославе! Обожаю ))

\iusr{Ольга Епифанович}

Артем Гриненко А вы, прошу прощения не случайно попали в Киев??

\iusr{Лариса Михальченко}

Спасибо,Максим!

\iusr{Наталия Литвиненко}

Не любят киевляне свой город, да и страну украинцы не любят. Любили бы-не
допустили этой вакханалии во всех сферах и не докатились бы до ручки.

Всем нравятся праздники, а не рано ли при этой коричневой чуме, которая
захватила Украину, торжества устраивать?!

Пир во время чумы, достижений нет, разруха и крахоборы у власти,
законодательство и Конституция попраны, а люди находятся в унынии и покорны
злу.

Думаю, что не о таком будущем, мечтали наши предки, когда отдавали свои жизни
за НАШЕ с вами будущее.

ПОЗОР нам ущербным, которые не смогли их Мечту воплотить в Жизнь.

Построили монстра, а не государство, которое НАС с вами и поедает.

А теперь ПРАЗДНУЙТЕ, если есть что и чем гордиться?!

\ifcmt
  pic https://scontent-frt3-2.xx.fbcdn.net/v/t1.6435-9/192824223_863193324283166_1271537555566119188_n.jpg?_nc_cat=101&ccb=1-3&_nc_sid=dbeb18&_nc_ohc=PIrxBBun0E8AX9FoyfQ&_nc_ht=scontent-frt3-2.xx&oh=0032f9831e0d7b5725d18b2d6957ffa1&oe=60D87C90
\fi

\iusr{Скрипка Николай}

Такое впечатление что в Киеве ставят какой-то злой эксперимент над жителями...
По-другому объяснить маразм и анархию которая происходит в Киеве сложно...

\iusr{Лиля Шрага}

Скрипка Николай Этот эксперимент уже 28лет ставят, с нашего молчаливого
согласия, не только над Киевом, а над всей страной.

\iusr{Скрипка Николай}

Лиля Шрага тут про Киев...

\iusr{Лиля Шрага}

Скрипка Николай Как вам ни покажется странным, я поняла.в Киев была влюблена с
1969 года. Мечтала там жить. В 1983 даже обмен нашла ,любя его по-прежнему, наш
Киев. Не сложилось в силу личных обстоятельств. А вот когда попала в Киев уже в
93 году, а позже-в 96, то поняла, что это не мой Киев. И люди там наполовину
уже не те...

\iusr{Елена Плитина}

Спасибо! Но, к сожалению, уже и наш исчезает, вернее - оккупировали его приезжие и изменяют под себя.

\iusr{Артем Гриненко}

Вам цього не зрозуміти, бо не киянин

\iusr{Макс Бужанский}

Артем Гриненко, а разве я выдаю себя за киевлянина?))

\iusr{Артем Гриненко}

Макс Бужанский слава Богу, що не видаєте...

\iusr{Андрей Данко}

Артем Гриненко, достаточно того, что Макс остается быть адекватным человеком и в жизни и в Парламенте.

\iusr{Иван Шолль}

Спасибо. Это лучшее поздравление с днём Города.

\iusr{Александр Васин}

Киева уже нет, остался КУев и куяне....

\iusr{Svetlana Ivanova}

Огромное спасибо. Тепло Ваших слов о Киеве дорогого стоит. Будьте здоровы и
счастливы не только в Киеве, но и на планете Земля.

\iusr{Alena Vladimirova}

Благодарю

\iusr{Natalia Lazarenko}

А вот \enquote{моего Хмельницкого} уже не осталось, ни в одном уголке моего города,
нигде не нашла. Того Тёплого. уютного, приятного и для посидеть почитать, и для
просто погулять, нет его. Его дома изменились(налеплены балконы разные,
разномастные), убрали плющ и виноград, увивавший кирпичные стены. А иной раз и
сами стены снесены, на их месте выстроены высотки, просто уродующие город и
делающие его просто каменными джунглями. Всё меньше мест, где птицы могут вить
гнёзда(срезаны огромные деревья). Грустно!


\end{itemize}
