% vim: keymap=russian-jcukenwin
%%beginhead 
 
%%file 27_05_2017.stz.news.ua.mrpl_city.1.istoria_realnoje_uchilische
%%parent 27_05_2017
 
%%url https://mrpl.city/blogs/view/realnoe-uchilishhe
 
%%author_id burov_sergij.mariupol,news.ua.mrpl_city
%%date 
 
%%tags 
%%title История: Реальное училище
 
%%endhead 
 
\subsection{История: Реальное училище}
\label{sec:27_05_2017.stz.news.ua.mrpl_city.1.istoria_realnoje_uchilische}
 
\Purl{https://mrpl.city/blogs/view/realnoe-uchilishhe}
\ifcmt
 author_begin
   author_id burov_sergij.mariupol,news.ua.mrpl_city
 author_end
\fi

Есть в Мариуполе приметный дореволюционной постройки трехэтажный дом. Его адрес
- улица Николаевская, № 7/17. Построен  он в начале ХХ века. История этого
трехэтажного дома довольно хорошо изучена. Известно, что он принадлежал
господину Оксюзову. То ли Харлампию, то ли Харитону Васильевичу. До наших дней
дошли лишь первые три буквы его имени – хар.

\ii{27_05_2017.stz.news.ua.mrpl_city.1.istoria_realnoje_uchilische.pic.1}

В 1907 году на дом  глаз положил учредитель реального училища Василий Иванович
Гиацинтов. Человек он был не простой.  Он  был гласным (депутатом)
Мариупольской городской думы. Но не это главное. Главное, что тестем Василия
Ивановича был Александр Антонович Гозадинов,  который более 25 лет возглавлял
Мариупольскую земскую управу, был самым крупным землевладельцем в уезде и,
кроме того, приходился  родственником самому митрополиту Игнатию. Читатель
догадался - Василий Иванович взял в аренду здание для реального училища.

Что такое реальное училище? Среднее учебное заведение, отличающееся от
классической мужской гимназии только тем, что в его программе не было древних
языков – греческого и латинского. Эта \enquote{мелочь} не позволяла реалистам поступать
в университеты. Им  была прямая дорога в институты гражданских инженеров,
горные и технологические.  Если, конечно, было кому платить за обучение.  

\ii{27_05_2017.stz.news.ua.mrpl_city.1.istoria_realnoje_uchilische.pic.2}

Известные выпускники мариупольского реального училища. Виктор Александрович
Синайский — выдающийся ленинградский скульптор и педагог. Константин Федорович
Челпан — конструктор,  руководитель конструкторского коллектива по созданию
двигателя  для танка Т-34 и других боевых машин. Георгий Антонович Костоправ —
румейский писатель, поэт и драматург, основатель национальной литературы греков
Приазовья.

Мариупольское реальное училище просуществовало до 1922 года, и было закрыто
из-за отсутствия средств на отопление. С сентября 1922 года в коридорах и
классах дома № 7/17 звучали не только голоса мальчишек, но и девочек. Теперь
это была 3-я трудовая школа.  Учили ребят бывшие преподаватели реального
училища и мариупольских гимназий. Среди выпускников 3-й школы был и Виктор
Васильевич Маевский – журналист-международник и дипломат, автор книг \enquote{Крепкие
люди}, \enquote{На Британских островах}, \enquote{Первый или пятый} и других. Эту же школу
окончил Иван Демьянович Кузема, кандидат технических наук, ученый-металлург,
посвятивший многие годы разработке новых марок стали для бронетехники и
судостроения.

\ii{27_05_2017.stz.news.ua.mrpl_city.1.istoria_realnoje_uchilische.pic.3}

Шли годы. В Мариуполе постепенно было восстановлено большинство строений,
сожженных отступающими гитлеровцами  с 7  по 10 сентября 1943 года. И только на
погорелке бывшего реального училища зияли пустые проемы окон. Лишь  в начале
60-х годов прошлого века   только что образованный  трест \enquote{Ждановжилстрой}
восстановил многострадальное здание.  В нем  разместилось управление трестом.
Однако  в начале 90-х годов жилищное строительство резко пошло на спад. И трест
\enquote{Ждановжилстрой} приказал долго жить. Комнаты его стали занимать разные
организации и учреждения. Постепенно строение пришло в упадок.

Но относительно недавно историческое здание было бережно отреставрировано
настоящими  патриотами  Мариуполя – семьей Сапаровых. Их патриотизм выразился
не в призывах и лозунгах, а в конкретном деле и любви к городу, где они живут.
Более того, они подарили городу часы с курантами, установленные на фасаде
бывшего реального училища.
