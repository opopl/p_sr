% vim: keymap=russian-jcukenwin
%%beginhead 
 
%%file letters.to_bereza
%%parent letters
 
%%url 
 
%%author_id 
%%date 
 
%%tags 
%%title 
 
%%endhead 

%19:49:49 08-03-23
Доброго вечора, пане Береза!

Слава Україні! Щиро вдячний за Вашу невтомну діяльність!

Хотів би тут написати щодо Маріуполя, звернути увагу на таку проблему.  Це
проблема насправді більш культурно-духовна та інформаційна, ніж військова.
Щодо військового аспекту, я впевнений, що Залужний обіцянок дарма не дає, і
Маріуполь дійсно скоро звільнять. Але... крім військового звільнення, потрібно
також звільнення культурне, духовне. Бо за місці колись щасливого, неймовірно
гарного Міста біля Моря - Маріуполя, зараз і на час звільнення, на той час коли
над Маріуполем замайорить знову наш прапор, буде Місто-Кладовище, Зруйноване
Місто, напівмертве Місто, Місто, де більшість будинків зруйновано, і місто, де
тисячі, десятки тисяч людей загинуло жахливою смертю. І потрібно буде дуже
багато зусиль, зусиль духовних, творчих - щоби знову це Місто засяяло, стало
таким же радісним і красивим, яким воно було всього навсього рік з гаком тому.
Так, існує зараз проблема Маріуполя, оскільки Маріуполь... фактично зник з
інформаційного та культурного простору країни, і це дуже боляче мені бачити. Я
взагалі то кажучи киянин, а не маріуполець, і тому мені легше відносно про це
писати. І тут є багато аспектів цієї проблеми, я зупинюсь зараз на проблемі
відсутності книжок про Маріуполь.

Справа у тому... що в Києві майже немає книжок про Маріуполь.
Практично немає, окрім декількох найменувань. І це неймовірно прикро... тому
що, по перше, сталась жахлива трагедія, яка вимагає якнайскорішої публікації
щоденників, свідчень, фото... - потрібні товсті томи, де все це би фіксувалось
би. На кшталт Книги Колективна Пам'ять, Голод 1946-1947, яка є в мене вдома, і
в якій більше ніж 1000 сторінок.  Але... всього цього немає щодо Маріуполя.
Звичайно, є в телеграмі, фейсбуці, але всі ці свідчення є неймовірно
розпорошені, і більш того, під загрозою зпрямого нищення і забуття, оскільки як
відомо, фейсбук є дуже ненадійне середовище для публікацій, і тим більше - для
довгострокового зберігання памяті про всі ці жахи, та про злочини росіян.
Достатньо одного натискання клавіші, щоби пост або аккаунт, той або інший,
просто зник, і назавжди... Як це сталось наприклад минулого року із аккаунтом
маріупольської журналістки Надії Сухорукової, в якому вона викладала уривки із
свого щоденника (він щойно був опублікований у вигляді невеличкої книжки, до
речі) А по друге... Маріуполь потребує не тільки військового звільнення, але
також духовного, культурного звільнення. Зараз... багато маріупольців в
розпачі... їхні домівки знищено, їхнє кохане місто знищено та перебуває в
окупації... Напів-мертве Місто... Місто-Кладовище... І потрібно також зберігати
пам'ять про мирний, щасливий, радісний Маріуполь! І це можна зробити також,
оскільки на даний момент є велика кількість фото, відео, але знову ж, таки, все
це розпорошене, розкидано... Я вже всім цим почав займатись - систематизацією,
записом доступної інформації. Як все це виглядає, можна подивитись ось тут... Я
поступово викладаю матеріали у готовому до перегляду і друку вигляді
https://archive.org/details/@kyiv_chronicler  Також тут
https://t.me/kyiv_fortress_1 і також на моїй фб сторінці www.facebook.com/ivan.ivan.kyiv Але то є окрема
величезна робота, якої я займаюсь у свій вільний час (я програміст). Крім
того... щодо проблеми культурного звільнення Маріуполя... відродження
Маріуполя. Потрібні нові книжки про Маріуполь... нові, цікаві книжки... багато
книжок. І я був би дуже радий, якщо як перший крок до духовного, культурного відродження Маріуполя,
про Маріуполь довоєнний, мирний, було би більше публікацій в телеграмі або фб,
щоби долею Маріуполя переймалась вся Україна, щоби маріупольці як спільнота не залишались би наодинці самі-по-собі,
і щоби це дійсно відчувалось по тому, що відбувається
в телеграмі або фб.

З повагою

Іван.


