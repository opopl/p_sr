% vim: keymap=russian-jcukenwin
%%beginhead 
 
%%file 17_09_2021.fb.pavlova_viktoria.1.vybory_rossia_dnr
%%parent 17_09_2021
 
%%url https://www.facebook.com/pavlova1975/posts/833085180744195
 
%%author_id pavlova_viktoria
%%date 
 
%%tags dnr,donbass,rossia,vybory
%%title Все еще будет. Это ведь первые в моей жизни выборы в статусе гражданки России, но не последние
 
%%endhead 
 
\subsection{Все еще будет. Это ведь первые в моей жизни выборы в статусе гражданки России, но не последние}
\label{sec:17_09_2021.fb.pavlova_viktoria.1.vybory_rossia_dnr}
 
\Purl{https://www.facebook.com/pavlova1975/posts/833085180744195}
\ifcmt
 author_begin
   author_id pavlova_viktoria
 author_end
\fi

Проголосовала @igg{fbicon.hand.ok} 

В последний раз я была на выборах 2 ноября 2014 года, когда выбирали главу ДНР.
Тогда война еще только начиналась, Донецк и пригороды обстреливали
круглосуточно, многие уехали из города, многие успели вернуться, многие были
еще живы.

\ifcmt
  ig https://scontent-lga3-1.xx.fbcdn.net/v/t1.6435-9/242122801_833085164077530_8701244749163744986_n.jpg?_nc_cat=101&ccb=1-5&_nc_sid=730e14&_nc_ohc=-AHXK2wqTLkAX-oKl-b&_nc_ht=scontent-lga3-1.xx&oh=99fdb59f7a8de146ecff2d47ffc69114&oe=6176545F
  @width 0.4
  %@wrap \parpic[r]
  @wrap \InsertBoxR{0}
\fi

Еще жива была моя родная тетя. На выборы мы ходили всей семьей, и она с нами.
Решили, что пойдем во второй половине дня, чтобы наплыв людей спал. Куда там -
очереди на избирательные участки стояли, как в мавзолей в советские времена.
Все терпеливо ждали, людская река потоком перемещала нас все ближе к кабинкам
для голосования.

Тогда я поставила галочку напротив одной из фамилий и облегченно выдохнула с
надеждой на светлое будущее и скорое окончание войны. Тогда я не знала, что
война затянется на долгие семь лет, что в следующем году не станет тети, что я
буду выть от боли, просматривая очередные сводки погибших и разрушений. 

Много воды утекло за семь лет. Я повзрослела, заматерела, стала циничней, но
надежда на светлое будущее никуда не делась. Это первое в моей жизни
голосование в статусе гражданки России. Для кого-то - рядовая процедура, а для
меня она волнительна, трогательна, торжественна. Ты чувствуешь себя пусть
маленькой, но все же частью огромной страны. 

Я голосовала через "Госуслуги". Пожалела, что не поехала очно -
зарегистрировалась на дистанционное электронное голосование раньше, чем узнала
о том, что будут организованы автобусные спецрейсы. А сегодня посмотрела
репортажи о наших людях, решивших голосовать лично, и чего-то так захотелось
быть среди них. 

Все еще будет. Это ведь первые в моей жизни выборы в статусе гражданки России,
но не последние. Хотя мне бы очень и очень хотелось, чтобы к следующему разу
границы России расширились, чтобы в ее составе появился еще один Федеральный
округ или республика. 

\#донбассголосует, \#выборывРоссии
