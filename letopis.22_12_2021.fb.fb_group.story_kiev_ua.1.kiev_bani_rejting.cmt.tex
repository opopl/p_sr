% vim: keymap=russian-jcukenwin
%%beginhead 
 
%%file 22_12_2021.fb.fb_group.story_kiev_ua.1.kiev_bani_rejting.cmt
%%parent 22_12_2021.fb.fb_group.story_kiev_ua.1.kiev_bani_rejting
 
%%url 
 
%%author_id 
%%date 
 
%%tags 
%%title 
 
%%endhead 
\zzSecCmt

\begin{itemize} % {
\iusr{Владимир Бондаренко}
на сковороды напротив церквы Мыколы Доброго. сейчас там БЦ

\iusr{Виктор Задворнов}

С легким паром! Посещал баню на Кременецкой (рядом с военкоматом Святошинского
района), на Брест-Литовском просп. - теперь реконструированное помещение
конторы бывшего \enquote{Брокбизнес банка}. Но легендарной считаю баню в поселке
Коцюбинский. Там отапливали банные \enquote{залы} и грели воду благодаря
функционированию кочегарки. Парадокс: идешь принять душ и попариться, а из
двери котельной выглядывает \enquote{темная} от угольной сажи личность - пьяненький
кочегар. Мыться пожалуйте! Цены в 60-е-70-е были весьма сносными.


\iusr{Sasha Jampolsky}

У меня немного менее исторические заметки про бани. Выросши в общей квартире на
Крещатике, без ванны, то есть ванна была но она не работала наверно с ещё
довоенных времен, ходили мы в Караваевские бани. Раз неделю, я с папой а мама с
бабушкой, ходили иногда в общее , а зачастую в номера. Кажется это стоило тогда
рупь за час. А потом, подросши и распознавши прелести противоположного пола, я
ходил туда с избранницами так как там паспорта у меня не просили потому что
знали с детства и ещё за два рубчика ( один сверху) можно было занять номер на
полные два часа.

А заодно и помыться


\iusr{Елена Мельникова}

В 60-70г. 20 века многие посещали общественные бани. Подол например ходил в баню
по Героев Триполья (ныне Спасская). Общие номера с парилкой, отдельные
номера-душ и ванная. Мылись как правило раз в неделю. Пиво и раков не помню, а
маникюр, парикмахер и замечательная косметичка Галина Григорьевна курившая
только \enquote{Беломор} по фронтовой привычке.

Может кто помнит?

\begin{itemize} % {
\iusr{Vladimir Dotsenko}
\textbf{Елена Мельникова}

С 1954 по 1961 посещали эти бани всей семьей, после в буфете для Отца было
бочковое пиво, а для нас, детишек - сладкий вкуснющая газировка с двойным
сиропом...


\iusr{Ольга Кирьянцева}
\textbf{Елена Мельникова} да, посещали эту баню). Номер с душем и ванной 1 рубль.
\end{itemize} % }

\iusr{Владимир Хворов}

у Нас на Андреевском спуске вода была по среди двора (колонка) Мы всей семьей
по субботам ходили в баню на Малой Житомирской Иногда в номера ст 1 рубль в час
Там были душ и ванная Ванная была просто роскошь Меня тщательно мыли а потом
разрешали плюхаться в ванной Это был кайф


\iusr{Катерина Силина}

Вы забыли о бане на Постышева. Мы всей семьей ходили туда, когда проживали на
Михайловской в коммунальной квартире. Я была ещё маленькая, но помню там был
бассейн в лягушками по углам.

\begin{itemize} % {
\iusr{Олег Чудновский}
\textbf{Катерина Силина} Да это сейчас ул. Малая Житомирская, там о ней есть.

\iusr{Катерина Силина}
\textbf{Олег Чудновский} Спасибо, теперь понятно, что не забыли нашу баню

\iusr{Светлана Николаевна}
\textbf{Олег Чудновский} в детстве жили в коммуналке на Тарасовской и тоже ходили в баню Караваевскую, память осталась на всю жизнь, спасибо напомнили, чудные были времена!!

\iusr{Олег Чудновский}
\textbf{Светлана Николаевна} Приятно, но спасибо надо сказать Юрию Бубнову.

\iusr{Лариса Кушниренко}
\textbf{Катерина Силина} Очень даже помню . Жили на Десятинной и это была наша баня. @igg{fbicon.face.happy.two.hands} 
\end{itemize} % }

\iusr{Петро Гарматюк}
А на вулиці Михайлівській баня була ?
Чи я помиляюся ?

\begin{itemize} % {
\iusr{Irena Visochan}
\textbf{Петро Гарматюк} это на Постышева рядом.

\iusr{Петро Гарматюк}
\textbf{Irena Visochan}
Спасибо.

\iusr{Давид Вайнблат}
\textbf{Петро Гарматюк} Була. Я жив на Прорізаний і туди ходів
\end{itemize} % }

\iusr{Владимир Шевцов}
Печерская Московская угол Суворова. Сейчас там Зоряный

\iusr{Алла Ковтун}
Троицкие до конца 80-х ещё работали.

\iusr{Татьяна Жалнина}
\textbf{Алла Ковтун} точно, я там была ,теперь только воспоминания

\iusr{Микола Ахаладзе}
А ви забули Лазню на проспекті Перемоги (бувший Брест-Литовський проспект).

\iusr{Анатолий Борозенец}

Маю сумнів щодо присутності \enquote{Московських бань} на вул. Ярославів Вал,40.
Будівлю, де нині інститут ім.Карпенка-Карого, звели для економічної школи в
1907-му.

\begin{itemize} % {
\iusr{Елена Гнеденко}
\textbf{Anatoly Borozenets} 

Похоже, что эти бани находились на ул. Бульварно-Кудрявской, приблизительно в
\#6 или8, (точно не помню). Там жили родственники ( и в пятидесятых годах 20
столетия), в чердачных помещениях. Сейчас там административное здание.


\iusr{Анатолий Борозенец}
\textbf{Елена Гнеденко} Ні. \enquote{Московські бані} знаходилися на вул. Ярославській, 17/22.
\end{itemize} % }

\iusr{Pasha Paolo}

Бани Каплера, которые \enquote{Московские} НИКОГДА не располагались на ЯрВалу.
Правильный адрес - Константиновская 22/17. Фактически, это гостиница \enquote{Сион},
сами бани были во дворе и теперь это действительно корпус Театрального
института.

\begin{itemize} % {
\iusr{Анатолий Борозенец}
Корпус Театрального інституту там колись був. Зараз, наскільки мені відомо, там стоматологічна клініка

\ifcmt
  ig https://scontent-frt3-1.xx.fbcdn.net/v/t39.30808-6/269778544_3072023223053391_3238025887301750879_n.jpg?_nc_cat=102&ccb=1-5&_nc_sid=dbeb18&_nc_ohc=V0PlqVYk-XgAX89i0sE&_nc_ht=scontent-frt3-1.xx&oh=00_AT-FdM49YrDS86Fw_dDpru5A7BreVmHPVhzJzqI_JzdLsA&oe=61CACF58
  @width 0.4
\fi

\begin{itemize} % {
\iusr{Александра Майорова}
\textbf{Анатолий Борозенец} 

клиника есть. РЯДОМ В ТОМ ЖЕ ДВОРЕ, КАК ВОЙТИ В АРКУ КОРПУС КГИТИ КАРПЕНКО -
КАРОГО В СТАРЫХ БАНЯХ С 1991 ГОДА. ЭТО ОТДЕЛЬНО СТОЯЩЕЕ 2 - Х ЭТАЖНОЕ ЗДАНИЕ
ВНУТРИ ДВОРА. ВХОД СО СТОРОНЫ СКВЕРА.


\iusr{Анатолий Борозенец}
\textbf{Александра Майорова} Дякую за інформацію!

\iusr{Александра Майорова}
\textbf{Анатолий Борозенец} Нина Апольская сейчас разместила фото с Адой Роговцевой в той самой Арке! Найдете легко в К.И. 52 ФОТОГРАФИИ ОТ НЕЕ....!

\end{itemize} % }

\iusr{Pasha Paolo}
Собственно, фото с вывеской бань.

\ifcmt
  ig https://scontent-frt3-2.xx.fbcdn.net/v/t39.30808-6/269859073_4663856050357469_709307188187473244_n.jpg?_nc_cat=101&ccb=1-5&_nc_sid=dbeb18&_nc_ohc=4urIWXH3oLQAX_0uW_i&_nc_oc=AQkyHhywyUMIYjOhbsiViHly2UBjREkQ6j_cAMX67B9l2o078KmSKIaYIU6gl0EIUvM&_nc_ht=scontent-frt3-2.xx&oh=00_AT-MyS_cxZLhdq4NSKFLk5aYtpahjbqXfmehlvtZL-Gu4w&oe=61CB0E7F
  @width 0.4
\fi

\end{itemize} % }

\iusr{Александр Сторчевой}
А на Ивана Кудри, кто помнит?

\begin{itemize} % {
\iusr{Кира Юкаева}
\textbf{Александр Сторчевой} назывались Боинские

\iusr{Олег Чудновский}
\textbf{Александр Сторчевой} Помним, хорошая баня была, там сейчас новый дом выстроили.

\begin{itemize} % {
\iusr{Алексей Шевнюк}
\textbf{Олег Чудновский} Олег а на Московской возле кинотеатра Авангард?

\iusr{Алексей Шевнюк}
Там в основном ущилище Связи парилось. Ещё с тазиками

\iusr{Elina Modenova}
\textbf{Алексей Шевнюк} нет, не только! Нас мама водила именно в эту баню, возле кинотеатра Авангард на Московской. Тазики- да, парилка- супер. Я старалась залезть на самый верх.

\iusr{Алексей Шевнюк}
\textbf{Elina Modenova}

\ifcmt
  ig https://i2.paste.pics/24f63d14e05ecec1dda838cb22983e41.png
  @width 0.2
\fi

\iusr{Олег Чудновский}
\textbf{Алексей Шевнюк} Помню, Леха и эту и которая на ул.Лейпцигская была!!!
\end{itemize} % }

\iusr{Светлана Власенко}
\textbf{Александр Сторчевой} я ходила с сестрой
\end{itemize} % }

\iusr{Yuriy Bubnov}

Всем, кто пишет: \enquote{а вы забыли упомянуть о бане... такой-то}. Замечу, эта
публикация - не перечень всех бань Киева. К тому же, для вас есть простор для
собственных воспоминаниях о любимых банях - пишите, уверен, все с удовольствием
почитают!


\iusr{Павел Левченко}
А СЕГОДНЯ ХОТЬ ОСТАЛИСЬ ПРИЛИЧНЫЕ БАНИ КИЕВА???

\iusr{Ірина Кравець}
А не Вы наследник бань Бубнова?

\begin{itemize} % {
\iusr{Yuriy Bubnov}
\textbf{Ірина Кравець} Прочтите до конца текст публикации. Там есть ответ на вопрос.

\iusr{Ірина Кравець}
\textbf{Yuriy Bubnov}  Извините, невнимательна
\end{itemize} % }

\iusr{Наталия Дудник}

Спасибо большое  @igg{fbicon.hands.pray} , очень интересно. Я помню в детстве ходила с сестрой в
Демеевские бани в 80-х. Вполне приличные парилки были. С них у меня и началась
большая любовь к этому делу @igg{fbicon.face.smiling.eyes.smiling}. Правда много десятилетий ходим с друзьями в
сауну

\begin{itemize} % {
\iusr{Станіслав Теліцький}
\textbf{Natalia Dudnik}, 

там була розкішна чайна, оформлена під старовину. О 10.00 (до закінчення першої
зміни) вона починала свою роботу. Ми, з друзями, полюбляли ходити туди уранці.
Слід згадати, що там були сауна і класична парна, а також - басейн і суха зала.
А, після процедур - свіжий чай з мармеладом і печивом. Незабутні відчуття!

\iusr{Катерина Силина}
\textbf{Наталия Дудник} Я тоже помню Димеевские бани.

\iusr{Натуся Натуся}
\textbf{Наталия Дудник} У меня детство тоже было связано с Димеевской баней!
С покойной Мамочкой ходили по выходным!
На 2 этаже была парикхматерская и после бани я ждала, когда Мама наведёт красоту!
Для меня, для ребёнка это был интересный мир!!!

\iusr{Наталия Дудник}
\textbf{Натуся Натуся} именно - интересный мир.. это что-то было отличное от текущей жизни

\iusr{Alexey Shahan Snigur}
\textbf{Наталия Дудник} А где именно находились Демеевские бани ?

\begin{itemize} % {
\iusr{Наталия Дудник}
\textbf{Alexey Shahan Snigur} напротив Центрального автовокзала, если для ориентировки...

\iusr{Натуся Натуся}
\textbf{Наталия Дудник} Точнее, напротив стеклозавода!

\iusr{Натуся Натуся}
\textbf{Alexey Shahan Snigur} Точнее, напротив стеклозавода.

\iusr{Наталия Дудник}
\textbf{Натуся Натуся} именно
\end{itemize} % }

\end{itemize} % }

\iusr{Irina Figurskaya}

Дійсно, \enquote{Каравпєвські} були найкращими. Це я знаю по своєму досвіду. Моя бабуся
мене туди водила. Хоча дома була ванна, але бабуся казала, що \enquote{митися у ванній
- не митися}. Дійсно, миття у бані приносило велике задоволення. Після неї -
таке почуття, що от от злітиш! Це цілий ритуал.

P.S. Один раз ми пішли у Троїцьки (не було білетів у Караваєвські). Нам не
сподобалось. Більше не зраджували Караваєвським.

\iusr{Александра Цымбал}

Демиевская баня стоит развалиной почти в самом конце Лобановского
(Краснозвёздного), рядом с бывшей синагогой Боришпольского, в которой сейчас
детские кружки.

\iusr{Катерина Силина}
\textbf{Александра Цымбал} Мы тоже посещали эту баню, а позже пользовались банно-прачечным комплексом самообслуживания

\iusr{Александра Цымбал}
У меня есть книга о киевских банях 19 века - один из томов о разных сферах деятельности в Киеве в то время.

\iusr{Надежда Лабик}
В бани Бубнова я ходила т.к. жила от них недалеко.

\iusr{Ирина Собко}
На малой Житомирской ещё работают?

\iusr{Nick Nesin}
З друзями пиво в цій  @igg{fbicon.face.smiling.hearts} бані пили

\iusr{Polina Feldman}

Мы ходили с бабушкой на Голосаеевсой была баня это был как сщастливый день
железные тазики и отельные кабинки а также кто платил больше отельный номер для
купания в моей памяте осталось это незнаю еть ли там еще это Сталинка это лет 50
тому  ❄ ️ ☃️ @igg{fbicon.bouquet} 

\iusr{Svitlana Plieshch}

В институте ходили с подружками в баню на пр. Победы в районе \enquote{Большевика}.
Учились в Политехе и в день, когда у мальчиков была военка, у нас был банный
день, а потом шли в гости к кому-то из у Участниц. Готовили что-то вкусное и
гуляли до вечера. И бани хорошие были и отдых прекрасный. Там, по-моему потом
был банк. Который сейчас обанкротился. Правда бани уже не восстановят.

\begin{itemize} % {
\iusr{Ганжа Тамара}
\textbf{Svitlana Plieshch} я тоже туда ходила.

\iusr{Вахтанг Кварелашвили}
\textbf{Svitlana Plieshch} Там сейчас Высший антикоррупционный суд.

\iusr{Зоя Ян Корогодские}
\textbf{Svitlana Plieshch} когда закончился ремонт я пошел узнать по чем теперь помыться и мы с охраняющим ментом никак не могли друг друга понять. Я не понимал что в бане на входе делает мент и чего не пускает меня, а он не понимал какую стоимость номеров я хочу узнать в банке.

\iusr{Olena Karelina}
\textbf{Svitlana Plieshch} Як називалась ця баня?...не можу згадати...

\begin{itemize} % {
\iusr{Вахтанг Кварелашвили}
\textbf{Olena Karelina} никак. Просто Лазня.

\iusr{Svitlana Plieshch}
\textbf{Olena Karelina} шукала в інеті, не знайшла. Якби хоча б номер будинку знала.

\iusr{Alexander Krivoshapka}
\textbf{Vakhtang Kvarelashvili} Да, сверху большими буквами выложена надпись была, невозможно не заметить
\end{itemize} % }

\iusr{Михайло Довгошея}
Так само ходив туди)))

\iusr{Виталий Наталич}
\textbf{Svitlana Plieshch} я у отца спрашивал; почему, если это баня, а называется \enquote{Лазня}, там не моются, там лазят?
\end{itemize} % }

\iusr{Павел Кондя}
Ще на Жилянській, навпроти Л. Кузні)))

\iusr{Наталия Педос}
А мы в студенческие годы ходили в баню банно-прачечного комбината на улице Артема.

\iusr{Надежда Владимир Федько}
Я жив по Мало-Житомирській, 5, з 1947 до 1966. В Центральні бані ходили дуже часто.

\iusr{Галина Зуева}
И на Демиевке были бани

\iusr{Екатерина маковецкая}

на Подоле на Почайнинской тоже была прекрасная баня)))) бабушка и дедушка
любили туда ходить, даже имея ванную, несколько раз в год ходили в баню!


\iusr{Акбар Ярматов}

В Караваевских \enquote{банях}, помню в 84 тых годах, по заказу Министерства бытового
обслуживания, наш конструкторский отдел установил турникеты, как в метро.., но
не прижилось...)), влага и прочее, ну и конечно же человеческий фактор,
наличные деньги...)))

\iusr{Тамара Сергиенко}

Какая интересная тема-Бани!! Да, на Большевике, была шикарная баня, а главное
доступная всем. Поход в баню с подругой, настоящий праздник. Две парилки -сауна и
терма. А после маникюр и т. д. Неожидала, что многим будет приятно вспомнить и
обменяться воспоминаниями. На Толстого тоже была баня, но не могу вспомнить как
она называлась.

\begin{itemize} % {
\iusr{Natalia Lysina}
\textbf{Тамара Сергиенко} на Толстого, бывшей Караваевской, были Караваевские бани. На углу Пушкинской.

\iusr{Тамара Сергиенко}
\textbf{Natalia Lysina} Троицкие бани на Красноармейской.

\iusr{Валентина Зражевская}
\textbf{Тамара Сергиенко} по-моему это были Троицкие, или путаю?

\iusr{Тамара Сергиенко}
Да, Троицкие! Спасибо напомнили!

\iusr{Елена Гнеденко}
\textbf{Valentina Zrazhevskaya} Троицкие бани были напротив стадиона Олимпийский а на углу Пушкинской и пл. Толстого - Караваевские бани

\iusr{Elena Kuba}
\textbf{Валентина Зражевская} Троицкие - на Красноармейской

\iusr{Elena Kuba}
\textbf{Тамара Сергиенко} да, Кароваевские бани на Толстого -это праздник души и тела! Воспоминания детства!

\iusr{Тамара Сергиенко}
Да, а у меня студенческие годы! Только вспомнила, у Зощенко есть рассказ \enquote{Баня} -очищает смехом!
\end{itemize} % }

\iusr{Alexey Novozhylov}

Троїцькі бані ще працювали в кінці 85 початку 86 року, ходив туди в цей час.

\begin{itemize} % {
\iusr{Алёна Михайлова}
\textbf{Alexey Novozhylov} и я, и с подругами, там уже и сауна была. Я на Жадановского 7 жила до 92. Весь банно- прачечный комбинат прекрасно помню, рубашки папины сдавали, постельное белье.
\end{itemize} % }

\iusr{Yuriy Austin}

Ну почему \enquote{потребность пропала}? Ещё в начале 70-х можно было за 3 рубля в лапу
банщику, снять на часок \enquote{семейный номер}. Правда 3 рубля, в то время, были
очень большими деньгами. Да и не каждый банщик ещё пускал, боялись потерять
место.

\iusr{Василь Коваль}

Громадські бані допомогли людству позбутися вошей.

\begin{itemize} % {
\iusr{Григорий Владимирович}
\textbf{Василь Коваль}  @igg{fbicon.laugh.rolling.floor} Но можно было подхватить мандавошек.

\iusr{Василь Коваль}
\textbf{Григорий Владимирович} Можна.
\end{itemize} % }

\iusr{Владимир Дубровский}
А как же \enquote{Пошел ты в баню!}

\begin{itemize} % {
\iusr{Олександр Сєдих}
\textbf{Владимир Дубровский} Л.Ставицька. \enquote{Українська мова без табу}. Якщо вам цікаво, прочитайте, там це слово має інше значення. Українською ж - лазня...

\iusr{Татьяна Панасюк}
Да некуда уже пойти  @igg{fbicon.frown} 

\iusr{Tania Borisenko}
\textbf{Владимир Дубровский} \enquote{Пошел ты в баню}- вираз прийшов з Римської імперії.

\iusr{Анатолий Борозенец}
\textbf{Владимир Дубровский} Тепер цей вираз трошечки змінився та набув іншого змісту: \enquote{Йди ти в бан!} )
\end{itemize} % }

\iusr{Анна Сидоренко}
Моя семья ходила в Троицкие бани, поскольку жили в 20-ти шагах от неё, успела там ещё и подработать, будучи в декретном отпуске.

\iusr{Liudmyla Kalynovska}

На ДВРЗ была баня, построенная в 1949 году. Работала как баня где-то до 2000г.
Здание сохранилось. Сейчас там офис фарм. Компании Кусум, которая дала зданию
новую жизнь.


\iusr{Петр Кузьменко}

То, что помню с детства. Баня на Спасской. Сначала мы с родителями мылись там,
дважды в неделю, приходя из коммуналки на Андреевском спуске 2. Потом, нас
приводили по ночам из Морполита, когда наша баня была на ремонте. Ещё пару раз
мы приходили туда в сауну с любимой женой, с нашей Почайнинской. Теперь там
подольский бизнес-центр. Ушла история и большая часть жизни. Жаль...

\begin{itemize} % {
\iusr{Тамара Сергиенко}
\textbf{Петр Кузьменко}, всё когда нибудь заканчивается! Будут новые истории... Хорошие!!

\iusr{Петр Кваша}

Жаль что автор не вспомнил о бане на Спаской, это было очень даже банное
место, ходил туда с 60-г. была тогда еще поговорка, где можно встретить
киевлянина, отвечали в коммунальных банях. жил на Жданова.


\iusr{Rita Titova}
Я помню Караваевские бани.

\iusr{Constantin Fal}
\textbf{Петр Кузьменко} Общественные бани Михельсона

\iusr{Лариса Дерюгина}
\textbf{Петр Кузьменко} а мы, живя на Спасской 15, ходили часто и любили парилку.  @igg{fbicon.wink} 

\iusr{Alexey Shahan Snigur}
\textbf{Петр Кузьменко} Подскажите адрес бани на Спасской, ну или ориентир какой-то, если здания уже нет

\iusr{Петр Кузьменко}
\textbf{Alexey Shahan Snigur}, вроде бы Спасская 8. Была баня. Сейчас это часть бизнес - центра.

\iusr{Alexey Shahan Snigur}
\textbf{Петр Кузьменко} спасибо

\end{itemize} % }

\iusr{Ирина Урилова}
На Постышева была чудная баня!

\begin{itemize} % {
\iusr{Dmitriy S. Tucker}
\textbf{Ирина Урилова} 

я туда стричься ходил. Чудные барышни там работали
@igg{fbicon.hearts.two}. Иногда стригли меня в 4 руки
@igg{fbicon.heart.eyes}{repeat=2} 

\iusr{Михаил Мень}
\textbf{Ирина Урилова} я туда с детства с дедушкой ходил @igg{fbicon.heart.red}
\end{itemize} % }

\iusr{Лариса Кушниренко}

Еще была баня возле м. \enquote{.Большевик}, напротив киностудии Довженко.

\begin{itemize} % {
\iusr{Игорь П.}
Думаю ту баню все студенты КПИ семидесятых- восьмидесятых помнят) А в девяностых понеслось -одни банки кругом(

\iusr{Юрий Подмогильный}
И назвали ее после \enquote{реконструкции} БРОКБИЗНЕСБАНЯ.

\iusr{Maryna Bilous}
\textbf{Лариса Кушниренко} я тоже её помню. На ней было написано на украинском языке \enquote{Лазня}.
\end{itemize} % }

\iusr{Татьяна Ховрич}
А ещё - на Краснозвездном проспекте (пр.Лобановского), почти - на Демеевской площади

\iusr{Sofia Shutaya}

Прекрасно помню Галицкие бани на Жилянского. Все детство провела с мамой в
очереди в женское отделение на 2м этаже. Практически все население Евбаза
пользовалось этими банями. Внизу, перед лестницей, была парикмахерская., в зале
отдыха мебель была в белых чехлах, в буфете продавалось пиво и газированная
вода. До сих пор помню запах одеколона и женщин, делающих перманент с
проводами, подключёнными к металлическим цилиндрам.

\iusr{Игорь П.}

Спасибо куму, приобщил к парилке. Я ему рассказывал о прелестях новомодных на
то время (начало девяностых) саун, но Феликс был категоричен - только русская
парилка, и только на Постышева! Да, парилка там была чудесная.

\iusr{Олена Казакова}

А совсем недавно, года 3-4 назад снесли Соломенскую баню на ул. Лыпкивского
(бывш. Урицкого), и на её месте построили очередной небоскреб

\begin{itemize} % {
\iusr{Vladimir Drabkin}
Вот это - беда! Много лет, каждое воскресенье - баня. Да, была компания...
\end{itemize} % }

\iusr{Андрей Крамаренко}

Ходили с мамой на Малую Житомирскую, так как жили на против, в отдельный номер
с ванной. Хотя дома в камуналке была ванная. Почему, не знаю, был еще
маленький. Во взрослом состоянии ходили в общую с пивом как положено.
Колоритная баня... была.

\begin{itemize} % {
\iusr{Ілона Луценко}
\textbf{Андрей Крамаренко} Я тоже с подружками и с пивком посещала - это заведение.

\iusr{Андрей Крамаренко}
\textbf{Ілона Луценко} Привет. Скинь свой телефон. А, то я так и не записал.

\iusr{Ілона Луценко}
\textbf{Андрей Крамаренко} 0631864637;0961314855. А думаю чего не звонишь? @igg{fbicon.face.eyes.star} 
\end{itemize} % }

\iusr{Анатолий Покотило}
А как же Галицкие?!

\iusr{Sofia Shutaya}
\textbf{Anatolii Pokotilo} смотрите выше

\iusr{Геннадий Гейченко}
В \enquote{бубновскую} ходили. И семьёй, а когда мы с братом подросли, с папой.
За счастье было попасть в \enquote{нумер}.

\iusr{Nikolaj Vorobej}
Ходив туди...

\iusr{Марина Зеленкова}
Троицкие бани подержались дольше - мы с подругой студентками ходили как-то где-то 1989/1990 году))

\iusr{Iryna Rudenko}
\textbf{Марина Зеленкова} Мы тоже.

\iusr{Mark Barkan}

Ах, Караваевские бани! Я туда ходил почти каждое воскресение примерно с года
1961 до самого отъезда в 1976 году. Бани открывались в 8 утра, но мы приходили
к 7-ми. Мы - это команда человек 7-10. Давали рабочему рубль, а вход стоил 50
копеек. Парились/ мылись часа 4., а потом по Пушкинской шли в маленький зал
ресторана Украина, на углу Пушкинской и бульвара Шевченко, завтракать.
Последние года 3 до отъезда кто-то из нас должен был идти на Бесарабку, благо
рядом, купить мяса. Ресторан мясо не получал. Но мы приносили с рынка и нам его
на кухне готовили. Зато водки и коньяка было вдоволь. Пишу и вроде как вчера
это было...

\begin{itemize} % {
\iusr{Любовь Алёшина}
\textbf{Mark Barkan} Кучеряво жили!

\iusr{Vladimir Drabkin}
Однако!
\end{itemize} % }

\iusr{Valentina Urban}

Помню в детстве 1958-1959 гг. у нас в коммунальной квартире сломалась газовая
колонка и мы вынуждены были пойти в баню. К моей маме и мне присоединилась наша
бабушка, проживающая на Красноармейской в частном доме.Это были Караваевские
бани на углу Площади Тостого и ул. Пушкинской. Никогда не забуду свой испуг
увидев в бане обнаженных женщин с растительностью на разных частях тела, , мне
было 4 года.

\begin{itemize} % {
\iusr{Лариса Кушниренко}
\textbf{Valentina Urban}

У меня тоже был испуг при первом посещении в 4. Потом привыкла. Для меня
баня-\enquote{всегда праздник}. Очень нравилось, особенно номера. @igg{fbicon.face.happy.two.hands} 

\end{itemize} % }

\iusr{Polina Feldman}

С легким паром всех Вас поздравляю дорогие Киевляне с наступающим годом живите
долго радуйтесь жизнью ведь годы мчатся вот и Мы впомнили про Лазнью это Баня
мне напомнили а то я забыла это слова  @igg{fbicon.mushroom}
@igg{fbicon.fallen.leaf}  @igg{fbicon.sun.behind.rain.cloud}
@igg{fbicon.snowflake} ️@igg{fbicon.droplet} 

\iusr{Polina Zagranichnaya}

Я ходила с мамой каждую не делю в баню на Жилянской. Мы брали отдельный номер и
никакие воришки нам не мешали

\iusr{Петро Гарматюк}

Постійно я ходив в баню біля площі Урицького. На згадаю як зараз.
Пл.Соломенська ?  То був ритуал - за день не розкажеш, але зловживань майже не
було.

\begin{itemize} % {
\iusr{Зоя Лабунская}
\textbf{Петро Гарматюк} о, я туди з Тарасівської 8 трамваєм їздила! Навіть душ Шарко був!
\end{itemize} % }

\iusr{Ed Edward}

Поряд Михайловський провулок 9......у 70-х бувал тут .........коли газова колонка не
працювала.........ще у дворі цього будинку за жіночим відділенням
спостерігали.........)))))

Як це було давно.........жах......))))

\iusr{Татьяна Корнилова}
А на Брест-Литовском?

\iusr{Alla Zgurzhnitsky}
\textbf{Tatiana Kornilova} караваевские

\iusr{Natalia Marevic}

В детстве в начале 50-х и мне пришлось походить в бани. Это были: баня на
Воровского ( теперь ул. Бульварно Кудрявская ) и на Малой Житомирской. В памяти
остались большие залы, мраморные лежаки и тазики.

\iusr{Татьяна Долгополова}
С детства помню походы в баню и жили мы как раз рядом во дворе Бубновских бань
целый ритуал еженедельный @igg{fbicon.face.smiling.eyes.smiling} 


\iusr{Анна Шустерман}
\textbf{Татьяна Долгополова}, а наш чёрный ход тоже выходил во двор бани, значит соседи.)

\iusr{Наталья Кулешова}

Жили в комуналке до 82 года, при отсутствии ванной комнаты, горячей
воды. Поэтому, посещали эту баню регулярно, брали семейный номер. Детей усаживали в
мраморную ванну на львиных ножках, сами мылись под душем. Похоже, номера
использовали не только в санитарных целях-на широких подоконниках, приходилось
убирать использованные презики. Захаживали туда, и вовсе странные
личности. Например, моюсь однажды в общей, намылилась, стою под душем, только глаза
промыла-смотрю, стоит ОНО, в костюмчике, с растегнутыми
брючатами, блаженный, дрочит! Девчата с визгом его быстро мочалками отметелили!
А, банщица сказала, что, он ей частенько приплачивает, чтобы пропустила!
-Мол, ничего, девочки, он же безобидный, да, и, у вас не убудет! Дааа, времечко было
веселое!

\iusr{Ігор Кулаков}
А знаете, почему \enquote{с лёгким паром}?  @igg{fbicon.smile} 

\begin{itemize} % {
\iusr{Irina Fon}
\textbf{Ігор Кулаков} почему?

\iusr{Ігор Кулаков}
\textbf{Irina Fon} есть много версий, напр., о том, что пар бывал тяжёлым для грешников и больных. Пар бывал также и смертельным. Но это если трубу не чистить

\iusr{Irina Fon}
\textbf{Ігор Кулаков} а! Интересно. Я наверное грешник, совершенно не переношу влажный пар.
\end{itemize} % }

\iusr{Татьяна Бондарь}

В 80х годах ходили в \enquote{Троицкие бани}. Не рассказали про баню на
Брест-Литовском, напротив к/студии ии.Довженко. большая бвла баня. Студентами
бегали туда.

\iusr{Алексей Иполитов}
\textbf{Татьяна Бондарь} 

Там сейчас Высший Антикорпуционный суд. А до этого был центральный офи
Брокбизнесбанка. Среди банкиров так и назывался - Брокбизнесбаня

\iusr{Vladimir Kot}
А была ещё баня на Сапёрно Слободском. проезде

\iusr{Rayisa Dmytrenko}
Відвідували.

\iusr{Irina Kolomiec}

До 65 года ходили с мамой в баню на Брест- Литовском, а затем уже стедентими
КПИ, уже било плюс 2 сауни на вибор.

\iusr{Valeria Sedova}

вау

\iusr{Нина Бондаренко}

Ой, пам'ятаю цю будівлю. Оце весь час думаю, чого це у пріснопам'ятні радянські
часи, баням приділяли величезну увагу. І фільми знімали, і будівлі були
найкращі, і потрапити туди могли тільки по-блату \enquote{потрібні люди}. Це так
радянська влада піклувалась про змивання гріхів чи це були її \enquote{культові
заклади}, де обговорювали таємничі партійні проблеми.


\iusr{Наталья Степанова}
Большое спасибо за воспоминания и жизнь киевлян, прекрасные фото истории города!

\iusr{Константин Голубев}
Да. С Парижской коммуны ходили в баню на Постышева. Рядом.

\begin{itemize} % {
\iusr{Катерина Силина}
\textbf{Константин Голубев} Мы с Вами были соседями: и я с Парижской Коммуны 22
\end{itemize} % }

\iusr{Inna Ivanova}

Ходили в баню на М. Житомирской, жили на Б. Житомирской, до тех пор, пока нам
не подключили колонку в 89м или 90м (не помню точно)


\iusr{Светлана Мороз}
жили на шота руставели караваевские любимая баня ходили всей семьей

\iusr{Alexey Shahan Snigur}

Насчёт Московских бань Каплера, Вы ничего не путаете? Они на подоле угол
Ярославской (а не ЯрВал) и Константиновской, здание до сих пор сохранилось во
дворе, там один из корпусов Карпенко- Карого ( неожиданно).

\begin{itemize} % {
\iusr{Yuriy Bubnov}
\textbf{Alexey Shahan Snigur} 

Да, я знаю об этом. Но у автора текста в моей публикации - А. Росовецкого -
написано именно так. Я не могу корректировать чужой текст. Действительно:
\enquote{Каплер содержал бани ближе к Днепру, на Подоле. На угловом участке по
Константиновской и Ярославской улицам, который принадлежал ему с 1899 года,
гражданский инженер Николай Яскевич выстроил для него двухэтажный корпус с
подвалом, в котором были предусмотрены всевозможные варианты мытья.}

\iusr{Alexey Shahan Snigur}
\textbf{Yuriy Bubnov} 

вот тут даже фото имеется, где видна надпись \enquote{бани} на крыше углового дома, в
котором размещалась гостиница Сион. Откуда же инфо про ЯрВал, были ли там бани
вообще, может какие-то другие?

\iusr{Yuriy Bubnov}
\textbf{Alexey Shahan Snigur} Это надо спросить у А. Росовецкого.

\end{itemize} % }

\iusr{Маргарита Мышанская}

Как коренной житель Соломенки смею упомянуть о Соломенских банях
(ул. Урицкого, 38), построенных в 1934 году. Ходили мы туда семьей, с соседями по
дому до 1973 года, пока не получили квартиру с удобствами. Снесены в последние
годы..

\iusr{Татьяна Оржеховская}
Я с родителями ходила в баню на Мало Житомирскую

\iusr{Владимир Зарубин}

В Троицкие бани ходил с отцом. После бани в буфете 150 и пиво отец брал себе а
мне крем соду и плавленый сырок Дружба. Был великолепный книжный киоск, Ильф и
Петров, Ярослав Гашек, Остап Вышня. Прелесть!

\iusr{Leonid Babayev}
Ещё забыли упомянуть баню на Верхнем Валу!

\iusr{Елена Седова}

На Печерске (ул. Московская) была бан, в 1950-1960гг. сюда ходили жители
близлежащих домов. На ней написано бы ло *Лазня*. И детям смешно было.

\begin{itemize} % {
\iusr{Галина Куцая}
\textbf{Елена Седова}

Она так и называлась \enquote{Московские бани} и входила в состав Лысогорского
банно-прачечного комбината, центр которого размещался на ул.
Красноармейской,там где были Троицкие бани. Кстати, автор ошибается, Троицкие
бани функционировали и в 80-х годах. Через них прошло очень много
эвакуированных с Припяти и Чернобыля, здесь они мылись и их переодевали в новую
одежду.


\iusr{Olena Vitvitska}
\textbf{Елена Седова} 

Я тоже туда с подругой ходила по субботам. В конце 70-х - начале 80-х.
Готовились тщательно, маски всякие сами делали и с собой приносили. Хотя зачем
они нам были нужны в том возрасте? Пар был хороший.

\end{itemize} % }

\iusr{Галина Полякова}

Ванна - это ванна, а вот баня - это совершенно иное. Это священнодействие. Папа
ходил в баню с друзьями. Веники готовились заранее и со знанием дела. Вся суть
была в парной. Бани исчезли, а современные сауны папу не привлекали.

\iusr{Елена Насонова}
Были еще бани на Брест Литовском проспекте в аккурат напротив парка Пушкина

\iusr{Alexander Krivoshapka}
Троицкие бани еще вконце 80-х работали, однажды парился в то время

\iusr{Татьяна Иванова}
Спасибо, очень интересно!

\iusr{Валентина Валентина}
Наш балкон был через стенку во дворе!

\begin{itemize} % {
\iusr{Анна Шустерман}
\textbf{Валентина Валентина}, ты не написала, что на Жилянской. @igg{fbicon.face.blowing.kiss} 
\end{itemize} % }

\iusr{Николай Татарчук}

Банька банька - здоровье, общение, ритуал, отдых не сравнить с современными
дорогими почасовыми саунами. Компанией друзей. как правило в суботу, ехали в
баньку до офицыального открытия , чтобы схватить первый парок, потом очередь в
парилку мойка и массаж на больших лежаках, обливание горячей и холодной водой с
тазика... потом по желанию чай, пиво с тараночкой кто соточку и душевный
разговор- дома к обеду с отличным настроением . С легким паром любители
настоящей баньки.

\ifcmt
  ig https://scontent-frt3-2.xx.fbcdn.net/v/t39.30808-6/269702668_3198011190516644_1969758554390333_n.jpg?_nc_cat=110&ccb=1-5&_nc_sid=dbeb18&_nc_ohc=ZSbzjxP_8c4AX-22UOA&_nc_ht=scontent-frt3-2.xx&oh=00_AT9xoOoG4k-u0uL8v7TKw8vkxJUsXzsTuAaGs-JTBF-6yg&oe=61CAC127
  @width 0.4
\fi


\iusr{Людмила Губанова}
Да, было много бань, мы туда ходили. Теперь их не найдёшь.  @igg{fbicon.thumb.down.yellow} 

\iusr{Александр Иванович Лавров}
Когда я жил на Пушкинской, ходил в Караваевские. Иногда там были актёры Леси Украинки. Давно это было.

\iusr{Неман Иванов}
На Урицкого были Соломенские бани.

\iusr{Наталья Василенко}

\ifcmt
  ig https://scontent-frx5-2.xx.fbcdn.net/v/t39.1997-6/p480x480/105941685_953860581742966_1572841152382279834_n.png?_nc_cat=1&ccb=1-5&_nc_sid=0572db&_nc_ohc=l_ypCwa5uwoAX-7eP_W&_nc_ht=scontent-frx5-2.xx&oh=00_AT8Tp9fQ4acsyscZKQFnA0VS3px7WoL_TXIMKiPxJkYjcg&oe=61CBC3CB
  @width 0.2
\fi


\iusr{Елена Петровна}
Завжди гидувала такими закладами..

\iusr{Клитко Галина Федоровна}

Живя на квартире мы тоже посещали баню на Бреслитовском.

\iusr{Tanya Podkova}
Т
Работая в Доме кино посещали Троицкие бани раз в неделю, вплоть до 90-го.

\iusr{Светлана Дубински}

\ifcmt
  ig https://i2.paste.pics/5007182f9f6bec6a6ace8a7865e78776.png
  @width 0.2
\fi

\iusr{Mila Pechenaya}

А я с бабушкой ходила на Подоле возле Днепра.Как давно это было 1960 год
@igg{fbicon.hands.pray}  @igg{fbicon.face.full.moon} 



\end{itemize} % }
