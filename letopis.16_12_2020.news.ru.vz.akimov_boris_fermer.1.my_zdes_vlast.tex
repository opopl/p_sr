% vim: keymap=russian-jcukenwin
%%beginhead 
 
%%file 16_12_2020.news.ru.vz.akimov_boris_fermer.1.my_zdes_vlast
%%parent 16_12_2020
 
%%url https://m.vz.ru/opinions/2020/12/16/1075845.html
 
%%author Акимов, Борис (фермер, кандидат философских наук)
%%author_id akimov_boris_fermer
%%author_url 
 
%%tags 
%%title Мы здесь власть
 
%%endhead 
 
\subsection{Мы здесь власть}
\label{sec:16_12_2020.news.ru.vz.akimov_boris_fermer.1.my_zdes_vlast}
\Purl{https://m.vz.ru/opinions/2020/12/16/1075845.html}
\ifcmt
	author_begin
   author_id akimov_boris_fermer
	author_end
\fi

Нация, пережившая советский XX век, находится в посттравматическом состоянии.
Нас почти сто лет приучали к бездействию и безынициативности. Все решено где-то
там. Решено глупо, нелепо, несправедливо. Но зато решено за нас.

\ifcmt
pic https://img.vz.ru/upimg/soc/soc_1075845.jpg
\fi

В деревне по соседству была добрая традиция. Вынести мусор из дома и положить
его около дороги. Через пару часов собаки мусор растаскивают, и он валяется по
всей улице. Старая добрая традиция всегда имела логичное объяснение: местные
власти не ставят контейнеры, а до ближайшей помойки мусор возить далеко (минут
пять). Эта версия казалась жителям невероятно убедительной. Нельзя сказать, что
замусоренные улицы были им дороги и симпатичны, но логика казалось неоспоримой.
Кто, если не другие, должны решить проблемы наших замусоренных улиц?

Нация, пережившая советский XX век, находится в посттравматическом состоянии.
Нас почти сто лет приучали к бездействию и безынициативности. Все решено где-то
там. Решено глупо, нелепо, несправедливо. Но зато решено за нас. Теперь можно
сидеть на кухне и материть этих самых, решивших все так криво и косо. В этом
смысле мы очень свободны. Именно свобода является чертой этой самой посттравмы.
Свобода от ответственности. Тех, кто брал, делал и не боялся ответственности, в
1917-м, а потом в 1929-м, 1937-м и так далее – упорно вырезали, высылали,
загоняли в подполье. Когда режим перестал быть кровожадным, безответственность
превратилась в модель устойчивого поведения, в удобную модель. Всегда кто-то
еще должен, только не я. 

Все более-менее слышали словосочетания «земский врач» или «земская школа». Но
почти никто не ассоциирует эти слова с проявлением личной ответственности
людей, живущих на одной территории. Ответственности местного сообщества за себя
и вверенную ему просто по факту проживания территорию. А было буквально так:
некоторые сознательные граждане брали на себя ответственность, сами облагали
себя добровольными сборами (налогами). То есть просто скидывались и содержали
школы, больницы, строили дороги и убирали мусор. Они не ждали от царского
правительства врачей и не сетовали на то, «куда же идут наши налоги». Они
действовали. Они самостоятельно и по доброй воле ограничивали свою свободу
ответственностью. И тем самым брали власть в свои руки.

Цифры земской деятельности пока кажутся современному российскому обществу,
склонному к диванной критике, чем-то из сферы унылой и пыльной статистики, не
имеющей за собой ничего героического и притягательного. Между тем можно себе
только представить, сколько тысяч примеров деятельных людей за всеми этими
цифрами стоит. В 1913 году в земских школах обучалось почти два миллиона детей.
Из 44 600 школ, существовавших в стране, более половины были земскими. Еще один
пример – библиотеки. В 1864 году в земских губерниях (там, где ввели земства)
России было 152 библиотеки, из которых 84 работали в столицах. За 50 лет их
число возросло до 12 627. Читательская аудитория составляла около двух
миллионов человек.

Земские врачи – воспетые Булгаковым и прочей литературой начала XX века –
главный, наверное, символ нового местного ответственного бытия. Один из моих
прадедов был земским врачом. По всем семейным легендам – считался главным
героем местности. К 1910 году в России было 3012 земских врачей. Всего же почти
40 тысяч человек медицинского персонала – врачи, аптекари, фельдшеры,
провизоры, заведующие санитарными и ветеринарными бюро и т. д. – работало в
органах земской медицины.

Такая бурная добровольная деятельность людей, взявших власть (то есть
деятельную ответственность) на себя, поражает. Еще более удивительно для
большинства наших современников то, что участие в этом акте гражданской
ответственности за сообщество и территорию касалось только тех, кто сам этого
захотел. Не хочешь ответственности – не будет у тебя её. Будешь сидеть на кухне
и ругаться. И были и такие. Но тех, кто был готов действовать и браться,
хватало на то, чтобы все эти сотни больниц, школ, дорог построить и содержать. 

По мне, гражданское общество – это общество тех, кто готов брать
ответственность на себя. Общество тех, кто готов действовать, не переваливая
(пусть и с праведным гневом) ответственность на других (жена, дворники, соседи,
чиновники и т. д). Сначала спросим с себя, а потом с окружающих.

«Кто здесь власть? Мы здесь власть!» Мне, при всем моем консерватизме и
антипатии к революционной и разрушительной риторике, нравится этот лозунг.
Власть – это в первую очередь ответственность. Во всяком случае, так должно
быть. Русские люди XIX века, пошедшие в земства и взявшие ответственность за
скучные дороги и больницы на себя, стали реальной властью. Они творили историю
своей земли, действовали и превращались в ту самую честную власть благодаря
своей решительной ответственности за окружающий мир (квартал, деревню, район).

В деревне по соседству совсем недавно случилась маленькая мирная революция.
Группа граждан вышла на улицу, вычистила ее от мусора и разбила на месте бывшей
помойки сад с туями. Вот уже три месяца ни один пакет не был выброшен в
новообразованный садик. Маленькое гражданское общество в деревне по соседству
решило за день проблему, о которой на кухне ругались многие годы. В другой
деревне по соседству местное гражданское общество самостоятельно построило
дорогу. А в третьей деревне чуть подальше умудрились сами организовать школу.

Кто здесь власть?

