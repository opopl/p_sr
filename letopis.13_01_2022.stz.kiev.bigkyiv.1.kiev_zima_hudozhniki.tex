% vim: keymap=russian-jcukenwin
%%beginhead 
 
%%file 13_01_2022.stz.kiev.bigkyiv.1.kiev_zima_hudozhniki
%%parent 13_01_2022
 
%%url https://bigkyiv.com.ua/kyyiv-zymovyj-ochyma-hudozhnykiv
 
%%author_id smolina_miroslava
%%date 
 
%%tags gorod,hudozhnik,isskustvo,kartina,kiev
%%title Поділ, Лавра та Печерська ковзанка: Київ зимовий очима художників
 
%%endhead 
 
\subsection{Поділ, Лавра та Печерська ковзанка: Київ зимовий очима художників}
\label{sec:13_01_2022.stz.kiev.bigkyiv.1.kiev_zima_hudozhniki}
 
\Purl{https://bigkyiv.com.ua/kyyiv-zymovyj-ochyma-hudozhnykiv}
\ifcmt
 author_begin
   author_id smolina_miroslava
 author_end
\fi

Щоб побачити по-справжньому зимовий київський пейзаж, зовсім не обов’язково
благати природу про щедрий снігопад. Навіть з дому можна не виходити.
Українські художники закарбували на своїх полотнах і засніжений Поділ, і зимову
Володимирську гірку, й тиху зиму київських околиць.

\ii{13_01_2022.stz.kiev.bigkyiv.1.kiev_zima_hudozhniki.pic.1}

Вільний учень Петербурзької академії мистецтв та добрий приятель Тараса
Шевченка Михайло Сажин прожив у Києві майже 10 років. Двоє друзів-художників
мріяли замалювати всі визначні місця міста, але реалізувати цю ідею завадив
арешт Шевченка. Та дещо в свій київський період Сажин все-таки встиг. Зокрема,
видати альбом \enquote{Види Києва}, до якого увійшли сепієві та акварельні
роботи, та написати олією картину з зимовим видом на Володимирську гірку.
