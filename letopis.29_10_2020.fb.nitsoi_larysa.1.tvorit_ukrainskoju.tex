% vim: keymap=russian-jcukenwin
%%beginhead 
 
%%file 29_10_2020.fb.nitsoi_larysa.1.tvorit_ukrainskoju
%%parent 29_10_2020
 
%%url https://www.facebook.com/larysa.nitsoi/posts/3732078833490888
 
%%author Ніцой, Лариса
%%author_id nitsoi_larysa
%%author_url 
 
%%tags 
%%title Творіть українською, будьте українцями.
 
%%endhead 
 
\subsection{Творіть українською, будьте українцями}
\label{sec:29_10_2020.fb.nitsoi_larysa.1.tvorit_ukrainskoju}
\Purl{https://www.facebook.com/larysa.nitsoi/posts/3732078833490888}
\Pauthor{Ніцой, Лариса}

Українські митці в Україні, які творять своє мистецтво російською мовою – творять російську культуру, не українську. 

Мене запитала журналістка, чи якісне у нас українське кіно, з огляду на україномовні фільми, які стали з’являтися нещодавно.

Я запитала у відповідь, а до російських фільмів у вас виникло питання, чи вони
якісні? Ні, таких питань до російськомовних фільмів не виникало, лише до
україномовних.

Але ж ці РОСІЙСЬКІ фільми російською мовою зняті в Україні, тими самими
українцями, що знімають тепер в Україні УКРАЇНСЬКЕ кіно українською мовою. Це
ті самі люди, і ті самі вміння, і такі ж фільми, лише мови різні. 

Одначе, наші фільми, виконані російською мовою МИ АВТОМАТИЧНО віднесли до
російського кіно, отже, до російської культури! Бо так воно й працює у світі:
фільм, знятий німецькою – ви назвете німецьким кіном, фільм, знятий французькою
мовою – ви назвете французьким кіном. Фільм, знятий російською мовою - ви
називаєте російським кіном, навіть якщо він знятий українцями в Україні. 

Раніше до цих російськомовних фільмів у вас не виникало питання, чи воно
якісне. Бо воно якісне. Отже, й українською мовою кіно якісне. Крапка.

Як і крапка в тому, що українські митці, які творять мистецтво російською -
АВТОМАТИЧНО відносять своє мистецтво до російської культури. 

Творіть українською, будьте українцями.
