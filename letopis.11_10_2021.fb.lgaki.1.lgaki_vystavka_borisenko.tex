% vim: keymap=russian-jcukenwin
%%beginhead 
 
%%file 11_10_2021.fb.lgaki.1.lgaki_vystavka_borisenko
%%parent 11_10_2021
 
%%url https://www.facebook.com/AkademiyaMatusovskogo/posts/4678615988869146
 
%%author_id lgaki
%%date 
 
%%tags donbass,hudozhnik,isskustvo,kartina,kultura,lgaki,lnr,lugansk,vystavka
%%title Сюрреалистичные «Отражения» Павла Борисенко: в Академии Матусовского — новая выставка
 
%%endhead 
 
\subsection{Сюрреалистичные «Отражения» Павла Борисенко: в Академии Матусовского — новая выставка}
\label{sec:11_10_2021.fb.lgaki.1.lgaki_vystavka_borisenko}
 
\Purl{https://www.facebook.com/AkademiyaMatusovskogo/posts/4678615988869146}
\ifcmt
 author_begin
   author_id lgaki
 author_end
\fi

Сюрреалистичные «Отражения» Павла Борисенко: в Академии Матусовского — новая
выставка

Заслуженный художник ЛНР, доцент кафедры станковой живописи Академии
Матусовского Павел Борисенко представил нетипичные для себя работы.

\ifcmt
  tab_begin cols=3

     pic https://scontent-mxp1-1.xx.fbcdn.net/v/t1.6435-9/245280657_4678615152202563_7214517853890473528_n.jpg?_nc_cat=105&ccb=1-5&_nc_sid=730e14&_nc_ohc=6ZvzLr59yA8AX9fJxCc&_nc_ht=scontent-mxp1-1.xx&oh=98e58d9e8333aeb915d21473b8325417&oe=61AFEB3B

     pic https://scontent-mxp1-1.xx.fbcdn.net/v/t1.6435-9/245209177_4678615148869230_5626939063726524633_n.jpg?_nc_cat=100&ccb=1-5&_nc_sid=730e14&_nc_ohc=3i2OtohiHb8AX9e9HB8&_nc_ht=scontent-mxp1-1.xx&oh=88559cf493c8f1c075226b179359eeb0&oe=61AFE0AE

     pic https://scontent-mxp1-1.xx.fbcdn.net/v/t1.6435-9/245131799_4678615275535884_1108173192740833829_n.jpg?_nc_cat=103&ccb=1-5&_nc_sid=730e14&_nc_ohc=Wv1T_NCYDn4AX_4QnCL&_nc_ht=scontent-mxp1-1.xx&oh=3a40c3a23f3a8cc2801ab253653cee39&oe=61B0436D


  tab_end
\fi

— Надоел формализм. Я почувствовал, что в нем достиг своей вершины, стал
повторяться. Надо идти дальше, — так объяснил свое обращение к сюрреализму,
символизму сам автор. — Надоест это, найдется что-то еще. Главное – не
останавливаться.

\ifcmt
  tab_begin cols=3

     pic https://scontent-mxp1-1.xx.fbcdn.net/v/t1.6435-9/245084001_4678615168869228_5357688638636204810_n.jpg?_nc_cat=109&ccb=1-5&_nc_sid=730e14&_nc_ohc=r8fwPQWMCxwAX932sZA&tn=lCYVFeHcTIAFcAzi&_nc_ht=scontent-mxp1-1.xx&oh=52e6da91ca7bfa9743160110f0746b19&oe=61AFE5BE

     pic https://scontent-mxp1-1.xx.fbcdn.net/v/t1.6435-9/245388104_4678615158869229_8886887099314284331_n.jpg?_nc_cat=109&ccb=1-5&_nc_sid=730e14&_nc_ohc=Q53i6f3U0nUAX_iVBsC&_nc_oc=AQmkZWgmYCEoYj7ynJhwA0PbZrpvoWy12a0TTbKmOcwY2YQ1mkEk_3427a0C2WdK7zU&_nc_ht=scontent-mxp1-1.xx&oh=3830efba38d63a2603ba0d7c5ac7acb3&oe=61AF922F


		 pic https://scontent-mxp1-1.xx.fbcdn.net/v/t1.6435-9/245436027_4678615438869201_3447793256841774827_n.jpg?_nc_cat=105&ccb=1-5&_nc_sid=730e14&_nc_ohc=N3IZguuIvNAAX9qQo6Y&_nc_ht=scontent-mxp1-1.xx&oh=17ce40cefff4b46245fc0ed54b6e6ca0&oe=61AF811B

  tab_end
\fi

Все картины, представленные в холле корпуса вуза на Красной площади, 7,
написаны Павлом Борисенко в последние годы. Большинство выставляется впервые.
Посмотреть их можно до конца октября.

— В течение всего учебного года раз в месяц будем открывать новую выставку, на
которых педагоги факультета покажут свои работы, — рассказала декан факультета
изобразительного и декоративно-прикладного искусства Академии Наталья Феденко.

Так что мы тоже останавливаться не планируем.

Фото – Марина Машевски

\ifcmt
  tab_begin cols=3

		 pic https://scontent-mxp1-1.xx.fbcdn.net/v/t1.6435-9/245269719_4678615332202545_107023007621095833_n.jpg?_nc_cat=100&ccb=1-5&_nc_sid=730e14&_nc_ohc=PQGyUkwIdEMAX8oDsvD&tn=lCYVFeHcTIAFcAzi&_nc_ht=scontent-mxp1-1.xx&oh=7dfbb96e20cd948e4801f39169baa3db&oe=61AF1F62

     pic https://scontent-mxp1-1.xx.fbcdn.net/v/t1.6435-9/245232618_4678615328869212_4401233495359483454_n.jpg?_nc_cat=110&ccb=1-5&_nc_sid=730e14&_nc_ohc=luQupbGrPfUAX-QRs0b&_nc_ht=scontent-mxp1-1.xx&oh=3c28fdb81ba84c0fa796abd09585ee37&oe=61AFE30A

  tab_end
\fi
