% vim: keymap=russian-jcukenwin
%%beginhead 
 
%%file 09_10_2021.fb.fb_group.story_kiev_ua.1.zanjkoveckoj_8
%%parent 09_10_2021
 
%%url https://www.facebook.com/groups/story.kiev.ua/posts/1770776166452510/
 
%%author_id fb_group.story_kiev_ua,borozenec_anatolij.kiev
%%date 
 
%%tags gorod,kiev
%%title ЗАНЬКОВЕЦЬКОЇ, 8
 
%%endhead 
 
\subsection{ЗАНЬКОВЕЦЬКОЇ, 8}
\label{sec:09_10_2021.fb.fb_group.story_kiev_ua.1.zanjkoveckoj_8}
 
\Purl{https://www.facebook.com/groups/story.kiev.ua/posts/1770776166452510/}
\ifcmt
 author_begin
   author_id fb_group.story_kiev_ua,borozenec_anatolij.kiev
 author_end
\fi

ЗАНЬКОВЕЦЬКОЇ,8

У моєму попередньому дописі мова йшла про будівлю колишнього театру на
вул.Заньковецької,8. Думаю, що доречно буде дізнатися що сталося після того, як
будівлю знесли. 

На його місці не стали зводити нового будинку. Хіба що, за рахунок звільненого
місця, трохи збільшили корпуси двох сусідніх будинків. А 9-поверховий житловий
будинок-сталінку звели в 1952-му позаду знесеного театру, значно відступивши
вглиб від червоної лінії забудови вулиці (архітектори Л.Каток, В.Гопкало,
А.Лобода). 

У спадок від будівлі театру сталінка отримала не тільки його 8-й номер.
Нагадаю, що театр був спалений, перебуваючи   у статусі Театру Київського
військового округу. Так от, випадково чи ні, але сталінку збудували для
проживання сімей працівників ...саме Київського військового округу. У колишній
підвідомчий театр вони, звісно, піти вже не могли. А от посидіти у скверику,
облаштованому на його місці, та подумати навіщо колеги його спалили у вересні
1941-го, мали можливість.   

\begin{multicols}{2}
\ii{09_10_2021.fb.fb_group.story_kiev_ua.1.zanjkoveckoj_8.pic.1}
\ii{09_10_2021.fb.fb_group.story_kiev_ua.1.zanjkoveckoj_8.pic.2}
\ii{09_10_2021.fb.fb_group.story_kiev_ua.1.zanjkoveckoj_8.pic.3}
\ii{09_10_2021.fb.fb_group.story_kiev_ua.1.zanjkoveckoj_8.pic.4}
\end{multicols}

Ще однією особливості цього будинку, як на мене, є те, що до вул. Заньковецької
він розміщений не самою привабливою своєю стороною, а тилом. Найбільш
презентабельною – фасадом – будинок дивиться у бік Хрещатика. Припускаю, що
спочатку планувалося, що він буде приписаний до вул. Хрещатик. На цю думку
натякає напис на проектному зображені будинку про його будівництво на
майданчику №9 вул. Хрещатик. До центральної вулиці столиці від будинку можна
було спуститися сходами. Спочатку це виглядало ефектно. Вид  будинку на схилі,
зі сходами до головної вулиці столиці чимось композиційно навіть нагадує мені
Будинок із зіркою (Хрещатик, 25). Однак, пізніше станція метро «Хрещатик» з
рестораном перекрила вигідну для сприйняття будинку картинку. На жаль, з
центральної частини Хрещатика будинок став майже невидимий. Можливо, що саме ця
обставина й обумовила його реєстрацію по вул. Заньковецької.

Чи можна було цього уникнути? Я не архітектор, але думаю, що можливість така
була. Треба було вестибюль входу на станцію метро опустити під землю. Або,
принаймні, збудувати лише одноповерховий вестибюль. Без ресторану з вар'єте
саме на цьому місці можна було якось і обійтися.

\ii{09_10_2021.fb.fb_group.story_kiev_ua.1.zanjkoveckoj_8.cmt}
