% vim: keymap=russian-jcukenwin
%%beginhead 
 
%%file 18_12_2021.stz.news.lnr.lug_info.1.novyj_god_teatr
%%parent 18_12_2021
 
%%url https://lug-info.com/news/luganskij-russkij-teatr-predstavil-novogodnee-predstavlenie-dvenadcat-mesacev
 
%%author_id 
%%date 
 
%%tags 
%%title Новогоднее представление "Двенадцать месяцев"
 
%%endhead 
\subsection{Новогоднее представление \enquote{Двенадцать месяцев}}
\label{sec:18_12_2021.stz.news.lnr.lug_info.1.novyj_god_teatr}

\Purl{https://lug-info.com/news/luganskij-russkij-teatr-predstavil-novogodnee-predstavlenie-dvenadcat-mesacev}

\headCenter{Луганский русский театр представил новогоднее представление \enquote{Двенадцать месяцев}.}

Луганский академический русский драматический театр имени Павла Луспекаева
представил новогоднее представление \enquote{Двенадцать месяцев} по мотивам сказки
Самуила Маршака. Об этом с места события передает корреспондент ЛИЦ.

\ii{18_12_2021.stz.news.lnr.lug_info.1.novyj_god_teatr.pic.1}

Постановщиком спектакля выступила заслуженная артистка ЛНР Светлана Степкина.

\enquote{Это Маршак, это красивый литературный язык, и это, все-таки, классика! А мы –
театр республиканский, у нас главная елка страны, и это удачный выбор}, -
пояснила она выбор драматургического материала.

\ii{18_12_2021.stz.news.lnr.lug_info.1.novyj_god_teatr.pic.2}

Представление наполнено ярким музыкальным и световым оформлением,
хореографическими номерами и мультимедийными эффектами. Несмотря на отсутствие
традиционной игровой программы вокруг елки, на представление детей все равно
приглашают Дед Мороз и Снегурочка.

\ii{18_12_2021.stz.news.lnr.lug_info.1.novyj_god_teatr.pic.3}

\enquote{Главные герои нашей сказки юному зрителю рассказывают о том, что добро все
равно победит, а жадность и злость будут наказаны рано или поздно. Мне как
режиссеру всегда хочется, чтобы сказка чему-то учила, и ведь не зря были
придуманы нашими предками былины и легенды – обучить будущее поколение}, -
сказала режиссер.

\ii{18_12_2021.stz.news.lnr.lug_info.1.novyj_god_teatr.pic.4}

Зрители первого утреннего сеанса оживленно комментировали происходящее на
сцене, общаясь с персонажами, эмоционально реагировали на появляющихся лесных
животных, прогоняли злого волка, жалели падчерицу и вместе с ней со страхом
кричали, когда от зимней спячки с ревом проснулся медведь.

Показы сказки продлятся до 14 января. На данный момент запланировано 39
представлений, но зрители продолжают заказывать билеты, поэтому могут появиться
дополнительные сеансы.

\enquote{И это замечательно, значит, у людей новогоднее настроение, и я думаю, что чем
больше людей мы порадуем, тем легче будет им жить на следующий год. Поэтому: не
сидеть дома, всем идти на сказку, поднимать настроение. Вот вчера мы сдали
(художественному совету) сказку, Дед Мороз ударил посохом и в Луганске пошел
снег. Поэтому наша сказка чудесная и настоящая, она подарит вам ощущение
радости и счастья!} - добавила Степкина.

Луганский академический русский драматический театр имени Луспекаева
располагается по адресу: Луганск, улица Коцюбинского, 9а. Дополнительную
информацию можно получить по телефонам: (0642) 50 20 31, (0642) 50 11 43, на
сайте театра, а также в его аккаунтах в социальных сетях "ВКонтакте" и
Instagram. 
