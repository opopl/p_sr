% vim: keymap=russian-jcukenwin
%%beginhead 
 
%%file slova.balet
%%parent slova
 
%%url 
 
%%author 
%%author_id 
%%author_url 
 
%%tags 
%%title 
 
%%endhead 
\chapter{Балет}
\label{sec:slova.balet}

%%%cit
%%%cit_head
%%%cit_pic
%%%cit_text
Повар с тяжелым русским акцентом ликует о том, что это будет лучший борщ в его
карьере, вдруг запевая \enquote{Volga Batman} на мотив знаменитой \enquote{Эй, ухнем} Федора
Шаляпина – в английском переводе эта песня называется The Volga Boatman, соль
шутки очевидна. Битва за свободу происходит под \enquote{Матросский танец} из \emph{балета}
Глиэра \enquote{Красный мак}, более известный как \enquote{Яблочко}, пока казак-повар
продолжает крошить свеклу в гигантский казан.  Учитывая данное очень непростое
обстоятельство даже не знаю, стоит ли настаивать на чисто украинской
принадлежности борща. Ведь его очевидно придется декоммунизировать и варить в
каком-нибудь другом цвете
%%%cit_comment
%%%cit_title
\citTitle{Прежде чем бороться за борщ, его бы надо декоммунизировать}, 
Дмитрий Заборин, strana.ua, 21.06.2021
%%%endcit

