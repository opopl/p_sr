%%beginhead 
 
%%file 26_08_2021.fb.arximisto.1.vorovstvo_ogradok_staroje_gorod_kladbische
%%parent 26_08_2021
 
%%url https://www.facebook.com/arximisto/posts/pfbid02pPPu7ibh32GzJ8r15nhL7KrJoKeuPMRsEQcXC3Z6LhnMe6MxgTGeTD9z3XHpvWXMl
 
%%author_id arximisto
%%date 26_08_2021
 
%%tags 
%%title Волонтеры зафиксировали воровство оградок на Старом городском кладбище
 
%%endhead 

\subsection{Волонтеры зафиксировали воровство оградок на Старом городском кладбище}
\label{sec:26_08_2021.fb.arximisto.1.vorovstvo_ogradok_staroje_gorod_kladbische}

\Purl{https://www.facebook.com/arximisto/posts/pfbid02pPPu7ibh32GzJ8r15nhL7KrJoKeuPMRsEQcXC3Z6LhnMe6MxgTGeTD9z3XHpvWXMl}
\ifcmt
 author_begin
   author_id arximisto
 author_end
\fi

Волонтеры зафиксировали воровство оградок на Старом городском кладбище

Вчера один из волонтеров по восстановлению Некрополя стал свидетелем воровства
оградок на Старом городском кладбище.

\#новости\_архи\_города

По словам волонтера Д., около 16:10 25 августа он увидел, как трое мужчин
вытащили из зарослей на кладбище (со стороны Новоселовки) массивную ограду и
погрузили ее в \enquote{Газель}. Он успел сфотографировать саму машину. Ее номер –
АН9695НМ.  На капоте – логотип службы такси и перевозок Maxim 27 77 (см. фото).
Водитель с двумя грузчиками уехали, а четвертый мужчина ушел в Новоселовку. 

К сожалению, волонтер не смог задержать воров, поскольку был с ребенком...

Мы обращаемся ко всем с просьбой выяснить принадлежность этой машины и
причастность упомянутой службы такси к разворовыванию кладбищенских оград, как
заявил Андрей Марусов, директор общественной организации \enquote{Архи-Город},
координирующей волонтерское движение по восстановлению Некрополя. 

Мы не разглашаем имя волонтера из соображений безопасности, но готовы
предоставить его контакты, оригиналы фото и свидетельства правоохранительным
органам.

В этом году кладбищенские мародеры стали настоящим бичом, как отмечает
А. Марусов. Из-за ливней многие участки кладбища превратились в джунгли. Этим
вовсю пользуются мародеры Неделю назад мы обнаружили, что они разломали ограду
безымянного захоронения на древнем \enquote{Хараджаевском} участке. В прошлом году мы
покрасили ее в черный цвет, придали ей ухоженный вид, надеясь тем самым уберечь
от разграбления. Наши усилия оказались тщетными... 

По мнению волонтеров, огораживание забором территории кладбища, усиленное
патрулирование, видео наблюдение по периметру, сторож, расчистка участков
кладбища – любая из этих мер могла бы значительно снизить риск разграбления
кладбища.

Контакты \enquote{Архи-Города} - 096 463 69 88, arximisto@gmail.com

Вниманию @MRPL.POLICE
