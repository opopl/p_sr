% vim: keymap=russian-jcukenwin
%%beginhead 
 
%%file 04_12_2020..ljubka_andrii.1.dym_nad_karabahom
%%parent 04_12_2020
 
%%url https://day.kyiv.ua/uk/blog/suspilstvo/dym-nad-karabahom
 
%%author Любка, Андрій
%%author_id ljubka_andrii
%%author_url 
 
%%tags 
%%title Дим над Карабахом
 
%%endhead 
 
\subsection{Дим над Карабахом}
\label{sec:04_12_2020..ljubka_andrii.1.dym_nad_karabahom}
\Purl{https://day.kyiv.ua/uk/blog/suspilstvo/dym-nad-karabahom}
\ifcmt
	author_begin
   author_id ljubka_andrii
	author_end
\fi

Війна в Нагірному Карабасі спонукала мене пильніше придивитися до тамтешніх
подій, історії конфлікту і геополітичного контексту. Чому? Відповідь проста: бо
я просто не розумів, кого із воюючих сторін підтримую.

З одного боку, Азербайджан воює за свою міжнародно визнану територію, тобто
виганяє окупантів, а Вірменія – проросійська держава, яка не підтримує Україну
в ООН. З іншого боку, історично ця територія – Арцах – була заселена вірменами,
і тільки з появою Совєтського Союзу цей регіон включено в склад Азербайджану;
до того ж Вірменія – це християнська країна, і цю війну можна розглядати також
і в світлі гантінґтонівської ідеї про конфлікт цивілізацій, бачити тут сутичку
хреста й півмісяця.

Коли цей воєнний конфлікт увійшов у нову гарячу фазу, мені чомусь пригадалася
прогулянка Єрусалимом. Три роки тому якраз у ці дні я гуляв цим містом і
розглядав його первісну структуру. Тому й пригадав, що воно поділено на 4
квартали: єврейський, християнський, мусульманський і вірменський. Цей поділ
красномовно підтверджує древність і впливовість вірменського народу, безумовно
– одного з найважливіших в історії останніх двох тисячоліть. Що там казати:
вірмени мали власну писемність ще в ті століття, коли в Європі й не пахло
слов’янами! А дослідники Біблії стверджують, що це саме до вершечка вірменської
гори Арарат причалив Ноїв ковчег…

Але історія була нещадною до вірменів – і тепер Арарат, наприклад, уже не
вірменський, бо знаходиться на території Туреччини. Протягом століть вірмени
зазнавали жахливих утисків і гонінь, квінтесенцією яких стала велика різанина
1915-го року, коли в Османській імперії було здійснено геноцид вірмен. Тоді
було винищено понад 1,5 мільйонів вірмен, а ці вбивства, гоніння й депортації
стали для турків також способом «етнічної чистки» територій східної Туреччини,
де історично проживали вірмени. Сучасна Вірменія – це лише клаустрофобічний
окраєць вірменських просторів. У цьому контексті й не дивно, що значно більше
вірмен живе в діаспорі по всьому світу, ніж у самій Вірменії. Такі реалії
народу з великою історією, який натерпівся лиха…

Щоб трохи краще зрозуміти вірмен, їхню ментальність і світогляд, я повернувся
до книжки, яка мене надзвичайно вразила кілька років тому. Мова про «Книгу
шепотів» Варужана Восґаняна; цей роман отримав Центральноєвропейську
літературну премію «Angelus» і саме в польському перекладі я його прочитав. Як
не дивно, оригінал написано румунською, бо пан Восґанян – нащадок вірменів, які
сто років тому втекли з Османської імперії, рятуючись від геноциду. З того часу
вони осіли в Румунії, а сам Варужан навіть зробив блискучу кар’єру і став свого
часу румунським міністром фінансів. Парадоксально, але фінансист і політик зміг
написати без перебільшень видатну книгу, яка допомагає зрозуміти його народ і
пройнятися до вірмен щирою симпатією.

Ця симпатія мимоволі переростає у співчуття. Бо неможливо зі спокійним серцем
дивитися на те, як вірмени перед від’їздом спалюють свої домівки, щоб вони не
дісталися ворогу. Дим над Карабахом – пекучий і страшний…

І можна скільки завгодно захоплюватися успіхами азербайджанської армії,
проглядаючи в цих подіях можливу модель для звільнення українських окупованих
територій, але цей дим над Карабахом і колони автомобілів, що вивозять людей –
не дають спокою. Коней і шаблі замінили безпілотники й автомати, але суть
історичних подій залишається та сама.

Мій висновок простий: добре, що я не мушу нікого підтримувати в цьому
конфлікті. Бо розум каже одне, а серце співчуває іншим.

