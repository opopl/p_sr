% vim: keymap=russian-jcukenwin
%%beginhead 
 
%%file 25_01_2022.yz.maj_dnr.1.gradusnik
%%parent 25_01_2022
 
%%url https://zen.yandex.ru/media/id/5f8f226b1fe36c1d9e02a36b/gradusnik-mojet-lopnut-61eec8834bdf480e9f9b4a0b
 
%%author_id yz.maj_dnr
%%date 
 
%%tags rossia,ugroza,usa,vojna
%%title Градусник может лопнуть
 
%%endhead 
 
\subsection{Градусник может лопнуть}
\label{sec:25_01_2022.yz.maj_dnr.1.gradusnik}
 
\Purl{https://zen.yandex.ru/media/id/5f8f226b1fe36c1d9e02a36b/gradusnik-mojet-lopnut-61eec8834bdf480e9f9b4a0b}
\ifcmt
 author_begin
   author_id yz.maj_dnr
 author_end
\fi

Градус отношений между Россией и Западом никогда не был комфортным, разве что в
то время, когда Горбачев, а за ним и Ельцин со своей командой сдавали все и вся
США и других бенефициарам упадка России. Температура росла скачками, не смотря
на титанические усилия МИД РФ (не только их, понятно, но первые удары принимали
и отражали именно они) воззвать к голосу разума американских и европейских
политиков. А сегодня накал «страстей» достиг такого уровня, что политический
«градусник» начал плавиться. Красные линии, говорите? Боюсь, что они уже
нарисованы жирным фломастером, независимо от того, перейдут ли стороны эти
красные линии, информационные «ястребы» сделали свое дело не только в
политической сфере, но и в цивилизационной.

\ii{25_01_2022.yz.maj_dnr.1.gradusnik.pic.1}

Я говорю о том, что в последнее время идеологический разрыв стал пропастью, и
сейчас мы имеем гораздо более острую фазу информационной битвы в отношениях
между Россией и Западом, чем даже во времена так называемой Холодной войны.
Конечно, прежде всего я говорю о США и Британии. И теперь вряд ли важно, кто
выстрелит первым, Россию обвинят в любом случае. «Ты виноват уж в том, что
хочется мне кушать», - так кажется? И уже не так важно, как именно будут
развиваться события.

Другой вопрос: осознали ли мы это? Готовы ли к глобальному перелому или нет. Я
не о военной сфере, я о другой – духовной. Потребность мира и покоя в нас
непреодолима, отсутствие привычки к зависти в наших генах, да и трудолюбие
тоже, а потому нам сложно понять тех же британцев, привыкших жить и богатеть за
счет своих колоний. Для нас немыслимой кажется сама идея смотреть в сторону с
намерением что-то забрать у других, будь то земля с ее ископаемыми или люди,
которых нужно сделать своими рабами. Мы – патриоты не воспитанные, а рожденные.

Оговорюсь: я не имею в виду либеральное сообщество, которое на каждом углу
признается, как ему стыдно быть русским, я говорю об обычных жителях
России-матушки, которые сейчас очень надеются на то, что ничего не произойдет
хотя бы потому, что нас много, и мы не понимаем, за что нас можно не любить.

Так вот, мы до сих пор не поняли, что реальность изменилась, мы до сих пор
верим, что либерально-западническое непотребство, которое мы видим в
образовании, культуре, в огромном количестве СМИ – это случайность, типа,
«заигрались мальчики». Но все будет хорошо, ничего такого-этакого не
происходит, одумаются, исправятся и все такое. Мы до сих пор пребываем в
твердой уверенности, что все еще можно изменить и вообще, плохое к нам не
пристанет, радужные флаги и дети в гей-парах не приживутся, и вообще, «при чем
здесь идеология»?

Но вот сейчас, именно сейчас, когда Россия решила сказать Западу «Хватит»,
когда разрыв стал слишком очевиден, время задуматься о том, нудны ли нам
западные стандарты и правила, согласны ли мы жить по их протоколам, к чему нас,
собственно, так долго приучали, но не смогли довести дело до конца, захотим ли
считать Родиной ту страну, где сытнее, или будет делать тучной свою
собственную.

Хотелось бы еще напомнить, что двигаясь в направлении либерально-глобалистской
парадигмы, мы окажемся в мире, где все уже занято, а для нас места нет, разве
только на задворках цивилизации, как у всех, кто потерял свою идентичность. И
ведь нам это уже показали! Вернее, мы сами себе это доказали в 90-х, когда наши
Горбачевы с Ельциными в десна целовались в Западом. Ведь они (западные лидеры)
праздновали наше поражение! Они так и говорили: «Россия пала», да и вели себя
не как с друзьями, а как с побежденными. Да, они расширяли НАТО на восток, но
делали это только для того, чтобы закрепить свою победу и не дать нам
отыграться. Он наступали нам на голые пятки, нарушая собственные обещания и
обязательства, данные в обмен на капитуляцию. Но разве колхозник, почесывая по
холке свою свинью, кормит ее не для своего будущего обеда?

Вот только обиды и возмущения тут ни к чему – Запад так поступает всегда.
Вспомните хотя бы стеклянные бусики для аборигенов Америки или Австралии....

Незавидной была бы наша общая судьба, если бы не пришел Путин. Я говорю не
только о Президенте России, а о команде, которой удалось сделать невероятно –
сделать Россию тем, чем она была раньше и тем, чем ей по праву должно быть –
полюсом притяжения. Конечно, Западу это не по нраву – Россия выскользнула,
сняла ошейник, освободилась из клетки.. можете сюда подставить любой эпитет,
обозначающий освобождение, и будете правы.

Но Запад, как видно, до сих пор считал (а может быть, еще и считает), что, если
Россию приструнить, проучить, заставить, загнать в угол (снова можете
подставить сюда слова – синонимов в русском языке хватает), то все вернется в
90-е. А любые знаки того, что Россия готова дружить, но именно дружить, а не
пресмыкаться, Запад считает неприемлемыми. И вот, свершилось – Запад осознал,
что Россия не готова и не будет мириться с «демократической» привычкой делить
мир на близких и тех, кто обязан подчиняться. И потому он решился пойти в
атаку. Ну а мы... Знаете, за много лет мы впервые столкнулись с наступлением, а
не с обороной, Путин прямо заявил, что его не устраивает то-то и то-то, и все
больше и больше, и главное – России есть, чем ответить. Долго запрягали...

Законы русской политики (и политики тоже) – долго запрягаем, но быстро едем.
Оттого и градус высок. Да настолько, что градусник, гляди, вот-вот лопнет...
