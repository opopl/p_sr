% vim: keymap=russian-jcukenwin
%%beginhead 
 
%%file 09_08_2021.fb.arestovich_alexei.1.myshlenie_chelovek
%%parent 09_08_2021
 
%%url https://www.facebook.com/alexey.arestovich/posts/4576522979078413
 
%%author Арестович, Алексей
%%author_id arestovich_alexei
%%author_url 
 
%%tags chelovek,filosofia,myshlenie
%%title В принципе, все, что нужно от человека - это внеконцептуальное мышление
 
%%endhead 
 
\subsection{В принципе, все, что нужно от человека - это внеконцептуальное мышление}
\label{sec:09_08_2021.fb.arestovich_alexei.1.myshlenie_chelovek}
 
\Purl{https://www.facebook.com/alexey.arestovich/posts/4576522979078413}
\ifcmt
 author_begin
   author_id arestovich_alexei
 author_end
\fi

- В принципе, все, что нужно от человека - это внеконцептуальное мышление.

Одна из самых зловещих концепций, отравляющий человечество, это мета-концепция
дисциплинарного общества:

\url{https://disciplinary-society.apeiron.school/}

И дворник, и глава ООН определяется тем, как он мыслит (если мыслит).

Большинство людей не мыслит, а думает.

Эти два процесса отличаются по критерию воли: мышление - это директно-волевой
процесс (направляется изнутри), думание - реактивный (направляется
обстоятельствами). 

Есть средний класс: люди, которые уже пробуют мыслить, но делают это в рамках ведущей  концепции.

Концепция - система представлений о действительности, которая состоит из
картины мира (явной и неявной), этики, из неё выведенной, стержневых принципов,
на которых она стоит и присущего ей символического ряда. 

Концепция, это то, что заканчивается на -...изм, хотя не обязательно. 

Все «беды» в мире - от людей, живущих в симбиозе с концепциями. 

Концепция даёт способность к мышлению (а это - огромный скачок духа, от
которого в здравом уме отказаться невозможно), но подменяет волю человека
собственной волей, и фильтрует входящие в сознание данные системой более-менее
жёстких фильтров - то, чего нет в картине мира данной концепции - отбрасывается
и репрессируется.

Концептуальные люди хорошо опознаются по остекленевшему лицу - в момент, когда
задеты концептуальные вещи, нечеловеческому (в буквальном смысле) упрямству,
при попытке сдвинуть их со своих позиций или хотя бы предложить посмотреть на
них критически, и - полному отсутствию чувства юмора в концептуальных для себя
вопросах. 

Подсадка на концепцию даёт «жертве» важнейшее во Вселенной качество - волю, и
способность трансформировать ее в намерение - т.е. умению наводить внутренний
прицел внимания на цель и не терять его ни при каких внешних воздействиях или
внутренних сомнениях.

Это настолько ценная способность (волей Бог творил мир), что о ней одной можно
прочитать цикл больших семинаров.

В рамках заданной темы, достаточно сказать, что человек, крепко подсевший на
концепцию, становится как бы сверх-человеком (ещё бы - он «едет» на недюжинных
ресурсах концепции) - а от этого удовольствия отказаться люди оказываются
способными уровня через два развития, не раньше.

Концептуалисты сражаются между собой - за победу единственно верного учения, и
с презрением относятся к людям, недоросшим до персонального внимания концепции.

А вот людей, переросших концепции, концептуалы ненавидят - это живое отрицание,
их главного источника жизненной силы, и вообще-то субъективно эти негодяи
переживаются концептуалами, как сволочи, пытающиеся оторвать голодного младенца
от материнской сиськи, а наркомана - от его шприца.

Европейская культура, к счастью, дала человечеству жизненную и интеллектуальную
практику, правила и органика которой предполагает обучение внеконцептуальному
мышлению.

Это - философия.

Философию питают три Музы и охраняют три Даймона:

\begin{itemize}
\item - внимание к сущему (вопрос о том, что есть, а чего нет в реальности),
\item - необходимость обоснования мысли и слова (логоса) аргументами, 
\item - критический анализ собственного (и чужого мышления и речи), т.е. - отсутствие
священных коров мысли, неприкосновенных для критики.
\end{itemize}

Эти три правила - универсальный трёхгранный нож, для  обрезания пуповины
зависимости от концепций. 

Концепции сильны, и часто оказывались способными переродить изнутри даже
философию - тогда тот или иной ее раздел становится -...измом.

Но соблюдение Трёх правил - это верная панацея от этой болезни.

Золотой век человечества (если ему суждено случится) придёт, когда критическое
количество людей освоит внеконцептуальное мышление. 

Значительная часть моей деятельности посвящена выработке у людей способности к
внеконцептуптуальному мышлению - за что меня ужасно любят -...исты с любого
фланга.)

Я стараюсь раскачивать концепции, как гнилые зубы, отрезать их, как щупальца
ядовитых медуз и вытаскивать людей из глубоких отравлений, накапливающихся
годами.

Одно из самых главных сражений - это моя попытка если не сломать, то хотя бы
ослабить дисциплинарное общество.

Ослабить в конкретном человеке, ломающее его изнутри, открыть человеку дорогу к
свободе. 

Пора этому молоху на свалку истории. 

Подробнее - в видеоанонсе к семинару, который состоится в эту субботу, 14
августа, с 17 до 20.00, онлайн.

Вторая часть семинара - киносеминар, 28 августа, по фильму «Поезд на Юму».

\ii{09_08_2021.fb.arestovich_alexei.1.myshlenie_chelovek.cmt}
