% vim: keymap=russian-jcukenwin
%%beginhead 
 
%%file 29_03_2022.fb.tregub_oleksandr.kiev.1.geroizm_on_hold.cmt
%%parent 29_03_2022.fb.tregub_oleksandr.kiev.1.geroizm_on_hold
 
%%url 
 
%%author_id 
%%date 
 
%%tags 
%%title 
 
%%endhead 
\zzSecCmt

\begin{itemize} % {
\iusr{Olga Khvorostetskaya}

А ще Франкл писав, що вижили ті - хто мав смисл, сенс \enquote{заради чого
жити}  @igg{fbicon.heart.yellow}  @igg{fbicon.heart.blue}  сама книга має назву \enquote{У пошуках смислу}. Це мабуть
найголовніше сьогодні.

\iusr{Inna Prihodko}
\textbf{Olga Khvorostetskaya} в гарний зошит написала плани на весну та літо, перемога і море в планах є!

\iusr{Оксана Баранова}

Теж прочитала цю книгу за рік до війни Книга допомогла зберегти себе і рухатись
далі.

\iusr{Denys Sudilkovsky}
Улюблена книга випадково вчасно прочитана за місяць до війни.

\begin{itemize} % {
\iusr{Alexander Tregub}

після мемуарів з задоволенням прочитав і його наукові роботи, теж гарні

\iusr{Yulia Storozhuk}
Раджу ще \enquote{The Choice} Edith Eger.
\end{itemize} % }

\iusr{Dima Boychuk}

Героям слава!

Саша, відновлюйте роботу. Мене вже люди питають де вчитись веб-дизайну.

\iusr{Alexander Tregub}
\textbf{Dima Boychuk} вже  @igg{fbicon.smile} 

\iusr{Yana Bohdanova}

Бачила що розповсюджують новину, начебто прожектор надав доступ до 12 курсів
безкоштовно. Хотіла знайомим скинути, але не можу знайти у вас ніякої інфи.
Фейк?

\begin{itemize} % {
\iusr{Alexander Tregub}

не фейк  @igg{fbicon.smile}  на час війни плюс місяць, ми відкрили безкоштовний доступ до 12
відеокурсів гуманітарного факультета — історія дизайну, композиція, сучукрліт і
тд. Тут в першому припіненому пості деталі і код доступа
\url{https://t.me/humanitarium_prjctr}

\iusr{Yana Bohdanova}
\textbf{Alexander Tregub} О, дякую! @igg{fbicon.heart.yellow}  @igg{fbicon.heart.blue} 

\end{itemize} % }

\iusr{Ella Vasylevska}

Саша, дякую, роздрукую і буду читати кожного ранку, щоб не забувати!

\iusr{Yuliia Kozlovets}
100\% підтримую. Продж  @igg{fbicon.hands.shake}  @igg{fbicon.thumb.up.yellow} 

\iusr{Oleksandr Taranenko}

Саме так, не час розслаблятися. Кожного дня методично робити свою справу.

\iusr{Mira Mironova}

Почала читати Франкла 22 лютого.

Радію зараз кожній робочій годині, кожній прочитаній сторінці і кожному новому
навчальному відео.

\iusr{Anna Anastasieva}

Уривки зі спогадів Віктора Франкла колись справили сильне враження, добре, що
тоді дізналась про нього і дякую, що ви зараз про це нагадали

\iusr{Mariya Lipyatskaya}
Підписуюсь під кожним словом  @igg{fbicon.heart.blue}  @igg{fbicon.heart.yellow} 

\iusr{Oleg Malamuzh}

Ми на вихідних шикарний воркшоп провели, намалювались, відволіклись,
переключились, запрошую на наступний)

\iusr{Kate Jacuszek}
підписуюся під кожним словом!!!)

\iusr{Oleksii Salio}
Тільки так, тільки вперед

\iusr{Oksana Semenik}
Дуже чекаю на курс по UX Writing  @igg{fbicon.heart.broken} 

\iusr{Nata Sha}
Ваші слова - Богові у вуха. Хай все збудеться!

\iusr{Max Sukach}
Доречний пост

\iusr{Olga Degtyarenko}
Правда!

\iusr{Inna Prihodko}

Я думаю, що ми зараз маємо підтримувати одне одного в міру своїх можливостей.
Тому я роблю манікюр, ходжу на укладку та массаж - всім треба заробляти і
годувати свої сім'ї. Якщо є можливість купити у людини послугу або річ, це не
зайва витрата, це дружня підтримка. І ще я чорнобривці посіяла в ящик на
лоджії...


\end{itemize} % }
