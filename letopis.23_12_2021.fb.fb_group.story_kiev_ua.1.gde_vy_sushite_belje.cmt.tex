% vim: keymap=russian-jcukenwin
%%beginhead 
 
%%file 23_12_2021.fb.fb_group.story_kiev_ua.1.gde_vy_sushite_belje.cmt
%%parent 23_12_2021.fb.fb_group.story_kiev_ua.1.gde_vy_sushite_belje
 
%%url 
 
%%author_id 
%%date 
 
%%tags 
%%title 
 
%%endhead 
\zzSecCmt

\begin{itemize} % {
\iusr{Сергій Огородник}
у нас і сьогодні люди сушать на вулиці

\begin{itemize} % {
\iusr{Марина Соболевская}
\textbf{Сергій Огородник} , якщо нема загазованного повітря. Бо ми вже на балконах боїмось сушити білизну..

\iusr{Сергій Огородник}
\textbf{Марина Соболевская} нам з повітрям пощастило. Від проспектів далеко, поряд - Виноградарський ліс.

\iusr{Марина Соболевская}
\textbf{Сергій Огородник} , от коли можна позаздрити віддаленому від великих магістралей життю.)
\end{itemize} % }

\iusr{Николай Гребенкин}
Было время: и на чердаках снимали. Бедность потому что...

\iusr{Марина Соболевская}
\textbf{Николай Гребенкин}, представляю, как это обидно хозяйке.. такой урон..

\iusr{Татьяна Вакуленко}

На гвоздяке, на чердаке)

На нашем стареньком патриархальном Сырце, на длинной улице Олены Телиги, а раньше
Коротченко, а еще раньше Ново-Окружной... До сих пор во дворах полощатся алые и
прочие разноцветные паруса пододеяльников и простыней и сигнальные флажки
всяких милых детских вещичек!

Но панталоны и бюстгалтеры все также прячут от чужих глаз, и это правильно, я считаю).

Мой папа, светлая ему память, учил вешать белье на балконе простынями наружу и
последний ряд лицом в комнату. А уж между ними, внутри прятать нижнее белье)

\begin{itemize} % {
\iusr{Марина Соболевская}
\textbf{Татьяна Вакуленко} , оказывается, были и наработки, как лучше развешивать белье!) Спасибо за подробности!)

\iusr{Татьяна Вакуленко}
\textbf{Марина Соболевская} о-о-о, сколько было патриархальных условностей...

\iusr{Вікторія Святненко}
\textbf{Татьяна Вакуленко} и мой дедуля меня учил «прятать флаги» @igg{fbicon.face.grinning.sweat} 

\begin{itemize} % {
\iusr{Марина Соболевская}
\textbf{Вікторія Святненко} , я в восторге от вашего дедули.)

\iusr{Вікторія Святненко}
\textbf{Марина Соболевская} даааа! Дедуля у меня был ещё тот юморист)

\iusr{Марина Соболевская}
\textbf{Вікторія Святненко} , как он вас учил? Расскажите?)

\iusr{Вікторія Святненко}
\textbf{Марина Соболевская} я поступила в институт и переехала жить к нему в частный дом. Зимой дедуля просил расчистить снег, ну а я студентка! В одну лопату стежку почищу и убегать! А он в след: «Я тебе что!!! Канатоходеееец? А ну ка нормально чисти!!!!»

\iusr{Марина Соболевская}
\textbf{Вікторія Святненко} , гениальный дедушка! Это же надо придумать! Канатоходец!)))

\iusr{Вікторія Святненко}
\textbf{Марина Соболевская} 

так и с бельём... Из под тишка показывает на серое вывешенное белье соседки и
говорит: «Витуля, это не белье, такое стыдно вывешивать на всеобщее обозрение!
Накрайняк, для улицы держи новый комплект постельного и нательного! Что б
думали «какая хозяйкаааа!»

\iusr{Вікторія Святненко}
\textbf{Марина Соболевская} ой, Мариночка! А его фраза: «я тебе что, лесоруб??? Что ты мне столько борщу насыпала???»

\iusr{Марина Соболевская}
\textbf{Вікторія Святненко} , а почему лесоруб?)

\iusr{Вікторія Святненко}
\textbf{Марина Соболевская} бо вони важко працюють і багато їдять @igg{fbicon.laugh.rolling.floor}{repeat=3} 

\end{itemize} % }

\end{itemize} % }

\iusr{Татьяна Петрюк}

 @igg{fbicon.laugh.rolling.floor} Вечный кадр ... Это Подоле улице Еленовская
 @igg{fbicon.face.grinning.big.eyes} ( не удержалась и засняла исторический
 кадр)

\ifcmt
  ig https://scontent-frx5-1.xx.fbcdn.net/v/t39.30808-6/269814219_2552458711552741_6760950649870775704_n.jpg?_nc_cat=110&ccb=1-5&_nc_sid=dbeb18&_nc_ohc=vic5Y-uu6zsAX-GU2gP&_nc_ht=scontent-frx5-1.xx&oh=00_AT-Dt6JxYU9P9ByLCebWNQS2Ea-mxkJWb8gqRPh-Dkk75w&oe=61CEA73C
  @width 0.4
\fi

\iusr{Марина Соболевская}
\textbf{Татьяна Петрюк} , может это тайный знак?)

\iusr{Вікторія Святненко}

Ой, Спасибо большое! С таким удовольствием прочитала! Да, да, застали и мы в
80-е такие столбы с веревками... а тетушка на Большой Житомирской вывешивала
белоснежное белье во внутреннем дворике... ах, как же оно пахло с мороза....


\iusr{Марина Соболевская}
\textbf{Вікторія Святненко} , искренне благодарна за доброе внимание и ваши воспоминания!)

\iusr{Валерий Стрельцов}
А, Ваш папа случай не с Татарки.

\iusr{Татьяна Вакуленко}
\textbf{Валерий Стрельцов} нет, мы евбазовские. Воровского 6.

\iusr{Катя Бекренева}
Нам с квартирой повезло в том смысле, что до дороги далековато, и дом во дворе. Постельное бельё с мороза это что-то)

\iusr{Марина Соболевская}
\textbf{Катя Бекренева} , искренне за вас радуюсь!)  @igg{fbicon.tulip} 

\iusr{Николай Бурчин}
Пахнет наволочка снегом, где-то капает вода..  @igg{fbicon.face.nerd} 

\begin{itemize} % {
\iusr{Марина Соболевская}
\textbf{Николай Бурчин} , философское видение.)

\iusr{Николай Бурчин}
\textbf{Марина Соболевская} , это ж из песни)
\end{itemize} % }

\iusr{Lidia Bualo}
На чердАке, на гамАке, чтоб не свиснули собаки

\iusr{Марина Соболевская}
\textbf{Lidia Bualo} , отлично! Уже третий вариант стихотворения.)

\iusr{Yurii Koval}
Саме так і казали: \enquote{Сохнєтє}.

\begin{itemize} % {
\iusr{Марина Соболевская}
\textbf{Yurii Koval} , так это не шутка?)

\iusr{Yurii Koval}
\textbf{Марина Соболевская} Єврейский говір.
\end{itemize} % }

\iusr{Evgen Zachepylenko}
У них в Венеции то же сушат ...

\ifcmt
  ig https://scontent-frt3-1.xx.fbcdn.net/v/t39.30808-6/269811084_1055423678365065_813436908500964475_n.jpg?_nc_cat=104&ccb=1-5&_nc_sid=dbeb18&_nc_ohc=20iNGTck2OkAX-_ho0R&_nc_ht=scontent-frt3-1.xx&oh=00_AT-FcFW4x05tMT_w5XCVR1i2rCmYqedcyMX8mwGNo1X5Ng&oe=61CF08ED
  @width 0.3
\fi

\begin{itemize} % {
\iusr{Татьяна Николаева}
\textbf{Evgen Zachepylenko} 11трусов и 2 майки - странная коллекция )

\begin{itemize} % {
\iusr{Evgen Zachepylenko}
\textbf{Татьяна Николаева} итальянцы наверно часто меняют такой народ...

\ifcmt
  ig https://scontent-frt3-1.xx.fbcdn.net/v/t39.30808-6/269808432_1055429585031141_8086476427554166830_n.jpg?_nc_cat=104&ccb=1-5&_nc_sid=dbeb18&_nc_ohc=ZFqV7LjdhbUAX_nq46t&_nc_ht=scontent-frt3-1.xx&oh=00_AT-eZwBtTFnxJ1bhxN5bH4vT-hW-ndmKVydmwarHVynD0Q&oe=61CDF804
  @width 0.3
\fi

\iusr{Татьяна Николаева}
\textbf{Evgen Zachepylenko} все постирала  @igg{fbicon.laugh.rolling.floor} 

\iusr{Evgen Zachepylenko}
\textbf{Татьяна Николаева} таки ДА

\iusr{Al Pol}
\textbf{Evgen Zachepylenko} 

Однажды у меня в доме была в гостях прелестная юная дева. В то время наш
коттеджный поселок ещё продолжал застраиваться и рядом на участке вовсю шли
работы по сооружению фундамента соседского дома.

Дева имела привычку разгуливать по дому с бокалом шампанского в костюме
Евы..... В задумчивости подошла к большому окну и наблюдая за рабочими
предалась раздумьям...

Я в это время после жарких утех задремал...

Разбудил меня дикий вопль. Работяга на соседском участке грохнул себя по ноге
200-килограммовой трамбовкой.... Засмотрелся бедолага на юную, нагую деву в
окне....

Женская красота - страшная сила ....

\end{itemize} % }

\iusr{Катя Семенкова}
\textbf{Evgen Zachepylenko} 

романтично потягивая кофе, а над тобой развиваются лёгкие труселя
@igg{fbicon.face.smiling.eyes.smiling}{repeat=3} 

\begin{itemize} % {
\iusr{Татьяна Николаева}
\textbf{Катя Семенкова} всегда можно посмотреть нарравление и силу ветра  @igg{fbicon.wink} 

\iusr{Evgen Zachepylenko}
\textbf{Татьяна Николаева} итальянская девушка устала от стирки

\ifcmt
  ig https://scontent-frt3-2.xx.fbcdn.net/v/t39.30808-6/269822623_1055455211695245_7074469432369854602_n.jpg?_nc_cat=101&ccb=1-5&_nc_sid=dbeb18&_nc_ohc=4MtHzstjDVwAX8amLHS&_nc_ht=scontent-frt3-2.xx&oh=00_AT8xn9YH87yMU2w-qBnrVF21Sv2jvAXJht6oDodvqnXoqw&oe=61CE88B2
  @width 0.3
\fi

\end{itemize} % }

\iusr{Марина Соболевская}
\textbf{Evgen Zachepylenko} , супер! И милиция их не гоняла!)))

\end{itemize} % }

\iusr{Владимир Баваровский}

У нас во дворе (Верхний Сырец) были бабушки, на которых оставляли белье. Они
четко сидели и смотрели за ним. И нас гоняли чтоб мы не бегали и не исгадили))
А вот зимой хозяйки как-то сами следили. Мама всегда сушила на открытом
балконе. Но в начале 2000х как-то после сушки она сняла и встряхнула простынь,
а от нее столб пыли( Больше у нас на балконе белье не сохло и веревок нет
@igg{fbicon.face.grinning.smiling.eyes}  А вот труселя с панталонами у нас
всегда на кухне на натянутой по диагонали веревке))

\begin{itemize} % {
\iusr{Lesya Paschenko-Startseva}
\textbf{Владимир Баваровский} Простите, вопрос не по теме: а что такое Верхний Сырец( живу на Щусева,36)?

\begin{itemize} % {
\iusr{Владимир Баваровский}
\textbf{Lesya Paschenko-Startseva}, 

массивы возле парка детской жд и выше в сторону парка нивки. Это условное
разделение. Больше в период подростковых разборок) \enquote{А, ты с вехнего Сырца? А мы
с нижнего) Вот тебе в дыню!}  @igg{fbicon.laugh.rolling.floor} 


\iusr{Владимир Баваровский}
\textbf{Lesya Paschenko-Startseva}, 

вот, кстати, кажется по Щусева, оно и делилось. Я точно уже не скажу. Я жил с
самого края между парками и мне было это не важно, а вот друзья обитавшие в
районе 97й школы на Телиги постоянно имели столкновения и им это было
принципиально. Знаю только, что нежелательно было без взрослых соваться дальше
Щусева) Если пройти по Грекова до техникума (или что там сейчас),то дальше
можно было схлопотать.


\iusr{Тома Храповицкая}
\textbf{Владимир Баваровский} Помню, помню... не участвовала, но слышала от мальчишек.

\iusr{Lesya Paschenko-Startseva}
\textbf{Владимир Баваровский} Ничего себе! Я про такое и не подозревала! Я выросла на Лукьяновке, училась на Татарке(СШ1). Спасибо за разъяснение.
\end{itemize} % }

\iusr{Марина Соболевская}
\textbf{Владимир Баваровский} , идеальный вариант! И белье высохнет, и охрана надежная!)

\begin{itemize} % {
\iusr{Владимир Баваровский}
\textbf{Марина Соболевская}, 

да! До сих пор помню, как баба Сима ходит по двору \enquote{руки в боки}, в
одной руке выбивалка для ковра и гоняет нас от белья) А нам ведь там медом
намазано @igg{fbicon.face.grinning.smiling.eyes}  Не соображали, что чумазые
пачкаем белье @igg{fbicon.face.grinning.smiling.eyes} 


\iusr{Марина Соболевская}
\textbf{Владимир Баваровский} , прямо увидела эту картинку! Спасибо!)
\end{itemize} % }

\end{itemize} % }

\iusr{Леонид Гапон}
Мы в духовке на вегёвке и немножко на чегдак, чтобы не было багдак.

\begin{itemize} % {
\iusr{Ли Бесс}
\textbf{Леонид Гапон} очень по-одесски ))

\iusr{Наталья Борчанинова}
\textbf{Ли Бесс} И по-белоцерковски!!!)))

\iusr{Ли Бесс}
\textbf{Наталья Борчанинова} та да )) надо ж понимать согт людей )))

\iusr{Марина Соболевская}
\textbf{Леонид Гапон} , спасибо за еще один вариант!)
\end{itemize} % }

\iusr{Al Pol}

Бельё, высушенное на морозе имеет неповторимый аромат. Никакая сушильная машина
так не высушит.

Вспоминаю улицы Неаполя вблизи ж/д вокзала, где прямо через всю улицу натянуты
веревки для сушки белья .... Там и трусы, и всякое другое бельё никого не
смущает.

\begin{itemize} % {
\iusr{Марина Соболевская}
\textbf{Al Pol} , теперь и я люблю Италию! За свободу в развешивании белья!)

\begin{itemize} % {
\iusr{Al Pol}
\textbf{Марина Соболевская} 

Италия это целый мир ! Флоренция, Неаполь, Рим, Милан, Венеция .... Обожаю
эти города .... Хотя у небольших городов своя прелесть ... Сорренто,
Монтекатини, Лидо-ди-Есоло, Амальфи, Позитано - просто сказка !

Однажды в Неаполе меня \enquote{обидели} .... В каком-то бутике изысканно-вежливый,
прилизанный продавец мне сказал на убогом английском: Dear sir, you are
wasting your time, in our stores there are no clothes of your size.

Несчастный не понял, что этим он сделал мне комплимент.

\iusr{Марина Соболевская}
\textbf{Al Pol}, спасибо! Мне ужасно интересно!)

\iusr{Al Pol}
\textbf{Марина Соболевская} 

Иногда, путешествуя, я снимаю короткие видеоролики, которые выкладываю в
YouTube..... Не всегда, но часто ...

Испания, Италия, Греция, Южная Франция, Эмираты, Норвегия, Черногория, Швеция, и т. д.

Могу скинуть ссылку на мой видеоролик об Италии....

\iusr{Марина Соболевская}
\textbf{Al Pol} , здесь нельзя делать ссылки. Можно в личку.)

\iusr{Al Pol}
\textbf{Марина Соболевская} 

Отправил ссылку в личку.... я редко пользуюсь этим сервисом, возможно что-то
пошло не так. Пишите, если не получите.

\end{itemize} % }

\iusr{Вера Мезенцева}

Волшебное чувство восхищения после каждого вывешивания во дворе частного
дома, это было в юности и почему то запомнился этот запах единственного
стирального порошка в те времена Лотос.

\begin{itemize} % {
\iusr{Марина Соболевская}
\textbf{Вера Мезенцева} , 

удивительно! У вас запах порошка, как триггер, возвращает вас мысленно в картинки из прошлого.)

\iusr{Al Pol}
\textbf{Марина Соболевская} 

Позволю себе вмешаться. Действие запахов в качестве триггера, запускающего
определенные эмоции, почувствовал на себе не однажды.

Вы, наверное, не знаете такие духи, как БЫТЬ МОЖЕТ. В конце 60-х, и в 70-х
прошлого века это был очень популярный аромат. Многие девушки пользовались
этими духами, в том числе и предмет моего пылкого, юношеского обожания....моя
первая любовь ....

С тех пор прошло более тридцати лет и, однажды, я почувствовал аромат БЫТЬ
МОЖЕТ.... Что со мной произошло, описать словами невозможно. Откуда появились
эти духи через столько лет - история заслуживающая отдельного повествования.

Потом была другая девушка и духи ШЕХЕРЕЗАДА ... И снова через много лет встреча
с запахом .... И снова эмоциональный шторм ....
\end{itemize} % }

\end{itemize} % }

\iusr{Марина Соболевская}

Весной 2016 года Мэрия инициировала закон о публичном пространстве, согласно
которому сушить белье на балконах будет запрещено — даже под угрозой штрафов.
Кроме того, инициатива должна была избавить город от уродливых балконов,
которые портят внешний вид города, и излишества рекламных билбордов.

Инициаторы проекта, заммэра Виталия Кличко Алексей Резников, обещали вынести
его на рассмотрение Киеврады в сентябре.)

\begin{itemize} % {
\iusr{Vladimir Popoff}
\textbf{Марина Соболевская} 

а про уродские кондиционеры там нет инициативы? Чтобы убрать их нафиг с
фасадов, хотя бы старых, зданий.


\iusr{Марина Соболевская}
\textbf{Vladimir Popoff} , честно говоря, я не знаю, чем закончилась эта история.)

\iusr{Владимир Баваровский}
\textbf{Марина Соболевская}, увы( Ничем

\iusr{Марія Рябоконенко}
\textbf{Марина Соболевская} 

Так ведь и советской власти было запрещено сушить белье на балконе, как пишет
автор \enquote{выше ватерлинии}. На центральных улицах за это штрафовали.

\end{itemize} % }

\iusr{Людмила Мудренко}
В 50-х годах у нас на Подоле сушили во дворе и охраняли.
Мама повесила мое поатьеце и рейтузы на чердаке. Утром было пусто.

\begin{itemize} % {
\iusr{Марина Соболевская}
\textbf{Людмила Мудренко} , вот да. Пишут, что в 50-ых послевоенных воровство было обычным делом.

\iusr{Вера Мезенцева}
\textbf{Людмила Мудренко} Ужасно
\end{itemize} % }

\iusr{Валентина Козачук}

Всегда сушила белье на балконе, дом стоит внутри массива, дороги далеко., но и
в ванной есть веревки, очень удобно досушивать белье и полотенца после душа. А
когда жили в хрущевке в Ново-Беличах, то часто сушили на улице на веревках,
воровства не было, так как все друг друга знали, многие работали на одном
предприятии, а зимой на снег всегда выносили коврики, дорожки и чистили снегом

\begin{itemize} % {
\iusr{Марина Соболевская}
\textbf{Валентина Козачук} , спасибо за ваш рассказ!)

\iusr{Володимир Денисов}
\textbf{Валентина Козачук} я свою Бабулю переконав років 7 тому, краще здати у чистку.... до того кожну зиму -« онучек, сніг впав, треба паласи почистити»
Але ... після снігу запах інший...
\end{itemize} % }

\iusr{Наталья Борчанинова}

В 80-е я жила на Березняках над о. Тельбин на 13 этаже. Балкон был не
застеклён. Вернулась с пляжа, пополоскала купальник (раздельный) и, когда
вывешивал сушить на верёвке, прищепка соскользнула в руке, трусы в другой руке
тоже соскользнули, не прищеплены были, и.... полетели вниз на чей - то балкон,
может быть и соседнего подъезда)))...

\begin{itemize} % {
\iusr{Марина Соболевская}
\textbf{Наталья Борчанинова} , сочувствую.) Нашлись? Или с концами?

\iusr{Наталья Борчанинова}
\textbf{Марина Соболевская} 

Второй вариант))) С концами...

Ни на деревьях, которые росли под домом, ни на земле их не было. А под моим
балконом было 12 балконов моего стояка и столько же смежных балконов соседнего
подъезда...

Объявлений о такой \enquote{пропаже} в подъездах я не писала)))

\iusr{Марина Соболевская}
\textbf{Наталья Борчанинова} , купальник- это важно. ) Потеря ужасная. Очень вас понимаю.

\iusr{Наталья Борчанинова}
\textbf{Марина Соболевская} 

А я тогда представила \enquote{картину}, когда хозяйка того балкона нашла
женские трусы и вела допрос мужу)))

\iusr{Марина Соболевская}
\textbf{Наталья Борчанинова} , это шедевральная ситуация! Представляю!)))
\end{itemize} % }

\iusr{Надежда Владимир Федько}

1970-1972-й... Подільський райвідділ міліції... Щотижнева нарада дільничих
інспекторів... Нараду проводить керівник служби дільничих інспекторів,
заступник начальника райвідділу... Підполковник Ш-ов.

Одне із стратегічних завдань - боротися з вивішенням білизни на балконах по вулицях (перелік вулиць)!

Керівна настанова - не ініціатива міліції (у дільничих багато більш важливих
проблем), а директивна вказівка райкому партії! Київ - зразкове місто!

\begin{itemize} % {
\iusr{Марина Соболевская}
\textbf{Надежда Владимир Федько} , отлично! Спасибо за подтверждение!)

\iusr{Надежда Владимир Федько}
\textbf{Марина Соболевская} 

В ті роки я працював у карному розшуку... Дільничі мого району обслуговування
постійно жалілися, що боротьба з \enquote{трусами на балконі} відволікає їх від
боротьби із злочинністю.

До речі, в ті роки була категорія злодіїв, які спеціалізувалися на крадіжках
білизни у дворах, з балконів 2 та верхнього поверхів, з горищ будинків.


\iusr{Марина Соболевская}
\textbf{Надежда Владимир Федько} , благодарю от всей души! Гениально про жалобы!! А про воровство. Все логично..
\end{itemize} % }

\iusr{Mariia Lashkevych}
Тиріли і котлети з пательні, що на примусі поки кухарка за чимось до кімнати йшла

\begin{itemize} % {
\iusr{Марина Соболевская}
\textbf{Mariia Lashkevych} , котлеты свежепожаренные, видимо, так вкусно пахли, что трудно было удержаться.))
\end{itemize} % }

\iusr{Елена Баскова}

1980 - спокойно сушили белье на балконах и никаких запретов не встречали.
помню еще соседей, сушивших свое белье на улице (балконов у них, видимо, небыло)

\begin{itemize} % {
\iusr{Владимир Баваровский}
\textbf{Елена Баскова}, смотря где.

\iusr{Марина Соболевская}
\textbf{Елена Баскова} , верно. Вы их уже не застали.)

\iusr{Марина Соболевская}
\textbf{Елена Баскова} , посмотрите коммент - \textbf{Надежда Владимир Федько}.) В 1970-ых милицию обязывали бороться с бельем на балконах.))

\begin{itemize} % {
\iusr{Елена Баскова}
\textbf{Марина Соболевская} я вижу коммент. Я пишу про 80-е.

\iusr{Олена Харченко}
\textbf{Марина Соболевская} якщо на хрещатику, то можливо. А на воскресенці не чіпали. Боролись зі застекленими балконами

\iusr{Раиса Лузан}
\textbf{Олена Харченко} 

да, было такое. Мы жили на Печерске на первом этаже в доме , который выходил
окнами на райком партии и в 73 году застеклили лоджию. Через какое-то время
папу вызвали в райком и сказали снять застекление. Не знаю, что ответил им
папа, но стекла остались. Потом постепенно на всех 16-ти этажах соседи начали
стеклить лоджии, также как у нас. Смотрелось очень гармонично, как будто так и
было при строительстве дома.

\end{itemize} % }

\iusr{Наталья Борчанинова}

А вот в начале 80-х на оз. Тельбин проводились международные соревнования по
катанию на водных лыжах... До начала соревнований с жильцов дома, у которых
окна выходили на озеро, участковый или какие-то \enquote{товарисчи} взяли расписку,
чтобы не вывешивали белье в дни соревнований...

\begin{itemize} % {
\iusr{Марина Соболевская}
\textbf{Наталья Борчанинова} , очень интересное событие! Спасибо!)

\iusr{Наталья Борчанинова}

Сейчас живу в г. Белая Церковь. Белье с соседями сушим на площадках с проволок
для сушки белья. Дом \enquote{хрущовка}, 88 квартир. Живу на 1 этаже. Летом сохнет
белье на солнце, а зимой на морозе!

За веревки соседи не ругаются, даже те, кто живут на верхних этажах, тоже
предпочитают сушить во дворе. Краж давно нет...

\end{itemize} % }

\end{itemize} % }

\iusr{Анна Цесорева}

А який чудовий спектакль з такою назвою я дивилась в театрі Колесо на
Андріївському узвозі! Хто ще не бачив - рекомендую \enquote{Одного разу на Подолі, або
гдє ві сохнітє бєльйо}

\iusr{Марина Соболевская}
\textbf{Анна Цесорева} , дякую за рекомендацію!)

\iusr{Ксенія Ковальська}

Трохи не в тему, проте дотично. Згадайте, як вибивали тріпачкою килими на
спеціально встановленій конструкції зі зварених труб. Це влітку. А взимку, коли
випадав сніг, з усіх квартир виносили килими, притрушували їх сніжком,
вибивали, топтали, змітали сніг віником і несли в квартиру разом з килимом
запах снігу. Минулої зими жодного килима на снігу я з вікна багатоповерхівки не
бачила.

\begin{itemize} % {
\iusr{Владимир Баваровский}
\textbf{Ксенія Ковальська}, 

это потому что уже ковров почти не осталось @igg{fbicon.smile}  Однако как раз
в прошлом году на Сырце, где живут родители, все дворы были в следах чистки
ковров и даже часть парка успели изгадить
@igg{fbicon.face.grinning.smiling.eyes}  Так что традиции неискоренимы.

\begin{itemize} % {
\iusr{Марина Соболевская}
\textbf{Владимир Баваровский} , сейчас помыть ковер- 120 грн. кв. метр.)

\iusr{Владимир Баваровский}
\textbf{Марина Соболевская}, я знаю, но папа брыкается и ни в какую)

\iusr{Ксенія Ковальська}
\textbf{Владимир Баваровский} дійсно, килимів на душу населення поменшало. Але в старих районах ще є. Я свій здавала чистити на фірму.

\iusr{Марина Соболевская}
\textbf{Ксенія Ковальська} , а за яку ціну? Й чи задоволені?)

\iusr{Ксенія Ковальська}
\textbf{Марина Соболевская} не пам'ятаю ціну. Задоволена. Зараз згадаю назву фірми. Здається \enquote{Мокра справа} чи \enquote{Мокрое дело}.

\iusr{Марина Соболевская}
\textbf{Ксенія Ковальська} , дякую!)
\end{itemize} % }

\iusr{Марина Соболевская}
\textbf{Ксенія Ковальська} , так.) Моя матуся переймалась в ті роки, як снігу не було, як килимки мити?)
\end{itemize} % }

\iusr{Nadia Matura}
Не слямзили воровки

\iusr{Марина Соболевская}
\textbf{Nadia Matura} , верно. Просто мне так запомнилось, ошибочно.)

\iusr{Татьяна Бартновская}

Замечательно! Сейчас живу в частном доме в селе и сушу белье всегда во дворе, на воздухе! Даже зимой, когда руки с прищепками за секунду примерзают к верёвке. И всегда, где бы я не жила, и дома, в Киеве, всегда сушилка белье на улице! И в двухтысячных охраняла его. Все точно так как в этом замечательном рассказе!!! А этот стишок - родом из нашей любимой Одессы!  @igg{fbicon.face.smiling.eyes.smiling}{repeat=3}  @igg{fbicon.face.blowing.kiss}{repeat=3} 

\begin{itemize} % {
\iusr{Марина Соболевская}
\textbf{Татьяна Бартновская} , с удовольствием прочла ваш комментарий! Думаю, одесские стишки передавались цыганской почтой.) А сушка белья на природе - постепенно забывается. Так радуюсь за вас, за вашу возможность продлить удовольствие!)

\begin{itemize} % {
\iusr{Larysa Karbysheva}
\textbf{Марина Соболевская} В старых районах ещё есть место во дворах. Много деревьев, между ними веревки натягивают..

\iusr{Марина Соболевская}
\textbf{Larysa Karbysheva} , но охранять, я так понимаю - еще надо.)

\iusr{Larysa Karbysheva}
\textbf{Марина Соболевская} Не знаю.... Из Окна вижу как развешивают. Охранников нет, может 1-ый этаж?

\iusr{Марія Рябоконенко}
\textbf{Марина Соболевская} Сестра живёт на Отрадном. Когда не приеду, во дворе сушится белье. Веревки постоянно натянуты, их не снимают.
\end{itemize} % }

\end{itemize} % }

\iusr{Татьяна Дзюба}
В духовке, на веровке и немножко на чердак!

\iusr{Марина Соболевская}
\textbf{Татьяна Дзюба} , благодарю. Уже четвертый вариант!)

\iusr{Лариса Кушниренко}

Сушили во дворе на Десятинной. Чтобы охраняли- не помню. Веревки подпирали
длинными палками, чтобы белье сушилось выше и не пачкалось.

\iusr{Ольга Dzhun}
\textbf{Лариса Кушниренко} и у нас так делали, чтобы белье не поовисало

\iusr{Марина Зонова}
Сушили во дворе на Десятинной. Чтобы охраняли- не помню. Веревки подперали

\begin{itemize} % {
\iusr{Марина Соболевская}
\textbf{Марина Зонова} , центр города, на минуточку.) Верно! Палкой веревки подпирали. Спасибо, что напомнили.)

\iusr{Tatyana Mazerati}
\textbf{Марина Соболевская} odin nol, u predidushih pisateley podpErali-navernoe uchilis u odnogo pedagoga russkogo yazika
\end{itemize} % }

\iusr{Всеволод Серов}

Житомир 80е. Пам'ятаю у дворі. Але хтось мав спостерігати... Аби десь не щезло

\iusr{Марина Соболевская}
\textbf{Всеволод Серов} , не замислювалась, а воно он як..)

\iusr{Лена Рытова}
В комнате конечно, планировка позволяет

\iusr{Люсянка Балашова}

на Соломенке... в соседнем доме сохранилась такая площадка, с метал. столбами и
перекладинами для ковров.. в нашем доме двора нет, подъезды выходят прямо на
дорогу... но столбы для сушки и для ковров были за домом... там сушила бельё
соседка с первого этажа-у неё был целый процесс-раз в три месяца она
вываривала, и при этом открывала дверь на цепочку... стирала и развешивала на
весь двор-всё что есть у них-по несколько штук-простыни
,пододеяльники, наволочки, мужс. рубашки, трусы всякие и т. д-и это всё собиралась
в квартире (двухкомнатная )..а соседи уже обсуждали-Валька стирает.... сейчас в
этой квартире живут другие люди

\begin{itemize} % {
\iusr{Марина Соболевская}
\textbf{Люсянка Балашова} , а ведь верно! Весь двор был в курсе постирушек. и кто что носит.)

\iusr{Лидия Щепкина}
\textbf{Люсянка Балашова} 

живем возле военного округа, в пятиэтажке, до сих пор на таких вот квадратиках
иногда сушат белье, лично мы за балконом, т. к. он выходит в сад, белье сохнет
хорошо, а зимой пахнет свежестью...


\iusr{Люсянка Балашова}
\textbf{Лидия Щепкина} возле КВО ?

\iusr{Лидия Щепкина}
\textbf{Люсянка Балашова} да,на Брюллова
\end{itemize} % }

\iusr{Любовь Емельянова}
И сейчас тырят и сушат во дворе на веревке. Есть еще места.

\begin{itemize} % {
\iusr{Марина Соболевская}
\textbf{Любовь Емельянова} , так вроде в городе такая загазованность, что не знаешь, где сушить лучше.)

\iusr{Любовь Емельянова}
\textbf{Марина Соболевская} на ДВРЗ еще хороший воздух.

\iusr{Олена Андурова}
\textbf{Марина Соболевская} 

Когда болела свекровь, пять лет назад, простыни сушили во дворе(балкона не
было).На Довженка. Вроде и трасса рядом, а белье было чистое, никто не охранял.

\end{itemize} % }

\iusr{Надежда Владимир Федько}

У 1974-му, коли ми побралися і я переїхав жити до дружини, то в сусідньому
будинку (навпроти) була пральня. Ми були молоді і заморочуватися з пранням нам
було не комільфо. Тому постільну білизну здавали в пральню. Коштувало дешево і
якість прання була високою. Білизна також була випрасувана.

Потім пральню закрили...

\iusr{Марина Соболевская}
\textbf{Надежда Владимир Федько} , так. Мої батьки теж постільну білизну здавали в пральню.)

\iusr{Наталія Громова}

Ніяких проблем, в рівень з перилами балкону були натягнуті вірьоаки, , так що
панталончикм не прикрашали будинок, з краю вішали простирадло, а далі що
заманеться, були ще вірьовки на кузні, хто жив на першому поверсі, ті вішали в
дворі, , Я жила на Подолі, будинок у дворі.

\begin{itemize} % {
\iusr{Марина Соболевская}
\textbf{Наталія Громова} , дякую, що розповіли!)

\iusr{Наталія Громова}
\textbf{Марина Соболевская} Да якось це не було проблемою, а от кума моя носила в пральню, на Спаській, в неі не було балкону, будинок на розі Хоривоі та Межигірської.

\iusr{Марина Соболевская}
\textbf{Наталія Громова} , так! Дякую, що нагадали! Батьки також постільну білизну вдносили в пральню! Було дуже зручно. Додому несли запаковану, прасовану та крохмалену білизну.)
\end{itemize} % }

\iusr{Ольга Почивалова}
Хорошо, что это непотребство закончилось

\begin{itemize} % {
\iusr{Марина Соболевская}
\textbf{Ольга Почивалова} , какое? Не соображу.)

\iusr{Ольга Почивалова}
\textbf{Марина Соболевская} вывешивание исподнего во дворе

\iusr{Марина Соболевская}
\textbf{Ольга Почивалова} , вон в Италии его вывешивают надо головами прохожих.))

\iusr{Ольга Почивалова}
\textbf{Марина Соболевская} не вижу в этом никакой эстетики

\iusr{Марина Соболевская}
\textbf{Ольга Почивалова} , кому что ближе, эстетика ограничений или эстетика разнообразия. Мне ближе разнообразие. А значит и ваша точка зрения- тоже.)  @igg{fbicon.tulip} 

\end{itemize} % }

\iusr{Valentina Urban}

В далеких теперь уже 60- х практически все хозяйки сушили белоснежное,
подсиненное, накрахмаленное постельное белье на собственных веревках, так как и
остальные вещи. Среди наших шести 5- то этажек на Печерск е имелись случаи
когда озорные мальчишки наверное ради интереса бритвенными лезвиями полосали
замерзшие на морозе простыни и пододеяльники. Даже нам детям было больно
смотреть как рыдали хозяйки обнаружив причиненный ущерб, однако никогда никто
пойман не был.

\iusr{Марина Соболевская}
\textbf{Valentina Urban} , не знала. Спасибо, что рассказали.

\iusr{Виктория Зайцева}

У нас в Мариуполе тоже была площадка для сушки белья во дворе хрущевки.
Бетонные столбы по углам с кручьями для веревок. Я ещё помню как вешала с
мамой. Потом, когда балкон застеклили и ветер не трепыхал белье, то вешали на
балконе. У нас в городе белье надо было сушить быстро, а то с неба могла сажа
посыпать, если ветер с завода. А их два, надо смотреть с какой стороны ветер.
Поэтому застекленный балкон стал спасением.

\begin{itemize} % {
\iusr{Марина Соболевская}
\textbf{Виктория Зайцева} , спасибо! Важны особенности города. В Мариуполе жили дедушка с бабушкой мужа. В Мариуполе жила двоюродная бабушка учитель музыки Соболевская Галина Казимировна.)

\begin{itemize} % {
\iusr{Виктория Зайцева}
\textbf{Марина Соболевская} 

Что-то знакомая фамилия. А в какой школе она преподовала? Я училась в
музыкальной школе, у нас в городе их было всего 5, так что можно было слышать
фамилии учителей на сводных концертах.


\iusr{Марина Соболевская}
\textbf{Виктория Зайцева} . 

кажется. 1 музыкальная школа. Здесь нельзя давать ссылки. Но вы погуглите. Про
нее есть статья в газете \enquote{Старый Мариуполь}.) И фото.)

\iusr{Виктория Зайцева}
\textbf{Марина Соболевская} 

Ясно, вот оно и понятно огкуда фамилия знакомая. Эта самая старая и самая
известная школа в городе. Я в ней один год училась, а потом нас перевели в
другую, ближе к месту жительства. Здание школы старое, ещё дореволюционное.
Кстати, до сих пор стоит. Там сейчас контора Электросетей, но ещё флигель
жилой. Если хотите, могу фото школы выложить, я была там осенью в этом году и
сфоткала это здание, я то школу помню. Ещё интерьер помню.

\iusr{Марина Соболевская}
\textbf{Виктория Зайцева} , 

Чудо! Была бы вам благодарна! А можно и адрес конторы Электросетей? Может у них
в архиве сохранилось личное дело за 1945-1949 годы (Технорук участка Донэнерго)
дедушки Соболевского Юрия Казимировича.)


\iusr{Виктория Зайцева}
\textbf{Марина Соболевская} 

Адркес, я сейчас гляну точный. Знаю, что переулок Кузнечный, кажется номер 1. Я
посмотрю, потом, в квитанциях, точный адрес, но фото школы ( ныне Электросети )
вот выложу сейчас.

\ifcmt
  ig https://scontent-frt3-1.xx.fbcdn.net/v/t39.30808-6/269859487_3021796321412777_7838891879967474200_n.jpg?_nc_cat=104&ccb=1-5&_nc_sid=dbeb18&_nc_ohc=b5vLh4_JzukAX8TB9Bc&_nc_ht=scontent-frt3-1.xx&oh=00_AT_ybZrSirZwmitRu-YYqUI_kcrGoQ2-T1BR4-ak7nCn4Q&oe=61CE13E0
  @width 0.3
\fi

\iusr{Виктория Зайцева}
\textbf{Марина Соболевская} 

Фото флигеля. Я люблю в старой части города бывать, тоже фотографировать. Есть
такие шедевры архитектурные, но экология.....

\ifcmt
  ig https://scontent-frt3-1.xx.fbcdn.net/v/t39.30808-6/269853253_3021797011412708_2347658220345955989_n.jpg?_nc_cat=108&ccb=1-5&_nc_sid=dbeb18&_nc_ohc=-LSQ7HPfW8MAX9EuqeG&_nc_ht=scontent-frt3-1.xx&oh=00_AT99p-XXKxmL5lR-A_q7h3cS8yvhYGa1ZDhz-BKUVyzRQg&oe=61CFCC82
  @width 0.3
\fi

\iusr{Марина Соболевская}
\textbf{Виктория Зайцева} , Чудо! Благодарю вас за подарок на католическое Рождество!) Тоже очень- очень люблю старинные дома, флигели и даже сарайчики.) Эти фото дороги мне.)

\iusr{Виктория Зайцева}
\textbf{Марина Соболевская} А Вы можете поинтересоваться у одного фотографа с Мариуполя, он историей города занимается. Может он Вам больше поможет. Он есть в ФБ. \href{https://www.facebook.com/I.Stanislavsky}{Иван Станиславский}.

\iusr{Марина Соболевская}
\textbf{Виктория Зайцева} , больше спасибо!!)  @igg{fbicon.rose} 

\end{itemize} % }

\iusr{Larysa Karbysheva}
\textbf{Виктория Зайцева} 

А завод Ильича был построен таким образом, что дым на город не шёл. Это потом
построили дома вокруг завода... Завод был построен в 19 веке, с учетом розы
ветров

\begin{itemize} % {
\iusr{Марина Соболевская}
\textbf{Larysa Karbysheva} , не знала. Спасибо за интересную подробность.)

\iusr{Larysa Karbysheva}
\textbf{Марина Соболевская} Хотели строить , конечно же , у моря. Но правительство царское запретило. А вот уже Азовсталь построен у моря...

\iusr{Марина Соболевская}
\textbf{Larysa Karbysheva} , как тщательно в том веке подходили к строительству!)

\iusr{Виктория Зайцева}
\textbf{Larysa Karbysheva} Просто завод был далеко от города в те времена, и мощность не та. А вот уже в 70-е годы достроили много новых цехов, и хоть центр города далековато - 5-6 км, северный ветер доносил сажу быстро. Азовсталь от нас ещё дальше был, но вост... Ещё

\iusr{Марина Соболевская}
\textbf{Виктория Зайцева} , очень интересно! Дедушка мужа до войны был главным инженером Мариупольской электростанции.)

\iusr{Виктория Зайцева}
\textbf{Марина Соболевская} Там я в комментах о муз школах писала...

\iusr{Larysa Karbysheva}
\textbf{Виктория Зайцева} В своё время жили в Мариуполе, около Центрального рынка, а с балкона вдалеке были видны трубы з Ильича. Всегда смотрели, как дым идёт в сторону от города. Но потом же домов настроили вокруг ....

\iusr{Larysa Karbysheva}
\textbf{Виктория Зайцева} Вы думаете он в Киеве белый? Как бы не так....

\iusr{Виктория Зайцева}
\textbf{Larysa Karbysheva} В Киеве, да, а вот за городом, в селах, в Броварах белый. Я как сюда переехала, к снегу привыкала, белому. Это для нас так необычно. И небо голубое, без разноцветных облаков. А сейчас климат сильно изменился, и зимы стали как в Мариуполе. Без снега, тёплые. Иногда с морозом. Но в основном без снега.

\iusr{Larysa Karbysheva}
\textbf{Виктория Зайцева} Вспоминаю своё студенчество, как на лыжах по выходным катались..( в Киеве)

\iusr{Виктория Зайцева}
\textbf{Larysa Karbysheva} Ага, а теперь снега нет. Вот точно зима в Мариуполе. Только сейчас в Марике нет снега, мороз ниже на 4 градуса и ветер дикий. Как сказала подруга, на улице - 17°, а ощущения как - 28°. Круто!

\end{itemize} % }

\end{itemize} % }

\iusr{Виктория Зайцева}

Видела как сушили белье в Марике на 5 этаже. И здесь, в Броварах у одних на 14
этаже увидела такую конструкцию. Называется такое устройство полиспаст,
кажется. С 5 этажа верёвки шли на дерево. А с 14 шли к какому-то выступу на
стене напротив. Наверное верхолаз крепил. Интересное приспособление для сушки
белья.


\iusr{Марина Соболевская}
\textbf{Виктория Зайцева} , которое можно передвигать?)

\iusr{Olena Korkodym}

А в чому прикол з цими мотузками? Невже це зручно? Зараз стiльки моделей
сушарок - i китайських, i eлектричних.

\begin{itemize} % {
\iusr{Марина Соболевская}
\textbf{Olena Korkodym} , 

прикол? История. История города. История семей. История поколений. Ведь это
очень интересно - сравнить, как было раньше? Как жили бабушки? Как изменилась
жизнь родителей? Как эти бытовые задачи решаем мы? А у наших детей и внуков еще
сто раз все изменится. Нет будущего без прошлого. Вы не сами по себе. За вами
стоят двое дедушек и бабушек. Четверо прадедушек и прабабушек. Восемь
прапрадедушек и прапрабабушек. И если мы любим свою семью, своих близких - то
нам хочется узнать о них побольше. Как они жили? Что любили? Что ненавидели?
Какие трудности испытывали? )

\begin{itemize} % {
\iusr{Olena Korkodym}
\textbf{Марина Соболевская} Так, ранiше треба було знайти простiр для бiлизни. А зараз дивлюся на будинок 'гостинку' навпроти -- всього пара балконiв з незавiшеними вiкнами. Це дуже дивно не змiнювати мислення.

\iusr{Марина Соболевская}
\textbf{Olena Korkodym} , чому? Хтось традиційний. А хтось постійно змінюється. Я обожнюю змінюватись. Та минуле мене дуже цікавить та заворожує.)

\iusr{Olena Korkodym}
\textbf{Марина Соболевская} 

Але бiлизна на рiвнi вiкон - це ж, мабуть, незручно для самих мешканцiв
квартир. Водночас, таке популярне задоволення. Що люди хочуть цим сказати? Що
вони приховують вiд мешканцiв сусiднiх будинкiв, завiшуючи балкони сокровенним?
Заради чого це таiнство? Як це - споглядати свiтанок крiзь махровий халат чи
сiру футболку?

\iusr{Марина Соболевская}
\textbf{Olena Korkodym} , можливо, щоб білизна побула на порозі. Це фантастичний спосіб. Я жалкую, що немаю можливості.)

\iusr{Olena Korkodym}
\textbf{Марина Соболевская} Тобто, Ви теж не подiляjeте iдеi, що сушарка пiдлогова на балконi менше впливае на психiку?

\iusr{Марина Соболевская}
\textbf{Olena Korkodym} , хай буде багато найкращих сушарок.) Я за комфорт.)

\iusr{Olena Korkodym}
\textbf{Марина Соболевская} Часом бачу, як люди зранку виходять на балкони. Спираються на пiдвiконня. Труселя-купальники лягають jiм на плечi. А вони п'ють каву, палять та привiтно махають руками мешканцям будинку навпроти. Теж переконана: в усьому винен Кличко  @igg{fbicon.grin} 

\iusr{Галина Чекина}
\textbf{Olena Korkodym} ми все життя сушили на балконі, в потім сусідка почала годувати голубів... тут без коментарів. Застеклили. Так під вікнами зробили автостоянку... яка вже свіжість. Наші страждання завершились відкладанням коштів та придбанням електросушки.
\end{itemize} % }

\iusr{Галина Чекина}
\textbf{Olena Korkodym} 

нормальна сушарка елекролюкс коштує немало. Але зручно, як у казці
@igg{fbicon.face.smiling.eyes.smiling} 

\end{itemize} % }

\iusr{Елена Ярмоленко}

До 1973 года жила на ул. Ленина, весь двор сушил белье во дворе. У каждого были
свои веревки и свое место, бабушка всегда следила из окна за бельем, чтоб никто
руками не трогал, веревки растягивали на всю ширину двора. Бабушка рассказывала
историю, что в 1957 году мама пришла и сообщила радостную новость, что выходит
замуж, за моего папу и пока на кухне все плакали, белье во дворе сняли

\begin{itemize} % {
\iusr{Марина Соболевская}
\textbf{Елена Ярмоленко} , браво улице Ленина. Она не обращала внимание на политику партии.) Утрата белья- сейчас это трудно представить. Но такое было время наших бабушек.)

\iusr{Олена Казакова}
\textbf{Елена Ярмоленко} Ще й як крали чашки з кав'ярень! Ви що не пам'ятаєте, як у чашок навмисно відбивали ручки саме для того, щоб ТАКУ чашку вже не було б бажання вкрасти

\iusr{Марина Соболевская}
\textbf{Олена Казакова} пам'ятаю сколоті чашки. Але не усвідомлювала, що це - навмисно! От наївна!)))
\end{itemize} % }

\iusr{Yevheniia Doroshuk}
И немножко на сердак....

\begin{itemize} % {
\iusr{Марина Соболевская}
\textbf{Yevheniia Doroshuk} , на чердак?)

\iusr{Yevheniia Doroshuk}
\textbf{Марина Соболевская} Да на чердак. Извините,.
\end{itemize} % }

\iusr{Мария Маслова}
У нас во дворе соседнего дома до сих пор так сушат:-)

\begin{itemize} % {
\iusr{Марина Соболевская}
\textbf{Мария Маслова} , я думала, такого уже нет. И мне было радостно, что лишь вид белья во дворе мог перенести меня мысленно в детство.)

\iusr{Larysa Karbysheva}
\textbf{Мария Маслова} И у меня во дворе сушат.

\end{itemize} % }

\iusr{Елена Хлистунова}
\enquote{Добрый вечер вам в окно, где вы сохните белье...} @igg{fbicon.grin} 

\iusr{Марина Соболевская}
\textbf{Елена Хлистунова} , спасибо! Количество вариаций растет!)

\iusr{Олена Харченко}
Бетонну площадку у нас зайняли авто

\iusr{Ирина Козина}

штанги такие во дворе есть, и периодически летом на них сохнет белье.
рачительные хозяйки сидят неподалеку на лавочке и бдят. некоторые бдят из
окошка и устрашающе кричат на мимопроходящих, заподозренных ими в покушении на
чужие подштанники

\iusr{Марина Соболевская}
\textbf{Ирина Козина} , представляю!)

\iusr{Ирина Кучер}
На чердаке , на гилляке шоб не украли те моряки

\iusr{Марина Соболевская}
\textbf{Ирина Кучер} , надо же! Еще один вариант.) Спасибо!)

\iusr{Iryna Huskova}

Кстати я недавно просто что «здрасьте вам через окно» значит тип ну и бред же
ты говоришь  @igg{fbicon.smile}  меня этому научили мои Подольские бабушки ))

\iusr{Марина Соболевская}
\textbf{Iryna Huskova} , супер! Буду знать.)

\iusr{Ольга Dzhun}
У духовке на веровке, на чердаке на ломаке

\iusr{Марина Соболевская}
\textbf{Ольга Dzhun} , спасибо за очередной вариант.)

\iusr{Almira Yusupov}

A у нас в Бруклине, между окнами домов выходящих в колодец была веревка на
ролике, там и сушили. надо было правда иметь хорошего соседа с другой стороны

\begin{itemize} % {
\iusr{Olena Korkodym}
\textbf{Almira Yusupov} У вас там до сих пор никто на английском не говорит? @igg{fbicon.grin} 

\iusr{Almira Yusupov}
\textbf{Olena Korkodym} никто кроме меня @igg{fbicon.face.smiling.eyes.smiling} . В основном старые не могли выучить а все кто работал выучили язык. Между собой конечно по -русски. Даже голосовать можно по русски. Но это тут нормально, Я работаю в Латиноамериканском районе, там все по испански, я тоже выучила элементарный испанский. В китайском районе между собой по китайски. За это и люблю Нью Йорк. Все культуры сохраняються

\iusr{Almira Yusupov}

\ifcmt
  ig https://scontent-frt3-1.xx.fbcdn.net/v/t39.30808-6/269870135_2535706029900252_1643166479412429205_n.jpg?_nc_cat=108&ccb=1-5&_nc_sid=dbeb18&_nc_ohc=i4NQAglVxS4AX-flmDC&_nc_ht=scontent-frt3-1.xx&oh=00_AT_sXFgY2Dk115v4S0vDoqBiEbP0b9iFGs6IrG10y7LdZw&oe=61CE60B1
  @width 0.3
\fi

\iusr{Марина Соболевская}
\textbf{Almira Yusupov} , чудесно!) Помощь, дружба и взаимовыручка.)

\end{itemize} % }

\iusr{Наталія Крюкова}

У нас во дворе жители трехэтажки точно также сушат всё, что постирано. От простыней до бабушкиных панталон.
Верёвки привязывают между деревьями.
А на бетонных плитах бывшей площадки для сушки белья теперь стоянка авто жителей дома.
Столбы для веревок давно срезаны и ушли во вторсырье.

\begin{itemize} % {
\iusr{Марина Соболевская}
\textbf{Наталія Крюкова} , спасибо, что поделились! По фрагменту складывается история городского быта.)

\iusr{Наталія Крюкова}
\textbf{Марина Соболевская} 

Когда несколько лет назад я меняла утепление на балконе с деревянных уже
отработавших рам на металлопластик, первое, что предложили мне рабочие, это
телескопические крепления веревок для сушки белья...

А для меня это нонсенс.

Особенно учитывая количество цветов и коллекцию кактусов...

\ifcmt
  ig https://scontent-frx5-2.xx.fbcdn.net/v/t39.1997-6/s480x480/68906835_916980898645292_1075146209586315264_n.png?_nc_cat=1&ccb=1-5&_nc_sid=0572db&_nc_ohc=0t35mwH6P-QAX8moRPT&_nc_ht=scontent-frx5-2.xx&oh=00_AT8p6p1FCc6W_-MiaYjTHVxzm77RF8C2CgdXV4Ln8gcrXQ&oe=61CEB912
  @width 0.2
\fi

\end{itemize} % }

\iusr{Виктория Русакова}
А у меня вид белья на верёвках тоску вызывает.

\begin{itemize} % {
\iusr{Almira Yusupov}
\textbf{Виктория Русакова} а у меня вызывает философски мысли о чистоте и новизне

\iusr{Марина Соболевская}
\textbf{Almira Yusupov} , вот у меня тоже мысли о чистоте. Вот если бы грязным бельем над головой трусили - это был бы ужас.)))

\iusr{Марина Соболевская}
\textbf{Виктория Русакова} , вполне возможно. У вас с этим видом связаны грустные события? Или у ваших предков. Даже подсознательно...
\end{itemize} % }

\iusr{Irina Kolomiec}
Дворик в Киеве возле цирка, всегда улибаюсь из- за белья сзади моей родни

\ifcmt
  ig https://scontent-frx5-1.xx.fbcdn.net/v/t39.30808-6/269744804_968862623988140_8942526538205517199_n.jpg?_nc_cat=111&ccb=1-5&_nc_sid=dbeb18&_nc_ohc=c0vWYz-YxcUAX_yNOn5&_nc_ht=scontent-frx5-1.xx&oh=00_AT_b3j8Et_vjL-vqn0Eoeq0U4XvJJiV9BeHb19M8JTNlZg&oe=61CEE4DE
  @width 0.3
\fi

\iusr{Марина Соболевская}
\textbf{Irina Kolomiec} , какое изумительное фото! Восторг! Огромное спасибо!)

\iusr{Галина Гурьева}
Соседка до сих пор во дворе сушит. Цокольный этаж, однушка...

\iusr{Марина Соболевская}
\textbf{Галина Гурьева} , и дома белье не задыхается.)

\iusr{Людмила Удкевич}
\textbf{Галина Гурьева} и у нас во дворе хрущёвки на Нивках жильцы первого этажа сушат на бельевой площадке

\iusr{Viktoria Terpylo}

Я помню у нас на Лесном во дворе обычной 9-этажки \enquote{чешки} была такая площадка
как на первом фото с такими же бетонными столбами и обсажена облепихой. Там
женщины сушили в основном простыни. А когда она была свободной мы играли в
Пионер-болл.

А ещё я ходила в школу и в соседнем доме как раз проходя через такую площадку
надо было наклоняться под простынями. У кого был балкон 3м а у кого то и на пол
дома. А ещё использовали крыши и вешали белье там. В сёлах сушили на чердаке под
крышей, когда был сезон дождей

\begin{itemize} % {
\iusr{Марина Соболевская}
\textbf{Viktoria Terpylo} , очень интересно! Спасибо, что поделились! Про сельский быт - не знала.)

\iusr{Галина Гурьева}

Я еще застала то время, когда свекровь моего мужа на чердак с бельем
отправляла. Я еще думала~чего он туда лазит постоянно? Белье щупал, высохло или
нет)))


\iusr{Марина Соболевская}
\textbf{Галина Гурьева} , изумительная семейная история!)
\end{itemize} % }

\iusr{Леся Сагайдачная}
У нас так до сих пор сушат

\iusr{Марина Соболевская}
\textbf{Леся Сагайдачная} , традиции сохраняются.)

\iusr{Светлана Светлана}
Да, всё так и было...

\iusr{Марина Соболевская}
\textbf{Светлана Светлана} , спасибо за одномоментность.)

\iusr{Olga Rossokha}

Там ещё были такие слова - \enquote{-Где вы сохнете бельё? -Мы? У кладовке на веровке,
чтоб не стыбрили воровки. А вы? - Мы на чердАке на ломаке, чтоб не слямзили
собаки}.))

\begin{itemize} % {
\iusr{Марина Соболевская}
\textbf{Olga Rossokha} , спасибо за еще один вариант фольклора.)

\iusr{Olga Rossokha}
\textbf{Марина Соболевская} , да, вспомнила, что наши мальчишки в школе пели эту песенку, шутники.

\iusr{Марина Соболевская}
\textbf{Olga Rossokha} , как и мой папа.)))
\end{itemize} % }

\iusr{Зоя Лабунская}
На двОре?, - Но там же воры!, - На чердАке на ломАке!  @igg{fbicon.laugh.rolling.floor}{repeat=3} 

\begin{itemize} % {
\iusr{Марина Соболевская}
\textbf{Зоя Лабунская} , а что такое \enquote{ломака}?)

\iusr{Зоя Лабунская}
\textbf{Марина Соболевская} я так понимаю, палка. Сама в детстве не могла понять.

\iusr{Марина Соболевская}
\textbf{Зоя Лабунская} , спасибо!)
\end{itemize} % }

\iusr{Zoya Golubeva}

Згадваю дитинство, коли мама заносила попрану білизну з морозу в хату, яки
стояв запах свіжісті, це відчуття забути не можливо. А влітку, білизна з
ароматом трав. Як це чудово.


\iusr{Марина Соболевская}
\textbf{Zoya Golubeva} , так! Як ви гарно розповіли! Дякую!)

\iusr{Алена Лозовая}

Где вы сохнете белье? В духовке на вероака,, а вы? на чердаке на ломака @igg{fbicon.face.grinning.smiling.eyes} так
говорила подруга моей мамы многого лет, когда мы действительно сушили его на
чердаке в доме на ул. Саксаганского @igg{fbicon.face.tears.of.joy} 

\begin{itemize} % {
\iusr{Марина Соболевская}
\textbf{Алена Лозовая} , очень интересно! А что такое \enquote{на ломаке}?)

\iusr{Галина Гурьева}

Ломака~такая большая палка, перекладина. Ломаку больше использовали для драк
местные хулиганы, папа рассказывал.

\iusr{Марина Соболевская}
\textbf{Галина Гурьева} , спасибо!) Буду знать.

\iusr{Алена Лозовая}
\textbf{Марина Соболевская} да, ломака это деревянная перекладина на которой сушили белье
\end{itemize} % }

\iusr{Людмила Мозговая}
Летом на балконе.... у нас чистый двор...

\iusr{Марина Соболевская}
\textbf{Людмила Мозговая} , это большая удача.)

\iusr{Мария Каменцова}

А я вот терпеть не могу, чтобы на мое белье и прочую стирку глазели
посторонние.  @igg{fbicon.smile}  Поэтому в нашей большой квартире высоко натянута по коридорам
невидимая лёска. Стирка всегда происходит на ночь, утром снимаю высохшее белье,
и никому оно не мешает. Ветерком оно – да, не пропитывается, но после глажки
пахнет всё равно замечательно! И о покраже никакой речи быть не может...  @igg{fbicon.wink} 

\begin{itemize} % {
\iusr{Марина Соболевская}
\textbf{Мария Каменцова} , 

от поколения к поколению быт меняется.) У вас свой собственный семейный лайфхак
с невидимой леской.) С домашней сушкой я даже не задумывалась, что белье раньше
крали.))

\iusr{Мария Каменцова}
\textbf{Марина Соболевская} . Видите, как ярко проявляется у каждого индивидуальность! И это прекрасно!  @igg{fbicon.smile} 

\iusr{Марія Рябоконенко}
\textbf{Мария Каменцова} 

А я терпеть не могу развешенное по квартире белье, поэтому сушу на раскладной
сушке на балконе и не выше закрытой части. Во времена, когда не было стиральной
машины, постельное сдавали в прачечную, а белье нательное сушили на балконе, но
ниже застеклённый части.

\iusr{Мария Каменцова}
\textbf{Марія Рябоконенко} . 

Как интересно поговорить на эту тему и обсудить проявления смекалки у различных
хозяек! У каждого свои приоритеты, так здорово узнать о них!  @igg{fbicon.smile}  Я тоже
прекрасно помню, что постельное белье все сдавали в прачечную, стоило это
копейки, и возвращалось оно белоснежное, хрустящее, выглаженное и душистое...

\end{itemize} % }

\iusr{Олена Казакова}

А у моей мамы был такой вариант: - Где ви сохнете билльо? - Кусочек - на
вировке, кусочек - на духовке, кусочек - на чегдак.

\iusr{Марина Соболевская}
\textbf{Олена Казакова} . спасибо за ваш вариант.)

\iusr{Татьяна Коризно-Мирская}

Ми? В духовке, на веровке и немножко на себе

\iusr{Марина Соболевская}
\textbf{Татьяна Коризно-Мирская} , отлично. Спасибо за еще один вариант.)

\iusr{Володимир Денисов}
Жил на Брест-Литовськом 20/1 и наш балкон как раз вьіходил на проспект, белье сушили строго ниже уровня перил

\begin{itemize} % {
\iusr{Марина Соболевская}
\textbf{Володимир Денисов} , строго.) Спасибо!) Так и предполагала, что зависело от того, расположен ли дом на стратегической улице.)

\begin{itemize} % {
\iusr{Володимир Денисов}
\textbf{Марина Соболевская} бьіло много плюсов, к примеру бьіло видно 9 точек салютов, на парад не ходили так как вся техника и многие колонньі проходили у нас под окнами, но в остальном с началом 90х стал жах от машин и мотоциклов, а так же от копоти. Уехал от туда в 2000 году

\iusr{Марина Соболевская}
\textbf{Володимир Денисов} , круто! Вы салют с балкона смотрели!)

\iusr{Володимир Денисов}
\textbf{Марина Соболевская} да

\iusr{Марина Соболевская}
\textbf{Володимир Денисов} , сажа - это и наша беда. На балконе белье не сушим.)
\end{itemize} % }

\iusr{Наталія Крюкова}
\textbf{Володимир Денисов} А я в тогда 49, сейчас стал 47???напротив киностудии.
Да не дай бог, чтобы тряпки развивались!
Кухня, коридор и ванная.
Или ниже перил.

\end{itemize} % }

\iusr{Александра Рудницкая}

Родилась и жила до замужества в Ивано-Франковске. В раннем детстве жили в
старом польском особняке, поделенном на три квпртиры. Чердак там был шикарный,
через все пространство особняка, пол дощатый. Там сушили белье, веревки не
снимали.

В 196.. не помню каком году нам и соседям дали квартиры в новостройке, а
особняк отдали многодетной семье (попробуйте сейчас так). В новом дворе были
стойки для сушки белья, веревки свои натягивали, один раз мама поленилась снять
веревку, чтобы утром не натягивать снова, но за ночь веревка исчезла. До сих
пор помню ее пожелание: \enquote{чтоб ты на ней повесился, гад}.

\begin{itemize} % {
\iusr{Марина Соболевская}
\textbf{Александра Рудницкая},

спасибо! Это очень интересно! Мы часто недооцениваем наш быт. А он и является
сердцем семьи. В Ивано- Франковске жил мой отец в детстве. Тоже в старинном
особняке на несколько квартир. Рассказывал, что очень красивые дома были. Адрес
забылся. А мне хотелось бы их увидеть, если сохранились.)

\begin{itemize} % {
\iusr{Александра Рудницкая}
\textbf{Марина Соболевская} Дома сохранились, по крайней мере некоторые. Это очень комфротные квартиры, большой площади, с высокими потолками, большими окнами, и всегда окружены садом. Мы жили на улице Гаркуши.

\iusr{Марина Соболевская}
\textbf{Александра Рудницкая} , спасибо! Посмотрю на гугл карте.  @igg{fbicon.tulip} 
\end{itemize} % }

\end{itemize} % }

\iusr{Natalia Zabrovski}

В нашем дворе, в Днепровском районе, на улице Бойченко (нынче Соловьяненко),
была такая площадка со столбами и между ними железные пруты натянуты , именно
для сушки белья. Никто не воровал, потому что грозная тётя Катя, соседка, днями
висела в окне пятого этажа, и орала на весь двор, если кто-то подходил к белью,
и неважно её это белье висело или соседское, боялись все! А зимой! Хрустящее,
стоячее от мороза, пахнущее свежестью, небом, снегом, зимнем солнцем, со
снежинками .... белее белого белье! Волшебно! Теперь это быль!

\iusr{Марина Соболевская}
\textbf{Natalia Zabrovski} , изумительная история!))) Благодарю вас!) Тетю Катю и я теперь буду помнить.)

\iusr{Марина Ковешникова}
Да, на Отрадном так еще сушат:)

\begin{itemize} % {
\iusr{Марина Соболевская}
\textbf{Марина Ковешникова} , по старой киевской традиции.)

\iusr{Марина Ковешникова}
\textbf{Марина Соболевская} и при всем том, даже, молодые хозяйки:)

\iusr{Марина Соболевская}
\textbf{Марина Ковешникова} , вот это удивительно! Обычно молодыми мы хотим делать все не так, как наши родители.) И меня глубоко трогает желание потомков сохранить семейные истории и традиции.)

\iusr{Марина Ковешникова}
В хрущевках на 1ом этаже нет балконов:). Если не пристроят.
\end{itemize} % }

\iusr{Elena Tartakovskaya}
Какой милый текст! Спасибо!

\iusr{Марина Соболевская}
\textbf{Elena Tartakovskaya} , благодарю Вас!)

\iusr{Татьяна Санько}

Да, мы жили на первом этаже и сушили белье во дворе, да, на шее прищепки на
веревочке. Действительно, за бельем надо было присматривать. Пару раз нам белье
испачкали мальчишки- подростки палками. Но воровства не было. Стирали руками,
вываривали, синькой подкрашивали, крохмалили. Все это было.

\iusr{Марина Соболевская}
\textbf{Татьяна Санько} , спасибо за интересные бытовые подробности! Это увлекательная история вашей семьи.)

\iusr{Stella Stavitsky}
Точно так !

\ifcmt
  ig https://scontent-frx5-1.xx.fbcdn.net/v/t39.1997-6/s180x540/245811551_429232528665793_9146110549457790740_n.png?_nc_cat=105&ccb=1-5&_nc_sid=ac3552&_nc_ohc=L7nIUUCtXGwAX88EnKA&_nc_ht=scontent-frx5-1.xx&oh=00_AT9iMeuEhVgLa3h2hHyV6qBnAZaBW_5HcHSXs79iYlCmGg&oe=61CED2E4
  @width 0.1
\fi

\iusr{Наталия Платонова}

У нас были два балкон: собственно балкон и лоджия с кирпичеыми колоннами. Мы на
велосипеде катались. И вот там сохло бельё и на ветерке, и на морозце. Запах
умопомрачительный!


\iusr{Марина Соболевская}
\textbf{Наталия Платонова} , это надо хоть раз испытать, чтобы осознать всю прелесть.)

\iusr{Алина Гриненко}

Я помню в детстве мама сушила белье во дворе,а меня назначали дежурной
,смотреть за бельем. Я созывала подружек и мы играли в резинки,пока все не
высохнет, было очень весело. Теплые воспоминания

\begin{itemize} % {
\iusr{Марина Соболевская}
\textbf{Алина Гриненко} , спасибо за ваши светлые воспоминания.)

\iusr{Вікторія Святненко}
\textbf{Алина Гриненко} какие тёплые воспоминания... Все таки мы были послушными, попробуй внуку моему сказать, что б белье стерёг! @igg{fbicon.laugh.rolling.floor} 

\iusr{Алина Гриненко}
\textbf{Вікторія Святненко} это было так весело) еще мы первоклашки и второклашки умудрялись втянуть в эту игру мальчишек старшекласников ,которые собирались на лавочке у подьезда,и что характерно они послушно держали нам резинку пока мы прыгали,как то проще все были в общении.

\iusr{Надежда Владимир Федько}
\textbf{Алина Гриненко} Чого тільки не зробиш заради дівчат...))
\end{itemize} % }

\iusr{Светлана Светлана}

У нас был маленький дворик на Волынской. В центре двора был тонкий железный
столб. От столба во все стороны тянулись веревки, которые привязывались к
деревьям по периметру площадки. Каждая хозяйка знала свою верёвку. Если своих
веревок на стирку не хватало, то спрашивали разрешения чтоб вывесить стирку на
соседних. Прикольно было, когда постельное белье висело как паруса. Ещё верёвку
по центру подпирали палкой, чтоб не прогибались к земле. Классно было под ним
бегать, пока хозяйка не увидит и не накричит, чтоб под \enquote{стиркой не бегали}.

Ещё помню как зимой мама заносила с мороза замёрзшие простыни и пододеяльники.
Как они пахли свежестью!

И да,таки был у нас случай воровства. Когда одна по тем меркам крутая семья
решила импортные синтетические трусики недельки просушить на верёвке. Потом все
смеялись,что незачем хвастать. И да,такие личные вещи с тех времён приучена
сушить под чем то более обыденным. Прописано в крови что белье на показ не
выставлять.

Ещё классно было зацепиться за верёвку и покататься. Но потом наступала кара.

\begin{itemize} % {
\iusr{Марина Соболевская}
\textbf{Светлана Светлана} , чудесные воспоминания! Целая семейная история! Благодарю вас!)

\begin{itemize} % {
\iusr{Светлана Светлана}
\textbf{Марина Соболевская} 

да, у нас во дворе выросло четыре поколения детей ( я из последнего). Вначале
жили в бараке, потом строили дом, а потом наш дом пошел под снос и наш двор
расселили. А так ,да, на праздники во дворе выставляли длинный стол.... И дети
знали всех родственников соседей... Таких дворов уже нет.

Недавно заходила, хотела увидеть свой дом, надеялась что его переоборудовали под
склад. ина месте нашего дома яма @igg{fbicon.face.pleading}  От двора осталось пару домиков одноэтажек на
четыре семьи, их переоборудовали под склад. А наш двухэтажный кирпичный на
восемь семей снесли.


\iusr{Марина Соболевская}
\textbf{Светлана Светлана} , многое из моего детства тоже в Киеве разрушено. Очень больно. Но что поделать, столица всегда будет перестраиваться.
\end{itemize} % }

\end{itemize} % }

\iusr{Надежда Владимир Федько}

Колись давно-давно... У 70-х... В нашому дворі була спеціальна площадка для
сушки білизни. Заасфальтована та з стовпчиками для вірьовок. Але на початку
90-х, коли літні люди почали відходити, дисципліна похитнулася і нахабні діти,
що були погано виховані, \enquote{приватизували} двір для стоянки своїх авто. Зараз у
дворі відстоюються більше 30 авто. А стовпчики позбивали, щоб не заважали
паркуватися. Один, правда, ще вцілів.

\ifcmt
  ig https://scontent-frt3-1.xx.fbcdn.net/v/t39.30808-6/269880452_4841181902607896_1031701839680260760_n.jpg?_nc_cat=107&ccb=1-5&_nc_sid=dbeb18&_nc_ohc=-RBL_VkHg0EAX_FS_WW&_nc_ht=scontent-frt3-1.xx&oh=00_AT9djg-1einCoDLQ9bVyQEK1jHOp3_2v08AKQWB-5Cyj5Q&oe=61CEAE61
  @width 0.2
\fi

\begin{itemize} % {
\iusr{Марина Соболевская}
\textbf{Надежда Владимир Федько} , дякую за фото. Цей стовпчик, як пам'ятка давній традиції.)

\iusr{Надежда Владимир Федько}
\textbf{Марина Соболевская} Залишається повісити табличку: \enquote{Тут у ** - **роках сушилися білизна видатних людей цього подвір'я (перелік)... }

\iusr{Марина Соболевская}
\textbf{Надежда Владимир Федько} , геніально!)
\end{itemize} % }

\iusr{Александр Асатуров}
Как говаривал Голохвастов....одно слово и уже не тот фасон) я про свистнули

\iusr{Марина Соболевская}
\textbf{Александр Асатуров} , кстати да.)))

\iusr{Юрий Панчук}

Панталоны бы наверно не стырили, а вот фирменные джинсы 100\%. У нас во дворе и
сейчас некоторые жильцы первых этажей сушат.

\begin{itemize} % {
\iusr{Марина Соболевская}
\textbf{Юрий Панчук} , храбрые соседи!)

\iusr{Татьяна Петрюк}
\textbf{Юрий Панчук} 

моя сокурсница перед выходом на новую работу постирала всю свою верхнюю одежду
( 1981 или 1982 год ) в надежде каждый день являться в чисто мужской коллектив
в новеньком и свеженьком ...: Повесила все на ночь сушиться на балконе первого
этажа ( Минский массив)- за ночь абсолютно все сняли ...

Пришла Валя на работу в спортивном костюме ..... и такое было ...

\iusr{Светлана Светлана}
\textbf{Татьяна Петрюк} 

Живем на первом этаже. На окнах стоят решетки. Стекли балкон, сняли решетки и
мастера на пару часиков уехали. Я осталась одна дома, без решёток на окнах...
Блин, стрёмно было...


\iusr{Татьяна Петрюк}
\textbf{Светлана Светлана} я всегда живу на первом этаже  @igg{fbicon.face.wink.tongue} ( частный дом) .. привычно ...

\end{itemize} % }

\iusr{Маргарита Воронова}

История))

\iusr{Валентина Світельська}
А в нашому дворі на Сирці чотири залишилось.

\iusr{Марина Полуэктова}

Сушили во дворе, конечно. И не обязательно в Киеве).
И сейчас сушим! Ленор отдыхает

\iusr{Марина Соболевская}
\textbf{Марина Полуэктова} , Ленор и пытается заменить альпийскую свежесть.))

\iusr{Defektolog Logoped}

В Одессе до сих пор сушат в старых домах.  @igg{fbicon.grin} В моем дворе такие сушилки украли и
сдали в металлолом в начале 2000г.

\begin{itemize} % {
\iusr{Марина Соболевская}
\textbf{Defektolog Logoped} , спасибо! Оказывается. это была общепринятая забота о хозяйках и их постирушках.) Я думала, это киевская особенность.)

\iusr{Defektolog Logoped}
\textbf{Марина Соболевская} 

Нет, видимо украинская. В других областных центрах тоже видела:Николаев,
Днепр, Одесса, Харьков.


\iusr{Марина Соболевская}
\textbf{Defektolog Logoped} , очень интересная особенность! Значит в те годы повсюду были обрудованы сушилки для проживающих.)

\iusr{Defektolog Logoped}
\textbf{Марина Соболевская} 

Да. При СССР и планировка дворов, квартир была одинаковая: сталинка (старая и
новая) хрущевка, брежневка, чешка и т. д.


\iusr{Марина Соболевская}
\textbf{Defektolog Logoped} , это понимаю.) Но оборудованное место для сушки белья? Чтобы не на деревьях и кустах - это была программа партии или самодеятельность жильцов?)

\iusr{Defektolog Logoped}
\textbf{Марина Соболевская} 

Да, видимо программа партии. в моем дворе во дворе сушили те, у кого не было
балкона(жильцы 1этажа и жильцы цоколя). Мы сушили на балконе. Я выросла в
сталинке.


\iusr{Марина Соболевская}
\textbf{Defektolog Logoped} , спасибо!)

\iusr{Марина Соболевская}
\textbf{Defektolog Logoped} , в нашей чешке, кажется, все были с балконами. И первый этаж - тоже. А сушка оборудована во дворе на наш дом.

\iusr{Defektolog Logoped}
\textbf{Марина Соболевская} да, чешки с балконами были


\end{itemize} % }

\iusr{Любовь Белоцерковец}

А ведь правда!! Тогда все сушили на улице! Помню, одни знакомые прямо с Окна к
дереву умудрились протянуть веревку, которая была двойной. Белье повесили -
потянули за один край веревки, и по колесиках белье растягивалось на большое
расстояние... Я тоже хотела, но у нас под окном на было такого огромного
дерева)))

Спасибо за рассказ... как будто в прошлое вернулась)))

\ifcmt
  ig https://scontent-frx5-2.xx.fbcdn.net/v/t39.1997-6/p600x600/106702127_953852515077106_5584524183393874628_n.png?_nc_cat=1&ccb=1-5&_nc_sid=0572db&_nc_ohc=TRmk_L3bNCwAX9FpYp-&_nc_ht=scontent-frx5-2.xx&oh=00_AT8me4bTn_Zpr0JdQqmIAa9adM2ApJ-iNEqppH3wY179pA&oe=61CE4BB0
  @width 0.2
\fi

\iusr{Марина Соболевская}
\textbf{Любовь Белоцерковец} , замечательная история! Спасибо, что поделились!)

\iusr{Dima Bilan}
Да ну так не только в Киеве делали, тут уж не сочиняйте

\begin{itemize} % {
\iusr{Марина Соболевская}
\textbf{Dima Bilan} , отвечаю за каждую букву!)))) На вранье еще не ловили.) Суть в том, что в столице этот вопрос строго регламентировали.) В комментариях неоднократно это подтвердили. А в регионах каждый мог вести свой быт, как хотел.)

\iusr{Dima Bilan}
\textbf{Марина Соболевская} вы о чем? Я про белье во дворе на веревках.
\end{itemize} % }

\iusr{Наталія Сігалова}

Очень весело сушат белье в Неаполе. Не удержалась сфотографировала многоэтажный
дом лохматый от развевающихся за каждым балконом и окном штанов и рубашек.

\iusr{Марина Соболевская}
\textbf{Наталія Сігалова} , сушка белья в Италии меня очень впечатлила! Вот уж люди не заморачиваются, а наслаждаются жизнью!)

\iusr{Ирина Иванченко}

Всё это весело весьма, спасибо.
Всю жизнь в частном секторе проживала, проблем таких не возникало, но у
знакомых, знаю, были, примерно так все и сушили

\iusr{Марина Соболевская}
\textbf{Ирина Иванченко} , спасибо!)

\iusr{Mila Shtermer}

Ни разу ничего не украли, а сейчас бы не рискнула повесить вещи на улице без
присмотра. Постельное даже на ночь оставляли. Чоколовка


\iusr{Марина Соболевская}
\textbf{Mila Shtermer} , удивительно! Сейчас я тоже, наверное, побоюсь оставить белье на улице.

\iusr{Надежда Владимир Федько}

Білизну крали в основному злодюжки, щоб зекономити гроші на бухло. Це не дивує.

Що мене дивує... В 70-ті люди жили досить бідно, але стакани з автоматів;
кружки з квасних та пивних цистерн; чашки з кав'ярень не крали.

Як прийшли 90-ті, то красти стали навіть півлітрові баночки, в які наливали
пиво. І у баночок стали фарбувати донця, щоб їх не приймали у пунктах прийому
склотари.

\begin{itemize} % {
\iusr{Марина Соболевская}
\textbf{Надежда Владимир Федько} , дякую! Цікаві спогади! Про фарбовані баночки не чула. А такий колоритний момент.)
\end{itemize} % }

\iusr{Владимир Дубровский}
Не только в Киеве, вспомните итальянские фильмы, там только это белье и показывали. Неореализм.

\begin{itemize} % {
\iusr{Марина Соболевская}
\textbf{Владимир Дубровский} , верно!) И белье это было символом, олицетворением. Может - дома, может - детства, может мамы..)
\end{itemize} % }

\iusr{Олена Андурова}

Мама в 60-х годах сушила постельное белье во дворе, веревки привязывали к
деревьям, у каждого были длинные палки, подпирать провисшие веревки с
бельем. дедушка сидел на стуле под деревом, читал газету и охранял. на балконе
пришлось поставить щитки, чтобы не была видна с улицы сушащаяся мелочь. сейчас
сушу на балконе. Как вкусно пахнет белье с мороза!

\begin{itemize} % {
\iusr{Марина Соболевская}
\textbf{Олена Андурова} , 

спасибо за чудесную семейную историю! Очень интересно! Представила себе вашего
дедушку с газетой, который сторожил белье.) А запрет на балконную сушку выше
ватерлини и вас коснулся. Видимо, поэтому и щит от посторонних взглядов.)

\end{itemize} % }

\iusr{Татьяна Тамакова}
На ветерке самая вкусная...!!!!

\iusr{Марина Соболевская}
\textbf{Татьяна Тамакова} , спасибо!)  @igg{fbicon.tulip} 

\iusr{Инна Радченко}
С мороза белье и вправду духмяное, никакая отдушка так не пахнет:)

\iusr{Марина Соболевская}
\textbf{Инна Радченко} , надо будет этой зимой поставить эксперимент!) В выходной, когда машин поменьше.))

\iusr{Ирина Петрова}

Спасибо! Да, согласна на все 200\% @igg{fbicon.wink}  ничего не может быть вкуснее белья с мороза.

Теперь, правда, и морозы не балуют.

Но, только на открытой полностью лоджии всегда сушу бельё. А вот на
Заньковецкой и правда, развешивали только ниже перил, балкон папа обтянул
каким-то серым материалом (даже не скажу, что это было..). Натягивали три
веревки во всю длину 4-метровую. Высушили - верёвки смотали. А в уголках были
постоянные небольшие верёвочки для мелочи). Дворов у нас там не было.

\begin{itemize} % {
\iusr{Марина Соболевская}
\textbf{Ирина Петрова} , спасибо за подробный рассказ! Очень интересно узнать традиции вашей семьи!) Вот и у вас были правила сушки белья на балконе. Думаю, это касалось всей улицы.)

\begin{itemize} % {
\iusr{Татьяна Петрюк}
\textbf{Марина Соболевская} это касается не улицы  @igg{fbicon.wink} .., это культура людей ...

\iusr{Марина Соболевская}
\textbf{Татьяна Петрюк} , 

не соглашусь с вами.)  @igg{fbicon.wink}  В милиции были разнарядки, следить за тем, чтобы белье
сушили ниже уровня перил. Дружинники ходили по квартирам и делали замечания.
Постепенно это вошло в привычку горожан.Распоряжение спускалось сверху. Я не
имею ввиду нижнее белье. Его действительно не вывешивали публично.)


\iusr{Татьяна Петрюк}
\textbf{Марина Соболевская} 

не буду спорить, тк зайдём в дебри .. но кто то же .. начал вешать веревки
вверху балконов и никто по сей день не объясняет , что это недопустимо  @igg{fbicon.face.pensive} 


\iusr{Ирина Петрова}
\textbf{Татьяна Петрюк} 

тут все дело в районе города. На Крещатике, в Пассаже, на КарлаМарла дворники
могли сделать замечание. У нас такое было. Потому что по улице проезжали те,
которые трудились на благо наше на Орджоникидзе, на Октябрьской революции.
\enquote{Потемкинские деревни} выдуманы не нами.

У меня тут есть публикация про приключения нашего балкона в ноябрьские дни)
\end{itemize} % }

\end{itemize} % }

\iusr{Людмила Смирнова}
\enquote{Где вы сохнете белье?} - это я отлично помню

\ifcmt
  ig https://scontent-frx5-2.xx.fbcdn.net/v/t39.1997-6/p480x480/105941685_953860581742966_1572841152382279834_n.png?_nc_cat=1&ccb=1-5&_nc_sid=0572db&_nc_ohc=TL6XGv0Fm3kAX9HsAKh&_nc_ht=scontent-frx5-2.xx&oh=00_AT9Q_7s4CMMi1ZKkn3DJDyxPDlXQSE35E-2X36JQIz-NFQ&oe=61CDBE0B
  @width 0.2
\fi

\iusr{Nina Steseva}
И продолжение: а мы на чердаки, чтобы не было бардаки.

\iusr{Марина Соболевская}
\textbf{Nina Steseva} , спасибо!)


\end{itemize} % }
