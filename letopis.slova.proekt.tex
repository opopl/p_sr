% vim: keymap=russian-jcukenwin
%%beginhead 
 
%%file slova.proekt
%%parent slova
 
%%url 
 
%%author 
%%author_id 
%%author_url 
 
%%tags 
%%title 
 
%%endhead 
\chapter{Проект}
\label{sec:slova.proekt}

%%%cit
%%%cit_head
%%%cit_pic
%%%cit_text
Думаю, мало кто в нашей стране понял, что в лице правящей элиты (после
женевской встречи Байдена и Путина) Украина приняла окончательное решение – к
какому \emph{глобальному проекту} она примкнула.  Итак, из двух
\emph{глобальных проектов}: - во-первых, \emph{проект} «трансгуманизм», где
лидеры США и ЕС будут формировать человека-потребителя с помощью
транснациональных цифровых гигантов; - во-вторых, \emph{проект} «социальный
рейтинг» под руководством лидеров КНР, в ходе реализации которого всем
действиям китайцев будут присуждаться положительные и отрицательные оценки, а
далее на их основе – рассчитываться конкретный рейтинг.  Ну, в советское время
что-то аналогичное вспоминается в связи с формированием заслуг и достижений для
принятия в пионеры, затем в ряды ВЛКСМ и, наконец, в члены КПСС
%%%cit_comment
%%%cit_title
\citTitle{Государство, в котором осуществляется внутренний раскол, обречено / Лента соцсетей / Страна}, 
Александр Гончаров, strana.ua, 27.06.2021
%%%endcit


