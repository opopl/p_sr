% vim: keymap=russian-jcukenwin
%%beginhead 
 
%%file 14_12_2021.fb.bilchenko_evgenia.1.idejnost
%%parent 14_12_2021
 
%%url https://www.facebook.com/yevzhik/posts/4539252476109827
 
%%author_id bilchenko_evgenia
%%date 
 
%%tags bilchenko_evgenia,kiev,rossia,st_peterburg,ukraina
%%title БЖ. Очень высокой идейности пост
 
%%endhead 
 
\subsection{БЖ. Очень высокой идейности пост}
\label{sec:14_12_2021.fb.bilchenko_evgenia.1.idejnost}
 
\Purl{https://www.facebook.com/yevzhik/posts/4539252476109827}
\ifcmt
 author_begin
   author_id bilchenko_evgenia
 author_end
\fi

БЖ. Очень высокой идейности пост

Какие всё-таки странные звери - люди... Я живу в самом красивом городе мира, в
самом сердце этого города, точнее, в сердце его сердца, в инфаркте миокарда в
стиле \enquote{ампир}, в шкатулке Петрушиной. Рядом со мной живут: Иосиф Бродский,
Борис Рыжий, Анна Ахматова, Владимир Маяковский, Сергей Есенин, Александр
Пушкин и - дальше, - сколько дойдешь в белых ботиночках по сугробам. Рядом со
мной - самый лучший мужчина на свете, потому что я его люблю навсИгда.

\ii{14_12_2021.fb.bilchenko_evgenia.1.idejnost.pic.1}

И, несмотря на это, мысленно, я всё ещё мечтаю уехать в Питер. Вот как так, а?

У меня есть тому объяснение: в Киеве остались мои пацаны. Когда я говорю
"пацаны", я буквально имею в виду тех четверых, что добухивали со мной горькую
до конца. Во всех, блин, ее смыслах. И я по ним очень скучаю, потому что я их,
нехристей таких асоциальных, люблю больше всего вмири.

Примечательно, что я так и не могу читать лекции. И примечательно, что первые
предложения мне сделали люди, которые не видят ничего зазорного в том, чтобы на
гранты США  или Канады писать о том, какая плохая Россия в ее самых крутых
вузах. Вот меня это возмутило, и мне сказали, что я - \enquote{травмат} (я травмат?) -
и вообще. 

В общем, муж очень точно назвал меня \enquote{идейной}. Ужасно тривиально, по старинке
это звучит, но я таки ИДЕЙНАЯ. И я не смогла бы продавать \enquote{украинстvo} (какой
кошмар с термином, да?) и читать лекции в центре града Петра, какой клёвый
мужик был Мазепа. Ничего с собой не могу поделать. Лучше - бомжом/фрилансером. 

Это всё потому, что выдался день повольнее, заказ чернорабочего в литературе -
продлен до декабря конца,  англоязычные воронежские  студенты-антиглобалисты
сегодня меня своими бойкими вопросами депутату ЕС об Ассанже порадовали и
знанием языка (Воронеж, бонус, именно тебе, у тебя либералов в системе высшего
поменьше с их цензурой), и я больше не шастаю, потому что мы шастали пять дней
подряд по всей суровой родне от Бранимира до Бледного, пока я не упалъ с ацкими
болями в животе (скорая? не скорая? аппендицит? СРК?) и не съел мужнины
макароны по-флотски (на мясо тянет, значит, пока терплю: у меня ещё нет крутого
полиса, чтобы тут ещё бесплатно по больничкам шастать, я всё понимаю, я терплю,
болезней у меня не один ипучий СРК).

А ещё меня на некоторым образом жесткий фест пригласили, как стар..., в общем,
как мастера пера, вот, прям, вообще не либеральный и не изящный фест, слишком
русский, к Марксу он не имеет отношения, это просто - кафешка крутая и фотка
тоже,  потому промолчу, а то интеллигентная публика на меня плеваЦа будет
дискурсами своими нев\#\#бенными.

\ii{14_12_2021.fb.bilchenko_evgenia.1.idejnost.cmt}
