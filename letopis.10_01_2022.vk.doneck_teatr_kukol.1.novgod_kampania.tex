% vim: keymap=russian-jcukenwin
%%beginhead 
 
%%file 10_01_2022.vk.doneck_teatr_kukol.1.novgod_kampania
%%parent 10_01_2022
 
%%url https://vk.com/teatrkukoldonetsk?w=wall-108142113_5488
 
%%author_id doneck_teatr_kukol
%%date 
 
%%tags doneck,novyj_god,dnr,teatr,kukla,deti,donbass,kultura
%%title Вот и подошла к концу наша новогодняя кампания 2021-2022
 
%%endhead 
\subsection{Вот и подошла к концу наша новогодняя кампания 2021-2022}
\label{sec:10_01_2022.vk.doneck_teatr_kukol.1.novgod_kampania}

\Purl{https://vk.com/teatrkukoldonetsk?w=wall-108142113_5488}
\ifcmt
 author_begin
   author_id doneck_teatr_kukol
 author_end
\fi

Вот и подошла к концу наша новогодняя кампания 2021-2022, которая проходила с
24 декабря по 9 января  @igg{fbicon.christmas.tree} 

От этого Нового года переполняют эмоции не только у зрителей, но и у нас @igg{fbicon.exclamation.mark}
Почему спросите вы?

\ii{10_01_2022.vk.doneck_teatr_kukol.1.novgod_kampania.pic.1}

Потому что:

 @igg{fbicon.round.pushpin} нам подарили ёлку, которая красуется на площади возле театра и радует не
только наших зрителей, а и всех жителей и гостей Донецка;

 @igg{fbicon.round.pushpin} у нас теперь два зрительных зала - большой и малый;

 @igg{fbicon.round.pushpin} фасад театра украшен настолько ярко, что поговаривают, вечерами его видно
даже из космоса  @igg{fbicon.face.grinning.squinting} ;

 @igg{fbicon.round.pushpin} в фойе театра всех встречает звездное небо;

 @igg{fbicon.round.pushpin} оформлены 5 разных, не похожих друг на друга тематических фотозон;

 @igg{fbicon.round.pushpin} уникальность этого года - не одно новогоднее представление, а целых два:
Новогоднее кукольное шоу «Двенадцать добрых дел» (4+) и Новогодний утренник для
малышей «Бука Новый год» (0+).

\ii{10_01_2022.vk.doneck_teatr_kukol.1.novgod_kampania.pic.2}

А теперь немного статистики:

 @igg{fbicon.chart.increasing}  40 показов Новогоднего кукольного шоу «Двенадцать добрых дел»;

 @igg{fbicon.chart.increasing}  24 показа Новогоднего утренника \enquote{Бука Новый год};

 @igg{fbicon.chart.increasing}  27 выездных показов новогоднего представления «Леди Баг и Супер-кот – самых
храбрый Новый год» по Донецку и городам Донецкой Народной Республики;

 @igg{fbicon.chart.increasing}  более 10000 счастливых зрителей;

 @igg{fbicon.chart.increasing}  около 1000 надутых шариков для маленьких зрителей \enquote{Бука Новый год};

 @igg{fbicon.chart.increasing}  120 сказок показал Кот Баюн;

 @igg{fbicon.chart.increasing}  24 раза Бука начинал жить по-новому;

 @igg{fbicon.chart.increasing}  около 1 т сладостей съели Бош и Крош в сладком королевстве;

\ii{10_01_2022.vk.doneck_teatr_kukol.1.novgod_kampania.pic.3}

 @igg{fbicon.chart.increasing}  более 500 желаний, стихотворений и песен были выслушаны Дедушкой Морозом;

 @igg{fbicon.chart.increasing}  5067 желаний были собраны в сачки для желаний из волшебных соцветий,
собранных на полях Полудницы;

 @igg{fbicon.chart.increasing}  24 раза внучка отправлялась в кругосветное путешествие в поисках весны;

 @igg{fbicon.chart.increasing}  40 раз была укрыта снежным одеялом поляна ярких впечатлений;

 @igg{fbicon.chart.increasing}  самое популярное новое имя для Буки - Константин, Коля и Сашка;

 @igg{fbicon.chart.increasing}  40 раз Хрустальда считала сделанные добрые дела Боша и Кроша;

 @igg{fbicon.chart.increasing}  24 раза звучала фраза \enquote{раз, два, три - Ёлочка, гори!} - и ровно столько же
она загоралась;

 @igg{fbicon.chart.increasing}  1348 выпущенных дымовых колечек из увеселительной пушки Каменными
стражниками, чтобы разогнать дурные мысли;

 @igg{fbicon.chart.increasing}  около 10000 снежков были брошены во время новогоднего утренника, из них 1852
раза попали в Буку, а не в корзину;

 @igg{fbicon.chart.increasing}  100 сотрудников, которые трудились для вас @igg{fbicon.heart.red}

\ii{10_01_2022.vk.doneck_teatr_kukol.1.novgod_kampania.pic.4}

Подробнее  @igg{fbicon.index.pointing.right}  
\href{https://teatrkukoldnr.ru/novosti/217-zakritie-ngkampanii-9-yanvarya}{%
В Донецком республиканском академическом театре кукол завершилась новогодняя кампания, %
teatrkukoldnr.ru, 10.01.2022%
}

С 24 декабря 2021 года по 9 января 2022 года в Государственном бюджетном
учреждении «Донецкий республиканский академический театр кукол» проходила
новогодняя кампания.

\ii{10_01_2022.vk.doneck_teatr_kukol.1.novgod_kampania.pic.5}

«В эту новогоднюю кампанию у нас в театре все по-новому и подготовка
кардинально отличается от предыдущих лет! Это и украшение фасада здания, и
внутреннее оформление театра, и ёлка на площади, и не одно новогоднее
представление на стационаре, а целых два: Новогоднее кукольное шоу «Двенадцать
добрых дел» и Новогодний утренник для малышей «Бука Новый год», оба
представления – интерактивные и главными помощниками являются сами зрители», –
прокомментировала директор Донецкого театра кукол Ольга Арутинова.

\ii{10_01_2022.vk.doneck_teatr_kukol.1.novgod_kampania.pic.6}

Донецкому театру кукол была презентована ёлка от жителей Ханты-Мансийского
автономного округа, которые приняли участие в добровольческой акции АНО
«Гуманитарный Добровольческий Корпус» по сбору средств «Подари чудо детям –
2021», благодаря им на площади возле театра стоит новогодняя ёлка, которая
дарит праздничную атмосферу всем сотрудникам и зрителям театра кукол.

\ii{10_01_2022.vk.doneck_teatr_kukol.1.novgod_kampania.pic.7}

«Главная идея нашего шоу заключается в том, что имеет значение не количество
добрых дел – семь или двенадцать – важно совершать их от чистого сердца, не
жалея сил и времени, ведь даже одним хорошим делом можно изменить чью-то
жизнь», – прокомментировали режиссеры новогоднего кукольного шоу Виталий
Маринков и Алевтина Шиманова.

\ii{10_01_2022.vk.doneck_teatr_kukol.1.novgod_kampania.pic.8}

Новогоднее кукольное шоу «Двенадцать добрых дел» – это интерактивное
представление, которое состоит из двух актов, в котором главными помощниками
главных героев являются зрители. Шикарные декорации сцены и эпизодов, необычные
экспериментальные куклы, захватывающих и непредсказуемый сюжет – всё это с
первых минут представления приковывает взгляд как маленьких зрителей, так и
взрослых.

\ii{10_01_2022.vk.doneck_teatr_kukol.1.novgod_kampania.pic.9}

«Спасибо большое всему коллективу театра за роскошную атмосферу праздника и на
улице и внутри помещения. У вас так красиво и чудесно, даже взрослые ощущают
волшебство, мысленно становясь детьми. Спасибо, что так стараетесь для своих
зрителей, вкладываете столько любви и тепла в работу театра. Приезжали впервые,
очарованы и собираемся приезжать ещё», – прокомментировал зритель новогоднего
кукольного шоу Светлана Якимова.

\ii{10_01_2022.vk.doneck_teatr_kukol.1.novgod_kampania.pic.10}

Внутреннее украшение дарило атмосферу волшебства и праздника по всему театру:
новогодняя ёлка, звездное небо, оформлены оба зрительных зала, а также 5
разные, не похожие друг на друга тематические фотозоны.

«Весь утренник наполнен волшебством и приключениями, в которое не только дети,
а и даже взрослые начинали верить и принимать активное участие. Но самое
главное новогоднее чудо – каждый желающий мог пообщаться с Дедушкой Морозом», –
прокомментировала режиссер новогоднего утренника для малышей София
Романенко-Куренная.

Новогодний утренник для малышей «Бука Новый год» – это интерактивное
представление для самой маленькой возрастной категории, в котором сюжетная
линия неразрывно связана с игровыми элементами, постоянно вовлекая зрителя в
происходящее на сцене.

«Мне и моему 3-хлетнему сыну очень понравилось. Я как будто снова стала
маленькой, и наивно ждала того самого чуда – когда появится Дед Мороз и
загорится огнями ёлка. Само представление очень интересное – интерактив с залом
на высшем уровне. Спасибо всей вашей команде за профессиональный подход к
своему делу!», – прокомментировала Анна Малаева.

В рамках новогодней кампании было проведено показ 64 представлений на
стационаре, из них: 40 показов «Двенадцать добрых дел» из которых 8
благотворительных, которые посетили около 1500 детей всех льготных категорий,
24 показа «Бука Новый год», а также 27 выездных показов новогоднего
представления «Леди Баг и Супер-кот – самых храбрый Новый год» по Донецку и
городам Донецкой Народной Республики, таким как: Ясиноватая, Докучаевск,
Горловка, Шахтерск, Торез, Иловайск, Зугрес, Харцызск, Старобешевский район.
Общее количество зрителей новогодней кампании составило более 10000 человек.
