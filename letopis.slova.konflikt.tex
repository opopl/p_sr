% vim: keymap=russian-jcukenwin
%%beginhead 
 
%%file slova.konflikt
%%parent slova
 
%%url 
 
%%author 
%%author_id 
%%author_url 
 
%%tags 
%%title 
 
%%endhead 
\chapter{Конфликт}
\label{sec:slova.konflikt}

%%%cit
%%%cit_head
%%%cit_pic
%%%cit_text
Будущее украинского и российского народов, как и любых европейских народов, ‒
во взаимовыгодном сотрудничестве и уважении. Да, сегодня между Россией и
Украиной есть \emph{конфликт}. Но наша задача, как двух зрелых европейских стран,
решить этот \emph{конфликт} с обоюдной выгодой. И моя принципиальная и
последовательная позиция, которой я никогда не изменял, лежит именно в
плоскости переговоров и нахождения взаимных интересов. Если украинцы построят
мост между Европой и Россией, то они будут нужны и Европе, и России. Если они
будут строить стену, то эту стену обойдут и оставят нас наедине с ненужной
стеной. Время на построение независимого государства не может быть бесконечным.
Мы либо построим что-то стоящее, либо нет. Все зависит от того, что и с кем мы
хотим строить
%%%cit_comment
%%%cit_title
\citTitle{О будущем украинского и русского народов}, 
Виктор Медведчук, strana.ua, 15.07.2021
%%%endcit

%%%cit
%%%cit_head
%%%cit_pic
%%%cit_text
Иными словами, одной из важнейших функций ЦПД является банальное стукачество,
надо полагать — за неправильное словоупотребление. А по сути — за
мыслепреступление, как у упомянутого Оруэлла. Например, заикнется кто-нибудь о
том, что на Донбассе имеет место самый банальный гражданский социальный
\emph{конфликт}, имеющий социальные причины, ставший результатом грабежа страны и
люмпенизации народонаселения, а Путин просто умело использовал этот \emph{конфликт},
вмешавшись в него, и все — можно доносить на заикнувшегося. Интересно,
грантоедскую премию за такое выписывают?..
%%%cit_comment
%%%cit_title
\citTitle{Краткий словарь грантоедов под редакцией СНБО / Лента соцсетей / Страна}, 
Александр Карпец, strana.news, 26.10.2021
%%%endcit

%%%cit
%%%cit_head
%%%cit_pic
%%%cit_text
Командир первой роты полка киевского "Беркута" Михаил Добровольский рассказал
"Стране", что заступил на службу 5 декабря 2013 года. По его мнению, милицию
провоцировали намеренно.  "Лично я как военный это понимал. Нас специально
поливали грязью, дергали, чтобы мы ответили. А дальше подготовленные люди с
камерами ждали "картинки", которую бы потом показали по телевидению", - говорит
экс-милиционер.  По его словам, "беркутовцы" не воспринимали участников Майдана
как врагов, но были агрессивные люди, которые, как считает Добровольский, были
"заряжены" на насилие. А мирных протестующих, по его мнению, просто
использовали.  Впрочем, то же самое он говорит о своих бывших коллегах.
"Милицию и протестующих сделали сторонами \emph{конфликта}. Остальные сидели по
ресторанам и смеялись", - дает Добровольский свое видение той ситуации
%%%cit_comment
%%%cit_title
\citTitle{Как убивали на Майдане и что он дал стране. Воспоминания участников событий 8 лет спустя}, 
Юлия Колтак, strana.news, 21.11.2021
%%%endcit
