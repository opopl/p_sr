% vim: keymap=russian-jcukenwin
%%beginhead 
 
%%file slova.kolonia
%%parent slova
 
%%url 
 
%%author 
%%author_id 
%%author_url 
 
%%tags 
%%title 
 
%%endhead 
\chapter{Колония}
\label{sec:slova.kolonia}

%%%cit
%%%cit_pic
%%%cit_text
Ну, вот так и получается, что в метрополию из \emph{колонии} сбегал весь цвет
украинства, включая и миллионы и миллионы трудолюбивых переселенцев на Кубань,
в Сибирь, Дальний Восток. И поэтому именно такая \emph{колониальная} политика России и
привела к тому, что Украина все годы своей независимости с 1991 года находится
не только в экономической и технологической, но и культурной нищете.
Интересно, а после 1991 года из независимой Украины, взявшей курс на
национальное процветание, разве не уехали миллионы украинских умов и дарований
в другие страны? Например, в Польшу, Германию, а то и вовсе в Америку? Я не
упоминаю про Россию, куда все эти дарования тоже сбегают из Украины, как и в
былые времена. Я акцентирую внимание на «дружественные» Украине страны Европы и
Америки (про Канаду не забываем, где огромная украинская диаспора, просто
переполненная всякими «талантами»). И если идти по логике современных
украинских пропагандистов, то Польша с Германией и Америкой проводят точно
такую же \emph{колониальную политику} высасывания мозгов из Украины, какую все время
проводила и Россия. Вы можете назвать хоть одного современного украинского
изобретателя или писателя, который сегодня работал бы на благо Украины в самой
Украине? Нет, все они сейчас находятся за пределами своей родины, и трудятся
исключительно на благо Европы и Америки
%%%cit_title
\citTitle{Украина никак не может выбраться из колониальной зависимости – то от России, то от Запада}, 
Исторический Понедельник, zen.yandex.ru, 01.06.2021
%%%endcit

%%%cit
%%%cit_pic
%%%cit_text
Но польско-немецкие националисты не хотят называть свою страну великим
историческим названием, чтобы исключить всякую ее ассоциацию с русским миром,
который они так не любят, хотя, по сути, имеют русские генетические корни, а не
польские или немецкие. И поэтому им слово Русь в названии – как серпом по
пальцам, а так как придумать ничего нового они не в состоянии, то пришлось
вытащить из загашника польское слово Украина, которое, по сути, обозначает
«зависимые территории», или точнее – \emph{«колония»}.  Ну вот, во что Украина
сегодня превратилась – мы прекрасно видим. Только к \emph{колониям} («окраинам»), как
известно, у метрополий отношение совсем другое, совсем не родственное. По мне,
так уж лучше называться все же Малороссией, чем \emph{Колонией}
%%%cit_title
\citTitle{Почему современную Украину назвали «окраиной», а не более престижно - Киевской Русью?}, 
Исторический Понедельник, zen.yandex.ru, 22.02.2021 
%%%endcit

%%%cit
%%%cit_head
%%%cit_pic
%%%cit_text
Наслідки \emph{колонізаційної} політики ми відчуваємо досі, адже купа людей, які
народилися й виросли в україномовному середовищі, з часом (найчастіше – після
переїзду) переходять на російську мову. Я думаю, ні в кого немає сумнівів, що
українська мова милозвучніша за російську, але це не зупиняє людей від
самовільної (а не з примусу!) русифікації. На щастя, з року в рік ми
спостерігаємо все менше таких добровільних переходів на російську, навпаки –
після Революції Гідності все більше людей обирають українську, хоча найбільші
міста країни все ще російськомовні
%%%cit_comment
%%%cit_title
\citTitle{Справа не в милозвучності}, Андрій Любка, day.kyiv.ua, 25.06.2021
%%%endcit

%%%cit
%%%cit_head
%%%cit_pic
%%%cit_text
Первый тянет нас назад, второй толкает вперед.  Украина разделена на два типа
мышления - \emph{колониальный} и цивилизационный.  \emph{Колониальный} трактует историю
Украины, как историю \emph{колонии}, которая переходила от одной империи к другой.
Цивилизационный видит Украину полноправной частью сменяющих друг друга империй.
\emph{Колониальный} смотрит в историю, цивилизационный - в будущее.  \emph{Колониальный} - то
журба та нескінченний плач. \emph{Колониальный} приемлет коррупцию и воровство, как
наследие проклятых империй. \emph{Колониальный} готов ждать десятилетиями перемен, и
"может наши дети увидят" светлое будущее. \emph{Колониальное мышление} - идеальное
підгрунтя для существование воровского правящего класса, для воровского
госаппарата. Ведь мы не заслужили ещё достойной жизни, для достойной жизни
нужно долго страдать и жертвовать. Да и народ не готов. Вот когда народ
перестанет быть патерналистским, тогда и госаппарат станет честным
%%%cit_comment
%%%cit_title
\citTitle{Украина разделена на два типа мышления - колониальный и цивилизационный}, 
Павел Себастьянович, strana.ua, 14.07.2021
%%%endcit

%%%cit
%%%cit_head
%%%cit_pic
%%%cit_text
Выход из ситуации очень простой – создавать современное правовое
мультикультурное государство, а не пытаться \emph{колонизировать} свои же территории.
Но очередная украинская власть не в состоянии это делать. А тогда неизбежна
борьба с \emph{колонизаторами} за национальное достоинство и угроза территориальной
целостности страны, чего нельзя допустить.  Если бы РФ строила свою
государственность по лекалам современной Украины, то она быстро бы лишилась
огромного количества территорий и населения. Я считаю, что украинцы и русские –
разные народы, но народы братские, славянские, православные, с единой
ментальностью, историей, святыми и традициями. Народы, которые веками
добивались величайших побед, когда шли рука об руку.
Мы должны поставить во главу угла построения нового европейского государства
Украина не национальное достоинство, когда одни нации объявляются ценными, а
другие подлежат ассимиляции, а права и свободы граждан. Нам нужна свободная и
справедливая держава, а не национальная резервация
%%%cit_comment
%%%cit_title
\citTitle{О будущем украинского и русского народов}, 
Виктор Медведчук, strana.ua, 15.07.2021
%%%endcit

