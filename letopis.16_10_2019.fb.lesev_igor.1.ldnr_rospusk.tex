% vim: keymap=russian-jcukenwin
%%beginhead 
 
%%file 16_10_2019.fb.lesev_igor.1.ldnr_rospusk
%%parent 16_10_2019
 
%%url https://www.facebook.com/permalink.php?story_fbid=2729445787086507&id=100000633379839
 
%%author_id lesev_igor
%%date 
 
%%tags dnr,donbass,lnr,minsk_dogovor,politika,ukraina,vojna
%%title О роспуске ЛДНР
 
%%endhead 
 
\subsection{О роспуске ЛДНР}
\label{sec:16_10_2019.fb.lesev_igor.1.ldnr_rospusk}
 
\Purl{https://www.facebook.com/permalink.php?story_fbid=2729445787086507&id=100000633379839}
\ifcmt
 author_begin
   author_id lesev_igor
 author_end
\fi

О роспуске ЛДНР

Украинская политика – это даже не инфантильно, непоследовательно или нелепо.
Это всегда и в первую очередь стыдно. Идти на выборы с тезисом о мире.
Мурыжиться на переговорах в Минске. Затрахать всех с давно согласованной на
международном уровне формулой Штанмайера. А затем заявиться на собственную
свадьбу и заявить, что невеста должна похудеть, иначе ничего не будет. Вот за
это и стыдно.

\ifcmt
  ig https://scontent-frx5-1.xx.fbcdn.net/v/t1.6435-9/72680778_2729445543753198_2875742713641172992_n.jpg?_nc_cat=100&ccb=1-5&_nc_sid=730e14&_nc_ohc=mQKGixOuyJYAX97EqdT&_nc_oc=AQm1Lh1lCAkABSqZg4AQpiMTwdmbprxNz59CdvqxtE_ypCPrs90CrRZU3kO-cVIUSFk&_nc_ht=scontent-frx5-1.xx&oh=119828ea888a71c2facd749ac715da72&oe=61B8B3D9
  @width 0.4
  %@wrap \parpic[r]
  @wrap \InsertBoxR{0}
\fi

Еще раз. Минск – он не безальтернативен. Также, как когда-то был не
безальтернативен Нантский эдикт или Сен-Жерменский мирный договор. Наверняка,
можно было чуть по-другому, а можно было даже и вообще никак.

Но в договорных отношениях есть одно непременное условие – стороны их
выполняют. Если они неприемлемы – стороны их не подписывают.

Да, бывают варианты, когда обстоятельства стоят выше собственных хотелок. Ты,
скажем, хочешь начать профессиональную карьеру футболиста, а тебе говорят, что
в 39 лет не подпускают подавать даже мячики.

Но Украина все же не проигравшая сторона на Донбассе. Даже с учетом того, что
обе военные кампании 14-го и 15-го были с треском провалены. Украина не
добилась генеральной хотелки, но не потеряла ничего принципиально нового, а с
учетом сепаратисткой волны, охватившей практически весь Донбасс, сумела
самопровозглашенные республики значительно сократить территориально.

Это именно стакан, который наполовину полон или пуст. Плюс – Мариуполь,
Славянск, Краматорск, Лисичанск и Северодонецк. Минус – памятник Ленину на
центральной площади Донецка и такой ненавистный русский язык в оставшихся
неподконтрольными городах Донбасса.

А теперь пришло время определяться. Для начала сказать людям – нам, простым
обывателям – какую модель Украины вы продолжаете лепить? Этническую? Украину
для украиноязычных украинцев? Если так, то, естественно, ни о каком Минске речи
быть не может. Но вы просто должны об этом сказать вслух. И здесь, и за
поребриком. Нельзя быть верным мужем и только по средам трахаться с
проститутками.

Если вы строите мультикультурную Украину, с признанием права за своими
согражданами в языковом разнообразии и культурных особенностях регионов, тогда
непонятно, что вас пугает в Минске. Просто озвучьте ваши страхи. Ваши «красные
линии», о которых вы чешете, но которые вы не конкретизируете.

Мне вот показательна в этом плане Испания. Ребята пережили страшную гражданскую
и получили по итогу очень действенную прививку от кровопролития. И научились
цивилизованно жить и также решать вопросы друг с другом, не испытывая при этом
какой-то взаимной любви и уважения.

Страна формально унитарная, а по факту состоит из автономий с разным уровнем
внутреннего самоуправления. Периодически там начинается движуха за отделение,
как в Каталонии образца 17-го. Потом приезжают «полицаи» из Мадрида и разгоняют
организаторов. Вот на днях девять фигурантов, организовавших референдум о
независимости, получили реальные крупные сроки.

Но ощутите разницу с Донбассом. Я даже не об АТО и обстрелах. В Каталонии, как
и на Донбассе, миллионы желающих свалить нафиг от центра. Но в Мадриде это
просто признают, как неприятный факт, и последовательно говорят «нет», не делая
при этом попыток сделать из каталонцев, басков, галисийцев или андалусийцев
испанцев кастильского типа. Где одна нация и один язык.

И реакция Киева на идентичные сепаратистские движения. Сотни тысяч своих же
сограждан называются сепарами и коллаборантами. А отстаивание языковых прав
провозглашается атакой на суверенитет страны. Так было при Порошенко, и так
продолжается при Зеленском.

Позиция абсолютно тупиковая, потому что она не о мире, не о примирении, и
вообще не о попытках интеграции. Отказ от Минска – это отказ от
мультикультурной модели построения государства. А дальше только армовир. Ну,
потому что идеологический вакуум не может быть пустым. Он обязательно чем-то
будет наполнен. И, похоже, будет снова наполнен вятровичами.

А от неподконтрольной части Донбасса Украина таки отказалась. Невеста им,
понимаешь, не такая. Невеста уже давно сношается с Ростовской областью,
де-факто ставшей ее продолжением. А женишок наш все еще думает, что он молод и
всем интересен.

\ii{16_10_2019.fb.lesev_igor.1.ldnr_rospusk.cmt}
