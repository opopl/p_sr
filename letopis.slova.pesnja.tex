% vim: keymap=russian-jcukenwin
%%beginhead 
 
%%file slova.pesnja
%%parent slova
 
%%url 
 
%%author 
%%author_id 
%%author_url 
 
%%tags 
%%title 
 
%%endhead 
\chapter{Песня}

Песня, петь, поющий

%%%cit
%%%cit_pic
\ifcmt
  pic https://avatars.mds.yandex.net/get-zen_doc/1706869/pub_60bcf367e8b8730bb2371439_60bcf4e93adb654fa55128f5/scale_1200
	caption Невероятная Людмила Сенчина очень органична и в русском, и в украинском костюме, и с русской, и украинской песнями. Фото Яндекс.Картинки.
\fi
%%%cit_text
Я выросла в \emph{поющей} семье. Где на всех праздниках за столом лились разные напевы
под баян и акапелла. Репертуар тоже был разный. \emph{Пели} народные \emph{песни}, песни
молодости родителей, песни из кинофильмов. И неважно русские ли это были песни
или украинские, якутские или \emph{песни} итальянской молодежи. \emph{Пели} очень красиво, с
многоголосьем и очень душевно. И все знали слова \emph{песен} наизусть, так как
оттачивали свое мастерство на праздниках друг у друга.  Многое стерлось из
детства, но слова этих \emph{песен} я помню до сих пор. Такое не забывается.  Хочу и
сегодня поделиться с вами самыми популярными в нашей семье и невероятно
лиричными \emph{народными украинскими песнями}, которые пелись наряду с русскими тогда
и поются сейчас в нашей семье, несмотря ни на что
%%%cit_title
\citTitle{5 задушевных украинских песен, которые пели наши родители, а теперь поем мы}, 
Кино Вояж И Не Только, zen.yandex.ru, 07.06.2021
%%%endcit

%%%cit
%%%cit_pic
%%%cit_text
Третье место - «Ридна мати моя». Исполнение этой \emph{песни} никого не
оставляет равнодушным. И хотя многие считают ее тоже народной, у этой
\emph{песни} есть авторы. Стихи написал А. Малышко. Они сами по себе очень
мелодичны. Но им подстать и написанная композитором П. Майборода музыка.
Впервые прозвучала \emph{песня}, официальное название которой \emph{«Песня про
рушник»}, в фильме «Годы молодые». Очень проникновенно в нем исполнил эту песню
А. Таранец. Кто только не исполнял в свое время эту песню: Н. Матвиенко, Д.
Гнатюк, И. Кобзон, Я. Евдокимов, Р. Бейбутов, Т. Повалий, и даже А. Пугачева.
Мне понравилось исполнение молодой исполнительницы, участницы шоу «Ты супер» А.
Панкратовой. Девочка лишена была материнской заботы. Может поэтому ее
исполнение трогает своей правдивостью, послушайте\footnote{Ты супер! — «Рiдна
мати моя» — Александра Панкратова, 17 лет, Краснодарский край, youtube,
19.05.2019},
%%%cit_title
\citTitle{5 задушевных украинских песен, которые пели наши родители, а теперь поем мы}, 
Кино Вояж И Не Только, zen.yandex.ru, 07.06.2021
%%%endcit


%%%cit
%%%cit_pic
%%%cit_text
Первое место - «Ой, у вишневом у саду».  Об этой украинской народной \emph{песне} могу
говорить долго, так как \emph{это моя любимая песня}. Мы ее с вами можем спеть вместе.
Вот ее слова:

Ой, у вишневому саду, там соловейко щебетав.
До дому я просилася, а вiн мене все не пускав.
До дому я просилася, а вiн мене все не пускав.

Ти, милий мій, а я твоя, пусти мене, зійшла зоря.
Проснеться матінка моя, буде питать де була я.
Проснеться матінка моя, буде питать де була я.

...

Ой, у вишневому саду, там соловейко щебетав.
До дому я пiшла одна, а вiн другую провожав....
До дому я пiшла одна, а вiн другую провожав...

Эта \emph{песня}, кто бы ее ни \emph{пел}, вызывает у меня всегда сначала мурашки, а
потом слезы: ее очень любила моя мама. Мама перед самым уходом лежала в
больнице и почти не ела. Нас спасала именно эта \emph{песня}. Я включала ее на
телефоне, клала на подушку. И у мамы появлялся какой-никакой аппетит и мне
удавалось ее покормить.  А еще она так улыбалась этой трогательной мелодии, что
даже пыталась \emph{подпеть}. Эти воспоминания всегда со мной, как и
\emph{песня}.  Я думаю, что у каждого свои воспоминания. Но точно знаю, что эта
\emph{песня} никого не оставила равнодушным. Я сегодня предложу вам послушать
три варианта этой \emph{песни}. А вы сами выберете, какой из трех вам по душе.
Первый вариант в исполнении талантливой молодой \emph{певицы} Алисы Игнатьевой
участницы шоу \enquote{Голос}. Завораживающее пение:
%%%cit_title
\citTitle{5 задушевных украинских песен, которые пели наши родители, а теперь поем мы}, 
Кино Вояж И Не Только, zen.yandex.ru, 07.06.2021
%%%endcit

%%%cit
%%%cit_pic
%%%cit_text
Мы выбрали только пять \emph{песен}, но их таких лиричных, добрых, поучительных,
лечащих души и сердца и создающих настроение, великое множество. И каждый свои
\emph{песни} определяет для себя сам. Но пока мы помним их и \emph{поем}, а с нами \emph{поют} наши
дети и внуки, значит у этих \emph{песен}, а значит и у нас, есть будущее. А вы, что
об этом думаете? Делитесь своими любимыми \emph{песнями}
%%%cit_title
\citTitle{5 задушевных украинских песен, которые пели наши родители, а теперь поем мы}, 
Кино Вояж И Не Только, zen.yandex.ru, 07.06.2021
%%%endcit

%%%cit
%%%cit_pic
%%%cit_text
Х...ох..лятские \emph{песни} просто не реальные, красота, мелодика, душевность.
Обалдеть можно от них. Обожаю!!
%%%cit_comment
Лариса
%%%cit_title
\citTitle{5 задушевных украинских песен, которые пели наши родители, а теперь поем мы}, 
Кино Вояж И Не Только, zen.yandex.ru, 07.06.2021
%%%endcit

%%%cit
%%%cit_pic
%%%cit_text
А я бы ещё в этот список добавил \emph{песню} Ехал казак за Дунай в исполнении
Московского казачьего хора и А. Панкратовой из финала второго сезона Ты Супер!
%%%cit_comment
Алексей Выгодин
%%%cit_title
\citTitle{5 задушевных украинских песен, которые пели наши родители, а теперь поем мы}, 
Кино Вояж И Не Только, zen.yandex.ru, 07.06.2021
%%%endcit

%%%cit
%%%cit_pic
%%%cit_text
РОдилась и выросла среди поющих родственников. Папа очень любил петь украинские
\emph{песни} хоть сам был русский. Его любимые песни \enquote{Гуцулка Ксеня}, и \enquote{Дывлюсь я на
небо}. Сама тоже очень люблю петь украинские \emph{песни}, а слушать еще больше
%%%cit_comment
Antonina
%%%cit_title
\citTitle{5 задушевных украинских песен, которые пели наши родители, а теперь поем мы}, 
Кино Вояж И Не Только, zen.yandex.ru, 07.06.2021
%%%endcit

%%%cit
%%%cit_pic
%%%cit_text
Две \emph{песенные премьеры} состоялись в сентябре 1970 года, в передаче Черновицкой
студии телевидения \enquote{Камертон хорошего настроения}. С Театральной площади города
Черновцы в прямом эфире они прозвучали на всю Украину.  При подготовке к записи
возникла проблема с исполнителями - все известные\emph{ певцы} и
вокально-инструментальные ансамбли оказались в то время на гастролях. Поэтому
Владимир Ивасюк исполнил \emph{песни} в дуэте с преподавательницей черновицкой
музыкальной школы №1 Еленой Кузнецовой
%%%cit_comment
%%%cit_title
\citTitle{Какая легенда вдохновила Владимира Ивасюка написать \enquote{Червону руту} 
(украинскому шлягеру - 50)}, 
Татьяна Кроп, zen.yandex.ru, 07.12.2020
%%%endcit

%%%cit
%%%cit_pic
%%%cit_text
\emph{Песни} прозвучали красиво и профессионально. И премьеры прошли успешно. Особенно
удачно сложилась судьба \emph{песни} \enquote{Червона рута}. После первого, сентябрьского
исполнения, она прозвучала в октябре 1970 года. На творческом вечере
Черновицкой областной организации Союза композиторов Украины ее \emph{спел} ВИА
\enquote{Смеричка} из Вижницы под руководством Левка Дутковского (в ансамбле в то время
работали Назарий Яремчук и Василий Зинкевич, будущие звезды украинской
эстрады). И вновь - успех
%%%cit_comment
%%%cit_title
\citTitle{Какая легенда вдохновила Владимира Ивасюка написать \enquote{Червону руту} 
(украинскому шлягеру - 50)}, 
Татьяна Кроп, zen.yandex.ru, 07.12.2020
%%%endcit

%%%cit
%%%cit_pic
%%%cit_text
Не один раз, вспоминая эту историю, однажды я принял для себя решение
переложить ряд широко известных в России народных и авторских украинских \emph{песен}
на русский язык. Решил и сделал.  Эти \emph{песни}, переложенные на русский язык,
предназначены, прежде всего, для жителей южных регионов России, граничащих с
Украиной. Притом, что в русских текстах \emph{песен} я постарался сохранить их
украинский колорит, переводы легко ложатся на существующие мелодии, так же
легко поются, доставляя радость исполнителям и слушателям.  Также \emph{песни} на
русском языке придутся к месту в миллионах смешанных русско-украинских семьях,
в которых, как правило, эти \emph{песни} пелись всегда
%%%cit_comment
%%%cit_title
\citTitle{Украинские Песни Русскими Словами}, 
БРАТИНА, zen.yandex.ru, 15.12.2020
%%%endcit

