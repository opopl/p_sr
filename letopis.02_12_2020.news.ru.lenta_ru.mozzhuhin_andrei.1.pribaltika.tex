% vim: keymap=russian-jcukenwin
%%beginhead 
 
%%file 02_12_2020.news.ru.lenta_ru.mozzhuhin_andrei.1.pribaltika
%%parent 02_12_2020
 
%%url https://lenta.ru/articles/2020/12/02/pribaltika/
 
%%author Мозжухин, Андрей
%%author_id mozzhuhin_andrei
%%author_url 
 
%%tags 
%%title «Трупы убитых выставляли на всеобщее обозрение»
 
%%endhead 
 
\subsection{«Трупы убитых выставляли на всеобщее обозрение»}
\label{sec:02_12_2020.news.ru.lenta_ru.mozzhuhin_andrei.1.pribaltika}
\Purl{https://lenta.ru/articles/2020/12/02/pribaltika/}
\ifcmt
	author_begin
   author_id mozzhuhin_andrei
	author_end
\fi
\index[rus]{Прибалтика!История}

\begin{center}
	\em\bfseries\color{red}\huge 
	Зачем Сталин завладел Прибалтикой\par и что с ней хотел сделать Гитлер
\end{center}

\ifcmt
pic https://icdn.lenta.ru/images/2020/10/23/11/20201023112738140/detail_e260fe16c4d6cf1f5b82485b01fd4c55.jpg
caption Немецкие солдаты смотрят на человеческий череп на броне огнеметного танка ОТ-34. Латвия, 1944 г. Фото: waralbum.ru
\fi

\def\lenta#1{\textbf{\em\color{blue} #1}}

\begin{leftbar}
	\bfseries
В издательстве РОССПЭН вышла монография «Прибалтика. 1939-1945 гг. Война и
память», посвященная одной из самых сложных и противоречивых тем европейской
истории XX века: включению Прибалтики в состав СССР после пакта Молотова —
Риббентропа, ее жизни под немецкой оккупацией в годы Великой Отечественной
войны и после освобождения Красной армией. Имели ли шанс прибалтийские страны
сохранить свою независимость во время Второй мировой войны? Почему к июню 1941
года их население возненавидело советскую власть? Какую участь Прибалтике
готовил Гитлер? Кто и почему во время нацистской оккупации уничтожил почти всех
евреев в Литве? Как СССР усмирял Прибалтику после войны и как теперь с ней быть
России? Обо всем этом «Ленте.ру» рассказала автор книги, доктор исторических
наук, главный научный сотрудник Санкт-Петербургского института истории РАН,
профессор РГПУ им. А.И. Герцена Юлия Кантор.
\end{leftbar}

\subsubsection{Подарок от Сталина}

\textbf{«Лента.ру»:} В своей книге вы указываете, что существующие в предвоенных
прибалтийских государствах «социально-политические условия в значительной
степени облегчили Советскому Союзу реализацию планов по включению Латвии, Литвы
и Эстонии в свою геополитическую орбиту». Что под этим подразумевается?

\textbf{Юлия Кантор:} 
\ifcmt
pic https://icdn.lenta.ru/images/2019/01/22/12/20190122125338421/preview_a51ce0042251b524973801d8db1f4dc6.jpg
cpx Юлия Кантор:
fig_env wrapfigure
width 0.2
\fi

Что почти во все годы межвоенного существования независимых прибалтийских
государств там функционировали жесткие авторитарные режимы, дрейфовавшие в
сторону диктатуры. После государственных переворотов в Литве в 1926 году, а
также в Латвии и в Эстонии в 1934 году, там фактически не работали парламенты,
активно действовала свирепая политическая полиция, существовала жесткая цензура
и были запрещены коммунистические партии.

Во всех трех странах накануне 1939-1940 годов не существовало условий для
демократического развития или какого-либо серьезного гражданского
самовыражения. Во многом именно это обусловило сравнительную безболезненность
включения Литвы, Латвии и Эстонии в орбиту Советского Союза. Ну и, безусловно,
податливость тамошних лидеров, которые тихо подчинялись нажиму из Москвы,
сдавали собственные позиции и независимость своих государств. И при этом вели
успокоительную риторику и внутри своих стран, и во внешний мир.

\ifcmt
pic https://icdn.lenta.ru/images/2020/10/20/16/20201020162102743/preview_4cf477194cab6288f216380170849d95.jpg
fig_env wrapfigure
width 0.3
\fi

\lenta{Каким термином, на ваш взгляд, было бы корректно обозначать включение Прибалтики в состав СССР в 1940 году? Сейчас в этих странах говорят об «оккупации» или об «аннексии», а нынешний российский президент Путин недавно употребил слово «инкорпорация».}

Я в книге использую термин «советизация». Под ним подразумевается весь комплекс
идеологических, военных, политических и экономических мер, с помощью которых
Советский Союз в 1939-1940 годах подготавливал присоединение Прибалтики,
формально осуществленное законными властями всех трех стран. А как именно все
это происходило, в книге достаточно подробно описано. Податливость
прибалтийских элит сочеталась с недвусмысленными угрозами применения военной
силы со стороны Москвы и одновременно щедрыми посулами от нее.

\lenta{Говоря про обещанные щедрые посулы, вы имеете в виду передачу Литве польского
города Вильно?}

Это самый яркий пример. В современной Литве не любят вспоминать, что ее
нынешняя столица Вильнюс стала для нее сталинским подарком — в обмен на
беспрепятственное размещение советских военных баз на литовской территории. Это
случилось в октябре 1939 года, тогда Литва еще была независимым государством.
Когда в сегодняшней Литве критикуют пакт Молотова – Риббентропа, как-то
«забывают», что обретение столицы литовского государства — его прямое
следствие.

\lenta{В книге вы рассказываете, как в октябре 1939 года после размещения в Прибалтике
советских военных баз, проведенного после заключения договоров с
правительствами независимых балтийских государств, Молотов с трибуны Верховного
Совета СССР официально заверил мировое сообщество об отсутствии планов их
советизации. Примерно то же самое несколькими днями раньше в личной беседе
Сталин заявил и лидеру Коминтерна Георгию Димитрову. Как вы думаете, насколько
кремлевские руководители в тот момент были искренни или вся их риторика
изначально была лживой?}

Мне трудно сказать, о чем на самом деле думал Сталин. Возможно, он просчитывал
варианты дальнейшего развития событий в Европе, внимательно наблюдая за
поведением прибалтийских государств и отслеживая реакцию Англии и Франции.

\ifcmt
tab_begin cols=2
	caption Книга «Прибалтика. 1939-1945 гг. Война и память», РГАСПИ

	pic https://icdn.lenta.ru/images/2020/10/20/15/20201020155708415/pic_963b14c11160b1511dbd0827cc14e62e.jpg
	caption Представители Красной армии и Краснознаменного Балтийского флота приветствуют депутатов Государственной думы Эстонии. Таллин, 22 июля 1940 г.

	pic https://icdn.lenta.ru/images/2020/10/20/15/20201020155711982/pic_d9a8d92843e3697bc579983044cf7390.jpg
	caption Девушка с острова Муху (Моон). Эстония, 1940 г. 
tab_end
\fi

Но западным демократиям, по-прежнему мечтавшим столкнуть лбами СССР и Германию,
тогда было не до Прибалтики, а правительства Латвии, Литвы и Эстонии после
заключения договоров о размещении у них советских войск выступили с примерно
такими же заявлениями, что и Молотов. Наверняка такое их податливое поведение
побудило Сталина к проведению более активной и брутальной политики.

\subsubsection{«Красный» регион}

\lenta{Но чем можно объяснить такую податливость и покорность? Почему политические
элиты Литвы, Латвии и Эстонии в конце 1939 года и в начале 1940 года вели себя
так вяло и пассивно?}

Во-первых, как я уже говорила, во всех трех государствах тогда были
авторитарные режимы. Но любые авторитарные режимы крайне негибки и в
критической ситуации неспособны держать удар.

Во-вторых, прибалтийские элиты после подписания пакта Молотова – Риббентропа и
особенно советско-германского договора о дружбе и границе были растеряны.
Тягостное впечатление произвела на них и Зимняя война СССР с Финляндией — во
всех трех странах решили, что для них подобное сопротивление может лишь
привести к неоправданно огромным жертвам.

\lenta{В те дни в Прибалтике действительно были многочисленные митинги и демонстрации
в поддержку действий СССР или это выдумки советской пропаганды?}

Просоветские митинги и демонстрации в июне 1940 года действительно проходили,
но все они были хорошо организованы, о чем свидетельствуют документы,
хранящиеся сейчас в РГАСПИ. Что показательно, во всех этих шествиях почти все
лозунги уже были на русском языке.

\lenta{Имелась ли в тогдашней Прибалтике социальная база для советизации?}

Имелась, причем очень широкая, о чем сейчас там тоже не любят вспоминать. Еще
со времен революции 1917 года Прибалтика считалась «красным» регионом бывшей
Российской империи. Наибольший процент голосов за большевистскую партию на
выборах во Всероссийское учредительное собрание в ноябре 1917 года дали именно
балтийские окраины. Немало рабочих и даже часть интеллигенции сочувствовали
коммунистическим идеям. Вспомним, например, про латышских стрелков, сыгравших
важную роль в ключевых российских событиях 1918 года.

\lenta{Особенно во время выступления левых эсеров в Москве 6 июля 1918 года, когда
благодаря именно латышам большевики удержали власть.}

Конечно. Более того, после этого подразделения латышских стрелков почти в
полном составе зачислили в ВЧК. Кстати, обо всем этом очень хорошо помнил
Гитлер, относившийся к прибалтийским народам с опасливым презрением. Не забыли
в Берлине и о том, как жители Прибалтики в том же 1918 году весьма активно
сопротивлялись немецкой оккупации после заключения Брестского мира.

\begin{leftbar}
\large
Если вернуться к ситуации 1940 года, то, как я уже говорила, советизация
сначала была мягкой и постепенной
\end{leftbar}

И хотя поспешно организованные выборы прошли с грубейшими нарушениями местных
избирательных законов и с явными фальсификациями, о чем в книге тоже подробно
рассказано, ни в одной из прибалтийских стран это не вызвало сколь-нибудь
серьезных протестов.

\subsubsection{В Стране Советов}

В книге вы указываете, что «жестокая сталинская советизация» настроила жителей
Прибалтики против новой власти.

Я убеждена, что если бы не жестокость тоталитарного сталинского режима, то
советизация Прибалтики в предвоенный период прошла бы гораздо мягче и
спокойнее. И реакция ее жителей на приход нацистов была бы совершенно иной.

\ifcmt
tab_begin cols=2
	caption Книга «Прибалтика. 1939-1945 гг. Война и память», РГАСПИ

	pic https://icdn.lenta.ru/images/2020/10/20/15/20201020155440480/pic_a8083072578dd234fcaddae15485a3f3.jpg
	caption Демонстранты проходят мимо полпредства СССР с приветствием товарищу Сталину и Советскому правительству. Таллин, 5 июля 1940 г. 

	pic https://icdn.lenta.ru/images/2020/10/20/15/20201020155444769/pic_6a293a4c1dd3b31a3e5491fdeafd277b.jpg
	caption Шведская молодежь деревни Свиби. Остров Вормси (Ормсё), Эстония, 1940 г.
tab_end
\fi

\lenta{Но чем были вызваны массовые репрессии против местного населения в
мае-июне 1941 года, которые закончились всего лишь за пять дней до начала
Великой Отечественной войны?}

Мера устрашения. Причем поначалу раздражение людей было вызвано не
политическими или идеологическими причинами, а неуклюжей и абсолютно
безграмотной экономической политикой советской власти в Прибалтике.
Национализация, на которую так надеялся местный пролетариат, прошла безобразно
— рабочие от нее ничего не получили. Предприятия стали работать хуже, так как
их прежние хозяева были в лучшем случае отстранены, в худшем — сразу
репрессированы. После грабительского для местного населения перехода на
советскую валюту зарплаты стали существенно меньше, а цены выше.

Шоком для жителей республик Прибалтики стало и первое столкновение с
ассортиментом и качеством ввозимой из СССР продукции, особенно одеждой и
другими товарами народного потребления. Как говорится в одном документе того
времени, эти изделия «собирали у витрин злорадствующие по адресу советской
власти толпы». После этого Москве пришлось пойти на попятную и «дать указание
союзным наркоматам и главкам о том, чтобы производство изделий потребительского
характера по прибалтийским республикам не подчинять союзной стандартизации».

Одновременно с этим раздражение вызвал ажиотажный спрос на местную продукцию
среди приехавших — для советских людей такое изобилие, продукты и одежда такого
качества были чудом. Тем не менее уровень жизни населения советской Прибалтики
очень быстро и резко упал, что породило среди людей естественное недовольство.
А советская власть сталинского образца реагировала на это единственным
известным ей способом: репрессиями и высылками в Сибирь.

\lenta{Можно ли все эти репрессивные сталинские акции трактовать как геноцид литовцев,
латышей и эстонцев? В современной Прибалтике именно так их и называют.}

Геноцид — это целенаправленное уничтожение какого-либо одного народа. Но
сталинский террор в предвоенной Прибалтике, как и во всем остальном СССР, был
интернационален.

Он затронул представителей самых разных народов, в том числе и местных русских,
и в неменьшей степени, чем представителей титульных народов. Если судить по
документам, то, например, в Латвии в пропорциональном отношении больше всех
пострадали евреи (11 процентов из пяти процентов ее предвоенного населения), а
в Литве — евреи и поляки (соответственно 12,5 процента из 9,6 процента и 11,4
процента из 10,3 процента предвоенного населения).

\lenta{Как вы считаете, помогла ли советизация Прибалтики в 1939-1940 годах улучшить
военно-стратегическое положение СССР и подготовиться к будущей войне с
Германией?}

Советское руководство исходило именно из этого постулата. Вопрос в том,
насколько это удалось реализовать. В критические для нашей страны первые недели
гитлеровского нашествия в советской Прибалтике происходили шокирующие события.
Значительная часть населения этих республик буквально стреляла в спину
отступающим частям Красной армии и всячески помогала стремительному наступлению
вермахта.

Уже в первые дни Великой Отечественной войны немецкая авиация уничтожила
значительную часть самолетов, базировавшихся на наших авиабазах в Прибалтике.
Хаотичная эвакуация из Таллина по Балтийскому морю в августе 1941 года, о
которой в книге тоже рассказывается, стала настоящей трагедией.

\ifcmt
pic https://icdn.lenta.ru/images/2020/10/20/15/20201020155824161/pic_b7aaedaa31c4ae8d288e2802bd5b3c1f.jpg
caption Военный парад в Таллине. 7 ноября 1940 г.  Книга «Прибалтика. 1939-1945 гг. Война и память», РГАСПИ
\fi

Здесь нужно еще учесть, и это хорошо видно по документам, что в предвоенный
период многие мероприятия по укреплению обороны территорий прибалтийских
республик, особенно их побережья, были фактически провалены. Причин тут
несколько: и плохая организация этой работы — в частности, отсутствие внятных
планов ее проведения и механизмов осуществления, и откровенный саботаж со
стороны местного населения, и тот факт, что прибалтийская промышленность, «со
скрипом» переходившая на советские рельсы, не смогла быстро и качественно
обеспечить оборонные нужды СССР.

\subsubsection{Рейхскомиссариат «Остланд»}

Почему после оккупации Прибалтики летом 1941 года Германия не позволила
литовцам, латышам и эстонцам иметь свои государства или хотя бы автономии, в
отличие от хорватов и словаков, которые вообще были «расово неполноценными»
славянами?

Как я уже говорила, Гитлер смотрел на прибалтийские народы с опасливым
презрением. Поэтому даже того подобия государственности, что получили от него
хорваты и словаки (для уроженца Австро-Венгрии те были «своими славянами»),
нацисты никогда бы не разрешили латышам, эстонцам и литовцам. Те надеялись, что
немцы помогут восстановить их независимость, но у Берлина имелись иные планы.
Даже на статус колонии Прибалтика не имела шансов претендовать. Ее территория
вошла в состав рейхскомиссариата «Остланд» (в переводе с немецкого — «восточная
земля») с административным центром в Риге.

Национальные флаги были запрещены, национальные языки — тоже. Радиовещание и
театры — только на немецком. Разумеется, прибалты не должны были иметь и своей
интеллигенции — музеи, университеты, библиотеки либо закрывались, либо
разорялись. Нацисты строго запретили в Прибалтике использовать слова «свобода»,
«независимость» и «государственность» даже в агитационных целях, когда они
набирали добровольцев в Латышский и Эстонский легионы СС.

Кстати, они появились только в феврале 1943 года, после поражения вермахта под
Сталинградом, а легионеры присягали на верность лично Гитлеру. До этого немцы
разрешали создавать в Прибалтике только парамилитарные вооруженные
формирования, выполнявшие вспомогательные и карательные функции в тылу, в
прифронтовой полосе и на фронте, в том числе на территории РСФСР и Белоруссии.

\lenta{Какую политику Третий рейх проводил в оккупированной Прибалтике? Там была
тотальная германизация?}

Да, и немцы это очень четко формулировали, совершенно не скрывая свои
намерения. Так, в «Меморандуме» Розенберга о вехах активности в отношении
Латвии, Литвы, Эстонии, территории которых воспринимались как будущая
территория немецкого расселения, говорилось следующее:
