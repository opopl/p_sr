% vim: keymap=russian-jcukenwin
%%beginhead 
 
%%file letters.mariupol.demidko_olga.5
%%parent letters.mariupol.demidko_olga
 
%%url 
 
%%author_id 
%%date 
 
%%tags 
%%title 
 
%%endhead 

17:37:39 20-09-23
Вітаю ще раз! Знаєте, Олю, я не здивований. Вашої байдужістю до моєї роботи. І взагалі, тим... що Ви фактично втратили совість,
втратили якісь базові людські якості...	 Я цілий місяць витратив, переписав
усі пости Бурова, зробив дві гарні книжки, також, закачав в Архів більше ста випусків Мариуполь Былое, купу ще інших речей зробив!
І продовжую робити кожного дня! ... Але ... я жалкую, що я Вам написав. Дуже жалкую. Бо я звик писати Людям,
не бездушним роботам...
Мені шкода, знаєте, у що Ви перетворились.... у якогось жалюгідного робота, знаєте...
Робота без душі, без Серця, без совісті. Так... робот без душі, без серця... Мені Ви нецікаві... зовсім. Я підходив до Вас на виставці
вибачився, від душі... Думаєте, мені легко було...  У відповідь лише перекошене від злості лице, нічого людського...
І нащо Ви зайнялись Буровим, я от не розумію... Не хочу Вас більше ніколи в своєму житті бачити, так, ніколи!
А із книжками про Сергія Давидовича якось я і сам розберусь, так!
Які Ви такі жалюгідні і нікчемні стали, я Вас раніше дуже поважав, а зараз не знаю взагалі за що поважати, от чесно!
Можете не відповідати, я не збираюсь читати Ваші відповіді, мені це зовсім нецікаво.
Таке відчуття, що Ви стали просто частиною Мережі і все.
А на бездушних роботів я не хочу витрачати час.
На все добре.

Знаєте, Олю, за своє життя я ні разу не підняв руку на жінку а тим більше на
дитину. І ніколи не робив і не збираюсь робити по відношенню ні до будь-якої
дитини або жінки. Те що ви тримаєте на компі - це наслідки того, що ми
спілкувались у чисто віртуальному просторі, не будучи попередньо знайомими у
реальності. І зараз ми так само продовжуємо спілкуватись чисто віртуально, хоча
певна зустріч в реальності все таки відбулась.... Але не знаю чи задумувались
Ви над тим, наскільки віртуальна реальність здатна викривлювати спілкування...
наскільки спілкування у фейсбуці або телеграмі не схоже на нормальне
спілкування у реальності... Не задумувались... я теж... Ось воно і вийшло... Я
зараз дивлюсь на все це і плачу... плачу не тому, що Ви мене не простили,
звісно, то Ваше право, прощати або не прощати... а тому що... цей давній цвях у
Вашому Серці... от знаєте... він в першу чергу шкодить Вам, а не мені...  Ви і
так пашете під три чорти, на Вас дитина і хворі батьки, а ще цей цвях на додачу
від мене...  А щодо Вашої родини я вже писав у листі і повторюю зараз, і
повторю віч-на-віч якщо треба, що ніколи не було у мене бажання зробити Вам
дійсно щось зле.  А Ви можете зберігати оті скріни скільки хочете, можете
роздрукувати собі і покласти на поличку... але чи стане Вам легше жити від
того? Ваша Воля. Я не маю ненависті або зла на Вас, ніколи не мав, і ніколи не
буду мати, і я мрію, так, я мрію, щоби у Вас і у Вашої родини все було добре,
особливо з погляду на те, скільки лиха Вам прийшлось пережити під час блокади,
і також наскільки Ви талановита та непересічна особистість.

Дякую і Вам за Ваші слова! Не тримаю! Знаєте, таке відчуття... навіть важко
передати словами, це щось неймовірне просто... що гора звалилась з плеч...
величезна така гора... була, здавлювала душу і серце... бо повірте, мені часом
було дуже важко думати щодо Вас і усвідомлення того, скільки турботи і негативних емоцій я Вам заподіяв...
а тепер немає цієї гори... чудо якесь просто!  ... я ще сьогодні думав на закриття
виставки встигнути, але запізнився. Там була якась в галереї вже зовсім інша подія. 
Але взяв із собою Бурова + листівки (я весь цей скарб кожен раз вже таскаю в рюкзаку...), врешті решт хоча б якісь фоточки зробив. 
Так от! Кияни теж вміють робити гарні книжки про Маріуполь ;)

радий, якщо Вас легше стало... Тут ще такий аспект... Вибачте, якщо я Вас
відволікаю... тут я почав думати... в результаті вийшла купа слів... Дивіться.
Декілька пунктів... Справа в тому... що поки ті страшні скріни існують, завжди
є небезпека, що все зіпсується. В тому сенсі, що Ви подивитесь на оті огидні
страшні штуки зверху і знову почнете думати про мене так само... Мені буде дуже
шкода, якщо таке станеться... Буде дуже шкода... Тому... якщо Ви в глибині душі
хочете, щоби дійсно наші відносини нормалізувались і моя присутність десь на
якійсь виставці (наприклад) Вас абсолютно не напрягала... треба все це витерти
назавжди... Я щодо Вас до речі ніяких скринів у себе не тримаю... Потім... якщо
Ви хочете в душі, в серці, щоби повністю і назавжди зникла оця вся психологічно
важка ситуація щодо мене...  рано чи пізно треба буде зустрітись в реальності -
або хоча б просто зізвонитись для початку. Тут справа не в тому, що я від Вас
чогось особливо хочу під час тої зустрічі або дзвінку, достатньо буде просто
познайомитись, про щось поговорити (наприклад, я міг би ті книжки показати,
або якщо Ви робите екскурсії, прийти до Вас на екскурсію)... а в тому... що все
пішло шкереберть тому що спілкування відбувалось через чисто віртуальні образи
нас обох, при відсутності попереднього знайомства в реальності. Розумієте... і
Ви і я не були знайомі в реальності ні як колеги або ні як друзі або ж ні як
просто знайомі, і це стало головною причиною всього цього... Мені дуже шкода
все це визнавати, але це так... Я достатньо довго думав влітку, а чому ж все це
сталось саме так...  і я прийшов саме до такого висновку. Я... знаєте... не
пригадую... щоби таки страшні і огидні штуки я комусь ще колись писав... А тому
що... зазвичай я спочатку знайомлюсь в реальності... а вже потім, як
поспілкувались з людиною... тоді вже можна і на фейсбуці додатись...
Розумієте... фактично спілкувались не люди... коли існує відповідний зоровий
контакт в фізичній реальності, ну тобто... як це пояснити... наприклад, ідемо
по вулиці і про щось говоримо... ну таке... а віртуальні образи людей,
представлені у вигляді відповідних фейсбук-профілів, Вашого і мого... Також...
додало ще те, що той мій профіль був без фото і під іншим ім'ям, і тут я
скажу... це повністю моя вина в цьому... І врешті решт це призвело до того,
що... Дуже соромно мені перед Вами за все це, коли я знову дивлюсь... І крім
того... дуже-дуже важлива така штука... Ми з різних світів фактично... Ви зі
світу Маріуполя, також... страшного лиха, яке пережили Ви та Ваша сім'я і іншої
спільноти... Я ж - киянин - у мене все відбувалось і відбувається зовсім по
іншому... Це все теж зробило свій внесок... тому я не усвідомлював, наскільки
різні обставини у нас обох... І Інколи я знаєте взагалі навіть думаю із
розпачем... що я і світ Маріуполя... це як марсіане і земляне, або навпаки...
Але я звісно не можу ні про що просити Вас, тут Ви робіть як Вам буде зручно.
