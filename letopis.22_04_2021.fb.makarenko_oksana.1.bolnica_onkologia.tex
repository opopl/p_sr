% vim: keymap=russian-jcukenwin
%%beginhead 
 
%%file 22_04_2021.fb.makarenko_oksana.1.bolnica_onkologia
%%parent 22_04_2021
 
%%url https://www.facebook.com/permalink.php?story_fbid=2905321183086518&id=100008259933050
 
%%author 
%%author_id 
%%author_url 
 
%%tags 
%%title 
 
%%endhead 
\subsection{В онкоцентрі два ліфта}
\Purl{https://www.facebook.com/permalink.php?story_fbid=2905321183086518&id=100008259933050}

В онкоцентрі два ліфта. Один вузький і довгий пасажирський, правда, туди
входить з десяток людей. А другий - вантажниий і його дверима і кнопками
управляє тіточка, яка принципово відмовляється перевозити просто людей. Черга
на ліфт є завжди, і величенька. А так як ми не просто люди, а людина на
візочку, то маємо ''привілеї'' - ми викликаємо вантажний ліфт і царюємо там
самі-одні. І якби ви знали, з якою заздрістю і, навіть, трошки з ненавистю на
нас дивляться інші. От уявили, стоять всі майже однаково безволосі, безброві,
виснажено-сині, але все ж на своїх двох, дивляться на мого безволосого чоловіка
на візочку і сичать - хаааарааашооо вааам. Пздць.
