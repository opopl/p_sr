% vim: keymap=russian-jcukenwin
%%beginhead 
 
%%file 25_04_2022.stz.pc.ua.dou.1.geroizm_pogljad_kapellana.5.istorii_vadyma
%%parent 25_04_2022.stz.pc.ua.dou.1.geroizm_pogljad_kapellana
 
%%url 
 
%%author_id 
%%date 
 
%%tags 
%%title 
 
%%endhead 

\subsubsection{Історії Вадима}

З дозволу Вадима я розкажу декілька історій чудес, які ми бачили в житті його
та нашої команди. Вадим є сміливою людиною. Він один з пасторів нашої церкви. В
нього троє дітей. Він міг би спокійно поїхати разом з сім'єю в Європу.

Але він тут. Боронить Україну. Його команда евакуювала більше ніж 2000 людей з
Бучі, Ірпеня, Ворзеля, Києва, Чернігова та Донбасу. Зараз він на Донбасі. Я
вважаю, що він має отримати зірку Героя України.

\ii{25_04_2022.stz.pc.ua.dou.1.geroizm_pogljad_kapellana.5.istorii_vadyma.pic.1}

На момент написання цієї статті, сьогодні, 22 квітня, Вадим підірвався на міні
біля Лисичанська. Вони вижили, але це було реальне чудо. Може в наступній
статті я це опишу.

Далі цитати Вадима, передаю його історії, як є.

\paragraph{Жінка з Сєвєродонецька}

Вчера одного человека в Северодонецке нужно было забирать с адреса,
женщина-инвалид. Туда отказались ехать военные, там всё простреливается,
сказали, идите сами если хотите. Мы рискнули и слава Богу забрали человека.

Эта женщина всю войну просидела одна, её дом крайний в городе, кроме неё там
никого не осталось, он полностью разбит, она говорит, что лежала и кричала
«помогите», когда слышала что кто-то ходит под окнами. Зимой когда было −12, в
её квартире не было окон, на кровать где она лежала прилетали горячие осколки
от ракет.

\ii{25_04_2022.stz.pc.ua.dou.1.geroizm_pogljad_kapellana.5.istorii_vadyma.pic.2}

Чтобы не сойти с ума, она вела календарь, где записывала как прожила день. Она
слышала крики русских и стрельбу из автоматов. Ну, короче, там полная жесть,
она реально чудом осталась жива!

В дороге она радовалась всему: дождю, людям, воде, птицам.

\paragraph{Історія евакуації дитини поліцейського з Чернігова}

Мы припарковались возле полицейского отделения в Чернигове (На моєму бусі
написано «Капелан», а на інших машинах «Евакуація»)

К нам подошёл полицейский и спросил не могли бы мы эвакуировать его грудного
ребёнка вместе с супругой.

\ifcmt
  ig https://s.dou.ua/storage-files/RZ5FHKT_1.jpeg
	@caption Немовля з Чернігова
  @wrap center
  @width 0.6
\fi

У него реально не было возможности в те дни, а именно в конце марта, просто
вырваться и отвезти свою семью в безопасное место. Этот человек до конца
выполнял свой долг!

\paragraph{Історія Евакуації Максима. Хлопчика з інвалідністю}

\begin{zzquote}
Четырёхлетний Максим вместе с мамой Настей и папой Ваней проживали в Буче.
Связь с семьей пропала 26 февраля. Второго марта семья вышла на связь и
сообщила, что находится в подвале своего дома и что у них уже неделю нет света,
тепла и воды также нам сказали, что у малыша второй день температура 40.

После молитвы было принято решение идти за Максимом в Бучу. Подошли к крайнему
посту ЗСУ, военные сказали, что не пропустят, поскольку идёт бой и вообще
никого не пускают только выпускают. Очень активно работала артиллерия, над
головой летели пули от трассировочного огня.

Немного подождав, удалось подсесть в авто к одному из местных и переправиться
через мост. Найдя дом, где проживала семья Вани, мы спустились в подвал и нашли
там Максима вместе с родителями, у малыша был сильный жар. Взяв необходимые
вещи, мы вышли с подвала и пошли в сторону моста, по которому можно было
перейти в Ирпень, по дороге было много разбитой техники, дворовые собаки,
которые бегали по улице и сильно скулили от взрывов артиллерии и пытались
просто прижаться к тебе.

Возле самого моста горел подбитый танк. Когда мы подошли к мосту, то поняли,
что он полностью виден врагу и простреливается со всех сторон. Выхода у нас не
было и, помолившись, мы вышли на мост и начали потихоньку двигаться в сторону
Ирпеня. На другой стороне моста уже стояли наши военные они собрались и затаив
дыхание смотрели, как мы идём, когда мы подошли к центру моста начались сильные
взрывы артиллерии, военные стали кричать «бегите к нам» и когда оставалось
пройти четверть моста мы все побежали. Добежав до наших военных, мы оказались в
безопасности, они угостили Максима конфетой и отвезли нас к переправе через
реку.

Переправившись, мы сели в свой автомобиль и отвезли ребят в наш лагерь для
переселенцев. Спустя несколько часов после эвакуации мне сообщили, что в дом,
где пряталась семья, прилетела ракета. Сейчас малыш в безопасности и полностью
здоров. Слава Иисусу Христу. Через неделю в тот дом в Буче попала ракета.
\end{zzquote}

В нас можна забрати все, але ніхто не зможе забрати нашу Волю, Віру в Бога і
Віру в Перемогу. Ця війна — є Великою Визвольною для нашого народу. І ми
переможемо!

\ifcmt
  ig https://s.dou.ua/storage-files/q7uR7d6.png
	@caption Праця Капеланів від ХСП
  @wrap center
  @width 0.8
\fi
