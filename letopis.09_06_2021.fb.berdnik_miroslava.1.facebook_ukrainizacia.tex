% vim: keymap=russian-jcukenwin
%%beginhead 
 
%%file 09_06_2021.fb.berdnik_miroslava.1.facebook_ukrainizacia
%%parent 09_06_2021
 
%%url https://www.facebook.com/permalink.php?story_fbid=1222284798223768&id=100013267772922
 
%%author 
%%author_id berdnik_miroslava
%%author_url 
 
%%tags 
%%title Пишет Юрий Ткачёв: Открыл для себя волшебный мир украинизаторских групп в Facebook
 
%%endhead 
 
\subsection{Пишет Юрий Ткачёв: Открыл для себя волшебный мир украинизаторских групп в Facebook}
\label{sec:09_06_2021.fb.berdnik_miroslava.1.facebook_ukrainizacia}
\Purl{https://www.facebook.com/permalink.php?story_fbid=1222284798223768&id=100013267772922}
\ifcmt
 author_begin
   author_id berdnik_miroslava
 author_end
\fi

Пишет Юрий Ткачёв:

Открыл для себя волшебный мир украинизаторских групп в Facebook. Идея такая:
они находят фирмы, использующие русский в общении с клиентами и устраивают им
травлю. В целом не ново (так-то Фейсбук за такое банит, но не политических
украинцев, им это разрешено).  Но дело не в этом.  Вот какую волшебную жалобу я
нашёл в одной из таких групп! Дети, видите ли, говорят \enquote{спасибо}, а не
\enquote{дякую}. 

\ifcmt
  pic https://scontent-lga3-2.xx.fbcdn.net/v/t1.6435-9/197313194_1222282528223995_2432787769612168550_n.jpg?_nc_cat=110&ccb=1-3&_nc_sid=730e14&_nc_ohc=tRZtwFX5WEkAX8Dw1AR&_nc_ht=scontent-lga3-2.xx&oh=1f406083a9122899495520ac6f8446d6&oe=60E8507B
\fi

А эти ДБ-доносители не знают, что украинское слово \enquote{спасибі} -
производное от \enquote{Спаси біг}?

Когда я в детстве у бабушки в селе говорила \enquote{дякую} (это городское
выражение) вместо  \enquote{спасибі}, мне всегда говорили \enquote{Дякую - по
спині ломакою}
