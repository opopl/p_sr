% vim: keymap=russian-jcukenwin
%%beginhead 
 
%%file 16_11_2020.news.ua.strana.1.sputnik_V_peshko
%%parent 16_11_2020
 
%%url https://strana.ua/news/301328-uchenyj-prizval-zelenskoho-kupit-vaktsinu-ot-koronavirusa-u-rf-2020.html
%%author 
%%tags 
%%title 

%%endhead 

\subsection{Вице-президент Академии экономических наук призвал Зеленского закупить российскую вакцину от коронавируса}
\label{sec:16_11_2020.news.ua.strana.1.sputnik_V_peshko}
\Purl{https://strana.ua/news/301328-uchenyj-prizval-zelenskoho-kupit-vaktsinu-ot-koronavirusa-u-rf-2020.html}

23:15, 16 ноября 2020

\ifcmt
pic https://strana.ua/img/article/3013/28_main-v1605560371.jpeg
caption Обращение к Зеленскому. Фото: yandexcloud
\fi

Первый вице-президент Академии экономических наук Украины Анатолий Пешко
призвал президента Владимира Зеленского купить российскую вакцину от
коронавируса \enquote{Спутник V},\Furl{https://strana.ua/news/300582-v-voz-vaktsinu-sputnik-v-vnesut-v-spisok-rekomenduemykh-dlja-ekstrennoho-ispolzovanija-.html} несмотря на продолжающуюся войну. 

Текст письма он опубликовал на своей странице в Facebook.

\enquote{Уважаемый Владимир Александрович! Эпидемия коронавируса в Украине набирает
угрожающих масштабов. Прирост заболевших в 11-12 тысяч человек за сутки очень
быстро поставит разваленную отечественную систему здравоохранения на колени.
Украинцев просто негде будет лечить и люди будут умирать, не дождавшись
медицинской помощи. Это ужасная перспектива, которую карантин выходного дня, к
сожалению, не сможет предотвратить.  Сегодня уже всему научному сообществу
очевидно, причем, не только в Украине, но и во всем мире, что для борьбы с
эпидемией необходимо принимать любые меры, которые позволят поднять Т-клеточный
иммунитет, что в свою очередь поможет привести к усилению защитной реакции
каждого человека}, \dshM указано в сообщении. 

По его словам, Израиль, ОАЭ, Саудовская Аравия, Венгрия, Бразилия, Китай, Южная
Корея, Турция и Индия уже якобы начали заключать договоры по поставке
российской вакцины, поэтому нужно отбросить политические мотивы и спасать жизни
украинцев. 

\enquote{По данным директора научно-исследовательского центра эпидемиологии и
микробиологии имени Гамалеи Александра Гинцбурга, чье учреждение и разработало
вакцину, клеточный иммунитет к коронавирусу после применения препарата \enquote{Спутник
V} может сохраняться до двух лет. Чего вполне достаточно, чтобы пресечь цепочку
распространения вируса. Кроме того, Россия уже зарегистрировала и вторую
вакцину \dshM \enquote{ЭпиВакКорона} производства научного центра \enquote{Вектор}. Она проходит
пострегистрационные исследования. А центр имени Чумакова РАН разработал свою
вакцину, который содержит цельный вирус. Данный препарат также проходит стадию
клинических испытаний. Как ожидается, зарегистрирована вакцина может быть до
конца года. При этом вакцина фирмы Pfizer, о которой так много говорят в
последнее время, имеет логистические проблемы, связанные с условиями хранения.
Кроме того, данная вакцина намного дороже российского аналога}, \dshM добавил он. 

Пешко призвал Зеленского купить образцы вакцины и производить их в Украине,
наращивая экономику страны. \enquote{чтобы не пострадал украинский народ}.

\enquote{Я призываю Вас Владимир Александрович, отбросив все политические разногласия,
подумать о спасении жизней тысяч украинцев! Ведь они могут умереть от
смертоносного вируса в ближайшие месяцы. Европейский Союз, несмотря на режим
санкций против РФ, может себе позволить разрешить использование российской
вакцины на своей територрии. Эту вакцину закупают такие страны как . Все мы
обязаны в первую очередь думать об украинском народе. Политика подождет. Люди
хотят жить. Пожалуйста, приобретите вакцину, дайте украинцам шанс выжить}, \dshM резюмировал ученый. 

\ifcmt
pic https://strana.ua/img/forall/u/11/58/photo5303197387289899381.jpg
pic https://strana.ua/img/forall/u/11/58/photo5303197387289899382.jpg
pic https://strana.ua/img/forall/u/11/58/photo5303197387289899383.jpg
\fi

Напомним, Казахстан и Россия достигли договоренности о строительстве
завода по производству российской вакцины от коронавируса.\Furl{https://strana.ua/news/301297-v-kazakhstane-postrojat-zavod-dlja-proizvodstva-rossijskoj-vaktsiny-ot-covid-19.html}

Как сообщала \enquote{Страна}, во Всемирной организации здравоохранения (ВОЗ)
спрогнозировали, что Украина сможет получить первую
партию зарегистрированной вакцины от коронавируса во втором квартале 2021
года.\Furl{https://strana.ua/news/300838-v-voz-nazvali-sroki-postavok-v-ukrainu-pervoj-partija-vaktsin-ot-koronavirusa.html}

