% vim: keymap=russian-jcukenwin
%%beginhead 
 
%%file 02_03_2023.fb.kipcharskij_viktor.mariupol.1.den_7___2_03_22__ser.cmt
%%parent 02_03_2023.fb.kipcharskij_viktor.mariupol.1.den_7___2_03_22__ser
 
%%url 
 
%%author_id 
%%date 
 
%%tags 
%%title 
 
%%endhead 

\qqSecCmt

\iusr{Наталия Павлова}

Нас безжалостно сдали. А новости были такие, что мы думали, что через 10 дней Донецк освободят.

\iusr{Марина Солошенко}

Чомусь депутати і їх родини не постраждали, як і родини великих бізнесменів,
вони знали і виїхали заздалегідь, наш Ярославський так поспішав, що збив
людину, але на літак сів! А для городян (за 2 неділі до війни) в підвалах
пологових будинків готувались до прийому дітей війни ( ми про це тільки зараз
узнали)!

\begin{itemize} % {
\iusr{Віктор Кіпчарський}
\textbf{Марина Солошенко} 

Зеленський казав, що з середини січня не було квітків на літаки з України.

Але мене бісить не це, а те, що ті самі \enquote{народні обранці}, що втекли від своїх
виборців і навіть не попередили, тепер засуджують тих, хто залишився і мусить
виживати на оті жалюгідні \enquote{пайки}, що їм дають \enquote{визволителі}, знімаючи
\enquote{відосики} про \enquote{вдячних} маріупольців...

\iusr{Vitaliy Snihur}
\textbf{Віктор Кіпчарський} 

квитків було багато і досить дешевих у лютому, принаймні у Wizzair. Я кілька
разів відкладав відпустку, мав квиток на 20 грудня в Ліссабон, але переніс на
початок квітня. Мав також квиток в Берлін на 6 березня. В Берлін квитки були
десь від 20Є з Києва

\iusr{Віктор Кіпчарський}
\textbf{Vitaliy Snihur} Цитую:

\enquote{Щодо політиків, дипломатів, бізнесменів... Ви сказали, 30 чартерів лише вчора
полетіли?... Це говорить, що у нас багата країна! Ось пан Шольц каже, що треба
інвестувати в Україну... Ось бачите, у нас вчора 30 чартерів відлетіло...
Головне, щоб вони порожніми летіли, а не вивозили щось... Я думаю, залишаються
надійні.}

Нагадаємо, у ЗМІ оприлюднили список нардепів, які покинули країну.

Також деякі видання повідомили про масовий відліт з України чартерів
бізнесменів та олігархів.

\href{https://varta1.com.ua/news/volodimir-zelenskiy-vidreaguvav-na-evakuaciyu-ta-vilit-diplomativ-deputativ-ta-oligarhiv-z-ukrayini_341215.html}{%
Володимир Зеленський відреагував на \enquote{евакуацію} та виліт дипломатів, депутатів та олігархів з України, %
varta1.com.ua, 14.02.2022%
}

\ifcmt
  igc https://i2.paste.pics/66be557ed6916570593afe2adb0b7899.png
	@width 0.5
\fi

\iusr{Віктор Кіпчарський}
\textbf{Vitaliy Snihur} 

Станом 14 лютого, в Україні відсутні 23 народних депутати.

Про це пише \enquote{Українська правда} з посиланням на свої джерела в правоохоронних органах

Зокрема, йдеться про таких народних обранців:

\begin{itemize}
  \item Кива Ілля (ОПЗЖ), 30 січня відправився в Аліканте (Іспанія);
  \item Королевська Наталія (ОПЗЖ), 9 лютого полетіла в Ригу (Латвія);
  \item Льовочкін Сергій (ОПЗЖ), 10 лютого полетів до Венеції (Італія);
  \item Львочкіна Юлія (ОПЗЖ), 26 січня відправилась у Ніццу (Франція);
  \item Рабінович Вадим (ОПЗЖ), 3 лютого полетів у Тель-Авів (Ізраїль);
  \item Новинський Вадим (позафракційний), 10 лютого відправився у Мюнхен (Німеччина);
  \item Устінова Олександра (\enquote{Голос}), 6 лютого полетіла у Дюссельдорф (Німеччина);
  \item Железняк Ярослав (\enquote{Голос}), 12 лютого відправився у Париж (Франція);
  \item Абрамович Ігор (ОПЗЖ), 12 лютого вилетів до Варшави (Польща);
  \item Аліксейчук Олександр (\enquote{Слуга народу}), 5 лютого вилетів у Доху (Катар);
  \item Аллахвердієва Ірина (\enquote{Слуга народу}), 4 лютого полетіла в Дубай (ОАЕ);
  \item Плачкова Тетяна (ОПЗЖ), 13 лютого полетіла у Відень (Австрія);
  \item Борт Віталій (ОПЗЖ), 3 лютого полетів до Стамбула (Туреччина);
  \item Пузанов Олександр (ОПЗЖ), 13 лютого вилетів у Доху (Катар);
  \item Іванісов Роман (позафракційний), 11 лютого вилетів у Париж (Франція);
  \item Кривошеєв Ігор (\enquote{Слуга народу}), 4 лютого відправився у Мадрид (Іспанія);
  \item Нагорняк Сергій (\enquote{Слуга народу}), 11 лютого полетів у Цюрих (Швейцарія);
  \item Пивоваров Євген (\enquote{Слуга народу}), 11 лютого полетів у Шарджу (ОАЕ);
  \item Солод Юрій (ОПЗЖ), 9 лютого відправився у Ригу (Латвія);
  \item Шпенов Дмитро (позафракційний), 12 лютого полетів до Женеви (Швейцарія);
  \item Столар Вадим (ОПЗЖ), 12 лютого відправився до Ніцци (Франція);
  \item Яковенко Євген (позафракційний), 12 лютого полетів у Стамбул (Туреччина);
  \item Волошин Олег (ОПЗЖ), 14 лютого залишив Україну, перетнувши на автомобілі кордон з Білоруссю.
\end{itemize}

\href{https://varta1.com.ua/news/u-zmi-oprilyudnili-spisok-nardepiv-yaki-pokinuli-krayinu_341214.html}{%
У ЗМІ оприлюднили список нардепів, які покинули країну, varta1.com.ua, 14.02.2022%
}

\ifcmt
  igc https://i2.paste.pics/7a8d13a5889c6734c2a37cf3eaac8f25.png
	@width 0.5
\fi

\iusr{Віктор Кіпчарський}
\textbf{Vitaliy Snihur}

\href{https://varta1.com.ua/news/masovo-tikayut-z-ukrayini-povilitati-charteri-biznesmeniv-ta-oligarhiv---zmi_341196.html}{%
Масово тікають: з України повилітати чартери бізнесменів та олігархів, — ЗМІ, varta1.com.ua, 14.02.2022% 
}

\ifcmt
  igc https://i2.paste.pics/fd361e0d2ca166786535c68cbb9046a3.png
	@width 0.5
\fi


\iusr{Vitaliy Snihur}
\textbf{Віктор Кіпчарський} я поділився своїм досвідом та спостереженнями.
те що писали ЗМІ - не читав 🙂

\iusr{Віктор Кіпчарський}
\textbf{Vitaliy Snihur} навряд чи олігархи та депутати купували квітки на Wizzair.
До того ж я писав про мера і депутатів нашого міста.
Можете поцікавитись, що було, наприклад, у Херсоні.

\iusr{Vitaliy Snihur}
\textbf{Віктор Кіпчарський} ок

\end{itemize} % }

\iusr{Елена Девина}

А ви в якому районі жили?

\begin{itemize} % {
\iusr{Віктор Кіпчарський}
\textbf{Елена Девина} наступні пару тижнів показали, що у найтихішому районі березня 2022 - на нижче Нептун, на Гонди.

\iusr{Елена Девина}
\textbf{Віктор Кіпчарський} На Гонди у приватному секторі-батьківська хата мого чоловіка(розтрощена.. )

\iusr{Віктор Кіпчарський}
\textbf{Елена Девина} Так, кілька прильотів навколо нас було - навіть у у сусідню п'ятиповерхівку (42-й будинок) влетів снаряд, але не вибухнув. У 46-й прилетів уламок, але в цілому було спокійно.
\end{itemize} % }

\iusr{Zhanna Savelyeva}

Так, вогонь прийшов до Східного. Потім з'ясувалося, як ні дивно, так же швидко
і пішов, тому деякі дома залишилися цілими...

\iusr{Zhanna Savelyeva}

Мамин двір 2 березня. Знайшла у ТГ.


\ifcmt
  tab_begin cols=3,no_fig,center,separate,amount=5,layout=3.2

	% 1 - 0.02
     pic https://i2.paste.pics/df34b8a36d60ec6b3da9e520f0222aec.png
	% 2 - 0.07
		 pic https://i2.paste.pics/dd2166f152f9f1baad60a7dd4cdcc5cf.png
	% 3 - 0.09
		 pic https://i2.paste.pics/3d4e85a5a259c59e6f8b25dc5020a7a5.png
  % 4 - 0.13
		 pic https://i2.paste.pics/cd3847dc000a5184963322f9558d0f4a.png
  % 5 - 0.15
		 pic https://i2.paste.pics/e42316476b44fd330e12bfacc1bf5e94.png

  tab_end
\fi

\iusr{Roy Evans}

Коли вже не було світла, та трохи згодом почалися проблеми зі зв'язком, я почав
у ручному режимі шукати gsm-мережі, то бачив тільки якісь 5 цифр. Вважав, що то
якась мережа для військових, але згодом виявилось, що це був той самий фенікс.
І це було як раз напочатку березня. Було страшно, але я ще їздив до мами та
бабусі, і побачивши що у місті, вмовив маму їхати з дому, бо 5й поверх
п'ятиповерхівки не саме надійне місце. Нащастя, це було зроблене вчасно, бо
потім туди прилетіло у балкон. А ще з цікавого: спостерігав за тг каналом, де
шукають рідних, та публікують де хто загинув, і от по датах збігаються райони
штурмів покидьків. Досі не можу зрозуміти, як їх можуть підтримувати.
