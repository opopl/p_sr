% vim: keymap=russian-jcukenwin
%%beginhead 
 
%%file 05_05_2021.fb.respublikalnr.1.lnr_sport_avtogonka
%%parent 05_05_2021
 
%%url https://www.facebook.com/groups/respublikalnr/permalink/807482823220844/
 
%%author 
%%author_id 
%%author_url 
 
%%tags 
%%title 
 
%%endhead 
\subsection{Газета \enquote{Республика} (№17, 2021г).  Спорт.  Побеждает самый быстрый}
\Purl{https://www.facebook.com/groups/respublikalnr/permalink/807482823220844/}

В парке «Патриот» прошёл второй этап Кубка и Первенства ЛНР по автомногоборью.

c открытых соревнованиях за право обладания Кубком ЛНР принимали участие
спортсмены старше 18 лет, в первенстве же свои навыки управления автомобилем
показывали юные водители от 10 до 17 лет.

– Некоторые дети и раньше принимали участие в организованных РАФ автогонках, но
вне конкурса, это были просто показательные выступления. В этом году мы
проводим первое официальное первенство среди детей, – рассказала президент
Республиканской автомобильной федерации Юлия Живолуп.

Справиться с поставленной задачей детям помогали наставники, которые
преодолевали маршрут вместе с юными спортсменами, но в роли штурмана – на
пассажирском сиденье.

– Молодые спортсмены едут либо с родителями, либо с тренером – это обязательное
требование первенства. Это дети, и как бы хорошо они ни ездили, должна быть
страховка, – объяснила Юлия Живолуп.

Самой юной участнице первенства, Насте Коверге, всего 10 лет. Девочка села за
руль два года назад и уже уверенно преодолевает крутые виражи наравне с
опытными автогонщиками. А ведь правила для детей и взрослых одни: своевременно
стартовать, проехать трассу по схеме, не задев конусы, и правильно
финишировать. При этом не стоит забывать, что и скорость в спринт-слаломе имеет
огромное значение.

Надежда ПЕРЕСВЕТ ГАЗЕТА "РЕСПУБЛИКА" (№17, 2021г).

\verb|#газета #республика #автомотоспорт #Первенство_ЛНР|
