% vim: keymap=russian-jcukenwin
%%beginhead 
 
%%file 30_10_2020.fb.max_buzhanskii.2.hmel_zhena
%%parent 30_10_2020
 
%%url https://www.facebook.com/permalink.php?story_fbid=937758929721920&id=100004634650264
%%author 
%%tags 
%%title 
 
%%endhead 

\subsection{Самая желанная баба - чужая}
\label{sec:30_10_2020.fb.max_buzhanskii.2.hmel_zhena}

\Purl{https://www.facebook.com/permalink.php?story_fbid=937758929721920&id=100004634650264}
\Pauthor{Бужанский, Максим}

Самая желанная баба - чужая.
Факт.

И Хмельницкий желал.

Трудно уже сейчас сказать, кто именно, Гетман или пан Чаплинский первым стал
мужчиной судьбы злополучной Гелены, но запала она им в душу крепко.

Не будь её, возможно история Украины, Польши с Литвой, а может быть и Царства
Московского, пошла бы по какому то другому пути.

Но она была, и всё пошло туда, куда пошло.

Говорят, что прибеднялся Хмельницкий настолько не бедно, что Гелена, шляхтянка
из семьи обедневших шляхтичей, служила у него нянькой детей.

И, будучи на тот момент православной, крестила одного из них.

А что ж, как говорится, за кума, что под кумом не была?
Была.

Нет никаких исторических свидетельств, что Хмельницкому отказала хоть
какая-нибудь баба, так что Гелену можем смело записывать на его счет.

На свою беду, и на беду Речи Посполитой, она, была единственной симпатичной дамой в округе, и нагрянувший грабить хутор Суботов Чаплинский, уволок её с собой.
Этого Хмельницкий стерпеть не мог.
И понеслась...
Под скрежет уже начавшей обрушение Речи Посполитой, Чаплинский перекрестил несчастную Гелену по католически, и обвенчался с ней.
А ведь в наших краях убивали и за меньшее.
Впервые выслушавший эту историю Тугай Бей был потрясён до глубины души!
Нет, конечно он с удовольствием пригнал свою орду на помощь Гетману, как договаривались, и искренне рассчитывал на победу, грабеж, добычу, пленных и прочие, чисто мужские радости.
Хан, провожая, так и сказал, хлопнув нагайкой по спине- держись там, и хорошего тебе настроения.
Но чтоб такая романтика!!- Тугай Бей был под впечатлением.
Цивилизация раскрывалась неведомыми гранями, неизведанными и волнующими.
Грянула война, закрутилось-завертелось, понесло, будто диким водоворотом, сражения, победы, пленные польские Гетманы и бесконечные трофеи.
Глаза разбегались, было очень трудно держать себя в руках, держать в них же запорожцев и показаченых, а уж Тугай Бея с его заезжей бандой так вообще, хоть привязывай на ночь.
Но вот, грянули Пилявцы, и Польша рухнула беззащитной, а вместе с ней,в широко расставленные руки Гетмана, и Гелена, к этому времени уже Чаплинская.
Зажили вместе, крепкой семьей.
Он, она, 60 тысяч казаков и тысяч десять тугай-беевских татар.
Всех очень беспокоил один вопрос.
Нет, не как жить дальше.
Как можно жить дальше, если Гетман живет нерасписанным, так сказать,с шляхтянкой-католичкой.
Вот всё было хорошо, а тут никак.
Хмельницкий обратился не куда-нибудь,а к Киевскому Митрополиту.
Мол так и так, вот...
Митрополит отшатнулся от него, как от сумасшедшего.
Обвенчать чужую жену с Гетманом, это было уже явно через край.
Наотрез.
Надо сказать, что православное местное духовенство вообще не сильно подпрыгивало от счастья, наблюдая успехи в национально-освободительной борьбе.
И меньше всего на свете хотело воссоединяться с кем-нибудь, имеющим отношение к Московскому Патриархату.
Хотели чтобы поляки прыть поумерили, не более.
Короче, уломать Сильвестра Коссова не получалось никак, а выход нужно было искать срочно.
И он нашелся.
Так вышло, что совсем недалеко, в молдавских Яссах, отирался Патриарх Иерусалимский Паисий.
Затравленный всеми, нищий и несчастный.
Круче не было уже никого, и к Паисию немедленно отправилась группа товарищей, с целью приглашения в Киев на предмет Свадьбы.
Паисий дал себя уговорить, едва взглянув на послов.
Живописно наброшенные на голые, не знавшие сорочек плечи трофейные жупаны, трофейные же сапоги, иногда разных цветов, кольца, серьги, усы, топорики.
Ну как таким откажешь, а?
Ударили по рукам, причем свою Паисий потом два дня носил на перевязи.
В начале зимы, с трудом сдерживающий нетерпение Хмельницкий отправил в Киев полковника Нечая, с приказом очистить город от поляков, униатов и евреев.
До последнего!- напутствовал Гетман Нечая.
Что вы собрались с ними делать? -тихо спросил Тугай Бей, бесцельно отиравшийся в гетманском шатре.
Отвезем в хорошее место- мрачно ответил Хмельницкий. 
Тугай Бей, сощурив и без того узкие глаза , смотрел на гетмана, будто видя его впервые.
Погром был диким.
Беззащитных резали с жутким остервенением, не щадя никого.
Городская власть, еще вчера такая толерантная и цивилизованная, встала на сторону погромщиков, чтобы самой не загреметь куда-нибудь, под шум войны за независимость.
Митрополит Киевский своего неудовольствия не выразил никак.
И вот, долгожданный момент.
Киев ждёт!!
Буквально перед самим въездом Хмельницкий обнаружил возле себя празднично одетого Тугай Бея, в окружении небольшой группы татар, наряженных, как цыганский табор.
Куда собрался?- машинально спросил Гетман, думая о чем то своем.
На Свадьбу, куда ж ещё?!- изумленно ответил Тугай.
Чью?- осторожно спросил Хмельницкий, почуяв недоброе.
Твою, чью ж еще?!- искренне изумился Тугай-Бей, любуясь своим отражением в трофейном панцире крылатого гусара, стоявшим в углу.

Видишь ли... Ну сам то подумай, христиане, киевляне, Патриархи, праздник... ну
куда там ты, мерзость иностранная..? - как можно более ласково  начал
Хмельницкий.

Тугай Бей остолбенел.

То есть, - обычно красная рожа татарина сделалась багровой,- всем можно, а мне нет??!! 

У тебя жена есть? - деловито осведомился Гетман.

Шесть, может семь, не помню!- злобно процедил татарин.

Ну вот, ты всё уже видел...- вкрадчиво приобнял мурзу за плечи вождь восставшего народа.

Тугай Бей постоял, молча скрипнул челюстями, пнул служивший стулом барабан и
рывком отшвырнув полог шатра убрался прочь.

Взвизгнули дудки, и спустя час, поднимая за собой гигантский столб пыли,
татары, снявшись с лагеря, двинулись прочь.

Свистели нагайки, разрывая кожу на спинах пленных, колонна, бесконечной змеей вытягивалась на дороге.

Тугай Бей молча раскачивался в седле, мрачно закусив губу.

Ему было обидно.

А на следующий день, грянули трубы!

Барабаны гремели, ржали кони, гул толпы вливался в эту смесь звуков и глушил её.

Распахнутые Золотые Ворота, тысячи горожан, Патриарх и Митрополит.

Шитые золотом рясы, сжатые в руках посохи.

На белом коне, медленным, торжественным шагом, въезжал в Киев Гетман
Запорожского Войска, победителем, триумфатором.

Бесконечный строй запорожской пехоты лился сквозь ворота, и граждане изумленно смотрели на эту невиданную мощь!

Студенты Киево-Могилянской коллегии хором читали стих на латыни.

В стихе ясно давалось понять, что рядом с Гетманом, Александр Македонский-жалкий червяк.

На глазах многотысячной толпы, Патриарх Иерусалимский Паисий обвенчал по прежнему состоявшую в браке с Чаплинским католичку Гелену с Гетманом Богданом Хмельницким.

Перекрестив её обратно в православие, предварительно, под кислыми взглядами
Митрополита Сильвестра и киевского духовенства.

Начиналась новая жизнь, торжественная, пугающая своей неизвестностью и ей же манящая!

Тут бы и сказать, что жили они долго и счастливо.

Но нет, не жили.

Спустя три года, к раздавленному разгромом под Берестечком, огромными потерями, гибелью старого друга, Тугай Бея, Гетману примчал гонец.

От старшего сына, Тимофея.

Со скупым известием, что Тимофей обвинил несчастную Гелену в супружеской измене
и краже из казацкой казны, которую она никогда в глаза не видела.

И повесил голой на воротах Чигирина.

Подробностей Хмельницкий не выяснял.

Колесо истории волокло всех за собой, наматывая на ось и нещадно давя.

Было уже не до подробностей.
