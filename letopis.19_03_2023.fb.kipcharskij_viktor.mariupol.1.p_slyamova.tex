%%beginhead 
 
%%file 19_03_2023.fb.kipcharskij_viktor.mariupol.1.p_slyamova
%%parent 19_03_2023
 
%%url https://www.facebook.com/permalink.php?story_fbid=pfbid02hUovwMtthJfa6uPQv58Et9hkHRx6jF4Qn4WceQGc4pMpxCMJRgbS9d8SRJ1S61nAl&id=100006830107904
 
%%author_id kipcharskij_viktor.mariupol
%%date 19_03_2023
 
%%tags mariupol,dnevnik,mariupol.war
%%title Післямова
 
%%endhead 

\subsection{Післямова}
\label{sec:19_03_2023.fb.kipcharskij_viktor.mariupol.1.p_slyamova}

\Purl{https://www.facebook.com/permalink.php?story_fbid=pfbid02hUovwMtthJfa6uPQv58Et9hkHRx6jF4Qn4WceQGc4pMpxCMJRgbS9d8SRJ1S61nAl&id=100006830107904}
\ifcmt
 author_begin
   author_id kipcharskij_viktor.mariupol
 author_end
\fi

Післямова

Рік тому ми прокинулися в рідному селі мого тата, в будинку моєї двоюрідної
сестри Люби та її чоловіка Миколи. Приїхали ми напередодні ввечері і
\enquote{населення} будинку збільшилося втроє.

На вечерю зібралися родичі: мої племінниці, чоловік старшої та дві їх дочки -
мої онучки. З усіх я знав лише Любу, як востаннє бачив півстоліття тому. 

\ii{19_03_2023.fb.kipcharskij_viktor.mariupol.1.p_slyamova.pic.1}

Мій недолік і моє нещастя в тому що я виріс у Заполярній Кандалакші, де служив
тато і де в нас не було жодних родичів, тож дідусі з бабусями не розповідали
нам з сестрою казки, не співали пісні. Тато час від часу заступав на чергування
на цілу добу, а мама викладала у вечірній школі, тож вкладалися ми самі. На
літо батьки влітку, \enquote{зібравши} відпустки, возили нас на оздоровлення на
Вінниччину - отоді я й бачився зі своїми родичами Коли тато у свої 42 (його
призвали у 17 плюс 25 років служби) роки пішов на пенсію (!), він забажав
поїхати до теплого моря, аби гріти на сонці \enquote{заслужений} радикуліт, тож ми
переїхали до Жданова, де в нас теж не було жодного родича. Моє нещастя в тому
що я не звик до спілкування з родичами, не знав і не ділив з ними їх радостей
та проблем. Тому та гостинність, з якою нас зустріли незнайомі родичі, мене
ошелешила.

Хазяї віддали нам свої кімнати: Люба з Миколою перейшли до малесенької кімнати
(скоріше - комірчини), де стояло лише вузьке ліжко і був вузенький прохід між
ним та стіною. Напевно, вони через мій храп не спали перші півночі.

Дітей з онуками розмістили у кімнаті хазяйської дочки Віталіни. Щоранку кіт
приходив під двері і нявчав, вимагаючи, аби йому відкрили двері. Потім він
наполовину заходив і дивився на диван, не розуміючи де його хазяйка. Потім
незадоволено фиркав і йшов шукати її - і так щоранку, поки ми там були. 

За звичкою (а може то сільські гени?) я прокинувся серед ночі і намагався
звикнути до тиші. 

З того часу пройшов рік і знову я не сплю серед ночі, переживаю спогади,
аналізую розповіді людей, що писали коментарі до моїх постів і намагаюсь
зрозуміти: чому нас оминули ті жахи, що випали на долю інших?

Одна дівчинка (сподіваюсь, я не ображу цим Світлану - вона молодша за моїх
синів і хоч має трьох діточок, але для мене за віком вона дівчинка) дозволила
мені об'єднати її коментарі і викласти їх, що я й зроблю тохи пізніше - я ще не
оговтався від своїх спогадів. Тож порівнюючи пережите нами із тим, що випало на
долю інших маріупольців, я намагаюсь зрозуміти, чому нам так пощастило: в
нашому дворі, оточеному трьома п'ятиповерхівками, не загинула, не була поранена
і навіть не померла жодна людина за три тижні.

Навколо гинули люди: на сусідній вулиці Покришкіна, 20, на проспекті
Металургів, 205 та 219, приватний сектор на Гонди, Кальміуська адмінистрація...
У 42-й будинок влучив снаряд, але, по-перше, він не вибухнув, а по-друге, це
було від вулиці, а не у дворі...

Чому? 

Навколо нас стояло кілька батарей, госпіталь, над нами літали літаки. Над
головами (в черзі біля Нептуна) літали міни.

Нещастя оминули наш двір.

Чому?

В той самий день, як ми виїхали, обстріляли колону з біженцями біля
Степногорська; на блок-пості за Мангушем БТР переїхав машину і вбив мого друга
та покалічив його дружину. 

Над вечір того самого дня на блок-пості перед Мангушем, через який нас
пропустив вояка, що хотів швидше випити, людей обшукували і знущалися над ними.
Нас це оминуло.

Чому?

Мені кажуть: ваш будинок побудували там, де раніше було поле: там не жили люди,
там не було церков, які руйнували у тридцяті роки, там не було кладовищ, на
яких будували будинки - там була "чиста енергетика". Але будинки, які я назвав
раніше, у які "прилітало" і у яких гинули люди теж будували серед поля.

Чому?

Я, механік за освітою, програміст за захопленням - людина далека від містики, -
не можу знайти логічні відповіді на ці "чому". Якось одна людина сказала мені:
"Ти намагаєшся зрозуміти Бога, а треба вірити." Та ж людина сказала мені: "Не
питай "Чому?" та "За що?" - питай "Для чого?".

Наступного дня, у неділю, натщесерце (не поснідавши), ми з Олею і онуком (його
ніхто не примушував - він сам сказав, що піде з нами і навіть відмовився хоча б
випити чаю) пішли до церкви - я подякував за те, що ми виїхали з того кошмару
про який напередодні дізналися з теленовин і попросив врятувати тих, хто
залишився там. 

Написалося: На іншій стороні війни...

\obeycr
На іншій стороні війни птахи співають на світанку,
Я зачарований стою. І дихаю. Радію ранку.
Як довго я на це чекав: так довго, що собі не вірю
В цю неймовірную весну, в цей ранок, спів і в цю надію,
Це не наснилися мені під гуркіт бомб, ракет, снарядів,
Що смерть несли в мій дім, в мій двір, в мою сім'ю на залпах Градів.
Я знаю, важко буде нам, та вірю: проженемо гадів, 
Що без запрошення прийшли щоб нас від нас "асвабажать".
І змусимо їх "руській мір" скрізь за собой поприбирать.
Ми відбудуємо міста, насадимо дерева й квіти,
Та не забудемо людей, нам дали можливість жити.
І тих Героїв й Героїнь, що нас собою захищали.
І, може, Господи, дай сил, я дітям внуків передам, 
Про те, як "братья" нас вбивали...
\smallskip
На іншій стороні війни ми б цілу ніч спокійно спали.
Коли не спогади про тих, с ким разом вдома виживали.
З ким ми ходили по дрова (і це - у промисловим місті!),
Носили воду, гріли сніг, на вогнищі варили їсти. 
Де ви? Що з вами? Напишіть - немає сил в безвісті жити.
В неділю в церкву я пішов - про захист людям попросити...
\restorecr

Навздогін

В жодному разі не пов'язую те, що Бог (не подобається - замініть на випадок,
доля, карма, сили природи чи ще щось) зберіг саме нашу родину - після того, як
ми поїхали, наш двір ще деякий час був \enquote{оазою спокою} у зруйнованому на 85
відсотків Маріуполі (вцілив лише один будинок з шести!). Так, сусіда, що
пригостив нас рибою, поранило в ногу, але ж не в нашому дворі - він ходив на
Кіровський перевірити квартиру тещі. Так, у \enquote{наш} павільйон, який ми не давали
розібрати на дрова, прилетіли міна, від якої повилітали шибки на перших двох
поверхах, але ж люди не загинули.

Малюнок: навколо нашого двору.

%\ii{19_03_2023.fb.kipcharskij_viktor.mariupol.1.p_slyamova.cmt}
