% vim: keymap=russian-jcukenwin
%%beginhead 
 
%%file 25_10_2019.news.pravda_com_ua.SPARTA
%%parent 25_10_2019
 
%%endhead 
\subsection{Республіка Щастя. Як в комуні під Харковом третій десяток років будують утопію}
\label{sec:25_10_2019.news.pravda_com_ua.SPARTA}
\url{https://www.pravda.com.ua/articles/2019/10/25/7230003/}
  
\vspace{0.5cm}
 {\ifDEBUG\small\LaTeX~section: \verb|25_10_2019.news.pravda_com_ua.SPARTA| project: \verb|letopis| rootid: \verb|p_saintrussia| \fi}
\vspace{0.5cm}

{\em
В селе Караван, в получасе езды от Харькова, находится коммуна "С.П.А.Р.Т.А."
или "Сельскохозяйственная Поэтизированная Ассоциация Развития Трудовой
Активности". 

Участники этого объединения напоминают одержимых персонажей фильмов Эмира
Кустурицы: они доят коров и заседают в "парламенте", штрафуют за курение и
употребление алкоголя, бегают 100 километров, в обязательном порядке сочиняют
стихи, возводят в культ Ленина и высчитывают уровень счастья по специальной
формуле. 

Несмотря на принудительное лечение, уголовные преследования и отсутствие
элементарных удобств, "С.П.А.Р.Т.А." существует уже 23 года, а само движение и
вовсе берет начало в Советском Союзе.

УП побывала в "С.П.А.Р.Т.Е." и поговорила с ее жителями об устоях и идеологии
их организации, коммунизме и демократии, Горбачеве и Ельцине, войне и мире, ЗОЖ
и вредных привычках, многолетнем давлении властей и поиске рецепта счастья.
}

\subsubsection{Дорога к С.П.А.Р.Т.Е.}

Учхоз с древнегреческим названием "С.П.А.Р.Т.А." находится в селе Караван в 25
километрах от Харькова. Автобус доезжает до небольшого городка Люботин. До
самой деревни дважды в день в не самое удобное время курсирует маршрутка.

"C.П.А.Р.Т.А.нцы" подхватывают нас на полпути на стареньком Volkswagen Golf. На
приборной панели машины корректором выведено слово "Тезис", ниже висит длинный
список под названием "Дефектная ведомость". За рулем сидит женщина средних лет
в соломенной шляпе, рядом на пассажирском сидении --- знакомая нам Ольга Широкая,
которая предварительно одобрила наш приезд.

– Томка я, --- радушно протягивает руку женщина в соломенной шляпе и тут же с
энтузиазмом начинает расспрашивать нас о том о сем.
  
\emph{"Томка" --- Тамара Костюк занимает в "С.П.А.Р.Т.Е." должность старшего гвардейца
Все фото Константина Буновского}

За окном простирается типичный деревенский пейзаж --- неприметные домики с
поросшими бурьяном участками и покосившимися заборами. Через несколько минут
тряски на грунтовой дороге доезжаем до пункта назначения.

– Как говорится, милости просим, чем богаты, тем и рады! --- улыбается Тамара и
приглашает нас в беседку "Спартак".

\subsubsection{Психбольница, КГБ и ``Братство Кандидатов в Люди''}

Будущий основатель "С.П.А.Р.Т.Ы." Юрий Давыдов родился в Саранске в 1954-м. До
16 лет жил в городе Веневе под Тулой, окончил училище по специальности
тракторист-механизатор. В 1973-м был призван в армию, служил в Латвии и в
Литве.

{\em
Прежде, чем создать свое движение, будущий основатель "С.П.А.Р.Т.Ы." выучился на тракториста-механизатора
\/}

"Юра говорил: "Я жил в шести республиках, в 15 населенных пунктах", –
рассказывает Тамара. --- На месте не сидел, потому что папа был
политрепрессирован, удерживался в ГУЛАГе в Кемерово с 1942 по 1952 годы.
Реабилитировали только в 1956-м --- мать ездила в Кремль, добивалась. 

Но джугашвилинисты никуда не делись, давление на политзаключенных было, и
родители частенько перемещались. Когда в армию шел, спросил военного комиссара,
что он может сделать для Родины, а тот сказал ему: "Разработай Теорию Счастья,
чтобы воспитать нового советского человека". С того момента он начал по
кусочкам собирать ее".

В 1978-м после возвращения из армии Давыдов поступил учиться на
промышленно-гражданское строительство в Мордовский университет в Саранске
заочно. Параллельно менял работы и города --- работал обрубщиком в Саранске,
докером в Клайпеде, станкостроителем в Харькове. 

Помимо учебы и работы занимался спортом: боксировал в Саранске, прыгал с
парашютом в Харькове, бегал марафоны в Ростовской области. В 1980-м женился, но
брак продлился всего год.

"Юра придерживался теории, что дамочек нужно проверять на верность, –
утверждает Тамара. --- Она жила в Саранске, он расписался, уехал в Харьков и
проверял, что она будет без него делать. Из этого у него родились исследования
"История Одной Любви", "История Шаблонной Любви" и "История Потребительской
Любви". На уровне бытовой семьи мыслителю, у которого глобальные мечты и цели,
невозможно реализоваться".

В это время Давыдов также составил первую версию опросника "Знаете ли Вы
окружающий мир и себя? Можете ли Вы построить собственное Счастье?", в который
в итоге войдет 1500 вопросов. Сегодня этим вопросником в "С.П.А.Р.Т.Е."
тестируют новичков.

"Перворазнику" даем анкету --- "откровенник" из 50 вопросов, --- объясняет Тамара.
– Как только ты его заполнил --- условно за час --- тебе выдается "Методика
заполнения теста", чтобы подготовить к заполнению опросника.

Методика расписана на 28 страницах текста. Был период, когда предлагалось за
два дня ее скопировать от руки --- уровень самопожертвования проверялся. Затем
дается опросник из 1500 вопросов. Ты его переписываешь, пусть месяц на это
уйдет, и дальше заполняешь ответник --- тоже условно за месяц. Получается
трансформация мышления за 90 дней".

Одновременно с социологическими исследованиями Давыдов работал над различными
техническими инновациями. Тем не менее, его эксцентричные идеи пришлись не по
нраву советской власти и в 1984-м его отправили в психбольницу.

"Кандидат в Настоящие Великие Люди придумал проект "Лужок", чтобы улучшить
снабжение жителей Киевского района Харькова, и презентовал его в обкоме, –
вспоминает Тамара. 

В обкоме вызвали медбратьев и те нагрянули на квартиру к его отцу. Юра сел на
поезд и поехал в Мордовию. Во второй раз они приперлись на квартиру к Юре.
КГБшники просили его подписать бумажку, что он не будет заниматься
социологическими исследованиями. Юра не подписал и его на три месяца поместили
в харьковскую психушку".

Советская власть пыталась лечить основателя движения Юрия Давыдова в психбольнице

Выйдя из лечебницы, Давыдов продолжил исследовать человеческую природу и
окружающий мир. К концу 80-х он написал шесть томов "Теории Счастья" –
причудливый симбиоз философии, поэзии и математики. 

Одновременно с этим работал над своей физической формой --- преодолевал
100-километровые сверхмарафоны, бегал босиком по углям, практиковал раджа- и
бхакти-йогу, занимался каратэ. Искал единомышленников и в 1987-м основал
предтечу "С.П.А.Р.Т.Ы." --- организацию "Б.К.Л." --- "Братство Кандидатов в Люди".

\subsubsection{Старший гвардеец Тамара}

Нынешний руководитель, или, как говорят в коммуне, старший гвардеец
"С.П.А.Р.Т.Ы." Тамара Костюк, родилась в 1970-м в Люботине "в роддоме, где
раньше было поместье люботинской дамочки, которая вышла замуж за философа
Бердяева". В школе она была комсоргом, параллельно с этим пела и пыталась
писать стихи.

В 1987-м Тамара в ХГУ им. Горького (сейчас --- Харьковский Национальный
Университет им. Каразина) познакомилась со сторонницами идей Давыдова, вступила
в "Б.К.Л." и начала участвовать в разработке Теории Счастья.

"Наш инструктор все время стремился к совершенству, занимался всю жизнь
исследованиями --- как устроены коллективы и мышление окружающих, что такое
дружба, любовь, верность, ревность, производительность труда, --- рассказывает
Тамара.

Нам тоже это было интересно, а рыбак рыбака видит издалека. Когда я после КПСС
и ВЛКСМ начала изучать феличологию (местная наука о счастье --- УП), это было как
переход в новый мир".

Вскоре после вступления Тамары в организацию у "Б.К.Л." появилась первая
штаб-квартира на улице Чигрина в Харькове. Давыдов и другие участники движения
стали жить в ней коммуной. На следующий год, сразу же после своего 18-го дня
рождения, к ним присоединилась и Тамара.

"Дамочки и ребята решили съехаться и жить в одной квартире, --- вспоминает
Тамара. --- Их было четверо --- это не было деление по парам, мы просто так
коллектив строили. 

Мама была в большом огорчении, падала на колени, спрашивала, на кого я ее
оставляю. Когда она закрыла хату, я пролезла в форточку, забрала свой паспорт и
побежала за Юрой. Я была экстремальной радикалкой или радикальной
экстремалкой".

ЗОЖ --- один из главных принципов организации Давыдова

В 1988-м в Одессе Тамара пробежала свои первые 100 километров, затем начала
вести свою группу. К тому моменту "Б.К.Л." насчитывало уже 40 участников. После
разочарования руководства коммуны в некоторых из участников организацию решили
переименовать в "Б.К.Н.Л." --- Братство Кандидатов в Настоящие Люди". На
следующий год коммуна перебралась в квартиру на Пушкинском въезде в центре
Харькова.

"Мы жили вшестером в одной комнате, --- утверждает Тамара. --- Хозяйке квартиры не
нравилось, что мы туда-сюда ходим и однажды она вызвала милицию. 

Юра был дома, его менты схватили, бросили в тюрьму на 15 суток, вменяли
хулиганство. Мы устроили на "политпяточке" (площади --- УП) пикет, собирали
подписи для его освобождения. Тогда Перестройка была на улицах, и решение суда
отменили через девять суток".

В 1989-м году во время традиционной "сотки" в Одессе у "Б.К.Н.Л." появилась
собственная символика --- четырехцветное знамя.

Среди множества аляповатых элементов на флаге в глаза бросается золотой силуэт
Ленина. К вождю мирового пролетариата в коммуне относятся с пиететом. Бюст
Владимира Ильича также покрывается пылью в одной из комнат, прозванной
Ленинской.

"Ленин на знамени, потому что всем знаком --- и правым, и левым, и красным, и
коричневым, и жовто-блакитным, --- пространно объясняет Тамара. --- Это один из
примеров того, что когда индивид много делает в масштабах нации и
межнациональной интеграции, след становится заметным, и когда он переходит в
мир иной, этот след исследуется. 

Мы поддерживаем две стороны: приходят демократы --- поддерживаем коммунистов.
Приходят коммунисты --- поддерживаем демократов. Мы считаем, что нужно помочь
остаться в живых и тем, и другим. Иначе одни других заколошматят. Наша задача –
сбалансированное счастливое общество".

{\em
В Ленинской комнате в "С.П.А.Р.Т.Е." раньше проживал "топ-менеджмент"
\/}

На знамени "С.П.А.Р.Т.Ы." изображены вождь мирового пролетариата и символика
эсперанто

В 1990-м коммуна в очередной раз переехала --- из центра Харькова в спальный
район ХТЗ. На следующий год развалился СССР, но на участниках "Б.К.Н.Л."
падение империи сильно не отразилось.

"СССР --- это и Советский Союз, и совдепия. Есть борцы, а есть палачи, стукачи и
ГУЛАГ, --- подчеркивает Тамара. --- С одной стороны Союз однозначно надо было
обновлять, Михаил Сергеевич (Горбачев) был за обновление, и мы тоже. Но то, что
развал так подействует --- до такой степени пойдет обнищание народа и падение
экономики... ".

Одновременно с распадом СССР "С.П.А.Р.Т.А.нцы" изобрели собственную иерархию
управления. С 1991-го в организации действует "двухпартийный парламент", в
состав которого входят "партии" дженералов и универсалов. 

Универсалами сейчас руководит Ольга Широкая, еще одна участница движения Юлия
Приведенная, с которой Тамара знакома уже 28 лет, курирует дженералов из
Москвы. Сама Тамара при этом координирует обоих.

"Дженералы и универсалы соревнуются на семи уровнях управления, --- объясняет
руководитель "С.П.А.Р.Т.Ы." Тамара. --- Они устроены как конструктивная
организация --- суть в том, что одна партия помогает другой выкарабкиваться. Мы
называем это "К.О.Р.О.Л". --- "Коллективный Объективный Разум Объединения
Любского (от слова Любовь)". 

Поскольку мы сделали парламент, бросились все в процесс Amikidji (в переводе с
эсперанто --- "подружиться"). У тебя команды не будет, если ты ее не соберешь.
Выбрали самый простой способ --- пошли по селу и говорили: "Люди, люди, хто зі
мною?" Amway отдыхает по сравнению с нами".

В начале 90-х Давыдов запустил свое движение в ряде городов России --- в Москве,
в Калуге и в Перми. Вскоре участники квазипарламента "Б.К.Н.Л." внезапно
оказались в эпицентре противостояния у парламента настоящего.

\subsubsection{П.О.Р.Т.О.С., Горбачев и Ельцин}

Пока Тамара много и долго говорит, периодически ныряя в лирические отступления,
ее коллега Ольга Широкая молча макает кисточку в банку с краской и скрупулезно
выводит на плакате буквы для очередного мотивационного плаката. 

Лозунги в духе марксизма-ленинизма --- неотъемлемая составляющая идеологии
"С.П.А.Р.Т.Ы.", равно как и сокращения в лучших-худших традициях СССР.

\emph{Отдающими советской пропагандой лозунгами в коммуне увешано все, вплоть до туалета}

Ольга представляется "временно исполняющей обязанности гвардейца", проще
говоря, она --- заместитель Тамары.

"Я в "Б.К.Н.Л." с 1991 года, --- вспоминает Ольга. --- Собиралась поступать в УЗПИ
(сейчас --- Украинская инженерно-педагогическая академия) в Харькове, шла на
консультацию по математике. 

Возле входа в метро "Университет" ко мне подошла девушка: "Мы проводим
небольшой социологический опрос, есть несколько минут?" Я глянула на часы –
16:40. Ехать до института 10 минут, значит, несколько минут есть. Она спросила:
"Что такое счастье?" Так мы проговорили четыре часа".

С 1993-го "Б.К.Н.Л." стало называться "Б.К.Н.Л. --- П.О.Р.Т.О.С."
("Поэтизированное Объединение Разработки Теории Общенародного Счастья") имени
офицера группы "Альфа" Геннадия Сергеева. 

Название организации видоизменилось после кровопролития возле Белого дома в
Москве 3-4 октября. В разгар политического кризиса в РФ президент Борис Ельцин
отдал приказ штурмовать здание правительства и стрелять по своим
оппонентам-коммунистам из танков. Юрий Давыдов и другие участники "Б.К.Н.Л."
вышли защищать Белый дом.

"Альфу" Ельцин послал на расстрел, чтобы они постреляли всех защитников, –
рассказывает Ольга. --- Когда "альфовцы" зашли в Белый дом, Юра начал спрашивать,
что такое счастье у одного из командиров Геннадия Сергеева. Командир принял
решение не подчиняться приказу и выводить из здания беззащитных людей. 

И они в своих бронированных автобусах и танках начали вывозить защитников.
Благодаря этому наши вышли живыми. Сергеева убил снайпер, и с тех пор мы носим
его имя. А звание Героя России ему дали через бог знает сколько лет, когда
ФСБшники увидели, что общественность все знает и замять (дело) не удалось".

Через три года участники "Б.К.Н.Л. --- П.О.Р.Т.О.С." вновь принимали участие в
политических процессах в России --- поддерживали в предвыборной кампании Михаила
Горбачева.

"Мы не хотели лишаться своей страны, но нам не хватило опыта, --- ностальгирует
по СССР Тамара. --- На метро "Пушкинской" в Москве стояли с пикетиком, непрерывно
собирали подписи в поддержку Горбачева, разговаривали с населением, музыку 60-х
крутили. 

Был момент, когда оргкомитет собирался в гостинице "Россия", 27 января 1996-го
я там выступила и сказала, что у нас есть девять козочек и три свинки и мы
поднимем сельское хозяйство. В тот год несколько раз встречались с Михаилом
Сергеевичем и Раисой Максимовной (Горбачевыми)".

На выборах 1996 года в РФ снова победил Борис Ельцин, а Горбачев получил всего
0,5\%. Тем временем, участники организации обустроили свое хозяйство в деревне
Караван за Харьковом. Так появился учхоз "С.П.А.Р.Т.А.".

Сегодня территория коммуны включает пять домов и занимает площадь в 1,4
гектара. В разное время в "С.П.А.Р.Т.Е." держали лошадей, свиней, коз, овец,
уток, разводили пчел и выращивали грибы. На данный момент остались только 9
коров --- участники коммуны продают надоенное молоко и иногда делают из него
адыгейский сыр.

{\em
Из всех животных в "С.П.А.Р.Т.Е." осталось только девять коров, да и тех собираются продавать
\/}

О бывшем президенте РФ и других исторических персонажах напоминает нейминг
всего имущества --- как движимого, так и недвижимого. Именем Ельцина, Сталина и
Иуды названы туалеты, в то время как Толстого, Пушкина, Есенина, Маяковского
увековечили в именах сараев, в честь Ярослава Мудрого и Наполеона назвали
погреба, а имена Ликурга и Кобзаря присвоили коровникам.

"В честь людей, на которых хотелось бы быть похожими в первую очередь, названы
конструктивные постройки, --- аргументирует Ольга.

Туалет называется Сталин. Сколько у Сталина, 67 млн погибших? Есть еще туалет
Ельцин. По неофициальным данным, за два дня возле Белого дома 5 тысяч человек
было убито" (по разным оценкам, в противостоянии возле Белого дома погибло от
124 до 350 человек --- УП).

\emph{Все помещения в коммуне названы именами известных исторических персонажей}

Лингвистические эксперименты не заканчиваются на туалетах и сараях. Помимо
Ленина на флаге "С.П.А.Р.Т.Ы." можно обнаружить пятиконечную звезду --- символ
языка эсперанто, который также входит в Теорию Счастья, а также надпись La
Paco, что в переводе с эсперанто означает "мир". Значок с такой же символикой
носит и Тамара.

В коммуне считают, что глобальная "эсперантизация" поможет улучшить
коммуникацию между народами, а эсперанто как язык международного общения "самый
простой и изящный".

\subsubsection{Высоцкий, Макаревич и ``Рідна мати моя''}

Внезапно из дома под названием Терем раздается задорный звук баяна. Разговор к
тому моменту продолжается уже три часа, поэтому решаем сделать музыкальную
паузу.

Главный зал Терема напоминает кабинет безумного ученого. Стены и потолок вдоль
и поперек увешаны бумагами с многочисленными формулами, графиками, диаграммами,
плакатными слоганами и стихотворениями. 

У входа в глаза бросается карта трезвых и пьяных стран по состоянию на 2006
год. Напротив двери стоит телевизор и большой стеллаж с книгами и
видеокассетами. На шкафу --- знамя "С.П.А.Р.Т.Ы.", на соседней стене –
перевернутый флаг Украины. С потолка свисают желтые липкие ленты с трупиками
мух. Воздух затхлый и отдает старостью.

Один из аксакалов коммуны Евгений Привалов играет на баяне "Рідна мати моя".
Остальные участники сидят за столом с песенниками под кодовым названием
"Э.Х.О". ("Эволюционное Хоровое Обеспечение"), и поют в унисон.

{\em
В перерывах между работой участники коммуны поют под баян
\/}

В песеннике "С.П.А.Р.Т.А.нцев" можно найти "Песню о Ленине" и другие
коммунистические шлягеры

Репертуар песенника своеобразен и разнообразен: от "Гимна СССР" и
"Интернационала" до песен Высоцкого и Макаревича.

"С.П.А.Р.Т.А.нцы" заканчивают импровизированный концерт своей коронной "Мы
желаем счастья вам". Музыкальная пауза переходит в обед. Кухарка Люба подает к
столу окрошку, пельмени, котлеты, оладьи, фрукты и компот.

– Мы еще называемся "П.О.Р.Т.О.С.", потому что любим покушать. Люба прошла
тест, чтобы нас не отравить. Это, конечно, шутка, но в каждой шутке, как
известно… --- смеется Тамара и предлагает нам чай каркаде.

Ланч весьма уместен перед началом разговора о деле всей их жизни --- вычислении
уровня счастья.

\subsubsection{Формула счастья}

Перед приездом в "С.П.А.Р.Т.У." Ольга Широкая попросила нас заполнить
"откровенник" из 50 вопросов, который в коммуне обычно выдают "перворазникам".
"Это из нашего начального теста по феличологии. Паспортные данные можно не
заполнять", --- написала Ольга.

"Откровенник", помимо прочего, содержит цепочку вопросов "Вы пьете?" –
"Почему?" --- "Вы намерены пить в дальнейшем?" --- "Почему?" --- "Сколько держится в
крови алкоголь?" --- "Откуда вы это узнали?", а также спрашивает "Чего Вам не
хватает для полного Счастья?". 

О счастье в "С.П.А.Р.Т.Е." готовы говорить бесконечно.

"Феличология --- это 77 наук, --- заверяет Ольга. --- Чтобы это все понять, нужна
недюжинная оперативка души. Учитывая, что у некоторых пропускная способность у
души маленькая, то у них просто процессор подвисает".

{\em
Уровень счастья в "С.П.А.Р.Т.Е." вычисляют по причудливым формулам
\/}

F(EST)O --- суммарный показатель счастья человека —"С.П.А.Р.Т.А.нцы" вычисляют по
специальной формуле.

"Лемма счастья Пифагора-Давыдова называется F(EST)O, в переводе с эсперанто  –
"праздник", --- просвещает нас Тамара. --- Счастье состоит из произведения Energio
– Энергии, Sociala --- Интеллекта, Tempo --- Времени и Organizeco --- количества
друзей. EST --- у всех условно одинаково, а О --- разное. Энергия измеряется в
Джоулях или Килокалориях. Интеллект вычисляем в ДАВах --- в честь нашего
основателя Давыдова. 

ДАВ расшифровывается как Дезинтеграл Активных Выживаний. Можно вычислить его
через электроэнцефалограф, а можно --- через написание поэзии. 

Считаем количество зарифмованных строк и строф. Рифмование обязательно для
входа в парламент. Норма --- условно 10 куплетов в неделю, если норму не
выполняешь, твой рейтинг понижается".

Уровень счастья и другие показатели в "С.П.А.Р.Т.Е." считают раз в три месяца
или в полгода. Свои подсчеты в коммуне называют "Н.О.В.А.Я. М.О.С.К.В.А." –
"Нахождение и Обучение Великих Альтруистов Явных --- Матрица Общего Соревнования
Коллективного Выполнения Amikidji". Людей с низким уровнем просят покинуть
организацию.

"Когда попытки поднять уровень счастья превращаются в пытки, не пытаемся, –
ухмыляется Тамара. --- Если индивид свыше 25 лет с нами и настойчиво нарушает
закон, напоминаю раз в три месяца или раз в полгода. Если "заблудился",
попытаюсь чаще. Но если сильно упирается --- отпускаю: куда праотцы позовут, туда
и отправляйся. Мы все Дети Галактики, куда пошлют --- туда и пойдешь".

\subsubsection{С.П.А.Р.Т.А. вне закона}

В конце 90-х --- начале 2000-х уровень счастья участников "С.П.А.Р.Т.Ы." вряд ли
был удовлетворительным --- у организации начались серьезные проблемы с законом. К
тому времени у "Б.К.Н.Л. --- П.О.Р.Т.О.С." под Люберцами в Московской области
появился "Городок Солнце" на 200 человек. 

Организация запустила бизнес --- 40 грузовиков занималось доставкой продуктов
питания, а в 2000-м выкупила территорию бывшего военного завода "Салют" в
Подмосковье.

Российским властям такие маневры не понравились. Генпрокуратура РФ обвинила 14
участников коммуны --- 10 граждан России и 4 украинцев --- в создании незаконного
вооруженного формирования. 5 из них, в том числе Юрий Давыдов, оказались за
решеткой, остальных объявили в межгосударственный розыск.

"Мы с Тамарой находились в межгосударственном розыске с 2000 года, --- утверждает
Ольга. --- Юру 7 декабря 2000-го посадили в тюрьму. У него было пять охотничьих
ружей. Еще в 90-е у некоторых наших были газовые пистолеты, в то время приняли
решение, что инструкторы их будут носить.

Когда Юру пытали, ему говорили: "Мы тебя посадим, посмотришь, теперь все твои
разбегутся". Он расписал это во втором томе "Теории Счастья" "Бутырская
баллада".

\subsubsection{История "Б.К.Н.Л. --- П.О.Р.Т.О.С." продолжается уже больше 30 лет}

После объявления в розыск Тамара написала письмо Путину, в котором разъяснила
суть деятельности организации и просила отнестись к ним справедливо. Не
помогло. Юрий Давыдов вышел на свободу только в мае 2006-го, а спустя три года
его не стало. 

Ольгу и Тамару сняли с розыска, закрыв дело за отсутствием состава
преступления, лишь в 2011-м. За это время объединение снова сменило название и
превратилось в "Ф.А.К.Э.Л. --- П.О.Р.Т.О.С." --- "Формирование Альтруистов
Кандидатов в Эволюционирующие Люди --- Поэтизированного Объединения Разработки
Теории Общенародного Счастья".

"В 2000-м у нас было 100 участников, --- вспоминает Тамара. --- Когда в Москве
понаехал РУБОП и СОБР, пришлось разбираться с решетками, с мусорской системой,
осталась четверть --- 25. К 2007-му мы по-новому выстроили экономику и снова
выросли: в Подмосковье было 42 человека, в "Подхарьковье" --- 37, в
"С.П.А.Р.Т.Е." --- 12. И еще 30 сменных. Итого 121 человек. 

Сегодня в Караване осталось пять матерых участников, которые тиражируют идеи
движения. В РФ наша организация сейчас находится в Одинцово под Москвой. Ждем
оттепель в России. Обычно раз в 20 лет бывает. Наше время пришло --- есть у нас
такой слоган и песня "Русь вперед". Каждому свое время, но наше время пришло".

\subsubsection{От оператора ПК к доярке}

Спрятав копну кислотных волос под косынкой, Алина идет доить коров. Путь в
коровник Ликург лежит через море грязи. "Слава Героям Труда!" --- гласит покрытая
слоем многолетнего налета надпись на входе.

В помещении царит зловоние. При этом даже коровы в "С.П.А.Р.Т.Е." названы
поэтично --- Ариадна, Красота, Поэма, Пальмира, Аура, Теза, Мирина и другими
изысканными именами.

– Маму Мира звали, бабушку --- Мойра, а она --- Мирина, --- объясняет Алина,
присоединяя к вымени коровы доильный аппарат.

\emph{Алина пришла в "С.П.А.Р.Т.У." как "оператор ПК", но стала дояркой}

Тем временем, свободолюбивая Аура пытается выбраться из стойла.

– Сука, Аура, на место! --- негодует девушка и вытирает пот со лба.

Через полчаса, после завершения процесса, Алина закуривает сигарету.

– А ты разделяешь их идеи? --- интересуемся у нее.

– Я не знаю, это их идеи, --- пожимает плечами Алина. --- Пить, курить нельзя, но я
курю, на праздники могу выпить, для них это неправильно. У них штрафы: за
курение --- 333 грн, за выпивку --- 999 грн. Требовали, чтобы я курить бросила, но
я не хочу. Если я курю, то не при них. 

Вообще курят все здесь, но чтоб не видели. И я не считаю неправильным на
праздник выпить шампанского, любой человек, даже президент делает это. У них
нельзя жениться, они незамужние и неженатые. После вступления в организацию
детей нет. Им нельзя иметь интимные отношения. Нам тоже, если мы тут работаем и
живем".

Алина тушит сигарету.

\emph{Вместе с двухлетней дочерью Алина снимает часть дома по соседству от "С.П.А.Р.Т.Ы."}

– Как давно ты в "С.П.А.Р.Т.Е." и как здесь оказалась? --- спрашиваем Алину.

– Я сама с Лозовой, с ними знакома четыре года. Пришла сюда по объявлению в
интернете. Было красиво написано: "Требуется оператор ПК". А какой тут оператор
ПК? Стала научным технологом, потом пастухом, последний год работаю дояркой,
потому что не было людей --- мало кто согласен с тем, что нельзя пить-курить.

– Ты ведь не за еду здесь работаешь?

– Зарплату платят, не за идею здесь. Хотелось бы больше, конечно. Но я понимаю,
шо це ферма. Самое хуже --- 150 грн в день. Бывало и 450 грн. Но если посчитать
по часам --- начинаешь с 4, заканчиваешь в 9-10, а бывает и позже. Когда
переработка молока, целую ночь сидишь. Работу хотела поменять, но жалко, если
распадется все. Они собираются продавать коров. Невыгодно, убыточные.

Последний год Алина с двухлетней дочкой снимают часть дома через несколько
дворов от "С.П.А.Р.Т.Ы." 

"Ей 23, Катька --- это у нее второй ребенок от второго мужа, --- характеризует
девушку Тамара. --- Одну девочку или мальчика у нее забрали уже. А муж такой был
– объелся груш. Делал у нас грядки, как-то она его намотала и он с переляку
переспал с ней и смотался, ребенка бросил. 

Научили ее коров доить. Она то в абрикосовый покрасится, то в еще какой-то.
Дочке два года, а мать такой пример подает. Чем она хороша? Мы абы кого не
берем. Если не умеет на бумаге писать --- не берем. А эта умеет. Но просто нужно
бегать за ней, чтобы заставить".

– Тут еще будет "сотка", интересное зрелище, --- улыбается Алина. --- Я не бегаю,
смысла не пойму. Захотел --- бегаешь каждый день по чуть-чуть. Но за раз
пробежать 100 километров --- смысл? Пробегают, то в обморок падают, то еще
что-то. Хоть бы призы были какие-то... А чисто за идею, за грамотку
нарисованную и медальку --- типа тех шоколадных, что в детстве дарили…

Алина приглашает нас на экскурсию в Крупскую --- женское общежитие. Интерьер
напоминает бедные постперестроечные квартиры 90-х в какой-нибудь глухой
провинции. Здесь как будто бы вечный Новый год: на выцветших стенах растянулись
старые гирлянды, а на полке пылится советский Дед Мороз. Под потолком висит
икона. Готовые декорации для неснятого фильма Балабанова.

\emph{В женском общежитии Крупской, как и всей коммуне, время как будто бы остановилось}

– Когда я рано встаю, здесь остаюсь, --- объясняет Алина. --- Лежишь ночью, не
спится --- на потолок смотришь: "О Боже!" А за закрытой дверью комната Ленинской
называется. 

Когда Томкин батя покойный закладывал, его закрывали в той комнате с Лениным,
отдавали поварихе ключ, чтоб его кормила. Он дядя такой был, шибанок. Каждой
дамочке, которая здесь находилась, предлагал интим за деньги. Короче,
интересный был мужчина. В прошлом году зимой помер --- замерз на остановке. 

\subsubsection{Молочник, тракторист, пастух и "девятка" Фидель}

Старожил "С.П.А.Р.Т.Ы." Евгений Привалов грузит банки с молоком в багажник
потрепанной временем "девятки", прозванной Фиделем в честь знаменитого
кубинского революционера. На стекле машины --- трещины, из приборной панели
торчат провода, салон набит всяким хламом.

– Молоко везу в Люботин на "большой рынок" --- по дачным домам развозить, –
объясняет Привалов с характерным русским акцентом.

Евгений родился в Димитровграде Ульяновской области РФ, долгое время жил в
Калуге. Один из близких товарищей Давыдова, в организации он состоит с начала
90-х. 

Вместе с другими участниками движения Привалов защищал Белый дом в Москве,
проходил по делу о создании незаконного вооруженного формирования и отбывал
срок в тюрьме.

В "С.П.А.Р.Т.Е." Евгений управляет хозяйством и играет на баяне. "И тракторист,
и дежурный, бывает и пастух", --- описывает функционал Привалова Тамара.

\emph{Евгений Привалов защищал вместе с основателем коммуны Белый дом и сидел в
тюрьме}

В 90-х и 2000-х Евгений занимался транспортными перевозками --- возил по России и
Украине корма. В Харьков Привалов передислоцировался 10 лет назад из-за
стечения обстоятельств.

– Я заехал сюда на неделю подменить Андрея Петрова, который здесь управлял
хозяйством, --- рассказывает Евгений. --- Неделя еще не закончилась, а 10 лет
прошло. Андрей спасал рабочего, который в колодец провалился, и сам погиб. Тот
с бодуна был --- руки тряслись, залез туда на метр, чтобы кран починить, не
удержался и упал. Неблагополучный колодец, там скопились какие-то яды --- дыхнул
один раз и готов.

– В какой вы партии? --- интересуемся у Привалова.

– Переходил несколько раз из партии в партию, уже и забыл. Скорей всего я в
партии дженералов стажер. Настроение зависит от количества сделанных добрых
дел. Успел сделать --- ты в тонусе, не успел --- настроение стало хуже. А звание
только отражает твои наработки. Заработал --- получил.

– Изначально чем вас привлекла организация Давыдова?

– Я товарища в третьем классе попросил: "Дай попробовать пиво". Попробовал –
дерьмо. Хуже, чем навозная жижа. С тех пор я не пью. Так и здесь --- если счастье
вещь нужная, то почему бы его не приобрести? 

– Мне кажется, у вашего движения много общего с хиппи --- они тоже
пропагандировали мир, счастье, любовь...

– Хиппи считать не умели. Счастье, которое не подсчитано, высоким не может
быть. Это как прибыли неподсчитанные. 

И потом, хиппи что, не курят и не пьют? И курят, и пьют. Живут ради
удовольствия. А как отказаться от такого удовольствия, как балдеж? Похмелье,
рак --- это потом. Поэтому нет у них ни мира, ни любви.

Пока мы разговариваем с Евгением, рабочие ремонтируют крышу коровника Кобзарь,
а Алина купает дочку в тазике.

\emph{Большинство зданий "С.П.А.Р.Т.Ы." нуждаются в капитальном ремонте}

– Следите ли вы за новостями? --- продолжаем расспрашивать Привалова.

– За политической ситуацией слежу, --- утвердительно кивает он. --- Сейчас, слава
Богу, новый президент. Надеемся, что воровство прекратится, сколько ж можно
воровать! Крадут вон через забор. Только отвернулся, смотришь --- аккумулятора
нет. Вроде и народ тут был, и сторож был --- никто ничего не видит.

– Вы считаете себя коммунистом?

– Мы и так при коммунизме живем давно. От каждого по способностям, каждому по
потребностям. Мне сколько надо, у меня все есть. Что такое коммунист? Если
дадите точное определение, тогда надо подумать. Иисус был первым коммунистом.
Можно назвать верующих, которые ходят в церковь, коммунистами?

– В какой-то степени да.

– Если в какой-то степени --- то и я в такой степени коммунист. Правда, им [в
церкви] до коммунизма далеко, они только амбары свои набивают и пузо. Почему их
попами называют? Потому что попы такие здоровые. А есть священники --- но их
мало. Большинство --- попы.

– Что думаете о конфликте России и Украины?

– Нет никакого конфликта между Россией и Украиной. Деньги отмывают умники.

– А гражданство у вас русское или украинское?

– У меня русский паспорт. Я написал заявление в Администрацию президента на
получение украинского гражданства, я 10 лет здесь как украинец в сердце, да и
раньше, как говорится… Мне уже можно в выборах участвовать. Проголосовал бы,
например, за партию "Слуга народа", так я не могу, у меня нет паспорта. А я бы
хотел поддержать умных ребят. 

– У вас есть семья, дети?

– Все, что есть у нас --- это и есть семья. Чужих детей не бывает. Наш главный
постулат --- это "П.О.Б.Е.Д.А." --- "Повсеместное Обеспечение Бесплатной Едой Детей
Альтруистов". Если обеспечить детей в школах бесплатным питанием, независимо от
того, родители зарабатывают или пьянствуют, у них не будет белкового голодания,
и они начнут соображать головой. 

Еще обязательно должно быть трудовое воспитание, чтобы могли заработать сами.
Не знаниям нужно учить, а навыкам --- чтобы умел, например, хоть дрова нарубить
или костер поджечь. В школах навыков практических не дают, там дают знания.
Пока новый президент, надо перестроить срочно образование. Хай Зеленский это
почитает.

В отличие от Евгения, который верит в конспирологию и винит в развязывании
войны на Донбассе украинские власти, Ольга прямо называет виновником Россию. 

Несмотря на то, что "С.П.А.Р.Т.А." позиционирует себя как пацифистская
организация, в условиях российской агрессии она готова поступиться своими
принципами --- по крайней мере, до тех пор, пока война не закончится.

"Я-то могу сказать, что я пацифист, но в данной ситуации я понимаю, что это
[война России и Украины] --- дело рук ФСБшников, --- объясняет Ольга. --- Я против
любого разделения территорий. И естественно, я против того, что отшманывают
Крым, пытаются хитрыми манипуляциями оттяпать часть Донбасса и т.д. К России я
всегда хорошо отношусь, к нерусям --- отвратительно. Русский --- значит "светлый",
а неруси --- это скоты. 

Лично я против развала Союза, потому что они по сути что сделали --- не стали
реформировать, а просто обворовали всех. Юра был в большей степени за то, чтобы
объединить славянские государства --- Украину, Россию и Белоруссию".

\subsubsection{Эсперанто говорили...}

Любые обвинения по поводу построения сектантской организации в "С.П.А.Р.Т.Е."
категорически отрицают. Старший гвардеец коммуны сравнивает деятельность учхоза
с инновациями Ford и Apple.

"Люди, которые говорят, что мы секта --- малограмотные дураки, --- заверяет Тамара.
– Это признак того, что они несовершенны и ищут причину вне себя. Я отношусь к
ним с сочувствием.

Учитывая, что единственной статьей доходов "С.П.А.Р.Т.Ы." сейчас является
продажа молока, возникает логичный вопрос, каким образом организации удается
оставаться на плаву. 

В коммуне объясняют, что помимо "С.П.А.Р.Т.Ы." участники задействованы "в
нескольких десятках проектов" в рамках "Ф.А.К.Э.Л. --- П.О.Р.Т.О.С." --- в
частности, поставляют продукты питания фермерам и конным клубам. Кроме того,
занимаются различными индивидуальными активностями на стороне.

– Мы 20 лет занимаемся доставкой продуктов питания и кормов, обеспечиваем
радиус 50-100 километров вокруг Харькова и по Харькову, --- рассказывает Ольга
Широкая. --- Нашу организацию мы между собой называем "Честность". У нас
несколько тысяч покупателей. Еще я работаю с трезвенниками как областной
координатор, в школах лекции по Теории Счастья и трезвости проводила в этом
году.

Тамара Костюк, между делом упомянувшая о своих путешествиях в Лондон и
Копенгаген, говорит, что работает "тайм-менеджером, мотиватором и коучем". При
этом в своих методиках также использует Теорию Счастья. Своими клиентами Костюк
называет айтишников, менеджеров по продажам и других.

– У меня всегда нестабильный доход --- много и очень много, но всегда много,
потому что я все время работаю, --- улыбается Тамара.

В этот момент из Терема выходит кухарка Люба с пакетами.

– Ну теперь вже все, не знаю, коли ми побачимся, завтра мені на работу, –
прощается со всеми Люба.



В лучшие времена организация Давыдова насчитывала сотню человек. Сегодня в коммуне остались единицы

– Спасибо, Люба! --- благодарит ее Тамара. --- Я тебе стишок написала неделю назад, пока ты будешь идти, успею прочитать. "Жили-были, ели, пили"…

– …эсперанто говорили… --- предлагает свой вариант Ольга.

Тамара на секунду запинается, но, собравшись с мыслями, начинает декламировать снова.

– Жили-были, ели, пили.

Пели, плакали, любили,

Счастье строили, творили,

Для друзей друзьями были.

По поэме написали,

Для кого легендой стали,

На вопросы отвечали,

По уму людей встречали.

Вы себе теперь ответьте:

Для чего вам жить на свете?

Польза в чем и в чем растрата,

За чей счет идет оплата?

Есть ли будущего шанс,

Кто ответит, кто за вас?

– Молодец, Томочка, мы будем за тебе голосовать в президенты! --- смеется Люба. –
Ты у нас будешь президентом, я тобі говорю! Ты всіх алкашів постреляешь! Их
надо стрелять!

– Может, одного мафиозу надо слегка припугнуть, а остальные перепугаются и
пойдут работать, --- предлагает компромисс Тамара, прищуривая глаза в лучах
заходящего солнца.

Перед тем, как уйти в закат, рискуем спросить ее о главном.

– И все же, как найти счастье?

– Первое --- узнать о нас, второе --- взять ручку в руки и начать учиться. Мы
сделали свою технологию, в которой чувствуем себя, как рыбы в воде. Мы за любые
технологии, которые помогают человечеству продуплицироваться (от лат. duplicate
– удвоение --- прим. УП). 

Некоторые приписывают Людовику XIV фразу "После меня хоть потоп". А я хочу,
чтобы после меня все продолжилось. Мы как 300 спартанцев --- будем бороться до
последнего.

Дмитрий Кузубов, фото --- Константин Буновский, для УП
