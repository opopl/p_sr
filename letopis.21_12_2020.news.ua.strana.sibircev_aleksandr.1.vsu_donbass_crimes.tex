% vim: keymap=russian-jcukenwin
%%beginhead 
 
%%file 21_12_2020.news.ua.strana.sibircev_aleksandr.1.vsu_donbass_crimes
%%parent 21_12_2020
 
%%url https://strana.ua/articles/rassledovania/307418-kak-narushajut-ustav-soldaty-i-ofitsery-v-zone-oos.html
 
%%author Сибирцев, Александр
%%author_id sibircev_aleksandr
%%author_url 
 
%%tags vojna,donbass,ukraina
%%title Офицер закапывал солдат живьем в землю, солдаты бьют офицеров. За что судят бойцов ВСУ на Донбассе
 
%%endhead 
 
\subsection{Офицер закапывал солдат живьем в землю, солдаты бьют офицеров. За что судят бойцов ВСУ на Донбассе}
\label{sec:21_12_2020.news.ua.strana.sibircev_aleksandr.1.vsu_donbass_crimes}
\Purl{https://strana.ua/articles/rassledovania/307418-kak-narushajut-ustav-soldaty-i-ofitsery-v-zone-oos.html}
\ifcmt
	author_begin
   author_id sibircev_aleksandr
	author_end
\fi

\ifcmt
  pic https://strana.ua/img/article/3074/kak-narushajut-ustav-18_main.jpeg
  caption В зоне ООС среди военных часто происходят непозволительные вещи, фото: facebook.com/pressjfo.news 
  width 0.5
  fig_env wrapfigure
\fi

В сводках пресс-служб ВСУ и штаба ООС на Донбассе редко упоминается о
чрезвычайных происшествиях в действующей армии и в тылу - хулиганствах, пьяных
драках, суицидах, несчастных случаях и неуставных отношениях.

А когда они тем или иным образом появляются в СМИ, их нередко списывают на
пропаганду противника.

Например, на днях произошел характерный случай.

На сайте украинской делегации в Трехсторонней контактной группе появилось
сообщение, что сепаратисты передали тело бойца ВСУ,\Furl{https://strana.ua/news/307243-soldat-skonchalsja-ot-intoksikatsii-alkoholem-ili-narkotikami-na-nepodkontrolnom-donbasse.html} которое было обнаружено на
неподконтрольной территории Донбасса. Причем сначала на сайте ТКГ написали, что
солдат скончался от отравления алкоголем или наркотиками. Но позже это
сообщение было удалено с официальных ресурсов.

Позже спикер ТКГ Алексей Арестович принял вину на себя,\Furl{https://strana.ua/news/307414-arestovich-nazval-tekhnicheskoj-oshibkoj-publikatsiju-posta-o-peredache-tela-voennoho.html} заявив, что по ошибке
транслировал "версию оккупантов". 

"Данное сообщение стало технической ошибкой, ответственность за которую несу я
лично. Перепутал версию представителей российской стороны, которую обсуждали во
внутренней переписке, и сообщения для СМИ, и фактически обнародовал версию
оккупантов", - заявил Арестович. 

"Страна" проанализировала судебные решения, которые были вынесены в последние
годы по поводу преступлений в ВСУ в районе зоны боевых действий. 

\subsubsection{Офицер расстреливал подчиненных }

С момента начала боевых действий на востоке Украины настоящим бичом ВСУ стали
пьянки и употребление наркотиков на службе.

"Страна" насчитала в открытом реестре судебных решений несколько сотен
документов, которые касаются "пьяных" преступлений людей в погонах. Стоит
отметить, что уголовные производства открываются лишь по малой части пьяных
преступлений бойцов ВСУ - согласно криминальной статистике, в поле зрения
правосудия попадает всего 10-15\% правонарушений. Остальные остаются за кадром.

Кстати, многие судебные дела по пьяным преступлениям первых лет АТО длятся до
сих пор.

Например, только в 2020 году капитан одного из подразделений ВСУ получил
приговор.\Furl{https://reyestr.court.gov.ua/Review/88760925} Историю этого офицера, даже изложенную сухими юридическими терминами,
вполне можно назвать историей серийного садиста-убийцы. 

В материалах дела описано, как капитан ВСУ расправлялся с подчиненными. Одного
солдата офицер расстрелял, причем первоначально хотел расстрелять двоих -
просто один из них упал на землю, имитировав смерть. А потом скрылся.

Еще одного подчиненного офицер просто закопал живьем в землю. Через полчаса
солдату удалось откопаться и сбежать. Солдат, пойманных за употреблением
спиртного, офицер нещадно избивал и сделал некоторых инвалидами. Примечательно,
что офицер избивал солдат, будучи сам под градусом.

После перехода на контрактную систему набора в боевые подразделения ВСУ, пьянок
стало меньше. Во всяком случае, на первой линии фронта. Зато уже во второй
линии обороны и в тылу люди в погонах пьют по-прежнему.

Вот что рассказал "Стране" А., боец подразделения, расположенного под Бахмутом:
"Большая часть народа здесь постоянно бухает. Квасят все. Пьянка в части
начинается, как только начинает темнеть. В ближайший поселок в лавку отряжаются
гонцы за бухлом. Гонцы иногда даже не переодеваются в гражданку. Пьют даже те,
кто на боевом дежурстве - в патруле или на постах, с оружием. Проверяющие и
начальники караулов тоже все время под градусом, поэтому не замечают, что их
подчиненные тоже все синие".

\subsubsection{Пьяные войны с местными }

Если учесть, что пьют люди с боевым оружием, последствия для непосредственных
участников и для окружающих часто плачевны. Например, в судебном документе \href{https://reyestr.court.gov.ua/Review/76648400}{№
229/4106/18} описано, как два украинских морских пехотинца осенью 2018 года
открыли стрельбу возле сельского кафе в пгт Очеретино Донецкой области.

В пьяном виде один из солдат прострелил ногу гражданскому. А для острастки
дополнительно еще семь раз выпалил поверх голов сельчан. Товарищ бойца решил
поддержать своего сослуживца огнем и выпустил в воздух магазин своего
пистолета. Обстреляв посетителей кафе, бойцы спокойно удалились.

В одном из судебных дел - № 665/568/19, есть история о том, как вооруженные
украинские морпехи из того же подразделения устроили операцию устрашения в 2018
году против местных обидчиков в селе Хлебодаровка Херсонской области.\Furl{https://reyestr.court.gov.ua/Review/80652216}

Предыстория конфликта такова - в часть вернулся избитый местными морпех,
который рассказал командиру, что получил по шее в местном баре. Исполняющий
обязанности командира батареи, услышав рассказ своего подчиненного, поручил
нескольким бойцам-сержантам и даже одному лейтенанту, подъехать к бару и
разобраться с обидчиками, как написано в тексте судебного документа "для
урегулирования конфликта, в связи с неоднократными конфликтами между
военнослужащими и местными жителями".

А для усиления эффекта "урегулирования" приказал бойцам вооружиться автоматами.
Кстати, патроны для автоматов, как следует из судебного документа, морские
пехотинцы взяли "не числящиеся на регистрации в части".

Подъехав к кафе, морпехи особо не задумывались и устроили "профилактическую"
стрельбу по уличным фонарям и поверх голов посетителей бара. Спасаясь, сельчане
забежали в бар и забаррикадировались от вооруженных морпехов. И оттуда вызвали
полицию. Морпехи попытались преследовать сельчан, но так и не сумели взломать
двери в бар.

Уходя, уже, видимо, для острастки, морпехи разбили стекла и фары в стоящем у
бара авто и заодно побили водителя машины. Интересно, что полиция предъявила
подозрение лишь одному морпеху - и.о. командира батареи, отдавшему приказ о
вооруженном налете на бар.

Ему было предъявлено подозрение по ч.2 ст.29, мера пресечения - домашний арест.
По загадочной причине, "левые" патроны к оружию, которым были вооружены
мстители, не заинтересовали следователей. Как, собственно, и другие
обстоятельства налета на сельский бар.

\subsubsection{Солдаты бьют офицеров, а командиры - солдат}

Нещадно бьют по пьяной лавочке солдаты даже своих командиров - об этом гласят
материалы дела № 359/4569/19.\Furl{https://reyestr.court.gov.ua/Review/82163771}

22 апреля 2019 года шесть сержантов и солдат жестоко избили трех командиров -
двух лейтенантов и младшего лейтенанта. Конфликт начался из-за замечания,
сделанного офицером одному из бойцов по поводу самовольной отлучки и нежелания
заступать в наряд. Боец очень огорчился и рассказал своим товарищам о
командирском "беспределе".

Не долго думая, шесть сержантов и солдат ворвались в комнату офицеров и начали
бойню. Одному из офицеров ворвавшиеся даже не дали подняться с постели и
избили, перекинув кровать.

В подразделении ВСУ под селом Троицким Луганской области группа солдат решила
"повоспитывать" своего командира в звании лейтенанта - дело № 423/334/20.\Furl{https://reyestr.court.gov.ua/Review/89812998}

15 декабря 2019 года солдаты обиделись на офицера за то, что тот "неправильно
оставил военнослужащих и поехал в пгт Мироновский". В два часа ночи солдаты
наведались к спящему офицеру и жестоко избили его ногами, выбив командиру зубы.

Но чаще воспитывают мордобоем не подчиненные командиров, а наоборот - офицеры
солдат. 20 апреля 2020 года под Одессой в зенитно-артиллерийском полку два
пьяных офицера избили младшего сержанта срочной службы, который в тот момент
охранял с оружием склад.\Furl{https://reyestr.court.gov.ua/Review/90219207}

Старший лейтенант с лейтенантом свалили подчиненного на землю, сломали младшему
сержанту нос и долго пинали ногами. Всего, судя по материалам дела, сержанту
срочной службы было нанесено не менее пятнадцати ударов!

\subsubsection{Потери по разгильдяйству приписывают к боевым }

Смерти многих украинских бойцов на фронте, которые первоначально были отнесены
пресс-службами ВСУ к боевым, при расследовании оказываются не боевыми, а лишь
свидетельствуют о непрофессионализме командиров.

Например, случай под Авдеевкой 20 марта 2019 года, когда на мине подорвалась
украинская боевая разведывательная машина.\Furl{https://reyestr.court.gov.ua/Review/91695956}

Тогда погиб один военнослужащий, тяжело ранены еще двое. Боевая машина была
уничтожена. Судя по делу № 234/9876/19, досудебное следствие пришло к выводу,
что, отдавая приказ на выезд боевой разведывательной машины, командиры просто
не проинформировали бойцов о существовании на маршруте минных полей, которые
были установлены украинскими же саперами. В результате произошла трагедия.

Но в релизе пресс-службы бригады о том, что смертельная мина была украинской,
не упомянуто.

Потери от огня на поражение, который часто открывают украинские бойцы по своим
товарищам, а также потери личного состава из-за непрофессионализма командиров
также остаются за рамками официальных релизов ВСУ.

"Страна" насчитала несколько десятков судебных дел 2019 - 2020 годов, в которых
идет речь о разборках с применением табельного оружия и смертях солдат из-за
разгильдяйства командиров.

Например, в деле № 221/9318/19.\Furl{https://reyestr.court.gov.ua/Review/86476402}

11 декабря 2019 года на опорном пункте под пгт Новотроицкое Донецкой области,
младший сержант расстрелял из автомата своего товарища. Стрельбу сержант начал
из-за личной ссоры. По материалам дела, он выстрелил очередью, попав в живот
сослуживцу.

К слову, конфликты военных происходят не только по пьянке. Многие солдаты
жалуются на настоящее вымогательство со стороны офицеров.

"В нашей части каждый второй офицер старается нагрузить солдат материально, на
взятки. Взятки требуют за увольнения, чтобы не ставили часто в наряды, за
очередные отпуски. Даже за выдачу новой формы и чтобы выплатили законные боевые
надбавки и зарплаты. Есть и другие поборы - например, требуют покупать за свой
счет лопаты и пилы взамен сломавшихся во время выполнения работ. Я купил за
свой счет шуруповерт, который присвоил один из офицеров, заявив при этом, что я
якобы сломал тот, который был до этого", - рассказал "Стране" солдат В.

Случаи вымогательства армейского начальства с подчиненных также есть в реестре
судебных решений.

Например, в деле №219/4821/19.\Furl{https://reyestr.court.gov.ua/Review/81645614}

Солдат съездил в отпуск с фронта, оставив бронежилет, каску, спальный мешок и
каремат на складе своего подразделения. Однако, вернувшись, своего снаряжения
солдат на складе не нашел. А лейтенант, начальник склада, потребовал от солдата
4 500 гривен за "утерю" имущества. Так как солдат сразу не расплатился,
лейтенант в лучших традициях бандитских 90-х поставил солдата на "счетчик".

Через пару месяцев сумма "долга" солдата возросла уже до 5500 гривен. Солдат
пожаловался в полицию, которая и задержала предприимчивого офицера на получении
денег.
