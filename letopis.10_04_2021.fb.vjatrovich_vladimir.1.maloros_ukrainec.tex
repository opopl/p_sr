% vim: keymap=russian-jcukenwin
%%beginhead 
 
%%file 10_04_2021.fb.vjatrovich_vladimir.1.maloros_ukrainec
%%parent 10_04_2021
 
%%url https://www.facebook.com/volodymyr.viatrovych/posts/10219326466131889
 
%%author 
%%author_id 
%%author_url 
 
%%tags 
%%title 
 
%%endhead 

\subsection{Про малороса і українця}
\label{sec:10_04_2021.fb.vjatrovich_vladimir.1.maloros_ukrainec}
\Purl{https://www.facebook.com/volodymyr.viatrovych/posts/10219326466131889}

Малорос - це лінь українця. Вона може бути наслідком втоми. Але необов‘язково.
Бути малоросом просто легше. Це спокійно, не викликає конфліктів  з сусідами.
Хоч звісно не викликає у них захоплення, чи навіть просто поваги. Але на рівні
малороса немає потреби в цьому - «лиш би не чіпали». Малорос - це українець,
який не хоче рости, якому не треба багато.

Бути українцем важко. Це постійний процес відстоювання своєї особливості перед
собою та іншими. Говорити не так як вони і не те, що вони хотіли б чути від
тебе. Дряпатися кудись вгору, де тебе ніхто не чекає. Звідки на тебе зневажливо
кидають погляди інші, мовляв «а тобі чого там унизу не сидиться». 

Малоросу простіше, у нього вже є все що потрібно для його ідентифікації іншими.
Всі його особливості (вареники, борщ, гарні пісні...) вже давно відібрані  цими
іншими за їх смаками. То ж вони не викликають ні в кого заперечень. 

Українець постійно відтворює свої особливості, він має  просувати їх у світі,
де вони сприймаються з подивом, а іноді навіть ворожістю, яку викликає чуже і
незрозуміле. 

Малорос вірить, що найважливіше можна випросити на колінах. Українець знає, що
мусить вирвати його в боротьбі. 

Для малороса головне у головній пісні - «згинуть наші вороженьки як роса на
сонці». Для українця - «душу й тіло ми положим за нашу свободу». 

Малоросом бути легше,  але щоб не було соромно перед дітьми маєм залишатись
українцями. 
