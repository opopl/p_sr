% vim: keymap=russian-jcukenwin
%%beginhead 
 
%%file 17_03_2018.stz.news.ua.mrpl_city.1.gosti
%%parent 17_03_2018
 
%%url https://mrpl.city/blogs/view/gosti
 
%%author_id burov_sergij.mariupol,news.ua.mrpl_city
%%date 
 
%%tags 
%%title Гости
 
%%endhead 
 
\subsection{Гости}
\label{sec:17_03_2018.stz.news.ua.mrpl_city.1.gosti}
 
\Purl{https://mrpl.city/blogs/view/gosti}
\ifcmt
 author_begin
   author_id burov_sergij.mariupol,news.ua.mrpl_city
 author_end
\fi

Помнится, в дедушкином доме, кроме домочадцев – дочерей, зятьев, внуков,
постоянно кто-то присутствовал из гостей. Не из тех, что, покалякав и испив
чаю, удалялись восвояси, а тех, кто, придя или приехав, ночевал две-три, а то и
более ночей, становился на время как бы членом семейства. Гости присутствовали
почти беспрерывно, они сменяли друг друга осенью, зимой, весной и особенно
летом.

\ii{17_03_2018.stz.news.ua.mrpl_city.1.gosti.pic.1}

Это были многочисленные родичи с малой родины дедушки и бабушки - села Варвы,
одного из древнейших поселений Черниговщины, вдова дедушкиного брата и ее дети
- из пригородного Володарска, двоюродная сестра папы с сынишкой - из
Ленинграда, первая жена папиного дядьки  (почти ровесника ему) молдаванка тетя
Муся и вторая жена с двумя сыновьями от первого брака и дочкой от дядьки из
Сталино, старинные друзья – семья рыбаков из Безыменного. Визиты эти были, как
правило, неожиданны.

\enquote{Зимними} гостями были родственники из Варвы. Поскольку для крестьян
вообще, а тем более для колхозников, лишь пара недель в один из зимних месяцев
были доступны для отдыха. Всех больше запомнилась бабушкина двоюродная
племянница – тетка Марина. Она была чернява волосами и очами, но белолица и
румяна. Шею ее украшали несколько рядов бус из стеклянных разноцветных шариков,
которые она называла кораллами. С тех пор при упоминании слова \enquote{коралл}
вспоминаются не скелеты полипов из теплых южных морей, а бусы тетки Марины.
После обмена троекратными поцелуями с бабушкой и сестрами, тетка Марина
начинала скороговоркой передавать приветы от многочисленных близких и дальних
родственников, попутно сообщая о житье-бытье каждого из них.  Вставить хотя бы
одно слово в поток ее приветов было совершенно невозможно.

В то время, когда гостья сообщала новости из Варвы, а затем распаковывала
поклажу – торбы с сушеными грушами и яблоками, табаком-самосадом и завязанными
в веночки чесночинами, предназначенными частью для даров хозяевам, а остальное
для продажи на базаре (чтобы оправдать расходы на поездку), женская половина
семьи уже хлопотала у печки. Наконец борщ, заправленный старым салом и
чесноком, был готов, куски сулы - азовской рыбы, в ту далекую эпоху
предназначавшуюся для скромных обедов, -  дожаривались на сковородке, и все
садились за большой дубовый стол. Граненые стаканчики заполнялись вишневой
наливкой, трапеза начиналась.

После застолья тетка Марина, прикрыв глаза, начинала тихонько петь: \enquote{Повій,
вітре, на ВкраЇну, де покинув я дівчину...} Ей начинали подпевать постепенно
крепнувшими голосами женщины. Пение прерывалось рыданиями бабушки. Бабушка к
тому времени жила в Мариуполе более сорока лет, но город этот Украиной не
считала, хотя ее окружало много земляков-украинцев. Для нее Украина была там,
на хуторе Рудом близ Варвы, где она родилась, где прошло детство, где она с
девчатами ходила босиком в церковь в соседнее село Ладан, а черевики надевала
лишь у входа в храм – обувь надо было беречь. Она рассказывала внукам, какие
огромные осокори растут по берегам речек Удай и Варвицы, обещала повезти туда и
показать все. Внуки взрослели, бабушка старела, и поездка так и не состоялась.

В канун нового сорок седьмого года приехал демобилизованный за месяц до того
племянник дедушки Иван Поликарпович. Он был молод, а по отчеству его величали
за глаза только для того, чтобы отличить от Иванов – сыновей других дедушкиных
братьев. В ладно скроенном гвардейском мундире (как оказалось, сшитом им самим)
он был чудо как хорош. Дядя Иван ни минуты не мог сидеть без дела. Наколов на
неделю вперед дров, он принимался отгребать снег во дворе. В ту зиму в нашем
городе было необычайно много снега. Вечерами он переточил всю наличность в доме
ножей, ножовок и ножниц. Закончив одну работу, он присматривался, как бы
выискивая новую, чтобы занять ею руки. К несчастью, сразу после нового года
дедушка умер. Иван, побывав на заводе, куда его собирался определить один из
родственников, вернулся в родное село, где проработал всю жизнь трактористом и
комбайнером и дослужился до звания Героя Соцтруда.

Дети рано умершего брата дедушки Аксентия, обосновавшегося в начале двадцатых
годов в селе Никольском, переименованном в 1925 году в Володарск, прибывая
поездом в Мариуполь, прежде чем отправиться к матери, останавливались в доме
бабушки. Время от времени приезжала тетя Надя - главный скульптор-художник
фаянсового завода, чьи работы были представлены на Всемирной выставке в
Брюсселе в павильоне Советского Союза. Захаживал дядя Александр, военный врач,
прошедший Отечественную войну, потом добросовестно служивший несколько лет в
Монголии, за что правительство этой страны наградило его орденом, а в запас он
вышел в звании подполковника. Запомнился дядя Володя, в кубанке, в шинели,
опоясанный портупеей, он ехал из Болгарии в отпуск, где стояла его часть, бывал
он и после войны, когда был студентом, а затем доцентом Ворошиловградского
педагогического института. Но чаще всего навещала дом тетя Маруся, выпускница
Днепропетровского государственного университета, преподаватель географии, а
затем директор школы в Володарске.  Современные молодые люди вряд ли поверят,
что рано овдовевшая колхозница сумела помочь всем четверым детям получить
высшее образование... 

Тетя Муся – небольшого роста, моложавая и подвижная не по годам брюнетка с
черными живыми глазами - квартировала подолгу. Она старалась помочь маме по
хозяйству, а все свободное время вышивала крестиком яркие узоры на наволочках
для подушек. Но как только из старенького репродуктора раздавалась молдавская
мелодия, она вскакивала и начинала приплясывать в такт музыке. Тетя Муся всегда
с большим почтением и даже любовью вспоминала бросившего ее мужа и сетовала,
что будь у нее дети, он бы ни за что на свете ее не покинул. Нужно отдать
должное ее бывшему супругу Александру Марковичу, он купил для бывшей жены
небольшой домишко неподалеку от Городского сада, где она и прожила до самой
кончины. Александр Маркович, когда бывал по делам в Мариуполе, тоже не обходил
дом. Его визиты были стремительны и кратковременны. Еще с порога он кричал,
обращаясь к маме: \enquote{Шура, нажарь бычков!} И одновременно бросал авоську с еще
трепещущими рыбешками, которая смачно шлепалась на стол. Его просьба
исполнялась в возможно короткие сроки. Когда лакомство было готово, он извлекал
из потертого, видавшего виды портфеля поллитровку \enquote{Московской} - и, боже упаси,
чтобы кто-нибудь из присутствующих мужчин не поддержал в его пагубной страсти.
Но чаще всего мужское население дома во время его наскоков находилось на
работе, а он знал, что женщины водку в рот не возьмут, Александру Марковичу
приходилось самому \enquote{бороться с зеленым змием}. Опустошив блюдо с бычками и
половину бутылки, он так же стремительно, как и появлялся, исчезал.

Частыми гостями были Яков Степанович и Мария Платоновна из Безыменного. Они
приезжали вечером с рыбой, чтобы, переночевав, отправиться с восходом солнца на
базар для ее продажи. Яков Степанович был высоким жилистым потомственным
рыбаком с лицом, обветренным всеми ветрами Азова, и вместе с тем, лицом
благообразным и добрым. Характер его был воплощением спокойствия, мудрости и…
покорности. Мария Платоновна же – полная противоположность мужу. Полная,
приземистая, экспансивная, хриплоголосая, она никогда не лезла в карман за
словом. Всеми своими повадками она показывала, кто в семье главный. Однако
семья эта была крепкая, вырастившая четверых достойных детей. Самое интересное
начиналось после ужина, когда домашние дела были сделаны и все собирались
вместе в столовой, служившей одновременно и кухней, и Якова Степановича просили
что-нибудь рассказать. Из его повествований запомнилось, как азовские рыбаки в
лютые морозы на санях выходили на лов рыбы, запуская сети через лунки, как он
прятался в плавнях от вербовщиков, чтобы не загреметь в армию Деникина. Якову
Степановичу не сразу поверили, когда он поведал, что в их каюке (суденышке, на
котором он и его собратья по труду выходили в море) никто, в том числе и он, не
умеют плавать. Мария Платоновна в своих воспоминаниях к тому времени всегда
возвращалась к тому периоду своей жизни, когда служила горничной у купчихи
Мокровой. Кульминацией ее рассказа был момент, когда она приняла, извините,
ночной горшок за супник.

Иногда в суровые зимы, хромая и тяжело опираясь на клюку, забредала, облаченная
в одеяния, состоявшие из одних заплат, убогая Феклуша, нищенка, просившая
подаяние на кладбище. Ее тело и правая половина лица были изуродованы
параличом. Правый глаз неестественно краснел вывернутым нижним веком, пугая
детей остекленевшим взором. После нехитрого угощения гостья пристраивалась на
маленькой скамеечке к печке, грела руки теплом, источающим поддувалом. Время от
времени Феклуша пыталась что-то сообщить словами, которые невозможно было
разобрать, но тон их говорил - она безмерно благодарна хозяевам за кров и еду.
Поговаривали, будто традиция давать приют Феклуше по очереди в жилищах
мариупольцев возникла по совету некоего священника. Священник этот, как и
многие другие пастыри, был расстрелян, а традиция существовала до тех пор, пока
живы были его прихожанки...

Все ушло в прошлое. Давным-давно продан дом. Почти все названные здесь гости
ушли туда, откуда не возвращаются. Остались только воспоминания.
