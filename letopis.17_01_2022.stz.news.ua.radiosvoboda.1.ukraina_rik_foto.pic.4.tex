% vim: keymap=russian-jcukenwin
%%beginhead 
 
%%file 17_01_2022.stz.news.ua.radiosvoboda.1.ukraina_rik_foto.pic.4
%%parent 17_01_2022.stz.news.ua.radiosvoboda.1.ukraina_rik_foto
 
%%url 
 
%%author_id 
%%date 
 
%%tags 
%%title 
 
%%endhead 


\ifcmt
  tab_begin cols=2,no_fig,center

		pic https://gdb.rferl.org/8c8e0000-0aff-0242-f286-08d9d90e8a47_w1023_s.jpeg
		@caption_begin
13 травня. Печерський райсуд Києва відмовив прокуратурі у клопотанні про взяття
під варту народного депутата від «Опозиційної платформи – За життя» Віктора
Медведчука. 11 травня на тлі проведення обшуків у депутата Віктора Медведчука генпрокурорка
Ірина Венедіктова повідомила, що підписала підозри двом депутатам – Віктору
Медведчуку і його соратнику Тарасу Козаку. Їм інкримінують «державну зраду» і
«готування до розграбування національних цінностей на окупованій території».
Більше про те, як суд відправив Медведчука під домашній арешт, читайте у цій
\href{https://www.radiosvoboda.org/a/news-sud-medvedchuk/31253770.html}{статті}
		@caption_end

		pic https://gdb.rferl.org/08810000-0a00-0242-6c05-08d9d90e8a37_w1023_s.jpg
		@caption 20 травня. Пресконференцію за підсумками другого року своєї роботи Володимир Зеленський зібрав у цеху державного підприємства «Антонов».

  tab_end
\fi


 
