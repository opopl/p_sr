% vim: keymap=russian-jcukenwin
%%beginhead 
 
%%file 28_12_2020.news.ua.strana.shahov_aleksandr.1.ukraina_covid_vaccine
%%parent 28_12_2020
 
%%url https://strana.ua/news/309159-koronavirus-prohnoz-na-2021-hod-dlja-ukrainy-rossii-i-vseho-mira.html
 
%%author 
%%author_id shahov_aleksandr
%%author_url 
 
%%tags covid_vaccine
%%title В ЕС эпидемия пойдет на спад после Пасхи, а Украина пролетает мимо вакцины. Что будет с ковидом в 2021 году
 
%%endhead 
 
\subsection{В ЕС эпидемия пойдет на спад после Пасхи, а Украина пролетает мимо вакцины. Что будет с ковидом в 2021 году}
\label{sec:28_12_2020.news.ua.strana.shahov_aleksandr.1.ukraina_covid_vaccine}
\Purl{https://strana.ua/news/309159-koronavirus-prohnoz-na-2021-hod-dlja-ukrainy-rossii-i-vseho-mira.html}
\ifcmt
	author_begin
   author_id shahov_aleksandr
	author_end
\fi
\index[rus]{Коронавирус!Вакцина!Украина, 28.12.2020}

\ifcmt
pic https://strana.ua/img/article/3091/koronavirus-prohnoz-na-59_main.jpeg
caption Маски никуда не денутся и в 2021 году. Фото Pixabay
\fi

Через несколько дней наступит новый, 2021-й год.

Его предшественник прошел под знаком коронавируса. И, судя по всему, ковид
будет главной темой и следующих двенадцати месяцев. 

Правда, есть надежда, что наступающий год станет последним ударным периодом для
пандемии во всех крупных государствах. Кроме Украины. Для нашей страны просвет
начинает виднеться где-то с 2022 года. 

"Страна" собрала основные прогнозы по развитию ковид-инфекции в 2021 году - как
в мире, так и в Украине. 

\subsubsection{Когда пандемия пойдет на спад в 2021 году?}

Если брать крупные страны, то самые пессимистические прогнозы звучат по
Соединенным Штатам. По данным центра глобальных исследований при Вашингтонском
университете, к марту число умерших от ковида в США превысит 500 тысяч человек.
Сейчас это 340 тысяч. То есть темпы смертности вырастут.

Один из факторов, который называют причиной роста, сугубо экономический -
окончание моратория на выселение людей во время пандемии.

Дело в том, что с 1 января владельцы недвижимости смогут попрощаться с
арендаторами, которые не платили вследствие потери работы или других причин.
Потерявшие кров американцы начнут селиться у родственников или снимать одно
жилье на несколько семей. Что повысит заражение через домашние контакты - самый
опасный вид передачи вируса. 

В Штатах называют две основных календарных точки улучшения ситуации

"Когда мы дойдем до конца 2021 года, мы, вероятно, сможем начать ослаблять меры
предосторожности", - рассказал доктор Том Фриден, бывший директор Центров по
контролю и профилактике заболеваний.

При этом доктор наук Майкл Сааг, замдекана по вопросам здравоохранения
Университета Алабамы, считает, что заболеваемость начнет так или иначе падать
летом.

Что, впрочем, наблюдалось и в уходящем году и может объясняться сезонностью. 

Отдельный всплеск заболевания ожидается по всему миру из-за череды праздников -
в первые месяцы 2021 года. Об этом предупредила ВОЗ. И рекомендовала носить
маски даже во время встреч в семейном кругу (чего, конечно, вряд ли кто-то
станет придерживаться). 

Вообще, прогнозы организации оптимизмом не пышут. Директор Европейского бюро
ВОЗ Ганс Клюге на днях заявил, что ударный период развития коронавирусов - два
года. По этой логике на улучшение можно будет выйти лишь в конце 2021-го. 

Впрочем, есть и более позитивные прогнозы. В России заявили, что в наступающем
году эпидемия пойдет на спад. Об этом заявил академик РАН, завотдела в НИИ
эпидемиологии Роспотребнадзора Вадим Покровский. По его словам, к концу 2021
года "оптимизм у всех поднимется", а экономическая ситуация станет улучшаться.

В Германии же считают, что уже летом 2021 года немцы постепенно смогут
вернуться к нормальной жизни. А серьезные ограничения продлятся как минимум до
Пасхи, заявляет глава Всемирной медицинской ассоциации Франк Ульрих Монтгомери.

Это связано, по его словам, с началом вакцинации, чей эффект наступит не сразу. 

"Хотя вакцинация начнется раньше, чем ожидалось, положительный эффект от нее
будет проявляться постепенно. Поэтому нам придется жить с различными
ограничениями как минимум до Пасхи", - сказал Монтгомери.

То есть в конечном счете улучшение ситуации специалисты ставят в прямую
зависимость от вакцин. Разберемся, сколько людей в мире смогут охватить ими. 

\subsubsection{Сколько людей вакцинируют в мире?}

Начнем снова с американцев.

Первые вакцины в США будут развернуты в ближайшее время. Министр
здравоохранения Алекс Азар считает, что достаточное количество доз для любого
американца, который захочет сделать вакцину, будет доступно в конце весны 2021
года.

Другие американские эксперты говорят, до конца марта вакцины получит треть
населения Штатов - до 100 миллионов человек. Главные препараты - Moderna и
Pfizer. 

Правда, они делают оговорку, что могут быть и проблемы: как с безопасностью
вакцин, так и с тем, насколько длительный они будут давать иммунитет. 

Ричард Мартинелло, директор по профилактике инфекций Йельской больницы в
Коннектикуте, заявил, что могут быть потери в скорости вакцинации. На это нужны
месяцы. При этом, по его мнению, чтобы темпы роста Covid-19 упали существенно,
с прививкой должны быть от 70\% до 80\% населения. А если вакцина сама по себе
будет давать не стопроцентный иммунитет - то и больше. 

В Германии правительство утверждает, что к лету прививки смогут себе сделать
все желающие. Правда, когда они их сделают - вопрос. Как заявил министр
здравоохранения Йенс Шпан, стабильное падение заболеваемости начнется после
того, как вакцинируется 55-60\% людей. Выше мы приводили прогнозы, что спад
ожидают уже летом. 

В РФ ожидают в 2021 году запуска массового производства еще двух своих вакцин -
"Вектор" и препарата центра им. Чумакова, который допущен до клинических
испытаний. А "Спутник V", которая уже производится, только за рубежом планирует
выпустить в наступающем году около миллиарда доз. 

Как подсчитали в компании Goldman Sachs, к апрелю 2021 года вакцину получит
половина населения США и Канады, к маю - половина ЕС, Австралии и Японии.

При этом уже сейчас понятно, что только богатые страны (или те, кто сами делают
вакцину, как Россия), получат прививки первыми. 

Исследователи американской Высшей школы здравоохранения пришли к выводу, что
каждый четвертый житель планеты сможет получить прививку от коронавируса не
ранее 2022 года. 

По их данным, более 51\% от общего числа доступных во всем мире доз вакцин
придется на страны, где проживают в общей сложности менее 15\% населения Земли.
Таким образом, остальные 85\% населения получат чуть менее половины всех доз
вакцин.

Видимо, в эту вторую категорию попадет и Украина. 

\subsubsection{Чего ждать Украине?}

В Украине улучшения ситуации с коронавирусом ожидают в апреле (министр
здравоохранения Степанов) и в мае (главный санврач Ляшко). Видимо, главным
образом в силу сезонности - поскольку никаких внятных планов по вакцинации
украинцев нет.

А те, что есть говорят о прививках для сравнительно небольшой прослойки
населения - от 2 до 4\% примерно до конца весны. Да и те - под вопросом. 

Тем не менее, власти, в отличие от своих коллег в Евросоюзе и США, дышат
оптимизмом. Например, если верить Виктору Ляшко, в Украине болезнь начнет
отступать даже раньше, чем в Германии - весной. 

"Ориентировочно, я считаю, что уже с мая 2021 года у нас будет такой адаптивный
карантин, который позволит дышать легче. Уже не будет такого как было в этот
период.

Но этому будет способствовать и сезонность вируса, я думаю, мы будем видеть
четкую сезонность и в весенне-летний период коронавирус уменьшит свою
агрессивность и циркуляцию.

Плюс, появится вакцина и с каждым днем все больше людей будет
прививаться против коронавируса, формировать иммунитет, который постепенно
станет коллективным, и мы забудем вообще об этой ужасной болезни", - заявил
Ляшко.

Правда, не очень понятно, как Украина достигнет "коллективного иммунитета" уже
весной, если прививать украинцев начнут в лучшем случае в феврале - да и то
микроскопическими дозами вакцин. Тогда как в Евросоюзе, обеих Америках и
России, а также других странах (например, в Израиле), уже начали массово делать
прививки для населения.\Furl{https://strana.ua/news/308981-vaktsinatsija-ot-koronavirusa-nachalas-v-evrope.html}

Более того, ни одной вакцины Украина еще у себя не зарегистрировала. Хотя без этого колоть людям прививки будет нельзя. 

Как уже рассказывала "Страна" - дело в том, что вакцины, на которые Украина
может рассчитывать в рамках бесплатной гуманитарной помощи, еще пока не прошли
всех этапов испытаний. И, соответственно, не получили сертификатов даже в
странах-производителях. Не говоря уже о ВОЗ, которая должна одобрить препараты,
поступающие как гуманитарка. А это - еще несколько месяцев на изучение. 

И, раз украинский Минздрав не регистрирует ведущих западных вакцин - Pfizer и
Moderna - значит нам их пока поставлять никто не собирается. И здесь не должны
обманывать заявления чиновников, что они уже подготовили холодильное
оборудование для закупок того же Pfizer. 

Детальнее о перспективах по вакцинам - в материале "Страны" Почему Украина
рискует в 2021 году остаться без вакцин от ковида.\Furl{https://strana.ua/articles/analysis/309131-pochemu-ukraina-riskuet-statsja-bez-vaktsin-ot-kovida.html}

То есть пока вообще под вопросом, что и когда нам поставят. А значит, расчеты
Минздрава по спаду коронавируса в апреле-мае или взяты с потолка, или
рассчитаны на сезонные факторы. 

\subsubsection{Смертность в Украине вырастет}

Но даже если представить, что Украине открыли доступ к вакцине, скорость такой
теоретической вакцинации будет крайне низкой. Как подсчитали в Киевской школе
экономики, сам процесс прививания на украинских реалиях может растянуться на
полгода и за это время покрыть только 36\% населения. 

"В Украине ориентировочно 13,5 миллионов людей старше 55 лет, которые есть в
группе риска по COVID-19. Их надо вакцинировать в первую очередь.

Мы подсчитали, что даже в идеальном мире, когда все 6 580 амбулаторий семейной
медицины будут вакцинировать пациентов по 8 часов в день без перерывов и
выходных, это продлится 6 месяцев. Это всего 36\% населения, что даже не
гарантирует установление коллективного иммунитета.

В реальном мире же обеспечить такую эффективность практически нереально. А для
развертывания дополнительных пунктов вакцинации, чтобы увеличить пропускную
способность, необходимы дополнительные медицинские кадры", - говорится в
аналитическом отчете КШЭ. \Furl{https://kse.ua/ua/about-the-school/news/vaktsinatsiya-ne-vryatuye-ukrayinu-vid-poshirennya-covid-19-u-2021-rotsi-tse-lishe-odin-iz-bagatoh-instrumentiv-analiz-kse/?fbclid=IwAR1999LDBJDft-ckWE2z7c4obqXY0h6cXBq6y0masCN0phDm6LRq8_dYg9U}

То есть, на украинских мощностях нужно гораздо более полугода, чтобы
вакцинировать чуть более трети населения. Хотя этого мало для коллективного
иммунитета (на Западе говорят, что нужно привить 60-80\% людей).

А с учетом задержек поставок, мы можем с уверенностью говорить - в 2021 году
действительно массовой вакцинации в Украине не будет. И решающего воздействия
на течение болезни украинцев она не окажет. 

С поправкой на это можно ждать "качелей", аналогичных этому году: остановка
роста заболеваемости летом и новый рост осенью-зимой. 

Также можно с уверенностью говорить, что смертей в наступающем году от
коронавируса будет больше. За 2020 год мы будем иметь 18-19 тысяч погибших в
Украине. Но фиксировать их начали только с февраля, и на нынешние величины -
100-200 человек в день - Украина вышла относительно недавно. 

А теперь мы с них будем начинать новый год. И пока тенденции к снижению до
уровня прошлой зимы-весны не просматривается. То есть, как и в США, Украина
может получить в 2021 году существенный рост смертей от ковида. 

И поломать этот сценарий может лишь массовая вакцинация, которую следует ударно
провести в первой половине следующего года. Пока это возможно лишь на
российской вакцине, которую можно производить в Украине (Москва согласилась
передать украинским предприятиям технологию). Но данный вариант зависит от
того, согласится ли Киев зарегистрировать "Спутник". 
