% vim: keymap=russian-jcukenwin
%%beginhead 
 
%%file 19_09_2019.fb.fb_group.zkkk.1.shuljavska_respublika.cmt
%%parent 19_09_2019.fb.fb_group.zkkk.1.shuljavska_respublika
 
%%url 
 
%%author_id 
%%date 
 
%%tags 
%%title 
 
%%endhead 
\subsubsection{Коментарі}

\begin{itemize} % {
\iusr{Konstantin Mozghovyi}

Ось із-за таких полковників і прийшли до влади лєніни, артьоми, алєксєєви, які
потім їх же і розстріляли, а родини у сталінські концтабори відправили.

\begin{itemize} % {
\iusr{Богдана Мурашова}
\textbf{Konstantin Mozghovyi}, 

це були дворяни - люди честі. А мудрий нарід повівся на популізм аголтєлтх
гопніків. А Ви, не шановний, не маєте жодного морального права давати оцінку
діям моїх пращурів. Розкажіть краще чим займались Ваші.

\begin{itemize} % {
\iusr{Денис Голубицкий}
\textbf{Богдана Мурашова} Костянтин не вартий і слова від Вас. Я таких в бан без дискусій відправляю

\iusr{Konstantin Mozghovyi}
\textbf{Богдана Мурашова}, і тому обзивав нижчих за званням "скотинами", на підставі свого дворянства?

\iusr{Konstantin Mozghovyi}
\textbf{Богдана Мурашова}, 

мої пращури жили на Шулявці. Не дворянствували, але мали будинок з балконом на
2-му поверсі, на Виборзькій. Всі ті будови знесли під Олімпіаду-80, збудували
будинки-книжки, тепер гуртожитки КПІ. Пам'ятаєте красивий дворянський будинок
біля кіно-ру Довженка ? Мене туди у музичну школу водили. А тітки жили на
Смоленській біля заводу "Більшовик". Так само, як і Ви, починаючи від дідів,
я є корінним киянином, але не живу у Києві давно.

\iusr{Maksym Oleynikov}
\textbf{Богдана Мурашова} 

Пані Богдано, дружньо раджу: не гайте час і нерви на дискусії з подібними
"знавцями", які мають абсолютно хибне, але при цьому залізобетонне уявлення про
вітчизняну (і не лише) історію. До того ж, як правило, хамство і безапеляційна
зверхність для таких - норма у спілкуванні....

\iusr{Богдана Мурашова}
\textbf{Konstantin Mozghovyi}, скотинами він називав жандармів. Вам їх шкода?

\iusr{Богдана Мурашова}
\textbf{Konstantin Mozghovyi}, до речі, а що Ваші пращури робили у 1905 році, а у 1917-му, а в 1937-му?

\iusr{Богдана Мурашова}
\textbf{Maksym Oleynikov}, дякую за підтриму. У мене з цим просто - буде хамити, піде в БАН.

\iusr{Konstantin Mozghovyi}
\textbf{Богдана Мурашова}, жандарм - це військовий для нагляду за порядком. І обзивати людину при виконанні службових обов'язків дуже не виховано, м'яко кажучи.

\iusr{Богдана Мурашова}
\textbf{Konstantin Mozghovyi}, у нас з Вами різне розуміння поняття честь і гідність. Ви прийшли в мою групу і будьте ласкаві поводити себе належним чином.

\iusr{Konstantin Mozghovyi}
\textbf{Богдана Мурашова}, 

я не знаю, що робили у 905 чи 917, просто жили собі спокійно, як і весь Київ,
ніяких революцій тут не було. У 1938 батька моєї мами забрали \enquote{як ворога
народу}, він був директором маслозаводу, моя мама так ніколи його і не бачила,
була 1938 р.н. Бабушку з трьома малими дітьми вивезли в Красноярський край,
повернулись у 1946 р. А займались \enquote{красним терором} якраз оті алєксєєви, яких
рятували Ваші родичі від жандармів. Ваші родичі потім де \enquote{сиділи} в яких
краях?

\iusr{Богдана Мурашова}
\textbf{Konstantin Mozghovyi}, 

пра-, прадід, про якого мова у пості, помер у 1918 році від старості. Його
донька, яка виводила Алєксєєва, з чоловіком - викладачем ьа двома доньками
(одна з яких моя бабуся) КПІ в 1920 році переїхала до села Маслівка, де було
організовано маслівський с/г інститут. Думаю це і врятувало родину від
репресій. Певернулись до свого будиночку вони по війні. Директорів у нас в
родині не було, членів КПРС також. А переховувались в нашому будинку багато
хто: і граіиня Наталія Брасова, і втікачі з сирецького концтабору:

\url{https://m.facebook.com/groups/2430033423886749/permalink/2472729492950475/}

\end{itemize} % }

\end{itemize} % }

\iusr{Виктория Зайченко}

Приблизно в цей час, можливо, трошечки пізніше, на 2 Дачній жила сім'я сестри
мого прадіда. Її чоловік, поляк, також приїхав працювати інженером на завод
Греттера і Криванека.

\begin{itemize} % {
\iusr{Богдана Мурашова}
\textbf{Виктория Зайченко}, дуже цікаво! А як їх звали, який номер будинку?

\iusr{Виктория Зайченко}
\textbf{Богдана Мурашова} 

Стрєлковськи Микола Владиславович і Анна Петрівна. Номера будинку я достеменно
не знаю, але він згадується в двох документах- перший раз у записі про смерть
Миколи Владиславовича (1926р), другий- в документі про те, що їх молодший син,
Володимир, пропав без вісті у 1944. Номери ці виявилися різні, в першому
документі- 2 Дачна, 10, а в другому- 13чи 18, запис друкований, погано видно

\end{itemize} % }

\end{itemize} % }
