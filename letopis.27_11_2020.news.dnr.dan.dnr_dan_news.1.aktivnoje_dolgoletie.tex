% vim: keymap=russian-jcukenwin
%%beginhead 
 
%%file 27_11_2020.news.dnr.dan.dnr_dan_news.1.aktivnoje_dolgoletie
%%parent 27_11_2020
 
%%url https://dan-news.info/obschestvo/okolo-180-000-zhitelej-stali-uchastnikami-aktivnogo-dolgoletiya-organizatory-zayavili-ob-uspexe-akcii.html
 
%%author Донецкое Агентво Новостей (ДАН)
%%author_id dnr_dan_news
%%author_url 
 
%%tags dnr,stariki
%%title Около 180 000 жителей стали участниками «Активного долголетия», организаторы заявили об успехе акции
 
%%endhead 
 
\subsection{Около 180 000 жителей стали участниками «Активного долголетия», организаторы заявили об успехе акции}
\label{sec:27_11_2020.news.dnr.dan.dnr_dan_news.1.aktivnoje_dolgoletie}
\Purl{https://dan-news.info/obschestvo/okolo-180-000-zhitelej-stali-uchastnikami-aktivnogo-dolgoletiya-organizatory-zayavili-ob-uspexe-akcii.html}
\ifcmt
	author_begin
   author_id dnr_dan_news
	author_end
\fi

\index[rus]{Проекты!ДНР!Активное Долголетие}

\ifcmt
pic https://dan-news.info/wp-content/uploads/2020/11/dan-news.info-2020-11-27_08-06-05_638167-dsc-9308-1024x682.jpg
\fi

Донецк, 27 ноя – ДАН. Проект «Активное долголетие» стал успешным в ДНР, за год
его участниками стали 180 тысяч человек. Об этом сегодня на брифинге в ДАН
сообщила замгендиректора по развитию и связям с общественностью Донецкой
республиканской универсальной научной библиотеки имени Н. Крупской Ольга
Арутинова.

«Проект «Активное Долголетие» − это в первую очередь поддержка, внимание и
небезразличие. Это стремление сделать активным образ жизни людей старшего
поколения, улучшить их психоэмоциональное состояние, укрепить здоровье. За год
реализации проекта около 180 000 человек стали участниками «Активного
долголетия», для них было проведено более 1000 мероприятий», – сказала
Арутинова.

Она добавила, что для пожилых людей работают кружки и клубы по интересам,
посвященные изобразительному искусству, литературному творчеству,
краеведческому туризму и путешествиям, рукоделию – вязанию, вышиванию и
кружевоплетению. Для тех, кто занимается подсобным хозяйством, работает кружок
по обсуждению садоводства и огородничества. Есть секции и для желающих изучить
иностранные языки или повысить компьютерную грамотность. Доступно и спортивное
направление, в частности занятия йогой.

«Члены клуба активно принимают участие в насыщенной культурно-просветительской
жизни библиотеки. Это и творческие встречи, презентации новых изданий, общение
с известными личностями, в том числе в рамках интеграционной деятельности. На
сцене лекционного зала библиотеки проходили показы литературно-музыкальных
постановок при содействии муздрамы, а также концертные программы артистов
филармонии», — отметила представитель библиотеки.

Все мероприятия проходят с соблюдением масочного режима и социальной дистанции.

Презентация «Активного долголетия» состоялась 28 ноября прошлого года. Проект
реализуется главной библиотекой ДНР и общественным движением «Донецкая
Республика» при поддержке министерства культуры. Подробную информацию о проекте
и ближайших мероприятиях можно узнать по телефону: 071 499 5293
