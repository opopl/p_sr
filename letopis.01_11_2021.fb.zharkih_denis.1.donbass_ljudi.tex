% vim: keymap=russian-jcukenwin
%%beginhead 
 
%%file 01_11_2021.fb.zharkih_denis.1.donbass_ljudi
%%parent 01_11_2021
 
%%url https://www.facebook.com/permalink.php?story_fbid=3112489442297785&id=100006102787780
 
%%author_id zharkih_denis
%%date 
 
%%tags chelovek,donbass,obschestvo,ukraina,vojna
%%title Донбасс - Люди
 
%%endhead 
 
\subsection{Донбасс - Люди}
\label{sec:01_11_2021.fb.zharkih_denis.1.donbass_ljudi}
 
\Purl{https://www.facebook.com/permalink.php?story_fbid=3112489442297785&id=100006102787780}
\ifcmt
 author_begin
   author_id zharkih_denis
 author_end
\fi

Раньше по работе часто ездил на Донбасс, особенно в Луганскую область. И люди
там хорошие, и атмосфера открытая. Отношения между людьми более жесткие, меньше
экивоков, меньше двоемыслия. Но за всей этой нестоличной грубостью чувствуется
надежность промышленных людей, их смелость, товарищество. Как этого давно не
хватает в Киеве. Я ведь вырос не в центре, а в промышленном районе,
обставленном различными заводами и заводиками. Сегодня все эти заводы закрыты.
Ни одного не осталось. Сегодня рабочим быть стыдно. Вот вором быть не стыдно, а
рабочим стыдно. 

Те, кто презирает Донбасс выросли на ином воспитании, они презирают труд, в
принципе, не работают, а приспосабливаются. У нас выросло поколение
профессиональных приспособленцев. И когда пошла война, они приспособились к
войне. Нет, воевать они не пошли, они извлекают из этого выгоду. Вот
приспособленцы хотят уничтожить дух трудового народа, который остался на
Донбассе. Жулики вообще не любят честных людей, как шлюхи честных женщин. 

Нам бы наоборот, сохранить промышленные традиции людей, эту культуру. Ну, не
нужно нашей стране столько лакеев, нужно же и работать кому-нибудь. А еще я
хочу спокойно приехать на Донбасс и открыто говорить с людьми, как это было
раньше. Но теперь я для них киевлянин, враг. И я ненавижу тех политических
шлюх, что сделали из меня врага для миллионов людей.

2015

\ii{01_11_2021.fb.zharkih_denis.1.donbass_ljudi.cmt}
