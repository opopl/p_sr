% vim: keymap=russian-jcukenwin
%%beginhead 
 
%%file slova.chelovek
%%parent slova
 
%%url 
 
%%author 
%%author_id 
%%author_url 
 
%%tags 
%%title 
 
%%endhead 
\chapter{Человек}

%%%cit
%%%cit_pic
%%%cit_text
Самое дорогое - это Люди
%%%cit_comment
%%%cit_title
\citTitle{Желая быть ангелом, превращаемся в демона}, 
Протоиерей Андрей Ткачёв, youtube.com, 04.06.2021
%%%endcit

У диплоїдних організмів, таких, як \emph{Людина}, присутні копії майже всієї
ДНК.  Зокрема, є пари хромосом по одній від кожного з батьків, однак для
дослідників цінність представляє сума тільки однієї копії пар хромосом.
Фактично, розмір геному \emph{Людини} являє собою приблизно 6 мільярдів пар
основ, однак вчені використовують тільки половину в якості
\enquote{репрезентативної} копії.  Вчені зазначили, що 8\% геному пропустили у
2003 році не через відсутність в них важливої інформації, а через технологічні
обмеження. Дослідники наголосили, що ці обмеження вдалося подолати завдяки
секвенуванню, яке розробили вчені з Pacific Biosciences і Oxford Nanopore,
\textbf{Вчені повністю розшифрували геном людини}, day.kiev.ua, 03.06.2021

В жизни любого \emph{человека} есть, как хорошие, так и отрицательные моменты.
Существует заблуждение, что психотерапия направлена на то, чтобы сделать ее
протекание счастливым и беззаботным. Но на самом деле, специалисты в этой сфере
работают над тем, чтобы последствия испытанного стресса, страха, неудач не
мешали вам в дальнейшем. Например, если вы не можете устроиться на хорошую
работу – психотерапевт не поможет. Но, если вы боитесь ходить на собеседования,
испытываете трудности в коммуникации с новыми людьми, рушите связи из-за
конфликтного характера – вполне,
\textbf{Для чего нужна психотерапия личности?}, beztabu.net, 02.06.2021

Фаина ищет и находит простые слова, образы, сюжеты. У нее, в отличие от
современных потребителей, болит душа. Не слышать ее может лишь тот, у кого души
нет. Для кого патриотизм это не жертвенность во имя страны, а способ
обогащения, для кого война только прибыльный бизнес, для кого своя рубашка
ближе к телу, а хата с краю.  Послушайте ее обращение в ООН. Никакой
театральности, надрыва, пафоса. Она, скорее, робеет, чем кого-то обвиняет,
требует. Ей как-то неудобно, ведь она понимает, что дети не помнят мира, а
большие дяди в политике не понимают. Точнее, все эти дяди понимают, но просто
дяди негодяи, и заняты другим. А ей, Фаине, за этих дядей неловко и стыдно.
Хотя стреляют не по им, а по ней. Вот это и есть крест настоящего
\emph{Человека}, испытывать стыд за чужую подлость. Маленький \emph{Человек},
но большая душа,
\textbf{Гибель людей на Донбассе говорит о том, что наше общество больное},
Денис Жарких, strana.ua, 04.06.2021

Ведь важно не то, что плетут политики, а важно, чтобы не стреляли, чтобы \emph{Люди}
не погибали, чтобы мы опять стали \emph{Людьми}, а не чинами и капиталами. Общество
больно, его покидает \emph{Человечность}. Очевидно, прохудилась общественная совесть.
Как вы думаете?,
\textbf{Гибель людей на Донбассе говорит о том, что наше общество больное},
Денис Жарких, strana.ua, 04.06.2021

Как жаль, как ужасно жаль, что не существует больше такого \emph{Человека}... Он
остался в прошлом. Он продолжает жить лишь в книгах, фильмах, рассказах наших
дедов...  \emph{Человек} утратил свое величие, и произошло это не в одно мгновение.
Ныне мы живем при капитализме, и нет больше тех великих идеалов - да вообще
никаких идеалов нет, а есть лишь личный интерес, не выходящий за рамки
материального: тепла, сытости и комфорта. Мы превратились в потребителей. Идея
низвергнута. У нас - тех, кто был когда-то велик как единый народ - ныне
раздор, вражда и отчужденность. Мы думаем по-другому, поступаем по-другому. Мы
извратили многие понятия. Мы упрекаем и обвиняем друг друга, завидуем и злимся,
предъявляем взаимные счеты. Мы больше не \enquote{Величество}...  Мы больше не братья.
Мы живем в разных государствах, между нами пролегли границы. Мы отдаляемся друг
от друга с огромной скоростью... ,

Каждый \emph{человек} — это целый огромный мир в себе, это сложная загадка с долгим
прошлым и бесконечным будущим, 

То що ж сталося у жовтні 2018 року? Можна з впевненістю стверджувати, що
«Портрет Едмунда Беламі» не має «фіксованого» автора, але має власників. Поміж
численними співавторами, які створили це зображення, має бути введена на рівних
нежива субстанція, якою є алгоритм. Тобто 2018 року у світ «високого мистецтва»
увійшла нова гібридна авторська сутність – \emph{людино-машина}. Це сталося
цілком очікувано. Технології, які значною мірою змінили навколишній світ, тепер
сфокусовані на \emph{Людині} і на всьому живому на планеті. Ми стоїмо перед
питаннями: Чи дійсно \emph{Людина} звільнилася від Бога і сама може впливати на
все живе? Чи може \emph{Людина} потрохи зникне на корить \emph{людино-машини}?
Чи не стане \emph{людське} життя лише виконанням програми, яку написали десь, у
науковій лабораторії? Яким буде поняття/розуміння \emph{Людини} в майбутньому?,
\citTitle{Мистецтво в добу «привидів у машинах»}, Богдан Шумилович, zaxid.net, 02.06.2020

Облишимо культурно зумовлені факти. Маркес, Борхес і Льйоса змусили мене
задуматися над іншим. Власне, над тим моментом зрілості, коли сам обираєш, як
ти називаєшся. Навіть прийнявши заготовлені родом варіанти, \emph{людина} має свободу
вибору. Хтось живе як Толік, хтось — як Анатолій Васильович, хтось взагалі стає
Анатоль. Кілька літер, додаткових звуків і відразу зовсім інший портрет. Навіть
те, як людина вимовляє власне ім'я, повільно й поважно, чи квапливо і голосно,
скаже чимало про власника і його ментально-душевний стан.  Напевне, в
дорослості, ми всі розуміємо — немає красивих і некрасивих імен. Є особистості,
що своїм досвідом і характером перетворили власне ім’я на символ. Символ сили
чи слабкості, перемоги чи підлості,
\citTitle{Як ти називаєшся?}, Анна Данильчук, day.kyiv.ua, 10.06.2021

