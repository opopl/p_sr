% vim: keymap=russian-jcukenwin
%%beginhead 
 
%%file 06_09_2021.fb.bryhar_sergej.1.rus_artisty_ukraina
%%parent 06_09_2021
 
%%url https://www.facebook.com/serhiibryhar/posts/1781229928743618
 
%%author_id bryhar_sergej
%%date 
 
%%tags kultura,muzyka,rossia,ukraina
%%title Власне, якого біса російські артисти мають годуватися в Україні? Резидентам ворожої держави тут взагалі не місце!
 
%%endhead 
 
\subsection{Власне, якого біса російські артисти мають годуватися в Україні? Резидентам ворожої держави тут взагалі не місце!}
\label{sec:06_09_2021.fb.bryhar_sergej.1.rus_artisty_ukraina}
 
\Purl{https://www.facebook.com/serhiibryhar/posts/1781229928743618}
\ifcmt
 author_begin
   author_id bryhar_sergej
 author_end
\fi

"Здесь надо отметить, что хотя Россия и основное поле заработка – это поле не
единственное. Есть еще страны Балтии, Украина, которая относится к нам
предельно доброжелательно. Мы сейчас не спеша обсуждаем возможность концертов в
этих странах", - розповів лідер гурту "Ногу Свело" Максим Покровський 15 травня
2021 року.

\ii{06_09_2021.fb.bryhar_sergej.1.rus_artisty_ukraina.pic}

Ну ось і "дообсуждалися". Маємо тепер міні-тур по Україні: Одеса, Дніпро,
Харків, Київ.

"А чьо, Макс - хороший русскій, наш чєловєк, пусть виступаєт", - скажуть мені
місцеві мегаліберальні мегаліберальчики, оці от, що "абнятьіплакать", які добрі
та чуйні... зокрема і до представників ворожої сторони. Та, власне, вони вже
кажуть. 

"Це утопія", - одразу ж парирую я. - "Шкідливі - всі. І оті, що буцімто
"хороші", бо не восхваляють Путіна, значно небезпечніші, ніж відверті
московські фашисти, адже "обрусєніє края" краще виходить саме в них. Дурненькі
й довірливі "тутешні" вважають їх "своїми", "нашими"... І це велика біда. В
людей, які й так мають проблеми з національною ідентичністю та світоглядними
основами, зводить не ноги, а голови...

Власне, якого біса російські артисти мають годуватися в Україні? Резидентам
ворожої держави тут взагалі не місце!

Чи може влада працює в напрямку трансформації статусу тієї держави - з ворожої
на... не ворожу в очах більшості населення?

З 2015 року "господін Покровскій" та його колектив неодноразово відвідували
окупований Крим, незаконно перетинаючи державний кордон України. Це вже привід
для заборони на в'їзд. І ця заборона існувала. Тепер її, чомусь, немає... В нас
тут взагалі коїться чортзнащо. 

Навіщо потрібні "кримські платформи", "меморандуми", "кримські ініціативи",
якщо знімаючи заборону на в'їзд і дозволяючи працювати в Україні людям, які
неодноразово незаконно в'їжджали до Криму,  влада фактично визнає анексовану
територію частиною іншої держави?

А ще є проблема з нашим протестним потенціалом. Якби щось подібне відбувалося у
2015-18 роках (а воно й відбувалося, недбалості та провокацій не бракувало й
тоді), думаю, вірогідність, що до самого виступу так і не дійде, була б значно
вищою, ніж тепер.

На жаль, загальна деградація триває...

\ii{06_09_2021.fb.bryhar_sergej.1.rus_artisty_ukraina.cmt}
