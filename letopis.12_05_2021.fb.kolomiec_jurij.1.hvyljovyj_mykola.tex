% vim: keymap=russian-jcukenwin
%%beginhead 
 
%%file 12_05_2021.fb.kolomiec_jurij.1.hvyljovyj_mykola
%%parent 12_05_2021
 
%%url https://www.facebook.com/permalink.php?story_fbid=949825109143132&id=100023469540550
 
%%author 
%%author_id 
%%author_url 
 
%%tags 
%%title 
 
%%endhead 

\subsection{Микола Хвильовий - 88 років}
\label{sec:12_05_2021.fb.kolomiec_jurij.1.hvyljovyj_mykola}
\Purl{https://www.facebook.com/permalink.php?story_fbid=949825109143132&id=100023469540550}


\ifcmt
  pic https://scontent-iad3-2.xx.fbcdn.net/v/t1.6435-9/185402249_949824635809846_3874647537587219034_n.jpg?_nc_cat=108&ccb=1-3&_nc_sid=730e14&_nc_ohc=Dbxl0KPfwfYAX8-Yij-&_nc_ht=scontent-iad3-2.xx&oh=04206229fe4cdae807ea84eeb176963f&oe=60C3C049
\fi


13 травня минає 88 років від дня загибелі (1933) українського прозаїка, поета,
публіциста, одного з основоположників пореволюційної української прози Миколи
Хвильового (справжнє ім'я – Микола Григорович Фітільов) (13 грудня 1893 р.,
Тростянець, нині Сумська область – Харків).

Життєпис. Народився в селищі Тростянець на Харківщині (нині райцентр Сумської
області) у родині вчителів. Батько, Григорій Олексійович Фітільов, був із
дворян, у повсякденний час, як писав сам М.Хвильовий «в высшей степени
безалаберным человеком» і п'яницею. Розмовляв російською, і саме завдяки йому
хлопець «рано перечитав російських класиків, добре познайомився із зарубіжними
– Дікенсом, Гюго, Флобером, Гофманом».

Навчався у початковій школі в селі Колонтаєві, де вчителювала його мати,
Єлизавета Іванівна (до заміжжя – Тарасенко), потім продовжив навчання в
Охтирській чоловічій гімназії, яку був змушений залишити через участь у «так
званому українському революційному гуртку», а згодом в Богодухівській гімназії,
звідки був виключений за зв'язки з соціалістами під час революційних
заворушень.

Підлітком мандрував у пошуках заробітку Донбасом і півднем України. Пізніше
працював слюсарем у ремісничій школі, у канцелярії волосної управи села
Рублівки, брав участь у роботі місцевої «Просвіти». Від 1916 р. – учасник
Першої світової війни. Вишкіл в окопах війни сформував у ньому переконаного
більшовика. На чолі повстанського загону, який він організував наприкінці 1918
р. на Харківщині, воював проти гетьманців, німців, дроздовців, армії УНР.

У квітні 1919 р. вступив до КП(б)У. Цього ж року одружився з учителькою
Катериною Гащенко, вона йому народила дочку Іраїду, але цей шлюб швидко
розпався.

На початку 1921 р. їде «завойовувати» столицю – Харків. Працював слюсарем на
заводі. Одружився з Юлією Уманцевою, яка мала дочку від першого шлюбу Любов, її
М.Хвильовий приймав як рідну і ніжно називав Любистком. Того ж року почав
друкуватися в газетах і журналах, в альманахах «Штабель», «На сполох». Активно
заявив про себе, як один із організаторів літературно-художнього життя,
член-засновник багатьох тогочасних літературних організацій: спочатку «Гарт»
(1923); саме в нього зародилась ідея створення напівофіційної студії «Урбіно»,
що збиралась у нього ж на квартирі і була прообразом ВАПЛІТЕ (1924); потім були
ВАПЛІТЕ (1926) (віцепрезидент), ВУСПП (1927), «Пролітфронт» (1930).

У 1920-ті рр. повністю підтримував і впроваджував у життя політику
«українізації», виступав проти русифікаційного і «просвітянського» векторів
розвитку української радянської культури. У 1925 р. в газеті «Культура і
побут», редактором якої був В.Еллан-Блакитний, М.Хвильовий опублікував статтю
«Про „сатану в бочці“, або про графоманів, спекулянтів та інших просвітян»,
якою фактично зініціював літературну дискусію 1925-28 рр. До кінця 1925 р.
опублікував книгу памфлетів під назвою «Камо грядеши?», а в 1926 р. - «Думки
проти течії»; того ж року в газеті «Культура і побут», додатку до газети «Вісті
ВУЦВК», надрукував наступну серію памфлетів «Апологети писаризму». Своєрідним
продовженням цих памфлетів стала знаменита стаття «Україна чи Малоросія?», яка
однак, за життя автора так і не з'явилася друком. У цих творах автор висловив
вимогу перед новою українською літературою припинити наслідувати Москву й
орієнтуватися на «психологічну Європу». Вважав, що на зміну провідній ролі
Європи в культурному процесі має прийти «євразійський Ренесанс», у якому
провідну роль відводив новій українській культурі.

З позицією М.Хвильового солідаризувались українські націонал-комуністи,
літературна група «неокласиків» на чолі з М.Зеровим та широкі кола національно
свідомої української інтелігенції. Влітку 1926 р., у розпал літературної
дискусії, з'явилася друком перша частина роману «Вальдшнепи», де його
персонажі, так само невтомно полемізуючи, дошукуються відповідей на найгостріші
питання доби, порушують болючі проблеми національного буття,
національно-культурного відродження України, осмислюючи непрості уроки
революції. Проте саме ці роздуми були піддані нищівній критиці. Незважаючи на
це, М.Хвильовий пристрасно заперечував так званий масовізм, профанацію
мистецтва, зведення його до ролі ідеологічного обслуговування партійної й
державної політики, наголошував на необхідності позбутися залежності від
«російського диригента», адже для більшості тогочасних літераторів було
властиве намагання копіювати готові форми і художні засоби російської
літератури. М.Хвильовий вважав таке безглузде копіюванням раболіпством, яке
ніколи не дасть справді високохудожнього твору: «Перед нами стоїть таке
питання: на яку зі світових літератур взяти курс? В будь-якому випадку, не на
російську. Від російської літератури, від її стихії українська поезія повинна
втікати як можна швидше. Справа в тому, що російська література віками тяжіє
над нами, як хазяїн положення, який привчив психіку до рабського наслідування.

Українське суспільство, зміцнівши, не примириться зі своїм фактичним гегемоном
– російським конкурентом. Ми повинні негайно стати на сторону молодого
українського суспільства, яке втілює не лише селянина, але і робітника, і цим
назавжди покінчити з контрреволюційною ідеєю створювати на Україні російську
культуру.

Європа – це досвід багатьох віків. Це не та Європа, якій Шпенглер оголосив
«присмерк», не та, що гниє, і до якої вся наша ненависть. Це – Європа
грандіозної цивілізації, Європа – Гете, Дарвіна, Байрона, Ньютона, Маркса і т.
д. Це та Європа, без якої не обійдуться перші фаланги азійського ренесансу.

М.Хвильовий намагався розтлумачити зміст висунутих ним закликів і гасел («Геть
від Москви!», «До психологічної Європи», «Азіатський ренесанс»), пояснював
опонентам, що зовсім не закликає до розриву політичного і економічного союзу з
радянською Росією. Однак дискусія набрала політичного звучання, тож
культурологічні проблеми вже не бралися опонентами до уваги. Був підданий
критиці радянською владою. У листі «Тов. Кагановичу та іншим членам ПБ ЦК
ВКП(б)У» від 26 квітня 1926 р. Й.Сталін вказав на виступи М.Хвильового як
прояви поширення антиросійських настроїв в Україні. Лист став сигналом для
гострої критики з боку московського та республіканського керівництва (виступи
та статті Л.Кагановича, А.Хвилі, В.Чубаря, Г.Петровського). Разом з О.Шумським
і М.Волобуєвим був трактований як провідний ідеолог однієї з трьох течій
«націонал-ухильництва» всередині комуністичної партії («хвильовізму»,
«шумськізму» і «волобуєвщини»). Намагаючись врятувати ВАПЛІТЕ від розпуску,
М.Хвильовий в 1926-28 рр. змушений був публічно засудити свої погляди та
відмовитися від них. У 1927 р. працівники ГПУ УСРР заводять справу-формуляр
С-183, починається стеження за діяльністю М.Хвильового. 

У грудні 1927р. – березні 1928 р. М.Хвильовий перебував у Берліні та Відні на
лікуванні й активно знайомився з досягненнями європейської культури, працював у
плані популяризації української літератури за межами України. Під час
перебування в Берліні та Відні М.Хвильовий веде активне листування з
А.Любченком, з якого видно, що німецька культура справила на письменника
надзвичайне враження. У листі А.Любченку від 16 грудня 1927 р. він пише, що
«подивитись є що. Це свідчить хоч би той же Берлін, що в ньому я живу уже
кілька днів». У наступній кореспонденції М.Хвильовий рекомендує А.Любченку, як
і всім «ваплітянам», відвідати Європу, при цьому наголошує, що «в Німеччину
обов'язково треба завітати, а далі вже Італія, Франція…» В одному з листів зі
столиці Австрії від 2 березня 1928 р. М.Хвильовий цікавиться у свого адресата:
«Як справа з перекладами на німецьку мову? За всяку ціну ми мусимо вивести нашу
літературу на широку європейську арену. Словом, треба мужатись – наше
„впереді“».

У січні 1928 р., перед поверненням до України, у листі до газети «Комуніст» він
вимушено засудив своє гасло «Геть від Москви!». Після повернення продовжував
втілювати попередню ідеологічну орієнтацію ВАПЛІТЕ у створених ним журналах
«Літературний ярмарок» (1928-30) та «Пролітфронт» (1930-31). Після закриття
обох журналів пробував писати, дотримуючись «партійної лінії», однак був майже
цілком ізольований від літературного життя радянським режимом.

Загибель. Навесні 1933 р. письменник разом з А.Любченком побував на Полтавщині,
де на власні очі бачив трагедію Голодомору. З цієї подорожі повернувся фізично
й морально розбитим. В атмосфері шаленого цькування, можливо, передчуваючи
наближення тотального терору, після арешту свого приятеля письменника М.Ялового
на знак протесту проти початку масових репресій проти української творчої
інтелігенції 13 травня 1933 р. в Харкові, у будинку письменників «Слово»,
покінчив життя самогубством. Смерть М.Хвильового стала символом краху ідеології
українського націонал-комунізму й кінця українського національного відродження
1920–1930-х рр. Твори та ім'я М.Хвильового залишалися забороненими аж до
останніх років існування тоталітарного режиму в Україні.

Вшанування пам'яті. У Харкові встановлено меморіальну дошка на місці будинку,
де з 1923 р. по 1930 р. жив Микола Хвильовий.  У місті Тростянець Сумської
області у 2003 р. було встановлено меморіальну дошку на будинку, де з 1893 р.
по 1904 р. він проживав.
