% vim: keymap=russian-jcukenwin
%%beginhead 
 
%%file 25_04_2022.fb.kopejka_gennadij.1.gambarov
%%parent 25_04_2022
 
%%url https://www.facebook.com/GennadiiKopeika/posts/5633067206737312
 
%%author_id kopejka_gennadij
%%date 
 
%%tags 
%%title Гамбаров Леонид Арамович (1940-2022)
 
%%endhead 
 
\subsection{Гамбаров Леонид Арамович (1940-2022)}
\label{sec:25_04_2022.fb.kopejka_gennadij.1.gambarov}
 
\Purl{https://www.facebook.com/GennadiiKopeika/posts/5633067206737312}
\ifcmt
 author_begin
   author_id kopejka_gennadij
 author_end
\fi

Сегодня утром, 25 апреля 2022 года, во время обстрела российскими окупантами
города Дергачи (севернее Харькова) погиб наш друг Гамбаров Леонид Арамович -
доктор технических наук, профессор института информатики и управления НАН
Украины, преподаватель ХПИ, Мастер спорта по альпинизму, инструктор–методист
2-й категории.

\ii{25_04_2022.fb.kopejka_gennadij.1.gambarov.pic.1}

До последних дней, несмотря на военное положение, он выполнял свои служебные
обязанности, читал он-лайн лекции. Подробности произошедшего выясняются…

Светлая память!

Гамбаров Леонид Арамович (1940-2022), альпинистская биография.

В 1961 в а/л «Алибек» восхождениями на вв. Семёнов-баши, Сулахат и М. Домбай
сразу выполнил 3-й разряд по альпинизму. В следующие сезоны – участник сборов
Харьковского ДСО «Буревестник» под рук. Мартынова И.А. в а/л «Баксан»,
«Безенги», «Эльбрус».

1966 – Всесоюзная школа инструкторов в а/л «Шхельда». С 1967 по 1989, около 40
смен работает с участниками в а/лагерях «Цей», «Баксан», «Торпедо», «Шхельда»,
«Джайлык», «Домбай». 

Совершил более сотни восхождений, 20 из которых высшей категории сложности:
(1968) п. Вильса, 5А; (1969) п. Щуровского, 5Б; (1970) Джайлык по Сев. ст.,
п/п, 5Б; (1972) Уллу-тау по «ромбу», 5Б; (1974) Джайлык, по Ю ст., 5Б, рук.;
(1976) Гл. Домбай–Ульген, 5Б; (1977) п. Пассионарии, 5Б; (1978) Уилпата, 5Б;
Дубль-пик, 5А; (1981) Уллу–кара, 6А; (1982) п. Вольной Испании, 5Б.

Неоднократно принимал участие в спасательных работах, наиболее сложные: (1972)
«крест» Ушбы (головной отряд), (1980) п. Кавказ (руководитель работ).

В 70-х, 80-х годах судил соревнования по скалолазанию в Крыму: первенства
Харькова, ЦС «Спартака», Украины и Союза.

\ii{25_04_2022.fb.kopejka_gennadij.1.gambarov.cmt}
\ii{25_04_2022.fb.kopejka_gennadij.1.gambarov.cmtx}
