% vim: keymap=russian-jcukenwin
%%beginhead 
 
%%file kiev.citaty
%%parent kiev
 
%%url 
 
%%author 
%%author_id 
%%author_url 
 
%%tags 
%%title 
 
%%endhead 

\def\kv#1{\textbf{\emph{#1}}}

В \kv{Киеве} митинг людей с хорошими лицами против произвола полиции, Владислав
Чечило, 22.05.2021

Неужели, вырастить картофель в пустыне в Израиле, а позже транспортировать в
Украину, оплатить все расходы - выгоднее, чем вырастить ее под \kv{Киевом} или
другим городом Украины? Василий Апасов, 22.05.2021

В Госдуме резко ответили на слова \kv{Киева} об условиях поставок воды в
Крым, 22.05.2021, РИА Новости.

Нацгвардия Украины задержала белоруса с мешками навоза у посольства в
\kv{Киеве}, 22.05.2021

Радянські війська встановили контроль над \kv{Києвом} 6 листопада 1943-го та
вигнали фашистів, ... На початку листопада німецькі окупанти почали палити
\kv{Київ}, ... У ніч на 6 листопада передові частини Червоної армії вступили
у майже порожнє палаюче \kv{Місто}. О 4.00 доповіли про звільнення
\kv{Столиці України} gazeta.ua, 06.11.2020

Очередная акция украинского маразма проявила себя в \kv{Киеве}. topwar.ru,
13.05.2015

На волне военной истерии в \kv{Киеве} с мая 2015 г. работает паб
\enquote{Каратель}, где можно отведать \enquote{ополченцев-гриль},
\enquote{жареных снегирей} и \enquote{вежливых людей}.  Для участников боевых
действий на Донбассе обещают скидки.  https://skeptimist.livejournal.com,
08.08.2015

Украинские СМИ: Взорванный в центре \kv{Киева} каратель ВСУ был личным врагом
Кадырова, spb.tsargrad.tv, 08.09.2017

Макрон выпил в Париже под тост \enquote{Будьмо!} и получил приглашение на бокал
вина в \kv{Киев}, strana.ua, 22.05.2021

В \kv{Киеве} с задержаниями и стычками прошел марш трансгендеров, obozrevatel.com, 22.05.2021

25 квітня 1910 року я вийшла заміж за Гумільова, — писала майбутня поетеса. —
Вінчалися ми за Дніпром, у сільській церкві. Того самого дня Уточкін летів над
\kv{Києвом} і я вперше бачила літак, Анна Ахматова й \kv{Київ},
day.kiev.ua, 25.06.2009.

Незважаючи на попередження \kv{Киян}, садибу Уткіна таки знесли — мерія досі не
прокоментувала інцидент, Жадібність знищить \kv{Київ}, day.kiev.ua, 12.04.2021

\kv{Київ} — місто контрастів. Найстаріша вулиця — Володимирська (понад тисячу
років), а найдовша — Броварський проспект (14 км). Тема \kv{Києва} нескінченна. Це
місто оспівано поетами і письменниками, його малюють та фотографують. Для
когось це місто дитинства, для когось — столиця України, а хтось захоплюється
його давньою історією...  Кожний бачить \kv{Київ} по-своєму, 
«Адреса: \kv{Київ}», day.kiev.ua, 19.06.2018

На виставці можна побачити \kv{Київ} таким, яким бачать і відчувають його митці
— Олена Яблонська, Олександр Павлов, Ганна Файнерман, Ернест Котков, Олександр
Найден, Олексій Орябинський, Зоя Орлова, Віктор Козик, Яким Левич, Владислав
Шерешевський, Олег Животков, Юрій Соломко, Любов Рапопорт...  «Адреса:
\kv{Київ}», day.kiev.ua, 19.06.2018

Вечірній \kv{Київ} - неймовірно прекрасний, 21.06.2017, day.kiev.ua

Выходит, что Булгаков – самый актуальный \kv{киевский} писатель, Олесь
Бузина, 03.05.2014

\kv{Киевские} силовики разместили бронетехнику вблизи жилья в Счастье – Народная
милиция, lug-info.com, 22.05.2021

Я ненавиджу комуністів, але при них \kv{Київ} був красивіший, - Дмитрий Гордон

Я кричала: «Ненавиджу тебе, \kv{Київ}!» і мені намагались допомогти –
донеччанка, radiosvoboda.org, 2015

Колишній народний депутат Ірина Фаріон назвала \enquote{зрадниками} і
\enquote{перевертнями} громадян України, які говорять російською мовою. Про це
вона заявила 8 травня в \kv{Києві} на форумі \enquote{За українську мову},
gordonua.com, 14.05.2018

Скандальна Фаріон розгромила ялинку в \kv{Києві}: \enquote{Презерватив замість
зірки}, kyiv.znaj.ua, 15.12.2020,

Натомість у Почаївський та \kv{Києво}-Печерській лаврах побачите перестарілих
зрусифікованих штурпаків з немитими бородами, які відверто можуть послати вас
на три веселі букви разом з вашою ``нєзалєжнай нєнькай'', Юрій Винничук,
Малоросійський мазохізм, 18.07.2011

Якби Богдан уклав Переяславську угоду з Туреччиною, то це б нам не загрожувало
ані мечетями в Бердичеві, ані гаремами в \kv{Києві}. Православна віра вчинила з
нас московських рабів, Юрій Винничук, Малоросійський мазохізм, 18.07.2011

Сайт Майдан був створений друзями Гонгадзе, програмістами та активістами
наметового містечка на Майдані Незалежності в \kv{Києві}, maidan.org.ua

\enquote{ПДР для нього не писані}. У \kv{Києві} велосипедист рухався
швидкісною смугою жвавої магістралі, gordonua.com, 17.05.2021

\kv{Киевлян} настойчиво готовят к подорожанию проезда в метро, Алексей
Гавриленко, 23.05.2021

Обмен удерживаемыми лицами также блокируется \kv{Киевом}, представитель
которого продолжает рассказывать анекдоты вместо ответов на вопросы
координатора о ситуации с \enquote{процессуальной очисткой}, Андрей Марочко,
lug-info.com, 23.05.2021

Страдают юные жители Донбасса не только в результате обстрелов территорий
Республик. Угрозу им несут и \enquote{освободители}, орудующие на
подконтрольной \kv{Киеву} территории. Только на этой неделе стало известно о
двух таких случаях, Андрей Марочко, lug-info.com, 23.05.2021

\kv{Киевские} силовики два раза за сутки нарушили \enquote{режим тишины},
lug-info.com, 23.05.2021

О серии обысков у чиновников \kv{КГГА}: ``За обыски у главного архитектора —
плюсую, эту собаку вообще судить надо за то, что он позволил сделать с внешним
видом \kv{Столицы}'', Макс Назаров, focus.ua, 2021

Вшанування Дня Героїв у \kv{Києві} – фоторепортаж, radiosvoboda.org, 23.05.2021

23 травня відзначають день пам'яті та вшанування всіх українців, які присвятили
життя боротьбі за свою країну. Заходи з нагоди Дня Героїв пройшли сьогодні в
\emph{Києві}, radiosvoboda.org, 23.05.2021

\enquote{Общерусскої} історії не може бути, як нема \enquote{общерусскої}
народності... Ми знаємо, що \kv{Київська} держава, право, культура були утвором
одної народності, українсько-руської, Володимиро-Московська – другої,
великоруської, Михаил Грушевский, \textbf{«Звичайна схема «русскої» історії й справа
раціонального укладу історії східного слов'янства»}, 1904

А не з «общерусской» колиски братніх народів – «\kv{Кієвской} Русі» чи
«Дрєвнєрусского государства» – як про це бреше Москва і в чому їй бездумно
вторує частина українського інформаційного простору, radiosvoboda.org,
23.05.2021

Дуже цікава особистість. Близький до Івана Франка, до Лесі Українки. Вона
називала його «Фотіком». У садибі Фотія у \kv{Києві} є дерева, посаджені й
Франком, і Лесею. Це племінчатий онук Шевченка, \textbf{У двох сестер і двох братів
Тараса Шевченка було 90 внуків – родичка Кобзаря}, radiosvoboda.org, 22.05.2021 

У \kv{Києві} пройшла акція на підтримку затриманого в Мінську опозиціонера,
pravda.com.ua, 23.05.2021

У \kv{Києві} за добу підтвердили 73 випадки коронавірусу, pravda.com.ua,
23.05.2021

Не так давно на Инвестиционном форуме в \kv{Киеве} мэр \kv{Столицы} Виталий
Кличко заявил, что готов работать едва ли не бодигардом для каждого инвестора
\kv{Города}, Сергій Чекальський, pravda.com.ua, 22.05.2021

Теми методами, которыми пользуются наши градоначальники, \kv{Киев} еще долго
останется токсичной средой для прихода международных корпораций, Сергій
Чекальський, pravda.com.ua, 22.05.2021

Я каждый день езжу разбитыми дорогами \kv{Киева}, застреваю в пробках, теряю
мотивацию из-за отсутствия элементарных коммунальных удобств и при этом не
понимаю, что именно зависит от меня и на что я лично могу повлиять, Сергій
Чекальський, pravda.com.ua, 22.05.2021

В \kv{Киеве} показали идеальную маршрутку, sharij.net, 21.05.2021

Во время марша ЛГБТ-сообщества, что проходил в субботу в \kv{Киеве},
представители ультраправой организации «Традиция и порядок» попытались помешать
мероприятию, sharij.net, 22.05.2021

Монумент жертвам украинских нацистов УПА напоминает крымчанам о преступлениях
\kv{Киева}, riafan.ru, 08.05.2021

А на Украине рушатся мосты, и текущей по улице в центре \kv{Киева} канализацией
– не удивить никого, \textbf{Украинская правда}, voskhod.info, 19.05.2021

Могу расшифровать: это касается всей Большой Новороссии с \kv{Киевом} в
придачу, \textbf{Украинская правда}, voskhod.info, 19.05.2021

Найбільший хмарочос \kv{Києва} таки збудують?, day.kiev.ua, 20.05.2021

В 1921 году семья переехала в \kv{Киев}. Благодаря уникальным математическим
способностям юный Николай Боголюбов уже в 14 лет стал полноправным участником
семинаров по математике, www.biblioatom.ru

В \kv{Киев} Боголюбов с семьей вернулся в 1944 году. Он вновь работал в
Институте теоретической физики и в \kv{Киевском} университете в качестве
заведующего кафедрой математической физики (1944—1950 гг.), www.biblioatom.ru

В ДТП на трассе \kv{Киев} - Одесса погибли два человека. Фото, strana.ua, 22.05.2021

Дебошира доставили в местное управление полиции. Оказалось, что ему 21 год, он
проживает в \kv{Киеве}, strana.ua, 24.05.2021

На \kv{Крещатике} неадекватный прохожий наплевал на патрульных и разбил
стекло служебной машины, strana.ua, 24.05.2021

В начале мая на борту самолета, летевшего из \kv{Киева} в Доминикану,
произошел конфликт между туристами, obozrevatel.com, 24.05.2021

ПЦУ, як правонаступниця \kv{Київської} митрополії, має повне право на отримання у
користування культових споруд \kv{Києво}-Печерського заповідника, radiosvoboda.org, 24.05.2021

\kv{Києво}-Печерська лавра – православний монастирський комплекс у \kv{Києві}. Визначна
пам'ятка історії та архітектури. Заснований у період України-Русі в 1051 році
як печерний монастир за межами \kv{Києва}, radiosvoboda.org, 24.05.2021

У грудні 2013 Липовой приїхав у \kv{Київ} на автівці, взявши із собою
мисливський карабін \enquote{Сайга М3}, на який мав офіційний дозвіл ще з 2003
року, та близько 60 патронів, Між двох вогнів. \textbf{Як розслідують вбивства
силовиків на Майдані}, pravda.com.ua, 24.05.2021

В \emph{Киеве} мужчина пытался закидать навозом посольство Беларуси. Ему
помешали нацгвардейцы. Видео, strana.ua, 22.05.2021

Украина в 2016 году также вернула в \emph{Киев} самолёт белорусских авиалиний,
угрожая поднять военную авиацию. Типа там на борту важный преступник, -
\textbf{У пилота были международные правила и приказ садиться}, Дмитрий Раимов,
strana.ua, 24.05.2021

Евросоюз может сегодня прекратить полеты над Беларусью. Рейсы из \emph{Киева} уже
облетают страну, strana.ua, 24.05.2021

Как сообщалось, 15 декабря, на Шулявском мосту в \emph{Киеве} обрушились три фонарных
столба. При этом они повредили девять автомобилей, но среди людей обошлось без
пострадавших. Момент падения трех столбов на Шулявский мост в \emph{Киеве} попал на
видео, strana.ua, 24.05.2021

В Голосеевском районе \emph{Киева} прорвало трубу, огромный фонтан повредил
почти два десятка авто. Фото и видео, strana.ua, 20.05.2021

Тож яку підтримку \emph{Київ} отримав після початку війни на Донбасі і на що
може розраховувати найближчим часом? 
radiosvoboda.org, 24.05.2021

Навіть попри те, що Путін не почав нову фазу завоювань в Україні – останній
епізод кілька тижнів тому змусив людей у Вашингтоні і \emph{Києві} понервувати,
radiosvoboda.org, 24.05.2021

\emph{Киевские} силовики разместили бронетехнику в районе Орехово – Народная
милиция, lug-info.com, 24.05.2021

«Слуга народа» объяснила, как получила квартиру в \emph{Киеве} за 165 тысяч
гривен, 24.05.2021 

\enquote{Работаем, затянув пояса}. В \emph{Киевском} метрополитене объяснили
требование повысить стоимость проезда, strana.ua, 13.05.2021

Найденный полицией блуждающим по \emph{Киевскому} метро 8-летний мальчик не
ходит в школу - омбудсмен, strana.ua, 24.05.2021

Нехорошая квартира Марьяны Безуглой. Как нардеп вчетверо занизила стоимость
своего жилья в \emph{Киеве}, strana.ua, 24.05.2021

У Міністерстві культури України відсутні відомості про погодження
науково-проєктної документації з демонтажу радянського герба на монументі
«Батьківщина-мати» у \emph{Києві} та надання дозволу на проведення робіт. Про це
повідомили у Мінкульті у відповідь на запит Радіо Свобода, radiosvoboda.org, 24.05.2021.


Шановні туристи, ми не заперечуємо вагомості внеску цього \emph{Київського}
духовного центру до скарбниці всіх східнослов'янських культур, але наголошуємо,
що Лавра – це передусім феномен української культури».
radiosvoboda.org, 06.05.2017

Ця традиція іде ще від апостола Андрія – дві тисячі років тому він першим
провістив сакральну природу \emph{Києва}. У ХІ столітті у знаменитій промові
«Слово про закон і благодать» перший руський митрополит Іларіон возвеличив
\emph{Київ} як «город святий всеславний», radiosvoboda.org, 09.10.2018

\emph{Київські} князі розмовляли українською. Музейники це замовчують,
radiosvoboda.org, 22.03.2017

В древности на территории \emph{Киевской} Руси было принято сжигать дома, если жилье
обволакивала \enquote{черная плесень}. Наши предки справедливо считали его настоящим
ядом, который способен убивать жителей, strana.ua, 24.05.2021

Мені соромно, що в \emph{Києві} гетьмана досі не вшановано. І це не є питання грошей,
це скоріше питання серця, umoloda.kyiv.ua, 25.05.2021

День в истории. 24 мая: в знаменитой \emph{киевской} тюрьме родился прадедушка
«главврача Майдана», Дмитрий Губин, ukraina.ru, 24.05.2021

Весной 1944 года Александр Богомолец вернулся в \emph{Киев}, где возглавил работы по
воссозданию АН УССР. Несмотря на застарелый туберкулёз, академик много курил.
Его здоровье было подорвано, и 19 июля 1946 года его не стало. Похоронен
Александр Богомолец в \emph{киевском} парке, посаженном им и его учениками, у дома,
где он жил, ukraina.ru, 24.05.2021

Однако в начале XXI века символ \emph{Киева} постигло несчастье — из-за
возросшей загазованности воздуха, реагентов, которыми посыпали дороги зимой, а
также каштановой моли эти деревья начали погибать, 
\textbf{Киев до и после. Украинская столица отметит грустный День рождения}, Георгий Лучников, ukraina.ru, 25.05.2021

И компания Порошенко очень ревниво относится к тому, что кто-то ещё использует
название «\emph{Киевский} торт» для своей продукции, как бы та ни
соответствовала созданному в СССР рецепту,
\textbf{Киев до и после. Украинская столица отметит грустный День рождения}, Георгий Лучников, ukraina.ru, 25.05.2021

Но ещё раньше — начиная с 2014 года — многие любители \emph{Киевского} торта из
числа противников и ненавистников Порошенко стали отказываться от этого
лакомства. У них оно прочно ассоциировалось с человеком, политика которого у
них вызывает отторжение. И из сладкого символа города на Днепре
\emph{«Киевский»} стал символом жадного и беспринципного человека, 
\textbf{Киев до и после. Украинская столица отметит грустный День рождения}, Георгий Лучников, ukraina.ru, 25.05.2021

Корпорация «Рошен» является собственником серии торговых марок «\emph{Киевский}
торт» как словесных, так и комбинированных, 
\textbf{Киев до и после. Украинская столица отметит грустный День рождения}, Георгий Лучников, ukraina.ru, 25.05.2021

«Видел много прекрасных парков в городах, но город в парке вижу впервые», —
заявил в 1966 году во время визита в \emph{Киев} президент Франции, лидер
Сопротивления Шарль де Голль,
\textbf{Киев до и после. Украинская столица отметит грустный День рождения}, Георгий Лучников, ukraina.ru, 25.05.2021

В общем, тот \emph{Киев}, который воспевали и которым восхищались поэты,
писатели и общественные деятели не только СССР, но и мира, тот \emph{Киев},
который мы знали и любим, на протяжении веков и десятилетий, потихоньку уходит
в небытие, 
\textbf{Киев до и после. Украинская столица отметит грустный День рождения}, Георгий Лучников, ukraina.ru, 25.05.2021

Однако теперь \emph{Киев} всё больше напоминает бетонные джунгли. Дома строят на тех
самых зелёных днепровских склонах, которые воспевали поэты и певцы,
\textbf{Киев до и после. Украинская столица отметит грустный День рождения}, Георгий Лучников, ukraina.ru, 25.05.2021

Цветы каштана изображены на \emph{«Киевском»} торте. Этот торт — ещё одна визитная
карточка \emph{Киева}, \textbf{Киев до и после. Украинская столица отметит грустный
День рождения}, Георгий Лучников, ukraina.ru, 25.05.2021

Каштан появился на изобретенном в 1956 году «\emph{Киевском} торте». А в 1969 году каштан появился на гербе \emph{Киева},
\textbf{Киев до и после. Украинская столица отметит грустный День рождения}, Георгий Лучников, ukraina.ru, 25.05.2021

Хотя он вылетал из \emph{Киева} и его ВЫПУСТИЛИ, после чего, уже в небе,
развернули самолёт обратно, \textbf{Никаких норм международного права нет - их
уничтожил Запад}, Александр Скубченко, strana.ua, 25.05.2021

В \emph{Киеве} заметили неуправляемую баржу, которая свободно дрейфует вниз по
течению Днепра. Видео, strana.ua, 25.05.2021

\emph{Киев} возглавил список регионов по заболеваемости COVID-19: 402 новых
случая, obozrevatel.com, 25.05.2021

Коли був флешмоб \emph{\#KyivNotKiev}, в коментарі завжди приходили росіяни,
котрі казали: \enquote{Альо, пацани, шо ви собі понапридумували, ну який
\emph{Київ}}. 

Як ідеться в історичних джерелах на сайтах столичних екскурсоводів, походження
назви пов'язують із легендою про перебування у \emph{Києві} російської
імператриці Катерини Другої, яка, проїжджаючи 1787 цією місцевістю, нібито,
сказала князю Потьомкіну: «Кинь грусть!», \textbf{Що таке \enquote{Кинь-Ґрусть}
і чому Киянам варто там побувати}, bigkyiv.com.ua, 22.05.2021

В центре \emph{Киева} под ногами у прохожих нашли килограммы взрывчатки,
гранаты и автомат Калашникова, strana.ua, 25.05.2021

Недавно \enquote{Страна} описывала дикий случай, имевший место на бульваре
Дружбы Народов в Печерском районе \emph{Столицы}, когда ранее судимый местный
житель, угрожал взорвать управляющую компанию \enquote{Печерскжитло} вместе с
сотрудниками, strana.ua, 11.04.2021

Велопаломники УПЦ с 1 по 4 июня преодолеют 450 километров из \emph{Киева} к
Почаевской лавре, fraza.com, 25.05.2021

Записки \emph{Киевлянки}: людей накручивают - «26 мая затмение, вот тогда
Россия и нападет! - kp.ru, 15.04.2021

То, что происходит на Востоке, \emph{Киевское} правительство, пришедшее к
власти на волне Майдана, прикрыло фиговым листом аббревиатуры АТО —
антитеррористическая операция, Олесь Бузина, www.segodnya.ua, 02.06.2014

Проєкт у 15 років і 10 томів: у Софії \emph{Київській} презентують видання про
графіті, bigkyiv.com.ua, 25.05.2021

Войска хана Батыя разрушили почти все древнерусские города, но они вскоре
восстали из пепла. И только столица Руси — \emph{Киев} — пришла в полный
упадок.  Почему? Олесь Бузина, segodnya.ua, 14.06.2014

\emph{Киевлянка} заставила попрошайничать малолетнюю дочь. Теперь ей грозит
3-летний срок, strana.ua, 25.05.2021

\emph{Киевские} попрошайки получают в неделю 3 тысячи гривен, их крышуют
полицейские - расследование, strana.ua, 12.03.2019

\enquote{Сплошной Тамбов}. Ницой возмутилась количеству русскоговорящих в
\emph{Киеве}, rian.com.ua, 26.02.2018

Тобто через \emph{Київ} ми утверджуємо на цілий світ свою ідентичність, і
кажемо: \enquote{Господи милосердний, Україна – це не Раша}, Ірина Фаріон,
obozrevatel.com, 16.06.2019

Кодекс \emph{киевлянина} представили Олег Скрипка и Андрей Курков, gazeta.ua,
22.09.2010

КОРОЛЕВА ЕЛИЗАВЕТА - ПРАВНУЧКА АННЫ ЯРОСЛАВНЫ.  Этих двух женщин разделяют
тридцать поколений и 900 лет, но они родственницы.  Королева Елизавета II
является правнучкой Анны \emph{Киевской} в 30-м поколении, Владимир Федорович,
facebook.com, 25.05.2021

Обращение упоротого \emph{киевлянина}, 10.04.2014, \url{https://www.youtube.com/watch?v=rnAaQB2BTzA}

В \emph{Киеве} закрыли два ресторана из-за несоблюдения карантинных мер. В одном из
них - ресторане ВАО в декабре 2017 года сняли видео, как нынешний замглавы МВД
Антон Геращенко вылизывает тарелку, strana.ua, 02.04.2021

\emph{Киевляне} вышли на первое место по скупке жилья в Москве, strana.ua, 25.05.2021

На території Музею у \emph{Києві} посадили Лесину грушу, volyn.com.ua, 22.05.2021

З цими словами український мент скрутив на асфальті білоруського активіста.
Тут і зараз - 22 травня 2021 року посеред \emph{Києва}. Посеред \emph{Києва}, який досі
знаходиться \enquote{на Украине} - на території, де москвороті мусора безкарно пакують
нас та наших білоруських союзників, Євген Дикий, gazeta.ua, 23.05.2021

Геращенко обвинил в нападении с битой на автомобиль накачанного
\emph{Киевлянина}. Тот говорит - это был не он, strana.ua, 25.05.2021

В \emph{Киеве} две женщины путем обмана под предлогом гадания и снятия порчи
отбирали сбережения у столичных пенсионеров, strana.ua, 25.05.2021

По его словам, в \emph{Киеве} ни в одном пункте вакцинации сейчас нет вакцин,
strana.ua, 25.05.2021

Нигде нет вакцин. Я знаю многих главврачей в \emph{Киеве} - вакцины от
коронавируса закончились, strana.ua, 25.05.2021

Облетая гнездо Бацьки. Как теперь долететь из \emph{Киева} в Москву и вернет ли
\enquote{Белавиа} деньги за билеты, strana.ua, 25.05.2021

Что думают \emph{Киевляне} о запрете полетов в Беларусь. Опрос
\enquote{Страны}, Антонина Белоглазова, strana.ua, 26.05.2021

А кстати, \enquote{ветерана}, который в \emph{Киеве} взорваться обещал на
мосту, посадили?  Игорь Правлоцкий, комментарий к видео, \textbf{Посадки
самолётов от Порошенко до Лукашенко. Что нужно понимать по скандалу с
задержанием Протасевича}, Олеся Медведева, strana.ua, 24.05.2021

О том, что Роман не просто наблюдал, говорит и эта фотография. Как
утверждается, она сделана у разбитого памятника Ленину на Бессарабке в
\emph{Киеве}.  Протасевич на ней улыбается, показывает пальцами знак победы (V)
и обернут в красно-белый флаг белорусской оппозиции, \textbf{Черное солнце
Романа Протасевича. Воевал ли экс-главред \enquote{Нехты} в \enquote{Азове}},
strana.ua, 26.05.2021

У \emph{Києві} презентували книгу спогадів про журналістку Радіо Свобода
Богдану Костюк, radiosvoboda.org, 26.05.2021

\enquote{Хотел убить всех. Надоели люди}. \emph{Киевлянин} рассказал, почему бросил бутылку с
бензином в троллейбус, strana.ua, 26.05.2021

Найденный полицией блуждающим по \emph{киевскому} метро 8-летний мальчик не
ходит в школу - омбудсмен, strana.ua, 24.05.2021

И тогда очищение власти произойдёт очень быстро, за несколько лет. Избираемые
будут ориентироваться на избирателя, а не на друг друга, и не на начальство в
\emph{Киеве}, \textbf{Не нужно никого лишать избирательного права}, Павел Себастьянович,
strana.ua, 27.05.2021

Прошло лишь полгода с тех пор, как официальный \emph{Киев} поддержал очередной Минский
майдан, а потом присоединился к числу стран, не признающих победу Лукашенко на
последних президентских выборах, \textbf{Нужен ли Украине «белорусский фронт»?}, Виктор
Дяченко, from-ua.com, 26.05.2021

Использование автомобиля в \emph{Киеве} должно стать невыгодным, Анатолий
Амелин, strana.ua, 27.05.2021

А через три года \emph{Киев} может начать менять общественные пространства
развивать велодорожки (которыми пользуются и самокатчики) и озеленять
освободившиеся пространства, \textbf{Использование автомобиля в \emph{Киеве}
должно стать невыгодным}, Анатолий Амелин, strana.ua, 27.05.2021

В порядке бреда как избавить \emph{Киев} от пробок.  \emph{Город} опять встал
из-за ремонтов. Куча рабочего времени просто теряется.  \emph{Город} в сизой
дымке выхлопных газов, \textbf{Использование автомобиля в \emph{Киеве} должно
стать невыгодным}, Анатолий Амелин, strana.ua, 27.05.2021

Убивший ранее жену \emph{киевлянин} перерезал горло матери и спрятал труп в
диван, на котором ел и спал, strana.ua, 22.12.2020

\enquote{Не может пройти 30 метров}. На вокзале \emph{Киева} ради удобства
Порошенко перегнали поезд на другую колею, strana.ua, 27.05.2021

Масштабна аварія перетворила вулицю \emph{Києва} на річку: відео НП, gazeta.ua,
27.05.2021

Через місяць каштани розцвіли й після перевірки виявилося, що дерева є
звичайними кінськими каштанами. Експертиза показала, що серед висаджених дерев
лише два каштани відповідають сорту Бріотті, 11 - сорт Аескупус Глобра, 18
дерев - сорт Баумана, які ніколи не висаджувалися на вулицях \emph{Києва}, і
157 - звичайні кінські каштани, \textbf{Звільнили через каштани: на \emph{Хрещатику}
висадили фальшиві дерева}, gazeta.ua, 27.05.2021

Возле ТЦ Ocean Plaza в \emph{Киеве} из-за порыва трубы затопило проезжую часть. Фото,
strana.ua, 27.05.2021

Порыв трубы на Шота Руставели в \emph{Киеве}. Как центр \emph{Столицы} залило
кипятком, strana.ua, 05.11.2019

Спустя 8 месяцев полиция так и не выяснила, почему погибла сотрудница
посольства США в \emph{Киеве}, strana.ua, 27.05.2021

Этого не ждал никто. Никому бы в голову не пришло, что настанет тот день, когда
кто-нибудь в \emph{Киеве} отдаст приказ, и боевая авиация появится в небе
Донецка убивая живущих в нем. Люди не верили своим глазам,
\url{www.facebook.com/roman.solodovchuk.3939}, 27.05.2021

Интрига продержалась недолго. Выяснилось, что Порошенко уехал в Донецкую
область,  воспользовавшись общим вагоном \enquote{Интерсити}
\enquote{\emph{Киев}-Константиновка}, который отправляется из столицы в 06:10,
strana.ua, 27.05.2021

\enquote{Что за дичь вообще?}. Появилось видео, как в \emph{Киеве} женщина
гуляла по проезжей части с лошадью на цепи, strana.ua, 27.05.2021

В \emph{Киеве} мужчина в одних трусах верхом на лошади заехал в супермаркет за пивом.
Видео, strana.ua, 29.06.2020

Как анонсировали в SETTERS, пятеро рекламщиков из Санкт-Петербурга уже поехали
в \emph{Киев}, чтобы \enquote{пилить сторис с Крещатика и вместе работать над
проектами}, \textbf{Популярные украинские рекламщики оскандалились
\enquote{дружбой} с компанией РФ, вывесившей флаг \enquote{Новороссии}},
obozrevatel.com, 27.05.2021

Честно говоря, мне тоже было немного не по себе, когда получил предложение
сыграть Воланда. Все вокруг говорили: «Мистика! Бог вас покарает!» Тогда я как
раз был в \emph{Киеве} на гастролях, вот и решил пойти в только что восстановленный
храм в \emph{Киево}-Печерской лавре попросить у Бога разрешения сниматься в «Мастере и
Маргарите», - Олег Басилашвили о роли Воланда в фильме  \enquote{Мастер и Маргарита.}

Про це у своєму Telegram-каналі повідомив активіст Сергій Стерненко.
\enquote{У \emph{Києві} у Гідропарку відкрили ще одне кафе імені Григорія
Лєпса.  Російський співак відкрито підтримав окупацію Криму й агресивну
політику Володимира Путіна} – написав він, прикріпивши відповідні фото та
скріншоти, kyiv.media, 27.05.2021

В понедельник, 28 сентября, Печерский райсуд \emph{Киева} должен был
рассмотреть постановление патрульной полиции, составленное за езду в нетрезвом
виде на мотоцикле в отношении известного музыканта и лидера группы
\enquote{Вопли Видоплясова} Олега Скрипки. Однако заседание так и не
состоялось, strana.ua, 28.09.2020

Відомий лікар, доктор медицини Йоганн Лерхе у 1770-71 роках перебував тут,
надаючи місцевій владі допомогу в боротьбі з чумою. За результатами тієї
робочої поїздки до \emph{Києва} він залишив занотовані спостереження не лише
медичного характеру, а й щодо особливостей місцевого клімату.  \enquote{З боку
Дніпра \emph{Київ} являє собою прекрасну картину; безліч церков, що вінчають
височини, роблять ландшафт цей чарівним. Майже біля кожного будинку є сад з
вишнями, сливами і горіховими деревами. 18 квітня почали дерева цвісти; але
через три дні випав сніг, який дуже багато пошкодив}, – пише Лерхе, \textbf{Київські
весняні градуси}, bigkyiv.com.ua, 06.05.2021

Регулярні метеоспостереження у \emph{Києві} ведуться з 1812 року. Цю справу
започаткував випускник \emph{Києво}-Могилянської академії, а згодом і викладач
географії та історії \emph{Київської} першої гімназії Максим Берлінський,
\textbf{\emph{Київські} весняні градуси}, bigkyiv.com.ua, 06.05.2021

Ось як ці події згадує \emph{киянин} Михайло Ясинський: «Ніч з 8-го на 9-те
липня була дуже тривожна… Почали палити різні документи, над \emph{Києвом}
стояв дим, в повітрі носились клапті перегорілого та недогорілого паперу, по
вулицях і тротуарах вітер носив ті ж чорні клапті. Палили партійні, професійні
матеріали, канцелярські, бухгалтерські. Накази, персональні справи.  Нищення
документів тривало днів три-чотири…», \textbf{Місто-герой: якою ціною? Історія
в цифрах}, bigkyiv.com.ua, 08.05.2021

8 травня 1965 року \emph{Київ} отримав звання «Місто-герой». Так українська
столиця ввійшла у список, що складається з 12 міст і однієї фортеці, які
проявили найбільшу мужність у Другій світовій війні. І дійсно на долю
\emph{Києва} випала неймовірна кількість страждань, які треба було подолати
заради життя, заради майбутнього. Щоб оцінити масштаб подій, варто
проаналізувати деякі цифри, які є красномовнішими, ніж слова...
\textbf{Місто-герой: якою ціною? Історія в цифрах}, bigkyiv.com.ua, 08.05.2021

В \emph{Киеве} агрессивный мужчина средь бела дня крушил банкомат, прыгал по
машинам и нападал на прохожих. Видео, strana.ua, 28.05.2021

Девушка-блогер в \emph{Киеве} за один день устроила стрельбу на улице и
разгромила съемную квартиру. Видео, strana.ua, 05.03.2021

В \emph{Киеве} пьяный мужчина стрелял из окна квартиры по прохожим, его жилье
штурмовал спецназ. Фото, strana.ua, 28.05.2021

\enquote{У меня был поезд в 2:00 ночи из Днепра в \emph{Киев}, я должна была ехать на передачу
к Леше Дурневу \enquote{Дурнев ищет парня}. И вот, мне уже пора было уезжать с дачи. Я
спросила у Валеры, даст ли он мне денег на такси, он сказал \enquote{не дам}. А, нет,
не так было. Я подошла к телефону, а он мне говорит: \enquote{Нет, ты поедешь через
час}. Говорю, - мне пора домой. Уже было около девяти вечера. Я ответила, что
поеду за свои деньги. Он начал возмущаться. И в этот момент я заматерилась,
сказала: \enquote{Короче, е**ла я в рот я такое}. Он начал говорить мне, что я
охренела и начал подходить ближе. Я уже забыла в этот момент, что мне нужно
звонить в такси. Я машинально позвонила папе и начала просто кричать в трубку:
\enquote{Пожалуйста, помоги, меня хотят избить!}. В этот момент Валера просто
выхватывает у меня телефон и начинает бить меня по голове}, - рассказывает
Чуприна, \textbf{Участница скандальной ню-фотосессии в Дубае заявила об
изнасиловании. Ее обвинили в вымогательстве}, strana.ua, 28.05.2021

Валерий утверждает, что с помощью заявления в полицию об изнасиловании, которое
Наталья подала уже в \emph{Киеве}, девушка просто пытается заработать. По
словам мужчины, она вымогала у него деньги в обмен на отзыв заявления,
\textbf{Участница скандальной ню-фотосессии в Дубае заявила об изнасиловании.
Ее обвинили в вымогательстве}, strana.ua, 28.05.2021

П'ять годин із Лукашенком. Пакет санкцій готує і \emph{Київ}, radiosvoboda.org, 28.05.2021

Кінець мінського формату: що означає заява \emph{Києва} про зміну переговорів по
Донбасу, pravda.com.ua, 28.05.2021

\emph{Киев} настолько рьяно пустился присваивать себе имущество Автономной
области Крым и в частности имущество Черноморского флота, что в Симферополе
решили, вполне в духе того времени, воспользоваться подобным Ельцинским
предложением, \textbf{Аннексия на аннексию?} Дмитрий Жук, zen.yandex.ru,
05.05.2021

Если Кучма пошёл на аннексию (а я считаю это аннексией) суверенного полуострова
с молчаливого согласия российского президента, то о какой такой аннексии,
начиная с 2014 года, ноет \emph{Киев}, если всё равно вынашивались планы
передачи полуострова в экстерриториальные владения США после избавления от
российского Черноморского Флота? \textbf{Аннексия на аннексию?} Дмитрий Жук,
zen.yandex.ru, 05.05.2021

Трижды ударил и плюнул в лицо. Как пьяный соратник Олега Ляшко избил
полицейского в \emph{Киеве}, strana.ua, 19.05.2021

Столица отмечает день рождения. За свою тысячелетнюю историю город повидал
немало. В разное время \emph{Киевом} правили Рюриковичи, татары, литовцы, поляки,
советы. Но все равно город остался истинно украинским. Шарм улиц, уют
двориков и даже царь-балконы делают \emph{Киев} неповторимым. Богатая история оставила
для нас многочисленные памятники архитектуры, культуры и природы, Василий
Михальчук, obozrevatel.com, 29.05.2021

Михаил Булгаков писал \enquote{Эх, \emph{Киев}-город! Красота!.. Вот так Лавра
пылает на горах, а Днепро, неописуемый свет! Травы! Сеном пахнет! Склоны!
Долы!..}. Этот невероятный образ сохранился до наших времен. Отправляемся на
прогулку в прошлое и сравниваем с нынешними временами, Василий Михальчук,
obozrevatel.com, 29.05.2021

В разных городах мира прошли акции солидарности с оппозицией Беларуси. В \emph{Киеве}
на нее пришли 70 человек. Фото, strana.ua, 29.05.2021

Моя остання відрада. Як знищення зелених зон \emph{Києва} впливає на адаптацію міста
до зміни клімату, Оксана Расулова, lb.ua, 27.05.2021

\enquote{Не хочу вмерти. Ковід забрав близьку людину} - як у \emph{Києві}
вперше масово кололи Coronavac, gazeta.ua, 29.05.2021

27 (16 по ст. ст.) мая 1753 года в \emph{Киеве} было завершено строительство
Андреевской церкви. Строилась по проекту архитектора Бартоломео Растрелли на
Андреевской горе в память посещения \emph{Киева} императрицей России Елизаветой
Петровной. В 1990-е годы храмом завладела неканоническая УАПЦ. В октябре 2018
года Верховная рада приняла закон «Об особенностях пользования Андреевской
церковью Национального заповедника \enquote{София \emph{Киевская}}», согласно
которому Андреевская церковь передана в постоянное и безвозмездное пользование
Константинопольскому патриархату, 27 мая – день в нашей истории,
odnarodyna.org, 27.05.2021

27 мая 2014 года нацбаты «Азов» и «Днепр» проводили очередную «зачистку» в
Мариуполе и штурмовали городской штаб ДНР.  Перестрелки шли в центре города на
улицах Греческая, Георгиевская, на пересечении улиц Ленина и Торговой, а также
в районе гостиницы «Спартак». На следующий день Мариуполь контролировали
военные группировки \emph{Киева}, odnarodyna.org, 27.05.2021

Сегодня у \emph{Киева} день рождения! Нашей столице исполнилось целых 1539 лет.
У \emph{Киева} есть очень много достопримечательностей, но главная ценность
города - очень много хороших людей, которые в нем живут. Есть вам
посчастливилось найти здесь близких по духу людей, они навсегда останутся
вашими друзьями, strana.ua, 30.05.2021

Редакция \enquote{Страны} поздравляет \emph{Киевлян} с днем города. Хотим вам
пожелать: даже если вы живете в \emph{Киеве} с самого рождения или очень давно
- не переставайте исследовать и изучать \emph{Город}. Каждый раз вы будете находить
здесь что-то особенное, открывать места, которые в итоге будут становиться
вашими любимыми, strana.ua, 30.05.2021

Впервые День рождения \emph{Киева} отпраздновали в 1982 году, когда городу
исполнилось 1500 лет, strana.ua, 30.05.2021

В \emph{Киеве} самое старое подземное метро в Европе с самой глубокой станцией
- Арсенальной (105 м), strana.ua, 30.05.2021

Взрыв в метро и захват заложников. Рассказываем, как в \emph{Киеве} провели
антитеррористические учения. Фото, видео, obozrevatel.com, 28.05.2021

\enquote{Спасатели сразу приступили к тушению пожара, вызванного взрывом.
Вместе с тем, они провели эвакуацию пострадавших и установили телескопические
опоры, чтобы не допустить обрушения конструкций станции метрополитена. В этом
им помогли спасатели \emph{Киевской} аварийно-спасательной службы}, –
рассказали в пресс-службе, \textbf{Взрыв в метро и захват заложников.
Рассказываем, как в \emph{Киеве} провели антитеррористические учения. Фото,
видео}, obozrevatel.com, 28.05.2021

Двадцять п'ять років тому в Україні з'явилися нові купюри. На десятигривневій
банкноті \emph{Київ} розмістив портрет Івана Мазепи. І був приречений
вислуховувати обурення Москви, radiosvoboda.org, 28.04.2021

Бросили файер и распылили слезоточивый газ. В \emph{Киеве} радикалы сорвали
закрытый показ ЛГБТ-фильма, strana.ua, 28.05.2021

\emph{Київ} відзначає 1539-річчя: Зеленський і Кличко привітали з Днем
\emph{Міста}, radiosvoboda.org, 30.05.2021

«Затишні та сповнені історією вулички Подолу й широкі проспекти спальних
районів – заворожують по-своєму. Та головне у \emph{Києві} те, що ти постійно
чуєш його серцебиття. Серцебиття, яке заряджає. Якби \emph{Київ} був людиною,
сьогодні я хотів би просто обійняти його і сказати: «Люблю тебе, старенький!» З
днем народження, \emph{Києве}!», – написав Зеленський у фейсбуці,
radiosvoboda.org, 30.05.2021

«Прокинутися в \emph{Місті}, яке любиш. Відкривати його для себе знову і знову.
Дбати про його красу і комфорт. Навчатися і працювати. Закохуватися і
відпочивати.  Щодня творити тут свою історію. Твій \emph{Київ} – твоя історія»,
– сказав Кличко, radiosvoboda.org, 30.05.2021

Перша половина звернення написана каліграфічним почерком, імовірно, її зробив
книжник, бо ж при Святій Софії у \emph{Києві} діяв скрипторій – майстерня з
перепису книг, \textbf{Графіті Софії \emph{Київської} і Володимир Великий.
Історик Нікітенко розповіла про «маркери» князя}, radiosvoboda.org, 27.09.2019

\emph{Кияни} за годину зібрали понад 5 тисяч для пенсіонерки, у якої шахраї
вкрали гроші з картки, kyiv.media, 29.05.2021

\enquote{Давайте в День \emph{Києва} подаруємо свято, закинемо на нову її карту
по 100-200 грн, хто скільки зможе, щоб компенсувати пенсію і повернути віру в
людей.  Поділіться цим постом і проведіть бесіди зі своїми рідними щоб вберегти
їх від подібних шахраїв}, – зазначив Олександр Барабошко, \textbf{\emph{Кияни}
за годину зібрали понад 5 тисяч для пенсіонерки, у якої шахраї вкрали гроші з
картки}, kyiv.media, 29.05.2021

\emph{Киевляне}, в вашем городе иногда везёт.  Везёт случайно забрести куда то
туда, где вы храните \emph{Киев} для себя, не для туристов, Максим Бужанский,
facebook.com, 30.05.2021

Такое впечатление что в \emph{Киеве} ставят какой-то злой эксперимент над
жителями...  По-другому объяснить маразм и анархию которая происходит в
\emph{Киеве} сложно...  \textbf{комментарий к посту Бужанского о Киеве за
30.05.2021}, Скрипка Николай, facebook.com, 30.05.2021

\emph{Киева} уже нет, остался \emph{КУев} и \emph{куяне}....
\textbf{комментарий к посту Бужанского о Киеве за 30.05.2021}, facebook.com,
30.05.2021

Огромное спасибо. Тепло Ваших слов о \emph{Киеве} дорогого стоит. Будьте
здоровы и счастливы не только в \emph{Киеве}, но и на планете Земля,
\textbf{комментарий к посту Бужанского о Киеве за 30.05.2021}, Svetlana
Ivanova, facebook.com, 30.05.2021

Был в прошлом году в \emph{Киеве} проездом из Жулян до Центрального ж/д
вокзала!  Прилетел перед закрытием авиадвиж. в связи с пандемией последним
рейсом из Ганновера. После Европы - удручающее впечатление, особенно столичные
дороги!  Тем не менее, \emph{Киевлян} с праздником! Будет и на их улицах
когда-то настоящее благоустройство и настоящий праздник!  Без всякого сомнения!
Дай Бог с нациками справиться! \textbf{комментарий к посту Бужанского о Киеве
за 30.05.2021}, Wl Mikriukov, facebook.com, 30.05.2021

Макс, \emph{Киева} нет. Есть ублюдок от соития застройщиков, отбитомозгой мрази
и понаехавшей галицайни, \textbf{комментарий к посту Бужанского о Киеве за
30.05.2021}, Egor Trofimoff, facebook.com, 30.05.2021

Названо локацію у \emph{Києві}, де покажуть оригінал Конституції Пилипа Орлика,
kyiv.media, 29.05.2021

У МЗС зазначили, що переговори щодо цього були дуже складними. У результаті все
ж досягнуто домовленості, що оригінал Конституції Пилипа Орлика, написаний 1710
року латиною покажуть у «Софії \emph{Київській}».  Виставлятимуть рукопис  у
Софійському соборі в \emph{Києві} з 16 серпня до 14 листопада 2021 року,
\textbf{Названо локацію у \emph{Києві}, де покажуть оригінал Конституції Пилипа
Орлика}, kyiv.media, 29.05.2021

У центрі \emph{Києва} стартував благодійний \enquote{Пробіг під каштанами}.
Кличко не зміг приєднатися через травму, kyiv.media, 30.05.2021

Зокрема, участь у благодійному етапі пробігу можуть брати команди адміністрацій
та підприємств. Так, на цю частину забігу прийшла команда \emph{Київської}
міської держадміністрації. Вони прийшли у футболках \enquote{Команда Кличка}.
На жаль сам мер \emph{Києва} Віталій Кличко приєднатися до традиційного забігу
цього року не зміг, \textbf{У центрі \emph{Києва} стартував благодійний
\enquote{Пробіг під каштанами}. Кличко не зміг приєднатися через травму},
kyiv.media, 30.05.2021

16.05.21 Возле памятника основателям \emph{Киева} неожиданно оказалось
многолюдно.  Напротив лестница на  Співоче поле, где все еще работала выставка
тюльпанов и потому вдоль дороги выстроилась длинная шеренга
автомобилей. Памятник открыли после реставрации. Новые лавочки, цветочная
клумба, фонтан с шикарной подсветкой, хотя  скорее всего любоваться ею будут
только посетители трех, расположившихся рядом ресторанов. Из-за развязки возле
моста Патона Наводницкий парк стал неуютным и мало посещаем, тем более, что
убрали остановки в сторону метро Дружбы Народов. А как было здорово, когда
вдоль Днепра ходил трамвай №5 - такой красивый маршрут! Алла Ильенко, Киевские истории,
facebook.com, 24.05.2021

Сьогодні відзначають День \emph{Києва} – основні локації та розваги, Перше
святкування Дня \emph{Києва} відбулося наприкінці травня 1982 року —
святкування 1500-річчя \emph{Києва}, bigkyiv.com.ua, 30.05.2021

\enquote{\emph{Киев} меняется. Но кое-что не меняется никогда - это наша любовь к
\emph{Киеву}. С днем рождения, любимый город}, - говорит Кличко в конце ролика,
\textbf{Кличко поздравил столицу старыми фото под песню \enquote{Как тебя не
любить, мой \emph{Киев}}. Видео}, strana.ua, 30.05.2021

Вася: зачем ты едешь в \emph{Киев}, если там такая \enquote{це-европа}?
ЗВЕЗДУЛЬКИ, zen.yandex.ru, 05.05.2021

Пущай над \emph{Киевом} звучит речь русская. Как и должно, кстати, быть.
Хотя... я бы не поехал. Не люблю быть там, где меня видеть не хотят.  А вот
артисты наши рвутся туда и я знаю зачем - это хайп. О них сразу напишут, что их
либо не пустили в страну, либо пустили, но хотят сорвать выступление.  Да,
видимо хайп. Так как удовольствие сомнительное - петь и думать,
\enquote{прилетит} тебе что-нибудь в голову или обойдётся?  Там ведь не один
такой Серюня - их много, зачарованных.  Когда же их перевоспитают - а главное
кто и как? А пора...пора уже))) \textbf{Вася: зачем ты едешь в \emph{Киев},
если там такая \enquote{це-европа}?} ЗВЕЗДУЛЬКИ, zen.yandex.ru, 05.05.2021

Появились фото очереди на вакцинацию в \emph{Киеве}, которая растянулась на сотни
метров, strana.ua, 30.05.2021

\emph{Киеву} - 1539. Москве - 700: цивилизацию на болота завезли
\emph{Киевляне}, Джон Смит, obozrevatel.com, 30.05.2021

\emph{Киеву} - 1539. Москве - 700. В два раза меньше.  Даже по официальной
российской версии Москву основал князь-\emph{киевлянин}.  То их
\enquote{Золотое кольцо России} - Ярославль, Владимир, Переяслав-Залесском -
все построено украинскими князьями из \emph{Киева}.  Цивилизацию на болота
завезли \emph{киевляне}.  Это наш царский территориальный подарок аборигенам.
С днем \emph{Киева}, Джон Смит, obozrevatel.com, 30.05.2021

\enquote{Киев, с днем рождения! \emph{Киев} это мое детство, это маршрут 18 троллейбуса
и прогулки Владимирской горкой. Церковь на улице Мостицкой, каток Льдинка, под
аркой Дружбы народов, где я так и не стала фигуристкой. Куреневский рынок, где
был куплен лучший друг детства – рыжий кот Макс. Все это мой \emph{Киев}. Я
очень жалею, когда вижу, как разрушают исторические здания, и в такие моменты
хочется посмотреть в глаза этим людям, кому новое блестящее из стекла и бетона
лучше, чем памятник архитектуры. Но несмотря на все не теряю надежду, что город
перестанут разрушать и застраивать хаотично и \emph{Киев} станет еще красивее,
чем был в моем детстве. Люблю тебя, \emph{Киев}}, – написала знаменитость,
\textbf{\enquote{Океан Ельзи}, Юлия Санина, Джамала: как украинские звезды
поздравили с Днем \emph{Киева}}, obozrevatel.com, 30.05.2021

\enquote{Признаюсь честно – этот город меня закалил! Научил быть честным, в первую
очередь с собой! Этот город придал мне уверенности! Я часто люблю колебаться и
долго принимать решение, но это о Карпе в \emph{Киеве} сейчас. Так, пять лет в моем
паспорте штамп с \emph{киевской} пропиской, хотя знакомство и жизнь здесь началась
гораздо раньше. Здесь живу я, моя семья, мои друзья и близкие люди. Этот город
ярких огней и неподдельных эмоций!} – заявила артистка,
\textbf{\enquote{Океан Ельзи}, Юлия Санина, Джамала: как украинские звезды
поздравили с Днем \emph{Киева}}, obozrevatel.com, 30.05.2021

Опубликовала свое поздравление в Facebook с Днем \emph{Киева} и украинская
группа, во главе с солистом Святославом Вакарчуком.  \enquote{С днем твоим, наш
вечерний, утренний, громкий, тихий, концертный, рабочий, ежедневный и
праздничный, желанный и мечтательный, наш ритм, наш дом, наш \emph{Киев}!} –
написал коллектив в соцсетях, \textbf{\enquote{Океан Ельзи}, Юлия Санина,
Джамала: как украинские звезды поздравили с Днем \emph{Киева}},
obozrevatel.com, 30.05.2021

\enquote{Хорошо помню нашу первую прогулку летом 1983 года. Марина мне открыла
невероятный ночной \emph{Киев}: \enquote{Шоколадный домик} на улице Либкнехта,
сейчас Шелковичная, \enquote{Дом с химерами} на Орджоникидзе, а ныне Банковой,
\enquote{Дом неутешительной вдовы} на улице Энгельса, теперь Лютеранской.
После этого я провожал Марину домой в переулок Ипсилантиивський – тогда он еще
назывался Аистова, где ее ждала взволнованная мама}, – поделился воспоминаниями
Порошенко, \textbf{Порошенко: \emph{Киев} – самое счастливое место, здесь я
встретил любовь моей жизни}, obozrevatel.com, 30.05.2021

Он также отметил: рад тому, что улицам удалось вернуть исторические названия,
\enquote{убрав коммунистический мусор из нашего дома}.  Порошенко написал, что
\emph{Киев} является особенным для каждого украинца в мире.  \enquote{В этом городе
выросли наши дети, а теперь растут внуки, здесь похоронены наши родители... Это
действительно наш дом}, – подытожил пятый президент, \textbf{Порошенко:
\emph{Киев} – самое счастливое место, здесь я встретил любовь моей жизни},
obozrevatel.com, 30.05.2021

Большой \emph{киевский} роман, Всеволод Непогодин о романе Алексея Никитина «От
лица огня», hvylya.net, 30.05.2021

\enquote{Завтра будем требовать другими методами}. Как в \emph{Киеве} прошла акция
сторонников Стерненко, strana.ua, 30.05.2021

У \emph{Киева} есть и хайп, и вайб и ему для этого не нужны ни \emph{киевляне},
ни гости столицы, Егор Гордеев, strana.ua, 30.05.2021

Как я уже писал в публикации ко дню рождения Булгакова, \emph{Киеву} досталась
трагическая судьба - стоять на геополитическом фронтире. Года этак с 1240-го
здесь нет покоя, Даниил Богатырев, strana.ua, 30.05.2021

Ну, а конформисты... Они будут себя нормально чувствовать даже если \emph{Киев}
переименуют в \enquote{\emph{Куив}}, а всех жителей заставят красить лица
гуталином, Даниил Богатырев, strana.ua, 30.05.2021

Сегодня - повод подумать о том, каким прекрасным городом \emph{Киев} мог бы
стать, Павел Вернивский, strana.ua, 30.05.2021 

Знаете, ситуация в \emph{Киеве} это ещё один пример когда краткосрочное личные
интересы множества наносят огромный ущерб долгосрочным интересам \emph{Городу}
в целом. Жаль только, что это никого не волнует, Павел Вернивский, strana.ua,
30.05.2021

Під Офісом президента України в \emph{Києві} відбулася акція на підтримку
одеського активіста Сергія Стерненка і його соратника Руслана Демчука, передає
кореспондент Радіо Свобода, radiosvoboda.org, 30.05.2021

З армії я повернувся в 1986 році, далі там себе не бачив. Радянська армія –
коли ти ніхто. Коли ти – порох, пил на чоботях командирських.  Я вступив у
\emph{Київський} педагогічний інститут на музично-педагогічний факультет.
Виявилося випадково, що в мене є неабиякий голос, природний матеріал
(сміється). Так мені сказали під час вступу, \textbf{Співаючий далекобійник. Як
український хорист став противником Путіна, зіркою YouTube та потрапив в
Канни}, pravda.com.ua, 28.05.2021

Накануне дня города в центре \emph{Киева} обвалился на тротуар кусок балкона.
Фото, strana.ua, 30.05.2021

В \emph{Киеве} на Лукьяновке обвалился балкон. Фото, strana.ua, 18.04.2021

В шахте лифта одной из многоэтажек на левом берегу Днепра в \emph{Киеве} произошел
пожар. Фото, strana.ua, 30.05.2021

\emph{Киевский} силовик дезертировал с оружием с позиций ВСУ у Золотого –
Народная милиция, lug-info.com, 30.05.2021

Пробег под каштанами 2021 - В День \emph{Киева} провели благотворительный забег
- фото и подробности, zn.ua, 30.05.2021

\emph{Київ} будується швидко і часто непрогнозовано. За висотками не встигають
зводитися школи та інші заклади соціальної інфраструктури. Автомобілісти у
столиці мають пріоритет перед велосипедистами, пішоходами та іншими учасниками
руху.  До Дня \emph{Києва} ми попросили фахівців із планування міст
прокоментувати ці та інші болі росту столиці та поміркувати, що з ними можна
вдіяти, \textbf{Болі росту. Урбаністи про головні проблеми розбудови
\emph{Києва} і те, як їх можна розв'язати}, hromadske.ua, 30.05.2021

І не в останню чергу – переборюючи настрої божевільного по суті чекання на
начебто чудодійну російську вакцину з боку не тільки пацієнтів (монологи на
честь \enquote{Спутника V}, мовляв, тільки йому можна довіряти, я чув на власні
вуха у стольному \emph{Києві}, що ж говорити про регіони з більшим
проросійським \enquote{градусом}), а й частини лікарів. Нехай не надто значної,
але реальної. Тобто людей, які встигли наслухатися \enquote{експертів} з відомої трійки
заблокованих телеканалів і їхніх новітніх іпостасей у YouTube або ще гірше –
надивитися російського телебачення (супутникові антени ніхто не забороняв).
Ясна річ, що \enquote{відловити} зазомбованих пропагандою лікарів майже неможливо, а
якщо і можливо, то що з ними робити? Віддавати під суд за державну зраду? Чи
просто \enquote{зрізати} кваліфікацію, а разом із тим – оплату? Не знаю, то справа
нового міністра, gazeta.ua

Ефір цьому телеканалу був перекритий у перший же день. Проте «Перший
незалежний» спокійно працює в Інтернеті з \emph{Києва}, чи то зберігши студії
своїх попередників, чи то створивши нові, та продовжуючи «славетні» традиції
попередників. Причому продовжуючи у дуже хитрий, а разом із тим простий спосіб,
який дозволяє мінімалізувати втрату аудиторії, \textbf{«Заблоковані» телеканали
та антидержавна пропаганда}, Сергій Грабовський, day.kiev.ua, 18.05.2021

Дорогой Кленя! Все ли ты еще дома? (Тебе уж надоел этот вопрос?) У нас после
двух недель весны была почти неделя какого­то междусезонья со страшным ветром,
тучами и даже дождем, во время которого я схватила кашель и насморк, но без
жару и без неприятных осложнений. Теперь погода опять поправилась, — значит, и
я поправлюсь.  Мне прислали немножко денег из \emph{Киева}, и это очень кстати,
т[ак] к[ак] иначе я была стеснена в сроке выезда, а теперь могу хоть и до Пасхи
пробыть, если надо будет. Целую вас всех. Л[еся], 219. До К. В.  Квітки,
09.03.1913, Хелуан, том. 14, Леся Українка, Повне академічне зібрання творів,
Листи (1907–1913), Луцьк, 2021

Хрещатик сивий, \emph{Київ} теж — красиві, бо ти в цей світ державити прийшла,
Василь Стус, II. 1964

Здається, чую: лопають каштани, жовтозелену викидають брость і зовсім поруч —
\emph{Київське} весняне пахуче небо гуком налилось.  Здається, бачу: рвуться
буйні трави, де вже відговорили всі струмки, а Україна, Лебединя, Слава за
сином назирає з-під руки, Василь Стус

Кругом — вона, геть доокруж — вона, та тільки терням поросла дорога.  Сосна
росте із ночі. Роєм птиць благословенна свінула Софія, і галактичний
\emph{Київ} бронзовіє у мерехтінні найдорожчих лиць, Василь Стус

\emph{Київ} — за ґратами. \emph{Київ} — весь у квадраті вікна. Похід почався
Батиїв?  ачи орда навісна?  Мороком горло огорне — ані тобі продихнуть.
Здрастуй, бідо моя чорна, здрастуй, страсна моя путь, Василь Стус.

\emph{Київ} святкує своє 1500-річчя. Реставровано Золоті ворота, через які
ніхто не в'їздить і не виїздить. Символом \emph{Києва} була для мене брама
Заборовського.  Замурована. Бо цей \emph{Київ} запечатано. Що кращий стає
\emph{Київ}, то він страшніший. Бо, замість живого міста, обернувся на
маскараду, машкару вампіра, що п'є кров своїх синів і дочок – і від того
кращає. Згадую жінку з «Соняшної машини», голова якої була схожа на зміїну.
Золотоголовий \emph{Київ} – змієголовий. Ніяк не позбудуся враження, що над
ювілейним \emph{Києвом} висить труна Івана Світличного (чи живий він?) – як
статуя Ісуса Христа над Римом. Хизування ювілеєм \emph{Києва} – гордість заброд
і підніжків. Бо пишатися вони не вміють, бо люблять тільки хамською любов'ю.
Право на офіційну любов до \emph{Києва} має тільки сонм чиновників – т. зв.
інтелігенція по-радянському, Запис 9-10, ТАБОРОВІ ЗАПИСКИ, Василь Стус.

Навколо блукали вівчарки й вівці, підходячи до вогню й беручи з рук газовиків
їжу. Було ще видно, тож вогонь палав зовсім невидимо в призахідних сонячних
променях. Газовики сиділи на витоптаному футбольному газоні й готували свою
баранину. Схожі були на монголо-татар, котрі відпочивали після вдалого набігу
на газові вишки \emph{Київської} Русі, \textbf{Ворошиловград}, Сергій Жадан

Між ними один був вищий од усіх цілою головою: то був їх поводатар, Степан
Воздвиженський. Це були тульські семінаристи. Вони йшли до \emph{Києва}. Деякі
були послані на скарбові гроші в \emph{Київську} духовну академію [1], деякі
були прості семінаристи, що, скінчивши курс, йшли до \emph{Києва} на прощу,
\textbf{Хмари}, Іван Нечуй-Левицький

Кандидатам до академії були видані скарбові гроші на поштові коні до самого
\emph{Києва}. Одначе вони не поїхали за ті гроші, а пішли пішки і гроші
постановили пропити дорогою, ще й товаришів-богомольців напоїти. Стаючи коло
корчом на спочинок, вони гуляли й пили й товаришів поїли. Останні поштові гроші
вони пропили в Броварах, недалечке од \emph{Києва}, хрестячись і молячись до
синіх святих \emph{київських} гір, на котрих біліли церкви й дзвіниці, блищали
проти сонця золоті хрести й бані церков. Хоч далека, зате ж весела була їм
дорога до \emph{Києва}!  Забачивши святий \emph{Київ}, вони зареклися більше
пити, щоб вступити по-християнській до святого міста, \textbf{Хмари}, Іван
Нечуй-Левицький

Стоять \emph{Київські} гори непорушне, заглядають в синій Дніпро, як і
споконвіку, несуть на собі пам'ятку про минувшість для того, хто схоче її
розуміть, і ждуть не діждуться, поки знов вернеться до їх слава старого
великого \emph{Києва}, поки знов заквітчають їх потомки давніх батьків свіжими
квітками історії... \textbf{Хмари}, Іван Нечуй-Левицький 

Тульські семінаристи стріли велику силу прочан, що йшли з далекого краю до
\emph{Києва}. Всі богомольці, побачивши \emph{Київ}, попадали навколішки,
хрестились, молились і били поклони, \textbf{Хмари}, Іван Нечуй-Левицький

Воздвиженський приїхав до \emph{Києва} дуже богобоящим. Він часто вставав серед ночі,
моливсь богу й не давав спать іншим своїми молитвами. Раз його товариші
прокинулись вночі й побачили щось біле й велике, що стояло коло шафи і
шепотіло. То був Воздвиженський, \textbf{Хмари}, Іван Нечуй-Левицький

Раз над \emph{Києвом} стояла чудова весняна ніч, що так надихувала духом поезії
Гоголя й Пушкіна. Повний місяць дивився в синій, гладенький, як дзеркало,
Дніпро. Небо було ясне й синє. На заході, над чорною смугою лісу й гір небо
блищало дуже пізнім рум'яним вечором. Було ясно, як удень. На Братській церкві
можна було читать золоті написи на стінах. Повітря було тихе, запашне.
Здається, не тіло, а сама душа ним дихала. На серці ставало легко, на душі —
спокійно, \textbf{Хмари}, Іван Нечуй-Левицький

І довго ще після молитви, сидячи перед одчиненим вікном, дивився Дашкович на
сонний, тихий \emph{Київ}, на високі гори, де блищали золоті верхи Михайлівського
монастиря, де на шпилі висів собор, Андріївський, а попід горою зеленіли садки,
біліли стіни будинків. Йому хотілось одного — не покинуть \emph{Києва}, а другого —
знайти тиху, добру, як голубка, і співучу жінку, \textbf{Хмари}, Іван Нечуй-Левицький

На стінах світлиці можна було читать історію нещасного \emph{Києва}, котрого шарпали й
перекидали з рук у руки сусіди, \textbf{Хмари}, Іван Нечуй-Левицький

Послалась ніч над \emph{Києвом}; вже погас червоний одлиск на високих верхах
Андрея й Михайлівського монастиря, а Марта з Степанидою все сиділи коло вікна.
Резеда розливала тонкі пахощі й сповнила ними всю хату. Вони знов заграли й
заспівали \enquote{Звук унылый фортепьяна}. І голоси їх були дуже гарні, і
пісня наче сама співалась в чистім повітрі, але думи брали перевагу. Кожній
хотілось марити мріями, \textbf{Хмари}, Іван Нечуй-Левицький

Ранок, свіжий і росяний, саме розгорювавсь. На самих вершечках усіх золотих
хрестів Старого \emph{Києва}, на високих горах, з'явились червоні, як жар,
проміння, а Поділ лежав сонний під горою в глибокій темній тіні,
\textbf{Хмари}, Іван Нечуй-Левицький

Я насмілююсь просить, щоб мене зоставили в \emph{Києві}, — просив
Воздвиженський, \textbf{Хмари}, Іван Нечуй-Левицький

Дашкович недовго й був в академії. Його добра слава пішла скрізь; і його
запросили перейти в Московський університет. Але, скучаючи додому й за
\emph{Києвом}, він потім перейшов в \emph{Київський} університет,
\textbf{Хмари}, Іван Нечуй-Левицький

Ого-го! Отак наші! Вдивляй світ. Боже тобі поможи! Честь і слава \emph{Києву} й
Дніпрові, коли об'явиться тутечки новий філософ! А мені здається, навіщо тобі
крячкою сидіть, мордувать себе? Адже ж батько твій прожив вік без
пересвідченнів, і я живу так само, і всі живуть так само. Одначе якось живемо,
хвалить бога, і земля нас держить, \textbf{Хмари}, Іван Нечуй-Левицький

Та не дури мене, бо я ще не зовсім дурна! Сухобрусів не було дурних ні в
\emph{Києві}, ні поза \emph{Києвом}, \textbf{Хмари}, Іван Нечуй-Левицький

В цім \emph{Києві} такі чуда творяться, громадянство таке дике, люди такі, що
не вміють навіть до ладу говорить по-руській, не то що по-французькій.  При тих
словах Турман знов спустила очі, почувши дуже примітну українську вимову в
розмові \emph{киянок}, \textbf{Хмари}, Іван Нечуй-Левицький

Дивовижний збіг неймовірно втішив Яроша — його запросили на конференцію,
присвячену літературі мертвими мовами, до Стамбула, це було цілком закономірно,
адже на території Туреччини були і Лідія, і Лікія, і Урарту, і Хетська та
Арканумська держави. У літаку, який вилітав з \emph{Києва}, Ярош несподівано
побачив біля себе Андрія Куркова і не міг стриматися, щоб не познайомитися.
Письменник летів на презентацію свого роману «Пікнік на льоду», Ярош читав цей
роман, особливо йому в пам'ять вбився сумний задумливий пінгвін, який жив у
головного героя. Дорогою вони розговорилися, а турецьке вино, яке подавали
стюардеси в маленьких стограмових пляшечках, виявилося дуже смачним,
\textbf{Танго Смерті}, Юрій Винничук 

Кожен з нас ставив себе на місце батька і намагався уявити і втечу, і помсту, і
повернення, а найбільше нашу уяву сколочувала така цікава деталь, про яку теж
ми довідалися з часописів: серед ночі один із поранених бійців випорпався з
могили, доповз до селянських хат, там його підлікували, а опісля помогли
перейти польський кордон. Хто був тим козаком 4-ї \emph{Київської} дивізії,
якому вдалося вижити і врятуватися? А що як це хтось із наших батьків?
\textbf{Танго Смерті}, Юрій Винничук

Він жив скромно, даючи приватні уроки музики, коштів на життя вистачало, але
Мількер був книгоманом, підтримував зв'язки з букіністами \emph{Києва}, Москви,
Вільнюса, а що на старі книги грошей бракувало, то займався ще й спекуляцією,
як тоді казали, бо відвідував нелегальні книжкові базари, які переслідувала
міліція, а тому задля конспірації у 1970—1980-х роках доводилося часто міняти
місце зібрання книголюбів, \textbf{Танго Смерті}, Юрій Винничук

Усе це дуже повільно. А тут мені дедалі частіше повідомляють про появу людей,
які рангом починають пригадувати щось із свого попереднього життя. Як на зло,
воно припало якраз на воєнні і повоєнні роки. Уже й з \emph{Києва} дзвонили,
питали, що це за хєрня. Мовляв, у нас же ж зараз нова політика, згладжуємо усі
гострі кути з Росією, а тут знову — Катинь, розстріли в'язнів, більшовицький
терор...  І все це з уст очевидців! Звідки це береться?  \textbf{Танго Смерті},
Юрій Винничук

На щастя, та родина, яка там поселилася, перейнялася моєю долею, і коли
господареві запропонували роботу в \emph{Києві}, він хутенько прописав мене в себе. І
отак я опинився знову вдома, \textbf{Танго Смерті}, Юрій Винничук

Джордж Орвелл, 1984. — \emph{Київ}: Видавництво Жупанського, 2015.

Під впливом усіх цих викриттів СРСР врешті - решт змушений був визнати факт
голоду 1932 – 1933 років. Перед розпадом Радянського Союзу, у 1988 – 1989
роках, в Україні встановлено перші пам'ятники жертвам Голодомору — на
\emph{Київщині} та в Харкові, \textbf{Голодомор - комуністичний геноцид в Україні},
memory.org.ua

Хоча, правду кажучи, він почував себе педофілом. Банзаю видавалося, що ті
нечисленні перехожі, яких вони стрічали, озиралися назад, кліпали очима й
пересвідчувалися, що це не сон; вони дивились услід молодому чоловікові, який
ішов вулицею, обіймаючи за плечі якусь неповнолітню дівчину, \enquote{малолєтку}, як
кажуть у \emph{Києві}, \textbf{Культ}, Любко Дереш

І ще він думав про лекцію, надану йому Соломією. Про те, що не варто занадто
прив'язуватись до людей, особливо – до \enquote{малолєток}, як кажуть у \emph{Києві},
\textbf{Культ}, Любко Дереш

Darwin, C. (1859). On the Origin of Species by Means of Natural Selection.
London: John Murray, - Дарвін Чарльз. Походження видів. — \emph{Київ}—Харків:
Державне видавництво сільськогосподарської літератури УРСР, 1949. — 444 с. —
Прим. перекл.

Емілі Дікінсон. Лірика. — \emph{Київ}: «Дніпро», 1991. — С. 248

Долина богів. Історії з Кремнієвої долини / Пер. з англ. Андрія Бондаря. —
\emph{Київ}: Yakaboo Publishing, 2018. — 288 стор.

\emph{Київський} вокзал буяв осінніми настроями. Люди гонили по перонах,
спекулянти здавали білети в неіснуючі вагони, начальники поїздів садили людей у
купе провідників. Одне слово, все йшло своєю чергою. Мелодійно-меланхолійний
голос диктора сповістив, що прибуває потяг No87 Мінськ—\emph{Київ} на шосту
колію. Маса людей, які зустрічали його, посунули на третій перон штурмувати
вагони в пошуках тих, кого вони чекали. Перон гудів майже десять хвилин і так
само несподівано затих, \textbf{Місто, в якому не ходять гроші}, Кузьма
Скрябін, Повне зібрання творів, Харків, 2019

Вся схема електронної системи міста була розроблена мирним дядюлькою, який в
\emph{Києві} розробляв конструкції луна-парків, а сюди загримів замість шести років
тюрми за те, що одна карусель була неправильно розрахована й обірвалася,
поховавши десять дітей. Таких спеців було тут з тридцять, 
\textbf{Місто, в якому не ходять гроші},
Кузьма Скрябін, Повне зібрання творів, Харків, 2019

Ти малий долбєнь! Ану закрий пащеку, бо я тобі розірву її сам. Якби вона
лишилася в нас до ранку, ти б лікувався вже в іншій лікарні. На Павлівській, у
\emph{Києві}, а я на Лук'янівці чекав розприділення в Зону, але в іншу Зону — строгого
режиму. Ти поняв, ублюдок сопливий? Закрий рот, мразь, і мовчи, бо підеш за
нею. Вона в лабораторіях на фабриці, і ти вже не пізнаєш її і не побачиш
ніколи, бо тобі то не треба. Ти поняв?
\textbf{Місто, в якому не ходять гроші},
Кузьма Скрябін, Повне зібрання творів, Харків, 2019

Чемпіоном в дисципліні «Найдебільніший переїзд року» став я. 28 травня —
переїзд автобусом \emph{Київ}—Ковель на концерт, вночі — Ковель — Луцьк на
концерт нічний, зранку автомобіль Луцьк—Ужгород, після нічного концерту літак
Ужгород—\emph{Київ} в 7.00 ранку, там живий ефір на М1, потім у 13.50 переліт в
Донецьк, там жваве спілкування із прикордонниками і з фляшкою «Джек Денієлз»,
потім машина до Маріуполя, там концерт, машина зранку в Донецьк, літак до Києва
30 травня в 6.50, звідти літак в Копенгаген, звідти всього годинка — іти — в
Осло. Правда, просто?  \textbf{Я, Шонік і Шпіцберген}, Кузьма Скрябін, Повне
зібрання творів, Харків, 2019

Дніпро, \emph{Київ} — місто невелике, ми зустрінемось в метро. І зранку знову
ниряю у підземні тунелі, І несе мене до тебе ця криклива канітєль, І бачу ті
самі джинси і той рюкзачок, І від радості зжимаю у кишені кулачок.  Ще чоловік
п'ятдесят у вагон забігає, Мене розмазує по тобі, ніби маслом коровай, І я
чекаю в душі на зупинку «Хрещатик», Щоби вибратись наверх і навіть пози не
міняти, \textbf{Говорили і курили}, Кузьма Скрябін, Повне зібрання творів,
Харків, 2019

Моя країна має \emph{Київ}, Волинь і Карпати, Донбас і Чорне море, Одесу і
Львів.  І в ній живуть на повну ногу лиш одні депутати, Всі інші грають ролі у
кіно про лохів.  Стиснувши зуби, всі стоять і пропускають кортежі, За місяць
заробляють тільки фляжку вина. Найбільша радість українця — дотягнути до
завтра, Ми всі давно в окопах, хоча це не війна, \textbf{Руїна}, Кузьма
Скрябін, Повне зібрання творів, Харків, 2019

Дніпро, Одеса, \emph{Київ} і Донбас, За нами у колибу завертає, І поки вистачає
на скіпас, То значить, ше біди в житті немає, \textbf{Буковель}, Кузьма
Скрябін, Повне зібрання творів, Харків, 2019

Так, в атмосфері взаємоповаги, під високоінтелектуальну бесіду і проходила наша
вечеря. Фінал усі любили найбільше. Батьки Роми привозили спиртне, яке стане
відомим широким масам співвітчизників тільки у далекому 2000 році. Коньяки,
віскі, ром, що ми пили в армії, я після дємбєля не бачив довгих 11 років, аж
поки цивілізація не докотилася до брам \emph{Києва}, а моя кишеня, нарешті, не
набула такого статусу, коли до супермаркету можна заходити за покупками, а не
спитати в касирки, яка година, \textbf{Я, паштет і армія}, Кузьма Скрябін,
Повне зібрання творів, Харків, 2019

До чого ж тут Медведчук? У 1965-му, коли Стус і В'ячеслав Чорновіл
\enquote{зривали} прем'єру (до речі, Сергій Параджанов згодом розцілував їх за
чесний і сміливий вчинок), Вітє, що виховувався у родині політзасланця (за
однією версією — батько був членом ОУН, інші кажуть — шуцман), було усього
дев'ять років.  У 72-му він ще вчився у школі і ще не вступив на юридичний
факультет \emph{Київського} університету. Отже, міг хіба чути про арешти
якихось \enquote{антирадянщиків}, \textbf{Справа Василя Стуса - Глава книги,
яку хоче заборонити Медведчук}, Василь Кіпіані

Уперше долі Медведчука та Стуса перетнулися, ймовірно, влітку 1980 року, коли
Віктор Володимирович працював пересічним \emph{київським} адвокатом, а Василь
Семенович перебував у камері слідчого ізолятора КГБ на вулиці Володимирській.
Чотирнадцятого травня 1980 року Стус був заарештований співробітниками слідчого
відділу Управління КГБ по м. \emph{Києву} та \emph{Київській} області за
звинуваченням в \enquote{антирадянській агітації та пропаганді}, 
\textbf{Справа Василя Стуса - Глава книги, яку хоче заборонити Медведчук},
Василь Кіпіані

Об'єктивним джерелом про поведінку Стуса та його адвоката Медведчука є 58-ме
число московського самвидавного журналу \enquote{Хроника текущих событий}. Саме
там ми й можемо прочитати звіт про судовий процес 29 вересня — 2 жовтня 1980
року, що проходив у залі Київського міськсуду, \textbf{Справа Василя Стуса -
Глава книги, яку хоче заборонити Медведчук}, Василь Кіпіані 

Вологого, як всяке природне збудження, весняного ранку правобережним
\emph{київським} проспектом в районі Печерська жваво рухався насторожений
молодик років 26-ти у бідненькій довгій чорній одежині: сягала надп'яткових
його кісток, тому по землі не волочилася, натомість відкривала геть не дешеві
черевики іспанської фірми Camper, і вже тільки один цей незрозумілий дисонанс
мав би викликати запитання в очах перехожих, \textbf{\#Галябезголови}, Люко
Дашвар

Філософ і барига Гавана, дужий дядько років сорока, жив у Козельці,
приторговував краденим і погодився не тільки скупити оптом увесь крам, який
Консуматенко поцупив у Хмельницькому, у монастирі і на \emph{київському} Печерську і
припер до Козельця сьомого березня під ранок, а й гостинно відчинив для Юрка
двері своєї дерев'яної, обкладеної цеглою хати: живи, їж, пий, розмірковуй про
недосяжну неосяжність життя!
\textbf{\#Галябезголови}, Люко Дашвар

Тобто шостого березня вона ще стояла! — резюмував слідчий. — Бо комунальники
мусили відтягувати її від баків, аби забрати з них сміття. А після восьмого —
зникла. І номери...
\begin{itemize}
\item — Що? — полковник оцінив зачіпку як слабеньку, тому і запитав без азарту.
\item — Номери не \emph{київські}...
\item — А «міні-купер» якого року? Вони зможуть його ідентифікувати за роком випуску?
\end{itemize}
\textbf{\#Галябезголови}, Люко Дашвар

Кажуть, у Чорнобая поселилася дівка! І требує, щоби він організував у місті
притулок для всіх бомжів \emph{київських}! Щоби зі столиці до Затятового їх
переселити! Уявляєш? \textbf{\#Галябезголови}, Люко Дашвар

От зараза! Ще на роботу запізнюся, — пришвидшила ходу і за перехрестям побачила
червону «мазду» з \emph{київськими} номерами. Стояла неподалік вагона-кав'ярні.
Поряд із автівкою розгублено тупцював такий стильний хлопець, що Лєні аж
дихалку перехопило, \textbf{\#Галябезголови}, Люко Дашвар

Видушував з власного бізнесу останню копійчину, але завалював жінку
подарунками, дивував щедрими романтичними витівками — то безліч троянд, то
Дніпром на орендованій яхті, то вальси Штрауса у виконанні \emph{київських} віртуозів
персонально для пані Юлії... «Якщо не так витрачати — нащо тоді взагалі
заробляти?!» — підтримував сам себе, та невдовзі пролунав другий дзвінок.
Побачив на зап'ястку жаданої жінки обнову: браслет білого металу з прозорими
білими камінцями, \textbf{\#Галябезголови}, Люко Дашвар

Чорнобай потиху вів позашляховик \emph{київськими} центральними вулицями, уважно
вдивлявся у постаті дівчат. Посеред них могла бути Галя. Дурне діло — отак
шукати дівчину, яка не хоче, аби її знайшли. А невістка не хоче, бо інакше б
відповідала на дзвінки. Та як зупинитися? Годиною раніше на Пушкінській біля
старовинного будинку Чорнобай проводжав поглядом роздратованого сина, який ішов
до червоної «мазди», \textbf{\#Галябезголови}, Люко Дашвар

«Швидка» повезла Анку у супроводі Сашка до \emph{київської} клініки. Перед тим
Чорнобай відкликав затятівського автомеханіка вбік, дав йому гроші: і на
операцію, і на реабілітаційний період мало вистачити.  — А ви хіба не з нами до
\emph{Києва}? — спитав Сашко, \textbf{\#Галябезголови}, Люко Дашвар

Скільки маршрутка пленталася до \emph{Києва}, стільки Галю не покидало
незрозуміле відчуття тривоги. Дивувалася: а чому? Подвійну добру справу
зробила: не просто врятувала кошеня, а й подарувала краплину радості і тепла
жінці, якій увесь білий світ — чорна яма, бо тваринка точно зробить її життя
оптимістичнішим, 
\textbf{\#Галябезголови}, Люко Дашвар

Я сказала: не в салонах \emph{Києва} чи області! Я кажу: в салонах краси і перукарнях
всієї України! Якщо дізнаюся, що ти ці мої слова не почула, твоє фото гулятиме
соціальними мережами з моїми особистими коментарями. І я не обмежуся
коментарями! Я знайду тебе і знищу власноруч! Почула?
\textbf{\#Галябезголови}, Люко Дашвар

На ДВРЗ, околичному районі \emph{Києва}, з вікон якого Бровари видно, Оля Корнійчук
курила на кухні орендованої малосімейки, хоч робити те було справою вкрай
ризикованою, бо хазяйка попередила: відчує запах табаку — вижене з квартири
того ж дня. Зазвичай Оля курила на балкончику, та сьогодні світ перевернувся,
старі його сенси категорично втратили вагу, і Оля відчувала просто-таки фізичну
потребу бути жорстоко покараною за слабкодухість і боягузство,
\textbf{\#Галябезголови}, Люко Дашвар

У Польщі курячі лапи перекручую на собачі консерви! — збрехав Юрко і рвонув на
жабці до \emph{Києва},
\textbf{\#Галябезголови}, Люко Дашвар

Уявив альтернативу: ось він за кермом «міні-купера», вже далеченько від'їхав
від \emph{Києва}, аж поліція. Стояти! Чи просто зупинився кави попити у придорожній
кафешці. Дівчина за стійкою подає йому каву і каже: «У вас, отче, ряса в
крові!»,
\textbf{\#Галябезголови}, Люко Дашвар

«Що в біса відбувається? — подумала. — Купа людей, машин!» І непростих машин,
бо, крім поліцейських «пріусів» і «швидкої», просто у скверику, який оточував
будинок, стояли два новіших за Юлин «мерседеси», «рендж-ровер» і рідкісний для
\emph{Києва} «бугатті»,
\textbf{\#Галябезголови}, Люко Дашвар

Дістався бази. Ходив берегом озера, чекав на пані Хитрук, усе намагався оцінити
результати своєї майже триденної подорожі до \emph{Києва}: чи прояснилося після неї у
мізках?!, 
\textbf{\#Галябезголови}, Люко Дашвар

Залишив Галю під наглядом Казидорівни, зірвався — скільки їхав до \emph{Києва},
стільки розмірковував: з якого боку починати проблему вирішувати? Логіка
підказувала: звернися до приватної клініки, де тебе самого знають, як
облупленого, надають кваліфіковану й ефективну допомогу щоразу, коли ти
звертаєшся!  \textbf{\#Галябезголови}, Люко Дашвар

\begin{itemize}
\item — Я не з Києва. Я — нізвідки. І малої батьківщини у мене немає. — Так не буває.
\item — Ще й як буває. Я народилася у машині, коли тата з однієї
військової частини переводили до іншої,
\end{itemize}
\textbf{\#Галябезголови}, Люко Дашвар

\begin{itemize}
\item — І нащо до \emph{Києва} приїхав?
\item — Мріяв стати слугою Господа! — відповів Консуматенко, обдаючи столичну
				компанію міцним горілчаним смородом. — Вступив до обителі послушником.
				Святий отець-настоятель так мене хвалив! Так хвалив! Казав: брате мій,
				Юрію, ти створений, аби стати ченцем! Скоро ми з братією розкриємо для
				тебе свої обійми і приймемо до своїх лав, а Господь те давно зробив.
\end{itemize}
\textbf{\#Галябезголови}, Люко Дашвар

\begin{itemize}
\item — А кіт? Кота, значить, у дівчини не було?
\item — Та ви дослухайте! Вона того ж дня виїхати з \emph{Києва} хотіла, а я
				того дня ще мав до обителі повернутися. Кажу ж — друг помер.
								Відспівували... І кажу їй: можу тільки завтра.
\end{itemize}
\textbf{\#Галябезголови}, Люко Дашвар

Бровари ми оминули, бо у дівчини плани змінилися. Вона спочатку хотіла у
Броварах ветеринара знайти, бо котові геть зле було. Ми на виїзді з Києва
заправилися, і вона мене все підганяла: скоріше! Щоби встигнути кота до лікаря
довезти. Так кіт помер, коли ми тільки з \emph{Києва} виїхали,
\textbf{\#Галябезголови}, Люко Дашвар

— Хіба зараз це має значення? — роздратувався Тьома. — Мене інше вражає! Ти
приїжджав до \emph{Києва}, говорив зі мною про Галю, але не сказав, що вона у тебе!
Чому, тату? А раптом у мене через те зараз — великі проблеми!
\textbf{\#Галябезголови}, Люко Дашвар

Поки їхали до \emph{Києва}, так і не закинув вудку, бо всю снагу перебила дивна
пригода. Вже рухалися Затятовим до траси, коли на шляху червоної «мазди» виник
хлопець на велосипеді. І як той Сашко Сулима дізнався, що Галя їде з міста?
Виїхав прямо під колеса автівки, і коли Тьома з переляку так сильно натиснув на
гальма, що «мазда» аж вискнула, Сашко кинув вєлік і не зважав на сина Чорнобая,
який вискочив з автівки, кричав та розмахував руками,
\textbf{\#Галябезголови}, Люко Дашвар

«Мазда» зірвалася з місця... Скільки їхали до \emph{Києва} — все мовчали.
Майбутнього психолога тільки і вистачило на те, щоби віддати дружині ключі від
трикімнатної квартири на Березняках: мовляв, усе як раніше, мила! Та настрій —
у дупу!  Вирішив відкласти розмову на завтра, а вранці наступного дня набрехав
Галі, що має бігти в універ, вискочив з дому, зателефонував спочатку Лисиці,
\textbf{\#Галябезголови}, Люко Дашвар

— А я тобі кажу! Забувай уже про свою «Леваду», повертайся до \emph{Києва}, —
усміхнувся Черпак, пропустивши повз вуха невдоволення друга. — Тут на тебе
чекає більше відкриттів, ніж посеред лісу біля озера,
\textbf{\#Галябезголови}, Люко Дашвар

Сашко хотів був переконати Казидорівну, що сам може довезти Галю до \emph{Києва}: хоч
на материному «мегані», хоч на військовому УАЗику, який врешті виторгував у
фермера. Та з Казидорівною не поторгуєшся: де сядеш, там і злізеш. І мати
затялася: будемо робити, як Казимира Теодорівна скаже!
\textbf{\#Галябезголови}, Люко Дашвар

Ото хай завтра зранку Казидорівна відтарабанить нашу турботливу Галю до \emph{Києва},
а ми перехрестимося і попросимо Бога, аби вона до Затятового дорогу забула! А
Чорнобая ми і без неї знайдемо! І врятуємо, якщо у тому буде потреба! —
вичерпалася, замовкла,
\textbf{\#Галябезголови}, Люко Дашвар

— З Казидорівною до \emph{Києва} поїхала, — Анка подумала, що хоч і ціною власного
здоров'я, але мети вона досягла: вберегла сина від тої Галі!
\textbf{\#Галябезголови}, Люко Дашвар

Звідти під чужим іменем в'їхав до України, зупинився в орендованій квартирі на
околиці \emph{Києва}, тиждень купався у любові свого коханого Льоні і ретельно
готувався до найвідповідальнішого кроку у своєму житті, 
\textbf{\#Галябезголови}, Люко Дашвар

Вечір уже запалив на вулицях \emph{Києва} тисячі ліхтарів: і з жовтими витратними
лампами розжарювання, і з економними яскравими ледами. Вони так обнадійливо
сяяли, що здавалося, у місті не лишилося жодної темної плями. Навпаки!
Непомітні вдень кутки, скверики, провулки увечері виглядали набагато
презентабельніше, і від того народжувалося викривлене враження про те, що
увечері в \emph{Києві} не менше перспектив, ніж удень! Мо', навіть більше!
\textbf{\#Галябезголови}, Люко Дашвар

Він приїхав після дзвінка Сашка, коли дізнався, що після інфаркту Анці потрібна
складна і дорога операція на серці. Мотнувся назад, до \emph{Києва}, спробував
знайти Антона Черпака, та друг перебував «поза зоною». Чорнобай поїхав по
лікарнях: інститут серця, клініка імені Амосова, Олександрівська лікарня,
лікарня на Червоному хуторі, приватні клініки, 
\textbf{\#Галябезголови}, Люко Дашвар

Увечері в Прищевій хаті лунали пісні церковного, а потім далеко не церковного
змісту. Деркуляни знали: хазяїн продав рибу святій братії
\emph{Києво}-Печерської лаври, \textbf{Знахідка на все життя, Повісті},
Огульчанський О. Я.  

Преподобный Исаия Печерский, - Преподобный Исаия в числе других
\emph{Киево}-Печерских святых подвизался в XI - начале XII вв. Основным
подвигом его жизни были безмолвие и неутомимый труд, за что он именуется
Трудолюбивым. Скончался святой подвижник в 1115 году, мощи его находятся в
Ближних пещерах \emph{Киево}-Печерской Лавры. Празднование преподобному Исаии
совершается 15 мая, 28 сентября и на второй Неделе Великого поста.

Усього через два тижні непопулярний Янукович утік з країни в розпал протестів,
що стали відомі як Євромайдан. Як доказ нової сили соціальних мереж, назва
цього українського повстання була взята від хештега в Twitter (він поєднував
«Європу» (бо демонстранти прагнули партнерства з Європою, а не Росією) та
«Майдан Незалежності», центральну площу \emph{Києва}, де відбувалися мітинги).
Але, як і революціонери, що використовували нову форму Інтернету для об'єднання
та усунення ворога, Росія тепер використовувала мережі, щоб розірвати Україну
на частини, \textbf{Війна лайків. Зброя в руках соціальних мереж}, П. В.
Сінґер, Емерсон Т. Брукінґ, Харків, 2019

Чорнай тримав удома потішного й напрочуд товариського коргі з кличкою Данте.
Рута чи не кожні вихідні перестрівала їх дорогою до супермаркету «ВОПАК», що на
\emph{Київській}. Із часом коргі навіть почав її впізнавати. Іванка також раз
чи двічі зустрічала вчителя з собакою, але клички того не знала, \textbf{Доки
світло не згасне назавжди}, Макс Кідрук

Парубок біля вікна, майнувши чубом, чіпляє сідло на гак, а батько:
– Поїдеш у \emph{Київ}, Грицьку, та той... Іди лишень сідлай коні.
– Де це я?! – аж скрикнув Григорій, нагло зводячись.
– Заспокойся, ляж, Бог з тобою! – це матінка, це стара, поклала руку на голову і схилила
йому її назад на подушку: – Лежи, лежи, серце. Ач який, мов з хреста знятий... А ви! І не тю на вас, лоботряси! Добре, що самі, нівроку, як воли, а хлопцеві не до розмов.
\textbf{Тигролови}, І. Багряний — «Фолио», 1944

– Ні-ні!.. – злякався, відчув, як у нього мороз пішов поза спиною: – Ні-ні! – В
пам’яті зринула Лук’янівська в’язниця... \emph{Київське} ОГПУ – НКВД... Отак!
Утікав, утікав і потрапив назад... Як же це?... Ні-ні! Він поривається йти: –
Пустіть... Пустіть... Я не хочу... Я піду собі... То я так, то я навмисне...
\textbf{Тигролови}, І. Багряний — «Фолио», 1944

Втім, повернемося до моменту її рішення «стати українкою». Адже подальша її
діяльність як члена ОУН, приїзд в окупований нацистами \emph{Київ}, перебування
у застінках ґестапо і навіть сама смерть були вже радше наслідками того зламу,
котрий у свідомості цієї сміливої жінки відбувся раніше, \textbf{Ростислав
Семків. Передмова, - Олена Теліга, Вибрані твори},

Там, за лісами, неспокійно спить,
В боях ранений, мій трагічний \emph{Київ},
Та біля мене не лише блакить —
Сліпуче сяйво розхиляє вії.
Здавалось все: і ліс, і я сама,
І це багаття в заграву злилося.
Ти мала димний і сосновий смак,
Моя п’ятнадцята прекрасна осінь!
\textbf{«Похмурий ліс у вересневім сні...», Вибрані твори}, Олена Теліга


\emph{Киев} не должен мериться древностью с другими столицами, \emph{Киев}
должен иметь современность, которой можно было бы гордиться, Константин
Бондаренко, strana.ua, 31.05.2021

Але скільки би не було років \emph{Києву} - варто не мірятися древністю з
іншими столицями, а думати про його сучасне обличчя і інфраструктуру. І просто
- любити. Як вчив Олександр Духнович: «Люби род свой не про то, что он славный,
а про то, что он твой», \textbf{\emph{Киев} не должен мериться древностью с
другими столицами}, Константин Бондаренко, strana.ua, 31.05.2021

В \emph{Киеве} вслед за тарифным нарастает земельный майдан, Александр
Гончаров, strana.ua, 31.05.2021

\emph{Киеву} можно пожелать одного - пережить этот грязный период, Елена
Лешенко, strana.ua, 31.05.2021

В центре \emph{Киева} мужчина расстрелял другого и скрылся на \enquote{Мерседесе}, strana.ua, 01.06.2021

Украинские националисты из организаций \enquote{Правый сектор} и
\enquote{Правая молодежь} по случаю Дня \emph{Киева} устроили в сквере Василия
Слипака у Андреевского спуска собственное праздничное мероприятие, в ходе
которого детей учили стрелять по бумажному макету московского Кремля из
автомата, \textbf{Правый сектор в центре \emph{Киева} устроил
\enquote{стрельбы} из АКМ по бумажному Кремлю}, strana.ua, 30.05.2021

Об этом свидетельствуют фотографии, размещенные на Facebook-странице
\enquote{Правый сектор \emph{Киев}} в воскресенье, 30 мая.  \enquote{Во время
мероприятия все желающие имели возможность пострелять по кремлю, колорадским
жукам, сфотографироваться в военном снаряжении и испытывать себя в разборке и
сборке автомата Калашникова}, - сообщили организаторы мероприятия,
\textbf{Правый сектор в центре \emph{Киева} устроил \enquote{стрельбы} из АКМ
по бумажному Кремлю}, strana.ua, 30.05.2021

\emph{Киевские} городские сумасшедшие одинокими точно не смотрятся, Светлана
Крюкова, strana.ua, 31.05.2021

Безумец счастлив в своём безумии. Ветер качает за его спиной тяжёлые ветки
столетних каштанов и клёнов, дирижируя в такт, делая нашего уличного героя
прохожим на фею с \emph{Киевских} холмов и парков, спустившуюся с гор на
асфальт в диковинном обличье, чтобы пошатать ветки и проветрить дороги любимого
города летней прохладой, \textbf{\emph{Киевские} городские сумасшедшие
одинокими точно не смотрятся}, Светлана Крюкова, strana.ua, 31.05.2021

Чего он только не исполняет в этом прикиде: здоровается, размахивает руками,
подмигивает, в упор выглядывая глазами знакомых водителей, желает и показывает
всеми известными невербальными жестами пальцами - удачи, победы, фортуны,
счастья и безопасности на дороге, окатывая \emph{Киевлян} душевным приветливым
взглядом и весёлостью, контрастирующей с сонным утром, изменчивой погодой и
настроением скучающих горожан, настолько разительно, что многие просыпаются,
машут в ответ и подбрасывают привычному герою дорожные чаевые. За что? За
настроение. В больших городах эмоции стоят денег, \textbf{\emph{Киевские}
городские сумасшедшие одинокими точно не смотрятся}, Светлана Крюкова,
strana.ua, 31.05.2021

Неизменные атрибуты городской жизни, без которых \emph{Киев} был не \emph{Киев}
- городские сумасшедшие, Светлана Крюкова, strana.ua, 31.05.2021

В \emph{Киеве} вороны напали на прохожего. Ему пришлось спасаться бегством.
Видео, strana.ua, 31.05.2021

Сегодня ректор \emph{Киевского} национального университета им. Тараса Шевченко
Владимир Бугров заявил, что 7 июня ученый совет КНУ лишит белорусского
президента Александра Лукашенко звания почетного доктора вуза.  Об этом Бугров
сообщил на пресс-конференции во время Всеукраинского форума \enquote{Украина
30.  Образование и наука}. \enquote{Я обратился к Ученому совету наработать
процедуру, ее действительно сегодня нет. Но уже три комиссии ученого совета
выработали такую процедуру. Там основаниями является нарушение прав человека,
неуважение к государству Украина. 7 июня будет заседание ученого совета
университета на котором будет два вопроса: утверждение процедуры и лишение
статуса почетного доктора Александра Лукашенко}, - рассказал ректор,
\textbf{Университет Шевченко лишит Лукашенко звания почетного доктора вуза},
strana.ua, 31.05.2021
