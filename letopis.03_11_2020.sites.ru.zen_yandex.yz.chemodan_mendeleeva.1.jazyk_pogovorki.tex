% vim: keymap=russian-jcukenwin
%%beginhead 
 
%%file 03_11_2020.sites.ru.zen_yandex.yz.chemodan_mendeleeva.1.jazyk_pogovorki
%%parent 03_11_2020
 
%%url https://zen.yandex.ru/media/mendeleev/zazrenie-zadorinka-oskomina-i-pantalyka-chto-eto-znachit-10-samyh-neponiatnyh-russkih-pogovorok-5fa115946f1ecc136f783866
 
%%author 
%%author_id yz.chemodan_mendeleeva
%%author_url 
 
%%tags 
%%title Зазрение, Задоринка, Оскомина и Панталыка. Что это значит? 10 самых непонятных русских поговорок
 
%%endhead 
 
\subsection{Зазрение, Задоринка, Оскомина и Панталыка. Что это значит? 10 самых непонятных русских поговорок}
\label{sec:03_11_2020.sites.ru.zen_yandex.yz.chemodan_mendeleeva.1.jazyk_pogovorki}
\Purl{https://zen.yandex.ru/media/mendeleev/zazrenie-zadorinka-oskomina-i-pantalyka-chto-eto-znachit-10-samyh-neponiatnyh-russkih-pogovorok-5fa115946f1ecc136f783866}
\ifcmt
	author_begin
   author_id yz.chemodan_mendeleeva
	author_end
\fi
\index[rus]{Русский язык!Поговорки!10 самых непонятных русских поговорок, 03.11.2020}

\ifcmt
  pic https://avatars.mds.yandex.net/get-zen_doc/3714606/pub_5fa115946f1ecc136f783866_5fa1164801bdcc483342be88/scale_1200
	caption Он такой...
  width 0.4
\fi

\subsubsection{Без зазрения совести}

Даже интуитивно не понятно, что значит зазрение. Связано ли это как-то с
глазами и глаголом узреть? Связано, при этом самыми крепкими узами. «Без
зазрения совести», это развитие устаревшего оборота «совесть зазрила» из
которого вытекает, что «зазрить» означает угрызение, иными словами: «совесть
загрызла». Однако, это уже более современное понимание, ведь изначально «зрить»
= «видеть, смотреть». Таким образом «зазрить» это вариация современного
«присмотр», но с некоторыми нюансами, а саму поговорку можно перевести, как
«игнорировать взгляд совести». Между прочим, устаревшее «зазрить» сегодня могло
бы пригодиться. Например, для обозначения камер наружного наблюдения. Сам факт
их наличия, как раз можно было бы назвать зазрением.

\subsubsection{Без сучка и задоринки}

\ifcmt
  pic https://avatars.mds.yandex.net/get-zen_doc/2907131/pub_5fa115946f1ecc136f783866_5fa117166f1ecc136f7aeffa/scale_1200
  width 0.4
\fi

Если с сучками все более или менее понятно, то, что за зверь «задоринка»? Те,
кого я успел пробежать с этим вопросом, отвечали вполне логично – шутка,
сарказм или ирония. Таким образом, данную поговорку можно трактовать, как
работу, сделанную без помех и подколов с чьей либо стороны. Однако, слово
«задоринка» к «задору» не имеет никакого отношение, это больше похоже на
омонимы. Истинным братом «задоринки» является глагол «задирать, драть,
надрывать» и так далее, а само слово означает шероховатость. Поговорка же
появилась в среде плотников, где фраза «без сучка и задоринки» означала гладкую
доску, без сучков и шероховатостей.

\subsubsection{Вверх тормашками}

\ifcmt
  pic https://avatars.mds.yandex.net/get-zen_doc/3993525/pub_5fa115946f1ecc136f783866_5fa116d16f1ecc136f7a73bb/scale_1200
  width 0.4
\fi

Сегодня это выражение встречается все реже, но когда-то любой перевернутый
предмет описывался именно этой фразой. Логично предположить, что тормашки, это
дно или нижнее основание чего бы то ни было. Например, описывая бардак в
комнате, можно сказать: «все вверх дном» или «все вверх тормашками». В данном
случае фразы равнозначны, что и подтверждает значение слова. Этимология
«тормашек» восходит к «тормам», так на одном из диалектов русского языка
называли ноги.

\subsubsection{И на старуху бывает проруха}

\ifcmt
  pic https://avatars.mds.yandex.net/get-zen_doc/3985561/pub_5fa115946f1ecc136f783866_5fa1174949e00863eb42b68c/scale_1200
  width 0.4
\fi

Значение поговорки, думаю, знают все: и опытный человек может совершить ошибку.
А вот откуда тут слово «проруха»? В интернете популярна версия, что проруха,
это искаженное поруха, что в древности означало «изнасилование». Звучит забавно
и даже изобретательно, но неверно. И проруха и поруха имеют общий корень — рух,
что на праславянском означало «движение». Ныне к однокоренным можно отнести
слова разруха и разрушение, и хоть смысл мало изменился, но глагольная форма
«прорушить» (прорубить, пробить, продолбить) до нас не добралась. Иными
словами, «И на старуху бывает проруха» буквально означает, что и у опытного
человека может что-то сломаться/рухнуть.

\subsubsection{Витать в облаках}

\ifcmt
  pic https://avatars.mds.yandex.net/get-zen_doc/3985748/pub_5fa115946f1ecc136f783866_5fa1179ba5c3e80d311b0e89/scale_1200
  width 0.4
\fi

Сам по себе глагол «витать» не вызывает вопросов, все же очевидно: витать,
значит, летать. Не так, как птица, а как бы незримо. С легкой руки классиков
это очень древнее славянское слово стало неизменным спутником дымки и запаха.

\begin{leftbar}
  \begingroup
    \em\Large\bfseries\color{blue}
Риза, безголовая, безрукая, горбом витала над толпой, затем утонула в толпе,
потом вынесло вверх один рукав ватной рясы, другой.

Михаил Булгаков, «Белая гвардия»
  \endgroup
\end{leftbar}

Злую шутку с бессмертными авторами русской литературы сыграла именно эта
поговорка. Ее они понимали ровно так же, как и мы, витать, значит, летать. Но
на самом деле «витать» у славян — это обитать, находиться в каком-либо месте.
При чем чаще это слово использовалось в значении гостить. Отсюда и
распространенное в славянских языках (чешский, словацкий, польский и др.)
приветствие, звучащее примерно так: vita(t, ti, с). Так что сегодня поговорку
«Витать в облаках» можно перевести на современный русский, как «Гостить на
облаках».

\subsubsection{Гол, как сокол}


\ifcmt
  pic https://avatars.mds.yandex.net/get-zen_doc/1675790/pub_5fa115946f1ecc136f783866_5fa117f949e00863eb43f94d/scale_1200
  width 0.4
\fi

Кажется, все просто, «гол» - голый, «сокол» - птица. Почему она голая? Ну, кто
поймет этих предков, может у них соколы голые были... Шутка в смене ударения, в
оригинале поговорка строго рифмованная: «Гол, как соко'л, остёр, как топор».
Согласен, рифма тут так себе, но, как минимум, ударения были соблюдены. Да и
птица сокол никакого отношения к данной поговорке не имеет. Соко'л — это
вариант стенобитного орудия, очищенный от сучков и веток ствол. Возможно
«соко'л» и «кол» когда-то были однокоренными словами. В любом случае, на
современный русский поговорку можно перевести так: «Голый, как очищенный от
веток и коры ствол дерева».

\subsubsection{Дать стрекача}

\ifcmt
  pic https://avatars.mds.yandex.net/get-zen_doc/3985748/pub_5fa115946f1ecc136f783866_5fa1181801bdcc4833460fec/scale_1200
  width 0.4
\fi

Вроде, все понимают, что это значит — убежать, при чем стремительно. Но ведь в
обиходе мы не используем слово «стрекач»! Кому в голову придет назвать,
например, спринтера стрекачом? А ведь это была бы вполне корректная адаптация
английского «sprinter», Владимир Жириновский остался бы доволен. Тем более что
в современном русском языке буквосочетание «стр» никуда не делось и сохранило
свой изначальный посыл — стрела, стремительность, стремглав (кстати, то же
странное слово) и так далее. Стрекач же, это отглагольное существительное, для
которого изначальной формой было слово «стрекать» - убегать. К сожалению, в
русском языке нет более современного отглагольного существительного, которое
могло бы заменить собой «стрекача». Ведь не скажем же мы «Дать бега»? Наиболее
близким к современному может быть только один перевод: «Дать дёру», но он уже
существует в качестве синонима и звучит не многим понятнее.

\subsubsection{Затрапезный вид}


\ifcmt
  pic https://avatars.mds.yandex.net/get-zen_doc/1907878/pub_5fa115946f1ecc136f783866_5fa118d46f1ecc136f7e1c86/scale_1200
  width 0.4
\fi

Это выражение встречается все реже и реже. Ему на смену пришло сленговое:
«стрёмный вид». Однако, если вы в приличном обществе или считаете себя
воспитанным и культурным человеком, то затрапезный — отличный вариант оскорбить
какого-нибудь гостя. Кажется, что «затрапезный» имеет какие-то древние корни и
описывает внешность человека после некой трапезы, мол, накушался и выглядишь не
прилично. Но история выражения куда интереснее. Во времена Петра I трудился
один предприниматель, Иван Затрапезников. От императора он получил текстильную
мануфактуру, на которой наладил производство материи. Правда в качестве
материала он использовал дешёвую пеньку, она же конопляное волокно, от чего
ткань получалась грубой и не особо качественной. Именно ее и прозвали
«затрапезом», от фамилии предпринимателя, а тех, кто щеголял в одежде из этой
ткани, получили эпитет: «Затрапезный вид».

\subsubsection{Набить оскомину}

\ifcmt
  pic https://avatars.mds.yandex.net/get-zen_doc/1579004/pub_5fa115946f1ecc136f783866_5fa119c001bdcc48334910ee/scale_1200
  width 0.4
\fi

Мало того, что «оскомина» совершенно не понятное слово, так еще и вся фраза
никак не согласуется со значением «сильно надоесть». Все просто, выражение на
столько древнее, что первое слово успело изменить свой смысл, а вот второе
вообще вышло из оборота. Сегодня слово «набивать» используется в двух смыслах:
набить подушку и набить морду. Первичное значение глагола сохранилось только в
устойчивой форме «набивать тату». При чем значение здесь близкое к «набитой
морде» - натирать, причинять болезненное раздражение. К примеру, в древности
можно было абсолютно спокойно сказать: «набил мозоль». Что же касается
оскомины, то это отглагольное существительное, в оригинале звучавшее так:
«скомить», что переводится на современный русский, как ныть (в контексте боли,
например, ноет зуб). Таким образом, «набить оскомину» можно перевести на
современный русский язык: «довести до ноющей боли». Кстати, сегодня словом
оскомина не редко называют гиперестезию - повышенную чувствительность зубов.

\subsubsection{Сбить с панталыку}

\ifcmt
  pic https://avatars.mds.yandex.net/get-zen_doc/1612125/pub_5fa115946f1ecc136f783866_5fa11a5d6f1ecc136f80f7c4/scale_1200
  width 0.4
\fi

Сбить с толку, привести в замешательство, запутать — все прекрасно знают, что
означает этот фразеологизм. А вот со значением слова «панталык» возникает
проблема. Более того, не сразу понятно, как его правильно писать в именительном
падеже единственного числа: панталык, панталыка или панталыку? Лингвисты пока
не пришли к единому мнению по поводу происхождения слова. Основная версия
гласит, что панталык (именно так), это искаженное название горы Пантелик в
Греции, со множеством пещер, в которых можно легко заплутать. Отыскать Пантелик
в Греции задача сложная (скорее всего невыполнимая), зато найти Некрополь
Пантелика в Италии на Сицилии труда не составит. Не знаю, как на счет пещер, но
там действительно много гробниц, вырубленных в скале. Но есть еще одна версия
происхождения слова панталык, от заимствованного из европейских языков корня
pantl (скорее всего, немецкого) — узелок/завязка. Pantl перешел к западным
славянам в значении ленточка (у нас, например, оттуда же слово «петля»). Вне
зависимости от версии происхождения, само слово «панталык» имеет значение
схожее с лабиринтом, узлом, завязкой. Например, Оксфордский словарь дает
определение слову панталык именно в значении «прийти в растерянность,
запутаться». Возможно, выражение появилось, как аналогия рассуждений человека с
плетением узлов. Как минимум, это выглядит логично.

