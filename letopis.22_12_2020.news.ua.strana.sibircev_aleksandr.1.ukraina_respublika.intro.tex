% vim: keymap=russian-jcukenwin
%%beginhead 
 
%%file 22_12_2020.news.ua.strana.sibircev_aleksandr.1.ukraina_respublika.intro
%%parent 22_12_2020.news.ua.strana.sibircev_aleksandr.1.ukraina_respublika
 
%%url 
 
%%author 
%%author_id 
%%author_url 
 
%%tags 
%%title 
 
%%endhead 
Уже появились самопровозглашенные \enquote{президент Украины} и \enquote{Львовский горсовет}.

\ifcmt
pic https://strana.ua/img/article/3079/kto-i-zachem-91_main.jpeg
caption В селе Верхняя Рожанка Львовской области создали свой "сельсовет" и приняли свой флаг
\fi

На Западной Украине неожиданно вспыхнул "сепаратистский" скандал. 

В декабре на Львовщине и Ровенщине группы активистов самопровозгласили
параллельные органы власти, за что полиция задержала шестерых из них,
предъявив им попытку "дестабилизации общественно-политической ситуации".

У "сепаратистов" провели обыски, во время которых нашли флэш-носители и
диски, протоколы собраний, газеты и документы об образовании незаконных
органов местного самоуправления. Возбуждено дело по ч.1 ст.3 53 УК Украины
(самовольное присвоение властных полномочий), по которому грозит до трех
лет. 

На самом деле, как выяснила "Страна", движение "самопровозглашенных"
советов охватило уже всю страну. Более того, избран даже "президент
Украины".

Об этом тренде "Страна" писала почти пять лет назад\Furl{https://longread.strana.ua/lyudi-vne-gosudarstva} - уже тогда в ряде
регионов появились так называемые "люди вне государства": граждане,
которые отказывались платить налоги и не признавали украинских законов,
организовывая свое самоуправление. Тогда на них провела облавы СБУ,
возбуждались уголовные дела.

Но к настоящему времени движение вновь возродилось. Причем в еще большем
объеме. Немало этому поспособствовал процесс укрупнения районов и создания
ОТГ, который вызвал сильное недовольство на местах.

Почему появляются "люди вне государства" и какие у них цели, выясняла
"Страна". 


