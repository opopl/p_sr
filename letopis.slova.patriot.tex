% vim: keymap=russian-jcukenwin
%%beginhead 
 
%%file slova.patriot
%%parent slova
 
%%url 
 
%%author 
%%author_id 
%%author_url 
 
%%tags 
%%title 
 
%%endhead 
\chapter{Патриот}
\label{sec:slova.patriot}

%%%cit
%%%cit_pic
%%%cit_text
Главное - не поступились принципами!  Оскандалиться везде - главная линия
поведения \emph{ура-патриотической} тусовки на международной арене. Пренебрегать
установленными правилами, если они мешают из любого международного мероприятия
делать перформанс для своей секты в социальных сетях - постоянный принцип. От
позорных выходок Гончаренко в ПАСЕ до шума по поводу националистического клича
на спортивной форме.  И если к ужасу настоящих \emph{патриотов} своей страны наша
сборная запомнится на Евро-2020 только этим, то та секта этому даже не
огорчится - зато они «не смогли поступиться принципами»
%%%cit_title
  
%%%endcit


%%%cit
%%%cit_pic
%%%cit_text
Трудно было не заметить, как старательно изображал из себя \emph{ультрапатриота} глава
федерации футбола Павелко. И трудно было с этого не поржать, понимая, что
мутный постсоветский бизнесмен Павелко космически далек от всей этой
бандеровщины. Его настоящее отношение к ней – что-то между равнодушием и
презрением
%%%cit_comment
%%%cit_title
\citTitle{Сев на украинские потоки, так важно одеть вышиванку},
Вячеслав Чечило, strana.ua, 13.06.2021
%%%endcit

%%%cit
%%%cit_pic
%%%cit_text
Теперь это возвращается бумерангом. Нищая перенаселенная Индия стала идеальным
инкубатором для формирования нового, более заразного штамма, который уже пришел
в Европу и США, провоцируя новую волну пандемии.  Впрочем, фармацевтические
корпорации от этого не расстроятся, а \emph{украинские патриоты} все равно не снимут
запрет на «Спутник». Потому что жизнь и здоровье простых людей заботят их
меньше всего на свете
%%%cit_comment
%%%cit_title
\citTitle{Индийский штамм коронавируса не испугает украинских патриотов}, 
Андрей Манчук, strana.ua, 13.06.2021
%%%endcit

%%%cit
%%%cit_head
%%%cit_pic
%%%cit_text
Послышались претензии от \emph{патриотов}, которые специализируются на видеоиграх.
Сообщество GameStreetUA, которое выступает за украиноязычную локализацию игр,
написало критический пост о \enquote{Сталкере-2} и его отношению к украинскому
языку. \enquote{Кто как не украинская компания должна понимать вред от
обрусевшей (в оригинале - зросійщеної) передачи украинских топонимов?}, -
задаются вопросом активисты. \enquote{И русским языком в трейлере, и
подзаголовком игры украинская компания способствует распространению ложных
стереотипов о российском Чернобыле среди иностранцев}, - бьет тревогу паблик.
Правда, он считает, что GSC сделала это \enquote{невольно}. Но тем не менее
авторам поста \enquote{грустно}. Они призывают исправить \enquote{ошибку} до
релиза игры
%%%cit_comment
%%%cit_title
\citTitle{Сталкер 2 Сердце Чернобыля - почему хейтят видеоигру украинских разработчиков}, 
Екатерина Терехова, strana.ua, 16.06.2021
%%%endcit

%%%cit
%%%cit_head
%%%cit_pic
%%%cit_text
И вреда от такого несистемного \enquote{санкционного протекционизма} намного больше,
чем от специального рамочного торгового закона, ведь вместо нулевого сальдо мы
получаем торговую дыру с РФ на 1,5-2 млрд долл. ежегодно. Зато это
\enquote{\emph{патриотичная} дыра}
%%%cit_comment
%%%cit_title
\citTitle{Отрицательное торговое сальдо Украины с РФ - патриотичная дыра / Лента соцсетей / Страна}, 
Алексей Кущ, strana.ua, 24.06.2021
%%%endcit

%%%cit
%%%cit_head
%%%cit_pic
%%%cit_text
Как там напутствовал Карнеги: не держите все яйца в одной корзине. И вот они
рады тому факту, что их откатили в отдельное лукошко. Орут что те резанные о
своей эксклюзивности и несхожести с русскими. Как смеют там в Москве называть
их одним народом. Они иные — бритые, а значит бриты им роднее и ближе. При этом
бреют их все и во всём, отвлекая их внимание на красивые эротические
(\emph{патриотические}) картинки. Теребят их напряжённые \emph{патриотические нервы} ложными
поводами. А покумекать возможности просто не существует. Природа так устроила.
Можно сказать, наказала или функцией такой наградила. Им сказали что Петро
Сидоренко не брат Петру Сидорову, при этом Зеленский, Шендерович, Коломойский и
прочие Нетаньяху одним народом быть могут
%%%cit_comment
%%%cit_title
\citTitle{Яйца, но не роковые}, Дмитрий Жук (ЦИНИК), zen.yandex.ru, 02.07.2021
%%%endcit

%%%cit
%%%cit_head
%%%cit_pic
%%%cit_text
Почему эта тема так важна для нас? Да потому что в нэзалежной после инцидента
наступила эйфория, и все \enquote{\emph{патриоты}} начали говорить о том, что Россия боится
развязать Третью мировую войну, а они с поддержкой так называемых западных
партнеров смогут теперь усилить давление на Россию в рамках попыток
\enquote{возвращения} Крыма и Донбасса. Для таких одурманенных украинцев как обухом по
лбу прозвучали слова российского президента о том, что даже гипотетическое
уничтожение нарушившего неприкосновенность российских границ британского
эсминца едва ли бы поставило мир на грань новой глобальной войны. Объяснил он
это тем, что те, кто осуществлял провокацию, прекрасно понимали, что в их
ситуации нельзя победить. По словам главы российского государства, это не
русские к ним пришли за тысячу километров, а \enquote{НАТО приперлось к границам
России}. Это был четкий месседж Западу и всем, кто считает Россию слабой:
попытаетесь нарушить границу - будете жестко наказаны
%%%cit_comment
%%%cit_title
\citTitle{Луганский Информационный Центр – НЕДЕЛЯ ГЛАЗАМИ ЭКСПЕРТА: 
Пеленг дозволенного, штампы внешнего управления и выходной бастард}, , lug-info.com, 04.07.2021
%%%endcit

%%%cit
%%%cit_head
%%%cit_pic
%%%cit_text
Україні давно вже час піднятися над проблемою мови і перейти з площини \enquote{Кращий
\emph{патріот} – україномовний} до діалогу про те, як нам об’єднатися і прийняти одне
одного, як відмінності перетворити на розмаїття, консерватизм – на збереження
традицій, а культуру не насаджувати, а творити у форматі космополітизму.
Мова – занадто конфліктний символ, щоб побудувати навколо нього ідею нації й
\emph{патріотизму}. Вона стала уособленням протистояння двох ідеологій, що виросли на
спільній історії, але являють собою дві сторони однієї правди. А правда завжди
багатолика
%%%cit_comment
%%%cit_title
\citTitle{Українці не розуміють одне одного не через мову, а через небажання слухати, чути і сприймати}, 
Юлія Мендель, www.pravda.com.ua, 07.07.2021
%%%endcit


%%%cit
%%%cit_head
%%%cit_pic
%%%cit_text
Реакцию большинства украинских политиков на статью Путина нельзя назвать адекватной.
Переполох, который поднялся в политических кругах Украины после выхода статьи
Владимира Путина «Об историческом единстве русских и украинцев», трудно
переоценить. Однако в целом реакцию на эту статью большинства украинских
политиков и аналитиков нельзя назвать адекватным ответом.
Обычно авторы ответов страдают тем, в чем сами обвиняют В. Путина, а именно:
идеологическими клише, ярлыками и отвлечением от основной темы. Как правило,
статья объявляется посредственной, неинтересной и такой, которой украинский
\emph{патриот} не заинтересуется. Так вот, я считаю, что украинец, который думает о
будущем своей страны и своего народа, должен эту статью внимательно изучить,
поскольку она дает достаточно жесткую, но абсолютно правдивую оценку нашей
общей истории и современной украинской государственности
%%%cit_comment
%%%cit_title
\citTitle{О будущем украинского и русского народов}, 
Виктор Медведчук, strana.ua, 15.07.2021
%%%endcit

%%%cit
%%%cit_head
%%%cit_pic
%%%cit_text
Да, украинцам присуще чувство \emph{патриотизма}, но это не есть замена
профессионализму. «Не читал, но осуждаю», - это принцип нынешних «шариковых».
Господи, сколько же этих дебилов набилось во власти?! По-моему, образумить их
уже невозможно, этимология понятий у нас, образованных людей, одна, для них –
другая, т.е. какая-то придуманная ими в одночасье.  К первоисточникам и
здравому смыслу их обращать бесполезно. По-моему, мы с ними уже никогда не
достигнем той точки динамического равновесия, когда еще можно дискутировать, а
не бить сразу в морду
%%%cit_comment
%%%cit_title
\citTitle{В Украине не осталось ни денег, ни защиты, ни даже мечты}, 
Александр Гончаров, strana.ua, 31.07.2021
%%%endcit

%%%cit
%%%cit_head
%%%cit_pic
%%%cit_text
Выделялся лишь алчевский токарь Юра Бардаш, который высказывался по этим
опасным темам на телешоу Дудя. Но его травили за это в Украине как только
можно.  Ложь не может родить талант, и украинская музыка остается глубоко
провинциальной, вторичной, скучной - она явно не отвечает исторической драме,
которую переживает в эти годы страна. Музыкальная тусовка представляет собой
загончик с \emph{патриотическими собаками} - только не в совковом космосе, а на
хуторе, где развалили колхоз, а сейчас дерибанят землю.  И случайная
вынужденная фронда Green Grey выглядит на этом безрыбье панковским бунтом
%%%cit_comment
%%%cit_title
\citTitle{Green Grey повели себя так, как, по идее, и полагается рокерам}, 
Андрей Манчук, strana.ua, 31.07.2021
%%%endcit

%%%cit
%%%cit_head
%%%cit_pic
%%%cit_text
Узколобые \emph{«патриоты»} возмущаются фото нашей Ярославы Магучих со
спортсменкой из России. Мол негоже Олимпийским чемпионкам и стоять рядом.  Даже
прочла, что Ярославу планирует вызвать «на ковёр». Отдельные персонажи требуют
отправить спортсменку на фронт.  Стыдно и мерзко. В условиях, когда государство
наплёвало на развитие спорта, каждая медаль - подвиг.  Кому в пору паковать
чемоданы - так это вот таком разжигателям.  А Ярослава - героиня. Как и все
наши олимпийцы
%%%cit_comment
%%%cit_title
\citTitle{Узколобые патриоты призывают отправить олимпийскую призерку на фронт}, 
Диана Панченко, strana.ua, 08.08.2021
%%%endcit

%%%cit
%%%cit_head
%%%cit_pic
%%%cit_text
Я читав некролог. Марко Федорович Сміян, видатний державний діяч, вірний син,
полум’яний \emph{патріот}... народився там-то й тоді-то… Вчився в школі, в інституті. В
комсомолі набув смаку до громадської роботи. Ну, так, набув. Але ж після
інституту — жодного дня на практичній роботі. Одразу — в керівники. Маленькі,
більші, ще більші, найбільші! Що ж тут писати? Перелічувати посади, титули,
ордени? Ага, знайшли-таки. — Після закінчення інституту Марко Федорович ще три
роки навчався в аспірантурі. Набуті в інституті спеціальні знання, звичка до
самостійного вивчення розвитку соціалістичної культури, глибоке ознайомлення з
досвідом відомих теоретиків і практиків, смак до роботи з першоджерелами — усе
це чимало важило для його подальшої роботи, сприяло власному самовизначенню...
%%%cit_comment
%%%cit_title
\citTitle{Тисячолітній Миколай}, Павло Загребельний 
%%%endcit

%%%cit
%%%cit_head
%%%cit_pic
%%%cit_text
Гениальность Гоголя, писателя и драматурга, настоящая. Поскольку с момента
написания «Ревизора» и «Мёртвых душ» прошло 200 лет, а всё понятно и временами
знакомо до колик в животе. Порода человеческая никуда не делась. Веришь в
картины великого мастера, понимая: он не придумал образы. Нашёл, пригляделся,
был со многими знаком лично. А сегодняшний социум… такой же. Чуть приправленный
сменой гардероба, научно-техническим прогрессом, да более примитивным языком
общения.  Но суть это не меняет. Как было «наверху тьма власти, внизу — власть
тьмы», так и осталось. Никуда не исчезли две беды: «дураки и дороги». Особенно
смешно, когда все «говорят о \emph{патриотизме}, небось проворовались…».
Занавес. И низкий поклон гению!
%%%cit_comment
%%%cit_title
\citTitle{Такие смешные: украинская политика... через призму творчества Гоголя}, 
Исторические напёрстки, zen.yandex.ru, 28.10.2021
%%%endcit

%%%cit
%%%cit_head
%%%cit_pic
%%%cit_text
Крайне редко впрягаюсь комментировать всякую хайповую дичь, но это случай
особый.  На фото не рядовой гопан, злой на мир с очередной похмелуги. Это, с
позволения сказать, епископ Адриан Кулик, один из самых \emph{патриотичных
патриотов} в ПЦУ.  Персонаж с яркой биографией. Сначала нахомутал с
американской церковной юрисдикцией, откуда смылся в 2001-м году в Киев. Затем,
после мытья и катанья с УАПЦ, в 2013-м, благополучно кинул и слабеющих
украинских автокефалов, присоединившись к более сильному на тот момент
Филарету. В общем, тот еще верный воин.  Вы думали, что после его недавних
призывов маркировать недостаточно \emph{патриотичных} украинцев «безродными
собаками» кто-то где-то сделал какие-то выводы? Ага))) И ведь дело не в том,
что теперь этот ангельский образ совершенно так лихо и по-гопнически испытывает
на прочность правую и левую щеки беззащитной женщины
%%%cit_comment
%%%cit_title
\citTitle{Когда священник начинает бить за русский язык - это уже предел / Лента соцсетей / Страна}, 
Юрий Молчанов, strana.news, 31.10.2021
%%%endcit
