% vim: keymap=russian-jcukenwin
%%beginhead 
 
%%file 12_01_2022.fb.fb_group.story_kiev_ua.1.kiev_restorany_sssr.cmt
%%parent 12_01_2022.fb.fb_group.story_kiev_ua.1.kiev_restorany_sssr
 
%%url 
 
%%author_id 
%%date 
 
%%tags 
%%title 
 
%%endhead 
\zzSecCmt

\begin{itemize} % {
\iusr{Tetyana Kilesso}

Цукерки \enquote{Надія} я ніколи не куштувала. Відбивні з свинини так, замовляла вцьому
ресторані. А зараз замовила б заливну осетрину з лимоном.


\iusr{Светлана Манилова}

В упоминаемый Вами период в рестораны я еще не ходила. @igg{fbicon.smile} Впервые в \enquote{Столичном}
оказалась в новогоднюю ночь 1988 года и второй раз в июне 1989 года, когда
отмечали с группой окончание университета. Впечатления хорошие, как и
обслуживание с кухней.

\iusr{Вадим Ляшенко}

Многие блюда сейчас у нас в диковину.

\iusr{Светлана Ветрова}

Приємні спогади цін за порції можна було спокійно посидіти со спиртним на
червонец на двох але треба туди попасти а це іноді було скрутно

\iusr{Вадим Ляшенко}
 @igg{fbicon.thumb.up.yellow} 

\iusr{Катя Бекренева}

Не часто посещала рестораны) В основном на свадьбах) По хронологии: Пектораль,
два раза Краков, Млын (защита), Верховина, Украина (просто так), Братислава,
два раза на юбилеях, Столичный (просто так)

Это советское время. Понравилась кухня в Млыне.

\end{itemize} % }
