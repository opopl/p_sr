% vim: keymap=russian-jcukenwin
%%beginhead 
 
%%file 14_04_2021.fb.respublikalnr.10.ryba_priroda
%%parent 14_04_2021
 
%%url https://www.facebook.com/groups/respublikalnr/permalink/796276577674802/
 
%%author 
%%author_id 
%%author_url 
 
%%tags 
%%title 
 
%%endhead 

\subsection{Более 71 тысячи мальков выпустили в водоемы ЛНР}
\Purl{https://www.facebook.com/groups/respublikalnr/permalink/796276577674802/}

С начала года в 15 водоёмов Республики было выпущено более 71 тысячи мальков.
Об этом сообщили в Министерстве природных ресурсов и экологической безопасности
ЛНР.

\ifcmt
  pic https://scontent-bos3-1.xx.fbcdn.net/v/t1.6435-0/p180x540/173704373_122212109958897_7541339614976083210_n.jpg?_nc_cat=105&ccb=1-3&_nc_sid=825194&_nc_ohc=CK_lniZX8oQAX8kUof9&_nc_ht=scontent-bos3-1.xx&tp=6&oh=741e4be820c642419bd45a603bed1ee4&oe=609B62EF
\fi

«С начала 2021 года 13 пользователями водных объектов вселено в 15 водоемов
Луганской Народной Республики более 71 тысячи штук малька карпа, толстолоба и
радужной форели», – говорится в сообщении.

В ведомстве рассказали, что зарыбление стало возможным благодаря улучшению погодных условий.

«Учитывая прошлогоднее маловодье и, как следствие, недостаточное количество
малька на водных объектах Республики также проводятся мероприятия по пересадке
маточного поголовья в подготовленные пруды для дальнейшего получения малька в
условиях естественного нереста. Из уже имеющегося маточного поголовья
планируется получение личинки и малька путем инкубации икры на аппаратах Вейса
для дальнейшего подращивания и реализации осенью 2021 года», – уточнил
начальник отдела биоресурсов Минприроды Юрий Просолупов.

Всего в 2021 году в водоёмы Республики планируют выпустить 905 тысяч мальков и
выловить 237 тонн товарной рыбы.

Пресс-служба Правительства Луганской Народной Республики
\url{https://sovminlnr.ru/novosti/24177-bolee-71-tysyachi-malkov-vypustili-v-vodoemy-lnr.html}
