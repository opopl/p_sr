%%beginhead 
 
%%file 07_07_2018.fb.fb_group.mariupol.biblioteka.korolenka.1.spletem_vmeste_kupal
%%parent 07_07_2018
 
%%url https://www.facebook.com/groups/1476321979131170/posts/1745187022244663
 
%%author_id fb_group.mariupol.biblioteka.korolenka,kibkalo_natalia.mariupol.biblioteka.korolenko
%%date 07_07_2018
 
%%tags mariupol,kultura,prazdnik.ivana_kupala,festival,dity
%%title Сплетем вместе купальский венок
 
%%endhead 

\subsection{Сплетем вместе купальский венок}
\label{sec:07_07_2018.fb.fb_group.mariupol.biblioteka.korolenka.1.spletem_vmeste_kupal}
 
\Purl{https://www.facebook.com/groups/1476321979131170/posts/1745187022244663}
\ifcmt
 author_begin
   author_id fb_group.mariupol.biblioteka.korolenka,kibkalo_natalia.mariupol.biblioteka.korolenko
 author_end
\fi

Сплетем вместе купальский венок

Увлекательный «иванокупальский» квест прошел 6 июля в парке им. Гурова в рамках
фестиваля «Купальский венок».  Его подготовили и провели для мариупольской
детворы сотрудники Централизованной библиотечной системы для взрослых. Идея
игры-квеста была такова: каждому участнику необходимо собрать цветок из цветной
бумаги, пройдя пять станций, а далее все цветы  соединить в едином венке.

На старте участники получали бумажную цветочную серединку и отправлялись «в
путь», как и положено на Ивана Купала, по высокой зеленой траве.  Все задания:
отгадай-ка, поиграй-ка, собери-ка, смастери-ка, разгадай-ка, были тематически
связаны с этим интересным мистическим летним праздником. Детям приходилось
«поломать голову» отгадывая загадки, собирая слова с недостающими слогами и
разгадывая зашифрованные  слова,  работать руками, создавая из цветной бумаги
прекрасных бабочек, а также разминать ручки-ножки, играя в спортивные игры. 

Выполнив на каждой станции задание, довольные  участники квеста наклеивали на
цветочную серединку лепесток определенного цвета. На финише все ребята отдали
свои «цветики-пятицветики» библиотекарям для единого «Купальского веночка» и
получили подарочный сертификат в «Игротеку» Центральной библиотеки им. В.Г.
Короленко.  Супер-приз – настольная игра «Эрудит» - досталась победителю,
определенного с помощью лотереи.  

Квест библиотекарей разнообразил праздник и добавил всеобщего позитива.

Материалы для квеста были предоставлены Центром культуры и досуга Старого Крыма
- организаторами фестиваля «Купальский венок», выигравшими грант по программе
«Громадський бюджет».  Игры предоставлены Центральной библиотеке им. В.Г.
Короленко Международной организацией по миграции (МОМ) в рамках реализации
проекта «MRPLife: Знайомимось, гуртуємось, розвіваємось».
