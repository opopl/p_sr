% vim: keymap=russian-jcukenwin
%%beginhead 
 
%%file 12_10_2020.news.pravda_com_ua.1_fedoriv_ukraine
%%parent 12_10_2020
 
%%endhead 

\subsection{Андрій Федорів: Україна --- це свобода, вольниця, бардак, хаос}
\label{sec:12_10_2020.news.pravda_com_ua.1_fedoriv_ukraine}

\url{https://www.pravda.com.ua/articles/2020/10/12/7269382/}

Кажуть, він різкий і категоричний. Якщо вважає, що проект --- лайно,
співпрацювати не буде.

За це його не люблять.

"Якщо ти думаєш, що світ --- це місце, де живуть виключно твої конкуренти, ти не
розвиваєшся сам. І не розвиваєш ринок", --- повторює він.

Те, що він має талант змінювати середовище навколо себе, визнають навіть ті,
хто його не любить.

Перелік того, що створив Андрій Федорів може бути задовгим. Він --- засновник
креативної інвестиційної компанії Fedoriv Group, маркетингової агенції Fedoriv
Agеncy, Фундації Дарини Жолдак, бізнес-простору Kooperativ, каналу Fedoriv Vlog
на YouTube, автор навчального курсу для власників бізнесу BrandFather.

У 2019 році спільно з Міністерством цифрової трансформації України його агенція
створила бренд цифрової держави "Дія", метою якого є оцифровка та автоматизація
всіх державних послуг.

Федорів --- один з найкращих в Україні фахівців з брендінгу. А ще він візіонер,
здатний відповісти не лише на питання "як?", але й на питання "куди?" і
"навіщо?".

Саме тому "Українська правда" запропонувала Андрію Федоріву розповісти, який
ребрендінг потрібен Україні і в чому наша конкурентна перевага.

В інтерв’ю УП Федорів розповів про Україну як стартап.

Про сильні ідеї, яким не будуть плескати в долоні, і які потрібно виборювати.

Про те, чому наші факапи є нашими можливостями.

Про те, чому шкодять порівняння на кшталт "Київ --- це новий Берлін".

Про прозорі стіни сучасного світу і про те, чому ребрендінг України не має
сенсу, допоки "Беркут" б’є людей.

Про своє бачення перемоги і фотографії майбутнього.

Обережно: спойлер. 

Під час розмови жодного разу не прозвучало слово "зрада". 

\subsubsection{"Україна --- це голодний підліток з рюкзачком"}

--- Пропоную рольову гру. Уявіть, що я ваша потенційна клієнтка. Прийшла до
вашого офісу і кажу: "Андрію, у нас велетенська корпорація. Майже 42 мільйони
співробітників. Величезні ресурси. Давні традиції. Але щось пішло не так.

Ми, як спільнота, не знаємо, куди йдемо і до чого прагнемо. Ми потребуємо
чарівного пенделю ззовні, щоб набути другого дихання, наростити м’язи і почати
завойовувати нові ринки. Нам потрібен ребрендінг. Залишаємо лише назву –
Україна".

Візьметеся за такий проєкт?

--- Чудово. Мені подобається така корпоративна рольова гра. Те, що ви описали, є
класичним ребрендінгом. Це коли компанія, що пройшла довгий шлях і має певні
активи, ставить собі саме ті питання, що й будь-який стартап.

Класичний ребрендінг --- завжди ре-брейнінг. Переосмислення. В цьому випадку ─
переосмислення України.

--- Тобто, ви погодилися би взятися за ребрендінг?

--- Не має значення. Ви прийшли до мене з пропозицією, яка потребує обговорення.

Будь-який ребрендінг --- в першу чергу історія про зміни.

Хто буде лідером змін? Хто ця людина, яка візьме на себе емоційну
відповідальність за проєкт?

Хто є команда, той спецназ, який буде втілювати ідеї?

Хто ми? В якому сегменті працюємо? Хто наші конкуренти? Проти кого ми граємо?
Хто забирає наші гроші?

Куди ми йдемо? Яка наша велика амбітна мета?

Як про неї розповісти, щоб люди захотіли до нас приєднатися?

Це перше, про що б я запитав.

А ще обов’язково попередив би: коли йдеться про зміни, готуйтеся до спротиву.
Будь-яка ідея, а особливо сильна, ризикована, радикальна ідея --- завжди
беззахисна. Вона як ембріон. Їй не будуть всі плескати в долоні. Її потрібно
виборювати.

- Які, на вашу думку, мають бути перші три кроки з ребрендінгу країни? 

--- З усією повагою до української історії та традицій, перше, що нам потрібно зробити, це визнати, що ми --- стартап. 

Визнати, що ми підліток, як країна, як держава, як економіка. До моменту, поки
ми намагаємось виглядати або діяти як хтось інший, в нас є проблеми.

Другий крок --- не намагатися реанімувати неефективні пострадянські гнилі
системи. Їх потрібно відключати. Не можна це міністерство зробити ефективнішим?
Окей, вимкнули і забули, що тут взагалі колись було міністерство.

Третій крок --- сфокусуватися на точках сили. Не намагатися бути
надуніверсальними. Мовляв, і в цьому, і в цьому ми розумні й гарні. Ні.

--- Ви часто повторюєте, що маркетинг --- це відповідь на просте питання: чим ви
відрізняєтеся від конкурентів? Чим ми, як спільнота, відрізняємося від інших? В
чому ця відмінність перспективна для позиціонування в світі?

--- Ми круто стрибаємо у висоту, бо в нас є Сергій Бубка.

У нас фантастичні архітектори, айтішники, у нас є музика. Питання як все це
інтегрувати в світовий контекст. Не намагаючись бути тими, ким ми не є.

В Україні нереальний сервіс, який можна отримати завдяки підприємцям.

Коли ми приходимо в Києві в будь-який ресторан і говоримо: будь ласка, можна
оцю страву без цього або без цього, чи можна нам просто зробити спагеті, в
дев’яноста відсотків зі ста скажуть: звичайно!

Коли ми з дітьми приходили до ресторанів у Нью-Йорку, нам казали --- неможливо.

Особистий випадок, що яскраво ілюструє відмінності.

Моя дружина лікувалася в шпиталі в Берліні. Я був поруч. Різко змінилася
погода, я раптом став почуватися жахливо. Дуже боліла голова, стало якось зле.
Ми покликали медсестру, попросили виміряти мені тиск. В неї в руках був
тонометр. Вона відповіла: "Ні. Бо ви не є пацієнт. Пацієнт --- ваша дружина. А ви
звертайтесь до лікаря".

Приватна клініка, за яку ми сплатили. Гроші не працюють. Працюють правила.В
Україні цього немає. 

В нашому ДНК немає культу раз і назавжди написаних правил світоустрою.

Ми стартап, і ми стартуємо знизу, а не зверху. Як кажуть інвестори, у нас
низька база.

Так, з моменту рестарту України в 1991 році ми багато часу змарнували. Або
нічого не робили. Або руйнували те, що було. Або забирала час необхідність
прийти до тями. Я, до речі, не впевнений, що ми до неї повною мірою прийшли.

Але є й хороші новини. Те, що ми не зробили колись --- це наші можливості
сьогодні. 

Знаєте, є такий вислів: "Чим гірше, тим краще". Чому? Уявіть: приходить прораб
на замовлення в будинок. А там все ідеально. Все, що йому потрібно зробити, це
підключити кілька розеток. Чи це добре? Для будинку і для його мешканців
напевне так. Для прораба --- ні. Бо він залишається без роботи.

--- Тобто наша головна конкурентна перевага в тому, що ми все ще на низькому
старті?

--- Неможливо на стартап накласти стандарти зрілої корпорації. Коли компанія
маленька і там працює 10-15 людей, у них не може бути чітко визначених
інструкцій і розписаних ролей. В будь яку мить можна сказати, окей, залишаємось
сьогодні на ніч. Ти менеджер, але ти сьогодні вночі будеш малювати макети. Бо
ми стартап з якостями стартапу.

Зрілі країни --- це пенсійного віку пани і пані. В них вже все є, в них є, що
втрачати, в них побудована інфраструктура.

А Україна --- це хлопець або дівчина з рюкзачком. В такої людини, окрім рюкзачка,
є лише одне --- мотивація, пов’язана з почуттям голоду. Це сильна мотивація. 

Чому стільки людей їдуть працювати за кордон? Тому що вони голодні. Немає
можливостей та альтернативи, є тільки ця мотивація.

Зрозумійте, не існує шорт-листу, в якому Україна стоїть поруч з Англією,
Німеччиною, США. Україна з інфраструктурної, технологічної, бізнесової точки
зору є недорозвиненою.

Мені здається, що ми часто використовуємо хибні порівняння, які стимулюють
грати проти сильних гравців на їхньому полі. Ми говоримо: "Ми майже як
Німеччина. Київ --- це новий Берлін".

Камон! По-перше, ми аж ніяк не Німеччина. Але й не треба казати "Ми –
недоНімеччина".

Нам потрібно шукати щось своє, свій осередок, своє повідомлення. Те, що ми
можемо запропонувати світові.

Ми граємо в іншій лізі. Це важко визнати, але це треба визнати.

Визнання цього вже є перемога.

--- Хто повинен це визнати?

--- Це складне питання. Я не вірю в зміни, що відбуваються в голові керівника.
Ефективні ті зміни, що відбуваються в головах критичної маси лідерів думок. Для
того, щоб це сталося, має бути чітка ідея, меседж, які спочатку підтримують
каталізатори, малі групи, після цього --- велика група, а згодом до неї
приєднуються решта. 

\subsubsection{"Фантастична їжа і креативний клас --- це вже точки опори"}

--- Питання, що ви згадали на початку розмови --- хто наші конкуренти? У вас є
відповідь?

--- На жаль, українське суспільство не є інтегрованим в світовий контекст. В
країни немає чіткого розуміння, що ми працюємо в глобальному конкурентному
середовищі.

Всі країни світу сьогодні конкурують за людей. Конкурують за туристів, багатих
і не багатих. За підприємців, які обирають країни для створення робочих місць
та інвестицій.

З кожним роком ця конкуренція стає більш публічною і цинічною. Людей хантять на державному рівні.

Це коли Меркель у відео звертається до всіх працьовитих і каже: "Німеччина
чекає на вас". 

Коли Португалія говорить: "Ось вам 500 тисяч євро згідно кредитної програми, і
ви наш резидент або громадянин" або "ось нерухомість, що дає шанси будь-кому
стати успішним".

Перш ніж починати ребрендінг, потрібно визначитися з конкурентами.

Якщо наш конкурент Німеччина, що ми хочемо забирати з Німеччини --- людей або інвестиції?

Або нашим конкурентом є Білорусь?

Наведу приклад. В Білорусі дуже крутий IT-кластер. Але зараз в країні виникли
політичні проблеми. Білоруські айтішники обирають Україну, бо тут їм наразі
вільніше працюється. В порівнянні з Білоруссю Україна --- фантастична країна.

Або. В Казахстані українські топ-менеджери вважаються суперінтелектуалами.
Казахи із великою повагою ставляться до української експертизи.

Це ті карти, які ми маємо покласти на стіл. Тверезо подивитися, в чому наша
сила, і в чому наша слабкість.

--- Одна з важливих ідей в сучасному маркетингу --- якщо якість продукту або
послуги --- лайно, ваш бренд не врятує жодна маркетингова стратегія. Що потрібно
змінити або посилити в продукті, що виробляє бренд "Україна"?

--- Я спілкувався з цього приводу з одним іноземним експертом. Він сказав
наступне: "Сучасний світ прозорий. Стіни --- прозорі. Якщо ви повісили гарний
логотип на скляну стіну, але крізь скло видно, як ваш "Беркут" б’є людей, ніхто
не зверне уваги на ваш красивий логотип".

Щодо продукту. Наш продукт не є незрілим. Але він потребує нових підходів.

Наприклад, в урбаністиці є термін ревіталізація. Наповнення новим життям старих
об’єктів. Kooperativ, де ми зараз спілкуємося, це колишня книжкова фабрика
"Жовтень". Тут друкували квиточки на тролейбус. Потім будівля стояла порожньою,
її багато разів приватизували, купували, продавали. 

Ми змогли орендувати три поверхи і дах на 10 років. Тут був залізобетон і все.
Зараз у нас працює більше 600 сучасних, молодих, успішних, усміхнених людей. В
красивих інтер’єрах. Кожен тиждень відбуваються події. Якби не коронавірус,
було б більше.

Подібної ревіталізації потребують готелі в Карпатах. Ресторани в Вінниці та
Херсоні. Куди не подивись, теоретично все можна зробити.

В Україні стільки роботи! Ви не уявляєте, скільки в Україні роботи. 

--- Як розказати світу про бренд "Україна" так, щоб він асоціювався не з
катастрофами та корупцією?

--- По-перше. Ми повинні визнати: Україну сьогодні сприймають в світі як
негативний бренд. В нас точка старту не нуль. Нижче нуля.

По-друге. Хороша стратегія бренду --- це іноді одне слово. Максимум 3-5 слів,
більше ніхто не запам’ятає.

Люди мислять поверхнево, шаблонами, в них немає часу розбиратися в нюансах.
Франція --- Ейфелева Башта, сири, вино, коханці. Швейцарія --- класні гори,
шоколад, годинник.

Україна? Чорнобиль, корупція, війна. Реальність сьогодні така. 

А має бути так. Україна? Фантастичні ресторани, природа, кайфові айтішники,
агро. 

Але говорити потрібно не про сільське господарство. Забудьте про сільське
господарство, ніхто не мріє мати справу з пострадянським колгоспником. 

Говорити потрібно про їжу. Їжа --- це світовий тренд. Їжа --- це мотивація. Їжа –
це перевага. Говоріть про натуральні продукти, які смачно готують в ресторанах. 

В нас є про що говорити. Фантастична їжа і креативний клас --- це вже точки
опори.

Ще питання, яке я одного разу ставив навіть в Кабміні. Чи готові ми в
комунікації говорити, наприклад, про ціни на їжу? Наші ціни --- це фантастична
перевага. Коктейль в крутому барі в центрі Києва коштує в десятки разів
дешевше, ніж те саме в Берліні.

В Європі всі рахують кожну копійку. Чи ми маємо визнати, що низькі ціни --- це
наша суперперевага? Ні, ми не маємо пишатися лише низькими цінами. Ми можемо
обрати інше формулювання: креативна їжа в поєднанні з низькими цінами.

\subsubsection{"Україна має стати Клондайком для підприємців, готових ризикувати всім, що мають"}
\subsubsection{"adsfdsf"}

--- Що заважає ребрендінгу України?

--- На класичний ребрендінг України відсутній замовник.

Щось замовить Міністерство закордонних справ, щось Міністерство культури. В
результаті ми отримуємо просто етикетку, наліпку, шпалери в новому дизайні.

Глибока історія, що насправді рухає вперед, не створюється.

Мені як інноватору бракує саме запиту держави на системну проривну ідею.

Якщо ми говоримо про класичну бізнесову історію, бренд-стратегія є донькою
бізнес-стратегії. 

Якщо ми говоримо про бренд країни, в його основі повинна бути державна
стратегія і національна ідея. Я не розумію зараз національної ідеї України.

"Ми йдемо в Європу". Хто йде, куди саме йде, з чим туди йде? Я не розумію.

--- Географічно Україна знаходиться в Європі, начебто йти немає куди в прямому
сенсі слів. Чи тут працюють приховані сенси?

--- Це формулювання я не вважаю правильним рішенням. Те, як я бачу Україну і те,
що мене надихає, можу описати так: Україна --- це країна з дуже молодою
енергетикою. Я маю на увазі не енергетичну сферу, а енергією, ви ж розумієте,
так? 

Україна яскрава, бурхлива, вона позитивна, вона жвава. Красива --- як нація. Всі
щось хочуть створювати. 

VISA і ROZETKA нещодавно проводили конкурс молодих бізнес-команд. 1200 заявок.
Коли я показував конкурсні відео в Німеччині, мені казали: це неможливо.

--- Чому? Молоді в Німеччині не цікавий бізнес?

--- Мені сказали так: підприємництво --- це вибір номер три. Ти можеш обрати роботу
на державу. Лакшері тобі не світить, але в тебе буде стабільне,
суперпередбачуване життя. Ти о п’ятій прийдеш з роботи, в тебе буде кайфова
прогулянка парком, в тебе повна соціальна захищеність.

Друге --- ти можеш піти в величезну корпорацію, не важливо в яку. Робити
зрозумілу і передбачувану кар’єру там. Ти знаєш, що буде 20 років потому, і
живеш відповідно до цього знання.

В Україні підприємництво --- це номер один чи два, це класна історія.

Питання --- чи є це державною політикою? Чи є це основою національного бренду?
Скільки мільйонів доларів ми вклали в те, щоб розповісти про це світові? Що
зробила держава, аби це підсилити?

Кілька разів ми намагалися працювати з державою на волонтерських засадах і
завжди отримували проблему. Одразу нас звинувачували в корупції. Відразу
починалися якісь лайки, сварки. Одразу з’являлася група людей, яка вишукувала
недоліки в нашій діяльності. Це швидко відбиває бажання співпрацювати з
державою.

Є цікавий приклад Південної Африки. В певний момент вони зробили бренд-комітет
країни, куди увійшли найкращі фахівці і найкращі агенції з маркетингу. Не
політики або громадські діячі, а фахівці, для яких маркетинг --- щоденна робота. 

Їм сказали приблизно таке: ми не хочемо, щоб на цьому полі ви конкурували між
собою. Ми хочемо, щоб ваші здібності стали точкою злиття ваших зусиль. Будь
ласка, об’єднайтеся і зробіть свою роботу разом.

Для України це була б хороша практика. Якщо ми маємо дезінтеграцію --- хтось
зробив логотип, хтось --- промо, хтось вигадав слоган --- все розсипається. Немає
простої ідеї, яка б все об’єднувала і пояснювала.

--- Своє бачення такої ідеї ви сформулювали одним словом, яке сьогодні вже
згадували --- свобода. 

Але свобода не існує сама по собі. Свобода завжди від когось або від чогось.
Історично Україна існувала і розвивалась так, що їй завжди було і є від кого
захищатись, вона вимушена виборювати собі волю. Чи не забагато драматичних
конотацій для національної ідеї в цьому слові? З точки зору маркетингу.

--- Розумієте, сильні бренди не йдуть проти течії. Вони фіксують органічні
ментальні речі і піднімають їх на п’єдестал. Сильний бренд обирає емоцію, з
якою хоче асоціюватися, потім розкриває її як поняття з різних боків.

Якщо ми маємо історію появи країни із Дикого Поля, якщо з 2004 року маємо дві
революції, що призвели до змін в суспільстві (я не маю на увазі їхні політичні
наслідки) --- це означає, що для України воля важлива. 

Україна --- це свобода, вольниця, бардак, хаос. В нашому випадку слова-синоніми.
Питання в тому, як спрямувати цю енергію. 

Питання в тому, що тобі скаже таксист, який буде везти від літака міжнародного
рейсу, і питання в тому, що ти йому відповіси.

Наприклад, скаже: "Чого ти сюди приїхав? Ми ж никакие".

Дуже хотілося б відповісти: "Ми --- "никакие"? Ми найвільніша країна Європи!".

--- Як цього добитися?

--- Повернутись до питання, що таке стратегія. А стратегія це відповідь на три
питання.

\begin{itemize}
	\item Перше. Де ми? Тобто визнати найприємніші факти про нас.
	\item Друге. Хто ми? Хто є наші конкуренти, в чому конкретно наші проблеми?
	\item Третє. Яка наша візія? Сформувати її, подумки, зробити фотографію майбутнього. Уявити, куди ми хочемо прийти.
\end{itemize}

В мене є улюблений анекдот про двох голлівудський режисерів. Один другого
запитує: "Як твій фільм?". Той відповідає: "Супер, фільм готовий. Залишилось
тільки зняти".

--- Скільки часу може взяти будування бренду Україна за умови, що робота над ним
починається сьогодні?

--- Дев’ять місяців стартапу і десять місяців реалізації.

--- Замовником ребрендінгу країни може бути лише держава?

--- Так.

--- Що особисто вам потрібно для того, щоб діставати з широких штанин синю
паспортину і відчувати щось на кшталт: "Дивіться і заздріть. Я --- громадянин"?

--- Підтримка творчості. Підтримка тих, кому це насправді потрібно. Максимум
справжніх, а не задекларованих, лібертаріанських підходів. 

Україна має стати Клондайком для підприємців, готових ризикувати всім, що
мають. 

Розмовляла Лариса Даниленко, для УП.
