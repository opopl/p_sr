% vim: keymap=russian-jcukenwin
%%beginhead 
 
%%file 18_12_2021.fb.menendes_enrike.1.kramatorsk_kiev.cmt
%%parent 18_12_2021.fb.menendes_enrike.1.kramatorsk_kiev
 
%%url 
 
%%author_id 
%%date 
 
%%tags 
%%title 
 
%%endhead 
\zzSecCmt

\begin{itemize} % {
\iusr{Антон Деч}
Киев после Питера выглядит, как обшарпанный бедный провинциальный городок.
Питер после Москвы выглядит как мелкий провинциальный городок.
Все познаётся в сравнении...

\begin{itemize} % {
\iusr{Андрей Викторович}
\textbf{Антон Деч} согласен.

\iusr{Энрике Менендес}
\textbf{Антон Деч} я не был в Питере, очень хочу однажды побывать. Киев для меня почти родной. Это моя страна. А так, мне и Москва нравится, и Берлин, и Вена.
Но люблю я только Донецк

\iusr{Антон Деч}
\textbf{Энрике Менендес} Киев мне не нравится по причине варварского отношения жителей к городу.
От прекрасных дореволюционных фасадов варварски изувеченных царь-балконами, граффити, аляповатой рекламой у меня начинается депрессия и нервозность на соматическом уровне. Особенно это втыкает после Питера. Это вообще кем надо быть, чтобы на фасад с майоликовым фризом воткнуть царь-балкон из вагонки и рубероида спилив фигурную кованную решётку.
Донецк я любил, так как вообще люблю города с промышленным населением.

\iusr{Сергей Основин}
\textbf{Антон Деч} теперь Донецк по сравнению с Краматорском выглядит как «обшарпанный бедный провинциальный городок».

\iusr{Антон Деч}
\textbf{Сергей Основин} Не знаю. После Майдана в Донецке не был. И вы, скорее всего, тоже...

\iusr{Лилия Винокурова}
\textbf{Антон Деч} мне после Дубая Италия показалась малость обшарпанной @igg{fbicon.grin} 

\iusr{Сергей Основин}
\textbf{Антон Деч} а мне и не надо. У меня там есть кому сравнить.  @igg{fbicon.smile} 

\iusr{Антон Деч}
\textbf{Сергей Основин} Политизированные люди под воздействием мозгобойной пропаганды постоянно выдают белое за чёрное.
Сам я, высказываюсь только о том, что видел сам. И вам того же желаю...

\iusr{Вадим Радионов}
\textbf{Anton Detch} киев после пригорода Киева, выглядит как большая деревня.

\iusr{Антон Деч}
\textbf{Вадим Радионов} Какого именно. Борисполя Броваров или Ирпеня?
Ставлю на Ирпень...

\iusr{Сергей Основин}
\textbf{Антон Деч} то есть сами себя вы считаете эталоном незаполитизированности. Понимаю.  @igg{fbicon.smile} 

\iusr{Антон Деч}
\textbf{Сергей Основин} Ни в коем случае. Когда я сужу о том, что видел сам, на это накладывается только моё субъективное восприятие.
Когда левый чел со слов других левых челов, судит о вещах, которые сам не видел - смысл такого суждения примерно ноль.
Поезжайте в Краматорск и Донецк. Сделайте репортаж. И я с удовольствием ознакомлюсь.

\iusr{Дмитрий Баевский}
\textbf{Anton Detch} если вы имеете в виду новостройки типа парнаса, то хорошо, что у нас провинция

\iusr{Антон Деч}
\textbf{Дмитрий Баевский} Я имел в виду центр, который просто варварски изуродован.
Чем плох Парнас? Туда в отличие от Троещины метро есть. И проблем с отоплением нет...

\iusr{Сергей Основин}
\textbf{Антон Деч} 

Я опираюсь на мнение знакомых и родных мне людей (не одного и не двух), которые
достаточно регулярно и по понятным причинам выбираются из Донецка на «материк».
В данном случае их мнение я просто транслирую, в чем и признаЮсь. И оно(мнение)
имеет ровно такой же вес, как и любое другое. Вы ведь для меня такой же «левый
человек», как и я для вас.  @igg{fbicon.wink} 

Что касается «сделайте репортаж» - спасибо, улыбнуло. Я вполне знаю, как
делаются подобные репортажи и насколько стОит им доверять.


\iusr{Антон Деч}
\textbf{Сергей Основин} Почему не доверять. Если частное лицо, а не проплаченный журналист куда-то едет и про что-то рассказывает, почему не доверять?
Это будут субъективные впечатления, но без материальной заинтересованности.

\iusr{Антон Деч}
\textbf{Сергей Основин} Лично я ценю впечатления от первого лица. По опыту, чьи-то впечатления, пропущенные через восприятие третьего лица полностью теряют объективность.

\iusr{IR Rena}
\textbf{Anton Detch} Та прям. Скажіть це тим, хто в Москві не був ніколи. Дешеві понти та бруд на кожному кррці. Північний Самарканд

\iusr{Сергей Основин}
\textbf{Антон Деч} ну, все же скорее не объективность, а просто своим глазам верится больше, чем чьим-то. Это - да. Согласен.

\iusr{Sergej Belous}
\textbf{Антон Деч} ну не сказал бы, что Питер выглядит мелким и провинциальным на фоне Москвы — скорее запущенным в плане благоустройства, чуть отойдешь от Невского — как будто в нулевые переносишься  @igg{fbicon.face.tears.of.joy} 

\iusr{Антон Деч}
\textbf{Sergej Belous} Это тоже не так. Дворы в Питере нормальные.
Москва берет развитием инфраструктуры.

\iusr{Sergej Belous}
\textbf{Антон Деч} я не про дворы, а улицы. Говорю о своём впечатлении.

\iusr{Антон Деч}
\textbf{Sergej Belous} Я бы не сказал. За фасада и следят.
Там есть несколько островков 90-х типа Апраксина двора. Но, вроде, проблему закрывают.

\iusr{Sergej Belous}
\textbf{Антон Деч} грязновато летом, а зимой снегом завалено. А в Москве моют/чистят оперативно.

\iusr{Ирина Целищева}
Смеющимся украинцам - вы очень далеки от реальности, отстали на много лет. Так и есть.

\iusr{Ольга Сидорова Сидорова}
\textbf{Anton Detch} расскажи по фашистов ))))

\iusr{Nata Platonova}
\textbf{Энрике Менендес} Лучше летом, конечно. Тогда, возможно, можно будет снять пуховик )) (шутка). Тут прекрасно.

\iusr{Svitlana Kovalova}
\textbf{Anton Detch} москвоцентричность

\iusr{Svitlana Kovalova}
\textbf{Irina Tselishcheva} если от вашей реальности, то это для нас удача

\iusr{Антон Деч}
\textbf{Svitlana Kovalova} На Украине гораздо большая киевоцентричность. Кажется 85\% авиа пассажиропотока идёт через Киев. Для Москвы эта цифра 40\%.
И все деньги тоже в Киеве. При этом Киев выглядит, как Саратов конца 90-х.
Я после последней поездки в Киев долго в себя приходил. Такое впечатление, что городом вообще не управляют.

\iusr{Ирина Целищева}
IR Rena Это говорит лишь о том, что вы давно не были в Москве. Впрочем, как и в крупных городах, столицах Западной Европы.

\iusr{Ирина Целищева}
Кто бывал везде и видел своими глазами, как Антон Деч - может сравнивать.

\iusr{Остап Петренко}
\textbf{Антон Деч} , постарайтесь взять себя в руки и, чтобы больше не расстраиваться, вообще не приезжать в Украину!

\iusr{Антон Деч}
\textbf{Остап Петренко} Понимаете в чем дело. Я езжу куда хочу и когда хочу. Что и вам желаю. Приезжайте хоть в Россию. Пишите любые мнения.
Это называется свобода.
Но людям из страны победившей демократии это не понять.

\iusr{Остап Петренко}
\textbf{Антон Деч} , да, смешно слышать разглагольствования о свободе от жителя автократии с полицейским режимом, где всё что можно запрещено, и с каждым днём всё больше запрещают. Мне ваша Россия не нужна, чего и вам желаю в отношении Украины. В общем, нэхай щастыть, в переводе для россиян!

\iusr{Ирина Целищева}
\textbf{Остап Петренко} Удивительная слепота к тому, что происходит в родной стране! К ограничениям свободы слова, к закрытию трех телеканалов, к беспределу и беззаконию, к режиму СБУ, к бесконечным делам, заводимых на людей из всех стран, только за то, что побывали в Крыму без разрешения.

\iusr{Антон Деч}
\textbf{Остап Петренко} Вы сейчас, сами того не понимая, демонстрируете образ мысли наиболее близкий к образу мысли классического раба в классической рабовладельческой системе.
Как вы узнаете, что ваша жизнь самая свободная, не побывав в других странах?
Удачи!

\iusr{Остап Петренко}
\textbf{Антон Деч} , да пошёл ты, холоп путинский

\iusr{Остап Петренко}
\textbf{Антон Деч} , твои ордынские собратья тебе лайки ставят, наяривают, плесень русскомирная, как грибница расползлась по миру

\iusr{Антон Деч}
\textbf{Остап Петренко} Ордынцы они такие. Ползают по всему миру! Ко всему испытывают интерес. Люблю я этот народ!

\iusr{IR Rena}
\textbf{Anton Detch} Якщо ви знаєте в Україні лише Київ, то це не означає, що я не можу літати куди завгодно зі свого міста )))

\iusr{Svitlana Kovalova}
\textbf{Anton Detch} я об этом и говорю, все с Москвой сравниваете.

\iusr{Антон Деч}
\textbf{Svitlana Kovalova} Что плохого в Москве? Например, Москва за три года приростает примерно на метрополитен Киева.
В Москву жить работать едут со всего мира. Только американцев там уже под сто тысяч. Внутри страны создан город с высокой концентрацией труда капитала и технологий. С крайне высокой конкуренцией, поражающей воображение инфраструктурой и транспортной системой.
Жителю Киева полезно туда съездить.

\iusr{Vlad Diachenko}
\textbf{Anton Detch} для нас \enquote{очень важно} мнение жлоба из путинбурга.

\end{itemize} % }


\iusr{Ἀθηνᾶ Σερβαρλι}
 @igg{fbicon.flame}  @igg{fbicon.hands.applause.yellow} Мирный, процветающий Донецк! Да будет так!

\begin{itemize} % {
\iusr{Storozhenko Sergey}
В составе Украины

\iusr{Остап Петренко}
\textbf{Storozhenko Sergey} , но без каких-либо особых условий

\iusr{Лилия Беликова}
\textbf{Storozhenko Sergey} В Хошимине хорошо? Ну и живите! А мы сами разберемся.

\iusr{Лилия Беликова}
\textbf{Остап Петренко} В джуржулеште и условия выдвигайте!

\iusr{Storozhenko Sergey}
\textbf{Lilia Belikova} Вы всем рассказали, что делать? Свободны.

\iusr{Storozhenko Sergey}
\textbf{Остап Петренко} На равных с остальными областями, конечно.

\iusr{Storozhenko Sergey}
\textbf{Lilia Belikova} Да вы наразбирались уже.

\iusr{Остап Петренко}
\textbf{Лилия Беликова} , так я и Джурджулештах никаких особых условий не требую! Таити, Таити... Нас и в Джурджулештах неплохо кормят!

\iusr{Евгения Савицкая}
\textbf{Storozhenko Sergey}
Не реально.

\iusr{Storozhenko Sergey}
\textbf{Евгения Савицкая} Нужно просто немножко поразмыслить и захотеть.

\iusr{Евгения Савицкая}
\textbf{Storozhenko Sergey}
Не захотят. Пора уже с этим смириться. И принять.

\iusr{Остап Петренко}
\textbf{Евгения Савицкая} , правильно, \enquote{просто нужно немного потерпеть}

\iusr{Евгения Савицкая}
\textbf{Остап Петренко}
Кому потерпеть?! Натерпелись уже за 8 лет!

\iusr{Storozhenko Sergey}
\textbf{Евгения Савицкая} Натерпелись потому что не думали и не хотели.

\iusr{Евгения Савицкая}
\textbf{Storozhenko Sergey}
Если кастрюля мешает мыслить, её следует снять.

\iusr{Storozhenko Sergey}
\textbf{Евгения Савицкая} Так снимайте скорей, восемь лет как-никак. Щи прокисли давно уже.

\iusr{Остап Петренко}
\textbf{Storozhenko Sergey} , там кокошник снимать надо, но он намертво прикипел

\iusr{Storozhenko Sergey}
\textbf{Остап Петренко} Потому что кастрюлей прижат
\end{itemize} % }

\iusr{Станіслав Федорчук}

Гудіти і говорити - різні речі. Тебе хтось призначив «голосом Донбасу»?
Скількох людей ти представляєш, крім власної родини?

\begin{itemize} % {
\iusr{Игорь Куприн}
\textbf{Станіслав Федорчук} Краматорск поддерживает!

\iusr{Станіслав Федорчук}
\textbf{Игорь Куприн} що саме підтримує? Порожню балаканину про повернення Донецька на умовах капітуляції перед Москвою?

\iusr{Illia Victorovich}
\textbf{Игорь Куприн} вам Краматорск лично сообщал о поддержке? Хочу напомнить, что точно также Краматорск совсем недавно кизяка Бабая \enquote{поддерживал}

\iusr{Сергей Проскуров}
федорчукам завидно. )

\iusr{Вадим Щетинин}
\textbf{Станіслав Федорчук} хто призначав... хто призначав....
Ахметка дал вказівку погудіти.

\iusr{Ирина Целищева}
\textbf{Illia Victorovich} А я вот совсем по-другому этот комментарий читаю. Что автор Энрике поддерживает Краматорск, где живет и работает сейчас. Ведь вопрос был про Энрике, кого представляет.

\iusr{Illia Victorovich}
\textbf{Irina Tselishcheva} тогда совсем мелко получается - кумир их вон страусов поддерживал, а тут всего лишь Краматорск)
\end{itemize} % }

\iusr{Александр Лапин}

В спальниках Киева всё то же самое, что и в регионах: ходишь в один
супермаркет, в одну пиццерию и т.д.  @igg{fbicon.wink}  А в центре Киева, в-основном, живут
приезжие, местные коренные давно съехали от шума и выхлопных газов в пригород.
Да и мира в центре маловато - постоянно какие-то акции протеста, драки с
полицией, избиения неформалов и зигующие малолетки, рисующие свастики на стенах
домов на Хрещатике. Хоть этого г..вна, уверен, в Краматорске нет.

\begin{itemize} % {
\iusr{Людмила Кивалина}
\textbf{Александр Лапин} почему такой бред пишут те, кто очень далёк от центра Киева и Киева вообще?

\iusr{Александр Лапин}
\textbf{Людмила Кивалина} , я добираюсь до Хрещатика за полчаса  @igg{fbicon.wink} 

\iusr{Артур Нискубин}
\textbf{Александр Лапин} Сказочник)

\iusr{Александр Лапин}
\textbf{Артур Нискубин} , 15 минут до метро и 15 на метро. Что тут сказочного?

\iusr{Людмила Кивалина}
\textbf{Александр Лапин} и за эти полчаса наблюдаете кучу свастики, избиений, стычек с полицией, зигующих малолеток? Или сейчас просто поддержать надо нужный образ для тех, кто в Киеве не живет?

\iusr{Александр Лапин}
\textbf{Людмила Кивалина} , избиения неформалов регулярно наблюдаю на Подоле, в районе Контрактовой площади, а марши, факельные шествия и прочие акции протеста - это Майдан, Грушевского и так называемый «правительственный квартал». Пять свастик малолетки буквально неделю назад на Хрещатике рисовали, снимая видео, и никто из полиции даже не почесался. На Хануку прямо на Майдане какой-то «патриот» разбил Ханукию. Это же всё в новостях было, есть и фото и видео. Ну об акциях протеста я вообще не говорю, каждый раз как минимум, полька 9 баллов, даже если мирно...

\iusr{Сергей Проскуров}
\textbf{Людмила Кивалина} выйдите на Подол (от Почтовой до Контрактовой) вечером. И воочию самостоятельно всё это узрите.

\iusr{Артур Нискубин}
\textbf{Александр Лапин} Сказочник вы по факту.

\iusr{Вера Победенова}
Сергей не устаёте из Канады на Подол выходить?

\iusr{Людмила Кивалина}
\textbf{Александр Лапин} я живу именно в центре, регулярно передвигаюсь, и мы наблюдаем какие-то два разных Киева

\iusr{Александр Лапин}
\textbf{Артур Нискубин} , да нет, вот ваш шеврон нашёлся, носите с гордостью  @igg{fbicon.wink} 

\ifcmt
  ig https://scontent-frt3-2.xx.fbcdn.net/v/t39.30808-6/268822216_137647408648450_2234761278274069866_n.jpg?_nc_cat=103&ccb=1-5&_nc_sid=dbeb18&_nc_ohc=GOVi8NNyEEsAX9lxSed&_nc_ht=scontent-frt3-2.xx&oh=00_AT88bAFmRbq908OxUceuYYbqOujsuK8JFOcs57T-7wy8lA&oe=61E3A979
  @width 0.2
\fi


\iusr{Сергей Проскуров}
\textbf{Вера Победенова} устаю. В основном, от таких, как вы, умников. Но приходится.

\iusr{Александр Лапин}
\textbf{Людмила Кивалина} , охотно верю. У «патриотов» свой Киев, с факельными шествиями, СУГСом и «сафари» на участников Маршей равенства. Только в последние годы полиция на сто процентов стала защищать от радикалов и то один день в году.

\iusr{Людмила Кивалина}
\textbf{Александр Лапин} я не очень понимаю ярлыков в кавычках, что именно вкладывается в это понятие, но в очередной раз убеждаюсь, что у каждого свой глобус.
И если такие вот люди - ца, то печаль вдвойне

\iusr{Антон Морозов}
\textbf{Александр Лапин} вы давно были в Киеве?

\iusr{Остап Петренко}
\textbf{Людмила Кивалина} , всё понятно в вашем комменте, но что это за \enquote{ца}, что-то из \enquote{Кин-дза-дзы}?

\iusr{Александр Лапин}
\textbf{Антон Морозов} , я из него и не уезжал  @igg{fbicon.face.tears.of.joy} 

\iusr{Александр Лапин}
\textbf{Людмила Кивалина} , 

а что тут вкладывать? Есть люди, считающие, что они вправе решить за других.
Взять и запретить родной язык других людей, точно таких же граждан во всех
сферах жизни, кроме кухни. Взять и возвеличить свою версию истории, унижая
историю других, и так далее. И всё это под соусом «патриотизма». Потому что для
меня патриотизм - это в первую очередь, уплата налогов в своей стране, а не
горлопанство.


\iusr{Остап Петренко}
\textbf{Александр Лапин} , 

а для меня патриотизм - это в первую очередь готовность идти защищать свою
Родину, будущее своих детей с оружием в руках, а если воевать не можешь по
каким-либо причинам, то всячески помогать тем, кто Родину защищает. А налоги
платить и захватчикам можно, захватят - и точно так же будешь им продолжать
платить, только тогда уже и пикнуть что-то против не сможешь, как крымчане
сейчас - живут и языки в з"дницы попрятали, и хотели бы попротестовать по
разным причинам, да боязно, это им не украинская вольница

\iusr{Александр Лапин}
\textbf{Остап Петренко} , мы же вроде бы договорились, что пока не покажете свою мордочку, я с вами не общаюсь  @igg{fbicon.wink} 

\iusr{Остап Петренко}
\textbf{Александр Лапин} , а я с вами и не общаюсь, я с вашими комментами общаюсь!

\iusr{Александр Лапин}
\textbf{Остап Петренко} , тогда продолжайте дальше с ними общаться, но без надежды на взаимность  @igg{fbicon.face.tears.of.joy} 

\iusr{Ольга Сидорова Сидорова}
\textbf{Александр Лапин} да в Киеве столько фашистов негде яблоку упасть ))))) как страшно жить , вокруг свастики и черепа(((((.

\iusr{Ольга Сидорова Сидорова}
\textbf{Александр Лапин} это кто говорит что украинцев не существует не твой ли фюрер Путлер . Найди мне в советской истории такие суждения

\iusr{Ольга Сидорова Сидорова}
\textbf{Александр Лапин} скажите если в Украине хунта, свастистики, почему вы представитель руссофашисткого отребья не в тюрьме ????

\iusr{Александр Лапин}
\textbf{Ольга Сидорова Сидорова} , 

вот скажите, зачем вы меня оскорбляете? Провоцируете ответить вам так же
по-хамски, чтобы подать жалобу? К сожалению, за семь лет в Киеве я видел в разы
больше свастик, нацистских рун и зигующих молодчиков, чем за 29 лет в Донецкой
области, где я эту дрянь видел только на исторических фото в книгах или в
фильмах. А здесь улицы переименовывают в честь коллаборантов с нацистами, не
говоря уже об уличном ультра-правом насилии...

\iusr{Ольга Сидорова Сидорова}
\textbf{Александр Лапин} 

колаборанти это боевики Лугандонии. А то что было 100 лет назад это просто
история. И колаборанты Лугандонии убивают украинских соллат. В 1941 году
интернета не было и когда после советских мракобесов которые убили и вывезли в
Сибирь тысячи украинцев ( за украинский язык, за то что не хотели колхоза и
т.д. ) и тут приходят немцы и они им кажутся освободителями после 2 лет НКВД
шного кошмара . А сейчас все знают кто такие боевики Лугандонии, как они
расстреливают пленных по законам 1943 года и т.д. Тебе смешно? А матерям
которые похоронили молодых сыновей и дочерей которых расстреляли Гиркины по
законам 1943 года не смешно

\iusr{Александр Лапин}
\textbf{Ольга Сидорова Сидорова} , 

Нюрнберский трибунал с вами не согласен. Кстати, события Второй Мировой
происходили как минимум 76 лет назад, но никак не сто. И военные преступления
не имеют срока давности. Украинского коллаборанта с нацистами Ивана Демьянюка
судили и в Израиле, и в Германии. В последний раз его взяли под стражу в 89
лет, за два года до смерти. На Нетфликсе есть отличный сериал о нём - «Дьявол
по соседству». И когда-нибудь так будут преследовать и судить каждого военного
преступника этого вооружённого конфликта, причём с обеих сторон.


\iusr{Ольга Сидорова Сидорова}
\textbf{Александр Лапин} там судили ОУН, УПА???

\iusr{Ольга Сидорова Сидорова}
\textbf{Александр Лапин} ну да будут сулить Путина и главарей Лугандонии

\iusr{Александр Лапин}
\textbf{Ольга Сидорова Сидорова} , 

также руководителей и бойцов НВФ (в том числе, так называемых «добробатов») с
обеих сторон, и украинских военных, допустивших применение оружие
неизбирательного действия в условиях городской застройки, тех, кто незаконно
лишал свободы, пытал и убивал мирное население ПО ОБЕ СТОРОНЫ от линии
разграничения. На скамье подсудимых должны быть не только Путин с теми вождями
«Л/ДНР», кто доживёт до суда, но и Порошенко с Турчиновым, а также все другие
руководители силового блока нашей страны с того периода по день, когда
прекратятся обстрелы городской застройки.

\iusr{Александр Лапин}
\textbf{Ольга Сидорова Сидорова} , 

ну и это по меньшей мере некорректно редактировать свой комментарий после того,
как оппонент уже выставил оценку его первоначальной редакции, добавив скорбных
подробностей, и затем обвинять оппонента в бесчувственности...

\iusr{Ольга Сидорова Сидорова}
\textbf{Александр Лапин} 

Порошенко, Аваков и Турчинов должны отбывать наказание в первую очередь за то
что не обеспечили полицейскими зачистку от мракобесов которые не давали проходу
военным, ну и за всяких Кив, Дейдеев, Семенченко, которые решали там шкурные
вопросы, особенно Кива, кторый вместе с Аваковым заработал миллионы на
наркотрафиках

\iusr{Ольга Сидорова Сидорова}
\textbf{Александр Лапин} в вашей России сафари на лгбт устраивает вообще государство

\iusr{Александр Лапин}
\textbf{Ольга Сидорова Сидорова} , 

ну мы же сейчас обсуждаем военные преступления, а не должностные или связанные
с оборотом наркотиков. Если у вас есть доказательства преступлений
вышеуказанных лиц, обратитесь в соответствующие органы, уверен, как минимум, в
отношении Кивы и Семенченко, эти доказательства под сукно не положат, они и так
сейчас фигуранты дел.

\iusr{Александр Лапин}
\textbf{Ольга Сидорова Сидорова} , 

сколько раз повторять, что живу в Украине? Вы в состоянии это запомнить? Или с
вашего аккаунта разные боты пишут? Так напишите памятку для коллег и
присобачьте её на монитор, а то каждый раз заново приходится знакомиться  @igg{fbicon.face.tears.of.joy} 


\iusr{Александр Шадрин}
\textbf{Александр Лапин}, видел недавно в Конотопе перед судом акцию Нацкорпуска. Сняли видосик, пофоткались и тихонько поснимали курточки с надписями.
Видно боятся местных небайдужих громадян))

\iusr{Ирина Целищева}
Вот это альтернативно одаренный уникум Ольга Сидорова Сидорова коза  @igg{fbicon.smile}  Явный бот.

\iusr{Остап Петренко}
\textbf{Ирина Целищева} , 

лучше уж бот, но украинский, чем комментатор сующий свой нос из далёкого
Екатеринбурга, откуда, помню видео 14-15 гг, отправлялись безмозглые простофили
воевать на Донбасс против украинцев, а их ещё православный поп московского
патриархата благославлял на это \enquote{святое дело}, ск@ты, бд!

\iusr{Ирина Целищева}

У меня достаточно друзей в Донецке, не ботов, которым я искренне сопереживаю. А
о Екатеринбурге вы, судя по всему, НИЧЕГО не знаете! Энрике Менендес, с такой
Украиной как Ольга Сидорова Сидорова вам не вернуть мирный и процветающий
Донецк.

\iusr{Остап Петренко}
\textbf{Ирина Целищева} , да мне и на фиг не нужен ваш Ебург, впрочем, как и остальные российские города

\iusr{Nikolaus Haufler}
\textbf{Александр Лапин} 

в центре Киева соотношение цена/качество жизни одно из самых высоких в Европе.
Я живу в хорошем районе Берлина, в этом году прожил в центре Киева два месяца,
кроме пробок не на что пожаловаться. Акции и демонстрации в повседневной жизни
не мешают, даже если жить прямо у Майдана.


\iusr{Александр Лапин}
\textbf{Nikolaus Haufler} , 

ну это кому как. Если нужно ехать в центр на машине, а там очередная акция
протеста, тем более, если с перекрытием улицы, как это у нас часто любят делать
- то хочется самому расстрелять этих \enquote{активистов} стоимостью 200 грн. из
реактивного г..вномёта  @igg{fbicon.wink} Если работаешь на улице Богомольца, скажем, и идёшь от
выхода из метро \enquote{Хрещатик}, который на Институтской, то минимум один митинг
\enquote{вкладчиков} будет под Нацбанком, с биотуалетами и автобусами, а второй на
самой Богомольца перед МВД. Вот через две, а то и три толпы приходится
протискиваться. А ещё может быть митинг на углу Орлика и Липской перед
Верховным Судом. Вот определили бы единственное разрешённое место для митингов
и собраний на Гидропарке - всем бы было хорошо. И \enquote{активистам}, которых никто
бы не разгонял, пусть себе беснуются на острове. И горожанам, которые бы
вздохнули с облегчением после семи лет буйства \enquote{народовластия}.

\iusr{Nikolaus Haufler}
\textbf{Александр Лапин} я пока не научился отличать обычные ежедневные пробки от пробок, связанных с акциями протеста)

\iusr{Александр Лапин}
\textbf{Nikolaus Haufler} , 

обычные пробки 7-8 баллов в часы пик, 9 - если дождь/снег/аварии. И длится это
максимум 2-3 часа. А если акция протеста с перекрытием улиц, то 9 баллов, пока
не расплатятся организаторы с последними «активистами». Если акция
анонсирована, то в это время лучше в центр не соваться. А по вечерам
националисты устраивают охоту на неформалов (волосы «не того цвета», пирсинг,
ЛГБТ и всё, что вызывает ненависть скрепных нациков) на Подоле, могут вломиться
в кафе и и избить его посетителей (как это неоднократно бывает в «HVLV»). Так
что, как говорится, не путайте туризм с эмиграцией...


\iusr{Nikolaus Haufler}
\textbf{Александр Лапин} 

я часто бываю в HVLV и в курсе всех этих событий, но согласитесь, что из многих
тысяч жителей центра своими глазами увидели всё это совсем немногие.
Повседневная жизнь в центре Киева никак не включает себя быть свидетелем
избиений в кафе, это вообще не фактор при оценке жизни в центре.

А плюс туриста — возможность сравнения. В любом большом городе Европы каждые
выходные больше происшествий, полиции и демонстраций, чем в центре Киева. Центр
Киева — один из самых безопасных центров городов Европы с сравнимым населением.

\end{itemize} % }

\iusr{Александр Богданов}
Это называется Родина.
Поэтому я в Украину и не приезжаю 14 лет, ту Родину разграбили.

\begin{itemize} % {
\iusr{IR Rena}
\textbf{Александр Богданов} І не приїжджайте.

\iusr{Остап Петренко}
\textbf{Александр Богданов} , нехай щастить!
\end{itemize} % }

\iusr{Вячеслав Беленький}

Не путайте туризм с эмиграцией (с)

В Киеве на \enquote{ПМЖ} точно так же будете ходить \enquote{в один супермаркет (ну ладно, в
два), в одно кафе, в одну пиццерию, в одном месте покупаю овощи, в одном мясо и
в одном вино}

А в центре появляться раз в два года случаем зарулив и проползая в пробке.

В остальное время за версту не подходя.

\begin{itemize} % {
\iusr{Игорь Лесев}
\textbf{Вячеслав Беленький} ещё и в одну и ту же квартиру возвращаться, и ложиться на один и тот же диван с одной и той же женой) жуть!

\iusr{Вячеслав Беленький}
\textbf{Игорь Лесев} ... и одной и той же любовницей  ☠ ️ 

\iusr{Антон Морозов}
\textbf{Vyacheslav Belenkiy} в центре тоже самое, стараюсь никуда из него не выбираться.

\iusr{Ирина Целищева}
И в театры и на прочие культурные события в центр не ездят?
\end{itemize} % }

\iusr{Stanislav Tsykalovskyi}
Мажор.
На каждую потребность по магазину. Ты точно с Донбасса?
Потому что мы, простые рабочие парни, живем проще.
Много проще.
С одним магазином для всего.

\ifcmt
  ig https://scontent-frx5-1.xx.fbcdn.net/v/t39.30808-6/269312332_4965151883519102_2025575336092810632_n.jpg?_nc_cat=100&ccb=1-5&_nc_sid=dbeb18&_nc_ohc=yykH5LOa4tYAX-COjON&_nc_ht=scontent-frx5-1.xx&oh=00_AT--qP2FKf8qeBBsbRGJnn5DLCzZCgfVYrEfaKF_7f_dUA&oe=61E39966
  @width 0.3
\fi

\begin{itemize} % {
\iusr{Алена Тодорович}
\textbf{Stanislav Tsykalovskyi} А что такое ?)))) На Донбассе нельзя ходить в разные магазины и по разным потребностям?

\iusr{Stanislav Tsykalovskyi}
\textbf{Алена Тодорович} кому можно, кому не нужно
\end{itemize} % }

\iusr{Сергей Перевозчиков}
Хотеть не вредно.

\iusr{Алена Тодорович}
\textbf{Сергей Перевозчиков} Вам -да, не надо хотеть))) Вам и так сойдет, а другие сами решат.


\iusr{Санта Иванова}
Мне по душе всё же небольшие городки с инфраструктурой. Может со временем такое предпочтение

\iusr{Halyna Pry}
в Донецьк треба, в Донецьк! там не вистачає \enquote{голосу Донбасу} та \enquote{ответственных граждан}.

\iusr{Юрий Лукшиц}
Мне в Киеве не понравилось. Слишком шумный город. Хотя в Москве ещё хуже. Не представляю, как люди там высыпаются.

\iusr{Дмитрий Баевский}
Только надо понимать, что вы не голос Донбасса, а один из голосов. Голоса разные есть

\begin{itemize} % {
\iusr{Энрике Менендес}
\textbf{Дмитрий Баевский} 100\%

\iusr{Mykola Skyba}
\textbf{Дмитрий Баевский} Це - голос, який найбільше подобається. Відчувається справжність, любов до людей, до Донбасу і України в цілому, відсутність адженд.
\end{itemize} % }

\iusr{Vitalii Ovcharenko}
Голос ахметовского(!) Донбасса

\iusr{Алена Тодорович}
\textbf{Vitalii Ovcharenko} Идиот ?

Напишите ответ...

Нажмите ENTER, чтобы отправить комментарий.
\iusr{Alyona Dovgan}
А я вообще в селе живу. После 2 больших городов, конечно, скучновато (зимой), но зато соседи не сидят на голове и природа красивая.

\iusr{Кирилл Полевой}
Могли бы окучивать избирательный округ наездами из Киева. Семье ведь лучше в Киеве

\iusr{Елена Вишневская}
Это так же нереально, как вернуться в детство...

\begin{itemize} % {
\iusr{Энрике Менендес}
\textbf{Елена Вишневская} не думаю, что настолько нереально

\iusr{Ирина Целищева}
Комментарии убеждают, что нереально.

\iusr{Алена Тодорович}
\textbf{Ирина Целищева} каждый вправе думать, как ему хочется и живут все по -разному

\iusr{Алена Тодорович}
\textbf{Энрике Менендес} Не слушаейте их, идите своим путем и как душа ведет. она лучше знает
\end{itemize} % }

\iusr{Галина Куриленко}

НЕ поверите, но и в мегаполисах люди посещают одни и те же магазины, ближайшие
к дому)) А вот в культурной жизни возможности в столицах очень широкие и я
испытываю белую зависть к его жителям. Переехав в Киев, Вы потеряете духовную
связь с Донбассом, это точно. Миша Павлив, прожив в Киеве какое-то время, стал
такую пургу гнать, что пришлось от него отписаться. Будьте рядом, будьте с
нами!)

\iusr{Ирина Целищева}
В Донецке интеллектуальная жизнь уж точно богаче, чем в Краматорске.

\begin{itemize} % {
\iusr{Ирина Целищева}

Можете смеяться сколько угодно. Не знаю, как ходят люди в театры в Киеве, но в
Донецке действуют 4 государственных театра и филармония с консерваторией, и еще
с десяток самодеятельных театральных коллективов, и люди ходят на спектакли и
концерты. А еще картинная галерея и музеи.


\iusr{Ольга Сидорова Сидорова}
\textbf{Irina Tselishcheva} да там корифеи русской культуры Моторола , Бабай ))))

\iusr{Ирина Целищева}
\textbf{Ольга Сидорова Сидорова} Не позорьтесь. Вы далеки от реальности. Академические театры оперы и балета, музыкально-драматический, юного зрителя и кукол. И кстати еще картинная галерея - художественный музей.

\iusr{Ольга Сидорова Сидорова}
\textbf{Irina Tselishcheva} вы не считаете их корифеями?

\iusr{Ирина Целищева}
Я считаю, что вы забыли в своем нике концовку \enquote{коза}. )))

\iusr{Ольга Сидорова Сидорова}
\textbf{Irina Tselishcheva} моторолле цветы носите ? Ответьте

\iusr{Евгения Савицкая}
\textbf{Ольга Сидорова Сидорова}
Вы бы помолчали лучше.
\end{itemize} % }

\iusr{Елена Лобанова}
Кому хотите вернуть?

\iusr{Александр Суховерхов}

Это удивительно! То есть вы хотите откатить назад?, типа революция гидности
нечто случайное, такое назойливое, от чего можно отмахнутся?? Что должно
произойти чтобы такие как вы поняли, что вы живёте в новом
государстве, государстве цивилизационного выбора (Бандерстан) и прошлого не
вернуть? Что должно произойти чтобы такие как вы отвернулись от мечтаний вернуть
всё назад, отвернулись от Бандерстана и выбрали украинскую Новороссию?? Что? У
меня нет ответа, у московской интеллектуальной элиты тоже. Кто знает?

\url{https://t.me/demjanjuki}

\begin{itemize} % {
\iusr{Александр Лапин}
\textbf{Александр Суховерхов} , 

а если мы хотим выбрать по-настоящему независимую Украину, не сателлит США, но
и не сателлит РФ. Лично мне импонирует пример Финляндии - член ЕС, но не член
НАТО, после войны с СССР, сейчас сохраняет нормальные отношения с РФ.

\iusr{Александр Суховерхов}
\textbf{Александр Лапин} 

Такая Украина невозможна, ибо Украины больше нет, после победы революций
гидности она развалилась на Бандерстан и украинскую Новороссию, где живут два
народа - демьянюки и украинцы. Вот это разделение.. .Это, пантеон героев(для
первых Бандера, для вторых Ковпак),9 мая(для первых поразка, так как ох предки
выбрали третий рейх, для вторых Победа), советский период историй(для первых это
оккупация и колонизация, для вторых подвиг, гордость и достижение, вершиной коих
была Мрия, ЮЖМАШ, Антонов и, однозначно, полёт Гагарина), русский язык (для первых
людей в принципе не может быть украинца с родным русским языком), и, самое
главное, ЦИВИЛИЗАЦИОННЫЙ выбор (для первых это выбор рабов (мысль Дацюка), выбор
себе благодетеля, что будет их содержать, учить и показывать путь в будущие, а
если что ни так воспитывать, для вторых это неприемлемо, ибо они Победители и
будушие творят своё, сами и им не нужен Хозяин, чтобы указывал) Чтобы их отличать
,первых предлагаю называть демьянюками, а вторых украинцами.

\iusr{Александр Лапин}
\textbf{Александр Суховерхов} , 

зря Вы так считаете. Истина, как обычно, посредине. Везде есть как адекватные
люди, так и неадекватные. Чёрно-белое мышление, конечно, многое упрощает, но
реальная жизнь - она более разноцветная  @igg{fbicon.wink} 


\iusr{Александр Суховерхов}
\textbf{Александр Лапин} Вот бы ваш комментарий прочитать через год, пять лет, десять лет.

\iusr{Александр Лапин}
\textbf{Александр Суховерхов} , нет ничего невозможного, пока мы живы  @igg{fbicon.wink} 
\end{itemize} % }

\iusr{Игорь Гомольский}
Вот со многим в Донецке не очень, но с культурной жизнью как-то даже слишком хорошо.

\begin{itemize} % {
\iusr{Ирина Целищева}
И я об этом написала. Но селянам смешно.

\iusr{Игорь Гомольский}
\textbf{Ирина Целищева}, им всегда смешно.
\end{itemize} % }

\iusr{Андриан Бережной}

Будучи в любом городе Украины посещаю одну сеть магазинов, покупаю привычную
торговую марку, хожу по знакомому маршруту (это в целях безопасности, почти все
города изучаю ночью, так удобно из-за транспорта). Многие хотят стать
правительством в эмиграции, понимая все плюшки этого положения, так что твое
высказывание \enquote{голос Донбасса от аборигена} считаю не особо хорошим рекламным
слоганом. Но братика своим голосом поддержу)))


\iusr{Евгения Савицкая}
Стесняюсь спросить...
А куда собираетесь вернуть Донецк?

\begin{itemize} % {
\iusr{Остап Петренко}
\textbf{Евгения Савицкая} , он наверное пропустил \enquote{...-ся в...}, ошибся

\iusr{Евгения Савицкая}
\textbf{Остап Петренко}
Ничего не поняла...

\iusr{Ирина Целищева}
В сегодняшнюю Украину не получится. \enquote{Хочу вернуться в мирный и процветающий Донецк.} - реальнее.

\iusr{Остап Петренко}
\textbf{Ирина Целищева} , да, вы правильно поняли, именно это я хотел сказать! Ну и слава Богу!
\end{itemize} % }

\iusr{Виктория Пасечник}
похоже, я живу в краматорске. ведь тоже хожу в один супермаркет, где покупаю все что в списке ))

\iusr{Алена Тодорович}
\textbf{Виктория Пасечник} Вы какая-то странная женщина)))) Так нельзя

\iusr{Михаил Михеев}

Энрике, спокойствие, обеспеченность, богатство материальное прямо противоречат
интеллектуальному богатству. Вы почти каждый день выдаете оригинальные мысли.
Для этого не нужно много магазинов. А Киев подписывает себе приговор. Мне его
не жалко

\begin{itemize} % {
\iusr{Евгения Савицкая}
\textbf{Михаил Михеев}
Мысль действительно оригинальная - вернуть Донбасс....

\iusr{Алена Тодорович}
\textbf{Михаил Михеев} Господи, чего только люди не выдумают ?))) Не примеряйте свое пальто на других .другим не мешает .
\end{itemize} % }

\iusr{Олександр Сосницький}
А що значить «вєрнуть»???
Ви багатьма постами заперечуєте факт російської окупації, наполягаючи на «гражданском характєрє».
Ну то просто живіть в Донецьку, раз ваши «гражданє» живуть там  @igg{fbicon.smile} 

\iusr{Rostyslav Pelekhovych}
Що значить \enquote{повернути}? Що за обезцінювання іншоі сторони? Вони мають право на свій Донецьк.

\iusr{Юлия Волчкова}

До сих пор помню, Вас Дедом Морозом в моем доме!!! Медвеженок на полке!!!
Спасибо за НГ в 14 году!!! Вы нас очень поддержали. И пусть, некоторые уникалы
брызжут слюной, но для моего, ребенка, испуганного взрывами тяжёлого оружия,
30 декабря было праздником!!! Спасибо.

\iusr{Андрѣй Полтавцев}
Как был провинциалом так и остался  @igg{fbicon.smile} 

\begin{itemize} % {
\iusr{Vadim Ivanov}
\textbf{Andriy Poltavtsev} Полтава - мегаполис )))))

\iusr{Андрѣй Полтавцев}
\textbf{Vadim Ivanov} у вас претензии ко мне или к Полтаве?  @igg{fbicon.smile} 

\iusr{Vadim Ivanov}
\textbf{Andriy Poltavtsev} я по поводу провинциальности )))

\iusr{Андрѣй Полтавцев}
\textbf{Vadim Ivanov} Вы альтерэго Менендеса?

\iusr{Vadim Ivanov}
\textbf{Andriy Poltavtsev} verus amicus est tamquam alter idem

\iusr{Андрѣй Полтавцев}
\textbf{Vadim Ivanov} оно и заметно что \enquote{похожий}  @igg{fbicon.smile}  Хотя скорее всего очередной бот

\iusr{Vadim Ivanov}
\textbf{Andriy Poltavtsev} длблдь

\iusr{Андрѣй Полтавцев}
\textbf{Vadim Ivanov} Ваша подпись без гласных? Самокритично  @igg{fbicon.smile} 

\iusr{Vadim Ivanov}
\textbf{Andriy Poltavtsev} мужчинка, вы показались себе очень смешным ? Вам показалось.

\iusr{Vadim Ivanov}
Андрюша , НЕ провинциал  @igg{fbicon.laugh.rolling.floor}{repeat=3} 

\iusr{Андрѣй Полтавцев}
\textbf{Vadim Ivanov} Завидуйте молча  @igg{fbicon.smile} 

\iusr{Алена Тодорович}
\textbf{Андрѣй Полтавцев} А вы из какой провинции? ))))

\iusr{Андрѣй Полтавцев}
\textbf{Алена Тодорович} А Вы тоже альтерэго Менендеса или просто бот?
\end{itemize} % }

\iusr{Лариса Пилашевич}
Так зависть или уверенность? Странный пост...

\iusr{Оксана Бучек}

Это только кажется, живя в Киеве точно так же вся инфраструктура
сосредотачивается в радиусе пары км от дома. Кстати, когда жила в Донецке, меня
расстояния меньше беспокоили, поехать к стоматологу, забрать продукцию или
отвезти документы клиентам в другой район было обычным делом, сейчас стараюсь,
чтоб все было рядом и поменьше передвижений и пробок, скорее воспользуюсь
доставкой, чем поеду на пол дня выстраивать

\iusr{Алексей Блюминов}
особенно богатая интеллектуальная жизнь в нынешнем киеве)))

\end{itemize} % }
