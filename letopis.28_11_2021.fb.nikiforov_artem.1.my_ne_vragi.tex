% vim: keymap=russian-jcukenwin
%%beginhead 
 
%%file 28_11_2021.fb.nikiforov_artem.1.my_ne_vragi
%%parent 28_11_2021
 
%%url https://www.facebook.com/Nikiforov.Artem/posts/4455807704487917
 
%%author_id nikiforov_artem
%%date 
 
%%tags druzhba,obschestvo,politika,strana,tv,ukraina,vrag,vrazhda
%%title Мы - не враги друг другу
 
%%endhead 
 
\subsection{Мы - не враги друг другу}
\label{sec:28_11_2021.fb.nikiforov_artem.1.my_ne_vragi}
 
\Purl{https://www.facebook.com/Nikiforov.Artem/posts/4455807704487917}
\ifcmt
 author_begin
   author_id nikiforov_artem
 author_end
\fi

Мы - не враги друг другу.

Одно из самых крупных заблуждений публичной политики Украины - «если воюете в
эфирах, то обязаны плевать друг другу в лицо и в жизни».

Нет, так не работает. 

\ifcmt
  ig https://scontent-frx5-1.xx.fbcdn.net/v/t39.30808-6/261995834_4455807621154592_6407111615623521924_n.jpg?_nc_cat=105&ccb=1-5&_nc_sid=8bfeb9&_nc_ohc=T1efTbFK1qsAX95bVLW&_nc_ht=scontent-frx5-1.xx&oh=a2457fb2042072ecee79795c3f2ddf17&oe=61ACC323
  @width 0.4
  %@wrap \parpic[r]
  @wrap \InsertBoxR{0}
\fi

Публичная политика была «придумана» для того, чтобы люди и социальные группы
переносили свое противостояние в «физически» безвредную плоскость.

Гладиаторские бои в свое время как раз и служили точкой выхода накопившейся
злобы: чтобы не воевали друг с другом все, воевало ограниченное число
«специально обученных людей».

К слову, гладиаторы не обязаны были ненавидеть друг друга и в свободное от
поединков время могли нормально общаться. 

Гладиаторские бои в современном мире - это парламентские дискуссии и ток-шоу.
Именно туда переносится весь запал противостояния и конкуренции. 

Мы часто слышим: «вот! В эфирах дерутся, а потом друг к другу на праздники
ходят». Да, дерутся. Да, ходят. Потому что в этом суть явления публичной
политики. 

Мы хотим, чтобы политики вне камеры били друг другу морды? Мы действительно
такие мелочные? 

Публичная политика - это площадка конкуренции идей и взглядов. Политики в этом
отношении - всего лишь рупоры. И это нормально. Уйдет один - прийдет другой. 

С политическими взглядами не рождаются. Политические взгляды формируются.
Первое, на что мы должны обращать внимание, это насколько качественно политик
делает свою работу по отстаиванию позиции доверявших ему свои голоса
избирателей.

Вопрос. Политик, не способный отстоять взгляды своих избирателей, но искренне
бьющий морды своим оппонентам в туалетах - хороший политик? Конечно, нет.

Да, у нас после эфиров обнимаются и националисты с ОПЗЖ, и Слуги народа с БПП.
Не потому что они плохие или нечестные, а потому что после шахматной партии
оппоненты жмут друг другу руки, а не огревают друг друга доской.

\ii{28_11_2021.fb.nikiforov_artem.1.my_ne_vragi.cmt}
