% vim: keymap=russian-jcukenwin
%%beginhead 
 
%%file 25_07_2020.fb.lnr.2
%%parent 25_07_2020
 
%%endhead 
\subsection{``ОХОТА НА РУССКИХ'': что делают западные наёмники на Донбассе.}
\url{https://www.facebook.com/groups/LNRGUMO/permalink/2875321085912780/}

«ОХОТА НА РУССКИХ»: что делают западные наёмники на Донбассе.

Вновь вернуться к теме иностранцев на Донбассе заставила информация о гибели
под Горловкой «Техаса» гражданина США Шона Фуллера, воевавшего за ВСУ.

Впрочем, вскоре выяснилось, что американский наемник жив и здоров, а за него
приняли другого украинского военного, внешне похожего.  После этого
командование ВСУ приняло решение отвести свой «иностранный легион» подальше от
линии соприкосновения.

Что же мистер Фуллер делал на Донбассе, так далеко от Родины?

Как известно, на стороне ДНР и ЛНР тоже воюет много иностранцев.  Не секрет,
что за свою службу они получают определенную плату, но она не настолько велика,
чтобы ради нее одной лезть под пули.

Так, немецкое издание DW публиковало историю про некоего Алекса Д., уроженца
Киргизии, который эмигрировал в ФРГ, а оттуда направился на Донбасс, где встал
на сторону ополчения. Зарабатывая в Германии 1100 евро, в ДНР он получал 150
евро. Скажем прямо, не те деньги, чтобы лишь ради них пойти воевать.  Нужен
дополнительный стимул, как минимум, в виде идеи.  Изначально на Донбассе идея
построения «Русского мира», социально-справедливого общества, была, но после
куда-то подевалась.

На «той» стороне тоже все неоднозначно.  Безусловно, есть там и свои «идейные»,
но идеи у них «не те».

Издание Guardian по этому поводу писало: По данным Hope Not Hate, по меньшей
мере два британца, как предполагается, выехали в охваченную войной
восточноевропейскую страну за последние месяцы, воодушевленные людьми,
связанными с батальоном «Азов», одиозным фашистским вооруженным отрядом.

В составе добробата «Киевская Русь» были замечены грузинские боевики.  Немецкий
Spiegel отмечал, что в рядах «Азова» воюют наемники из Германии и других
европейских стран, привлеченные своими украинскими единомышленниками-радикалам.

Упомянутый выше мистер Фуллер как раз из таких американских ультраправых,
симпатизирующих идеям Гитлера и методам СС.

Денег больших нищая Украина им платить не может, рассказывают про зарплаты в 4
тысячи гривен (порядка девяти тысяч рублей).

Поэтому на Донбассе давно говорят, что некоторые иностранцы приезжают на
«сафари на русских».

Фактически они получают от Киева индульгенцию на убийства и любые преступления
против местного населения. Американский наемник считался «гордостью ВСУ».

Помимо «любителей пострелять по русским» в Незалежной воюют и настоящие
профессионалы. Например, ЧВК Greystone, считающая своих бойцов лучшими в мире.
Очевидно, что их «труд» оплачивается уже по отдельному тарифу.  Ничего личного,
только бизнес.

Кроме того, Киев привлек много профессиональных инструкторов из США, Канады,
Великобритании и Израиля, которые обучают украинских солдат. В ДНР и ЛНР
считают, что высокопоставленные иностранные военные участвуют в разработке
операций против непризнанных республик.

Почву для подобной ситуации создал во многом лично президент Порошенко,
подписавший в 2015 году закон, разрешающий служить в ВСУ иностранцам и лицам
без гражданства
  
