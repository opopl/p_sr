% vim: keymap=russian-jcukenwin
%%beginhead 
 
%%file 24_01_2022.stz.news.ua.zaxid.1.lviv_detsad_kniga_skazki
%%parent 24_01_2022
 
%%url https://zaxid.net/lvivskiy_ditsadok_vipustiv_blagodiynu_knigu_kazok_n1534465
 
%%author_id balandjuh_oleksandra
%%date 
 
%%tags blagotvoritelnost,deti,detsad,kniga,lvov,skazka,ukraina
%%title Львівський дитсадок випустив благодійну книгу казок
 
%%endhead 
 
\subsection{Львівський дитсадок випустив благодійну книгу казок}
\label{sec:24_01_2022.stz.news.ua.zaxid.1.lviv_detsad_kniga_skazki}
 
\Purl{https://zaxid.net/lvivskiy_ditsadok_vipustiv_blagodiynu_knigu_kazok_n1534465}
\ifcmt
 author_begin
   author_id balandjuh_oleksandra
 author_end
\fi

\begin{zznagolos}
Виручені від продажу книжки гроші підуть на придбання апарату для плазмоферезу.
\end{zznagolos}

Львівський приватний дитячий садок «Фрузя» розпочав продаж благодійної збірки
казок «Подаруй казку», всі гроші від якого будуть перераховані
Західноукраїнському спеціалізованому дитячому медичному центру. За зібрані
гроші планують придбати апарат для проведення плазмоферезу «Гемофенікс», який
щороку рятуватиме життя 50 хворих дітей з західного регіону країни. Вартість
цього апарата – 775 тис. грн.

\ii{24_01_2022.stz.news.ua.zaxid.1.lviv_detsad_kniga_skazki.pic.1}

Цей апарат, як повідомили у Центрі, конче потрібний лікарні для порятунку дітей
з нирковою недостатністю, бронхіальною астмою, різного роду отруєннями тощо.
Цей, що є у Центрі, вже відпрацював свій термін придатності.

Долучитися до цього благодійного проекту може кожний, придбавши збірку казок
«Подаруй казку». Аби придбати дороговартісний апарат потрібно продати 3200
книжок. Вартість однієї - 325 грн. Станом на 24 січня продано 150 книжок.

Цю книгу казок писали співачка Оксана Муха із сестрою Уляною, співачка та
ведуча Наталка Карпа, музикант та саундпродюсер Роман Черенов (Morphom),
Олександр Фоззі Сидоренко (учасник гурту «ТНМК») та інші.

Діти і працівники дитячого садка «Фрузя» просять львів’ян та всіх небайдужих
підтримати їх, придбати книжку і стати одним із благодійників. Стати
благодійником і придбати книжку можна за посиланням.

