%%beginhead 
 
%%file 08_02_2023.fb.rojz_svitlana.kyiv.1.ya_z_brala_v_odnomu_
%%parent 08_02_2023
 
%%url https://www.facebook.com/svetlanaroyz/posts/pfbid023MB38advRiK3N95matTiKB157FhSe6BM5JEs2dQcQKYpFbEh6gBpza6FQ4dxWsWRl
 
%%author_id rojz_svitlana.kyiv
%%date 08_02_2023
 
%%tags psihologia,psihika,vojna,vojna.2022
%%title Я зібрала в одному місці проекти, які створювала під час війни із різними командами
 
%%endhead 

\subsection{Я зібрала в одному місці проекти, які створювала під час війни із різними командами}
\label{sec:08_02_2023.fb.rojz_svitlana.kyiv.1.ya_z_brala_v_odnomu_}

\Purl{https://www.facebook.com/svetlanaroyz/posts/pfbid023MB38advRiK3N95matTiKB157FhSe6BM5JEs2dQcQKYpFbEh6gBpza6FQ4dxWsWRl}
\ifcmt
 author_begin
   author_id rojz_svitlana.kyiv
 author_end
\fi

Я зібрала в одному місці проекти, які створювала під час війни із різними
командами. Всі вони зроблені з турботою про дітей та дорослих. Всі дуже прості
по формі. Всі можна використовувати дома, в школах та садочках, в роботі
психолога. В будь-який час. Всі вони створювались, щоб впоратись із
навантаженнями стресом, щоб додати стійкості в наших випробуваннях, відновити
близькість. Всі проекти вільному доступі. І все це робота команд. 

☀️ 1. «Стресостійке дитинознавство» - практикум для дорослих. (Брошура з
інформацією по різним темам, ігри) (макет \href{https://www.facebook.com/alexander.koreshkov.7}{Alexander Koreshkov} )
\url{https://drive.google.com/file/d/1AD4PKDvH-deylbhquN0Qz2ItJdH6tWEt/view}

☀️ 2. Стійкість для стійких – наш з \href{https://www.facebook.com/htytysh}{Halyna Tytysh} анімаційний проект, 20 відео з
тілесними практиками \url{https://bit.ly/3I3dk3r}

☀️ 3. Крокуй до укриття - пісня - руханка - для зняття напруги у дітей під час
повітряної тривоги (з \href{https://www.facebook.com/TNMKband}{ТНМК}, Олександр Корешков)

\url{https://youtu.be/DxS_dV9T55U}

☀️ 4. Практикуми стійкості та відновлення - прості тілесні вправи для дітей та
дорослих  - 64 відео - кожне по 15 хвилин. (\href{https://www.facebook.com/smartosvita}{ГО \enquote{Смарт освіта}}, 
\href{https://www.facebook.com/UNICEFUkraine}{UNICEF Ukraine})

Всі відео практикумів тут:

\url{https://bit.ly/3C6SV9O}

Також тут:

\url{https://bit.ly/3jxZ}

5. Дихальні картки - дихальні вправи для дітей для вгамування тривоги (із ГО
Смарт Освіта, Unicef Ukraine)

\url{https://www.facebook.com/171923206205552/posts/5708204639244020/}

Опис 

☀️ 6. Картки сили - практикум для дітей (картки з зображенням тварин, природних
явищ, різних об'єктів із пропозицією тілесних вправ, метафор сили, ігр) із ГО
Смарт Освіта та Unicef Ukraine

\url{https://www.facebook.com/171923206205552/posts/5508937402504079/}

☀️ 7. Твоя сила - практикум для дітей від 8-16 років. Збірка із 22 практик. Із
ГО Смарт Освіта та Unicef Ukraine \url{https://bit.ly/3WP5DlC}

☀️ 8.  Онлайн курс для освітян - повітряна тривога, там є багато алгоритмів і
практичних знань (із \href{https://www.facebook.com/eef.org.ua}{Фонд Східна Європа} )

\url{https://www.facebook.com/100000708430176/posts/5884458454921044/}

9. Онлайн курс для батьків \enquote{Як підтримати дитину під час війни}. (Із Фонд Східна Європа )

\url{https://courses.zrozumilo.in.ua/courses/course-v1:eef+EEF-024+May2022/about}

\ifcmt
  ig https://i.paste.pics/91dba7cebde61205827bd20c1bea170d.png
  @wrap center
  @width 0.6
\fi

10. Обійманці - практикум близькості (для відновлення контакту, повернення до
тілесності), . (Із ГО Смарт Освіта) Завантажити 12 карток і розгорток можна
тут: \url{https://bit.ly/3VbNZqR}

Стикерпаки з обійманцями для месенжерів

Viber: \url{https://bit.ly/3jgJFJu} 

Телеграм: \url{https://bit.ly/3BPvhhR}

☀️ 11. Ігри в темряві - 40 ігр, в які можна грати з дітьми, коли темно

\url{https://www.unicef.org/ukraine/stories/games-in-the-dark}

\ifcmt
  ig https://i.paste.pics/c7f9edfbcbdc6c38b0cdd983eeb14a37.png
  @wrap center
  @width 0.6
\fi
