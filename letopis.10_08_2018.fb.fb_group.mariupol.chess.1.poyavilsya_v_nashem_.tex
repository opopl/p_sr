%%beginhead 
 
%%file 10_08_2018.fb.fb_group.mariupol.chess.1.poyavilsya_v_nashem_
%%parent 10_08_2018
 
%%url https://www.facebook.com/groups/mariupol.chess/posts/1165939460231415
 
%%author_id fb_group.mariupol.chess,jarmonov_igor.chess.mariupol
%%date 10_08_2018
 
%%tags mariupol,mariupol.pre_war,chess
%%title Появился в нашем городском саду прекрасный уголок для шахматистов
 
%%endhead 

\subsection{Появился в нашем городском саду прекрасный уголок для шахматистов}
\label{sec:10_08_2018.fb.fb_group.mariupol.chess.1.poyavilsya_v_nashem_}
 
\Purl{https://www.facebook.com/groups/mariupol.chess/posts/1165939460231415}
\ifcmt
 author_begin
   author_id fb_group.mariupol.chess,jarmonov_igor.chess.mariupol
 author_end
\fi

Появился в нашем городском саду прекрасный уголок для шахматистов. Именно там
вчера состоялась встреча с юными воспитанниками Городского шахматного клуба
(группа Владимира Пономарчука) и КДЮСШ №4 (тренер - Кулак С.С.). Был проведен
конкурс по решению шахматных задач. Я предложил детям решить несколько своих
миниатюр. Шахматная миниатюра в композиции - задача или этюд с числом фигур не
более 7. Первую миниатюру, которая к слову была первой моей публикацией в
газете \enquote{Приазовский Рабочий} в 1982 г., ребята решали около получаса.
Они очень старались, интерес был необыкновенный! Расставив на трех столах
позицию, разбившись на группы, ребята живо и наперебой предлагали свои варианты
решения.  Наконец, самый опытный из них, Евгений Перцев, указал верный первый
ход.  Совместно были найдены все защиты черных и матующие ходы белых. Два
остальных орешка в виде миниатюр были расколоты значительно быстрее. И опять
отличился сильнейший! Три лучших решателя были награждены медалями и Грамотами!
Это Перцев Евгений, Пархоменко Дмитрий, Королев Иван. 

Приглашаем всех любителей шахмат приобщаться к занятиям!
