% vim: keymap=russian-jcukenwin
%%beginhead 
 
%%file poetry.rus.ruh_taras.realnist
%%parent poetry.rus.ruh_taras
 
%%url http://maysterni.com/publication.php?id=5702
%%author 
%%tags 
%%title 
 
%%endhead 

\subsubsection{Реальність}
\label{sec:poetry.rus.ruh_taras.realnist}
\Purl{http://maysterni.com/publication.php?id=5702}

Народе, подивись навкруги!
Хіба для цього наших пращурів кров лилась?!
Ах, чортові наслідники,
А ви всеж Клялись.

Дніпро так сумно тече...
Через краіну чорноти,
А біль за Украіну серце пече!
Що,Народе, не чуємо ми запах прірви.

Да піднєміть же свої очі
Невже до цього нас закликав святий Тарас?!
Що лежимо на зручному боці!
І не хвилює, що у душі та в голові в нас!

В Великий смітник перетворилась Украіна!
Хіба зникла любов?
Що байдужа стала земля рідна?!
...в мене у судинах вже кипить кров!

Неосвіченість панує нашим краєм
Лихослов"я в нашу мову пророста,
А ще про прекрасне життя розмовляєм
Не помічаючи як йдемо до цвинтаря!

Ми поскудимо наш рід козачій
Михайло,Софія,Андрію припадаю до вас чолом!
Лише здержіть цей темний вирій
Чи,Народе, може соромимось бути хохлом?

Народе, подивись навкруги!
Поля,села,міста,небо синіє...
Тай давайте за Украіну всі поставимо свічки.
І хай вона молодим деревом зеленіє!

Хай зеленіє в світі праці та пізнань!
І віддамо нарешті нашим пращурам дань!
