% vim: keymap=russian-jcukenwin
%%beginhead 
 
%%file 19_01_2019.stz.news.ua.mrpl_city.1.sosnovsky_take_krasyve_zhyttja
%%parent 19_01_2019
 
%%url https://mrpl.city/blogs/view/mariupolets-evgen-sosnovskij-take-krasive-zhittya
 
%%author_id demidko_olga.mariupol,news.ua.mrpl_city
%%date 
 
%%tags 
%%title Маріуполець Євген Сосновський: таке красиве життя
 
%%endhead 
 
\subsection{Маріуполець Євген Сосновський: таке красиве життя}
\label{sec:19_01_2019.stz.news.ua.mrpl_city.1.sosnovsky_take_krasyve_zhyttja}
 
\Purl{https://mrpl.city/blogs/view/mariupolets-evgen-sosnovskij-take-krasive-zhittya}
\ifcmt
 author_begin
   author_id demidko_olga.mariupol,news.ua.mrpl_city
 author_end
\fi

\begin{center}
\color{blue}\em
\textbf{Вже давно настав час писати не тільки про багате минуле Маріуполя, але й про
його унікальних сучасників. Дорогі читачі, саме з цього матеріалу
розпочинається серія нарисів, присвячених маріупольцям, закоханих у своє місто.
Ми будемо дізнаватися більше про тих, хто захоплює, надихає і завдяки своїй
творчості, діяльності змінює Маріуполь в кращу сторону.}
\end{center}

\ifcmt
  ig https://mrpl.city/uploads/posts/resize/w/750x750/uspht6aq9qsjxkm1.jpg
	@caption Фото Олега Пашка
  @wrap center
  @width 0.9
\fi

Зовсім не випадково все починається саме з Євгена Сосновського, адже він
здатний показати Маріуполь з кращих сторін. Фотограф невтомно працює не тільки
на користь рідного міста, але й продовжує змінювати думки багатьох українців
про Маріуполь. Завдяки його майстерності маріупольські будинки, вулиці
отримують нове життя. Євгена можна побачити майже на всіх міських заходах, його
світлини викликають захоплення і бажання переглядати їх знову і знову. Думаю,
буде цікаво нарешті познайомитися ближче з талановитим маріупольським
фотохудожником.

\textbf{Читайте також:} 

\href{https://mrpl.city/news/view/yadovitye-tsveta-i-vyveski-na-polfasada-pochemu-v-mariupole-nelzya-fotografirovat-istoricheskie-zdaniya-foto}{%
Ядовитые цвета и вывески на полфасада: Почему в Мариуполе нельзя фотографировать исторические здания?, %
Яна Іванова, mrpl.city, 30.07.2018%
}

З самого народження Євген живе у Маріуполі. Закінчив школу в 1982 році і
вступив тоді ще у Ждановський металургійний інститут на зварювальний факультет,
звідки після другого курсу був покликаний до армії. Але після армії знову
повернувся до інституту. Саме в інституті Євген познайомився зі своєю
майбутньою дружиною. І після закінчення навчання у них були не тільки два
дипломи і розподіл в один з маріупольських проектних інститутів, а й чудовий
піврічний син. Разом з дружиною працював в проектному інституті до 2015 року.
Сьогодні Євген працює приватним підприємцем. Найголовнішим для нашого героя є
його сім’я, яка завжди підтримає і зрозуміє. Рік тому він разом з дружиною
відзначив 30 років спільного життя. Головне їхнє досягнення – це, звичайно ж,
діти. Обидва закінчили школу із золотими медалями. Син зараз працює
програмістом і вже сам нещодавно став батьком. Дочка закінчила Школу
журналістики у Львівському УКУ і працює в Києві. Але і там вона не забуває про
Маріуполь і брала активну участь в розробці бренду Маріуполя. Євген відверто
говорить, що йому подобається місто, в якому він живе. Навіть незважаючи на всі
його проблеми. І перебуваючи в Києві, Львові чи навіть в Парижі або Амстердамі
він нерідко ловив себе на думці, що тягне додому, до Маріуполя.

Фотографувати Євген почав з дитинства. Він упросив батьків, і вони подарували
йому найпростішу камеру \enquote{Етюд}. Так він і став \enquote{фотографом}. Свої перші
фотографії друкував так званим контактним способом – прикладав негатив до листа
фотопаперу, накривав його склом, вмикав настільну лампу і відраховував секунди
витримки. Але перші ж отримані знімки викликали у нього дикий захват. Він
носився з ними по кімнатах і показував татові й мамі. Найбільшим досягненням
для нього тоді було, якщо на фотографії можна було визначити, хто саме
зображений на знімку. Пізніше у нього з'явилася інша камера – \enquote{Любитель}, назва
якої повністю відповідала її можливостям, а потім і легендарний \enquote{Зеніт}. Він і
сьогодні лежить у нього на полиці. 

\ii{19_01_2019.stz.news.ua.mrpl_city.1.sosnovsky_take_krasyve_zhyttja.pic.1}

Загалом до недавнього часу він знімав тільки \enquote{для внутрішнього споживання}, для
сім'ї. А взяти камеру в руки для інших цілей його спонукали події, які почали
відбуватися в Маріуполі в грудні 2013 року, коли після розгону студентів на
Майдані з півсотні маріупольців вийшли на наш місцевий Майдан. Тоді і були
зняті перші кадри \enquote{новітньої історії} Маріуполя і він почав освоювати новий для
себе жанр репортажу. Зима і весна 2014-го – проукраїнські мітинги. 13 липня
того ж року – найбільший прапор України в українському Маріуполі. Щирі,
непідробні емоції людей, сльози радості на їхніх очах, коли вони тримали в
руках це величезне жовто-синє полотнище. Євген підкреслив, що багато
центральних ЗМІ в той час показували Маріуполь однобоко, не помічаючи того, що
в нашому місті багато людей, які не чекали \enquote{русского мира} і робили все
можливе, щоб він сюди не прийшов. Саме таких людей він і намагався показати на
своїх знімках в надії хоч якось, хоч трохи змінити думку про Маріуполь і
маріупольців. Потім був серпень 2014-го, коли наше місто опинилося під загрозою
захоплення російськими окупантами, які вже зайняли Новоазовськ. І все ті ж
маріупольці вийшли живим ланцюгом на східну околицю міста, щоб ще раз, під
гуркіт зовсім недалекої канонади, показати, що Маріуполь – це Україна...

\textbf{Читайте також:} 

\href{https://mrpl.city/news/view/fotografii-knigi-tetrapody-iz-mariupolya-stali-chastyu-blagotvoritelnoj-aktsii-v-kieve-foto}{%
Фотографии, книги, тетраподы из Мариуполя стали частью благотворительной акции в Киеве, %
Ігор Романов, mrpl.city, 02.11.2018%
}

Багато подій відбулося за ці п'ять років. І багато з того, що відбувалося,
вдалося зберегти на фото Євгена. Серед них, в основному, портрети, живі емоції
людей, яким не байдуже все, що відбувається навколо. І після перегляду чудового
фільму Дмитра Мойсеєва \enquote{Такі красиві люди} з його дозволу наш герой використав
назву цього фільму для своєї сторінки в Facebook. Так і з'явився своєрідний
фотопроект \#ТакіКрасивіЛюди, який триває у нього з 2014 року і який матиме
продовження, адже в Маріуполі дійсно багато по-справжньому красивих людей,
причому мається на увазі не тільки їхня зовнішність. Це ті люди, які надихають
Євгена і яких йому хотілося б показати всій Україні. І саме ці люди спонукають
його брати в руки камеру і знімати їх протягом ось вже п'яти років. 

\ii{19_01_2019.stz.news.ua.mrpl_city.1.sosnovsky_take_krasyve_zhyttja.pic.2.divchynka}

Минулого року Євген Сосновський отримав приз \enquote{Золотий День} на XX Міжнародній
фотовиставці газети \enquote{День} за знімок \enquote{Доця}. Він дуже цінує цю нагороду і
сподівається, що дівчинка на цьому знімку скоро обійме свого батька, який
повернеться додому з війни.

\textbf{Читайте також:} 

\href{https://mrpl.city/news/view/mariupolskij-fotograf-stal-prizerom-mezhdunarodnogo-konkursa-foto}{%
Мариупольский фотограф стал призером международного конкурса, Роман Катріч, mrpl.city, 29.10.2018}

\ii{19_01_2019.stz.news.ua.mrpl_city.1.sosnovsky_take_krasyve_zhyttja.pic.3.divcha_tato}

Важливо, що родина фотографа – дружина і діти – розуміють, для чого це все
робиться. Рік тому вони разом навіть подарували йому новий фотоапарат, оскільки
старий не витримував великого навантаження.

Євген Сосновський - дуже різностороння і цікава людина. Відомо, що він
займається і театральною діяльністю – грає у Народному театрі \enquote{Театроманія}, у
який потрапив завдяки зйомці театральних постановок. Сам жартує з цього
приводу: \enquote{дознімався}. До речі, в юності він грав у Народному театрі ПК
\enquote{Азовсталь} (нинішній ПК \enquote{Молодіжний}), яким тоді керував Леонід Олександрович
Бессарабов.

\ii{19_01_2019.stz.news.ua.mrpl_city.1.sosnovsky_take_krasyve_zhyttja.pic.4}

Після зйомки однієї з вистав \enquote{Театроманії} Євген розговорився з
режисером Наталією Гончаровою, і вона, дізнавшись про те, що він колись теж
виходив на цю сцену, сказала фатальну для нього фразу: \enquote{А приходьте до нас...}
Рішення він прийняв не одразу, але тоді готувався спектакль \enquote{Королівські ігри}
за п'єсою Г. Горіна, а перед таким автором він не зміг встояти. Так наш герой
опинився не на звичному місці під сценою, де зазвичай бігав з фотоапаратом, а
безпосередньо на сцені в ролі кардинала Вулсі. Євгену і надалі хотілося б
продовжувати цікаві, різнопланові ролі, будь-то невдаха Сем з \href{https://mrpl.city/blogs/view/v-dk-molodezhnyj-pohitili-dzhenni}{%
\enquote{Викрадення Дженні}} або той же кардинал Вулсі з \href{https://mrpl.city/blogs/view/korolevskie-igry-narodnogo-teatra-teatromaniya}{\enquote{Королівських ігор}}.

Наостанок хотілося б розповісти про історію створення фотоальбому, присвяченого
старовинному Маріуполю. Зокрема, фотохудожник наголошує, що є у Маріуполя своя
неповторність, є своя особлива краса, яка нерідко захована від наших очей, а
нерідко ми її просто не помічаємо і проходимо повз. Коли міська влада оголосила
про свій намір підвищити туристичну привабливість Маріуполя, багато містян
поставилися до цієї ідеї, м'яко кажучи, скептично. Мовляв, що цікавого може
показати Маріуполь потенційним туристам – труби заводів, сірі багатоповерхівки.
Євген не став дискутувати, просто взяв свій фотоапарат - і відправився на
прогулянку вулицями старого Маріуполя. Так з'явилася перша серія з декількох
знімків архітектурних та історичних пам'яток нашого міста. Коли з'являється
вільний час, він знову вирушає на таке \enquote{фотополювання}, і з часом ця серія
знімків також вилилася в фотопроект, який він по аналогії з попереднім назвав
\#ТакеКрасивеМісто.

\textbf{Читайте також:} \href{https://mrpl.city/news/view/ischezayushhuyu-arhitekturu-mariupolya-sobirayut-po-krupitsam-na-pamyatfoto-plusvideo}{%
Исчезающую архитектуру Мариуполя собирают по крупицам на память, Денис Росін, mrpl.city, 15.10.2018}

Особисто мені було дуже цікаво дізнатися, чи є у Євгена якісь секрети в
техніці, адже його знімки Маріуполя для мене, як і для багатьох маріупольців,
дійсно відрізняються високим професіоналізмом. Але він відповів, що особливих
секретів і якоїсь специфічної техніки немає. Просто для кожного знімку він
шукає цікавий ракурс, обирає час для кращого освітлення і чекає відповідної
погоди. Буває, що на одне і теж місце повертається по кілька разів, поки не
отримає бажаного результату.

На жаль, Євгену нерідко дорікали, що не можна називати красивими ті руїни, в
які перетворилися багато пам’ятників архітектури нашого міста. Але для нього
вони, як і раніше, зберігають свою красу. І він дуже сподівається, що хоч якісь
з них вдасться зберегти і по можливості реставрувати, щоб повернути їм справжню
красу. Це теж одна з цілей його проекту \#ТакеКрасивеМісто – привернути увагу до
історії та архітектури нашого міста не тільки маріупольців, а й міської влади.
Він вважає, що якщо хоча б когось його знімки змусили відірватися від смартфона
і подивитися навколо, вийти і просто прогулятися старими вулицями Маріуполя,
значить час не було витрачено даремно.

\textbf{Читайте також:} \href{https://mrpl.city/news/view/luchshie-fotografy-mariupolya-poluchili-po-3-tys-grn}{%
Лучшие фотографы Мариуполя получили по 3 тыс. грн, Ганна Хіжнікова, mrpl.city, 10.05.2018}

\textbf{Улюблені книги Євгена Сосновського:} \enquote{12 стільців} і \enquote{Золоте теля} І.
Ільфа і Є.  Петрова, \enquote{Майстер і Маргарита} і \enquote{Собаче серце} М. Булгакова,
новели О'Генрі, п'єси Г. Горіна. Із сучасних українських письменників
подобається Сергій Жадан.  З \enquote{нехудожньої} літератури – Малкольм Гладуелл і
його \enquote{Переломний момент}.

\textbf{Улюблений фільм:} \enquote{Той самий Мюнхгаузен} за однойменною п'єсою Григорія
Горіна.

\textbf{Курйозний випадок з життя:} \enquote{Коли у мене з'явився мій перший фотоапарат
і я відзняв першу плівку, ми з татом вирішили подивитися, що ж я там назнімав.
Відкрили кришку фотоапарата і, звісно, нічого на плівці не побачили. Це потім
вже я з узятих в бібліотеці книг дізнався, як з відзнятої плівки отримати
фотографію. Почав сам змішувати реактиви, готувати проявники, закріплювачі
тощо. Але, як кажуть, \enquote{перший блин дійсно виявився комом}}.

\textbf{Порада фотографам-початківцям:} \enquote{Знімати. Знімати якомога більше. Не
боятися експериментувати і відходити від стандартів. Але для початку освоїти
абетку фотографії, щоб розуміти технічну сторону процесу зйомки, основи
композиції.  Зараз це зробити набагато простіше. Є інтернет. Є фотошколи. Ну а
творча складова вже залежить від того, як фотограф сприймає навколишній світ і
від внутрішнього світу самого фотографа. Цьому навчити вже складніше...}

\begin{center}
\color{blue}\em\textbf{А на завершення пропоную переглянути фото нашого талановитого і непересічного героя.}
\end{center}

\ii{19_01_2019.stz.news.ua.mrpl_city.1.sosnovsky_take_krasyve_zhyttja.pic.5.fotos}

%\ii{insert.author.demidko_olga}
\clearpage
