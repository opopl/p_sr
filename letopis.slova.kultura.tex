% vim: keymap=russian-jcukenwin
%%beginhead 
 
%%file slova.kultura
%%parent slova
 
%%url 
 
%%author 
%%author_id 
%%author_url 
 
%%tags 
%%title 
 
%%endhead 
\chapter{Культура}
\label{sec:slova.kultura}

%%%cit
%%%cit_pic
%%%cit_text
Оскільки було силове впровадження росмовної \emph{субкультуру}, має бути силове витіснення її з України
%%%cit_comment
Ярослав Трінчук
%%%cit_title
\citTitle{...ми ще на стадії формування українського музичного ринку - відвойовуємо його у Росії}, 
Про Мову, facebook, 18.06.2021
%%%endcit

%%%cit
%%%cit_head
%%%cit_pic
\ifcmt
  pic https://img.vz.ru/upimg/m11/m1104023.jpg
\fi
%%%cit_text
Когда Путин в послании Федеральному собранию сказал фразу о Киплинге, которая
сразу стала мемом, он, с одной стороны, отсылал, понятно, к нравам «Книги
джунглей», прозрачно намекая на ситуацию в современной международной политике.
Но, с другой стороны (и этот более глубокий месседж также, думаю, был считан
всеми, кому он предназначался), реагировал на ту беспрецедентную атаку на
традиционную христианскую европейскую \emph{культуру}, которую мы наблюдали весь 2020
год, и которая сегодня продолжает лавинообразно нарастать
%%%cit_comment
%%%cit_title
\citTitle{Россия должна жить по заветам Киплинга}, 
Владимир Можегов, vz.ru, 18.06.2021
%%%endcit

%%%cit
%%%cit_head
%%%cit_pic
%%%cit_text
Не менее сложной и весьма затратной задачей является суверенизация
информационного пространства и установление государственного контроля над ним.
Феномен \emph{культурного влияния} на общество через социальные сети, видеоплатформы,
которые ранее заявлялись «свободными», а оказались четко регулируемыми и жестко
модерируемыми инструментами информационного и \emph{культурного воздействия}, доказал,
что власти Китая были полностью правы, когда распространения такого влияния на
своей территории не допустили
%%%cit_comment
%%%cit_title
\citTitle{Революция Духа – единственный путь спасения России}, 
Юрий Барбашов, voskhodinfo.su, 30.06.2021
%%%endcit

%%%cit
%%%cit_head
%%%cit_pic
%%%cit_text
Результаты незаконченной \emph{культурной вестернизации} России в 19-начале 20 века
были полностью уничтожены во времена сталинской деспотии. Начиная с хрущевских
реформ, страна опять стала открываться миру, \emph{культурная} вестернизация
постепенно возобновилась.  Новой попыткой движения в Европу стала Перестройка.
\enquote{Жить как на Западе} - это в то время была идея-фикс советских людей
%%%cit_comment
%%%cit_title
\citTitle{Как разведенный диктатор-отравитель собирается защищать \enquote{традиционные ценности}}, 
Игорь Эйдман, opinions.glavred.info, 06.07.2021
%%%endcit

%%%cit
%%%cit_head
%%%cit_pic
%%%cit_text
Ключовим тут є слова «\emph{одна культура}». У свідомості президента українська
культура за імперської доби не існувала окремо від російської, \emph{культура} була
спільною для обох народів. Фраза про популярність в Росії «неймовірної
української літератури» належить до порожньої, позбавленої сенсу риторики, бо
українськими творами цікавились у РСФСР хіба що поодинокі українці, що там
мешкали
%%%cit_comment
%%%cit_title
\citTitle{Українська мова в культурному просторі держави. Протистояння триває}, 
Лариса Масенко, www.radiosvoboda.org, 11.07.2021
%%%endcit
