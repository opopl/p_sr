% vim: keymap=russian-jcukenwin
%%beginhead 
 
%%file 07_03_2023.fb.fb_group.mariupol.pre_war.2.istoriya_shkoli_na_k.eng
%%parent 07_03_2023.fb.fb_group.mariupol.pre_war.2.istoriya_shkoli_na_k
 
%%url 
 
%%author_id 
%%date 
 
%%tags 
%%title 
 
%%endhead 

History of school on Karasievka

In 1883, the sons and daughters of Karasiev residents entered a one-story,
ground-breaking house, where they were to learn reading, writing and the
beginnings of mathematics.

The report of the Mariupol County Land Administration for 1884 gives a
description of the first Karasiev school. It consisted of two rooms, the total
area of which consisted of 130 square arshins (65.75 sq.m. meters). The same
document noted that the floors at the school were "quite bad" that in the
winter time a gun came out of the cellar.

In 1892, or a little earlier, Mariupol City Duma petitioned the superior
authorities to assign a school named after Metropolitan Ignatius. The exact
date of acceptance of the petition could not be established, but the fact that,
at least in 1905 she was already named Ignatievskaya, is confirmed by the
documents of the Mariupol County Land Administration.

In the 1905-1906 academic year, 249 children attended the school: 136 boys and
113 girls.

In different years, the educational institution on Karasievka was called in
different ways. In 1910: City primary school named after Metropolitan Ignatius
(Karasievskoe 1st). As is often the case, a change in the name was not
necessarily accompanied by changes in substance. As before, the Law of God,
Russian and Church-Slavic languages, arithmetic, geography, natural sciences
and singing were taught here.

In 1908, another school was opened in Karasievka - Ignatievskaya II. One
ancient source says that "it was formed from an evening independent school
located in the building of Ignatievskaya 1st Land School".

Ignatievskaya (Karasevskaya 1st) were housed in a one-story building on
Kosmo-Damianovskaya Square in the immediate vicinity of the Church of the
Nativity of the Blessed Virgin. This building, however, has been rebuilt
several times, existed back in 1968. Later in its place, the repair and
construction trust built a two-story office for itself, which is now occupied
by the Tax Administration.

After the revolution, in 1918 Ignatievskaya School lost its own name and
received a serial number. Now she is in elementary school. By that time, it had
five classrooms. In 1928, she was transformed into a seven-year-old.

Some time later after the destruction of the Karasiev Church in the desert near
the old school building, where schoolchildren organized a skating rink in the
winter, construction of a four-story building began. On September 1, 1938, its
doors were opened to students who were studying in the seven-year-old school
No. 11 and the guys from nearby neighborhoods. Enrollment from first to eighth
grades was produced then. There were two in the eighth grade.

On June 19, 1941 a tenth-graders prom took place at a school on Karasievka -
the first since the opening of a new building. His participants, who lived to
this day, remembered him as bright, joyful and cheerful... And three days
later, the war began.

From the first days of the war, a hospital was deployed at the school. From the
first days of the occupation, Hitler set up a Soviet POW camp instead.

Later, the survivors of the camp were moved to an even more horrible place -
the former training plant named after Ilyich. The German headquarters moved
into the vacated building for some time. And then at school they set up a grain
warehouse. All classes were filled with corn, millet, wheat. And so it was
before the retreat of the occupiers in September 1943. Before leaving, they set
the school on fire. The wooden coverings caught fire, and then collapsed under
a load of grain blackened by fire.

Shortly after the release of Mariupol, classes in the 11th school resumed.
She's kind of back to her roots. Now it was an elementary school housed in a
one-story building where it existed until 1938. Thank goodness the fire spared
him.

In 1948, an order was issued to transfer the premises to an orphanage. A large
group of children of different ages were brought from Stalin (that's how
Donetsk was called then), and the students of the school were moved to other
schools. Only in 1951 the rebuilt school opened its doors to the children of
the Karasievka district. Before 1953, the school was seven years old. in 1961,
she became eleven-year-old and was given the conscientious name of "Labor
Polytechnic".

From 1965 to 1980 the school was converted into English. It was in 1965 that
children from all over the city began to be admitted to the first classes of an
unusual, meanwhile, educational institution. ... (From the materials of the
site Old Mariupol)
