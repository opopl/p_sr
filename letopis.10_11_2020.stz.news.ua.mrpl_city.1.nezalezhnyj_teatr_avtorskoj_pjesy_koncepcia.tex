% vim: keymap=russian-jcukenwin
%%beginhead 
 
%%file 10_11_2020.stz.news.ua.mrpl_city.1.nezalezhnyj_teatr_avtorskoj_pjesy_koncepcia
%%parent 10_11_2020
 
%%url https://mrpl.city/blogs/view/nezalezhnij-teatr-avtorskoi-pesi-kontseptsiya
 
%%author_id demidko_olga.mariupol,news.ua.mrpl_city
%%date 
 
%%tags 
%%title Незалежний театр авторської п'єси "Концепція"
 
%%endhead 
 
\subsection{Незалежний театр авторської п'єси \enquote{Концепція}}
\label{sec:10_11_2020.stz.news.ua.mrpl_city.1.nezalezhnyj_teatr_avtorskoj_pjesy_koncepcia}
 
\Purl{https://mrpl.city/blogs/view/nezalezhnij-teatr-avtorskoi-pesi-kontseptsiya}
\ifcmt
 author_begin
   author_id demidko_olga.mariupol,news.ua.mrpl_city
 author_end
\fi

Пропоную ближче познайомитися, ще з одним унікальним колективом, який став
учасником міського фестивалю \enquote{Маріуполь театральний – 2020}. Це \emph{Незалежний
театр авторської п'єси \enquote{Концепція}}, в якому можна знайти сучасну авангардну
режисуру і драматургію. Колектив був заснований \emph{1 лютого 2019 року} за
ініціативою творчого тандему поетеси \emph{\textbf{Марії Грецької}} та режисера \emph{\textbf{Олексія
Гнатюка}}.

\ii{10_11_2020.stz.news.ua.mrpl_city.1.nezalezhnyj_teatr_avtorskoj_pjesy_koncepcia.pic.1}

Спочатку це був проєкт навчального плану – театральна студія, куди
приходили всі охочі займатися акторською майстерністю. Пізніше Олексій
зрозумів, що студія готова стати справжнім театром і почав писати п'єсу, назва
якої – \emph{\enquote{Рентген усміхнених сердець}} – йому наснилася. Колектив почав працювати
над виставою і вже під час роботи народилася ідея створити театр саме
авторської п'єси.

У жовтні 2019 р. у театру з'явився свій сценічний простір, що відрізняється в
звичному розумінні класичного театру своєю камерною моделлю. Подібні театри,
популярні в Європі і в столичних містах, де глядачам надається унікальна
можливість стати учасниками вистави. У приміщенні театру було 30 місць, глядачі
проживали разом з акторами все, що відбувалося на сцені, мимоволі стаючи
учасниками спектаклів. Сценічний простір є продовженням зали для глядачів, що
дає можливість глядачам повністю занурюватися в атмосферу вистави.

Нещодавно театр переїхав до Маріупольської камерної філармонії, де йому
виділили декілька кімнат і завдяки зусиллям колективу це приміщення вже стало
досить комфортним і затишним. На стінах можна побачити світлини з вистав,
погляд притягують і меблі, стилізовані під старовину. Тепер у глядацькій залі
буде 50 місць для глядачів.

У складі трупи близько 20 осіб, серед них є досвідчені актори і ті, які
займаються в студії при театрі. В об'єднання театру входять актори з
багаторічним стажем, організатори культурно-дозвіллєвої діяльності, телеведучі,
бізнесмени, поети та студенти. Важливо, що Незалежний театр авторської п'єси
\enquote{Концепція} не має вікових обмежень, головний критерій потрапити до
трупи – це талант і бажання працювати. Актори наголошують, що театральне
мистецтво займає в їхньому житті найголовніше місце. А \emph{актриса \textbf{Ганна Лафазан}}
навіть переїхала до Маріуполя з Харкова, щоб грати саме в цьому театрі.

\ii{10_11_2020.stz.news.ua.mrpl_city.1.nezalezhnyj_teatr_avtorskoj_pjesy_koncepcia.pic.2}

При театрі функціонує студія, де можна навчитися нелегкому ремеслу актора і
закріпити набуті навички на показах. Завдання студії – виховати актора, який
досконало володіє своїм тілом, як образним інструментом, голосом та уявою. Для
цього в процесі навчання акторській справі застосовуються різні методики і
акторські тренінги.

Наразі модно осучаснювати класику, прикладом чого служить відома формула
\enquote{Шекспір в джинсах}. Але колективу \enquote{Концепції} набагато цікавіші \emph{нова
драматургія, нові думки, нове сприйняття реальності}.

Незалежний театр авторської п'єси \enquote{Концепція} можна назвати \enquote{сучасним} перш за
все в європейському розумінні – як театр сучасної драматургії. Під \enquote{сучасним}
маються на увазі не тільки і не стільки п'єси із сучасного життя, а твори, що
відображають дихання часу, дух нашої епохи.

\ii{10_11_2020.stz.news.ua.mrpl_city.1.nezalezhnyj_teatr_avtorskoj_pjesy_koncepcia.pic.3}
\ii{10_11_2020.stz.news.ua.mrpl_city.1.nezalezhnyj_teatr_avtorskoj_pjesy_koncepcia.pic.4}

В репертуарі театру на теперішній час є такі вистави як: трагікомедія \enquote{Рентген
усміхнених сердець} та оперна театральна вистава за мотивом однойменної опери
В. А. Моцарта \enquote{Дон Жуан}. Вже в листопаді маріупольці зможуть побачити дві
прем'єри від Незалежного театр авторської п'єси \enquote{Концепції} – психологічну
драму \enquote{Страхи в стилі НЮ} та фантасмагорію \enquote{Другий шанс}. Особисто мене
інтерв'ю з режисером театру Олексієм Гнатюком переконало, що цей колектив
готовий і надалі завойовувати любов маріупольських глядачів, адже актори
постійно вдосконалюють свою майстерність і завдяки креативному баченню
режисера будуть приємно вражати своїми новими виставами.

