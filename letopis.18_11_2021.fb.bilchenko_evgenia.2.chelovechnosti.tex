% vim: keymap=russian-jcukenwin
%%beginhead 
 
%%file 18_11_2021.fb.bilchenko_evgenia.2.chelovechnosti
%%parent 18_11_2021
 
%%url https://www.facebook.com/yevzhik/posts/4452125138155895
 
%%author_id bilchenko_evgenia
%%date 
 
%%tags bilchenko_evgenia,chelovechnost,chelovek,deti,donbass,film,film.deti_donbassa,interview,rossia,ukraina,vojna
%%title БЖ. Человечности!
 
%%endhead 
 
\subsection{БЖ. Человечности!}
\label{sec:18_11_2021.fb.bilchenko_evgenia.2.chelovechnosti}
 
\Purl{https://www.facebook.com/yevzhik/posts/4452125138155895}
\ifcmt
 author_begin
   author_id bilchenko_evgenia
 author_end
\fi

БЖ. Человечности! 

Мое краткое интервью "Народным Новостям" на фильм Герман Владимиров и его
съёмочной группы "Дети Донбасса", названный мной в рецензии ниже "фильмом
Доброты", можно прочитать по ссылке \#НН, спасибо ребятам за эмоциональное
общение:

\href{https://nation-news.ru/673148-sbezhavshaya-iz-ukrainy-poetessa-bilchenko-predlozhila-luchshii-sposob-pokonchit-s-voinoi-v-donbasse}{%
Сбежавшая из Украины поэтесса Бильченко предложила лучший способ покончить с войной в Донбассе, %
nation-news.ru, 18.11.2021%
}

\ifcmt
	ig https://external-frt3-2.xx.fbcdn.net/safe_image.php?d=AQEIsbIVSSJttrv7&w=500&h=261&url=https%3A%2F%2Fstatic.nation-news.ru%2Fuploads%2F2021%2F11%2F18%2Forig-1637219439e2e6568f7da61443eec53e520bdc64a7.jpeg&cfs=1&ext=jpg&_nc_oe=6f0ee&_nc_sid=06c271&ccb=3-5&_nc_hash=AQEpxjJhYL4eDYmG
  @width 0.7
  @wrap \parpic[r]
\fi

Добавлю: в экранной культуре есть американское по происхождение понятие
"мокьюментари" - художественного вымысла в псевдодокументальной форме, как это
любят делать украинские официальные СМИ (оппозиционные прикрыли почти все, даже
самые ня), а здесь - real documentary, причем я ночью перед "Детьми Донбасса"
посмотрела исполненную жёстких сцен насилия "Ополченочку", а потом - кино
Германа и его товарищей. На "Ополченочке" я не плакала, только гидазепамила: не
потому что насилия нет, а в кино - ложь, насилие есть и в кино - не ложь, но
кино - это всё-таки Символическое, а документалка - это Реальное. Любые
настоящие глаза настоящего ребенка - сильнее любого художественного фильма,
или, как говорят мои бывшие студенты, "трушнее". От этого - больно, но доброта
- это и есть нормальная такая боль, а не ми-ми-ми про мир, дружбу и жвачку. 

Сейчас размышляю над проектом Горловки и моего христианского обета, даденного
мной  прекрасной русской/украинской девочке Алине и ее бабушке, которая...
понимает мой рэп. Вот вспомнился недавний текст, написанный мной после научной
конференции о "человечности" милых, но ничего не понимающих в этой жизни,
умозрительных людей, во время которой на моей последней прокуренной кух/зне
сидело живое чудо человечности - Алина. Текст, сдобренный алининой горловской
бабушкой, проходящей сейчас подвалы, добавляю как поэтическую рецензию на фильм
"Дети Донбасса" (прозой я ужасно заикалась на интервью, с заплаканной мордой,
стеснялась и выглядела не айс): 

Шоколадка

\obeycr
Много-премного пришло учёных: рассуждали о человечности.
О нашей увечности и о вечности, о времени быстротечности.
О бесконечности космоса. Всяк говорил своё,
Ибо то была конференция, а не абырвалг, ё моё.
Я тоже там умное говорила, снобствуя и цитируя.
У меня же на кухне, в самом холодном углу квартиры,
Кутаясь в мою кофту - драную, серую, шерстяную, - по горло,
Сидела Алёнка, Алёшка, Алёнушка - шоколадка моя из Горловки.
Её волосы, блеском каштановым окутывающие скулы,
Напоминали конфеты советские - "Кара-Кум".
А глаза её, карие, золочёные Тицианом,
От горя сами себя очищали, от калия будто цианистого.
Алёнушка ехала в Киев через Россию - тридцать восемь часов пути.
Её бабушка ехала столько же, чтоб для пенсии получить id.
На путь туда и обратно даже в плацкартах 
Уходило всё, что капало на проклятую эту карту.
В вагоне Алёнушка-шоколадка откровенно борзела,
Отвечая военным из Украины: "Зачем вы стреляете в нашу землю?"
У Алёнки было целых три деда, и о них я хочу отдельно
Рассказать дельно: не в пример конференции, ибо истина - неподдельна.
Первый дед её умер в сорок девять. Танкист. Живым дошёл до Берлина.
Потом он стал журавлём, вышибая клин песенный птичьим клином.
Второй дед её умер тоже в сорок девять: судьбе решать-то,
Кому и с кем совпадать: подорвал дыхалку на ртутной шахте.
А третий - самый красивый - из шахты угольной
Гарцевал Ванюшей, завидным парнем, по главной улице.
И дожил он до девяноста, сохранив при обстрелах крылья.
Похоронить Алёнке его не дали: блок-посты в тот день перекрыли.
А потом пришло много-премного учёных, и всяк испытал кАтарсис,
Рассуждая о человечности: к чему она там, блин, катится.
Я вышла из зума, будто из комы, мат загнав под щёки и веки,
И вернулась на кухню... Алё, Алён, шоколадка моя - навеки.
\restorecr
