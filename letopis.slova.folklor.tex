% vim: keymap=russian-jcukenwin
%%beginhead 
 
%%file slova.folklor
%%parent slova
 
%%url 
 
%%author 
%%author_id 
%%author_url 
 
%%tags 
%%title 
 
%%endhead 
\chapter{Фольклор}

%%%cit
%%%cit_pic
%%%cit_text
Но в том-то и дело, что образ взят из легенд Буковины - родного края
композитора и автора слов песни Владимира Ивасюка. Он долго и кропотливо
работал над песней. Отец композитора, Михаил Федорович Ивасюк в книге
\enquote{Монолог перед лицом сына} вспоминал, что впервые Владимир Ивасюк
встретил выражение \enquote{червона рута} в редком научном издании
\emph{фольклорных} миниатюр (Львов, 1907). А через три года, находясь в
\emph{фольклорной} экспедиции в одном из сел Черновицкой области, услышал
гуцульскую легенду об этом таинственном цветке. Она гласит: \enquote{Обычно
цветки руты - желтые, но изредка бывают красные. Девушке, которой удастся
отыскать и сорвать волшебный лепесток, повезет и самой расцвести, как цветок, и
своей красотой приворожить любого парня}
%%%cit_comment
%%%cit_title
\citTitle{Какая легенда вдохновила Владимира Ивасюка написать \enquote{Червону руту} 
(украинскому шлягеру - 50)}, 
Татьяна Кроп, zen.yandex.ru, 07.12.2020
%%%endcit

