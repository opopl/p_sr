% vim: keymap=russian-jcukenwin
%%beginhead 
 
%%file 06_12_2021.fb.fb_group.story_kiev_ua.2.knigoizdateli_bratja_novickie
%%parent 06_12_2021
 
%%url https://www.facebook.com/groups/story.kiev.ua/posts/1812638948932898
 
%%author_id fb_group.story_kiev_ua,jeremenko_evgenia
%%date 
 
%%tags istoria,kiev,knigoizdanie,tipografia
%%title КНИГОИЗДАТЕЛИ БРАТЬЯ КОРЧАК-НОВИЦКИЕ (1)
 
%%endhead 
 
\subsection{КНИГОИЗДАТЕЛИ БРАТЬЯ КОРЧАК-НОВИЦКИЕ (1)}
\label{sec:06_12_2021.fb.fb_group.story_kiev_ua.2.knigoizdateli_bratja_novickie}
 
\Purl{https://www.facebook.com/groups/story.kiev.ua/posts/1812638948932898}
\ifcmt
 author_begin
   author_id fb_group.story_kiev_ua,jeremenko_evgenia
 author_end
\fi

\index[rus]{Типография}

Дорогие одногруппники, удалось упорядочить немногое, что доступно в
интернетресурсах, об известных киевлянах, державших типографии и принесших
немалую пользу.

КНИГОИЗДАТЕЛИ БРАТЬЯ КОРЧАК-НОВИЦКИЕ (1)

1 декабря 1937 г. в далёком Казахстане был вынесен приговор отцу и сыну:
75-летнему отцу ‒ ВМН (высшая мера наказания, расстрел), 38-летнему сыну ‒ 10
лет ИТЛ (исправительно-трудовые работы, лагерь). Через пять лет, 25 января 1943
г., была на 6 лет ИТЛ осуждена 70-летняя мать. Все трое ‒ по 58-й статье УК
(уголовный кодекс) РСФСР. Эта статья содержала в себе обвинение в пропаганде
или агитации, содержащие призыв к свержению, подрыву, ослаблению советской
власти... Отец и сын были приговорены даже не судом, а тройкой УГБ (управление
государственной безопасности) по ЗКО (Западно-Казахстанскому Округу). Мать же
предстала перед Облсудом ЗКО. Отец и сын были реабилитированы 31 декабря 1959
г. за недоказанностью состава преступления, мать ‒ 21 апреля 1993 г. Указом
Президента СССР от 13.08.1990 как жертва политического террора в СССР. Фамилия
их КОРЧАК-НОВИЦКИЕ.

\ii{06_12_2021.fb.fb_group.story_kiev_ua.2.knigoizdateli_bratja_novickie.pic.1}

Для представителей тогдашней власти они были некими жителями (очевидно
ссыльными) Казахстана с неизвестными им образованием и прежним родом
деятельности, уроженцами Украины. Отец работал сторожем, сын инструктором, мать
не работала.

На самом деле они ‒ представители известной киевской семьи Корчак-Новицких,
трудившихся в книгоиздании с конца 1870-х несколько десятилетий. 

Репрессирован и расстрелян был младший брат, Николай Трофимович (1862-1937),
уроженец Киева, перед арестом проживавший с семьёй в западно-казахстанском
городке Уральске. Многолетние сроки получили его сын Николай Николаевич
(1899-19?), «уроженец Киева, украинец, образование незаконченное высшее
(профиль неизвестен)» (из следственного дела) и супруга Елена Фёдоровна
(1872-194?). Из её следственного дела: «русская, уроженка Украина, Черниговская
обл район, Ново-Зыбково с. Образование: среднее».

Старший брат, Георгий Трофимович (1859? - после 1915?), до революции не дожил.
В метрической книге Ольшанского кладбища в Праге есть запись об упокоившемся в
1979 г. инженере Георгии (без отчества) Корчак-Новицком, но принадлежал ли он к
этой ветви рода, сказать пока что сложно. Как неизвестно, имеет ли отношение к
братьям И.Н. Корчак-Новицкий, который в 1933 г. был представителем Союза
Младороссов в Америке. 

В Киеве братья-издатели проживали в центре, имели собственные дома и участки в
дачном посёлке Святошин. 

Николай Тимофеевич был купцом 2-й гильдии, владел типолитографией в собственном
доме на ул. Меринговской (сейчас Заньковецкой), 6. Это здание не сохранилось,
на его месте в 1914 году был построен доходный жилой дом в стиле модерн.
Владел также домом на ул. Совской, 67 (ул. Физкультуры, район метро
Олимпийская). 

В Святошине у его семьи была дача на участке №370 по ул. Южной, 66 (или
Петропавловской 65: участки между этими улицами расположены в 1 ряд, поэтому
имеют двойной адрес; сейчас примерно ул. Верховинная 59, недалеко от
пересечения с бывшей Парковой ‒ Ф.Кричевского, примыкает к территории
онкоцентра). В элитном дачном Святошине эта часть была привилегированной. Рядом
парк (территория Гимназии восточных языков) с его
увеселительно-развлекательными заведениями, с другой стороны храм. Пересечение
Брест-Литовского проспекта с улицами Парковой к югу и Торговой  к северу (м.
Житомирская) считалось центром посёлка. Но от невольного оживления магазинов,
ресторанов, фотографического салона Раштанова и других публичных мест участок
Н.Т.Корчак-Новицкого хоть и находился неподалёку, но был защищён непременно
сохранявшейся зеленью, знаменитыми святошинскими соснами и дубами. Соседями
были известные купцы Б.А.Семадени, С. и Н.Н. Николаевы, семьи Рофе, Дуван и др.
На 366 участке располагался дом причта Свято-Николаевской Святошинской церкви. 

На то время в нашем городе (где первая книга была отпечатана в 1616 г.)
ситуация с книгоизданием была такова.  

 «В Киеве… на 1 января 1887 г. было 17 типографий  и 2 скоропечатных заведения.
 Из них в четырёх ‒ в типографиях Г.Корчак-Новицкого, С.Кульженко, И.Кушнерева
 и в Лаврской ‒ в производстве применялись паровые двигатели. Имелись также 17
 литографий, одна металлография и 4 словолитни. На всех типографских
 производствах трудились 550 рабочих. В их составе ‒ 10 факторов (управляющих
 технической частью типографии, 205 наборщиков, 75 печатников, 36 наладчиков,
 10 гравёров, 25 машинистов, 24 батырщика (они покрывали краской типографский
 набор), 20 литографов и 145 учеников. В целом на типографском производстве в
 1887 г. были задействованы 62 скоропечатные машины, 42 типографских станка и
 105 тонн шрифта.

В Киеве в тот год числилось 165 тыс. жителей, и к их услугам было 7 кабинетов
для чтения, 11 библиотек, 25 книжных лавок и магазинов и ещё 17 чел. продавали
книги в разнос»,‒ сообщает  Виталий Ковалинский.

Через 10 лет, к 1 января 1897 года, всех заведений печатного дела (типографий,
литографий, металлографий, ксилографий, фототипий, фотоцинкографий и отдельных
ручных типографских и литографских станков) в Российской Империи (без
Финляндии) было 1958, в том числе в Петербурге 255, Москве — 212, Варшаве —
143, Вильне — 22, Киеве — 22, Одессе — 55.

Братья Корчак-Новицкие, взявшись за типографское дело, довольно быстро
оказались в числе известных издателей.

На рубеже веков младший брат ‒ Николай Тимофеевич ‒ возглавлял Типографию
Императорского Университета Св. Владимира. 

Издательскому делу в Университете изначально придавалось серьёзное значение.
Уже через год после его открытия, в 1835, заработала типография. Инициатором
был первый ректор и декан историко-филологического факультета Михаил
Александрович Максимóвич (1804‒1873) ‒ филолог, фольклорист, историк, поэт,
ботаник, член-корреспондент Санкт-Петербургской академии наук (1871). Его труды
и исследования тоже печатались в этой типографии. Уже за 50 лет объём изданных
книг и статей, авторами которых были ведущие учёные, превышал 600 названий. 

На книгах и журналах, изданных в кон. XIX ‒ нач. ХХ значилось: «Типография
Императорского Университета Св. Владимира, Акционерного общества печати и
издательского дела Н.Т. Корчак-Новицкого». С 1850 года и до революции в
университете было 4 факультета: историко-филологический, физико-математический,
юридический и медицинский. Естественно, издания университетской типографии
соответствовали различным специальностям этих факультетов. Назовём наугад лишь
несколько из них:

«Из Апостола: (Трудные места)», «Объяснение важнейших пасхальных песнопений с
указанием связи между ними»; «Из первых лет жизни Пушкина на юге России»;
«Сказка о немце, казаке и русском скипидаре»; Л. Струнина «Первые воскресные
школы в Киеве»; «Св. архистратиг Михаил: Всё, что можно знать о нём на
основании Св. Писания, предания и соображений разума, — и о церковном
чествовании св. Архангела»; Крыловский А. «Львовское Ставропигиальное Братство»
(Опыт церковно-исторического изследования); Житецкий П. «Теория поэзии»; М.
Скабалланович «О пасхальных напевах»; Гиляров А. Н. «Философия в её существе,
значении и истории: практическое пособие»;  Малороссийские переписные книги
1666 года; Паше-Озерский Н.Н. «Об „опасном состоянии" личности преступника»;
В.С. Иконников «Киев в 1654‒1855. Исторический очерк», Караваев В. А. «Поездка
на остров Яву (впечатления натуралиста)» …

В типографии Н.Т.Корчак-Новицкого печатались каталоги имеющихся во многих
заведениях Киева и губернии, прежде всего, в Университете Св. Владимира, книг:
библиотеки И.Н. Костомарова, пожертвованной А. Л. Костомаровой; студенческого
отдела библиотеки; отдела имени Н.Х. Бунге; обзор рукописей и другие. Кроме
этого, издавался каталог книг Киевской губернской земской управы, периодические
издания, получаемые из библиотек мира, описание рукописей Императорской
Александровской Киевской гимназии, книг Киевского окружного педагогического
музея, Киевского благотворительного общества…

В университетской типографии издавали и периодику: «Университетские Известия»
— ежемесячный журнал, выходящий с сентября 1861 г. до революции. Он состоял из
двух частей: официальной (протоколы, отчеты и т. п.) и неофициальной со
статьями научного содержания, разделявшимися на критико-библиографические и
научную хронику. А также "Труды Киевской духовной Академии" (1897-99), «Записки
Киевского общества естествоиспытателей» и другие. Со второй половины 1900 до
1906 г. в типографии Николая Трофимовича Корчак-Новицкого выходил журнал
"Кіевская Старина", который прежде весьма длительное время (1882–1900)
печатался в типографии его старшего брата Георгия Трофимовича.

Вот что пишет об этом солидном журнале Энциклопедический словарь Брокгауза и
Эфрона: «Киевская Старина» — исторический журнал, выходит с 1882 г.,
ежемесячно, в Киеве. Посвящён разработке местной истории, выяснению
характеристических особенностей народного мировоззрения и веками выработавшихся
бытовых отношений в Южной России. Состоит из трёх отделов: I. Оригинальные
статьи. II. Документы, известия и заметки и III. Критика и библиография. «К.
Старина» дала целый ряд весьма ценных статей, исследований, мемуаров,
дневников, заметок и известий по археологии, истории, литературе и этнографии.
Сотрудники журнала: профессор В. Б. Антонович, профессор Д. И. Багалей. В. П.
Горленко, П. В. Голубовский, А. Голомбиовский, профессор Н. П. Дашкевич, П. С.
Ефименко, А. Я. Ефименко, П. И. Житецкий, И. П. Житецкий, профессор В. С.
Иконников, В. А. Мякотин, профессор Н. Ф. Сумцов, М. К. Чалый и многие др.
Журнал иллюстрируется портретами замечательных деятелей, снимками с монастырей,
церквей, древнейших гравюр и т. п.». Нужно добавить, что с конца 1890-х гг. в
журнале печатались произведения украинских писателей И. Франко, М.
Коцюбинского, Леси Украинки, Панаса Мирного и др.

Помимо издания научных трудов и журналов типография обзавелась хромолитографией
для печатания видов, преимущественно, святых мест. 

Согласно журналу «Антиквар», в киевском «Музее Шереметьевых собрана
замечательная коллекция хромолитографий с видами Киево-Печерской лавры,
производство которых было налажено с середины ХІХ в. преимущественно при
монастырях. Вскоре к их выпуску подключились такие крупные издатели, как Н.Т.
Корчак-Новицкий, Е. И. Фесенко, И. Д. Сытин. За качеством церковной печатной
продукции, выходившей тысячными тиражами, строго следили, поэтому на
хромолитографиях рубежа ХІХ ‒ ХХ вв. неизменно присутствует надпись: «Дозволено
цензурой… года». Невысокая цена изображений обеспечила им огромную популярность
среди местного населения и паломников, начинавших с весны хождение по святым
местам. Из таких путешествий они привозили иконки, крестики, хромолитографии и
фарфор, декорированный изображениями древних обителей». 

Было и ещё одно направление деятельности типографии, причём на первых порах
вызвавшее отрицательное отношение других издателей: лубок. 

Началось с того, что из Москвы в Киев переехал один из братьев Губановых ‒
Тимофей, к тому времени уже известный лубочный издатель, прославившийся
чрезвычайно низкими требованиями к выпускаемой им продукции. Он открыл на
Подоле издательство, нанял третьесортных авторов-борзописцев и стал выпускать
низкопробные псевдоисторические романы, сонники, оракулы и т.п. В них было
очень много картинок, совсем мало текста, множество грамматических ошибок;
выходили они на радость необразованным простакам огромными тиражами и были
очень дёшевы. Большинство солидных киевских типографий отказались их печатать.
Для Н.Т. Корчак-Новицкого же такой опыт стал ступенькой к тому, что
продиктовала дальнейшая жизнь.

Начало ХХ века ознаменовалось многими нестроениями, этот ряд открыла
русско-японская война 1904-05. Для поднятия боевого и патриотического духа
простого народа известные издатели империи (10 ‒ в Москве, 13 ‒ в
Санкт-Петербурге, 2 ‒ в Киеве: И.И.Чоколов и Н.Г.Корчак-Новицкий, 1 ‒ в
Ростове-на-Дону) по заказу свыше в огромных количествах печатают плакаты,
листовки, открытки в лубочном стиле. 

Акционерное общество Н.Т.Корчака-Новицкого выпускает серию хромолитографий
«Русско-японская война». В эту серию входит, например, литография «Геройское
отражение русскими японской эскадры и потопление японских пароходов».

К 1914 году дело Т.А. Губанова продолжает его сын, И.Т.Губанов. И
Н.Т.Корчак-Новицкий не прекращает с ним сотрудничества. С началом I Мировой
войны его АО издаёт серию литографий «Война России с Австрией и Германией». На
каждой из них стоит дата, пометка «От штаба Верховного Главнокомандующего»,
указано также: Киев. Издание И. Т. Губанова. Хромолитография Акционерного
общества Н. Т. Корчак-Новицкого, № листа, год ‒ 1914. И обязательно «Цензурное
разрешение от такой-то даты». К тому же под каждой литографией есть подпись,
призванная укрепить веру в скорую победу: «Австрiйскiе арьергарды, пытавшiеся
задержать наше наступленiе въ районе Барановъ-Ранишевъ, отброшены съ большими
потерями», «…и въ районе къ северу отъ реки Пилицы, отброшены и находятся въ
полномъ отступленiи» и т.п. Размер литографии ок. 44х58 см.

Помимо своего доходного предприятия Николай Тимофеевич, как и все люди его
круга, участвовал в общественных начинаниях. В частности, распространяя свой
профессиональный интерес, был попечителем библиотеки Киевского
благотворительного общества. 

На рубеже XIX – XX веков для состоятельных людей участие в делах, приносящих
пользу неимущим, было явлением привычным и естественным. Накануне I-й Мировой
войны в Киеве насчитывалось около 50 обществ, целью которых была разнообразная
по-мощь нуждающимся. И Общество помощи бедным (Киевское благотворительное
обще-ство), основанное в 1834 году и включавшее 16 учреждений сообразно
направлению их деятельности, играло ведущую роль. К 1914 году его недвижимое
имущество составляло почти три миллиона рублей, ежегодные расходы приближались
к 150 тысячам. Помимо членских взносов и частных пожертвований, весьма
существенным источником дохода товарищества были операции с недвижимостью. Так,
принадлежавшие Обществу земли по Бибиковскому бульвару и Кадетскому шоссе
(бульвар Тараса Шевченко и Воздухофлотский проспект) были переданы в аренду
организации велосипедистов-любителей, дача и пруд на 12-й версте Житомирского
шоссе (пруд на 5-й просеке в Святошино) – также в аренду под устройство
мельницы).

Направления деятельности Общества, помимо оплаты лечения, обучения, жилища и
других бытовых нужд, отражали и культурно-образовательные цели. В этой-то сфере
и трудился Н.Т. Корчак-Новицкий. Такой была жизнь до 1917 года.

К сожалению, о мытарствах его и семьи после революции известно немного. Их
постигла обычная для того времени участь: лишение работы, места жительства,
человеческих условий существования. И приговор.

Старший брат ‒ Георгий Трофимович ‒ до революции не дожил. О нём следующий
рассказ.

\ii{06_12_2021.fb.fb_group.story_kiev_ua.2.knigoizdateli_bratja_novickie.cmt}
