% vim: keymap=russian-jcukenwin
%%beginhead 
 
%%file topics.vojna.my.7.matrica.nashe.dugina.4.zhertva
%%parent topics.vojna.my.7.matrica.nashe.dugina
 
%%url 
 
%%author_id 
%%date 
 
%%tags 
%%title 
 
%%endhead 

\paragraph{17:27:42 23-08-22 Царь Просто Царь}

Итак, смерть Дугиной уже используется для нагнетания антиукраинской истерии. Но
до 12 сентября потребуются новые поводы. Поэтому прогнозы о том, что в России
готовится череда терактов, один другого ужаснее, выглядят довольно
правдоподобными.

Кто станет жертвой ответного террора

Убийство Дугиной (или ее отца, если он случайно избежал покушения) вполне
укладывается в сценарий сакральной жертвы: во-первых, фигура довольно
приметная, но, во-вторых, не особо значимая. Для современной России она вполне
может стать аналогом Хорста Весселя для нацистской Германии.

Вопрос в том, как Кремль будет ее использовать. Можно предположить два
варианта. Во-первых, это введение военного положения и объявление всеобщей
мобилизации. Путин не может сказать народу: спецоперация проваливается,
украинцы наносят нам сокрушительные удары. А вот сказка о том, что мифические
«украинские террористы» подорвали дочь Дугина и затем устроили еще несколько
терактов, вполне может сгодиться как основание для объявления «войны до
победного конца». Во-вторых, хоть и с меньшей вероятностью, это использование
неконвенциальных вооружений, таких как химическое или ядерное оружие, на фронте
и для ударов «по центрам принятия решений», то есть по столице и другим
городам. Это именно те варианты, на которых настаивают Дугин и «Незыгарь».

Впрочем, высказываются и другие предположения. Та же Юлия Латынина
спрогнозировала, что убийство дочери Дугина «запустит волну ответного террора,
как убийство Кирова». «Неделю назад я спросила: "Как Путин ответит на удары по
Крыму», — и похоже, что Путин, не имея, чем ответить снаружи, ответит массовым
террором внутри. А если не ответит он, это сделают за него другие», — добавила
она.
