% vim: keymap=russian-jcukenwin
%%beginhead 
 
%%file 21_05_2021.fb.nicoj_larisa.1.vyshyvanka_marker_ukrainy
%%parent 21_05_2021
 
%%url https://www.facebook.com/nitsoi.larysa/posts/919214851978221
 
%%author Ницой, Лариса
%%author_id nicoj_larisa
%%author_url 
 
%%tags vyshyvanka,ukraina,nicoj_larisa,kultura,tradicii
%%title Вишиванка - маркер України. Василь Вишиваний - приклад того, як українцями СТАЮТЬ
 
%%endhead 
 
\subsection{Вишиванка - маркер України. Василь Вишиваний - приклад того, як українцями СТАЮТЬ}
\label{sec:21_05_2021.fb.nicoj_larisa.1.vyshyvanka_marker_ukrainy}
\Purl{https://www.facebook.com/nitsoi.larysa/posts/919214851978221}
\ifcmt
 author_begin
   author_id nicoj_larisa
 author_end
\fi

Вишиванка - маркер України. Василь Вишиваний - приклад того, як українцями СТАЮТЬ. 

Ерцгерцог австрійський, Вільгельм Габсбурзький, в часи Першої світової війни,
побувавши в Україні в Карпатах, яка належала тоді Австро-Угорщині, лишився тут,
полюбивши цю землю і її народ. Він вивчив українську мову і підтримав Україну в
її боротьбі за незалежність, ставши полковником легіону Українських Січових
Стрільців (УСС). Не плутати з німецькими каральними підрозділами (сс) в часи ІІ
Світової війни. Українці, ще в часи І Світової мали легіони СС (Січових
Стрільців) і спільного між нашим сс і німецьким сс є хіба що дві співзвучні
літери. 

Так от, вивчивши українську мову, нащадок австро-угорської королівської
династії розділив з українськими січовими стрільцями повсякденне життя, їжу,
боротьбу і прагнення до незалежної України. Січові стрільці любили і поважали
Вільгельма за його сміливість і відданість українській справі. Вони любили його
вірші, написані українською мовою, які кликали українців до боротьби.
Українські січові стрільці вважали його побратимом і подарували йому вишиванку.
З тих пір Вільгельм ходив лише у вишиванках, а українці стали його називати
Василем Вишиваним. 

\ifcmt
  pic https://scontent-lga3-2.xx.fbcdn.net/v/t1.6435-9/188142682_919214825311557_7253608119817178258_n.jpg?_nc_cat=110&ccb=1-3&_nc_sid=8bfeb9&_nc_ohc=_8yTGOFJCPcAX8h4NU0&_nc_ht=scontent-lga3-2.xx&oh=71e24d10d1fda16ce19a658a692830b4&oe=60CC3ABF
	fig_env wrapfigure
	width 0.4
\fi

Василь Вишиваний міг стати на український престол. Ми, українці, забуваємо, що
ми королівство зі своєю династією королів. Король Данило - не єдиний король, як
нам нав'язують думку. У короля Данила, були нащадки, які теж були українськими
королями. Українські королі дружили з європейськими королями. Сварилися.
Мирилися. Їздили на бали один до одного. Разом полювали. Влаштовували лицарські
турніри. Одружувалися. Родичалися. Українські королеви народжували своїм
європейським королям династії нащадків. Ми були одним світом. Василь Вишиваний,
принц габсбурзький, 100 років тому міг стати українськими королем, зійшовши на
наш престол, який українцям дістався в спадок ще з часів Русі. І як тільки
перед українцями в часи І Світової війни замайоріла перспектива здобути
незалежну державу, відновилися й розмови про монархію і королівство. Якби так
сталося, ми б мали своїх королів, як Британія має англійську королеву. Однак,
більшовики, захопивши Україну, поховали таку можливість, як і нашу молоду
незалежність. Україна знову була роздерта між світами. Василь Вишиваний
емігрував у Європу, і, де міг, використовував свій дипломатичний хист на захист
України, а Україну накрив червоний терор, куфайки і балалайки. Вишиванки
поховали на дно в скрині. За вишиванку могли впаяти статтю за звинувачення в
націоналізмі…

А потім була ІІ Світова війна, в якій ерцгерцог вижив і залишився впливовою в
дипломатичних колах людиною. А в Україні після німецької окупації настала нова
совєцько-московська окупація. НКВД не дрімало, рискало по світу в пошуках
знакових українців і знищувало їх. У 1947 році енкаведисти викрадають в Європі
Василя Вишиваного, герцога Габсбурзького, і вивозять в ссср. В Україні (за волю
якої він не шкодував життя) його садять в лук'янівську тюрму в Києві. Чужинська
московська власть в нашій країні звинувачує нашого героя Василя Вишиваного у
співпраці з українськими націоналістами і засуджує на 25 років таборів. Але й
розкоші жити в радянських гулагах московити йому не подарували. 1948 року
Василя Вишиваного в Лук'янівській тюрмі вбивають. Де вони його поховали до
сьогодні не відомо. Так в українців відбирали могили героїв і пам'ять про них. 

...Пройшли роки. Українці на 30-му році незалежності з величезними потугами
відновлюють свою мову. А з мовою приходить пам'ять. Важко приходить і дуже
повільно. Частина українців, яка мріє лише про повний холодильник,  затьмареним
розумом не розуміє, нащо прибирати з вулиць московитські назви і пам'ятники
московитським тиранам. Травмований народ вважає тих тиранів своїми героями, і
цьому народові треба пояснювати, що то так чужа влада чужого народу мітить нашу
землю своїми мітками і нам ці мітки, як рабське клеймо, треба всюди стерти і
помаркувати нашу землю нашими маркерами, нашими пам'ятниками і нашими назвами.
Бо без цього ми лишимося рабами, а не господарями - тому не збудуємо сильної
країни – і, відповідно, не матимемо повного холодильника... Важкою і довгою
дорогою українці повертаються до себе. І повертають свою пам'ять. 

Сьогодні у Всесвітній День вишиванки в Києві, у символічному місці - на
Лук'янівці, було урочисто відкрито пам'ятник Василю Вишиваному. І дуже
символічно цвіли навколо вишиванки на людях, вигравали нашивки-тризубчики на
курсантах, служили українські священники, і лилася Києвом пісня, написана
колись в Карпатах Василем Вишиваним, виконана сучасним українським гуртом
«Друже, Музико» не з Києва, і не з Карпат - з Одеси… Там теж Україна, яка шанує
Вишиваного. 

На підніжжі пам'ятника викарбувані слова Вільгельма фон Габсбурга - Василя
Вишиваного: \enquote{...і я уперто вірю в самостійність і соборність суверенности
нашої, нам всім так милої держави Україна}.

Дякую Генералу Суботіну, Олександрові Юркову, «Музичному батальйону» та всім,
кого гуртуєте навколо такої потрібної української справи.

\subsubsection{Комментарии}

\begin{itemize}
\iusr{Lyubov Gudkova}

Дякую, пані Ларисо, за любов до України, за історичну правду, за Вашу
відданість нашій країні!

\iusr{Светлана Колесникова}

Нехай Бог береже Вас пані Ларисо. Пишаюся Вами, тішуся з того що в Україні є
такі патріоти, читаю Ваші дописи і чекаю нові. З такими патріотами Україні
бути.

Слава Україні!
Слава нації!

\iusr{Sergey Litvinov}

Светлана Колесникова таких «патриотов» хватит лишь на один хутор или одно
отделение в психиатрической больнице. ) Это и будет ваша «вкрайина». )

\iusr{Лейла Сидоренко}

Дякую!

\ifcmt
  pic https://scontent-lga3-2.xx.fbcdn.net/v/t1.6435-9/188685183_963940854351593_7552445637493786563_n.jpg?_nc_cat=111&ccb=1-3&_nc_sid=dbeb18&_nc_ohc=Y_bw4wytlM8AX-9gWQ3&_nc_ht=scontent-lga3-2.xx&oh=3968167428c31977bca96c3ad5d4acfd&oe=60CC49F2
	width 0.3
\fi

\iusr{Lina Kacmar}

Дуже дякую за цю розповідь.

\iusr{Лариса Гіс}

Дякуємо за цікаву розповідь!

Вітаю Вас з Днем Вишиванки!

\iusr{Лиска Лілія}

Дякуємо вам всім за відкриття цього пам'ятника,нашому забутому Герою! Шана
величезна і В Вишиваному,і вам-Українцям!!!

\iusr{Людмила Гаценко}

23 травня День героїв України. Василя Вишиваного вважаймо одним із них.

\iusr{Вікторія Лопата}

Не знала деталей його смерті( аж заплакала(

За Конашевича, за Сірка, за Хмельницького, за Мазепу... а список довшає і довшає( всі могили потрощили(

Про Сагайдачного як прочитала, що спокійно 300 років пролежав у серці Подолу і
тільки в 1935 нелюди разом з собором, де був похований підірвали... Це була
остання крапля. Той пам'ятник і та вулиця здаються нікчемними, коли розумієш,
що там міг досі поряд стояти неймовірний мазепинський перебудований у камені
собор із справжньою могилою легендарного Петра Конашевича-Сагайдачного.... А
ні, має бути все куценьке дешевеньке, зекономдєноє на нас від і до.

\iusr{Володимир Просянко}

Така у нас історія, яку ми не знаємо, яку нам не дають знати, яку де-хто й не
хоче знати... Особисто мені болячі відчувати, що виїхавши з дамбасу, ніби-то до
вільної бандерівської землі, зрозумів, що вільних, незалежних, сміливих
українців сралін вивіз до Магадану, залишивши тут тих, хто міг писати (і таким
залишився) доноси на сусіда, а сусід на нього. Мені тут деякі "інтелігенти"
(про яких таваріщ члєнін говорив, що вони не мозг нації, а її гівно) пишуть,
доводять, аж всераються, що у Львові нема суржика, навіть відсутні
язикогаварящіє... Я спілкуюся з так званим простим народом і здивований його
вихованням, світоглядом. Тепер починаю розуміти, чому зрадили Мазепу, чому з
1917 року ми, як завжди, опинилися під паскудною мацквою, бо ми просто
пєрєсталі стрєлять, патаму шта какая разніца... Знаю і впевнений, що цю статтю
читають з мацкви (а воно їм треба?!!), щоб взяти і просто попаскудити, бо вони
такими паскудами вродилися, але головне і болюче, що їх підтримає савєтскоє (а
тепер- зелене) бидло! Ось така моя особиста характеристика нашій державній
дійсності!

\iusr{Alla Allova}

Вас, упоротых, хоть с головы до ног вышей - так упоротыми и останетесь

\iusr{Natalia Mykytyn}

Alla Allova рузькомірське - в рашку !!!

\end{itemize}
