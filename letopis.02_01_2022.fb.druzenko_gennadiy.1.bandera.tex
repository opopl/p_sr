% vim: keymap=russian-jcukenwin
%%beginhead 
 
%%file 02_01_2022.fb.druzenko_gennadiy.1.bandera
%%parent 02_01_2022
 
%%url https://www.facebook.com/gennadiy.druzenko/posts/10158693288318412
 
%%author_id druzenko_gennadiy
%%date 
 
%%tags bandera_stepan,ideologia,kult,obschestvo,ukraina
%%title СТЕПАН БАНДЕРА: ІСТОРИЧНА ПОСТАТЬ І СУЧАСНИЙ КУЛЬТ
 
%%endhead 
 
\subsubsection{СТЕПАН БАНДЕРА: ІСТОРИЧНА ПОСТАТЬ І СУЧАСНИЙ КУЛЬТ}
\label{sec:02_01_2022.fb.druzenko_gennadiy.1.bandera}
 
\Purl{https://www.facebook.com/gennadiy.druzenko/posts/10158693288318412}
\ifcmt
 author_begin
   author_id druzenko_gennadiy
 author_end
\fi

СТЕПАН БАНДЕРА: ІСТОРИЧНА ПОСТАТЬ І СУЧАСНИЙ КУЛЬТ

Якщо мене спитати, як я ставлюсь до Степана Бандери, напевно я відповім:
стримано позитивно. Він – точно не мій герой, але мені цілком зрозуміло, що в
такий час і в такому віці, в які Бандера потрапив у вир національно-визвольної
боротьби, він майже приречений був стати саме таким радикалом-екстремістом,
яким він увійшов в історію. 

\ii{02_01_2022.fb.druzenko_gennadiy.1.bandera.pic.1}

Врешті-решт, це постать з довгого ряду революціонерів-радикалів, що творили
історію ХХ століття. Російські більшовики, італійські фашисти, німецькі
націонал-соціалісти, радикальні сіоністи, кубинські революціонери, ірландські
та баскські націоналісти і навіть південноафриканські борці проти апартеїду.
Усіх їх об‘єднувало дві речі: віра в насильство як прийнятний засіб політичної
боротьби та претензія на ексклюзивність своєї візії  майбутнього. Іншими
словами, їхній світогляд ґрунтувався на двох засадничих  принципах:

1. Хто не з нами, той проти нас.

2. Велика мета виправдовує будь-які засоби її досягнення.

У міжвоєнний час, коли Бандера вийшов на історичну сцену, ці принципи були
радше мейнстримом, а не девіацією. І тому не дивно, що молода людина сприйняла
та практикувала терор як засіб політичної боротьби не тільки проти поляків
(атентат Перацького) чи совєтів (вбивство Майлова), а й проти українців, які
дотримувалися альтернативного погляду на українську справу. 

Не мені судити людину, що готова була накласти власним життям в боротьбі за
українську державність як вона її розуміла. Попри те, що мені глибоко чужі
методи політичної боротьби, які практикував Бандера, я вважаю, що Степан
Андрійович точно заслужив на розділ в підручниках з української історії та
місце в пантеоні борців за українську державність. Врешті-решт, терор був (і
залишається) зброєю слабких далеко не тільки в українському контексті...

Але я переконаний, що сучасний апофеоз культу Бандери в Україні – це найбільш
успішна, підступна та стратегічна операція Луб‘янки за весь час існування
ЧК-ОГПУ-НКВД-КГБ. І далеко не тільки тому, що глорифікація Бандери отруює
взаємини з нашими закордонними друзями: Польщею та Ізраїлем насамперед, але не
тільки. Вона руйнує зсередини українську справу як таку. Бо боротьба за
державність та розбудова державності – це дві кардинально відмінні стратегії.

Боротьбу за незалежність визначає емансипаційний наратив. За своєю суттю він
нагадує юнацтво: людина прагне довести собі і батькам, що вона може жити і
чинити як їй заманеться. Так і нації: в якісь час вони переростають
ембріональний розвиток в імперському лоні і прагнуть народитись на світ. Для
цього потрібна максимальна мобілізація широких мас, жертовність та гуртування
навколо ідеї незалежності. Всі, хто виступає проти незалежності (розбіжності
можуть бути лише тактичного характеру), – вороги. Це за своєю природою
ексклюзивний наратив: нація (як і конкретна особистість) гуртується навколо
відчуття своєї іншості, окремішності, ексклюзії з більшого суспільного та
державного (як правило імперського) тіла.

Але щойно державність стає фактом, ексклюзію має замінити інклюзія: всі хто не
проти незалежності, всі – наші. Всі – співгромадяни, члени тієї самої політії,
навіть якщо їхня візія змісту та розвитку державності кардинально відмінна. Бен
Ґуріон чортом дихав на Менахема Беґіна, послідовника Жаботинського та очільника
радикальної бойової організації (щось на кшталт ОУН(б)) \enquote{Ірґун}, між бійцями
\enquote{Хагани} (яка підпорядковувалась Бен Ґуріону) та \enquote{Ірґуна} не раз траплялись
збройні сутички, але(!!!) Беґін став депутатом першого Кнесета та шостим
Прем‘єр-міністром Ізраїлю, а не політичним емігрантом чи політв‘язнем у
єврейській державі. Бо навіть для такого затятого та авторитарного лідера як
Бен Ґуріон було зрозуміло: політичні опоненти – це не вороги, а союзники у
боротьбі за незалежний Ізраїль. 

Даруйте, але не можу уявити Мельника та Бульбу-Боровця (не кажучи вже про
поміркованіших українських діячів) в парламенті України, в якій переміг Степан
Бандера... 

І тому сучасна глорифікація Степана Бандери вкрай контрпродуктивна для
української справи. Бо вона переносить імператив боротьби за українську
державність (\enquote{Здобудеш Українську Державу, або згинеш у боротьбі за Неї!}) у
внутрішньополітичний дискурс і тим руйнує національний компроміс, який і є
найпотужнішою гарантією нашої незалежності. За українську державність потрібно
бути готовим вмирати і вбивати (перевірено на практиці в 2014-20 роках), але
вбивати і вмирати за своє ексклюзивне бачення України, змістовного наповнення
цієї державності, – це найкоротший шлях до Руїни. Врешті-решт ми це проходили і
в другій половині 17 століття, і в перебігу Визвольних змагань століття 20-го.
Обидва рази непримиренність українських гетьманів та віра в ексклюзивність
своєї правди завершувалась для України руїною та бажанням посполитих знайти
хоч-якусь стабільність в обіймах Московії...

Глорифікація Бандери – це етос саморуйнування української політичної нації
зсередини. Не даремно найбільш затятими \enquote{бандерівцями} в сучасній Україні є
екс-комуністка Фаріон \url{t.ly/8fSj}, та політичний порученець (і вірогідно
утриманець) Кривецього-\enquote{Пупса} Олег Тягнибок. Пам‘ятаєте принцип радикалів:
велика мета виправдовує будь-які засоби її досягнення? Велич мети \enquote{Свободи} та
ОУН звичайно не порівняти, як і масштаб особистості Бандери і Тягнибока. А от
абсолютна нерозбірливість у засобах залишилась...

Отже, Степана Андрійовича слід згадати незлим вдячним словом у підручниках
історії та знайти йому відповідне місце у національному пантеоні борців за
українську незалежність. Він на це однозначно заслужив. 

А от з сучасним культом Бандери, \enquote{бандеризацією} сучасного українського
дискурсу будь-який свідомий український патріот має з усіх сил боротися. Бо під
личиною мобілізації українців на боротьбу проти російської навали, сучасний
\enquote{бандеризм} вносить в українську політику нетерпимість до опонентів, нездорову
віру в єдиноправильність своєї візії України, легітимацію насильства як
прийнятного способу політичної боротьби та абсолютну нерозбірливість у засобах
досягнення політичних цілей. В цьому сенсі Петро Порошенко – архетипичний
виразник сучасного бандерівства. Яке, на відміну від самого Бандери, готове
класти на вівтар своєї боротьби чужі життя, але не готове – свої власні.

Поклавши Степана Андрійовича до національного пантеону борців за українську
державність та віддавши йому належну шану, нам конче потрібно шукати інших
прикладів для наслідування у розбудові цієї державності. Мудрих, стратегічних,
якими рухає любов до свого, а не ненависть до чужого. Які мають талант
будувати, а не вбивати.

Але в цьому є засаднича проблема: рівнятись на Бандеру набагато легше, аніж
рівнятись на Шептицького. Сповідувати декалог українського націоналіста значно
простіше, аніж намагатись жити так, як проповідував Митрополит Андрей чи
заповідав Василь Стус:

\begin{zznagolos}
\enquote{Коли ти озлобишся, коли повіриш, що світ — жорстокий, і щоб жити в ньому ти
маєш стати злим, світ закриється від тебе, бо твоя жорстокість перетворить
широко відкриті до світу очі на бійниці ненависти, із яких вона струмуватиме до
світу. Тоді й від світу ти матимеш те саме. Попри всю несправедливість, треба
вміти любити, вірити, зберігати надію, бо наш світ — добрий, але обмеженість
людських поглядів викривила його в нашому сприйнятті. Знай це, або, поки, вір}.
\end{zznagolos}

Любити завжди важче, аніж ненавидіти. Але ми не для того такою дорогою ціною
розгризали пуповиння, що єднало нас з русским миром, аби побудувати на
українських теренах ресентиментний український \enquote{мирок}, в якому замість вікон –
бійниці. Задушливу атмосферу імперського потягу було сенс залишати лише за
умови, що ми спроможні не просто перефарбувати наш радянський вагон у
жовто-блакитний колір, а створити посутнісну альтернативу Росії. Здатні замість
провінційного недо-Риму побудувати на берегах Дніпра \enquote{руський Єрусалим}, який
століттями прозрівали наші предки як справжню місію України...

Втім щоб будувати український Єрусалим, нам потрібно наслідувати – як би важко
це не було – Шептицького та Стуса, а не будувати Україну за рецептами Степана
Бандери. Його час давно минув...

\ii{02_01_2022.fb.druzenko_gennadiy.1.bandera.cmt}
