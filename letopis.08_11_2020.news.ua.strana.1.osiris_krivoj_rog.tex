% vim: keymap=russian-jcukenwin
%%beginhead 
 
%%file 08_11_2020.news.ua.strana.1.osiris_krivoj_rog
%%parent 08_11_2020
 
%%url https://strana.ua/news/299793-podrobnosti-ubijstva-sovershennoho-7-nojabrja-v-krivom-rohe-2020.html
%%author 
%%tags 
%%title 
 
%%endhead 

\subsection{"Осирис" устроил резню после возвращения с заработков в Европе. Подробности убийств в Кривом Роге}
\label{sec:08_11_2020.news.ua.strana.1.osiris_krivoj_rog}
\Purl{https://strana.ua/news/299793-podrobnosti-ubijstva-sovershennoho-7-nojabrja-v-krivom-rohe-2020.html}

\Pauthor{Войко, Дмитрий}

\ifcmt
img_begin 
	url https://strana.ua/img/article/2997/93_main-v1604857789.jpeg
	caption Момент задержания "Осириса" в Кривом Роге. Фото: Нацполиция 
	width 0.7
img_end
\fi

Вторые сутки не утихает резонанс вокруг жестоких и бессмысленных убийств,
совершенных вчера в Кривом Роге - на малой родине президента Владимира
Зеленского. Сам глава государства уже назвал эти события абсолютным
терроризмом, вероятно наследуя президента Франции Эммануэля Макрона,
обозначившего так убийство учителя под Парижем на религиозной почве.

Завтра в местном суде будут избирать меру пресечения задержанному по подозрению
в этих жестоких преступлениях. Им оказался 29-летний местный житель Тарас
Усенко.

Сам подозреваемый называет себя "Осирис" (древнеегипетское божество загробного
мира - Ред.). Откуда эти отсылки к мифологии Египта, еще предстоит выяснить.

Так, вчера Тарасу Усенко сообщили о подозрении в двойном убийстве и покушении
на убийство еще восьмерых людей, в том числе ребенка. Криворожская местная
прокуратура ввиду совершения особо опасных преступлений  будет настаивать на
содержании под стражей, без возможности получить альтернативное наказание.

Судя по информации, полученной "Страной" из источников в прокуратуре, несмотря
на распространившуюся информацию о невменяемости Усенко, в распоряжении органов
досудебного расследования на сегодняшний день не было данных, что он стоял на
учете в психоневрологическом диспансере. По случаю выходного дня выяснить это
из-за отсутствия электронного учета душевнобольных нет возможности. Так как
подобная информация по старинке хранится в бумажном виде и выдается даже
полиции только в рабочий день.

По ходу расследования было решено не ходатайствовать перед местным судом о
применении к обвиняемому специальной меры пресечения, которая предусмотрена для
психов (содержание в психиатрической больнице в условиях, исключающих
общественно опасное поведение). Также было решено просить арест без денежного
залога, а потом уже провести сначала амбулаторную, а если потребуется и
стационарную судебно-психиатрическую экспертизу.

В ходе обследований и интервью с психиатрами уже и будет дан ответ на самый
главный вопрос, интересующий следствие - отдавал ли "Осирис" во время
совершения убийств и покушений отчет своим действиям и мог ли ими управлять.
Если психиатры скажут, что Тарас Усенко не может быть осужден, тогда в суд
будет направлен не обвинительный акт, а ходатайство о применении к нему
принудительных мер медицинского характера. А вопрос вины в суде вообще не будет
рассматриваться. В случае признания невменяемости, его ждет принудительное
лечение в Днепропетровской государственной психиатрической больнице закрытого
типа.

В данный момент заместитель министра МВД Антона Геращенко сообщает только
"околесицу" в исполнении подозреваемого Усенко. Он говорит, что является
"посланником ада", "Осирисом" и "должен был убить 15 человек". Стоит напомнит,
что задержали мужчину в момент, когда тот пытался зарезать женщину с ребенком.
При нем был большой кухонный и перочинный складной нож.

Во время неотложного обыска в квартире Усенко, где он жил с родственниками
нашли какие-то непонятные рисунки на стене в его комнате в виде оккультных
знаков. Близкие говорят, что Тарас стал каким-то странным после возвращения из
Чехии, где был на заработках. Возможно там связался с какими-то сектантами.
Однако родные Усенко не говорят, почему не был поднят вопрос о принудительном
обследовании своего домочадца у психиатра, как только заметили изменения.

Если комиссия экспертов придет к тому, что "Криворожский Осирис" вменяем и
попросту валяет дурака, пытаясь выдать себя за психа, ему грозит пожизненное
заключение.

Сам подозреваемый утверждает на аудиозаписи, распространенной в интернете, что
обращался в полицию чтобы ему помешали убивать. Сейчас эта информация
проверяется. Был ли звонок на "102" и если был, то почему на него не
отреагировали должным образом - выясняется. Не исключена еще и халатность
полицейских в дежурной части полиции города, проигнорировавших странный звонок.

"Полиция в ходе расследования будет проверять круг общения подозреваемого в
двойном убийстве и ранении 8 человек для понимания --- не было ли эта трагедия
вдохновлена кем то другим. Не было ли здесь факта внушения, подстрекательства",
– подытожил замминистра Геращенко.

Напомним, 7 ноября около 15:20 в полицию поступило сообщение о неизвестном
мужчине, который на улице в Металлургическом районе Кривого Рога бросался на
людей с ножом. Жертвами злоумышленника стали случайные прохожие, которым
мужчина нанес многочисленные ножевые ранения. В результате два человека
скончались на месте - это был отец и сын. Еще восемь человек, в том числе и
маленький мальчик, доставлены в больницу, где находятся в данный момент.

Правоохранители задержали нападавшего и доставили в Криворожское отделение
полиции. Заместитель министра внутренних дел Антон Геращенко в своем Facebook
написал, что "им оказался 29-летний житель Кривого Рога --- Тарас Усенко. Факта
убийств он не отрицал и прямо заявлял, что он "пришел из ада", что "он Бог",
что некий "голос" приказал ему убить 15 человек".

Ранее "Страна" сообщала, что Зеленский назвал терактом двойное убийство в
Кривом Роге. По словам Зеленского, у него нет сомнений, что сегодняшние
убийства в Кривом Роге являются "проявлением абсолютного терроризма - по
смыслу, по сути, по результатам, по ужасу, который пришел в наши дома".

Также сообщалось, что в Кривом Роге мужчина зарезал двоих прохожих и ранил еще
8 человек, среди которых школьник. Против него возбудили уголовное производство
по статье 115 Уголовного кодекса Украины (умышленное убийство), сейчас
правоохранители выясняют все обстоятельства инцидента.
