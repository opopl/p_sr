% vim: keymap=russian-jcukenwin
%%beginhead 
 
%%file 17_04_2021.fb.kotova_ksenia.1.harizma
%%parent 17_04_2021
 
%%url https://www.facebook.com/xeniya.kotova/posts/4318861481506808
 
%%author 
%%author_id 
%%author_url 
 
%%tags 
%%title 
 
%%endhead 
\subsection{Харизма - Мальчик - Дедушка}
\Purl{https://www.facebook.com/xeniya.kotova/posts/4318861481506808}

Знакомая дама отвезла четырехлетнего сына в гости к бабушке с дедушкой,
надолго, типа мама от внука без ума, а папа хоть суров и не склонен к
сантиментам, но тоже не надышится.

\ifcmt
  pic https://scontent-mxp1-2.xx.fbcdn.net/v/t1.6435-9/174214996_4318860264840263_6248645805026167826_n.jpg?_nc_cat=105&ccb=1-3&_nc_sid=8bfeb9&_nc_ohc=CDtgASOuvTQAX8OBqgX&_nc_ht=scontent-mxp1-2.xx&oh=1b086d3d436068bab2bf0c4492d251c3&oe=60A2635D
\fi

И вот забрала она малютку оттуда и зашла с ним в магазин.

А там споткнулась обо что-то и врезалась в стеллаж, с которого упало несколько предметов — игрушек, что ли.

На что малютка брутальным дедовским басом заметил:

— Ну даваааай, разнеси тут все к х@@м!

Эта история, разумеется, не про вредность обсценной лексики в присутствии
малолетних, а про мощную харизму, которая не может не заразить все податливые
натуры вокруг.

Н. Кочелаева
