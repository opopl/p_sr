% vim: keymap=russian-jcukenwin
%%beginhead 
 
%%file 11_11_2020.news.ua.pravda.1.rusynstvo.final_allegory
%%parent 11_11_2020.news.ua.pravda.1.rusynstvo
 
%%url 
%%author 
%%tags 
%%title 
 
%%endhead 

\subsubsection{Фінальна алегорія}

\begin{itemize}
  \item \dshM Ну що, знайшов русинських радикалів? \dshM дещо в’їдливо питає мене Тарас, коли я сідаю до нього у авто.
  \item \dshM Ні, \dshM відповідаю.
  \item \dshM З чого почнеш свій репортаж?
  \item \dshM З фрази \enquote{Українство в Закарпатті \dshM це вірус}.
  \item \dshM Ого, \dshM дивується Тарас. \dshM Провокаційний початок. А назву вже придумав?
  \item \dshM Так, можливо, \enquote{Русинська карта}.
\end{itemize}

У політичному сенсі на Закарпатті є дві карти, котрі час від часу розігруються
Росією \dshM русинська та угорська.

І дехто з політиків сприймає ці карти як \enquote{туза} і \enquote{короля}.

Але, чесно кажучи, ця пара нагадує комбінацію у покері, яку називають \enquote{Анна
Курнікова} \dshM не лише тому, що ініціали російської тенісистки \enquote{АК} збігаються з
позначеннями туза і короля в колоді.

А ще й тому, що ця комбінація, як і дружина Енріке Іглесіаса, виглядала
привабливо, але дуже рідко вигравала.
