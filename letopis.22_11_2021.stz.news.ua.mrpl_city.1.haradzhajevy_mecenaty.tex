% vim: keymap=russian-jcukenwin
%%beginhead 
 
%%file 22_11_2021.stz.news.ua.mrpl_city.1.haradzhajevy_mecenaty
%%parent 22_11_2021
 
%%url https://mrpl.city/blogs/view/haradzhaevi-veliki-hlibotorgovtsi-i-grandiozni-metsenati
 
%%author_id demidko_olga.mariupol,news.ua.mrpl_city
%%date 
 
%%tags mariupol,mariupol.istoria
%%title Хараджаєви: великі хліботорговці і грандіозні меценати
 
%%endhead 
 
\subsection{Хараджаєви: великі хліботорговці і грандіозні меценати}
\label{sec:22_11_2021.stz.news.ua.mrpl_city.1.haradzhajevy_mecenaty}
 
\Purl{https://mrpl.city/blogs/view/haradzhaevi-veliki-hlibotorgovtsi-i-grandiozni-metsenati}
\ifcmt
 author_begin
   author_id demidko_olga.mariupol,news.ua.mrpl_city
 author_end
\fi

%\ii{22_11_2021.stz.news.ua.mrpl_city.1.haradzhajevy_mecenaty.pic.1}

Продовжуючи цикл, присвячений видатним маріупольцям, які поховані на
найстарішому міському кладовищі, хочу розповісти про видатного діяча, який
першим був удостоєний звання \enquote{Почесний громадянин міста Маріуполя}. Це
Олександр Давидович Хараджаєв – міський голова Маріуполя (1860–1894), купець
першої гільдії і меценат. І сам Олександр Давидович, і його син, і інші члени
родини зробили унікальний внесок в розвиток торговельної справи, будівельної
галузі, освіти і медицини Маріуполя.

Життєвий шлях, роль і внесок Хараджаєвих у своїх розвідках у різний час
дослідували історики та краєзнавці С.Буров, Л. Яруцький, С.Новікова, В.Волонець
та інші.

Олександр Хараджаєв народився в Маріуполі 15(27) жовтня 1826 року у сім'ї
грецького купця \textbf{\emph{Давида Антоновича Хараджи}} (1775–1857). Завдяки участі у
торгівлі зерновим хлібом Давиду Антоновичу вдалося зміцнити добробут родини та
виділяти значні суми на благодійність. Займаючи посаду церковного старости, у
1838 році він пожертвував 5 000 руб. для будівництва собору Святого Харлампія –
найбільшого храму Маріуполя. А у 1852 році направив 5 150 крб. на спорудження
іконостасу головного храмового престолу.

\ii{22_11_2021.stz.news.ua.mrpl_city.1.haradzhajevy_mecenaty.pic.1.large}

\emph{\textbf{Олександр Давидович Хараджаєв}} зміг не тільки продовжити, а й примножити справу
батька.  Він заснував торгову контору із закупівлі зерна у селах Приазов'я та
продажу його за кордон і став власником понад десятка невеликих вітрильних
суден та кількох пароплавів. Серед них - 6 вітрильників каботажного та
закордонного плавання: шхуни \enquote{Агіос Георгіос}, \enquote{Десна}, \enquote{Мімі Мембелі}, \enquote{Святий
Антоній}, \enquote{Святий Миколай} та бот \enquote{Свята Надія}. Під його прапором ходив цілий
приватний торговий флот. О. Хараджаєв сприяв проведенню через Маріуполь
телеграфної лінії Ростов-на-Дону – Одеса для покращення умов торгівлі на
Азовському морі.

Крім того, значні капітали він вкладав у нерухомість. Придбав у
Слов'яно-сербському повіті (нині – Луганська область) кам'яно\hyp{}вугільну копальню.
За рівнем доходів Олександр Хараджаєв був зарахований до купців першої гільдії.
У період із 1860 до 1864 років був міським головою Маріуполя.

Олександр Давидович брав активну участь у громадському житті міста, надавав
благодійну допомогу різним закладам. Працював у комісіях зі створення у місті
гімназій, шкіл та інших закладів. У 1874 році він був обраний Маріупольським
земством опікуном земських народних шкіл.

Олександр Хараджаєв був одружений з дочкою голови Грецького суду, купця Івана
Чебаненка Софії (1833–1915 рр.). У них було четверо дітей: син Давид (нар. 1853
р.), дочки Єлизавета (нар. 1850 р.) Катерина (нар. 1866 р.), і Єпістімея (нар.
1868 р.).

Крім того, значні капітали він вкладав у нерухомість. Придбав у
Слов'яно-сербському повіті (нині – Луганська область) кам'яно\hyp{}вугільну копальню.
За рівнем доходів Олександр Хараджаєв був зарахований до купців першої гільдії.
У період із 1860 до 1864 років був міським головою Маріуполя.

Олександр Давидович брав активну участь у громадському житті міста, надавав
благодійну допомогу різним закладам. Працював у комісіях зі створення у місті
гімназій, шкіл та інших закладів. У 1874 році він був обраний Маріупольським
земством опікуном земських народних шкіл.

Олександр Хараджаєв був одружений з дочкою голови Грецького суду, купця Івана
Чебаненка Софії (1833–1915 рр.). У них було четверо дітей: син Давид (нар. 1853
р.), дочки Єлизавета (нар. 1850 р.) Катерина (нар. 1866 р.), і Єпістімея (нар.
1868 р.).

\ii{22_11_2021.stz.news.ua.mrpl_city.1.haradzhajevy_mecenaty.pic.2}

\textbf{\emph{Софія Іванівна Хараджаєва}}, як і її чоловік, активно займалася благодійністю та
розвитком освіти в Маріуполі. Була почесною опікункою Маріїнської жіночої
гімназії, для учениць якої заснувала стипендію імені своїх батьків – Івана та
Катерини Чебаненко, а також постійно жертвувала гроші на придбання навчальних
посібників та інші потреби. У 1893 році за рішенням училищної ради
Олександрівської чоловічої гімназії було відкрито історико-археологічний музей.
Він розміщувасяу будинку О.Д. Хараджаєва. Головною метою було зібрати матеріали
давнини, що перебувають у місцевих установах та у приватних осіб. Загалом
вдалося зібрати 419 експонатів.

Пішов з життя Олександр Давидович 6(18) червня 1894 року в рідному Маріуполі,
де й похований на старому міському цвинтарі. Олександр Хараджаєв першим був
удостоєний звання \enquote{Почесний громадянин міста Маріуполя}. 

Син Олександра Давидовича – \emph{\textbf{Давид Хараджаєв}} – залишив дуже яскравий слід в
історії Маріуполя. Його активна участь у громадському та культурному житті
міста неодноразово були відзначені не лише маріупольцями, а й державою. Він
продовжив розвивати експортний бізнес батька, розширив операції із закупівлі на
сусідні повіти та став найбільшим хліботоргівцем Приазов’я. Давиду
Олександровичу вдалося вивести сімейний бізнес на міжнародний рівень, і
торговий дім Хараджаєвих увійшов до шістки найбільших фірм Маріуполя, які
займалися продажем зерна за кордон, серед яких тільки його фірма була місцевого
походження, а інші – іноземні. Давид Олександрович Хараджаєв побудував новий
маріупольський порт, нові склади та провів телефон у порту. 1910 року його
обрали головою біржового комітету Маріуполя. Він відіграв велику роль у
місцевому самоуправлінні міста, був гласним міської думи, повітового
маріупольського та губернського катеринославських земських зборів. Його обирали
почесним мировим суддею, він був головою Товариства допомоги бідним, Товариства
спасіння на водах, членом правління Товариства тверезості. На пожертвування Д.
Хараджаєва було збудовано міську лікарню та протитуберкульозний санаторій.
Кошти на його утримання виділялися Давидом Олександровичем і збиралися на
благодійних святах \enquote{білої квітки}.

Особливу опіку мецената мала Маріупольська чоловіча гімназія. Він заснував
стипендії для незаможних учнів, збудував  церкву при гімназії. Давид Хараджаєв
був опікуном гімназії 30 років. Він вислужив чин дійсного світського радника,
був нагороджений російськими та грецькими орденами за внесок у зміцнення
з'язків двох країн, був віце-консулом Греції у Маріуполі. У Маріупольському
повіті з ініціативи та за підтримки Д. О. Хараджаєва було відкрито лазарети при
заводах Нікополь-Маріупольського товариства та \enquote{Російський Провіданс}, міському
акцизному винному складі, в селах Михайлівка, Ігнатівка, Нова Каракуба, Темрюк,
Волноваха.

\ii{22_11_2021.stz.news.ua.mrpl_city.1.haradzhajevy_mecenaty.pic.3}

Відомо, що Давид Хараджаєв сприяв розвитку будівельної галузі у місті. Йому
належав Олександрівський завод з виробництва цегли та черепиці, який почав
роботу в 1898 році. Це підприємство вважалося найбільшим у будівельній галузі в
Маріуполі. Воно виробляло продукцію на 15 000 руб. на рік, з кількістю
робітників 75–82 особи.

Завдяки успішній комерційній діяльності Давид Олександрович мав найвищий
авторитет у діловому світі Маріуполя, тому з 1910 до 1919 років очолював
біржовий комітет і арбітражну комісію Маріупольської товарної біржі. Входив до
складу повітового та губернського сільськогосподарських комітетів, представляв
інтереси міста різних нарадах і з'їздах з питань хлібної торгівлі. Ця
діяльність було відзначена Маріупольським земством і Давиду Хараджаєву було
винесено подяку допомогу у боротьбі з неврожаєм 1902 року.

У 1905 р. відбулося вшанування почесного громадянина Маріуполя, гласного Давида
Хараджаєва з нагоди 25-річчя його громадської діяльності.

Роки Першої світової війни Хараджаєв був головою маріупольського комітету
Червоного Хреста, який допомагав пораненим солдатам. Революція 1917 року
похитнула бізнес Хараджаєва, він опинився в таборі прихильників встановлення
твердої консервативної влади. У 1918 році після приходу до влади гетьмана Павла
Скоропадського Хараджаєв був призначений повітовим старостою Маріупольського
повіту. Навесні 1919 року Давид Хараджаєв разом із сім'єю репатріювався до
Греції, його власність була конфіскована більшовиками (в одному з його
особняків розташувався комітет партії, зруйнований у роки Другої світової
війни). На всіх фото представлений саме \emph{\textbf{Давид Хараджаєв}}.

\ii{22_11_2021.stz.news.ua.mrpl_city.1.haradzhajevy_mecenaty.pic.4}

Сьогодні пам'ятник Олександру Давидовичу Хараджаєву частково відновлений
завдяки зусиллям Андрія Марусова та інших маріупольських волонтерів ГО
\enquote{Архі-місто}. Проте він потребує подальшої реставрації та упорядкування. Як
бачимо, статус міського голови на кладовищі не дає жодних привілеїв. І все ж
пам'ять про широку і багатогранну діяльність великих хліботорговців та
грандіозних меценатів – купців Хараджаєвих – маріупольці мають зберегти і
шанувати...
