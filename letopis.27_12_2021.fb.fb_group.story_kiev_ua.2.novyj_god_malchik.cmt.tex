% vim: keymap=russian-jcukenwin
%%beginhead 
 
%%file 27_12_2021.fb.fb_group.story_kiev_ua.2.novyj_god_malchik.cmt
%%parent 27_12_2021.fb.fb_group.story_kiev_ua.2.novyj_god_malchik
 
%%url 
 
%%author_id 
%%date 
 
%%tags 
%%title 
 
%%endhead 
\zzSecCmt

\begin{itemize} % {
\iusr{Альбина Орел}
Помню, что мама сама мне шила пачку снежинки и белые тапочки к утреннику)))

\iusr{Наталия Луц}
\textbf{Альбина Орел} мне тоже

\iusr{Наталия Луц}
Мне тоже

\iusr{Lora Berdichevsky}
\textbf{Наталия Луц} и мне @igg{fbicon.face.grinning.smiling.eyes} 

\iusr{Катя Бекренева}
Маленький мальчик символизировал Новый год), к концу года он превращался в Деда Мороза)

\begin{itemize} % {
\iusr{Maksym Oleynikov}
\textbf{Катя Бекренева} 

Такого не пам'ятаю ні тоді, ні пізніше. Було таке, що Дід Мороз із Снігуркою
забавляють дітей, аж тут звляється хлопчик - Новий рік, і це символізувало
очікуваний прихід нового року, і Дід Мороз прощався.... Мабуть, різноманітні
були сценарії.

\end{itemize} % }

\iusr{Ирина Иванченко}
Приємні ваші спогади, до душі, дякую.

Ніколи не користувалася \enquote{казенними}, у дитинстві новорічні костюми шили
самостійно, завжди вигадували щось оригінальне, не як у всіх.


\iusr{Татьяна Кривицкая}

Гарний петрушка. Трохи пізніше Схоже і в мене було. Я була сніжинкою, а
\enquote{чешки} мамуля пошила за ніч, з білоі тканини, плаття прикрасили з мішури
сніжинками. Я була найщасливіша і найкраща сніжинка! А, саме головне, в наступному
році був Мир і молоді батьки.

\iusr{Анна Доманская}

Новорічні спогади з дитинства самі теплі і чудові. Саморобні новорічні костюми і
прикраси на ялинку, все було казкове в дитинстві.

\iusr{Elena Marijchuk}

А мне мама тапочки - балетки ВЯЗАЛА КРЮЧКОМ. Получались САМЫЕ КРАСИВЫЕ.

\iusr{Inna Radzinsky}

Дякую за приємні спогади. Мені того ж 1959 року в дитячому садку на вулиці Анрі
Барбюса довелося бути Снігуронькою. Незважаючи на те, що я була бодай
найменшого зросту в групі. Мене вповноважили на цю посаду тому що я була здатна
запам'ятати і проголосити досить довгу новорічну промову. Біло-блакитний
кожушок і такий самий капелюшок були оздоблені ватою з безліччю блискучих
плямок з потовчених кольорових кульок. Десять хвилин акторської слави в
невимовно красивому одязі - один з найяскравіших спогадів дитинства.

\begin{itemize} % {
\iusr{Maksym Oleynikov}
\textbf{Inna Radzinsky} 

Інночко, ти й у дорослому віці була аж ніяк не дилдою....
@igg{fbicon.face.smiling.eyes.smiling} 

\iusr{Наталя Будзинська}
А мене ось так відягнули в садку в 1958

\ifcmt
  ig https://scontent-frt3-1.xx.fbcdn.net/v/t39.30808-6/270317983_4523673071075751_8336366262052736821_n.jpg?_nc_cat=104&ccb=1-5&_nc_sid=dbeb18&_nc_ohc=eiBh2RBsqqwAX-QsWH-&_nc_ht=scontent-frt3-1.xx&oh=00_AT8C6q4AXmaypwDKHyRWpP1U5mBppYUcSe3mWNmtBnpjYQ&oe=61D3C636
  @width 0.3
\fi

\begin{itemize} % {
\iusr{Maksym Oleynikov}
\textbf{Наталя Будзинська} О!!! Звідки така розкіш? З якої скрині?

\iusr{Наталя Будзинська}
\textbf{Maksym Oleynikov} 

Садок був відомчий, від заводу - ДВРЗ. Нас вдягали й годували дуже добре, я
цього не пам'ятаю, бо мені на цьому фото ще не виповнилося 3-років. Мама
розповідала, завод виготовляв і ремонтував вагони з усієї України, і , мабуть,
Союзу. До речі, кілька років тому зробили фільм про дільничного з ДВРЗ,
прототипом якого був справжній дільничий, гроза всіх хуліганів на ДВРЗ -
прізвище Маценко. От з його дочкою я була в одній групі в цьому самому садку


\iusr{Olena Korenets Nebuchadnezzar}
\textbf{Nataly Budzynska} Яка ж гарнесенька!

\iusr{Inna Radzinsky}
\textbf{Наталя Будзинська} \textbf{Maksym Oleynikov} 

Цікаво - мій садок теж був відомчий, від Київського хлібозаводу. Теж дуже добре
годували, від чого я плакала бо не любила їсти (тоді) @igg{fbicon.face.wink.tongue} 

\end{itemize} % }

\end{itemize} % }

\iusr{Alexey Novozhylov}

Колись в дитинстві пам'ятаю як дуже злякався якогось родича, перевдягнутого в
дєда мороза, здається то був 67 або 68 рік. Подумалось чомусь, як українці
Канади або Америки тоді Різдво святкували, колядки співали, Св. Миколай, Санта
Клаус, цікаво мабуть...  @igg{fbicon.face.smiling.eyes.smiling}{repeat=2} 

\iusr{Тамара Мельничук}
Але все було стабільно при дефіцитах всього

\iusr{Єлизавета Бойко}
\textbf{Tamara Melnichuk} Дійсно,що стабільно не було нічого.

\iusr{Владимир Мазур}

В послевоенные годы такие костюмы петрушки шили из марли, колпаки делали из
ватмана и украшали обвёртками от конфет. Все были счастливы

\iusr{Тамара Ар}
Сейчас очень похожие детские костюмы есть в детском мире, но по приличной цене

\iusr{Аліса Забой}

1956 рік - перший в історії СРСР міжнародний фестиваль молоді та студентів!
Небачене і неймовірне для післясталінської доби...

\begin{itemize} % {
\iusr{Анна Дубницкая}
\textbf{Alisa Zaboy} 1957 рік.

\iusr{Аліса Забой}
\textbf{Анна Дубницкая} дякую, але в пам'яті відбився 1956 й... Може двадцятий з'їзд?

\iusr{Аліса Забой}
\textbf{Анна Дубницкая} але, враження і відчуття від того не забути...

\iusr{Анна Дубницкая}
\textbf{Alisa Zaboy} Так, ХХ з'їзд КПРС.

\iusr{Аліса Забой}
\textbf{Анна Дубницкая} Вже перевірила... Для дітей-школярів - фестиваль, для батьків - з'їзд...
\end{itemize} % }

\iusr{Людмила Терещенко}
Все девочки снежинки, мальчики зайцы)))

\iusr{Natasha Kovalchuk}

Новогодний костюм @igg{fbicon.snowflake} из марли, до сих пор хранится у Мамы! Вот уже 44
года @igg{fbicon.man.facepalming}  Помню, как мама его шила, а Папа вырезал самые красивые
снежинки.

\ifcmt
  ig https://scontent-frx5-1.xx.fbcdn.net/v/t39.30808-6/270204999_10220224308108390_3115706692024135972_n.jpg?_nc_cat=100&ccb=1-5&_nc_sid=dbeb18&_nc_ohc=_YQdK8odr1YAX961-kQ&_nc_ht=scontent-frx5-1.xx&oh=00_AT9EVIw2rXSC45hE_YTjJbvVc3f9JnzpraX2iHPLrp-xwQ&oe=61D3DD1B
  @width 0.3
\fi

\begin{itemize} % {
\iusr{Наталья Чекотун}
\textbf{Natasha Kovalchuk} А у меня был даже не из марли,а из широкого бинта. Юбка в несколько ярусов. Накрохмалено,канечно
\end{itemize} % }

\iusr{Нина Бондаренко}

От і я пам'ятаю ті білі тапки, що мама шила для щороку під Новий рік. А ще
пам'ятаю, віршик:

Открываем календарь -

Начинается - январь...

А я з натхненням кричала: \enquote{...Комендант... Комендант..}, тому що Дідом Морозом
одягли коменданта гуртожитку. Садочки були відомчі, всі жили у відомчих (
заводських) будинках, всі знали один одного...


\iusr{Наталья Борчанинова}

Ось що є в моєму архіві))) Це Запрошення моє і моєі сестри ( м. Біла Церква)

Зустріч НР 1968!!!

\ifcmt
  ig https://scontent-frx5-1.xx.fbcdn.net/v/t39.30808-6/269977182_1549087795458726_7137437500624395470_n.jpg?_nc_cat=105&ccb=1-5&_nc_sid=dbeb18&_nc_ohc=7IYsy7t_5XoAX8ZE2On&_nc_ht=scontent-frx5-1.xx&oh=00_AT-oRq2zJ0ed1v_R5CH_LwE1_c4rUHLwL07rXVdkAbuI9g&oe=61D530E2
  @width 0.4
\fi

\iusr{Валентина Мірошниченко}
НЕ завжди хлопчик. У 1964 році в Жовтневому палаці НР була я.  @igg{fbicon.laugh.rolling.floor} 

\begin{itemize} % {
\iusr{Maksym Oleynikov}
\textbf{Валентина Мірошниченко} 

Чудово! Але на теренах СРСР від 50-х і фактично до 80-х це був хлопчик, бо рік
- чоловічого роду, тож на сцені могла бути і дівчинка, але вдягнена саме так,
як хлопчик-Новий рік... @igg{fbicon.face.savoring.food} 


\iusr{Валентина Мірошниченко}
\textbf{Maksym Oleynikov} Так.
\end{itemize} % }

\iusr{Larisa Dmitrakova}
Нищета..

\begin{itemize} % {
\iusr{Maksym Oleynikov}
\textbf{Larisa Dmitrakova} Добре, що Ви це розумієте. Бо у нас тьма-тьмуща тих, хто сприймає ті часи як щось незрівнянно краще, аніж сьогодення...

\iusr{Mike Kaufman-Portnikov}
\textbf{Maksym Oleynikov} пропоную вважати це психіатричним захворюванням. Це просто жах.

\iusr{Раиса Сухарева}
\textbf{Maksym Oleynikov} ото ж тьма -тьмуща.

\iusr{Юлия Александровна}
\textbf{Maksym Oleynikov} 

для многих - это всеря детской и юношеской беззаботности, где небыло проблем,
родители живы, праздники всем миром праздновались небыло по телику злобы, а в
людях зависти, потому что все были бедны)

Вспоминают счастливые для себя времена, вспоминибт счастливых себя)

\end{itemize} % }

\iusr{Larisa Dmitrakova}

\ifcmt
  ig https://scontent-frt3-2.xx.fbcdn.net/v/t39.30808-6/270102197_1021946631723143_8662830794233552072_n.jpg?_nc_cat=103&ccb=1-5&_nc_sid=dbeb18&_nc_ohc=SEy_o48Nb3oAX9VHH2u&_nc_ht=scontent-frt3-2.xx&oh=00_AT_ouPq0FW4A-SGHhXbf5JTSTvO-C1OiLMOt9g9Vv4EjGA&oe=61D487D1
  @width 0.4
\fi

\end{itemize} % }
