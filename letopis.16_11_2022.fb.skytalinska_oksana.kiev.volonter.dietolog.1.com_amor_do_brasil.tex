% vim: keymap=russian-jcukenwin
%%beginhead 
 
%%file 16_11_2022.fb.skytalinska_oksana.kiev.volonter.dietolog.1.com_amor_do_brasil
%%parent 16_11_2022
 
%%url https://www.facebook.com/O.Skytalinska/posts/pfbid0211AEF9zLQst1VANpNTeDgUx8AoMAjjmKdZbdq5wKQkLyBUrahiSAybqWicLnLZwVl
 
%%author_id skytalinska_oksana.kiev.volonter.dietolog
%%date 
 
%%tags 
%%title COM AMOR DO BRASIL (з любов'ю з Бразилії)
 
%%endhead 
 
\subsection{COM AMOR DO BRASIL (з любов'ю з Бразилії)}
\label{sec:16_11_2022.fb.skytalinska_oksana.kiev.volonter.dietolog.1.com_amor_do_brasil}
 
\Purl{https://www.facebook.com/O.Skytalinska/posts/pfbid0211AEF9zLQst1VANpNTeDgUx8AoMAjjmKdZbdq5wKQkLyBUrahiSAybqWicLnLZwVl}
\ifcmt
 author_begin
   author_id skytalinska_oksana.kiev.volonter.dietolog
 author_end
\fi

COM AMOR DO BRASIL (з любов'ю з Бразилії)
🇧🇷@igg{fbicon.heart.red} @igg{fbicon.flag.ukraina}
Незважаючи на те, що Україна і Бразилія знаходяться далеко одна від одної -- ми  близькі серцями, тому що ми однаково думаємо.
Вчора Бойові Бджоли побували в гостях у пана Norton Rapesta, Надзвичайного та Повноважного Посла Бразилії в Україні на його запрошення.
Перше, що сказав Пан Посол, що любить Україну всім серцем.
До речі, бразильські добровольці служать у складі Інтернаціонального Легіону ЗСУ. В червні бразильські добровольці ЗСУ "віджали" у росіян БМП-2. Двоє бразильців: модель Таліта До Валле та Буглас Бурріго, загинули, захищаючи нашу країну, вічна пам'ять Героям.
Бразилія та Україна мають і спільну історію: в Бразилії є потужна українська діаспора, в якій навіть через 5 поколінь бразильські українці пам'ятають українську мову та традиції.
Пан Посол -- проста і щира людина; живе життям звичайних киян, знає, що таке відключення світла і звик до виття сирен.
Наша зустріч відбувалась під час повітряної тривоги.
Пан Посол подарував і особисто заніс в машину 4 коробки із сотнею пакуночків з цукерками для дітей, які живуть у деокупованих та прифронтових зонах. Це частинка любові від бразильських дітей. Маленька листівочка "Від дітей Бразилії для відважних дітей України" вкладена у кожному пакуночку. Нехай слово "Бразилія" асоціюється з чимось приємним вже з дитинства.
Це лише один із кроків на довгому шляху співпраці наших двох народів. На початку грудня запланували відвідини діток, які проходять лікування  та реабілітацію у  дитячій лікарні. 
День обіцяв бути прекрасним, якби не обстріли. Ми проїжджали по Печерську і бачили пожежні машини та швидкі допомоги, та сірий дим за будинками. 
Так і живемо. Спілкуємось із людьми, близькими по духу, робимо маленькі щоденні внески у Велику Перемогу і віримо в Україну. 
Любіть Україну у сні й наяву,
вишневу свою Україну,
красу її, вічно живу і нову,
і мову її солов'їну.
Любіть у труді, у коханні, у бою,
як пісню, що лине зорею...
Всім серцем любіть Україну свою -- 
і вічні ми будемо з нею!
Велика подяка всім причетним до цієї зустрічі: Валентина Бенедь, Olga Saltykova @igg{fbicon.heart.red}

\ii{16_11_2022.fb.skytalinska_oksana.kiev.volonter.dietolog.1.com_amor_do_brasil.orig}
\ii{16_11_2022.fb.skytalinska_oksana.kiev.volonter.dietolog.1.com_amor_do_brasil.cmtx}
