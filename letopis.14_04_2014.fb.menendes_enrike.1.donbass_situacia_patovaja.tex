% vim: keymap=russian-jcukenwin
%%beginhead 
 
%%file 14_04_2014.fb.menendes_enrike.1.donbass_situacia_patovaja
%%parent 14_04_2014
 
%%url https://www.facebook.com/e.menendes/posts/691600497549046
 
%%author_id menendes_enrike
%%date 
 
%%tags donbass,ukraina,vojna
%%title В Донбассе ситуация конечно патовая
 
%%endhead 
 
\subsection{В Донбассе ситуация конечно патовая}
\label{sec:14_04_2014.fb.menendes_enrike.1.donbass_situacia_patovaja}
 
\Purl{https://www.facebook.com/e.menendes/posts/691600497549046}
\ifcmt
 author_begin
   author_id menendes_enrike
 author_end
\fi

У меня такая масса информации – хочу ей делиться. Написал много текста, немного
сумбурно. Если что – спрашивайте, буду рассказывать то, что знаю то, как вижу
я. 

В Донбассе ситуация конечно патовая. Никто реально ничего не понимает, я тоже
не уверен, что понимаю лучше кого-то, но всё же попробую разложить всё по
полочкам. В таком положении  прогнозы дело не благодарное – всё меняется в
момент одним непредсказуемым действием. Я думал, что пик мы прошли, а на
выходных случилось резкое обострение. Поехали:

1) Главная проблема ситуации в том, что заинтересованные стороны не хотят
слышать друг друга уже очень давно. Говорят, что евреи потому так хорошо живут,
что умеют договариваться. А залог успешных переговоров – это принятие и
уважение интересов оппонента, даже если они тебе противны. Так вот – там, где
не договариваются – там стреляют. Увы – и тут уже кто-кого сильнее.

2) Давайте мыслить логически, какие игроки сейчас представлены на арене (это
здорово проясняет ситуацию):

Новая власть в Киеве – она не имеет ни ресурсов, ни понимания, что делать на
Востоке. Их задача усмирить регион – за последние три недели задача
эволюционировала от «сделать Донбасс верным вассалом» до «ладно, мы согласны на
ваши условия, только не бунтуйте». Я упростил специально, сейчас детали не
важны. Одновременно с этим, Киев не всегда понимает, что в регионе 80\% проблемы
– это «местные», в чьих глазах Киев не легитимен и лишь 20\%, а то и меньше,
Россия и лично ВВП).  Так вот, смена власти вызвала здесь реальное возмущение.
Майдан выпустил джина из бутылки и несмотря на все благородные побуждения
именно он привёл к накалению на Востоке. Нужно это просто принять. Только время
лечит такие вещи.

В Киеве на Власть давит Майдан – та его радикальная часть, которая не верит в
переговоры, а считает, что нужно выжигать несогласных калёным железом. На мой
взгляд, это деструктивная сила, которая мешает переговорам. Она нам сейчас не
союзник. Те, кто называет весь Донбасс биомассой – фашисты и с ними нужно
бороться. Не кормите эту силу агрессивными постами в фейсбуке. Не создавайте
дополнительные причины для конфликта.

А еще есть ЮЛЯ. Она очень не любит Донбасс, который отвечает ей взаимностью. У
неё нет шансов на президентство. Зато она очень комфортно чувствует себя, когда
вокруг полный ад. Как у себя дома, прям. Может поэтому Киев частно ведёт себя
так странно? Ей сейчас это выгодно.

В Донбассе есть народ (как бы странно это сейчас не звучало) – в большой своей
массе люди напуганы, не понимают что происходит, над ними довлеют мифы разной
степени идиотичности (от автобусов с бандеровцами, до искажения слов Фарион,
сказанных взаправду). Реальная поддержка ухода в Россию среди населения где-то
18\% (в то время как за единую Украину более 60\%). Еще плюс 12-15\% просто не
любят новую власть в разной степени жёсткости, но в РФ не хотят. С этим,
вообще-то, нужно было работать все 23 года независимости, а не грести всех под
одну гребёнку (Вы прямо сейчас со мной не согласны, но вспомните про
уважение/принятие оппонента). Именно РЕАЛЬНАЯ поддержка среди населения, а не
ростовские гастролёры делают возможным контр-революцию в Донбассе.

В том же самом народе 10\% очень патриотически настроенной прослойки, которая
может стать опорой в регионе. Не для новой власти, а вообще для строительства
нового государства (не ДНР само-собой).

Еще 50-60\% вообще уже потеряны и нифига не понимают. Они уже хотят, чтобы всё
закончилось. Они сейчас не субъект. Но среди них нужно тоже проводить
информационную работу и искать союзников.

Есть местная элита. Она тоже разная. Начнём с тех, кто “мутит воду” – это
ментовско-прокурорская тусовка, плотно засевшая в регионе и привыкшая считать
его своим. Именно полная лояльность этой тусы к Семье и лично ВФЯ лежит в
основе всего, что здесь происходит. Батя крепко держит их за яйца и именно это
так невыгодно отличает нас от Днепропетровска и Харькова. При этом рядовой и
даже средний состав силовиков – это тоже простые люди с которыми надо вести
диалог, объясняя, что реально происходит. Например, сказать, что в состав
России никто не войдёт и надеяться на это глупо. Их руководители по итогу
сбегут, а им тут жить. Это отрезвляет. Что делать с этой тусой –
договариваться. Вести просвещение. Им нужно напоминать, чем они рискуют. Не
угрожать им силой – силы не хватает для такого, а просто вежливо говорить, что
в Приднестровье элиты нет. Есть криминалитет. Они хотят такого даунгрейда?

Криминалитет, кстати, есть и у нас. В камуфляже в Славянске стоит енакиевская
братва и местные криминалы, а не российские ГРУшники (может они и есть, но
реального пруфа я не видел). 

Вторая часть элиты – это лично РЛА (плюс его группа влияния). Он уже далеко не
бандит и вообще он очень умён. Его бизнес полностью легален. Он абсолютно не
заинтересован в отделении региона. В его бизнесе ключевым для генерирования
денег является доступ к дешевому сырью с ГОКов в центре Украины. Отделение
Донбасса может перерезать эту схему. Ахметов этого точно не хочет. Равно, как
не хочет и в Россию. На его уровне вообще не нужно резких движений. У него есть
пряник и для Путина, и для Яценюка, и для Януковича. Тем не мене, даже Ахметов
чувствует себя проигравшим от Майдана. С Батей поступили не по-пацански. Думаю
РЛА и БВК это не одобрили. Открыто кинуть ВФЯ они тоже не могу (не знаю почему,
лично не общался) – и кроме всего наверное они тоже не решили свои переговорные
задачи с Киевом. РЛА – ключевое лицо переговоров. Кстати, эти люди, насколько
мне известно, достаточно спокойны. Причина этого мне не ясна.

Остальная часть местной элиты – я их всех запихнул в один сегмент. Тут много
разных персонажей. Им просто нужно озвучивать те же риски, что и остальным.
Сбежать отсюда они конечно смогут. Но уже будут не респектабельными
бизнесменами, а эмигрантами.

Россия. Тут всё сложно. С одной стороны глупо не признавать, что интересы РФ в
регионе были, есть и будут, а противодействовать им с точки зрения силы мне
представляется мало-возможным. С другой – сейчас они страшилка для страны и
новая власть делает на это слишком большой упор. Не было бы тут местных «тёрок»
- никакая Россия не смогла бы здесь мутить воду. Главное, что нужно знать – это
то, что один Донбасс им не нужен. Во-первых, реально не нужен. Во-вторых, есть
Осетия и Абхазия, которые будут очень удивлены, если примут Донбасс, не приняв
их.

3) Что делать-то? Призывы вооружить патриотов и послать стрелять в сепаратистов
лично я расцениваю, как призыв к гражданской войне. Натравить армию на
восставших, сказав, что это не местные, а российский спецназ? Представляете,
что тут начнётся после того, как погибнет первый местный ополченец? Тогда
Россия войдёт сюда уже не с вторжением, а с реальной миротворческой миссией.

Короче, как бы странно это не звучало, учитывая ситуацию – нужно ждать и
договариваться. Обещать локальной договороспособной элите то, что она хочет
услышать (без фанатизма), не раскачивать эскалацию. С силовиками вести
пропаганду – объяснять, что кроме профита в жизни еще бывают риски и иногда
лучше об этом подумать сразу, а не потом. Сепаратисты перегниют внутри. Там не
все оголтелые – радикалы устали сидеть без движения, а тот движ, которого они
хотят нужен не всем. Турчинов согласен на референдум и децентрализацию – это
многих успокоит. А тех, кого не успокоит нужно держать на мушке. Как только они
захотят экшна – а это случиться – нужно их нейтрализовывать или ликвидировать
(это в крайнем случае). Пару таких ситуций должны здорово отрезвить людей.

Кстати о людях – время лечит. Истерика не может быть вечной. Успокойте людей.
Говорите с ними, каждый на доступном ему уровне.

Если не делать всего этого – тогда точно война.

Простым людям нужно думать о себе – быть осторожными и запастись необходимым.
Не паниковать. Не нагнетать. Продолжать жить, пока ничего непоправимого не
случилось.

Это просто информация – вам для размышления. Есть вопросы - задавайте. Есть
информация, которая противоречит моей – сообщите, только без истерики и
обвинений.

P.S. Всё написанное не отрицает возможности деструктивного вмешательства
России. Увы.
