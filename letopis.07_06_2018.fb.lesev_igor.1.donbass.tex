% vim: keymap=russian-jcukenwin
%%beginhead 
 
%%file 07_06_2018.fb.lesev_igor.1.donbass
%%parent 07_06_2018
 
%%url https://www.facebook.com/permalink.php?story_fbid=1943151002382660&id=100000633379839
 
%%author_id lesev_igor
%%date 
 
%%tags avakov_arsen,donbass,putin_vladimir,ukraina,vojna
%%title Два рассинхрона в один день
 
%%endhead 
 
\subsection{Два рассинхрона в один день}
\label{sec:07_06_2018.fb.lesev_igor.1.donbass}
 
\Purl{https://www.facebook.com/permalink.php?story_fbid=1943151002382660&id=100000633379839}
\ifcmt
 author_begin
   author_id lesev_igor
 author_end
\fi

Два рассинхрона в один день. Утром Аваков говорит о полицейской миссии на
Донбассе. И так загадочно намекает, что у него имеется козырный инсайд,
позволяющий «приоткрыть карты». Например, что зачистить республики можно будет
и без армии, используя всего-то две тысячи правильных полицейских.

\ifcmt
  ig https://scontent-frt3-1.xx.fbcdn.net/v/t1.6435-9/34640512_1943150775716016_6842888921307676672_n.jpg?_nc_cat=102&ccb=1-5&_nc_sid=730e14&_nc_ohc=FJhuVhTXCIoAX-xj0RF&_nc_ht=scontent-frt3-1.xx&oh=db1fdc742b95fd22a9628f5118ca7516&oe=61B95C34
  @width 0.4
  %@wrap \parpic[r]
  @wrap \InsertBoxR{0}
\fi

Чуть позже, к обеду, от Путина уже звучит совершенно противоположный месседж. А
по сути, угроза. Наступление ВСУ на Донбассе – это последствия для всей
украинской государственности. Достаточно специфические слова, особенно если
учитывать, что ранее Москва неоднократно подчеркивала принадлежность территории
ЛДНР именно за Украиной.

Но что мы имеем по итогу? Аваков начинает комментировать те направления, где
его функционал не распространяется. Донбасс за военными, а военные за
Порошенко. МВД работает только на внутриукраинских территориях Донбасса, да и
то, с определенными оговорками в пользу СБУ и ВСУ. А учитывая достаточно
агрессивное заявление Путина, которое Аваков точно не ожидал, можно сделать
вывод, что глава МВД смачно так блефует. Нет никакого у него инсайда о скорых
договоренностях по Донбассу.

А что тогда у него есть? Есть желание пропетлять через год. Избирательная
кампания уже началась и Арсену Борисовичу крайне важно определиться с местом
сидения. И есть версия, что он аккуратно подстраивается под Юлию Владимировну.
Месседжи Авакова направлены на:

а) подчеркивание неэффективности АП и лично Порошенко на Донбассе и 

б) наличие хитрого плана, который позволит правильным профессионалам разрулить
ситуацию с учетом всех интересов патриотической спильноты.

Очевидно, что у Авакова нет не только работающего плана, но даже рычагов
влияния на ситуацию в зоне конфликта. Но когда именно ярчайший представитель
«партии войны» делает подобные заявления о неэффективности действующей методы
на Донбассе, он в первую очередь плюет в огород Порошенко. И подыгрывает
Тимошенко, которая как раз в силу электоральной скованности, не может делать
резкие выпады и давать однозначные оценки по ситуации на востоке.

Хотя Авакову эти раскладывания яиц все равно не помогут. Он слишком токсичен и
способен выживать именно в Украине, исключительно находясь на верхушке пищевой
цепи. Как только потеряет пост главы МВД, его схарчит даже не Юля (пусть у них
и будут договоренности), а то же НАБУ. Как раз в следующем году заработает
новый модный суд, и Аваков первым пойдет на рассмотрение.

Ну а комментарий Путина с таким же издевательски самориторическим вопросом –
зачем вы признали Порошенко – выдает раздражение. Вне всяких сомнений, ЛДНР
останется в таком виде до 2019 года. Потом возможны варианты, но обсуждать их
будут уже не с Петром Алексеевичем и, тем более, не с Арсеном Борисовичем. Но
ведь и у них те еще хлопоты на носу. Привести в порядок дома в Испании и
Италии, разбить виноградники, порешать визовые вопросы, написать мемуары. И это
не значит, что Путин после 2019-го получит Украину. Ни хрена он не получит. Но
и от нас «немытая Россия» с нашим чудесным шизофреническим сообществом дальше
через год не станет. Просто мы все потеряем бесцельно еще один год.

\ii{07_06_2018.fb.lesev_igor.1.donbass.cmt}
