% vim: keymap=russian-jcukenwin
%%beginhead 
 
%%file 12_02_2019.stz.news.ua.mrpl_city.2.parki_mrpl_skulptury_ukrashennie_detmi
%%parent 12_02_2019
 
%%url https://mrpl.city/news/view/parki-mariupolya-ukrasyat-skulptury-pridumannye-detmi-foto
 
%%author_id ivanova_jana.mariupol,news.ua.mrpl_city
%%date 
 
%%tags 
%%title Парки Мариуполя украсят скульптуры, придуманные детьми (ФОТО)
 
%%endhead 
 
\subsection{Парки Мариуполя украсят скульптуры, придуманные детьми}
\label{sec:12_02_2019.stz.news.ua.mrpl_city.2.parki_mrpl_skulptury_ukrashennie_detmi}
 
\Purl{https://mrpl.city/news/view/parki-mariupolya-ukrasyat-skulptury-pridumannye-detmi-foto}
\ifcmt
 author_begin
   author_id ivanova_jana.mariupol,news.ua.mrpl_city
 author_end
\fi

\ii{12_02_2019.stz.news.ua.mrpl_city.2.parki_mrpl_skulptury_ukrashennie_detmi.pic.1}

В Мариуполе представят коллекцию детских эскизов, по которым мастера создадут
скульптуры. Кованые изделия в будущем планируют установить в парках и на
детских площадках города.

Инициатором творческого проекта выступил
\href{https://www.facebook.com/events/1093822007489272}{\enquote{Коллегиум Анны
Ярославны Восток}}.%
\footnote{\url{https://www.facebook.com/events/1093822007489272}}
На выставке зрители проголосуют за самые интересные
изображения юных художников Мариуполя, Мангуша и Сартаны.

\begin{quote}
\em\enquote{Экспозиция будет доступна к просмотру всего два часа, и тем интереснее с нею
ознакомиться. Все желающие увидят эскизы металлических скульптур, созданные
учениками художественных школ. Вы можете отдать свой голос за привлекательные,
на ваш взгляд, эскизы. Из 100 рисунков выберут 20, которые в дальнейшем могут
быть воплощены в парковые скульптуры профессиональными мастерами-кузнецами со
всей страны}, 
\end{quote}
- отмечают организаторы проекта.

\ifcmt
  ig https://i2.paste.pics/PTC1T.png?trs=1142e84a8812893e619f828af22a1d084584f26ffb97dd2bb11c85495ee994c5
  @wrap center
  @width 0.9
\fi

\ifcmt
  ig https://i2.paste.pics/PTC1V.png?trs=1142e84a8812893e619f828af22a1d084584f26ffb97dd2bb11c85495ee994c5
  @wrap center
  @width 0.9
\fi

Уточняется, что на детских площадках Мариуполя и в парковых зонах металлические
скульптуры по эскизам детей планируют установить в 2019 году.

Выбрать будущие скульптуры, придуманные детьми, горожане могут 13 февраля с
13:00 до 15:00 в Мариупольском филиале Национальной академии изобразительного
искусства по ул. Георгиевской, 69 (вход с ул. Архитектора Нильсена).

Проект \enquote{Коллегиума Анны Ярославны Восток} и Мариупольской организации
Национального союза художников Украины станет участником конкурса от
Украинского культурного фонда.

Ранее в Мариуполе по инициативе мастеров в Мариуполе открыли \href{https://mrpl.city/news/view/simbioz-metalla-i-kamnya-v-mariupolskom-parke-otkryli-novye-art-obekty-foto-plusvideo}{музей под открытым
небом}%
\footnote{Симбиоз металла и камня. В мариупольском парке открыли новые арт-объекты, Анастасія Селітріннікова, mrpl.city, 22.11.2018, \par\url{https://mrpl.city/news/view/simbioz-metalla-i-kamnya-v-mariupolskom-parke-otkryli-novye-art-obekty-foto-plusvideo}}
\enquote{Обереги Украины} с коваными скульптурами.

Напомним, что в Мариуполе предлагают построить \href{https://mrpl.city/news/view/v-mariupole-predlagayut-postroit-krepost-muzej-domaha-foto}{музей-кре\hyp{}пость \enquote{Домаха}},%
\footnote{В Мариуполе предлагают построить крепость-музей \enquote{Домаха}, Яна Іванова, mrpl.city, 11.02.2019, \par\url{https://mrpl.city/news/view/v-mariupole-predlagayut-postroit-krepost-muzej-domaha-foto}}
который может стать символом патриотизма и новым туристическим объектом.

%\ii{12_02_2019.stz.news.ua.mrpl_city.2.parki_mrpl_skulptury_ukrashennie_detmi.txt}
