% vim: keymap=russian-jcukenwin
%%beginhead 
 
%%file 01_05_2021.fb.bilchenko_evgenia.1.bog_velosipedist
%%parent 01_05_2021
 
%%url https://www.facebook.com/yevzhik/posts/3850449031656845
 
%%author 
%%author_id 
%%author_url 
 
%%tags 
%%title 
 
%%endhead 
\subsection{БЖ. Теодицея}
\Purl{https://www.facebook.com/yevzhik/posts/3850449031656845}


\ifcmt
  pic https://scontent-bos3-1.xx.fbcdn.net/v/t1.6435-9/180771951_3850448981656850_2643685417531839383_n.jpg?_nc_cat=101&ccb=1-3&_nc_sid=8bfeb9&_nc_ohc=laeAP2ncUFgAX-fp7YG&_nc_ht=scontent-bos3-1.xx&oh=b5dc85461db8a5e5e8a2c2fd62f2dc14&oe=60B25A6D
\fi


Бог существует, и в качестве сущего Он - велик:
Так на вЕлик садится маленький гражданин,
Начитавшийся материнско-отцовских книг
Про каверинский путь планеты от трав до льдин.

Он уверен, что сможет всё до конца пройти:
Любые пути - нипочём за его плечом.
Маленький гражданин едет в большой горсти
Того, кто не видим ни миром и ни мечом.

Маршрут его - запределен, предельно чист,
Как всё, что бывает в детстве, пока мы не
Перестаём длить веру, что Бог - велосипедист
Из книги про капитана с полочки на стене.

1 мая 2021 г.
PS. Theodicea «богооправдание» от др. -греч. θεός «бог, божество» + δίκη «право, справедливость») — совокупность религиозно-философских доктрин, призванных оправдать управление Вселенной добрым Божеством, несмотря на нал

