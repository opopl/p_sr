%%beginhead 
 
%%file 25_01_2023.fb.fb_group.mariupol.pre_war.1.naprik_nts__1970_kh_
%%parent 25_01_2023
 
%%url https://www.facebook.com/groups/1233789547361300/posts/1396262157780704
 
%%author_id fb_group.mariupol.pre_war,kipcharskij_viktor.mariupol
%%date 25_01_2023
 
%%tags mariupol,mova,afrika,student,jumor
%%title Наприкінці 1970-х у Жданові (так тимчасово - 40 років - звався Маріуполь) з'явилися перші афростуденти
 
%%endhead 

\subsection{Наприкінці 1970-х у Жданові (так тимчасово - 40 років - звався Маріуполь) з'явилися перші афростуденти}
\label{sec:25_01_2023.fb.fb_group.mariupol.pre_war.1.naprik_nts__1970_kh_}
 
\Purl{https://www.facebook.com/groups/1233789547361300/posts/1396262157780704}
\ifcmt
 author_begin
   author_id fb_group.mariupol.pre_war,kipcharskij_viktor.mariupol
 author_end
\fi

Наприкінці 1970-х у Жданові (так тимчасово - 40 років - звався Маріуполь)
з'явилися перші афростуденти. Вони 9 місяців (чому саме 9? Може тому, щоб
встигли побачити своїх нащадків?) вивчали російську мову у Ростові і вже звикли
до того, що викликали до себе підвищену увагу, яку час від часу підігрівали
різними витівками.

Та от у магазині \enquote{Паляниця} біля гуртожитку вони розіграли таку сценку. Питає
такий продавчинь ламаною російською:

- Скільки коштує сантиметр ковбаси?

Вона серйозно пояснює, що у нас ковбасу не продають на сантиметри.

Він уважно слухає, киває головою, а потім питає:

- А скільки коштує метр ковбаси?

От я зараз гадаю: а може то був мандрівник у часі і він просто зарано з'явився?
Бо ж тільки зараз яйця покупають на кілограми...  🤔🤓😁

Отут колись була \enquote{Паляниця}...

Фото надав Google Maps
