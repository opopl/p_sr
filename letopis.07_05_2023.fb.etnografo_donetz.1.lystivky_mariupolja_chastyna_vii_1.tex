%%beginhead 
 
%%file 07_05_2023.fb.etnografo_donetz.1.lystivky_mariupolja_chastyna_vii_1
%%parent 07_05_2023
 
%%url https://www.facebook.com/etnografo.donetz/posts/pfbid0u6Hq266kSZzTNgqgEkPNhws1m8FogUePmDyshofohRDe1oYbJSPtJLdN3HHFZQjvl
 
%%author_id etnografo_donetz
%%date 07_05_2023
 
%%tags 
%%title Листівки Маріуполя - Частина VII/1
 
%%endhead 

\subsection{Листівки Маріуполя - Частина VII/1}
\label{sec:07_05_2023.fb.etnografo_donetz.1.lystivky_mariupolja_chastyna_vii_1}

\Purl{https://www.facebook.com/etnografo.donetz/posts/pfbid0u6Hq266kSZzTNgqgEkPNhws1m8FogUePmDyshofohRDe1oYbJSPtJLdN3HHFZQjvl}
\ifcmt
 author_begin
   author_id etnografo_donetz
 author_end
\fi

Частина VII/1

\#КольороваДонеччина
\#Маріуполь

Лідерами  по кількості кольорових листівок є повітове місто Маріуполь
Катеринославської губернії, та позаштатне місто Слов'янськ Ізюмського повіту
Харківської губ, тому листівки з цими містами буду викладати частинами, бо їх
багато. Сьогодні Маріуполь-порт, Маріуполь рибацький, Азовське море і річка
Кальміус. Воно і не дивно, що по цим населеним пунктам їх найбільше. Маріуполь
центр повіту. Маріуполь це порт. Це важка промисловість. Слов'янськ це
позаштатне місто,  Слов'янськ це курорт. Слов'янськ це залізничний вузол де
крім двох станцій Слов'янськ та Слов'янський Курорт є ще міська залізнична
гілка, ну і звісно це Сіль заводи.
