% vim: keymap=russian-jcukenwin
%%beginhead 
 
%%file 30_12_2020.news.lnr.lug_info.lugansk_inform_center.1.vsu_samoubijstva
%%parent 30_12_2020
 
%%url http://lug-info.com/news/one/dva-boitsa-vsu-pokonchili-s-soboi-iz-za-konfliktov-s-komandirami-narodnaya-militsiya-63376
 
%%author 
%%author_id lugansk_inform_center
%%author_url 
 
%%tags 
%%title Два бойца ВСУ покончили с собой из-за конфликтов с командирами – Народная милиция
 
%%endhead 
 
\subsection{Два бойца ВСУ покончили с собой из-за конфликтов с командирами – Народная милиция}
\label{sec:30_12_2020.news.lnr.lug_info.lugansk_inform_center.1.vsu_samoubijstva}
\Purl{http://lug-info.com/news/one/dva-boitsa-vsu-pokonchili-s-soboi-iz-za-konfliktov-s-komandirami-narodnaya-militsiya-63376}
\ifcmt
	author_begin
   author_id lugansk_inform_center
	author_end
\fi

\index[rus]{АТО!Самоубийства!Два бойца ВСУ, 30.12.2020}

Два киевских силовика в зоне так называемой операции объединенных сил ("ООС")
покончили с собой из-за конфликтов с командирами. Об этом на брифинге сообщил
офицер пресс-службы управления Народной милиции ЛНР Иван Филипоненко.

"29 декабря, находясь на взводном опорном пункте в районе населенного пункта
Выскрива, покончил с собой военнослужащий 2-го батальона 92-й бригады рядовой
Ефременюк. Со слов сослуживцев, командир батальона не отпускал военнослужащего
в отпуск по личным обстоятельствам", - сказал офицер пресс-службы.

Он отметил, что в подразделение прибыли сотрудники военной службы правопорядка
для выяснения причин происшествия. 

"В этот же день, находясь в суточном наряде, произвел выстрел в голову из
автомата военнослужащий 3-го батальона 80-й отдельной десантно-штурмовой
бригады старший сержант Гончар, - продолжил Филипоненко. - В ходе
предварительного разбирательства установлено, что боевик не выдержал давления
со стороны командира роты из-за отказа сдачи денежных средств на новогодний
подарок комбригу (полковнику Владимиру) Швораку".

Представитель оборонного ведомства Республики добавил, что в связи с
участившимися случаями суицидов среди военнослужащих, находящихся в зоне
проведения "ООС", в период новогодних праздников запланирована работа мобильных
групп военных психологов в соединениях оперативно-тактической группировки
"Север".

"Целью является анкетирование военнослужащих для выявления реального уровня
морально-психологического состояния и выявление военнослужащих, склонных к
нервному срыву и готовых совершить самоубийство", - подытожил Филипоненко.

В Народной милиции неоднократно отмечали, что морально-психологическое
состояние киевских силовиков ухудшается, они продолжают мародерствовать и
устраивать "пьяные разборки", участились случаи неуставных взаимоотношений,
которые приводят к дезертирству и самоубийствам среди военнослужащих ВСУ. В
течение 2020 года из зоны проведения "ООС" дезертировали более 250
военнослужащих ВСУ.\Furl{http://lug-info.com/news/one/bolee-250-boitsov-vsu-dezertirovali-iz-zony-oos-za-god-narodnaya-militsiya-63201}

Власти Украины начали силовую операцию против Донбасса в апреле 2014 года.
Урегулирование конфликта базируется на Комплексе мер по выполнению Минских
соглашений, подписанном 12 февраля 2015 года в белорусской столице участниками
Контактной группы и согласованном с главами стран - участниц "нормандской
четверки" (Россия, Германия, Франция и Украина). Документ, в частности,
предусматривает прекращение огня и отвод тяжелых вооружений от линии
соприкосновения.

{\bfseries 
ЛуганскИнформЦентр — 30 декабря — Луганск
}

