% vim: keymap=russian-jcukenwin
%%beginhead 
 
%%file slova.voda
%%parent slova
 
%%url 
 
%%author_id 
%%date 
 
%%tags 
%%title 
 
%%endhead 
\chapter{Вода}

%%%cit
%%%cit_head
%%%cit_pic
%%%cit_text
Не следовало мне так делать. Все дело в том, что я очень долго живу в стране
людей и начал уже утрачивать чувство различия между собой и людьми. Но для
отшельника и, наверное, для многих других людей я напоминание о другом мире, к
которому люди все еще чувствуют отвращение и неприязнь, и, вероятно, не без
основания. Веками люди и мой народ жестоко боролись друг с другом без
милосердия и, как предполагаю, без чести. В результате отшельник, который, как
вы говорите, самый добрый из людей, не знал, как себя со мной вести. Он, должно
быть, знал, что я безвреден и не представляю никакой угрозы ни для него, ни для
его расы, и все же чувствовал беспокойство. Если я был бы дьяволом или
каким-нибудь демоном, он знал бы, как действовать, брызнул бы \emph{святой водой} и
произнес бы заклинание. Хоть я не дьявол, но, однако, каким-то неведомым путем
мысли обо мне связываются с дьяволом. Все эти годы я сожалел, что навестил его
%%%cit_comment
%%%cit_title
\citTitle{Зачарованное паломничество}, Клиффорд Саймак
%%%endcit

%%%cit
%%%cit_head
%%%cit_pic
%%%cit_text
– В это я, конечно, не верю, – сказал Голан Тревиз, стоя на широких ступенях
Селдон-Холла и глядя сверху на сверкающий в солнечном свете город.  Терминус
был мягкой планетой с высоким коэффициентом \emph{водного} пространства.
Введение контроля над погодой сделало планету более комфортабельной и – как
часто думал Тревиз – менее интересной
%%%cit_comment
%%%cit_title
\citTitle{Кризис основания}, Айзек Азимов
%%%endcit

%%%cit
%%%cit_head
%%%cit_pic
%%%cit_text
Гукові було дуже цікаво слухати дивного супутника, але він досі не міг втямити,
до чого тут його особиста трагедія, яке вона має відношення до незвичайних
здібностей, про які він натякнув. Думав запитати про це, але рюкзак нестерпно
різав плечі, хотілося трохи відпочити й напитися. Гук зупинився й жестом
запропонував присісти. Юнак глянув на спітніле обличчя супутника, допоміг йому
розв’ючитися, а потім, усміхаючись, дивився, як той умивався біля потоку та
жадібно ковтав прохолодну \emph{воду}. Сівши під високим багатовіковим берестом,
химерний хлопець, заплющивши очі, прислухався до шелесту листя. Гук вернувся на
стежку, сів поряд з ним, попросив оповідати далі
%%%cit_comment
%%%cit_title
\citTitle{Вогнесміх}, Олесь Бердник
%%%endcit

%%%cit
%%%cit_head
%%%cit_pic
%%%cit_text
И вот они ушли, добрые товарищи по паломничеству, исчезли вместе с машиной
Джоунза.  Корнуэлл, тяжело ступая, пошел вслед за остальными по ночному лугу к
прекрасному зданию, сверкавшему в лунном свете. Суетились маленькие существа,
среди которых не шел, а как бы плыл Сторож. Немного в стороне своей неровной
походкой ковылял Жестянка.  «Вот и пришел конец, — подумал Корнуэлл, — конец
долгого пути, который начался в Вайлусинге, когда я нашел спрятанную книгу».
Такого конца он не мог себе вообразить. Он искал древних, но теперь древние его
уже не интересовали: они оказались совсем не такими, как он думал.  Он вспомнил
вечер, когда они впервые после Сожженной равнины нашли \emph{воду}. Он тогда винил
себя за то, что увлек всех в паломничество. Он знал, что возвращение той же
дорогой сулит верную гибель. Теперь же все кончено, и нет необходимости
возвращаться. Перед ним работа на целую жизнь и даже больше, чем на одну жизнь.
Здесь, если прав Сторож, есть возможность слить три великие культуры в одну еще
более великую, возможно, с помощью ученых из других миров, вооруженных
неизвестными знаниями и философиями. К тому же тут есть и неизвестный фактор —
этот Жестянка. И даже намека нет на то, к чему все это может привести
%%%cit_comment
%%%cit_title
\citTitle{Зачарованное паломничество}, Клиффорд Саймак
%%%endcit
