% vim: keymap=russian-jcukenwin
%%beginhead 
 
%%file man.errors
%%parent body
 
%%endhead 
\section{DEALING WITH ERRORS, PROBLEMS, ETC}
  
\vspace{0.5cm}
 {\ifDEBUG\small\LaTeX~section: \verb|man.errors| project: \verb|latexmk| rootid: \verb|p_saintrussia| \fi}
\vspace{0.5cm}
  
Some possibilities:

a. If you get a strange error, do look carefully at the output that  is
on  the  screen  and  in  log  files.   While  there  is  much  that is
notoriously verbose in the output of latex (and that is added to by la-
texmk),  the verbosity is there for a reason: to enable the user to di-
agnose problems.  Latexmk does repeat some messages at the end of a run
that  it  thinks would otherwise be easy to miss in the middle of other
output.

b. Generally, remember that latexmk does its work by running other pro-
grams.  Your first priority in dealing with errors should be to examine
what went wrong with the individual programs.  Then you need to correct
the  causes of errors in the runs of these programs.  (Often these come
from errors in the source document, but they could also be about  miss-
ing LaTeX packages, etc.)

c. If latexmk doesn't run the programs the way you would like, then you
need to look in this documentation at the list of command line  options
and  then at the sections on configuration/initialization files.  A lot
of latexmk's behavior is configurable to deal  with  particular  situa-
tions.  (But there is a lot of reading!)

The  remainder  of  these notes consists of ideas for dealing with more
difficult situations.

d. Further tricks can involve replacing the standard commands that  la-
texmk runs by other commands or scripts.

e.  For possible examples of code for use in an RC file, see the direc-
tory  example_rcfiles  in  the  distribution  of  latexmk   (e.g.,   at
http://mirror.ctan.org/support/latexmk/example_rcfiles).  Even if these
examples don't do what you want, they may provide suitable inspiration.

f. There's a useful trick that can be used when you  use  lualatex  in-
stead  of  pdflatex  (and  in some related situations).  The problem is
that latexmk won't notice a dependency on a file, bar.baz say, that  is
input  by  the  lua code in your document instead of by the LaTeX part.
(Thus if you change bar.baz and rerun latexmk, then latexmk will  think
no  files have changed and not rerun lualatex, whereas if you had '\in-
put{bar.baz}' in the LaTeX part of the document, latexmk  would  notice
the  change.)   One  solution is just to put the following somewhere in
the LaTeX part of the document:

\begin{verbatim}
	\typeout{(bar.baz)}
\end{verbatim}

       This puts a line in the log file that latexmk will  treat  as  implying
       that  the file bar.baz was read.  (At present I don't know a way of do-
       ing this automatically.)  Of course, if the file has a different  name,
       change bar.baz to the name of your file.

       g. See also the section "Advanced Configuration: Some extra resources".

       h.    Look   on   tex.stackexchange,   i.e.,   at   http://tex.stackex-
       change.com/questions/tagged/latexmk  Someone may  have  already  solved
       your problem.

       i. Ask a question at tex.stackexchange.com.



                                 17 April 2020                              24








       j.  Or ask me (the author of latexmk).  My e-mail is at the end of this
       documentation.



ALLOWING FOR CHANGE OF OUTPUT FILE TYPE
       When one of the latex engines is run, the usual situation is that latex
       produces  a .dvi file, while pdflatex and lualatex produce a .pdf file.
       For xelatex the default is to produce a .pdf file, but to optimize pro-
       cessing  time  latexmk  runs xelatex its -no-pdf option so that it pro-
       duces an .xdv file.  Further processing by  latexmk  takes  this  as  a
       starting point.

       However, the actual output file may differ from the normal expectation;
       and then latexmk can adjust its processing to accommodate  this  situa-
       tion.   The  difference in output file type can happen for two reasons:
       One is that for latex, pdflatex and lualatex the  document  itself  can
       override  the defaults. The other is that there may be a configuration,
       or misconfiguration, such that the program that latexmk invokes to com-
       pile  the  document is not the expected one, or is given options incom-
       patible with what latexmk initially expects.  (E.g.,  the  -output-for-
       mat=...   option  could  be used with lualatex, or xelatex gets invoked
       without the -no-pdf option.)

       Under latex and pdflatex, control of the output format by the  document
       is  done by setting the \pdfoutput macro.  Under lualatex, the \output-
       mode macro is used instead.

       One example of an important use-case for document control of the output
       format  is  a document that uses the psfrag package to insert graphical
       elements in the output file. The psfrag package achieves its effects by
       inserting postscript code in the output of the compilation of the docu-
       ment.  This entails the use of compilation to a .dvi file, followed  by
       the  use  of  conversion  to  a postscript file (either directly, as by
       dvips or implicitly, as an intermediate step by dvipdf).   Then  it  is
       useful  to  force output to be of the .dvi format by inserting \pdfout-
       put=0 in the preamble of the document.

       Another example is where the document uses graphics file of  the  .pdf,
       .jpg,  and  png types.  With the default setting for the graphicx pack-
       age, these can be processed in compilation to .pdf but not with  compi-
       lation  to  .dvi.  In this case, it is useful to insert \pdfoutput=1 in
       the preamble of the document to force compilation to .pdf  output  for-
       mat.

       In all of these cases, it is needed that latexmk has to adjust its pro-
       cessing to deal with a mismatch between the actual output  format  (out
       of  .pdf,  .dvi,  .xdv) and the initially expected output, if possible.
       Latexmk does this provided the following conditions are met.

       The first is that latexmk's $allow_switch configuration variable is set
       to  a  non-zero value as it is by default.  If this variable is zero, a
       mismatch of filetypes in the compilation results in an erro.




                                 17 April 2020                              25








       The second condition for latexmk to be able to handle a change of  out-
       put  type is that no explicit requests for .dvi or .ps output files are
       made.   Explicit  requests  are  by  the  -dvi  and  -ps,   -print=dvi,
       -print=ps,  -view=dvi,  and -view=ps options, and by corresponding set-
       tings of the $dvi_mode, $postscript_mode, $print_type, and  $view  con-
       figuration  variables.   The print-type and view-type restrictions only
       apply when printing and viewing are explicitly requested, respectively.
       For  this  purpose,  the use of the -pdfdvi and -pdfps options (and the
       corresponding setting of the $pdf_mode variable) does not count  as  an
       explicit  request  for the .dvi and .ps files; they are merely regarded
       as a request for making a .pdf file together with an  initial  proposal
       for the processing route to make it.

       Note that when accommodating a change in output file type, there is in-
       volved a substantial change in the network of rules that  latexmk  uses
       in  its  actions.  The second condition applied to accommodate a change
       is to avoid situations where the change in the rule network is too rad-
       ical to be readily handled automatically.



CONFIGURATION/INITIALIZATION (RC) FILES
       In  this section is explained which configuration files are read by la-
       texmk. Subsequent sections "How  to  Set  Variables  in  Initialization
       Files",  "Format  of  Command  Specifications",  "List of Configuration
       Variables Usable in Initialization Files", "Custom  Dependencies",  and
       "Advanced  Configuration"  give  details  on what can be configured and
       how.

       Latexmk can be customized using initialization files, which are read at
       startup in the following order:

       1) The system RC file, if it exists.
          On a UNIX system, latexmk searches for following places for its sys-
       tem RC file, in the following order, and reads the first it finds:
          "/opt/local/share/latexmk/LatexMk",
          "/usr/local/share/latexmk/LatexMk",
          "/usr/local/lib/latexmk/LatexMk".
          On a MS-Windows system it looks for "C:\latexmk\LatexMk".
          On a cygwin system (i.e., a MS-Windows system in which Perl is  that
       of cygwin), latexmk reads the first it finds of
          "/cygdrive/c/latexmk/LatexMk",
          "/opt/local/share/latexmk/LatexMk",
          "/usr/local/share/latexmk/LatexMk",
          "/usr/local/lib/latexmk/LatexMk".

       In addition, it then tries the same set of locations, but with the file
       name replaced "LatexMk" replaced by "latexmkrc".

       If the environment variable LATEXMKRCSYS is set, its value is  used  as
       the name of the system RC file, instead of any of the above.

       2) The user's RC file, if it exists.  This can be in one of two places.
       The traditional one is ".latexmkrc" in the user's home directory.   The



                                 17 April 2020                              26








       other  possibility  is "latexmk/latexmkrc" in the user's XDG configura-
       tion home directory.  The actual file read is the first  of  "$XDG_CON-
       FIG_HOME/latexmk/latexmkrc"  or  "$HOME/.latexmkrc" which exists.  (See
       https://specifications.freedesktop.org/basedir-spec/basedir-spec-lat-
       est.html for details on the XDG Base Directory Specification.)

       Here  $HOME  is  the  user's  home  directory.  [Latexmk determines the
       user's home directory as follows:  It is the value of  the  environment
       variable  HOME,  if this variable exists, which normally is the case on
       UNIX-like systems (including Linux and OS-X).  Otherwise  the  environ-
       ment  variable USERPROFILE is used, if it exists, which normally is the
       case on MS-Windows systems. Otherwise a blank string is used instead of
       $HOME, in which case latexmk does not look for an RC file in it.]

       $XDG_CONFIG_HOME  is  the  value  of  the environment variable XDG_CON-
       FIG_HOME if it exists.  If this environment variable  does  not  exist,
       but  $HOME  is  non-blank,  then $XDG_CONFIG_HOME is set to the default
       value of $HOME/.config.  Otherwise $XDG_CONFIG_HOME is blank,  and  la-
       texmk does not look for an RC file under it.


       3)  The  RC  file  in  the current working directory.  This file can be
       named either "latexmkrc" or ".latexmkrc", and the first of these to  be
       found is used, if any.

       4) Any RC file(s) specified on the command line with the -r option.

       Each RC file is a sequence of Perl commands.  Naturally, a user can use
       this in creative ways.  But for most purposes, one simply  uses  a  se-
       quence of assignment statements that override some of the built-in set-
       tings of Latexmk.  Straightforward cases can be handled without  knowl-
       edge  of  the  Perl  language by using the examples in this document as
       templates.  Comment lines are introduced by the "#" character.

       Note that command line options are obeyed in the order  in  which  they
       are written; thus any RC file specified on the command line with the -r
       option can override previous options but can be  itself  overridden  by
       later  options on the command line.  There is also the -e option, which
       allows initialization code to be specified in latexmk's command line.

        For possible examples of code for in an RC file, see the directory ex-
       ample_rcfiles  in  the  distribution  of  latexmk (e.g., at http://mir-
       ror.ctan.org/support/latexmk/example_rcfiles).


HOW TO SET VARIABLES IN INITIALIZATION FILES
       The important variables that can be configured  are  described  in  the
       section  "List  of  configuration  variables  usable  in initialization
       files".  (See the earlier  section  "Configuration/Initialization  (rc)
       Files"  for  the  files where the configurations are done.)  Syntax for
       setting these variables is of the following forms:

                           $bibtex = 'bibtex %O %B';




                                 17 April 2020                              27








       for the setting of a string variable,

                           $preview_mode = 1;

       for the setting of a numeric variable, and

                           @default_files = ('paper', 'paper1');

       for the setting of an array of strings.  It is possible  to  append  an
       item to an array variable as follows:

                           push @default_files, 'paper2';

       Note  that  simple  "scalar"  variables  have names that begin with a $
       character and array variables have names that begin with a @ character.
       Each statement ends with a semicolon.

       Strings  should  be  enclosed  in single quotes.  (You could use double
       quotes, as in many programming languages.  But then the  Perl  program-
       ming  language  brings  into  play some special rules for interpolating
       variables into strings.  People not fluent in Perl will want  to  avoid
       these complications.)

       You  can do much more complicated things, but for this you will need to
       consult a manual for the Perl programming language.




FORMAT OF COMMAND SPECIFICATIONS
       Some of the variables set the commands that latexmk uses  for  carrying
       out  its  work, for example to generate a .dvi file from a .tex file or
       to view a postscript file.  This section describes some important  fea-
       tures of how the commands are specified.  (Note that some of the possi-
       bilities listed here do not apply to the $kpsewhich variable;  see  its
       documentation.)

       Placeholders:  Supposed you wanted latexmk to use the command elatex in
       place of the regular latex  command,  and  suppose  moreover  that  you
       wanted  to  give  it the option "--shell-escape".  You could do this by
       the following setting:

            $latex = 'elatex --shell-escape %O %S';

       The two items starting with the % character  are  placeholders.   These
       are  substituted by appropriate values before the command is run.  Thus
       %S will be replaced by the source file that elatex will be applied  to,
       and  %O will be replaced by any options that latexmk has decided to use
       for this command.  (E.g., if you used the -silent option in the invoca-
       tion  of  latexmk,   it  results in the replacement of %O by "-interac-
       tion=batchmode".)

       The available placeholders are:




                                 17 April 2020                              28








       %A     basename of the main tex file.  Unlike %R, this is unaffected by
              the setting of a jobname by  the -jobname option or the $jobname
              configuration value.

       %B     base of filename for current command.   E.g.,  if  a  postscript
              file  document.ps  is being made from the dvi file document.dvi,
              then the basename is document.

       %D     destination file (e.g., the name of  the  postscript  file  when
              converting a dvi file to postscript).

       %O     options

       %P     If  the  variable $pre_tex_code is non-empty, then %P is substi-
              tuted  by  the  contents  of  $pre_tex_code  followed  by   \in-
              put{SOURCE},  where  SOURCE  stands  for  the name of the source
              file.  Appropriate quoting is done.  This enables TeX code to be
              passed  to  one  of the *latex engines to be executed before the
              source file is read.

              If the variable $pre_tex_code is the empty string,  then  %P  is
              equivalent to %S.

       %R     root filename.  This is the base name for the main tex file.

              By  default  this is the basename of the main tex file.  However
              the value can be changed by the use of the  -jobname  option  or
              the $jobname configuration variable.

       %S     source  file  (e.g.,  the name of the dvi file when converting a
              .dvi file to ps).

       %T     The name of the primary tex file.

       %U     If the variable $pre_tex_code is non-empty, then  its  value  is
              substituted  for %U (appropriately quoted).  Otherwise it is re-
              placed by a null string.

       %Y     Name of directory for auxiliary output files (see the configura-
              tion variable $aux_dir).  A directory separation character ('/')
              is appended if $aux_dir is non-empty and does not end in a suit-
              able character, with suitable characters being those appropriate
              to UNIX and MS-Windows, i.e., ':', '/' and '\'.   Note  that  if
              after  initialization,  $out_dir is set, but $aux_dir is not set
              (i.e., it is blank), then latexmk  sets  $aux_dir  to  the  same
              value $out_dir.

       %Z     Name  of directory for output files (see the configuration vari-
              able $out_dir).  A directory separation character ('/')  is  ap-
              pended  if  $out_dir is non-empty and does not end in a suitable
              character, with suitable characters being those  appropriate  to
              UNIX and MS-Windows, i.e., ':', '/' and '\'.

       If  for  some  reason you need a literal % character in your string not



                                 17 April 2020                              29








       subject to the above rules, use "%%".

       Appropriate quoting will be applied to the filename  substitutions,  so
       you  mustn't  supply them yourself even if the names of your files have
       spaces in them.  (But if your TeX filenames have spaces in them, beware
       that  some  older  versions  of the TeX program cannot correctly handle
       filenames containing spaces.)  In case latexmk's quoting does not  work
       correctly  on your system, you can turn it off -- see the documentation
       for the variable $quote_filenames.

       The distinction between %B and %R needs a bit of care, since  they  are
       often  the same, but not always.  For example on a simple document, the
       basename of a bibtex run is the same as for the texfile.  But in a doc-
       ument  with  several bibliographies, the bibliography files will have a
       variety of names.  Since bibtex is invoked with  the  basename  of  the
       bibliography  file, the setting for the bibtex command should therefore
       be

            $bibtex = 'bibtex %O %B';

       Generally, you should use %B rather than %R.  Similarly for  most  pur-
       poses, the name %T of the primary texfile is not a useful placeholder.

       See  the default values in the section "List of configuration variables
       usable in initialization files" for what is normally the most appropri-
       ate usage.

       If you omit to supply any placeholders whatever in the specification of
       a command, latexmk will supply what its author thinks  are  appropriate
       defaults.  This gives compatibility with configuration files for previ-
       ous versions of latexmk, which didn't use placeholders.

       "Detaching" a command: Normally when latexmk runs a command,  it  waits
       for the command to run to completion.  This is appropriate for commands
       like latex, of course.  But for previewers, the command should normally
       run  detached,  so that latexmk gets the previewer running and then re-
       turns to its next task (or exits if there is nothing else to  do).   To
       achieve  this  effect  of  detaching a command, you need to precede the
       command name with "start ", as in

            $dvi_previewer = 'start xdvi %O %S';

       This will be translated to whatever is appropriate for  your  operating
       system.

       Notes: (1) In some circumstances, latexmk will always run a command de-
       tached.  This is the case for a previewer in preview  continuous  mode,
       since  otherwise previewing continuously makes no sense.  (2) This pre-
       cludes the possibility of running a command named start.   (3)  If  the
       word  start  occurs  more  than  once  at  the beginning of the command
       string, that is equivalent to having just one.  (4) Under cygwin,  some
       complications  happen, since cygwin amounts to a complicated merging of
       UNIX and MS-Windows.  See the source code  for  how  I've  handled  the
       problem.



                                 17 April 2020                              30








       Command names containing spaces: Under MS-Windows it is common that the
       name of a command includes spaces, since software is often installed in
       a subdirectory of "C:\Program Files".  Such command names should be en-
       closed in double quotes, as in

            $lpr_pdf  =  '"c:/Program  Files/Ghostgum/gsview/gsview32.exe"  /p
       %S';
            $pdf_previewer   =   'start   "c:/Program   Files/SumatraPDF/Suma-
       traPDF.exe" %O %S';
            $pdf_previewer = 'start "c:/Program  Files/SumatraPDF  (x86)/Suma-
       traPDF.exe" %O %S';


       (Note  about  the  above  example: Under MS-Windows forward slashes are
       equivalent to backslashes in a filename under almost all circumstances,
       provided  that  the  filename is inside double quotes.  It is easier to
       use forward slashes in examples like the one above, since then one does
       not  have  to worry about the rules for dealing with forward slashes in
       strings in the Perl language.)

       Command names under Cygwin: If latexmk is executed  by  Cygwin's  Perl,
       be particularly certain that pathnames in commands have forward slashes
       not the usual backslashes for the  separator  of  pathname  components.
       See  the  above  examples.  Backslashes often get misinterpreted by the
       Unix shell used by Cygwin's Perl to execute external commands.  Forward
       slashes don't suffer from this problem, and (when quoted, as above) are
       equally acceptable to MS-Windows.

       Using MS-Windows file associations: A useful trick  under  modern  ver-
       sions of MS-Windows (e.g., WinXP) is to use just the command 'start' by
       itself:

            $dvi_previewer = 'start %S';

       Under MS-Windows, this will cause to be run whatever program the system
       has  associated  with  dvi  files.   (The same applies for a postscript
       viewer and a pdf viewer.)  But note that this trick is not always suit-
       able  for the pdf previwer, if your system has acroread for the default
       pdf viewer.  As explained elsewhere, acroread under MS-Windows does not
       work well with latex and latexmk, because acroread locks the pdf file.

       Not using a certain command: If a command is not to be run, the command
       name NONE is used, as in

            $lpr  = 'NONE lpr';

       This typically is used when an appropriate command does  not  exist  on
       your system.  The string after the "NONE" is effectively a comment.

       Options to commands: Setting the name of a command can be used not only
       for changing the name of the command called, but also to add options to
       command.   Suppose  you  want latexmk to use latex with source specials
       enabled.  Then you might use the following line  in  an  initialization
       file:



                                 17 April 2020                              31








            $latex = 'latex --src-specials %O %S';

       Running  a  subroutine instead of an external command: Use a specifica-
       tion starting with "internal", as in

            $latex = 'internal mylatex %O %S';
            sub mylatex {
                my @args = @_;
                # Possible preprocessing here
                return system 'latex', @args;
            }

       For some of the more exotic possibilities that then  become  available,
       see  the  section "ADVANCED CONFIGURATION: Some extra resources and ad-
       vanced tricks". Also see some of the examples in  the  directory  exam-
       ple_rcfiles in the latexmk distribution.

       Advanced  tricks:  Normally one specifies a single command for the com-
       mands invoked by latexmk.  Naturally, if there is some complicated  ad-
       ditional  processing  you need to do in your special situation, you can
       write a script (or batch file) to do the processing, and then configure
       latexmk to use your script in place of the standard program.

       You  can  also  use a Perl subroutine instead of a script -- see above.
       This is generally the most flexible and portable solution.

       It is also possible to configure latexmk to run multiple commands.  For
       example,  if  when  running  pdflatex to generate a pdf file from a tex
       file you need to run another program after pdflatex to perform some ex-
       tra processing, you could do something like:

            $pdflatex  =  'pdflatex  --shell-escape %O %S; pst2pdf_for_latexmk
       %B';

       This definition assumes you are using a  UNIX-like  system  (which  in-
       cludes  Linux  and  OS-X), so that the two commands to be run are sepa-
       rated by the semicolon in the middle of the string.

       If you are using MS-Windows, you would replace the above line by

          $pdflatex = 'cmd /c pdflatex --shell-escape %O %S'
                      . '&& pst2pdf_for_latexmk %B';

       Here, the UNIX command separator ; is replaced  by  &&.   In  addition,
       there is a problem that some versions of Perl on MS-Windows do not obey
       the command separator; this problem is overcome by explicitly  invoking
       the MS-Windows command-line processor cmd.exe.


LIST OF CONFIGURATION VARIABLES USABLE IN INITIALIZATION FILES
       In  this  section  are  specified the variables whose values can be ad-
       justed to configure latexmk.   (See  the  earlier  section  "Configura-
       tion/Initialization  (rc) Files" for the files where the configurations
       are done.)



                                 17 April 2020                              32








       Default values are indicated in brackets.  Note that for variables that
       are  boolean  in character, concerning whether latexmk does or does not
       behave in a certain way, a non-zero value, normally 1, indicates  true,
       i.e.,  the behavior occurs, while a zero value indicates a false value,
       i.e., the behavior does not occur.


       $allow_switch [1]

              This controls what happens when the output extension  of  latex,
              pdflatex,  lualatex  or  xelatex  differs from what is expected.
              (The possible extensions are .dvi, .pdf, .xdv.)  This can happen
              with  the use of the \pdfoutput macro in a document compiled un-
              der latex or pdflatex, or with the use of the \outputmode  macro
              under lualatex.  It can also happen with certain kinds of incor-
              rect configuration.

              In such a case, latexmk can appropriately adjust its network  of
              rules.  The adjustment is made if $allow_switch is on, and if no
              request for a dvi or ps file has been made.

              See the section ALLOWING FOR CHANGE OF OUTPUT EXTENSION.


       $always_view_file_via_temporary [0]
              Whether .ps and .pdf files are initially to be made in a  tempo-
              rary  directory and then moved to the final location.  (This ap-
              plies to dvips, dvipdf, and ps2pdf operations, and the filtering
              operators on .dvi and .ps files.  It does not apply to pdflatex,
              unfortunately, since pdflatex provides no way  of  specifying  a
              chosen name for the output file.)

              This use of a temporary file solves a problem that the making of
              these files can occupy a substantial time.  If a viewer (notably
              gv) sees that the file has changed, it may read the new file be-
              fore the program writing the file has not yet finished its work,
              which  can cause havoc.

              See the $pvc_view_file_via_temporary variable for a setting that
              applies only if preview-continuous mode (-pvc option)  is  used.
              See $tmpdir for the setting of the directory where the temporary
              file is created.


       $analyze_input_log_always [1]

              After a run of latex (etc), always analyze .log for input  files
              in  the  <...>  and (...) constructions.  Otherwise, only do the
              analysis when fls file doesn't exist or is out of date.

              Under normal circumstances, the data in the fls  file  is  reli-
              able, and the test of the log file gets lots of false positives;
              usually $analyze_input_log_always is best set to zero.  But  the
              test  of  the  log  file  is  needed  at  least in the following



                                 17 April 2020                              33








              situation: When a user needs to persuade latexmk that a  certain
              file  is  a  source file, and latexmk doesn't otherwise find it.
              Then the user can write code that causes a line with (...) to be
              written  to log file.  One important case is for lualatex, which
              doesn't always generate lines in the .fls  file  for  input  lua
              files.   (The  situation  with lualatex is HIGHLY version depen-
              dent, e.g., there was a big  change  between  TeXLive  2016  and
              TeXLive 2017.)

              To  keep  backward compatibility with older versions of latexmk,
              the default is to set $analyze_input_log_always to 1.


       $auto_rc_use [1]
              Whether to automatically read the standard  initialization  (rc)
              files, which are the system RC file, the user's RC file, and the
              RC file in the current directory.  The command line option -norc
              can  be  used to turn this setting off.  Each RC file could also
              turn this setting off, i.e., it could set $auto_rc_use  to  zero
              to prevent automatic reading of the later RC files.

              This  variable does not affect the reading of RC files specified
              on the command line by the -r option.

       $aux_dir [""]
              The directory in which auxiliary files (aux, log, etc) are to be
              written  by  a  run of (pdf)latex.  If this variable is not set,
              but $out_dir is set, then $aux_dir is set to $out_dir, which  is
              the directory to which general output files are to be written.

              Important  note:   The  effect  of  $aux_dir,  if different from
              $out_dir, is achieved by giving (pdf)latex  the  -aux-directory.
              Currently  (Dec.  2011  and later) this only works on the MiKTeX
              version of (pdf)latex.

              See also the documentation of $out_dir for some complications on
              what directory names are suitable.

              If  you also use the -cd option, and $out_dir (or $aux_dir) con-
              tains a relative path, then the path is interpreted relative  to
              the document directory.

       $banner [0]
              If  nonzero, the banner message is printed across each page when
              converting the dvi file to postscript.   Without  modifying  the
              variable  $banner_message,  this is equivalent to specifying the
              -d option.

              Note that if $banner is nonzero, the $postscript_mode is assumed
              and the postscript file is always generated, even if it is newer
              than the dvi file.

       $banner_intensity [0.95]
              Equivalent to the -bi option, this is a decimal number between 0



                                 17 April 2020                              34








              and  1 that specifies how dark to print the banner message. 0 is
              black, 1 is white.  The default is just right if your toner car-
              tridge isn't running too low.

       $banner_message ["DRAFT"]
              The banner message to print across each page when converting the
              dvi file to postscript.  This is equivalent to the -bm option.

       $banner_scale [220.0]
              A decimal number that specifies how  large  the  banner  message
              will  be printed.  Experimentation is necessary to get the right
              scale for your message, as a rule of thumb the scale  should  be
              about  equal  to 1100 divided by the number of characters in the
              message.  The Default is just right for  5  character  messages.
              This is equivalent to the -bs option.

       @BIBINPUTS
              This  is  an array variable, now mostly obsolete, that specifies
              directories where latexmk should look for .bib  files.   By  de-
              fault  it  is set from the BIBINPUTS environment variable of the
              operating system.  If that environment variable is  not  set,  a
              single  element list consisting of the current directory is set.
              The format of the directory names depends on your operating sys-
              tem, of course.  Examples for setting this variable are:

                      @BIBINPUTS = ( ".", "C:\\bibfiles" );
                      @BIBINPUTS = ( ".", "\\server\bibfiles" );
                      @BIBINPUTS = ( ".", "C:/bibfiles" );
                      @BIBINPUTS = ( ".", "//server/bibfiles" );
                      @BIBINPUTS = ( ".", "/usr/local/texmf/bibtex/bib" );

              Note  that  under  MS  Windows,  either a forward slash "/" or a
              backward slash "\" can be used to separate pathname  components,
              so  the  first  two  and the second two examples are equivalent.
              Each backward slash should be doubled to avoid running afoul  of
              Perl's rules for writing strings.

              Important note: This variable is now mostly obsolete in the cur-
              rent version of latexmk, since it has a better method of search-
              ing  for  files  using  the kpsewhich command.  However, if your
              system is an unusual one without the kpsewhich command, you  may
              need to set the variable @BIBINPUTS.

       $biber ["biber %O %S"]
              The biber processing program.

       $biber_silent_switch ["--onlylog"]
              Switch(es)  for the biber processing program when silent mode is
              on.

       $bibtex ["bibtex %O %S"]
              The BibTeX processing program.





                                 17 April 2020                              35








       $bibtex_fudge [1]
              When using bibtex, whether to take special action to allow  bib-
              tex  to  work  when  $out_dir  or $aux_dir is specified.  In May
              2018, there was planned an update to bibtex to  correct  a  bug,
              after  that  update,  the  special  action will no longer be re-
              quired.

       $bibtex_silent_switch ["-terse"]
              Switch(es) for the BibTeX processing program when silent mode is
              on.

       $bibtex_use [1]
              Under what conditions to run bibtex or biber.  When latexmk dis-
              covers from the log file that one (or more)  bibtex/biber-gener-
              ated  bibliographies  are used, it can run bibtex or biber when-
              ever it appears necessary to regenerate  the  bbl  file(s)  from
              their  source  bib  database  file(s).   But  sometimes, the bib
              file(s) are not available (e.g., for a document obtained from an
              external archive), but the bbl files are provided.  In that case
              use of bibtex or biber will result in incorrect  overwriting  of
              the  precious  bbl  files.   The  variable  $bibtex_use controls
              whether this happens, and also  controls  whether  or  not  .bbl
              files are deleted in a cleanup operation.

              The possible values of $bibtex_use are:
                0:  never  use  BibTeX  or biber; never delete .bbl files in a
              cleanup.
                1: only use bibtex or biber if  the  bib  files  exist;  never
              delete .bbl files in a cleanup.
                1.5:  only  use bibtex or biber if the bib files exist; condi-
              tionally delete .bbl files in a cleanup (i.e., delete them  only
              when the bib files all exist).
                2: run bibtex or biber whenever it appears necessary to update
              the bbl files, without testing for  the  existence  of  the  bib
              files; always delete .bbl files in a cleanup.

              Note  that  the value 1.5 does not work properly if the document
              uses biber instead of bibtex.  (There's a long story why not.)


       $cleanup_includes_cusdep_generated [0]
              If nonzero, specifies that cleanup also deletes files  that  are
              generated by custom dependencies.  (When doing a clean up, e.g.,
              by use of the -C option, custom dependencies are those listed in
              the .fdb_latexmk file from a previous run.)

       $cleanup_includes_generated [0]
              If  nonzero,  specifies that cleanup also deletes files that are
              detected in the fls file (or failing that, in log file) as being
              generated.   It  will  also  include files made from these first
              generation generated files.

              This operation is somewhat dangerous, and  can  have  unintended
              consequences,  since the files to be deleted are determined from



                                 17 April 2020                              36








              a file created by (pdf)latex, which can contain erroneous infor-
              mation.  Therefore  this  variable is turned off by default, and
              then files to be deleted are restricted to those explictly spec-
              ified   by  patterns  configured  in  the  variables  clean_ext,
              clean_full_ext, and @generated_exts, together  with  those  very
              standard  cases  that  are  hardwired  into  latexmk (e.g., .log
              files).

       $cleanup_mode [0]
              If nonzero, specifies cleanup mode: 1 for full  cleanup,  2  for
              cleanup  except  for .dvi, .ps and .pdf files, 3 for cleanup ex-
              cept for dep and aux files.  (There is also  extra  cleaning  as
              specified by the $clean_ext, $clean_full_ext and @generated_exts
              variables.)

              This variable is equivalent to specifying one of the  -c  or  -C
              options.   But there should be no need to set this variable from
              an RC file.

       $clean_ext [""]
              Extra extensions of files for latexmk to remove when any of  the
              clean-up  options  (-c  or  -C)  is selected.  The value of this
              variable is a string containing the extensions separated by spa-
              ces.

              It is also possible to specify a more general pattern of file to
              be deleted, by using the place holder %R, as in commands, and it
              is also possible to use wildcards.  Thus setting

                  $clean_ext  =  "out  %R-blx.bib  %R-figures*.log  pythontex-
              files-%R/*";

              in an initialization file will imply that when a clean-up opera-
              tion  is  specified,  not  only  is  the  standard  set of files
              deleted, but also files of the form  FOO.out,  FOO-blx.bib,  %R-
              figures*.log,  and  pythontex-files-FOO/*,  where FOO stands for
              the basename of the file being processed (as in FOO.tex).

              The files to be deleted are relative to the directory  specified
              by  $aux_dir.   (Note  that if $out_dir but not $aux_dir is set,
              then in its  initialization,  latexmk  sets  $aux_dir  equal  to
              $out_dir.   A normal situation is therefore that $aux_dir equals
              $out_dir, which is the only case supported  by  TeXLive,  unlike
              MiKTeX.)

              The  filenames  specfied  for a clean-up operation can refer not
              only to regular files but also to directories.  Directories  are
              only deleted if they are empty.  An example of an application is
              to pythontex, which creates files  in  a  particular  directory.
              You  can  arrange  to remove both the files and the directory by
              setting

                  $clean_ext = "pythontex-files-%R pythontex-files-%R";




                                 17 April 2020                              37








              See also the variable @generated_exts.

       $clean_full_ext [""]
              Extra extensions of files for latexmk to remove when the -C  op-
              tion  is  selected, i.e., extensions of files to remove when the
              .dvi, etc files are to be cleaned-up.

              More general patterns are allowed, as for $clean_ext.

              The files specified by $clean_full_ext to be deleted  are  rela-
              tive to the directory specified by $out_dir.


       $compiling_cmd [""], $failure_cmd [""], $warning_cmd [""], $success_cmd
       [""]

              These variables specify commands that are  executed  at  certain
              points  of  compilations.  One motivation for their existance is
              to allow very useful convenient visual indications  of  compila-
              tion  status even when the window receiving the screen output of
              the compilation is hidden.  This is particularly useful in  pre-
              view-continuous mode.

              The  commands  are  executed  at  the following points: $compil-
              ing_cmd at the start of compilation, $success_cmd at the end  of
              a  completely successful compilation, $failure_cmd at the end of
              an unsuccessful compilation, $warning_cmd at the of an otherwise
              successful compilation that gives warnings about undefined cita-
              tions or references or about multiply defined references. If any
              of  above  variables  is  undefined or blank (the default situa-
              tion), then the corresponding command is not executed.

              However, when $warning_cmd is not set, then in  the  case  of  a
              compilation  with  warnings  about  references or citations, but
              with no other error, one or  other  of  $success_cmd  or  $fail-
              ure_cmd  is  used  (if  it  is  set) according to the setting of
              $warnings_as_errors.

              An example of a simple setting of these variables is as follows

                  $compiling_cmd = "xdotool search  --name  \"%D\"  set_window
              --name \"%D compiling\"";
                  $success_cmd    =  "xdotool  search --name \"%D\" set_window
              --name \"%D OK\"";
                  $warning_cmd   = "xdotool search --name \"%D\" ".
                                   "set_window --name \"%D CITE/REF ISSUE\"";
                  $failure_cmd   = "xdotool search  --name  \"%D\"  set_window
              --name \"%D FAILURE\"";

              These  assume  that  the  program xdotool is installed, that the
              previewer is using an X-Window system for display, and that  the
              title  of the window contains the name of the displayed file, as
              it normally does.  When the commands are  executed,  the  place-
              holder  string  %D  is  replaced  by the name of the destination



                                 17 April 2020                              38








              file, which is the previewed file.  The above commands result in
              an appropriate string being appended to the filename in the win-
              dow title: " compiling", " OK", or " FAILURE".

              Other placeholders that can be used are %S, %T, and %R, with  %S
              and  %T normally being identical. These can be useful for a com-
              mand changing the title of the edit window. The  visual  indica-
              tion  in a window title can useful, since the user does not have
              to keep shifting attention to the (possibly hidden)  compilation
              window to know the status of the compilation.

              More  complicated  situations  can best be handled by defining a
              Perl subroutine to invoke the necessary commands, and using  the
              "internal"  keyword  in the definitions to get the subroutine to
              be invoked.  (See the section "Format of Command Specifications"
              for how to do this.)

              Naturally,  the  above  settings that invoke the xdotool program
              are only applicable when the X-Window system  is  used  for  the
              relevant window(s).  For other cases, you will have to find what
              software solutions are available.


       @cus_dep_list [()]
              Custom dependency list -- see section on "Custom Dependencies".

       @default_excluded_files [()]
              When latexmk is invoked with no files specified on  the  command
              line, then, by default, it will process all files in the current
              directory with the extension .tex.  (In general, it will process
              the files specified in the @default_files variable.)

              But sometimes you want to exclude particular files from this de-
              fault list.  In that case you can specify the excluded files  in
              the array @default_excluded_files.  For example if you wanted to
              process all .tex files with the exception of  common.tex,  which
              is a not a standard alone LaTeX file but a file input by some or
              all of the others, you could do

                   @default_files = ("*.tex");

                   @default_excluded_files = ("common.tex");

              If you have a variable or large number of files to be processed,
              this method saves you from having to list them in detail in @de-
              fault_files and having to update the list every time you  change
              the set of files to be processed.

              Notes:  1.  This variable has no effect except when no files are
              specified on the latexmk command line.  2. Wildcards are allowed
              in @default_excluded_files.

       @default_files [("*.tex")]
              Default list of files to be processed.



                                 17 April 2020                              39








              If  no filenames are specified on the command line, latexmk pro-
              cesses all tex files specified in the  @default_files  variable,
              which  by  default is set to all tex files ("*.tex") in the cur-
              rent directory.  This is a convenience: just run latexmk and  it
              will  process  an  appropriate  set of files.  But sometimes you
              want only some of these files to be processed.  In this case you
              can  list the files to be processed by setting @default_files in
              an initialization file (e.g., the file "latexmkrc" in  the  cur-
              rent  directory).  Then if no files are specified on the command
              line then the files you specify by  setting  @default_files  are
              processed.

              Three examples:

                   @default_files = ("paper_current");

                   @default_files = ("paper1", "paper2.tex");

                   @default_files = ("*.tex", "*.dtx");

              Note  that more than file may be given, and that the default ex-
              tension is ".tex".  Wild cards are allowed.  The parentheses are
              because @default_files is an array variable, i.e., a sequence of
              filename specifications is possible.

              If you want latexmk to process all .tex files with a few  excep-
              tions, see the @default_excluded_files array variable.

       $dependents_phony [0]
              If  a  list  of dependencies is output, this variable determines
              whether to include a phony target for each source file.  If  you
              use  the  dependents  list  in  a Makefile, the dummy rules work
              around errors make gives if you remove header files without  up-
              dating the Makefile to match.

       $dependents_list [0]
              Whether  to  display  a  list(s) of dependencies at the end of a
              run.

       $deps_file ["-"]
              Name of file to receive list(s) of dependencies at the end of  a
              run,  to be used if $dependesnt_list is set.  If the filename is
              "-", then the dependency list is set to stdout  (i.e.,  normally
              the screen).

       $do_cd [0]
              Whether  to  change working directory to the directory specified
              for the main source file before processing it.  The default  be-
              havior  is  not to do this, which is the same as the behavior of
              latex and pdflatex programs.  This variable is set  by  the  -cd
              and -cd- options on latexmk's command line.

       $dvi_filter [empty]
              The  dvi  file  filter  to be run on the newly produced dvi file



                                 17 April 2020                              40








              before other processing.  Equivalent to specifying the  -dF  op-
              tion.

       $dvi_mode [See below for default]
              If  nonzero, generate a dvi version of the document.  Equivalent
              to the -dvi option.

              The variable $dvi_mode defaults to 0, but  if  no  explicit  re-
              quests  are made for other types of file (postscript, pdf), then
              $dvi_mode will be set to 1.  In addition, if  a  request  for  a
              file  for  which  a  .dvi file is a prerequisite, then $dvi_mode
              will be set to 1.

       $dvi_previewer ["start xdvi %O %S" under UNIX]
              The command to invoke a dvi-previewer.   [Under  MS-Windows  the
              default  is "start"; then latexmk arranges to use the MS-Windows
              start program, which will cause to be run whatever  command  the
              system has associated with .dvi files.]

              Important  note:  Normally you will want to have a previewer run
              detached, so that latexmk doesn't wait for the previewer to ter-
              minate  before continuing its work.  So normally you should pre-
              fix the command by "start ", which  flags  to  latexmk  that  it
              should  do  the  detaching  of the previewer itself (by whatever
              method is appropriate to the operating system).   But  sometimes
              letting latexmk do the detaching is not appropriate (for a vari-
              ety of non-trivial reasons), so you should put the "start "  bit
              in yourself, whenever it is needed.

       $dvi_previewer_landscape ["start xdvi %O %S"]
              The command to invoke a dvi-previewer in landscape mode.  [Under
              MS-Windows the default is "start"; then latexmk arranges to  use
              the  MS-Windows  start program, which will cause to be run what-
              ever command the system has associated with .dvi files.]

       $dvipdf ["dvipdf %O %S %D"]
              Command to convert .dvi to .pdf file.  A common  reconfiguration
              is  to  use  the dvipdfm command, which needs its arguments in a
              different order:

                   $dvipdf = "dvipdfm %O -o %D %S";

              WARNING: The default dvipdf  script  generates  pdf  files  with
              bitmapped fonts, which do not look good when viewed by acroread.
              That script should be modified to give  dvips  the  options  "-P
              pdf" to ensure that type 1 fonts are used in the pdf file.

       $dvipdf_silent_switch ["-q"]
              Switch(es) for dvipdf program when silent mode is on.

              N.B.  The  standard  dvipdf program runs silently, so adding the
              silent switch has no effect, but is actually innocuous.  But  if
              an  alternative program is used, e.g., dvipdfmx, then the silent
              switch has an  effect.   The  default  setting  is  correct  for



                                 17 April 2020                              41








              dvipdfm and dvipdfmx.

       $dvips ["dvips %O -o %D %S"]
              The  program to used as a filter to convert a .dvi file to a .ps
              file.  If pdf is going to be generated from pdf, then the  value
              of  the  $dvips_pdf_switch  variable -- see below -- will be in-
              cluded in the options substituted for "%O".

       $dvips_landscape ["dvips -tlandscape %O -o %D %S"]
              The program to used as a filter to convert a .dvi file to a  .ps
              file in landscape mode.

       $dvips_pdf_switch ["-P pdf"]
              Switch(es)  for  dvips  program when pdf file is to be generated
              from .ps file.

       $dvips_silent_switch ["-q"]
              Switch(es) for dvips program when silent mode is on.

       $dvi_update_command [""]
              When the dvi previewer is set to be updated by  running  a  com-
              mand,  this is the command that is run.  See the information for
              the variable $dvi_update_method for further information, and see
              information  on  the  variable $pdf_update_method for an example
              for the analogous case of a pdf previewer.

       $dvi_update_method [2 under UNIX, 1 under MS-Windows]
              How the dvi viewer updates its display when  the  dvi  file  has
              changed.    The  values  here  apply  equally  to  the  $pdf_up-
              date_method and to the $ps_update_method variables.
                  0 => update is automatic,
                  1=> manual update by user, which may only mean a mouse click
              on the viewer's window or may mean a more serious action.
                  2  =>  Send  the  signal,  whose  number  is in the variable
              $dvi_update_signal.  The default value under  UNIX  is  suitable
              for xdvi.
                  3  => Viewer cannot do an update, because it locks the file.
              (As with acroread under MS-Windows.)
                  4 => run a command to do the update.  The command is  speci-
              fied by the variable $dvi_update_command.

              See  information on the variable $pdf_update_method for an exam-
              ple of updating by command.

       $dvi_update_signal [Under UNIX: SIGUSR1, which  is  a  system-dependent
       value]
              The  number of the signal that is sent to the dvi viewer when it
              is updated by sending a signal -- see  the  information  on  the
              variable  $dvi_update_method.   The default value is the one ap-
              propriate for xdvi on a UNIX system.

       $failure_cmd [undefined]
              See the documentation for $compiling_cmd.




                                 17 April 2020                              42








       $fdb_ext ["fdb_latexmk"]
              The extension of the file which latexmk generates to  contain  a
              database  of information on source files.  You will not normally
              need to change this.

       $filetime_causality_threshold  [5];   $filetime_offset_report_threshold
       [30]. (Units of seconds.)

              These variables control how latexmk deals with the following is-
              sue, which can affect the use of files  that  are  on  a  remote
              filesystem (network share) instead of being on a file system lo-
              cal to the computer running latexmk.  Almost users will not have
              to  worry about these settings, and can ignore the following ex-
              planation.

              In almost all situations, latexmk does not need to use the  time
              stamps  of the files it works with.  However, there are a couple
              of situations when it needs to know whether a certain  file  was
              created in the current run of a program (e.g., (pdf)latex) or is
              a leftover file from a previous run. It does this  by  comparing
              the  modification time of the file with the system time just be-
              fore the program was started. If the modification time  is  ear-
              lier  than  when the program was started, the file is a leftover
              file, which latexmk treats as if it were not  created.   If  the
              filetime  is at least the program start time, then it can be as-
              sumed that the file was created in the current run.

              Unfortunately, this test can fail if the file  is  on  a  remote
              system,  since  its  system time is not necessarily synchronized
              with that of the local system;  the  timestamps  on  the  remote
              files  are set by the remote system, not the local system.  Gen-
              erally, modern operating  systems  regularly  synchronize  their
              time  with  a server, so the non-synchronization is mostly small
              (a second or so, or a few seconds).  But even a small difference
              can mess up latexmk's test.

              Latexmk measures the time difference between the time on the two
              systems and compensates for this.  But  the  measurement  (in  a
              system-independent way) is only accurate to a second or two.  So
              latexmk allows for a threshold on the  difference  between  file
              and  system  time  before it concludes that a file is a leftover
              file from a previous  run.  The  configuration  variable  $file-
              time_causality_theshhold,  which  in units of seconds, specifies
              this threshold.  Luckily high precision is not needed.  The pre-
              vious  run  is normally the previous run in a human run-edit-run
              cycle, and is at least many seconds  back.   A  few  seconds  is
              therefore   appropriate  for  the  threshold,  $filetime_causal-
              ity_theshhold; it should be non-negative always, and  should  be
              bigger than 2 if a remote filesystem or network share is used.

              If  the  difference in system times on the two systems is large,
              it normally indicates that at least one of the systems  is  mis-
              configured.    The   variable  $filetime_offset_report_threshold
              specifies the smallest size of the  difference  (or  offset)  in



                                 17 April 2020                              43








              seconds  between the times of the local and remote system beyond
              which the offset is reported.  This is reported at the point  in
              the  latexmk's  progress that it measures the offset. The report
              is made if silent mode is used and diagnostic mode is not on.


       $force_mode [0]
              If nonzero, continue processing past minor latex errors  includ-
              ing unrecognized cross references.  Equivalent to specifying the
              -f option.

       @generated_exts [( aux , bbl , idx , ind , lof , lot  ,  out  ,  toc  ,
       $fdb_ext )]
              This  contains a list of extensions for files that are generated
              during a LaTeX run and that are read in by LaTeX in later  runs,
              either directly or indirectly.

              This  list  specifies  files known to be generated by latex (and
              pdflatex, etc).  It is used in two ways: (a) The specified files
              are deleted in a cleanup operation (with the -c, -C, -CA, -g and
              -gg options), and (b) It affects the determination of whether  a
              rerun of (pdf)LaTeX is needed after a run that gives an error.

              (Concerning item (b): Normally, a change of a source file during
              a run should provoke a rerun.  This includes a file generated by
              LaTeX,  e.g.,  an  aux file, that is read in on subsequent runs.
              But after a run that results in an error, a new run  should  not
              occur  until  the  user has made a change in the files.  But the
              user may have corrected an error in a source  .tex  file  during
              the run.  So latexmk needs to distinguish user-generated and au-
              tomatically generated files;  it  determines  the  automatically
              generated  files as those with extensions in the list in @gener-
              ated_exts.)

              A convenient way to add an extra extension to the list,  without
              losing  the already defined ones is to use a push command in the
              line in an RC file.  E.g.,

                              push @generated_exts, "end";

              adds the extension "end" to the list of predefined generated ex-
              tensions.   (This  extension  is used by the RevTeX package, for
              example.)

       $go_mode [0]
              If nonzero, process files regardless of timestamps, and is  then
              equivalent to the -g option.

       %hash_calc_ignore_pattern
              !!!This variable is for experts only!!!

              The  general rule latexmk uses for determining when an extra run
              of some program is needed is that one of the  source  files  has
              changed.   But  consider for example a latex package that causes



                                 17 April 2020                              44








              an encapsulated postscript file (an "eps" file) to be made  that
              is  to  be read in on the next run.  The file contains a comment
              line giving its creation date and time.  On  the  next  run  the
              time  changes,  latex  sees  that  the eps file has changed, and
              therefore reruns latex.  This causes an infinite loop,  that  is
              only  terminated  because  latexmk  has a limit on the number of
              runs to guard against pathological situations.

              But the changing line has no real effect, since it is a comment.
              You can instruct latex to ignore the offending line as follows:

                 $hash_calc_ignore_pattern{'eps'} = '^%%CreationDate: ';

              This creates a rule for files with extension .eps about lines to
              ignore.  The left-hand side is a Perl idiom for setting an  item
              in  a hash.  Note that the file extension is specified without a
              period.  The value, on the right-hand side, is a string contain-
              ing  a  regular expresssion.  (See documentation on Perl for how
              they are to be specified in general.)  This  particular  regular
              expression  specifies that lines beginning with "%%CreationDate:
              " are to be ignored in deciding whether a file of the given  ex-
              tension .eps has changed.

              There  is  only one regular expression available for each exten-
              sion.  If you need more one pattern to specify lines to  ignore,
              then  you need to combine the patterns into a single regular ex-
              pression.  The simplest method is separate the different  simple
              patterns  by  a vertical bar character (indicating "alternation"
              in the jargon of regular expressions).  For example,

                 $hash_calc_ignore_pattern{'eps'} = '^%%CreationDate:  |^%%Ti-
              tle: ';

              causes lines starting with either "^%%CreationDate: " or "^%%Ti-
              tle: " to be ignored.

              It may happen that a pattern to be ignored is specified in,  for
              example,  in  a system or user initialization file, and you wish
              to remove this in a file that is read later.  To  do  this,  you
              use Perl's delete function, e.g.,

                  delete $hash_calc_ignore_pattern{'eps'};


       $jobname [""]

              This  specifies the jobname, i.e., the basename that is used for
              generated files (.aux, .log, .dvi, .ps,  .pdf,  etc).   If  this
              variable  is a null string, then the basename is the basename of
              the main tex file.  (At present, the string in  $jobname  should
              not contain spaces.)

              The  placeholder  '%A' is permitted. This will be substituted by
              the basename of the TeX file.  The primary  purpose  is  when  a



                                 17 April 2020                              45








              variety  of tex files are to be processed, and you want to use a
              different jobname for each but one that is  distinct  for  each.
              Thus  if you wanted to compare compilations of a set of files on
              different operating systems, with distinct filenames for all the
              cases, you could set

                 $jobname = "%A-$^O";

              in  an initialization file.  (Here $^O is a variable provided by
              perl that contains perl's name for the operating system.)

              Suppose you had .tex files test1.tex and test2.tex.   Then  when
              you run

                 latexmk -pdf *.tex

              both  files  will  be  compiled.  The .aux, .log, and .pdf files
              will have basenames test1-MSWin32 ante test2-MSWin32  on  a  MS-
              Windows system, test1-darwin and test2-darwin on an OS-X system,
              and a variety of similar cases on linux systems.


       $kpsewhich ["kpsewhich %S"]
              The program called to locate a source file when the  name  alone
              is  not  sufficient.  Most filenames used by latexmk have suffi-
              cient path information to be found directly.  But sometimes, no-
              tably when a .bib or a .bst file is found from the log file of a
              bibtex or biber run, only the base name of the  file  is  known,
              but not its path. The program specified by $kpsewhich is used to
              find it.

              (For advanced users: Because of the different way in  which  la-
              texmk uses the command specified in $kpsewhich, some of the pos-
              sibilities listed in the FORMAT OF COMMAND SPECIFICATIONS do not
              apply.   The  internal  and  start keywords are not available. A
              simple command specification with possible options and then "%S"
              is  all  that  is  guaranteed to work.  Note that for other com-
              mands, "%S" is substituted by a single source file. In contrast,
              for $kpsewhich, "%S" may be substituted by a long list of space-
              separated filenames, each of which is  quoted.   The  result  on
              STDOUT of running the command is then piped to latexmk.)

              See  also  the  @BIBINPUTS variable for another way that latexmk
              also uses to try to locate files; it applies only in the case of
              .bib files.

       $kpsewhich_show [0]
              Whether  to show diagnostics about invocations of kpsewhich: the
              command line use to invoke it and the results.   These  diagnos-
              tics  are shown if $kpsewhich_show is non-zero or if diagnostics
              mode is on.  (But in the second case, lots of other  diagnostics
              are  also  shown.)   Without  these diagnostics there is nothing
              visible in latexmk's screen output about  invocations  of  kpse-
              which.



                                 17 April 2020                              46








       $landscape_mode [0]
              If nonzero, run in landscape mode, using the landscape mode pre-
              viewers and dvi to postscript converters.  Equivalent to the  -l
              option.  Normally not needed with current previewers.

       $latex ["latex %O %S"]
              Specifies  the  command  line  for the LaTeX processing program.
              Note that as with other programs, you can use this variable  not
              just  to  change  the name of the program used, but also specify
              options to the program.  E.g.,

                                  $latex = "latex --src-specials %O %S";

              To do a coordinated setting of all of $latex, $pdflatex, $luala-
              tex, and $xelatex, see the section "Advanced Configuration".


       %latex_input_extensions
              This  variable specifies the extensions tried by latexmk when it
              finds that a LaTeX run resulted in an error that a file has  not
              been  found,  and  the file is given without an extension.  This
              typically happens when LaTeX commands of the  form  \input{file}
              or  \includegraphics{figure}, when the relevant source file does
              not exist.

              In this situation, latexmk searches for custom  dependencies  to
              make  the  missing  file(s),  but restricts it to the extensions
              specified by the variable %latex_input_extensions.  The  default
              extensions are 'tex' and 'eps'.

              (For  Perl experts: %latex_input_extensions is a hash whose keys
              are the extensions.  The values are  irrelevant.)   Two  subrou-
              tines  are  provided for manipulating this and the related vari-
              able %pdflatex_input_extensions,  add_input_ext  and  remove_in-
              put_ext.   They are used as in the following examples are possi-
              ble lines in an initialization file:

                  remove_input_ext( 'latex', 'tex' );

              removes the extension 'tex' from latex_input_extensions

                  add_input_ext( 'latex', 'asdf' );

              add the extension 'asdf to  latex_input_extensions.   (Naturally
              with such an extension, you should have made an appropriate cus-
              tom dependency for latexmk, and should also have done the appro-
              priate  programming  in the LaTeX source file to enable the file
              to be read.  The standard extensions are handled  by  LaTeX  and
              its graphics/graphicx packages.)

       $latex_silent_switch ["-interaction=batchmode"]
              Switch(es)  for the LaTeX processing program when silent mode is
              on.




                                 17 April 2020                              47








              If you use MikTeX, you may prefer the results if  you  configure
              the  options  to include -c-style-errors, e.g., by the following
              line in an initialization file

                $latex_silent_switch  =  "-interaction=batchmode  -c-style-er-
              rors";


       $lpr ["lpr %O %S" under UNIX/Linux, "NONE lpr" under MS-Windows]
              The command to print postscript files.

              Under  MS-Windows (unlike UNIX/Linux), there is no standard pro-
              gram for printing files.  But there are ways you can do it.  For
              example, if you have gsview installed, you could use it with the
              option "/p":

                  $lpr = '"c:/Program Files/Ghostgum/gsview/gsview32.exe" /p';

              If gsview is installed in a different directory, you  will  need
              to  make the appropriate change.  Note the combination of single
              and double quotes around the name.  The  single  quotes  specify
              that  this is a string to be assigned to the configuration vari-
              able $lpr.  The double quotes are part of the string  passed  to
              the  operating  system to get the command obeyed; this is neces-
              sary because one part of the command name ("Program Files") con-
              tains a space which would otherwise be misinterpreted.

       $lpr_dvi ["NONE lpr_dvi"]
              The printing program to print dvi files.

       $lpr_pdf ["NONE lpr_pdf"]
              The printing program to print pdf files.

              Under  MS-Windows you could set this to use gsview, if it is in-
              stalled, e.g.,

                  $lpr = '"c:/Program Files/Ghostgum/gsview/gsview32.exe" /p';

              If gsview is installed in a different directory, you  will  need
              to  make  the appropriate change.  Note the double quotes around
              the name: this is necessary because one part of the command name
              ("Program Files") contains a space which would otherwise be mis-
              interpreted.

       $lualatex ["lualatex %O %S"]
              Specifies the command line for the LaTeX processing program that
              is  to be used when the lualatex program is called for (e.g., by
              the option -lualatex.

              To do a coordinated setting of all of $latex, $pdflatex, $luala-
              tex, and $xelatex, see the section "Advanced Configuration".






                                 17 April 2020                              48








       %lualatex_input_extensions
              This  variable specifies the extensions tried by latexmk when it
              finds that a lualatex run resulted in an error that a  file  has
              not  been  found,  and  the  file is given without an extension.
              This typically happens when LaTeX  commands  of  the  form  \in-
              put{file}  or \includegraphics{figure}, when the relevant source
              file does not exist.

              In this situation, latexmk searches for custom  dependencies  to
              make  the  missing  file(s),  but restricts it to the extensions
              specified by the variable %pdflatex_input_extensions.   The  de-
              fault extensions are 'tex', 'pdf', 'jpg, and 'png'.

              See details of the %latex_input_extensions for other information
              that equally applies to %lualatex_input_extensions.

       $lualatex_silent_switch ["-interaction=batchmode"]
              Switch(es) for the lualatex program (specified in  the  variable
              $lualatex) when silent mode is on.

              See  details  of  the $latex_silent_switch for other information
              that equally applies to $lualatex_silent_switch.

       $make ["make"]
              The make processing program.

       $makeindex ["makeindex %O -o %D %S"]
              The index processing program.

       $makeindex_silent_switch ["-q"]
              Switch(es) for the index processing program when silent mode  is
              on.

       $max_repeat [5]
              The  maximum number of times latexmk will run latex/pdflatex be-
              fore deciding that there may be an infinite  loop  and  that  it
              needs to bail out, rather than rerunning latex/pdflatex again to
              resolve cross-references, etc.  The  default  value  covers  all
              normal cases.

              (Note  that the "etc" covers a lot of cases where one run of la-
              tex/pdflatex generates files to be read in on a later run.)

       $MSWin_back_slash [1]
              This configuration variable only has an effect when  latexmk  is
              running  under MS-Windows.  With the default value of 1 for this
              variable, when a command is executed under  MS-Windows,  latexmk
              substitutes  "\"  for the separator character between components
              of a directory name.  Internally, latexmk uses "/" for  the  di-
              rectory  separator  character,  which  is  the character used by
              Unix-like systems.

              For almost all programs and for almost all filenames  under  MS-
              Windows,  both  "\"  and  "/"  are  acceptable  as the directory



                                 17 April 2020                              49








              separator character, provided at least that filenames are  prop-
              erly  quoted.   But it is possible that programs exist that only
              accept "\" on the command line, since that is the  standard  di-
              rectory  separator  for MS-Windows.  So for safety latexmk makes
              the substitution from "/" to "\", by default.

              However there are also programs on MS-Windows for which  a  back
              slash "\" is interpreted differently than as a directory separa-
              tor; for these the directory separator should be "/".   Programs
              with  this  behavior  include  all  the  *latex  programs in the
              TeXLive implementation  (but  not  the  MiKTeX  implementation).
              Hence  if  you use TeXLive on MS-Windows, then $MSWin_back_slash
              should be set to zero.


       $new_viewer_always [0]
              This variable applies  to  latexmk  only  in  continuous-preview
              mode.  If $new_viewer_always is 0, latexmk will check for a pre-
              viously running previewer on the same file, and if one  is  run-
              ning  will  not  start a new one.  If $new_viewer_always is non-
              zero, this check will be skipped, and latexmk will behave as  if
              no viewer is running.

       $out_dir [""]
              If  non-blank,  this  variable  specifies the directory in which
              output files are to be written by a run of (pdf)latex.  See also
              the variable $aux_dir.

              The  effect of this variable (when non-blank) is achieved by us-
              ing the -output-directory option of (pdf)latex.  This exists  in
              the  usual current (Dec. 2011 and later) implementations of TeX,
              i.e., MiKTeX and TeXLive. But it may not  be  present  in  other
              versions.

              If  you also use the -cd option, and $out_dir (or $aux_dir) con-
              tains a relative path, then the path is interpreted relative  to
              the document directory.

              Commonly,  the  directory specified for output files is a subdi-
              rectory of the current working directory.  However, if you spec-
              ify  some  other  directory, e.g., "/tmp/foo" or "../output", be
              aware that this could cause problems, e.g.,  with  makeindex  or
              bibtex.   This  is because modern versions of these programs, by
              default, will refuse to work when they find that they are  asked
              to  write  to  a  file in a directory that appears not to be the
              current working directory or one of its subdirectories.  This is
              part  of  security  measures by the whole TeX system that try to
              prevent malicious or errant TeX documents from incorrectly mess-
              ing with a user's files.  If for $out_dir or $aux_dir you really
              do need to specify an absolute pathname (e.g., "/tmp/foo") or  a
              path (e.g., "../output") that includes a higher-level directory,
              and you need to use makeindex or bibtex, then you need  to  dis-
              able  the  security measures (and assume any risks).  One way of
              doing this is to temporarily set an operating system environment



                                 17 April 2020                              50








              variable  openout_any  to "a" (as in "all"), to override the de-
              fault "paranoid" setting.

       $pdf_mode [0]
              If zero, do NOT generate a pdf  version  of  the  document.   If
              equal  to 1, generate a pdf version of the document using pdfla-
              tex, using the command specified by the $pdflatex variable.   If
              equal  to  2, generate a pdf version of the document from the ps
              file, by using the command specified by  the  $ps2pdf  variable.
              If  equal  to 3, generate a pdf version of the document from the
              dvi file, by using the command specified by  the  $dvipdf  vari-
              able.  If equal to 4, generate a pdf version of the document us-
              ing lualatex, using the command specified by the $lualatex vari-
              able.   If  equal  to 5, generate a pdf version (and an xdv ver-
              sion) of the document using xelatex, using the  commands  speci-
              fied by the $xelatex and xdvipdfmx variables.

              In  $pdf_mode=2,  it is ensured that .dvi and .ps files are also
              made.  In $pdf_mode=3, it is ensured that a .dvi  file  is  also
              made.  But this may be overridden by the document.

       $pdflatex ["pdflatex %O %S"]
              Specifies the command line for the LaTeX processing program in a
              version that makes a pdf file instead of a dvi file.

              An example use of this variable is to add certain options to the
              command line for the program, e.g.,

                   $pdflatex = "pdflatex --shell-escape %O %S";

              (In  some  earlier versions of latexmk, you needed to use an as-
              signment to $pdflatex to allow the use of  lualatex  or  xelatex
              instead of pdflatex.  There are now separate configuration vari-
              ables for the use of lualatex or  xelatex.   See  $lualatex  and
              $xelatex.)

              To do a coordinated setting of all of $latex, $pdflatex, $luala-
              tex, and $xelatex, see the section "Advanced Configuration".

       %pdflatex_input_extensions
              This variable specifies the extensions tried by latexmk when  it
              finds  that  a pdflatex run resulted in an error that a file has
              not been found, and the file  is  given  without  an  extension.
              This  typically  happens  when  LaTeX  commands of the form \in-
              put{file} or \includegraphics{figure}, when the relevant  source
              file does not exist.

              In  this  situation, latexmk searches for custom dependencies to
              make the missing file(s), but restricts  it  to  the  extensions
              specified  by  the variable %pdflatex_input_extensions.  The de-
              fault extensions are 'tex', 'pdf', 'jpg, and 'png'.

              See details of the %latex_input_extensions for other information
              that equally applies to %pdflatex_input_extensions.



                                 17 April 2020                              51








       $pdflatex_silent_switch ["-interaction=batchmode"]
              Switch(es)  for  the pdflatex program (specified in the variable
              $pdflatex) when silent mode is on.

              See details of the $latex_silent_switch  for  other  information
              that equally applies to $pdflatex_silent_switch.

       $pdf_previewer ["start acroread %O %S"]
              The command to invoke a pdf-previewer.

              On  MS-Windows, the default is changed to "cmd /c start """; un-
              der more recent versions of Windows, this will cause to  be  run
              whatever command the system has associated with .pdf files.  But
              this may be undesirable if this association is  to  acroread  --
              see the notes in the explanation of the -pvc option.]

              On  OS-X  the  default is changed to "open %S", which results in
              OS-X starting up (and detaching) the viewer associated with  the
              file.   By  default, for pdf files this association is to OS-X's
              preview, which is quite satisfactory.

              WARNING:  Problem under MS-Windows: if acroread is used  as  the
              pdf  previewer,  and  it is actually viewing a pdf file, the pdf
              file cannot be updated.  Thus makes acroread  a  bad  choice  of
              previewer  if you use latexmk's previous-continuous mode (option
              -pvc) under MS-windows.  This problem does not occur if, for ex-
              ample, SumatraPDF or gsview is used to view pdf files.

              Important  note:  Normally you will want to have a previewer run
              detached, so that latexmk doesn't wait for the previewer to ter-
              minate  before continuing its work.  So normally you should pre-
              fix the command by "start ", which  flags  to  latexmk  that  it
              should  do  the  detaching  of the previewer itself (by whatever
              method is appropriate to the operating system).   But  sometimes
              letting latexmk do the detaching is not appropriate (for a vari-
              ety of non-trivial reasons), so you should put the "start "  bit
              in yourself, whenever it is needed.

       $pdf_update_command [""]
              When  the  pdf  previewer is set to be updated by running a com-
              mand, this is the command that is run.  See the information  for
              the variable $pdf_update_method.

       $pdf_update_method [1 under UNIX, 3 under MS-Windows]
              How  the  pdf  viewer  updates its display when the pdf file has
              changed. See the information on the variable  $dvi_update_method
              for the codes.  (Note that information needs be changed slightly
              so that for the value 4, to run a command to do the update,  the
              command  is  specified  by the variable $pdf_update_command, and
              for the value 2, to specify update  by  signal,  the  signal  is
              specified by $pdf_update_signal.)

              Note that acroread under MS-Windows (but not UNIX) locks the pdf
              file, so the default value is then 3.



                                 17 April 2020                              52








              Arranging to use a command to get a previewer explicitly updated
              requires three variables to be set.  For example:

                  $pdf_previewer = "start xpdf -remote %R %O %S";
                  $pdf_update_method = 4;
                  $pdf_update_command = "xpdf -remote %R -reload";

              The  first  setting  arranges for the xpdf program to be used in
              its "remote server mode", with the server name specified as  the
              rootname  of  the TeX file.  The second setting arranges for up-
              dating to be done in response to a command, and the  third  set-
              ting sets the update command.

       $pdf_update_signal  [Under  UNIX:  SIGHUP,  which is a system-dependent
       value]
              The number of the signal that is sent to the pdf viewer when  it
              is  updated  by  sending  a signal -- see the information on the
              variable $pdf_update_method.  The default value is the  one  ap-
              propriate for gv on a UNIX system.

       $pid_position[1 under UNIX, -1 under MS-Windows]
              The  variable  $pid_position  is  used  to specify which word in
              lines of the output from $pscmd corresponds to the  process  ID.
              The  first word in the line is numbered 0.  The default value of
              1 (2nd word in line) is correct for Solaris 2.6, Linux, and OS-X
              with their default settings of $pscmd.

              Setting  the  variable  to -1 is used to indicate that $pscmd is
              not to be used.

       $postscript_mode [0]
              If nonzero, generate  a  postscript  version  of  the  document.
              Equivalent to the -ps option.

              If  some  other  request  is made for which a postscript file is
              needed, then $postscript_mode will be set to 1.

       $pre_tex_code ['']

              Sets TeX code to be executed before inputting the  source  file.
              This  works  if the relevant one of $latex, etc contains a suit-
              able command line with a %P or %U substitution.  For example you
              could do

                   $latex = 'latex %O %P';
                   $pre_tex_code = '\AtBeginDocument{An initial message\par}';

              To  set  all  of  $latex, $pdflatex, $lualatex, and $xelatex you
              could use the subroutine alt_tex_cmds:

                   &alt_tex_cmds;
                   $pre_tex_code = '\AtBeginDocument{An initial message\par}';





                                 17 April 2020                              53








       $preview_continuous_mode [0]
              If nonzero, run a previewer to view the document,  and  continue
              running latexmk to keep .dvi up-to-date.  Equivalent to the -pvc
              option.  Which previewer is run depends on the  other  settings,
              see the command line options -view=, and the variable $view.

       $preview_mode [0]
              If nonzero, run a previewer to preview the document.  Equivalent
              to the -pv option.  Which previewer is run depends on the  other
              settings,  see the command line options -view=, and the variable
              $view.

       $printout_mode [0]
              If nonzero, print the document using the  command  specified  in
              the $lpr variable.  Equivalent to the -p option.  This is recom-
              mended not to be set from an RC file, otherwise you could  waste
              lots of paper.

       $print_type = ["auto"]
              Type  of  file  to  printout:  possibilities  are "auto", "dvi",
              "none", "pdf", or "ps".   See the option -print= for the meaning
              of the "auto" value.

       $pscmd Command used to get all the processes currently run by the user.
              The -pvc option uses  the  command  specified  by  the  variable
              $pscmd  to  determine  if there is an already running previewer,
              and to find the process ID (needed if latexmk  needs  to  signal
              the previewer about file changes).

              Each line of the output of this command is assumed to correspond
              to one process.  See the  $pid_position  variable  for  how  the
              process number is determined.

              The  default  for  pscmd  is  "NONE" under MS-Windows and cygwin
              (i.e., the command is not used), "ps -ww  -u  $ENV{USER}"  under
              OS-X,  and  "ps  -f -u $ENV{USER}" under other operating systems
              (including Linux).  In these specifications "$ENV{USER}" is sub-
              stituted by the username.

       $ps2pdf ["ps2pdf %O %S %D"]
              Command to convert .ps to .pdf file.

       $ps_filter [empty]
              The postscript file filter to be run on the newly produced post-
              script file before other processing.  Equivalent  to  specifying
              the -pF option.

       $ps_previewer ["start gv %O %S", but start %O %S under MS-Windows]
              The  command  to  invoke a ps-previewer.  (The default under MS-
              Windows will cause to be run whatever command the system has as-
              sociated with .ps files.)

              Note  that  gv  could be used with the -watch option updates its
              display whenever the postscript file changes, whereas  ghostview



                                 17 April 2020                              54








              does  not.  However, different versions of gv have slightly dif-
              ferent ways of writing this  option.   You  can  configure  this
              variable appropriately.

              WARNING: Linux systems may have installed one (or more) versions
              of gv under different names, e.g.,  ggv,  kghostview,  etc,  but
              perhaps not one actually called gv.

              Important  note:  Normally you will want to have a previewer run
              detached, so that latexmk doesn't wait for the previewer to ter-
              minate  before continuing its work.  So normally you should pre-
              fix the command by "start ", which  flags  to  latexmk  that  it
              should  do  the  detaching  of the previewer itself (by whatever
              method is appropriate to the operating system).   But  sometimes
              letting latexmk do the detaching is not appropriate (for a vari-
              ety of non-trivial reasons), so you should put the "start "  bit
              in yourself, whenever it is needed.


       $ps_previewer_landscape  ["start gv -swap %O %S", but start %O %S under
       MS-Windows]
              The command to invoke a ps-previewer in landscape mode.

       $ps_update_command [""]
              When the postscript previewer is set to be updated by running  a
              command,  this  is the command that is run.  See the information
              for the variable $ps_update_method.

       $ps_update_method [0 under UNIX, 1 under MS-Windows]
              How the postscript viewer updates its display when the .ps  file
              has  changed.  See  the  information  on  the  variable $dvi_up-
              date_method for the codes.   (Note  that  information  needs  be
              changed slightly so that for the value 4, to run a command to do
              the update, the command is specified  by  the  variable  $ps_up-
              date_command,  and for the value 2, to specify update by signal,
              the signal is specified by $ps_update_signal.)

       $ps_update_signal [Under UNIX:  SIGHUP,  which  is  a  system-dependent
       value]
              The  number of the signal that is sent to the pdf viewer when it
              is updated by sending a signal --  see  $ps_update_method.   The
              default value is the one appropriate for gv on a UNIX system.

       $pvc_timeout [0]
              If  this  variable  is  nonzero, there will be a  timeout in pvc
              mode after a period of inactivity.  Inactivity  means  a  period
              when  latexmk  has  detected  no  file changes and hence has not
              taken any actions like compiling the document. The period of in-
              activity is in the variable $pvc_timeout_mins.


       $pvc_timeout_mins [30]
              The period of inactivity, in minutes, after which pvc mode times
              out.  This is used if $pvc_timeout is nonzero.



                                 17 April 2020                              55








       $pvc_view_file_via_temporary [1]
              The same as $always_view_file_via_temporary, except that it only
              applies in preview-continuous mode (-pvc option).

       $quote_filenames [1]
              This specifies whether substitutions for placeholders in command
              specifications  (as  in  $pdflatex)  are  surrounded  by  double
              quotes.   If this variable is 1 (or any other value Perl regards
              as true), then quoting is done.  Otherwise quoting is omitted.

              The quoting method used by latexmk is tested to  work  correctly
              under  UNIX systems (including Linux and Mac OS-X) and under MS-
              Windows.  It allows the  use  of  filenames  containing  special
              characters, notably spaces.  (But note that many versions of la-
              tex and pdflatex cannot correctly  deal  with  TeX  files  whose
              names  contain spaces.  Latexmk's quoting only ensures that such
              filenames are correctly treated by the operating system in pass-
              ing arguments to programs.)

       $recorder [1]
              Whether  to  use the -recorder option to latex and pdflatex. Use
              of this option results in a file of extension .fls containing  a
              list  of  the  files  that these programs have read and written.
              Latexmk will then use this file  to  improve  its  detection  of
              source  files and generated files after a run of latex or pdfla-
              tex.

              It is generally recommended to use this option (or to  configure
              the $recorder variable to be on.)  But it only works if (pdf)la-
              tex supports the -recorder option, which is true for  most  cur-
              rent implementations

              Note  about  the  name of the .fls file: Most implementations of
              (pdf)latex produce an .fls file with the same  basename  as  the
              main  document's LaTeX, e.g., for Document.tex, the .fls file is
              Document.fls.  However,  some  implementations  instead  produce
              files  named  for  the program, i.e., latex.fls or pdflatex.fls.
              In this second case, latexmk  copies  the  latex.fls  or  pdfla-
              tex.fls  to a file with the basename of the main LaTeX document,
              e.g., Document.fls.

       $search_path_separator [See below for default]
              The character separating paths in the environment variables TEX-
              INPUTS,  BIBINPUTS, and BSTINPUTS.  This variable is mainly used
              by latexmk when the -outdir, -output-directory, -auxdir,  and/or
              -aux-directory  options are used.  In that case latexmk needs to
              communicate  appropriately  modified  search  paths  to  bibtex,
              dvipdf, dvips, and (pdf)latex.

              [Comment  to technically savvy readers: (pdf)latex doesn't actu-
              ally need the modified search path.  But,  surprisingly,  dvipdf
              and  dvips do, because sometimes graphics files get generated in
              the output or aux directories.]




                                 17 April 2020                              56








              The default under MSWin and Cygwin is ';'  and  under  UNIX-like
              operating  systems  (including Linux and OS-X) is ':'.  Normally
              the defaults give correct behavior.  But there can be  difficul-
              ties  if  your operating system is of one kind, but some of your
              software is running under an emulator for the other kind of  op-
              erating  system;  in  that  case you'll need to find out what is
              needed, and set $search_path_separator  explicitly.   (The  same
              goes,  of  course, for unusual operating systems that are not in
              the MSWin, Linux, OS-X, Unix collection.)

       $show_time [0]
              Whether to show CPU time used.

       $silence_logfile_warnings [0]
              Whether after a run of (pdf)latex to summarize warnings  in  the
              log file about undefined citations and references.  Setting $si-
              lence_logfile_warnings=0 gives the summary of warnings (provided
              silent  mode isn't also set), and this is useful to locate unde-
              fined citations and references  without  searching  through  the
              much  more  verbose log file or the screen output of (pdf)latex.
              But the summary can also be excessively annoying.   The  default
              is  not  to  give these warnings.  The command line options -si-
              lence_logfile_warning_list  and   -silence_logfile_warning_list-
              also set this variable.

              Note  that multiple occurrences for the same undefined object on
              the same page and same line will be compressed to a single warn-
              ing.

       $silent [0]
              Whether  to run silently.  Setting $silent to 1 has the same ef-
              fect as the -quiet of -silent options on the command line.

       $sleep_time [2]
              The time to sleep (in seconds) between checking for source  file
              changes when running with the -pvc option.  This is subject to a
              minimum of one second delay, except that zero delay is also  al-
              lowed.

              A  value  of  exactly 0 gives no delay, and typically results in
              100% CPU usage, which may not be desirable.

       $texfile_search [""]
              This is an obsolete variable,  replaced  by  the  @default_files
              variable.

              For   backward   compatibility,  if  you  choose  to  set  $tex-
              file_search, it is a string of  space-separated  filenames,  and
              then latexmk replaces @default_files with the filenames in $tex-
              file_search to which is added "*.tex".

       $success_cmd [undefined]
              See the documentation for $compiling_cmd.




                                 17 April 2020                              57








       $tmpdir [See below for default]
              Directory to store temporary files  that  latexmk  may  generate
              while running.

              The  default  under MSWindows (including cygwin), is to set $tm-
              pdir to the value of the first of whichever of the system  envi-
              ronment  variables  TMPDIR or TEMP exists, otherwise to the cur-
              rent directory.  Under other operating systems (expected  to  be
              UNIX/Linux,  including  OS-X),  the  default is the value of the
              system environment  variable  TMPDIR  if  it  exists,  otherwise
              "/tmp".

       $use_make_for_missing_files [0]
              Whether to use make to try and make files that are missing after
              a run of latex or pdflatex, and for which  a  custom  dependency
              has  not been found.  This is generally useful only when latexmk
              is used as part of a bigger project which is built by using  the
              make program.

              Note that once a missing file has been made, no further calls to
              make will be made on a subsequent run of latexmk to  update  the
              file.   Handling  this  problem is the job of a suitably defined
              Makefile.  See the section "USING latexmk WITH make" for how  to
              do  this.   The intent of calling make from latexmk is merely to
              detect dependencies.

       $view ["default"]
              Which kind of file is to be previewed if a  previewer  is  used.
              The  possible  values  are  "default",  "dvi", "ps", "pdf".  The
              value of "default" means that the "highest" of the kinds of file
              generated is to be used (among .dvi, .ps and .pdf).


       $warnings_as_errors [0]
              Normally  latexmk copies the behavior of latex in treating unde-
              fined references and citations and multiply  defined  references
              as  conditions  that give a warning but not an error.  The vari-
              able $warnings_as_errors controls whether this behavior is modi-
              fied.

              When  the  variable  is  non-zero, latexmk at the end of its run
              will return a non-zero status code to the  operating  system  if
              any  of  the files processed gives a warning about problems with
              citations or references (i.e., undefined citations or references
              or multiply defined references).  This is after latexmk has com-
              pleted all the runs it needs to try and resolve  references  and
              citations.   Thus  $warnings_as_errors  being nonzero causes la-
              texmk to treat such warnings as errors, but only when they occur
              on  the last run of (pdf)latex and only after processing is com-
              plete.  A non-zero value $warnings_as_errors can be set  by  the
              command-line option -Werror.

              The default behavior is normally satisfactory in the usual edit-
              compile-edit cycle.  But, for example, latexmk can also be  used



                                 17 April 2020                              58








              as  part  of  a build process for some bigger project, e.g., for
              creating documentation in the build of a  software  application.
              Then  it is often sensible to treat citation and reference warn-
              ings as errors that require the  overall  build  process  to  be
              aborted.   Of course, since multiple runs of (pdf)latex are gen-
              erally needed to resolve references and citations, what  matters
              is  not  the  warnings on the first run, but the warnings on the
              last run; latexmk takes this into account appropriately.

              In addition, when preview-continuous mode is  used,  a  non-zero
              value  for  $warnings_as_errors  changes the use of the commands
              $failure_cmd, $warning_cmd, and $success_cmd  after  a  complia-
              tion.  If there are citation or reference warnings, but no other
              errors, the behavior is as follows. If $warning_cmd is  set,  it
              is  used.  If it is not set, then then if $warnings_as_errors is
              non-zero and $failure_cmd is set, then $failure_cmd.   Otherwise
              $success_cmd  is used, if it is set.  (The foregoing explanation
              is rather complicated, because latexmk has to deal with the case
              that one or more of the commands isn't set.)


       $xdvipdfmx ["xdvipdfmx -E -o %D %O %S"]

              The  program  to  make a pdf file from an xdv file (used in con-
              junction with xelatex when $pdf_mode=5).

       $xdvipdfmx_silent_switch ["-q"]
              Switch(es) for the xdvipdfmx program when silent mode is on.

       $xelatex ["xelatex %O %S"]
              Specifies the command line for the LaTeX processing  program  of
              when  the  xelatex program is called for.  See the documentation
              of the -xelatex option for some special properties of  latexmk's
              use of xelatex.

              Note  about xelatex: latexmk uses xelatex to make an .xdv rather
              than .pdf file, with the .pdf file being created in  a  separate
              step.  This is enforced by the use of the -no-pdf option.  If %O
              is part of the command for invoking xelatex, then  latexmk  will
              insert the -no-pdf option automatically, otherwise you must pro-
              vide the option yourself.  See the documentation for the  -pdfxe
              option for why latexmk makes a .xdv file rather than a .pdf file
              when xelatex is used.

              To do a coordinated setting of all of $latex, $pdflatex, $luala-
              tex, and $xelatex, see the section "Advanced Configuration".


       %xelatex_input_extensions
              This  variable specifies the extensions tried by latexmk when it
              finds that an xelatex run resulted in an error that a  file  has
              not  been  found,  and  the  file is given without an extension.
              This typically happens when LaTeX  commands  of  the  form  \in-
              put{file}  or \includegraphics{figure}, when the relevant source



                                 17 April 2020                              59








              file does not exist.

              In this situation, latexmk searches for custom  dependencies  to
              make  the  missing  file(s),  but restricts it to the extensions
              specified by the variable  %xelatex_input_extensions.   The  de-
              fault extensions are 'tex', 'pdf', 'jpg, and 'png'.

              See details of the %latex_input_extensions for other information
              that equally applies to %xelatex_input_extensions.

       $xelatex_silent_switch ["-interaction=batchmode"]
              Switch(es) for the xelatex program (specified  in  the  variable
              $xelatex) when silent mode is on.

              See  details  of  the $latex_silent_switch for other information
              that equally applies to $xelatex_silent_switch.




CUSTOM DEPENDENCIES
       In any RC file a set of custom dependencies can be set up to convert  a
       file with one extension to a file with another.  An example use of this
       would be to allow latexmk to convert a .fig file to .eps to be included
       in the .tex file.


   Defining a custom dependency:
       The old method of configuring latexmk to use a custom dependency was to
       directly manipulate the @cus_dep_list array that  contains  information
       defining  the  custom  dependencies.   (See  the section "Old Method of
       Defining Custom Dependencies" for details.) This  method  still  works,
       but is no longer preferred.

       A better method is to use the subroutines that allow convenient manipu-
       lations of the custom dependency list.  These are

           add_cus_dep( fromextension, toextension, must, subroutine )
           remove_cus_dep( fromextension, toextension )
           show_cus_dep()

       The arguments are as follows:

       from extension:
              The extension of the file we are converting from  (e.g.  "fig").
              It is specified without a period.

       to extension:
              The extension of the file we are converting to (e.g. "eps").  It
              is specified without a period.

       must:  If non-zero, the file from which we are converting  must  exist,
              if  it doesn't exist latexmk will give an error message and exit
              unless the -f option is specified.  If must is zero and the file



                                 17 April 2020                              60








              we  are  converting from doesn't exist, then no action is taken.
              Generally, the appropriate value of must is zero.

       function:
              The name of the subroutine that latexmk should call  to  perform
              the  file  conversion.   The first argument to the subroutine is
              the base name of the file to be converted without any extension.
              The  subroutines  are declared in the syntax of Perl.  The func-
              tion should return 0 if it was successful and a  nonzero  number
              if it failed.


       Naturally  add_cus_dep adds a custom dependency with the specified from
       and to extensions.  If a custom dependency has been previously  defined
       (e.g.,  in an rcfile that was read earlier), then it is replaced by the
       new one.

       The subroutine remove_cus_dep removes the specified custom  dependency.
       The subroutine show_cus_dep causes a list of the currently defined cus-
       tom dependencies to be sent to the screen output.


   How custom dependencies are used:
       An instance of a custom dependency rule is created whenever latexmk de-
       tects  that a run of latex/pdflatex needs to read a file, like a graph-
       ics file, whose extension is the to-extension of a  custom  dependency.
       Then  latexmk  examines  whether  a file exists with the same name, but
       with the corresponding from-extension, as specified in  the  custom-de-
       pendency.   If it does, then a corresponding instance of the custom de-
       pendency is created, after which the rule is invoked whenever the  des-
       tination  file  (the one with the to-extension) is out-of-date with re-
       spect to the corresponding source file.

       To make the new destination file, the Perl subroutine specified in  the
       rule is invoked, with an argument that is the base name of the files in
       question.  Simple cases just involve a subroutine invoking an  external
       program;  this  can  be  done by following the templates below, even by
       those without knowledge of the Perl programming language.   Of  course,
       experts could do something much more elaborate.

       One  item  in the specification of each custom-dependency rule, labeled
       "must" above, specifies how the rule should be applied when the  source
       file fails to exist.

       When  latex reports that an input file (e.g., a graphics file) does not
       exist, latexmk tries to find a source file and a custom dependency that
       can be used to make it.  If it succeeds, then it creates an instance of
       the custom dependency and invokes it to make the  missing  file,  after
       which the next pass of latex etc will be able to read the newly created
       file.

       Note for advanced usage: The operating  system's  environment  variable
       TEXINPUTS can be used to specify a search path for finding files by la-
       tex etc.  Correspondingly, when a missing  file  is  reported,  latexmk



                                 17 April 2020                              61








       looks  in the directories specified in TEXINPUTS as well as in the cur-
       rent directory, to find a source file from which an instance of a  cus-
       tom dependency can be used to make the missing file.


   Function to implement custom dependency, traditional method:
       The  function  that implements a custom dependency gets the information
       on the files to be processed in two ways.  The first is through its one
       argument;  the argument contains the base name of the source and desti-
       nation files.  The second way is described later.

       A simple and typical example of code in an initialization rcfile  using
       the first method is:

           add_cus_dep( 'fig', 'eps', 0, 'fig2eps' );
           sub fig2eps {
               system( "fig2dev -Leps \"$_[0].fig\" \"$_[0].eps\"" );
           }

       The  first  line adds a custom dependency that converts a file with ex-
       tension "fig", as created by the xfig program, to an encapsulated post-
       script  file,  with extension "eps".  The remaining lines define a sub-
       routine that carries out the conversion.   If  a  rule  for  converting
       "fig"  to  "eps"  files already exists (e.g., from a previously read-in
       initialization file), the latexmk will delete this rule  before  making
       the new one.

       Suppose  latexmk  is  using this rule to convert a file "figure.fig" to
       "figure.eps".  Then it will invoke the fig2eps  subroutine  defined  in
       the  above  code with a single argument "figure", which is the basename
       of each of the files (possibly with a path component).   This  argument
       is  referred to by Perl as $_[0].  In the example above, the subroutine
       uses the Perl command system to invoke the program fig2dev.  The double
       quotes around the string are a Perl idiom that signify that each string
       of the form of a variable name, $_[0] in this case, is  to  be  substi-
       tuted by its value.

       If  the  return  value of the subroutine is non-zero, then latexmk will
       assume an error occurred during the execution of  the  subroutine.   In
       the  above  example, no explicit return value is given, and instead the
       return value is the value returned by the last  (and  only)  statement,
       i.e., the invocation of system, which returns the value 0 on success.

       If  you use pdflatex instead of latex, then you will probably prefer to
       convert your graphics files to pdf format, in which case you would  re-
       place the above code in an initialization file by

           add_cus_dep( 'fig', 'pdf, 0, 'fig2pdf' );
           sub fig2pdf {
               system( "fig2dev -Lpdf \"$_[0].fig\" \"$_[0].pdf\"" );
           }

       Note  1: In the command lines given in the system commands in the above
       examples, double quotes  have  been  inserted  around  the  file  names



                                 17 April 2020                              62








       (implemented  by '\"' in the Perl language).  They immunize the running
       of the program against special characters  in  filenames.   Very  often
       these  quotes  are not necessary, i.e., they can be omitted.  But it is
       normally safer to keep them in.  Even though the rules for quoting vary
       between  operating  systems,  command  shells  and individual pieces of
       software, the quotes in the above examples do not cause problems in the
       cases I have tested.

       Note  2:  One  case in which the quotes are important is when the files
       are in a subdirectory and your operating system is  Microsoft  Windows.
       Then  the  separator character for directory components can be either a
       forward slash '/' or Microsoft's more usual backward slash  '\'.   For-
       ward  slashes  are  generated  by  latexmk, to maintain its sanity from
       software like MiKTeX that mixes both directory  separators;  but  their
       correct use normally requires quoted filenames.  (See a log file from a
       run of MiKTeX (at least in v. 2.9) for an example of the  use  of  both
       directory separators.)

       Note  3:  The subroutines implementing custom dependencies in the exam-
       ples given just have  a  single  line  invoking  an  external  program.
       That's  the usual situation.  But since the subroutines are in the Perl
       language, you can implement much more  complicated  processing  if  you
       need it.


   Removing custom dependencies, and when you might need to do this:
       If  you  have some general custom dependencies defined in the system or
       user initialization file, you may find that for  a  particular  project
       they  are  undesirable.  So you might want to delete the unneeded ones.
       A situation where this would be desirable is where there  are  multiple
       custom  dependencies with the same from-extension or the same to-exten-
       sion. In that case, latexmk might choose a different one from  the  one
       you want for a specific project.  As an example, to remove any "fig" to
       "eps" rule you would use:

           remove_cus_dep( 'fig', 'eps' );

       If you have complicated sets of custom dependencies, you  may  want  to
       get  a  listing  of the custom dependencies.  This is done by using the
       line

           show_cus_dep();

       in an initialization file.


   Function implementing custom dependency, alternative methods:
So far the examples for functions to implement custom dependencies have
used the argument of the function to specify the base name of converted
file.  This method has been available since very old  versions  of  la-
texmk, and many examples can be found, e.g., on the web.

However  in later versions of latexmk the internal structure of the im-
plementation of its "rules" for  the  steps  of  processing,  including
custom  dependencies,  became  much more powerful.  The function imple-
menting a custom dependency is executed within a special context  where
a number of extra variables and subroutines are defined.  Publicly doc-
umented ones, intended to be long-term stable, are listed below,  under
the heading "Variables and subroutines for processing a rule".

Examples  of  their  use is given in the following examples, concerning
multiple index files and glossaries.

The only index-file conversion built-in to latexmk is  from  an  ".idx" file
written on one run of latex/pdflatex to an ".ind" file to be read in on a
subsequent run.  But with the index.sty package,  for  example, you  can
create extra indexes with extensions that you configure.  La- texmk does not
know how to deduce the extensions from  the  information it  has.  But you can
easily write a custom dependency.  For example if your latex file uses the
command  \verb|"\newindex{special}{ndx}{nnd}{Special index}"|  you  will need to get
latexmk to convert files with the extension .ndx to .nnd.  The most
elementary method is to  define  a  custom dependency as follows:

           add_cus_dep( 'ndx', 'nnd', 0, 'ndx2nnd' );
           sub ndx2nnd {
               return system( "makeindex -o \"$_[0].nnd\" \"$_[0].ndx\"" );
           }
           push @generated_exts, 'ndx', 'nnd';

       Notice  the  added line compared with earlier examples.  The extra line
       gets the extensions "ndx" and "nnd" added to the list of extensions for
       generated files; then the extra index files will be deleted by clean-up
       operations

       But if you have yet more indexes with yet different  extensions,  e.g.,
       "adx"  and  "and", then you will need a separate function for each pair
       of extensions.  This is quite annoying.   You  can  use  the  Run_subst
       function to simplify the definitions to use a single function:

\begin{verbatim}
	add_cus_dep( 'ndx', 'nnd', 0, 'dx2nd' );
	add_cus_dep( 'adx', 'and', 0, 'dx2nd' );
	sub dx2nd {
	  return Run_subst( "makeindex -o %D %S" );
	}
	push @generated_exts, 'ndx', 'nnd', 'adx', 'and';
\end{verbatim}

You could also instead use

\begin{verbatim}
	add_cus_dep( 'ndx', 'nnd', 0, 'dx2nd' );
	add_cus_dep( 'adx', 'and', 0, 'dx2nd' );
	sub dx2nd {
	  return Run_subst( $makeindex );
	}
	push @generated_exts, 'ndx', 'nnd', 'adx', 'and';
\end{verbatim}

This  last example uses the command specification in $makeindex, and so
any customization you have made for the standard index also applies  to
your extra indexes.

Similar techniques can be applied for glossaries.

Those of you with experience with Makefiles, may get concerned that the \verb|.ndx|
file is written during a run of latex/pdflatex and is always later than  the
\verb|.nnd|  last read in.  Thus the .nnd appears to be perpetually out-of-date.  This
situation, of circular dependencies, is  endemic  to latex, and is one of the
issues that latexmk is programmed to overcome.  It examines the contents of the
files (by use of a checksum), and  only does a remake when the file contents
have actually changed.

Of  course  if you choose to write random data to the .nnd (or the .aux
file, etc) that changes on each new run, then you will have a  problem.
For  real experts: See the \verb|%hash_cal_ignore_pattern| if you have to deal
with such problems.

   Old Method of Defining Custom Dependencies:
       In much older versions of latexmk, the only method of  defining  custom
       dependencies  was  to directly manipulate the table of custom dependen-
       cies.  This is contained in the @cus_dep_list array.  It is an array of
       strings,  and each string in the array has four items in it, each sepa-
       rated by a space, the  from-extension,  the  to-extension,  the  "must"
       item,  and the name of the subroutine for the custom dependency.  These
       were all defined above.

       An example of the old method of defining custom dependencies is as fol-
       lows.  It  is  the code in an RC file to ensure automatic conversion of
       .fig files to .eps files:

           push @cus_dep_list, "fig eps 0 fig2eps";
           sub fig2eps {
               return system( "fig2dev -Lps \"$_[0].fig\" \"$_[0].eps\"" );
           }

       This method still works, and is almost equivalent  to  the  code  given
       earlier  that used the add_cus_dep subroutine.  However, the old method
       doesn't delete any previous custom-dependency for the same  conversion.
       So the new method is preferable.


ADVANCED CONFIGURATION: Some extra resources and advanced tricks
       For  most purposes, simple configuration for latexmk along the lines of
       the examples given is sufficient.  But  sometimes  you  need  something
       harder.   In this section, I indicate some extra possibilities.  Gener-
       ally to use these, you need to be fluent in the  Perl  language,  since
       this is what is used in the rc files.

       See  also the section DEALING WITH ERRORS, PROBLEMS, ETC.  See also the
       examples in the directory example_rcfiles in the latexmk distributions.
       Even if none of the examples apply to your case, they may give you use-
       ful ideas

   Utility subroutines
       ensure_path( var, values ...)

              The first parameter is the name of one of the system's  environ-
              ment  variables  for search paths.  The remaining parameters are
              values that should be in the variable.  For each  of  the  value
              parameters,  if  it  isn't  already  in the variable, then it is
              prepended to the variable; in that case the environment variable
              is  created  if it doesn't already exist. For separating values,
              the character appropriate the the operating system  is  used  --
              see the configuration variable $search_path_separator.

              Example:

                ensure_path( 'TEXINPUTS', './custom_cls_sty_files//' );

              (In this example, the trailing '//' is documented by TeX systems
              to mean that latex, pdflatex, etc search for files in the speci-
              fied directory and in all subdirectories.)

              Technically   ensure_path   works  by  setting  Perl's  variable
              $ENV{var}, where var is the name of the  target  variable.   The
              changed  value  is then passed as an environment variable to any
              invoked programs.


   Variables and subroutines for processing a rule
       A step in the processing is called a rule. One possibility to implement
       the  processing  of a rule is by a Perl subroutine.  This is always the
       case for custom dependencies. Also, for any other rule, you can  use  a
       subroutine  by  prefixing the command specification by the word "inter-
       nal" -- see the section FORMAT OF COMMAND SPECIFICATIONS.

       When you use a subroutine for processing a rule, all the  possibilities
       of Perl programming are available, of course.  In addition, some of la-
       texmk's internal variables and subroutines  are  available.   The  ones
       listed  below  are  intended  to  be available to (advanced) users, and
       their specifications will generally have stability under upgrades. Gen-
       erally,  the  variables  should be treated as read-only: Changing their
       values can have bad consequences, since it is liable  to  mess  up  the
       consistency of what latexmk is doing.

       $rule  This  variable  has  the  name of the rule, as known to latexmk.
              Note that the exact contents of this variable for a  given  rule
              may be dependent on the version of latexmk

       $$Psource
              This gives the name of the primary source file.  Note the double
              dollar signs.

       $$Pdest
              This gives the name of the main output file if  any.   Note  the
              double dollar signs.


       rdb_ensure_file( $rule, file )
              This  a subroutine that ensures that the given file is among the
              source files for the specified rule.  It is typically used when,
              during  the  processing of a rule, it is known that a particular
              extra file is among the dependencies that latexmk  should  know,
              but its default methods don't find the dependency. Almost always
              the first argument is the name of the rule currently being  pro-
              cessed, so it is then appropriate to specify it by $rule.

              For  examples of its use, see some of the files in the directory
              example_rcfiles of latexmk's distribution.  Currently the  cases
              that  use  this  subroutine  are bib2gls-latexmkrc, exceltex_la-
              texmkrc and texinfo-latexmkrc.  These illustrate  typical  cases
              where  latexmk's normal processing fails to detect certain extra
              source files.

       rdb_remove_files( $rule, file, ... )
              This subroutine removes one or more files  from  the  dependency
              list for the given rule.

       rdb_list_source( $rule )
              This  subroutine returns the list of source files (i.e., the de-
              pendency list) for the given rule.

       rdb_set_source( $rule, file, ... )

       rdb_set_source( $rule, @files )
              This subroutine sets the dependency list for the given  rule  to
              be the specified files.  Files that are already in the list have
              unchanged information.  Files that were  not  in  the  list  are
              added to it.  Files in the previous dependency list that are not
              in the newly specified list of files are removed from the depen-
              dency list.

       Run_subst( command_spec )
              This subroutine runs the command specified by command_spec.  The
              specification is a string in the format listed  in  the  section
              "Format  of Command Specifications".  An important action of the
              Run_subst is to make substitutions of placeholders, e.g., %S and
              %D  for  source and destination files; these get substituted be-
              fore the command is run.  In addition, the command after substi-
              tution  is  printed  to  the screen unless latexmk is running in
              silent mode.


Coordinated Setting of Commands for *latex

To set all of \verb|$latex|, \verb|$pdflatex|, \verb|$lualatex|, and \verb|$xelatex|  to  a  common
pattern,  you  can  use one of the following subroutines, \verb|std_tex_cmds|,
\verb|alt_tex_cmds|, and \verb|set_tex_cmds|.

They work as follows

\begin{verbatim}
	&std_tex_cmds;
\end{verbatim}

This results in \verb|$latex = 'latex %O %S'|, and  similarly  for  \verb|$pdflatex|,
\verb|$lualatex|,  and  \verb|$xelatex|.   Note the ampersand in the invocation; this
indicates to Perl that a subroutine is being called.

\begin{verbatim}
	&alt_tex_cmds;
\end{verbatim}

This results in \verb|$latex = 'latex %O %P'|, and  similarly  for
\verb|$pdflatex|, \verb|$lualatex|,  and  \verb|$xelatex|.   Note the ampersand
in the invocation; this indicates to Perl that a subroutine is being called.

\begin{verbatim}
	set_tex_cmds( CMD_SPEC );
\end{verbatim}

Here \verb|CMD_SPEC| is the command line without the program  name.  This  results in
\verb|$latex = 'CMD_SPEC'|, and similarly for \verb|$pdflatex|,
\verb|$lualatex|, and \verb|$xelatex|.
An example would be

\begin{verbatim}
	set_tex_cmds( '--interaction=batchmode %O %S' );
\end{verbatim}


Advanced configuration: Using latexmk with make This section is targeted only
at advanced users who use the  make  program for complex projects, as for
software development, with the dependencies specified by a Makefile.

Now the basic task of latexmk is to run  the  appropriate  programs  to make  a
viewable version of a LaTeX document.  However, the usual make program is not
suited to this purpose for at least two reasons.   First is that the use of
LaTeX involves circular dependencies (e.g., via .aux files), and these cannot
be handled by the standard make program.  Sec- ond  is  that  in  a  large
document the set of source files can change quite frequently, particularly with
included graphics  files;  in  this situation  keeping a Makefile manually
updated is inappropriate and er- ror-prone, especially when the dependencies
can be determined automati- cally.  Latexmk solves both of these problems
robustly.

Thus  for  many  standard LaTeX documents latexmk can be used by itself without
the make program. In a complex project it simply needs  to  be suitably
configured.  A standard configuration would be to define custom dependencies to
make graphics files from their source files  (e.g., as  created  by  the  xfig
program).  Custom dependencies are latexmk's equivalent of pattern rules in
Makefiles.

Nevertheless there are projects for which a  Makefile  is  appropriate,
and it is useful to know how to use latexmk from a Makefile.  A typical
example would be to generate documentation for a software project.  Po-
tentially  the  interaction  with the rest of the rules in the Makefile
could be quite complicated, for example if some of the source files for
a LaTeX document are generated by the project's software.

In this section, I give a couple of examples of how latexmk can be usefully
invoked from a Makefile.  The examples use specific  features  of current
versions  of  GNU make, which is the default on both linux and OS-X systems.
They may need modifications for other versions of make.

The simplest method is simply to delegate all the relevant tasks to latexmk, as
is suitable for a straightforward LaTeX document.  For this a suitable Makefile
is like

\begin{verbatim}
	           .PHONY : FORCE_MAKE
	           all : try.pdf
	           %.pdf : %.tex FORCE_MAKE
	               latexmk -pdf -dvi- -ps- $<
\end{verbatim}

(Note: the last line must be introduced by a tab for  the  Makefile  to
function  correctly!)  Naturally, if making try.pdf from its associated
LaTeX file try.tex were the only task to be performed, a direct use  of
latexmk  without  a  Makefile would normally be better.  The benefit of
using a Makefile for a LaTeX document would be  in  a  larger  project,
where lines such as the above would be only be a small part of a larger
Makefile.

The above example has a pattern rule for making a .pdf file from a .tex
file,  and it is defined to use latexmk in the obvious way.  There is a
conventional  default  target  named  "all",  with  a  prerequisite  of
try.pdf.   So  when  make is invoked, by default it makes try.pdf.  The
only complication is  that  there  may  be  many  source  files  beyond
try.tex, but these aren't specified in the Makefile, so changes in them
will not by themselves cause latexmk to be invoked.  Instead, the  pat-
tern  rule is equipped with a "phony" prerequisite FORCE_MAKE; this has
the effect of causing the rule to be always out-of-date,  so  that  la-
texmk  is always run.  It is latexmk that decides whether any action is
needed, e.g., a rerun of pdflatex.  Effectively the Makefile  delegates
all  decisions  to  latexmk, while make has no knowledge of the list of
source files except for primary LaTeX file for the  document.  If there
are, for example, graphics files to be made, these must be made by cus-
tom dependencies configured in latexmk.

But something better is needed in more complicated situations, for  ex-
ample, when the making of graphics files needs to be specified by rules
in the Makefile.  To do this, one can use a Makefile like  the  follow-
ing:

\begin{verbatim}
              TARGETS = document1.pdf document2.pdf
              DEPS_DIR = .deps
              LATEXMK = latexmk -recorder -use-make -deps \
                    -e 'warn qq(In Makefile, turn off custom dependencies\n);' \
                    -e '@cus_dep_list = ();' \
                    -e 'show_cus_dep();'
              all : $(TARGETS)
              $(foreach file,$(TARGETS),$(eval -include $(DEPS_DIR)/$(file)P))
              $(DEPS_DIR) :
                     mkdir $@
              %.pdf : %.tex
                     if [ ! -e $(DEPS_DIR) ]; then mkdir $(DEPS_DIR); fi
                     $(LATEXMK) -pdf -dvi- -ps- -deps-out=$(DEPS_DIR)/$@P $<
              %.pdf : %.fig
                     fig2dev -Lpdf $< $@
\end{verbatim}

(Again,  the  lines  containing  the  commands  for the rules should be started
with tabs.)  This example was inspired by how GNU automake handles automatic
dependency tracking of C source files.

After  each  run of latexmk, dependency information is put in a file in
the .deps subdirectory.  The Makefile causes these dependency files  to
be read by make, which now has the full dependency information for each target
.pdf file.  To make things less trivial it  is  specificed  that two  files
\verb|document1.pdf| and \verb|document2.pdf| are the targets.  The dependency files are
.deps/document1.pdfP and .deps/document2.pdfP.

There is now no need for the phony prerequisite for the  rule  to  make .pdf
files from .tex files.  But I have added a rule to make .pdf files from .fig
files produced by the xfig program; these are  commonly  used for  graphics
insertions  in  LaTeX documents.  Latexmk is arranged to output a dependency
file after each run.  It is given the \verb|-recorder| option,  which  improves its
detection of files generated during a run of pdflatex; such files should not be
in the dependency list.  The -e  options  are  used  to  turn off all custom
dependencies, and to document this.  Instead the \verb|-use-make| is used to delegate
the making of  missing files to make itself.

Suppose  in  the LaTeX file there is a command \verb|\includegraphics{graph}|, and an
xfig file "graph.fig" exists.  On a first run, pdflatex  reports a  missing
file, named "graph". Latexmk succeeds in making "graph.pdf" by calling "make
graph.pdf", and after completion of its work, it lists "fig.pdf" among the
dependents of the file latexmk is making.  Then let "fig.fig" be updated, and
then let make be  run.   Make  first  remakes "fig.pdf", and only then reruns
latexmk.

Thus  we  now  have  a method by which all the subsidiary processing is
delegated to make.



