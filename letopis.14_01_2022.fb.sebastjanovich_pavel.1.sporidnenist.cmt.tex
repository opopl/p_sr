% vim: keymap=russian-jcukenwin
%%beginhead 
 
%%file 14_01_2022.fb.sebastjanovich_pavel.1.sporidnenist.cmt
%%parent 14_01_2022.fb.sebastjanovich_pavel.1.sporidnenist
 
%%url 
 
%%author_id 
%%date 
 
%%tags 
%%title 
 
%%endhead 
\zzSecCmt

\begin{itemize} % {
\iusr{Andrey Kravets}
Работа, семья, ипотека....

\iusr{Queen Jane}
Футбол

\iusr{Dmitry Belous}
Это признак инфантильности нациции

\begin{itemize} % {
\iusr{Александра Лутохина}
\textbf{Dmitry Belous} Именно. Как научила воспитателька в детском саду, так и делать. Вместо воспитательки правительство, всего-то.

\iusr{Павел Гурнік}
\textbf{Dmitry Belous} насамперед ознака інфантильності і патерналізму най-активних майже-громадян.
Інші беруть з них приклад
\end{itemize} % }

\iusr{Олена Пархуня}

Ну, т. е, если коротко... мы все разные, и всегда такие были.. А политика-это
способ найти что-то для объединения людей в группы.. Найти вопросы(и ответы),
которые устроят большинство. И только.. Сегодня (21 век, однако) просто все
быстрее выходит наружу, и очень много информационного д\#рьма Но, с этим надо
учиться жить..(21 век, однако)). Думаю, телевизор, 70 лет назад, тоже вносил
пугающие изменения в инфополе.. прошлое уже..

\iusr{Roman Sozansky}

А це нормально для будь-якого суспільства. Тому що всі дуже різні, культурно,
освітньо, за способом мислення, за особистими якостями і т.і. Я вже вам казав,
вам особисто подам руку, незважаючи на різні погляди в гуманітарній сфері. Про
це я готовий з вами дискутувати скільки завгодно і вважати, що ви погано
розбираєтесь в гуманітарних питаннях. Маю на це право. Але це не заперечує
того, що як людина, ви мені видаєтесь гідним чоловіком і я завжди про це
пам'ятатиму перед усім. А суперечки, дискусії, незгоди - це нормально. Цим і
відрізняється здорове суспільство від хворого. В здоровому організмі кров
кипить і лейкоцити з еритроцитами постійно влаштовують гармидер. А в важко
хворому - штиль і ніхто ні з ким не сперечається. Подивіться на Росію, Північну
Корею, Білорусь. Ви такого \enquote{єднання} хочете? Такого камертону? Тут в рамках
однієї сім'ї, стосунків кількох людей важко найти повне порозуміння, порою, а
він хоче, щоб 40 млн людей думали і дихали в унісон. Дивак-чоловік.


\iusr{Ігор Можайко}
\enquote{Бла-асловения!.. Всем сердцем!.. Нежность испытывал, нежность, понимаешь ты, мужицкая морда?}

\iusr{Володимир Брунько}

ЛОМы не те. Если технологично, по-деловому, как в бизнесе, то нужно искать
ЛОМов (продавцы), покупать их, обучать, сделать все компоненты маркетинга,
отработать бюджет на продвижение \& go!

Покупать не за деньги. Хотя и это нужно. Покупать - это перевербовывать.

\iusr{Святослав Дубина}
Сегодня в 16:30 будем проявлять волю  @igg{fbicon.face.wink.tongue}  Паша, не дрейфь, настроим мы общий камертон

\iusr{Зиновій Бокійчук}

...органічно було би, коли такий тон виробляється в суспільстві, а не
сподіватися на вождів. Джерелом гармонії завжди є стосунки між людьми. Тому я
би поставив питання: Як повернути цим стосункам природну гармонію?

\begin{itemize} % {
\iusr{Олена Пархуня}
\textbf{Зиновій Бокійчук} 

она всегда с людьми.. Не может кто-то забрать гармонию, или вернуть её.. Просто
это не ваша гармония.. Вы хотите только (!)такую, как вам удобно. Мне она зачем?


\iusr{Зиновій Бокійчук}
\textbf{Олена Пархуня} Якщо Вас влаштовує так як є зараз, то вам гармонія дійсно не потрібна. Я допускаю, що знайдуться люди яким подобатися какафонія.

\iusr{Павел Себастьянович}
\textbf{Зиновій Бокійчук} да, камертон должен начать звучать из общества. Может он и звучит, не слышим.
\end{itemize} % }

\iusr{Oleksandr Diubanov}
Ценностных приоритетов для себя никто не выделяет.

\begin{itemize} % {
\iusr{Олена Пархуня}
\textbf{Oleksandr Diubanov} нит. Просто у каждого они свой. Свои, не ваши.. Это и напрягает, вас..

\iusr{Oleksandr Diubanov}
\textbf{Олена Пархуня} поэтому и каждый лишь сам за себя. Но вы ошибаетесь в том что меня это напрягает. Другое слово.
\end{itemize} % }

\iusr{Марьяна Шевцова}

Разделяй и властвуй... именно этим занимаются все правители... они разделили
народ, науськали, натравили запад на восток, объявили войну русскому языку, на
котором говорит почти вся Украина, допустили, изобрели позорную гибридную войну
в Донбассе, ограбили народ налогами тарифами и погнали нищих и обездоленных на
войну защищать свои сверхприбыли ... они выдавили лучших работников в Европу и
хотят теперь их ограбить налогами, они установили себе космические зарплаты за
узаконенное безделье...и этому беспределу наших избранников не видно края...

\iusr{Людмила Халанская}
\textbf{Марьяна Шевцова} Вы забыли добавить: а мы всё это терпим и местами даже поощряем.

\iusr{Олена Пархуня}

Вы умеете слова складывать в стройные конструкции.. Но, сегодня, это уже не
сильно работает.. т. е призывы не работают совсем.. Работают действия,
харизма, деньги... В итоге будет так:

Если просто... конечно(стихи не мои)
Но в тему @igg{fbicon.monkey.see.no.evil} 
"Я почистил карму,
Выхожу в астрал..
Но пришлось вернуться-
Кот опять наср@ лал..

\begin{itemize} % {
\iusr{Микола Чепурний}
\textbf{Олена Пархуня} 

Вам осталось задаться вопросом: а почему народ клюёт на харизмы, деньги (как
пряник) и действия (как кнут) ? Вопрос риторический и ответ очевиден: пряник
\enquote{на халяву}, а кнут - что бы безмозглые не разбежались.

\end{itemize} % }

\iusr{Татьяна Нанкина}
Бедность, одна из причин.

\iusr{Микола Чепурний}
\textbf{Татьяна Нанкина} А причина бедности - глупость ?!

\iusr{Татьяна Нанкина}
\textbf{Микола Чепурний}, так

\iusr{Алена Петренко}

Утеряна духовность, превалирует материальный мир, \enquote{разучились} читать и
думать, гаджеты, вместо помощи, используются для праздной развлекаловки в
основном, ушло живое общение между нами, нас пожирает одиночество и плюс к этому
два года масочной шизофрении, страха.....

\iusr{Наталия Давиденко}

Все нации одинаковые, единства нет нигде. В любом вопросе. Хотя я за то, чтобы
мы были терпимее и добрее друг к другу.

\begin{itemize} % {
\iusr{Микола Чепурний}
\textbf{Наталия Давиденко} Ой ли -одинаковые ?! Есть нации, наученные пророками \enquote{терпеть и быть добрыми взаимно}

\iusr{Людмила Шумлянська}
\textbf{Микола Чепурний} кто эти святые?
\end{itemize} % }

\iusr{Виктор Вовин}

Спорідненість это про родство, наверное, больше, как явное так и возможное.

А вот эмпатии в нашем обществе критично не хватает, ее проявление многие
воспринимают с недоверием или удивлением, в зависимости от собственного уровня
таковой..


\iusr{Natali Korchinskaya}

Меня поражает одно, за 8 лет ни одного движения не было против войны, ни
активистов, ни родителей детей которых потеряли, никого и это страшно!!

\begin{itemize} % {
\iusr{Iren Semivolos}
\textbf{Natali Korchinskaya} до росіян-загарбників звертайтесь. Чи ви хочете, щоб українців просто вбивали, а вони не захищались?!
\end{itemize} % }

\iusr{Ольга Михайлова}

После массированного наступления украинских активистов на права (и обычаи
коммуникации!) русскоязычных, с хамством, хейтингом, отчуждением по типу
\enquote{чемодан, вокзал, Россия}... - сложно представить душевное единение. Мол, мы
добились своего - а теперь не просто смиряйтесь, а радуйтесь и нас любите?


\end{itemize} % }
