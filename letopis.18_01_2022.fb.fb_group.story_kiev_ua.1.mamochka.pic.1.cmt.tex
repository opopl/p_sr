% vim: keymap=russian-jcukenwin
%%beginhead 
 
%%file 18_01_2022.fb.fb_group.story_kiev_ua.1.mamochka.pic.1.cmt
%%parent 18_01_2022.fb.fb_group.story_kiev_ua.1.mamochka
 
%%url 
 
%%author_id 
%%date 
 
%%tags 
%%title 
 
%%endhead 

\iusr{Ирина Кравцова}
Красивая с младенчества.

\iusr{Владимир Новицкий}
\textbf{Ирина Кравцова} Спасибо Ирочка

\iusr{Ирина Кравцова}
\textbf{Владимир Новицкий} 

Ваш рассказ прекрасен. Часто бываю на Павловской, перехожу туда через, наверное
единственный \enquote{сквозняк} там сохранилось несколько старых домов со старыми
акациями и шелковицами. Там живёт старичок по имени Христофор, сидит на порожке
своей маленькой квартирки, хочет с кем то поговорить. Это просто портал в
прекрасное прошлое. Жаль, и туда уже \enquote{добрались}.

\iusr{Татьяна Демиденко}

Какой рассказ @igg{fbicon.hands.pray} пролетело все перед глазами как фильм
!Спасибо @igg{fbicon.hands.pray} 

\iusr{Владимир Новицкий}
\textbf{Татьяна Демиденко} Спасибо!!

\iusr{Татьяна Демиденко}
Всем здоровья  @igg{fbicon.hands.pray} 

\iusr{Olena Klymenko}

Спасибо большое, за прекрасный, душевный рассказ о жизни Вашей семьи. Будьте
здоровы и счастливы.

\iusr{Владимир Новицкий}
\textbf{Olena Klymenko} Спасибо

\iusr{Раиса Карчевская}
Очень красивая фотография

\iusr{Михайло Наместник}
Поразительно, как вам удалось сохранить семейньій фотоархив

\iusr{Владимир Новицкий}
\textbf{Михайло Наместник} 

Брали с собой самое дорогое, память о наших близких и эти фото - которые нам о
них напоминали. К сожалению все остальные вещи - дедушкины иконы, сахарницу
Фаберже, папины ордена и медали и многое другое на которое мы имели разрешение-
отобрали у нас на Московской таможне при посадке в самолёт.
