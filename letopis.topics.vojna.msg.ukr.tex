% vim: keymap=russian-jcukenwin
%%beginhead 
 
%%file topics.vojna.msg.ukr
%%parent topics.vojna.msg
 
%%url 
 
%%author_id 
%%date 
 
%%tags 
%%title 
 
%%endhead 

https://www.facebook.com/groups/story.kiev.ua/posts/1914763742053751/

Kiev Fortress Валентина Фищук Відкрийте будь-ласка, твори Григорія Саввича
Сковороди - хоча він й є наш великий український філософ, але писав свої твори
про Бога та Людину російською та латиною. Його теж треба викорінити? І щодо
латини та мови. Специфіка будь-якої мови полягає в тому, що її неможливо просто
так "викорінити", як би те комусь не хотілось би. Будь-яка мова, і російська в
тому числі, є живим організмом, який живе за власними законами, і який
неможливо просто так загнати в якісь межі законами або ж циркулярами. Ось та же
латинська мова. Римська імперія у свій час жорстоко насаджувала свій лад у
поневолених країнах, не менш жорстокіше ніж росімперія в Україні або Польщі або
британська імперія в індії. Римська імперія давно вже зникла, а латина живе.
Хоча й мертва мова в тому, що нею ніхто не розмовляє в звичайному житті, але
вона живе в словах, живе в творах, живе як мова богослужіння. Латина - офіційна
мова Ватикану. Також, "специфіка" , календар, інформація, реформація - всі ці
слова є спадком жорстокої, аморальної римської імперії, в якій Нерон вбив свою
матір та спалив Рим, а Калігула вїжджав в сенат на коні. Тим не менш,
незважаючи на все це, латинська мова пережила 15 сторіч, і вже ніколи не
зникне, хоча й існує лише як писемна мова. Тим більше російська мова, як мова
живого спілкування, як мова, що розповсюджена далеко за межами московії,
переживе і криваву зловісну брехливу московію, переживе і минулу свою імперську
провину перед Україною, і буде жити далі, і в Україні зокрема. А якщо вам не
подобається росмова під час війни - адресуйте, будь-ласка, зокрема свої
занепокоєння до конкретних українців, що розмовляють росмовою і широко відомі в

Відкрийте будь-ласка повість Захар Беркут Івана Франка... і перші рядки це
будут рядки з Пушкіна (цитата з поеми Руслан і Людмила). як це не дивно. Пушкін
не обмежується лише Полтавою або ж Клеветникам России, і зокрема наш Каменяр,
Іван Франко досить високо цінував творчість Пушкіна. Але звісно, гнів та
ненависть з боку українців до сучасної московії є абсолютно законна річ,
злочини сучасної московії просто перетнули всі можливі людські межі, але все ж
таки тут важливо підходити до таких речей, як знесення памятників та
перейменування вулиць, з більшим розумом, ніж це зараз відбувається. На наш
погляд, це є явне розпорошення розумових та емоційних сил, які можна було би
більш продуктивно витратити на допомогу нашим хлопцям, зокрема Маріуполю.

https://www.facebook.com/kyjeslav.borshchahivets/posts/5195164247219634

У вас тута трохи якась каша. То парк чудовий, то він слугував русифікації. Ну -
слугував, ну і шо. росімперія давно вже вмерла - а чудовий парк залишився. От
ви коли ідете в Київскому метро, яке відкрили в 1960 році, ви ж не
задумувались, що його будувала компартія, яка винна в багатьох злочинах. срср
вмер - а чудове гарне метро залишилось, хоч би якою тоталітарною державою совок
не був. Або ж гімн Києва, Як тебе не любити, Києве Мій. Теж був створений в
совку в 1962, і взагалі його першим виконавцем був росіянин, Юрій Гуляєв. І
тому подібне. А щодо Пушкіна, відкрийте Захар Беркут Івана Франка, і там ви
побачите цитату Пушкіна. Іван Франко взагалі досить високо оцінював творчість
Пушкіна. Так само в ширшому значенні і творчість поетів та письменників. Можете
знести хоч всі памятники, всі до одного, але творчість поетів та письменників,
зокрема Пушкіна, буде продовжувати жити в серцях багатьох людей, буде
продовжувати жити також на сторінках книг (через цитати, згадки і тому
подібне), і зокрема також творів українських письменників, таких як Іван
Франко, бо ж Пушкін не обмежується лише Полтавою або ж Клеветникам России, і
звісно справжня цінність Пушкіна не вимірюється кількістю памятників. Щодо
порівняння московії і російської імперії, знаєте, сучасна московія, тобто
путінська московія, навіть гірша і за росімперію і за срср, бо московія, бридке
хиже утворення на тілі Землі, зараз нищить те, що будували таким великим коштом
і такими великими зусиллями і росімперія, і срср, як-от найбільший у світі
літак Мрія АН-225. московія не має ніякого відчуття цінності і власне Пушкіна,
бо вона бомбить Місто, Вічне Місто Київ, де той самий памятник і знаходиться
(або ж Одесу, де навіть є музей Пушкіна).

розумієте, тут ви теж впадаєто в крайнощі, тому що не ви єдиний, хто любить
Україну. І кожен любить Україну трохи по-своєму. Жодного відношення не мають,
чи мають... Ви що у нас геній на кшталт Франка, що є такі самовпевнені і у
всьому на світі розбираєтесь? Все у світі так чи інакше має відношення до
України, тому что Україна є частиною цього великого світу. Крім того, у вас
немає ніякої особистої монополії вирішувати, що має відношення до України, а що
ні. Це справи громади або віча, де збираються люди, що живуть безпосередньо у
Києві, наприклад. Ось громада збирається і вирішує, бути тим чи іншим
памятниками чи ні в цьому районі чи ні. І судячи з досить жвавої дискусії,
якраз досить велика частина української громади вважає, що або це питання не на
часі, або ж памятник треба залишити, так, як він і був. Суть України полягає ж
не тільки у Пушкіні, чи Шевченкові, а у тому, чи здатні ми сукупно та мирно
вирішувати ти чі інші спірні питання. В цьому і є суть демократії. Ми самі
вважаємо, що це питання абсолютно не на часі, бо це тільки розпорошує наші
сили, зараз війна зі страшним аморальним ворогом, ви хіба забули. А після війни
- проголосуємо, обговоримо - якщо що, знесемо, нема питань.

визволяти. справа не зовсім у цьому. це лише один із словесних приводів до
вторгнення, і не самий найголовніший. Основною тезою роспропаганди, на нашу
думку, є міф про те, що в Україні "неонацисти", які служать Заходу, який хоче
знищити расєю-матушку. І вбивця - якщо він хоче вбити - він привід знайде.
Стоїть Пушкін, є ті, хто розмовляє російською - йдемо визволяти молодших
братів. Пушкіна звідусіль знесли, всі перейшли на українську - йдемо мститися
за Пушкіна, Україна вже повністю бандерівська стала, так чому б не знищити її
взагалі одним ядерним ударом. Розумієте. Вони там просто всі як скажені звірі
стали. От ви самі можете зайти на однокласники, і перевірити. Як барани без
мізків, що їм вкладуть, у те вони й вірять. І привід до насильства завжди
можна знайти. І тут справа не в тому, що ми персонально Пушкіна любимо чи ні.
Справа в принципі демократії та громадянського суспільства. Україна зможе
протистояти московії тоді, коли вона стане по справжньому згуртованою, а одним
із чинників цієї згуртованості є в тому, що треба навчитися спірні питання
вирішувати по-дружньому, злагоджено. Іще головне питання щодо московії є в
тому, як зламати, знищити саму ідеологію московії. Але це складне питання, яке
потребує напружених інтелектуальних зусиль.

ви не туди дивитесь. Розумієте, можна спалити книжки (що є поганий прецендент
сам по собі, і тут справа не в змісті цих книжок, а в самому сутті цього явища
спалювання книжок,  що ставить інформаційно для світу нас як варварів ),
викинути памятники пушкіну, але саму мову - в сенсі інструменту людського
спілкування  по собі неможливо просто так знищити. Сама історія заборон
української мови тут є найкращим свідченням. Її забороняли, принижували - а
українська мова все це пережила, і нарешті розквітає по справжньому в
незалежній Україні. І тут справа не в російській мові як такій, а в самому
понятті мови як перш за все інструменту спілкування. Так, можна казати, що
росмова це мова путіна, мова оккупанта, давайте все заборонимо, все спалимо, і
тому подібне, але від того нікому краще не буде. Такі штампи є просто
неправильним осмисленням самого факту росмови в Україні, і що з нею далі
робити. І тут також проблема в тому, що люди не усвідомлюють суть самого явища
мови. Будь-яка мова, українська, російська, англійська - це живий організм сам
по собі, який живе за своїми законами. Так що ви можете бути персонально проти
росмови в Україні, але факт є факт. росмова в Україні є і буде на час нашого
життя, і напевне буде ще надовго, а може і назавжди. Вам напевне це не
подобається, і ви цього не хочете, але це факт. Ось наприклад, ви ж знаєте, що
існує латина - мова Римської імперії. Хіба Рим був краще за московію. Може й
гірше у свої часи, аморальна, жорстока імперія... Але латина надовго пережила
Рим, і продовжує жити в медицині та запозиченнях, і навіть є офіційною мовою
Ватикану. Сковорода наприклад писав латиною (крім росмови), як і багато інших
філософів і вчених. А росмова - це жива мова, і нею говорять не тільки хуйло з
скабеєвою, а наприклад скажімо також той герой родом з Харкова або Одеси, що
вчора підбив черговий ворожий танк десь на Донбасі. Так що питання не в
існуванні росмови як такої, а питання в тому, які меседжі, які смисли генерують
ті люди в Україні, що розмовляють росмовою. Якщо досить велика частина
російськомовних людей в Україні є патріотами України, а таких дуже-дуже багато,
дуже гарних, розумних, хоробрих людей, щирих серцем і душею, якщо вони
підтримують та захищають Україну, так в чому питання? І більш того. Якщо ми
хочемо знищити московію, ми в тому числі повинні знищити її ідеологічно, тобто
перепрошити мізки зомбованим баранам запоребріком, щоби вони більш не
загрожували Україні. А цю контр-пропаганду неможливо згенерувати інакше як
росмовою, бо москалі в більшості своїй просто ніяких інших мов просто не
знають.

% Охтирка оперативна
https://www.facebook.com/groups/498652738278706/?multi_permalinks=552444902899489&comment_id=553493676127945&notif_id=1652342891110460&notif_t=feedback_reaction_generic&ref=notif

Kiev Fortress Julie Tolmacheva так, звичайно. Але ви неуважно читали, що ми
хотіли донести. Так, це мова ворога, звісно. Але це також і мова друзів. До
біса українців, справжніх патріотів України, і не тільки українців, а й друзів
України по всьому світу, розмовляють росмовою, збирають гроші, виходять на
демонстрації на підтримку України, знімають кліпи, пишуть в телеграмах,
фейсбуках, при цьому абсолютно будучи патріотами України. Наприклад, ви
телеграм канал Казанського не читаєте, напевно. Або хроніки Харкова Соловйова.
А чому. Нащо оце зациклення на московії. Що, тільки московія розмовляє
росмовою, чи що. Що, хуйло персонально винайшов росмову, чи написав хоч один
словник росмови, що кажуть, що це мова путіна. Он наш Григорій Саввич Сковорода
набагато гарніше писав на росмові, ніж уся потворна московія разом. Або ми їх,
або вони нас. Абсолютно вірно. Ми маємо справу з жахливим, брехливим,
аморальним ворогом. Але перемогу України над московією, якщо брати ідеологічний
вимір, неможливо досягти простим тотальним відторгненням усього російського.
По-перше, це неможливо в принципі, як би вам не хотілося цього досягти. А
по-друге. А що ж потрібно, власне кажучи. Тут потрібен розум. Крім емоцій і
законної люті, потрібно дуже багато розуму. Дуже-дуже багато розуму, науки.
Вивчення філософії, психології, математики, фізики, суспільних наук. Ми повинні
стати настільки розумними, наскільки можемо. Бо якщо ми не станемо розумними,
ми не зробимо нову якісну зброю, яка буде захищати нас від ворога. Якщо ми не
станемо розумними, ми не побудуємо кращий суспільний лад в Україні. Якщо ми не
станемо розумними, ми не створимо нові технології, та нові робочі місця в
Україні. Якщо ми не станемо по справжньому розумними як суспільство в цілому,
ми так і залишимося найбіднішою наскрізь корумпованою країною Європи, поруч із
злим жахливим сусідом, із постійною небезпекою нових нападів та нових ракетних
ударів. Так що тут напевне варто нагадати. Учітеся, читайте, і чужому
навчайтесь, і свого не цурайтесь.

Kiev Fortress Катя Сосна як ми вибрали не Кравчука, а Чорновола. Якби то, якби
те. Якби ми б всі розмовляли б українською, то путін би не прийшов визволяти.
Якби влада була б українською, то все було б добре. Знаєте, питання не у тому,
а у чому. Приказка є така. Якби бабуся була б дідусем, то у неї була б борода.
Розумієте. Оці осі нескінченні якби б то, якби б це - це ментальна пастка.
Нескінченна демагогія, яка ні до чого не веде. Кравчук вже помер, залиште його
у спокої. І крім того, у свій час він власне і став одним із засновників
незалежної України, підписавши угоди в Біловежській пущі. За це йому честь і
хвала. Також за те, що отримання незалежності пройшло безкровно. У свій час він
був на своєму місці, так, комуняка звичайно, але також і хитрий лис. В Україні,
якщо вам нагадати, не було кровопролиття, незалежність була отримана фактично
безкровно.
