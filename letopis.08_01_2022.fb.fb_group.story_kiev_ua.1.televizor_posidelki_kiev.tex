% vim: keymap=russian-jcukenwin
%%beginhead 
 
%%file 08_01_2022.fb.fb_group.story_kiev_ua.1.televizor_posidelki_kiev
%%parent 08_01_2022
 
%%url https://www.facebook.com/groups/story.kiev.ua/posts/1835913833272076
 
%%author_id fb_group.story_kiev_ua,atojev_konstantin.kiev
%%date 
 
%%tags kiev,televizor,tv
%%title ТЕЛЕВИЗИОННЫЕ ПОСИДЕЛКИ В КИЕВЕ
 
%%endhead 
 
\subsection{ТЕЛЕВИЗИОННЫЕ ПОСИДЕЛКИ В КИЕВЕ}
\label{sec:08_01_2022.fb.fb_group.story_kiev_ua.1.televizor_posidelki_kiev}
 
\Purl{https://www.facebook.com/groups/story.kiev.ua/posts/1835913833272076}
\ifcmt
 author_begin
   author_id fb_group.story_kiev_ua,atojev_konstantin.kiev
 author_end
\fi

ТЕЛЕВИЗИОННЫЕ ПОСИДЕЛКИ В КИЕВЕ

На рубеже 50-60 х годов телевизор в Киеве перестал быть признаком достатка и
стал доступен многим. До этого обычно ходили его смотреть к соседям. С покупкой
телевизора эти хождения прекращались. Соседи обижались, они привыкали к этим
почти ежедневным совместным просмотрам фильмов, матчей киевского
\enquote{Динамо}, КВНов. Так постепенно живое общение заменялось общением
виртуальным. Наш сосед работал на Киевском телевидении и время от времени
посвящал нас в тайны телевизионного закулисья. 

Именно у соседей я посмотрел «матч века» 1963 г., в честь столетия английского
футбола с Яшиным, Эйсебио, Копа, Лоу, Хенто, Чарльтоном, Гривзом. Потом пришел
черед чемпионата Европы по фигурному катанию в Москве 1965 г. Эмерих Данцер,
Ален Кальма, Ондрей Непела, Хана Машкова. Меня тогда влюбила в себя английская
танцевальная пара Диана Таулер и Бернард Форд, ставшая четвертой. Победили
тогда чехи – Ева и Павел Романовы. Но уже на следующий год англичане стали
чемпионами. Летом смотрели фестиваль песни из «Сопота». После того как в 1963
г. в нем победила Тамара Миансарова («Пусть всегда будет солнце») он был
довольно популярен. Но вот и мы, наконец, обзавелись телевизором - харьковской
«Березкой». Близился чемпионат мира 1966 г. по футболу и игнорировать этот факт
было невозможно, ведь его впервые должны были показывать по ТВ. И вот смотрим
первый фильм по нашей «Березке» - «Двое». 

\ii{08_01_2022.fb.fb_group.story_kiev_ua.1.televizor_posidelki_kiev.pic.1}

Сорокаминутная короткометражка, снятая в 1965 г., была дипломной работой
выпускника ВГИКа Михаила Богина, уволенного с \enquote{Мосфильма} за
\enquote{профнепригодность}. В главных ролях снялись Валентин Смирнитский и 19-ти
летняя Виктория Федорова, удивительно профессионально и органично сыгравшая
роль, в которой не было слов. Богин долго не мог найти студию, которая
согласилась бы поставить этот малобюджетный фильм-диплом. Наконец, он добрался
до Риги и, переночевав на вокзале, отправился на студию. И тут ему повезло, то
ли сыграло роль, что директор был два дня как назначен, то ли рекомендательное
письмо С. Герасимова. Фильм рассказывал о том, как студент-музыкант всячески
пытается познакомиться с понравившейся ему на улице девушкой, и когда, наконец,
ему это удается, выясняется, что она глухонемая. Федоровой даже пришлось
выучить язык глухонемых. Тогда, о проблемах людей с ограниченными возможностями
не то, что снимать фильмы, говорить было не принято. Это было свежим дуновением
эмпатии, прилетевшим из будущего. Фильм был снят в эстетике совершенно не
свойственной советскому кинематографу, неспроста его многие сравнивали с
\enquote{Мужчиной и женщиной} Лелюша. Ну и совсем молодые Федорова и Смирнитский были
великолепны. 

Ну с Федоровой давно все ясно - звезда, а вот когда сегодня смотришь на ее
партнера в этом фильме, думаешь - на Западе Смирнитский в то время мог бы стать
на один уровень с такими звездами как Ален Делон. Фильм сразу получил в 1965 г.
приз ФИПРЕССИ и Золотой приз в разделе короткометражных фильмов Московского
международного кинофестиваля. Причем настаивал на присуждении ему приза, не
кто-нибудь, а сам Микеланджело Антониони, бывший гостем киносмотра. Картину с
успехом показывали более чем в 80 странах мира, в том числе в США. Как говорил
М. Богин в одном из интервью: \enquote{Когда я закончил \enquote{Двое}, то
понял, что весь свой идеализм, хрупкость и застенчивость, сидевшие во мне, и
душевные переживания на эту тему, я вложил в этот фильм. Я думаю, что они
\enquote{просвечиваются}, излучаются из этого фильма, и многие люди ощущают
это...}.

И правда, фильм получился на удивление тонкий, лиричный и трогательный. Ну, и
почему-то в память врезалась фраза, произносимая героем при знакомстве, когда
девушка на все комплименты отвечает молчанием – он еще не знает, что она не
слышит: \enquote{Как говорил Сервантес, ничто не дается нам так дешево, и ценится так
дорого, как вежливость}. Почему она врезалась в память – не знаю, видимо, как
руководство по знакомству с глухонемыми девушками.

\ii{08_01_2022.fb.fb_group.story_kiev_ua.1.televizor_posidelki_kiev.cmt}
