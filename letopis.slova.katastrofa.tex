% vim: keymap=russian-jcukenwin
%%beginhead 
 
%%file slova.katastrofa
%%parent slova
 
%%url 
 
%%author 
%%author_id 
%%author_url 
 
%%tags 
%%title 
 
%%endhead 

%%%cit
%%%cit_head
%%%cit_pic
%%%cit_text
А после \emph{катастрофы} немецкой армии под Сталинградом в 1943 году работа над
\enquote{Планом Ост} и вовсе была свернута.  В то же время совершенно понятно, что в
\enquote{чистовом} виде планы немцев на Украину могли быть еще страшнее, чем их
рисовали нацисты в 1942 году. Это видно хотя бы по тому, что происходило на
оккупированных территориях.  Помимо зверств, грабежа и угона людей в рабство,
нацисты уже присылали своих колонистов - в Крыму и под Житомиром
организовывались экспериментальные колонии немцев, откуда предварительно
выгонялось местное население
%%%cit_comment
%%%cit_title
  \citTitle{22 июня - 80 лет нападения на СССР. Что немцы готовили для украинцев}, Максим Минин, strana.ua, 22.06.2021
%%%endcit

%%%cit
%%%cit_head
%%%cit_pic
%%%cit_text
При этом, даже не достигнув западного уровня всеобщего достатка, переняв лишь
модель, Россия под западным культурным влиянием заимствует и дегенеративные
модели поведения в обществе. И это неизбежно, так как при всей своей кажущейся
«инаковости» последние несколько столетий развивается в русле западной
философской мысли и культуры. Продолжение следования в этом русле грозит
неминуемой, предсказуемой и практически уже видимой \emph{катастрофой}
%%%cit_comment
%%%cit_title
\citTitle{Революция Духа – единственный путь спасения России}, 
Юрий Барбашов, voskhodinfo.su, 30.06.2021
%%%endcit

%%%cit
%%%cit_head
%%%cit_pic
%%%cit_text
Седьмая нерешаемая проблема России, отчасти родственная предыдущей, – это
советское варварство, уничтожившее здешний культурный слой.  Человек живёт в
современной России весело и беззаботно, пока не задумывается о том, каких
размеров разрушение случилось здесь в середине прошлого века. Что толку
напоминать о человеческих жертвах, сколько о них сказано, но – какое количество
городов, храмов, кладбищ, икон и даже просто библиотек, садов, обстановки в
домах – сгинуло быстро и бессмысленно. А ведь любому, кто имел счастье ездить
по Европе, понятно, что сохранение древнего стола и стула, везде и повсюду у
них стоящего, этой узкой улицы, этого мощного дуба, этого барочного дома,
собора – это позвоночник европейского величия и обаяния, то, на чём они
держатся и чем они нам до сих пор так милы, несмотря на всю новейшую политику.
И – возвращаясь – изучая всё то же самое, что было и тут, в каждом уездном
городе, и что пропало, не оставив среды для наследования, – можно только
оплакивать эту \emph{грандиозную катастрофу}
%%%cit_comment
%%%cit_title
\citTitle{Семь нерешаемых проблем России}, 
Дмитрий Ольшанский, vz.ru, 15.07.2021
%%%endcit
