% vim: keymap=russian-jcukenwin
%%beginhead 
 
%%file 09_02_2023.fb.suhorukova_nadia.mariupol.1.tot__kto_vibralsya_i.cmt
%%parent 09_02_2023.fb.suhorukova_nadia.mariupol.1.tot__kto_vibralsya_i
 
%%url 
 
%%author_id 
%%date 
 
%%tags 
%%title 
 
%%endhead 

\qqSecCmt

\iusr{Nataliia Movchan}

Я перечитываю всё утро и плачу... от бессилия... как сказать хоть какое-то
слово, которое на минуту-две ослабит вашу боль

\iusr{Инесса Алейник}

К сожалению о Мариуполе начинают забывать 😥 Знаю таких , кто вообще не знает
что город оккупирован и разрушен, хотя живут в Украине и считают себя
украинцами. Сирия, Турция, Днепр да это страшно сочувствую, соболезную,
понимаю, но Мариуполь, там люди погибали под завалами, от бомбежек, умирали
от ран, голода, были погребены заживо и их никто не спасал. Спасибо Вам,
Надежда о том что вы говорите, пишете, находите в себе силы не молчать об
этом. Нас Мариупольцев действительно никто не понимает и мы не такие как все.
Мы выжили в этом кошмаре для того что бы, быть живыми свидетелями этой подлой
и жестокой войны.

\begin{itemize} % {
\iusr{Татьяна Чугай}
\textbf{Инесса Алейник} 

мы никогда не поймём что Вы пережили, но мне, как и большинству украинцев,
всегда будет \enquote{болеть} Мариуполь!!! Наш ДОЛГ кричать об этом на весь мир!!! В
память о тех людях, которые остались там навсегда! Воимя нашего светлого
будущего! Чтобы нигде и никогда в мире больше не повторились эти зверства!!!

\iusr{Алла Алимова}
\textbf{Инесса Алейник} 100\%

\iusr{Елена Кадевич}
\textbf{Инесса Алейник} так и есть, дорогая моя коллега(((

\iusr{Nataliia Movchan}
\textbf{Инесса Алейник} 

нет, нет, с кем бы я не говорила о войне, Мариуполь звучит, и женщины начинают
плакать, не одна моя харьковская подруга постоянно делится голосами мирных, там
много о Мариуполе, нет слов, как страшно и было и есть

\iusr{Наталья Ушакова}
\textbf{Инесса Алейник} 

как не хочется это признавать, но по- моему, про Мариуполь забыли уже как
только весь этот ад начался.......

Но в Европе про Мариуполь знают!!! Никогда не забуду глаза волонтера-поляка,
который помогал нам в Польше. У него были такие глаза, когда услышал, что мы из
Мариуполя, и он смотрел на нас так, как будто перед ним воскресшие мертвецы😔.
Очень долго меня расспрашивал. Он посадил нас на поезд и ооооочень долго махал
нам вслед. Взрослый мужчина плакал как ребенок при этом🤗

\iusr{Тамара Белавина Белавина}
\textbf{Наталья Ушакова} 

Мы в Польше живём в селе Смогли попасть туда только 5апрел А там оказывается
весь март только о Мариуполе и показывали На нас смотрели как на оживших
мертвецов Тут думали что в Мариуполе уже всех нас убили Мы были тут первые из
Мариуполя Остальные из мирных городов К нам даже приезжали какие то чины ихние
Опрашивали как "всідків війни" До нас тут таких не было

\iusr{Наталья Ушакова}
\textbf{Тамара Белавина Белавина} Мы с дочкой ехали проездом через Польшу второго апреля......

\iusr{Инна Ивашутина}
\textbf{Наталья Ушакова} 

ні, ми не забули про Маріуполь! Це - біль нестерпний, жах! Про це не можна
забути ніколи!!! Я про це буду пам'ятати завжди завдяки вам-свідкам і людям,
хто вижив у цьому пеклі... Обіймаю вас міцно, мої любі українці! Хорони вас
Бог!!!! Тримаймося!!!

\end{itemize} % }

\iusr{Інна Солодчук}

Читаю і реву...😭😭😭 Такий біль, такий страх, таке зло... Господи, за що? Чому? Навіщо?😭😭😭

\iusr{Людмила Парфилова}

Ми обов'язково вернемося у наш рідний український Маріуполь. Я в це вірю. Дякую вам.

\iusr{Виктория Костоглодова}

Ми обов'язково вернемося у наш рідний український Маріуполь. Я в це вірю. Дякую вам. ♥️♥️♥️🇺🇦🇺🇦🇺🇦

\iusr{Irina Linnik}

Как же хочется домой в родной и любимый Мариуполь

\iusr{Assol Shulzhenko}

Дом похож на тот, в котором жила тетя, на Азовстальской. Очень больно смотреть
на такие фото

\iusr{Оксана Грицай}
😭😭😭

\iusr{Олена Шувалова}

Смотрю фотографии нашего родного города до войны в одной группе, туда постоянно
добавляются новые и новые участники, у всех одинаковые чувства, одна большая
боль, мне иногда представляется, как я еду по городу на автобусе и смотрю в
окно, рассматриваю свой город после возвращения, но потом вспоминаю, что от
города мало осталось: от центра, от нашего Левого берега практически ничего не
осталось, и останавливаю себя в своей виртуальной поездке, потому что осознание
того, что города нет, так и не пришло😭

\begin{itemize} % {
\iusr{Наталья Чмиль}
\textbf{Олена Шувалова}

\iusr{Татьяна Прудникова}
\textbf{Олена Шувалова} 

Я добавилась в группу Мариуполь довоенный, я никогда не была ранее в Мариуполе,
фото в этой группе смотрю с болью, очень больно за Мариуполь, очень жаль людей,
боль и ненависть.... Смотреть на фото и понимать каким прекрасным был город и
что с ним сделали .....

\iusr{Олена Шувалова}
\textbf{Татьяна Прудникова}, я об этой группе и говорила
\end{itemize} % }

\iusr{Алла Алимова}
💔

\iusr{Alena Hriy}

Все так і є...

\enquote{Найулюбленіший пост} він про мене точно!

\iusr{Юлия Рагозина}

Боль, которая всё время со мной, постоянно плачу...., а когда вижу фотографии
родного города, кажется что хожу по любимым улицам !!!! Я вся ещё там, хотя
будет год, а очень, очень больно....

\iusr{Natali KaraDeniz}

Я из Симферополя, стадия принятия того что у меня отняли так и не наступила.
Надежда на возврат очень призрачная. У вас ещё страшнее. 🙁

\iusr{Наталия Мороз}

Сколько боли и страданий!!!

\iusr{Виктория Аверченко}

Мы сидим в подвале. В бомбоубежище уже нет мест. С нами двое детей. Дом кричит,
визжит, содрогается, корчится от страшных ударов. Вместе с ним трясусь от ужаса
я и кричу молитвы, всё которые знаю. Я не кричу, я ору, чтобы в этом грохоте
самой слышать себя, чтобы Бог меня услышал. Я сижу спиной к тонкой стене и одна
мысль стучит в висках: \enquote{Только не голова! Только чтобы не снесло голову!}.
Почему? Не знаю. Сейчас вспоминаю и опять в груди всё замирает от ужаса. Я уже
почти не смотрю фотографии нашего города сейчас. Только иногда заглядываю. Как
будто прикрыла дверь и оставила маленькую щёлочку. Смотрю, как посторонний
наблюдатель. Я до сих пор не верю, или не хочу верить, что моего, нашего города
больше нет.

\begin{itemize} % {
\iusr{Алла Алимова}
\textbf{Виктория Аверченко} 💔😪
\end{itemize} % }

\ifcmt
  igc https://i.paste.pics/a730708a29337f443683c00506d45798.png
  @width 0.2
\fi

\iusr{Ирина Богатчук}

Нет слов..............

\iusr{Natali Korovanenkova}

Я сейчас в Никополе, на той стороне @рки, по городу каждый день прилеты. У меня
единственная мысль \enquote{второго марш броска я не вынесу}... и снится один и тот же
сон, как я выгребаю себя с под завалов, стою держусь за провода под потолком и
смотрю вниз в дыру с7 этажа..

\iusr{Вера Цыганенко}

\ifcmt
  igc https://i.paste.pics/3887ea622617f5049b6273f7d77e26ce.png
  @width 0.1
\fi

\iusr{Наталья Никулина}

Душа як цей дім на фото, уся в ранах, які не залатати ...

\iusr{Татьяна Шалашная}

Это те слова и мысли которые у всех выехавших из города. У меня половина сердца
боль жуткая, не проходящая, а половина сердца ненависть. Наверное тоже будет не
проходящая. 😪🇺🇦🙏🙏🙏

\iusr{Виктория Владимировна Трипольская}

А мы кто остался в Мариуполе, остались потому что не было транспорта, не было
возможности так как у меня лежачая мама,красный крест не приехал, в частном
доме не смогла её уберечь прилёт в окно. вот и все,.. а теперь находясь в городе
с дочерью и внуком (4 года), живём на мою пенсию по группе, везде бешенные
очереди и боль когда видишь что нет района стадиона, Московской, три дома на
Комсомольском бульваре и мой дом частный на проспекте Ленинградский который
разбит и никто его не ремонтирует а средств на восстановление нет вот и все
....сплошная боль и ужасы во сне

\begin{itemize} % {
\iusr{Лилия Савицкая}
\textbf{Виктория Владимировна Трипольская} я тоже в Мариуполе, но душа в 23 февраля

\iusr{Олена Махиня}
\textbf{Виктория Владимировна Трипольская} 

да, а кто то хвалит супермаркеты и совести хватает, снимается в роликах про
восстановление, лучше побольше тех, кто говорит правду

\iusr{Людмила Зевако}
\textbf{Виктория Владимировна Трипольская} 

забыли добавить, что город вроде бы родной, но чужой... Когда бомбили, было
только одно-выжить. Вода, дрова, приготовление еды, походы к родным - потому что
ту сторону бомбили( идёшь и думаешь, дом и родные целы? И эти мысли сводили с
ума, пока дойдешь.) А сейчас... Просто потерялась в этой жизни и своём родном
городе.

\iusr{Victoria Podkaura}
\textbf{Лилия Савицкая} 

Лилия, подскажите, пожалуйста, а вы с VPN сюда заходите? А то у меня подруга в
Мариуполе не может 3 месяца уже в Фейсбук зайти...

\iusr{Марина Пушкина}
\textbf{Виктория Владимировна Трипольская} 

А мы кто уехал, очень вас любим и не капельки не осуждаем, а всегда думаем как
вам там холодно, голодно да, нам душевно очень тяжело но мы сыты и в тепле, а вы
продолжаете жить в этом аду и нам вас жалко ещё больше, не все люди могут
уехать, держитесь там без нас до времени!!! Любим !!!

\iusr{Наталья Евчук}
\textbf{Victoria Podkaura} надо заходить через VPN

\iusr{Лилия Савицкая}
\textbf{Victoria Podkaura} да, через vpn, бесплатные и ключик и глобус

\ifcmt
  igc https://scontent-fra3-1.xx.fbcdn.net/v/t39.30808-6/329224948_1633949523741186_5240831088576560188_n.jpg?_nc_cat=105&ccb=1-7&_nc_sid=dbeb18&_nc_ohc=_w7An2XRZQ8AX-Ac5Io&_nc_ht=scontent-fra3-1.xx&oh=00_AfBnaIe9Yyoil3s6C5wiKSsd63MlC7Iu7Jf2NqhBYJ1gBA&oe=63F8FE2E
  @width 0.4
\fi

\iusr{Лилия Савицкая}
\textbf{Наталья Евчук} вернее, не глобус, а Сатурн

\iusr{Лилия Савицкая}
\textbf{Марина Пушкина} 

хочется поклониться Вам до земли, за Вашу поддержку. Я каждый день плачу, не живу-существую

\end{itemize} % }

\iusr{Оксана Спивак}

Очень, очень хочется в свой город Мариуполь!💖🙌🙏✌

\iusr{Ольга Нарбут}

В моем кругу знакомых, нет человека, который не знал бы, что произошло в
Мариуполе, о таком не возможно забыть. Мариуполь в Украине - это символ
мужества. То что вы пишите, это очень важно, для следующих поколений.

\iusr{Валентина Самарина}

Моя жизнь разделилась на до и после, днем вроде как то забывается а ночь....
постоянно снится ужас, у меня в доме небылр подвала вернее он был но у меня
клаустрофобия. Бомбили постоянно прятались просто на этажах или у соседей там 3
стены. Да стены спасали только от осколков а от прямых попаданий нет оставалось
только молится прошло уже 7 мес как я выехала из Мариуполя но боль не уходит а
тоска все сильней до сих пор перед глазами поток людей черных перемотаных кто с
пакетиком кто с тачкой идут в сторону молокозавода там обещают эвакуацию люди
как зомби просто поток я пытаюсь это забыть но ...может со временем станет
легче но никонда я этого прстить не смогу

\iusr{Анна Малеванная}

Мы остались так как несмогли выехать с нами больной но душа разорвана на части
сердце разбито но жыть надо ради того чтобы увидеться с детками которых
разбросало по свету надежда умирает последней

\iusr{Людмила Озерская Караберова}

Все сказано очень точно, высказано то что я не смогла сказать и многим нас не
понять, особенно тем беженцам которые ввехали со Львова и пытались приравнятся
к нам мариупольцам.

\begin{itemize} % {
\iusr{Ольга Адаменко}
\textbf{Людмила Озерская Караберова} 

Да, я бы сказала что это особая каста людей. Вот я нахожусь в Ирландии, в
Дублине, мы живем в огромном спортзале нас 6о человек. Я - одна из
Мариуполя(приехала сюда в сентябре), все остальные из разных
городов-Харьков, Днепр, Одесса, Киев и мн. другие. Но среди нас есть парочка-пожилые
, они со Львовской обл. Вот они домики свои сдают а сами дернули в Ирл. Те ещё
старикашки!!!!

\end{itemize} % }

\iusr{Lludmila Ekzarkhova}

Дужэ боляче. Я тэж з много любiмого, та рiдного до болю Марiуполя. Марiуполь
чэкае ЗСУ. Всэ Будэ Украiна.

\iusr{Галина Лосева}
\textbf{Lludmila Ekzarkhova} так, ми чекаємо ЗСУ. І це буде. Без сумніву. Маріуполь це Україна! Повернемось!

\iusr{Надежда Брагуц}

\ifcmt
  igc https://i.paste.pics/3887ea622617f5049b6273f7d77e26ce.png
  @width 0.1
\fi

\iusr{Оксана Григорьева}

Мы выехали, пол года просто сходила с ума, хотела вернуться. Вернулась, сердце
рвало на части от вида сгоревшей квартиры и разбитого города. Уехала, болит.....

\iusr{Оксана зозуля}

Очень сильно, то что в душе у меня и каждого

\iusr{Игорь Ганенко}

Все сказаное - ПРАВДА

\iusr{Лидия Репина}

Очень хочется домой!!! А до ма нет. И города нет практически. И многих
друзей, знакомых соседей нет! И понимание того, что ничего не вернуть и не будет
так как прежде никогда, просто выворачивает наизнанку. Безумно скучаю, хожу на
море в другой стране, чтобы хоть что то похожее увидеть. Но и море не то, и запах
не тот, и чайки кричат по другому. Эта боль потери останется в нас до
конца.😔💔💔💔

\begin{itemize} % {
\iusr{Анжела Белоконь}
\textbf{Лидия Репина} про море четко в ТОЧКУ, такие же ощущения от ЧУЖОГО моря
\end{itemize} % }

\iusr{Елена Смирнова}

Те же чувства по истечению 9 лет бегства из Донецка. Хочется вернуться домой.
Но ты понимаешь, что вернутся в довоенный город невозможно. А в сегодняшний -
самоубийство. А год назад снова пришлось уехать, потому что \enquote{летало} прямо над
головой и \enquote{падало} прямо перед глазами, разрушая все вокруг...

\iusr{Tatiana Polievets}

Конечно, это поймут только те, кто такое пережил или их близкие, которые
переживали за своих родных, которые там находились. Чьи дома, города, прошлая
жизнь разрушена, не только в Мариуполе. А многим людям вообще фиолетово, они
даже элементарно посочувствовать не в состоянии, не то, чтобы понять. Живут
своей жизнью да и всё, и мало что для них изменилось

\iusr{Оксана Рыхлик}

Всё именно так. Каждая буковка...

\iusr{Лилия Шевченко}

Все, что написано выше - каждое слово, точно ! В душу, в сердце.

\iusr{Светлана Губина}

Мариуполь разбит, там живут настоящие люди, которые любят Мариуполь, поверьте
даже в разбитых квартирах, без тепла, без света и газа, у людей всё сохранено
но они покинули Мариуполь, почему? Поехали в Европу в чужие комнаты, на
социалку, почему.....?

\begin{itemize} % {
\iusr{Olga Piven}
\textbf{Светлана Губина} Потому что покориться оккупантам ниже их достоинства

\iusr{Наталья Шиян}
\textbf{Светлана Губина} 

Значит так его любили, а город живой, зализал свои раны и готов к новой жизни! И
не важно как я тут власть, нужно жить дальше, а так скулить и скитаться можно
бесконечно! Жизнь она продолжается, и так как было уже не будет, живите здесь и
сейчас!

\iusr{Татьяна Сальникова}
\textbf{Светлана Губина} Почему!?! Вы серьёзно?! Потому что мы не хотим жить с убийцами и окупантами!

\iusr{Svitlana Orlova}
\textbf{Светлана Губина} 

как жить рядом с теми, кто убил твоих родных, кто разрушил твой дом, кто
уничтожил твою жизнь, твои мечты растоптал? Оккупанстский хлеб глотку не дерёт?
И не надо говорить, что мы не любим Мариуполь, мы любим наш родной украинский
Мариуполь! И мы обязательно в него вернёмся!

\iusr{Olga Piven}
\textbf{Наталья Шиян} Вот сами и живите там сейчас, где богуют чеченцы и буряты, а коренные мариупольцы - третий сорт.
Наш Мариуполь - европейский город, а то что там сейчас, трудно назвать цивилизацией.

\iusr{Ann Ann}
\textbf{Светлана Губина},

потому что мне физически плохо находиться даже рядом с этими пидарасами
оккупантами, меня физически тошнит при виде их, понятно? у меня нет стокгольмского
синдрома и я чётко знаю кто нас пришёл убивать и чётко знаю, кто поломал нашу
жизнь

\iusr{Ludmila Karpova}
\textbf{Наталья Шиян} с кем жить. Наступит время мариуполь освободится. А жить с теми, кто 24 часа бомбил город.

\end{itemize} % }

\iusr{Elena Zholudeva}

Каждое слово в точку.

\iusr{Max Mara}

😭😭😭

\iusr{Svetlana Prigorodova}

У моих родных ничего в Мариуполе, кроме памяти, не осталось! Дом был разбит,
сейчас снесли... Остался страх, звук разрывов и грохот орудий....

\iusr{Надя Бондаренко}

Я знаю что Бог услышит наши молитвы и мы верь нёмся домой аминь

\iusr{Надя Бондаренко}

\ifcmt
  igc https://i.paste.pics/be0f5a0a1c43803605c8eb2edec8cdfd.png
  @width 0.2
\fi

\iusr{Marina Moroz}

Моє місто, моє море.. Я, навіть, знову зі свого 8 поверху хочу бачити труби з
чорним димом працюючої \enquote{Азовсталі}..

\begin{itemize} % {
\iusr{Nadia Sukhorukova}
\textbf{Marina Moroz} і я теж хочу побачити місто як до вторгнення окупантів

\iusr{Marina Moroz}
\textbf{Nadia Sukhorukova} Те, кто утверждает, что \enquote{жизнь налаживается} похожи на жертву, которая поёт песни и дифирамбы своему насильнику.

\iusr{Nadia Sukhorukova}
\textbf{Marina Moroz} согласна. Они не похожи, они такие и есть. Их город убили, сравняли с землёй, вокруг них могилы и разруха, по их земле ходят оккупанты и устанавливают свои порядки, а они рассказывают, что им все равно какая власть. Это даже не предательство. Это диагноз. Украину продолжают убивать, а этим пофиг, что они поддерживают убийц. Противно.
\end{itemize} % }

\iusr{Ирина Яксманицкая}

Живём только надеждой, что вернёмся в свой родной украинский Мариуполь. Даже
разрушенный, уничтоженный, но свой, родной, любимый. Будем восстанавливать его,
строить, благоустраивать. Это наш город. Будем сильными. Держимося, земляче, ще
трохи почекаємо, і назад додому.

\begin{itemize} % {
\iusr{Nadia Sukhorukova}
\textbf{Ирина Яксманицкая} спасибо! Держимся!
\end{itemize} % }

\iusr{Виктория Костоглодова}

Сегодня общалась с девчонкой с Мариуполя, она говорит город стал чужим и
холодным. Понаехали всякие строители из мордоворотии, очень много таджиков в
городе. Цены космос на всё, работы толком нет.

\begin{itemize} % {
\iusr{Nadia Sukhorukova}
\textbf{Виктория Костоглодова} 

Виктория, так многие адекватные люди рассказывают. Но есть и другие. Я двух
особей уже заблокировала, но их точку зрения можно прочитать чуть выше. Они
пишут, что им все равно при какой власти жить. Можно и с убийцами. Это для них
не принципиально.

\iusr{Виктория Костоглодова}
\textbf{Nadia Sukhorukova}, 

им не всё равно при ком жить, Танюша в прямую многим говорит, что они пришли в
чужую жизнь в чужой город и забрали у нас всё самое ценное 😭 Я её очень прошу
быть очень аккуратный

\iusr{Nadia Sukhorukova}
\textbf{Виктория Костоглодова} привет передавайте Танюше! Мы обязательно увидимся
\end{itemize} % }

\iusr{Елена Смирнова}

Насколько методично расстрелян хотя бы этот дом на снимке. И насколько надо
быть циничными, чтобы утверждать, что те, кто расстреливал, \enquote{освобождали}...
Кого?От кого - чего? А теперь убитый город отстроят? Для кого? Для тех, кого
убили, а потом сгребли бульдозером, вместе с руинами? А вы знаете, что
Мариупольский драмтеатр сравняли с землей именно в международный День театра?
Это я пишу для тех, кому все равно какая власть, а город себе \enquote{живет и
обновляется}. Он \enquote{обновляется} для оккупантов.

\begin{itemize} % {
\iusr{Irina Dernovaya}
\textbf{Елена Смирнова} 

всё верно, и как прежде город не будет, отдельное Спасибо, тем, кто бросил его,
смылся, как говорится вовремя, почему за это ни слова... За Драм тоже знаем не
понаслышке, за людей, которых нет.... Наш Дом здесь, и кто сюда понаехал это их
сложности. Это Наш Город, наша Боль, и никто другой не поймёт этого, каждый в
этой Войне переживает так сказать 'своё', и сколько бы разногласий не было, мы,
из Мариуполя одна семья....

\iusr{Sasha Sasha}
\textbf{Елена Смирнова} 

у них один циничный ответ на все свои преступления - это сделали ВСУ. Театр,
дома уничтожил Азов, а они только \enquote{освобождали} и \enquote{спасали}...

\iusr{Елена Смирнова}
\textbf{Sasha Sasha} та да... \enquote{Азов} - с самолетов... Я тоже самое слышала по поводу разрушений в Ирпене, да везде. \enquote{Спасители}, епть...
\end{itemize} % }

\iusr{Виктория Костоглодова}

Но они с семьёй ждут нас с леопардиками ) говорит что людей заставляют получать
их недо паспорта 🤬

\iusr{Ірина Нуралієва}

Спасибо за эти слова.

Это то, что зачастую невозможно произнести вслух, потому что оно всё ещё свежо
и болит.

\iusr{Юрий Бабенко}

росія - недоімперія терористів має бути знищена!!!

\iusr{Вячеслав Пархоменко}

\ifcmt
  igc https://i.paste.pics/5a616c3fffca72346974b50ff365dc77.png
  @width 0.1
\fi

\iusr{Нина Комкова}

Мы обязательно вернём в свой родной город, по другому не может быть.

До встречи всем и городу тоже

\iusr{Галина Гордиенко}

\ifcmt
  igc https://i.paste.pics/e1396c6e6442536a64f0d7b3f02d038f.png
  @width 0.2
\fi

\iusr{Юлия Юлия}

Я сейчас в Киеве. И местные постоянно говорят, что мы одинаковые. Что их также
бомбили, они голодали. Сначала я пыталась спорить. Сейчас молчу. Им не
докажешь, они считают себя самыми пострадавшими и постоянно упрекают помощью
для ВПО \enquote{вам помогают, а нам нет}, забывая, что у них есть свой дом, свои вещи,
своя кровать....

\begin{itemize} % {
\iusr{Oksana Cherdantseva}

Не звертайте уваги, нам в Києві від початку війни теж страшно, але це не
зрівняється з вашим болем. Не всі в Києві проти ВПО. Знаю багато людей, які
допомогали і допомогають Маріупольцям. Коли в квітні у місцевих групах з'явилось
повідомлення, що везуть групу з дітей з Маріуполя, які залишились без
батьків, будуть везти за кордон, але їм потрібен відпочинок кілька днів і
догляд(писали, що 70 з чимось дітей), ви не уявляєте скільки було людей, які
хотіли допомогти взяти до себе кількох діток. Повідомлення гуляло по групах з
тиждень, і виявилось спамом. Тримайтеся🇺🇦

\end{itemize} % }

\iusr{Kylish Igor}

Марік з нами

\iusr{Надя Бондаренко}

Я люблю свой город Мариуполь и скучаю по нём

\ifcmt
  igc https://i.paste.pics/6b348b6383e2c77050b2cfee8267f9fb.png
  @width 0.2
\fi

\iusr{Надя Бондаренко}

\ifcmt
  igc https://i.paste.pics/1206029adb2bf7b478e583e949b156b7.png
  @width 0.2
\fi

\iusr{Татьяна Коровина}

Боль

\iusr{Марина Кучеренко}

Я тоже молчу, когда говорят мне, что вы из Мариуполя, а мы же тоже из Украины и
нам тоже было страшно: сирены гудят, а мы боялись и мы тоже выехали...

\iusr{Олег Гусак}

Сколько комментарий... Очень много...Так дуже тяжко. Одні сльози..Але жодного
коментаря не має про те, що багато з нас винні перед містом, бо саме через своє
мовчання, \enquote{колебание} в ту, чи іншу сторону, ми допустили проведення
демонстрацій під червоними та смугастими прапорами \enquote{імперії} зла, закривали
очі на проведення референдумів, не підганяли палицями керівників та
\enquote{прикормлених} голів кишенькових профкомів своїх підприємств, що мовчи
спостерігали та підігрували пропаганді \enquote{руського миру} ( а ро суті
фашизму)...І так було в кожному регіоні та місті Донбасу. ...А тепер плачемо...
\enquote{Це не ми... а що ми могли зробити!?} Тепер хоть дійшло, що треба було робити?
Те саме зараз відбувається з людьми і в оркостані. Там теж так кажуть...

\begin{itemize} % {
\iusr{Андрей Моисеенков}
\textbf{Олег Гусак} 

Яку дурню ти написав. Чи зупинило війну в Чернігівщині, чи Сумщині, те що вони
розмовляють українською. Таки недоумки, як пан Гусак, сіють розбрат серед
українців! Я з сім'єю в Дніпрі, теж чув від поважної пані, що вона не буде
пускать біженців за шалені гроші, по міркам коштовності оренди квартир в
Маріуполі, до себе. Бо, якщо б вони розмовляли українською, то путін злякався
б і не напав на Україну. Яке ж дурне ! І таких багато, на жаль. До речі,
погрози їдуть до всього Заходу, не лякає рашистів той факт, що там мало хто
знає російську!

\iusr{Любовь Мудрик}
\textbf{Олег Гусак} 

стесняюсь спросить, а вы с Мариуполя? И вы все видели с самого начала и до
конца трагедию Мариуполя и Мариупольцев, которые сидели в холодных подвалах,
без тепла, воды, света, связи, без надежды на жизнь ! Вы их теперь учите на
какой мови говорить. Вы власть научите любить и ценить людей, которая их
обрекла на смерть ! Которая все знала, и бросила людей на смерть !

\iusr{Julia Vasil'evna}
\textbf{Олег Гусак} такая чушь в каждом вашем слове

\iusr{Kateryna Korolkova}
\textbf{Олег Гусак} яке ж ти дурне 🤯

\end{itemize} % }

\iusr{Наталья Годун}

\ifcmt
  igc https://i.paste.pics/5a616c3fffca72346974b50ff365dc77.png
  @width 0.1
\fi

\iusr{Нина Дегтярева}

\ifcmt
  igc https://scontent-fra3-1.xx.fbcdn.net/v/t39.1997-6/64553989_2575276562483107_8940181216414400512_n.png?stp=dst-png_s168x128&_nc_cat=1&ccb=1-7&_nc_sid=ac3552&_nc_ohc=HlVGuS2Vi1oAX-7G-bC&_nc_ht=scontent-fra3-1.xx&oh=00_AfBuDx7h2x0HQjZEWInSHG0lzGiNI8-CIKSYNLSzGWq3-g&oe=63F8CACA
\fi

\iusr{Елена Муравская}

Искренне, проникновенно, точно и ненадуманно, как зеркало тоскующей души...
Самое смешное, что только сейчас понимаешь как любил Марик и продолжаешь любить
и восхищаться Мариупольцами, где бы они сейчас не были... Мы все от туда... мы
особенные...

\iusr{Вячеслав Приходченко}

Таким в начале войны был мой дом. Теперь его снесли совсем. А сегодня мой
аккаунт \enquote{забанили} на 27 дней за неприемлимые политические высказывания.

\ifcmt
  igc https://scontent-fra3-1.xx.fbcdn.net/v/t39.30808-6/330853385_2578356628972521_5201132732504076881_n.jpg?_nc_cat=105&ccb=1-7&_nc_sid=dbeb18&_nc_ohc=5V1jc_lCWoIAX8ROp48&_nc_ht=scontent-fra3-1.xx&oh=00_AfD8OBrr_uHB5W1wcyUlRu9UgRT20cCFM3aCvR7EKbanBg&oe=63F94675
	@width 0.4
\fi

\begin{itemize} % {
\iusr{Nadia Sukhorukova}
\textbf{Вячеслав Приходченко} Господи! Я даже не узнаю где это...

\iusr{Вячеслав Приходченко}
\textbf{Nadia Sukhorukova} Киевская, 92, Восточный мк.р. В пристройке поликлиника и детская библиотека.
\end{itemize} % }

\iusr{Julia Vasil'evna}

Мы уже никогда не сможем вернуться в Мариуполь. Нет больше Мариуполя.

Мы можем вернуться в город, но не в Мариуполь. Есть такое понятие ТОЧКА
НЕВОЗВРАТА. Она наступила где-то в конце апреля

\begin{itemize} % {
\iusr{Elena Barteneva}
\textbf{Юлия Васильевна} 100\% точка неозврата

\iusr{Alena Andrianova}
\textbf{Julia Vasil'evna} це так, нікуди їхати, але душа болить за рідне Місто Маріуполь
\end{itemize} % }

\iusr{Olichka Svoja}

Если мы улыбаемся, это не значит, что мы прежние - наедине с самим собой это
дикая боль и жуткие воспоминания

\iusr{Лина Колесниченко}

Момент принятия никогда не настанет...

Ми просто будем жити, посміхатись, кормити котиків і висаджувати квіти...

Але, якщо наша помста не розчавить тих виродків - нам буде важче це робити...

\iusr{Ксения Антропова}

\ifcmt
  igc https://i.paste.pics/2a00dcc14f14b2e0677e4e0ecf9a8aa7.png
	@width 0.1
\fi

\iusr{Людмила Бабич}

Да, все ещё болит и душа и сердце. Никак не могу смириться, и не смогу
забыть.😭

\iusr{Loredana Picierro}

Mamma come fa male. Mariupol era gemmellata con la mia citta taranto si giocava
a calcio tutto era normale. Ora orrore solo orrore mi sento vicina a voi tanto
con tutto il cuore

\iusr{Ilona Borovlova}

Про або вижити, або одразу насмерть - просто 100\%

\iusr{Людмила Нагорна}

Яка нестерпна біль, чи буде за це хтось покараний?

\iusr{Elvira Zbanca}

Очень больно за всех, кого убила эта безумная война, кого покалечила физически, сломала эмоционально.

Но страшнее ещё от того, сколько сторонников этого безумия радуются этому
безмерному горю. У сожалению, российская пропаганда уничтожила все человеческое
в своих слушателях. Страшно, что все это горе только усиливает безумие
агрессора, что столько людей продолжают погибать.  Мариуполь, весь мир с тобой.
Горе объединило психически нормальных людей или просто людей, но ожесточило
захватнические амбиции агрессора.

\iusr{Марина Сергийчук}

Не надо называть Мариуполь мертвым, он выжил и живет, и те кто выжил здесь
продолжают жить в своем родном городе, а кто скучает пусть возвращаются домой!

\begin{itemize} % {
\iusr{Татьяна Прудникова}
\textbf{Марина Сергийчук} 

Я видела в других видео в Мариуполе сносят дома, много домов..... Это больно,
хоть я не из Мариуполя, сносят вместе с домов чью то жизнь. А куда же людям
возвращаться, если их квартиры снесены!?..... Боль сильная....

\iusr{Марина Сергийчук}
\textbf{Татьяна Прудникова} 

Дома сносят, но и многие ремонтируют и строят. Поверьте нам тоже очень больно
смотреть на наш родной город в таком виде. Но скажу люди на улице не живут и не
готовят возле подъезда на костре, как это было сразу во время войны!

\end{itemize} % }

\iusr{Лариса Алексеевна}

Уважаемые \enquote{страдающие}, выехавшие из Мариуполя! Сейчас, прямо, модным стало
выражение \enquote{Рашка строит дома на костях, на земле пропитанной кровью и т д}.
Хочу спросить, а когда придёт Украина (если придёт) она, что будет строить на
чём-то другом, на какой-то другой земле? Или вообще строить не будет?

\begin{itemize} % {
\iusr{Натали Козуб}
\textbf{Лариса Алексеевна} Украина пройдёт, даже не сомневайся !

\iusr{Лариса Алексеевна}
\textbf{Натали Козуб} вы сделали вид, что не поняли вопрос?

\iusr{Натали Козуб}
\textbf{Лариса Алексеевна}, я всё поняла ! Сейчас в Мариуполе рашка будет закатывать кости в асфальт. Вас это устраивает?!

\iusr{Лариса Алексеевна}
\textbf{Натали Козуб} прочитайте внимательно вопрос (прежде, чем отвечать)

\iusr{Натали Козуб}
\textbf{Лариса Алексеевна}, я умею читать и думать. А вот у вас наблюдаю отсутствие мозга ))

\iusr{Артем Климов}
\textbf{Лариса Алексеевна} 

Вам персонально для информации: в Украине прежде чем приступить к
восстановительным работам после разрушений, аварийно-спасательные отряды
специального назначения ГУ МЧС Украины занимаются поиском и извлечением тел
погибших из-под завалов, поиски не прекращаются до последнего погибшего. Вместе
со спасателями под завалами ищут с тренированными спасательно-поисковые
собаками. А в Мариуполе, оккупанты вместо поиска и излечения спецсредствами
погибших, перемалывают их шредерами на полигонах - на Мухина, в Сартане и в
Тяжмаше, получая на выходе ФБСы с людским духом.

\iusr{Nadia Sukhorukova}
\textbf{Артем Климов} 

все так. Багато хто загинув, навіть могил не має. Їх особи не идентифіковані.
Їх потрібно поховати. Але спочатку дізнатися - хто ці люди. Вони мають право на
повернення їм ім'я, нехай і після смерті. Рашистам пофіг на людей. Вони
спеціально все руйнують. Щоб приховати свої злочини. Спасибі вам за відповідь
цієї дамі. А вона йде в бан. Таких як вона тут не буде.

\end{itemize} % }

\iusr{Вера Ольчедаевская}

Слов нет только слезы

