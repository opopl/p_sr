% vim: keymap=russian-jcukenwin
%%beginhead 
 
%%file letters
%%parent body
 
%%url 
 
%%author_id 
%%date 
 
%%tags 
%%title 
 
%%endhead 

\ii{letters.27_01_2023}
\ii{letters.06_02_2023}
\ii{letters.14_02_2023}
\ii{letters.18_02_2023}

%https://www.facebook.com/permalink.php?story_fbid=pfbid0sbMsAukawJAnZeA8fdWBvWhfxh4pusncRbwfHWN2sQYRxNnjbGn4FkQ5fDmWm4ajl&id=100087641497337

Надіє, дуже важлива книга! Мені, якщо чесно, як людині, що на щастя не бачила
наочно всього цього пекла, хоча я багато читав і читаю постів про Маріуполь, і
я відчуваю ваш біль, який так, відчуваєш, навіть якщо всього цього не бачив,
важко коментувати, оскільки подумки питаєш себе, що і як варто казати людям, що
все це пережили. Але... я мушу зізнатись в такому. Ще до війни я бачив вже
приблизно все те, що було в Маріуполі... у фільмі. Польському фільмі. Фільм
називається Місто 44 (Город 44). Страшний фільм насправді, і я в той момент
зовсім не думав, що всі ті жахіття, що були показані у тому фільмі, колись
будуть повторені вже в Україні. І цей фільм про трагедію Варшавського
повстання. Так, трагедія такого ж масштабу...  Хоча інша країна, інша історична
епоха, інші конкретні обставини. Так, у 1944 році, коли Варшава була під
німецькою окупацією, поляки підняли повстання проти нацистів, яке було в
результаті надзвичайно жорстоко придушене. Надзвичайно жорстоко. Місто було в
вогні, зруйновано, близько 100 000 поляків було в результаті вбито. Поляки
дотепер пам'ятають про це. Але... Варшава... якщо ходиш вулицями цього міста -
красивого, гарного міста... не відчуваєш, що тут колись коїлось... Так
заведено, що Міста мають внутрішню, так, внутрішню велику Силу... яка все
подолає і переможе! І так! В кінці фільму - якщо знайдете час колись подивитись
- дуже зворушливо. По-перше... (якщо подивититесь, побачите самі... - все
розказувати наперед напевне не треба... )  І по друге, що стосується власне
Міста Варшава. Кінець фільму... Палаюче знищене місто зникає... І постає
Варшава наших днів... І оцей перетворення із палаючого міста у сучасне мирне
усміхнене європейське місто колись мене надзвичайно зворушив, аж до сліз... І
так само Маріуполь. Місто залікує свої рани, так само як і маріупольці. І
колись, через десятиліття те, що відбувалось, залишиться лише у книжках та у
відео та фільмах, та пам'яті народу. Але приклад поляків дуже показовий. Вже
скільки років пройшло, а вони зняли цей фільм зовсім недавно, хоча стільки
років вже пройшло! Але вони пам'ятають, так, і дотепер, і будуть пам'ятати
завжди! 

от ви кажете, що зоряні війни, то є нудно... знаєте. А я почну здалека... з
Природи. Ось природа... виходите на вулицю... вдихаєте свіже повітря...
радієте... Сонечко світить... чи не задумувались ви про те, як все прекрасно,
взаємопов'язано між собою. Сонце світить - дощ ллє. Земля всмоктує в себе
вологу, і дає життєдайні сили, щоби прекрасна квіточка - скажімо, десь у Києві
на Співочому Полі - або у бабусі в селі -  проросла. Потім з'являються люди,
висаджують квіти, створюють парки, як от парк Кейкенхов в Голландії, або ж
дитячий парк Веселка в Маріуполі - справжня міні Голандія у прекрасному місті
біля Моря... Так от. Сонце світить, вітер віє... дощ ллє... Саме по собі Сонце
- то є величезна розжарена куля, на якій немає ніякого життя, там панує лише
Вогонь і Смерть... Так, Величезний Страшний Вогненний Океан під назвою Сонце,
розміром із тисячу таких планет як Земля... Але... світло від Сонця досягає
Землі, і дає в результаті життя прекрасній тендітній квіточці... Одне - велике,
неймовірно велике, дуже гаряче і смертоносне - дає в результаті життя чомусь
дуже малому, тендітному, крихкому... Так... В Природі, у Всесвіті все
взаємопов'язане... мале пов'язане з великими... мертве - з живим, гаряче - з
холодним... А як можна назвати те, что все це уособлює собою? Одні кажуть -
Матір-Природа, інші кажуть - Бог, Творець. А в цьому фільмі центральним
лейтмотивом проходить ідея Сили... Сили, що все об'єдную між собою... Дуже
філософський фільм насправді. І показовий діалог з передостанньої, восьмої серії
фільму (Останні Джедаї)... 

Таймкод - 49-10

Люк - тянись до неї своєю душею...  що ти бачиш?
Рей - острів... Життя... Смерть та тлін... початок нового Життя... тепло...
холод... мир... насилля... 
Люк - а що їх об'єднує?
Рей - баланс... і енергія... Сила...
Люк - а в тобі самій...
Рей - і в мені самій... та ж сама Сила... 
Люк - це і є твій урок... Сила не є власністю джедаїв...
ідея, що зі смертю джедая вмирає Світло, є хибною... ти відчуваєш це!? 

Ну от. У Всесвіті все пов'язано. Як колись писав англійський поет-романтик Вільям Блейк, 

Бачити світ у зерні піску, Небо — у квітці синій, В своїй долоні безкрайність усю І вічність в одній годині.

А Сила - то є назва тому, що все об'єднує
в одне ціле, і далекі галактики, і розжарене Сонце, і тендітну квіточку десь у бабусі на городі.
Ось у цьому власне і є основна ідея фільму, ідея Сили, так, ідея Сили.
