% vim: keymap=russian-jcukenwin
%%beginhead 
 
%%file 01_02_2022.fb.fb_group.story_kiev_ua.1.chelovek_kiev_knigi
%%parent 01_02_2022
 
%%url https://www.facebook.com/groups/story.kiev.ua/posts/1589040201292775
 
%%author_id fb_group.story_kiev_ua
%%date 
 
%%tags andersen_gans_kristian,kiev,kniga,knigoizdanie,knigopechatanie,literatura
%%title Жил в Киеве человек, который больше всего на свете любил книги
 
%%endhead 
 
\subsection{Жил в Киеве человек, который больше всего на свете любил книги}
\label{sec:01_02_2022.fb.fb_group.story_kiev_ua.1.chelovek_kiev_knigi}
 
\Purl{https://www.facebook.com/groups/story.kiev.ua/posts/1589040201292775}
\ifcmt
 author_begin
   author_id fb_group.story_kiev_ua
 author_end
\fi

Жил в Киеве человек, который больше всего на свете любил книги. Еще ребенком он
дал себе обещание всю свою жизнь посвятить книгам и он свое обещание выполнил.
Звали этого человека Стефан Кульженко.

11-летний мальчик-сирота, он шел однажды по Крещатику и заглянул в книжную
лавку на углу с улицей Институтской. Лавка принадлежала Иосифу Вальнеру,
который сам издавал литографии, вся лавка была увешена репродукциями картин и
мальчику показалось, что он попал в сказку. Он был решительный мальчик и тут же
попросился в ученики к хозяину и был принят с полным обеспечением. Он не только
хорошо работал, он еще и учился: самостоятельно, например, выучил немецкий
язык. Несмотря на то, что он был самым молодым учеником – на шестом году
обучения он стал получать самую большую зарплату и именно его хозяин выбрал в
свои помощники. Поработав несколько лет в книжном магазине, Стефан очень
заинтересовался издательским делом и устроился управляющим типографии Ивана
Давиденко, одной из лучших в Киеве, и опять его взяли сразу и без протекции,
так он понравился хозяину. А когда Давиденко решил делать бизнес в Петербурге,
именно Стефану Кульженко он и отдал свою типографию в аренду.

И первое, что сделал Кульженко – взял кредит и купил новое современное
оборудование. И первая книга, которая вышла на этом новом оборудовании – сказки
Ганса Христана Андерсена на украинском языке в переводе Михаила Старицкого и с
иллюстрациями нашего известного художника Николая Мурашко. У меня есть мечта –
когда-нибудь увидеть эту книгу вживую и подержать ее в руках!

Со временем нужно было расширять производство и Стефан Кульженко покупает
участок на улице Новоелизаветенской ( сейчас улица Пушкинская),  и на его
деньги был построен 3-х этажный дом, на первом этаже которого была сама
типография, на втором – контора (офис) и склады, а на 3-м – квартира самого
хозяина, где он жил с семьей, в которой росли 5 детей. Вот так мальчик-сирота
стал одним из самых богатых людей Киева. О нем можно рассказывать очень много.
Он одним из первых начал издавать фотоальбомы с видами Киева – и это могли быть
и роскошные альбомы с папиросной бумагой и шелковыми лентами, и скромные
небольшие и недорогие альбомы, которые гимназисты дарили знакомым барышням. Он
издавал путеводители по Киеву и открытки с киевскими видами.

Печально, что в конце жизни человек, который больше всего на свете любил книги,
почти ослеп. А когда он умер, его похоронили на Байковом кладбище и его могила
сохранилась. А руководить типографией стал его старший сын Василий и он же
написал прекрасные воспоминания об отце «Полвека у печатного станка». И именно
в типографии Кульженко в январе 1918 году были напечатаны первые украинские
деньги – Карбованцы. После захвата Украины большевиками типография  Кульженко
была национализирована и прекратила существование. И показательно, наверное,
что последней книгой, которую Василий Кульженко издал в своей типографии, стал
уникальный путеводитель «Киев» Константина Шероцкого, без которого я бы не
смогла сделать не одну свою экскурсию по Киеву.
