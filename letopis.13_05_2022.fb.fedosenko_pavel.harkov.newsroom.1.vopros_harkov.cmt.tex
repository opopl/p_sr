% vim: keymap=russian-jcukenwin
%%beginhead 
 
%%file 13_05_2022.fb.fedosenko_pavel.harkov.newsroom.1.vopros_harkov.cmt
%%parent 13_05_2022.fb.fedosenko_pavel.harkov.newsroom.1.vopros_harkov
 
%%url 
 
%%author_id 
%%date 
 
%%tags 
%%title 
 
%%endhead 
\qqSecCmt

\begin{itemize} % {
\iusr{Yulia Yarmolenko}

Ми, звичайно, не в метро живемо, але й на голові родичів сидіти набридло.
Будемо повертатися. Поки що без дитини

\iusr{Maryna Khoroshenko}
Ми вирішили почекати ще, десь тижнів шість, ближче до липня будемо повертатися скоріше за всього.

\iusr{Олексій Жабін}
в ще громадський транспорт не працює

\iusr{Pavel Fedosenko}
\textbf{Олексій Жабін} с понедельника планируют запустить

\iusr{Iva Stishun}

Моё любимое место в Ха. Спасибо. И за фото и за посыл. Так и есть. Все мы
разные и по разному реагируем и действуем. Мне кажется никто не знает, что
будет, если сейчас многие вернутся и возникнет новое напряжение на разбитую
инфраструктуру и уставшую психику. Поэтому продолжается время сложных
выборов/решений и ответственности за них...

\iusr{Andrey Poyarkov}
Милый дом

\iusr{Igor Saldyga}
Нескорений Харків чекає на своїх мешканців

\ifcmt
  ig https://scontent-mxp1-1.xx.fbcdn.net/v/t39.30808-6/279740210_5256014951123908_2812767106461061280_n.jpg?_nc_cat=111&ccb=1-6&_nc_sid=dbeb18&_nc_ohc=SDRjQe18YUkAX-ScYkt&_nc_ht=scontent-mxp1-1.xx&oh=00_AT_rS6xn-KKn3BB-szuKr7LwLMHOpFDKbcjEK2R-kGDCfg&oe=628680CC
  @width 0.3
\fi

\iusr{Willy White}
100\%

\iusr{Олег Федоров}
саме так!

\iusr{Людмила Анненко}

Так, транспорт запускають, але не метро(, доїхати з Салтівки на Холодну Гору на
роботу треба буде дивитися з пересадками як(

\begin{itemize} % {
\iusr{Pavel Fedosenko}
\textbf{Людмила Анненко} метро запустят еще очень не скоро

\iusr{Людмила Анненко}
\textbf{Pavel Fedosenko} 

так, то ясно треба думати як добиратися, і біля нас на Салтівці магазини та
ринки порозбивали(, думаємо про повернення, бо викликають на роботу, але ще
дали відпустку до кінця травня, будемо дивитися як далі? Або на ровері їздити.

\iusr{Oksana Yarosh}
\textbf{Людмила Анненко} 

враховуючи, що впродовж перших двох тижнів обстрілювали надземний міст, саме
Салтівську лінію запустять останньою(

\iusr{Ihor Palchykivskyi}
\textbf{Павел Федосенко} судя по новостям, что скоро из метро выселяют - то к августу запустят

\iusr{Алёна Иванова}
\textbf{Людмила Анненко} велосипед всему голова

\iusr{Pavel Fedosenko}
\textbf{Ihor Palchykivskyi} если выселят в ближайшее время и если реально успели подготовить все за три месяца

\iusr{Oksana Yarosh}
\textbf{Ihor Palchykivskyi} 

ахах)) та ясно, спочатку всіх нажахати, що \enquote{неможливо} запустити метро раніше,
ніж за три місяці, а потім героїчно запустити його раніше озвучених строків
(йопта, харків'янчики, дивіться, яка у вас міська влада, п'ятирічку за три
роки!")

\iusr{Ihor Palchykivsky}
\textbf{Павел Федосенко} та там то 2 месяца то 3  @igg{fbicon.face.grinning.sweat} 

\iusr{Лариса Гнатченко}
Реально, все зависит от внутренней планки тревожности - ну, и от местоположения дома.
\end{itemize} % }

\iusr{Ekaterina Ivanitskaya}
Дом милый дом  @igg{fbicon.house.with.garden}  @igg{fbicon.heart.blue}  @igg{fbicon.heart.yellow} 

\iusr{Татьяна Прохорова}
Сумно але так і є((

\emph{Анна Ерошкина}

Если совсем невыносимо там, где вы сейчас находитесь, то можно вернуться, жить
сможете. Но если вы в комфортных условиях для себя и детей, лучше подождать.

\emph{Валерия Величко}
Одним постом ответил на все мои вопросы по поводу возвращения. Спасибо)

    Reply
    See Translation
    2d

Микита Соловйов  ·
Я немного иначе отвечаю. МОжно возравщаться (можно было и не уезжать), если понимаешь зачем.

\end{itemize} % }
