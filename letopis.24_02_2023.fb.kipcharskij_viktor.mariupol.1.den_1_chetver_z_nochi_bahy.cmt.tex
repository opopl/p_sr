% vim: keymap=russian-jcukenwin
%%beginhead 
 
%%file 24_02_2023.fb.kipcharskij_viktor.mariupol.1.den_1_chetver_z_nochi_bahy.cmt
%%parent 24_02_2023.fb.kipcharskij_viktor.mariupol.1.den_1_chetver_z_nochi_bahy
 
%%url 
 
%%author_id 
%%date 
 
%%tags 
%%title 
 
%%endhead 

\qqSecCmt

\iusr{Віктор Кіпчарський}

Рік тому, у Маріуполі я почав писати щоденник, який я назвав \enquote{Хроніка}.

Минув рік. Багато дрібниць пішло з пам'яті.

Сподіваюсь, хтось (не лише з маріупольців) виправить мене, або доповнить.

\iusr{Елена Девина}

В нас теж перший дзвінок був з Волонтерівці...

\iusr{Марина Солошенко}

В Харкові оптимізму було мало, бо танки вже їхали по окружній дорозі, був шок!

\begin{itemize} % {
\iusr{Віктор Кіпчарський}
\textbf{Марина Солошенко} Оптимізм?
Він був хіба що на \enquote{Марафоні}.
В нас була надія на те, що за 8 років ми захистили своє місто.
\end{itemize} % }

\iusr{Таісія Пославська}

Прокинулась о 4.35. Бахало так, що одразу зрозуміла війна...

\begin{itemize} % {
\iusr{Віктор Кіпчарський}
\textbf{Таісія Пославська} Який район?

\iusr{Таісія Пославська}
\textbf{Віктор Кіпчарський} Курчатово

\iusr{Віктор Кіпчарський}
\textbf{Таісія Пославська} Напевно, Вам чути було з двох боків: і з Лівого, і з Сартани з Талаковкою...

\iusr{Таісія Пославська}
\textbf{Віктор Кіпчарський} так
\end{itemize} % }
