%%beginhead 
 
%%file 10_11_2021.fb.fed_greek_comm_ukr.1.ob_istorii_sozdania_obschestva_grekov_v_mariupole
%%parent 10_11_2021
 
%%url https://www.facebook.com/federation.of.greek.communities.of.ukraine/posts/pfbid02FT5B5WVXAuJPcWMuWTZM8V6jF1tWwZoXFdjuBQzVaVCPJhV5xngKuSAxiuHj7PiPl
 
%%author_id fed_greek_comm_ukr
%%date 10_11_2021
 
%%tags 
%%title Об истории создания общества греков в Мариуполе…
 
%%endhead 

\subsection{Об истории создания общества греков в Мариуполе...}
\label{sec:10_11_2021.fb.fed_greek_comm_ukr.1.ob_istorii_sozdania_obschestva_grekov_v_mariupole}

\Purl{https://www.facebook.com/federation.of.greek.communities.of.ukraine/posts/pfbid02FT5B5WVXAuJPcWMuWTZM8V6jF1tWwZoXFdjuBQzVaVCPJhV5xngKuSAxiuHj7PiPl}
\ifcmt
 author_begin
   author_id fed_greek_comm_ukr
 author_end
\fi

@igg{fbicon.flag.ukraina} @igg{fbicon.flag.greek} @igg{fbicon.flag.ukraina} Історія грецького національного руху в Україні

Об истории  создания общества греков в Мариуполе...

\ii{10_11_2021.fb.fed_greek_comm_ukr.1.ob_istorii_sozdania_obschestva_grekov_v_mariupole.pic.1}

В последнее время выходит в свет много печатных работ (книг, брошюр, газетных
статей), касающихся истории греков Украины или их участия в исторических
событиях. Это радует. Но при отсутствии опубликованных добросовестных
исследовательских работ по этой теме возникает иногда, вполне возможно и не
умышленное, искажение сути происходивших не так уж давно событий. Возникает это
ещё и потому, что фиксирующие эти события материалы находятся в личных архивах,
в архивах отдельных общественных организаций, либо вообще отсутствуют, и их
надо восстанавливать. 

Поэтому столь важно добросовестное исследование вопроса перед о\hyp{}публикованием,
так как в противном случае опубликованная и многократно переходящая из одной
публикации в другую ошибка приобретает значение трудно опровержимого факта. В
связи со сказанным выше я попытаюсь осветить некоторые вопросы, касающиеся
Мариупольского общества греков и вызывающие среди общественности различные
толкования.

Первый такой \enquote{спорный} момент: где и когда возникло первое греческое
общество (имеется в виду - после 1937 г.)?

Спор между греками вёлся, по крайне мере, о двух городах: Мариуполе и Донецке,
приоритет иногда болезненно отстаивался сторонниками версий.

Трудность для постороннего (исследователя)  заключалась  в  том, что в то время
(начало1989 г.) ещё не было закона о регистрации общественных организаций,
вследствие чего не было единых документов, фиксирующих такое событие, или же
они могут находиться (в виде протокола) в архивах обществ, и поиск этих бумаг
уже предполагает некоторые усилия и трудности для исследователя. При этом надо
ещё отметить, что в то время мало кого беспокоило скрупулезное сохранение всех
архивов и фиксирование событий на бумаге. Кроме этого, не было и мыслей о том,
что происходят некие исторические события.

Как это было?

Шёл 1988 год – год больших надежд: в городе, в частности, в Мариупольском
металлургическом институте (ныне ПГТУ), повеяло духом демократии и свободы.
Совершенно не случайно именно здесь возникает инициативная группа по
возвращению городу его исторического названия. В составе инициативной группы
значительную работу выполняли греки – жители города. Этот процесс логично
предвосхитил последующие события.

В период, когда инициативная группа пыталась внести на рассмотрение сессии
городского Совета вопрос о переименовании города, я и познакомилась с
Александром Ашлой. Хотя он не был членом группы, однако всегда присутствовал на
наших митингах по поводу переименования города. Именно он во время одной
встречи в декабре 1988 г. сформулировал и  предложил то, что во мне пока ещё
бродило и зрело: создать национальную организацию греков. 

В городе, где греки составляли значительное сообщество в сравнении с другими
национальными меньшинствами, необходимость (политическая, социальная,
культурная) создания организации совершенно иного типа, имеющей и иные кроме
просветительских целей (в сравнении с действующим просветительским обществом
им. Хартахая) витала в воздухе. 

Где, как не в Мариуполе должно было возникнуть такое этническое греческое
обществ в Украине, как только появилась возможность? Ведь Мариуполь был
культурным центром греческой диаспоры ещё в дореволюционной России.

В горкоме партии разъяснили, что общество может быть признано только в случае,
если оно создано при двух каких-либо официальных учреждениях.  Решено было
просить о предоставлении \enquote{крыши} Игоря Жежеленко – ректора института, где я
работала. Мы (А. Ашла и автор статьи) обратились к нему с соответствующим
письмом и получили согласие.  А. Ашла переговорил также с директором ДК
\enquote{Азовсталь} Аркадием Проценко и тоже получил положительный ответ. И хотя две
организации согласились нас как бы приютить, помещения у нас не было. Иногда мы
проводили свои заседания Совета прямо в вестибюле металлургического института,
чаще – у автора статьи на квартире.

После получения согласия И. Жежеленко и А. Проценко надо было найти около 20 (так
нас инструктировали) человек греческой национальности, согласных выступить в
качестве учредителей греческого общества. А. Ашла уехал в Москву. Мы с сестрой
переговорили с нашими знакомыми – греками, и, надо сказать, далеко не все
согласились участвовать в этом деле. Сказывался страх после ужаса 1937 года.
Пришлось прибегнуть даже к телефонному справочнику в поисках греческих фамилий.
Некоторые по телефону говорили: \enquote{живем – и хорошо, зачем на свою голову и на
голову всех греков искать беды?}, \enquote{давно ли забыли старые раны?}, \enquote{хотите, чтоб
нас опять поставили к стенке?} и т.д. Были и такие, которым всё равно, более
того, некоторые сетовали на то, что судьбе было угодно, чтоб они родились
греками: это испортило им карьеру и прочее.      

После приезда А. Ашлы в январе состоялось ещё одна встреча. Назначили день сбора
учредителей (у автора статьи на квартире) – 25 января 1989 года. Мы с Еленой
Коноп сумели уговорить к тому времени только 15 человек – греков. Хочу здесь с
благодарностью отметить, что наш друг Сергей Буров, не будучи греком,
согласился участвовать 16-ым \enquote{для массовости}, как он шутил.

25 января 1989 года собрались: Александр Ашла, Елена Коноп, Лель Кузьминков,
Игорь Налчаджи, Василий Чапни, Владимир Харакоз, Константин Диамантопуло,
Дмитрий Топалов, Фёдор Бахчисарай, Алексей Сагирь, Георгий Цивадиц, Константин
Малакуцко, Николай Токий, Фёдор Харабаш и автор этих строк.

Повестка дня состояла из трех вопросов: 

\begin{itemize}
  \item 1. Об организации Общества греков. 
  \item 2. Выборы председателя Общества и членов Совета. 
  \item 3. Выборы делегата на областное учредительное собрание \enquote{Товариства рідної мови ім. Т. Г. Шевченка}.
\end{itemize}

В то время на волне пробуждающегося национального самосознания масс партийные
органы Украины, стремясь развивающиеся движения сделать управляемыми и
подконтрольными, даже содействовали созданию обществ в составе \enquote{Товариства
рідної мови}. Предварительно обсудив эту тему до учредительного собрания, мы
решили не оставаться в стороне и, в случае создания греческого общества,
избрать сразу же делегата на областное учредительное собрание \enquote{Товариства}, а
затем уже решать вопрос о возможном вступлении общества в его состав.

Если просмотреть протокол учредительного собрания греческого общества
(сохранился только черновик), вырисовывается довольно противоречивая картина.
Всплыло два спорных момента. Первый состоял в том, что некоторые члены собрания
никак не соглашались с тем, что надо что-то создавать, когда \enquote{всё уже есть}.
Имелось в виду общество \enquote{Меотида}, более известное как просветительское
объединение им. Хартахая. Такую точку зрения особенно настойчиво защищали
Л. Кузьминков и И. Налчаджи (руководитель объединения). 

Доводы о том, что имеющееся объединение не является национальным, этническим,
не имеет Устава, четкой структуры и положения о членстве и что оно ставит лишь
узкие просветительские цели, не сразу были приняты. Вопрос решался неожиданно
(для автора статьи) болезненно, с трудом.

Второй момент разногласий возник по поводу одного из основных направлений
работы будущего общества – языка. Некоторые считали, что необходимо изучать и
вводить в школах новогреческий язык, другие – диалекты греков Приазовья. 

Споры по этому поводу привели однозначно к тому, что даже с целью объединения
урумов и румеев надо изучать литературный язык современной Греции, не забывая,
однако, о сохранении и развитии диалектов греков Приазовья.

Результатом учредительного собрания было постановление: \enquote{заявить о создании
Общества греков на базе двух предприятий – Мариупольского металлургического
института и ДК \enquote{Азовсталь}}.

В результате обсуждения второго вопроса в состав Совета вошли: 

Александр Ашла, Игорь Налчаджи, Валентина Коноп-Ляшко, Аркадий Проценко,
Валерий Киор. Председателем первого общества греков был избран А. Ашла.

Аркадий Проценко и Валерий Киор были избраны по предложению А. Ашлы. В. Киор на
заседаниях Совета так и не появлялся и участия в его работе не принимал.

На следующий день была составлена выписка из протокола учредительного собрания
и письмо председателю правления Донецкого отделения Советского фонда культуры с
просьбой зарегистрировать общество, и А. Ашла уехал с документами в Донецк.
Поскольку основной задачей общества признавалось сохранение развитие языка и
культуры, общество решили назвать: \enquote{Общество греческого языка и культуры}. 

Кроме того, в то время такое название более логично вписывалось в представление
членов горкома партии города: ведь мы имели в начальный период своей
деятельности постоянного куратора – каб. № 4, горком партии, В. Б. Старчук, а
Донецкое отделение фонда культуры выступало в это время как учредитель Общества
родного языка и культуры им. Т. Шевченко. 

В газете \enquote{Приазовский рабочий} (№ 27 (14584), среда 8 февраля 1989 г.) вышла
статья А. Ашлы \enquote{Культура древняя и молодая}, где он сообщает: \enquote{Недавно при
Мариупольском металлургическом институте и Дворце культуры и техники комбината
\enquote{Азовсталь} создано общество греческого языка и культуры, направленное на
возрождение и развитие национальной культуры.} Статья подписана: \enquote{А. Ашла,
председатель общества греческого языка и культуры, преподаватель музыкальной
школы № 3}. В этой же заметке-статье он приглашает греков на собрание, которое
состоится 9 февраля 1989 года в 19:00 в малом зале ДК \enquote{Азовсталь}.     

Так было создано наше общество. Многие при общении вспоминают как первое именно
это собрание греков в ДК \enquote{Азовсталь}. Но ведь это было позже, когда наше
общество было уже зарегистрировано в Донецком отделении Советского фонда
культуры, после значительной подготовительной работы, после выбора Совета и
председателя общества. 

В сентябре этого же 1989 года мы по решению собрания, прошедшего также в ДК
\enquote{Азовсталь}, решили вступить во Всесоюзное общество советских греков, которое
предполагалось создать с центром в Донецке. 

Сразу же А. Ашла поставил задачу перед Советом нашего общества – создать  его
Устав и разработать структуру.  В этой работе приняли участие также Виктор
Харабет и Владимир Коссе. 

У нас не было тогда каких-либо аналогов, поэтому работа продвигалась вперед
медленно, многие вопросы вызывали споры, работа растянулась на месяцы. Но уже в
первом \enquote{Уставе общества греческого языка и культуры при Мариупольским
металлургическом институте и Дворце культуры и техники завода \enquote{Азовсталь}} (так
он назывался) была заложена демократическая основа национального объединения,
предусматривающая возможность создания первичных организаций и последующего
объединения как индивидуальных, так и коллективных членов.

Как только весть об образовании греческого общества в Мариуполе достигла,
благодаря СМИ, Киева, с нами связались представители украинского национального
движения и пригласили двух наших представителей на I Учредительный съезд
\enquote{Руховцев} в Киеве. Он проходил в атмосфере удивительного подъема и эйфории. 

На съезде выступали  Л. Кравчук, В. Яворивский, О. Гончар, Б. Олейник.
Присутствовали представители всех национальных общественных организаций,
образовавшихся к тому времени в Украине. Было много сказано слов о том, что все
национальные меньшины получат возможность развивать свою культуру.

Но когда я спросила у Яворивского (а он был избран в руководство \enquote{Руха}), как
он видит, в частности, развитие греческой культуры в Украине и проблему
возрождения языка, он, похлопав меня по плечу, сказал, что, дескать, скоро в
Киеве будет фестиваль национальных искусств, и они обязательно пригласят наши
национальные творческие коллективы. Мне стало абсолютно ясно, что дальше песен
и плясок дело развитие греческой культуры и образования (по планам \enquote{Руховцев})
не сдвинется, и надо эту проблему самим, как говорится, \enquote{брать в руки}.

На счету созданного греческого общества при А. Ашле в течение первых полутора
лет существования были такие мероприятия, как: 

📌создание первых в Украине в период 80-90-х годов курсов новогреческого языка; 

📌первое широкое празднование Дня независимости Греции в ДК \enquote{Азовсталь} 28 марта 1989 года;

📌 создание оргкомитета по проведению в Мариуполе первых \enquote{Аристотелевских
чтений} с участием в нём Ю. Ю. Хотлубея (тогда первого секретаря горкома партии)
и В. В. Жежеленко (ректора металлургического института);

📌 успешные переговоры с Донецком и с Киевом  о регулярных (оплачиваемых) греческих радиопередачах в Мариуполе;

📌 первые поиски возможности обеспечить общество учебниками греческого языка
(первые 50 штук учебников М. Рытовой были получены из Грузии после неоднократных
просьб); 

📌переговоры с Грузией и Московой о направлении на учебу (новогреческий язык)
учителей для обеспечения Мариуполя квалифицированными кадрами и направления
преподавателей общества на учебу;

📌 I-ая региональная конференция в апреле 1989 года по вопросам преподавания
греческого языка в Приазовье. 

На примере \enquote{информации} в газете об этом (последнем) случае я хочу показать,
как непреднамеренно, но шаблонно искажаются события. В \enquote{Приазовском рабочем} №
81 (14638) за 28 апреля 1989 года вышла статья Анатолия Балджи под заголовком
\enquote{С уважением к родному языку}. В ней автор, справедливо отмечая, что в
Мариуполе и окрестных селах некому преподавать греческий язык, пишет (да
простят мне читатели длительную цитату по тексту): \enquote{Сложившаяся ситуация,
естественно, известна и в органах народного образования. Вот почему областное
управление народного образования, областной институт усовершенствования
учителей (привожу цитату) сочли необходимым провести в Мариуполе семинар
учителей...}. 

И в конце статьи \enquote{...пока готовился к печати материал, позвонила заведующая
Ильичевским районным отделом народного образования и сообщила: в этом году на
отделение греческого языка Тбилисского университета направляется одна из
выпускниц СШ № 49... Что ж, лед тронулся, наступило время практических дел, что и
требуется в ходе подготовки к предстоящему Пленуму ЦК КПСС по межнациональным
вопросам}. 

Умолчим об ошибке в том, кто направлялся в Тбилиси, но что мы действовали в
русле подготовки к Пленуму ЦК - мы не ведали. Но не ведал и автор этой статьи о
том, сколько усилий, писем, звонков, поездок в Донецк и т.д. необходимо было
сделать Мариупольскому греческому обществу, чтобы областное управление
народного образования \enquote{сочло необходимым} провести этот семинар учителей. 

Надо сказать, что председатель общества А. Ашла всегда строил нашу работу таким
образом, чтоб планы общества \enquote{вплетались} естественным образом в планы
официальных властных структур, и я считаю, что это правильный подход, потому
что властям чаще приходится вспоминать о греках, которые составляют
значительное сообщество в нашей области. 

Вот таким образом при недобросовестном подходе к теме публикации искажается
история событий и правда о них. Ведь посторонний человек, прочитав статью,
подумает, что инициатива (уж не говорю об исполнении) полностью принадлежала
органам образования (власти). 

Увы! И инициатива, и все исполнение мероприятия полностью лежало на плечах
общества, представители власти только присутствовали и одобряли (иногда и
бумагами) наши действия после длительных переговоров с ними.  

Валентина Коноп-Ляшко

(г. \enquote{Эллины Украины}, май 2010 г.)

\#Україна \#Греція \#греки\_України \#ФГТУ \#Греція\_Україна
\#ФЕДЕРАЦІЯ\_ГРЕЦЬКИХ\_ТОВАРИСТВ\_УКРАЇНИ  \#Україна\_ФГТУ \#греки
\#ФГТУ\_історія
