% vim: keymap=russian-jcukenwin
%%beginhead 
 
%%file 28_12_2020.news.radiosvoboda.rozdobudko_igor.1.moskva_ukraina_gerb_mazepy_kreml
%%parent 28_12_2020
 
%%url https://www.radiosvoboda.org/a/ivan-mazepa-harmata-moskva-kreml/31021407.html
 
%%author 
%%author_id rozdobudko_igor
%%author_url 
 
%%tags ukraina,istoria,moskva,kreml,mazepa_ivan
%%title Українські святині в серці Москви. Герб Івана Мазепи на території Кремля
 
%%endhead 
 
\subsection{Українські святині в серці Москви. Герб Івана Мазепи на території Кремля}
\label{sec:28_12_2020.news.radiosvoboda.rozdobudko_igor.1.moskva_ukraina_gerb_mazepy_kreml}
\Purl{https://www.radiosvoboda.org/a/ivan-mazepa-harmata-moskva-kreml/31021407.html}
\ifcmt
	author_begin
   author_id rozdobudko_igor
	author_end
\fi

\ifcmt
  pic https://gdb.rferl.org/E066E0A0-EDAD-44BE-9CEE-2460978B7BA4_w1597_r1_s.jpg
	caption Гармата гетьмана України Івана Мазепи на території московського Кремля
\fi

\begin{leftbar}
  \begingroup
    \em\Large\bfseries\color{blue}
Чи знаєте ви, друзі, що, гуляючи по московському Кремлю, можна побачити герб
Івана Мазепи, а вийшовши звідти на Червону площу російської столиці та
підійшовши до першого встановленого в Москві скульптурного пам'ятника,
прочитати на його постаменті назву українського міста, яку викарбував тут,
вдячний своїй батьківщині України, автор?
  \endgroup
\end{leftbar}

Розумію, друзі, що багатьом з вас і згадувати ту Москву не хочеться, не то, що
по ній гуляти. Але я тут народився, і тут живу, так що гуляти іноді доводиться.
Та я і тут, у Москві, завжди пам'ятаю про Україну, про її славу, і добре знаю,
яка сторінка московської історії і як пов'язана з Україною. Хочу і вас з
деякими з тих сторінок ознайомити.

Мало хто знає, але герб українського гетьмана Івана Мазепи можна побачити (і
навіть двічі!), в самому серці Росії, в московському Кремлі. Розташований герб
на стволі та лафеті старовинної української гармати, яка виставлена поблизу
кремлівського Арсеналу як один із російських військових трофеїв.

\ifcmt
  pic https://gdb.rferl.org/C7410BB4-E46B-4387-90F8-DBDA6BE60CC6_w1023_r0_s.jpg
  caption Герб гетьмана Івана Мазепи на лафеті його гармати на території московського Кремля
	width 0.4
	
	pic https://gdb.rferl.org/64378D7E-6337-417E-9BFC-3359464D898F_w1023_r0_s.jpg
	width 0.4
\fi

Автор «мазепиної гармати», український ливар із Глухова Карпо Балашевич, зробив
її у 1705 році. От що пише про цю гармату російський дослідник М.В. Гордєєв: «З
робіт українського майстра гарматного литва XVII століття Карпа Балашевича в
Кремлі є гармата «Лев», відлита в місті Глухові в 1705 році. На казенній
частині ствола латинський напис і литий герб гетьмана Малоросії Івана Мазепи,
оточений буквами «А.Е.Г.І.М.В.Ц.Б.З.», які позначають початкові літери слів
його повного титулу. Калібр 125 мм, вага 3 тис. кг, загальна довжина ствола 3
890 мм. Гармата розташована на декоративному чавунному лафеті, виконаному в
1835 році, біля західного фасаду Арсеналу, з лівого боку від вхідної арки».

Як бачимо, другий герб гетьмана Мазепи, той що на лафеті, зробили вже
московські майстри у 1835 році, за царювання імператора Миколи І. Цікаво
зазначити, що російські священники проклинали в цей час Мазепу анафемою, а
кремлівські майстри зберігали пам'ять про нього на гетьманській зброї. Не
забуваймо і ми!

\ifcmt
  pic https://gdb.rferl.org/0B295ECA-98D5-4FCB-92AA-16586F824075_w1023_r0_s.jpg
	caption Гармата Івана Мазепи на тлі Арсеналу московського Кремля
	width 0.4

	pic https://gdb.rferl.org/A7A2FC4D-E8DB-4FF2-A7F6-4025EABEEAF1_w1023_r0_s.jpg
	caption Автор статті біля Мазепиної гармати. 2014 рік
	width 0.4
\fi

Відомо також, що Карпо Балашевич у 1699 році вилив на честь Івана Мазепи так
званий «Мазепин дзвін», з гербом та зображенням самого гетьмана на боці дзвону.
Цей дзвін був у гетьманській столиці Батурині, а в радянські роки в
Чернігівському історичному музеї. Після Другої світової війни з України він
зник, а віднайшовся раптово 2015 року в російському місті Оренбурзі, без язика
та з тріщиною.

Вийшовши з московського Кремля на Червону площу, підійдемо до першого в
російській столиці пам'ятника, встановленого тут у 1818 році. Це пам'ятник
визволителям Кремля та Москви від польсько-литовсько-козацького війська 1612
року Кузьмі Мініну та Дмитру Пожарському. Автор твору – Іван Мартос
(1754–1835), український та російський скульптор, із відомого козацького роду.

\ifcmt
  pic https://gdb.rferl.org/F26AAF23-DD0A-4F00-8D9A-B04C0F5209A9_w650_r0_s.jpg
	caption Іван Мартос (1754–1835) – скульптор-монументаліст, уродженець міста Ічні (Чернігівщина). Народився в українській козацько-старшинській родині
\fi

Народився Іван Мартос в містечку Ічня на сучасній Чернігівщині, де його батько
був сотенним отаманом, а дядько відомим різьбярем, що різьбив іконостас
Ічнянської церкви та скульптури для її іконостасу, і таким чином, потяг до
скульптурного мистецтва майбутній перший московський скульптор відчув, скоріш
за все, ще на рідній Україні, спостерігаючи за творчістю свого дядька. З малих
років потрапивши до Росії та проживши тут усе своє подальше життя, Іван Мартос
разом з тим ніколи не забував про Україну, мріяв побувати у Києві, а для рідної
землі став автором надгробку гетьмана Кирила Розумовського у Батурині та
відомого пам'ятника «дюку» Рішельє на Приморському бульварі в Одесі.

\ifcmt
  pic https://gdb.rferl.org/DD3BA154-1556-4BF1-82AA-B78A505492F5_w1023_r0_s.jpg
	caption Пам'ятник «дюку» Рішельє в Одесі. Автор Іван Мартос із української козацько-старшинської родини
\fi

Пам'ятник Мініну і Пожарському спочатку мав стояти у Нижньому Новгороді, біля
мурів місцевого кремля, там, де мешканці міста збирали гроші 1612 року для
організації «ополченія» для визволення Москви від польсько-литовсько-козацького
війська. Але захоплення від пам'ятника Мартоса було настільки велике, що
імператор Олександр І видав рескрипт про встановлення скульптури в центрі
Москви.

\ifcmt
  pic https://gdb.rferl.org/224270B6-BA45-408E-80A0-54553A113F03_w1023_r0_s.jpg
	caption Пам'ятник Мініну й Пожарському на Червоній площі у Москві. Створений за проєктом скульптура Івана Мартоса з відомого українського козацького роду, уродженця міста Ічні (Чернігівщина)
\fi

А українець Іван Мартос (в Росії його називали «хохлом»), вирішив зробити так,
аби кожна людина, яка захоче уважно вивчити цей пам'ятник, могла довідатися, що
зроблений він уродженцем українського міста Ічні. Зайшовши до пам'ятника з
тилу, прочитаємо на основі скульптури такий напис: «СОЧИНИЛЪ И ИЗВАЯЛЪ ІОАННЪ
ПЕТРОВИЧЬ МАРТОСЪ РОДОМЪ ИЗЪ ИЧНИ».

\ifcmt
  pic https://gdb.rferl.org/444DBA61-147A-427F-B122-A68BC388DDD6_w1023_r0_s.jpg
	caption Напис про українське місто Ічню розташований на задній частині пам'ятника, внизу скульптури

	pic https://gdb.rferl.org/D720C701-F7FD-449F-83CC-9EE1A9C9F94A_w1597_r0_s.jpg
	caption Напис на задній частині пам’ятника Мініну й Пожарському на Червоній площі у Москві: «СОЧИНИЛЪ И ИЗВАЯЛЪ ІОАННЪ ПЕТРОВИЧЬ МАРТОСЪ РОДОМЪ ИЗЪ ИЧНИ»
\fi

А в Нижньому Новгороді у 2005 році була встановлена копія пам'ятника Мартоса,
авторства відомого грузинського скульптора Зураба Церетелі, того самого, що
створив для українського міста Харкова пам'ятник козака Харка. Щоправда, на
нижньогородському пам'ятнику, українське місто Ічня вже не згадується.

\ifcmt
  pic https://gdb.rferl.org/E5163591-6595-4152-9A2D-F669D4649EF9_w1023_r0_s.jpg
	caption Копія пам'ятника Мініну й Пожарському в Нижньому Новгороді. Копію скульптури Івана Мартоса, уродженця міста Ічні (Чернігівщина), виконав Зураб Церетелі
\fi

Ігор Роздобудько – історик, перекладач, член Малої Ради Громади українців Росії

Думки, висловлені в рубриці «Точка зору», передають погляди самих авторів і не
конче відображають позицію Радіо Свобода

