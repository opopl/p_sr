% vim: keymap=russian-jcukenwin
%%beginhead 
 
%%file 15_08_2023.fb.mariupol.muzej.mkm.1.robota_suchasnyh_mariupolskih_arheologiv
%%parent 15_08_2023
 
%%url https://www.facebook.com/100093184796939/posts/pfbid02s5Rk7n7vYYVP4CGX81xFTMDmiW2i7bqLRxLgJm1wWBcpjsizpsY1WCy8iUKdgYSZl
 
%%author_id mariupol.muzej.mkm
%%date 
 
%%tags 
%%title Робота сучасних маріупольських археологів
 
%%endhead 
 
\subsection{Робота сучасних маріупольських археологів}
\label{sec:15_08_2023.fb.mariupol.muzej.mkm.1.robota_suchasnyh_mariupolskih_arheologiv}
 
\Purl{https://www.facebook.com/100093184796939/posts/pfbid02s5Rk7n7vYYVP4CGX81xFTMDmiW2i7bqLRxLgJm1wWBcpjsizpsY1WCy8iUKdgYSZl}
\ifcmt
 author_begin
   author_id mariupol.muzej.mkm
 author_end
\fi

🎉✨️Вітаємо фахівців археології з їх професійним святом! Бажаємо цікавих
відкриттів та цінних знахідок. А нашим читачам хочемо розповісти про роботу
сучасних маріупольських археологів.

⏳️Маріупольський музей має майже 100-річний досвід археологічних досліджень,
які розпочались з перших років роботи установи. Час від часу музейники брали
участь у роботі археологічних експедицій та отримували на збереження знахідки.
А після 1980-90-х років встановилися вже міцні дружні відносини між місцевим
музеєм та представниками маріупольської археологічної школи. Здобутками
багаторічної співпраці були й регулярне поповнення знахідками музейної
колекції, й наукові праці та монографії дослідників.

Сьогодні можна говорити вже про третє покоління професійних маріупольських
археологів, що присвятили себе виявленню, вивченню й охороні археологічних
пам'яток Приазов'я. 

🏕📚Перші дослідники, захоплені таємничою наукою, жагою відкриттів та
романтикою польових робіт, брали участь у розкопках ще на початку ХХ ст., це
музейники Іван Коваленко з дружиною Ніною Семенівною, Неоніла Єгорова та інші.
У 1970-80-і роки нова хвиля молодих і завзятих маріупольських дослідників
отримували досвід у маститих науковців представників Інституту археології АН
УРСР, таких як С. Н. Братченко та М. І. Гладких. Серед них був і В. К.
Кульбака, завдяки якому значно поповнилась археологічна колекція
Маріупольського краєзнавчого музею, розпочалась робота археологічної експедиції
Маріупольського державного університету, при інституції формувалось зібрання
артефактів, що переросло у музей археології та історії.

☺️🤓Цікавість до давнини та розшифрування стародавніх артефактів завжди буде
актуальною. На території Маріуполя та його околиць було зареєстровано більше 30
курганів, які мали статус охоронних пам’яток, проте ще десятки залишаються
недостатньо дослідженими та потребують уваги. В останні роки маріупольські
археологи продовжували робити нові унікальні відкриття.

✅️ Майже щороку Маріупольський державний університет проводить польові
дослідження, в тому числі і тих курганів, які розпочали вивчати ще 20-30 років
тому. Археологічні експедиції МДУ очолює В’ячеслав Олегович Забавін, кандидат
історичних наук, доцент кафедри історії та археології. Під його керівництвом
викладачі і студенти, випускники історичного факультету і волонтери з Києва та
різних регіонів України, щороку на кілька спекотних літніх тижнів занурюються у
непередавану атмосферу польових розкопок. Учасники групи згадують, що їх завжди
об’єднує дух згуртованості у доланні труднощів та прийнятті рішень, а відчуття
усвідомлення знахідки чогось нового окриляє і веде вперед.

📍Так, у 2015-16 роках археологічні експедиції МДУ проходили в районі
приазовської Ялти. А у липні 2021 року працювали на розкопках двох із п'яти
курганів епохи бронзи – раннього середньовіччя, поблизу села Комишувате
Мангуського району Донецької області. Обстеження цього курганного комплексу
проходило у 1989 році, але В'ячеслав Забавін з'ясував, що тоді було досліджено
лише три кургани. В результаті, рішення поновити експедицію було не даремним.

⭐️👏Об'єкт дослідження виявився не звичайним курганом. Вдалося відкрити два
особливих поховання, що мали форму кам'яних гробниць і датуються приблизно
1500-1300 роками до н.е. Знайдені предмети дали змогу припустити, що поховання
належало жрецю, і може бути відносено до розряду елітарних. Про це говорить
дерев'яна чаша із кованою бронзовою накладкою, яку археологи знайшли біля
голови похованого чоловіка. Саме дерев'яний посуд призначався у наших давніх
предків для вживання священних напоїв – сома-хаома. До речі, склад цього
галюциногенного ритуального напою довів ще Володимир Кульбака. За думкою
вченого він був сумішшю коров'ячого молока і соку маку і використовувався
шаманами для досягнення певного стану під час спілкування з духами.

📎Рідкісною виявилась і знайдена бронзова накладка на чаші. Річ у тім, що з
1500 поховань на території Північного Приазов'я подібні накладки були знайдені
лише у семи. А знахідка команди В. О. Забавіна ще й за розмірами відрізняється
від решти артефактів, що підкреслює її унікальність.

Після завершення експедиції та наукової обробки виявлених матеріалів, знахідки
поповнили колекцію Маріупольського краєзнавчого музею. На превеликий жаль нові
предмети не встигли потрапити до експозиції. На сьогодні доля цих нових
надходжень, як і більшої частини фондового зібрання достовірно невідома. Проте
співробітники музею не втрачають надії на повернення та відновлення колекції.🙏

☺️✨️Що стосується діяльності маріупольських археологів, вони не припиняють своєї
важливої справи. В'ячеслав Олегович Забавін разом з колегами, заслуженими
істориками та науковцями виховує наступне покоління дослідників давнини. Цього
року маріупольські студенти-археологи долучились до вивчення нового досвіду
досліджень у Словаччині. Разом із польськими та словацькими колегами спільнота
Маріупольського університету приймає участь у розкопках городища та курганного
некрополя біля селища Бойна. Віримо, що отриманий експедиційний досвід
дозволить маріупольцям удосконалити свої професійні навички та знадобиться у
майбутніх дослідженнях незвіданих пластів давнього минулого Приазов'я.

\#mkmmariupol \#mariupol \#маріупольськиймузей \#маріуполь
\#відродимомаріупольськиймузей \#деньархеолога
