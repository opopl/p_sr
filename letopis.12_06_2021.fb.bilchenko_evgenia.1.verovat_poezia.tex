% vim: keymap=russian-jcukenwin
%%beginhead 
 
%%file 12_06_2021.fb.bilchenko_evgenia.1.verovat_poezia
%%parent 12_06_2021
 
%%url https://www.facebook.com/yevzhik/posts/3973781912656889
 
%%author Бильченко, Евгения
%%author_id bilchenko_evgenia
%%author_url 
 
%%tags bilchenko_evgenia,literatura,poezia,ukraina
%%title БЖ. Веровать
 
%%endhead 
 
\subsection{БЖ. Веровать}
\label{sec:12_06_2021.fb.bilchenko_evgenia.1.verovat_poezia}
\Purl{https://www.facebook.com/yevzhik/posts/3973781912656889}
\ifcmt
 author_begin
   author_id bilchenko_evgenia
 author_end
\fi

БЖ. Веровать.

У Ассоли не было выбора: 
только мнительность и кураж.
По сути, это - абсурднейший персонаж.
Нет никого противнее девок, балуемых отцом:
Из них получаются либо дурочка, либо Цой.
А вокруг - селение - мелкое дерьмецо,
И каждый желает знать, где сидит лицо:
По её чертам охота гналась рысцой,
Но фазан исчез, а вместо мишени - блажь.
Юродивые глазища, фокус их - никаков:
Астигматизм разъезжающихся зрачков.
Жилкой висок разболтан, и цель - смещаема.
Ошарашены лучники, выбешены мещане.
Она разбивается о скалу - и опять на счастье.
"Прощайте, - кричит, - меня заберут, прощайте,
Родные мои, пещерные и песчаные!
Я вас люблю!" - как можно любить стрелков?
Её обсессия - это вымышленная страна,
Где ей обещали всё до рожна сполна,
Где она нужна, как никто из них, как из них никто:
Длилось из года в год перформативное шапито.
На нее ходили, как в цирк. Кто-то платил по сто.
Заезжий сектант её огласил святой.
После надоела всем экскурсия на постой,
Кроме двух старикашек, но вскорости тех и то,
Перестало смешить. Осталась Ассоль одна.
Но каждое утро она влезала на брег морской.
Тасовала скалы худенькою рукой.
Раздвигала гроты, чтоб виден был горизонт.
Собирала тучи - выжать из них озон.
Овладела дайвингом: ныряла под мезозой,
Перекопала шельфы до Атлантид: резон
Был минимален. Море имело зоб
Глубже, чем в том учебнике про покой,
Что написал креатор один, буддист.
Море и вглубь, и вширь, и в обхват, и ввысь
Глядело одновременно, как Ассолевые глаза.
Непонятно, оно было против неё ли, за?
Оно говорило всё, ничего так и не сказав:
В их посёлке таким же каверзным был вокзал,
Где кассир просиживал свои будущее и зад,
Хотя, если честно, здесь каждый был сердцем чист.
Всё дело - в её батяне: он сам ловил
Севастопольской хваткой Чехова звёзды по мере сил
И рассыпал их в мемориалах, в пиалах, откуда чай
С детства пила Ассоль, плача и хохоча,
Когда на дне кипятковой кружки кантовская свеча
Начинала выделывать стрекача,
Особенно по ночам.
Господи, как же Ты их любил!
У Ассоли не было выбора: самомнение и скандал.
Бог, видя, что дело - плохо, терпения преподал.
Умер папа. Ассоль, старея, сделалась чуть добрей.
У моря выросли дети, внуки, - и стало их семь морей.
В старом жилфонде дна потрескались жерди рей.
Морякам обломалось вешаться, кончив рейс.
Чайки спадали, словно бельё с досочек на заре...
А потом появился Грей.
12 июня 2021 г.

\ifcmt
  tab_begin cols=2

     pic https://scontent-lga3-2.xx.fbcdn.net/v/t1.6435-9/198673163_3973781722656908_4659239578717233196_n.jpg?_nc_cat=103&ccb=1-3&_nc_sid=8bfeb9&_nc_ohc=VgqIRoZ3EFIAX_l0Xxn&tn=ntrKbsW_7ChXu3v-&_nc_ht=scontent-lga3-2.xx&oh=f2d9dfdeb139cdcac0d2b6eb8ee1e838&oe=60C985A2

     pic https://scontent-lga3-2.xx.fbcdn.net/v/t1.6435-9/199271588_3973781832656897_1273880116914418013_n.jpg?_nc_cat=109&ccb=1-3&_nc_sid=8bfeb9&_nc_ohc=ta3Ty9wRaSgAX-kPGFz&_nc_ht=scontent-lga3-2.xx&oh=91e56b012e6ad66b1a96f37e9b1fb593&oe=60CA0244

  tab_end
\fi

