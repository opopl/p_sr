% vim: keymap=russian-jcukenwin
%%beginhead 
 
%%file 02_08_2021.fb.volkov_vladimir.1.zelenskii_torg_umesten_kartina
%%parent 02_08_2021
 
%%url https://www.facebook.com/permalink.php?story_fbid=2678403028973071&id=100004102000556
 
%%author 
%%author_id volkov_vladimir
%%author_url 
 
%%tags devushka,kartina,ukraina,zelenskii_vladimir
%%title ЗЕЛЕНСКИЙ И УКРАИНА: "ТОРГ УМЕСТЕН".
 
%%endhead 
 
\subsection{ЗЕЛЕНСКИЙ И УКРАИНА: \enquote{ТОРГ УМЕСТЕН}.}
\label{sec:02_08_2021.fb.volkov_vladimir.1.zelenskii_torg_umesten_kartina}
 
\Purl{https://www.facebook.com/permalink.php?story_fbid=2678403028973071&id=100004102000556}
\ifcmt
 author_begin
   author_id volkov_vladimir
 author_end
\fi

ЗЕЛЕНСКИЙ И УКРАИНА: "ТОРГ УМЕСТЕН".

Картина художника Грибкова Сергея Иванович (1822-1893) «В лавке» 1882 года
представляет яркий сюжет, актуальность которого не меркнем с веками. Молодая
босоногая крестьянка с вожделением смотрит на украшения, предлагаемые ей
лавочником, осознавая, что никогда не сможет это купить. Тема социального
неравенства звучит в картине особенно явственно, и блестяще разыграна по ролям.
Фигура лавочника находится в тени и похожа на серую тень, а образ главной
героини вынесен на первый план, освещен солнцем. Зритель любуется обликом
молодой девушки в красивом украинском наряде, и не сразу замечает
льстиво-любезного лавочника.

\ifcmt
  pic https://scontent-cdg2-1.xx.fbcdn.net/v/t1.6435-9/229075475_2678402985639742_5493259168765615826_n.jpg?_nc_cat=100&ccb=1-5&_nc_sid=730e14&_nc_ohc=Mr5M82tmJHcAX8pdSjE&_nc_ht=scontent-cdg2-1.xx&oh=d3a1502a352036f59ed1c2198e9c9c1a&oe=6144E7A8
  width 0.4
\fi
