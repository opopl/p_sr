% vim: keymap=russian-jcukenwin
%%beginhead 
 
%%file 07_12_2020.fb.fb_group.story_kiev_ua.1.taras_shevchenko_pamjatnyk
%%parent 07_12_2020
 
%%url https://www.facebook.com/groups/story.kiev.ua/permalink/1533166546880141/
 
%%author Киевские Истории
%%author_id fb_group.story_kiev_ua
%%author_url 
 
%%tags 
%%title Пам'ятник Тарасу Шевченку
 
%%endhead 
 
\subsection{Пам'ятник Тарасу Шевченку}
\label{sec:07_12_2020.fb.fb_group.story_kiev_ua.1.taras_shevchenko_pamjatnyk}
\Purl{https://www.facebook.com/groups/story.kiev.ua/permalink/1533166546880141/}
\ifcmt
	author_begin
   author_id fb_group.story_kiev_ua
	author_end
\fi
\index[writers.rus]{Шевченко!Тарас, Григорович!Памятник}

Кожний народ має своїх національних героїв, які стали символами втілення
народного духу і боротьби за незалежність. В українському пантеоні таких осіб
поза сумнівом чільне місце належить Тарасові Шевченку.

Відразу після смерті великого сина України виникла ідея увічнення його пам’яті
у камені. Але до того дня, коли фігура Кобзаря нарешті з’явилася у парку
напроти університету, було аж п’ять невдалих спроб встановити пам’ятник
Шевченку в місті.

Вперше про створення пам’ятника заговорили ще у 1904р. Ініціатором виступило
земське зібрання міста Золотоноша на Черкащині. При київській міській Думі був
створений спеціальний комітет, почався збір коштів, але до задуму повернулися
тільки через два роки. Саме тоді офіційні представники Петербургу все ж дали
добро на увічнення пам’яті бунтівного поета. Був створений новий комітет по
встановленню пам’ятника Т. Шевченку у Києві. Одними з перших до нього увійшли
Микола Лисенко, Борис Грінченко, Михайло Грушевський. Очолив його міський
голова Іпполіт Дьяков.

Місцем для встановлення пам’ятника обрали невеликий скверик між комплексом
Михайлівського Золотоверхого монастиря і реальним училищем (зараз це будівля
Дипломатичної академії при МЗС України). Відкриття монументу планували
приурочити до 50-річчя з дня смерті поета, у 1911р. Влада, формально погодивши
на відкриття, доволі довго зволікала з останнім словом. Коли згода нарешті була
отримана, комітет оголосив конкурс на кращий пам’ятник. Розглянули більше аніж
півсотні проектів і навіть видали грошові премії переможцям. Однак надалі серед
членів комітету виникли деякі суперечки (наприклад, дехто виказав побажання
одягнути Шевченка в національний одяг). Долучився і Клуб російських
націоналістів, члени якого звинуватили українських діячів у сепаратизмі. Через
все це час згаяли, встановити пам’ятник на Михайлівській площі не вдалося, на
тому місці з’явився добре відомий киянам пам’ятник княгині Ользі, Кирилові і
Мефодію і Андрію Первозванному. Подейкують, що Іпполіт Дьяков висловився з
цього приводу так: «Кавалер должен уступить место даме».

Пам’ятник Шевченку запропонували поставити на Караваєвській площі (зараз площа
Льва Толстого). Але члени комітету по створенню пам’ятника з цим не погодились:
неохайний і захаращений закуток у центрі міста - а саме такою тоді була ця
площа - не може бути місцем увічнення такої видатної особистості. Як нову
локацію стали розглядати Бібіковський бульвар, Бесарабську площу,
Золотоворітський сквер, Петровську алею, Володимирську гірку, Театральну площу
і ще кілька місць. Внаслідок довгих дискусій комітет все ж змушений був
погодитись із пропозицією міської Думи і зупинитися у виборі на Караваєвській
площі.

У 1913р. був оголошений новий, вже четвертий конкурс. При визначенні переможця
знову виникли суперечки, а Перша світова війна і взагалі відсунула реалізацію
на невизначений час. Сумна доля спіткала і кошти, зібрані на встановлення
пам'ятника: їх було конфісковано та витрачено на інші справи

В роки воєнних дій і численної зміни влади у Києві достойно увічнити Шевченка
не встигли ані Грушевський, ані Петлюра, ані Скоропадський. В жовтні 1918р.
нова влада спробувала реанімувати роботу комітету із встановлення пам’ятника,
але численні засідання і гарячі обговорення врешті звели його діяльність
нанівець.

Чимось подібним до пам’ятника був виготовлений з фанери бюст поета роботи
Бернарда Кратко на місці, де пам’ятник колись планувався – на Михайлівської
площі, на рештках знесеного пам’ятника княгині Ользі (фігуру якої скинули,
розбили, а уламки закопали поряд). Щодо бокових постатей Андрія Первозванного
та просвітників Кирила і Мефодія, то їх сховали під дерев’яними кубами.

Наступною спробою було невелике гіпсове погруддя роботи скульптора Федора
Балавенського – його 1 травня 1919р. на нинішній Європейській площі (напроти
міського музею) встановила пролетарська влада. Але вже наприкінці серпня, коли
місто захопили війська Денікіна, бюст був розбитий в процесі нищення усіх
слідів перебування в Києві більшовиків.

А з 1923р. знову почалися тривалі розмови про встановлення достойного монументу
в центрі міста. Тоді вперше і прозвучала пропозиція зробити це у сквері напроти
університету на місці поваленого пам’ятника імператору Миколі І.

За сучасний пам’ятник Тарасові Шевченку, який сьогодні стоїть у парку його
імені, взялися лише у 1930-ті. Роботу доручили ленінградському скульптору
Матвію Манізеру, автору пам’ятника Шевченку, вже відкритого на той час у
Харкові.

Первісний варіант пам’ятника був інший, цікавіший за образним вирішенням,
динамічніший, пластично виразніший. Пам’ятник міг бути багатофігурним, як,
наприклад, пам’ятник поетові у Харкові того ж автора.

У постаменті мала бути барельєфна композиція на тему поеми «Гайдамаки». У
другому варіанті – ті ж персонажі вже окремими фігурами біля підніжжя
пам’ятника. Твір стояв уже повністю готовий для відливання у бронзі у майстерні
М. Манізера в Ленінграді, коли в справу втрутилося більшовицьке керівництво на
чолі з заступником голови РНК СРСР Л.Кагановичем.

За спогадами канадського скульптора українського походження Л. Молодожанина,
який тоді навчався в Академії мистецтв у М. Манізера, скульптору наказали
прибрати гайдамаків, а постать Шевченка переробити так, щоб він виглядав
зажуреним-засмученим-задумливим. Авторові дорікали, що повсталих гайдамаків
можна трактувати як протест сучасного (тобто 1930-х рр.) українського селянства
проти колгоспного ладу і навіть проти радянської влади. Л.Каганович «порадив»
Манізеру зробити постать Т.Шевченка із закладеними за спину руками, на які
повішено важкий одяг, і щоб голова поета була опущена. Тому скульптор був
змушений спростити і примітизувати пам’ятник. Саме в такому вигляді його було
відлито на Ленінградському заводі художнього лиття і таким його затвердив новий
керівник УРСР Микита Хрущов.

Бронзову 7-метрову статую поета на постаменті з червоного граніту урочисто
відкрили 6 березня 1939 року.

Опальний Лев Троцький у статті «Об украинском вопросе» написав: «Сталинская
бюрократия возводит памятники Шевченко, но с тем, чтобы покрепче придавить этим
памятником украинский народ и заставить его на языке Кобзаря слагать славу
кремлевской клике насильников»...

\ifcmt
tab_begin cols=2
	caption Памятники Кобзарю в Киеве

%1
pic https://scontent.fiev6-1.fna.fbcdn.net/v/t1.0-9/130299555_755591228363915_8490219185339348150_o.jpg?_nc_cat=101&ccb=2&_nc_sid=b9115d&_nc_ohc=y8--ONBsU7MAX9_mkDB&_nc_ht=scontent.fiev6-1.fna&oh=538140a9fe7bd11a9e98e8e8e7341667&oe=5FF5EFED
caption 1913 рік. Обраний комітетом проект Антоніо Шортіно.
width 0.3

%2
pic https://scontent.fiev6-1.fna.fbcdn.net/v/t1.0-9/130499092_755591278363910_1630800495884540185_o.jpg?_nc_cat=102&ccb=2&_nc_sid=b9115d&_nc_ohc=1Ic_He6ejkwAX80uC87&_nc_oc=AQkbFHt6k6DIb7qnPpAqFBIGFDqVjvJarpS1IQ8g04ArPdVmeddb0b1RO6m9V4gQQ3s&_nc_ht=scontent.fiev6-1.fna&oh=478225e185d2aa40e4e3a32a5747b828&oe=5FF580CB
caption Такою у 1900-х роках була Караваєвська площа, на якій збирались встановити пам’ятник поету.

%3
pic https://scontent.fiev6-1.fna.fbcdn.net/v/t1.0-9/130686032_755591381697233_8845223622627564903_n.jpg?_nc_cat=110&ccb=2&_nc_sid=b9115d&_nc_ohc=jeyaWbPQHu4AX89_Dru&_nc_ht=scontent.fiev6-1.fna&oh=44804f93e571eca7f1642600c161c001&oe=5FF45C5F
caption 1919 рік. Знесення пам’ятника княгині Ользі

%4
pic https://scontent.fiev6-1.fna.fbcdn.net/v/t1.0-9/130274035_755591501697221_90887913020564128_o.jpg?_nc_cat=103&ccb=2&_nc_sid=b9115d&_nc_ohc=sXSffM3-UHAAX-zVD4v&_nc_ht=scontent.fiev6-1.fna&oh=385ef74d8978448a62941825ce8b57d0&oe=5FF2D7A0
caption 1918 рік. Відкриття бюсту Т. Шевченка на Михайлівській площі на місці знесеного пам’ятника княгині Ользі.
width 0.3

%5
pic https://scontent.fiev6-1.fna.fbcdn.net/v/t1.0-9/130307275_755591565030548_4540429129546623935_o.jpg?_nc_cat=101&ccb=2&_nc_sid=b9115d&_nc_ohc=KBHf6EtYYloAX8IdIut&_nc_ht=scontent.fiev6-1.fna&oh=54f9e9f919fb97402ce299925b887f41&oe=5FF44744
caption 31 серпня 1919 року. Розбитий денікінцями бюст Т. Шевченка на Європейській площі.

%6
pic https://scontent.fiev6-1.fna.fbcdn.net/v/t1.0-9/130403990_755591628363875_4121649297607623715_o.jpg?_nc_cat=101&ccb=2&_nc_sid=b9115d&_nc_ohc=xUuA9idfj74AX-ANfTC&_nc_ht=scontent.fiev6-1.fna&oh=df5d53dd4af5a5edcf6917bec8804f73&oe=5FF4E35F
caption 1936 рік. Початковий проект М. Манізера.

%7
pic https://scontent.fiev6-1.fna.fbcdn.net/v/t1.0-9/130307259_755591795030525_5864466884513512769_n.jpg?_nc_cat=108&ccb=2&_nc_sid=b9115d&_nc_ohc=n5Fw0uYS3PEAX-j2mrV&_nc_oc=AQno32WnfT4MTrhH3zzTONj6p0Pi2KsYBkeGBZCWL_wBj4Q6KvYKwdWyNQoLQxfG4Kg&_nc_ht=scontent.fiev6-1.fna&oh=1ef220dd4b5f1c0d9471569218504468&oe=5FF4748C
caption Другий варіант проекту М. Манізера.

%8
pic https://scontent.fiev6-1.fna.fbcdn.net/v/t1.0-9/130731651_755591858363852_7787689418184113607_n.jpg?_nc_cat=104&ccb=2&_nc_sid=b9115d&_nc_ohc=xTnQgvSTGa0AX9nJteh&_nc_ht=scontent.fiev6-1.fna&oh=3e9e5dab1c326fd48da524e12f876c9e&oe=5FF28A37
caption Матвій Генріхович Манізер (1891-1966).

tab_end
\fi

