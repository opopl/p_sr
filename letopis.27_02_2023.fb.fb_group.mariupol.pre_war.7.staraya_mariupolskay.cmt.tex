% vim: keymap=russian-jcukenwin
%%beginhead 
 
%%file 27_02_2023.fb.fb_group.mariupol.pre_war.7.staraya_mariupolskay.cmt
%%parent 27_02_2023.fb.fb_group.mariupol.pre_war.7.staraya_mariupolskay
 
%%url 
 
%%author_id 
%%date 
 
%%tags 
%%title 
 
%%endhead 

\qqSecCmt

\iusr{Татьяна Кизимова(Козаченко)}

Она нас спасла, пока не уехали 19 марта. Но уже опасно было ходить из-за
обстрелов. Сын 18 марта чудом остался жив.

\iusr{Марина Клок}

А где она находится?

\begin{itemize} % {
\iusr{Олена Сугак}
\textbf{Марина Клок} ул Малофонтанная. Вниз от Фонтанной, по лестнице.

\iusr{Марина Клок}
\textbf{Олена Сугак} благодарю

\iusr{Vadim Melnik}
\textbf{Марина Клок} 

где ПЭС знаете за досаафом? Он как бы на возвышенности а это за ПЭС, вниз за
частный сектор, это будет Малофонтанная улица. Метров 100 влево по улице и там
источник.

\iusr{Марина Клок}
\textbf{Vadim Melnik} поняла. Благодарю.
\end{itemize} % }

\iusr{Alexandra Osovskaya}

Я и сейчас из нее набираю, уж получше, чем вода из крана ...

\iusr{Таня Примак}

Под обстрелами бегали за водой к ней спасибо криничка, что спасала людей

\iusr{Вика Ермолаева}

Тоже Малофонтанная, только немножко дальше. Март 2022.

\ifcmt
  igc https://scontent-frt3-2.xx.fbcdn.net/v/t39.30808-6/333040066_519488933673521_4108052245324997248_n.jpg?_nc_cat=108&ccb=1-7&_nc_sid=dbeb18&_nc_ohc=2LhvvUM4IvsAX_CpyRP&_nc_ht=scontent-frt3-2.xx&oh=00_AfBZ3u2WSEsKgnMiPpvGqCdI4N0oYL4O7T6PWupxErjbKg&oe=642A5C42
	@width 0.6
\fi

\begin{itemize} % {
\iusr{Олена Сугак}
\textbf{Вика Ермолаева} это возле старой подстанции

\begin{itemize} % {
\iusr{Вадим Коробка}
\textbf{Олена Сугак} Тут близько було перша міська електростанція, збудована 1909р.

\iusr{Олена Сугак}
\textbf{Вадим Коробка} це не вона?

\ifcmt
  igc https://scontent-frt3-2.xx.fbcdn.net/v/t39.30808-6/334049292_993035334994380_4034671482528906316_n.jpg?_nc_cat=110&ccb=1-7&_nc_sid=dbeb18&_nc_ohc=OVMTesrqTzgAX9sjrKG&_nc_ht=scontent-frt3-2.xx&oh=00_AfDQZo0c9Qhd38_nMtovFTHvG7HdAN7TV47N4mcM6bqiyg&oe=642B33EE
	@width 0.6
\fi

\iusr{Вадим Коробка}
\textbf{Олена Сугак} Не можу сказати. Ту електростанцію висадили в повітря перед відступом РСЧА з Маріуполя в 1941 р. Я в 1960-х бачив її рештки.

\iusr{Олена Сугак}
\textbf{Вадим Коробка} 

Казали що це стара водокачка, яка ще працює постачаючи воду на Гавань та Аджахи.
До неї ведуть білі сходи. Але все було за парканом... Тому подивитися що там
найсправді, не смогла.

\iusr{Вадим Коробка}
\textbf{Олена Сугак} Кругла будівля, ймовірно, - каптаж у ній пристрій, що дозволяють збирати та виводити підземні води на поверхню для їхнього використання. Думаю, що будувавалась за пректом Нільсена. Утім, колись чув, що чи то будівництва, чи то до їх відновлення після ДСВ долучився архітектор та художник Микола Йосипович Нікаро-Карпенко.

\ifcmt
  igc https://scontent-fra3-1.xx.fbcdn.net/v/t39.30808-6/332904746_936911067479807_7350200194477075931_n.jpg?_nc_cat=104&ccb=1-7&_nc_sid=dbeb18&_nc_ohc=5FtxIDU4umoAX_avqUv&_nc_ht=scontent-fra3-1.xx&oh=00_AfDcIqGmcNU0XL7VHC3_qdzAmdGJOazxMADsMtdd2aScww&oe=6429ACCD
	@width 0.4
\fi

\end{itemize} % }

\iusr{Вадим Коробка}

Віка!, Хто автор фото?

\begin{itemize} % {
\iusr{Вика Ермолаева}
\textbf{Вадим Коробка}, автор - Ермолаев Иван, мой сын. Сфотографировал, как я с мужем черпаю...

\iusr{Вадим Коробка}
\textbf{Вика Ермолаева} Яке число?

\iusr{Вика Ермолаева}
\textbf{Вадим Коробка}, 8 марта.

\iusr{Олена Сугак}
\textbf{Вика Ермолаева} фото б треба в групу \textbf{Маріуполь історичний}. З датою і місцем для історії

\iusr{Вадим Коробка}
\textbf{Вика Ермолаева} Чи можна поширити в групі?

\iusr{Вика Ермолаева}
\textbf{Вадим Коробка}, так.
\end{itemize} % }

\end{itemize} % }

\iusr{Марина Бусенко}

А нас спасла креничка у соседа во дворе по подгорной улице. каждый раз молилась
чтоб из за обстрелов не пострадал наш источник. много людей приходило поводу. а
17 попал снаряд многих убило

\iusr{Hanna Krushynska}

Мы набирали воду в криничке на Львовской, элеваторские дома. Вода там
горьковато-солоноватая.

\begin{itemize} % {
\iusr{Наталия Павлова}
\textbf{Hanna Krushynska} жуткая вода.

\iusr{Hanna Krushynska}
\textbf{Наталия Павлова} ну однако ничего не случилось, привыкаешь к вкусу, она хотя бы чистая. Только с 1 апреля уже нельзя было ходить, такие плотные обстрелы начались

\iusr{Олена Сугак}
\textbf{Наталия Павлова} а дождевая ?

\iusr{Наталия Павлова}
\textbf{Hanna Krushynska} 

у меня потом в Эстонии и дальше в лагере в Германии были жуткие почёчные
приступы. Утром в Нарве проснулись в гостинице, надо вставать на автоьус, а я
не могу пошевелиться. Ношпа и нимесил как то спасли. Мой муж иногда проявочет
какие то предвидческие способности. За неделю др войны он начал кипятить и
собирать воду в 20 литровый баклагах. А также на улице набрал две кастрюли
дождевой воды. Потом эти баклаги мы стянули в погреб. Былр примерно 250 литров
плюс дождевая. А у нас большая собака. Ей кашу надо варить. На костре. После 31
марта варили прямо в подвале на самогоне манку, потому что она быстро варилась.
Так что во время обстрелов нам не пришлось ходить за водой. Самое страшное у
нас началось с 31 марта. Собаки проходили к люку подвала на кухне прямо через
дыру в стене и выбитые двери. Наша и соседская. Пришлось и дожлевуб пить. Она
замёрзла. Мы её отбивали в 3 часа ночи в полной темноте, вокруг рыскали орки и
днровцы. А воду из криницы в первый раз приносили уже после того, как орки
заняли Черемушки. А наш сосед как раз попал под самолёты обстрел ракетами ул.
Федорова. Видел, как летели две серебристые ракеты и попали в дом за 100 метров
от него. Когда мы могли выскочить из подвала, соседи нам варили макароны в
своём подвале на печке.

\iusr{Hanna Krushynska}
\textbf{Олена Сугак} 

у нас был колодец с дождевой водой, хватило на две недели, дождей ведь не было
почти, а набрать его под завязку из шланга пока была водопроводная вода я не
догодалась, потом ходили на криничку, но это всегда была рулетка

\iusr{Наталия Павлова}
\textbf{Олена Сугак} собрали дожлевую доя собак в основном ещк в первые дни войны. Тогда дожди все время были. А она замёрзла в начале марта до дна. Воду мы накопили и ходить ща ней пришлось уже после 12 апреля. После кипячёной двухсантиметровый слой серого осадка. И горько солёный вкус. Пили её почти три недели. Мне хватило.

\iusr{Hanna Krushynska}
\textbf{Наталия Павлова} 

наверное это были две ракеты как раз в наш дом, потому что летели с Черёмушек.
У меня тоже сейчас периодически почечные приступы, даром не прошло. Хоть мы её
и кипятили с содой сначало, потом отстаивали и сливали с осадка, потом уже
готовили на ней и заваривали чай. Ещё дома были кварц, шунгит и кремень,
настаивали воду, вкус был лучше. Дома была печка, жгли книги, у нас была
большая библиотека, всю жизнь мама и я собирали. Одной книги хватало чтобы
закипятить чайник.

\iusr{Наталия Павлова}
\textbf{Hanna Krushynska} а нам хватало трех крышек самогона 80 градусов, чтобы пожарить 6 лепешек. Ещё сварить собаке манку и нам макароны. Пока мы сообразила, как готовить еду после 31 марта, когда нельзя былр голову из подвала высунуть мы не ели 7 дней. Мед, вино сухое, изюм.

\end{itemize} % }

\iusr{Maksim Bondarenko}

Никогда её не видел (

\iusr{Вадим Коробка}

До 1911 р. водопостачання маріупольців здійснювалося з природного джерела на
вул. Малофонтанній – «Великого фонтану», над яким було зведено кам’яну
водорозбірну споруду, що мала три вмістилища для питної води, прання та водопою
худоби. Варте уваги те, що забір води був безкоштовний. Обкладанню на користь
міста підлягали тільки тачки, на яких здійснювалась доставка споживачам
ємностей з водою (5 руб. на рік з одиниці). У 1908 р. саме таких візків в
Маріуполі нараховувалось майже 220. До речі, з цього випливає, що певна група
міських жителів займалася водовозним промислом. Проте такий примітивний спосіб
водопостачання не міг безконечно задовольняти всіх городян й міське
самоврядування.

Перед початком будівництва міського водопроводу 15 вересня до 8 жовтня 1909 р.
під керівництвом В. Нільсена шляхом відкачки вод було встановлено, що
середньодобовий дебіт (потужність) «Великого фонтану» складає 193000 відер (1
відро = 12,299 л) або 2374 м3. На підставі цього показника міська управа дійшла
висновку про те, що такий обсяг води достатній для населення міста «не тільки в
теперішній час, але й у майбутньому».

Міський водопровід почав функціонувати 12 березня 1911 р. воду почали
відпускати за плату 7 діючих водорозбірних будок (3 ще були недобудовані).
Одночасно для городян був закритий забір води з природного джерела «Великий
фонтан». Становище недоступності до виходу підземних вод мало примусити
водовозів й значну частину городян користуватися платною водопровідною водою.
Саме 12 березня в газеті «Мариупольская жизнь» було вміщено оголошення міської
управи, яким маріупольці оповіщались про те, що з 12 березня 1911 р. вода буде
відпускатися із водорозбірних будок за розцінкою: 1/4 коп. з відра, за водопій
худоби по 1 коп. з голови. Пиття кружками із особливих кранів було
безкоштовним.

\iusr{Сергей Лукьянчук}

Вкусная там вода помню

\iusr{Leonid Ehdelshteyn}

Хорошая вода в той криничке, на улице Малофонтанной. Когда жил на Слободке,
иногда брал эту воду. 20 - 30 лет назад. Приезжал на тачке с бидоном. Добрая
вода. И до людей добрая. Страшно читать о том, как погибали люди, которые
просто шли за водой. А многие пили воду из батарей или топили снег, когда он
был. Даже во время Второй Мировой, ни наши ни немцы, никогда не стреляли в
водоносов, если те набирали, по очереди, воду из одного источника. Сколько лет
криничке, никто не знает?

\iusr{Наталія Артюх}

Нас тоже спасала, один раз чуть не погибли наши мужчины

\iusr{Андрей Грибоедов}

После прилёта она чуть изменилась там мою знакомую осколком тогда ранело в ногу
и голову хорошо наши её нашли и спасли

\iusr{Андрей Грибоедов}

Мы воду с подвала черпали а так бегали под обстрелами до последнего два раза
чуть под авио бомбы не попали они перед нами упала одна а на следующий день
также упала вторая как ити от швейки по дороге

\iusr{Андрей Грибоедов}

Срачька правда от той выды была у всего Мариуполя 😆и пох главное было выжить у
нас 60+человек сидело а воду и еду добывали 4,5 человек максимум ну и + 20
грудничков было но они эвакуировались дольнейшая судьба их не известна

\iusr{Андрей Грибоедов}

Если КТО ЕСТЬ КТО СИДЕЛ НА (ПОЖ,ЗАЩИТЕ ) ПИШИТЕ В ЛИЧЬКУ ХОЧУ ЗНАТЬ ЧТО ВСЕ КТО ЭВАКУИРОВАЛСЯ ЖИВЫ!!
