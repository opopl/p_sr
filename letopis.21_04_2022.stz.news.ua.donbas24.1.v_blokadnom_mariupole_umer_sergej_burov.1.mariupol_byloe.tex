% vim: keymap=russian-jcukenwin
%%beginhead 
 
%%file 21_04_2022.stz.news.ua.donbas24.1.v_blokadnom_mariupole_umer_sergej_burov.1.mariupol_byloe
%%parent 21_04_2022.stz.news.ua.donbas24.1.v_blokadnom_mariupole_umer_sergej_burov
 
%%url 
 
%%author_id 
%%date 
 
%%tags 
%%title 
 
%%endhead 

\subsubsection{\enquote{Мариуполь. Былое}}

Его знали как неравнодушного, теплого и доброго человека, который никогда не
отказывал в помощи и консультациях. Отличительной чертой Сергея Бурова была
скромность. Свое время краевед посвящал выпускам еженедельной программы
\enquote{Мариуполь. Былое} и публикациям. Телепрограмма, которую хоть раз посмотрел
едва ли не каждый горожанин, выходила на ТК \enquote{Сигма} на протяжении 27 лет.

\ii{21_04_2022.stz.news.ua.donbas24.1.v_blokadnom_mariupole_umer_sergej_burov.pic.2}

\begin{leftbar}
	\begingroup
		\bfseries
\qbem{Его авторская программа выходила в эфир с 24 июля 1994 года. Он снимал
ее вместе с моим мужем Виктором Дедовым, который погиб в Мариуполе 11
марта 2022 года. Они вместе работали и сняли очень много программ. Он
был очень замечательным, интеллигентным, умным, очень тактичным и
стеснительным. Всегда извинялся, если просил нас помочь ему скачать
видео. Был удостоен звания \enquote{Мариуполец года}. Очень любил город и его
жителей, всегда находил интересные истории о художниках, театралах,
заводчанах, снимал сюжеты об улицах города, домах и их историях. В
октябре 2020 года мы проводили Сергея Бурова на пенсию, это был как раз
опасный период пандемии коронавируса}, — рассказала выпускающий
редактор, журналист, режиссер, ведущая телеканала \enquote{Сигма} Наталья
Дедова.
	\endgroup
\end{leftbar}

