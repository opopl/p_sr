% vim: keymap=russian-jcukenwin
%%beginhead 
 
%%file 17_04_2022.fb.malyshko_irina.1.mariupol_muzej
%%parent 17_04_2022
 
%%url https://www.facebook.com/irina.malyshko/posts/5132874920084687
 
%%author_id malyshko_irina
%%date 
 
%%tags 
%%title Мариуполь - это огромная историческая потеря
 
%%endhead 
 
\subsection{Мариуполь - это огромная историческая потеря}
\label{sec:17_04_2022.fb.malyshko_irina.1.mariupol_muzej}
 
\Purl{https://www.facebook.com/irina.malyshko/posts/5132874920084687}
\ifcmt
 author_begin
   author_id malyshko_irina
 author_end
\fi

Не могу в себя прийти. Мариуполь - это огромная историческая потеря. Это
уничтоженный пласт нашей истории, о которой, к сожалению, мы так мало знали. 

Я много ездила по стране, но Мариупольский краеведческий.., в него было вложено
столько любви, столько уважения к прошлому, в Марике смогли сохранить очень
много, там хранились уникальные документы по Мариупольскому уезду. У них был
потрясающий отдел этнографии. Восток Украины - это мультикультурный котел,  это
и греческая, и немецкая, и еврейская, и итальянская история. Евреев в Марике до
1941 проживало больше, чем греков. В музее хранились уникальные документы по
Мариупольскому уезду.  И всего этого нет. Этот город должны оплакивать все - и
украинцы, и немцы, и греки, и евреи, и итальянцы. 

\ii{17_04_2022.fb.malyshko_irina.1.mariupol_muzej.pic.1}

Все греческое наследие уничтожено. Уничтожены все материальные носители
истории, остатки того, что можно было сохранить.  Остатки того, что не успела
уничтожить Советская власть. Ведь в годы революции 1917 погибли почти все
православные святыни, вывезенные греками из Крыма. Нет уже ни слов, ни слез.  У
меня такое чувство, что с этим музеем  и этим городом умерла часть меня.

\ii{17_04_2022.fb.malyshko_irina.1.mariupol_muzej.pic.2}

на фото, 

убранство немецкого  дома - к Мариупольскому уезду относился Мариупольский
немецкий колонистский округ и Мариупольский меннонитский округ. 

остатки Мариупольской синагоги

Греческий рыбный завод, рыбные промысли держали исключительно греки, но
рыбаками были - украинцы,

дворик в краеведческом музее.

Теперь думаю - почему я не снимала каждый свой шаг.

\ii{17_04_2022.fb.malyshko_irina.1.mariupol_muzej.cmt}
