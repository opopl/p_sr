% vim: keymap=russian-jcukenwin
%%beginhead 
 
%%file 28_02_2022.fb.reznichenko_nina.kiev.1.spali_doma
%%parent 28_02_2022
 
%%url https://www.facebook.com/nina.athena.wac/posts/5053916081318194
 
%%author_id reznichenko_nina.kiev
%%date 
 
%%tags __feb_2022.vtorzhenie
%%title Спали дома, было тихо
 
%%endhead 
 
\subsection{Спали дома, было тихо}
\label{sec:28_02_2022.fb.reznichenko_nina.kiev.1.spali_doma}
 
\Purl{https://www.facebook.com/nina.athena.wac/posts/5053916081318194}
\ifcmt
 author_begin
   author_id reznichenko_nina.kiev
 author_end
\fi

Спали дома, было тихо.

Сегодняшний день был посвящен лекарственному забегу. Под сирены вдалеке. Где-то
в моем любимом Киеве шли бои в это время.

\ii{28_02_2022.fb.reznichenko_nina.kiev.1.spali_doma.pic.1}

Маршрут был Дарницкий жд вокзал - Дарницкая площадь - Русановская набережная -
метро Левобережная - Дарницкий жд вокзал. В двух местах мне передали добрые
люди по пластинке моего антидепрессанта, в третьем в очереди в аптеку стояла
наша Валя, а я ей несла онкологический препарат. 

\ii{28_02_2022.fb.reznichenko_nina.kiev.1.spali_doma.pic.2}

Многие магазины и редкие аптеки работают, в супермаркеты и аптеки огромные
многочасовые очереди. В маленькие магазинчики попасть попроще, но там мало что
есть. На Азербайджанской частично окрыт базарчик. Трамваи ходят кое-как, видела
несколько, автобусы - совсем мало, но тоже есть.

Встретила на Левобережной зоотовары без очереди (о, чудо!), прихватила пару кг
котам. 

Русановку охраняют, мосты перекрыты, всюду наши ребята в желтых повязках,
молодцы. А с нашей стороны почему-то патрулей мало. 

На обратном пути на Дарницкой площади встретила мародерствующего бухаря, не
нашла кого на него натравить, а сама уже еле ползла, наорала, но он не обратил
внимания.

Три часа быстрого и не очень шага - и вот я дома. Совершенно мокрая. Болит всё
(сказывается год лечения и малоподвижного образа жизни). Судороги в стопах и
голенях, спина огнем горит. Но я вырвала себе 10 дней без клинической
депрессии. Можно жить!

К сожалению, многие онкопациенты не могут сказать о себе того же, поскольку
из-за войны полностью отрезаны от лекарств и возможности получения лечения.

Русские солдаты, валите домой, вам здесь не рады!!!

\ii{28_02_2022.fb.reznichenko_nina.kiev.1.spali_doma.cmt}
