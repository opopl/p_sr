% vim: keymap=russian-jcukenwin
%%beginhead 
 
%%file 24_02_2023.stz.news.ua.donbas24.1.rik_vijny
%%parent 24_02_2023
 
%%url https://donbas24.news/news/rik-povnomasstabnoyi-viini-podiyi-yaki-uviisli-v-istoriyu-ukrayini
 
%%author_id demidko_olga.mariupol,news.ua.donbas24
%%date 
 
%%tags 
%%title Рік повномасштабної війни: події, які увійшли в історію України
 
%%endhead 
 
\subsection{Рік повномасштабної війни: події, які увійшли в історію України}
\label{sec:24_02_2023.stz.news.ua.donbas24.1.rik_vijny}
 
\Purl{https://donbas24.news/news/rik-povnomasstabnoyi-viini-podiyi-yaki-uviisli-v-istoriyu-ukrayini}
\ifcmt
 author_begin
   author_id demidko_olga.mariupol,news.ua.donbas24
 author_end
\fi

\ii{24_02_2023.stz.news.ua.donbas24.1.rik_vijny.pic.front}
\begin{center}
  \em\bfseries\Large
Рік тому, 24 лютого, розпочалася повномасштабна кривава війна в Україні
\end{center}

Пройшов рік з початку повномасштабного вторгнення Росії в Україну. Цей рік,
сповнений важливих дат і трагедій, переламних подій і омріяних зустрічей. Він
точно змінив кожного українця, адже ми вчилися під звуки сирен без тепла, води,
світла та зв'язку. Це був рік сліз, мужності, болю, надії, витримки, єдності та
незламності. Українці об'єднались, вистояли та показали всьому світу приклад
єдності та сили, які є у нашої країни. Донбас24 зібрав головні подій, які за
цей рік запам'яталися найбільше та точно увійдуть в підручники з історії
України.

\ii{insert.read_also.jana_ivanova.ljudy_zdyvuvaly_za_rik_vijny_dobrym_sercem}

\subsubsection{Кінець лютого 2022 року}

Рівно рік тому о 4:55 ранку \textbf{24 лютого} Україна прокинулася від перших ворожих
ракет на своїй території — путін оголосив, що ухвалив рішення про початок
проведення \enquote{спецоперації} на Донбасі після прохання угруповань \enquote{днр} та \enquote{лнр}.
Звуки вибухів чули жителі багатьох міст України, зокрема Харкова, Одеси та
Києва. Того ж дня у Верховній Раді депутати проголосували й ввели воєнний стан
на всій території України. Водночас міністр оборони закликав усіх, хто вміє і
готовий тримати зброю, стати до лав територіальної оборони. Цей день вже
увійшов у історію, адже Україна остаточно розірвала дипломатичні відносини з
Росією.

Вже наступного дня, 25 лютого, стало відомо, що захисники Зміїного гідно
відповіли ворогові, який закликав прикордонників скласти зброю і здатися.
Загалом відомий вираз \enquote{Русский военный корабль, иди нах*й!} став лейтмотивом
подальшого спротиву України. До речі, саме його обрали словом року. Крім того,
лютий запам'ятався знищенням української \enquote{Мрії}. \textbf{27 лютого} росіяни обстріляли
аеропорт \enquote{Антонов} під Києвом.

\href{https://archive.org/details/video.25_02_2022.ukrainska_pravda.russkij_voennyj_korabl_idi_nahuj}{%
Відео: \enquote{Русский военный корабль, иди нах@й} - прикордонники Зміїного росіянам, Українська правда, 25.02.2022}%
\footnote{\url{https://archive.org/details/video.25_02_2022.ukrainska_pravda.russkij_voennyj_korabl_idi_nahuj}} %
\footnote{\url{https://www.youtube.com/watch?v=LDrFVdms8yk}}

\ifcmt
  ig https://i2.paste.pics/PSHUC.png?trs=1142e84a8812893e619f828af22a1d084584f26ffb97dd2bb11c85495ee994c5
  @wrap center
  @width 0.9
\fi

\subsubsection{Весна 2022 року}

У березні масованих атак рф зазнавали багато українських міст. Дата 9 березня
стала трагічною в історії Маріуполя, адже окупанти скинули бомби на міський
пологовий будинок. Поранення отримали щонайменше 17 людей, зокрема вагітні
жінки. А 13 березня у повітряному бою над Києвом загинув майор Степан
Тарабалка, якого українці вважали \enquote{Привидом Києва}, символом хороброго
захисника українського неба. 16 березня російський літак скинув авіабомбу на
будівлю Маріупольського драматичного театру, де ховалися сотні людей, а поруч з
будівлею був напис \enquote{Діти}. Ця трагедія й досі відгукується у серцях
маріупольців, адже наприкінці 2022 року окупанти знесли пошкоджену будівлю, яка
колись була серцем культурного життя Маріуполя.

\textbf{Читайте також:} \href{https://donbas24.news/news/tvarini-geroyi-ukrayini-yak-cotirilapi-ryatuyut-ta-dopomagayut-podolati-voroga-foto}{\emph{Тварини-герої: як чотирилапі рятують та допомагають подолати ворога}}%
\footnote{Тварини-герої: як чотирилапі рятують та допомагають подолати ворога, Яна Іванова, donbas24.news, 22.02.2023, \par%
\url{https://donbas24.news/news/tvarini-geroyi-ukrayini-yak-cotirilapi-ryatuyut-ta-dopomagayut-podolati-voroga-foto}%
}

\ii{24_02_2023.stz.news.ua.donbas24.1.rik_vijny.pic.1}

У квітні були і щасливі дати. Зокрема, \textbf{2 квітня} ЗСУ вдалося звільнити всю
Київську область від російських загарбників. За кілька годин після деокупації
весь світ дізнався про звірства росіян у Бучі, Ірпені та Гостомелі. Навесні
росіяни продовжували вчиняти геноцид. Так, \textbf{8 квітня} ворожа ракета поцілила у
залізничний вокзал Краматорська, де на евакуацію очікували сотні людей,
переважно жінки з дітьми, в результі чого загинула 61 людина та більше сотні
отримали поранення. \textbf{12 квітня} СБУ затримала проросійського політика та кума
президента рф Віктора Медведчука. Це була досить радісна подія для всіх, що
стало привідом для створення популярних мемів. А \textbf{13 квітня} відбулося \enquote{бойове
хрещення} української крилатої ракети \enquote{Нептун}. Попри погану погоду вдалося
потопити крейсер \enquote{Москва}. Знищення крейсера \enquote{Москва} дозволило Україні перейти
до подальшої операції з деокупації острова Зміїний. Флот рф втратив флагман —
уперше з часів війни з Японією. Наступного дня, \textbf{14 квітня}, Сполучені Штати
Америки анонсували передачу Україні сучасних гаубиць М777. На той момент Захід
відмовлявся постачати важкі зразки озброєння ЗСУ, щоб уникнути ескалації
конфлікту. 155-мм гаубиця \enquote{пробила} перший невидимий бар'єр, відкривши шлях для
інших артилерійських установок. Ще однією радісною новиною стала перемога
України на Євробаченні-2022 в Турині. \textbf{15 травня} італійська сцена підкорилася
гурту Kalush Orchestra, які, попри діючі заборони, закликали зі сцени врятувати
захисників Маріуполя.

\textbf{Читайте також:} \href{https://donbas24.news/news/proisov-polon-pomer-komandir-iz-azovstali-oleg-mudrak}{\emph{Пройшов полон: помер командир із \enquote{Азовсталі} Олег Мудрак}}%
\footnote{Пройшов полон: помер командир із \enquote{Азовсталі} Олег Мудрак, Яна Квітка, donbas24.news, 23.02.2023, \par%
\url{https://donbas24.news/news/proisov-polon-pomer-komandir-iz-azovstali-oleg-mudrak}%
}

\ii{24_02_2023.stz.news.ua.donbas24.1.rik_vijny.pic.2}

\textbf{17 травня} стало відомо, що оборонці \enquote{Азовсталі} отримали від
вищого командування сигнал про право вийти з території меткомбінату, аби
зберегти своє життя. З \textbf{20 травня} сотні захисників перебували в
російському полоні.

\subsubsection{Літо 2022 року}

У червні Україна наблизилася до мети стати членом Європейського Союзу, адже \textbf{23
числа} разом із сусідньою Молдовою отримала статус кандидата на вступ. Цього дня
Україна отримала перші американські HIMARS. За місяць ці реактивні
артилерійські системи продемонстрували себе якнайкраще — знищили низку
логістичних об'єктів росіян. \textbf{7 липня} на острові Зміїний знову замайорів
український стяг. Свою втечу росіяни назвали \enquote{жестом доброї волі}. А \textbf{14 липня}
відбувся черговий теракт — росіяни обстріляли центр Вінниці. Загинуло
щонайменше 20 людей, понад півсотні отримали важкі поранення. Згодом \textbf{29 липня}
була цинічно обстріляна колонія в Оленівці Донецької області, де утримувалися
українські полонені й заявили, що \enquote{ЗСУ прицільно били по бараках}. А \textbf{20 серпня}
українці частково здійснили мрію країни-агресорки — на Хрещатику в Києві
провели \enquote{парад} російської техніки, проте вже знищеної.

\ii{24_02_2023.stz.news.ua.donbas24.1.rik_vijny.pic.3}

\textbf{Читайте також:} \href{https://donbas24.news/news/zittya-pid-postiinimi-obstrilami-u-baxmuti-shho-zrobili-rosiyani-iz-mistom-z-rik-foto}{\emph{Життя під постійними обстрілами у Бахмуті: що зробили росіяни із містом за рік}}%
\footnote{Життя під постійними обстрілами у Бахмуті: що зробили росіяни із містом за рік, Тетяна Веремєєва, donbas24.news, 14.02.2023, \par%
\url{https://donbas24.news/news/zittya-pid-postiinimi-obstrilami-u-baxmuti-shho-zrobili-rosiyani-iz-mistom-z-rik-foto}%
}

\subsubsection{Осінь 2022}

Відомо, що з початком осені ЗСУ мали одразу декілька значних перемог на полі
бою. Українські війська перейшли в контрнаступ на Харківщині й \textbf{8 вересня}
звільнили місто Балаклія. 10 вересня ЗСУ змогли звільнити Куп'янськ, а окупанти
покинули місто Ізюм на Харківщині, яке було окуповане ними з 2 квітня. Через
кілька днів у звільненому Ізюмі виявили масове поховання на понад 440 тіл.

Наприкінці вересня відбувся найбільший обмін полоненими. Так, \textbf{21 вересня}
Україна повернула 215 військових, серед яких були захисники Азовсталі. Серед
них: парамедикиня Катерина Поліщук з позивним Пташка, старший сержант Михайло
Діанов та\par\noindent Дмитро Козацький — Орест — автор популярного фото з Азовсталі.

\ii{24_02_2023.stz.news.ua.donbas24.1.rik_vijny.pic.4}

\textbf{7 жовтня} Україна вперше отримала Нобелівську премію миру. Нагороду присудили
українській організації \enquote{Центр громадянських свобод}. \textbf{8 жовтня} почався зі слів:
\enquote{Благодатний вогонь зійшов}. Цього дня зазнав пошкоджень Кримський міст — на
залізничній гілці сталася потужна пожежа. Внаслідок цього обвалилися дві
півсекції автомобільної частини мосту, а також сталася пожежа на його
залізничній частині. Ця атака ускладнила логістику російських військ на півдні
України та позитивно позначилася на моральному дусі українців. Росія та
особисто володимир путін звинуватили у вибуху на Кримському мосту українські
спецслужби. А \textbf{10 жовтня} відбувся перший масований ракетний обстріл Росією
енергетичної інфраструктури України. Російські війська запустили 84 ракети, 43
з яких збили сили ППО. У Росії назвали удари відповіддю на вибух на Кримському
мосту, а також заявили, що об'єкти інфраструктури нібито є \enquote{легітимною ціллю}.
Вже через тиждень, \textbf{17 жовтня}, рф цинічно атакувала Київ іранськими дронами. \textbf{11
листопада} сталася найщасливіша подія з початку повномасштабного вторгнення —
українські військові зайшли в Херсон, який росіяни окупували на початку
березня. Звільнене місто заполонила українська символіка, а місцеві жителі
розповідали, як ховали прапори у підвали і чекали українських військових.

\textbf{Читайте також:} \href{https://donbas24.news/news/z-kvitucix-mist-u-sucilni-ruyini-yak-zminilisya-sjevjerodoneck-ta-lisicansk-za-rik-viini-foto}{\emph{З квітучих міст — у суцільні руїни: як змінилися Сєвєродонецьк та Лисичанськ за рік війни}}%
\footnote{З квітучих міст — у суцільні руїни: як змінилися Сєвєродонецьк та Лисичанськ за рік війни, Наталія Сорокіна, donbas24.news, 24.02.2023, \par%
\url{https://donbas24.news/news/z-kvitucix-mist-u-sucilni-ruyini-yak-zminilisya-sjevjerodoneck-ta-lisicansk-za-rik-viini-foto}%
}

\ii{24_02_2023.stz.news.ua.donbas24.1.rik_vijny.pic.5}

\subsubsection{Зима 2022 — 2023 року}

У грудні президент вперше за 8 місяців залишив Україну й з візитом відвідав
США. \textbf{21 грудня} у Вашингтоні Володимир Зеленський провів двосторонню зустріч із
Джо Байденом, а \textbf{22 грудня} виступив у Конгресі США. Вранці \textbf{14 січня} російська
ракета влучила у багатоповерхівку в Дніпрі. Замість 72-х квартир залишилося
повітря і величезна порожнеча всередині у тих, хто втратив дім та рідних.
Загинуло 46 людей. Серед них — шестеро дітей. Українці ще не відійшли від
трагедії у Дніпрі, коли \textbf{18 січня} всіх сколихнула звістка про авіакатастрофу в
Броварах, де впав гелікоптер ДСНС. На борту перебувало керівництво Міністерства
внутрішніх справ, їхні помічники, члени екіпажу Ніжинського спеціального
авіаційного загону ДСНС.

Гелікоптер впав поблизу дитячого садка. Загалом загинуло 14 осіб, одна з них —
дитина.

Радісною для всіх українців став візит в Україну президента США — Джозефа
Байдена. Сталося це 20 лютого, коли українці вшановують пам'ять Героїв Небесної
Сотні. Це був перший візит президента США в Україну за останні 15 років.

\ii{24_02_2023.stz.news.ua.donbas24.1.rik_vijny.pic.6}

Текст підготовлено за наступними джерелами: focus.ua,\par\noindent life.pravda.com.ua,
radiosvoboda.org

Нагадаємо, раніше Донбас24 розповідав, які \href{https://donbas24.news/news/yaki-realni-sansi-peremogi-ukrayini-v-cyomu-roci-budanov-dav-ocinku}{\emph{реальні шанси перемоги}}%
\footnote{Які реальні шанси перемоги України в цьому році — Буданов дав оцінку, Яна Іванова, donbas24.news, 23.02.2023, \par\url{https://donbas24.news/news/yaki-realni-sansi-peremogi-ukrayini-v-cyomu-roci-budanov-dav-ocinku}}
України у війні вже цього року.

Ще більше новин та найактуальніша інформація про Донецьку та Луганську області
в нашому телеграм-каналі Донбас24.

Фото: з відкритих джерел.

\ii{insert.author.demidko_olga}
%\ii{24_02_2023.stz.news.ua.donbas24.1.rik_vijny.txt}
