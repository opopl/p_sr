% vim: keymap=russian-jcukenwin
%%beginhead 
 
%%file 2016.uinp.22_sichnja_den_sobornosti_ukrainy.2.istor_dovidka
%%parent 2016.uinp.22_sichnja_den_sobornosti_ukrainy
 
%%url 
 
%%author_id 
%%date 
 
%%tags 
%%title 
 
%%endhead 

\subsubsection{Історична довідка}

Текст IV Універсалу датований 22 (9) січня 1918 року. Ухвалили його вночі 24
(11) січня 1918 року на засіданні Малої Ради Української Центральної Ради.
Документ містив чотири головні напрями: проголошення самостійності Української
Народної Республіки; доручення Раді Народних Міністрів укласти мир з
Центральними державами; оповіщення оборонної війни з більшовицькою Росією;
декларування основ внутрішнього соціально-еконо\hyp{}мічного будівництва й окреслення
заходів для припинення війни з Центральними державами.

Перед проголошенням IV Універсалу виступив голова Української Центральної Ради
Михайло Грушевський. Голосували за документ поіменно: \enquote{за} було 39, \enquote{проти}  –
4, \enquote{утрималось} – 6 осіб.

\textbf{Уперше в XX столітті Україна проголошувалася незалежною суверенною державою.}

Революційні події на Наддніпрянській Україні, проголошення української
державності сприяли піднесенню національного руху в підавстрійській Галичині. 1
листопада 1918 року постала Західно-Українська Народна Республіка. Її лідери
ініціювали переговори про об'єднання Наддніпрянської України з
Наддністрянською. Їх наслідком стало підписання 1 грудня 1918 року у Фастові
\enquote{передвступного} договору  між УНР і ЗУНР про злуку обох республік в одну
велику державу. 3 січня 1919 року Українська Національна Рада ЗУНР у
Станіславові (нині – Івано-Франківськ) ратифікувала цей договір і прийняла
ухвалу про наступне об'єднання двох частин України в одну державу.

22 січня 1919 року на Софійському майдані в Києві в урочистій атмосфері
відбулося проголошення Акта злуки УНР та ЗУНР в єдину незалежну державу. У
зачитаному на зборах \enquote{Універсалі соборності}, зокрема, відзначалося: \enquote{Однині
воєдино зливаються століттями одірвані одна від одної частини єдиної України –
Західноукраїнська Народна Республіка (Галичина, Буковина, Угорська Русь) і
Наддніпрянська Велика Україна. Здійснились віковічні мрії, якими жили і за які
умирали кращі сини України. Однині є єдина незалежна Українська Народна
Республіка}. Наступного дня Акт злуки майже одностайно був ратифікований
Трудовим конгресом України.

В роки радянського тоталітарного режиму проголошення незалежності УНР і День
Соборності не відзначалися. Із утвердженням влади російських більшовиків ці
\enquote{контрреволюційні свята} стерли із суспільної свідомості. Однак, пам'ять про
об'єднання УНР і ЗУНР в єдину Українську Державу зберігали мешканці Західної
України й українська політична еміграція в країнах Європи й Америки.

Перше офіційне відзначення свята Соборності на державному рівні відбулося 22
січня 1939 року в столиці Карпатської України м. Хусті. Це була наймасовіша за
20 років перебування краю у складі Чехословаччини демонстрація українців —
понад 30 тис. осіб.

У 71-у річницю Акта злуки (22 січня 1990 року) в Україні відбулася одна з
найбільших у Центральній і Східній Європі масових акцій – \enquote{живий ланцюг} як
символ єдності східних і західних земель та знак ушанування подій Української
революції. Більше мільйона людей, узявшись за руки, створили безперервний
ланцюг від Києва до Львова. Акція стала одним зі свідчень того, що українці
подолали страх перед комуністичним режимом і готові протистояти політиці
комуністичної партії.
