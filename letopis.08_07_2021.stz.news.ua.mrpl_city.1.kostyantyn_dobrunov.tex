% vim: keymap=russian-jcukenwin
%%beginhead 
 
%%file 08_07_2021.stz.news.ua.mrpl_city.1.kostyantyn_dobrunov
%%parent 08_07_2021
 
%%url https://mrpl.city/blogs/view/pamyati-kostyantina-dobrunova
 
%%author_id demidko_olga.mariupol,news.ua.mrpl_city
%%date 
 
%%tags 
%%title Пам'яті Костянтина Добрунова
 
%%endhead 
 
\subsection{Пам'яті Костянтина Добрунова}
\label{sec:08_07_2021.stz.news.ua.mrpl_city.1.kostyantyn_dobrunov}
 
\Purl{https://mrpl.city/blogs/view/pamyati-kostyantina-dobrunova}
\ifcmt
 author_begin
   author_id demidko_olga.mariupol,news.ua.mrpl_city
 author_end
\fi

\ii{08_07_2021.stz.news.ua.mrpl_city.1.kostyantyn_dobrunov.pic.1}

9 липня 2015 року пішов з життя головний режисер маріуполь\hyp{}ського драматичного
театру і видатна людина \textbf{Костянтин Володимирович Добрунов}. Його ім'я відомо в
професійних колах не тільки в Маріуполі і в Донецькій області, а й за межами
України. Його вистави стали улюбленими для маріупольських глядачів. Я
зустрілася з близькими, колегами і учнями Костянтина Володимировича, щоб
згадати творчий і життєвий шлях великого майстра.

Творчість Добрунова відрізняла глибина дослідження матеріалу і режисури
спектаклів. Цьому сприяла отримана академічна освіта. У 1989 році він закінчив
Вище театральне училище ім. Щукіна, при державному академічному театрі
ім. Вахтангова. Режисер відрізнявся бездоганним професіоналізмом і найвищим
рівнем ерудиції. Для доньки Наталі батько був справжнім взірцем для
наслідування. Вона наголосила, що
\begin{quote}
\em\enquote{Костянтин Володимирович був людиною, у якої
можна було вчитися всьому – навіть подиху, погляду. Настільки вони були
багатозначними. Кожна мить проведена з ним, як з батьком, вчителем – безцінна}.
\end{quote}
\textbf{Наталі Добруновій} дуже шкода, що й досі вона не може знайти такого талановитого
і професійного майстра, як її батько. Цікаво, що вступ Наталі до акторського
відділення для Костянтина Добрунова став справжнім сюрпризом. Він не очікував,
що буде приймати іспит у власної доньки, але вона змогла приємно вразити батька
і він підтримав її вибір.

\ii{08_07_2021.stz.news.ua.mrpl_city.1.kostyantyn_dobrunov.pic.2}

Дружина Костянтина – \textbf{Анжеліка Добрунова} – спочатку теж була його
ученицею.  Майбутні чоловік і дружина познайомилися у 1986 році. У Анжеліки
Арганівни була давня мрія зв'язати своє життя з мистецтвом. Але одразу не
виходило. Спочатку вона закінчила технікум за спеціальністю
\enquote{Архітектура}, але незважаючи на червоний диплом, зрозуміла що це не
зовсім те, чим хотілося б займатися. Тоді дівчина прийшла до Горлівського
народного театру \enquote{Юність} в студію, щоб почати вивчати акторську
професію. Костянтин Володимирович став викладачем Анжеліки.  Вона одразу ж
зрозуміла, що це дуже талановита і неординарна людина. Зі свого боку Костятин
Добрунов також відзначив для себе Анжеліку. Вже тоді вони почали реалізовувати
разом творчі проєкти і зрозуміли, що їх поєднує не тільки творчість. Їхні
стосунки спочатку – вчитель – учениця, режисер – актриса – привели їх обох до
розуміння, що вони працюють як єдине ціле і дивляться в одному напрямку.
Анжеліка Арганівна підкреслила:

\begin{quote}
\em\enquote{це був не тільки мій чоловік,
це була більша частина мене самої. Мені в житті дуже пощастило, що я потрапила
мистецтво саме так... Завдяки такому турботливому вчителю, такому талановитому
режисеру і, звичайно, дуже великій Людині}.
\end{quote}

\ii{08_07_2021.stz.news.ua.mrpl_city.1.kostyantyn_dobrunov.pic.3}

Колеги пам'ятають режисера як професіонала своєї справи, людину з
енциклопедичними знаннями... Костянтин Добрунов\par\noindent запам'ятався багатьом акторам не
лише як головний режисер, а й як чудовий педагог.

Головний художник Донецького академічного обласного драматичного театру (м.
Маріуполь), лауреатка премії ім. В. Клеха \textbf{Світлана Канн} розповіла, що вперше
зустрілася з Добруновим ще в Донецькому театрі ім. Артема. Там вони робили
разом виставу, а потім Костянтин Володимирович прийшов працювати до
маріупольського театру. Світлана Юріївна відзначила, що головний режисер був
дуже цікавою людиною:

\begin{quote}
\em\enquote{він справжній педагог. Колись викладав англійську мову в
школі. Він вмів спілкуватися з артистами, налаштувати їх на роботу, завжди
шукав їхню найкращу та водночас нову сторону акторського дарування. Не часто
зустрічаються такі режисери... Він міг виховати артистів. Сам Костянтин
Володимирович був дуже талановитим актором. Він зіграв в декількох виставах,
просто замінюючи когось, але зіграв блискуче. А як він співав... У нього був
чудовий голос...}.
\end{quote}

\ii{08_07_2021.stz.news.ua.mrpl_city.1.kostyantyn_dobrunov.pic.4}

Актриса Донецького академічного обласного драматичного театру (м. Маріуполь)
\textbf{Ірина Топчиєва} почала свій творчий шлях у маріупольському театрі саме зі
знайомства з Костянтином Володимировичем. За словами Ірини, цей шлях був
довгий, плідний, щасливий. Саме в цей час відбувалося становлення Топчиєвої як
актриси. Вона дуже вдячна головному режисеру:

\begin{quote}
\em\enquote{Костянтин Володимирович як
вчитель, як людина, як режисер дуже багато зробив і для цього театру, і для
багатьох акторів, зокрема і для мене. Весь той багаж, який він нам залишив, ми
несемо далі по життю. Ми його використовуємо в наших подальших роботах. Це
досвід, це майстерність і це любов до нашої професії}.
\end{quote}

А акторка \textbf{Анна Німайєр} зауважила, що Костянтина Володимировича зараз дуже не
вистачає: 

\begin{quote}
\em\enquote{це була та людина, яка добре знала свою справу та могла зібрати
навколо себе колектив. Він мів надихнути, у всіх горіли очі. З будь-якої п’єси
йому вдавалося зробити шедевральну виставу}. 
\end{quote}

Головний режисер був для Анни Німайєр чудовим вчителем, педагогом, який допоміг
її розкритися, знайти нові грані в собі, що посприяло професійному зростанню
актриси.

\begin{quote}
\em\enquote{Я дуже вдячна Богу
і долі, що працювала з таким талановитим режисером. Костянтин Володимирович
дуже багато давав. Мабуть, все, що я зараз маю, все завдяки йому. Він міг
сказати одне слово і ти вже починаєш робити, йому вдавалося швидко направити.
Це було дивовижно}, 
\end{quote}
– наголосила Анна.

\ii{08_07_2021.stz.news.ua.mrpl_city.1.kostyantyn_dobrunov.pic.5}

Завдяки зусиллям головного режисера у Маріуполі було створене акторське
відділення при музичному училищі. Наразі випускники цього відділення, гарно
зарекомендувавши себе, поповнили трупу маріупольського театру.

Одним зі студентів Костянтина Володимировича став артист \textbf{Дмитро Нестеренко},
який зазначив, що 

\begin{quote}
\em\enquote{вклав він в нас дуже багато. вклав своє зерно, яке проросло
і дало свої плоди. Школа Добрунова – це дійсно професійна школа, яка в
подальшому допомагає нам йти по творчому шляху. Але такі майстри, на превеликий
жаль, йдуть дуже рано...}. 
\end{quote}

Костянтин Добрунов був головним режисером з 2000 року і за цей час поставив
десятки вистав, які стали резонансними подіями в культурному житті Приазов’я і
надовго увійшли в репертуар театру. Серед них – \enquote{Театр часів Нерона і Сенеки},
\enquote{Роман про дівчаток}, \enquote{Тартюф}, \enquote{Анна Кареніна}, \enquote{Дерева вмирають стоячи},
\enquote{Чорне і червоне, або Маріупольський скарб Нестора Махна}, \enquote{Приборкання
норовливої}, \enquote{Маргарита}, \enquote{Майстер}, \enquote{Примадонни}. Костянтин Володимирович, без
перебільшення, залишив яскравий слід в театральному житті нашого міста і завжди
житиме  в серцях своїх близьких, учні, колег та відданих глядачів.

\textbf{\emph{Всі світлини з Особистого архіву Анжеліки Добрунової}}
