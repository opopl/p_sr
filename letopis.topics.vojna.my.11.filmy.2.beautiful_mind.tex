% vim: keymap=russian-jcukenwin
%%beginhead 
 
%%file topics.vojna.my.11.filmy.2.beautiful_mind
%%parent topics.vojna.my.11.filmy
 
%%url 
 
%%author_id 
%%date 
 
%%tags 
%%title 
 
%%endhead 

\section{О Фильме Игры Разума (Beautiful Mind)}

Недавно я посмотрел и даже несколько раз пересмотрел прекрасный фильм Игры
Разума (Beautiful Mind - буквально, Прекрасный Разум - 2001 год - удостоенный
Оскара).  Это фильм про математика-ученого, Джона Неша, реально жившего,
который известен в науке, тем, что создал теорию игр (game theory),
математическую теорию, которая ныне активно используется во многих областях
человеческой жизни, за которую ему в 1994 присудили Нобелевскую Премию по
экономике.  Джон Неш также известен тем, что на какой то период своей жизни он
заболел шизофренией, но он смог побороть в итоге свой душевный недуг, в
конечном итоге вернувшись к активной жизни в науке. Вот про жизнь Неша в науке,
а также про то, как Неш заболел, боролся со своим недугом, и как он научился
справляться со своей болезнью, и был снят этот фильм. А касательно содержания
фильма, и впечатлений после просмотра, я перейду немного позже, сейчас хотелось
бы сделать небольшое философское отступление по поводу роли точных наук, и в
частности математики, в нашей современной жизни.

... Говорят, что математика - это царица наук, и это действительно так, хотя
тут есть парадокс, потому что математика - это в каком то смысле вообще не
наука, если принять точку зрения, что наука - это все, что проверяется и
подтверждается опытом, и что наука исследует явления природы. Ну ладно,
математика царица наук или нет, это не так важно, как важно то, что математика
лежит в основе нашего сегодняшнего понимания мира, потому что физические
законы, которые управляют всем, что происходит вокруг, выражены через
математические уравнения и формулы. И исторически развитие наших представлений
об окружающем мире, шло параллельно с развитием математики.  Более того,
технологии, которые мы пользуемся каждый день, даже например простой
выключатель света, покоятся на действии законов физики, записанных
математическим языком, как например уравнения электродинамики Максвелла.  В
общем, математика повсюду, она не видна совершенно, ее не замечают, но она
работает - и в том, как работает компьютер или смартфон, и в том, как и почему
движется вагон Киевского метро, и почему лифт в нашем доме едет вверх или вниз,
и почему, если набрать ванну полную воды в своей квартире, а потом плюхнуться
туда, то весь пол будет мокрым, и почему, если бросить яблоко с балкона нашей
квартиры на десятом этаже, оно неизбежно упадет через несколько секунд на
асфальт, громко шмякнувшись, и до смерти испугав рядом проходящую старушку, и
даже также в том, почему самолет взлетает, когда мы летим в отпуск в Египет,
хотя казалось бы, если он такой тяжелый, так как он способен летать? 

\ifcmt
  tab_begin cols=3,no_fig,center

     pic https://i2.paste.pics/fa62372d549789835275d055915e7ef5.png
		 @caption Исаак Ньютон, английский физик и математик (1642 - 1727)

		 pic https://i2.paste.pics/b0451b1c82d51957a71bfad9bb089f0c.png
		 @caption Джеймс Кларк Максвелл, шотландский физик и математик (1831 - 1879)

		 pic https://i2.paste.pics/60a463d15e5875fe82582334fa50512f.png
		 @caption Людвиг Больцман, австрийский физик-теоретик (1844-1906)

  tab_end
\fi

Вот...  математика и физика лежит в основе нашего сегодняшнего технологичного и
цифрового мира, и без Исаака Ньютона, открывшего закон всемирного тяготения,
Джеймса Кларка Максвелла, записавшего уравнения электродинамики, или же Людвига
Больцмана, внесшего вклад в понимание законов термодинамики, а также многих-многих других
ученых, внесших самый разнообразный и неоценимый вклад в копилку наших знаний о
законах, управляющих поведением природных явлений, мы бы по прежнему бы бегали
туда-сюда с факелами, свечками, ездили бы на конях, и также мерли бы пачками
немеряно от оспы, чумы, и прочих напастей, как это и происходило например в
Средневековье.  А чтобы поехать из Винницы, скажем в Киев на паломничество к
Святым Церквям, да, как это было в столетии этак восемнадцатом, нам бы пришлось
трястись целую неделю в бричке, потому что купить билет на поезд Винница-Киев
было никак нельзя, потому что и поездов, и вокзалов тогда никаких не было! А
насчет того, чтобы созвониться по скайпу со старым школьным товарищем, ныне
загорающим на своем собственном ранчо в Техасе, так речи вообще не было! Так
что слава математике! Слава физике!  Слава точным наукам, и ученым, благодаря
которым у нас есть столько всяких технологий, облегчающих и делающих приятней
нашу жизнь!

... А теперь, после такого философско-исторически-лирического отступления, (
которое возможно некоторым покажется довольно забавным и смешным ) перейду
собственно к фильму.

Фильм, как уже было сказано, о жизни американского математика Джона Неша.
Начало отводит нас к тому времени (конец сороковых годов двадцатого столетия),
когда Неш приезжает в первый раз в Принстонский Университет (Princeton
University), один из самых известных и мощных университетов в США, по
стипендии, чтобы учиться на степень доктора философии (PhD), то есть кандидата
наук по нашему.  Здесь он встречает своих будущих друзей Бендера, Сола, и также
Мартина Хансона, - все они математики. Он также попадает на приветственную
лекцию декана факультета, где декан говорит о важности математики и точных
наук, а также экономики и медицины в начинающемся противостоянии и холодной
войне глобальных сверхдержав - США и СССР.  Лекция про то, что математика
помогла выиграть вторую мировую войну (взлом нацистских шифров), и про то, что
противостояние между сверхдержавам происходит не только в плане гонки
вооружений, но также по линии интеллектуальных свершений на ниве точных наук, и
что математики в Принстоне должны сыграть важную роль в этой новой холодной
войне. 

... Далее, Неш, молодой аспирант, уже имеющий репутацию гения, отправляется
обустраиваться в свою комнату в университетском общежитии. Далее происходит
такая сцена. Неш, весь в мечтах о своих будущих научных свершениях, и горящий
желанием тут же начать творческую работу над своими идеями, в своей комнате
придвигает свой письменный стол к окну, потом садится за него, намереваясь как
следует поработать.  Он видит, что снаружи внизу в университетском дворе
собирается команда студентов поиграть в мяч, есть две команды, которые
расходятся по разные стороны перед началом игры. Неша интересует математическое
осмысление движения команд, его ключевая мысль, то, что его занимает все время
- управляющая динамика - теория, которую ему только предстоит разработать.  Так
вот, Неш придвигает письменный стол к окну, смотрит на команды, и начинает на
окне рисовать математическое отображение того, что происходит снаружи между
командами - линию, разделяющую визуально команды, какие то другие знаки и
символы.  Он поглощен этим занятием, все его мысли об этом. 

\ifcmt
  tab_begin cols=3,no_fig,center

		 pic https://i2.paste.pics/c539b7cb15972a8126ed817f671437e7.png
		 @caption Чарльз вваливается в комнату Неша, - давай поделим сию келию на двоих!

		 pic https://i2.paste.pics/090cffb8d99ee7998c636bd30c8ecfad.png
		 @caption Неш удивленно смотрит на незваного развязного гостя, - на двоих?! только не это!

     pic https://i2.paste.pics/e65bf043c8d68ab79b88767d5e7497c7.png
		 @caption Комната Неша, Неш сидит за столом, позади фамильярный вальяжный Чарльз Гармен

  tab_end
\fi

Внезапно в комнату
врывается какой то молодой полураздетый рубаха-парень, вальяжный весь такой,
навеселе, фамильярный; его зовут Чарльз Гармен, он видит Неша возле окна, и тут
же знакомится с ним.  Неш удивленно на него смотрит, потом садится за стол,
намереваясь продолжать свою математическую работу. Не тут то было!  Чарльз все
время болтает, о каком то приеме, о какой то горячей встрече с какой то
красоткой, в общем, без умолку из него сыпятся слова.  Потом, видя, что Неш на
него не обращает внимание, Чарльз пытается привлечь его внимание, сначала
пытаясь украсть печенье из коробки Неша, стоящей на столе, а когда Неш нервно
захлопывает своей правой рукой эту самую коробку с печеньем, Чарльз вообше
запрыгивает на стол перед Нешем, и насмешливо смотря на него, говорит, - так
оказывается, мой сосед зануда! Ну что ж, надо растопить лед! - И достает
фляжку, очевидно, с каким то крепким напитком внутри.  Интересно, что пьют
студенты в Принстоне при первом знакомстве, - водяру? или коньяк? или может
быть виски или же бренди?  Я не знаю, к сожалению, а фильм об этом умалчивает,
что же именно было внутри этой самой небольшой железной фляжки. 

\ifcmt
  tab_begin cols=3,no_fig,center

     pic https://i2.paste.pics/c35fef1cb2fed886e4f23f45e31876f2.png
		 @caption Неш рисует формулы на окне
		 
		 pic https://i2.paste.pics/69eeed5ffeb48dd4db1105f957a49aae.png
		 @caption Чарльз бегает по комнате, болтает, пытается привлечь внимание Неша

		 pic https://i2.paste.pics/5bca779e8b8e635195b0692d5759d758.png
		 @caption Прелюдия к доверительной беседе на свежем воздухе с фляжкой с неизвестной жидкостью

  tab_end
\fi

Далее... Неш и Чарльз на крыше болтают дружески, а фляжка им в этом очень помогает. Неш признается,
что не любит людей, и что это взаимно, и что его учительница ему когда-то даже говорила, что Джон
родился с двойной порцией мозгов, и в половиной сердца... Он делится своими мечтами о своей научной работе,
взволнованно говорит, что ему нет дела до семинаров, и до тупой зубрежки, ему позарез нужна идея, оригинальная идея,
чтобы состояться в его работе, чтобы его заметили в научных кругах... 

\ifcmt
  tab_begin cols=4,no_fig,center

     pic https://i2.paste.pics/3b55aa0648932b7fe4623d4ee60507bc.png
		 @caption Чарльз - раз трудно сломать лед, может растопим?

		 pic https://i2.paste.pics/d8f97a55d88f71f3439ac74407c43098.png
		 @caption Беседа на крыше

		 pic https://i2.paste.pics/4facd95bbd7a6f0a0b2d63c7071a7cf9.png
		 @caption Беседа на крыше

		 pic https://i2.paste.pics/077c17da891de9424e2f8044d2c0b21b.png
		 @caption Беседа на крыше

  tab_end
\fi

Далее... идет забавная сцена... На улице... Друзья Неша играют в какую то игру
на доске с округлыми шашками, тут рулит веселый Мартин Хансон с сигаретой в
зубах, который все время всех обыгрывает, - кто следующий, - давай!
Проигравшие игроки плюются и встают из-за доски. Тут они все замечают Неша,
который бегает туда-сюда с карандашом и записной книжкой, задом наперед,
туда-сюда... Эй Неш! - что ты делаешь?! - Неш отвечает - я исследую движение
голубей, я пытаюсь разгадать законы движения птиц... Мартин зовет его, давай
сюда, будем играть! 

\ifcmt
  tab_begin cols=3,no_fig,center

     pic https://i2.paste.pics/d106a3652d2412f62eb445889bee52a8.png
		 @caption Мартин Хансон - трусы, все трусы, никто не примет мой вызов!

		 pic https://i2.paste.pics/f7497efe00b9efad49206ec16c252297.png
		 @caption Неш исследует поведение голубей
		 
		 pic https://i2.paste.pics/bc69d1f4217c891d574282e4a188aaf2.png
		 @caption Голуби, движение которых исследует Неш

  tab_end
\fi

Мартин подкалывает Неша, - Неш подавляет всех своим гением, а со мной сразиться
ему смелости не хватает!  Неш, в конце концов не выдерживает и идет сражаться с
Мартином, и что вполне естественно, - потому что Мартин очевидно в ударе, - и
он просто рвет на части всех, кто садится с ним играть, - Мартин выигрывает, -
походу во время игры задавая Нешу ехидные вопросы. Неш, раздраженный,
огорченный проигрышом, рывком встает из-за доски, и нервно, опрокинув доску с
шашками по дороге, уходит, не оглядываясь...

\ifcmt
  tab_begin cols=3,no_fig,center

     pic https://i2.paste.pics/06e098e0879eeba082ded5da924542be.png
		 @caption Мартин вызывает Неша на игру

		 pic https://i2.paste.pics/39c59fd90aeadfa5698f04c6efaf2e08.png
		 @caption Мартин Хансон и Неш играют, друзья смотрят

		 pic https://i2.paste.pics/57abf680a22c8bba94f3dfd3602a716f.png
		 @caption Игра, вид сверху

  tab_end
\fi
