% vim: keymap=russian-jcukenwin
%%beginhead 
 
%%file 30_10_2021.fb.chornoguz_jaryna.1.ubogist_mova_servis.cmt
%%parent 30_10_2021.fb.chornoguz_jaryna.1.ubogist_mova_servis
 
%%url 
 
%%author_id 
%%date 
 
%%tags 
%%title 
 
%%endhead 
\subsubsection{Коментарі}

\begin{itemize} % {
\iusr{Назар Чорношличник}

Були спроби робитись "українською еміграцію" у власній країні, зі своїми
крамницями, кредитовими спілками, кав'ярнями. Але тоді узкій мір безперешкодно
своїм морем заповнює ВСЕ

\iusr{Назар Чорношличник}
Так

\iusr{Тетяна Чеховська-Косцова}

Дві доби маю власну війну зі сторінкою \url{https://www.facebook.com/shopkinsu/}

На зауваження про мову отримала і грубість, і знецінення, і шквал "однодумців",
які прибігли і наставили моїм постам то гнівні, то смішні смайли.

Але знаєте, якщо то така ціна перемоги, то і нехай. Бо сьогодні власниця
сторінки пані Таня Пось напевно вперше в житті на сторінці свого магазину
написала пост українською))

\begin{itemize} % {
\iusr{Олеся Ковтун}
\textbf{Тетяна Чеховська-Косцова} написала в приват.

\iusr{Михайло Шахтьор Жадан}

Так. Проте потім - продовжила писати дописи масковскім нарєчієм...

Але що показово: її дописи зазвичай мають 3-10 вподобаєк. Максимум - 20.
Українськомовний допис станом на 10:52 31 жовтня має 16!

Якою мовою вона писатиме надалі? якщо звісно ж - там не вата...

\iusr{Тетяна Чеховська-Косцова}
\textbf{Михайло Жадан}
Ну, отак кнутом і пряником можливо і покажемо як робити бізнес))
Україномовним поставила вподобаку. Язичним - гнівні))
\end{itemize} % }

\iusr{Анастасия Олейникова}
З огляду на Історію, нескінченно.  @igg{fbicon.smile} 

\iusr{Васек Духновський}

Даааа. Попри те, що ситуація явно покращується, нам ще роками вигрібати цю
фігню. Днями бачив протилежну, але не менш сумну картину: продавець у м'ясному
відділі красивою українською перепитує "вам стегно", а женщіна така "а што
такоє стєгно?".

Думаю, блядь, хуй його знає, як твоєю російською буде те їбане стегно. Справді
ще хвилин п'ять намагався згадати, як же воно перекладається)

\begin{itemize} % {
\iusr{Nemo Skal}
Тій шерепі москворотій треба було ще про гомілку сказати.

\iusr{Михайло Шахтьор Жадан}
Треба було відповісти: "Те саме що гомілка, лише вище знаходиться"
 @igg{fbicon.beaming.face.smiling.eyes} 
\end{itemize} % }

\iusr{Євген Рижак}

Ярин, для того, щоб це скінчилося і ця малоросня щось та допетрала, треба аби
ти і всі вояки пішли до дому... Прийдешнє пуйло допоможе прозріти, я гадаю...

\iusr{Людмила Малая}
Люди знають російську мову, і то не факт,+мат, а на ще щось в їх мозку місця не вистачає

\iusr{Олег Ттт}
Ох, мабуть, ще довго

\iusr{Alla Svyrydyuk}
Мабуть, не так швидко, як хотілось би... допоки не прийде чітке розуміння , що мова насамперед ідентифікує нас як украінців

\iusr{Юрій Лозенко}
ЕщьЕ немнОга падАждіте....
Впевнений, скоро переможемо .!) Вода камінь точить..)

\iusr{Олександр Нікольський}
Трохи триватиме, будь ласка, чекайте ...

\begin{itemize} % {
\iusr{Олекса Мар}
Олександрe! Я вже понад 30 років чекаю, скільки ще?
\end{itemize} % }

\iusr{Валерій Попович}
+++

\iusr{Дмитрий Громов}

Ти знаєш Яринко, мова звісно дуже важлива. Це навіть вже і не повинно
обговорюватися. Війна розставила все на свої місця. Я знаю дуже добре російську
мову. Я нею завжди розмовляю "с русскоязычными" спеціально, щоб там в мізках щось
клацнуло і спрацювало на користь нашої держави. Не завжди це звісно
допомагає, але все одно працює. В мене є один хороший знайомий. Росіянин. Живе в
Україні ще з часів СССР. Мову він сам вчив, але все одно смішно в нього
виходить, хоча він і дуже старається. Так от він для фронту допомагав фінансово
більше ніж будь хто, кого я знаю зі своїх знайомих. І зараз дивлячись на цю
"більшість 73\%",серед яких багато хто розмовляє українською чи носить вишиванку
- я тепер розумію, що окрім нашої мови, дуже важливо що на серці і в душі
людини. Тому що, якщо там погань та ворог, то й мова не допоможе зробити з нього
справжню людину  @igg{fbicon.thinking.face} 

\iusr{Анатолій Бер}
\textbf{Ярина Чорногуз} треба їх на всіх рівнях карати! Як законом так і морально!

\iusr{Ірина Якимів}
Так може варто зробити цей магазин відомим..?

\iusr{Joanna Corsun}
Завжди(((

\iusr{Роман Дейнего}

Щє довго, нажаль.... але це нормально! Нічого швидко не буває.... Зараз це
виглядає, як битись головою об стіну.... @igg{fbicon. man. facepalming}  але це принесе свої плоди у
майбутньому!!!

\begin{itemize} % {
\iusr{Олекса Мар}
\textbf{Roman Deynego} Ні, це не нормально! У сусідніх правових державах за відсутність норми або базових речей, згаданий бізнес вбивають, а всіх українофобів жорстоко карають!

\iusr{Роман Дейнего}
\textbf{Олекса Мар} Вати багато щє.... а також любителів радянського союзу, для яких росія.... це по типу і є срср... @igg{fbicon. man. facepalming}  Їх валом і швидко ми від них ніколи не позбавимось..... Але процес в дії і все з часом буде норм!))
\end{itemize} % }

\iusr{Наталя Литвиненко}
Ага, Ярино, дякую/ нагадала. В рибному магазині ще піду та буду догризати свою правоту. Та сама «пісня» + хамство.

\iusr{Татусь Че}
Доки ми будем соплі жувати і бути толерастами! Вони вперто будуть стояти на своєму і заявляти про себе, а ми?

\iusr{Олексій Фомін}
Ще з років 30-50 в гіршому випадку ((((

\iusr{Оксана Городинська}
Вперто робити свою справу. Вони сподіваються, що нам набридне

\iusr{Ігор Науменко}
Покі буде какаяразніца накакомйазикє ліш би на руском

\end{itemize} % }
