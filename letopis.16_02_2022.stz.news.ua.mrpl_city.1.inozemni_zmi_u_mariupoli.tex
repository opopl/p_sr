% vim: keymap=russian-jcukenwin
%%beginhead 
 
%%file 16_02_2022.stz.news.ua.mrpl_city.1.inozemni_zmi_u_mariupoli
%%parent 16_02_2022
 
%%url https://mrpl.city/blogs/view/inozemni-zmi-u-mariupoli
 
%%author_id demidko_olga.mariupol,news.ua.mrpl_city
%%date 
 
%%tags 
%%title Іноземні ЗМІ у Маріуполі
 
%%endhead 
 
\subsection{Іноземні ЗМІ у Маріуполі}
\label{sec:16_02_2022.stz.news.ua.mrpl_city.1.inozemni_zmi_u_mariupoli}
 
\Purl{https://mrpl.city/blogs/view/inozemni-zmi-u-mariupoli}
\ifcmt
 author_begin
   author_id demidko_olga.mariupol,news.ua.mrpl_city
 author_end
\fi

\ii{16_02_2022.stz.news.ua.mrpl_city.1.inozemni_zmi_u_mariupoli.pic.1}

Останнім часом до Маріуполя все частіше приїжджають іноземні журналісти. Для
декількох кореспондентів я провела екскурсії старою частиною міста, ознайомила
їх з історією регіону, з головними міськими історико-культурними пам'ятками і
дізналася, що найбільше цікавить іноземні ЗМІ. Спочатку я зустрілася з празьким
журналістом Ондрою Соукупом. Він розпитував мене про головні особливості
історичного розвитку Маріуполя, ментальні риси містян та загальні настрої
мешканців щодо нинішньої ситуації.

Більше часу я спілкувалася з кореспондентом чеського радіо Мартіном Доразіним.
Він працює кореспондентом радіо і телебачення вже 30 років. Увага чоловіка
здебільшого зосереджена на Польщі, Прибалтиці та Україні. За час своєї роботи
він побував у багатьох гарячих точках світу і для того, щоб висвітлити сучасну
інформацію максимально правдиво, вирішив приїхати до України і зокрема до
Маріуполя. Насправді за всіма політичними подіями, що відбувалися в Україні,
Мартін намагався спостерігати завжди зсередини. Чоловік був на першому майдані
та висвітлював події Помаранчевої революції. У 2014 році він вперше відвідав
Маріуполь. Мартін сподівається, що широкомасштабної війни проти України не
буде. Проте вважає, що наразі повинен бути саме на Сході України.

\ii{16_02_2022.stz.news.ua.mrpl_city.1.inozemni_zmi_u_mariupoli.pic.2}

Чоловік приїхав з колегою – фотографом – Івою Зімовою. На шляху до Маріуполя
вони відвідали Харків, Слов'янськ, Мар'їнку. Кореспондент добре знає російську
мову, тому з легкістю спілкується з жителями міст. У Маріуполі журналісти
відвідують істо\hyp{}рико-архітектурні об'єкти, спілкуються з активістами та
волонтерами щодо змін, які відбулися у місті та впливу нинішньої ситуації на
містян. За словами Мартіна, тема збройного конфлікту в Україні займає у чеських
медіа дуже високе місце. Він навіть не очікував, що інтерес буде настільки
великий. До речі, чеських журналістів в Маріуполі найбільше. Зокрема, у місті
працює чеське телебачення (вже другий редактор приїхав на зміну першому),
друкарські видання, радіо, фотографи та окремі блогери. Мартін такий інтерес
пояснює великою кількістю українців, які працюють у Празі. Наразі у Чехії, за
даними чеського уряду, легально працює і проживає близько 140 тисяч українців.
Загалом між Чехією та Україною існують довгі історичні традиції, які об'єднують
дві країни.

\ii{16_02_2022.stz.news.ua.mrpl_city.1.inozemni_zmi_u_mariupoli.pic.3}

Водночас з великою обережністю до візитів іноземних журналістів наразі
ставляться представники міської влади, адже існує побоювання, що в результаті
може відбуватися перекручування контексту. Так радник міського голови з питань
промоції Петро Андрющенко повідомив, що у Маріуполі вже були найбільш імениті
видання, серед яких і \enquote{Financial Times}, \enquote{The Washington Post}. Були й інші
авторитетні європейські видання, зокрема іспанські, французькі, чеські та інші.
Однак в результаті виходили матеріали з перекрученими контекстами, де акценти
були розставлені зовсім інакше. Так сталося з коротким інтерв'ю міського голови
Вадима Бойченка британському виданню, на яке відреагували і російські ЗМІ. Одне
за одним вони почали публікувати повідомлення про те, що мер Маріуполя вважає
Росію другом. Слідом за російськими ЗМІ, у соціальних мережах українського
сегмента почали з'являтися посилання на російські ЗМІ з коментарями про \enquote{зраду}
Вадима Бойченка. Проте відбулося справжня перекручування інформації. Міський
голова Маріуполя назвав агресивних сусідів \enquote{друзями} саркастично, і журналісти
\enquote{Financial Times} під час перекладу не поставили лапки на цих словах. У
результаті інформацію все ж виправили, проте тепер до іноземних кореспондентів
ставляться з великою засторогою. Петро наголосив, що \emph{\enquote{хотілося б, щоб іноземні
журналісти бачили наше місто таким, яким воно є на сьогоднішній день, як воно
змінилося за вісім років, як містяни не панікують, як Маріуполь є прикладом
спокою та виваженого ставлення до загрози, яка є сьогодні, була вчора і напевно
залишатиметься ще не один рік уперед}}.

\ii{16_02_2022.stz.news.ua.mrpl_city.1.inozemni_zmi_u_mariupoli.pic.4}

Наразі в Маріуполі перебувають нідерландське телебачення, чеське радіо, газети
з Німеччини, польські та англійські видання, можна зустріти і медіа із  США
(зокрема NBC) та Аргентини. Проте далеко не всі містяни готові йти на контакт з
іноземними журналістами. На думку кореспондента чеського радіо Мартіна
Доразіна, це пов'язано з тими, що багато маріупольців перебувають у стресі,
вони вже дуже втомилися від ситуації, зовсім не хочуть думати про це, тим
більше розмовляти з іноземцями на ці теми. Чоловік підкреслив, що людям все
дуже се це набридло, їм боляче. Він добре їх розуміє, тому ні на кого не тисне.
Спілкується більше з маріупольськими активістами, які готові до діалогу. 

Особисто я сподіваюся, що чеські журналісти, які на мене справили гарне
враження, висвітлюватимуть інформацію об'єктивно та правдиво і ніяких
перекручувань інформації в їхніх матеріалах не буде. Зі свого боку я намагалася
показати унікальність Маріуполя, розповісти про його багату, але непросту
історію, особливий приазовський менталітет та познайомити з найбільш цікавими і
неповторними культурними та архітектурними пам'ятками.
