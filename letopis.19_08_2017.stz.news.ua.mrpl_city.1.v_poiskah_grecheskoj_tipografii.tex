% vim: keymap=russian-jcukenwin
%%beginhead 
 
%%file 19_08_2017.stz.news.ua.mrpl_city.1.v_poiskah_grecheskoj_tipografii
%%parent 19_08_2017
 
%%url https://mrpl.city/blogs/view/v-poiskah-grecheskoj-tipografii
 
%%author_id burov_sergij.mariupol,news.ua.mrpl_city
%%date 
 
%%tags 
%%title В поисках греческой типографии
 
%%endhead 
 
\subsection{В поисках греческой типографии}
\label{sec:19_08_2017.stz.news.ua.mrpl_city.1.v_poiskah_grecheskoj_tipografii}
 
\Purl{https://mrpl.city/blogs/view/v-poiskah-grecheskoj-tipografii}
\ifcmt
 author_begin
   author_id burov_sergij.mariupol,news.ua.mrpl_city
 author_end
\fi


В свое время были записаны воспоминания Любови Саввичны Тохтамыш – бывшей
литературной сотрудницы редакции греческой газеты \enquote{Колехтивистис}. Обратили на
себя внимание ее слова: \enquote{У нас была своя типография}. Где была эта типография?
Об этом позже. А пока несколько абзацев о языках мариупольских греков.

\ii{19_08_2017.stz.news.ua.mrpl_city.1.v_poiskah_grecheskoj_tipografii.pic.1}

Исторические судьбы языков мариупольских греков складывались таким образом, что
на протяжении длительного времени они оставались неписьменными. (Для справки: в
научной литературе  мариупольскими греками принято называть греков,
переселенных из Крыма в Северное Приазовье в 1778 году.) Это название  касается
не только жителей города Мариуполя, но и греческих сел, находившихся некогда в
составе Мариупольского уезда Азовской, а затем Екатеринославской губернии.

Понятно, что ни крымские ханы, ни царское правительство России не были
заинтересованы в национальном развитии своих подданных-инородцев, в том числе
культурном. А ведь письменность является инструментом национальной культуры,
самосохранения самосознания  народа, его культуры. Более того, власть
предержащие старались ассимилировать нерусские народы. Положение изменилось с
установлением на территории  бывшей Российской империи советской власти. В
апреле 1923 года на XII съезде РКП(б) было принято решение о внедрении так
называемой политики коренизации. \enquote{Коренизация – кадровая политика,
заключавшаяся в подготовке, выдвижении и использовании в национальных
образованиях национальных кадров (в особенности из \enquote{титульных национальностей})
для работы в государственных и общественных органах, в хозяйственных и
культурных учреждениях. Проводилась она в СССР в 1920-1930-е гг. Одной из целей
коренизации было развитие национальных языков, внедрение их в сферу
деятельности государственного аппарата}. (Словарь социолингвистических
терминов. — М.: РАН. Институт языкознания. Российская академия лингвистических
наук, 2006.)

\ii{19_08_2017.stz.news.ua.mrpl_city.1.v_poiskah_grecheskoj_tipografii.pic.2}

Нужно отметить, что на Мариупольщине коренизация была встречена патриотическими
силами греческого населения  с одобрением. Однако сразу же возникли трудности
при создании письменности. В силу исторически сложившихся обстоятельств для
части мариупольских греков родным языком был и остается румейский, восходящий,
по мнению некоторых филологов,  к средневековому греческому,  для другой части
языком был крымскотатарский, каждый из этих языков имеет несколько диалектов.
Эта тема изложена в фундаментальном труде профессора, доктора исторических наук
Ирины Семеновны Пономаревой \enquote{Етнічна історія греків Приазов'я}.

Большой  вклад  в развитие письменного румейского языка внес поэт, драматург и
журналист Георгий Антонович Костоправ. В основу письменного румейского языка
был положен сартанский диалект.

Переход на димотику – язык Греции - создал условия для развития культуры греков
СССР. А замена традиционной греческой  орфографии на орфографию фонетическую
существенно облегчила обучение детей в созданных национальных школах греческих
сел. В 1928 году на Всеукраинском совещании греческих
административно-территориальных районов было отмечено, что проблема румейской
письменности решена. Новой  письменностью пользовались журналисты, писатели,
поэты, фольклористы, а также авторы учебников для греческих школ. Известно, что
с октября 1930 года на румейском  языке издавалась греческая газета
\enquote{Колехтивистис}.  Издавались альманахи \enquote{Флогоминитрес спитес}, \enquote{Неотита},
детский журнал \enquote{Пионерос}, учебники для греко-эллинских школ-семилеток. К
сожалению, далеко не все эти печатные издания сохранились. Все это печаталось в
специальной типографии, оснащенной греческими шрифтами и персоналом, владеющим
греческим языком.

В ходе греческой операции НКВД, которая началась 15 декабря 1937 года в
Украине, было репрессировано, читай – расстреляно,  от  5700  до  6500
человек, в том числе многие представители интеллигенции.  Прекратилось издание
газеты, альманахов, учебников на румейском языке. Были закрыты Греческий
педагогический техникум, Греческий театр, греческие школы...

Известный далеко за пределами нашего города газетчик и журналист Николай
Григорьевич Руденко в 2004 году опубликовал исследование под названием \enquote{История
мариупольских типографий}, в котором приведены интересные данные об истории
печатного дела в г. Мариуполе. К сожалению,  в этом весьма полезном труде
адреса типографий не приведены. Как правило, местные типографы не указывали на
продукции  адреса своих заведений. Не удалось найти местоположение типографий,
действовавших после 1917 года, в том числе и греческой.

В интервью  Любови Саввичны Тохтамыш говорится : \enquote{Редакция располагалась на
улице Апатова, 24. Это здание сохранилось целиком (я после войны там была). Вот
идешь вниз по Советской улице и - слева находился ресторан, а поперек шла улица
Апатова. Совсем  недалеко от пересечения была наша редакция... У нас была и
своя типография. Там не только наша газета печаталась, печатались книги,
журналы, учебники. Располагалась она тоже на Апатова, но ниже, ближе к морю.
Иногда мне приходилось идти туда с материалами в сумерках, так страшновато
было. Место не было людным}. Небольшое пояснение. В 1989 году мариупольским
улицам были возвращены исторические названия. Улица Советская – теперь
Харлампиевская, улица Апатова – теперь Итальянская.

Воспоминания Л.С.Тохтамыш  подсказали путь поиска адреса греческой типографии
Мариуполя. В начале сентября 1943 года немецко-фашист\hyp{}ские оккупанты сожгли
значительную часть жилых домов и общественных зданий Мариуполя. Но строения на
Итальянской  улице на участке от улицы Торговой, с которой видно море, до
гавани Шмидта остались нетронутыми огнем. Это обстоятельство помогло в поисках
типографии.  Итак, на нечетной стороне  Итальянской улицы не было достаточно
больших помещений, в которых можно было бы поместить типографское оборудование.
Четная сторона состояла и состоит из одноэтажных жилых домов, кроме одного –
двухэтажного особняка, принадлежавшего  судовладельцу Петру Региру.
Современный адрес – ул. Итальянская, 12. С большой степенью вероятности можно
утверждать, что именно там помещалась греческая типография.

Перед Второй мировой войной  и сразу после нее в нем находился филиал швейной
фабрики. Затем управление № 118 треста Донбасстальконструкция.  В 90-е годы
здание оказалось практически бесхозным и разгромлено вандалами.  В настоящее
время от него остался  лишь остов.
