% vim: keymap=russian-jcukenwin
%%beginhead 
 
%%file 09_07_2021.fb.fb_group.story_kiev_ua.1.osvobozhdenie_kieva
%%parent 09_07_2021
 
%%url https://www.facebook.com/groups/story.kiev.ua/posts/1703048666558594/
 
%%author Киевские Истории
%%author_id fb_group.story_kiev_ua
%%author_url 
 
%%tags germania,gorod,istoria,kiev,mirvojna2,nacizm,semja,sssr,ukraina,vojna,vov
%%title Продолжение - Часть шестая - как я запомнил освобождение родного Киева от немцев
 
%%endhead 
 
\subsection{Продолжение - Часть шестая - как я запомнил освобождение родного Киева от немцев}
\label{sec:09_07_2021.fb.fb_group.story_kiev_ua.1.osvobozhdenie_kieva}
 
\Purl{https://www.facebook.com/groups/story.kiev.ua/posts/1703048666558594/}
\ifcmt
 author_begin
   author_id fb_group.story_kiev_ua
 author_end
\fi

Продолжение Часть шестая. В предыдущих частях я подробно рассказал Вам о жизни
нашей семьи в условиях немецкой оккупации Киева. Продолжу свой рассказ о том,
как я запомнил освобождение родного Киева от немцев. Это было как раз в канун
праздника, годовщины революции. 

Постоянно была слышна артиллерийская  канонада со стороны Днепра. Мама уже
боялась ходить зарабатывать на Евбаз, так как немцы, чувствуя, что им
приходиться оставить Киев — хватали всех без разбору и отправляли в Германию на
работы. Немцы начали откровенно грабить, отбирая у людей  всё ценное, что можно
было увести. 

Вокзал был забит людьми, так как с немцами пытались убежать и те , которые
сотрудничали с  ними  во время оккупации.  

Сбежала и наша соседка, заявившая немцам на  маминого брата Колю, которого
расстреляли у нас во дворе.  Воспользовавшись её бегством — мы вернулись
буквально за несколько дней до освобождения Киева в свою квартиру, которую она
занимала со своим сожителем полицаем, на Дмитриевскую №42.

Повсюду распространялись слухи, что при отступлении немцы будут всё взрывать,
оставляя за собой выжженную землю и действительно был взорван мост через Днепр.

А в Бабьем Яру продолжали строчить пулемёты, это немцы уничтожали всех тех,
которые были задержаны и сидели в тюрьмах, а также военнопленных.  Дело в том,
что немцы находились на высоком берегу Днепра, а нашим войскам пришлось
наступать со стороны низкого берега и это создавало особую трудность для штурма
Киева. Кроме того необходимо было преодолеть  реку Днепр, при переправе которой
погибло очень много наших солдат, немцы расстреливали их сверху прямо в упор.  

Несколько дней в Киеве не  было никакой власти, все сидели по домам и боялись
выходить.  Вот так сидя у окна я увидел как по нашей улице проехал  танк  Т-34
в сторону Бульвара Шевченка. Мой брат Жора, который был на 9 лет старше меня
выбежал и побежал за танком. На углу Дмитриевской и Бульвара Шевченко, как раз
у Аптеки он остановился  и произошёл стихийный митинг. Жора как то смог
забраться на танк и потом в газете, к сожалению не помню её название  был
опубликован рассказ и помещено фото, где чётко был виден мой брат на танке. У
нас в доме долго хранилась эта газета.

Запомнилось ещё то, что сразу после освобождения Киева, заработала почта и люди
стали бояться почтальонов.  Как только почтальон заходил во двор — наступала
напряжённая тишина. Все смотрели к кому он идёт. Все боялись получить похоронку
— известие о гибели близких людей. А таких похоронок приходило много. Ведь
почти три года люди не  знали ничего о судьбе своих близких, которые сражались
на фронтах. 

К счастью для нас пришло письмо от папы, который писал, что жив и здоров.  Мы
стали получать, как семья офицера — паёк и другие льготы. Жить стало немножечко
легче.              

Мама пошла на старую работу на завод Большевик, а по выходным продолжала
выходить на Евбаз и немного подрабатывать на продаже перешитых ею вещей. Мы с
братом пошли учиться в Хореографическое Училище, которое было тогда на углу
улиц Воровского и Тургеневской, рядом с нашим домом. 

Запомнилось ещё — как маму посадили в тюрьму и не за какие то махинации или
воровство, а за то, что она несколько раз опоздала на работу, а один раз не
пришла, так как заболела, но не смогла доказать это. Такие в то время были
законы, за несколько  опоздания на работу  и  прогул  - садили в тюрьму. 

Так она просидела в тюрьме на Лукьяновке две недели. И только папино
ходатайство и то, что она жена офицера фронтовика, помогло ей выйти  досрочно.
Помню как мы её встречали, как она нам рассказывала о тех страхах и тюремных
ужасах. 

Если люди на свободе жили впроголодь, то представляете, что было в тюрьме. Она
рассказывала, что многие  женщины были без одежды и когда она выходила, она всю
одежду с себя отдала им, а оделась в то, что бабушка ей принесла. Но были и
приятные моменты, это когда мы с мамой и Жорой  ездили  на летние каникулы к
папе в Вену, где он служил в штабе Советских оккупационных войск в Австрии. 

Детская память конечно несовершенна, но мне очень хорошо запомнились эти
поездки, ведь я был уже школьник.  Отлично помню как мы на поезде доезжали до
Будапешта, где нас встречал папа. Крики на вокзале лоточников «
Цукерки-чоколад», « Цукерки - чоколад» и папа покупал нам всякие вкусности.
Потом он нас на машине возил по Будапешту, на озеро Балатон  и дальше в Вену.

Самое интересное, что через 30 лет, я ездил, фактически по этому же маршруту в
составе туристической  группы, но для этого я должен был выучить имена и
фамилии всех руководителей Коммунистических партий мира, а также прослушать
лекцию как мне нужно себя вести за границей, чтобы не опозорить выпускающую
меня в другой мир  Родину. Старшее поколение хорошо  помнят эти собеседования
перед поездкой за рубеж. 

В Вену меня правда, даже с такими знаниями не пустили, так как она уже была не
социалистической — как Венгрия, а капиталистической страной. Кстати она  была
единственной страной, которая не стала « народной демократией» после войны и
откуда были выведены все войска.  Последний Советский солдат покинул Австрию в
1955 году.

Но вернёмся в те времена, когда нас встретил счастливый папа. 

Хорошо запомнился разрушенный Будапешт и практически уцелевшая красавица Вена.
Австрия и после войны оставалась не только красивой но и благоустроенной и
ухоженной страной где сохранились постройки столетней давности, замки
феодальных времён, где многие носили одежду — фасон которой не изменился за
сотни лет. 

Мне было странно видеть, как возле нашего дома, где мы жили в Вене, постоянно
проезжал на велосипеде пожилой мужчина  в кожаных коротких штанишках ( шортах),
о которых мы тогда не имели понятия и  в  шляпке с вставленным в неё красивым
пером какой то экзотической птицы. Это был —австриец из горного Тироля, как мне
папа объяснил.

Было очень интересно из нашей нищеты  и примитивной жизни, без удобств  и воды
в доме  - пожить в том двухэтажном особнячке со всеми возможными удобствами,
где жили офицеры штаба. Хозяйка особняка была очень приветливой и довольной,
что к ней подселили не солдат а офицеров, так как ежедневно передвижная кухня
привозила еду прямо в особняк и этой еды вполне хватало не только семьям
офицеров , но и оставалось для  семьи хозяйки. А с продуктами в тот период ,
местные жители имели проблемы.

У папы был ординарец, который водил машину BMW, на которой ездил папа в штаб и
выполнял другие его поручения. Кстати на этой машине , после демобилизации папа
приехал в Киев прямо из Вены.

На ней мы путешествовали по Вене и её окрестностям.  Вся Австрия была разделена
на четыре части, как и Вена. Но в отличии от Берлина границы между зонами были
условными. Единственное ограничение было то,  что офицеры не имели права носить
оружие выезжая в город. Оружие можно было брать только , когда офицер ехал
домой в отпуск.. 

Поскольку оккупация Австрии  после войны  союзными войсками происходила в
особом режиме, временному правительству Австрии была дана гарантия, что Австрия
останется независимой страной, считалось, что служить в капиталистической
стране невыносимо трудно, поэтому каждый офицер имел в году два отпуска по 45
суток. Часть денег папа получал в местной валюте, а остальные деньги ему
перечисляли на расчётную книжку. Таким образом к отпуску собиралась приличная
сумма. 

Но были и неприятные моменты, особенно в первоначальный период , сразу после
окончания войны с Германией. Несмотря, на то, что война в Европе закончилась,
на Дальнем Востоке  она продолжалась. И вот представьте себе состояние
офицера или солдата, который выжил в этой бойне, а тут приходит разнарядка
отправить столько то офицеров и солдат на Дальний Восток, с перспективой там
погибнуть. Было очень страшно попасть в этот список. Слава Богу папа в него не
попал.

Из мест, которые мне запомнились в Вене особо могу отметить уголок
композиторов, где они  были похоронены и установлены  им памятники, особенно
Штраусу со скрипкой. В этом парке организовывались гулянья и выступления
музыкантов.  Дворец императора Франца Иосифа , в котором в одном крыле был
организован Дом офицеров, куда мы ходили смотреть фильмы.   В остальном дворец
сохранил всю свою первозданную прелесть. В залах царила обстановка красоты и
богатства.  Ездили мы на пикники и охоту на фазанов в Венский лес, красоту
которого воспел в своей музыке гениальный Штраус. 

Ходили мы  в Оперный театр, который внешне очень нам напоминал наш Киевский.
Нам с Жорой, уже  выступающим в Киевском  Оперном театре в спектаклях
Хореографического училища , в котором мы учились — было очень интересно с ним
ознакомиться.  

\ifcmt
  tab_begin cols=3

     pic https://scontent-cdt1-1.xx.fbcdn.net/v/t1.6435-9/215869908_4113691078678147_3140303228500611780_n.jpg?_nc_cat=109&ccb=1-3&_nc_sid=b9115d&_nc_ohc=PzhIjrexq14AX9iosW6&_nc_ht=scontent-cdt1-1.xx&oh=16c6b8fc2bf9ee4dce4d8bb595d5b93f&oe=612BDB46

     pic https://scontent-cdt1-1.xx.fbcdn.net/v/t1.6435-9/216391927_4113691825344739_5808365251433741193_n.jpg?_nc_cat=103&ccb=1-3&_nc_sid=b9115d&_nc_ohc=_pEZ3M4RvnkAX8JEGMI&_nc_ht=scontent-cdt1-1.xx&oh=27cfcbde838b7a55376498b31d901cc0&oe=612D3C3B

		 pic https://scontent-cdg2-1.xx.fbcdn.net/v/t1.6435-9/214035008_4113692702011318_180752225766621218_n.jpg?_nc_cat=100&ccb=1-3&_nc_sid=b9115d&_nc_ohc=VgLLwA50pHsAX9mzqoL&_nc_ht=scontent-cdg2-1.xx&oh=bebb12ebd380645d116c934b65d9c9c3&oe=612E586B

  tab_end
\fi


Посетили мы и  впервые сооружённый на территории освобождённой Европы памятник
Советским воинам. Он был открыт 19 августа 1945 года. Было удивительно как за
столь короткий строк могли соорудить такой величественный памятник, который
состоял из бронзовой скульптуры солдата  20 метров высоты, который стоял на
фоне полукруглой колонады из белого мрамора. Надпись на нём гласила « Вечная
слава воинам Красной Армии павшим в боях с немецко — фашистскими захватчиками
за свободу и независимость народов Европы»  Всех мест в которые нас папа возил
— не перечесть. 

Но всё хорошее, когда то кончается. Так и у нас наступило время
возвращаться домой в Киев. И если бы мы не видели своими глазами  как живут
побеждённые австрийцы, может бы мы и не замечали  разницы,  как нам надо за
водой выходить к колонке и в 20 градусный мороз ходить в туалет во дворе с
крысами. Чтобы покупаться надо сходить в баню. Стоять по 5-6 часов в очереди за
мукой раз в месяц. Отоваривать карточки на продукты. Всё как говориться —
познаётся в сравнении.

\ifcmt
  tab_begin cols=3

     pic https://scontent-cdg2-1.xx.fbcdn.net/v/t1.6435-9/210918811_4113694898677765_4773281424643118257_n.jpg?_nc_cat=107&ccb=1-3&_nc_sid=b9115d&_nc_ohc=VEUxAZD7G8cAX-28hXw&tn=lowUrFCbCbt-jOWu&_nc_ht=scontent-cdg2-1.xx&oh=c006a9c2b399c916ae0aebb4993e7948&oe=612EA40C

     pic https://scontent-cdt1-1.xx.fbcdn.net/v/t1.6435-9/210688699_4113696542010934_7404518115414815650_n.jpg?_nc_cat=103&ccb=1-3&_nc_sid=b9115d&_nc_ohc=n49Y1xqd-BcAX8LYt1k&_nc_ht=scontent-cdt1-1.xx&oh=0a5a814a0b7ffa03cd32925461e10656&oe=612C4D94

		 pic https://scontent-cdg2-1.xx.fbcdn.net/v/t1.6435-9/202244941_4113697945344127_6874840990414466027_n.jpg?_nc_cat=104&ccb=1-3&_nc_sid=b9115d&_nc_ohc=KYyenxWlX_EAX8gD5_X&tn=lowUrFCbCbt-jOWu&_nc_ht=scontent-cdg2-1.xx&oh=4342dd46c5c68bb4bdd6d314959d3ff3&oe=612EB1E5

  tab_end
\fi

На фото папа и его товарищ, тоже из Киева с мамой у машины BMW.

Фото — символизирующее дружбу воинов освободителей Австрии.

Памятник воинам освободителям в Вене. Открыт 19 августа 1945 года

Памятник композитору  Штраусу в Вене.

Оперный театр в Вене, который очень похож на наш Киевский театр.

Афиша Киевского Оперного театра с нашими именами.
