% vim: keymap=russian-jcukenwin
%%beginhead 
 
%%file slova.korabl
%%parent slova
 
%%url 
 
%%author_id 
%%date 
 
%%tags 
%%title 
 
%%endhead 
\chapter{Корабль}

%%%cit
%%%cit_head
%%%cit_pic

\ifcmt
  tab_begin cols=3

     pic https://regnum.ru/uploads/pictures/news/2021/10/22/regnum_picture_16348942292527105_normal.JPG
     pic https://regnum.ru/uploads/pictures/news/2021/10/22/regnum_picture_16348943002812285_normal.JPG
		 pic https://regnum.ru/uploads/pictures/news/2021/10/22/regnum_picture_16348942172373491_normal.JPG

  tab_end
\fi
%%%cit_text
Иван Федорович Крузенштерн — мореплаватель, учёный, один из учредителей
Русского географического общества, человек, совершивший первое русское
кругосветное плавание. Экспедиция 1803−1806 годов под командованием И. Ф.
Крузенштерна и Ю. Ф. Лисянского на \emph{кораблях} «Надежда» и «Нева» расширила
горизонты российской науки во многих областях знания — гидрологии и
гидрографии, картографии, биологии и этнографии
%%%cit_comment
%%%cit_title
\citTitle{«Крузенштерн. Вокруг света» — фоторепортаж}, 
Наталья Стрельцова, regnum.ru, 22.10.2021
%%%endcit

%%%cit
%%%cit_head
%%%cit_pic
%%%cit_text
Он добрался до \emph{корабля} «Космическая ловушка» без хлопот и приготовился
взлететь: протонный микрореактор был активирован, двигатель и средства
управления проверены, сверху чистое пространство – и тут в стратосфере появился
с трудом двигающийся грузовой \emph{корабль}, поврежденный метеоритом и не имевший
возможности посылать радиосообщения, отчаянно сигналящий, умоляя освободить
пространство.  Был подан сигнал об аварийной блокировке. Все \emph{корабли} в порту
были тут же задержаны. Любому \emph{кораблю} на старте, если только он уже не
находился в полете, предписывалось отменить взлет.  «Космической ловушке»
следовало отменить взлет, но она этого не сделала. Лакки Старр догадывался,
какие чувства должен был испытывать на борту агент Х. Он владел новейшей
информацией о Солнечной системе и каждая секунда была на счету.  Он не мог
полагаться на то, что пройдет достаточно много времени, прежде чем Совет будет
у него на хвосте. Отмени он свой взлет, и время задержки трудно предсказать.
Пока еще поврежденный грузовой \emph{корабль} медленно опустится, и машины «скорой
помощи» примут членов экипажа и возможных пассажиров... Потом, когда взлетное
поле снова освободят, понадобится реактивация микрореактора и повторная
контрольная проверка. Нет, он не мог позволить себе задержку.  Итак, его
реактивный двигатель выбросил струю, и он стартовал
%%%cit_comment
%%%cit_title
\citTitle{Лаки Старр и Кольца Сатурна}, Айзек Азимов
%%%endcit

