% vim: keymap=russian-jcukenwin
%%beginhead 
 
%%file 26_10_2020.sites.ru.zen_yandex.asov_alexandr.1.dvoeverie_hramov_vladimirskoj_rusi
%%parent 26_10_2020
 
%%url https://zen.yandex.ru/media/id/5eeaf725a3dca453cfdd4b58/dvoeverie-hramov-vladimirskoi-rusi-5f8f27a86dc8f92edac21d47
 
%%author 
%%author_id asov_alexandr
%%author_url 
 
%%tags ancient_russia
%%title Двоеверие храмов Владимирской Руси
 
%%endhead 
 
\subsubsection{Двоеверие храмов Владимирской Руси}
\label{sec:26_10_2020.sites.ru.zen_yandex.asov_alexandr.1.dvoeverie_hramov_vladimirskoj_rusi}
\Purl{https://zen.yandex.ru/media/id/5eeaf725a3dca453cfdd4b58/dvoeverie-hramov-vladimirskoi-rusi-5f8f27a86dc8f92edac21d47}
\ifcmt
  author_begin
   author_id asov_alexandr
  author_end
\fi

\index[rus]{Русь!История!Двоеверие храмов Владимирской Руси, 26.10.2020}
\index[rus]{Русь!Северная!Храм Покрова на Нерли, 26.10.2020}

\ifcmt
pic https://avatars.mds.yandex.net/get-zen_doc/3828082/pub_5f8f27a86dc8f92edac21d47_5f91563a2c1a69338e7fd751/scale_1200
caption Храм Покрова на Нерли
\fi

Храмы Владимиро-Суздальской Руси, возведенные в XII-XIII веках, называют песней
в камне... Но только песней, слова коей утрачены, неясны. В самом деле, что за
странные образы изображены на стенах Дмитровского собора во Владимире,
Георгиевского собора в Юрьев-Польском, на вратах и стенах Суздальских храмов?

Какие только мнения не высказывались на сей счёт. Чаще всего искали смысл
образов в апокрифической и ветхозаветной литературе. И лишь Б.А. Рыбаков
справедливо заметил, что эти образы имеют отношение к дохристианской культуре
Руси.

Казалось бы, чего проще? Обратиться именно к литературным памятникам XII века,
да и к более ранним. К примеру, к "Слову о полку Игореве", пронизанному
языческими образами. Ведь сам поход Игоря произошёл за десять лет до начала
строительства Дмитровского собора. Одного этого обращения достаточно, чтобы
ощутить народную стихию тех лет.

Да и сами образы, имеющиеся на стенах храмов, повторяют те же образы, имеющиеся
на вполне языческих предметах, наручах и прочем...

К примеру, можно сравнить их с образами на языческой братине из Черниговского
кургана... Вот посмотрим и сравним...

\ifcmt
pic https://avatars.mds.yandex.net/get-zen_doc/2359038/pub_5f8f27a86dc8f92edac21d47_5f9154dd2c1a69338e7d5108/scale_2400
caption Вверху 1) Дажьбог (А. Македонский) на грифоне с русальным жезлом, рядом Вавила; 2) в центре Жива с опущенными рукавами, рядом лучник-Ярила и князь Яр, отождествляемый с Гераклом; 3) в Центре Бус Белояр с дудочкой, и сражение, вероятно Златояра с Германарехом...
width 0.4
\fi

\subsubsection{Двоеверие на Владимирщине}

Ныне время до завоевания Руси монголами называют эпохой двоеверия. Полагают,
что князья и пришлое греческое духовенство насильно обращало народ в
христианскую веру, а народ стойко держался за старину, помещал новых
христианских святых рядом со старыми богами.

Но конечно же, всё было много сложнее. И если поначалу в народе христианство
приживалось с трудом, то спустя годы, после смены поколений, забвения старых
обрядов, вера, понимаемая только через обрядовое действо, стала внешне
христианской.

Древняя духовная ведическая традиция оказалась хранимой только образованным
сословием: частью священства, которое ратовало за соединение русской ведической
и византийской христианской вер, а также некоторыми князьями, чтившими старину.
Одичавшее же колдовское, ведовское язычество продолжало хорониться по окраинам,
оно ещё долго давало о себе знать, выливалось в восстания и языческую вольницу.

Так было и во Владимирской Руси рубежа XII-XIII веков, когда при князе Андрее
Боголюбском, а потом и при его брате Всеволоде Большое Гнездо стали возводится
удивительные храмы.

\ifcmt
pic https://avatars.mds.yandex.net/get-zen_doc/3633274/pub_5f8f27a86dc8f92edac21d47_5f9145d26dc8f92eda2e7634/scale_1200
caption Дмитриевский собор
width 0.4
\fi

Тогда среди Владимирского священства князем Андреем Боголюбским был выделен и
выдвинут в епископы отец Феодор. Современники его прямо называли волхвом, ибо
он выступал за сохранение прадедовских обычаев, с частности ратовал за
мясоедение в православные праздники, совпадавшие с древними ведическими. В 1169
году отец Феодор был проклят митрополитом Константином, греком. Он был признан
"волхвом" и "еретиком", а потом казнён в Киеве, тогда были сожжены и некие
писанные им книги. Следует сказать, что дядей Феодора был боярин Петр
Бориславович, предполагаемый автор "Слова о полку Игореве", — как видим, в сём
семействе свято чтили древлюю веру. Очевидно, духовная жизнь во Владимирской
Руси тех лет была своеобразной, епископами были... бывшие волхвы, более
почитавшие традицию, восходившую к Бояну, чем к Моисею.

И сии волхвы-епископы были поддерживаемы князьями. Так Андрей Боголюбский
вместе с поддержавшими его берендеями-язычниками (в летописях всегда
указывалось, что берендеи были "поганой веры") из-за казни отца Феодора пошёл
войной на Киев, и сжёг его, и особо тогда пострадали "от поганых" киевские
монастыри и церкви.

Продолжил ту же линию и Всеволод Большое Гнездо, сменивший брата Андрея на
престоле. Известно, что даже сына своего Юрия (будущего святого правителя
сокровенного Китеж-града) он постриг по языческому (ведическому) обряду при
вступлении в совершеннолетие.

И о том же мировоззрении, в коем соединились образы Евангелий и Вед, говорят и
стены храмов, возведённых в те годы.  

\subsubsection{Каменные Песни Владимирских храмов}

\paragraph{1. Песнь о Вышне и Свароге.}

Один из древнейших храмов Владимирской Руси — Георгиевский собор в
Юрьев-Польском. Начато его строительство было при Юрии Долгоруком, а закончено
при его внуке, сыне Всеволода III Святославе накануне монгольского завоевания.

Георгиевским сей собор был назван в честь небесного покровителя Юрия
Долгорукого святого Георгия. Но когда, при Всеволоде III и Святославе, он обрел
каменное убранство, вероятно, в народе его стали называть также и храмом
Триглава.

Кроме многочисленных львов и китоврасов, персонажей как Вед, так и библейских
сказаний, мы можем видеть угловую колонну собора, украшенную Триглавом. Это
явно не христианская Троица, ибо её не изображали так. Мы видим даже два
Триглава: один (Малый) порождает другой (Большой).

\ifcmt
pic https://avatars.mds.yandex.net/get-zen_doc/3986710/pub_5f8f27a86dc8f92edac21d47_5f9145842c1a69338e625336/scale_1200
caption Триглавы Георгиевского собора
width 0.4
\fi

Здесь в камне явлен важнейший образ Ведической Традиции: образ Миропроявления
Триединого Вышня, Большого Триглава. Вышень родился в нашей Вселенной из
Триединого Отца, Творца-Твастыря, или Сварога-Твастыря, то есть из Малого
Триглава. Вышень — Творец всех миров, вначале Он Сам родил Своего Отца, а потом
стал Его Сыном.

Ибо Вышень "Сын, воздвигнувший в Яви с Навью два великих имя Сварога"; Он также
"возвысил и третье имя во Сварожьей небесной выси" ("Книга Коляды" II а). Так
Триединый Сварог породил Триединого Вышня, который "шагнул три раза - широко
чрез простор Вселенной". Вышний — "Юноша — Сын Закона, Явь и Навь и Навь и
Правь перешедший. Тот, в следах чьих - источник меда, в высшем следе - сияет
Сурья. Тот, следы чьи соединяют триедино Землю и Небо" ("Книга Коляды" II а).
Вышний проявлен в Трёх ликах: Явью, Навью и Правью.

Этот текст Вед и иллюстрирует колонна Георгиевского собора в Юрьев-Польском.

\paragraph{2. Песнь о Дажьбоге.}

Более всех украшен каменными образами Дмитровский собор. Он был построен как
дворовый храм Всеволода III в 90-х годах XII-го века. Назван он был Дмитровским
в память святого Дмитрия Солунского. Дмитрием был назван при крещении и сам
князь Всеволод Большое Гнездо, но всё равно князя мы знаем только под его
славянским (языческим) именем.

Это может показаться удивительным, но большая часть рельефов храма со дня его
постройки и до недавнего времени не выставлялась на обозрение. Сии рельефы,
именно те, что изображали героев Вед, были закрыты пристройками, галереями и не
были видны никому. А иные рельефы, в том числе и христианские, были скрыты от
горожан высокими стенами детинца, ибо скульптуры и рельефы не были приняты
византийским духовенством, боровшимся и с античным наследием.

Очевидно, указание скрыть рельефы дал ещё сам Всеволод, дабы не вызвать
неудовольствия у ортодоксального священства. Раскрыты рельефы были только после
реставрационных работ в 1837-1839 годах, когда смысл изображённого был уже
неясен.

И что же мы видим нас стенах храма кроме христианских сюжетов: Крещения
Господня, изображений святых и прочего? Во-первых изображения самого Дажьбога и
его священного животного Царственного Льва (на рельефах Дмитровского собора 125
раз изображены львы).

\ifcmt
pic https://avatars.mds.yandex.net/get-zen_doc/2376594/pub_5f8f27a86dc8f92edac21d47_5f91515a2c1a69338e772d36/scale_1200
caption Дажьбог (Давид)
  width 0.3
\fi

В центральной закомаре западного фасада мы видим Дажьбога с гуслями (или
свитком законов?). Особенно ценно то, что рельеф сопровождает надпись: ДА-БЪ.
Эта надпись означает именно Дажьбог (или Дабог), прочтение Давид вторично и
сомнительно. Замечу, крамольную надпись до сего времени тщательно затушевывали
при всех изданиях фотографий сего рельефа, но её можно наблюдать просто подойдя
к храму.

На южном фасаде (правая закомара) вновь изображен Дажьбог, летящий на волшебных
птицах. Обычно, это изображение толковали как изображение летящего на птицах
Александра Македонского (за которым Б.А. Рыбаков разглядел и самого Дажьбога).
Хотя с тем же успехом его можно было толковать и как изображение Кея Кавуса из
иранских преданий.

Все сии образы восходят к одному и тому же праобразу: летящему на волшебных
весенних птицах Тарху Дажь-богу (или Аполлону Таргелию, Вишну).

\ifcmt
pic https://avatars.mds.yandex.net/get-zen_doc/3518430/pub_5f8f27a86dc8f92edac21d47_5f915a222c1a69338e86c7d6/scale_1200
caption Дажьбог-Аполлон-Александр Македонский
  width 0.3
\fi

Да и в "Книге Коляды" (XII в) говорится, что Дажьбог прилетает весной на
"златой колеснице Солнца". Колесницу ту "Книга Коляды" (I а) называет также и
"золотой лодкой". "Гамаюн колесницу ту принесла. Как летит Гамаюн в небе синем
— приклоняются все дубравы". А в "Книге Велеса" (Крынь 7:1) сказано: "Дажьбог
на лодке своей плывет в Сварге синей (в небе синем)... И мы видим, как золото
то Огнебогом распалено, И это дух всякой жизни, и прибежище тварям земным. И
всякий благой муж может видеть это, а злому Бог не дает зрения..."

Этот же образ Дажьбога-Аполлона мы видим и на братине из Черниговского клада
(см. выше). Более подробно о символике сего образа сказано также в статьях
"Космопорт Дажьбога"\Furl{https://zen.yandex.ru/media/id/5eeaf725a3dca453cfdd4b58/kosmoport-dajboga-5ef7989421ac3b028e165bf9?integration=morda_zen_lib&place=export} и "Как Аполлон по небу летал"\Furl{https://zen.yandex.ru/media/id/5eeaf725a3dca453cfdd4b58/kak-apollon-po-nebu-letal-i-dlia-chego-emu-byl-nujen-trenojnik-5f884ec932cf03146245d695?integration=morda_zen_lib&place=export} и проч...

\paragraph{3. Песнь о Перуне и детях его.}

На стенах Дмитровского собора можно найти и иные языческие (ведические) сюжеты.
В середине правой арки западного фасада мы видим такую группу: Лучник пускает
стрелу в огромную Птицу, которая преграждает Лучнику путь к некому Чудовищу.

\ifcmt
pic https://avatars.mds.yandex.net/get-zen_doc/759807/pub_5f8f27a86dc8f92edac21d47_5f9165546ca69c7802b5dcdd/scale_1200
  width 0.3
\fi

То, что этот рельеф изображает одну из песен "Русских Вед", и не может быть
отнесён к сюжетам античного язычества, легко показать, сопоставив сей рельеф с
изображением Лучника и Птицы с оковки турьего рога из Чёрной Могилы (русское
языческое захоронение). Кажется, что эти изображения скопированы с одного
рельефа, насколько они близки. Преемственность именно русской языческой
изобразительной традиции здесь очевидна.

\ifcmt
pic https://avatars.mds.yandex.net/get-zen_doc/50335/pub_5f8f27a86dc8f92edac21d47_5f916653c2b29d22941fa522/scale_1200
caption Оковка Турьего рога из Чёрной могилы
  width 0.3
\fi

Этот сюжет хорошо известен по "Русским Ведам". Речь здесь идёт о подвиге
Перуна, который отправился в Навь, дабы поразить Скипера-зверя, царя загробного
мира. Путь ему преградила Птица Могол (или Гамаюн). Перун пустил громовую
стрелу в Птицу, и та пропустила Перуна. "Тут снимал Перун лук тугой с плеча,
натянул тетивочку шелковую — и пустил стрелу позлачёную. Прострелил Птице
правое крылышко — из гнезда тотчас птица выпала..." (Книга Коляды V б).

Потом этот подвиг повторяли дети и внуки Перуна: Дажьбог, Яр-Арий, а потом и
Зарян (у греков он же Геракл), затем Рус. На соседнем рельефе мы видим того же
героя, который булавой поражает льва. 

\ifcmt
pic https://avatars.mds.yandex.net/get-zen_doc/1888987/pub_5f8f27a86dc8f92edac21d47_5f91676c6ca69c7802b9ab49/scale_1200
  width 0.3
\fi

В герое ныне обычно видят Геракла, но мы и в нём скорее готовы признать
опять-таки Перуна, который также сражался со Львом (своим сыном Дажьбогом,
обернувшимся в Льва). Тем более, что булава его мало напоминает дубину Геракла,
она более походит на громовую "тоегу" Перуна (смотри, например, Перуновы тоеги
из храмов Ретры и Свинторога).

В поздних русских былинах с Птицей-Соловьем у речки Смородины сражается Илья
Муромец, а с Птицей Моголом Иван-Царевич. Но это поздние переложения того же
сюжета. Тем более, в белорусских сказах место сих героев занимает опять таки
Царь Гром или Пярун. Так что, у нас есть все основания видеть в сих рельефах
Дмитровского собора изображение Перуна.

\paragraph{Песнь о Велесе и Змее Огненном Волхе.}

Ведическое имя Всеволода значит "всем владеющий". Таковое имя даёт носящему его
покровительство Велеса — бога властителя, владыку, а также покровительство
Волха (Вольги, Вука), воинской ипостаси Велеса.

Велес почитался и покровителем самого города Владимира, Изображения Велеса
находили на древних урочищах близ Владимира, здесь немало Волосовых мест,
Ярилиных полян (Ярила — весенняя ипостась Велеса).


\ifcmt
  pic https://avatars.mds.yandex.net/get-zen_doc/1347728/pub_5f8f27a86dc8f92edac21d47_5f916c276ca69c7802c31acb/scale_1200
  caption Велес и птица Матерь Сва
  width 0.3
\fi

Потому Велеса и Волха мы также находим среди изображений на стенах Дмитровского
собора. И прежде всего лик Велеса, помещенный рядом с Птицей, можно выделить на
восточных апсидах. Лик Велеса весьма похож на известный Велесов лик Збручского
идола.

В"Велесовой книге", Велес и Птица Матерь Сва-Слава также являются рядом, ибо
древние русичи почитали Велеса "как Отца Божьего: Отцом нашим, а Матерью —
Славу" (Род, Пров:6).

А воинский лик Велеса — бога Змея Огненного Волха мы находим на западном
фасаде, левой арке. Там изображен бог — полу-рыба, полу-дракон, полу-птица,
полу-человек, который сражается с Волком. В таком виде являлся только Волх,
ставший впоследствии былинным богатырем Вольгою. Волх, как Вольга (и
южнославянский Змей Огненный Вук) умел "обертываться Ясным Соколом и парить
легко по подоблачью", мог также "превращаться в Волка серого — рыскать Волком в
лесах дремучих", обращался он и "быстрой Щукой" и гулял по морю синему,
принимал и иные образы ("Книга Коляды" III д).


\ifcmt
  pic https://avatars.mds.yandex.net/get-zen_doc/4004066/pub_5f8f27a86dc8f92edac21d47_5f916cef6dc8f92eda749a2b/scale_1200
  caption Велес-Волх
  width 0.3
\fi

Изображенное на рельефе сражение Волха-Велеса с Волком суть сражение его с
Чёрным богом, являвшимся в виде Волка, который готов поглотить весь мир.

Венчает же ведические сюжеты Дмитровского собора Уточка Нави и Яви, Уточка
Велеса и Рода-Седыя, которая творила мир вместе со Сварогом в начале
Миротворения. Она находится на самой вершине, на кресте собора.

\ifcmt
  pic https://avatars.mds.yandex.net/get-zen_doc/1889318/pub_5f8f27a86dc8f92edac21d47_5f916de86dc8f92eda765fb4/scale_1200
  width 0.3
\fi

И эта Уточка, как и образ Велеса, сражающегося с Волком, представляет вечное
сражение между силами Нави и Яви, которое и являет саму Правь, Небесный Закон.
Об этом нам говорят все рельефы Дмитровского, Покровского и Георгиевского
соборов.

\ii{26_10_2020.sites.ru.zen_yandex.asov_alexandr.1.dvoeverie_hramov_vladimirskoj_rusi.comments}


