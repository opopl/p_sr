% vim: keymap=russian-jcukenwin
%%beginhead 
 
%%file 07_01_2022.vk.doneck_hud_muzej_posetiteljam.1.rubrika_pro_plastilin
%%parent 07_01_2022
 
%%url https://vk.com/naych_prosvet_otdel_drhm?w=wall-206859301_673
 
%%author_id doneck_hud_muzej_posetiteljam
%%date 
 
%%tags donbass,zhizn,kultura,doneck,dnr,hudozhnik,kartina,isskustvo
%%title МУЗЕЙНАЯ РУБРИКА «PROПЛАСТИЛИН»
 
%%endhead 
\subsection{МУЗЕЙНАЯ РУБРИКА «PROПЛАСТИЛИН»}
\label{sec:07_01_2022.vk.doneck_hud_muzej_posetiteljam.1.rubrika_pro_plastilin}

\Purl{https://vk.com/naych_prosvet_otdel_drhm?w=wall-206859301_673}
\ifcmt
 author_begin
   author_id doneck_hud_muzej_posetiteljam
 author_end
\fi

МУЗЕЙНАЯ РУБРИКА «PROПЛАСТИЛИН»

Михаил Нестеров - выдающийся русский художник, его картины можно увидеть во
многих музеях России.

\ii{07_01_2022.vk.doneck_hud_muzej_posetiteljam.1.rubrika_pro_plastilin.pic.1}

В работах мастера чувствуется «русская тема», одухотворённая, наполненная,
богатая. Многие русские художники успешно работали в этом ключе, но Нестерову
удалось нащупать свою интонацию, без которой невозможно творчества крупного
художника. Оттого картины живописца затрагивают струны души до сих пор, волнуя
воображение. И мы, свидетели суетного технологичного XXI века, восхищаемся и
тоскуем о временах которые не застали, но которые смотрят на нас с полотен
Нестерова, о временах далёкой и давно ушедшей Руси, прекрасной и искренней…

Если бы нужно было одним словом описать творческий метод Нестерова, то это было
бы слово «упорство». Свои картины он создавал подолгу, много раз переписывая
даже, казалось бы, незначительную деталь. Часто оставлял работу и принимался за
нее вновь, но уже на новом холсте. Своими работами был редко когда доволен.
Многие из картин уничтожил. При этом не был лишен тщеславия. Признавался, что
любит приходить в Третьяковскую галерею, чтобы посмотреть реакцию публики на
свои картины.

В Донецком республиканском художественном музее находится \enquote{Зимний
пейзаж} Михаила Нестерова, который является этюдом к картине \enquote{Святая
Русь} 1905 г., находящейся в ГРМ.

\enquote{Святая Русь} - одно из самых масштабных и ключевых произведений
живописи Серебряного века. На полотне изображены Иисус Христос и пришедшие к
нему богомольцы. Основой, на которой стоит сюжет этой картины, являются слова
Христа из Евангелия: «Приидите ко Мне все труждающиеся и обремененные и Аз
успокою вы». Эта же фраза считается неофициальным вторым названием работы.

На этюде \enquote{Зимний пейзаж} который находится в нашем музее мы видим
заснеженную холмистую местность с высоким горизонтом.  На первом плане справа
от центра ствол березки,  окруженный редкой порослью,  справа - маленькая
елочка. На втором плане  справа опушка - спускающегося с холма
серого-коричневого леса с вкраплением темных,  стройных елей.  Вдали белая
равнина,  на горизонте темная полоска леса.  Небо белесое.

Работа была закуплена МК СССР у московского коллекционера Гордона Л. В. и
передана на постоянное хранение в Донецкий художественный музей в 1968 году.

\ii{07_01_2022.vk.doneck_hud_muzej_posetiteljam.1.rubrika_pro_plastilin.pic.2}

Сегодня я поделюсь с вами как раскатать тонкие нити-жгутики для ветвей
деревьев.

Для начала мы разминаем пластилин руками, чтобы он был эластичным и податливым
в работе. Тонкие нити-жгутики можно создать, аккуратно вытягивая мягкую массу
пальцами.

Постепенно ладонями, а затем и пальцами рук, в большей степени, средними и
верхними фалангами раскатать жгутик. Маленькие тонкие жгутики раскатывают
кончиками пальцев и чем сильнее давить на пластилин, тем тоньше получится
жгутик. При этом важно следить за тем, чтобы движения были мягкими, плавными, а
жгутик раскатывался равномерно.

Ждем вас в музее на мастер-классе по пластилинографии каждую субботу и
воскресенье в 11.00 и 13.00.

Запись по телефонам: (062) 381-00-64, (071) 489 65 51, (071) 468 54 27.
 
ЮЛИЯ МИХАЙЛЕНКО
 
\begin{verbatim}
	#ДРХМ
	#Музейная_рубрика_ДРХМ
	#Музейная_рубрика_PROПЛАСТИЛИН
	#музейные_предложения_ДРХМ
\end{verbatim}
