% vim: keymap=russian-jcukenwin
%%beginhead 
 
%%file slova.materia
%%parent slova
 
%%url 
 
%%author_id 
%%date 
 
%%tags 
%%title 
 
%%endhead 
\chapter{Материя}

%%%cit
%%%cit_head
%%%cit_pic
%%%cit_text
Проводячи свій експеримент, я запитував сам себе: чому тоді, коли наша наука й
техніка здатна творити дивовижні суперлайнери — земні, повітряні, космічні,
вакуумні — для мандрів у буттєвих сферах, — я маю користуватися архаїчними
човниками мислення у спробі збагнути своє покликання? Схилятися перед
відкриттями Евкліда, Птолемея, Кеплера, коли вже навіть осяяння Ейнштейна та
Ціолковського старіють на наших очах?  \emph{Матерія} — найпластичніша
субстанція самотворення, це показує еволюція світу: навіть погляд дилетанта
може відзначити, що одна й та ж у суті своїй клітина несе в собі здатність бути
й органом бачення — оком, і органом слуху — вухом, і мозком, і роговицею, і
епітелієм шкіри, і м’язом, і легенею, і кров’яним тільцем, і всім, чим треба...
\emph{Матерія} — дивовижний самотворець, мати й основа для творення водночас.
Вона — ідея й вияв, задум і втілення, творець і утвір, батько й син, музичний
інструмент і прекрасна мелодія, слово— логос і слухач космічної поеми віків...
%%%cit_comment
%%%cit_title
\citTitle{Вогнесміх}, Олесь Бердник
%%%endcit
