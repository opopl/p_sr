% vim: keymap=russian-jcukenwin
%%beginhead 
 
%%file 06_04_2020.stz.news.ua.mrpl_city.1.istoria_odnogo_dnja_chastyna_3
%%parent 06_04_2020
 
%%url https://mrpl.city/blogs/view/istoriya-odnogo-dnyakniga-znahidka-chastina-iii
 
%%author_id demidko_olga.mariupol,news.ua.mrpl_city
%%date 
 
%%tags 
%%title "Історія одного дня" – книга-знахідка (Частина III)
 
%%endhead 
 
\subsection{\enquote{Історія одного дня} – книга-знахідка (Частина III)}
\label{sec:06_04_2020.stz.news.ua.mrpl_city.1.istoria_odnogo_dnja_chastyna_3}
 
\Purl{https://mrpl.city/blogs/view/istoriya-odnogo-dnyakniga-znahidka-chastina-iii}
\ifcmt
 author_begin
   author_id demidko_olga.mariupol,news.ua.mrpl_city
 author_end
\fi

{\em
Дорогі читачі, саме цього дня, 6 квітня, у Маріуполі розгорталися дуже цікаві і
незвичні події. Щоправда, це відбулося 157 років тому... Пропоную продовжити
ознайомлюватися зі змістом унікальної книги, кульмінаційний момент якої,
впевнена, здивує багатьох з вас.

Нагадую, що уривки подані мовою оригіналу:
}

\ii{06_04_2020.stz.news.ua.mrpl_city.1.istoria_odnogo_dnja_chastyna_3.quote.1}

\emph{В наступних наказах Ільяшенко вимагав одразу ж заарештували Логафетова, поголити його і передати під суд.}

\ii{06_04_2020.stz.news.ua.mrpl_city.1.istoria_odnogo_dnja_chastyna_3.quote.2}

\begingroup
\em От так вирішив Григорій Ільяшенко помститися за свого знайомого Поддубню. Треба
визнати, він мав неабиякі акторські здібності. Але він і уявити не міг, що його
план виявиться дійсно успішним, адже всі в місті повірили, що він є
високопосадовцем і всі накази справжні. Навіть капітан команди Іван Павлович
Лисенко не впізнав почерк свого писаря і виконував всі накази лженачальника.
Більше того на суді над Логофетовим Ільяшенко настільки сміливо себе поводив,
що всі остаточно переконалися – перед ними справжній Уповноважений Государя.
Логафетов змирився зі своєю участю і одразу ж був поголений. Тепер вже в ролі
арештанта він чекав на виїзд з міста. А маріупольці робили все, щоб догодити
високому начальнику, Уповноваженому Государя Імператора Всеросійського
Ільяшенку, який на застіллі, влаштованому на його честь, так багато випив, що
виїхати з міста одразу не зміг і залишився ще відпочивати. Мабуть, якщо виїхав
би раніше, ця історія закінчилася б для нього цілковитим тріумфом. Однак в
Маріуполі знайшлися ті, хто зрозумів, що він – справжній самозванець і всі його
накази є вигадкою. Хто зміг викрити головного героя книги і чи вдалося йому
врятуватися, ви дізнаєтеся згодом... 
\endgroup

\textbf{\emph{Читайте також:}} 

\begin{itemize} % {

\item \href{https://archive.org/details/04_03_2020.olga_demidko.mrpl_city.knyga_istoria_odnogo_dnja}{%
\enquote{Історія одного дня} – книга-знахідка про Маріуполь (I Частина)}%
\footnote{\enquote{Історія одного дня} – книга-знахідка про Маріуполь (I Частина), Ольга Демідко, mrpl.city, 04.03.2020, \par%
\url{https://mrpl.city/blogs/view/istoriya-odnogo-dnyakniga-znahidka-pro-mariupol}, \par%
Internet Archive: \url{https://archive.org/details/04_03_2020.olga_demidko.mrpl_city.knyga_istoria_odnogo_dnja}
}

\item \href{https://archive.org/details/23_03_2020.olga_demidko.mrpl_city.1863_istoria_odnogo_dnja_chastyna_2}{\enquote{1863: Історія одного дня} – книга-знахідка про Маріуполь (Частина II)}%
\footnote{\enquote{1863: Історія одного дня} – книга-знахідка про Маріуполь (Частина II), Ольга Демідко, mrpl.city, 23.03.2020, \par%
\url{https://mrpl.city/blogs/view/1863-istoriya-odnogo-dnyakniga-znahidka-pro-mariupol-chastina-ii}, \par%
Internet Archive: \url{https://archive.org/details/23_03_2020.olga_demidko.mrpl_city.1863_istoria_odnogo_dnja_chastyna_2}
}

\end{itemize} % }

%\end{minipage}
%\restoregeometry
