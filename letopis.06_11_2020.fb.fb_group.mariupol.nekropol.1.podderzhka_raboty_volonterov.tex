%%beginhead 
 
%%file 06_11_2020.fb.fb_group.mariupol.nekropol.1.podderzhka_raboty_volonterov
%%parent 06_11_2020
 
%%url https://www.facebook.com/groups/278185963354519/posts/405404460632668
 
%%author_id fb_group.mariupol.nekropol,arximisto
%%date 06_11_2020
 
%%tags 
%%title Поддержка работы волонтеров по восстановлению Мариупольского Некрополя
 
%%endhead 

\subsection{Поддержка работы волонтеров по восстановлению Мариупольского Некрополя}
\label{sec:06_11_2020.fb.fb_group.mariupol.nekropol.1.podderzhka_raboty_volonterov}
 
\Purl{https://www.facebook.com/groups/278185963354519/posts/405404460632668}
\ifcmt
 author_begin
   author_id fb_group.mariupol.nekropol,arximisto
 author_end
\fi

%\vspace{0.5cm}
\bigskip

Друзья, мы \textbf{обращаемся к вам с просьбой поддержать работу волонтеров по
восстановлению Мариупольского Некрополя!} К сожалению, мы исчерпали все наши
ресурсы.

\textbf{Нужно 5-6 тысяч гривен} для последнего рывка перед холодами – чтобы обустроить
древний участок Некрополя и создать хотя бы один \enquote{уголок цивилизации} посреди
разрухи...

На этом участке – самая древняя плита 1834-го года, склепы Александра
Хараджаева, городского головы в 1860-64 годах и рода Гофов, могила сестер
Гозадиновых, родственниц митрополита Игнатия, целый ряд древних плит без дат
или имен...

\textbf{Что мы сделаем при вашей поддержке?} Проложим дорожки (нужно плотное
агроволокно), высадим цветы и \enquote{благородные деревья} (мускари, крокусы; рябины,
клены), покрасим кресты и ограды, а также расширим аллею, которая ведет к
участку (спилим и уберем несколько сухостойных \textbackslash\ выродившихся деревьев).

Мы будем искренне благодарны за ваши пожертвования на карточку Приватбанка №
5168 7554 5285 3746 (получатель Андрей Марусов, директор ГО \enquote{Архи-Город}).

Что мы предлагаем в знак нашей благодарности? Бесплатную экскурсию по
Некрополю, включая демонстрацию того, как мы потратили ваши деньги. И, конечно
же, - публичное признание!

Поэтому мы очень просим каждого благотворителя сообщить нам свое имя и фамилию
в мессенджере \enquote{Архи-Города} (зачастую имя отправителя в системе Привата
неизвестно).

Мы также будем отчитываться об их использовании – фотографии сделанного, чеки.

Заранее благодарим вас за поддержку!

А теперь - \emph{детальный бюджет}

\emph{Плотное агроволокно} – для укладки дорожек в древней части и уничтожения поросли
сирени. Последнюю можно уничтожить и ядами, но тогда мы убьем почву на
десятилетия... Это неприемлемо.

Цена – 20-25 грн. за погонный метр. Нужно метров 40-50, около 1 000 грн.

\emph{Саженцы} \enquote{благородных} лиственных деревьев (клены, рябины, липы, осины и т.д.) –
обустройство Некрополя невозможно без его очистки от зарослей. Это означает,
что после очистки он превратится в пустыню из разрушенных надгробий и оград.
Чтоб сократить \enquote{пустынное время} - деревья нужно садить уже сейчас!

Нужно не менее 8 саженцев. Цены на саженцы рябины скандинавской \textbackslash\ клена
варьируют от 100 грн. (однолетний, на ЦТ) до 370 грн. (рябина уже плодоносит;
питомник возле телебашни). По средней цене в 200 грн. – нужно 1 600 грн.

Знаете, где купить дешевле? Подскажите! (выкапывать в посадках – все же не
самый лучший вариант...)

\emph{Мускари и крокусы.} Мы хотим, чтобы уже следующей весной вокруг памятников
Хараджаева и Гофов расцвели цветы. Цена в супермаркете за 1 луковицу – 5 грн.
Планируем высадить 200 цветов. Стоимость – \emph{1000} грн.

\emph{Расходники для пилы} (бензин, два вида масла) – 750 грн. (три выходных по 250
грн. каждый цикл).

\emph{Другие товары} (краска, ацетон, кисти, перчатки, щетки по металлу...) – около \emph{900}
грн.

\emph{Итого:} необходимо не менее \textbf{\emph{5 250 грн}}.

p.s. Если вы можете подарить что-то из вышеперечисленного - пожалуйста,
напишите \textbackslash\ позвоните! Участница одной из экскурсий по Некрополю предлагала
тачку - мы не успели взять контакты, пожалуйста, отзовитесь!

Контактный телефон - 096 463 69 88. 
