% vim: keymap=russian-jcukenwin
%%beginhead 
 
%%file literatura
%%parent body
 
%%url 
 
%%author 
%%author_id 
%%author_url 
 
%%tags 
%%title 
 
%%endhead 

14.05.2021
\url{https://t.me/authors_anonymous/1203} 

ЩО РАДИТЬ ПОЧАТКІВЦЯМ ДЖ. Р. Р. ТОЛКІН?

На зв'язку "Поради класиків". 
Що ж радить авторам-початківцям творець "Хоббіта" й "Володаря перснів"?

☝️ Хвилювання марні.
Справді, Токін писав свої книги, щоб догодити собі і письменникові всередині нього. Він очікував, що вони підуть "у макулатуру", коли покинуть його стіл. Насправді ж вони набули нечуваної популярності. Тож якщо ви розважаєте себе, ви вже знаєте одну людину, яка насолоджується вашою книгою.

☝️ Продовжуйте писати, долаючи труднощі.
Письменникові знадобилося СІМ років, щоб написати «Хоббіта». Він збалансував вимогливу денну роботу, хворобу та хвилювання за свого сина, який був далеко від Королівського флоту. 

☝️ Слухайте критиків, яким довіряєте.
Коли його редактор сказав: «Зробіть це краще», Толкін не відкинув поради. Він читав і перечитував, і намагався з усіх сил.
Автор послухав обізнаних відгуків та працював над тим, щоб покращити твір. Хто ж був редактором, якого він слухав? С. С. Льюїс, творець "Хронік Нарнії".

☝️ Нехай ваші інтереси керують вашим письмом.
Спочатку Толкін цікавився мовами. Він скористався цим і створив нові мови, а потім і цілу культуру навколо цього. Чим цікавитесь ви? Перетворіть це на фантастичну історію.

☝️ Поезія може привести до великої прози.
Коли Толкін не міг висловити свої думки бажаною прозою, він писав багато у віршах. Наступного разу, коли застрягнете, можете спробувати фокус Толкіна: спершу напишіть сцену у формі вірша.

☝️ Щасливі аварії.
Скільки б ви не планували, на сторінках кожної книги трапляються щасливі аварії. І до цього треба ставитися як до справжнього скарбу для вашої історії.

☝️ Мрії дають нам натхнення.

У всіх нас є сильні мрії. Але як бути з буквальними мріями?

Коли Толкін мріяв втопитися, він перетворив цей досвід на мотиви та прозу для
своїх оповідань. У його "листах" це описується, як сон, а згодом це
перетворюється на відчуття вторгнення Мордора в Середзем’я та потоплення
Ізенгарда.

☝️ Реальні люди стають чудовими персонажами.

Толкін спирався на реальних людей як на прототипи для населення Середзем’я. Ви
також можете орієнтуватися на своїх знайомих для створення історій. Реальні
люди роблять дивовижні речі, як великі, так і малі, а потім рідко впізнають
себе на сторінці. Це безпрограшний варіант для авторів.

☝️ Ви можете стати наступним автором бестселерів.

Толкін не очікував визнання, яке він отримав від своєї першої книги "Хоббіт".
Він відчував, що це "щаслива аварія". Ви не дізнаєтесь, чи матимете успіх, поки
не спробуєте. Можливо, наступним бестселером стануть ваші супергерої-однороги
(як варіант).

☝️ Книги, які ви пишете, можуть здатися банальними.

Ми не можемо оцінити власну роботу. Сцена, яку ми вважаємо мелодраматичною, для
читача може видатися зворушливою. Толкін вірив, що якщо ви навчитесь якогось
ремеслу і виллєте своє серце та фантазію на сторінку, робота отримає резонанс.
Я теж у це вірю.
