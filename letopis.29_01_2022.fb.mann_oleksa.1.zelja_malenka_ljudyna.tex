% vim: keymap=russian-jcukenwin
%%beginhead 
 
%%file 29_01_2022.fb.mann_oleksa.1.zelja_malenka_ljudyna
%%parent 29_01_2022
 
%%url https://www.facebook.com/olexa.mann/posts/5563446683671545
 
%%author_id mann_oleksa
%%date 
 
%%tags ukraina,zelenskii_vladimir
%%title Зеля – це яскрава квінтесенція трагедіі маленької людини
 
%%endhead 
 
\subsection{Зеля – це яскрава квінтесенція трагедіі маленької людини}
\label{sec:29_01_2022.fb.mann_oleksa.1.zelja_malenka_ljudyna}
 
\Purl{https://www.facebook.com/olexa.mann/posts/5563446683671545}
\ifcmt
 author_begin
   author_id mann_oleksa
 author_end
\fi

Зеля – це яскрава квінтесенція трагедіі маленької людини. В його варіанті просто
ну дуже маленької.

Все, як прописано в скрижалях класичних творів ХІХ століття, але якимось дивом
перенесене в ХХІ-ше. 

Але тут ще і трагедія маленької людини абсолютно не на своєму місці. Яка не
відповідає викликам часу, з регресивним і консервативним мисленням минулого (у
його варіанті совковим), де пошлий анекдотєц гундосим голосом заміняє швидку
реакцію на події, бо саме смішне в ситуації це ж і є, судячи з усього, цей
гундосий голос. І де абсолютно логічне, в рамках типажа, оце ось «зазирання у
вічі» і «я вам нічого не винний» і постійне ляпання язиком мимо каси. І,
звичайно, хамство, бо може. Бо так має спілкуватись «начальство», бачив, знає,
прогибався і не раз. І виніс цей єдиний урок. І, головне, відставання від подій
на дні, тижні, місяці і роки, незважаючи на стимулятори. Бо стимулятори, як і
технології на менталітет не впливають, а тільки підкреслюють. Дай персонажу
Гоголя пару дорожек на бачку унітаза і записати відосик, він і не таке нам
запише. Бо є в наявності лише дріб’язковість та вузкість поглядів, але ж
амбіція пре. Хочеться пійти вприсядку там де не можна по протоколу. І, як
наслідок, огида у мислячих людей, задовго до висування маленької людини на
президента і всієї цієї колотнечі, яка послідувала за цим штучно-технологічним
процесом. 

Всі розмови про «тварь дрожащую ілі право імею» закінчились просто твар’ю
дрожащей. Бо «право» ніхто ніколи і не вимагав. Бо навіщо. Від прав проблем
більше. 

Коротше кажучи, все по-класиці. І все давно описано. Просто заїбали вже вкрай
ці маленькі люди, як з каталогу персонажів. Їм би назад під перепльот і
шурудіти там собі між строк, свою пилюку нюхати разом з книжковими хробаками.
Так ні, на авансцену хочеться, що аж труситься. А потім розргібати це все
лайно.
