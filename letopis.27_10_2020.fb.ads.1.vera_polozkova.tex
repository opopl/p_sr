% vim: keymap=russian-jcukenwin
%%beginhead 
 
%%file 27_10_2020.fb.ads.1.vera_polozkova
%%parent 27_10_2020
%%url https://www.facebook.com/events/579375156174693/
 
%%endhead 

\subsection{Вера Полозкова в Киеве}
\label{sec:27_10_2020.fb.ads.1.vera_polozkova}

\url{https://www.facebook.com/events/579375156174693/}

“Не приехать в Киев на концерт, а придумать концерт, чтобы приехать 10 лет
назад впервые я оказалась в Киеве. Это вопрос жесткой зависимости от города.
Были времена, когда я приезжала каждый месяц. Моя жизнь делилась ровно на две
половины. Была Москва, в которой много работала и жила безрадостную одинаковую
жизнь, и Киев, где было все: друзья, любовь, слезы, вечеринки, эмоции, вещи,
книжки, концерты… Вся жизнь была сосредоточена в Киеве, вся работа была в
Москве. Я была здесь около сотни раз. Все серьезно. Конечно, это было и
продолжает быть местом силы, одним из главных в моей жизни."

Вера Полозкова выступит 2 декабря в МЦКИ Октябрьском с долгожданным концертом и презентацией новой книги.

Вера: «Работа горя» --- это книга, которая собиралась, копилась, нарастала семь
лет подряд: тяжелых лет, прекрасных лет, в течение которых родилось трое моих
детей, был заключен и рассыпался мой брак, скрежетала и лопалась по шву
привычная картина мира, начинались войны между самыми дорогими мне странами,
уходили из жизни мои друзья, люди вокруг меня являли чудеса душевной щедрости
или выбирали предать все, что провозглашали и любили; когда все, что я знала о
себе, все, каким я видела свое будущее, время несколько раз опрокинуло и
вывернуло наизнанку. Это книга о том, как тесно переплетены в нас все эти
состояния: бессилия, горечи, любви, ярости, смирения, отчаяния, упоения
настоящим. О том, что взросление --- это именно то, как мы пытаемся отгоревать,
что мы больше не всемогущи. То, как мы встречаемся со своей тенью, со своей
слабостью, то, как мы принимаем собственное несовершенство, неравенство своим
ожиданиям. Это книга о том, что ремесло --- спасает. О том, что моя душа нашла
свой дом, и это --- комнатка с верандой с видом на океан. Это книга о том, что
равновесие, знание, возможность полагаться только на себя стоят гораздо дороже,
чем мы думали в юности. Но сколько бы мы ни отдали за них, они самое главное
наше обретение.
