% vim: keymap=russian-jcukenwin
%%beginhead 
 
%%file 21_02_2023.stz.news.ua.donbas24.2.mizhn_den_ridnoj_movy
%%parent 21_02_2023
 
%%url https://donbas24.news/news/miznarodnii-den-ridnoyi-movi
 
%%author_id demidko_olga.mariupol,news.ua.donbas24
%%date 
 
%%tags 
%%title Міжнародний день рідної мови
 
%%endhead 
 
\subsection{Міжнародний день рідної мови}
\label{sec:21_02_2023.stz.news.ua.donbas24.2.mizhn_den_ridnoj_movy}
 
\Purl{https://donbas24.news/news/miznarodnii-den-ridnoyi-movi}
\ifcmt
 author_begin
   author_id demidko_olga.mariupol,news.ua.donbas24
 author_end
\fi


\begin{center}
  \em\color{blue}\bfseries\Large
  21 лютого в Україні святкують Міжнародний день рідної мови
\end{center}

\ii{21_02_2023.stz.news.ua.donbas24.2.mizhn_den_ridnoj_movy.pic.front}

В Україні вже 23 роки поспіль відзначають Міжнародний день рідної \href{https://donbas24.news/news/slovo-2022-roku-yakii-viraz-stav-golovnim-v-rik-viini}{\emph{мови}}.%
\footnote{Слово 2022 року: який вираз став головним в рік війни, Яна Іванова, donbas24.news, 10.01.2023, \par\url{https://donbas24.news/news/slovo-2022-roku-yakii-viraz-stav-golovnim-v-rik-viini}}

Це свято бере початок від трагічних подій 1952 року в Бангладеш. А офіційно
дата була встановлена у 1999 році, коли ЮНЕСКО прийняла рішення підтримати
мовне різноманіття у світі. З огляду на щорічне зменшення мов у світі, яке
пов'язано зі зникненням носіїв мови та втратою культури певного народу, це
свято набуває все більшої важливості та актуальності.

\textbf{Читайте також:} \emph{Як швидко перейти на українську мову: корисні поради}%
\footnote{Як швидко перейти на українську мову: корисні поради, Ольга Демідко, donbas24.news, 28.07.2022, \par%
\url{https://donbas24.news/news/yak-svidko-pereiti-na-ukrayinsku-movu-korisni-poradi}%
}

\subsubsection{Історія і традиції дати}

Міжнародний день рідної мови започаткували у листопаді 1999 року на ХХХ сесії
Генеральної конференції ЮНЕСКО з метою захисту мовної й культурної
багатоманітності. Головним гаслом свята є: \emph{\enquote{Мови — це важливо!}}.
Дату обрали не випадково. Саме 21 лютого 1952 року в Бангладеш відбувся
збройний конфлікт через мову. Тоді загинуло 4 студенти. Рідна мова — це носій
знань і цінностей, які передаються від одного покоління іншому в якості
нематеріального культурного спадку. З 1971 року, після проголошення
незалежності Бангладеш, цей день відзначають у країні як день мучеників, які
загинули за рідну мову. За пропозицією цієї країни ЮНЕСКО проголосило 21 лютого
Міжнародним днем рідної мови. На жаль, близько 43\% мов сьогодні перебувають на
межі зникнення. А 40\% людей не мають змоги отримати освіту рідною мовою.
Загалом у Європі під загрозою перебувають 30 мов, 13 із яких — на межі
зникнення.

Цього дня в Україні проводяться різноманітні освітні заходи, присвячені рідній
мові в школах, університетах та бібліотеках. Також цього дня на виставках та
майстер-класах популяризують народне мистецтво.

\textbf{Читайте також:} \emph{Мовні норми вступають в силу — яких правил треба дотримуватися}%
\footnote{Мовні норми вступають в силу — яких правил треба дотримуватися, Еліна Прокопчук, donbas24.news, 15.07.2022, \par%
\url{https://donbas24.news/news/movni-normi-vstupayut-v-silu-yakix-pravil-treba-dotrimuvatisya}%
}

\ii{21_02_2023.stz.news.ua.donbas24.2.mizhn_den_ridnoj_movy.pic.1}

Уповноважений із захисту державної мови Тарас Кремінь привітав всіх українців
із цим чудовим святом і наголосив, що \emph{\enquote{за українську мову, національну
ідентичність та право бути господарями на своїй землі гинуть найкращі сини та
доньки країни}.}

\begin{leftbar}
\em\enquote{За українську мову, національну ідентичність та право бути господарями на своїй землі гинуть кращі сини та доньки країни}.
\end{leftbar}

\subsubsection{Цікаві факти про українську мову}

\begin{itemize} % {
\item 1. Сучасна українська мова за останніми підрахунками має близько 256 тисяч слів.

\item 2. Особливістю української мови є наявність великої кількості
зменшувально-пестливих форм. Навіть слово \enquote{вороги} може звучати як
\enquote{вороженьки}. А назви всіх дитинчат тварин є іменниками середнього роду:
теля, котеня, жабеня.

\item 3. Українську мову забороняли близько 134 разів.

\item 4. Найдовше слово має аж тридцять літер. Воно означає хімікат для боротьби зі шкідниками: \enquote{дихлордифенілтрихлорметилметан}.

\item 5. Найближчою до української мови за лексичним запасом є білоруська — 84%
спільної лексики. Далі йдуть польська і сербська (70\% і 68\% відповідно),
потім — російська (62\%). Якщо порівнювати фонетику й граматику, то
українська має від 22 до 29 спільних рис з білоруською, чеською, словацькою
й польською мовами, а з російською — тільки 11.

\end{itemize} % }

\textbf{Читайте також:} \emph{Слово 2022 року: який вираз став головним в рік війни}%
\footnote{Слово 2022 року: який вираз став головним в рік війни, Яна Іванова, donbas24.news, 10.01.2023, \par\url{https://donbas24.news/news/slovo-2022-roku-yakii-viraz-stav-golovnim-v-rik-viini}}

\ii{21_02_2023.stz.news.ua.donbas24.2.mizhn_den_ridnoj_movy.pic.2}

\begin{itemize} % {
\item	6. Українська мова, на відміну від інших слов'янських мов, зберегла кличний відмінок. Тому в нас є цілих сім відмінків.

\item 7. Цікаво, що у 448 році візантійський історик Пріск Панійський під час
перебування в таборі гунського володаря Аттіли на території сучасної
України, записав слова \enquote{мед} і \enquote{страва}. Це була перша згадка
українських слів. Риси української мови наявні ще в писемних
пам'ятках XIV — XV століття.

\item 8. У різні історичні періоди українську мову називали по-різному: про́ста,
руська, русинська, козацька тощо. Історично найуживанішою назвою
української до середини XIX століття була назва \enquote{руська мова}.

\item 9. Найбільша кількість слів в нашій мові розпочинається на літеру \enquote{п}, а найменша — на літеру \enquote{ф}.

\item 10. Наша мова дуже багата на синоніми. Зокрема, слово \enquote{горизонт} має 12
синонімів: обрій, небозвід, небосхил, крайнебо, круговид, кругозір,
кругогляд, виднокруг, видноколо, виднокрай, небокрай, овид.
\end{itemize} % }

Текст підготовлено за наступними джерелами: \href{https://suspilne.media/106911-miznarodnij-den-ridnoi-movi-so-mi-znaemo-pro-ridnu-ukrainsku}{suspilne.media},%
\footnote{Міжнародний день рідної мови. Що ми знаємо про українську, Тетяна Яворович, suspilne.media, 21.02.2023, \par\url{https://suspilne.media/106911-miznarodnij-den-ridnoi-movi-so-mi-znaemo-pro-ridnu-ukrainsku}}
\href{https://np.pl.ua/2023/02/mizhnarodnyy-den-ridnoi-movy-istoriia-vynyknennia-tsikavi-fakty-ta-lystivky-pryvitannia}{np.pl.ua}%
\footnote{Міжнародний день рідної мови: історія виникнення, цікаві факти та листівки-привітання, Новини Полтавщини, 21.02.2023, \par\url{https://np.pl.ua/2023/02/mizhnarodnyy-den-ridnoi-movy-istoriia-vynyknennia-tsikavi-fakty-ta-lystivky-pryvitannia}}

Нагадаємо, раніше Донбас24 розповідав про безкоштовні \href{https://donbas24.news/news/bezkostovni-kursi-ukrayinskoyi-ucasnikam-dopomozut-pereiti-na-derzavnu-movu}{курси української мови}.%
\footnote{Безкоштовні курси української - учасникам допоможуть перейти на державну мову, Еліна Прокопчук, donbas24.news, 03.12.2022, \par\url{https://donbas24.news/news/bezkostovni-kursi-ukrayinskoyi-ucasnikam-dopomozut-pereiti-na-derzavnu-movu}}

Ще більше новин та найактуальніша інформація про Донецьку та Луганську області
в нашому телеграм-каналі Донбас24.

Фото: з відкритих джерел

\ii{insert.author.demidko_olga}
%\ii{21_02_2023.stz.news.ua.donbas24.2.mizhn_den_ridnoj_movy.txt}
