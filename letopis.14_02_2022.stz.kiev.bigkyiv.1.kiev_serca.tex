% vim: keymap=russian-jcukenwin
%%beginhead 
 
%%file 14_02_2022.stz.kiev.bigkyiv.1.kiev_serca
%%parent 14_02_2022
 
%%url https://bigkyiv.com.ua/kyyiv-obyednuye-serczya-osoblyvi-istoriyi
 
%%author_id nizhynska_tetjana
%%date 
 
%%tags istoria,kiev,ljubov
%%title Київ об’єднує серця: особливі історії
 
%%endhead 
 
\subsection{Київ об’єднує серця: особливі історії}
\label{sec:14_02_2022.stz.kiev.bigkyiv.1.kiev_serca}
 
\Purl{https://bigkyiv.com.ua/kyyiv-obyednuye-serczya-osoblyvi-istoriyi}
\ifcmt
 author_begin
   author_id nizhynska_tetjana
 author_end
\fi

Київ – місто, котре дарує кохання, об’єднує серця та розпалює пристрасті.
Навіть важко злічити, скільки інтимних і прекрасних історій кохання
розгортається на тлі Дніпровських схилів. Однак, є декілька особливих історій,
в яких Київ зіграв роль романтичного міста, котре дарує взаємність та
довгоочікувана возз’єднання. Тут можна почути «так» після трьох відмов,
зустрітися вдруге після багатьох років розлуки, закохатися у зірку та
наважитися на втечу... 

\subsubsection{Нарешті почуте «так»}

Історія кохання двох майстрів поезії Анни Ахматової та Миколи Гумільова вражає
складністю і кількістю відмов. Першу пропозицію руки та серця 17-ти річний юнак
зробив Анні ще в Петербурзі, буквально після декількох днів знайомства.
Розгублена і нажахана 14-ти річна дівчинка відповіла «ні».

Минуло декілька років, Микола Гумільов закінчив гімназію і вирушив до столиці
Франції. Саме у Парижі він і прочитав листа від Анни, котра писала про
самотність і відчуття непотрібності. Микола покинув все і стрімголов полетів до
Криму, де в цей час жила кохана.

Здавалося, що ця зустріч точно закінчиться весіллям. Однак, прогулюючись
берегом моря Анна та Микола побачили двох мертвих дельфінів. Дівчина сприйняла
знахідку як знак долі в вдруге сказала «ні».
