% vim: keymap=russian-jcukenwin
%%beginhead 
 
%%file 01_12_2020.fb.kirichenko_sergii.1.mysli_ukraincev_zhelanija
%%parent 01_12_2020
 
%%url https://www.facebook.com/permalink.php?story_fbid=3436313686487362&id=100003262938234
 
%%author Кириченко, Сергей
%%author_id kirichenko_sergii
%%author_url 
 
%%tags 
%%title МЫСЛИ УКРАИНЦЕВ 01.12.1991 г. и ИХ ЖЕЛАНИЯ 01.12.2020 г.
 
%%endhead 
 
\subsection{Мысли украинцев 01.12.1991 г. и их желания 01.12.2020 г.}
\label{sec:01_12_2020.fb.kirichenko_sergii.1.mysli_ukraincev_zhelanija}
\Purl{https://www.facebook.com/permalink.php?story_fbid=3436313686487362&id=100003262938234}
\ifcmt
	author_begin
   author_id kirichenko_sergii
	author_end
\fi

1 декабря 1991 года большинство, почти все жители Украины, на тот год еще
бывшей одной из самых развитых республик Союза, думали о том, что новая страна,
в которой они неожиданно оказались, в короткий срок станет одной из самых
успешных в мире и в Европе.

Ведь промышленный, сырьевой, научный, кадровый, образовательный и трудовой
потенциал той Украинской республики объективно давал старт для многих
достижений в самых разнообразных отраслях, о которых весь мир мог только
мечтать. Локомотивами этих мечтаний и планов были космическая,
авиастроительная, машиностроительная, комбайностроительная, судостроительная,
автомобилестроительная, полупроводниковая, ткацкая, пищевая и многие другие
отрасли народного хозяйства.

Многие тогда думали: «Вот совсем немного, и все не только наладится, а и пойдет
успешнее! Ведь то, что расходилось на весь Союз, теперь будет оставаться в
одной Украине».

Не дождавшись от Горбатого «Обновленного Союза», жители Украинской республики
проголосовали за референдум по Акту о независимости Украины и за первого
президента, который буквально на следующий день нас-ал на весь народ и сразу
принялся грабить Нэньку.

Выходит, что уже тогда что-то пошло не так… или выбор неудачный, или выбирали
не тех…

Прошло 29 лет!

Сегодня те, кому посчастливилось выжить за этот период, думает о том, когда бы
прекратился тот кошмар, который с каждым годом делает жизнь украинцев все
невыносимее и тяжелее. Теперь люди мыслят о том, чтобы пощезли те, кто дорвался
до власти и ежедневно грабит свой же народ. 

Сегодня люди мечтают о том, чтобы случилось какое-то чудо, и настали в стране
перемены, после которых не станет олигархов, их цепных псов, а к людям вернется
ПРАВДА И СПРАВЕДЛИВОСТЬ! Без которых невозможен никакой успешный
государственный строй.

1-го декабря 2020 г. люди думают о таком: «Выжить бы! Дожить бы до счастливого
часа освобождения от кошмара, навязанного лжецами, насильниками, грабителями и
ворами от власти!»

Люди хотят реального НАРОДОВЛАСТИЯ! Украинский народ устал тащить на своем
горбу паразитов!

Да будет ВОЛЯ! Долой рабство! Да здравствует СВОБОДА!

Вот, где-то так. Такая вот разница в мыслях жителей Украины в два разных,
далеких друг от друга дня: 01.12.1991 г. и 01.12.2020 г.

\subsubsection{Commentaries}

\begin{itemize}
\item \fbusr{Светлана Никончук}
Дійсто, адже ніхто недумав що народиться такий собі мутант!!!
\item \fbusr{Nikolay Potapov}
Перелом сознания жителей Украины начался 1 апреля 1992 г., когда из
официального обращения были выведены рубли и в качестве
платёжного средства остались только купоны. Я как раз в этот
период был в командировке на Украине. Было очень страшно, что
моя наличность превратится в пустые бумажки. Но практика
показала, что мои опасения были напрасными. От рублей не
отказывался никто.
\item \fbusr{Вадим Божевой Літвінов}
\ifcmt
pic https://scontent-waw1-1.xx.fbcdn.net/v/t1.0-9/129252565_3318088888289653_2694158741366389275_n.jpg?_nc_cat=107&ccb=2&_nc_sid=dbeb18&_nc_ohc=hXLHyF1LAYUAX_mavIY&_nc_ht=scontent-waw1-1.xx&oh=8d99c1426bb698fd618f383df0dbb207&oe=5FF6C797
\fi
\item \fbusr{Liubov Boichenko}
Так что граждане, надо выработать действенный план по борьбе с этими мутантами.

\item \fbusr{Ігор Демчук}
не бывать ворам рыгыанало-мертведчуковцам у власти 

\item \fbusr{Сергій Кириченко}
На этой земле есть коренное население, которому власть и должна принадлежать!
Сколько можно позволять пришлым инородцам грабить народ Украины?!

\item \fbusr{Лариса Щепеткина}
Вы испытали богатое и чудесное духовное наследие. В прошлом вы были
благословлены тем, что среди вас было много великолепных
Существ Света — мастеров, чьи учения оставили вам чудесные
руководящие принципы для проживания добродетельной жизни.

Однако пришло время объединить прошлое с будущим, и человечество испытывает
глубокую потребность в руководстве, вдохновении и поддержке
друг друга.

Мы собираемся предложить новый путь. Мы очень надеемся, что некоторые из вас
возьмут знамя и станут в будущем духовными защитниками
зарождающейся новой духовности среди Осознанного человечества.
Мы просим вас начать объединяться, чтобы учиться, поддерживать
и вдохновлять друг друга. Мы просим вас делиться друг с другом
своими знаниями, мудростью, видениями и опытом. Мы осознаем
сильное чувство одиночества среди народов мира — чувство
изоляции.

\url{https://gromada-ks.blogspot.com/2020/12/2020_4.html}

\item \fbusr{Вадим Божевой Літвінов}
\ifcmt
pic https://scontent.fiev6-1.fna.fbcdn.net/v/t1.0-0/p228x119/130244339_3325690107529531_7335983212875488238_n.jpg?_nc_cat=104&ccb=2&_nc_sid=dbeb18&_nc_ohc=rfPUp-z3LNgAX8zVZD4&_nc_ht=scontent.fiev6-1.fna&tp=6&oh=3167857878d48288d0e1a02668ba178b&oe=5FF380BE
\fi

\item \fbusr{Ігор Демчук}
Вадим Божевой Літвінов
руССкамиравцы так терпилами и дойдут до своего конца .не повторят подфиг куликовцев 

\end{itemize}
