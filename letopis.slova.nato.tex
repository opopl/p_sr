% vim: keymap=russian-jcukenwin
%%beginhead 
 
%%file slova.nato
%%parent slova
 
%%url 
 
%%author 
%%author_id 
%%author_url 
 
%%tags 
%%title 
 
%%endhead 
\chapter{НАТО}

\enquote{Слуга народа} Безуглая назвала провокационным вопрос о количестве стран в \emph{НАТО},
Елена Вьюн, strana.ua, 02.06.2021

В ходе программы \enquote{Хард с Влащенко} телеведущая Наталья Влащенко
спросила у Безуглой, сколько стран входят в \emph{НАТО}.  \enquote{Смотрите, я вообще
сразу по эфиру и по нашей беседе прошу не задавать конкретных вопросов, потому
как мы разговариваем про то, что происходит}, - ответила нардеп.  После чего
Влащенко добавила, \enquote{Ну мы же про \emph{НАТО} сейчас говорим?}.  В ответ на это
Безуглая заявила, что такие вопросы являются провокационными,
\textbf{\enquote{Слуга народа} Безуглая назвала провокационным вопрос о количестве стран в НАТО},
Елена Вьюн, strana.ua, 02.06.2021

В \emph{НАТО} входит 30 государств: Албания, США, Бельгия, Болгария, Эстония, Испания,
Голландия, Хорватия, Исландия, Италия, Канада, Греция, Литва, Люксембург,
Латвия, Норвегия, Польша, Португалия, Франция, Румыния, Германия, Cеверная
Mакедония, Словакия, Словения, Великобритания, Дания, Чехия, Турция, Венгрия,
Черногория,
\textbf{\enquote{Слуга народа} Безуглая назвала провокационным вопрос о количестве стран в НАТО},
Елена Вьюн, strana.ua, 02.06.2021

\emph{НАТО} – не защищает, Евросоюз – сыпется, но есть план Б. Что заявил Зеленский в интервью немецкой газете,
Максим Минин, strana.ua, 01.06.2021

Ех, не вміє молоде покоління політиків тримати удар. Згадується, свого часу в
розпал дискусії про те, чи варто Україні вступати у \emph{Північно-Атлантичний
Альянс}, міністр оборони Анатолій Гриценко запитав у Леоніда Кравчука: - А Ви
можете сказати, скільки держав входять в \emph{НАТО}?  Леонід Макарович миттєво
відповів питанням на питання: - А Ви можете назвати основні засади
діалектичного матеріалізму?,
\textbf{Молодым политикам стоит поучиться держать удар у Леонида Кравчука}, Константин Бондаренко, strana.ua, 02.06.2021

Несмотря на протесты Польши, в преамбуле текста единой конституции Евросоюза,
который удалось согласовать минувшей ночью в Брюсселе, не будет упоминания о
христианских корнях вошедших в него государств. Зато там будет упоминание о
том, что НАТО было и останется «фундаментом европейской безопасности»,
\textbf{Европа сносит свою идентичность}, Мак Сим, zen.yandex.ru, 04.06.2021

