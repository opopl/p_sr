% vim: keymap=russian-jcukenwin
%%beginhead 
 
%%file 15_01_2022.fb.ovcharenko_vitalij.1.japonia_interview_ukr_rossia
%%parent 15_01_2022
 
%%url https://www.facebook.com/vitalii.ovcharenko.5/posts/3182416495355053
 
%%author_id ovcharenko_vitalij
%%date 
 
%%tags dostoinstvo,interview,japonia,napadenie,rossia,ugroza,ukraina,cennosti,psihologia
%%title Інтерв'ю японцям стосовно війни України з Росією
 
%%endhead 
 
\subsection{Інтерв'ю японцям стосовно війни України з Росією}
\label{sec:15_01_2022.fb.ovcharenko_vitalij.1.japonia_interview_ukr_rossia}
 
\Purl{https://www.facebook.com/vitalii.ovcharenko.5/posts/3182416495355053}
\ifcmt
 author_begin
   author_id ovcharenko_vitalij
 author_end
\fi

Кілька днів тому давав інтерв’ю якомусь японському ТБ, стосовно війни України з
Росією.

\ii{15_01_2022.fb.ovcharenko_vitalij.1.japonia_interview_ukr_rossia.pic.1}

Перше питання від японської ведучої було таке, що максимально розкриває
культурний код японців:

- чи готові українці воювати проти Росії (увага) захищаючи честь і достоїнство
своєї країни?

Окей, вловив я, заплановані відповіді про демократію, на сьогодні відміняються
- будемо сьогодні про «захист честі та достоїнства».  @igg{fbicon.smile} .

Наступне ж питання було у подібному ж стилі: 

- чи готові ви всі воювати до остаточного кінця, як би це вам пояснити, ну всі
померти за честь своєї великої країни?

Не знаю чи очікувала вона від мене відповіді у стилі: ми готові з криками
«Банзай» всі стати «камікадзе», але прийшлось, звісно ж згадуючи про честь і
достоїнство, відповідати що вже 8 років ми воюємо і таке інше.

Але цікавий досвід спілкування з японцями і такі їхні питання, які йдуть від
світосприйняття.

Вкінці побажав гарного дня і здоров’я Імператору і його родині. 

Не знаю чи можна таке бажати правда...

\ii{15_01_2022.fb.ovcharenko_vitalij.1.japonia_interview_ukr_rossia.cmt}
