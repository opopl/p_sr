% vim: keymap=russian-jcukenwin
%%beginhead 
 
%%file moje.kremlevskie_narrativy
%%parent moje
 
%%url 
 
%%author_id 
%%date 
 
%%tags 
%%title 
 
%%endhead 

\subsection{Кремлевские нарративы vs нарративы Киева}

Кремлевские нарративы. Здесь - кремлевский нарратив, там - кремлевский
нарратив. Ущемление прав русскоязычных - Кремль. Геноцид на Донбассе -
кремлевский нарратив. Обнищание населения и гражданская война - Кремль. Высокие
тарифы - Кремль снова. Бунтуют против вакцинации - снова Кремль.  Все вокруг -
агенты Кремля. Вот этот - агент. И этот тоже - агент. И вот эта милая
светловолосая девушка, что сейчас в наушниках у себя слушает Газманова, уже
пол-часа тщетно ожидая маршрутку на Троещину - конечно, тоже, агент Кремля. А
водитель маршрутки, включивший Владимирский Централ, - или, боже упаси, -
самого Иосифа Кобзона, вообще полный зашквар, - это точно личное доверенное
лицо Путина. И даже более того, как сказала где-то Лариса Ницой, в Киеве
Украины нет, тут вообще повсюду Путин. Вы думали, Путин в Кремле? Да нет, он
давно уже в Киеве, ведь Лариса Ницой ошибаться не может, и ведь действительно
тут всюду говорят на русском. И так далее. Вон, еще, сайт Миротворец - более
чем 100 000  человек вроде уже записали в агенты Кремля, включая подло убитого
Олеся Бузину, неутомимую Олесю Медведеву, спевшую песню про Киев, талантливую
профессора-поэтессу Евгению Бильченко, обаятельную Антонину Белоглазову с ее
опросами простых Киевлян, умнейшего академика Петра Толочко, написавшего кучу
классных книг по Киеву, Любовь Титаренко, художественного руководителя
Киевского театра Браво, конечно, Анатолия Шария (Киевлянина в изгнании) и также
Фаину Савенкову, девочку писателя-фантаста из Луганска, и даже Президента
Хорватии. Просто поразительно! И столько Киевлян на Миротворце! Скоро наверное
и вообще весь Киев целиком запишут в Миротворец, как самый главный оплот
Русского Мира, Киев же как-никак Матерь Городам Русским.

\ifcmt
  tab_begin cols=2,no_fig,center
     pic https://avatars.mds.yandex.net/i?id=31ae2249ba1e06400bddba166a204810-5874640-images-thumbs&n=13
		 pic https://s00.yaplakal.com/pics/pics_original/0/7/4/4208470.jpg
  tab_end
\fi

Слушайте. Ну надоело уже. Все время несется изо всех утюгов... Кремль, Кремль,
Кремль. Кремль сделал то, Кремль сделал это. Путин мечтает о том, Путин хочет
это. Скажите вот по-честному, Вы правда влюблены в Кремль, в Путина, или что? В
каких-то жалких кремлевских рабов превратились на ментальном уровне. Ни дня не
можете прожить без Кремля и Путина, Лаврова и Шойгу. Путин - просто какой-то
бог у вас, причем бог противоречивый. С одной стороны, это кремлевский карлик,
какой-то просто сумасшедший, а с другой стороны, вы все его всегда внимательно
слушаете и боитесь, и даже песни про него постоянно поете!

\ifcmt
  tab_begin cols=3,no_fig,center
     pic http://photos.wikimapia.org/p/00/03/86/83/55_full.jpg
		 pic https://upload.wikimedia.org/wikipedia/commons/5/5e/%D0%A2%D1%80%D1%91%D1%85%D1%81%D0%B2%D1%8F%D1%82%D0%B8%D1%82%D0%B5%D0%BB%D1%8C%D1%81%D0%BA%D0%B0%D1%8F_%D1%83%D0%BB%D0%B8%D1%86%D0%B0_%D0%9A%D0%B8%D0%B5%D0%B2_2012_02.JPG
		 pic https://avatars.mds.yandex.net/i?id=3e72a0b484b3f8631780ad2d1e399331-4077532-images-thumbs&n=13 
  tab_end
\fi

А как же Святая София, Киев, Днепр, университет Шевченко и институт
философии имени Сковороды? Где ваши собственные Киевские нарративы? Где ваше
Киевское слово, твердое, уверенное и решительное, основанное на тысячелетней
мудрости и истории Стольного Града Киева? 

\ifcmt
  tab_begin cols=2,no_fig,center
     pic https://avatars.mds.yandex.net/i?id=4f5bfd24d995c3a5f31ee9005e89b8de-4322265-images-thumbs&n=13
		 pic https://avatars.mds.yandex.net/i?id=2c2f0377c8a3c42d841b62fdee445adb-5468554-images-thumbs&n=13
  tab_end
\fi

Разве ж Киев не старше Москвы, не
мудрее, не умнее Москвы?  Почему так много Москвы в информационном пространства
и так удручающе мало Киева? Везде Москва, Москва... также везде Украина,
Украина и еще раз Украина... НАТО, ЕС, лучшие в мире западные партнеры...  а
бедный Киев... Спрятался где-то в красивых альбомах и фейсбук-группах по
истории Киева... Как же так можно, елки-палки, обидно же! Такой же удивительный
и прекрасный Город, куда там Москве до него!  И вы же каждый день ездите не в
Московском метро, а в Киевском метрополитене.

\ifcmt
  tab_begin cols=2,no_fig,center
		 pic https://ic.pics.livejournal.com/tov_tob/21010671/2766179/2766179_original.jpg
		 pic https://avatars.mds.yandex.net/i?id=39c3b676331a509accd7cdfa13574214-5870057-images-thumbs&n=13
  tab_end
\fi

В красивейшем киевском метро вы
едете от Арсенальной до Гидропарка, а может быть - от Золотых Ворот до
Выдубичей. Берете Киевское такси, например, Уклон или Болт, если надо быстро
куда-то доехать, а не московское такси - как оно называется, извините, - мы не
в курсе, наверное, надо позвонить в Кремль и узнать у Пескова; кладете ребенка
в Киевскую больницу, например, Охматдет, а не в институт Склифосовского (или
какие там еще есть больницы?). Почему так мало гордости за Киев, за Столицу,
почему такое безумное преклонение перед Москвой.  Киев же уже как тридцать лет
имеет собственное политическое значение, времена, когда весь центр принятия
важных решений по Украинской ССР находился в Москве, давно уже в прошлом, так
почему же вы не пускаете в ход наш родной Киев? Слово Киева, ценности Киева,
мудрость Киева? Почему же все время говорят проукраинские политики (партии,
газеты, сми); проукраинские активисты (блоггеры) - и это считается чем-то
хорошим и правильным, хотя то, что человек себя просто назвал проукраинским
патриотом, еще ни о чем не говорит, собственно говоря, поскольку украинцы как
раз известны своим умением быстро перекрашиваться в нужные цвета... Сегодня
повесил один флажок, а завтра пришли другие - быстро повесил другой...
Вчерашние бравые комсомольские лидеры сегодня бодро машут флажками и увлеченно
поют про Украину, Бандеру, даже про Путина поют песенки... Что же вы будете петь завтра, интересно... 

\ifcmt
  tab_begin cols=3,no_fig,center
     pic https://pbs.twimg.com/media/Cd6YwEHUAAAUk79.jpg
		 pic https://s.44.ua/s/22/section/newsInText/upload/images/news/intext/000/054/450/7_6214feee4bd38.jpeg
		 pic https://sharij.net/wp-content/uploads/2019/12/photo_2019-12-19_13-13-46.jpg
  tab_end
\fi

С другой
стороны, говорят, пророссийский - и сразу как бы чувствуешь что-то нехорошее,
неправильное. Все, что носит оттенок или корень -рус-, это все плохо,
некачественно.  Украина - это априори хорошо, а вот Россия - это конечно плохо.
Хотя... если вспомнить, что Украина - это вообще говоря и есть Россия тоже,
поскольку Россия - это греческий вариант слова Русь, только с двумя буквами С,
то есть, Киевская Россия, Киевская Русь, Kiev Russia, то возникает когнитивный
диссонанс, поскольку получается, что Украина - это одновременно и хорошо, и
плохо. 

Таким образом, есть люди за Украину, есть за Россию, есть за движение на Запад,
есть на Восток, а вот тех, кто прежде всего за Киев, а значит, за самое Сердце
Украины, за самый исток Украины, а значит, за Украину втройне, - за
удивительный, многоликий, неповторимый Город на Днепре, - не слышно совершенно,
не видно совсем в этом многоголосии. И здесь мы бы хотели в общем немножко
улучшить эту ситуацию, и постараться сформулировать свою, собственную, Киевскую
позицию по животрепешущим проблемам современной Украины. Что такое кремлевские
нарративы, все уже и так знают, а какие могли бы быть наши собственные,
Киевские нарративы?

\ifcmt
  tab_begin cols=2,no_fig,center

		pic https://scontent-lhr8-1.xx.fbcdn.net/v/t39.30808-6/274300563_256216460015591_6695519426888780594_n.jpg?_nc_cat=111&ccb=1-5&_nc_sid=5cd70e&_nc_ohc=Wo7vYQe7uW4AX_ICmC6&_nc_oc=AQlWGUWb-PbCysQ1xmXHcJNwmCqULxMp_8W9Y9IE9bDPLvK4-FEWqMDwHJi4-jltaRE&_nc_ht=scontent-lhr8-1.xx&oh=00_AT_xld6qA-3uWtk5ko1TPU-TBbElZ9Lyn5UeCi5330yuSg&oe=621BEAB6

		pic https://i2.paste.pics/c8cbbd49e226f77d37eca320b6448a75.png

  tab_end
\fi

(1) \textbf{Положение русского языка в Киеве, в частности, и вообще в Украине}. Тут, к
сожалению, ситуация совершенно удручающая. Язык, на котором писал свои
сочинения Григорий Саввич Сковорода, наш великий философ-просветитель, который
в свое время учился в Киевской Академии, и чей памятник стоит на Подоле, как
раз возле Андреевского Спуска, - по факту систематически изгоняется из сферы
образования, науки, культуры.  Знаете, сейчас как раз уже 300 лет нашему
великому философу, и это поразительно, просто поразительно, что язык, на
котором он писал о самых сокровенных вещах - Человеке, Боге, Душе -
систематически оплевывается и унижается. Некоторые недалекие люди говорят, а
иные даже исступлено кричат, испуская потоки желчи по бескрайнему интернету,
что это язык московских оккупантов, говорят, что это московский (москвинский,
мокшанский) язык. Чушь собачья! Русский язык, раз уж вспомнили о Сковороде, это
- прежде всего язык Сковороды, поскольку на 90 процентов или более язык его
произведений совпадает со современным русским языком. Не верите? Ну так
почитайте сами, откройте в оригинале Сад Божественных Песен, глаза ж вам даны
не только для того, чтобы смотреть передачи ТСН, не так ли? Конечно, в его
произведениях есть примеси украинских (малороссийских) слов, и грамматические
конструкции немного старомодны, все-таки писал он более двух столетий назад; но
по факту, язык его произведений - с нашей (Киевской) точки зрения, это самый настоящий русский язык, это уж
будьте уверены. И значит, если унижают русский язык, то по факту унижают язык
Григория Саввича, а значит, это плевок в самого Сковороду.  Задумайтесь об
этом, господа патриоты. Вот, например, песнь 18-ая, красиво, правда?

\raggedcolumns
\begin{multicols}{2} % {
\setlength{\parindent}{0pt}
\obeycr
Ой ты, птичко жолтобоко,
Не клади гнезда высоко!
Клади на зеленой травке,
На молоденькой муравке.
\smallskip
От ястреб над головою
Висит, хочет ухватить,
Вашею живет он кровью,
От, от! кохти он острит!
\smallskip
Стоит явор над горою,
Все кивает головою.
Буйны ветры повевают,
Руки явору ломают.
\smallskip
А вербочки шумят низко,
Волокут мене до сна.
Тут течет поточок близко;
Видно воду аж до дна.
\smallskip
На что ж мне замышляти,
Что в селе родила мати?
Нехай у тех мозок рвется,
Кто высоко в гору дмется,
\smallskip
А я буду себе тихо
Коротати милый век.
\smallskip
Так минет мене все лихо,
Щастлив буду человек.
\restorecr
\end{multicols} % }

\ifcmt
tab_begin cols=2
  @caption Памятник Григорию Сковороде в городе Киеве

  ig https://avatars.mds.yandex.net/get-altay/2776464/2a00000170f4a99f4b8134bb851eb03ec81d/XXL
	ig https://ic.pics.livejournal.com/s_a_plotnikov/85688553/71932/71932_original.jpg

tab_end
\fi

Далее, что касается русского языка, является ли он своим или чужим, мы скажем,
что русский, - это один из двух основных Киевских языков, да, языков Города
Киева, наравне с украинским, на котором, кстати, написан гимн Киева, Як тебе не
любити, Києве Мій. Ничуть не желая унизить или умалить украинский язык (мы про
него напишем в другом месте), мы тем не менее скажем, что русский - это Язык,
великий язык на котором говорят миллионы Киевлян каждый день. В разговорах в
семье, на улицах; в школах и на свиданиях; во время телефонных разговоров и
также во время разговоров по скайпу, во время переписки по вайберу или по
телеграму. На русском языке в Киеве говорят уборщицы и депутаты Верховной Рады;
говорят сотрудники СБУ и врачи скорой; говорят везде, и тут, и там. В
маршрутках говорят и в метро тоже; говорят в институтах Академии Наук, и в
Милицейской Академии тоже; Говорят зимой, и летом; осенью, и весной. На русском
языке матерится Петр Порошенко, когда ему вручают повестку, и на русском же
языке Антон Геращенко призывает россиян остановить Путина. Да, представте себе,
дамы Фарион и Ницой, в Киеве говорят на прекрасном, удивительном русском языке,
и воистину поразительно, что нам нужно вообще об этом писать и напоминать,
поскольку обилие и разнообразие разговорного и письменного русского языка в
Киеве - это очевидный факт.  Русский язык абсолютно Киеву органичен и Киеву
совершенно свой. На русском говорят и пишут Антон Геращенко и Арсен Аваков,
Петр Толочко и Олеся Медведева, Алексей Арестович и Борислав Береза, Олег
Волошин и даже Юрий Бутусов. На русском языке был написан призыв Мустафы Найема, с которого начался Майдан-2... 
На русском языке говорили воины Красной Армии,
которые ценой огромных жертв освободили Киев, и на русском языке писал свои
книги Михаил Булгаков. Это наш - Киевский язык, корни которого идут прямиком в
Киевскую Русь, во времена, когда на стенах Софиевского Собора простые Киевляне
оставляли свои надписи. Вы хотите сказать, что это все староукраинский язык, а
русский язык - это миф путинской пропаганды, что это все навязанное, не свое,
родное. Что была именно Украина-Русь, и именно так, а не иначе.  Ну а мы тогда
ответим, что Украина-Русь - это какой-то непонятный конструкт, какой-то вообще
выдуманный мираж. Был Киев, Стольный Град Киев, который столько раз грабили и
уничтожали, а он все возрождался, Киев, объективный, реальный, вот он - здесь -
прямо под ногами! Град Киев, куда язык человеческий вел, стоило только
спросить...  а как нам добраться до Киева, не подскажете? И тут же путнику
охотно указывали, - сначала пройди вон тот лес, потом поверни направо, потом -
налево, потом будет еще один лес, - только гляди, такой тебе совет, - не попади
ты волкам на обед, - ну аж потом ты наконец-то увидишь Киев...  И он здесь
стоит уже тысячу лет, наш Киев, в котором были киевляне и князья, бояре и
смерды, тиуны и дружинники. И все это было связано со Псковом и Новгородом,
Суздалем и Ярославлем, и другими городами Северо-Восточной Руси.  И все это
вместе именовалось Русь. Просто Русь. А потом, - Русь разделилась, раскололась
на части, и каждая часть пошла своей дорогой, - одна путем вольницы и
освободительных войн, а другая - путем жесткой, централизованной
государственности и имперскости, - потом соединившись вместе, а потом снова
разойдясь... А пресловутая Украина-Русь?  Вот именно так, Украина-Русь, или же
Русь-Украина, через дефис, и никак иначе, а то Грушевский обидится?  Да чепуха
это все. Был Киев - а потом все остальное. И что сейчас важно, по-настоящему
важно, это не то, насколько именно сегодняшний Киевский русский язык подобен
языку графити Софии Киевской, а то, что Киев сейчас является невероятно
замусоренным, загрязненным, разбитым городом, чье нынешнее запущенное состояние
(можно еще вспомнить про Чернобыль рядом) абсолютно не соответствует его
истинной значимости.

\ifcmt
  tab_begin cols=3,no_fig,center

     pic https://gordonua.com/img/article/15213/29_big.jpg?v1601729183
		 pic https://avatars.mds.yandex.net/i?id=515efdebec88624056a69b00ecc6e3b0-5886792-images-thumbs&n=13
		 pic https://avatars.mds.yandex.net/i?id=a951e988dcb2b7168b7301d363197d0a-5747194-images-thumbs&n=13

  tab_end
\fi

Далее, касательно притеснений русского языка и в Киеве, и по Украине, это
действительно правда, и Кремль здесь особо ни причем, поскольку же именно ты,
государство Украина, управляющее Городом Киевом последние 30 лет, хорошо
постаралась в этом, закрывая русские школы, вводя глупые украинизаторские
законы, и вообще систематически унижая русский язык - кровный язык Киева! - как
второстепенный, иностранный язык.  Знаешь, Украина, есть такая пословица:
насильно мил не будешь. Это касательно развития украинского языка. К сожалению,
тоталитарная репрессивная политика в отношении статуса русского языка и русской
культуры не привела к настоящему, духовному подъему украинской культуры и
языка, как ты хотела, не привела. Совсем наоборот.

\ifcmt
  tab_begin cols=3,no_fig,center
     pic https://i2.paste.pics/c4004e2585224bc31f681c37e8d849bc.png
		 pic https://i2.paste.pics/5de7ab7812afc20b9788023f41992509.png
		 pic https://i2.paste.pics/b7b038ded8df6885ce858fc8fc4054af.png
  tab_end
\fi

Alas! Украинский язык - певучий, удивительный язык, - проникающий в самую глубь
сердца, озаряющий самые потаенные уголки души, - стоит только открыть томик
Леси Украинки или же Олександра Олеся, или послушать какую-нибудь песню
Русланы, - ну например, старую добрую песню Свитанок, или же, Знаю Я -
превратился в государственный фетиш, в мертвый символ, в канцелярский язык, в
язык, на котором заставляют говорить насильно, в язык, который многие люди
вообще абсолютно искренне, от души ненавидят, поскольку он ассоциируется с
ненавистью, грабежами, убийствами и насилием, творимым незаконными военными
действиями ВСУ, а также с людоедскими заявлениями Фарион, Дроздова, Ницой, и их
многочисленных единомышленников, которые готовы резать москалей и малороссов,
которые повсюду видят предателей, живя во мраке мысленных чудовищ, порожденных
ими самими... Знаешь, Украина, апостол Павл сказал когда-то, что, \enquote{если
буду говорить красиво, но не буду иметь любви, то буду подобен меди звенящей},
- то есть, даже если ты говоришь самым изысканным украинским языком, но при
этом ты зол и глуп, и душа твоя пуста и не имеет сострадания и сочувствия, а
только бесконечное - росия-ворог, путин..., москалей-на-ножи, то какой толк
тогда в твоем украинском? Как ты думаешь, Украина, если бы например Леся Украинка перенеслась
машиной времени в наше время, она бы тоже орала кричалки, которые орут фанаты
на матче Украина-Россия?

\ifcmt
  tab_begin cols=2,no_fig,center

     pic https://avatars.mds.yandex.net/i?id=2a00000179f5e312985d9263bd58639efd1b-4613271-images-thumbs&n=13
		 pic https://ye.ua/images/news/poster_photo_title-1496732268.jpeg

  tab_end
\fi
