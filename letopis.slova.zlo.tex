% vim: keymap=russian-jcukenwin
%%beginhead 
 
%%file slova.zlo
%%parent slova
 
%%url 
 
%%author 
%%author_id 
%%author_url 
 
%%tags 
%%title 
 
%%endhead 
\chapter{Зло}
\label{sec:slova.zlo}

%%%cit
%%%cit_pic
%%%cit_text
Нам як ніколи потрібні тривкі та справедливі правила гри замість управління
країною в ручному режимі. Нам потрібні відкритість та прозорість влади замість
плівок майора Мельниченка чи генерала Кондратюка. Нам потрібно нарешті вчитись
\enquote{жить не по лжи}. Навіть якщо цей вираз належить Солженіцину. Бо етика
важливіша за айдентику. А етичність важливіша за ідентичність. Бо як казав наш
великий земляк Борис Антоненко-Давидович, \enquote{\emph{Зло}, навіть під українським
національним прапором, виголошене українською мовою, все ж лишається \emph{Злом}...}
%%%cit_comment
%%%cit_title
\citTitle{У політику треба повернути мораль. Інакше ми приречені залишатися сталіністами}, 
Геннадій Друзенко, gazeta.ua, 14.06.2021
%%%endcit

%%%cit
%%%cit_head
%%%cit_pic
%%%cit_text
\enquote{Сегодня в Украине власть заведомо потакает возрождению нацизма, и последние
марши пособников и сторонников дивизии СС Галичина - прямое этому
подтверждение. Своим движением мы обязаны обратить внимание мировой
общественности на то \emph{зло}, которое сегодня под патронатом власти даёт первые
ростки в нашей стране. Наш долг, не дать возможность коричневой чуме вновь
распространится по миру}, - заявил Кива.  Нардеп от \enquote{Слуги народа} Максим
Бужанский прокомментировал слова Зеленского  и высказал мнение, что не
упоминание Великой отечественной войны - это отказ от своих родных, которые в
ней воевали
%%%cit_comment
%%%cit_title
\citTitle{22 июня в Киеве - что говорили украинские политики}, Екатерина Терехова, strana.ua, 22.06.2021
%%%endcit

