% vim: keymap=russian-jcukenwin
%%beginhead 
 
%%file 16_03_2023.fb.kipcharskij_viktor.mariupol.1.r_k_tomu__den_21__16.cmt.2
%%parent 16_03_2023.fb.kipcharskij_viktor.mariupol.1.r_k_tomu__den_21__16.cmt
 
%%url 
 
%%author_id 
%%date 
 
%%tags 
%%title 
 
%%endhead 

\begin{itemize} % {
\iusr{Sergey Drovorub}

\href{https://www.delfi.lt/ru/news/live/oni-otvetyat-i-tot-pilot-kotoryj-sbrasyval-bomby-na-mirnye-doma-te-kto-strelyali-v-nas-cvidetelstvo-nadezhdy-kotoraya-vyzhila-v-dramteatre-mariupolya.d?id=89935973}{%
"Они ответят: и тот пилот, который сбрасывал бомбы на мирные дома, те, кто
стреляли в нас": cвидетельство Надежды, которая выжила в Драмтеатре Мариуполя, %
Анастасия Федченко, delfi.lt, 11.04.2022%
}

\ifcmt
  ig https://i.paste.pics/144b0f464f5a1f8dd6a4e4ac65be76d9.png
  @wrap center
  @width 0.4
\fi

\iusr{Sergey Drovorub}

\href{https://news.liga.net/politics/news/v-mariupole-samolet-okkupantov-sbrosil-bombu-na-zdanie-dramteatra-gde-pryatalis-sotni-lyud}{%
В Мариуполе самолет оккупантов сбросил бомбу на драмтеатр, где прятались сотни людей, news.liga.net, 16.03.2022%
}

\ifcmt
  tab_begin cols=2,no_fig,center,separate

     pic https://i.paste.pics/970111145676d0aa39f42b2a25c65c56.png
		 pic https://i.paste.pics/b57d98bc614f907af21d83a6fc1de4b1.png

  tab_end
\fi

\iusr{Віктор Кіпчарський}
\textbf{Sergey Drovorub} Дякую за інформацію.
От тільки точного часу там нема: мені пишуть про час з 8:30 по 10:05.
Може про наш виїзд о 9:45 я помилився, коди писав щоденник - натиснув сусідню цифру.

\iusr{Sergey Drovorub}
\textbf{Віктор Кіпчарський} у вікіпедії:

Около 10 часов утра 16 марта 2022 года ВКС России нанесли бомбовые авиаудары по
Драматическому театру Мариуполя и плавательному бассейну «Нептун»[19][20][21].
Amnesty International сообщила, что рядом с театром находились сотни
гражданских[11].

Согласно расследованию Amnesty International, бомбы пробили крышу в восточной
части театра и сдетонировали в зрительском зале, скорее всего, на уровне
легкого настила сцены. Предполагается, что бомба не проникла на самый нижний
уровень здания театра из-за взрывателя, который был установлен на взрыв при
контакте или с небольшой задержкой. В результате взрыва были разрушены
ближайшие внутренние стены, образующие крылья зрительского зала, и внешние
несущие стены. Обвал кровли произошёл в основном в северо-восточной части
здания; несущая конструкция крыши обвалилась и упала в зрительский зал, накрыв
место взрыва[11].

\iusr{Sergey Drovorub}
\textbf{Віктор Кіпчарський} сусід по телефону мені казав, що о 9, 9:30 проходив повз, то театр ще не бомбили. Але він не дуже достовірне джерело.

\iusr{Віктор Кіпчарський}
\textbf{Sergey Drovorub} Своїм прикладом я майже підтвердив старе міліцейське спостереження: бреше, як очевидець.
Майже - тому що я не був очевидцем а записував з теленовин.
Дякую за допомогу: в своєму пості я вже зробив приписку.

\iusr{Sergey Drovorub}
\textbf{Віктор Кіпчарський} будь ласка.
Прикольна приказка 🙂 не чув такої.

\iusr{Sergey Drovorub}
\textbf{Віктор Кіпчарський} 

в очевидців, колись читав, можуть відбуватися дивні метаморфози пам'яті. Точно
не передам, але щось типу такого: окрім різного бачення ситуації в реальному
часі, ще і пам'ять може домальовувати щось таке, чого насправді не відбувалося,
або навпаки витісняти якісь спогади і люди абсолютно відверто і щиро можуть
казати різне.

Ще згадався Бендер, що намагався врятувати Паніковського від розправи і
закликав людей записуватися у свідки. 🙂

\iusr{Віктор Кіпчарський}
\textbf{Sergey Drovorub} А таке:
Кричить, як потерпілий!

\iusr{Sergey Drovorub}
\textbf{Віктор Кіпчарський} таке чув.

\iusr{Sergey Drovorub}
\textbf{Віктор Кіпчарський} в мене напарник був. У човні. То він казав на суржику: \enquote{оре як потерпівший} 🙂

\iusr{Віктор Кіпчарський}
\textbf{Sergey Drovorub} 

На поліцейській конференції до зали засідань забіг чоловік якого переслідував
інший з ножем, за яким біг третій з пістолетом. Вони пробігли через зал і
вибігли через інші двері.

Опитані свідки (поліцейські!) вказували різні прикмети: колір волосся, одяг і так далі.

\iusr{Sergey Drovorub}
\textbf{Віктор Кіпчарський} так. Щось типу того. Ще пригадався ефект Мандели. Так званий.

\end{itemize} % }
