% vim: keymap=russian-jcukenwin
%%beginhead 
 
%%file 07_06_2021.fb.bilchenko_evgenia.1.recenzia_khat.cmt
%%parent 07_06_2021.fb.bilchenko_evgenia.1.recenzia_khat
 
%%url 
 
%%author 
%%author_id 
%%author_url 
 
%%tags 
%%title 
 
%%endhead 
\subsubsection{Комментарии}
\label{sec:07_06_2021.fb.bilchenko_evgenia.1.recenzia_khat.cmt}

\begin{itemize}
\iusr{Катарина Синчилло}

Спасибо за рецензию! Каждый зритель и каждый критик имеет право на своё мнение.
Так же, как и каждый художник имеет право на своё видение. Этот спектакль был о
тебе. Новая версия создавалась под влиянием травли одного из моих любимых
киевских поэтов. Я высказалась об этом со сцены. И Слава Богу, что зал меня
услышал, понял, почувствовал, и как режиссёра, и как Йоганну! В одном мы точно
очень похожи - я до последнего вздоха буду защищать свои убеждения и своё право
на свободу. Ни в своём доме, ни в своём театре я не хочу грязных отношений,
предательства, закулисных игр - нельзя построить храм из навоза, поэтому ищу
актёров- единомышленников. И это тоже моё право. А что касается рецензии, то я
не хочу, чтобы она была последней. Я приглашаю на спектакли не за тем, чтобы
меня хвалили.

\begin{itemize}

\iusr{Евгения Бильченко}
\textbf{Катарина Синчилло} Катя, никто не услышал последнего источника твоей
новой версии. Люди слушают, но не слышат, пока не говоришь в лоб. Я очень долго
думала, что надеть, какой цветок взять, как скрыть лицо (не потому что боюсь, а
потому что грублю в ответ на конформизм), и вышел жёлтый цвет. У вас его на
сцене было много. Это меня поразило. Я все время старалась быть в кепке и в
маске, но в кепке сидеть в театре нельзя, а маску я уронила на подмостки. Ее
надо выкинуть. Я ее уже раз надела.

\iusr{Катарина Синчилло}
\textbf{Евгения Бильченко} На сцене жёлтого цвета не было вообще! Жёлтый и
золотой - это не одно и тоже! А тему буллинга все прочитали ещё на Премьере
новой версии. И перед каждым спектаклем я напоминаю актёрам, о чём он. И о чём
КХАТ.

\iusr{Евгения Бильченко}
\textbf{Катарина Синчилло} Золотой купол и жёлтый одуванчик - все Божье творение.

\iusr{Катарина Синчилло}
\textbf{Евгения Бильченко} Все мы Божьи творенья, и очень часто, к сожалению, не ведаем, что творим...

\iusr{Евгения Бильченко}
Лучше не ведать корни добра и зла, чем ведать всё и не сделать ничего для сопротивления злу.

\iusr{Катарина Синчилло}

\textbf{Евгения Бильченко} А ещё есть вариант - ведать всё, стать на сторону
зла, начать его защищать и пропагандировать. Всем нам воздастся за всё
сотворённое. Я учусь уважать свободу выбора каждого, ведь каждому отвечать за
своё.

\iusr{Людмила Кузема}

Браво, Катарина!

\iusr{Евгения Бильченко}

\textbf{Катарина Синчилло} есть ещё свобода выбора у Чикатило: он имеет право
зарезать твоего родственника, демократия же. PS. Чужие \enquote{браво}, как и свои, на
солдата Бильченко не влияют, ибо в славе уже не нуждаюсь: я поэт народа,
пролетарий, а не примадонна. Но я не вру и не льщу, поощряя чужие грехи.

\iusr{Аркадий Веселов}

\textbf{Катарина Синчилло} мои 5 копеек: я сразу понял сцену травли, всё вполне читается
) Костюмы-наряды золотого-роскошного стало больше - зрелищно и красиво! Ещё
очень понравились восточные танцы - красивые девочки и шикарная музыка - это
браво!! Осталось яркое впечатление от спектакля! И вообще я очень рад был вас
видеть ))

\iusr{Евгения Бильченко}

А я бы предпочла молодых мускулистых парней на сцене.

\iusr{Евгения Бильченко}
И много диалогов с \textbf{Виктор Кошель} о психоанализе и философии. Вот.

\iusr{Евгения Бильченко}

Давайте сделаем гендерный переворот). Мужские полуголые тела и спящий Аркадий.

\iusr{Катарина Синчилло}

\textbf{Аркадий Веселов} Аркадий, спасибо, мы рады, если этим нашим спектаклем
удалось подзарядить тебя позитивом и энергией, помочь восстановить духовные
силы, которые тебе нужны сейчас, как никогда. Береги Женю!

\end{itemize}

\iusr{Елена Приходько}

Я Вас не знаю лично, но Вами восхищаюсь. Вы не такая как все, очень умная,
начитанная. Приятно читать ваши комментарии и стихи

\end{itemize}
