% vim: keymap=russian-jcukenwin
%%beginhead 
 
%%file 10_11_2022.fb.chitaja_tamara.ua.1.mykluho_maklaj
%%parent 10_11_2022
 
%%url https://www.facebook.com/tamarachitaia/posts/pfbid0Yy1B8NS6SHYCS6JJvQ6SSCv9KfuA2BzS9qwtq62vAuKT1TR84nob2PGfdxE4Jb9jl
 
%%author_id chitaja_tamara.ua
%%date 
 
%%tags 
%%title Ім'я українця носить узбережжя острова Нова Гвінея - хто він?
 
%%endhead 
 
\subsection{Ім'я українця носить узбережжя острова Нова Гвінея - хто він?}
\label{sec:10_11_2022.fb.chitaja_tamara.ua.1.mykluho_maklaj}
 
\Purl{https://www.facebook.com/tamarachitaia/posts/pfbid0Yy1B8NS6SHYCS6JJvQ6SSCv9KfuA2BzS9qwtq62vAuKT1TR84nob2PGfdxE4Jb9jl}
\ifcmt
 author_begin
   author_id chitaja_tamara.ua
 author_end
\fi

@igg{fbicon.fleur.de.lis} Ім'я українця носить узбережжя острова Нова Гвінея - хто він? @igg{fbicon.fleur.de.lis}

МИКОЛА МИКЛУХО-МАКЛАЙ - Так, панове! Це наш з вами співвітчизник! Великий
українець, чиїм ім'ям ми повинні пишатися! Його біографія та життя є
надзвичайно надихаючими для мене як дослідниці, етнографа, мандрівниці, тож
сьогодні розповім про нього детально…

У суботу, 14 квітня 1888 р. о 8-й годині 30 хвилин у лікарні Вілліє при
Військово-медичній академії, на лікарняному ліжку помер 41-річний мандрівник у
вишиванці Микола Миклухо-Маклай, вчений-антрополог, знавець тринадцяти мов, чиє
ім`я носить 300 км узбережжя о. Нова Гвінея - Берег Миклухо-Маклая.

Народився нащадок козаків Микола Миклухо-Маклай 17 липня 1846 р в Малині на
Житомирщині. Це не так далеко від Києва. 

\ii{10_11_2022.fb.chitaja_tamara.ua.1.mykluho_maklaj.pic.1}

За часів Хмельницького його пращур шотландський лицар Майкл МакЛей одружився з
дочкою курінного отамана Охріма Макухи (Миклухи). Того Охріма, який разом із
синами Омельком, Назаром і Хомою воював проти поляків; у якого Назар, панночку
покохав і на бік ворога став. Вночі брати пробралися до фортеці, щоб викрасти
Назара та покарати козацьким судом, але наткнулися на польську варту. Хома
(молодший) наказав братові доставити зрадника батькові, а сам бився до загину.
Охрім таки покарав Назара. Цей епізод став основою повісті «Тарас Бульба», яку
Микола Миклухо-Маклай завжди возив із собою. Так друзі, з цим вікопомним твором
великий український письменник Микола Гоголь увійшов в класику російської
літератури… І цікаво, що рідний дядько майбутнього вченого, Григорій Ілліч,
який у 1824-1828 роках навчався у Ніжинській гімназії вищих наук і товаришував
там з юним Миколою Гоголем, не раз розповідав йому цю сімейну легенду й до
кінця життя був абсолютно переконаний, що саме вона була покладена геніальним
письменником у основу одного з найдраматичніших епізодів \enquote{Тараса Бульби}! А
батько Миклухи-Маклая завжди тримав у себе на столі портрет Тараса Бульби
(свого символічного предка)...

\ii{10_11_2022.fb.chitaja_tamara.ua.1.mykluho_maklaj.pic.2}

Прапрадіду Миколи Миколайовича, Степану, (який в юні роки мав прізвисько
Махлай), за бойові заслуги було дароване потомствене дворянство. Тоді Степан
Макуха і вирішив \enquote{облагородити} своє прізвище й замінити неблагозвучне \enquote{Макуха}
на \enquote{Миклуха}, а \enquote{Махлая} – на не дуже зрозумілого \enquote{Маклая}.

Батько Миколи Миклухо-Маклая – випускник ніжинського ліцею, начальник
петербурзької залізниці, був добре знайомий із творами Т.Шевченка. Він вислав
150 карбованців (половину річної зарплати – це дуже великі кошти!) Шевченкові у
вигнання, за що був звільнений з роботи і мав йти під суд, але 40-річний
українець помер від туберкульозу, лишивши удову та п'ятеро дітей, з
польсько-німецької родини лікаря Беккера, її родичами були Гете і Міцкевич.

\ii{10_11_2022.fb.chitaja_tamara.ua.1.mykluho_maklaj.pic.3}

Сам Микола Миклухо-Маклай 1884 року дав таку відповідь кореспондентові газети
\enquote{Сідней морнінг гералд}: \enquote{Моя особа – то живий приклад того, як благочинно
з'єдналися три одвічно ворожі сили. Гаряча кров запорожців мирно злилася з
кров'ю їхніх, здавалося б, непримиренних гордих ворогів, ляхів, розбавленою
кров'ю холодних германців}. І справді, по материнській лінії (мати – Катерина
Семенівна, уроджена Беккер) в його жилах текла й німецька, й польська кров…

\ii{10_11_2022.fb.chitaja_tamara.ua.1.mykluho_maklaj.pic.4}

Хворобливий трирічний Миколка, вражений смертю татка, почав заїкатися. Втім,
крихке здоров'я не завадило Миколі отримати освіту. Він вступити до вишу, але у
1864 р. за участь у виступах студентів був виключений із Петербурзького
університету без права вступати до іншого вищого навчального закладу Росії. І
ніколи б не одержав він дозволу на виїзд за кордон, якби не роман українця з
фрейліною царя, на яку імператор поклав око. Довелося Миколі здобувати вищу
освіту за кордоном: Гайдельберг, Лейпціг, Єн, а потім був асистентом відомого
природознавця Ернста Геккеля.

\ii{10_11_2022.fb.chitaja_tamara.ua.1.mykluho_maklaj.pic.5}

У 1866-67 рр. відвідав Канарські острови та Марокко, три весняні місяці 1869 р.
провів на узбережжі Червоного моря. Перші спостереження вченого стосувалися
зоології, потім - географії, антропології й етнографії. Від 1870 р. почалися
одіссеї великого мандрівника: Нова Гвінея, острови Малонезії та Мікронезії,
півострів Малакка, Австралія. 1880 р. українець у Сіднеї заснував австралійську
біологічну наукову станцію. Він розробив проект створення на Новій Гвінеї
незалежної держави - Папуаського Союзу. Пізніше безуспішно домагався від
царського уряду дозволу організувати на Новій Гвінеї «вільну російську
колонію». Він від імені десятка тисяч людей звернувся до уряду. І сталося диво!
Британія відмінила колонізацію Нової Гвінеї. Через десять років повернувся до
Європи, читав лекції в Лондоні, Берліні, Парижі. Приїхав на батьківщину,
досліджував фауну Чорного моря та південного узбережжя Криму.

\ii{10_11_2022.fb.chitaja_tamara.ua.1.mykluho_maklaj.pic.6}

Миклухо-Маклай мав власну думку й умів її відстоювати: виступав на захист
рідної мови. З цього приводу звертався до Бісмарка та російського царя і у
листопаді 1882 р. Миклухо-Маклай зустрічався в Гатчині з Олександром III.

У 1883 р. він отримав дозвіл царя на одруження. Потім повернувся до Сіднею,
одружився з Маргарет-Еммою Робертсон - дочкою прем'єр-міністра Нового
Південного Велсу. Три роки прожив дослідник серед папуасів, вивчив їхній побут
і діалекти. 1885 р. виступив на захист проти німецької анексії північно-східної
Нової Гвінеї. Сорокарічним повернув із дружиною та двома синами додому. Поїхав
до столиці, щоб приготувати до друку свої праці. Двічі він приїздив до Малина в
маєток матері, де вивчав побут і традиції поліщуків (українське та білоруське
населення Полісся), цікавився походженням древлян, їх історією. У цей час
петербурзьку квартиру Миклухо-Маклая відвідував Яворницький і немало експонатів
випросив у нього. У Дмитра Яворницького був талант «циганити», сяде й заявляє:
«Як цей панцир черепахи прикрасить наш музей! Не піду звідси – поки не дасте!».
І таки сидів.

Окрім цікавих артефактів, що за життя назбирав Микола Миклухо-Маклай, цікавими
були і його думки та висловлювання. Зокрема, мандрівник дав жорсткий і вельми
об'єктивний аналіз розвитку \enquote{російської ідеї} та державності наприкінці ХІХ
сторіччя.

Ось він: \enquote{На Русі, коли ми простежимо історію Російської держави від Івана
Грозного до наших днів, відкинувши хіба що добу Петра I, не допускалося під
страхом смерті чи тюремного ув'язнення не тільки мати іншу думку, а навіть
сумніватися в чомусь такому, що було усталеним і прийнятим у державі. Дика
азійська орда з її лютою жорстокістю, зневагою цінностей духовних і поділом
суспільства на рабів і вождів принесла й укоренила на віки в Російській державі
становище, коли право думати діставали тільки ті з нижчої суспільної верстви,
хто, думаючи, у своїх міркуваннях угадував бажання вождя}.

Сьогодні 12 онуків, правнуків, праправнуків Миколи Миклухо-Маклая живуть в
Австралії та вважають себе українцями, очолюють Товариство українців Австралії.
Старійшина цього роду – диктор та ведучий австралійського державного
телебачення Роберт Миклухо-Маклай. Ще за радянських часів (1980 і 1988 рр.) був
у Києві, він таємно вночі на таксі за долари промчав до Малина, щоб
сфотографувати будинок, де народився його дід-українець, легенда роду
Миклухо-Маклаїв.

@igg{fbicon.writing.hand} За матеріалами Ганни Черкаської та книги \enquote{МИКЛУХО-МАКЛАЙ} В.Колесникова

\ii{10_11_2022.fb.chitaja_tamara.ua.1.mykluho_maklaj.orig}
\ii{10_11_2022.fb.chitaja_tamara.ua.1.mykluho_maklaj.cmtx}
