% vim: keymap=russian-jcukenwin
%%beginhead 
 
%%file 12_01_2021.fb.buzhanskii.1.korolev_uletel
%%parent 12_01_2021
 
%%url https://www.facebook.com/permalink.php?story_fbid=1849827221848415&id=100004634650264
 
%%author 
%%author_id 
%%author_url 
 
%%tags 
%%title 
 
%%endhead 
\subsection{Королёв улетел}
\Purl{https://www.facebook.com/permalink.php?story_fbid=1849827221848415&id=100004634650264}
\ifcmt
  author_begin
   author_id buzhanskii_max
  author_end
\fi

\obeycr
Долго стоял, решаясь, мялся.
Крутил в руках папиросу, прикусил уже было зубами, сплющил мундштук, снова покрутил пальцами.
Кинул взгляд на ракету, потом снова на папиросу, хмыкнул.
Есть что то общее.
Оглянулся.
Всё было чужим.
Он смотрел по сторонам, и не узнавал тут ничего.
Какие то люди рассказывали о нём, и он не узнавал себя.
Пытки?
Да что они знают о пытках, он видел тех, кого пытали.
Работал с ними, смеялись, вот так вот, как он, прикусывая папиросы железными зубами, свои выбили.
Они не думали о пытках, думали о ракетах.
О космосе.
Мечтали о нем, жрали баланду и думали о звёздах, стояли, задрав голову к небу и смотрели на них.
Война.
Закидали трупами?
Королёв горько усмехнулся.
Да?
Так просто взяли и закидали, и всё?
Все те, кто по двадцать часов стоял у станка, это они закидали?
Или вон те, которые создавали оружие, и они закидали?
Королев смахнул пепел со щеки, вспомнил одного толстого конструктора, артиллериста, вечно сердитого, совсем седого.
Кричал на мальчишек, лупивших из его орудий на полигоне по мишеням.
Вспомнил, как тот плакал потом на параде, в 45м.
Стоял и рыдал под проливным дождём, все счастливы, а он рыдал.
Седой толстый старик с мокрыми волосами.
Королев только потом, через 10 лет узнал, что ему было тогда 35, жена и сын сгорели в вагоне под Харьковом, попали под бомбы.
Закидали трупами, да?
Королев сглотнул застывший в горле комок, закурил вторую.
Разруха.
Тут всё лежало в руинах, целые города битого в крошку кирпича, ходили, высоко поднимая ноги, словно цапли.
Шесть дней в неделю они чертили эту ракету, растили её, словно ребёнка, рылись в трофейных немецких чертежах.
А в воскресенье шли, и таскали эти кирпичи, разбирали завалы.
Где то сломанные очки, под камнями, где то кукла, с оторванной рукой...

Не хватало всего, всё заново, а они чертили эту ракету, закрывали глаза и видели, как она взлетает.

А оказалось, построили не такие дома.

Как не такие?

Вся страна в руинах, люди вернулись из эвакуации вникуда, он вспомнил, как
вырастали эти серые пятиэтажки, как дети орали во дворе, играя в войну.

Пусть играют, уже можно только играть...

Ааа... вот эти тоже победили нацизм, да?

Тоже воевали с нацизмом, и вон тот, вон в той форме?

Королев затягивается, прижимает воротник к горлу, морозное зимнее солнце слепит
глаза.

Он тут никому не нужен.

Где то висит памятная доска, где то даже памятник есть, он его видел, приходил
посмотреть, постоял, не узнал.

Он им не нужен, ломает всё, всю картину.

Ну разве ж он был рабом?

Это он то боялся?

Да ему закурить некогда было, Королев смеётся, хлопает рукой по карману,
вытаскивает спичечный коробок, нет времени на страх.

Постоял ещё пару минут, просто закрыл глаза, набрал полную грудь воздуха, замер.

Открыл.

Постоял с минуту, глядя, как навсегда молодой Гагарин шагает к ракете по залитому солнцем бетону, побежал за ним, неуклюже путаясь в длинном пальто и придерживая рукой шляпу.
Догнал, закурил, минуту помолчал.
Лечу с тобой.
Мы ведь никогда больше не вернёмся, -тихо сказал Гагарин.
Я знаю,- Королев щурится от дыма, - махни им рукой, поехали.
\restorecr
