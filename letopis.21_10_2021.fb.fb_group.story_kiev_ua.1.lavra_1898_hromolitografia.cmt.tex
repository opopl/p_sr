% vim: keymap=russian-jcukenwin
%%beginhead 
 
%%file 21_10_2021.fb.fb_group.story_kiev_ua.1.lavra_1898_hromolitografia.cmt
%%parent 21_10_2021.fb.fb_group.story_kiev_ua.1.lavra_1898_hromolitografia
 
%%url 
 
%%author_id 
%%date 
 
%%tags 
%%title 
 
%%endhead 
\zzSecCmt

\begin{itemize} % {
\iusr{Виктор Киркевич}

В 1993 году в Вашингтоне познакомился с сыном Е.И. Фесенко - Андреем. Он мне
подарил свои книги. Его супруга Татьяна Фесенко написала очень интересные
воспоминания, где очень много о довоенном и военном Киеве. Они у меня есть и я
их использую в своих книгах.

\iusr{Оксана Дубинина}
\textbf{Виктор Киркевич} Вы у нас энциклопедия по Киеву и не только@igg{fbicon.heart.red}

\iusr{Петр Кузьменко}
Очень интересное изображение. Благодарю.

\iusr{Виктор Киркевич}
Повесть кривых лет

By Фесенко, Татьяна Павловна [1915 - 1995]

Воспоминания непосредственной участницы описанных событий, документальная
повесть о жизни семьи автора в Киеве до Второй мировой войны и в страшные
военные годы. Воспоминания тем более ценные, что они написаны профессиональным
литератором, филологом. Фесенко Татьяна Павловна, родилась в 1915 г. в Киеве.
Весной 1941 года окончила аспирантуру Киевского университета по специальности
английская филология. Во время войны и немецкой оккупации была отправлена
немцами в лагерь в Верхней Силезии. Прошла весь путь Ди-Пи (\enquote{перемещенных лиц})
в немецких лагерях. После разгрома Германии эмигрировала в Америку. Работала в
Библиотеке конгресса до выхода на пенсию. Автор прозаических и поэтических
произведений на русском языке. К материалам по истории русской печати за
рубежом. Мало известные имена, неизученные материалы (литература второй
послевоенной эмиграции). \enquote{Книга Татьяны Фесенко - это в какой-то мере история
борьбы советского человека за себя, борьбы за человечность} (из рецензии
эпохи). Для славистов, историков русской литературы, библиографов. Большая
редкость.

\begin{itemize} % {
\iusr{Оксана Дубинина}
\textbf{Виктор Киркевич} Виктор Геннадиевич, благодарю за Ваши интересные комментарии! @igg{fbicon.heart.red}

\iusr{Виктор Киркевич}
\textbf{Оксана Дубинина} Я расскажу о своих встречах с Татьяной и Андреем Фесенко в Вашингтоне.

\iusr{Оксана Дубинина}
\textbf{Виктор Киркевич} очень интересно!) ждем @igg{fbicon.face.happy.two.hands} 
\end{itemize} % }

\iusr{Ігор Печерський}
Підкажіть, а хіба ця хромолітографія не Амвросія Ждахи?

\begin{itemize} % {
\iusr{Оксана Дубинина}
\textbf{Ігор Печерський} Ні, ця хромолітографія Фесенко

\ifcmt
  ig https://scontent-frt3-1.xx.fbcdn.net/v/t39.30808-6/246185403_6535908139782731_2820470564335575748_n.jpg?_nc_cat=102&ccb=1-5&_nc_sid=dbeb18&_nc_ohc=OzqZ4w8yFd8AX-Kmpt1&_nc_ht=scontent-frt3-1.xx&oh=00_AT8NsNFzNbSJxibbYUa_70ikyV41OOagA3K2dBb7sIi19A&oe=61E4A8B3
  @width 0.2
\fi

\iusr{Ігор Печерський}
\textbf{Оксана Дубинина} 

подивився на одному з аукціонів. У назві її подано як хромолітогафію
видавництва Корчак-Новицького з картини Амвросія Ждахи. Таким чином, напевно це
хромолітографія типографії Фесенко.

\ifcmt
  ig https://scontent-frx5-1.xx.fbcdn.net/v/t39.30808-6/246359481_1217673928715048_6448157651537375860_n.jpg?_nc_cat=110&ccb=1-5&_nc_sid=dbeb18&_nc_ohc=8ydJ1AvI8MgAX_LknrG&_nc_ht=scontent-frx5-1.xx&oh=00_AT_08xnUIVQYbv1zVN3QcXFv0g4eZTXn7_TLQgFqj-5FqQ&oe=61E48525
  @width 0.4
\fi

\iusr{Оксана Дубинина}
\textbf{Ігор Печерський} по рокам таки Амвросій копію зробив з роботи Фесенка)

\iusr{Ігор Печерський}
\textbf{Оксана Дубинина} 

і хоча це не дуже принципово, але все ж таки вважаю про автора картини
(ілюстрації) забувати не слід.

Юх́им Фесéнко (Е.И.Фесенко) (1850 — 1926) — український друкар і видавець родом
із Чернігівщини. (від себе - не художник).

1883 — заклав в Одесі друкарню, видавав серед іншого ікони, лубкові картини в
українському стилі, листівки з текстами народних пісень і нотами, ілюстровані
митцем Амвросієм Ждахою та книги етнографічно-побутового змісту («Весілля»
Миколи Лисенка)...

Вікіпедія.

\iusr{Оксана Дубинина}
\textbf{Ігор Печерський} дуже вдячна за інформацію про Фесенка!

\end{itemize} % }

\iusr{Раиса Карчевская}
Очень интересно. Спасибо большое

\iusr{Ирина Иванченко}
\enquote{Лепота!}

\iusr{Junghänel Tetyana}

Я в июле, проехалась по Украине на машине, и была очень удивлена, огромному
количеству церквей ... Прям через каждые пару метров.

Зачем это?? Зачем такое количество ? Не ужели вера в Бога от этого зависит?

Моя бабушка ходила в соседнюю деревню, и ничего.

Объясните пожалуйста.

P.S. спасибо за интересный пост. Извините за отклонение от темы.

\begin{itemize} % {
\iusr{Владимир Бахтурин}
\textbf{Junghänel Tetyana} до 1917 года в Киеве было более 300 храмов различных реллигий...
А я был недавно в Праге, так там их ещё больше на квадратный километр...

\begin{itemize} % {
\iusr{Junghänel Tetyana}
\textbf{Владимир Бахтурин} я чаще в Праге и вообще в Чехии .. такого количества, я не видела.

\iusr{Junghänel Tetyana}
\textbf{Владимир Бахтурин} 

я вот просто думаю, что церковь за счёт прихожан живёт. Правильно? А если на
каждом километре церковь, храм..

\iusr{Оксана Дубинина}
\textbf{Junghänel Tetyana} прихожане под неусыпным контролем...
Я тоже удивляюсь огромному количеству новых церквей
\end{itemize} % }

\iusr{Наталья Снигирь}
\textbf{Junghänel Tetyana} я думаю, это политика московского патриархата.

\iusr{Дар'я Густіліна}
\textbf{Junghänel Tetyana}, ну вообще каждая религиозная община любого вероисповедания хочет иметь свой храм обычно. Поэтому ищут спонсоров, добиваются выделения земли, сами скидываются и строят.
\end{itemize} % }

\iusr{Оксана Кулікова}
О, в мене в бабці така висіла!

\iusr{Оксана Дубинина}
\textbf{Оксана Кулікова} теж розміром приблизно 100*50см? Можливо у Вашої бабці ще оригінальний друк був?

\iusr{Оксана Кулікова}
\textbf{Оксана Дубинина} не підкажу. Її вже немає. Були Братья Карамазови 1897 року друку з ятями і ця картина. Бабця 19 року народження.

\end{itemize} % }
