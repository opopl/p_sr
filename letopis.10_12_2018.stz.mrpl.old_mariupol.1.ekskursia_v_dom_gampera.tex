% vim: keymap=russian-jcukenwin
%%beginhead 
 
%%file 10_12_2018.stz.mrpl.old_mariupol.1.ekskursia_v_dom_gampera
%%parent 10_12_2018
 
%%url http://old-mariupol.com/ekskursiya-v-dom-gampera
 
%%author_id mrpl.old_mariupol
%%date 
 
%%tags 
%%title Экскурсия в дом Гампера
 
%%endhead 
 
\subsection{Экскурсия в дом Гампера}
\label{sec:10_12_2018.stz.mrpl.old_mariupol.1.ekskursia_v_dom_gampera}
 
\Purl{http://old-mariupol.com/ekskursiya-v-dom-gampera}
\ifcmt
 author_begin
   author_id mrpl.old_mariupol
 author_end
\fi

\begin{quote}
\em\bfseries
Маленькая экскурсия в легендарный дом Гампера, связанная с изданием (к
сожалению, мизерным тиражом) книжечки, посвящённой этому дому. В книгу вошли
все статьи мариупольских краеведов о доме Гампера и окрестностях. Книга издана
на средства жительницы дома.
\end{quote}

\ii{10_12_2018.stz.mrpl.old_mariupol.1.ekskursia_v_dom_gampera.pic.1}

Впечатление от экскурсии – дом в плачевном состоянии. Чтобы его спасти, нужны
огромные средства, которых в обозримом будущем не предвидится, слишком много
других проблем. А поэтому – дом не спасти. Ну, еще лет 50, конечно, он
протянет, даже больше, но будет достраиваться, переделываться и, в конце
концов, от дома Гампера ничего не останется. И всё.
