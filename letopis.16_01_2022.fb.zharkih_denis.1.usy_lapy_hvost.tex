% vim: keymap=russian-jcukenwin
%%beginhead 
 
%%file 16_01_2022.fb.zharkih_denis.1.usy_lapy_hvost
%%parent 16_01_2022
 
%%url https://www.facebook.com/permalink.php?story_fbid=3167621070117955&id=100006102787780
 
%%author_id zharkih_denis
%%date 
 
%%tags obschestvo,politika,prognoz,strana,ukraina,vojna
%%title Усы, лапы, хвост
 
%%endhead 
 
\subsection{Усы, лапы, хвост}
\label{sec:16_01_2022.fb.zharkih_denis.1.usy_lapy_hvost}
 
\Purl{https://www.facebook.com/permalink.php?story_fbid=3167621070117955&id=100006102787780}
\ifcmt
 author_begin
   author_id zharkih_denis
 author_end
\fi

Усы, лапы, хвост. 

Думаю, война будет. Вот пережил несколько революций, и понял, что революция
совершается, не когда революционеры пламенные, а когда власть тупая до
крайности. Дедушка Ленин точно описал формулу - верхи не могут, а низы не
хотят. Уже только за это памятник заслужил. Ведь как мудро - верхи не могут,
импотенты, занимаются черт знает чем, устают, бегают, ног не чуют, но не могут. 

Я в 2013 году в коридоры власти доступ имел. Не могли, и все. Не с кем там было
разговаривать, не то чтобы толковых людей совсем не было, но они были настолько
разбавлены массой случайных и дезориентированных людей, которые
дезориентировали других и все пускали на волю случая.  Вон Лукашенко и Токаев
смогли, а Витя Янукович бросил и страну, и родной Донбасс, который по сей день
кровью умывается. Впрочем, на Януковиче политические импотенты не кончились. 

Война продолжение политики другими средствами, хотя это тоже говорил Ленин, но
до него это сказал Карл фон Клаузевиц. Другими средствами, Карл, точнее не
скажешь. То есть, политики стали тупыми, но народ пока хочет (ага, революции не
будет) и угорает от патриотизма. 

Так вот, вместо того, чтобы заниматься политикой, президент начинает воевать, и
вся властная камарилья играет в войнушки. Играть в войнушки во властных
кабинетах хорошо и приятно, это тебе не на фронте в буквально смысле
разрываться на части, и твои дети не прячутся в подвалах, а отдыхают на
Мальдивах. 

За 30 лет мы вырастили совершенно новую управленческую элиту, которая ровным
счетом ни за что не отвечает, которая не просто плевать хотела на результаты
своего управления, но и гордиться именно именно своим управлением, несмотря на
то, что оно привело к нищете и крови страны.

А народу до поры до времени это все нравится, патриоты они, в общем. И войнушка
тоже захватывает, пока не грянет основная война. А тогда уже два сценария: 1.
Элитка поджимает хвост и сдается (очень, кстати, европейский вариант, так
Европа Гитлеру сдавалась). 2. Часть элиты расстреливают, часть разбегается по
теплым местам, а на смену приходят толковые люди, которые спасают страну (та же
ситуация в СССР). Поскольку мы идем по европейскому пути, а советский опыт
предан анафеме, то вероятность поджать хвост у нашей элиты все же больше, чем
переродиться. 

Ну, да они хвост подожмут, а лапы в руки, и рванут к своим офшорам, зарубежным
счетам и мамам диджеям. И в ус не будут дуть. А вся эта кровавая ситуация будет
на плечах простого народа, который по глупости своей эту музыку заказал.
Думали, что так патриотичнее, им так подсказали, кто ж мог знать, что так все
выйдет... А ведь Карл предупреждал, что другими средствами. Средства
закончились, пора другие средства применять. Раньше народ уничтожали нищетой,
плохой медициной, фальшивым образованием, продажными СМИ, а теперь дешево и
сердито уничтожаем народ быстро и с помощью военных действий. Народишко за 30
лет всего-то миллионов 15 потерял, а вот за пару лет так, глядишь, всего 10-15
миллиончиков останется, да и то, половина не здесь. Средства другие, а политика
та же - народ уморить, а самим карман набить. Вот и вся политика. Хотите? А
если не хотите, то это революция (она, кстати, войне не помеха), но о революции
я писал выше.
