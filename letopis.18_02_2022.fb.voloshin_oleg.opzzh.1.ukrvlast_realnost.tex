% vim: keymap=russian-jcukenwin
%%beginhead 
 
%%file 18_02_2022.fb.voloshin_oleg.opzzh.1.ukrvlast_realnost
%%parent 18_02_2022
 
%%url https://www.facebook.com/oleg.voloshin.7165/posts/5224693640896988
 
%%author_id voloshin_oleg.opzzh
%%date 
 
%%tags politika,rossia,rusmir,strana,ukraina,vlast
%%title Украинская власть продолжает жить в выдуманной реальности
 
%%endhead 
 
\subsection{Украинская власть продолжает жить в выдуманной реальности}
\label{sec:18_02_2022.fb.voloshin_oleg.opzzh.1.ukrvlast_realnost}
 
\Purl{https://www.facebook.com/oleg.voloshin.7165/posts/5224693640896988}
\ifcmt
 author_begin
   author_id voloshin_oleg.opzzh
 author_end
\fi

Украинская власть продолжает жить в выдуманной реальности. В 2014 году
искусственно радикальными методами в Украине был нарушен баланс сил. Россию,
симпатизирующих ей, просто принадлежащих к русской (для кого-то советской)
культуре стремительно задвинули на задворки, поставили перед выбором: смирение
или эмиграция. 

С тех пор дисбаланс несколько раз пробовали скорректировать эволюционным путём.
К 2021 году очевидный крах попыток построить успешное (в
социально-экономической сфере) националистическое государство открыл
возможность вхождения во власть через широкую коалицию носителей идей
домайданной Украины. Но в ответ начались никаким правовыми рамками не
ограничиваемые репрессии, санкции, ликвидация плюрализма в СМИ. 

Такой курс не просто встречал непротивление, он полностью стимулировался США и
Великобританией. Ради зачистки пространства на Западе терпели даже союз с
откровенно опасными группировками неонацистского толка, периодически
срывавшимися в антисемитские или расистские выходки и нападения на
ЛГБТ-активистов. Цель для англосаксов оправдывала средства даже в разгар борьбы
с «белыми супермасистами» у себя дома. 

Уже в разгар кризиса безопасности также по беспределу закрыли телеканал «Наш».
При этом нацики открыто хвастали, что «загнали людей русского мира под лавку и
головы им поднять не дадут». В итоге Москве оставляют один вариант -
радикальными методами восстановить баланс. 

Хотя ещё можно все остановить, пойдя на уступки. Признав невозможность
построения моноэтнической, жестко антироссийской формы государственности в
таких границах. Но этот шанс упорно продолжают упускать.
