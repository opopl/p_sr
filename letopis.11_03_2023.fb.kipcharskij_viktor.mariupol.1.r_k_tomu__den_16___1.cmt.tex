% vim: keymap=russian-jcukenwin
%%beginhead 
 
%%file 11_03_2023.fb.kipcharskij_viktor.mariupol.1.r_k_tomu__den_16___1.cmt
%%parent 11_03_2023.fb.kipcharskij_viktor.mariupol.1.r_k_tomu__den_16___1
 
%%url 
 
%%author_id 
%%date 
 
%%tags 
%%title 
 
%%endhead 

\qqSecCmt

\iusr{Zhanna Savelyeva}

В ці дні ми навчилися молитися. Молилися за маріупольців, разом з вами. Ми
знали що смерть вирує і зупинити її неможливо вже(

\begin{itemize} % {
\iusr{Віктор Кіпчарський}
\textbf{Zhanna Savelyeva} There are no atheists in foxholes.
В окопах не буває атеїстів.
Автор невідомий.
\end{itemize} % }

\iusr{Светлана Водзянская-Живогляд}

Я бачіла своїми очима як утворилися ті вирви на Куїнджі, лежала у 100 метрах,
бачила як летять и раділа що на мені ярко жовте взуття, тобто якщо буде повна
\enquote{каша} то впізнають по взуттю... отакі незрозумілі роздуми з'являються коли
лежишь і здається що то кінець...

\iusr{Светлана Водзянская-Живогляд}

Кілька днів не писала свої спогади бо я не пам'ятаю що і за чим йшло, якась
мішанина ... в якийсь із днів вийшло сонечко, і я витягла дітей у двір подихати
повітрям, хвилин 10 вони топталися по дворі з бажанням залізти назад у підвал,
але було тихо і сонце необхідне дітям. Почули гул літаку, кинулися до будинку і
не встигли.... В спину вдарила взривною хвиля, я закрила собою середню доньку,
чоловік сина, старша сіла і закричала так що я почула навіть через «вату» у
вухах . Заштовхнула середню у сарай, чоловік заніс молодшого і обійняв
перелякану Даринку(середня), я вискочила забирати старшу, вона все ще на тому ж
видоху продовжувала кричати зриваючи зв'язки і зажимаючі вуха руками, очі
скляні. Я вчепила її за \enquote{грудки} бо по іншому не підніму, вона вища за мене.
Бігом оглядаю - крові нема, голос пропав, вона труситися і сипить, веду в
будинок, вона наче лялька дерев'яна. В будинку пробую припинити істерику і
паніку, тим більше що чоловік зайшов з молодшими вони побачили і почали плакати
всі разом. За 5 хвилин зрозуміла що безсила це зробити по доброму і відвісила
їй ляпаса, нарешті очі втратили \enquote{скло}, в них звична для підлітка якого
образили ненависть до матері. Розберуся з цим потім, головне вижити. Діти з
підвалу більше не виходили взагалі, перші 2 дні після цього вони навіть у
туалет не піднімались, ми бігали з відпали та горщиком для малого.

Це була авіабомба яка лягла на котельню у 2 будинках від нас.. після цього весь
город \enquote{блищав} від уламків, тент над машинами у камінні та землі.

\begin{itemize} % {
\iusr{Віктор Кіпчарський}
\textbf{Светлана Водзянская-Живогляд} Не дай Боже пережити таке...
\end{itemize} % }

\iusr{Roy Evans}

Вже не перший раз читаю, що люди не знають чим загасити пожежу без води.
Сподіваюсь, такі знання більше не будуть нікому потрібні, але краще знати. Так
от землею дуже добре можна загасити пожежу. Якщо йдеться про багатоповерхівки,
то суміжні квартири від тої, же пожежа, треба під стінами перестати землею. Ми
так робили постійно, бо пожежі були разів 10, не меньше.

\begin{itemize} % {
\iusr{Віктор Кіпчарський}
\textbf{Roy Evans} У Вас вдома була лопата? У мене було дві штикові, три совкові, але, нажаль, тільки у гаражі.
До того ж на вулиці було -6, вночі до -12.
Коли копали яму під вогнище - землю грали багаттям.
Особисто в нас був запас води у ванні, у каністрах, у відрах та у каструлях.
А от сусід вирішив помитися, випустив воду з ванни і саме тут вода зникла...

\iusr{Roy Evans}
\textbf{Віктор Кіпчарський} це все було вже наприкінці березня, тож землю просто копали лопатами біля під'їздів. На то час вже ніяких запасів води не було, ми зливали навіть систему опалення, щоб хоч якась була, бо ходити набирати воду не було можливості.

\iusr{Віктор Кіпчарський}
\textbf{Roy Evans} У спогадах за два дні буде як робітники комунальник служб прийшли копати могили... без лопат.
У мене у багажнику машини були сокира та пилка. А от лопат не було.

\iusr{Roy Evans}
\textbf{Віктор Кіпчарський} нам з цим пощастило, інструменти були.
\end{itemize} % }
