% vim: keymap=russian-jcukenwin
%%beginhead 
 
%%file 13_10_2021.fb.fb_group.pro_movu.1.detsad_kiev_mova.cmt
%%parent 13_10_2021.fb.fb_group.pro_movu.1.detsad_kiev_mova
 
%%url 
 
%%author_id 
%%date 
 
%%tags 
%%title 
 
%%endhead 
\subsubsection{Коментарі}

\begin{itemize} % {
\iusr{Олег Миколів}
У Львові тепер на кожному кроці чути кацапів та їхній мат

\iusr{Наталія Колесніченко-Братунь}
Рекомендую «Кияночку»

\begin{itemize} % {
\iusr{Nik Rudenko}
\textbf{Наталія Колесніченко-Братунь} а це сепаратизм,, Київ завжди терпів рахітів,,
\end{itemize} % }

\iusr{Валентин Боговик}
На мій погляд, треба раптом перестати розуміти російську всім українцям, і тоді
побачимо, як поведуть себе російськомовні!

\begin{itemize} % {
\iusr{Наталія Вишневецька}
\textbf{Валентин Боговик} біда в тому , що більшість хохлів міркує саме так, як написала авторка. " в чьом плохо знаніє русскава,,, ,,нє заастряйтє,, і т.п.

\iusr{Милочка Романенко}
\textbf{Валентин Боговик} 100\% !

\iusr{Маргарита Сидоренко Дарда}
\textbf{Валентин Боговик}, 

у тому й проблема, що російськомовні тримаються за цей статус з упертістю
баранів, бо «мьі же с детства разговариваем на русском».

До такої міри «совєти», а ще раніше імперські москалі так затуркали дурних
хохлів, що і через 30 років незалежності не хотять перейти на мову дідів, бо
«єто же жлобский язьік селюков и кугутов».

В українців, на відміну від хохлів, Слава Богу, вистачило розуму повернутись до
спілкування рідною мовою.

На мій погляд, з лагідної українізації, що відбувалась протягом 30-ти років
нічого не вийшло та й вийти не могло. Тільки жорсткий закон про мову і штрафи
за недотримання його, бо до шлункоголових достукатись можна тільки через гроші
та шлунок!

\begin{itemize} % {
\iusr{Валентин Боговик}
\textbf{Маргарита Сидоренко} Дарда абсолютно згоден!
\end{itemize} % }

\iusr{Людмила Трохименко}
\textbf{Валентин Боговик} так і роблю, більшість у відповідь переходять на гарну українську.

\iusr{Vira Mazurok}
\textbf{Валентин Боговик} я так і кажу російськомовним: Не розумію, прошу українською

\end{itemize} % }

\iusr{Олександр Харчук}

Нажаль Київ так загажений 'понаєхавшимі ',що виправити ситуацію можливо
перемістивши столицю в інше місто, або побудувавши заново,такі приценденти в
світі були!

\begin{itemize} % {
\iusr{Tetjana Sibukowa}
\textbf{Олександр Харчук} 100\%!

\iusr{Віктор Коваленко}
Вороги України того й хочуть. Щоб українці втекли з Києва. Але тікати зі своєї столиці - це не вихід. Слава Україні!  @igg{fbicon.heart.red}  @igg{fbicon.heart.sparkling}  @igg{fbicon.flag.ukraina} @igg{fbicon.heart.yellow}  @igg{fbicon.heart.blue} 

\begin{itemize} % {
\iusr{Ольга Дмітрікова}
\textbf{Віктор Коваленко} Героям Слава

\iusr{Nik Rudenko}
\textbf{Віктор Коваленко} Погані висновки,,,а де маркплан,,,
\end{itemize} % }

\iusr{Генаїда Галас}
\textbf{Олександр Харчук} ,особливо тепер, які переїхали зі Сходу і всім не
задоволені, я не про всіх пишу, спілкувалася з хорошими людьми.

\begin{itemize} % {
\iusr{Nik Rudenko}
\textbf{Генаїда Галас} Їх треба ставить в ,,-а де були ,і що робили,,,і чому приїхали!!!!

\iusr{Генаїда Галас}
\textbf{Nik Rudenko} ,так, багато сєпарів.
\end{itemize} % }

\iusr{Маргарита Сидоренко Дарда}
\textbf{Олександр Харчук}, 

столицею України є і буде Київ.

А от з «понаехавшьіми» боротись треба на рівні законів та штрафів.

Не виконуєш закон про мову, маєш штраф і позбавлення можливості займатися
бізнесом в Україні. Через шлунок до таких доходить дуже швидко.

\end{itemize} % }

\iusr{Татьяна Бурдун}
Заборонити вчителям розмовляти російською ,адже вони на роботі і мають говорити українською

\iusr{Зоя Вячеславовна}
Ганьба київлянам

\begin{itemize} % {
\iusr{Євген Діброва}
\textbf{Зоя Чаплинська} Правильно буде "киянам" @igg{fbicon.wink} . Але як для Херсона - то піде...

\begin{itemize} % {
\iusr{Зоя Вячеславовна}
\textbf{Євген Діброва} вибачаюсь помилково

\iusr{Александр Васильевич}
\textbf{Євген Діброва} Все знаєте. А як українською будуть мешканці Краматорську?

\iusr{Олександр Ніко}
\textbf{Александр Васильевич} Краматорськ-краматорці, краматорчани. Луганськ-луганці, луганчани. Якщо є сумніви, -мешканці Краматорська, Горішніх Плавнів...(особливо стосується мешканок Кривого Рогу)

\iusr{Євген Діброва}
\textbf{Олександр Ніко} Дякую, Олександре. Вважаю, що "Алєксандр Васільєвіч", прочитавши Ваш коментар, залишиться задоволеним. @igg{fbicon.wink} 

\iusr{Євген Діброва}
\textbf{Александр Васильевич} Знаю, на жаль, не все. Але намагаюся пізнавати якамога більше. До речі, почитайте відповідь на своє запитання у коментарі Вашого тезки Олександра Ніко.
\end{itemize} % }

\iusr{Василь Корпан}
Ганьба тим хто наволоч у владу обирає-а це наслідки...

\end{itemize} % }	

\iusr{Julie Tolmacheva}
Не переходити на російську у спілкуванні з російськомовними. Поширення
української залежить і від нас, носіїв мови.


\iusr{Юрій Дмитрович Федірчик}

Це - "тихий" саботаж руського міра, а риба гниє з голови... Якщо ця влада не
взмозі, чи не хоче ....  Думайте!!!

\begin{itemize} % {
\iusr{Галя Климчук}
\textbf{Юрій Дмитрович Федірчик} Чому все винна влада? Починати треба кожному з себе. Всі розуміють українську, але спілкуються російською.

\begin{itemize} % {
\iusr{Євген Діброва}
\textbf{Галя Климчук} Так, починати з себе, але якщо з трибуни Верховної Ради та інших, рангом нижче, лунають проросійські виступи - то як?

\iusr{Галина Кохмат}
\textbf{Євген Діброва} 

Чому я маю дивитись на якихось депутатів чи неуків чиновників? я українка і
розмовляю на всій території України тільки українською. москалів не розумію. все!

Чому в Росії всі переходять на російську, а в Україні-ні. це ж як треба себе не
поважати, принижуватиме перед чужим!

\iusr{Євген Діброва}
\textbf{Галина Кохмат} Тому що ті депутати типу Бужанського, Дубінського та інші - є вороги України. І їх необхідно ставити на місце. А не помічати таких виродків - то нагадує притчу про страуса та його голову, що занурена у пісок.
\end{itemize} % }

\end{itemize} % }

\iusr{Lars Strashevsky}

Все рускє викінути на СМІТНИК. їх музику, книжкі на якізє, все що ВИРОБЛЕНО В
СССР, або в росії... ВИКІНУТИ НА СМІТНИК. Щоб навіть не нагадувало. І все
зміниться.

\iusr{Александр Глушко}
Начасі зазвучить. Не так швидко, у вас там теж ракша є.

\iusr{Віктор Коваленко}
А чому не піти таким шляхом : Обрати україномовного Президента.

@igg{fbicon.heart.red} ️  @igg{fbicon.heart.sparkling} @igg{fbicon.heart.red} @igg{fbicon.heart.yellow}  @igg{fbicon.heart.blue} 

\iusr{Nadja Korobchuk}

Сталіца, одне слово.... Сумно. Технічні українці!!!!!

\iusr{Василь Воскобійник}

що зробиш коли риба гніє з голови. закони наче і прийняли, щоб не гавкали, а
віз і досі там. як це буде виконуватись якщо половина слуг кацапляться деякі
пуйла вітають відкрито і славлять цього катюгу перед нашими людьми, от і маємо
бо при цій владі всяке гімно спливає на поверхню і смердить і цим пишається.


\iusr{Олександр Чигрин}

Нас спасе від знишкодження тіки українська нація, непереплутайти дибіли,
націоналізм і нацизм, це велика різниця і неосвідченість нашого люду це наша
біда, тому задача не в зелених чи голубих , а наш прогрес в збереженні
світового музею нації УКРЇНА, це воля народа , не цар, не імператор, не король,
це діалектика розвиток вільної мислі

\begin{itemize} % {
\iusr{Nik Rudenko}
\textbf{Олександр Чигрин} А мораль не стабільна ,,,держись,,
\end{itemize} % }

\iusr{Стефанія Кучерко}
Так і є.

\iusr{Тетяна Дехтяр}

80\% депутатів ВР, поза парламентом, дають інтерв'ю російською......
Передивіться Трускавець. Там жодного українця. Тому скрізь «какая разніца».

\begin{itemize} % {
\iusr{Лариса Ерещенко}
\textbf{Тетяна Дехтяр} , 

от мені цікаво а хто все ж таки голосував за цих весільних фотографів -
невігласів, " дітей Трускавця", вони ж потрапили у парламент не самі по собі!
Чи дивилися люди їхні передвиборчі програми, їх біографії, їхні трудові
здобутки, чи задавали питання на зустрічах з кандидатами?..

Доречі, там би і побачили, як володіє кандидат державною мовою...

Наша біда - це несерйозне ставлення до виборів самих виборців. Бабусі голосують
за гречку чи відремонтовану у дворі лавочку, молодь часто взагалі не цікавиться
політикою, байдужість до свого майбутнього у цій країні. ... Людей
відповідальних, небайдужих, пасіонарних - малий відсоток.

А по-друге - неповага до владних структур як інституцій - у тих, хто туди
приходить. Тому такі зашквари, невиконання мовного і інших законів.

\begin{itemize} % {
\iusr{Тетяна Дехтяр}
\textbf{Лариса Ерещенко} Дивилися серіали «Свати» і «Слуга народу». Слухали 95-й квартал. Тепер маємо, що маємо  @igg{fbicon.cry} 
\end{itemize} % }

\iusr{Nik Rudenko}
\textbf{Тетяна Дехтяр} це не надовго,,,

\begin{itemize} % {
\iusr{Nik Rudenko}
Не спіть,,,вчить дітей і долбаних родичів,,,та і сусідів,,, щоб ми були нацією,
\end{itemize} % }

\iusr{Nadiia Huzhel}
\textbf{Тетяна Дехтяр} 

При чому тут депутати ? "Риба гниє з голови, але чистять її з хвоста" - не чули
? "Хвіст"-це громадяни, і українці в тому числі, які привели до влади цих
невігласів і дилетантів. Українці, які в своїй більшості не здатні усвідомити
себе нарешті українцями і навчити цьому своїх дітей.........

\begin{itemize} % {
\iusr{Nik Rudenko}
\textbf{Nadiia Huzhel} з вашої голови,,,

\iusr{Nadiia Huzhel}
\textbf{Nik Rudenko} По-перше, "Вашої" при звертанні до незнайомої людини пишеться з великої букви, по-друге, я зелену плісняву не приводила до влади......

\iusr{Nik Rudenko}
\textbf{Nadiia Huzhel} да велика буква буде ,,коли ваша корисна робота буде зроблена для задоволення вашої сім'ї і друзів,,,

\iusr{Тетяна Дехтяр}
\textbf{Nadiia Huzhel} І то правда, при чому ті депутати????? Та так, пані, ні при чому. Тільки вони порушують закон про мову відкрито, нагло, цинічно. Тому спитати з інших вже не має що  @igg{fbicon.cry} 

\iusr{Nadiia Huzhel}
\textbf{Тетяна Дехтяр} 

Ви мабуть не зрозуміли про Що я написала: ці люди, яких наш народ привів до
влади нагло і цинічно порушували закони ще до приходу до влади, і не тільки
закони, але всі моральні та етичні засади, принижуючи українців у своїх
дебільних шоу і серіалах, тому питати перш за все треба з тих, хто дав їм всю
повноту влади...

\end{itemize} % }

\end{itemize} % }

\iusr{Владимир Александров}
Зразу перейти на украiнську мову не .можливо

\begin{itemize} % {
\iusr{Тетяна Дехтяр}
\textbf{Владимир Александров} 30 років незалежності — це ще «зразу»????? А коли вже зможем?????

\iusr{Роксолана Роксолана}
\textbf{Владимир Александров} добре , що на кац"пську легко переходили

\iusr{Mariia Piatkivska}
\textbf{Владимир Александров} Все можливо, було б тільки бажання

\iusr{Валентина Малярова}
\textbf{Владимир Александров} А вакцинуватись під час пандемії в самий раз !!!

\iusr{Svitlana Khmelivska}
\textbf{Владимир Александров} 

А все життя прожити в Україні і не знати української мови, це як назвати? Мої
однокласники перейшли на москальську зразу, як лише переїхали в Київ. То це
можливо бігом "перевзутися "і стати перевертнями?

\end{itemize} % }

\iusr{Лариса Товкач}

Така ситуація не тільки в Києві. Практично скрізь в Запоріжжі та області
говорять російською або суржиком. Це прикро, адже знайти україномовне
середовище для дитини практично неможливо.


\iusr{Nik Rudenko}
І це правда,,

\begin{itemize} % {
\iusr{Nik Rudenko}
Треба щось робити,,,чому всі заснули
\end{itemize} % }

\iusr{Viktor Bondarets}
тре поміняти Клічка

\begin{itemize} % {
\iusr{Богдан Валерійович Щербак}
\textbf{Viktor Bondarets} це не є сферою відповідальності мера
\end{itemize} % }

\iusr{Ігор Р.}

Коли мої малята ще були малятами, категорично наполіг на українській групі у
дитсадку, а потім у школі. Було то у Донецьку і дружина висловлювала
занепокоєння стосовно подальшого навчання у виші, потім роботи. Мовляв, не
знатимуть термінології і зтикатимуться з проблемами, специфічними для Сходу в
цілому. Я не добирав аргументи, просто сказав, що буде так і діти благополучно
довчилися хоч в такому варіанті, коли предмети викладали українською, а на
перервах, самі розумієте (доня закінчила школу вже в Києві). Я задоволений
своїм авторитетним рішенням. Діти володіють рідною без східняцького акценту. А
отой діалект, на якому писав Пушкін, причепився автоматично.

Мені до вподоби тактика: забути москальську. Зі мною більшість кацапомовних
переходить на українську. На днях працюю в компанії з двома колегами, які у
побуті розмовляють на "опщепанятнам". Звісно, під моїм впливом вони таки
нормально складають речення українською і, о диво, я чую, як знаходячись у
сусідній кімнаті вони й між собою спілкуються солов'їною. Закликаю всіх ніколи
не перемикатися з рідної хвилі. Ми не зобов'язані це робити.

\iusr{Микола Микола}

Агресор напав не на конкретний адрес, вулицю, будинок чи місто! Агресор напав
на Україну!  Війна не Сході нашої Держави!!! Війна в Україні!!! А скільки ще
ВОРОГІВ в тилу, в кожному місті, селі, на кожній вулиці!!!  Боже допоможи
справитись з цією навалою!  Облагорозуми їх...

\begin{itemize} % {
\iusr{Nik Rudenko}
\textbf{Микола Микола} 22 війни,,і ви думаєте що вас відпустять,,,,

\iusr{Микола Микола}
\textbf{Nik Rudenko}
Ти про що?

\end{itemize} % }

\iusr{Світлана Юшицина}

Є у наших людей щось плебейське - відмовлятись від свого рідного без найменшого
супротиву, не встигнувши вийти з рідної хати, все забувають(((


\iusr{Богдан Петрусь}
Важко .Какая разніца ,ніяк не розуміють хто ворог і хто товариш

\begin{itemize} % {
\iusr{Nik Rudenko}
\textbf{Богдан Петрусь} да нарешті індефікуйся,,,
\end{itemize} % }

\iusr{Lars Strashevsky}

Ніколи НЕ ПЕРЕХОДЬ НА ВОРОЖУ МОВУ, ТИ ПОКАЗУЄШ СВОЮ РАБСЬКУ ПОКІРНІСТЬ, ТА
СОРОМ У СЕБЕ ВДОМА НА РІДНІЙ ЗЕМЛІ, і своїм вчинком показує рускоязичніку що
тут він пан а ти раб.

\begin{itemize} % {
\iusr{Nik Rudenko}
\textbf{Lars Strashevsky} вивчи мову,,,

\iusr{Lars Strashevsky}
Вчу пан і не одну. Завжди рад коли мене виправляють. Дякую.

\begin{itemize} % {
\iusr{Nik Rudenko}
\textbf{Lars Strashevsky} виправся сам,а потім напрям ок \_давай!!!!
\end{itemize} % }

\iusr{Nik Rudenko}
\textbf{Lars Strashevsky} А ви не дружете з собою ,,,

\iusr{Lars Strashevsky}
Рускій троль не нервуйся.

\end{itemize} % }

\iusr{Валерий Мазуркевич}

\obeycr
Степи мої запродані
Жидові, німоті,
Сини мої на чужині,
На чужій роботі. /253/
Дніпро, брат мій, висихає,
Мене покидає,
І могили мої милі
Москаль розриває...
Нехай риє, розкопує,
Не своє шукає,
А тим часом перевертні
Нехай підростають
Та поможуть москалеві
Господарювати,
Та з матері полатану
Сорочку знімати.
Помагайте, недолюдки,
Матір катувати».
\restorecr

\iusr{Лариса Речун}

Комплекс меншовартості нав'язували московити українцям сотнями років , причому
методів не підбирали . Повага до себе , знання своєї історії - добрі помічники
, щоб того комплексу позбутись і бути гідними у своїй державі , розмовляючи
рідною українською . А ще - мова - дуже потужна зброя , здатна зупинити війну .


\iusr{Петро Васьків}

Сумна картина. Найбільша руїна, яку заподіяли московити - духовна.

\iusr{Валерий Мазуркевич}

\obeycr
І знов Тараса згадав. Штовхаюсь я; аж землячок,
Спасибі, признався,
З циновими ґудзиками:
«Де ты здесь узялся?»
«З України». — «Так як же ти
Й говорить не вмієш
По-здешнему?» — «Ба ні, — кажу, —
Говорить умію,
Та не хочу». — «Экой чудак!
\restorecr

\iusr{Микола Мудроха}

\ifcmt
  ig https://scontent-frt3-1.xx.fbcdn.net/v/t39.30808-6/245119368_10216067020307306_8140290882546146684_n.jpg?_nc_cat=102&ccb=1-5&_nc_sid=dbeb18&_nc_ohc=vvUJonjstgoAX_zWpp0&_nc_ht=scontent-frt3-1.xx&oh=240fd6c6fb90d497b06852a0756db757&oe=616F2142
  @width 0.4
\fi

\iusr{Oksana Klokova}

Неможливо уникнути. Нас замало, щоб змінити ситуацію. Я працюю у художній
школі, і не вдається умовити дітей розмовляти українською хоча б на уроках.
Хтось просто не вміє, хтось свідомо відмовляється. Є, звісно, і такі, які зі
мною спілкуються українською, але між собою - майже завжди дитина "ламається",
бо хоче бути своєю... Сумно це...

\begin{itemize} % {
\iusr{Наталія Зубенко}
\textbf{Oksana Klokova}
\obeycr
Мови рідна,
СЛОВО рідне,
Хто вас забуває,
Той у грудях ,
Не серденько,
Але, камінь має.
\restorecr

\iusr{Тетяна Гиря}
\textbf{Oksana Klokova} ,дуже сумно, але потрібно лагідно стояти на своєму.
\end{itemize} % }

\iusr{Микола Мудроха}

\ifcmt
  ig https://scontent-frx5-1.xx.fbcdn.net/v/t39.30808-6/245146629_10216067025507436_5922179028594262923_n.jpg?_nc_cat=110&ccb=1-5&_nc_sid=dbeb18&_nc_ohc=DJ-kLn8IrxoAX8O6B78&_nc_ht=scontent-frx5-1.xx&oh=0ba81f0e41f199bcd12ef08e31d79bb4&oe=616DCD6A
  @width 0.4
\fi

\begin{itemize} % {
\iusr{Nik Rudenko}
\textbf{Микола Мудроха} мові. Та не переймайся,,друже -все буде гарно(файно)
\end{itemize} % }

\iusr{Микола Мудроха}

\ifcmt
  ig https://scontent-frt3-1.xx.fbcdn.net/v/t39.30808-6/245077092_10216067029707541_2633546302083409546_n.jpg?_nc_cat=102&ccb=1-5&_nc_sid=dbeb18&_nc_ohc=SSnogk-gdQIAX-nmPk_&_nc_ht=scontent-frt3-1.xx&oh=d98116cd366f326314e066d46693c2af&oe=616E63C6
  @width 0.4

	ig https://scontent-frx5-1.xx.fbcdn.net/v/t39.30808-6/244969003_10216067031107576_7246712607138162629_n.jpg?_nc_cat=105&ccb=1-5&_nc_sid=dbeb18&_nc_ohc=2TAl97qOBNgAX_EvzJk&_nc_ht=scontent-frx5-1.xx&oh=6289dc214540e660b893844d06773456&oe=616E12BC
  @width 0.4
\fi

\begin{itemize} % {
\iusr{Nik Rudenko}
\textbf{Микола Мудроха} ні ,,це розум ,,віра ,,і інтелект,,,
\end{itemize} % }

\iusr{Іван Репей}

ЯК ЦЕ ДУЖЕ ПРОСТО РОБИТЬСЯ, ЗАКОН ПРО МОВУ Є, Є, ТАК ОСЬ, СЛІДУЮЧИЙ ЗАКОН МАЄ
БУТИ ДРУГИМ. РАЗ В УКРАЇНІ Є ЗАКОН ПРО МОВУ ТО МАЄ БУТИ ЗАКОН ПРО ВИКЛАДАННЯ У
ВСІХ ЗАКЛАДАХ ОСВІТИ НА ТЕРЕНАХ УКРАЇНИ, УКРАЇНСЬКОЮ МОВОЮ А ВСІ ОСТАЛЬНІ МОВИ
МАЮТЬ ПОЗНАЧАТИСЯ У ВИКЛАДАНІ ЯК ІНОЗЕМНА МОВА. ЩО ЩЕ ТРЕБА, А ТРЕБА ПЕРЕД
ВЕРХОВНОЮ РАДОЮ ЗРОБИТИ МІТИНГ НА ЗАХСТ УКРАЇНСЬКОЇ МОВИ. І ДОБИТИСЯ ПРИЙНЯТЯ І
ЗАТВЕРДЖЕНЯ ЗАКОНУ ПРО МОВУ В ОСВІТНІХ ЗАКЛАДАХ УКРАЇНИ ТА ТРЕБА ПОСПІШАТИ ЩОБ
В НАСТУНОМУ 2022 році ПО УКРАЇНІ ВЖЕ ВЕЛОСЯ ВИКЛАДАННЯ УКРАЇНСЬКОЮ МОВОЮ. І ХАЙ
БИ ЯК ТАМ МОСКАЛІ НЕ ВЕРИЩАЛИ, УКРАЇНСЬКІ ЗАКОНИ ПРИЙМАЮТЬСЯ НА УКРАЇНСЬКІЙ
ЗЕМЛІ А НЕ В РОСІЇ. БО ЧИМ БІЛЬШЕ БУДУТЬ В УКРАЇНІ РОЗМОВЛЯТИ І ВИКЛАДАТИ
НЕДРЖАВНОЮ МОВОЮ ТИМ БІЛЬШОЮ СПОКУСОЮ БУДЕ ДЛЯ РОСІЙСЬКИХ ЗАГАРБНИКІВ ЗАХИЩАТИ
РОСІЙСЬКОМОВНИХ ГРОМАДЯН В УКРАЇНІ ЯКИХ БУДЕ З КОЖНИМ РОКОМ ВСЕ БІЛЬШЕ
МНОЖИТИСЯ. Ось так мої любі києвляни і всі інші українці СВОЮ МОВУ ЯК І ДЕРЖАВУ
ТРЕБА ВІДСТОЮВАТИ, БОРОНИТИ ТА ЗАХИЩАТИ. І НЕВІДТЯГУЙТЕ НА ДОВГО, ПІДНІМАЙТЕ
НАРОД НА МИРНИЙ ЗАХИСТ УКРАЇНИ І УКРАЇНСЬКОЇ МОВИ ЦЕ ДВА В ОДНОМУ, ЯКЕ ТРЕБА В
ПЕРШУ ЧЕРГУ ЗАХИСТИТИ. НЕБУДЕ ЗАХИТУ МОВИ НЕБУДЕ ДЕРЖАВИ УКРАЇНА. ...а, тепер
хто зімною незгодний можите мене шматувати і зїсти, тільки знайте я вже занадто
старий то можите зубки поламати.

\begin{itemize} % {
\iusr{Ольга Гнатко}
\textbf{Іван Репей} та соромно за вчителів викладачів, бо вони впершу чергу
спілкуються російською, на перервах 100\% на уроках, ще може і дотримуються.
\end{itemize} % }

\iusr{Микола Мудроха}

\ifcmt
  ig https://scontent-frt3-2.xx.fbcdn.net/v/t39.30808-6/245587939_10216067037907746_6886189683109500277_n.jpg?_nc_cat=101&ccb=1-5&_nc_sid=dbeb18&_nc_ohc=mW_f4k9JmP0AX_8mfcR&_nc_ht=scontent-frt3-2.xx&oh=c900a7ed618836a60cb55257758aba74&oe=616E01CD
  @width 0.4
\fi

\iusr{Petro Zelenycia}
Не русифікація, а росій-щеня=москов-щеня.

\iusr{Люба Кубара}
Парадокс ,,україномовна дитина відчуває себе іноземцем" у Києві-столиці України!

\begin{itemize} % {
\iusr{Nik Rudenko}
\textbf{Люба Кубара} так вам і треба ,,,слабі,,,,

\iusr{Люба Кубара}
\textbf{Nik Rudenko} а ви ,,здорові"

\iusr{Nik Rudenko}
\textbf{Люба Кубара} років ви хворі ,,я не здоров,,,

\iusr{Nik Rudenko}
\textbf{Люба} ,,\_я здаюсь ,,почитав ваш ф,б, тттттак і треба боронити,,
\end{itemize} % }

\iusr{Ігор Попович}

Київ - україномовне місто! А хто не розмовляє українською, той або окупант, або
хо..хол дресерований московським ґаспадіном на руссскам пантовам язикє
разбалаківать і соромитися рідної мови!

\iusr{Любов Бойчук}
Україна- має розмовляти УКРАЇНСЬКОЮ!

\iusr{Svitlychnyi Olexandr}
і цев Києві, а що казати за ще більш зросійщені міста...

\iusr{Марія Метелюк}

Коли мої діти та онуки ще 16 років назад перебралися на Полтавщину, де
українською теж майже ніхто не розмовляв, то я дуже переживала. Одного разу,
перед Різдвом ,приїхала забрати чотирирічного онука до нас ,а виховательки
каже:"Як мені стало легше, що ваш онук знає українську. Він дітей у групі
виправляє: не конєшно, а звичайно, не счас, а зараз, не понімаю, а
розумію"....Потім діти переїхали в Київ ..І знову переживала, але, слава Богу,
говорять з усіма і діти ,і онуки лише українською. Хоча в садочку пробувала
говорити російською, та онук, який старший на сім років, її постійно вчив:"Ти де
живеш? В Україні! То говори своєю мовою! Українською!". Прикро, що в Києві,
столиці України, в транспорті рідко почуєш державну мову, на перерві вчителі
теж мало її використовують, а варто починати саме із садочків, шкіл виправляти
дітей. Тоді це буде освіта українською, а не лише на папері.


\iusr{Олексій Сухий}
В столиці України складно виховувати дитину україномовною... який сором(((

\begin{itemize} % {
\iusr{Lyudmyla Mlosch}
\textbf{Олексій Сухий} погоджуюсь, який сором!
Нема на то ради... Ганьба!

\iusr{Наталія Зубенко}
\textbf{Lyudmyla Mlosch} Є рада!!! Дотримання закону про українську ДЕРЖАВНУ мову!!!!
\end{itemize} % }

\iusr{Віктор Кримський}
Записати в якусь спортивну секцію, україномовну, НЕМОЖЛИВО!!!!

\iusr{Тетяна Гиря}

Ми - не меншоварті, ми - народ України !!!.Хай ті ,що живуть в Україні, і ті
,що приїхали жити до нас і не хочуть вчити нашу мову відчувають себе
меншовартими. Бо позаймали робочі місця і "чтокають",а українцям доводиться
їхати за кордон на заробітки. Пора уже не боротися зі своїм народом, а навести
лад в країні, яка бореться з російським окупантом, а російськомовні вольготно
себе відчувають і закривають рота, м'яко кажучи, українцям.

Хай вивчають мову! А школи ,садочки просто зобов'язати говорити українською, хто
з вчителів не може, чи цурається говорити українською - хай шукає іншу
роботу. Досить панькатись, бо ми так довіку не позбудемось агресора, який
всілякими способами вносить розбрат в Україну. МИ - УКРАЇНЦІ!!!!!

Поважаймо себе!!

\begin{itemize} % {
\iusr{Lyudmyla Mlosch}
\textbf{Тетяна Гиря} 100відсотково погоджуюсь з Вами!
За незнання державної мови, порушення закону- звільняти з посад. Можливо запровадити штрафи..., а новоприбулих працівників
без знання мови не приймати на роботу..
\end{itemize} % }

\iusr{Лидия Беленченко}
Спочатку створити україномовне середовище в сім'ї

\iusr{Надія Зборщик}
Мамі респект, що тримається мужньо! Отак би побільше!

\iusr{Надія Чухіна}
Подавіться що зі Львовом робиться, панаєхалі...

\iusr{Олег Касяненко}

Дика ситуація, хоча і не унікальна. Ірландці здобули незалежність в 1928 році,
але майже втратили свою мову бо інтелектуальна еліта вихована колонізаторами не
захотіла відмовлятися від англійської. Не будемо, як ірландці. Україна повинна
бути тільки українською.

\begin{itemize} % {
\iusr{Наталія Зубенко}
\textbf{Олег Касяненко} Саме так!!! УКРАЇНА ПОВИННА БУТИ УКРАЇНСЬКОЮ.
\end{itemize} % }

\iusr{Galina Bogarada}

У нас теж така проблема в Києві, але боремось- обидві внучки розмовляють тільки
українською, а російська для них- мова ворога

\begin{itemize} % {
\iusr{Юлія Клименко}
\textbf{Galina Bogarada} 

і мої батьки боролись, ще з 60 их розмовляли завжди з нами лише рідною, і в часу
тотальної русифікації, коли в моїй,"українській"школі, з горем пополам на уроках
звучала українська та й не завжди..... мої батьки скрізь і на роботі і в
оточенні розмовляли лише українською.... це було не просто, але ми вижили!!))

\end{itemize} % }

\iusr{Маня Чепурна}
Гідна позиція батьків!

\iusr{Світлана Ларійчук}

Дякую за
Вашу позицію і наполегливість.
Сподіваюсь,
Ви таки переможете, бо справедливість і закон на
Вашій стороні.

\iusr{Oleg Kozak}
І як на фоні цієї зневаги, може відбутись держава, у повному розумінні цього слова? Одним словом ,Моголи!

\iusr{Люба Бандрівська}

\textbf{Іван Репей} 

Я з вами повністю згідна, але без образ ,вам би не завадило підовчити трохи нашу
мову, бо неграмотність у вашому коменті зашкалює.

\iusr{Наталія Зубенко}
 @igg{fbicon.sunflower}{repeat=3}  @igg{fbicon.cherries}{repeat=3}  @igg{fbicon.sunflower}{repeat=3} 

\iusr{Юлія Клименко}

Знаю особисто людей, які живучи в селі під Києвом, також у містечку за 70 км від
Києва навмисне вчать дітей російською.... коли я задала питання чому, почула
відповідь - бо дитині легше було вчитися рощмовляти російською..... То про що
мова ...у мене таких приводів ніколи в сім"ї не було і ось вже в енному
поколінні.... А тут, сумно і соромно за таких батьків і прикро за наш народ.....




\end{itemize} % }
