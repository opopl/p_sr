% vim: keymap=russian-jcukenwin
%%beginhead 
 
%%file slova.deputat
%%parent slova
 
%%url 
 
%%author 
%%author_id 
%%author_url 
 
%%tags 
%%title 
 
%%endhead 
\chapter{Депутат}
\label{sec:slova.deputat}

%%%cit
%%%cit_head
%%%cit_pic
%%%cit_text
Таков лозунг \emph{народных депутатов} от Евросолидарности.  Народные
\emph{депутаты} ЕС Порошенко зарегистрировали в Верховной Раде проект
постановления № 5674 о заявлении парламента о прекращении любых дипломатических
отношений Украины с Беларусью. В постановлении Украина призывает
демократические страны мира к продолжению и усилению политического и
экономического давления на Беларусь и до введения персональных санкций в
отношении должностных лиц Беларуси. Докладывать относительно проекта
постановления в Раде будет \emph{нардеп} \enquote{Евросолидарности} Алексей
Гончаренко
%%%cit_comment
%%%cit_title
\citTitle{Фракция Европейская солидарность намерена прервать все отношения с Беларусью}, 
Валерий Песецкий, strana.ua, 24.06.2021
%%%endcit

%%%cit
%%%cit_head
%%%cit_pic
%%%cit_text
Тем не менее, парламент не постеснялся уполномочить его своим постановлением.
Иными словами, Конституцию тогда отменили постановлением Верховной Рады. Одна
сплошная рубрика «Очевидное и невероятное».  В 2019-м \emph{депутаты} абсолютно
по-шулерски тиснули в Конституцию курс на евроинтеграцию и НАТО. Причем в
первый раздел это «счастье» пихать не стали, чтобы референдум не проводить.
Закрепили в преамбуле и успокоились.  И если вам кажется, что это уже дно, то
советую подумать еще раз. Ранее, чтобы придать своим действиям видимость
законности, власти меняли Конституцию. А вот президент Зеленский приличия не
блюдет. Он просто делает вид, что никакой Конституции в природе не существует.
Захотел и ограничил украинцев в правах своими пятничными санкциями
%%%cit_comment
%%%cit_title
\citTitle{Ранее, чтобы выглядеть прилично, власть меняла Конституцию / Лента соцсетей / Страна}, 
Максим Могильницкий, strana.ua, 28.06.2021
%%%endcit

%%%cit
%%%cit_head
%%%cit_pic
\ifcmt
  pic https://strana.ua/img/forall/u/0/36/2021-07-03_14h36_25.png
\fi
%%%cit_text
Еще один участник опроса также считает, что для спортсменов не слишком важно,
на каком языке говорить и давать интервью, в отличие от чиновников или
\emph{депутатов}, например.  \enquote{Ну, скажем так, если вопрос на украинском
языке, а он отвечает... Но он же не совсем публичная личность. Это спортсмен.
По-хорошему, по-правильному, было бы на украинском ответить. Но, опять-таки, не
знаю. Депутаты и те, кто на виду, те обязаны. А футболист не настолько
публичная персона}, - делает киевлянин скидку для спортсменов
%%%cit_comment
%%%cit_title
\citTitle{Что говорят украинцы о пресс-конференциях футболистов на русском языке. Опрос Страны}, 
Антонина Белоглазова, strana.ua, 03.07.2021
%%%endcit
