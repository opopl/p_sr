% vim: keymap=russian-jcukenwin
%%beginhead 
 
%%file 14_10_2017.stz.news.ua.mrpl_city.1.jakov_gugel_k_80_letiju_tragicheskoj_gibeli
%%parent 14_10_2017
 
%%url https://mrpl.city/blogs/view/yakov-gugel-k-80-letiyu-tragicheskoj-gibeli
 
%%author_id burov_sergij.mariupol,news.ua.mrpl_city
%%date 
 
%%tags 
%%title Яков Гугель. К 80-летию трагической гибели
 
%%endhead 
 
\subsection{Яков Гугель. К 80-летию трагической гибели}
\label{sec:14_10_2017.stz.news.ua.mrpl_city.1.jakov_gugel_k_80_letiju_tragicheskoj_gibeli}
 
\Purl{https://mrpl.city/blogs/view/yakov-gugel-k-80-letiyu-tragicheskoj-gibeli}
\ifcmt
 author_begin
   author_id burov_sergij.mariupol,news.ua.mrpl_city
 author_end
\fi

\ii{14_10_2017.stz.news.ua.mrpl_city.1.jakov_gugel_k_80_letiju_tragicheskoj_gibeli.pic.1}

Позвонила Галина Михайловна Захарова - известная правозащитница, автор
нескольких книг и огромного количества публикаций о сталинских репрессиях,
которая возвратила из небытия имена невинно погубленных людей:

- Вы, знаете, что в ближайшее воскресенье, 15 октября, исполняется 80 лет, как
в Киеве, в застенках НКВД был расстрелян Яков Семенович Гугель? Это ли не
повод, чтобы вспомнить человека, чья деятельность оставила глубокий след в
истории Мариуполя?

Действительно, в почетном списке личностей, внесших весомый вклад в развитие
промышленности в Мариуполе, Я. С. Гугель занимает далеко не последнее место. Он
руководил восстановлением ильичевского завода \enquote{Б} (бывший \enquote{Русский Провиданс}),
строительством Мариупольского трубопрокатного завода, вошедшего в 1958 году в
состав  завода им. Ильича. Но главное его достижение, если говорить о
мариупольском периоде его жизни и деятельности, – это возведение на Левом
берегу реки Кальмиус завода \enquote{Азовсталь}, ввод его в эксплуатацию и освоение
производства на первых порах чугуна. Люди, которые хотя бы косвенно участвовали
в строительстве крупных объектов черной металлургии, знают, сколько моральных,
физических сил при этом затрачивают руководители строительства всех уровней: от
прорабов до управляющих строительных трестов. А ведь Яков Семенович Гугель был
на самом верху пирамиды сложнейшего управленческого механизма.

% стройка азовстали
\ii{14_10_2017.stz.news.ua.mrpl_city.1.jakov_gugel_k_80_letiju_tragicheskoj_gibeli.pic.2}

К сожалению, в публикациях о Якове Гугеле не всегда говорится о том, какими
техническими и людскими ресурсами он располагал при строительстве завода
\enquote{Азовсталь}. Чтобы ответить на этот вопрос, приведем отрывок из книги \enquote{30
пламенных лет} (Донецк, Донбасс, 1964): \enquote{В самом уголке левобережья, омываемом
рекой и морем, первая и последняя горсти земли из котлована под домну
выбрасывались обычной лопатой. Сотни лопат ежеминутно вонзались в нетронутый
грунт. Сотни одноконных подвод передвигались по левобережью, перевозя к отвалам
вынутый грунт. За день сообща едва выполняли такой объем, с которым современный
экскаватор легко управится за несколько часов}. Рабочих-землекопов называли
грабарями. В цитируемой книге о них сказано, что это были главным образом
крестьяне-единоличники, прибывавшие на стройку ради приработка. На самом же
деле это были раскулаченные хлеборобы из близлежащих сел. Не случайно же
появились близ стройки так называемые грабарские поселки. Грабарей не хватало.
На стройке было всего четыре грузовых автомобиля. Чтобы ускорить работы по
рытью котлована, окружком партии обязал ряд городских организаций  выделить на
строительство доменной печи пятьдесят одноконных подвод.

Это задание, конечно, было выполнено.

Какие грузоподъемные механизмы применялись на строительстве \enquote{Азов\hyp{}стали}?  В уже
упоминавшейся книге сказано, что в цехе стальных конструкций, где решался успех
стройки, работы велись под открытым небом, чаще всего вручную.  Кранов не было,
и на подноске металла трудилось сто двадцать рабочих. Около семисот котельщиков
вручную  клепали конструкции. Сварки не было и в помине. Как выдающееся событие
обозначили приобретенный в Германии вантовый стреловой кран грузоподъемностью
десять тонн. К средствам механизации того времени можно отнести одноколесные
ручные тачки. При бетонировании фундаментов по узкому настилу вереница рабочих
катила перед собой одноколесные тачки, наполненные цементным раствором.
Содержимое тачек вываливали в котлован. А там, внизу, люди, взявшись за плечи
по трое, по четверо, трамбовали ногами колышущуюся массу.

Кадры. На стройке широко использовался труд заключенных, среди которых были
иногда и люди с высшим и средним техническим образованием. Для заключенных были
построены бараки на Правом берегу. На стройку было привлечено из городских
предприятий и учреждений, а также их окрестных сел полторы тысячи комсомольцев.
Устраивались массовые выходы на стройку жителей города в нерабочие для них дни.
Инженерно-технический состав формировался из молодых выпускников технических
вузов. Кстати, многие из них, пройдя закалку на стройке, стали прекрасными
специалистами и руководителями. На летние каникулы приезжали  студенты,
обучающиеся в Харькове, Киеве, Ростове-на-Дону, Новочеркасске, чтобы
подзаработать.

И в этих неимоверно сложных условиях Яков Семенович и его соратники сделали все
возможное и невозможное, чтобы 12 августа 1933 года доменная печь № 1 выдала
первую плавку чугуна, и завод \enquote{Азовсталь} вступил в число действующих
предприятий страны. 

Приходилось слышать, что именно Гугель разместил \enquote{Азовсталь} без учета розы
ветров. Но решение  вопроса, где размещать предприятие, - не в компетенции
директора завода. Тем более что Яков Семенович прибыл на стройку \enquote{Азовстали},
когда там уже шли работы по сооружению будущего гиганта индустрии. Среди
некоторой части молодежи существует мнение, мол, зачем надо было развивать
индустрию в Мариуполе, здесь надо было устраивать курорт. Может быть.
Встречаются публикации, где говорится, что Яков Семенович будто разделял
взгляды троцкистской оппозиции, за что и так жестоко пострадал. Вряд ли. Он был
увлечен конкретным делом, где с блеском раскрылись его способности
организатора. Да и было ли у него свободное время, чтобы разбираться во
внутрипартийных дрязгах. Хватало ли у него времени на сон.

В этой заметке нет биографических данных Якова Семеновича Гугеля. Какой смысл
переписывать то, о чем уже написано, написано толково и на основании
документов. А еще проще – набрать \enquote{Гугель Яков Семенович}.
