% vim: keymap=russian-jcukenwin
%%beginhead 
 
%%file 07_12_2020.news.ua.obozrevatel.raguckaja_lilia.1.maidan2
%%parent 07_12_2020
 
%%url https://www.obozrevatel.com/society/poetessa-majdana-vsyu-romantiku-i-nashu-naivnost-smyilo-krovyu-pervogo-pogibshego.htm
 
%%author Рагуцкая, Лилия
%%author_id raguckaja_lilia
%%author_url 
 
%%tags 
%%title Поэтесса Майдана: всю романтику и нашу наивность смыло кровью первого погибшего
 
%%endhead 
 
\subsection{Поэтесса Майдана: всю романтику и нашу наивность смыло кровью первого погибшего}
\label{sec:07_12_2020.news.ua.obozrevatel.raguckaja_lilia.1.maidan2}
\Purl{https://www.obozrevatel.com/society/poetessa-majdana-vsyu-romantiku-i-nashu-naivnost-smyilo-krovyu-pervogo-pogibshego.htm}
\ifcmt
	author_begin
   author_id raguckaja_lilia
	author_end
\fi

\ifcmt
pic https://i.obozrevatel.com/news/2020/12/7/filestoragetemp-62.jpg?size=972x462
\fi

\index[rus]{Майдан!Второй}

Когда началась Революция Достоинства, молодой поэтессе Злате-Зоряне Паламарчук
было чуть больше 20 лет. Она изучала журналистику в одном из киевских ВУЗов и
писала нежные стихи о любви.

Злата-Зоряна была на Майдане до самого конца. Революция Достоинства изменила и
ее саму, и ее поэзию, девушка начала писать о Революции, а впоследствии – и о
войне. Именно на слова молодой поэтессы были созданы песни \enquote{Небесна
Сотня} (группа \enquote{TaRuta}), \enquote{На відблисках мого меча} (группа
\enquote{Тінь Сонця}), \enquote{Час ітиме як йшов} (группа \enquote{Тобі
здалося}), \enquote{Дай мені так} (группа \enquote{Колір Ночі}) и другие.

Воспоминаниями о Майдане, его феномене и ярких моментах, а также своей
собственной Революции, Злата-Зоряна рассказала OBOZREVATEL.

– Злата-Зоряна, когда вы впервые пришли на Майдан?

– В первую ночь. Тогда, когда Мустафа Найем пригласил всех на встречу на
Майдане. Когда я приехала, людей было совсем мало, но они постоянно приезжали и
приезжали... Кажется, уже в первую ночь была машина озвучки, выступали
политики... Хотя, возможно, я ошибаюсь, и политиков еще не было – сейчас очень
трудно хронология событий вспоминается. Но и первая ночь, когда все начиналось,
врезалась в память.

Следующее яркое воспоминание – когда собрался первый миллион. Очень четко
помню, как Святослав Вакарчук давал свой концерт, когда включали все телефоны –
и эти кадры облетели все мировые СМИ.

\ifcmt
pic https://i.obozrevatel.com/gallery/2020/12/7/1zvnm200cd1y1vbuc1xzeqg.jpeg
pic https://i.obozrevatel.com/gallery/2020/12/7/image.jpg
\fi

– Чем вы занимались на Майдане?

– Мне посчастливилось быть среди людей, которые регистрировали прибывших на
Майдан. Это было 26 декабря, у нас открылась палатка, одна из трех, кажется,
где мы выписывали такие \enquote{удостоверения майдановца}. Люди приходили в те
палатки, заполняли символическую анкету и получали бумажное удостоверение
участника Майдана. Мне потом рассказывали, что некоторые перевозчики в западных
областях благодаря этому символическому удостоверению давали скидку на проезд в
Киев или вообще не брали денег за проезд.

\ifcmt
pic https://i.obozrevatel.com/gallery/2020/12/7/1268324054859257790509557062316644763234967n.jpg
pic https://i.obozrevatel.com/gallery/2020/12/7/1266256281054177125022347453507272940138460n.jpg
\fi

В первый день СМИ снимали очереди огромные, которые стояли у этой палатки.
Тогда к символическим удостоверениям мы добавляли еще и небольшие значки с
тризубом – на память. Кто-то нам их принес. И у меня есть фото, где я цепляю
тот значок совсем маленькому ребенку, которому еще и года не было... Это же
сейчас тому малышу уже лет 8 исполнилось...

\ifcmt
pic https://i.obozrevatel.com/gallery/2020/12/7/12650424628176393485081142047087738236869372n.jpg
pic https://i.obozrevatel.com/gallery/2020/12/7/1265204691784583972818644958793539716043238n.jpg
\fi

Люди были очень разные. Интеллигенты, студенты... Это не выдумка, это не миф:
там были настоящие люди, которые ехали, потому что действительно хотели
свергнуть режим Януковича, действительно хотели не в Россию, они хотели в
Европу. Эта вот возможность регистрировать людей мне показала Майдан с
внутренней стороны. И это было невероятно.

А потом произошел перелом. 22 января, в День соборности. Тогда на мосту Патона
проводили символическую акцию – объединение левого и правого берега ко Дню
соборности. И я впервые туда поехала. Там было о-о-очень холодно. Когда
вернулась оттуда в нашу палатку регистрации – узнала о первой смерти на
Майдане: убили Сергея Нигояна.

\ifcmt
pic https://i.obozrevatel.com/gallery/2020/12/7/sergejnigoyan-cropped.jpg
\fi

Это все изменило. Всю ту романтику, всю ту наивность, с которой мы шли на
мирный протест, смыло первой кровью. Ты понимаешь, что рискуешь потерей жизни
даже. Все стало слишком серьезно. В моем понимании, это был билет в один конец.
Вот мы идем к чему-то – и обратно мы уже не вернемся.

18 февраля, когда произошел массовый расстрел Небесной сотни, все началось с
наступления \enquote{Беркута}. Они пошли в атаку с Крепостного переулка. Мы с
майдановцами там были. Никто не понимал и не знал, что будет такой приказ, вот
сейчас стена \enquote{Беркута}, которая просто стоит, двинется на людей, они начнут
стрелять...

И вот мы стояли в этом Крепостном переулке, близко к \enquote{беркутовцам}, когда все
началось.

\ifcmt
pic https://i.obozrevatel.com/gallery/2020/12/7/1266180358497326458363476276626339936935337n.jpg
\fi

Я помню, как люди просто начали бежать. Помню, как побежала вместе со всеми.

Это просто было как в каком-то фильме. Ты психологически к этому никак не
готова... Тебе двадцать с небольшим хвостиком лет – и ты никак не готова к
тому, что сейчас по тебе начнут стрелять... Что тебя будут загонять, как
животное просто, – и стрелять по тебе...

Люди бежали с Крепостного переулка в сторону Арсенальной. Это такая широкая
очень дорога, и \enquote{Беркуту} было удобно прицельно стрелять, попадать в людей...
Расстрел начался с Крепостного переулка, с этого наступления... Тогда погибло 8
человек.

Мне удалось выбраться оттуда невредимой. А вечером подожгли Дом профсоюзов...
Было как-то не до того, чтобы думать, куда мои вещи подевались, где моя сумка,
в которой лежал диктофон, блокнот, много бутылочек воды, документы... У меня
только телефон в руках был. А сумка, как оказалось, осталась в машине озвучки,
стоявшей в Крепостном переулке. Ее затем разбили, сравняли с землей...

А тогда как раз уже приняли эти \enquote{драконовские законы} об участниках революции,
которые позволяли приходить к людям в квартиры. Была уверена, что ко мне точно
придут. Ведь документы в той машине остались... И я помню тот вечер, когда
вернулась в квартиру – я начала убирать дома, чтобы было чисто...

– Готовились к приходу чужаков?

– Да. Я очень перепугалась. А потом мне позвонили ведущие со сцены, сказали,
что кто-то принес мои документы к сцене. Те, кто меня знал, даже подумали, что
я погибла или сильно ранена – иначе от чего бы еще документы мои кто-то нашел и
передал.

Так документы вернулись ко мне. А сумка, вещи, диктофон – все это пошло по
людям где-то там.

Интересная история случилась с моим блокнотом. Он для меня был очень важен. Для
каждого его ежедневник – это личное. А я в свой еще и стихи записывала... И
можете себе представить, что еще в соцсети Вконтакте через некоторое время мне
пишет подросток-девочка, говорит \enquote{у меня есть ваш блокнот, и я могу вам его
переслать. Мой отец был на Автомайдане, и у него ваш блокнот, я могу вам его
переслать}. Я прочитала, думаю – что-то не то... Перечитываю снова, а там не
\enquote{на Автомайдане}, а \enquote{на Антимайдане}. И девочка пишет из Донецка. И
прикладывает фото моего блокнота...

\ifcmt
pic https://i.obozrevatel.com/gallery/2020/12/7/1268130143954866984716851641776003414548827n.jpg
\fi

– Вернула?

– Я написала ей, что это мой блокнот. А дальше она ограничила доступ к своему
профилю. Может, родителям проговорилась, что мне написала, может, ей запретили.
Но я поняла, что наши вещи оказались у участников Антимайдана.

Но на этом эпопея с блокнотом не закончилась. Я сейчас не пользуюсь Вконтакте,
но профиль там остался. И как-то я зашла и прочитала сообщение, что мой блокнот
теперь где-то в музее, в оккупированном Крыму.

Мне эта история очень неприятна, я не уверена, что там люди могут вообще понять
нас. И когда ты осознаешь, что твои вещи, твой блокнот, твои личные мысли
оказались в чужих руках, когда ты не знаешь, что это за музей, в каких целях
твой блокнот там читается... Такое впечатление, что кто-то копается в твоем
белье. Это очень неприятно. Но это все – такие мелочи по сравнению с тем, что
переживают люди, которые там потеряли своих родных, близких, любимых.

То есть моя история – совсем обычная.

– Революция Достоинства очень вас изменила?

– Чрезвычайно. Она разделила жизнь на \enquote{до} и \enquote{после}. До Революции я была
сознательной, но не настолько. Революция Достоинства обострила чувство
справедливости, понимание, что за свое достоинство и свободу надо бороться, что
если на тебя давят, ты должен сопротивляться. Она меня сделала сильнее. Открыла
новые грани моего творчества: я начала писать о Майдане. Далее, началась
аннексия Крыма и война, вся эта поэзия стала основой для написания новой, уже
другой, посвященной войне.

Я часто думаю о том, что нам не дали в полной мере прожить и ощутить Небесную
сотню, нам не дали оценить ее жертву, у нас очень быстро начались другие бурные
события. У нас отобрали Крым. Россия вторглась на Донбасс. Украина начала нести
огромные жертвы.

И эта Небесная сотня – ее не отпели, она как-то осталась, по-моему, обделена
вниманием, потому что сразу пришла война.

Поэтому я очень рада, например, что у меня есть поэзия, посвященная Небесной
Сотне. Что в общем люди начали и тогда, и сейчас об этом писать, что это не
остается без внимания, что мы начинаем об этом больше говорить – и помнить. Вот
просто подумайте: семь лет назад с нами случилась Революция Достоинства. Для
меня, честно, это как семь дней, я не считаю, что много уже времени прошло,
очень уж быстро эти годы пролетели. Но вот, например, детям, которые сейчас уже
совершеннолетние, тогда было по 11 лет. Они Революцию Достоинства могут и не
помнить, или помнить так, как я – Оранжевую революцию, просто с телевизоров.
Они могут не знать ее истинную историю.

\ifcmt
pic https://i.obozrevatel.com/gallery/2020/12/7/image-1.jpg
\fi

– Они и не знают в большинстве, это правда.

– Поэтому очень важно, чтобы происходили встречи в школах с участниками. Я
знаю, что иногда школьники сами даже просят: \enquote{Расскажите нам о Революции
Достоинства}. Потому что если не мы начнем об этом рассказывать, те, кто в ней
принимал непосредственное участие, то за нас это начнут делать другие,
возможно, не участники, возможно, даже те, кто какие-то совсем другие выводы
сделал из тех событий. Ведь даже сразу после Революции Достоинства, когда я
общалась с лицеистами, выступала перед студентами, читала им свою поэзию, – по
завершении встречи ко мне подходили отдельные преподаватели и спрашивали: \enquote{Ну и
скажите сейчас, а надо нам была эта Революция Достоинства, если в конце концов
она привела к войне?}...

И таких мифов – множество. Даже на таких ранних этапах, когда, казалось бы,
можно было бы пообщаться с очевидцами, обратиться к первоисточникам – люди не
разобрались. И сознательно или нет, распространяют эти мифы, будто война
началась из-за Революции Достоинства. Что аннексия Крыма стала прямым
следствием Майдана. Хотя мы знаем, что сценарий аннексии Крыма разрабатывался
задолго до того известного поста Мустафы Найема. И Россия просто искала повод,
чтобы начать его реализовывать.

Нет, никакая Небесная Сотня не виновата в том, что Путин всю жизнь смотрел на
Крым, который, видимо, нужен России был, потому что давал выход через Черное
море. И мифы, которые породила российская пропаганда тогда, в самом начале, с
годами только растут и умножаются. Вот почему важно рассказывать правдивую
историю Майдана. А правдивая история – это история каждого, кто там был.
Революция Достоинства – это история каждого, кто там был. Так складывается
общая история – из тысяч личных историй. И, к примеру, я никогда не посмотрю на
Майдан другими глазами, я знаю этих участников, я знаю, что их вело туда. Я
знаю, как они этим проникались. Я никогда не скажу, что это что-то было
фейковое, что это кто-то создал специально или что на Майдане платили людям,
которые туда шли.

Ясно, что один человек не может увидеть все. Но то, что видела я, позволяет мне
категорически утверждать: тем участникам, которых знала я, никогда никто денег
не платил. Они (в частности, мои сверстники) просто хотели выйти из-под режима
Януковича – и все.

– На Майдан вышло очень много людей. Но еще больше, в частности, в восточных и
южных регионах, – его не поддержали, не поняли. Как думаете, почему?

– Одной из первых палаток, которые появились на Майдане, была палатка крымчан.
Поэтому я так не могу сказать, что они не поддерживали. Находились те, кто
поддерживал. Но у нас всех разная историческая память – и эта разность нас
разделяет. Мы, как мне кажется, живем по разным ценностями. "Бандеровцев" вон
уважают на Западной Украине, а на востоке и в Крыму ими детей и взрослых
пугают. Возможно, это говорит историческая память, которая является одной из
основ наших ценностей.

Она нас больше всего и разъединяет.

Хотя вполне возможно, больше всего нас разделяет даже не история, а незнание
подлинных исторических событий. Это порождает массу искаженных \enquote{историй}, часть
из которых, как известный миф о русскоязычном младенце, съеденном кровожадными
бандеровцами, искусно подогревается пропагандой.

Так рождаются и крепнут мифы, которые нас разъединяют.

– Но речь шла не о прошлом, а о будущем. И, казалось бы, людям свойственно
стремиться к лучшей жизни. Считаете ли вы, что страх перед прошлым, перед
историей, перед мифами пропагандистскими для части украинского общества
оказался сильнее желания лучшего будущего?

– Вполне возможно. А может, сыграло роль еще и то, что за президента Януковича
голосовал все-таки преимущественно восток. И восстать против его режима во
время Революции Достоинства для жителей восточных регионов могло означать
признание своей ошибки, своего поражения. А вдобавок еще и экономические связи
восточных регионов с Россией сказались, территориальная близость к РФ... Трудно
сказать. Я не специалист, поэтому могу разве предполагать.

– Вы очень хорошо сказали о том, что история Майдана – это тысячи конкретных
историй конкретных людей. Запомнились ли вам такие истории? Знали ли кого-то из
тех, кто впоследствии погиб и стал Героем Небесной Сотни?

– Я не знала лично ни одного Героя Небесной Сотни. Но каждый раз, когда
появлялись фотографии убитых, когда приносили их вещи, среди которых было и
\enquote{удостоверение майдановца}, – я все думала: а может быть, этот человек был у
меня. Может, мы даже перекинулись парой слов. Всматривалась, не мой ли почерк
на том удостоверении...

Самое яркое и ужасное одновременно мое воспоминание о Майдане – это гробы с
Героями, которые на руках несли к сцене. \enquote{Пливе кача}, которую мне до сих пор
невыносимо трудно слушать – и гробы, плывущие по морю людей к сцене... Мы
сейчас привыкли к войне, к потерям, к статистике о погибших. Слушая новости, мы
понимаем, что кто-то погиб, и сочувствуем. Но эти ощущения трудно сравнить с
ощущениями от тех смертей. Когда ты, не в последнюю очередь из-за своей
наивности, молодости, никогда в жизни не видел таких столкновений и такого
количества жертв. А здесь вдруг начинает происходить с тобой, на твоих глазах.
Ты пропускаешь это через себя, понимаешь, что ты – часть этого, поддерживаешь
это движение... И, конечно, воспринимаешь это все как свою потерю, каждого
погибшего там воспринимаешь как собственную потерю. Даже если не знал его
лично.

Очень жаль, что сейчас не ценится эта жертва. Жаль, что люди не задумываются,
что никто не хотел умирать в центре Киева. Никто не хотел умирать в принципе.
Люди вышли на мирную акцию. И попали в ситуацию, когда у них отняли жизни.

Сейчас вспомнила еще один яркий момент. Я тогда жила недалеко от станции метро
\enquote{Героев Днепра}. Как-то возвращалась домой с Майдана. И на плечах у меня, как
всегда, был флаг. Он согревал и душу, и тело, когда шел дождь или мокрый
снег... Этот флаг всегда был со мной. Я к нему привыкла... Когда он был со
мной, было как-то спокойнее, что ли...

И вот, помню, что уже выхожу на станции метро своей – и ко мне подходит
женщина. Красивая такая, видно, что едет с работы, где-то из офиса... Она
подходит, сжимает краешек моего флага в кулак – и говорит: \enquote{Снимите с себя это
барахло и езжайте в свою Галичину}. Я не помню, что я ей ответила, но это
\enquote{Снимите с себя это барахло и езжайте в свою Галичину} запомнила навсегда. Были
даже такие мнения наших граждан, отношение их к Майдану.

– А о людях? Возможно, какие-то люди вам запомнились? Какие-то поступки,
которые вас удивили?

– Почему-то запомнилось, как нам в палатку, где мы регистрировали людей,
притащили откуда-то буржуйку – хотя нам было тепло, там постоянно была куча
людей. А еще, помню, приехал какой-то человек, из гостиницы \enquote{Рэдиссон},
кажется, принес очень вкусные обеды и ужины. Настоящие, пятизвездочные, очень
хорошо упакованные и оформленные. Это было очень странно видеть, и приятно.

А вообще еды было много всегда, постоянно кто-то что-то приносил и кормил тех,
кто там был на Майдане. Это помню.

Помню, как в аптеку заехала, это уже после расстрела, уже где-то конец
февраля... Заехала покупать различные лекарства, все, на что хватало денег. И
заходят еще какие-то люди, видят, что я это все пакую и \enquote{мы также едем, мы
также везем}, и я просто сажусь в машину к незнакомым людям и мы просто едем на
Майдан, чтобы завезти эти лекарства. Это к Главпочтамту, к месту, где собирали
все медикаменты... Просто так: вы - на Майдан, я – на Майдан, и мы – на Майдан,
поехали – поехали!

На самом деле, – это ужасные события, ужасна жертва людей в этих событиях. Но
это такие события, когда ты встречаешь очень много хороших людей. Большая
концентрация хороших людей в одном месте. Это было невероятное ощущение.

И я считаю, что каждая из предыдущих революций – Революция на граните,
Оранжевая революция – они подпитывали то, из чего родился затем Майдан. Майдан
состоялся еще с памятью тех революций. И думаю, что это очень важно – помня о
Майдане, не забывать и о событиях и людях, благодаря которым он в конце концов
и произошел.
