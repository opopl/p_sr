% vim: keymap=russian-jcukenwin
%%beginhead 
 
%%file 15_12_2020.sites.biospectrumasia.1.sputnik_v_efficacy
%%parent 15_12_2020
 
%%url https://www.biospectrumasia.com/news/90/17294/sputnik-v-vaccine-shows-91-4-efficacy-in-phase-3-clinical-trials.html
 
%%author 
%%author_id 
%%author_url 
 
%%tags 
%%title Sputnik V vaccine shows 91.4% efficacy in phase 3 clinical trials
 
%%endhead 
 
\subsection{Sputnik V vaccine shows 91.4\% efficacy in phase 3 clinical trials}
\label{sec:15_12_2020.sites.biospectrumasia.1.sputnik_v_efficacy}
\Purl{https://www.biospectrumasia.com/news/90/17294/sputnik-v-vaccine-shows-91-4-efficacy-in-phase-3-clinical-trials.html}

The cost of one dose of lyophilized (dry) form of the vaccine is less than \$10
for international markets

\ifcmt
  pic https://www.biospectrumasia.com/uploads/articles/close_up_doctor_holding_covid_vaccine_syringe_23_2148747822-17294.jpg
\fi

The National Research Center for Epidemiology and Microbiology named after N.F.
Gamaleya of the Ministry of Health of the Russian Federation (Gamaleya Center)
and the Russian Direct Investment Fund (RDIF, Russia’s sovereign wealth fund),
on 14 Dec 2020 announced the efficacy of over 90\% of the Russian Sputnik V
vaccine as demonstrated by the final control point data analysis of the largest
double-blind, randomized, placebo-controlled Phase III post-registration
clinical trials of the Sputnik V vaccine against novel coronavirus infection in
Russia’s history. Sputnik V is the world’s first registered vaccine against
coronavirus based on a well-studied human adenoviral vectors platform.

The data analysis at the final control point of the trials demonstrated a
91.4\% efficacy rate. By now over 26,000 volunteers have been vaccinated as
part of double-blind, randomized, placebo-controlled Phase III
post-registration clinical trials of Sputnik V in Russia.

As of December 14, no unexpected adverse events were identified as part of the
research. Some of those vaccinated had short-term minor adverse events such as
pain at the injection point and flu-like symptoms including fever, weakness,
fatigue, and headache. Health conditions of the participants will be monitored
for at least 6 months after receiving the first immunization.

During the clinical trials, the safety of the vaccine is constantly being
monitored; information is analyzed by the Independent Monitoring Committee
comprising leading Russian scientists. Collection, quality control and data
processing is conducted in line with ICH GCP standards and involves the active
participation of Moscow’s Health Department and Crocus Medical, the contract
research organization (CRO).

Sputnik V has a unique set of parameters making it one of the most competitive
vaccines globally. The efficacy rate is over 90\% and the vaccine is based on a
safe and proven platform of human adenoviral vectors. The uniqueness of the
Russian vaccine lies in the use of two different human adenoviral vectors as a
delivery mechanism of the outer coat genetic material of coronavirus to human
body. This approach provides for the creation of a stronger and long-term
immunity as compared to vaccines, using one and the same component for both
doses.

Cost of one dose is less than \$10 for international markets while the
production of the lyophilized (dry) form of the vaccine, which is stored at a
temperature of +2 to +8 degrees Celsius, enables easier distribution of the
vaccine in international markets.

Requests for vaccination of more than 1.2 billion people (2.4 billion doses)
with the Sputnik V vaccine came from more than 50 countries. The vaccine
supplies for the global market will be produced by RDIF’s international
partners in India, Brazil, China, South Korea and other countries.

On August 11, the Sputnik V vaccine developed by the Gamaleya Center was
registered by Russia’s Health Ministry and became the world’s first registered
vaccine against COVID-19.


