% vim: keymap=russian-jcukenwin
%%beginhead 
 
%%file 24_09_2019.stz.news.ua.mrpl_city.1.kiseljova_nina_mykolaivna_mrpl_moje_zhyttja
%%parent 24_09_2019
 
%%url https://mrpl.city/blogs/view/kiselova-nina-mikolaivna-mariupolmoe-zhittya
 
%%author_id demidko_olga.mariupol,news.ua.mrpl_city
%%date 
 
%%tags 
%%title Кисельова Ніна Миколаївна: "Маріуполь - моє життя!"
 
%%endhead 
 
\subsection{Кисельова Ніна Миколаївна: \enquote{Маріуполь - моє життя!}}
\label{sec:24_09_2019.stz.news.ua.mrpl_city.1.kiseljova_nina_mykolaivna_mrpl_moje_zhyttja}
 
\Purl{https://mrpl.city/blogs/view/kiselova-nina-mikolaivna-mariupolmoe-zhittya}
\ifcmt
 author_begin
   author_id demidko_olga.mariupol,news.ua.mrpl_city
 author_end
\fi

Наша наступна героїня присвятила своє життя благородній і важливій місії.
Самовіддана багаторічна праця \textbf{Ніни Миколаївни Кисельової} на користь Маріуполя,
що вимагала особливої самовіддачі і відповідальності, сьогодні заслуговує
найглибшої поваги і вдячності.

\ii{24_09_2019.stz.news.ua.mrpl_city.1.kiseljova_nina_mykolaivna_mrpl_moje_zhyttja.pic.1}

Маленька Ніна разом з мамою приїхала в Маріуполь, коли їй було всього 10 років.
Сталося це у 1954 році. Місто сподобалося одразу. Але спочатку жити було
складно. Довгий час доводилося змінювати місце проживання. Лише у 1970 році у
Ніни та її мами з'явилося власне помешкання. Дитинство було нелегким, проте
змалечку мама прищеплювала доньці, що жити потрібно по совісті і неважливо
скільки у тебе грошей, головне, щоб совість була чистою. Водночас Ніна
Миколаївна завжди керувалася внутрішнім чуттям: \emph{\enquote{ти повинна це зробити!}}.
Можливо, саме це і привело її до настільки високої посади. За освітою вчитель
російської мови та літератури, вона і уявити на могла, що буде вирішувати
стільки важливих питань як для Маріуполя, так і України загалом.

\textbf{Читайте також:} \emph{Художественные чтения, презентации книг, интерактивное пространство: в Мариуполе пройдет масштабный книжный фестиваль}%
\footnote{\enquote{Ореn Book} в Мариуполе: художественные чтения, презентации, интерактивное пространство, mrpl.city, 24.09.2019, \par%
\url{https://mrpl.city/news/view/hudozhestvennye-chteniya-prezentatsii-knig-interaktivnoe-prostranstvo-v-mariupole-projdet-masshtabnyj-knizhnyj-festival}}

Займаючи посаду секретаря міськкому з ідеології і будучи членом Центральної
виборчої комісії, саме Ніна Миколаївна просувала на рівні Маріуполя, Донецька,
Києва ініціативу щодо відкриття в нашому місті гуманітарного коледжу, з якого
згодом виріс єдиний класичний (багатопрофільний) університет у Донецькій
області – \textbf{Маріупольський державний університет}. На той час у місті не
було вчителів історії та української мови. А це був час національного
відродження.  Двадцять років Ніна Миколаївна слухала про те, що дівчаткам
Маріуполя ніде вчитися і, коли з'явилася можливість це виправити, вона діяла
дуже рішуче.  Переконувала місцевих партійців, що будинок політосвіти потрібно
віддати студентам. Тоді вона наголошувала в Києві: \emph{\enquote{щоб цей птах
продовжував свій політ, потрібно віддати будівлю політосвіти на баланс
коледжу}}. Разом з тим зустрічалася з ректором Донецького державного
університету \textbf{Володимиром Павловичем Шевченком}. Зверталася за допомогою
до колег по ЦВК, і ті організували зустріч з \textbf{Ігорем Рафаїловичем
Юхновським}, головою комітету Верховної Ради України з питань науки і освіти.
Вона наполегливо стукала у всі двері і все ж таки досягла поставленої мети.
Незважаючи на те, що спочатку було неймовірно важко працювати (було всього 11
викладачів, які не мали навчальних планів і були зовсім не підготовлені до
нової професії; аудиторії були не пристосовані до навчання), проте початок був
покладений. У 2005 році за значний особистий внесок у розвиток вищої освіти
Ніна Миколаївна була нагороджена заохочувальною відзнакою Міністерства освіти і
науки України - \textbf{нагрудним знаком Петра Могили}.

Ніну Миколаївну завжди відрізняли витончений смак і вимогливість до себе,
вишукані манери справжньої Леді і доброзичливість. Її однаково поважали всі:
вчені і артисти, служителі церкви різних конфесій та лідери різноманітних
політичних об'єднань, студенти та пенсіонери, депутати і міліціонери,
прихильники і опоненти. У 1998 року написала заяву про звільнення за власним
бажанням з посади керуючого справами виконкому міської ради. Після цієї події
багато маріупольців були розгублені, адже саме двері Ніни Миколаївни були
завжди відчинені... Вона завжди, незважаючи на власні обставини, негаразди на
роботі, намагалася вислухати кожного, дати мудру пораду, на справі допомогти у
вирішенні проблемного питання.

Однак жінка, що звикла віддаватися роботі, не уявляла, як можна тепер жити без
самовідданої праці. Дуже вчасно ректор ДонНУ запропонував очолити
Маріупольський навчально-консультацій\hyp{}ний центр Донецького національного
університету для підготовки фа\hyp{}хівців в області економіки. З усією самовіддачею
Ніна Миколаївна занурилася в цю справу. За визнанням декана економічного
факультету, доктора економічних наук, професора \textbf{Геннадія Олександровича
Черниченка}, Маріупольський навчально- консультаційний центр став найкращим
центром ДонНУ. Через нього пройшли багато начальників цехів, заступники
директорів бюджетоутворюючих підприємств міста, представники малого та
середнього бізнесу.

Наша героїня розуміє, що без підтримки мами і чоловіка багато чого не вдалося б
реалізувати. Мама підтримувала у всьому. Всю господарську роботу брала на себе.
Чоловік – \textbf{Михайло Данилович} – справжній ерудит, завжди захоплювався
літературою, може цитувати цілі фрагменти з книг.

Наразі Ніна Миколаївна знаходиться на заслуженому відпочинку. Вона розуміє, що
Маріуполь для неї – це ціле життя, і не уявляє себе без рідного міста.
Найбільше подобається гуляти по набережній, де легше думати. Також дуже любить
приходити до свого дітища – МДУ.

\textbf{Читайте також:} \emph{Мариупольская студентка создала игру о событиях 1917 года в родном городе}%
\footnote{Мариупольская студентка создала игру о событиях 1917 года в родном городе, Богдан Коваленко, mrpl.city, 23.09.2019, \par%
\url{https://mrpl.city/news/view/mariupolskaya-studentka-sozdala-igru-o-sobytiyah-1917-goda-v-rodnom-gorode-foto}
}

\textbf{Улюблена книга:} всі твори Валентина Савовича Пікуля.

\textbf{Улюблені фільми:} старі англійські та італійські фільми.

\textbf{Порада маріупольцям:} 

\begin{quote}
\em\enquote{Кожному треба щось робити для міста. Пам'ятати, що ти маріуполець, і живеш в
особливому місті. Саме по собі нічого не відбудеться. Потрібно виховувати в
собі самодисципліну і працювати на користь рідного Маріуполя!}.
\end{quote}

\emph{Автор фото - Валерій Комаров}.
