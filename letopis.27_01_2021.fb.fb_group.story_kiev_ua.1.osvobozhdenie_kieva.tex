% vim: keymap=russian-jcukenwin
%%beginhead 
 
%%file 27_01_2021.fb.fb_group.story_kiev_ua.1.osvobozhdenie_kieva
%%parent 27_01_2021
 
%%url https://www.facebook.com/groups/story.kiev.ua/posts/1585575148305947
 
%%author_id fb_group.story_kiev_ua,kamencova_maria.kiev
%%date 
 
%%tags 1943,1943.kiev.osvobozhdenie,istoria,kiev,kiev.okupacia.1941,osvobozhdenie,vov
%%title ОСВОБОЖДЕНИЕ КИЕВА
 
%%endhead 
 
\subsection{ОСВОБОЖДЕНИЕ КИЕВА}
\label{sec:27_01_2021.fb.fb_group.story_kiev_ua.1.osvobozhdenie_kieva}
 
\Purl{https://www.facebook.com/groups/story.kiev.ua/posts/1585575148305947}
\ifcmt
 author_begin
   author_id fb_group.story_kiev_ua,kamencova_maria.kiev
 author_end
\fi

\textbf{ОСВОБОЖДЕНИЕ КИЕВА}

Благодарю всех-всех-всех, кто откликнулся на мой рассказ «Кукла-цыганка»!

Вот еще воспоминания моей мамы Нины Николаевны Каменцовой, записанные, когда ей
было 90 лет.

Киев, осень 1943 года.

Последние недели фашистской оккупации.

Когда немцам стало ясно, что город не удержать, они применили тактику
«выжженной земли». Все, что можно было уничтожить, должно было быть взорвано,
ценности – вывезены, все жители – убиты.

Люди стали уходить в малонаселенные места, прятаться в подвалах, пещерах,
канализационных трубах...

Специальные отряды немцев тщательно выслеживали эти убежища и, найдя,
беспощадно расстреливали всех.

В Кмитовом яру, среди оврагов, уцелел небольшой заброшенный хутор. Туда сбилось
довольно много беженцев – человек 25, в основном женщины и дети.

Верховодил там бывший военный, ветеран еще Первой мировой войны, без ноги, на
протезе.

Он установил железную дисциплину воинской части: охрана периметра, сигнализация
о приближении противника, отряд снабжения, повара (готовили в основном
гороховую похлебку – без хлеба, зато горячую!), уборщики территории (надо было
все время придавать ей заброшенный, нежилой вид). Часто устраивал учения.

Эта организованность всех и спасла.

Когда появился карательный отряд немцев, своевременно был дан сигнал, и все
успели надёжно укрыться в убежищах и замести следы.

Мама пряталась в хибарке, за полуразрушенной русской печью. Вбежала последней и
еле успела туда втиснуться. Народу набилось – не пошевельнуться! Но никто не
издавал ни звука, дышали тихо. Немцы прошли по комнатам, никого не обнаружили,
но один засунул руку з печку и нащупал мамино плечо в пальто.

\raggedcolumns
\begin{multicols}{3} % {
\setlength{\parindent}{0pt}

\ii{27_01_2021.fb.fb_group.story_kiev_ua.1.osvobozhdenie_kieva.pic.1}
\ii{27_01_2021.fb.fb_group.story_kiev_ua.1.osvobozhdenie_kieva.pic.1.cmt}

\ii{27_01_2021.fb.fb_group.story_kiev_ua.1.osvobozhdenie_kieva.pic.2}
\ii{27_01_2021.fb.fb_group.story_kiev_ua.1.osvobozhdenie_kieva.pic.2.cmt}

\ii{27_01_2021.fb.fb_group.story_kiev_ua.1.osvobozhdenie_kieva.pic.3}
\ii{27_01_2021.fb.fb_group.story_kiev_ua.1.osvobozhdenie_kieva.pic.3.cmt}

\end{multicols} % }

Она стояла, окаменев.

Немец поскреб пальцами по холодной неподвижной ткани, – решил, что это
какой-нибудь хлам или старые мешки. Убрал руку, и они вышли.

Если бы мама от испуга дернула плечом, вскрикнула или хотя бы глубоко
вздохнула, всех бы перестреляли. Но воля у нее была железная.

Прождав около часа, поняли, что опасность миновала, вышли и стали хозяйничать
по-прежнему. Каждый день посылали гонцов в город – как оно там?

Но весть принесли не они.

По тропинке, отчаянно размахивая руками, мчался незнакомый мальчишка. Полы
расстёгнутого пальто реяли у него за спиной, как крылья ангела!

Не останавливаясь, он проорал:

– Танки! На Крещатике наши танки!!! – и, не останавливаясь, побежал дальше.

Мама рассказывала:

– Как мы летели! Земли под ногами не чувствовали!

Но это были не «танки».

На Крещатике стоял всего один танк. Не похожий на неуклюжие немецкие:
небольшой, ладный, стремительный!

Все люки были закрыты.

Вокруг были одни развалины. Мёртвый город!

И вдруг оттуда стали выходить люди. По одному, по двое, по трое... Они тихо
подходили к танку. Образовалась небольшая толпа.

Тогда внутри танка что-то лязгнуло, отворился люк, и вылез танкист: чумазый, в
ребристом шлеме, замасленном комбинезоне.

Он стал и, неловко опустив натруженные руки, смотрел на людей. Они были тоже
грязные, оборванные, еле живые.

Подошли еще поближе и, бережно разворачивая в руках тряпицы, стали предлагать
ему самое драгоценное, сбереженное на крайний день: черный сухарик, кусочек
сала, пару кубиков сахара...

Танкист посмотрел на все это и заплакал.

В толпе тоже многие утирали слезы.

Так был освобождён Киев.
