% vim: keymap=russian-jcukenwin
%%beginhead 
 
%%file 13_02_2022.yz.istoricheskie_naperstki.1.ajda_putin_molodec
%%parent 13_02_2022
 
%%url https://zen.yandex.ru/media/id/5ef8896c0d13dd78e21972de/ai-da-putin-ai-da-molodec-kak-razygrana-specoperaciia-agressiia-rossii2022-620765b46e69cf54234233f4
 
%%author_id yz.istoricheskie_naperstki
%%date 
 
%%tags agressia,napadenie,putin_vladimir,rossia,ugroza,ukraina
%%title Ай да Путин, ай да... молодец! Как разыграна спецоперация «Агрессия России-2022»
 
%%endhead 
 
\subsection{Ай да Путин, ай да... молодец! Как разыграна спецоперация «Агрессия России-2022»}
\label{sec:13_02_2022.yz.istoricheskie_naperstki.1.ajda_putin_molodec}
 
\Purl{https://zen.yandex.ru/media/id/5ef8896c0d13dd78e21972de/ai-da-putin-ai-da-molodec-kak-razygrana-specoperaciia-agressiia-rossii2022-620765b46e69cf54234233f4}
\ifcmt
 author_begin
   author_id yz.istoricheskie_naperstki
 author_end
\fi

Кажется, всё встало на свои места. После оглушительной дипломатической
затрещины, прописанной Сергеем Лавровым главе МИД Соединённого Королевства Лиз
Трасс, капкан должен захлопнуться. Такого унижения перед всем честным миром не
прощают, «глухая» любительница покататься на танке вывернется из лифчика, но
воплотит в жизнь свои санкционные угрозы России. Очень хочется надеяться, во
всяком случае.

\ii{13_02_2022.yz.istoricheskie_naperstki.1.ajda_putin_molodec.pic.1}

Демонстрация уровня IQ державной дамы принадлежностью Украине Воронежской и
Ростовской областей, монголами да татарами, захватывающих Киев, оказанием
военной помощи Прибалтике через Чёрное море... это за гранью. Впервые за много
лет с огромным удовлетворением читаю английские паблики, где граждане
Королевства прозревают на глазах, осознав, кто именно руководит островной
ядерной державой...

\subsubsection{Странное поведение России...}

Крепился долго, ещё с декабря. Не ожидал, что часть коллективного политического
Запада ... столь напоминает деревяшку по имени Буратино, которого два хитрых
хищника отправили на Поле Чудес растить золотые сольдо. Всё началось три месяца
назад, когда президент России потребовал две странные декларативные вещи.
Гарантии нераспространения НАТО на Восток и более резко: недопустимость
включения в сферу Альянса — Украины.

Которую вообще-то... туда не примут даже в алкогольно-наркотическом бреду. По
столь длинному списку объективных причин, где только одного «пункта №5»
(агрессия в сторону участника Альянса — коллективный военный ответ всей
организации) достаточно, чтобы началась третья мировая из-за Крыма. Вот если
полуостров неожиданно признают российским... тогда другое дело, повод Москве
серьёзно занервничать.

То есть, вопрос Украины был приглашением к некой весёлой провокации. А
банальщина типа «расширения НАТО» — организация глухого дипломатического
тупика, чтобы любые переговоры обнулить, сделать неприемлемыми для победившего
в «холодной войне» Запада. Уверен, письменные предложения Москвы были
составлены так, что начать их обсуждать — потерять лицо, признаться в
агрессии... уже не России.

Но вопросы НАТО-Украина были иезуитски увязаны между собой. Это стало пачкой
дрожжей в дачном сортире жарким летним днём. Признаюсь, оплеуху в сторону Киева
сначала не раскусил. Зачем так обидно-то было? Ясно даже последнему диванному
эксперту: «Минск-2» не будет исполняться никогда, это смертный приговор
киевскому режиму от собственных бандеровцев. А штурмовать Донбасс дурных нэма.

\ii{13_02_2022.yz.istoricheskie_naperstki.1.ajda_putin_molodec.pic.2}

Но политики и СМИ англосаксов наживку заглотили так глубоко, что крючок
показался из... понятно какого места. Довольно грубый психологический трюк
удался, даже удивил Кремль эффектом и скоростью реакции. Как оказалось,
нынешние политические элиты стран НАТО допускают абсолютно всё.

\subsubsection{Как организуются настоящие «агрессии».}

Итак, западный истеблишмент уверовал: будет немотивированное «вторжение России
на Украину». Без всякого повода, многочисленных агрессивных дипломатических
нот, объявления жёстких санкций и отзыва признания легитимности майданной
власти. Без объявления военных сборов резервистов, отзывом дипломатов из Киева,
призыва граждан РФ покинуть Украину, закрытых пунктов пропуска, прекращение
всякого товарообмена.

Только малая часть этих мер могла лишить Запад всех запасов валерьянки, а
Украина полыхнула бы новым Майданом. Агрессия так не ведётся, даже в
информационном поле. Она разыгрывается так:

\begin{itemize} % {
\item Россия нежданно-негаданно признаёт ЛНР и ДНР, там начинается срочный референдум о независимости или вообще присоединению к «русскому брату»;
    
\item появляются российские миротворцы для «обеспечения мирного волеизъявления», гремят страшные угрозы в сторону возможных агрессоров и провокаторов;
    
\item над территорией ЛНР и ДНР объявляется бесполётная зона, демонстративно разворачиваются ЗРК вдоль всей российско-украинской границы;
    
\item все военные округа РФ поднимаются «внезапной проверкой боеготовности» от
				Калининграда до Находки, воинские эшелоны стучат колёсами в европейскую
				часть РФ;
    
\item в крупных городах Малороссии открываются «центры приёма добровольцев»,
				журналистам со вкусом дают интервью ражие русские и белорусские мужики,
				«приехавшие в отпуск по зову сердца, в память о дедах, воевавших с
				нацистами»;

\item неожиданно весь атомный подплав ВМФ России торопится на дальние морские
				учения, а Гарант (не вставая с кресла) «откошмаривает» Минобороны за
				недостаточные показатели оснащения Армии РФ «гиперзвуком»,
				«Посейдонами» и прочей баллистической радостью. Сергей Шойгу клянется
				всё исправить до весенней распутицы;

\item как только в страны-соседи России прибывают даже единичные американские
				солдаты, оттуда начинают (в большой информационной панике) вывозить
				семьи наших дипломатов, звучат настоятельные просьбы к гражданам РФ не
				посещать «потенциальную зону опасности»;

\item начинаются лихорадочные переводы туда-сюда активов российских банков и
				госкорпораций в странах Североатлантического Альянса.

\end{itemize} % }

\ii{13_02_2022.yz.istoricheskie_naperstki.1.ajda_putin_molodec.pic.3}

Ничего этого не нужно делать, просто балаболить в СМИ, намекать, допускать
оговорки, манипулировать «вбросами». Потом опровергать или вообще зловеще
молчать, пряча рвущийся наружу смех. Перечислять «признаки неминуемой агрессии»
можно бесконечно, чтобы в Совбезе ООН начался коллективный обморок под ехидные
шуточки Василия Небензи и монументальное спокойствие представителя Китая.

\subsubsection{Скучная реальность.}

Но... звенящая тишина. Кремлю не пришлось даже организовывать «утечки» с
полунамёками. Пропагандистские штампы Запада сами по себе стали причиной
глобальной истерики. Оказалось, англосаксы за долгие годы демонизации России и
Путина... сами поверили в собственные выдумки. А дзюдоист Владимир
продемонстрировал главный принцип этой борьбы — используй импульс противника
против него самого.

Запад придумал «коварного диктатора и агрессора Путина», реальный Путин на том
сыграл. Каждое пропагандистское лыко легло в нужную строку, тысячу раз
написанную в речах западных политиков, напечатанную миллионными тиражами в
газетах.

Что русские стремятся восстановить СССР в прежних границах, опять сделать
сателлитами неофитов Евросоюза из Восточной Европы, перекрыть «газовый кран».
Россия дика и агрессивна, с ней не нужно приличному человеку считаться,
политических противников она травит полонием и «Новичками» и так далее, и тому
подобное...

Ну, до-демонизировали? И сами поверили: Путин — самый опасный человек на свете,
отец углеводородной бомбы, ага. «Даже Трампа избрал», одним движением брови
Крым присоединил, Ближний Восток под себя подмял... боже, от него можно ждать
чего угодно! Так, за дюжину лет (со времён Мюнхенской речи) большая часть
западного истеблишмента поверила собственным мантрам пропаганды.

\ii{13_02_2022.yz.istoricheskie_naperstki.1.ajda_putin_molodec.pic.4}

Поэтому вежливое и недвусмысленное предложение России Западу отступиться от
Украины, не переходить границ «русского мира» — поставило англосаксов под
ружьё. Как так!? Это уже наша сфера влияния, мы никуда не уйдём. Сунули тем
самым головёшку в капкан.

Сегодня весь мир с большим юмором наблюдает, чего стоят заверения Госдепа и
Брюсселя, которые всеми конечностями отбрыкиваются от официального
присоединение Украины к западному военному блоку. Ну да, перейдя нарисованную
нашим Гарантом «красную черту», можно гарантированно нарваться на настоящий, не
вымышленный конфликт. Ловушка захлопнулась.

Владимир Путин прекрасно подыграл западной пропаганде, моё почтение. Горько
посетовав на позорное бегство Вашингтона из Афганистана, с глубоким сожалением
указав, чего на самом деле стоят «американские гарантии безопасности»
создаваемым марионеточным режимам. Киев это принял на свой счёт. Потом получил
прямой отказ «союзников» даже рассматривать возможность ввода войск НАТО для
«отражения агрессии». Всем стало ясно — ни за кого Америка не впишется.

Осенью 2021-го Гарант продолжал нагнетать обстановку, удивлённо сообщив: Запад
перегибает палку в наращивании сил и средств у границ РФ. Выдвинул требования
гарантий безопасности на этом направлении, исключения даже намека об интеграции
в НАТО Украины. Потом появились женевские «ультиматумы» с требованием письменно
на них ответить по всем пунктам.

Как и ожидалось, не очень сообразительные политики англосаксов это посчитали
оскорблением, стали всех убеждать… война скоро. А шаркающий пенсионер Байден
договорился до конкретной даты намедни — всё случится 16 февраля, старчески
забыв поднять по тревоге вооружённые силы страны. Вот так, уже в течение двух
последних месяцев Вашингтон и Брюссель занимаются собственной дискредитацией в
глазах ещё вменяемой публики.

То «земля не промерзла» для гусеничных траков агрессора, то «Путин боится
испортить Олимпиаду китайскому кормчему Си», то крупа просроченная в солдатских
рационах и так далее. Признаться, что Путин блефовал и откровенно насмехался
над демонизацией себя и России... это позор, полное обнуление всей многолетней
лжи и пропаганды против Москвы.

Ждать нападения? Только и остаётся, дальше выставляя себя на посмешище с каждым
новым днём. А Гарант не останавливается, всё более толсто тролит. То
венгерскому Орбану лекцию прочитает об энергетической безопасности Европ и как
Вашингтон до нитки оберёт «зелёную энергетику» за поставляемый сжиженный газ.
То Макрона напрямую спросит: «ну, готовы французы воевать? А за чьи интересы,
сможешь… народу объяснить, дорогой мусью?».

Что делать Байдену иже с ним? Только вводить «адские санкции». С таким
раскладом категорически не согласна Европа, это чревато страшным финансовым и
энергетическим кризисами. Ввести что-то вегетарианское… гомерический хохот
сотрясёт остальной мир. О, за «агрессии» теперь положен запрет на ввоз турецких
помидоров?

Из путинской ловушки выход только один: садиться за стол переговоров, начиная с
самого безболезненного для репутации вопроса — украинского. Первые нужные слова
уже прозвучали: будем наращивать присутствие войск в восточных странах НАТО, но
ни под каким соусом не станем воевать с Россией из-за Киева. Один-ноль, что и
требовалось.

Поверит Гарант «гарантиям» невступления Украины в Североатлантический Альянс?
Не смешите. Даже письменные заверения не нужны. Всё необходимое Кремлю уже
свершилось: налицо полный раскол и раздрай внутри коллективного Запада,
политическое похмелье и коллапс государственности в Киеве. А Зеленский
витиевато-матерно разругался с «гарантами» майданной демократии, осознав
простую вещь — Украину просто толкают в пасть раздраженного медведя.

Подведя себя к придуманной ими же пропасти, многие на Западе поняли:
«российской агрессии» нет даже в отдалённой перспективе, манипуляторы
откровенно заигрались и заврались. Америка очень желает воевать до последнего
украинского солдата, чтобы окончательно подмять под себя политику и экономику
Евросоюза «адскими секторальными санкциями» против России, от которых сама не
пострадает ни на цент.

Это было понятно давно, но Владимиру Путину требовалось зафиксировать публично
данный расклад. Чтобы как можно больше европейских «Буратин» бросились в глазах
избирателей на Поле Чудес закапывать золотые сольдо, сидеть и ждать… «агрессии
России». Но это всего лишь первый слой весёлой кремлевской провокации, попутно
решалась...

\subsubsection{Проблема олигархов.}

Глубже всех увяз коготок британского дряхлого льва, когда в парламенте
прозвучал призыв принять Закон о санкциях против активов российского бизнеса и
собственности русскоговорящих олигархов на территории Англии. Автор этого
судьбоносного для России документа, как не трудно догадаться, — великий географ
и историк, глава МИД Соединённого Королевства Лиз Трасс.

\begin{zznagolos}
«Мы обеспечим, чтобы те, кто разделяет ответственность за дестабилизирующие
действия Кремля, заплатили большую цену... Любая компания, представляющая
интерес для Кремля и режима в России, сможет стать объектом санкций. Так что
путинским олигархам или российским компаниям, поддерживающим власти России,
будет негде спрятаться!».	
\end{zznagolos}

Кремль немедленно подыграл, пресс-секретарь Песков пригрозил ответными
«зеркальными мерами», но политически грамотный народ покатился со смеха. Лондон
решил взяться за олигархов из России? Наконец-то, набрался смелости
отреагировать на доклад правительства Её Величества от 2016 года, где прямо
написано:

\begin{zznagolos}
«объём средств русского бизнеса, имеющих коррупционное происхождение, что
поступают в Соединенное Королевство, — оценивается в 100 млрд фунтов стерлингов
в год».	
\end{zznagolos}

Намерения британских властей в отношении имущества олигархов российская сторона
полностью поддержала. Особенно ядовито высказалась официальный представитель
МИД России Мария Захарова:

\begin{zznagolos}
«Главное, ни шагу назад, миссис Трасс! Мы на вас очень надеемся. Никакие
сиюминутные британские финансовые интересы не могут быть важнее демократии и
свободы! Доведёте свою мысль до реализации – сможете претендовать на получение
знака «За взаимодействие».	
\end{zznagolos}

Попутно были сообщены фамилии коррупционеров, которых Россия просила Лондон
выдать в первую очередь. А все официальные СМИ, как по команде, выпустили
глубокие аналитические материалы, где главным было давнее предупреждение
Гаранта отечественной элите относительно рисков зарубежного размещения средств.

\ii{13_02_2022.yz.istoricheskie_naperstki.1.ajda_putin_molodec.pic.5}

Много их там скопилось? Никто не знает. На территории Соединённого Королевства
проживает не менее 100 тысяч наших очень небедных соотечественников. В конце
прошлого года агентство Reuters сообщило: только «инвестиционными визами»,
которые выдаются для получения вида на ПМЖ (в дальнейшем — подданства
Великобритании), обладают свыше полутора тысяч россиян.

На минуточку, «инвест-виза» — это требование единовременно вложить в экономику
Королевства сумму ... от 2 млн. фунтов-стерлингов (202 млн. рублей) каждому
соискателю британского паспорта.

Имена самых обеспеченных олигархов хорошо известны: основатель группы USM
Алишер Усманов, Роман Абрамович и Михаил Фридман. Все вкладываются в футбольные
клубы, ценные бумаги, элитную и раритетную недвижимость. Их особняки стоят
неподалёку от баснословно дорогих домов в самых престижных районов Лондона:
вице-президента «ФосАгро» Андрея Гурьева, экс-главы Банка Москвы Андрея
Бородина, бывшего топ-менеджера Газпрома Андрея Гончаренко.

Каким будет Закон о «русских активах и олигархах» — не очень понятно (и будет
ли вообще). Но судя по крайне обоснованным утечкам из «Гардиан», под удар
попадут собственники пакетов акций российских энергетических компаний. От
самого крупного миллиардера Михаила Прохорова — до представителей «низшей лиги
олигархов» Евгения Чичваркина. Средства будут арестованы или заморожены на
неопределённый срок.

На счёт имущества олигархов власти туманного Альбиона предпринять что-либо
бессильны. Разве что ограничить право залога и продажи. А судиться можно
десятилетиями с каждым нашим коррупционером, доказывая незаконность обретения
капиталов. Это мертворождённая инициатива, но нервы способна потрепать.

\subsubsection{Выводы...}

Если все капканы и силки расставлены Гарантом верно, нужно запасаться
поп-корном и ждать, куда именно потекут вскоре финансовые потоки. Проблема
выборов 2024 года будет решаться полным перетряхиванием олигархических элит.
Если за Россию — выводи скорее награбленное домой. Против, готов финансировать
оппозиционные кампании (это явно является условием «разморозки» или
неприкосновенности активов) — отправляйся в Лондон.

\ii{13_02_2022.yz.istoricheskie_naperstki.1.ajda_putin_molodec.pic.6}

Конечно, риск велик. Если закон будет принят, Лондон немедленно лишится статуса
«тихой гавани» для уведённых капиталов со всего мира, потеряет доверие всех без
исключения заморских олигархов. А на счёт российских… это вообще подарок новой
избирательной кампании Владимира Путина, долгие и продолжительные аплодисменты
всего российского народа принципиальности «английских коллег».

Особенно если появятся данные о действующих чиновниках и сотрудниках
госкорпораций, которые сегодня скрыты за именными счетами родственников,
друзей, близких — величайший скандал может выйти. Полное и законное право
Гаранта засучить рукава, начать заселять Сибирь новыми колонистами в лучших
сталинских традициях.

То есть, истерика с «агрессией России» довела не очень сообразительных
политиканов до полного цугундера. Ядовитые английские комментаторы уже
спрашивают отшлёпанную Лавровым главу МИД Лиз Трасс: Вы действуете по заданию
Путина, очищаете его страну от коррупционеров и воров? Лишаете нашу экономику
миллиардов фунтов, если многие примут решение «вернуть награбленное» и
покаяться?

Жаль, что это лишь мечты. Наше справедливое раздражение в адрес обладателей
«криминальных капиталов» не будет удовлетворено. Представляю, насколько сейчас
перегружены все спецслужбы России, наблюдающие за панической лихорадкой
отечественных «Буратин», перепрятывающих золотые сольдо. С каждым, наверняка,
спокойно поговорят, укажут новое Поле Чудес для их возделывания.

Жаль, что консенсус элит не будет нарушен, всё произошедшее за эти месяцы
останется за семью кремлёвскими печатями. Но спецоперация «Агрессия России»
вызывает уважение по всем пунктам. Как в собственный чан с лживой пропагандой
угодили западные политические элиты, как мир полюбовался на «гарантии
безопасности» и настоящее отношение к марионеточным режимам, насколько «сплочён
Североатлантический Альянс» при запахе жаренного.

Да и нашим олигархам придётся повнимательнее посмотреть на политиков стран
пребывания их капиталов. Насколько рухнул уровень профессионализма, умение
вести государственные дела. Как легко можно манипулировать и загонять в
безнадёжные тупики великую когда-то дипломатию держав-гегемонов, выставлять на
посмешище перед всем миром президентов, премьер-министров, глав МИД и публику
рангом пожиже.

Как они сами делают Россию сильнее своими руками, подпитывают наше чувство
юмора и сплоченность, экономят президенту Путину расходы на избирательную
кампанию. Умно...
