
\isubsec{2_4_Generating_PDF_directly_from_TeX}{Generating PDF directly from \TeX}

The purpose of Han The Thanh's \pdfTEX\  project \footnote{ We are
grateful to Han The Thanh for considerable help with this section. }
was to create an extension of \ \TeX\  that can create PDF directly from
TEX source files and possibly actually enhance the result of \TeX\ 
typesetting with the help of PDF. The pdfTEX program contains \ \TeX\
as a subset: VVhen PDF output is not selected, \pdfTEX produces normal
DVI output, otherwise it produces PDF output that looks identical to the
DVI output. The next stage of the project is to investigate alternative
justification algorithms, possibly making use of multiple master fonts.
We will not, however, discuss that aspect of the program in this book. 

The \pdfTEX\  program is based on the original \TeX\ sources and \verb|web2c| and has 
been successfully compiled on UNIX, Macintosh, Amiga, VV1n32, and DOS systems. 

%%page page_90                                                  <<<---3

\isubsubsec{2_4_1_Setting_up_pdfTeX}{Setting up \pdfTEX}

The \pdfTEX\ program [PDFTEXS] is distributed with many of the free TEX
packages, including MikTeX and fpTEX for Windows 32, teTEX for UNIX,
CMacTeX for Macintosh, and the general \verb|web2c| system on which most
of these packages are based. 

In addition to the normal \ \TeX\  fonts and macros, a \pdfTEX distribution consists 
of the following items: 

\begin{itemize}
  \item \verb|pdftex.pool| pool file, needed for creating formats; 
  \item \verb|ttf2afm| an external program to generate AFM files from TrueType fonts, needed 
    to create \ \TeX\  font metric files; 
  \item \verb|pdftex.cfg| \pdfTEX\  configuration file (see \refsec{2_4_1_1_The_pdfTEX_configuration_file}); and 
  \item \verb|map| PostScript and TrueType font maps (see \refsec{2_4_1_3_Map_files}). 
\end{itemize}

When \pdfTEX\  is running, some extra search paths beyond those normally
requested by \TeX\ itself are used: 

\begin{itemize}
    \item VFFONTS the path where \pdfTEX\  looks for virtual fonts; 
    \item T1FONTS the path where \pdfTEX\  looks for Typel fonts; 
    \item TTFONTS the path where \pdfTEX\  looks for TrueType fonts; 
    \item PKFONTS the path where \pdfTEX\  looks for PK fonts; and 
    \item TEXPSHEADERS the path where \pdfTEX looks for its configuration file 
(\verb|pdftex.cfg|), font mapping files (\verb|map|), encoding files (\verb|enc|), and graphics 
files (see \refsec{2_4_2_3_Graphics_inclusion}). 
\end{itemize}

\ipar{2_4_1_1_The_pdfTEX_configuration_file}{The \pdfTEX\ configuration file}

When \pdfTEX\  starts, it reads a configuration file called
\verb|pdftex.cfg|, searched for in the  \verb|TEXPSHEADERS| path.
Because \verb|web2c| systems commonly specify a private tree for
\pdfTEX\  where configuration and map files are located, this allows
individual users or projects to maintain customized versions of the
configuration file. It also means that individual \ \TeX\  input files
need not set any \pdfTEX-specific macros. 

The configuration file is used to set default values for the following parameters, 
all of which can be overridden in the \ \TeX\  source file: 

\begin{itemize}
  \item \verb|output_format| Integer parameter specifying the output. A value greater than 
    zero means PDF output, otherwise DVI output. 
  \item \verb|compress_level| Integer parameter specifying the level of text compression (using 
    \verb|zlib|). Zero means no compression, 1 means fastest, 9 means best, 2.8 means 
    something in between. 
  \item \verb|decimal_digits| Integer parameter specifying the precision of real numbers in PDF 
    code. Valid values are in the range 0..5. A higher value means more precise 
    output, but it may also mean a larger size and more time taken to display or 
    print. In most cases the optimal value is 2. 

  \item \verb|image_resolution| Integer parameter specifying the default
    resolution of bitmap
    image files that contain no resolution information. 

  \item \verb|page_width, page_height| Dimension parameters specifying the page width and 
    height of PDF output. If not specified, then page width is calculated by 
    taking the width of the box being shipped out and adding $2\times\ $ \verb|(horigin + \hoffset)|. The page height is calculated in a similar way. 
  \item \verb|horigin, vorigin| Dimension parameters specifying the offset of the \ \TeX\  output 
    box from the top left corner of the ``paper.''
  \item \verb|map| The name of the font mapping file (similar to those used by many DVI to 
    PostScript drivers); more than one map file can be specified, using multiple map 
    lines. If the name of the map file is prefixed with a +, its values are appended 
    to the existing set, otherwise they replace it. If no map files are given, a default 
    file psfonts .map is searched for. 
\end{itemize}

A typical \verb|pdftex.cfg| file looks like the following. It sets up output for A4 paper 
size and the standard \ \TeX\  offset of 1 inch, and loads two map files for fonts. 

\begin{verbatim}
  
  output_format 1 
  compress_level O 
  decimal_digits 2 
  _width 210mm 
  _height 297mm 
  horigin lin 
  vorigin 1in 
  map standard.map 
  map +cm.map 

\end{verbatim}

\ipar{2_4_1_2_Setting_up_fonts}{Setting up fonts}

The \pdfTEX\  program normally works with Type 1 and TrueType fonts; a
source must be available for all fonts used in the document, except for
the 14 base fonts supplied by Acrobat (the Times, Helvetica, Courier,
Symbol, and Dingbats families). It is possible to use METAFONT-generated
fonts in \pdfTEX. It is strongly recommended, however, not to do so if
an equivalent is available in Type 1 or TrueType format, since the
resulting Type 3 fonts render very poorly in current versions of
Acrobat. Given the free availability of Type 1 versions of all the
Computer Modern fonts and the ability to use standard PostScript fonts
without further ado, this is not usually a problem. 

%%page page_92                                                  <<<---3

\ipar{2_4_1_3_Map_files}{Map files}

The \pdfTEX\  program reads \emph{map files} specified in the
\emph{configuration file} (see
\refsec{2_4_1_1_The_pdfTEX_configuration_file}), in which reencoding and
partial downloading for each font are specified. Every font needed must
be listed, each on a separate line, apart from PK fonts. 
\pindx{map files}

The syntax of each line is similar to the \verb|dvips| map files \footnote{
Most dvips map files can be shared with \pdfTEX without problems. 
} 
and can contain up to six space-separated fields: \emph{texname},
\emph{basename}, \emph{fontflags}, \emph{fontfile}, \emph{encoding}, and
\emph{special}.  The only mandatory field is \emph{texname}, and it must
be the first field. The rest of the fields are optional, but if
\emph{basename} is given, it must be the second field. Similarly, if
\emph{fontflags} is given, it must be the third field (if
\emph{basename} is present) or the second field (if \emph{basename} is
left out). It is possible to vary the positions of \emph{fontfile},
\emph{encodingfile}, and \emph{special}, but the first three fields must
be given in fixed order. 

\begin{itemize}

  \item \textbf{texname:} the name \ \TeX\  uses, that is, the name of the TFM file. This must be 
  present. 
  
  \item \textbf{basename:} the PostScript font name. If not given, then it will be taken from the 
  font file. Specifying a name that does not match the name in the font file will 
  cause \pdfTEX\  to produce a warning, so it would be best not to use this field 
  if the font resource is available (this is the most common case). This option is 
  primarly intended for use of base fonts and for compatibility with dvips map 
  files. 
  
  \item \textbf{fontflags}: flags specifying some characteristics of the
    font. The following description is taken (with some modification)
    from the PDF specification (Bienz et al.  (1996)), Section 7.9.2
    (Font descriptor flags). 
  
  The value of the Flags key in a font descriptor is a 32-bit integer that
  contains a collection of Boolean attributes. These attributes are true
  if the corresponding bit is set to 1 in the integer. The following
  specifies the meanings of the bits, with bit 1 being the least
  significant. Reserved bits must be set to zero. 
  
  \begin{center}
    \begin{tabular}{ccc}
      \hline \\
      \emph{Bit position} & \emph{Semantics} & \emph{Sample}  \\
      \hline \\
      1     & Fixed-width font               & Sample Text \\
      2     & Serif font                     & Sample Text \\
      3     & Symbolic font                  &             \\
      4     & Script font                    &             \\
      5     & Reserved                       &             \\
      6     & Uses the Adobe Standard Roman  &             \\
            & Character Set                  &             \\
      7     & Italic                         & Sample Text \\
      8-16  & Reserved                       &             \\
      17    & All-cap font                   & SAMPLE TEXT \\
      18    & Small-cap font                 & SAMPLE TEXT \\
      19    & Force bold at small text sizes &             \\
      20-32 & Reserved                       &             \\
      \hline \\
    \end{tabular}
  \end{center}
  
  All characters in a fixed-width font have the same width, while
  characters in a proportional font have different widths. Characters in a
  serif font have short strokes drawn 
  at an angle on the top and bottom of character stems, while sans serif fonts do not 
  have such strokes. A symbolic font contains symbols rather than letters and numbers. 
  Characters in a script font resemble cursive handwriting. An all-cap font, typically used 
  for display purposes such as titles or headlines, contains no lowercase letters. It differs 
  from a small-cap font in that characters in the latter, while also capital letters, have been 
  sized and their proportions adjusted so that they have the same size and stroke weight 
  as lowercase characters in the same typeface family. 
  
  \noindent Bit 6 in the flags field indicates that the font's character
  set is the Adobe Standard Roman Character Set, or a subset of that, and
  that it uses the standard names for those characters. 
  
  \noindent Finally, bit 19 is used to determine whether bold characters are drawn with extra pixels 
  even at very small text sizes. Typically when characters are drawn at small sizes on very 
  low resolution devices such as display screens, features of bold characters may appear 
  only one pixel wide. Because this is the minimum feature width on a pixel-based device, 
  ordinary nonbold characters also appear with one-pixel-wide features and cannot be 
  distinguished from bold characters. If bit 19 is set, features of bold characters may be 
  thickened at small text sizes. 
  
  \noindent If no font flags are given, \pdfTEX\  treats the font as 3, a symbol font. If we do 
  not know the correct value, it would be best not to provide one; specifying a 
  wrong Value of the font flags may cause Acrobat some problems. 
  
  \item \textbf{fontfile:} the name of the font source file. This must be a Type 1 or TrueType font 
  file. The font file name can be preceded by one or two special characters that 
  say how the font file should be handled. 
  
  \begin{itemize}
    \item If it is preceded by a \verb|<|, then the font file will be
      partially downloaded, meaning that just those glyphs (characters)
      used in the document are extracted and put into a new subset font,
      which is then embedded in the output.  This is the most common use
      and is strongly recommended for any font, as it ensures portability
      and reduces the size of the PDF output. 
    \item If the font file name is preceded by a double \verb|<<|, the whole font file will 
      be included-all glyphs of the font are embedded, including the ones that 
      are not used in the document. Apart from increasing the size of the PDF 
      file, this option may cause problems with TrueType fonts too, so it is not 
      recommended. It is useful in cases where the font is somehow strange and 
      cannot be subsetted properly by \pdfTEX. 
    \item If no character precedes the font file name, the font file is
      read, but nothing is embedded. Only the font parameters are
      extracted to generate a font descriptor that is used by Acrobat to
      simulate the font if needed.  This Option is useful when we do not
      want to embed the font (that is, to reduce the output size) but do
      wish to use the font metrics
      and let Acrobat generate an instance that looks close to the
      original font, provided that resource is not installed on the system
      where the PDF output is viewed or printed.  To use this feature, the
      font flags must be specified and have bit 6 set on, meaning that
      only fonts with the Adobe Standard Roman Character Set can be
      simulated. The only exception is the Adobe Symbol font, which is not
      very useful. 
    \item If the font file name is preceded by a \verb|!|, the font is not read at all and is 
      assumed to be available on the target system. This option can be used to 
      create PDF files that do not contain any embedded fonts. The PDF output 
      then works only on systems where the font is available. It is not very useful 
      for document exchange, because the file is not portable. On the other hand, 
      it is very useful when we wish to speed up the running of \pdfTEX\  while 
      testing a document. This feature requires Acrobat to have access to all the 
      installed fonts, including those that are only in the \ \TeX\  support tree. 
  \end{itemize}
  
  Note that the standard 14 fonts are never embedded, even if they are marked 
  for download in map files. 
  
  \item \textbf{encoding:} name of a file containing an encoding vector to be used for the font. 
    The file name may be preceded by a \textbf{<}, but the effect is the same. The format 
    of the encoding vector is identical to that used by \textbf{dvips} (see Goossens et al. 
  (1997), Section 11.2.4). If no encoding is specified, the font's built-in default 
  encoding is used. It may be omitted if we are sure that the font resource has 
  the correct built-in encoding. In general, this latter option is highly preferred 
  and is required to subset TrueType fonts. 
  \item \textbf{special:} special instruction for font transformation as
    for dvips. Only specifications of SlantFont and Ext endFont are read;
    other instructions are ignored. 
\end{itemize}

If \pdfTEX\  cannot locate a font in a map file, it will look first for a source with 
the extension pgc, a PGC source (PDF Glyph Container). \footnote{
This is a text file containing a PDF Type 3 font, usually created using METFI POST with some 
utilities by Hans Hagen. In general, PGC files can contain whatever is allowed in a PDF page description 
to support fonts. At present PGC fonts are not very useful, since vector Type 3 fonts are not displayed 
very well in Acrobat. They may be more useful when Type 3 font handling gets better. 
} If no PGC source is 
available, \pdfTEX\  will try to use PK fonts in the same way as normal DVI drivers. 

Lines containing nothing apart from \emph{texname} indicate that a scalable Type 3 
font should be used. For font types as Type 1, TrueType, and scalable Type 3, all 
requests for the font at any size will be provided by just one font in the PDF output. 
Thus if a font, for example, \textbf{csr10}, is listed in a map file, it will be treated as scalable. 
The font \textbf{csr10} will be downloaded only once for
\textbf{csr10}, \textbf{csr10 at 12pt}, etc. 

It does not hurt much if a scalable Type 3 font is not listed in a map file, except 
that the font source will be downloaded multiple times for different sizes, meaning 
the PDF output is larger. On the other hand, if a font is listed in a map file as scalable 
Type 3 and its PGC source is not scalable or not available (in this case \pdfTEX\  will 
use PK fonts instead), the PDF output will be valid. However, some fonts may look 
ugly because bitmaps will be scaled. 

%%page page_95                                                  <<<---3
 
\ipar{Some_sample_map_file_entries}{Some sample map file entries}

Use a built-in font with font-specific encoding, that is, neither a downloaded font 
nor an external encoding is given. SlantFont is specified in the same way as for 
dvips. 

\begin{verbatim}
psyr Symbol 
psyro Symbol ".167 SlantFont" 
\end{verbatim}

Use a built-in font with an external encoding (\verb|8r.enc|). The \verb|<| preceding the 
encoding file name may be omitted. 

\begin{verbatim}
ptmri8r Times-Italic <8r.enc 
ptmro8r Times-Roman <8r.enc ".167 SlantFont" 
\end{verbatim}

Use a partially downloaded font with an external encoding: 

\begin{verbatim}
putr8r Utopia-Regular <8r enc <putr8a pfb 
putro8r Utopia-Regular <8r.enc <putr8a.pfb ".167 SlantFont" 
\end{verbatim}

Use the Type 1 font name taken from the downloaded font itself: 

\begin{verbatim}
logo8 <logo8.pfb 
\end{verbatim}

Adjust the width but not the stroke thickness: 

\begin{verbatim}
logodlo logobflo <logobf10.pfb ".913 ExtendFont" 
\end{verbatim}

 
Use entire downloaded font without reencoding: 
\begin{verbatim}
pgsr8r Gillsans <<pgsr8a.pfb 
\end{verbatim}

Use partially downloaded font without reencoding: 

\begin{verbatim}
pgsr8r Gillsans <pgsr8a pfb 
\end{verbatim}

Do not read the font at all--the font must be available on the target system: 

\begin{verbatim}
pgsr8r Gillsans !pgsr8a.pfb 
\end{verbatim}

Use an entire downloaded font with reencoding: 

\begin{verbatim}
pgsr8r GillSans <<pgsr8a.pfb 8r.enc 
\end{verbatim}

Use a partially downloaded font with reencoding: 

\begin{verbatim}
pgsr8r Gillsans <pgsr8a.pfb 8r.enc 
\end{verbatim}
 
Do not include the font, but extract parameters from the font file and reencode: \footnote{
This only works for fonts with Adobe Standard Encoding. The font flags say what this font is like, 
so Acrobat can generate a similar instance if the font resource is not available on the target system. 
}

\begin{verbatim}
pgsr8r GillSans 32 pgsr8a.pfb 8r.enc 
\end{verbatim}

Use a TrueType font in the same way as a Type 1 font: 

\begin{verbatim}
verdana8r Verdana <verdana.ttf 8r.enc 
\end{verbatim}

\ipar{2_4_1_4_TrueType_fonts}{TrueType fonts}

As we have seen, \pdfTEX\  can work with TrueType fonts, and adding the font names 
to map files is straightforward. The only extra task for TrueType fonts is to create 
TFM font metric files. There is a program, \verb|ttf2afm|, in \pdfTEX\  distributions that 
can be used to extract AFM font metrics from TrueType fonts. Usage is simple: 

\begin{verbatim}
ttf2afm ttf-fi Le [encoding] 
\end{verbatim}

The name of the TrueType font file is \emph{ttf-file}, and the optional
\emph{encoding} specifies an 
encoding file, which is the same as those used in map files for \pdfTEX\  and \verb|dvips|. 
If the encoding is not given, all the glyphs in the AFM output will be mapped to 
\verb|/.notdef|. The \verb|ttf2afm| program writes the output AFM to standard output. From 
this we can make a TFM from the AFM file (Goossens et al. (1997, Section 10.5)). 
If we need to know which glyphs are available in the font, we can run ttf2afm 
without encoding to get all the glyph names. 

To use a new TrueType font (\verb|times.ttf|), the minimal steps (assuming that 
\verb|test.map| is included in \verb|pdftex.cfg|) on a UNIX system might be 

\begin{verbatim}

  ttf2afm times.ttf 8r.enc >times.afm 
  afm2tfm times.afm -T 8r.enc 
  echo "times TimesNewRomanPSMT <times.ttf <8r.enc" >>test.map 

\end{verbatim}

The PostScript font name, \verb|TimesNewRomanPSMT|, is reported by \verb|afm2tfm| 
but is not strictly needed in the \pdfTEX\  map file. 

\verb|ExtendFont| and \verb|SlantFont| also work for TrueType fonts. 

\isubsubsec{2_4_2_New_primitives}{New primitives}

The \pdfTEX\  program adds a set of new primitives to TEX; they are described in the 
following sections, and allow the user access to features of the PDF format. 

 
%%page page_97                                                  <<<---3
 
\ipar{2_4_2_1_Document_setup}{Document setup}

\begin{verbatim}
\pdfoutput=n 
\end{verbatim}

This integer parameter specifies whether the output format should be DVI
or PDF.  A value greater than zero means PDF output, otherwise DVI
output. This parameter cannot be specified \emph{after} shipping out the first
. In other words, it must be set before \pdfTEX\  ships out the
first page if we want PDF output. This is the only parameter that must
be set to produce PDF output; all other parameters are optional. 

\begin{verbatim}
\pdf compresslevel=n 
\end{verbatim}

This integer parameter specifies the level of text compression via \verb|zlib|. Zero means 
no compression, 1 means fastest, 9 means best, 2..8 means something in between. 
A value out of this range will be adjusted to the nearest meaningful value. 

\begin{verbatim}
\pdfpagewidth=dimen 
\pdfpageheight=dimen 
\end{verbatim}

These dimension parameters specify the page width and page height of PDF output. 
If they are not given, the page dimensions will be calculated as described in Section 
2.4.1.1. 

\begin{verbatim}
\pdfpagesattr={tokens} 
\end{verbatim}

This token list parameter specifies optional attributes for every page of the PDF 
output file. These attributes can be \verb|MediaBox| (rectangle specifying the natural size 
of the page), \verb|CropBox| (rectangle specifying the region of the page being displayed 
and printed), and \verb|Rotate| (number of degrees the page should be rotated clockwise 
when it is displayed or printed-must be 0 or a multiple of 90). 

\begin{verbatim}
\pdfpageattr={tokens}
\end{verbatim}

This is similiar to \verb|\pdfpagesattr|, but it takes priority over it. It can be used to 
overwrite any attributes given by \verb|\pdfpagesattr| for individual pages. 

\ipar{2_4_2_2_The_document_information_and_catalog}{The document information and catalog}

\begin{verbatim}
\pdfinfo{info keys} 
\end{verbatim}

This allows the user to add information to the document info section; if this is 
provided, it can be seen in \verb|Acrobat Reader| with the menu option
\verb|File -> Document Info -> General|. The \emph{info keys} parameter is a set of data pairs (a key, and a value). 
The key names are preceded by a \verb|/|, and the values are in parentheses; all keys 
are optional. The possible keys are \verb|/Author|, \verb|/CreationDate| (defaults to current 
date), \verb|/ModDate|, \verb|/Creator| (defaults to ``TeX''),
\verb|/Producer| (defaults to ``pdfTeX''), 
\verb|/Title|, \verb|/Subject|, and \verb|/Keywords|. 

\verb|/CreationDate| and \verb|/ModDate| are expressed in the form
\verb|D:YYYYMMDDhhmmss|, 
where \verb|YYYY| is the year, \verb|MM| is the month, \verb|DD| is the day, \verb|hh| is the hour, \verb|mm| is 
the minutes, and \verb|ss| is the seconds. 

Multiple uses of \verb|\pdfinfo| are permitted; if a key is given more than once, the 
first appearance will take priority. An example of the use of \verb|\pdfinfo| follows: 

\begin{verbatim}
\PDFinfo{ 
  /Title (example.pdf) 
  /Creator (TeX) 
  /Producer (pdfTeX) 
  /Author (Tom and Jerry) 
  /CreationDate (D:19980212201000) 
  /ModDate (D:19980212201000) 
  /Subject (Example) 
  /Keywords (cat;mouse) 
} 
\end{verbatim}

\begin{verbatim}
\pdfcatalog{catalog keys} openaction {action} 
\end{verbatim}

The document catalog is similar to the document info section, and the available 
keys are \verb|/URI|, which provides the base URL of the document, and \verb|/PageMode|, 
which determines how Acrobat displays the document on startup. The possibilities 
for the latter are: 

\begin{center}
  \begin{tabular}{cc}
    \verb|/UseNone|     & Open document with neither outlines nor thumbnails visible. \\
    \verb|/Useoutlines| & Open document with outlines visible.                        \\
    \verb|/UseThumbs|   & Open document with thumbnails visible.                      \\
    \verb|/Fullscreen|  & 0pen document in full-Screen mode. In full-screen mode,
    there are no menu bar, window controls, or any other window present.              \\
  \end{tabular}
\end{center}

The default is \verb|/UseNone|. 

The \emph{action} is the action to be taken when opening the document; it is specified in the same way as for internal links (see \refsec{2_4_2_6_Destinations_and_links}), for example 
\verb|goto page 3 {/Fit}|. 

\ipar{2_4_2_3_Graphics_inclusion}{Graphics inclusion}

\begin{verbatim}
\pdfimage width WIDTH height HEIGHT depth DEPTH {FILENAME} 
\end{verbatim}

Insert an image, optionally changing the width, height, depth, or any combination 
of these attributes. The default values are zero for depth and the image’s natural 
size for height and width. If all of them are given, the image will be resized to fit 
the specified values. If some of them (but not all) are given, the rest will be scaled 
proportionally to keep the aspect ratio the same as that of the natural size. If none 
of them is given, then the image will be set at its natural size. The dimension of 
the image can be accessed by putting the \verb|\pdfimage| command into a \ \TeX\  box and 
checking the dimensions of that box. 

The image type is determined by the extension of the file name. Thus png 
means PNG format and pdf means it is a PDF file; otherwise, the image is treated 
as JPEG. 

\begin{verbatim}
\pdfimageresolution=RESOLUTION 
\end{verbatim}

If the image is a bitmap file and contains resolution information, then that is used; 
otherwise, \verb|\pdfimageresolution| can be used to specify it. The default is 72 dpi. 

\begin{verbatim}

\pdffontprefix{PREFIX STRING} 
\pdfimageprefix{PREFIX STRING} 
\pdfformprefix{PREFIX STRING} 

\end{verbatim}

Sometimes there are problems including a PDF file as an illustration, because of 
conflicts in fonts, image names or PDF form objects. These three commands allow 
you to change the default prefixes for names. So if an included PDF file has a font 
resource named \verb|/F34|, and you find it conflicts with an \verb|/F34| in the current file, you 
can use \verb|\pdffontprefix| to name it, for example, /FF34. Using these commands 
is recommended only for experts. 

\ipar{2_4_2_4_XObject_Forms}{XObject Forms}

\begin{verbatim}
\pdfform NUMBER 
\end{verbatim}

writes out the \ \TeX\  box \verb|NUMBER| as an XObject Form to the PDF file. 

\begin{verbatim}
\pdflastform 
\end{verbatim}

returns the object number of the last XObject Form written to the PDF file 

\begin{verbatim}
\pdfrefform \NAME 
\end{verbatim}

puts in a reference to the XObject Form called \verb|\NAME|. 

These macros support ``object reuse'' in \pdfTEX. The content of the XObject 
Form object corresponds to the content of a \ \TeX\  box, which can contain text, 
pictures, and references to other XObject Form objects. The XObject Form can 
be used by simply referring to its object number. This can be useful in a large 
document with a lot of similar elements, since it avoids the duplication of identical 
objects. A common example is a document style that places an identical graphic or 
text in the header of every page. 

%%page page_100                                                  <<<---3

\ipar{2_4_2_5_Annotations}{Annotations}

\begin{verbatim}
\pdfannot width WIDTH height HEIGHT depth DEPTH {TEXT} 
\end{verbatim}

attaches an annotation at the current point in the text. The annotation content will 
be raw PDF code, as specified in \verb|TEXT|. 

\begin{verbatim}
\pdflastannot 
\end{verbatim}

returns the object number of last annotation created by \verb|\pdfannot|. These two 
primitives allow the user to create any annotation that cannot be created by 
\verb|\pdfannotlink| (see following). 

\ipar{2_4_2_6_Destinations_and_links}{Destinations and links}

\begin{verbatim}
\pdfdest ( num {NUM} | name {NAME} ) UPPEARANCE 
\end{verbatim}

establishes a destination for links and bookmark outlines. The link must
be identified by either a \emph{NUMBER} 
or a symbolic \emph{REFNAME} and the way Acrobat is to display the 
must be specified. \emph{APPEARANCE} must be one of the following: 

\begin{center}
  \begin{tabular}{cc}
    \verb|fit|   & fit whole page in window                 \\
    \verb|fith|  & fit whole width of page                  \\
    \verb|fitv|  & fit whole height of page                 \\
    \verb|fitb|  & fit whole Bounding Box page              \\
    \verb|fitbh| & fit whole width of Bounding Box of page  \\
    \verb|fitbv| & fit whole height of Bounding Box of page \\
    \verb|xyz|   & keep current zoom factor                 \\
  \end{tabular}
\end{center}

\verb|xyz| can optionally be followed by zoom \emph{factor} to provide a
fixed zoom-in. The \emph{factor} 
is like \ \TeX\  magnification; that is, 1000 is the ``normal'' page view. 

\begin{verbatim}
\pdfannotlink height {HEIGHT} depth {DEPTH} attr {ATTR} ACTION 
\end{verbatim}

starts a hypertext link. If the optional dimensions are not specified, they will be 
calculated from the box containing the link. The attributex (explained in great detail 
in Section 6.6 of the PDF manual) determine the appearance of the link. Typically 
they are used to specify the color and thickness of any border around the link. 
Thus \verb|/C [O.9 O 0] /Border [0 0 2]| specifies a color (in RGB) of bright red 
and a border thickness of 2 points. 

The \emph{action} can do many things; some of the possibilities are listed in \reftab{2-13}. 

\begin{verbatim}
\pdfendlink 
\end{verbatim}

ends a link; all text between \verb|\pdfannotlink| and \verb|\pdfendlink| will be treated as 
part of this link. The \pdfTEX\  program may break the result across lines (or pages), 
 
\ipar{2_4_2_7_Bookmarks}{Bookmarks}

\begin{verbatim}
\pdfoutline ACTION count {COUNT} {TEXT} 
\end{verbatim}

creates an outline (or bookmark) entry. The first parameter specifies
the action to be taken and is the same as that allowed for
\verb|\pdfannotlink|. The \emph{COUNT} specifies the number of direct
subentries under this entry; it is zero if this entry has no subentries
(in which case it may be omitted). If the number is negative, then all
subentries will be closed, and the absolute value of this number
specifies the number of subentries.  The \emph{TEXT} is what will be
shown in the outline window; note that this is limited to characters in
the \verb|PDFDocEncoding| vector. 

\ipar{2_4_2_8_Article_threads}{Article threads}

\begin{verbatim}
\pdfthread num {NUM} name {NAME} 
\end{verbatim}

starts an article thread. The corresponding \verb|\pdfendthread| must be in the box in 
the same depth as the box containing \verb|\pdfthread|. All boxes in this depth level will 
be treated as part of this thread. An identifier (\emph{NUMBER} or
\emph{REFNAME}) must be specified; 
threads with the same identifier will be joined together. 

\begin{verbatim}
\pdfendthread 
\end{verbatim}

finishes the current thread. 
 
\begin{verbatim}
\pdfthreadhoffset=DIMEN 
\pdfthreadvoffset=DIMEN 
\end{verbatim}

specify thread margins. 

\ipar{2_4_2_9_Miscellaneous}{Miscellaneous}

\begin{verbatim}
\pdfliteral {PDF TEXT} 
\end{verbatim}

Like \verb|\special| in normal \ \TeX, this command inserts raw PDF code
into the output. It allows support of color and text Uansformation and
is used in the standard graphics package's \verb|pdftex| driver. 

\begin{verbatim}
\pdfnames{DATA} 
\end{verbatim}

puts \emph{DATA} in the names dictionary in the catalog. 

\begin{verbatim}
\pdfobj STREAM {TEXT} 
\end{verbatim}

is similar to \verb|\pdfliteral|, but the text is inserted as contents of an object. If the 
optional keyword stream is given, the contents will be inserted as a stream. 

\begin{verbatim}
\pdflastobj 
\end{verbatim}

returns the object number of the last object created by \verb|\pdfobj|. These primitives 
provide a mechanism allowing insertion of a user-defined object in PDF output. 

\begin{verbatim}
\pdftexversion 
\end{verbatim}

returns the version of \pdfTEX\  multiplied by 100; for example, for version \verb|0.13b| it 
returns 13. 

\begin{verbatim}
\pdftexrevision 
\end{verbatim}

returns the revision of \pdfTEX; for example, for version \verb|0.13b| it returns \verb|b|. 

\isubsubsec{2_4_3_Graphics_and_color}{Graphics and color}

The \pdfTEX\  program supports inclusion of pictures in PNG, JPEG, and
PDF format. The most common technique-the inclusion of Encapsulated
PostScript figures-is replaced by PDF inclusion. EPS files can be
converted to PDF by Ghostscript, Acrobat Distiller, or other
PostScript-to-PDF convertors. The bounding box of a PDF file is taken
from the CropBox if available, otherwise from MediaBox. To get the right
MediaBox, it is necessary to transform the EPS file before conversion so
that the start point is at the (0,0) coordinate and the page size 
is set exactly corresponding to the \verb|BoundingBox|. A Perl script [EPSTOPDF] for 
this purpose has been written by Sebastian Rahtz. 

Other alternatives for graphics in \pdfTEX\  are 

\begin{itemize}
  \item \LaTeX\  picture mode Since this is implemented simply in terms of font characters, it works in exactly the same way as usual. 
  \item XY-pic If the PostScript back-end is not requested, XY-pic uses its own Type 1 
    fonts and needs no special attention. 
  \item \verb|tpic| The \textbf{tpic} \verb|\special| commands (used in some macro packages) can be 
    redefined to produce literal PDF, using macros by Hans Hagen. 
  \item \verb|METAPOST| Although the output of METAPOST is PostScript, it is in a 
    highly simplified form, and a METAPOST-to-PDF conversion (written by Hans 
    Hagen and Tanmoy Bhattacharya) is implemented as a set of macros that read 
    METRPOST output and support all of its features. The type \verb|mps| is supported 
    by the \LaTeX\  \verb|graphics| package for this purpose. 
\end{itemize}

The last two macro files are part of the Con\TeX\ T macro package (\verb|supp-pdf.tex| 
and \verb|supp-mis.tex|), but they also work with \LaTeX\ and are available separately. 

The inclusion of raw PostScript commands-the technique utilized by the
\textbf{pstricks} package (Goossens et al. (1997), Chapter 4) --- cannot be
supported.  Although PDF is a direct descendant of PostScript, it lacks
any programming language commands and cannot deal with arbitrary
PostScript. 

The standard \LaTeX\  \verb|graphics| and \verb|color| packages have
\verb|pdftex| options, which allow use of normal color, text rotation,
and graphics inclusion commands.  The implementation of graphics
inclusion makes sure that however often a graphic is used (even if it is
used at different scales or transformed in different ways), it is
embedded only once. 

A number of samples of \pdfTEX\  output can be found on the TUG Web server 
[PDFTEXEX]. 
