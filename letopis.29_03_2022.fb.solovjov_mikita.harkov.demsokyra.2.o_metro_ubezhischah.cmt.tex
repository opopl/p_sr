% vim: keymap=russian-jcukenwin
%%beginhead 
 
%%file 29_03_2022.fb.solovjov_mikita.harkov.demsokyra.2.o_metro_ubezhischah.cmt
%%parent 29_03_2022.fb.solovjov_mikita.harkov.demsokyra.2.o_metro_ubezhischah
 
%%url 
 
%%author_id 
%%date 
 
%%tags 
%%title 
 
%%endhead 
\zzSecCmt

\begin{itemize} % {
\iusr{Anna Baklygina}
Полностью согласна. Проблемы с психикой у людей уже катастрофически а будет ещё хуже..

\begin{itemize} % {
\iusr{Ирина Булгакова}
\textbf{Анна Баклыгина} заметила, что у тех, кто сидит ночью дома под обстрелами, с психикой ещё хуже

\iusr{Anna Baklygina}
\textbf{Ирина Булгакова} , я нахожусь дома и вроде ещё могу адекватно реагировать. Метро замкнутое пространство и концентрация людей большая и это не есть хорошо. Временно да, постоянно нет.

\iusr{Ирина Булгакова}
\textbf{Анна Баклыгина} расскажите это моей знакомой, которая звонит мне сейчас с ППоля и дрожащим голосом вперемежку со слезами рассказывает про то, как прилетело возле её дома

\iusr{Anna Baklygina}
\textbf{Ирина Булгакова} , вы идиотка. Писать в соц.сети место и время прилёта уголовно наказуемое преступление сейчас. Посему, идите на хуй. Микита Соловйов, прошу прощения что у Вас под постом.

\iusr{Roman Frolov}
\textbf{Anna Baklygina} та вроде никто ничего не вьідал
тут же про реальную проблему, как люди реагируют на близкую опасность (замените мьісленно тот район на любой другой) и не надо посьілать.

\iusr{Anna Baklygina}
\textbf{Roman Frolov}, А Вы изначальный пост читали или Вам просто поговорить и вставить свой пятак? Мужчина с закрытым профилем..
\end{itemize} % }

\iusr{Yuliia Pimenova}

Ну это ж не цветочки дарить в метро, а целые задачи и процессы. Сильно
сомневаюсь, что возьмутся

\iusr{Oleg Chernyak}

Ты шо, а где же герой обороны Харькова Терехов пиар-концерты проводить будет?

\begin{itemize} % {
\iusr{Andrey Fedorinin}
\textbf{Oleg Chernyak} 

безотносительно Терехова и без сарказма - метро это убежище, точка. К
сожалению, в нашем с тобой городе нет достаточного количества бомбоубежищ (не
нужно рассказывать про подвалы и т.д. - это не бомбоубежища). Да, проблема с
отсутствием быстрого и безопасного транспорта существует, но на данный момент
метро решает проблему банального здорового сна для многих людей. Относительно
того, что не выходят на улицу - бред сивой кобылы, из того, что я вижу в своем
районе. Респираторные заболевания - я лично заболел после полуторачасовой
очереди на кассу в классе, а не потому, что спускаюсь спокойно поспать в
бомбоубежище. На счет эвакуации... Ну камон - реально переселить людей в
села... А по ним ничего не прилетает? Про психологические проблемы вообще
промолчу... выводить на 500-600 метров погулять... Да ну еб вашу машу... Может
где-то это и нужно делать, не спорю, но не везде - из того что вижу конкретно
тут - отсутствуют такие, кто месяц сидит на жопе в метро и разводит
антисанитарию. Вообще можно написать все, что угодно по этому поводу, но
хочется сказать одно - если в городе все так безопасно - хочу видеть работающий
наземный транспорт.

\end{itemize} % }

\iusr{Нина Юхименко}

Все равно не понимаю, почему россии можно стрелять по Украине, а Украине по
россии нельзя?

\begin{itemize} % {
\iusr{Viktoriia Levizka}
\textbf{Нина Юхименко} потому, что НЕЧЕМ! у нас практически нет наступательного вооружения. скажите спасибо зегандонам

\iusr{Kostyantyn Filonenko}
\textbf{Нина Юхименко} там де змогли - там в'єбалі. Наприклад, Мілерово

\iusr{Pavlo Vinnyk}
\textbf{Нина Юхименко} думаю проблема в том что нам нечем стрелять на такие расстояния

\iusr{Григорий Степанов}
\textbf{Viktoriia Levizka}
ЧЕМ СТРЕЛЯТЬ?!

\iusr{Sergiy Fakas}
\textbf{Нина Юхименко} 

Самое дальнее наше оружие єто советская ракета Точка-У. Дальность 120 км.
Именно ей ударили по Миллерово и потопили БДК \enquote{Саратов} в Бердянске.

\iusr{Serge Bikhunenko}
\textbf{Sergiy Fakas} 

я так понимаю, что проблема еще и с точностью - \enquote{точка} старая, без жпс и
прочих современностей наведения. А то так вполне можно было бы по скоплениям в
белгородской области кинуть.

И мало их осталось  @igg{fbicon.frown} 

\iusr{Sergiy Ryabykin}

Тому що більшість, в тому числі харків'яни, обрали Зелю. А він позакривав
проекти розвитку ракет типу Вільха (130км) та Нептун (25км).

Озерніться навкруги і подякуйте.

\iusr{Sergiy Fakas}
\textbf{Serge Bikhunenko} Та нормально там с точностью, Саратову хватило  @igg{fbicon.smile} . Но мало их скорее всего.

\iusr{Oleksii Fesenko}
\textbf{Нина Юхименко} 

потому что мы не они. Мы не убиваем мирных просто из ненависти из за того что
они не такие как мы. Тот факт что с территории рф можно ударить по Харькову с
ураганч означает что с Харькова можно ударить по рф, но зачем?  @igg{fbicon.smile} 

\iusr{Andriy Trushevskyi}
\textbf{Sergiy Ryabykin}, 

ніхто нічого не закривав. Вільха вже на озброєнні, навіть вільха-м, що на 180км
стріляє (хоча мабуть не так багато як хотілось би), а нептун мали прийняти на
озброєння в квітні.

Ось про Вільху з вікипедії: \enquote{РСЗВ «Вільха» використовувалась з перших днів
російського вторгнення в Україну 2022 року. Станом на 2 березня було зроблено
близько 50 влучних ракетних ударів.}

\url{uk.wikipedia.org/wiki/Вільха_(ракетний_комплекс)}

\iusr{Khrystyna Kuznietsova}
\textbf{Oleksii Fesenko} в смысле, зачем? Люди не должны платить за свои поступки?

\iusr{Oleksii Fesenko}
\textbf{Кристина Кузнецова} 

для этого вводятся санкции и репарации. Мы не в ХV веке чтобы в ответ на
убийства убивать, как бы сильно кому-то этого не хотелось.

У меня тоже такие мысли бывают но с ними нужно бороться

\iusr{Oleksii Fesenko}
\textbf{Кристина Кузнецова} 

а еще смотрите, психика ломается не только у жертв, но и у убийц, как рф
образца 90х. Нам бороться со своими демонами, им со своими. Зачем нам лечиться
сначала от \enquote{нас убивали} а потом от \enquote{мы убивали}?

\iusr{Марійка Мороз}
\textbf{Нина Юхименко} 

тому, що війна не об'явлена. Де-юре. Будь-який постріл по території рф розв'яже
хуйлу руки на тотальну примусову мобілізацію. Він зараз потрапив у свою ж
пастку: війни нема, \enquote{спецоперація}, тому він не може поставити у військо усіх
чоловіків країни. Поки він це робитиме завуаліровано, а значить - в рази
повільніше. Час грає на нас: в них закінчуватимуться ресурси, а економіку
обвалюють санкції. Але якщо він оголосить в рашці воєнний стан, це вийде нам не
на користь.

\iusr{Viktoriia Levizka}
\textbf{Григорий Степанов} вы МЕНЯ спрашиваете?

\end{itemize} % }

\iusr{Женя Дисс}
Повністю згодна. І пустити метро.

\iusr{Женя Дисс}
Давно думаю про це.

\iusr{Sergii Gizelo}

І є ще один нюанс:

Я от бажаю запустити своє виробництво, маленьке, але все ж таки.

А зібрати людей до праці без метро майже неможливо, пішки по двадцять
кілометрів на день не находиш, автівками - варіант економічно безглуздий

\iusr{Artem Korotenko}

Чимось схожа історія з залишками майдану, який простояв до літа 2014 року. Всі
просто затягують рішення проблеми, що назріла

\iusr{Svitlana Polyakova}

Есть две стороньі медали.

Я сидела до 20. 03 в Киеве (понимаю, что єто не Харьков), периодически спускаясь в
б/у в соседнем доме. А потом вьіехала в євакуацию, через Польщу.

І, хочу сказать, что моя психика больше страдает вдали от опасности
,т. к. євакуация, єто то еще испьітание. При том, что я об'ездила (поездки на 50\%
бьіли связаньі с работой) более 10 стран на разньіх континентах-Азия Америка
,Европа.

Євакуация + беженство -не туризм!

\begin{itemize} % {
\iusr{Микита Соловйов}

Так сидеть дома и периодически спускаться в подвал это совершенно другой
вариант. Сейчас в Харькове так делают многие. А вот сидеть сутками в метро, это
и по психике, и по санитарии и по всему остальному жесть полная.

\iusr{Оля Хабарова}
\textbf{Свитлана Полякова} 

а почему, объясните,пожалуйста.

Моя дочка в эвакуации неделю плакала, конечно, а сейчас, вроде бы все в
порядке.

Может просто времени ещё мало прошло?

\iusr{Ирина Булгакова}
\textbf{Микита Соловйов} 

да никто и не сидит, в 6.00 платформа стремительно пустеет, все разбегаются по
домам, именно чтобы помыться, поесть, заниматься ежедневными делами.
Большинство приходит только переночевать.

\iusr{Svitlana Polyakova}
\textbf{Оля Хабарова} Напишите, плз, где ваша дочь и сколько ей лет?
Єто тоже имеет большое значение.

\iusr{Оля Хабарова}
\textbf{Свитлана Полякова} в Германии, дочке 32, внуку 3,5.

Приютили очень хорошие люди. Пара примерно моего возраста, у них тоже двое
внуков, живут отдельно, но недалеко.

Добирались дочка с внуком очень тяжело, эвакуационным поездом, потом
переночевали в Варшаве и поехали дальше

\iusr{Svitlana Polyakova}
Да, Германия отличается неожиданной доброжелательностью .

\iusr{Svitlana Polyakova}
\textbf{Оля Хабарова} 

Но вьі же понимаете, кто остался в метро? К єтим людям необходим особьій
подход. И да, возможно и психолог.

Более решительньіе давно вьіехали.

\iusr{Оля Хабарова}
\textbf{Свитлана Полякова} это я понимаю, конечно.

\iusr{Оля Хабарова}
\textbf{Свитлана Полякова} 

немцы даже называют украинцев гостями, а не беженцами. Принимают, как членов
семьи. Это потрясающе, на самом деле

\iusr{Marina Novoselskaya}
\textbf{Svitlana Polyakova} 

я вас розумію, я думала, якщо поїду, буде трохи легше морально, а вийшло так,
що мені ще важче, хоча дякувати Богу, в нас неймовірна кількість прекрасних
людей, які підтримують та допомагають. А додому тягне нереально. Коли бачу, що
у людей розбиті домівки і їм нема куди вертатися, мені аж ножем по серцю (

\end{itemize} % }

\iusr{Olexandr Burlaka}

Метро \enquote{Тракторний завод} Якщо вийти з нього, то жодної зруйнованої будівлі
знайти важко. Навіть розбите скло припадає переважно натабачні кіоски, що стали
жертвою мародерів

\iusr{Andrey Bogdanovich}

С первого дня войны живу дома, на салтовке, студенческая. В подвале провел
только первый день, только день, ночевали уже дома. Если чесно, то я вообще не
понимаю как в метро можно столько высидеть?! У кого психика слабая давно
уехали, мои жена и дочь в их числе, в глубь страны, где потише. Мы с мамой
остались. Жизнь продолжается, а не рабочее метро приносит определенные
неудобства, и вспышки какой-нибудь заразы вполне реальны! Стопроцентно
согласен! Рискуем нажить еще проблем. Короче говоря я двумя руками \enquote{за}!

\begin{itemize} % {
\iusr{Elena Hizhnaya}
\textbf{Андрей Богданович} 

я тоже на Студенческой живу, но временно уехала. Скажите пожалуйста, там же уже
не так страшно? Очень хочу домой, но боюсь.

\iusr{Elena Hizhnaya}
\textbf{Андрей Богданович} 

в первый день я спустилась в метро, увидела толпу людей с животными, детьми на
полу и у меня начался приступ панической атаки. Я не смогла там находиться
из-за концентрации страха. В подвале тоже не ночевала, там холодно и сыро

\iusr{Микита Соловйов}
\textbf{Андрей Богданович} Вот спускаться или нет в подвал, я отказываюсь обсуждать. Тем более на Салтовке.

\iusr{Elena Hizhnaya}
\textbf{Микита Соловйов} я читала что в подвале нужно иметь два выхода, иначе это братская могила будет. У нас часть подвала под какой-то спортзал захвачена

\iusr{Ирина Кравченко}
\textbf{Elena Hizhnaya} Живу недалеко от Студенческой. Недавно в районы поблизости на Академика Павлова было несколько прилетов. Разнесло несколько домов в частном секторе. Бахает очень сильно постоянно. Хотя сегодня по городу ездила, в других районах было тихо.

\iusr{Elena Hizhnaya}
\textbf{Ирина Кравченко} я только собралась домой ехать, прочитала про случай на новой почте и осталась. Хотя каждый день думаю о городе. Не хочу нигде жить кроме Харькова

\iusr{Viktoriia Levizka}
\textbf{Elena Hizhnaya} там и не было ТАК страшно. ул Барабашова. пески

\iusr{Andrey Bogdanovich}
\textbf{Елена Хижная} 

страшно, потому-что прилетает, и ты не знаешь куда прилетит. Но прилетает везде, безопасных мест в городе нет, с точки зрения прилета. Просто нужно принять это и жить дальше, насколько это возможно в этой ситуации. Переходить дорогу иногда тоже страшно, масса случаев тому подтверждение. Но страх этот притуплен, т.к. мы не обращаем на него внимания в силу его повседневности, примерно так-же нужно поступить со страхом прилета, прилет так-же возможен как и вылетевший на красный автомобиль, нужно просто научиться воспринимать сложившуюся реальность, ее мы уже имеем и нам с ней жить.

\iusr{Elena Hizhnaya}
\textbf{Андрей Богданович} спасибо большое за ответ.

\iusr{Andrey Bogdanovich}
\textbf{Елена Хижная} не за что. Учимся жить в новой реальности! Израиль в ней давно живет. Привыкли.

\end{itemize} % }

\iusr{Viktoriia Levizka}
на Студняке то же люди нормально живут

\begin{itemize} % {
\iusr{Elena Hizhnaya}
\textbf{Viktoriia Levizka} как я рада это слышать!!! Я тоже оттуда

\iusr{Viktoriia Levizka}
\textbf{Elena Hizhnaya} 

иногда проблема с коммуникациями. отопление, газа не было 4 дня. ни разу не
ходили даже в подвал. во двор было 2 прилета еще в начале. все. а про НП - то
была дальняя. это очень редко и нигде не застраховано(( просто не нужно торчать
на открытых местах. даже в квартирах правило второй стены работает

\iusr{Elena Hizhnaya}
\textbf{Viktoriia Levizka} спасибо большое за ответ!

\iusr{Viktoriia Levizka}
\textbf{Elena Hizhnaya} 

у меня все на Салтовке. в районе Краснодарской и то было более стремно, чем в
районе Студняка. на Мавзолее, еще меньше проблем. только вода было с пониженным
тиском. ну и какое-то время ночевали в коридоре

\end{itemize} % }

\iusr{Олена Монова}

Все логічно і правильно, але гнилі помідори ловитимеш все одно

\iusr{Микита Соловйов}
Ну если бы я этого боялся, то не писал бы примерно 90\% текстов )

\iusr{Игорь Дубровський}

Групу потрібно створювати. Не мережеву. Місто, метро, нуо. Чия ініціатива буде
сприйнята? Не знаю. Але згоден на сто відсотків

\iusr{Yulia Vepritskaya}
Поддерживаю. Их нужно вытаскивать оттуда и эвакуировать желающих.

\iusr{Anton Bondarev}

Ууу.. От чую... Шо скоро набегуть, обвинять, проклянуть, изувером и аспидом
окоянным кликать стануть, у изуверы запишуть.Ну и вестимо, опосле сожгуть твоё
чучело... На рельсах промеж станций @igg{fbicon.face.wink.tongue}{repeat=3} 

\begin{itemize} % {
\iusr{Svitlana Polyakova}

Єто не смешно, и нет причин для стеба.

\iusr{Anton Bondarev}
\textbf{Svitlana Polyakova} 

да ни кто не стебеца. это скорее грустная ирония. Тема актуальная и для города
злободневная. Но я на днях на странице знакомой которая написала на эту же тему
столько проклятий, оскорблений и угроз прочёл от оппонентов что был в шоке...

\iusr{Микита Соловйов}
\textbf{Антон Бондарев} Удивляет, что до сих пор не.

\iusr{Svitlana Polyakova}
\textbf{Anton Bondarev} Просто не пришел еще тот....

\iusr{Anton Bondarev}
\textbf{Микита Соловйов} 

может просто потому шо у Вашего Благородия слишком уж интелегентная и утонченно
изысканная публика/почитатели?  @igg{fbicon.beaming.face.smiling.eyes}{repeat=3} Вот и хамить и дерзить не изволят @igg{fbicon.face.wink.tongue} 

\iusr{Микита Соловйов}
\textbf{Антон Бондарев} А еще очень помогает висящий у входа топор и большая табличка \enquote{Осторожно! Здесь могут послать нахуй!}
\end{itemize} % }

\iusr{Женя Дисс}

У нас одного разу зламалась автівка, десь на Сході. Зима, холодно, не супер,
десь -5.

За кілька хвилин біля нас зупинилась автівка, люди запропонували допомогу( і
допомогли). Пропонували пересісти до них, погрітись.

І я зрозуміла, як люди замерзають насмерть, коли загубились, наприклад.

Бо мені так не хотілось виходити, взагалі змінювати позу. Я навіть розуміла, що
так не можна, але виходити із зони дискомфорту не було ніяких сил.

І це відбулося зі мною за 20 хвилин. А люди так сидять місяць майже. Це дупа.
Але потрібно дуже акуратно їх виводити тепер

\iusr{Оксана Масалітіна}

Я узнала что оказывается в Киеве, где можно сказать вообще спокойно, и уж точно
намного безопаснее пойти домой, собрать вещи и сесть потом на поезд в
эвакуацию, так вот, что в Киеве неделями в метро живут люди. Я не представляю
что там у них с психикой и как это вообще потом можно будет починить ((( но с
этим точно нужно что-то делать.

\iusr{Olena Kryzhanivska}

Абсолютно аналогічна ситуація у Києві. І транспортні проблеми через
розірваність транспортних потоків величезного працюючого міста, та ще й з
непрацюючими мостами через Дніпро! І таки-да дикі проблеми з психікою у тих,
хто там живе другий місяць. І санітарні проблеми. Дякую за підняття цієї теми.

\begin{itemize} % {
\iusr{Микита Соловйов}
\textbf{Олена Крижанівська} 

Честно говоря, в Киеве для меня это вообще непонятно. Там же реальная угроза
даже по сравнению с Харьковом микроскопическая. Но лезть со своими советами к
киевлянам я не буду, сами без меня разберутся.

\iusr{Оксана Масалітіна}
\textbf{Olena Kryzhanivska} 

от я взагалі не розумію нащо у Києві жити тижнями у метро((( ну буває лячно,
буває дуже дуже лячно, особливо якщо нема кому надати психологічну допомогу чи
\enquote{сусіди по платформі} накрутять. Але можна ж поїхати з міста. Там вже матраци,
намети (!) поставили і оселилися ледь не назавжди.

\iusr{Roman Frolov}
\textbf{Olena Kryzhanivska} в Киеве вроде метро работает по одной стороне как челнок, но не на всех линиях

\iusr{Olena Kryzhanivska}
\textbf{Микита Соловйов} 

Саме тому і дякую. Що без порад, але тема - архіважлива. Я розумію, що Києву ще
дістанеться десь наприкінці квітня - орки будуть біснуватися до свого
прибацаного \enquote{пабєдобєсія}. Але все одно - уявити в Києві людину з нормальною
психікою, що сидить другий місяць у метро, дуже важко. Ще раз вдячна за допис
та тему, про яку всі бояться згадати.

\iusr{Оксана Масалітіна}
\textbf{Микита Соловйов} все правильно ты говоришь
Но они как зашли туда в конце февраля, так и живут там

\iusr{Оксана Масалітіна}
\textbf{Roman Frolov} поезда раз в час, не останавливается на станциях в центре города и нет сообщения левого берега с правым.

\iusr{Roman Frolov}
\textbf{Оксана Масалітіна} 

ну вот у нас хотя бы начать - по самой длинной линии, чтоб обязательно была
остановка на южд, это сразу уберет проблему как вывозить людей с района хтз и
дальше на вокзал

\iusr{Оксана Масалітіна}
\textbf{Roman Frolov} да, с пролета и ХТЗ на вокзал или работу добираться это вообще жесть

\iusr{Марійка Мороз}
\textbf{Оксана Масалітіна} намети у нас в метро теж стоять! Побачила вчора фотографію м.Київська - офігєла.

\iusr{Марла Сингер}
\textbf{Олена Крижанівська} 

алкоголт з 1 квітня (якщо не жарт, звісно) запустять, може й метро хоча б не
раз в півтори години... бо кажуть працювати бізннсу а як

\end{itemize} % }

\iusr{Svitlana Polyakova}

Конечно, их нужно вьіводить из метро. Главное сейчас - не навредить больше! Єто
огромная проблема-куда везти? В Закарпатье - негде жить. Нужньі временньіе лагеря,
В Польшу - переполнена.

И добавится к имеющимся психическим проблемам, еще и стресс при перемещении + язьіковьій барьер.

\begin{itemize} % {
\iusr{Оксана Масалітіна}
\textbf{Svitlana Polyakova} да, это уж точно не те люди, которые легко приспособятся к жизни в другой стране

\iusr{Микита Соловйов}
\textbf{Свитлана Полякова} 

Негде жить" по сравнению с чем? Наши ребята занимаются расселением на западе
Украины. И условия далеко не 5*. Но уж точно в разы лучше, чем в метро.

А Польша, Чехия, Германия и т.д. это возможность вообще возвращаться к
нормальной жизни. Там везде сейчас дают право на работу.

\iusr{Elena Babenko}
\textbf{Микита Соловйов} 

право то дають, але роботи в Польщі немає - тільки для чоловіків. Або саджати
дерева. Не все так райдужно в Польщі

\iusr{Svitlana Polyakova}
\textbf{Микита Соловйов} 

Ладно, подкину вам \enquote{веселой жизни в ШВЕЙЦАРИИ}.

Группа -\enquote{Українці в Швейцарії}

Окунитесь в среду .\url{https://www.facebook.com/groups/679158755543458/?ref=share}

\iusr{Tetyana Sfandex}
\textbf{Микита Соловйов} 

\obeycr
право на работу не равно получить работу
Но ещё примерно месяц - и можно будет жить в летних домиках и даже палатках до осени
На дачах всяких
Собственно мы жили около Днепра на дачах в начале матра когда выехали
Да +13 в домике вначале, но явно не хуже чем метро
Знакомые тоже в летних домиках на ЗУ сейчас
Уже нет -15 ночью
\restorecr

\iusr{Svitlana Polyakova}
\textbf{Микита Соловйов} Єто вьі пишите, не оказавшись в \enquote{шкуре} беженца.
Страньі разньіе и условия пребьівания тоже разньіе

\iusr{Viktoriia Levizka}
\textbf{Микита Соловйов} вот только работы нет. та, что до войны стоила 1200-1300, теперь за 500 с руками и ногами((

\iusr{Olga Geraschenko}
\textbf{Свитлана Полякова} расскажите про свой опыт. Очень интересно

\iusr{Svitlana Polyakova}
\textbf{Olga Geraschenko} 

Я вьіпадаю из \enquote{правила}.

У меня есть опьіт 6 лет проживания в Женеве, в среднем по полгода в году. Моя
дочь 7 лет работала в Женеве, а я приезжала периодически помогать ей с
ребенком.

Т. к у дочери много друзей мьі не имеем проблем с проживанием.

Дочь с внуком, так сложилось, приехали сюда раньше.

А я вьіезжала из Киева 20.03.

Автомобилем до Львова. Там поезд на Пшемьісль. А оттуда 24 часа автобусом(за
свои деньги) до Женевьі.

Живем у друзей, на программу S подали, но швейцарцьі все умеют считать, и я еще
не встречала человека, которьій получил интервью, по программе для
временноперемещенньіх лиц.

Некоторьіе ждут уже 4недели.

При єтом не знаю, за что они питаются. Но в лагерях для перемещенньіх лиц как -
то кормят.

\iusr{Svitlana Polyakova}
В разньіх кантонах по разному.

\iusr{Olga Geraschenko}
\textbf{Свитлана Полякова} спасибо. Очень познавательно

\iusr{Микита Соловйов}
\textbf{Свитлана Полякова} 

Так я никого не призываю уезжать, кажется. Я вот сижу в Харькове и мне, честное
слово, хватает чем заниматься. Но аккуратно позволю себе предположить, что
найти работу в Швейцарии сейчас не сложнее чем в Харькове.  @igg{fbicon.smile} 

\iusr{Oleksii Fesenko}
\textbf{Svitlana Polyakova} Румыния, Венгрия, Болгария...

\iusr{Svitlana Polyakova}
\textbf{Микита Соловйов} 

Да, владеющим франц-немецким язьіком ,реже англ ,имеющим соответствующее
образование или работу, которую можно делать руками -маникюр, парикмахер и
прочее.

\iusr{Svitlana Polyakova}
\textbf{Олексій Фесенко} Согласна ,при условии наличия работьі.

\iusr{Женя Дисс}
\textbf{Svitlana Polyakova} Польша знаходить місця все ще. Ми туди возимо.

\iusr{Маша Бахтігозіна}
\textbf{Микита Соловйов} 

какую работу? Нелегальную? Ок, большой риск вылететь из страны сразу после того
как Вы предложите кому-то работать без оформления. Легальную? Без языка
довольно проблематично и нужно ждать, пока получишь право на работу. Мы
приехали в Дрезден 16 марта, регистрацию мне назначили на 20 мая. До этого я и
рыпнуться в сторону работы не могу. Выучить польский - ок, могу себе
представить. Выучить немецкий или французский за эти два месяца - ну Вы сами
понимаете...

\iusr{Микита Соловйов}
\textbf{Маша Бахтігозіна} 

Естественно, я говорю о легальном трудостройстве. И естественно, пока не выучен
язык, ни о какой работе по специальности или около речь не идет. Но
невозможность получить работу по специальности и близко не равно невозможности
получить какую-то работу вообще. Низкоквалифицированной (и низкооплачиваемой,
конечно) работы же, не требующей особенного общения за пределами пиджин, по
моей информации в большинстве европейских стран хватает.

\end{itemize} % }

\iusr{Ирина Федотова}

\obeycr
Згодна на всі 200\%.
Я працюю із дітками в метро.
Це просто жахливо, в яких вони там в і умовах.
Треба негайно зачиняти метро у якості сховища, та відкривати у якості транспорту.
Більш того.
Майже всі мешканці метро вдень зараз чимчикують додому і повертаються тільки на ночівлю.
Чому?
Бо в метро завжди є їжа, є корм для тварин, є безкоштовна електроенергія.
Справді наляканих людей, які бояться звідти виходити - одиниці.
Більшості там просто зручно і безплатно.
\restorecr

\begin{itemize} % {
\iusr{Светлана Прокопенко}
\textbf{Ирина Федотова} 

ще можливо що деяким мешканцям насправдi краще спати в метро, бо вдома вони всю
нiч в станi напруги та панiки чекають на на бомби та ракети.

\iusr{Женя Дисс}
\textbf{Ирина Федотова} якщо так, то добре. Нехай людям буде зручно і безплатно. Головне- щоб не вивчена безпомічність.

\iusr{Маша Бахтігозіна}
\textbf{Светлана Прокопенко} 

я не могла спати вдома. Просто уявляла собі, що за це доведеться заплатити
власним життям або життям моїх дітей і не могла залишитися. Правда, усього 4
дні, але нам просто пощастило, що друзі запропонували виїхати до Мерефи. А так
-не знаю, бігти було нікуди і нічим.

\iusr{Ирина Федотова}
\textbf{Светлана Прокопенко} 

розумієте у чому справа... Тепер вся Україна вимушена буде жити в постійному
очікуванні бомбардування, як зараз Ізраіль, бо в нас ж божевільний сусід з
купою боєприпасів. Але ж неможливо постійно використовувати метро, як
сховище...

Метро потрібно місту, аби працювати. Багато хто з бізнесу готовий відкриватися,
але вони елементарно не можуть зібрати на роботу працівників бо ті не можуть
дістатися до роботи.

Треба шукати інші варіанти. Бомбосховища у школах та ін. А \#метроповиннопрацювати

\iusr{Ирина Федотова}
\textbf{Женя Дисс} 

ви не розумієте, що на \enquote{безкоштовно} потрібні чиїсь гроші?! Принаймні, гроші
громади або міста. Якщо в місті не може працювати бізнес з-за того, що
працівники не можуть дістатися до роботи, це \enquote{безплатно} може дуже швидко
закінчитися.

\end{itemize} % }

\iusr{Igor Feldman}
Без транспорта Харьков не оживет. А никакого транспорта сейчас нет.

\begin{itemize} % {
\iusr{Микита Соловйов}
\textbf{Игорь Фельдман} Это отдельная тема. И через день-два подробно распишу.

\iusr{Igor Feldman}
\textbf{Микита Соловйов} 

наземный транспорт сейчас проблема, и потому, что это опасно, и потому, что
судя по всему его стало сильно меньше из-за обстрелов, в том числе депо. Так
что только метро сейчас способно выполнять роль городского транспорта. А оно
стоит. Значит стоит и бизнес, который мог бы работать

\end{itemize} % }

\iusr{Наталія Пахніна}
Підтримую, спілкувалась з тими, хто сидить в метро, наявні ознаки психічних розладів в них є

\iusr{Anna Slavutskaya}
\textbf{Наталія Пахніна} Може, це не причина, а наслідок (в метро залишилися тількі супертривожні люди)

\iusr{Наталія Пахніна}
\textbf{Anna Slavutskaya} , можливо

\iusr{Анна Кошелева}

Харькову нужен наземный и подземный транспорт! Людям-подземникам помощь в
психиатрическом лечении и социальной адаптации. Хотя кто этим будет заниматься.
Моим пенсионерам горсовет всё никак не довезет обещанную гумпомощь и валерьяну
с цитрамоном..

\iusr{Ірина Форстер}
Абсолютно согласна! Не такой страшный черт как его рисуют.

\iusr{Виталий Буняев}

Кстати, о санитарии.

Пункты выдачи гуманитарки нужно оборудовать передвижными биотуалетами в
достаточном количестве или чем-то подобным. Потому что одновременные 500-1000
человек, стоящие по нескольку часов в очереди, и ежедневные несколько тысяч,
проходящие через пункт, превращают в адовый звиздец окрестные улицы, переулки и
дворы. Как минимум, в общественный сортир и свалку. И так изо дня в день, из
недели в неделю.

Образовалась целая прослойка, которая кочует от пункта к пункту, а потом с
благоприобретенным спускается в метро и там просто живет.... их легко опознать
со стороны.

С этим всем точно нужно что-то делать.

\iusr{Anna Korol}
Ви очевидець, ви там живете, все бачите, значить так і є.

\iusr{Светлана Иванова}

Ч полторы недели просидела в метро, но я выходила на несколько часов домой.) В
метро ходила только на ночь, не могла заснуть под гупанье

\iusr{Dmytro Kurilo}

А какая подготовка нужна для того, чтобы запустить метро? Я думаю в случае
штатной работы эволюционно решаются сразу обе проблемы: 1. Метро как транспорт
реально нужен городу. 

2. \enquote{Сидельцы} за пару дней, посмотрев на живых людей, сами потихонечку
начнут выдвигаться к своим домам или к эвакуаторам.

Что-то мне подсказывает, что немногих оставшихся будет легче социализировать.

В принципе, это тоже самое, про что и пост, но только чуть меняется порядок
действий.

Но, в принципе, поддерживаю автора обеими руками - зачем оттягивать
неизбежное?!

\begin{itemize} % {
\iusr{Ерофей Подольский}
\textbf{Dmytro Kurilo} надо сделать как в Киеве. Метро работает днём.

\iusr{Dmytro Kurilo}
\textbf{Ерофей Подольский} Для старта - неплохой ход

\iusr{Микита Соловйов}
\textbf{Дмитрий Курило} Нет, так не получится. Для того, чтобы возвобновить работу метро, придется СНАЧАЛА выселить сидельцев. И там будет еще не на один день работы после этого.

\iusr{Dmytro Kurilo}
\textbf{Микита Соловйов} не поясните? Разве метро нельзя запускать по частям? Линиями, станциями? Другие варианты?

\iusr{Ерофей Подольский}
\textbf{Микита Соловйов} тем не менее в Киеве так всё работает. Есть люди которые сидят постоянно, есть люди которые ночуют. Метро работает только по одной стороне, вагон едет раз в час...

\iusr{Olga Suvorova}
\textbf{Дмитрий Курило} люди в том числе и в вагонах живут, которые стоят на станциях.

\iusr{Dmytro Kurilo}
\textbf{Olga Suvorova} Ужас какой. Перебор...
\end{itemize} % }

\iusr{Viktoria Nesterenko}
Підтримую на всі 100\%! І дякую, що пишете про це  @igg{fbicon.smile} 

\iusr{Irina Nesvitaylo}

Поддерживаю, надо запускать метро, это самый безопасный и быстрый транспорт для
Харькова. Но там есть люди, которые лишились жилья совсем....с ними будет
проблема, куда им идти?

\begin{itemize} % {
\iusr{Tetyana Sfandex}
\textbf{Irina Nesvitaylo} к друзьям, знакомым. Спрашивать кто может пустить. Многие выехали и в принципе жильё по городу есть.

\iusr{Микита Соловйов}
\textbf{Ирина Несвитайло} Если людям негде жить в Харькове, то тогда уж лучше в эвакуацию, чем в метро.
\end{itemize} % }

\iusr{Natalie Neviasky}

Микutа, я пережила длительную бомбардировку в Израиле в 91м году (но у вас
ситуация намного хуже сейчас). Этих людей просто так уговорить в этой стадии
будет крайне трудно. Здесь нужно собрать оставшихся психологов и психиатров, и
всех туда забросить. Большинству из находящихся там сейчас уже необходима
психологическая и психиатрическая помощь.

\begin{itemize} % {
\iusr{Микита Соловйов}
\textbf{Natalie Neviasky} Я человек довольно резкий. И в данном случае говорю не столько об убеждении, сколько о принудительном выселении. Естественно, эту неделю дать на то, чтобы свыкнуться с мыслью.
Да, и я Микита, а не Микола.

\iusr{Natalie Neviasky}
\textbf{Микита Соловйов} izvinite, ya znayu- typo (typing through Translit- no Cyrillic here - at work)

\iusr{Natalie Neviasky}
\textbf{Микита Соловйов} im nuzhna psihologicheskaya pomoshch, a ne rezkost', oni navernoe tam vse s det'mi ili starikami
\end{itemize} % }

\iusr{Лена Полторацкая}

Встречаю в саду Шевченко жителей метро. Бледные, с потеряным
взглядом, оцепенелые. Им реально необходимо выходить наружу, участвовать в любых
процессах.

\iusr{Boris Sevastyanov}

На Гертруде был сегодня 2 часа. Ну да, слышно, и приходы и уходы. Но
адаптируется быстро. Главное научить людей различать. А так - тишь, благодать.

\iusr{Дмитрий Баевский}
\textbf{Boris Sevastyanov} на северной 2 люди живут. Не только в убежище. Но то уже - сами говорят, на Бога полагаются..

\iusr{Boris Sevastyanov}

И ещё говорили сегодня как раз про запуск метро. Хоть раз в час электричка.
Людям по делам надо иногда, что-то открывать заново, работать. На такси не
наездишься. Хотя вчера сегодня и в этом смысле не ощутил какой-то обдираловки.
Нормальные цены в онтакси. Но, это уже мой вопрос нелюбви к метро )

\iusr{Микита Соловйов}
\textbf{Борис Севастьянов} 

Там масса своих технических вопросов будет с запуском. Обсуждаем несколько
последних дней. Но пока этот режим убежища не отменить, запуск в принципе
нереален. В реверсном варианте как в Киеве шансов нет чисто технически на двух
из трех веток.

\iusr{Igor Prykhodko}

Есть одна инфекционная хрень, которую никто не отменял: ковид. Хоть он сейчас и
не такой страшный, как в начале пандемии, но всё равно доставляет много хлопот
врачам, оставшимся в городе. Плюс сезонные вирусные инфекции, которые могут
быть очень неприятными. Плюс обычая бытовая антисанитария...

Хотя бы с этой точки зрения с убежищами в метро надо что-то решать.

\begin{itemize} % {
\iusr{Микита Соловйов}
\textbf{Игорь Приходько} 

"Количество новых случаев ковида у Украине достигло значения "абсолютно похуй!" (с)

Но я согласен, что и ковид, и масса других заболеваний в тех условиях могут
дать очень сильную вспышку. Об этом и пишу. Это же чашка Петри в чистом виде.

\iusr{Roman Frolov}
\textbf{Igor Prykhodko} 

сложно сказать, есть ли ковид в наших уловиях сейчас, или нет. Но длительное
пребывание в подвалах или метро еще скажется на здоровье (возможно чуть позже),
сейчас экстремальная ситуация придает сил на короткий период. Я про город пишу,
где есть электрика и тепло в подвалах.

\iusr{Igor Prykhodko}
\textbf{Микита Соловйов}, 

вот это вот \enquote{абсолютно похуй} скажите, пожалуйста, заведующим терапевтическими
отделениями в двух-трёх работающих стационарах, а потом быстренько бегите, пока
вам глаза не выцарапали. А ещё лучше пообщайтесь с врачами. Это серьёзнее, чем
вам кажется. И не \enquote{может дать}, а уже. Просто страшных поражений лёгких меньше.

Ковид нынче имеет неприятные осложнения, которые отражаются на
работоспособности. Нам сейчас эта двойная нагрузка на психику точно не нужна.

Я сегодня был в \enquote{Росте} и не мог не заметить, что очень многие в
масках. В нашем АТБ тоже. Люди вспоминают об этой проблеме.

\iusr{Марійка Мороз}
\textbf{Igor Prykhodko} 

я сегодня тоже в РОСТе была (возможно, в другом). Масок не увидела ни на ком,
зато нарешті увидела женщин с помадой на губах. И такое свидетельство весны и
побеждающей жизни мне очень по душе.

\iusr{Микита Соловйов}
\textbf{Игорь Приходько} Не поверите, с врачами сейчас общаюсь регулярно. В основном из еще работающих стационаров.

\iusr{Igor Prykhodko}
\textbf{Микита Соловйов}, 

не поверю. Точнее, поверю, но немного опасаюсь вашей привычки обращать внимание
только на то, что сейчас находится в \enquote{светлом поле} вашего сознания. А я вырос
в абсолютно медицинской семью (вдобавок у папы была специальность \enquote{военная
санитария}) и воспринимаю такие вот визиты политиков немного иначе.

Просто хотел попросить, чтобы вы внимательнее слушали тех, с кем общаетесь @igg{fbicon.wink} 

\iusr{Микита Соловйов}
\textbf{Игорь Приходько} 

вопрос не в светлом поле и т.д. А в том, что сейчас существует до черта
значительно более опасных вещей. И в готовности признать, что в обозримом
будущем безопасности нет и не будет. Да, жить в Харькове опасно. Да, есть риск
заболеть, получить ранение, погибнуть. Это фон. И дальше все строится из оценок
рисков, не имеющих ничего общего с уровнем оценки рисков мирного времени. И
когда я разговариваю с врачами, то в 90\% случаев я не о скитаниях вечных и о
земле с ними говорю, а о потребностях их в каких-то вещах. Не для красивого
быта, а без которых умрет за следующую неделю столько-то человек. Одни риски
всегда нужно сравнивать с другими рисками. И вот я о том, что риск ковида в
сегодняшней ситуации перестал быть одним из главных.

\iusr{Igor Prykhodko}
\textbf{Микита Соловйов}, 

не уверен, что у кого-то сейчас есть полная статистика. Судя по тому, что время
от времени рассказывают мне, ковид отодвинулся с первого места на второе. На
первом сердечно-сосудистые. На третьем - бытовые травмы. На четвёртом -
\enquote{специфические повреждения} (ранения в результате обстрелов).

Насчёт смертности не знаю. И знать не хочу, если честно.

\iusr{Микита Соловйов}
\textbf{Игорь Приходько} 

Так а есть еще большая группа причин смерти, связанных с эффектами второго
порядка от войны. Вот простой пример. Сколько людей умрет от того, что в городе
не работает общественный транспорт? Уверен, это довольно значимое количество.
Потому что это кратно снижает возможность занятости. Кратно снижает
мобильность, в том числе по некритичным показаниям. Кратно затрудняет
возможность завезти продукты дальним родственникам для человека без авто.

Я упорно об одной вещи пишу. Вот только что даже отдельный пост накатал
здоровый. Безопасности в понимании мирного времени нет и не будет до конца
войны. Это не гипотеза, это окночатлеьное утверждение. И нужно вообще все,
связанное с безопсаностью, переставать мерять мерками мирной жизни.

\iusr{Igor Prykhodko}
\textbf{Микита Соловйов}, 

уже не меряют. Все прекрасно понимают, что сердечно-сосудистое сейчас
обостряется в первую очередь не в силу типичных причин. В частности, мой
вчерашний гипертонический криз был обусловлен... терактом в Рамат-Гане и
Бней-Браке.

Парадоксальным образом плохие новости издалека (Мариуполь, Киев, Изюм, Северная
Салтовка для обитателя центра) действуют дольше и, увы, сильнее, чем взрыв
где-то неподалёку. При том условии, разумеется, что не задело осколком и не
травмировало взрывной волной.

В Харькове был выдающийся медик по фамилии Генис. В последние годы жизни он
преподавал скучную медицинскую статистику, но превратил это в нечто
увлекательное. Я тогда был в щенячьем возрасте, но любил его слушать просто во
дворе или за чаем. Так вот, он бы вам наверняка сказал, что от отмены
общественного транспорта не умер и не умрёт никто. И устанавливать такие связи
неграмотно. Почему? Одним из его любимых тезисов было то, что он нажатия курка
никто не умирает. Нет ПРЯМОЙ связи между нажатием курка и чьей-то гибелью, но
может быть прямая связь между нажатием курка и выстрелом. Для статистики очень
важно ГРАМОТНО (профессионально) выстраивать систему связей и зависимостей.

\iusr{Микита Соловйов}
\textbf{Игорь Приходько} 

Прочтите, но это уже чистая игра в слова. Каждая сотня недостающих рабочих мест
это жизни. Каждый миллион гривен это жизни. И так далее.


\end{itemize} % }

\iusr{Marina Novoselskaya}

Я трохи розумію цих людей, поки ми були вдома, то почувалися набагато краще,
ніж коли перебралися у підвал, як це не дивно, бо у підвалі дійсно безпечніше.
Але... після короткого перебування у підвалі нам стало морально важче
знаходитися вдома, переслідували думки, що ми щось робимо не так, що наражаємо
себе на небезпеку, ще й рідня почала атакувати нас, що треба евакуюватися.
Морально це підкосило чи не більше, ніж вибухи авіабомб ( З іншого боку - а як
взяти на себе відповідальність за таке рішення - хто зна, чим закінчиться
закриття метра як укриття для тих людей, що там зараз. Важке рішення, дійсно
важке

\iusr{Ірина Гаєва}
Согласна

\iusr{Ольга Ладия Щербакова}

Все верно. Чем дольше такой образ жизни, тем больше последствий для психики.
Определенная степень директивности в такой ситуации допускается - объявить
решение, дать время на сборы ( принятие информации) и реализация. По хорошему
раздать контакты для оказания психологической помощи. Например,

\url{https://www.facebook.com/ildcua/photos/a.1612616195655681/3014569788793641/?type=3}

\ifcmt
  ig https://scontent-mxp1-1.xx.fbcdn.net/v/t39.30808-6/277297785_3014569782126975_4736528590093258169_n.png?_nc_cat=100&ccb=1-5&_nc_sid=730e14&_nc_ohc=H1NzpVHL0o0AX9Qupw_&_nc_ht=scontent-mxp1-1.xx&oh=00_AT-jeBGcrhPxnJ4Y_iq5bBsLqUlBE0_osi_QbEOz9se7OQ&oe=626D1FAB
  @width 0.3
\fi

\iusr{Anna Pirozhenko}
А что по этому поводу думает Терехов?

\begin{itemize} % {
\iusr{Микита Соловйов}
\textbf{Анна Пироженко} Науке в моем лице не известно.

\iusr{Микита Соловйов}
Выяснил. Терехов гордится метро-убежищами и призывает всех в них жить.

\ifcmt
  ig https://scontent-mxp1-1.xx.fbcdn.net/v/t39.30808-6/277534031_7323789217691428_3122316868394712898_n.jpg?_nc_cat=106&ccb=1-5&_nc_sid=dbeb18&_nc_ohc=phBbBM9SfP4AX-0F7S4&_nc_ht=scontent-mxp1-1.xx&oh=00_AT-EPDdu5rvxhERpcAnHsVxhXN-pIQbg32wW3rYuO_R8uQ&oe=626D6A23
  @width 0.2
\fi

\iusr{Anna Pirozhenko}
\textbf{Микита Соловйов} печально..

\iusr{Светлана Прокопенко}
\textbf{Микита Соловйов} он предлагает там ночевать, а не жить.
И возможно это правильно

\end{itemize} % }

\iusr{Viktoriia Levizka}

сейчас на верху безопаснее, чем под землей. тепло, птички поют, свежий воздух и
звуки в основном, нашей арты. и желательно поменьше общения в очередях

\iusr{Людмила Даниличева}
Я виїхала з Харкова в Лозову. Тут людей приймають і розподіляють. Навіщо їхати далеко?

\iusr{Микита Соловйов}
\textbf{Людмила Даниличева} 

Вариантов масса и каждый решает для себя сам. Но скажем так:вероятность
обострения и переноса боевых действий в район Лозовой заметно выше, чем в район
Праги.

\iusr{Людмила Даниличева}
\textbf{Микита Соловйов} 

Но до Праги еще нужно доехать. Мне, с моим здоровьем, и до Лозовой было
проблемно доехать.

\iusr{Виктория Любецкая-Котенко}

Привет от харьковчанки. Я не знаю как был налажен быт на других станциях метро
Харькова, но я обитала в метро 15 дней с начала войны. Жители метро
Алексеевская, исходя из рассказов в посте, вообще, походу, самые адекватные
пользователи метро как убежища. Т.е. пользовались как убежищем, а не постоянным
жильём. Как только комендантский час заканчивался, почти вся станция оставалась
пустой до начала следующего комендансткого часа. Ходили мыться домой, кушать
предпочитали дома, да и очереди за продуктами и лекарствами сами себя не
отстоят. Люди видели, что снаружи всё цело, просто в метро спать спокойнее,
потому что: в метро не слышно взрывов, метро хорошо охраняется ночью от
внезапных гостей и 100\% не оставят голодным. У нас на станции никто не
паниковал, на конфликты не триггерил, напротив, народ знакомился и
объединялся. Для тех же вариантов эвакуации. \enquote{Костяк} станции, люди, что там с
самого начала, это от силы человек 20, а так на станцию постоянно прибывали
новые люди и постоянно уезжали старые. С нашего метро собрали очень много групп
людей, которых по тоннелям метро проводили до вокзала. И люди сами просили
работников метро, а не наоборот, что работники метро за уши тащили перепуганных
людей по тоннелями на вокзал. В общем, люди сидели, но не без дела, а каждый
день узнавали способы уехать, помогали носить гуманитарку по тоннелям, дети
тоже не скучали, с дома многие поприносили книжки/настолки/игрушки, а когда
уезжали, оставляли их тем, кто ещё остался сидеть в метро. Так и я с мужем в
итоге, как только нашли знакомого, что согласился приютить в Днепре, а потом
нашли знакомого, согласившегося отвезти в Днепр, мы поехали. Нельзя закрывать
метро как убежище. А то и убежищ то факту не остаётся. В подвалы
нельзя-завалит, приспособденных бомбоубежищ почти нет, теперь ещё и метро
нельзя. Такое...я в метро хоть отдыхала и не слушала вои реактивных самолётов,
как моя мама, что оставалась в квартире. Вот её психологическое состояние точно
под вопросом. А я благодаря метро ещё держусь и чувствую себя в целом хорошо.
Метро изначально задумывались и как убежище. В Днепре за попытку сфоткать
станцию сразу прибежит охрана, т.к. важный объект, убежище, говорили мне ещё в
2018м. А теперь, когда их время поработать как убежище настало-предлагают их
так не использовать. Не вижу тут логики. Тут проблема не метро, а отдельных
людей у которых и в хорошее время психика была чувствительной. Повторюсь, на
Алексеевской никаких таких настроений замечено не было. Люди каждое утро
выходили погулять с детьми, собаками, покурить, в магазин сходить, никто в углы
метро не забивался безвылазно. И не страдайте от синдрома выжившего. Никакой
гарантии, что вот прям сейчас не прилетит ночью в ваш дом. Лучше пользоваться
убежищами и метро-отличное убежище, особенно если оно рядом с вами.

\begin{itemize} % {
\iusr{Roman Frolov}
\textbf{Виктория Любецкая-Котенко} 

не отменять как убежище а в дневное время запускать поезда - хотя бы по одной
стороне, так и до вокзала не надо будет пешком 10+ км идти по туннелю.

\iusr{Виктория Любецкая-Котенко}
\textbf{Roman Frolov} но судя по пунктам 1) и 6), автор предлагает именно закрыть как убежище и вот я пытаюсь защищать концепцию метро-убежища)

\iusr{Виктория Любецкая-Котенко}

кроме безопасного сна, во время воздушной тревоги, стоя в очереди в магазин я
скорее в метро побегу, чем к себе домой, хотя расстояние и до одного и до
другого, в моём случае, 5мин.

\iusr{Igor Solomadin}
\textbf{Виктория Любецкая-Котенко} абсолютно согласен с Вами. Прожил с собакой в метро на Студенческой неделю.
Но эвакуироваться по мере возможностей необходимо!

\iusr{Alena Zolotareva}
\textbf{Виктория Любецкая-Котенко} кстати, да, на Научке примерно так же )
\end{itemize} % }

\iusr{Володимир Скорик}

О, лише вчора-сьогодні говорив про це і з волонтерами, і серед своїх. Вже час
запускати метро.

Транспорт в місті зараз дуже необхідний.

Натомість метро поступово перетворюється в притон.

\iusr{Vladimir Brodetsky}
Абсолютно с Вами согласен.

\iusr{Konstantin Dvornichenko}

Сегодня работали в городе на улице Шевченко, я удивился насколько много людей
идут пешком по каким-то делам, а расстояния у нас в городе не слабые. Что бы
город начал оживать метро должно работать как метро. Для начала запустить
челноки, и сразу дети подземелья начнут видеть нормальных людей и постепенно
приходить в себя.

Так же сегодня опросил водителей Касанов, готовы ли они выйти на работу. Готовы
всё без исключений. Транспорт в Харькове должен начинать работать.

\iusr{Anna Vergeles}
согласна

\iusr{Светлана Прокопенко}

Единственное, я бы оставила возможность пережидать в метро воздушные тревоги и
ночевать с начала и до окончания комендантского часа.

\iusr{Борис Шестопалов}

Предлагаю рассмотреть также вариант переселения желающих в близлежащие села. Мы
здесь тоже живём жизнью, начали весенние работы, помогаем приезжим
обустроиться.

Такое решение имеет некоторые преимущества.

\begin{itemize}
  \item 1. Возможность обеспечить себе продуктовый безопасность.
  \item 2. Смена рода деятельности.
  \item 3. Весна на природе благотворно влияет на психику.
  \item 4. Возвращаться в Харьков, при желании, не далеко.
  \item 5. Человеки становятся более самостоятельными, обретают новые навыки и умения.
\end{itemize}

\begin{itemize} % {
\iusr{Tetyana Sfandex}
\textbf{Борис Шестопалов} смотря в какие, в некоторых россияне стоят до сих пор, а некоторые под угрозой оккупации

\iusr{Борис Шестопалов}
\textbf{Tetyana Sfandex} а некоторые на территории Украины под контролем ВСУ. Это жизнь и, похоже, гарантии отсутствуют.
\end{itemize} % }

\iusr{Наталья Безрученко}
УСЕ ВІРНО, але я не харків’янка, і мене закидають яйцями за мою думку

\iusr{Igor Onopchuk}

Думаю, пункт 3 можна влаштовувати шляхом примусового виходу зі станції для
прибирання і мінімальної стерилізації. Думаю, ні у кого не повинно виникати
проблем із тим, щоб станцію тупо прибрали.

Плюс, команду психіатричної допомоги.

За досвідом Києва: якщо метро запускати (періодично), то добре принаймні
розклад відправлення з кінцевих станцій оголосити. А то я 2 рази заходив якраз
в момент, коли поїзд відправився в ту сторону, куди мені потрібно було. 2
години на станції стояти точно не комільфо

\begin{itemize} % {
\iusr{Ірина Вихованець}

Якщо честно, то і в Києві було б не погано, щоб люди виходили (хоча тут такого
жаху як в Харкові не було, і метро хоч і в урізаному вигляді по маршруту, і з
очікуванням в 1.5-2 години, але ходить)

\iusr{Igor Onopchuk}
\textbf{Ірина Вихованець} ну важко це назвати «ходить»
\end{itemize} % }

\iusr{Alexander Bogatyrev}
+100500 городу сейчас нужно метро в режиме транспорта. И похоже срочно

\iusr{Ольга Закопайло}
Підтримую. Все логічно і розумно

\iusr{Маша Бахтігозіна}

Все прекрасно. ТОлько куда они эвакуируются? В Черновцы снимать квартиру за 600
долларов в месяц? Или во Львов за 900 долларов?

Страны ЕС - супер! Я сейчас в Германии: люди живут неделями в спортзалах, где
цветет буйным цветом ротавирус и откуда с очень большим трудом можно вырваться,
т.к. социального жилья нет, а тех денег, которое государство дает в качестве
компенсации аренды жилья, на жилье не хватает. Да, кому-то везет, а кто-то
ночует на вокзалах. Чем это лучше?

\begin{itemize} % {
\iusr{Olena Gritsay}
\textbf{Маша Бахтігозіна} 

в село. Я в селе сейчас. Земли дохрена, домов дохрена, селяне рады до усёру,
они аж оживают. Ой, работать надо? Дрова рубить, печку топить... Нееее, мы в
метро посидим, испортим психику детям, себе, поимеем артриты, и тд. Всруся, но
не покорюся @igg{fbicon.face.eyes.crossed.out}  @igg{fbicon.dizzy} 

\iusr{Анюта Тарасова-Черенкова}
\textbf{Маша Бахтігозіна} 

на заході облаштовують для біженців приміщення. В містах безкоштовно можна
отримати їжу кожного дня. Просто подати по приїзду документи. Існує державна
підтримка по житлу. Люди здають квартири біженцям безкоштовно, держава за
кожного дає 450 грн. Це звичайно не квартира за 900 доларів. Але люди можуть
жити. Звичайно хто хоче може і сам знімати квартири.

\iusr{Iryna Yahodina}
\textbf{Olena Gritsay} 

Добрий день. Будь ласка, якщо можете, напишіть в месенджер в якому селі ви, або
хоча б напрямок, де шукати. За кордони я не хочу, та й розумію, що й у наших
містах не маю змоги оселитися. Але ж, якщо крайній випадок таки настане, то
потрібно буде кудись їхати.

\iusr{Маша Бахтігозіна}
\textbf{Olena Gritsay} 

Ну, мы уехали после 4 дней сидения в метро в Мерефу. Поспали 1 ночь спокойно, а
на следующий день соседнее село разбомбили полностью (Яковлевка), а через
неделю и саму Мерефу разбомбили. Вы знаете какое-то село, в которое не
прилетит? Я не знаю. Мы уехали на запад, проехали Винницу ночью, а утром туда
прилетело.

Я не говорю, что надо сидеть в метро. Но мне кажется, что просто выгнать оттуда
людей на все четыре стороны - это не оптимальный вариант.

\iusr{Маша Бахтігозіна}
\textbf{Анюта Тарасова-Черенкова} 

\enquote{облаштовують приміщення} звучить дуже добре. Але в реальності це часто
спортзали. Або актові зали в школах, бо справжнього житла навіть на кшталт
гуртожитків не вистачає. Плюс в деякі не беруть з тваринами. Коротше, Ви все
правильно кажете, але я теж знаю, про що кажу, бо я в Дрездені і кожного дня в
німецьких групах читаю про те, в яких люди опинилися умовах. І я розумію, що
психологічно їм комфортніше в харківському метро. Бо не кожен витримає і шлях
на захід, і проблеми, з якими тут люди зіштовхуються.

\iusr{Микита Соловйов}
\textbf{Маша Бахтігозіна} 

Ну для начала. Во Львове, это конечно круто. Но вот в селах Под Львовом,
Хмельницким, Фраником и т.д. жилье стоит от \enquote{ничего} от переданного волонтерам
до 100 долларов. Наши ребята из Харьков-СОС расселяют эвакуировавшихся и таких
вариантов хватает. За 250 приятели сняли на две больших семьи большой
двухэтажный дом. И вариантов таком диапазоне море.

В Чехии от приятеля знаю, ситуация аналогичная. Если уперся и хочешь в Праге,
то в лучшем случае комната в общежитии. Если готов ехать туда, где предоставят
жилье, то социальная квартира на первые то ли два, то ли три месяца будет.
Право на работу оформляется заметно быстрее.

И да, прилететь в Украине может куда угодно. Вопрос только в вероятности и
плотности обстрелов. Если к этому не готов, то нужно выезжать из страны. И
найти вариант можно сейчас в очень многих странах. Но да, это будет не столица
земли, не Дрезден или Мюнхен, если мы говорим о Германии. Но явно на порядок
более комфортные условия, чем метро.

\iusr{Анюта Тарасова-Черенкова}
\textbf{Маша Бахтігозіна} 

виїхати в західну Україну краще ніж нічого. Розумію якщо без дітей, сидіти в
метро ще як ніяк можна. А з дітьми вже набагато тяжче. Так прилетіти може в
будь яке місце зараз, але всім або сидіти під бомбами, або виїхати закордон не
вдастся. Тому хтось таки залишиться, хтось поїде на західну і працює в тилу, бо
країна має жити, для тих хто в аду, і для економіки. А вже хтось і виїзжає за
кордон.

\iusr{Jane Dukhopelnykova}
\textbf{Iryna Yahodina} 

Спробуйте Кременчук, наскільки мені відомо, вони активно розсіляють по
гуртожитках і навколішніх селах, тиждень тому точно посілитися не було
проблемою. Ну так, прилетіти будь-де може, але тьху-тьху там тихо.

\iusr{Iryna Yahodina}
\textbf{Jane Dukhopelnykova} Кременчук - рідне місто моєї мами. Я там кожне літо проводила у бабусі. Але в селах навколо вільних будинків майже немає.

\iusr{Jane Dukhopelnykova}
\textbf{Iryna Yahodina} ну мені здається, гуртожиток теж краще чим метро... Добре, можливо я невірно зрозуміла запитання, в надто широкому сенсі

\iusr{Iryna Yahodina}
\textbf{Jane Dukhopelnykova} 

Я живу в Києві в своїй квартирі, але може ж бути будь-що. У нас околиці дуже
активно обстрілюються. На сусідню вулицю ще на третій день ракета ціла
прилетіла, відео і фото весь світ облетіли. Тому потрібно мати хоча б якийсь
план по евакуації.

\iusr{Andrey Lysachenko}
\textbf{Маша Бахтігозіна} 

Не преувеличивай на счет Мерефы. Повреждено меньше 2\% зданий. Я только что
созванивался с друзьями которые сейчас там находятся. Все гораздо спокойней чем
в Харькове

\end{itemize} % }

\iusr{Евгений Мякушка}

Метро городу нужно, как средство передвижения! Однозначно! Но,в метро сейчас
свозят возрастных одиноких людей с разбитых домов,без документов и денег. Кому
они нужны вне города? И если их вывозить далеко,нужно сопровождение! Но и метро
нужно...

\iusr{Alyona Krishtofovitch}

Розумію... Мені після шести днів в підвалі здавалося що вийти можна тільки щоб
приготувати їжу і скоро помитися. Щоб іти далі 20 м від під'їзду... ви шо! А
після трьох тижнів... Повернення до життя буде дуже болісним і важким(

\iusr{Vaclav Reset}

Київ вже давно цю лавочку прикрив.

За остані 2 тиждні вже стало зрозуміло, що ті, хто сидять в метро - розподілені
на 2 категорії, з них, одній просто нікуди вже йти, а другій - потрібен
психотерапевт.

Ці питання може вирішити тільки адмінистрація міста, бо метрополітен вміщує
мізерний процент мешканців, і ті декілька тисяч, що зараз там мешкають - проста
блокують пересування містом для всих інших сотен тисяч.

\iusr{Евгений Бурцев}

Могу много написать про жизнь на Героев, но не публично. Каждый день там рядом
нахожусь

\iusr{Наталия Новицкая}

Как хорошо писать сидя дома попивая кофеек, а люди лишились квартир, денег нет
,чтобы уехать, только не надо писать, что все бесплатно, заботливые вы наши, метро
единственное место где люди могут пережить эту беду, временные трудности, люди не
паникуют и не жалуются, зато у кого и вода и свет и интернет пишут ......,читать
стыдно!!!

\begin{itemize} % {
\iusr{Ольга Калениченко}
\textbf{Наталия Новицкая} сколько? А если это на полгода? А если на два года? Будете жить в метро?

\iusr{Наталия Новицкая}
\textbf{Ольга Калениченко} спросите у путина

\iusr{Ольга Калениченко}
\textbf{Наталия Новицкая} так не он же в метро живет...

\iusr{Наталия Новицкая}
\textbf{Ольга Калениченко} 

А почему люди живут в метро? Кто приказы отдаёт бомбить город? Или вы
думаете, что это игра на выживание, квест такой? Стыдно должно быть, хотя, какой
умишко,....

\iusr{Ольга Калениченко}
\textbf{Наталия Новицкая} все ясно. Беда.

\iusr{Наталия Новицкая}
\textbf{Ольга Калениченко} Действительно, беда, и вам всего хорошего

\iusr{Ольга Калениченко}
\textbf{Наталия Новицкая} срочно нужен кислород...

\iusr{Наталия Новицкая}
\textbf{Ольга Калениченко} Кому? вам? Все так запущено? Ну кислород не поможет!

\iusr{Elena Baliuk}
\textbf{Наталия Новицкая} а если это все затянется не на один месяц? Так и будут людьми подземелья? Сейчас школы пустуют, так может там поселить тех кто остался без крова

\iusr{Svitlana Kovalova}
\textbf{Наталия Новицкая} вайфай вижу работает

\iusr{Andriy Utevsky}
\textbf{Ольга Калениченко} 

А если, а если? Только месяц прошел, а уже ныть начали. У вас что, обычная
жизнь или война? Куда Вам там едить постоянно нужно? Я знаю там людей, у
которых убежищ в доме нет и им далеко бегать туда-сюда.

\iusr{Ольга Калениченко}
\textbf{Andriy Utevsky} 

войне не месяц, а 8й год) а разводить в метро антисанитарию это тупость, если
есть две ноги и мозг, и такой сильный страх, то нужно выехать в мирный регион,
а не жить грязным как бомж.

\iusr{Andriy Utevsky}
\textbf{Ольга Калениченко} 

Какие мы смелые. Харьеов не 8-й год бомбят. А в вас еще ничего не прилетело?
Сидите дома и мойтесь постоянно. Свой мозг используйте для полезных дел.


\iusr{Ольга Калениченко}
\textbf{Andriy Utevsky} я 5 лет на войне отработала, не переживайте за меня))) да, я предпочитаю мыться, а не сидеть вонючей))

\iusr{Andriy Utevsky}
\textbf{Ольга Калениченко} Я за Вас не переживаю. Но не все такие как Вы. Поэтому не приставайте к людям. Вы не знаете кто там и почему.

\iusr{Ольга Калениченко}
\textbf{Andriy Utevsky} 

так я из опыта говорю, что чем раньше люди осознают сто это не «временно» и это
надолго, тем легче им будет начать новую жизнь. У нас никто не спрашивал, хотим
мы этого или нет. Но жить в метро долго не выход, а попытка спрятаться от
реальности.

\end{itemize} % }

\iusr{Yaroslava Serdyuk}

Нормальное решение днем как транспорт, на ночь - как убежище.

\iusr{Александр Ткач}

У многих людей которые в метро есть жилье. Маниакальный эгоистичный психоз. А
вдруг прилетит именно в него. Таких надо лечить. Метро надо открывать и
налаживать жизнь в военных условиях. Надо работать.

\iusr{Andriy Utevsky}
\textbf{Александр Ткач} Желание сохранить жизнь - это эгоизм? Круто!

\iusr{Дмитрий Беляев}
Салтовка и горизонт разбит в хлам, где людям ночевать?

\begin{itemize} % {
\iusr{Аліна Грудзинська}
\textbf{Беляев Дмитрий} Эвакуироваться.

\iusr{Дмитрий Беляев}
\textbf{Алина Грудзинская} куда? В одном доме 64 квартиры. Во всём районе сколько людей? И метро в первую очередь стратегический объект и только потом это транспорт
\end{itemize} % }

\iusr{Sveta Kris}
Я думаю это самый безопасный вид транспорта

\iusr{Svitlana Kovalova}
А что они там едят?

\begin{itemize} % {
\iusr{Наталия Товстуха}
\textbf{Svitlana Kovalova} Их кормят волонтеры.

\iusr{Вадим Сусидко}
\textbf{Наталия Товстуха} Одной Мивной кормят. Мне моя сестра рассказываол как привозили только мивину

\iusr{Наталия Товстуха}
\textbf{Вадим Сусидко} Сходите и посмотрите сами. И мивина тоже есть. Помимо полноценного первого и второго.

\iusr{Tetyana Sfandex}
\textbf{Вадим Сусидко} ну как бы никто вообще не обязан их кормить

В Западной Украине, да даже уже в Днепре, магазины работают и можно купить что
угодно. Эвакуационные поезда каждый день и уже без давки.

Зачем сидеть и ждать чтобы кто-то привозил еду, если не тяжело больной человек

\iusr{Roman Frolov}
\textbf{Tetyana Sfandex} это уже психология начинает работать, мир сужается до размеров лавочки в вагоне метро
\end{itemize} % }


\end{itemize} % }
