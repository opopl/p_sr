% vim: keymap=russian-jcukenwin
%%beginhead 
 
%%file 12_08_2022.stz.news.ua.donbas24.1.azovstal_fortecja_mrpl_narys_metalurg_kombinat.4.fortecja
%%parent 12_08_2022.stz.news.ua.donbas24.1.azovstal_fortecja_mrpl_narys_metalurg_kombinat
 
%%url 
 
%%author_id 
%%date 
 
%%tags 
%%title 
 
%%endhead 

\subsubsection{\enquote{Азовсталь} — фортеця Маріуполя}

Після початку АТО у 2014 році велику увагу було приділено бомбосховищам і
підземним тунелям на території \enquote{Азовсталі}. В \href{https://archive.org/details/video.03_05_2022.bbc_news_ukraina.dyrektor_azovstali_azov_tuneli_gotuv_2014}{інтерв'ю}%
\footnote{Відео: Директор \enquote{Азовсталі} про \enquote{Азов} і таємні тунелі: \enquote{Готувалися з 2014 року}, BBC News Україна, 03.05.2022, %
\par\url{https://www.youtube.com/watch?v=0Ki_9jyEV-M}, \par%
Internet Archive: \url{https://archive.org/details/video.03_05_2022.bbc_news_ukraina.dyrektor_azovstali_azov_tuneli_gotuv_2014}%
} ВВС генеральний директор
\enquote{Азовсталі} Енвер Цкітішвілі розповів, що на комбінаті налічується 36
бомбосховищ, в яких одночасно може укриватися близько 12 000 осіб.

Після 24 лютого 2022 року, коли на околицях Маріуполя вже були чутні обстріли
росіян, прийняли рішення відкрити прохідні \enquote{Азовсталі}, аби забезпечити вільний
доступ цивільних до бомбосховищ. На той час в укриттях був зроблений запас
питної води та сухпайків.

Після повномасштабного вторгнення росії були підняті радянські архіви
будівництва азовстальських бомбосховищ і виявлено, що 5 з них є найбільш
захищеними від ворожих атак (можуть витримати навіть потрапляння одного прямого
ядерного удару).

Чому цивільні шукали прихистку саме в укриттях \enquote{Азовсталі}? Можливо, і через
те, що з трьох боків \enquote{Азовсталь} оточено водою — річка Кальміус і Азовське море
не дозволяли російській армії одразу захопити територію комбінату, перебування
в азовстальських бомбосховищах здавалося цивільним маріупольцям надійнішим,
аніж перебування у підвалах житлових будинків.

На початку березня російські війська оточили місто, заблокувавши можливість
покинути територію заводу. Тоді про подвиг і оборону Маріуполя говорили у
всьому світі. Мужні воїни \enquote{Азову}, 36-ї бригади морської піхоти, прикордонники,
поліцейські, бійці ТрО та співробітники СБУ тримали оборону на території
металургійного гіганта.

\href{https://archive.org/details/video.05_05_2022.azov.shturm_azovstali_tryvaje}{%
Відео: Штурм \enquote{Азовсталі} триває!, АЗОВ, 05.05.2022}%
\footnote{\url{https://www.youtube.com/watch?v=iM99s81dzag}} %
\footnote{\url{https://archive.org/details/video.05_05_2022.azov.shturm_azovstali_tryvaje}}

\ifcmt
  ig https://i2.paste.pics/PT1WM.png?trs=1142e84a8812893e619f828af22a1d084584f26ffb97dd2bb11c85495ee994c5
  @wrap center
  @width 0.9
\fi

До середини травня росіяни зосередили всі свої сили на знищенні останньої
цитаделі українців в Маріуполі. Обстріли \enquote{Азовсталі} не припинялися ні на
хвилину: колосальна кількість авіаударів, фосфорних бомб, ракет, усіх видів
артилерії припадала на територію комбінату. Увесь світ страждав від безсилля та
відчаю, люди в багатьох країнах виходили на мирні акції з закликом врятувати
цивільних і військових з \enquote{Азовсталі}. \enquote{Help Mariupol, help Azovstal}, —
\href{https://archive.org/details/video.15_05_2022.eurovision_ukraine.kalush_orchestra_oleg_psjuk_pro_azovstal}{закликали} 
\footnote{Відео: \enquote{Такий шанс я б собі не пробачив втратити} – фронтмен Kalush Orchestra Олег Псюк про \enquote{Азовсталь}, Євробачення Україна, 15.05.2022, \par%
\url{https://www.youtube.com/watch?v=mCQROQ__SW0}, \par%
Internet Archive: \url{https://archive.org/details/video.15_05_2022.eurovision_ukraine.kalush_orchestra_oleg_psjuk_pro_azovstal}}
зі сцени Євробачення 2022 Kalush Orchestra.

\ifcmt
  ig https://i2.paste.pics/PT25S.png?trs=1142e84a8812893e619f828af22a1d084584f26ffb97dd2bb11c85495ee994c5
  @wrap center
  @width 0.9
\fi

6 травня почалася евакуація цивільних з \enquote{Азовсталі}, а 16 травня почалася
операція з порятунку військових, які були заблоковані на території заводу
\enquote{Азовсталь}. За \href{https://archive.org/details/video.17_05_2022.24_kanal.ganna_maljar_pro_operaciu_evakuacia_oboronci_azovstal}{словами}%
\footnote{Відео: Ганна Маляр про операцію з евакуації оборонців Азовсталі, 24 Канал, 17.05.2022, %
\par\url{https://www.youtube.com/watch?v=M2X5LbBAQy4}, \par%
Internet Archive: \url{https://archive.org/details/video.17_05_2022.24_kanal.ganna_maljar_pro_operaciu_evakuacia_oboronci_azovstal}%
} Ганни Маляр, евакуація військових проходила для
подальшого повернення українських захисників і захисниць додому, але цинізм і
жорстокість росіян після \href{https://donbas24.news/news/koloniyu-v-olenivci-obstrilyali-rosiyani-perexoplennya-sbu-video}{теракту}%
\footnote{Колонію в Оленівці обстріляли росіяни — перехоплення СБУ, Наталія Сорокіна, donbas24.news, 29.07.2022, %
\par\url{https://donbas24.news/news/koloniyu-v-olenivci-obstrilyali-rosiyani-perexoplennya-sbu-video}
} в Оленівці 29 липня 2022 року ставлять під
сумнів усі угоди, домовленості та гарантії з боку міжнародних організацій.

Історії героїв і героїнь \enquote{Азовсталі} пролунали на весь світ. Їхні трагічні
долі, мужність і сміливість, одруження в азовстальському бункері, надія та віра
день за днем вражали та змушували затамовувати подих. Варто не забувати, що
порятунок героїв \enquote{Азовсталі} продовжується і зараз, в той час, коли окупанти
цинічно влаштовують на руїнах металургійного гіганта недолугі показові
концерти.

\textbf{Які сторінки стануть наступними в історії видатного підприємства — покаже час.
Але його розквіт, відродження та відбудова з попелу можлива лише під
українським прапором.}

Нагадаємо, раніше Донбас24 розповідав, як \href{https://donbas24.news/news/po-foto-oresta-stvorili-kartinu-za-nomerami-vsi-kosti-z-prodazu-idut-na-volonterstvo}{по фото \enquote{азовця} створили картину за
номерами — всі кошти з продажу ідуть на волонтерство}.%
\footnote{По фото \enquote{азовця} створили картину за номерами — всі кошти з продажу ідуть на волонтерство, Еліна Прокопчук, donbas24.news, 11.08.2022, \par\url{https://donbas24.news/news/po-foto-oresta-stvorili-kartinu-za-nomerami-vsi-kosti-z-prodazu-idut-na-volonterstvo}
}

Ще більше новин та найактуальніша інформація про Донецьку та Луганську області
в нашому телеграм-каналі Донбас24.

ФОТО: малюнок на обкладинці АrtLizabeta; pastvu.com, з відкритих джерел

