% vim: keymap=russian-jcukenwin
%%beginhead 
 
%%file slova.zhenschina
%%parent slova
 
%%url 
 
%%author 
%%author_id 
%%author_url 
 
%%tags 
%%title 
 
%%endhead 
\chapter{Женщина}
\label{sec:slova.zhenschina}

%%%cit
%%%cit_head
%%%cit_pic
%%%cit_text
Есть просто исторически сложившаяся роль мужчин.  Удивляет когда начинают
говорить про гендерную дискриминацию в Украине.  Да, если говорить про Украину,
то мужчин больше на руководящих должностях. Их чаще приглашают на разные ивенты
в качестве спикеров.  Только разве это говорит о дискриминации \emph{женщин}?
Почему никто не пытается разобраться в причино-следственных связях?  Никто же
не запрещает и не мешает \emph{женщине} развиваться профессионально, создавать
карьеру или строить бизнес.  Проблема в том, что большинство \emph{женщин} в
Украине этого не хотят!!! Им это не нужно. \emph{Женщины}, в подавляющем
большинстве, видят для себя совершенно другое предназначение! В основном,
родить и воспитать ребёнка.  И при этом функцию добытчика они возлагают на
мужчин. По мнению большинства \emph{женщин} именно мужчина должен содержать и
обеспечивать семью. Даже если она работает, то она считает, что мужчина должен
зарабатывать больше неё
%%%cit_comment
%%%cit_title
\citTitle{Никакой гендерной дискриминации в Украине не существует}, 
Павел Вернивский, strana.ua, 18.06.2021
%%%endcit

