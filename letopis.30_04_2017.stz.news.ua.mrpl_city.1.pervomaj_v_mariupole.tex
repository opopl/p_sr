% vim: keymap=russian-jcukenwin
%%beginhead 
 
%%file 30_04_2017.stz.news.ua.mrpl_city.1.pervomaj_v_mariupole
%%parent 30_04_2017
 
%%url https://mrpl.city/blogs/view/pervomaj-v-mariupole
 
%%author_id burov_sergij.mariupol,news.ua.mrpl_city
%%date 
 
%%tags 
%%title Первомай в Мариуполе
 
%%endhead 
 
\subsection{Первомай в Мариуполе}
\label{sec:30_04_2017.stz.news.ua.mrpl_city.1.pervomaj_v_mariupole}
 
\Purl{https://mrpl.city/blogs/view/pervomaj-v-mariupole}
\ifcmt
 author_begin
   author_id burov_sergij.mariupol,news.ua.mrpl_city
 author_end
\fi

\enquote{1 мая 1899 года в садах у реки Кальчик была проведена первая в Мариуполе
маевка}- это цитата из книги Д. Н. Грушевского \enquote{Имени Ильича}. Дмитрий
Николаевич, как настоящий ученый (он был доцентом и кандидатом экономических
наук), снабдил эту фразу ссылкой на совершенно конкретный документ в Донецком
областном архиве. В приведенной выше цитате есть слово, которое может
оказаться не­знакомым представителям молодого поколения. Объясним его значение,
воспользовавшись словарем С. И. Ожегова: \enquote{Маевка. В дореволюционной России:
нелегальное революционное собрание рабочих 1 мая}.

\ii{30_04_2017.stz.news.ua.mrpl_city.1.pervomaj_v_mariupole.pic.1}

Мариупольские рабочие и в последующие годы после 1899 года отмечали праздник
солидарности трудящихся маевками, а иногда и шествиями-демонстра\hyp{}циями. Первая
легальная первомайская демонстрация в нашем городе состоялась на Екатерининской
улице (теперь про­спект Мира) в 1917 году. С установлением советской власти
Екатерининскую улицу нарекли проспектом Республики. Что же касается коренных
жителей, то они продолжали называть эту городскую магистраль неофициальным
именем - Большая. Именно по Большой после войны гражданской до самой
Отечественной шли первомайские колонны от Торговой улицы вверх к городскому
скверу. Заметим, что до Великой Отечественной войны и не­сколько лет после нее
рабочие и служащие завода имени Ильича, ученики школ из прилегающих заводских
поселков проводили праздничные манифестации в своем районе, так сказать, у себя
\enquote{дома}.

\ii{30_04_2017.stz.news.ua.mrpl_city.1.pervomaj_v_mariupole.pic.2}

У каждого из мариупольцев свои воспоминания о первомайских днях. Может быть,
даже об одних и тех же. Вот 1 мая 1945 года. Из черной тарелки
радиорепродуктора голосом Юрия Левитана сообщили, что Берлин пал. Однако
боевые действия продолжаются. О празднике напоминают лишь редкие красные флаги,
вывешенные кое-где на фасадах домов да бравурные марши, раздающиеся из уличных
громкогово­рителей - пирамидок. На всю Торговую улицу единственное украшение:
гирлянда флажков, натянутая от электрического столба у дома, где, по легенде,
якобы жил митрополит Игнатий, до столба на противоположной стороне улицы,
стоящего по диагонали. Флажки нарезаны из старых газет и окрашены \enquote{подручными
красками}: зеленкой, разбавленными красными и фиолетовыми чернилами, каким-то
ярко-оран­жевым химическим веществом. Изготовители гирлянды - ученики старших
классов неполных средних школ - № 1 (женской) - ее ученицы учатся в первую
смену, и №3 (мужской), ее мальчишки занимаются - во вторую. Теперь двухэтажное
здание бывшей школы занимает один из департаментов исполкома.

\ii{30_04_2017.stz.news.ua.mrpl_city.1.pervomaj_v_mariupole.pic.3}

Конец сороковых годов. 1 мая. Шествие начинают колонны трудовых резервов.
Строгие шеренги подростков в синих сатиновых гимнастерках, в черных брюках, на
головах фуражки с лакированными козырьками, на околышах \enquote{молотки}. Они похожи
на маленьких солдат, впрочем, почему похожи? Они и были солдатами, принявшими
на свои еще детские плечи все тяготы военных и послевоенных лет. 

\ii{30_04_2017.stz.news.ua.mrpl_city.1.pervomaj_v_mariupole.pic.4}

За ремесленниками и учащимися школ ФЗО маршируют школяры. Ряды их были
нестройные. Физруки, комиссованные по ранениям офицеры, пятясь и срывая голоса,
кричат в жестяные раструбы: \enquote{Правый фланг, подтянись, держи дистанцию}.
\enquote{Иванов, не отставать!}. Украшением демонстрации был четкий строй курсантов
Школы морского обучения.  Бравые парни в бескозырках, в форменках с синими
воротниками-гюйсами. Марширующие колонны достигали пика своего совер­шенства,
поравнявшись с трибуной. Трибуна стояла чуть ниже местного отделения Госбанка.
Подобие драпировки из красных флагов.  Создавали фон для руководителей города,
стоящих на трибуне.

Затем начинается шествие колонн предприятий.  Во главе азовстальцев – грузовик,
задекорированный кумачом.  Спереди большой портрет Орджоникидзе и крупные буквы
- \enquote{Азовсталь}.  Затем шествует большой духовой оркестр клуба металлургов
предприятия, знаменосец с бархатным знаменем завода. А уж потом шагает шеренга
руководителей: директор, парторг, комсорг, лидер профсоюзов. За ними –
нестройными рядами движутся работники цехов и отделов. У некоторых в руках
флажки или ветки с самодельными бумажными цветами. И, конечно же, портреты
руководителей партии и правительства, которые несут по два человека.
Транспаранты из красной материи, на которых праздничные призывы.

За азовстальцами следуют труженики Коксохима, треста \enquote{Азовстальстрой}, завода
металлоконструкций, других предприятий Орджоникидзевского района. Оформление
колонн мало чем отличается от описанных выше азовстальских.  Но вот появляются
на Большой представители предприятий и учреждений центральной части города. Все
здесь скромнее. И транспарантов маловато. И портреты вождей государства,
наклеенные на фанерки и притороченные к наскоро остроганным палкам, легко несут
даже женщины. Вместо духовых оркестров - баянист. Народ оживленный. Судя по
всему, люди уже \enquote{согрелись} у походных буфетов – столов, выставленных у
продовольственных магазинов....

Демонстрация завершалась, местные вожди, облаченные в парусиновые \enquote{сталинки}, в
застегнутые на все пуговицы шеви­отовые темные костюмы, покинули свой
торжественный пост. На площади перед Госбанком на брусчатой мостовой начались
танцы под баян. Девушки танцуют друг с дружкой, их кавалеры в большинстве
своем сложили головы на фронте, а те, кто избежал печальной участи, продолжали
служить в армии, дожидаясь демобилизации.

1 мая было не только днем проведения демонстраций, но и рубежным сроком для
ввода в строй различных объектов, основания организаций и газет. Для
подтверждения этого приведем \enquote{кусочек} мариупольской хронологии.

1 мая 1929 года вышел в свет первый номер газеты \enquote{Ильичевец}.

1 мая 1933 года пущен первый трамвай по маршруту \enquote{Гавань Шмидта - улица
Франко}.

1 мая 1957 года сдана в эксплуатацию больница завода им. Куйбышева.

1 мая 1958 года открыт Дворец культуры коксохимзавода...

У каждого свои воспоминания о Первомае... 
