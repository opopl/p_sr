% vim: keymap=russian-jcukenwin
%%beginhead 
 
%%file 27_08_2021.fb.kocaba_ruslan.1.ivan_franko_165_let
%%parent 27_08_2021
 
%%url https://www.facebook.com/ukrpacificus/posts/1258244357954057
 
%%author Коцаба, Руслан
%%author_id kocaba_ruslan
%%author_url 
 
%%tags birthday,franko_ivan,literatura,patriotizm,ukraina
%%title Сьогодні 165 річниця від дня народження Івана Франка
 
%%endhead 
 
\subsection{Сьогодні 165 річниця від дня народження Івана Франка}
\label{sec:27_08_2021.fb.kocaba_ruslan.1.ivan_franko_165_let}
 
\Purl{https://www.facebook.com/ukrpacificus/posts/1258244357954057}
\ifcmt
 author_begin
   author_id kocaba_ruslan
 author_end
\fi

Сьогодні 165 річниця від дня народження Івана Франка. Хочу нагадати усім
держимордам і професійним патріотам, які нині будуть на цій даті піаритися, -
розлогу та надзвичайно актуальну зараз цитату Каменяра, написану ще в далекому
1895 році. В інтернеті повну версію можна знайти за назвою "Не люблю
русинів"... Отож, думайТЕ-читайТЕ! 😉 

«Насамперед признаюся в тому гріху, що його багато патріотів уважає смертельним
моїм гріхом: не люблю русинів. […] Признаюсь у ще більшому гріху: навіть нашої
Русі не люблю так і в такій мірі, як це роблять або вдають, що роблять,
патентовані патріоти. Що в ній маю любити?

\ifcmt
  pic https://scontent-cdg2-1.xx.fbcdn.net/v/t1.6435-9/240678554_1258244327954060_755075377744679330_n.jpg?_nc_cat=104&_nc_rgb565=1&ccb=1-5&_nc_sid=8bfeb9&_nc_ohc=X-wuxTW2kNoAX-evVNt&_nc_oc=AQlQohPCJsEl1tZtzTpWOik5g6nT8G5Wkwgdiecy3zFDqP4t6ga6ZBoxuFq2b1vRTXQ&_nc_ht=scontent-cdg2-1.xx&oh=fca63d4493800ae98fef07fd93b6e73e&oe=6151EC75
  width 0.4
	fig_env wrapfigure
\fi

Щоб любити її як географічне поняття, для цього я занадто великий ворог
порожніх фраз, забагато бачив я світу, щоби запевняти, що ніде нема такої
гарної природи, як на Русі. Щоб любити її історію, для цього досить добре її
знаю, занадто гаряче люблю загальнолюдські ідеали справедливості, братерства й
волі, щоб не відчувати, як мало в історії Русі прикладів справжнього
громадянського духу, справжньої самопожертви, справжньої любові.

Ні, любити цю історію дуже тяжко, бо майже на кожному кроці треба б хіба
плакати над нею. Чи, може, маю любити Русь як расу – цю расу обважнілу,
незграбну, сентиментальну, позбавлену гарту й сили волі, так мало здатну до
політичного життя на власному смітнику, а таку плідну на перевертнів
найрізнороднішого сорту? Чи, може, маю любити світлу будущину тієї Русі, коли
тої будущини не знаю і для світлості її не бачу ніяких основ?

Коли, незважаючи на те, почуваю себе русином і по змозі й силі своїй працюю на
Русі, то, як бачиш, шановний читачу, цілком не з причини сентиментальної
натури. До цього примушує мене почуття собачого обов’язку. Як син
селянина-русина, вигодований чорним селянським хлібом, працею твердих
селянських рук, почуваю обов’язок панщиною всього життя відробити ті шеляги,
які видала селянська рука на те, щоб я міг видряпатись на висоту, де видно
світло, де пахне воля, де ясніють вселюдські ідеали. Мій руський патріотизм –
то не сентимент, не національна гордість, то тяжке ярмо, покладене долею на мої
плечі.

Я можу здригатися, можу тихо проклинати долю, що поклала мені на плечі це ярмо,
але скинути його не можу, іншої батьківщини шукати не можу, бо став би підлим
перед власним сумлінням. І якщо щось полегшує мені нести це ярмо, так це те, що
бачу руський народ, який, хоч гноблений, затемнюваний і деморалізований довгі
віки, який хоч і сьогодні бідний, недолугий і безпорадний, а все-таки поволі
підноситься, відчуває в щораз ширших масах жадобу світла, правди та
справедливості і до них шукає шляхів. Отже, варто працювати для цього народу, і
ніяка праця не піде на марне» (Передмова до збірки «Obrazki galicyjskie»,
написана 1895 р.).

\ii{27_08_2021.fb.kocaba_ruslan.1.ivan_franko_165_let.cmt}
