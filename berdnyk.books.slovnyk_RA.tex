% vim: keymap=russian-jcukenwin
%%beginhead 
 
%%file books.slovnyk_RA
%%parent body
 
%%endhead 

\section{СЛОВНИК РА (ВОСКРЕСІННЯ СЛОВА)}

Містерія

Від упорядника

«Словник РА» (інша назва — «Воскресіння Слова»[2]) — не науково-фантастичний
трактат про витоки мови. За допомогою «символічної етимології» Олесь Бердник
шукає в сучасній українській мові слова, що якнайкраще зберегли залишки
праслов’янського сонячного культу (автор називає його «Ра»). Тож цей твір можна
назвати культурно-світоглядною мовознавчою розвідкою, яка не претендує на
науковий статус.

В Україні «Словник РА» публікується вперше.

Спочатку було Мовчання,

З Мовчання з’явилося Слово,

Слово стало Матір’ю Мови,

Мова народила Людину.

Людина — батько Духу,

Дух у сяйві Слова

Повертається до Великого Мовчання.

Де зачинається Слово?

Хто дав йому путь і гарт?

З якої виходять основи

Пісня, роздум і жарт?

Стежки не паперові

Слово вперед ведуть, —

Буйне дерево Мови

Корінням сягає в Суть…

Суть — Материнське Лоно

Дії, світла, життя,

Те, що над всі кордони

Творить Вічне Життя!..

Слово — Дитя Великого Мовчання — заблукало на манівцях людської псевдоісторії.
Сини Хаосу нав’ючили на Слово вантаж безглуздя, жорстокості та суєтності,
перетворили Боже Дитя на віслюка для своїх забаганок, знущань, підлих замірів
та мізерної втіхи. Туман нерозуміння оповив свідомість людей, бо правдива суть
Слова затьмарена брехнею і відносністю. А затьмарене Слово жити не може у світі
облуди й ганьби: воно задихається і вмирає на хресті щоденного рабства, у
темниці безпросвітності.

Зруйнувати домовину Слова, повернути Боже Дитя Людині, воскресити в слові
безцінну вартість Материнської Першосуті — мета і обов’язок творчого мислення.

Ходімо ж тисячолітніми стежками рідного народу, спробуємо розшукати на них
загублені скарби материнської полум’яної Мови.

Слово — диво Всесвіту.

Його появу жадають пояснити хитромудрими науковими теоріями, начіплюють на
Слово безліч одежин різних епох і народів, аж доки воно забуває про своє
первородство і стає кріпаком того чи іншого мовного насильника.

Слово — не раб потреби, не дитя необхідності, а втаємничене серце Вічного
Буття. Кореневі слова в будь-якій мові не просто стихійне поєднання звуків, а
природне звучання тої Першооснови, котра стоїть за явищем, річчю, істотою,
суттю. Видозміна слів у різних народів — лише різне забарвлення того ж таки
вияву завдяки переломленню в розмаїтих скельцях прагнень і психік.

Зовнішній Космос (природній всесвіт) і Внутрішній Космос душі тісно, нерозривно
пов’язані. Той зв’язок виявляється в слові, а опісля — в думці, у дусі.

Абетка — ніби чарівна арфа. Кожна буква — струна. Сонце, зорі, дерева, квіти,
птахи, ріки, моря, вітри, люди, звірі, гори, поля — весь розмаїтий світ Великої
Матері — торкається вібруючих струн, творячи чаклунську симфонію Мови. У
природному всесвіті ми знаємо ту першооснову, котра лягла підвалиною буття,
стала чарівною ниткою, що з неї зіткано наше життя: то є дарунок
Світовида-Ярила, невидима Божа вібрація, вічне джерело радості, енергії, сили —
ПРОМІНЬ СВІТЛА.

Променю в мові відповідає звукосполучення РА. Він і є вібраційною основою
абетки, котра збуджує до дії, до життя інших своїх братів, напоює їх силою і
певністю, радістю і повнотою вияву. Доки людина не оволоділа в дитинстві звуком
РА, вона інтуїтивно відчуває, що мова її ще неповноцінна.

Могутній вібраційний звук РА і стає по праву першонародження символом сонця,
світла, радості.

Бог Сонця РА відомий багатьом народам.

Самоназва сонця — РА — могла зникати, замінювалася іншими (Світовид, Ярило,
Сонце, Дажбог і т. д.), але всі мовні сонячні діти залишилися в душі народу
протягом тисячоліть.

Гляньмо…

Радість (Ра-дість, Ра-дає… тобто це почуття дає Ра, світло, сонце).

РОБОТА (рос. РАБОТА) — (Ра-бути, бути в обіймах Ра, буття радості, світла,
натхнення). Так високо пращури оцінили роботу, творчість, як продовження місії
сонячного променя на землі. Ось чому РОБІТНИК, ТВОРЕЦЬ — істинні діти Сонця,
його воїни, його виконавці, його промені. Ось чому за ними майбуття, а не за
тими, хто віками визискував їх, не маючи повноважень РА-СВІТОВИДА. Оскільки РА
є основою самого буття, то перемога робітників, творців неминуча, а поразка
синів темряви — лише справа часу.

РОЗУМ (рос. РАЗУМ) — дивовижне поєднання понять УМ (уміти, збагнути, оволодіти)
і РА/РО. Ра з Ум дають Людину Мислячу, котра дарунок світла — РА сполучує з
умінням творити, поєднувати хаотичні елементи стихій у гармонію осмисленого
буття. Це слово, крім ознаки наявності осмислення, усвідомлення, творення, —
означає і об’єднання (РАЗОМ, тобто воєдино, вкупі). Отже, місія Розуму є
всеоб’єднання світу, всіх його проявів. Людині Мислячій доручено бути
господарем, будівником Космосу.

РАЗ. Початок лічби. Вона природно бере початок від Сонця. Кожен бачить ЄДИНЕ
джерело буття щоденно, тому звук РА законно очолює число.

Разом з тим пращури інтуїтивно відчували небезпеку роздрібнення світу, тому

РАХУВАТИ — це Ра-ховати, затемнити єдиний промінь, подрібнити його в
нескінченних проявах розділеного світу (Розділити — це РА — ділити).

РАДА — від слова РАДІСТЬ. Радитись — це опромінитись РА, радісно об’єднатися в
спільному прагненні знайти основу, прийнятну для всіх. Не може бути «злої
ради», «поганої ради», бо це вже ЗРАДА, тобто відхід від РАДИ, РАДОСТІ, від РА.
Отже, зрадники прирікають себе на вічну загибель, бо вони перестали бути синами
Світла, розірвали пуповину життя.

Не вживайте поєднань «погана порада», «недобра рада» тощо. Рада, порада можуть
бути лише світоносні, добрі, щирі, сердечні.

РАЯТИ, РОЗРАЯТИ — вернути в душу людини РА, світло, Радість. Ра — це також і
країна Світовида, Сонця, Ра, казкова країна щастя.

РАЙДУГА — світоносна дуга, ворота Ра, брама Сонця.

РАНОК — схід Ра над світом.

РАЛО, РАЛИТИ — орати землю (О-РА-ТИ), підготувати її до весняної сонячної пори,
пори Ра.

ПОРА — по-Ра, після Ра (час після сходу Ра, коли людина зобов’язана встати до
праці. Пізніше слово «пора» стало універсальним, кажеться, мовиться навіть
«нічна пора», а це неможливо по суті слова: ПОРА — лише сонячний ранок. Ось
чому «на порі» — це означає «в розквіті сил, радості, творчості, життя».

ПАРУБОК — юнак, котрий на порі, готовий до одруження, що шукає собі пари.

ПАРА — подружжя, об’єднане променем РА, освячене вогнем. Хлопець і дівчина
прадавніх часів питали одне одного:

— Ти радий мені?

— Ти рада мені?

— Стрибнеш зі мною через РА?

— Стрибну.

Вони бралися за руки, стрибали через буйне багаття і ставали ПАРОЮ, обвінчаною
богом РА.

ПРА — найдревніший, найдавніший. Той, що почався ще в епоху Ра — першосвітла. Ось чому

ПРАЩУР (П-РА-ЩУР, Пра-чур). Чур — хранитель роду, предок. Пращур — сонячний предок.

РАМА — той, хто несе промінь РА Прекрасне поєднання понять МА (мати, воліти, бути в наявності) і світла Ра. У нашій мові лишилося лише побутове значення цього слова (віконна рама, що пропускає світло в хату). Певно, візантійська навала винищила первородне поняття, котре, все-таки, лишилося на сході в епосі про героїчного Рама, посланця сонячного бога Вішну, котрий утвердив сонячний культ у герці з втіленням темряви і зла — Раваною3. (Вішну — це Вищий, Вишній, Горішній, тобто знову той же таки Ра.)

ПРАВДА — слово, котре вживають усі жерці релігій, усі соціологи, всі філософи, всі апологети містичних чи політичних вчень. Мається на увазі (здебільшого) правда як точна інформація, відомості про те чи інше явище, ситуацію, подію. Але це лише поверхове значення дивовижного слова, котре складається з ПРА — тобто найдавніший, той, що народжений від РА, і ВЕДА — знання, розуміння.

3 Виникає запитання: хто ж тоді РА-ВАНА? Але це запитання риторичне, над яким можуть подумати ті, хто захоче продовжити мовознавчі розвідки автора. — Прим. упор.

Отже, правда — це глибинне, вічне, первородне знання, притаманне самому життю.

Разом з тим:

КРИВДА — це і криве знання, і прикрите (КР), і вкрадене.

РА, РО, РУ входило в мову пращурів не лише як основа вогненності, чистоти,
вірності, сонячності, а й як єдність, об’єднання, зведення докупи різних
чинників або як ознака спільноти, рідності, єдинокровності.

Наприклад.

БРАТ, СЕСТРА, безсумнівне родичання через РА. Тим більше, що родити — це ДАТИ у
світ нове РА (Ра — дати), звідси — рід (род), народ, родина.

РЯД — як зведення роздрібненості в спільну шеренгу, випрямлення.

ПРЯСТИ — готувати спільну нить з розтріпаного волокна.

СТАРАТИСЯ — діяти для РА.

ОБРАТИ — взяти за основу РА.

Можна знайти ще багато слів у рідній мові, у мовах інших народів, котрі
ствердять вібраційну суть втаємниченого звуку РА (РО, РУ, РЕ, РІ), але справа
не в нагромадженні прикладів, а в знаходженні ключа розуміння того, що повинен
збагнути кожний творець: мова є першожиття народу, і саме вона формує дух
нації. Хто зрікається рідної мови, той перестає бути Людиною.

Ходімо далі. Розглянемо, яка основа лягає в динаміку буття? Безумовно, рух.

РУХ — основний прояв РА у зовнішньому світі. Звідси — РУКА (ру-х-а) — орган
людини, який рухає, діє, творить,

РУШАТИ, РУШИТИ, РУШЕННЯ — мінлива гра поняття, як у напрямку творчому, так і в
руйнівному.

РІКА — як рух води.

РІК — рух часу.

РІЧ, РЕКТИ — рух мови.

РОК, РОКОВАНІСТЬ — течія долі.

КРАЙ — межа, до якої сягає РА. Звідси, КРАЇНА — земля, напоєна силою, полум’ям,
щедрістю РА, земля, рідна для даного народу. Ось чому УКРАЇНА, УКРАЙНА —
дивовижна, сокровенна самоназва, в якій безодня значень. Це і земля, не
обмежена видимими кордонами, а та, котра кличе до РА, вслід за сонцем, це і
РАЇНА — багата, сонячна; це і УКРИТА, невидима, прихована в лоні РА. УКРАЇНА —
це не тільки географічний терен, держава, сукупність степів, лісів та рік, а,
передусім, духовна країна Свободи, Радості, Мужності, Пошуку, небесна країна
РА.

Саме тому козаки Січі Запорізької боронили не тільки межі рідної землі, а й
головну суть буття України — ВОЛЮ, РАДІСТЬ НЕПІДЛЕГЛОСТІ. Вони могли
відтворити, відродити НОВУ УКРАЇНУ де завгодно — на Дунаї, в Малій Азії, на
Кубані, в Америці, аби лише вдалося зберегти Зерно Свободи. Як відомо, цього не
сталося, земні козаки загинули навіки, а небесні, легендарні, пісенні рушили
шляхом РА у вічну мандрівку до Небесної УКРАЇНИ…

РА Святий, прийди у нашу хату,

Розбуди нещадно і зігрій.

Хай всміхнеться Вічно Суща Мати

Усміхом кохання і надій!

Хай у небі, на землі, в безмежжі

Загримить всепереможний РА.

Темні розпадаються мережі —

Народилась Радості пора!

А тепер перейдемо до укладення невеликого словника, в якому спробуємо
прослідкувати утворення пишних китиць, гірлянд, віток слів від первісних,
кореневих понять, котрі випливають із самої суті буття.

За яким принципом укладати словник? За абетковим? Але ж у житті, у природі
такого принципу немає. Космічна Першосуть, котра розгортається в бутті Людини й
Природи, не знає і не має деспотичних програм: вітки її Древа розкриваються за
глибинним законом родинної послідовності, любовної спорідненості, а насамперед
— важливості тої чи іншої частки буття, котра має проявитися тепер, у цю мить,
у цьому місці. Наприклад, коли розпукується зерно, то воно запускає в землю
корінець, щоб дати паросткові опертя: змінити послідовність цих проявів
неможливо. У мові — те саме: є слова-корені, на них спираються інші слова,
котрі творять стовбур, гілки, а потім і листя та квіти велетенського Древа. Це
такі слова, як БУТТЯ, СУТЬ, ВЕЛИКИЙ, МАЛИЙ, РУХ, ГОДІ і т. д. Значення РА ми
вже збагнули в заспіві до словника.

Коли пращури вийшли з лона Великої Матері-Лади, то первісні парості звуків
гуртувалися, наростали довкола урочистого кореня Р: усі голосні просто-таки
липнули до цієї могутньої вібрації, творячи найважливіші поняття: АРА, УРУ,
ОРО, ІРІ, ЕРЕ. З цих звукосполучень ми й почнемо наш СЛОВНИК РА, Словник
Радості.

Ар’ї, райці чи ірійці, де ви?

Лиш у мові пломенять сліди.

Зерна обтрусило буйне Древо,

А саме пропало назавжди.

Не пропало! Повстає у Слові,

У піснях, легендах і казках.

Скоро, скоро загримить ізнову

Під полками Рами битий шлях!

Віто, Віто — життєдайна сило,

Заховалась Ти у глибину.

У словесну запхано могилу

Душу Світу вічно-вогняну.

Воскресай, пресвітла Диво-Мати,

І життя нетлінне розбуди!

В зореносних, огнетканих шатах

Із глибин незмірності гряди!

*

А-а-а! — Первонароджена голосна. Перший паросток мови, котрий одразу спирається
на корінець «Р».

Ар, Ара. Перша тріада, священна трійця Мови, таємниче зерно Внутрішнього
Космосу душі. Від цього зерна починається мова.

А оскільки мова формує Людину, то Людина й називає себе дитиною АРА, арійцем,
сином Вічного Вогню, Світлоликого Сонця-Ра, променем Незримого багаття.

Але звук «А» — Первонароджений, він — Дитя Боже, його не можна оголювати перед
стихіями ще хаотичного світу. Інтуїтивно розуміючи це, наші пращури заховали
священну тріаду в покривала вторинних звукосполучень (жар, кара, мара, дар і т.
д.), а там, де «А» мусило стояти попереду, вони поставили один з останніх
звуків, котрий означав саму людину, Особу — «Я». Це й означало, що «А» — Син
Бога, схований в «Я»2, прикритий ним, захищений від навали стихій.

Отже, «Я» прикрило своєю плоттю «А», захистило його заборолом, твердинею інших
звуків. Рідна мова УКРАЙНИ священно берегла цей заповіт пращурів, і лише
пізніше, коли голос першопредків загубився в далині, ми впустили в огненні
шеренги Мови чужаків, таких як: авторитет, аборт, агностик, актуальний, акція,
акт, антимонія і так далі, і тому подібне…

Отже, пращури не називали себе арійцями. Це можна бачити з того, що всі
споріднені з АРА слова прикриті символом Особи — «Я».

ЯРИЛО — сонце, бог Світовид, Дажбог, пломеніюче джерело життя в небі. Яке
могутнє слово! І які повнокровні, нерушимі гірлянди слів воно породило!

ЯРИЙ — сонячний, вогняний, нестримний.

ЯРІТИ — пломеніти.

ЯРИНА — сонячний, весняний посів.

ЯРКИЙ — сліпучо-ясний (ми даремно відмовилися від цього чарівного слова).

ЯРМО — навіть це слово, що стало ознакою рабства, означає засіб для оранки
(ЯРАНКИ) землі під ярину.
