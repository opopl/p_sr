% vim: keymap=russian-jcukenwin
%%beginhead 
 
%%file slova.bazar
%%parent slova
 
%%url 
 
%%author 
%%author_id 
%%author_url 
 
%%tags 
%%title 
 
%%endhead 
\chapter{Базар}
\label{sec:slova.bazar}

%%%cit
%%%cit_head
%%%cit_pic
%%%cit_text
Натомість \emph{базар} символізує життя у всій його повнокровності, лише там ви
зустрінете найбідніших і найбагатших, тих, що взялися за гендель з мрією
розбагатіти й стати володарем світу, і тих, що хочуть вести тихе й спокійне
життя, тому тепер стоять за прилавком з городиною. На \emph{базарі} ви
довідаєтеся про те, що їдять місцеві мешканці, що вирощують тут і в околицях,
що вважається цінним і популярним, тож його привозять з інших країн, як місцеві
люди вдягаються і наскільки дбають про чистоту й гігієну. Це вже не кажучи про
традиції усміхатися під час розмови, обмінюватися кількома репліками й
наріканнями на погоду, торгуватися і дякувати після покупки – усе це блискуче
демонструє рівень повсякденної культури. Навіть, не знаю, який з усіх
\emph{базарів} світу є для мене найулюбленішим. Напевно, все ж ужгородський
"\emph{Зелений базар}" у центрі, бо саме тут я купую продукти для себе, саме
тут знаю, в кого добрий хліб, у кого варто взяти домашній сир чи шматок сала.
Хоча не менш важливим для мене є виноградівський "малий" \emph{базар}, де я сам
колись продавав яблука і на виручені гроші купував собі ланґош, і "великий"
\emph{базар}, де в радісному настрої ставав між рядами на шматок картону, щоб
поміряти новий одяг перед початком навчального року
%%%cit_comment
%%%cit_title
\citTitle{Базар і цвинтар – найважливіші точки в місті. Між ними й треба шукати його серце}, 
Андрій Любка, gazeta.ua, 26.07.2021
%%%endcit
