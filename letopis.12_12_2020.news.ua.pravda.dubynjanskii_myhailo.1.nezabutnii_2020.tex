% vim: keymap=russian-jcukenwin
%%beginhead 
 
%%file 12_12_2020.news.ua.pravda.dubynjanskii_myhailo.1.nezabutnii_2020
%%parent 12_12_2020
 
%%url https://www.pravda.com.ua/articles/2020/12/12/7276572/
 
%%author Дубинянський, Михайло
%%author_id dubynjanskii_myhailo
%%author_url 
 
%%tags year_2020
%%title Незабутній 2020-ий
 
%%endhead 
 
\subsection{Незабутній 2020-ий}
\label{sec:12_12_2020.news.ua.pravda.dubynjanskii_myhailo.1.nezabutnii_2020}
\Purl{https://www.pravda.com.ua/articles/2020/12/12/7276572/}
\ifcmt
	author_begin
   author_id dubynjanskii_myhailo
	author_end
\fi

\ifcmt
pic https://img.pravda.com/images/doc/4/4/44da3f5-2020.jpg
\fi

\enquote{В истории США были и худшие годы, и, конечно, были худшие годы в мировой
истории. Но большинство из живущих сегодня не видели ничего подобного.

Вам должно быть больше 100 лет, чтобы помнить о разрушениях Первой мировой
войны и пандемии гриппа 1918 года; примерно 90 лет, чтобы иметь представление
об экономических лишениях, вызванных Великой депрессией; и 80, чтобы сохранить
память о Второй мировой войне и ее ужасах.

Остальные из нас не были подготовлены к такому}.

Так рассуждает журнал Time, поместивший на обложке перечеркнутую цифру \enquote{2020} и
провокационную надпись \enquote{The worst year ever}.  

Действительно ли уходящий год стал худшим на памяти нынешнего поколения? Видел
ли современный мир что-то более трагическое, чем пандемия, унесшая более
полутора миллионов жизней и не сбавляющая оборотов?

Что-то более катастрофическое, чем переполненные больницы, повсеместные маски,
закрытые границы, неработающие рестораны и кафе, пустующие офисы и торговые
центры, безлюдные аэропорты и мировые достопримечательности?.. 

Разумеется, с оценкой Time согласны далеко не все. В отечественных соцсетях
тоже нашлись критики журнала – видимо, обиженные тем, что у современных
американцев  не сохранилось болезненных воспоминаний о Чернобыле, о бандитских
девяностых или о российском вторжении в Украину.

\begin{leftbar}
	\begingroup
		\em Впрочем, население нашей многострадальной страны вряд ли придет к
				консенсусу о том, бывало ли тяжелее, чем в 2020-м: а если нет, то
				какова худшая сторона 2020-го.
	\endgroup
\end{leftbar}

И в этом вся суть уходящего года. Бесконечный спор. Нарастающее непонимание.
Взаимное отчуждение.

Новый коронавирус, перевернувший нашу реальность в 2020 году, выглядит не
идеальным убийцей, но идеальным разъединителем людей.

Он не только запер человечество по национальным квартирам и временами обрекал
нас на изоляцию. Он напомнил нам, что мы – прежде всего отдельные биологические
единицы, слишком сильно отличающиеся друг от друга.

Тяжелейшее поражение легких, борьба за каждый вздох, непоправимый ущерб
здоровью даже в случае выживания – это новая коронавирусная инфекция. Слабые
простудные симптомы, потеря обоняния на пару дней, обнаруженные впоследствии
антитела – это тоже новая коронавирусная инфекция.

И если пожилые по умолчанию попадают в группу риска, а юные могут не особенно
тревожиться, то для человека среднего возраста COVID-19 оказывается лотереей с
абсолютно непредсказуемым исходом.

\begin{leftbar}
	\begingroup
		\em Читайте также: \href{https://www.pravda.com.ua/rus/articles/2020/11/21/7274195/}{Этика пандемии}
	\endgroup
\end{leftbar}

Наши мысли, чувства, мечты, идейные убеждения и политические предпочтения не
значат ничего. Наша индивидуальная биология решает все.

Новая болезнь слишком опасна, чтобы обойтись без ломки привычного жизненного
уклада. Новая болезнь недостаточно опасна, чтобы каждый смог лично убедиться в
необходимости чрезвычайных мер.

Трудно представить ситуацию, более способствующую разделению рода
человеческого. Затронув всех без исключения, глобальный коронакризис затронул
всех по-разному.

\begin{leftbar}
	\begingroup
		\em У  одних пандемия отнимает работу или досуг. У других – здоровье или
				жизнь. Для одних самым страшным словом уходящего года стал \enquote{локдаун}. У
				других куда более сильные эмоции связаны со словом \enquote{сатурация}.
	\endgroup
\end{leftbar}

Этой осенью кто-то публиковал  трогательные твиты вроде \enquote{Больше всего разбивают
сердце пустые кинотеатры}. А в это же время чье-то сердце разрывалось из-за
гибели самого близкого и дорогого человека.

Травматический опыт тех, кто пострадал только от карантина, несопоставим с
травматическим опытом другой части общества – лично столкнувшейся с тяжелым
течением COVID-19.

Это напоминает войну, которая для кого-то оборачивается продуктовыми карточками
и очередями, а для кого-то – артобстрелами и пулеметным огнем.

В украинском случае напрашиваются аналогии с нашей жизнью после 2013 года:
когда одни страдали лишь от финансового кризиса и обесценивания гривни – а
другие сражались под Иловайском и Дебальцево, бежали из Донецка и Луганска,
теряли друзей и родных.

Правда, даже по официальной статистике число жертв пандемии  в Украине уже
выше, чем число погибших на Донбассе с обеих сторон. И в ближайшие месяцы война
с новым коронавирусом оставит гибридную войну с Кремлем далеко позади.

Ситуацию усугубляет то, что часть пострадавших от локдауна изначально нашла
выход в отрицании неудобной биологической реальности. А замечание о том, что
вирус  поражает и не верящих в его опасность, справедливо лишь отчасти.

\begin{leftbar}
	\begingroup
		\em
			В силу своей непредсказуемой избирательности COVID-19 переубеждает одних,
				но помогает другим утвердиться в собственном мнении.
	\endgroup
\end{leftbar}

Даже в странах  с наибольшей смертностью будет прослойка граждан, достаточно
везучих, чтобы не переболеть в тяжелой форме и не потерять никого из близких, –
и достаточно упрямых, чтобы отбрасывать любую нежелательную  информацию,
поступающую извне.  

Для этой части общества пандемия навсегда останется глобальным обманом,
COVID-19 – обычной  простудой, а любые противоэпидемические ограничения –
преступной тиранией правительств.

\begin{leftbar}
	\begingroup
		\em
				Читайте также: \href{https://www.pravda.com.ua/rus/articles/2020/09/12/7266157/}{Разделенные вирусом}
	\endgroup
\end{leftbar}

Соотечественники, не разделяющие подобный взгляд на вещи, предстанут в лучшем
случае одураченными паникерами, а в худшем – вражескими пособниками,
помогавшими отравить жизнь коронаскептика.

В свою очередь сам несгибаемый коронаскептик тоже не сможет рассчитывать на
понимание и сочувствие с их стороны.

Этот раскол сравним уже не с пребыванием в тылу и на передовой – а скорее с
нахождением по разные стороны фронта. Человек, воевавший с кремлевскими
боевиками на востоке Украины, никогда не поймет обывателя, щеголяющего с
георгиевской ленточкой.

Человек, осведомленный о сбитии \enquote{Боинга} российским \enquote{Буком}, не найдет общего
языка с аудиторией Дмитрия Киселева. А человек, чьи родные или друзья стали
жертвами COVID-19, не сможет договориться с теми, кто продолжает твердить о
\enquote{выдуманной пандемии} и \enquote{простом гриппе}.

Воспоминания о невзгодах 2020 года останутся в каждом уголке планеты, в каждом
доме, в каждой семье.

Психологическая травма, связанная с пережитым в 2020-м, окажет влияние на
миллионы людей.

Но общей памяти и общей травмы не будет. А память, разделяющая и разъединяющая
нас, вряд ли поможет проложить дорогу к лучшему будущему.

\textbf{\em Михаил Дубинянский}
