% vim: keymap=russian-jcukenwin
%%beginhead 
 
%%file 07_10_2020.news.ru.aif.kolokolceva_arina.1.habarov_smert_russia_altai
%%parent 07_10_2020
 
%%url https://altai.aif.ru/society/lyubite_rossiyu_v_barnaule_skonchalsya_izvestnyy_uchenyy_stanislav_habarov
 
%%author Колокольцева, Арина
%%author_id kolokolceva_arina
%%author_url 
 
%%tags altai,russia,habarov_stanislav
%%title Влюбленный в сады. В Барнауле скончался известный ученый Станислав Хабаров
 
%%endhead 
 
\subsection{Влюбленный в сады. В Барнауле скончался известный ученый Станислав Хабаров}
\label{sec:07_10_2020.news.ru.aif.kolokolceva_arina.1.habarov_smert_russia_altai}
\Purl{https://altai.aif.ru/society/lyubite_rossiyu_v_barnaule_skonchalsya_izvestnyy_uchenyy_stanislav_habarov}
\ifcmt
	author_begin
   author_id kolokolceva_arina
	author_end
\fi

\index[deaths.rus]{Хабаров, Станислав!Академик РАН, Барнаул, Россия, 07.10.2020}

\ifcmt
pic https://aif-s3.aif.ru/images/021/526/a504affa7268f69a00ccd7f98095fa4c.jpg
caption  Станислав Хабаров продолжал приходить в свой рабочий кабинет даже тогда, когда и ходить стало тяжело © / НИИ садоводства им. Лисавенко / Пресс-служба 
\fi

Ушел из жизни академик РАН Станислав Хабаров. Человек, на счету которого 430
научных работ, более трех десятков патентов, считал важным, чтобы изобретения
были не просто изобретениями, а внедрялись в жизнь.

«Это я в молодости думал, что патенты - главное,  а главное то, чтобы они
воплощались, приносили пользу России», - говорил Станислав  Николаевич.

«Он занимался разработкой эффективных агроэкосистем промышленных садов в
условиях юга Западной Сибири, индустриальных технологий возделывания плодовых и
ягодных культур», - рассказывают в институте садоводства им. Лисавенко.

Творческая работа Станиславу Хабарову всегда нравилась больше, чем чиновничья.
Даже в те годы, когда он был на посту руководителя НИИ Лисавенко, ученый
продолжал совершать открытия и заниматься профессорской деятельностью.

\begin{leftbar}
	\begingroup
				\em \textbf{Биографическая справка}\par
Станислав Николаевич родился в 1938 году в селе Тулата Чарышского района. Всю
свою жизнь он провел в родном регионе. Здесь же окончил сельхоз
институт, аспирантуру Алтайской опытной станции садоводства. С 1969
года работал заместителем директора Алтайской опытной станции, а в 90-х
возглавил НПО «Сады Сибири», руководил НИИ садоводства Сибири им. М.А.
Лисавенко. Параллельно с этим был заведующим кафедрой консервирования
плодов и овощей АГТУ, профессором кафедры плодоводства АГАУ. Под его
научным руководством были подготовлены 12 кандидатских и две докторские
диссертации. Станислав Хабаров опубликовал более 430 научных работ,
получил 30 патентов и два авторских свидетельства. 
	\endgroup
\end{leftbar}

«Сидеть в президиумах — неделями, в правительственных кругах вращаться —
месяцами… Мне странно, что людям это интересно может быть, мне вот абсолютно
нет. Потеря времени»- говорил он  в одном из интервью.

В трудные годы, когда пространства института Лисавенко стали сокращаться,
Хабаров пытался  с этим сражаться. Одним из больных моментов была гибель
земляничных полян. Солнечные гектары, где горожане собирали до 15 тонн ягод,
отдали под застройку элитными коттеджами. Хабаров тогда пытался отстоять
уникальное место, не получилось. И агроном с большой буквы очень переживал
из-за этого.

\begin{leftbar}
	\begingroup
		\em Сидеть в президиумах — неделями, в правительственных кругах вращаться —
				месяцами… Мне странно, что людям это интересно может быть, мне вот
				абсолютно нет. 
	\endgroup
\end{leftbar}

Станислав  Хабаров продолжал приходить в свой рабочий кабинет даже тогда, когда
и ходить стало тяжело. Но поднимался по лестнице, заходил и садился за рабочий
стол, над которым висел портрет его учителя, гениального садовода Михаила
Лисавенко. И делал записи, надеясь, что они пригодятся тем, кто так же, как он,
будет влюблен в сибирские сады.

Своим коллегам, ученикам он часто говорил: «Любите Россию». Сам он Россию, ее
сады любил безгранично.

Прощание  со Станиславом  Хабаровым состоится 7 октября с 10 до 12 часов в
ритуальном зале по адресу: г. Барнаул, ул. Германа Титова, 9А (отправление
автобуса от института в 9:00). Похороны пройдут на Черницком кладбище.
