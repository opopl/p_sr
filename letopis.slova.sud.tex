% vim: keymap=russian-jcukenwin
%%beginhead 
 
%%file slova.sud
%%parent slova
 
%%url 
 
%%author 
%%author_id 
%%author_url 
 
%%tags 
%%title 
 
%%endhead 
\chapter{Суд}
\label{sec:slova.sud}

%%%cit
%%%cit_head
%%%cit_pic
%%%cit_text
Во-вторых, речь тут идет о классической иллюзии выбора. Если бы депутаты не
проголосовали «как следует», то президент наложил бы вето. Иными словами, закон
в любом случае приняли бы в том виде, в каком его желают видеть наши «западные
партнеры». Либо так, либо вообще никак.  А без него ведь тоже нельзя. \emph{Судебная}
система уже теперь агонизирует, задыхаясь от недостатка \emph{судей}. До конца года в
отставку уйдут еще семь сотен. Вот и пришлось выбирать между коллапсом и сдачей
суверенитета
%%%cit_comment
%%%cit_title
\citTitle{Чего ради независимость отдаем? На что меняем?  / Лента соцсетей / Страна}, 
Максим Могильницкий, strana.ua, 30.06.2021
%%%endcit

%%%cit
%%%cit_head
%%%cit_pic
%%%cit_text
«\emph{Суд} констатував, що володіти українською мовою як мовою свого громадянства –
обов’язок кожного громадянина України. При цьому кожен громадянин є вільним у
виборі мови або мов для приватного спілкування».
Это что же, за незнание украинского языка можно теперь лишать гражданства?!
Ведь теперь это обязанность каждого гражданина!
А многочисленные иностранцы в наб советах и других органах с гражданством
Украины, не разговаривающие на украинском, как с ними тогда?)
А «частное общение» - это как? у себя дома на кухне?!)
Ну и ничего, что это решение является нарушением Конституции самим
\emph{Конституционным Судом}?!
%%%cit_comment
%%%cit_title
\citTitle{Что получается - за незнание украинского языка теперь будут лишать гражданства?}, 
Елена Лешенко, strana.ua, 16.07.2021
%%%endcit

