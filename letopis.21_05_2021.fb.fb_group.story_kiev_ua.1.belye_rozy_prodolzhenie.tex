% vim: keymap=russian-jcukenwin
%%beginhead 
 
%%file 21_05_2021.fb.fb_group.story_kiev_ua.1.belye_rozy_prodolzhenie
%%parent 21_05_2021
 
%%url https://www.facebook.com/groups/story.kiev.ua/permalink/1667166693480125
 
%%author Киевские Истории
%%author_id fb_group.story_kiev_ua
%%author_url 
 
%%tags 
%%title «Белые розы и одно венское кресло» (Продолжение)
 
%%endhead 
 
\subsection{«Белые розы и одно венское кресло» (Продолжение)}
\label{sec:21_05_2021.fb.fb_group.story_kiev_ua.1.belye_rozy_prodolzhenie}
\Purl{https://www.facebook.com/groups/story.kiev.ua/permalink/1667166693480125}
\ifcmt
 author_begin
   author_id fb_group.story_kiev_ua
 author_end
\fi

\textbf{Svetlana Kievljanka}

Друзья! Вы уж простите! Сегодня ФБ чудит)))То отклоняет,то одобряет мои публикации.Ну,бывает! Мы снесли две предыдущие и делаем новую! Тут продолжение рассказа,а ссылку на начало я дам в первом комментарии.Надеюсь,что в будущем все будет ок! Простите! Завтра - окончание)))Кто успел прочитать ,молчим,держим интригу)))


\ifcmt
  pic https://scontent-bos3-1.xx.fbcdn.net/v/t1.6435-9/p843x403/182160457_2929031500703826_1977846157718336999_n.jpg?_nc_cat=107&ccb=1-3&_nc_sid=825194&_nc_ohc=b0AVPobHdsEAX-aSUmN&_nc_ht=scontent-bos3-1.xx&tp=6&oh=03da0f274bc933d20743408b0e6c2931&oe=60D1891A
\fi


«Белые розы и одно венское кресло»
(Продолжение)
Юджин сидел за мольбертом в Централ - парке Нью-Йорка и понимал,что за ним кто-то наблюдает.Он просто кожей ощущал на себе чей-то пристальный взгляд.Это была женщина,скорее всего.У мужчин взгляд конкретный,у женщин - оценивающий.Его сейчас оценивали.Сто процентов.
Жека, ставший Юджином сразу по приезду в США , писал постоянно .Даже в самолете,уносившем его в новую жизнь,он пытался запечатлеть облака.Правда,тогда это было , скорее , бегство от наполнявших его эмоций , просто надо было поменять вектор мыслей . Облака были подходящим объектом.
До Америки им пришлось поколесить по Европе,много неприятных моментов пришлось пережить .И вот они в центре мира.В городе,который не спит никогда.Некоторое время они жили на пособие,но отцу повезло найти работу,а мама быстренько устроилась в ближайшую парикмахерскую.И пусть это был «салон» , но выходцы из СССР все равно называли его парикмахерской.Трудно было привыкнуть к дичайшей смеси русско-английских слов.Очень.Местные жители , как правило,начинали фразу по-английски,а заканчивали ее  чисто по-русски,с вкраплениями русского же мата.Ибо половины слов не знали.Это не мешало жить и дружить армянам,евреям,украинцам и одесситам! Одесситы во все времена считались здесь отдельной нацией! 
Так и жили - Женька поступил на курсы живописи,мама стригла , папа сидел в какой-то конторе сутками.Вечерами Жека получал уроки живописи,а днём работал то в МакДональдсе,то в закусочной на колёсах.О получении высшего образования с их доходами можно было только мечтать...
«Я стану художником! Я стану знаменитым художником! Обязательно! Моя семья будет обеспечена! Я заберу Нельку! Мы родим троих детей и будем счастливы»,- повторял эту мантру Женя,улыбаясь и протягивая гамбургер темнокожему ньюйоркцу.
————————-
Он обернулся,это действительно была женщина.Он видел ее не раз и не два.Иногда она сидела вон там,справа,на скамейке возле куста магнолии.Иногда под широким зонтом на плетёном кресле метрах в двадцати от него.Но смотрела она всегда в его сторону.Он впервые обернулся и ... не отвёл взгляд.И тогда она встала.И подошла.
- А Вы талантливы,юноша.Вы очень талантливы.Я давно наблюдаю за Вами.У Вас неповторимая манера письма.Поверьте,я понимаю толк в живописи.Позвольте представиться .Меня зовут Люси.Мне 65 лет.И ничего дурного в мыслях у меня нет.Просто хочу поговорить с Вами,пообщаться.Мне кажется,мы можем быть полезны друг другу.Ну что? Вы не против? Я могу украсть немного Вашего времени? Тогда приглашаю Вас на кофе.Сейчас за мной придёт машина и мы отправимся в одно чудное заведение.Складывайте пока свой скарб,я буду Вас ждать.
Женька обомлел от неожиданного предложения,но покорно встал и собрал свои пожитки.Что же ей от меня надо?!
———————-
Семья королевского нотариуса господина Ристича была одной из самых уважаемых в Субботице.И одной из самых состоятельных.Ожидаемый приход к власти в Сербии коммунистов ставил крест на многолетней карьере главы семьи.Перспектива стать конторщиком никак не привлекала.На семейном совете решение было принято единогласно.Чем ждать неизвестности здесь,в Сербии,лучше ехать за неизвестностью в США.Продав огромный салаш в несколько гектаров за городом,дом в центре Субботицы и квартиру в Белграде,собрав все средства с банковских счетов,коих было немало,семья тронулась в путь накануне Второй Мировой войны.
Блестяще образованный,владеющий семью языками глава семьи,достаточно быстро нашёл себе применение в далекой стране.Купив прекрасную квартиру прямо возле Централ-Парка,он вскоре открыл адвокатскую контору.И достаточно быстро дела пошли вгору.Бизнес процветал.А трудиться глава семьи умел.Пока остальные родственники пасли овец в далекой Австралии,а другие наслаждались жизнью и любовались волнами  Босфора в Стамбуле,его семья начинала свой путь к процветанию в Америке.При всем этом,детям не давали повода расслабиться.Достаточно быстро все члены семьи поняли - лень и Америка - понятия несовместимые.А потому трудились и учились все.Старшие сыновья успешно закончили престижные университеты и продолжили семейную адвокатскую традицию,а дочь мечтала о карьере художницы.Но мечте не суждено было сбыться.И тогда к адвокатской практике прибавился ещё один бизнес - Милица открыла художественную галерею.Именно эта галерея и перешла по наследству к ее дочери Люси.Поиск молодых и талантливых художников - вот , чем занималась она в свободное от работы время.Жека стал одним из тех , кому посчастливилось предстать на ее пути.И сегодня перед ним открывались невероятные перспективы.Но кто об этом знал тогда,в далеких 90-х....
—————-
- У меня был сын.Йошко,- начала Люси.Я расскажу Вам о нем позже.А пока расскажите-ка мне о себе.Откуда Вы? И что это ?! Где Вы это писали? Я не видела такой красоты сроду.Это что-то необыкновенное.Но ведь это не Америка.Тут нет таких церквей.Что-то подобное.но не столь величественное , есть у меня на историческо й Родине , в Сербии.А это где?
Люси рассматривала рисунки Жени.Среди многочисленных пейзажей,в основном,нью-йоркских,она заметила несколько,написанных им по памяти....
- Это Киев.Украина.Страна такая.Когда-то она была в СССР.И это одна из жемчужин Киева - Киево-Печерская Лавра.Я пытаюсь вспоминать и писать.Там невероятно красиво.Особенно весной,когда цветёт сирень.А это склоны Днепра.Киев расположен на этой реке. Это фантастически красивый город.И люди там фантастические.А ещё там осталась моя первая любовь.... 
Люси вглядывалась в рисунки,переворачивала листки,возвращалась к предыдущим и вдруг начала свой рассказ...
- Я уже говорила Вам,Юджи,у меня был сын.Талантливый художник.Я не льщу ему,это правда,он был талантлив.Здесь,в НЮ,он познакомился с девушкой.Итальянкой.Мы не стеснены в средствах,он зарабатывал неплохо,а она была из семьи достаточно необеспеченных эмигрантов.Но сын у меня один ,  и его счастье для меня было превыше всего.Они стали снимать квартиру неподалёку и жить вместе.Хоть и виделись мы редко,я все равно радовалась его любви.Одно время я путешествовала по Европе.Меня не было почти полгода.Приехав, позвала их на обед.От увиденного я чуть не потеряла сознание.Джулия исхудала до невозможности.На неё страшно было смотреть.И это несмотря на безупречный мейк-ап.Я заподозрила неладное.Но сын успокоил , мол, все нормально.Просто тусили в клубе до утра.А потом наши встречи стали все реже и реже.Сын находил кучу предлогов,чтобы отодвинуть их.Через полгода Джули умерла.Наркотики.Я чувствовала,я подозревала,но гнала эту мысль.А все оказалось правдой...
Люси остановилась,потянулась к изящной сумочке,вынула портсигар и длинный мундштук.Закурила. «Безупречный маникюр,безупречный внешний вид,безупречные манеры.Все в ней безупречно! Что ей надо от меня»,- думал Жека.
- Да.Так вот....Сын после ее смерти упал на самое дно.Я не успевала вытаскивать его с разных сомнительных вечеринок,где экстази и всякая прочая дрянь была главным атрибутом.Я потратила огромные средства,но...все напрасно.Он падал и падал все ниже.Спасла его не я - церковь.Уж не знаю,кто привёл его туда,но через несколько месяцев он стал другим.Таким же угрюмым и нервным,но с наркотиками было покончено.Я не находила себе места от счастья! Но это продлилось недолго.Он снова пропал.В этот раз надолго.Три месяца я искала его всюду.Все бесполезно.А однажды я получила письмо со странным штемпелем - Республика Сербия.Оттуда родом моя семья.Писал какой-то военный.На непонятном мне языке.Последней в нашем доме по-сербски говорила моя мама.Я практически языка не знаю.Но я нашла переводчика.Он прочитал мне то,от чего я поседела.Мой сын на войне.Он принимал участие в битве под Вуковаром.Ему оторвало ногу...Оказывается,сын уехал в Сербию....Так,видимо,решил затушить свою боль от потери любимой.А,может,просто хотел убить себя.Но самоубийство не вязалось с его характером.Война - вот это показалось ему выходом.Пока шло письмо,сын уже был в дороге домой.Естественно,вернулся он не в свою квартиру,я забрала его к себе,но общего языка с самым родным человеком на Земле я так и не смогла найти.Он окончательно замкнулся в себе.И , самое страшное, снова вернулись наркотики.Однажды он не проснулся.Слишком большой была доза.С тех пор я говорю «У меня был сын...» 
Люси замолчала,отложила сигарету.Потянулась к сумке,раскрыла ее и долго смотрела вовнутрь.Затем вынула из неё фотографию.
- Знаете,почему я подошла к Вам? Вот,смотрите....Это он.Мой Йошко.
Женя осторожно потянулся за фото,почему-то было страшно.Предчувствие,что ли...Он не ошибся.Со старого снимка , на котором с распростертыми руками был изображён молодой улыбающийся парень,на него смотрел ....он,Женька.В воздухе повисла тишина.Молчали оба.Казалось,что и птицы замолчали,и шума транспорта не было слышно.Они - Люси и Жека были единственными людьми в этом вязком вакууме.....Каждый боялся произнести хоть слово.Первой заговорила она.
- Я увидела Вас и едва не упала.Такого не бывает.Вы полная копия моего сына - та же манера склонять голову набок,та же манера трясти левую руку,когда нервничаете.Вы даже кофе пьёте одинаково , держа чашку двумя руками.Как это возможно? А уж мольберт.Когда я увидела Вас за мольбертом,сомнений не осталось.Я должна была познакомиться с Вами.И вот мы сидим рядом.Вы и я.И у меня есть предложение к Вам.
Женя начал приходить в себя.Вся эта история прилично встряхнула его.Пока он ловил каждое слово,не в силах произнести что-то в ответ.А потому выдохнул и 
кивнул.
- Я уже говорила Вам , что Вы талантливы.А у меня дар открывать таланты и помогать им.Я в этом деле профи,поверьте.Так вот.Сначала я просто хотела предложить Вам работать со мной.Через мои руки прошло немало талантов,многие сегодня очень состоятельные и знаменитые художники.Из Вас я сделаю самого успешного и богатого.Обещаю! Нет,денег я  Вам не дам.Я дам Вам образование,дам площадку - у меня шикарная галерея.Я введу Вас в определенные круги , остальное Вы сделаете сами.Я это вижу.Но только у меня есть условие.
Женя напрягся....Нет.Господи.Только не это.Не надо.Так все хорошо начиналось.........
- Среди Вашего длинного рассказа о Родине,- продолжала Люси,- я услышала одну фразу - там,в далёком Киеве Вы оставили свою любовь.Пообещайте мне,когда Вы станете на ноги настолько крепко,что скажете себе «Все! Я успешен!», первое,что Вы сделаете,поедете к ней и будете стоять на коленях,умоляя быть с Вами.Причем,неважно,где.Лучше,конечно,здесь.Главное,чтобы она простила Вас.Ибо любовь - это смысл жизни.Вы согласны? Я делаю это ради себя  и своего сына.Вы сделаете это ради себя и неё.Ну? По рукам? 
Ох,ничего себе поворот....Ещё утром он раскладывал гамбургеры на углу в закусочной,а сейчас незнакомая  женщина предлагает перевернуть всю его жизнь с ног на голову.От такого отказываются? Или это сон? Стоп.В чем подвох?! Так не бывает......
- Люси.Откровенно говоря,Вы повергли меня в шок.Я пока не могу все это сопоставить,переварить и ответить сходу.Нет! То,что Вы предлагаете - это волшебство,это лучшее предложение в моей жизни.Но...
- Так.Давайте без истерик.Я вынашивала этот план не один месяц.Я много раз видела Вас в той забегаловке на углу.Я знаю,где работают Ваша мама и Ваш отец.Все продумано до мелочей.К тому же я не даю Вам рыбу.Я даю Вам удочку.Знаете такую притчу? Вот! Единственное,что я сделаю - усажу Вас на прикормленное место.Остальное Вы сделаете сами.Итак? Я жду.По рукам? Ну и славно.Вот бумаги,почитайте,у Вас на размышление сутки.Иначе опоздаете на учебу.И не на тот «кружок» самодеятельности,который Вы посещали в свободное время.Это серьезное учебное заведение.Готовьтесь.Джеймс ,мой секретарь,заедет за Вами завтра к 20:00.Да,кстати,сейчас он как раз беседует с Вашими родителями.Не волнуйтесь,он убедит их в искренности и честности моих намерений.А сейчас спать.Я тоже подустала.До завтра.Спокойной ночи.
Женя вскочил первым,успел протянуть руку даме,отодвинул стул и кинулся открывать дверь кафе.Зря.Судя по всему,Люси была здесь частым гостем,потому что двери с улыбкой ей открыл официант.Она грациозно кивнула головой и вышла.Женя вышел следом .Домой ехать через пол-города.Но он шёл пешком и думал.Слишком все это выглядело сказочным и неправдоподобным.Ладно.Ночь на размышления у меня есть.А там....Там видно будет.
—————————
Вначале она ещё ждала от Женьки писем.Месяц ждала,второй,третий.А потом перестала.Нет,она не забыла его.Это он ее забыл.Он вычеркнул Нелю из своей американской жизни.Ну и пусть.У неё вступительные экзамены прошли.Она студентка.Начинается новый этап в ее жизни.И все же.....как он там? Все ли с ним в порядке? Господи! А вдруг он голодает где-нибудь под мостом?! А вдруг они бомжи вообще?! Ну почемуууууу он не пишет! Господи! Помоги ему! Пожалуйста!!! 
——————
Он не курил.Терпеть не мог эту идиотскую привычку.Сейчас он держал в руках какую-то новомодную штучку - «электронную сигарету», чтобы казаться своим среди снующих вокруг с умными лицами людей.Практически все они были ему незнакомы,это Люси устроила вечеринку.Светский раут.Шампанское,икра,дамы в вечерних платьях,дорогие парфюмы,лимузины...Все это было так непохоже на его предыдущую жизнь,на окружавших его однокурсников,на соседей по Брайтон-бич.Но он здесь.Сейчас.И он - один из них.Слишком хорошо,чтобы быть правдой.А.Главное.Это первая выставка .Его выставка.И она в одной из лучших галерей Нью-Йорка.Господи,это правда?! Да.Это правда.И уже завтра во многих газетах появится имя Юджин Клин .А ещё через полгода за полотнами молодого, талантливого и ужасно дорогого художника из далекого Советского Союза (так было надо для придания имиджа «беглеца из нищеты») будут выстраиваться очереди.Ну что?! «Я успешен!» Но сейчас,стоя на террасе с сигаретой,толку от которой никакого,он ещё всего этого не знал.Он стоял,оглушенный успехом,неузнанный никем (тоже часть плана) и вспоминал тот последний день в Киеве.И почему-то в голове звучали те «Белые розы! Белые рооозы!» По-моему,это любимая песня Нели.Интересно,где в Киеве можно купить штук 100, нет! 101 белую розу?
—————-
Ненавижу! Ненавижу эту дрянную песню.Под неё мы расставались.Неля вытянулась на простыне,коньяк кружил голову,но в теле присутствовала какая-то необыкновенная легкость,как будто она парила где-то над постелью)))Тю.Ну надо же.Напилась?)))Уже засыпая,она вспомнила все,что произошло с ней за эти долгие десять лет...
—————
- Нелька! Слушай! Тут такое дело.....Ты фильм «ХХХХХХХ» смотрела? Ну,нашумевший такой! Культовый,можно сказать.Нет? Ты с ума сошла? Сейчас о нем только и говорят.Ой,нет.Сейчас уже не говорят.Пару лет назад говорили.Вернее,лет восемь назад.Короче.Там главный герой - красавец необыкновенный! Ну Ален Делон отдыхает и тихо плачет в уголочке! В него все влюблены были! Глаза голубые-голубые! Волосы чёрные! Короче,отпад!!!
Ленка тараторила и тараторила,Неля понять не могла,при чем тут она...
- Короче.Там такая история...Мы познакомились пару дней назад.В компании.Я,конечно,чуть не рухнула,когда его увидела.Сидели,разговаривали.Он выпил и давай о жизни своей горемычной - женился,разводился.Тетки на шею вешались,как бусы...И остался он в результате один-одинёшенек! Ну ты представляешь?! Он и один! Невероятно! 
Ещё непонятнее....Я при чем???
- Так вот.Тут две проблемы нарисовались.И ты нужна позарез!!! Я,конечно,тебе его уже не отдам,это понятно,он мой навеки,но....не могла бы ты его приютить хоть на время?! У тебя же квартира пустая.А у меня букет - муж,сын.Ну и свекровь,дай ей Бог здоровья.
Здрасьти,пожалуйста.Ничего себе,предложеньице,называется.
- И ещё...Он там подрался , вроде как.Короче,зуб ему выбили.А ты ж у нас стоматолог.Это ж по твоей части? Я заплачу.Денег,как я понимаю,у него тоже негусто...
- Лен! Ты вообще в своём уме?! Ты хоть поняла,что мне предложила?! Поселить у себя немолодого,пьющего,беззубого , бездомного и безденежного любимца всех дам середины 90-х! Ты это серьёзно? 
- В смысле? Конечно, серьёзно.А что тут такого? Ты обеспеченная барышня,я в долгу не останусь,ну почему бы не помочь человеку.Тем более,Пашка его уже выгоняет...Ну Нель.Ну пожалуйста.Клянусь.Я что-то придумаю! Ну хоть ненадолго,а? 
- Так.Ни о каком поселении и речь не идёт.С зубами разберёмся.Остальное - нет! Категорически! Понятно?
- Ну Нель.....
- Категорически! 
- Ладно.Когда хоть в клинику прийти?
- Завтра давай.И не опаздывайте! Терпеть не могу опозданий!
Шесть лет,как от Женьки ни звука...Все.Тема закрыта.И все же! Господи! Храни его там! Пожалуйста! 
————————-
Потихоньку привыкаешь ко всему.К славе быстрее.Но и бьет она по самому больному.Эйфория успеха поднимает тебя над Землёй.Синяки от падений болят долго.Звезда,упавшая с неба прямо Женьке в руки,была очень горячей,руки обожгла.Да и по душе прошлась.В общем,почувствовал он то,что называется «занесло», даже Люси позванивать перестал.Правда,потом долго стоял перед ней на одном колене,испрашивая прощения.Простила,поняла.Проходила это не раз и не два с другими - молодыми и зелёными.Дал слово вычеркнуть тот «звездный» период из жизни .И стал подумывать о поездке в Киев.Но плотный график выставок не оставлял времени.Да и заказов от многочисленных компаний и частных лиц было выше крыши.Все! Через год еду! Боже...Снова забыл письмо написать...Это ж сколько лет прошло?! Пять? Шесть? Все! Завтра напишу.Кровь из носу напишу! Клянусь!
———————
Ну хоть кофе он сварить может? Или все я?! 
- Саша! Тебе с сыром бутерброд или с рыбкой? Саааш! Ты меня слышишь? 
Спит.Он все время спит...Я на работу опаздываю.А завтра в командировку! 
- Саш! Кофе на плите! Я умчалась.
Она и сама не поняла,как так получилось.Сначала он оказался в ее рабочем кабинете.С раскрытым ртом.Конечно,она его узнала.Ну как же.Ленка права.Да,в него влюблялись сразу,с первого взгляда.Не все,но многие.И он действительно был мега-популярен.И не только у нас.Многие зарубежные кинокомпании приглашали его на съемки.Такое лицо было нарасхват.А потом перерыв.Длительный и болезненный - вот вчера ты кумир поколения,а сегодня .... никто.И началось - друзья,алкоголь,вечеринки.За чужой счёт,разумеется.Вот и итог - жены вразбежку,детей нет.Один,как перст.А тут Неля - тоже одна.Как перст...
Поженились они через полгода.Просто привыкли быть вместе.Не любовь - привычка.Она нужна ему,он ...Не то,чтобы нужен.Куда деть-то его? Короче.Так получилось.
А ещё через месяц она нашла ему роль.Как и у революционеров,у продюсеров зубы тоже иногда болели.И началась новая,светлая полоса в до того задрипанной жизни кумира женщин бальзаковского возраста.Сериалы сыпались один за другим.Снова фестивали и конкурсы! Причём он,опытный и безумно популярный, уже член жюри.Сначала Неля была желанным гостем на всех мероприятиях,потом муж стал вежливо уговаривать ее оставаться дома,а в конце концов просто перестал приглашать,иногда и в известность не ставил,куда и зачем идёт.
 Неля поначалу обижалась,а потом просто перестала его замечать - жили,как соседи.Не выгонишь же.Вроде,как супруг...
Сегодня она оделась особенно торжественно и нарядно.Жаль,что он не видит - спит.Вечером в Премьер-паласе конференция.Международная.Она делает доклад.Первый в своей жизни.Еще вечером за ужином она аккуратно намекнула мужу,что хотела бы пойти с ним,представить своим коллегам бы хотела,но он уткнувшись в телевизор , махнул рукой ,мол,вот завтра и поговорим.А потом,спохватившись,вдруг хлопнул себя ладонью по лбу. 
 - Ой! Прости! Никак! Завтра же финал фестиваля! Я никак! В жюри же.Прости! В другой раз.Обязательно.Обещаю! 
Ну,нет,так нет.Хотя,обидно.
Доклад прошёл на «ура» в буквальном смысле этого слова.Молодому доктору,одному из самых молодых докторов,допущенных на столь высокую трибуну,аплодировали маститые коллеги из разных стран.Это был успех.
Взяв бокал холодного шампанского,Неля вышла на балкон отеля,вдохнула вечерней прохлады, чувствуя необыкновенный прилив энергии.Эмоции зашкаливали! Она была на седьмом небе! Где-то недалеко,на этом же длинном балконе,заливалась смехом молоденькая барышня.Ее смех иногда прерывал до боли знакомый хрипловатый баритон.Все ещё улыбаясь,Неля заставляла себя вспомнить,где она слышала этот голос.Вспомнила.Дома.У себя дома.Улыбка сползла с ее лица,она медленно обернулась и пошла на голос.Вдали стояла, обнявшись,парочка.Он ласкал длинные волосы своей девушки,она ласково гладила его по спине.Видно было,что люди влюблены и не могут оторваться друг от друга.Появление Нели в непосредственной близости вызвало шок у мужчины.Девушка все ещё продолжала водить рукой по его спине,он пытался остановить ее руки.Неля судорожно допивала своё шампанское.Вообще-то стоило что-то сказать,хоть сцену закатить,скандальчик небольшой,но...она не умела.Просто глотала шипучий напиток и беззвучно плакала.Потом поставила бокал на стоящий перед парочкой стол,развернулась и ушла,откуда пришла.Ничего не понимающая барышня спрашивала «Саааш.Это кто ваще? Что это было,Сааш?» Но он уже освободился от ее ласковых рук и мчался вслед за женой.
Вечер был испорчен.У двоих.Нет.У троих.
Тогда-то она и открыла тот самый коньяк,который допивала сегодня,сидя в венском кресле на балконе своей квартиры.
Ну а Сашу она выселила.С огромным трудом,правда,но выселила.Кумир дам 90-х снова был в поисках счастья.В который раз.Кстати! Говорят,подумала Неля почти засыпая,что он совсем недавно в каком-то нашумевшем фильме снялся.О балете,что ли.Говорят,и выглядел не очень...Ну,то уже его проблемы.Все! Спать! Спать! Спать!

Вадим Гомберг
Сумасшедше здорово.!!!

Ирина Сушко

Потрясающе 😍 вовремя я увидела Ваш рассказ и сегодня прочитала все части не дожидаясь продолжения👍🏻😍, правда немного в разброс 😉, но на одном дыхании! Очень увлекательно! Представила себе как сценарий фильма! Если также талантливо подойти к этому вопросу, то и фильм 🎥 получился бы прекрасный 😍! «О жизни. О любви! И, конечно, о любимом городе!»
БЛАГОДАРЮ

Наталия Слободян
Наваждение этот Ваш «рассказ»,Светлана...
Я это вижу.
Девичий виноград,переливающийся шёлк,Люси,её руку,протягивающую фотографию сына...даже пощипывающее рот шампанское!..
Следовательно,в моём восприятии это киносценарий.
Как же сочно Вы пишете!..
И лифт лязгает знакомо,и ночь у Днепра колдует собственными воспоминаниями...
Спасибо вам.
