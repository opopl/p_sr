% vim: keymap=russian-jcukenwin
%%beginhead 
 
%%file 29_01_2022.fb.novohackij_oleksandr.kiev.1.lovushka_privychnosti
%%parent 29_01_2022
 
%%url https://www.facebook.com/oleksandr.novokhatskyi.7/posts/976672339907917
 
%%author_id novohackij_oleksandr.kiev
%%date 
 
%%tags chelovechnost,obschestvo,psihologia,ukraina
%%title ЛОВУШКА ПРИВЫЧНОСТИ
 
%%endhead 
 
\subsection{ЛОВУШКА ПРИВЫЧНОСТИ}
\label{sec:29_01_2022.fb.novohackij_oleksandr.kiev.1.lovushka_privychnosti}
 
\Purl{https://www.facebook.com/oleksandr.novokhatskyi.7/posts/976672339907917}
\ifcmt
 author_begin
   author_id novohackij_oleksandr.kiev
 author_end
\fi

ЛОВУШКА ПРИВЫЧНОСТИ

В ловушке неразделимости понятий страна и государство (держава) находится
огромное количество людей украинских. Видимо так хотелось своего
государственного образования, что не имело значения какое оно и для кого
именно.

Украинцам необходимо научиться жить в стране, в громаде, в собственном
пространстве личных отношений и категорически выйти(!) из пространства державы.
Держава, мир чиновников. Мир, не имеющий ни малейшей власти над людьми, а
совсем наоборот – всецело подчинённый выполнению функций по обеспечению жизни
людей и их благополучия.

Необходимо мужество и взрослость чтобы выйти из державы (всецело подчинённого и
безвольного пространства) в своё живое пространство краины (края Любви и
человекости). 

Огромное мужество и воля требуются заставить себя воспринимать мир людей не
через хотелки чиновников, а понимать всю природу образования коллективных
пространств сожития.

Государственными образованиями владеют только те, кто своей жизнью гарантирует
его существование. За счёт кого существует государственное образование Украина,
кто оплачивает его своими жизнями?

В украинском людском пространстве ярко выраженный конфликт между
государством-державой (машиной Власти) и человеком (народом). Сама структура
системы сожития есть противочеловековой. Направленна против человека и на
подавление его Воли. Наша «плекана та омріяна», на самом деле есть машиной
канализации Воли, криминализации Свободы и пожирания состояния людского
добробуту. Вся эта система заточена на возвеличивание самой машины (насилия)
государства.

Ведь у нас реализована схема, в которой именно государственные органы порождают
людское пространство отношений. Смотрите в нашу историю проявления – Верховный
Совет Украинской Социалистической Республики объявил/провозгласил о создании
независимого государства Украина. Ничего не смущает?

При этом, Основной закон (он же техническое описание машины сожития) утверждает
в ультимативной форме, что именно народ-люди есть единственным источником
самого пространства отношений в землях подконтрольных. При этом от имени
народа-людей всё-всё осуществляют некие особи под названием «Власть».

Подобно некому регенту над несовершеннолетним правителем, такой себе
юридический субъект «Власть» осуществляет всю хозяйственную и прочую
деятельность вместо людей. Ничего не смущает?

Украина выстроена таким образом, что буквально является раем для чиновника.
Пространство отношений, где источником самих отношений является якобы чиновник
в виде юридического лица «Власть». А юридическое лицо «Народ» лишено любого
права влиять на ситуацию и решения чиновников. Только и исключительно через
противогосудасртвенные действия - исключительно через бунт и всяческие майданы.
Все отношения между человеком-народом и «Властью» (корпусом чиновников) подобны
отношениям человека и банды вооружённых разбойников.

Такой конфликт легко снимается/разрешается принятием простой истины. Не народ
для государства, а государство для народа. Конфликт будет устранён, когда всё
станет на свои места. Когда статус Народа будет выше статуса Власти. При этом
Народ будет проявлен в виде огранизованных/формализованых структур громад,
которые владеют землёй государственного образования. 

Нас ждёт весьма яркий период нашей жизни. Период осознания себя и своего права
на свою землю. Период революции элит. Период восстановления и проявления
нормального порядка вещей.

Украинцам необходимо выходить из государственного пространства, избегать
словесных формул типа – погода в державі, моя держава, національні інтереси,
державні пріоритети ... 

Люди не живут в государствах, там живут только чиновники. Не стоит записывать
всю страну людей в мир чиновников, особенно от самого рождения ...

Вопрос лишь в Воле и понимании серьёзности всего написанного.
