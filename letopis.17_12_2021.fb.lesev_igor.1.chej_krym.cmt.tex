% vim: keymap=russian-jcukenwin
%%beginhead 
 
%%file 17_12_2021.fb.lesev_igor.1.chej_krym.cmt
%%parent 17_12_2021.fb.lesev_igor.1.chej_krym
 
%%url 
 
%%author_id 
%%date 
 
%%tags 
%%title 
 
%%endhead 
\zzSecCmt

\begin{itemize} % {
\iusr{Кирилл Рыжанов}
ПолитВера что ли?))

\begin{itemize} % {
\iusr{Игорь Лесев}
\textbf{Кирилл Рыжанов} вижу специалиста)

\iusr{Кирилл Рыжанов}
это уже мем в определённых кругах))

\iusr{Aleksandr Bilogur}
\textbf{Игорь Лесев} Вы к Вере ходите? Чего ж тогда удивляться. Куда там этому несчастному Соловьёву.
\end{itemize} % }

\iusr{Yuri Petrenko}
Отличный текст!

\iusr{Марина Фокина}

Это предсказуемый итог. Джангиров ещё держится ими в приличных людях, а так
никакой критики в их адрес.)Я забанена давно по причине недостаточной любви к
России и Путину. Хотя отношу себя не просто к русофилам, а к имперцам. Но
недостаточно, по их мнению)) Я тоже считаю, что Крым отжали, но я за крымчан рада.
Правду говорить легко и приятно. А это просто секта какая-то. Никакой
объективности.

\begin{itemize} % {
\iusr{Михаил Максимчук}
\textbf{Марина Фокина} 

Правду какую??? То что Крым был русский на протяжение веков после его
завоевание это вам что нибудь говорит?? А урод Хрущев его так просто отдал это
как ??? А крымчане молодцы и при чем здесь Россия был референдум...


\iusr{Марина Фокина}
\textbf{Михаил Максимчук} 

Историческая справедливость восторжествовала. Но сделано это было вне закона.
На референдуме, согласно Конституции Украины, должна была голосовать вся
Украина, а не только Крым. Так что отжали. А Донбасс не захотели отжимать, вот мы
и мучаемся 8 лет. И меньше пафоса и вопросительных знаков.

\end{itemize} % }

\iusr{Tatyana Meylakhs}
Все верно, так оно и есть. А я думала, почему мне нравятся украинские ватники и вызывают ужас российские. Теперь понимаю.

\iusr{Любовь Чуб}

Первое. Люди везде одинаковы. Особенно в диалоге. Каждая сторона считает только
себя правой и ей очень не нравится, когда ей возражают. И пусть кто-то один
будет хоть тысячу раз прав (а так не бывает: колокольни разные), ничего не
изменится. И это не зависит ни от национальности, ни от политических
пристрастий. Почитайте диалоги наших, приверженцев Наших и неНаших, к примеру).

Второе. Великодержавный шовинизм, зародившийся и укрепившийся ещё в годы Союза,
никуда не исчез. Ни на каком уровне.

Третье. Донбасс тоже все!

Четвёртое. Не обижайтесь. У них есть причины к нам так относиться.

И пятое. Вы все равно молодец и умница!

\iusr{Валерий Дейнека}

Говорить можно что угодно, но по факту Россия на подъеме и развивается, а
Украина чахнет и катится в пропасть.

\iusr{Валя Бабич}

Все точно.Но украинская власть испытывает комплекс неполноценности по отношению
к России. Вместо того, чтобы проанализировать все, что произошло в 14 году,
сделать выводы и решить стратегические вопросы развития страны- у нас сплошные
лозунги и призывы. (Усрамся- не дамся).

\iusr{Юрий Зыбин}

Тут форумчане приятно удивили деликатностью и вежливым обращением друг к другу
выражая своё мнение по этому архиважному и сложному вопросу Как всегда Игорь
великолепно как говорят всё по полочкам разложил на составляющие понимая что
удары могут последовать с обеих сторон Неожиданно для всех Вас я начну свой
рассказ по Крыму


\iusr{Святослав Крижевич}

Отлично разложенно, на удивление. Взвешенно, грамотно, по полочкам. Не согласен
с пунктом 6, там есть нюансы, а в остальном приятно удивили. Логика по нынешним
смутным временам это запредельная роскошь:)


\iusr{Святослав Крижевич}
Кстати а видео эфира есть ? Можете выложить?

\iusr{Евгений Языков}
Давно убедился, адекватный человек везде будет костью поперек горла.

\iusr{Natasha Varetskaya}
Мы уходим (безвозмездно) в период \enquote{кто сильнее и рациональнее, того и Крым}.
да, и рулит симпатия..... кто интереснее и загадочнее : Путин или условный укропрезидент или и.о.
Укропрезиденты вот совсем не альфа... путин - альфа, даже при всех косяках... поэтому многое народ прощает.
украинским ничего не простят.

\iusr{Николай Воронин}
Даже не начав читать комменты, сразу понял, что к Верке сходили ))

\iusr{Николай Воронин}
Да и Донбасс похоже тоже всё. Просто об этом по разным причинам никто не говорит... пока не говорит..

\iusr{Влад Винница}

Игорь, у меня (как у чёткого адекВатника) тоже вызывает удивление: почему в
союзном государстве Республика Казахстан задрачивают русскоязычных по полной,
но Российская Федеоация даже и не пикнет!

\begin{itemize} % {
\iusr{Ирина Порохненко}
\textbf{Влад Винница} Там задрачивать пытаются местные ницои и стерненки, но там на одного такого завели уголовное дело за разжигание и он смылся из страны.
То есть государство в корне пресекло подобное.
А у нас сам президент говорит \enquote{чемодан вокзал}.
Нормальному государству не нужны разжигание, ненавись, распри на национальной почве, это путь к гражданской войне. Любое нормальное государство такое будет пресекать, если только....
Если только государство само не исповедует идеологии нацизма.
\end{itemize} % }


\iusr{Дмитрий Коломийченко}
О, так вы к Вере попали. Да, там не забалуешь. )))

\begin{itemize} % {
\iusr{Игорь Лесев}
\textbf{Дмитрий Коломийченко} я так понял, это ру-вариант Цензор.нет

\iusr{Дмитрий Воронин}
Вера правее Путина, вообще Украину не признает)))

\iusr{Марина Фокина}
\textbf{Дмитрий Воронин} 

Вера специалист широкого профиля) \href{https://youtu.be/dm6KKMRG_QI}{%
Вера Бытко Стиль, имидж, красота Мода и Стиль с Верой Бытко, youtube, 24.06.2015%
}

\end{itemize} % }

\iusr{Берой Егоров}

Всё не совсем так. То что на момент развала Союза всё его население, страдало
разжижением мозга - это факт. При этом, если быть обьективными, то
локомотивами развала, особо буйно помешанными его заинтересантами, были, если
опустить Прибалтику, именно Грузия и Украина. Говоря образно, слабоумие части
России, тогда называемой РСФСР, было скорее пассивным, а не деятельным. Но не
суть. Наиболее показательными же стало постразвальное развитие осколков страны.
С момента распада страны, Украина уже единолично несёт ответственность за то
что именно она всегда была тормозом всех попыток интеграции государств
постсоветского пространства


\iusr{Александр Готвальд}
Похоже на попытку оправдаться.

\iusr{Natalia Zadorozhnaya}

Грамотно изложенные аргументы, обоснованная фактами оценка событий,
последовательное и логическое мышление - всегда красная тряпка туполобым
самонадеянным идиотам. И там и здесь.

Ещё раз порадовалась тому, что читаю. Хотя бы в силу полного согласия с каждым пунктом.
К сожалению, даже с шестым. Не вернётся. Очевидно.

А вот с Турчинова за это (и не только с него) спросить было бы не лишне. Как и
со всех авторов переворота и кошмарных его итогов по самому масштабному
уничтожению нашей страны.

\iusr{Дмитрий Воронин}
Гмм... уход чуждого Украине полуострова Крым создаёт неразрешимые проблемы???
Насильно устранить  "агрессию"?
Да,как однако глубоко 60 лет впитались и отбрасывается лихо 170 лет

\iusr{Вячеслав Савченко}

хороший текст, спасибо. Небесспорный, но тем не менее сильный.

Можно проговорить, к примеру, \enquote{госпереворот}.

Игорь, не бывает \enquote{госпереворотов} в государстве, когда все силовики под
контролем, все на местах, а всего-то на одной микроскопической площади и в
одном городе творится чего-то непонятное.

Игорь, это нонсенс. Это что-то другое, но никак не \enquote{госпереворот}.

Кроме того - легитимная и законодательная ветвь власти даже при этом на месте.
Очень странные пошли \enquote{госперевороты}, я вам скажу.

То есть. Третий тур извращенных выборов в 04-м - это еще не \enquote{переворот}.
Подтасовки ЦВК им. Кивалова в 10-м - вовсе не \enquote{переворот}. Отмена
дваждынесидельцем Конституции страны - вовсе не \enquote{переворот}.

А вот перманентная драка двух сотен активистов с тремя сотнями беркутОв - упс. Вот оно, да.

Кстати, а никто не выяснял у ростовского сидельца - чё тот вообще на лыжи-то
стал?  @igg{fbicon.wink} 

\iusr{Ирина Думчева}

Все верно, по пунктам.. хоть и меня, как \enquote{ру} с украинской родиной, коробит. но
каждый имеет право на свой глоссарий.

\end{itemize} % }
