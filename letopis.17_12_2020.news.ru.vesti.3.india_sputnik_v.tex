% vim: keymap=russian-jcukenwin
%%beginhead 
 
%%file 17_12_2020.news.ru.vesti.3.india_sputnik_v
%%parent 17_12_2020
 
%%url https://www.vesti.ru/article/2500201
 
%%author 
%%author_id 
%%author_url 
 
%%tags 
%%title Получены первые образцы "Спутника V", сделанного в Индии
 
%%endhead 
 
\subsection{Получены первые образцы \enquote{Спутника V}, сделанного в Индии}
\label{sec:17_12_2020.news.ru.vesti.3.india_sputnik_v}
\Purl{https://www.vesti.ru/article/2500201}

\ifcmt
pic https://cdn-st1.rtr-vesti.ru/vh/pictures/xw/307/883/5.jpg
\fi

Россия получила первые образцы вакцины от коронавируса\Furl{https://smotrim.ru/theme/2366} "Спутник V", которая
произведена в Индии. Об этом глава Российского фонда прямых инвестиций Кирилл
Дмитриев сообщил в эфире телеканала "Россия 24".

У РФПИ есть договоренности с четырьмя индийскими производителями. Всего Индия
произведет для России около 300 миллионов доз вакцины в следующем году.

Также Дмитриев рассказал о "Спутнике V лайт".\Furl{https://smotrim.ru/article/2500209} Он пояснил, что "Спутник – V"
является самым главным препаратом, "это некая платформа вакцин". "Спутник V
лайт" – это компонент, который будет предоставлен компании AstraZeneca. Он
позволит многим странам давать временное, но быстрое решение – останавливать
пандемию.

Ранее Владимир Путин во время Большой пресс-конференции рассказал о
лайт-вакцине\Furl{https://smotrim.ru/article/2500192} с эффективностью 85\%, которой можно будет привить сразу десятки
миллионов россиян.
