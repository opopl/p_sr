% vim: keymap=russian-jcukenwin
%%beginhead 
 
%%file 11_12_2020.sites.ru.diletant_media.kotelnikov_konstantin.1.poxod_batyja_russia
%%parent 11_12_2020
 
%%url https://diletant.media/articles/45300006/?utm_medium=kartoteka
 
%%author Котельников, Константин
%%author_id kotelnikov_konstantin
%%author_url 
 
%%tags batu_han,ancient,russia
%%title Поход Батыя на Русь навсегда изменил русские земли
 
%%endhead 
 
\subsection{Поход Батыя на Русь навсегда изменил русские земли}
\label{sec:11_12_2020.sites.ru.diletant_media.kotelnikov_konstantin.1.poxod_batyja_russia}
\Purl{https://diletant.media/articles/45300006/?utm_medium=kartoteka}
\ifcmt
	author_begin
   author_id kotelnikov_konstantin
	author_end
\fi

Русь после Батыя, казалось, едва ли воскреснет. Прошли годы, но путешественники
то и дело видели останки тысяч павших и руины навсегда разрушенных городов.

\index[rus]{Русь!История!Поход Батыя на Русь, 11.12.2020}


\ifcmt
  pic https://diletant.media/upload/iblock/e0b/e0b6fe09be1e834b7cd7d4076f9f16ec.webp
\fi

\subsubsection{Монгольское нашествие на Русь (1237 — 1240)}

Пойти на Русь монголы решили ещё в 1235 году. Первое столкновение русских и
половцев с завоевателями на р. Калка в 1223 году показало, что и на западном
направлении тьмы кочевников — непобедимы. Внук Чингисхана Батый (он же
Бату-хан) подготовил и возглавил поход в 1236 году. Войско его, по подсчётам
историка В. Кучкина, вероятно, насчитывало около 135 тысяч воинов, что для того
времени — не просто много… Никто в мире не мог противостоять такой армии и
победить, а раздробленная Русь — тем более.

Разгромив Волжскую Булгарию, Батый\Furl{http://diletant.media/articles/45265948/} зимой 1237 г. вторгся в пределы русских
княжеств. Взяли Рязань\Furl{http://diletant.media/articles/43681345/} (21 декабря) и умертвили её князя Юрия Ингваревича,
затем Коломну, Суздаль, Владимир\Furl{http://diletant.media/articles/39130002/} (7 февраля 1238 г.), Москву, князя
владимирского Юрия Всеволодовича одолели и убили в сражении на р. Сити.
Покорение русских земель сопровождалось жестокими битвами, грабежами и
поджогами. Города и монастыри превращались в обугленные руины. Ростов,
Ярославль, Торжок, «град злый» Козельск, что оказал ожесточённое сопротивление,
Переяславль, Юрьев, Тверь, Чернигов, Муром, Гороховец и другие места — три года
(1238 — 1240) монголы побеждали, захватывали, насиловали, порабощали, пытали и
убивали, часто без разбору пола и возраста. «А град пожегше, и люди избиша, и
манастырь паграбиша», — типичная картина (из летописного описания произошедшего
в Чернигове). 

\ifcmt
  pic https://diletant.media/upload/medialibrary/b37/b37ae90be659561436048eca1baaf82a.webp
  caption Поход Батыя. Источник: Google Sites

	pic https://diletant.media/upload/medialibrary/b7c/b7c4c21758b494a14d8adef4df7363ba.webp
	caption Оборона Рязани в 1237 г. Источник: poputi.su
\fi

После долгой подготовки Батый взял Киев (ноябрь 1240 г.). И снова: Каменец,
Владимир Волынский, Галич… К концу 1240 года монголы покорили большую часть
Руси (уцелели Новгород, Смоленск, Пинск, Витебск, Полоцк). Только после этого
монголы вторглись в Венгрию и дальше на Запад — в страны, которые они избрали
конечной целью похода. Об этих кровавых событиях писал магистр Рогерий и
другие хронисты.

Варварское опустошение — вот что учиняли завоеватели всюду, где оказывались. И
чем сильнее им сопротивлялись, тем хуже обходились с поверженными. В Козельске
они истребили всех жителей, включая младенцев — «избив си и не пощаде от
отрочат до сосущих млеко». Пепелище и трупы на месте прежде цветущих городов.
Современники и летописцы оставили нам страшные, прямо апокалиптические описания
Руси после Батыя. 

