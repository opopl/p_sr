% vim: keymap=russian-jcukenwin
%%beginhead 
 
%%file 25_05_2021.fb.bilchenko_evgenia.4.spasibo_piter_pushkin_i_klincy
%%parent 25_05_2021
 
%%url https://www.facebook.com/yevzhik/posts/3924743984227349
 
%%author Бильченко, Евгения
%%author_id bilchenko_evgenia
%%author_url 
 
%%tags 
%%title БЖ. Спасибо, Питер, Пушкин и Клинцы!
 
%%endhead 
 
\subsection{БЖ. Спасибо, Питер, Пушкин и Клинцы!}
\label{sec:25_05_2021.fb.bilchenko_evgenia.4.spasibo_piter_pushkin_i_klincy}
\Purl{https://www.facebook.com/yevzhik/posts/3924743984227349}
\ifcmt
 author_begin
   author_id bilchenko_evgenia
 author_end
\fi

Ну, вот. Теперь можно. Я поздравляю абсолютно всех победителей этого
сильнейшего в этом году Международного фестиваля. Все достойны. Но я очень хочу
выделить из всех авторов двух людей, которые запали мне в душу тем, что
выражают в своих текстах мою веру в метафизику русской души. Они выразили наш
архетип в форме жёсткого рэпа и рэп-тоники. Они сделали традицию Живой. Они не
забыли, что дух важнее жизненного комфорта. 

Это я к сегодняшней дискуссии, что есть вещи важнее жизни. Отдать свою жизнь за
ближнего, за брата, за будущее. За Родину. За духовность, да, есть такое слово.

\ii{25_05_2021.fb.bilchenko_evgenia.4.spasibo_piter_pushkin_i_klincy.pic}

И я надеяться не смела, что девушка из города Пушкин и парень из города Клинцы
Брянской области, так радикально синтезировав фольклор, тему Великой
Отечественной войны, Бога в содержании и ультрамодерн в подаче, смогут пройти в
победители на официальном мероприятии и подарить мне глоток надежды, что не все
молодое поколение пиитов пропиталось сладко-депрессивным богемным либерализмом
генерации Z. 

Евгения Удальцова и Андрей Бондаренко, спасибо за победу над пластмассовым миром. 

Мое любимое стихотворение фестиваля привожу ниже. Это мой личный выбор.

Евгения Удальцова. "Дом со звездой"
Он быстро родился, выпустил мамкин подол, 
Скоро заговорил, скоро пошёл, 
Ловил на лету стрижей, хватал на бегу ремёсла. 
Мать говорила: с Лёшкой мне очень просто. 
Лёшка хотел успеть. Лёшка спешил расти. 
Лёшка дожил до двадцати. 
В сорок четвёртом вручили матери похоронку. 
Она замерла: нет моего ребёнка. 
Время бежало, за Лёшкой отправив мать, 
за матерью - память. Некому вспоминать. 
Что после себя Лёшка успел оставить? 
Звезду на торце избы в чёрной оправе. 
Мало ли Лёшек стало такой звездой? 
Война настигла у Белого моря, у Чёрного-забрала, 
Два ордена наколола - и за руку увела, 
И в море Своём купала, и воду лила неживую, 
Неистово смерть плясала, рвалась на передовую. 
Лёшка стрелял и бился, вынул своих из плена, 
Забрал языка, крови пролил по-колено, 
Вывел отряд, но всё же, в бою неравном, 
Пал смертью храбрых. 
Плакали крепкие мужики: Лёшка, покойся с миром. 
Красной землёй укрыли юного командира. 
Жалко парнишку, жалко. 
Дожил до двадцати. 
Было куда расти. 
Что делали в двадцать мы? А ну-ка, повспоминаем. 
Расскажем Лёшке, что было в двадцать с нами? 
Расскажем Лёшке? Да лучше уж промолчим. 
За что ты, Лёшка, смертью такой почил? 
Зачем не пожил? Зачем ты себя не спас? 
Неужто погиб за нас? 
Нам говорят: они воевали зря. 
Но дети смеются, но в окнах огни горят, 
Где смерть танцевала на каждом клочке земли, - 
Выросли травы, ландыши расцвели. 
Как будто сквозь горе, сквозь боль и мрак, 
Нам Лёшка кричал: держи! 
И ни за что-то, за просто так, 
Вложил в наши руки жизнь. 
Как будто злу отводил глаза, к себе приковав беду. 
И этот подарок никак нельзя потратить на ерунду. 
Такая жертва. Нам некогда судьбу проглядеть, как сон. 
Нам нужно разумно потратить дар, который внутри несём. 
Нам нужно вышколить, приручить могучую нашу силу. 
И так пожить, от души пожить. 
Чтоб Лёшка сказал: спасибо.

\begin{verbatim}
  #Петербургскийангел #мойличныйвыбор #мокшейбытьмодно
\end{verbatim}
