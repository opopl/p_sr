% vim: keymap=russian-jcukenwin
%%beginhead 
 
%%file slova.nacia
%%parent slova
 
%%url 
 
%%author 
%%author_id 
%%author_url 
 
%%tags 
%%title 
 
%%endhead 
\chapter{Нация}
\label{sec:slova.nacia}

%%%cit
%%%cit_pic
\ifcmt
  pic https://avatars.mds.yandex.net/get-zen_doc/901899/pub_5ff4cffafe4e686f6a93f46c_5ff4d048bb14d54ffb01aba6/scale_1200
	caption Святой равноапостольный князь Владимир. Икона конца XIX века
\fi
%%%cit_text
А вот на самом деле самое главное лицо украинской \emph{нации} – это киевский
князь Владимир Красно Солнышко, это тот, который 1000 лет назад Русь крестил.
Это, по сути, единственная личность, которую современные украинцы приписывают
себе полностью, и не хотят делиться со своими братьями россиянами и белорусами,
утверждая, что это исключительно их герой.  Даже когда несколько лет назад в
Москве был открыт новый памятник Владимиру, как крестителю Руси, украинские
власти и всякие патриоты возмутились так, будто кто-то чужой покусился на их
самое святое. А на самом ли деле киевский князь Владимир Креститель был
украинским? Или хотя бы киевским?
%%%cit_title
\citTitle{Почему киевский хан Владимир Креститель считается украинским князем, когда он даже славянином не был?},
Исторический Понедельник, zen.yandex.ru, 05.01.2021 
%%%endcit

%%%cit
%%%cit_head
%%%cit_pic
%%%cit_text
Если спросить гражданина РФ, кто он по \emph{национальности}, все от тувы до питера
скажут \enquote{да, мы русские}. Здесь давно вытеснили понятие \emph{национальности} и
\emph{национальной гордости}, все малые народности, из которых состоит \enquote{русский
человек}, граждане Российской Федерации, стесняются того, что они таджики,
тувинцы, нанайцы, адыгейцы, кабардины и прочие малые народности. В СССР было
стыдно быть \enquote{нерусским}, поэтому люди стыдились своей \emph{национальности}. Так
стиралась \emph{национальная идентичность}. Вместо нее пришло \enquote{советский человек},
\enquote{русский}, как прилагательное к чему-то.  Ну, не существует \enquote{русских}. Носители
языка – да, но РФ населена не русскими, а различными малыми народностями,
\emph{национальностями}, объединенных в \enquote{гражданин Российской Федерации}
%%%cit_comment
%%%cit_title
\citTitle{Страна без исторических корней всегда будет оккупантом}, 
Олена Степова, news.obozrevatel.com, 17.06.2021
%%%endcit

%%%cit
%%%cit_head
%%%cit_pic
%%%cit_text
Это означает, что единственный путь спасения для российского государства перед
лицом явных угроз – создание собственной \emph{национальной} модели диктатуры. Однако
важнее того – выработка собственного этического кодекса, который мог бы стать
основой для такой \emph{национальной модели}
%%%cit_comment
%%%cit_title
\citTitle{Революция Духа – единственный путь спасения России}, 
Юрий Барбашов, voskhodinfo.su, 30.06.2021
%%%endcit

%%%cit
%%%cit_head
%%%cit_pic
%%%cit_text
Четвёртая нерешаемая проблема России – это отсутствие \emph{нации}.
\emph{Политической нации}, конечно, а не того изначального этноса, который её
формирует, с его узнаваемым фольклорным, литературным или бытовым миром. Многие
западные – и не только западные – \emph{нации} родились и выросли благодаря
революционному национализму и либеральному капитализму лет сто, а то и двести
назад. Там всё было просто и шаблонно: старая империя, её аристократия, а
заодно Ватикан, господство каких-нибудь высокомерных чужаков, но – в противовес
им романтические интеллигенты и бородатые промышленники сочиняли, пользуясь
человеческим материалом окрестных крестьян, их сказок и рабочих рук, новую
общность, которая чуть позже, на баррикадах или в результате проигранных
прежними королями войн, оказывалась наследником распавшихся или преобразившихся
государств. У нас – иначе. У нас \emph{национализм} и капитализм мелькнули где-то в
эпизоде, быстро ушли в кулисы, а главная роль создателя будущего и знаменосца
прогресса досталась большевикам, которые, в свою очередь, были озабочены не
Россией и уж точно не русским народом, а всем человечеством. И когда их корабль
утонул, страна осталась в печальной неопределённости, словно выживший в океане:
мы – кто? Мы – страна какого народа? Мы – чьи наследники? Царей,
революционеров, интеллигентов, крестьян? Мы господа или слуги? Мы – победители
внутри своей истории, или же мы вечные страдальцы? И кто наш враг – Запад?
Соседи? Собственное начальство? 12 июня – день нашей независимости от кого? А 7
ноября – это повод для радости или траура? \emph{Национальная} история – за вычетом
разве что единодушно принимаемой памяти о войне 1941 года – есть одно сплошное
противоречие и конфликт. И миром пока не пахнет
%%%cit_comment
%%%cit_title
\citTitle{Семь нерешаемых проблем России}, 
Дмитрий Ольшанский, vz.ru, 15.07.2021
%%%endcit

%%%cit
%%%cit_head
%%%cit_pic
%%%cit_text
Колективну українську душу не тривожать протиріччя та обмеження, вона їх просто
не відчуває. Колективний інтелект не шукає причин матеріальних проблем, він
просто не може цього робити. Рушійною силою у нас виступає українська плоть,
яка реагуючи на обмеження на тілесний дискомфорт, як то загроза війни чи
зменшення поживного субстрату, починає випльовувати з себе потоки пропозицій,
називаючи їх ідеями, які нічого не балансують і не долають, а додатково до
існуючих створюють ще більші нові проблеми, за якими губляться попередні.
Причім зрозуміти звідки виходять ці пропозиції і хто відповідає за їх втілення
в Україні неможливо.  Глобально можна говорити про українське \emph{національне} ніщо,
яке не має свого місця, не має цілі, не має метафізичної сили, рухається у
різних напрямках одночасно, вислизає і веде цілий народ у небуття
%%%cit_comment
%%%cit_title
\citTitle{Сформировалось украинское национальное ничто / Лента соцсетей / Страна}, Анатолий Якименко, strana.news, 08.09.2021
%%%endcit


%%%cit
%%%cit_head
%%%cit_pic
%%%cit_text
«Гоголь про Росію писав. Це не про нас!» Да ладно! Именно про вас, как это ни печально прозвучит для
\emph{национального} самосознания. Потому что в России, худо-бедно... да сажают. С
завидной регулярность: министров с их «товарищами», губернаторов с «дружками»,
городничих, полицмейстеров, столоначальников всех калибров и «попечителей
богоугодных заведений». Околоточные уже взятки не вымогают, упрашивать
приходится на коленях, позорище!  Мало, конечно, сажают. В лениво расставленные
сети попадают совсем страх потерявшие, да не шибко сообразительные. Но ничего,
процесс идёт. А решить вопрос по бизнесу, раньше стоивший взятками не меньше
мульёна, теперь не выходя из дома по интернету можно. За скромную госпошлину
%%%cit_comment
%%%cit_title
\citTitle{Такие смешные: украинская политика... через призму творчества Гоголя}, 
Исторические напёрстки, zen.yandex.ru, 28.10.2021
%%%endcit

%%%cit
%%%cit_head
%%%cit_pic
%%%cit_text
В начале XIV века Филипп IV, король,  прославившийся  своей  редкостной
красотой, был неограниченным повелителем Франции, Он  смирил  воинственный пыл
властительных баронов, покорил восставших фламандцев, победил Англию в
Аквитании, повел  успешную  борьбу  даже  с  папством,  закончившуюся  так
называемым Авиньонским пленением пап. Парламенты были в его  распоряжении, а
соборы - на его содержании.  У Филиппа было три совершеннолетних сына, так что
он  мог  рассчитывать на продолжение рода. Свою дочь он выдал за короля Англии
Эдуарда II. Среди своих вассалов он числил шесть иностранных королей, а  союзы,
заключенные им, связывали его со многими государствами, вплоть до России.  Он
прибирал к рукам любые капиталы и состояния. Постепенно  он  обложил налогом
церковную казну и земли, обобрал евреев, нанес удар по объединению ломбардских
банкиров. Чтобы  удовлетворять  нужды  казны,  он  прибегал  к выпуску
фальшивых денег. День ото дня золотые монеты становились все легче весом  и
стоили  все  дороже.  Ужасающе  тяжелым  было   бремя налогов; королевские
соглядатаи буквально наводнили страну.  Экономические  кризисы вели к разорению
и голоду,  что,  в  свою  очередь,  вело  к  возмущениям, которые король топил
в крови. Бунты кончались длинной  вереницей  виселиц.  Все и вся должны были
покоряться, гнуть спину или  разбивать  себе  лоб  о твердыню королевской
власти.  Этот невозмутимый и жестокий владыка  вынашивал  мысль  о
\emph{национальном} величии Франции. При  его  правлении  Франция  была
великой державой,  а французы - несчастнейшими из людей
%%%cit_comment
%%%cit_title
\citTitle{Железный король}, Морис Дрюон
%%%endcit

%%%cit
%%%cit_head
%%%cit_pic
%%%cit_text
Ще одна унікальна технологія, яку застосовує «День» у своїх виданнях, — відхід
від типового для України песимістичного, депресивного історичного патріотичного
дискурсу й пропонування оптимістичного погляду на наше майбутнє. Навіть розбір
помилок і проблем, які були й, на жаль, будуть у нашому житті, тут завжди
супроводжується якісною аналітикою і перспективами, яким чином ми можемо
уникнути в майбутньому таких помилок, як ми повинні поводитися, щоб із жертви
перетворитися на сильну європейську державу. Звичайно, природно, що після
століть колоніальної історії, після радянської окупації, Голодомору-геноциду,
«Розстріляного Відродження», Чорнобильської катастрофи і ще сотень трагічних
подій новітньої історії України комплекс жертви міг сформуватися. І ми
проживаємо й проговорюємо дуже багато травматичних досвідів.  Однак разом із
тим «Детокс» пропонує багато прикладів успішних українських історій, багато
прикладів сильних особистостей, які послужили становленню не завжди Української
держави, іноді інших імперій, але це свідчить про те, що українська
\emph{нація} має якісь особливі риси, які всупереч багатьом історичним
несправедливостям і трагедіям допомогли нам вистояти. І сьогодні в нас є
українська культура, українські традиції, Українська держава. Наприклад, УПА,
яка в часи Другої світової війни боролася проти двох зол — Гітлера й Сталіна.
Мало які інші армії без держави спромоглися такої важливої місії. А Київська
Русь, яка є прикладом дуже демократичної й сучасної як на ті реалії держави?!
Навіть сама назва Київська Русь, Україна-Русь — усе тут тлумачиться і
наповнюється додатковими сенсами
%%%cit_comment
%%%cit_title
\citTitle{«Книга сильних сенсів»}, 
Марія Чадюк, day.kyiv.ua, 28.10.2021
%%%endcit
