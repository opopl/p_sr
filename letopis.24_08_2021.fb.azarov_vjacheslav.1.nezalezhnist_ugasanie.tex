% vim: keymap=russian-jcukenwin
%%beginhead 
 
%%file 24_08_2021.fb.azarov_vjacheslav.1.nezalezhnist_ugasanie
%%parent 24_08_2021
 
%%url https://www.facebook.com/vyacheslav.azarov/posts/4012107725562652
 
%%author_id azarov_vjacheslav
%%date 
 
%%tags nezalezhnist,ukraina
%%title Неуклонное угасание проекта Украина
 
%%endhead 
 
\subsection{Неуклонное угасание проекта Украина}
\label{sec:24_08_2021.fb.azarov_vjacheslav.1.nezalezhnist_ugasanie}
 
\Purl{https://www.facebook.com/vyacheslav.azarov/posts/4012107725562652}
\ifcmt
 author_begin
   author_id azarov_vjacheslav
 author_end
\fi

Неуклонное угасание проекта Украина зависело в целом от одного главного фактора
– самостоятельное государство рассматривалось политической элитой, как источник
обогащения, власть виделась, как самый выгодный бизнес. Это не зависело от
флагов или названий партий, а на протяжении всех 30 лет было общей тенденцией,
для чего политики рвались к власти. Соответственно, когда советское наследие
иссякло и больше нечего было разворовывать из созданного предшественниками,
власть стала торговать суверенитетом, продавать страну, как площадку для
реализации интересов других государств. Также она стала бешенными темпами брать
кредиты, обворовывая таким образом будущие поколения украинцев, которым эти
деньги придется отдавать. И в рамках подобной элитарной модели ситуация в
Украине будет только ухудшаться, потому что, кроме личного обогащения, другой
главной цели у наших политиков нет. В том числе и поэтому я анархист.
