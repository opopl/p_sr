% vim: keymap=russian-jcukenwin
%%beginhead 
 
%%file 10_11_2017.stz.news.ua.mrpl_city.1.arhitekturna_vizitivka_mariupolja
%%parent 10_11_2017
 
%%url https://mrpl.city/blogs/view/arhitekturna-vizitivka-mariupolya
 
%%author_id demidko_olga.mariupol,news.ua.mrpl_city
%%date 
 
%%tags 
%%title Архітектурна візитівка Маріуполя
 
%%endhead 
 
\subsection{Архітектурна візитівка Маріуполя}
\label{sec:10_11_2017.stz.news.ua.mrpl_city.1.arhitekturna_vizitivka_mariupolja}
 
\Purl{https://mrpl.city/blogs/view/arhitekturna-vizitivka-mariupolya}
\ifcmt
 author_begin
   author_id demidko_olga.mariupol,news.ua.mrpl_city
 author_end
\fi

Коли постає питання, що є архітектурним символом Маріуполя, зазвичай виникає
образ водонапірної вежі, яка вже давно стала справжньою архітектурною візитною
карткою міста.

\ii{10_11_2017.stz.news.ua.mrpl_city.1.arhitekturna_vizitivka_mariupolja.pic.1}

107 років вона прикрашає Маріуполь, вигідно виділяючись у навколишньому
урбаністичному пейзажі, має насичену історію.

Спроектував майбутній маріупольський символ німець, лютеранин Віктор
Олександрович Нільсен у 1908 році. Він приїхав на постійне проживання до міста
у 1900 р. і став головним міським архітектором Маріуполя. Нині на його честь
названа вулиця, на якій розташована вежа.

Роботи по будівництву вежі розпочалися в грудні 1909 року. Сьогодні – це
пам'ятник архітектури, а тоді – \enquote{об'єкт життєвої необхідності}, адже вона була
одним з ключових елементів першого міського водопроводу, який запрацював у
липні 1910 року. Як уже зазначалося, автором проекту вежі був архітектор В. О.
Нільсен. Десятьма роками раніше він побудував дуже схожу вежу на берегу Волги в
місті Рибінськ (1899 р.), тому у нього вже був потрібний досвід. Система
працювала до 1932 року. Вода з бака  на четвертому поверсі вежі подавалася до
спеціальних колонок, куди маріупольці приходили, аби купити води. Довжина
водогону складала 21 кілометр. Водночас вода була проведена до будинків
забезпечених маріупольців. За відомостями міської управи, в 1911 р. водопровід
обслуговував дві третини населення міста, близько 30 000 жителів. Решта
користувалася колодязями та іншими джерелами.

Водонапірна вежа створена в формах еклектики, тобто поєднує елементи різних
стилів, що знайшли своє втілення в романських і готичних мотивах віконних
прорізів і в системі прикрас фасаду вежі. Вона має чотири яруси й вісім граней
у периметрі. Із зовнішнього декору водонапірної вежі Маріуполя варто
підкреслити напівкругле оформлення прорізів здвоєних вікон четвертого ярусу, що
нависає над стінами елементи переходу карниза над верхнім ярусом в площину
скатів даху і рельєф парапету оглядового майданчика на її вершині. І хоча у
\emph{\textbf{маріупольської водонапірної вежі}} є сестра-близнюк – \emph{\textbf{Рибінська водонапірна вежа}},
все ж вона унікальна. Старша Рибінська сестра – більш сувора і зовні скромна, а
молодша маріупольська – яскрава і святкова. 

Водонапірна вежа пропрацювала за прямим призначенням до 1932 року. Пізніше
служила пожежною дзвіницею, адже висота споруди становить 33 метри, розташована
вона на височині, тож це був ідеальний варіант для оглядового посту.

Існують легенди, що під час Другої світової війни вежа не зазнала серйозних
пошкоджень завдяки оригінальній методиці, використаній під час будівництва.
Мешканці міста вважали, що її врятували курячі яйця, які міцніше поєднували
цемент, ніж цегла. Правда це, чи тільки легенда, для нас важливо, що вона все ж
таки вціліла.

\ii{10_11_2017.stz.news.ua.mrpl_city.1.arhitekturna_vizitivka_mariupolja.pic.2}

Довгий час водонапірна вежа знаходилася в запустінні. У 1983 році, попри на її
непривабливий вигляд їй присвоїли статусу пам'ятки архітектури місцевого
значення під охоронним номером \enquote{31-ДЦ}. Наприкінці 80-х років минулого століття
виникла ідея реставрації будівлі та розміщення в ньому музею містобудування
Маріуполя. Хоча, на жаль, ця ідея так і не була втілена в життя.

У 1987–1988 роках фахівцями Київського інституту \enquote{Укрпроектреставрація} було
проведено обстеження будівельних конструкцій башти та розроблений проект її
реставрації. У ході реконструкції її вигляд не зазнав значних змін. Єдине, що
було зроблено – розширена і засклена оглядова вежа, а поверхню будівлі покрили
захисним шаром.

У 1996 р. будівлі було розташовано відділення банку. А після спалення
приміщення Маріупольської міськради до вежі переїхали кілька структурних
підрозділів міськради. Пізніше покинула вежу і міськрада, після чого вона
відкрила свої двері для культурно-освітнього проекту VEZHA.

\ii{10_11_2017.stz.news.ua.mrpl_city.1.arhitekturna_vizitivka_mariupolja.pic.3}

Сьогодні водонапірна вежа – архітектурна візитівка, символ Маріуполя, що
прикрашає листівки, запрошення, календарі. Вона об'єднує всіх маріупольців,
незважаючи на світогляд, віросповідання, політичні симпатії, ідеологічні
переконання і національні відмінності.

%\emph{Джерело: \url{http://mrpl.city}}
%\ii{insert.author.demidko_olga}
