% vim: keymap=russian-jcukenwin
%%beginhead 
 
%%file 31_12_2021.fb.fb_group.story_kiev_ua.2.vulkan
%%parent 31_12_2021
 
%%url https://www.facebook.com/groups/story.kiev.ua/posts/1829948543868605
 
%%author_id fb_group.story_kiev_ua,krysjkov_sergij.kiev
%%date 
 
%%tags kiev,shkola,vulkan
%%title Вулкан набирал силу...
 
%%endhead 
 
\subsection{Вулкан набирал силу...}
\label{sec:31_12_2021.fb.fb_group.story_kiev_ua.2.vulkan}
 
\Purl{https://www.facebook.com/groups/story.kiev.ua/posts/1829948543868605}
\ifcmt
 author_begin
   author_id fb_group.story_kiev_ua,krysjkov_sergij.kiev
 author_end
\fi

Эту историю, много лет назад, мне поведал коренной киевлянин, мой давнишний
приятель, Евгений Х. На самом деле, не было никакой выставки, не было жюри и
\enquote{вышестоящих товарищей}. Всё происходило в классе одной из киевских
школ, где присутствовали лишь Женины одноклассники и учительница по труду. Я,
конечно же, приукрасил и добавил эмоциональности. Итак, слушайте.

\ii{31_12_2021.fb.fb_group.story_kiev_ua.2.vulkan.pic.1}

Ландшафтный макет, представленный на выставке работ старшеклассников, был
выполнен красиво, умело и с любовью. Было понятно, что в эту красоту вложено
немало труда, времени и творческого огонька. Макет изображал мирную, спокойную
картину деревенского быта. Хуторок из нескольких миниатюрных домиков,
слепленных из разноцветного пластилина, приютился у подножия горы,
пластилиновые склоны которой правдоподобно изображали серый базальт с красными
гранитными прожилками.

Кроны малюсеньких деревьев были старательно вырезаны из цветной бумаги. Чёрный
пластилин изображал вспаханные участки, зелёный - травяные лужайки. Технологию
создания фактур, имитирующих \enquote{пашню} и \enquote{траву} знал только
автор макета - Женя, присутствовавший на выставке и дающий ответы на вопросы
учеников, родителей, учителей и работников РайОНО.

Женин экспонат был одним из самых популярных на выставке, рядом с ним всегда
находилось несколько посетителей. Подошли, наконец, и члены жюри, сразу
залюбовавшиеся макетом. Председатель, солидный дядька из районного отдела
народного образования, дал свою оценку:

- Красиво, очень красиво. Несомненно, заслуживает поощрения.

Директор школы тут же подхватил:

- Молодец, Женя. Созидать, конечно же, лучше, чем хулиганить.

Посыпались похвалы от учителей. Снизошёл до одобрения и Ваня Петров -
председатель Совета Дружины, очень правильный пионер. Все были уверены, что
Женина работа получит, если не первое, то хотя бы, одно из призовых мест.

Между тем, директор продолжил расспросы:

- Фигурки людей и животных изображают бегущих. Куда, и почему они бегут?

Глухим голосом, Женя медленно проговорил:

- Они спасаются от вулкана.

Директор потрепал Женю по плечу:

- Ну, ты и фантазёр. Извержения нет, а они уже бегут.

В голосе Жени появились едва заметные нотки скрытой угрозы и мрачного
торжества:

- Оно будет.

Оттенок беспокойства и озадаченности, быстро прошёл, едва появившись, поскольку
все знали о Жениной эксцентричности. Продолжились вопросы:

- А что это за красный колпачок?

Действительно, в самом углу макета был пристроен колпачок из красного пластика,
никак не вписывающийся в пейзаж.

- Колпачок нужен для того, чтобы никто случайно, раньше времени, не нажал на
кнопку. Под ним находится волшебная кнопка, которая приведёт макет в действие.

- Так это ещё и действующий макет?! И можно посмотреть, как он работает?

- Конечно - Женя снял колпачок - пожалуйста.

Председатель жюри, стоявший ближе всех, протянул руку и нажал на кнопку... Пару
секунд ничего не происходило и, вдруг - началось! Вершина горы разверзлась в
жерло, из которого стали вздыматься тёмно-красные искры и посыпался
вулканический пепел. В одно мгновение, идиллия превратилась в бедствие и мрак.
Бегущие фигурки теперь, очень даже гармонично, вписывались в пейзаж.

Все замерли, кто-то сдавленно вскрикнул, одна из женщин издала стон. Как будто
под гипнозом, никто не мог проронить ни слова. В павильоне воцарилась тишина,
нарушаемая лишь лёгким шуршанием реакции термического разложения бихромата
аммония.

Вулкан набирал силу. Потоки пластилиновой лавы текли по склонам горы, стали
валить деревья, поглощать домики и фигурки спасающихся бегством. Тёмно-зелёный
оксид хрома, изображающий вулканический пепел, тонким слоем накрыл окружающий
ландшафт. Извержение было в разгаре и слой пепла становился всё толще, потоки
лавы всё шире...

Через несколько минут всё кончилось. Печальная и мрачная картина
мини-апокалипсиса навсегда осталась в памяти присутствовавших. В несколько
минут, Женя, с садистским удовольствием, уничтожил то, что сам же создавал
долгие дни и недели. Красочный пейзаж превратился в выжженную, засыпанную
серо-зелёным пеплом пустыню, над которой возвышались оплавленные, дымящиеся
остатки пластилиновой горы. Все были тогда настолько шокироавны, что даже не
отругали Женю. Но тройку по поведению, по итогам четверти, он таки получил,
вместо обычной четвёрки.

Ну вот, начал во здравие, окончил - заупокой. Аминь.

P.S. По словам Евгения, когда вулкан затих - у учительницы труда случилась
истерика. Вот так.

(с)Айнцвайдрайченко, 2019

Примечание: фотографию \enquote{вулкана} я взял из журнала \enquote{Химия и химики}

\ii{31_12_2021.fb.fb_group.story_kiev_ua.2.vulkan.cmt}
