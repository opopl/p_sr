% vim: keymap=russian-jcukenwin
%%beginhead 
 
%%file slova.poroshenko
%%parent slova
 
%%url 
 
%%author 
%%author_id 
%%author_url 
 
%%tags 
%%title 
 
%%endhead 
Сила емоцій і капітуляція розуму, Засліплені \emph{Порошенком} в пориві емоцій готові
пробачити все навіть Медведчуку, zaxid.net, 28.05.2021

%%%cit
%%%cit_head
%%%cit_pic
\ifcmt
  pic https://avatars.mds.yandex.net/get-zen_doc/1328418/pub_617a5f15aa83b66f09805d56_617a5ffe7dd5651901555df9/scale_1200
  @width 0.4
\fi
%%%cit_text
Многие заметили в своё время: бывший президент \emph{Порошенко} крайне
смахивает на Чичикова. Даже в иллюстрациях дореволюционных и советских изданий.
Либо в исполнении Александра Калягина «Мёртвых душ». Тут вообще полное
попадание.  Мимика, жесты, манера держаться и особенно — говорить.  Образ
хитрющего рвача и корыстолюбца. Немного образован и красноречив. Но всегда
наружу прорывается какой-то запредельный цинизм. Неумолимая воля к
поступательному движению, шаг за шагом приближающая индивида к цели. Когда
можно «взять много». Даже сюжетно многое сходится: Чичиков поначалу
категорически «не брал», рисуя образ честного чиновника. Терпеливо ждал своего
часа. Дождался. Крышу снесло напрочь...  Но \emph{Порошенко} скучен и тускл,
вороватый купчина с замашками турецкого бея, угодливо расстилающегося перед
любыми султанами. Не забывая пустить кровушку беззащитным подданным. Куда более
интересен в сатирико-художественном сравнении нынешний президент Украины.
Никого не напоминает? Думаем!
%%%cit_comment
%%%cit_title
\citTitle{Такие смешные: украинская политика... через призму творчества Гоголя}, 
Исторические напёрстки, zen.yandex.ru, 28.10.2021
%%%endcit
