% vim: keymap=russian-jcukenwin
%%beginhead 
 
%%file 07_04_2021.fb.morits_junna.1.bethoven_bomzh_dovlatov
%%parent 07_04_2021
 
%%url 
 
%%author 
%%author_id 
%%author_url 
 
%%tags 
%%title 
 
%%endhead 

\subsection{Сергей Довлатов - Ахматова - Бомж - Бетховен}
\label{sec:07_04_2021.fb.morits_junna.1.bethoven_bomzh_dovlatov}

Шли мы с приятелем из бани. Останавливает нас милиционер. Мы насторожились, спрашиваем:

— В чем дело?
А он говорит:
— Вы не помните, когда были изданы «Четки» Ахматовой?
— В тысяча девятьсот четырнадцатом году. Издательство «Гиперборей», Санкт-Петербург.
— Спасибо. Можете идти.
— Куда? — спрашиваем.
— Куда хотите, — отвечает. — Вы свободны…
Смесь обыденности и безумия. Оказалось, вовсе не художественный вымысел. Вот заехал я сегодня на заправку, пристроился в очередь. Жду, слушаю по радио музыку. А в двух шагах на перевернутом деревянном ящике сидит бомж. Санитарно запущенный. В каком-то блохастом балахоне. Сидит, жмурится на солнышке, ковыряет в носу, прислушивается к музыке из открытого окна моей машины. И вдруг так, между прочим, спрашивает:
— Девятая симфония Бетховена? Дирижирует кто? Не Аббадо?

\ifcmt
  pic https://scontent-bos3-1.xx.fbcdn.net/v/t1.6435-0/p526x296/169233140_2000806150057178_5718654610813934478_n.jpg?_nc_cat=102&ccb=1-3&_nc_sid=730e14&_nc_ohc=WgKmMKq5nfMAX_Rt-Oc&_nc_ht=scontent-bos3-1.xx&tp=6&oh=be337c7396049389e26287077b3a32c4&oe=6094BB2A
\fi

