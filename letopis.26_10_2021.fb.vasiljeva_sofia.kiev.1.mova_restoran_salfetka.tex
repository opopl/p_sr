% vim: keymap=russian-jcukenwin
%%beginhead 
 
%%file 26_10_2021.fb.vasiljeva_sofia.kiev.1.mova_restoran_salfetka
%%parent 26_10_2021
 
%%url https://www.facebook.com/permalink.php?story_fbid=825019731372994&id=100015949240603
 
%%author_id vasiljeva_sofia.kiev
%%date 
 
%%tags chelovechnost,gorod,jazyk,kiev,maloros,mova,restoran,salfetka,ukraina,ukrainizacia
%%title Спілкуватися українською мовою = знищити малоросів? Такая себе мотивация, знаете ли
 
%%endhead 
 
\subsection{Спілкуватися українською мовою = знищити малоросів? Такая себе мотивация, знаете ли}
\label{sec:26_10_2021.fb.vasiljeva_sofia.kiev.1.mova_restoran_salfetka}
 
\Purl{https://www.facebook.com/permalink.php?story_fbid=825019731372994&id=100015949240603}
\ifcmt
 author_begin
   author_id vasiljeva_sofia.kiev
 author_end
\fi

Итак, спасибо фейсбуку, что открывает новый взгляд на людей. Недавно поняла,
что быть не подписанным на своих друзей, знакомых и родных - значит быть
лучшего мнения о них. Увы. 

\ifcmt
  ig https://scontent-mxp1-1.xx.fbcdn.net/v/t1.6435-9/122396347_5503645812994960_8412838023053838581_n.jpg?_nc_cat=109&ccb=1-5&_nc_sid=730e14&_nc_ohc=j0tXD4364q0AX_BHK84&_nc_ht=scontent-mxp1-1.xx&oh=8b999888efe87f454c0ae83c55ac1711&oe=61AE672E
  @width 0.4
  %@wrap \parpic[r]
  @wrap \InsertBoxR{0}
\fi

Скажу сразу, в политику не лезу, сплетни и интриги не интересны. Но и Вы все не
политические эксперты, как бы это грустно для Вас не звучало. 

Утром попался мне данный пост и затянул. Вплоть до прочтения всех 7 тысяч
комментариев. Значит задело. 

\href{https://www.facebook.com/permalink.php?story_fbid=5503649732994568&id=365125280180398}{%
У Києві поширюється флешмоб, який почався з однієї серветки, Дмитрий Чекалкин, facebook, 24.10.2020%
}

Почему? Работая в ресторанной сфере, таких случаев, как на салфетке, было мало.
Но попадались и было обидно. 

Безусловно, каждый из нас очень важный и нужный пиксель в обществе. Но знания
истории Украины и украинского языка недостаточно. Нужно что-то больше. 

Больше критического мышления, больше доброты и толерантности, больше
взаимопонимания, больше принятия других картин мира, больше умения промолчать,
когда не спрашивают, больше чтения книг, а не статей на Ukr.net и просмотра
политических каналов. 

Всегда нужно больше (...). 

Интересно. А что ещё «дорослі, свідомі, з чіткою громадянською позицією люди»
могут ещё делать кроме как репостов в фейсбуке политических мемов,
выплескивания ядом «з приводу української мови», писанин на салфетках и
самоутверждений за чужой счёт? 

Ограниченность и только. Вы же не вкладываете никакого позитивного намерения в
свои действия. Спілкуватися українською мовою = знищити малоросів? Такая себе
мотивация, знаете ли. 

Предлагаю Вам ещё один тотальный флешмоб: сидит компания, общается на русском
языке, а Вы мимо проходите. Сразу же внедряетесь и кричите «ЧОМУ НЕ СПІЛКУЄТЕСЬ
УКРАЇНСЬКОЮ МОВОЮ? МАЛОРОСИИ!» и уходите. 

Или только на салфетках писать будем?)

Только не бейте за пост на русском. Украинский язык знаю. 

Я думаю українською.
