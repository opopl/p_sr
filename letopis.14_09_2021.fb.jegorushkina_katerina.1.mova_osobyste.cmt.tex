% vim: keymap=russian-jcukenwin
%%beginhead 
 
%%file 14_09_2021.fb.jegorushkina_katerina.1.mova_osobyste.cmt
%%parent 14_09_2021.fb.jegorushkina_katerina.1.mova_osobyste
 
%%url 
 
%%author_id 
%%date 
 
%%tags 
%%title 
 
%%endhead 
\subsubsection{Коментарі}

\begin{itemize} % {
\iusr{Ірина Соловей}
Така сильна історія. Вражена і глибоко зворушена.

\begin{itemize} % {
\iusr{Катерина Єгорушкіна}
\textbf{Iryna Solovey} Окрема сила —дати собі право розповідати ці історії. Я не завжди її мала.

\iusr{Ірина Соловей}
\textbf{Kateryna Yegorushkina} це правда, святкую разом з тобою її ріст 
@igg{fbicon.fist.raised}  @igg{fbicon.sunflower}  @igg{fbicon.woman.dancing} 
\end{itemize} % }

\iusr{людмила вихристюк}

\iusr{Serhiy Bochechka}
Дуже романтично ) Прям згадав героїв 'Мини Мазайла' Куліша @igg{fbicon.heart.suit} Тре буде свій мовний пост написати )

\iusr{Natalie Valevska}

у мене, україномовної з народження, з прапрадідів, україномовної завжди - і
доки жила на Галичині, і всі двадцять років у Києві, - від цього тексту мурахи
табунами, біжать і біжать, ніяк не спиняться

\begin{itemize} % {
\iusr{Катерина Єгорушкіна}
\textbf{Natalie Valevska} у мене теж. Я вже не могла тримати це в собі.

\iusr{Катерина Єгорушкіна}
\textbf{Natalie Valevska} дякую @igg{fbicon.heart.red}
\end{itemize} % }

\iusr{Оксана Собко}

Так, у кожного є свої спогади і важливі Люди в житті які були віхами на
стежечках, шляхах, щось таке що вивело до тих вершин... А далі вже ти сам. І ти
сам може теж для когось став віхою, важливою людиною в спогадах..

Моє улюблене слово - одне з - бабусине : "пильнуйся"

"Пильнуйся, квіточко"

(Тобто бережи себе, від "пильнувати"- піклуватися, доглядати, турбуватися про
когось)

\begin{itemize} % {
\iusr{Катерина Єгорушкіна}
\textbf{Оксана Собко} як красиво! Дякую @igg{fbicon.heart.red}
\end{itemize} % }

\iusr{Оксана Собко}
Дякую за чисте джерело чистого почуття

\iusr{Oksana Myroshnychenko}
як же я тебе розумію...
і мене з першого курсу питають про Львів і дивуються що я з Півдня.)
Суголосне останнім часом особливо полюбила, воно за звучанням таке глибоке і за рахунок цього влучно відповідає суті)

\begin{itemize} % {
\iusr{Катерина Єгорушкіна}
\textbf{Oksana Myroshnychenko} о, то ти з Півдня? (здивований смайлик)))

\iusr{Oksana Myroshnychenko}
\textbf{Kateryna Yegorushkina} 
))))) - "Ні, я не зі Львова, я з Миколаєва" - "Але ж льівського Миколаєва?"
Курсі на третьому чи четвертому на літніх фестах я так
втомилась від цих пояснень що коли питали "звідки ти" одразу
казала що з Могилянки)))

\iusr{Катерина Єгорушкіна}
\textbf{Oksana Myroshnychenko} насправді я пам‘ятаю, що ти з Півдня. Усі ми в якомусь розумінні родом з Могилянки @igg{fbicon.face.smiling.eyes.smiling} 
\end{itemize} % }

\iusr{Svitlana Ryzhykova}
Дуже вам дякую, що розповіли. Для мене - це надзвичайно важливо @igg{fbicon.heart.red}

\iusr{Татуся Бо}

Дуже люблю бабусине слово "макошенний" це майже синонім до "навіжений" чи "скажений".

\begin{itemize} % {
\iusr{Катерина Єгорушкіна}
\textbf{Татуся Бо} о, як цікаво! А звідки бабуся?

\iusr{Татуся Бо}

\textbf{Kateryna Yegorushkina} Полтавщина. У мене вся родина звідти. Був дід, якого я не знала, він з Кубані, але його мати переїхала туди з Полтавщини.
\end{itemize} % }

\iusr{Andriy Kachor}
Пам’ятаю, як з мене в школі сміялись, коли вирішив перейти на українську.
«А потім ми перемогли»  @igg{fbicon.smile} 

\begin{itemize} % {
\iusr{Катерина Єгорушкіна}
\textbf{Andriy Kachor} невимовно тішуся, коли перемагає справжність @igg{fbicon.heart.blue} 
\end{itemize} % }

\iusr{Halyna Halyna}

\obeycr
Дякую вам за щирість, за частинку вашоі життєвої сповіді
Скористаюсь нагодою і скажу ВЕЛИЧЕЗНЕ дякую за вашу творчість (перевірено на
дітях  @igg{fbicon.face.smiling} ️ )
Душа моя світиться, коли іноземці розмовляють українською, навіть з акцентом @igg{fbicon.face.smiling.eyes.smiling} 
А на зараз одне з моїх улюблених слів - любисток
\restorecr

\begin{itemize} % {
\iusr{Катерина Єгорушкіна}
\textbf{Halyna Halyna} яке прекрасне! Дякую Вам @igg{fbicon.heart.red}
\end{itemize} % }

\iusr{Наталія Меда}
Катя, шикарно. Мені відгукнулося до глибини душі. Ось такі роздуми треба масово
транслювати нашій молоді

\begin{itemize} % {
\iusr{Катерина Єгорушкіна}
\textbf{Наталія Меда} дякую, Наталю)
\end{itemize} % }

\iusr{Kasya Hvozdova}
 @igg{fbicon.hands.applause.yellow}{repeat=3}  пестощі, широчинь, виднокіл, кохання

\begin{itemize} % {
\iusr{Катерина Єгорушкіна}
\textbf{Kasya Hvozdova} клас! А ще крайнебо пригадалося:-)

\iusr{Kasya Hvozdova}
\textbf{Kateryna Yegorushkina} точно
\end{itemize} % }

\iusr{Olena Gorokhovska}
Тепер, коли в мене є свої діти, я все частіше згадую Галину Василівна, і ... бажаю своїм дітям зустріти таких чудових педагогів на своєму шляху!
Привіт @igg{fbicon.wink} Ти така молодець, Катрусе!

\begin{itemize} % {
\iusr{Катерина Єгорушкіна}
\textbf{Olena Gorokhovska} дякую, Оленочко! Справді, з роками все більше цінуєш таких учителів @igg{fbicon.heart.blue} 
\end{itemize} % }

\iusr{Vasyl Kolinko}

Це так. Розмовляючи російською я розмовляю мізками. Розмовляючи українською я
розмовляю серцем. Давно це помітив. Українська мова мого тіла та єства, а
російська тренованого розуму. Важко інколи через цю двомовність. Ще й через те,
що в минулому житті був російськомовним поетом, душа якого тим не менш обрала
містом свого наступного втілення Україну.


\iusr{Mariana Burlak}

Затишок @igg{fbicon.heart.yellow}  @igg{fbicon.heart.blue} 
Я колись довго не могла зрозуміти сенс слова "обратно"  @igg{fbicon.face.grinning.sweat}  а щеодним із досягнень було те, що коли я прийшла в бібліотеку на Трьохсвятительській просити якусь книгу з філософії (пам'ять підвела) українською, той чоловік так подивився на мене,наче я прошу переписати його майно. Я тоді йому сказала, що не розумію російської. На наступний раз він сказав,що вони з дружиноюперейшли на українську мову, і сквзав, що я пробудила у нього любов до рідної мови. @igg{fbicon.heart.yellow}  @igg{fbicon.heart.blue} 
\begin{itemize} % {
\iusr{Катерина Єгорушкіна}
\textbf{Mariana Burlak} ти молодець, що лишилася собою. Часом спостерігаю, як люди із Західної України, переїжджаючи на Київщину, переходять на російську і їхні діти стають російськомовними. Це дуже прикро спостерігати @igg{fbicon.face.disappointed} 

\iusr{Mariana Burlak}
\textbf{Kateryna Yegorushkina}

\ifcmt
  ig https://scontent-frx5-2.xx.fbcdn.net/v/t39.1997-6/p480x480/91521538_1030933857302751_5093925307199520768_n.png?_nc_cat=1&ccb=1-5&_nc_sid=0572db&_nc_ohc=wxCpbfgwU4YAX-omxAp&_nc_ht=scontent-frx5-2.xx&oh=00e57bac50425277b4d1b20f10ae7112&oe=614CFE37
  @width 0.2
\fi

\iusr{Лера Худякова}
\textbf{Катерина Єгорушкіна} про Західну Україну. Сестра нещодавно була у Львові і каже, що він став більш російськомовним. Це мене так здивувало... @igg{fbicon.face.flushed} 

\end{itemize} % }

\iusr{Наталія Гридіна}
Катю, це вражаюче та мотивуюче! Аж собі захотілося винайти декілька новеньких
слів.

\begin{itemize} % {
\iusr{Катерина Єгорушкіна}
\textbf{Natalia Grydina} втішена, що надихнула @igg{fbicon.face.smiling.halo} 
\end{itemize} % }

\iusr{Ирина Курас}
Дякую! Дуже близьке..
\ifcmt
  ig https://scontent-frx5-2.xx.fbcdn.net/v/t39.1997-6/p480x480/91521538_1030933857302751_5093925307199520768_n.png?_nc_cat=1&ccb=1-5&_nc_sid=0572db&_nc_ohc=wxCpbfgwU4YAX-omxAp&_nc_ht=scontent-frx5-2.xx&oh=00e57bac50425277b4d1b20f10ae7112&oe=614CFE37
  @width 0.2
\fi

\iusr{Катерина Павелко}
Люблю за лаконічність: просто неба, горілиць, натщесерце )))

\begin{itemize} % {
\iusr{Катерина Єгорушкіна}
\textbf{Katerina Pavelko} прекрасні слова @igg{fbicon.heart.red}

\end{itemize} % }

\end{itemize} % }
