% vim: keymap=russian-jcukenwin
%%beginhead 
 
%%file 06_01_2022.tg.lesev_igor.1.bejonse
%%parent 06_01_2022
 
%%url https://t.me/Lesev_Igor/265
 
%%author_id lesev_igor
%%date 
 
%%tags 
%%title О Бейонсе в/на Украине
 
%%endhead 
\subsection{О Бейонсе в/на Украине}
\label{sec:06_01_2022.tg.lesev_igor.1.bejonse}

\Purl{https://t.me/Lesev_Igor/265}
\ifcmt
 author_begin
   author_id lesev_igor
 author_end
\fi

О Бейонсе в/на Украине.

Иногда я задрачиваюсь всяким рейтинговым говном. Иногда это бывает запредельно
серьезно. Например, лет 10 назад я поставил себе цель посмотреть все фильмы,
которые были номинированы на «Оскар». И не просто посмотреть. Написать на
каждый из них рецензию, ну и выложить на своем сайте. В итоге, это переросло в
сумасшествие, которое длится до сих пор. Дошел на сегодня до 1992 года (иду
хронологически с самого начала еще немых 1927-го).

В прошлом году поставил себе более щадящее начинание – пересмотреть/переслушать
самый последний рейтинг от журнала «Роллинг Стоун» топ-500 «лучших песен всех
времен». Рейтинг от 2020 года. Сразу скажу, подборка – говно невероятное. Уж
как я далек от англоязычных песен, - а я от них дальше, чем сегодня от Алма-Аты
– но даже такой профан понимает, что рейтинг говяный по отношению к самим
носителям английского языка.

Вот просто приведу аналогию. Представьте, что какая-то респектабельная солянка,
типа Минкульта РФ, «Русского радио», Шнура и примкнувшего к ним Макаревича
составит список 500 лучших песен на русском языке за все времена. И вот в те
500 не войдет ни одной песни Иосифа Кобзона и Марка Бернеса. Представили? И как
вы к такой подборке отнесетесь? Так вот, в этом «роллин-стоуновском» рейтинге
нет ни одной песни Фрэнка Синатры и Бинга Кросби. А это легенды Америки.

И я уже понял, как составлен рейтинг. На момент написания этих строчек я дошел
до 16 места. Практически, прошел уже всё. Остались типа сливки. И там типаж
такой. Расовый. Три негро-исполнителя, один белый. Три негра и снова белый. И
так почти все 500 песен. Вот без гугла знаете кто такие Dr. Dre feat. Snoop
Doggy Dogg с их песней 'Nuthin' but a 'G' Thang' не самого древнего 1992 года?
А это 29 место в рейтинге. А Notorious B.I.G., 'Juicy' с 32 местом?

Нет, я знаю, что рейтинги составляются для того, чтобы о них потом говорили,
спорили, иногда высмеивали. Но все-таки в кинематографе/музыке есть вполне
сложившиеся критерии, которые халтурно не перебить. Но в негро-зависимой
Америке и это засрать сумели.

Но я сейчас о другом. Дошел я до 16 места, а там Бейонсе. Ну и говорю жене, что
она на открытие «Донбасс-Арены» в Донецк приезжала. А жена у меня девчонка до
безобразия молодая, и еще более вызывающе красивая. Когда Бейонсе приезжала в
Донецк, ей было столько, что тогда пошел бы по статье «педофилия». Короче,
молодая красотка усомнилась, я полез в ютуб и откопал видос открытия донецкой
арены от 2009 года.

Видео час пятьдесят, Бейонсе выступает в самом конце – хедлайнер, хуле. И я вот
пролистывая, вспоминаю всё то. Вроде бы еще страна остопиздевшего всем Ющенко,
который уже не хромая утка, а хромота без утки. Но, сука, какая же это была
родная и потерянная Украина!

Ведущих два. Один на украинском, другой на русском. На стадионе аниматоры (или
как они там?) выводят слово «Донецк» без мягкого знака, и никакое говно не
видит в этом «акта сепаратизма». Показывают кадры истории «Шахтера» - и это не
только недавний кубок УЕФА – главное футбольное достижение независимой Украины
– но и целый пласт советской истории. Где-то пляшут казачки, где-то шахтеры, на
арене Бейонсе и это показатель, что страна не живет прошлым, а прошлое
воспринимает как должное (ВСЁ прошлое, без «расово правильной» фильтрации) и
движется вперед.

Знаете, я никогда не придавал значения тому, что живу в Советском Союзе, хотя
бы потому, что был тогда ребенком. Но когда я увлекся историей, я стал сожалеть
о потере той страны. И в 2009 году я не ценил ту Украину. У меня даже не было
еще своей квартиры и машины, я вообще чувствовал себя нищебродом и часто
занимал до получки, но только злоебучий Майдан дал понимание того, какую же
страну я потерял. Нет, точно не «лучшую на свете». Но ведь сейчас я именно о
ней вспоминаю.

И, наверное, чтобы оценить по заслугам Украину Порошенко-Зеленского, нужно
оказаться в концлагере. Вот как всё это говно любить, когда у тебя забрали
чувство достоинства и цинично объявили это мародерское изъятие «Революцией
Достоинства»?

В общем, пост был о Бейонсе.  

\url{https://www.youtube.com/watch?v=B7TVK9dEHr8}
