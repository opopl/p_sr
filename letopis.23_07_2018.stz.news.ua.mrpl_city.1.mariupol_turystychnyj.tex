% vim: keymap=russian-jcukenwin
%%beginhead 
 
%%file 23_07_2018.stz.news.ua.mrpl_city.1.mariupol_turystychnyj
%%parent 23_07_2018
 
%%url https://mrpl.city/blogs/view/mariupol-turistichnij
 
%%author_id demidko_olga.mariupol,news.ua.mrpl_city
%%date 
 
%%tags 
%%title Маріуполь туристичний
 
%%endhead 
 
\subsection{Маріуполь туристичний}
\label{sec:23_07_2018.stz.news.ua.mrpl_city.1.mariupol_turystychnyj}
 
\Purl{https://mrpl.city/blogs/view/mariupol-turistichnij}
\ifcmt
 author_begin
   author_id demidko_olga.mariupol,news.ua.mrpl_city
 author_end
\fi

Питання про туристичний потенціал Маріуполя не втрачає своєї актуальності, адже
воно піднімалося на круглих столах, у публікаціях (Сергій Буров, Вадим Джувага,
Сергій Дороговцев), на зібраннях місцевої влади та активістів.

\ii{23_07_2018.stz.news.ua.mrpl_city.1.mariupol_turystychnyj.pic.1}

Цікаво, що протягом своєї історії Маріуполь ніколи не втрачав туристичної
привабливості. На карті Маріуполя 1888 року позначені купальні, що
розташовувалися на сучасному місці яхт-клубу металургійного комбінату
\enquote{Азовсталь}. На той час купальні встановлювалися окремо як для чоловіків, так і
для жінок. Сьогодні це дивна і для когось неймовірна інформація, але наприкінці
XIX ст. відпочиваючі Азовського моря купалися без одягу. За відомостями Вадима
Джуваги, до революції (1917–1921 рр.) після торгівлі та промисловості найбільше
прибутків приносив прийом відпочиваючих в місті.

\ii{23_07_2018.stz.news.ua.mrpl_city.1.mariupol_turystychnyj.pic.2}

Дивно, що протягом радянського періоду Маріуполь не мав офіційного статусу
курортного міста, адже у 70 – 80 роки XX століття на міських пляжах ніде було
яблуку впасти. У Маріуполь за три літніх місяці приїжджало близько 200 тис.
відпочиваючих. Проте в роки незалежності України розвиток туризму в Маріуполі
занепав: число туристів скоротилося до 30-40 тис. чоловік. Мабуть, не останню
роль в цьому зіграла незацікавленість держави у розвитку туристичного
потенціалу Маріуполя та незадовільний екологічний стан. Ще більшої шкоди
принесли військові дії та статус прифронтового міста. Однак сьогодні ситуація
значно покращилася. За останніми відомостями міськради, щодоби маріупольські
пляжі відвідує близько 26 тисяч осіб. До того ж у санаторіях не знайти вільних
місць, що свідчить про повернення Маріуполю неофіційного статусу курортного
міста.

\textbf{Читайте також:} \emph{Туристов в Мариуполь будут привлекать брендированными ярмарками}
\footnote{Туристов в Мариуполь будут привлекать брендированными ярмарками, mrpl.city, 31.05.2018, \url{https://mrpl.city/news/view/turistov-v-mariupol-budut-privlekat-brendirovannymi-yarmarkami}}

Що саме в будь-якому місті може приваблювати туристів? Це архітектура, історія,
природа і клімат, географічне положення, національний колорит, видатні містяни,
ремісничі традиції, підприємства, мова, кухня, легенди, творчі колективи,
менталітет, місцеві традиції дозвілля. Всі перелічені чинники Маріуполь має,
проте є над чим працювати, при цьому системно і дуже плідно. Поверхневе
ставлення до міста, його головних культурних, архітектурних пам'яток, стану
пляжів призводить до негативних наслідків.

Як це не прикро визнавати, але саме війна на Сході України стала потужним
стимулом до пробудження і своєрідним рушієм до активних дій у різних напрямах
щодо розвитку міста. Мені пощастило особисто взяти участь у багатьох творчих
ініціативах, присвячених розвитку туристичного потенціалу Маріуполя. Так, з
травня 2015-го і до червня 2016 року у місті було проведено загальноміський
проект \textbf{\emph{\enquote{Маріуполь – це Україна!}}}, головною метою якого стало знайомство з
різноманіттям і унікальністю міста, пробудження у маріупольців інтересу до
архітектурної та культурної спадщини Маріуполя. Впродовж реалізації проекту
було проведено конкурс серед учнів та студентів Маріуполя, які надсилали на
сайт проекту вірші, есе, малюнки чи наукові дослідження, присвячені рідному
місту.

Було обрано 90 переможців, для яких організатори проекту провели різні види
екскурсій по Маріуполю, а саме: \textbf{\emph{\enquote{Археологічне сафарі}}},
\textbf{\emph{\enquote{Екскурсія по старим вулицям Маріуполя}}},
\textbf{\emph{\enquote{Порт}}} та \textbf{\emph{\enquote{Маріуполь
індустріальний}}}. У результаті проекту маріупольська молодь познайомилася з
археологічними артефактами міста, його історією, архітектурою, відкрила для
себе нові сторони Маріуполя. Твори переможців увійшли до
\emph{\enquote{Альманаху конкурсних робіт учнівської та студентської молоді
загальноміського конкурсу \enquote{Ма\hyp{}ріуполь – це Україна!}}} (2016 р.). Разом
з тим було виготовлено 14 табличок, присвячених найбільш унікальним
архітектурним спорудам Маріуполя. Цей проект, безперечно, став важливим кроком
у розвитку туристичної складової, адже після нього інтерес до архітектурної
спадщини значно підвищився. Окремі школи почали звертатися з проханням провести
екскурсії, виникли нові маршрути.

\ii{23_07_2018.stz.news.ua.mrpl_city.1.mariupol_turystychnyj.pic.3}

\textbf{Читайте також:} \emph{В Мариуполе пройдет \enquote{Неделя туризма} с экскурсиями по городу}%
\footnote{В Мариуполе пройдет \enquote{Неделя туризма} с экскурсиями по городу (ФОТО), Яна Іванова, mrpl.city, 03.03.2017, \url{https://mrpl.city/news/view/v-mariupole-projdet-nedelya-turizma-s-e-kskursiyami-po-gorodu-foto}}

\ii{23_07_2018.stz.news.ua.mrpl_city.1.mariupol_turystychnyj.pic.4}

Також у 2016 році на волонтерських засадах були розроблені розмальовки
\emph{\textbf{\enquote{Розфарбуй місто}}} (художниця - Анастасія Пономарьова), в яких маріупольська
архітектура представлена в антистресовому варіанті. Біля кожної архітектурної
споруди на сторінках розмальовки можна знайти історичну довідку. Головна мета
видання: зацікавити маріупольців та гостей міста історією, архітектурою
Маріуполя, виховати любов та повагу до архітектурної спадщини міста. У
результаті юні маріупольці, розмальовуючи архітектуру власного міста, дізналися
про неї більше, зрозуміли, що і в Маріуполі є чим пишатися.

\ii{23_07_2018.stz.news.ua.mrpl_city.1.mariupol_turystychnyj.pic.5}

Ще одним заходом, який мав велике значення на шляху становлення Маріуполя як
міста туристичного, стало проведення фестивалю \textbf{\emph{\enquote{Обличчя міст}}} (2017 рік), де
можна було ознайомитися з різними локаціями, відвідати квести, екскурсії,
фотозони та найбільш креативні маріупольські майданчики. Такої вражаючої
кількості екскурсантів мені ще не доводилося бачити. На одну екскурсію
приходило понад 80 бажаючих, серед яких були різні вікові групи: від 7 років до
85 - як маріупольців, так і приїжджих гостей. Ще більше вражало те, що на один
вид екскурсії могли прийти ті ж самі екскурсанти, які хотіли ще раз прослухати
та пройтися по місцях, щоб краще запам'ятати. Цей проект став вирішальним у
моєму розумінні, що Маріуполь – це дійсно туристичне місто, потенціал якого
треба лише підтримувати та розвивати. Разом з тим останнім часом Маріуполь дуже
стрімко змінюється, тому є всі підстави сподіватися, що кількість туристів у
місті буде з кожним роком лише зростати.
