% vim: keymap=russian-jcukenwin
%%beginhead 
 
%%file 11_01_2022.tg.edinaja_rusj.1.rus_narod_sborka
%%parent 11_01_2022
 
%%url https://t.me/edinarus_chat/950
 
%%author_id tg.edinaja_rusj
%%date 
 
%%tags rusmir,rossia,future,narod.russkij,kazahstan
%%title Надо найти точку сборки - представление о мире и будущем, которое объединит нас
 
%%endhead 
\subsection{Надо найти точку сборки - представление о мире и будущем, которое объединит нас}
\label{sec:11_01_2022.tg.edinaja_rusj.1.rus_narod_sborka}

\Purl{https://t.me/edinarus_chat/950}
\ifcmt
 author_begin
   author_id tg.edinaja_rusj
 author_end
\fi

Назначение Аскара Умарова (\url{https://t.me/holmogortalks/15230}) резко
отрезвило оптимистов, освежив в коллективной памяти аксиому о том, что РФ не
является русским государством (\url{https://t.me/russ_orientalist/9626}).
Русский мавр сделал дело – обслужил олигархический интернационал
(\url{https://t.me/lorcencov/3911}) – русский мавр может уходить. 

Глядя на всё это с чувством глубокого горького удовлетворения, поднимаю за всех
нас бокал желчи и в который раз повторяю то ли тост, то ли призыв, то ли мантру
– русский народ должен кристаллизовать из себя национальную политическую
организацию и превратиться из рассеянной пыли в единый многомиллионный кулак.
Ну хотя бы волну. Пока мы социальный газ из людей-атомов, ничем не связанных
друг с другом, мы лишены возможности влиять на происходящее, мы можем только
как мошки колотиться об стекло нефтегазового телевизора, который показывает нам
постепенное наступление несветлого будущего. Противостоять RF corporation может
только объединенный народ.

Политика – надстройка не только над экономикой, но еще и над культурой и
сознанием. Наш социальный мир есть отражение наших представлений о том, каким
он должен быть. Для создания русской организации не надо пытаться сначала найти
деньги – сейчас все деньги в системе, которой невыгодно существование русских,
они для неё – балластная нагрузка на экономику трубы. Денег не будет. Надо
найти точку сборки - представление о мире и будущем, которое объединит нас.

Сейчас, когда уже почти всё ясно, этой точкой может оказаться и просто поднятый
флаг. А может, человек. А может – мечта о Единой Руси от моря до моря, о земле
русских людей без границ, таможен и воров во власти. У нас русских так много
умных людей, неужели они не смогут общими усилиями сформулировать такое
представление. Но надо его искать. Пока мы вместо этого в стиле акына
комментируем происходящее, оно так и будет происходить.

Истории на нас плевать – она хоронила скифов и римлян, похоронит и русских.
Вопрос, будем ли мы помогать ей хоронить нас, нашу идентичность, чтобы жить
дальше со спокойной совестью мертвецов. Или совесть заставит все-таки бороться
за своё национальное имя – а значит и за политическое выражение этого имени.
Потому что пока мы сторонимся политики, непрошенные дизайнеры наших судеб кроят
нам будущее, в котором нас нет.
