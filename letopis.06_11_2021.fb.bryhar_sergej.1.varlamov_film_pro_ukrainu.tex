% vim: keymap=russian-jcukenwin
%%beginhead 
 
%%file 06_11_2021.fb.bryhar_sergej.1.varlamov_film_pro_ukrainu
%%parent 06_11_2021
 
%%url https://www.facebook.com/serhiibryhar/posts/1827865924080018
 
%%author_id bryhar_sergej
%%date 
 
%%tags film,rossia,ukraina,varlamov_ilja.rossia.blogger
%%title Варламов - фільм про Україну
 
%%endhead 
 
\subsection{Варламов - фільм про Україну}
\label{sec:06_11_2021.fb.bryhar_sergej.1.varlamov_film_pro_ukrainu}
 
\Purl{https://www.facebook.com/serhiibryhar/posts/1827865924080018}
\ifcmt
 author_begin
   author_id bryhar_sergej
 author_end
\fi

Розмови довкола російського блогера Варламова з його пропагандонським фільмом
про Україну і, зрозуміло, переважно нищівна критика тієї маячні, це з одного
боку добре, адже свідчить про здорові процеси в нашому суспільстві, а з іншого
- погано, тому що означає, що ми, як спільнота, надто повільно ростемо, а отже
серед нас досі чимало тих, хто шукає "хороших русскіх".

І тут у мене є дві важливі думки. 

Перша (тут усе цілком очевидно): "не треба їх, тих "хороших русскіх", шукати -
їх просто немає, - і російський ліберал дійсно  закінчується на українському
питанні (хоча, насправді, - на багатьох національних питаннях, бо ці люди, як і
решта співвітчизників, заражені великодержавним шовінізмом)".

Друга - геть від Москви. Це ніби й теж банальність, але так само і об'єктивна
необхідність, ключова умова виживання української держави.

І тут, звісно ж, можна довго та прискіпливо розмірковувати над концептуальними
особливостями цього "відходу", однак я - про основне. Наполягаю на тому, що він
починається з повсюдного захисту та підтримки української мови та, відповідно,
відходу від російської. Хто б там що не розповідав, але зв'язок між російською
мовою, російською культурою і відповідними, шкідливими для українства
наративами - прямий!

Вважаю всі спроби виправдання споживання російськомовного культурного продукту
ознаками ліні та небажання виходити із зони комфорту. Хоч і можу зрозуміти,
чому все так, а не інакше. І погляд мій на цю проблематику загалом досить
песимістичний...

Мені, наприклад, теж багато в чому складно. Англійську на слух я розумію так
собі (пишу ще непогано, а от на слух... гірше)... І думаю, не я один вважаю, що
розумного кіно (як художнього, так і документалістики) в перекладі українською
замало. Доводиться викручуватися різними способами. Іноді рятують субтитри.
Так, це взагалі правильно - це, власне, повсюдна практика цивілізованого світу.
Але і їх мало. Іноді дивлюся субтитровані переклади іншими слов'янськими мовами
(або оригінальні фільми цими ж слов'янськими мовами).

Останнім часом моя діяльність усе тісніше пов'язана з управлінням сайтами - і
ось тут криється найбільша проблема. Підказки найпростіше розшукувати
російською... З українською все дуже сумно.

Виходи тут два - вдосконалювати англійську і дивитися та читати (або відразу ж
перекладати) потрібні матеріали іншими доступними мовами. Власне, я так і
роблю.

Отже, сподіваюся, скоро і цього гріха (так, ще дивлюся... нема, де правду діти)
теж не матиму.

Але ми - це ми... Годі й сподіватися, що люди, яким зараз принаймні за 30, і
які звикли споживати російськомовний контент, масово відійдуть від звичної для
себе практики.

На жаль, цього не станеться (я розмовляю з багатьма адекватними українцями, які
все-все розуміють, але звички перемагають... так, вони зменшують кількість того
російського контенту, так, вони аж ніяк не хизуються тим, що його споживають,
але все одно це роблять).

Однак у нас є головне завдання. Ключове. Я б навіть сказав - два завдання.
Паралельні. Формулюються просто: "не заражати російським контентом наших дітей,
і розвивати (звісно ж, популяризувати) український культурно-інформаційний
простір".

Чим більше українських дітей ростимуть без російського і російськомовного
культурно-інформаційного супроводу, тим краще виглядає наше майбутнє.

І ніяких вам варламових та іншої отруйної московської нечисті...

Все і складно і просто водночас.

\ii{06_11_2021.fb.bryhar_sergej.1.varlamov_film_pro_ukrainu.cmt}
