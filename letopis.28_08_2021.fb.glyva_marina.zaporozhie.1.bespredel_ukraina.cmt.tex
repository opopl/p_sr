% vim: keymap=russian-jcukenwin
%%beginhead 
 
%%file 28_08_2021.fb.glyva_marina.zaporozhie.1.bespredel_ukraina.cmt
%%parent 28_08_2021.fb.glyva_marina.zaporozhie.1.bespredel_ukraina
 
%%url 
 
%%author_id 
%%date 
 
%%tags 
%%title 
 
%%endhead 
\subsubsection{Коментарі}

\begin{itemize} % {
\iusr{Natalia Shabanina}
Сумма большая?

\begin{itemize} % {
\iusr{Marina Glyva}
\textbf{Natalia Shabanina} Не очень. Дело же не в сумме, а в том, что меня насильно и незаконно принуждают ее вернуть, угрожая судом. И эти деньги мне по закону заплатили. Отдать мне их не за что, я лучше ребенка накормлю. Меня этот Департамент все лето терроризирует.

\iusr{Natalia Shabanina}

С этим бороться бесполезно. Можно продлить статус одинока мати, и чтобы они
частями удерживали эту сумму из следующих выплат. Ещё вариант: оформи
единоразовую материальную помощь. У них же в УСЗН. Скажем, на подготовку
ребенка к школе.

\iusr{Marina Glyva}
\textbf{Natalia Shabanina} 

Статус есть, социальную выплату матерям-одиночкам в прошлом году отменили, с 22
июля 2020 года, поэтому и требуют выплату обратно, хотя она была до принятия
этого закона и не имеют права забирать. Все новые законы вступают в силу с
момента их принятия согласно 94 статье Конституции Украины, ни один закон не
может вступать в силу прошедшим временем, когда его еще не существовало. А
социальное пособие я уже год не получаю.

\iusr{Natalia Shabanina}

Я лично не оформляю больше ничего. Уже лет пять. Когда мне говорят, вы же
многодетные, у вас льготы, это просто смешно. Они только на бумажке. На
практике они не работают.

\iusr{Marina Glyva}
\textbf{Natalia Shabanina} Матерям-одиночкам Не положено НИЧЕГО у нас в государстве, ни льготного питания в садиках и школах, никакой материальной помощи, мне в собесе сказали, что я даже как малообесреченная оформить не могу. Когда я работала, социальное пособие матери-одиночки мне тоже не платили.

\iusr{Marina Glyva}
\textbf{Natalia Shabanina} Многодетным только льготное питание в садике и в начальной школе, это реально добиться. Но ты ведь действительно многодетная, 6 детей, должны быть льготы, просто социальная служба не хочет их давать, они вообще обнаглели!!!!

\iusr{Natalia Shabanina}
Если не малообеспеченный, то, да. Ничего.

\iusr{Marina Glyva}
\textbf{Natalia Shabanina} Так социальная служба отказывается оформлять статус малообеспеченный, я еще когда в декрете была и у меня был доход 420 грн: 300 как мать-одиночка и 120 декретных, мне отказали, сказали, что мне денег хватает и я не малообеспеченная. Так я тогда официально работала. Они ищут поводы, чтобы не платить, потом скажут, что у меня глаза не такого цвета и цвет кожи не соответствует, поэтому ничего не положено.

\iusr{Natalia Shabanina}

Нет, Мариш, я не оформляю ничего, потому что не хочу тратить на это свое время,
силы, нервы. Они мне нужны для другого. К тому же у меня дети уже взрослые.
Через год я даже многодетной мамой считаться не буду!  @igg{fbicon.face.grinning.smiling.eyes} 

Там такой прикол: если у тебя доход выше хоть на копейку, то будь ты трижды
многодетная и одинокая, не оформить ни субсидию, ни 50 процентную скидку на
коммуналку. Льгот на самом деле остаётся - бесплатный проезд (вы пробовали в
некоторых маршрутках бесплатно проехать??) И у нас была бесплатная музыкальная
школа. За это спасибо. Все выучились. Ну, как бы все...

\iusr{Marina Glyva}
\textbf{Natalia Shabanina} 

Если у женщины 5 детей, то ей уже пенсия автоматически есть, так по крайней
мере раньше было до прихода Зеленского к власти. Я Это пособие всего-то чуть
больше двух лет получала и то потому, что на тот момент не было работы и
доходов. Когда началась пандемия, осталась вообще без денег, государство
отобрало соцпособие. А в России, которую наши власти ненавидят, на каждого
ребенка во время пандемии давали по 10000р.помощи. А у нас для чего государство
нужно, чтобы над людьми издеваться? Люди со стажем 40 лет получают пенсию 1800
грн (воспитатели, медсестры, библиотекари и др.), ниже минимальной зарплаты и
прожиточного минимума.

\iusr{Natalia Shabanina}

У мамы моей тоже минималка  @igg{fbicon.face.sad.but.relieved}  До сих пор работает. Хорошо помню один год, когда
дома жрать, извините, было нечего, я ходила за бумажкой в УСЗН узнать сколько
пособия назначили. 4 гривны 22 копейки. Шла и плакала.

Ничего, прожили сами...

\iusr{Natalia Shabanina}
На всю жизнь запомнила. 4гривны 22 копейки!!!!! Поразила меня эта сумма...

\iusr{Marina Glyva}
\textbf{Natalia Shabanina} ужас!!!!

\iusr{Marina Glyva}
\textbf{Natalia Shabanina} 

Прожить можно, если есть работа и зарплата позволяет кушать и коммуналку
оплатить. А сейчас в Украине только официально 1,8 млн.безработных (данные со
страницы Азарова), 10 миллионов уехали из Украины на заработки в Россию, Польшу
и другие страны. Хотя когда у меня была работа, жить тоже было не за что,
зарплата маленькая была и уходила за коммуналку, садик и скромно на питание,
даже одежду ребенку покупать было не за что. Про себя я вообще молчу, себе
ничего не покупаю. Были бы нормальные пенсии и зарплаты в Украине, не нужны
тогда были бы ни субсидии, ни соц.пособия.


\iusr{Natalia Shabanina}

Ещё есть моральный аспект. Понимаешь, я не буду унижать детей бесплатным
проездом в маршрутке, когда на тебя орет водитель или тихо ненавидит, потому
что много льготников развелось. Я плачу за всех....

Я не буду унижать их, когда надо платить фонды в школе и садике. Я их плачу...

Я не позволю никому заставить их усомниться в том, что у них есть достоинство.

Мои дети свободные люди и у них есть достоинство и свобода. Без всяких льгот.

Но это не благодаря государству. Я лучше на 25-ю работу устроюсь.

\iusr{Marina Glyva}
\textbf{Natalia Shabanina} 

Если у тебя много денег, плати, конечно, же, я работала в школе, многодетные
фонды не платят и питание у них бесплатное. А вообще-то моя публикация была не
об этом, а о том, как нарушаются законы в стране. Многодетным сейчас по 2
тысячи на каждого ребенка дают к школе, не пиши только, что ты от них
откажешься, п что это унижает достоинство детей, не верю. От денег никто не
откажется.


\iusr{Marina Glyva}
\textbf{Natalia Shabanina} А чем бесплатный проезд унижает детей? По закону до 6 лет дети бесплатно ездят, и никто не считает себя униженным.

\iusr{Natalia Shabanina}

Во-первых, у меня нет школьников. Во-вторых, это дают малообеспеченным
многодетным. В-третьих, унижает не бесплатный проезд, а отношение окружающих к
этому. Ты разве не видишь, как разговаривают в маршрутках даже с пенсионерами и
не берут их в транспорт? Невозможно на практике посадить в одну маршрутку 6
детей, если водитель вопит - у меня одно льготное место. Мне их что , по
очереди катать? Я должна это выслушивать? Поэтому я оплачиваю проезд и все. А
по поводу фондов... Я не готова сесть на голову другим родителям, чтобы они
платили за моих детей, и чтобы потом на них косо смотрели. И денег у меня не
много. Каждый занимает свою жизненную позицию.

\iusr{Natalia Shabanina}

Когда мои учились в школе, никто ничего не давал, никаких денег. Только один
раз был картонный рюкзак от депутата и пачка гнилых цветных карандашей. А в
садике, когда мы жили ещё на Космосе, каждое утро меня у ворот встречала
воспитатель и требовала оплату фондов. 20 лет назад. Мы говорим о разных вещах.

\iusr{Natalia Shabanina}
Удачи тебе в отстаивании своих прав.  @igg{fbicon.thumb.up.yellow} 

\iusr{Marina Glyva}
\textbf{Natalia Shabanina} 

Тебе просто не повезло с воспитателем. Я работала в школе и была классным
руководителем, и у меня были в классе дети из многодетных семей, я не брала с
них фонды, не имеет право педагог вымогать деньги с родителей, вообще все фонды
по закону ДОБРОВОЛЬНЫЕ! У Сережи сейчас в школе хорошая администрация: и
директор в 106 школе хорошая, и завуч, и к учителю хорошему его перевели в
прошлом году, директор мне разрешила фонд школы не сдавать, а фонд класса сдаю,
конечно же.

\iusr{Marina Glyva}
\textbf{Natalia Shabanina} Спасибо, Наташа. Но я сомневаюсь, что в нашей стране можно добиться правды, у нас в стране прав тот, у его есть должность и связи, поэтому они могут спокойно нарушать закон.

\iusr{Marina Glyva}
\textbf{Natalia Shabanina} Я живу возле конечной остановки 87 автобуса, здесь всегда стоит очередь из пенсионеров, спокойно они садятся в маршрутку и никто на них не орет, все зависит от водителя.

\iusr{Marina Glyva}
\textbf{Natalia Shabanina} 

У каждого свой опыт, Наташа. У меня тоже была история в садике, когда мама из
родительского комитета орала на меня, оскорбляла и вымогала деньги
дополнительно кроме фондов, хотя я все платила как все, она просто обнаглела и
за родительский счет покупала каждый год мебель, подарки воспитателям и т.д. Я
просто поставила ее на место и отказалась выполнять ее прихоти, она бесилась,
но ничего не могла сделать. Воспитатели тоже были против подарков, которые она
придумывала и с родителей сдирала на них деньги. Нужно наглость пресекать и
людей ставить на место!

\iusr{Natalia Shabanina}

 @igg{fbicon.face.grinning.smiling.eyes}  примерно так я и поступала. С первым ребенком. А потом - на втором, третьем
и т.д мне просто надоело бороться и тратить свою единственную жизнь на этот
хлам, в государстве, в котором ничего не меняется ни 20 лет назад, ни сейчас.
Лишь в отдельных случаях можно восстановить справедливость и то, если у тебя
есть связи.

\iusr{Marina Glyva}
\textbf{Natalia Shabanina} У меня выхода нет, меня жизнь ставит в такие условия, что остается только бороться за свои права.

\iusr{Natalia Shabanina}
Пусть все сложится, Марин... От всей души... @igg{fbicon.heart.red} @igg{fbicon.hands.pray}  @igg{fbicon.face.blowing.kiss} 

\iusr{Marina Glyva}
\textbf{Natalia Shabanina} Спасибо!

\end{itemize} % }

\iusr{Наталья Кузьменко}
Марина, я вернусь в город, займемся опять твоей проблемой.

\iusr{Marina Glyva}
\textbf{Наталья Кузьменко} Спасибо!!!!

\iusr{Ольга Ольга}
Марина, желаю тебе удачи, чтоб все же в твоей жизни все сложилось.

\begin{itemize} % {
\iusr{Marina Glyva}
\textbf{Ольга Ольга} Спасибо, Оля! @igg{fbicon.heart.red}

\iusr{Алексей Дубовый}
\textbf{Marina Glyva} Вы уже решили Свой вопрос? Хочется Вам помочь.

\iusr{Marina Glyva}
\textbf{Алексей Дубовый} 

Нет, мне из Министерства социальной политики (его отделения - Социальной
сервисной службы) вчера пришло письмо, они поддерживают нарушение закона и
требуют вернуть социальное пособие за июль 2020 года, а мою жалобу полностью
проигнорировали. Я вчера опять звонила на урядову горячу линию 15-45, мне
сказала оператор, что многие люди обращались к ним с подобным вопросом и что по
закону возвращать средства не нужно. Но Департамент социальной защиты все равно
требует. Я не знаю, как с ними бороться.

\end{itemize} % }

\end{itemize} % }
