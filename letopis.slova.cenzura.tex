% vim: keymap=russian-jcukenwin
%%beginhead 
 
%%file slova.cenzura
%%parent slova
 
%%url 
 
%%author 
%%author_id 
%%author_url 
 
%%tags 
%%title 
 
%%endhead 
\chapter{Цензура}
\label{sec:slova.cenzura}

%%%cit
%%%cit_head
%%%cit_pic
%%%cit_text
Из того же смыслового ряда - 15-я статья Конституции. В ней говорится, что
никакая идеология не является обязательной, а \emph{цензура} запрещена.
Однако в Украине события развиваются в последние годы совсем в обратном
направлении.
\emph{Цензуру} осуществляют, во-первых, националисты, которые атакуют неугодных им
артистов, блогеров или простых людей, высказывающих публично или в соцсетях
\enquote{неправильную} позицию. Радикалам в этом помогает власть - хотя бы тем, что
закрывает глаза на их действия и не привлекает к ответственности.
Второй пласт \emph{цензуры} - это списки неугодных лиц, куда входят писатели, актеры,
музыканты. Их ведут Минкульт и СБУ. Таких людей запрещено не только пускать в
страну, но и показывать по телевидению. Это \emph{прямая цензура}
%%%cit_comment
%%%cit_title
\citTitle{День Конституции Украины 28 июня - какие статьи нарушаются сильнее всего}, 
Оксана Малахова; Максим Минин, strana.ua, 28.06.2021
%%%endcit

%%%cit
%%%cit_head
%%%cit_pic
%%%cit_text
Жёсткая антидонбассовская \emph{цензура} и одна известная социальная сеть -
синонимы уже давно. Стоит выложить ссылку на значимое неудобное интервью по
Донбассу, реакция следует незамедлительно. Но на этот раз под нож цензуры
попали не новости, не статья, а стихи поэта Анны Ревякиной. Два года назад за
честный рассказ о женщине-художнике, женщине-снайпере с позывным Юла, ушедшей
воевать в бескрайние донбасские степи, я попала в месячный \emph{бан}. Юла — прототип
главной героиня поэмы «Шахтёрская дочь». В прошлом году за упоминание о другом
снайпере Скрипаче я снова оказалась в месячном \emph{бане} с предупреждением, если
буду продолжать в том же донбасском духе, то мою страницу и вовсе ликвидируют,
зачистят
%%%cit_comment
%%%cit_title
\citTitle{От политической до поэтической цензуры всего один шаг}, Анна Ревякина, ukraina.ru, 08.07.2021
%%%endcit


