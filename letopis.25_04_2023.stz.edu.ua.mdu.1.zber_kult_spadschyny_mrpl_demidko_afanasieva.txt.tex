% vim: keymap=russian-jcukenwin
%%beginhead 
 
%%file 25_04_2023.stz.edu.ua.mdu.1.zber_kult_spadschyny_mrpl_demidko_afanasieva.txt
%%parent 25_04_2023.stz.edu.ua.mdu.1.zber_kult_spadschyny_mrpl_demidko_afanasieva
 
%%url 
 
%%author_id 
%%date 
 
%%tags 
%%title 
 
%%endhead 

Представниці Маріупольського державного університету доцентка кафедри
культурології Ольга Демідко та магістрантка спеціальності «Культурологія»
Наталія Афанасьєва долучилися до конкурсу наукових проєктів у межах
Всеукраїнської науково-практичної конференції «Дослідження молодих вчених: від
ідеї до реалізації». Організатором події виступив Київський університет імені
Бориса Грінченка.

Конкурс об'єднав студентів, аспірантів і викладачів закладів вищої освіти
Києва, Харкова, Ніжина, Луцька, Одеси, Херсона, Маріуполя. Власне бачення
культурологині МДУ представили за напрямом «Наука та культура: збереження і
популяризація історико-культурної спадщини», успішно пройшовши відбірковий етап
конкурсу.

На конференції Ольга Демідко презентувала проєкт пересувної фотовиставки
«Літопис театрального життя Маріуполя». Він дозволить проаналізувати розвиток
театральної культури міста-героя протягом середини XIX – кінця XX століття,
висвітлити діяльність професійних та самодіяльних театральних колективів
приазовського краю, а також приїжджих труп.

Сьогодні для багатьох українців залишається невідомим, що найстаріший театр
на Лівобережній Україні був заснований саме у Маріуполі. Проєкт спрямований
змінити ставлення широкої громади до історії міст Східної України та
наблизити до театрального мистецтва героїчного міста. Крім того, він
спрямований зруйнувати радянську «парадигму закритості» музеїв та архівів і
захистити найбільш вразливу культурну спадщину – паперові об’єкти,
історичні фотографії та документи,

– доцентка Ольга Демідко.

Унікальність та інноваційність проєкту визначає особистий оцифрований архів
дослідниці МДУ, який нараховує понад 1000 світлин. Всі вони були зібрані з 2013
року під час написання дисертації, присвяченій театральному життю Північного
Приазов'я. Колекція містить програмки, афіші та фотодокументи, що зберігалися в
маріупольських театрах, музеях і приватних архівах. 

За результатами конференції Ольга Демідко стала переможницею за обраним
напрямом конкурсу. Це дозволило дослідниці отримати фінансову підтримку на
реалізацію власного проєкту. Викладачка сподівається, що відкриття виставки
відбудеться саме у Маріупольському університеті, після чого буде представлена в
інших освітніх закладах, музеях і театрах Києва. 

У свою чергу, магістрантка Наталія Афанасьєва представила історико-культурний
проєкт віртуального музею пам’яті «Архітектурна спадщина Маріуполя: проблеми
окупації та випробування часом». Його основна мета полягає у створенні
веб-сайту, який міститиме інтерактивну мапу значущих архітектурних споруд
героїчного міста із супровідними матеріалами про них: дореволюційними
листівками, світлинами різних періодів, аудіо- та відео-матеріалами, спогадами.

Основна ідея полягає в тому, щоб дослідити архітектурну спадщину Маріуполя,
зокрема у реаліях сьогодення, зберегти та актуалізувати пам’ять про місто,
про його історико-культурні особливості. У межах проєкту буде створена база
важливих архітектурних споруд і проведена фотофіксація їхнього поточного
стану, збір текстових і візуальних матеріалів, що транслюють значення
кожного архітектурного об’єкту для маріупольців, міста, регіону,

– магістрантка Наталія Афанасьєва.

Студентка Маріупольського університету не отримала призового фонду на
реалізацію авторського проєкту, тому додатково шукатиме джерела фінансування.
Не дивлячись на це дівчина вже втілює у життя власну ідею і планує довести
справу до кінця навіть без стороннього бюджетування. 

Читайте також: Студенти МДУ отримали фінансування на реалізацію власних проєктів
