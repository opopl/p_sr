% vim: keymap=russian-jcukenwin
%%beginhead 
 
%%file 24_05_2018.fb.lesev_igor.1.ukraina_posle_poroshenko.cmt
%%parent 24_05_2018.fb.lesev_igor.1.ukraina_posle_poroshenko
 
%%url 
 
%%author_id 
%%date 
 
%%tags 
%%title 
 
%%endhead 
\subsubsection{Коментарі}

\begin{itemize} % {
\iusr{Serhiy Vorobey}
Ти таки розумний хоч і по инший бік барикад.
Для твоєї втіхи, багато українців Тимоху сприймають уже як лайт-версію Яника. Тож ми спробуємо аби таки Порох відмотав і другий строк. Оскільки трохи він зачистив поле від якісних кандидатів. Та й не було їх практично.

\begin{itemize} % {
\iusr{Игорь Лесев}
ага, ну спробуйте))

\iusr{Игорь Лесев}
все по бизнес-плану, теперь, оказывается, надо Петю сохранить)

\iusr{Станислав Бочкур}
\textbf{Ira Gavrylova} Гетьманат от слова "геть"...

\iusr{Олег Резник}
Та не нахер его с его спробами.

\iusr{Дмитрий Марунич}
какие баррикады? на дворе полнейшая реализация идей майдана

\iusr{Serhiy Vorobey}
\textbf{Дмитрий Марунич}, ви що таке алегорія знаєте?
Хоча схоже ми живемо в одному місті, та в різних країнах.

\end{itemize} % }

\iusr{German Gorozhanski}

Пётр Пятый, Пётр Шестой, чтоб не нарушать отчетности. Или Юля Шестая - из
рукава. Тогда условный Онищенко должен уже засесть за составление плана нового
триллера. "Это будет славная охота", в смысле, чтиво. С элементами эротики.

\iusr{Дмитрий Коломийченко}

1. Инерционный сценарий. Без учета системного кризиса в самой Украине и
вероятной волны мирового экономического кризиса. 

2. В России нет ожидания
прихода "адекватных сменщиков".

\begin{itemize} % {
\iusr{Игорь Лесев}
1) + 2) -

\iusr{Дмитрий Коломийченко}
\textbf{Игорь Лесев} 2) Какие ваши доказательства?

\iusr{Игорь Лесев}
1) Легализация Порошенко путем признания выборов 2014, 2) Минские соглашения, отдающие республики Украине, 3) инвестиционная, экономическая деятельность с ЮА 4) отказ от резких шагов, угроз, демонстрации силы. Все это свидетельствует именно об ожидании 2019 года, как панацеи

\iusr{Дмитрий Коломийченко}
\textbf{Игорь Лесев} 

1) Россия в своей политике исходит из того, что ей крайне невыгоден кризис
украинской государственности. Поэтому Москва не оставляла попытки диалога и с
готовностью пошла на Минск. Собственно Минск и надо рассматривать, как попытку
России, Франции и Германии сохранить Украину в нынешнем виде (2014). 2)
Экономическая деятельность одна из форм взаимодействия. Война её никак не
отменяет. США и Великобритания продолжают экономическую деятельность, несмотря
на все преступления русских. 3) Исходя из п1, любые резкие шаги провоцировали
усиление кризиса украинской государственности, и до тех пор пока Минск мог быть
в принципе исполнен хоть наполовину, России не было никакого смысла вести
активную антиукраинскую деятельность. Исходя из того, что ответная акция должна
решительно прекратить силовое противостояние, её форма лежит в военной
плоскости. До завершения активной части сирийской кампании нежелательно
увеличивать оперативную нагрузку. Управление операциями всегда стремится уйти
от одновременности к последовательности. Полноценное развертывание 8 армии
требовало времени. С колес они в Чечне навоевались. 4) Как только участники
процесса убедятся в принципиальной невозможности Минска и сирийская кампания
окончательно перейдёт в состояние стабилизации акция со стороны России станет
вопросом предлога. 5) Стоит еще учитывать фактор близости волны мирового
финансового кризиса.

\end{itemize} % }

\iusr{Олег Резник}
Это самый реалистичный прогноз. А как оно может быть по другому за такой исторически короткий срок?

\iusr{Василий Стоякин}
Астанавитесь! Мы еще к Украине без Кучмы не привыкли...

\iusr{Марина Прохорова}

Мне кажется, что каждый раз, когда "северный сосед" собирается (ну вдруг!)
как-то вмешаться в происходящее вна, текущая укровлада откалывает такой
пердимонокль, что сосед отступается и только машет руками: Сама, всё сама!

Это конечно грустно, но неисчерпаемый идиотизм - лучшая защита от "вторжений" и
вмешательств

\iusr{Михайло Бойченко}
России не будет из-за тьісяч убитьіх на Донбасе

\iusr{Михайло Бойченко}
А как вариант жесткой культурной революции - неужели совсем не просматиривается @igg{fbicon.wink} 

\begin{itemize} % {
\iusr{Игорь Лесев}
профессор, шо вы с этой культурной революцией зачастили? тут вам не Мао, мы сами умеем культурно отдохнуть

\iusr{Михайло Бойченко}
Не бойса, аньі тебе ньі болна зарежут @igg{fbicon.face.tears.of.joy} 
\end{itemize} % }

\iusr{Дмитрий Марунич}
очень похоже на правду

\iusr{Сергей Копылов}
В реальности будет совсем другой сценарий. Одно останется неизменным - Украины, как суверенного государства нет и не будет.

\iusr{Маргарита Городничая}

Граждане, не забывайте, что у нас есть легитимный президент. Он жив, здоров,
ему не был объявлен импичмент. Зовут В. Ф. Янукович...

\begin{itemize} % {
\iusr{Игорь Лесев}
я вас умоляю, а Владимир Владимирович в мае 2014 кого с победой поздравлял? самозванца?)

\iusr{Маргарита Городничая}
Нельзя же все списывать в Владимира Владимировича...

\iusr{Михайло Бойченко}
Даже по старой логике, его каденция закончилась несколько лет тому. Он же не прижизненньій, а ток лехитимньій @igg{fbicon.wink} 

\iusr{Маргарита Городничая}
Если не восстановить легитимность, порядка не будет, будут постоянные перевороты и революции...
\end{itemize} % }

\iusr{Матвей Кублицкий}

две ошибки во втором абзаце. 1. В России уверены что в 2019 году точно НЕ
придут адекватные. это здесь знают все. 2. Общая украинская ошибка - если
Россия вернет Крым - то отменят санкции. это полная глупость. Санкции вводились
не из-за Крыма. Это просто форма давления на РФ и с Крымом не связано. Крым -
лишь предлог

\begin{itemize} % {
\iusr{Василий Стоякин}
1. Во всяком случае, подают сигналы. 2. Нет такой "украинской ошибки". Тут просто наплевать на то, что будет с Россией после возвращения Крыма. 3. Если вы считаете, что Игорь - украинский пропагандист, то у меня для вас плохие новости...

\iusr{Василий Стоякин}
\textbf{Игорь Лесев} Или за шумеров? Да нет... за монголов.

\iusr{Матвей Кублицкий}

 @igg{fbicon.face.tears.of.joy}{repeat=3} Тогда давайте уж за гуннов или ост-готов. кстати, именно последние
полностью истребили сарматов и скифов, сделав псевдоисторика Витренко
посмешищем

\end{itemize} % }

\iusr{Владислав Клочков}
Ага, добавят недобытое, как футбольный арбитр...

\iusr{Ольга Антоненко}
Грустно это - шутка 2019 года заключается в том, что ничего принципиально не изменится.

\iusr{Михаил Касимов}

Ох, наломали вы дров, громадяне. Нет, Путин вам точно не поможет. Вам даже Ким
Чен Ын не поможет. Его дедушка Ким Чен Ир мог бы помочь, но увы уже умер

\end{itemize} % }
