% vim: keymap=russian-jcukenwin
%%beginhead 
 
%%file 19_11_2021.fb.fb_group.story_kiev_ua.1.brezhnev.cmt
%%parent 19_11_2021.fb.fb_group.story_kiev_ua.1.brezhnev
 
%%url 
 
%%author_id 
%%date 
 
%%tags 
%%title 
 
%%endhead 
\zzSecCmt

\begin{itemize} % {
\iusr{Андрей Павлов}
Настоящий мужик! Любил охоту, машины, женщин, застолья. Другим давал жить.

\begin{itemize} % {
\iusr{Sergej Grishakov}
\textbf{Андрей Павлов} ..члены только и жили...

\iusr{Виктор Задворнов}
\textbf{Sergej Grishakov} а дочка-то лучше всех
\end{itemize} % }

\iusr{Анна Николаевна}
Светлая память Леониду Ильичу Брежнєву  @igg{fbicon.hands.pray} 

\begin{itemize} % {
\iusr{Igor Neronov}
\textbf{Анна Николаевна} Нам солнца не надо, нам Партия светит, нам хлеба не надо- Работу давай!!!)))

\iusr{Igor Neronov}
\textbf{Анна Николаевна} 

Эй, любители, а особенно любительницы эпохи Брежнева, ну- ка, вспомните, чем Вы
пользовались вместо прокладок, как десятками раз штопали колготки и носили их
под брюки, какого качества мышьяк закладывали Вам в пломбы, синие куры в пайке
и обязательно 300 грамм в додачу костей на кило замороженного мяса, родителей,
горбатившихся всю жизнь на квартиру и проживание всей семьей с бабушками и
дедушками на 50 метрах? Вспомнили??? Будь проклято и забыто это время!!!!!!

\begin{itemize} % {
\iusr{Арт Юрковская}
\textbf{Igor Neronov} 

не смешите. Сейчас ползарплаты за еду надо отдать, в больницу лучше не попадать
,а то никто без денег лечить не будет. За тепло не может пол Украины
заплатить. Чтобы квартиру купить - надо взять кредит, если еще дадут, а если не
платишь - и квартиру потеряешь и всем на тебя начхать. Когда слышат, что тебе
56-57 лет - на работу уже не возьмут. Пенсии отменилив будущем вообщ, про
8-часовый день с двумя выходными можно забыть. Вот это время - будь проклято! А
советским есть, что вспомнить и сравнить.


\iusr{Igor Neronov}
Так, работайте!!! Сидя жопой на диване, будете голодать в любое время!!

\iusr{Светлана Манилова}
\textbf{Igor Neronov}, Вы сильно не расходитесь, пожалуйста. Комментарии не должны нарушать правил группы...

\iusr{Natasha Levitskaya}
\textbf{Igor Neronov}

Вам уже делала замечание за грубость, но комментарий ваш удалили. Видимо, вы не
успели прочитать. Ещё раз вам предупреждение - лексику измените в комментариях!


\iusr{Igor Neronov}
\textbf{Natasha Levitskaya} ок, но очень затруднительно ))

\iusr{Natasha Levitskaya}
\textbf{Igor Neronov}

Здесь не ваша личная территория. И не надо устраивать диванные \enquote{бои без
правил}, даже если вам и затруднительно. В группе более 100 тыс. участников.


\iusr{Catherine Murashchyk}
\textbf{Igor Neronov} особенно войну в Афганистане вспомните.

\iusr{Igor Neronov}
\textbf{Vit Naz} кто на кого учился ...)))

\end{itemize} % }

\iusr{Алексей Назаров}

Войну в Афганистане вспомню я. Потому, что на неё загнали моего отца. Путём
тупого шантажа. Когда подошла его очередь на квартиру, ему сказали: или в
Афганистан, или забудь о квартире. Прапощику без жилья, у которого жена и двое
детей. И вернулся он оттуда с хорошим таким посттравматическим синдромом.
Хорошо, хоть вернулся относительно целым. Уже одного этого хватило,чтобы навеки
проклясть эту сучью власть, маразмирующего бровеносца и его наследников. А у
меня за время моей жизни при ней накопилось неплохое количество счетов к ней.


\iusr{Янина Ерошкина}
\textbf{Igor Neronov} А ты работаешь? @igg{fbicon.wink}  @igg{fbicon.face.tears.of.joy} 

\begin{itemize} % {
\iusr{Igor Neronov}
\textbf{Янина Ерошкина} и день и ночь..)

\iusr{Янина Ерошкина}
\textbf{Igor Neronov} На невидимом фронте? @igg{fbicon.face.tears.of.joy}  @igg{fbicon.laugh.rolling.floor} 

\iusr{Igor Neronov}
\textbf{Янина Ерошкина} наша Служба и опасна и трудна, и на первый взгляд как - будто не видна...
\end{itemize} % }

\end{itemize} % }

\iusr{Григорий Владимирович}

2 раза в Борисполе видел на расстоянии около 20 метров, потом с Олейником и
Недригайлом сопровождали в Залесье.

\begin{itemize} % {
\iusr{Катя Шерман}
Вы прикоснулись к вечности  @igg{fbicon.smile}  Григорий Владимирович

\iusr{Григорий Владимирович}
\textbf{Катя Шерман} Хорошо когда есть что вспомнить хорошее.
\end{itemize} % }

\iusr{Max Gopencko}
До указа «Об усилении борьбы с пьянством и алкоголизмом» оставалось 3,5 года....

\iusr{Николай Бурчин}
А сигареты с ментолом в те годы были в продаже? @igg{fbicon.face.nerd} 

\begin{itemize} % {
\iusr{Igor Neronov}
\textbf{Николай Бурчин} Нет, только газеты \enquote{Правда} и
\enquote{Известия} разрезанные на четвертушки, вместо туалетной бумаги, лежали
в сортире...))

\begin{itemize} % {
\iusr{Николай Бурчин}
\textbf{Igor Neronov} жаль, жаль.. у нас была бумага  @igg{fbicon.smile} 

\iusr{Igor Neronov}
\textbf{Николай Бурчин} Бумага -то была, но газетку подкладывали в унитаз, чтобы фекалии не пачкали дорогущий чешский, а кому повезет, финский фаянс..))))

\iusr{Sergej Grishakov}
\textbf{Igor Neronov} \enquote{Правды} нет, остался \enquote{Труд} в неограниченном колличестве.

\iusr{Надежда Владимир Федько}
\textbf{Igor Neronov} 

Ми жили в комунальній квартирі. Коли мені було 5 років, то в мої обов'язки
входило рвати на шматки газети для туалету. При цьому була категорична
настанова - портрети \enquote{вождів} виривати і класти в стопку окремо. Одного разу я
застав сусідку за знищенням портретів - вона рвала їх на дрібні клаптики і
кидала у відро зі сміттям, яке потім виносилося у дворовий сміттєзбірник.

\end{itemize} % }

\iusr{Виктор Анисков}
\textbf{Николай Бурчин}
Были  @igg{fbicon.wink}  Я в 77-78 курил Данхил )

\begin{itemize} % {
\iusr{Николай Бурчин}
\textbf{Виктор Анисков} круто! добывали как-то или продавался в табачных киосках?

\iusr{Виктор Анисков}
\textbf{Николай Бурчин}
Бывший сотрудник предложил купить. Я и иногда ему заказывал )
\end{itemize} % }

\iusr{Татьяна Ховрич}
\textbf{Николай Бурчин}, 

появились в период Олимпиады-80. Потом не исчезали (Sallem, New Port). Правда,
стоили намного дороже, чем наш \enquote{Космос}. Если не ошибаюсь, по 1.5 руб. А может,
ошибаюсь в цене.

\begin{itemize} % {
\iusr{Illya Davydov}
\textbf{Tatyana Hovrych} а вот если простые Столичные прокапать корвалолом,то будут ментоловые

\iusr{Татьяна Ховрич}
\textbf{Illya Davydov}, небольшая поправочка: жидким валидолом из круглой капсулки.

\iusr{Ирина Левина}
\textbf{Татьяна Ховрич} \enquote{MORE} - 110мм (зелененькая пачка) продавались повсеместно, а вот Данхил привозили моряки из загранки))

\ifcmt
  tab_begin cols=4,no_fig,center

     pic https://scontent-frx5-1.xx.fbcdn.net/v/t39.30808-6/258336884_972076243379545_7103405831031703650_n.jpg?_nc_cat=111&ccb=1-5&_nc_sid=dbeb18&_nc_ohc=nO2hp_QJlM8AX8LiGEd&_nc_ht=scontent-frx5-1.xx&oh=00_AT_SSi7MA2NS3XJI2kyt9Z8IeELa36WvsN3PYnhm3C4RiA&oe=61C95645

		 pic https://scontent-frt3-1.xx.fbcdn.net/v/t39.30808-6/257406640_972076786712824_9094888357543117531_n.jpg?_nc_cat=106&ccb=1-5&_nc_sid=dbeb18&_nc_ohc=oqy6h5YwG0oAX90Ats2&_nc_ht=scontent-frt3-1.xx&oh=00_AT_rnLtPQELHwE7ObdJ8AK60o-7q4UvZ6NFodstON-c9ZA&oe=61C95879

		 pic https://scontent-frt3-2.xx.fbcdn.net/v/t39.30808-6/258775559_972077903379379_2210019873837989737_n.jpg?_nc_cat=103&ccb=1-5&_nc_sid=dbeb18&_nc_ohc=zF6D4a7dawkAX-Sqz1Z&_nc_ht=scontent-frt3-2.xx&oh=00_AT8D7F7bj9XCpC69UFpBeaR5YTzjJ55pHYt8mewr4HwQaw&oe=61C91D68

		 pic https://scontent-frx5-2.xx.fbcdn.net/v/t39.30808-6/258751539_972078156712687_5108058481969474308_n.jpg?_nc_cat=109&ccb=1-5&_nc_sid=dbeb18&_nc_ohc=FTuDj6aKhpoAX_jirzs&_nc_ht=scontent-frx5-2.xx&oh=00_AT9NKKfueKoTECbeHusBV2TNxWPgaVOw8fTBPrNUiV_pOw&oe=61C8D56C

  tab_end
\fi

\iusr{Татьяна Ховрич}
\textbf{Ирина Левина}, 

да, именно. Баловали себя в дни \enquote{студенческой ЗП}: покупали в каждую стипендию
разные импортные сигареты. Но зелёные \enquote{Мore} - самые любимые: во-первых,
ментоловые, во-вторых, длинненькие, долго курятся. @igg{fbicon.face.wink.tongue} 


\iusr{Надежда Владимир Федько}
\textbf{Татьяна Ховрич} Дякую пані Тетяно за чудову колекцію і інформацію!

\iusr{Татьяна Ховрич}
\textbf{Надежда Владимир Федько} , колекція - не моя, а Ірини Левіної.

\iusr{Надежда Владимир Федько}
\textbf{Татьяна Ховрич} Моя подяка пані Ірині))

\iusr{Надежда Владимир Федько}
\textbf{Ирина Левина} Дякую за колекцію і підтримку))

\iusr{Татьяна Ховрич}
Ирина, Вы - коллекционеруете?

\iusr{Ирина Левина}
\textbf{Татьяна Ховрич} нет, но люблю путешествовать по интернету во времени и пространстве )))

\end{itemize} % }

\iusr{Alexander Nejuvoj}
\textbf{Николай Бурчин} Были наши помню год 65-й. \enquote{Пчёлка} назывались.

\begin{itemize} % {
\iusr{Аркадий Израилевский}
\textbf{Alexander Nejuvoj} \enquote{Пчёлка} были ароматизированые, а не ментоловые.

\iusr{Alexander Nejuvoj}
\textbf{Аркадий Израилевский} Я был маленький и ещё не курил. Но пахли они вкусно. Ароматизированные или с ментолом,-это уже технология???
\end{itemize} % }

\iusr{Николай Бурчин}

Я совсем не помню ментоловые сигареты в СССР, наверное таки не очень часто
встречались или не пользовались особой популярностью. В начале 90-х помню пошла
мода на них. Помню длинные черные More...  @igg{fbicon.smile} 

\iusr{Нина Светличная}
\textbf{Николай Бурчин} Моя приятелька \enquote{робила} будь-які сигарети ментоловими. На сигарету наносила трішки бальзаму \enquote{Звездочка})

\iusr{Igor Neronov}
Не так.. правды- нет, известия- закончились, остался труд за три копейки..)

\iusr{Виталий Колычев}
Явские \enquote{золотое руно} и 100мм тоже были \enquote{золотое руно}

\iusr{Igor Popell}
\textbf{Mykola Burchin} я демобилизовался в ноябре 1981г. В продаже какое-то время были
финские сигареты. Видимо, остатки поставок на олимпиаду 80г. Помню, что иногда
покупал ментоловый \enquote{Newport} (или \enquote{Salem}?) за 1,50р. или за 1р. Были еще
Marlboro и Bond.

\begin{itemize} % {
\iusr{Irena Visochan}
\textbf{Igor Popell} 

Какая-то финская водка была? с клюквой? или ликёр? И, конечно, а як жеж, без
сигаретки с ментолом? (в сумочке, Salem,, Да, это после олимпиадные остатки
роскоши.)


\iusr{Igor Popell}
\textbf{Irena Visochan} водку не застал. Видимо, всю выпили до моего дембеля ))) А вот Fanta была. Жвачки Kalev появились.

\iusr{Надежда Владимир Федько}
\textbf{Irena Visochan} Фінську горілку з клюквою пам'ятаю...

\iusr{Irena Visochan}
\textbf{Igor Popell} 

Проконсультировалась с главным спецом, с моим Игорем. Так вот-' финской водки
не было, а был финский клюквенный ликёр.


\iusr{Irena Visochan}
\textbf{Надежда Владимир Федько} мой муж сказал, что это был ликер, финский клюквенный,

\iusr{Igor Popell}
\textbf{Irena Visochan} профессионализм не убьешь!

\iusr{Irena Visochan}
\textbf{Igor Popell}  @igg{fbicon.hearts.two} 

\iusr{Надежда Владимир Федько}
\textbf{Irena Visochan} У Вашого чоловіка чудова пам'ять!

\iusr{Irena Visochan}
\textbf{Надежда Владимир Федько} Да. Надюша, спасибо!

\iusr{Надежда Владимир Федько}
\textbf{Irena Visochan} Надійки вже нема! Сторінку веду я, її чоловік, Володимир.
\end{itemize} % }

\iusr{Олег Дурбалов}
Были, ST Moritz

\iusr{Юрий Мисик}
\textbf{Николай Бурчин} да были

\iusr{Catherine Murashchyk}
\textbf{Mykola Burchin} были! Сама курила

\iusr{Ivan Tsurkan}
\textbf{Николай Бурчин} в магазинах «Березка», «Каштан»

\iusr{Лариса Павловская}
\textbf{Mykola Burchin} В киоске на углу Прорезной(Свердлова тогда) и Крещатика продвали импортные сигареты, помню болгарские

\iusr{Татьяна Шаповалова}
\textbf{Николай Бурчин} Нет. Но мы их курили. Всё можно было \enquote{достать}.

\iusr{Иванна Кудрина}
\textbf{Николай Бурчин}
Были... в Москве точно... More, в зелёной упаковке - ментоловые.

\iusr{Gennady Henry Sergienko}
\textbf{Mykola Burchin} Кто помнит \enquote{Запашнi} - нашу альтернативу ментолу?  @igg{fbicon.smile} 

\begin{itemize} % {
\iusr{Генадий Пинский}
\textbf{Gennady Henry Sergienko} Несколько не хуже.

\iusr{Gennady Henry Sergienko}
\textbf{Генадий Пинский} Любимые в школьные годы  @igg{fbicon.smile} 

\iusr{Ирина Левина}
\textbf{Gennady Henry Sergienko} Запашнi (ГОСТ-58, Прилуксткая ТФ) @igg{fbicon.wink} 

\ifcmt
  ig https://scontent-frt3-1.xx.fbcdn.net/v/t39.30808-6/259557211_972132860040550_5959224906510151616_n.jpg?_nc_cat=106&ccb=1-5&_nc_sid=dbeb18&_nc_ohc=VFC1bixzbK8AX9XQ1qQ&_nc_ht=scontent-frt3-1.xx&oh=00_AT92NB4IZozJu8Lvc6ElHZ7iGK2j9RD75SMN3m4XouCJqw&oe=61CA1DAA
  @width 0.4
\fi

\end{itemize} % }

\iusr{Геннадий Дудко}
\textbf{Николай Бурчин} склоне всего это были остатки с Олимпиады 80. Цена была 1,50 р. Да если хотелось такие покурить, то ,, достать,, можно было без проблем.

\end{itemize} % }

\iusr{Sergej Grishakov}
Видел \enquote{настоящего мужика} на расстоянии вытянутой руки в Киеве на проспекте Воссоединения во время приезда Никсона.
Народу \enquote{выгнали} тогда на улицы-немерянно.

\begin{itemize} % {
\iusr{Tanya Podkolzina}
\textbf{Sergej Grishakov} , ага. Тоже там стояли, не по доброй воле)

\iusr{Анна Николаевна}
\textbf{Tanya Podkolzina} 

Я тоже видела Леонида Ильича и Щербицкого на привокзальной площади, когда он
приезжал открьівать знаменитьій монумент Родина- мать. Я бьіла молодой, мне
бьіло прикольно и интересно. А потом мьі пошли с друзьями вьіпить по чашечке
кофе с пироженьім Картошка. Бьіло вкусно и запомнилось.

\end{itemize} % }

\iusr{Валерій Чистяков}
За здоровье Ильича - накатили сгоряча!

\begin{itemize} % {
\iusr{Luda Draganova}
\textbf{Валерій Чистяков} ...Накатили сгоряча - получили строгача. Вот такие именины Леонида Ильича!
\end{itemize} % }

\iusr{Светлана Дубински}
Лучшие времена переживала страна.

\begin{itemize} % {
\iusr{Надежда Владимир Федько}
\textbf{Светлана Дубински} Я так не вважаю!

\begin{itemize} % {
\iusr{Светлана Дубински}
\textbf{Надежда Владимир Федько} Все конечно условно. Не сажали и не убивали. Вот в этом \enquote{большая заслуга} этих времен

\iusr{Надежда Владимир Федько}
\textbf{Светлана Дубински} 

Угу. У 1972 пройшла перша хвиля арештів і посадок української інтелігенції.
Напередодні Олімпійських ігор у Москві (1980) друга \enquote{зачистка}...

В 1974-му Солженіцина викинули з СРСР. Така ж доля і Віктора Некрасова.
Лупиноса, Плюща і Григоренко закрили в психіатричну лікарню... Буковський
відсидів, а потім його обміняли на Луїса Корвалана. І це тільки окремі особи з
величезного списку репресованих за політичні погляди.

\iusr{Zoya Sharykova}
\textbf{Надежда Владимир Федько} 

А більшості людей було так зручно не замислюватись над тим... З‘явилися «фантики»
у вигляді Фанти, сигарет (не палю, тому їх не перелічую), копченої ковбаски,
яку частіше стали «викидати» у зв‘язку з Олімпіадою - от і здалося, що щасливі
роки... Прізвища тих, кого ув’язнювали, забороняли навіть вимовляти. Запитати б
про ті «щасливі» роки у їхніх родин...


\iusr{Надежда Владимир Федько}
\textbf{Zoya Sharykova} 

Запитати треба також у родин тих 15 000 солдат, що полегли в Афганістані. І тих
десятків тисяч, що залишилися інвалідами.


\iusr{Zoya Sharykova}
\textbf{Надежда Владимир Федько} 

Так, це взагалі окрема сторінка, цей мартиролог на їхній совісті... Мені
доводилось зустрічатися з хлопцями-афганцями, писати про них (я журналіст), і я
ті їхні рани, фізичні й душевні, бачила зблизька. Скількох уже немає, а ті, хто
є, так і не позбулися синдрому... За що їхні життя поламані? Я особисто ще можу
шукати і знаходити приводи для оптимізму, а вони так і не навчились... Тих, хто
зумів спекулювати на цій темі, не маю на увазі, війна відкривала й підленькі
душі, але вони в меншості.

\iusr{Надежда Владимир Федько}
\textbf{Zoya Sharykova} 

Згідно довідника \enquote{Военно-политическая спецоперация СССР в Афганистане}
(2008) безповоротні втрати - 14 453, з яких 1739 покінчили життя самогубством!

Загальні санітарні втрати 469 685 осіб. (стор. 215)

\iusr{Надежда Владимир Федько}
\textbf{Zoya Sharykova} Як журналісту можу порекомендувати Вам такий довідник:

С.В. Червонопиский, А.А. Костыря, В.Г. Сироштан. Военно-политическая операция
СССР в афганистане (25 декабря 1979 - 15 февраля 1989 гг.). Словарь-справочник.
2-е изд. переработанное и дополненное. Киев - 2008.

\ifcmt
  ig https://scontent-frx5-1.xx.fbcdn.net/v/t39.30808-6/259683616_4716323241760430_7488078940293679502_n.jpg?_nc_cat=110&ccb=1-5&_nc_sid=dbeb18&_nc_ohc=WB6BReHzdQYAX-h4j7q&_nc_ht=scontent-frx5-1.xx&oh=00_AT88Zqyi1zAxY2gtmYzYfO_GdvBl4o_rg_05sZlvDcIB8Q&oe=61C89D55
  @width 0.4
\fi

\iusr{Zoya Sharykova}
\textbf{Надежда Владимир Федько} 

Дякую! Червонописького знаю дуже мало й офіційно (після двох заходів), а Володю
Сіроштана добре й особисто, правда, давно не спілкувались. Знайду довідник,
хоча вже давно про це не пишу. Скажу Вам відверто про те, про що мало кому
говорила: коли сталось оте путлерівське вторгнення до нас під істеричним
девізом «спасти русский народ», якось воно печально нагадало те радянське
бажання «помочь афганскому народу»... Тільки тепер ми помінялися місцями, ми їх
не запрошували і зараз змушені відбиватися від «порятунку»... Якось нагадує
кармічну відповідь... Але ж ні, ту сусідню країну ця карма не наздогнала. Хіба що
все попереду...

\iusr{Надежда Владимир Федько}
\textbf{Zoya Sharykova} 

Форму покарання обирає Господь! Наша проблема - це \enquote{внутрішня еміграція},
тобто, люди, які живуть в Україні, але вважають, що повинні жити в Росії.

\iusr{Олег Курилов}
\textbf{Надежда Владимир Федько} 

число погибших в Афганистане преувеличено немного. Вы путаете военные потери и
общие. Самоубийц никогда не вносили в военные потери, даже сейчас. Ввод войск в
Афганистан был ошибкой СССР, но обвинять его за это... это геополитика, и не
важно какой строй был тогда в СССР, это зависило от множества разных
факторов.... Скажем так если страна входит в тройку-пятёрку ведущих стран. Она
обязана поступать так или перестанет быть ведущей страной.

\iusr{Надежда Владимир Федько}
\textbf{Олег Курилов} Рекомендую відкрити на с. 215 і прочитати! Всі претензії до авторів довідника!

\ifcmt
  ig https://scontent-frx5-1.xx.fbcdn.net/v/t39.30808-6/259683616_4716323241760430_7488078940293679502_n.jpg?_nc_cat=110&ccb=1-5&_nc_sid=dbeb18&_nc_ohc=WB6BReHzdQYAX-h4j7q&_nc_ht=scontent-frx5-1.xx&oh=00_AT88Zqyi1zAxY2gtmYzYfO_GdvBl4o_rg_05sZlvDcIB8Q&oe=61C89D55
  @width 0.4
\fi

\iusr{Олег Курилов}
\textbf{Надежда Владимир Федько} 

спасибо за рекомендацию, но в сети не смог найти это издание. Поискал в других
источниках. В принципе цифры приводятся похожие на Ваши ( те, что вы привели).
В общие санитарные потери входят заболевшие разными инфекционными
заболеваниями.

\end{itemize} % }

\iusr{Нина Светличная}
\textbf{Светлана Дубински} 

І де ж ці часи \enquote{переживались}? У таборах, де сидів Стус? Чи у
психлікарнях, куди засаджували незгодних?

\iusr{Надежда Калинская}
\textbf{Светлана Дубински} \enquote{Лучшие}, але переживала, виживала!

\end{itemize} % }

\end{itemize} % }
