% vim: keymap=russian-jcukenwin
%%beginhead 
 
%%file 17_12_2021.stz.news.lnr.mirlug.1.shkolniki_risunki_zaschitniki
%%parent 17_12_2021
 
%%url https://mir-lug.info/novosti-proektov/luganskie-shkolniki-peredali-risunki-dlya-zashhitnikov-rodiny-v-ramkah-akczii-s-novym-godom-soldat
 
%%author_id 
%%date 
 
%%tags 
%%title Луганские школьники передали рисунки для защитников Родины в рамках акции «С Новым годом, солдат!»
 
%%endhead 
\subsection{Луганские школьники передали рисунки для защитников Родины в рамках акции «С Новым годом, солдат!»}
\label{sec:17_12_2021.stz.news.lnr.mirlug.1.shkolniki_risunki_zaschitniki}

\Purl{https://mir-lug.info/novosti-proektov/luganskie-shkolniki-peredali-risunki-dlya-zashhitnikov-rodiny-v-ramkah-akczii-s-novym-godom-soldat}

Ученики Луганского общеобразовательного учреждения – средней
общеобразовательной школы № 17 имени Валерия Брумеля 17 декабря передали
активистам проекта «Волонтёр» Общественного движения «Мир Луганщине» рисунки и
открытки для защитников Республики. Также волонтёры провели для детей
развлекательную программу.

\ii{17_12_2021.stz.news.lnr.mirlug.1.shkolniki_risunki_zaschitniki.pic.1}

Помощник координатора проекта «Волонтёр» ОД «Мир Луганщине» Алёна Назина
отметила, что дети подготовили открытки и рисунки в рамках ежегодной акции
проекта «С Новым годом, солдат!».

– Многие военнослужащие встретят новый год на передовой, а не в кругу семьи,
поэтому мы хотим подарить им тепло, заботу и внимание. Перед новогодними
праздниками мы сами поедем на передовую, чтобы передать военным детские
открытки, рисунки, а также новогодние посылки. Мы проводим эту акцию каждый год
и уже знаем, что защитники нашей Республики очень ждут, когда мы привезём им
открытки от детей. Они всегда внимательно читают пожелания от детей, им приятно
такое внимание, – подчеркнула Алёна Назина.

\ii{17_12_2021.stz.news.lnr.mirlug.1.shkolniki_risunki_zaschitniki.pic.2}

Первоклассница Адель Дрозденко рассказала, что нарисовала открытку военному,
чтобы поддержать его, поблагодарить за то, что он стоит на защите Родины.
Девочка пожелала солдату, чтобы война закончилась, а каждый Новый год он
отмечал вместе со своей семьёй.

Своим мнением об акции «С Новым годом, солдат!» поделилась мама Адель Екатерина
Демьяненко:

– Очень важно проводить такие акции, ведь нужно с детства учить ребят делать
добрые дела, благодарить тех людей, которые стоят на защите рубежей нашей
Родины. Я очень рада, что моя дочь сама проявила инициативу нарисовать открытку
для военного. Я помогала Адель в создании открытки, но пожелания она писала
сама, это только её мысли.

Алёна Назина подчеркнула, что учащиеся Луганского общеобразовательного
учреждения – средней общеобразовательной школы № 17 имени Валерия Брумеля
передали около 40 открыток и рисунков.

– Сегодня крайний день приёма рисунков и открыток в рамках акции «С Новым
годом, солдат!». Все рисунки обязательно дойдут до военных, а самые лучшие
работы будут поощрены – дети получат от Общественного движения «Мир Луганщине»
подарки, – напомнила Алёна Назина.
