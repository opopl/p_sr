% vim: keymap=russian-jcukenwin
%%beginhead 
 
%%file 05_02_2018.stz.news.ua.mrpl_city.1.arhitekturna_prykrasa_mariupolja
%%parent 05_02_2018
 
%%url https://mrpl.city/blogs/view/arhitekturna-prikrasa-mariupolya-1
 
%%author_id demidko_olga.mariupol,news.ua.mrpl_city
%%date 
 
%%tags 
%%title Архітектурна прикраса Маріуполя
 
%%endhead 
 
\subsection{Архітектурна прикраса Маріуполя}
\label{sec:05_02_2018.stz.news.ua.mrpl_city.1.arhitekturna_prykrasa_mariupolja}
 
\Purl{https://mrpl.city/blogs/view/arhitekturna-prikrasa-mariupolya-1}
\ifcmt
 author_begin
   author_id demidko_olga.mariupol,news.ua.mrpl_city
 author_end
\fi

\ii{05_02_2018.stz.news.ua.mrpl_city.1.arhitekturna_prykrasa_mariupolja.pic.1}

Однією з найцікавіших пам'яток архітектури в нашому місті є будівля Палацу
культури \enquote{Молодіжний} по вул. Харлампіївській, 17/25. Ця будівля вражає не
тільки яскравою архітектурою, але й неймовірно позитивною енергетикою, адже у
різні часи в її стінах проходили репетиції та виступи як місцевих, так і
українських, зарубіжних талановитих і обдарованих акторів, співаків,
музикантів, творчих колективів.

Спочатку будівля, яка сьогодні по праву вважається архітектурною прикрасою
міста, була побудована як готель \enquote{Континенталь}. Будівництво завершили в 1897
р., на жаль, автор проекту невідомий, документи втрачені. У першокласному
готелі \enquote{Континенталь} проводили свій вільний час маріупольське купецтво і
місцеве міщанство. Господарем особняка був В. Томазо. На першому поверсі
розташовувалися елітні магазини та модні салони, зал купецьких зборів.

\ii{05_02_2018.stz.news.ua.mrpl_city.1.arhitekturna_prykrasa_mariupolja.pic.2}

Архітектура готелю вражає, в ній простежуються риси модерну та еклектики. Не
дивно, що будинок є архітектурною пам'яткою місцевого значення, адже перехожих
він радує своєю красою та вишуканістю. При будівництві використовувався
матеріал місцевого виробництва: червона формувальна цегла хараджаєвського
заводу. В інтер'єрі корпусів використовувалися плити та стійки сходових
прольотів, відлиті на маріупольському заводі В. С. Сойфера. Ажурні чавунні сходи
з написом \enquote{Готель Томазо} на боковій частині кожної сходинки збереглися до
нашого часу. Незважаючи на міцність чавунних сходів, при інтенсивних
навантаженнях пошкодження можуть з'явитися навіть на них. Проте, на жаль,
сьогодні при проведенні ремонту про історичне й архітектурне значення сходів
всі забули, що негативно позначається на їхньому зовнішньому вигляді.

\ii{05_02_2018.stz.news.ua.mrpl_city.1.arhitekturna_prykrasa_mariupolja.pic.3}
\ii{05_02_2018.stz.news.ua.mrpl_city.1.arhitekturna_prykrasa_mariupolja.pic.4}

Пізніше будинок готелю \enquote{Континенталь} було розширено, однак, як зазначають
співробітники краєзнавчого музею, його вигляд майже не змінився. Краєзнавці
згадують, що на початку ХХ ст. будинок був нефарбованим, з блакитними
дерев'яними рамами на вікнах.

Цікаво, що саме для освітлення готелю \enquote{Континенталь} призначалася перша в
Маріуполі електростанція: вона була відкрита у грудні 1898 року, знаходилася на
вул. Харлампіївській і належала В. Томазо.

12 січня 1918 р. готель \enquote{Континенталь} став центром битви за встановлення
Радянської влади в м. Маріуполі, про що свідчить меморіальна дошка на будинку.
З 1933 р. тут розміщується Клуб металургів заводу \enquote{Азовсталь}. У робітничому
клубі організували самодіяльний драматичний гурток. Його учасниками були
звичайні робітники.

Будівля пережила важкі роки окупації під час Другої світової війни і, як багато
інших, була зруйнована. Після відновлення у 1946 р. в ній працював Палац
культури та техніки металургійного комбінату \enquote{Азовсталь}. У стінах
палацу розпочав діяльність робітничий самодіяльний театр, у творчості якого
були відчутні характерні риси колективу – повага до автора, увага до
внутрішнього світу людини. Саме ці якості визначили всю подальшу його
діяльність, а також посприяли отриманню звання \enquote{Народний самодіяльний
театр} ПК \enquote{Азовсталь} у 1960 р. Завдяки збереженим афішам вдалося
встановити, що на сцені театру можна було побачити зразки російської класики:
\enquote{Горе від розуму} О. Грибоєдова, \enquote{Одруження Бєлугіна} О.
Островського та М. Соловйова. Багато вихованців Народного театру почали
працювати в професійному театрі, з них деякі здобули спеціальну театральну
освіту. Це В. Шкурін, Г. Федющенко, І. Гуревич, О. Яковлєва, В.  Безлепко та
ін.

У концертному залі перед жителями нашого міста виступали чимало знаменитостей:
Любов Орлова, Марк Бернес, Євген Леонов, Борис Андрєєв, Павло Кадочников. У
липні 1971 р. у стінах палацу відбувся незабутній концерт видатного музиканта
Мстислава Ростроповича. У липні 1991 року в палаці пройшла перша в історії
міста виставка оригінальних робіт Архипа Івановича Куїнджі.

\ii{05_02_2018.stz.news.ua.mrpl_city.1.arhitekturna_prykrasa_mariupolja.pic.5}

Новою сторінкою в історії будівлі стало відкриття 23 жовтня 2010 року в його
стінах Палацу культури \enquote{Молодіжний}, який об'єд\hyp{}нав під своїм дахом
маріупольську творчу молодь для її творчої самореалізації та розвитку.

Справжнім відкриттям для маріупольських глядачів стали вистави Народного театру
\enquote{Театроманія}, який відкрився у 2011 р. на базі ПК
\enquote{Молодіжний}. Засновником та керівником клубу є Тельбізов Антон
Миколайович. За шість років існування театрального клубу в його складі було
більше ніж 200 учасників. Проте основний склад трупи залишається незмінним – це
18 найбільш відданих учасників колективу.

Клуб відіграє важливу роль у художньо-творчому та духовно-естетичному розвитку
молоді міста. Колектив \enquote{Театроманії} неодноразово брав участь у різноманітних
фестивалях, конкурсах, міських заходах. Проте головний напрямок роботи –
здійснення вистав на сцені рідного Палацу культури, гастролі. У театрі працюють
три режисери (А. Тельбізов, Н. Гончарова та О. Самойлова), які надають перевагу
сучасній та авторській драматургії: \enquote{13 хвилин тиші}, \enquote{Омлет з гірчицею},
\enquote{Дорогою до щастя}, \enquote{Готель двох світів} тощо. Робота режисерів і гра акторів
\enquote{Театроманії} зачаровує, пробуджує свідомість глядачів, надихає на нові
звершення і серйозні роздуми.

З серпня 2016 року завдяки загальнонаціональному проекту у ПК \enquote{Молодіжний}
працює служба реєстрації шлюбу за одну добу. Молодята можуть вранці подати
заяву і вже через кілька годин стати подружжям. \enquote{Шлюб за добу} має великий
попит у Маріуполі і ще раз підтверджує, що світом дійсно керує кохання. Поряд з
цим у стінах не один рік вражає своєю діяльністю подружжя Сладкових, Олександр
і Марія, які натхненно і з неабияким ентузіазмом працюють над розвитком
української культури Маріуполя.

\ii{05_02_2018.stz.news.ua.mrpl_city.1.arhitekturna_prykrasa_mariupolja.pic.6}

Сьогодні ПК \enquote{Молодіжний} є однією з найбільш потужних куль\hyp{}турно-просвітницьких
платформ міста. А традиційні та креативні гуртки для дітей і молоді, дружня
атмосфера та величезна кількість концертів, творчих виступів, тренінгів відомих
українських та зарубіжних діячів перетворили \enquote{Молодіжний} в одно з найбільш
улюблених місць для багатьох маріупольців.
