% vim: keymap=russian-jcukenwin
%%beginhead 
 
%%file 22_06_2021.fb.fb_group.story_kiev_ua.1.dovbychka_mamochka_kepochka.cmt
%%parent 22_06_2021.fb.fb_group.story_kiev_ua.1.dovbychka_mamochka_kepochka
 
%%url 
 
%%author_id 
%%date 
 
%%tags 
%%title 
 
%%endhead 
\subsubsection{Коментарі}

\begin{itemize} % {
\iusr{Jael Jael}
Браво, на одном дыхании прочла
Концовка-супер

\iusr{Natalie Michael}
Так и ждала подвоха! Класс!

\iusr{Ирина Петрова}
\textbf{Natalie Michael} радію, що сподобалось)

\iusr{Наталья Козакова}

Хороший рассказ! Интересно, что времена меняются, а "мамочки" всё так же
существуют на свете, как будто их не тронули изменения внешнего мира. Может,
это классические признаки какой-то болезни?

\begin{itemize} % {
\iusr{Ирина Петрова}
\textbf{Natalya Kozakova} але тішить, що є константи в цьому, не найкращому зі світів @igg{fbicon.wink} 
\end{itemize} % }

\iusr{Лариса Захарченко}
Спасибо, было интересно и смешно, и немного грустно...

\begin{itemize} % {
\iusr{Ирина Петрова}
\textbf{Лариса Захарченко} дякую за схвальні слова @igg{fbicon.wink}{repeat=2}  @igg{fbicon.face.happy.two.hands} 
\end{itemize} % }

\iusr{Людмила Краснюк}

Интересный рассказ! @igg{fbicon.heart.red} Описание яркое ! Нарисовалась реальная картинка в
автобусной поездке! Браво!  @igg{fbicon.hands.applause.yellow} 

\ifcmt
  ig https://scontent-frx5-2.xx.fbcdn.net/v/t39.1997-6/s168x128/17634213_1652591098100624_731967241620291584_n.png?_nc_cat=1&ccb=1-5&_nc_sid=ac3552&_nc_ohc=G0LACNpAiIoAX9Yk1e_&_nc_ht=scontent-frx5-2.xx&oh=e9f6b9add98d42ee1137101bc3470104&oe=61946877
  @width 0.1
\fi

\begin{itemize} % {
\iusr{Ирина Петрова}
\textbf{Людмила Краснюк} дякую щиро! Все так і було! @igg{fbicon.face.smiling.eyes.smiling} 
\end{itemize} % }

\iusr{Ксения Бычкова}
Уух, как хорошо написано! Браво автору!

\iusr{Ирина Петрова}
\textbf{Ксения Бычкова} дуже приємно, що сподобалось! @igg{fbicon.face.happy.two.hands} 

\iusr{Ирина Ивасенко}
Замечательно, но купаться или нет в Довбичке всё-таки ?!

\begin{itemize} % {
\iusr{Ирина Петрова}
\textbf{Irina Ivasenko} звісно!!! Купатись, плавати, сидіти у прохолодній водичці по шию, краще - з холодною скляночкою смачного напою!

\iusr{Галина Кардаш}
\textbf{Ирина Ивасенко} и меня интересует, можно или нет купаться в Довбычке! И почему "да" или "нет")))

\begin{itemize} % {
\iusr{Ирина Петрова}
\textbf{Galina Kardash} можна! Я впевнена стовідсотково! Бо сама перевіряла! Там ще дуже особливий пісок - світлий, та зернистий, як цукор

\iusr{Ксения Зубакова}
\textbf{Галина Кардаш} потому что там нудиииисты)))
\end{itemize} % }

\iusr{Ксения Зубакова}
\textbf{Ирина Ивасенко} конечно, купаться!

\end{itemize} % }

\iusr{Марина Лазарева}
 @igg{fbicon.hands.applause.yellow}  это прекрасно!

\iusr{Viktoria Terpylo}

Мне почему то нарисовалась в моем воображении такая себе еврейская мамочка
которая задушит чадо в своих объятиях....

А ведь когда то на Довбычку лапоть плавал, потом там был нудистский пляж... а
когда совсем девченками мы переплывали с Молодежного пляжа на матрасе на
Довбычку)

\begin{itemize} % {
\iusr{Maryna Chemerys}
\textbf{Viktoria Terpylo} Нудистський пляж існує і зараз, а Молодіжний пляж оновили,і там приємно відпочивати, як і колись)

\iusr{Viktoria Terpylo}
\textbf{Maryna Chemerys} лет двадцать не была на Молодежном....там в 2000х открыли уже цивильный платный пляж

\iusr{Ксения Зубакова}
\textbf{Viktoria Terpylo} Молодёжный – бесплатный, но толпы. Лодка оттуда на Довбычку – 20 грн. с человека

\iusr{Ирина Петрова}
\textbf{Viktoria Terpylo} та й зараз можна переплисти на човнику, які біля рятувальної станції забирають)))
\end{itemize} % }

\iusr{Анна Загорулько}
Из-за такой мамочки, то есть моей свекрови мы так и не расписались со вторым мужем)

\begin{itemize} % {
\iusr{Ирина Петрова}
\textbf{Anna Zagorulko} як шкода...

\iusr{Анна Загорулько}
\textbf{Ирина Петрова} так, але це уже все в минулому)
\end{itemize} % }

\iusr{Alexander Iyerusalimov}
Такой крутой перец ездит в маршрутке? Не верю.

\begin{itemize} % {
\iusr{Kristinka Medvedeva}
\textbf{Alexander Iyerusalimov} 

авто могло находится на СТО, например.. Да и вообще, разве внешние данные, такие
как бицепс/трицепс и дорогая одежда- это обязательный атрибут людей, которые не
пользуются общественным транспортом?

\begin{itemize} % {
\iusr{Alexander Iyerusalimov}
\textbf{Kristinka Medvedeva} совсем не эксперт по пассажирам маршруток, вам виднее. Полагаюсь на ваше мнение и соглашаюсь с автором.

\iusr{Kristinka Medvedeva}
\textbf{Alexander Iyerusalimov}  @igg{fbicon.thumb.up.yellow} 
\end{itemize} % }

\iusr{Ирина Петрова}
\textbf{Alexander Iyerusalimov} 

чому? Ви бачили у громадському транспорті тільки неохайно та бідно вдягнутих
людей? Попередня читачка висловила вже припущення, та таких випадків може бути
багато) ) це - життя!

\begin{itemize} % {
\iusr{Alexander Iyerusalimov}
\textbf{Ирина Петрова} совсем не хотел что-то плохое сказать о пассажирах маршруток, тем более об их внешнем виде.
Просто так "сложился образ".
\end{itemize} % }

\end{itemize} % }

\iusr{Елена Салата}
Гениально!

\iusr{Ирина Петрова}
\textbf{Olena Salata} щиро дякую! Приємно! @igg{fbicon.face.smiling.eyes.smiling} 

\iusr{Anna Stepanyuk}
Класс!  @igg{fbicon.hands.applause.yellow} 

\begin{itemize} % {
\iusr{Ирина Петрова}
\textbf{Anna Stepanyuk} радію, що сподобалась маленька замальовка)
\end{itemize} % }

\iusr{Dimitri Statnikov}

Разговор людей на диктофон писали или это такое богатое воображение?

\begin{itemize} % {
\iusr{Ирина Петрова}
\textbf{Dimitri Statnikov} ні, в мене гарна пам'ять  @igg{fbicon.wink}  але, й на уяву гріх скаржитись)))

\iusr{Колісник Олександр}
\textbf{Ирина Петрова} ...ну, ну..
\end{itemize} % }

\iusr{Dee Covan}
многа буков. фоткi- зачьот.

\begin{itemize} % {
\iusr{Ирина Петрова}
\textbf{Dee Covan} якось так вийшло... @igg{fbicon.wink} 

\iusr{Тоня Яблонева}
\textbf{Ирина Петрова} 

Прекрасна замальовка! Не розумію, чому Ви виправдовуєтеся? Група "КИ" для тих,
кого не лякають тексти більше абзаца, бо емоції. спогади, почуття не висловиш,
як в анекдоті, лаконічним: "Выехаем-выехайте".

\iusr{Светлана Манилова}
\textbf{Dee Covan}, с меньшим их количеством было бы не так круто. @igg{fbicon.smile} 
\end{itemize} % }

\iusr{Александра Симоненко}
Супер! Классный рассказ  @igg{fbicon.hands.applause.yellow} 

\iusr{Ирина Петрова}
\textbf{Aleksandra Simonenko} щиро дякую @igg{fbicon.face.happy.two.hands} 

\iusr{Анна Березка}
Шикарно!!!

\iusr{Ирина Петрова}
\textbf{Анна Березка} потішили серце автора @igg{fbicon.wink} 

\iusr{Лариса Олейникова}

Ой, спасибо! Прочитала на одном дыхании! Ох уж эти мамочки, умеют чад своих
лишить прекрасных мгновений жизни! Так и будут они подружек выгуливать по
промзоне!

\begin{itemize} % {
\iusr{Ирина Петрова}
\textbf{Larisa Oleynikova} мамочки бувають різними, як, загалом, і синочки! Як казав наш славетний земляк ( мовою оригінала)
"Всяка імієт свой ум голова". @igg{fbicon.face.happy.two.hands} 
\end{itemize} % }

\iusr{Мария Константиновская}
Прелесть!!!

\begin{itemize} % {
\iusr{Ирина Петрова}
\textbf{Мария Константиновская} тішить схвалення читачів @igg{fbicon.face.happy.two.hands}{repeat=2}  @igg{fbicon.hearts.two} 
\end{itemize} % }

\iusr{Раиса Карчевская}

Рассказ - супер, написан с прекрасным юмором, и очень жизненный и отличные фотографии.
Спасибо большое

\begin{itemize} % {
\iusr{Ирина Петрова}
\textbf{Раиса Карчевская} дякую! Приємно, коли людоньки отримують задоволення @igg{fbicon.face.happy.two.hands} 
\end{itemize} % }

\iusr{Мария Елинская}
Чудова розповідь, але мамочьку треба в лісах Пущі 'загубити'

\begin{itemize} % {
\iusr{Ирина Петрова}
\textbf{Мария Елинская} 

ну, якщо в кэпоЧке... @igg{fbicon.laugh.rolling.floor}  Але, такі мамочки виживають суперськи!


\iusr{Ирине Вильчинская}
\textbf{Мария Елинская} 

УЖЕ не поможет... и дело совсем уже не в мамочке... А девчонку жаль, скрутится.
смелется и превратится в бесцветную, выгоревшую тень этого бархатного тирана и
кровопийцы... страшный, в общем-то, рассказ, и написан просто блестяще!

\begin{itemize} % {
\iusr{Ирина Петрова}
\textbf{Ирине Вильчинская} 

дякую! Маємо надію, що дівчина відкриє поширше оченята, глянеться навкруги -
21-ше століття навколо, "Домострой" лишився у минулому))) Хай їй щастить!

\iusr{Ирине Вильчинская}
\textbf{Ирина Петрова} 

домострой-то остался там, а домашнее насилие никуда не делось. И проявляться
может в самых разных формах. Хорошо бы, чтоб девочка во-время прозрела... и
спасибо за ваше неравнодушие. Это теперь тоже в диковинку.

\end{itemize} % }

\iusr{Ирина Петрова}
\textbf{Irina Vilchinskaya} 

тут соглашусь, насчёт домашних тиранов. Увы... из пещер материальных вышли, а
вот из душевных - не все @igg{fbicon.face.disappointed} 

\end{itemize} % }

\iusr{Евгения Верченко}
Біжи Ларіса, біжи!!!
І не оглядуйся!!!

\begin{itemize} % {
\iusr{Ирина Петрова}
\textbf{Евгения Верченко} хто зна, може, вона нас і почула вже  @igg{fbicon.laugh.rolling.floor} 

\iusr{Евгения Верченко}
\textbf{Ирина Петрова} нехай почує...
\end{itemize} % }

\iusr{Игорь Кокарев}
Браво! Ваш фельетон в прошлые времена обязательно напечатали бы в журнале "Перець".

\begin{itemize} % {
\iusr{Ирина Петрова}
\textbf{Igor Kokarev} ого! Звісно, там був непоганий колектив авторів @igg{fbicon.face.happy.two.hands}  мені до них далеко. Дякую!
\end{itemize} % }

\iusr{Марина Прокопенко}
Шикарно написано...
 @igg{fbicon.thumb.up.yellow}{repeat=3} 

\iusr{Ирина Петрова}
\textbf{Марина Прокопенко} Дякую сердечно!

\iusr{Marina Lavrow}
Ну просто чудо!

\iusr{Ирина Петрова}
\textbf{Marina Lavrow} дуже приємно чути схвальні відгуки! @igg{fbicon.face.happy.two.hands} 

\iusr{Тамара Ар}
Вот они, мамкины сынки!,,,,

\begin{itemize} % {
\iusr{Ирина Петрова}
\textbf{Toma Ar} якщо мама не виховає маминого, тоді буде сам свій @igg{fbicon.wink} 

\iusr{Тамара Ар}
\textbf{Ирина Петрова} 

часто, женщина растит сына одна, поэтому он для нее становится всем, и
привыкает к этому,,,, с таким человеком любой другой женщине
сложно,,,,,,, воспитание в плане воспитанности, тут значения не имеет,,,,,,, он
воспитан, но, только по отношению к ней

\end{itemize} % }

\iusr{Владимир Новицкий}
Рассказ хорош, с юмором написан, просто класс.

\begin{itemize} % {
\iusr{Ирина Петрова}
\textbf{Vladimir Novitsky} дякую, пане Володимире, Ваше схвалення тішить @igg{fbicon.face.happy.two.hands} 

\iusr{Владимир Новицкий}
\textbf{Ирина Петрова}
\ifcmt
  ig https://scontent-frx5-2.xx.fbcdn.net/v/t39.1997-6/s168x128/16781161_1341101952618574_7704631035023065088_n.png?_nc_cat=1&ccb=1-5&_nc_sid=ac3552&_nc_ohc=7VVbYisllDsAX_Wd3Vk&_nc_ht=scontent-frx5-2.xx&oh=26fbb3769d691fc4de785a969568c35c&oe=61946C6B
  @width 0.1
\fi

\end{itemize} % }

\iusr{Ирина Снижко}
Рассказ СУПЕР  @igg{fbicon.hands.applause.yellow}  . Спасибо  @igg{fbicon.hands.pray} 

\iusr{Ирина Петрова}
\textbf{Ирина Снижко} автору завжди приємно схвалення @igg{fbicon.face.happy.two.hands}  щиро дякую!

\iusr{Петр Кузьменко}
Прекрасное саркастичное повествование! Вы подарили мне отличное настроение и сделали мой вечер! @igg{fbicon.laugh.rolling.floor}  @igg{fbicon.hands.applause.yellow}  @igg{fbicon.thumb.up.yellow} 

\iusr{Ирина Петрова}
\textbf{Петр Кузьменко} радію! Дуже приємно чути добрі слова читачів! @igg{fbicon.wink}  @igg{fbicon.hearts.two} 

\iusr{Andrei Dvoynos}
Ну с купанием просто... нормальному мужику на этом пляже придётся лежать все время на животе  @igg{fbicon.smile}   @igg{fbicon.smile}   @igg{fbicon.smile} 

\iusr{Ирина Петрова}
\textbf{Andrei Dvoynos} чесно кажучи, якось мої друзі цілком нормально себе почували тут, були і жінки, і чоловіки.

\iusr{Taras Rybaruk}

Саркастична розповідь. Я проживаю в Пущі, тому знаю практичну кожну стежину
повсюди. Купаюсь на 7- лініі з іншого берега за дамбою, з цього боку людей
багато, особливо на вихідні та свята. Пробував на ЦК - 14- та лінія внизу там
озеро не чищене більше 30-ти років, ніхто не купається крім собак, але людей
багато просто відпочиває з пікніком, а ось як йти до Артека вверх на ЦК,
відчуття там не дуже, якось моторошно, вода практично стояча, за часів “совка”
там були прикрі випадки з купанням, крім того там більше дерев навколо, на 4-й
лініі якось також не сподобалось. Тиждень тому знайшов чимало білих грибів,
знаю де й коли збирати. Нажаль відпочиваючі залишають багато сміття повсюди за
собою, ментальність й культура вражають. Також прикро, що Пущу почали
забудовувати останнім часом, завдяки \enquote{зелені}. Якось боремось, але ж.
Венедіктовій та команді \enquote{Зе} схоже Пуща приглянулась для збагачення. Люди
відпочивають в місцях постійно там, де сподобалось. Хтось просто полюбляє десь
полежати на природі чи посидіти. Велосипедистів особливо багатенько за останні
роки. Джерел близько 10-ти по Пущі, більше хороших відгуків про 7-му та 13-ту
лінію, хоча яка вода кому смакує краще, там і набирає. Можливо така коротка
розповідь комусь стане в нагоді.

\begin{itemize} % {
\iusr{Тамара Ар}
\textbf{Taras Rybaruk} 

пущу водицу стали застраивать при данной команде, кончу заспу при предыдущей
команде, исторический центр застраивать многоэтажками при пред предвдущих
командах,,,,,,,,все одно и тоже! И не в лучшую сторону за 30 лет


\iusr{Ирина Петрова}
\textbf{Taras Rybaruk} 

Дякую за розлогу розповідь. Пуща - то не дуже знайома для мене місцевість, так
склалось по життю. Не часто там бували й в дитинстві, з друзями-студентами
їздили, інколи зараз теж буваємо, але, здебільшого восени, і у парку на 4-1
лінії. Тепер буду знати більше з Ваших слів.

\begin{itemize} % {
\iusr{Анна Анна}
\textbf{Ирина Петрова} ! Тарас написал про Пущу, потому что у Вас вторая и третья фотографии под рассказом с Пущи.
А рассказ чудесный! Я кстати его год или два назад уже читала. Еще и посылала его своим нефейсбучным подружкам

\iusr{Ирина Петрова}
\textbf{Анна Анна} дякую
\end{itemize} % }

\end{itemize} % }

\iusr{Людмила Мозговая}
Обожали там купаться!

\iusr{Oksana Shukayeva}
Свят-свят-свят @igg{fbicon.thinking.face} 

\iusr{Юляшка Белкова}
Як колоритно!!! Посміялась від душі!!!

\iusr{Ирина Петрова}
\textbf{Юляшка Белкова} дуже дякую за сміх! Для гарного настроя читача й пишу ☺ ️ 

\iusr{Анатолий Золотушкин}
Замечательный стиль. Какая интрига!

\iusr{Ирина Петрова}
\textbf{Анатолий Золотушкин} щиро дякую, пане Анатолію. Ваше схвалення тішить мене, бо Ви - людина творча!  ☺ ️ 

\iusr{Леся Лагуна}
Ирина, спасибо, получила удовольствие от вашего легкого, с тонким юмором, рассказа ! И фотки классные!

\iusr{Ирина Петрова}
\textbf{Елена Лагуна} дякую! Радію, коли читачі задоволені! @igg{fbicon.hearts.two} 

\iusr{Юлия Прибыткова}
Супер!!!!!!!! Спасибо огромное, море удовольствия @igg{fbicon.face.wink.tongue}  @igg{fbicon.laugh.rolling.floor}  @igg{fbicon.face.smiling.hearts} 

\iusr{Ирина Петрова}
\textbf{Juls Nuw} дуже приємні схвальні відгуки!! Серце автора тішиться @igg{fbicon.face.happy.two.hands}  @igg{fbicon.hearts.two} 

\iusr{Tamara Silenko}
интрига, это ДА!

\iusr{Ирина Петрова}
\textbf{Tamara Silenko} життя - ще той майстер неочікуваних та інтригуючих сценарієв! @igg{fbicon.face.grinning.smiling.eyes} 

\iusr{Victor Orlik}

Довбычка! Незабываемая радость и счастье моего детства и юности, когда приезжал
на каникулы к маме из холодной Москвы, лодка, пустынный пляж вдали от
многолюдья главного пляжа, голубое небо и солнце-со мной всю жизнь.

\begin{itemize} % {
\iusr{Ирина Петрова}
\textbf{Victor Orlik} як я Вас розумію! Довбичка - місце сили! Не можу уявити й собі гарнішого та милішого місця!

\iusr{Victor Orlik}
\textbf{Ирина Петрова} Щиро дякую! Привет от потомственного киевлянина, застрявшего с юности в чужих пределах

\iusr{Ирина Петрова}
\textbf{Victor Orlik} нашого цвіту по всьому світу @igg{fbicon.hearts.two}  @igg{fbicon.face.happy.two.hands} 
\end{itemize} % }

\iusr{Татьяна Васильевна Зубко Маркина}
Зацепило одно слово Довбычка и все посыпались воспоминания

\iusr{Ирина Петрова}
\textbf{Татьяна Зубко} сподіваюсь, що приємні, літні, теплі? Якщо так - радію!

\iusr{Ольга Григор'єва}
Ви автор цього тексту? Читала його пару років тому. Дуже сподобався

\begin{itemize} % {
\iusr{Ирина Петрова}
\textbf{Ольга Григор'єва} так, це вже було опубліковано, але тоді у групі було близько 25 тис. учасників. Зараз вже більше 90 тис. Вирішила поділитись літнім оповіданнячком, сподіваюсь воно зараз на часі - літо, купання)

\iusr{Ольга Григор'єва}
\textbf{Ирина Петрова} чудово написано! Знову прочитала з задоволенням

\iusr{Ирина Петрова}
\textbf{Ольга Григор'єва} дякую! Ніщо так не тішить серце автора, як схвальні відгуки читачів @igg{fbicon.hearts.two}  @igg{fbicon.face.happy.two.hands} 
\end{itemize} % }

\iusr{Ирен Рум}
Придумали?) Честно?)

\begin{itemize} % {
\iusr{Ирина Петрова}
\textbf{Ирен Рум} ні, на жаль ( або на щастя) в нашій родині професіональний письменник - мій старший брат. Я - чисто конкретно інженер-сантехнік. Тому щось вигадати, як того потребує закон літератури, не можу. А ось описати реалії трішки вдається).
\end{itemize} % }

\iusr{Alik Perlov}
Вроде я уже слышал це оповидання)))

\begin{itemize} % {
\iusr{Ирина Петрова}
\textbf{Alik Perlov} Друже, тішить твоя чітка пам'ять! Це означає відсутність склерозу - це радісно!  @igg{fbicon.laugh.rolling.floor}  @igg{fbicon.heart.eyes} 

\iusr{Alik Perlov}
\textbf{Ирина Петрова} Вооот)

\iusr{Ирина Петрова}
\textbf{Alik Perlov} я цим не можу похвалитись, якісь каструлі неслухняні, самі не можуть вимкнути плиту @igg{fbicon.face.weary} 
\end{itemize} % }

\iusr{Tamara Avdeeva}

На Довбычке нудисткий пляж, уже давно. Купаться там можно. Но с пешеходного
моста через Труханов остров пешком далеко идти, а с Гидропарка лодочники
перевозят за деньги.

\iusr{Лариса Бригинец}
Отлично, великолепно! Огромное удовольствие от стиля, содержания, интриги. Спасибо!

\iusr{Ирина Петрова}
\textbf{Лариса Бригинец} щиро дякую за чудовий відгук! Тішить, що порадувала @igg{fbicon.face.happy.two.hands} 

\iusr{Татьяна Сирота}
Замечательный рассказ!
Спасибо, сделали мне настроение! @igg{fbicon.wink} 

\iusr{Ирина Петрова}
\textbf{Татьяна Сирота} дуже цьому радію! @igg{fbicon.face.happy.two.hands} 

\iusr{Светлана Гурневич}
Причудово !

\iusr{Ирина Петрова}
\textbf{Светлана Гурневич} дякую щиро!

\iusr{Ольга Смаль}
Я их прямо представила/)

\begin{itemize} % {
\iusr{Ирина Петрова}
\textbf{Olga Glushchenko Smal} це найліпша хвала від читача, коли все перед очима повстає! Дякую! @igg{fbicon.face.happy.two.hands} 

\iusr{Ольга Смаль}
Дякую вам!
\end{itemize} % }

\iusr{Ольга Лубягина}
Отличный рассказ: как можно терпеть такого " Нудилу"?

\iusr{Ирина Петрова}
\textbf{Ольга Лубягина} оце й мене вразило! Та так, що вилилось у оповіданнячко @igg{fbicon.face.smiling.eyes.smiling} 

\iusr{Alla Stehachova}
Какая прелесть!!!

\iusr{Ирина Петрова}
\textbf{Alla Stehachova} дяку4ю!

\iusr{Sasha Ty}
Здається, новий стиль оповідань зародився, чи як?
 @igg{fbicon.smile} 

\iusr{Ирина Петрова}
\textbf{Sasha Ty} оце так! Здивована приємно! Дякую!

\iusr{Александр Молибог}

Ирина!!! Спасибо!!!! Прекрасно !!!! Улыбка не сходила с лица!!! Сколько
талантливых людей, благодаря, "Киевские истории" имеют возможность писать чудо
рассказы! Ира Молибог

\begin{itemize} % {
\iusr{Ирина Петрова}
\textbf{Александр Молибог} Шановна Ірино! Щиро дякую за теплі слова! Це найкраща та найбажаніша реакція читача! Дуже радію Вашій посмішці! Робіть це частіше! Якщо що - то є тут ще декілька моїх оповіданнячок))) може, й ще посміхнетесь)
\end{itemize} % }

\iusr{Геннадий Романов}
Давно не было такого - дочитал до конца! Хорошо читается, прекрасный слог - талантливо!

\begin{itemize} % {
\iusr{Ирина Петрова}
\textbf{Геннадий Романов} це просто бальзам та єлей на душу автора! Ніщо так не тішить, як схвальні відгуки)! В КІ є ще декілька оповіданнячок...
\end{itemize} % }

\iusr{Kasyan Servetsky}
А у меня как-то сразу в голове визуализировлся этот рассказ. Неплохая бы сцена для какого-нибудь фильма получилась бы!

\iusr{Svetlana Loguinova}

Прекрасный рассказ! Захватывающий, интригующий! Вы превосходный рассказчик,
Мастер! Читая рассказ, я так ясно себе представляла персонажей, согласно Вашему
описанию! Спасибо!

\begin{itemize} % {
\iusr{Ирина Петрова}
\textbf{Svetlana Loguinova} щиро дякую! Щось придумати я не дуже можу, а ось описати дійсність вдалось!)))

\iusr{Svetlana Loguinova}
\textbf{Ирина Петрова}

\ifcmt
  ig https://scontent-frx5-2.xx.fbcdn.net/v/t39.1997-6/s168x128/16781161_1341101952618574_7704631035023065088_n.png?_nc_cat=1&ccb=1-5&_nc_sid=ac3552&_nc_ohc=7VVbYisllDsAX_Wd3Vk&_nc_ht=scontent-frx5-2.xx&oh=26fbb3769d691fc4de785a969568c35c&oe=61946C6B
  @width 0.1
\fi

\end{itemize} % }

\iusr{Наталья Слуцкая}

Довбычка, это хорошо. А пруды в Пуще - грязные лужи. Это в Пуще только
купаться, когда больше негде, а хочется! Еще в прудах Пущи хорошо изучать
зеленых лягушек, если они там еще есть, конечно. Ну и вообще там изучать
болотную фауну Украины и влияние на нее человека.


\iusr{Ирина Петрова}
Тиждень тому були на 14 лінії... чудове озеро, купались)

\iusr{Наталия Черницкая}
 @igg{fbicon.bouquet}  @igg{fbicon.thumb.up.yellow}{repeat=4} восторг!

\iusr{Alisa Brekhova}
Замечательный стиль и отличное чувство юмора!

\iusr{Ирина Петрова}
\textbf{Alisa Brekhova} дякую! Дуже приємно

\iusr{Елизавета Шур}
Неймовірно смачно і талановито!  @igg{fbicon.face.happy.two.hands}   @igg{fbicon.face.happy.two.hands}   @igg{fbicon.face.happy.two.hands} 

\iusr{Ирина Петрова}
\textbf{Елизавета Шур} дякую! Серце автора радіє від теплих слів)

\end{itemize} % }
