% vim: keymap=russian-jcukenwin
%%beginhead 
 
%%file 05_05_2021.fb.dugin_aleksandr.1.russkie
%%parent 05_05_2021
 
%%url https://www.facebook.com/alexandr.dugin/posts/4334925886517418
 
%%author 
%%author_id 
%%author_url 
 
%%tags 
%%title 
 
%%endhead 
\subsection{Раз...  и нерусское превращается в Русское. Это русское чудо}
\label{sec:05_05_2021.fb.dugin_aleksandr.1.russkie}
\Purl{https://www.facebook.com/alexandr.dugin/posts/4334925886517418}

Человек, живущий в России (и не только), всегда сам выбирает: кем быть -
русским, россиянином или кем-то еще. Русский - это культурная и историческая
идентичность - от древнего Новгорода и Киева и дальше вглубь восточных славян
как древней общности до ХХ и XXI веков. Это форма тотального самосознания. И
если уж вы - русский, то вы берете на себя ответственность за Русское, а
Русское берет ответственность за вас. Значит, вы принимаете на себя Русскую
Судьбу. Это очень много. И здесь множество уровней - от глубинной диалектики,
метафизики, религии до языка, сновидений (Русские Сны) и даже до самого
простого и телесного начала (русская плоть отличается от всего остального,
потому что Судьба некоторым образом записана и в ней). Но даже телесный уровень
- в каком-то смысле допускает выбор. Тело может преобразится в русское тело.
Потому что у русских тело это дух (хотя многие об этом и не подозревают).

Россиянин это, скорее, признание принадлежности к современной России (как к
государству) и некоторая дистанция в отношении "русскости". Это тоже возможно.
Дистанция так дистанция. При этом вполне можно из россиянина сделаться русским
и обратно. 

И наконец, прямое разотождествление и с народом, и с историей, и с
государством, и с языком, и с культурой, может привести к самоопределению -
\enquote{нерусский}. Ну \enquote{нерусский}, и ладно. Учтем. Но и здесь не навсегда, не
необратимо. Если внутренняя молния озарит сознание, Русское может открыться как
откровение. Раз...  и нерусское превращается в Русское. Это русское чудо. Оно
может быть, а может и не быть.

