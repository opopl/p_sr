% vim: keymap=russian-jcukenwin
%%beginhead 
 
%%file 30_11_2021.fb.voloshyna_larysa.1.skandal_bloggery_golodomor
%%parent 30_11_2021
 
%%url https://www.facebook.com/lolawolsh/posts/4654599894635723
 
%%author_id voloshyna_larysa
%%date 
 
%%tags blogger,golodomor,lvov,skandal,ukraina
%%title Давайте поглянемо на фактаж скандалу з блогерами
 
%%endhead 
 
\subsection{Давайте поглянемо на фактаж скандалу з блогерами}
\label{sec:30_11_2021.fb.voloshyna_larysa.1.skandal_bloggery_golodomor}
 
\Purl{https://www.facebook.com/lolawolsh/posts/4654599894635723}
\ifcmt
 author_begin
   author_id voloshyna_larysa
 author_end
\fi

Давайте поглянемо на фактаж скандалу з блогерами

1. 27 листопада у День пам’яті жертв Голодоморів українські блогери-мільйонними
влаштовують у Львові вечірку. Привод цілком побутовий - день народження одного
з них, а саме Instagram-блогера з Чернівців Олександра Заліска.

2. Ведучий постить у соцмережах фото зі свята з приписом: «Блогерська голодна
туса», що викликає справедливе обурення громадськості, адже непрозоро натякає
на Голодомор. 

3. Після того, як відомі українські бренди заявляють про припинення контрактів
із блогерами, які брали участь у вечірці, останні виходять на свої мільйонні
аудиторії та починають вибачатися. Але, і тут увага, не за принизливий для
національної гідності жарт ведучого, а за те, що не подивилися у календар, коли
приймали запрошення на святкування дня народження друга.

4. Сам «винуватець торжества» публікує емоційне воззваніє на тему: «Так шо
тепер не жить?», яке повністю витискає з інформаційного просторупричину
скандалу і переводить суспільну дискусію у площину - чи можна святкувати:
уродини, хрестини, або проводити інші персональні свята у дні національної
жалоби? 

5. Суспільство розділяється на тих, хто проти і тих, хто за. Ба більше!
Відбувається певна ідеологічно-ціннісна трансформація мільйонної аудиторії і
самих блогера. Якщо раніше вони просто не мали позиції, про що свідчить їх
нерозуміння неприпустимості жартувати на тему голоду в такий день, то тепер у
них позиція є: «Задовбали зі своїми голодоморами!».  

Висновок. Маємо вдалу інформаційну операцію. Легким помахом маніпулятивної руки
моральний постулат: «Не можна сміятися над національною трагедією» у свідомості
невибагливої публіки перетворюється на лякалку: «Їх національна пам’ять заважає
вам жити повним життям, святкувати, радіти, бути щасливими». Тепер це питання
стає для них особистим і можна не сумніватися, що думку невдовзі у всіх цих
мільйонів з’явиться думка про те, що ворожою для них є саме українська, а не
імперська ідея.  

Наостанок хочу нагадати про те, хто першим вирішив пожартувати про «Сеньйора
Голодомора». І де він тепер? А де поціновувачі його жартів?

\ii{30_11_2021.fb.voloshyna_larysa.1.skandal_bloggery_golodomor.cmt}
