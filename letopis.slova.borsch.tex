% vim: keymap=russian-jcukenwin
%%beginhead 
 
%%file slova.borsch
%%parent slova
 
%%url 
 
%%author 
%%author_id 
%%author_url 
 
%%tags 
%%title 
 
%%endhead 
\chapter{Борщ}
\label{sec:slova.borsch}

%%%cit
%%%cit_pic
%%%cit_text
При этом гадят с каким-то маниакальным удовольствием, умудряясь любое начинание
превратить в совково-колхозный императив. Защита (от кого?) украинского языка у
них как-то случайно закончилась штрафами за использование русского языка.
Декоративная рубаха-вышиванка у них вдруг превратилась в их «генетический код»
с обязательным прогибом детворы надевать ее на первый звонок. Майдан – это у
них революция, только не спрашивай, кто их революционеров вальнул на
Институтской. Не на часi. На Донбассе у них непременно российско-украинская
война, а в Крыму нет линии фронта, потому что то – другое. А борщ? В МИДе будут
орать, что \emph{Борщ} может быть только украинским и затрахают этим даже гид
«Мишлен»
%%%cit_comment
%%%cit_title
\citTitle{Бандеровцы при власти создали в Украине уютненькую Уганду}, 
Игорь Лесев, strana.ua, 13.06.2021
%%%endcit

%%%cit
%%%cit_pic
\ifcmt
  pic https://avatars.mds.yandex.net/get-zen_doc/4787029/pub_60c4a1a87c4727120350b555_60c4d9b903088a3a1a384357/scale_1200
\fi
%%%cit_text
Не андертальцы, наевшись \emph{борща} и согревшись у костра в уютной пещере обладали
безграничным полётом фантазии. Глядя на согревающие языки пламени, они мечтали
о своём прекрасном будущем, о будущем удобстве пещеры и окружающей её
территории. Только и того. Что на фантазировали себе мы не узнаем никогда,
потому что мова тогда только зарождалась, а китайцев, придумавших бумагу, ещё
не родилось. Египтяне с папирусом ещё были на подлёте где-то в районе Альфы
Центавра. Когтем на граните писать не получалось. Вероятнее всего изжили сами
себя в мечтаниях от голодухи и срача. Что косвенно доводят и нынешние потомки
%%%cit_comment
%%%cit_title
\citTitle{От \enquote{псеродактиля} в \enquote{андертальца} и далее в венец человечества: рассуждения никудышного антрополога}, 
Дмитрий Жук, zen.yandex.ru, 13.06.2021 
%%%endcit



%%%cit
%%%cit_head
%%%cit_pic
%%%cit_text
\verb|#знаетеливы| что очень долгое время \emph{борщ} в западном мире был
символом коммунизма. Почему \emph{борщ}? Потому что красный. В многочисленных
голливудских фильмах, начиная с 40-х годов, когда впервые появился шпион мистер
\emph{Борщ}, \emph{страшная овощная похлебка} была сначала знаком
противостояния на земле, в воздухе и даже в космосе. А потом стала неотъемлемой
принадлежностью русской мафии.  В одном из эпизодов супергеройского сериала
60-х годов \enquote{Бэтмен} главные герои попали в лапы Ольги, королевы
\enquote{бессаро́вских казаков}, и она немедленно собралась приготовить из них
традиционный бессаро́вский \emph{свадебный борщ}. По традиции его всегда варят
из пленных, поясняет Ольга, которая называет Бэтмена и Робина на свой лад
Батюшкой и Робончиком
%%%cit_comment
%%%cit_title
\citTitle{Прежде чем бороться за борщ, его бы надо декоммунизировать}, 
Дмитрий Заборин, strana.ua, 21.06.2021
%%%endcit
