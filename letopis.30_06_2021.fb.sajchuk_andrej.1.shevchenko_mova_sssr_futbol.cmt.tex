% vim: keymap=russian-jcukenwin
%%beginhead 
 
%%file 30_06_2021.fb.sajchuk_andrej.1.shevchenko_mova_sssr_futbol.cmt
%%parent 30_06_2021.fb.sajchuk_andrej.1.shevchenko_mova_sssr_futbol
 
%%url 
 
%%author 
%%author_id 
%%author_url 
 
%%tags 
%%title 
 
%%endhead 
\subsubsection{Коментарі}

\begin{itemize}
\iusr{Leonid Kalchenko}
Він раніше десь обіцяв виправитися і виступати українською.

\iusr{Анна Діденко}
Не знала, що він родом з села під Яготином. Феномен дацького #к@ц@пізму, не інакше.

\iusr{Lesya Vakulyuk}
Від авторки ідеї поста - невеличке уточнення: він з Яготинського району) Це Київщина, яка по селах і містечках українською розмовляє з домішками суржика.

\iusr{Юрій Голубінка}
Їжте, не обляпайтесь\par
{\em У Росії вважають, що Україна обіграла Швецію завдяки російській мові}\par
\url{https://shlyahta.com.ua/u-rosii-vvazhaiut-shcho-ukraina-obihrala-shvetsiiu-zavdiaky-rosiiskii-movi}

\iusr{Alina Bodnar}
\textbf{Юрій Голубінка} чому тоді вилетіла росія?)))

\iusr{Joana Madzestes}

Тут нема ніякої України, тут довбана малорось. Я сюди повернулась на два місяці
і засрягла на 17 міс. Люди, скажіть як мені тут вижити? В цьому тоталітарному
неосовку? Я претерпіваю агресію, напади та дискримінацію. Я не можу отримати
адекватну медичну допомогу, я не можу працювати. Я виходячи на вулицю не знаю
чи вернусь жовою додому. Це неосовок!

\iusr{Joana Madzestes}

Пане Андрію, вам смішно з мого коменту? А розказати вам як мені ніж до горлянки
приставляли, чи як ледь не вкокошили даунбасята бо мою київську адресу росіяни
оприлюднили. Не до сміху.

\iusr{Евгений Дьяконов}
Отцепитесь от Андрея.

\iusr{Maksym Kostetskyi}
А відколи це його діяльність оплачується з державного бюджету?)

\iusr{Василь Самохвалов}
Не з державного бюджету, якшошо

\begin{itemize}

\iusr{Andrii Saichuk}
тут справді цікава річ. я чесно не знав цей нюанс фінансування УАФ (не сильно
цікавлюся футболом - швидше як культурним явищем). А хто в такому випадку
платить 60 тис євро в місяць пану тренеру? І чи є він тоді тренером
\enquote{національної збірної}? В чому її \enquote{національність}?

\iusr{Василь Самохвалов}
\textbf{Andrii Saichuk} Ти будеш здивований, але у всьому світі так. Навіть
більше - міжнародні федерації суворо стежать за тим, щоб держава не могла
втручатися у діяльність національних федерацій. Стосовно зарплат - футбол
прибуткова річ, загалом. Виступ збірної на Євро - зокрема.

\iusr{Andrii Saichuk}
\textbf{Vasyl Samokhvalov} виходить це \enquote{світла} сторона українського олігархату
- наявність сильних клубів і потужної національної збірної ! \Laughey[1.0] Але то все не
знімає питання в чому тоді \enquote{національність} збірної, якщо і її тренер і гравець
проводять брифінг мовою - офіційно - держави-агресора 🤷

\iusr{Василь Самохвалов}
\textbf{Andrii Saichuk} питання до уєфа)

\iusr{Andrii Saichuk}
\textbf{Vasyl Samokhvalov} або до українських вболівальників, яким я такими постами, схоже, псую настрій \Smiley[1.0][yellow]

\iusr{Василь Самохвалов}
\textbf{Andrii Saichuk} ну, футболом ти явно не цікавишся)

\iusr{Andrii Saichuk}
\textbf{Vasyl Samokhvalov} футбол мені цікавий - просто в іншому вимірі.

\iusr{Василь Самохвалов}
\textbf{Andrii Saichuk} почни з хокею, там цікавіше

\iusr{Andrii Saichuk}
\textbf{Vasyl Samokhvalov} не в цій країні. Тут має значення футбол, бокс, трохи біатлон

\iusr{Василь Самохвалов}
просто склад команд нашої хокейної ліги за паспортами тебе здивує ще більше, ніж мова шевченка)

\iusr{Василь Самохвалов}
втім, і тренери з біатлону теж)

\iusr{Andrii Saichuk}
\textbf{Vasyl Samokhvalov} та хай буде хоч з Суринаму, аби не москаль)

\iusr{Василь Самохвалов}
\textbf{Andrii Saichuk} ти здивуєшся)

\iusr{Andrii Saichuk}
\textbf{Vasyl Samokhvalov} слухай, я і так знаю, шо всі найліпші українці грали за збірну Канади! )\par
\url{https://www.youtube.com/watch?v=TflsumHdevU}

\iusr{Василь Самохвалов}
\textbf{Andrii Saichuk} 

\url{https://hcdonbass.com/komanda/donbass/}

\iusr{Andrii Saichuk}
так то ж легіонери? вони за збірну не гратимуть.

\iusr{Василь Самохвалов}
\textbf{Andrii Saichuk} вони грають у чемпіонаті україни. а дехто бере й
українське громадянство, для збірної

\end{itemize}

\iusr{Olena Nedoshytko}
Нічого дивного. Андрій Шевченко - малорос і аж ніяк не українець.

\iusr{Ростислав Гудь}
\textbf{Olena Nedoshytko}, оппа, як багато гнівних смайликів))

\iusr{Vladislav Maistrouk}

А причём тут государственный бюджет к зарплате Шевченко, которому платит УАФ,
которая в свою очередь общественная организация не имеющая к государству
никакого отношения, особенно финансового?

\begin{itemize}
\iusr{Andrii Saichuk}

власне намагаюся тоді зрозуміти яке має відношення ця організація до України
(як держави і нації). Чи може презентувати її на національному рівні просто
тому, що купка впливових і багатих людей скинулися і платять зокрема з/п
шевченку? якщо при цьому їм нецікава українська мова і національне
законодавство 🤷

\iusr{Oleksii Bratushchak}
  · 
какаяразніца!)

\iusr{Andrii Saichuk}
Oleksii Bratushchak вот імєнно!)

\iusr{Vladislav Maistrouk}
\textbf{Andrii Saichuk} общение частных лиц никак не регламентируется законом. Что касается языкового вопроса, то это вопрос времени. Перейдут на украинский когда это будет для них естественно. А так это полемика ради полемики

\iusr{Olena Nedoshytko}
Багато питань назбирюється.
А можливо нема ніякої української збірної. Це лише форма їхня, а зміст зовсім інший.
Для себе вирішила, не вболіватиму за них. Чекаю на появлення нового футболу, справжнього, гідного.
😉

\iusr{Ростислав Гудь}
\textbf{Vladislav Maistrouk}, інтєрєсно, пачєму такая ріторіка васнавном пішеца ілі праізносіца на масковском язикє?

\iusr{Vladislav Maistrouk}
\textbf{Ростислав Гудь} патамушта

\iusr{Ростислав Гудь}
\textbf{Vladislav Maistrouk}, ну да, ну да... какаяразніца на каком язикє, проста главнає штоб етат язик бил русскім...

\iusr{Vladislav Maistrouk}
\textbf{Ростислав Гудь} нет, конечно есть разница. Но зависит от контекста.
Когда я в магазине или ресторане, не говоря про госучреждения, я предпочитаю
украинский. От футболистов, зная исторический контекст их формирования, у меня
нет претензий по языку. Да, было бы круто если бы они говорили на украинском,
но как есть. Не за их умение говорить мы за ними следим, и главное что в
головах там нет колорадских жуков. А тараканы то такое...

на войне же никто не делил своих на русскоязычных и україномовних, потому что в
том контексте, как и во многих других, это не повод для разногласий

\iusr{Ростислав Гудь}
\textbf{Vladislav Maistrouk}, Ви там були, на тій війні? А Шевченко Андрюха
був? Може Зінченко? Шо, теж нє? Тоді не маніпулюйте ок?

\iusr{Vladislav Maistrouk}
\textbf{Ростислав Гудь} а вы можете довести обратное? Я общался со многими
ветеранами, волонтёрами который русскоязычные. Это для них не повод кича, но и
не повод стыдится.

\iusr{Ростислав Гудь}
\textbf{Vladislav Maistrouk}, я Вам ніц доводити не збираюсь. Як мінімум тому,
що не зобов‘язаний. Тим паче, ця ситуація ніяким боком не стосується ні війни,
ані військових чи волонтерів. Тому ще раз - залиште при собі свої маніпуляції.


\end{itemize}

\iusr{Іванна Кобєлєва}
Пропонуємо рішення\par 
\url{https://www.facebook.com/100001976791486/posts/4075913742484479/?d=n}

\iusr{Olga Klymko}

Я думаю, він не мав сили говорити взагалі...буває такий стан. Я знаю тих, хто
важко воював,зараз важко живе і досі говорить російською. І я дуже поважаю цих
людей, хто попри все - серцем обрав Україну. І Шевченко тренує не чиюсь збірну,
а нашу.

\begin{itemize}

\iusr{Lesya Vakulyuk}
\textbf{Olga Klymko} він же життям не ризикує, як ті, хто воюють

\iusr{Ірина Рубанова}
\textbf{Olga Klymko} та ні, говорити він сили все ж мав, але говорити
російською. А от українською говорити йому несила все життя. Італійською,
англійською — навчився, бо ж грав за італійський клуб, а от українською — ніяк,
хоч очолює збірну України. Ваші виправдання недолугі, як і тренер збірної
України, етнічний українець, що народився, виріс і навчився грати у футбол в
Україні, але не говорить українською бодай публічно.

\iusr{Ірина Рубанова}

а що до \enquote{тренує не чиюсь збірну, а нашу}, не факт. Росіяни вважають, що раз
одна мова, то одиннарод, а відтак і перемога їхня також. І об'єктивно мають
підстави так вважати, бо культурна і мовна перемога таки за ними, якщо
українська збірна дає інтерв'ю російською.

\ifcmt
  pic https://scontent-mia3-2.xx.fbcdn.net/v/t1.6435-9/209267916_4066428210099695_4499684572777142017_n.jpg?_nc_cat=102&ccb=1-3&_nc_sid=dbeb18&_nc_ohc=E_43G413RJwAX8ll0ko&_nc_ht=scontent-mia3-2.xx&oh=dcfe3c8db5491085e99a6eba2cf89bac&oe=60E52A82
  width 0.4
\fi

\iusr{Olga Klymko}
я не читаю вкиди політтехнологів росіян, і вам не раджу. Шевченко -
молодець,він прославив Україну грою !!! А зарпопонувати йому допомогу у
вивченні - це ОК !

\iusr{Ірина Рубанова}
\textbf{Olga Klymko} які вкиди. Я даю скріншот під моїм коментарем. Про те, що
йому варто би говорити хоч іноді українською говорили, щойно він став головним
тренером, допомоги валом - від Безкоштовних курсів української до цілком
доступних репетиторів на будь-який смак.  Вам не набридло писати бздури?

\iusr{Andrii Saichuk}
\textbf{Olga Klymko} людину, якій 45 років на носі і вона досі не вивчила
українську мову (при цьому вившивши три іноземні) - мало просто запропонувати
допомогу. Тут потрібний не пряник, а батіг.

\iusr{Olga Klymko}
\textbf{Andrii Saichuk} ви були в такому стані виснаження, як Шевченко ? у мене був такий схожий стан один раз, я досягла одного чудового результату, мені дали слово, а я не можу говорити, як каменюки у голові, а не слова. мені дуже жаль, що ви так сприйняли ц... Ещё

\iusr{Olga Klymko}
\textbf{Iryna Rubanova} ці слова від росіян, то і є вкиди російських політтехнологів, невже ви цьогоне розумієте ? чи той пан з рашки щиро так думає на вашу оцінку ?

\iusr{Olga Klymko}
\textbf{Andrii Saichuk} ще трошки відвлліку вашуувагу - на мій погляд саме тут зараз закладається ставлення до мови, але уваги до цього процесу майже ніякої. \par
\url{https://www.facebook.com/sqe.poltava/posts/560642931977572}

\iusr{Саша Воеводенко}
\textbf{Lesya Vakulyuk} а що ви скажете щодо всього, що він зробив для України
за свою кар‘єру? Чи знаєте взагалі його досягнення? 🤦🏻♀️

\iusr{Lesya Vakulyuk}
\textbf{Sasha Voevodenko} ви запитуєте, бо вам справді цікава моя думка? Чи ви
запитуєте риторично, на що вказує «рука/лице”? Якщо перше, то я відповім. Якщо
друге, то не тратитиму ні ваш, ні свій час, бо у вас й так є своя відповідь на
своє ж запитання

\end{itemize}

\iusr{Виталий Байко}

Більшість гравців збірної СССР були українці. Якщо врахувати ватників котрі вже
грають за нашу збірну, то виходить що і нас вже команда СССР


\iusr{Саша Воеводенко}

Я завжди за підтримку української мови, але гнати на Шевченка, людину, яка
зробила для України так багато... тим паче після такої перемоги 🤦🏻♀️🤦🏻♀️🤦🏻♀️

\begin{itemize}

\iusr{Andrii Saichuk}
він сьогодні не в статусі футболіста. він - тренер національної збірної . а це
означає трохи інші вимоги до людини. Йому би точно не відвалилося провести
брифінг якщо не українською, то бодай англійською. і коли шевченко йшов в
політику з регіоналами - чи мали ми його підтримувати? не треба змішувати
політику і спорт (хоча у випадку з футболом це майже неможливо 😉)


\iusr{Roman Oleksyn}
\textbf{Саша Воєводенко}
\ifcmt
  pic https://scontent-mia3-2.xx.fbcdn.net/v/t1.6435-9/209067432_4140696932703414_772126818500342_n.jpg?_nc_cat=105&ccb=1-3&_nc_sid=dbeb18&_nc_ohc=LExvj6gLlwAAX9L22qk&_nc_ht=scontent-mia3-2.xx&oh=26e84587b3c825c4bc7ab96801103512&oe=60E55AEF
  width 0.4
\fi

\iusr{Саша Воеводенко}
\textbf{Andrii Saichuk} моя думка, що тренера збірної зараз доречно тільки
вітати, бути вдячними за перемогу, критикувати за спортивні штуки, бажати
успіху в наступному матчі. Решта (політика) - може зачекати Знову ж таки, на
мою думку, нам (українцям) ще треба вчитися радіти перемогам і бути вдячними
загалом. У вічній боротьбі знецінюються досягнення та втрачається насолода від
життя

\iusr{Саша Воеводенко}
\textbf{Roman Oleksyn} чому мене має турбувати що ці скажені пишуть і
думають?))) у мене є моя збірна, якій наразі я вдячна за досягнення

\iusr{Andrii Saichuk}
\textbf{Саша Воєводенко} багато в чому слушна думка. Але я собі залишаю право "гавкати раз в раз щоби вона не спала" \Smiley[1.0][yellow]

\iusr{Саша Воеводенко}
\textbf{Andrii Saichuk} розумію цю позицію і вона має бути також. Але конкретно
в цьому випадку, у пості точно не вистачило чогось хорошого про перемогу та
досягнення. А це ж можна і треба поєднувати.

У менеджменті є поняття критики і зворотнього зв‘язку. Останній вважається
ефективним методом комунікації. І є техніки, як правильно його давати.
Перепрошую, якщо звучить, як повчання. Але ці техніки роблять дива і я також
намагаюся про них згадувати при нагоді (проф деформація)

\iusr{Andrii Saichuk}
\textbf{Саша Воєводенко} чув, що це також можна назвати способом спілкування з
ідіотами) на жаль мені самому часто бракне розуму для цього)

\iusr{Саша Воеводенко}
\textbf{Andrii Saichuk} взагалі не погоджуюсь. Маю десятирічний досвід на
керівних посадах. Люди різні, по-різному сприймають та передають інформацію.
Правильний та вчасний зворотній зв‘язок - мистецтво, яке вирішує дуже багато
проблем. Працює в будь-яких життєвих ситуаціях, де задіяні двоє та більше людей

\iusr{Andrii Saichuk}
\textbf{Саша Воєводенко} якби ми з вами були керівниками нашого шановного
тренера національної збірної, ми обов'язково застосували до нього всі доступні
види мистецтва комунікацій в управлінні)

\iusr{Andrii Saichuk}
Але на жаль ним керують інші люди. Суркіси різні, Ахмєтови)

\end{itemize}

\iusr{Наталка Федечко}
може хтось з коментаторів захоче конструктивно пообурюватись, то запрошую підтримати українізацію спорту\par
\url{https://www.facebook.com/permalink.php?story_fbid=4462192660458277&id=100000027613671}

\iusr{Lyubomyr Ferens}
Шевченко, який не розмовляє українською. Італійці теж про це писали.

\ifcmt
  pic https://scontent-mia3-2.xx.fbcdn.net/v/t1.6435-9/210436233_10225858861637132_1090462802167181318_n.jpg?_nc_cat=103&ccb=1-3&_nc_sid=dbeb18&_nc_ohc=TAO_D9rbEigAX--Kph9&_nc_ht=scontent-mia3-2.xx&oh=946dc583b00ad46da8b72b95418b863a&oe=60E5E772 
  width 0.4
\fi

\iusr{Andrii Saichuk}
\textbf{Lyubomyr Ferens} встид (

\iusr{Den Alampa}
\enquote{Я не знав... Не надто цікавлюся футболом} – це основний меседж цього посту.

\begin{itemize}

\iusr{Andrii Saichuk}
\textbf{Den Alampa} футбол, це дуже добре. Але для мене є важливішим є питання
української ідентичності. І це мій основний месидж.

\iusr{Den Alampa}
\textbf{Andrii Saichuk} може ще Марлосу пред'яву кинеш, що він не такий, як треба?

\iusr{Тихолаз Ігор}
\textbf{Andrii Saichuk} кажуть \enquote{навчи дурня Богу молитись - лоба розіб'є}... 
не конкретно про вас, Андрію, але загалом про ситуацію...

\iusr{Andrii Saichuk} А я би краще запропонував вам, шановні Den і Ігор
застановитися над тим, що спілкування зі мною в коментарях під моїм постом - це
радше привілей. Я би так радив до цього ставитися. Кажу це з симпатією)

\end{itemize}

\iusr{Алексей Контарёв}

Когда государству надо убить своих граждан оно вспоминает про патриотизм. В
данном случае государство даже на это не способно.

\begin{itemize}
\iusr{Andrii Saichuk}
\textbf{Алексей Контарёв} ну вы же, осмелюсь предположить, патриот какой то отчизны?

\iusr{Алексей Контарёв}
\textbf{Andrii Saichuk} конечно, в 2021 так и есть. Отчизна, отец нации и вся эта ерунда. Куда ж без неё.

\iusr{Andrii Saichuk}
\textbf{Алексей Контарёв} ну что ж, уважая ваше мнение могу лишь выразить
надежду, что возможно однажды преподобный Илон и будущая колония на Марсе
пробудят у вас сантименты, которые безусловно позволят вам проявить толику
эмпатии относительно сих устаревших и вздорных концепций.
\end{itemize}

\iusr{Юрий Борисенко}
Иди спать

\begin{itemize}
\iusr{Andrii Saichuk}
\textbf{Юрий Борисенко} не зрозумів до кого ви це кажете. Уточніть будь ласка.

\iusr{Юрий Борисенко}
\textbf{Andrii Saichuk} Все ты понял. Хорошего дня!

\iusr{Andrii Saichuk}
\textbf{Юрий Борисенко} отак вже і на \enquote{ти} перейшли. Шкода, що на сім наша дружба си скінчила
\end{itemize}

\iusr{Денис Сницкий}

Якщо рідна мова українських олігархів - російська, то чому футбольні клуби, які
належать олігархам мають проводити тренування чи спілкуватися українською, яка
не є рідною для них?

\begin{itemize}
\iusr{Andrii Saichuk}
\textbf{Dennis Snitskiy} абсолютно з вами погоджуюсь. Скажу більше - ми всі в
якомусь сенсі належимо клубу російськомовних олігархів.

\iusr{Денис Сницкий}
\textbf{Andrii Saichuk} логічно, бо бізнес можна робити в Україні тільки в межах дозволеного олігархатом

\iusr{Andrii Saichuk}
\textbf{Dennis Snitskiy} справа навіть не в бізнесі - еліти залишаються здебільшого неукраїнські.

\iusr{Денис Сницкий}
\textbf{Andrii Saichuk} свої наймають на роботу депутатами своїх водіїв,
охорону, прибиральниць колишніх, коханок, кодло друзів та кумів знімає для
українців серіал з ролями «опозиція проти влади», наріт любить таке, ну і
савік теж
\end{itemize}

\iusr{Ростислав Гудь}
Господи, Андрію, який в тебе тут лютий трешак... І ці всі «миру голуби»... курва
🤦♂️

\begin{itemize}
\iusr{Andrii Saichuk}
\textbf{Ростислав Гудь} а кожному жеж не за Шевченка абідна, а за своє право
бути рускоязичєскім украйонцем. От чому кожне таке зауваження викликає гнів і
заперечення. Захисна реакція.
\end{itemize}

\iusr{Nadiya Pavlyk-Vachkova}

Дуже сумно, друзі. Привіт вам з Лос-Анджелеса. Тут майже ніхто не розмовляє
українською. А так би хотілося.

\begin{itemize}
\iusr{Andrii Saichuk}
\textbf{Nadiya Pavlyk-Vachkova} привіт і тобі - в нас тут теж майже ніхто не
розмовляє українською. Особливо в збірній України - в так би хотілося!

%\emph{Nadiya Pavlyk-Vachkova}
%https://www.facebook.com/events/5154-de-longpre-ave-los-angeles-ca-90027-5702-united-states/%D1%83%D0%BA%D1%80%D0%B0%D1%97%D0%BD%D0%B0-%D1%96-%D0%B0%D0%BD%D0%B3%D0%BB%D1%96%D1%8F-%D1%87%D0%B2%D0%B5%D1%80%D1%82%D1%8C-%D1%84%D1%96%D0%BD%D0%B0%D0%BB%D1%83-ukraine-vs-england/692770451555122/

\iusr{Andrii Saichuk}
\textbf{Nadiya Pavlyk-Vachkova} буду. тіки пфайзером вколюся і прилечу)

\iusr{Joana Madzestes}
\textbf{Nadiya Pavlyk-Vachkova} Як можна попросити в США політичного притулку коли тебе в Україні намагаються вбити?

\iusr{Nadiya Pavlyk-Vachkova}
\textbf{Joana Madzestes} Не знаю. Думаю, через адвоката.
\end{itemize}

\iusr{РОМАН КУХАРУК}
І всі футболісти теж мали би. Один Іван Яремчук мовить.

\iusr{Наталка Федечко}
\textbf{Alex Glyvynskyy} Що думаєте з цього приводу?

\iusr{Mykhailo Fradkin}

\enquote{За 5 років УАФ не отримала жодної гривні бюджетного фінансування - Верховна
Рада – новини на УНН | 2 червня 2020, 14:06}\par
\url{https://www.unn.com.ua/uk/news/1872682-za-5-rokiv-uaf-ne-otrimala-zhodnoyi-grivni-byudzhetnogo-finansuvannya-verkhovna-rada}

Умовами контракту не
прописано, що головний тренер має давати коментар на прес-конференції
українською, головний тренер національною збірної не є державною посадовою
особою, він не подає декларацію, він не звітує нікому, окрім керівництву УАФ
про результати роботи. Він може спілкуватися будь-якою мовою, чимало
національних збірних мають тренерів громадян інших країн, які взагалі
користуються перекладачами, дають коментарі англійською, французською,
іспанською, німецькою, італійською. Я знаю Миколу Василькова, він мій друг, і
Олександра Гливинського, які працюють в збірній, 99\% гравців спілкуються
російською, іноді українською розмовляє Яремчук, він львів'янин. В збірній не є
актуальним питання спілкуватися українською, головне тренуватися, грати,
досягати результатів. Все.

\begin{itemize}

\iusr{Andrii Saichuk}
\textbf{Mykhailo Fradkin} все це сумно. дякую вам за цю інформацію. Співчуваю
Яремчуку, який в 30-ту річницю Незалежності мусить підлаштовуватись під
російськомовне середовище. Я би волів, щоб була збірна, яка не вийшла з групи,
але була з патріотів.

\iusr{Mykhailo Fradkin}
\textbf{Andrii Saichuk} вони патріоти, які розмовляють російською, поїдьте на
Донбас на передову, там чимало бійців розмовляють в побуті російською і
боронять Україну. Чим успішніше грає збірна України з футболу, тим більше вона
підвищує імідж України на міжнародній арені, навіть якщо її гравці спілкуються
російською. Це особисто для вас краще, що українські футболісти розмовляли
українською і грали на слабкому рівні, для інших українців головне, щоб
національна команда здобувала перемоги

\end{itemize}

\iusr{Joana Madzestes}

Просрали країну, просрали душу. Створили пекло.

\end{itemize}

