% vim: keymap=russian-jcukenwin
%%beginhead 
 
%%file 10_12_2016.stz.news.ua.mrpl_city.1.magazin_bratjev_adabashevyh
%%parent 10_12_2016
 
%%url https://mrpl.city/blogs/view/magazin-bratev-adabashevyh
 
%%author_id burov_sergij.mariupol,news.ua.mrpl_city
%%date 
 
%%tags 
%%title Магазин братьев Адабашевых
 
%%endhead 
 
\subsection{Магазин братьев Адабашевых}
\label{sec:10_12_2016.stz.news.ua.mrpl_city.1.magazin_bratjev_adabashevyh}
 
\Purl{https://mrpl.city/blogs/view/magazin-bratev-adabashevyh}
\ifcmt
 author_begin
   author_id burov_sergij.mariupol,news.ua.mrpl_city
 author_end
\fi

\ii{10_12_2016.stz.news.ua.mrpl_city.1.magazin_bratjev_adabashevyh.pic.1}

Лет, пожалуй, пятнадцать назад в офис ТРК \enquote{Сигма} зашел парнишка, теребя в руках небольшой сверток.

- У меня документ есть, может, кому понадобится?

- Продать хочешь?

- Ну да.

- Покажи.

Документом оказался небольшой свиток из старинной плотной бумаги. Одна его
сторона светло-серая, другая – желтая. Обе стороны исписаны четким
каллиграфическим почерком.

- Где взял?

- Показали: здесь ломай кирпичи. Пару раз стукнул ломом, кирпичи обвалились -
внутри ниша. Пошарил рукой – бутылка вся в паутине и пыли. Думал - клад,
золото. Бутылку разбил, а там бумажка эта. Хотел выбросить, но вот принес...

- Что за дом, где ты работал?

- Не знаю, как сейчас называется, но старшой говорил – комбинат \enquote{Ждановстрой}.

- На углу проспекта и Торговой?

- Ну да.

\ii{10_12_2016.stz.news.ua.mrpl_city.1.magazin_bratjev_adabashevyh.pic.2}

Действовать надо было быстро. Торг может не состояться, и текст, интересный для нас, пропадет.

- Ты тут присядь, а мы у начальства спросим, что и как.

А сами – бегом в соседнюю комнату к сканеру...

Что дальше было, не заслуживает особого внимания, а текст оказался интересным. Вот выдержки из него. 

\enquote{1895 года, апреля 9-го дня. Царствование Императора Николая II. При закладке
дома братьев Сергея и Ильи Карповичей Адабашевых присутствовали служащие
мануфактурного мариупольского отделения торгового дома \enquote{Бр. С. и И. Адабашевых}
следующие лица: управляющий Василий Паза, его помощник Феодор Пичахчи,
конторщик Барякин, приказчики Петр Браташевский, Ефим Кумброглу, Пантелей
Глущенко, Порфирий Губа, Александр Горшков, Леонтий Карабаш, Петр Маевский,
Зиновий Горпынченко, Моисей Шапиро, Макар Адабашев, Стефан Литвиненко, Вениамин
Хмельницкий, Савелий Капланов, Павел Безручкин, Кузьма Конпанченко, Феодор
Паза}.

Полный список присутствующих при закладке здания  приведен здесь в надежде, что
кто-нибудь из мариупольцев узнает своего пращура. 

На обратной стороне документа названы подрядчик каменных работ Челядников и
столярных – Иван Алексеев. Кроме того, сообщено, что на момент закладки нового
дома братьев Адабашевых  в Мариуполе жителей было до 30 тысяч, семь
православных церквей, одна католическая церковь и две еврейские синагоги, две
гимназии – мужская и женская. Перечислены имена крупных мариупольских купцов.
Не забыты и городской голова того времени Попов, исправник (по нашим понятиям –
начальник милиции) Вильнер, директор гимназии Тимошевский. Этот поистине
исторический документ заканчивается фразой: \enquote{Писал крестьянин Смоленской
губернии Бельского уезда села Татева Лука Никифоров}.

И тут возникает вопрос – о закладке какого из двух строений, принадлежавших
братьям Адабашевым, засвидетельствовал найденный документ. О том ли, на
фундаменте которого после войны было построено административное здание  треста
\enquote{Азовстальстрой}, или пристройки, сооруженной вдоль Торговой улицы,  торец
которой обращен в сторону улицы Николаевской? Вот тема для размышлений.

Кто такие Адабашевы? Сергей (Сероп) и Илья (Егия) Карповичи  Адабашевы
появились в наших краях, а точнее в Таганроге, в конце 60-х или в начале 70-х
годов ХIХ века. Приехали они из Москвы, где занимались торговлей мануфактурой.
Дела у них на новом месте пошли как нельзя лучше. Позже открыли свой магазин и
у нас в Мариуполе.

Чем же торговали предприимчивые братья? Ответ на этот вопрос дает объявление,
помещенное в 1910 году в раритетном издании \enquote{Адрес-календарь. Весь Мариуполь}.
Вот содержание объявления: \enquote{Оптово-розничные склады шелковых, шерстяных,
суконных, полотняных, бумажных, платочных и меховых товаров. Готовое мужское и
дамское платье. Прием заказов. Торговый дом бр. С. и И. Адабашевых. В Мариуполе
и Таганроге собственные дома. Главная контора в Москве. Телефон №82}.

Мариупольские старожилы рассказывали байку. В одном из адабашевских магазинов
случился пожар. Все приказчики, в том числе и  старший, бросились гасить пламя.
Примчался пожарный обоз. Пожарные принялись обильно поливать очаг огня, а с ним
и тюки с мануфактурой. Один из братьев - Сергей ли, Илья ли - увидев всю суету,
рявкнул: \enquote{Бросьте все! Быстро нанимайте извозчиков, и галопом в Таганрог за
товаром! Чем завтра торговать будете?!}
