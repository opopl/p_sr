% vim: keymap=russian-jcukenwin
%%beginhead 
 
%%file slova.pravilo
%%parent slova
 
%%url 
 
%%author 
%%author_id 
%%author_url 
 
%%tags 
%%title 
 
%%endhead 
\chapter{Правило}
\label{sec:slova.pravilo}

%%%cit
%%%cit_head
%%%cit_pic
%%%cit_text
Сергей Фурса написал длинный пост о том, что у украинцев не нужно ничего
спрашивать, потому что они ни в чем не разбираются. Это очень понятный и
рукопожатный пост – не спрашивать же, в самом деле, мнения людей, которые живут
в \emph{неправильных} регионах, разговаривают на \emph{неправильном} языке, голосуют за
\emph{неправильных} политиков, празднуют \emph{неправильные} праздники и имеют \emph{неправильных}
предков?  Но вслух об этом можно было и не говорить. Тем более что большинство
«реформ» в стране и так вполне официально предусматривают передачу прав и
полномочий международным структурам, которые «лучше знают как надо»
%%%cit_comment
%%%cit_title
\citTitle{В этой Украине живут какие-то неправильные украинцы}, Вячеслав Чечило, strana.ua, 07.07.2021
%%%endcit

%%%cit
%%%cit_head
%%%cit_pic
%%%cit_text
Вот я здесь пожила почти два месяца и поняла, что \emph{правильно} российская
пропаганда запугивает людей Украиной, \emph{правильно} им впаривает, что здесь пьют
кровь младенцев и всякую другую, \emph{правильно} им навешивает жуткой
нацистско-бандеровской лапши на их доверчивые уши, \emph{правильно} их доводит до того
состояния, что они реально боятся сюда ехать (я сейчас исключаю всякого рода
вирусную составляющую и тот естественный факт, что их в принципе сюда никто
особо и не приглашает).
Они боятся, потому что их научили бояться. И ненавидеть. И \emph{правильно}. Потому
что если россияне опять станут ездить в Украину так же массово, как и раньше,
они прямо носом почуют, насколько же здесь свободнее
%%%cit_comment
%%%cit_title
\citTitle{Два месяца живу в Украине и теперь понимаю, почему россиянам навешивают лапшу}, 
Елена Рыковцева, news.obozrevatel.com, 05.07.2021
%%%endcit
