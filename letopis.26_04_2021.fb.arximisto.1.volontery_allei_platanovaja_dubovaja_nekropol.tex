%%beginhead 
 
%%file 26_04_2021.fb.arximisto.1.volontery_allei_platanovaja_dubovaja_nekropol
%%parent 26_04_2021
 
%%url https://www.facebook.com/arximisto/posts/pfbid02Hj22xi8mJnwkVaUKj4sRn9ZB2fmKEPoRd1rQNCHZpNgjMM9dGNd56NXhTRcW9wNXl
 
%%author_id arximisto,golovnjova_maryna.mariupol,marusov_andrij.mariupol
%%date 26_04_2021
 
%%tags 
%%title Волонтеры заложили Платановую и высадили Дубовую аллеи в Некрополе
 
%%endhead 

\subsection{Волонтеры заложили Платановую и высадили Дубовую аллеи в Некрополе}
\label{sec:26_04_2021.fb.arximisto.1.volontery_allei_platanovaja_dubovaja_nekropol}
 
\Purl{https://www.facebook.com/arximisto/posts/pfbid02Hj22xi8mJnwkVaUKj4sRn9ZB2fmKEPoRd1rQNCHZpNgjMM9dGNd56NXhTRcW9wNXl}
\ifcmt
 author_begin
   author_id arximisto,golovnjova_maryna.mariupol,marusov_andrij.mariupol
 author_end
\fi

Волонтеры заложили Платановую и высадили Дубовую аллеи в Некрополе

\#новости\_архи\_города

На прошедших выходных волонтеры заложили Платановую и высадили Дубовую аллеи в
Мариупольском Некрополе, посадив двадцать саженцев деревьев.

Мариупольский Некрополь – это потенциальные \enquote{зеленые легкие} центра
города, как считает Андрей Марусов, директор ГО \enquote{Архи-Город}. Но сейчас
16 гектаров его территории заполнены сухостоем, дикой сиренью и выродившимися
деревьями. 

Волонтеры посадили четырнадцать дубов вдоль дорожки, которая ведет от входа в
Некрополь к усыпальнице Найденовых, рядом с участком, очищенном Комбинатом
коммунальных предприятий в прошлом году (см. карту). Они были подарены
лесниками \enquote{Азовского лесничества} в Никольском.

Платановую аллею волонтеры заложили вдоль самой древней аллеи Некрополя (на ней
до сих пор сохранилась брусчатка). Когда-то она вела от Марьинки к алтарю
церкви Всех Святых. 

Платан был выбран не только из-за его красоты и засухоустойчивости. Он имеет
символическое значение. Это едва ли не самое древнее дерево из ныне живущих на
Земле (зародился около 100 млн. лет назад).

Платан ассоциируется с Грецией. Создание Платановой аллеи - это дань памяти и
уважения к грекам-первопоселенцам Мариуполя. Самое древнее датированное
захоронение Некрополя находится на этой аллее, и оно принадлежит Гавриилу
Сахаджи, греку-первопоселенцу (1833-34 гг.).

Волонтеры посадили пять саженцев платанов. Они были закуплены за средства,
пожертвованные благотворителями, в дендроферме семьи Бережнюк подо Львовом
(один саженец обошелся в 78 грн., общая стоимость составила 452 грн. с
доставкой).

Для создания полноценной Платановой аллеи нужно еще около 35 саженцев (длина
аллеи – около 200 метров).

Следует подчеркнуть, что волонтеры сажали деревья только на очевидно пустых
местах, чтобы не нарушать права родственников мариупольцев, похороненных на
кладбище.

Напомним, что в прошлом году Комбинат коммунальных предприятий полностью
очистил от зарослей значительный участок кладбища. С апреля прошлого года
волонтеры очищают древний участок Некрополя и одновременно высаживают на нем
клены и дубы (уже посажено 18 деревьев).

Контактная информация: Андрей Марусов (096 463 69 88).

Мы искренне благодарим волонтеров, участников закладки аллей Maryna  Holovnova,
Елену Сугак, Надежду Андреевну Чапни, Илью Луковенко, Андрея Никифоренко,
Александра Сальникова и Андрея Клепикова.

И - огромное спасибо нашим благотворителям!

Мы будем благодарны за любую помощь в дальнейшем озеленении Мариупольского
Некрополя!

\#mariupol\_necropolis\_report

\#платановая\_аллея
