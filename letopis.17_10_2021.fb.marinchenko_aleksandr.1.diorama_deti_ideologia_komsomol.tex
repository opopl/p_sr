% vim: keymap=russian-jcukenwin
%%beginhead 
 
%%file 17_10_2021.fb.marinchenko_aleksandr.1.diorama_deti_ideologia_komsomol
%%parent 17_10_2021
 
%%url https://www.facebook.com/alex.marinchenko.9/posts/6266982653376893
 
%%author_id marinchenko_aleksandr
%%date 
 
%%tags istoria,komsomol,marazm,ukraina
%%title Як казав колись професор Ковальський "Маразм крепчал" або комсомол і досі живий
 
%%endhead 
 
\subsection{Як казав колись професор Ковальський \enquote{Маразм крепчал} або комсомол і досі живий}
\label{sec:17_10_2021.fb.marinchenko_aleksandr.1.diorama_deti_ideologia_komsomol}
 
\Purl{https://www.facebook.com/alex.marinchenko.9/posts/6266982653376893}
\ifcmt
 author_begin
   author_id marinchenko_aleksandr
 author_end
\fi

\headCenter{рухаємось по колу, а \enquote{маразм крепчает}}

Як казав колись професор Ковальський "Маразм крепчал" або комсомол і досі
живий.

14-го в Діорамі було багато відвідувачів, особливо школярів. Особисто мене
вразив 2-й клас. Поводилися дуже добре, уважно слухали, ставили цікаві
запитання. Дівчат особливо вразив той факт, що в лавах Червоної армії службу
проходило до 1 млн жінок (оцінки різняться, але все одно мова йде про дуже
велику кількість осіб), дві з яких зображені на полотні Діорами - учасниці
форсування Дніпра в районі сіл Військове-Вовніги у вересні 1943 р. Неля
Кожухова та Зоя Серовікова. Наприкінці екскурсії вже немолода вчителька збирає
клас і починає бравурну циркову виставу:

"Дівчата, збираємося навколо мене і будемо гучно вітати хлопців з чоловічим
днем! Всі вони майбутні герої, які успадкували славетні традиції захисту
Батьківщини!" 

Далі просто атас і класичний приклад еклектичного і надзвичайно
заідеологізованого мислення (хоча шкарпетки все ж таки не дарували!):

"Наші славетні пращури боронили українську землю в часи Київської Русі від
татаро-монголів і половців; козаки під проводом Хмельницького відбивали атаки
поляків і татар; Петлюра захищав Вітчизну від червоних банд; бійці Червоної
армії зламали хребет фашистським загарбникам в роки Великої Вітчизняної, які
вчинили Голокост; герої УПА до останнього чинили спротив сталінському
тоталітарному режиму, а сьогодні наші військові стоять на захисті нашого миру і
спокою на Сході України!" І все це одночасно, скоромовкою і в одному флаконі,
наше читання завченого вкрай невдалого віршу з претензією на гарну оцінку (!!!)
Я називаю таку позицію "І нашим, і вашим", щоб ніхто раптом не підкопався. 

Я ледь не впав від настільки еталонного маразму.

Дівчат це обурило, вони не розуміли, чому мають вітати виключно чоловіків, коли
у лавах ЗСУ перебуває величезна кількість жінок і що вони самі хочуть піти на
військову службу. Тітонька після гучного протесту нарешті вгамувалася і пішла
пудрити носик. А я ще довго стояв і чухав потилицю, розмірковуючи на тему "Що у
таких людей коїться в голові?"

Якісь нашарування різних епох, справжній вінегрет.

Мораль цієї басні така: рухаємось по колу, а "маразм крепчает".

Гарного всім дня.

\ii{17_10_2021.fb.marinchenko_aleksandr.1.diorama_deti_ideologia_komsomol.cmt}
