% vim: keymap=russian-jcukenwin
%%beginhead 
 
%%file 28_06_2018.fb.lesev_igor.1.illjuzii_moskva_pribaltika
%%parent 28_06_2018
 
%%url https://www.facebook.com/permalink.php?story_fbid=1971818859515874&id=100000633379839
 
%%author_id lesev_igor
%%date 
 
%%tags geopolitika,illjuzia,nato,pribaltika,rossia
%%title О навязчивых иллюзиях
 
%%endhead 
 
\subsection{О навязчивых иллюзиях}
\label{sec:28_06_2018.fb.lesev_igor.1.illjuzii_moskva_pribaltika}
 
\Purl{https://www.facebook.com/permalink.php?story_fbid=1971818859515874&id=100000633379839}
\ifcmt
 author_begin
   author_id lesev_igor
 author_end
\fi

О навязчивых иллюзиях.

Последние пару лет регулярно натыкаюсь в достаточно солидных западных изданиях
на «планы России» по захвату Прибалтики. Русские «намерены» захватить или часть
балтийских стран (Эстонию, чтобы обезопасить Питер), или «сувалкский коридор»
(чтобы выйти сухопутно к Калининграду), или все скопом. Натыкался даже на такие
«планы», как захват датского острова Борнхольм (без него, оказывается, никак
нельзя обезопасить в сохранности балтийские газопроводы) и даже шведского
острова Готланд. Швеция, напомню, вообще нейтральная страна.

\ifcmt
  ig https://scontent-frt3-1.xx.fbcdn.net/v/t1.6435-9/36363158_1971818699515890_2667229511778566144_n.jpg?_nc_cat=106&ccb=1-5&_nc_sid=730e14&_nc_ohc=1b3t4F0NSKYAX9Ih9nH&_nc_ht=scontent-frt3-1.xx&oh=bf56b673eba5ae2eef75a047aaf3db2b&oe=61BC35BF
  @width 0.4
  %@wrap \parpic[r]
  @wrap \InsertBoxR{0}
\fi

И озвучивают эти концепции не украинские СМИ, а вполне себе именитые издания
вроде «Вашингтон пост» или западные аналитические институты.

Что характерно, во всех этих «планах» нет базового объяснения, зачем русским
Прибалтика. Вернее, даже так – зачем русским НАСТОЛЬКО Прибалтика, чтобы
рисковать ее захватить? План по захвату балтийских стран подается как само
собой разумеющееся.

Давайте посмотрим на вариант захвата Прибалтики со стороны Москвы.

Первое. Самое страшное УЖЕ произошло, и произошло давно. Прибалты в НАТО. Т.е.,
«предотвращать ужасное», как это было в случае с Крымом, смысла нет никакого.

Второе. Помимо смысла, нет НИКАКИХ возможностей. Русские сокращают свои военные
расходы, потому что нихрена не тянут по куче направлений. В текущем году у них
военный бюджет составляет 46 ярдов зелени. Для сравнения, в США более 700
миллиардов, а мелкая Британия заложила 60 миллиардов. Абсолютно по всем видам
вооружений Россия в разы отстает от стран НАТО, а единственный сдерживающий
фактор у Москвы, позволивший ей пережить 90-е – это наличие ЯО.

Третье. Захват Прибалтики ничего принципиально не меняет для России в
логистическом плане, а также военно-стратегическом. Выход к морю у русских и
так есть. И даже есть кусочек своей милитаризированной Прибалтики в виде
Калининградской области.

Четвертое. Зато гипотетический захват Прибалтики окончательно сформирует в
мировой тусовке мнение о России и Кремле, как о сумасшедшей стране. Со всеми
вытекающими. Москва пытается как-то пропетлять с Крымом, а тут еще сверху
довесок в виде Прибалтики. Это как выстрелить в голову, когда та чешется.

И пятое. Присоединять что-то, где жизненный уровень в несколько раз выше – это
всегда неэффективно. А ведь это Что-то нужно будет включать во внутреннюю
хозяйственную цепочку, формировать лояльное население и куй знает сколько
бюджетировать, бюджетировать, бюджетировать. Это стране, которая уже лет 10
находится в перманентной рецессии.

Это мы посмотрели на Прибалтику из-за поребрика. Но у американцев и их
маленьких нервных союзников тоже есть свои резоны разгонять тему «оккупации
Прибалтики».

Первый и самый главный резон – это бюджетные няшки. Русские сокращают военные
расходы, доведя их уже ниже 50 ярдов. Для «империи» более чем скромные расходы.
А у главной страны добра эти расходы превышают 700 ярдов. В год. При текущих
уровнях расходов, России нужно 15 лет на то, что Америка в оборонке тратит за
один год.

Отставание русских дикое. Для примера, доходы всего федерального бюджета РФ в
2017 году чуть превысили 250 миллиардов долларов. Короче, американским
ВПК-лоббистам все сложнее объяснять, почему они на оборонку тратят 700, а не
500 или даже 200 ярдов зелени. Для этого нужно придумывать байки из склепа.
Выбрали «захват Прибалтики». А могли ведь и потерей Аляски пугать. Или планами
по захвату японского Хоккайдо.

Второй резон – это мобилизационный. Он уже для восточноевропейских элит. В
Прибалтике, Польше и Украине есть целый набор партий, которые всю свою
избирательную стратегию выстраивают на практически единственном месседже –
«восточной угрозе». В общем-то, это удобно. Хотя работает со скрипом даже в
Украине, которая реально потеряла Крым. Но это позволяет надрачивать население
до состояния стеклянных глаз. И уходить от темы реальной экономики и социалки.

А теперь самое интересное. Мы имеем классическую ситуацию, когда хвост виляет
собакой. Озвучивая «планы» по захвату Прибалтики и пугая обывателя, который и
сам пугаться рад, американцы вынуждены для проформы предпринимать «ответные
действия» по «предотвращению русской угрозы». Для этого начинаются вялые
перебрасывания каких-то частей в Прибалтику. Вот 900 десантников прибыло. А вот
начинаются воздушные патрули. А тут мы 50 «Абрамсов» подтянули. Иными словами,
чтобы пилить эти 700 ярдов зелени, нужно показывать какие-то телодвижения.

Но ведь и за поребриком на все это начинают смотреть с выпученными глазами. В
русских СМИ уже появляются материалы, что под предлогом ру-вторжения,
американцы сами готовят вторжения в Россию со стороны Прибалтики. И там тоже
начинаются телодвижения, только уже перепуганного характера. Переброска новых
частей в Псковскую и Ленинградскую области. Накачка ракетами Калининграда. И –
САМОЕ-САМОЕ – составление настоящих планов в случае войны со странами НАТО. И в
этих планах, конечно же, уже будут появляться НАСТОЯЩИЕ планы по захвату
Прибалтики.

Такое вот кино. «Русские идут – 2».

\ii{28_06_2018.fb.lesev_igor.1.illjuzii_moskva_pribaltika.cmt}
