% vim: keymap=russian-jcukenwin
%%beginhead 
 
%%file 19_06_2019.stz.news.ua.mrpl_city.1.ilona_arhangelska_ne_bijtes_buty_inshymy
%%parent 19_06_2019
 
%%url https://mrpl.city/blogs/view/ilona-arhangelska-ne-bijtesya-buti-inshimi
 
%%author_id demidko_olga.mariupol,news.ua.mrpl_city
%%date 
 
%%tags 
%%title Ілона Архангельська: "Не бійтеся бути "іншими""
 
%%endhead 
 
\subsection{Ілона Архангельська: \enquote{Не бійтеся бути \enquote{іншими}}}
\label{sec:19_06_2019.stz.news.ua.mrpl_city.1.ilona_arhangelska_ne_bijtes_buty_inshymy}
 
\Purl{https://mrpl.city/blogs/view/ilona-arhangelska-ne-bijtesya-buti-inshimi}
\ifcmt
 author_begin
   author_id demidko_olga.mariupol,news.ua.mrpl_city
 author_end
\fi

\ii{19_06_2019.stz.news.ua.mrpl_city.1.ilona_arhangelska_ne_bijtes_buty_inshymy.pic.1}

На мою думку, герої нашого часу – це ті, хто розуміють, що треба не боятися
діяти, володіють неабиякою силою духу, чітко усвідомлюють мету або знають, для
чого потрібні ті або інші нововведення. Продовжуючи розглядати життєвий шлях
непересічних маріупольців, пропоную познайомитися з жінкою, чия громадська
діяльність давно стала відома за межами міста. Вона пройшла довгий шлях від
звичайного волонтера до голови громадської організації і зробила чималий внесок
в соціальний та культурний розвиток Маріуполя. Мова піде про \textbf{Ілону
Архенгельську} – голову ГО \enquote{Клуб любителів японської культури},
організаторку багатьох загальноміських суспільно-важливих заходів.

Ілона народилася в Черкаській області. Батько працював інженером на
радіозаводі, мати викладала машинопис. Сім'я Ілони переїхала до Маріуполя у
середині 80-х, але до травня 1986 року жили на 2 міста: Київ та Маріуполь
(батько навчався в аспірантурі та працював у КПІ). Після Чорнобильської
трагедії остаточно переїхали до Маріуполя.

У 2005 році Ілона отримала диплом спеціаліста \enquote{Математик. Викладач математики
та основ економіки} Бердянського Державного Педагогічного Університету. З 2006
по 2008 рік вивчала англійську, японську мови та міжнародні відносини у
Токійському відділенні американського університету TEMPLE. У 2009 році
повернулась до Маріуполя, вийшла заміж і народила доньку – Софію. У Маріуполі
найбільше любить проводити час із родиною в Міському саду. Вона радіє, що сім'я
завжди підтримає і допоможе у всьому.

\ii{insert.read_also.demidko.zajceva}

\ii{19_06_2019.stz.news.ua.mrpl_city.1.ilona_arhangelska_ne_bijtes_buty_inshymy.pic.2}

З 2013 до 2015 року Ілона Архангельська працювала в освіті та соціальному
секторі. У 2014 році вона побачила на сторінці заступника міського голови
прохання надати допомогу одягом та речами першої необхідності
дітям-переселенцям, які переїхали до Маріуполя з території проведення АТО. Вона
спочатку відвезла речі своєї дитини до тимчасового притулку, а потім почала
збирати речі по знайомим та родичам. Так почалась її волонтерська діяльність.
Восени 2014 року вона стала лауреатом конкурсу \textbf{\enquote{Маріуполець (маріупольчанка)
року}}.

\ii{19_06_2019.stz.news.ua.mrpl_city.1.ilona_arhangelska_ne_bijtes_buty_inshymy.pic.3}

На початку 2015 року громадська активістка разом з Маріупольським відділенням
товариства Червоний Хрест Україна організувала благодійну виставку у приміщенні
Маріупольського крає\hyp{}знавчого музею. Ілона надала свою колекцію японських
предметів культури та побуту. Всі кошти від продажу білетів були переведені на
банківський рахунок товариства Червоного Хреста з метою закупівлі ліків для
дітей-переселенців.

\ii{19_06_2019.stz.news.ua.mrpl_city.1.ilona_arhangelska_ne_bijtes_buty_inshymy.pic.4}

У 2015 році маріупольчанка зареєструвала \textbf{ГО \enquote{Клуб любителів японської
культури}}. У тому ж році завдяки громадський активістці \textbf{Афіні Хаджиновій}
познайомилась з волонтерською програмою \enquote{Активні громадяни} від Британської
ради, та почала проводити разом з Афіною та іншими волонтерами тренінги зі
сприяння сталому розвитку та розбудові довіри і порозуміння в громаді.
Допомагала з написанням та реалізацією проектів різним громадським діячам. Один
з таких проектів був \textbf{\enquote{1000 журавликів миру}}. Вона з активістами проводила
майстер-класи з приготування їжі для дітей-сиріт, що мешкають у соціальному
гуртожитку, привозила кінопокази роликів на тему єдності та толерантності,
проводила чаювання з лекціями та вікторинами у Територіальному центрі
соціального обслуговування людей похилого віку та в адаптаційному центрі для
колишніх алко- та наркозалежних жінок (а з 2014 року й для тимчасових
переселенок) \enquote{Маленька мама}.

\ii{19_06_2019.stz.news.ua.mrpl_city.1.ilona_arhangelska_ne_bijtes_buty_inshymy.pic.5}

Наразі третій рік поспіль вона з волонтерами проводить молодіжний фестиваль
східних мистецтв, який збирає протягом 3 годин більш ніж 1000 осіб. На цьому
фестивалі гості можуть безкоштовно взяти участь у дуже корисних майстер-класах,
подивитися концерт, виставки робіт маріупольської молоді, побачити, як завдяки
фестивалю переселенці та люди з інвалідністю, які беруть в ньому участь,
адаптуються до життя в Маріуполі та знаходять нових друзів і однодумців.

У 2018 році завдяки підтримці громади вдалося зібрати кошти на поїздку
дівчинки-сироти з інвалідністю \textbf{Софії Ревякіної} на змагання з пара-карате до
Будапешту.

\textbf{Читайте також:} М\emph{ариупольские каратисты триумфально выступили на чемпионате мира}%
\footnote{Мариупольские каратисты триумфально выступили на чемпионате мира, Георгий Федоренко, mrpl.city, 30.04.2018, \par%
\url{https://mrpl.city/blogs/view/mariupolskie-karatisty-triumfalno-vystupili-na-chempionate-mira}
}

\ii{19_06_2019.stz.news.ua.mrpl_city.1.ilona_arhangelska_ne_bijtes_buty_inshymy.pic.6}

Багато різних людей допомагають Ілоні втілити в життя різноманітні ідеї та
проекти. Хтось допомагав з тренінгами, хтось надавав експонати для благодійної
виставки, хтось приміщення для тренінгів, хтось з реалізацією проектів чи
порадами з проведення заходів.

У майбутньому маріупольчанка планує й надалі проводити заходи, що об'єднують
громаду, освітні заходи для молоді та впроваджувати у місті ідеї, що
сприятимуть розвитку, які дуже добре показали себе за кордоном.

Нашу героїню надихають громадські активісти та волонтери зі всього світу, що
займаються соціальним сектором. У 2017-му році вона їздила у Лондон на з'їзд
Активних громадян. Почула там багато історій про корисні та життєво необхідні
справи, що роблять ці волонтери у своїх країнах та зрозуміла, що цей досвід
необхідний і Україні.

\ii{19_06_2019.stz.news.ua.mrpl_city.1.ilona_arhangelska_ne_bijtes_buty_inshymy.pic.7}

\textbf{Улюблені книги:} оповідання О. Генрі.

\textbf{Хобі:} колекціонує предмети японської культури та побуту.

\textbf{Порада маріупольцям:} 

\begin{quote}
\em\enquote{Не бійтеся брати участь у громадському житті міста, місцевих конкурсах,
проектах, ініціативах. Генерувати ідеї та втілювати їх у життя разом з іншими
однодумцями. Не бійтеся бути \enquote{іншими}! Повірте, нас таких багато і завжди
знайдуться люди, які підтримають цікаву ініціативу. Не бійтеся звертатися до
органів місцевої влади за допомогою}.
\end{quote}

\textbf{Читайте також:} \emph{Аниме, айкидо и чайная церемония: в Мариуполе прошел \enquote{СхідFest}}%
\footnote{Аниме, айкидо и чайная церемония: в Мариуполе прошел \enquote{СхідFest}, Ярослав Герасименко, mrpl.city, 12.05.2019, \par%
\url{https://mrpl.city/news/view/anime-ajkido-i-chajnaya-tseremoniya-v-mariupole-proshel-shidfest-foto}
}
