% vim: keymap=russian-jcukenwin
%%beginhead 
 
%%file 30_03_2022.stz.news.ua.pravda.2.ohtyrka.7.kim
%%parent 30_03_2022.stz.news.ua.pravda.2.ohtyrka
 
%%url 
 
%%author_id 
%%date 
 
%%tags 
%%title 
 
%%endhead 

\subsubsection{Я не можу бути, як Кім}
\label{sec:30_03_2022.stz.news.ua.pravda.2.ohtyrka.7.kim}

Я щодня ходжу на похорони, на всі похорони. 

Мене часто питають, чому я такий сумний, не посміхаюся, як Кім. Не можна ходити
на похорони і через півгодини бути веселим.

Я ж 20 років лікував цих людей, я допоміг майже кожній родини в місті. Дітям,
дорослим, бабусям, дідусям. Я їх усіх знаю.  Я знаю майже кожну людину, яка
плаче над домовиною. Я не можу бути, як Кім.

Зараз ми намагаємося хоронити не по одному, а по десять чоловік. На похоронах у
нас багато військових і дуже мало цивільних. 

Сьогодні я був у селі, куди звозять хлопців, які десь під Харковом загинули, то
там на поховання приходить по півтори тисячі людей. Для них смерть – це ще
неприйнятно. Якщо біда прийшла в сім'ю, значить, вона прийшла в селище.

А в нас у місті вже люди \enquote{згоріли}. Розумієте?

\ifcmt
  ig https://img.pravda.com/images/doc/7/e/7e6f643--------.jpg
	@caption Попри обстріли й бомбардування, український прапор розвівається на флагштоці в Охтирці. Позаду – пам'ятник захисникам України, встановлений на місці пам'ятника Леніну. Фото: Наталія Гуменюк
  @wrap center
  @width 0.7
\fi

** 

Що перше я зроблю після перемоги? 

Мабуть, я зроблю собі вихідний.

Ольга Кириленко, УП
