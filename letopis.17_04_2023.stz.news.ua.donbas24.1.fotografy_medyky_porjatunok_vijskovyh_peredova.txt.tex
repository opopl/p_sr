% vim: keymap=russian-jcukenwin
%%beginhead 
 
%%file 17_04_2023.stz.news.ua.donbas24.1.fotografy_medyky_porjatunok_vijskovyh_peredova.txt
%%parent 17_04_2023.stz.news.ua.donbas24.1.fotografy_medyky_porjatunok_vijskovyh_peredova
 
%%url 
 
%%author_id 
%%date 
 
%%tags 
%%title 
 
%%endhead 

Тетяна Веремєєва
17_04_2023.tetjana_veremeeva.donbas24.fotografy_medyky_porjatunok_vijskovyh_peredova
Ukraine,War,УкраїнаВійна,УкраїнаВійнаПередова,date.17_04_2023

Фотографи показали, як медики ризикують життям заради порятунку військових на
передовій (ФОТО)

Костянтин та Влада Ліберови разом із військовими лікарями були на передовій

Повномасштабна війна в Україні триває другий рік. Поки християни східного
обряду в усьому світі святкують Великдень, росіяни обстрілюють територію нашої
країни. Захисники України кожного дня, за будь-яких умов ризикують власним
життям.

Фотографи Костянтин та Влада Ліберови показали, як лікарі медичної служби Ульф
працюють на передовій та рятують життя військових.

Читайте також: В Україну повернулися ще 130 захисників — подробиці Великоднього
обміну полоненими (ФОТО)

Вони розповідають, що військові медики відразу після команди по рації виїхали
на передову, щоб врятувати двох поранених воїнів, які перебували в сирому окопі
під обстрілом ворога.

Після важких зусиль та ризиків вони успішно дісталися до поранених воїнів, які
перебували в сирому окопі. У броньованій гусеничній машині їх перевезли на
стабілізаційний пункт, де їм було надано необхідну медичну допомогу.

Читайте також: Спецназ ГРУ втратив 90−95% в Україні — The Washington Post

Лікарі ретельно виконали кожну процедуру, забезпечуючи належне лікування
поранених. Хоча ситуація була дуже напруженою, вони робили все необхідне для
порятунку захисників нашої країни.

В результаті важких зусиль обидва воїни були врятовані. Медики продемонстрували
готовність ризикувати своїм життям, щоб врятувати інших.

Читайте також: Росіяни посилили наступ на Бахмут — британська розвідка

Раніше Донбас24 розповідав, як українські мінометники знищують росіян під
Бахмутом.

Ще більше новин та найактуальніша інформація про Донецьку та Луганську області
в нашому телеграм-каналі Донбас24.

Фото: Костянтин та Влада Ліберови
