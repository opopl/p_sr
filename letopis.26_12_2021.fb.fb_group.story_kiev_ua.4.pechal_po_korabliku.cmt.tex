% vim: keymap=russian-jcukenwin
%%beginhead 
 
%%file 26_12_2021.fb.fb_group.story_kiev_ua.4.pechal_po_korabliku.cmt
%%parent 26_12_2021.fb.fb_group.story_kiev_ua.4.pechal_po_korabliku
 
%%url 
 
%%author_id 
%%date 
 
%%tags 
%%title 
 
%%endhead 
\zzSecCmt

\begin{itemize} % {
\iusr{Вадим Горбов}
Спасибо. За передышку.

\iusr{Людмила Козак}
Очень сожалею за теми домиками. Такие уютные. Как внутри не знаю. Не приходилось побывать.

\iusr{Anatoliye Anatoliy}
Спасибо. Там на Бажова жила хорошая знакомая нашей семьи

\iusr{Александр Асатуров}
Бажова + БВС это уже просто Дарница

\begin{itemize} % {
\iusr{Галина Полякова}
\textbf{Александр Асатуров} Конечно, Дарница! Но для нас, старых киевлян, это еще и Соцгород. Хоть и без Кораблика.

\iusr{Александр Асатуров}
\textbf{Галина Полякова} ну я киянин молодой) но Кораблик помню. У вас хорошо получилось, спасибо!

\iusr{Нина Светличная}
\textbf{Александр Асатуров} Ні, Соцмістечко.

\iusr{Александр Асатуров}
\textbf{Нина Светличная} хай так
\end{itemize} % }

\iusr{Алла Парадня}

Не знаю, кто там высказывает недовольство, я с большим удовольствием и
интересом читаю Ваши миниатюры и даже боюсь пропустить что-то. В любом рассказе
всегда есть что-то от личности рассказчика, но меня это не смущает. Личность
устраивает.)

\iusr{Елена Корниенко}

Ну у нас в Дарнице есть ещё один такой райончик со старыми \enquote{немецкими} домиками.
И большими \enquote{сталинскими} домами.

И был когда-то гастроном ,который все называли тэмп. Не знаю почему, но был. И
молочный, где продавали в поллитровых бутылочках куфир, сливки по 51 коп., иногда
сметану в баночках, но в основном развесную, как и творог. Молоко бывало и в
литровых, потом в треугольничках. Ох, как давно.

\iusr{Дмитрий Даен}

Соцгород однообразен в отличие от Дорогожицкой и парка детской ЖД. Я не говорю
об осколках детства: Дмитриевской, Воровского, Сенном рынке, Львовской площади,
Артёма, Обсерваторной. Нельзя сваять за 60 лет продолжение
полуторатысячелетнего города. Особенно, если прожил в нем всю осознанную (67
лет из 70-ти) жизнь. A propos, хрущевки строились на Дорогожицкой, а 2-х и 3-х
этажные сталинки - на Грекова, Максима Берлинского, Щусева. Так что не только
Соцгород.

\begin{itemize} % {
\iusr{Галина Полякова}
\textbf{Дмитрий Даен} 

Есть ли смысл сравнивать жилой район с парком детской ЖД? Соцгород уникально
спланирован: детские садики, спрятанные подальше от проезжей части, уютные
скверики, тоже удаленные от улиц, просторные дворы со спортивными и детскими
площадками. Я ни в коей мере не умаляю достоинств названных Вами улиц. Но они
формировались веками. А соцгород был построен в одночасье и по конкретному
плану. В этом его особенность и прелесть. А \enquote{хрущовок} в Киеве полным-полно.
2-х этажных \enquote{сталинок} я, честно, говоря знаю очень немного. Это по большей
части все же немецкие постройки.

\begin{itemize} % {
\iusr{Дмитрий Даен}
\textbf{Галина Полякова} Не будем спорить. Каждому - своё. С наступающим и всего наилучшего!

\iusr{Natala Vashchuk}
\textbf{Galina Poliakova} 

это вы плохо знаете Сырец и район вокруг Щусева *вернее - справа и слева от
Щусева, 4-5 кварталов от Телиги к станции метро Сырец): двухэтажные домики,
дворы с погребами, детские сады внутри района, микорайоны там так и строились -
по одному плану практически в одночасье

\iusr{Галина Полякова}
\textbf{Natala Vashchuk} 

Даю Сырец я плохо знаю. Но я ведь ничего плохого ни об одном районе Киева не
сказала. Я написала о Кораблике!

\iusr{Natala Vashchuk}
\textbf{Galina Poliakova} 

не-не, я не в качестве обвинения  @igg{fbicon.face.rolling.eyes}  просто я вот Соцгород очень поверхностно
знаю, а Сырец хорошо, вы наборот, я написала только о том, что знаю @igg{fbicon.grin}  оценивать
районы, какой плохой, какой хороший и в голову бы не пришло. Я так думаю, во
многих районах есть одинаковые микрорайоны по времени застройки и типовым
планам

\iusr{Дмитрий Даен}
\textbf{Галина Полякова} 

Давайте закроем дискуссию. Всё я знаю хорошо. Уж точно не хуже Вас. Хочу
отметить, что на Дорогожицкой 13, нынче Парково-сырецкая в доме, где почта, я
прожил с 9-ти лет до 14-ти. Район строился на моих глазах. Впрочем я Вам ничего
не пытаюсь доказать. Ну не повезло мне полюбить Соцгород. Простите, бога ради.

\iusr{Галина Полякова}
\textbf{Natala Vashchuk} 

Вы, безусловно, правы. Но, полагаю, для того и существует эта группа: мы можем
познакомить друг друга с различными уголками Киева.

\iusr{Lola Madino}
\textbf{Дмитрий Даен} Дорогожицька так і є Дорогожицькою.
Парково-Стрецька - це колишня Шамрила. І пошта саме на цій вулиці, а Дорогожицька до неї перпендикулярна.

\iusr{Дмитрий Даен}
\textbf{Галина Полякова} У 1960 році там була Дорогожицька. Нема про що сперечатися.
\end{itemize} % }

\iusr{Ольга Морозова}
\textbf{Дмитрий Даен} 

Полутаратысялетния история состоит из шестидесятилетних историй, каждая из них
является продолжением. И последние 30 лет трагическое продолжние.

\iusr{Natala Vashchuk}
\textbf{Galina Poliakova} именно:)

\iusr{Нина Светличная}
\textbf{Дмитрий Даен} 

Вы ошибаетесь, Соцгород очень разнообразен касательно архитектуры. Здесь
сочетание классических хрущевок и хрущевок улучшенных - с большимы коридорами,
8-ми метровыми кухнями. Здесь сочетаються \enquote{сталинки} и дома по спецпроектам,
которые не бросаются в глаза - снаружи обычная 9-ти этажка, а планировка -
ого-го! Ну, и сразу послевоенные дома по совершенно необычным проектам. И,
конечно же, высотки \enquote{от сан саныча} на украденных стадионах.

\begin{itemize} % {
\iusr{Дмитрий Даен}
\textbf{Нина Светличная} 

Ничего я не ошибаюсь. 40 лет назад - пустынная улица Строителей с хилыми
деревцами. Архитектура... Также как проспект Мира, улица Гагарина и
Ленинградская площадь. Что может уютнее однообразия?


\iusr{Галина Полякова}
\textbf{Дмитрий Даен} 

А Вы сверните с улицы в дворы. И увидите, что задумано было очень неплохо.
Много сквериков, детских и спортивных площадок, много зелени... Конечно,
архитектурных изысков Вы не найдете, Разве что остатки фонтанов во дворах
\enquote{сталинок}.

\iusr{Нина Светличная}
\textbf{Дмитрий Даен} Ну ви і порівняли!
\end{itemize} % }

\end{itemize} % }

\iusr{Наташа Семечева}

\ifcmt
  ig https://i2.paste.pics/e9cd4f9ab2f7603492285c7cf674084d.png
  @width 0.2
\fi

\iusr{Нина Светличная}

Мы жили рядом - Бажова, 1, бывший ведомственный дом. Поэтому Кораблик был совсем
рядом. Жаль, что уничтожили, кому мешал(

\begin{itemize} % {
\iusr{Галина Полякова}
\textbf{Нина Светличная} 

Время меняет многое. Но мне кажется, что новые хозяева магазина могли бы
сохранить сам кораблик, даже видоизменив его. От этого они бы только выиграли.

\end{itemize} % }

\iusr{Наташа Семечева}

В Соцгород попала после жизни на метро \enquote{Арсенальная}, с немеренным количество
машин. Просмотрела много квартир по городу и вдруг предложили посмотреть
квартиру на Красноткацой, квартира понравилась, но хозяева нашли размен. Но мне
уже понравился район, метро рядом, до центра 15 минут. И квартира на Бажова
неплохая в сталинке. Живу тут около тридцати лет, о Печерске не жалею, и
однообразности не ощущаю, Рынок, новые торговые центры, магазины район
становится современным, да и кино постоянно снимают

\iusr{Ольга Глуздань}

\ifcmt
  ig https://i2.paste.pics/2c0ca05ee4c0fc38ec344ceee0efd6d3.png
  @width 0.2
\fi

\iusr{Мария Кар}

Я родилась на Соцгороде, на ул. Краковской, в 5ти этажной сталинке. Одни окна
выходили на Краковскую, вторые - во двор - отличный уютный двор, с площадкой,
деревьями, стадионом, на котором зимой заливался каток. И гастрономы назывались
кодами - например, 33ий, сейчас уже и не вспомню. Парочка была на Строителей,
на Краковской, и один вон тот, на Бажова. Намуглу Краковской и б-ра Труда были
обувный магазин и фотоателье, а во дворах - Галантерея. Множество магазинов
было по Попудренко. Кстати, в фотоателье сейчас частный загс, эта деталь меня
очень порадовала, фотографии там получаются красивые и светлые.


\iusr{Олена Голяницька}

Директор Галантереи жила в моем подъезде, в доме рядом с этим магазином.
Магазин-№28 располагался на углу Краковской и Строителей, №32 на углу
Строителей и Попудренка, ТЭЦевский на Строителей, ближе к магазину Тканей и 14
гастроном..-пересечение ул. Красноткацкой и ул. Бажова. А еще популярный
Молочный на Строителей и рядом аптека. В аптеке большая банка с пиявками, мы
детьми бегали на них посмотреть. Давно там не живу, однако пару раз видела свой
дом в нескольких фильмах-сериалах. \enquote{Беги, не оглядывайся} и \enquote{Ничего не
случается дважды}.

\begin{itemize} % {
\iusr{Мария Кар}
\textbf{Олена Голяницька} 

точно, 32 и 28 Гастрономы. А Галантерея была во дворе одного из домов на углу
Строителей, она была на крыльце с лестницей.


\iusr{Олена Голяницька}
\textbf{Мария Кар} 

..да, на крыльце... я родилась и жила в доме рядом.. оранжевый такой с шпилем на
крыше... замечательный такой дом был, двор дружный, детей много, лавочки,
старушки каждую весну на газонах цветики сажали, зимой в большом дворе дворник
нагребал огромные сугробы, мы по этим сугробам прыгали, устраивали снежные
крепости, все друг друга знали, друг другу помогали... замечательное детство...

\end{itemize} % }

\iusr{Denis Kostjuk}

Откуда у всхех такое устойчивое мнение что \enquote{аварийный} строили пленные немцы?
Читайте историю кто конкретно и как строил. А не -\enquote{одна баба казала}.

\end{itemize} % }
