%%beginhead 
 
%%file 11_02_2022.fb.loskutova_natalia.mariupol.1.uchast_vikladach_v_m
%%parent 11_02_2022
 
%%url https://www.facebook.com/permalink.php?story_fbid=pfbid02gNzqSLNtqdJyLSfMiS8b1jSUqXgnkiktgtYavJqFJPxfcEL9ZFC6umc5XrszronFl&id=1427894275
 
%%author_id loskutova_natalia.mariupol
%%date 11_02_2022
 
%%tags mariupol,francia
%%title Участь викладачів МДУ у роботі французької делегації
 
%%endhead 

\subsection{Участь викладачів МДУ у роботі французької делегації}
\label{sec:11_02_2022.fb.loskutova_natalia.mariupol.1.uchast_vikladach_v_m}

\Purl{https://www.facebook.com/permalink.php?story_fbid=pfbid02gNzqSLNtqdJyLSfMiS8b1jSUqXgnkiktgtYavJqFJPxfcEL9ZFC6umc5XrszronFl&id=1427894275}
\ifcmt
 author_begin
   author_id loskutova_natalia.mariupol
 author_end
\fi

Участь викладачів МДУ у роботі французької делегації

10 лютого 2022 відбувся черговий візит президента інжинірингової компанії Beten
Жана Роша до Маріуполя. Метою візиту було підписання меморандуму з головою
Мангушської селищної громади Володимиром Караберовим стосовно облаштування
пляжів, за прикладом Маріуполя. Окрім цього відбулася зустріч пана Роша з
головою Бердянської райдержадміністрації Олексієм Бакаєм. Сторони обговорювали
питання розвитку водозабезпечення населення якісною питною водою і розвитку
зрошення в Запорізькій та Донецькій областях. Обговорення таких важливих
проблем неможливо без участі перекладача, тому для перекладу традиційно
запросили доцентку кафедри німецької та французької філології МДУ Лоскутову
Наталію. Завдяки допомозі у спілкуванні сторони досягли порозуміння. Наступний
візит французької делегації заплановано на березень, а це означає, що у
викладачів МДУ знову з'явиться можливість підвищити свій фаховий рівень знання
французької мови.
