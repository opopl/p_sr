% vim: keymap=russian-jcukenwin
%%beginhead 
 
%%file 27_01_2021.fb.fb_group.story_kiev_ua.1.osvobozhdenie_kieva.cmt
%%parent 27_01_2021.fb.fb_group.story_kiev_ua.1.osvobozhdenie_kieva
 
%%url 
 
%%author_id 
%%date 
 
%%tags 
%%title 
 
%%endhead 
\zzSecCmt

\begin{itemize} % {
\iusr{Елена Мирошниченко}
Плачу.

\begin{itemize} % {
\iusr{Мария Каменцова}
\textbf{Елена Мирошниченко} . 

Я сама плакала, когда писала... И мама, когда мне рассказывала. Я постаралась
воспроизвести ее рассказ с фотографической точностью. Все детали...


\iusr{Лариса Заболотная}
\textbf{Елена Мирошниченко} и я...

\end{itemize} % }

\iusr{Александр Лесик}
кошмар. моей бабке повезло она в каком то другом Киеве оккупацию пережила.

\begin{itemize} % {
\iusr{Мария Каменцова}
\textbf{Александр Лесик} . Да жить в оккупацию, мама рассказывала, можно было... Только когда немцев стали выбивать из города, они начали свирепствовать!

\begin{itemize} % {
\iusr{Александр Лесик}
\textbf{Мария Каменцова} 

а моя говорила что тихо ушли. за пару дней до освобождения. просто вдруг много
бродячих собак появилось. (немцы собак беспризорных отстреливали). не стало
немцев. пришли собаки. а потом коммунисты начали Киев обстреливать в котором
кроме жителей не было уже никого (но это понятно они ж предатели не
самоубились). а еще говорила - когда советские Киев сдавали то повзрывали все
эшелоны с зерном на Дарницком вокзале. чтоб не досталось никому. а фашисты когда
бежали-склады с едой (которую не успели вывезти) не взрывали. наверно взрывчатку
пожалели.

\iusr{Мария Каменцова}
\textbf{Александр Лесик} . 

Я так понимаю, что немцы тихо ушли, но до этого пытались уничтожить то и тех,
кого могли. Передаю просто рассказ моей мамы. Она девочкой была, рассказала то,
что сама видела. Может быть, другие видели и знали больше!

\iusr{Александр Лесик}
\textbf{Мария Каменцова} наверно.у мамы медали были ?

\iusr{Мария Каменцова}
\textbf{Александр Лесик} . Ей было 14 лет. Медали у нее были, но не боевые, юбилейные, и довольно много! Они сохранены.

\iusr{Александр Лесик}
\textbf{Мария Каменцова} 

ага у всех так. и дед и бабушка вся грудь в медалях. как начал вникать оказалось
все юбилейные. 20 лет победы 30. и т д. даже орден вов. и тот юбилейный. а за
реальные боевые достижения дед ..благодарность.. от товарища Сталина.. (письмецо
на тетрадном листе написанное) получил. ъ

\iusr{Tatiana Thoene}
\textbf{Александр Лесик} 

моя мама тоже вспоминала, что немцы ушли быстро и вроде «тихо». У них во дворе,
на Коминтерна, была немецкая полевая кухня. Когда наши стали приближаться к
Киеву, собрали манатки и пошли строем к вокзалу. Наверное, не везде
свирепствовали и кому-то повезло.

\end{itemize} % }

\end{itemize} % }

\iusr{Aleksandra Gorovaya}
Огромное спасибо! Слов нет - только слезы.

\begin{itemize} % {
\iusr{Мария Каменцова}
\textbf{Aleksandra Gorovaya} . Я, пока писала (вначале на компьютере), тоже целый день плакала...

\iusr{Мария Каменцова}
\textbf{Aleksandra Gorovaya} . Мама и папа, оба пережившие оккупацию в Киеве, рассказывали много, я здесь еще покажу! Некоторые рассказы прямо необыкновенные! Хотя они почти дети еще были, 14 лет...

\begin{itemize} % {
\iusr{Aleksandra Gorovaya}
\textbf{Мария Каменцова} Мои смогли уехать из Киева, хотя соседки по коммунальной квартиры, уговаривали бабушку остаться. (Как потом выяснилось , дворник выдал немцам всех евреев в доме). Школьная подруга моей матери, которая осталась со своей мамой в оккупированом Киеве, рассказывала, что немцы разрешили учиться в школе только детям, которые занимались балетом.
\end{itemize} % }

\end{itemize} % }

\iusr{Тома Храповицкая}
Спасибо... больше нет слов, Спасибо Вам и Вашей Маме за такие тяжёлые но нужные всем Нам воспоминания!!!!!!!!!

\iusr{Мария Каменцова}
\textbf{Тома Захарова} . Без памяти нет Родины.

\iusr{Ирина Иванова}

Машуня, до слез!!!! И какой же нежной, чистой как первая снежинка на фото твоя
мама. По фото не скажешь, что у этой девочки столько силы..... .Светлая,
добрая память живет во мне о тете Нине

\begin{itemize} % {
\iusr{Мария Каменцова}
\textbf{Ирина Иванова} . Вот она и побыла с нами как будто воочию... Вечная память! Рассказов как ее, так и папы множество. Я тут еще покажу...

\iusr{Оксана Тернавская}
\textbf{Ирина Иванова} мы все помним и её, и всех наших близких с ул.Герцена. Тем более, жизнь так связала наши судьбы.
\end{itemize} % }

\iusr{людмила петрухина}

Моя бабушка тоже пережила этот ужас, пряталась в Голосеевском лесу со своей
подругой, тоже рассказывала об освобождении Киева, о том, что кругом были одни
развалины, сколько же им пришлось пережить.

\iusr{Мария Каменцова}
\textbf{людмила петрухина} . Мы обязаны об этом помнить и рассказывать. Чтобы не повторилось...

\iusr{Giorgi Begeluri}
Kakaia krasivaiajenshchina!

\begin{itemize} % {
\iusr{Мария Каменцова}
\textbf{Giorgi Begeluri} . Потом она немного снималась в кино, ее пригласили... Стала еще красивее!

\iusr{Giorgi Begeluri}
\textbf{Мария Каменцова} ona vasha mama?

\iusr{Мария Каменцова}
\textbf{Giorgi Begeluri} . Да, конечно!  @igg{fbicon.smile} 
\end{itemize} % }

\iusr{Виктория Аксененко}

Мой дедушка был военный и успел вывезти бабушку с моей мамой и тетей. А как
только освободили Киев они вернулись в свою квартиру на ул. Чкалова. Мама тоже
много расказывала уже про освобожденный Киев. Страшное время было..... Мы должны
знать и помнить те года.....


\iusr{Татьяна Чернышева}
Мама такая красивая@igg{fbicon.heart.red}

\iusr{Мария Каменцова}
\textbf{Татьяна Чернышева} . Спасибо! Я еще буду рассказывать, покажу ее фото постарше, когда расцвела – она была прямо кинозвезда!

\iusr{Marina Chernaya}
Как ценны воспоминания очевидцев, когда так нещадно переписывают историю!

\begin{itemize} % {
\iusr{Мария Каменцова}
\textbf{Marina Chernaya} . Критерий всего – ИСТИНА.

\iusr{Marina Chernaya}
\textbf{Мария Каменцова} 

истину знает только Бог, порядочные историки, используя научные методы, вне
всякой политики и пропаганды, пытаются к ней приблизиться. В истории всех
народов и стран были светлые и темные страницы, нельзя этим манипулировать. Как
сказал А. Мень, правда одна, остальное - ложь.

\end{itemize} % }

\iusr{Елена Свец Баркан}

Моя бабушка с двумя детьми 36 и 41 года рождения (мой дядя и мама) была все
время оккупации в Киеве. Не могу даже себе представить, каково ей было

\begin{itemize} % {
\iusr{Александр Лесик}
\textbf{Елена Свец Баркан} может надо было спросить?

\begin{itemize} % {
\iusr{Елена Свец Баркан}
\textbf{Александр Лесик} 

кое что бабушка рассказывала, но она умерла, когда мне было только 16. Что
успела - рассказала. Как ходила по селам под Киевом менять вещи на продукты.
Когда проходила через посты если стояли немцы, то проверяли документы и
пропускали, а если были полицаи, то часто забирали вещи или продукты.

\iusr{Александр Лесик}
\textbf{Елена Свец Баркан} ага. а энкаведисты с 1918 до 39 просто стреляли и даже не закапывали

\iusr{Ольга Жук}
\textbf{Александр Лесик} вас, лично стреляли?
\end{itemize} % }

\iusr{Лариса Заболотная}
\textbf{Елена Свец Баркан} и мои... Бабушка 1905 года, мама 1929...

\iusr{Наталия Погорелова}

Моя бабушка 1908 года с двумя дочками 1928 и 1933 тоже остались в оккупации.
Жили на Сталинке в подвале деревянного домишки на 4 семьи. Мама очень много
рассказывала о том времени...

Перед освобождением Киева немецкий офицер, занимавший в их доме пустую
квартиру, предупредил, что город будет взорван и чтоб уходили поскорее. Мамина
семья с соседями, со своим каким-то скарбом, ушли в сторону Мышеловки. Ночевали
и дневали (в ноябре!) под открытым небом. Но ночами мама с её тетей (1920 г.р.,
8 лет разницы всего, так что были практически подругами) ходили в Киев
покормить лежачую бабушку тетиного жениха (он воевал, а тётя за ней
присматривала)...

Тетю эту, кстати, отправляли в начале оккупации в Германию, но она как-то
умудрилась сбежать - выпрыгнула из вагона и вернулась в Киев.

\iusr{Victor Rozhkov}

Мой отец тоже был в оккупации. Но такого как расстрел мирных не было. Наоборот
немцы уводили людей чтоб они не пострадали. А хуже всех в окупации были полицаи
из наших. Это слова отца. И он ещё жив.

\end{itemize} % }

\iusr{Тетяна Бондаренко}
Читала про саме цей момент в одній книзі про Київ в окупації!

\iusr{איליאנה ז׳וקוב}
 @igg{fbicon.face.downcast.sweat}  @igg{fbicon.flag.izrail}

\iusr{Lola Madino}

Щось я чула, та й документи свідчать - німці ВИСЕЛЯЛИ мирне населення. А не
вбивали... Так, за непідкорення наказу - розстріл. Але можна було скоритись
наказу та піти.

\begin{itemize} % {
\iusr{Тетяна Бондаренко}
\textbf{Lola Madino} Піти куди?
А хто не міг?

\iusr{Наталия Погорелова}
\textbf{Тетяна Бондаренко} хто не міг через хворобу, наприклад - залишався. Мама розповідала мені.

\iusr{Юлия Ильенко}
А про Бабий Яр вы чулы?

\iusr{Lola Madino}
\textbf{Julia Ilienko} а ви? Що саме чули?

\iusr{Оксана Волкова}
\textbf{Lola Madino} в Бабий яр тоже \enquote{добровольно} отправляли @igg{fbicon.cry} 

\begin{itemize} % {
\iusr{Lola Madino}
\textbf{Oksana Volkova} 

Так, сусіди, літня єврейська пара - пішли. \enquote{Добровільно} Хоч знали, що на
смерть. Їх умовляли - \enquote{не може бути, це ж культурна нація, вас вивезуть} - вони
сказали \enquote{ні, ми знаємо, нас уб'ють... Коли повернуться сини (з Червоної армії)
- скажіть, де наші кісточки}

Але сини не повернулись.
\end{itemize} % }

\iusr{Юлия Ильенко}

\ifcmt
  ig https://scontent-frt3-1.xx.fbcdn.net/v/t1.6435-9/142419677_3602589599859654_11061196312617109_n.jpg?_nc_cat=102&ccb=1-5&_nc_sid=dbeb18&_nc_ohc=diCpdINf764AX8MwH_W&_nc_ht=scontent-frt3-1.xx&oh=00_AT_ibODpXN11YVSJO4gLVrJrMC0nQlfw2wQMp150ux9ghw&oe=6211A5CB
  @width 0.2
\fi

\iusr{Юлия Ильенко}
А отсюда мой мирный дед с мирным другом успели сбежать

\ifcmt
  ig https://scontent-frt3-1.xx.fbcdn.net/v/t1.6435-9/142566451_3602590373192910_4530757805676079825_n.jpg?_nc_cat=107&ccb=1-5&_nc_sid=dbeb18&_nc_ohc=CBHwWoi3WcAAX-W7o53&_nc_ht=scontent-frt3-1.xx&oh=00_AT8arrKZezOxD-Zd9xUHdzu8VmMS4Gi27m-A3SQPs-bbTA&oe=620F9842
  @width 0.2
\fi

\begin{itemize} % {
\iusr{Владислав Проноза}
\textbf{Юлия Ильенко} это фото сделано вашим дедушкой? Он бежал и фотографировал?
\end{itemize} % }

\iusr{Lola Madino}
Мої батьки пережили окупацію обоє в Києві.

Коли німці оголосили наказ перед наступом Червоної армії у листопаді 43-го -
\enquote{для уникнення зайвих жертв серед населення звільнити зону військових дій} -
взяли, що змогли та пішли. У Мотовилівку наприклад.

\begin{itemize} % {
\iusr{Мария Каменцова}
\textbf{Lola Madino} . Ангелы кротости. Что-то люди от них прятались по ярам, лесам и канализационным трубам...  @igg{fbicon.wink} 

\iusr{Lola Madino}
\textbf{Maria Kamentsova} 

\enquote{тактику випаленої землі} застосували спочатку совєти, відходячи з
Києва. Наприклад спалили запаси продовольства а не роздали людям, яких залишали
напризволяще.

Німці були ворогами, це без сумніву.

Але \enquote{вбити всіх жителів} - це ви загнули...


\iusr{Оксана Волкова}
\textbf{Lola Madino} в предыдущем комменте было \enquote{щось чула} , теперь уже родители мирно ушли из зоны боевых действий  @igg{fbicon.thinking.face} 

\iusr{Lola Madino}
\textbf{Oksana Volkova} 

Так, вони виконали наказ, за невиконання якого загрожував розстріл. І від них я
це чула. Не \enquote{ВСІМ ЖИТЕЛЯМ}, а тим, хто порушував наказ.

А ще я читала документи. Чого і вам бажаю. А не \enquote{савєцкую історію вєлікай
атєчествєннай}

\iusr{Оксана Волкова}
\textbf{Lola Madino} 

\enquote{советская история Великой отечественной} написана кровью моей семьи, моих
дедов и бабушек, которые воевали с первого до последнего дня войны!

\iusr{Lola Madino}
\textbf{Oksana Volkova} менше патетитки. Більше документів.

\iusr{Оксана Волкова}
\textbf{Lola Madino} начинайте выкладывать пруфы. Только не книжечки, а реальные доки.

\iusr{Ирина Попова}
\textbf{Lola Madino} бот без фото и истории.

\iusr{Игорь П.}
\textbf{Lola Madino} ссылку на документы можно? Только вову вятровича не предлагать.
\end{itemize} % }

\iusr{Мария Каменцова}
\textbf{Lola Madino} . Щось я чула – это не свидетельство очевидца, а сплетни и домыслы.

\begin{itemize} % {
\iusr{Lola Madino}
\textbf{Maria Kamentsova} мені лінь шукати документи для вас, спробуйте самі.
Або почитайте Малакова. Може - допоможе.

\iusr{Tatiana Thoene}
\textbf{Maria Kamentsova} да бросьте, здесь полно коментов - пересказы воспоминаний. Не все претендуют на историческую достоверность.
\end{itemize} % }

\iusr{Юлия Ильенко}
\textbf{Lola Madino}, то, что вашим родственникам повезло выжить, не даёт вам право плевать на тысячи убитых немцами родственников других людей

\iusr{Lola Madino}
\textbf{Julia Ilienko} можете вказати пальцем - де я на когось плюнула?

\iusr{Наталия Погорелова}
\textbf{Lola Madino} 

мама моя теж казала, та й їх сім'ї німецький офіцер, що мешкав у сусідній
квартирі (сусіди встигли виїхати вчасно), порадив швиденько забратися з Києва.
То й зібрали, що могли, й пішли у Мишоловку. І їх сусіди всі з вулиці - теж. Не
розстрілювали німці, а видали наказ звільнити мешканцям місто, бо все буде
знищено.


\iusr{Юлия Ильенко}
\textbf{Lola Madino}, а почему вы упорно избегаете темы Бабьего Яра? И как эта тема сочетается с вашим утверждением, что нiмцi не вбивали?

\iusr{Людмила Новікова}

Так, медсестра літнього віку, що працювала в лікарні номер 1, в якій я лежала,
що навпроти Володимирського собору, (Тамарою її звали), розповідала, що їх німці
виселили з Ірпеня, бо там проводили бойові дії, в Мироцьке

\end{itemize} % }

\iusr{Nata Pylypenko Shmorgun}

Взрослые дети войны.. до слёз и мурашек, это ж какой ужас и страх был и все
вынесли!!! У меня вдруг возникла ассоциация с детками нынешней нашей войны,
дети которые видели и слышали обстрелы, взрывы и многие дети сироты, какой
страх и ужас пережили и они и ведь также будут рассказывать своим детям и
внукам.... война, оккупация всегда страшно, и ценна каждая минута с близкими! И
важно помнить!

\begin{itemize} % {
\iusr{Ольга Авраменко}
\textbf{Nata Pylypenko Shmorgun} Вы правы. Даже в самом страшном сне не могло присниться, что русские будут воевать с украинцами.

\iusr{Sokolova Oksana}
\textbf{Ольга Авраменко} то есть история ни чему не учит?) Столетиями воевали, а тут вдруг не могло приснится.
Не было и вот опять.

\iusr{Оксана Тернавская}
\textbf{Sokolova Oksana} кто с кем воевал?

\iusr{Арт Юрковская}
\textbf{Sokolova Oksana} украинцы с поляками воевали.

\iusr{Sokolova Oksana}
\textbf{Art Yurkovska} воевали. Но не один львовский музей поляки не ограбили.

\iusr{Sokolova Oksana}
Дамы я вас оставлю при вашем мнение, а себя при своем.
Вы можете опять про единую страну, а я порадуюсь что этой страны больше никогда не будет.
И если вам не нравится жить в Украине, то чемодан, вокзал и знакомое направление.
\end{itemize} % }

\iusr{Тетяна Бондаренко}

А ще, в родині мого зятя був такий життєвий момент: чоловік, військовий в
чинах, покинув свою дружину з маленьким сином через те, що вони були в окупації
в Ірпіні. Ось так!

Батьком для хлопця стала зовсім інша людина, яка і дала йому своє прізвище!

\iusr{Надежда Василенко}
\textbf{Тетяна Бондаренко} это был подонок, который просто нашел предлог. Не было бы окупации, он бы просто так ушел. Мало ли таких, которые после войны поменяли жен

\iusr{Лариса Чепурна}
Рассказ похож на какой то бред и кич

\begin{itemize} % {
\iusr{Александр Лесик}
\textbf{Лариса Чепурна} пустая страница с полупустой спорит.полк победосесов проснулся

\iusr{Оксана Тернавская}
\textbf{Александр Лесик} страница закрыта для таких, как вы

\iusr{Александр Лесик}
\textbf{Оксана Тернавская} ваша открыта)) вы искренний совкодрот победобес

\iusr{Оксана Тернавская}
\textbf{Александр Лесик} 

мне на Ваше мнение искренне наплевать. Во времена СССР я искренне его
критиковала и, кстати, не имела с этим никаких проблем. И в то, и в нынешнее
время, я жила хорошо, но всё это благодаря моим родителям, моим способностям,
и, безусловно, обстоятельствам. Никогда не буду охаивать своё прошлое, так как
без него нет будущего. А вот за Победу глотку перегрызу!!! Хорошего Вам дня!

\iusr{איליאנה ז׳וקוב}
\textbf{Лариса Чепурна} не то написала не в тот день .

\iusr{Таня Краснова}
\textbf{Лариса Чепурна} Лариса, это ты ещё про куклу цыганку «мыло» не читала @igg{fbicon.laugh.rolling.floor} 

\iusr{Мария Каменцова}
\textbf{Таня Краснова} . Из таких, как вы, немцы как раз и варили мыло. После удушения в газовых камерах.
\end{itemize} % }

\iusr{Ефросиния Батурина}
Такая Красивая !

\iusr{Мария Каменцова}
\textbf{Ефросиния Батурина} . Спасибо!  @igg{fbicon.smile} 

\iusr{Lubov Vladimirovna}

Моя свекровь Панченко София Филипповна освобождала Киев. Зенитные войска.
Святая была женщина!!! Много рассказывала... страшные были годы...светлая ей
память! На фото справа. Награждена орденом Красной звезды. Орден храним как
реликвию.

\ifcmt
  ig https://scontent-frx5-1.xx.fbcdn.net/v/t1.6435-9/142832175_418898539374085_6058853585748773206_n.jpg?_nc_cat=110&ccb=1-5&_nc_sid=dbeb18&_nc_ohc=Dun_vjzbOiwAX8phGDI&_nc_ht=scontent-frx5-1.xx&oh=00_AT_REmr7nl0idPoNO44AWyv6TNsoXgBvGlFDvWnHtywkDg&oe=620F69E9
  @width 0.2
\fi

\begin{itemize} % {
\iusr{Ирина Владиславовна Остапчук}
\textbf{Lubov Vladimirovna} доземний уклін Вашій свекрусі!

\iusr{Lubov Vladimirovna}
\textbf{Ирина Владиславовна Остапчук} Дуже вдячна Вам за такі слова! Які вони були молоді дівчатка і скільки випало на їхню долю... але це вже інша історія...
\end{itemize} % }

\iusr{Инна Игоревна}
Читаешь, аж мурашки по телу бегут и слезы наворачиваются на глазах.

\iusr{Мария Каменцова}
\textbf{Инна Игоревна} . 

Я, когда вспоминала рассказ мамы и писала его на компьютере, прежде чем тут
разместить, тоже весь день плакала... Сколько им пришлось пережить!!!

\iusr{Ирина Горобец}
Сколько же им пришлось пережить! Благодарю, что делитесь!

\iusr{Лариса Волошина}

Моя мама пережила оккупацию под Киевом. Кошмар... Когда началась война, ей было
5 лет. Отступление наших помнит эпизодически, но помнит. Чем дальше, тем больше
воспоминаний...

Ещё при наступлении немцев их семью пытались расстрелять. Бабушка была
верующая, а дед - ярый атеист. Для немцев - атеист значит коммунист. Кто-то
донёс. Немцы всю семью - бабушка, дед, четверо детей (от 1 года до 7),
поставили к стенке (к печке), собирались расстрелять... Но кто-то увидел,
уговорил подождать старосту, которого нашли и привели... Стояли так, может час,
может два, может больше... Староста помог. В общем отстояли их. ..

Но самое страшное, говорит мама, это - голод. Постоянный, никуда не исчезающий
голод... Во время налетов, это уже 43-й год, отступление немцев, они прятались
в погребе. Но как только самолёты улетали, дети, 6, 7, 8.....лет, сразу же
бежали к воронкам от бомб, надеясь в сумках, карманах убитых найти хоть что-то
съедобное, корочку, крошки... УЖАС...

Жили в непосредственной близи от железной дороги. Пришлось видеть взорванные
партизанами немецкие поезда - нечеловеческие крики, кровь, изувеченные тела,
части тел.... Мама не может простить немцев до сих пор, не может слышать
немецкую речь, но картина человеческого месива, пусть и врагов, потрясла её на
всю жизнь.

Отступление немцев, наступление наших, это же не единовременный момент, помнит
хорошо...

Налеты, бои, бесконечное количество убитых (хотя это правый берег Киева, здесь
наступление с форсированием Днепра все же не сравнить)...

Много чего было, сразу не перессказать...

Мама забыть весь этот ужас не может...

После войны тоже было не легко, ох как нелегко, но с войной, оккупацией не
сравнить. Так говорит мама.

Не хочу провоцировать политические споры, но не сказать не могу, для неё День
победы - это праздник, самый большой праздник.

Не буду продолжать...

\begin{itemize} % {
\iusr{Мария Каменцова}
\textbf{Лариса Волошина} . 

Я не ставила себе задачу – осветить ВСЁ освобождение Киева. У меня – камерный
рассказ только одного отдельного человека, непосредственные впечатления
14-летней девочки. Мне представилось, это ценно... и в самый раз для группы
Киевские истории, это истории О НАС, киевлянах!

\begin{itemize} % {
\iusr{Лариса Волошина}
\textbf{Мария Каменцова} 

Спасибо за Ваш рассказ. Очень проникновенный и настоящий. У меня тоже возникло
желание поделиться воспоминаниями, воспоминаниями, которые за пределы семьи не
выносились. Наверное пришло время, спасибо Вам...

\end{itemize} % }

\iusr{Чорна Галина}
\textbf{Лариса Волошина}
Душевно, эмоционально написали о войне...

\iusr{Лидия Суворовская}
\textbf{Лариса Волошина} Ее можно понять - день Победы стал для нее днем надежды на жизнь без голода и ежеминутного страха смерти

\begin{itemize} % {
\iusr{Алла Тихонова}
\textbf{Лидия Суворовская}, \enquote{для неё} только?! И для меня тоже праздник. Для полицаев и фашистов - нет.
А для Вас?

\iusr{Светлана Ковтун}
\textbf{Алла Тихонова} 

для меня день победы скорее день скорби, чем праздник у меня оба деда погибли
на войне, а бобкука с новорожденной мамой пережили окупацию Киева, война это
ужас, но не человеческое отношение к своим гражданам только в СССР было
,поэтому не вижу я чем особо гордится - сколько мирных жителей, сколько содат мы
потеряли, скольких победителей после войны в лагеря сослали, все не так
однозначно..

\iusr{Rimma Turovskaya}
\textbf{Светлана Ковтун} Что же Вы во всем ищете только плохое?

\iusr{Светлана Ковтун}
\textbf{Rimma Turovskaya} почему плохое?, просто пытаюсь смотреть без розовых очков..
\end{itemize} % }

\iusr{Ирина Попова}
Страннно....вы даже как то извиняетесь за то, что для вашей мамы День Победы -
Праздник?

\end{itemize} % }

\iusr{Зоя Майстренко}

Мои мама и бабушка тоже пережили оккупацию, в их дом попала бомба, они выбежали
с него в последние минуты. Бабушка осталась с двумя детьми, дедушка погиб при
обороне Киева.

\iusr{Ольга Васильевна}

Мою маму освобождали американцы под Кельном в небольшом городке! Когда она
рассказывала о своих чувствах, я рыдала! Попав в то рабство и не имея никакой
информации о ситуации, они просто старались выжить. А с падением Сталинграда
заметили удрученность немцев, у многих появились траурные ленты, после этого
ждали освобождения!

\iusr{Лариса Маковецкая}

Моего отца в 14 лет во время облавы, возле памятника Богдана Хмельницкого,
немцы забрали на работу в Германию, моя бабушка даже не знала где он и что с
ним, но вот каким то чудом она через год получила открытку \enquote{я живой, только
очень хочется жрать} эта открытка была вся в слезах радости и горя.

\iusr{Wiktoria Arto}

Какой необычный рассказ.. очень интересный, но я никогда не слышала, о том,
моменте, когда немцы отходили, что люди прятались.. и вот такое потрясающее
воспоминание, как люди выживали. Это же описала свидетель событий, тогда
девочка, а потом пожилая женщина, ведь помнила эти волнующие моменты всю свою
жизнь! Спасибо большое @igg{fbicon.heart.red}

\begin{itemize} % {
\iusr{Мария Каменцова}
\textbf{Wiktoria Arto} . 

Прятались, потому что немцы расстреливали всех оставшихся. Советской армии не
должно было достаться НИЧЕГО. Выжженная земля и мертвые жители! Получите вашу
победу!!! Но не вышло... Люди выжили и город остался. Мама рассказывала – после
освобождения жители сами, без принуждения и приглашения, каждый день ходили на
Крещатик разбирать завалы, оставшиеся от разрушенных зданий.

\begin{itemize} % {
\iusr{Наталия Погорелова}
\textbf{Мария Каменцова} и моя мама ходила разбирать завалы Крещатика... Тоже говорила, что люди сами шли.

\iusr{Алекс Вельт}
\textbf{Мария Каменцова} 

Вы бредите ?? Киев был объявлен Кампфзоной - моя бабушка помнила это. Зашли
двое солдат - сказали \enquote{5 минут - пошел вокзал- шнель}. Они думали что их будут
убивать и плакали. У бабушки двое детей было и сестра с ними ее. Собрали всех в
колону и привели на Пост-Волынский. Там всех в вагонах вывезли в Ирпень и
Бородянку. Вернулись они через три дня пешком все домой. В городе были уже
советы. Никто никого не стрелял. Только голодные были страшно. Все соседи
вернулись живыми.

\iusr{Алла Тихонова}
\textbf{Алекс Вельт} , может это для полицаев \enquote{шнель-вокзал}?
И потом, почему \enquote{пошел вокзал}? Зачем Бородянка?
И все здоровы, никто не бредит. Если Вы не умеете общаться вежливо, то лучше не вступать в разговор вообще. Ферштейн?

\iusr{Мария Каменцова}
\textbf{Алекс Вельт} . Их просто не успели расстрелять, была неразбериха при оставлении города. Повезло! Я за них рада!
\end{itemize} % }

\end{itemize} % }

\iusr{Ирина Архипович}
До слёз!!.. Мои тоже пережили оккупацию!!!!....

\iusr{Наталия Багинская}

Моя мама родилась в Киеве в 1933 году (голодомор), всю войну с бабушкой в Киеве
находились, прятались по ярам. Много чего рассказывали, голод сильный, на Подоле на
Красной площади сидели семьи умершие от голода. Слышали каждый выстрел в Бабьем
яру, они жили на Куреневке. Мкиа всю жизнь была подвержена этим страхам. Ей было 8
лет когда началась война.

\begin{itemize} % {
\iusr{Мария Каменцова}
\textbf{Наталия Багинская} . Жуть... Вот такого мама мне не рассказывала!

\iusr{Наталия Багинская}
\textbf{Мария Каменцова} Рассказывали, когда начался голод, были такие, которые ели людей, это наши украинцы, немцы сами их расстреляли.

\iusr{Наталия Багинская}

Ходили на обмен по селам. Мама болела тифом, 40 днейбыла без сознания, это когда
наши уже вошли в Киев. Соседи донесли и по ехала санитарная машина из больницы (
из нашей больницы и наши врачи),хотели маму забрать в больницу. Бабушка завалила
маму периной и подушками и сказала, что ее нет. И сама выходила, ходила брала
молоко козье и вливала теплое ей в рот. Потом мама выздоровела. Рассказывпла
хотела зажечь Лампадку, начали гореть волосы. Отгодода мама теряла
сознание, бабушка от голода опухла по пояс, если бы опухоль дошла до сердца ,она
бы умерла. Баьушкп работала в брд нице Павлова, главврач узнал, что она свой обед
не ест, а носит домой ребенку, дал распоряжение выдавать обед и на ребенка. Очень
много рассказывали. Всего и не помню уже.

\iusr{Марья Заславская}
\textbf{Наталия Багинская} хорошо, что рассказывали... мои мне мало рассказывали. хотя пережили всё
\end{itemize} % }

\iusr{Ирина Нищимная}
Сердце разрывается от боли,как же они выжили,, если выжили,,, война самое страшное, что может быть на этой планете,,,

\iusr{Lola Madino}
***Когда немцам стало ясно, что город не удержать, они применили тактику «выжженной земли».*** - правда
***Все, что можно было уничтожить, должно было быть взорвано*** - правда.
***ценности – вывезены*** - правда
***все жители – убиты*** - очень трагично и патетично - НО НЕПРАВДА.
На этом у меня все.

\begin{itemize} % {
\iusr{Валентина Бабченко}
\textbf{Lola Madino} ,а факты?

\begin{itemize} % {
\iusr{Таня Куликова}
\textbf{Валентина Бабченко} 

Киев не был мёртвым накануне освобождения и это известный факт. Многие мои
киевские соседи пережили оккупацию. Моя бабушка Куликова Елизавета
Савельевна, входившая в группу киевских подпольщиков, была расстреляна в Бабьем
Яру, по доносу соседей, за день до освобождения Киева. Все годы оккупации город
жил своей тяжёлой, голодной жизнью

\end{itemize} % }

\end{itemize} % }

\end{itemize} % }
