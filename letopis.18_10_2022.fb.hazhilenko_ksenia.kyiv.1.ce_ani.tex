% vim: keymap=russian-jcukenwin
%%beginhead 
 
%%file 18_10_2022.fb.hazhilenko_ksenia.kyiv.1.ce_ani
%%parent 18_10_2022
 
%%url https://www.facebook.com/permalink.php?story_fbid=pfbid02fa8aYvZxLvcWnWm5CWKEJWE7GrXu5aVBrSvGFn2bicCg3FaCpiw8pEpa7db38U8al&id=100006566076129
 
%%author_id hazhilenko_ksenia.kyiv
%%date 
 
%%tags volonter
%%title Знайомтеся, це Ані
 
%%endhead 
 
\subsection{Знайомтеся, це Ані}
\label{sec:18_10_2022.fb.hazhilenko_ksenia.kyiv.1.ce_ani}
 
\Purl{https://www.facebook.com/permalink.php?story_fbid=pfbid02fa8aYvZxLvcWnWm5CWKEJWE7GrXu5aVBrSvGFn2bicCg3FaCpiw8pEpa7db38U8al&id=100006566076129}
\ifcmt
 author_begin
   author_id hazhilenko_ksenia.kyiv
 author_end
\fi

Знайомтеся, це Ані.

Ані вдвох генерують майже 80\% продукціі нашого «пошивочного цеху»

Їхні кмітливі голівки невпинно винаходять та конструюють, ручки кроять, а
машинки строчать безкінечні теплі рукавиці, бафи, устілки, подушки та ковдри, і
навіть троха - термобілизну.

Коли вимикають світло, я спершу думаю: твою...!, в Ань зупинилися машинки! Але
потім ся заспокоюю - вони принаймні троха перепочинуть

Особисто я готова жити при свічках, прати вручну і навіть інколи обходитися без
фена 😫, аби машинки дівчат мали енергоживлення, а Ані - сили та натхнення для
своєї волонтерської роботи.

Будь як Ані,

Допомагай ЗСУ щохвилини,

В будь-який доступний спосіб

\ii{18_10_2022.fb.hazhilenko_ksenia.kyiv.1.ce_ani.orig}
