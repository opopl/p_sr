% vim: keymap=russian-jcukenwin
%%beginhead 
 
%%file 10_09_2020.fb.himich_roman.1.rjabchuk_statja_malorossia
%%parent 10_09_2020
 
%%url https://www.facebook.com/roman.khimich/posts/3681437128553365
 
%%author_id himich_roman
%%date 
 
%%tags malorossia,statja,ukraina
%%title Cтаття Миколи Рябчука "Від Малоросії"
 
%%endhead 
 
\subsection{Cтаття Миколи Рябчука \enquote{Від Малоросії}}
\label{sec:10_09_2020.fb.himich_roman.1.rjabchuk_statja_malorossia}
 
\Purl{https://www.facebook.com/roman.khimich/posts/3681437128553365}
\ifcmt
 author_begin
   author_id himich_roman
 author_end
\fi

\href{https://www.facebook.com/gennadiy.druzenko}{Gennadiy Druzenko}
наполегливо порадив мені ознайомитися із свіжою статтею Миколи Рябчука \enquote{Від
Малоросії}, в якій він нібито \enquote{намагається вивести українські спільноти з їхніх
гетто на ширший шлях}.

\href{https://zbruc.eu/node/100091}{%
Від Малоросії, Микола Рябчук, zbruc.eu, 04.09.2020%
}

Прізвище Рябчука знайоме мені давно і дуже добре. На зламі століть, коли мене
захопила суспільно-політична тематика, я перечитав чи не всі номери часопису
"Критика", в якому регулярно друкувався пан Микола. Він запам'ятався мені своєю
концепцією "креольства", за допомогою якої він пропонує аналізувати феномен
російськомовного українства. За двадцять років потому, як завжди несподівано
виявилося, що ця метафора дійсно вдала і корисна, але в геть іншому сенсі, аніж
на це сподівався її автор. Про це дещо пізніше, а поки повернуся до
обговорюваної статті. 

Я не бачу сенсу детально її аналізувати через безліч логічних хиб та інших вад
тексту. Пан Рябчук намагається обґрунтувати свою позицію за допомогою
силогізмів настільки неоковирних, що якось навіть незручно за такого поважного,
добре освіченого пана. Як це зазвичай буває із націонал-орієнтованими
інтелектуалами, основним інструментом пізнання соціальної реальності йому
послуговує побутова телепатія. 

"Що мали на увазі ті люди, які голосували за незалежну Україну?" - задається
питанням шановний метр. Притомна людина з вищою освітою висунула б кілька
гіпотез та пошукала б відповідні дані. Рябчукові це не потрібно, йому і так все
зрозуміло. 

"Українці проголосували не тільки за незалежність, а й за секретаря ЦК КПУ
Леоніда Макаровича Кравчука. Ці 90\% голосів за незалежність розпадаються: дві
третини – за Кравчука, приблизно третина – за дисидента В’ячеслава Чорновола,
українського Гавела чи Валенсу. Тобто дві третини вибирають незалежну радянську
Україну як продовження Української РСР зі всіма інституціями і кадрами. І лише
третина обирає модель, за якою розвивалися Польща, Чехо-Словаччина, Литва,
Латвія", стверждує він.

Мушу нагадати, що в 1991 велика, якщо не більша частина освіченого міського
класу мала погляди виразно антирадянські та антикомуністичні. Політика
гласності, яка на той час налічувала вже років п'ять, призвела до оприлюднення
величезного обсягу інформації про дійсно жахливі речі. Було втрачено ілюзії
щодо "ленінських витоків", повернутися до яких мріяла творча інтелігенція ще на
початку Перебудови. 

Ні про яку "незалежну радянську Україну як продовження Української РСР" вже не
могло бути й мови. Так, люди мріяли про незалежність і окремішність від Москви,
але без номенклатури, ідеократії, КДБ та решти радянського антуражу. Більше
того, без соціалізму, радянської уравніловки, держплану та інших елементів
радянської економічної системи.  

Кравчук був першим, хто вміло виконав роль Зеленського, ефективно створюючи
"позитивні очікування" в електорату. Він справляв враження людини розважливої
та хитрої. На той момент країна вже занурилася у важку економічну кризу і люди
вірили, що досвідчений керівний кадр швидше знайде спосіб відновити виробничі
процеси, перезапустити економіку. Проти Чорновола грала його очевидна
позасистемність, через яку було складно уявити, як він зможе "вирішувати
питання" із старими номенклатурниками, що очолили нові державні утворення на
пострадянському просторі. 

Неприємне враження справляла і значна частина його ентузіастів. Підстаркуваті
дядьки у вишиванках із "козацькими" вусами, оцей типаж націонал-стурбованого
ідіота з "Просвіти", який періодично верзе якусь відверту маячню, не викликав
симпатій в значної частини містян.

Опозиція "правильні проєвропейські україноцентричні українці - дещо дефектні
прорадянські малороси", яку вкотре багатослівно викладає Рябчук, висмоктана з
пальця. Варіантів української ідентичності було і є суттєво більше. Це,
щонайменше, галичанський (консервативний), малоросійський, прорадянський,
ліберальний та анархічний. П'ять, а не два. При цьому галичанська ідентичність,
щоб там не співав пан Рябчук, чудово поєднується із мрією про затишне тепле
місце в імперії, якщо розуміти під нею наддержавне утворення. Так званий
"проєвропейський вибір" прямо це передбачає.

Міркування Рябчука переповнені специфічної пихи, без якої, здається, не буває
націонал-орієнтованих інтелектуалів. Запропоноване ним розрізнення "України
української" та "України амбівалентної" викликає в мене регіт. Амбівалентність
з багатьох питань - характерна риса тієї публіки, яку Рябчук записує до першої
категорії. Приклад, який ми нещодавно обговорювали, це ставлення
націонал-стурбованої публіки до BLM, антиколоніальних рухів в Третьому світі і
навіть до Великого Голоду в Ірландії.


\href{https://focus.ua/opinion/opinions/459279-irlandskii_golodomor_pochemu_ukrainskie_uchebniki_istorii_sochuvstvuiut_britanskomu_imperializmu}{%
Ирландский Голодомор: почему украинские учебники истории сочувствуют британскому империализму, %
Роман Химич, focus.ua, 21.07.2020%
}

Резюмую.

Рябчук та його однодумці нічого не забули і нічому не навчилися. Вони ніяк не
можуть полишити свого інтелектуального схрона, в якому вже тридцять років
розповідають одне одному різню маячню про "титульну націю" та її особливі
права. Про жахи радянського колоніалізму та свій постколоніальний синдром.

Мене, до речі, вражає, наскільки ця публіка позбавлена фантазії та аналітичних
здібностей. Наприклад, жаліючись на утиски радянського режиму,
націонал-схвильовані навіть не замислюються, наскільки б їм було важче надувати
свої бульбашки зі шмарклів, якби Гітлер та Сталін не позбавили їх великих,
заможних та впливових єврейської та польської громад. 

Просто уявіть собі, які пісні б співав Микола Рябчук та його однодумці, якби в
кожному місті Західної України 10-30\%-50\% населення становили поляки - із своєю
мовою, потужною культурою та впливовою історичною Батьківщиною просто за
дверима. А в містах Правобережної України аналогічну долю населення складали б
євреї. І не ті зручні, тихенькі секулярні "жидобандерівці", які навіть кіпу не
ризикують надягати, аби не зачепити вічно набряклі "національні почуття"
аборигенів, а люди, скажімо так, набагато більш яскраві та розкуті. Це про
ступінь адекватності самої метафори аборигенного населення та його одвічних
російськомовних ворогів чи то конкурентів. 

При всьому цьому, як з'ясувалося абсолютно несподівано завдяки подіям в США,
метафора креольства може бути використана і навпаки. У спосіб, про який її
автор навряд чи замислювався. 

Як вже стало абс. очевидно, український етнічний націоналізм змістовно та
структурно подібний до BLM-руху. Подібні їхні риторика, культурні та політичні
практики. Чорнопикі, перепрошую, чорноброві адепти постколоніальних студій
вимагають від білих російськомовних людей роззброїтися, стати на коліна,
покаятися за століття панування, негайно запровадити найрізноманітніші форми
позитивної дискримінації на користь хлопців та дівчат із гетто і платити,
платити, платити. Бо так є правильно, так є гідно. Ukrainian Identity Matters! 

Причому ступінь і сам факт наявності оцієї української ідентичності, правила її
визначення та вимірювання чорні побратими та посестри визначатимуть самостійно.
Білим та прирівнюваним до білих треба стулити пельку і підкоритися.

Чи є в статті Рябчука бодай щось корисне? Так. В ній, нарешті, визнається той
факт, що після 30 років націонал-стурбоване товариство, вся ця ні\%\%ерська
спільнота не спроможна предявити "креолам" нічого, що могло б надихнути
зрадників повернутися до тубільних джерел. Зняти огидні колонізаторські брюки
та краватки, прикрасити себе елегантними татуюваннями та намистами, заспівати
та затанцювати як ото сторіччями діди співали-танцювали, поки не прийшли білі
чорти і не загнали їх на плантації. 

Особливо показовими в цьому плані є останні шість років. Що б не намагалися
змайструвати в царині правосуддя тубільні ентузіасти -
САП/НАБУ/НАЗК/ДБР/ВАС/АРМА, - в результаті все рівно виходить Львівський
апеляційний адміністративний суд, де взірцевий галичанський патріот Сергій
Зварич колядує собі та діточкам на подарунки. Якщо хтось сумнівається в
націонал-патріотичних чеснотах брата Сергія, просто гляньте анонс його Opus
Magnum

\href{https://web.archive.org/web/20180412002424/http://zvarych.info/index/biografija_igorja_zvaricha/0-95}{%
Біографія Ігоря Зварича, zvarych.info%
}

Пан Рябчук не висловлює жодних ідей щодо причин такої інституційної
неспроможності його однодумців. На мій погляд, причиною цієї імпотенції є те,
що їхніми фактичними, а не декларативними цінностями є а) трайбалізм б)
паганство (Україна поднад усе!) та в) магічне мислення в усьому різноманітті
його проявів.

Завершуючи свою рецензію, повторю ще раз ключову тезу: варіантів української
ідентичності було і є суттєво більше, аніж два. Це, щонайменше, галичанський,
малоросійський, прорадянський, ліберальний та анархічний. 

Модель соціальних процесів в українському суспільстві, яку багато років
просуває Рябчук, страждає на редукціонізм настільки радикальний, що виникає
питання про інтелектуальну спроможність метра. Я просто не бачу сенсу серйозно
обговорювати погляди, наріжним каменем яких є уявлення про флогістон та
епіцикли.

\ii{10_09_2020.fb.himich_roman.1.rjabchuk_statja_malorossia.cmt}
