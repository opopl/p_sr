%%beginhead 
 
%%file 31_03_2022.fb.lafazan_natalia.mariupol.1.malchik_vladlen
%%parent 31_03_2022
 
%%url https://www.facebook.com/permalink.php?story_fbid=pfbid02wcnqEPnF2Pynavt89v3JWtaZPgTWtsNRXS6kiBEpVJDVQRoAfD7uYgNWdpdLgzr5l&id=100030592628843
 
%%author_id lafazan_natalia.mariupol
%%date 31_03_2022
 
%%tags mariupol.war,mariupol
%%title Этого мальчика зовут Владлен
 
%%endhead 

\subsection{Этого мальчика зовут Владлен}
\label{sec:31_03_2022.fb.lafazan_natalia.mariupol.1.malchik_vladlen}

\Purl{https://www.facebook.com/permalink.php?story_fbid=pfbid02wcnqEPnF2Pynavt89v3JWtaZPgTWtsNRXS6kiBEpVJDVQRoAfD7uYgNWdpdLgzr5l&id=100030592628843}
\ifcmt
 author_begin
   author_id lafazan_natalia.mariupol
 author_end
\fi

Этого мальчика зовут Владлен. Он уже взрослый. Но он ровесник моей старшей
дочери. Он пришел и спросил можно ли ему остаться у нас. Я растерялась. Не
понимала что именно ему кажется безопасным и комфортным в нашем доме. Мне
рассказали его историю он сирота и жил с бабушкой. Бабушка умерла и лежала в
квартире. Конечно я ответила да. Я его всегда теряла. Я вспоминала о нем не
сразу. Я не успела к нему привыкнуть. Благо я любитель одеял и почти увлеченный
фанатик пледов. Я вязала их всегда, как только руки были не заняты. Первый я
связала безцельно. Белый, огромный, крупной вязки. Не думала, что его оценят.
Но дети оценили. Я связала второй. Почти такой. Потом в ход пошли серые нитки.
Это был плед для папы. Он сгорел первым. Белые ещё грели людей больше трёх
недель. Владлен у я могла дать только теплые куртки. Я спросила его, может ли
он захватить из дома одеяло. Именно когда он пошёл за одеялом, был прилет к
Наташе Дедовой. Они соседи по площадке. Я корила себя за то, что отправила
ребенка в неизвестность. Он вернулся и стал мне родным. Он всегда был рядом. Мы
делили еду и когда я спрашивала нужна ли добавка, Владлен удивлено спрашивал:
"А что, можно?" Он замечательный парень с чистым сердцем и светлыми мыслями.
После пожара я его потеряла из виду. Вроде он пошёл в нашу девятиэтажку к
соседям на первый этаж. Я очень надеюсь, что мы с ним встретимся в мирном
Мариуполе и я смогу ему помочь. Я не смогла в силу обстоятельств. И таких людей
без крова в Мариуполе очень много. И количество их увеличивается с каждым днём.

%\ii{31_03_2022.fb.lafazan_natalia.mariupol.1.malchik_vladlen.cmt}
