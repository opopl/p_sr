% vim: keymap=russian-jcukenwin
%%beginhead 
 
%%file 02_03_2018.stz.news.ua.mrpl_city.1.gotel_spartak
%%parent 02_03_2018
 
%%url https://mrpl.city/blogs/view/mariupolskij-gotel-spartak
 
%%author_id demidko_olga.mariupol,news.ua.mrpl_city
%%date 
 
%%tags 
%%title Маріупольський готель "Спартак"
 
%%endhead 
 
\subsection{Маріупольський готель \enquote{Спартак}}
\label{sec:02_03_2018.stz.news.ua.mrpl_city.1.gotel_spartak}
 
\Purl{https://mrpl.city/blogs/view/mariupolskij-gotel-spartak}
\ifcmt
 author_begin
   author_id demidko_olga.mariupol,news.ua.mrpl_city
 author_end
\fi

\ii{02_03_2018.stz.news.ua.mrpl_city.1.gotel_spartak.pic.1}

З виникненням у людей бажання подорожувати, з'явилися і перші готелі. Епоха
готельного бізнесу почалася з заїжджих дворів, таверн і трактирів. В кожну
епоху вони відповідали своїм головним вимогам – надавали гостям можливість
зупинитися на ніч. Цікаво, що перші готелі з'явилися близько двох тисяч років
тому в Стародавній Греції та в Стародавньому Римі. Проте сучасний вигляд
готелі придбали в США. Дослідники вважають, що це сталося приблизно в середині
XIX століття, саме тоді почали з'являтися повноцінні номери.

\ii{02_03_2018.stz.news.ua.mrpl_city.1.gotel_spartak.pic.2}

Сьогодні готелі відрізняються між собою не тільки ціновою політикою,
індивідуальною атмосферою, але й власною історією. У Маріуполі один з
найбільших і комфортабельних готелів міста \enquote{Спартак}  розташований на розі
Харлампіївської та Георгіївської вулиць. Обидві вулиці є історично важливими
для Маріуполя. Вулиця Харлампіївська, одна з перших вулиць нашого міста,
названа так на честь священномученика Харлампія. Старі джерела розповідають, що
під час переселення греків з Криму, коли епідемія стала косити паству
митрополита Ігнатія, переселенці звернули свої молитви до святого Харлампія.
Вважають, що саме завдяки заступництву Харлампія хвороби припинилися. Вулиця
упорядковувалася дуже повільно. Спочатку вона складалася в основному з
глинобитних і частково кам'яних будинків, критих черепицею. Пізніше, в середині
XIX століття, почала забудовуватися кам'яними особняками. Вулиця Георгіївська
була названа на честь особливо шанованого греками святого великомученика.

\ii{02_03_2018.stz.news.ua.mrpl_city.1.gotel_spartak.pic.3}

Повернемося до історії готелю \enquote{Спартак}. Є декілька версій його будівництва. За
першою, яка є більш розповсюдженою в інтернеті, будівля готелю, увінчана
куполом, була побудована за проектом головного архітектора міста Віктора
Нільсена наприкінці XIX століття (приблизна дата – 1884 рік). За іншою,
підтвердженою місцевою періодикою, яку підтримують краєзнавці (Буров С. Д.) та
музейні працівники (Божко Р. П.), приміщення було побудовано паном Рапопортом з
метою здавати в оренду в 1915 р.

\ii{02_03_2018.stz.news.ua.mrpl_city.1.gotel_spartak.pic.4}

Під час Першої світової війни міська влада хотіла розмістити в приміщенні
госпіталь для поранених, чи поселити біженців із західних губерній. Проте, що
саме знаходилося в приміщенні в роки Першої світової війни, достовірно
невідомо. У роки окупації в будівлі знаходилася казарма для німецьких та
словацьких офіцерів. Готель у стінах споруди відкрився вже за радянських часів.

Цікаво, що будівля готелю була добудована і сучасний її вигляд відрізняється
від початкового. Проте, щоб зробити прибудову, довелося прибрати залишки
згорілого в війну поштового відділення, яке розміщувалося з 1930 року в
будинку, що належав Еммануїлу Спиридоновичу де Полоне – судновласнику і
консульському агенту відразу двох держав: Італії та Австро-Угорщини. Для
виконання своїх консульських обов'язків панові де Полоне нікуди ходити або
їздити не доводилося: відвідувачів і прохачів він приймав у власному будинку.
Є версії, що в приміщенні консульства в 1878 році народився Антон Петрович
Боначіч - видатний оперний співак (тенор), соліст Большого театру в Москві.
Обдарований співак і артист, Боначіч чудово володів мистецтвом перевтілення.
Він співав Германа в \enquote{Піковій дамі} Чайковського, Садко в однойменній опері
Римського-Корсакова, Голіцина в \enquote{Хованщині} Мусоргського. З 1928 року жив і
працював у Мінську.

\ii{02_03_2018.stz.news.ua.mrpl_city.1.gotel_spartak.pic.5}

У різний час у готелі \enquote{Спартак} знаходили притулок під час ще довоєнних
гастролей скрипалі Давид Ойстрах та Михайло Ерденко, відомі свого часу на всю
країну співаки Вадим Козін, Георгій Виноградов, Клавдія Шульженко і Леонід
Утьосов зі своєю дочкою Едіт.

Також у \enquote{Спартаку} ночували артисти театру і кіно Борис Андрєєв, Марк Бернес,
Іван Миколайчук, Микола Крючков, Раїса Не\hyp{}дашківська, Віталій Поліцеймако.
Навіть сама Фаїна Раневська відпочивала в одному з номерів після зустрічей з
маріупольською публікою. Цей список поповнюється перебуванням знаменитих людей
й досі. Ось тільки деякі з них: Валерій Леонтьєв, Алла Пугачова, Михайло
Свєтін, Анна Самохіна, Ада Роговцева, Івар Калнінш, Володимир Конкін, Михайло
Жаров, Олег Блохін, Олексій Михайличенко, Анатолій Кашпіровський і багато
інших.

Однак це лише мала частка списку \enquote{зірок} різної величини та різних років, які
вшанували своєю присутністю маріупольський готель, який прикрашає історичний
центр міста і пропонує безпечний і комфортний відпочинок всім гостям і туристам
Маріуполя.

\clearpage
