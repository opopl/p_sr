% vim: keymap=russian-jcukenwin
%%beginhead 
 
%%file 10_01_2023.fb.lafazan_natalia.mariupol.1.miroslava.cmt
%%parent 10_01_2023.fb.lafazan_natalia.mariupol.1.miroslava
 
%%url 
 
%%author_id 
%%date 
 
%%tags 
%%title 
 
%%endhead 

\qqSecCmt

\iusr{Victoria Mironenko}

Нам на выезде из токмака женщина дала пончики в пакетике.... Не забудем
никогда.... А ещё одна семья нас забрала к себе ночевать.. Мы даже номера тел
не записали (((

\begin{itemize} % {
\iusr{Наталья Лафазан}
\textbf{Виктория Мироненко} 

да, Викуль. В том страшном пути из ада, встречались, только хорошие люди. Не
считая мудаков на блокпостах. Но номера телефонов в том момент мозг совсем не
подумал взять(((

\iusr{Victoria Mironenko}
\textbf{Наталья Лафазан} 

однозначно... Но мы все помним, возвращаться будем той же дорогой, обязательно
заедем... А на блокпостах стояли земляки моей мамы.. Из новоазовска и
обрыва.... Твою мать, как это может быть??? 🤦🤦🤦

\iusr{Наталья Лафазан}
\textbf{Виктория Мироненко} 

мир сошел с ума. Не весь. Но одна огромная страна утратила способность к
мышлению, сочувствию, любви. И начала сеять это по миру. Вот семена и к
землякам твоей мамы попали. Читали в детстве такую сказку. Оказалось прям
реально возможно. Лёд в сердце.

\iusr{Людмила Резник}
\textbf{Наталья Лафазан} після війни, заїдете і подякуєте).

\iusr{Наталья Лафазан}
\textbf{Людмила Резник} аби вони всі вистояли у окупації

\iusr{Людмила Резник}
\textbf{Наталья Лафазан} аби лишились живі.

\end{itemize} % }

\iusr{Olga Yuferova}

Наташа, читаю вас со слезами...

\begin{itemize} % {
\iusr{Наталья Лафазан}
\textbf{Ольга Юферова} все буде Україна. А ці страшні згадки залишаться у минулому. Я на це дуже сподіваюся.

\iusr{Olga Yuferova}
\textbf{Наталья Лафазан} так!

\iusr{Людмила Резник}
\textbf{Наталья Лафазан} так і буде. Але я своїм внукам буду розказувати про вас як ви вижили в Маріуполі, про Азовсталь.
\end{itemize} % }

\iusr{Евгений Котлубей}

Было очень круто, когда Дима пришел к нам с Паркового с пятилитровчиком воды в
подарок

\begin{itemize} % {
\iusr{Наталья Лафазан}
\textbf{Евгений Котлубей} 

вот с такими подарками до войны, наверное, дурку бы вызвали. А во время войны
ничего ценнее не было ( Страшно это даже вспоминать. А как мы это пережили,
мозг мой отказывается понимать. Жень. Прости. Может неуместно. Прийми мои
соболезнования. Мне очень жаль, что я не знала твою старшую дочь и ее маму. Ещё
больше жаль, что и не узнаю. Пусть покоятся с миром.

\iusr{Евгений Котлубей}
\textbf{Наталья Лафазан} дякую... я остаточно не вірю, плачу іноді

\iusr{Наталья Лафазан}
\textbf{Евгений Котлубей} 

це зрозуміло. Але, будь ласка, тримайся. Життя триває. Ми повинні бути
сильними. Та потім, в пам'ять про загиблих маріупольців, повернутися туди.
Можливо у нас не витримає нервова система і ми втечемо від страшних згадок. Але
повернутися ми ради них забов'язані! Ще на руїнах мого дому шашлику зробимо з
всіма друзями та маленьким Микитою. Ми там, в підвалі, домовлялися, що коли
йому виповниться рік, поставимо великі столи у дворі, та будемо святкувати
величезною компанією. Коли всіх знайдемо. Чому п'ятого березня у дворі - гадки
немаю) але дома немає -двір є. Збіг? Знак? Чекаю на п'яте березня!

\iusr{Marija Ruda}
\textbf{Наталья Лафазан} 

не змогла змовчати... Читаю завжди Ваші пости... Я захоплююся силою і духом
Вашим... Іноді, от зараз, я думаю, а чи змогла б я це все перенести так, як
Ви... Я зі Львова, в порівнянні у нас тихо.... Є іноді прильоти, світло зникає
на добу, після того... В мене тоді ступор, я якась розчавлена... Ви сильні,
мужні. Дай Боже щоб усе так і сталося, щоб ви повернулися, та і всі українці
додому, на свою землю ❤️🙏🙏🙏

\end{itemize} % }

\iusr{Галина Сергеева}

Мы сколько ехали по Украине- нам все старались помочь. В Винницу вообще
приехали к незнакомым людям и нас встретили как родных. А потом вообще
уговаривали до окончания войны жить у них.. Для меня Александр и Зоя стали
родными. И верю, что они обязательно приедут к нам в Мариуполь. И телефон в
сервисе отремонтировали бесплатно. Хороших людей в нашей стране очень много,
подавляющее большинство. И в Польше, и в Германии тоже.

\iusr{Людмила Резник}

Читаю ваші розповіді і не можу сліз задержати, через що вам довелось пройти, і
це в 21 столітті!!! Що радіти так хлібові, воді!!!! У нас все було і ми були
щасливі, поки одному дебілоїду маразм в голову не стукнув!!!

\begin{itemize} % {
\iusr{Валентина Камець}
\textbf{Людмила Резник} 

на жаль там ціла країна таких бо інакше б роздавали б того маразматика як
комаху, щоб і сліду не залишилось. А вони мовчки тупо підтримують, як стадо
вівців і дітей своїх посилають до нас вбивати, гвалтувати, мародерити,
руйнувати все навколо. Для мене вони всі одинакові, кінчені двуногі істоти, які
не мають права існувати на землі.

\end{itemize} % }

\iusr{Natalia Ostapenko}

До нашого маленького штабу в центрі Житомира під'їхало авто. Молода пара
збиралася повертатися у звільнену Бородянку і звернулися до нас про допомогу.
Просили завантажити їх........ ХЛІБОМ. Бо люди, які лишилися там, забули його
смак. В той період ми готували щоранку 2000 бутербродів на блокпости. Ми
скупили весь хліб, який був у магазинах на всіх сусідніх вулицях. І додали все,
що було, до хліба. Це була краплинка щастя, бо вони поверталися додому. Вірю,
що маріупольці теж повернуться додому і завжди будуть з хлібом. А нечисть
згине.

\ifcmt
  igc https://scontent-fra3-1.xx.fbcdn.net/v/t39.30808-6/323893645_1589445444818723_3134255765319139791_n.jpg?stp=cp6_dst-jpg&_nc_cat=103&ccb=1-7&_nc_sid=dbeb18&_nc_ohc=ACmNxwlZxpkAX9LtZry&_nc_ht=scontent-fra3-1.xx&oh=00_AfB4XTpFkBIwDoD3jX87exJRkVM289xgtqlp5hCStqRFjQ&oe=6439EF95
	@width 0.5
\fi

\begin{itemize} % {
\iusr{Валентина Камець}
\textbf{Natalia Ostapenko} 

поки ця країна, в нинішніх масштабах і кордонах, буде існувати нам ніколи не
буде спокою. Усім наступним поколінням також бо ця історія буде повторюватись
постійно.

\end{itemize} % }

\iusr{Tatiana Zhuridova}

Спасибо за пост. Читала со слезами. Сколько вы пережили! Обнимаю

\iusr{Ирина Давыдюк}

В місті Радомишль оформлювала допомогу як впо. Побачила пару з малою дитиною,
може місяць-два. Жінка пішла стяти в черзі, а чоловік з крихіткою вийшов з
задушливого приміщення на вулицю. Я вийшла покурити.

Стою,курю і щось мене потягнуло йому сказати, щоб не чекав, що дружина швидко
повернеться, адже черга велика. Дитинка плакала і він з за пазухи дістав
пляшечку з сумішшю і почав її годувати. Спитала звідки вони, а почувши дивилася
на нього і плакала.. Почула - Маріуполь...

Він дивився на мене, на мої сльози з великими очима. Я спитала чи є в них де
жити, чи є все необхідне для дитини, та чи є суміш.Сказав,що те, що п'є дитина,
то остання, що далі невідомо. Я вже знала про волонтерські центри, де є
харчування для немовлят та розповіла. Також дізналася, що є де жити, адже
приїхав до товариша, який одружився в Радомишлі та осів в цьому місті.
Сподіваюся, що в ції сім' ї зараз все добре, та шкодую, що не взяла номер
телефону. До цих пір не можу забути його вираз обличчя, та те болюче
"Маріуполь", вимовлене з такою тугою і біллю.

\iusr{Елена Шепель}

Завжди читаю і плачу, не можу до цього звикнути, хоча своє вже притупилося,
ми вже дома, хоча за цей час що читала в Харків два прильота було і на них уже
не реагуєш.

\begin{itemize} % {
\iusr{Nataliia Chaus}
\textbf{Елена Шепель}
\end{itemize} % }

\iusr{Валентина Мацієвська}

Просто 😭😭😭...

Ніколи не буде, як було...

Нехай буде краще і рани Ваші загояться.

\iusr{Раїса Марковець}

Наталя! В черговий раз прочитала ваш пост і плачу. Слава Богу, ви в безпечному
місті. Мені сьогодні волонтери скинули прохання : Помолитись за захисників
Бахмута. Там зараз пекло. Вони просять нашої молитви.

\iusr{Юлия Дейнеко}

Після сліз, хочу сказати, що завжди свято вірю в те, що добрих людей більше,
ніж злих. Адже добро повертається сторицею.

\iusr{Оксана Бублик}

Нажаль діло не тільки в одному діктаторі - там зазомбоване населення, яке
повністю його підтримує...

\iusr{Helen Tkalichenko}

Боже... Ваші розповіді розривають мені серце. Сижу, реву. Який жах...
тримайтесь, рідненькі, тільки тримайтесь. Все повернемо та відбудуємо.
Впевнена, що ви повернетесь в рідний Маріупіль.

\iusr{Світлана Волошина}

Ну а я то чого реву? Волинь. Було всього кілька прильотів-влучань. Люди не
голодують, святА всі відсвяткували. Але згадую розповіді моєї бабусі, як ходили
нашим краєм голодні з Брянщини. Просили хоч щось поїсти. За окраєць хліба
готові були працювати на будь-яких роботах. І - знову. Голодні люди. В країні,
яка, схоже, своїми продуктами пів світу годує. Боляче.

\iusr{Горбуля Николай Галина}

Дякуємо за Ваші дописи! Все буде Україна! Обіймаємо!

\iusr{Alena Hriy}

У Мангуші жила подруга тому як добралися до неї, вона нам цілий пакет продуктів
винесла... А ми як дурні стоїмо регочемо ..Вона в шоці каже ти чого регочеш??
Ніколи цього не забуду....

А у Бердянську підходили місцеві на палаці спорту давали їжу і таблетки від кашлю сину 💔 я всім так вдячна.....

А в Запоріжжі коли нас розселили волонтер спита звідки ми, а потім впав перед
сином і плакав💔 таке не забувається..

\iusr{Олена Залізняк}

реву....

\iusr{Людмила Иванко}

Без сліз вас читать неможливо... Пишіть, щоб ми знали правду ...
