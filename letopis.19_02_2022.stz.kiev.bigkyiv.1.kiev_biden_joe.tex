% vim: keymap=russian-jcukenwin
%%beginhead 
 
%%file 19_02_2022.stz.kiev.bigkyiv.1.kiev_biden_joe
%%parent 19_02_2022
 
%%url https://bigkyiv.com.ua/my-dumayemo-shho-rosiya-napade-na-kyyiv-najblyzhchymy-dnmy-dzho-bajden
 
%%author_id senenko_tetjana
%%date 
 
%%tags biden_joe,kiev,rossia,ugroza,ukraina
%%title Ми думаємо, що Росія нападе на Київ найближчими днями – Джо Байден
 
%%endhead 
 
\subsection{Ми думаємо, що Росія нападе на Київ найближчими днями – Джо Байден}
\label{sec:19_02_2022.stz.kiev.bigkyiv.1.kiev_biden_joe}
 
\Purl{https://bigkyiv.com.ua/my-dumayemo-shho-rosiya-napade-na-kyyiv-najblyzhchymy-dnmy-dzho-bajden}
\ifcmt
 author_begin
   author_id senenko_tetjana
 author_end
\fi

\begin{zznagolos}
Президент США Джо Байден заявив, що в американської розвідки є підстави вважати
можливим напад Росії на Київ найближчим часом.
\end{zznagolos}

За словами президента США, провокації Росії останніми днями відповідають тим
\enquote{методичкам}, які вона використовувала раніше. Це відповідає тим сценаріям, про
які попереджали США та партнери раніше.

\ii{19_02_2022.stz.kiev.bigkyiv.1.kiev_biden_joe.pic.1}

\begin{zzquote}
«У нас є підстави говорити, що Росія хоче напасти на Україну, наступними днями.
Ми думаємо, що вони нападуть на Київ. Там 2,8 млн населення. Ми говоримо про це
не тому, що хочемо конфлікту, а для того, щоб усунути будь-які підстави, які
могли б бути використані росіяни для виправдання своїх дій. Якщо Росія так
зробить це буде війною, яку вона обрала, за її вибором... Росія ще може обрати
дипломатію. Ще не пізно обрати деескалацію і повернутись за стіл переговорів»,
— заявив Байден.
\end{zzquote}

Росія оточила Україну зі сторони Білорусі також. У нас є підстави думати, що в
наступні дні вони нападуть на Київ. Там 2,8 мільйона населення, – сказав
Байден.

Перед зверненням він наголосив, що спілкувався із партнерами щодо ситуації в
Україні. Всі залишаються зі своєю твердою позицією у підтримці України.

Тим часом радник глави Офісу президента Михайло Подоляк заявив, що Україна
скористається шансом для дипломатії. Так він у суботу, 19 лютого, прокоментував
слова президента США, який заявив про можливе швидке вторгнення Росії в
Україну, пише CNN.  Подоляк додав, що \enquote{ніхто не знає, які думки в
голові у президента Росії Володимира Путіна}.
 
Він наголосив, що позиція президента США Джо Байдена, \enquote{безперечно,
заснована на інформації розвідувальної спільноти}, заявивши, що \enquote{це не
питання згоди чи незгоди із заявами}.

Нагадаємо, раніше держсекретар США Ентоні Блінкен попередив, що протягом
найближчих днів країна-агресор може вигадати привід та напасти на українську
територію. Головною ціллю Росії може бути Київ.
