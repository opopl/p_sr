% vim: keymap=russian-jcukenwin
%%beginhead 
 
%%file 22_02_2022.fb.zozulja_roman.1.mamkin_istorik
%%parent 22_02_2022
 
%%url https://www.facebook.com/permalink.php?story_fbid=1636727226663575&id=100009789399084
 
%%author_id zozulja_roman
%%date 
 
%%tags nacia,putin_vladimir,ukraina
%%title Мамкін історик не розуміє, що своїми фантастичними казками лише загартовує і згуртовує нашу націю
 
%%endhead 
 
\subsection{Мамкін історик не розуміє, що своїми фантастичними казками лише загартовує і згуртовує нашу націю}
\label{sec:22_02_2022.fb.zozulja_roman.1.mamkin_istorik}
 
\Purl{https://www.facebook.com/permalink.php?story_fbid=1636727226663575&id=100009789399084}
\ifcmt
 author_begin
   author_id zozulja_roman
 author_end
\fi

Мамкін історик не розуміє, що своїми фантастичними казками лише загартовує і
згуртовує нашу націю. 

Вони можуть скільки завгодно рукоблудити на забальзамований труп в мавзолеї,
але для нас він був, є і буде катом, як і його наступник. І наша справа, що
робити з їхніми пам’ятниками і як розвивати свою країну.

Ми ніколи не були братськими народами, бо братам не намагаються нав’язати
комплекс меншовартості і при нагоді закинути петлю на шию. Ми завжди були
кісткою в горлі для більшості правителів «великої і могучої». В різні історичні
періоди вони прагнули нас підкорити, морити голодом, переслідувати, саджати і
вбивати за право говорити рідною мовою, висловлювати позицію та прагнути
незалежності. І чим більше вони стискали свій зашморг, тим сильніше ми з нього
виривалися. У нас ніколи не було стокгольмського синдрому, у нас завжди було
тотальне несприйняття їхнього насилля і контролю. В новітній історії України ми
це демонстрували раз на десятиліття на Майданах. Натомість, наші сусіди бояться
підняти голову і висловити протест душевнохворому царю. Звісно, є свідомі
росіяни, які неодноразово висловлювали нам свою підтримку, але їх критично
мало. Переважна більшість - це зомбована  пропагандою біомаса. Але це їхній
вибір - жити в страху і пригніченні. І нам з ними не по дорозі. 

Ми мирна нація, яка не має в планах ні на кого нападати. Нам вистачає людей і
своїх земель, але і віддавати їх якомусь божевільному ми не будемо. 

На 8-му році війни нарешті весь світ прозрів і усвідомив, що в Україні немає
ніякого громадянського конфлікту, що ми 8 років стримуємо російську агресію і
не даємо їй поширитися на всю Європу. Ціною цього стало 15 тисяч життів… 

Тому світ має усвідомити і іншу істину, що путін - це реінкарнація гітлера. І
його плани більш амбітні, ніж захоплення  України. Тому, зараз як ніколи
важливо, щоб світ об’єднався і запровадив економічні санкції проти цього
проклятого режиму. 

А з фізичним опором, впевнений, наші ЗСУ справляться, бо характер наших
військових викутий зі сталі. Наша перемога вже близько, як і феєричний кінець
кремлівсього зла. 

Вірю в Україну, вірю в перемогу, вірю в українську армію! @igg{fbicon.hands.pray} @igg{fbicon.flag.ukraina}
