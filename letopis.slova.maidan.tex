% vim: keymap=russian-jcukenwin
%%beginhead 
 
%%file slova.maidan
%%parent slova
 
%%url 
 
%%author 
%%author_id 
%%author_url 
 
%%tags 
%%title 
 
%%endhead 
\chapter{Майдан}
\label{sec:slova.maidan}

В це відео слід тикати носиками як щенят, що нашкодили, тицяють у їхні какахи,
всіх наших (і не наших по ходу також) \enquote{білих пальтечок}, всіх тих, кому
наш \emph{Майдан} видавався \enquote{надто брутальним}, кому не подобався запах
\enquote{коктейлів} та шин, хто засуджував стрілянину та закликав нас до
ненасилля та всякого гандизму. А особливо тих, хто до останнього намагався про
щось \enquote{домовитись} із Овочем, вів нескінченні \enquote{перемовини}, коли
у нас на вулицях вже йшла війна.  Якби ми тоді їх послухались - були б зараз на
місці Протасевича, без варіантів.  Бо ось так, як на сьогоднішньому відео, і
лише так, закінчується \enquote{гандізм}, якщо твої опоненти - не британські
джентльмени, а породження ГУЛАГу. Проти цих потвор - лише вогонь на ураження,
\textbf{Хто на місці Протасевича не був, будь ласка, стуліть писки},
Євген Дикий, gazeta.ua, 04.06.2021

Сохранившие разум в \emph{майданном} безумии. В августе 2017 года в телепередаче
«Время покажет» на «Первом канале» киевлянка Светлана Пикта, мама трех детей,
которая тогда была на шестом месяце беременности (в декабере она родила двойню
— сына и дочь), на личном примере рассказала, как живется на Украине людям,
разумом которых не овладел \emph{майданный морок}. Светлана Пикта высказывала свое
мнение в социальных сетях: «Мы не относимся к украинским гражданам, которые
писают от восторга от режима Порошенко. Наша позиция не совпадает с
\emph{промайдановской} позицией. Мы пацифисты, христиане. Мы – многодетная семья»,
\textbf{Мужественные киевляне стали нашими земляками}, rweek.ru, 20.07.2018


%%%cit
%%%cit_head
%%%cit_pic
%%%cit_text
Про український характер у порівнянні з московитами а ляхами мовлено у праці
Володимира Антоновича ТРИ НАЦІОНАЛЬНІ ТИПИ НАРОДНІ:
\url{http://www.litforum.com.ua/index.php?r=7&a=5631} Українські \emph{майдани}
1990, 2004, 2013 були вибухом саме такої української стихії.  Вороги не тямлять
ані української нації, ані українського характеру, ані української стихії. Саме
тому вони не мають стовідсоткового успіху в Україні.  Жодна окупація в Україні
не досягнула і не досягне успіху, бо се вербовий народ (як влучно зауважила у
романі ВІДЛУННЯ відома україеська письмачка Лариса Денисенко), він гнеться але
не ламаєсь, і вельми вельми живучий.  Коли буде \emph{четвертий майдан}?
Скоро.  Десь 2024 рок
%%%cit_comment
%%%cit_title
\citTitle{Роман Кухарук - УКРАЇНСЬКА СТИХІЯ  А ВОРОГИ - Блог}, 
, romankuxaruk.com.ua, 08.04.2021
%%%endcit


%%%cit
%%%cit_head
%%%cit_pic
%%%cit_text
Но к заявлению РФ о том, что Украина добровольно и первой вышла из Минских
соглашений нашему МИДу стоило бы подготовиться, а президенту соответственно - к
звонкам от глав государств Нормандской четверки, которые, поставив подписи под
Минском, теперь, получается, разделяют за это все ответственность. Ведь
Украина, получается, отказывается от дипломатического пути решения конфликта и
переходит к военному. Но больше всего удивляет не парубийство действующей
власти и не верность \emph{идеалам майдана}. Первый раз вижу законопроект о тайных
военных спецоперациях, диверсиях и социальных гарантиях для участников этих
мероприятий. Конечно же, экономический и инфекционный кризис в стране, но чтоб
deep state авансом признавался в своих военных преступлениях и анонсировал свои
действия, - это мировая премьера и заявка на чемпионство
%%%cit_comment
%%%cit_title
\citTitle{Закон Об основах национального сопротивления - нечто уникальное / Лента соцсетей / Страна}, 
Елена Дьяченко, strana.ua, 29.06.2021
%%%endcit

%%%cit
%%%cit_head
%%%cit_pic
\ifcmt
  pic https://avatars.mds.yandex.net/get-zen_doc/4864152/pub_60d89638f47c9a12d7c67b5f_60d8965827b03102425e2d2f/scale_1200
\fi
%%%cit_text
Объяснение простое: работы нэмае... Когда был в одном украинском селе ещё в
далёком \emph{домайданном} 2012 году, то бросилось в глаза практически полное
отсутствие людей работоспособного возраста, по улицам ходили в основном дети и
и пенсионеры. Читатель может обвинить меня в предвзятости и будет не прав, ведь
я ТОЧНО ЗНАЛ где и как живёт и трудится большинство из них, ведь я всё детство
провел в том селе
%%%cit_comment
%%%cit_title
\citTitle{Страшная тайна демографии Украины}, 
Мак Сим, zen.yandex.ru, 28.06.2021
%%%endcit


%%%cit
%%%cit_head
%%%cit_pic
%%%cit_text
Вот как нам вернуть чувство, если хотите, общего национального достоинства? И
чтобы это не только для одних, потому что кого-то \emph{Майдан} – это чувство
достоинства, а для кого-то нет, а вот, наверное, для всех по возможности. Вот с
чего его сплести?  В одну реку нельзя войти дважды. Достоинство вернуть нельзя.
Можно. Мы же входили.  Нет.  \emph{Два Майдана}.  Нет. Только через жертвы и
наказания. Вот на \emph{Майдане} можно было, после Майдана сразу можно было,
если бы мы пошли на несколько важных шагов. Если бы начали воевать и воевали бы
до конца, пока не отвоевали Крым и Донбасс. Можно было бы удержать достоинство.
Отказались? Предоставили возможность этой вороватой и коррумпированной элите?
Теперь только через наказание, через жертву. Иначе нельзя. Сейчас только
жертва. Понимаете?
%%%cit_comment
%%%cit_title
\citTitle{Сергей Дацюк: Украина сегодня - не просто попрошайка, она на мусорнике истории}, 
Сергей Дацюк; Людмила Немыря, hvylya.net, 28.06.2021
%%%endcit

%%%cit
%%%cit_head
%%%cit_pic
%%%cit_text
С приближением осени сгущается страшная тень третьего \emph{Майдана}. ТРЕТИЙ \emph{МАЙДАН}:
ПРЕДЧУВСТВИЕ ГРАЖДАНСКОЙ КАТАСТРОФЫ.  Так уж сложилось, что я много читаю и
общаюсь с умными людьми. Поэтому понимаю закономерности общественного развития.
И сейчас меня откровенно пугает предстоящее. Я — певец первого \emph{Майдана} и
сочувствующий второму, с ужасом ожидаю третьего \emph{Майдана}.  С какими бы
прекрасными лозунгами он ни собрался, пускай даже главной темой выступлений со
сцены будет необходимость нового Общественного Договора и новой Конституции,
пускай на знаменах новых \emph{майдановцев} будут золотом вышиты принципы прихода во
власть людей достойных и профессиональных, все равно протест неминуемо
перерастет в беду. В грабеж магазинов, мародерство, кровавые столкновения с
перестрелками и прочими прелестями люмпен-революции — читайте книгу «1917» Ян
Валетов
%%%cit_comment
%%%cit_title
\citTitle{Пружина общественного терпения сжимается, но не бесконечно}, 
Александр Кочетков, strana.ua, 11.07.2021
%%%endcit

