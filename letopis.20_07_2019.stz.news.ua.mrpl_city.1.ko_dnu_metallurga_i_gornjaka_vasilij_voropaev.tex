% vim: keymap=russian-jcukenwin
%%beginhead 
 
%%file 20_07_2019.stz.news.ua.mrpl_city.1.ko_dnu_metallurga_i_gornjaka_vasilij_voropaev
%%parent 20_07_2019
 
%%url https://mrpl.city/blogs/view/ko-dnyu-metallurga-i-gornyaka-vasilij-voropaev
 
%%author_id burov_sergij.mariupol,news.ua.mrpl_city
%%date 
 
%%tags 
%%title Ко Дню металлурга и горняка: Василий Воропаев
 
%%endhead 
 
\subsection{Ко Дню металлурга и горняка: Василий Воропаев}
\label{sec:20_07_2019.stz.news.ua.mrpl_city.1.ko_dnu_metallurga_i_gornjaka_vasilij_voropaev}
 
\Purl{https://mrpl.city/blogs/view/ko-dnyu-metallurga-i-gornyaka-vasilij-voropaev}
\ifcmt
 author_begin
   author_id burov_sergij.mariupol,news.ua.mrpl_city
 author_end
\fi

О Воропаеве, азовстальском инженере, потерявшем зрение и нашедшем силы не
только не бросить профессию, но и защитить кандидатскую диссертацию, в свое
время журналисты писали много. Не обошли его вниманием кинохроникеры и
телевизионщики, а друг и коллега по заводу, инженер и талантливый поэт Николай
Берилов посвятил ему поэму.

Но и журналисты, на бегу бравшие интервью у Василия Алексеевича, и Берилов, не
один десяток лет бок о бок проработавший с ним в мартеновском цехе и в
лаборатории, как бы вскользь касались жизни Воропаева до того момента, когда
мир для него навсегда погрузился в кромешную тьму. Но именно в той жизни, до
несчастья, ковалась его несгибаемая воля и твердый характер. Читая черновики
анкет и прилагаемых к ним биографий, нетрудно заметить, что судьба не уставала
испытывать этого человека на прочность духа. А последующие события показали:
испытания продолжались до последнего его часа.

\ii{20_07_2019.stz.news.ua.mrpl_city.1.ko_dnu_metallurga_i_gornjaka_vasilij_voropaev.pic.1}

В биографии, которую неведомо по какому случаю записали со слов Василия
Алексеевича в 1982 году, читаем: \emph{\enquote{Родился я в селе Шехмань Тамбовской губернии
31 декабря 1909 года. Родители мои были крестьяне. Во время пожара все
имущество сгорело, отец и мать пошли работать по найму. В 1914 году началась
Первая мировая война, отца взяли на фронт, там он вскоре погиб. Моя мать,
Наталья Павловна, осталась с тремя детьми: старший Михаил, я – средний и
младший Сергей}}. Нужно ли обладать изощренной фантазией, чтобы представить, в
какой нужде пребывала семья вдовы-солдатки и погорелицы, когда вокруг полыхали
войны: мировая, затем гражданская, которая раздирала страну, судьбы, тела и
души людей, когда царила разруха, свирепствовали голод и эпидемии. Но вдова не
только спасла свое семейство и себя от погибели, но и поставила сыновей на
ноги. Уже в советское время Наталья Павловна ценою невероятных усилий приобрела
профессию учительницы. В этом качестве она работала то в школе, то в детском
доме, переезжая время от времени из села в село, из городка в городок. 

Вася Воропаев, закончив семилетку, пошел на подготовительные курсы и вскоре его
приняли в Липецкий металлургический техникум. Он выбрал специальность, где
требовалась и крепость мышц, и сила воли. А между тем, - об этом мало кто знал,
- еще в отрочестве будущий металлург ослеп на один глаз. Это не помешало
Василию успешно освоить программу техникума, пройти все предусмотренные
практики в цехах, защитить дипломный проект и поступить в литейный цех
Сталинградского завода \enquote{Красный Октябрь}. Работу в цехе он совмещал с учебой на
вечернем факультете института.

1938 год. Василий Воропаев, только что получивший диплом инженера, был
направлен в Мариуполь на завод \enquote{Азовсталь}. С того времени, если не считать
трех лет напряженного труда в эвакуации в Пермской области на оборонном заводе
в г. Лысьве, его жизнь была связана с этим предприятием. В мартеновском цехе
\enquote{Азовстали} довелось ему быть начальником смены, начальником ОТК. Потом перешел
в центральную заводскую лабораторию. Как пригодился там богатый
производственный опыт.

Вернувшись из эвакуации, Воропаев решил обосноваться в Мариуполе всерьез и
надолго. Еще в первые послевоенные годы, как только представилась возможность,
приступил к строительству собственного дома. Взял ссуду, получил \enquote{план} на
пустыре неподалеку от главного корпуса нынешнего технического университета. На
заводе выписывал кирпич, отслуживший свой срок в металлургических печах, но
вполне пригодный для строительства жилья. Все, от кладки стен до посадки
деревьев в саду, делал собственноручно вечерами, в выходные и праздничные дни,
во время плановых отпусков...

\textbf{Читайте также:} 

\href{https://mrpl.city/news/view/v-mariupole-pozdravlyali-silnyh-lyudej-so-stalnym-harakterom-foto}{%
В Мариуполе поздравляли сильных людей со стальным характером, Ігор Романов, mrpl.city, 15.07.2019}

Инженер Воропаев, как по основной службе, так и по общественной работе, был
человеком безотказным. Когда его просили провести экскурсию для школьников или
студентов по цехам завода, охотно соглашался. В тот роковой для него день
показывал ученикам тридцать первой школы рельсобалочный цех. Вдруг увидел, как
нагретый добела рельс мчится по рольгангу к дисковой пиле. Рядом девчонка. Еще
мгновение и острые зубья врежутся в металл, сноп искр накроет ее. Рывок - и
Воропаев закрыл собой ребенка. По роковой случайности раскаленная стальная
стружка поразила тот глаз, который видел. Слепота. Ни врачи столичных клиник,
ни даже знаменитый одесский профессор Филатов не смогли вернуть Воропаеву
зрение.

Быть может, если бы рядом не было жены, \textbf{Екатерины Федоровны}, - коротать бы ему
свой век инвалидом, делающим в цехе общества слепых щетки или собирающим на
ощупь простенькие устройства для электроприборов. Это она вселила в мужа веру в
себя. На долгие годы, до самой своей кончины стала его глазами, его научным и
техническим секретарем, а главное - духовной силой и опорой. Часами читала
вслух порой совершенно непонятные для нее технические тексты, обсчитывала
тысячи цифр – результаты рядовых и экспериментальных плавок, тщательно
записывала под диктовку Василия Алексеевича главу за главой диссертационной
работы.

Воропаев и его жена, занятые многотрудной научной работой, не были изолированы
от внешнего мира. Наоборот, им охотно оказывали помощь азовстальцы. И работники
мартеновского цеха, и исследователи Центральной заводской лаборатории проводили
по рекомендациям Василия Алексеевича опытные плавки, систематизировали
полученные результаты. Не отказывали ему в консультациях и советах
преподаватели металлургического института. Был среди его помощников и
преподаватель русского языка и литературы \textbf{Б. В. Петриков}.

Позже Борис Васильевич рассказывал:

\begin{quote}
\em
"Почти два года довелось мне заниматься
правкой рукописей Воропаева. Вначале стилистика сильно хромала. Мы обсуждали
сложные места текстов. Поражала его память. Все мои советы, воспринятые только
на слух, брались им на вооружение. Постепенно изложение мыслей на бумагу
незрячего инженера резко улучшилось и достигло того уровня, когда моя
\enquote{стилистическая} помощь уже была не нужна. Однажды я пришел к Воропаеву в
неурочный час. Я был поражен: мой подопечный ремонтировал пол на веранде –
заменял подгнившую доску новой. Он работал так, как будто все видит. Василий
Алексеевич, уже будучи незрячим, построил в своем саду решетчатую беседку. Да
такую, что соседка, увидев ее, в сердцах выпалила: \enquote{Та он
придуривается, он все видит. \enquote{Мой} с глазами и то такого не сделает}. 
Встречи с этим необыкновенным
человеком запомнились мне на всю жизнь".
\end{quote}

Василий Алексеевич работал над проблемой использования высокофосфористых
железных руд керченского месторождения для производства сталей ответственного
назначения. Специалисты знают, что вовлечение в металлооборот таких руд было и
есть задачей общегосударственного масштаба. В своей диссертации Воропаев смог
сделать свой вклад в решение этой задачи. Защита диссертации в ученом совете
Днепропетровского металлургического института прошла успешно. Члены совета без
всякого снисхождения к необычному соискателю единодушно проголосовали за
присуждение инженеру Воропаеву ученой степени кандидата технических наук.
Поздравления в его адрес заслуженно разделила и Екатерина Федоровна.

Через несколько лет после этих событий в семью Воропаевых вошел новый человек –
муж их дочери Наташи. Случилось так, что именно ему, \textbf{Борису Рассолову}, судьба
определила быть самым близким человеком Василия Алексеевича на долгие годы.
Борис Григорьевич вспоминал:

\begin{quote}
\em
\enquote{Мне пришлось находиться рядом и знать его жизнь более тридцати лет. Мой тесть
отдал металлургии более четверти века. Считал себя все годы после беды,
случившейся в 1956 году, нормальным и уверенным в себе человеком. Он не был
членом общества слепых. Это обнаружилось, когда пришлось воспользоваться
льготой, которая предоставлялась людям, лишенным зрения, при оплате за
пользование телефоном. Василий Алексеевич работал научным сотрудником кафедры
металлургии стали в нашем институте, проводил научные исследования, читал
лекции, участвовал в работе государственной комиссии по защите дипломных
проектов. Им были написаны две книги, посвященные переделу фосфористых чугунов,
множество статей в научные журналы. Несмотря на полную потерю зрения, он был
полноценным здоровым человеком, который сам за собой ухаживал, работал в саду,
выезжал на дачу. До самых последних дней сохранил прекрасную память. Он никогда
никому не сказал, что чем-то обделен жизнью. Даже тогда, когда рок наносил ему
удар за ударом}.
\end{quote}

А удары были тяжелее один другого. Решили возвести на улице Апатова три
девятиэтажки-общежития. Его дом и сад попали в зону застройки. То, что
создавалось многие годы ценой невероятных усилий, было снесено в считанные дни.
Он не роптал, спокойно перебрался в стандартную квартиру многоэтажного дома. В
1971 году случилась невосполнимая потеря – умерла его друг, его жена Екатерина
Федоровна. Казалось, что горестная чаша испита им до дна, что он уйдет из жизни
раньше, чем кто-нибудь из его семьи. Увы, он ошибся. Ему было уготовано
пережить смерть правнука, а через короткое время и горячо любимой дочери...

Воропаев не жаловался на судьбу. Он оставался внешне спокойным. Он считал себя
обязанным поддержать в горе своих близких.

\textbf{Читайте также:} 

\href{https://mrpl.city/news/view/na-mariupolskom-metkombinate-vnedryayut-novye-e-nergosberegayushhie-tehnologii}{%
На мариупольском меткомбинате внедряют новые энергосберегающие технологии, Ігор Романов, mrpl.city, 16.07.2019}
