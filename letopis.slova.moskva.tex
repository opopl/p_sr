% vim: keymap=russian-jcukenwin
%%beginhead 
 
%%file slova.moskva
%%parent slova
 
%%url 
 
%%author 
%%author_id 
%%author_url 
 
%%tags 
%%title 
 
%%endhead 
\chapter{Москва}
\label{sec:slova.moskva}

Болтовня о \emph{промосковскости} и \enquote{пропутинскости} Лукашенко есть полная ерунда,
поскольку все эти годы бацька только то и делал, что балансировал между \emph{Москвой}
и Западом, всячески уворачивался от цепких объятий \emph{Москвы} и лично Путина, с
которым у него очень непростые отношения, пытаясь сохранить максимум
самостийности, но не разрушить при этом, по украинскому образцу, экономику и
социалку, во многом зависящую от рынков бывшего Союза и вообще от традиционных
связей. Один только герб Советской Белоруссии чего стоит на фоне российского
триколора!,
\textbf{Европа впала в иррациональный истеричный лукашизм}, Александр Карпец,
strana.ua, 03.06.2021

Колектив Національного культурного центру України у м. \emph{Москві} вітає з Міжнародним днем захисту дітей!, 
\textbf{Наші діти!}, ukrcentr.ru, 01.06.2021

27 травня в Національному культурному центрі України у м. \emph{Москві} відбувся
концерт лауреата міжнародних конкурсів Української народної хорової капели
\emph{Москви}. На барвисту концертну програму «Пісенне джерело натхнення» завітали
шанувальники й друзі Української народної хорової капели \emph{Москви}, які протягом
багатьох років відвідують численні музичні вечори в Культурному центрі України
та, насамперед, щомісячний Український музичний салон. Відкриваючи концертну
програму творчого колективу, зі словами вдячності до гостей вечора звернулася
художній керівник та диригент капели, начальник служби з питань культури і
діаспори НКЦУ, заслужений працівник культури України та Росії, голова РГО
«Культурно-просвітницький центр українців у м. \emph{Москві}» Вікторія Скопенко,
\textbf{Пісенне джерело натхнення}, ukrcentr.ru, 28.05.2021

Конечно, мне было безумно страшно переезжать. Младшим двойняшкам на тот момент
исполнилось полгодика, третья дочь тоже была кнопкой, второй было 10, старшему
только исполнилось 12, вся эта «команда» была непригодна для бегства, мягко
говоря. Мы не могли взять с собой ничего, кроме детей, трое из которых ехали в
колясках,  и нескольких чемоданов с их же одеждой: чтобы нанять безумно дорогую
грузоперевозку, нужно было хотя бы знать, где мы будем жить. На пересадке с
самолёта на самолёт в Минске мы так долго заполняли декларации на всех семерых,
что чуть не опоздали на второй борт до \emph{Москвы}. Добрались до Ярославля
одним днём. Нам чудом удалось в тот же день подписать договор аренды
квартиры... и никогда не забуду, как я в этот первый день лежала в темноте на
голом матрасе и чужой подушке, рядом сопели младшие, а я смотрела широко
открытыми глазами в потолок. Скоро 40 лет, а приходится мне, многодетной
матери, всё начинать с нуля,
\textbf{Ярославль глазами Киевлянки},
odnarodyna.org, 03.06.2021

В Смутное время монастырь выдержал 16-месячную осаду польско-литовских войск,
подошедших к нему в сентябре 1608 года. В конце 1609 года в осаждённом
монастыре началась цинга, во время которой умерло свыше двух тысяч человек.
Людей, способных защищать монастырь осталось около 200. В январе 1610 года
осада была снята русскими войсками. Монастырские укрепление быстро восстановили
и при этом стены надстроили в высоту и увеличили в ширину, а башни обрели
дошедший до наших дней вид. Последний раз враги были под стенами монастыря в
1618 году, во время похода на \emph{Москву} польского королевича Владислава,
после чего наступило время процветания монастыря. В 1689 году в монастыре
укрывался спасшийся из \emph{Москвы} бегством Пётр I, и из него он уехал уже
полновластным правителем. При Петре I в обители появилась великолепная барочная
трапезная с храмом преподобного Сергия Радонежского, так называемая Трапезная
церковь. Это самая необычная церковь, которую я видел,
\citTitle{Место силы}, Ленинградский Наблюдатель., zen.yandex.ru, 03.06.2021

\emph{Москва} не может такого простить, потому что это наносит страшный удар по ее
государство- и нациеобразующей мифологии. Поэтому Россия всячески будет
дискредитировать такой законопроект о коренных народах. Ведь Путин мировым
лидерам говорил, что Украина – это \enquote{русская земля}, а украинцы и русские – это
\enquote{один народ}, тем более, что украинцев никогда не было и нет, есть только
\enquote{русские люди}, \enquote{русский язык}, да и вообще там, где Украина, \enquote{Русью пахнет}. А
тут вдруг оказывается, что русские не являются коренным народом... ,
\citTitle{Как законопроект о коренных народах Украины выбил почву из-под ног Москвы - Главред}, 
Григорий Перепелица, opinions.glavred.info, 10.06.2021

%%%cit
%%%cit_pic
%%%cit_text
Украинская детская писательницы Лариса Ницой раскритиковала Украинскую сборную
по футболу за то, что они не включили в свой плейлист ни одной украинской
песни, а также захотела, чтобы они полностью исключили из своего обихода
русский язык.  Об этом она написала на своей странице в Facebook.  \enquote{Это
просто трэш какой-то. Сначала Сборная Украины по футболу публикует плейлист с
русскими песнями. Тогда говорит, что это не их плейлист - и публикует новый,
уже их, куда не вошла ни одна украинская песня. Украинские футболисты
демонстрируют публично свою \emph{московитисть} - а тогда возмущаются, что
\emph{московиты} им указывают на форму... Вы действуете как \emph{конченые
московиты}. Поэтому не удивляйтесь, что \emph{Московия} запрещает вам форму. Вы
их подданные. Вы их - со всеми потрохами, то есть, мозгами}, - написала Ницой
%%%cit_title
\citTitle{Писательница Ларица Ницой назвала украинских футболистов моковитами}, Эллина Либцис, strana.ua, 11.06.2021
%%%endcit

%%%cit
%%%cit_pic
%%%cit_text
Почему-то \emph{Москва} считается чуждым современной Украине городом, но мало
кто знает, что она была основана не какими-то там «москалями», а натуральными
великими киевскими князьями, и была заселена «украинцами», бежавшими из
Киевского княжества на север от междоусобиц, набегов половцев, а затем и
монголо-татар. Вот интересен такой парадокс украинской истории: почему-то все
украинцы, и особенно западные, считают, что российские «москали» - это
совершенно чужой народ, а их столица \emph{Москва} – город совершенно далекой от
Украины ментально страны
%%%cit_title
\citTitle{Кто основал Москву? Вы даже удивитесь - ее основал великий киевский князь в XII веке и заселил ее украинцами}, 
Исторический Понедельник, zen.yandex.ru, 18.02.2021
%%%endcit

%%%cit
%%%cit_pic
%%%cit_text
Ну, западенцев понять можно – они и их предки на Руси никогда не жили,
постоянно находясь то под властью литовцев, поляков, османов, австрийцев,
венгров и даже румын. Поэтому для них \emph{Москва} – и на самом деле чуждый мир,
такой же, как для европейца – дебри Соломоновых островов.  Но западенцев в
Украине – раз-два и обчелся, а что же остальные украинцы? Вот бы они удивились,
что \emph{Москва} – это чисто украинский город, он был основан предками украинцев,
возвышен ими и прославлен на весь мир тоже ими. Не верите? А ну-ка давайте
вспомним генеалогию Рюриков, которые правили Русью
%%%cit_title
\citTitle{Кто основал Москву? Вы даже удивитесь - ее основал великий киевский князь в XII веке и заселил ее украинцами}, 
Исторический Понедельник, zen.yandex.ru, 18.02.2021
%%%endcit

%%%cit
%%%cit_pic
\ifcmt
  pic https://avatars.mds.yandex.net/get-zen_doc/1873427/pub_5fdbe7737200780de15215b3_5fdbf08a7200780de155d7c7/scale_1200
	caption Великий киевский князь Юрий Долгорукий - основатель Москвы. Портрет из "Царского титулярника" за 1672 г.
\fi
%%%cit_text
\emph{Москва} была основана в середине XII века русским князем Юрием Долгоруким.
Может, кто не знает, то этот персонаж русской истории был киевским князем, и не
просто князем, а великим (по титулу). Одновременно он княжил и во Владимире –
городе, который основал его отец – великий князь киевский Владимир Мономах. И
все последующие потомки этого киевского Рюриковича включительно до Александра
Невского тоже были великими киевскими князьями
%%%cit_title
\citTitle{Кто основал Москву? Вы даже удивитесь - ее основал великий киевский князь в XII веке и заселил ее украинцами}, 
Исторический Понедельник, zen.yandex.ru, 18.02.2021
%%%endcit

%%%cit
%%%cit_pic
%%%cit_text
И это тоже мало кто знает, но причинами основания \emph{Москвы} как раз и был наплыв
русского («украинского») населения из «Киевской Руси» на северные земли, как
раз туда, где сходятся истоки Волги, Днепра и \emph{Москвы-реки}. Начало это
происходить еще в начале XII века, когда Киев потерял все свое столичное
значение и от него политически и юридически откололись все русские княжества –
и северные, и западные.  Одновременно с юга на Русь начались массированные
нападения половцев, и через 100 с лишним лет, как раз к началу
монголо-татарского нашествия почти 70\% населения Киевского княжества
перебралось на север. Когда в Киев пришли монголы, в северные земли (в \emph{Москву})
же бежали и остальные «украинцы» (или скорее – «окраинцы»)
%%%cit_title
\citTitle{Кто основал Москву? Вы даже удивитесь - ее основал великий киевский князь в XII веке и заселил ее украинцами}, 
Исторический Понедельник, zen.yandex.ru, 18.02.2021
%%%endcit

%%%cit
%%%cit_pic
\ifcmt
  pic https://avatars.mds.yandex.net/get-zen_doc/2804475/pub_5fdbe7737200780de15215b3_5fdbf12e7c919e46c07eee04/scale_1200
	caption Князь великия всея Руси Иван I Калита, сделавший Москву столицей Древнерусского государства
\fi
%%%cit_text
\emph{Москва} так быстро стала разрастаться из-за этого «окраинского» наплыва, что ее
пришлось выделить в отдельное княжество. У Александра Невского (не забываем –
великого киевского князя) был сын Даниил, которого он и поставил управлять этим
княжеством. После Даниила \emph{московским князем} стал Иван Калита, потомственный
киевский Рюрикович. Под его руководством к началу XIV века \emph{Москва} стала одним
из самых крупных городов Руси, а \emph{Московское княжество} стало главенствующим на
всех русских землях, присоединяя к себе все другие ранее раздробленные
княжеские владения. Как результат – Иван I Калита получил титул «Князь великия
всея Руси»
%%%cit_title
\citTitle{Кто основал Москву? Вы даже удивитесь - ее основал великий киевский князь в XII веке и заселил ее украинцами}, 
Исторический Понедельник, zen.yandex.ru, 18.02.2021
%%%endcit

%%%cit
%%%cit_pic
%%%cit_text
Не лишним будет напомнить, что «всея Русь» - это была именно ВСЯ РУСЬ, а не
только \emph{«Московия»} или что-то там еще. Таким образом Иван I Рюрикович, ближайший
потомок всех великих киевских князей, получил право на ВСЮ РУСЬ, от Киева и до
Новгорода, от Смоленская и до Твери, от Черного моря до самого Ледовитого
океана. Это право было подтверждено перемещением из разоренного
монголо-татарами Киева резиденции митрополита киевского, который в конечном
итоге стал \emph{митрополитом московским}. Соответственно в \emph{Москве} сосредоточилась и
вся духовная власть, а не только политическая
%%%cit_title
\citTitle{Кто основал Москву? Вы даже удивитесь - ее основал великий киевский князь в XII веке и заселил ее украинцами}, 
Исторический Понедельник, zen.yandex.ru, 18.02.2021
%%%endcit

%%%cit
%%%cit_pic
%%%cit_text
Конечно, не сразу \emph{Москва} стала столицей «всея Руси», до этого центром огромной
страны был Владимир (как помните – основанный великим киевским князем
Владимиром Мономахом). Но в 1331 году все признаки высшей власти были
перенесены из Владимира в \emph{Москву}, так \emph{Москва} и стала столицей Руси, которая
впоследствии превратилась в Россию.  Итак, что же мы видим?
%%%cit_title
\citTitle{Кто основал Москву? Вы даже удивитесь - ее основал великий киевский князь в XII веке и заселил ее украинцами}, 
Исторический Понедельник, zen.yandex.ru, 18.02.2021
%%%endcit

%%%cit
%%%cit_pic
%%%cit_text
А мы видим, что все сегодняшние \emph{«москали»} - это не потомки каких-то там
финно-угров или монголо-татар, как твердят нам украинские (?) историки. Это
потомственные русичи, которые бежали из разоренного Киевского княжества на
север, в \emph{Москву}, которую, кстати, они и основали. По свидетельствам летописцев,
все земли «Киевской Руси» очень сильно обезлюдели еще до монголов, а после
монголов вообще превратились в «дикое поле» в полном смысле этого слова. Киев
потерял все свое торговое значение и населения в нем осталось чуть больше 1\% от
того количества, которое в нем проживало в лучшие времена
%%%cit_title
\citTitle{Кто основал Москву? Вы даже удивитесь - ее основал великий киевский князь в XII веке и заселил ее украинцами}, 
Исторический Понедельник, zen.yandex.ru, 18.02.2021
%%%endcit

%%%cit
%%%cit_pic
%%%cit_text
Кто и когда основал \emph{Москву} никому не ведомо. Из летописей в 1147 году
князь Юрий Долгорукий в \emph{Москве} назначил встречу с соседним князем,
следовательно город к этому времени уже стоял. Через два года после встречи
князь Ростово-Суздальский Юрий Долгорукий возьмет Киев штурмом, поубивает
киевлян и станет великим князем Киевским. Почти все домонгольские правители
Киева были родом с северных княжеств и брали город силой.  В 1299 году монголы
в очередной раз совершили нападение на Киев. Все кто выжил вместе с
митрополитом киевским и всея Руси Максимом убежали во Владимир. В результате
киевские земли опустели на несколько веков, превратившись в пустыню.  А
Владимир и \emph{Москва} за счет перетока населения возвысились. Предки современных
украинцев понаехали на киевские земли в 15-16 веках и в памяти пришлого народа
нет сведений о былинных киевских богатырях
%%%cit_comment
Алексей Четин
%%%cit_title
\citTitle{Кто основал Москву? Вы даже удивитесь - ее основал великий киевский князь в XII веке и заселил ее украинцами}, 
Исторический Понедельник, zen.yandex.ru, 18.02.2021
%%%endcit

%%%cit
%%%cit_pic
%%%cit_text
Алексей, город стоял но на тот момент административного значения не имел. В
ходе междоусобицы во время феодальной раздробленности он имел равновеликое
значение с остальными. Так было до 1380 года. Примечательно что в \emph{Москве}
есть белорусский вокзал и есть киевский. Украинского нет. Отрицать что Киев был
административным культурным и религиозным центром нельзя. Отрицать состоявшийся
факт Украины которую большевики признали в 1922м тоже. Это все равно что
продукты рвоты обратно в пищевод запихивать. По некоренным но однородным коими
были малоросс и белорусы национальности надо определиться с правовым статусом.
Либо это присягнувшие верноподданные либо согнанные в диаспоры под
цивилизованный апартеид категории граждан. Думаю что Россия окончательно
дальновидность не утратила
%%%cit_comment
Максим Олейник
%%%cit_title
\citTitle{Кто основал Москву? Вы даже удивитесь - ее основал великий киевский князь в XII веке и заселил ее украинцами}, 
Исторический Понедельник, zen.yandex.ru, 18.02.2021
%%%endcit

%%%cit
%%%cit_pic
%%%cit_text
Во-первых, конечно, он неправ. В \emph{Москве} тоже найдены берестяные грамоты.
На сегодняшний день их найдено четыре штуки, причем последняя обнаружена не так
давно - в 2015 году.  Кстати, одна из этих грамот считается самой
содержательной из всех известных. Не из московских, а вообще из всех найденных!
В ней порядка 370 слов. Это уже не записочка, это большой связный рассказ.  Но,
конечно, если сравнивать с новгородскими находками, все меркнет. Пусть там
тексты намного короче, зато их огромное количество - 1093 экземпляра! На таком
фоне любой другой город с его четырьмя, пятью, десятью грамотами покажется
беспросветной глушью, где и писать-то толком не умели.  Однако ни один археолог
такое суждение вынести и не подумает. Потому что знает, какое принципиальное
значение имеют условия почвы, где находят старинные артефакты. Новгородские
слои - уникальны
%%%cit_comment
%%%cit_title
\citTitle{Почему в Москве нашли всего четыре берестяные грамоты, а в Новгороде - свыше тысячи?}, 
Русичи, zen.yandex.ru, 18.05.2021
%%%endcit

%%%cit
%%%cit_pic
%%%cit_text
Идея этой заметки родилась из одного комментария к нашей статье про
новгородские берестяные грамоты. Поинтересовался мимо проходящий читатель - а
чего это, дескать, \enquote{в \emph{Московии}} оных грамот не сыщут никак? Уж не из-за
тотальной ли неграмотности \emph{лапотных москалей}? Хитро так поинтересовался, с
намеком. С некой проекцией на далеко идущие выводы.  Что ж, давайте попробуем
объяснить этому товарищу, в чем тут соль. Мы начнем, а вы в комментариях потом,
если хотите, дополните
%%%cit_comment
%%%cit_title
\citTitle{Почему в Москве нашли всего четыре берестяные грамоты, а в Новгороде - свыше тысячи?}, 
Русичи, zen.yandex.ru, 18.05.2021
%%%endcit

%%%cit
%%%cit_pic
%%%cit_text
Зато сохранилось немалое количество так называемых писал - инструментов, с
помощью которых буквы процарапывались на бересте. А если были писала, значит
люди ими активно пользовались. \enquote{Косвенным подтверждением выводов ученых о
грамотности русского населения являются находки писал (часть средневековой
\enquote{ручки}), которые нередки в \emph{Москве}. Последняя из таких письменных
принадлежностей была найдена в Зарядье}, - сообщает, например, руководитель
департамента культурного наследия \emph{правительства Москвы} Алексей Емельянов.
Собственно, на этом все. Вопрос о том, какой город был грамотным, а какой
безграмотным, можно закрыть. Грамотность на Руси была весьма широкой и, уж
конечно, одним Новгородом не ограничивалась
%%%cit_comment
%%%cit_title
\citTitle{Почему в Москве нашли всего четыре берестяные грамоты, а в Новгороде - свыше тысячи?}, 
Русичи, zen.yandex.ru, 18.05.2021
%%%endcit

%%%cit
%%%cit_pic
%%%cit_text
Далее - всесоюзный триумф. Песни \enquote{Червона рута} и \enquote{Водограй} (\enquote{Музыкальный
фонтан}) становились лауреатами всесоюзных фестивалей \enquote{Песня-1971} и
\enquote{Песня-1972} и звучали в \emph{Москве}, на заключительных концертах этих фестивалей
%%%cit_comment
%%%cit_title
\citTitle{Какая легенда вдохновила Владимира Ивасюка написать \enquote{Червону руту} 
(украинскому шлягеру - 50)}, 
Татьяна Кроп, zen.yandex.ru, 07.12.2020
%%%endcit

%%%cit
%%%cit_head
%%%cit_pic
\ifcmt
  pic https://avatars.mds.yandex.net/get-zen_doc/3612047/pub_60bdf74e7b0ba72b6fd0d2cb_60bdf83ae14854002eba8461/scale_1200
\fi
%%%cit_text
В Публичной библиотеке Конгресса США обнаружил замечательные карты.  Начнём с
\emph{Московии} и Сибири.  Середина 16 века. В \emph{Москве} страшный пожар,
Царём работает Иван 4. Присоединена Казань, Астрахань. Начинается освоение
Сибири.  Через 7 лет Ермак пройдется по Тартарии(?).  Англия открыла Северный
проход к России через Белое море. Учреждена \emph{Московская торговая компания}
для взаимной торговли с Англией.  В \emph{Московии} начинают работать
иностранные инженеры, врачи, военные и тд.  Первая карта больше похожа на
схему. Обозначена Америка. Обозначена Тартария. Некоторые народности обозначены
правильно, там, где они находятся сейчас
%%%cit_comment
%%%cit_title
\citTitle{Карты 1570 года. Тартария, Америка и материк на Северном Полюсе}, 
Нешкольная История, zen.yandex.ru, 08.06.2021
%%%endcit

%%%cit
%%%cit_head
%%%cit_pic
\ifcmt
  pic https://avatars.mds.yandex.net/get-zen_doc/1892973/pub_60bdf74e7b0ba72b6fd0d2cb_60bdf9e0931ee526e1264f0a/scale_1200
\fi
%%%cit_text
Не отходя от кассы, вторая карта-схема Тартарии. Тут есть \emph{Московия}, Астрахань,
Казань. Я к тому, что в 16 веке это были реальные объекты. И Тартария, и
Московия.  А сейчас тартары \enquote{тщательно забыты}
%%%cit_comment
%%%cit_title
\citTitle{Карты 1570 года. Тартария, Америка и материк на Северном Полюсе}, 
Нешкольная История, zen.yandex.ru, 08.06.2021
%%%endcit

%%%cit
%%%cit_head
%%%cit_pic
%%%cit_text
... В Україні внутрішню окупацію здійснюють колишні \emph{московські окупанти} та
колонізатори - \emph{московська} п'ята колона у вигляді різних \enquote{червоних} та
\enquote{лівоцентристських} партій, п'ята колона західного неолібералізму у вигляді
різних \enquote{центристських}, \enquote{правоцентристських}, \enquote{демократичних},
\enquote{національно-демократичних} і навіть \enquote{теж-націоналістичних} організацій, а всі
вони обслуговують основного грабіжника й узурпатора влади – чужорідний і
космополітичний транснаціональний олігархічний капітал кримінального
походження" (В. Іванишин. Внутрішня окупація – війна на знищення).  Режим
внутрішньої окупації не дає можливості для розвитку як всієї країни, так і
корінної нації
%%%cit_comment
%%%cit_title
\citTitle{Сучасний стан української нації та держави Україна}, 
, pravyysektor.info, 25.05.2017
%%%endcit

%%%cit
%%%cit_head
%%%cit_pic
%%%cit_text
Як ти там кажеш? «Я виступаю проти цькування.... і далі по тексту» так от,
виступай ти певно за поріг своєї квартири і тримай напрям на вокзал, там сідай
у потяг разом зі своєю подругою, та їдьте десь у бік \emph{Москви}. Іііі вже там,
виступайте проти цькування скільки вам сили вистачить. А тут всім коло сраки,
подобається це тобі чи ні, але ви шмати лайна, не зважаючи на академічний
ступінь :) Хочете жити тут, мовчки сидіть і все буде добре, ніхто вас цькувати
не буде)
%%%cit_comment
Артем Карташов, facebook
%%%cit_title
\citTitle{Кучма Олексій Андрійович. Виступаю проти цькування - професора Є. В. Більченко},
Алексей Кучма, facebook, 28.01.2021
%%%endcit

%%%cit
%%%cit_head
%%%cit_pic
%%%cit_text
26 червня виповнюється 100 років з дня народження закарпатського прозаїка і
драматурга Юрія Керекеша (1921–2007 ). Його ім’я не надто відоме за межами
рідного краю, але на Закарпатті це помітна постать літературного процесу, а
його оповідання, сповнені життєвої драматургії та художньої довершеності, і
нині беруть за душу. Однак в даному разі хочеться глянути на нього як на
представника міжвоєнного покоління, яке починало свій творчий шлях із...
російської мови. Щоб зрозуміти закоріненість закарпатського
\emph{москвофільства}, яке й донині дотривало у вигляді найчисельнішого в
Україні осередку \emph{московського} православ’я, слід збагнути покручену
історію Закарпаття
%%%cit_comment
%%%cit_title
\citTitle{100 років поколінню закарпатських москвофілів, які обрали українську мову}, 
Олександр Гаврош, www.radiosvoboda.org, 20.06.2021
%%%endcit


%%%cit
%%%cit_head
%%%cit_pic
%%%cit_text
Я направду мовив про неприпустимість уживання \emph{московитського} ЦВЄТНОСТІ, на що
пан промовець здобувся на мисль – же се від Бога. Його донька Леся, же
виявилася курваторкою сего дійства, катеґорично заборонила будь яку дискусію –
чисто по московитськи – і вказала мені на двері. Я сій дамочці дав зрозуміти,
же я не в неї дома, а у ЦЕНТРІ УКРАЇНСЬКОЇ КУЛЬТУРИ, же існує на мої податки, і
тому не потерплю \emph{московської} пропаґанди у центрі княжокого Київа.  Мене
здивувала присутність на заході адвоката Олега Березюка. У приватній розмові
він повідомив, же був там випадково – Янишин, як колишній десантник, знайшов го
по базі даних десантників і запросив.  Отже, українці, будьте обережні – ворог
мімікрує і наступає звідусіль.  До речі, виставці відмовили у прихистку – до
їхньої чести – Музей історії Київа а Софія Київська. А от Центр української
культури позарився на \emph{московську макуху}.  Шкода.  Але будьмо пильні і знищуймо
ворога у зародку скрізь, де побачимо
%%%cit_comment
%%%cit_title
\citTitle{Роман Кухарук - ЦВЄТНОСТІ ВІД ЙУХЛА ТРАНЗИТОМ З ГАЛИЧИНИ У КНЯЖОМУ КИЇВІ - Блог}, , romankuxaruk.com.ua, 26.06.2020
%%%endcit

