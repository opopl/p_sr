% vim: keymap=russian-jcukenwin
%%beginhead 
 
%%file 01_12_2018.stz.news.ua.mrpl_city.1.gorodskie_chasy_gde_ranshe_uznavali_vremja_mariupolcy
%%parent 01_12_2018
 
%%url https://mrpl.city/blogs/view/gorodskie-chasy-gde-ranshe-uznavali-vremya-mariupoltsy
 
%%author_id burov_sergij.mariupol,news.ua.mrpl_city
%%date 
 
%%tags 
%%title Городские часы: где раньше узнавали время мариупольцы
 
%%endhead 
 
\subsection{Городские часы: где раньше узнавали время мариупольцы}
\label{sec:01_12_2018.stz.news.ua.mrpl_city.1.gorodskie_chasy_gde_ranshe_uznavali_vremja_mariupolcy}
 
\Purl{https://mrpl.city/blogs/view/gorodskie-chasy-gde-ranshe-uznavali-vremya-mariupoltsy}
\ifcmt
 author_begin
   author_id burov_sergij.mariupol,news.ua.mrpl_city
 author_end
\fi

Сейчас узнать точное время - пара пустяков. Можно взглянуть на наручные часы.
Благо, часов нынче много. Самых разных. Механических и электронных,
сверхплоских и с огромными циферблатами, в корпусах из стали, из золота и даже…
из дерева, наручных, карманных, изящных дамских, декорированных под браслет.
Часы можно встретить иногда и на тонких детских ручках. Вместе с тем
продвинутая молодежь определяет часы и минуты текущего времени, заглянув на
экраны смартфонов, с которыми, кажется, их владельцы нигде и никогда не
расстаются. Если по какой-то причине вы остались без часов и не располагаете ни
смартфоном, ни простенькой мобилкой, достаточно послушать одну из местных
радиостанций. Время от времени они сообщают: \emph{\enquote{В Мариуполе 18 часов... -или-...
12 часов}}. А раньше?

Долгие годы и десятилетия после основания Мариуполя его жители \enquote{часов не
наблюдали}. И не потому, что были так уж счастливы. Просто часами, как
механизмами для определения точного времени, они не располагали. Да, если с
делом разобраться, в них и не было особой нужды. Спешить мариупольцам было
некуда. О том, что наступило утро, они замечали по восходу солнца, заход
дневного светила за горизонт знаменовал собой начало ночи. Предутренние часы
различали по крику петухов. Когда нужно было идти в храм на молитву - узнавали
по колокольному перезвону. Каждый горожанин хорошо знал \enquote{голоса} своей церкви.

\textbf{Читайте также:} 

\href{https://archive.org/details/17_11_2018.sergij_burov.mrpl_city.staraja_gazeta_chto_volnovalo_100_let_nazad}{%
Старая газета: Что волновало мариупольцев более 100 лет назад?, Сергей Буров, mrpl.city, 17.11.2018}

Немало утекло воды в Кальмиусе, прежде чем в домах местных обывателей появились
часы. У кого - простенькие, грубо раскрашенные \enquote{ходики}, у кого - настенные,
французского изготовления часы с боем в шикарном резном футляре из красного
дерева. Состоятельные горожане могли похвастаться карманными \enquote{хронометрами}, их
носили в кармане жилета, а серебряная или даже золотая цепочка, притороченная к
головке часов, другой своей оконечностью прикреплялась к жилетной пуговице. В
начале XX века не такой уж большой редкостью стали и будильники.

\ii{01_12_2018.stz.news.ua.mrpl_city.1.gorodskie_chasy_gde_ranshe_uznavali_vremja_mariupolcy.pic.1}

Главным православным храмом в Мариуполе был Харлампиевский собор. Он находился
на том месте, где сейчас возвышается здание бывшего оборонного ДОСААФ. В
известном историческом источнике \enquote{Мариуполь и его окрестности, Мариуполь, 1892
г.} написано об этой церкви следующее: \emph{\enquote{\enquote{благоговейным иждивением прихожан}, как
сказано было в описи 1849 г., заложен при протоиерее Феодосьеве, продолжал
строиться при протоиерее Демьянове и окончен в 1845 г. при протоиерее Моторном
новый, величественный, в византийском стиле трехпрестольный храм}}. Через
какое-то время на его колокольне установили часы, циферблаты которых были
обращены на четыре стороны. Куранты звонко отбивали четверть, половину и каждый
час. Харлампиевский собор не только украшал город, но и служил своеобразным
маяком для мореплавателей, его колокольня была видна за несколько миль до
подхода к Мариупольскому порту. Однако к середине 30-х годов городские власти
сочли, что это культовое сооружение ни исторической, ни архитектурной ценности
не представляет, а посему подлежит сносу. Сказано - сделано. Храм был взорван.
Лишь его часовой механизм посчитали полезным для нового строя. Решили
пристроить к зданию горисполкома башню, а на нее взгромоздить часы, снятые с
колокольни собора. Скажем, что до войны горисполком располагался в особняке,
некогда принадлежавшем семье Хараджаевых. Он стоял на углу улицы Артема (ныне
Куинджи) близ сквера, и сейчас его место занято частью дома № 48 по той же
улице. Но ломать - не строить. На башню денег не нашлось, механизм главных
городских часов неведомо куда исчез... Потом были война, оккупация. Здание
горисполкома гитлеровцы сожгли, как, впрочем, и многие, многие другие
мариупольские строения.

\textbf{Читайте также:} 

\href{https://mrpl.city/blogs/view/vtracheni-hrami-mariupolya}{%
Втрачені храми Маріуполя, Ольга Демідко, mrpl.city, 16.03.2018}

В 50-х годах прошлого, ХХ века на месте горисполкома по проекту киевского
архитектора Яновицкого и его коллег построили один из двух известных домов со
шпилями. Если всмотреться в башни, можно заметить, что место для часов было
оставлено - оно обозначено лепными кругами. Почему этими местами не
воспользовались, трудно сейчас сказать. Может, по той причине, что в ту пору
городские часы уже были. Они находились на выступающей части дома № 42 по
проспекту Ленина. Часы эти оказались недолговечными. Уже много десятилетий
место, где некогда был их циферблат, закрыто щитом...

\ii{01_12_2018.stz.news.ua.mrpl_city.1.gorodskie_chasy_gde_ranshe_uznavali_vremja_mariupolcy.pic.2}

1 ноября 1965 года раздался первый удар часов, установленных на только что
построенном Доме связи, циферблат и стрелки были изготовлены из нержавеющей
стали в цехах завода \enquote{Тяжмаш}, смонтировали их рабочие СУ-7 треста
\enquote{Ждановжилстрой}, монтаж механизма выполнили электрики городского узла
связи.  Отсчет времени через каждые 15 минут и мелодичный перезвон перед этим
исполняли 8 мощных репродукторов. Но, видимо, круглосуточное напоминание о
неудержимом течении времени надоело обитателям близлежащих домов, и часы на
Доме связи умолкли. В 70-х годах соорудили новое здание железнодорожного
вокзала. Его украшением стал большой циферблат башенных часов...

\ii{01_12_2018.stz.news.ua.mrpl_city.1.gorodskie_chasy_gde_ranshe_uznavali_vremja_mariupolcy.pic.3}

Вроде бы приборами для измерения времени мариупольцы были оснащены в
достаточном количестве. Но мечта обрести главные часы города не оставляла их. И
в самом начале XXI века на доме №75 по проспекту Мира у подземного перехода
появились электронные часы, которые несколько раз переделывались.

\textbf{Читайте также:} 

\href{https://mrpl.city/news/view/mariupol-za-ravnopravie-gorod-prisoedinitsya-k-evropejskoj-hartii-ravenstva-zhenshhin-i-muzhchin}{%
Мариуполь за равноправие: город присоединится к Европейской хартии равенства женщин и мужчин, Анастасія Папуш, mrpl.city, 29.11.2018}

\ii{01_12_2018.stz.news.ua.mrpl_city.1.gorodskie_chasy_gde_ranshe_uznavali_vremja_mariupolcy.pic.4}

Пару лет назад за счет семьи Сапаровых были приобретены и установлены на здании
бывшего реального училища (ул. Николаевская, 7/17) часы с боем и традиционным
для башенных часов круглым циферблатом.

\ii{01_12_2018.stz.news.ua.mrpl_city.1.gorodskie_chasy_gde_ranshe_uznavali_vremja_mariupolcy.pic.5}

Николай Сапаров рассказал, что они изготовлены в Германии. Специальное
электронное устройство обеспечивает высокую точность их хода. Более того, это
устройство в случае аварийного отключения электрического питания привода
стрелок при возобновлении подачи показывает реальное время. Как будто ничего не
произошло. Звук курантов натуральный, а не магнитофонная запись, как иногда
бывает...

\ii{01_12_2018.stz.news.ua.mrpl_city.1.gorodskie_chasy_gde_ranshe_uznavali_vremja_mariupolcy.pic.6}
