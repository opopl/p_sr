%%beginhead 
 
%%file 27_04_2021.fb.arximisto.2.pervoj_gimnazii_mrpl_status_pamjatki_initiative
%%parent 27_04_2021
 
%%url https://www.facebook.com/arximisto/posts/pfbid02wmB4AyjWVm4q8QKvNNdqQFnqhympj4uno4eDuBkKhcVFgC29EM3MQWDMZCwr6D1zl
 
%%author_id arximisto
%%date 27_04_2021
 
%%tags 
%%title Первой гимназии Мариуполя – статус памятки архитектуры!
 
%%endhead 

\subsection{Первой гимназии Мариуполя – статус памятки архитектуры!}
\label{sec:27_04_2021.fb.arximisto.2.pervoj_gimnazii_mrpl_status_pamjatki_initiative}

\Purl{https://www.facebook.com/arximisto/posts/pfbid02wmB4AyjWVm4q8QKvNNdqQFnqhympj4uno4eDuBkKhcVFgC29EM3MQWDMZCwr6D1zl}
\ifcmt
 author_begin
   author_id arximisto
 author_end
\fi

Первой гимназии Мариуполя – статус памятки архитектуры! Инициатива общественных
активистов

\#новости\_архи\_города

Общественные активисты инициируют предоставление Первой гимназии (бывшей
Мариинской женской гимназии) статуса памятки архитектуры и истории. Об этом
заявили Ярослав Федоровский, руководитель проекта реставрации старинных дверей
Мариуполя, и Андрей Марусов, директор общественной организации
\enquote{Архи-Город}.

В этом году Первая гимназия отмечает свое 145-летие. Она является самой старой
постоянно действующей средней школой Мариуполя. Ее здание, воздвигнутое в 1894
году, является замечательным архитектурным памятником города, как считает А.
Марусов.

Однако киевские разработчики историко-архитектурного опорного плана Мариуполя
почему-то не включили его в список потенциальных памяток архитектуры. Хотя
авторы первого опорного плана 1990-х годов рекомендовали это сделать.

Между тем, статус памятки архитектуры – это охранная грамота для здания, как
поясняет А. Марусов. Его нельзя будет разрушать или изменять, без согласования с
департаментом культурно-общественного развития горсовета. Такой статус также
делает здание и всю гимназию \enquote{видимыми} и узнаваемыми как для горожан,
так и для гостей города.

Как подчеркивают общественные активисты, горожане имеют право самостоятельно
инициировать предоставление Мариинке такого статуса. Соответствующая процедура
прописана в законе \enquote{Об охране культурного наследия} (ст.14) и Порядке учета
объектов культурного наследия, утвержденном Минкультуры.

Заявку на предоставление статуса может подавать общественная организация, среди
уставных задач которой стоит \enquote{охрана культурного наследия} (например,
ГО \enquote{Архи-Город}). Заявка должна включать историческое и архитектурное
обоснование.  Его авторами могут быть только специалисты с учеными степенями.

Финальный документ – учетную документацию на здание-па\hyp{}мятку – может
разрабатывать только организация с соответствующим опытом и специалистами.
Стоимость ее услуг зависит от сложности объекта. Например, учетная документация
на Дом Нильсена обошлась городскому бюджету в 29 тыс. грн.

Мы нашли опытных одесских архитекторов, которые согласились разработать
необходимую документацию для Мариинки всего за 16 700 грн., включая проезд
специалистов, как сообщил Я. Федоровский. 

Более того, мы уже заручились предварительной поддержкой благотворителей на сумму в 15 000 грн.

Вот со-авторы подарка к 145-летию Мариинки: общественная организация
\enquote{Національна вимога} (6500 грн.; глава организации Алина Юнге, мариупольчанка)
и коммунальное коммерческое предприятие Мариуполя \enquote{m.EHUB} (5 000 грн.;
директор предприятия Мария Слета). Я также внесу личные 3 500 грн. для этой
цели, как сказал Я. Федоровский.

Мы чрезвычайно признательны нашим благотворителям за помощь, как заявил А. Марусов.

Осталось собрать 1 700 грн.

Мы обращаемся ко всем неравнодушным мариупольцам – станьте со-авторами подарка
Мариинке! Любой взнос – ценен. Все средства будут собираться на счете
общественной организации \enquote{Архи-Город}.

Реквизиты счета следующие: 

Получатель: ГО \enquote{Архі-Місто}, код ЄДРПОУ 41902283

р \ р UA453351060000026003300127486 

в Донецьке обласне управління АТ \enquote{Ощадбанк}

МФО банку 335106

Призначення платежу: \enquote{благодійний внесок на збереження Першої гімназії}
(обязательно укажите предназначение платежа, чтоб у нас не возникли проблемы с
налоговой!!!).

Заранее благодарим всех благотворителей!

Мы будем информировать о поступлении пожертвований на странице ГО \enquote{Архи-Город}.
\footnote{\url{www.fb.com/arximisto}}

Мы также будем рады сообщить более подробную информацию об инициативе. Наши
контакты: 096 463 69 88 (телефон Андрей Марусов), Yaroslav Fedorovskyi 

\#перша145річниця 

\#мариинке\_охранная\_грамота

======================

\begin{itemize} % {
\item Видео программы \enquote{Ранок Маріуполя}, посвященной инициативе \footnote{\url{https://cutt.ly/dvYC45z}}
\item ФБ сторінка громадської організації \enquote{Національна вимога} \footnote{\url{https://www.facebook.com/nacvymoga}}
\item Коммунальное коммерческое предприятие \enquote{m.EHUB} \footnote{\url{https://cutt.ly/vbw0YSO}}
\item Закон України \enquote{Про охорону культурної спадщини} \footnote{\url{https://cutt.ly/XvYVmx8}}
\item Порядок обліку об'єктів культурної спадщини, затверджений наказом Міністерства культури України від 11.03.2013 № 158 %
\footnote{\url{https://cutt.ly/mvYNqpF}}
\end{itemize} % }

Групповое фото взято на Инстаграм аккаунте программы \enquote{Ранок Маріуполя}. Слева
направо: @Dmitry Gritsenko, ведущий программы \enquote{Ранок Маріуполя}, Андрей
Марусов, директор ГО \enquote{Архи-Город}, Ганна Дежец, директор музея Первой гимназии,
Ярослав Федоровский, руководитель проекта реставрации старинных дверей
Мариуполя, Ксения Мисюревич, ведущая программы \enquote{Ранок Маріуполя}
