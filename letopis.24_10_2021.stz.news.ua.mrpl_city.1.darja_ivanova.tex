% vim: keymap=russian-jcukenwin
%%beginhead 
 
%%file 24_10_2021.stz.news.ua.mrpl_city.1.darja_ivanova
%%parent 24_10_2021
 
%%url https://mrpl.city/blogs/view/darya-ivanova-nikoli-ne-mriyala-buti-aktrisoyu
 
%%author_id demidko_olga.mariupol,news.ua.mrpl_city
%%date 
 
%%tags 
%%title Дар'я Іванова: "Ніколи не мріяла бути актрисою!"
 
%%endhead 
 
\subsection{Дар'я Іванова: \enquote{Ніколи не мріяла бути актрисою!}}
\label{sec:24_10_2021.stz.news.ua.mrpl_city.1.darja_ivanova}
 
\Purl{https://mrpl.city/blogs/view/darya-ivanova-nikoli-ne-mriyala-buti-aktrisoyu}
\ifcmt
 author_begin
   author_id demidko_olga.mariupol,news.ua.mrpl_city
 author_end
\fi

\ii{24_10_2021.stz.news.ua.mrpl_city.1.darja_ivanova.pic.front}

Нещодавно в рубриці \emph{\enquote{Маріуполь театральний}} в рамках розважальної програми
\enquote{Ранок} на Маріупольському телебаченні я розповідала про актрису Донецького
академічного обласного драматичного театру (м. Маріуполь) Дар'ю Іванову. Її
роль у виставі \enquote{Біла ворона} давно підкорила всіх найвибагливіших театралів
міста. До речі, за талановите виконання ролі Жанни д'Арк вона отримала перемогу
на Другому обласному відкритому фестивалі театрального мистецтва Театральна
брама – 2021 в номінації \emph{\enquote{За високохудожнє виконання ролі}}. Унікальний життєвий
і творчий шлях \emph{\textbf{Дар'ї Іванової}} не може не вразити, тому я вирішила присвятити їй
окремий матеріал і в своєму блозі.

\ii{24_10_2021.stz.news.ua.mrpl_city.1.darja_ivanova.pic.1}

Актриса народилася в Маріуполі. Батько працював слюсарем, мати – медичною
сестрою. З дитинства Дар'я була дуже активною і творчою дівчинкою. Ще в
дитячому садочку та школі на всіх святах вона завжди щось декламувала. На
уроках літератури за виразне читання віршів Дар'я отримувала лише п'ятірки з
трьома плюсами та написом \emph{\enquote{молодець!}}. Цікаво, що в дитинстві вона зовсім не
мріяла стати актрисою. Насправді дуже хотіла співати на великих сценах. У
середній школі вчителька дала завдання написати твір про те, ким учні хочуть
стати в майбутньому. Даша написала, що хоче стати співачкою. Але коли вчителька
зачитала її твір, деякі однокласники почали глузувати з дівчинки та її мрії.
Дар'ю це дуже образило і вона вирішила більше ніколи не зізнаватися ким хоче
бути насправді. Відтоді майбутня актриса у творах зазначала, що стане
вчителькою чи лікаркою, тільки щоб більше ніхто з неї не глузував.

\ii{24_10_2021.stz.news.ua.mrpl_city.1.darja_ivanova.pic.2}

На театральний шлях дівчина стала завдяки мамі. Коли їй виповнилося 7 років,
мама відвела її до Палацу культури \enquote{Металургів} (зараз – МПК \enquote{Український
дім}), щоб донька займалася спі\hyp{}вом, але жінка помилися дверцятами і Дар'я
потрапила до театрального гуртка \emph{\enquote{Ребята}}. Керівницею тоді була \emph{Валентина
Михайлівна Киреєва}, яка одразу ж помітила винятковий талант дівчини. До гуртка
актриса ходила досить тривалий час – десь 9 років. Їй давали переважно ролі
гарних дівчаток, але Даші в той час хотілося зіграти якогось характерного
персонажа – відьму чи бабу Ягу. Проте вона була маленькою і тендітною з дуже
мелодійним, тонким і дзвінким голосом, тому ролі давали під її зовнішність –
світлі і добрі. У цьому гуртку дівчина часто співала. Голос Дар'ї Іванової і
досі можна почути в будинку культури Український дім, адже фонограма деяких
пісень збереглася і її включають і сьогодні на новорічні свята.

Пізніше майбутню актрису запросив до себе у вокально-інстру\hyp{}ментальний ансамбль
\enquote{Mід} в Будинок культури ім. Карла Маркса \emph{Михайло Давидов}. Деякий час Дар'я
намагалася поєднувати і гурток, і цей ансамбль але вже в старших класах вона
почала готуватися до іспитів, часу залишалося менше, тому вона вирішила ходити
тільки до ансамблю.  

Мати Дар'ї, хотіла, щоб її доньки (у Даші є старша сестра) отримали вищу
освіту, тому наша героїня вступає до Маріупольського державного університету
(тоді – Маріупольський гуманітарний інститут) на історичний факультет. Але дуже
швидко вона зрозуміла, що історія – це зовсім не її покликання. За весь час
навчання вона займалася купою різних справ, тільки не історією. Дівчина
закінчила курси шиття одягу, співала в ресторані і вже під час практики в школі
вона остаточно зрозуміла, що це не її професія, і що вчителькою історії вона
точно ніколи не буде.

\ii{24_10_2021.stz.news.ua.mrpl_city.1.darja_ivanova.pic.3}

Під час декрету Даша зрозуміла, що дуже сильно скучила за своїми виступами на
сцені, тому почала шукати можливості, як все ж таки повернутися до заняття, яке
робило її по-справжньому щасливою. Якось Дар'я сиділа зі своєю дочкою на
лавочці у парку ім. Петровського і до них підійшла ростова лялька, яка
роздавала листівки. Дівчина побачила в них запрошення до Першої театральної
школи-студії, куди актриса вирішила одразу ж відправитися. Керівниками студії
були \emph{Олексій Гнатюк та Лідія Хаджинова}. Тоді студія знаходилися в приміщенні,
де були сольові печери, а оскільки у доньки нашої героїні був хронічний
бронхіт, вона вирішила, що буде дуже правильно і корисно, якщо дочка
перебуватиме там, в той час, як сама актриса займатиметься справою, що
приносить їй справжнє задоволення. За три роки, поки Дар'я навчалася в студії,
колектив поставив три мюзикли. Для жінки це був дуже насичений і цікавий час.

І все ж доля повела Дар'ю далі, адже їй судилося стати актрисою професійного
театру. У 2014 році драматичний театр міста набирав у свою трупу акторів без
освіти і наша героїня наважилася спробувати свої сили на найбільшій сцені
Маріуполя. Відтоді для жінки почався зовсім новий етап в житті. Серед 15–18
осіб, які хотіли потрапити до театру обрали п'ятьох і Дар'я потрапила в цю
п'ятірку. Спочатку актриса була задіяна майже в кожній виставі в якості
масовки, адже артистів не вистачало.

\ii{24_10_2021.stz.news.ua.mrpl_city.1.darja_ivanova.pic.4}

Гра багатьох акторів, серед яких \emph{\textbf{Анатолій Шевченко}}, \emph{\textbf{Сергій Забогонський}}, \emph{\textbf{Олена
Біла}}, \emph{\textbf{Артур Войцеховський}}, \emph{\textbf{Вадим Єрмішин}}, дуже імпонувала Дар'ї. Спостерігаючи
за колегами вона намагалася у них вчитися. Насправді паралельно з роботою
актриса постійно навчалася. Зокрема, 2 роки Дар'я отримувала знання в студії
\emph{\textbf{Анжеліки Добрунової}}. А потім здобула й вищу акторську освіту. Вона закінчила
Харківську державну академію культуру за спеціальністю \enquote{Магістр зі сценічного
мистецтва}. 

Ключовими актриса вважає для себе свої ролі у виставах \emph{\enquote{Біла ворона}}, \emph{\enquote{Жах}},
\emph{\enquote{Maidan inferno}}, \emph{\enquote{Останній подвиг Ланцелота}}. Наразі у актриси є невеличкі
ролі і в кіно (серіали \enquote{Слід}, \enquote{Сліпа}, \enquote{Люся інтерн}).

Є одна роль, яку актрисі дуже хотілося б  ще зіграти. Це роль Меррі Поппінс.
Актриса уявляє мюзикл з піснями, де вона грає цю витончену Леді Досконалість.

Цікаво, що Дар'я ще й сама пише пісні та створює музику. Вона віддає ці пісні в
студію, де пишуть аранжування. Можливо, вже наприкінці 2021 року ми зможемо
почути пісні Дар'ї Іванової.

Насправді наша героїня абсолютно не вміє відпочивати, вона постійно у русі і,
як це не дивно, зовсім не боїться чоловічої роботи. Самотужки може зробити
ремонт, при цьому результат приємно вражає. Багато чого в квартирі актриси
зроблено її руками. Оскільки у неї є і кішка, і собака, часто псувалися
шпалери, тому жінка вирішила віднайти новий спосіб прикрасити стіни – вона сама
почала їх шпаклювати в дуже цікавих техніках, завдяки яким кімнати стали дуже
яскравими і стильними.

\ii{24_10_2021.stz.news.ua.mrpl_city.1.darja_ivanova.pic.5}

Донька актриси – Ніка – наразі навчається у 10 класі. Її теж цікавить акторська
професія. Дар'я любить подорожувати, мріє потрапити до Барселони. Також вона
вміє готувати випічку, до речі, дуже смачну. Також Даша займається хупінгом
(заняття з хулахупом, спортивними обручами). У виставі \enquote{Прибульчик} актриса
використала цю навичку.

Колеги Дар'ю Іванову цінують як за людські якості, так і за унікальний талант.
Зокрема, \emph{\textbf{Артур Войцеховський}} наголосив, що 
\begin{quote}
\em\enquote{вона добра і чесна людина, чудовий
друг і справжнє джерело та генератор творчих проєктів. Дар'я може заразити
якоюсь ідеєю і віх надихнути на її реалізацією. Так вийшло з проєктом \enquote{Ваша
Леся}}. 
\end{quote}
Цієї осені Артур і Дар'я закінчують спільну короткометражку, яку ми всі скоро
побачимо.

Актриса дуже любить подорожувати і обожнює рідне місто. Дуже сподівається, що
найближчим часом Кальміуський район теж отримає більший розвиток на рівні з
іншими районами. Загалом вважає Маріуполь одним з найкращих міст в Україні.

\emph{\textbf{Хобі:}} хупінг, пише пісні та музику.

\emph{\textbf{Улюблена книга:}} \emph{Кен Кізі \enquote{Над зозуленим гніздом}, Джордж Орвелл \enquote{1984}.}

\emph{\textbf{Улюблений фільм:}} 

\begin{quote}
\em\enquote{Одного разу в Голлівуді}, \enquote{Готель Гранд Будапешт}, \enquote{Москва
сльозам не вірить}, \enquote{Втеча з Шоушенка}, \enquote{Біжи, Форест, біжи}.
\end{quote}

\emph{\textbf{Побажання маріупольцям:}}

\begin{quote}
\em\enquote{Якщо у вас є мрія, треба обов'язково іти до неї. Вона може здійснитися одразу,
а може – через кілька років, якщо не здаватися.  Відсутність часу, грошей,
можливостей, допомоги ззовні не можуть бути перепоною. При бажанні, все це
можна знайти, якщо діяти. Мрії обов'язково треба переводити із стану духовного
в стан матеріальний. І ще, хочеться побажати бути більш людяними один до
одного. Посмішка, рука допомоги і добре слово, можуть зробити щасливою будь-яку
людину}.
\end{quote}
