% vim: keymap=russian-jcukenwin
%%beginhead 
 
%%file slova.znanie
%%parent slova
 
%%url 
 
%%author 
%%author_id 
%%author_url 
 
%%tags 
%%title 
 
%%endhead 
\chapter{Знание}

%%%cit
%%%cit_pic
%%%cit_text
Сегодня, в День России, выкладываю два программных текста. Программными я
называю тексты, которые написаны не мною, но которые я запасливо сохранила и
время от времени повторяю, ибо повторение - мать учения. Текст номер один
называется \enquote{Что должен \emph{знать} каждый русский}. Первоисточник его давно
затерялся, но копия сохранена у меня в основном блоге, а сегодня я переношу её
и сюда. Итак, что же должен знать каждый русский с младых ногтей? Спрашивали -
отвечаем:
\begin{itemize}
  \item 1. Государства Киевская Русь никогда не существовало. Этот термин был введён в обиход для временной градации периодов истории Древнерусского государства.
  \item 2. Первая столица Руси Ладога была основана словенами около 700 года н э.
	\item 3. Из 7 городов Рюриковой Руси, упоминаемых в Повести временных лет, 6
					городов находится на территории России, 1 - на территории Белоруссии.
								На украине - нет ни одного. Города Древней Руси это: Ладога,
								Исборск, Белоозеро, Новгород, Ростов, Муром, Полоцк
\end{itemize}
%%%cit_comment
%%%cit_title
\citTitle{Что должен знать каждый русский}, 
Школа Экзорцистов, zen.yandex.ru, 12.06.2021
%%%endcit

