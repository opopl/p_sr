%%beginhead 
 
%%file 13_01_2023.fb.arestovich_alexei.1.prodolzhaem_temu_sra
%%parent 13_01_2023
 
%%url https://www.facebook.com/alexey.arestovich/posts/pfbid0agrQLYKMR6K8g1z1HS2yqgeCkoW5J2wM4o1ooQTMUX5yYaKxXe8xX4Un27JcwV42l
 
%%author_id arestovich_alexei
%%date 13_01_2023
 
%%tags 
%%title Продолжаем тему срачей
 
%%endhead 

\subsection{Продолжаем тему срачей}
\label{sec:13_01_2023.fb.arestovich_alexei.1.prodolzhaem_temu_sra}

\Purl{https://www.facebook.com/alexey.arestovich/posts/pfbid0agrQLYKMR6K8g1z1HS2yqgeCkoW5J2wM4o1ooQTMUX5yYaKxXe8xX4Un27JcwV42l}
\ifcmt
 author_begin
   author_id arestovich_alexei
 author_end
\fi

- Продолжаем тему срачей.

Сейчас скажу пару слов на простонародном, так, чтобы дошло. 

———-

Чем больше у человека претензий к другим, тем худшего он мнения о себе.

Доебываются до других те, кто сомневается в своей правоте.

Человеку плохо с собой.

Человек не выдерживает остаться с собой наедине.

Невозможность терпеть внутреннее напряжение требует выбросить его наружу.

Психика работает путём подбора и обращения к символам, она просто не умеет
работать иначе.

Мы выбираем себе виноватого, вся «вина которого» в том, что он демонстрирует
тип поведения, который мы себе запретили (из-за чего себя и не переносим).

Можно забить внимание к себе зависанием в экранах, наркотиками, играми, и
другими способами заглушить негативные внутренние сигналы, а можно сбежать в
претензии в другим.

Задача встречи с собой, задача - главная в жизни человека.

Потому, что без этого не будет ни жизни, ни человека.

А большинство из нас не имеет опыта, прямого, чувственного опыта, не то что
«быть собой», даже знакомства с собой.

Спросите себя - какие чувства я испытываю, когда я поступаю, как я?

Так вот дорога к себе начинается с того, чтобы отъебаться от других.

Люди, сумевшие это сделать, бывают шокированы количеством ресурса (времени,
денег, внимания), которые они сливали в доебки до других.

Все эти миллионы долларов драгоценнейшего из ресурсов во Вселенной - внимания,
были Вами выброшены под ноги человека, который может даже не знать о Вашем
существовании. 

А могли бы быть потрачены Вами на самих себя.

Любить нужно себя.

Вкладывать нужно в себя.

Внимание должно распределяться сначала на себя, потом на других.

Человек себя не знает. И прежде чем заниматься другими должен себя найти.

Все это - время. Все это труд. Все это - учеба.

Все это воля и силы на то, чтобы дойти, познакомиться, стать и начать жить
по-новому.

А где из взять, если все они выброшены псам?..

Теперь хочу сказать пару слов тем, кто сильно переживает из-за негативного
мнения других. 

Вам нужно понять одну простую вещь.

Если человек до Вас (или кого-то другого) доебался, это означает, что он сам
себя подсознательно считает чмом.

Короли (и просто психически здоровые люди) не доебываются.

А сильно ли Вас должна беспокоить оценка человека, который сам себя считает
чмом?..

Да еще и неудачником - ровно в той сфере, по которой и доебался.

Если до Вас доебались, это всегда (услышите меня) всегда означает, что
доебавшийся - чмо, по факту доеба. 

И обращаться Вам с ним нужно, как с чмом.

Не придавайте ему значения, большего, чем он сам себе предает, это ему
повредит.

Не позволяйте себе поверить, что доеб - это критика, а «…критика - полезна». 

Здоровая психика не критикует людей.

Здоровая психика может критиковать действия, людьми она интересуется.

Признаки здорового человека - здоровый интерес к людям, интерес к их
разнообразию.

Критика другого человека - это всегда болезнь Вашей психики. 

Вашей неспособности мыслить эволюционно, в логике развития. 

Критика Вас другими людьми - это их болезнь, признак психической
ненормальности. 

С пациентами не спорят. 

Пациентам не позволено оценивать врачей.

Относитесь к себе, как к врачу.

Врач не позволяет пациентам оценивать себя.

Никогда, никогда, никогда не сомневайтесь в своей правоте.

В каждый момент времени Вы поступаете наилучшим образом - на условии наличной
на тот момент у Вас информации.

Если Вы хотите поступать лучше, Вам нужно не ругать себя (и тем более, не
позволять ругать себя другим) а улучшать качество информации, которая Вам
доступна и качество Вашей работы с ней.

Это все - труд. Это - годы работы.

И на этот труд Вам понадобится очень много сил, внимания и энергии, которые Вы
сливаете в самобичевание или «бичевание» других.

Когда обнаружите свою жизнь, слитой в унитаз в результате подобных действий,
претензий предъявлять будет некому.
