% vim: keymap=russian-jcukenwin
%%beginhead 
 
%%file 20_09_2021.fb.nikonov_sergej.1.bilchenko_pirat
%%parent 20_09_2021
 
%%url https://www.facebook.com/alexelsevier/posts/1586017738410202
 
%%author_id nikonov_sergej,bilchenko_evgenia
%%date 
 
%%tags bilchenko_evgenia,poezia
%%title БЖ. Пират
 
%%endhead 
 
\subsection{БЖ. Пират}
\label{sec:20_09_2021.fb.nikonov_sergej.1.bilchenko_pirat}
 
\Purl{https://www.facebook.com/alexelsevier/posts/1586017738410202}
\ifcmt
 author_begin
   author_id nikonov_sergej,bilchenko_evgenia
 author_end
\fi

Ещё один стих Евгении Витальевны Бильченко. А я не комментирую и не реагирую на комменты.


\ifcmt
  tab_begin cols=2

     pic https://scontent-yyz1-1.xx.fbcdn.net/v/t1.6435-9/242542980_1586016351743674_7645798819020589357_n.jpg?_nc_cat=110&_nc_rgb565=1&ccb=1-5&_nc_sid=730e14&_nc_ohc=NQW6mqs-RNQAX8Fp5UF&_nc_ht=scontent-yyz1-1.xx&oh=86677ef8375de97d23cf93a03a1e6fa3&oe=616D0E3E

     pic https://scontent-yyz1-1.xx.fbcdn.net/v/t1.6435-9/242536433_1586016538410322_7915976875507664061_n.jpg?_nc_cat=103&_nc_rgb565=1&ccb=1-5&_nc_sid=730e14&_nc_ohc=HP6DyX7_RPgAX9rXcry&_nc_ht=scontent-yyz1-1.xx&oh=86c607bca269f01c16e02ddef09fef27&oe=616CB8C4

  tab_end
\fi

БЖ. Пират

И не потому, что ты - придворный поэт и близок, блин, к Императору.
И не потому, что я - безработный бомж и падаю с эскалатора.
И не потому, что мне от тебя помощь нужна разведческая.
И не потому, что мы - ты и я - последнее человечество.
И не потому, что в далёкой Праге, где Голем и трдло нелепое,
Я приняла тебя в полумраке трдельника за Прилепина.
И не потому, что глава твоя, мой Розенбаум, - лысая.
И не потому, что рыжая грива моя от седин вся высохла.
И не потому, что ты пьешь, как конь: стопки в рядочек с бехеровкой.
И не потому, что ты - Пушкин, а я гожусь тебе в Кюхельбекеры.
И не потому, что на нашей Украйне, той ещё, - хата-мазанка.
И не потому, что перед начальством я тебя, чмо, отмазывала.
И не потому, что твои стихи - мужские и настоящие,
Как Святой Георгий перед ударом в темя Кощея ящура.
И не потому, что в ящики нас обоих сыграет Родина.
И не потому, что у Расторгуева - Растеряева - смерть - смородиновая.
А потому, что бомж - это ты, в рубище незалатанном.
Это ты, выпив транквилизатор, падаешь с эскалатора.
Это ты, получив удары в живот, в ответ на фашню смеялся так,
Как учили нас дедушка Ваня, Карлсон и Зоя Космодемьянская.
А ещё я знаю твой страшный секрет: что собака была "Пират" у тебя.
А ещё ты знаешь мой страшный секрет: та собака меня порадовала.
А ещё мы помним ночной трамвай, где, Егора-царя глашатаев,
Кидало нас, нажранных, правдой в прах: пражцев от нас отшатывало.
Вот и всё, что мне надо, блин, от тебя: блинчики со сметаною.
И не потому, что я умереть боюсь, я же ведь вся - метановая:
Из моего кармана шахтера, оттопыренного взрывчаткою,
Проступает моя поэзия - русская, непочатая.
И не потому, что не дорасти мне до МХАТа, тебя и гения
Твоего, перед коим я - жалкий червь, гнида неразумения.
Я дорасту до твоих глубин, в высоту "Москва-Сити" выспренных...
Но ранят меня, как мешок с дерьмом, не ссы, Пират, - сдюжи выстрелить.
19 сентября 2021 г.
Фото: Влад Маленко, retried from: Rustars.tv
