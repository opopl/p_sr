%%beginhead 
 
%%file 01_03_2023.fb.kipcharskij_viktor.mariupol.1.r_k_tomu_bulo_take__
%%parent 01_03_2023
 
%%url https://www.facebook.com/permalink.php?story_fbid=pfbid0WZwvdJBJhNaBHe7aPUZDqf3GcXvQDNX5Pe48Yr9FcaCYERraWdN5cMmivm82xpQPl&id=100006830107904
 
%%author_id kipcharskij_viktor.mariupol
%%date 01_03_2023
 
%%tags 01.03.2022,mariupol,mariupol.war,dnevnik
%%title Рік тому було таке:  День 6 - 1.03.22. Вівторок
 
%%endhead 

\subsection{Рік тому було таке:  День 6 - 1.03.22. Вівторок}
\label{sec:01_03_2023.fb.kipcharskij_viktor.mariupol.1.r_k_tomu_bulo_take__}

\Purl{https://www.facebook.com/permalink.php?story_fbid=pfbid0WZwvdJBJhNaBHe7aPUZDqf3GcXvQDNX5Pe48Yr9FcaCYERraWdN5cMmivm82xpQPl&id=100006830107904}
\ifcmt
 author_begin
   author_id kipcharskij_viktor.mariupol
 author_end
\fi

Рік тому було таке: 

День 6 - 1.03.22. Вівторок.

З ночі не було зв'язку. Від слова "зовсім": інтернету, телефону, телевізору,
FM-радіо.

На ранок зв'язок повернувся: рашисти у Урзуфі...

Як??? Там же база Азову!!! Не може такого бути!!!

Серед усіх смартфонів лише Нокія 5310 (напередодні війни віддав онуку, аби
звикав бути на зв'язку) тримала зв'язок. Отож, шукайте старі кнопкові телефони,
перехідники з міні-мікро, заряджайте. Бо воно - реально: Nokia connected
People!!!

Якщо знайдете зайві: передайте хлопцям.

Гупало.

Ну це вже як звичка: наші захисники нас боронять.

Недобра звістка: хто не купить алкоголь сьогодні до 12:00 - той буде жити на
тверезу голову. Оце мене розбурхало!!!

Ну, на власних запасах чоловіки  ще з тиждень можуть протриматися.., а потім?
З'являться тверезі, злі чоловіки - хана нападникам!

Сьогодні армія терористів рф цинічно обстріляла житлові райони Маріуполя.
Внаслідок військового теракту постраждали кілька будинків по бул. Шевченка та
школа №16 у тому ж районі. Крім того, обстріл знову зазнали житлові будинки в
Лівобережному районі, а також територія КП «Комунальник». Число постраждалих
уточнюється.

Телеграм міськради 

P.S. Від нас до тої школи 3 км по прямій

Що ще? 

На випадок відключення газо- елетропостачання умовив жінок зазделегідь зварити
м'ясо - варене може лежати в морозильнику не гірше за сире, бульйон також, а от
якщо щось - на вогнищі зварити супчик із бульйону буде набагато швидше, ніж із
сирого.  Головне - швидше! Бо останні роки пішла мода робити паркани із
негорючих матеріалів.

Мало не весь день невістка пекла бісквіти.

На разі все - маю допомогати онучечці робити браслет із бісеринок.

А за відсутності зв'язку написалося отаке:

Чудовий сон мені наснився:

Я в мирним часі опинився -

Не чути пострілів гармат

І "град" - лише звичайний град.

Сирени не ревуть тривоги

І танки не псують дороги...

Не гинуть люди, не страждають,

Дітей в підвали не ховають...

Тиша навкруги...

Здохли вороги...

А в душі луна:

Скінчилася війна.

Дай Боже, щоб збулися сни

З початком нової весни...

%\ii{01_03_2023.fb.kipcharskij_viktor.mariupol.1.r_k_tomu_bulo_take__.cmt}
