% vim: keymap=russian-jcukenwin
%%beginhead 
 
%%file 08_12_2021.tg.tkachev_jurij.1.vtorzhenie_putin_usa.cmt
%%parent 08_12_2021.tg.tkachev_jurij.1.vtorzhenie_putin_usa
 
%%url 
 
%%author_id 
%%date 
 
%%tags 
%%title 
 
%%endhead 
\subsubsection{Коментарі}

\ifcmt
  ig https://i2.paste.pics/2156ebd5fe0b7d3b676930279b757062.png
  @width 0.4
\fi

\begin{itemize} % {
\iusr{Ruslan}

Допустим, не только Путина. Но да, Путин один из главных защитников Украины.
Без Путина Украины давно уже не было бы.

\iusr{Вячеслав Шмарин}

Так наоборот, весь запад всеми доступными способами подталкивает Путина - ну
давай уже нападай, давай захвати ... Это и есть основная идея. Потому что
противостояние уже идет , но оно идет на политическом, экономическом  и
идеологическом фронте. И Штатам очень надо завязать нас на конкретике, чтобы
все остальные фронты были оголены.

\iusr{Kіевлянинъ}

Они этого не понимают. Политические украинцы до сих пор считают США нашими
партнёрами и братьями. У этих людей мозг испарился и пустая черепная коробка
которая рефлексирует на клич Слава Украини. За эти 8 лет народ деградировал и
оскотинился в край. Плюс пропаганда по ТВ и интернет СМИ помогает им верить в
параллельную реальность где Украина воюет с РФ и ее уважают в мировом масштабе.
Мои родители живо тому пример.

\iusr{Виталий Куртов}

Я вообще не понимаю в какой стране \enquote{Слуги}  живут) Но явно мы с ними в разных
Украинах живём) У них украинцев всё ещё 40 миллионов и против РФ партизнанить
будут старики, младенцы, жители Крыма и Донбасса)

\iusr{Вячеслав Шмарин}
\textbf{Kіевлянинъ}

Не так все просто. В 90-е Россию тоже захватили штаты. Их люди сидели во всех
кабинетах. И чтобы выкинуть их оттуда надо было очень кропотливо их
выдавливать. Потому что открыто им противостоять было невозможно. Сейчас такая
же ситуация на Украине.

\iusr{Vladique86}

Худшее, чем он бы столкнулся в этом случае - \enquote{экономические санкции}.

Эх, Юра-Юра! Как ты мог, о циник, забыть о КОПРОТИВЛЕНИИ, которое окажет в
едином порыве население чудо-государства? Как будут отчаянно бросаться на
\enquote{вторженцев} (правда, тех ещё уговорить надо вторгнуться, лол) наши земляки -
после многократно возросших тарифов, отката \enquote{активистам} с бизнеса  и битв за
урожай, украинизации, блокпостов \enquote{лиц без опознавательных знаков}, замерзающих
квартир и мега-пенсий! 

Как ты мог, о маловерный, усомниться в отчаянной борьбе за идеалы Майдана - то
есть за сайт \enquote{Миротворец}, пятилетние сроки за герб СССР на Facebook, Роттердам
Плюс и сгоревшие в банкопаде сбережения?

\iusr{Yury M}

Та ладно!!!? Как так то?? мы ж самую сильную армию в Европе создали! Мы же.. Мы
же сдерживаем российские войска, защищаем там всех в Европе от злого Путина!
вот.

\iusr{Александра Кормалева}

У меня складывается такое впечатление, что это игра штатов! Они своей игрой
толкают Украину на войну! И об этом знает Зеленский! Чем это может закончится
страшно и подумать! Но и бесконечно жить в подвешенном состоянии долго
продолжаться не может! Только один путь будет верным, это диалог Киева и
Донбасса!  Все посредники хотят что-то своего! Так что САМИ!

\iusr{Roma}

Нет, Западу нужно, чтобы Путин не мешал строить Антироссию. И это она нападет,
когда удобный момент возникнет. А пока её к этому готовят.

\iusr{motor unknown}

да всё будет хорошо, спите  спокойно)

\iusr{Вячеслав Шмарин}

Это вы оцениваете по западным меркам - выгодно/невыгодно. А Россия и Путин
живут по меркам справедливости. А если с этой точки зрения смотреть, то все
несколько сложнее. С одной стороны конечно и народ несет ответственность за то
что творится. Но далеко не полную.

\iusr{Вячеслав Шмарин}

Чтобы строить Антироссию - ее надо содержать, кормить и обогревать. А у запада
уже на себя денег не хватает. Поэтому он хочет чтобы тут все подрались, а запад
подливал бензинчику в огонь.

\iusr{MORGENSHTERN}

Вряд ли политические украинцы читают ваш блог.

\iusr{Александр}

\url{https://t.me/Shulyavskiy/190}

\obeycr
Не помню, но кажись, в тринадцатом году,
Транзитом Зрада Киев проезжала,
Дай, думает, по городу пройду,
Бывала раньше, но конкретно не гуляла,
Короче, ходит, смотрит, тратит наличман,
Красивый город, но видала интересней,
И Зрада выбрела, случайно, на Майдан,
А тот встречал ее лампадками и песней,
Что с нею сделалось, пером не описать,
Когда увидела на площади всю шоблу,
Бомжи, жлобы, на сцене мразей рать,
И все пропитано печалью, грустью, болью,
Пришлось уехать, много дел, и в том беда,
Но не прикажешь - не имеет сердце меры,
Вернулась Зрада, чтоб остаться навсегда,
И, кстати, сделала отменную карьеру...
\restorecr

\iusr{Telema}

0.01\% бандеровцев которые будут бухать в крыивках за деньги Запада. Вот какая
партизанская война ждет Путина. Но у Путина нет возможности. Тянуть всех это
придётся в социализм уйти.

\iusr{Roma}

Они строят. Каждый день это наблюдаю. И готовят её к войне.

\iusr{Telema}
И это плохо. Потому что я за дружбу, равенство, братство!

\iusr{Roma}
А кто будет разбираться, когда все ресурсы в руках нацистского государства?

\iusr{Вячеслав Шмарин}

Если мы считаем что народ братский, значит надо помогать. Но чтобы помогать
народ сам должен об этом заявить.

\iusr{Roma}

Да как бы Янукович в феврале 2014 года заявил. С нулевым результатом.

\iusr{Сергей Ившин}

\ifcmt
  ig https://i2.paste.pics/bab21fcf95151198379d570ca9588426.png
  @width 0.4
\fi

\iusr{Вячеслав Шмарин}

Идея неверна, долги надо набирать. И потом уже послать всех подальше. ПОтому
что прецеденты есть, например Великобритания подрезала деньги Венесуэлы. Так
что набирайте долгов и кредитов  - это пойдет западу на пользу, в другой раз не
будут лезть.

\iusr{Roma}

Народ? Как вы себе это представляете, всеукраинское Вече? Народ свою власть
через представительные органы власти осуществляет. Президент им и является.

\iusr{Telema}

1. Кому вам? А вам не надоело олигархов кормить?

2. То есть вы не считаете украинцев с русскими единым народом? Окей, критерий
различия?

\iusr{Юри Ткачёв}

куда писать? на чьё имя и адрес? от каждого представителя народа надо или хотя
бы от большинства представителей?

\iusr{Вячеслав Шмарин}

Нет ни разу не является. Все знают что реальной демократии нет и выбирают того
кого проплатили и подрекламировали. А вот когда народ заявит своим права и свое
желание мы обязательно увидим.

\iusr{Esve}

Каким образом заявить?

\iusr{Esve}
))))

\iusr{Кот в пальто}
Слава Путину и Украине тоже @igg{fbicon.face.smiling.eyes.smiling} 
Или так, Слава Путину, а потом уже Украине @igg{fbicon.beaming.face.smiling.eyes} 

\iusr{Roma}
И Путина в том числе?  @igg{fbicon.face.flushed}  А я догадывался.

\iusr{Александра Кормалева}

Юра, не дай Бог произнести имя Путина! Да еще в контексте спасителя! А агрессию
куда деть? Будут же друг на друга бросаться! Свидомые они такие!

\iusr{Telema}

Уже заявляет, все чаще. Но без поддержки народ задавят. Да, только начинается
процесс профсоюзов, их мало, но они уже есть! Но без поддержки нереально
сбросить ярмо Запада.

\iusr{Виктор}
\textbf{Kіевлянинъ}

Не согласен. Старшее поколение 60+ может ещё и верит в эту галиматью, но очень
многие из тех кто верил в Майдан, сейчас открыто его называют переворотом, и
перестали верить в искренность Запада

\iusr{Вячеслав Шмарин}

Старая испытанная схема. Подполье, группы, консолидация, и уже когда
действительно будет массовость - выступление. Сейчас власть слаба как никогда,
любой сильный толчок и посыпется. И дальше будет только слабеть, потому что
держится за счет штатов, а они сами слабеют. Поэтому и войну не надо начинать,
власть сама осыпется как в Афганистане.

\iusr{Вячеслав Шмарин}

да все верно. своих сбросить сможете, а вот запад отогнать только с помощью
России. Но Россия поддержит, не сомневайтесь

\iusr{Юри Ткачёв}
\textbf{Вячеслав Шмарин}
и всё-таки вы не уточнили, как именно должно выглядеть обращение? Расскажите всё-таки

\iusr{Ivan Petrov}
\textbf{Виктор}

\enquote{Искренность Запада} - это оксюморон. Запад столетиями весь мир грабит под
сказки о приносимой им цивилизации, свободе и демократии.

\iusr{Roma}
\textbf{Вячеслав Шмарин}
Да как-то сомневаемся после 2014 года.

\iusr{Artyom}
\textbf{Юрий Ткачёв}

Само собой. Как однажды сказал Президент РФ: \enquote{Проснулся утром - подумай, что ты
сделал для Украины}

\iusr{Artyom}

Обращение Виктора Владмировича Медведчука к Владимиру Владимировичу Путину :)

\iusr{Telema}

Ну значит вы тоже в фашизм удариться хотите, как наши политические украинцы?
Так запросто! Только, сдаётся мне, вы просто бредите

\iusr{Юри Ткачёв}
\textbf{Вячеслав Шмарин}
именно деревни нужно? Города не годится?

\iusr{Artyom}
\textbf{Вячеслав Шмарин}

Юрий как-то описывал обращение в консульство РФ с просьбой поддержать
прорусское движение в Одессе, там ответили, что все подобные вопросы должны
решаться через Партию Регионов

\iusr{Юри Ткачёв}
\textbf{Artyom}

ну это давно было, 2011 год ещё, если не раньше

\iusr{Вячеслав Шмарин}
\textbf{Юри Ткачёв}

не именно, это же просто пример. Город тоже подойдет, только по началу в городе
будет опасно.

\iusr{Юри Ткачёв}
\textbf{Вячеслав Шмарин}

ОК, в Харькове сдлелали, вернули проспект Маршала Жукова. И где реакция?

\iusr{Artyom}
\textbf{Юри Ткачёв}
Теперь вместо Януковича Медведчук)

\iusr{Вячеслав Шмарин}
\textbf{Юри Ткачёв}

Пункт 2. Набрать статистику. Должно быть понятно что это мнение народа.

\iusr{Юри Ткачёв}
\textbf{Вячеслав Шмарин}

какую статистику? какой она должна быть?

\iusr{Artyom}
\textbf{Юри Ткачёв}

Дать власти набить фрагов. От нескольких сотен до нескольких тысяч

\ifcmt
  ig https://i2.paste.pics/b522e0c099b0dff7dd1901e4ace42284.png
  @width 0.4
\fi

\iusr{Вячеслав Шмарин}
\textbf{Юри Ткачёв}

Один эпизод показали. Покажите 100 эпизодов. ЭТо статистика

\iusr{Юри Ткачёв}
\textbf{Вячеслав Шмарин}

Ага, т.е. нужно 100 эпизодов. ОК. А не переименования обратно, а отказы переименовывать сойдут?

\iusr{Ivan Petrov}
\textbf{Вячеслав Шмарин}

Wishful thinking. Там у большинства населения в головах такого насрано, что ни
приведи Господи. Сперва бандеровское говно должно перебродить, а потом уже
видно будет.

\emph{Вячеслав Шмарин}, [08.12.2021 21:47]
\textbf{Юри Ткачёв}
[In reply to Юри Ткачёв]
Это уже сложно. ПОтому что тут неясно , может это бумага в канцелярии застряла

Евгений, [08.12.2021 21:47]
[In reply to Юрий Ткачёв]
Такая забавная агитка. Путин придёт, всех победит.

\end{itemize} % }
