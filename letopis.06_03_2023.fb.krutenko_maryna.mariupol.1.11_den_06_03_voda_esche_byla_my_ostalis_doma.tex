%%beginhead 
 
%%file 06_03_2023.fb.krutenko_maryna.mariupol.1.11_den_06_03_voda_esche_byla_my_ostalis_doma
%%parent 06_03_2023
 
%%url https://www.facebook.com/marinakrytenko/posts/pfbid036quBXQvBaNMT5mNJuJqoy3Uk3oAENny4aqXPJbHy5UFBk5dt7md6TE4Hn1jMctD6l
 
%%author_id krutenko_maryna.mariupol
%%date 06_03_2023
 
%%tags mariupol,mariupol.war,dnevnik,06.03.2022
%%title ОДИННАДЦАТЫЙ ДЕНЬ ВОЙНЫ. 06.03.22 Вода у нас ещё была, мы остались дома
 
%%endhead 

\subsection{ОДИННАДЦАТЫЙ ДЕНЬ ВОЙНЫ. 06.03.22 Вода у нас ещё была, мы остались дома}
\label{sec:06_03_2023.fb.krutenko_maryna.mariupol.1.11_den_06_03_voda_esche_byla_my_ostalis_doma}

\Purl{https://www.facebook.com/marinakrytenko/posts/pfbid036quBXQvBaNMT5mNJuJqoy3Uk3oAENny4aqXPJbHy5UFBk5dt7md6TE4Hn1jMctD6l}
\ifcmt
 author_begin
   author_id krutenko_maryna.mariupol
 author_end
\fi

ОДИННАДЦАТЫЙ ДЕНЬ ВОЙНЫ. 06.03.22

Вода у нас ещё была, мы остались дома. 

Мужчины поехали развозить мясо и термобелье знакомым и друзьям. 

Приехали мужчины и сказали что бы мы садились в машину и уезжали. Формируются
две колонны для эвакуации из города. Одна колонна должна была выезжать от
Драмтеата, а вторая от стадиона \enquote{Ильичёвец}. От неожиданности начали плакать,
целоваться и прощаться с мужьями, как будто мы никогда больше не увидимся. Я
садилась в машину в состоянии сильного стресса, меня трясло и я не понимала как
и куда ехать. По дороге увидела сгоревшие машины. Я поняла, что по Куприна я не
выеду и на пожарной части повернула на 17 микрорайон.  На улице с односторонним
движением посередине дороги стоял подбитый автомобиль. Я включила аварийный
сигнал и поехала по встречному движению. 

На пересечении Шевченко и Строителей, те машины которые ехали на выезд из
города в сторону Запорожье, на большой скорости ехали обратно.... Начался обстрел
города..... Я развернулась и поехала в сторону \enquote{Ильичёвец}. 

Меня все так же трясло, я решила что мне нужно успокоится и сконцентрироваться
на дороге, знаках и пешеходах, чтоб не убить других и себя на дороге.  . 

Возле \enquote{Обжоры} все так же лежал труп женщины. Люди все так же тащили какие-то
продукты или вещи. Подъезжая к стадиону мы услышали объявление по рации
полиции: \enquote{враги обстреливает город. Эвакуации не будет!} 

Через 30-40 минут, мы вернулись домой к мужьям. Достали из машины вещи.
Наплакавшись и успокоившись продолжили своё существование. 

Продолжение следует.

%\ii{06_03_2023.fb.krutenko_maryna.mariupol.1.11_den_06_03_voda_esche_byla_my_ostalis_doma.cmt}
