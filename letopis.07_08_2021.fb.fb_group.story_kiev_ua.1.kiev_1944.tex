% vim: keymap=russian-jcukenwin
%%beginhead 
 
%%file 07_08_2021.fb.fb_group.story_kiev_ua.1.kiev_1944
%%parent 07_08_2021
 
%%url https://www.facebook.com/groups/story.kiev.ua/posts/1724117931118334/
 
%%author Киевские Истории
%%author_id fb_group.story_kiev_ua
%%author_url 
 
%%tags 1944,gorod,istoria,kiev,sssr,ukraina,ussr,vov
%%title Киев в 1944 году
 
%%endhead 
 
\subsection{Киев в 1944 году}
\label{sec:07_08_2021.fb.fb_group.story_kiev_ua.1.kiev_1944}
 
\Purl{https://www.facebook.com/groups/story.kiev.ua/posts/1724117931118334/}
\ifcmt
 author_begin
   author_id fb_group.story_kiev_ua,ljashenko_vadim
 author_end
\fi

Киев в 1944 году.

В 1943 году советские войска вошли в оставленный немцами Киев и увидели, что
значительная часть Крещатика и некоторые соседние кварталы повреждены или вовсе
разрушены. Разбирать завалы жители города, солдаты и военнопленные начали еще
во время войны. Половина Крещатика была разрушена, а район Софийской площади, к
счастью, почти не пострадал. В Киев в 1944 году ежедневно приезжают и уезжают
разнообразные группы социального и национального состава. На улицах можно было
встретить представителей войсковой верхушки (генералов, полковников), рядовых и
различных служащих: литераторов, редакторов, директоров и т.д. Люди различной
национальности: россияне, украинцы, грузины, евреи и т.д. Многие здесь находят
постоянное место жительства. На улицах постоянно проверяют документы, особенно
у военных. 

\ifcmt
  tab_begin cols=2
		width 0.45

     pic https://scontent-cdg2-1.xx.fbcdn.net/v/t1.6435-9/232557385_849904308964626_3809479595322066620_n.jpg?_nc_cat=102&ccb=1-4&_nc_sid=b9115d&_nc_ohc=3DSaCoUTcBYAX8GZcGN&_nc_ht=scontent-cdg2-1.xx&oh=4a3a4b95128af34f12f7587eb6493e28&oe=6135CA2A

     pic https://scontent-cdg2-1.xx.fbcdn.net/v/t1.6435-9/234442400_849904345631289_141349241147921342_n.jpg?_nc_cat=111&ccb=1-4&_nc_sid=b9115d&_nc_ohc=Wi3WDY-nipEAX9Y9-ss&_nc_ht=scontent-cdg2-1.xx&oh=a53f642e0f061c305c81bde1aaac860d&oe=6136D291

  tab_end
\fi

На прошлом пленуме писателей присутстовали: Тычина, Корнейчук, Остап Вишня,
Сосюра, Ванда Василевская, Фефер, Соболев а также представители прессы и
студентства. Тычина открыл пленум. Соболев отметил, что молодежь, пережившая
оккупацию прониклась духом врага и необходимо найти лекарства, чтобы излечить
её от вражеского влияния. Корнейчук выступил против Соболева и сказал, что
молодежь перенесла все тяготы фашистского гнёта и помогала партизанскому
движению.

\ifcmt
  tab_begin cols=2
		width 0.45
	 		pic https://scontent-cdg2-1.xx.fbcdn.net/v/t1.6435-9/233680944_849904405631283_563303171452349838_n.jpg?_nc_cat=108&ccb=1-4&_nc_sid=b9115d&_nc_ohc=j0lpnYj1jdgAX9xghar&_nc_ht=scontent-cdg2-1.xx&oh=adc185638483b03cf3b5475ffe4478fa&oe=613632C0

		  pic https://scontent-cdt1-1.xx.fbcdn.net/v/t1.6435-9/232471524_849904475631276_7033771259621414859_n.jpg?_nc_cat=101&ccb=1-4&_nc_sid=b9115d&_nc_ohc=Fi1MLC8F_TIAX9t1pMn&_nc_ht=scontent-cdt1-1.xx&oh=18a2d6ac6d7fe04cc6091896f19f0cf9&oe=6137BD75

  tab_end
\fi

Через несколько дней Корнейчука сняли и министром иностранных дел назначили
Мануильского. 

В поездах прибывает множество военных и командированных. Милиция проверяет
документы, попадаются преступники и дезертиры. 

Военные часто отказываются предъявлять документы и с ненавистью смотрят на
милицию.  

ФОТО:

\begin{itemize}
\item 01. План-схема хронологии разрушения Крещатика. По материалам публикации
Семена Широчина 'Как исчез довоенный Крещатик'. Красный цвет-разрушения
огнем и фугасом.

\item 02. Столица после начала Второй мировой войны. Фото: Музей истории Киева.
\item 03. Пострадавшее от бомбежек здание киевского Вокзала, ноябрь 1943 года.
\item 04. Понтонный мост через Днепр, построенный возле быков моста им. Е.Бош.
Фотография ноября-декабря 1943 года.
\end{itemize}

\ii{07_08_2021.fb.fb_group.story_kiev_ua.1.kiev_1944.cmt}
