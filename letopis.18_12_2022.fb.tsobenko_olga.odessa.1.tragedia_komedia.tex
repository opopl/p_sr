% vim: keymap=russian-jcukenwin
%%beginhead 
 
%%file 18_12_2022.fb.tsobenko_olga.odessa.1.tragedia_komedia
%%parent 18_12_2022
 
%%url https://www.facebook.com/permalink.php?story_fbid=pfbid0P4gjER9ZQPqazbpfQfCUiCQGpp9VDZsbf2MhWHLB2dW69ZenPCGYHukvtS64wwdsl&id=100001605241971
 
%%author_id tsobenko_olga.odessa
%%date 
 
%%tags 
%%title Там, где много трагедии, начинается комедия...
 
%%endhead 
 
\subsection{Там, где много трагедии, начинается комедия...}
\label{sec:18_12_2022.fb.tsobenko_olga.odessa.1.tragedia_komedia}
 
\Purl{https://www.facebook.com/permalink.php?story_fbid=pfbid0P4gjER9ZQPqazbpfQfCUiCQGpp9VDZsbf2MhWHLB2dW69ZenPCGYHukvtS64wwdsl&id=100001605241971}
\ifcmt
 author_begin
   author_id tsobenko_olga.odessa
 author_end
\fi

Мне долго не давалось понять бабушкину фразу о том, что там, где много
трагедии, начинается комедия... 

Как это? Комедия в трагедии? 

Потом поняла. 

Психика устаёт бояться. 

В этом роковой просчёт тех, кто делает на массовый страх свои главные ставки. 

Бабушка же часто повторяла, что и на войне был патефон. И был он вовсе не для
того, чтобы устраивать пир во время чумы, а просто оттого, что жизнь однажды
тоже устаёт бояться и отказывается находиться в хроническом страдании, в
хроническом ужасе, в хроническом запрете на самую малую радость. 

Мудрость тёмных времён - не относиться слишком серьёзно ни к себе, ни к другим,
ни к происходящему...как бы парадоксально это ни звучало. 

В какой-то момент стоит понять, что на эту планету не завозили ни
гарантированной безопасности, ни сверхлюдей, ни прав на исключительность. 

Приняв условия этой простой задачи, получается жить в доступном. 

В том доступном, где необходимость сообразить кастрюлю щей ничем не менее
значима, чем высокие нематериальные мысли. 

Где проплакавшись, заводишь тот же патефон и проходишься в неидеальном танце. 

Где ворчишь по утрам, но и радуешься простому факту, что работа пока есть. 

Где и мечтаешь немного о завтра, но и ощущаешь уже особый кайф однодневного
пофигизма: вот оно сейчас так, а потом и не важно, как будет. 

Где вдруг понимаешь, что вчерашняя прокрастинация - это уже роскошь уходящего
старого мира, а сегодня ничего не откладываешь, и делаешь это без танцев с
бубнами вокруг себя, изобретая работающую мотивацию (отчего мне настолько
ненавистно это дурацкое слово, так хорошо рифмующееся с канализацией?) 

Где и вправду очень остро осознаёшь, что всё, что по-настоящему важно, надо
делать сейчас, а не когда-нибудь. 

Сейчас говорить о любви, сейчас творить, сейчас мириться, сейчас расставаться с
опостылевшим, сейчас идти гулять под снегопадом, сейчас позвонить родителям,
сейчас перестать докапываться до близких... 

И сейчас перестать бояться. 

Для этого, к слову, нужна не какая-то немыслимая храбрость, а обыкновенный
здравый смысл, подсказывающий что одержимые тем, чтобы их боялись, на самом
деле не страшны, а комичны. 

И сейчас начать жить так, чтобы самому было понятно, что и вправду живёшь... 

Там, где много трагедии, начинается комедия. 

Комедия бестолковой, непонятной, горькой и счастливой одновременно человеческой
жизни, в которой так хочется ещё побыть. 

Лиля Град
