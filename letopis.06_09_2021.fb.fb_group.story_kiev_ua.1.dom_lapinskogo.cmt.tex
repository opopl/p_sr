% vim: keymap=russian-jcukenwin
%%beginhead 
 
%%file 06_09_2021.fb.fb_group.story_kiev_ua.1.dom_lapinskogo.cmt
%%parent 06_09_2021.fb.fb_group.story_kiev_ua.1.dom_lapinskogo
 
%%url 
 
%%author_id 
%%date 
 
%%tags 
%%title 
 
%%endhead 
\zzSecCmt

\begin{itemize} % {
\iusr{Олена Цюкевич}

Ой, і наш дворик! яке цінне фото!

\begin{itemize} % {
\iusr{Ольга Чекрыгина}
\textbf{Олена Цюкевич} а дворик дома 57? Я там часто бывала. Подруга жила в двухэтажном домике, стоявшем вдоль улицы

\iusr{Олена Цюкевич}
\textbf{Olga Chekrygina} ,58
\end{itemize} % }

\iusr{Vladimir Aniskov}
Превосходно

\iusr{Оксана Денисова}
\textbf{Vladimir Aniskov} Спасибо большое!

\iusr{Раиса Карчевская}
Оксана!
Большое спасибо за пост, очень познавательно и интересно

\iusr{Оксана Денисова}
\textbf{Раиса Карчевская} Спасибо огромное!

\iusr{Вячеслав Стрелец}

В детстве лазили по отселеным домам, их активно сносили в конце 70х, этот дом
был между ними.

Там под домом огромная каретная в подвале, тогда была заброшена, забита хламом.

\begin{itemize} % {
\iusr{Оксана Денисова}
\textbf{Вячеслав Стрелец} Как интересно! Здорово!

\iusr{Сергій Бєлах}
\textbf{Вячеслав Стрелец} 

а от уявіть, як би зараз так активно зносили, як в кінці 70-х! Скільки б було
крику. А тоді, нічого, норм. І зараз ніхто не згадує, скільки позносили старого
Київа при савєтах. Але все одно Клічко знищує Київ.

\begin{itemize} % {
\iusr{Владимир Писаренко}
\textbf{Сергій Бєлах} , тоді зносили старий фонд - який не можна було відремонтувати. Новобудови не порушували норм містобудування (зараз цим грішить кожна споруда!).
Впрочем, \enquote{какая разница} (с)

\iusr{Вячеслав Стрелец}
\textbf{Владимир Писаренко} на Ленина 41 снесли дом в котором жила оперная певица Мирошниченко, дом был 2 этажный но кирпичный, добротный, дружил с её сыновьями , живу через дорогу.
Не все дома были деревянными.
По сей день во дворах 70-76 по Богдана есть двух этажки

\iusr{Сергій Бєлах}
\textbf{Владимир Писаренко} 

да, да, не порушували норм містобудування... Панельні дев'ятиповерхівки в
історичному центрі на Шовковичній та Басейній зовсім не порушують норм
містобудування. Весь Поділ затиканий хрущовками ніразу не порушує норм
містобудування. Ага. Та, авжеж, \enquote{какая разніца} (с). Зате пломбир який смачний
був.


\iusr{Владимир Писаренко}
\textbf{Сергій Бєлах} , 

то тепер давайте все запаскудимо, якщо раніше, виявляється, було щось подібне?!
Мова про інше - висотність та відсутність належної інфраструктури. Знести
хрущовку - плове діло. Ви \enquote{Halmolnd}, \enquote{Oazis} або \enquote{монстра на Подолі} знесіть!

\end{itemize} % }

\iusr{Ольга Чекрыгина}
\textbf{Вячеслав Стрелец} мы тоже там лазили в конце 70-х. Дорога из 91 школы занимала гораздо больше времени, чем в школу

\iusr{Вячеслав Стрелец}
\textbf{Ольга Чекрыгина} 

я, в 58 ходил по Ленина, где и живу, там точно такие двух этажки, но иногда
ехал на 71 от шапито до Чкаловского  @igg{fbicon.laugh.rolling.floor} ,

\end{itemize} % }

\iusr{Татьяна Максимец}
Улица моего детства - каждый день ходила мимо этого прекрасного дома в школу
@igg{fbicon.heart.sparkling} 

\begin{itemize} % {
\iusr{Оксана Денисова}
\textbf{Татьяна Максимец} По- хорошему завидую Вам, люблю эту улицу!

\iusr{Игорь Запольский}
\textbf{Татьяна Максимец} А я напротив него работаю )

\iusr{Диана Ишунина}
\textbf{Игорь Запольский} Здание напротив тоже требует ремонта

\ifcmt
  ig https://scontent-frx5-1.xx.fbcdn.net/v/t39.30808-6/241627143_1744214125768742_7555882249011876194_n.jpg?_nc_cat=100&ccb=1-5&_nc_sid=dbeb18&_nc_ohc=O2mQZc8WQXMAX-3ac4j&_nc_ht=scontent-frx5-1.xx&oh=00_AT_WUFOF0cL56f1s4LOacwrM6Eo0afHnGRyGzFEs_VV0Eg&oe=62020CB7
  @width 0.3
\fi

\iusr{Valerii Gontarenko}
\textbf{Татьяна Максимец} Та й 91 школа була непогана. Шкода, що парк за будинком ортопедія знищила

\iusr{Ольга Чекрыгина}
\textbf{Татьяна Максимец} любимая улица, тот же путь в 91 школу мимо этого дома

\end{itemize} % }

\iusr{Anna Kievatz}
Как же так, что архитектор неизвестен?

\begin{itemize} % {
\iusr{Оксана Денисова}
\textbf{Anna Kievatz} Так бывает... в Киеве это не единственный красивый дом, архитектор которого неизвестен.

\iusr{Anna Kievatz}
\textbf{Оксана Денисова} Увы...
\end{itemize} % }

\iusr{Alya Mykhailova}
Оксаночка, вчера были на Вашей замечательной экскурсии! Благодарим за чудесные 3 часа! До встречи!@igg{fbicon.heart.red} @igg{fbicon.hands.applause.yellow}  @igg{fbicon.face.happy.two.hands} 

\iusr{Оксана Денисова}
\textbf{Alya Mykhailova} Спасибо Вам огромное за отзыв! И буду ждать Вас на других моих экскурсиях!!!

\iusr{Таня Сидорова}
Дякую Оксана, як завжди цікаво. Плануєте екскурсію?

\begin{itemize} % {
\iusr{Таня Сидорова}
\textbf{Оксана Денисова} дякую

\iusr{Наташа Сазонова}
\textbf{Оксана Денисова} спасибо!!!

\iusr{Ksenia Sereda}
\textbf{Оксана Денисова}, спасибо

\iusr{Таня Сидорова}
\textbf{Оксана Денисова} дякую

\iusr{Alla Manzhelii}
\textbf{Оксана Денисова} спасибо, тогда до встречи, зашла в Ваши публикации, оторваться не могу.

\iusr{Оксана Денисова}
\textbf{Alla Manzhelii} Спасибо Вам большое!
\end{itemize} % }

\iusr{Наталія Крюкова}
Спасибо за публикацию!

\iusr{Татьяна Оксаненко}

Очень интересно и познавательно. Помню, что родители так и называли этот дом,
домом Лапинского, чуть дальше, ближе к площади Победы, жил друг детства моего
отца, все мы часто его навещали. Как же так случилось, что неизвестно имя
архитектора такой роскошной усадьбы Спасибо, Вам огромное! ВЫ КАК ВСЕГДА на
высоте!

\begin{itemize} % {
\iusr{Оксана Денисова}
\textbf{Татьяна Оксаненко} Спасибо Вам! Вот такой загадочный дом, не сохранилось имя архитектора в архивах.

\iusr{Татьяна Оксаненко}
\textbf{Оксана Денисова} очень жаль..
\end{itemize} % }

\iusr{Елена Шевченко}
 @igg{fbicon.hands.pray} @igg{fbicon.heart.red}{repeat=3}

\iusr{Татьяна Абрамчук}
Спасибо за публикацию

\iusr{Nina Danilko}
Работаю рядом и всегда восхищаюсь этим домом сейчас такого не строят, а старое ломают

\iusr{Оксана Денисова}
\textbf{Nina Danilko} Будем надеяться, что такие красивые дома сохранятся!

\iusr{Ирина Иванченко}

Как вы умеете так \enquote{вкусно} преподать киевскую старину, Оксана, просто
прелесть. Каждый ваш пост читаю с удовольствием! Каюсь, обещала быть на
экскурсию, но всё никак не соберусь \enquote{с мыслями}... Но не теряю оптимизЬму всё же
быть... Благодарю вас, очень интересно.

\iusr{Оксана Денисова}
\textbf{Ирина Иванченко} Спасибо Вам! И приходите на экскурсии, на них интересно @igg{fbicon.grin} 

\iusr{Петр Кузьменко}

Благодарю! Очень интересно и познавательно! Это нужно знать киевлянам и всем, кто любит Город. @igg{fbicon.hands.applause.yellow} 

\iusr{Оксана Денисова}
\textbf{Петр Кузьменко} Спасибо Вам!

\iusr{Ольга Кожедуб}
Спасибо, Оксана! Очень красиво написано и, как всегда познавательно. Как попасть к Вам на \enquote{Тайны Золотоворотского квартала}?

\iusr{Оксана Денисова}
\textbf{Светлана Шварц} Напишу обязательно!

\iusr{Светлана Шварц}
\textbf{Оксана Денисова} спасибо @igg{fbicon.heart.red}

\iusr{Savelyeva Olena}
Очень интересно. Спасибо.

\iusr{Оксана Денисова}
\textbf{Savelyeva Olena} Спасибо!

\iusr{Людмила Прищепа}
Дякую за ваш труд ! Пізнавально і цікаво !@igg{fbicon.heart.red}

\iusr{Оксана Денисова}
\textbf{Людмила Прищепа} Спасибо!

\iusr{Roman Sytnyk}
А чому нема згадки, що в цьому домі мешкав Симон Петлюра?

\iusr{Тома Храповицкая}

Спасибо Вам огромное Оксана! Вы как всегда великолепны! Так преподнести,
преподать историю, красоту и величество Дома и его жителей! Восхитительно!!!

\iusr{Оксана Денисова}
\textbf{Тома Храповицкая} Спасибо Вам огромное!

\iusr{Людмила Сидоренко}

Я була в цьому будинку в комунальній квартирі. Запам'яталися високі стелі, десь
біля 4 метрів. Дуже дивно виглядали сучасні меблі в кімнаті 36 кв.м. До стелі
було ще 2м.


\iusr{Оксана Денисова}
\textbf{Людмила Сидоренко} да, там внутри очень интересно!

\iusr{Дмитрий Сливный}
Часто рядом прохожу. В маленьком флигеле сейчас костюмерная

\iusr{Эллина Перельман}
\textbf{Дмитрий Сливный} раньше флигель занимали скульпторы

\iusr{Juri Stefan Dubrovski}

в этом доме жила близкая подруга моей матери и в моем далеком детстве мы с
мамой часто бывали у нее... я хорошо помню фойе лестничной группы с лифтами,
если не изменяет память оставшиеся еще со времени постройки, квартиры в этом
доме конечно были перестроены властью рабочих и крестьян, но выветрить дух
романтики, таинственности, дух другой эпохи никакая власть не в состоянии ... и
конечно это архитектурная удача, взять детали дома, поэтажное разнообразие
оконных проемов и сандриков над ними в 3-х стилях: готики, ренессанса и
рационального модерна,... могу сказать, что архитектор был крепкий профессионал
будучи на ты с разными стилями архитектуры, но четко подчиняя разнообразие форм
единству замысла, а еще там потрясающая лестница подиума ведущая ко входу в дом

\begin{itemize} % {
\iusr{Оксана Денисова}
\textbf{Juri Stefan Dubrovski} спасибо Вам за такие интересные воспоминания! Да, архитектор был явно очень талантлив! И я надеюсь, что может быть все- таки эта тайна - кто строил этот дом- будет разгадана!

\iusr{Juri Stefan Dubrovski}
\textbf{Оксана Денисова} в архивах Софии должно быть

\iusr{Оксана Денисова}
\textbf{Juri Stefan Dubrovski} Когда делали «Свод памятников архитектуры» собирали данные из всех архивов, но так фамилии архитектора и не нашли.

\iusr{Juri Stefan Dubrovski}
\textbf{Оксана Денисова} жаль, а есть там имя арх. автора проекта дома Ричарда на Андреевском?

\iusr{Оксана Денисова}
\textbf{Juri Stefan Dubrovski} Проект дома был украден у петербургского архитектора Роберта Марфельда, кто- то из Киевских архитекторов его переделал, но кто поставит свою подпись под украденным проектом?

\iusr{Juri Stefan Dubrovski}
логишь, а архитектор Марфельд, вы знаете о нем больше, он не фортификационный проектировщик?

\iusr{Оксана Денисова}
\textbf{Владимир Федорович} Не знаю...

\iusr{Оксана Денисова}
\textbf{Juri Stefan Dubrovski} Мало что о нем знаю, только то, что написано в Википедии
\end{itemize} % }

\iusr{Maksim Pestun}

Окна папиной квартиры, где он вырос, выходят прямо на этот дом.

\begin{itemize} % {
\iusr{Оксана Денисова}
\textbf{Maksim Pestun} Завидую! По- хорошему @igg{fbicon.grin} 

\iusr{Олена Цюкевич}
\textbf{Maksim Pestun}, а в каком доме он жил? Можете это наш сосед?

\iusr{Maksim Pestun}
\textbf{Олена Цюкевич} Чеховский 3. Квартира до сих пор сохранилась за нами

\iusr{Олена Цюкевич}
Нет...

\iusr{Monya Feldman}
\textbf{Maksim Pestun} а я смотрел с окна на квартиру твоего папы.. Давно это было...
\end{itemize} % }

\iusr{Евгения Верченко}
Дякую дуже!
Цікаво дуже, про братів Булгакових не знала...

\iusr{Оксана Денисова}
\textbf{Евгения Верченко} Дякую!

\iusr{Elena Dubovenko}

Спасибо за подробности(!), о Булгаковых, связанных с Лапинским совсем не знала.
И фотография прекрасна! Парадные лестницы в доме мраморные, в комнатах под
высокими потолками изящная лепка, в квартирах были печки, потом их убрали и
появилось отопление, а на месте печей были хозяйственные ниши. Благодаря
высоким потолкам, из кухни лестница ведёт на довольно объемные антресоли,
используемые как дополнительное хоз. место.

\iusr{Оксана Денисова}
\textbf{Elena Dubovenko} Спасибо Вам за такие чудесные подробности о доме!

\iusr{Калерия Ходос}
Красивый дом ,,самое главное его сохранить и не дать снести и поставить на этом месте новый монстр стеклянную многоэтажку.

\iusr{Оксана Денисова}
\textbf{Калерия Ходос} Будем надеяться, что этот чудесный дом сохранится!

\iusr{Оксана Велкова}
Мой муж вырос в доме напротив. Загадочное место

\iusr{Ирина Нецора}
Спасибо, очень интересно.

\iusr{Оксана Денисова}
\textbf{Ирина Нецора} Спасибо!

\iusr{Лариса Артемчук}
Ул. Гончара 33?
Там ведь и Столыпин жил? Так?

\iusr{Оксана Денисова}
\textbf{Лариса Артемчук} 

Нет, это Гончара 60. А в доме 33 была клиника Маковского, где Столыпин умер
после покушения на него в Городском театре.

\iusr{Евгения Ерёменко}
Спасибо, Оксана  @igg{fbicon.face.smiling.eyes.smiling} 

\iusr{Оксана Денисова}
\textbf{Евгения Ерёменко} Спасибо Вам!

\iusr{Татьяна Сирота}

Прекрасная публикация @igg{fbicon.hearts.two} 
Спасибо огромное!

\iusr{Оксана Денисова}
\textbf{Татьяна Сирота} Спасибо Вам!

\iusr{Lyudmila Kravcova}
Удивительный дом!.. спасибо!

\iusr{Георгий Готесман}

Во! А я всё ждал - когда же в жтой рубрике (\enquote{КИ}) он возникнет! Наконец-то... В
возрасте лет 12, будучи в гостях у двоюродного дедушки Наума (бывшего военного
врача, а в мирное время - имевшего лицензию для приема на дому больных с
кожно-венерическими заболеваниями) заприметил на тогдашней улице Чкалова (дед
жил на Гоголевской) это чудо. Проник в него и даже прокатился на крохотном
антикварном лифте. Фасад также был светло-жёлтого цвета. Это здание врезалось в
память навсегда!

\begin{itemize} % {
\iusr{Оксана Денисова}
\textbf{Георгий Готесман} Рада, что оправдала Ваши ожидания и дом возник @igg{fbicon.grin}  и как Вам повезло, что Вы были внутри!

\iusr{Георгий Готесман}
\textbf{Оксана Денисова} Спасибо, Оксана! Но тогда это было совсем не сложно: год стоял 1966, либо 1967-й.

\iusr{Оксана Денисова}
\textbf{Георгий Готесман} Да, сейчас внутрь дома попасть нелегко  @igg{fbicon.face.pensive} 

\iusr{Александра Лобынцева}
\textbf{Оксана Денисова} А что там сейчас?

\iusr{Оксана Денисова}
\textbf{Александра Лобынцева} Просто жилой дом, дверь закрыта.
\end{itemize} % }

\iusr{Liudmila Boiko}

Дякую за прекрасну публікацію. Мов близькі родичі прожили в цьому домі все
життя, а ж по не почали продавати квартири які перетворили на комунальні, щоб
\enquote{роз'іхатись}. Мені теж пощастило на протязі року спостерігати
мармурові білі сходи в парадному, скляну стелю, до якої підіймався найстаріший
в Києві ліфт, привезений з Америки. Квартири по 250 кВ. Метрів із стінами по
півметра товщиною. Каміни оздоблені чеськими кахлями, але більшість була
замурована. Я жила в кімнаті, яка колись належала прислузі. З неі був
додатковий вихід в коридор, що вів до роздільних санвузлів і кухні, з якої був
вихід на \enquote{чорний} хід , до якого за будинком доставляли дрова і продукти. Перед
містком був \enquote{паліладнік} - місце відпочинку, садок. Під мостиком
окремий вхід в квартиру двірника. Все продумано, комфортно. Всього 12 квартир.

\begin{itemize} % {
\iusr{Оксана Денисова}
\textbf{Liudmila Boiko} Какие интересные подробности! Спасибо Вам огромное!!!

\iusr{Elena Dubovenko}
\textbf{Liudmila Boiko} щиро дякую, все так!

\iusr{Эллина Перельман}
В каждой из квартир 7 комнат.
Каждая комната( если повезло, 2) принадлежали разным семьям.
\end{itemize} % }

\iusr{Александр Муратов}

Спасибо! С 1952 года, с первог курса мединмтитута этот дом был для меня
загадкой, Я тогда ходли в дом напротив, где изучал \enquote{Латынь} и \enquote{Микробы}. . Эот
дом удивлял могие годы. Кому ни задавал вопросы, никто ничего вразумительног
сазвть не мог. Только став нерабтающим песионром, овладев компом за бгром,
начал встречать в паутине сообщения о братьях Штенгелях, об их домах. Узнал ,
что в молодости изучал ортопедию и травматологию в дном из их домов на
Бульв.-Кудрявской, примыкавшем \enquote{с тыла} к этому загдочному дому.

\begin{itemize} % {
\iusr{Оксана Денисова}
\textbf{Александр Муратов} Да, Вы изучали ортопедию в доме Барона Рудольфа Штейнгеля!

\iusr{Олег Верхоградов}
У Булгакова в \enquote{Мастер и Маргарита} был персонаж барон Майгель. Возможно, прообраз взят отсюда.

\iusr{Оксана Денисова}
\textbf{Олег Верхоградов} Может быть...
\end{itemize} % }

\iusr{Олена Кириченко}
В этом доме рос мой папа!

\iusr{Элина Юрчак}

Присоединяюсь к многочисленным благодарностям за познавательный пост автору, а
также спасибо за не менее интересные комментарии.

\iusr{Оксана Денисова}
\textbf{Элина Юрчак} Спасибо Вам!

\iusr{Людмила Петровна}

Я возле этого дома жила всю жизнь. И дом этот мне нравился своей загадочностью.
Спасибо вам за такую историю. Вы молодец.

\iusr{Оксана Денисова}
\textbf{Людмила Петровна} Спасибо огромное!

\iusr{Olena Klymenko}
Очень интересный пост!!! Спасибо большое, Оксана! Как всегда, очень познавательно!!!!

\iusr{Оксана Денисова}
\textbf{Olena Klymenko} Спасибо большое!

\iusr{Наталия Лебедь}

Давно в журнале Юность был напечатан очерк известного киевлянина Виктора
Некрасова \enquote{Городские прогулки}, где он описывал время учебы на архитектурном
факультете, который тогда находился на Большой Владимирской недалеко от этого
дома и каждый раз, проходя мимо, конечно же любовался им. Так вот он в этом
очерке написал, что автор толи сам Городецкий, толи кто-то из его учеников.
Много раз я пыталась найти этот очерк и перечитать, но не смогла найти, а в
книге с почти одноименным названием об этом ничего не сказано. Действительно
ЗАГАДКА.

\begin{itemize} % {
\iusr{Оксана Денисова}
\textbf{Наталия Лебедь} Да, версия о Городецком была, но она не подтвердилась.

\iusr{Juri Stefan Dubrovski}
\textbf{Наталия Лебедь} 

стиль действительно похож на Городецкого и его творчество, по годам и положению
заказчика проекта это мог сделать только он, как выдающийся архитектор того
времени, которому было интересно и у него были полностью развязаны руки, не
стоит забывать про большой земельный участок под проект, а такие возможности
можно доверить человеку с большим именем, результатами и известными объектами

\end{itemize} % }

\iusr{Олена Кириченко}

Моего папу в этот дом принесли с роддома в 1949 году и жил он там по 1962 год в
квартире номер 3

\begin{itemize} % {
\iusr{Оксана Денисова}
\textbf{Олена Кириченко} Как Вашему папе повезло жить в таком красивом доме!

\iusr{Олена Кириченко}
\textbf{Оксана Денисова} это да, спасибо Оксана

\iusr{Олена Кириченко}
Столько интересных и теплых воспоминаний
\end{itemize} % }

\end{itemize} % }
