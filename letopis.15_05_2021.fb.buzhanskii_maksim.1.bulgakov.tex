% vim: keymap=russian-jcukenwin
%%beginhead 
 
%%file 15_05_2021.fb.buzhanskii_maksim.1.bulgakov
%%parent 15_05_2021
 
%%url https://www.facebook.com/permalink.php?story_fbid=1944968515667618&id=100004634650264
 
%%author 
%%author_id 
%%author_url 
 
%%tags 
%%title 
 
%%endhead 
\subsection{Булгакову 130 лет.  И надо же, он все предугадал}
\label{sec:15_05_2021.fb.buzhanskii_maksim.1.bulgakov}
\Purl{https://www.facebook.com/permalink.php?story_fbid=1944968515667618&id=100004634650264}

Булгакову 130 лет. И надо же, он все предугадал.

Всё, абсолютно всё, они с Гоголем и Ильфом и Петровым выписали рамками, из
которых мы никуда не выпрыгнули ни за сто, ни за двести лет.

Частенько слышу - Булгаков не киевлянин, уж не знаю по каким меркам.

Не киевлянин, нет.

С тех пор, как киевлянами стали Парубий, Вятрович и Ницой, больше нет.

Навсегда нет, будем откровенны.

Но ему и не нужно, бал закончен, у каждого желающего есть справка о том, что
послужил в качестве транспортного средства, а то и в набсовете целом, страна
полна говорящих костюмов без головы, подписывающих распоряжения, граждане поют
хором не грузовиками, но областями целыми, а швондеровщину официально и
неоднократно признали высшим критерием патриотизма.

Иваны Бездомные успешно громят оппонентов на всех эфирах каждый вечер, и
никакой Кант им не доказательство, \enquote{Грибоедов} работает даже в локдаун, а к
некоторым приезжают таки наряды на мотоциклах.

Нехорошая квартира, да?

И квартирный вопрос всё так же стоит, и всё так же портит, хоть триста раз
допили остатки обеих империй, затравленным рабом которых, Булгаков так и не
выглядит.

Взять всё, и запретить, взять всё, и поделить, ничего прогрессивней мы пока не
придумали.  Но...

Рукописи уверенно летят в камин, ещё какие-нибудь десять, нет, пятнадцать лет,
и никто не узнает, что это уже было, всё уже было.

И сразу станет легче, и наверно, не так обидно.

Анна Войтенко

Легче почему-то не становится... От осознания всего происходящего хочется видеть свет в конце тоннеля

Наташа Тарчевська

Як добре, що залишається книжковий на Петрівці, з гарними радянськими виданнями Михайла Опанасовича, оригінальною російською

Макс Бужанский

Наташа Тарчевська , да вроде нет у нас проблем и с современными изданиями)

Lana Brahina

Макс Бужанский
Есть проблемы.
Я как мама читающего ребёнка 12 лет, говорю. Есть большие проблемы с книгами на
русском языке. Особенно хороших изданий. Русских авторов, по русски, у нас, уже
не издают. И привезти из Лабиринта тяжело.  Зачем читать Пушкина, Лермонтова и
пр. в переводе?  Современных зарубежных авторов издают только на украинском

Igor Kirzhner

Lana Brahina прекрасные советские издания сегодня легко купить в интернете. Стоят там очень не дорого. Намного дешевле новых.

Alex Fromoscow
Ильф с Петровым это тоже Булгаков, кстати)

Инна Яценко
Ваши рукописи не сгорят, остальное в топку)

Макс Бужанский
Инна Яценко, не горят, но и не греют)))

Инна Яценко
Макс Бужанский кого-то греют, кого-то жарят

Элина Гаврилюк-Конорезова

Макс Бужанский Спасибо вам за пост!!!

Оксана Макухо

И Аннушка мне кажется уже разлила масло Спасибо вам Максим за вши труды

Виктория Куколкина

А с гениями всегда так, они не вписываются ни в привычные рамки, ни в страну в
смартфоне. Зато они вечны, в отличие от рамок и смартфонов... да и стран, как
оказалось, тоже

Александр Романович

Виктория Куколкина страна Смартфония

Iryna Latysheva

Александр Романович я бы сказала \enquote{земля Смартфония}...

Александр Романович

Есть у нас один мемориал в Буче. Пойдем сегодня отдадим дань!

Всеволод Степанюк

Перечисленные понаехавшие, это обычные оккупанты нашей земли, а Булгаков всегда
будет великим кивеляниным.

Nadegda Nadosha

Вот вот, особенно «Бег» Булгакова - просто на злобу дня

Ирина Порохненко

\enquote{А швондеровщину признали официальным и высшим критерием патриотизма}.
Фраза супер!

Александр Сорин

Замечательно!

Oleg Popov

Ой, Максим Аркадьевич, я думаю что и в Вашей шобле, ой простите-фракции, тоже
немало тех, для которых фамилия Булгаков-мало что означает. Не говоря уже об
Ильфе и Петрове. Спросите их настоящие фамилии, и посмотрите как бывшие
свадебные фотографы, продавцы лотерейных билетов и прочие безработные, а ныне
Ваши коллеги по фракции \enquote{поплывут}. Ну а если зададите вопрос об авторстве
\enquote{Белой гвардии}-вобще увидите панику в их глазах. Вам же-респект и уважуха

Natella Basilaya

...И все равно, очень обидно и за Страну и за целое поколение нормальных людей
.... Булгаков - гений!!!! Спасибо за память за пост!

Іванна Яворська

Скоро не то что Булгакова, никого не будут знать современные дети. В Вашем
округе №25 в Чечеловском районе оставляют всего 3 лицея на целую кучу
гимназий-девятилеток. В десятые классы поступят от силы 20-25\% от всех
девятиклассников. Остальные - или на клубнику в Польшу, или в мелкий криминал.
Зачем им Булгаков?

Лара Крофт

Очень ёмко и в самую точку. Прям заколдованный лес, а клубочка все нет и нет.

Игорь Сидоренко

Лара, клубочек в действующей системе не предусмотрен...

Григорий Шульга

Да конечно. А как иначе. Если слуга сатаны. А Сатана все это ему показал.  И
вот что произошло. Все кто ипрочитал и услышал. ПОВЕРИЛИ и все произошло.  По
вере Вашей будет Вам.  Кто кому верит.  Я рекомендую покаяться за то что
связаны с сатаной. И примиритесь с Богом

Іванна Яворська

Вы хоть что-нибудь сделайте для образования нашего Чечеловского района,
повлияйте на бессовестный гумдепартамент Днепра. Потому что мы видим, что Вы
только о Булгакове разглагольствуете, а реального решения проблем по
образованию в Вашем избирательном округе нет.

Света Шевченко

Вся наша страна превратилась в \enquote{нехорошую квартиру}. Страшно, что
теперь навсегда...

Игорь Сидоренко

\enquote{Тьма, пришедшая со Средиземного моря, накрыла ненавидимый прокуратором город.
Исчезли висячие мосты, соединяющие храм со страшной Антониевой башней,
опустилась с неба бездна и залила крылатых богов над гипподромом, Хасмонейский
дворец с бойницами, базары, караван-сараи, переулки, пруды... Пропал Ершалаим -
великий город, как будто не существовал на свете. Все пожрала тьма, напугавшая
все живое в Ершалаиме и его окрестностях...}(с)

Irina Gruminskaya

Мой Киев - это Киев именно Булгакова! А вот Киев Парубия, Вятровича и Ницой....
никогда МОИМ не станет(((

Vladimir Chabanovskiy

С точки зрения констатации - все верно. Можно даже более красочно.
Только напомните мне, Макс в какой фракции?
В ОПЗЖ? в Батькивщине? На худой конец не к ночи буть помянутой ЕС?
Нет, он в активной части парламентской фракции слуг себя, составляющих монобольшинство, сформировавших и зянявших ВСЕ властные места и полномочия имеющая своего президента которого опять же активно и постоянно поддерживает Макс.
Поэтому чтобы хоть как-то понять и объяснить о чем Макс пишет напрашивается только заезженная тема про крестик и трусы.

Stefan Mashkevich

Общие принципы ведь действительно те же, что двести, что сто лет назад.
Поэтому действительно ничего принципиально и не меняется

Михаил Варшавский

Четко замечена хронология событий в Мастере и Маргарите и сегодня пока будут
ебанутые на всё голову ницои ветровичи и им подобные иметь возможность что то
решать и вещять в СМИ будет бардак....

Евгения Аброскина

Михаил Варшавский а разве вятровичи и ницои решают? Они просто демонстрируют
указующую \enquote{линию партии}. Была бы не такая политика правящего монобольшинства
во главе с Главным решателем, все эти швондеры сидели бы ровно на своих 5- х
точках и молчали.

Юлия Кораблева
Вы молодчина. Уважаю людей, которые говорят правду. Удачи Вам

Oleg Voloshin

Блестяще. Деградация общества очевидна. Скоро не найдётся в этой стране
человека, способного написать так, как ты, Макс. А ты, конечно, пишешь не так,
как те, кого ты здесь упоминаешь. И даже Ухудшанский будет уже не уровень - ему
некуда будет писать, когда все только смотрят. В этом смысле все уже потеряно.
Ты прав. Село победило Город. Потому Булгаков стал в своём Городе чужим. А
умный еврейский мальчик из США дал ещё этому селу право высказаться в любую
минуту. Восстание масс сделало интеллект столь же обременительным, как высшее
образование при большевиках. Лишь повод поставить редкого такого индивида к
стенке. Но большевики хоть сами были интеллектуалы, и могли эту массу двигать и
жать, как им хотелось, во имя свершений. Целый роман Булгаков просвещал одному
читателю - Сталину. Сейчас можно обойтись презентацией на пяти слайдах.

Пустота.

Макс Бужанский

Oleg Voloshin, это общий тренд, на упрощенность, не только наш(

Oleg Voloshin

Макс Бужанский да, естественно. Он начат там, откуда нами управляют. Но там у
них есть сверхэлита, которая этот тренд движет. И уровнями ниже элита проще. Я
пока не встречал сотрудника американского посольства в Киеве, которому бы в
разные годы при описании мотивов поступков Януковича ли или Зеленского я не мог
бы сказать про «дилемму Раскольникова про тварь и право», и чтоб он при этом не
знал, о чем речь.

Макс Бужанский

Я этих сотрудников пока вообще не встречал, к счастью)))

Oleg Voloshin

Макс Бужанский знаешь, иногда иностранцы в Киеве, и есть самые разумные
собеседники. Многое понимают. Но их боссы все равно ставят на Лещенков и
Наемов. Именно потому, что четко понимают, в каком виде им нужна Украина

Валенин Хмырев

Oleg Voloshin думаю, что сейчас, в некогда одной из самых грамотных и читающих
наций, немного найдётся людей, способных правильно написать слово \enquote{длинношеее}, а
процент, что-либо знающий о Булгакове, и того ниже. Сегодня прочитал пост одного
из \enquote{лидеров мнений}, где утверждалось, что Булгаков-украинофоб.

Oleg Voloshin

Валенин Хмырев естественно. Потому что он был хамофобом. А для многих это одно
и то же. Меня тоже коробит от суржика и ограниченности суждений в духе паствы
Пороха. С гордостью готов считаться «украинофобом»

Ольга Чечулина

Нет уж, рукописи не горят! А Родина не умирает. Это сам Булгаков написал, а
1919-м году: \enquote{Грешно думать. что Родина умирает}.

Leon Lax

Булгаковская белая гвардия обнуляет все достижения украинской историографии и
института национальной памяти за последнии 30 лет. Поэтому его так не любит
киевская власть

Tatyana Inozemtseva

Булгаков мастер, умело стирающий грань между реальностью, вымыслом и мистико!!!
Вы, Макс, мастер аналитики, ассоциаций и своего стиля написания!!

Надежда Тольева

Проблема этого мира в том, что глупцы и фанатики слишком уверены в себе, а
умные люди полны сомнений.

Alexander Abend

Булгаков не просто \enquote{не киевлянин}. Его произведения в украинских школах проходят в рамках курса \enquote{Зарубежная литература}.

Сергеев Олег

А ещё кроме того, что Булгаков КИЕВЛЯНИН, он врач и все его гениальные творения
начинались в большей степени из профессии и глубокого анализа человеческих душ.


\ifcmt
  pic https://scontent-frt3-2.xx.fbcdn.net/v/t1.6435-9/186530608_4286946391354685_6734841365437790967_n.jpg?_nc_cat=103&ccb=1-3&_nc_sid=dbeb18&_nc_ohc=KxS8y9d7FyoAX92T08t&_nc_ht=scontent-frt3-2.xx&oh=d7bca5dc1bc7cbd2ebfaec663f166243&oe=60C57F91
\fi

Nicolas Petrov
ФИНАЛ

Город прекрасный, город счастливый. Над разлившимся Днепром, весь в зелени
каштанов, весь в солнечных пятнах.  Сейчас в нем великая усталость после
страшных громыхавших лет. Покой.  Но трепет новой жизни я слышу. Его отстроят,
опять закипят его улицы, и станет над рекой, которую Гоголь любил, опять
царственный город. А память о Петлюре да сгинет[12].

Вадим Закк

На мой взгляд -- именно майдан кардинально изменил количество Швондеров и
Шариковых в Киеве , именно майдан размножил и умножил рагулизм и абсурд и
глупость в любой точке страны

Иван Шолль

Паустовский, Булгаков, Вертинский - все киевляне, более того, летописцы Киева.
Не важно кто что говорит, все уже сказано в их произведениях, наполненных
любовью к городу. А Ницой... Для нее Киев - Тамбов. Мы же не считаем киевлянами
немцев, проживавших здесь во времена Скоропадского.

Larisa Slutskaya

Иван Шолль интересно КЕМ же Вы считаете этих немцев или поляков и их потомков,
проживших в Киеве сотню лет? А потомков Ивановских ткачей или Питерских
матросов, поселившихся в Киеве в эти же годы, или после 45го Вы тоже киевлянами
не считаете?.

Konstantin Kurenov

Иван Шолль ты наш Тамбов не трожь, у вас Жмеринка есть!

Volodymer Volk

Действительно, Макс, иногда стыдно называться киевлянином стало. Столько дерьма
за последние годы столица впитала.

Геннадий Аксаментов
Рукописи не горят.

Александр Маковецкий

Какая чудесная избирательность. Игорь Сикорский - киевлянин. Булгаков - нет. В
связи с этим у меня вопрос к якобы \enquote{патриотам} Украины. Вы запретили советскую
символику. ДАЖЕ на 9 мая. А чё не отказались от Днепрогэса? Это ж палач Сталин
\enquote{на костях украинцев} лютовал. Электричество совковое в паху не жмёт? Чё не
отказались от символов совка - хрущёвок? Чё не выселяемся? Да и остальные 80\%
жилого фонда - проекты ненавистного совка. Ну хорошо хоть совковые предприятия
развалили. Антонов - исчадие совка когда уже закроете? Там ведь сама суть -
крамола! Антонов же ярый москаль! Из московской области мужик! Ну хорошо хоть
тут вы последовательны. Скоро закроете. Но с совковым жилым фондом нужно что то
решать. Срочно!

Stanislaw Mikhailov

Александр Маковецкий, да и Сикорский, будучи в эмиграции, состоял сразу в
нескольких русских монархических сообществах. Цэ Зрада, ящетаю)

Nicolas Petrov

К 130-летию Михаила Булгакова.
О Киеве его юности
\url{https://youtu.be/0PyQtY5pneA}

Sergey Bykov

Все зеркально у нас по Булгакову.

Наталия Подгурская

Он не угадал, он призван был отдать нам правду через фантастику и был членом
параллельной цивилизации.

Elina Frolova

Как всегда, на высоте!

Олександр Секрета

Одразу видно, теж науковець, як і Ківа

Игорь Федоров

Ницой куда лучше Булгакова! И пишет по украински! А не читают ее просто из
вредности. Ничего-еичего, старуха это еще возьмет свое! Куда там какому-то
Булгакову до нее. Он ведь рыбу выращивает и ей торгует.

Stanislaw Mikhailov

Игорь Федоров, да еще и мелочью в людей не кидается, как какой-нибудь буфетчик)

Игорь Федоров

Stanislaw Mikhailov Ницой, она почти как Астрид Линдгрен! Да чего там, она
почти как сами Гога и Магога!

Ирена Березник

Игорь Федоров, она Грета Тунберг от литературы!

Игорь Федоров

Ирена Березник и она и мать ее!

Светлана Гераскина Самохина

результат наступившей реальности-все меньше людей того \enquote{ужасного, преступного
совдеповского} образования и все больше и везде везде \enquote{специалисты}
реформированного - \enquote{взлет-посадка}!

Oskar Kontorskiy

Светлана Гераскина Самохина согласен, но к сожалению , возвратиться к этому
образованию вам не дадут..... они бояться этого. Увидите на выборах кого народ
выберет. Народ доведён до уровня рабов. Главное это наполнить желудок. Когда
это происходит страна и с ней нац. культура гибнет

Олег Голиков

Если бы то, к чему призывает Макс Бужанский, свершилось, многие бывшие граждане
Украины вернулись бы. Большая их часть.

Vadik Kulish

Это нормально. Когда НЕ ВСЕ понимают. Просто, ждут РАЗОБЛАЧЕНИЯ.

Oskar Kontorskiy

Величество народа определяется умением слушать и анализировать то что
предсказали лучшие сыны этого народа. А не умение это делать приведёт к
трагедии. Ведь это правда? Не так ли?????

Станислав Стасевский

Макс, как всегда супер. Добавил бы только одно, что Вы либо не разделяете, либо
не считаете нужным сказать. Мы семимильными шагами скатываемся к большевизму,
только с другим знаком или цветом, не знаю как правильнее сказать.

Sergey Bykov

Булгаков гениален. Он опередил время. Мы, киевляне, очень им гордимся!

Елена Гордеева

Первый Ваш текст, который понравился.

Макс Бужанский

Елена Гордеева, давайте считать все предыдущие - несуществующими ))

Игорь Федоров

Булгаков не актуален, это всем понятно. Необходимо срочно убедить Парубия
написать такое...чтобы все поняли, кто он, а кто Булгаков. Пусть напишет о
Таньке Черновол (чем не Маргарита?) и о великом мастере...да хоть Яценюке! Чем
плох? Нет, так есть еще Луценко со своей бабой третьего тысячелетия, ну или
казак Гавриляк? Или козюк Гаврилюк - как правильно?

Sergey Dorodnykh

— Сволочь он, — с ненавистью продолжал Турбин, — ведь он же сам не говорит на
этом языке! А? Я позавчера спрашиваю этого каналью, доктора Курицького, он,
извольте ли видеть, разучился говорить по-русски с ноября прошлого года. Был
Курицкий, а стал Курицький… Так вот спрашиваю: как по-украински «кот»? Он
отвечает «кит». Спрашиваю: «А как кит?» А он остановился, вытаращил глаза и
молчит. И теперь не кланяется.

Николка с треском захохотал и сказал:

— Слова «кит» у них не может быть, потому что на Украине не водятся киты, а в
России всего много. В Белом море киты есть…

Alexandr Kolotilov

И Некрасов,

Надіія Троусковетска

Олег Ольжич тоже киевлянин. Но его не любил Сталин, ему пришлось уехать. Он был
учёный археолог,поэт,но его убили в Зальцнхаузене за идею независимой родной
страны.Убили нацисты.У него был юбилей не так давно.Зачем нам русские
писатели?обласканные режимом?ведь все его пьесы шли в театрах,фильмы
экранизированы.Последний роман был проблемным,но он не для широкой тогдашней
публики.Придет русский мир-Вас лично уберут с первой линии,то не старайтесь так
сильно.Кстати,был перевод Двенадцати стульев на укр.мову ,вышел в конце 60 х.Во
где ржака.

Сергей Удовик

Вятровичи порубии и иже с ними это носители рустикальный кодов, враждебных
буржуазному Городу, как Булгаков назвал Киев с большой буквы. Это типичная
оккупация, которая началась в 1917-20 и продолжается поныне. А о киевлянах
настоящих я написал в книге, многие из них стали культуртрегерами на западе

\ifcmt
  pic https://scontent-frt3-1.xx.fbcdn.net/v/t1.6435-9/187310947_1438512589816286_2104480853816567366_n.jpg?_nc_cat=104&ccb=1-3&_nc_sid=dbeb18&_nc_ohc=NDBX3qzLeCwAX-lEW8_&_nc_ht=scontent-frt3-1.xx&oh=26724800dbc75e42e0cc477aace44f6b&oe=60C41BC0
\fi

Людмила Поддубцева

Максим ну вы и сравнили с Ницой, Булгаков, я думаю, не был такой истеричный.
