% vim: keymap=russian-jcukenwin:fenc=utf8
%%beginhead 
 
%%file 09_11_2020.fb.vladimir_tutenko.1.mova_nestor
%%parent 09_11_2020
 
%%url https://www.facebook.com/BatushkaV/posts/1607649782748997
%%author 
%%tags 
%%title 
 
%%endhead 

\subsection{День української писемності та мови - отець Володимир Тютенко}
\label{sec:09_11_2020.fb.vladimir_tutenko.1.mova_nestor}
\Purl{https://www.facebook.com/BatushkaV/posts/1607649782748997}
\Pauthor{Тютенко, Володимир}

\ifcmt
pic https://scontent.fiev6-1.fna.fbcdn.net/v/t1.0-0/p180x540/124827802_1607649722749003_2618732290872547487_n.jpg?_nc_cat=110&ccb=2&_nc_sid=730e14&_nc_ohc=QY5uqXQB85kAX93d_js&_nc_oc=AQmygnE6FZDR4s4emAigjWBs8QqEzaRdJm6LND_e3Qs7-ELJfEfvHSWLsT1Mmtw080o&_nc_ht=scontent.fiev6-1.fna&tp=6&oh=5beb793646ead242ee4b37ab37b125a0&oe=5FDAAAD4
\fi

День української писемності та мови --- свято розвитку державної мови, яке щороку
відзначається в Україні 9 листопада. 

За православним календарем --- це день вшанування пам'яті Преподобного
Нестора-Літописця --- послідовника творців слов'янської писемності Кирила і
Мефодія.  За традицією відбувається покладання квітів до пам'ятника прп.
Нестора-літописця. 

Саме прп. Нестор в своїй праці \enquote{Повесть временных лет} оставив для нас історію
творення нашого давньоруського народу: \enquote{Вот повести минувших лет, откуда пошла
Русская земля, кто в Киеве стал первым княжить и как возникла Русская земля}. З
цієї мови і мови Києво-Могилянської академії свт. Петра Могили і вийшли сучасні
російська та українська мови, тому їх потрібно і тепер знати разом, щоб нам не
відриватися від свого духовного спадку. Я за двомовність і за те, щоб ми мали
повагу до кожної особистості, незалежно від того, яка мова в неї рідна і на
якій мова вона розмовляє. З українською мовою не має проблем, вона вивчається
по всій Україні та об'єднує всіх українців, тому не зрозуміло, як це ми
докотились до того, що боротьба та приниження  російськомовних українців стало
держаною політикою. Україна складається з регіонів, які мають свою історію
розвитку, а тому закономірно, що є свої регіональні історичні, культурні та
мовні особливості. Повинно бути поєднання загальноукраїнських цінностей з
культурною автономією регіональних  особливостей.  Основою здорової єдності
громадян України може бути тільки  відчуття комфортності життя в країні,
комфортності, яка залежить не тільки від економіки, але і відчуттям внутрішньої
свободи свого духовного життя. Тому я занепокоєний тим, що нам нав'язали
боротьбу з російською мовою, Українською Православною Церквою, героями Великої
Вітчизняної війни - це нас тільки роз'єднує!  

Тому давайте будемо людьми при любих обставинах і незалежно від державної
політики! Давайте своєю добротою і терпимістю в транспорті, магазині, установах
будувати комфортне життя, труднощів і так хватає!  Преподобний отче Несторе,
моли Бога о нас!
