% vim: keymap=russian-jcukenwin
%%beginhead 
 
%%file slova.terror
%%parent slova
 
%%url 
 
%%author 
%%author_id 
%%author_url 
 
%%tags 
%%title 
 
%%endhead 
\chapter{Террор}
\label{sec:slova.terror}

%%%cit
%%%cit_pic
%%%cit_text
Убитые \emph{украинскими террористами} военнослужащие Народной милиции ЛНР защищали
свой родной Донбасс, свои дома и семьи. Об этом ЛИЦ заявил советник главы ЛНР
по вопросам казачества, кадетского образования и военно-патриотического
воспитания молодежи Сергей Юрченко. Украинская диверсионно-разведывательная
группа 11 июня совершила нападение на наблюдательный пост Народной милиции в
районе поселка Голубовское, погибли пятеро защитников Республики. Похороны
погибших прошли сегодня на Аллее Славы центрального кладбища Кировска
%%%cit_comment
%%%cit_title
\citTitle{Луганский Информационный Центр — Убитые террористами воины защищали свой родной Донбасс, дома и семьи – советник главы ЛНР}, , lug-info.com, 14.06.2021
%%%endcit

%%%cit
%%%cit_pic
%%%cit_text
\enquote{Подонки убили наших воинов во время перемирия. Уверен, что это было не
простое подразделение, не простые диверсанты, а именно \emph{террористы}. Ведь
это (погибшие защитники Республики) были не какие-то российские морские котики,
а простые бывшие шахтеры и рабочие – жители Донбасса, которые здесь родились и
с оружием в руках защищали свой дом, свои семьи, свою землю}, - сказал он.
Юрченко подчеркнул, что ополченцы Донбасса, а затем военнослужащие Народных
милиций Республик Донбасса за все время конфликта никогда не совершали таких
гнусных бесславных поступков, как это сделала украинская ДРГ
%%%cit_comment
%%%cit_title
\citTitle{Луганский Информационный Центр — Убитые террористами воины защищали свой родной Донбасс, дома и семьи – советник главы ЛНР}, , lug-info.com, 14.06.2021
%%%endcit


%%%cit
%%%cit_head
%%%cit_pic
%%%cit_text
Национальная полиция Украины хочет взыскать с "\emph{мостового террориста}" Алексея
Белько, которого признали невменяемым и поместили в психиатрическую больницу,
имущественный ущерб в размере 69 871 грн за сбитый полицейский коптер.  Об этом
рассказал адвокат подсудимого Маси Найем на своей странице в Facebook.
"Закончилось досудебное расследование по делу ветерана АТО, Алексея Белько.
Спасибо Службе безопасности Украины за адекватное отношение к делу ветерана. Не
благодарю Национальную полицию Украины. Потому что они, зная уже результаты
экспертизы, подали гражданский иск в рамках уголовного. А там просят взыскать с
Алексея имущественный ущерб в размере 69 871 грн за сбитый коптер Dji inspire 1
pro", - написал он
%%%cit_comment
%%%cit_title
\citTitle{С \enquote{мостового террориста} Белько хотят взыскать 70 тысяч за сбитый дрон}, 
, strana.ua, 29.01.2020
%%%endcit
