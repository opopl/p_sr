% vim: keymap=russian-jcukenwin
%%beginhead 
 
%%file 15_07_2021.fb.bielskaja_elizaveta.1.peredoz_rosijskoju
%%parent 15_07_2021
 
%%url https://www.facebook.com/elizabeth.bielska/posts/4161112480590886
 
%%author Бельская, Елизавета
%%author_id bielskaja_elizaveta
%%author_url 
 
%%tags jazyk,kiev,mova,ukraina,ukrainizacia
%%title ПЕРЕДОЗУВАННЯ РОСІЙСЬКОЇ
 
%%endhead 
 
\subsection{Передозування Російської}
\label{sec:15_07_2021.fb.bielskaja_elizaveta.1.peredoz_rosijskoju}
 
\Purl{https://www.facebook.com/elizabeth.bielska/posts/4161112480590886}
\ifcmt
 author_begin
   author_id bielskaja_elizaveta
 author_end
\fi

ПЕРЕДОЗУВАННЯ РОСІЙСЬКОЇ.

Бувають дні, коли в Києві отримую передозування російської у сфері
обслуговування. І хоча зазвичай переконую себе не псувати власний настрій
створенням конфліктної ситуації, все ж намагаюся дихати глибше, щоб вгамувати
серце, котре вистрибує із грудей, і максимально доброзичливо й тактовно знову і
знову заводжу розмову про мову.

\ifcmt
  pic https://scontent-cdg2-1.xx.fbcdn.net/v/t1.6435-9/217563348_4161112380590896_917119908598819775_n.jpg?_nc_cat=102&ccb=1-4&_nc_sid=8bfeb9&_nc_ohc=kTCqdKA9KyYAX_Bqj1s&_nc_ht=scontent-cdg2-1.xx&oh=89b90dba4ece140ce2beee5b4133f863&oe=6134E34A
  width 0.4
\fi

Крамниця жіночого одягу. За касою дівчина років двадцяти. На мою українську
відповідає російською. Запитую:

\obeycr
- Скажіть, будь ласка, чому ви обслуговуєте мене російською? Ви не знаєте про закон? Чи не чуєте, що  мова вашого клієнта українська?
- Патамушта я всьо время разгаваріваю на русском. По-украінскі тоже магу, но только тагда, кагда мєня попросят.
- Тобто я маю просити?
- Ну да.
- Гаразд, у такому разі я вас прошу.
- Добре (інтонація зневажлива). З вас 1800 гривень. Пакет потрібен?
\restorecr

За мною пані, що чула нашу розмову. Усміхається. Мабуть, їй цей діалог видався
маленькою перемогою "наших". Вітається з касиром українською. Чує відповідь
російською. Її усмішка зникає.

Це скидається на якесь зачароване коло. Так втомливо почуватися у власній
країні, наче в діаспорі, і щодня приймати 5-6 рішень про те, чи витрачати час і
зусилля на вимогу дотримуватися закону і не порушувати права українськомовних
громадян в Україні. Звісно, є ті, кому я кажу: "Дякую вам сердечно за
українську". Але рахунок надто нерівний.

Друзі, а що ви робите, коли ваші мовні права споживача порушують? Мовчите і
намагаєтеся не реагувати? Просите обслуговувати вас рідною мовою? Що ви кажете,
щоб бути переконливими, але не спричинити скандал? Чи краще спричинити? 🙂 Що є
дієвим для вас?

\ii{15_07_2021.fb.bielskaja_elizaveta.1.peredoz_rosijskoju.cmt}
