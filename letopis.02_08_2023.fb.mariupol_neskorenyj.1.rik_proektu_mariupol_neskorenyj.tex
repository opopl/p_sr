%%beginhead 
 
%%file 02_08_2023.fb.mariupol_neskorenyj.1.rik_proektu_mariupol_neskorenyj
%%parent 02_08_2023
 
%%url https://www.facebook.com/100066312837201/posts/pfbid0gcZE3UkVpSYFzZZcX21HrcGZbwmSprMPQzGiS9kwkEArwRLjuziynzHuPGapFnBrl
 
%%author_id mariupol_neskorenyj
%%date 02_08_2023
 
%%tags 
%%title Рік проєкту "Маріуполь нескорений"
 
%%endhead 

\subsection{Рік проєкту \enquote{Маріуполь нескорений}}
\label{sec:02_08_2023.fb.mariupol_neskorenyj.1.rik_proektu_mariupol_neskorenyj}

\Purl{https://www.facebook.com/100066312837201/posts/pfbid0gcZE3UkVpSYFzZZcX21HrcGZbwmSprMPQzGiS9kwkEArwRLjuziynzHuPGapFnBrl}
\ifcmt
 author_begin
   author_id mariupol_neskorenyj
 author_end
\fi

📎 Виставка в рамках проєкту \enquote{Маріуполь нескорений}, що виникла як прояв
підтримки України маріупольськими художниками та митцями з усього світу, за рік
об’їхала кілька країн Європи.

Проєкт започатковано в червні 2022 року. Тоді в Київському будинку художника
відкрилась виставка, що включала дві колекції: \enquote{Маріуполь – душа
України} та \enquote{Нескорена Україна}. Представлені у виставці роботи
символізують боротьбу та незламність українського народу. Окремі полотна
переносять глядачів у спокійну довоєнну реальність, інші демонструють жахливі
випробування, нав'язані війною, та українських захисників.

📑 За словами голови Національної спілки художників України Костянтина
Чернявського, презентована виставка особисто для нього тісно пов'язана з
Маріуполем та Художнім музеєм ім. А. І. Куїнджі. Адже жорстока війна,
розпочата 24 лютого 2022 року Російською Федерацією, змінила життя кожного
українця і не може залишити байдужим жодного митця.

👨🎨 Художники з Києва, Дніпра, Одеси, Франківська та інших куточків України
об’єднались, щоб підтримати маріупольських колег. Організована в Києві виставка
мала актуальну та благородну мету – через витвори мистецтва розкрити світу
жахливу та незміриму трагедію України, катастрофічну долю Маріуполя.

👏Саме так була започаткована серія експозицій \enquote{Маріуполь нескорений}.
Ініціатором та куратором  цього задуму, який тепер вже перетворився на
колосальний проєкт міжнародного рівня, виступив Валерій Юричковський,
маріупольський митець та промоутер, почесний член Національної спілки
художників України.

Після експонування в Києві виставка подорожувала західною частиною України та
навіть за межі держави. Правда, щоразу на різних локаціях варіювалась і
кількість картин в експозиції. Проте головне наповнення та меседж проєкту, а
саме, продемонструвати світові актуальні події сучасної епохи, міць, моральну
силу та любов до Батьківщини всього українського народу, залишались незмінними.

🗓28 липня 2022 року, до Дня Української Державності проєкт було презентовано в
Івано-Франківську у Музеї мистецтв Прикарпаття. Окрім картин на виставці
представили дві арт-інсталяції: державний прапор, привезений військовими із
Маріуполя, по якому пройшлися ворожі кулі, та металевий простріляний лист з
\enquote{Азовсталі}, що символізує понівечене місто. 

Експозиція в Івано-Франківську тривала кілька місяців. Далі географія виставки
значно розширилась, колекція стала доступна відвідувачам у Європі.

Вихід арт-проєкту на міжнародний рівень розпочався з Риги, де виставка
експонувалась з 5 по 25 листопада 2022 року у Галереї Спілки художників Латвії.

✅️ Глядачам були представлені роботи 40 художників України, створені ще до
початку повномасштабної війни та у похмурі дні лютого - березня, коли
демократичний світ спостерігав за незбагненною жорстокістю російських
загарбників та неймовірною мужністю українських захисників \enquote{Азовсталі} у
Маріуполі.

З кінця березня до середини квітня 2023 року роботи митців демонструвались у
Вільнюській Ратуші. Колекція включала вже картини не тільки українських
художників. Зокрема, литовський автор Юрій Григорович представив портрет свого
співвітчизника документаліста Мантаса Кведаравічюса, який був убитий у квітні
2022 року в оточеному Маріуполі під час документування вторгнення. Ця робота
під назвою \enquote{Триптих}  слугує пронизливим нагадуванням про військові злочини,
скоєні російськими окупантами.😪

Міжнародна подорож проєкту \enquote{Маріуполь нескорений} продовжилась у Варшаві.
Презентацію виставки, що відбулась 13 квітня 2023 року у головній будівлі Сейму
Польщі, відвідали дипломати багатьох країн, які на власні очі побачили страшну
правду про війну та про звірства росіян, відображені у художніх полотнах.

13 травня 2023 року картини маріупольських, українських та світових художників
презентували у Художній Галереї Везер в німецькому місті Бремен. Для всіх
відвідувачів було передбачено вільний вхід, тож, широка публіка змогла
долучитись до мистецьких робіт, що розповідають про біль та трагедію українців,
нашу боротьбу за вільне і світле майбутнє.

📍На кожній міжнародній виставці колекція проєкту \enquote{Маріуполь нескорений}
поповнювалась роботами художників Латвії, Литви, Польщі та інших країн. Так
митці висловлювали підтримку та солідарність з українськими художниками. У
серпні поточного року виставка, головна мета якої популяризувати міжнародну
підтримку боротьби України за незалежність через мовою художніх образів,
готується до презентації у київській міській галереї мистецтв \enquote{Лавра}. Там буде
представлено близько 60 художніх полотен.❗️

🗺Географія виставки \enquote{Нескорений Маріуполь}:

28 червня - 17 липня 2022 – Київський будинок художника, виставка що включала
дві колекції \enquote{Маріуполь – душа України} та \enquote{Нескорена Україна}.

28 липня 2022 – відкриття виставки в Івано-Франківську, у Музеї мистецтв
Прикарпаття. Експонування тривало кілька місяців.

5-25 листопада 2022 – Рига (Латвія), Галерея Спілки художників Латвії.

30 березня 2023 – Вільнюс (Литва), відкриття виставки у Вільнюській Ратуші.

13-18 квітня 2023 – Варшава (Польща), головна будівля польського Сейму.

13 травня - 10 червня 2023 – Бремен (Німеччина), Художня Галерея Везер.

\#Маріупольнескорений \#виставка \#культурнадеокупація \#Маріуполь \#Київ
\#географіявиставки
