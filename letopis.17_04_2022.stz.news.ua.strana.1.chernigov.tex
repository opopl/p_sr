% vim: keymap=russian-jcukenwin
%%beginhead 
 
%%file 17_04_2022.stz.news.ua.strana.1.chernigov
%%parent 17_04_2022
 
%%url https://strana.news/articles/special/386709-chernihov-posle-vojny-reportazh-i-rasskazy-mestnykh-zhitelej-o-zhizni-pod-bombami.html
 
%%author_id news.ua.strana,tovt_anastasia
%%date 
 
%%tags 
%%title "И стреляют, и стреляют, ночами не спишь". Как Чернигов пережил войну. Репортаж "Страны"
 
%%endhead 
 
\subsection{\enquote{И стреляют, и стреляют, ночами не спишь}. Как Чернигов пережил войну. Репортаж \enquote{Страны}}
\label{sec:17_04_2022.stz.news.ua.strana.1.chernigov}
 
\Purl{https://strana.news/articles/special/386709-chernihov-posle-vojny-reportazh-i-rasskazy-mestnykh-zhitelej-o-zhizni-pod-bombami.html}
\ifcmt
 author_begin
   author_id news.ua.strana,tovt_anastasia
 author_end
\fi

Тихо. Такая устрашающая тишина бывает только на войне.

Угрюмое безмолвие нарушает ветер, холодный и пронизывающий. Слышно, как железо
бьется о железо — это ветер играет остатками дырявого металлического забора,
перекошенного и изрешеченного пулями. Сквозняк гуляет по обвалившимся стенам
домов и воет, как брошенная собака.

\ii{17_04_2022.stz.news.ua.strana.1.chernigov.pic.1}

Всюду разбросаны обломки кирпичей, досок и порванные детские игрушки. Окна
домов, в которых еще месяц назад горел свет, стали чёрными бесформенными
пятнами. Оконные рамы обуглились, стекла выбиты, крыши обвалились. Дома
выглядят так, будто их прожевали и выплюнули, а затем еще и растоптали. Там,
где раньше стояли чьи-то жилища, сейчас — груда камней и пепел. 

Сплошная череда обломков и развалин — вот то, что осталось от села Новоселовка
под Черниговом. Ему пришлось принять на себя сильнейший удар противника.
Новоселовке не повезло встать на пути российских войск, которые пытались через
нее прорваться в Чернигов. А когда не смогли, практически стерли село с лица
земли. 

Еще несколько недель назад здесь не стихали канонады: постоянные выстрелы,
взрывы, обстрелы. Но враг отбит — российские войска ушли в начале апреля.
Сейчас в Новоселовке тихо. Даже слишком. 

Местные признаются, что жить под бомбежками было страшно. Но привыкли. А
сейчас, когда стало тихо — еще страшнее. "Все время думаешь: а вдруг опять?.."

Снова слышен звук металла. Но другой. Ритмичный скрип… Оборачиваюсь и не верю
глазам: это мальчик катается на качели. 

Удивительно, как в этом полуразрушенном селе под бомбежками уцелела детская
площадка. Совсем новая, цветная, желто-красная — сюрреалистичное яркое пятно на
фоне черно-серого послевоенного пейзажа. Как будто эту площадку вырезали в
фотошопе из снимка довоенного времени и искусственно вставили сюда, словно все
это — картина неизвестного постмодерниста. Но реальность превосходит любой
художественный вымысел. О том, что здесь кровавой поступью прошлась война,
красноречиво напоминает воронка от снаряда прямо возле детской горки. 
