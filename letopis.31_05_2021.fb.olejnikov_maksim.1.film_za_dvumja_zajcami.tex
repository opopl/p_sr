% vim: keymap=russian-jcukenwin
%%beginhead 
 
%%file 31_05_2021.fb.olejnikov_maksim.1.film_za_dvumja_zajcami
%%parent 31_05_2021
 
%%url https://www.facebook.com/maksym.oleynikov.7/posts/851715062084864
 
%%author Олейников, Максим
%%author_id olejnikov_maksim
%%author_url 
 
%%tags 
%%title До Дня Києва... - За двома зайцями, фильм
 
%%endhead 
 
\subsection{До Дня Києва... - За двома зайцями, фильм}
\label{sec:31_05_2021.fb.olejnikov_maksim.1.film_za_dvumja_zajcami}
\Purl{https://www.facebook.com/maksym.oleynikov.7/posts/851715062084864}
\ifcmt
 author_begin
   author_id olejnikov_maksim
 author_end
\fi

До Дня Києва...

Це один із найбільш «київських» кінофільмів, вивчений напам'ять кількома
поколіннями, у якому чи не кожний епізод  дихає місцевими інтонаціями і
топонімами…


\ifcmt
tab_begin cols=2

  pic https://scontent-frt3-2.xx.fbcdn.net/v/t1.6435-9/193098260_851713625418341_6929497534006904041_n.jpg?_nc_cat=101&ccb=1-3&_nc_sid=730e14&_nc_ohc=goLi9byP5E0AX9XOcy-&_nc_ht=scontent-frt3-2.xx&oh=b66af10793486c624531a3781232f5bb&oe=60D9757F

	pic https://scontent-frt3-2.xx.fbcdn.net/v/t1.6435-9/193108870_851713668751670_7021062697217859518_n.jpg?_nc_cat=103&ccb=1-3&_nc_sid=730e14&_nc_ohc=VTCLrRqiLegAX8F2JqG&_nc_ht=scontent-frt3-2.xx&oh=bd09f4f8b10f32505bd8a06c16889277&oe=60D92358
	caption Віктор Михайлович Іванов (1909-1981), кінорежисер і сценарист, Заслужений діяч мистецтв УРСР.

tab_end
\fi

Прем'єра фільму «За двома зайцями» відбулась 21 грудня 1961 року у Будинку
культури залізничників Дарницького залізничного вузла м.Києва (зараз це палац
культури «Дарниця» на вул.Заслонова, 18). Чому прем'єра пройшла у такому зовсім
непрестижному залі на тодішній окраїні міста? Тому, що Держкіно СРСР присвоїло
фільму всього лише дуже скромну 2-у категорію, що зазвичай означало – фільм
вийшов невдалим, але витрачені кошти як-небудь треба повернути. Така категорія
передбачала обмежений прокат (тільки на території Української РСР), друк
невеликої кількості копій і показ переважно у будинках культури, заводських
клубах і периферійних кінотеатрах. 

\ifcmt

tab_begin cols=2
  pic https://scontent-frt3-1.xx.fbcdn.net/v/t1.6435-9/193108801_851713718751665_5600690036629108979_n.jpg?_nc_cat=109&ccb=1-3&_nc_sid=730e14&_nc_ohc=JhaKkpWL56wAX-5U-6u&_nc_ht=scontent-frt3-1.xx&oh=407d2ee85e39473492c8114888cbd967&oe=60DA576D

	caption Таким був Микола Гриценко у той час, коли міг зіграти у фільмі Голохвостого. Фотолистівка 1950-х рр.

	pic https://scontent-frt3-2.xx.fbcdn.net/v/t1.6435-9/194469716_851713762084994_4838039009458364042_n.jpg?_nc_cat=103&ccb=1-3&_nc_sid=730e14&_nc_ohc=xVpZMTcPazwAX_v5YwU&_nc_ht=scontent-frt3-2.xx&oh=09fb6b5d31657d1fd13ebbf2345af7ca&oe=60DAB1CF

	caption Майя Булгакова, яка могла зіграти у фільмі Проню Прокопівну. Фото 1957р.

tab_end
\fi

Однак успіх фільму у глядачів був шаленим, люди буквально штурмували каси. Така
популярність змінила прокатну долю фільму. Його переозвучили російською мовою
(ті ж виконавці) і випустили у всесоюзний прокат. Початкова українська версія
фільму майже піввіку вважалася втраченою і була випадково знайдена у Маріуполі
лише у 2013 році. 

Народження фільму було теж непростим. Режисер Київської кіностудії
ім.О.Довженка Віктор Іванов не відразу отримав дозвіл на екранізацію п'єси
Михайла Старицького, написаної у 1883 році (Старицький з дозволу автора
переробив малосценічну п'єсу І.Нечуя-Левицького «На Кожум'яках» 1875р.). П'єса
висміювала життя українських русифікованих міщан Києва наприкінці ХІХст. 


\ifcmt
tab_begin cols=2

  pic https://scontent-frt3-1.xx.fbcdn.net/v/t1.6435-9/193469955_851713822084988_9053323408823773592_n.jpg?_nc_cat=106&ccb=1-3&_nc_sid=730e14&_nc_ohc=OLHCzAtpEwUAX883C4P&_nc_ht=scontent-frt3-1.xx&oh=a3a581cc23f48e31e7fabfc133307b60&oe=60DC1A57
	caption Такою Була Маргарита Криницина на момент зйомок фільму «За двома зайцями». Фотолистівка початку 1960-х рр.

	pic https://scontent-frt3-1.xx.fbcdn.net/v/t1.6435-9/193246647_851713942084976_5087790600398064550_n.jpg?_nc_cat=104&ccb=1-3&_nc_sid=730e14&_nc_ohc=82Lu9jjZRQ8AX89IDaS&_nc_ht=scontent-frt3-1.xx&oh=e1a6395c49c055d1e8fe9195d576b299&oe=60D97BC3
	caption Режисер Віктор Іванов на зйомках фільму «За двома зайцями».

tab_end
\fi


Чи то сюжет здавався несерйозним і неактуальним, чи то режисер - недостатньо
досвідченим (В.Іванов до цього працював лише на «провінційних» студіях –
Свердловській, Вільнюській, Каунаській), але кінематографічне начальство заявку
відхиляло. Врешті В.Іванов пішов на хитрість: в черговий раз подаючи на розгляд
керівництва свій сценарій, він вказав, що хоче сатирично зобразити «стиляг» -
так тоді називали молодь, яка на думку партійного керівництва занадто
цікавилась модою, стильними закордонними речами, західною музикою і т.п. Тоді
як раз набирала оберти кампанія боротьби із «стилягами», і пропозиція Іванова
таки отримала погодження. 

Робота над фільмом почалась у 1960 році. Більшість сцен знімали в павільйоні
кіностудії, деякі епізоди – на київських вулицях. На екрані з легкістю
упізнається Андріївський узвіз, Контрактова площа, старий Житній ринок,
Володимирська гірка, старі квартали Подолу - Боричів тік, Гончарна,
Воздвиженська і т.д. 

\ifcmt
tab_begin cols=2

  pic https://scontent-frt3-1.xx.fbcdn.net/v/t1.6435-9/193348335_851714002084970_1403036111960610280_n.jpg?_nc_cat=107&ccb=1-3&_nc_sid=730e14&_nc_ohc=grEXWV8e37gAX9X2Fq3&tn=ntrKbsW_7ChXu3v-&_nc_ht=scontent-frt3-1.xx&oh=968ed0ec9c55f9d25e5fd238c3a69c57&oe=60D89A3F
	caption Меморіальна дошка на честь В. Іванова в м.Козятині, де народився майбутній кінорежисер.

	pic https://scontent-frx5-1.xx.fbcdn.net/v/t1.6435-9/193857239_851714092084961_8072224705344342781_n.jpg?_nc_cat=100&ccb=1-3&_nc_sid=730e14&_nc_ohc=aZysap7sh4EAX-5ylzc&_nc_ht=scontent-frx5-1.xx&oh=bee88b63cf3343996541ae72ad064014&oe=60DC2968
	caption Меморіальна дошка на честь В.М.Іванова. Встановлена на адміністративному корпусі кіностудії імені Олександра Довженка.

tab_end
\fi


На роль Голохвостого запросили Миколу Гриценка – тодішнього провідного актора
Московського театру ім.Вахтангова. М.Гриценко мав у активі чимало вдалих ролей
негативних персонажів, крім того, народився на Донеччині, вчився в Макіївці і
Києві – тому на думку режисера, розумів «місцеві реалії» і володів українською. 

На роль Проні Прокопівни спочатку затвердили Майю Булгакову – в той час вона
була більш відома не як кіноактриса (ця слава у неї була ще попереду), а як
солістка естрадного оркестру Леоніда Утьосова. Голос у неї був на рівні кумирів
тих років – Гелени Веліканової чи Ніни Дорди. М.Булгакова першою в Союзі стала
виконувати пісні Едіт Піаф, у 1957р. як співачка отримала Срібний приз на
Всесвітньому фестивалі молоді і студентів в Москві. Майя була родом з Київської
області (село Буки), закінчила школу у Краматорську, тож українська мова була
її рідною. (Пробувалась на цю роль і актриса Київського театру російської драми
Віра Предаєвич). 


\ifcmt
  pic https://scontent-frt3-1.xx.fbcdn.net/v/t1.6435-9/193188953_851714182084952_6784782404135835857_n.jpg?_nc_cat=102&ccb=1-3&_nc_sid=730e14&_nc_ohc=EYE-ttbyZg0AX9Zo9_y&_nc_oc=AQnOMD4ApofQSOboffRdHANMAxtSFfe21YDRYEzw-4waFa_fO4DLLBeoQRc3sWSd50M&_nc_ht=scontent-frt3-1.xx&oh=142986ae3e041570140540cd05abd01d&oe=60DA2CA5
	caption Народна артистка України Маргарита Василівна Криницина (1932-2005)

	pic https://scontent-frt3-2.xx.fbcdn.net/v/t1.6435-9/193900706_851714255418278_2158760877360360455_n.jpg?_nc_cat=101&ccb=1-3&_nc_sid=730e14&_nc_ohc=DbgUO7PZtl8AX-fa7TZ&_nc_ht=scontent-frt3-2.xx&oh=cc9f5e6768053fb6c4062a4350da50af&oe=60D8F641

	caption В кадрі фільму видно млин Бродського на Поштовій площі.

	pic https://scontent-frx5-1.xx.fbcdn.net/v/t1.6435-9/193246590_851714345418269_8354702157553275839_n.jpg?_nc_cat=111&ccb=1-3&_nc_sid=730e14&_nc_ohc=3GTg3QU83ygAX8gvIlB&_nc_ht=scontent-frx5-1.xx&oh=48a0cbdee1c8698ee78e35d2761b8517&oe=60DB8BE9

	caption Більша частина «атмосферних» сцен у фільмі зняті на Подолі. Однак цей прохід Голохвостого з приятелями знімали на Батиєвій горі.

	pic https://scontent-frx5-1.xx.fbcdn.net/v/t1.6435-9/193906830_851714435418260_3098667388314989428_n.jpg?_nc_cat=111&ccb=1-3&_nc_sid=730e14&_nc_ohc=70VXg9BZhJoAX86ISvR&_nc_ht=scontent-frx5-1.xx&oh=72a1aff38963627b7a1288493ef56bde&oe=60DBD8EE

	caption Місце дії – вул.Воздвиженська на повороті з Андріївського узвозу. Під час зйомок на Воздвиженській поряд з кіномайданчиком перекладали бруківку. В.Іванов цим скористався і увічнив себе в фільмі в ролі роботяги (в цьому кадрі з молотом – не він😊). Тоді у кадр потрапили і справжні робітники служби благоустрою.

	pic https://scontent-frt3-1.xx.fbcdn.net/v/t1.6435-9/193563796_851714492084921_1974186226744131806_n.jpg?_nc_cat=109&ccb=1-3&_nc_sid=730e14&_nc_ohc=VXE3YnV9gr8AX_D-0Vp&_nc_oc=AQlfZ4SsYqpWA989FOIANHQqBLOdMdXsCPfLB32eUSCmaxATgHOScbuX9AshCeV7Tis&tn=ntrKbsW_7ChXu3v-&_nc_ht=scontent-frt3-1.xx&oh=25d0acb9d4e146bc8f2c3f17a25f02c7&oe=60D88248

	caption Сірки везуть Проню з пансіону вздовж вул.Ігорівської. Всі будиночки у тих кадрах не збереглися, але останній особнячок на розі Ігорівської і Братської вцілів.

	pic https://scontent-frt3-1.xx.fbcdn.net/v/t1.6435-9/193736616_851714568751580_6487800132156798000_n.jpg?_nc_cat=104&ccb=1-3&_nc_sid=730e14&_nc_ohc=PZDLC1FJh9kAX9WEkc9&_nc_ht=scontent-frt3-1.xx&oh=2a9aafc27bf8578254b9ed4131453a12&oe=60DAC383

	caption Кадр біля Кокоревської бесідки на нижній терасі Володимирської гірки. Справжня бесідка не збереглася, на її місці зараз – не відреставрована, а фактично відтворена заново.

	pic https://scontent-frx5-1.xx.fbcdn.net/v/t1.6435-9/193246132_851714655418238_7637552219804805890_n.jpg?_nc_cat=111&ccb=1-3&_nc_sid=730e14&_nc_ohc=F69bVtcwpC4AX_wd3Ho&_nc_ht=scontent-frx5-1.xx&oh=de138d8465167fca687617e66c0318f2&oe=60D96CF6

	caption Голохвостий ховається від Секлети на вул.Гончарній.

	pic https://scontent-frt3-1.xx.fbcdn.net/v/t1.6435-9/194798944_851714732084897_8998312194952700631_n.jpg?_nc_cat=109&ccb=1-3&_nc_sid=730e14&_nc_ohc=TqjugmJSMS8AX__jbdC&_nc_ht=scontent-frt3-1.xx&oh=a6b4bbacf71f5e7f29be775bf557599d&oe=60DA7CF9

	caption Галя біжить сходами з Борічевого току на вул.Андріївську.

	pic https://scontent-frt3-1.xx.fbcdn.net/v/t1.6435-9/194544410_851714845418219_278213155605059984_n.jpg?_nc_cat=102&ccb=1-3&_nc_sid=730e14&_nc_ohc=8WbrCXuJL3oAX9U9s0V&_nc_ht=scontent-frt3-1.xx&oh=4b8317795d044329a62d95ab3b933372&oe=60D86A0F

	caption Голохвостий їде в екіпажі на власне весілля. Ліворуч на розі будинок Сірків - № 46/2 – і Воздвиженська у перспективі.

	pic https://scontent-frt3-1.xx.fbcdn.net/v/t1.6435-9/194343228_851714925418211_7348888900175058863_n.jpg?_nc_cat=108&ccb=1-3&_nc_sid=730e14&_nc_ohc=7rJMGi0fVsAAX8Q8giD&_nc_oc=AQkHximA4aONCaS_zcgkYtUONbaoBZLCYg1Iv7XvW6MOEa0AdGGT-jucBXANh-yrsNc&_nc_ht=scontent-frt3-1.xx&oh=290fc5c2ecf06dab18d51be1a5eb3204&oe=60DA4DE4

	caption В кадрі фільму – вулиця Покровська. Між сучасним ліцеєм № 100 «Поділ» і дзвіницею церкви Миколи Доброго тоді стояв двоповерховий будиночок.

	pic https://scontent-frt3-1.xx.fbcdn.net/v/t1.6435-9/194120237_851715042084866_2598711042493823982_n.jpg?_nc_cat=104&ccb=1-3&_nc_sid=730e14&_nc_ohc=hSmI0Skjr9EAX8HCVQB&_nc_ht=scontent-frt3-1.xx&oh=05ab0d55c4c5dfbe0fb99bf931cc579e&oe=60DA2BEC

	caption Драматичний фінал фільму розігрується на чавунних сходах Андріївської церкви. Вона нічим не відрізняється від сучасної, хиба що лавочки стоять.
\fi


Про те, як замість М.Булгакової на роль Проні Прокопівни потрапила Маргарита
Криницина – існує як мінімум дві версії. Перша – у викладі самої Кринициної:
потрапила випадково, запросили підіграти на пробах акторам Яковченко і
Кушніренко, які зіграли батьків Проні, а режисеру і всій знімальній групі вона
сподобалась, бо у кадрі виглядала органічно і природно. Друга – у спогадах
інших  учасників зйомок, зокрема Таїсії Литвиненко, виконавиці ролі Химки:
режисера змусили взяти на головну жіночу роль Маргариту Криницину. В її активі
тоді були тільки чотири невеличкі ролі у малопомітних фільмах, і жодної
головної. В той час чоловік Кринициної сценарист Євген Онопрієнко був головним
редактором кіностудії і начебто сказав, що якщо його дружину не візьмуть у
фільм, то й самого фільму не буде. 

Як було насправді – зараз не так вже й важливо, бо Криницина у ролі Проні стала
справжнім діамантом фільму. Але сам В.Іванов вважав, що йому цю кандидатуру
нав'язали, і стосунки режисера і виконавиці головної жіночої ролі не склалися.
Казали, що навіть її комічна зовнішність – результат своєрідної помсти
режисера. Щоб зробити Проню негарною, її навмисне спотворили: заклеїли зуб,
засунули в ніс пробки, щоб він став не лише кирпатим, а й опуклим, знебарвили
брови. Чоловік М.Кринициної, переглянувши стрічку, на це дуже розсердився і
більше фільм ніколи не дивився.

Із заміною Майї Булгакової на Маргариту Криницину почали шукати і нового
Голохвостого, бо М.Гриценко видався художній раді старим для цієї ролі (на той
час йому було 48). Серед запрошених на проби був і актор Київського театру
російської драми Олег Борисов. Побачивши його проби на екрані, В.Іванов
зрозумів, що більше нікого шукати не треба. 

Якщо для Олега Борисова фільм приніс всесоюзну славу, то для Маргарити
Кринициної роль Проні стала дещо фатальною. В її акторському доробку понад 70
ролей, але всі пам'ятають лише Проню. Певний час актрису просто відмовлялись
знімати із вбивчим формулюванням: «не відповідає ідеалу жінки»… Після
закінчення зйомок у фільмі «За двома зайцями» М.Криніцина отримала премію – 400
рублів. Розповідала, що дуже зраділа такій сумі і накупила дочці дорогих
німецьких м'яких іграшок…

А ще у фільмі можна почути голос самого режисера В.Іванова – він озвучив
папугу, що повинен був кричати «Химка дура!». Часу вчити птаха не було, і
Віктор Іванов сам вимовив цю фразу з потрібною інтонацією…

Фільм отримав Державну премію імені Олександра Довженка тільки у 1999 році,
коли режисера Віктора Іванова і актора Олега Борисова вже не було в живих…
