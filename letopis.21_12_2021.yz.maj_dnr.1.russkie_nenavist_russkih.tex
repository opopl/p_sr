% vim: keymap=russian-jcukenwin
%%beginhead 
 
%%file 21_12_2021.yz.maj_dnr.1.russkie_nenavist_russkih
%%parent 21_12_2021
 
%%url https://zen.yandex.ru/media/id/5f8f226b1fe36c1d9e02a36b/pochemu-russkie-nenavidiat-russkih-61c22398095ed628a9db06dd
 
%%author_id 
%%date 
 
%%tags 
%%title Почему русские ненавидят русских?
 
%%endhead 
\subsection{Почему русские ненавидят русских?}
\label{sec:21_12_2021.yz.maj_dnr.1.russkie_nenavist_russkih}

\Purl{https://zen.yandex.ru/media/id/5f8f226b1fe36c1d9e02a36b/pochemu-russkie-nenavidiat-russkih-61c22398095ed628a9db06dd}

Мне природой не дано умение ненавидеть, даже врагов. Возможно, по этой причине
я до их пор не разобралась с природой этого чувства, но, получая письма от
некоторых украинских читателей моего телеграм канала, полные ненависти и злобы,
я не могу не задуматься над тем, что с ней делать, как ее истребить? Но
особенно поражает та, которую не скрывают к нам некоторые представители
соседней страны.

\ii{21_12_2021.yz.maj_dnr.1.russkie_nenavist_russkih.pic.1}

Сегодня на заседании гуманитарной подгруппы на Минских переговорах я слушала
Галину Третьякову и вдруг поняла, что ей безумно тяжело находиться по ту
сторону экрана и видеть нашу удивительно красивую и умную Дарью Морозову. Я
вдруг отчетливо поняла природу поведения Третьяковой - она нас ненавидит! До
глубины души, до колик, до срыва. Знаете, мне даже стало как-то не по себе...
Необъяснимое чувство...

Говорят, ненависть разрушает. Но опыт показывает, что строить она тоже умеет,
правда, какое это строительство – из каких кирпичей и на чем замешано – вот в
чем вопрос.

\ii{21_12_2021.yz.maj_dnr.1.russkie_nenavist_russkih.pic.2}

В последние годы я все чаще задаюсь вопросом о дне или хотя бы годе рождения
ненависти, и чем дальше ковыряю исторические источники, тем больше убеждаюсь в
том, что ненависть родилась вместе с человеком. Потому что животным не знакомо
это чувство, они хотят есть, и боятся, что их кто-нибудь съест, вот и вся
недолга. А человеку дано ненавидеть, и кстати, это еще одна причина моего
абсолютного неверия в теорию Дарвина.

Но я закончу со вступлением и спрошу вас: когда вы сами стали замечать, что
русские по рождению способны ненавидеть русских?

Ко мне лично это осознание пришло с наблюдениями за тем как, сперва на Западной
Украине, а потом все восточнее шагал по стране махровый национализм,
насаждаемый со всех сторон и из каждого утюга. И что интересно: многие и многие
считают, что украинский национализм и порожденная им ненависть к русским —
абсолютно закономерное явление. А я с этим категорически не согласна.
Украинский национализм придумали не на территории Российской Империи, в
которую, собственно, входила почти вся современная Украина, его придумали и
выпестовали в зачаточной форме поляки и германцы. Потом он был в коме до 1941
года, находясь на «медикаментозной седации», организованной западными
спецслужбами, и «проснулся» летом 1941 года, а тогда уже проявил себя во всей
красе. После 1945-го снова вынужденный сон, прерванный территориальным
размежеванием Империи, попросту – развалом СССР. А тогда уж украинский
национализм развернулся по полной...

\headTwo{Как русские стали украинскими националистами?}

Но мне совершенно непонятно, почему так быстро, с исторической точки зрения,
галицкий, т.е. западноукраинский национализм, который всегда был направлен
против русских, переместился в русскую среду исторической Новороссии,
Малороссии и других территорий Большой России. Ну правда, я ведь понимаю, что
западноукраинская ненависть к русским объясняется вполне практичными вещами –
ей нужна территория. Но Днепропетровск, Николаев, Запорожье, Киев, в конце
концов, который издревле называли матерью городов русских? Да ведь даже в
Донецке у украинских националистов были свои ячейки, весьма деятельные, между
прочим, и возможно, если бы американцы не поторопились бы так с майданом, а
занимались бы «воспитанием» молодежи подольше, да посерьезнее, кто знает, как
бы дело обернулось? Впрочем, я уже говорила, по-моему: я думаю, те, кто
методично и настойчиво делали из Украины Анти-Россию, с самого начала не делали
ставку на Донбасс и Крым. Но это только одна из моих конспирологических версий.

Но вот тем фактом, что многие сотни тысяч русских очень быстро стали
украинскими националистами, по-моему, должен заинтересовать всех ученых мира, в
первую очередь, России. И не только пропаганда в этом виной, я уверена.

\headTwo{Письмо оттуда}

В Днепропетровске (никак не смирюсь называть этот город «Днепром») у меня жил
одноклассник, и с самого 2014 года часто писал мне о том, что и как там
происходит. Так вот, поначалу он поражался и недоумевал, откуда появились вот
эти вот новоявленные бандеровцы, пристающие к людям на улицах, а в 2018 прислал
мне короткое письмо, которое много объяснило. Публикую его, теперь не боясь за
моего друга. Потому, что Лешку в 2020-м унес коронавирус а его семья переехала
в Россию и им уже не страшны ни представители Правого Сектора, и СБУ-шники, ни
кто другой. Вот это письмо:

\begin{zzquote}
\bfseries\large\color{magenta}
«Май, боюсь, в России прекрасно понимают ситуацию и ее смог ы исправить только
глобальный катарсис, иначе никак. Еще три года назад наш сосед ненавидел
Филатова за его «Обещайте им все, что угодно, а вешать мы их будем потом», а
сегодня он шпрехает на мове и в выходные ходит с друзьями в черно-красных
повязках «отлавливать сепаров». И повсюду так. У нас уже, почитай, друзей не
осталось, к которым можно зарулить в гости без страха, что тебя завтра сдадут
куда надо. Так что, думаю, русские просто отгородятся от нас той самой стеной
Сени (Яценюка, прим. мое) и забудут нас лет эдак на сто. Если, конечно, нашему
фюреру не придет в голову вдарить не по Донбассу, а сразу по России. Но это
вряд ли....»	
\end{zzquote}

Вот и скажите мне, откуда? А я вам еще добавлю горяченького: наибольшее число
бойцов в добровольческие националистические батальоны дала Днепропетровская
область. За ней Западная Украина, хотя она безнадежно отстала, и т сказать, ее
представители больше мародерили, чем геройствовали. И вообще, территории
исторической Новороссии и Малороссии, накачанные невероятной пропагандой,
изменились очень сильно, и явно не в сторону признания себя русским народом, а
как раз наоборот.

О повальности не говорю – это было бы неправдой, и мне крайне сложно
представить себе, каково живется тем, кто до сих пор деятельно не верит
украинским «учебным пособиям по русофобии».

\headTwo{Чуток про олигархов}

Однако, есть еще одно «но». Прежде, чем случился майдан, Украину раздерибанили
на части олигархи. А для них национальный вопрос – это вопрос власти – русский
по духу человек свободолюбив и горд, оттого не позволит обращаться с собой, как
с рабом. А значит, национализм под соусом независимости Украины – это их
главная тема. Только в такой ситуации их мир не рухнет. Собственно, потому-то
украинские олигархи, все, до единого и поддержали майдан. Потому и содержат
каждый по националистической армии, называемой «Азов», «Днепр», «Айдар», «Сич»,
«Донбасс» и так далее. И к мирняку методы у них сплошь Филатовские, те самые
«вешать потом будем». Ну да, плюс СМИ! Именно их СМИ обеспечивают ту самую
информационную пищу для роста ненависти к русским.

\headTwo{Почему об том так мало говорят?}

Об этом действительно мало говорят и пишут. Мало того, этим вопросом почти не
занимаются ученые, разве что последние несколько лет. Или национализм – такая
скользкая и опасная тема? Нет, особенно украинский. Но наши отечественные
историки, в том числе и российские, до недавнего времени уделяли этому вопросу
катастрофически мало внимания. А напрасно. Если бы нынешним 30-40 летним
рассказывали на урока истории о том, что творили бандеровцы во время Великой
Отечественной Войны, хотя бы в том дико урезанном варианте, в котором этим
событиям было уделено пару часов в советские времена, уверена, они бы не стали
рукоплескать героизации Бандеры, Шухевича и им подобных сегодня. А ведь имен за
незнанием истории тянется ее вольное изменение в интересах определенных кругов,
за которыми начинается изменение сознания. И именно из переделанной истории,
как из пены, рождается национализм и ненависть., которые прокладывают
непреодолимую пропасть противоречий между частями одного и того же народа.
Сколько русских сегодня воюет под черно-красным флагом и знаком трезубца –
символа ОУН? Спросите их, кто они и как они относятся к русским и к России.
Сомневаюсь, что вам понравится их ответ. А ведь они русские!

\headTwo{Почему я об этом думаю}

Потому что точно знаю – еще не поздно. Еще не достигнут максимум в уровне
ненависти жителей Украины к России и ко всему русскому. Потому что еще не все
на Украине готовы бросаться словами «ватник» и «колорад», потому что каждое
воскресенье к Одесскому Дому Профсоюзов приходят люди с цветами почтить память
погибших 2 сентября 2014 года за Русский мир. Потому что большинство украинцев
понимают, что ни русские и, даже не смотря на пропаганду, не верят своей
власти, клокочущей ненавистью и обвиняющей Россию во всех проблемах Украины.
Потому что пока еще слишком много осталось на Украине русских. Н это не будет
длиться вечно, это даже не будет длиться долго – подросло поколение,
воспитанное на ценностях ненависти. Следующее просто не будет иметь никаких
других.

Потому что пришло время задуматься, как нам справиться с этой ненавистью. Пока
еще не поздно...
