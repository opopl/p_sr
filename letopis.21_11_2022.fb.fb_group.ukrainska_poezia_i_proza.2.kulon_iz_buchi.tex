% vim: keymap=russian-jcukenwin
%%beginhead 
 
%%file 21_11_2022.fb.fb_group.ukrainska_poezia_i_proza.2.kulon_iz_buchi
%%parent 21_11_2022
 
%%url https://www.facebook.com/groups/255425469158429/posts/724013995632905
 
%%author_id fb_group.ukrainska_poezia_i_proza,dashkivska_iryna
%%date 
 
%%tags 
%%title Кулон із Бучі (новела)
 
%%endhead 
 
\subsection{Кулон із Бучі (новела)}
\label{sec:21_11_2022.fb.fb_group.ukrainska_poezia_i_proza.2.kulon_iz_buchi}
 
\Purl{https://www.facebook.com/groups/255425469158429/posts/724013995632905}
\ifcmt
 author_begin
   author_id fb_group.ukrainska_poezia_i_proza,dashkivska_iryna
 author_end
\fi

\obeycr
Кулон із Бучі
(новела)
Зранку Інна поспіхом згребла Ванькині речі та жбурнула за поріг. Духом його руським щоб тут і не тхнуло! Розбіжності у них почалися ще за Майдану, коли віджали Крим - стали геть чужі. А там, біля моря, колись все й спалахнуло. Звідти привезла в утробі їхню Ілону. Нині спогади анексовано разом з місцями, де все сталося. Жінці боліла у грудях Україна, а він ледь долоні від захоплення не стер:
- при Украине там порядка не было, наши наведут.
Тригером стала витівка в ювелірці.
Коли Інна саме народила Ілону, Іван, намірений женитися, приїхав до неї зі своєї Москви. Замість обручки блиснув ланцюжком із білого золота. Добротним, міцним, сріблястим. Хлопця прийняла, а дарунок не носила, все хтіла дібрати кулон, і гроші відкладала, та завше щось важніше вистрибувало наперед, відсовуючи той кулон на потім. Аж тут Інні запала підвіска у формі контурів України. І її Івана прорвало:
- Украины скоро вообще не будет. Возьми лучше «Ловца снов», красиво и по феншую.
Як у голову врізалось – не місце йому ні в її країні, ні в житті.
Двічі ще приїздив до доньки, далі став боятися бандерівців. Та й дзвонив радше для галочки, на свята.
Коли окупанти зайняли Бучу, Ілона сама набрала батька:
- Тату, рашисти нас вбивають, Кіриного діда розстріляли прямо на вулиці.
- Солнышко, не бойся, наша армия хочет вам помочь! Спасти! По мирным бить не будут.
- Та вони вже півміста роздовбали, тіла скрізь.
- Это страшилки. Скажи, что папа твой русский, тебя и пальцем не тронут!
- Кому сказати? Танку, Граду? - обірвала дівчинка.
Згодом Івана звістили, що дружину з донькою відкопали з братської могили. Обхідними стежками їхав у Бучу, що стріла його могильним холодом. Суцільна труна!
За ним до веранди ввійшла баба Валя, сусідка з іншої частини будинку. Розказувала все, що бачила, чула і до віку не забуде: як набіг сюди цей збрід, дівчата голосили, а мала все кричала «пустіть».
- Глянула у вікно - краще б не дивилась!. Інку прив'язали, а поряд, на підлозі … тварюки над Ілоною… Виродки! Ґвалтівники!
- Она ведь еще ребенок, - тільки й видушив із себе.
- Одурілі, бігали курити, іржали, смакували те звірство, спорили, чия далі черга, а щоб воно у вас навіки всохло, - продовжувала стара.
- Не верю! Их убили бандеровци, потому что я русский!
- Дурак, ти Ванька! А Сашко, онук мій ще до війни поставив у дворі камеру. Все розбили, а та і досі пише.
Глянув і прозрів. Бачив те, що було за межею сприйняття. Як вояки у формі обожнюваної ним армії несли два згортки, під ковдрою впізнав Ілонине волосся. Жбурнули, як непотріб в Урал із Zетками, хтось нив, що «малая быстро откинулась, могли еще порезвиться». Чув здичавілий регіт стада.
- Йди ліпше, ще взнають… аби хто гріха на душу не взяв, - гукнула з двору сусідка.
Крізь сльозу щось блиснуло в кутку за диваном. Простяг руку - Інкин триклятий ланцюжок, обірваний, із запеченими слідами крові…і кулон мигнув на сонці всіма своїми кордонами. Цілими, непокроєними! З особливою ніжністю поклав у кишеню сорочки і так запекла йому біля серця та нелюба Україна.
- ПЄду і сАбі вбивати рускІх, бо вАни нелюди, - сказав ломаною українською.
\restorecr

\ii{21_11_2022.fb.fb_group.ukrainska_poezia_i_proza.2.kulon_iz_buchi.orig}
\ii{21_11_2022.fb.fb_group.ukrainska_poezia_i_proza.2.kulon_iz_buchi.cmtx}
