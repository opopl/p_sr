% vim: keymap=russian-jcukenwin
%%beginhead 
 
%%file narod
%%parent _main_
 
%%url 
 
%%author 
%%author_id 
%%author_url 
 
%%tags 
%%title 
 
%%endhead 

https://33kanal.com/news/63759.html

А я з болем і жалем констатую: «Із гнилим запашком, з глибокою червоточиною».
Як не прикро це визнавати, а переглядаючи нашу історію, споглядаючи сьогодення,
саме такої думки доходжу. Так, є серед українців лицарі духу й свободи, для
яких Україна – понад усе, які готові віддати за неї все: добробут, спокій,
статки, навіть життя. Це Марк Паславський, котрий покинув ситу Америку, високу
армійську посаду, приїхав на землю своїх батьків, прийняв українське
громадянство й загинув під Іловайськом. Це всесвітньо оперний співак Василь
Сліпак, у якого була і всесвітня слава, і гроші, і Париж біля його ніг. А він
пішов у окопи заради України і теж поліг. Це тисячі інших хлопців і дівчат,
котрі щохвилини ризикують життям на фронті, не питаючи, що їм дала Україна.
Таких у нас справді багато. Але, на жаль, не більшість.

Огляньтеся довкола, скільки серед нас скигліїв. Все канючать щось у держави:
«Дайте! Дайте!», а зробити для неї, насамперед для творення своєї долі, свого
нормального життя, діяти не хочуть. Тільки винять тих, хто щось прагне зробити
для прогресу, для покращення. Ганять на всі заставки й кричать: «Не треба нам
нічого нового, верніть те, що було». Ось один з таких ходить нашим селом і
бубонить: «Ну коли вже той Путін прийде». Ти хоч уявляєш, що мелеш? Невже всі
клепки в голові порозсихалися?» Невже не чуєш, не бачиш, що твориться після
«Путін, пріді!» в Криму, на Донбасі. Ти сьогодні на всю губу картаєш Україну,
її президента. І, думаєш, й при Путіну так говоритимеш без наслідків. «Будемо
смирні – все буде добре», – відповідає. Тобто, перетворимося на червяків,
слизняків? Тільки в такому стані Путін визнає нас смирними. І де гарантія, що
він гидливо не розчавить тебе як того слизняка? Он навіть проросійські «до
мозга костей» кримчани й севастопольці починають стогнати від путінського
режиму, все частіше запитують себе: «Кого прикликали?».

А тут ще й деякі попи, дують в ту ж дудку. Поотримували сан за десяток кабанів
(знаю, що кажу) і розказують нам про «русскій мір», про один народ, про єдину
віру, прославляють Кіріла, який зі своєю братією, благословляв бойовиків разом
із російськими найманцями «очищать Украину от хунты и бандеровцев», закликав
українців не чинити опору окупантам. А ми, як барани, сліпо слідуємо за ними. І
хулимо, на чім світ стоїть, тих, хто хоче свій, український шлях знайти до
Бога.

А як вибираємо депутатів і президентів? Одній жіночці, сама казала, приглянувся
Янукович, бо високий, гарний, ставний. Інші голосували за пишний чуб, за гарну
усмішку. Ви чим думаєте, кого обираєте, керівника країни, проводиря нації чи
коханця на ніч?

Зовсім скоро вибори до Верховної Ради. І бачу, як на перший погляд
інтелігентні, розумні, освічені люди переконують голосувати за Бойка –
Медведчука – Рабіновича, мовляв, вони і мир, і багатство, і благоденствіє
принесуть українцям, на чім світ клянуть Порошенка за війну. Так, до колишнього
президента і в мене багато претензій, багато чого зробив не так, як треба було.
Але звинувачувати його в розв’язанні війни – глузд зовсім втратити. Це ж ті,
хто ратує сьогодні за мир, а точніше за капітуляцію перед Росією, самі й
посприяли виникненню і сепаратизму на Донбасі, і війни. А тепер все звалюють з
хворої голови на здорову. А ви їм плещете й підтакуєте. Покопирсайтеся у своїх
мізках і згадайте, як все починалося. Вибори покажуть, наскільки ми розумні й
далекоглядні, хто ми є: нащадки славних козаків чи «раби, подножки, грязь
Москви», готові стати «розхідним матеріалом» для Путіна.
