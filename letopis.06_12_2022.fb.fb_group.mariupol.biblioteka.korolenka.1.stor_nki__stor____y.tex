%%beginhead 
 
%%file 06_12_2022.fb.fb_group.mariupol.biblioteka.korolenka.1.stor_nki__stor____y
%%parent 06_12_2022
 
%%url https://www.facebook.com/groups/1476321979131170/posts/5617182125045114
 
%%author_id fb_group.mariupol.biblioteka.korolenka,lisogor_viktoria.mariupol
%%date 06_12_2022
 
%%tags mariupol,istoria,mariupol.istoria
%%title Сторінки історії - Як розвивалося місцеве самоврядування Маріуполя
 
%%endhead 

\subsection{Сторінки історії - Як розвивалося місцеве самоврядування Маріуполя}
\label{sec:06_12_2022.fb.fb_group.mariupol.biblioteka.korolenka.1.stor_nki__stor____y}
 
\Purl{https://www.facebook.com/groups/1476321979131170/posts/5617182125045114}
\ifcmt
 author_begin
   author_id fb_group.mariupol.biblioteka.korolenka,lisogor_viktoria.mariupol
 author_end
\fi

\textbf{Сторінки історії}

\textbf{\emph{ЯК РОЗВИВАЛОСЯ МІСЦЕВЕ САМОВРЯДУВАННЯ МАРІУПОЛЯ}}

Першим органом місцевого самоврядування в Маріуполі був Маріупольський грецький
суд, створений 1780 року для греків, що переселилися з Криму. Він виконував
адміністративні, поліцейські та судові функції. Суд складався з голови та
чотирьох засідателів. Усі обрані особи затверджувалися губернатором.

Канцелярія грецького суду спочатку складалася з секретаря, 2 канцеляристів, 2
підканцеляристів, 2 копіїсти, перекладача з турецької мови (толмача), 2
розсильних та 2 сторожів. Діяльністю канцелярії керував секретар. Весь штат
призначався губернським (пізніше – намісницьким) правлінням.

Вже після міської реформи 1785 року у Маріуполі було створено органи міського
самоврядування: 1790 року почала діяти міська дума, очолювана міським головою.
Цікавий факт: за місцевою традицією до її складу входили виключно греки.

Маріупольська міська дума складалася із розпорядчої «спільної думи» та
виконавчої «шестигласної думи». Незвичайна назва останньої була зумовлена тим,
що вона складалася із шести голосних (депутатів) – по одному від кожного стану
міського населення.

Засідання шестиголосної думи проходили раз на тиждень. Виглядало це, як
регламентував закон, так: «У міській думі сидить міський голова на стільці
посередині; навпроти міського голови сидять: на лаві праворуч – голос цехових,
ліворуч – голос посадських; біля міського голови у правому завороті на лавці –
голос справжніх містових обивателів та голос іногородніх та іноземних гостей;
біля міського голови в лівому завороті на лавці ж голос іменитих громадян і
голос гільдійський».

У веденні думи перебувало міське господарство, благоустрій, продовольчу справу,
розвиток торгівлі, промисловості.

\textbf{\emph{Історія місцевого самоврядування Маріуполя у XIX столітті}}

Згідно з історичною довідкою в 1870 було введено нове Містеве Положення.

Цей документ визначав функції та форму міського управління. У Маріуполі органи
місцевого самоврядування за новим законом були сформовані навесні 1872 року.
Виборчі права тоді отримали 1064 маріупольці, що становило 13,9\% населення
міста. Проте скористалися ними лише 180 осіб. З них було обрано 60 голосних
міської думи та три члени міської управи. Мундир міського голови нового зразка
першим одягнув Олександр Трандафілов. Діяльність членів управи та голови
оплачувалася. Платня міського голови, наприклад, становило 3000 рублів.

Міська дума була розпорядчим органом. Наділена певними правами, вона
представляла інтереси всього міського суспільства, і мала своє коло обов'язків.
На засіданнях голосні розглядали питання щодо суспільного життя. Вони
встановлювали розмір міських зборів, вирішували питання про позики та інші
фінансові операції, керували муніципальним майном, розглядали та затверджували
кошториси, звіти. Безпосереднє завідування справами покладалося на членів
міської управи, дві з яких переобиралися кожні два роки. «Управа веде поточні
відносини з міському господарству, вишукує заходи для її поліпшення, виконує
визначення думи, збирає необхідні їй відомості, складає проекти міських
кошторисів, стягує і витрачає міські збори, ...представляє звіти про діяльність і
про стан підвідомчих їй частин...», - гласило Городове становище, висуваючи на
чільне місце економічні завдання.

Чергова реформа місцевого самоврядування була проголошена 11 червня 1892 після
підписання Олександром III нового Міського Положення. Зміни переважно
торкнулися системи формування органів управління. За новим Положенням обирати і
бути обраним до міської думи могли лише маріупольці, нерухомість яких оцінили
щонайменше ніж 300 крб., купці I - ой і II - ой гільдій і власники підприємств
перших п'яти разрядов. Жінки, а також молоді люди, які не досягли 25 років або
перебувають під опікою, могли висловити свою думку через довірених осіб. А от
городяни єврейської національності вносилися до окремого списку. У тому числі
посадовими особами призначалися голосні у кількості трохи більше однієї десятої
частини загального складу думи.

В серпні1911 року відбулися чергові і, як виявилося, останні вибори у міську
думу за Положенням 1892 року. Вибрані голосні замість чотирьох років працювали
майже два повні терміни. Причиною цього була перша світова та громадянська
війна.

\url{https://mariupolrada.gov.ua}
