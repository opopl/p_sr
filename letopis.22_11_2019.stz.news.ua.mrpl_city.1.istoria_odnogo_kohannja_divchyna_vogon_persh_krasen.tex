% vim: keymap=russian-jcukenwin
%%beginhead 
 
%%file 22_11_2019.stz.news.ua.mrpl_city.1.istoria_odnogo_kohannja_divchyna_vogon_persh_krasen
%%parent 22_11_2019
 
%%url https://mrpl.city/blogs/view/istoriya-odnogo-kohannya-divchina-vogon-ta-pershij-krasen
 
%%author_id demidko_olga.mariupol,news.ua.mrpl_city
%%date 
 
%%tags 
%%title Історія одного кохання: дівчина-вогонь та перший красень
 
%%endhead 
 
\subsection{Історія одного кохання: дівчина-вогонь та перший красень}
\label{sec:22_11_2019.stz.news.ua.mrpl_city.1.istoria_odnogo_kohannja_divchyna_vogon_persh_krasen}
 
\Purl{https://mrpl.city/blogs/view/istoriya-odnogo-kohannya-divchina-vogon-ta-pershij-krasen}
\ifcmt
 author_begin
   author_id demidko_olga.mariupol,news.ua.mrpl_city
 author_end
\fi

\begingroup
\color{blue}
Восени хочеться більше проводити час вдома, поруч з коханими і близькими
людьми. Насолоджуватися затишком вдома, коли на вулиці холодно, грати в слова
чи монополію всією сім'єю, чи просто дивитися разом фільм - що може бути
кращим?! Проте останнім часом все частіше помічаю, що сьогодні міцна і велика
родина стає унікальним і далеко нечастим явищем. Жінки займаються кар'єрою, а
чоловіки не поспішають створювати сім'ю. Хоча інколи, вже одружившись, молоді
люди швидко розлучаються, вважаючи, що якщо не виходить, то краще одразу все
закінчити. Саме це явище надихнуло мене на створення нової серії нарисів,
присвячених маріупольцям, що вміли кохати по-справжньому, цінувати своїх
близьких і створювати міцні і щасливі сім'ї. Такі історії можуть стати
прикладом для наслідування і нагадати, що за своє щастя, не зважаючи на всі
труднощі та життєві негаразди, варто завжди боротися.
\endgroup

\ii{22_11_2019.stz.news.ua.mrpl_city.1.istoria_odnogo_kohannja_divchyna_vogon_persh_krasen.pic.1}

Героїня мого першого нарису – \textbf{Єрьоменко Антоніна Петрівна} (1937 року
народження) – змалечку знала, що обов'язково буде щасливою. Їй довелося багато
пережити, але неабияка сила волі, природжена загартованість та надзвичайна
кмітливість завжди допомагали досягати поставлених цілей і мати все, що
потрібно для справжнього щастя. Хоча маленька Тося пережила і роки окупації, і
смерть брата під час Другої світової війни, і важкі післявоєнні роки, та вона
не втратила свого оптимізму і віри у світле майбутнє. Юна Антоніна встигла
попрацювати дояркою у селі Стародубівка, де вона і жила разом з батьками та
двома сестрами. В обов'язки доярок входило годування, прибирання, догляд за
тваринами. Одна доярка обслуговувала 15 корів.

\ii{22_11_2019.stz.news.ua.mrpl_city.1.istoria_odnogo_kohannja_divchyna_vogon_persh_krasen.pic.2}

У 1957 році, помітивши здібну працівницю, Антоніну обрали депутатом до
сільської ради. А у 1958 році їй запропонували посаду секретаря сільської ради.
Вона погодилася і дуже швидко вивчила свої обов'язки, до яких входило збирати
податки, реєструвати шлюби, новонароджених, працювати з документами. Посада
секретаря була досить відповідальною і передбачала роботу з місцевим
населенням, що приносило повагу і авторитет жителів села.

\ii{insert.read_also.burov.portnihi}

Антоніна жила в часи, коли було складно придбати нові речі і хизуватися гарними
сукнями, але дівчина вирішила самотужки впоратися з цією проблемою, тому
закінчила курси по крою шиття, після чого більшість одягу собі шила сама.

\ii{22_11_2019.stz.news.ua.mrpl_city.1.istoria_odnogo_kohannja_divchyna_vogon_persh_krasen.pic.3}

Працюючи секретарем, вона вступила на заочне відділення математичного
факультету Бердянського педагогічного інституту, адже мріяла стати вчителем.
Отримавши професію вчителя математики, Антоніна Петрівна одразу починає
користуватися ще більшою повагою серед односельчан, які встигли полюбити Тосю
за її працелюбність, відповідальність, доброту і унікальне почуття гумору. І
дійсно - дівчина полюбляла жартувати, чим дивувала всіх подружок, які
здебільшого переживали про свою дівочу долю, адже раніше виходити заміж
потрібно було й заради свого статусу. А от чарівна Тося завжди вважала, що все
у неї буде, просто слід зачекати.

\ii{22_11_2019.stz.news.ua.mrpl_city.1.istoria_odnogo_kohannja_divchyna_vogon_persh_krasen.pic.4}

Свята в селах на той час відзначали весело. Часто збиралися у когось вдома, де
був найбільший двір. У дворі ставили дерев'яні столи і лавки. Кожен, хто
приходив, приносив щось з собою. Гармошки, баян, застільні пісні і танці
створювали незабутню атмосферу. На одному з таких свят Антоніна вирішила, що
обов'язково зустріне свою долю. Так і сталося. Подружки розповіли, що на свято
завітає багато хлопців, які шукають собі пару. Але одного красеня – \textbf{столяра
Бориса} – вже ділили між собою всі знайомі Тосі, адже такого стрункого,
ввічливого і вродливого хлопця в селі не бачили давно. Тоня вирішила подивитися
на нього сама, а потім вирішити, чи варто за нього боротися. Але, коли
побачила, довелося погодитися, він і дійсно був надзвичайно гарним.

\ii{22_11_2019.stz.news.ua.mrpl_city.1.istoria_odnogo_kohannja_divchyna_vogon_persh_krasen.pic.5}

Смілива і рішуча Антоніна вирішила не гаяти час. Під час танців запросила його
і спитала, чи хоче він відправитися з нею у найбільш незабутню подорож в його
житті. Хлопець був здивований таким питанням і рішучістю дівчини, але не
розгубившись, відповів, що спочатку хоче дізнатися, про яку ж таку подорож
йдеться. Тося одразу додала: \emph{\enquote{У подорож, що зветься життя}}. Тоді Борис
Іванович, що завітав у Стародубівку з Білосарайки, не усвідомлював, що ця
зухвала дівчина середнього зросту, з русявим волоссям і з неймовірним вогником
в очах, якого він раніше ні в кого не бачив, заволодіє його серцем назавжди.
Він закохався в неї з першого погляду. А от Тося не думала, що зможе так швидко
закохати в себе першого красеня, на якого дівчата оголосили справжнє полювання.

\ii{22_11_2019.stz.news.ua.mrpl_city.1.istoria_odnogo_kohannja_divchyna_vogon_persh_krasen.pic.6}

Після того свята Борис з друзями став частіше приходити до Стародубівки,
бажаючи знайти Антоніну. Він отримував від дівчат села безліч подарунків,
запрошень, але його нічого не цікавило. Натомість Тося вражала своїми знаннями,
начитаністю і кмітливістю. Так, вона запропонувала Борі пограти з нею. Умови
гри були простими. Треба було писати по одному слову, яке характеризувало їхнє
ставлення один до одного і шукати листівки десь в селі: в магазині, на вулицях,
де завгодно. Але знайомі могли підказувати і допомагати знайти сховані
листівки, які долітали з Білосарайки до Стародубівки і навпаки. Знаходячи ці
маленькі аркуші білого паперу зі словами: \emph{\enquote{сумую}, \enquote{натхнення},
\enquote{весна}, \enquote{щастя}, \enquote{сонечко}, \enquote{єдина},
\enquote{пишаюся}, \enquote{справжній}, \enquote{сильний}, \enquote{мужній},
\enquote{майбутнє}} тощо, вони і не помітили, як створили навколо себе
неймовірно романтичну атмосферу. Потім, під час зустрічі, вони пояснювали
значення цих слів. Для Бориса Антоніна була справжньою весною, він називав її
сонечком і пишався її досягненнями. А для Антоніни Борис став натхненням, її
сильним і мужнім чоловіком, з яким вона хотіла розділити майбутнє життя. Ці
листівки мали продовження протягом наступних 30 років. Саме завдяки цій грі
Борис зізнався дівчині в коханні і запропонував їй стати дружиною.

\textbf{Читайте також:} \emph{Проверено временем: мариупольские пары повторно женятся в годовщины своих свадеб}%
\footnote{Проверено временем: мариупольские пары повторно женятся в годовщины своих свадеб, Анастасія Папуш, mrpl.city, 14.11.2019, \par%
\url{https://mrpl.city/news/view/provereno-vremenem-mariupolskie-pary-povtorno-zhenyatsya-v-godovshhiny-svoih-svadeb-foto}
}

\ii{22_11_2019.stz.news.ua.mrpl_city.1.istoria_odnogo_kohannja_divchyna_vogon_persh_krasen.pic.7}

У 1960 році вони одружилися. Антоніна переїхала до Білосарайки, де почала
працювати вчителем математики у Білосарайській восьмирічній школі. Антоніна
Петрівна вела господарство, розподіляла фінанси, шукала додаткові заробітки.
Вміло поєднувала як берегиню вогнища, лагідну матір, віддану дружину, так і
інтелігентну жінку, яка знайшла своє покликання в житті... Борис Іванович їй
допомагав у всьому. Коли навантажена зошитами дружина приходила втомлена
додому, на неї чекала смачна вечеря і щасливий чоловік, який швидко ховав серед
зошитів маленький аркуш паперу з новим словом, що обов'язково підніме настрій
коханій жінці. У них народилися дві гарні донечки, які обожнювали разом з
батьками насолоджуватися домашнім затишком і точно знали - якщо хтось з них
залишиться вдома через хворобу, сумно точно не буде, адже батьки завжди вміли
розважити своїх дітей. Вони стали міцною і щасливою сім'єю. Коли у 1971 році
переїхали до Маріуполя, походи в кіно, театр, музей стали частим явищем.
Антоніна разом зі своїми класами часто подорожувала і з собою брала чоловіка та
дочок. Учні Антоніни Петрівни теж полюбили Бориса Івановича, який завжди
жартував з ними і розповідав повчальні історії. Діти називали свою вчительку
другою мамою, а її чоловіка - другим татом, що може свідчити про довірливі та
неймовірно теплі стосунки.

\ii{22_11_2019.stz.news.ua.mrpl_city.1.istoria_odnogo_kohannja_divchyna_vogon_persh_krasen.pic.8}

Антоніна і Борис  бралися за руки за кожної можливості. Вони потайки
цілувалися, коли стикалися один з одним в їхній крихітній кухні. Часто
закінчували фрази один одного, щодня разом відгадували кросворди і ребуси.
Перед кожним прийомом їжі вони схиляли свої голови і дякували Богу,
захоплюючись тим, що Він дав їм: чудову сім'ю, прекрасну долю і один одного.

\ii{22_11_2019.stz.news.ua.mrpl_city.1.istoria_odnogo_kohannja_divchyna_vogon_persh_krasen.pic.9}

Звісно, і у них були свої проблеми. Інколи не вистачало коштів, а інколи були
непорозуміння, але довго сваритися вони не вміли. Примирення проходило дуже
цікаво. Дівчата дуже любили, коли всі речі в домі ставали місцем для схованок.
Адже їхні батьки вибачалися чарівним чином. Не було кінця місцям, де могли
раптом з'явитися їхні листівки. Вони з'являлися в ящику з інструментами, у
взутті, під подушками. Ці загадкові слова, завдяки яким з'являлася посмішка на
обличчях Антоніни і Бориса, були невід'ємною частиною обстановки в їхньому
будинку.

Разом вони прожили рівно 30 років (до смерті Бориса Івановича). Але і тепер у
доньок Тоні і Борі можна знайти безліч маленьких аркушів з різними словами,
кожне з яких є маленькою частинкою великого і сильного почуття, яке їм вдалося
створити і, найголовніше, зберегти...

\textbf{Читайте також:} \emph{История любви, подарившая мариупольцам Париж}%
\footnote{История любви, подарившая мариупольцам Париж, Анастасія Селітріннікова, mrpl.city, 08.07.2019, \par%
\url{https://mrpl.city/news/view/istoriya-lyubvi-podarivshaya-mariupoltsam-parizh-foto}
}
