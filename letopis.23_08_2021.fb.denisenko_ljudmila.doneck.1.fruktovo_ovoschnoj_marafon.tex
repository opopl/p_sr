% vim: keymap=russian-jcukenwin
%%beginhead 
 
%%file 23_08_2021.fb.denisenko_ljudmila.doneck.1.fruktovo_ovoschnoj_marafon
%%parent 23_08_2021
 
%%url https://www.facebook.com/LED65/posts/4533385250045434
 
%%author Денисенко, Людмила (Донецк)
%%author_id denisenko_ljudmila.doneck
%%author_url 
 
%%tags dnr,donbass,doneck,eda,kuhnja
%%title Мои дорогие друзья и читатели! Продолжаем наш летний фруктово-овощной марафон
 
%%endhead 
 
\subsection{Мои дорогие друзья и читатели! Продолжаем наш летний фруктово-овощной марафон}
\label{sec:23_08_2021.fb.denisenko_ljudmila.doneck.1.fruktovo_ovoschnoj_marafon}
 
\Purl{https://www.facebook.com/LED65/posts/4533385250045434}
\ifcmt
 author_begin
   author_id denisenko_ljudmila.doneck
 author_end
\fi

Доброе утро, мои дорогие друзья и читатели!

Продолжаем наш летний фруктово-овощной марафон. И сегодня у нас краснокачанная
капуста. Именно эту капусту в детстве я считала «цветной». И в самом деле,
обычная капуста – белая, а эта – фиолетовая! И странно, что она
«краснокочанная», даже руки после её нарезки становятся не красными, а
фиолетовыми. А причиной тому – антоцианы - природные антиоксиданты,
содержащиеся в больших количествах в овощах и фруктах, имеющих темную окраску
(от буро-красной до темно-фиолетовой). 


\ifcmt
  tab_begin cols=2

     pic https://scontent-cdt1-1.xx.fbcdn.net/v/t39.30808-6/240401093_4533385046712121_7618711676577710432_n.jpg?_nc_cat=106&_nc_rgb565=1&ccb=1-5&_nc_sid=730e14&_nc_ohc=R2OMz-v9t9oAX8f1En8&_nc_oc=AQn8sBz_R51IUn-Pznm3jZTcnLcwEOvB3hAr8COJdCaJJLZU2PYp40fRfeSRQfkegTg&_nc_ht=scontent-cdt1-1.xx&oh=a182b3e6da099bf70726fbe416206e02&oe=61301307

     pic https://scontent-cdg2-1.xx.fbcdn.net/v/t39.30808-6/240169098_4533385066712119_5985685861779718845_n.jpg?_nc_cat=104&_nc_rgb565=1&ccb=1-5&_nc_sid=730e14&_nc_ohc=C7Y3nKkghJsAX_S5ADt&tn=lCYVFeHcTIAFcAzi&_nc_ht=scontent-cdg2-1.xx&oh=a225e29f30b71c1afe761fe05b6998f4&oe=6130A88A

  tab_end
\fi


Типичные примеры продуктов, содержащих
большое количество антоцианов: черника, черная смородина, черная малина, кора
баклажанов и краснокочанная капуста. Оттенок листьев краснокочанной капусты
зависит от сорта и от кислотности почвы, на которой она выращивается. Так,
капуста, растущая на щелочных почвах, синеет, а на кислых – приобретает
красноватые тона. По-видимому, мне всегда попадался сорт с щелочной почвы!


\ifcmt
  tab_begin cols=2

     pic https://scontent-cdt1-1.xx.fbcdn.net/v/t39.30808-6/240219744_4533385193378773_1057866238446918378_n.jpg?_nc_cat=103&_nc_rgb565=1&ccb=1-5&_nc_sid=730e14&_nc_ohc=7tnjNl1rYTYAX8MCjT7&_nc_ht=scontent-cdt1-1.xx&oh=e1f871bed09b78ab10625a5e98cd05c6&oe=612F8D42

		 pic https://scontent-cdg2-1.xx.fbcdn.net/v/t39.30808-6/240158703_4533385026712123_5797372823898910651_n.jpg?_nc_cat=100&_nc_rgb565=1&ccb=1-5&_nc_sid=730e14&_nc_ohc=jzLi-1iQUJEAX807Ulz&tn=lCYVFeHcTIAFcAzi&_nc_ht=scontent-cdg2-1.xx&oh=c4fbbf16b0021721f48d840311378492&oe=612F60B0

  tab_end
\fi


Итак, сегодня -  Краснокочанная капуста против рака.

И да, дорогие мои читатели, реклама на сайте не для красоты - это ваша реальная
помощь нуждающимся людям и животным военного Донецка, потому, потратьте пару
минут - ознакомьтесь с заинтересовавшей картинкой! Спасибо!
