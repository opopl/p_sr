% vim: keymap=russian-jcukenwin
%%beginhead 
 
%%file 08_09_2021.fb.arestovich_alexei.1.pereimenovyvanie_rusj_ukraina
%%parent 08_09_2021
 
%%url https://www.facebook.com/alexey.arestovich/posts/4666750366722340
 
%%author_id arestovich_alexei
%%date 
 
%%tags future,gosudarstvo,ideologia,istoria,obschestvo,rusj,rusj.ukraina,strana,ukraina
%%title Для справки: мы переименовывались множество раз, только в двадцатом веке: Гетьманщина, УНР, УССР, Украина
 
%%endhead 
 
\subsection{Для справки: мы переименовывались множество раз, только в двадцатом веке: Гетьманщина, УНР, УССР, Украина}
\label{sec:08_09_2021.fb.arestovich_alexei.1.pereimenovyvanie_rusj_ukraina}
 
\Purl{https://www.facebook.com/alexey.arestovich/posts/4666750366722340}
\ifcmt
 author_begin
   author_id arestovich_alexei
 author_end
\fi

\ifcmt
  ig https://scontent-frt3-2.xx.fbcdn.net/v/t1.6435-9/241568700_4666791603384883_8592175662162559979_n.jpg?_nc_cat=103&_nc_rgb565=1&ccb=1-5&_nc_sid=8bfeb9&_nc_ohc=ctnlhBEVJfUAX8fI3qn&_nc_ht=scontent-frt3-2.xx&oh=3b38ee9948f56572647270c71ca23e2c&oe=615D5D8F
  @width 0.4
  @wrap \parpic[r]
\fi
\obeycr
- Для справки: мы переименовывались множество раз, только в двадцатом веке: Гетьманщина, УНР, УССР, Украина. 
Не считая ЗУНР, и всяких «Донецко-Криворожских республик». 
Что-то я не слышал рефлексии по этому поводу у великих знатоков переименований.)
То есть быть сначала «Украинской народной республикой», а потом «Советской социалистической республикой», а потом «независимой Украиной» и объяснить всему миру и собственным гражданам эту разницу и необходимость переименования в глазах наших великих «историков» нормально, а объяснить разницу между «Русь» и «Россия», между «руськими» и «россиянами» "сложно", «…мы не сможем».
За 30 лет я подустал от людей, чьи мысли и действия продиктованы страхом и тупостью.
Сначала они боялись "Запада", потом «совка», теперь боятся обидеть США/ЕС или раздраконить Кремль.
Ни сил, ни понимания, ни размаха.
Бледная спирохета вместо мозгов, заросшая вишней.
И эти люди ещё обижаются на сравнение с афганскими беженцами.)
Вся мировая история - это нарративы храбрецов, которые творили невероятное, невыполнимое.
Все, кто трусил больших решений - или в рабстве, или стёрты из истории.
Подучились бы у своих врагов великим делам, пресмыкающиеся. 
Даже казахи придумали «казахских русских». 
Тимоти Снайдер талдычит, предлагает изящные решения. А уж он-то историк и специалист по Восточной Европе всяко покруче.)
Я сказал про Русь-Украину в шутку, общаясь с Ромой Цимбалюком, но шутить на эту тему не собираюсь.
Между свинопасами и Святославом, я выбираю Святослава. 
Долбанные селюки. 
——
Вы прослушали короткий стихотворный эпиграф к лёгкому, изящному и ироничному диалогу Голованова, Гордона и Арестовича.
Подробности - ниже:
\url{https://youtu.be/et9xigZET6g}
\restorecr
