% vim: keymap=russian-jcukenwin
%%beginhead 
 
%%file 31_08_2021.fb.akunin_boris.1.kniga_istoria_rossia
%%parent 31_08_2021
 
%%url https://www.facebook.com/borisakunin/posts/2016277245188945
 
%%author_id akunin_boris
%%date 
 
%%tags istoria,kniga,rossia
%%title Сегодня дописал последний, девятый том моей «Истории»
 
%%endhead 
 
\subsection{Сегодня дописал последний, девятый том моей «Истории»}
\label{sec:31_08_2021.fb.akunin_boris.1.kniga_istoria_rossia}
 
\Purl{https://www.facebook.com/borisakunin/posts/2016277245188945}
\ifcmt
 author_begin
   author_id akunin_boris
 author_end
\fi

Сегодня дописал последний, девятый том моей «Истории». Выйдет он не очень
скоро. Впереди редактура, подбор иллюстраций, рисунки, карты, снятие вопросов с
рецензентами и всякая другая предпечатная подготовка, а потом я еще буду писать
исторический роман, тоже последний. 

\ifcmt
  ig https://scontent-frt3-1.xx.fbcdn.net/v/t1.6435-9/241037491_2016271311856205_3995226371625247474_n.jpg?_nc_cat=106&ccb=1-5&_nc_sid=730e14&_nc_ohc=vjZMOvx8k9IAX-8pUmz&_nc_ht=scontent-frt3-1.xx&oh=09fb6d2fad9beb1449465522a8243eea&oe=615D519D
  @width 0.4
  @wrap \parpic[r]
\fi

И все же главная работа закончена. Она заняла почти 10 лет. За этот срок
изменилась страна, изменились обстоятельства моей жизни, изменились мои
читатели. Ибо речено: в России всё меняется за десять лет, и ничего – за
двести. 

Ощущения у меня странные. Это как однажды  ехал я поездом на Дальний Восток.
Привык к дороге, вроде как она и есть жизнь, и вдруг здрасьте, Хабаровск.
Чемодан-вокзал-с приездом.  

(Книги на картинке – не мои исторические источники, а пресс для только что
склеенной хронологической таблицы. И напоминание о смиренном прошении древнего
летописца: «Где описал, или переписал, или не дописал - чтите, исправляя, Бога
деля [Бога ради], а не кляните, занеже книги ветшаны, а ум молод – не дошел»).
