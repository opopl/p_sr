% vim: keymap=russian-jcukenwin
%%beginhead 
 
%%file 12_01_2021.fb.fb_group.story_kiev_ua.1.kukla_cyganka_1941
%%parent 12_01_2021
 
%%url https://www.facebook.com/groups/story.kiev.ua/posts/1571725889690873
 
%%author_id fb_group.story_kiev_ua,kamencova_maria.kiev
%%date 
 
%%tags 1941,babij_jar,istoria,kiev,kievljane,semja
%%title КУКЛА-ЦЫГАНКА. Осень, 1941 год
 
%%endhead 
 
\subsection{КУКЛА-ЦЫГАНКА. Осень, 1941 год}
\label{sec:12_01_2021.fb.fb_group.story_kiev_ua.1.kukla_cyganka_1941}
 
\Purl{https://www.facebook.com/groups/story.kiev.ua/posts/1571725889690873}
\ifcmt
 author_begin
   author_id fb_group.story_kiev_ua,kamencova_maria.kiev
 author_end
\fi

КУКЛА-ЦЫГАНКА.

Осень, 1941 год. 

Моей маме Нине Николаевне Каменцовой – 12 лет.

Дом наш на улице Герцена, 21 находился недалеко от печально известного Бабьего
яра. Всего через несколько недель после взятия Киева фашистскими войсками там
начались массовые расстрелы евреев и военнопленных. Стрельба из пулеметов
слышна была целый день... Вначале никто не понимал, в чем дело, но когда ночью в
окна стали стучаться «недостреленные», жуткая картина полностью прояснилась.

Несмотря на то, что за укрывательство и помощь евреям был объявлен расстрел, им
помогали. Прятали в погребах и на чердаках, делились куском хлеба и одеждой,
носили записки их знакомым, которые могли помочь уйти из города...

\ii{12_01_2021.fb.fb_group.story_kiev_ua.1.kukla_cyganka_1941.pic.1}

После нескольких дней гибели людей в Бабьем яру евреи, которые должны были идти
на сборные пункты, уразумели, что их ведут на смерть. Они стали скрываться, не
дожидаясь приглашения.

Бабушка моя была известная портниха, у нее была богатая клиентура. Мама
рассказывала – особенно она любила завуча соседней школы, которая у нас
называлась запросто «тетя Лия». Это была красивая, роскошно одетая дама, всегда
благоухавшая – даже слишком – какими-то изумительными духами. Мама шутила, что
«мы были уверены, что эти духи пропитали не только ее одежду, но и нижнее
бельё!» Она иногда получала посылки из-за границы, которые привозили ее
знакомые, и даже дарила маме небольшие подарки.

Однажды она расщедрилась и вручила ей необычную куклу – смуглую, с очень
длинными черными волосами, которые можно было заплетать и делать из них
прически. Кукла вызвала потрясение на всей улице. Приходили посмотреть на нее
не только дети, но и взрослые! Мама страшно берегла куклу, разрешала на нее
только поглядеть и даже сама с ней не играла: прекрасная Цыганка одиноко сидела
на почетном месте среди ее игрушек. Бабушка шила ей наряды из лоскутков,
которые оставались от состоятельных клиентов.

Голова и бюст у Цыганки были фарфоровые, а тело матерчатое и чем-то туго
набито. Пластмассовых кукол тогда еще не знали...

И вот эта самая тетя Лия, осунувшаяся и утратившая весь свой шик, пришла поздно
вечером в наш дом. На ней была драная телогрейка, голова укутана в вязаный
платок... О чем она говорила с бабушками и тетками – маме не было известно, но
через час она явилась со всем семейством: мужем, двумя сыновьями и маленькой
дочкой. Они спустились в погреб, который находился в прихожей под полом, и
затихли...

Школы были закрыты, никто не учился. Дети бегали по знакомым, по соседним
улицам и приносили всевозможные известия. Мама отлично понимала, что ни слова
говорить о постояльцах не следует. Но другие дети понемногу болтали. Постепенно
стало известно, что почти во всех домах на нашей небольшой улице скрываются
евреи...

Дети тогда были воспитаны на революционной литературе, имели понятие о
конспирации. Мысль о том, что могут нагрянуть немцы, и начнутся расстрелы,
приводила всех в ужас. Поэтому составился тайный заговор и учреждено расписание
дежурств, которое строго выполнялось. В начале и конце улицы затевалась игра.
Заметив фашистов, кто-то один стремглав летел вдоль улицы, оповещал всех, и
начиналась страшная «игра в прятки». Для евреев были заготовлены надежные
тайники, куда они мгновенно скрывались.

Не знаю, как было устроено в других домах, но в нашем на крышке погреба
расстилалась дерюга, и мама с соседским мальчиком начинали на ней играть.
Расставлялись куклы и самодельные машинки, гордо восседала неприкасаемая
кукла-цыганка... Евреев разыскивали. Немецкий патруль входил в дом, обозревал все
комнаты, потом, бывало, милостиво перекидывался словом с миловидной девочкой
вполне арийской внешности, которая играла в прихожей... Мама отвечала на том же
языке, тогда несколько слов на немецком знали все, и патруль успокоенно уходил.

Но однажды этот порядок оказался нарушен. То ли немцам кто-то «стукнул», что в
домах скрываются евреи, то ли просто они были в дурном настроении, но они
обшарили все комнаты, чуть ли не принюхивались...

А может быть, среди них попался какой-то бывший работник парфюмерной фабрики
или просто человек с очень острым обонянием... Во всяком случае, уже выходя из
дома, он остановился неподалеку от погреба, долго пофыркивал носом, потом
поднял палец и зловеще процедил:

— Юде!

И тут мама, ни жива ни мертва, поняла, что он уловил запах духов, который
пропитывал всю одежду тети Лии и, вероятно, сохранился даже после ее
переодевания и заточения в погреб! Дом был бедный, и роскошный запах дорогих
духов был совершенно неуместен...

Немец еще потянул носом и с удовлетворением отметил:

— Шанель номер пять! Дас ист юде! Девочка, где евреи?!

Как бы не понимая, вся поглощенная игрой, мама замахнулась на своего партнера,
начиная ссору, что-то закричала по-русски, а потом схватила священную
куклу-цыганку и стала бить ее головой о стену, восклицая в ярости:

— Дас ист юде! Дас ист юде! Юде шлехт, юде капут!!!

Смуглая фарфоровая головка тут же раскрошилась, мама схватила куклу за волосы и
стала ею бить мальчика, который сидел в полном оцепенении...

— Ну-ну, девочка, не надо бить мужчину! – загоготали немцы, которые тоже
немного уже выучили язык покоренного народа. Умиротворенные, они потянулись к
выходу, а один даже вынул из кармана маленькую шоколадку и дал ее маме.

Если тут так беспощадно колотят куклу-еврейку или цыганку, то евреев здесь
точно ненавидят, а значит, их здесь нет!

... Мама рыдала в голос, она никого не хотела слушать, забившись в дальние
комнаты. Но выпущенная из погреба тетя Лия недаром была в свое время завучем
школы. Даже сейчас она не растеряла своих великолепных манер. Она величественно
приблизилась к ревущей, как белуга, маме и снисходительно произнесла:

— Ты будешь великой актрисой!

И всё. Слезы сразу высохли, рев пропал...

Шоколадку мама вышвырнула в сад.

Через несколько дней пришел ответ от знакомых тети Лии, и они, выразив глубокую
благодарность, гуськом покинули наш дом: ее муж, она сама, два сына и дочь...

Как ни странно, все они уцелели, и после войны она как ни в чем не бывало
вернулась на свое место завуча. Сталинские гонения на евреев их не коснулись:
должно быть, у них были связи...

А через двадцать лет, в первой же волне еврейской эмиграции, все они уехали в
Израиль. Было их уже не пятеро, а много больше... С тех пор мы о них ничего не
слышали.

А маму действительно, остановив на улице, пригласили на киностудию, она немного
поснималась в эпизодах, – но потом приставания режиссеров отвратили ее от
кинематографа. Она поступила в институт учиться на бухгалтера, которым и
проработала всю жизнь.

Когда мама, уже в преклонных летах, рассказала мне эту историю, я долго не
могла прийти в себя:

— Ма, так это ж получается – ты у нас ПРАВЕДНИК МИРА! Можно было бы найти эту
тетю Лию, в Израиле есть организации, которые всё находят, у нас в окрестностях
не так много было школ...

— Ну и зачем – через столько лет?! – пожала она плечами. – Я не только их, я и
себя спасала! Куклу эту до сих пор жалко, – призналась она. – Больше такой у
меня не было...

\ii{12_01_2021.fb.fb_group.story_kiev_ua.1.kukla_cyganka_1941.cmt}
