% vim: keymap=russian-jcukenwin
%%beginhead 
 
%%file 06_12_2021.fb.zharkih_denis.1.vojny_ne_budet_huzhe.cmt
%%parent 06_12_2021.fb.zharkih_denis.1.vojny_ne_budet_huzhe
 
%%url 
 
%%author_id 
%%date 
 
%%tags 
%%title 
 
%%endhead 
\subsubsection{Коментарі}

\begin{itemize} % {
\iusr{Саша Тихилов}

Зато сегодня отличался подельник Медведчука, бывший кандидат в президенты
бойко, похвалив всу. Он точно решил перейти в партию потроха

\begin{itemize} % {
\iusr{Денис Жарких}
\textbf{Саша Тихилов} в том-то и дело, что не подельник. Может, поэтому  @igg{fbicon.thinking.face} 

\iusr{Дмитрий Воронин}
Бойко это не зачтется... напрасно и глупо

\iusr{Светлана Гнатьева}
\textbf{Саша Тихилов}
Это хорошо, что Бойко снимает маску.

\enquote{Подельником} Медведчука он никогда не был. Это ситуативное \enquote{сотрудничество},
они скорее конкуренты. Разве вы не заметили, что всё это время, после избрания
Зели, Медведчук и Бойко шли, как отдельные лидеры и кандидаты в президенты?
Бойко и Лёвочкие - подельники.


\iusr{Саша Тихилов}

Я более - менее смотрю новости и прекрасно помню, как медведчук вместе с бойко
приезжали в Москву к премьеру Медведеву и договаривались о газе, потом эта
двоица приезжала на Петербургский международный экономический форум и там
умничала. Медведчук не такой человек, которого можно назвать неосторожным или
глупым. Хотя оба они не проходные как кандидаты в президенты Украины. Лохторат
этот можно бесконечно кормить обицянками про светлое будущее в Европе, в нато,
под вопли америка с нами и завтра будут зарплаты в 1000 евро и пенсии в 500. Я
смотрю на ютубе соцопросы киян на очередную годовщину майдана - большинство так
ничего и не поняли и ничему не научились. Они неизлечимы и не исправимы. Я ещё
в 2014 говорил, что майдан головного мозга- это неизлечимое заболевание.
\end{itemize} % }

\iusr{Andrey Kichatov}

\obeycr
ЦИКЛ:ВЕЩИ
ВАРЕНЬЕ
ВАРЕНЬЕ ВАРИТСЯ В КАСТРЮЛЕ
КИПИТ ДОХОДИТ ВОТ УЖЕ
ПО БАНКАМ РАЗЛИВАТЬСЯ БУДЕТ
ГОТОВЫ КРЫШКИ НА МЕЖЕ
ВСЕ ЗАКРУТИВ ПЕРЕВЕРНУЛИ
ПОД ОДЕЯЛО В ПЛОТНЫЙ РЯД
ЧТОБ ПАРИЛИСЬ НЕ ПРОДОХНУЛИ
И ЧТОБ НИ СЛОВА НЕВПОПАД
24.11.2021
\restorecr

\iusr{Вадим Павский}
Солдат ребенка не обидит. Потому войны не будет. Увы.

\iusr{Дмитрий Мундштуков}

Это всё безумно интересно. Но неужели это не было очевидно 2,5 года назад,
когда на пост президента 40-миллионного европейского государства совершенно
сознательно избрали клоуна?  @igg{fbicon.face.confused} 

\begin{itemize} % {
\iusr{Виктория Ромина}
\textbf{Дмитрий Мундштуков} 30-миллионного

\iusr{Дмитрий Мундштуков}
\textbf{Виктория} , ну я образно, поскольку точно сейчас никто не знает.  @igg{fbicon.beaming.face.smiling.eyes} 
\end{itemize} % }

\iusr{Наталья Наталья}

Вопрос вообще изначально \enquote{зачем}? Украина с разрушенной экономикой, с кучей
долгов. Долги выплачивать потом за Украину? Восстанавливать экономику? Зачем
России все это?

\begin{itemize} % {
\iusr{Лариса Святодух}
\textbf{Наталья Наталья} конечно незачем. Но не понимают. Когда говорю об этом знакомым, вижу - им это даже в голову не приходит.

\iusr{Саша Тихилов}

Разруха прежде всего в головах. Заводы и дороги можно в конце концов построить
с нуля, больной мозг не заменить. Они реально больны майданом и русофобией.
Именно этот народ жаждет на троне клоунов и мошенников

\end{itemize} % }

\iusr{Александр Овчинников}
Денис отлично сказано ! просто супер .

\iusr{Олег Хавич}
Не переживайте, идиотский СММ офиса Медведчука - не хуже войны. Пишите  @igg{fbicon.wink} 

\iusr{Петр Вдович}

России война, конечно не выгодна. Она выгодна ...нашей власти.
В этом я и вижу опасность.

\begin{itemize} % {
\iusr{Виктория Ромина}
\textbf{Петр Вдович} все это понимают, кроме кастрюль..

\iusr{Петр Вдович}
\textbf{Виктория Ромина} кастрюли...это особые люди и особый статус.
Надеяться на то что они что то поймут бессмысленно и бесполезно.

\iusr{Виктория Ромина}
\textbf{Петр Вдович} увы
\end{itemize} % }

\iusr{Светлана Головина}
Хороший пост.

\iusr{Вячеслав Беленький}

Это называется План Моргентау для Украины в одновременном заказе Запада и
России, при непосредственном исполнении самой постмайданной властью, взявшей
курс на ускоренный суицид страны.

ВВМ всегда складно излагает, но \enquote{как далеки они от народа} (с)

\iusr{Владимир Павлович}
Просто бла бла, все может резко изменится, поэтому и риторика о войне

\iusr{Ира Гаврилова}
Короче, третьей мировой не будет. Как, похоже, и У. в ее нынешнем виде.

\begin{itemize} % {
\iusr{Елена Ланская}
\textbf{Ира Гаврилова}, увы..! Вчера услышал от одного из первых лиц в ГД, что Россия брезгует вести переговоры с нынешним типа правительством Украины. Не было такого НИКОГДА.! И вот ... пожалуйста (((

\iusr{Вячеслав Горловский}
\textbf{Ира Гаврилова} а У уже и так нет

\end{itemize} % }

\iusr{Александр Вербицкий}
Да, это очень интересно. Не менее интересно, чем Трипольская культура.

\iusr{Сергей Гайдаш}

Если развалится, то будут решены многие проблемы. Исчезнут многие вопросы и
острые проблемы. Ведь по сути, человек хочет лучше жить. Это на первом месте.
Ему не важно, как будет называться его государство.

\iusr{Sergei Cherny}
Лучше не скажешь.  @igg{fbicon.thumb.up.yellow} 

\iusr{Ольга Горлатая}

Написано отлично, всё по полочкам, но самое обидно, что это понимаем
мы, нормальные люди, но они, властьпридержащих, этого понять не хотят, у них
другие задачи.

\begin{itemize} % {
\iusr{Вика Малышева}
\textbf{Ольга Горлатая} всё они понимают, но им платят за другое..... а страдает народ(((

\iusr{Ольга Горлатая}
\textbf{Вика Малышева} Вот я и говорю, что у них задачи другие, не потушить а разжечь поярче..

\iusr{Lara Grigoryeva}
\textbf{Ольга Горлатая} все они понимают, но работают на публику и переигрывают постоянно заигрывая с западом и американцами. Назад хода нет и остается капкан
\end{itemize} % }

\iusr{Алексей Ч.}

В последние времена \enquote{веселый антихрист} приходит, как \enquote{черт из табакерки} к
маловерам, к \enquote{иванам не помнящим родства} по их души окаянные, закланные от
начала века для инферно.

\iusr{Sergey Raykov}

Что может быть хуже войны? Я думал, что ничего... Но оказывается
\enquote{экономическое удушение со стороны России?}

Но позвольте....

Украина в блокаде? Украина изолирована? У Украины кроме России нет соседей?

Россия предупреждала, что если Украина вступит в ассоциацию с ЕС и откроет свой
рынок для товаров из ЕС, Россия будет вынуждена ввести таможенный контроль и
регулировать(ограничивать) товары с Украины. Так оно и случилось.

Потом пошли взаимные санкции. Не хочу говорить о том, кто первый начал. Стороны
понесли взаимные убытки, причем Украина большие в относительных значениях. Это
тоже удушение?

Прекращена кооперация и сотрудничество в оборонно- промышленной сфере. По
инициативе Украины.

По поставкам нефти, газа, электроэнергии сказано много.

Осталось прекратить поставку кокса для металлургии Украины и картина будет
полная.

Разумеется, если к власти придут Медведчук, Азаров с опорой на Ахметова, то
России следует перейти от \enquote{удушения} к субсидированию? Наверное...так

Ещё Кучма сказал, что если из 130 млрд.куб. газа из России в Европу забрать 1-2
млрд. никто не заметит, с Россией это можно, это не ЕС..

\begin{itemize} % {
\iusr{Сергей Безгин}
\textbf{Sergey Raykov} вы правильно описали суть - "забрать...никто не заметит". И как с такими работать?

\iusr{Sergey Raykov}
\textbf{Сергей Безгин} надо выстраивать другие отношения

\iusr{Сергей Безгин}
\textbf{Sergey Raykov} надо. надо твёрдо сказать-"мы заметили. "Мы хотим вот этого и этого, а взамен можем дать столько то и по таким вот ценам. Всё просто, вопрос торга. Но главное-вопрос доверия. А доверие разрушено. И не только между странами, но и внутри стран

\iusr{Sergey Raykov}
\textbf{Сергей Безгин} Отношения уже не будут прежними, к прошлому возврата нет. Они другие и мы другие. Надо четко сказать. Не будет никакого союзного государства.

\iusr{Сергей Безгин}
\textbf{Sergey Raykov} а какие были прежние? Без союза уже дети выросли и сами родителями стали, кому тот союз теперь нужен? Хотя бы доверие нужно.

\iusr{Sergey Raykov}
\textbf{Сергей Безгин} было СНГ и Украина этим пользовалась...вырос до новое поколение, не знавшее Союза, вы правы..но и СНГ не надо...

\iusr{Александр Новиков}
\textbf{Sergey Raykov} сначала вам надо перекрыть поставки кокса для адмнистрации вашего т.н. президента  @igg{fbicon.face.tears.of.joy} 

\iusr{Sergey Raykov}
\textbf{Александр Новиков} кого вы называете вашим?

\iusr{Александр Новиков}
\textbf{Sergey Raykov} украиньскаво

\iusr{Sergey Raykov}
\textbf{Александр Новиков} у меня другой президент

\iusr{Александр Новиков}
\textbf{Sergey Raykov} у нас другой президент. И слава Богу!
\end{itemize} % }

\iusr{Наталья Панфилова}

Таких стран куча, та же Прибалтика, которые сдыхают но не сдаются придуманным
ворогам. Всрався но не сдався, это оно. Так что будет и дальше гнить, а народ
молчать и тикать.

\iusr{Елена Буданова}

\enquote{.... дурить народу голову бесконечно не получится, нельзя бесконечно называть
черное белым, войну миром, а нищету процветанием. Нельзя бесконечно ограбление
страны выдавать за европейский выбор, а установление диктатуры победой
демократии.} - цитата из Вашего текста.

Не согласна абсолютно!

Получится. Можно.

Последние 8 лет тому доказательство.

Кто может уехать из этого дурдома - тот уже уехал или, в конце концов, уедет.

Остальные, что бы они там себе не понимали, будут вынуждены жить в этом треше.
А с учётом того, что треш ощущают все, живущие в Украине, но большинство из
тех, кто его ощущает на себе, не понимают кому они этим трешем обязаны -
тянуться эта бодяга может до бесконечности....

И ещё.

Опять же цитата: \enquote{Россия идет по пути экономического удушения Украины не первый
год.}.

Да ладно Вам!  @igg{fbicon.face.tears.of.joy}  А куда прикажете засунуть тот факт, что Россия входит в тройку
основных торговых партнеров Украины? .

\begin{itemize} % {
\iusr{Петр Вдович}
\textbf{Елена Буданова} по торговле с Россией...здесь сложнее. Они все правильно делают. Нельзя было сразу обрубить связи.
И даже то что украинские танки ездят на российском дизеле...это тоже правильный ход со стороны России.
Стратегически Кремль делает всё правильно.

\iusr{Елена Буданова}
\textbf{Петр Вдович} Для себя - да! Только я вот не в РФ живу.... Как и миллионы моих соотечественников. И нам эти, принимаемые Россией и правильные для неё, решения вылазят боком!

\iusr{Петр Вдович}
\textbf{Елена Буданова} для многих вылазят боком...
Посмотрим к чему сегодня договоряться дедушка Джо и дедушка Пу)

\iusr{Елена Буданова}
\textbf{Петр Вдович} 

Такие встречи (даже в видео-формате) не проходят без предварительных
договорённостей об их результатах.

Но мы не узнаем этих договорённостей ни сегодня, ни когда-нибудь потом.... Так
что нет, НЕ посмотрим о чём сегодня договорятся \enquote{дедушка Джо и дедушка Пу} ))

\iusr{Ирена Березник}
\textbf{Елена Буданова} ,

так ведь каждое нормальное правительство заботится именно об интересах СВОЕЙ
страны, а не соседней! Иначе оно будет предателем для своих жителей. А поскольку
интересы РФ и Украины сейчас диаметрально противоположны, то то, что хорошо одной
стороне, всегда будет плохо для другой. И вообще, если вдаваться в аллегории, то
представляется картинка, как щупленького боксера легкого веса поставили против
чемпиона-тяжеловеса. И тренер его подначивает-давай, ты сможешь! Судья
наш-подыграет, и мысленно мы с тобой. Тяжеловес тоже делает разные финты, чтобы
не уложить с первого раза, а показать картинку. Но финал очевиден. Как-то так.

\iusr{Елена Буданова}
\textbf{Ирена Березник} 

Да, каждое нормальное правительство заботится о своём государстве. Но я не
гражданка РФ и меня бы очень устроило, чтобы правительство РФ заботилось об
интересах России исключительно в пределах территории России, а не за мой счёт.
И не за счёт моих друзей и близких.

Или Вы реально не понимаете, что решая свои геополитические вопросы в Крыму и
на Донбассе в 2014 году, Россия лишила меня и моих единомышленников, живущих на
подконтрольной Киеву части Украины, шанса связь реванш в 2019 году? И боюсь, не
только в 2019, а вообще в обозримой перспективе....

И да, я верю, что крымчане счастливы. Но меня вот только никто не спросил, а
хочу ли я счастья крымчанам ценою собственного счастья, ценою будущего своих
детей, друзей, не живущих в Крыму и, тем более, ценою жизней и сломанных судеб,
дончан?

Как-то так, Ирина!


\iusr{Ирена Березник}
\textbf{Елена Буданова} ,

ну, вообще-то я живу там же, где и вы. И вы задали интересный вопрос-насчет
того, спросили ли вас крымчане насчет того, хотите ли вы чтобы они были счастливы
ценой вашего счастья? Так ведь палка о двух концах, и в противном случае они
могли бы задать вам такой же вопрос. Это напоминает ситуацию с брошенной женой
(мужем), которая(ый), уходит к любовнику(це). Один из них по-любому будет
несчастным, но ни один из них не сможет изменить ситуацию-один не сможет, второй
не захочет.

\iusr{Елена Буданова}
\textbf{Ирена Березник} 

Это НЕ напоминает ситуацию с женой - не повторяйте банальности из соц. сетей.
Далеко ушёл Донбасс? Вот ровно туда же и Крым бы дошёл, несмотря на все желания
или нежелания крымчан, если бы не геополитические планы России. И я, искренне
радуясь, что чаяния большинства крымчан совпали с интересами РФ, не могу не
понимать, что свои интересы Россия решала мне в ущерб. Мне и миллионам таких,
как я - нам она только добавила проблем. А ещё я не могу не понимать, что
Россия решала именно свои проблемы, свои, а не проблемы крымчан. \enquote{Русские своих
не бросают} годится только как яркий лозунг - в жизни всё, как мы видим,
иначе.... Крымчанам просто сказочно повезло, что их желания совпали с желанием
Путина. Ну а не совпали бы - это были бы такие же неразрешимые проблемы
крымчан, как и те, которые нам, живущим на подконтрольной Киеву территории,
создал майдан и добавил (вдруг нам показалось мало?!) Путин.

\iusr{Ирена Березник}
\textbf{Елена Буданова} ,

вообще не спорю, что в Крыму стратегические интересы РФ. И что они совпали с
интересами жителей. Ситуацию, описанную ранее, не повторяю, она мне пришла в
голову сразу после события. А Донбасс, разумеется жаль, впрочем, это глупое слово
никак не передает настоящих чувств, но тут можно бесконечно перечислять, кто
виноват, но сегодняшнюю ситуацию это не изменит.


\iusr{Елена Буданова}
\textbf{Ирена Березник} 

Жаль - это когда ноготь сломала, только выйдя из салона, а миллионы разрушенных
судеб (что на подконтрольной Киеву территории, что не на подконтрольной) - это
называется иначе... Это трагедия. И да, ничего уже не изменишь.... Майдан этот
ад сделал реальностью, а Кремль, решая свои вопросы, только подлил масла в
огонь....

\iusr{Петр Вдович}
\textbf{Елена Буданова} да, Донбасс оказался между молотом и наковальней. Но именно киевские власти усугубили гуманитарную и военную ситуацию.
Как сейчас это разрешить...я уже незнаю. Похоже что поезд с Донбассом уходит.
РС. Не Майдан сотворил там ад, а госпереворот.
Я лично эти две категории разделяю.

\iusr{Елена Буданова}
\textbf{Петр Вдович} 

Пётр, для меня киевские власти - не повод для разговора. От слова вообще. Что
они там усугубили, что не усугубили.... Смысл по сотому кругу перетирать то,
что и так понятно про тех, кто пришёл к власти в 2014 и в 2019? (и как бы Вы
там для себя не разделяли понятия госпереворот и майдан - это слова-синонимы.
На все миллион процентов!).

Речь шла о роли России.

Я не хочу сейчас тратить своё время, доказывая ни Вам, ни кому бы то ни было,
что вода мокрая, но я просто всё, что происходило на протяжении этих 8 лет,
очень хорошо помню. И о том, как Россия признала вооружённый переворот
демократической сменой власти в Украине и прислала Зурабова к Порошенко на
инаугурацию я тоже помню... Как и многое другое.

Чего и всем желаю. Помнить. Думать. Анализировать. И не очаровываться.

\end{itemize} % }

\iusr{Виктория Фёдорова}
Ну почему же нельзя бесконечно дурить, называть чёрное белым и пр.?
Можно.
Доказано двумя майданами и их последствиями. Или перестали быть те, кто верить в чёрное на белом?

\iusr{Ирина Новикова}
Те, кто этого не понимает, уже в Польше давно, а голосят те, кому платят

\iusr{Ева Раш}
Будет. Будет. Иначе для чего все это вообще затевалось?

\iusr{Oleksa Karma}
Мир так хрупок! Помните, как все войны начинались?
С провокаций. Нужно болтун, провокаторов, подстрекателей изолировать от общества.

\iusr{Марк Зоряний}

Вылизывание Пресс-служба Виктора Медведчука глубоко и во все дыры. Дениска
написал очередной фашистский говновброс на радость кремлеботам и вате

\begin{itemize} % {
\iusr{Илья Кирьязиев}
\textbf{Марк Зоряний} не ругайся, боженька язык отрежет!

\iusr{Марк Зоряний}
\textbf{Илья Кирьязиев} за своим языком следи. Я не ругался, а написал правду

\iusr{Татьяна Павлович}
\textbf{Марк Зоряний} Для больного шизофренией его бред есть правда. На учёте состоишь?

\iusr{Марк Зоряний}
\textbf{Татьяна Павлович} сразу видно, что предки в нквд и кгб работали, всех в дурку готова ватница отправить. Фашистка с русскоязычной челюстью

\iusr{Ната Кондратьева}
\textbf{Марк Зоряний}  @igg{fbicon.wheelchair} 

\iusr{Марк Зоряний}
\textbf{Ната Кондратьева} что ватница и кремлеботка в Эстонии скучно заниматься стало антиэстонской пропагандой?

\iusr{Ната Кондратьева}
\textbf{Марк Зоряний}  @igg{fbicon.gorilla} 
\end{itemize} % }

\iusr{Юрий Радковский}

Виктор Владимирович хороший юрист, но проблема в том, что он не учитывает тот
факт, что пол страны не грамотные в этом вопросе, а остальные сильно умные. В
итоге и те и другие ни во что не ставят писанные законы и \enquote{работают по
беспределу.} А потом дружными рядами валят с прикарманенным баблом на ПМЖ. И за
30 лет за это наказали только одного персонажа - Павла Лазоренко. И не у нас, а
в США. После этого мероприятия, я так думаю, Сорос и воспитал у нас в стране
поколение \enquote{СОРОСЯТ} и в итоге, как говорил один наш презик, маемо ТЭ, що маемо.
Ну а в итоге те, кто не имеет желания или возможности свалить из страны -
должны идти на войну.


\iusr{Wlad Sankin}

Вот насчет нищеты. Картинка из жизни. Аэропорт в турецком городе Анталья. Кто
стоит в очереди за бутербродами по 5 евро? Пассажиры, летящие по направлению
Киев, Запорожье, Харьков. Везде слышна русская речь с характерным южным
говорком. Компании дам в возрасте 50 плюс распивают (время - восемь утра)
вермуты из дьюти фри. Мужья сидят рядом. Народ простой, больше смахивает на
рабочих или мелких служащих ну или предпринимателей типа лавочников или
таксистов. Получается, не все деньги в отпуске прокутили...

\iusr{Дмитрий Конончук}
Я так и знал, что Пономарь был прав. Анаконда существует.

\iusr{Сергей Падалкин}

Да ладно. 30 лет получалось дурить головы - получится и дальше. Тем паче,
растут поколения зомбированные, не знающие истории, не умеющие мыслить и
задавать вопросы

\begin{itemize} % {
\iusr{Александр Новиков}
\textbf{Сергей Падалкин} вот-вот, медиадебилы в виртуальной реальности.
\end{itemize} % }

\iusr{Татьяна Сидоренко}

Мое мнение, человека из простого народа. Денис, не в обиду будь сказано, у вас
есть а странице практически все люди из среды творческой, близко знакомой в
узких кругах, единомышленники, гости и работники СМИ и политических ток шоу на
ТВ. Думаю, Медведчук ещё более удален от простого смертного работяги.

Потому соглашусь с Елена Буданова -народ выживает каждый в своей норке. Кто
смог уехать -уехал, кто-то это планирует, а остальные будут жаловаться на
зарплаты, коммуналку, здравоохранение и образование, будут скрипеть зубами и
ждать \enquote{покращення}. Не будет протестов! Все эти марши под ОП или за
деньги или, как у мелких предпринимателей, от большого отчаяния. Но когда они
начинают возмущаться их не поддерживают ни шахтеры, которым не платят месяцами,
ни медики, ни учителя! Бюджетники вообще зашуганы-там увольнение грозит за
лишний вздох!

Те, кто понимал, что гiднiсть пройдет катком по каждому, сейчас или часть
России или живут восьмой год под обстрелами. Остальные приспособились к
реалиям. Это не осуждение, я на это не имею права, это жизнь.

Но она не стоит на месте, она меняется. Все мы лягушки, которых медленно варят.
И то, что в 2013 было не допустимо, стало обыденностью нашей жизни.  Медведчук
прав - будет хуже, но народ и к этому приспособится.

\iusr{Игорь Красиков}

Господи, ну настолько же все очевидно, что даже непонятно, о чем спорят-то...

России выгодна слабая и застрявшая в войне Украина. Ergo, никаких военных
действий не будет, как не будет и присоединения Донбасса к России. Еще —
поддерживать Украину в раздробленном и растерзанном состоянии могут только
националисты, которым экономика и прочее до лампочки. Ergo — их будут
прикармливать и дальше, не доводя ситуацию в стране до того накала, когда может
резко поменяться власть — потому что может прийти здравомыслящая власть,
которая легко объединит страну — да просто вернув статус русского языка — и
может начать ее доставать из ямы. А это очень опасно для всех, кто ставит на
слабую Украину.

Кстати, этого не понимают в Америке. Впрочем, возможно, в их парадигме тоже
нужен именно такой враг России, который реально ничего сделать не в состоянии,
и поэтому тоже поддерживаются именно те лица во власти, которые страну поднять
не смогут.

Поэтому — не волнуйтесь. Войны не будет. Умереть не дадут, нормально жить тоже
не позволят. Если не произойдет чего-то из ряда вон выходящего, пока существует
режим обнуления — никаких надежд на изменение ситуации нет в принципе. Разве
что кто-то из игроков решит пойти ва-банк... но это очень рискованная игра, так
что я бы на это не рассчитывал.


\iusr{Yurii Demchenko}
Печально все это ! А мы ? А что мы ? Лежим и думаем...Доколе ???

\iusr{Дмитрий Васильевич}
Это все понятно. А дальше что? Нет дальше..

\iusr{Любовь Сорокина}

ВОЙНА С РОССИЕЙ- ЭТО ТЫСЯЧА ПЕРВАЯ СКАЗКА ЗЕЛЕНСКОГО И ЕГО ВЛАСТИ ..СОВМЕСТНО С
США. ОТВЛЕКАЮЩИЙ МАНЕВР.. ОТ ТОГО, ЧТО СТРАНУ ПРОДОЛЖАЮТ РАЗВОРОВЫВАТЬ, УБИВАТЬ,
УНИЧТОЖАТЬ МЫ Ж ЭТО ВСЕ ПОНИМАЕМ... И ВЕДЕМСЯ НА ЭТО


\iusr{Константин Камаев}

Какая то заварушка будет, скорее всего под Олимпиаду в Китае, это 4 февраля
2022. Так уже было в 08.08.08.


\iusr{Борис Чипак}

Все, и США, и наши дебилы знают, что России не выгодно нападать на Украину,
но в планы это не входит. Для США нужна война и банде Порошенко тоже мир не
нужен, но все единогласно Россия нападе.

\iusr{Александр Ста́лив}

Денис, здравствуйте.

Всё, что происходит и существует внутри и вокруг Украины, всё это
осуществляется с подачи заокеанских и западноевропейских финансовых мошенников,
их подневольных наёмников и пособников.

Именно и только из этого следует исходить.

Всё остальное - впустую.

Вы слышите, Денис?

Украина - одна из жертв преступлений заокеанских и западноевропейских
финансовых мошенников. Финансовых мошенников, Денис, заостряю на этом особо
пристальное Ваше внимание.

И украинцы в придурков-потребителей превратились за последние три десятилетия
именно с подачи заокеанских и западноевропейских финансовых мошенников.

Украинцев злоумышленно, хитроумно, изощрённо, ловко, незаметно и в последние
годы внаглую, открыто многоходовками потребляют. Внаглую и открыто потому, что
украинцы уже настолько отупели, лишились человечности, что не соображают, в
какой халэпе (в беде) оказались и находятся.

Осуществляется самоистребление и самоизничтожение населения Украины, Денис.
Самоистребление и самоизничтожение. Своими собственными руками.

Вот эти все внутренние межусобные разборки, распри, вражда, раздоры, дошедшие
до бойни-войны, человеконенавистничества и живодёрства, смертоубийств, разрухи
отраслевого хозяйствования (экономики) государства - всё это следствие
злоумышленного разрушения Украины заокеанскими и западноевропейскими
финансовыми мошенниками. Мошенниками, Денис, вы только об этом задумайтесь.
Мошенниками.

Никакой правовой (нормативной) деятельности в Украине не осуществляется. Сплошь
преступления.

Повсеместно грязь, замусоренность, бесхозяйственность вопиющая - всё это
преступления, Денис. Преступления на бытовом уровне. Вы только в это
вдумайтесь.

Население Украины, люди - обездоленные и обиженные, злые, зловредные в общении
и отношениях. И к этому привыкли. Привыкли к звероподобному межусобному
отношению и общению.

Понимаете Вы это или нет?

И в такой звериной хищнической несостоятельности, Денис, политическая
деятельность отсутствует. Что очевидно.

Отсюда придурки на всех уровнях общественной деятельности. Нет ни единого
доброго, целеустремлённого, волевого человека. Ни единого, Денис.

Никто отраслевым хозяйствованием (экономикой) государства не занимается. Потому
что в этом не соображает.

Сами подумайте, когда соображают, разве настолько разваливают общество?

В том числе и Медведчук, и он пустослов, Денис. Ответственно об этом написал.

Деловой человек, он прежде всего и завсегда добрый, приветливый, приятный,
вежливый и способен находить и поддерживать общий язык с каждым. И этим он и
осуществляет любую деятельность, в том числе и экономическую, и политическую. В
этом состоит суть идеологии. В доброте. В чести. В правде. В совести. В
справедливости. В равенстве. Всё - во благо, Денис.

Вот это и есть политик. Это и есть политика.

Политика это сполна доброе благородное занятие и дело, Денис.

В политике, запомните это, не существует борьбы.

И там, где существует борьба, там отсутствует политика, отсутствует
политическая деятельность.

Борьба, Денис, это слабоумное (дементивное) потребительское политиканство.

И украинцев заокеанские и западноевропейские финансовые мошенники злоумышленно
превратили в слабоумных (дементивных) политиканов. Чтобы потреблять Украину.

Вы это понимаете?

Засилье внутри и вокруг Украины с подачи заокеанских и западноевропейских
финансовых мошенников.

Это главное. Суть.

Все межусобные внутренние разборки в Украине, все выяснения отношений с другими
государствами - всё это осуществляется с подачи заокеанских и
западноевропейских финансовых мошенников. Мошенников, Денис. Мошенников. То
есть - преступников. Преступников огромного масштаба.

Вот из чего следует исходить, Денис.

(Продолжение и окончание ниже в комментарии).
\end{itemize} % }
