% vim: keymap=russian-jcukenwin
%%beginhead 
 
%%file 20_07_2020.fb.lnr.5
%%parent 20_07_2020
 
%%endhead 

\subsection{Путин согласился перенести акцию «Бессмертный полк» на 2021 год}
\url{https://www.facebook.com/groups/LNRGUMO/permalink/2862149767229912/}
  
\vspace{0.5cm}
{\small\LaTeX~section: \verb|20_07_2020.fb.lnr.5| project: \verb|letopis| rootid: \verb|p_saintrussia|}
\vspace{0.5cm}

Андрей Аркадьев 20.07.2020

Ранее организаторы акции предложили отменить ее из-за эпидемиологической
ситуации в регионах России

Президент России Владимир Путин согласился перенести всероссийское шествие
«Бессмертный полк» на следующий год.

«Как бы нам ни хотелось, но придется перенести мероприятие на следующий год», -
сказал глава государства во время общения с рабочими судостроительного завода
«Залив» в Керчи.

Путин принимает в Крыму участие в церемонии закладки боевых кораблей для ВМФ.
Рабочая поездка изначально планировалась на 16 июля, но была перенесена, в тот
день глава государства провел совещание по бюджетному процессу.

В этом году из-за пандемии традиционные массовые мероприятия, которые ежегодно
проводятся 9 Мая, были отменены. Состоялись только пролет авиации и салюты.
Главный парад на Красной площади в Москве, а также парады в других городах
прошли 24 июня, в день, когда 75 лет назад состоялся самый первый Парад Победы.

Всероссийская акция «Бессмертный полк» была запланирована на 26 июля, в День
Военно-морского флота России. 
