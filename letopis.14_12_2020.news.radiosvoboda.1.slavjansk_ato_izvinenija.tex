% vim: keymap=russian-jcukenwin
%%beginhead 
 
%%file 14_12_2020.news.radiosvoboda.1.slavjansk_ato_izvinenija
%%parent 14_12_2020
 
%%url https://www.radiosvoboda.org/a/30998591.html
 
%%author 
%%author_id 
%%author_url 
 
%%tags 
%%title «Вам мало не покажется»: после скандальной сессии горсовета ветераны АТО добились извинений
 
%%endhead 
 
\subsection{«Вам мало не покажется»: после скандальной сессии горсовета ветераны АТО добились извинений}
\label{sec:14_12_2020.news.radiosvoboda.1.slavjansk_ato_izvinenija}
\Purl{https://www.radiosvoboda.org/a/30998591.html}

\ifcmt
pic https://gdb.rferl.org/9c18d7f6-e5f6-4bce-9311-2536c5ab226b_w1023_r1_s.jpg
caption Во время автопробега, Славянск, 13 декабря 2020 года
width 0.4
fig_env wrapfigure
\fi

\textbf{После скандальной (с полицией и потасовкой) первой сессии городского совета
Славянска некоторым депутатам и ветеранам АТО звонили и угрожали. Об этом
заявили участники заседания, которые 11 декабря возмущались участием в нем
партии ОПЗЖ и Нели Штепы. В воскресенье бывшие военные из разных городов
Донецкой области собрались в Славянске требовать извинений от присутствовавшего
на той сессии предпринимателя, от которого, как уверяют, и получили угрозы. Что
вышло, видело Радио Донбасс.Реалии.}

На месте сбора ветераны включают запись телефонного разговора. Говорят, что
голос на записи предположительно принадлежит предпринимателю из Славянска Олегу
Берко.

«Я вас в рот … всех. И атошников твоих. Вызывай их во вторник (во вторник
продолжится после перерыва сессия горсовета Славянска)…», – говорит голос,
похожий на голос предпринимателя Берко.

%\ifcmt
%pic https://scontent-frx5-1.xx.fbcdn.net/v/t15.13418-10/p526x395/129767660_454190442517841_633542160826243393_n.jpg?_nc_cat=110&amp;ccb=2&amp;_nc_sid=08861d&amp;_nc_ohc=XQCamgYKKckAX9ISZHn&amp;_nc_ht=scontent-frx5-1.xx&amp;tp=6&amp;oh=75788c8a9bc55fd116644890afd7a42f&amp;oe=5FFE2B79
%\fi

В пятницу, 4 декабря, во время перерыва в работе сессии, после потасовки, Олег
Берко уже высказывался в адрес ветеранов АТО: «У меня работает 150 «атошников».
Люди стоят за станками. А которые кричат, – где они были, «атошники». Да у нас
полстраны «атошников». Только у Свинарчуков они не спрашивают, куда деньги
делись, и куда жизни подевались».

Перед стартом авопробега ветераны АТО решили так: если предприниматель говорил
эти слова, то он должен публично принести извинения.

\subsubsection{Храм Василия Блаженного и файеры в цветах УПА}

Первый пункт назначения – офисное здание. Ветераны говорят, что тут расположен
офис Олега Берко и заодно собираются на совещания члены партии ОПЗЖ. Внутри,
кроме сторожа, никого нет. Всеобщее внимание привлекает макет храма Василия
Блаженного (расположен в Москве на Красной площади). Рядом с макетом стоит
флажок ОПЗЖ.

\ifcmt
pic https://gdb.rferl.org/c83ac5f9-bdc4-4cdd-be88-693be1a0d62b_w650_r1_s.jpg
width 0.4
fig_env wrapfigure
\fi

Возле офиса ветераны зажигают красно-черные файеры. На окнах появляются
надписи: «Штепа геть», «ОПЗЖ на нары», «2014 не пройдет».

\ifcmt
  pic https://gdb.rferl.org/d293884a-1fb5-4e3a-b568-089431e430d1_w1597_n_r1_st.jpg
	width 0.4
	fig_env wrapfigure
\fi

\subsubsection{Гараж и начальник полиции}

Второй пункт назначения – место, где живет Павел Придворов, депутат горсовета
от ОПЗЖ. Павел Придворов призывал восстанавливать экономические отношения с
Россией, говорил о гражданской войне. Соседи судились с Придворовым из-за
капитального гаража, который он построил во дворе многоэтажного дома.

Неподалеку от дома, где живет депутат, участников автопробега встречает
начальник отделения полиции Славянска Станислав Загурский.

Ветераны включают Загурскому запись телефонного разговора с нецензурной бранью
и угрозами.

«Да мне не интересно, кто кому что сказал. Потом будем разбираться: есть
славянский отдел полиции. Поедем в отдел полиции, напишете заявление – и будем
в правовом поле разбираться со всеми вопросами, которые возникают», – сказал
участникам автопробега главный полицейский Славянска.

«Мы не хотим писать заявление, мы хотим публичного извинения человека, который
этой сказал», – ответил Станиславу Загурскому ветеран АТО Сергей Кошуков.

После общения с полицией ветераны зашли во двор многоэтажного дома, посмотрели
на окна квартиры и гараж Павла Придворова и поехали к дому Олега Берко.

\subsubsection{Потасовка и автомат}

Возле дома Олега Берко автопробег уже ожидали правоохранители из полиции охраны
и съемочная группа местного телеканала. Договорились, что на территорию
особняка для переговоров пройдут несколько ветеранов. По бокам у Берко были
предметы, похожие на два автомата. Во дворе дома он ударил рукой в голову
активного жителя Славянска Василия Хоменко, говоря, что тот провокатор.

После потасовки Берко ушел на переговоры с Сергеем Кошуковым и еще одним
ветераном.

\ifcmt
  pic https://gdb.rferl.org/5c3ce3e6-d288-470d-b6d7-c23d9537f5bc_w650_r1_s.jpg
  caption Олег Берко на переговорах с ветеранами АТО
\fi

Издалека было слышно, как предприниматель сказал: «Перед вами за «мнимые
атошники» я извиняюсь».

Потом Берко дают телефон, чтобы тот послушал запись разговора. Предприниматель
начинает что-то объяснять, но тут полиция охраны закрывает металлическую
калитку.

\subsubsection{Извинения}

Через несколько минут Олег Берко выходит за калитку к участникам автопробега.
Он здоровается и говорит, что к ветеранам АТО относится с большим уважением.

«Если я взял на себя что-то лишнее – я приношу извинения перед вами, приношу
извинения. Но прошу и ко мне тоже соблюдать какое-то уважение. Я плачу,
поверьте, очень большие налоги, чтобы с этих налогов те же «атошники» получали
зарплаты».

После нескольких минут разговора Олег Берко озвучил свою версию, при каких
обстоятельствах был записан тот самый разговор с ругательствами:

– Ребята, если я вас обидел, сгоряча наговорил – я прошу у вас прощения, –
сказал предприниматель.

– А в рот кого вы собирались? – спросил один из ветеранов.

– Ребята, да когда звонил мне человек, я подумал, что это начальник областного
штаба ЕС (партия «Европейская солидарность» – ред.). Когда я разговаривал, я
думал, что это «еэсник». Потому что я ненавижу этих «свинарчуков», – уточнил
Олег Берко.

Подвел итог это встречи ветеран АТО Геннадий из Доброполья.

«Извинения не принимаем. Давайте так: будут высказывания в сторону «атошников»,
будет неуважение в сторону «атошников», – нам, честно, надоело уже это все. Нам
терять уже нечего. Если мы соберем всех людей, – мало вам не покажется. Не
трогайте нас, относитесь к нам с уважением. Если они еще раз нас оскорбят, – мы
их сносим всех», – предупредил ветеран. После этого Олег Берко пожал руки
нескольким ветеранам.

\ifcmt
  pic https://gdb.rferl.org/860b4c66-a6a5-4c59-9641-d12e6c1da706_w650_r1_s.jpg
  caption Участники автопробега в Славянске
\fi

А на вторник в Славянске планируют продолжить заседание первой сессии
городского совета нового созыва.
