%%beginhead 
 
%%file 08_05_2023.fb.etnografo_donetz.1.chastyna_vii_2_mariupol_metalurgijnyj_zavod_nikopol_ta_zavod_providans
%%parent 08_05_2023
 
%%url https://www.facebook.com/etnografo.donetz/posts/pfbid023kYztWXGQmnDEMN348RrtZmxpNsdzmxdJCYYHFDs8HohRdGYBY15L4xfVKbeJ9Dul
 
%%author_id etnografo_donetz
%%date 08_05_2023
 
%%tags 
%%title Частина VII/2 - Маріуполь металургійний. Завод Нікополь та завод Провіданс
 
%%endhead 

\subsection{Частина VII/2 - Маріуполь металургійний. Завод Нікополь та завод Провіданс}
\label{sec:08_05_2023.fb.etnografo_donetz.1.chastyna_vii_2_mariupol_metalurgijnyj_zavod_nikopol_ta_zavod_providans}

\Purl{https://www.facebook.com/etnografo.donetz/posts/pfbid023kYztWXGQmnDEMN348RrtZmxpNsdzmxdJCYYHFDs8HohRdGYBY15L4xfVKbeJ9Dul}
\ifcmt
 author_begin
   author_id etnografo_donetz
 author_end
\fi

Частина VII/2

\#КольороваДонеччина
\#Маріуполь

Маріуполь металургійний. Завод Нікополь та завод Провіданс

Завод Нікополь-Маріупольського Гірничого та Металургійного

Товариства розташований на високому північному березі Азовського моря, за 5
верст від Маріуполя за 1,5 версти від станції Сартана. 

19 квітня 1896 року А. В. Ротштейн підданий Пруссії та Е. Д. Сміт із США
звернулися до уряду з проханням дозволити їм заснувати
\enquote{Нікополь-Маріупольське гірниче та металургійне товариство} для
розробки руд у районі Нікополя та інших корисних копалин у всій імперії, а
також для будівництва трубних та суднобудівних заводів у Маріуполі. Перше
обладнання для заводів було привезено пароплавом із США із Сієтлу. 13 лютого
1897 року трубний цех випустив першу продукцію.

Зважаючи на те, що завод розташований від міста на відстані близько 5 верст,
товариство було змушене побудувати для робітників досить значну колонію.

Спершу побудували 47 кам'яних одноповерхових будинків, критих черепицею; 4
двоповерхові будинки по 16 квартир у кожному (рахуючи окрему кімнату та кухню),
а також 6 кам'яних одноповерхових казарм для не\hyp{}одружених робітників. Біля
кожного будиночка розбитий садок і влаштований загальний великий бульвар.

1899 року неподалік \enquote{Нікополя} запустили бельгійський завод
\enquote{Провіданс}.  У порту часто були затримки+ оплата роботи порту.

Тому було вирішено на р. Кальміус, за 6 верст від заводу, поглибивши річку на
14 ф. до моря, влаштувати свій власний порт і з'єднати його з заводом
залізничною гілкою. Ціна руди знизилася з 4.7 коп. до 3,7 коп. з пуду, тобто на
одну копійку. Поглиблення проводилося допомогою 2 драг, доставлених із Бельгії
морем. 

У березні 1920 року два заводи об'єднали в один. І зараз він відомий як
Маріупольський меткомбінат Ілліча.
