% vim: keymap=russian-jcukenwin
%%beginhead 
 
%%file 05_04_2021.fb.smith_helg.1.kiev
%%parent 05_04_2021
 
%%url https://www.facebook.com/groups/story.kiev.ua/permalink/1633652560164872/
 
%%author 
%%author_id 
%%author_url 
 
%%tags 
%%title 
 
%%endhead 

\subsection{это был мой Родной и добрый Киев}
\label{sec:05_04_2021.fb.smith_helg.1.kiev}
\Purl{https://www.facebook.com/groups/story.kiev.ua/permalink/1633652560164872/}

В незабываемые времена детства, когда открывалась речная навигация, а весна
стремительно набирала силу, вступая в свои законные права, мама водила меня на
работу к своей маме, моей бабушке, которая служила матросом на малом судне,
обслуживающем круизные лайнеры. Впервые на борт я поднялся в возрасте
трёх-четырёх лет. Это были невероятные впечатления, подъем на «настоящий
корабль» по качающемуся на волнах трапу. Разве такое забудешь? Наплыв ярких
эмоций переполняли незабываемыми впечатлениями юного моряка, как-то
по-особенному запомнился первый вдох, когда мои легкие впервые наполнились
пропитанным соляркой воздухом, запах которого по сей день ассоциируется с
чем-то родным и близким. 

\ifcmt
  pic https://scontent-amt2-1.xx.fbcdn.net/v/t1.6435-9/169062240_3912260678835058_8896785963236904610_n.jpg?_nc_cat=102&ccb=1-3&_nc_sid=825194&_nc_ohc=iHuC6iUTDR0AX8ya4uD&_nc_ht=scontent-amt2-1.xx&oh=ace25a7c35725b3690938dd0e2ac40fe&oe=6090017F
  width 0.4
\fi

В те далекие 80-е киевский речной транспорт находился на пике своего развития.
Большие и малые суда то и дело скользили в разные стороны по Днепру, создавая,
как казалось тогда, бесконечный речной трафик. Вдоль пристаней стояли большие
белоснежные красавцы — круизные теплоходы. Ракеты и метеоры, обеспечивающие
транспортный пассажиропоток по всей акватории Днепра и прилегающей к ней
территории Киевской области, отчаливали от пристаней каждые 15 минут. Большие
баржи и сухогрузы гордо проходили мимо зелёно-златоверхих холмов. Волны
игривыми блесками, переливая, отсвечивали лучи весеннего солнца, раздражая
незащищенные глаза. Все это создавало впечатления чего-то сказочного и
неповторимого, что ещё больше наполняло атмосферу тех счастливых дней.

За всем этим я любил наблюдать с борта «бабушкиного корабля», выполняющего свою
рутинную работу по обслуживанию больших круизных судов.

Я ловлю себя на мысли, что тогда было все по-другому, это был мой Родной и
добрый Киев, только в другой жизни, с заботливыми и веселыми близкими людьми,
которых, к сожалению, уже нет рядом...

Вот таким было детство маленького горожанина, выросшего под зелёными склонами у
берегов могучего и, в то же время, ласкового дедушки Днепра.

Автор Коваль Олег

Фото Жак Дюпакье


