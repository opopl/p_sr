% vim: keymap=russian-jcukenwin
%%beginhead 
 
%%file 24_12_2020.fb.ukc_ukrainianculturalcenter.1.skovoroda_year_2021
%%parent 24_12_2020
 
%%url https://www.facebook.com/UKC.UkrainianCulturalCenter/posts/2751450525120539
 
%%author 
%%author_id 
%%author_url 
 
%%tags 
%%title 
 
%%endhead 
\subsection{2021 в Україні - рік Григорія Сковороди!}
\Purl{https://www.facebook.com/UKC.UkrainianCulturalCenter/posts/2751450525120539}

\ifcmt
  pic https://scontent-mad1-1.xx.fbcdn.net/v/t1.6435-9/132891902_2751450485120543_3581447399995558370_n.jpg?_nc_cat=107&ccb=1-3&_nc_sid=8bfeb9&_nc_ohc=i9167UZMu2oAX-3UGJq&_nc_ht=scontent-mad1-1.xx&oh=762b96574746ef4f12486d69d7740be9&oe=6095547C
\fi

Верховна Рада 280 голосами ухвалила постанову "Про відзначення 300-річчя від
дня народження Григорія Савича Сковороди" (№4084).

Документом пропонується урочисто відзначити на державному рівні 300-річчя від
дня народження Григорія Сковороди.

Як зазначається у пояснювальній записці, пропонується рекомендувати Кабінету
Міністрів та відповідним міністерствам провести організаційні роботи з
підготовки та проведення заходів щодо відзначення на державному рівні 300-річчя
від дня народження Сковороди.

Держкомітету телебачення і радіомовлення України пропонується організувати
тематичні теле- і радіопередачі, присвячені життю і творчості Григорія
Сковороди, та забезпечити висвітлення через засоби масової інформації заходів,
що проводитимуться у зв'язку з відзначенням 300-річчя від дня його народження.

«Укрпошті» запропоновано видати серію поштових марок, присвячених 300-річчю від
дня народження Сковороди.

Окрім того, пропонується рекомендувати Національному банку України виготовити і
ввести в обіг ювілейну монету, присвячену 300-річчю з дня народження Григорія
Сковороди.

Місцевим державним адміністраціям та органам місцевого самоврядування
запропоновано розробити план заходів із відзначення цієї дати.

\verb|#УКЦ|
