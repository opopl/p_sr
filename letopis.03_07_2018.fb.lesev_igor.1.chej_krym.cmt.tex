% vim: keymap=russian-jcukenwin
%%beginhead 
 
%%file 03_07_2018.fb.lesev_igor.1.chej_krym.cmt
%%parent 03_07_2018.fb.lesev_igor.1.chej_krym
 
%%url 
 
%%author_id 
%%date 
 
%%tags 
%%title 
 
%%endhead 
\subsubsection{Коментарі}

\begin{itemize} % {
\iusr{Михаил Подоляк}

просто блестяще, Игорь... все по полочкам. По России и по Украине. Очень
прагматично и жёстко... если хочешь все понять о Крыме, достаточно этого твоего
текста)...

\begin{itemize} % {
\iusr{Игорь Лесев}
думаю, понимать более детально начнем, когда начнутся независимые судебные процессы

\iusr{Станислав Бочкур}
\textbf{Игорь Лесев} А есть на то шансы?

\iusr{Игорь Лесев}
\textbf{Станислав Бочкур} конечно. После иррациональных вопросов "чей Крым" обязательно появятся вопросы уголовного характера - "кто виноват". А потом и "что делать". Чернышевский ведь тоже наш)
\iusr{Станислав Бочкур}
\textbf{Игорь Лесев} Да просто сдаётся мне, что ваше высшее руководство поголовно настолько замазано в этом просёре, что независимого процесса провести не даст. Назначат пару козлов отпущения, скормят толпе и за старое возьмутся.

\iusr{Игорь Лесев}
\textbf{Станислав Бочкур} все так. Катынский расстрел тоже ждал своей правды много лет. А так да, виноватых у нас никогда нет. Ну, кроме Путина, естессна

\iusr{Матвей Кублицкий}
\textbf{Игорь Лесев} заметь: по Катынскому расстрелу никто не наказан и даже не обозначен как виновный. я про физ лиц. По Крыму тоже не будет виновных. да и по всей нынешней катавасии не будет. есть востребованность народа на такое судилище, но нет политической необходимости у правящих элит

\iusr{Станислав Бочкур}
\textbf{Matvey Kublitsky} Ну, "стрелочника" скормить могут для отвода глаз...
\iusr{Игорь Лесев}
\textbf{Матвей Кублицкий} так а кого уже могли наказать через почти 50 лет после событий?

\iusr{Матвей Кублицкий}
\textbf{Игорь Лесев} хотя бы попытаться выяснить виновного. я это просто к тому, что крайнего после правления пороха не найдут. для вида посуетятся - создадут, так сказать, инфоповод. и затихнут.... и на любые претензии скажут - этож народ крыма виноват а не мы. они ж голосовали

\iusr{Матвей Кублицкий}
\textbf{Игорь Лесев} да и предъявлять то откровенно говоря нечего

\iusr{Игорь Лесев}
ну да, просто расстреляли двадцать тысяч военнопленных, а так - сущий пустяк
\iusr{Матвей Кублицкий}
\textbf{Игорь Лесев} времена такие были. до этого поляки около ста тысяч русских пленных сгнобили. бесконечная история польско-русских претензий.... а про предъявлять нечего - я про историю с крымом. там реально юридически некому и нечего предъявить

\iusr{Игорь Лесев}
\textbf{Матвей Кублицкий} даже если бы поляки сгнобили несколько миллионов красноармейцев, считаю, что расстрел военнопленных - это гнусность... это, кстати, наша общая история, одна из самых гнилых ее страничек

\iusr{Матвей Кублицкий}
\textbf{Игорь Лесев} не спорю. гнусность. тогда вообще жизнь кровавой была. в результате тупой революции, откинувшей страну на сто лет назад. вот сталин кровью и наверстывал

\iusr{Антон Кораблев}
\textbf{Игорь Лесев} ну да, типично. Наших убили (притом зверски) 100.000- нормально. Мы хлопнули 20.000, какой вывод? Платить и каяться.
Знаешь что возьми уже поедь в Польшу и покайся.

\iusr{Игорь Лесев}
\textbf{Антон Кораблев} а мы что, на "ты" уже?

\iusr{Антон Кораблев}
\textbf{Игорь Лесев} да, тк я поляк. На колени, смерд. Кайся

\iusr{Игорь Лесев}
давно дурака никакого не банил, вот первый зашел

\iusr{Матвей Кублицкий}
\textbf{Игорь Лесев} оставь... все ж поржать

\iusr{Игорь Лесев}
я за чистоту сарая

\iusr{Матвей Кублицкий}
\textbf{Игорь Лесев} ну если с такой точки зрения....

\end{itemize} % }

\iusr{Алексей Нечаев}

Интересное мнение, особенно в плане того, что для России и для Украины Крым -
это две разных истории. Но вот с последним тезисом относительно того, что
"просранное не возвращается" можно поспорить, так как всё та же Россия Крым
посырала не один раз...)

\begin{itemize} % {
\iusr{Игорь Лесев}

Родина российскости (не путайте с Россией) - 1991 год. Мы все из одного
советского теста вышли. И с точки зрения именно российскости, Крым Россия не
просирала, а очень даже смиренно приняла то, что все получили по факту в 1991
году. Поэтому, следует говорить о приобретении РФ, а не о "возврате". Это как в
футболе. Победы сборной СССР автоматически ведь не плюсуются сборной России.

\iusr{Алексей Нечаев}
\textbf{Игорь Лесев}, вот кстати ФИФА плюсует советские достижения сборной России, иногда с пометками, иногда без, но плюсует.

\iusr{Алексей Нечаев}
\textbf{Игорь Лесев}, заодно зашёл проверить - УЕФА тоже плюсует, включая победу сборной СССР на ЧЕ-1960.

\iusr{Игорь Лесев}

буду знать, но вообще, во всех ру-СМИ выход сбРФ в четверть-финал называют
"историческим", хотя он не может таким быть, хотя бы потому, что в 66-м СССР
заняли 4-е место и, естественно, проходили соперников в плей-оф... но это уже
детали, мы же не о футболе) 

\iusr{Алексей Нечаев}
\textbf{Игорь Лесев}, а это следствие того, что у каждого отдельно взятого россиянина своё представление о прекрасном, о 1991 годе, о Крыме, о достижения советского футбола, и так далее.
Я тут за 3,5 года насмотрелся на эту дичь, которая мешает и им самим, и нам, пытающимся хоть как-то наладить взаимодействие с Россией.

\iusr{Матвей Кублицкий}
\textbf{Алексей Нечаев} а меня заинтересовало: просирало не один раз. Вы про ВОВ и гражданскую? других вариантов не помню

\iusr{Алексей Нечаев}
\textbf{Матвей Кублицкий}, это я про Крымскую войну и растянувшийся с 1954 по 1991 год слив Крыма.

\iusr{Матвей Кублицкий}
\textbf{Алексей Нечаев} крымская война середины 19 века не привела к потере крыма. а в 1954 году россия не теряла крым - просто передача региона с одного баланса на другой. фактическая потеря крыма 1991 год. кстати, вот что борька-алкаш точно проебал - это севастополь. он не входил в крым и всегда имел прямое подчинение москве

\iusr{Алексей Нечаев}
\textbf{Матвей Кублицкий}, опять таки, спорно. По итогам Крымской войны на какое-то время и сам Крым был потерян - его удалось вернуть лишь дипломатическими усилиями, но как хромую территорию; и возможность нормального функционирования флота аж до, если не ошибаюсь, 1871 года. А это приличный срок.

\iusr{Матвей Кублицкий}
\textbf{Алексей Нечаев} крым по итогам войны забрали полностью. а вот с флотом - да, проблемка была. да и еще куча разных ущемлений. в общем, серьезная тема для исследования

\iusr{Sergei Tatarintsev}

По итогам Крымской войны Россия в территориальном плане утратила в пользу
Молдавского княжества до 1878 г. Буджак (юг Бессарабии и выход к дельте Дуная).
Более, кажется, ничего.

\end{itemize} % }

\iusr{Gavrilova Irina}
Никто виновных не накажет.. ибо виноват ун ас Путин

\iusr{Марина Прохорова}

Когда нынешняя дэмократычна влада внезапно запретила ко ввозу Записки княгини
Дашковой, я не просто удивилась (не верю я, что ницоеподобные читали Дашкову)),
но даже решила проверить одну гипотезу, благо в электронке эта книга есть в
свободном доступе.

И действительно, слово "Крым" в Записках княгини встречается. Ровно один раз. Но какой!

"В июле мой сын возвратился из действующих войск, посланный оттуда с депешами,
оттуда возвестившими об окончательном признании подданства Крыма Русской
империи".

Вот и всё))

\iusr{Матвей Кублицкий}

повредничаю: двойка за знание индийско-пакистанского вопроса @igg{fbicon.face.tears.of.joy}{repeat=3} . Пакистан не
отнимал в 40ые у Индии ничего. это два государства, появившихся одновременно в
1947 году на обломках английской колонии. Просто при разделе страны на
буддийскую и мусульманскую части неувязочка с границами вышла

\begin{itemize} % {
\iusr{Игорь Лесев}
слабо вредничаешь. Азад-Кашмир как раз и был создан в 47-м во время первой индо-пакистанской. Гугл в помощь)

\iusr{Матвей Кублицкий}
она - эта первая война и была разделом пакистана и индии. До 47 года этих государств не существовало

\iusr{Матвей Кублицкий}
\textbf{Игорь Лесев} хотя близко к твоей статье. Наши государства тоже практически только что появились

\iusr{Игорь Лесев}
так а в чем претензии? я же написал в тексте - "с 40-х хапанул". Или 1947-й - это не 40-е?

\iusr{Матвей Кублицкий}
не хапанул - а отделился. полностью отделился.

\iusr{Игорь Лесев}
а Кашмир по разделу де-юре был закреплен за индусами не полностью отделился, а только горная северо-восточная часть. Учи географию, студент)

\iusr{Матвей Кублицкий}
\textbf{Игорь Лесев} пошел учить географию.
\end{itemize} % }

\iusr{Евгений Водовозов}

Просранные немцам полмира полностью вернулись на место в 45, включая просранные Судеты.
Просранная Прибалтика вернулась сама к себе.

\begin{itemize} % {
\iusr{Игорь Лесев}

вы упорно повторяетесь в своих одних и тех же абсолютно нерабочих примерах.
Судеты стали возможными только в результате полного военного поражения
Германии. Прибалтика не потеряла своей субъектности при включении в Союз, а
сама конституция СССР предполагала выход республик. А других примеров у вас
нет, потому что плохо историю знаете


\iusr{Евгений Водовозов}

У вас было написано безаппеляционно - "не возвращается", без упоминания каких
либо причин или условий. Я привёл пример, что возвращается. А по поводу полного
поражения Германии - я думаю году так в 43 немцы и не подозревали о нём. Так
что всё возможно и сейчас...

А насчёт "Прибалтика не потеряла своей субъектности" и "конституция СССР
предполагала выход республик" - это ж вы несерьёзно, я надеюсь).

Какая субъективность и какая конституция в совке?

\iusr{Игорь Лесев}

Первое. Если Россия потерпит глобальное военное поражение в войне, тогда она
потеряет не только Крым, но и еще несколько миллионов кв. км. территории. Это
чтобы лично вам утешить. Правда, далеко не факт, что Крым непременно победители
вернут Украине. Но это уже такое. Второе. Субъектность - это уровень союзных
республик, который той же Украине позволял быть членом-основателем ООН и иметь
даже свой МИД. Во времена СССР. Эстония, Латвия и Литва имели свои национальные
республики со столицами, латентным продвижением умеренного национализма и
прочими няшками. Идеологическое падение Союза автоматически давал шанс этим
республикам выйти из СССР. Не повезло, кстати, Карелии, которая когда-то была
16-й республикой, а затем ее субъектный статус был понижен. По этому же
принципу независимыми стали республики бывшей Югославии и произошел распад
Чехословакии. Но здесь нет "возврата к старому". В мире такого было полно.
Египет и Сирия когда-то были несколько лет одним государством. Сенегал и
Гамбия, тоже. Была конфедерация Центральноамериканских стран. Но если кто-то у
кого-то что-то захватил и мир это схавал - как и произошло с Крымом - возврата
к обратному уже нет.

\iusr{Игорь Лесев}

а вообще, общее оздоровление общества и страны начнется, когда вы и такие как
вы - я так понимаю, украинские патриоты, - отпустите фантомные боли по Крыму и
начнете ставить другие вопросы. Как так произошло, что украинская элита
безвольно отдала целый регион? Кто конкретно в этом виноват? И почему жители
отдельного региона в своей массе с радостью поменяли прописку?

\iusr{Ferg Kamusson}

Этот альтернативный повторяет давно поросшую мохом байку про "Судеты".
Поскольку собственных мозгов нет, ему трудно логически осмыслить сказанное.
Если бы Гитлер ограничился воссоединением с Судетами, Мемелем, Австрией - он
остался бы в истории рядом с Бисмарком, который тоже в свое время объединял
Германию железом и кровью. Но поскольку целью Гитлера было не воссоединение
германской нации, а завоевание "жизненного пространства"- с начала 1939 года
пошли захваты земель, никакого отношения к Германии никогда не имевших. Что не
могло не породить большую войну. Есть ли тут какие-то аналогии с возвращением
Крыма - может сам судить каждый сколько-нибудь разумный человек.

\iusr{Евгений Водовозов}

Как всегда - главные русские патриоты сидят в Израиле и Штатах, оттуда легче
родину любить)

А далее попробуем заменить в оригинальном тексте некоторые слова:

Если бы путин остановился на Чечне..., или Молдавии..., или Грузии..., или
Крыме.., может быть Донбассе?

\iusr{Евгений Водовозов}

Игорь, ну во первых мир пока формально не схавал аннексию, о чём говорят
специальные пакеты санкций, постоянно продлевающиеся. Конечно это больше похоже
на отмазку, но и говорить что санкции совсем уж не действуют тоже не совсем
честно - иначе бы великий и ужасный не пытался их прекратить.

А по поводу патриотизма и на что обратить внимание - конечно наша нынешняя
власть, это проблема, как и любая - но не главная. Главная проблема рядом, это
странно не замечать.

И конечно я уверен, что были бы к примеру вы у власти - то и Крым бы отвоевали
с несуществующей армией, и Донбасс бы давно вернули, окончательно наваляв
захватчикам

\iusr{Матвей Кублицкий}
\textbf{Евгений Водовозов} 

как Путин мог остановится на Чечне если это была внутрироссийская война? Грузия
и Молдавия - конфликты там начались еще когда Путин был офицером КГБ в
Германии. Он просто поддерживает статус кво в этих странах. Про Донбасс уже
надоело... все умные знают что пока Киев не сядет за стол переговоров с
Донбассом - мира не будет. И тот же Путин уже не раз говорил об этом

\iusr{Евгений Водовозов}

Ну, с Молдовой я может хватанул, путин там не причём, но Россия поучаствовала.
А вот Грузия-Осетия - это 2008 год, путин во всей красе

\iusr{Ferg Kamusson}

В общем, помимо потуг припаять Путину инициативу битвы на Калке и рассуждений о
порочном месте жительства другого человека - возразить особо нечего.

Если б альтернативный был способен к собственному мышлению, то он вспомнил бы,
что признанную после войны систему права, не позволяющую изменять границы
государства без его согласия, уничтожила не Россия, а как раз "империя добра"
на Балканах. Она ухитрилась даже протащить решение "международного суда", что у
государства, члена ООН, нет права препятствовать собственному расчленению. Но
поскольку альтернативный есть альтернативный, обсуждать абстракции не стоит.

\iusr{Всеволод Саксонов}
\textbf{Евгений Водовозов} простите, а что бы вы сделали на его месте, если напали на ваших, законно размещённых миротворцев? Умыли бы руки?

\iusr{Ferg Kamusson}

Уж не говоря о том, что и осетины и абхазы вели против грузинской мини-империи
свои собственные национальные войны. В 90-е Россия, сама раздерганная и
плывущая по течению, в эту свару прямо не вмешивалась. Когда же в 00-е пришел
откровенный агент влияния Мишико и начал провоцировать войну - решили
"преподать урок". Кстати, Абхазия и Юго-Осетия признаны как независимые страны,
к России их никто не присоединял. Так что аналогий с Германией опять не
проскальзывает.

\end{itemize} % }

\iusr{Елена Несветайлова}
Хорошо.

\begin{itemize} % {
\iusr{Игорь Лесев}
старался

\iusr{Елена Несветайлова}
Чувствуется. Молодец.
\end{itemize} % }

\iusr{Дмитрий Коломийченко}

Всё так кроме того, что вы Миллера не читали и не слушали по вопросу почему
российскойсть, а не русскость выбрали.

\end{itemize} % }
