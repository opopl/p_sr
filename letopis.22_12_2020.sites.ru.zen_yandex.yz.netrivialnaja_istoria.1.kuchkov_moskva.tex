% vim: keymap=russian-jcukenwin
%%beginhead 
 
%%file 22_12_2020.sites.ru.zen_yandex.yz.netrivialnaja_istoria.1.kuchkov_moskva
%%parent 22_12_2020
 
%%url https://zen.yandex.ru/media/id/5f6a1d4eaddbe610281941b4/pochemu-kuchkov--eto-vsetaki-moskva-esce-raz-o-berestianoi-gramote-xii-veka-5fe1bf4db681cf2cacd78f43
 
%%author 
%%author_id yz.netrivialnaja_istoria
%%author_url 
 
%%tags moskva,istoria,beresta_gramoty
%%title Почему Кучков – это все-таки Москва? Еще раз о берестяной грамоте XII века
 
%%endhead 
 
\subsection{Почему Кучков – это все-таки Москва? Еще раз о берестяной грамоте XII века}
\label{sec:22_12_2020.sites.ru.zen_yandex.yz.netrivialnaja_istoria.1.kuchkov_moskva}
\Purl{https://zen.yandex.ru/media/id/5f6a1d4eaddbe610281941b4/pochemu-kuchkov--eto-vsetaki-moskva-esce-raz-o-berestianoi-gramote-xii-veka-5fe1bf4db681cf2cacd78f43}
\ifcmt
	author_begin
   author_id yz.netrivialnaja_istoria
	author_end
\fi

В начале декабря я написал статью про древнейшее письмо с упоминанием Москвы:\Furl{https://zen.yandex.ru/media/id/5f6a1d4eaddbe610281941b4/drevneishee-pismo-s-upominaniem-moskvy-5fc9267feb95a53734b106bf?integration=site_desktop&place=layout}
новгородскую берестяную грамоту третьей четверти XII века. Летописи известны
лишь в более поздних копиях, так что берестяная грамота №723 – это единственный
текст XII века о Москве, написанный рукою современника. Кроме того, грамота
интересна тем, что называет Москву Кучковым. Это название ранее встречалось
лишь в одном пассаже из Ипатьевской летописи и косвенно подтверждает, что
легенда о первом владетеле Москвы – боярине Кучке – имеет историческую основу.

%\ifcmt
%pic https://avatars.mds.yandex.net/get-zen_doc/1535103/pub_5fe1bf4db681cf2cacd78f43_5fe1c056230da257b74c6709/orig
%caption Грамота №723. Прорисовка (gramoty.ru)
%\fi

Признаться, написав статью, я ожидал, что прилетят такие комментарии:

– И что? Что нового в вашей грамоте с упоминанием Москвы? Про нее знают уже
тридцать лет. Тоже мне, сенсация!

Такой (признаться, вполне справедливой) претензии не появилось. Зато возникло
множество других. Фоменковцев, давно «пробивших дно», упоминать не буду.

Кроме них пришли «знатоки древнерусского языка»:

– Текст грамоты переведен неправильно! Правильно – так-то и так-то.

А также «историки»:

– Автор из пальца высосал сенсацию! Кто может доказать, что Кучков – это именно
Москва? Если автор письма «пошел в Кучков», то это, наверное, какая-то
деревенька рядом с Новгородом. Зачем вообще ездить в Москву – новгородцу?

Я должен всех их разочаровать! Берестяных грамот (как и других источников по
истории Древней Руси) очень немного, и находят их нечасто. Они вдоль и поперек
истолкованы и переведены. Интерпретировать их заново имеет смысл только в том
случае, если вы являетесь узким специалистом по Древней Руси.

Все, что расскажет о берестяных грамотах неспециалист, будет или неправильно
или неново.

\begin{leftbar}
  \begingroup
    \em\Large\bfseries\color{blue}
		+ покланѧние ѿто душиль ко нѧсть шьль ти есьмъ кучькъву ажь ти хътѧ жьдати
				ааи ти нь хътѧ жьдати а оу ѳьдокь обруць ее водадѧ а свое възьму
  \endgroup
\end{leftbar}

Перевод:

\begin{leftbar}
  \begingroup
		\em\Large\bfseries\color{blue} «Поклон от Душилы Нясте (?). Я пошел в
				Кучков. Хотят ли ждать (не указано, кто) или не хотят (другой вариант:
				Захочу ли я ждать или нет), а я у Федки, отдав ей браслет, свое
				возьму».
  \endgroup
\end{leftbar}

Так вот! Моя позиция ненова, позиция оппонентов – неправильна. Перевод грамоты
я взял с сайта gramoty.ru, интерпретацию – из монографии двух академиков: В.Л.
Янин, А.А. Зализняк. Новгородские грамоты на бересте. Из раскопок 1990-1996
годов. М., 2000. Не я придумал, что Москва равна Кучкову.

Почему версия двух академиков мне кажется вполне убедительной, а возражения –
неубедительными? Рассмотрим эти возражения по пунктам.

1. Что новгородцы делали в Москве? Понятно что. Новгород – город торговый, и
его жители часто посещали по своим делам другие города Древней Руси. Кроме
Москвы в берестяных грамотах упоминаются Киев, Переславль Южный, Полоцк,
Смоленск, Углич, Ростов, Ярославль. Москва стояла на торговом пути из Новгорода
в Рязань. Перевалочным пунктом на пути из Новгорода в Москву был Волок Ламский
(Волоколамск).

Кучков-Москва к 1170-м годам была уже самым значительным поселением в среднем
течении Москвы-реки. Об этом говорят многочисленные упоминания Москвы в
летописном рассказе о междоусобной войне, которая вспыхнула в Северо-восточной
Руси после смерти Андрея Боголюбского. Москва в этот момент была уже городом.

\ifcmt
pic https://avatars.mds.yandex.net/get-zen_doc/3384412/pub_5fe1bf4db681cf2cacd78f43_5fe1bfe8b681cf2cacd8b267/scale_1200
caption Новгородский торг. Картина Аполлинария Васнецова (Public Domain, Wikimedia)
\fi

2. Недоумение читателей вызвала сама фраза «пошел в Кучков». Пошел? Неужели
пешком? В церковнославянском и древнерусском языках глагол «идти» был крайне
многозначен и подразумевал не только пешее передвижение (что очевидно и из
летописей). Впрочем, и пеший переход из Новгорода в Москву не исключен. Бедные
люди в прежние века путешествовали именно так. Вспомним, как Ломоносов с рыбным
обозом в разгар зимы прошел пешком тысячу километров из Холмогор в Москву…

3. Может быть, упоминаемый в грамоте «Кучков» – не Москва, а какое-то другое
поселение? В принципе, такого варианта исключить нельзя (хотя других Кучковых
нет в древнерусских источниках). Другое дело, велика ли эта вероятность?

В современной России есть несколько населенных пунктов с названием Москва.
Однако если вы услышите «еду в Москву» без уточнения контекста и в случайном
разговоре – велика ли вероятность, что речь идет о чем-либо, кроме поездки в
столицу России?

Есть множество типичных, широко распространенных названий населенных пунктов:
Великое, Иваново, Заречье, Спас, Борок, Починок и т. п. Их много десятков и
даже сотен. Понятно, как эти названия появились и почему они часты. Топоним
Кучков НЕ принадлежит к таким названиям. Среди 150 тысяч современных населенных
пунктов в России, кажется, только два Кучкова. Это неудивительно. Кучко (от
которого произошло название) – редкое, уникальное имя.

Поэтому крайне мала вероятность, что в новгородской грамоте идет речь о
чем-либо, кроме Кучкова-Москвы.
