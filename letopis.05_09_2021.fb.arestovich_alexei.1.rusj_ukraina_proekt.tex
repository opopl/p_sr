% vim: keymap=russian-jcukenwin
%%beginhead 
 
%%file 05_09_2021.fb.arestovich_alexei.1.rusj_ukraina_proekt
%%parent 05_09_2021
 
%%url https://www.facebook.com/alexey.arestovich/posts/4658749960855714
 
%%author_id arestovich_alexei
%%date 
 
%%tags future,identichnost',obschestvo,projekt,rusj,rusj.ukraina,strana,ukraina
%%title Что нужно понимать про мой проект «Русь-Украина»?
 
%%endhead 
 
\subsection{Что нужно понимать про мой проект «Русь-Украина»?}
\label{sec:05_09_2021.fb.arestovich_alexei.1.rusj_ukraina_proekt}
 
\Purl{https://www.facebook.com/alexey.arestovich/posts/4658749960855714}
\ifcmt
 author_begin
   author_id arestovich_alexei
 author_end
\fi

- Что нужно понимать про мой проект «Русь-Украина»?

1. Это моя личная инициатива и пока находится на уровне личной инициативы.
Хотя сторонников, а также людей и сообществ, которые независимо предлагают то же самое - немало. Ой, немало.)

2. Конечно же, ни смена названия государства сама по себе, ни даже победа над
Москвой в борьбе за первородство, ни являются исчерпывающими целями
предложения.

\ifcmt
  ig https://scontent-lhr8-2.xx.fbcdn.net/v/t39.30808-6/241423730_4658750274189016_5083545654889615676_n.jpg?_nc_cat=102&ccb=1-5&_nc_sid=8bfeb9&_nc_ohc=iaKNgzvIgCYAX-LFlgn&_nc_ht=scontent-lhr8-2.xx&oh=9d5d0891a567e8f09eba554b2cdb239a&oe=613D5433
  @width 0.5
  @wrap \parpic[r]
\fi

\obeycr
Нет - речь идёт о куда более важных, масштабных и принципиальных вещах.
Это проект грандиозного изменения нашей исторической судьбы.
Задача - исторического прыжка в будущее, разрыва с бледной немочью вечного провинциализма, вечного «не получится» и постоянного «да кому мы нужны», с первобытной замкнутостью мысли и чувства в пределах родного забора. 
Одним из символов проекта является князь Святослав - единственный настоящий геополитик в нашей истории, который не стеснялся рубить через всю геополитическую доску, сокрушать государства и переносить столицы.
Я хочу вернуть всем нам достоинство: себе, наших предкам и потомкам.
Достоинство игроков общеземного масштаба. 
Я хочу, чтобы мы мыслили дерзко и действовали программно.
Я предлагаю нацию, для которой нет невозможного, которая не знает слово «предел» в своём воображении и планах, действует мощно,  инновационно, широко, шагая фигурами через всю доску, готовую идти за пределы воображения: в науке, искусстве, политике, бизнесе, в космосе и на Земле.
Я хочу чтобы мы стали другими - такими какими мы были рождены на заре нашей цивилизации Руси. 
Такими, какими заслужила Украина тысячелетием своей истории: кровью, потом и талантами наших предков. Нашей с вами кровью. 
Переименование - один из двадцати и  далеко не самый важный (хотя и необходимый) пункт проекта. 
Важно другое.
Дополнить нашу идентичность маршевой ступенью прорыва.
Исторического, цивилизационного,  судьбоносного прорыва - который мы давно заслужили.
Разорвать бледную немочь кармы бескрылых, сжиравшую нас до сих пор.
Взлететь над собой, над Землей, над судьбой.
Оказаться сильнее рока и предначертанной нам кем-то упаднической траектории. 
Мы это можем. Мы это заслужили. Мы это сделаем.
Конечно, такие вещи готовятся основательно.
Конечно, действовать необходимо со строгим расчётом, планово и последовательно. 
Конечно, предстоит этап масштабного мифодизайна и борьбы за место под солнцем в новых условиях - настоящей цивилизационной драки за право на собственную историческую роль. 
Конечно, сопротивление будет: не только внешнее, в новую очередь - сопротивление внутренне: сопротивление узкого мышления и узкой индентификации, боящейся не то, что полёта - лишнего движения по земле. 
Оно уже есть.)
Вся несостоятельность этого сопротивления видна в его жалких возражениях на Проект.
Жалость заключается в том, что в этих возражениях нет ни программы, ни предложений, ни содержания - кроме «…как бы ничего не случилось» и «…давайте остаёмся догнивать в нашем привычном болоте», «…не нужно ничего менять». 
Этот тип самоидентификации обречён. 
Он технически не в состоянии выжить перед новыми вызовами времени и те, кто держится желания остаться в бледной немочи коротких горизонтов - враги и проекта, и Украины-Руси, они обрекают ее на новый виток исторических поражений. 
Именно из-за отказа мечтать, мыслить и летать. Действовать в историческом достоинстве. 
Мы не вытянем, не изменившись. Просто не потянем все ускоряющуюся цивилизационную гонку. 
Но мы - изменимся. Во имя прадедов, во имя правнуков, поднимется снова Русь - Украина. 
Прошу, помните - КТО мы. 
Тысяча поколений предков смотрит на нас с новой надеждой. Руський период, козацкий, все те, кто боролся за нашу собственное право и волю. 
Наша задача - поднять Украину в полный рост, зашагать широко и уверено. 
Полететь, как «Мрія» - вот наша задача. 
Наша главная идея - воля. Воля - как свобода для свершений и воля, как волевое движение души. 
Это - наше уникальное предложение в копилку общей цивилизации Земли.
Самое свободное общество. Самые свободные люди - вот, что мы должны построить. 
Воля и красота. Красота страны, людей, поступков, искусства и решений. Красота действий. 
Нам нужна республика талантов и героев. 
Мы - сможем.
И давайте начинать. 
\restorecr
——-
Мое включение по теме в эфир «Украинского радио»:
\url{https://youtu.be/yZww6jkxyqI}
