% vim: keymap=russian-jcukenwin
%%beginhead 
 
%%file slova.vokzal
%%parent slova
 
%%url 
 
%%author 
%%author_id 
%%author_url 
 
%%tags 
%%title 
 
%%endhead 
\chapter{Вокзал}
\label{sec:slova.vokzal}

%%%cit
%%%cit_pic
%%%cit_text
Во всем этом движняке прекрасно все. Напор, целеустремленность, отсутствие
нравственных и умственных барьеров. Не хватает только одного. Ощущения, что они
не одни на этом \emph{Вокзале}. Что вокруг здесь полно других людей с таким же
самым паспортом, как и у них. И что у этих других может быть иное мнение
%%%cit_comment
%%%cit_title
\citTitle{Бандеровцы при власти создали в Украине уютненькую Уганду}, 
Игорь Лесев, strana.ua, 13.06.2021
%%%endcit

%%%cit
%%%cit_head
%%%cit_pic
%%%cit_text
Итак, я выпускаю все то, чему полагается быть «за завесой». Начало моего
рассказа — вокзал. Мне было сказано явиться на такой-то \emph{вокзал} такого-то города
в такой-то стране, такого-то числа, в таком-то часу. Там за столиком будет
сидеть молодой человек, т. е. средних лет. Красивый, в полу-пальто с серым
мехом, мягкой шляпе. Я должен буду стать рядом с ним за общим столом и через
некоторое время спросить у него по-русски, есть ли у него спички. Если он
подаст мне спичечную коробку определенной марки, то это будет именно тот
человек, который мне нужен, и больше мне ни о чем заботиться не полагается.  Я
приехал на \emph{вокзал}, и все прошло очень точно. На углу стола сидел человек,
которого нельзя было не узнать по данному мне описанию. Я спросил спички, и он
подал мне их, улыбнувшись при этом добродушно и грустно, как улыбаются только
русские. Он был усталый, хотя молодой и неизможденный. Он давно устал и, должно
быть, навсегда.  Марка на коробке оказалась та самая, а усталый человек сказал
мне
%%%cit_comment
%%%cit_title
\citTitle{Три Столицы}, В. В. Шульгин
%%%endcit
