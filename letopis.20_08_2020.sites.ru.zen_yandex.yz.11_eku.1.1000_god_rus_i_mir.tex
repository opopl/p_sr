% vim: keymap=russian-jcukenwin
%%beginhead 
 
%%file 20_08_2020.sites.ru.zen_yandex.yz.11_eku.1.1000_god_rus_i_mir
%%parent 20_08_2020
 
%%url https://zen.yandex.ru/media/11ecu/1000i-god-rus-i-mir-5f3a41d44f2e4f5b77ddee5e?utm_source=yandex.zen&utm_medium=article&utm_campaign=5f8012d35dbc67260c3d0961
 
%%author 
%%author_id yz.11_eku
%%author_url 
 
%%tags 1000_god
%%title 1000-й год. Русь и мир.
 
%%endhead 
 
\subsection{1000-й год. Русь и мир.}
\label{sec:20_08_2020.sites.ru.zen_yandex.yz.11_eku.1.1000_god_rus_i_mir}
\Purl{https://zen.yandex.ru/media/11ecu/1000i-god-rus-i-mir-5f3a41d44f2e4f5b77ddee5e?utm_source=yandex.zen&utm_medium=article&utm_campaign=5f8012d35dbc67260c3d0961}
\ifcmt
  author_begin
   author_id yz.11_eku
  author_end
\fi

\begin{leftbar}
  \begingroup
    \em\Large\bfseries\color{blue}
Многие ждали, что в 1000-м году всё и закончится. Но не закончилось даже и в
2000-м. Есть даже большая вероятность, что и 2020-й переживем. Ну а пока до
наших насущных проблем оставалось ещё более десяти веков, мир оказался в 1000-м
году.
  \endgroup
\end{leftbar}

Москва. Как много в этом звуке... Но... до её основания еще долго. Не там весь
цвет политической жизни Руси пока.

Всего 12 лет прошло, с тех пор как наш правитель Владимир Святославич Русь
покрестил. Великому князю киевскому сейчас немного за 40. Он еще бодр и
деятелен. В этот год его дочь Премислава вышла замуж за венгерского князя
Ласло. Сыновья его растут и крепнут, скоро скоро грянет междоусобица. 

\ifcmt
pic https://avatars.mds.yandex.net/get-zen_doc/3985629/pub_5f3a41d44f2e4f5b77ddee5e_5f3d1d4734bdd8299c81a120/scale_1200
caption Князь Владимир с сыновьями беседует.
\fi

Владимир узнает, что умерла Рогнеда. Та самая, которая не хотела его в мужья,
которая считала, что он, сын служанки, не достоин дочери Рогволода. Ну а потом.
Да что уж теперь вспоминать. Последние годы мирно жила она там, где сегодня
белорусский Заславль. Крестилась. Молилась за себя, Владимира, и детей их.

Владимир после крещения и женитьбы на Анне Византийской укрепил отношения с
Константинополем. В 1000-м году дает войска для имперского похода в Армению и в
Болгарию. Он и сам лично идет в походе. Ну а как иначе? Дружина не поймет, если
в Киеве отсиживаться будет.

\ifcmt
pic https://avatars.mds.yandex.net/get-zen_doc/3950500/pub_5f3a41d44f2e4f5b77ddee5e_5f3d1f3342a7490ad3e6eb02/scale_1200
caption Европейское искусство тех лет
\fi

Ну а в Константинополе всё еще правит шурин Владимира. Император Василий II
получил прозвище Болгаробойца. При нем и крестился русский князь. Помимо выше
перечисленных походов еще в 1000-м году совместный с Венецией рейд в Далмацию.
Да и на востоке и юге неспокойно. В общем, дел у византийской власти
невпроворот. Только за счет союзных войск и успевает справляться.

Армению мы вспоминали. Не было там единого государства тогда. Были на той
территории небольшие царства, которые и между собой не всегда ладили. В Грузии
тоже были княжества, а в Тбилиси так и вовсе в 1000-м году существовал эмират.
Армянский эмират тоже некогда существовал, но к рассматриваемому периоду армяне
отстояли себя.

Эмираты тогда дело не редкое. Ислам, молодая религия, которой всего несколько
веков. Только набирает мощь, только проверяет возможные границы. 1000-й год -
период максимального расцвета Кордовского халифата, это территория современной
Испании. В те годы ключевым политическим игроком в регионе был Аль-Мансур. Его
десятки успешных военных походов против христиан привели к укреплению позиций
исламского государства на полуострове.

Говоря об исламском мире в тот год, нельзя не упомянуть про Авиценну. Да, этот
ученый, философ, врач, да много кто ещё, современник той эпохи. Но живет он
гораздо восточнее. Там, где сегодня Узбекистан.

\ifcmt
pic https://avatars.mds.yandex.net/get-zen_doc/1912454/pub_5f3a41d44f2e4f5b77ddee5e_5f3d1dd8849b7949ee7f01d2/scale_1200
caption Авиценна
\fi

Ему всего 20, но он уже призван лечить бухарского эмира. Доверить такое
почетное тело, явно могли не каждому. Вообще, в тот период исламский мир был на
подъеме в плане науки и культуры. Совсем недавно умер поэт Рудаки, в Каире Ибн
Юнус занимается сферической тригонометрией. В общем, стоит поговорить об этом
регионе в те века отдельно.

А пока вернемся в Европу, ведь мы не вспомнили про Священную Римскую Империю.
Там у нас императорствует Оттон III. Ну чем он славен? Прожил он всего 22 года,
был очень религиозным. Часто до фанатизма. Занимался истязанием плоти во
избежание греховных помыслов. Успел поставить двух Римских Пап из числа
доверенных ему лиц. Должен был жениться на Зое Константиновне Византийской, но
не успел. Его вместе со ставленным им Папой изгонят из Рима. Да и умер он
вскоре, и тем самым завершил Саксонскую династию.

\ifcmt
pic https://avatars.mds.yandex.net/get-zen_doc/3938527/pub_5f3a41d44f2e4f5b77ddee5e_5f3d1e1334bdd8299ca3b6f1/scale_1200
caption Оттон III. Именно таким вы себе его и представляли
\fi

Папа Сильвестр II ненадолго переживет императора, а тогда в 1000-м году он вел
активную борьбу с процветающими среди духовенства системой продажи должностей и
тайных сожительств высокопоставленных епископов с женщинами. Уж не за это ли их
с Оттоном не любили?

\textbf{"Почему вы умолчали про Торгейра Льосветнингагода???"} - будут писать
возмущенные читатели, не нашедшие в нашем опусе упоминания про Исландию.
Поэтому этот абзац специально для них.

А ещё мы очень много про какие государства не упомянули. Уж такой формат.
Захотите, ещё напишем. Много же интересного происходило в тот славный год
Средневекового Миллениума. 

\begin{itemize}
  \item 20 год
  \item 1420 год
  \item 1520 год
  \item 1620 год
  \item 1720 год
  \item 1820 год
  \item 1920 год
\end{itemize}

***Кстати, в тот период существовала Рашка. О её Великом жупане более позднего
периода, мы, кстати уже писали.
