% vim: keymap=russian-jcukenwin
%%beginhead 
 
%%file 13_11_2021.fb.kyshtymov_aleksandr.1.russkie_i_ukraincy.cmt
%%parent 13_11_2021.fb.kyshtymov_aleksandr.1.russkie_i_ukraincy
 
%%url 
 
%%author_id 
%%date 
 
%%tags 
%%title 
 
%%endhead 
\subsubsection{Коментарі}

\begin{itemize} % {
\iusr{Павел Резник}

Когда в дело влезает политика, всегда потом влезает язык, очень советую в Ютубе
беседа Вассермана и полиглота Дмитрия Петрова, кто хочет думать, посмотрите эту
передачу


\iusr{Tatjana Voronina}

Не понимаю одного: если украинцы не хотят быть родственниками, то почему
русские так настойчиво хотят этого? И так радуются, когда находится возможность
опереться на чей-то авторитет...

Ну хотят люди жить отдельно, зачем лезть к ним в родственники? Живут же
отдельно немцы и австрийцы, швейцарцы - французы - итальянцы - хотя там уже
тоже веками все перемешано. И никто не говорит, что мы более великие, потому мы
главные и отдайте нам то, что принадлежало нам когда-то... Последним таким был
Гитлер кажется...

\begin{itemize} % {
\iusr{Александр Кыштымов}
\textbf{Tatjana Voronina} 

немцы, австрийцы и швейцарцы живут как дружеские страны, без противопоставления
друг другу. Что же касается проекта Украины, то он предусмотрен как антипод
России и поэтому, пока Украина будет идти по этому пути конфликты неизбежны


\iusr{Maxim Gerasimenko}
\textbf{Tatjana Voronina} 

К сожалению, в ситуации Украины с Россией сошлись два основных фактора -
синдром "младшего брата" и нужда местной элиты во внешнем враге, чтобы отвлечь
народ от катастрофической внутренней коррупции. (Есть, правда, и другие
факторы)


\iusr{Tatjana Voronina}

Ну вот объясните мне, даже если это проект ( хотя почему на допустить, что
просто страна, которая хочет самостоятельно и независимо жить в оговорённых
границах ), даже если как антипод России, - то зачем конфликтовать пт этому
поводу? Пусть живет, как хочет и как знает. Почему ей обязательно надо быть
похожей на Россию? Зачем это надо России? Почему это ее так волнует? Ведь
действительно в самой России есть очень много мест, куда можно было бы
приложить усилия с пользой для своих граждан, разве нет?

\iusr{Tatjana Voronina}
\textbf{Maxim Gerasimenko}

Я бы ещё, наверное, добавила «синдром старшего брата». Посмотрите на
комментарии. Хотя я и русская, но мне всегда стыдно и неприятно видеть, с каким
пренебрежением люди, считающие себя наследниками великой русской культуры,
называют граждан соседнего государства самыми унизительными кличками.

И эти, конечно, на добавляет желания соседям признать себя родственниками.

Я живу в Словакии и вижу, как словаки и чехи относятся друг к другу после
распада их общей страны. И тихо им завидую...


\iusr{Татьяна Пашаева}
\textbf{Tatjana Voronina} Наблюдаю за русскими и украинцами. Должна сказать, что украинцы не отстают, а даже опережают "москалей" в оскорблениях. Так что все хороши.

\iusr{Tatjana Voronina}
\textbf{Татьяна Пашаева}
А вам никогда не казалось, что они так защищаются?..

% -------------------------------------
\ii{fbauth.koreljakova_tatjana.rossia.hudozhnik}
% -------------------------------------

\textbf{Tatjana Voronina} 

у вас какая-то память короткая. Вам не кажется, что все 'клички' - это ответ на
десятилетия русофобии. Не вспомните клички, которые потоками лились сначала от
украинцев? За что боролись, на то и напоролись. И может начинать надо именно с
этого - свое поведение скорректировать. А то двойные стандарты - мы вам будем
дули крутить, а вы нам кланяйтесь в пояс.

\iusr{Alena Morgan}
\textbf{Татьяна Корелякова} а вас-то кто и где обозвать успел? Что вам украинцы вообще плохого сделали? Или по телеку "слова передали"?!

\iusr{Татьяна Корелякова}
\textbf{Alena Morgan} 

Так я думаю, что и вы лично ни от одного русского никаких оскорблений не
слышали!  @igg{fbicon.grin}  а только ваш телек вам передаеь и передает уже 30 лет. А вы хоть
одну передачу антиукраинскую найдите на российском ТВ раньше 2014? Вот то то
же.

\iusr{Татьяна Корелякова}
\textbf{Alena Morgan} 4 года когда?

\iusr{Татьяна Корелякова}
\textbf{Alena Morgan} ну так это своим соотечественникам и их художествам в ножки
поклонитесь. Я 30 лет в Московском регионе, уроженка Украины - ни разу, нигде
никогда не сталкивалась с унижением украинцев. И государство в котором жили не
надо поносить задним числом - это некрасиво.

\iusr{Alena Morgan}
\textbf{Татьяна Корелякова} 

звучит так, как будто это мы тонны грязи каждый день на российском ТВ выливаем.
Смешно. Вам ничего не говорят, потому что не знают, что вы \enquote{уроженка Украины}.
Видно, что вы никак себя даже и не ассоциируете. Представьте себе, когда вам в
лицо говорят что вашей страны не существует))

\iusr{Татьяна Корелякова}
\textbf{Alena Morgan} 

а вы как ассоциируете интересно? В какой форме? @igg{fbicon.grin} видимо, именно за это
неправильное проявление идентичности и получаете негатив.

\iusr{Alena Morgan}
\textbf{Татьяна Корелякова} я украинка и люблю свою страну. Вот вам закат с пожеланиями доброй ночи.

\ifcmt
  ig https://scontent-frt3-1.xx.fbcdn.net/v/t39.30808-6/255260481_10223336280509760_7726368349278552201_n.jpg?_nc_cat=107&ccb=1-5&_nc_sid=dbeb18&_nc_ohc=otQUH4h2UQMAX-70qTm&_nc_ht=scontent-frt3-1.xx&oh=eeef6dce38852fc052a46c8737a7c32b&oe=61A6DC94
  @width 0.4
\fi

\iusr{Татьяна Корелякова}
\textbf{Alena Morgan} по поводу ТВ - встречный вопрос - а кто льет по украинскому ТВ тонны грязи уже 30 лет?

\iusr{Alena Morgan}
\textbf{Татьяна Корелякова} до аннексии Крыма мы все, к сожалению, считали Россию братской страной. Это было огромной ошибкой.

\iusr{Татьяна Корелякова}
\textbf{Alena Morgan} и тем не менее по ТВ шли антироссийские передачи и массированно. Так что на себя надо смотреть.

\iusr{Alena Morgan}
\textbf{Татьяна Корелякова} приведите примеры таких передач.

\iusr{Татьяна Корелякова}
\textbf{Alena Morgan} вы сами можете погуглить.

\iusr{Alena Morgan}
\textbf{Татьяна Корелякова} я не знаю, про что вы вообще. Поэтому и спрашиваю. Что именно я должна гуглить? Ваши фантазии?))

\iusr{Татьяна Корелякова}
\textbf{Alena Morgan} 

ну фантазии не фантазии, а знаменитые скачки не в павильонах Мосфильма были
сняты. И в повестке каждого Майдана была тема противостояния с Россией. Вот
тоже удивительно почему? Вроде внутренние вопросы решали, а в итоге вся политика
строилась именно на противостоянии России. Я этому удивлялась, потому что до
2014 ниеакой украинской повестки не было в России нигде. И вы слепая? Вы не
видите что ли какой вред вашей любимой стране от того, что разорваны связи с
Россией? Кредиты, упадок производства и отток населения - и все это собственными
руками, бездумно и беспощадно. Зато кричитечто любите Украину - странная какая-то
любовь, от которой все разрушается.

\iusr{Анна Снегина}
\textbf{Tatjana Voronina} 

там сжигают и расстреливают людей как это делают чубатые, там пропагандируют
откровенный фашизм? Или уроки ВОВ ничему не учат?

\iusr{Юнна Мориц}
\textbf{Tatjana Voronina} 

А Вам не стыдно и приятно видеть и слышать, как скачут в\textbackslash на Украине с воплями
\enquote{Москаляку на гиляку!}, как там в школьном буфете стоит компот в кувшинах с
надписью \enquote{Кровь славянских младенцев}, как там напоказ выставляют консервные
банки с надписями о том, что в этих банках - мясо, тушёнка из людей Донбасса???
И не притворяйтесь, не врите, что Вы слышите об этом впервые!.. Всё Вы
прекрасно знаете.

\iusr{Максим Степанюк}
\textbf{Yunna Morits} 

на щет воплей я не поспорю, но остальное бред сивой кобылы. Меньше телек
смотрите, он на вас, россиян, имеет какое-то странное влияние.

\iusr{Максим Степанюк}
\textbf{Tatyana Korelyakova} 

странно, что не владея истинными фактами, Вы, россияне, так верите в свои
слова. Ни один майдан не был направлен на агрессию с Россией. Даже в 2013 не
было призывов убивать россиян. Боролись за вступление в Евросоюз, для уравнения
рыночных цен и привлечения инвесторов. А Россия испугалась, когда лояльного к
ней призидента выперли, за убийства по его приказу его не выдают. Почему
Россия, которая судя по Вашему, хотела мира, не выдала преступника? И напала
подло, как нацисты в 41. Это указывает на то, что Россия - агрессор, и ни чем
не отличается от Германии, периода Гитлера. Спросите у любого ветерана, сильно
ли они любили нацистов? Тогда не удивляйтесь всем кричалкам и остальному
происходящему после 14. Россияне нормально живут в Украине, никто не трогает
русскоговорящих, это выдумки ваших Пиз....Ду...нов. Российские ведущие уже
тонут во враньё, потому и соответственно такое отношение. Так что перестаньте
писать чушь

\iusr{Дммтрий Кулаков}
\textbf{Максим Степанюк}

Если нормально русские живут в Украине, то что же вы со своей демократической
властью не сядете за стол переговоров с Донбассом и не решите проблемы его
автономии, использования там русского языка наравне с государственным,
отчисления налогов Киеву и пр.?

\iusr{Дммтрий Кулаков}

А решаете вопросы авиабомбежками и обстрелами из Градов жилых кварталов Донецка
и Мариуполя?

% -------------------------------------
\ii{fbauth.stepanjuk_maksim.kiev.ukraina}
% -------------------------------------

\textbf{Дммтрий Кулаков} 

я задам Вам вопрос сначала, это будет и ответ на Ваш! Почему в 41 после
нападения нацистов на СССР и захвата половины государства, Сталин не сел за
стол переговоров с Гитлером и не договорился \enquote{полюбовно}? Почему выгнав с
територии СССР нацистов, их гнали до самого Берлина? Почему всех тех, кто
служил нацистам, судили как предателей, а не дали им возможности решить, в
какой стране жить? И не дали им территорию со своей автономией? У нас Крым был
автономным. А Донецкие не хотят автономии, а хочет передачи земли в
собственность РФ. А это уже нарушение территориальной целостности, а за
территорию будет воевать любой народ.

\end{itemize} % }

\end{itemize} % }
