% vim: keymap=russian-jcukenwin
%%beginhead 
 
%%file 16_01_2022.stz.news.ua.hvylya.1.anatomia_vraga
%%parent 16_01_2022
 
%%url https://hvylya.net/analytics/245483-anatomiya-vraga-chto-ukraina-nedoocenivaet-v-rossii
 
%%author_id savruckij_bogdan,news.ua.hvylya
%%date 
 
%%tags 
%%title Анатомия врага. Что Украина недооценивает в России
 
%%endhead 
\subsection{Анатомия врага. Что Украина недооценивает в России}
\label{sec:16_01_2022.stz.news.ua.hvylya.1.anatomia_vraga}

\Purl{https://hvylya.net/analytics/245483-anatomiya-vraga-chto-ukraina-nedoocenivaet-v-rossii}
\ifcmt
 author_begin
   author_id savruckij_bogdan,news.ua.hvylya
 author_end
\fi

\begin{zznagolos}
Россия является системным врагом именно украинской государственности. Причем,
врагом во всех предыдущих исторических периодах, включая современный.
\end{zznagolos}

\ii{16_01_2022.stz.news.ua.hvylya.1.anatomia_vraga.pic.1}

Современная Россия, наряду с нашей внутренней украинской, \enquote{аутентичной}
неконсолидированностью, общественным инфантилизмом, антисистемностью и
неорганизованностью, как государственная модель, безусловно, является нашим
настоящим внешним врагом. Прежде всего, Россия является системным врагом именно
украинской государственности. Причем, врагом во всех предыдущих исторических
периодах, включая современный.

Исходя из исторических фактов, имевших место в нашей истории, я считаю, что
важнейшим элементом в долгосрочной стратегии противостояния с РФ, является
\textbf{правильное понимание мотивации и особенностей противника}, с которым Украина
имеет дело. Кстати, в своей войне с Украиной, русские используют такое же
китайское правило - \enquote{для того, чтобы проблему понять, ее сперва нужно
правильно назвать}.

\textbf{О очень упрощенном понимании характера нашей гуманитарной войны с РФ за
идентичность рекомендую прочесть в моей статье на \enquote{Хвиле}, ссылку на
которую я обязательно оставлю для Вас в конце этого текста.}

(ССЫЛКА) 

\href{https://hvylya.net/analytics/242912-kakuyu-voynu-vedet-rossiya-s-ukrainoy-chto-nuzhno-znat-o-vtorzhenii-v-yanvare-2022}{%
Какую войну ведет Россия с Украиной: что нужно знать о «вторжении» в январе 2022, Богдан Савруцкий, hvylya.net, 04.12.2021%
}

В массовом сознании современных украинцев, бытует искаженное и примитивное
представление о нашем извечном противнике, которое мешает нам правильно понять
суть нашего противостояния с Россией и которое в самом же Кремле успешно
используют против нас.

В Украине, в силу общего государственного и общественного инфантилизма,
сознание социума перегружено \enquote{политической метафизикой} в отношении совершенно
осязаемых и грозных факторов. Все чаще дежурные \enquote{мольфары} разного калибра и
разного же уровня примитивизма, являются \enquote{спикерами} наших политических
программ и \enquote{политических прогнозов} о том, что, дескать, \enquote{Путин скоро помрет},
Россия - \enquote{это тьма}, это \enquote{дикая монгольская орда необразованных дикарей, а мы
тут все европейцы с магдебургским правом}, поэтому украинцы, как \enquote{абсолютный
свет}, в любом случае как то там победят \enquote{абсолютную тьму}. Иными словами, наши
извечные враги обязательно, сами собой растают, \enquote{як роса на сонці}. По другому
быть просто не может.

Ну, во-первых, в истории и реальной политике не бывает \enquote{абсолютного света}, как
и \enquote{абсолютной тьмы}, бывает лишь наша эффективная стратегия или наше поражение,
являющимися следствием чужой эффективной стратегии. Фейл эпохи Руїни конца XVII
века, глупое поражение проекта Мазепы в начале XVIII века и крах УНР в первой
половине XX века - это, прежде всего, более эффективная стратегия Москвы, а не
происки метафизических \enquote{темных} сил.

Наш детский миф о \enquote{свете}, который должен победить \enquote{тьму}, безусловно, является
наивным упрощением проблемы и инфантильным зашучиванием серьезности опасности,
которую на самом деле представляет собой Россия и ее внешняя политика

\ii{16_01_2022.stz.news.ua.hvylya.1.anatomia_vraga.1.substrat}
\ii{16_01_2022.stz.news.ua.hvylya.1.anatomia_vraga.2.triggery_starta}
\ii{16_01_2022.stz.news.ua.hvylya.1.anatomia_vraga.3.stratagema_vyzhyvania}
\ii{16_01_2022.stz.news.ua.hvylya.1.anatomia_vraga.4.sakralnost_vlasti}
\ii{16_01_2022.stz.news.ua.hvylya.1.anatomia_vraga.5.votchinnoje_soznanie}
\ii{16_01_2022.stz.news.ua.hvylya.1.anatomia_vraga.6.effektivnost_stratagemy}
\ii{16_01_2022.stz.news.ua.hvylya.1.anatomia_vraga.7.endogennye_sily_razvitia_strategii}
\ii{16_01_2022.stz.news.ua.hvylya.1.anatomia_vraga.8.predely_ekspansii}
\ii{16_01_2022.stz.news.ua.hvylya.1.anatomia_vraga.9.ukrainskie_stolpy_identichnosti}
\ii{16_01_2022.stz.news.ua.hvylya.1.anatomia_vraga.10.kult_vojny_i_panteon_pobed}
\ii{16_01_2022.stz.news.ua.hvylya.1.anatomia_vraga.11.itogi}


