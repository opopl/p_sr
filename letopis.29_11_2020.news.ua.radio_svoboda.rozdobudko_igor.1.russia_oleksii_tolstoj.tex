% vim: keymap=russian-jcukenwin
%%beginhead 
 
%%file 29_11_2020.news.ua.radio_svoboda.rozdobudko_igor.1.russia_oleksii_tolstoj
%%parent 29_11_2020
 
%%url https://www.radiosvoboda.org/a/30974372.html
 
%%author Роздобудько, Ігор
%%author_id rozdobudko_igor
%%author_url 
 
%%tags 
%%title Олексій Толстой (1817–1875) писав про західний вектор розвитку України та східний – Росії
 
%%endhead 
 
\subsection{Олексій Толстой (1817–1875) писав про західний вектор розвитку України та східний – Росії}
\label{sec:29_11_2020.news.ua.radio_svoboda.rozdobudko_igor.1.russia_oleksii_tolstoj}
\Purl{https://www.radiosvoboda.org/a/30974372.html}
\ifcmt
  author_begin
   author_id rozdobudko_igor
  author_end
\fi


  

(Рубрика «Точка зору»)

\begin{leftbar}
  \bfseries
Правнук українського гетьмана Розумовського, письменник Олексій Толстой багато
уваги у своїй творчості приділяв минулому та прийдешньому Росії та України,
намагаючись висвітлити у своїх творах все те, що відрізняє Україну від Росії.
\end{leftbar}

Слідом за своїм учителем в історичних питаннях та другом \textbf{Миколою Костомаровим,
Олексій Толстой} вважав, що давня Русь-Україна розвивалася за європейським
вектором розвитку, мала розвинену демократичну форму правління, коли влада
князя була суттєво обмежена народним вічем. Сучасна ж Толстому Росія перейняла
свій суспільний лад не від княжого Києва, а від деспотичної Золотої Орди,
звідки пішла в Росію жорстока авторитарна царська влада.

\ifcmt
pic https://gdb.rferl.org/A3CC6F76-23D8-432B-A9B7-DB6CB3F5811A_w650_r0_s.jpg
caption Нащадок українського гетьмана Розумовського письменник Олексій Толстой (1817–1875)
\fi

Якби мешканець України-Русі часів Володимира Великого раптом би опинився в
Москві Івана Грозного чи Петербурзі ХІХ століття, він ніколи б не визнав цю
державу спадкоємницею давнього Києва, вважав Олексій Толстой. Цю свою ідею він
яскраво висловив у баладі \verb|«Потік-богатир»|.

Молодий боярин \verb|Київського князя Володимира| на ім’я Потік засинає на княжому
бенкеті і прокидається в Москві Івана Грозного:

\begin{multicols}{2}
  \obeycr
Їде цар на коні, в зипуні із парчі,
А навколо кати, в руки взявши мечі, –
Його милість збираються тішити,
Там когось чи рубати, чи вішати.

І у гніві за меч свій вхопився Потік:
«Що за хан на Русі так сваволить?»
Але чує слова: «Це земний їде Біг,
Це наш батько вбивати нас зволить!»
І на вулиці, скільки було там товпи,
Воєводи, бояри, ченці і попи,
Мужики, що стояли навколо –
Перед ним повалилися долу.

І не може Потік ніяк вникнути в суть:
«Якщо князь він, чи вдягся в порфіру,
Що ж вони бородою так землю метуть?
Князів й ми шанували, та в міру!
Та і дійсно, чи справді я тут на Русі?
Від земного нас бога Господь всіх спаси!
Нам Святим Письмом велено строго
Визнавати небесного Бога!»

І питає зустрічного він молодця:
«Де тут, дядьку, збирається віче?»
Та на тім з переляку немає лиця:
«Цур мене, – каже він, – чоловіче!»
  \restorecr
\end{multicols}

\ifcmt
pic https://gdb.rferl.org/70817CF9-F28A-46C9-8813-44BA74DFB0FF_w1023_r0_s.jpg
caption Деспотизм правлячої верхівки Росії викривали й прогресивні художники. Ілля Рєпін (Ріпин) «Іван Грозний і син його Іван 16 листопада 1581 року»
\fi

\ifcmt
pic https://gdb.rferl.org/D6E4BD62-80A5-42B9-A9AB-5EE25D78D61B_w1023_r0_s.jpg
caption Василь Суриков «Ранок стрілецької страти» 
\fi

Від усього побаченого Потік знову губить свідомість та просинається вже у
сучасній Толстому Росії. Але і тут не бачить він нічого для себе корисного:

\obeycr
«Безліч різних буває на світі чудес!
Я не знаю, що значить якийсь там прогрес,
Та до справжнього руського віча
Вам, панове, ще довго стовбичити!»
\restorecr

\ifcmt
tab_begin cols=2
  caption Ілюстрації до сатиричної «Історії держави Російської...» Толстого. Художник В. Порфир’єв, журнал «Стрекоза», 1906 рік
  pic https://gdb.rferl.org/AFBD3D3C-6B83-4384-9CC6-5BFF1DDA4846_w650_r0_s.jpg
  pic https://gdb.rferl.org/0F54699F-8E31-4E21-B271-BE503D9513EE_w1597_n_r0_st.jpg
tab_end
\fi

Замислювався Толстой і над майбутньою долею України – чи має вона можливість
зберегти свій європейський вектор розвитку, чи навпаки, приречена назавжди
залишитися в сфері російського політичного впливу? Цій проблематиці присвячує
він іншу баладу, \textbf{«Змій Тугарин»}. Тут на бенкет до князя Володимира
потрапляє представник дикого Степу Тугарин, який пророкує князеві про таке
майбутнє:

\begin{multicols}{2}
        \obeycr
І він заспівав: «Не прошу нагород,
Послухай, відважна громадо!
Не маєте ви у війні перешкод,
Тремтів перед вами і сам Царгород –
Ой ладо, ой ладоньки-ладо!

Та рід ваш не вічно пишається тим,
Я знаю про це достеменно,
Обіймуть ваш Київ і пломінь і дим,
І ваші сини будуть дітям моїм
Тримати покірно стремено!»

«Народ ваш на вічах судитися звик,
Образи змиває з вас поле –
Та прийдуть нові дні на руський поріг,
І честь вам, панове, замінить батіг,
А віче – кагана сваволя!»

Продовжив співець: «Час настане такий –
Піддасться наш хан християнам,
Та шлях до свободи ваш буде важкий,
І землю збере із вас самий цупкий,
Та сам же на ній стане ханом!

Сидітиме він за князівським столом,
Як ідол сидить серед храму,
І спини він битиме вам батогом,
А ви йому стукати в землю чолом –
Ой, сраму дізнаєтесь, сраму!»

Та далі пророчить чужинець-співак:
«Ви зрадите навіть матусю!
Не буде у вас слова честі ознак,
І ви, наковтавшись татарщини всмак,
Її називатиме Руссю!

І рідну забудете ви старину,
І предкам великим на сором,
Самі проти себе почнете війну,
Сказавши: \zqq{Зведемо варягів в труну,
Обійми розкриєм обдорам!}»
        \restorecr
\end{multicols}

\ifcmt
pic https://gdb.rferl.org/4F4C9E3F-8E14-41D7-B379-7F62B191ABC0_w650_r0_s.jpg
width 0.4
caption Київський князь Володимир Великий. Малюнок Олексія Штанка 
\fi

Тут Толстой під «варягами» має на увазі європейські народи, а «обдорами»
називає східні, азійські племена. А князь Володимир у цій поезії Толстого,
покаравши Тугарина за його зухвалість, дає ординцям гідну відсіч, і ці слова
Володимира Великого, варто думаю прочитати і сучасним українцям, аби не
забували та були гідні слави власних предків.

\begin{multicols}{2}
        \obeycr
Сказав Володимир: «Ич, вигадав нам
Грозити бідою, потворо!
Щоб ми від Тугарина прийняли срам!
Щоб спини підставили ми батогам!
Обійми розкрили обдорам!

Тому не бувати! Живе руська Русь!
Татарська нам Русь не до ладу!
Я, друзі мої, перед вами клянусь,
Що вірю в Вітчизну, й за неї молюсь –
Ой ладо, ой ладоньки-ладо!

Якщо б вже над нею біда і стряслась,
Нащадки біду переможуть!
Буває, – промовив свят-сонечко-князь, –
Неволя примусить пройти через грязь –
Купатись в ній свині лиш можуть!

Подайте ж ту чару велику мою,
Яку я здобув в лютій січі,
Із ханом хозарським в запеклім бою, –
За руський наш звичай до дна її п’ю,
За давній наш суд і за віче!

Я п’ю за варягів, відважних, лихих,
Русі побратимів завзятих,
Ким славен наш Київ, а грек ким притих,
За синє п’ю море, яке здавна їх
З країн принесло тридесятих!

Сказав Володимир – і разом за ним,
Як плеск лебединого стада,
Як вихор, що з неба злітає, як грім,
Народ відповів: «Ми за князя стоїм!
Ой ладо, ой ладоньки-ладо!

Керує по-руськи він руський народ,
А хан хай мандрує до аду!
Якщо ж вже насуне година незгод,
Ми вірим, що Русь їх мине без пригод, –
Ой ладо, ой ладоньки-ладо!»

Почув Володимир, що кажуть кругом,
І в серці палає відрада,
Він вірить: незгоди минуть вихором,
І весело чути йому над Дніпром:
«Ой ладо, ой ладоньки-ладо!»

Підвівсь Володимир у колі бояр,
І встали посадники града,
Піднісся весь Київ, і молод і стар,
І чути далеко дзвін кованих чар –
Ой ладо, ой ладоньки-ладо!
        \restorecr
\end{multicols}

Цілком мої переклади українських поезій Олексія Толстого ви можете прочитати на
сайті «Літературної України».\Furl{https://litukraina.com.ua/2020/06/15/oleksij-tolstoj-pravnuk-kirila-rozumovskogo/}

\ifcmt
pic https://gdb.rferl.org/34DB0665-1F10-4601-90BC-B9C3C7FA828A_w1023_r0_s.jpg
caption Київ. Пам'ятник богатирю Іллі Муромцю часів України-Русі. Він часто згадувався в українському фольклорі. Разом із двома іншими українськими богатирями Добринею Микитичем та Олешком Поповичем оберігав рідну землю від ворогів. Один із поетичних творів Олексія Толстого присвячений й Іллі Муромцю
\fi

\paragraph{Автор - Ігор Роздобудько}


\ifcmt
pic https://gdb.rferl.org/4333B0F7-ECBC-43A6-9DBD-C957C0CF2CF9_cx0_cy17_cw0_w144_r5.jpg
width 0.3
fig_env wrapfigure
\fi
Народився у Москві. Історик, перекладач, член Малої Ради Громади українців
Росії. В українському русі Росії з 1990-х років. Колишній пресаташе
Українського історичного клубу в Москві. Автор творів про історію
української діаспори в Росії: «Стародубщина. Нарис українського життя
краю», «Східна Слобожанщина. Українці навколо України», «Донщина та
далі на Схід». 
