% vim: keymap=russian-jcukenwin
%%beginhead 
 
%%file 15_10_2021.fb.kazakov_igor.1.kotjata_bajden_rasskaz
%%parent 15_10_2021
 
%%url https://www.facebook.com/Robinzon448/posts/4503532673062982
 
%%author_id kazakov_igor
%%date 
 
%%tags biden_joe,jumor,poezia,rossia,stihi,ukraina,usa
%%title Да, котята удивили, но послушайте наш сказ. Есть про Байдена рассказ
 
%%endhead 
 
\subsection{Да, котята удивили, но послушайте наш сказ. Есть про Байдена рассказ}
\label{sec:15_10_2021.fb.kazakov_igor.1.kotjata_bajden_rasskaz}
 
\Purl{https://www.facebook.com/Robinzon448/posts/4503532673062982}
\ifcmt
 author_begin
   author_id kazakov_igor
 author_end
\fi

Сегодня у моего товарища родился политический шарж на актуальную тему. Пришлось
на него ответить...

Борис:

А у нас на кухне газ, а у вас? А у нас сегодня Путин перекрыл газопровод - вот,
а у вас? А у нас сегодня кошка родила вчера котят, котята выросли немножко но
жить без газа не хотят.

Игорь:

Да, котята удивили, но послушайте наш сказ.

Есть про Байдена рассказ.

Как во сне геройски он побеждал Российский газ.

Снится Байдену, что он любезно избран выполнить задание ООН.

Он космический спецназ, а по избранному делу - превосходный водолаз. 

Исключительное право мир спасти, поверьте – достойнее сему нет нрава. 

Вот, незамедля он костюм одев, в пучину моря погрузился. Схватив трубу, как
голову дракона, пылающего пламенем Российского Орла, пленил весь газ для
праведной расправы. 

Огню предал его посланец мира, торжественно ликуя, как царь Нейрон. 

Так, достигая недр земли, вражину он низвел до ада. 

Но, Боже, видит он как подымается из ада пучина пепла, дыма, смрада – огнём
пылающий дракон. Проснулся разогретый в чреве вулкан, что спал в прохладной
неге. И имя тому зверю Йеловстоун. Куда проснувшемуся зверю девать вскипевшую
лавину? Он рад обнять собой всю землю, раскинув лапы с подземелья. 

И тут глаза «спасителя» прозрели. Сожженный газ был для вулкана на заказ.
Тушить он бросился пожар – вновь героический поступок. Но чтобы завершить
рассказ, я вспомню о котятах милых. Они в тушеньи рьяно помогали, обильно
простынь поливали, не дав огню хозяина во сне сгубить. Котята выросли немножко,
по прежнему тепла хотят и ласки, и сны хозяина особенно хранят от новых
подвигов и приключений, дабы угрозы миру миновать.
