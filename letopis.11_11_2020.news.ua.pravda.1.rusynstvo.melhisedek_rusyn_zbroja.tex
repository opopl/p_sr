% vim: keymap=russian-jcukenwin
%%beginhead 
 
%%file 11_11_2020.news.ua.pravda.1.rusynstvo.melhisedek_rusyn_zbroja
%%parent 11_11_2020.news.ua.pravda.1.rusynstvo
 
%%url 
%%author 
%%tags 
%%title 
 
%%endhead 

\subsubsection{Мельхіседек і русинська ядерна зброя}

Про те, що закарпатські русини вважають себе окремим народом і навіть
претендують на автономну республіку, газета The Washington Post писала ще в
1995 році.

У статті йшлося про такий собі Домініон Мельхіседека, віртуальну державу, яку
згодом звинуватили у причетності до банківських махінацій.

Ця ніким не визнана держава нібито оголошувала своєю територією ділянки землі
по всьому світу. Починаючи від французького атола Кліппертон і закінчуючи
Закарпаттям, яке в документах Домініона називалося "Підкарпатська Русь" або
"Рутенія".

Марк Педлі, творець Домініона Мельхіседека, навіть погрожував Франції ядерною
зброєю, яка нібито залишилася на Закарпатті після Радянського Союзу і
знаходилася на той момент у розпорядженні русинів.

Все це нагадувало гротескну комедію з ухилом у повну нісенітницю. А проблеми
закарпатських русинів були представлені у настільки безглуздому контексті, що
примушували згадати голлівудські бойовики на кшталт "Червоної спеки" з
Арнольдом Шварценеггером у ролі радянського капітану міліції Івана Данко. 

Лише згодом стало зрозумілим, що ці проблеми справді існують. Більш того, час
від часу вони загострюються настільки, що навіть таке поняття як "русинський
сепаратизм" вже не сприймається оксюмороном.

"Ти занадто драматизуєш русинське питання", \textemdash каже мій друг Тарас, кинувши
погляд на монітор ноутбука і прочитавши останню фразу.

Авжеж, дуже хотілося би зосередитись на "культурному русинстві", згадати
відомих русинів як-от суперзірку поп-арту Енді Воргола, поринути у тонкощі
русинської кухні.

\ifcmt
img_begin 
    url https://img.pravda.com/images/doc/e/a/eaacac6-dscf1200.jpg
    caption Енді Воргол (Андрій Варгола) \textemdash найвідоміший русин у світі. В Ужгороді його також шанують, хоча батьки митця родом не з Закарпаття, а з села, що зараз знаходиться на території Словаччини
    width 0.7
img_end
\fi

Але саме "політичне русинство" привертає найбільшу увагу суспільства. Останнім
часом у цьому казані знов забулькотіло.

Засуджений ще при Януковичі за сепаратизм священник УМЦ МП Димитрій Сидор на
останніх виборах балотувався від ОПЗЖ в Ужгородську міську раду. Цього разу
пролетів, як фанера над Говерлою, але нагадав про себе.

\ifcmt
img_begin 
        url https://img.pravda.com/images/doc/7/3/735c943-dscf1069.jpg
        caption Кафедральний Хресто-Воздвиженський собор в Ужгороді належить до Української православної церкви Московського патріархату. Настоятелем храму є сумнозвісний священик Димитрій Сидор. Саме Московський патріархат на Закарпатті є головним натхненником сепаратистських настроїв у русинській громаді.
        width 0.7
img_end
\fi

12 років тому проти нього відкрили кримінальне провадження за статтею
"Зазіхання на територіальну цілісність України". Голова так званого "Cойму
підкарпатських русинів" Сидор 1 грудня 2008 року проголосив незалежність так
званої "Підкарпатської Русі". В 2012 році Ужгородський міськрайонний суд виніс
вирок: три роки позбавлення волі – умовно, та сплату 1840 гривень штрафу.

Найкращі друзі отця Сидора, бізнесмени Василь Джуган та Михайло Тяско, які
представляються керівниками "Всесвітньої ради підкарпатських русинів" та мають
тісні зв'язки з Росією і проросійськи налаштованим президентом Чехії Мілошем
Земаном, нещодавно були заарештовані чеською поліцією за звинуваченням у
несплаті податків.

Про це писали медіа, і русинська громада у Закарпатті починає "миригувати",
тобто "нервувати".

"Поспілкуйся з місцевими русинами і зрозумій, що насправді ніякого екшна,
жодної драми у цій темі не існує. Лише мляві віртуальні бійки у соцмережах і
нагнітання політичної істерики у тих самих медіа", – заспокоює Тарас.

Щоб переконатися в цьому, прямую на зустріч з істориком Володимиром Феничем,
автором монографії "\enquote{Угорська Русь і Ганнібалова присяга} Михайла Драгоманова:
початки інтелектуального завоювання Закарпаття Україною".

Ця книга 5 років тому викликала справжній скандал і коштувала Феничу місця
декана історичного факультету Ужгородського університету. 

