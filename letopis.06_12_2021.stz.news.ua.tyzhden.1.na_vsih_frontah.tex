% vim: keymap=russian-jcukenwin
%%beginhead 
 
%%file 06_12_2021.stz.news.ua.tyzhden.1.na_vsih_frontah
%%parent 06_12_2021
 
%%url https://tyzhden.ua/Culture/253786
 
%%author_id chornogor_jaroslav
%%date 
 
%%tags 
%%title На всіх фронтах
 
%%endhead 
\subsection{На всіх фронтах}
\label{sec:06_12_2021.stz.news.ua.tyzhden.1.na_vsih_frontah}

\Purl{https://tyzhden.ua/Culture/253786}
\ifcmt
 author_begin
   author_id chornogor_jaroslav
 author_end
\fi



Матеріал друкованого видання № 48 (732) від 1 грудня.

\headTwo{Що Україна може протиставити гуманітарній агресії Кремля}

Чинна Стратегія національної безпеки України чітко вказує українському
суспільству на необхідність «рішуче протистояти гуманітарній агресії, розвивати
українську культуру як основу консолідації української нації та зміцнення її
ідентичності». І це дуже правильна вимога, адже ми майже щодня постаємо перед
виявами російського інформаційного впливу. 

\ii{06_12_2021.stz.news.ua.tyzhden.1.na_vsih_frontah.pic.1}

І якщо з прямими гібридними атаками вдається впоратися шляхом спростуванням
фейків, то загрози, схожі на «міни уповільненої дії», закладені імперською
політикою Москви за кілька сотень років у наш культурний шар, долати набагато
важче. Важливим також є той вплив, який діє на нашу державу опосередковано,
через інші країни, що знають Україну поверхово й нерідко сприймають її історію
та культуру через призму російських пропагандистських штампів. За 30 років
незалежності Україна не змогла достатньо віддалитися від РФ, лише окупація
Криму й початок бойових дій на території східних областей України дещо вплинули
на українське суспільство. Російська імперія, а згодом і СРСР розпалися, але
РФ, що утворилася на їхньому місці, не збирається втрачати «імперський спадок»
і активно бореться за нього. Відповідно політика Кремля підпорядкована
геополітичній стратегії та спрямована на реалізацію широкомасштабних цілей. Її
визначають пропагандистські завдання, що використовують культурно-історичні
наративи для обґрунтування зовнішньої політики держави.

\headTwo{Асиметрична боротьба}

Російську політику щодо України необхідно розглядати в контексті реалізації
такого геополітичного проєкту, як «русский мир», що має на меті реінтеграцію
пострадянського простору, а в перспективі — повну інкорпорацію нашої країни до
РФ. Київ протистоїть цим планам і відбиває весь спектр гібридних атак з боку
Москви. З огляду на вкоріненість у різних країнах світу не завжди сприятливих,
спотворених або поверхових уявлень про Україну, а також на значну асиметрію у
силах і засобах із Росією, нам варто обрати ключові точки докладання зусиль. І
культурна сфера здатна стати зручним плацдармом у цій боротьбі. Її результат
визначить майбутнє нашої державності, адже успіх щодо відновлення РФ статусу
«наддержави» неможливий без України.


То які загрози для нас найнебезпечніші? Їх є кілька, і спрямовані вони на
позбавлення українців культурно-історичного базису, без якого
соціально-економічний розвиток утрачає значення, та й потреба у власній державі
зникає.

Однією з таких загроз є російська історична політика, покликана утвердити
російську суб’єктність коштом інших народів. І український наратив відіграє в
цьому процесі фундаментальну роль, адже керівництво РФ намагається привласнити
нашу культурно-історичну спадщину. Росія використовує всі аспекти: всю історію
як таку, окремі історичні події та постаті, релігійне життя, літературу й
науку, постаті письменників, художників, науковців, спортсменів, а також вияви
нематеріальної культури, такі як пісні, колядки та щедрівки й навіть кулінарні
страви.

\headTwo{Украдена «велич»}

Без України, без сакрального значення Києва, без людських і матеріальних
ресурсів нашої країни неможливо вибудувати чітку структуру історичного розвитку
Росії як «великої держави». Тому Кремль вдається до історичних маніпуляцій, які
нібито мають обґрунтувати право РФ на певні території. Це і спадок Русі (в
основі легітимації російської держави є твердження, що єдиний спадкоємець
Київської Русі — середньовічне Московське князівство, з якого постало «Русское
царство», а згодом і Російська імперія), на міжнародному рівні озвучується
«русская» приналежність київських Рюриковичів: Ольги, Володимира Великого,
Ярослава Мудрого, Анни Ярославівни, Володимира Мономаха. Це й так зване
возз’єднання України з Росією за часів Богдана Хмельницького та «зрада» Івана
Мазепи (сумніви в законності та справедливості приєднання частини українських
земель позбавляє легітимності вимог Росії щодо Криму та всього чорноморського
узбережжя разом з Кубанню). Це й боротьба нашого народу впродовж усього ХХ
століття за власну державу (факт існування українських державних утворень у
1918–1921 роках і збройна боротьба українців проти російських окупантів
руйнують кремлівські історичні міфи про «братній народ»). Для звеличення Росії
та її «великої культури» продовжується привласнення представників різних
народів та їхній творчий спадок. Так, усіх діячів літератури науки та спорту,
які проживали в межах Російської імперії / Радянського Союзу автоматично
оголошено надбанням «русской культуры». У такий спосіб Україну намагаються
позбавити Петра Гулака-Артемовського та Івана Котляревського, Миколи Гоголя та
Володимира Короленка, Миколи Миклухо-Маклая та Єгора Ковалевського, Володимира
Вернадського й Сергія Корольова, Івана Піддубного та Іллю Рєпіна й багатьох
інших.

\begin{zznagolos}
в українців є один важливий фактор — це правда, і вона на нашому боці.
Донесення об’єктивної інформації здатне серйозно протидіяти експансіоністським
планам «русского мира». Потрібно лише правильно визначити слабкі місця
російської політики	
\end{zznagolos}

Яскравим виявом такого привласнення стала постать українського богатиря Івана
Піддубного. У 2014 році в Росії вийшов у прокат художній фільм «Поддубный».
Головну роль зіграв Міхаіл Порєчєнков, який згодом, у жовтні того самого року,
відкрито підтримав російських терористів на Донбасі й навіть стріляв з кулемета
по українських позиціях. В Україні фільм було заборонено через
пропагандистський контент. Але з боку Москви таке рішення викликало значне
обурення, а тодішній міністр культури РФ Владімір Мєдінський заявив, що
«Україні логічно надалі заборонити Булгакова, Гоголя й Шевченка».

У 2019–2020 роках розгорівся міжнародний скандал через спробу РФ привласнити
українську національну кулінарну страву — борщ. Розпочалося це з повідомлення
на офіційному урядовому акаунті Росії у твіттері з рецептом борщу, який назвали
Russian borsch. Ця суперечка навіть потрапила на шпальти світових ЗМІ, зокрема
про неї написала американська газета The New York Times. Також французька
компанія Groupe Michelin мусила вибачитися за визначення в публікації,
присвяченій московським ресторанам, українського борщу як страви російської
кухні, та виправити помилку. Відмова України та світової спільноти сприймати
борщ як російську страву викликає в РФ гостре обурення. І для відстоювання
своєї правоти доходить до того, що якщо вже не вдається закріпити борщ за
Росією, то можна засумніватися в його українському походженні й приписати
винайдення страви римським легіонерам. Наразі боротьба за визнання українського
походження борщу триває, на щастя, за широкої суспільної підтримки.

\headTwo{М’яка сила Кремля}

Ще однією загрозою для України є так звана зовнішня культурна політика РФ. Її
завдання — просувати «русскую культуру» у світі, а пріоритетом визначено
пострадянський простір зі збереженням російського впливу на ньому та подальшою
розбудовою «русского мира». Головними провідниками цього процесу є Російський
центр міжнародного наукового та культурного співробітництва («Росзарубежцентр»)
й Федеральне агентство зі справ СНД, співвітчизників, які проживають за
кордоном, і з міжнародного гуманітарного співробітництва — широковідоме
«Россотрудничество». Прикриваючись насамперед так званою високою культурою —
балетом, оперою, літературою, а також поширенням і популяризацією російської
мови, вони активно впроваджують вплив РФ.

Не можна оминути увагою роботу фонду «Русский мир», який вкладає кошти в
розвиток російських ресурсних центрів, що надають доступ до великих масивів
навчально-методичної та науково-популярної інформації з РФ і про неї та
розташовані в понад 40 країнах світу. Також є ще мережа організацій і структур,
які працюють на «культурній ниві». Зокрема, це й міжнародні російськомовні ЗМІ,
які доносять інформацію в такому вигляді, який максимально вигідний Кремлю.
Серед них телеканали «Первый канал. Всемирная сеть», «РТР-Планета», «НТВ-Мир»,
а також уже одіозні телевізійні мережі RT та Sputnik. Вони містять канали, що
мають доступ до багатомільйонної аудиторії в будь-якій точці земної кулі, й
Кремль часто використовує їх як рупор пропаганди. І це далеко не вичерпний
інструментарій зовнішньої культурної політики РФ.

Також ще одна загроза для України міститься в релігійній сфері. Релігійні
концепти-симулякри, як-от «русский мир», «духовные скрепы», «собирание русских
земель», «Святая Русь» тощо стали ідеологічним підґрунтям російсько-української
війни. На щастя, створення Православної церкви України та отримання Томосу від
Вселенського патріарха стало потужнім кроком для нашої держави у справі
утвердження незалежності. Протидія становленню та визнанню української церкви
ще триває, тому й культурну політику, і вплив на православний світ РФ активно
використовує безпосередньо й через структури РПЦ у світі, зокрема із залученням
російськомовної діаспори.

Є ще спорт і російська естрада — інструменти, спрямовані на великі маси
населення. Українське суспільство майже беззахисне перед ними. Це питання дуже
широке, потребує глибокого вивчення й заслуговує на окрему статтю.

\headTwo{Захистити своє}

Розглянуті безпекові загрози в культурно-гуманітарній сфері є серйозними
ризиками і потребують активних заходів із протидії. Проте в українців є один
важливий фактор — це правда, і вона на нашому боці. Донесення об’єктивної
інформації здатне серйозно протидіяти експансіоністським планам «русского
мира». Потрібно лише правильно визначити слабкі місця російської політики й
докласти зусиль для викриття кремлівської брехні та дезінформації, одночасно
демонструючи наявні матеріальні й духовні скарби української нації.

Певні сприятливі передумови для цього вже є. Від початку російської агресії
проти України в міжнародному середовищі посилився критичний до РФ дискурс, а з
ним похитнувся й рівень довіри до меседжів Москви, з’явилися застороги до
кремлівської дезінформації. Про це зокрема свідчить кейс із повернення
артефактів з кримських музеїв, відомих як «скіфське золото», коли в судовому
порядку вдалося спростувати претензії РФ і добитися визнання української
приналежності цих історичних пам’яток.

У світі є інтерес до внутрішньополітичної ситуації в Україні, окремих аспектів
війни та можливостей мирного врегулювання, перебігу окремих реформ і релігійної
політики. Цей момент варто використовувати, бо увага світової спільноти може
перемкнутися на іншу країну чи регіон і вікно можливостей швидко зачиниться.
Також у самій Україні з’явилися дієві інструменти й установи для реалізації
культурної політики, здатні посилити мистецьку та літературну присутність,
показати нашу сучасну культуру, сприяти донесенню правильних і правдивих
меседжів і до власного населення, і до цільових аудиторій по всьому світу.
Виклик для нашої держави полягає в тому, що потрібно перейти від окремих дій до
комплексної системи реалізації культурної політики, яка впливатиме на питання
національної безпеки та міжнародного іміджу, а також відкриє перспективи
соціально-культурного й економічного розвитку, позитивно позначиться на
внутрішній політичній стабільності, спроможності бачити й планувати майбутнє. І
цим самим надасть Україні внутрішню та зовнішню підтримку у протистоянні з РФ,
а також забезпечить стійкість нашої держави до будь-яких загроз. 
