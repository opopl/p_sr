

\documentclass[a4paper,11pt]{extreport}



\usepackage{titletoc}

\usepackage{xparse}

\usepackage{p.core}

\usepackage{p.fancyquote}

\usepackage{p.env}

\usepackage{p.rus}

\usepackage{p.secs}

\usepackage{p.toc}

\usepackage{p.fb}

\usepackage{p.sechyperlinks}

\usepackage[xindy]{imakeidx}

\usepackage{p.hyperref}

\usepackage[hmargin={1cm,1cm},vmargin={2cm,2cm},centering]{geometry}

\usepackage{color}

\usepackage{xcolor}

\usepackage{colortbl}

\usepackage{graphicx}

\usepackage{tikz}

\usepackage{tikzsymbols}

\usepackage{pgffor}

\usepackage[export]{adjustbox}

\usepackage{longtable}

\usepackage{multicol}

\usepackage{filecontents}

\usepackage[useregional]{datetime2}

\usepackage{mathtext}

\usepackage{nameref}



\begin{document}
Лариса Ніцой
28 июня, 15:11 ·
День Конституції.
Так уже склалося, що я пишу казки про наболілі українцям теми, і навіть про Конституцію в мене розповідається в "Казці про славного Орленка" - книжці для старшокласників та їхніх батьків.
Так от, найбільша проблема нашої Конституції - це не Конституція, а ми, громадяни України, які не дотримуємося Конституції. У нас кожен другий - конституційний професіонал. Усі ми великі знавці, що з нашою Конституцією не так, і як її треба поправити, забуваючи, що можна внести купу правок, але, якщо продовжувати їх не дотримуватися, то ... нічого не зміниться, будемо знову розповідати, що винна Конституція, а не ми.
От, наприклад, є в Конституції стаття 10 про державну мову, російську мову і мови меншин. 25 років ми сперечаємося навколо цієї статті. Але КРАПКУ в цій суперечці давно, ще 22 роки тому, поставив Конституційний Суд (КС).
Ця уставнова (КС) існує саме для того, щоб давати роз'ясннення, кому що не зрозуміло в Конституції. Що цікаво, якщо є роз'яснення КС по якійсь статті, то хто б ти не був, навіть якщо ти "господь Бог", то трактувати цю статтю Конституції треба так, як сказав КС. А інше трактування вважається неконституційним і карається штрафами і навіть в Кримінальному кодексі передбачені штрафи або 3-5 років тюрми.
Так от, усім українцям Конституційний Суд ще в 1999 році сказав: "Досить сперечатися наколо мови і Конституції, статті 10. Українська мова є обов'язковою, а інші мови національних меншин, у тому числі і російська мова - є НЕОБОВ'ЯЗКОВИМИ. Усі мови МОЖУТЬ вільно розвиватися ("можуть" означає: хочуть - розвиваються, не хочуть - не розвиваються, ніхто їх не переслідуватиме), але ОБОВ'ЯЗКОВА лише одна - УКРАЇНСЬКА мова. І держава повинна забезпечувати (тобто фінансувати) лише державну мову". Отак сказав Конституційний Суд.
Чи припинилися після такого Рішення КС суперечки в нашому суспільстві навколо статті 10? Та нічого подібного. Не припинилися.
Маючи таке рішення, таке пояснення Конституційного суду, ми й далі розповідаємо один одному на телебаченні, на нарадах, у міністерствах, урядам і президентам, як треба розуміти цю статтю - і це розуміння, всупереч Рішенню КС, є різним.
Ми всі є порушниками закону. Зверніть увагу, порушниками є не лише 5 колона, яка ініціює такі антизаконні дискусії, множачачи пояснення КС на нуль, а порушниками є всі ми, які беремо 22 роки участь у цих дискусіях, замість дати рішучу відсіч. Ми винні, що товчемо воду в ступі - і спонукаємо своєю нерішучістю далі розводити незаконні розмови порушниками.
То хто нам винен? Конституція? Що ми своїми дискусіями (замість жорстких дій) лише заохочуємо порушників порушувати Конституцію і самі стаємо співчучасниками їхніх порушень.
Висновок. Ми повинні рішучо утвердити в Україні 10 статтю Конституції - українську мову. Це повинно стати завдянням кожного українця, бо від цього залежить не лише національна безпека, а й економічне процвітання України. Спільна мова формує спільний світогляд. Спільний світогляд допомагає сформувати спільне бачення нашого майбутнього. А спільнобачення формує народ як спільну команду та спільні шляхи побудови майбутнього до економічного процвітання. Мовний роздрай створює роздрай в країні і приводить до неуспіху.
#КонституцІя_Ніцой
#День_Конституції_Ніцой
\section{Comments}
\begin{itemize}
\iusr{Victor Topol}
\url{https://www.facebook.com/profile.php?id=100007802321342}

Тут усе просто: ст.10 Конституції України говорить, що держава забезпечує функціонування української мови в УСІХ сферах СУСПІЛЬНОГО життя. Усі інші мови можуть функціонувати виключно у сфері ПРИВАТНОГО життя. Сфери суспільного життя це магазини, громадський транспорт, банк, телебачення, радіо, пляж, вулиця, органи влади, ресторани, кафе, театри і т.д. І т.п. - громадські місця, де збираються незнайомі люди. А у себе в сім'ї, серед знайомих і друзів або у приватній сфері життя людей ніхто не забороняє говорити іншими мовами.

\begin{itemize}
\iusr{Александр Синявский}
\url{https://www.facebook.com/profile.php?id=100065769780799}

Victor Topol в фашистском свинарнике и конституция - фашистская.Все по заветам вашего главного хэроя Гитлера.

\iusr{Игорь Олевский}
\url{https://www.facebook.com/profile.php?id=100031911082888}

Victor Topol
Це брехня, нема такого в Конституцii. В Конституцii написано, що мовою повиннi користуватись лише держслубовцi i тiльки в держустановах. В iнших випадках якою хочеш.

\end{itemize}
\iusr{Nadya Skorpion}
\url{https://www.facebook.com/nadya.skorpion}

Висловлюю щиру подяку пані Ларисі за правильні, мудрі слова щодо Конституції і мови. Так, нам усім потрібно на кожному кроці захищати свою рідну мову, а не переходити одразу на "общепонятну", демонструючи тим самим свою меншовартість. Якщо вони тут живуть роками і не можуть вивчити мову землі, котра їх годує, то чому ми маємо принижуватись перед такими недолугими.....

\iusr{Cvetik Nazarenko}
\url{https://www.facebook.com/cvetik.nazarenko.9}

В нашій країні порушуються багато статей.
В мене особисто з вчителями був конфлікт за 35 ст. КУ. Я поважаю інші релігії але нав'язувати дітям в школах християнство - це не допустимо.

\iusr{Константин Попов}
\url{https://www.facebook.com/butamil}

ми товчемо воду в ступі

\iusr{Анат Юхименко}
\url{https://www.facebook.com/profile.php?id=100017478342701}

,,Батьки,, від влади, завжди хитро маніпулюють Конституцією.
От мова, віра...
Будьте ласкаві.
Користуйтеся.
Тут не потрібно, аж ніяких витрат.
А де мова заходить про безкоштовну медицину і освіту...Там, активно помалкують.

\iusr{Топовый поклонник
Tetyana Rabchevska}
\url{https://www.facebook.com/tetyana.rabchevska}

Згідна з пані Ларисою на всі сто !

\iusr{Светлана Колесникова}
\url{https://www.facebook.com/profile.php?id=100004128582810}

Светлана Колесникова

\iusr{Maryana Kozak}
\url{https://www.facebook.com/mariana.kozak.31}

Maryana Kozak

\iusr{Шарупа Олександр}
\url{https://www.facebook.com/olexasha}

+

\iusr{Nata Ukra}
\url{https://www.facebook.com/profile.php?id=100068367502671}

https://www.youtube.com/watch?v=R0soYudCTLI

\iusr{Nata Ukra}
\url{https://www.facebook.com/profile.php?id=100068367502671}

https://www.ar25.org/article/oksana-bilozir-pro-proekt-gartlend-my-tilky-teper-pochynayemo-pysaty-svoyu-istoriyu.html?fbclid=IwAR0sWhXq5ykiZuTBpIqHlVOPj59Gz3ZopRMZjDMcfSDETdyby3cQAqhDwKA

\iusr{Лиска Лілія}
\url{https://www.facebook.com/profile.php?id=100023476873611}

Вітаю з Днем Конституції славних,супер-Українців--- пані Л.Ніцой та її мудру половинку--пана А.Ніцой!!! Дай Боже Вам --Здоров'я та Успіхів у всіх Справах!!!

\iusr{Ivan Galenko}
\url{https://www.facebook.com/ivan.galenko.908}

З Днем Перемоги в Конотопській битві! Зі Святом!
Можливо, саме тому у ці дні я думаю не про Конституцію, а про інструменти її дотримання. В НАТО - "стратегічні комунікації". 👫 В США - концепт "СИНХРОНІЗАЦІЯ ЗУСИЛЬ" уже як два роки. 🚀 А в Україні - "МУТЬ" 😡 . https://kiberdjura.blogspot.com/2021/06/blog-post_26.html Чи треба з цим щось робити? 🤫

\end{itemize}
\end{document}