% vim: keymap=russian-jcukenwin
%%beginhead 
 
%%file 09_11_2021.fb.nitsoi_larysa.1.movna_stijkistj.cmt
%%parent 09_11_2021.fb.nitsoi_larysa.1.movna_stijkistj
 
%%url 
 
%%author_id 
%%date 
 
%%tags 
%%title 
 
%%endhead 
\subsubsection{Коментарі}
\label{sec:09_11_2021.fb.nitsoi_larysa.1.movna_stijkistj.cmt}

\begin{itemize} % {
\iusr{Eduard Karpo}
Героям Слава! Дякую за працю!

\iusr{Джулія Рагімова}
Вітаю @igg{fbicon.bouquet} 

\iusr{Наталя Сосюк}
Героям Слава!

\iusr{Алла Рубан}
Тільки так! Будьмо сильними мовно і культурно перед окупантами, і вистоїмо у цій нерівній боротьбі.

\iusr{Анатолій Джигирис}
ДЯКУЄМО !

\ifcmt
  ig https://scontent-mia3-1.xx.fbcdn.net/v/t1.6435-9/254328516_1788946398161813_2589389871618593823_n.jpg?_nc_cat=100&ccb=1-5&_nc_sid=dbeb18&_nc_ohc=PC3DmdLJ-AoAX-PR1cz&_nc_ht=scontent-mia3-1.xx&oh=8fc14ccdc39c3b47df57c4bf44db13ca&oe=61B0B07D
  @width 0.3
\fi

\iusr{Nataliia Grygorieva}
+++++++++++

\ifcmt
  ig https://scontent-mia3-1.xx.fbcdn.net/v/t39.30808-6/255763736_547249406730693_4518070795514478421_n.jpg?_nc_cat=100&ccb=1-5&_nc_sid=dbeb18&_nc_ohc=e5YqJqzY-bQAX9QQzxp&_nc_ht=scontent-mia3-1.xx&oh=c818a5e50b289ecdca0c8d84129d060e&oe=618F2F0C
  @width 0.3
\fi

\iusr{Ivan Myrutenko}

Підтримую! Але не на всі 100, а не 99, бо видалити зі своєї бібліотеки 50-60\%
книг не можу, хоча вони й російською!) Перекладіть на українську, тоді я
поміняю!)

\begin{itemize} % {
\iusr{Володимир Олівець}
Вбийте в собі московита, пане \textbf{Myrutenko}, вони давно перекладені. Можу приїхати й видалити!

\iusr{Ivan Myrutenko}
\textbf{Володимир Олівець} Видалити не проблема! Ви привезіть мені аналогічну літературу українською, а я вам віддам російською!
\end{itemize} % }

\iusr{Любов Янішевська}

\ifcmt
  ig https://scontent-mia3-1.xx.fbcdn.net/v/t39.1997-6/p480x480/91521538_1030933857302751_5093925307199520768_n.png?_nc_cat=1&ccb=1-5&_nc_sid=0572db&_nc_ohc=0Z8I-jLPbCkAX8jTp0n&_nc_ht=scontent-mia3-1.xx&oh=3ff304ff66b82c549317db2d8f757a4d&oe=618E4077
  @width 0.1
\fi

\iusr{Vitaliy Lazorkin}
Вітаю!

\ifcmt
  ig https://scontent-mia3-1.xx.fbcdn.net/v/t1.6435-9/255194102_4869156833147870_3179498037323889331_n.jpg?_nc_cat=101&ccb=1-5&_nc_sid=dbeb18&_nc_ohc=LbtrxMqR8BcAX-_xZ12&_nc_ht=scontent-mia3-1.xx&oh=d29c48fb58b298fcc67f654e38bfd096&oe=61AF2800
  @width 0.3
\fi

\iusr{Ольги Мусієнко-Боровик}
Хай наша МОВА чарує світ.

\iusr{Лілія Коваль-Терлецька}
\textbf{\#Бандеризація\_України} головний посил, слова, які належать Ірина Фаріон

\iusr{Лариса Котляревська}
Вітаю! Дякую за Вашу працю і небайдужість! Сил і здоров'я Вам!
@igg{fbicon.heart.red} @igg{fbicon.heart.yellow}  @igg{fbicon.heart.blue} 

\iusr{Тамара Мозгова}

Писала диктант в онлайн трансляції. Надто швидко диктував Юрій Андрухович. Мені
доводилось залишати місце для слів, потім вписувати. Але, написала.

\begin{itemize} % {
\iusr{Larysa Nitsoi}
\textbf{Тамара Мозгова} так само.
\end{itemize} % }

\iusr{Володимир Олівець}

Загалом поділяю. Але один день на рік замало, звичка формується місяць! Це за
логікою схоже на позицію тих, що не споживають москвинського лише під час
війни. На жаль, навіть серед ваших друзів (і моїх теж) купа охочих до
споживання москвинського і навіть поширення такого (іноді зі словом
"перепрошую"). На зауваження такі рідко реагують адекватно. Вже скільки днів
української писемности та мови минуло, днів рідної мови, що я вже не вірю в
їхню ефективність...

\begin{itemize} % {
\iusr{Larysa Nitsoi}
\textbf{Володимир Олівець} ну я ж пишу, що сьогодні день, в який варто ПОЧАТИ )), хто цього й досі не зробив.

\iusr{Володимир Олівець}
Ну це инша справа! @igg{fbicon.thumb.up.yellow}  Тільки ви, як вчителька, щодня нагадуйте і в разі чого ставте двійки...  @igg{fbicon.face.tears.of.joy} 
\end{itemize} % }

\iusr{Ганна Широкова}

\ifcmt
  ig https://scontent-mia3-1.xx.fbcdn.net/v/t39.1997-6/s168x128/47270791_937342239796388_4222599360510164992_n.png?_nc_cat=1&ccb=1-5&_nc_sid=ac3552&_nc_ohc=jhk2sJotzV8AX9Mr9H7&_nc_ht=scontent-mia3-1.xx&oh=2902c1323ce26a9456f2bd7451261fdf&oe=618F95B6
  @width 0.1
\fi

\iusr{Лариса Федорчук}
Наша мова найкраща, наймилозвучніша.

\iusr{Natalia Natalia}
Дякую @igg{fbicon.heart.red} @igg{fbicon.heart.blue}  @igg{fbicon.heart.yellow} 

\iusr{Маряша Лексі}
Вуйко Штефко інакше б казав !

\end{itemize} % }
