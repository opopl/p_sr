% vim: keymap=russian-jcukenwin
%%beginhead 
 
%%file 24_12_2021.stz.sport.ua.dynamokiev.1.mova
%%parent 24_12_2021
 
%%url https://dynamo.kiev.ua/news/381926-artem-frankov-vyipolnyaya-sootvetstvuyuschij-zakon-ukrainyi-perejdem-na-ukrainskij-yazyik-gosudarstvennyij-poputno-udivlyayas
 
%%author_id 
%%date 
 
%%tags 
%%title Артем Франков: «Выполняя соответствующий закон Украины, перейдем на украинский язык. Государственный. Попутно удивляясь…»
 
%%endhead 
\subsection{Артем Франков: «Выполняя соответствующий закон Украины, перейдем на украинский язык. Государственный. Попутно удивляясь...»}
\label{sec:24_12_2021.stz.sport.ua.dynamokiev.1.mova}

\Purl{https://dynamo.kiev.ua/news/381926-artem-frankov-vyipolnyaya-sootvetstvuyuschij-zakon-ukrainyi-perejdem-na-ukrainskij-yazyik-gosudarstvennyij-poputno-udivlyayas}

\begin{zznagolos}
Главный редактор еженедельника «Футбол» Артем Франков рассказал об изменениях в
политике издания, начиная с середины января 2022 года.
\end{zznagolos}

«Это — последний номер «Футбола» в 2021 году, но не последний вообще. Как
говорится, не дождетесь — или дождетесь, но много позже.

\ii{24_12_2021.stz.sport.ua.dynamokiev.1.mova.pic.1}

Правда, начиная с 2022 года, нас ожидают определенные перемены. С первого же
номера, который увидит свет 10 января, мы возвращаемся к основательно
подзабытому формату‑1999 «мелованная обложка + 32 полосы газетки», сохраняя
выход два раза в неделю. Этот номер, 10‑го, станет одновременно последним
русскоязычным — после этого мы, выполняя соответствующий закон Украины,
перейдем на украинский язык. Государственный. Попутно удивляясь, как наша
метрополия, то бишь США, ухитряется вовсе обходиться без такого понятия...

Законопослушность действием, разумеется, ничуть не мешает далеко не мне одному
считать этот закон идиотским, вредным и опасным как для украинского языка, как
и для самого государства Украина. Вы мою позицию знаете, и я ее ничуть не
скрываю. Я — человек, для которого родной язык русский, но учил украинский со
второго по десятый класс средних своих школ и всю последующую жизнь, а посему
не хуже многих в этом вопросе. Ненавижу, когда меня принуждают к чему‑то —
хреновастенькая это свобода, когда не для всех.

Для особо одаренных и озабоченных Журнал «Футбол» в любом лице был и остается
сторонником территориальной целостности Украины, в том числе не признает
отторжения Крыма и части Донбасса, более того, клянет соседей за такую подставу
русскоязычного населения — дуракам ведь не объяснишь, что язык тут ни при чем!
Это ни в коем случае не отменяет сомнений по поводу эффективности действий
нынешней и прошлой власти, а также бардака в голове так называемых «патриотов»:
так вы чего хотите — вернуть эти земли с заблуждающимися и невинно
пострадавшими людьми или выжечь их на хрен, попутно прикончив всех предателей
Украины, то есть почти всех?! И пока сама наша держава не разберется в себе,
чего она хочет на самом деле, толку не будет ни малейшего. Военным способом
даже с помощью метрополии это всё решается только в случае развала России. А
она, стервь, почему-то не собирается разваливаться, несмотря на придурочные
прогнозы икспердов и прочей публики... Так что делать, а?

Все, прикрыли тему, тем более я могу быть не прав», — написа́л Франков в
передовице 98-го номера издания.
