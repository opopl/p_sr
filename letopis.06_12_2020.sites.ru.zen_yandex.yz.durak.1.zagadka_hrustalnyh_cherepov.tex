% vim: keymap=russian-jcukenwin
%%beginhead 
 
%%file 06_12_2020.sites.ru.zen_yandex.yz.durak.1.zagadka_hrustalnyh_cherepov
%%parent 06_12_2020
 
%%url https://zen.yandex.ru/media/durak_pro/zagadka-hrustalnyh-cherepov-dlia-chego-ih-delali-5fcbac75c26ad131b64af75e
 
%%author Дурак (Яндекс Zen)
%%author_id yz.durak
%%author_url 
 
%%tags istoria
%%title Загадка хрустальных черепов, для чего их делали?
 
%%endhead 
 
\subsection{Загадка хрустальных черепов, для чего их делали?}
\label{sec:06_12_2020.sites.ru.zen_yandex.yz.durak.1.zagadka_hrustalnyh_cherepov}
\Purl{https://zen.yandex.ru/media/durak_pro/zagadka-hrustalnyh-cherepov-dlia-chego-ih-delali-5fcbac75c26ad131b64af75e}
\ifcmt
	author_begin
   author_id yz.durak
	author_end
\fi

\ifcmt
  pic https://avatars.mds.yandex.net/get-zen_doc/1880741/pub_5fcbac75c26ad131b64af75e_5fcbaca28f8c7853ed295d1a/scale_1200
  caption Загадка хрустальных черепов, для чего их делали?
\fi

\begin{leftbar}
	\begingroup
		\em Первый хрустальный череп был найден в центральной Америке в 1927 году.
				Его изготовление приписывается культуре Майя. По преданиям самих Майя,
				всего хрустальных черепов 13. Через эти черепа жрецы якобы
				контактировали с богами.
	\endgroup
\end{leftbar}

\subsubsection{Магические свойства горного хрусталя}

Горный хрусталь – не самый удобный материал для изготовления скульптур. Редкий
скульптор, даже в наше технологически-продвинутое время, выберет этот материал
для собственного творчества. И это неспроста, сами попробуйте что-нибудь высечь
из стекла. Не вылить (выплавить) или выдуть, а именно высечь. Другое дело лёд,
скульптуры из которого сильно походят на стеклянные.

Однако, игнорируя все эти технологические сложности, древние люди зачем-то
высекали хрустальные черепа. Спрашивается, зачем? И почему именно черепа? И
именно человеческие?

\ifcmt
  pic https://avatars.mds.yandex.net/get-zen_doc/1860332/pub_5fcbac75c26ad131b64af75e_5fcbad027e300d7ccaf8fbff/scale_2400
  caption Загадка хрустальных черепов, для чего их делали?
\fi

Известно, что горный хрусталь во все времена считался особым материалом у
шаманов. К примеру, многие начинающие колдуны Мезоамерики шли в горы, чтобы
отыскать там самородки горного хрусталя, из которых, в последствие,
изготавливались предметы силы.

Найденный кусок хрусталя иногда подрабатывался, подгонялся под параметры его
обладателя, ведь особенно сильными кристаллами считались те, что по размеру
были схожи с пальцами обладателя.

Считалось, что горный хрусталь помогает человеку концентрироваться и
останавливать внутренний диалог.

\subsubsection{Прототипы предметов силы}

Во всех религиях мира существуют и применяются предметы силы, которые
представлены разнообразными святынями и ритуальной атрибутикой. Например, в
христианстве такими предметами являются: мощи, иконы, нательные и наперсные
кресты, предметы, некогда принадлежавшие Христу, апостолам или святым.

\ifcmt
  pic https://avatars.mds.yandex.net/get-zen_doc/3956291/pub_5fcbac75c26ad131b64af75e_5fcbad9b7e300d7ccaf9e6cd/scale_2400
  caption Представители танковой дивизии СС «Мёртвая голова» и Ацтекский бог
\fi

Черепами и костями животных древние люди украшали свои одежды и храмы. Воин с
ожерельем из зубов тигра, демонстрировал всем, что он однажды победил тигра, а
значит он искусный, сильный и ловкий боец.

Но зачастую, это делали не только для устрашения противника, но и в других –
магических целях. Однако совершенно непонятно зачем было изготавливать из
хрусталя копию человеческого черепа? И является ли такой хрустальный предмет
копией конкретного человеческого черепа, или представляет из себя образ черепа
вообще, иными словами – самостоятельный предмет в форме черепа?

Скорее всего, первые хрустальные черепа представляли собой точные копии черепов
настоящих, возможными прототипами которых были великие колдуны, чья сила из
тленных мощей переходила в нетленный хрусталь. И только потом эти черепа
сделались образами, подобными эмблеме дивизии СС «Мёртвая голова».

\ifcmt
  pic https://avatars.mds.yandex.net/get-zen_doc/1567436/pub_5fcbac75c26ad131b64af75e_5fcbad47702d845a132aed29/scale_2400
  caption Эмблема дивизии СС «Мёртвая голова» на фоне странной каменной скульптуры Майя
\fi

На образе конкретного человека построена концепция африканской магии Вуду, где
для осуществления колдовства, направленного на какого-нибудь конкретного
человека, колдун изготавливает специальную куклу, отождествляемую с этим
человеком - объектом колдовства. Кукла обязательно должна иметь внешнее
сходство с этим человеком. Для усиления эффекта к такой кукле прикрепляются
волосы, или какие-нибудь личные вещи этого человека, к примеру – кусок его
одежды.

И если в магии Вуду кукла является объектом, через который колдун воздействует
на жертву, то хрустальный череп, скорее всего - не что иное, как предмет силы,
из которого колдун её черпает.

