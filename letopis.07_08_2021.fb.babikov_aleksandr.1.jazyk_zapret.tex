% vim: keymap=russian-jcukenwin
%%beginhead 
 
%%file 07_08_2021.fb.babikov_aleksandr.1.jazyk_zapret
%%parent 07_08_2021
 
%%url https://www.facebook.com/permalink.php?story_fbid=338597734606594&id=100053691102017
 
%%author 
%%author_id babikov_aleksandr
%%author_url 
 
%%tags jazyk,kultura,mova,ukraina,ukrainizacia
%%title Чому ми маємо (за)боронити російську мову в Україні?!
 
%%endhead 
 
\subsection{Чому ми маємо (за)боронити російську мову в Україні?!}
\label{sec:07_08_2021.fb.babikov_aleksandr.1.jazyk_zapret}
 
\Purl{https://www.facebook.com/permalink.php?story_fbid=338597734606594&id=100053691102017}
\ifcmt
 author_begin
   author_id babikov_aleksandr
 author_end
\fi

Чому ми маємо (за)боронити російську мову в Україні?!

Чудернацькі вислови мовного омбудсмена не могли залишити осторонь висловлених
сентенцій щодо спрямування прихильників та користувачів російської мови. Не
бучу сенсу загромаджувати допис посиланнями на законодавство, бо навіть
положення ст. 10 Конституції України, якою гарантовано вільний розвиток,
використання і захист російської та інших мов національних меншин України,
викликають лише жаль щодо безпорадності нашого Основного закону. Європейські
стандарти, закладені у ратифікованій Європейсьій хартії регіональних мов або
мов меншин, згідно якої ми зобов’язалися (як кумедно звучить!) вчиняти рішучі
дії на їх розвиток з метою їх збереження, сприяння їх використанню у
суспільному житті і/або заохочення такого використання, вже стали минулим.

\ifcmt
  pic https://scontent-cdt1-1.xx.fbcdn.net/v/t1.6435-9/231360654_338597674606600_8444581935028482089_n.jpg?_nc_cat=103&_nc_rgb565=1&ccb=1-5&_nc_sid=8bfeb9&_nc_ohc=EtDBltdE5tAAX-Hq8JL&_nc_ht=scontent-cdt1-1.xx&oh=c5dec1cf1421b06e5d65796db477d995&oe=614EF02B
  width 0.6
\fi

То ж як ми маємо ставитися до російської мови та інших мов? Ми, ті, хто
любить свою Батьківщину і пишається її історією, а не шукає у ній поразок і
зневіри. Поглянемо на мовне питання іронічно і з історичної ретроспективи. Як
говорили наші пращури? Та не будемо обирати певних осіб з власних уподобань, бо
питання надзвичайно складне і багатогранне. Натомість візьмемо за основу плеяду
найвідоміших та беззаперечно шанованих людей, визначених Національним банком.

Князь Володимир Великий (960-1015 роки). Син князя київського Святослава
Ігоревича та рабині Малуші (древлянської княжни), онук княгині Ольги.
Виховувався своєю бабусею, а згодом дядьком Добринею. Історики переконані, що
розмовляв він на старослов’янській (церковнослов’янській), яку інколи називають
староболгарською, хоча інші стверджують, що на ній переважно  писали. Письмових
пам’яток за підписом Володимира Великого не лишилося і достовірної інформації,
на якій мові він розмовляв, не маємо. Проте, відповідь на це питання ми можемо
одержати ознайомившись із умовами, в яких він зростав. Існують гіпотези, що
його мати Малуш з балтійського міста Любека, звідки і його дядько - вихователь
Добриня. Бабуся княгиня Ольга родом з Пскова і, вірогідно, походженням з варяг.
Після смерті батька і міжусобної війни Володимир втік до Швеції, де на той час
перебував на службі Добриня. Протягом кількох років він проживав у цій країні,
де одружився та набрав військо варяг, з якими відвоював Київ. Така доля не
могла не вплинути на його ставлення до мов. Більш того, князь Володимир Великий
(ще той козак) поряд з кількома сотнями наложниць різних національностей та
країн, які постійно дислокувалися у містах Вишгород та Білогород (нині с.
Білогородка) під Києвом, був одружений на полоцкій княжні Рогнеді (матері
Ярослава Мудрого), чехині Малфріді, візантійській царівні Анні, свеї або
норманкі Олові. То ж можемо припустити, що поряд з старослов’янською, для
спілкування з дружинами, військом та наложницями, певною мірою Володимир
Великий мав бути поліглотом.

Ярослав Мудрий (980-1054). Син Володимира Великого та полоцької княгині
Рогнеди. Одружений на доньці шведській короля Інгігерді. Час його правління
відзначився не лише розбудовою шкіл, храмів, бібліотек, ним також було
започатковано переклади цінних рукописів і книг з європейських мов.Вів успішну
дипломатію з більшістю монарших дворів Європи та поріднився з багатьма з них
улаштуванням чисельних шлюбів. Складно сказати на якій мові розмовляв Ярослав
Мудрий. Письменною мовою на той час залишалася церковнослов’янська, на якій
відбувалося богослужіння та писемність не лише на території Київської Русі, а й
в Болгарії та Хорватії. Проте, на мою думку, такі обставини, як найм та
використання варязьких військ, що обумовлювало спілкування з їх ватажками, -
конунгами, налагодження тісних взаємовідносин з Візантією для розвитку церкви,
активна дипломатія з країнами Європи та укладання чисельних шлюбів, що
передбачало проведення певних перемовин і спілкування, організація перекладу
рукописів, вказує, що такий державний діяч не міг не володіти як мінімум
кількома мовами.

Богдан Хмельницький (1595-1657). Гетьман  Війська Запорозького, очільник
Українського гетьманату. Вільно, поряд з рідною, володів латинською, польською,
турецькою та французькою мовами. Перебуваючи у турецькому полоні, знаходився
поруч з командувачем османського флоту у якості перекладача, тривалий час
перебував на службі польського короля.

Іван Мазепа (1639-1709). Гетьман Війська Запорозького, Голова козацької держави
на Лівобережжі. Навчався в Києво-Могилянській академії, згодом вивчав військову
справу в Голандії та філософію у Франції. Тривалий час служив при дворі
польського короля Яна ІІ Казимира. І. Мазепа, поряд з рідною, прекрасно володів
польською, російською, татарською, французькою, латинською, італійською та
німецькою мовами.

Іван Франко (1856-1916). Знав 14 мов, перекладав твори 48-ми мовами. Мав
феноменальну здатність запам’ятовування іноземних слів, багато практикував.
Перший професійний український письменник, який заробляв на життя літературною
працею, номінувався на Нобелівську премію, писав твори не лише українською, а й
польською, німецькою та іншими мовами. Перекладав на українську у тому числі з
давньовавилонської, давньоарабської, давньогрецької, східних мов.

Михайло Грушевський (1866-1934) по батьківській лінії належав до козацького
роду Грушей-Грушевських, що згадується з ХУІІ сторіччя, народився в сім’ї
професора філології С.Ф. Грушевського, який викладав «рускую словесность» в
греко-католицькій гімназії, автора підручника церковнослов’янської мови. М.
Грушевський народився у м. Холм (Польща) дитячі роки провів на Кавказі. З
студентського періоду займався вивченням історії Русі-України, публікуючи твори
на російській та українській мовах, захистив магістерську дисертацію під назвою
«Барское староство». У 1906 році Харківський Університет присвоїв Грушевському
ступінь почесного доктора руської історії. В 1914 р. обраний членом Чеської АН,
в 1923 р. - член кореспондент Всеукраїнської АН, а в 1929 році обраний дійсним
членом Академії наук СРСР. Автор понад 2000 наукових праць.

Тарас Шевченко (1814-1861) народився у селі Моринці Київської губернії, у 1829
році переїжджає у с. Чижово під Смоленськом, а згодом - у Вільнюс, де навчався
малярству. З 1830 у Петербурзі навчається на 4-річних курсах у художника В.
Ширяєва. Багатогранна людина, яка досягла висот майстерності як художник, поет,
прозаїк, етнограф, громадський діяч. За виготовлення аквафортів для ілюстрації
книг російських та зарубіжних письменників отримав звання Академіка по
гравіюванню. Вільно і органічно писав твори українською та російською мовами.
Більшість прозаїчних творів, щоденник та значна частина листів Т. Шевченко
писалися російською мовою.

Леся Українка. (1871-1913) Зараз би її назвали вундеркіндом або
дитиною-міленіалом. У п’ять років почала писати музикальні п’єси, а з
восьмирічного віку - вірші. Поряд з українською та російською, володіла
болгарською, давньогрецькою, латиною, іспанською, італійською, польською,
німецькою, англійською та французькою мовами. Переважна більшість її творів –
українською. Лише прозу «Лист у далечінь» - німецькою, «Голос однієї російської
ув’язненої» - французькою, та російською - «Когда цветет никотиана…».

Григорій Сковорода. (1722-1794). На мою думку найбільш унікальна та недооцінена
історична постать України. Самобутній філософ, поет, педагог та мандрівник,
людина неординарних поглядів. Свої твори писав російською, латиною,
давньогрецькою, інколи використовував українську, проте особливості написання
творів, що вміщували у себе різнобарвність різних мов, і на даний час
залишається предметом активних досліджень науковців.

То ж у чому сенс забороняти російську мову? Як ставитися до творів, що виконані
нашими митцями російською? Ми маємо відрізати від себе по живому шмат своєї
історії і культури, чи скористатися цією перевагою?

І на останок лише слова Кузьми Скрябіна: «Мені до одного місця, на якій мові
людина говорить. Головне – то, що вона говорить. Якщо він знає узбецьку мову,
але за Україну згідний піти на амбразуру – то нам такі люди потрібні в країні».
