%%beginhead 
 
%%file 02_09_2021.fb.fb_group.mariupol.nekropol.1.otchet_volonter_rabota_21_24_avgusta
%%parent 02_09_2021
 
%%url https://www.facebook.com/groups/278185963354519/posts/591646288675150
 
%%author_id fb_group.mariupol.nekropol,arximisto
%%date 02_09_2021
 
%%tags 
%%title Отчет о волонтерской работе в Некрополе 21-24 августа 2021
 
%%endhead 

\subsection{Отчет о волонтерской работе в Некрополе 21-24 августа 2021}
\label{sec:02_09_2021.fb.fb_group.mariupol.nekropol.1.otchet_volonter_rabota_21_24_avgusta}
 
\Purl{https://www.facebook.com/groups/278185963354519/posts/591646288675150}
\ifcmt
 author_begin
   author_id fb_group.mariupol.nekropol,arximisto
 author_end
\fi

\vspace{0.5cm}
\textbf{Отчет о волонтерской работе в Некрополе 21-24 августа 2021}

Открытие древних плит, высадка пяти платанов, очистка Хараджаевского участка и
покраска крестов брата Михаила Улахова – возобновление волонтерской работы в
Некрополе оказалось очень результативным!

\textbf{Открытия и находки}

Обнаружение старинных плит Ивана Сахаджи (1804-1883) и Анастасии Сахаджи (?)
(род.1838 г.) стало настоящей сенсацией для волонтеров. Дело в том, что обе
находятся на древнем Хараджаевском участке, который волонтеры исследуют и
убирают уже больше года. Никто не ожидал, что на этом участке еще возможны
находки. Подробнее о них см. здесь \url{https://cutt.ly/sWcHWyR}
\footnote{Волонтеры открыли в Мариупольском Некрополе две древние плиты XIX века, Andrei Marusov, 30.08.2021, \url{https://www.facebook.com/groups/278185963354519/posts/589598428879936}}

\textbf{Благоустройство}

Волонтеры посадили еще пять саженцев платанов на древней, Платановой аллее.
Вместе с ними аллея уже насчитывает 15 деревьев. Все платаны, посаженные
весной, хорошо прижились и быстро растут. Мы планируем продолжить приобретение
и высадку саженцев, одновременно с уборкой и исследованием аллеи.

Valentina
Evseenko\footnote{\url{https://www.facebook.com/valentina.evseenko.7}} привела
в порядок кресты на могилах Пелагеи и Александра Улахова, брата Михаила
Улахова, знаменитого мариупольского фотографа. В начале июля мы очистили могилы
от зарослей, но завершить работу не смогли из-за ливней и жары.

Кроме того, мы продолжили очистку от зарослей Хараджаевского участка. Обильные
ливни превратили его в джунгли 🙁

\textbf{Использование пожертвований}

Мы потратили 531 гривну из пожертвований мариупольцев. На 471 грн. мы купили пять саженцев платанов; на 60 грн. – перчатки для работы.

Мы будем рады любым пожертвованиям на продолжение волонтерской работы в
Некрополе! Средства можно перечислить на карточку Привата 5168 7554 5285 3746.
Контактный телефон 096 463 69 88.

Мы благодарим волонтеров Александра и Наталия Шпотаковская\footnote{\url{https://www.facebook.com/natalie.shya}}, Илья Луковенко и
Валентину Евсеенко за самоотверженный труд!

Огромное спасибо всем и каждому благотворителю!

Приглашаем всех желающих принять участие в дальнейших волонтерских экспедициях!

\#mariupol\_necropolis\_report

\#платановая\_аллея
