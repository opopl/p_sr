% vim: keymap=russian-jcukenwin
%%beginhead 
 
%%file slova.razdelenie
%%parent slova
 
%%url 
 
%%author 
%%author_id 
%%author_url 
 
%%tags 
%%title 
 
%%endhead 
\chapter{Разделение}
\label{sec:slova.razdelenie}

%%%cit
%%%cit_head
%%%cit_pic
%%%cit_text
Конечно, украинская правящая элита выбрала первый глобальный проект, т.е.
трансгуманизм. Но тогда возникает закономерный вопрос, а что делать в ходе его
реализации с примитивным \emph{разделением} украинского общества по
культурно-историческим, религиозным о языковым проблемам? Ведь эти линии
\emph{разлома} становятся всё более глубокими и масштабными между различными
регионами. И хуже всего то, что наше общество пока молча соглашается с
\emph{разделением} на «правильных» и «неправильных» граждан.  Зато в Евангелие от
Матфея еще 2000 лет назад Иисус объяснял, что будет с государствами, где
осуществляется внутренний \emph{раскол}: «Иисус, зная помышления их, сказал им: всякое
царство, \emph{разделившееся} само в себе, опустеет; и всякий город или дом,
\emph{разделившийся в себе}, не устоит»
%%%cit_comment
%%%cit_title
\citTitle{Государство, в котором осуществляется внутренний раскол, обречено / Лента соцсетей / Страна}, 
Александр Гончаров, strana.ua, 27.06.2021
%%%endcit

