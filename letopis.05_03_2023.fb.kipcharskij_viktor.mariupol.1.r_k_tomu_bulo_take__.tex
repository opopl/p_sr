%%beginhead 
 
%%file 05_03_2023.fb.kipcharskij_viktor.mariupol.1.r_k_tomu_bulo_take__
%%parent 05_03_2023
 
%%url https://www.facebook.com/permalink.php?story_fbid=pfbid024BFtLFGsQ2UXtfDsZHjd1A8MuJuKsGriXUSi2iGaSSknqq5Wa8XtMCTJ98ww3kZil&id=100006830107904
 
%%author_id kipcharskij_viktor.mariupol
%%date 05_03_2023
 
%%tags mariupol,mariupol.war,dnevnik,05.03.2022
%%title Рік тому було таке: День 10  -  5.03.22. Субота
 
%%endhead 

\subsection{Рік тому було таке: День 10  -  5.03.22. Субота}
\label{sec:05_03_2023.fb.kipcharskij_viktor.mariupol.1.r_k_tomu_bulo_take__}

\Purl{https://www.facebook.com/permalink.php?story_fbid=pfbid024BFtLFGsQ2UXtfDsZHjd1A8MuJuKsGriXUSi2iGaSSknqq5Wa8XtMCTJ98ww3kZil&id=100006830107904}
\ifcmt
 author_begin
   author_id kipcharskij_viktor.mariupol
 author_end
\fi

Рік тому було таке:

День 10  -  5.03.22. Субота. 

Серед ночі, схоже на Лівий прилетіла ракета: шум, свист, потім вибух. На Лівому
щось горить. Близько першої мої пішли ховатися у підвал. Сиділи до третьої.

Новини поміняли тон: чиніть опір і т.д. Зелених коридорів нема. Маріуполь "в
блокаді", тобто в оточенні.

6:45. Мої сплять. Чомусь гадаю: як на Мирному та Новотроїцькому цвинтарях? Чи
цілі могили та пам'ятники нашім батькам?

7:00. Поодинокі рідкі, але дуже гучні постріли с чогось важкого.

7:03. Гул літака (ніби їде машина, звук зростає все гучніше, швидко переходить
у рев-свист), 4-6 вибухів. Де саме - не видно. Мої побігли у підвал - вдруге.
Онук спросоння плутався - пішов від дверей.

7:15. Газ є, акумулятор 12,3 вольта.

\ii{05_03_2023.fb.kipcharskij_viktor.mariupol.1.r_k_tomu_bulo_take__.pic.1}

Після 9:00 сказали, що буде "зелений" коридор: Маріуполь, Нікольське, Розівка і
далі на Запоріжжя. Виїзд автобусів об 11:00. Своїм ходом - до 16:00. Я метнувся
у гараж зливати з Дев'ятки бензин. Від'єднав шланг до бензонасосу і натягнув
трубку, яку вдома зняв з клапану бойлера. Ще якусь трубку на ганчірку натягнув
на штуцер "обратки". Стоячи в цікавій позі дмухав у другу трубку, витискаючи
бензин - добре що коли возився з газом, заправився. Погано, що не можна забрати
газ, бо невідомо, чи вистачить  палива нам на дорогу. Десь дуже близько гучно
гупало так, що аж вуха закладало. З носа пішло кров - мабуть, від натуги. А
може - трохи "приглушило". Бензин зливав у каністри з-під тосола, омивайки і
таке інше (а якби я не зберігав усякий мотлох?). Об 12:25 повернувся до
будинку. Наспіх хапали речі та виносили. Оля роздавала сусідам овочі, картоплю,
риб'ячий суп і інше.  Сусіди запропонували їхати разом - виїзд о 13:00. Десь о
12:35 прийшли свати і сказали, що коридору не буде, тож на блок-посту на
випускають.

Почали заносити найнеобхідніші речі. Сусіди приносили роздану їжу. Сусідка з
4-го поверху дуже вибачалась, що встигла нагодувати супом хлопця з п'ятого -
тому повертає не весь суп.

\ii{05_03_2023.fb.kipcharskij_viktor.mariupol.1.r_k_tomu_bulo_take__.pic.2}

Частково занесли речі - в мене "сіли батарейки". У Олі теж.

У цій метушні якось не звернули уваги на те,  що мародери почали "чистити"
магазини та аптеки - тягнуть геть усе: оселедці мариновані у діжках з кригою,
розрубані навпіл свинячі голови, калорифери, електробатареї, одяг, взуття -
все, що було в магазинах. З натугою тягнуть магазинні візки. Почало гучно
гупати: дві тітки бігли через наш двір з пральною машинкою!!! На руках несли
пральну машинку!!!

Для мене день закінчився...

Навздогін:

Раніше (день на другий чи третій) почали зупиняти доменні печі заводів, тому
спочатку вони перестали шуміти, а потім і димити. Тому за вікном от такий
незвичний пейзаж: завод без диму...

Для порівняння додав фото заводу з пилюкою

%\ii{05_03_2023.fb.kipcharskij_viktor.mariupol.1.r_k_tomu_bulo_take__.cmt}
