% vim: keymap=russian-jcukenwin
%%beginhead 
 
%%file 07_07_2020.fb.opzzh.1.jazyk
%%parent 07_07_2020
 
%%url https://www.facebook.com/OppositionPlatformForLife/posts/2837369116484124
 
%%author 
%%author_id 
%%author_url 
 
%%tags 
%%title 
 
%%endhead 

\subsection{Право говорить, учиться и получать информацию на родном языке — неотъемлемые естественные права каждого человека}
\label{sec:07_07_2020.fb.opzzh.1.jazyk}
\Purl{https://www.facebook.com/OppositionPlatformForLife/posts/2837369116484124}

Право говорить, учиться и получать информацию на родном языке — неотъемлемые естественные права каждого человека. В демократическом государстве их обеспечение является обязанностью для власти. Но в Украине Зеленского в основе языковой и культурной политики лежат не права и Конституция, а националистические догмы и прихоть чиновника. 
Сегодня по данным социологических исследований в Украине 54\% граждан постоянно говорят на русском языке или используют его в быту наряду с украинским. Но ведущий член Зе-команды Алексей Данилов считает, что вторым языком в Украине должен быть не русский, а английский. Об этом он заявил в интервью одному из украинских СМИ. Причем, использование английского языка для украинских граждан он считает обязательным. Видимо, чтобы легче было понимать желания кредиторов и новых хозяев украинских предприятий и земли. 
Позиция Данилова – свидетельство откровенно антидемократической и дискриминационной политики, которую нынешняя власть проводит в отношении русскоязычных граждан Украины. На выборах Владимир Зеленский обещал «не давить» русский язык. Президент Зеленский был готов отменить языковую дискриминацию и обеспечить права русскоязычных граждан. Еще летом 2019 года допускал особый статус русского языка для Донецка и Луганска. Но сегодня давление на русский язык только усиливается. Принятые Зе-командой и подписанные президентом Зеленским законы превращают украинцев, говорящих по-русски, в людей второго сорта. 
ОППОЗИЦИОННАЯ ПЛАТФОРМА – ЗА ЖИЗНЬ требует прекратить языковую дискриминацию. Обеспечить языковые и культурные права русскоязычных украинских граждан. Внести изменения в законодательство, чтобы гарантировать право свободно использовать русский и языки других национальных общин в культурной и общественной жизни Украины. Обеспечить право получать образование на русском языке для детей, по желанию их родителей. Прекратить политику принуждения, штрафов и наказаний в языковой и культурной политике.
Не Данилову или любому другому чиновнику решать на каких языках общаться гражданам Украины. Государственные служащие обязаны соблюдать Конституцию, а не навязывать свои фантазии и догмы, пользуясь служебным положением.
Наша политическая сила уже внесла в парламент проекты Законов в языковой, гуманитарной и образовательной сфере, которые восстановят права русскоязычных украинцев и представителей других национальных общин. Мы будем добиваться их принятия, проведения гармоничной и демократичной языковой политики в Украине.
Обеспечение прав русскоязычных украинских граждан – требование Конституции!


\ifcmt
  pic https://scontent-yyz1-1.xx.fbcdn.net/v/t1.6435-9/107686165_2837369083150794_6814997555157556824_n.jpg?_nc_cat=101&ccb=1-3&_nc_sid=8bfeb9&_nc_ohc=QmvRCdhD-GUAX8LKbdw&_nc_ht=scontent-yyz1-1.xx&oh=51a5a72c83982f371010275a4e7f6bf1&oe=60902325
\fi

