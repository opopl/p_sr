% vim: keymap=russian-jcukenwin
%%beginhead 
 
%%file 14_04_2021.fb.bilchenko_evgenia.1.aprel_vesna_stihi_dobro
%%parent 14_04_2021
 
%%url https://www.facebook.com/yevzhik/posts/3803885372979878
 
%%author 
%%author_id 
%%author_url 
 
%%tags 
%%title 
 
%%endhead 

\subsection{Доброе утро! Пусть апрель не будет серым! Доброты!}
\label{sec:14_04_2021.fb.bilchenko_evgenia.1.aprel_vesna_stihi_dobro}
\Purl{https://www.facebook.com/yevzhik/posts/3803885372979878}

\ifcmt
  pic https://scontent-bos3-1.xx.fbcdn.net/v/t1.6435-9/172941956_3803885249646557_4003605051244040086_n.jpg?_nc_cat=105&ccb=1-3&_nc_sid=8bfeb9&_nc_ohc=tk-c1TX64-oAX_PueKL&_nc_ht=scontent-bos3-1.xx&oh=d7d0098e1e8922710d2744128b8ddb1f&oe=609D284E
\fi

Завтра встану – за окном весна.
Половина неба – в молоке.
Мысли, как подростки, налегке.
На душе, как в детстве, – тишина.

Отступает время трезвых слов.
Наступает время пьяных снов.
С голых веток каплет Благодать –
Не могу ни думать, ни гадать.

Не мешает даже суета:
Я сегодня мирная почти.
Потому что на своем пути
Повстречала рыжего кота.

Он соседкину селедку жрал
И вполне доволен был собой.
На уме – божественный аврал
Под банальным именем «любовь».

Половина – в вечных голубях.
Половина – в быстротечных днях.
Половина неба – у тебя.
Половина неба – у меня.

(Из страшно деЦкого).
