% vim: keymap=russian-jcukenwin
%%beginhead 
 
%%file 08_09_2021.fb.neborakova_bogdana.1.jazyk_institut_francia_kursy.cmt
%%parent 08_09_2021.fb.neborakova_bogdana.1.jazyk_institut_francia_kursy
 
%%url 
 
%%author_id 
%%date 
 
%%tags 
%%title 
 
%%endhead 
\subsubsection{Коментарі}

\begin{itemize} % {
\iusr{Евгения Кужавская}

трешова ситуація, я теж термінологію російською знаю гірше в рази і не хочу
вчитись російською. Але і розумію, що вибороти право і далі ходити туди з
задоволенням в мене б не вийшло. Співчуваю

\iusr{Yaryna Tsymbal}

Сьогодні вранці Ліда вперше сказала, що діти в групі питають її, чому вона
говорить українською. Мабуть, починається( Я вже думала, що не доведеться
боротися, аж ні.

\begin{itemize} % {
\iusr{Юрій Марченко}
\textbf{Yaryna Tsymbal} Welcome on board!

\iusr{Yaryna Tsymbal}
\textbf{Юрій Марченко} Це замість підтримати?

\iusr{Юрій Марченко}
\textbf{Yaryna Tsymbal} 

Так це і є підтримка – ти не сама, "нас багато, і нас не подолати". Ми все це
проходили. І лише минулого року (7-й клас) – бінго! Однокласники почали вголос
казати Мирославу, "хорошо тєбє, ти так добре українську знаєш!". То ж Ліда може
відповідати, що вона говорить українською і буде в школі першою ученицею, а їм
списати не дасть!

\iusr{Yaryna Tsymbal}
\textbf{Юрій Марченко} 

Коли Ліда навчиться вимовляти кілька звуків, то вона так говоритиме, що й
дорослим фору дасть  @igg{fbicon.wink}  Вона вже поправляла шкільну вчительку укрмови, сусідку
її бабусі в Дяківцях. Почула, що та вживає рос. слова, і сказала: Треба казати
не так-то, а так!

\iusr{Юрій Марченко}
\textbf{Yaryna Tsymbal} Чи поправляти вчителів – це окрема тема...

\iusr{Yaryna Tsymbal}
\textbf{Юрій Марченко} Поки що Ліда вважає, що правильно так, як вона каже  @igg{fbicon.wink} 

\iusr{Bohdanna Dmytriv}
Ось і хто кого дискримінує в Україні....це жах один !!!!

\iusr{Маргарита Єгорченко}
\textbf{Yaryna Tsymbal} переходьте в наш садочок) здається, у нас таке неможливо
\end{itemize} % }

\iusr{Marko Halanevych}
Російсько-французький інститут в Україні)))

\iusr{Марта Госовська}

Я так само відмовилася від занять з французької у школі Лазур, бо українською
там навіть ніхто не намагався пояснювати, аргументуючи тим, що це міжнародна
група і всі і так розуміють російську. Але чому я зі свого боку маю робити
подвійні зусилля, а школа – робити, як їй зручно, нехтуючи моїми правами
клієнта? На це мені не відповіли, то ми й попрощалися на не дуже приємній ноті.

\iusr{Vira Pankiv}

У мене схожий досвід був зі школою акторського мистецтва. Коли і викладачі, і
вся група говорили російською, я на цьому фоні виділялась і це ставало умовною
стіною. Просто стало некомфортно і я в якийсь момент зійшла з дистанції

\iusr{Oleh Kotsarev}

Як би це не було кепсько, все-таки добре, що хоч якісь викладачі й групи є  @igg{fbicon.smile} 
Зважаючи на бекграунди-контексти, могло б не бути й цього  @igg{fbicon.smile} 

\iusr{Marko Halanevych}
Російсько-французький інститут в Україні)))

\iusr{Марта Госовська}

Я так само відмовилася від занять з французької у школі Лазур, бо українською
там навіть ніхто не намагався пояснювати, аргументуючи тим, що це міжнародна
група і всі і так розуміють російську. Але чому я зі свого боку маю робити
подвійні зусилля, а школа – робити, як їй зручно, нехтуючи моїми правами
клієнта? На це мені не відповіли, то ми й попрощалися на не дуже приємній ноті.

\iusr{Vira Pankiv}

У мене схожий досвід був зі школою акторського мистецтва. Коли і викладачі, і
вся група говорили російською, я на цьому фоні виділялась і це ставало умовною
стіною. Просто стало некомфортно і я в якийсь момент зійшла з дистанції

\iusr{Oleh Kotsarev}

Як би це не було кепсько, все-таки добре, що хоч якісь викладачі й групи є  @igg{fbicon.smile} 
Зважаючи на бекграунди-контексти, могло б не бути й цього  @igg{fbicon.smile} 

\begin{itemize} % {
\iusr{Bohdana Neborak}
\textbf{Oleh Kotsarev} знаєш, водночас мені гірко, що в наших головах є це формулювання – а могло би бути гірше)))

\iusr{Oleh Kotsarev}
Волію радіти навіть меншим, ніж хотілося б, досягненням  @igg{fbicon.wink} 
\end{itemize} % }

\iusr{Pavlo Shved}

Цікава мовна політика в \textbf{Французький інститут в Україні / Institut français
d'Ukraine} Біда-печаль. Цікаво, що думає з цього приводу посол Франції в
Україні. Йому такі практики за французькі гроші норм чи все таки нє? Якщо ні,
то хотілося б бачити реакцію!

\iusr{Yana Sotnyk}

Мені за рік вивчення французької в Києві жодного разу не трапився викладач,
який би давав пояснення українською. Моя рідна мова російська, тож для мене це
не було проблемою, але разом з тим, якби викладач був україномовний, я була би
тільки рада. Але в цій сфері з цим чомусь традиційно проблемно. В кращому
випадку, вони відповідають українською, якщо ставити питання українською. Не
розумію, чому так. Адже більшість з них, це молодь, яка випускається з
українських університетів.  @igg{fbicon.shrug} 

\iusr{Viktoriya Ma}

От прям в цю конкретну секунду стикаюся з аналогічною ситуацією. в Музеї-аптеці
екскурсовод відмовилася для учнів вести екскурсію українською, бо вона
російськомовна. Я написала в книзі побажань , щоб дотримувалися Закону України.
Слів нема!

\begin{itemize} % {
\iusr{Olga Pogynaiko}
\textbf{Viktoriya Ma} Може, почати тегати уповноваженого із захисту укрмови?

\iusr{Viktoriya Ma}
\textbf{Olga Pogynaiko} А є такий?

\iusr{Olga Pogynaiko}
\textbf{Viktoriya Ma} є. \textbf{Уповноважений із захисту державної мови - Тарас Кремінь}

\iusr{Ярослав Попович}
\textbf{Viktoriya Ma} Ось тут слід заповнити \par\url{https://mova-ombudsman.gov.ua/povidomiti-pro-porushennya}

\iusr{Сергей Лунин}

Це не закон, а псевдозакон, що суперечить щонайменше шести статтям Конституції,
та одній — кримінального кодексу.  Сподіваюся, справа дійде колись до ЄСПЛ і
українська держава цілком заслуженно зганьбиться.

\end{itemize} % }

\iusr{Olga Pogynaiko}

Найцікавіше, що, очевидно, в них нічого не зміниться, поки не станеться гучного
скандалу.

\begin{itemize} % {
\iusr{Bohdana Neborak}
\textbf{Olga Pogynaiko} з мого боку його не станеться, бо «доказів» я не збирала

\iusr{Olga Pogynaiko}
\textbf{Bohdana Neborak} мислиш як юрист)) Але для скандалу вони не дуже й треба. Тут грає роль story, а не докази  @igg{fbicon.smile} 
\end{itemize} % }

\iusr{Євгенія Завалій}

І це Київ( у мене взагалі в Одесі часто буває запрошують до дітей щоб
презентувати книжку, а тоді кажуть - а ви можете російською, а у мене ж книжка
українською. А ви що не можете перекласти? Ну от я можу перекласти, але на
хочу. І я тоді почула: ви неадекватна(((

\begin{itemize} % {
\iusr{Yana Sotnyk}
\textbf{Yevgenia Zavaliy} О боже!

\iusr{Bohdana Neborak}
\textbf{Yevgenia Zavaliy} це, звісно, взагалі жесть

\iusr{Євгенія Завалій}
\textbf{Yana Sotnyk} так уяви що не один раз  @igg{fbicon.man.facepalming}  і це до того, що у нас тут вважають, що письменниця має бути клоуном, співати, малювати і розважати всіх. І всі роблять велику послугу, що тебе вислуховують  @igg{fbicon.shrug}  але ну ми маємо працювати в тих умовах які є.

\iusr{Yana Sotnyk}
\textbf{Yevgenia Zavaliy} Мене теж регулярно питають, чи є книга російською. І дуже дивуються, що нема. Ти молодець, що все це робиш.
\end{itemize} % }

\iusr{Iryna Tsilyk}

Все це просто неймовірно дістало. Ти молодець, що підняла цю тему. Чудово
розумію тебе, що не хочеться псувати стосунки з викладачами, але скільки ж
можна все це терпіти (

\iusr{Ірина Забіяка}
Офтоп, але - мало би бути все французькою, від початку.

\begin{itemize} % {
\iusr{Bohdana Neborak}
\textbf{Ірина Забіяка} 

напевно, хоча я не знаю, бо ходила на курси лише німецької та англійської на
значно вищі рівні, де все було іноземною. Насправді викладачі старалися робити
багато що французькою, але частина пояснень таки відбувається
українською/російською

\iusr{Hanna Kyriyenko}
\textbf{Ірина Забіяка} є різні методики для початківців.

\iusr{Natalia Slipenko}
Необов'язково. Для початківців такий метод не завжди ефективний.

\iusr{Ірина Забіяка}

Я ходила у Фр.інститут саме з таким викладачем, це важко, але ефективно. Є й
інші точки зору, але я - як викладач іншої іноземної- переконана, що так краще.
Не легше, але краще.


\iusr{Marina Marchenko}
\textbf{Ірина Забіяка} Я, наприклад, зараз арабську мову вивчаю. Можете собі уявити все арабською від початку?

\iusr{Ірина Забіяка}
\textbf{Marina Marchenko} можу.

\iusr{Marina Marchenko}
\textbf{Ірина Забіяка} وفقك الله

\iusr{Natalia Tovarisch Chi}
\textbf{Ірина Забіяка} 

у мене так і було 11 років тому. викладач, як виявилося десь наприкінці
семестру, був російськомовний, але їм було заборонено використовувати будь-яку
іншу мову, крім французької. спочатку у всіх у групі був сильний шок, але через
пару уроків звикли. зате я таким чином дійшла до впевненого В1 менше, ніж за
рік

\end{itemize} % }

\iusr{Nadiika Pototska}

Як же це дратує  @igg{fbicon.face.shoked.head.exploding}{repeat=2}  щоб у Матвія була повністю (і дійсно) україномовна школа —
ми щодня возимо його з Осокорків на Поділ. А про різні дитячі курси взагалі
мовчу. Ще той квест із перешкодами

\begin{itemize} % {
\iusr{Олена Мохник}
\textbf{Nadiika Pototska} курси для дітей - це постійний стрес для дітей пояснювати, що вони не розуміють мову ворога. «Как ета нє панімают?»

\iusr{Halya Kerosina Shyyan}
\textbf{Nadiika Pototska} а куди на Поділ?
\end{itemize} % }

\iusr{Hanna Kyriyenko}

Раніше могли прикриватися, що більшість клієнтів хоче російською (насправді,
ні). Зараз є закон. \textbf{Французький інститут в Україні / Institut français
d'Ukraine} порушує законодавство України. Все.


\iusr{Nadiika Pototska}

А про французьку — подивись у Olga Shurova. Вона саме відкрила школу) сайт та
інші деталі там точно українською \url{https://musiquedelangue.com}

А Оля дуже крута

\begin{itemize} % {
\iusr{Ольга Шурова}
Дякую! @igg{fbicon.heart.red}
У мене поки що лише курс для початківців, але згодом будуть і інші рівні.
Дуже розумію Богдану, це у нас всюди так, на жаль( для мене було принципово свій курс вести саме французькою та українською. На закиди знайомих, що я таким чином відсіюю величезну СНД-аудиторію, я відповідала, що мені вони не потрібні) я щиро вважаю, що потрібно розвивати україномовний продукт, а не гнатися за рублями 
\end{itemize} % }

\iusr{Halyna Herasym}

Я якраз думала тобі написати про курси альянс франсез у Львові, я ними
назвичайно задоволена і викладачки там просто чудові. Мені важко давався
онлайн-формат в групі, тож я зрештою пішла до окремої репетиторки і займаюсь
індивідуально, але це в жодному разі не провина викладачок.

\textbf{Alliance française de Lviv}, мої вам сердечка.

\begin{itemize} % {
\iusr{Bohdana Neborak}
\textbf{Halyna Herasym} мерсі, з понеділка починаю)

\iusr{Constantin Nelep}
\textbf{Bohdana Neborak} Bon courage  @igg{fbicon.face.smiling.eyes.smiling} 
\end{itemize} % }

\iusr{Oksana Karpiuk}

Ходила туди у 2012-2013. Троє різних викладачів і всі україномовні за
замовчуванням. Навіть не думала, що там такі проблеми можуть бути. Дуже цікаво
почути, це мені щастило, чи у Інституту останніми роками змінилися
"налаштування"?

\begin{itemize} % {
\iusr{Bohdana Neborak}
\textbf{Oksana Karpiuk} можливо, це стосується мого рівня – а11, а12. я пишу в пості про них

\iusr{Oksana Karpiuk}
\textbf{Bohdana Neborak} я теж на початкові ходила, правда прізвищ викладачів не згадаю вже. Зрештою, пройшло таки досить багато часу.

\iusr{Ivanka Chupak}
\textbf{Oksana Karpiuk} я теж ходила туди у 2012-2013 і теж закинула і демотивувалась саме через російськомовну викладачку, якої зовсім не розуміла. Мого рівня вечірніх груп більше не було на той час. Так ще туди і не повернулась через брак часу, але теж не витримала б, якби були пояснення російською саме через незнання аніскілечки усіх граматичних структур російською. Та й, загалом, wtf?!((

\iusr{Oksana Karpiuk}
\textbf{Ivanka Chupak} оскільки в мене були заняття до обіду, то може україномовних викладачів якраз на менш популярний денний час перекидали, а російськомовних на активніші і більш затребувані вечірні години? Хоча це вже якась теорія змови виходить...

\iusr{Ivanka Chupak}
\textbf{Oksana Karpiuk} не думаю, що там прям щось було ''задумано''. Просто так сталося і співпало, напевно. Але факт залишається фактом, на жаль

\iusr{Natalia Hlotova}
\textbf{Oksana Karpiuk} чоловік ходив туди на ранковий інтенсив у 2016, викладач був родом з Сенегалу, вже довго жив в Україні, навіть вуз тут закінчив. Але українською не володів.
\end{itemize} % }

\iusr{Roksolyana Iskra}
Стало дуже сумно від прочитаного. @igg{fbicon.face.pleading} 

\iusr{Radoslava Kabachiy}
Це жесть!! Bohdana , про це не лише можна, про це ТРЕБА говорити!! Бо це жесть!!

\iusr{Olexander Shtepan}
Максимально чесна позиція. Правильно, що пишете й комунікуєте про це. Україна —
не російська корпорація. Україна також і для українців.

\iusr{Oksana Schur}

У мене була така ситуація з німецькою в Гете. Я підняла руку і сказала: так, я
одна проти того, щоб мати пояснення російською. І це виявилося ефективним )

\begin{itemize} % {
\iusr{Bohdana Neborak}
\textbf{Oksana Schur} у мене в Гете не було російської взагалі, до речі
\end{itemize} % }

\iusr{Анна Процук}
Та блін, у мене російська перша і був курс російської в універі, а тієї термінології теж не розумію.

\iusr{Jan Navrátil}
\textbf{Michal Lebduška} Ось Тобі гарне (а насправді сумне) продовження (і один
із безлічі прикладів на які можна натрапити щодня) нашої давньої дискусії про
мовну ситуацію в Україні, або "хто тут насправді до кого мерзенний
(ненависний)", на яку Ти так і досі мені не відповів...

\iusr{Olia Barnett}

Так само відмовилася від більше сотні компаній викладацьких , туристичних і
комерційних які хотіли надавати мені постуги в Україні російською мовою .

\iusr{Стефанія Яцків}

Вам хоч не кричать в спину - в очі теж часто: то ж було як закон - «мало ми вас
убівалі бандєровцев». Я за майже сорок років свого буття киянкою наслухалась.
Наболілась.

Тепер ж впевнено заходжу в супермаркети, бо Закон на моїм боці!

Сльози ... стискають (чомусь) груди, як згадую.. Сентиментальність! Хай загине і слід...

Вам успіхів й успіхів, дорога панно Богданко @igg{fbicon.face.happy.two.hands} 

\begin{itemize} % {
\iusr{Bohdana Neborak}
\textbf{Стефанія Яцків} який жах, пані Стефаніє, мені дуже прикро читати
\end{itemize} % }

\iusr{Тарас Кремінь}
Богдано, напишіть нам про це:\par 
\url{https://mova-ombudsman.gov.ua/povidomiti-pro-porushennya}

\begin{itemize} % {
\iusr{Bohdana Neborak}
\textbf{Тарас Кремінь} дякую, я сьогодні напишу

\iusr{Владимир Токач}
Пане Тарасе, може одним махом надіслати всім мовним курсам?
Можна з реєстру отримати контакти всіх СПД, що надають послуги по відповідних КВЕДах і надіслати їм нагадування про мову викладання електронкою та/або на мобільні, ці дані надають при реєстрації підприємства.

\iusr{Тарас Кремінь}
\textbf{Volodymyr Tokach}, звичайно можна! На жаль, не всі курси є офіційними.

\iusr{Владимир Токач}
Усе ж багато хто зареєстрований, особливо великі.
\end{itemize} % }

\iusr{Зоя Звиняцковская}

" не відчувала себе нав’язливою надокучливою проблемою" - це дуже точне
визначення відчуттів людини, яка хоче в нас щось зробити в законним спосіб в
колнктиві. дуже добре відчувається в батьківських спільнотах. ц6 просто ключове
визначення і найголовніша перепона для того, щоб таких людей ставало більше

\begin{itemize} % {
\iusr{Halya Kerosina Shyyan}
\textbf{Zoya Zvinyatskovskaya} але долати це в собі дуже сильно б‘є по нервовій системі, тому далеко не завжди виправдано.

\iusr{Зоя Звиняцковская}

\textbf{Halya Kerosina Shyyan} я нікого не штовхаю до цього, це особисте рішення кожного. і так, це дуже травмуючий досвід, знаю особисто. але з іншого боку - це єдиний шлях змінити щось

\iusr{Halya Kerosina Shyyan}
\textbf{Zoya Zvinyatskovskaya} та, інший вихід - ескапізм і страусизм. Теж мають дофіга побочок.

\iusr{Bohdana Neborak}
\textbf{Zoya Zvinyatskovskaya} тут ці почуттєві штуки мене найбільше виснажили. Я ходила і домовлялася з собою, своїм ставленням – щирим і своїм ставленням – виваженим і осмисленим – довго. Вивело остаточно це мене тоді, коли мені припинили відповідати

\iusr{Зоя Звиняцковская}

\textbf{Bohdana Neborak} 

та я розумію. але тут хоч порівняно просто можна піти. а коли ти замкнений в
батьківському чаті в аже даєш собі слово промовчати, але кожнаа пропозиція тм -
це просто знущання))) і вааріанти два: або ти не будеш поважати себе, або тебе
не будуть поважати інші. ммм все таке смачне...

\iusr{Bohdana Neborak}
\textbf{Zoya Zvinyatskovskaya} насправді я з жахом думала весь цей час про друзів, у яких діти в садках і школах

\iusr{Зоя Звиняцковская}

\textbf{Bohdana Neborak} на жаль

\iusr{Halya Kerosina Shyyan}
\textbf{Zoya Zvinyatskovskaya} 

я от висловлювалася в тих чатах рази три на рік, розлого і конструктивно.
Повагу, здається, здобула, при тому не таку «ліпше не чіпати», а навіть з
певною симпатією, але типу «сильно мудра і дистанціюється від колективу». Але
алхімія тих чатів така, що фоловити їх - само по собі джерело неврозу, бо є
багато такого з чим типу не особливо погоджуєшся, але можеш піти на компроміс,
бо воно а) нешкіливе; б) і так зроблять (типу лакування підлоги в класі).
Врешті, та, все дуже смаченьке і всі пересварились.


\iusr{Halya Kerosina Shyyan}
\textbf{Bohdana Neborak} ну ігнор - це ж, як відомо, найпекельніша зброя!

\iusr{Vera Karasjova}
\textbf{Zoya Zvinyatskovskaya} є ще третій варіант - стати ворогом народу, бо "доводиш до нервових зривів вчительку англійського, якій важко вивчити українську(????), і знущаєшся над людиною, ризикуючи її здоров'ям"((

\end{itemize} % }

\iusr{Tetiana SmiLetko}
Підтримую.

\iusr{Anatoliy Dnistrovyi}
класика жанру майже на всіх мовних курсах у Києві, до речі, в автошколах подібна історія

\begin{itemize} % {
\iusr{Ганна Улюра}
\textbf{Anatoliy Dnistrovyi} а як іспити складати потім? Правила ж і терміни відповідні українською

\iusr{Оксана Цюпа}
\textbf{Ganna Uliura Бугага}, якось завели про важкість медикам читати лекції українською, бо терміни. Коли я спитала, якою ж мовою студентам дипломи писати, то це не викликало у відписувачів когнітивного дисонансу, бо писати треба українською.

\iusr{Halya Kerosina Shyyan}
\textbf{Anatoliy Dnistrovyi} ну в автошколах «газку давай!», то ше такоє. ;))))

\iusr{Vera Karasjova}
\textbf{Anatoliy Dnistrovyi} 

мене особисто мовні курси набагато більше тригерять, бо там україномовні в
жорстокому напрязі не лише через саму ситуацію відокремлення від групи.
Дорослим ще якось, а от дітям доводиться одночасно вчити фактично дві мови - бо
мови такі "однакові-однакові", а от "швидкий-fast", не знаючи "бьістрьій", на
рос.мовних курсах дитина фіг вивчить(

\end{itemize} % }

\iusr{Natalia Vasura}

Питання ще і в конкретній установі - це єдині курси за все моє життя (а вчилася
я багато  @igg{fbicon.face.tears.of.joy} ), які я просто покинула на передостанньому занятті

\begin{itemize} % {
\iusr{Bohdana Neborak}
\textbf{Natalia Vasura} ого
\end{itemize} % }

\iusr{Dzvinka Pinchuk}

Мала подібний неприємний досвід з цими ж курсами чотири роки тому. Окрім того,
рівень викладання був доволі посередній. Засмутилась страшенно. Дякую, що ти
підняла цю тему.

\begin{itemize} % {
\iusr{Bohdana Neborak}
\textbf{Dzvinka Pinchuk} мені прикро, Дзвінко
\end{itemize} % }

\iusr{Bohdana Neborak}
\textbf{Dzvinka Pinchuk} мені прикро, Дзвінко

\begin{itemize} % {
\iusr{Віра Семенова}
я вчора на власні оці вуха чула футбольного тренера українською! років 25 йому на вигляд, малих тренує у мене під домом.

\iusr{Ольга Спіжова}
\textbf{Віра Семенова} о, а де саме? Буду вишукувати контакти!

\iusr{Віра Семенова}
\textbf{Ольга Спіжова}, можу дізнатись. біля 259 школи.

\iusr{Ольга Спіжова}
\textbf{Віра Семенова} о, ще й моя рідненька школа! Дізнайся будь ласка, буду вдячна дуже
\end{itemize} % }

\iusr{Іванна Кобєлєва}
Тегнули вже \textbf{Уповноважений із захисту державної мови}?

\iusr{Julia Galeta}

це так дико, що люди реально не розуміють, що хтось може не вловлювати "падєжи
і спряженія", бо ніколи не вчив/ла в україномовній школі цієї термінології. а
ще їхній відділ курсів славиться своїми проблемами із комунікцією, тому не
отримати відповідь на лист або запит ніколи або запізно - це фішка, про яку я
чула неодноразово, та й сама стикалась

\iusr{Alisa Bondarenko}

Я навчалась там кілька років тому, загалом мені подобалось. Але там була одна
викладачка, з якою в мене пов'язані змішані почуття. Зі мною в группі навчалися
"хлопчики" з гурту "Авіатор" і вся увага (іноді аж занадто улеслива) вчительки
була зосереджена на них — чи то тому що "зірки", чи то тому що це чоловіки в
переважно жіночій групі. от такий дивний непрофесійний був підхід в неї). Але
потім я змінила викладачів і була цілком задоволена.

\iusr{Mariana Khemii}

Мала схожий досвід на курсах англійської. Добре, що побувала на пробному, бо
російські терміни востаннє чула у 5-му класі, тому, звісно, нічого не розуміла.
Пішла на приватні заняття


\iusr{Oksana Semenik}

Не розумію чому на курсі копірайтингу в Україні, де мова є інструментом роботи,
досі мова викладання та роботи російська  @igg{fbicon.shrug} 

\begin{itemize} % {
\iusr{Olena Sharhovska}
\textbf{Oksana Semenik} теж не розумію. А коли підняла це питання в себе на сторінці - виявилось, що таких більшість

\iusr{Mariana Khemii}
\textbf{Oksana Semenik} або формально українська, і в описі чи відео курсів купа помилок  @igg{fbicon.monkey.see.no.evil}  Страшно уявити, чого там вчать
\end{itemize} % }

\iusr{Olena Pavlova}
Богдано, ти молодець. з такими, як ти, ми неодмінно переможемо

\iusr{Біда Софія}
Bonne chance!

\iusr{Halya Kerosina Shyyan}

З мовною термінологією це все взагалі дуже дивно. В школах ж її давно не
вивчають і уже декілька поколінь, навіть російськомовних, по ідеї не оперує
глаголами и спряжениями, якщо не має окремо уроків російської. Тобто навіть
викладачі мали б не дуже де мати цьому понавчовуватися, якщо їм менше 45-ти і
вони росли в Україні. Хіба то нейтіви, але ці б ні на яку не мали переходити
тоді.

\begin{itemize} % {
\iusr{Bohdana Neborak}
\textbf{Halya Kerosina Shyyan} мені важливо співставляти з категоріями і термінами української мови, до речі, я так починаю розуміти мову в принципі

\iusr{Halya Kerosina Shyyan}
\textbf{Bohdana Neborak} 

це от, до речі, теж цікава фішка, яку дуже добре розумію, як філолог. Але у
нефілологічних групах де вивчала мову чітко помічала, що більшості воно тільки
ускладнює процес, особливо на початкових рівнях. Моя улюблена цитата
одногрупніці на німецькій в Гьоте, коли було узгодження ім/прикм: «то то до
того не відноситься?!».

Коротше, трохи не по темі, але для загальних груп того таки, мабуть, має бути
мінімум, а таким, як ми, або приватні уроки/роз‘яснення (не третьою мовою
obviously), або філологія. По темі ж, до речі, якщо вже вчити французьку через
третю мову, то таки через англійську. Зловила себе на тому, що в гугл
транслейті завжди англ. пишу, коли треба фр., але укр., коли польську. Що
природньо, бо шукаєш ту мову, у якій логіка ближча.

\iusr{Анна Цяцько}
\textbf{Halya Kerosina Shyyan} 

Мені 32, закінчувала відділення журналістики на філфаку в Харківському
національному унівверситеті. Після повністю україномовоної школи довелося зі
словником опанувати всю термінологію російською, бо окремо викладалася і
сучасна російська мова, і стилістика російської мови, і російська література.
Рік випуску — 2011. Припускаю, що мій приклад, на жаль, не поодинокий.

\iusr{Halya Kerosina Shyyan}
\textbf{Анна Цяцько} ну але це ж були окремі предмети пов‘язані, власне, з російською філологією на окремому факультеті. А на ці курси приходять люди різних фахів і значно більше шансів, що вони вчили українську граматику, а російську ні.

\iusr{Анна Цяцько}
\textbf{Halya Kerosina Shyyan} я мала на увазі саме викладачів і тезу про те, що їм має бути не менше 45-ти

\iusr{Olena Derevska}

Мені якраз 45, і я вчилася у російській школі (українська як друга мова була).
Я добре пам'ятаю, в принципі, подвійну термінологію. Але, що цікаво, зараз
дитині пояснюю ту ж українську і не можу згадати іноді російські еквіваленти.
Основна мова спілкування витісняє іноді навіть першу. Щодо неможливості
викладачу "заговорити" українською термінологію, то це така маячня. Я в 1999 р.
викладала англійську мову діткам на курсах при церкві. І робоча мова була
українська. Хоча я до того мала кількарічний досвід репетиторства і викладання
на курсах в форматі англійська/російська, для мене не було жодної проблеми
перебудувати свої матеріали і своє викладання на українську термінологію! Люди,
які в Україні заявляють про неможливість перебудування, лицеміри і яскраві
ненависники всього українського. Ну тільки якщо їм не 60 і вони дійсно
українську не вивчали. Але тут вже вступає в діло закон і вимога адміністрацій
курсів його дотримуватися.

\end{itemize} % }

\iusr{Любов Якимчук}
Співчуваю. Добре, що на наступному рівні вже можна буде вчити французьку французькою. Але це все жесть, звісно.

\iusr{Andriy Lyubka}

я взагалі не розуміюся в тій лексиці - падєжи, спряженіє, глагол, тобто навіть
за бажання я б не зміг через російську вчити іншу мову, бо для цього мусив би
для початку професійно опанувати російську. вапрос: зачем?  @igg{fbicon.smile} 

\begin{itemize} % {
\iusr{Bohdana Neborak}
\textbf{Andriy Lyubka} так, мені також нічого це робити
\end{itemize} % }

\iusr{Andriy Borodavko}
\textbf{Французький інститут в Україні / Institut français d'Ukraine}
Скажіть, будь ласка, а чому ви дискримінує те українців? Ви сповідуєте нацистські погляди, чи ксенофобські?

\iusr{Елизавета Цареградская}

просто рускій спасать надо, он уміраєт. І найжахливіше - шо все ніяк не умрьот.

\begin{itemize} % {
\iusr{Olena Voronetska}
\textbf{Єлизавета Цареградська} то може добить, щоб не мучився?)
\end{itemize} % }

\iusr{Оксана Плахотнюк}

Маю абсолютно ідентичну ситуацію, тому в разі чого буду вдячна за контакт
хорошого викладача або викладачки французької  @igg{fbicon.face.relieved} @igg{fbicon.heart.suit}

\begin{itemize} % {
\iusr{Bohdana Neborak}
\textbf{Oksana Plakhotniuk} у понеділок починається дистанційний курс у французькому альянсі (Львів), але А2. проте мені хвалили там усіх – тому я вирішила оплатити 1 заняття, аби зрозуміти рівень групи і тд. напишу контакт тобі у приват

\iusr{Оксана Плахотнюк}
\textbf{Bohdana Neborak} супер, дуже дякую!

\iusr{Liliya Dechkoba}
\textbf{Оксана Плахотнюк} \textbf{Bohdana Neborak} Підтримую вашу позицію! Пропоную свою допомогу з французькою граматикою. Зі свого досвіду знаю, що таки набагато ефективніше пояснити/засвоїти базову структуру рідною мовою.

\iusr{Маріам Найем}
\textbf{Anna Kaliukh} якщо що ідеальна викладачка)

\end{itemize} % }

\iusr{Gennadiy Gnyp}
Правильно робиш, що виносиш це у публічну площину

\iusr{Yaryna Tsymbal}

Я вчила російську з першого класу, бо це була радянська школа, з українською
мовою навчання. Проте я впевнена, якщо ми зараз запитаємо містян/ок, які
закінчили російськомовні школи і все життя говорять російською, про різницю між
причастием і дєєпричастием, нам відповість хіба кожна десята людина!

\begin{itemize} % {
\iusr{Ганна Улюра}
\textbf{Yaryna Tsymbal} ой сонце, ти якою мовою їх не назви, про ту різницю тобі розкаже навряд і кожен десятий)

\iusr{Yaryna Tsymbal}
\textbf{Ganna Uliura} Ну, тут усі особливо плуталися, напр. Микола після своєї рос. школи.

\iusr{Роман Губа}
\textbf{Yaryna Tsymbal} і що? через це вони перестануть бути російськомовними?)

\iusr{Yaryna Tsymbal}
\textbf{Роман Губа} Ну, звісно, ні, бо мені не про це йдеться. Коли ти ніколи не вчила лінгвістичну термінологію російською, це дуже ускладнює процес навчання.

\iusr{Роман Губа}
\textbf{Yaryna Tsymbal} це правда. я російськомовний, але майже не вчив російської у школі, втім я припускаю, що декому вчити іноземну мову з російськомовною викладачкою буде легше.

\iusr{Oleksandr Stukalo}
\textbf{Yaryna Tsymbal} ну, щодо дієприслівника буде те саме:))
\end{itemize} % }

\iusr{Юрій Марченко}

Читаю коментарі і розумію, що "Me too" по українськи – це про мовний харасмент.
Bohdana Neborak, все правильно робиш. А буде потрібна репетиторка – звертайся
 @igg{fbicon.smile} , знаю одну.

 \begin{itemize} % {
\iusr{Bohdana Neborak}
\textbf{Юрій Марченко} дякую дуже! напишу вам наступного тижня, коли зрозумію, чи зможу іти на А2, чи не витягну  @igg{fbicon.smile}  
 \end{itemize} % }

\iusr{Marina Marchenko}

Я викладаю французьку і в мене зворотна ситуація: не всі учні розуміють/хочуть
українську. А сьогодні через освітню платформу, де мене знаходять учні, написав
мені француз: "Я щойно переїхав в Україну на роботу і мені треба якнайшвидше
вивчити російську, щоб влитися в колектив, а моїй дружині треба вивчити
російську, щоб знайти тут роботу". Завіса.

\begin{itemize} % {
\iusr{Bohdana Neborak}
\textbf{Marina Marchenko} дякую, що розказали, бо це теж багато свідчить про те, як бачать зовні ситуацію окремі французи. Закон би тепер мав працювати на користь української у виборі

\iusr{Юлія Джугастрянська}
\textbf{Marina Marchenko} дякую за цей коментар, теж втомилася дивуватися цій ситуації: іноземець приїздить сюди і вчить російську, бо тут нею розмовляють.

\iusr{Halya Kerosina Shyyan}
\textbf{Юлія Джугастрянська} 

тут теж треба розуміти, що багато, хто приїжджає в будь-яку країну
пост-радянського простору бачить це, як один з епізодів кар‘єри і мову обирає
ту, яка покриє більші території в перспективі. Естонську чи грузинську ніхто не
буде вчити більше ніж із ввічливості для смолток, просто є країни де робочу
перевагу над російською може мати англійська. Коротше, це все реалії
постколоніальної епохи, коли їдучи в Індію на декількарічний контракт, навряд
чи хтось би заморочився більше, ніж англійською. Тих, кого знаю, що бачать сенс
вчити місцеву мову, або просто люблять лінгвістику, або спонукають особисті
обставини.

\iusr{Катерина Кладик}
\textbf{Marina Marchenko} 

буквально післявчора продавець (за кордоном) влаштував скандал, бо я говорила
до нього англійською, хоча він чув, що з подругою спілкуюсь українською, отже
«я можу говорити до нього російською». Звинуватив мене в тому, що я
гіперболізую політичний контекст.


\iusr{Юлія Джугастрянська}
\textbf{Halya Kerosina Shyyan} тут питання дуже просте: якою мовою спілкується з приїжджим близьке оточення - в колективі, персонал у побуті, діти з його дітьми в навчальних закладах і на ігрових майданчиках.

\iusr{Maryna Zelenyuk}
Ви йому пояснили, що в Україні краще вивчити українську?

\iusr{Вікторія Таланко}
В україні всі знають українську, жоден іноземець тут з українською мовою не пропадав. Тому це суто питання поваги до нас і до нашої мови.

\iusr{Olena Derevska}
\textbf{Kateryna Kladyk} я б адміністратора викликала або, якщо в тому гадюшнику нема адміністратора, ославила б їх на весь ФБ, бо така хамська поведінка не личить сфері обслуговування - домислювати за клієнта, що і як тому робити і розуміти.

\iusr{Halya Kerosina Shyyan}
\textbf{Юлія Джугастрянська} 

звісно. І, ясна річ, є дуже різні історії іноземців в Україні. Але, якщо трохи
узагальнити, ті, хто приїжджають на пару років попрацювати у великій компанії,
міжнародній організації, на дипслужбу, переважно варяться в доволі герметичному
середовищі, а діти їхні, якщо сім‘я не мішана, де один(а) з батьків місцеві
ходять в англомовні школи для експатів з такими ж дітьми. І з ними ж тусуються.
Самі ж іноземці зазвичай обдаровані тим привілеєм, що з ними спілкуються мовою,
які ті розуміють, зазвичай таким собі globish. Повторюсь, це просто дві сторони
медалі сучасного субтильного колонізму і неуникного глобалізму.


\end{itemize} % }

\iusr{Оксана Цюпа}
Обнімаю, співчуваю. Бажаю вивчити французьку без зайвих нервів та утисків

\iusr{Olga Tokariuk}
обіймаю і співчуваю, дуже відгукується ваш пост

\iusr{Valentyna Klymenko}

ще є прекрасний варіант, коли є носій мови, який вивчив російську, а українську
ні. і викладач хороший, а тобі некомфортно, бо треба з двох іноземних
перекладати


\iusr{Viktoria Baurdo}

Це справа не тільки французької, я не можу знайти йогу, або спортзал, курси по
дизайну, або що завгодно українською. Це дуже злить.

\begin{itemize} % {
\iusr{Oksana Gadzhiy}
\textbf{Viktoria Baurdo} йога і дизайн — дуже добре розумію. більшість залів і контенту російською. реально іноді доводиться перекладати в голові і це злить

\iusr{Kateryna Holovko}
Йога украінською це Катря Чуквінська і Марта Чорна. Теж онлайн
Спортзал. залежить де
\end{itemize} % }

\iusr{Rostyslav Nyemtsev}

Як на мене, мова найкраще засвоюється, коли її викладають тією ж мовою.
Французьку краще вчити французькою, без опори на рідну, а тим паче іноземну.
Тоді краще відбувається занурення у мовне середовище і мозок менше шукає
аналогій. Я повністю підтримую вас у тому, що викладання французької через
російську в Україні неприпустиме.

\begin{itemize} % {
\iusr{Hanna Kyriyenko}
\textbf{Rostyslav Nyemtsev} не завжди, я вважаю, що починати треба якраз з опорою на рідну. Але, звісно, кожен вибирає сам методику.

\iusr{Rostyslav Nyemtsev}
\textbf{Hanna Kyriyenko} я ж написав: як на мене. Саме так вчив французьку. І після 9 місяців зміг піти на maîtrise, а згодом і на PhD  @igg{fbicon.smile} . Думаю, тут багато залежить від викладача і власного бажання.

\iusr{Bohdana Neborak}
\textbf{Rostyslav Nyemtsev} згідна, пане Ростиславе, що це добрий спосіб вчити мову)

\iusr{Julia Harbuz-Montoya}
Вчила у Франції мову з нуля. Ні в зуб ногою. Прийшла в групу, яка вже займалася,. Викладачка говорила лише французькою. Вкінці дня я вже розуміла завдання з прочитання тексту про сніданок на 3 абзаци. Не потрібна там мова ще одна. Що не ясно-картинка втгуглі шукається.
На сайті фр телебачення. Tv5 monde є уроки французької. Відео з текстом всього сказаного і вправами до них. На всі рівні

\iusr{Natalia Tovarisch Chi}
\textbf{Rostyslav Nyemtsev} підтримую. теж мала найкращі результати саме за такого підходу. діти ж якось примудряються вивчити мову без знання граматики, причому зазвичай дуже швидко
\end{itemize} % }

\iusr{Oryna Vyshnia}

Чомусь я не дивуюся, що це сталося саме у французів... Я пів року працювала в
Новарці, спільний україно-фрагцузький проєкт, який будував Арку над ЧАЕС. Поки
- державний. Ніхто, жодна людина не говорила українською. Вся документація
включно з бухгалтерією велася трьома мовами - англійською, французькою і
російською. Там взагалі багато чого було... З тих пір я з французами не працюю.
Французи і повага - це протилежні поняття.

\iusr{Ірина Троскот}
Те саме було кілька років тому в мого малого в Гете-інституті. Завершилося тим, що відмовився вивчати мову.

\iusr{Петро Яценко}
Поволі шукаю в Києві кулінарні курси українською. І досі не знайшов.

\end{itemize} % }
