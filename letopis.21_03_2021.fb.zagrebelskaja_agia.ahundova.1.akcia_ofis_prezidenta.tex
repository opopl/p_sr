% vim: keymap=russian-jcukenwin
%%beginhead 
 
%%file 21_03_2021.fb.zagrebelskaja_agia.ahundova.1.akcia_ofis_prezidenta
%%parent 21_03_2021
 
%%url https://www.facebook.com/permalink.php?story_fbid=1822639701230493&id=100004534422805
 
%%author 
%%author_id zagrebelskaja_agia.ahundova
%%author_url 
 
%%tags ofis_prezidenta,pogrom.20_03_2021,protest,rossia,sternenko_sergei,ukraina
%%title Мені начхати, що подумають в Росії про акцію під ОП
 
%%endhead 
 
\subsection{Мені начхати, що подумають в Росії про акцію під ОП}
\label{sec:21_03_2021.fb.zagrebelskaja_agia.ahundova.1.akcia_ofis_prezidenta}
 
\Purl{https://www.facebook.com/permalink.php?story_fbid=1822639701230493&id=100004534422805}
\ifcmt
 author_begin
   author_id zagrebelskaja_agia.ahundova
 author_end
\fi

1. Мені начхати, що подумають в Росії про акцію під ОП.

2. Я не вважаю будівлю ОП символом України.

3. Щиро впевнена, що людина красить крісло, а не навпаки. Тому поваги до
інституту президенства не маю,  лише до особи, якщо звичайно вона того варта
(до речі, перші звинувачення у неповазі до інституту президенства почула від
прихильників Януковича).

4. Мирні протести тривають вже більше року. Хлопці та дівчата, як на роботу,
ходять на судові засідання, під парламент, ОП та генеральну прокуратуру. І все
безрезультатно. Глупо постійно робити одне і те саме, очікуючи на інший
результат.  

5. Президент ігнорує вимоги громадян. ОАСК працює, Портнов рішає в судах,
Татаров сидить в ОП. На ключові посади призначені друзі, сусіди, куми,
однокурсники, працівники кварталу. Більшість корупційних схем повернулися, але
вже у виконанні «нових облич». Жоден клієнт РНБО не за ґратами. Не вважаю, що
президент не знає, що робить дуже погано (вчиняє злочини та покриває злочинців)
та йому треба продовжувати пояснювати це з квіточками під ОП. Це щонайменш
принизливо, якщо звичайно є гідність. 

6. Кривосуддя працює. Вбивці Каті Гандзюк отримали від 3 до 6 років за
вбивство. Стерненко за легкі тілесні ушкодження та 300 грн - 7 років та 3
місяці. 

7. Парламент цинічно піклується про шкурні інтереси. Щоб розібратися з
розмальованою будівлею ОП та забезпечити собі спокій і недоторканість, вони
скликають позачергове засідання. Бо для них важливо, щоб як і вчора, вони
відвідували перукаря, елітні ресторани, паркували під парламентом свої дорогі
авто... та їх гармонії не загрожували маргінальні людішки. 

На прийняття закону, який розблокує процес проти вбивць Небесної сотні, місяці
як вони не мають часу. Прохання та сльози родин не допомагають. Депутати
зайняті. Не до цього зараз. 

8. ...

Цей перелік довгий. І я чесно не знаю, на що очікував президент? Що пару
вихідних ще вийдемо на вулиці, а потім якось змиримось? Що здебільшого всім
«какая разница»? Ахметов університет в Маріуполі будує, РНБО санкції накладає,
з «Дому» розповідають як ми добре живемо... і все «стерпится-слюбится»? Але ж
змиритися готові не всі. Й тих, що не готові - немало. А значить, вони будуть
шукати нові й нові можливості бути почутими.

\verb|#СтерненкуВолю|
