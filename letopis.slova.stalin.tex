% vim: keymap=russian-jcukenwin
%%beginhead 
 
%%file slova.stalin
%%parent slova
 
%%url 
 
%%author 
%%author_id 
%%author_url 
 
%%tags 
%%title 
 
%%endhead 
\chapter{Сталин}

\emph{Сталин} приказал расстреливать детей! Да-да, именно так поздравил страну с Днем
Защиты Детей журналист Невзоров. И даже сослался на одно любопытное
постановление Совнаркома. Как обычно, ложь чудовищная!  У аппарата Ярослав
Бушмицкий, давайте разбираться. Зная волчью повадку либеральных журналистов и
не имея привычки верить им на слово полез читать то самое постановление. И
волосы дыбом!  Журналист Невзоров сообщает, что в тридцать пятом году товарищ
Сталин издал указ о расстрелах детей с двенадцати лет. И начали массово его
применять, причём по самым незначительным поводам. То поезда расцепят, то
венерической болезнью заболеют. И всех детей за это к стенке. И даже номер
называет – постановление Совнаркома номер три дробь пятьсот девяносто восемь.
Открыл, а там совсем не про то!

%%%cit
%%%cit_pic
%%%cit_text
Звичайно масштаб особистості \emph{Сталіна} і Порошенка важко порівняти. Як і їхні
злочини проти власного народа. Але логіка виправдання мерзотних вчинків свого
вождя у вірних порошенківців така сама, як і у вірних \emph{сталіністів}: проте армію
відродив, Путіна зупинив, безвіз підписав... Нагадаю, що за \emph{Сталіна} був не
тільки Голодомор та Розстріляне відродження, а й український культурний
ренесанс, перший академічний словник української мови, розщеплення атому в
Українському фізико-технічному інституті, об'єднання України в сучасних
кордонах (окрім Криму) та входження УРСР до складу держав-засновників ООН. Що
аж ніяк не виправдовує жахливих злочинів \emph{сталінізму}
%%%cit_comment
%%%cit_title
\citTitle{У політику треба повернути мораль. Інакше ми приречені залишатися сталіністами}, 
Геннадій Друзенко, gazeta.ua, 14.06.2021
%%%endcit
