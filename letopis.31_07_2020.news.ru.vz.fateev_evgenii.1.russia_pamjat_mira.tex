% vim: keymap=russian-jcukenwin
%%beginhead 
 
%%file 31_07_2020.news.ru.vz.fateev_evgenii.1.russia_pamjat_mira
%%parent 31_07_2020
 
%%url https://m.vz.ru/opinions/2020/7/31/1052577.html
 
%%author Фатеев, Евгений
%%author_id fateev_evgenii
%%author_url 
 
%%tags russia,memory,world
%%title России пора становиться хранилищем памяти мира
 
%%endhead 
 
\subsection{России пора становиться хранилищем памяти мира}
\label{sec:31_07_2020.news.ru.vz.fateev_evgenii.1.russia_pamjat_mira}
\Purl{https://m.vz.ru/opinions/2020/7/31/1052577.html}
\ifcmt
	author_begin
   author_id fateev_evgenii
	author_end
\fi
\index[rus]{Мысли!России пора становиться хранилищем памяти мира, 31.07.2020}

\textbf{Если на Западе начнут перетруску и чистку музеев, библиотек и архивов, нужно
предложить им отдавать все, что не кажется политкорректным, нам. Пусть мы
станем в их глазах свалкой неполиткорректной рухляди. Ничего! Мы не гордые. Мы
сложные.}

\ii{pic.author.fateev_evgenii}

По всему западному миру валят памятники. Пока нет вопиющих новостей про музеи,
но есть основания полагать, что музейные новости у нас еще впереди. Они только
недавно открылись после карантина. Да и то не все. В музеях хранится старое
искусство, живет старая вещность, которая буквально вопиет о своей
неполиткорректности.

Сегодня музеи – это одна сплошная зона поражения всяческими «измами». В старой
европейской живописи живут насилие и расизм, сексизм и… там Иисус Христос
сплошь белый. И стоит ожидать попыток поменять музейные экспозиции, развенчать
и низвергнуть великих представителей «белого» искусства. А может случиться и
нечто похуже. 

\ifcmt
pic https://img.vz.ru/upimg/soc/soc_1052577.jpg
\fi

Учитывая сегодняшнюю всеобщую взвинченность, найдут ли правовые системы
западных стран в себе силы жестко наказать какого-нибудь вандала, возжелавшего
надругаться над произведениями «неправильного» искусства? Насколько сегодняшние
музеи застрахованы от невежественных, но расово, гендерно соответствующих
геростратишек, эдаких невежественных сверхчеловечков, которые заявятся в музей
с чем-нибудь колюще-режущим, распыляющим или растворяющим? Как это сделала в
1914 году, например, откровенная идиотка, суфражистка Мэри Ричардсон,
искромсавшая ножом шедевр Веласкеса «Венера с зеркалом». Кстати, ее
биографическая траектория весьма показательна. Вскоре она вступила в Британский
союз фашистов и стала лидером его женского подразделения.

И хватит ли западным музеям пороху отстоять право находиться в экспозиции для
неполиткорректных, по сегодняшним меркам, картин? Сможет ли Метрополитен-музей
защитить «Спящую Терезу» Балтю, которую обвиняют в «романтизации сексуализации
ребенка»? Три года назад удалось. Получится ли это сейчас?

И не была ли акция, случившаяся в Художественной галерее Манчестера в 2018
году, открытием ящика Пандоры? Напомню, руководство галереи на время убрало из
постоянной экспозиции картину Джона Уильяма Уотерхауса «Гилей и нимфы», шедевра
прерафаэлитской живописи, за «откровенный сексизм» и несоответствие
представлению сегодняшних феминисток о прекрасном.

Все больше набирает обороты отыскивание в истории старой европейской живописи
художниц, работы которых можно более или менее обоснованно повесить в больших
музеях. Следует признать, что в истории старого европейского искусства женских
имен очень немного. Так получилось. Скажу более крамольную вещь – до XVIII века
хороших художниц в Европе вообще не было. Но во имя равенства и других
ценностей феминистского фундаментализма, боюсь, продолжится натягивание совы на
глобус, вытаскивание за уши старых средненьких художниц в ущерб более
талантливым художникам-мужчинам, которые могут пасть жертвами перемен в
музейных экспозициях. 

А еще разворачивается борьба с европоцентризмом. Так, в январе 2020 года
Йельский университет отменил пользовавшийся много лет большим успехом базовый
курс «Введение в историю искусства: от Ренессанса до настоящего времени» именно
по причине его европоцентризма.

В меню всемирной аналитики появилось теперь такое блюдо, как расизм. Поскольку
ребята на этой ниве работают системно и страстно, не удивлюсь тому, что через
какое-то недолгое время Веронезе и Пуссена в Лувре еще придется поискать, а
может там им уже и не найдется места.

Сегодня, на мой взгляд, происходит подтачивание некой надгендерной, надрасовой,
надклассовой, всеобщей шкалы оценок. В этом корень современного варварства,
подкрепленного спекуляциями пустых и подлых апостолов постмодернизма, злые и
темные идеи которых выбрались из академических книжек и мутировали в кричалки
звереющих на глазах участников уличных демонстраций. Сегодняшние варвары хотят
вносить смешные и идиотские поправки к всеобщему: красоте, эффективности,
истинности, чести и прочему.

И вот что я предлагаю. Если на Западе начнут перетруску и чистку институтов
социальной памяти (музеев, библиотек и архивов), нужно предложить им отдавать
все, что не кажется политкорректным, нам. Пусть мы станем, в их глазах, свалкой
неполиткорректной рухляди. Ничего! Мы не гордые. На самом же деле, пора нам
становиться великим Русским ковчегом, хранилищем памяти мира. Нам необходимо
хранить способность человека к цивилизационной сложности. Как Бран Старк из
«Игры престолов», мы должны стать всемирным трехглазым вороном.

Это какая-то патология – люди сегодня стремительно становятся
маловместительными, неспособными к цивилизационной сложности. А в нас вся
сложность мира способна поместиться. На территории современной России в разное
время существовали более сотни различных государств. Своими историческими
кульбитами мы сформировали в себе навыки такого существования. В нас, даже
сегодняшних, вмещаются и Красный проект, и Романовская империя, и место
наказания Прометея, и кантианский Кенигсберг, и Византия, и древнегреческие
колонии, и Золотая Орда, и много всего другого – в нас все это может
поместиться. 

И многое мы уже умеем или учимся уметь хранить. Дуньхуанские рукописи и
библиотеку Вольтера, едва ли не лучшее в мире собрание еврейских инкунабул и
самую большую в мире коллекцию рукописей Карла Маркса…

Если опять вспомнить о реалиях популярного сериала «Игра престолов», в конечном
итоге король ночи приходит, чтобы убить трехглазого ворона, хранителя всемирной
памяти. И к нам придет. В сериале удалось отбиться. И мы отобьемся. 
