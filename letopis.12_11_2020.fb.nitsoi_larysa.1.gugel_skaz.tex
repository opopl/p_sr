% vim: keymap=russian-jcukenwin
%%beginhead 
 
%%file 12_11_2020.fb.nitsoi_larysa.1.gugel_skaz
%%parent 12_11_2020
 
%%url https://www.facebook.com/larysa.nitsoi/posts/3772708999427871
 
%%author Ніцой, Лариса
%%author_id nitsoi_larysa
%%author_url 
 
%%tags 
%%title Гугел сказився
 
%%endhead 
 
\subsection{Гугел сказився}
\label{sec:12_11_2020.fb.nitsoi_larysa.1.gugel_skaz}
\Purl{https://www.facebook.com/larysa.nitsoi/posts/3772708999427871}
\Pauthor{Ніцой, Лариса}

Гугел сказився. Вислів \enquote{частина україномовного населення України} перекладає як
\enquote{часть рускоязичного насєлєнія}. Уявляєте?  Можете самі перевірити. 

\ifcmt
pic https://scontent.fiev6-1.fna.fbcdn.net/v/t1.0-9/124844906_3772708812761223_5778789950154595917_o.jpg?_nc_cat=110&ccb=2&_nc_sid=730e14&_nc_ohc=EShN1SDEHZMAX9bzW99&_nc_ht=scontent.fiev6-1.fna&oh=a9e22bb8c0199ecdab5ddf46a61e960b&oe=5FE58C91
\fi

Коротше. У вас є можливість зробити безкоштовно добру справу і отримати плюсик
у карму. Відкриваєте гуглперекладач. В лівій частині обираєте українську мову -
у правій частині російську. В українську колонку вводите \enquote{частина
україномовного населення України} -  і перекладаєте на російську. Отримуєте
дурний переклад. 

Під дурним перекладом написано курсивом \enquote{надіслати відгук}. Треба їм написати
сотні тисяч відгуків. Тоді по той бік увімкнеться не цифрова програмка, якій
усе добре, а людський мозок. Треба те паскудство виправити.
