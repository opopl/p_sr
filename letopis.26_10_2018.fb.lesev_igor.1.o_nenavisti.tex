% vim: keymap=russian-jcukenwin
%%beginhead 
 
%%file 26_10_2018.fb.lesev_igor.1.o_nenavisti
%%parent 26_10_2018
 
%%url https://www.facebook.com/permalink.php?story_fbid=2142889785742113&id=100000633379839
 
%%author_id lesev_igor
%%date 
 
%%tags facebook,ideologia,nenavist,obschenie,obschestvo,psihologia,raskol,ukraina
%%title О ненависти
 
%%endhead 
 
\subsection{О ненависти}
\label{sec:26_10_2018.fb.lesev_igor.1.o_nenavisti}
 
\Purl{https://www.facebook.com/permalink.php?story_fbid=2142889785742113&id=100000633379839}
\ifcmt
 author_begin
   author_id lesev_igor
 author_end
\fi

О ненависти.

Пережил тут еще один странный бан. Вроде «бортанули» на месяц, но потом как-то
по непонятной причине «откинулся» через недели две.

Алгоритм все тот же. Жалоба на статью. Но тут уже не прокатило. У меня уже НЕ
ОСТАЛОСЬ материалов, за которые можно даже формально натягивать. Впрочем, к
жалобе жалобщики прикрепили еще два моих комментария от 2013 года. Комментарии,
естественно, с матюками. Но ведь ТОГДА за это не наказывали. И вот тут я решил
срадикальничать и потереть все, где я насрал с обсценной лексикой, вульгарщиной
и прочей неполиткорретностью.

\ifcmt
  ig https://scontent-frt3-1.xx.fbcdn.net/v/t1.6435-9/44810194_2142889692408789_4738209767783137280_n.jpg?_nc_cat=106&ccb=1-5&_nc_sid=730e14&_nc_ohc=IrC4LbKc6DgAX_f4Gj5&_nc_ht=scontent-frt3-1.xx&oh=fc8d59a1b889c40117a21dbab64e1828&oe=61B6A53E
  @width 0.4
  %@wrap \parpic[r]
  @wrap \InsertBoxR{0}
\fi

Собственно, чего я все это рассказываю? Изучая в ускоренном темпе воспоминания
из виртуальной жизни одного единственного человека – т.е. самого себя – я
невольно пришел к некоторым не самым приятным выводам.

Во-первых, я увидел себя со стороны. Например, в общении с девушками я
совершенно вульгарный тип. Все шутки – ниже пояса. Все темы сводятся к сексу.
Любые споры заканчиваются сексистской категоричностью о тупости женщины в этом
подлунном мире. Это настолько неприятно, что будь я женщиной, я бы с таким как
я никогда бы не встречался. И уж точно не переписывался.

Во-вторых, в общении с политическими оппонентами я часто делал то, что сам
демонстративно презираю – навешивание клише и ярлыков. Раз-два, и собеседник, с
которым ты споришь (условно) о длине Кордильер, уже «тупой колхозник с низкой
социальной ответственностью». Поэтому, искренне поражаюсь своим виртуальным
френдам-оппонентам, которые столько лет меня терпят.

В-третьих, сам по себе «Фейсбук» - это редчайшее говно, если, конечно, мы не
говорим о сети, как о бизнес-проекте для определенных категорий лиц. Столько
времени тратится на идиотские споры о тех же Кордильерах, и только для того,
чтобы по итогу назвать собеседника тупым.

Но это было вступление. Извините, что маленькое. Лично моя история в ФБ
началась сразу же после событий на Майдане 2013 года. До этого мне эта сеть
была абсолютно не интересна. Оставлял 2-3 комментария в месяц, и два десятка
лайков хрен пойми чего. Но именно с Майданом ФБ становится настоящей площадкой,
где схлестнулись наши соотечественники.

Критический момент в общении между нами всеми произошел где-то в ноябре
2013-январе 2014 года. Мы тогда срались, подначивали друг друга, стебались, но
все это было ржакой-аттракционом. Мы называли друг друга придурками и
постепенно вешали на своих оппонентов ярлыки.

Но самое главное, во всем этом веселом процессе у нас всех не было взаимной
НЕНАВИСТИ. Проще говоря, если я вступал в спор с экс-коллегой-девчушкой с того
же «Обозревателя», в первую очередь она для меня оставалась носительницей
сисек, а не сторонницей или противницей Майдана. Мы НЕдегуманизировали друг
друга. Наши срачи заканчивались именно в ФБ. Но вот начинается ужин, ты бежишь
на кухню за сосиской, и война окончена.

А потом произошел невидимый слом. Это как скандал в семье. Семьи могут
скандалить годами и оставаться вместе. А может быть только один скандал, даже
не очень громкий, но в нотках друг друга появляется ненависть и взаимное
неуважение. И тогда балаган превращается в бригады, работающие на взаимное
уничтожение.

То что каждый из нас своей риторикой приложил усилия к взаимному расколу нашего
общества – это понятно. Общетеоретическая софистика. Но есть водораздел, когда
пошла точка невозврата. Это захват Киевсовета. Именно с этого момента хаос стал
доминировать над системой. Именно это отличает Майдан 13/14 годов от Майдана
04/05. Именно тогда власть стала терять свою сакральность. А те, кто пришли
следующими – не приобрели до конца легитимности. Вот тогда насилие и стало
базовым залогом успеха.

И именно с того момента произошло непоправимое и по-настоящему преступное –
взаимные страчи в «Фейсбуке» переросли во взаимную ненависть между людьми,
которые к власти вообще никакого отношения не имеют.

\ii{26_10_2018.fb.lesev_igor.1.o_nenavisti.cmt}
