% vim: keymap=russian-jcukenwin
%%beginhead 
 
%%file 07_04_2022.fb.gin_anna.harkov.1.dnepr_sirena.cmt
%%parent 07_04_2022.fb.gin_anna.harkov.1.dnepr_sirena
 
%%url 
 
%%author_id 
%%date 
 
%%tags 
%%title 
 
%%endhead 
\zzSecCmt

\begin{itemize} % {
\iusr{Evgeny Gendin}

А где у нас в городе бомбоубежище хоть?

\begin{itemize} % {
\iusr{Анна Гин}
В моем случает - подвал в доме, пятиэтажка. Но я в него не заходила, только курю рядом.

\iusr{Evgeny Gendin}
\textbf{Анна Гин} как это знакомо)

\iusr{Юлія Мензелівська}
\textbf{Анна Гин} Что самое грустное, подвалы не надежные..

\iusr{Лариса Гнатченко}

у нас и подвалы закрыты были. почему и уехали. некуда прятаться. а когда бомбят
не где-то, а вот прям рядом, у тебя шатаются стены старенького дома, летят
стекла вместе с рамами, а ты в коридоре - то еще удовольствие.

\iusr{Mariya Kushka}
Еще метро.
\end{itemize} % }

\iusr{Андрей Козачек}
А Московский район как был, так и остался...((

\begin{itemize} % {
\iusr{Анна Гин}
\textbf{Андрей Козачек} думаю теперь уже это вопрос пары недель/месяцев.

\iusr{Андрей Козачек}
\textbf{Анна Гин} ну теперь то уже да....

\iusr{Ирина Семенчина}
\textbf{Андрей Козачек} і бюст ката Жукова досі стоїть за якого ще донедавна мер дупу рвав...

\iusr{Сергій Каднай}
\textbf{Андрей Козачек} Питання про зміну назви вже опрацьовується в міськраді. Десь читав про це. Не пригадаю зараз.

\iusr{Ирина Семенчина}
\textbf{Андрей Козачек} но осадок от борьбы мера за этот бюст остался...

\iusr{Светлана Михайловская}
А Московский проспект еще не переименовали?

\end{itemize} % }

\iusr{Наталья Канунникова}

Да, война казалась далеко....эти уроки НВП, противогазы, какие-то странные
слова зарин, иприт...казалось, мы с этим не столкнемся, такого быть не может!
Ну с кем нам воевать???? И да, хочется проснуться, это страшный сон

\begin{itemize} % {
\iusr{Marina Degterewa}
\textbf{Наталья Канунникова} 

Да. И только наш военрук, на наивный вопрос: \enquote{а реально спрятаться от атомной
бомбардировки, как на плакатах нарисовано?}, ответил: \enquote{нет. На самом деле
погибнут все...}(простите)

\iusr{Наталія Глюз}
\textbf{Marina Degterewa} та наш казав \enquote{береш білу простень і біжиш в напрямку кладовища, хоч по адресу потрапиш.}  @igg{fbicon.thinking.face}  @igg{fbicon.face.tears.of.joy} 
\end{itemize} % }

\iusr{Юлія Висоцька}
Тоже не могу поверить ...

\iusr{Нина Дитрих}

\obeycr
Тогда, в царстве - государстве \enquote{дракона} получилось отрубить ему часть.
Пришло время рубить \enquote{голову}.
Держись, моя хорошая.
Страшно, тревожно, громко...
Для нас, как для наших настоящих дедов и воинов, это будет бой и победа.
Для фашистов... Как для фашистов - уничтожение.
И вечный позор. На века. Несмываемый.
\restorecr

\iusr{Victoria Mashynetc}

Да у нас тоже в школе были такие учебные тревоги и я тоже вспомнила это все
сейчас, очень правильно сказано...и очень страшно, что возможно Днепр ждёт тоже
самое, что в Харькове, молюсь, чтобы не было этого!

\iusr{Анна Кипаренко}

Да, после Харькова тяжело привыкнуть к сиренам. У нас на Салтовке их не было,
нас просто бомбили. А в Днепре бывают дни, когда они воют одна за другой. И
тогда сразу хватаешь телефон, смотришь новости... и сердце сжимается. Неужели
опять Харьков? Опять Салтовка? До сих пор не могу понять, что эта страшная
реальность происходит с нами сейчас....

\iusr{Stasia Koblents}

Анна... Вы так все описали, я вернулась в своё детство... коробочка... звуки... 116
школа... как будто вы писали про неё... ещё помню, как нас водили в какое-то
помещение на тринклера, цокольный этаж, там что-то было о том, если не
ошибаюсь, что в этом помещении запускают сирену... смутные воспоминания, но вы
так визуализировали, что вспомнилось давно забытое...

Спасибо, что пишите! Берегите себя, пожалуйста  @igg{fbicon.hands.pray}
@igg{fbicon.heart.yellow}  @igg{fbicon.heart.blue} 

\begin{itemize} % {
\iusr{Навроцкая Юлия}
\textbf{Stasia Koblents} у меня 132. Между нашими школами было срперничество))

\iusr{Stasia Koblents}
\textbf{Юліа Навроцька} я только помню, что нашу школу ремонтировали и мы около года ходили в 132 @igg{fbicon.hands.shake}  @igg{fbicon.person.raising.hand} 

\iusr{Навроцкая Юлия}
\textbf{Stasia Koblents} а у вас часто проводили конкурс \enquote{строя и песни} на 23е февраля))
Наш класс 2 года побеждал по школе, а потом по району, в вашей школе...

\iusr{Olga Chegodaeva}
\textbf{Stasia Koblents} а у меня 131, а дети отучились в 116. Муж подобрал осколок от того что прилетело в дом писателей.
\end{itemize} % }

\iusr{Ирина Бурцева}
То же чувство. Все время напоминаю себе: это летит настоящая ракета, несёт смерть. Это делают русские.

\iusr{Ella Shevchenko}

ух. узнаю поколение 1972 - 1980 с подвалами - тревогами с не боюсь этого слова
уроками с военруком - и с метанием гранат на школьном стадионе. И да тогда это
фейк и совок. Сейчас реальность и расия и совок - напали на нас. Тоже отхожу,
что все нереально и начинаю что то делать

\begin{itemize} % {
\iusr{Polyakova Marina}
\textbf{Ella Shevchenko} 

а нам в 90-х во время учебной тревоги больше Чернобыль вспоминали - акцент был
на противогазы, радиацию... война вообще осталась только в книгах для нашего
поколения на тот момент

\iusr{Vira Dankova}
\textbf{Ella Shevchenko} в то время мы верили, что больше НИКОГДА не будет войны.
\end{itemize} % }

\iusr{Наташа Булатова}
Ничего прорвёмся !!Мы Харьковчане сильная нация @igg{fbicon.flag.ukraina}!!!

\iusr{Александра Василенко}

Сегодня я поймала себя на мысли, что расстроилась засохшему куску хлеба.
Кусочек маленький, но выбросить не могу.. В точности как наши бабушки. В
Харькове мы были первую неделю войны. К 1 марта за хлебом стояли огромные
очереди, муки на полках тоже не было. Я спрятала в холодильник оставшиеся от
мирной жизни булочки, которые мы отрезали маленькими кусочками, и грустно
шутили о том, что теперь для нас стало деликатесным товаром. В Виннице много
хлеба, много булочек и другой выпечки, но я не могу забыть то ощущение. В моем
городе не хватает хлеба... И уж тем более не могу забыть, что прямо сейчас есть
Мариуполь, Рубежное и многие другие наши города, прямо сейчас есть люди,
которые давно не ели и не пили. И не могу выбросить этот кусок @igg{fbicon.face.pensive}  43-й день
войны. И я тоже все еще не могу в это поверить

\begin{itemize} % {
\iusr{Olga Chegodaeva}
\textbf{Александра Василенко} 

у меня в морозилке лежит хлеб купленный в первый день войны. Он какой то
мажорный бездрожжевой и цены в магазине не было, а другого тоже не было. Но
через пару дней стали печь и раздавать тем кто не мог заплатить. Сейчас с
хлебом все как раньше, плюс каждый день раздают бесплатно. Люди даже не берут.
Я в Харькове в центре.

\iusr{Лариса Гнатченко}
\textbf{Александра Василенко} 

а у меня в холодильнике остался целый хлеб, купленный на рынке аккурат перед
тем, как его разбомбили... наверное, когда приедем, такое увидим...

\iusr{Александра Василенко}
\textbf{Ольга Чегодаева} 

люди приспосабливаются, город приспосабливается. Меня добавили в винницкий чат
поиска недвижимости. Многие ищут жилье и в качестве срока указывают «до
победы». Я хочу и дальше верить, что наши города нас дождутся

\iusr{Olga Chegodaeva}
\textbf{Александра Василенко} обязательно дождутся
\end{itemize} % }

\iusr{Aleksandra Kalinichenko}

Провівши декілька днів в Дніпрі зрозуміла, яке було щастя відсутність серен в
Ха. Вони б не замовкали на Салтівці. Спати під них неможливо так, як під, майже
звичний, гуркіт обстрілів

\iusr{Марина Филимонова}
Обнимаю вас

\iusr{Елена Казачук}

Уроки НВП, противогазы, зарницы.... Мы веселились.... Это было лучше, чем
сидеть на уроке точных наук. Я так же вспоминаю как мы подсмеивались над
учителем, который пытался научить нас, как вести себя при военной угрозе...

\iusr{Яна Хворост}
Тяжело поверить и принять такую реальность

\iusr{Оксана Матийко}
Берегите, пожалуйста, себя !

\iusr{Олена Житар}
Да, все время хочеться проснуться.

\iusr{Криволапова Елена}
И я.... Параллельная реальность какая-то... так не может быть  @igg{fbicon.face.crying.loudly} 

\iusr{Екатерина Тыжнова}

У нас слышно одновременно четыре сирены с разных точек... Жуть... Друг друга
догоняют, сливаются, распадаются в звуке и снова по кругу...

Теперь привыкли. Киев.

\iusr{Наталія Наталія}
Приветствуем в Днепре. Сейчас любой город Украины родной каждому. -)

\iusr{Oxana Razumey}
Обнимаю Сердцем
Боже Поможи в Час Добрий

\iusr{Julia Leshchenko}

Такое ощущение, что сирена тут в Днепре заедает. В Харькове мы её вообще не
слышали. За 12 дней войны пару раз и то совсем тихо. А тут орёт очень громко и
иногда так и не отключается. Только перестаёт звучать по
возрастающей-затухающей, а остаётся от неё на одной ноте такое зловещее
\enquote{ууууууу}... Кажется, это \enquote{уууууу} встроились мне в мозг прям и так там и
останется. (((

\begin{itemize} % {
\iusr{Nonna Vetrova}
\textbf{Julia Leshchenko} 

а я вообще в Днепре не слышу сирену. То есть слышу очень тихо. Первое время
вообще не было слышно. Потом они ее усилили и на нашей окраине стало слабенько,
но слышно. Это вам с районом не повезло. Или наоборот.

\iusr{Юлія Мензелівська}
\textbf{Юлия Лещенко} 

Видно в Харькове быстро перебили звязь сирены..(

Сочувствую Вам.

Так же в Мариуполе читала что не слышали люди...ранена Украина.

Реzкий с маленькой мир пожаловал.. эти завистливые убийцы и сами не умеют жить и
другим не дают.. закрыть этих нациссстов от цивилизации.. что
больше не делали боли людям.

Извините.

\iusr{Виктория Меркулова}
\textbf{Юлия Лещенко} Сейчас в Харькове ревуны работают исправно. Очень даже слышно.

\iusr{Виктория Илюхина}
\textbf{Юлия Лещенко} 

за то сейчас в Харькове сирена воет практически постоянно, уже сил нет от этого
воя и бабахов, остатки нервов на пределе

\iusr{Vyacheslav Matveev}

А у нас в школе на тренировках НВП отрабатывали хим атаки. Выше над школой жд
колеи и стандартная беда для тренировок жд катастрофа с цистернами. И вот нам
долго на уроке рассказывали о том что при аварии если одно вещество то вся
гадость опускается к земле и надо подниматься вверх, а если другое наоборот
пары уходят вверх и надо спускаться вниз... и потом сирена обьявляют аварию с
тем что стелится по земле.... и... спускают всех вниз на плац.... и мы стояли и
ржали что учителя убили всех детей... весело всем было и никто не думал а если
бы по настоящему...

\end{itemize} % }

\iusr{Светлана Косенко}

А я с родным братом накануне войны разговаривала, он на Кубани живет, заверил,
что полномасштабной войны не будет, я поверила!!!

\iusr{Svitlana Gekko}

А я дитинство провела в кілька км від АЕС, тому сирени в нас завжди навіювали
лютий страх \enquote{атомка взірвалась!}.... а от в Харкові за два тижні жодного разу
не чула, сигналом обійматися міцніше і молитися активніше були вибухи. А під
авіанальотами падали хто де, навіть 9-річна доня знала.... А в Дніпрі сирени
гучні... але не такі страшні як інші звуки....

\iusr{Лариса Гнатченко}

а у меня сиренный синдром появился. как начинает выть - мне просто жутко
хочется спать. прям вот бери и ложись. вот щас воет, у меня ж разгар рабочего
дня.... а в глаза - хоть спички. думаю, организм так защищает

\iusr{Анна Ерошкина}
\textbf{Лариса Гнатченко} а у меня головная боль сразу появляется

\iusr{Natalia Kamaeva}
И оч по-другому вспоминаются рассказы бабушки...

\begin{itemize} % {
\iusr{Елена Мартыненко}
\textbf{Наталья Камаева} в точку! Тоже хотела об этом написать. Разве я могла знать расспрашивая бабушку про войну, что и мы и наши дети будем когда-то это переживать?!
\end{itemize} % }

\iusr{Maryna Lvovska}

Значит в вас живёт дух сопротивления! И вы обязательно выстоете! И добро
победит зло обязательно!

\iusr{Marina Degterewa}

После событий 2014 была \enquote{волна учений по спуску в бомбоубежище} среди учебных
заведений. На одном из таких \enquote{учений} мой средний сын спросил у директора
школы: \enquote{а у нас возле дома нет убежища. Моим родителям можно будет здесь
находиться?} На что педагог тактично промолчала. А после \enquote{волны учений}
старший, будучи сталкеров, спускался в такие - заводские и городские - убежища.
\enquote{Разворовали за несколько лет, всё, что было до 14 года. Все. Ни одно не
пригодно было для пребывания людей. А сейчас и подавно. Даже трубы воздуховодов
на металл спилили.}

\iusr{Kamilla Kostenko}

В Испании в школах звонок на урок - это сирена. Пусть короткая, но у меня кровь
замирает в венах. Может это одна из причин того, что не много украинских детей
ходят в испанские школы  @igg{fbicon.frown} 

\iusr{Elena Belokon}
Да.

\iusr{Kateryna Pokatska}
Абсолютно такие же воспоминания и мысли

\iusr{Irina Fedorova}
У меня такие же воспоминания. И ощущения. Только с разницей, что я в родном Харькове...

\iusr{Oksana Sapir}

 @igg{fbicon.hands.pray} держитесь это больше не слово... словом не описать  @igg{fbicon.face.pensive}  хочется схватить и
держать... и гладить по голове... и принеси чайку... всем  @igg{fbicon.face.pleading}  @igg{fbicon.face.pensive}{repeat=5} 

\iusr{Dmitry Bezugly}

Видимо, мой папа мне не улыбался и я ребенком насмотревшись фильмов о войне ( а
их было много) воспринимал всерьез и со страхом, как и вероятность нападения
Америки ядерными ракетами(

\iusr{Ирина Бережная}
Да, к этому вою невозможно привыкнуть. Он проникает в самую глубь

\iusr{Tetiana Zharikova}
Читаю и вижу свою 122 школу на Салтовке

\iusr{Tala Ka}
+1000

\iusr{Влада Николенко}
Никто не может

\iusr{Денис Скриль}
Дуже сильно - дякую! @igg{fbicon.flag.ukraina} @igg{fbicon.sun.with.face}  @igg{fbicon.hands.pray} 

\iusr{Владимир Рожков}
Не переймайся

\iusr{Alexey Lukash}
Тримайтеся!

\iusr{Natalya Bolshak}
 @igg{fbicon.cry} 

\iusr{Навроцкая Юлия}

Мы одногодки, с разницей в месяц. Я помню про \enquote{учебные тревоги} и военрук,
рассказывающий, что при ядерной атаке, надо \enquote{упасть в сторону противоположную}
и \enquote{убыть в золочевский район}. Мы весело ржали, класс был литературный и
\enquote{убыть} нас страшно веселило...

Я помню 24 февраля. Как мы с мужем разом сели в постели и молча посмотрели друг
другу в глаза. Это было... ну страшно.

При близких прилетах, я \enquote{падаю в сторону противоположную} и широко открываю
рот... я вспоминаю военрука и как мы над ним радостно ржали...

Моей школы уже нет... её разрушили, ибо место было очень коммерческим...
русские тут ни причём, но когда я засыпаю... а в Харькове, нынче, сон мой
зыбок, мне сниться моя старая школа и губастенький военрук, который меня так
веселил... я не помню, как его звали, в моем юношеском сознании, он был \enquote{тупым
салдофоном} со шрамом от фуражки поперёк лба... я хочу Вам сказать спасибо.
Ваши дебильные уроки помогли мне выжить при бомбежках моего Харькова...

Спасибо, за то, что не смотря на наше ржание и хамство, Вы вбивали нам в нашу
пустую голову \enquote{правило 2х стен}, и что рот надо широко открывать... и что
должно быть в доме всегда, как резерв - вода и сухари...

\begin{itemize} % {
\iusr{Анна Корнилова}
\textbf{Юліа Навроцька} школа не 132, случайно?

\iusr{Навроцкая Юлия}
\textbf{Анна Корнилова} точно ))
\end{itemize} % }

\iusr{Ольга Товченко}

Я то же помню как в детстве проверяли сирену. А вот когда началась война, в
нашем районе Харькова сирена не работала и о бомбежках мы узнавали, когда
реально начинали бомбить и бежали прятаться в погреб. Страшно было когда
слышали гул летящих истребителей и уже через несколько секунд они летели низко
над головой. Дети бежали в укрытие, а мой сын кричал ложитесь. Нас хватило на 8
дней. Потом мы выехали из Харькова.

\begin{itemize} % {
\iusr{Valeriia Zabolotna}
\textbf{Ольга Товченко} 

Все именно так и было! Самолёт над домом и взрыв совсем рядом. Сидя в подвале
молились, понимая что ни на что не можем повлиять, только молится. Это чувство
останется, как мне кажется, навсегда.

\iusr{Сергій Каднай}
\textbf{Ольга Товченко} Сірен нема і зараз. Ну то можна пояснити.

\iusr{Виталий Гуденко}
\textbf{Сергій Каднай} Подекуди є, але не усюди. В мене сирену лише собака чує - по ній і орієнтуємось  @igg{fbicon.smile} 

\iusr{Ирина Волович}
\textbf{Ольга Товченко} 

Да я только в центре, когда уезжала на 10-й день войны, её и услышала. А
так-то, самолёты прямо над домами летели, чуть ли не на бреющем полёте. А
канонада у нас в районе надолго не утихала, потому я в какой-то момент
перестала обращать на неё внимание, готовила, хлеб пекла. Ночевали только в
кладовке и коридоре поначалу. А вот когда снаряды начали рваться прямо перед
окнами, уехали. А муж оставался. Пока в него осколок не прилетел.

\iusr{Ольга Товченко}
\textbf{Сергій Каднай} пояснить можно все, але забути, наврядчi...
\end{itemize} % }

\iusr{Елена Филиппова}

А теперь мой Харьков, мою Салтовку, нелюди бомбят и без сирен! Ненавижу! Их
деды переворачиваются в могилах от того, что творят их выродки! Этим фашистам
нет места ни на земле, ни на небе! @igg{fbicon.anger} 

\iusr{Лариса Сосницкая}

Мы взрослые девочки. В это придется поверить. И с этим как - то жить. Харьков.
Два часа ночи. Бомбят уже часов 5. Либо я сама ебнусь, либо выживу и они
получат от меня такой пиз@ды, которую я ещё даже не придумала

\iusr{Anastasiya Kostina}
А нормальный человек не может в это поверить, потому что в его мире все по-человечески. А это ТВАРИ

\iusr{Светлана Баруткина Бурак}

Выбачаюся. Але ня мне патрэбны лекі. Падрабязнасьці на старонцы. 

\url{https://m.facebook.com/story.php?story_fbid=1562989744071471&id=100010813809799}

\iusr{Max Golovachev}
Тут появилось мнение, что Z - это сложенные семерки, ненавижу этих пидарасов, за то что сделали фарс трагедией...

\iusr{Юлія Мензелівська}
Правда ..так же не могу и не хочу поверить что они пытаются уничтожить нас..

\iusr{Michael TavDa}

\raggedcolumns
\begin{multicols}{2} % {
\setlength{\parindent}{0pt}

\obeycr
КРЫЛАТЫЕ, но не КАЧЕЛИ
.
В этом месяце букеты
Не цветут еще – февраль.
Смертоносные ракеты
Запускает подлый враль.
.
Позабыто все на свете,
Сердце замерло в груди:
Только взрывы рядом где-то,
Только смерти впереди!
Только взрывы близко где-то,
Только горе впереди!
.
Совсем не взрывпакеты –
Не ведая преград,
Крылатые ракеты
На мальчиков летят!
Крылатые ракеты
На девочек летят!
.
Детство кончилось когда-то,
Ведь оно не навсегда,
Стали взрослыми ребята,
Путин их послал сюда.
.
В Украине тоже дети,
Им расти еще, расти:
Только взрывы рядом где-то,
Только смерти впереди!
Только взрывы близко где-то,
Только горе впереди!
.
Совсем не взрывпакеты –
Не ведая преград,
Крылатые ракеты
На мальчиков летят!
Крылатые ракеты
На девочек летят!
.
Шар земной весь возмутила
Вероломность россиян.
Путин – изверг, зверь, мудила,
Варваров-манкуртов хан.
.
Позабыто все на свете,
Сердце замерло в груди:
Только взрывы рядом где-то,
Только смерти впереди!
Только взрывы близко где-то,
Только горе впереди!
.
Совсем не взрывпакеты –
Не ведая преград,
Крылатые ракеты
На деточек летят!
Крылатые ракеты
На деточек летят!
\restorecr
\end{multicols} % }


\end{itemize} % }
