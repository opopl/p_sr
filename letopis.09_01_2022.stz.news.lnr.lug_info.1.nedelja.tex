% vim: keymap=russian-jcukenwin
%%beginhead 
 
%%file 09_01_2022.stz.news.lnr.lug_info.1.nedelja
%%parent 09_01_2022
 
%%url https://lug-info.com/news/nedel-ya-glazami-eksperta-rassadnik-pandemii-obezvozhivanie-i-evroodnobokost
 
%%author_id news.lnr.lug_info
%%date 
 
%%tags 
%%title НЕДЕЛЯ ГЛАЗАМИ ЭКСПЕРТА: Рассадник пандемии, обезвоживание и еврооднобокость
 
%%endhead 
\subsection{НЕДЕЛЯ ГЛАЗАМИ ЭКСПЕРТА: Рассадник пандемии, обезвоживание и еврооднобокость}
\label{sec:09_01_2022.stz.news.lnr.lug_info.1.nedelja}

\Purl{https://lug-info.com/news/nedel-ya-glazami-eksperta-rassadnik-pandemii-obezvozhivanie-i-evroodnobokost}
\ifcmt
 author_begin
   author_id news.lnr.lug_info
 author_end
\fi

\begin{zznagolos}
Своим видением событий прошедшей недели, так или иначе связанных с Луганской
Народной Республикой, с ЛИЦ делится военный эксперт, общественный деятель,
подполковник запаса Народной милиции ЛНР Андрей Марочко.	
\end{zznagolos}

\ii{09_01_2022.stz.news.lnr.lug_info.1.nedelja.pic.1}

\subsubsection{НА ЛИНИИ СОПРИКОСНОВЕНИЯ}

Минувшая неделя для жителей Луганской Народной Республики прошла относительно
спокойно, несмотря на это обстановка на линии соприкосновения по-прежнему
оставалась стабильно напряженной.

По данным наблюдателей представительства ЛНР в СЦКК, нарушений режима
прекращения огня со стороны вооруженных формирований Украины не зафиксировано.

Всего с 00:00 часов 21 июля 2019 года режим всеобъемлющего устойчивого и
бессрочного прекращения огня со стороны вооруженных формирований Украины был
нарушен 1000 раз, из них 666 раз – после вступления в силу допмер; 42 защитника
Республики погибли, 24 получили ранения. Среди мирного населения 5 человек
погибли, 38 получили ранения, повреждено 192 объектов гражданской
инфраструктуры.

\subsubsection{РАССАДНИК ПАНДЕМИИ}

Первый день наступившего года в нэзалежной традиционно отпраздновали факельными
шествиями в честь дня рождения пособника фашистов и главаря ОУН Степана
Бандеры. В шабаше в Киеве участвовало не менее тысячи человек, среди которых
было большое количество молодежи и даже детей. Причем дети не уступали своим
старшим побратимам, наравне с ними несли горящие факелы и выкрикивали
антисемитские и антироссийские лозунги.

Такое омоложение не случайно, ведь украинские радикалы и националисты тщательно
выпестывают свое \enquote{будущее}. Уже на протяжении многих лет националистическая
\enquote{Свобода}, радикалы \enquote{Правого сектора}, С14 и другие группировки занимаются
\enquote{просвещением} молодежи, а Запад их активно поддерживает и спонсирует,
прикрываясь ширмочкой заботы о демократии.

Результатом \enquote{правильного} образования стало ежегодное увеличение в
геометрической прогрессии пособников фашизма. Примечательно, что за все это
время страны, за исключением России, которые пострадали от рук бандеровцев,
преступно молчат, и лишь изредка кто-то на уровне общественности выражает
обеспокоенность. Вот и в этот раз антисемитский плакат с призывом к
\enquote{Нюрнбергу-2} в отношении \enquote{жидовско-московского коммунизма}
вызвал не более чем возмущение еврейской общины, а посольство Израиля на
Украине традиционно осудило проведение факельного марша в Киеве без каких-либо
последствий для нэзалежной.

Только по официальной статистике во время Великой Отечественной Войны УПА
зверски убила и замучала более миллиона евреев, и спустя 80 лет Телль-Авив
постыдно продолжает общаться с теми, кто героизирует виновных в этой трагедии.
Также прикусили язык поляки, чехи, венгры, словаки, французы и многие другие
народы, до которых дотянулись бандеровские каратели. Более того, эти страны
сейчас активно поддерживают Киев в его стремлении идти по пути \enquote{национальной
идентичности}. 

Уже в ближайшем будущем детишки, марширующие сегодня по Хрещатику, подрастут, и
большинство из них в поисках заработка поедут в другие страны, где будут
заражать остальных идеями нацизма. Видимо, этого и добиваются соросовские
фонды, используя Украину как рассадник для будущей коричневой пандемии. Ведь
для того, чтобы получать сверхприбыли и сохранять контроль над ресурсами,
американцам нужно повсеместное, тотальное состояние конфликта, а искры, которые
могут разжечь пламя новой глобальной войны, все интенсивней выбиваются
нацистским огнивом на Украине.

\subsubsection{ОБЕЗВОЖИВАНИЕ}

То, что от Киева можно ждать любую подлянку, знает каждый житель Донбасса.

В новом году Украина решила не нарушать традиции и с 21:00 2 января устроила
водную блокаду жителям ряда населенных пунктов ЛНР, прекратив подачу воды с
Западной фильтровальной станции.

В связи с этим прифронтовой Первомайск остался полностью без воды, а подача в
Стаханов, Кировск, Алчевск, Брянку, и Перевальский район была значительно
сокращена. По официальной версии причиной стала авария на магистральном
водоводе, по крайней мере, так сообщили оттуда нашему предприятию
"Лугансквода", но у многих возникли сомнения.

Бесспорно, техническое состояние водопровода, который был построен в середине
прошлого столетия, оставляет желать лучшего, однако жители пострадавших
населенных пунктов отмечают странную закономерность. Как только праздник или
какое-то торжественное событие в ЛНР - у украинских поставщиков воды находятся
причины чтобы отключить воду. Причем они прекрасно знают, что без воды остаются
прежде всего коммунальные предприятия, которые обеспечивают жизнедеятельность
населенных пунктов, но моральный аспект тут - второстепенный.

Если жители еще как-то могут запастись водой, то для больниц, детских садов и
других учреждений это весьма проблематично, и нужен подвоз в большом
количестве. Конечно, МЧС ЛНР и другие службы организовали экстренную доставку
воды, а предприятие \enquote{Лугансквода} начало использовать резервы, но разве можно в
одночасье обеспечить всех?

Также нужно отметить, что денежные средства за потребленную воду наша
Республика выплачивает Украине без задержек и в полном объеме, а вот качество
воды и услуг поставщика вызывают массу вопросов. Буквально меньше месяца назад
поставляемую жидкость с характерным специфическим запахом трудно было назвать
водой, и этот вопрос даже был вынесен на заседание Контактной группы. Украина
обещала исправиться, но по факту стала еще и отключать техническую воду. Самое
печальное, что на данный момент отказаться от потребления воды с оккупированной
территории нельзя по многим причинам, а люди в этой ситуации становятся
заложниками.

5 января подача воды была возобновлена. Хотелось бы надеяться, что таких
\enquote{сюрпризов} больше не будет, однако печальный опыт говорит о том, что пока не
будет решен вопрос с собственными источниками водоснабжения, водная блокада со
стороны Украины будет вводиться регулярно.

\subsubsection{ЕВРООДНОБОКОСТЬ}

Верховный представитель Европейского союза по иностранным делам и политике
безопасности Жозеп Боррель в период новогодних и рождественских праздников
решил нарушить культурно-политические европейские традиции и вместо того, чтобы
попивать вермут в кругу семьи, решил потрудиться и посетить Украину.

Свою первую в новом году командировку глава европейской дипломатии начал с
линии соприкосновения у Станицы Луганской. Событие знаковое, ведь крайний раз
главный европейский дипломат добирался до Украины еще в 2014 году.

На совместном брифинге в Станице с министром иностранных дел Украины Дмитрием
Кулебой Боррель произнес традиционные мантры про поддержку мирных инициатив,
снижение напряженности и недопустимость \enquote{российской агрессии}. Из новых камней
в огород Москвы был небольшой окатыш с критикой якобы попыток Кремля снова
поделить европейский континент на сферы влияния. Как говорится, кто как
обзывается...

Из самой поездки и всего выше сказанного нетрудно сделать ряд выводов. Прежде
всего, Боррель продемонстрировал то, что страны Евросоюза заняли однобокую
позицию, и судьба жителей Донбасса, которые не приняли госпереворот на Украине,
Запад не интересует. ЕС поддерживает тех, кто восьмой год убивает в своей
стране непокорных, используя армию и запрещенные методы ведения войны. Также
объединенная Европа предпочитает не замечать нарушения Украиной правовых,
культурных, этических и других норм, которые принято называть европейскими
ценностями.

Как отметил наш представитель на Минских переговорах Родион Мирошник, в речи
представителя Европейского союза ни разу не прозвучали слова ни об ущемленных
правах людей, живущих в Донбассе, и нарушенных Киевом, ни об игнорируемых
Киевом Минских обязательствах, ни о деструктивном подходе Украины к диалогу
между сторонами конфликта.

Боррель, по сути, расставил все точки над вопросами европейской безопасности за
неделю до проведения переговоров между РФ и США. До этого Россия прочертила
красные линии, и нам лишь остается дождаться начала следующей недели, когда
геополитический натюрморт начнет явственно прорисовываться.
