%%beginhead 
 
%%file 17_10_2022.fb.galushko_denis.1.znayu_shcho_u_vr_lez
%%parent 17_10_2022
 
%%url https://www.facebook.com/HalushkoDenis/posts/pfbid02qfWPPdfGiSDhEa8Y5ncS4fEwMV2KrqrMtEpyH4QmQt8kRW9mUjs3C6fZbtDUUtfXl
 
%%author_id galushko_denis
%%date 17_10_2022
 
%%tags 
%%title Знаю що у ВР лежить проект закону N8090 від Serhiy Rudyk
 
%%endhead 

\subsection{Знаю що у ВР лежить проект закону N8090 від Serhiy Rudyk}
\label{sec:17_10_2022.fb.galushko_denis.1.znayu_shcho_u_vr_lez}

\Purl{https://www.facebook.com/HalushkoDenis/posts/pfbid02qfWPPdfGiSDhEa8Y5ncS4fEwMV2KrqrMtEpyH4QmQt8kRW9mUjs3C6fZbtDUUtfXl}
\ifcmt
 author_begin
   author_id galushko_denis
 author_end
\fi

Знаю що у ВР лежить проект закону N8090 від Serhiy Rudyk

Який ще не розглядав комітет.

Короче лежить і далі там, чекає на погоджувальну раду. 

Серед нас є ті, які як я були на Вуглегірській Тес, Пісках, під Марїнкою,
тепер Бахмут, цього для тих козарлюг було достатньо, щоб декілька разів
бувало, що і на день подумки попрощатися з сім'єю.

Тому хочу звернутися принаймі до того, кого ще пам'ятаю з 14-го, з
волонтерських засад і ініціатив David Braun ви можете якось посприяти цій
законотворчості щоб можливість виїзду військових у відпустку до своїх родин
закордон стала реальністю? 

Згідно законадавства європейських країн родини, що отримали допомогу і
притулок закордоном не мають також права повертатися до України до певного
періоду часу. 

Тому ситуація наступна, або ми змінемо радянське законодавство щодо
військових або варто вступати в ЄС, щоб наші родини мали змогу приїхати до
нас. 

Я пішов добровольцем, зараз в ЗСУ, в мене троє дітей, щоб одразу полишити
якісь хворі фантазії на тему, залишити мою країну, яку я пішов боронити за
власним бажанням. 

Дякую за увагу...

Далі не основне, але про мою реальність сьогодні і важливе особисто для мене...

Мої діти забувають мене.

Я не бачив їх від 23 лютого. 

Не торкався не обіймав.  

Ми не говорили зранку з коханою за війну, так вона далеко.

Вона прислала мені сьогодні відео, де наймолодша доця, каже, що не знає як
намалювати тата:

«Це буня, це мама, а як намалювати тата я не знаю...» 

Вже майже закінчується восьмий місяць повномасштабної війни, мої діти забувають
мене і серед наших є багато, тих хто не бачив своїх, від 23 лютого і їхати їм у
відпустку нема до кого.

Мій син відсвяткував свій др без мене, через три дні др у старшої доньки, а
за два тижні моїй молодшій мацьопці виповниться чотири роки, з яких майже рік
вона мене не бачила.

Ось така реальність, бійців ЗСУ, кращі роки життя, дитинства, наших дітей, ми
присвячуємо тому, щоб у всіх, а не тільки в них, було майбутнє, краще за цю
реальність.

Служу народу України!
